%\documentclass[conference]{IEEEtran}
%\IEEEoverridecommandlockouts
\documentclass[preprint,12pt]{elsarticle}

\usepackage{placeins}
\usepackage{afterpage}
\usepackage{cite}
\usepackage{amsmath,amssymb,amsfonts}
\usepackage{algorithmic}
\usepackage{graphicx}
\usepackage{textcomp}
\usepackage{xcolor}
\def\BibTeX{{\rm B\kern-.05em{\sc i\kern-.025em b}\kern-.08em
    T\kern-.1667em\lower.7ex\hbox{E}\kern-.125emX}}

\usepackage{pdflscape}
\usepackage{pifont}
\usepackage{amssymb}
\usepackage[numbers]{natbib}
\usepackage[hyphens]{url}
\usepackage{color, colortbl}
\usepackage{caption}
\usepackage{subcaption}
\usepackage{lipsum}
\usepackage{booktabs}
\usepackage{multirow}
\usepackage{makecell}
\usepackage[utf8]{inputenc}
\providecommand{\tightlist}{%
  \setlength{\itemsep}{0pt}\setlength{\parskip}{0pt}}

\usepackage{pgfplotstable} % For reading tables from csv
\usepackage{booktabs} % For better table lines
\usepackage{rotating}
\usepackage{numprint} % For proper number formatting
\usepackage{pgf}
\pgfkeys{/pgf/number format/.cd ,precision=2,sci generic={exponent={\times 10^{#1}}}}
\newcommand\convert[1]{\pgfmathprintnumber{#1}}
\pgfplotsset{compat=1.18}

\definecolor{Gray}{gray}{0.9}
\newcommand{\cmark}{\ding{51}}
\newcommand{\xmark}{\ding{55}}
\newcommand{\val}[1]{\begin{tabular}[c]{@{}l@{}} #1 \\~\\~ \end{tabular}}
%\newcommand{\fullcite}[1]{\citeauthor{#1}~\citep{#1}}
\newcommand{\fullcite}[1]{\citet{#1}}
\newcommand{\tildex}[1]{$_{\widetilde{~}}$#1}
\newcommand{\GHicon}[1]{\raisebox{-.15\height}{\,\includegraphics[scale=0.02]{img/github-mark-logo.png}}~\texttt{#1}}
\newcommand{\so}{\textsc{Stack Overflow}}
\newcommand{\gh}{\textsc{GitHub}}
\newcommand{\se}{Software Engineering}

\newcommand\spaceBelowAuthors{-0.7cm}

\journal{Information and Software Technology}

\begin{document}

\begin{frontmatter}

\title{A Lot of Talk and a Badge: An Exploratory Analysis of Personal Achievements in GitHub}

%\author{\IEEEauthorblockN{Fabio Calefato}
%\IEEEauthorblockA{%\textit{dept. name of organization (of Aff.)} \\
%\textit{University of Bari}\\
%Bari, Italy \\
%0000-0003-2654-1588}
%\and
%\IEEEauthorblockN{Luigi Quaranta}
%\IEEEauthorblockA{%\textit{dept. name of organization (of Aff.)} \\
%\textit{University of Bari}\\
%Bari, Italy \\
%0000-0002-9221-0739}
%\and
%\IEEEauthorblockN{Filippo Lanubile}
%\IEEEauthorblockA{%\textit{dept. name of organization (of Aff.)} \\
%\textit{University of Bari}\\
%Bari, Italy \\
%0000-0003-3373-7589}
%}

\author[inst1]{Fabio Calefato}
\author[inst1]{Luigi Quaranta}
\author[inst1]{Filippo Lanubile}

\affiliation[inst1]{organization={University of Bari, Dipartimento di Informatica},%Department and Organization
            addressline={Via E. Orabona, 4}, 
            city={Bari},
            postcode={70125}, 
            country={Italy}}


%\maketitle


\begin{abstract}
\textit{\textbf{Context}.} \gh\ has introduced a new gamification element through personal achievements, whereby badges are unlocked and displayed on developers' personal profile pages in recognition of their development activities.
\textit{\textbf{Objective}.} In this paper, we present an exploratory analysis using mixed methods to study the diffusion of personal badges in \gh, in addition to the effects and reactions to their introduction. 
\textit{\textbf{Method}.} First, we conduct an observational study by mining longitudinal data from more than 6,000 developers and performed correlation and regression analysis.
Then, we conduct a survey and analyze over 300 \gh\ community discussions on the topic of personal badges to gauge how the community responded to the introduction of the new feature.
\textit{\textbf{Results}.} We find that most of the developers sampled own at least a badge, but we also observe an increasing number of users who choose to keep their profile private and opt out of displaying badges. 
Besides, badges are generally poorly correlated with developers' qualities and dispositions such as timeliness and desire to collaborate.
We also find that, except for the \texttt{Starstruck} badge (reflecting the number of followers), their introduction does not have an effect.
Finally, the reaction of the community has been in general mixed, as developers find them appealing in principle but without a clear purpose and hardly reflecting their abilities in the current form.
\textit{\textbf{Conclusions}.} We provide recommendations to \gh\ platform designers on how to improve the current implementation of personal badges as both a gamification mechanism and as sources of reliable cues of ability for developers' assessment.

\end{abstract}

%\begin{IEEEkeywords}
%gamification, achievements, badges, social translucence, signaling theory, open source software
%\end{IEEEkeywords}

\begin{keyword}
%% keywords here, in the form: keyword \sep keyword
gamification \sep achievements \sep badges \sep social translucence \sep signaling theory \sep open source software \sep mining software repositories

%% PACS codes here, in the form: \PACS code \sep code
%\PACS 0000 \sep 1111
%% MSC codes here, in the form: \MSC code \sep code
%% or \MSC[2008] code \sep code (2000 is the default)
%\MSC 0000 \sep 1111
\end{keyword}

\end{frontmatter}

\section{Introduction}

Gamification refers to the practice of incorporating elements and mechanics of game design into non-game contexts~\citep{hamari2007,hunter2012}.
The goal of gamification is to increase user engagement and motivation, as well as drive desired behaviors by creating a sense of play and/or competition~\citep{deterding2011,robson2015}.
There have been several attempts to propose frameworks for the gamification of \se\ activities~\citep{garcia2017,sasso2017} and methods to build gamified software~\citep{morschheuser2018}.
Previous research has also found evidence that gamification mechanisms are in part responsible for the success of websites such as \so~\citep{vasilescu13}.
Gamification, in fact, has always been a staple of the technical Q\&A site whose members gain badges, medals, and privileges by being good community citizens (e.g., answering questions, editing poorly written posts, etc.).
Another popular source of user-generated content in the \se\ domain is \gh, which, in contrast, has always incorporated several social features (e.g., developers have their personal profiles and can follow each other, repositories can be starred), but only  implemented few and simple gamification elements on its website, such as code streaks counters, which were removed from the contribution graph in May 2016~\citep{vessels2016github}.
However, this suddenly changed in June 2022, when a blog post~\citep{brooks2022github} announced the introduction of \textit{personal achievements}, a new feature whereby \textit{badges} are unlocked in recognition of developers' activities and displayed on their personal profile pages.
However,  unlike \so, neither the list of available badges is publicly disclosed by \gh\ nor their meaning and the rules to unlock them.
Previous research has shown that implementing gamification elements in collaborative software platform may steer the behavior of developers in unexpected and unwanted directions~\citep{Moldon2021}.
In this work, We find evidence that it is not entirely clear to the \gh\ community how to interpret the meaning of personal achievements.
As shown in Figure~\ref{fig:discussion}, a user is asking the community for help (1) on how to interpret the \texttt{Pull Shark} badge (2), and two other users agree with the original poster that the meaning of the badge is ambiguous and not intuitive (3).

%% add here the motivation about GH as recruitment helper for recruiters and job seekers and the risk of sending unintended or misreading signals
\gh\ has not only earned the reputation of being the one-stop-shop for software development~\citep{metz2015} but has also quickly become a useful platform for job seekers~\citep{superhire-blog,hacker-news-post,anti-pattern-github-resume}.
Employers access applicants' \gh\ profile pages to supplement resumes and further assess the knowledge, skills, and abilities of candidates~\citep{marlow2013activity}.
\fullcite{marlow2013activity} found that \gh\ design strongly influenced the activity traces available through profile pages, which employers look at during applicants' assessment.
In particular, they observed that employers look at those activity traces that provide reliable cues and require lower effort to access and evaluate them. 
Because personal achievement badges are gamification elements that provide quick and low-effort access to activity traces, \gh\ users who display them on their profile pages may be unintentionally `giving off' cues that have unexpected repercussions, for example, to managers assessing them as job candidates and to other OSS developers who review their contributions~\citep{singer2013,tsay2014}.
In this study, we find evidence that few of the activity cues conveyed through personal badges are actually associated with the intended developers' abilities.

In this paper, we investigate the effects of introducing a new gamification element into \gh.
Specifically, we conduct a mixed-methods study of personal achievement badges (in short, \textit{personal badges} hereinafter).
We characterize the distribution of badges in \gh\ (RQ$_1$), investigate the qualities of developers they are intended to signal and how well they correlate with these qualities (RQ$_2$), and look into the effects, if any, that the introduction of this feature has had on developers' activities (RQ$_3$).
Finally, we gauge how the \gh\ community feels about the new feature (RQ$_4$).

\begin{figure*}[t]
    \centering
    \includegraphics[width=0.7\textwidth]{img/discussion.png}
    \caption{An excerpt taken from a \gh\ discussion on the difficulty of interpreting the meaning of badges.}
    \label{fig:discussion}
\end{figure*}

After mining longitudinal data from over 6,000 \gh\ developers, we find that while most own at least one badge, the number of those who chose to have a private page or opted out of displaying badges on their profile has been steadily increasing.  
%After mining longitudinal data from over 6,000 \gh\ developers, we analyzed the distribution of badges over a six-month time span (from Jun. to Dec. 2022) and found that after an initial interest in Jul., when the number of users with at least one badge reached  85.74\%, the same number started  decreasing while the number of those who chose to have a private page or opted out of displaying badges on their profile steadily increased, reaching 17.34\% in Nov.  
Furthermore, the correlation and regression analyses reveal that, respectively, only one out of the five types of personal badges analyzed (i.e., \texttt{Starstruck}) provides a reliable signal of the associated developer quality (i.e., popularity) and that its introduction has positively affected the number of followers for developers displaying the badge.
Finally, the analysis of 54 responses to a survey and 312 community discussions shows that, while some users find badges somewhat appealing and nice to display, most find them unreliable as visual cues of developers' activity or skills.
%because, unlike \so, they are too easily achieved and not properly `earned.'


The remainder of this paper is organized as follows.
In Section~\ref{sec:background}, we illustrate the theoretical background of the paper.
In Section~\ref{sec:res-quest}, we present the four research questions.
The empirical study is described in Section~\ref{sec:study}, together with the dataset and the methods used to conduct the investigation.
The results are presented in Section~\ref{sec:results} and discussed in Section~\ref{sec:discussion} together with related work.
Finally, we discuss the limitations in Section~\ref{sec:limitations} and conclude in Section~\ref{sec:conclusions}.



\section{Background}\label{sec:background}

\subsection{Social Translucence}
One of the reasons behind the success of \gh\ is that it has been designed as a `socially translucent' system~\citep{erickson2000}, which makes socially salient information (e.g., stars, followers) and activities of participants (e.g., contribution calendar) visible, thus providing the basis for inferences, planning, and coordination.
On the contrary, previous OSS forges like SourceForge were generally opaque to social information. 

According to \fullcite{erickson2000}, there are three building blocks to designing socially translucent knowledge platforms, namely \textit{visibility}, \textit{awareness}, and \textit{accountability};
visibility and awareness ensure that users modulate their actions appropriately (e.g., seeing who made the latest changes to a file and who is assigned to fix a bug report), and accountability ensures that norms and rules come into play as effective mechanisms of social control (e.g., feeling the pressure of fixing a build-breaking commit).

We argue that profile pages, which show developers' salient personal information, activity traces, and now also personal badges, act as a hub for socially translucent\footnote{\citeauthor{erickson2000} use the adjective \textit{translucent} instead of \textit{transparent} to highlight privacy concerns and constant tension between showing just enough and too much information on socially augmented platforms. We use the term consistently.} \textit{signals} that support visibility, awareness, and accountability in \gh.

\subsection{Signaling Theory}

When choosing with whom to collaborate, especially online, much of what we want to know about others is not directly observable. Instead, we rely on \textit{signals}, i.e., perceivable features and actions that indicate the presence of the qualities of interest~\citep{donath2008}.
For example, on collaborative development platforms like \gh\, we can rely on the overall number of code-review comments as a signal of a developer's timeliness and commitment; similarly, a large number of contributed pull requests can signal technical skills and high productivity.

\textit{Signaling theory}, initially developed in the fields of economics~\citep{spence1978job} and biology~\citep{zahavi1975mate}, is useful to model the relationship between signals and qualities, and to explain how only certain signals can be considered reliable.
At its core, signaling theory is fundamentally concerned with reducing information asymmetry between the \textit{signaler}---i.e., the party who possesses some pieces of information/personal attributes and benefits %(e.g., a mate or a job applicant selected over some alternatives) 
from the actions taken by the other party---and the \textit{receiver}---i.e., an outsider who lacks information and would in turn benefit from making decisions based on it.  
For example, in biology, a peacock (the signaler) shows off the size and shape of his tail to be selected in favor of alternative partners (the signaler's benefit) by a peahen (the receiver) because a larger tail is a signal of a healthy bird and a better chance of healthy offspring (the receiver's benefit)~\citep{zahavi1975mate}. 

However, not all signals are \textit{reliable}. To be considered reliable, the cost of deceiving a signal must outweigh the potential benefits.
In terms of reliability, we distinguish two classes of signals.
The \textit{assessment signals}, which are inherently reliable because the signal can be produced only if the signaler possesses the indicated quality.
For example, being able to lift a heavy weight in a gym is an assessment signal of someone's strength---a person without enough strength simply would not be able. 
The other class of signals, named \textit{conventional signals}, is not inherently reliable because the link between signal and quality is arbitrary, and is established by social conventions.
For example, owning a gym t-shirt is a conventional signal of strength, as it is easily accessible, even if the wearer is weak.
These types of signal are widespread in online interaction; for example, the descriptions on online profile pages are conventional signals, which are unreliable because they are more open to deception.
Therefore, for signals to be reliable, they have to be \textit{observable}---receivers need to be able to notice them--- and they have to be \textit{costly} to produce.
In the labor market scenario, where employers lack information on job applicants, high-quality prospective employees distinguish themselves from the other low-quality applicants by highlighting in their resumé their education;
holding a degree with honors from a prestigious institution is a signal that is easily observable for recruiters and is also costly to produce and hard to fake without being smart and hard working~\citep{spence1978job}. 

Previous research has shown that easily observable signals conveyed through online profiles in translucent, social coding platforms (e.g., \so\ reputation points, \gh\ activity graph) can act as proxies of expertise~\citep{shami2009,marlow2013impression}, awareness~\citep{dabbish2012}, and commitment~\citep{tsay2013,tsay2014}.
Accordingly, here we want to investigate the extent to which personal badges in \gh\ can be considered reliable as signals (whether assessment or conventional) of desirable yet directly unobservable qualities and personal dispositions of developers (e.g., timeliness, popularity, desire to collaborate, etc.), useful in the scenarios of open source software development collaboration (i.e., `people sensemaking')~\citep{shami2009} and developer hiring~\citep{marlow2013activity}. 

\subsection{Gamification in SE}

%Gamification is the application of game elements and mechanics to non-game activities, to improve users’ engagement, motivation, and participation~\citep{deterding2011}.
Gamification in software engineering is common, especially in the social programmer software ecosystem~\citep{singer2013}, where several types of elements of game design are often used~\citep{depaulaport2021}.
\fullcite{pedreira2015} conducted a systematic mapping study on gamification in software engineering and found that badges, i.e., tokens that users can display only after completing specified activities~\citep{anderson2013,fanfarelli2015}, is the second most used gamification element after point-based reputation.
For example, \so\ implements a reward system that combines reputation scores and leaderboards with badges to publicly reward users for their contributions to the Q\&A community. 

\gh\ has recently implemented personal badges to reward developers for the achievements unlocked through their activities on the coding platform.
This is the first work to study the effects of badges and the reactions of the community to this new gamification element since its implementation in \gh.
Previous research on \gh, in fact, has focused on badges as visual cues added by developers to README pages to give a quick look at repository and source code statistics~\citep{trockman2018,Legay2019,Venigalla2022}.

Although intended to increase user participation and steer user behavior in desired directions, gamification can have unwanted effects, even resulting in lower engagement, satisfaction, and performance~\citep{hammedi2021}.
\fullcite{Moldon2021} examined how the behavior of \gh\ developers changed after the removal of the code streak counters from the activity graphs displayed on their profile pages.
They found that long-running streaks became less common, thus highlighting the power of gamification as a channel for social influence and raising awareness on the potential consequences of adding gamification elements to collaborative software development platforms.

Finally, \so\ provides a complete list of existing achievements, as well as the rules and even community guidelines~\citep{calefato2018} for detailing how to obtain them.  
However, an undesirable side effect of knowing which achievements are there and how to unlock them is that the behavior of users may change after obtaining a badge. 
\fullcite{grant2013} observed that the number of edits to posts in \so\ drastically drops once users obtain related achievements.
In \gh, neither the list of achievements nor the rules for obtaining them are publicly disclosed.
We speculate that this design choice may be intended to help avoid the risk that making the list of achievements public steers the developers' coding behavior toward the game and, conversely, ensures that the regular coding activity on the platform is rewarded with achievement badges.

\section{Research Questions}\label{sec:res-quest}
The overall goal of this paper is to characterize the diffusion, usefulness, and perception of personal badges in \gh.
Consequently, we define and answer the following four research questions.

\textit{RQ$_1$ -- What is the distribution of personal badges unlocked on GitHub?}\\ 
With the first research question, we want not only to uncover the distribution of unlocked personal badges by type but also to assess whether and how many developers opted out of displaying them on their public profile page.

\textit{RQ$_2$ -- Are personal badges reliable signals of developers' qualities?}\\
%The second research question builds on the signaling theory and seeks to understand whether personal badges are reliable signals of developers' characteristics, such as timeliness and popularity.
The second research question builds on the signaling theory.
Because there is no official description of personal badges, we first seek to understand what developers' qualities they are signaling (e.g., timeliness, popularity, desire to collaborate).
Then, we assess whether these signals are reliable.

\textit{RQ$_3$ -- Did the introduction of personal badges affect developers' activities?}\\
The unlocking rules and the badge list are not disclosed by \gh. 
However, there are several unofficial resources, such as repositories\footnote{\url{https://github.com/Schweinepriester/github-profile-achievements}} and community discussions,\footnote{e.g., \url{https://github.com/community/community/discussions/28656}} which provide insights on the badges discovered so far and how to obtain them.
As such, we seek to understand the effects of the introduction of personal badges on the activities that developers typically perform in \gh.
As detailed in Section~\ref{sec:res-rq2}, we define activities in terms of several selection and collaboration scenarios that are typical of OSS development.

\textit{RQ$_4$ -- How does the \gh\ community feel about personal badges?}\\
Finally, with the fourth research question, we investigate the response of the \gh\ community to the introduction and perceived usefulness of personal badges. 


\section{Empirical Study}\label{sec:study}


\subsection{Data}\label{sec:data} 
We collected a multidimensional longitudinal dataset of badges and other activities of \tildex{6k} \gh\ developers, using the following procedure.

Data were collected from a sample of organizations used in the preparation of the follow-up to a previous study of core developers who abandon \gh\ projects~\citep{Calefato2022}.
At the time of writing, \gh\ hosts \tildex{28} million public repositories.
\fullcite{Kalliamvakou2014} reported several perils when mining \gh, as many repositories do not contain source code or are not software-engineered projects.
Following their recommendations, we devised a selection approach that would filter out personal or inactive projects that do not contain source code (e.g., static websites, documentation projects) or do not have a sufficient development activity history (e.g., commits, pull requests).
Consequently, we started with the selection of organizations and, from the \textit{Topics} section on the \gh\ website, we identified the ten most trending topics. 
Then, we moved on to select five to six organizations per topic. 
We ended up selecting 58 organizations. 
Finally, for each organization, we used the \gh\ API to retrieve the list of organization members.
The API returned an error when retrieving the members of 7 organizations; hence, we ended up with 6,022 developers from 51 organizations. 
Then, for each organization, we selected either `reference project' (e.g., \GHicon{rails/rails}, \GHicon{nodejs/node}); when unclear, we chose the largest in terms of contributors (i.e., \GHicon{laravell/framework}) and, if more projects had a similar number of contributors, we chose the one with most stars. 
With this approach, we were able to generate a heterogeneous sample of projects that vary in terms of size (contributors, pull requests, LOC), history (age), and programming language.
Refer to Table~\ref{tab:projects} in~\ref{appendix:projects} for a breakdown of the characteristics of the selected projects.


%We started from the opportunistic selection of 51 \gh\ organizations. 
%The sampling was guided by the intention to select projects varying in terms of size (i.e., number of files and developers), development language (e.g., Python, Java, JavaScript), and application domain (e.g., npm, docker, rails, TensorFlow); then, we retrieved the list of all organizations' members (6,022).
Next, we detail the data collected and measures for each research question.
%%%%%%%%%%%%%%%%%%%%%%%%%%%%%%%%%%%%%%%%%%%%     Data RQ1    %%%%%%%%%%%%%%%%%%%%%%%%%%%%%%%%%%%%%%%%%%%%%%% 
To investigate RQ$_1$ (diffusion of personal badges), because the GitHub API does not support the retrieval of personal badges, we developed a custom scraper in Python. 
For each user, we scraped the profile page to retrieve the list of unlocked personal badges and the time when the achievement was first obtained.
\gh\ introduced personal badges in June 2022; we repeated scraping monthly for six months, from June to December.
We also counted how many \gh\ developers chose not to have a public profile page or opted out of displaying personal badges from their public profile page.


%%%%%%%%%%%%%%%%%%%%%%%%%%%%%%%%%%%%%%%%%%%%     Data RQ2 + RQ3   %%%%%%%%%%%%%%%%%%%%%%%%%%%%%%%%%%%%%%%%%%%%%%% 

To investigate the reliability of badges as signals of developers’ qualities (RQ$_2$) and the potential effects of badges on developers' activities (RQ$_3$), we collected several repeated measures (see Section~\ref{sec:methods}) concerning their development activities and popularity.
Regarding development-related activities, we used the \gh\ API to mine the number of Issues and Pull Requests (PRs) opened, closed (merged), and worked on (as assignee) by developers in our sample, the time to close issues and merge pull requests, and the number of commits. % and repositories contributed to. 
As a measure of the popularity of developers in the community \gh, we also collected the number of followers. % and following; 
All these metrics were collected monthly for twelve months, from January to December 2022.

%%%%%%%%%%%%%%%%%%%%%%%%%%%%%%%%%%%%%%%%%%%%     Data RQ4   %%%%%%%%%%%%%%%%%%%%%%%%%%%%%%%%%%%%%%%%%%%%%%%% 

To gauge the perceptions of personal badges in \gh\ (RQ$_4$), we used Google Forms to design an online survey consisting of 15 questions (both closed- and open-ended). 
Survey questions (reported in~\ref{appendix:questionnaire}) are partially based on similar work by~\fullcite{trockman2018} who investigated repository badges in \gh.
After collecting basic demographic data, the questions focused on why respondents choose (not) to display badges on their profile, whether they consider them indicators of development skills, and what inferences they make about other developers who display badges on their profile pages.
The survey was advertised on social media and posted as a question to the \gh\ community discussions space.\footnote{\url{https://github.com/community/community/discussions/37346}}
%Overall, we received 33 answers. 
Respondents received no monetary compensation.
An anonymized copy of the survey is available in the replication package.\footnote{\url{https://doi.org/10.5281/zenodo.7501582}}
%After providing an overview of profile badge adoption in \gh\ and collecting through the survey the hypotheses about their effects, we now test the degree to which the presence of such badges correlates with certain expected qualities (i.e. productivity) of \gh\ developers, which we operationalize with measures such as the number of followers and commits.
Furthermore, to complement the survey and also further assess how the \gh\ community reacted to the introduction of personal badges, we mined the questions and answers posted to the discussions space of the \GHicon{community/community} repository. 
Specifically, we retrieved all posts listed in the \texttt{profile} category\footnote{\url{https://github.com/community/community/discussions/categories/profile}} that also contained the keywords \texttt{badge} or \texttt{achievement.}
%Overall, we mined X,xxx Q\&A threads containing XXX questions and YYY answers.


\subsection{Methods}\label{sec:methods}

%%%%%%%%%%%%%%%%%%%%%%%%%%%%%%%%%%%%%%%%%%%%     Methods RQ1    %%%%%%%%%%%%%%%%%%%%%%%%%%%%%%%%%%%%%%%%%%%%%%% 
To answer RQ$_1$ (diffusion of personal badges), we compute the number and percentages of badges unlocked by developers in our experimental sample for each month, from June to December 2022, together with the number of developers who opted out of visualizing their badges on the profile page or made their profile page private.
In addition, we plot the monthly progression of the number of unlocked badges.

%%%%%%%%%%%%%%%%%%%%%%%%%%%%%%%%%%%%%%%%%%%%     Methods RQ2  %%%%%%%%%%%%%%%%%%%%%%%%%%%%%%%%%%%%%%%%%%%% 
%With respect to RQ$_2$ (reliability of badges as signals), we looked for correlations between the presence of personal badges and differences in the quality they are signaling. 
Regarding the RQ$_2$ (reliability of badges as signals), we first hypothesize and then search for correlations between the presence of personal badges and differences in the qualities they might signal. 
We collect several measures, such as the number of followers, commits, and open and closed issues and PRs.
After verifying the presence of non-normal distributions with the Shapiro-Wilk test, we use the non-parametric Wilcoxon-Mann-Whitney (WMW) test to compare distributions and Cliff's $\delta$ to assess the effect size.



%%%%%%%%%%%%%%%%%%%%%%%%%%%%%%%%%%%%%%%%%%%%     Methods RQ3   %%%%%%%%%%%%%%%%%%%%%%%%%%%%%%%%%%%%%%%%%%%%

Personal badges may be correlated with various qualities of developers.
However, the analysis performed for the second research question cannot assess whether and how the introduction of personal badges has any effect on developer activities.
Consequently, for RQ$_3$ we perform a difference-in-differences (DiD) regression analysis~\citep{angrist2009}, a quasi-experimental approach that uses longitudinal data from observational studies to compare changes that occur over time in the outcome variable (e.g., number of commits) between the treatment group (i.e., developers who unlocked the \texttt{Galaxy Brain} badge) and the control group (i.e., developers who did not).

%Regarding (), for each of the metrics computed (e.g., the number of PRs merged), we perform a trend analysis, that is comparing the historical values of such variables to establish and compare the trends before and after a badge (e.g., \texttt{Pull Shark}) has been obtained.


%%%%%%%%%%%%%%%%%%%%%%%%%%%%%%%%%%%%%%%%%%%%     Methods RQ4   %%%%%%%%%%%%%%%%%%%%%%%%%%%%%%%%%%%%%%%%%%%% 

Finally, to answer RQ$_4$ (perceptions of personal badges), we first quantitatively and qualitatively analyze the responses to the survey; then, we perform a thematic analysis of the discussion posts mentioning personal badges.
%afterward, we qualitatively analyzed those entire threads (i.e., questions and answers),
Specifically, we focus on the questions that express positive / negative feedback, report shortcomings, and provide suggestions to improve personal badges.
%Then, we performed both automatic sentiment analysis and manual inspection of the Q\&A discussion threads collected from community discussions related to badges to gauge further the positive/negative reactions to the introduction of badges. 
%To perform the sentiment analysis task, we used the sentiment analysis module~\citep{calefato2018} of the EMTk toolkit~\citep{calefato2019}, which is specifically trained in the software engineering domain.
%statistically representative sample (95\% confidence level, 5\% margin of error) 

\section{Results}\label{sec:results}

%%%%%%%%%%%%%%%%%%%%%%%%%%%%%%%%%%%%%%%%%%%%     Results RQ1    %%%%%%%%%%%%%%%%%%%%%%%%%%%%%%%%%%%%%%%%%%%% 

\begin{table*}[h!]
%\vspace{-0.1cm}
\caption{Categories of personal achievements badges unlocked as of December 2022 and their distribution in our dataset. The badges highlighted in gray are excluded from the empirical analysis.}
\label{tab:badges_tiers}
\tiny
\begin{tabular}{clclr}
\multicolumn{5}{c}{\textsc{Achievements}}                                                                                                                                                                                               \\ \hline
\textbf{Badge}                     & \textbf{Title}                                                         & \textbf{Earnable?} & \textbf{Description}                                                      & \textbf{Unlocked (\%)} \\ \hline \vspace{-3mm}
\includegraphics[width=7mm, height=7mm]{img/badges/pair-extraordinaire-default.png}                         & \val{Pair Extraordinaire}                                                      & \val{\cmark}             & \val{Coauthored a merged pull request}                                       & \val{847 (4.99\%)}   \\ \vspace{-4mm}
\includegraphics[width=7mm, height=7mm]{img/badges/quickdraw-default.png}                                   & \val{Quickdraw}                                                                & \val{\cmark}             & \begin{tabular}[c]{@{}l@{}}\val{Closed an issue or a pull request within\\5 min of opening}\end{tabular} 
              & \val{1441 (8.48\%)}  \\ \vspace{-3mm}
\includegraphics[width=7mm, height=7mm]{img/badges/starstruck-default.png}                                  & \val{Starstruck}                                                               & \val{\cmark}             & \val{Created a repository that has 16 stars}                                 & \val{1,122 (6.60\%)} \\ \vspace{-3mm}
\includegraphics[width=7mm, height=7mm]{img/badges/galaxy-brain-default.png}                                & \val{Galaxy Brain}                                                             & \val{\cmark}             & \val{2 accepted answers}                                                     & \val{192 (1.13\%)}   \\ \vspace{-3mm}
\includegraphics[width=7mm, height=7mm]{img/badges/pull-shark-default.png}                                  & \val{Pull Shark}                                                               & \val{\cmark}             & \val{2 pull requests merged}                                                 & \val{838 (4.93\%)}   \\ 
\rowcolor{Gray}\vspace{-3mm}
\includegraphics[width=7mm, height=7mm]{img/badges/yolo-default.png}                                        & \val{YOLO}                                                                     & \val{\cmark}             & \val{Merged a pull request without code review}                              & \val{1324 (7.79\%)}  \\ 
\rowcolor{Gray}\vspace{-4mm}
\includegraphics[width=7mm, height=7mm]{img/badges/arctic-code-vault-contributor-default.png}               & \val{Arctic Code Vault Contributor}                                            & \val{\xmark}             & \begin{tabular}[c]{@{}l@{}}\val{Contributed code to repositories in the\\2020 \gh\ Archive Program} \end{tabular} 
      & \val{4,591 (27.02\%)}\\ 
\rowcolor{Gray}\vspace{-4mm}
\includegraphics[width=7mm, height=7mm]{img/badges/public-sponsor-default.png}                              & \val{Public Sponsor}                                                           & \val{\cmark}             & \begin{tabular}[c]{@{}l@{}} \val{Sponsoring open source work via\\ \gh\ Sponsors} \end{tabular}                          & \val{118 (0.69\%)}   \\
\rowcolor{Gray}
\includegraphics[width=7mm, height=7mm]{img/badges/mars-2020-contributor-default.png}                      & \val{Mars 2020 Contributor}                                                     & \val{\xmark}             & \begin{tabular}[c]{@{}l@{}} \val{Contributed code to repositories used in\\Mars 2020 Helicopter Mission} \end{tabular}   & \val{282 (1.66\%)}   \\
\multicolumn{5}{c}{\textsc{Tiers}}                                                                                                                                                                                                      \\ \hline
\multicolumn{1}{l}{\textbf{Badge}} & \textbf{Title}                                                         & \textbf{Tier}      & \textbf{Description}                                                        & \textbf{Unlocked (\%)}  \\ \hline \vspace{-3mm}
\includegraphics[width=7mm, height=7mm]{img/tiers/pair-extraordinaire-bronze.png}                           & \val{Pair Extraordinaire x2}                                                   & \val{Bronze}             & \val{Coauthored 10 merged pull requests}                                     & \val{455 (2.68\%)}     \\ \vspace{-3mm}
\includegraphics[width=7mm, height=7mm]{img/tiers/pair-extraordinaire-silver.png}                           & \val{Pair Extraordinaire x3}                                                   & \val{Silver}             & \val{Coauthored 24 merged pull requests}                                     & \val{469 (2.76\%)}     \\ \vspace{-3mm}
\includegraphics[width=7mm, height=7mm]{img/tiers/pair-extraordinaire-gold.png}                             & \val{Pair Extraordinaire x4}                                                   & \val{Gold}               & \val{Coauthored 48 merged pull requests}                                     & \val{53 (0.31\%)}      \\ \vspace{-3mm}
\includegraphics[width=7mm, height=7mm]{img/tiers/starstruck-bronze.png}                                    & \val{Starstruck x2}                                                            & \val{Bronze}             & \val{Created a repository w/ 128 stars}                                      & \val{598 (3.52\%)}     \\ \vspace{-3mm}
\includegraphics[width=7mm, height=7mm]{img/tiers/starstruck-silver.png}                                    & \val{Starstruck x3}                                                            & \val{Silver}             & \val{Created a repository w/ 512 stars}                                      & \val{507 (2.98\%)}     \\ \vspace{-3mm}
\includegraphics[width=7mm, height=7mm]{img/tiers/starstruck-gold.png}                                      & \val{Starstruck x4}                                                            & \val{Gold}               & \val{Created a repository w/ 4096 stars}                                     & \val{158 (0.93\%)}     \\ \vspace{-3mm}
\includegraphics[width=7mm, height=7mm]{img/tiers/galaxy-brain-bronze.png}                                  & \val{Galaxy Brain x2}                                                          & \val{Bronze}             & \val{8 accepted answers}                                                     & \val{41 (0.24\%)}      \\ \vspace{-3mm}
\includegraphics[width=7mm, height=7mm]{img/tiers/galaxy-brain-silver.png}                                  & \val{Galaxy Brain x3}                                                          & \val{Silver}             & \val{16 accepted answers}                                                    & \val{18 (0.11\%)}      \\ \vspace{-3mm}
\includegraphics[width=7mm, height=7mm]{img/tiers/galaxy-brain-gold.png}                                    & \val{Galaxy Brain x4}                                                          & \val{Gold}               & \val{32 accepted answers}                                                    & \val{12 (0.07\%)}      \\ \vspace{-3mm}
\includegraphics[width=7mm, height=7mm]{img/tiers/pull-shark-bronze.png}                                    & \val{Pull Shark x2}                                                            & \val{Bronze}             & \val{16 pull requests merged}                                                & \val{1,382 (8.13\%)}   \\ \vspace{-3mm}
\includegraphics[width=7mm, height=7mm]{img/tiers/pull-shark-silver.png}                                    & \val{Pull Shark x3}                                                            & \val{Silver}             & \val{128 pull requests merged}                                               & \val{1,897 (11.17\%)}  \\ \vspace{-3mm}
\includegraphics[width=7mm, height=7mm]{img/tiers/pull-shark-gold.png}                                      & \val{Pull Shark x4}                                                            & \val{Gold}               & \val{1024 pull requests merged}                                              & \val{645 (3.80\%)}     \\ \hline
\end{tabular}
%\vspace{-0.2cm}
\end{table*}



\subsection{RQ$_1$ -- Diffusion of Personal Badges}\label{sec:res-rq1}

Table~\ref{tab:badges_tiers} shows the distribution of unlocked badges in our data set as of December 2022.
The statistics in the table reveal that the most popular badge is \texttt{Arctic Code Vault Contributor} (4591, 27.02\%).
This result is `inflated' because this one is among the badges that already existed before the launch of the personal badges in June 2022; besides, this badge is not earnable anymore as it was awarded to anyone contributing code to \gh\ in 2020. 
The other previously existing badges are \texttt{Mars 2020 Contributor} (282, 1.66\%), and \texttt{Public Sponsor} (118, 0.69\%),  unlocked respectively by developers who contributed code to the repositories used in the Mars 2020 helicopter mission and by those who have provided sponsorships to projects via the \gh\ \textsc{Sponsors} program.
Because these three badges predate the introduction of personal achievement badges in June 2022, they were excluded from subsequent analyses.

Regarding the other recently introduced types of badges, the most common (aggregated by tier) is \texttt{Pull Shark} (4,762, 28.03\%), unlocked by merging a certain number of pull requests.
The second most common type of badge in our dataset is \texttt{Starstruck} (2,385, 14.03\%), which is awarded by owning repositories that receive more and more stars from other users.
\texttt{Pair Extraordinaire} (1,824, 10.74\%) and \texttt{Galaxy Brain} (263, 1.55\%) are unlocked respectively by coauthoring merged pull requests and by answering questions in project discussions. 
Finally, \texttt{YOLO} (1,324, 7.79\%) is the personal badge awarded to those who have merged a pull request without performing a code review.
Because the action that earns developers this badge is a development practice that should not be encouraged, it was also excluded from subsequent analyses.
We also discuss the bad coding practice promoted by the \texttt{YOLO} badge during the analysis of the survey responses in Section~\ref{sec:res-rq4:survey}.

As of December 2022, most of the 6,022 developers in our dataset had at least one personal badge displayed on their profile page (4,977, 82.65\%), only 1 (0.02\%) had no personal badge, and 1,044 (17.34\%) had opted out of displaying badges or having a public profile.
Figure~\ref{fig:plotrq1} plots over a six-month time span the distributions of the number of developers with badges (with pre-existing badges excluded) versus those who either opted out of displaying badges on their profile page or without a public profile altogether.
We observe an initial increase in the number of users who display badges from June (4,825, 80.12\%) to July (5,163, 85.74\%); afterward, we notice a decreasing trend (4,452, 73.93\% in November) with more and more developers choosing not to display badges, either opting out or because they chose not to have a public profile page (1,412, 23.48\% in November).

\begin{figure}[t]
    \centering
    %\vspace{-0.75cm}
    \includegraphics[scale=0.58]{img/results/rq1_users_badges.png}
    %\vspace{-0.7cm}
    \caption{Number of users with badges vs. those who opted out of displaying badges or without a public profile page (the filter refers to the exclusion of pre-existing badges.)}
    %\vspace{-0.3cm}
    \label{fig:plotrq1}
\end{figure}

%%%%%%%%%%%%%%%%%%%%%%%%%%%%%%%%%%%%%%%%%%%%     RQ2    %%%%%%%%%%%%%%%%%%%%%%%%%%%%%%%%%%%%%%%%%%%%%%% 

\begin{figure}[t]
     \centering
     \vspace{-2cm}
     \begin{subfigure}[b]{0.45\textwidth}
         \centering
         \includegraphics[width=\textwidth]{img/results/rq2_violinplot_pair-extraordinaire_opened_issues_prs.png}
         \caption{Pair Extraord. (\# issues and PRs opened)}
         \label{fig:rq2_vp_pairextra_opened_issues_prs}
     \end{subfigure}
    \begin{subfigure}[b]{0.45\textwidth}
         \centering
         \includegraphics[width=\textwidth]{img/results/rq2_violinplot_pair-extraordinaire_authored_commits.png}
         \caption{Pair Extraordinaire (\# authored commits)}
         \label{fig:rq2_vp_pairextra_authored_commits}
     \end{subfigure}
     \begin{subfigure}[b]{0.45\textwidth}
         \centering
         \includegraphics[width=\textwidth]{img/results/rq2_violinplot_quickdraw.png}
         \caption{Quickdraw (time to close issues \& PRs)}
         \label{fig:rq2_vp_quickdraw}
     \end{subfigure}
     \begin{subfigure}[b]{0.45\textwidth}
         \centering
         \includegraphics[width=\textwidth]{img/results/rq2_violinplot_starstruck.png}
         \caption{Starstruck (\# followers)}
         \label{fig:rq2_vp_starstruck}
     \end{subfigure}
     \begin{subfigure}[b]{0.45\textwidth}
         \centering
         \includegraphics[width=\textwidth]{img/results/rq2_violinplot_galaxy-brain_opened_issues_prs.png}
         \caption{Galaxy Brain (\# issues and PRs opened)}        \label{fig:rq2_vp_galaxybrain_opened_issues_prs}
     \end{subfigure}
     %\begin{subfigure}[b]{0.20\textwidth}
     %    \centering
     %    \includegraphics[width=\textwidth]{img/results/rq2_violinplot_galaxy-brain_authored_commits.png}
     %    \caption{Galaxy Brain\\(\# authored commits)}
     %    \label{fig:rq2_vp_galaxybrain_authored_commits}
     %\end{subfigure}
     \begin{subfigure}[b]{0.45\textwidth}
         \centering
         \includegraphics[width=\textwidth]{img/results/rq2_violinplot_pull-shark_authored_commits.png}
         \caption{Pull Shark (\# authored commits)}
         \label{fig:rq2_vp_pullshark_authored_commits}
     \end{subfigure}
     \begin{subfigure}[b]{0.45\textwidth}
         \centering
         \includegraphics[width=\textwidth]{img/results/rq2_violinplot_pull-shark_committed_commits.png}
         \caption{Pull Shark (\# committed commits)}
         \label{fig:rq2_vp_pullshark_committed_commits}
     \end{subfigure}
     \caption{Distributions of response variables w/ and w/o badges. WMW $U$, $p$-value, and Cliff's $\delta$ statistics below each figure.}
     \label{fig:rq2_violinsplots}
\end{figure}


\subsection{RQ$_2$ -- Reliability of Personal Badges as Signals}\label{sec:res-rq2}

\textbf{Correlation analysis}. To discover what badges might be signaling, we investigated the associations of their presence with desirable qualities and dispositions of OSS developers.
The results of the correlation analysis between the badges and their hypothesized qualities are reported below.
The distributions are illustrated in Figure~\ref{fig:rq2_violinsplots}, where we report the results of the WMW tests with $p$-values. 
Cliff's $\delta$ values are also reported to gauge the effect size (the magnitude is assessed using the thresholds provided in~\citep{romano2006}, i.e., $|\delta|<.147$ `\textit{negligible}', $|\delta|<.33$ `\textit{small}', $|\delta|<.474$, `\textit{medium}', otherwise `\textit{large}').

To gauge how \gh\ developers perceive the signals sent by personal badges, we defined the following question in the survey (further described in Section~\ref{sec:res-rq4}): ``\textit{Q$_{14}$. For each badge, please report the intended signal that you think it sends to others and, if different, the actual signal it conveys to people visualizing it.''}
As detailed in Section~\ref{sec:res-rq4:survey}, 54 \gh\ developers participated in the survey.
However, for question Q$_{14}$ we received 31 answers.
The somewhat low number of answers suggests that, in general, the \gh\ community may not have a clear understanding of what signals, if any, personal badges send to others; as P$_{30}$ wrote: ``\textit{they're supposed to show me some kind of evidence, but I don't know}.'' 
Consistently, in many cases the respondents replied that they had no clue about the meaning of some badges; however, when they did, their answers showed some consistency in describing the perceived signals.
After analyzing and grouping the mentions of similar signals in the answers, which we considered only if they appeared at least three times, we hypothesized associations between the perceived signal sent by each personal badge and some code development metrics available through the \gh\ API, as described next.

The \texttt{Pair Extraordinaire} badge is awarded to developers who coauthor merged pull requests.
According to the survey responses ($n=6$), owning the badge might signal an increased \textit{desire to collaborate} ($P_{37}$: ``\textit{[the badge] suggests that they like collaborating with other developers to create some pieces of code}'').
Accordingly, we tested its correlation with the number of authored commits, as well as the number of PRs and Issues opened, as pull-based
development has been popularized by \gh\ to collaborate and integrate developers' contributions to project repositories~\citep{Gousios2015,Gousios2016}; also, we aggregated the number of PRs and Issues opened, due to the interlinked nature in \gh\ and because opening an Issue (e.g., for reporting bugs or requesting features) is a valuable form of non-coding collaboration~\citep{Izquierdo2022}.
The results of the WMW tests show the existence of a statistically significant difference between developers with and without the badge and the number of opened Issues and PRs (Figure~\ref{fig:rq2_vp_pairextra_opened_issues_prs}, $p<.001$) with a negligible effect size ($\delta=.132$).
Instead, there is no statistically significant difference in the case of the number of authored commits between developers with and without the \texttt{Pair Extraordinaire} badge (Figure~\ref{fig:rq2_vp_pairextra_authored_commits}).

\afterpage{\FloatBarrier}

Regarding the \texttt{Quickdraw} badge, gained by developers who close Issues and PRs within 5 minutes, according to the survey response ($n=7$) its presence could be a signal of \textit{timeliness}  ($P_{14}$: ``\textit{it may indicate that you're fast, committed};'' $P_{19}$: ``\textit{The word [\texttt{Quickdraw}] says is it...  they're quick};'' $P_{40}$: ``\textit{I guess it signals you're on time, which means dedication too}'').
Consequently, we tested its association with an overall shorter time taken to close Issues and PRs compared to those who do not own the badge.
The result (Figure~\ref{fig:rq2_vp_quickdraw})
shows a statistically significant difference in the WMW test but a negligible effect size ($p<.001$, $\delta < -0.031$).

The most substantial finding is observed for the \texttt{Starstruck} badge, which, according to the respondents ($n=9$) is perceived as a signal of \textit{popularity} since it is unlocked when repositories are starred ($P_{41}$: ``\textit{It tells others that you are a coding rockstar};'' $P_{50}$: ``\textit{[It] means one is popular, I guess};''); therefore, we tested the correlation of its presence on developers' profile pages with the number of their followers, a commonly used proxy measure of developer popularity in \gh~\citep{Blincoe2016}; the results in Figure~\ref{fig:rq2_vp_starstruck} show that there is a statistically significant difference between the distributions of developers with and without the badge in terms of the number of followers ($p<.001$) and that this difference has a large effect size ($\delta=.693$).

The \texttt{Galaxy Brain} badge is unlocked when the answers posted to project discussions are accepted.
According to respondents ($n=4$), its presence could signal an increased \textit{willingness to help} projects grow ($P_{8}$: ``\textit{Maybe support? That you want to help a \gh\ project and its community beyond `just' coding?}'').
\textsc{Discussions}~\citep{Hata2022} is a new feature for asking questions or discussing topics in \gh\ projects, outside of specific Issues or PRs, in a fashion similar to Question-Answering sites such as \so.
The analogy with \so\ suggests that developers may be motivated to answer posts in \gh\ for similar reasons, most notably the desire to help fellow programmers~\citep{calefato2018}, contribute to the  community~\citep{Lu2022}, and build a reputation~\citep{Bosu2013}.
As such, we tested the correlation between its presence and the number of opened Issues and PRs, which again are the main features of \gh\ used by developers to contribute to projects~\citep{Gousios2016}. 
The results in Figure~\ref{fig:rq2_vp_galaxybrain_opened_issues_prs} show that there is a statistically significant difference ($p<0.001$) and a small effect size ($\delta=0.156)$ between developers with and without the badge in terms of opened Issues and PRs.%, whereas there is no significant result in terms of the number of authored commits.

Finally, the \texttt{Pull Shark} badge is obtained when PRs are merged into a project repository.
Based on the survey responses ($n=5$), its presence could represent a signal of \textit{willingness to contribute code} to projects ($P_{29}$: ``\textit{As a maintainer, [\texttt{Pull Shark}] gives me the general idea of someone who wants to write code, for their own or others' repos};'' $P_{42}$: ``\textit{[It] signals that you're an active code contributor}'').
Consequently, we tested its correlation with the number of both authored and committed commits.
In the case of authored commits, the results of the WMW tests show a statistically significant difference, but a negligible effect size between developers with and without the badge (Figure~\ref{fig:rq2_vp_pullshark_authored_commits}, $p<0.01$, $\delta=0.023$). 
Regarding committed commits, we observe a lack of statistical significance (see Figure~\ref{fig:rq2_vp_pullshark_committed_commits}).

\textbf{Correlation analysis by tier}. In general, the correlation analysis revealed limited significant results (see Table~\ref{tab:wmw}, column \textit{All}).
Most differences, in fact, are significant but with negligible or small effect sizes, with the exception of the \texttt{Starstruck} badge, which exhibits a strong association with the number of followers, as a signal of popularity.
One reason for the negligible effect sizes may be due to having conducted the correlation analysis by aggregating all badge tiers (see Table~\ref{tab:badges_tiers}), thus distinguishing only between developers \textit{with} and \textit{without} badges.
Therefore, it might be possible that the tier-2 (\textit{bronze}), tier-3 (\textit{silver}), and tier-4 (\textit{gold}) badges, which are harder to achieve, might send more reliable signals than the basic tier-1 badges of the same type.
Consequently, in Table~\ref{tab:wmw}, we replicate the correlation analyses by tier.
However, we observe only a few variations.
We found that the difference in the number of followers between developers without the \texttt{Starstruck} badge and those who unlocked the basic tier is significant but with a negligible effect size ($\delta = -0.031$); on the contrary, the effect sizes remain large for the other tiers.
Furthermore, for the \texttt{Galaxy Brain} badge, we observe a significant difference with a medium effect size ($\delta = 0.341$) only for the \textit{silver} tier.



\begin{sidewaystable}
\scriptsize
\centering

\caption{Results of the WMW test for each badge, aggregated and by tier. Significant results with medium or large effect sizes are shown in bold.}
\label{tab:wmw}
%\rotatebox{90}{
\begin{tabular}{lcccccccccc}
\hline
\multirow{2}{*}{\textbf{Badge}}                                                         & \multicolumn{2}{c|}{\textbf{All}}                                 & \multicolumn{2}{c|}{\textbf{Basic tier (x1)}}         & \multicolumn{2}{c|}{\textbf{Bronze tier (x2)}}                   & \multicolumn{2}{c|}{\textbf{Silver tier (x3)}}                    & \multicolumn{2}{c}{\textbf{Gold tier (x4)}}   \\
                                                                                        & WMW U                       & \multicolumn{1}{c|}{$\delta$}       & WMW U                 & \multicolumn{1}{c|}{$\delta$} & WMW U                      & \multicolumn{1}{c|}{$\delta$}       & WMW U                       & \multicolumn{1}{c|}{$\delta$}       & WMW U                        & $\delta$       \\ \hline
\begin{tabular}[c]{@{}l@{}}Pair Extraordinaire\\ (\# Issues \& PRs opened)\end{tabular} & \convert{4816418}***          & \multicolumn{1}{c|}{0.132}          & \convert{2109379}***    & \multicolumn{1}{c|}{-0.031}   & \convert{1222880}***         & \multicolumn{1}{c|}{0.146}          & \convert{1336223}***          & \multicolumn{1}{c|}{0.218}          & \convert{147936}***            & 0.229          \\
\begin{tabular}[c]{@{}l@{}}Pair Extraordinaire\\ (\# authored commits)\end{tabular}     & \convert{4565373.0}***         & \multicolumn{1}{c|}{0.073}         & \convert{2065755.0}***   & \multicolumn{1}{c|}{-0.031}   & \convert{1157481.5}***          & \multicolumn{1}{c|}{0.085}         & \convert{1207913.5}         & \multicolumn{1}{c|}{0.101}         & \convert{134223.0}*           & 0.115         \\
\begin{tabular}[c]{@{}l@{}}Quickdraw\\ (time to close Issues \& PRs)\end{tabular}       & \convert{1536895942}***       & \multicolumn{1}{c|}{-0.031}         & \convert{1536895942}*** & \multicolumn{1}{c|}{-0.031}   & -                          & \multicolumn{1}{c|}{-}              & -                           & \multicolumn{1}{c|}{-}              & -                            & -              \\
\begin{tabular}[c]{@{}l@{}}Starstruck\\ (\# followers)\end{tabular}                     & {\convert{4809523}***} & \multicolumn{1}{c|}{\textbf{0.693}} & \convert{2003887.5}***  & \multicolumn{1}{c|}{-0.031}   & \textbf{\convert{1261846}***} & \multicolumn{1}{c|}{\textbf{0.755}} & \textbf{\convert{1166782}***} & \multicolumn{1}{c|}{\textbf{0.880}} & \textbf{\convert{377007.5}***} & \textbf{0.958} \\
\begin{tabular}[c]{@{}l@{}}Galaxy Brain\\ (\# Issues \& PRs opened)\end{tabular}        & \convert{934257}***  & \multicolumn{1}{c|}{0.156} & \convert{670655}***     & \multicolumn{1}{c|}{-0.031}   & \convert{151057}***          & \multicolumn{1}{c|}{0.236}          & \textbf{\convert{75607.5}***} & \multicolumn{1}{c|}{\textbf{0.341}} & \convert{36937.5}            & 0.072          \\
\begin{tabular}[c]{@{}l@{}}Pull Shark\\ (\# authored commits)\end{tabular}              & \convert{4405480.0}**          & \multicolumn{1}{c|}{0.023}          & \convert{748130.5}      & \multicolumn{1}{c|}{-0.031}   & \convert{1247632.5}          & \multicolumn{1}{c|}{0.002}          & \convert{1779396.5}***           & \multicolumn{1}{c|}{0.032}          & \convert{630320.5}***            & 0.082         \\ 
\begin{tabular}[c]{@{}l@{}}Pull Shark\\ (\# committed commits)\end{tabular}              & \convert{4325628.5}          & \multicolumn{1}{c|}{0.004}          & \convert{744374.0}**      & \multicolumn{1}{c|}{-0.031}   & \convert{1235150.5}          & \multicolumn{1}{c|}{-0.008}          & \convert{1737803.0}           & \multicolumn{1}{c|}{0.007}          & \convert{608301.0}***             & 0.045         \\\hline
\multicolumn{11}{l}{*** $p<0.001$,  ** $p<0.01$, * $p<0.05$}                                                                                                                                                                                                                                                                                                                                              
\end{tabular}
\end{sidewaystable}



%%%%%%%%%%%%%%%%%%%%%%%%%%%%%%%%%%%%%%%%%%%%     RQ3   %%%%%%%%%%%%%%%%%%%%%%%%%%%%%%%%%%%%%%%%%%%%%%% 


\subsection{RQ$_3$ -- Effects of Personal Badges on Developers’ Activities}\label{sec:res-rq3}

This section presents the results of the difference-in-differences (DiD) regression analysis. 
The fundamental assumption for applying a DiD regression is that in the absence of treatment (i.e., before the introduction of personal badges) the dependent variable trend would be the same in both the treatment group (i.e., the developers who will eventually unlock a given type of badge) and the control group (i.e., the developers without the badges).
In Figure~\ref{fig:rq3_starstruck_boxplot}, through the boxplots we visually confirm the \textit{parallel} (or \textit{common}) \textit{trends} assumption for the \texttt{Starstruck} badge; Figure~\ref{fig:rq3_quickdraw_boxplot} shows instead a violation of the premise for the \texttt{Quickdraw} badge, which is therefore excluded from this analysis.
The other badge excluded from this analysis for the same reason is \texttt{Pull Shark}, in terms of both the number of authored and committed commits.
Refer to~\ref{appendix:did} for the other figures.

A difference-in-differences regression model is built to estimate equations such as (\ref{eq:dd})
 
\begin{equation}\label{eq:dd}
y_{it} = \alpha + \beta B_i + \gamma BA_t + \delta (B \times BA)_{it} + \epsilon_{it}
\end{equation}
\\
where $y_{it}$ is the dependent variable (outcome) for developer $i$ at time $t$, $B_i$ is the binary variable for developers who own a badge (the treatment group, i.e., $HasBadge_i=1$) or not (the control group, i.e., $HasBadge_i=0$), and $BA_t$ is a time dummy that switches on for observations obtained after June 2022 when personal badges became available (i.e., $BadgeAvail_t=1$ when $t > June$, otherwise $BadgeAvail_t=0$). 
Finally, $\epsilon_{it}$ is the residual term.


\begin{figure}
    %\vspace{-0.1cm}
    \centering
    \begin{subfigure}[b]{0.7\textwidth}
        \centering
        \includegraphics[width=\textwidth]{img/results/rq3_starstruck_followers_grouped_boxplot.png}
        \vspace{-0.9cm}
        \caption{Starstruck (Jan.-Dec. '22)}
        \label{fig:rq3_starstruck_boxplot}
        \vspace{0.9cm}
    \end{subfigure}    
    \begin{subfigure}[b]{0.7\textwidth}
        \centering
        \includegraphics[width=\textwidth]{img/results/rq3_quickdraw_time_close_issues_prs_grouped_boxplot.png}
        \vspace{-0.9cm}
        \caption{Quickdraw (Jan.-Dec. '22)}
        \label{fig:rq3_quickdraw_boxplot}
    \end{subfigure}
    %\vspace{-0.2cm}
    \caption{Examples of visual confirmation (a) and violation (b) of the parallel trends assumption before the introduction of personal badges.}
    %\vspace{-0.3cm}
    \label{fig:rq3_boxplots_parallel}
\end{figure}

The resulting DiD model is an interaction model interpreted as follows~\citep{angrist2009}.
The intercept estimate $\alpha$ is interpreted as the mean of the outcome (i.e., the dependent variable $y$) for the control group (i.e., the developers without the badge) in the month(s) before the introduction of the personal badges feature in \gh\ (i.e., $HasBadge_i=0$ and $BadgesAvail_t=0$).
The coefficient $\beta$ of $HasBadge$ is the expected mean change in the outcome $y$ between the treatment and control groups in the pre-treatment period ($HasBadge_i=1$ and $BadgesAvail_t=0$); this can be viewed as the `baseline difference' in the outcome variable between the two groups before treatment.
The coefficient $\gamma$ of $BadgesAvail$ is the expected mean difference in $y$ before and after the introduction of the personal badges among the control group; 
the main effect for $BadgesAvail$ %(i.e., the post-treatment variable) 
is the effect of the simple passage of time in the absence of the treatment.
Finally, the estimated coefficient $\delta$ of the interaction term is an estimate of the treatment effect. This is the DiD coefficient and the focus of this analysis because the interaction is to test whether the expected mean change in the outcome $y$ before and after the introduction of the new feature is different for the treatment and control groups, respectively, the developers with ($HasBadge_i=1$) and without ($HasBadge_i=0$) the badge of interest.

\begin{table}
\centering
%\vspace{-0.1cm}
\caption{Results of the difference-in-difference regression for each badge. Significant results are highlighted in bold.}
\label{tab:dd-regressions}
\footnotesize
\begin{tabular}{llcc}
\hline
\multicolumn{4}{l}{\textbf{Pair Extraordinaire, log(y=no. issues and PRs opened)}}                          \\ \hline
                             & \multicolumn{1}{c}{\textbf{Coef (S.E.)}} & \textbf{[0.25}  & \textbf{0.75]}  \\ \cline{2-4} 
Intercept ($\alpha$)         & \textbf{0.295 (0.142)*}                  & \textbf{0.015}  & \textbf{0.574}  \\
HasBadge ($\beta$)             & 0.054 (0.198)                            & -0.335          & 0.443           \\
BadgeAvail ($\gamma$)        & \textbf{0.913 (0.169)***}                & \textbf{0.581}  & \textbf{1.245}  \\
HasBadge:BadgeAvail ($\delta$) & 0.374 (0.229)                            & -0.077          & 0.825           \\ \cline{2-4} 
                             & \multicolumn{3}{l}{Adj. R$^2$=0.212}                                         \\ \addlinespace
\multicolumn{4}{l}{\textbf{Pair Extraordinaire, log(y=no. authored commits)}}                          \\ \hline
                             & \multicolumn{1}{c}{\textbf{Coef (S.E.)}} & \textbf{[0.25}  & \textbf{0.75]}  \\ \cline{2-4} 
Intercept ($\alpha$)         & \textbf{0.8.972 (0.053)***}                  & \textbf{8.904}  & \textbf{9.041}  \\
HasBadge ($\beta$)             & -0.107 (0.067)                            & -0.239          & 0.025           \\
BadgeAvail ($\gamma$)        & \textbf{-0.200 (0.040)***}                & \textbf{-0.277}  & \textbf{-0.122}  \\
HasBadge:BadgeAvail ($\delta$) & 0.072 (0.076)                            & 0.344          & -0.077           \\ \cline{2-4} 
                             & \multicolumn{3}{l}{Adj. R$^2$=0.004}                                         \\ \addlinespace\hline
%\multicolumn{4}{l}{\textbf{Quickdraw, log(y=time to close issues and PRs)}}                                 \\ \hline
%                             & \multicolumn{1}{c}{\textbf{Coef (S.E.)}} & \textbf{[0.25}  & \textbf{0.75]}  \\ \cline{2-4} 
%Intercept ($\alpha$)         & \textbf{10.158 (0.014)***}               & \textbf{10.131} & \textbf{10.186} \\
%HasBadge ($\beta$)             & \textbf{0.090 (0.020)***}                & \textbf{0.051}  & \textbf{0.130}  \\
%BadgeAvail ($\gamma$)        & \textbf{0.079 (0.016)***}                & \textbf{0.049}  & \textbf{0.110}  \\
%HasBadge:BadgeAvail ($\delta$) & \textbf{-0.205 (0.022)***}               & \textbf{-0.249} & \textbf{-0.161} \\ \cline{2-4} 
 %                            & \multicolumn{3}{l}{Adj. R$^2$=0.002}                                         \\ \hline
\multicolumn{4}{l}{\textbf{Starstruck, log(y=no. followers)}}                                               \\ \hline
                             & \multicolumn{1}{c}{\textbf{Coef (S.E.)}} & \textbf{[0.25}  & \textbf{0.75]}  \\ \cline{2-4} 
Intercept ($\alpha$)         & \textbf{3.618 (0.011)***}                & \textbf{3.596}  & \textbf{3.639}  \\
HasBadge ($\beta$)             & \textbf{1.386 (0.016)***}                & \textbf{1.354}  & \textbf{1.418}  \\
BadgeAvail ($\gamma$)        & \textbf{-0.100 (0.016)***}               & \textbf{-0.131} & \textbf{-0.070} \\
HasBadge:BadgeAvail ($\delta$) & \textbf{0.169 (0.023)***}                & \textbf{0.124}  & \textbf{0.213}  \\ \cline{2-4} 
                             & \multicolumn{3}{l}{Adj. R$^2$=0.228}                                         \\ \addlinespace \hline
\multicolumn{4}{l}{\textbf{Galaxy Brain, log(y=no. issues and PRs opened)}}                                               \\ \hline
                             & \multicolumn{1}{c}{\textbf{Coef (S.E.)}} & \textbf{[0.25}  & \textbf{0.75]}  \\ \cline{2-4} 
Intercept ($\alpha$)         & \textbf{0.322 (0.104)**}                 & \textbf{0.117}  & \textbf{0.527}  \\
HasBadge ($\beta$)             & 0.007 (0.399)                            & -0.778          & 0.792           \\
BadgeAvail ($\gamma$)        & \textbf{1.166 (0.121)***}                & \textbf{0.929}  & \textbf{1.404}  \\
HasBadge:BadgeAvail ($\delta$) & -0.135 (0.436)                           & -0.991          & 0.721           \\ \cline{2-4} 
                             & \multicolumn{3}{l}{Adj. R$^2$=0.187}                                         \\ \hline
%\multicolumn{4}{l}{\textbf{Pull Shark, log(y=no. issues closed)}}                                   \\ \hline
%                             & \multicolumn{1}{c}{\textbf{Coef (S.E.)}} & \textbf{[0.25}  & \textbf{0.75]}  \\ \cline{2-4} 
%Intercept ($\alpha$)         & 0.641 (0.336)                            & -0.019          & 1.300           \\
%HasBadge ($\beta$)             & 0.129 (0.367)                            & -0.592          & 0.850           \\
%BadgeAvail ($\gamma$)        & \textbf{2.122 (0.370)***}                & \textbf{1.395}  & \textbf{2.850}  \\
%HasBadge:BadgeAvail ($\delta$) & -0.504 (0.407)                           & -1.303          & 0.296           \\ \cline{2-4} 
%                             & \multicolumn{3}{l}{Adj. R$^2$=0.193}                                         \\ \hline
\multicolumn{4}{l}{***$p<0.001$, **$p<0.01$, *$p<0.5$}                                                     
\end{tabular}
%\vspace{-0.4cm}
\end{table}

\afterpage{\FloatBarrier}

Table~\ref{tab:dd-regressions} reports the results of the DiD regressions.
They show that the interaction term is significant only for the \texttt{Starstruck} badge, on which we focus on the presentation of findings.
The DiD coefficient $\delta$ is significant and different from zero (0.169); this means that the introduction of the new feature of the \texttt{Starstruck} badge, which counts the number of starred repositories of a developer, caused an increase in the average number of followers; specifically, because the regression is in logs, the average count of followers in the post-treatment time window is 18.4\% (i.e., $(e^{coef} -1 )\times 100$) higher than it would have been without the introduction of the \texttt{Starstruck} badge in \gh~\citep{cameron2005}.
In addition, %the intercept coefficient $\alpha$ (the constant) is significant and different from zero; this indicates that developers in the control group have an average number of followers of 3.618 in the pre-treatment time window.
we observe that the coefficient $\beta$ (treatment group) is significant and different from zero, which means that in the pre-treatment time window the developers in the treatment and control groups had a different number of followers; specifically, developers with the badge had on average 299.9\% more followers than those without. % and the average number of followers in the treatment group was 5.004 ($\alpha+\beta$).
Finally, the coefficient $\gamma$ is also significant and different from zero (-0.1), meaning that the average number of followers in the control group from the pre-treatment to the post-treatment time window decreased by 9.5\%. % to 3.518 ($\alpha+\gamma$).
%Finally, based on these results, we can calculate the counterfactual ($\alpha + \beta + \gamma = 4.904$), which represents the average number in logs of followers in the treatment group, had the introduction of the \texttt{Starstruck} badge never happened; however, because the treatment did occur, and the actual average number of followers in logs is equal to 5.073 (the counterfactual  $+ \delta$).

%%%%%%%%%%%%%%%%%%%%%%%%%%%%%%%%%%%%%%%%%%%%     RQ4   %%%%%%%%%%%%%%%%%%%%%%%%%%%%%%%%%%%%%%%%%%%%%%%% 


\subsection{RQ$_4$ -- Community Perception of Personal Badges}\label{sec:res-rq4}

\subsubsection{Survey analysis}\label{sec:res-rq4:survey}
We received 57 responses.
Survey analysis was carried out by the first two authors. 
They analyzed the survey results reading and annotating the responses in a spreadsheet to extract interesting excerpts from open-ended questions and highlight commonalities.
After that, the entire team discussed and consolidated the extracted excerpts.
 
Regarding the demographics (Q$_{1}$-Q$_{4}$), participants in the survey reported having on average 9 years of experience (min. 1, max. 31). 
The participants also maintain an average of 6 open source software repositories (min. 0, max. 18) and have contributed to \tildex{22} repositories (min. 1, max. \tildex{100}).
Finally, they reported having an average of \tildex{28} followers (min. 0, max. \tildex{100}).
Three respondents were disqualified because they reported (Q$_{5}$) that they had opted out of displaying badges (1) or did not have a public profile page (2). 
They motivated their choices (Q$_{15}$) by saying that they ``\textit{simply don't like the idea of badges}'' (P$_{22}$) and that \gh\ ``\textit{may be a social coding site but it's not (or shouldn't be or become) like social media}'' (P$_{33}$).
The other users who reported keeping the badge feature enabled (Q$_{6}$, $n=54$), cited various motivations for doing so. The primary reason was curiosity (19 respondents), followed by having observed badges on other developers' profiles (18 respondents), the aesthetic appeal (12), as a suggestion by fellow \gh\ users (3), and in recognition of the feature's utility in conveying valuable information (2). 
In the following, we analyze the remaining valid responses.

Question Q$_{7}$ ($n=53$, see Figure~\ref{fig:survey_q7-10}) aimed to assess whether participants consider the presence of personal badges an indicator of their own coding skills in general.
Most of the respondents either strongly disagreed ($n=24)$ or disagreed ($n=8)$.
Then, questions Q$_8$-Q$_{10}$ ($n=53$) asked participants to report on the effects that badges may have on collaboration with other developers.
The results are also reported in Figure~\ref{fig:survey_q7-10}.
Although the distribution of responses is varied, overall the figure shows that about half of the respondents feel that badges can be neither an indicator of others' ability nor are they influenced by their presence during collaboration with others.
Question Q$_{11}$ asked the respondents to justify their previous responses. 
They were critical of the current implementation of badges. 
Participant P$_7$ found that ``\textit{these badges can be achieved too easily, not like in Stack Overflow},'' and P$_{2}$ that there are too few of them (``\textit{we all have almost the same badges}''); others added that they fail to adequately capture developers' experience -- e.g., ``\textit{I have 30+ years of development experience and almost none is visible through [badges]}'' (P$_{1}$), ``\textit{I have lots of experience and almost no badges}'' (P$_{8}$).
Furthermore, P$_5$ answered with a provocative question: ``\textit{Are developers with these badges better developers? Vice versa and more critical: is a developer bad just because they don't have some of these badges?}''

\begin{figure}
    \centering
    %\vspace{-0.2cm}
    \includegraphics[height=4.2cm, width=11cm]{img/results/rq4_q7-10.png}
    %\vspace{-1.5cm} 
    \caption{The extent to which participants find personal badges to be an indicator of coding skills (Q$_{7}$) and their perceived effects on collaboration (Q$_{8}$-Q$_{10}$).}
    %\vspace{-0.3cm}
    \label{fig:survey_q7-10}
\end{figure}

\afterpage{\FloatBarrier}

Next, we examine how respondents feel about specific badges.
Question Q$_{12}$ ($n=41$) asked participants to indicate the most relevant badges.
The answers are clustered around two main groups.
The first group contains answers (15) from users who feel that none of these badges can tell us anything useful about others.
The other group of remaining answers (26) instead believes that the badges give information about who unlocks them.
Specifically, 6 respondents mentioned the badge \textsc{Galaxy Brain} (awarded to those who have accepted answers in project discussions) because it reflects ``\textit{how serious [one is] in the community''} (P$_{28}$).
A group of 11 participants mentioned the \textsc{Starstruck} badge (unlocked by receiving project stars) because it gives tangible ``\textit{evidence of acknowledgment and interest by the community}'' (P$_{17}$).
The remaining respondents (9) highlighted the importance of \textsc{Pair Extraordinaire} (awarded to those who coauthor merged pull requests) because the badge may reflect ``\textit{a positive attitude [toward] collaboration with others''} (P$_{19}$).


Finally, question Q$_{13}$ ($n=42$) asked participants to indicate any missing badge, in addition to those available.
The suggestions converged on the following ideas for new badges that would display: i) the years of work in a project, ii) how many repositories you are a maintainer of, iii) how many repositories you have contributed to, iv) more badges in general, and v) more badges as long as they are challenging to unlock.
%Finally, the last two questions asked whether the participants noticed any differences when interacting with others after displaying any personal badge (Q$_{16}$) and any difference in developers' practices or ability depending on the presence of personal achievement badges on their profile page(Q$_{17}$). They strongly disagreed.


\subsubsection{Community discussions analysis}

Overall, we mined all the Q\&A threads related to personal badges from June (i.e., the launch of the new feature) to December 2022. 
We retrieved 312 questions, containing 937 answers and 781 replies. 
Subsequently, we performed a thematic analysis of the questions we had retrieved.

To carry out the thematic analysis, we divided the entire set of 312 posts into subsets of about 100 elements; then, the first two authors independently analyzed the first subset.
We identified common and different categories. 
The differences were resolved through discussions with the entire team. After completing the analysis of the first subset, we obtained an initial coding schema, which was applied independently to the second subset of posts. 
A similar approach was followed to resolve differences and update the coding schema with new or refined categories. 
When necessary, the changes were propagated back to the previous subset. 
After resolving all the differences, we moved on to the third (final) subset, replicating the same approach. 
We used \texttt{Taguette}\footnote{\url{https://www.taguette.org}} to support the annotation process and a spreadsheet to define the coding schema.

Thematic analysis revealed four main themes (see Table~\ref{tab:badge_discussions}): \emph{Info requests}, \emph{Feedback}, \emph{Improvements}, and \emph{Other}. 
In the following paragraphs, we describe each theme and the corresponding codes, exemplifying the most relevant concepts with excerpts from the discussion threads.

The prevalent theme that emerged from our analysis is the request for information about personal badges -- namely, \emph{Info requests}. 
This theme encompasses 122 distinct questions, a number that doubles those recorded for the next two main themes, i.e., \emph{Improvements} and \emph{Feedback}. 
Arguably, the prevalence of questions from the theme \emph{Info requests} can be explained by the choice of \gh\ not to disclose explicit information on personal badges; this choice has led several users of the platform to seek unofficial information on the community forum.
Within this theme, the most frequent code is \emph{How to get a badge}, assigned to 79 questions regarding the requirements needed to earn a specific badge (e.g., \emph{``How do you get the Galaxy Brain Badge?''}). 
Similarly, 17 more questions were coded as \emph{How to get badges}; these represent requests for help on how to earn more badges in general (\emph{``How to get a new badge on GitHub?''}). 
The remaining 26 questions from this theme were coded as \emph{General info}; these are generic requests for information about the badges feature, the most common being the full list of the currently available personal badges in \gh\ (e.g., \emph{``How can I see all available badges in GitHub? Please Help!!''}).

The next major theme identified is the \emph{Improvements} category, which includes 60 questions and 2 codes: \emph{Badge(s) proposals} and \emph{UX suggestions}.
Under \emph{Badge(s) proposals}, we collected 22 questions that represent requests for the addition of new personal badges. 
For instance, three users recommended adding an additional badge to reward code-review activity (e.g., \emph{``I've learned a lot from code reviewing others' code and being code reviewed by others''}); %{[}\ldots{]} So I think it's fair to have one {[}badge{]} for the code review as well''}; 
furthermore, three other users proposed the addition of organization-specific badges (e.g., \emph{``to celebrate the first PR merge of a new employee''}). 
Some users even suggested creative names for their new badge ideas, such as the \emph{``Issue Muncher''} badge---keeping track of \emph{``how many issues one has helped to close through linked PRs''}---and the \emph{``Bug Hunter''} badge---rewarding reports of bugs in \gh. 
Also, a user suggested that the personal badge feature be extended to showcase actual professional achievements, such as the Linux Foundation Certified Kubernetes Administrator certificate or the Google Certification Academy certificate. 
Finally, a couple of more questions with this code were concerned about how to suggest new types of badge, showing that some users are willing to participate in the definition of future personal achievements.
Regarding the \emph{UX suggestions} code, we identified 40 threads. 
In particular, 21 questions concerned a feature request, the most popular (with 8 questions) being the possibility to selectively hide/show the earned badges on user profile pages; other recurring feature requests are: enabling custom badge ordering in profiles (2), taking historical data into account when defining badge unlocking rules (2), and enabling a summary view displaying all possible achievements (2). Another common type of UX-related question (11 questions) represents negative feedback on the graphical appearance of badges (e.g., \emph{``These new badges are too cartoonish''}, \emph{``The designs could be more professional''}) or suggestions on how to improve the display of badges (e.g., \emph{``Make the old badges look like the new ones.''})

\begin{table}[t]
    \centering
    \caption{Results from the thematic analysis of questions.}
    \label{tab:badge_discussions}
    \footnotesize
    \begin{tabular}{clc}
    \hline
    \textbf{Category} & \textbf{Code} & \textbf{Frequency} \\
    \hline
    \addlinespace
    \multirow{3}{*}{\makecell{\textbf{\textit{Info requests}}\\ (122)}} & General info & 26 \\
    & How to get badges & 17 \\
    & How to get a badge & 79 \\
    \addlinespace
    \multirow{3}{*}{\makecell{\textbf{\textit{Feedback}}\\ (53)}} & Positive & 12 \\
    & Negative (in general) & 25 \\
    & Negative (badge-specific) & 16 \\
    \addlinespace
    \multirow{2}{*}{\makecell{\textbf{\textit{Improvements}}\\ (60)}} & Badge(s) proposal & 23 \\
    & UX suggestion & 37 \\
    \addlinespace
    \multirow{2}{*}{\makecell{\textbf{\textit{Other}}\\ (77)}} & Technical issue & 42 \\
    & Miscellaneous & 4 \\
    & Excluded & 31 \\
    \addlinespace
    \hline
    \end{tabular}
    \vspace{-0.4cm}
\end{table}

\afterpage{\FloatBarrier}

The most significant theme in relation to RQ4 is \emph{Feedback}, encompassing the codes \emph{Positive}, \emph{Negative (in general)}, and \emph{Negative (specific)}, for a total of 53 questions. 
Although some questions (12) report positive feedback, in the form of short and generic statements of appreciation for the feature of personal badges, most of the questions collected within this category were coded as negative feedback, either generic (25 questions) or feature-specific (16 questions). 
Regarding generic negative feedback, several users reported disregarding the new feature, and 17 out of 25 asked how to opt out (or declared their intention to do so).
Frequent causes for the negative feedback were: i) aversion to the gamification of a professional environment like \gh\ and the \emph{``childish''} style %and ``trivial'' nature 
of badges -- e.g., \emph{``[they] look like Candy Crash Saga. These weird cartoon emoji are massive, and distract the eye from actual content''} %\emph{``Pull Shark? That's pretty patronizing. I do this for a living.''}; \emph{``{[}\ldots{]} turning GitHub into a social network style site is a terrible idea''} 
(11 questions); ii) skepticism about the ability of personal badges to convey developers' skills -- e.g.,  \emph{``unnecessary gamification that doesn't help assess the performance/dedication/talent/achievement of a developer''} (2 questions); iii) concerns about the potential negative effects of badges on user behaviors -- e.g., \emph{``PRs created not for the purpose of making a meaningful contribution, but simply to get another badge checked off their list.''}, \emph{``the achievements system sets up terrible incentives for folks to create PRs, not for the purpose of making a meaningful contribution, but simply to get another badge checked off their list.''} -- and the whole community 
-- e.g., \emph{``{[}Badges{]} risk becoming a driver for ivory tower superiority and community tribalism, which is something the developer community already struggles with''}, \emph{``I'm not listening to your feature request because I have badges and you don't''} (5 questions).

Regarding badge-specific negative feedback, most complaints related to the badge \texttt{YOLO} (7 questions), which is perceived by several developers as a source of shame and an unfair mark to obtain, especially if earned within single-person projects (e.g., \emph{``I feel having the YOLO badge does not send the right message, especially to recruiters''}). Another common criticism (4 questions) concerned the unclear requirements for unlocking the \texttt{Pull Shark} badge, sometimes perceived as biased (e.g., \emph{``Pull Shark achievement should work {[}only{]} for repositories I don't own. {[}\ldots{]} we will always merge our own pull requests, but not everyone will merge our pull request and that's the point''}). % 

Finally, the theme \emph{Other} gathered 77 questions, mostly coded as requests of support for a \emph{Technical issue} (e.g., \emph{%``Pull Shark details not correct {[}\ldots{]} 
Pull Shark shows the wrong merged PR number for me''}), and 4 questions coded as \emph{Miscellaneous}.
In addition, we excluded 31 questions when their content was deemed off-topic or when the text body was empty, not in English, or incomprehensible.



\section{Discussion}\label{sec:discussion}

In this paper, we studied personal achievement badges, a new gamification feature introduced in \gh, with unknown effects.

\textbf{Research questions}.
We answered four research questions.
First (RQ$_1$), we explored the diffusion of badges among a sample of 6,022 \gh\ developers as of December 2022 (Table~\ref{tab:badges_tiers}) and their evolution over six months, since their introduction in June 2022 (Figure~\ref{fig:plotrq1}).
We found that all developers except one own at least one badge and that the number of those who opted out of displaying badges and chose to make their profile pages private has been steadily increasing.
Then (RQ$_2$), we investigated whether owning badges is associated with signaling certain hypothesized developers' qualities and dispositions. 
The initial results of our exploratory analysis (Figure~\ref{fig:rq2_violinsplots}) showed only a large difference for developers owning the badge \texttt{Starstruck}, gained by those who get their repositories starred and a signal of increased popularity.
Furthermore, the regression analysis (RQ$_3$) showed that the only effect of implementing personal badges in \gh\ has been a large increase in the number of followers of developers who own the \texttt{Starstruck} badge (Table~\ref{tab:dd-regressions}).
Finally (RQ$_4$), the analysis of the survey responses showed several shortcomings in the current implementation of badges, such as the limited types of badges currently existing and the lack of those more accurately reflecting contributions to projects and years of experience.
Besides, the analysis of community discussions about personal badges revealed that \gh\ users are willing to know more about badges and the requirements set out for earning them; also, part of the community is willing to participate in the definition of future achievements.
However, the discussions analyzed showed the clear prevalence of negative opinions on badges, which confirms the skepticism of several users about their ability to adequately convey the skills of a developer and the concerns of the community about the potential drawbacks of gamifying \gh.

\textbf{Badges as signals}.
The results of the user survey confirmed that \gh\ developers do look at profile pages and are aware of the elements therein---so much so that several even decided to opt out of displaying badges.
As such, these pages have the potential to act as hubs that make socially translucent signals visible and readily available, and support awareness of collaborators’ behavior~\citep{erickson2000}.
This result is consistent with those reported by \fullcite{shami2009} who used the signaling theory as a conceptual framework to investigate how users of online social platforms rely on digital artifacts for `people sensemaking,' i.e., use portions of profile pages as proxies used to infer unknown coworkers' expertise.

However, the visibility of personal badges is not sufficient to ensure their reliability.
According to signaling theory~\citep{donath2008}, to be reliable, assessment signals must be both observable and costly to produce.
Our findings suggest that most of the currently available personal badges, although they reflect the properties possessed, do not send reliable assessment signals because they may not be costly enough to produce. %; one possible explanation is that \gh\ users are not aware of the different tiers for the same badge.
The only exception to general unreliability is the \texttt{Starstruck} badge signaling popularity.
One potential explanation for this finding is that the \texttt{Starstruck} badge is awarded to a developer as a result of the interest of others (stars) in their repositories: creating a repository implies an effort broader in scope, if not harder, as compared to other badges, which require the developers to take one action (e.g., answer a question, merge a pull request).
This result is somewhat consistent with those reported by \fullcite{trockman2018} who found that popularity-related repository badges are more reliable as assessment signals.
One major difference with their work is that repository badges are \textit{chosen}---maintainers select what they intend to signal by adding them to README pages (e.g., the count of downloads badge to signal popularity, up-to-date dependency badge to signal the lack of known security risks).
Conversely, personal badges are a truer gamification element---they are unlocked and displayed on personal pages---and the only control developers have over them is choosing whether to opt out of showing them altogether.

\textbf{Badges as a gamification mechanism}.
%Compare to the papers discussing gamification frameworks for \se. 
%Because repository badges are not gained based on developers' activity, personal badges are a more proper type of gamification mechanism.
According to \fullcite{hunter2012}, users of a gamified environment go through a journey, from \textit{onboarding} to \textit{scaffolding} and, finally, \textit{mastery}.
As argued by~\fullcite{dalsasso2017}, implementing game mechanisms such as badges to gamify a software development environment is an iterative and far from trivial undertaking.
Our analyses suggest that the leveling process of personal badges in \gh\ needs to be adjusted, especially during the onboarding and mastery stages.
On the one hand, some junior developers among the survey respondents reported that it \textit{``felt nice}'' to discover the first badge on their profile.
However, the easier-to-achieve onboarding badges (i.e., \texttt{YOLO} and \texttt{Quickdraw}) are obtained through practices that are not to be encouraged in software development (i.e., closing issues and PRs quickly or without code review), and in the community discussions some users referred to these badges as ``\textit{shameful}'' and ``\textit{trivial}.''
Conversely, more senior developers argued that they feel that current badges are ``\textit{too easily achieved, not like in Stack Overflow}'' and complained about the lack of badges reflecting their years of experience and project contributions (\emph{``They don't give the impression that I'm a professional developer, but that I'm a novice coder who's just finished my first project.''}).

\textbf{Implications for \gh\ platform designers}.
The negative feedback collected from survey responses and discussion analysis highlight some criticalities in the current design of personal badges.
\gh\ platform designers should consider using the following insights to make informed decisions about future improvements. 

%consciousness %selectively disable % new features
In our study, we found an increasing number of developers (up to \tildex{23}\%) who made their profile page private or opted out of displaying personal badges.
Albeit neither the survey or the discussion analyses helped us to understand the motivations, one possible explanation is that \gh\ developers are conscious that the elements on their profile page may influence the processes of impression formation~\citep{marlow2013impression,singer2013} and people sensemaking~\citep{shami2009}.
For instance, more and more employers look at information gleaned from social networking sites during the hiring process to assess job applicants~\citep{careerbuilder-social-media-survey}, and their social media image has an influence on the chance of being hired~\citep{bohnert2010}.
However, \fullcite{marlow2013activity} found that, albeit  conscious, only few \gh\ developers engaged in efforts to clean up their profile page because deemed too costly. 
Our findings show that \gh\ developers may have grown even more conscious over time and that the decision to allow them to easily opt out of displaying personal badges is a step in the right direction. 
Nonetheless, \gh\ platform designers should  consider implementing the ability to selectively hide/show badges on user profile pages -- a popular feature requests mentioned by several survey participants; this might alleviate the discontent caused by badges like \texttt{YOLO} and \texttt{Quickdraw}, which mark and publicly expose undesirable and unprofessional user behaviors. 

In terms of usability, to meet user expectations, platform designers should also consider making the progress toward the achievement of badges more explicit, especially in the case of debated badge requirements (e.g., those of the \texttt{Pull Shark} badge), and graphic designers should also revise the appearance of badges to give them a more professional and consistent `look and feel.' % (e.g., inspired by the look of actual professional achievements).

Furthermore, the current implementation of badges in \gh\ appears underdeveloped.
First, as discussed earlier, the current number of badges and the implementation of the leveling system do not seem to fit well with the onboarding and mastery stages typical of a gamified environment. 
As such, we suggest platform designers consider adding more: i) \textit{onboarding badges}, that are appealing to `newcomers' (e.g., junior developers) without fostering bad software development practices such as no or quick code reviews; ii) \textit{mastery badges}, that reflect long-running, broad-in-scope achievements that are appealing to and feel `earned' by more experienced, senior developers.
Second, platform designers should consider rolling out more of them all at once rather than slowly, as some users complained that they ``\emph{all have almost the same badges}.''
\fullcite{sarma2016} developed Visual Resume, a tool that aggregates activity traces collected from \gh\ and \so\ to aid employers in the hiring process; they evaluated the tool with managers and other technical personnel and found that they prefer cues in aggregated form (e.g., activity summaries) and community-generated endorsements (e.g., reputation) to inform hiring decisions and form impressions. 
Their findings may provide platform designers with useful ideas to implement more reliable personal badges, as participants in that study expressed a desire to have quick access to features indicating whether commits are central to the code base and whether they make significant changes rather than tweak/fix code.
Third, platform designers should further reflect on the unintended side effects of the decision not to provide any official documentation regarding badges.
Our findings show that this decision implies that no clear indication is given of what cues the badges are intended to signal beyond their name and image, with a potential impact on impression formation and people sensemaking, for example, in collaborative development~\citep{marlow2013impression,singer2013}.
As of this writing, two more badges have been discovered, namely \texttt{Heart On Your Sleeve} and \texttt{Open Sourcerer} but the community still has not confirmed what they mean or how to achieve them.\footnote{\url{https://github.com/drknzz/GitHub-Achievements}}$^,$\footnote{\url{https://github.com/github-profile-achievements/english}}

%\textbf{Implications for recruiters and job seekers}. % reliability and cost of verification
Finally, we urge platform designers to reflect about the potential side effects that displaying personal badges may have on the hiring process, as revealed by our analysis of community discussions (\emph{``A potential employer glancing at my profile, perhaps unfamiliar with this very niche GitHub-specific feature, might assume a negative impression of me''}).
Albeit \gh\ is designed as a platform to support collaborative software development, employers and job seekers also use it  as a recruitment tool~\citep{capiluppi2013,greene2016}.
\fullcite{marlow2013activity} analyzed the signals given off by activity traces displayed on \gh\ profile pages through the lens of signaling theory.
After interviewing managers and developers who use \gh\ as part of the hiring and job application process, they found that interviewees valued the insights provided by \gh\ accounts as more reliable and verifiable than a static list of individuals' skills.
In fact, according to~\fullcite{walther2009}, when forming an impression, information provided by third parties (i.e., \gh\ in this case), is more reliable than self reports.
Nonetheless, \citeauthor{marlow2013activity} found that the reliability of the activity traces as signals varied and their use was directly related to the evaluation cost.
In particular, they found that activity traces send reliable assessment signals when they reflect (i) \textit{history of activity} (e.g., commitment to an open source project over years) and (ii) \textit{networking} (e.g., contributing commits to high-status OSS projects).
Still, employers only chose to look at those activity traces that are easy to verify quickly.
Hence, consistently with this evidence and the signaling theory, we suggest that platform designers overhaul personal badges so that they provide quick access to aggregated traces of activity history and collaboration in high-status projects, which are costly to produce for job seekers and, at the same time, affordable for employers and fellow developers to verify.
%Or, to put into the words of John Resig, the creator of JQuery, ``\textit{when it comes to hiring, I'll take a GitHub commit log over a resume any day}.''\footnote{\url{https://twitter.com/jeresig/status/33968704983138304}}
%\so\ has become a platform for recruitment.

\section{Limitations}\label{sec:limitations}
First, in this study, we analyzed the activities of a sample of over 6,000 \gh\ developers retrieved from 51 organizations.
Because \gh\ currently hosts a steadily increasing number of OSS projects that are the result of the contributions of millions of developers,\footnote{\url{https://octoverse.github.com/#the-world-of-open-source}} we acknowledge the potential threat to external validity.
However, we emphasize that our study is the first to analyze personal badges; it was initiated in a timely way as soon as the new feature was released in June 2022 and completed within six months after its release. 
As such, we argue that the novelty and timeliness of this work compensate for the limited number of sampled developers.
Furthermore, we point out that this is a study that compares the activity and profiles of developers \textit{before} and \textit{after} the introduction of the features of personal badges and therefore it was possible to conduct it only because we already had the necessary data collected for the \textit{before} stage as part of a previous study. 
In other words, without the already available data regarding the \tildex{6k} developers -- limited or not -- this study would simply not have been possible. 
We also point out that the sampled developers contribute to large and active software-engineered  projects that were carefully mined from \gh\ following the recommendations provided in~\citep{Kalliamvakou2014}.
Finally, given the exploratory nature of our study, we argue that the current findings help us provide an initial understanding of both the effects of badges and the developers’ perception of the new gamification element introduced in \gh. 
By making the complete replication package available, we encourage the community to further our understanding of personal badges through replications and extensions of this study.

We also acknowledge the somewhat limited number of survey participants (54). 
However, to counteract this limitation and properly answer RQ4 (gauge community perception of personal badges), we complemented the survey analysis with the thematic analysis of over 300 discussions where \gh\ community members shared their opinions about the new feature.

Another limitation is that in our quantitative analyses (RQ1-3) we only analyzed the activities of sampled developers in public repositories. Developers can choose (i.e., not the default setting) to also include their activities in private repositories as contributing to unlocking personal achievements. 
However,this is a piece of information that is impossible to retrieve and therefore represents an intrinsic limitation that our study shares with all the works that rely on data retrieved from public \gh\ repositories.


Finally, \gh\ provides no official information about personal badges and how to unlock them. 
As such, all descriptions provided in this study regarding badges have been inferred by \gh\ community members and contributed to public repositories.
One potential limitation affecting the correlation and regression analyses is that we hypothesized the cues that personal badges might send as signals based on the perceptions reported by survey participants in their responses.
Albeit to build the hypotheses we considered only the perceived signals that were reported by two or more respondents, alternative hypotheses not included in the DiD regression analysis are still possible.
For example, based on survey responses, in the study we hypothesized that displaying the \texttt{Quickdraw} badge might be a cue of timeliness.
Instead, one might also read the very short time taken to close an issue or a pull request as a negative indication that the developer might not always perform thorough code reviews.
Still, we notice that the potential `arbitrariness' of the cues sent by badges is an intrinsic limitation of the current implementation of the available personal badges, which lack official documentation.

%Finally, also related to the limited significant results obtained from the correlation and regression analyses, we acknowledge that this could be the effect of aggregating all the badge tiers (see Table~\ref{tab:badges_tiers}) and conducting the analyses distinguishing only between developers with and without badges.
%Therefore, it is possible that the tier-3 badges, which are harder to achieve, might send more reliable signals than the tier-1 badges of the same type.


\section{Conclusions}\label{sec:conclusions}
In this paper, we analyzed the introduction of a new gamification feature in \gh, namely personal achievement badges.
We studied the distribution of badges, the signals they send, and the effects of their introduction.
We also collected evidence of the reaction of the developers' community to the new feature.
We concluded by providing recommendations to \gh\ platform designers on how to improve personal badges as a gamification mechanism and as reliable cues of developers’ ability.
As the implementation of the feature appears to be in the initial stage, we will consider furthering this study as changes to badges are hopefully introduced in \gh.

\section*{Acknowledgment}
This research was co-funded by projects DARE (PNC0000002, CUP: B53C22006420001), FAIR (PE00000013, CUP: H97G22000210007), SERICS (PE0000014, CUP: H93C22000620001), and QualAI (PRIN 2022, CUP: H53D23003510006).
The authors thank the anonymous respondents to the survey and Prof. Claudia Capozza for her guidance and feedback on regression modeling.

%\bibliographystyle{IEEEtranN}
\bibliographystyle{elsarticle-num-names}
\bibliography{biblio}

\newpage
\appendix

\section{Sampled Projects}\label{appendix:projects}
\pgfplotstableread[col sep=semicolon]{tables/projects.csv}\datatable

\begin{table*}
    \centering
    \caption{Breakdown of sampled projects' characteristics. All values have been retrieved using the \gh\ API, except for LOC (calculated using the tool \texttt{tokei}).}
    \label{tab:projects}
    \tiny
    \pgfplotstabletypeset[
        string type,
        columns/Repository/.style={column name=\textbf{Repository}, column type={l}},
        columns/Main language/.style={column name=\textbf{Main language}, column type={l}},
        columns/Age/.style={column name=\textbf{Age}, column type={c}},
        columns/Contributors/.style={column name=\textbf{Contributors}, column type={c}},
        columns/LOC/.style={column name=\textbf{LOC}, column type={c}},
        columns/PRs/.style={column name=\textbf{PRs}, column type={c}},
        columns/Stars/.style={column name=\textbf{Stars}, column type={c}},
        every head row/.style={before row=\toprule, after row=\midrule},
        every last row/.style={after row=\bottomrule},
    ]{\datatable}
\end{table*}

\afterpage{\FloatBarrier}

\newpage
\section{Survey: How do GitHub users feel about Personal Badges?}\label{appendix:questionnaire}

\footnotesize
Greetings! Thank you for taking an interest in our survey. We understand
how precious your time is, and that is why we made sure this survey would
only take less than 5 minutes. Do not worry, your data will be collected
just for research purposes: it will be anonymously stored, and once
analyzed, it will be reported in aggregated form. After that, it will be
deleted. We promise!

Thanks again for your help!

\hfill \break

\textbf{Background information}

\emph{Tell us something about you.}

\begin{enumerate}
\def\labelenumi{\arabic{enumi}.}
\tightlist
\item
  How many years of experience do you have in open source software?
  \_\_\_\_\_\_\_\_\_\_\_\_\_\_\_\_\_\_\_
\item
  How many OSS projects are you a maintainer of?
  \_\_\_\_\_\_\_\_\_\_\_\_\_\_\_\_\_\_\_\_\_\_\_\_\_\_\_\_\_\_\_\_\_\_\_\_\_\_\_\_\_
\item
  How many OSS projects have you contributed to?
  \_\_\_\_\_\_\_\_\_\_\_\_\_\_\_\_\_\_\_\_\_\_\_\_\_\_\_\_\_\_\_\_\_\_\_\_\_\_\_\_
\item
  How many followers do you have?  \_\_\_\_\_\_\_\_\_\_\_\_\_\_\_\_\_\_\_\_\_\_\_\_\_\_\_\_\_\_\_\_\_\_\_\_\_\_\_\_\_\_\_\_\_\_\_\_\_\_\_\_\_\_\_\_\_\_\_
\item
  Do you have a public profile, and have you kept enabled the default
  feature of displaying personal achievement badges on your profile
  page?
\end{enumerate}

\emph{Mark only one oval.}

\begin{itemize}
\tightlist
\item
  Yes
\item
  No
\end{itemize}

\hfill \break

\textbf{Badges in general}

\begin{enumerate}
\def\labelenumi{\arabic{enumi}.}
\setcounter{enumi}{5}
\tightlist
\item
  What influenced your decision to display personal achievement badges
  on your profile page? 
\end{enumerate}

\emph{Check all that apply.}

\begin{itemize}
\tightlist
\item
  They look nice
\item
  They display useful information
\item
  I was inspired by other GitHub users
\item
  I was curious about trying out a new feature
\item
  It was suggested to me by other GitHub users
\item
  Other:
  \_\_\_\_\_\_\_\_\_\_\_\_\_\_\_\_\_\_\_\_\_\_\_\_\_\_\_\_\_\_\_\_\_\_\_\_\_\_\_\_\_\_\_\_\_\_\_
\end{itemize}

\vspace{1mm}
\begin{enumerate}
\def\labelenumi{\arabic{enumi}.}
\setcounter{enumi}{6}
\tightlist
\item
  Do you consider the presence of personal achievement badges in general
  to be an indicator of your own coding skills?
\end{enumerate}

\emph{Mark only one oval.}

\begin{itemize}
\tightlist
\item
  1 Not really
\item
  2
\item
  3
\item
  4
\item
  5 Very much
\end{itemize}

\hfill \break

\textbf{Badges and collaboration}

\begin{enumerate}
\def\labelenumi{\arabic{enumi}.}
\setcounter{enumi}{7}
\tightlist
\item
  When looking for someone to collaborate with, I notice personal
  achievement badges displayed on their profile page
\end{enumerate}

\emph{Mark only one oval.}

\begin{itemize}
\tightlist
\item
  1 Not really
\item
  2
\item
  3
\item
  4
\item
  5 Very much
\end{itemize}

\vspace{1mm}
\begin{enumerate}
\def\labelenumi{\arabic{enumi}.}
\setcounter{enumi}{8}
\tightlist
\item
  I consider the presence of personal achievement badges in general to
  be an indicator of a developer's ability
\end{enumerate}

\emph{Mark only one oval.}

\begin{itemize}
\tightlist
\item
  1 Not really
\item
  2
\item
  3
\item
  4
\item
  5 Very much
\end{itemize}

\vspace{1mm}
\begin{enumerate}
\def\labelenumi{\arabic{enumi}.}
\setcounter{enumi}{9}
\tightlist
\item
  The presence of personal achievement badges influenced my decision to
  collaborate with someone
\end{enumerate}

\emph{Mark only one oval.}

\begin{itemize}
\tightlist
\item
  1 Not really
\item
  2
\item
  3
\item
  4
\item
  5 Very much
\end{itemize}

\vspace{1mm}
\begin{enumerate}
\def\labelenumi{\arabic{enumi}.}
\setcounter{enumi}{10}
\tightlist
\item
  Please explain your previous answer:
  \_\_\_\_\_\_\_\_\_\_\_\_\_\_\_\_\_\_\_\_\_\_\_\_\_\_\_\_\_\_\_\_\_\_\_\_\_\_\_\_\_\_\_\_\_\_\_\_\_\_\_\_\_\_\_\_
\end{enumerate}

\hfill \break

\textbf{Specific Badges}

Below are the personal achievement badges discovered and unlocked so far
by the GitHub community members.

\vspace{1mm}
\begin{enumerate}
\def\labelenumi{\arabic{enumi}.}
\setcounter{enumi}{11}
%\item
%  Which personal achievement badges (displayed above) have you unlocked so far? 
%  %\_\_\_\_\_\_\_\_\_\_\_\_\_\_\_\_\_\_\_\_\_\_\_\_\_\_\_\_\_\_\_\_\_\_\_\_\_\_\_\_\_\_\_\_\_\_
\item
  What personal achievement badges (displayed above) do you find most important? And what do they tell you about a GitHub user?
  \_\_\_\_\_\_\_\_\_\_\_\_\_\_\_\_\_\_\_\_\_\_\_\_\_\_\_\_\_\_\_\_\_\_\_\_\_\_\_\_\_\_
%\item
%  What effects do you expect each personal achievement badge to have on your collaborations, collaborators, and projects?
%  %\_\_\_\_\_\_\_\_\_\_\_\_\_\_\_\_\_\_\_\_\_\_\_\_\_\_\_\_\_\_\_\_\_\_\_\_\_\_\_\_\_\_\_\_\_\_
\item
  Are there any personal achievement badges that you feel are missing from those displayed above? What do you think these missing personal achievement badges should reflect?  
  \_\_\_\_\_\_\_\_\_\_\_\_\_\_\_\_\_\_\_\_\_\_\_\_\_\_\_\_\_\_\_\_\_\_\_\_\_\_\_\_\_\_\_\_\_\_\_\_\_\_\_\_\_\_\_\_\_\_\_\_\_\_\_\_\_\_\_\_\_\_\_\_\_\_\_\_\_\_\_\_\_\_\_\_\_\_\_\_\_\_\_
\item
  For each badge in the above picture, report the intended `signal' that you think it sends to others and, if different, the actual `signal' it
  conveys to people who visualize it.  \_\_\_\_\_\_\_\_\_\_\_\_\_\_\_\_\_\_\_\_\_\_\_\_\_\_\_\_\_\_\_\_\_\_\_\_\_\_\_\_\_\_\_\_\_\_\_\_\_\_\_\_\_\_\_\_\_\_\_\_\_\_\_\_\_\_\_\_\_\_\_\_\_\_\_\_\_\_\_\_\_\_\_\_\_\_\_\_\_\_\_\_\_\_\_\_\_
\end{enumerate}

\hfill \break

\textbf{Disqualification}

\begin{enumerate}
\def\labelenumi{\arabic{enumi}.}
\setcounter{enumi}{14}
\tightlist
\item
  Please, tell us why you decided to keep your profile page private; or,
  if public, why you specifically opted out of having personal
  achievements badges displayed on your profile.  
  \_\_\_\_\_\_\_\_\_\_\_\_\_\_\_\_\_\_\_\_\_\_\_\_\_\_\_\_\_\_\_\_\_\_\_\_\_\_\_\_\_\_\_\_\_\_\_\_\_\_\_\_\_\_\_\_\_\_\_\_\_\_\_\_\_\_\_\_\_\_\_\_\_\_\_\_\_\_\_\_\_\_\_\_\_
\end{enumerate}

\hfill \break

\textbf{Thank you, you answered all the questions!}

We will delete this information as soon as the study is over and will
never share it with anyone.

\section{Boxplots}\label{appendix:did}

Through the boxplots below, we visually confirm the presence/absence of the parallel (or common) for applying a DiD regression, that is, in the absence of treatment (i.e., before the introduction of personal badges), the dependent variable trend would be the same in both the treatment group (i.e., the developers who will eventually unlock a given type of badge) and the control group (i.e.,
the developers without the badges). 


\begin{figure*}[ht]
    \vspace{-1cm}
     \centering
        \begin{subfigure}[b]{0.45\textwidth}
         \centering
         \includegraphics[width=\textwidth]{img/results/rq3_pair-extraordinaire_issues_prs_opened_grouped_boxplot.png}
         \vspace{-0.8cm}
         \caption{Pair Extraord. (\# Issues \& PRs opened)}
     \end{subfigure}
     \begin{subfigure}[b]{0.45\textwidth}
         \centering
         \includegraphics[width=\textwidth]{img/results/rq3_pair-extraordinaire_authored_commits_grouped_boxplot.png}
         \vspace{-0.8cm}
         \caption{Pair Extraord.(\# authored commits)}
     \end{subfigure}
     \begin{subfigure}[b]{0.45\textwidth}
         \centering
         \includegraphics[width=\textwidth]{img/results/rq3_quickdraw_time_close_issues_prs_grouped_boxplot.png}
         \vspace{-0.8cm}
         \caption{Quickdraw (\# authored commits)}
     \end{subfigure}
     \begin{subfigure}[b]{0.45\textwidth}
         \centering
         \includegraphics[width=\textwidth]{img/results/rq3_starstruck_followers_grouped_boxplot.png}
         \vspace{-0.8cm}
         \caption{Starstruck (\# followers)}
     \end{subfigure}
     \begin{subfigure}[b]{0.45\textwidth}
         \centering
         \includegraphics[width=\textwidth]{img/results/rq3_galaxy-brain_commits_grouped_boxplot.png}
         \vspace{-0.8cm}
         \caption{Galaxy Brain (\# Issues \& PRs opened)}
     \end{subfigure}
     \begin{subfigure}[b]{0.45\textwidth}
         \centering
         \includegraphics[width=\textwidth]{img/results/rq3_galaxy-brain_commits_grouped_boxplot.png}
         \vspace{-0.8cm}
         \caption{Pull Shark (\# authored commits)}
     \end{subfigure}
     \begin{subfigure}[b]{0.45\textwidth}
         \centering
         \includegraphics[width=\textwidth]{img/results/rq3_galaxy-brain_commits_grouped_boxplot.png}
         \vspace{-0.8cm}
         \caption{Pull Shark (\# committed)}
     \end{subfigure}
     \caption{Visual confirmations and violations of the parallel (common) trends assumption before the introduction of personal badges.}
\end{figure*}



\end{document}
