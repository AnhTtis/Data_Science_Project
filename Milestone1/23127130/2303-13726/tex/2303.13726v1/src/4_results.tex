\section{Results}\label{sec:results}
In this section, we first show the impact of considering the robot's full-body dynamics when selecting footstep placements and contact surfaces (\sref{sec:frbd_for_footstep_planning}).
This demonstrates the benefits of our approach compared to other state-of-the-art methods that use simplified models.
We then validate our approach in an~\gls{mpc} scheme on the ANYmal robot (\sref{sec:robot_experiment}).
Finally, we showcase the potential highly-dynamic maneuvers that our approach can generate by exploiting limb dynamics in simulations (\sref{sec:potential_dynamic_motion_with_high_torque}).

\begin{figure}[b!]
    \centering\begin{tabular}{cc}
    \rowname{a} & {\raisebox{-.5\height}{\includegraphics[width=0.93\columnwidth]{figs/visualization_normal.png}}}\\\\
    \rowname{b} & {\raisebox{-.5\height}{\includegraphics[width=0.93\columnwidth]{figs/visualization_torque_reduced.png}}}\\\\
    \rowname{c} & {\raisebox{-.5\height}{\includegraphics[width=0.93\columnwidth]{figs/visualization_surface_slippery.png}}}
    \end{tabular}
    \caption{
    Visualization of footstep placements and contact surfaces planned by our~\gls{mpc} when changing the joint torque limit and friction coefficient.
    The figures in the left and right columns of each row show the start and end of the crossing of the second foot, respectively.
    (a) ANYmal's default torque limits of \SI{40}{\newton\meter} and high friction coefficient of $1.0$.
    (b) Reduced torque limits of \SI{20}{\newton\meter}.
    (c) Low friction coefficient of $0.14$.
    When the torque limit is reduced in (b), our approach moves the footstep closer to the robot's hip and selects contact surfaces that require lower torque commands.
    Conversely, when the friction coefficient is low, our approach stretches the robot's leg perpendicular to the ground to maintain the reaction force within the small friction cone.
    }
    \label{fig:frbd_for_footstep_planning}
\end{figure}

\begin{figure}[!hbt]
    \centering
    \includegraphics[width=\linewidth]{figs/plot/torque_plot.pdf}
    \caption{
    Torque commands on the knee joints when the robot crosses the patches. 
    The red line represents the torque commands computed with the ANYmal's default torque limit of \SI{40}{\newton\meter}, while the blue line represents the torque commands computed with the reduced torque limit of \SI{20}{\newton\meter}.
    The black dashed line indicates the reduced torque limit.
    With the default torque limit, the torque command for the right front knee computed by our~\gls{mpc} reaches its peak at~\SI{2.7}{\second}.
    When the torque limit is lowered, our~\gls{mpc} changes the patch crossing moment, as shown in \fref{fig:frbd_for_footstep_planning}b.
    The command torque for the left front knee reaches the limit at~\SI{2.9}{\second}, but it is maintained within it during that gait.
    }
    \label{fig:torque_plot}
\end{figure}

\subsection{Footstep Planning with the Robot's Full-Body Dynamics}
\label{sec:frbd_for_footstep_planning}
As explained in \sref{sec:related_work}, neither~\gls{srbd} nor~\gls{cd} can enforce joint torque limits.
Furthermore, ~\gls{srbd} cannot account for limb dynamics and kinematics.

Here, we present results that justify our full-body dynamics~\gls{mpc} that is able to consider the robot's joint torque limits and limb dynamics when planning footsteps placements and contact surfaces.
To strictly focus on the independent variables (i.e., joint torque limits and friction coefficients), we set up the simulation as shown in~\fref{fig:frbd_for_footstep_planning}.

\subsubsection{Joint Torque Limits}\label{sec:joint_torque_limits}
We investigated the ability of the robot to adjust to changing torque limits.
The result will allow us to consider having the robot carry a payload (e.g., an arm), or perform more dynamic movements (e.g., crossing large gaps), without exceeding the robot's torque limit.
First, we used the ANYmal's default torque limits of \SI{40}{\newton\meter} to have the robot walk with a trotting gait on the footstep surface of two pallets with a gap of \SI{30}{\centi\meter}.
Next, we reduced the torque limits by half.
\fref{fig:frbd_for_footstep_planning}a and \ref{fig:frbd_for_footstep_planning}b show the selection of footstep placements and contact surfaces for the default and reduced torque limits, respectively.
These demonstrate that our~\gls{mpc} with the reduced torque limit positions the robot's legs closer together.
As a result, the joint torque commands are reduced and kept within the torque limit, as shown in~\fref{fig:torque_plot}.

\begin{figure*}[htp]
    \centering\begin{tabular}{cc}
    \rowname{a} &
     {\raisebox{-.5\height}{\includegraphics[width=0.965\textwidth]{figs/robot_experiment.png}}}\\\\
    \rowname{b} & {\raisebox{-.5\height}{\includegraphics[width=0.965\textwidth]{figs/simulation_stair_climbing_trot.png}}}\\\\
    \rowname{c} & {\raisebox{-.5\height}{\includegraphics[width=0.965\textwidth]{figs/simulation_stair_climbing_pace.png}}}\\\\
    \rowname{d} & {\raisebox{-.5\height}{\includegraphics[width=0.965\textwidth]{figs/simulation_dynamic_jump.png}}}
    \end{tabular}
    \caption{
    Snapshots of various locomotion maneuvers computed by our topology-based~\gls{mpc} with automatic footstep placement and contact surface selection.
    All experimental and simulation trials were conducted using an~\gls{mpc} with an optimization horizon of \SI{0.85}{\second} and a control frequency of \SI{50}{\hertz}.
    To further explore the potential for more dynamic maneuvers that could be achieved with high-torque actuators, we increased the torque limit for simulations.
    (a) Experimental validation of a gap-crossing maneuver with a gap of~\SI{10}{\centi\meter}.
    (b) Climbing up a staircase of \SI{30}{\centi\meter} depth and \SI{10}{\centi\meter} height with a trotting gait.
    (c) Climbing up the same staircase with a pacing gait.
    (d) Jumping over a gap of \SI{40}{\centi\meter}.
    }
    \label{fig:robot_experiment}
\end{figure*}

\subsubsection{Limb Dynamics}
We varied the friction coefficient to investigate how the robot leverages limb dynamics when determining footstep placements and contact surfaces.
We used the ANYmal's default torque limits and the same reference velocity used in \sref{sec:joint_torque_limits}.
As shown in~\fref{fig:frbd_for_footstep_planning}c, our approach stretches the robot's leg perpendicular to the ground, allowing it to maintain the contact forces within the friction cone (with a friction coefficient of $0.14$).

\subsection{Validation of MPC on the ANYmal Robot}\label{sec:robot_experiment}
We validated our~\gls{mpc} approach on the ANYmal robot.  
The robot did not perceive its surroundings, but it received the candidate contact surface information extracted from the predefined environment, as shown in \fref{fig:control_pipeline}.
The robot moved with a walking gait.
It selected the next contact surface and planned its footstep placement within it, considering the full-body dynamics and kinematics, as shown in \fref{fig:robot_experiment}a.
With our topology-based~\gls{mpc}, the ANYmal robot successfully traversed a discrete terrain with a gap of \SI{10}{\centi\meter}.
This demonstrates that our approach can operate fast enough in an~\gls{mpc} scheme on real robots, with an optimization horizon of \SI{0.85}{\second} and a control frequency of \SI{50}{\hertz}.

\subsection{Achieving Dynamic Locomotion}\label{sec:potential_dynamic_motion_with_high_torque}
We conducted simulations to further evaluate our~\gls{mpc} scheme in achieving highly dynamic quadrupedal locomotion by utilizing limb dynamics.
To showcase the potential of our approach in generating highly dynamic maneuvers with robots equipped with high-torque actuators in the future, we disabled the robot's torque limits.

\subsubsection{Stair Climbing}
We extracted the contact surfaces of a staircase with \SI{30}{\centi\meter} depth and \SI{10}{\centi\meter} height.
Our~\gls{mpc} approach selected footstep placements and contact surfaces to climb stairs in real time.
The approach demonstrated the ability to handle both trotting and pacing dynamics, as shown in~\fref{fig:robot_experiment}b and~\ref{fig:robot_experiment}c.

\subsubsection{Dynamic Jumping}
We tested an even more dynamic jumping gait on terrain with a \SI{40}{\centi\meter} gap.
\fref{fig:robot_experiment}d shows that our~\gls{mpc} can achieve highly dynamic motions and adjust the footstep placements and contact surfaces to land reliably on the next contact surface.

% Add more detail about why this is an important validation result:
% 1. Steps are different convex patches
% 2. The proposed cost function is only 2D which means that when combined with the other cost functions it can be used on non-coplanar surfaces.
% 3. Steps are often occluded so replanning foot position based on newly detected contact surfaces provides more robustness.
