\section{Potential Field and Winding Number in Contact Surfaces}\label{sec:methodology}
In this section, we explain in detail our novel approach to footstep placement and contact surface selection.
It borrows the concepts of electric potential (\sref{sec:electric_potential} and \ref{sec:potential_regions}) and winding number (\sref{sec:winding_number} and \ref{sec:winding_number_regions}) from classical electrostatics and topology, respectively.
These concepts enable us to design a contact-surface penalty function, allowing our~\gls{mpc} to choose optimal footstep placement and contact surface (\sref{sec:harmonic_cost}).

\subsection{Potential Field}\label{sec:electric_potential}
We begin with an introduction to the electric potential that arises from a charged particle.
This is also called a Coulomb potential and is defined as
\begin{equation}
    V_E(\mathbf{r}) = \frac{Q}{4\pi\varepsilon_0}\frac{1}{\left|\mathbf{r} - \mathbf{p}\right|},
\end{equation}
where $Q$ is the charge of the particle, $\varepsilon_0$ is the permittivity of vacuum, $\mathbf{r}$ is the point at which the potential is evaluated, and $\mathbf{p}$ is the point at which the charged particle is located.
Since the physical meaning of the electric potential is not relevant to the definition of our cost function, we drop the scaling factor $\frac{Q}{4\pi\varepsilon_0}$ and make the potential unitless.
Moreover, the potential measured across a closed curve $\boldsymbol{\gamma}(s)$ can be obtained by integrating the potential field:
\begin{equation}\label{eq:potential}
    V(\boldsymbol{\gamma}) = \int\frac{1}{\left|\boldsymbol{\gamma}(s) - \mathbf{p}\right|} ds.
\end{equation}

\subsection{Potential Field in Contact Surfaces}\label{sec:potential_regions}
We describe the contact surfaces as polygons, whose boundaries can be represented by a sequence of linear segments.
Similar to~\eref{eq:potential}, we calculate the potential measured across the $i$-th segment of the polygon as follows:
\begin{equation}\label{eq:potential_segment}
    p_i = \int_0^1\frac{1}{\left|\mathbf{a}_i + (\mathbf{b}_i-\mathbf{a}_i)s - \mathbf{\hat{p}}\right|} ds,
\end{equation}
where $\mathbf{a}_i, \mathbf{b}_i\in\mathbb{R}^2$ are the endpoints of the segment and $\mathbf{\hat{p}}\in\mathbb{R}^2$ is the position of the robot's foot in the \textit{horizontal plane}.
It has an analytical solution, and the resulting potential produced by each potential segment $p_i$ can be calculated as follows:
\begin{equation}\label{eq:potential_region}
    V_{\boldsymbol{\gamma}} = -\sum_i^{\boldsymbol{\gamma}}\frac{\atan(\frac{c_i}{e_i}) - \atan(\frac{d_i}{e_i})}{e_i},
\end{equation}
with\vspace{-1.5em}
\begin{align*}
c_i &= (\mathbf{a}_i\cdot\mathbf{b}_i) - (\mathbf{a}_i\cdot\mathbf{\hat{p}}) + (\mathbf{b}_i\cdot\mathbf{\hat{p}}) - (\mathbf{b}_i\cdot\mathbf{b}_i), \\
d_i &= - (\mathbf{a}_i\cdot\mathbf{b}_i) - (\mathbf{a}_i\cdot\mathbf{\hat{p}}) + (\mathbf{b}_i\cdot\mathbf{\hat{p}}) + (\mathbf{a}_i\cdot\mathbf{a}_i), \\
e_i &= [\mathbf{a}_i\times\mathbf{b}_i]_z - [\mathbf{a}_i\times\mathbf{\hat{p}}]_z + [\mathbf{b}_i\times\mathbf{\hat{p}}]_z,
\end{align*}
where $[\mathbf{a}\times \mathbf{b}]_z$ returns the $z$-coordinate of the cross product of the two vectors $\mathbf{a} = [a_x, a_y]$ and $\mathbf{b} = [b_x, b_y]$ in the XY plane (i.e., $a_x b_y - a_y b_x$).

\subsection{Winding Number}\label{sec:winding_number}
To represent discrete terrain as a topological space, we leverage the concept of \textit{winding number}.
The winding number is a measure of how many times a curve is wound around a point in the 2D plane.
There are different ways to define the winding number.
For instance, from the perspective of differential geometry, we can associate this number with the polar coordinates for a point at the origin ($\mathbf{\hat{p}=0}$), i.e.,
\begin{equation}
    \text{wind}(\boldsymbol{\gamma}, \mathbf{\hat{p}=0}) = \frac{1}{\pi}\oint_{\boldsymbol{\gamma}}\left(\frac{x}{r^2}dy + \frac{y}{r^2}dx\right),
\end{equation}
where $x, y$ defines the parametric equation of a continuous closed curve $\boldsymbol{\gamma}$, with $r^2 = x^2 + y^2$.

\subsection{Winding Number in Contact Surfaces}\label{sec:winding_number_regions}
As in~\sref{sec:potential_regions} and using the formula derived by~\cite{O'Rourke:1998:CGC:521378}, we calculate the winding number of the contact surface by summing over the $i$-th linear segments, as follows:

\begin{equation}\label{eq:winding}
    \text{wind}(\boldsymbol{\gamma}, \mathbf{\hat{p}}) = \sum_i^{\boldsymbol{\gamma}} \frac{\atantwo(c_i,d_i)}{2\pi},
\end{equation}
with\vspace{-1.5em}
\begin{align*}
    c_i &= [(\mathbf{a}_i-\mathbf{\hat{p}})\times(\mathbf{b}_i-\mathbf{\hat{p}})]_z, \\
    d_i &= (\mathbf{a}_i-\mathbf{\hat{p}})\cdot(\mathbf{b}_i-\mathbf{\hat{p}}). \nonumber 
\end{align*}
Then, we can determine if a point $\mathbf{\hat{p}}$ is inside the contact surface by checking if $\text{wind}(\boldsymbol{\gamma}, \mathbf{\hat{p}})\geq\frac{1}{2}$.

\subsection{Contact-Surface Penalty Function}\label{sec:harmonic_cost}
To enable the robot to select a placement $\mathbf{p}_{\mathcal{C}_k}$ (with $\mathcal{C}_k$ as each foot) within a contact surface, we assign a zero cost when the placement is inside the surface.
To achieve this, we use an indicator function based on the winding number to compute the \textit{contact-surface penalty} cost $\ell_{\mathbf{p}_{\mathcal{C}_k}}$ as follows:
\begin{align}\label{eq:indicator_function}
    \ell_{\boldsymbol{\mathbf{p}}_{\mathcal{C}_k}} = 
    \begin{cases}
        \quad 0 & \text{if } \sum_j V_{\boldsymbol{\gamma}_j} = 0, \\
        \sqrt{\frac{\sum_j N_j}{\sum_j V_{\boldsymbol{\gamma}_j}}} & \text{if } \sum_j \text{wind}(\boldsymbol{\gamma}_j,\mathbf{\hat{p}}_{\mathcal{C}_k})<\frac{1}{2}, \\
        \quad 0 & \text{if } \sum_j \text{wind}(\boldsymbol{\gamma}_j,\mathbf{\hat{p}}_{\mathcal{C}_k})\geq\frac{1}{2},
    \end{cases}
\end{align}
where $V_{\boldsymbol{\gamma}_j}$ is the potential, $N_j$ is the number of linear segments, and $\text{wind}(\boldsymbol{\gamma}_j,\mathbf{\hat{p}}_{\mathcal{C}_k})$ is the winding number obtained for the polygon $j$.
The potential and winding number are computed using~\eref{eq:potential_region} and~\eref{eq:winding}.
The analytical derivatives of this penalty function can be computed using the chain rule.

As shown in~\eref{eq:indicator_function}, we define the penalty function by using a square root (second line).
However, to avoid division by zero, we set the penalty function to zero when the potential is zero (first line).
This happens when $\mathbf{p}_{\mathcal{C}_k}$ lies on the boundary of the polygon. 
\fref{fig:harmonic_cost} illustrates our penalty function when we have two contact surfaces.

\begin{figure}[t]
    \centering
    \includegraphics[width=\linewidth]{figs/cost_function.png}
    \caption{
    Contact-surface penalty function for two contact surfaces.
    The cost is zero inside the two candidate contact surfaces outlined in black and positive elsewhere.
    To determine if the robot's feet are inside the contact surfaces, we compute the winding number.
    }
    \label{fig:harmonic_cost}
\end{figure}

Our penalty function is based on electric potential, which results in a harmonic field that remains harmonic for convex polygons, non-convex polygons, collections of polygons, self-intersecting curves, and overlapping polygons.
This means that it contains the fewest possible number of local minima and saddle points.
Our topology-based representation allows us to enforce all of these properties using the winding number.
Furthermore, inspired by~\cite{ivan-ijrr2013}, our penalty function exploits invariances within a homology class defined by the winding number.
In other words, this function captures containment, which makes it an ideal candidate for numerical optimization.