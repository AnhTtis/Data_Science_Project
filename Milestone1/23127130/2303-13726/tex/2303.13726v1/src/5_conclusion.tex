\section{Conclusion}
In this work, we introduced a novel topology-based approach that enables full-body dynamics~\gls{mpc} to automatically select footstep placements and contact surfaces for locomotion over discrete terrains.
Specifically, we proposed a contact-surface penalty function that uses potential field and winding number to optimize both footstep placement and contact surface.
With our method, we first justified the importance of considering full-body dynamics, which includes joint torque limits and limb dynamics, in footstep planning.
We evaluated the planned footsteps and showed that our full-body dynamics~\gls{mpc} can effectively adapt to variations in joint torque limits and friction coefficients.
Second, to demonstrate the practical implementation of our method in an~\gls{mpc} scheme on real robots, we conducted hardware experiments on discrete terrain using the ANYmal quadruped robot.
Finally, we showcased the potential capabilities of our approach in various dynamic locomotion maneuvers with robots equipped with high-torque actuators through stair-climbing and gap-crossing simulations.

For future work, we plan to implement a perceptive locomotion pipeline using our~\gls{mpc} to demonstrate the practical usefulness of our approach in solving real-world problems with changing environments. Furthermore, we will conduct empirical comparisons between our full-body dynamics~\gls{mpc} approach and reduced-order dynamics~\gls{mpc} to highlight the significance of our proposed approach.
Lastly, as demonstrated in \fref{fig:footstep_planning}, we showed the potential of our method for humanoid locomotion. Although the control complexity is higher, our method can be considered a future research direction for humanoid footstep planning.