\section{Introduction}\label{sec:introduction}
To \textit{traverse discrete terrains} such as stepping stones, legged robots need to carefully plan their footsteps and motions ~\cite{kuindersma-ar16,mastalli-tro20,jenelten-tro22}.
In previous works, footsteps and motions were computed separately~\cite{winkler15icra,mastalli-icra20,tonneau18tro} to reduce the combinatorial complexity of these nonlinear problems~\cite{posa14ijrr,mastalli16icra}.
However, in doing so, assumptions need to be introduced into the gait pattern, kinematics, and dynamics model.
Alternatively, to focus on the footstep planning (i.e., combinatorial problem), other approaches neglect limb dynamics, thereby allowing for formulating this problem as a mixed-integer convex problem~\cite{aceituno_cabezas-ral18,tonneau-icra20}.
However, for robots with heavy limbs or limited actuation torque, this assumption does not hold, nor can the full reachability (i.e., kinematics) of the robot be exploited.
These limitations also lead to errors in footstep tracking, which are caused by improper tracking of angular momentum~\cite{mastalli-underreview22}.
The above-mentioned limitations can be addressed by considering the robot's full-body dynamics, as it also takes into account limb dynamics and joint torque limits, and computing motions and footsteps together.
However, this results in a large optimization problem that is challenging to solve within a control loop of a few milliseconds.

\begin{figure}[t]
    \centering
    \includegraphics[width=\linewidth]{figs/footstep_planning.png}
    \caption{
    Visualization of the planned motions and footsteps, taking into account the robot's full-body dynamics (including joint torque limits and limb dynamics), friction cones, and all potential contact surfaces.
    Swing-foot trajectories are represented in different colors, while candidate contact surfaces are indicated by dark gray squares.
    Our approach can be used to plan footstep placements and contact surfaces for both quadruped and humanoid robots.
    A video demonstrating our approach is available at \texttt{\url{https://youtu.be/uweesAj5_x0}}.}
    \label{fig:footstep_planning}
\end{figure}

In our recent works, we demonstrated an~\gls{mpc} that takes into account the full-body dynamics of the robot~\cite{mastalli-underreview22,mastalli-invdynmpc22}.
A key advantage of our previous approaches is the ability of our~\gls{mpc} to generate agile and complex maneuvers through the use of feasibility-driven search~\cite{mastalli22auro} and optimal policy tracking.
However, it requires a predefined sequence of footstep placements and contact surfaces.
To address these limitations, this paper borrows concepts from topology and classical electrostatics to enable the automatic selection of footstep placements and contact surfaces in real time.
Compared to other state-of-the-art approaches, our footstep plans ensure joint torque limits, friction-cone constraints, and full-body kinematics and dynamics (\fref{fig:footstep_planning}).
To the best of our knowledge, our work is the first to \textit{introduce full-body dynamics~\gls{mpc} that optimizes footstep placement and contact surface} as well.

\subsection{Related Work}\label{sec:related_work}
Recent methods for footstep planning can optimize footstep placement and gait pattern~\cite{aceituno_cabezas-ral18, dai-ichr14, winkler-ral18, jiayi20iros}.
These approaches can handle discrete terrains by smoothing their geometry~\cite{posa14ijrr, dai-ichr14, winkler-ral18} or using discrete variables in mixed-integer optimization~\cite{aceituno_cabezas-ral18, tonneau-icra20, jiayi20iros, deits14ichr}.
However, their computational complexity makes it infeasible to deploy them online.
This is because they result in a combinatorial explosion of hybrid modes, which cannot be resolved by employing simplified models to cast the problem as mixed-integer convex optimization~\cite{aceituno_cabezas-ral18, deits14ichr}.
Alternatively, it can be formulated as a continuous problem with complementary constraints.
However, the ill-posed nature of the complementary constraints increases computation time, making it difficult to deploy this approach online~\cite{posa14ijrr,mastalli16icra,dai-ichr14}.
To avoid combinatorial complexity or ill-conditioning, our work \textit{focuses on optimizing footstep placement without optimizing gait pattern}.

The narrow focus on optimizing footstep placement enables online re-planning.
For instance, as shown in~\cite{herdt10ar}, \gls{mpc} can generate a walking motion with automatic footstep placement.
However, despite the impressive achievement, this approach assumes that the robot behaves as a linear inverted pendulum, which cannot account for its kinematic feasibility, joint torque limits, orientation, or the effect of non-coplanar contact conditions and limb and vertical motions.
Later, Di Carlo et al.~\cite{dicarlo-iros18} proposed a convex relaxation of the~\gls{srbd} that can accommodate the robot's orientation.
As in~\cite{herdt10ar}, this boils down to a linear~\gls{mpc} that can be solved with a general-purpose quadratic programming solver.
More recently, other~\gls{mpc} approaches employ~\gls{srbd} or~\gls{cd}, using direct transcription and general-purpose nonlinear programming solvers~\cite{bledt19iros,romualdi-icra22}.
Although these approaches can address non-coplanar contacts and vertical motions,~\gls{srbd} still cannot account for the robot's kinematic limits and the effects of limb dynamics.
Moreover, neither~\gls{srbd} nor~\gls{cd} can account for joint torque limits.
One may think that these limitations are not critical for robots with lightweight legs; however, a recent study shows that they still have a substantial impact on the control~\cite{corberes-icra21}.
This justifies why, in our work, we \textit{compute motions that ensure the robot's full-body dynamics}.

Real-time handling of both body and leg kinematics and dynamics in an~\gls{mpc} manner can be achieved by taking advantage of the temporal structure of the optimal control problem.
\Gls{ddp}~\cite{mayne-66} takes advantage of this structure by factorizing a sequence of smaller matrices, rather than employing sparse linear solvers~\cite{HSL} commonly done in nonlinear programming~\cite{nocedal-optbook}.
This reduction in computational complexity makes it feasible to use~\gls{mpc} with full-body dynamics, as shown in simulation results presented in~\cite{tassa-iros12}.
Inspired by these results, recent research has shown the application of~\gls{mpc} with~\gls{srbd} and full kinematics~\cite{farshidian-ichr17,grandia19iros,grandia21icra} and full-body dynamics~\cite{mastalli-underreview22,mastalli-invdynmpc22,koenemann-iros15,neunert-ral18,katayama-underreview}.
However, these approaches do not address the issue of selecting footstep placement and contact surface with full-body dynamics.
While algorithms based on~\gls{ddp}, such as iLQR~\cite{li-icinco04}, \textsc{Box-FDDP}~\cite{mastalli22auro}, and \textsc{ALTRO}~\cite{howell-19}, have been also developed, they have not yet been applied to address the aforementioned problems.
In contrast to other~\gls{mpc} approaches, our topology-based approach selects optimal footstep placement and contact surface within the framework of full-body dynamics~\gls{mpc} by \textit{creating a continuous cost function for terrain}.

\subsection{Contribution}
The main contribution of our work is an~\gls{mpc} that simultaneously plans full-body motions, torque commands, feedback policies, footstep placements, and contact surfaces in real time.
Specifically, we identify three technical contributions as follows:
\begin{enumerate}[label=(\roman*)]
    \item A novel topology-based~\gls{mpc} that uses potential field and winding number to plan footstep placements and contact surfaces in real time;
    \item Demonstration of the advantages of our topology-based~\gls{mpc} that takes into account full-body dynamics;
    \item Experimental validation of our topology-based~\gls{mpc} on the ANYmal robot and simulations that showcase its potential capabilities.
\end{enumerate}
In the next section, we introduce the concepts of potential field and winding number that we use to \textit{create a contact-surface penalty function}.