\section{Background}

\subsection{Quantum Circuits}
Quantum computation focuses on the use and manipulation of the qubit states $\ket{0}$ and $\ket{1}$, which can exist in a superposition of these states as $\ket{\psi} = \alpha\ket{0} + \beta\ket{1}$ prior to measurement. $N$ qubits exist in a superposition of $2^N$ basis states given by bitstrings of length $N$. These states are manipulated through the use of quantum logic gates in quantum circuits.

In principle, gates can act on any number of qubits. We mainly focus on single-, two- and three-qubit gate.  Multi-qubit gates often use controls, meaning the state of another qubit only changes when the value of the other qubits is in a specific state. For three-qubit gates, this can mean there are multiple controls, like the Toffoli gate shown in Figure \ref{fig:qubit-vs-mixed-radix}.  Similarly, multiple qubits can be controlled by one qubit. For comprehensive review of quantum gates we refer to \cite{nielsen_quantum_2011}.  

\subsection{Higher Radix Computation}
Most abstractions of quantum computing are binary, focusing on the superposition of only two computational states. Many physical quantum technologies have access to higher energy levels which can be used to represent additional logical states as qu\textit{dits} which use the lowest $d-1$ energy states which are increasingly harder to control. In this work, we constrain ourselves to at most four logical states, a \textit{ququart}, which balances the potential computational benefit with its increasing error and time cost. In its naive use-case, additional levels have the same computational benefit as in classical - at most constant reductions in circuit depth and gate counts \cite{pavlidis_arithmetic_2021}.

Some work \cite{gokhale_asymptotic_2019, litteken_communication_2022} has demonstrated specific applications that take advantage of extra computational states to reduce space requirements and improve execution time.  These strategies are not generally applicable as it requires hand optimization for those circuits.  Other work \cite{baker_efficient_2020} attempted to generalize these improvements through compression, which stores multiple qubits worth of information in a smaller number of qudits. However, the usefulness of this strategy for general applications has not been explored and did not consider direct-to-pulse implementations of multi-qudit gates.

\begin{table*}[htbp]
    \centering
    \caption{Durations for one-qubit, two-qubit and $i$Toffoli gates synthesized in qubit-only, mixed-radix and full-ququart environments.}
    \renewcommand{\arraystretch}{1.2}
    \begin{tabu}{l r|l r|[2pt]l r|[2pt]l r|l r|[2pt]l r|l r}
        \multicolumn{4}{c|[2pt]}{\textbf{(a) Qudit (ns)}} & \multicolumn{2}{c|[2pt]}{\textbf{(b) Qubit Only (ns)}} & \multicolumn{4}{c|[2pt]}{\textbf{(c) Mixed-Radix (ns)}} & \multicolumn{4}{c}{\textbf{(d) Full-Ququart (ns)}}\\
        \hline
        
        U & 35 & U$^0$ & 87 & CX$_2$ & 251 & CX$^{0q}$ & 560 & CX$^{q0}$ & 880 & CX$^{00}$ & 544 & CX$^{01}$ & 544 \\

        U$^1$ & 66 & U$^{0,1}$ & 86 & CZ$_2$ & 236 & CX$^{1q}$ & 632 & CX$^{q1}$ & 812 & CX$^{10}$ & 700 & CX$^{11}$ & 700 \\

        CX$^0$ & 83 & CX$^1$ & 84 & CS$^\dagger_2$ & 126 & CZ$^{q0}$ & 384 & CZ$^{q1}$ & 404 & CZ$^{00}$ & 392 & CZ$^{01}$ & 488 \\
        
        SWAP$^{in}$ & 78 & & & SWAP$_2$ & 504 & SWAP$^{q0}$ & 680 & SWAP$^{q1}$ & 792 & CZ$^{11}$ & 776 & SWAP$^{00}$ & 916 \\

        &&&& $i$Toffoli$_3$ & 912 & ENC & 608 &&& SWAP$^{01}$ & 892 & SWAP$^{11}$ & 964 

    \end{tabu}
    \label{tab:two-qubit-gates}
    \vspace{-0.4em}
\end{table*}

\subsection{Quantum Optimal Control}
The state of qudits is manipulated through external hardware-specific control fields $f_k(t)$. We consider superconducting devices, so these control fields are analog microwave pulses. Given a target unitary operation $U$, quantum optimal control finds controls $f_k$ which realize $U$. Many optimal control algorithms and toolboxes have been developed \cite{khaneja_optimal_2005, sklarz_loading_2002, petersson_optimal_2021, gunther_quantum_2021}, and here we make use of the open-source software package Juqbox \cite{petersson_discrete_2020, petersson_optimal_2021}. We find control pulses of shortest duration which realize gates of interest up to competitive fidelity, $0.99$ for two-qudit gates and $0.999$ for single-qudit gates. Juqbox achieves this by minimizing the objective $J[f_k] = 1 - F[f_k] + L[f_k]$ where

\begin{equation}
    F[f_k] = \frac{1}{h^2} \abs{\Tr{U^\dagger_T[f_k] \, V}}^2
\end{equation}
quantifies the gate fidelity between target unitary $V$ and the applied transformation $U_T[f_k]$. Here $h$ is the Hilbert space dimension of the logical subspace (in our case $h=d$) and $T$ denotes the allotted gate time. This task is solved by repeatedly solving the Schrödinger equation and adjusting the control fields to minimize $J$. Higher energy levels are sometimes included in the simulation in order to accurately capture their effect on the state evolution and reduce errors from truncating high-dimensional systems. These guard states are not logical states, therefore populating them is penalized with a leakage term $L[f_k]$. Currently, Juqbox only allows pulse optimization for a fixed gate time $T$, therefore we minimize pulse durations by applying an iterative re-optimization technique \cite{seifert_time-efficient_2022}.  