\documentclass[sigconf]{acmart}
\usepackage{amsmath,amsfonts}
\usepackage{bbm}
\usepackage{physics}
\usepackage{algorithmic}
\usepackage{graphicx}
\usepackage{textcomp}
\usepackage{xcolor}
%\usepackage[hyphens]{url}
\usepackage{tabu}
\usepackage{qcircuit}
\usepackage{enumitem}

\AtBeginDocument{%
  \providecommand\BibTeX{{%
    Bib\TeX}}}

\DeclareUnicodeCharacter{2009}{ }

\copyrightyear{2023} 
\acmYear{2023} 
\setcopyright{acmlicensed}
\acmConference[ISCA '23] {Proceedings of the 50th Annual International Symposium on Computer Architecture}{June 17--21, 2023}{Orlando, FL, USA.}
\acmBooktitle{Proceedings of the 50th Annual International Symposium on Computer Architecture (ISCA '23), June 17--21, 2023, Orlando, FL, USA}
\acmPrice{15.00}
\acmISBN{979-8-4007-0095-8/23/06} 
\acmDOI{10.1145/3579371.3589106}

%\usepackage[final]{changes}  % Uncomment to view final
%\usepackage[draft,addedmarkup=uline]{changes}  % Uncomment to view draft
% Set diff style
%\newcommand{\addcolor}{blue!70!green}
%\setaddedmarkup{\textcolor{\addcolor}{\uwave{#1}}}
%\setdeletedmarkup{\textcolor{red!70!black}{\sout{#1}}}
%\setdeletedmarkup{}  % Uncomment to only show additions

\def\BibTeX{{\rm B\kern-.05em{\sc i\kern-.025em b}\kern-.08em
    T\kern-.1667em\lower.7ex\hbox{E}\kern-.125emX}}

%%%%%%%%%%% Show marked up changes %%%%%%%%%%%
% Usage:
%   ... \added{text only in the new version} ...
%   ... \deleted{text no longer in the new version} ...
%   ... \replaced{NEW version of text}{OLD text} ...
%\usepackage[final]{changes}  % Uncomment to view final
%\usepackage[draft,addedmarkup=uline]{changes}  % Uncomment to view draft
% Set diff style
%\newcommand{\addcolor}{blue!70!green}
%\setaddedmarkup{\textcolor{\addcolor}{\uwave{#1}}}
%\setdeletedmarkup{\textcolor{red!70!black}{\sout{#1}}}
%\setdeletedmarkup{}  % Uncomment to only show additions
% Uncomment to reverse the added/deleted order
%\renewcommand{\replaced}[2]{\deleted{#2}\added{#1}}
%%%%%%%%% End show marked up changes %%%%%%%%%


% Ensure letter paper
\pdfpagewidth=8.5in
\pdfpageheight=11in

%%%%%%%%%%%---SETME-----%%%%%%%%%%%%%
%%%%%%%%%%%%%%%%%%%%%%%%%%%%%%%%%%%%

\pagenumbering{arabic}

\settopmatter{printacmref=true}
\begin{document}

%%%%%%%%%%%---SETME-----%%%%%%%%%%%%%
\title[Dancing the Quantum Waltz: Compiling Three-Qubit Gates on Four Level Architectures]{Dancing the Quantum Waltz:\\Compiling Three-Qubit Gates on Four Level Architectures}
%%%%%%%%%%%%%%%%%%%%%%%%%%%%%%%%%%%%

\author{Andrew Litteken}
\email{litteken@uchicago.edu}
\orcid{0000-0001-5676-1747}
\affiliation{%
  \institution{University of Chicago}
  \city{Chicago}
  \state{Illinois}
  \country{USA}
}

\author{Lennart Maximilian Seifert}
\email{lmseifert@uchicago.edu}
\orcid{0000-0002-2605-3720}
\affiliation{%
  \institution{University of Chicago}
  \city{Chicago}
  \state{Illinois}
  \country{USA}
}

\author{Jason D. Chadwick}
\email{jchadwick@uchicago.edu}
\orcid{0000-0002-7932-1418}
\affiliation{%
 \institution{University of Chicago}
 \city{Chicago}
  \state{Illinois}
 \country{USA}
}

\author{Natalia Nottingham}
\email{nottingham@uchicago.edu}
\orcid{0000-0003-3824-074X}
\affiliation{%
  \institution{University of Chicago}
  \city{Chicago}
  \state{Illinois}
  \country{USA}
}

\author{Tanay Roy}
\email{roytanay@fnal.gov}
\orcid{0000-0003-0438-012X}
\affiliation{%
  \institution{Fermilab}
  %\institution{University of Chicago}
  \city{Batvia}
  \state{Illinois}
  \country{USA}
}

\author{Ziqian Li}
\email{ziqianli@stanford.edu}
\orcid{0000-0002-3419-2333}
\affiliation{%
  \institution{Stanford University}
  %\institution{University of Chicago}
  \city{Stanford}
  \state{California}
  \country{USA}
}

\author{David Schuster}
\email{dschus@stanford.edu}
\orcid{0000-0002-0012-3874}
\affiliation{%
  \institution{Stanford University}
  %\institution{University of Chicago}
  \city{Stanford}
  \state{California}
  \country{USA}
}

\author{Frederic T. Chong}
\email{chong@cs.uchicago.edu}
\orcid{0000-0001-9282-4645}
\affiliation{%
  \institution{University of Chicago}
  \city{Chicago}
  \state{Illinois}
  \country{USA}
}

\author{Jonathan M. Baker}
\email{jonathan.baker@duke.edu}
\orcid{0000-0002-0775-8274}
\affiliation{%
  \institution{Duke University}
  \city{Durham}
  \state{North Carolina}
  \country{USA}
}

\IEEEtitleabstractindextext{%
\begin{abstract}
Vision transformer (ViT) expands the success of transformer models from sequential data to images. The model decomposes an image into many smaller patches and arranges them into a sequence. Multi-head self-attentions are then applied to the sequence to learn the attention between patches. 
Despite many successful interpretations of transformers on sequential data, little effort has been devoted to the interpretation of ViTs, and many questions remain unanswered. For example, among the numerous attention heads, which one is more important? 
How strong are individual patches attending to their spatial neighbors in different heads? What attention patterns have individual heads learned? 
In this work, we answer these questions through a visual analytics approach. Specifically, we first identify \textbf{\textit{what}} heads are more important in ViTs by introducing multiple pruning-based metrics. 
Then, we profile the spatial distribution of attention strengths between patches inside individual heads, as well as the trend of attention strengths across attention layers.
Third, using an autoencoder-based learning solution, we summarize all possible attention patterns that individual heads could learn. Examining the attention strengths and patterns of the important heads, we answer \textbf{\textit{why}} they are important. 
Through concrete case studies with experienced deep learning experts on multiple ViTs, we validate the effectiveness of our solution that deepens the understanding of ViTs from \textit{head importance}, \textit{head attention strength}, and \textit{head attention pattern}.
\end{abstract}

% Note that keywords are not normally used for peerreview papers.
\begin{IEEEkeywords}
Vision transformer, multi-head self-attention, deep learning, explainable artificial intelligence, visual analytics.
\end{IEEEkeywords}}

\renewcommand{\shortauthors}{A. Litteken, L. Siefert, J. Chadwick, N. Nottingham, T. Roy, Z. Li, D. Schuster, F. Chong, J. Baker}

%%
%% Keywords. The author(s) should pick words that accurately describe
%% the work being presented. Separate the keywords with commas.
\keywords{quantum computing, qudit, compilation}

\begin{CCSXML}
<ccs2012>
<concept>
<concept_id>10010583.10010786.10010813.10011726</concept_id>
<concept_desc>Hardware~Quantum computation</concept_desc>
<concept_significance>500</concept_significance>
</concept>
<concept>
<concept_id>10010520.10010521.10010542.10010550</concept_id>
<concept_desc>Computer systems organization~Quantum computing</concept_desc>
<concept_significance>500</concept_significance>
</concept>
</ccs2012>
\end{CCSXML}

\ccsdesc[500]{Hardware~Quantum computation}
\ccsdesc[500]{Computer systems organization~Quantum computing}

%\begin{teaserfigure}
%    \textbf{a)}
%    \hfill
%    \textbf{b)}
%    \quad\quad\quad\quad\quad\quad\quad\quad\quad\quad\quad\quad\quad\quad\quad\quad\quad\quad\quad\quad\quad\quad\quad\quad\quad\quad\quad\quad
%    \\
%    \begin{minipage}{0.45\linewidth}
%    \includegraphics[width=\linewidth]{figures/hero-image}
%    \end{minipage}
%    \hfill
%    \begin{minipage}{0.45\linewidth}
%    \includegraphics[width=\linewidth]{figures/HH_opt_IRB_edited.pdf}
%    \end{minipage}
%    \caption{a) A comparison of executing a Toffoli gate on a three-qubit-only system versus a Toffoli gate between a ququart and qubit in a mixed-radix system. In a qubit-only system, we must use a decomposition that uses eight two-qubit gates that can be reduced to one two-qudit gate that has a shorter duration. b) Interleaved Randomized Benchmarking for an optimal control $H \otimes H$ pulse on a superconducting transmon ququart following our qubit encoding. We use two-qubit Clifford sequences of gate depth up to 100 and average each data point over 10 samples. Error bars show the standard deviation of the mean but they are smaller than the mean markers. Red: Standard two-qubit Randomized Benchmarking to estimate the average Clifford gate fidelity to be $F_\mathrm{RB} \approx 95.8\%$. Blue: Interleaving the $H \otimes H$ pulse between the RB Cliffords yields a combined per-operation fidelity of $F_\mathrm{IRB} \approx 92.1\%$, resulting in an $H \otimes H$ fidelity of $F_{H \! H} \approx 96.0\%$.}
%    \label{fig:hero-images}
%\end{teaserfigure}

\maketitle

%%%%%% -- PAPER CONTENT STARTS-- %%%%%%%%

\section{Introduction}
\label{sec:introduction}

Suppose that we want to \emph{fit} and \emph{validate} a model on the basis of a single dataset.  Two example scenarios are as follows:
\begin{list}{}{}
\item{\emph{Scenario 1.}} We  want to use the data both to generate and to test a hypothesis. 
\item{\emph{Scenario 2.}} We want to use the  data both to fit a complicated model, and to obtain an accurate estimate of the expected prediction error. 
\end{list}
In either case, it is clear that a naive approach that fits and validates a model on the same data is deeply problematic. In Scenario 1, testing a hypothesis on the same data used to generate it will lead to hypothesis tests that do not control the Type 1 error, and to confidence intervals that do not attain the nominal coverage \citep{fithian2014optimal}.   And in Scenario 2, estimating the expected prediction error on the same data used to fit the model will lead to massive downward bias  \citep[see][for recent reviews]{tian2020prediction,oliveira2021unbiased}.

In the case of Scenario 1, recent interest  has focused on \emph{selective inference}, a framework that enables a data analyst to generate and test a hypothesis on the same data \citep[see, e.g.,][]{taylor2015statistical}. The main idea is as follows: to test a hypothesis generated from the data, we should condition on the event that we selected this particular hypothesis. Despite promising applications of this framework to a number of problems, such as inference after regression \citep{lee2016exact}, changepoint detection \citep{jewell2022testing,hyun2021post}, clustering \citep{gao2020selective,chen2022selective,yun2023selective}, and outlier detection \citep{chen2020valid}, it suffers from some drawbacks: 
\begin{enumerate}
\item To perform selective inference, the  procedure used to generate the null hypothesis must be fully-specified in advance.  For instance, if a researcher wishes to cluster the data and then test for a difference in means between the clusters, as in \cite{gao2020selective} and \cite{chen2022selective}, then they must fully specify the clustering procedure (e.g., hierarchical clustering with squared Euclidean distance and complete linkage, cut to obtain $K$ clusters) in advance. 
\item Finite-sample selective inference typically requires an assumption of multivariate Gaussianity, though in some cases this can be relaxed to obtain asymptotic results \citep{taylor2018post,tian2017asymptotics,tibshirani2018uniform,tian2018selective}.
\end{enumerate}
Thus, it is clear that selective inference does not provide a flexible, ``one-size-fits-all" approach to Scenario 1. 

In the case of Scenario 2, proposals to de-bias the ``in-sample" estimate of expected prediction error  tend to be specialized to simple models, and thus do not provide an all-purpose tool that is broadly applicable to complex contemporary settings \citep{oliveira2021unbiased}.

\emph{Sample splitting} \citep{cox1975note} is an intuitive  approach that is broadly applicable to a variety of settings, including Scenarios 1 and 2; see the left-hand panel of Figure~\ref{fig:samplesplit_vs_datathin}. We split a dataset containing $n$ observations into two sets, containing $n_1$ and $n_2$ observations, respectively (where $n_1+n_2=n$). Then we can generate a hypothesis based on the first set and test it on the second set (Scenario 1), or we can fit a model to the first set and estimate its error on the second set (Scenario 2). Sample splitting also forms the basis for cross-validation, an important tool for a practicing data scientist \citep{hastie2009elements}. 

While sample splitting often can 
 adequately address both Scenarios 1 and 2, it also suffers from some drawbacks: 
\begin{enumerate} 
    \item If the data contain outliers, then each outlier is assigned to a single subsample. %Again, this may not be desirable.
    \item If the observations are not independent (for instance, if they correspond to a time series) then the subsamples that result from sample splitting are not independent, and so sample splitting does not provide a solution to either Scenario 1 or Scenario 2.
    \item If one is interested in drawing conclusions at a per-observation level, then sample splitting is unsuitable.  For example, if sample splitting is applied to a dataset consisting of the 50 states of the United States, then one can only conduct inference or perform validation on those states not used in fitting.
    \item If the model of interest is fit using  unsupervised learning, then  sample splitting may  not provide an adequate solution in either Scenario 1 or 2.  The issue relates to \#3 above. This is discussed in \cite{gao2020selective,chen2022selective}, and \cite{neufeld2022inference} in the context of Scenario 1. 
\end{enumerate}

In recent work, \cite{neufeld2023data} proposed an approach for \emph{convolution-closed data thinning} that addresses these drawbacks. They consider splitting, or \emph{thinning}, a random variable $X$ drawn from a convolution-closed family into $K$ independent random variables $\Xt{1},\ldots,\Xt{K}$ such that
$X=\sum_{k=1}^K\Xt{k}$, 
and $\Xt{1},\ldots,\Xt{K}$ come from the same family of distributions as $X$ (see the right-hand panel of Figure~\ref{fig:samplesplit_vs_datathin}). 
For instance, they show that $X \sim N(\mu, \sigma^2)$ can be thinned into two independent $N(\epsilon \mu, \epsilon \sigma^2)$ and $N((1-\epsilon) \mu, (1-\epsilon) \sigma^2)$ random variables that sum to $X$. 
Finally, and most critically, if  $X$ is drawn from a Gaussian, Poisson, negative binomial, binomial, multinomial, or gamma distribution, then they can thin it  \emph{even when parameters of its distribution are unknown}. Because the thinned random variables are independent, this
provides a new approach to tackle Scenarios 1 and 2:  
one thins the dataset into two independent datasets.  One then fits a model to one dataset, and  validates it on the other. 




On the surface, it is quite remarkable that one can break up a random variable $X$ into two or more {\em independent} random variables that sum to $X$ without knowing some (or sometimes any) of the parameters.  In this paper, we seek to explain the underlying principles that make this possible.  In doing so, we show that convolution-closed data thinning can be generalized so as to make it more flexible and much more widely applicable. The convolution-closed data thinning property $X=\sum_{k=1}^K\Xt{k}$ is desirable because it ensures that no information has been lost in the thinning process. However, clearly this would be equally true if we were to replace the summation by any other deterministic function.  Likewise, the fact that $\Xt{1},\ldots,\Xt{K}$ are from the same family as $X$, while convenient, is nonessential. 

Our generalization of convolution-closed data thinning is thus a procedure for splitting $X$ into $K$ random variables such that  the following two properties hold: 
$$
\text{  (i) } X=T(\Xt{1},\ldots,\Xt{K}); \text{ and  (ii) }\Xt{1},\ldots,\Xt{K} \text{ are mutually independent}.
$$
This generalization is broad enough to simultaneously encompass both convolution-closed data thinning and sample splitting. Furthermore, it greatly increases the scope of distributions that can be thinned. In the $K=2$ case, this generalized goal has been stated before \citep[see][``P1'' property]{leiner2022data} as we will describe later.  However, we are the first to develop a widely applicable strategy for achieving this goal.   Not only are we able to thin exponential families that were not previously possible (such as the  beta family), but we can even thin outside of the exponential family.  For example, generalized thinning enables us to thin $X \sim \text{Unif}(0, \theta)$ into
 $\Xt{k} \overset{\text{iid}}{\sim} \theta \cdot\text{Beta}\left(\frac{1}{K},1\right)$, for $k=1,\dots,K$, in such a way that $X=\max\{\Xt{1},\ldots,\Xt{K}\}$.




The primary contributions of our paper are as follows:
\begin{enumerate}
\item We propose \emph{generalized data thinning}, a general strategy for thinning a single random variable $X$ into two or more independent random variables, $\xo,\ldots,\Xt{K}$, without knowledge of the parameter value(s).  
%Unlike in \cite{neufeld2023data}, we recover the original random variable $X$ using the function $T(\xo,\ldots,\Xt{k})$, where $T(\cdot)$ need not be addition, and the distributions of the $\Xt{k}$'s need not be the same as $X$.  
Importantly, we show that {\em sufficiency} is the key property underlying the choice of the function $T(\cdot)$.
\item We show that it is possible to apply generalized data thinning to distributions far outside the scope of consideration of \cite{neufeld2023data}: these include the beta, uniform, and shifted exponential distributions, among others.  A summary of distributions covered by this work is provided in Table~\ref{table:maintable}.  In light of results by \cite{darmois, koopman}, and \cite{pitman_1936} (see the end of Section~\ref{sec:method}), we believe our examples are representative of the full range of cases to which this sufficiency-based approach can be applied. 
\item We show that sample splitting --- which, on its surface, bears little resemblance to convolution-closed data thinning --- is in fact based on the same principle: both are special cases of generalized data thinning with different choices of the function $T(\cdot)$.  In other words, our proposal is a direct \emph{generalization} of sample splitting. 
\end{enumerate}



We are not the first to propose generalizations of sample splitting.  Inspired by \cite{tian2018selective}'s use of randomized responses, \cite{rasines2021splitting} define what they call the $(U,V)$-decomposition, which injects independent noise $W$ to create two independent random variables $U=u(X,W)$ and $V=v(X,W)$ that together are jointly sufficient for the unknown parameters.  However, they do not describe how to perform a $(U,V)$-decomposition other than in the special case of a Gaussian random vector with known covariance.  Our generalized thinning framework achieves the goal set out in their paper, providing a concrete recipe for finding such decompositions in a broad set of examples.  Another paper with a similar goal is \cite{leiner2022data}.  They define ``data fission'', which seeks to find random variables $f(X)$ and $g(X)$ for which the distributions of $f(X)$ and $g(X) \mid f(X)$ are known and for which $X=h(f(X),g(X))$. When these two random variables are independent (which they describe as the ``P1'' property), their proposal aligns with generalized thinning.  However, like \cite{rasines2021splitting}, they do not provide a general strategy for performing P1-fission, and the only two examples they provide are the Gaussian vector with known covariance and the Poisson.

The rest of our paper is organized as follows. In Section~\ref{sec:method}, we define generalized data thinning, present our main theorem, and provide a simple recipe for thinning that is followed throughout the paper. In Section~\ref{sec:natural-exp-fam}, we consider the case of thinning natural exponential families; this section also revisits the convolution-closed data thinning proposal of \cite{neufeld2023data}, and clarifies the class of distributions that can be thinned using that approach. In Section~\ref{sec:general-exp}, we show that we can apply data thinning to  general exponential families. We consider distributions outside of the exponential family in Section~\ref{sec:outside-exp-fam}. Section~\ref{sec:counterexamples} contains examples of distributions that \emph{cannot} be thinned using the approaches in this paper; these examples provide insight into the fact that sufficiency is the key property needed for (generalized) data thinning to ``work".  We verify our results numerically in Section~\ref{sec:experiments}.  Finally, we close with a discussion in Section~\ref{sec:discussion}; derivations and additional technical details are deferred to the appendix.  


\begin{figure}
  \hspace{39mm}  Sample splitting    \hspace{10mm} Generalized data thinning 
  \vspace{-3mm}
\begin{center} 
\centering
\includegraphics[scale=0.18,trim={3cm 8cm 39cm 8cm},clip]{figures/schematics-002.png}
\includegraphics[scale=0.18,trim={3cm 8cm 40cm 8cm},clip]{figures/schematics-v3-003.png} 
\caption{\emph{Left:} Sample splitting assigns each observation into either a training set or a test set. 
\emph{Right:} Generalized data thinning, the proposal of this paper, splits each observation into two parts that are independent and that can be used to recover the original observation $T(\xo, \Xt{2})=X$. In some cases, they are drawn from the same distributional family as $X$.  
\label{fig:samplesplit_vs_datathin}}
\end{center}
\end{figure}

\setlength{\tabcolsep}{2.5pt}
\begin{table*}[t]
\small
\centering
\caption{Experimental results of proposed method.}
\begin{tabular}{lcccccccccccccccclccccc}
\hline
                              &  &               &              &  & \multicolumn{7}{c}{Total   Power Consumption {[}mW{]}}                &  & \multicolumn{2}{c}{}          &  &                                                                           &  & \multicolumn{1}{l}{}                                                            \\ \cline{6-12}
                              &  & \multicolumn{2}{c}{Accuracy} &  &
			      \multicolumn{3}{c}{Standard HW} &  &
			      \multicolumn{3}{c}{Optimized HW} &  &
			      \multicolumn{2}{c}{\#Selected} &  &
			      \multirow{2}{*}{\begin{tabular}[c]{@{}c@{}}Max
				Delay\\ Red.\end{tabular}} &  &
				\multirow{2}{*}{\begin{tabular}[c]{@{}c@{}}Voltage
				  Scaling\\ Factor\end{tabular}} &
				  \multirow{2}{*}{\begin{tabular}[c]{@{}c@{}}
				    \\ V\_SHW\end{tabular}} &
				    \multirow{2}{*}{\begin{tabular}[c]{@{}c@{}}\\
				    V\_OHW\end{tabular}}\\ \cline{3-4} \cline{6-8} \cline{10-12} \cline{14-15}
Network-Dataset               &  & Orig.         & Prop.        &  & Orig.   & Prop.     & Red.      &  & Orig.   & Prop.     & Red.      &  & Wei.            & Act.          &  &                                                                           &  &                                                                                 \\ \hline
LeNet-5-CIFAR-10              &  & 80.6\%        & 78.5\%       &  & 375.5   & 149.6   & 60.2\%  &  & 360.7    & 78.3    & 78.3\%  &  & 35            & 210           &  & 40 ps                                                                    &  & 0.71/0.8  & 10.1\%  & 5.4\%                                                                       \\
ResNet-20-CIFAR-10            &  & 91.9\%        & 89.6\%       &  & 718.9   & 361.0   & 49.8\%  &  & 663.9    & 288.3   & 56.6\%  &  & 35            & 210           &  & 40 ps                                                                    &  & 0.71/0.8    & 13.2\%  & 11.5\%                                                                     \\
ResNet-50-CIFAR-100           &  & 79.9\%        & 78.5\%       &  & 708.7   & 293.8   & 58.5\%  &  & 701.8    & 157.1   & 77.6\%  &  & 41            & 223           &  & 30 ps                                                                    &  & 0.73/0.8  & 8.1\%  & 4.2\%                                                                      \\
EfficientNet-B0-Lite-ImageNet &  & 73.8\%        & 69.7\%       &  & 21.2    & 19.3    & 9.0\%   &  & 2.4      & 1.9     & 20.8\%  &  & 50            & 236           &  & 20 ps                                                                    &  & 0.75/0.8  & 6.4\%  & 6.5\%                                                                      \\ \hline
\end{tabular}
\label{tab:results}
\end{table*}






\section{Background}

\subsection{Quantum Circuits}
Quantum computation focuses on the use and manipulation of the qubit states $\ket{0}$ and $\ket{1}$, which can exist in a superposition of these states as $\ket{\psi} = \alpha\ket{0} + \beta\ket{1}$ prior to measurement. $N$ qubits exist in a superposition of $2^N$ basis states given by bitstrings of length $N$. These states are manipulated through the use of quantum logic gates in quantum circuits.

In principle, gates can act on any number of qubits. We mainly focus on single-, two- and three-qubit gate.  Multi-qubit gates often use controls, meaning the state of another qubit only changes when the value of the other qubits is in a specific state. For three-qubit gates, this can mean there are multiple controls, like the Toffoli gate shown in Figure \ref{fig:qubit-vs-mixed-radix}.  Similarly, multiple qubits can be controlled by one qubit. For comprehensive review of quantum gates we refer to \cite{nielsen_quantum_2011}.  

\subsection{Higher Radix Computation}
Most abstractions of quantum computing are binary, focusing on the superposition of only two computational states. Many physical quantum technologies have access to higher energy levels which can be used to represent additional logical states as qu\textit{dits} which use the lowest $d-1$ energy states which are increasingly harder to control. In this work, we constrain ourselves to at most four logical states, a \textit{ququart}, which balances the potential computational benefit with its increasing error and time cost. In its naive use-case, additional levels have the same computational benefit as in classical - at most constant reductions in circuit depth and gate counts \cite{pavlidis_arithmetic_2021}.

Some work \cite{gokhale_asymptotic_2019, litteken_communication_2022} has demonstrated specific applications that take advantage of extra computational states to reduce space requirements and improve execution time.  These strategies are not generally applicable as it requires hand optimization for those circuits.  Other work \cite{baker_efficient_2020} attempted to generalize these improvements through compression, which stores multiple qubits worth of information in a smaller number of qudits. However, the usefulness of this strategy for general applications has not been explored and did not consider direct-to-pulse implementations of multi-qudit gates.

\begin{table*}[htbp]
    \centering
    \caption{Durations for one-qubit, two-qubit and $i$Toffoli gates synthesized in qubit-only, mixed-radix and full-ququart environments.}
    \renewcommand{\arraystretch}{1.2}
    \begin{tabu}{l r|l r|[2pt]l r|[2pt]l r|l r|[2pt]l r|l r}
        \multicolumn{4}{c|[2pt]}{\textbf{(a) Qudit (ns)}} & \multicolumn{2}{c|[2pt]}{\textbf{(b) Qubit Only (ns)}} & \multicolumn{4}{c|[2pt]}{\textbf{(c) Mixed-Radix (ns)}} & \multicolumn{4}{c}{\textbf{(d) Full-Ququart (ns)}}\\
        \hline
        
        U & 35 & U$^0$ & 87 & CX$_2$ & 251 & CX$^{0q}$ & 560 & CX$^{q0}$ & 880 & CX$^{00}$ & 544 & CX$^{01}$ & 544 \\

        U$^1$ & 66 & U$^{0,1}$ & 86 & CZ$_2$ & 236 & CX$^{1q}$ & 632 & CX$^{q1}$ & 812 & CX$^{10}$ & 700 & CX$^{11}$ & 700 \\

        CX$^0$ & 83 & CX$^1$ & 84 & CS$^\dagger_2$ & 126 & CZ$^{q0}$ & 384 & CZ$^{q1}$ & 404 & CZ$^{00}$ & 392 & CZ$^{01}$ & 488 \\
        
        SWAP$^{in}$ & 78 & & & SWAP$_2$ & 504 & SWAP$^{q0}$ & 680 & SWAP$^{q1}$ & 792 & CZ$^{11}$ & 776 & SWAP$^{00}$ & 916 \\

        &&&& $i$Toffoli$_3$ & 912 & ENC & 608 &&& SWAP$^{01}$ & 892 & SWAP$^{11}$ & 964 

    \end{tabu}
    \label{tab:two-qubit-gates}
    \vspace{-0.4em}
\end{table*}

\subsection{Quantum Optimal Control}
The state of qudits is manipulated through external hardware-specific control fields $f_k(t)$. We consider superconducting devices, so these control fields are analog microwave pulses. Given a target unitary operation $U$, quantum optimal control finds controls $f_k$ which realize $U$. Many optimal control algorithms and toolboxes have been developed \cite{khaneja_optimal_2005, sklarz_loading_2002, petersson_optimal_2021, gunther_quantum_2021}, and here we make use of the open-source software package Juqbox \cite{petersson_discrete_2020, petersson_optimal_2021}. We find control pulses of shortest duration which realize gates of interest up to competitive fidelity, $0.99$ for two-qudit gates and $0.999$ for single-qudit gates. Juqbox achieves this by minimizing the objective $J[f_k] = 1 - F[f_k] + L[f_k]$ where

\begin{equation}
    F[f_k] = \frac{1}{h^2} \abs{\Tr{U^\dagger_T[f_k] \, V}}^2
\end{equation}
quantifies the gate fidelity between target unitary $V$ and the applied transformation $U_T[f_k]$. Here $h$ is the Hilbert space dimension of the logical subspace (in our case $h=d$) and $T$ denotes the allotted gate time. This task is solved by repeatedly solving the Schrödinger equation and adjusting the control fields to minimize $J$. Higher energy levels are sometimes included in the simulation in order to accurately capture their effect on the state evolution and reduce errors from truncating high-dimensional systems. These guard states are not logical states, therefore populating them is penalized with a leakage term $L[f_k]$. Currently, Juqbox only allows pulse optimization for a fixed gate time $T$, therefore we minimize pulse durations by applying an iterative re-optimization technique \cite{seifert_time-efficient_2022}.  

\section{Compression and Gate Set}

\subsection{Information Compression}
Information \textit{compression} in the context of this work refers to the storage of many qubits worth of information which deviates slightly from the typical classical understanding. The goal of this compression is to reduce the total number of physical units required to realize a given quantum algorithm. 

Rather than designing algorithms that specifically use higher level states, we encode the data of two individual qubits into one four-level computational unit, called a ququart, given as $\ket{\psi}_4 = \alpha\ket{0} + \beta\ket{1} + \gamma\ket{2} + \delta\ket{3}$. This can be seen as equivalent to $\ket{\psi_1}_2 \otimes \ket{\psi_2}_2 = \alpha_1\alpha_2\ket{00} + \alpha_1\beta_2\ket{01} + \beta_1\alpha_2\ket{10} + \beta_1\beta_2\ket{11}$ by the following mapping:
\begin{align*}
       \ket{00} \rightarrow \ket{0} \ \  \ket{01} \rightarrow \ket{1} \ \ \ket{10} \rightarrow \ket{2} \ \ \ket{11} \rightarrow \ket{3}
\end{align*}
Therefore, $\alpha = \alpha_1\alpha_2$, $\beta = \alpha_1\beta_2$ etc.  This compression does not result in the loss of any information since the transformation from $\ket{\psi_1}_2 \otimes \ket{\psi_2}_2$ to $\ket{\psi}_4$ is unitary and therefore invertible.  This follows a modification of the scheme from \cite{baker_efficient_2020}.
 
This compression does not require a circuit to explicitly use the $\ket{2}$ and $\ket{3}$ states for the compiler to make use of ququarts like in \cite{gokhale_asymptotic_2019, ivanov_time-efficient_2012, wang_qudits_2020}.  We are able to adapt qubit-compiler pipelines to compile a circuit and encode qubits into a ququart and keep track of the original qubits without requiring changes in the original circuit.

\subsection{Qubit Gates on Ququarts}
\label{sec:qubit-gates-on-ququarts}
Past work has studied higher level systems by generalizing operations on a qubit circuit. For example, the X gate, is generalized to a $+1 \mod d$ instead, where $d$ is the dimension of the qudit. Multi-qubit gates generalize similarly; for example a CNOT can be viewed as a $\ket{1}$-controlled $+1 \mod 2$ gate and therefore in general we can consider $\ket{c}$-controlled $+m \mod d$ gates, $0 \leq c, m \leq d - 1$ \cite{luo_universal_2014}. 

While possible to use this generalized gate set to perform computation, it is not concise. For example, to perform a CNOT between the second encoded qubits encoded in different ququarts we would need to apply two $\ket{1}$-controlled $+1$ gates and two $\ket{3}$-controlled $+1$ gates. We could instead generate and calibrate a more expressive gate set that directly performs this operation.

We develop a gate set which performs qubit operations directly on ququarts.  For a single-qubit gate $U$ acting on two encoded qubits in the state $\ket{q_0q_1}$, we use the unitary $U^0 = U \otimes \mathbbm{1}$ to act on qubit $q_0$, $U^1 = \mathbbm{1} \otimes U$ to act on qubit $q_1$, and $U^{0,1} = U \otimes U$ to act on both qubits simultaneously.

For two-qubit gates, there are several important classes of operations. The first is the interaction between the two compressed qubits which we call an \textit{internal} operation. For example, a $\text{CX}^0$ is a CNOT controlled on the second qubit targeting the first; this is equivalent to the single ququart gate which swaps the states $\ket{1}$ and $\ket{3}$. $\text{CX}^1$ controls on the first and targets the second encoded qubit. A SWAP operation exchanges the order of the encoding, i.e. SWAP$\ket{q_1q_2} = \ket{q_2q_1}$. The second are gates which act on qudits in different, but adjacent, physical locations. These \textit{partial} gates interact a non-encoded qubit and a qubit in an encoded pair in adjacent locations; all gates of this type we call \textit{mixed-radix} gates. For these gates, order matters, i.e. the gate behaves differently depending on which qubit is the target.
  
The four CX gates are $\{\text{CX}^{q0}, \text{CX}^{q1}, \text{CX}^{0q}, \text{CX}^{1q}\}$ where the first index indicates the control and the second the target object, and $q$ is the qubit.  We also define two mixed-radix SWAPs $\{\text{SWAP}^{q0}, \\ \text{SWAP}^{q1}\}$ which are the same regardless of direction.  The \textit{ full-ququart} gates follow from the mixed-radix gates defining the four CX gates: $\{\text{CX}^{00}, \text{CX}^{01}, \text{CX}^{10}, \text{CX}^{11}\}$ and three SWAPs: $\{\text{SWAP}^{00}, \\ \text{SWAP}^{01}, \text{SWAP}^{11}\}$.

\subsection{Generating Pulses}\label{sec:pulse-generation}
Using quantum optimal control we directly synthesize each of the gates in our new mixed-radix and full-ququart gate set and baseline comparisons. We use a realistic superconducting device Hamiltonian inspired by IBM hardware \cite{sheldon_procedure_2016}.

We consider up to three weakly coupled, anharmonic transmons \cite{koch_charge-insensitive_2007}:
\begin{align}
    &H(t) = ~\sum_{k=1}^3 \qty[\omega_k a_k^\dagger a_k + \frac{\xi_k}{2} a_k^\dagger a_k^\dagger a_k a_k] \\
    &+ \sum_{k=1}^3\sum_{l>k}J_{kl} (a_1^\dagger a_2 + a_2^\dagger a_1) \notag 
    + \sum_{k=1}^3 f_k(t) (a_k + a_k^\dagger).
    \label{eq:ham_rot}
\end{align}

The static terms describe the individual qudits and their pairwise couplings, while the last term captures the effect of driving the system through external control fields $f_k(t)$. The transmons are designed with $\ket{0}$-$\ket{1}$ transition frequencies $\omega_1/2\pi = 4.914 \,\mathrm{GHz}$, $\omega_2/2\pi = 5.114 \,\mathrm{GHz}$, and $\omega_3/2\pi = 5.214 \,\mathrm{GHz}$, and with equal anharmonicities $\xi_k/2\pi = -330 \,\mathrm{MHz}$. We consider linear connectivity with static couplings given by $J_{12}/2\pi = J_{23}/2\pi = 3.8\,\mathrm{MHz}$. The drive power is limited to $f_\mathrm{max} = 45\,\mathrm{MHz}$ to avoid substantial leakage into higher energy states, and we restrict ourselves to the $k=1$ subspace when synthesizing single-qudit gates. 

A full list of the gates synthesized and the minimal found duration of these gates can be found in Table \ref{tab:two-qubit-gates}. We reiterate the importance of short gate times - quantum systems are subject to a variety of both coherent and incoherent errors. By minimizing the total execution time of any given gate we reduce the circuit duration, reducing the effects of incoherent noise.

The closed system considered does not account for the full dynamics of a real quantum device. We have not specifically optimized these pulses under a more detailed model due to the increased computational cost of these optimizations, especially for the large Hilbert spaces involved in two-qudit operations.

In Section \ref{sec:experiment}, we use similar optimal control techniques to implement a single-ququart operation on an experimental device, showing that our methods and assumptions are realistic given a well-characterized machine.

\subsection{Properties of Qubit Gates on Ququarts}
Our gate set and mixed-radix architecture provides real advantages over typical qubit-only versions. Within each ququart, we have a pair of encoded qubits between which gates are 5x faster and 10x higher fidelity than qubit-only schemes. By using a single computation device, the total amount of control hardware required is reduced (at most by half).  Additionally, we have much higher connectivity between qubits once they are encoded in ququarts. In a ququart-qubit pair, there are three computational qubits directly connected to one another. Between two ququarts, there are four fully connected computational qubits. This is higher relative connectivity compared to industry standards for superconducting: lines, grids, and heavy hex architectures. Improved connectivity reduces expensive qubit movement operations.  These increased connections are demonstrated in Figure \ref{fig:mixed-radix-and-full-encoded}.

Compression is not without its downsides. In Table \ref{tab:two-qubit-gates}, we see mixed-radix and ququart gates take much longer than qubit based gates. Pulses must be more carefully designed, and leakage between states is more prominent, resulting in the longer gates times. Each increasing energy level has a shorter coherence time scaling with $1/k$ where $k$ is the energy level. Shorter decoherence, combined with longer gate times, means using mixed-radix and ququart based gates is a delicate balancing act between increasing fidelity due to gate execution while not increasing error due to decoherence.


\subsection{Experimental Demonstration of Single-Ququart Control}\label{sec:experiment}

Driven by advantages found in theoretical studies \cite{pavlidis_arithmetic_2021}, experimental researchers have explored the implementation of these higher-dimensional systems, leading to realizations of qutrit devices which manipulate the third energy level \cite{galda_implementing_2021, hill_realization_2021, roy_realization_2022, goss_high-fidelity_2022, morvan_qutrit_2021, wu_high-fidelity_2020}. These works show that including higher levels is possible although challenging due to higher susceptibility to noise and lower coherence times.

Motivated by the findings for ququart-specific applications we have been studying control of four energy levels in experiment on a physical device. We extend the capabilities of one qudit of the superconducting transmon device presented in \cite{li_autonomous_2023, li_hardware_2023, roy_realization_2022} to include the fourth state. We implement two-qubit Randomized Benchmarking (RB) \cite{magesan_scalable_2011} on this single ququart following our encoding scheme. RB is a common method to characterize the average Clifford gate fidelity. This is achieved by executing Clifford circuits of varying depth, which perform the identity operation in the ideal case, and measuring the probabilities of the system returning to the ground state (survival probabilities). The fidelity can be extracted from exponential regression. The RB circuits are generated using Qiskit \cite{anis_qiskit_2021}.

We additionally implement Interleaved Randomized Benchmarking (IRB) \cite{magesan_efficient_2012} to specifically find the fidelity of the single-ququart gate $H \otimes H$, which performs a Hadamard gate on each encoded qubit in parallel, used by the compiler below. The gate control pulse is designed using similar optimal control methods as discussed in Section \ref{sec:pulse-generation} adapted to this experimental device.

Results from this work are shown in Fig. \ref{fig:HH_IRB}. We find an average Clifford gate fidelity of $F_\mathrm{RB} \approx 95.8\%$ from normal RB while interleaving with the $H \otimes H$ gate yields $F_\mathrm{IRB} \approx 92.1\%$ fidelity per operation. From that the specific gate fidelity, $F_{H \! H} \approx 96.0\%$, can be extracted. This first study shows that ququarts can be realized in experiment and optimal control yields high-quality pulses to manipulate their state. At the time of this writing we are not aware of any comparable demonstration. We are convinced that the fidelities can be improved with more carefully engineered ququart devices and more sophisticated pulse design methods.

% Note: should we say that manuscript will be out soon? No, then it makes it sound like we should have waited until that was out. Clever
\begin{figure}
    \centering
    \includegraphics[width=0.7\linewidth]{figures/3qb_connectivity.pdf}
    \caption{Visualization of connectivity advantages in qubit-ququart systems. Encoding qubits in ququarts (light blue) enables triangle connectivity between triples of qubits, where two of which are encoded in the same ququart and one appears either in a bare qubit or encoded in a neighboring ququart.}
    \label{fig:mixed-radix-and-full-encoded}
\end{figure}

\section{The Quantum Waltz: Three Qubit Gates on Ququarts}

Three-qubit gates are widely used in arithmetic operations, such as the Cuccaro adder \cite{cuccaro_new_2004} and multi-controlled-CNOT \cite{barenco_elementary_1995}, as smaller pieces in larger quantum algorithms such as \cite{grover_fast_1996}.  While QAOA and VQE see more use in the current quantum algorithm space, some QAOA based algorithms still use three-qubit gates \cite{hadfield_quantum_2019}.  Additionally, some error correction schemes make heavy use of three-qubit gates \cite{yoder_universal_2017}.  Current hardware platforms typically decompose these gates. Using higher radix we can reduce gate times mitigating the issue of reduced coherence times of higher-energy levels enabling more efficient execution of quantum circuits containing these gates.

\subsection{Connectivity Advantage}

The set of two qubit gates laid out in Section \ref{sec:qubit-gates-on-ququarts} are enough to universally perform general qubit computation on ququarts \cite{luo_universal_2014, nielsen_quantum_2011}, but simply compiling to two-qubit gates would not take advantage of the flexibility of this abstraction. When we encode two qubits in a ququart, we virtually increase the connectivity between qubits, see Figure \ref{fig:mixed-radix-and-full-encoded}.  Each of the encoded qubits in a ququart is connected to an adjacent qubit, or both of the encoded qubits in an adjacent ququart. As highlighted by each of the different colors this creates many triangle subgraphs between encoded qubits . Triangle subgraphs are uncommon in current hardware due to the increased probability of crosstalk \cite{mundada_suppression_2019, ding_systematic_2020}. But, triangle-based interactions are common in many different circuits that use three-qubit gates.  Here, we increase the number of virtual connections without increasing number of physical connections to create four interactions between encoded qubits.

\begin{figure}
    \centering
    \includegraphics[width=0.85\linewidth]{figures/state_evs.pdf}
    \caption{Visualization comparing the evolution of a $\ket{3}$-controlled $X$ gate in a mixed-radix environment for a CCX gate in (a) and a CX gate in (b).}
    \label{fig:state-evolutions}
\end{figure}

It is not fundamentally harder to interact three or four qubits worth of information than two qubits worth with a single operation on ququarts. These gates are equivalent to either mixed-radix or full-ququart gates. For example, if we have a fully encoded ququart next to a bare qubit and perform a Toffoli gate targeting the qubit, it is equivalent to a $\ket{3}$-controlled X  gate on the qubit. This is computationally simpler than the several $\ket{1}$- and $\ket{3}$-controlled X required in the decomposition and can be seen in the state evolutions in Figure \ref{fig:state-evolutions}. This gate implementation gives superconducting qubits more natural access to the native multi-qubit gates,  avoids decompositions that add extra gates and performs three-qubit interactions between two physical quantum devices, reducing the complexity of implementing such a three-qubit pulse across three devices and two couplers. Used in conjunction with the previously generated one- and two-qubit gates, we can more efficiently perform circuits that include three-qubit gates.

\subsection{Generated Pulses}

\begin{table}[htbp]
    \centering
    \caption{Mixed-Radix and Full-Ququart Three-Qubit Gate Durations}
    \renewcommand{\arraystretch}{1.2}
    \begin{tabu}{l r|[2pt]l r|l r}
        \multicolumn{2}{c|[2pt]}{\textbf{(a) Mixed-Radix (ns)}} & \multicolumn{4}{c}{\textbf{(b) Full-Ququart (ns)}} \\
        \hline
        CCX$^{q01}$ & 619 & CCX$^{01,0}$ & 536 & CCX$^{01,1}$ & 552 \\
        CCX$^{1q0}$ & 697 & CCX$^{0,01}$ & 785 & CCX$^{0,10}$ & 785 \\
        CCX$^{01q}$ & 412 & CCX$^{1,10}$ & 785 & CCX$^{1,01}$ & 680 \\
        \hline
        CCZ$^{01q}$ & 264 & CCZ$^{01,0}$ & 232 & CCZ$^{01,1}$ & 310 \\
        \hline
        CSWAP$^{01q}$ & 684 & CSWAP$^{01,0}$ & 680 & CSWAP$^{01,1}$ & 744 \\
        CSWAP$^{10q}$ & 762 & CSWAP$^{10,0}$ & 758 & CSWAP$^{10,1}$ & 822 \\
        CSWAP$^{q01}$ & 444 & CSWAP$^{0,01}$ & 510 &  CSWAP$^{1,01}$ & 432
    \end{tabu}    
    \label{tab:three-qubit-gate-times}
\end{table}


\subsubsection{Multi-control Gates}
Native three-qubit gates on two physical units have the potential to offer a significant improvement in gate fidelity and execution time.  In Table \ref{tab:three-qubit-gate-times}, we show pulse durations of the three-qubit Toffoli gate in several mixed-radix and full-ququart configurations. These gates were synthesized using the same fidelity targets and pulse generation techniques as the two-qubit gates, a higher fidelity than if decomposed with many gates of the same target fidelity. After synthesizing the different configurations of Toffoli gates, we find that there is a substantial difference in the gate duration depending on which qubits are controls and which is the target.
% as it relates to the configuration of the controls and the target of the Toffoli gate across the three or four qubits encoded in two physical units.

Consider the mixed-radix example where both control qubits are encoded in the same ququart, and the target qubit is in the bare qubit, or the $\text{CCX}^{01q}$ gate, seen in Figure \ref{fig:three-qubit-configs}a.  This configuration is about two-thirds the time of the $\text{CCX}^{0q1}$, seen in Figure \ref{fig:three-qubit-configs}b, where the control qubits are split across the bare qubit and the ququart.  The reason for this difference is twofold.  The first follows from the two-qubit only gates. Gates which use the ququart as a control and the qubit as a target are generally faster, the pulse only induces population changes between the $\ket{0}$ and $\ket{1}$ state of the qubit, rather than between $\ket{0}$ and $\ket{1}$, and $\ket{2}$ and $\ket{3}$ of the ququart.  The second is that the entire ququart acts as the control, only changing the state of the bare qubit if the ququart is in the $\ket{3}$ state.  In the split-control case, the ququart must control on both the $\ket{2}$ and $\ket{3}$ state.

The same concept of separation of controls and targets follows for the full-ququart Toffoli gates as well.  Regardless of whether the target qubit is in the first or second encoding of the ququart, it is substantially faster to keep the controls encoded in the \textit{same} ququart with the target encoded in a separate ququart. 

\subsubsection{Target-Independent Gates}
Separating the controls and targets into different devices yields more efficient gate execution; however, compiling circuits to conform to this configuration is unnecessarily constraining. Instead, we consider a situation where all multi-qudit gates are \textit{target-independent} and only affect the global state when all three qubits are in $\ket{1}$. For example, the Toffoli gate, or CCX, is locally equivalent to CCZ which is target-independent, as seen in Figure \ref{fig:toffoli_decomp}c. 

When pulses are synthesized, CCZ is much more efficient as seen in Table \ref{tab:three-qubit-gate-times}, remarkably on par with the speed of the qubit only gates. In addition, we only need to define three configurations: $\text{CCZ}^{q,01}, \text{CCZ}^{0,01}, \text{CCZ}^{1,01}$, as opposed to the nine possible CCX configurations, reducing computational overhead. We postulate the short duration of these gates is because CCZ only changes the phase of the entire three-qubit state rather than the population. This makes the CCZ a valuable tool when compiling three-qubit gates.

\begin{figure}
    \centering
    \scalebox{0.85}{
    \includegraphics[width=\linewidth]{figures/three-qubit-gate-configs.png}
    }
    \caption{Examples of mixed-radix two-control and two-target gates. a) A configuration where both controls are encoded in the ququart and the target is mapped to a qubit. b) A configuration where the controls are split across the qubit and the ququart and the target is encoded in the ququart. c) A configuration where both targets are encoded in the ququart and the control is mapped to the qubit. d) A configuration where the targets are split across the qubit and the ququart and the control is encoded in the ququart.}
    \label{fig:three-qubit-configs}
    \vspace{-0.5em}
\end{figure}

\subsubsection{Multi-target Gates}
\label{sec:multitarget}
We also consider gates that use one control qubit to affect the state of some number of other qubits, for example the CSWAP. With our methods we synthesize gates to the same fidelity targets as before and show their times in Table \ref{tab:three-qubit-gate-times}. We find benefits when separating the control qubit from the target qubits as depicted in Figure \ref{fig:three-qubit-configs}c versus Figure \ref{fig:three-qubit-configs}d.  When both targets are encoded the same ququart, we limit the state changes to be between $\ket{1}$ and $\ket{2}$ in that ququart.
\section{Compilation Strategies}

\subsection{Using Three-Qubit Gates}
In our qubits-on-ququarts compilation strategy, we expand the physical connectivity graph between the ququarts on a given architecture and treat each ququart as two connected qubits.  Each qubit in the expanded ququart is fully connected to the qubits in the neighboring ququarts as shown in Figure \ref{fig:mixed-radix-and-full-encoded}.  We call this new graph the interaction graph; it maintains a mapping of where circuit qubits are mapped to on this graph. When one or fewer of the qubits in the expanded ququart is mapped to, the entire ququart is in a qubit state. Otherwise, it is considered to be in the ququart state.

To execute three-qubit gates, circuit qubits must be routed into a connected subgraph of the interaction graph, e.g. for CCZ$(q_0, q_1, q_2)$ requires $q_0\sim q_1$ and $q_1\sim q_2$ but it is not guaranteed that $q_0\sim q_2$, where $\sim$ defines adjacency. We develop a compiler optimization which appropriately performs routing and gate selection based on this adjacency and use of higher dimension. While we are able to perform any configuration of three qubit gates directly in mixed-radix or full-ququarts scenarios, we take care to use best configurations to minimize time in the less stable $\ket{2}$ or $\ket{3}$ states.

\subsubsection{Qubit-Only}
In a qubit-only regime we can use a decomposition into eight CX operations \cite{shende_cnot-cost_2008}.  This decomposition has the flexibility of being target-independent from a compilation standpoint.
%We simply surround the target qubit with Hadamard gates, similar to a CCZ gate. 
This is an expensive compilation, requiring eight two-qubit gates and 14 one-qubit gates.  But, it does not use the less stable $\ket{2}$ and $\ket{3}$ state.  Alternatively, we can use a directly-optimized three-qubit pulse sequence. QOC software failed to find a solution for a direct CCZ operation, so we synthesize a pulse implementing the $i$Toffoli gate using a three-qubit version of our quantum optimal control software that only uses the first two levels of the qubits and use the decomposition shown in Figure \ref{fig:toffoli_decomp}d inspired by \cite{kim_high-fidelity_2022} to execute a complete Toffoli gate.

\begin{figure}
    \centering
    \scalebox{0.9}{
        \input{figures/toffoli-decomp.qcircuit}
     }
     \caption{Different decompositions for the Toffoli Gate. a) is the base Toffoli circuit. b) is Toffoli circuit with a swapped second control and target from the original.  By surrounding the control and the target with Hadamards, we perform the same operation. c) The Toffoli gate constructed from a CCZ gate which can be used as a Toffoli by surrounding the target with Hadamard gates. (d) The Toffoli gate constructed from an iToffoli gate, which requires an controlled $S^\dagger$ gate in addition to Hadamard gates.}
    \label{fig:toffoli_decomp}
    \vspace{-0.5em}
\end{figure}

\subsubsection{Intermediate Mixed-Radix}
We also permit \textit{temporary} use of the higher energy levels to perform an operation.  By performing an encoding gate (ENC) followed by the three-qubit gate and a final decode (ENC$^\dagger$) operation, we get temporary access to full connectivity to perform fast three-qubit gates.

The compiler should opt to encode qubits of similar \textit{type}, i.e. both controls together or both targets together. Let $U(q_0, q_1, q_2)$ be the operation with $q_0, q_1$ the same (either both controls or both targets). In some cases, encoding is simple because the routing strategy (prior work) results in $q_0\sim q_1$ as in  \ref{fig:three-qubit-configs}(a). However, it may fail to do this by default and we may have $q_0\sim q_2$ and $q_1\sim q_2$ as in \ref{fig:three-qubit-configs}(b). 

We have three options to compile to a favorable configuration.  First, we could enforce the ideal relationship through additional gates by adding an additional SWAP($q_0$, $q_2$). Second, in the special case where $U = X$ Toffoli we can change which pair is the same type with Hadamard gates as in Figure \ref{fig:toffoli_decomp}b to use the most efficient implementation; we call this \textit{re-targeting}. Third, if $U$ permits, we transform $U$ into $U'$ so that $q_0, q_1, q_2$ are all the same type; for example we transform CCX to CCZ so each operand is a ``control,'' Figure \ref{fig:toffoli_decomp}c. While the additional re-targeting or transformation gates add both error and duration, they enable the shortest duration version of $U$ to be used for an overall net increase in fidelity. We consider the special cases of $U \in \{CCV | V \in SU(2)\}$, i.e the set of locally equivalent gates to $CCX$. We leave the generalized case to future work in circuit synthesis.


\subsubsection{Full Ququart}
Mixed-radix three-qubit gate strategies apply for full-ququart compilation as well. However, the router by default, described below, does not distinguish control or target. When executing three-qubit gates, we ensure only qubits of the same type are encoded if it does not require an extra swap operation.

\subsection{Mapping and Routing}
Our compilation for encoded qubits on ququart architectures is similar to previous compilation strategies for qubits as seen in many previous works \cite{cowtan_qubit_2019, murali_noise-adaptive_2019, duckering_orchestrated_2021} and adapts them to three-qubit gates on ququart architectures.  However, unlike these prior works, we take into account the varying fidelities and durations of internal ququart versus mixed-radix versus full-ququart inter-ququart gates, similar to \cite{litteken_communication_2022}.

The first step is to decompose the operations in the circuit to native gates supported by the device. Our compiler handles the native execution of three-qubit gates, we decompose to the CX, CCX, CCZ or CSWAP along with a parameterized single-qubit rotation gate.

Qubits are mapped onto the interaction graph with the goal of maximizing locality.  We assign a weight between each pair of qubits in the original circuit according to: $w(i, j) = \sum_{t \in C} o(i, j, t)/{t}$, where the sum is over each time step $t$ in the circuit $C$ and $o(i, j, t) = 1$ if qubits $i, j$ interact in time step $t$ and $0$ otherwise. This weight includes lookahead functionality by weighting future interactions (larger $t$) smaller. 
%This adds a weighted value that decreases as the time step $t$ in the circuit increases if the qubits interact during time step $t$ as determined by $o(i, j, t)$. 
The first qubit is mapped according to which has greatest total weight to all other qubits: \\ $\underset{i}{\text{argmax}} W(i) = \sum_{j \in Q_c / \{i\}} w(i, j)$.  This qubit is placed in the first encoded location of the center-most qudit on the connection graph.  For each other qubit, we choose the circuit qubit that has the greatest $W$ with respect to the placed qubits.  For each adjacent qubit, $n$, to the placed qubits, we compute $\sum_{j \in Q_P} w(i, j) d(n, \varphi(j))$ where $\varphi$ is the mapping of circuit qubits to physical qubits, and $d$ is a specialized fidelity function between the qubits estimating the possibility of error along the communication path.  We then map the qubit to the minimizing location.

When routing, we track the circuit qubits on the interaction graph and use SWAP gates until the interacting qubits are adjacent. We attempt to disrupt advantageous qubit layouts as little as possible by using adaptive weights that change as operations are scheduled based on \cite{baker_time-sliced_2020}.  This strategy attempts to keep qubits interacting in the near future close to one another where the disruption of each potential SWAP between circuit qubits $i, j$ is calculated by $D(i, j) = \sum_{k \in Q_c} w(i, k) (d(\varphi(i), \varphi(k)) - d(\varphi(j), \varphi(k))) +  w(j, k) (d(\varphi(j), \varphi(k)) - d(\varphi(i), \varphi(k)))$. However, rather than using simple distances, we use the same specialized distance metric incorporating the previous function $d$.  We choose the SWAP candidate that minimizes this value while always moving the qubit closer to the other qubits it needs to interact with.  To generalize to three-qubit based routing we modify the cost function to $C(i) = \sum_{j \in Q_o / i}D(i, \varphi^{-1}(n)) (d(\varphi(i), \varphi(j)) - d(n, \varphi(j))$  where $Q_o$ is now a set of all operands.

It would be reasonably simple to extend this compiler design to accomodate k-qubits on n-d-level-qudits, where we pack each qudit with $log_2(d)$ qubits, and ensure that there is no way to move any one qubit closer to another in a fully connected set of qubits. However, we only explore three-qubit gates on a maximum of two, four-level devices, or three, two-level devices in this work.  This is for design and practical reasons.  From a design point of view, our gate set and compiler are intended to be used after a circuit has been translated into qubit-based gates. Compiling natively to higher-radix qudit operations would require a much larger set of basis gates than the qubit-based set we use here.  Additionally, there is not a standard set of four-or-more qubit gates that are typically used in circuits, meaning there would have to be some arbitrary decomposition to four-qubit gates, rather than three qubits.  Choosing a basis gate set is a time intensive process and has to be done selectively \cite{gokhale_optimized_2020}. We therefore expand the normal one and two-qubit framework used in many compilers to include the most commonly used three-qubit gates as it is the most common multi-qubit gate.

It should be noted that all translation to higher-radix devices occurs during this compilation step.  The general programmer still writes a program in terms of qubits.  The compiler translates the program into the correct sequence of qubit-on-ququart operations to perform the same computation.  In the case of full-ququart operation, the measured state would be decoded according to the compression strategy.
\section{Evaluation}

\subsection{Circuits}

We examine five three-qubit based circuits that can be parameterized by number of qubits with different constructions. The first is the Generalized Toffoli (CNU) circuit \cite{baker_decomposing_2019}, which flips the state of a target qubit if all the controls are one. This circuit uses exclusively Toffoli gate based decomposition and is highly parallel. The Cuccaro Adder \cite{cuccaro_new_2004} is nearly entirely serialized using $2n+2$ qubits with a mix of three-, two- and single-qubit gates to add two $n$-bit numbers. Third is a QRAM circuit which uses primarily CSWAP gates to retrieve data from or move data into a set of qubits \cite{gokhale_quantum_2020}.  The fourth is a Select circuit, which is a preparation mechanism used in Quantum Phase Estimation (QPE) \cite{low_hamiltonian_2019}.  It performs a particular Pauli operation on $n$ qubits for each potential $2^m$ states of $m$ index qubits \cite{babbush_encoding_2018}.  For our case, the choice of Pauli string does not affect compilation.  To keep the fidelity of circuit simulation within comparable bounds, we only select on two random values rather than all of the potential $2^m$ values the index qubits could be in.  The fifth is a purely synthetic circuit to study relative strength of our architecture on potential distributions of CX versus CCX gates.

\subsection{Baselines, Hardware Topology and Error}
We compare against two strategies.  The first is a compilation that routes the circuit with three-qubit gates, before decomposing them to one- and two-qubit gates only.  This is in line with current practices for most compilation pipelines.  The second baseline does not decompose to these smaller gates.  Instead, the $i$Toffoli-based decomposition directly on qubits similar to \cite{kim_high-fidelity_2022}.  This is a more challenging gate to synthesize as discussed previously. For simulation, this gate has a 99\% fidelity and with 912 ns duration determined via the same quantum optimal control strategies as the mixed-radix and full-ququart gates.  Additionally, we use the Hadamard-based retargeting technique to ensure that we are applying the Toffoli gate to the correct qubit without an extra SWAP.  This allows us to always use the demonstrated $i$Toffoli gate where the target qubit is the center of three connected qubits.

We consider the same underlying hardware topology for each comparison point - a 2D mesh. This type of grid architecture has relative density on the upper end of realized superconducting connectivity graphs, reflective of Google's Sycamore chip \cite{arute_quantum_2019} and more dense than IBM's heavy-hex \cite{gambetta_expanding_2022}. We consider a  grid design with dimensions $\lceil\sqrt{n} \rceil \times \frac{n}{\lceil\sqrt{n}\rceil}$ with nearest neighbor connectivity.

We use a realistic T1 time from an IBM device of $163.45 \mu s$ \cite{ibm_ibm_nodate}.  Higher energy levels decohere more quickly.  In theory, each state decays at a rate of $o(1/k)$ where $k$ is the energy level as discussed in \cite{younis_berkeley_2021}. We therefore use $81.73 \mu s$ and $54.15 \mu s$ as the T1 times for the $\ket{2}$ and $\ket{3}$ states.  As any transmon technically has access to these higher energy states, we do not expect that a device designed to access these higher-energy states will reduce the base T1 time. 

\subsection{Circuit Estimation}
We use two metrics to estimate the fidelity of a circuit without simulation to extrapolate how compiled circuits may perform by comparing simulation to estimation. The first is the product of all of the gate success rates in the circuit, called the gate expected probability of success (gate EPS).  Since there are multiple classes of multi-qubit gates, some of which have higher fidelity than others, we use the product of these success rates. 

Second, we model decoherence as an exponential decay where the probability of no decoherence is $\prod_{k=1}^3 \text{exp}(k*t_k/T_1)$ where $t_k$ is the time the qubit spends in state $k$.  When we construct the circuit we keep track of how long each qudit exists in the $\ket{1}$ or $\ket{3}$ state as the maximum state and calculate the probability of not decohering over the course of the execution for each qudit.  The product of the expected success of each qudit is the EPS due to coherence for the entire circuit.  When multiplied by the gate EPS, we have the EPS for the entire circuit.

\subsection{Circuit Simulation}
Since access to ququart devices at this scale are limited, we must use simulation to evaluate the performance of our approach. We use the trajectory method \cite{brun_simple_2002} for improved scalability compared to full density simulation.  This work simulates circuits of up to 24 qubits (or, equivalently, 12 ququarts). For this work, for each circuit, we generate at least 1000 random quantum states and for each we simulate once and compute the average fidelity over all random states. We emphasize the use of random \textit{quantum} states as classical inputs are not always affected by quantum errors. 

In the past, prior work on simulation of qudit systems neglects the realistic duration differences between gates which results in drastically different usage patterns and simply injecting errors on a moment-to-moment basis can skew results. For example, in this work our direct-to-pulse compilation of CCX and CCZ gates have significantly different execution times. We modify the trajectory method simulation slightly to account for this difference. Rather than inserting many idle gates during each time step, before each gate, we insert one idle gate using the exact time that qudit has been idle.  This is a more accurate representation of from which state these qudits could be decohering.

\begin{figure*}
    \centering
    \scalebox{0.9}{
    \includegraphics[width=\linewidth]{svg-inkscape/simulation-results.pdf}
    }
    \vspace{-0.5em}
    \caption{Simulated results for QRAM, Generalized Toffoli, Cuccaro Adder and Select Circuit from 5 to 21 qubits with different mixed-radix and full-ququart compilation strategies. The mixed-radix strategies do not have complete error bars due to the requirement to simulate a four-level system for every qubit which would require more than 86 GB of memory per circuit in our simulation framework. The final graph is the average fidelity improvement for each compilation method over the qubit-only compilation method as the size of the circuit increases.}
    \label{fig:main-simulation-results}
    \vspace{-0.5em}
\end{figure*}

\subsection{Noise Model for Qudit Systems}
For qubits we consider both symmetric depolarizing and amplitude damping errors. There are four possible single-bit %error 
channels: no error ($I$), bit flip errors ($X$), phase flip errors ($Z$) and bit and phase flip errors ($Y = ZX$). In simulation each error channel is drawn with probability $p/3$. Two-qubit errors are given as the product of single-qubit errors, e.g. $X \otimes X$ for a bit flip on both interacting qubits; there are 16 possible channels of this type so each error occurs with probability $p/15$ and no error ($I \otimes I$) occurs with probability $1 - 15p$. 

For a general qudit system, we consider a generalized form of these errors. The ``bit-flip" type gates become $X_{+1 \text{mod } d}$ and the ``phase-flip'' errors become $Z_d = \text{diag}(1, \exp{\omega}, \exp{\omega^2}, ..., \\ \exp{\omega^{d-1}})$ where $\omega^j$ is the $j-th$ root of unity. The product of $\{I, X_{+1 \text{mod } d}, ..., X_{+1 \text{mod } d}^{d-1}\}$ and $\{I, Z_d, Z_d^2, ..., Z_d^{d-1}\}$ is a basis for all $d\times d$ Pauli matrices which allows us to construct a general symmetric qudit depolarizing channel. This explains the expected increase in error for using qudit systems: For a two-qubit gate the chance of \textit{no} error is $1 - 15p$ while for a ququart this chance diminishes to $1 - 255p$ let alone possible differences in $p$ \cite{miller_propagation_2018}. 

Amplitude damping for qubits can be described as non-unitary transformations on the quantum state with operators \\ $K_0 = \text{diag}(1, \sqrt{1 - \lambda_1})$ and $K_1 = \sqrt{\lambda_1}e_{0, 1}$. Here $e_{i,j}$ refers to a matrix with all 0's except for a $1$ in the $i$-th row and $j$-th column and is of appropriate dimension. %(here $2 \times 2$). 
In the general qudit case we have \\ $K_0 = \text{diag}(1, \sqrt{1 - \lambda_1}, \sqrt{1 - \lambda_2}, ... \sqrt{1 - \lambda_{d}})$, $K_1 = \sqrt{\lambda_1}e_{0, 1}$, ... $K_d = \sqrt{\lambda_{d}}e_{0, d-1}$. Since we primarily focus on a superconducting system in this study we take $\lambda_m = 1 - \exp{-m\Delta t / T_1}$ where $\Delta t$ is the idling duration and $T_1$ is the coherence time of the qubit \cite{khammassi_qx_2017}. 

In this work we are also concerned with the manipulation of mixed-radix systems. When drawing an error for such a system, for example a qubit-ququart interaction, we consider only relevant errors for the respective participant. For instance, a two-qudit error is drawn from $P_2 \otimes P_4$ and not from $P_4 \otimes P_4$ (where $P_d$ is the set of $d$-dimensional Paulis, exactly the set of potential errors described above). Similarly, for two-qubit gates on encoded qubits, we consider only single \textit{ququart} errors since gates on encoded systems are equivalent to single-ququart gates.
\section{Results}\label{section:results}

\subsection{\pt-differential cross section of heavy-flavour hadron decay electrons in pp and \pPb collisions}
The \pt-differential production cross section of electrons from semileptonic decays of heavy-flavour hadrons at midrapidity
in pp collisions at $\sqrt{s} = 13$ TeV measured in the transverse momentum interval $\mbox{$0.2 < p_{\rm{T}} < 35$ GeV$/c$}$ is shown in Fig.~\ref{Fig:ppHFESpectra}. The statistical uncertainties are represented as vertical lines while the total systematic uncertainties are displayed as boxes. 
In the top left panel of Fig.~\ref{Fig:ppHFESpectra}, the cross sections measured with the TPC--TOF detectors and the two different data sets collected with different magnetic fields are plotted together with the spectra obtained using the TPC--EMCal detectors with MB triggered events, as well as with EMCal triggered events, EG1, and EG2. 
The ratios of the different analyses in the overlapping \pt~intervals are shown in the bottom left panel of Fig.~\ref{Fig:ppHFESpectra}.  
For $\mbox{0.5 $<$ \pt $<$ 4~\GeVc}$, the ratio of the result from the TPC--TOF analyses with $B = $ 0.5 T to the one obtained with $B = $ 0.2 T is displayed. In 3 $<$ \pt $<$ 4 \GeVc, the ratio of the cross section obtained from the TPC--TOF analysis to that obtained from the TPC--EMCal analysis is shown for MB triggered events. At higher \pt, namely 6 $<$ \pt $<$ 10 \GeVc (12 $<$ \pt $<$ 18 \GeVc), the ratio of the TPC--EMCal results for MB and EG2 (EG2 and EG1) triggered events is reported. All ratios are consistent with unity within statistical and systematic uncertainties, which demonstrates that the different analyses are in agreement with each other. The final cross section in the \pt intervals 0.2--0.5 GeV/$c$, 0.5--4 GeV$/c$, 4--6 GeV$/c$, 6--12 GeV$/c$, and 12--35 GeV$/c$ was obtained from the TPC--TOF low-$B$ field analysis, the TPC--TOF nominal-$B$ field analysis, and from the results obtained with the TPC--EMCal detectors using MB, EG2 and EG1 triggered events, respectively. In this way, for each \pt range, the measurement with the smallest total uncertainty (quadratic sum of statistical and systematic uncertainty) is used.


The $p_{\rm{T}}$-differential cross section measurement was compared with FONLL~\cite{Cacciari:2012ny} and GM-VFNS~\cite{Bolzoni:2012kx} pQCD calculations
\footnote{${\rm Central~values:}~~{\rm FONLL:}~\mu_{\rm F}~=~\mu_{\rm R}~=~\sqrt{m_{\rm Q}^2+p_{\rm T}^2},~m_{\rm b}~=~4.75~{\rm GeV},~m_{\rm c}~=~1.5~{\rm GeV};~~{\rm GM-VFNS:}~\mu_{\rm F}~=~0.49~\mu_{\rm R},\\\mu_{\rm R}~=~\sqrt{4m_{\rm Q}^2+p_{\rm T}^2},~m_{\rm b}~=~4.5~{\rm GeV},~m_{\rm c}~=~1.5~{\rm GeV};~~{\rm where}~\mu_{\rm R}~=~{\rm renormalization~scale,}~\mu_{\rm F}~=~{\rm factorisation~scale}$},
as shown in the right panel of Fig.~\ref{Fig:ppHFESpectra}.
 The uncertainties of the FONLL calculations reflect different choices for the charm- and beauty-quark masses, and for the factorisation and renormalisation scales as well as the uncertainty on the set of parton distribution functions (PDF) (CTEQ6.6~\cite{Nadolsky:2008zw}). The FONLL calculations describe the measurements within the uncertainties, although the theoretical uncertainties are large, up to a factor of two. The data are found to be close to the upper edge of the FONLL prediction, which can be clearly seen in the right bottom panel of Fig.~\ref{Fig:ppHFESpectra}, where the ratio of the data points to the FONLL calculations is shown. 
Similar observations were made for the measurements of electrons from heavy-flavour hadron decays in pp collisions at lower energies at the LHC~\cite{Abelev:2014gla, ALICE:2012mzy, Acharya:2018upq, Acharya:2019mom} and at RHIC~\cite{STAR:2011bqq, PHENIX:2006tli}. The measurement of the cross section of D mesons is also consistent with upper bound of FONLL pQCD calculations in pp collisions at LHC~\cite{Abelev:2012vra,ALICE:2017olh,ALICE:2019nxm,Acharya:2021cqv,ATLAS:2015igt,LHCb:2015swx, LHCb:2016ikn} and RHIC~\cite{STAR:2012nbd}, as well as in $\rm{p \bar{p}}$ collisions at Tevatron energies~\cite{CDF:2003vmf}. 
The FONLL calculations use fragmentation functions tuned on e$^{+}$e$^{-}$ data and assume that all charm quarks fragment only into  D$^+$ and D$^0$ mesons (and their antiparticles). 
Recent measurements of charm-baryon production at midrapidity in pp and \pPb collisions from ALICE show a baryon-to-meson ratio significantly higher than that in e$^{+}$e$^{-}$ collisions, suggesting that the fragmentation of charm quark is not universal across different collision systems~\cite{ALICE:2021dhb,ALICE:2021npz}.
As a consequence, calculations taking properly into account the latest open-cham baryon measurements at midrapidity to constrain the charm fragmentation are expected to predict a smaller yield of heavy-flavour hadron decays by about 9\% compared to the FONLL spectrum.
The largest source of uncertainties in the GM-VFNS prediction is due to scale variation, and hence PDF related uncertainties and variations of the bottom and charm mass are not considered. The GM-VFNS framework includes leptoproduction from the following three steps: beauty quark to beauty hadrons (b $\rightarrow$ B), transition from  beauty quark to  charm hadrons (b $\rightarrow$ B $\rightarrow$ D), and charm quark to charm hadrons (c $\rightarrow$ D). The GM-VFNS calculations describe the data within the uncertainties for \pt greater than 5 \GeVc, but largely underestimate the cross section for lower \pt, up to a factor of five at 1 \GeVc, as seen in the right middle panel of Fig.~\ref{Fig:ppHFESpectra}. Similar observations were reported for the non-prompt D meson measurements at $\sqrt{s}~=~5.02~{\rm TeV}$~\cite{Acharya:2021cqv}. For prompt D mesons at $\sqrt{s}~=~5.02~{\rm TeV}$, however, the GM-VFNS predictions describe the cross section within the uncertainties~\cite{Acharya:2021cqv}. Electrons from heavy-flavour hadron decays are dominated by semileptonic decays of beauty hadrons for \pt~$>5$ ~GeV$/c$~\cite{Abelev:2012sca, ALICE:2014aev}. Therefore, the cross section measured up to 35 GeV$/c$ can provide important information to beauty hadron production. 

\begin{figure}[!ht]
\centering

\includegraphics[width=0.48\linewidth]{figures/Results/HFE_ppNormalB/HFE_Comparison_1_CR1.pdf}
\includegraphics[width=0.48\linewidth]{figures/Results/HFE_ppNormalB/Data_GMVFNS_FONLL_1.pdf}
\caption{Left, top: $p_{\rm T}$-differential cross section of electrons from heavy-flavour hadron decays in pp collisions at $\sqrt{s} =$ 13 TeV measured at midrapidity with different detectors and data sets. Left, bottom: Ratios of the different measurements in the overlapping \pt intervals. Right: $p_{\rm{T}}$-differential cross section compared with Fixed Order with Next-to-Leading-Log resummation (FONLL)~\cite{Cacciari:2012ny} and General-mass-variable-flavour-number-Scheme (GM-VFNS)~\cite{Bolzoni_2014} predictions and its ratios with respect to FONLL and GM-VFNS central values in the two lower panels. Vertical bars and boxes denote statistical and systematical uncertainties, respectively.}        
\label{Fig:ppHFESpectra}
\end{figure}

\begin{figure}[!ht]
\centering
\includegraphics[width=0.5\linewidth]{figures/Results/HFE_pPb/HFE_InvariantCrossSection_pPb_withNewRuns_CR1_v1.pdf}
\caption{Top: $p_{\rm T}$-differential cross section of electrons from heavy-flavour hadron decays in \pPb  collisions at \sqrtsNN $= 8.16~\rm TeV$ measured at midrapidity with different detectors. Bottom: Ratios of the different measurements in the overlapping \pt intervals.}        
\label{Fig:pPbHFESpectra}
\end{figure}



 The $p_{\rm{T}}$-differential production cross section of electrons from semileptonic heavy-flavour hadron decays at midrapidity in \pPb collisions at $\sqrt{s_{\rm{NN}}} = 8.16$ TeV measured in the transverse momentum interval $0.5 < p_{\rm{T}} < 26$~GeV$/c$ is shown in Fig.~\ref{Fig:pPbHFESpectra}. In the upper panel of Fig.~\ref{Fig:pPbHFESpectra}, the cross sections measured with the TPC--TOF detectors are plotted together with the measurements obtained using the TPC--EMCal detectors with MB and EMCal EG2 and EG1  triggered events. On the bottom left panel of Fig.~\ref{Fig:pPbHFESpectra}, the ratios of the cross sections obtained from the different measurements are calculated in the overlapping \pt intervals. For 3 $<$ \pt $<$ 5 \GeVc, the ratio of the result obtained from the TPC--TOF analysis with respect to that from the TPC--EMCal is shown for  MB triggered events. For 6 $<$ \pt $<$ 10 \GeVc (12 $<$ \pt $<$ 14 \GeVc) the ratio of the TPC--EMCal results obtained with MB and EG2   triggered events (EG2 and EG1) is reported. All ratios are consistent with unity within statistical and systematic uncertainties. The same strategy as in pp collisions was used to get the final cross section in  \pPb collisions. The final cross section in the \pt intervals 0.5--4 GeV$/c$, 4--6 GeV$/c$, 6--9 GeV$/c$, and 9--26 GeV$/c$ was obtained from the TPC--TOF nominal-$B$ field analysis and from the results using the TPC--EMCal detectors with MB, EG2, and EG1 triggered events, respectively.


\subsection{Nuclear modification factor of electrons from heavy-flavour hadron decays in \pPb collisions}


The nuclear modification factor of electrons from heavy-flavour hadron decays, $R_{\rm{pPb}}$, is defined as
\begin{equation}
    R_{\rm{pPb}}(p_{\rm{T}},\it{y}) = \frac{\rm 1}{ A} \frac{{\rm {d}}^2 \sigma_{\rm{pPb}}/{\rm{d}} p_{\rm{T}}\rm{d}\it{y}}{{\rm{d}}^2\sigma_{\rm{pp}}/{\rm{d}} p_{\rm{T}}\rm{d}\it{y}},
\end{equation}
where  ${\rm {d}}^2 \sigma_{\rm{pPb}}/{\rm{d}} p_{\rm{T}}\rm{d}\it{y}$ is the cross section of electrons from heavy-flavour hadron decays measured in {$\mbox{\pPb}$} collisions at \sqrtsNN $= 8.16~\rm TeV$ and ${\rm {d}}^2 \sigma_{\rm{pp}}/{\rm{d}} p_{\rm{T}}\rm{d}\it{y}$ is the cross section of electrons from heavy-flavour hadron decays in pp collisions at the same centre-of-mass energy, scaled with the number of nucleons ($A$) in the lead ion.
The reference cross section in pp collisions was obtained using the measurement at \sqrts $= 13~\rm TeV$, presented here. The cross section at \sqrts $= 13~\rm TeV$ was scaled to \sqrts $= 8.16~\rm TeV$ using pQCD calculations. The $p_{\rm{T}}$-dependent scaling factor was obtained by calculating the ratio of the production cross sections of electrons from heavy-flavour hadron decays from FONLL calculations~\cite{Cacciari:2012ny} at $\sqrt{s}= 8.16$ TeV to $\sqrt{s}=13$ TeV. The systematic uncertainty on the pp reference includes the systematic uncertainties on the measured cross section at $\sqrt{s}=13$ TeV, which was described above, and the ones on the $p_{\rm{T}}$-dependent scaling factor. The uncertainty on the scaling factor ranges between 11\% and 1\% going from \pt $= 0.2$ \GeVc to \pt $= 26$ \GeVc. This includes the uncertainties on the PDFs, quark masses, and factorisation and renormalisation scales, as described in Ref.~\citenum{Averbeck:2011ga}. The two contributions were added in quadrature leading to a total systematic uncertainty of 5-15\%, depending on $p_{\rm{T}}$. In addition, a global normalisation systematic uncertainty of 2.3\% from the pp analysis at $\sqrt{s}=13$ TeV was also considered.
The $p_{\rm{T}}$-differential cross section of electrons from heavy-flavour hadron decays in pp collisions at \sqrts $= 13~\rm TeV$ scaled to \sqrts $= 8.16~\rm TeV$ using the aforementioned procedure is shown together with the $p_{\rm{T}}$-differential cross section of electrons from heavy-flavour hadron decays in \pPb collisions at \sqrtsNN $= 8.16~\rm TeV$ in Fig.~\ref{Fig:HFESpectraComparison}.

The nuclear modification factor of electrons from heavy-flavour hadron decays as a function of transverse momentum at $\sqrt{s_{\rm{NN}}} = 8.16$ TeV is presented in Fig.~\ref{Fig:RpPb}. The statistical and systematic uncertainties of the spectra in p–Pb and pp collisions were propagated as uncorrelated. The normalisation uncertainties are shown as a solid box at $R_{\rm{pPb}} = 1$. The $R_{\rm{pPb}}$ is consistent with unity within statistical and systematic uncertainties over the whole $p_{\rm{T}}$ range of the
measurement. Modifications of the cross section of electrons from
heavy-flavour hadron decays in \pPb collisions due to different cold nuclear matter effects, are small compared to the current uncertainties of the measurement in the probed $p_{\rm{T}}$ range.
The sample of electrons from heavy-flavour hadron decays is dominated by beauty-hadron decays for  $\mbox{$p_{\rm{T}} > 5$ GeV$/c$ }$ ~\cite{Abelev:2012sca, Abelev:2014hla}. The $R_{\rm{pPb}}$ was fitted with a constant function above 5 \GeVc and the value was $\rm 0.95 \pm 0.02(stat.) \pm 0.13(sys.)$, thus consistent with unity within 13\%. The $R_{\rm{pPb}}$ of unity indicates that the beauty production is not modified in \pPb collisions within the kinematic range of this measurement, which is also consistent with the measurement of $R_{\rm{pPb}}$ of beauty-decay electrons up to $p_{\rm{T}} = 8$ GeV$/c$ at $\sqrt{s_{\rm{NN}}} = 5.02$ TeV~\cite{Adam:2016wyz}. In the right panel of Fig.~\ref{Fig:RpPb}, the $R_{\rm{pPb}}$ at $\sqrt{s_{\rm{NN}}} = 8.16$ TeV is compared with that at $\sqrt{s_{\rm{NN}}} = 5.02$ TeV and different theoretical models provided for $\sqrt{s_{\rm{NN}}} = 5.02$ TeV ~\cite{Acharya:2019hao}. The $R_{\rm{pPb}}$ is observed to be independent of the centre-of-mass energy. The data disfavour the enhancement trend at low \pt predicted by the model calculations which are based on incoherent multiple scatterings ~\cite{KANG201523}. Model predictions which are based on coherent multiple scattering and energy loss in the CNM, pQCD calculations using FONLL 
framework and EPS09NLO for the nuclear modification of the PDF, as well as calculations which assume the formation of a hydrodynamical expanding medium in \pPb collisions at $\sqrt{s_{\rm{NN}}} = 5.02$~TeV within the Blast wave framework predict an $R_{\rm{pPb}}$ close to unity and are in agreement with the measurements.

 

\begin{figure}[!ht]
\centering
\includegraphics[width=0.5\linewidth]{figures/Results/HFE_pPb/ppScaledReference_wNewRuns_CR1.pdf}
\caption{$p_{\rm{T}}$-differential cross section of electrons from heavy-flavour hadron decays measured in \pPb collisions at \sqrtsNN $= 8.16~\rm TeV$ compared with the pp reference at the same centre-of-mass energy obtained from the measurement in pp collisions at \sqrts $= 13~\rm TeV$ scaled to \sqrts $= 8.16~\rm TeV$.}        
\label{Fig:HFESpectraComparison}
\end{figure}

\begin{figure}[!ht]
      \begin{center}
      \includegraphics[width=0.48\linewidth]{figures/Results/HFE_pPb/RpPb_Updated_wNewRuns_CR1_v1.pdf}
      \includegraphics[width=0.48\linewidth]{figures/Results/HFE_pPb/RpPb_8TeV_ComparedWith5TeV_wNewRuns_CR1_v1.pdf}
      \end{center}
\caption{ The nuclear modification factor $R_{\rm{pPb}}$ of electrons from heavy-flavour hadron decays in \pPb collisions at
\sqrtsNN $= 8.16~\rm TeV$ (left) compared with that at \sqrtsNN $= 5.02~\rm TeV$ and theoretical models at \sqrtsNN $= 5.02~\rm TeV$ (right) ~\cite{Acharya:2019hao}.}    
\label{Fig:RpPb}
\end{figure}

\subsection[Self-normalised yield of electrons from heavy-flavour hadron decays vs. normalised multiplicity]{Self-normalised yield of electrons from heavy-flavour hadron decays vs. normalised multiplicity in pp and \pPb collisions}
The self-normalised yield of electrons from heavy-flavour hadron decays as a function of the self-normalised charged-particle pseudorapidity density  at midrapidity, i.e., ${\rm d}^{2} {N}/{\rm d}\pt {\rm d}{y} / \langle {\rm d}^{2}{N}/{\rm d}\pt{\rm d}{y}\rangle_{{\rm INEL>0}}$ vs. $\dnchdeta/\left<\dnchdeta\right>$, in pp collisions at $\sqrt{s} =$ 13 TeV is presented in Fig.~\ref{Fig:SelfnormalisedYield}.  The results are self-normalised to the INEL $>$ 0 event class. The measurements were performed in five \pt intervals from 0.5 to 30 GeV$/c$. 
The dashed line shown in the figure is a linear function with a slope of unity. The available data samples allow us to examine events with a multiplicity more than six times larger than the average multiplicity in pp collisions. The self-normalised yield of electrons from heavy-flavour hadron decays grows faster than linear with the self-normalised multiplicity. The measurement in intervals of \pt shows that this increase is more pronounced for high-\pt electrons. The yield of heavy-flavour decay electrons increases by approximately a factor of nine with respect to its multiplicity-integrated value for the lowest measured \pt~interval ($0.5 < \pt < 1.5$ \GeVc) and a factor of 29 for the highest measured \pt interval $\mbox{$(20 < \pt < 35\ \GeVc)$}$ for multiplicities of six times the average multiplicity. 

\begin{figure}[!ht]
\centering
\includegraphics[width=0.6\linewidth]{figures/Results/HFE_ppNormalB/SNYNo_Uncert_CR1_2.pdf}
\caption{Self-normalised yield of electrons from heavy-flavour hadron decays as a function of normalised charged-particle pseudorapidity density at midrapidity computed in \pp collisions at \sqrts $= 13~\rm TeV$ in different \pt intervals.}
\label{Fig:SelfnormalisedYield}
\end{figure}


\begin{figure}[!h]
%\centering
\includegraphics[width=0.48\linewidth]{figures/Results/HFE_ppNormalB/SNY_Pt_Ratio_0_CR1_2.pdf}
\includegraphics[width=0.48\linewidth]{figures/Results/HFE_ppNormalB/SNY_HFE_Fit_Diagonal_Linear_CR2.pdf}
\caption{ Ratio of the  self-normalised yields
in different \pt intervals with respect to that in the $6 < \pt < 12~{\rm GeV}/c$ interval (left) and double ratio of the self-normalised yields of  electrons to the self-normalised multiplicity (right) in pp collisions at $\sqrts=13$ TeV for three \pt ranges.}
\label{Fig:SNY_Ratio_pp}
\end{figure}

In the left panel of Fig.~\ref{Fig:SNY_Ratio_pp}, the ratios of the self-normalised yields of electrons from heavy-flavour hadron decays in various \pt intervals with respect to the one measured in the $6 < \pt < 12$ GeV$/c$ interval are shown. The yield of lower-$\pt$ electrons is higher in low  multiplicity events, while it decreases in higher multiplicity events. 
An opposite trend is observed for electrons at higher \pt, where the yield is lower in low  multiplicity events and increases at higher multiplicities.
The increase of the slope with \pt is influenced by the momentum dependence of jet fragmentation affecting the measured multiplicity at midrapidity, and the momentum dependence of the fraction of electrons from charm and beauty hadron decays. The relative fraction of electrons from beauty hadron decays increases with $\pt$ and becomes the main source of heavy-flavour hadron decay electrons at high \pt ($\pt > 5$ GeV$/c$)~\cite{Adam:2015ota,ALICE:2020msa,Weber:2018ddv}.




In the right panel of Fig.~\ref{Fig:SNY_Ratio_pp}, the double ratio of the self-normalised  electron yield to the self-normalised multiplicity in pp collisions is presented. The double ratio is observed to increase with multiplicity. The increase is weaker for low-\pt electrons than for high-\pt electrons. A linear function was used to fit the multiplicity dependence of the double ratio, 
which was found to describe the data reasonably well for all \pt intervals. This indicates that in the measured \pt range the yield grows approximately with the square of the multiplicity with a slope increasing with \pt.


\begin{figure}[!h]
%\centering
\includegraphics[width=0.5\linewidth]{figures/Results/HFE_ppNormalB/SNY_PYTHIA_Default_CR1_1.pdf}
\includegraphics[width=0.5\linewidth]{figures/Results/HFE_ppNormalB/SNY_PYTHIA_CR2_CR1_1.pdf}
\caption{Comparison of the self-normalised yield of electrons from heavy-flavour hadron decays as a function of multiplicity measured in \pp collisions at \sqrts $= 13~\rm TeV$ for different \pt intervals with PYTHIA 8.2 Monash tune (left) and PYTHIA 8.2 with CR mode 2 (right). The width of the band is the statistical uncertainty from PYTHIA simulations. The bottom panel shows the ratio of data with respect to the MC predictions. The vertical bars correspond to the propagated statistical error from the data and the MC predictions, and the boxes correspond to systematical uncertainties from the data. }
     \label{Fig:SelfnormalisedYield_pp_PYTHIA}
\end{figure}

The self-normalised yield of electrons from heavy-flavour hadron decays is compared in Fig.~\ref{Fig:SelfnormalisedYield_pp_PYTHIA} with PYTHIA 8.2 simulations using different tunes. 
In the PYTHIA~8.2 framework, multiparton interactions (MPI) and the colour reconnection (CR) mechanism are implemented, which reproduce the charged-particle multiplicity distribution measured at the LHC~\cite{Adam:2016mkz, ATL-PHYS-PUB-2017-008}. These mechanisms are important in order to describe the stronger than linear increase of charm and beauty production with multiplicity as demonstrated in ~\cite{Adam:2015ota}. 
The charged-particle multiplicity also includes particles directly produced in the same hard partonic scattering process in which the heavy quark is created, making them strongly related. These dependencies come from the initial- and final-state radiations, decays of heavy-flavour hadrons, and charged particles produced in the jet fragmentation and are known as auto-correlation effects.
A study of the self-normalised yield of heavy-flavour particles using the PYTHIA 8.2 generator shows that the stronger than linear increase of the yield of heavy-flavour particles is mainly driven by auto-correlation effects. In the absence of auto-correlation effects the increase of the yield of particles produced in hard scattering processes is weaker than linear for multiplicities exceeding about three times the mean multiplicity~\cite{Weber:2018ddv}.
In PYTHIA 8.2, the \pt dependence of the increase of the self-normalised yield with multiplicity is also due to auto-correlation effects introduced by the parton fragmentation because high momentum partons are accompanied by a larger number of fragments which contribute to the multiplicity. In the case of electrons from heavy-flavour hadron decays, the high-\pt part of the spectra is dominated by beauty decay electrons, whose yield was demonstrated to have a more pronounced increase with multiplicity due to the larger jet activity~\cite{Weber:2018ddv}.
In the left panel of Fig.~\ref{Fig:SelfnormalisedYield_pp_PYTHIA}, the measured self-normalised yield of electrons from heavy-flavour hadron decays is compared to calculations with the PYTHIA 8.2 Monash tune that describe the overall trend in data, but the slope is overestimated at high \pt.
In the right panel of Fig.~\ref{Fig:SelfnormalisedYield_pp_PYTHIA}, an improved tune which includes string formation beyond the leading-colour approximation i.e. PYTHIA 8.2 with  CR mode 2~\cite{Sjostrand:2014zea, Christiansen:2015yqa}, is shown to reproduce the \pt dependence,  however the slope is underestimated at high \pt.



\begin{figure}[!h]
\centering
\includegraphics[width=0.5\linewidth]{figures/Results/HFE_ppNormalB/SNY_EPOS_CR1_1.png}
\caption{Comparison of self-normalised yield of electrons from heavy-flavour hadron decays as a function of multiplicity measured in \pp collisions at \sqrts $= 13~\rm TeV$ for different \pt intervals with EPOS 3 hydro calculations. The width of the band is the statistical uncertainty from EPOS simulations. The bottom panel shows the ratio of data with respect to the MC predictions. The vertical bars correspond to the propagated statistical error from the data and the MC predictions, and the boxes correspond to systematical uncertainties from the data. The ratio for the lowest multiplicity point  for 15 $<$ \pt $<$ 30 \GeVc~(not shown in the figure) is 13$\pm$18.}
     \label{Fig:SelfnormalisedYield_pp_EPOS}
\end{figure}

Calculations with the EPOS 3 event generator~\cite{Werner:2013tya} are able to reproduce the data well, except for the highest measured \pt interval, as can be seen in Fig.~\ref{Fig:SelfnormalisedYield_pp_EPOS}. In the EPOS 3 model, the elementary scattering objects are pomerons, which are exchanged between the partons participating in the collision. The pomerons consist of a hard pQCD scattering vertex, accompanied by initial (space-like) and final (time-like) state parton emission.
The production of a hard probe is more likely from events with hard pomeron exchanges. 
This implies that for a given charged-particle multiplicity the presence of heavy-flavour hadrons favours events with fewer but harder pomerons, which leads to a stronger than linear increase of heavy-flavour production with charged-particle multiplicity. The increase also gets stronger with the increasing \pt, which, as discussed above for the case of PYTHIA 8.2 simulations, is related to the hardness of the partonic scattering and the accompanying  jet activity in the event. The subsequent hydrodynamic evolution of the system then amplifies the increase because the charged-particle multiplicity is reduced by the hydrodynamic expansion, in contrast to the heavy-flavour production.  The charged-particle multiplicity is reduced because part of the available energy goes into flow rather than particle production~\cite{Werner:2016nsq}. 


\begin{figure}[!h]
\centering
\includegraphics[width=0.45\linewidth]{figures/Results/HFE_ppNormalB/SNY_JPsi_0_CR1_2.pdf}
\includegraphics[width=0.45\linewidth]{figures/Results/HFE_ppNormalB/SNY_ChargedParticles_High_6_10_0_CR1_2.pdf}
\includegraphics[width=0.45\linewidth]{figures/Results/HFE_ppNormalB/SNY_Dmeson_0_CR1_2.pdf}
\includegraphics[width=0.45\linewidth]{figures/Results/HFE_ppNormalB/SNY_StrangeParticle_0_CR1_2.pdf}
\caption{Comparison of the self-normalised yield of electrons from heavy-flavour hadron decays measured in \pp collisions at \sqrts $= 13~\rm TeV$  with the self-normalised yields of J/$\psi$ in \pp collisions at \sqrts $= 13~\rm TeV$ (top left), charged particles in \pp collisions at \sqrts $= 13~\rm TeV$ (top right), D mesons in \pp collisions at \sqrts $= 7~\rm TeV$ (bottom left) and strange particles in \pp collisions at \sqrts $= 13~\rm TeV$ (bottom right), in comparable \pt bins.}
\label{Fig:SNY_CompOtherPart_pp}
\end{figure}


The trend of the self-normalised yield of electrons in pp collisions as a function of self-normalised multiplicity is compared in Fig~\ref{Fig:SNY_CompOtherPart_pp} with the self-normalised yield of other particles measured by the ALICE Collaboration, namely J/$\psi$~\cite{ALICE:2020msa}, charged particles~\cite{Acharya:2019mzb}, strange hadrons~\cite{Acharya:2019kyh} in pp collisions at $\sqrt{s}$ $= 13~\rm TeV$, and D mesons~\cite{Adam:2015ota} in pp collisions at $\sqrt{s}$ $= 7~\rm TeV$. The self-normalised yields for strange hadrons were calculated using the multiplicity-dependent cross section measurements reported in~\cite{Acharya:2019kyh}. These self-normalised yields allow a direct comparison of multiplicity-dependent production of different particle species, with the advantage that the charged-particle pseudorapidity density is measured using the same detector and procedure. 
The \pt ranges of electrons are selected to be similar to the measured \pt range of the compared particles, with a caveat that the \pt interval of electron parents (heavy-flavour hadrons) is considerably broader and shifted towards higher \pt values compared to the
one of the electrons. The slope of the increase of the self-normalised yield of electrons from heavy-flavour hadron decays as a function of self-normalised multiplicity at midrapidity is similar to that measured for J/$\psi$, charged particles, strange mesons, and D mesons in similar \pt ranges. 
At high and intermediate \pt, the production of hadrons is dominated by hard partonic scattering processes, independent of the particle species, accompanied by jet activity in the event. For heavy-flavour particles this is also true at low \pt due to the large charm and beauty quark masses. As it was discussed above, in PYTHIA 8.2, the particle production associated with jet activity leads to strong auto-correlation effects, which give rise to the observed stronger than linear increase of particle yields, making the self-normalised yield of the different particles reported here compatible with each other.

The self-normalised yield of electrons from heavy-flavour hadron decays 
as a function of the self-normalised charged-particle pseudorapidity density 
  for \pPb collisions at $\mbox{\sqrtsNN $= 8.16~{\rm TeV}$}$ is presented in Fig~\ref{Fig:SelfnormalisedYield_pPb}. The results are self-normalised to the INEL$>0$ event class, similarly to pp collisions. The dashed line is a linear function with a slope of unity as shown in the figure. The measurements were performed in five $p_{\rm T}$ intervals from 0.5 \GeVc to 26 \GeVc. 
Events with multiplicity more than four times larger than the average multiplicity in \pPb collisions are studied. 
The self-normalised yield of electrons from heavy-flavour hadron decays grows faster than linear with the self-normalised multiplicity. The measurements in $\pt$ intervals show no $\pt$ dependence within the uncertainties of the measurement. The yield increase is approximately a factor of seven for multiplicities four times larger than the average multiplicity.

\begin{figure}[!ht]
\centering
\includegraphics[width=0.6\linewidth]{figures/Results/HFE_pPb/HFESelfNormalisedYield_pPb8TeV_CR2.pdf}

\caption{Self-normalised yield of electrons from heavy-flavour hadron decays as a function of self-normalised charged-particle pseudorapidity density at midrapidity measured in \pPb collisions at \sqrtsNN $= 8.16~\rm TeV$ in different \pt intervals. The position of the points on the $x$-axis are shifted horizontally by $\rm \delta x$  to improve the visibility.}
\label{Fig:SelfnormalisedYield_pPb}
\end{figure}

In the left panel of Fig.~\ref{Fig:SNY_Ratio_pPb}, the ratios of the self-normalised yield of electrons from heavy-flavour hadron decays 
in various \pt intervals with respect to the one measured in the 3 $ < \pt < $ 6 \GeVc interval are shown. Contrary to the pp collision case,  within the uncertainties no \pt dependence is observed. The right panel of Fig.~\ref{Fig:SNY_Ratio_pPb} shows the double ratio of the self-normalised heavy-flavour hadron decay electron yield to the self-normalised multiplicity. The double ratio increases with multiplicity, with no dependence on \pt. The double ratio was fitted with a linear function, which reasonably describes the data for all \pt intervals. This indicates that in the measured \pt range the yield increases approximately with the square of the multiplicity with a similar coefficient for all \pt intervals.

\begin{figure}[!h]
%\centering
\includegraphics[width=0.48\linewidth]{figures/Results/HFE_pPb/pT_Ratio_pPb_8TeV_CR1_v1.pdf}
\includegraphics[width=0.48\linewidth]{figures/Results/HFE_pPb/Double_Ratio_pPb_8TeV_CR2.pdf}

\caption{Ratio of the  self-normalised yield
in different \pt intervals with respect to that in the $3 < \pt < 6~{\rm GeV}/c$ interval (left). Double ratio of the self-normalised yield of heavy-flavour hadron decay electrons to the self-normalised multiplicity in \pPb collisions at \sqrtsNN $= 8.16~\rm TeV$ in three \pt ranges (right).}
\label{Fig:SNY_Ratio_pPb}
\end{figure}

Though the self-normalised yields of electrons from heavy-flavour hadron decays  in pp and \pPb collisions show similar features in their increase with multiplicity, a quantitative comparison of the measurements between the two systems is not straightforward. In pp collisions, a high multiplicity event arises mostly from hard events, with multiparton interactions and jets fragmenting in multiple hadrons.  In \pPb collisions, the multiplicity dependence of heavy-flavour production is also driven by the presence of multiple binary nucleon--nucleon interactions, which make the contribution from possible auto-correlation effects smaller in such collisions. In \pPb collisions, an event with a high multiplicity value similar to those in pp collisions can come from the superposition of a few soft nucleon--nucleon collisions. Therefore, for similar multiplicity, the hardness of the event is not the same in the two systems.
 

The self-normalised yield of electrons from heavy-flavour hadron decays is compared in Fig.~\ref{Fig:SNYComparisonWithEPOS_pPb8TeV} with EPOS 2.592 simulations ~\cite{Werner:2013tya, Werner:2010aa}. The measurements in two \pt intervals are compared with the EPOS model without the hydrodynamic component, as provided by the authors. The EPOS model shows no \pt dependence similar to the observations in the data, but underpredicts the data at high multiplicity, showing an almost linear increase. 

\begin{figure}[!h]
\centering
\includegraphics[width=0.55\linewidth]{figures/Results/HFE_pPb/SelfNormaliseYieldHFEwEPOS_CR1_v2.pdf}
\caption{ Self-normalised yields of electrons from heavy-flavour hadron decays as a function of self-normalised charged-particle pseudorapidity density at midrapidity measured in \pPb collisions at \sqrtsNN $= 8.16~\rm TeV$ compared with EPOS 2.592 without-hydrodynamics in two \pt intervals 0.5 $< \pt <$ 3 \GeVc and 3 $< \pt <$ 6 \GeVc. The width of the band is the statistical uncertainty from EPOS simulations. The bottom panel shows the ratio of data with respect to the MC predictions. The vertical bars correspond to the propagated statistical uncertainties from the data and the MC predictions, and the boxes correspond to systematical uncertainties from the data.}
\label{Fig:SNYComparisonWithEPOS_pPb8TeV}
\end{figure}

The self-normalised electron yields in \pPb collisions in different \pt ranges are also compared with the normalised yields of D mesons~\cite{Adam:2016mkz} in \pPb collisions at \sqrtsNN $= 5.02~\rm TeV$ in Fig.~\ref{Fig:SNP_CompOtherPart_pPb}. Similar to the observation in pp collisions, the self-normalised yield of electrons from heavy-flavour hadron decays as a function of the self-normalised multiplicity shows a trend compatible with the one of D mesons. Also the multiplicity dependence of D meson yields in \pPb collisions does not show a \pt dependence, which gives a hint that the production mechanisms of charm and beauty as a function of the multiplicity in \pPb collisions are similar.

\begin{figure}[!ht]
\centering
\includegraphics[width=0.55\linewidth]{figures/Results/HFE_pPb/HFESelfNormaliseCompDmeson_CR2.pdf}
\caption{Self-normalised yields of electrons from heavy-flavour hadron decays measured in \pPb collisions at \sqrtsNN $= 8.16~\rm TeV$ for different \pt intervals compared with self-normalised yields of D mesons in \pPb collisions at \sqrtsNN $= 5.02~\rm TeV$. The position of the points for \pPb collisions at \sqrtsNN $= 5.02~\rm TeV$ on the $x$-axis are shifted horizontally by $\rm \delta x$ to improve the visibility.}
      \label{Fig:SNP_CompOtherPart_pPb}
\end{figure}






\section{Related Work}

While this work is the first we are aware of to explore ququart-based execution, there have been studies using existing superconducting qubit technology to execute native three qubit gates.  In particular, Kim et al. \cite{kim_high-fidelity_2022} and Gokhale et al. \cite{gokhale_quantum_2020} have explored driving two connections between three qubits in a line to perform three qubit gates in a superconducting architecture.  Gokhale developed a technique to execute two CX gates in parallel in a single CXX gate.  These gates did not find improved fidelity, but achieved similar goals of faster parallel gate execution than serial execution as described in this work.  It should be noted that this was done without explicit calibration for this sort of operation.  While this work may seem similar through the application of multiqubit gates across many qubits, it mainly focuses on non-superconducting devices, which are able to make use of a global operator gate.  It touches on performing three-qubit gates on superconducting devices, but is unable to generate gates that are more successful than the serialized decomposition. Our work focuses more on superconducting devices and direct synthesis, and must also contend with the issue of communication.

Kim et al. \cite{kim_high-fidelity_2022} developed a 98.2\% fidelity iToffoli gate between three superconducting qubits.  In this case, both controls induce a state change in a center qubit by driving the connections between the qubits.  This gate is performed very quickly with a gate duration of 392 ns.  While an impressive result, it is difficult to compare this work to our own as the device has a substantially different Hamiltonian than this work assumes.  Additionally, this required significant manual calibration between each of the three qubits, a process which may not scale well to larger systems. This work is the basis for our $i$Toffoli baseline, which we found to be similar in performance to the mixed-radix strategy.  However, the computational complexity of generating these pulses is much higher, requiring additional optimizable controls and a larger simulated Hilbert space (when taking into account the simulation of additional ``guard'' energy levels).  Additionally, the mixed-radix scheme presented here only requires calibration between each pair of qudits, which is similar to the processes already in use on quantum computers. The $i$Toffoli scheme would require calibration between each trio of qubits, which would add a significant additional overhead.

There have also been several physical realizations of the $i$Toffoli gate that use the $i$SWAP gate, a CPHASE gate, and a reverse $i$SWAP gate using the $\ket{2}$ state to change the state of a qubit in \cite{hill_realization_2021, fedorov_implementation_2011}. Galda et. al. \cite{galda_implementing_2021} explores using qutrits on IBM's Jakarta device to implement a Toffoli gate with 78\% fidelity.  This is similar to our work, using the more accessible $\ket{2}$ state to perform a three-qubit gate.  These are conceptually similar to the encode, mixed-radix Toffoli, and decode scheme that was laid out in this work. We believe that a machine specifically designed with qudits in mind could enable much higher fidelities for similar experiments.


\section{Conclusion}
The architecturally imposed requirement to decompose more complex three-qubit gates into component one- and two-qubit gates is an extreme hurdle for realizing quantum computing.  Decomposing these gates increases both the number of error-prone gates that need to be executed, and the execution time of the circuit on devices with short coherence times.  However, many architectures have access to higher level states beyond the traditional two-level system.  While more prone to decoherence and error, this extra computational space can be used to compress quantum data, encoding two qubits into one physical device called a ququart.

This work takes advantage of increased connectivity and interaction potential when we have encoded qubits into a four-level system.  Encoding qubits in this way allows for the interaction of three to four qubits across a single physical connection, and we synthesize a library or efficient three-qubit gates via optimal control that take advantage of this virtual connectivity and are much faster and higher fidelity than performing the decomposition of a three-qubit gate.  We also demonstrate the viability of this encoding scheme and gate set via the execution of a $H \otimes H$ gate on real superconducting hardware. We then use these gates to develop compilation strategies, the quantum waltz, that use the most efficient configurations of three-qubit gates on mixed-radix and full-ququart systems to produce circuits that achieve 2x and 3x better simulated fidelities in mixed-radix and full-ququart environments, respectively compared to two-qubit based strategies.  We also demonstrate that ququart-based gates are a viable alternative to $i$Toffoli based three-qubit pulse strategies with potential practical upsides. Despite the difficulty of accessing and performing operations on higher level states, this efficient implementation of three-qubit gates provides worthwhile benefits for quantum computation.

Mixed-radix and full-ququart implementations of three-qubit gates makes ququart computation an invaluable piece of the quantum computing repertoire. It is more flexible than previous hand optimized circuits to improve circuit execution via higher radix devices, does not require the use of quaternary-based logic, and can be selectively applied to certain sections of larger circuits.  Realized implementations of these gates provide a massive opportunity to improve near-term execution of quantum circuits and expand the capabilities of quantum computers.

% Reiterate iToffoli in conclusion

% Acknowledge RCC


%%%%%%% -- PAPER CONTENT ENDS -- %%%%%%%%
\begin{acks}
This work is funded in part by EPiQC, an NSF Expedition in Computing, under award CCF-1730449; in part by STAQ under award NSF Phy-1818914; in part by NSF award 2110860; in part by the US Department of Energy Office of Advanced Scientific Computing Research, Accelerated Research for Quantum Computing Program; and in part by the NSF Quantum Leap Challenge Institute for Hybrid Quantum Architectures and Networks (NSF Award 2016136) and in part based upon work supported by the 
U.S. Department of Energy, Office of Science, National Quantum Information Science Research Centers.  FTC is Chief Scientist for Quantum Software at ColdQuanta and an advisor to Quantum Circuits, Inc.

We would like to thank Casey Duckering for his input in early discussion of compiler development for ququarts. We would like to thank Stefanie Günther and N. Anders Petersson for valuable advice on using the quantum optimal control software packages Juqbox and Quandary.

This work was completed in part with resources provided by the University of Chicago’s Research Computing Center.
\end{acks}

%%%%%%%%% -- BIB STYLE AND FILE -- %%%%%%%%
\bibliographystyle{ACM-Reference-Format}
\balance
\bibliography{references3}
%%%%%%%%%%%%%%%%%%%%%%%%%%%%%%%%%%%%

\end{document}

