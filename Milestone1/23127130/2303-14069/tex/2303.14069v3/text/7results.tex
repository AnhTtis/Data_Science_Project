\section{Results}

When we are able to perform native implementations of three-qubit gates via ququarts, we significantly reduce the number of gates that need to be executed, reducing failure rate.  However, the reduction in time to execute these gates may not be enough to overcome the reduced coherence time of higher radix states. In Figure \ref{fig:main-simulation-results}a-d, we examine the simulation fidelities for three-qubit compilation strategies across different sized circuits using Toffoli gate based decompositions.  Each point represents the average fidelity of 1000+ different initial states run once, with randomly inserted error.  The error bars are the standard error, which is the standard deviation of the all the trials divided by the square root of the number of trials.  The mixed-radix compilation schemes stop at 12 qubits due to memory-based computational limitations.  While mixed-radix circuits start in an all-qubit state, we must model them as if they are entirely on ququarts, since we must be able to model the higher levels at all times.  This restricts the number of physical devices we are able to simulate in this scheme to 12 ququarts.

\begin{figure}
    \centering
    \scalebox{0.85}{
    \includegraphics[width=\linewidth]{svg-inkscape/circuit-stats-cnx_log.pdf}
    }
    \caption{EPS statistics for the generalized Toffoli circuit.  We show the gate and coherence EPS on the left and the product EPS on the right.}
    \label{fig:circuit-estimation}
    \vspace{-1em}
\end{figure}

The first difference is that all of our mixed-radix and full-ququart compilation strategies exceed the fidelity of our baseline two-qubit gate, qubit-only compilation scheme.  From a pure gate error perspective, this should not be unexpected, each of these schemes greatly reduces the number of gates required to execute the same operation. Figure \ref{fig:circuit-estimation} demonstrates how the EPS for gate error is substantially improved by using three two-ququart gates gates or one two-ququart gate for full-ququart computation.  And, as hoped, the simulation finds that the idle time potentially spent in the $\ket{2}$ and $\ket{3}$ state does not outweigh the benefits of the using fewer gates and the shorter circuit duration.  The shorter duration of the gates counteracts the increased decoherence rate of the ququarts. Figure \ref{fig:circuit-estimation} also demonstrates the same point. The coherence EPS between all of the mixed-radix strategies and the qubit-only baseline are nearly the same, and is improved for full-ququart strategies. The general trend of our simulation results is mirrored in Figure \ref{fig:circuit-estimation} where the total EPS of the circuit is shown.  As the EPS trends match what we find in our simulated results, we are able to infer that the scaling of simulation will match the scaling of the EPS results.  While we only show examples of the generalized Toffoli circuit, the results are similar for other circuits.  However, we note that the mixed-radix strategies only marginally outperform or match the simulated results of the qubit-only $i$Toffoli based strategy. This makes sense when we examine the $i$Toffoli based decomposition.  We must insert an extra SWAP gate to perform the corrective Controlled-S gate, resulting in a similar number of gates, and duration, for both decompositions.  Additionally, this SWAP may result in extra corrective SWAPs later on. The extra communication disrupts the layout of the circuit further than was intended, and can require extra gates.

\begin{figure*}
    \centering
    \scalebox{0.85}{
    \includegraphics[width=\linewidth]{svg-inkscape/sensitivity-simulation.pdf}
    }
    \vspace{-0.5em}
    \caption{The results of several sensitivity studies. (a) Sensitivity in simulation by using CSWAP gates in different orientations instead of decomposing to Toffoli gates. (b) Changes in CCZ compilation strategies' fidelities as gate error ququarts increases.  (c) Changes in CCZ compilation strategies' fidelities as coherence error for the $\ket{2}$ and $\ket{3}$ level states changes. (d) Differences in fidelities between mixed-radix and full-ququart compilation strategies as the distribution of CX gates to CCX gates in a circuit changes. In all graphs The black line represents the qubit-only fully-decomposed compilation method.  The red line represents the qubit-only iToffoli-based decomposition. Below those points mixed-radix or full-ququart methods are more error prone than using only qubits. Please note the different scaling on the y-axis.}
    \label{fig:sensitivity-studies}
    \vspace{-0.5em}
\end{figure*}

Digging into the difference between the higher-radix strategies, we find that the mixed-radix strategies are all relatively similar to one another, with some additional separation as the size of the circuit increases.  We first compare the mixed-radix Hadamard corrected CCX gates, shown in light blue, to the mixed-radix gates without this correction, shown in pink.  While there is some cancellation between the single qubit gates, the extra serialization and marginal gate error of the correction gates is a drawback to using this correction strategy based on the simulated results. The reduction in time from the better configuration of the CCX gate is not always enough to overcome these additional costs.  If we instead use CCZ decomposition, shown in green, we consistently achieve the same, or better, fidelity, especially as the size of the circuit increases and the reduction from CCZ gates is more pronounced.  In these cases, the benefits found by using shorter target-independent gates from the start rather than retargeting improves the fidelity of the circuit in this mixed-radix regime.  In Figure \ref{fig:main-simulation-results}e we find that the mixed-radix gates achieve 2x better fidelities for circuit size 12, which is a significant improvement over two-qubit gate computation.  This alone would be a important optimization for three-qubit gate based circuits.

We find that the ququart compilation scheme, shown in grey, has higher fidelity improvement, up to 3x reductions as seen in Figure \ref{fig:main-simulation-results}e, and 50\% improvement over the $i$Toffoli baseline and mixed-radix strategy.  The reasoning behind this is two-fold.  The first is that we no longer need to encode and decode gates before each three-qubit gate in this scheme reducing gate error.  Gate reduction is important, and this reduces the number of gates.  The second is reduction of communication.  With the higher connectivity at all times, we reduce the qubit communication required to perform certain gates. Both of these factors add to the reduction in time, keeping the full-ququart based circuits under the coherence limits and maintaining higher circuit fidelity.  We further reduce the overall circuit time by using faster, target-independent CZ gates in place of CX.

There are cases where full-ququart compilation does not outperform mixed-radix compilation to the same degree.  For instance, the QRAM circuit.  There are more than double the CX gates as Toffolis in this circuit.  The serialization induced by ququarts with slower two-qubit gates reduces the effectiveness of ququarts. Additionally, these benchmarks are only kernels of computations that could be used within the context of larger circuits.  In such cases, we will not have the benefit of a perfect mapping to start.  This would not affect the improvement in fidelity from the qubit only to the mixed-radix strategies, but the effort to encode the qubits into a full-ququart regime before execution may outweigh the benefits.

\subsection{Special Gate Case Study: CSWAP}
As detailed in Section \ref{sec:multitarget} we could instead decompose to a different-three qubit gate in the circuit.  In the case of QRAM, this is the CSWAP gate.  In Figure \ref{fig:sensitivity-studies}a, we explore the differences in fidelity when we use CSWAP gate alongside the original results using CCZ gates.  A CSWAP can be constructed from two CX gates and one CCX gate, but cannot be re-targeted in the same way.  Regardless, in the mixed-radix state, by orienting the CSWAP such that the targets are separate from the controls when possible and like qubits are with like, we see improvements over the CCZ decomposition.  In fact it is able to beat the full-ququart CCZ compilation in some cases because of the reduced number of CX gates.  While we can always attempt to encode the qubits favorably in a mixed-radix environment, this is not as natural a change when compiling for ququarts, and could lead to bad configurations if we solely focus on the disruption of qubits on ququarts.  If we focus on the CSWAP in a full-ququart regime, the basic version shown in blue, and instead use the strategy that places the targets in the same ququart, shown in bright pink, we find even more improvement to our full-ququart encoding regime.  This further indicates the importance using the best decomposition possible by separating the targets and the controls of certain gates.

%\begin{figure}
%    \centering
%    \scalebox{0.95}{
%    \includesvg[width=\linewidth]{figures/cuccaro-gate-simulation.svg}
%    }
%    \caption{Changes in CCZ compilation strategies' fidelities as gate error ququarts increases.  The black line represents the %qubit-only compilation method.  Below this point mixed-radix or full-ququart methods are more error prone than using only qubits.}
%    \label{fig:gate-error-sensitivity}
%\end{figure}
\subsection{Sensitivity to Ququart Gate Error Rate}
While we synthesized our gates using a realistic Hamiltonian, it is still more difficult to physically realize gates that access higher energy levels.  In Figure \ref{fig:sensitivity-studies}b we explore how the simulated fidelity changes as the error on ququart and mixed-radix gate increases for an 11-qubit Cuccaro Adder.  Both strategies see a very fast drop off as the gate error increases, crossing over the qubit-only baseline fidelity when the ququart error rate is between two and four times worse than qubit gates for mixed-radix compilation (97\% fidelity), and between four and six times worse for full-ququart compilation (94\%).  We also find that the $i$Toffoli strategy outperforms the full-ququart strategy at three times worse ququart gates than qubit gates as well.  While these are still high fidelity targets, it does indicate that we do not need our three-qubit gates exceed the fidelity of two-qubit gates for these strategies to be successful.

%\begin{figure}
%    \centering
%    \scalebox{0.95}{
%    \includesvg[width=\linewidth]{figures/qram-coherence-simulation.svg}
%    }
%    \caption{Changes in CCZ compilation strategies' fidelities as coherence error for the $\ket{2}$ and $\ket{3}$ level states changes.  The black line represents the qubit-only compilation method.  Below this point mixed-radix or full-ququart methods are more error prone than using only qubits.}
%    \label{fig:coherence-error-sensitivity}
%\end{figure}
\subsection{Sensitivity to Ququart Coherence Error Rate}
In this work we selected the expected theoretical decrease in coherence time as the grounding for most of our simulation experiments.  However, physical realizations don't always meet reality.  Accessing higher energy levels may prove to be more costly in terms of coherence time due to lack of control, or it may be less of an issue as has been found by some testbeds when accessing the qutrit state \cite{cervera-lierta_experimental_2022}.  In Figure \ref{fig:sensitivity-studies}c we demonstrate the effects of changing the rate that the $\ket{2}$ and $\ket{3}$ levels decohere for an 12-qubit QRAM circuit.  The main detail to note is that as the rate increases, the distance between mixed-radix and full-ququart fidelities decreases until mixed-radix becomes higher fidelity.  Mixed-radix gates do not spend as much time in the higher level states, so as machines are developed and these level are more unstable, it may be better to avoid using full ququart encodings for larger circuits.

%\begin{figure}
%    \centering
%    \scalebox{0.95}{
%    \includesvg[width=\linewidth]{figures/random_distribution-distribution-simulation.svg}
%    }
%    \caption{Differences in fidelities between mixed-radix and full-ququart compilation strategies as the distribution of CX gates to CCX gates in a circuit changes.}
%    \label{fig:cx-vs-ccx-distribution-sensitivity}
%\end{figure}
\subsection{Ratio of Three Qubit Gates to Two Qubit Gates}
It may not always be the case that the number of three qubit gates greatly exceeds the number of two qubit gates. Future applications may have a higher mix of two-qubit gates to three-qubit gates, or may only require a few three qubit gates to perform the desired operation.  In Figure \ref{fig:sensitivity-studies}d, we example how the fidelity of different mixes of two-qubit to three-qubit gates is effected by compilation using a full-ququart strategy versus a mixed-radix strategy for an 11-qubit circuit.  As the ratio of two-qubit to three-qubit gates increases, it becomes less and less profitable to use a full ququart encoding.  At 60\% CX gates it becomes more profitable to remain in the mixed-radix regime. Using CX gates on ququarts requires more serialization, since we cannot perform two separate operations on qubits encoded in the same ququart.  This increases the time, and we start seeing the effects of reduced coherence times.  This changes the calculus about when mixed-radix is better than full-ququart compilation. In cases where we don't need to use as many three qubit gates, it does not make as much sense to use ququarts for the entirety of the circuit.  While this indicates that quantum circuits that only use two-qubit gates do not benefit from this encoding scheme, we can use resynthesis tools \cite{younis_berkeley_2021} to automatically insert three-qubit gates into the circuit, such as in \cite{patel_geyser_2022}.  However, resynthesis can introduce additional error as a perfect direct translation is not always possible and is better explored in a future work. We also include the $i$Toffoli strategy in this analysis as well. We find that it matches the mixed-radix strategy, further solidifying that these strategies have similar performance characteristics.