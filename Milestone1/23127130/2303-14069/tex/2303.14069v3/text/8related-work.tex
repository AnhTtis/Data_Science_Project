\section{Related Work}

While this work is the first we are aware of to explore ququart-based execution, there have been studies using existing superconducting qubit technology to execute native three qubit gates.  In particular, Kim et al. \cite{kim_high-fidelity_2022} and Gokhale et al. \cite{gokhale_quantum_2020} have explored driving two connections between three qubits in a line to perform three qubit gates in a superconducting architecture.  Gokhale developed a technique to execute two CX gates in parallel in a single CXX gate.  These gates did not find improved fidelity, but achieved similar goals of faster parallel gate execution than serial execution as described in this work.  It should be noted that this was done without explicit calibration for this sort of operation.  While this work may seem similar through the application of multiqubit gates across many qubits, it mainly focuses on non-superconducting devices, which are able to make use of a global operator gate.  It touches on performing three-qubit gates on superconducting devices, but is unable to generate gates that are more successful than the serialized decomposition. Our work focuses more on superconducting devices and direct synthesis, and must also contend with the issue of communication.

Kim et al. \cite{kim_high-fidelity_2022} developed a 98.2\% fidelity iToffoli gate between three superconducting qubits.  In this case, both controls induce a state change in a center qubit by driving the connections between the qubits.  This gate is performed very quickly with a gate duration of 392 ns.  While an impressive result, it is difficult to compare this work to our own as the device has a substantially different Hamiltonian than this work assumes.  Additionally, this required significant manual calibration between each of the three qubits, a process which may not scale well to larger systems. This work is the basis for our $i$Toffoli baseline, which we found to be similar in performance to the mixed-radix strategy.  However, the computational complexity of generating these pulses is much higher, requiring additional optimizable controls and a larger simulated Hilbert space (when taking into account the simulation of additional ``guard'' energy levels).  Additionally, the mixed-radix scheme presented here only requires calibration between each pair of qudits, which is similar to the processes already in use on quantum computers. The $i$Toffoli scheme would require calibration between each trio of qubits, which would add a significant additional overhead.

There have also been several physical realizations of the $i$Toffoli gate that use the $i$SWAP gate, a CPHASE gate, and a reverse $i$SWAP gate using the $\ket{2}$ state to change the state of a qubit in \cite{hill_realization_2021, fedorov_implementation_2011}. Galda et. al. \cite{galda_implementing_2021} explores using qutrits on IBM's Jakarta device to implement a Toffoli gate with 78\% fidelity.  This is similar to our work, using the more accessible $\ket{2}$ state to perform a three-qubit gate.  These are conceptually similar to the encode, mixed-radix Toffoli, and decode scheme that was laid out in this work. We believe that a machine specifically designed with qudits in mind could enable much higher fidelities for similar experiments.

