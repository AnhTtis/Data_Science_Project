\begin{figure}
    \centering
    \includegraphics[width=0.7\linewidth]{figures/3qb_connectivity.pdf}
    \caption{Visualization of connectivity advantages in qubit-ququart systems. Encoding qubits in ququarts (light blue) enables triangle connectivity between triples of qubits, where two of which are encoded in the same ququart and one appears either in a bare qubit or encoded in a neighboring ququart.}
    \label{fig:mixed-radix-and-full-encoded}
\end{figure}

\section{The Quantum Waltz: Three Qubit Gates on Ququarts}

Three-qubit gates are widely used in arithmetic operations, such as the Cuccaro adder \cite{cuccaro_new_2004} and multi-controlled-CNOT \cite{barenco_elementary_1995}, as smaller pieces in larger quantum algorithms such as \cite{grover_fast_1996}.  While QAOA and VQE see more use in the current quantum algorithm space, some QAOA based algorithms still use three-qubit gates \cite{hadfield_quantum_2019}.  Additionally, some error correction schemes make heavy use of three-qubit gates \cite{yoder_universal_2017}.  Current hardware platforms typically decompose these gates. Using higher radix we can reduce gate times mitigating the issue of reduced coherence times of higher-energy levels enabling more efficient execution of quantum circuits containing these gates.

\subsection{Connectivity Advantage}

The set of two qubit gates laid out in Section \ref{sec:qubit-gates-on-ququarts} are enough to universally perform general qubit computation on ququarts \cite{luo_universal_2014, nielsen_quantum_2011}, but simply compiling to two-qubit gates would not take advantage of the flexibility of this abstraction. When we encode two qubits in a ququart, we virtually increase the connectivity between qubits, see Figure \ref{fig:mixed-radix-and-full-encoded}.  Each of the encoded qubits in a ququart is connected to an adjacent qubit, or both of the encoded qubits in an adjacent ququart. As highlighted by each of the different colors this creates many triangle subgraphs between encoded qubits . Triangle subgraphs are uncommon in current hardware due to the increased probability of crosstalk \cite{mundada_suppression_2019, ding_systematic_2020}. But, triangle-based interactions are common in many different circuits that use three-qubit gates.  Here, we increase the number of virtual connections without increasing number of physical connections to create four interactions between encoded qubits.

\begin{figure}
    \centering
    \includegraphics[width=0.85\linewidth]{figures/state_evs.pdf}
    \caption{Visualization comparing the evolution of a $\ket{3}$-controlled $X$ gate in a mixed-radix environment for a CCX gate in (a) and a CX gate in (b).}
    \label{fig:state-evolutions}
\end{figure}

It is not fundamentally harder to interact three or four qubits worth of information than two qubits worth with a single operation on ququarts. These gates are equivalent to either mixed-radix or full-ququart gates. For example, if we have a fully encoded ququart next to a bare qubit and perform a Toffoli gate targeting the qubit, it is equivalent to a $\ket{3}$-controlled X  gate on the qubit. This is computationally simpler than the several $\ket{1}$- and $\ket{3}$-controlled X required in the decomposition and can be seen in the state evolutions in Figure \ref{fig:state-evolutions}. This gate implementation gives superconducting qubits more natural access to the native multi-qubit gates,  avoids decompositions that add extra gates and performs three-qubit interactions between two physical quantum devices, reducing the complexity of implementing such a three-qubit pulse across three devices and two couplers. Used in conjunction with the previously generated one- and two-qubit gates, we can more efficiently perform circuits that include three-qubit gates.

\subsection{Generated Pulses}

\begin{table}[htbp]
    \centering
    \caption{Mixed-Radix and Full-Ququart Three-Qubit Gate Durations}
    \renewcommand{\arraystretch}{1.2}
    \begin{tabu}{l r|[2pt]l r|l r}
        \multicolumn{2}{c|[2pt]}{\textbf{(a) Mixed-Radix (ns)}} & \multicolumn{4}{c}{\textbf{(b) Full-Ququart (ns)}} \\
        \hline
        CCX$^{q01}$ & 619 & CCX$^{01,0}$ & 536 & CCX$^{01,1}$ & 552 \\
        CCX$^{1q0}$ & 697 & CCX$^{0,01}$ & 785 & CCX$^{0,10}$ & 785 \\
        CCX$^{01q}$ & 412 & CCX$^{1,10}$ & 785 & CCX$^{1,01}$ & 680 \\
        \hline
        CCZ$^{01q}$ & 264 & CCZ$^{01,0}$ & 232 & CCZ$^{01,1}$ & 310 \\
        \hline
        CSWAP$^{01q}$ & 684 & CSWAP$^{01,0}$ & 680 & CSWAP$^{01,1}$ & 744 \\
        CSWAP$^{10q}$ & 762 & CSWAP$^{10,0}$ & 758 & CSWAP$^{10,1}$ & 822 \\
        CSWAP$^{q01}$ & 444 & CSWAP$^{0,01}$ & 510 &  CSWAP$^{1,01}$ & 432
    \end{tabu}    
    \label{tab:three-qubit-gate-times}
\end{table}


\subsubsection{Multi-control Gates}
Native three-qubit gates on two physical units have the potential to offer a significant improvement in gate fidelity and execution time.  In Table \ref{tab:three-qubit-gate-times}, we show pulse durations of the three-qubit Toffoli gate in several mixed-radix and full-ququart configurations. These gates were synthesized using the same fidelity targets and pulse generation techniques as the two-qubit gates, a higher fidelity than if decomposed with many gates of the same target fidelity. After synthesizing the different configurations of Toffoli gates, we find that there is a substantial difference in the gate duration depending on which qubits are controls and which is the target.
% as it relates to the configuration of the controls and the target of the Toffoli gate across the three or four qubits encoded in two physical units.

Consider the mixed-radix example where both control qubits are encoded in the same ququart, and the target qubit is in the bare qubit, or the $\text{CCX}^{01q}$ gate, seen in Figure \ref{fig:three-qubit-configs}a.  This configuration is about two-thirds the time of the $\text{CCX}^{0q1}$, seen in Figure \ref{fig:three-qubit-configs}b, where the control qubits are split across the bare qubit and the ququart.  The reason for this difference is twofold.  The first follows from the two-qubit only gates. Gates which use the ququart as a control and the qubit as a target are generally faster, the pulse only induces population changes between the $\ket{0}$ and $\ket{1}$ state of the qubit, rather than between $\ket{0}$ and $\ket{1}$, and $\ket{2}$ and $\ket{3}$ of the ququart.  The second is that the entire ququart acts as the control, only changing the state of the bare qubit if the ququart is in the $\ket{3}$ state.  In the split-control case, the ququart must control on both the $\ket{2}$ and $\ket{3}$ state.

The same concept of separation of controls and targets follows for the full-ququart Toffoli gates as well.  Regardless of whether the target qubit is in the first or second encoding of the ququart, it is substantially faster to keep the controls encoded in the \textit{same} ququart with the target encoded in a separate ququart. 

\subsubsection{Target-Independent Gates}
Separating the controls and targets into different devices yields more efficient gate execution; however, compiling circuits to conform to this configuration is unnecessarily constraining. Instead, we consider a situation where all multi-qudit gates are \textit{target-independent} and only affect the global state when all three qubits are in $\ket{1}$. For example, the Toffoli gate, or CCX, is locally equivalent to CCZ which is target-independent, as seen in Figure \ref{fig:toffoli_decomp}c. 

When pulses are synthesized, CCZ is much more efficient as seen in Table \ref{tab:three-qubit-gate-times}, remarkably on par with the speed of the qubit only gates. In addition, we only need to define three configurations: $\text{CCZ}^{q,01}, \text{CCZ}^{0,01}, \text{CCZ}^{1,01}$, as opposed to the nine possible CCX configurations, reducing computational overhead. We postulate the short duration of these gates is because CCZ only changes the phase of the entire three-qubit state rather than the population. This makes the CCZ a valuable tool when compiling three-qubit gates.

\begin{figure}
    \centering
    \scalebox{0.85}{
    \includegraphics[width=\linewidth]{figures/three-qubit-gate-configs.png}
    }
    \caption{Examples of mixed-radix two-control and two-target gates. a) A configuration where both controls are encoded in the ququart and the target is mapped to a qubit. b) A configuration where the controls are split across the qubit and the ququart and the target is encoded in the ququart. c) A configuration where both targets are encoded in the ququart and the control is mapped to the qubit. d) A configuration where the targets are split across the qubit and the ququart and the control is encoded in the ququart.}
    \label{fig:three-qubit-configs}
    \vspace{-0.5em}
\end{figure}

\subsubsection{Multi-target Gates}
\label{sec:multitarget}
We also consider gates that use one control qubit to affect the state of some number of other qubits, for example the CSWAP. With our methods we synthesize gates to the same fidelity targets as before and show their times in Table \ref{tab:three-qubit-gate-times}. We find benefits when separating the control qubit from the target qubits as depicted in Figure \ref{fig:three-qubit-configs}c versus Figure \ref{fig:three-qubit-configs}d.  When both targets are encoded the same ququart, we limit the state changes to be between $\ket{1}$ and $\ket{2}$ in that ququart.