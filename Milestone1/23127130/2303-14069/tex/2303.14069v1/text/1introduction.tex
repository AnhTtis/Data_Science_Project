\begin{figure}[H]
    \centering
    \scalebox{0.98}{
    \includegraphics[width=\linewidth]{figures/hero-image.png}
    }
    \caption{A comparison of a Toffoli gate execution on a three-qubit-only system versus a Toffoli gate execution on a ququart and qubit in a mixed-radix system. In a qubit-only system, we must use a decomposition that uses eight two-qubit gates that can be reduced to one two-qudit gate that has a shorter duration.}
    \label{fig:qubit-vs-mixed-radix}
    \vspace{-0.5em}
\end{figure}

\section{Introduction}

Quantum systems are rapidly developing - stimulating the design and optimization of available hardware to maximize utilization of current resources for near-term quantum algorithms and the transition to quantum error correction \cite{gambetta_expanding_2022, chapman_scaling_2020}. Error-prone gates, sparse connectivity and low coherence times (the approximate computation time of any given device) are challenges currently facing quantum systems \cite{preskill_quantum_2018}. Even smaller quantum algorithms which fit on current hardware push devices to their limit. These limitations require improved optimization frameworks to make them useful in the near-term, prior to quantum error correction.

Success of quantum algorithms depends on how many error-prone gates are used and the total program duration. In most competitive quantum systems, e.g. superconducting systems, trapped ions, and neutral atoms, gates which act on many qubits simultaneously ($\ge 3$ operands) must be decomposed, increasing both gate counts and circuit depth. In this work, we focus primarily on superconducting systems, where limited connectivity between devices further exacerbates the decomposition problem.  Many circuits include gates, such as the Toffoli gate, to perform reversible arithmetic calculations; thus, three-qubit operations are common across implementations of quantum algorithms \cite{baker_efficient_2020, grover_fast_1996, cuccaro_new_2004, gokhale_quantum_2020}. Finding gate implementations without having to reduce them to more elementary gates saves valuable computational resources.

Currently, most quantum devices use qu\textit{bits}, which have two energy levels, used to represent the $\ket{0}$ and $\ket{1}$ state. Recently, there have been several explorations into using the higher energy levels such as $\ket{2}$ and $\ket{3}$ to reduce the number of gates required to perform computation. While there are many examples of exploiting this concept of qu\textit{dits}, such as using qutrits (3 logical states) to implement the multi-control Toffoli gate \cite{gokhale_asymptotic_2019, litteken_communication_2022}, implementing higher-radix adders \cite{taheri_monfared_quaternary_2019}, and other applications \cite{ivanov_time-efficient_2012}, these use cases are the result of hand optimization, making their general use limited.

Another proposed use of higher-radix states is to fully encode data from two qubits into one physical unit with four logical levels, called a ququart.  Previously, this strategy was avoided due to more error-prone operations \cite{chi_programmable_2022, wang_qudits_2020} and lower coherence times. 
Though coherence time is a limited resource, we can solve this problem by developing a set of operations which make better use of additional logical levels.

In this work, we observe that one ququart is equivalent to two qubits; thus the information of two qubit devices can instead occupy a single device which has access to four logical states \cite{baker_efficient_2020}. This has significant advantages on the relative connectivity of the qubit information: by performing this compression, we can access three qubits worth of information by interacting only two physical devices in a single operation rather than directly interacting three physical devices in a single operation. This type of \textit{mixed-radix} gate (four-level system interacting with an adjacent two-level system) is equivalent to performing a three-qubit gate. Similarly, we can consider two adjacent ququarts which allows us to perform interactions on up to four qubits worth of information by controlling only two physical devices; we call these \textit{full-ququart} gates. Our strategy could remove the need to perform expensive decompositions of three- or four-qubit gates, as visualized in Figure \ref{fig:qubit-vs-mixed-radix} potentially improving circuit fidelity of circuits containing multiqubit gates through the direct execution of three-qubit gates. We are primarily focused on common three-qubit interactions since they appear more commonly in real applications, unlike four-qubit gates.
%While native three-qubit gates have been demonstrated \cite{kim_high-fidelity_2022}, the calibration overhead and engineering requirement for general use makes them unfavorable.  Our abstraction would also circumvent these requirements, only requiring calibration between each pair of qudits rather than each set of three connected qubits.

\begin{figure}
    \centering
    \scalebox{0.85}{
    \includegraphics[width=\linewidth]{figures/HH_opt_IRB_edited.pdf}
    }
    \caption{Interleaved Randomized Benchmarking for an optimal control $H \otimes H$ pulse on a superconducting transmon ququart following our qubit encoding. We use two-qubit Clifford sequences of gate depth up to 100 and average each data point over 10 samples. Error bars show the standard deviation of the mean but they are smaller than the mean markers. Red: Standard two-qubit Randomized Benchmarking to estimate the average Clifford gate fidelity to be $F_\mathrm{RB} \approx 95.8\%$. Blue: Interleaving the $H \otimes H$ pulse between the RB Cliffords yields a combined per-operation fidelity of $F_\mathrm{IRB} \approx 92.1\%$, resulting in an $H \otimes H$ fidelity $F_{H \! H} \approx 96.0\%$.}
    \label{fig:HH_IRB}
    \vspace{-1.0em}
\end{figure}

We examine using ququarts to dynamically encode and decode gates to perform native three-qubit gates on ququarts on a simulated superconducting device in a compilation pipeline called the Quantum Waltz, a dance done in three-four time.  In particular, the major contributions are the following:
\begin{itemize}[leftmargin=*]
    \item A collection of mixed-radix and full-ququart gates that are logically equivalent to qubit-only gates, allowing for translation between qubit and mixed-radix operation.
    \item Demonstrating viability of ququart operations via pulses generated optimal control on hardware not previously designed for ququart pulses, Figure \ref{fig:HH_IRB}
    \item Identifying specific relationships between the controls and targets of three-qubit gates that allow for more efficient execution of mixed-radix and full-ququart gates with a compiler that choreographs three-qubit gates into particular configurations on ququarts for better performance and as a viable alternative to qubit-only strategies.
    \item Demonstrating, in simulation, how three-qubit gates on ququarts can achieve a 2x improvements in simulation in a mixed-radix environment and up to 3x fidelity improvements in a full ququart environment, as well as insights into the right situations to implement these gates.
\end{itemize}
