\section{Evaluation}

\subsection{Circuits}

We examine five three-qubit based circuits that can be parameterized by number of qubits with different constructions. The first is the Generalized Toffoli (CNU) circuit \cite{baker_decomposing_2019}, which flips the state of a target qubit if all the controls are one. This circuit uses exclusively Toffoli gate based decomposition and is highly parallel. The Cuccaro Adder \cite{cuccaro_new_2004} is nearly entirely serialized using $2n+2$ qubits with a mix of three-, two- and single-qubit gates to add two $n$-bit numbers. Third is a QRAM circuit which uses primarily CSWAP gates to retrieve data from or move data into a set of qubits \cite{gokhale_quantum_2020}.  The fourth is a Select circuit, which is a preparation mechanism used in Quantum Phase Estimation (QPE) \cite{low_hamiltonian_2019}.  It performs a particular Pauli operation on $n$ qubits for each potential $2^m$ states of $m$ index qubits \cite{babbush_encoding_2018}.  For our case, the choice of Pauli string does not affect compilation.  To keep the fidelity of circuit simulation within comparable bounds, we only select on two random values rather than all of the potential $2^m$ values the index qubits could be in.  The fifth is a purely synthetic circuit to study relative strength of our architecture on potential distributions of CX versus CCX gates.

\subsection{Baselines, Hardware Topology and Error}
We compare against two strategies.  The first is a compilation that routes the circuit with three-qubit gates, before decomposing them to one- and two-qubit gates only.  This is in line with current practices for most compilation pipelines.  The second baseline does not decompose to these smaller gates.  Instead, the $i$Toffoli-based decomposition directly on qubits similar to \cite{kim_high-fidelity_2022}.  This is a more challenging gate to synthesize as discussed previously. For simulation, this gate has a 99\% fidelity and with 912 ns duration determined via the same quantum optimal control strategies as the mixed-radix and full-ququart gates.  Additionally, we use the Hadamard-based retargeting technique to ensure that we are applying the Toffoli gate to the correct qubit without an extra SWAP.  This allows us to always use the demonstrated $i$Toffoli gate where the target qubit is the center of three connected qubits.

We consider the same underlying hardware topology for each comparison point - a 2D mesh. This type of grid architecture has relative density on the upper end of realized superconducting connectivity graphs, reflective of Google's Sycamore chip \cite{arute_quantum_2019} and more dense than IBM's heavy-hex \cite{gambetta_expanding_2022}. We consider a  grid design with dimensions $\lceil\sqrt{n} \rceil \times \frac{n}{\lceil\sqrt{n}\rceil}$ with nearest neighbor connectivity.

We use a realistic T1 time from an IBM device of $163.45 \mu s$ \cite{ibm_ibm_nodate}.  Higher energy levels decohere more quickly.  In theory, each state decays at a rate of $o(1/k)$ where $k$ is the energy level as discussed in \cite{younis_berkeley_2021}. We therefore use $81.73 \mu s$ and $54.15 \mu s$ as the T1 times for the $\ket{2}$ and $\ket{3}$ states.  As any transmon technically has access to these higher energy states, we do not expect that a device designed to access these higher-energy states will reduce the base T1 time. 

\subsection{Circuit Estimation}
We use two metrics to estimate the fidelity of a circuit without simulation to extrapolate how compiled circuits may perform by comparing simulation to estimation. The first is the product of all of the gate success rates in the circuit, called the gate expected probability of success (gate EPS).  Since there are multiple classes of multi-qubit gates, some of which have higher fidelity than others, we use the product of these success rates. 

Second, we model decoherence as an exponential decay where the probability of no decoherence is $\prod_{k=1}^3 \text{exp}(k*t_k/T_1)$ where $t_k$ is the time the qubit spends in state $k$.  When we construct the circuit we keep track of how long each qudit exists in the $\ket{1}$ or $\ket{3}$ state as the maximum state and calculate the probability of not decohering over the course of the execution for each qudit.  The product of the expected success of each qudit is the EPS due to coherence for the entire circuit.  When multiplied by the gate EPS, we have the EPS for the entire circuit.

\subsection{Circuit Simulation}
Since access to ququart devices at this scale are limited, we must use simulation to evaluate the performance of our approach. We use the trajectory method \cite{brun_simple_2002} for improved scalability compared to full density simulation.  This work simulates circuits of up to 24 qubits (or, equivalently, 12 ququarts). For this work, for each circuit, we generate at least 1000 random quantum states and for each we simulate once and compute the average fidelity over all random states. We emphasize the use of random \textit{quantum} states as classical inputs are not always affected by quantum errors. 

In the past, prior work on simulation of qudit systems neglects the realistic duration differences between gates which results in drastically different usage patterns and simply injecting errors on a moment-to-moment basis can skew results. For example, in this work our direct-to-pulse compilation of CCX and CCZ gates have significantly different execution times. We modify the trajectory method simulation slightly to account for this difference. Rather than inserting many idle gates during each time step, before each gate, we insert one idle gate using the exact time that qudit has been idle.  This is a more accurate representation of from which state these qudits could be decohering.

\begin{figure*}
    \centering
    \scalebox{0.9}{
    \includegraphics[width=\linewidth]{svg-inkscape/simulation-results.pdf}
    }
    \vspace{-0.5em}
    \caption{Simulated results for QRAM, Generalized Toffoli, Cuccaro Adder and Select Circuit from 5 to 21 qubits with different mixed-radix and full-ququart compilation strategies. The mixed-radix strategies do not have complete error bars due to the requirement to simulate a four-level system for every qubit which would require more than 86 GB of memory per circuit in our simulation framework. The final graph is the average fidelity improvement for each compilation method over the qubit-only compilation method as the size of the circuit increases.}
    \label{fig:main-simulation-results}
    \vspace{-0.5em}
\end{figure*}

\subsection{Noise Model for Qudit Systems}
For qubits we consider both symmetric depolarizing and amplitude damping errors. There are four possible single-bit %error 
channels: no error ($I$), bit flip errors ($X$), phase flip errors ($Z$) and bit and phase flip errors ($Y = ZX$). In simulation each error channel is drawn with probability $p/3$. Two-qubit errors are given as the product of single-qubit errors, e.g. $X \otimes X$ for a bit flip on both interacting qubits; there are 16 possible channels of this type so each error occurs with probability $p/15$ and no error ($I \otimes I$) occurs with probability $1 - 15p$. 

For a general qudit system, we consider a generalized form of these errors. The ``bit-flip" type gates become $X_{+1 \text{mod } d}$ and the ``phase-flip'' errors become $Z_d = \text{diag}(1, \exp{\omega}, \exp{\omega^2}, ..., \\ \exp{\omega^{d-1}})$ where $\omega^j$ is the $j-th$ root of unity. The product of $\{I, X_{+1 \text{mod } d}, ..., X_{+1 \text{mod } d}^{d-1}\}$ and $\{I, Z_d, Z_d^2, ..., Z_d^{d-1}\}$ is a basis for all $d\times d$ Pauli matrices which allows us to construct a general symmetric qudit depolarizing channel. This explains the expected increase in error for using qudit systems: For a two-qubit gate the chance of \textit{no} error is $1 - 15p$ while for a ququart this chance diminishes to $1 - 255p$ let alone possible differences in $p$ \cite{miller_propagation_2018}. 

Amplitude damping for qubits can be described as non-unitary transformations on the quantum state with operators \\ $K_0 = \text{diag}(1, \sqrt{1 - \lambda_1})$ and $K_1 = \sqrt{\lambda_1}e_{0, 1}$. Here $e_{i,j}$ refers to a matrix with all 0's except for a $1$ in the $i$-th row and $j$-th column and is of appropriate dimension. %(here $2 \times 2$). 
In the general qudit case we have \\ $K_0 = \text{diag}(1, \sqrt{1 - \lambda_1}, \sqrt{1 - \lambda_2}, ... \sqrt{1 - \lambda_{d}})$, $K_1 = \sqrt{\lambda_1}e_{0, 1}$, ... $K_d = \sqrt{\lambda_{d}}e_{0, d-1}$. Since we primarily focus on a superconducting system in this study we take $\lambda_m = 1 - \exp{-m\Delta t / T_1}$ where $\Delta t$ is the idling duration and $T_1$ is the coherence time of the qubit \cite{khammassi_qx_2017}. 

In this work we are also concerned with the manipulation of mixed-radix systems. When drawing an error for such a system, for example a qubit-ququart interaction, we consider only relevant errors for the respective participant. For instance, a two-qudit error is drawn from $P_2 \otimes P_4$ and not from $P_4 \otimes P_4$ (where $P_d$ is the set of $d$-dimensional Paulis, exactly the set of potential errors described above). Similarly, for two-qubit gates on encoded qubits, we consider only single \textit{ququart} errors since gates on encoded systems are equivalent to single-ququart gates.