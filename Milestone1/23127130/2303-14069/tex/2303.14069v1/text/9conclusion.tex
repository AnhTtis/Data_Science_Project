\section{Conclusion}
The architecturally imposed requirement to decompose more complex three-qubit gates into component one- and two-qubit gates is an extreme hurdle for realizing quantum computing.  Decomposing these gates increases both the number of error-prone gates that need to be executed, and the execution time of the circuit on devices with short coherence times.  However, many architectures have access to higher level states beyond the traditional two-level system.  While more prone to decoherence and error, this extra computational space can be used to compress quantum data, encoding two qubits into one physical device called a ququart.

This work takes advantage of increased connectivity and interaction potential when we have encoded qubits into a four-level system.  Encoding qubits in this way allows for the interaction of three to four qubits across a single physical connection, and we synthesize a library or efficient three-qubit gates via optimal control that take advantage of this virtual connectivity and are much faster and higher fidelity than performing the decomposition of a three-qubit gate.  We also demonstrate the viability of this encoding scheme and gate set via the execution of a $H \otimes H$ gate on real superconducting hardware. We then use these gates to develop compilation strategies, the quantum waltz, that use the most efficient configurations of three-qubit gates on mixed-radix and full-ququart systems to produce circuits that achieve 2x and 3x better simulated fidelities in mixed-radix and full-ququart environments, respectively compared to two-qubit based strategies.  We also demonstrate that ququart-based gates are a viable alternative to $i$Toffoli based three-qubit pulse strategies with potential practical upsides. Despite the difficulty of accessing and performing operations on higher level states, this efficient implementation of three-qubit gates provides worthwhile benefits for quantum computation.

Mixed-radix and full-ququart implementations of three-qubit gates makes ququart computation an invaluable piece of the quantum computing repertoire. It is more flexible than previous hand optimized circuits to improve circuit execution via higher radix devices, does not require the use of quaternary-based logic, and can be selectively applied to certain sections of larger circuits.  Realized implementations of these gates provide a massive opportunity to improve near-term execution of quantum circuits and expand the capabilities of quantum computers.

% Reiterate iToffoli in conclusion

% Acknowledge RCC