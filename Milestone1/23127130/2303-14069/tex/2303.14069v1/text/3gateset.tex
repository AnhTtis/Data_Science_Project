
\section{Compression and Gate Set}

\subsection{Information Compression}
Information \textit{compression} in the context of this work refers to the storage of many qubits worth of information which deviates slightly from the typical classical understanding. The goal of this compression is to reduce the total number of physical units required to realize a given quantum algorithm. 

Rather than designing algorithms that specifically use higher level states, we encode the data of two individual qubits into one four-level computational unit, called a ququart, given as $\ket{\psi}_4 = \alpha\ket{0} + \beta\ket{1} + \gamma\ket{2} + \delta\ket{3}$. This can be seen as equivalent to $\ket{\psi_1}_2 \otimes \ket{\psi_2}_2 = \alpha_1\alpha_2\ket{00} + \alpha_1\beta_2\ket{01} + \beta_1\alpha_2\ket{10} + \beta_1\beta_2\ket{11}$ by the following mapping:
\begin{align*}
       \ket{00} \rightarrow \ket{0} \ \  \ket{01} \rightarrow \ket{1} \ \ \ket{10} \rightarrow \ket{2} \ \ \ket{11} \rightarrow \ket{3}
\end{align*}
Therefore, $\alpha = \alpha_1\alpha_2$, $\beta = \alpha_1\beta_2$ etc.  This compression does not result in the loss of any information since the transformation from $\ket{\psi_1}_2 \otimes \ket{\psi_2}_2$ to $\ket{\psi}_4$ is unitary and therefore invertible.  This follows a modification of the scheme from \cite{baker_efficient_2020}.
 
This compression does not require a circuit to explicitly use the $\ket{2}$ and $\ket{3}$ states for the compiler to make use of ququarts like in \cite{gokhale_asymptotic_2019, ivanov_time-efficient_2012, wang_qudits_2020}.  We are able to adapt qubit-compiler pipelines to compile a circuit and encode qubits into a ququart and keep track of the original qubits without requiring changes in the original circuit.

\subsection{Qubit Gates on Ququarts}
\label{sec:qubit-gates-on-ququarts}
Past work has studied higher level systems by generalizing operations on a qubit circuit. For example, the X gate, is generalized to a $+1 \mod d$ instead, where $d$ is the dimension of the qudit. Multi-qubit gates generalize similarly; for example a CNOT can be viewed as a $\ket{1}$-controlled $+1 \mod 2$ gate and therefore in general we can consider $\ket{c}$-controlled $+m \mod d$ gates, $0 \leq c, m \leq d - 1$ \cite{luo_universal_2014}. 

While possible to use this generalized gate set to perform computation, it is not concise. For example, to perform a CNOT between the second encoded qubits encoded in different ququarts we would need to apply two $\ket{1}$-controlled $+1$ gates and two $\ket{3}$-controlled $+1$ gates. We could instead generate and calibrate a more expressive gate set that directly performs this operation.

We develop a gate set which performs qubit operations directly on ququarts.  For a single-qubit gate $U$ acting on two encoded qubits in the state $\ket{q_0q_1}$, we use the unitary $U^0 = U \otimes \mathbbm{1}$ to act on qubit $q_0$, $U^1 = \mathbbm{1} \otimes U$ to act on qubit $q_1$, and $U^{0,1} = U \otimes U$ to act on both qubits simultaneously.

For two-qubit gates, there are several important classes of operations. The first is the interaction between the two compressed qubits which we call an \textit{internal} operation. For example, a $\text{CX}^0$ is a CNOT controlled on the second qubit targeting the first; this is equivalent to the single ququart gate which swaps the states $\ket{1}$ and $\ket{3}$. $\text{CX}^1$ controls on the first and targets the second encoded qubit. A SWAP operation exchanges the order of the encoding, i.e. SWAP$\ket{q_1q_2} = \ket{q_2q_1}$. The second are gates which act on qudits in different, but adjacent, physical locations. These \textit{partial} gates interact a non-encoded qubit and a qubit in an encoded pair in adjacent locations; all gates of this type we call \textit{mixed-radix} gates. For these gates, order matters, i.e. the gate behaves differently depending on which qubit is the target.
  
The four CX gates are $\{\text{CX}^{q0}, \text{CX}^{q1}, \text{CX}^{0q}, \text{CX}^{1q}\}$ where the first index indicates the control and the second the target object, and $q$ is the qubit.  We also define two mixed-radix SWAPs $\{\text{SWAP}^{q0}, \\ \text{SWAP}^{q1}\}$ which are the same regardless of direction.  The \textit{ full-ququart} gates follow from the mixed-radix gates defining the four CX gates: $\{\text{CX}^{00}, \text{CX}^{01}, \text{CX}^{10}, \text{CX}^{11}\}$ and three SWAPs: $\{\text{SWAP}^{00}, \\ \text{SWAP}^{01}, \text{SWAP}^{11}\}$.

\subsection{Generating Pulses}\label{sec:pulse-generation}
Using quantum optimal control we directly synthesize each of the gates in our new mixed-radix and full-ququart gate set and baseline comparisons. We use a realistic superconducting device Hamiltonian inspired by IBM hardware \cite{sheldon_procedure_2016}.

We consider up to three weakly coupled, anharmonic transmons \cite{koch_charge-insensitive_2007}:
\begin{align}
    &H(t) = ~\sum_{k=1}^3 \qty[\omega_k a_k^\dagger a_k + \frac{\xi_k}{2} a_k^\dagger a_k^\dagger a_k a_k] \\
    &+ \sum_{k=1}^3\sum_{l>k}J_{kl} (a_1^\dagger a_2 + a_2^\dagger a_1) \notag 
    + \sum_{k=1}^3 f_k(t) (a_k + a_k^\dagger).
    \label{eq:ham_rot}
\end{align}

The static terms describe the individual qudits and their pairwise couplings, while the last term captures the effect of driving the system through external control fields $f_k(t)$. The transmons are designed with $\ket{0}$-$\ket{1}$ transition frequencies $\omega_1/2\pi = 4.914 \,\mathrm{GHz}$, $\omega_2/2\pi = 5.114 \,\mathrm{GHz}$, and $\omega_3/2\pi = 5.214 \,\mathrm{GHz}$, and with equal anharmonicities $\xi_k/2\pi = -330 \,\mathrm{MHz}$. We consider linear connectivity with static couplings given by $J_{12}/2\pi = J_{23}/2\pi = 3.8\,\mathrm{MHz}$. The drive power is limited to $f_\mathrm{max} = 45\,\mathrm{MHz}$ to avoid substantial leakage into higher energy states, and we restrict ourselves to the $k=1$ subspace when synthesizing single-qudit gates. 

A full list of the gates synthesized and the minimal found duration of these gates can be found in Table \ref{tab:two-qubit-gates}. We reiterate the importance of short gate times - quantum systems are subject to a variety of both coherent and incoherent errors. By minimizing the total execution time of any given gate we reduce the circuit duration, reducing the effects of incoherent noise.

The closed system considered does not account for the full dynamics of a real quantum device. We have not specifically optimized these pulses under a more detailed model due to the increased computational cost of these optimizations, especially for the large Hilbert spaces involved in two-qudit operations.

In Section \ref{sec:experiment}, we use similar optimal control techniques to implement a single-ququart operation on an experimental device, showing that our methods and assumptions are realistic given a well-characterized machine.

\subsection{Properties of Qubit Gates on Ququarts}
Our gate set and mixed-radix architecture provides real advantages over typical qubit-only versions. Within each ququart, we have a pair of encoded qubits between which gates are 5x faster and 10x higher fidelity than qubit-only schemes. By using a single computation device, the total amount of control hardware required is reduced (at most by half).  Additionally, we have much higher connectivity between qubits once they are encoded in ququarts. In a ququart-qubit pair, there are three computational qubits directly connected to one another. Between two ququarts, there are four fully connected computational qubits. This is higher relative connectivity compared to industry standards for superconducting: lines, grids, and heavy hex architectures. Improved connectivity reduces expensive qubit movement operations.  These increased connections are demonstrated in Figure \ref{fig:mixed-radix-and-full-encoded}.

Compression is not without its downsides. In Table \ref{tab:two-qubit-gates}, we see mixed-radix and ququart gates take much longer than qubit based gates. Pulses must be more carefully designed, and leakage between states is more prominent, resulting in the longer gates times. Each increasing energy level has a shorter coherence time scaling with $1/k$ where $k$ is the energy level. Shorter decoherence, combined with longer gate times, means using mixed-radix and ququart based gates is a delicate balancing act between increasing fidelity due to gate execution while not increasing error due to decoherence.


\subsection{Experimental Demonstration of Single-Ququart Control}\label{sec:experiment}

Driven by advantages found in theoretical studies \cite{pavlidis_arithmetic_2021}, experimental researchers have explored the implementation of these higher-dimensional systems, leading to realizations of qutrit devices which manipulate the third energy level \cite{galda_implementing_2021, hill_realization_2021, roy_realization_2022, goss_high-fidelity_2022, morvan_qutrit_2021, wu_high-fidelity_2020}. These works show that including higher levels is possible although challenging due to higher susceptibility to noise and lower coherence times.

Motivated by the findings for ququart-specific applications we have been studying control of four energy levels in experiment on a physical device. We extend the capabilities of one qudit of the superconducting transmon device presented in \cite{li_autonomous_2023, li_hardware_2023, roy_realization_2022} to include the fourth state. We implement two-qubit Randomized Benchmarking (RB) \cite{magesan_scalable_2011} on this single ququart following our encoding scheme. RB is a common method to characterize the average Clifford gate fidelity. This is achieved by executing Clifford circuits of varying depth, which perform the identity operation in the ideal case, and measuring the probabilities of the system returning to the ground state (survival probabilities). The fidelity can be extracted from exponential regression. The RB circuits are generated using Qiskit \cite{anis_qiskit_2021}.

We additionally implement Interleaved Randomized Benchmarking (IRB) \cite{magesan_efficient_2012} to specifically find the fidelity of the single-ququart gate $H \otimes H$, which performs a Hadamard gate on each encoded qubit in parallel, used by the compiler below. The gate control pulse is designed using similar optimal control methods as discussed in Section \ref{sec:pulse-generation} adapted to this experimental device.

Results from this work are shown in Fig. \ref{fig:HH_IRB}. We find an average Clifford gate fidelity of $F_\mathrm{RB} \approx 95.8\%$ from normal RB while interleaving with the $H \otimes H$ gate yields $F_\mathrm{IRB} \approx 92.1\%$ fidelity per operation. From that the specific gate fidelity, $F_{H \! H} \approx 96.0\%$, can be extracted. This first study shows that ququarts can be realized in experiment and optimal control yields high-quality pulses to manipulate their state. At the time of this writing we are not aware of any comparable demonstration. We are convinced that the fidelities can be improved with more carefully engineered ququart devices and more sophisticated pulse design methods.

% Note: should we say that manuscript will be out soon? No, then it makes it sound like we should have waited until that was out. Clever