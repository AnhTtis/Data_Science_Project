\documentclass{article}
\usepackage[utf8]{inputenc}

\title{ISCA23-Native Ququart Toffolis}
\author{Andrew Litteken}
\date{October 2022}

\begin{document}

\maketitle

\section{Introduction}

\section{Background}

\section{Mechanics of Three qubit operators with Ququart and mixed radix}
\subsection{Three qubit gates are not substantially harder than two qubit gates on ququarts}
\subsection{Mixed Radix Toffolis are not Surpassingly Expensive}
\subsection{What is the control and what is the target matters for ququarts}
\subsection{What if nothing is the target}
\subsection{What if you only have to change populations}

\section{Compilation}
We have all these different schemes for how the qubits can be laid out, all qubits or all ququarts
\subsection{Architectures}
\subsection{How do you design for three qubit gates?}
\subsubsection{Trios with Lookahead}
\subsection{How do we make sure we honor the controls the best we can?}
\subsubsection{Hadamards}
\subsubsection{Preemptive Swap Fixing}
\subsection{What if your circuit has no three qubit gates?}

\section{Evaluation}
\subsection{Circuits}
\subsection{Simulation}

\section{Results}
\subsection{Gate and Time Reductions}
\subsection{Simulation Results}

\section{Conclusion}


\end{document}
