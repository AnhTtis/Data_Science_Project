\section{Introduction}
\label{introduction}

AND is an important task in digital libraries that aims to properly link each publication to its respective co-authors so that author-level metrics can be accurately calculated and authors' publications can be easily found. However, this task is extremely challenging due to the high number of authors sharing the same names. In this paper, \emph{author name} denotes a sequence of characters referring to one or several authors~\footnote{It is estimated that about 114 million people share 300 common names.}, whereas \emph{author} refers to a unique person authoring at least one publication and cannot be identified only by his/her \emph{author name}~\footnote{In the DBLP database, there are 27 exact matches of ‘Chen Li’, 23 reverse matches and more than 1000 partial matches} but rather with the support of other identifiers such as ORCID, ResearchGate ID and Semantic Scholar author ID. 

Although relying on these identifiers almost eliminates any chance of mislinking a publication to its appropriate author, most bibliographic sources do not include such identifiers. This is because not all of the authors are keen to use these identifiers and if they are, there is no procedure or policy to include their identifiers when they are cited. Therefore, in bibliographic data (e.g. references), authors are commonly referred to by their names only. Considering the high number of authors sharing the same names (i.e. homonymy), it is difficult to link the names in bibliographic sources to their real-world authors especially when the source of the reference is not available or does not provide indicators of the author's identity. The problem is more critical when names are substituted by their initials to save space, and when they are erroneous due to wrong manual editing. Disciplines like social sciences and humanities suffer more from this problem as most of the publishers are small and mid-sized and cannot ensure the continuous integrity of the bibliographic data.  

Table~\ref{tab:illust} demonstrates real examples of reference strings covering the above-mentioned problems. The homonomy issue shows an example of two different papers citing the name \emph{J M Lee} which refers to two different authors. In this case, it is not possible to disambiguate the two authors without leveraging other features. The Synonymy issue shows an example of the same author \emph{Jang Myung Lee\orcidlink{0000-0003-4290-8087}} cited differently in two different papers as \emph{Jang Myung Lee} and \emph{J Lee}. Synonymy is a serious issue in author name disambiguation as it requires the awareness of all name variates of the given author. Moreover, some name variates might be shared by other authors, which increases homonymy. 

% \item \orcidlink{https://orcid.org} is an ORCID link.


\begin{table}[ht]
    \caption{Illustrative examples of author name ambiguity and incorrect author names}
    \label{tab:illust}
    \centering
    \begin{tabular}{|c|c|m{8cm}|}
    
    \hline
        \textbf{Issue Type} &
        \textbf{Source} &
        \multicolumn{1}{c|}{\textbf{Citations}} \\
    \hline
    
    \multirow{3.7}{*}{Synonyms} 
        & See~\footnotemark 
        & T. Jin, \href{https://dblp.org/pid/130/8653.html}{\textbf{J. Lee}}, and H. Hashimoto, “Internet-based obstacle
        avoidance of mobile robot using a force-reflection,” in Proceedings of the 2004 IEEE/RSJ International Conference on
        Intelligent Robots and Systems, (Sendai, Japan), pp. 3418–
        3423, October 2004.\\
        
        \cline{2-3}
        
        & See~\footnotemark 
        & TasSeok Jin, \href{https://dblp.org/pid/130/8653.html}{\textbf{JangMyung Lee}}, and Hideki Hashimoto, “Internet-based
        obstacle avoidance of mobile robot using a force-reflection,” IEEE/RSJ International Conference on Intelligent Robots and Systems, pp. 3418-3423. 2004.\\
    \hline
    
    \multirow{4.5}{*}{Homonyms} 
        & See~\footnotemark 
        & T.S. Jin, \href{https://dblp.org/pid/130/8653.html}{\textbf{J.M. Lee}}, and H. Hashimoto. Internet-based obstacle
        avoidance of mobile robot using a force-reflection. In Proceedings
        of the 2004 IEEE/RSJ International Conference on Intelligent Robots
        and Systems, pages 3418–3423, Sendai, Japan, October 2004.\\
        
        \cline{2-3}
        
        & See~\footnotemark  
        & H-J Kim, \href{https://dblp.org/pid/53/6517.html}{\textbf{J-M Lee}}, J-A Lee, S-G Oh, W-Y Kim, "Contrast Enhancement
        Using Adaptively Modified Histogram Equalization", Lecture Notes in
        Computer Science, Vol.4319, pp.1150 - 1158, Dec. 2006.\\ 
    \hline
    
    
    
    \end{tabular}
\end{table}

\addtocounter{footnote}{-3}
\footnotetext{Xu, Zhihao, et al. "Teleoperating a formation of car-like rovers under time delays." Proceedings of the 30th Chinese Control Conference. IEEE, 2011.}
\addtocounter{footnote}{1}
\footnotetext{Shi, Pu, Jianning Hua, and Yiwen Zhao. "Posture-based virtual force feedback control for teleoperated manipulator system." 2010 8th World Congress on Intelligent Control and Automation. IEEE, 2010.}
\addtocounter{footnote}{1}
\footnotetext{Xu, Zhihao, Lei Ma, and Klaus Schilling. "Passive bilateral teleoperation of a car-like mobile robot." 2009 17th Mediterranean Conference on Control and Automation. IEEE, 2009.}
\addtocounter{footnote}{1}
\footnotetext{
Lu, Ching-Hsi, Hong-Yang Hsu, and Lei Wang. "A new contrast enhancement technique by adaptively increasing the value of histogram." 2009 IEEE international workshop on imaging systems and techniques. IEEE, 2009.}




Since these problems are known for decades, several studies~\cite{muller2017semantic,kim2018web,foxcroft2019name2vec,hussain2017survey,ferreira2012brief,qian2015dynamic,zhang2016bayesian,khabsa2014large,khabsa2015online} have been conducted using different machine learning approaches. This problem is often tackled using supervised approaches such as Support Vector Machine (SVM)~\cite{han2004two}, Bayesian Classification~\cite{zhang2016bayesian} and Neural networks (NN)~\cite{tran2014author}. These approaches rely on the matching between publications and authors which are verified either manually or automatically. Unsupervised approaches~\cite{liu2014author,kim2020learning,fan2011graph} have also been used to assess the similarity between a pair of papers. Other unsupervised approaches are also used to estimate the number of co-authors sharing the same name~\cite{zhang2018name} and decide whether new records can be assigned to an existing author or a new one~\cite{qian2015dynamic}. Due to the continuous increase of publications, each of which cites tens of other publications and the difficulty to label this streaming data, semi-supervised approaches~\cite{louppe2016ethnicity,zhao2013semi} were also employed. Recent approaches~\cite{zhang2017name,xu2018network} leveraged the outstanding efficiency of deep learning on different domains to exploit the relationship among publications using network embedding. All these approaches use the available publication data about authors such as titles, venues, year of publication and affiliation. Some of these approaches are currently integrated into different bibliographic systems. However, all of them require an exhausting manual correction to reach an acceptable accuracy. In addition, most of these approaches rely on the metadata extracted from the papers which are supposed to be correct and complete. In real scenarios, the source of the paper is not always easy to find and only the reference is available. 



In this paper, which builds upon our earlier work~\cite{boukhers2022whois}, we aim to employ bibliographic data consisting of publication records to link each author's name in unseen records to their appropriate real-world authors (i.e. DBLP identifiers) by leveraging their co-authors and area of research embedded in the publication title and source. Note that the goal of this paper is to disambiguate author names in newly published papers that are not recorded in any bibliographic database. Therefore, all records that are considered unseen are discarded from the bibliographic data and used only for testing the approach. The assumption is that any author is most likely to publish articles in specific fields of research. Therefore, we employ articles' titles and sources (i.e. Journal, Booktitle, etc.) to bring authors close to their fields of research represented by the titles and sources of publications. We also assume that authors who already published together are more likely to continue collaborating and publishing other papers.

For the goal mentioned above, our proposed model \emph{WhoIs} is trained on a bibliographic collection obtained from DBLP, where a sample consists of a target author, pair of co-authors, title and source. For co-authors, the input is a vector representation obtained by applying Char2Vec which returns character-level embedding of words. For title and source, the BERT model is used to capture the semantic representations of the sequence of words. Our model is trained and tested on a challenging dataset, where thousands of authors share the same atomic name variate. The main contributions of this paper are: 
\begin{itemize}[leftmargin=*]

\item We proposed a novel approach for author name disambiguation using semantic and symbolic representations of titles, sources, and co-authors.
\item We provided a statistical overview of the problem of author name ambiguity. 
\item We conducted experiments on challenging datasets simulating a critical scenario. 
\item The obtained results and the comparison against baseline approaches demonstrate the effectiveness of our model in disambiguating author names.
\end{itemize}




The rest of the paper is organized as follows. Section~\ref{related_work} briefly presents related work. Section~\ref{method} describes the proposed framework. Section~\ref{experiments} presents the dataset, implementation details and the obtained results of the proposed model. Finally, Section~\ref{conclusion} concludes the paper and gives insights into future work.







