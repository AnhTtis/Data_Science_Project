\section{Related Work}
\label{related_work}



In this section, we discuss recent approaches softly categorized into three categories, namely unsupervised-, supervised- and graph-based; 


\subsection{Unsupervised-based:} Most of the studies treat the problem of author name ambiguity as an unsupervised task~\cite{kim2020learning,zhang2018name,khabsa2015online,khabsa2015online,qian2015dynamic} using algorithms like DBSCAN~\cite{khabsa2015online} and agglomerative
clustering~\cite{wu2014unsupervised}. Liu et al.~\cite{liu2014author} and Kim et al.~\cite{kim2020learning} rely on the similarity between a pair of records with the same name to disambiguate author names on the PubMed dataset. Zhang et al.~\cite{zhang2018name} used Recurrent Neural Network (RNN) to estimate the number of unique authors in the Aminer dataset. This process is followed by manual annotation. In this direction, Ferreira et al.~\cite{ferreira2010effective} have proposed a two-phase approach applied to the DBLP dataset, where the first one is obtaining clusters of authorship records and then disambiguation is applied to each cluster. Wu et al.~\cite{wu2014unsupervised} fused features such as affiliation and content of papers using Shannon’s entropy to obtain a matrix representing pairwise correlations of papers which is in return used by Hierarchical Agglomerative Clustering (HAC) to disambiguate author names on Arnetminer dataset. Similar features have been employed by other approaches~\cite {yang2011author,arif2014author}.

\subsection{Supervised-based:} 
Supervised approaches~\cite{han2004two,qian2011combining,sun2011detecting,tran2014author,zhang2016bayesian} are also widely used but mainly only after applying to block that gathers authors sharing the same names together. Han et al.~\cite{han2004two} present two supervised learning approaches to disambiguate authors in cited references. Given a reference, the first approach uses the Naive Bayes model to find the author class with the maximal posterior probability of being the author of the cited reference. The second approach uses SVM to classify references from DBLP to their appropriate authors. Sun et al.~\cite{sun2011detecting} employ heuristic features like the percentage of citations gathered by the top name variations for an author to disambiguate common author names. Neural networks are also used~\cite{tran2014author} to verify if two references are close enough to be authored by the same target author or not. Hourrane et al.~\cite{hourrane2018using} propose a corpus-based approach that uses word embeddings to compute the similarity between cited references. In~\cite{ebraheem2018distributed}, an Entity Resolution system called the DEEPER is proposed. It uses a combination of bi-directional recurrent neural networks (BRNN) along with Long Short Term Memory (LSTM) as the hidden units to generate a distributed representation for each tuple to capture the similarities between them. Zhang et al.~\cite{zhang2016bayesian} proposed an online Bayesian approach to identify authors with ambiguous names and as a case study, bibliographic data in a temporal stream format is used and the disambiguation is resolved by partitioning the papers into homogeneous groups.



\subsection{Graph-based:}
As bibliographic data can be viewed as a graph of citations, several approaches have leveraged this property to overcome the problem of author name ambiguation~\cite{hoffart2011robust,han2011collective,zhang2017name,xu2018network}. Hoffart et al.~\cite{hoffart2011robust} present a method for collective disambiguation of author names, which harnesses the context from a knowledge base and uses a new form of coherence graph. Their method generates a weighted graph of the candidate entities and mentions to compute a dense sub-graph that approximates the best entity-mention mapping.  Xianpei et al.~\cite{han2011collective} aim to improve the traditional entity linking method by proposing a graph-based collective entity linking approach that can model and exploit the global interdependence, i.e., the mutual dependence between the entities. In~\cite{zhang2017name}, the problem of author name ambiguity is overcome using relational information considering three graphs: person-person, person-document and document-document. The task becomes then a graph clustering task with the goal that each cluster contains documents authored by a unique real-world author. For each ambiguous name, Xu et al.~\cite{xu2018network} build a network of papers with multiple relationships. A network-embedding method is proposed to learn paper representations, where the gap between positive and negative edges is optimized. Further, HDBSCAN is used to cluster paper representations into disjoint sets such that each set contains all papers of a unique real-world author. 

