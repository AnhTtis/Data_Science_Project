
\documentclass[aps,prb,notitlepage,twocolumn,noshowpacs,10pt,longbibliography]{revtex4-1}

\usepackage{graphicx}
\usepackage{inputenc}
\usepackage{epstopdf}
\usepackage{siunitx}
\usepackage{amsmath,amsfonts}

%\usepackage{mathabx}
%\usepackage{lineno}

%\usepackage[columnwise]{lineno}
%\usepackage[switch]{lineno} 
%\usepackage[switch,columnwise]{lineno}

%\usepackage[colorlinks=true,allcolors=blue,breaklinks=true]{hyperref}



\begin{document}

\title{Quasiparticle transport in a two-dimensional boundary superfluid}
%\title{Probing excitation transport in the quantum well surrounding superfluid $^3$He}
\author{S.~Autti$^1$}
\email{s.autti@lancaster.ac.uk}
\author{R. P. Haley$^1$}
\author{A. Jennings$^1$}
\email{Current address: RIKEN Center for Quantum Computing, RIKEN, Wako, 351-0198, Japan}
\author{G. R. Pickett$^1$}
\author{M. Poole$^1$}
\author{R. Schanen$^1$}
\author{A. A. Soldatov$^2$}
\author{V. Tsepelin$^1$}
\author{J. Vonka$^1$}
\email{Current address: Paul Scherrer Institute, Forschungsstrasse 111, 5232 Villigen PSI, Switzerland}
\author{V. V. Zavjalov$^1$}
\author{D. E. Zmeev$^1$}
\affiliation{$^1$Department of Physics, Lancaster University, Lancaster LA1 4YB, UK. }
\affiliation{$^2$P.L. Kapitza Institute for Physical Problems of RAS, 119334 Moscow, Russia}

\maketitle


 
{\bf
The B phase of superfluid $^3$He can be cooled into the ``pure’’ superfluid regime, characterised by negligible thermal quasiparticle density. Here, the bulk superfluid is bounded by a two-dimensional quantum well at the boundaries of the container, where creating quasiparticles requires much less energy.\cite{autti2020fundamental,Nagato19981135,Nagai2008,Murakawa2011,Okuda2012,Murakawa2011,Aoki2005,Zheng2016} In this Article, we carry out experiments where we create a non-equilibrium state within the quantum well and show that the induced quasiparticle currents flow diffusively in the two-dimensional system. We conclude that the bulk of superfluid $^3$He is wrapped by an independent two-dimensional superfluid that interacts with mechanical probes instead of the bulk superfluid, only providing access to the bulk superfluid if given a sudden burst of energy. That is, superfluid $^3$He at the lowest temperatures and applied energies is thermo-mechanically two dimensional. Our work opens this two-dimensional quantum condensate and the interface it forms between the observer and the bulk superfluid for exploration, and provides the possibility of creating two-dimensional condensates of arbitrary topology.
}

\begin{figure}[htb!]
	\includegraphics[width=1\linewidth]{fig1_pot_well_v2.pdf}
	\caption{{\bf The two-dimensional quasiparticle quantum well:}  ({\bf a}) The superfluid gap is suppressed as the container boundary is approached. This yields a potential well in which bound quasiparticles (red halo) can exist to arbitrary low energies. At low temperatures the density of quasiparticles in the bulk superfluid is vanishingly low. ({\bf b}) The probe wire (cross-section shown by the grey disk) is surrounded by the bound quasiparticle potential well (red halo). The thickness of the potential well $a \xi$ ($a\sim 1$, see Methods) determines the bound quasiparticle mean free path as indicated by the double arrow. Here $R$ is the radius of the wire cross bar. When the cross bar is moving, the local superflow velocity around it depends on polar angle $\theta$.  Green notches indicate the span of  $\theta=\pm 80^\circ$ where the quasiparticle escape condition $v_{\rm fl}>v_{\rm c}$ is satisfied locally for the wire velocity of $v$=\SI{45}{\milli\meter\second}$^{-1}$.}
	\label{quantum_well}
\end{figure}

At the lowest temperatures (and here in zero field and pressure) bulk superfluid $^3$He exists in the B phase. Here, the minimum energy required to create a quasiparticle, or one half of a broken Cooper pair, is the energy gap $\varDelta\approx 1.6\,$mK. These quasiparticles are responsible for the macroscopic transfer of momentum and energy in the superfluid, and the quasiparticle density decreases exponentially with decreasing temperature. Therefore, at temperatures below a quarter of the superfluid transition temperature $\approx1$\,mK, the bulk superfluid only conducts heat efficiently from sources hot enough to create new quasiparticles. However, within roughly a coherence length $\xi\approx80$\,nm of the sample container walls, the energy gap is suppressed, as shown in Fig.~\Ref{quantum_well}a. The coherence length is the smallest length scale across which the superfluid wave function can change, and therefore the gap suppression region is effectively two-dimensional. This suppression gives rise to a quantum well allowing quasiparticles to exist at arbitrarily small energies. We know that these bound quasiparticles can be pushed to the bulk superfluid where they become regular thermal quasiparticles\cite{autti2020fundamental}, but basic physics in the absence of bulk emission such as transport within the quantum well remain unexplored. Thus, it has not been clear whether the two-dimensional system is an independent superfluid condensate. Here we argue that the bound quasiparticles are decoupled both from phonons in the container wall and from thermal quasiparticles in the bulk superfluid. They thus have their own mode of transport by diffusion within the two-dimensional superfluid, forming the primary system at low energies that a probe rod inserted in the superlfuid interacts with. 


In the simplest description, the bound quasiparticles have a linear dispersion as a function of their in-plane momentum $p_{||}$, $E=v_\mathrm{L} p_{||}$.\cite{silaev2014andreev,Nagato19981135,Nagai2008,Murakawa2011,Okuda2012,Aoki2005,Zheng2016}  Here $v_\mathrm{L}=27$~mm/s is the Landau critical velocity, which means that such bound quasiparticles move at a uniform group velocity $v_\mathrm{qp}=v_\mathrm{L}$ in the plane of the surface\cite{volovik2009fermion,Murakawa2011}. Since this dispersion is best approximated when the scattering from the containing wall is (partially) specular \cite{Murakawa2011}, the primary experiments in this article were carried out with approximately two monolayers of solid $^4$He coating all surfaces\cite{autti2020fundamental}, yielding a specularity in the range between 0.2 and 0.8, where one is full specularity and zero means diffuse surface scattering. We can also measure the effect of removing the $^4$He coating\cite{dmitriev2005separation} so that specularity is approximately zero. In this case the bound quasiparticles have a more complicated dispersion, but the order of magnitude of $v_\mathrm{qp}$ remains the same \cite{Murakawa2011,volovik2009fermion}.


We are able to induce interaction between the two- and three-dimensional superfluids when a probe inserted in the bulk superfluid is accelerated to a velocity exceeding the critical velocity $v_\mathrm{c}=v_\mathrm{L}/3$ (in this case, for a cylindrical rod), promoting the most energetic quasiparticles bound on the probe surface to escape into the bulk. The probe we use is the crossbar of a cylindrical goalpost-shaped wire\cite{Bradley2016} (radius $R=\SI{63.5}{\micro\meter}$, crossbar 8\,mm, legs 25\,mm), as shown in Fig.~\ref{goalpost} in Methods. When the wire is moved, driven by the Laplace force in a magnetic field, the superfluid flow velocity with respect to the crossbar surface follows $v_{\rm fl}=2v \cos(\theta)$, where $v$ is the velocity of the crossbar in the laboratory frame, and the polar angle $\theta$ is defined in Fig.~\ref{quantum_well}b. The bound-state dispersion is Doppler shifted in energy by $\pm v_{\rm fl} p_\mathrm{F}$ ($p_\mathrm{F}$ being the Fermi momentum). By changing the direction and amplitude of the superflow along the surface we can manipulate the bound quasiparticle dispersion curves, allowing us to create a non-equilibrium state within the two-dimensional superfluid. 



\begin{figure}[htb!]
	\includegraphics[width=1\linewidth]{fig3_ramps_v2.pdf}
	\caption{{\bf Schematic illustration of the cross bar motion:} The wire can be moved across 0.5\,mm with a steady velocity, here \SI{45}{\milli\metre \,\second^{-1}}, and then paused (green line). After a wait of $\Delta t$,  the process can be repeated with a further up movement (``up cycle" blue line), or reversed back to the starting point (``down cycle", red line). The combination of the green and blue lines is denoted ``up-up cycle" and that of the green and red lines ``up-down cycle". The vertical bands of colour indicate where surface quasiparticles are emitted from the wire during acceleration and deceleration.  The emitted quasiparticles increase the temperature of the bulk superfluid, which is detected using a separate thermometer.}
	\label{flow_schematic}

\end{figure}


As the wire velocity $v$ is increased from zero beyond $v_\mathrm{c}$, the most energetic bound quasiparticles escape to the bulk superfluid as schematically shown in Fig.~\ref{flow_schematic} and detailed in Fig.~\ref{bands_fig} in Methods. While the wire is moving at a uniform velocity\cite{Zmeev2014}, a new equilibrium of bound quasiparticle distribution is established and no further quasiparticles are released into the bulk. A similar process takes place during the deceleration phase at the end of the motion, where a further pulse of bound quasiparticles can escape.


\begin{figure}[htb!]
	\includegraphics[width=1\linewidth]{fig5_cartoon_bands.pdf}
\caption{{\bf A cartoon of the quasiparticle emission in an up-up cycle:} ({\bf  a}) As the superflow accelerates, the bound quasiparticle bands are Doppler-shifted up for quasiparticles with momenta against the superfluid and down for momenta along the flow direction (red pillars). This raises the most energetic bound quasiparticles above the minimum energy in the bulk superfluid, allowing the quasiparticles to escape into the bulk as a sudden burst (yellow star and arrow). During the motion the quasiparticle deficit created on the right-hand side band can only be filled slowly from transport of quasiparticles along the potential well. When the motion ceases the bands return to their original positions allowing another sudden burst of quasiparticles to escape. At the end of the whole velocity cycle the left-hand side band is left with a deficit. ({\bf b}) If the initial up cycle is followed with no delay by another up cycle, the second cycle will start with a quasiparticle deficit allowing no quasiparticle emission during the acceleration. Therefore, only one pulse of quasiparticles is emitted, that from the deceleration. Allowing for (partial) recovery and measuring the amount of emitted quasiparticles thus allows us to take snapshots of the bound quasiparticle diffusion. }
    \label{cartoon}
 \end{figure}

The pulses of quasiparticles released into the bulk increase the temperature of the bulk liquid.  We can infer the dependence between the temperature increase and the heat released by the motion of the wire. Moreover, because each quasiparticle in the bulk liquid carries an energy very close to  $\varDelta$, this measured temperature rise yields the number of quasiparticles that have escaped during the motion of the wire. We label the direction of the initial acceleration-deceleration cycle ``up". The quasiparticle release process in an up cycle is schematically illustrated in Fig.~\ref{cartoon}a. 

We can measure the heat released by such cycle of the cross bar by varying the bulk quasiparticle density. That is, the crossbar also scatters bulk quasiparticles while it moves, with the resulting drag force $F$ yielding the heating $Q= F d$ ($d$ is the distance travelled). This drag force can be varied by changing the temperature of the bulk superfluid and measured independently using a separate thermometer wire whose resonance width $\Delta f \propto F$. The proportionality constant is calculated in Methods, yielding the black line in Fig.~\ref{exp_relax}a, in good agreement with the measured $Q$. The effect of $Q$ can thus be removed by extrapolating the measured heat release linearly to $\Delta f =0$. The heat released from the bound quasiparticles extracted this way from a single acceleration is $q=12\pm 3~$pJ, which is in good agreement with the theoretical estimate $\sim10\,$pJ as detailed in Methods.



\begin{figure}[tb!]
	\includegraphics[width=1\linewidth]{temp_dep_v3.pdf}
	\caption{{\bf Direct measurement of bound quasiparticle heat release:} The heat $Q_\mathrm{tot}$ released in an up cycle (green points) depends linearly (green dashed line%, slope $0.74\pm0.03$~pJ/Hz
) on the thermal bulk drag force, proportional to the resonance width $\Delta f$ of the vibrating wire thermometer in the same superfluid volume. The black line shows an estimate of the heating from the drag force due to thermal bulk quasiparticles (Methods) in good agreement with the slope of the green data. Extrapolation to zero $\Delta f=0$ yields the bound state contribution $q=14\pm 3~$pJ (intersection with the y-axis is at $2q$). For an up-up cycle with $\Delta t >25~$ms (blue points), the slope is doubled because the distance travelled is twice longer (blue dashed line% , slope $1.4\pm0.1$~pJ/Hz
), and the $\Delta f=0$ intersection yields $4 q=60\pm 4~$pJ. These experiments were carried out with $^4$He preplating. Up-up cycles measured without $^4$He preplating (magenta points) show a 10\% reduction in the slope %($1.2\pm0.2$~pJ/Hz)
and the linear fit (dashed magenta line) extrapolates to $4 q\approx 28~\mathrm{pJ}$. }
	\label{exp_relax}
\end{figure}

Importantly for the current experiment, at the end of the up cycle, the dispersion curves of the quasiparticles have returned to their zero-velocity profile, but leaving on one branch a deficit where the highest energy quasiparticles have energies well below zero (see Fig.~\ref{cartoon}a). Since at low temperatures the equilibrium density of quasiparticles in the bulk liquid is very low, the mechanism for the deficit of quasiparticles to be replaced by quasiparticles coming from the bulk is too slow to be directly observed\cite{autti2020fundamental}. Therefore, equilibrium for these particles can be reestablished only by the flow of bound quasiparticles along the potential well around the cross bar.



Now we can progress to measuring bound quasiparticle transport in the surface system. Referring back to the cycle of cross bar motion shown in Figure~\ref{cartoon}a, the end state is not the same as the initial state as we are left with a quasiparticle deficit in the left-hand branch. The key concept of the experiment is to repeat the acceleration-deceleration cycle once more after a wait of $\Delta t$ (up-up cycle).  This can be done for two extrema. If we repeat the cycle immediately after the first has ended, that is, setting $\Delta t$ to zero then the second cycle follows that shown in Fig.~\ref{cartoon}b where we start already with the full deficit in the left-hand branch. However, this second process only yields one burst of quasiparticles, that on the deceleration. Thus, the combined series of two cycles yields three bursts of quasiparticles into the bulk. Alternatively, after the first cycle we can wait for an ``infinite’’ period, i.e. longer than the time taken for the deficit to fill by flow of localised quasiparticles round the wire periphery, in which case we have the same starting conditions as shown in Fig.~\ref{cartoon}a. Thus, a series of two cycles with a long wait between will emit four equal bursts of bound quasiparticles.  Between these two extremes we can explore the situation with intermediate values of the delay $\Delta t$ and can map out how rapidly the deficit fills.  That is the basis of the measurement which provides directly a measure of the diffusion rate of the localised quasiparticles through the potential well around the wire.


To interpret the data we now need to devise a model for diffusive quasiparticle transport in the two-dimensional surface quantum well, which we do following the theoretical lead of Refs.~\citenum{PhysRevB.75.024514,silaev2013dissipative,wu2013majorana}. This model is based on three assumptions (all well justified). First, there is no interaction between surface-bound and bulk quasiparticles, as has been observed across two orders of magnitude in bulk quasiparticle density\cite{autti2020fundamental}.  Secondly, the thermal Kapitza resistance between phonons in the solid boundary material and the superfluid quasiparticles is very large at the lowest temperatures. A simple estimate for the exponential decay of the energy of the bound quasiparticles into the material of the cross bar yields a time constant $\tau_\mathrm{R_\mathrm{K}} > 10$~s (Methods). These two assumptions imply that energy in the bound quasiparticle system is conserved. Third, the distribution of the bound quasiparticle system reflects the  history of cross bar motion as detailed below and in Ref.~\citenum{autti2020fundamental}, meaning that also the bound quasiparticle momentum distribution equilibrates with the time constant $\tau$. Thus, quasiparticle transport acts as the equilibration mechanism.


The diffusion coefficient for quasiparticle transport can be estimated as $D\sim v_\mathrm{qp} l_{||}$. Here $l_{||}$ is the mean free path which, in the absence of quasiparticle-quasiparticle collisions (this assumption is confirmed below), is determined by the thickness of the quantum well, that is, the distance between the wire surface and the edge of the surface layer, $\l_{||}\sim$\SI{10}{\micro\meter} (Fig.~\ref{quantum_well}b and Methods). When the cross bar is stopped, the density of bound quasiparticles carrying the momentum imbalance, $n$, reflects the flow velocity along the cross bar surface immediately before the wire was stopped. We estimate this distribution as $n \propto v_\mathrm{fl}\propto \cos \theta$. Solving the diffusion equation for this initial state yields exponential recovery of equilibrium $n$ with the time constant $\tau=R^2/D \sim 10~$ms  (see Methods). We emphasise that the time constant is tied to the population gradient arising from the flow velocity profile and, hence, to non-local diffusion.

We can probe the diffusive bound-state redistribution by varying the delay $\Delta t$ between the two cycles of motion. Fewer quasiparticles are released in the second cycle if the susceptible population has been depleted during the first cycle. The replenishment of the deficit begins already during the deceleration and continues until the velocity reaches zero velocity again. According to the  diffusion picture, the energy released in the second acceleration is $ q \left[1 - \exp(-(\Delta t + t_0) /\tau)\right]$, where $t_0=6$\,ms is the combined acceleration and deceleration time. The total energy release from the up-up cycle is detailed in Methods. The measured $\Delta t$ dependence, shown in Fig.~\ref{exp_decay1}, confirms the exponential equilibration with the fitted value $\tau=6\pm 3$~ms\cite{autti2020fundamental} in good agreement with the theoretically estimated time constant.



\begin{figure*}[htb!]
	\includegraphics[width=1\linewidth]{exp_decay_fig5.pdf}
	\caption{{\bf Diffusion in the two-dimensional superfluid:} ({\bf a}) We record the total energy released to the bulk superfluid as a function of the recovery time $\Delta t$. Blue circles show the result for an up-up cycle at two different bulk temperatures and red circles for corresponding up-down cycle. The fitted exponential time dependencies describe the diffusion process that redistributes quasiparticles until equilibrium is recovered. This allows inferring the magnitude of the bound quasiparticle depletion (or excess for the up-down cycles) that results from the first up cycle. Fitted parameter values $q$ and $q_\mathrm{down}$ are indicated in the figure with errors corresponding to 68\% confidence intervals. Data in this panel was measured with applied $^4$He coating on the cross bar surface. ({\bf b}) The number of bound quasiparticles released is approximately halved when the cross bar surface is not coated with solid $^4$He. We vary the bulk temperature to show that $-q$ (blue points) and $q_\mathrm{down}$ (red points) are independent of the quasiparticle density in the bulk. The top x-axis shows the thermometer wire resonance width which is proportional to density of bulk quasiparticles. Dashed horisontal lines are a guide to the eye corresponding to $-q=-6$\,pJ (blue line) and $q_\mathrm{down}=9$\,pJ (red line). At temperatures higher than 0.19\,mK, $Q$ becomes larger than $q$ or $q_\mathrm{down}$, thus causing difficulties in extracting $q$ and $q_\mathrm{down}$ reliably. Error bars show the 68\% confidence interval of the exponential fits. Additionally, the crossing of the exponential tails seen for the 0.22\,mK data in panel a is due to a hysteresis effect\cite{autti2020fundamental}, proportional to $\Delta f$. This acts to increase the apparent $q_\mathrm{down}$ . The magnetic field was 136~mT and $v=45$~mm/s in both panels.}
	\label{exp_decay1}
\end{figure*}


When the first motion cycle depletes the bound quasiparticles available for bulk escape, none will be released if the cycle follows with no intermediate recovery time. That is, the fitted $q$ should be consistent with that extracted by temperature extrapolation above. The blue lines in Fig.~\ref{exp_decay1} are fits with $\tau=6~$ms at two different temperatures, with the average value ${q} = 11\pm 5\,$pJ. This is in good agreement with $q=12\pm 3\,$pJ obtained above with the temperature extrapolation. Combining this with the observation detailed below that $q$ does not depend on bulk temperature yields two immediate conclusions. First, starting the second cycle before the bound state recovery is complete provides a quantitative snapshot of bound quasiparticles' dynamics. Second, the bound state population available for bulk escape at this velocity is fully depleted by the first cycle of motion. 

We also note that none of the measured dissipation originates from direct creation of bulk quasiparticles. That is, the temperature extrapolation tells us the total magnitude of dissipation at zero bulk temperature, which matches with the magnitude of the diffusion process. This is despite the fact that also a large bulk superfluid volume flows at speeds exceeding the Landau critical velocity when the wire is moved, and naively we might expect a direct pair breaking mechanism to arise. This observation provides a new perspective to experiments carried out in the polar phase of superfluid $^3$He, \cite{Mineev2016,PolarDmitriev,PhysRevLett.121.025303} where exceeding the bulk Landau velocity in a large volume potentially causes no observable dissipation either, contrary to theoretical expectation\cite{autti2020exceeding,HQVs_prl}. 

Reversing the direction of motion for the second cycle (up-down cycle) results in a temporary excess of bound quasiparticles available for escape during the acceleration of the down cycle. This scenario is illustrated schematically in Figure~\ref{ramps_details}. The measured data shown in  Figure~\ref{exp_decay1} confirms this excess. Applying the diffusion picture, we expect the excess bound quasiparticle emission to be removed by diffusion as ${q}_{\mathrm{down}}\exp(-(\Delta t + t_0) /\tau)$, in good agreement with the data shown in Figure~\ref{exp_decay1}. The low-temperature fitted $q_\mathrm{down}\approx30$\,pJ is consistent with a temporary population excess that is removed by diffusion, as detailed in Methods. At higher temperatures the hysteretic effect explained in Ref.~\citenum{autti2020fundamental} distorts the measurement. 



We can decrease the number of bound quasiparticles susceptible to the bulk flow by removing the $^4$He coating of the surface and thus taking the specularity close to zero \cite{Murakawa2011}. Reducing the surface specularity moves the density of states towards states with lower momenta in the plane of the wire surface. States with zero in-plane momentum gain no energy from $v_\mathrm{fl}$ and therefore cannot escape to the bulk. Repeating the extrapolation to $\Delta f =0$ as before yields $q\approx7~$pJ (Fig.~\ref{exp_relax}). Assuming the escape process is equally effective with and without $^4$He, this implies that the susceptible bound state density is roughly twice higher with $^4$He coating than without it. This is qualitatively in line with the theoretical prediction in Ref.~\citenum{Murakawa2011}. The slope of $Q(T)$ is 10\% larger with $^4$He coating than without it. This is because surface specularity does not change the bulk quasiparticle density at a given temperature, and the drag force $F$ at large velocities $v$ increases slowly with growing specularity \cite{fisher1991microscopic,enrico1996specular,enrico_thesis}. Change in the surface specularity has no effect on the diffusion time constant process within experimental uncertainties\cite{autti2020fundamental}. That is, quasiparticle-quasiparticle collisions remain negligible regardless of changes in the density of states, substantiating the assumption that the mean free path is determined geometrically by the thickness of the surface layer.

To add further evidence for the independence of the surface dynamics, we can extract $q$ and $q_\mathrm{down}$ by varying $\Delta t$ at different bulk temperatures (no $^4$He, Fig.~\ref{exp_decay1}b). The bulk quasiparticle density changes by two orders of magnitude over the temperature range studied but, remarkably, the bound state process remains undisturbed, indicated by the constant energy release. This shows that the snapshot technique reliably probes the bound quasiparticles' dynamics, and that there is no coupling between the bound quasiparticles and thermal bulk quasiparticles. 

We conclude that the two-dimensional superfluid manifests independent quasiparticle transport dynamics, carrying energy and momentum. At energies below the superfluid gap, the surface provides a preferential path for heat flow between hot and cold objects immersed in the superfluid and a direct interaction channel between immersed mechanical probes. This means that zero-temperature superfluid $^3$He is thermo-mechanically two dimensional.

Confining a fermion gas at a low temperature to a high-purity two-dimensional solid-state system has led to the discovery of a variety of quantum Hall phases and topological quantum states. Similarly, spontaneous formation of ultra-cold superfluid $^3$He, an extremely pure fermionic system, into a two-dimensional surface system is likely to yield a diversity of new physics. For example, the bound qausiparticles' possible interactions with the sub-gap bosonic excitations or bulk topological defects span at least 7 orders of magnitude in energy below the superfluid gap and some 18 degrees of freedom\cite{VolovikBook,vollhardt2013superfluid}.  Our results imply that devising suitable nanoprobes that fit within the two-dimensional superfluid should reveal long-range quasiparticle transport that can be studied in varying topological configurations, such as across different bulk superfluid phases and interfaces, via controlled confinement provided by engineered nanostructures \cite{lotnyk2020thermal,heikkinen2021fragility,todoshchenko2020mobile,Levitin841,shook2020stabilized,PhysRevLett.122.085301,levitin2014study,zhelev2017ab}, or across the free surface\cite{autti2021ac,autti2021nonlinear}. It may also be possible to access these phenomena by engineering the topology of the mechanical probes\cite{arrayas2021design,arrayas2022progress}. Finally, the surface layer is also expected to host Majorana zero modes \cite{Murakawa2011,Shapiro2013} that detailed transport measurements may reveal. These research avenues has the potential to transform our understanding of this versatile macroscopic quantum system.

%This energy range could be probed using levitating probes not in contact with container walls\cite{arrayas2021design,arrayas2022progress}.

\section*{Author contributions}

The measurements and data analysis were carried out by S.A., A.J., M.P., R.S., J.V., A.A.S., and D.E.Z. The experimental protocols were developed by R.S., J.V. and D.E.Z. Theoretical work, interpretation of the results, and background investigations were done by S.A. with contributions from R.P.H., G.R.P., V.T., V.V.Z. and D.E.Z.  The manuscript was prepared by S.A. with help from G.R.P., while all authors provided comments and clarifications. D.E.Z. supervised the experiments.


\section*{Acknowledgements}
We thank Grigory Volovik, Jaakko Nissinen and Petri Heikkinen for stimulating discussions. This work was funded by UKRI EPSRC (EP/P024203/1, EP/X004597/1, and EP/W015730/1) and STFC (ST/T006773/1), as well as EU H2020 European Microkelvin Platform (Grant Agreement 824109). We acknowledge M.G. Ward and A. Stokes for their excellent technical support. S.A. acknowledges financial support from the Jenny and Antti Wihuri Foundation.

\section*{Competing interests}
 The authors declare no competing interests

 \section*{Data availability}
 The data that supports the findings of this study are available in Lancaster University data repository at [link/reference to be added]

 \section*{Code availability}
No specialised software was used in analysing the data.

\section*{Methods}


All experimental data and parameter values in this Article are for the saturated vapour pressure, which is vanishingly small at ultra-low temperatures. The superfluid transition temperature at this pressure is $T_\mathrm{c}\approx\, $\SI{920}{\micro\kelvin}. The superfluid sample is contained in a box made from Stycast paper composite containing sintered silver heat exchangers, surrounded by a guard cell also filled with cold $^3$He.\cite{Sinters2020} The motion of the goal-post wire\cite{Bradley2011114} in the cell is illustrated schematically in Fig.~\ref{goalpost}.

\begin{figure}[htb!]
	\includegraphics[width=1\linewidth]{fig2_goalpost.pdf}
	\caption{{\bf The goal-post shaped wire:} We can move the cross bar of the goal-post wire at constant speed over a distance of several mm, enabling the current experiments\cite{Zmeev2014}. The wire is surrounded by a volume of superfluid used as a bolometer for detecting the heat released from the motion of the wire\cite{autti2020fundamental}. The detection coils are used for calibrating the velocity of the wire cross bar.}
	\label{goalpost}
\end{figure}

\subsection*{Vibrating wire thermometry and bolometry}

The changes in the bulk thermal quasiparticle density can be measured using a superconducting \SI{4.5}{\micro\meter}-thick vibrating wire, whose resonance width $\Delta f \propto \exp(-\varDelta/k_B T)  \propto F$.\cite{fisher1991microscopic,PhysRevB.57.14381,Guenault1986} Here $\varDelta$ is the superfluid gap and $k_B$ the Boltzmann constant, $F_{\SI{4}{\micro\meter}}$ is the drag force acting on the vibrating wire resonator, $F$ is the drag force acting on the cross bar of the goalpost-shaped wire, and $T$ is the superfluid temperature. This implies that $Q(T)\propto \Delta f$ with the proportionality constant determined by the ratio of the probe diameters and the thermal quasiparticle scattering efficiency, which depends on probe velocity. For further confirmation we scale the measured $F_{\SI{4}{\micro\meter}}$ by the ratio of cross-sectional areas and according to Equation~17 in Ref.~\citenum{fisher1991microscopic} (see also Ref.~\citenum{enrico_thesis}). Estimating $Q(T)$ this way yields the black line in Fig.~\ref{exp_relax}, in good agreement with measured $Q(T)+2 q$ (green points and line) considering that no fitting parameters were used and only the drag force experienced by the cross bar is included in the estimate (wire legs are excluded).
 
The heat release due to ejected bound quasiparticles is monitored by using the surrounding superfluid volume as a bolometer. Changes in temperature are measured using the thermometer wire resonator. Typical bolometer data curves are shown in Figure~\ref{bulk_measurement}. The temperature is stable before the goalpost wire is moved at $t=0$, and the peak of the bulk temperature increase occurs well after the wire motion ends. This is because the bolometer readout time, determined by the thermometer wire resonance width $\Delta f$ and the thermalisation time of the bolometer,\cite{bib:Winkelmann:2007} is of the order of a second. Comparing two data curves with the delay between the two cycles of crossbar motion  0 and 30\,ms. The temperature peak shifts according to delay $\Delta t$ added between the two cycles. 

The bolometer is calibrated using resonant AC measurements of the goalpost-shaped wire\cite{MSkyba_phd}, where the energy output can be directly recorded with a four-point measurement. The data obtained is fitted with the BCS heat capacity\cite{vollhardt2013superfluid} using the effective volume of the sample as a fitting parameter. The fitted volume is $16~\mathrm{cm}^3$, which falls between the free volume of the sample container, $15~\mathrm{cm}^3$, and the total volume of the sample container including the volume within the heat exchangers, $32~\mathrm{cm}^3$.

%% HERE  %%

\begin{figure}[htb!]
	\includegraphics[width=1\linewidth]{fig6_temp_pulse.pdf}
	\caption{{\bf Temperature evolution of the bulk liquid after a double cycle:} The superfluid temperature is stable before the cross bar motion starts at $t=0$. The cross bar motion takes less than $30$\,ms for $\Delta t=0$, but the superfluid bolometer reacts slowly to this sudden quasiparticle release. If we look first at the $\Delta t =0$\,ms up-up data, only three pulses of bound quasiparticle are emitted, producing the smallest rise in temperature. For the  $\Delta t = 30$\,ms data, four pulses of quasiparticles are emitted giving a larger temperature swing of the bulk liquid.  For comparison, reversal of the direction results in increased (as opposed to depleted) bound quasiparticle release, as shown in the $\Delta t = 0$\,ms up-down data. We can use these curves as transfer functions to infer the level of filling of the left-hand bound state band after the first cycle as a function of the delay time. The right y-axis is drawn so that the peak value of each curve corresponds to the total energy release to the bulk liquid. The temperature of the bulk superfluid was $0.22$\,mK in this measurement. }
	\label{bulk_measurement}
\end{figure}



\subsection*{Escape condition}

All fermionic quasiparticles in superfluid $^3$He move at the Fermi velocity ($\approx50$\,m/s). However, the bound quasiparticles are nearly perfectly retroreflected from the edge of the surface quantum well owing to Andreev reflection. If reflection from the wire surface is specular, the resulting bound quasiparticle dispersion is the Dirac dispersion $E=v_\mathrm{L} p_{||}$. The corresponding group velocity $v_\mathrm{L}$ is the drift speed arising from the minute misalignment of the inbound and outbound trajectories in the Andreev reflection process\cite{mizushima2015symmetry}. 

Let us assume that the bound quasiparticle system is in equilibrium at zero temperature. This means that the quasiparticle dispersion bands are filled up to the Fermi energy, selected to be equal to zero for simplicity in the schematic Figures \ref{cartoon}, \ref{bands_fig} and \ref{ramps_details}. If the wire is accelerated instantaneously to velocity $v$, the energetic escape condition for the highest-energy bound quasiparticles in vector form is $\mathbf{v}_\mathrm{fl} \cdot \mathbf{\hat{p}}_\mathrm{in} \ge \varDelta/p_\mathrm{F} + \mathbf{v}\cdot \mathbf{\hat{p}}_\mathrm{out}$ where $\mathbf{\hat{p}}_\mathrm{in}$ is a unit vector that corresponds to the bound quasiparticle's momentum during the increase in $\mathbf{v}_\mathrm{fl}$ and $\mathbf{\hat{p}}_\mathrm{out}$ is the direction of escape to bulk. The local flow velocity near the wire surface follows $\mathbf{v}_\mathrm{fl}=2\cos(\theta) v \hat{\mathbf{\theta}}$, where $\hat{\mathbf{\theta}}$ is the azimuthal unit vector perpendicular to the cylinder radius.

For increasing $v$, the escape condition is first satisfied for $\mathbf{\hat{p}}_\mathrm{in}$ and $ \mathbf{v}_\mathrm{fl} $ in the same direction while  $\mathbf{\hat{p}}_\mathrm{in}$ is antiparallel with $\mathbf{\hat{p}}_\mathrm{out} $ for quasiparticles at the wire vertices ($\theta = 0$) where $v_\mathrm{fl} $ reaches its maximum. In this case we get the well-known critical cross-bar velocity $v_\mathrm{c}=\varDelta/(3 p_\mathrm{F})=v_\mathrm{L}/3$. This process is schematically illustrated in Fig.~\ref{bands_fig}. We can use similar arguments for deceleration from a steady state configuration at nonzero $v_\mathrm{fl}$, obtaining the critical velocity $v_\mathrm{c}'=\varDelta/(2 p_\mathrm{F})=v_\mathrm{L}/2$. 

At velocities significantly higher than $v_\mathrm{c}$, the vector picture is more complicated. For simplicity, the main text and Figs.~\ref{bands_fig} and \ref{ramps_details} only discuss scalar quantities along the direction of the external flow. At $v=45~$mm/s, the bulk escape process concerns about 90\% of the cross bar surface (see Fig.~\ref{quantum_well}b) , but the largest contribution of the bound state escape originates from the vicinity of the vertices where $v_\mathrm{fl}$ is the largest. Precise calculation of the distribution requires a three-dimensional treatment of the system, and such numerical simulations are left for the future. 

%With high-enough ${v}_\mathrm{fl}$ the energetical escape condition is satisfied in all directions $\mathbf{\hat{p}}_\mathrm{in}$ and $\mathbf{\hat{p}}_\mathrm{out} $, that is, also without the scattering step. However, the observation that closing the gap in the bulk superfluid by a flow that exceeds $v_\mathrm{L}$ results in no drag force or dissipation hints that the scattering step is nevertheless required to create dissipation.

We note that oscillatory motion has been speculated to overheat the bound state system enough to result in observable bound state escape below $v_\mathrm{L}/3$, \cite{LAMBERT1992294} but no such ``pumping" is observed in our experiments. The pumping of quasiparticles towards higher energies is unlikely because the quasiparticle redistribution is governed by diffusion, as discussed below.


\begin{figure}[htb!]
	\includegraphics[width=1\linewidth]{fig4_bands.pdf}
	\caption{{\bf Schematic presentation of the quasiparticle dispersion curves:}  The panels represent dispersion curves for the bound quasiparticles and for quasiparticles in the bulk superfluid during an acceleration, steady velocity, and deceleration cycle with the indicated times referring to the initial ``up" sequence shown in Fig.~\ref{flow_schematic}. In panel ({\bf a}) the wire is stationary. As we apply an increasing superflow along the cross bar surface (in the wire frame) the quasiparticle bands become tipped  until as in ({\bf b})  those bound quasiparticles with energies above the minimum in the bulk liquid escape into the bulk. When the acceleration ceases, no more bound quasiparticles can escape, and equilibrium in the wire frame as shown in panel ({\bf c}) is re-established by diffusion. On deceleration, a second burst of bound quasiparticles escapes into the bulk. Finally, with the superflow velocity again zero as shown in panel ({\bf d}), the bands return to their initial state, but leaving the bound quasiparticles to redistribute via diffusion. A more detailed description of the same process is shown in Fig.~\ref{ramps_details}. Note that this figure is for illustration purposes only. The local dispersion curves cannot be really represented in this way, especially for the depletion situation, but it gives the gist of the idea. }
	\label{bands_fig}
\end{figure}


\subsection*{Diffusion rate}

Assuming there are no quasiparticle-quasiparticle collisions, we can estimate the mean free path in the surface layer as the longest distance that a quasiparticle can travel without changing the direction of the group velocity. This distance is $l_{||}=2 \sqrt{2 R a \xi}$, where $R$ is the cross bar radius, $\xi$ is the coherence length, and $a \xi$ is the effective thickness of the surface layer (see Fig.\ref{quantum_well}b). If we assume $a=3$ as an estimate of the surface layer edge where the most energetic quasiparticles (the ones that the experiment is sensitive to) would be reflected\cite{Nagato19981135}, then $l_{||}\approx$\SI{12}{\micro \meter} at zero pressure. This yields the diffusion constant $D\sim l_{||} v_\mathrm{qp}=l_{||} v_\mathrm{L}$. Note that along the length of the cross bar cylinder the system is homogeneous, and the diffusion experiment is not sensitive to the mean free path in this direction.

We can now solve the diffusion equation $\partial_t n = D \nabla^2 n$. Here $n$ stands for the quasiparticle population which is out of equilibrium, $\nabla^2$ is the Laplace operator and $\partial_t$ stands for time derivative. The Fermi sea seen in Fig.~\ref{ramps_details} at $t=0$ at energies below zero is uniformly filled and there are no quasiparticles above zero energy. Thus, in that case $n=0$. The population carrying the momentum imbalance when the wire is stopped depends on the details of the dispersion relation, but we estimate that it initially follows the local flow velocity so that $n(\theta) \sim \cos \theta$. The resulting time dependence is $n(t,\theta)=n(t=0,\theta) \exp(-t  D/R^2)$, yielding the diffusion time constant $\tau=R^2/D$. Inserting $R=\,$\SI{63.5}{\micro\meter} gives $\tau\approx 13$~ms. On the other hand, fitting $a$ to the experimentally observed time constant gives $a\approx10$, which exceeds the theoretical expectation by a factor of $\sim2$.\cite{Nagato19981135} The feasibility of these values of $a$ is discussed in the next section.

%In a flat geometry diffusion is accelerated. However, calculations of the mean free path for similar bound quasiparticles in vortex cores yield $l_{||} \sim$\SI{1}{\micro\meter} along the vortex due to quasiparticle-quasiparticle collisions \cite{silaev2013dissipative}. This estimate is not directly applicable to a two-dimensional system, but it hints that the flat-surface mean free path may not be only limited by geometry. Probing this process directly requires facilitating local thermometry in the bound-state system. 

\subsection*{Heat release from bound quasiparticles}

The total energy released during the measurement cycle ``up-up" (green+blue lines in Fig.~\ref{flow_schematic}) is $Q_\mathrm{tot}^\mathrm{up-up}= [Q+2 q] + [Q_\mathrm{up}+ q_\mathrm{up}+  q] $, where the brackets separate the contributions from the first and second phases of cross bar motion. Here $Q$ and $Q_\mathrm{up}$ are due to the drag force from collisions with thermal bulk quasiparticles. The bulk thermal quasiparticle density increases by less than 10\% in the course of a typical up cycle, thus $Q_\mathrm{up} \approx Q$. Note also that $Q_\mathrm{up}$ is independent of $\Delta t$, which means that the small difference between $Q$ and $Q_\mathrm{up}$ can be ignored in the analysis of the bound state dynamics. 

The first acceleration phase and each deceleration phase releases approximately the same amount of heat originating from the bound quasiparticle system, denoted $q$. That is, the critical velocity for quasiparticle release from the acceleration is $v_\mathrm{L}/3$ while that from the deceleration is $v_\mathrm{c}'=v_\mathrm{L}/2$. This means that the deceleration should release less quasiparticles than the acceleration, but the difference can be ignored if the wire moves at a speed much higher than $v_c$ (here $5 v_c$). In the above expression for $Q_\mathrm{tot}^\mathrm{up-up}$ we have thus approximated that the heat released from the deceleration is equal to that from the acceleration. 

The heat release from the second acceleration phase is $q_\mathrm{up} = q (1- \exp(-(\Delta t + t_0) /\tau))$. That is, the bound state population available for ejection during acceleration is depleted by the first cycle. Full depletion of the bound quasiparticles that are available for bulk escape implies $q_\mathrm{up}=0$, and after a long enough recovery time $q_\mathrm{up}=q$. For intermediate values of $\Delta t$ the bound state population available for bulk ejection recovers exponentially owing to the diffusion with time constant $\tau$. Note that the bound state recovery starts as soon as the deceleration at the end of the first up cycle begins, and continues until the same velocity is reached again during the following acceleration. This process is approximately accounted for by adding $t_0=6$\,ms (sum of the acceleration and deceleration times) to the recovery time $\Delta t$ in the exponential decay expression. 

For an ``up-down" cycle (green+red lines in Fig.~\ref{flow_schematic}), the energy release is expected initially to be larger than for up-up cycles (see Fig.~\ref{ramps_details}), decreasing as a function of $\Delta t$ and reaching the same level as the ``up-up" cycles at large $\Delta t$. The time dependence is determined by the same diffusion process as detailed above. The total energy released is therefore $Q_\mathrm{tot}^\mathrm{up-down}= [Q+2 q]+[Q_\mathrm{down} + \bar{q}_\mathrm{down} + 2 q] $. With similar arguments as above, $Q_\mathrm{down} \approx Q$. Note, however, that $Q_\mathrm{up}>Q_\mathrm{down}$ owing to a hysteresis in the distance covered by the cross bar\cite{autti2020fundamental} . This effect is proportional to the density of thermal bulk quasiparticles and therefore vanishes at the lowest temperatures measured in this Article. 

The diffusion picture does not determine the magnitude of the time dependent excess of bound quasiparticles ejected by the acceleration in the up-down cycle. Separating the asymptotic bound state contribution from the decaying excess, the excess heat release from the second phase of motion is denoted $\bar{q}_\mathrm{down} =  q_\mathrm{down}\exp(-(\Delta t + t_0) /\tau)$. The number of bound quasiparticles released is a faster-than-linear function of velocity $v$ at $v>v_c$ ,\cite{autti2020fundamental} which hints that $q_\mathrm{down}>q$, as confirmed in the main text.

We can estimate the available quasiparticle energy release from the surface layer as follows. The gap suppression region around the cross bar has the volume $V=2 \pi R  a \xi L$ ($L$ is the cross bar length). Taking the diffusion calculation above literally yields the self-consistent thickness of the layer $a \xi$ with $a \approx 10$. The Doppler shift energy bridges the bulk escape at cross bar velocity $v=v_\mathrm{c}$. If we assume the surface layer is populated according to the normal state density of states $N(0)$, then the energy release from the entire cross bar surface is $q \sim V N(0) \varDelta^2 (v/v_\mathrm{c}-1)^2 \approx 13~$pJ for $v=45~$mm/s, in decent agreement with both the measured value with (12\,pJ) and without (7\,pJ) $^4$He preplating of the wire surface. Choosing a more conservative $a=3$ yields $q\approx 4~$pJ. We emphasise that this estimate neglects many important contributions, such as the quasiparticle release from the legs of the moving superconducting wire (inclusion of which would increase the energy) and the fact that a significant part of the cross bar surface has a smaller flow velocity than the maximum at the vertices, which corresponds to $v_\mathrm{c}$ (thus, decreasing the released energy). Furthermore, according to Ref.~\citenum{LAMBERT1992294} page 296, the gap suppression near the wire is significantly expanded by the increased density of quasiparticles when $v> 2 v_\mathrm{c}$, which would act to increase the quasiparticle emission. This effect may explain why the fitted parameter $a$ appears so large.


\subsection*{Kapitza resistance and bound quasiparticle life time}

The experimentally determined thermal Kapitza resistance between phonons in a metal and superfluid $^3$He quasiparticles in the bulk superfluid is in the range $R_\mathrm{K} \sim10^4$ to $10^5$\SI{}{\kelvin\meter^2\watt^{-1}}. Here the upper end of the range corresponds to a $^4$He preplated and the lower end to pure $^3$He interface between the fluid and the solid. These values are obtained by extrapolation to \SI{1}{\milli\kelvin} using the data in Ref.~\citenum{PhysRevB.54.R9639}. The temperature of the surface-bound quasiparticles is not above \SI{1}{\milli\kelvin} because otherwise they would escape to the bulk. For lower temperatures the Kapitza resistance increases as $R_\mathrm{K} \sim 1/T$.\cite{Sinters2020} For a pessimistic estimate of the decoupling of the cross bar phonons and the bound quasiparticles, we therefore take the measured pure $^3$He $R_\mathrm{K}$, and assume the bound quasiparticle system is at \SI{1}{\milli\kelvin} temperature during the decay described by $\tau$ in the main text. 

The decay of energy via the Kapitza resistance is exponential with the time constant $\tau_{R_\mathrm{K}} = R_\mathrm{K}  C $. Here $C$ is the heat capacity of the body of heat equilibrating via $R_\mathrm{K}$. If we approximate the bound quasiparticle system to be a layer of normal fluid of thickness $a \xi$ at \SI{1}{\milli\kelvin} right after the cross bar has stopped (consistent with the temperature chosen above), the resulting quasiparticle life time (decay rate of energy) in the bound quasiparticle system is $\tau_{R_\mathrm{K}} \approx 12$~s for $a=3$. Thus, even in pessimistic estimate the bound quasiparticles exchange a negligible amount of energy with the wire phonons. 




%\bibliography{Flopper_bibliography}
%\bibliographystyle{naturemag}

\begin{thebibliography}{10}
\expandafter\ifx\csname url\endcsname\relax
  \def\url#1{\texttt{#1}}\fi
\expandafter\ifx\csname urlprefix\endcsname\relax\def\urlprefix{URL }\fi
\providecommand{\bibinfo}[2]{#2}
\providecommand{\eprint}[2][]{\url{#2}}

\bibitem{autti2020fundamental}
\bibinfo{author}{Autti, S.} \emph{et~al.}
\newblock \bibinfo{title}{Fundamental dissipation due to bound fermions in the
  zero-temperature limit}.
\newblock \emph{\bibinfo{journal}{Nature Communications}}
  \textbf{\bibinfo{volume}{11}}, \bibinfo{pages}{4742} (\bibinfo{year}{2020}).

\bibitem{Nagato19981135}
\bibinfo{author}{Nagato, Y.}, \bibinfo{author}{Yamamoto, M.} \&
  \bibinfo{author}{Nagai, K.}
\newblock \bibinfo{title}{Rough surface effects on the p-wave {Fermi}
  superfluids}.
\newblock \emph{\bibinfo{journal}{Journal of Low Temperature Physics}}
  \textbf{\bibinfo{volume}{110}}, \bibinfo{pages}{1135--1171}
  (\bibinfo{year}{1998}).

\bibitem{Nagai2008}
\bibinfo{author}{Nagai, K.}, \bibinfo{author}{Nagato, Y.},
  \bibinfo{author}{Yamamoto, M.} \& \bibinfo{author}{Higashitani, S.}
\newblock \bibinfo{title}{Surface bound states in superfluid $^3${He}}.
\newblock \emph{\bibinfo{journal}{Journal of the Physical Society of Japan}}
  \textbf{\bibinfo{volume}{77}}, \bibinfo{pages}{111003}
  (\bibinfo{year}{2008}).

\bibitem{Murakawa2011}
\bibinfo{author}{Murakawa, S.} \emph{et~al.}
\newblock \bibinfo{title}{Surface {Majorana} cone of the superfluid $^3${He}
  {B} phase}.
\newblock \emph{\bibinfo{journal}{Journal of the Physical Society of Japan}}
  \textbf{\bibinfo{volume}{80}}, \bibinfo{pages}{013602}
  (\bibinfo{year}{2011}).

\bibitem{Okuda2012}
\bibinfo{author}{Okuda, Y.} \& \bibinfo{author}{Nomura, R.}
\newblock \bibinfo{title}{Surface {Andreev} bound states of superfluid $^3${He}
  and {Majorana} fermions}.
\newblock \emph{\bibinfo{journal}{Journal of Physics Condensed Matter}}
  \textbf{\bibinfo{volume}{24}}, \bibinfo{pages}{343201}
  (\bibinfo{year}{2012}).

\bibitem{Aoki2005}
\bibinfo{author}{Aoki, Y.} \emph{et~al.}
\newblock \bibinfo{title}{Observation of surface {Andreev} bound states of
  superfluid $^3${He} by transverse acoustic impedance measurements}.
\newblock \emph{\bibinfo{journal}{Physical Review Letters}}
  \textbf{\bibinfo{volume}{95}}, \bibinfo{pages}{075301}
  (\bibinfo{year}{2005}).

\bibitem{Zheng2016}
\bibinfo{author}{Zheng, P.}, \bibinfo{author}{Jiang, W.},
  \bibinfo{author}{Barquist, C.}, \bibinfo{author}{Lee, Y.} \&
  \bibinfo{author}{Chan, H.}
\newblock \bibinfo{title}{Anomalous damping of a microelectromechanical
  oscillator in superfluid $^3${He-B}}.
\newblock \emph{\bibinfo{journal}{Physical Review Letters}}
  \textbf{\bibinfo{volume}{117}}, \bibinfo{pages}{195301}
  (\bibinfo{year}{2016}).

\bibitem{silaev2014andreev}
\bibinfo{author}{Silaev, M.} \& \bibinfo{author}{Volovik, G.}
\newblock \bibinfo{title}{Andreev-{Majorana} bound states in superfluids}.
\newblock \emph{\bibinfo{journal}{Journal of Experimental and Theoretical
  Physics}} \textbf{\bibinfo{volume}{119}}, \bibinfo{pages}{1042--1057}
  (\bibinfo{year}{2014}).

\bibitem{volovik2009fermion}
\bibinfo{author}{Volovik, G.~E.}
\newblock \bibinfo{title}{Fermion zero modes at the boundary of superfluid
  $^3${He-B}}.
\newblock \emph{\bibinfo{journal}{JETP letters}} \textbf{\bibinfo{volume}{90}},
  \bibinfo{pages}{398--401} (\bibinfo{year}{2009}).

\bibitem{dmitriev2005separation}
\bibinfo{author}{Dmitriev, V.~V.}, \bibinfo{author}{Levitin, L.~V.},
  \bibinfo{author}{Zavjalov, V.~V.} \& \bibinfo{author}{Zmeev, D.~Y.}
\newblock \bibinfo{title}{Separation of $^3${H}e from $^3${H}e-$^4${H}e mixture
  by means of adsorption}.
\newblock \emph{\bibinfo{journal}{Journal of Low Temperature Physics}}
  \textbf{\bibinfo{volume}{138}}, \bibinfo{pages}{877--880}
  (\bibinfo{year}{2005}).

\bibitem{Bradley2016}
\bibinfo{author}{Bradley, D.} \emph{et~al.}
\newblock \bibinfo{title}{Breaking the superfluid speed limit in a fermionic
  condensate}.
\newblock \emph{\bibinfo{journal}{Nature Physics}}
  \textbf{\bibinfo{volume}{12}}, \bibinfo{pages}{1017--1021}
  (\bibinfo{year}{2016}).

\bibitem{Zmeev2014}
\bibinfo{author}{Zmeev, D.~E.}
\newblock \bibinfo{title}{A method for driving an oscillator at a quasi-uniform
  velocity}.
\newblock \emph{\bibinfo{journal}{Journal of Low Temperature Physics}}
  \textbf{\bibinfo{volume}{175}}, \bibinfo{pages}{480--485}
  (\bibinfo{year}{2014}).

\bibitem{PhysRevB.75.024514}
\bibinfo{author}{Kopnin, N.~B.} \emph{et~al.}
\newblock \bibinfo{title}{Enhanced vortex heat conductance in mesoscopic
  superconductors}.
\newblock \emph{\bibinfo{journal}{Phys. Rev. B}} \textbf{\bibinfo{volume}{75}},
  \bibinfo{pages}{024514} (\bibinfo{year}{2007}).

\bibitem{silaev2013dissipative}
\bibinfo{author}{Silaev, M.~A.}
\newblock \bibinfo{title}{Dissipative vortex motion in fermi superfluids at
  ultra low temperatures}.
\newblock \emph{\bibinfo{journal}{Journal of Low Temperature Physics}}
  \textbf{\bibinfo{volume}{171}}, \bibinfo{pages}{539--550}
  (\bibinfo{year}{2013}).

\bibitem{wu2013majorana}
\bibinfo{author}{Wu, H.} \& \bibinfo{author}{Sauls, J.}
\newblock \bibinfo{title}{Majorana excitations, spin and mass currents on the
  surface of topological superfluid $^3${He-B}}.
\newblock \emph{\bibinfo{journal}{Physical Review B}}
  \textbf{\bibinfo{volume}{88}}, \bibinfo{pages}{184506}
  (\bibinfo{year}{2013}).

\bibitem{Mineev2016}
\bibinfo{author}{Mineev, V.~P.}
\newblock \bibinfo{title}{{NMR} properties of the polar phase of superfluid
  $^3${He} in anisotropic aerogel under rotation}.
\newblock \emph{\bibinfo{journal}{J Low Temp. Phys.}}
  \textbf{\bibinfo{volume}{184}}, \bibinfo{pages}{1007--1014}
  (\bibinfo{year}{2016}).

\bibitem{PolarDmitriev}
\bibinfo{author}{Dmitriev, V.~V.}, \bibinfo{author}{Senin, A.~A.},
  \bibinfo{author}{Soldatov, A.~A.} \& \bibinfo{author}{Yudin, A.~N.}
\newblock \bibinfo{title}{Polar phase of superfluid $^{3}\mathrm{He}$ in
  anisotropic aerogel}.
\newblock \emph{\bibinfo{journal}{Physical Review Letters}}
  \textbf{\bibinfo{volume}{115}}, \bibinfo{pages}{165304}
  (\bibinfo{year}{2015}).

\bibitem{PhysRevLett.121.025303}
\bibinfo{author}{Autti, S.} \emph{et~al.}
\newblock \bibinfo{title}{Bose-einstein condensation of magnons and spin
  superfluidity in the polar phase of $^{3}\mathrm{He}$}.
\newblock \emph{\bibinfo{journal}{Physical Review Letters}}
  \textbf{\bibinfo{volume}{121}}, \bibinfo{pages}{025303}
  (\bibinfo{year}{2018}).

\bibitem{autti2020exceeding}
\bibinfo{author}{Autti, S.} \emph{et~al.}
\newblock \bibinfo{title}{Exceeding the {Landau} speed limit with topological
  {Bogoliubov Fermi} surfaces}.
\newblock \emph{\bibinfo{journal}{Physical Review Research}}
  \textbf{\bibinfo{volume}{2}}, \bibinfo{pages}{033013} (\bibinfo{year}{2020}).

\bibitem{HQVs_prl}
\bibinfo{author}{Autti, S.} \emph{et~al.}
\newblock \bibinfo{title}{Observation of half-quantum vortices in topological
  superfluid $^{3}\mathrm{He}$}.
\newblock \emph{\bibinfo{journal}{Physical Review Letters}}
  \textbf{\bibinfo{volume}{117}}, \bibinfo{pages}{255301}
  (\bibinfo{year}{2016}).

\bibitem{fisher1991microscopic}
\bibinfo{author}{Fisher, S.}, \bibinfo{author}{Pickett, G.} \&
  \bibinfo{author}{Watts-Tobin, R.}
\newblock \bibinfo{title}{A microscopic calculation of the force on a wire
  moving through superfluid $^3${He-B} in the ballistic regime}.
\newblock \emph{\bibinfo{journal}{Journal of Low Temperature Physics}}
  \textbf{\bibinfo{volume}{83}}, \bibinfo{pages}{225--235}
  (\bibinfo{year}{1991}).

\bibitem{enrico1996specular}
\bibinfo{author}{Enrico, M.} \& \bibinfo{author}{Watts-Tobin, R.}
\newblock \bibinfo{title}{Specular and diffuse scattering of quasiparticles by
  a macroscopic object moving through superfluid $^3${He-B}}.
\newblock \emph{\bibinfo{journal}{Journal of Low Temperature Physics}}
  \textbf{\bibinfo{volume}{102}}, \bibinfo{pages}{103--116}
  (\bibinfo{year}{1996}).

\bibitem{enrico_thesis}
\bibinfo{author}{Enrico, M.~P.}
\newblock \emph{\bibinfo{title}{Ballistic excitation scattering processes in
  superfluid $^3${He-B} and spin-lattice relaxation in copper down to
  \SI{7}{\micro\kelvin}}}.
\newblock Ph.D. thesis, \bibinfo{school}{Lancaster University, UK}
  (\bibinfo{year}{1995}).

\bibitem{VolovikBook}
\bibinfo{author}{Volovik, G.~E.}
\newblock \emph{\bibinfo{title}{The Universe in a Helium Droplet}}
  (\bibinfo{publisher}{Oxford University Press}, \bibinfo{year}{2003}).

\bibitem{vollhardt2013superfluid}
\bibinfo{author}{Vollhardt, D.} \& \bibinfo{author}{W{\"o}lfle, P.}
\newblock \emph{\bibinfo{title}{The Superfluid Phases of Helium 3}}
  (\bibinfo{publisher}{Dover Publications}, \bibinfo{year}{2013}).

\bibitem{lotnyk2020thermal}
\bibinfo{author}{Lotnyk, D.} \emph{et~al.}
\newblock \bibinfo{title}{Thermal transport of helium-3 in a strongly confining
  channel}.
\newblock \emph{\bibinfo{journal}{Nature communications}}
  \textbf{\bibinfo{volume}{11}}, \bibinfo{pages}{4843} (\bibinfo{year}{2020}).

\bibitem{heikkinen2021fragility}
\bibinfo{author}{Heikkinen, P.} \emph{et~al.}
\newblock \bibinfo{title}{Fragility of surface states in topological superfluid
  $^3${He}}.
\newblock \emph{\bibinfo{journal}{Nature communications}}
  \textbf{\bibinfo{volume}{12}}, \bibinfo{pages}{1--8} (\bibinfo{year}{2021}).

\bibitem{todoshchenko2020mobile}
\bibinfo{author}{Todoshchenko, I.} \emph{et~al.}
\newblock \bibinfo{title}{Mobile solid $^3${He} on carbon nanotube promoted by
  topological frustration} (\bibinfo{year}{2020}).
\newblock \eprint{2010.13748}.

\bibitem{Levitin841}
\bibinfo{author}{Levitin, L.~V.} \emph{et~al.}
\newblock \bibinfo{title}{Phase diagram of the topological superfluid $^3${He}
  confined in a nanoscale slab geometry}.
\newblock \emph{\bibinfo{journal}{Science}} \textbf{\bibinfo{volume}{340}},
  \bibinfo{pages}{841--844} (\bibinfo{year}{2013}).

\bibitem{shook2020stabilized}
\bibinfo{author}{Shook, A.} \emph{et~al.}
\newblock \bibinfo{title}{Stabilized pair density wave via nanoscale
  confinement of superfluid $^3${He}}.
\newblock \emph{\bibinfo{journal}{Physical review letters}}
  \textbf{\bibinfo{volume}{124}}, \bibinfo{pages}{015301}
  (\bibinfo{year}{2020}).

\bibitem{PhysRevLett.122.085301}
\bibinfo{author}{Levitin, L.~V.} \emph{et~al.}
\newblock \bibinfo{title}{Evidence for a spatially modulated superfluid phase
  of $^{3}\mathrm{He}$ under confinement}.
\newblock \emph{\bibinfo{journal}{Physical Review Letters}}
  \textbf{\bibinfo{volume}{122}}, \bibinfo{pages}{085301}
  (\bibinfo{year}{2019}).

\bibitem{levitin2014study}
\bibinfo{author}{Levitin, L.} \emph{et~al.}
\newblock \bibinfo{title}{Study of superfluid $^3${He} under nanoscale
  confinement}.
\newblock \emph{\bibinfo{journal}{Journal of Low Temperature Physics}}
  \textbf{\bibinfo{volume}{175}}, \bibinfo{pages}{667--680}
  (\bibinfo{year}{2014}).

\bibitem{zhelev2017ab}
\bibinfo{author}{Zhelev, N.} \emph{et~al.}
\newblock \bibinfo{title}{The {A-B} transition in superfluid helium-3 under
  confinement in a thin slab geometry}.
\newblock \emph{\bibinfo{journal}{Nature communications}}
  \textbf{\bibinfo{volume}{8}}, \bibinfo{pages}{1--9} (\bibinfo{year}{2017}).

\bibitem{autti2021ac}
\bibinfo{author}{Autti, S.} \emph{et~al.}
\newblock \bibinfo{title}{{AC} {Josephson} effect between two superfluid time
  crystals}.
\newblock \emph{\bibinfo{journal}{Nature Materials}}
  \textbf{\bibinfo{volume}{20}}, \bibinfo{pages}{171--174}
  (\bibinfo{year}{2021}).

\bibitem{autti2021nonlinear}
\bibinfo{author}{Autti, S.} \emph{et~al.}
\newblock \bibinfo{title}{Nonlinear two-level dynamics of quantum time
  crystals}.
\newblock \emph{\bibinfo{journal}{Nature communications}}
  \textbf{\bibinfo{volume}{13}}, \bibinfo{pages}{3090} (\bibinfo{year}{2022}).

\bibitem{arrayas2021design}
\bibinfo{author}{Array{\'a}s, M.}, \bibinfo{author}{Trueba, J.~L.},
  \bibinfo{author}{Uriarte, C.} \& \bibinfo{author}{Zmeev, D.~E.}
\newblock \bibinfo{title}{Design of a system for controlling a levitating
  sphere in superfluid $^3${H}e at extremely low temperatures}.
\newblock \emph{\bibinfo{journal}{Scientific Reports}}
  \textbf{\bibinfo{volume}{11}}, \bibinfo{pages}{1--7} (\bibinfo{year}{2021}).

\bibitem{arrayas2022progress}
\bibinfo{author}{Array{\'a}s, M.} \emph{et~al.}
\newblock \bibinfo{title}{Progress on levitating a sphere in cryogenic fluids}.
\newblock \emph{\bibinfo{journal}{Journal of Low Temperature Physics}}
  (\bibinfo{year}{2023}).

\bibitem{Shapiro2013}
\bibinfo{author}{Rosenstein, B.}, \bibinfo{author}{Shapiro, I.} \&
  \bibinfo{author}{Shapiro, B.~Y.}
\newblock \bibinfo{title}{Effect of nanoholes on the vortex core fermion
  spectrum and heat transport in p-wave superconductors}.
\newblock \emph{\bibinfo{journal}{J. Phys.: Condens. Matter}}
  \textbf{\bibinfo{volume}{25}}, \bibinfo{pages}{075701}
  (\bibinfo{year}{2013}).

\bibitem{Sinters2020}
\bibinfo{author}{Autti, S.} \emph{et~al.}
\newblock \bibinfo{title}{Effect of the boundary condition on the {Kapitza}
  resistance between superfluid $^{3}${He-B} and sintered metal}.
\newblock \emph{\bibinfo{journal}{Phys. Rev. B}}
  \textbf{\bibinfo{volume}{102}}, \bibinfo{pages}{064508}
  (\bibinfo{year}{2020}).

\bibitem{Bradley2011114}
\bibinfo{author}{Bradley, D.} \emph{et~al.}
\newblock \bibinfo{title}{A new device for studying low or zero frequency
  mechanical motion at very low temperatures}.
\newblock \emph{\bibinfo{journal}{Journal of Low Temperature Physics}}
  \textbf{\bibinfo{volume}{165}}, \bibinfo{pages}{114--131}
  (\bibinfo{year}{2011}).

\bibitem{PhysRevB.57.14381}
\bibinfo{author}{B\"auerle, C.}, \bibinfo{author}{Bunkov, Y.~M.},
  \bibinfo{author}{Fisher, S.~N.} \& \bibinfo{author}{Godfrin, H.}
\newblock \bibinfo{title}{Temperature scale and heat capacity of superfluid
  ${}^{3}${He-B} in the $100\ensuremath{\mu}\mathrm{K}$ range}.
\newblock \emph{\bibinfo{journal}{Phys. Rev. B}} \textbf{\bibinfo{volume}{57}},
  \bibinfo{pages}{14381--14386} (\bibinfo{year}{1998}).

\bibitem{Guenault1986}
\bibinfo{author}{Gu{\'e}nault, A.~M.}, \bibinfo{author}{Keith, V.},
  \bibinfo{author}{Kennedy, C.~J.}, \bibinfo{author}{Mussett, S.~G.} \&
  \bibinfo{author}{Pickett, G.~R.}
\newblock \bibinfo{title}{The mechanical behavior of a vibrating wire in
  superfluid $^3${He-B} in the ballistic limit}.
\newblock \emph{\bibinfo{journal}{Journal of Low Temperature Physics}}
  \textbf{\bibinfo{volume}{62}}, \bibinfo{pages}{511--523}
  (\bibinfo{year}{1986}).

\bibitem{bib:Winkelmann:2007}
\bibinfo{author}{Winkelmann, C.} \emph{et~al.}
\newblock \bibinfo{title}{Bolometric calibration of a superfluid $^3${He}
  detector for dark matter search: Direct measurement of the scintillated
  energy fraction for neutron, electron and muon events}.
\newblock \emph{\bibinfo{journal}{Nuclear Instruments and Methods in Physics
  Research Section A: Accelerators, Spectrometers, Detectors and Associated
  Equipment}} \textbf{\bibinfo{volume}{574}}, \bibinfo{pages}{264--271}
  (\bibinfo{year}{2007}).

\bibitem{MSkyba_phd}
\bibinfo{author}{Skyba, M.}
\newblock \emph{\bibinfo{title}{Critical flow velocity in superfluid
  $^3${He-B}}}.
\newblock Ph.D. thesis, \bibinfo{school}{Lancaster University, UK}
  (\bibinfo{year}{2016}).

\bibitem{mizushima2015symmetry}
\bibinfo{author}{Mizushima, T.}, \bibinfo{author}{Tsutsumi, Y.},
  \bibinfo{author}{Sato, M.} \& \bibinfo{author}{Machida, K.}
\newblock \bibinfo{title}{Symmetry protected topological superfluid
  $^3${He-B}}.
\newblock \emph{\bibinfo{journal}{Journal of Physics: Condensed Matter}}
  \textbf{\bibinfo{volume}{27}}, \bibinfo{pages}{113203}
  (\bibinfo{year}{2015}).

\bibitem{LAMBERT1992294}
\bibinfo{author}{Lambert, C.}
\newblock \bibinfo{title}{Theory of pair breaking by vibrating macroscopic
  objects in superfluid $^3${He}}.
\newblock \emph{\bibinfo{journal}{Physica B: Condensed Matter}}
  \textbf{\bibinfo{volume}{178}}, \bibinfo{pages}{294 -- 303}
  (\bibinfo{year}{1992}).

\bibitem{PhysRevB.54.R9639}
\bibinfo{author}{Hu, Y.}, \bibinfo{author}{Stecher, G.~J.},
  \bibinfo{author}{Gramila, T.~J.} \& \bibinfo{author}{Richardson, R.~C.}
\newblock \bibinfo{title}{Magnetic coupling in thermal-boundary resistance
  between thin silver films and liquid $^{3}\mathrm{He}$ in the millikelvin
  regime}.
\newblock \emph{\bibinfo{journal}{Phys. Rev. B}} \textbf{\bibinfo{volume}{54}},
  \bibinfo{pages}{R9639--R9642} (\bibinfo{year}{1996}).

\end{thebibliography}


\clearpage
\onecolumngrid


\setcounter{figure}{0}
\renewcommand{\theequation}{S\arabic{equation}}
\renewcommand{\thefigure}{S\arabic{figure}}

\setcounter{equation}{0}
\section*{Extended data}

\begin{figure}[htb!]
	\includegraphics[width=1\linewidth]{ramps_schematic_suppl_v3.pdf}%
	\caption{{\bf Bound quasiparticle dynamics:} The cross bar up-up cycle shown in Fig.~\ref{flow_schematic} starts at $t=0$~ms with an equilibrium bound state population (red and blue disks illustrate quasiparticles). At zero temperature, no quasiparticles are above the Fermi energy (here at $E=0$). After the acceleration ($t=3~$ms), the superflow $v_\mathrm{fl}$ along the surface has shifted the states by $\pm p_\mathrm{F} v_\mathrm{fl}$. The quasiparticles energetic enough to escape to bulk scatter away, and diffusion equilibrates the rest of the spectrum. Here we have assumed this process is finished at $t=10~$ms. The wire is then brought to a halt ($t=13~$ms). The diffusive redistribution of the quasiparticles is determined by the time constant $\tau$ ($t=19~$ms). Moving the wire soon thereafter in either the same direction or the opposite direction ($t=23$~ms) and recording the resulting quasiparticle emission reveals the population dynamics. At $t=35~$ms the wire is stationary again. Note that the spectral evolution is shown here for a Dirac spectrum at zero temperature, the effect of the bulk escape is neglected for simplicity, and the direction of momentum is assumed to be uniformly along the surface. These choices allow for compact visualisation, but they should not be used for detailed calculations.
	}
	\label{ramps_details}
\end{figure}




\end{document}


















