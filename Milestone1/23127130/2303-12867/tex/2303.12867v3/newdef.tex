%\bibliographystyle{unsrt}
\let\lambdabar\relax
%\usepackage{newpxtext,newpxmath}
\let\openbox\relax
\let\Bbbk\relax
\let\coloneqq\relax
\let\eqqcolon\relax

\usepackage[latin1]{inputenc}
\usepackage{amsthm}
\usepackage{amssymb}
\usepackage{amsmath}
\usepackage{bbold}
\usepackage{bbm}
\usepackage{nonfloat}
\usepackage[pdftex]{hyperref}
\usepackage{braket}
\usepackage{dsfont}
\usepackage{mathdots}
\usepackage{mathtools}
\usepackage{enumerate}
\usepackage[shortlabels]{enumitem}
\usepackage{csquotes}
\usepackage{stmaryrd}
\usepackage[cal=boondox]{mathalfa}
\usepackage{graphicx}
\usepackage{stackengine}
\usepackage{scalerel}
\usepackage{xr}
%\usepackage[dvipsnames]{xcolor}
%\usepackage{xcolor}
\usepackage{array}
\usepackage{makecell}
\newcolumntype{x}[1]{>{\centering\arraybackslash}p{#1}}
\usepackage{tikz}
\usepackage{pgfplots}
\usetikzlibrary{shapes.geometric, shapes.misc, positioning, arrows, arrows.meta, decorations.pathreplacing, decorations.pathmorphing, patterns, angles, quotes, calc}
\usepackage{booktabs}
\usepackage{xfrac}
\usepackage{siunitx}
\usepackage{centernot}
\usepackage{comment}
\usepackage{chngcntr}
%\usepackage{caption}
%\usepackage{subcaption}
%\usepackage{ulem}

\newcommand{\ra}[1]{\renewcommand{\arraystretch}{#1}}

\newtheorem{thm}{Theorem}
\newtheorem*{thm*}{Theorem}
\newtheorem{prop}[thm]{Proposition}
\newtheorem*{prop*}{Proposition}
\newtheorem{lemma}[thm]{Lemma}
\newtheorem*{lemma*}{Lemma}
\newtheorem{lemma_app}{Lemma}
\newtheorem{cor}[thm]{Corollary}
\newtheorem*{cor*}{Corollary}
\newtheorem{cj}[thm]{Conjecture}
\newtheorem*{cj*}{Conjecture}
\newtheorem{Def}[thm]{Definition}
\newtheorem*{Def*}{Definition}
\newtheorem{question}[thm]{Question}
\newtheorem{problem}[thm]{Problem}
\newtheorem{definition_app}{Definition}
\newtheorem{remark}{Remark}

% The following is necessary to make "\begin{thm}[{\cite{X}}]" print "Theorem [X]." instead of "Theorem ([X)]." It works also for prop, lemma, cor, and so on.
\makeatletter
\def\thmhead@plain#1#2#3{%
  \thmname{#1}\thmnumber{\@ifnotempty{#1}{ }\@upn{#2}}%
  \thmnote{ {\the\thm@notefont#3}}}
\let\thmhead\thmhead@plain
\makeatother

\theoremstyle{definition}
\newtheorem{rem}[thm]{Remark}
\newtheorem*{note}{Note}
\newtheorem{ex}[thm]{Example}
\newtheorem{axiom}[thm]{Axiom}
\newtheorem{fact}[thm]{Fact}

\newtheorem{manualthminner}{Theorem}
\newenvironment{manualthm}[1]{%
  \renewcommand\themanualthminner{#1}%
  \manualthminner \it
}{\endmanualthminner}

\newtheorem{manualpropinner}{Proposition}
\newenvironment{manualprop}[1]{%
  \renewcommand\themanualpropinner{#1}%
  \manualpropinner \it
}{\endmanualpropinner}

\newtheorem{manuallemmainner}{Lemma}
\newenvironment{manuallemma}[1]{%
  \renewcommand\themanuallemmainner{#1}%
  \manuallemmainner \it
}{\endmanuallemmainner}

\newtheorem{manualcorinner}{Corollary}
\newenvironment{manualcor}[1]{%
  \renewcommand\themanualcorinner{#1}%
  \manualcorinner \it
}{\endmanualcorinner}

\newcommand{\bb}{\begin{equation}\begin{aligned}\hspace{0pt}}
\newcommand{\bbb}{\begin{equation*}\begin{aligned}}
\newcommand{\ee}{\end{aligned}\end{equation}}
\newcommand{\eee}{\end{aligned}\end{equation*}}
\newcommand*{\coloneqq}{\mathrel{\vcenter{\baselineskip0.5ex \lineskiplimit0pt \hbox{\scriptsize.}\hbox{\scriptsize.}}} =}
\newcommand*{\eqqcolon}{= \mathrel{\vcenter{\baselineskip0.5ex \lineskiplimit0pt \hbox{\scriptsize.}\hbox{\scriptsize.}}}}
\newcommand\floor[1]{\left\lfloor#1\right\rfloor}
\newcommand\ceil[1]{\left\lceil#1\right\rceil}
\newcommand{\texteq}[1]{\stackrel{\mathclap{\scriptsize \mbox{#1}}}{=}}
%\renewcommand{\textleq}[1]{\stackrel{\mathclap{\scriptsize \mbox{#1}}}{\leq}}
\newcommand{\textl}[1]{\stackrel{\mathclap{\scriptsize \mbox{#1}}}{<}}
\newcommand{\textg}[1]{\stackrel{\mathclap{\scriptsize \mbox{#1}}}{>}}
%\renewcommand{\textgeq}[1]{\stackrel{\mathclap{\scriptsize \mbox{#1}}}{\geq}}
\newcommand{\ketbra}[1]{\ket{#1}\!\!\bra{#1}}
\newcommand{\ketbraa}[2]{\ket{#1}\!\!\bra{#2}}
\newcommand{\ketbraaa}[3]{\ket{#1}_{{#2}} \!\!\bra{#3}}
\newcommand{\ketbrasub}[1]{\ket{#1}\!\bra{#1}}
\newcommand{\ketbraasub}[2]{\ket{#1}\!\bra{#2}}
\newcommand{\sumno}{\sum\nolimits}
\newcommand{\e}{\varepsilon}
\newcommand{\G}{\mathrm{\scriptscriptstyle G}}

\newcommand{\tmin}{\ensuremath \! \raisebox{2.5pt}{$\underset{\begin{array}{c} \vspace{-4.1ex} \\ \text{\scriptsize min} \end{array}}{\otimes}$}\!}
\newcommand{\tmax}{\ensuremath \! \raisebox{2.5pt}{$\underset{\begin{array}{c} \vspace{-4.1ex} \\ \text{\scriptsize max} \end{array}}{\otimes}$}\! }
\newcommand{\tminit}{\ensuremath \! \raisebox{2.5pt}{$\underset{\begin{array}{c} \vspace{-3.9ex} \\ \text{\scriptsize \emph{min}} \end{array}}{\otimes}$}\!}
\newcommand{\tmaxit}{\ensuremath \!\raisebox{2.5pt}{$\underset{\begin{array}{c} \vspace{-3.9ex} \\ \text{\scriptsize \emph{max}} \end{array}}{\otimes}$}\!}

\newcommand{\tcr}[1]{{\color{red} #1}}
\newcommand{\tcb}[1]{{\color{blue} #1}}
\newcommand{\tcrb}[1]{{\color{red} \bfseries #1}}
\newcommand{\tcbb}[1]{{\color{blue} \bfseries #1}}
\newcommand{\ludo}[1]{{\color{green!60!black} #1}}
\newcommand{\vitt}[1]{{\color{blue!60!black} #1}}

\newcommand{\id}{\mathds{1}}
\newcommand{\R}{\mathds{R}}
\newcommand{\N}{\mathds{N}}
\newcommand{\C}{\mathds{C}}
\newcommand{\E}{\mathds{E}}
\newcommand{\PSD}{\mathrm{PSD}}

\newcommand{\cptp}{\mathrm{CPTP}}
\newcommand{\lo}{\mathrm{LO}}
\newcommand{\onelocc}{\mathrm{LOCC}_\to}
\newcommand{\locc}{\mathrm{LOCC}}
\newcommand{\sep}{\mathrm{SEP}}
\newcommand{\ppt}{\mathrm{PPT}}
\newcommand{\gocc}{\mathrm{GOCC}}
\newcommand{\het}{\mathrm{het}}
\renewcommand{\hom}{\mathrm{hom}}

\newcommand{\CN}{\mathrm{CN}}
\newcommand{\PIO}{\mathrm{PIO}}
\newcommand{\SIO}{\mathrm{SIO}}
\newcommand{\IO}{\mathrm{IO}}
\newcommand{\DIO}{\mathrm{DIO}}
\newcommand{\MIO}{\mathrm{MIO}}

\newcommand{\df}{\textbf{Definition.\ }}
\newcommand{\teo}{\textbf{Theorem.\ }}
\newcommand{\lm}{\textbf{Lemma.\ }}
\newcommand{\prp}{\textbf{Proposition.\ }}
\newcommand{\cll}{\textbf{Corollary.\ }}
\newcommand{\prf}{\emph{Proof.\ }}

\DeclareMathOperator{\Tr}{Tr}
\DeclareMathOperator{\rk}{rk}
\DeclareMathOperator{\cl}{cl}
\DeclareMathOperator{\co}{conv}
\DeclareMathOperator{\cone}{cone}
\DeclareMathOperator{\inter}{int}
\DeclareMathOperator{\Span}{span}
\DeclareMathAlphabet{\pazocal}{OMS}{zplm}{m}{n}
\DeclareMathOperator{\aff}{aff}
\DeclareMathOperator{\ext}{ext}
\DeclareMathOperator{\pr}{Pr}
\DeclareMathOperator{\supp}{supp}
\DeclareMathOperator{\spec}{sp}
\DeclareMathOperator{\sgn}{sgn}
\DeclareMathOperator{\tr}{tr}
\DeclareMathOperator{\Id}{Id}
\DeclareMathOperator{\relint}{relint}
\DeclareMathOperator{\arcsinh}{arcsinh}
\DeclareMathOperator{\erf}{erf}
\DeclareMathOperator{\dom}{dom}
\DeclareMathOperator{\diag}{diag}

\newcommand{\HH}{\pazocal{H}}
\newcommand{\T}{\pazocal{T}}
\newcommand{\D}{\pazocal{D}}
\newcommand{\K}{\pazocal{K}}
\newcommand{\B}{\pazocal{B}}
\newcommand{\Sch}{\pazocal{S}}
\newcommand{\Tsa}{\pazocal{T}_{\mathrm{sa}}}
\newcommand{\Ksa}{\pazocal{K}_{\mathrm{sa}}}
\newcommand{\Bsa}{\pazocal{B}_{\mathrm{sa}}}
\newcommand{\Schsa}{\pazocal{S}_{\mathrm{sa}}}
\newcommand{\NN}{\mathcal{N}}
\newcommand{\MM}{\mathcal{M}}

\newcommand{\lsmatrix}{\left(\begin{smallmatrix}}
\newcommand{\rsmatrix}{\end{smallmatrix}\right)}
\newcommand\smatrix[1]{{%
  \scriptsize\arraycolsep=0.4\arraycolsep\ensuremath{\begin{pmatrix}#1\end{pmatrix}}}}

\stackMath
\newcommand\xxrightarrow[2][]{\mathrel{%
  \setbox2=\hbox{\stackon{\scriptstyle#1}{\scriptstyle#2}}%
  \stackunder[5pt]{%
    \xrightarrow{\makebox[\dimexpr\wd2\relax]{$\scriptstyle#2$}}%
  }{%
   \scriptstyle#1\,%
  }%
}}

\newcommand{\tends}[2]{\xxrightarrow[\! #2 \!]{\mathrm{#1}}}
\newcommand{\tendsn}[1]{\xxrightarrow[\! n\rightarrow \infty\!]{#1}}
\newcommand{\tendsk}[1]{\xxrightarrow[\! k\rightarrow \infty\!]{#1}}
\newcommand{\ctends}[3]{\xxrightarrow[\raisebox{#3}{$\scriptstyle #2$}]{\raisebox{-0.7pt}{$\scriptstyle #1$}}}

\def\stacktype{L}
\stackMath
\def\dyhat{-0.15ex}
\newcommand\myhat[1]{\ThisStyle{%
              \stackon[\dyhat]{\SavedStyle#1}
                              {\SavedStyle\widehat{\phantom{#1}}}}}

\makeatletter
\newcommand*\rel@kern[1]{\kern#1\dimexpr\macc@kerna}
\newcommand*\widebar[1]{%
  \begingroup
  \def\mathaccent##1##2{%
    \rel@kern{0.8}%
    \overline{\rel@kern{-0.8}\macc@nucleus\rel@kern{0.2}}%
    \rel@kern{-0.2}%
  }%
  \macc@depth\@ne
  \let\math@bgroup\@empty \let\math@egroup\macc@set@skewchar
  \mathsurround\z@ \frozen@everymath{\mathgroup\macc@group\relax}%
  \macc@set@skewchar\relax
  \let\mathaccentV\macc@nested@a
  \macc@nested@a\relax111{#1}%
  \endgroup
}

\newcommand{\fakepart}[1]{
 \par\refstepcounter{part}
  \sectionmark{#1}
}

\counterwithin*{equation}{part}
\counterwithin*{thm}{part}
\counterwithin*{figure}{part}

\newenvironment{reusefigure}[2][htbp]
  {\addtocounter{figure}{-1}
   \renewcommand{\theHfigure}{dupe-fig}
   \renewcommand{\thefigure}{\ref{#2}}
   \renewcommand{\addcontentsline}[3]{}
   \begin{figure}[#1]}
  {\end{figure}}

\tikzset{meter/.append style={draw, inner sep=10, rectangle, font=\vphantom{A}, minimum width=30, line width=.8, path picture={\draw[black] ([shift={(.1,.3)}]path picture bounding box.south west) to[bend left=50] ([shift={(-.1,.3)}]path picture bounding box.south east);\draw[black,-latex] ([shift={(0,.1)}]path picture bounding box.south) -- ([shift={(.3,-.1)}]path picture bounding box.north);}}}
\tikzset{roundnode/.append style={circle, draw=black, fill=gray!20, thick, minimum size=10mm}}
\tikzset{squarenode/.style={rectangle, draw=black, fill=none, thick, minimum size=10mm}}

\definecolor{Blues5seq1}{RGB}{239,243,255}
\definecolor{Blues5seq2}{RGB}{189,215,231}
\definecolor{Blues5seq3}{RGB}{107,174,214}
\definecolor{Blues5seq4}{RGB}{49,130,189}
\definecolor{Blues5seq5}{RGB}{8,81,156}

\definecolor{Greens5seq1}{RGB}{237,248,233}
\definecolor{Greens5seq2}{RGB}{186,228,179}
\definecolor{Greens5seq3}{RGB}{116,196,118}
\definecolor{Greens5seq4}{RGB}{49,163,84}
\definecolor{Greens5seq5}{RGB}{0,109,44}

\definecolor{Reds5seq1}{RGB}{254,229,217}
\definecolor{Reds5seq2}{RGB}{252,174,145}
\definecolor{Reds5seq3}{RGB}{251,106,74}
\definecolor{Reds5seq4}{RGB}{222,45,38}
\definecolor{Reds5seq5}{RGB}{165,15,21}

\usepackage{pgfplots}
\newtheorem{theorem}{Theorem}
\newtheorem{conjecture}{Conjecture}
\newtheorem{definition}{Definition}
\newtheorem{example}{Example}