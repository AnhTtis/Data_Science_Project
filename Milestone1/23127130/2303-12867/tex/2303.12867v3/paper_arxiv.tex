\documentclass[aps,twocolumn,tightenlines,superscriptaddress]{revtex4-1}

%\bibliographystyle{unsrt}
\let\lambdabar\relax
%\usepackage{newpxtext,newpxmath}
\let\openbox\relax
\let\Bbbk\relax
\let\coloneqq\relax
\let\eqqcolon\relax

\usepackage[latin1]{inputenc}
\usepackage{amsthm}
\usepackage{amssymb}
\usepackage{amsmath}
\usepackage{bbold}
\usepackage{bbm}
\usepackage{nonfloat}
\usepackage[pdftex]{hyperref}
\usepackage{braket}
\usepackage{dsfont}
\usepackage{mathdots}
\usepackage{mathtools}
\usepackage{enumerate}
\usepackage[shortlabels]{enumitem}
\usepackage{csquotes}
\usepackage{stmaryrd}
\usepackage[cal=boondox]{mathalfa}
\usepackage{graphicx}
\usepackage{stackengine}
\usepackage{scalerel}
\usepackage{xr}
%\usepackage[dvipsnames]{xcolor}
%\usepackage{xcolor}
\usepackage{array}
\usepackage{makecell}
\newcolumntype{x}[1]{>{\centering\arraybackslash}p{#1}}
\usepackage{tikz}
\usepackage{pgfplots}
\usetikzlibrary{shapes.geometric, shapes.misc, positioning, arrows, arrows.meta, decorations.pathreplacing, decorations.pathmorphing, patterns, angles, quotes, calc}
\usepackage{booktabs}
\usepackage{xfrac}
\usepackage{siunitx}
\usepackage{centernot}
\usepackage{comment}
\usepackage{chngcntr}
%\usepackage{caption}
%\usepackage{subcaption}
%\usepackage{ulem}

\newcommand{\ra}[1]{\renewcommand{\arraystretch}{#1}}

\newtheorem{thm}{Theorem}
\newtheorem*{thm*}{Theorem}
\newtheorem{prop}[thm]{Proposition}
\newtheorem*{prop*}{Proposition}
\newtheorem{lemma}[thm]{Lemma}
\newtheorem*{lemma*}{Lemma}
\newtheorem{lemma_app}{Lemma}
\newtheorem{cor}[thm]{Corollary}
\newtheorem*{cor*}{Corollary}
\newtheorem{cj}[thm]{Conjecture}
\newtheorem*{cj*}{Conjecture}
\newtheorem{Def}[thm]{Definition}
\newtheorem*{Def*}{Definition}
\newtheorem{question}[thm]{Question}
\newtheorem{problem}[thm]{Problem}
\newtheorem{definition_app}{Definition}
\newtheorem{remark}{Remark}

% The following is necessary to make "\begin{thm}[{\cite{X}}]" print "Theorem [X]." instead of "Theorem ([X)]." It works also for prop, lemma, cor, and so on.
\makeatletter
\def\thmhead@plain#1#2#3{%
  \thmname{#1}\thmnumber{\@ifnotempty{#1}{ }\@upn{#2}}%
  \thmnote{ {\the\thm@notefont#3}}}
\let\thmhead\thmhead@plain
\makeatother

\theoremstyle{definition}
\newtheorem{rem}[thm]{Remark}
\newtheorem*{note}{Note}
\newtheorem{ex}[thm]{Example}
\newtheorem{axiom}[thm]{Axiom}
\newtheorem{fact}[thm]{Fact}

\newtheorem{manualthminner}{Theorem}
\newenvironment{manualthm}[1]{%
  \renewcommand\themanualthminner{#1}%
  \manualthminner \it
}{\endmanualthminner}

\newtheorem{manualpropinner}{Proposition}
\newenvironment{manualprop}[1]{%
  \renewcommand\themanualpropinner{#1}%
  \manualpropinner \it
}{\endmanualpropinner}

\newtheorem{manuallemmainner}{Lemma}
\newenvironment{manuallemma}[1]{%
  \renewcommand\themanuallemmainner{#1}%
  \manuallemmainner \it
}{\endmanuallemmainner}

\newtheorem{manualcorinner}{Corollary}
\newenvironment{manualcor}[1]{%
  \renewcommand\themanualcorinner{#1}%
  \manualcorinner \it
}{\endmanualcorinner}

\newcommand{\bb}{\begin{equation}\begin{aligned}\hspace{0pt}}
\newcommand{\bbb}{\begin{equation*}\begin{aligned}}
\newcommand{\ee}{\end{aligned}\end{equation}}
\newcommand{\eee}{\end{aligned}\end{equation*}}
\newcommand*{\coloneqq}{\mathrel{\vcenter{\baselineskip0.5ex \lineskiplimit0pt \hbox{\scriptsize.}\hbox{\scriptsize.}}} =}
\newcommand*{\eqqcolon}{= \mathrel{\vcenter{\baselineskip0.5ex \lineskiplimit0pt \hbox{\scriptsize.}\hbox{\scriptsize.}}}}
\newcommand\floor[1]{\left\lfloor#1\right\rfloor}
\newcommand\ceil[1]{\left\lceil#1\right\rceil}
\newcommand{\texteq}[1]{\stackrel{\mathclap{\scriptsize \mbox{#1}}}{=}}
%\renewcommand{\textleq}[1]{\stackrel{\mathclap{\scriptsize \mbox{#1}}}{\leq}}
\newcommand{\textl}[1]{\stackrel{\mathclap{\scriptsize \mbox{#1}}}{<}}
\newcommand{\textg}[1]{\stackrel{\mathclap{\scriptsize \mbox{#1}}}{>}}
%\renewcommand{\textgeq}[1]{\stackrel{\mathclap{\scriptsize \mbox{#1}}}{\geq}}
\newcommand{\ketbra}[1]{\ket{#1}\!\!\bra{#1}}
\newcommand{\ketbraa}[2]{\ket{#1}\!\!\bra{#2}}
\newcommand{\ketbraaa}[3]{\ket{#1}_{{#2}} \!\!\bra{#3}}
\newcommand{\ketbrasub}[1]{\ket{#1}\!\bra{#1}}
\newcommand{\ketbraasub}[2]{\ket{#1}\!\bra{#2}}
\newcommand{\sumno}{\sum\nolimits}
\newcommand{\e}{\varepsilon}
\newcommand{\G}{\mathrm{\scriptscriptstyle G}}

\newcommand{\tmin}{\ensuremath \! \raisebox{2.5pt}{$\underset{\begin{array}{c} \vspace{-4.1ex} \\ \text{\scriptsize min} \end{array}}{\otimes}$}\!}
\newcommand{\tmax}{\ensuremath \! \raisebox{2.5pt}{$\underset{\begin{array}{c} \vspace{-4.1ex} \\ \text{\scriptsize max} \end{array}}{\otimes}$}\! }
\newcommand{\tminit}{\ensuremath \! \raisebox{2.5pt}{$\underset{\begin{array}{c} \vspace{-3.9ex} \\ \text{\scriptsize \emph{min}} \end{array}}{\otimes}$}\!}
\newcommand{\tmaxit}{\ensuremath \!\raisebox{2.5pt}{$\underset{\begin{array}{c} \vspace{-3.9ex} \\ \text{\scriptsize \emph{max}} \end{array}}{\otimes}$}\!}

\newcommand{\tcr}[1]{{\color{red} #1}}
\newcommand{\tcb}[1]{{\color{blue} #1}}
\newcommand{\tcrb}[1]{{\color{red} \bfseries #1}}
\newcommand{\tcbb}[1]{{\color{blue} \bfseries #1}}
\newcommand{\ludo}[1]{{\color{green!60!black} #1}}
\newcommand{\vitt}[1]{{\color{blue!60!black} #1}}

\newcommand{\id}{\mathds{1}}
\newcommand{\R}{\mathds{R}}
\newcommand{\N}{\mathds{N}}
\newcommand{\C}{\mathds{C}}
\newcommand{\E}{\mathds{E}}
\newcommand{\PSD}{\mathrm{PSD}}

\newcommand{\cptp}{\mathrm{CPTP}}
\newcommand{\lo}{\mathrm{LO}}
\newcommand{\onelocc}{\mathrm{LOCC}_\to}
\newcommand{\locc}{\mathrm{LOCC}}
\newcommand{\sep}{\mathrm{SEP}}
\newcommand{\ppt}{\mathrm{PPT}}
\newcommand{\gocc}{\mathrm{GOCC}}
\newcommand{\het}{\mathrm{het}}
\renewcommand{\hom}{\mathrm{hom}}

\newcommand{\CN}{\mathrm{CN}}
\newcommand{\PIO}{\mathrm{PIO}}
\newcommand{\SIO}{\mathrm{SIO}}
\newcommand{\IO}{\mathrm{IO}}
\newcommand{\DIO}{\mathrm{DIO}}
\newcommand{\MIO}{\mathrm{MIO}}

\newcommand{\df}{\textbf{Definition.\ }}
\newcommand{\teo}{\textbf{Theorem.\ }}
\newcommand{\lm}{\textbf{Lemma.\ }}
\newcommand{\prp}{\textbf{Proposition.\ }}
\newcommand{\cll}{\textbf{Corollary.\ }}
\newcommand{\prf}{\emph{Proof.\ }}

\DeclareMathOperator{\Tr}{Tr}
\DeclareMathOperator{\rk}{rk}
\DeclareMathOperator{\cl}{cl}
\DeclareMathOperator{\co}{conv}
\DeclareMathOperator{\cone}{cone}
\DeclareMathOperator{\inter}{int}
\DeclareMathOperator{\Span}{span}
\DeclareMathAlphabet{\pazocal}{OMS}{zplm}{m}{n}
\DeclareMathOperator{\aff}{aff}
\DeclareMathOperator{\ext}{ext}
\DeclareMathOperator{\pr}{Pr}
\DeclareMathOperator{\supp}{supp}
\DeclareMathOperator{\spec}{sp}
\DeclareMathOperator{\sgn}{sgn}
\DeclareMathOperator{\tr}{tr}
\DeclareMathOperator{\Id}{Id}
\DeclareMathOperator{\relint}{relint}
\DeclareMathOperator{\arcsinh}{arcsinh}
\DeclareMathOperator{\erf}{erf}
\DeclareMathOperator{\dom}{dom}
\DeclareMathOperator{\diag}{diag}

\newcommand{\HH}{\pazocal{H}}
\newcommand{\T}{\pazocal{T}}
\newcommand{\D}{\pazocal{D}}
\newcommand{\K}{\pazocal{K}}
\newcommand{\B}{\pazocal{B}}
\newcommand{\Sch}{\pazocal{S}}
\newcommand{\Tsa}{\pazocal{T}_{\mathrm{sa}}}
\newcommand{\Ksa}{\pazocal{K}_{\mathrm{sa}}}
\newcommand{\Bsa}{\pazocal{B}_{\mathrm{sa}}}
\newcommand{\Schsa}{\pazocal{S}_{\mathrm{sa}}}
\newcommand{\NN}{\mathcal{N}}
\newcommand{\MM}{\mathcal{M}}

\newcommand{\lsmatrix}{\left(\begin{smallmatrix}}
\newcommand{\rsmatrix}{\end{smallmatrix}\right)}
\newcommand\smatrix[1]{{%
  \scriptsize\arraycolsep=0.4\arraycolsep\ensuremath{\begin{pmatrix}#1\end{pmatrix}}}}

\stackMath
\newcommand\xxrightarrow[2][]{\mathrel{%
  \setbox2=\hbox{\stackon{\scriptstyle#1}{\scriptstyle#2}}%
  \stackunder[5pt]{%
    \xrightarrow{\makebox[\dimexpr\wd2\relax]{$\scriptstyle#2$}}%
  }{%
   \scriptstyle#1\,%
  }%
}}

\newcommand{\tends}[2]{\xxrightarrow[\! #2 \!]{\mathrm{#1}}}
\newcommand{\tendsn}[1]{\xxrightarrow[\! n\rightarrow \infty\!]{#1}}
\newcommand{\tendsk}[1]{\xxrightarrow[\! k\rightarrow \infty\!]{#1}}
\newcommand{\ctends}[3]{\xxrightarrow[\raisebox{#3}{$\scriptstyle #2$}]{\raisebox{-0.7pt}{$\scriptstyle #1$}}}

\def\stacktype{L}
\stackMath
\def\dyhat{-0.15ex}
\newcommand\myhat[1]{\ThisStyle{%
              \stackon[\dyhat]{\SavedStyle#1}
                              {\SavedStyle\widehat{\phantom{#1}}}}}

\makeatletter
\newcommand*\rel@kern[1]{\kern#1\dimexpr\macc@kerna}
\newcommand*\widebar[1]{%
  \begingroup
  \def\mathaccent##1##2{%
    \rel@kern{0.8}%
    \overline{\rel@kern{-0.8}\macc@nucleus\rel@kern{0.2}}%
    \rel@kern{-0.2}%
  }%
  \macc@depth\@ne
  \let\math@bgroup\@empty \let\math@egroup\macc@set@skewchar
  \mathsurround\z@ \frozen@everymath{\mathgroup\macc@group\relax}%
  \macc@set@skewchar\relax
  \let\mathaccentV\macc@nested@a
  \macc@nested@a\relax111{#1}%
  \endgroup
}

\newcommand{\fakepart}[1]{
 \par\refstepcounter{part}
  \sectionmark{#1}
}

\counterwithin*{equation}{part}
\counterwithin*{thm}{part}
\counterwithin*{figure}{part}

\newenvironment{reusefigure}[2][htbp]
  {\addtocounter{figure}{-1}
   \renewcommand{\theHfigure}{dupe-fig}
   \renewcommand{\thefigure}{\ref{#2}}
   \renewcommand{\addcontentsline}[3]{}
   \begin{figure}[#1]}
  {\end{figure}}

\tikzset{meter/.append style={draw, inner sep=10, rectangle, font=\vphantom{A}, minimum width=30, line width=.8, path picture={\draw[black] ([shift={(.1,.3)}]path picture bounding box.south west) to[bend left=50] ([shift={(-.1,.3)}]path picture bounding box.south east);\draw[black,-latex] ([shift={(0,.1)}]path picture bounding box.south) -- ([shift={(.3,-.1)}]path picture bounding box.north);}}}
\tikzset{roundnode/.append style={circle, draw=black, fill=gray!20, thick, minimum size=10mm}}
\tikzset{squarenode/.style={rectangle, draw=black, fill=none, thick, minimum size=10mm}}

\definecolor{Blues5seq1}{RGB}{239,243,255}
\definecolor{Blues5seq2}{RGB}{189,215,231}
\definecolor{Blues5seq3}{RGB}{107,174,214}
\definecolor{Blues5seq4}{RGB}{49,130,189}
\definecolor{Blues5seq5}{RGB}{8,81,156}

\definecolor{Greens5seq1}{RGB}{237,248,233}
\definecolor{Greens5seq2}{RGB}{186,228,179}
\definecolor{Greens5seq3}{RGB}{116,196,118}
\definecolor{Greens5seq4}{RGB}{49,163,84}
\definecolor{Greens5seq5}{RGB}{0,109,44}

\definecolor{Reds5seq1}{RGB}{254,229,217}
\definecolor{Reds5seq2}{RGB}{252,174,145}
\definecolor{Reds5seq3}{RGB}{251,106,74}
\definecolor{Reds5seq4}{RGB}{222,45,38}
\definecolor{Reds5seq5}{RGB}{165,15,21}

\usepackage{pgfplots}
\newtheorem{theorem}{Theorem}
\newtheorem{conjecture}{Conjecture}
\newtheorem{definition}{Definition}
\newtheorem{example}{Example}

\DeclareMathOperator{\arccosh}{arccosh}

%\newtheorem{lemma}{Lemma}
\pgfplotsset{width=10cm,compat=1.9}
\usetikzlibrary{decorations.markings}


%\usepackage{lineno}
%\linenumbers


%\usepackage{caption}
%\captionsetup[figure]{%
%   justification=justified,
%  singlelinecheck=false 
%}

\begin{document}


%\title{Improved lower bound on two-way quantum capacities of Gaussian channels}
%\title{Improved lower bound on two-way quantum and secret-key capacity of Gaussian channels}
\title{Maximum tolerable excess noise in CV-QKD and improved lower bound on two-way capacities}

\author{Francesco Anna Mele}
\email{francesco.mele@sns.it}
\affiliation{NEST, Scuola Normale Superiore and Istituto Nanoscienze, Consiglio Nazionale delle Ricerche, Piazza dei Cavalieri 7, IT-56126 Pisa, Italy}

\author{Ludovico Lami}
\email{ludovico.lami@gmail.com}
\affiliation{QuSoft, Science Park 123, 1098 XG Amsterdam, the Netherlands}
\affiliation{Korteweg-de Vries Institute for Mathematics, University of Amsterdam, Science Park 105-107, 1098 XG Amsterdam, the Netherlands}
\affiliation{Institute for Theoretical Physics, University of Amsterdam, Science Park 904, 1098 XH Amsterdam, the Netherlands}
\affiliation{Institut f\"{u}r Theoretische Physik und IQST, Universit\"{a}t Ulm, Albert-Einstein-Allee 11, D-89069 Ulm, Germany}

\author{Vittorio Giovannetti}
\email{vittorio.giovannetti@sns.it}
\affiliation{NEST, Scuola Normale Superiore and Istituto Nanoscienze, Consiglio Nazionale delle Ricerche, Piazza dei Cavalieri 7, IT-56126 Pisa, Italy}
 

\begin{comment}
	\begin{abstract}
		%Building a large-scale quantum network requires to perform quantum communication tasks --- e.g. distributing secret-keys, sending private information, sending qubits, and distributing entanglement --- over long distances, which is unfortunately prevented in any point-to-point optical link (e.g.~optical fibres or free-space links) because of thermal attenuation noise.
		In order to overcome this limitation, it is possible to exploit quantum repeaters along the optical link. However, since quantum repeaters will likely be expensive, it is important to calculate the maximum achievable communication rates --- called the capacities --- in the absence of these devices. In the best-studied model of quantum optical communication, the thermal attenuation noise which affects signals travelling across an optical link is schematised as the thermal attenuator channel $\mathcal{E}_{\lambda,\nu}$. Mathematically, $\mathcal{E}_{\lambda,\nu}$ acts by mixing the input signal in a beam splitter of transmissivity $\lambda\in[0,1]$ with an external environment initialised in a thermal state of mean photon number $\nu$.
		In this paper, we provide an improved lower bound on the secret-key capacity, two-way private capacity, two-way quantum capacity, and two-way entanglement distribution capacity of the thermal attenuator. Here, ``two-way" means that the sender and the receiver can exploit a free, noiseless, public, classical communication line during their quantum communication strategies. In addition, we also provide an improved lower bound on the corresponding energy-constrained (EC) two-way capacities. Specifically, we prove our bounds by devising an entanglement distribution protocol, which consists in performing projective measurements and a suitable number of steps of a recurrence entanglement distillation protocol prior to apply an improved version of the hashing protocol. Moreover, we characterise the zero-capacity parameters region: all the (EC and unconstrained) two-way capacities of $\mathcal{E}_{\lambda,\nu}$ vanish if and only if the transmissivity $\lambda$ and the environmental mean photon number $\nu$ satisfy $\lambda<\frac{\nu}{\nu+1}$, independently of the energy constraint.
		
		In the best-studied model of quantum optical communication, the thermal noise which affects signals travelling across an optical link (e.g.~optical fibres or free-space links) is schematised as the thermal attenuator channel $\mathcal{E}_{\lambda,\nu}$. Mathematically, $\mathcal{E}_{\lambda,\nu}$ acts by mixing the input signal in a beam splitter of transmissivity $\lambda\in[0,1]$ with an external environment initialised in a thermal state of mean photon number $\nu$. In order to build a large-scale quantum network, it is crucial to calculate the maximum achievable communication rates --- called the capacities --- in performing quantum communication tasks across the thermal attenuator, e.g. distributing secret-keys, sending private information, sending qubits, and distributing entanglement. In this paper, we provide an improved lower bound on the secret-key capacity, two-way private capacity, two-way quantum capacity, and two-way entanglement distribution capacity of the thermal attenuator. Here, ``two-way" means that the sender and the receiver can exploit a free, noiseless, public, classical communication line during their quantum communication strategies. In addition, we also provide an improved lower bound on the corresponding energy-constrained (EC) two-way capacities. Specifically, we prove our bounds by devising an entanglement distribution protocol, which consists in performing projective measurements and a suitable number of steps of a recurrence entanglement distillation protocol prior to apply an improved version of the hashing protocol. Moreover, we characterise the zero-capacity parameters region: all the (EC ) two-way capacities of $\mathcal{E}_{\lambda,\nu}$ vanish if and only if the transmissivity $\lambda$ and the environmental mean photon number $\nu$ satisfy $\lambda<\frac{\nu}{\nu+1}$, independently of the energy constraint.
	\end{abstract}
\end{comment}



%The two-way capacities of quantum channels determine the ultimate limits of quantum communication which are achievable by two distant parties who can freely perform local operations and public classical communication. These quantities are important since provide benchmarks for physical implementations of a quantum repeater, which is a device that could be inserted between the two parties to increase the communication rates. In addition, since quantum repeaters will likely be expensive, it is important to calculate the maximum achievable communication rates in the absence of these devices. In this paper, we find a new lower bound on the energy-constrained and unconstrained two-way quantum and secret-key capacities of the thermal attenuator, thermal amplifier, and additive Gaussian noise, collectively dubbed  ``phase-insensitive Bosonic Gaussian Channels'' (piBGCs), which are realistic models for the noise that affects optical signals in optical fibres or free-space links. Specifically, we prove our bound by devising an entanglement distribution protocol, which consists in performing projective measurements and a suitable number of steps of a recurrence entanglement distillation protocol prior to apply an improved version of the hashing protocol, and by calculating its rate of distributed ebits. We do this calculation by introducing and exploiting a simple Kraus representation of piBGCs, derived by making use of the associated Lindblad master equations. Moreover, we find the parameters regions where the energy-constrained two-way capacities of piBGCs are strictly positive and, additionally, we show that they are equal to the parameters regions in which the piBGCs are not entanglement breaking.
%\ludo{The two-way capacities of quantum channels determine the ultimate entanglement distribution rates achievable by two distant parties that are connected by a noisy transmission line, in absence of quantum repeaters. Since repeaters will likely be expensive to build and maintain, a central open problem of quantum communication is to understand what performances are achievable without them. In this paper, we find a new lower bound on the energy-constrained and unconstrained two-way quantum and secret-key capacities of all phase-insensitive bosonic Gaussian channels [all of them, right?], namely thermal attenuator, thermal amplifier, and additive Gaussian noise, which are realistic models for the noise affecting optical fibres or free-space links. Ours is the first nonzero lower bound in the parameter range where the reverse coherent information becomes negative, and it shows explicitly that entanglement distribution is always possible when the channel is not entanglement breaking. In addition, our construction is fully explicit, i.e.\ we devise and optimise a concrete entanglement distribution and distillation protocol that works by combining recurrence and hashing protocols. A key technical tool needed for the analysis is a Kraus representation of phase-insensitive Gaussian channels derived by making use of the associated Lindblad master equations.}
\maketitle
\textbf{The two-way capacities of quantum channels determine the ultimate entanglement and secret-key distribution rates achievable by two distant parties that are connected by a noisy transmission line, in absence of quantum repeaters. Since repeaters will likely be expensive to build and maintain, a central open problem of quantum communication is to understand what performances are achievable without them. In this paper, we find a new lower bound on the energy-constrained and unconstrained two-way quantum and secret-key capacities of all phase-insensitive bosonic Gaussian channels, namely thermal attenuator, thermal amplifier, and additive Gaussian noise, which are realistic models for the noise affecting optical fibres or free-space links. Ours is the first nonzero lower bound on the two-way quantum capacity in the parameter range where the (reverse) coherent information becomes negative, and it shows explicitly that entanglement distribution is always possible when the channel is not entanglement breaking. This completely solves a crucial open problem of the field, namely, establishing the maximum excess noise which is tolerable in continuous-variable quantum key distribution. In addition, our construction is fully explicit, i.e.~we devise and optimise a concrete entanglement distribution and distillation protocol that works by combining recurrence and hashing protocols}.
 

%\section{Introduction}
Quantum information~\cite{NC}, and in particular quantum communication, will likely be a key element of next-generation technologies. Possible applications of a global quantum internet~\cite{quantum_internet_Wehner,Pirandola20} include unconditionally secure communication~\cite{bennett1984quantum}, entanglement and qubit distribution, quantum sensing improvements~\cite{Sidhu}, distributed and blind quantum computing~\cite{Distributed_QC,secure_access_qinternet}, and new experiments in fundamental physics~\cite{Sidhu}. Since long optical links are very sensitive to noise, quantum repeaters~\cite{repeaters, Munro2015} may be required~\cite{Die-Hard-2-PRL,Die-Hard-2-PRA} in order to establish quantum communication over long distances. However, today's physical implementations of quantum repeaters are still at the level of proof of principle experiments. In addition, quantum repeaters will likely be expensive and thus it will be desirable to use them sparingly. Consequently, an important problem is to devise reliable quantum communication protocols that do not use quantum repeaters.
%determine the ultimate limits of point-to-point quantum communication without quantum repeaters in order to provide benchmarks for quantum repeaters and to understand when these devises are actually needed.

%\ludo{[Nature Photonics \href{https://www.nature.com/nphoton/content}{guidelines} recommend about 50 references for the main text. Right now we cite 67 papers. This introduction seems a good place to edit out some of them.]} \tcb{[Francesco: Ora abbiamo 52 referenze di cui una e' un footnote. Inoltre, le ultime cinque referenze (da [48] a [52]) sono solo nel SM.}

%\ludo{[Watch out: names in reference titles should be enclosed in $\{\}$ in order for capitalisation to be preserved by BibTeX. I corrected `Wheeler' in the above paragraph. Also, we should uniform the reference list: some first names and journal names are abbreviated, others aren't.]}

The above problem can be formalised in the framework of quantum Shannon theory~\cite{MARK,Sumeet_book}. %, which studies how to efficiently transmit information from a sender (Alice) to a receiver (Bob) by exploiting the properties of quantum mechanics. Information is encoded in the quantum state of signals sent by Alice to Bob across a possibly noisy communication line, mathematically modelled as a quantum channel.
In that setting, the ultimate limits of point-to-point quantum communication across a quantum channel $\Phi$ are given by its capacities, %which 
quantifying the maximum amount of information that can be transmitted per use of the channel in the asymptotic limit of many uses. A plethora of capacities have been defined, depending on the type of information to be transmitted --- e.g.\ qubits %and secret-key 
or private bits --- and the %type of extra resource required to implement the communication protocol. 
available extra resources. In this work we focus on the so-called two-way capacities, modelling the technologically realistic scenario in which the sender (Alice) and the receiver (Bob) can use two-way (public) classical communication to assist the transmission. 
%In this work we focus on the technologically realistic scenario in which Alice and Bob can use two-way (public) classical communication to assist the transmission.
%--- e.g.~two-way noiseless classical communication between Alice and Bob (i.e.~they can freely and publicly send bits to each other) or pre-shared entanglement. In this work we focus on the so-called two-way capacities, which corresponds to the extra resource of two-way noiseless classical communication.
We study two-way capacities for transmission of qubits and secret-key bits, respectively called the two-way quantum capacity $Q_2(\Phi)$ and the secret-key capacity $K(\Phi)$. Note that $Q_2(\Phi)$ and $K(\Phi)$ are also the proper figure of merits for transmission of ebits %entanglement
and private bits, respectively, thanks to quantum teleportation~\cite{teleportation} and to one-time pad, respectively. In order to take into account that in practice Alice can exploit only a certain finite amount of energy to produce her signals, one usually considers the so-called energy-constrained (EC) two-way capacities~\cite{Davis2018}, denoted by $Q_2(\Phi, N_s)$ and $K(\Phi, N_s)$. Here, $N_s\geq 0$ is the maximum average photon number per input signals to the channel $\Phi$. Since an ebit can generate a secret-key bit~\cite{Ekert91}, it holds that $Q_2(\Phi,N_s)\le K(\Phi,N_s)$.

%\ludo{[Should we draw at all a distinction between $Q_2$ and $D_2$, and $P_2$ and $K$, given that $Q_2=D_2$ (teleportation) and $P_2=K$ (one-time pad)? I would perhaps prefer to simplify the notation and keep only two distinct objects, $Q_2$ and $P_2$ (or we can also call it $K$, no problem). We could then explain in words that $Q_2$ is also the entanglement distribution capacity, and $P_2$ the secret key distribution capacity.]}

Future long-distance quantum links will most likely rely on optical technologies~\cite{Pirandola20}. It is thus of key importance to investigate the transmission of continuous variable (CV) systems, modelling finite ensembles of electromagnetic modes, across CV quantum channels~\cite{BUCCO}. Realistic models to describe the input-output relation of a single-mode state which is sent across an optical fibre or a free-space link are the thermal attenuator, thermal amplifier, and additive Gaussian noise~\cite{BUCCO}, collectively called ``phase-insensitive bosonic Gaussian channels'' (piBGCs). Both the thermal attenuator $\mathcal{E}_{\lambda,\nu}$ and the thermal amplifier $\Phi_{g,\nu}$ depend on two parameters, $(\lambda,\nu)$ in the former case and $(g,\nu)$ in the latter. The first parameters, $\lambda\in [0,1]$ and $g\in [1,\infty)$, represent the transmissivity and the gain, respectively. The second one, $\nu\in [0,\infty)$, quantifies the added thermal noise. %Although this is negligible (i.e.~$\nu\simeq 0$) at room temperature for telecom wavelengths, it cannot be ignored for microwave wavelengths, as well as for infrared lasers and radio waves~\cite{Rosati2018}. %The action of these channels on an input state $\rho$ can be written as
By letting $S, E$ be single-mode systems with $a, b$ denoting their annihilation operators, the action of piBGCs on an input state $\rho$ is
\bb
    \mathcal{E}_{\lambda,\nu}(\rho)&\coloneqq\Tr_E\left[U_\lambda^{SE} \big(\rho^S \otimes\tau_\nu^E \big) {U_\lambda^{SE}}^\dagger\right]\,,\\
    \Phi_{g,\nu}(\rho)&\coloneqq\Tr_E\left[U_g^{SE} \big(\rho^S\otimes\tau_\nu^E\big) {U_g^{SE}}^\dagger\right]\,,\\
    \Lambda_\xi(\rho)&\coloneqq\frac{1}{\pi\xi}\int_{\mathbb C} \mathrm{d}^2 {z}\, e^{-\frac{|z|^2}{\xi}}  D(z)\,\rho\,  D(z)^\dagger\,,
\ee
where $\tau^E_\nu$ is the thermal state on $E$ with mean photon number $\nu$, $U_\lambda^{SE} \coloneqq e^{\arccos(\sqrt{\lambda})(a^\dagger b-a\, b^\dagger)}$ is the beam splitter unitary of transmissivity $\lambda$, $U_g^{SE} \coloneqq e^{\arccosh(\sqrt{g}) (a^\dagger b^\dagger-a\, b)}$ is the two-mode squeezing unitary of gain $g$, and $D(z) \coloneqq e^{z a^\dagger-z^\ast a}$ is the displacement operator. In the following, the piBGCs are understood to map Alice's input systems, denoted as $A'$, to Bob's output systems, denoted as $B$. Moreover, Alice's ancillary systems are denoted as $A$.
 


The two-way capacities of the thermal attenuator and thermal amplifier have been determined for all $\lambda$ and $g$ in the case in which $\nu=0$ and there is no energy constraint~\cite{PLOB}. Except for this very special case, the two-way capacities of piBGCs are not known, although upper~\cite{PLOB,Davis2018,Goodenough16,TGW,holwer,MMMM,squashed_channel} and lower bounds~\cite{holwer,Pirandola2009,Noh2020,Ottaviani_new_lower,Pirandola18,Wang_Q2_amplifier} have been established. For nonzero values of $\nu$, the gap between such bounds is however relatively large for many values of $\lambda$ and $g$. In particular, for a certain range of values of $\lambda$ and $g$ all lower bounds available prior to our work vanish%\ludo{[Ludo: it seems to me from the discussion below Eq.~(116) in~\cite{Pirandola18} that they still have a range where they can't prove faithfulness, even for $K$. Is that correct?]} \tcb{[Francesco: Yes, this is correct]}
, while the upper bounds do not. The same is true for the additive Gaussian noise channel. This entails that the precise noise threshold above which quantum communication or key distribution become impossible had not been determined in the prior literature. Indeed, it has been recently identified as a crucial open problem of the field~\cite[Section~7]{Pirandola18}.


%We show our lower bound by introducing an entanglement distribution protocol, which combines and optimises known entanglement distillation protocols~\cite{Horodecki-review,Bennett-distillation,Bennett-error-correction,Bennett-distillation-mixed,reviewEDP_dur,p1orp2,Improvement-Hashing}, and evaluating its rate.
%To achieve this, we generalise the `master equation trick', introduced in~\cite{Die-Hard-2-PRA}, in order to easily deal with calculations involving thermal attenuator in Fock basis,
%\ludo{On the technical side, our analysis is made possible by a generalisation of the `master equation trick' of~\cite{Die-Hard-2-PRA}}  to the thermal amplifier and additive Gaussian noise. 
%\ludo{[Ho (ri)trovato questo articolo https://arxiv.org/abs/1012.4266 dove calcolano una Kraus representation di molti canali (quantum-limited, per\`o). Non \`{e} che combinando due quantum-limited rappresentati come li rappresentano loro si ottiene la rappresentazione che proponiamo noi?]}
%In addition, we determine the parameter region where the EC two-way capacities of piBGCs vanish. This allows us to solve the open problem stated in~\cite{Pirandola18,Pirandola20}, determining the maximum excess noise which is tolerable by QKD protocols~\cite{Pirandola18,Pirandola20}. 




%\tcb{Additionally, an important aspect of our method is that it yields a \emph{faithful} lower bound, i.e.~which vanishes if and only if the corresponding capacity vanishes, unlike all previously available protocols~\cite{Pirandola2009,Noh2020,Ottaviani_new_lower,Pirandola18,Wang_Q2_amplifier}.}

%\ludo{[I believe that we should stress heavily that our lower bound is not only quantitatively better in many cases, but also conceptually superior because it is \emph{faithful}.]}

\section*{Results}%\ludo{[Sembrerebbe appropriato dividere questa sezione in due parti, per evidenziare i nostri due main results: (i) il protocollo ottimizzato con tutti i plot numerici; (ii) il teorema con la dimostrazione della faithfulness della $Q_2$. Che dici?]}
In this paper, we find a new lower bound on all two-way capacities of piBGCs, which constitutes a significant improvement with respect the state-of-the-art lower bounds~\cite{Ottaviani_new_lower,Pirandola2009,Pirandola18,Wang_Q2_amplifier,Noh2020} in many parameter regions. In particular, we completely solve the above open problem, quantitatively establishing that piBGCs can be used to distribute pure entanglement --- and hence also secret key --- whenever they are not entanglement breaking. 
%in the following parameters regions: low transmissivity for the thermal attenuator with $\nu>0$, high amplification for the thermal amplifier with $\nu>0$, and low temperature for additive Gaussian noise.
Notably, our results are fully constructive: to prove them, we introduce and analyse a concrete entanglement distribution scheme that combines and optimises known protocols~\cite{Horodecki-review,Bennett-distillation,Bennett-error-correction,Bennett-distillation-mixed,reviewEDP_dur,p1orp2,Improvement-Hashing}.
%\tcb{In section~\ref{subsec_a} we devise an entanglement distribution protocol whose ebits rate constitutes our new lower bound on the two-way capacities of the piBGCs. In section~\ref{subsec_b} we find that the parameter region where the two-way capacities vanish is precisely the entanglement-breaking region.}

\subsection{Improved lower bound on the two-way capacities}~\label{subsec_a}
Before presenting our improved lower bound on the two-way capacities of piBGCs, let us see the main ideas underpinning it. 
The best known lower bounds on the two-way capacities of any piBGC $\Phi$, prior to our work, are the (reverse) coherent information lower bounds, which are reported in Eq.~\eqref{lowQ2_main_att}--\eqref{lowQ2_main_noise} in the Methods. They can be established by applying the inequality 
\bb\label{ineq_Q2_Ed}
Q_2(\Phi,N_s)\ge E_d\left(\Id_{A}\otimes\Phi(\rho_{AA'}) \right)
\ee
for a suitable choice of $\rho_{AA'}$, and then by resorting to the hashing inequality~\cite{devetak2005} in order to lower bound $E_d$ with the (reverse) coherent information of $\Id_{A}\otimes\Phi(\rho_{AA'})$. Here, $E_d(\rho_{AB})$ denotes the distillable entanglement of a bipartite state $\rho_{AB}$~\cite{reviewEDP_dur}. Importantly, if $\rho_{AB}$ is such that its (reverse) coherent information is non-positive, %the hashing inequality is trivial (i.e.~the right-hand side of~\eqref{hashing_ineq_main} is negative),\tcb{gives a non-positive lower bound on $E_d(\rho_{AB})$, }
one may try to obtain a %non-trivial
positive lower bound on $E_d(\rho_{AB})$ by adopting a sufficiently large number of iterations of an entanglement distillation recurrence protocol~\cite{Bennett-error-correction,Bennett-distillation-mixed,reviewEDP_dur} on $\rho_{AB}$ before applying the hashing inequality. This idea was exploited in~\cite{Bennett-error-correction,Bennett-distillation-mixed} to find a positive %non-trivial 
lower bound on the distillable entanglement of qubit Werner states having non-positive coherent information. In order to find a new lower bound on the two-way capacities of any piBGC $\Phi$ which outperform the best known lower bounds at least in the parameter region where the latter vanish, our idea is to apply a recurrence protocol on a suitably chosen $\Id_{A}\otimes\Phi(\rho_{AA'})$ before initiating the hashing protocol. %where $\rho_{AA'}$ is suitably chosen.

%Let $\Phi$ be the piBGC which connects Alice to Bob. 
Let us introduce a protocol to distribute ebits across a piBGC $\Phi$. Our protocol is composed of $6$ steps named S1--S6, and it depends on three parameters: $M\in\N^+$, $c\in(0,1)$, $k\in\N$. %Our lower bound on the two-way capacities of $\Phi$ is the rate of distributed ebits achievable by our protocol optimised over $M$, $c$, and $k$. %\ludo{[Should we change the names of these parameters into something more friendly, such as $M,c,k$, or $M,c,k$?]}


\bigskip
\noindent \textbf{\emph{Entanglement distribution protocol}}:  
\vspace{0ex}
\begin{enumerate}[leftmargin=3.76ex]
\item[\textbf{S1}:] Alice prepares many copies of the state 
\bb\label{state_our_protocol}
    \hspace{3.5ex} \ket{\Psi_{M,c}}_{\!AA'}\coloneqq c\ket{0}_{\!A}\!\otimes\!\ket{0}_{\!A'}+\sqrt{1\!-\!c^2}\ket{M}_{\!A}\!\otimes\!\ket{M}_{\!A'}\,,
\ee
and sends the subsystems $A'$ to Bob through the channel $\Phi$. Here,  $\ket{0}$ and $\ket{M}$ denote the vacuum and the $M$th Fock state, respectively. Now Alice and Bob share many pairs in the state $\Id_{A}\otimes\Phi(\ketbra{\Psi_{M,c}})$. If there is an energy constraint $N_s$, %the mean photon number of $\ket{\Psi_{M,c}}$ has to be less or equal to $N_s$, i.e.\
the parameters $c$ and $M$ have to satisfy $(1-c^2)M\le N_s$.

\item[\textbf{S2}:] Bob performs the local POVM $\{\Pi_M, \mathbb{1}-\Pi_M\}$ on each pair, where $\Pi_M\coloneqq \ketbra{0}+\ketbra{M}$. If Bob finds the outcome associated with $\Pi_M$, then the pair is kept, otherwise it is discarded. Hence, by re-mapping $\ket{M}\to \ket{1}$, each pair is now in an effective two-qubit state.

%Alice and Bob have transformed each pair in a two-qubit state. %the state into a two-qubit state. which lives in $\HH_2\otimes\HH_2$, where $\HH_2$ is the Hilbert space spanned by $\{\ket{0},\ket{1}\}$. We have thus reduced the problem in distilling entanglement from a two-qubit state.

%\item[\textbf{S3}:] Both Alice and Bob apply the unitary $\sum_{i\ne\{0, M\}}^\infty\ketbra{i}+\ketbraa{1}{M}+\ketbraa{M}{1}$ on each pair. This step allows them to transform the state into a two-qubit state which lives in $\HH_2\otimes\HH_2$, where $\HH_2$ is the Hilbert space spanned by $\{\ket{0},\ket{1}\}$. Hence, we have reduced the problem in distilling entanglement from a two-qubit state. \ludo{[I would be less formal here and simply say that we re-map $\ket{M}\to \ket{1}$, with the understanding that the right-hand side stands for a qubit state. Like this it's a bit confusing tbh.]}

%\item[\textbf{S3}:] For each of the remaining pairs, Alice and Bob apply a Pauli-based twirling map \bb\hspace{3.5ex} X_{AB} \to \frac{1}{4}\sum_{\mu=0}^3 P_\mu \otimes P_\mu\, X_{AB}\, P_\mu \otimes P_\mu\,,\ee with $\{P_\mu\}_{\mu=0,1,2,3}$ being the qubit Pauli operators.  
%This step allows them to obtain a state which is diagonal in the Bell basis $\{\ket{\psi_{ij}}\}_{i,j\in\{0,1\}}$ .  


\item[\textbf{S3}:] Alice and Bob apply the Pauli-based twirling, reported in~\eqref{def_twirling_map} in the Methods, in order to transform each of the remaining pairs in a Bell-diagonal state.


\item[\textbf{S4}:] Alice and Bob run $k$ times the following sub-routine, which is a recurrence protocol dubbed \emph{P1-or-P2}~\cite{p1orp2}.  
 
\medskip
\noindent \textbf{\emph{P1-or-P2 sub-routine}}:  
     \begin{itemize}
         \item \textbf{Step 4.1}: At this point of the protocol, the two-qubit state $\rho_{AB}$ of each pair is of the form
         \bb\label{bell_diagonal_state_step4}
            \qquad\quad  \rho_{AB}=\sum_{i,j=0}^{1}p_{ij}\,\mathbb{1}_A\otimes  X^j Z^i \ketbra{\psi_{00}}(\mathbb{1}_A\otimes   X^j Z^i)^\dagger\,,
         \ee
        where $\ket{\psi_{00}}$ denotes the ebit state (see~\eqref{Bell_states_main} of the Methods), $X,Z$ denote the well-known Pauli operators, and $\{p_{ij}\}_{i,j\in\{0,1\}}$ is a probability distribution.
        
        Alice and Bob collect all pairs in groups of two.  
        For a given group, call $A_1B_1$ and $A_2B_2$ the four qubits involved.
        %Let $\rho$ be the two-qubit state of each of the remaining pairs at this point of the protocol. Alice and Bob collect all the pairs in groups of two pairs: $A_1B_1$ and $A_2B_2$. For all $i,j\in\{0,1\}$ let us introduce the Bell state \bb\ket{\psi_{ij}}\coloneqq \frac{1}{\sqrt{2}}\sum_{m=0}^1 (-1)^{im}\ket{m}\otimes\ket{m\oplus j}\,.\ee 
        %If $\rho$ satisfies $\bra{\psi_{10}}\rho\ket{\psi_{10}}<\bra{\psi_{01}}\rho\ket{\psi_{01}}$, then both Alice and Bob apply the CNOT gate on $A_1A_2$ and $B_1B_2$, respectively, where $A_1,B_1$ are the control qubits and $A_2,B_2$ are the target qubits.
        If $p_{10}<p_{01}$, they apply the CNOT gate on $A_1A_2$ and $B_1B_2$, respectively, where $A_1,B_1$ are the control qubits and $A_2,B_2$ are the target qubits.
        Otherwise, they apply first the Hadamard gate on each qubit, then the CNOT gate as in the above case, and finally the Hadamard gate on $A_1$ and $B_1$.
        
        \item \textbf{Step 4.2}: For each of the above group of two pairs, Alice and Bob perform a projective measurement with respect to the computational basis on $A_2$ and $B_2$, respectively, and then they discard the pair $A_2B_2$. If the outcomes are different, they discard also the pair $A_1B_1$.
     \end{itemize}

The sub-routine tends to increase the value of $p_{00}$, bringing the state $\rho_{AB}$ closer to the ebit state $\ket{\psi_{00}}$. Step 4.1's $p_{10}<p_{01}$ condition means that the $X$ error in~\eqref{bell_diagonal_state_step4} is more prominent than the $Z$ error. The sub-routine allows one to reduce $X$ (resp.~$Z$) errors when such a condition is (resp.~is not) satisfied. This comes at the price that $Z$ (resp.~$X$) errors may be amplified. By selectively applying the two procedures alternately, both errors are corrected~\cite{p1orp2}.


 

%\ludo{[Allora, da lettore inesperto di questa sub-routine quale sono, forse posso immedesimarmi nel lettore medio. E quindi, non ho capito bene qual \`e lo scopo di questo step. Cosa stiamo cercando di ottenere? Come mai questi passaggi specifici? Sarebbe bene cercare di dare un po' di intuizione...]}
%Step 4 allows Alice and Bob to increase the fidelity between the two-qubit Bell-diagonal state $\rho$ of the remaining pairs and the Bell state $\ket{\psi_{00}}$ \ludo{[Usiamo il termine `ebit' per il target della distillazione]}. This is useful since if such a fidelity is sufficiently close to one \ludo{[we can alo say how much: the threshold for hashing is approximately $\braket{\psi_00|\rho|\psi_00}\geq 0.811$]}, then the coherent information of $\rho$ is positive and hence the hashing inequality is not trivial. \ludo{[Qui sei stato molto preciso, tagliamo un po' anche a costo di essere pi\`{u} vaghi.]}

\item[\textbf{S5}:] Depending on the state $\rho$ of the remaining pairs (see the SM for details~\footnote{The supplemental material contains detailed derivations for interested readers.}), both Alice and Bob apply on each qubit one of the following unitaries: the qubit rotation around the $y$-axis of an angle $\pi/2$, the Hadamard gate, the identity.
 
\item[\textbf{S6}:] In the end, Alice and Bob run the improved version of the hashing protocol introduced in~\cite{Improvement-Hashing} in order to generate ebits.
\end{enumerate}

Our lower bound on the two-way quantum capacity $Q_2(\Phi)$ is the supremum over $M\in\N^+$, $c\in(0,1)$, and $k\in\N$ of the ebit rate of the above protocol, which is calculated in the SM~\cite{Note1}. Since the secret-key capacity $K$ is always larger than $Q_2$, our bound is also a lower bound on $K(\Phi)$. Our lower bound on the EC two-way capacities with energy constraint $N_s$ can be obtained by optimising with the additional condition $(1-c^2)M\le N_s$. 

%The main hurdle in calculating the ebits rate of the above protocol is that it requires to calculate the action of piBGCs on non-Gaussian operators, which can be very cumbersome in general since it involves complicated infinite series~\cite{Die-Hard-2-PRA}. To deal with this difficulty and to find the exact sum of these series, we generalise the ``master equation trick''\ludo{of~\cite{Die-Hard-2-PRA}} 
%, which was introduced in~\cite{Die-Hard-2-PRA} to derive a simple Kraus representation of the thermal attenuator, to the thermal amplifier and additive Gaussian noise. %Namely, 
%\ludo{By using this tool,} in Theorem~\ref{gen_master_eq_trick} of the SM we %provide %\ludo{construct} simple Kraus representations of piBGCs, which can allow one to obtain simple expression of the action of piBGCs on generic operators. This is done by solving the Lindblad master equation associated with the piBGCs. 

We plot our bounds on the two-way capacities of the thermal attenuator $\mathcal{E}_{\lambda,\nu}$ in Fig.~\ref{fig_main}(a), of the thermal amplifier $\Phi_{g,\nu}$ in Fig.~\ref{fig_main}(b), and of the additive Gaussian noise $\Lambda_\xi$ in Fig.~\ref{fig_main}(c). These plots demonstrate that our bound is strictly tighter than all known lower bounds on both the two-way quantum and secret-key capacity of piBGCs in a large parameter region.  


In the SM~\cite{Note1} we introduce an additional entanglement distribution protocol that combines and optimises the multi-rail protocol from~\cite{Winnel} and the qudit P1-or-P2 protocol from~\cite{p1orp2}. The ebit rate of this protocol constitutes an additional lower bound on the two-way capacities of the piBGCs, which we have found to be tighter than our previously discussed lower bound on $Q_2(\mathcal{E}_{\lambda,\nu})$ for $\nu\lesssim1$ (see the SM for details~\cite{Note1}).




\begin{figure}
\begin{tabular}{c}
\includegraphics[width=1.0\linewidth]{Figures/att_main.pdf} \\  
(a)\\
 \includegraphics[width=1.0\linewidth]{Figures/amp_main.pdf} \\ 
(b) \\
 \includegraphics[width=1.0\linewidth]{Figures/add_main.pdf} \\ 
(c)
\end{tabular}
\caption{\textbf{(a).}~Bounds on the two-way quantum capacity $Q_2$ and secret-key capacity $K$ of the thermal attenuator $\mathcal{E}_{\lambda,\nu}$ plotted with respect to $\lambda$ for $\nu=10$. The red line is our lower bound, the black line is the reverse coherent information lower bound in~\eqref{lowQ2_main_att}, the blue line is the lower bound on $K(\mathcal{E}_{\lambda,\nu})$ discovered by Ottaviani et al.~\cite{Ottaviani_new_lower} (in general this is not a lower bound on $Q_2$, since in general it only holds that $Q_2\le K$), and the green line is the PLOB upper bound in~\eqref{PLOB_Q2_main}. \textbf{ (b).}~Bounds on the two-way quantum capacity $Q_2$ and secret-key capacity $K$ of the thermal amplifier $\Phi_{g,\nu}$ plotted with respect to $g$ for $\nu=10$. The red line is our lower bound, the black line is the coherent information lower bound in~\eqref{lowQ2_main_amp}, the blue line is the WOGP lower bound on $K(\Phi_{g,\nu})$~\cite{Ottaviani_new_lower}, and the green line is the PLOB upper bound in~\eqref{PLOB_Q2_main}.  \textbf{ (c).}~Bounds on the two-way quantum capacity $Q_2$ and secret-key capacity $K$ of the additive Gaussian noise $\Lambda_\xi$ plotted with respect to $\xi$. The red line is our lower bound, the black line is the coherent information lower bound in~\eqref{lowQ2_main_noise}, and the green line is the PLOB upper bound in~\eqref{PLOB_Q2_main}.}
\label{fig_main}
\end{figure}

\begin{comment}
    \ludo{[Forse i paragrafi qui sotto starebbero meglio nella Discussion section, no? Alla fine \`e l\`i che dobbiamo spiegare come mai i nostri risultati sono innovativi/potenti.]}\tcb{[Okay, li sposto e li modifico di conseguenza.]}

First, except for the special case of the pure loss channel $\mathcal{E}_{\lambda,0}$ and pure amplifier channel $\Phi_{g,0}$, it was an open question whether the (reverse) coherent information lower bounds could equal the true two-way quantum capacities of piBGCs: our new lower bounds in Fig.~\ref{fig_main} provide a negative answer to this question.

Second, let $\lambda(\nu)$ (resp.~$g(\nu)$) be the upper (resp.~lower) endpoint for the $\lambda$-range (resp.~$g$-range) for which the coherent information lower bound on $Q_2(\mathcal{E}_{\lambda,\nu})$ (resp.~$Q_2(\Phi_{g,\nu})$) 
vanishes. %Second, fixed $\nu>0$, the transmissivity value $\lambda(\nu)\coloneqq 1-2^{-h(\nu)}$ is the upper endpoint for the $\lambda$-range for which the reverse coherent information lower bound on $Q_2(\mathcal{E}_{\lambda,\nu})$ %reported in~\eqref{lowQ2_main_att} vanishes. Analogously, the gain value $g(\nu)\coloneqq \big(1-2^{-h(\nu)}\big)^{-1}$ is the lower endpoint for the $g$-range for which the coherent information lower bound on $Q_2(\Phi_{g,\nu})$ %reported in~\eqref{lowQ2_main_amp} vanishes.
Importantly, our lower bounds on $Q_2(\mathcal{E}_{\lambda(\nu),\nu})$ and $Q_2(\Phi_{g(\nu),\nu})$ do not vanish for $\nu>0$ and, for large $\nu$, they approach $14\%$ of the tightest known \emph{upper} bound, due to PLOB~\cite{PLOB} and reported in~\eqref{PLOB_Q2_main} in the Methods. 
%In contrast, we numerically observe that our lower bounds on $Q_2(\mathcal{E}_{\lambda(\nu),\nu})$ and $Q_2(\Phi_{g(\nu),\nu})$ are strictly positive for $\nu>0$ and, for sufficiently large $\nu$, approach $14\%$ of the tightest known \emph{upper} bound, due to PLOB~\cite{PLOB} and reported in~\eqref{PLOB_Q2_main} in the Methods.
In this sense, our results constitute significant progress over the current state of affairs, although a relatively large gap between upper and lower bounds on two-way capacities remains.



Third, in the energy constraint scenario our lower bound on the EC two-way capacities of piBGCs can outperform the best known lower bounds (see the SM)~\cite{Noh2020}. %, even for the values of the energy constraint where the NPJ bounds are larger than the reverse coherent information and coherent information bounds .
\end{comment}


\subsection{Two-way capacities of piBGCs vanish if and only if piBGCs are entanglement breaking}\label{subsec_b}

%In contrast to the (reverse) coherent information lower bounds on the $Q_2$ of piBGCs, we numerically observe (see the SM) that our lower bounds are strictly positive if and only if the PLOB upper bounds are strictly positive. 

In contrast to the (reverse) coherent information lower bound on the $Q_2$ of any piBGC, we numerically observe (see the SM~\cite{Note1}) that our lower bound is strictly positive if and only if the tightest known \emph{upper} bound, due to PLOB~\cite{PLOB} and reported in~\eqref{PLOB_Q2_main} in the Methods, is strictly positive. 
The latter holds if and only the piBGC is not entanglement breaking~\cite{PLOB,Ent_breaking_Gaussian, Holevo-EB}. This would imply that the two-way capacities of a piBGC vanish if and only if the piBGC is entanglement breaking. We can mathematically prove this by showing the forthcoming Theorem~\ref{th1_main} (proved in the SM~\cite{Note1}).
\begin{thm}\label{th1_main}
    Let $N_s>0$, $\nu\ge0$, $\lambda\in[0,1]$, $g\ge 1$, and $\xi\ge0$. The energy-constrained two-way quantum capacity of the thermal attenuator $\mathcal{E}_{\lambda,\nu}$, thermal amplifier $\Phi_{g,\nu}$, and additive Gaussian noise $\Lambda_{\xi}$ vanish if and only if they are entanglement breaking. That is,
    \bb\label{param_reg}
    Q_2(\mathcal{E}_{\lambda,\nu},N_s)>0 &\text{ $\Leftrightarrow$ } \lambda\in\left(\frac{\nu}{\nu+1},1\right]\,,\\
    Q_2(\Phi_{g,\nu},N_s)>0 &\text{ $\Leftrightarrow$ } g\in\left[1,1+\frac{1}{\nu}\right)\,,\\
    Q_2(\Lambda_{\xi},N_s)>0 &\text{ $\Leftrightarrow$ } \xi\in[0,1)\,.
    \ee
    The same holds for the secret-key capacity $K(\cdot, N_s)$ as well as for the unconstrained capacities.% $Q_2(\cdot, N_s=\infty)$ and $K_2(\cdot, N_s=\infty)$.
    %In particular, the (unconstrained) two-way quantum capacity satisfies: \ludo{[Instead of repeating the equations, I would say something along the lines of `The same holds for the secret-key capacity $K_2(\cdot, N_s)$ as well as for  the unconstrained capacities $Q_2(\cdot, N_S=\infty)$ and $K_2(\cdot, N_s=\infty)$.']} \bbQ_2(\mathcal{E}_{\lambda,\nu})>0 &\text{ $\Leftrightarrow$ } \lambda\in\left(\frac{\nu}{\nu+1},1\right]\,;\\Q_2(\Phi_{g,\nu})>0 &\text{ $\Leftrightarrow$ } g\in\left[1,1+\frac{1}{\nu}\right)\,;\\Q_2(\Lambda_{\xi})>0 &\text{ $\Leftrightarrow$ } \xi\in[0,1)\,. \ee In addition, the same properties are satisfied by the secret-key capacity $K$. 
\end{thm}

\begin{comment}
\ludo{[Anche il paragrafo qui sotto forse starebbe meglio nella Discussion section?]}
Theorem~\ref{th1_main} solves the open question stated in~\cite[Section 7]{Pirandola18}%, deemed ``crucial'' by the authors of~\cite{Pirandola18}
: ``What is the maximum
excess noise that is tolerable in QKD? I.e., optimizing over all QKD protocols?"
Such maximum tolerable excess noise is defined as
\begin{equation}\label{exc}
    \epsilon(\lambda)\coloneqq \frac{1-\lambda}{\lambda}\max\{\nu\ge0\,:\,K(\mathcal{E}_{\lambda,\nu})>0\}\,,
\end{equation} 
and Theorem~\ref{th1_main} establishes that $\epsilon(\lambda)=1$ for all $\lambda\in(0,1)$. In~\cite{Pirandola18,Pirandola20} it was found that $\varepsilon(\lambda)\le1$ by applying the PLOB bound, and it was found a lower bound on $\varepsilon(\lambda)$, which was far from $1$, by applying the best known lower bound on $K(\mathcal{E}_{\lambda,\nu})$~\cite{Ottaviani_new_lower}. 
\end{comment}
\begin{proof}[Proof sketch] %of Theorem~\ref{th1_main}]
    Let $\Phi$ be a piBGC. The entanglement distribution protocol's Steps S1 and S2 imply that $Q_2(\Phi,N_s)\ge E_d\left(\rho_{AB} \right)$, where $\rho_{AB}$ denotes the two-qubit state shared by Alice and Bob after completing Step S2. By exploiting that any two-qubit state is distillable if and only if it is not PPT~\cite{2-qubit-distillation}, {in the SM~\cite{Note1}} we show that $\rho_{AB}$ is distillable if and only if $\Phi$ is not entanglement breaking. Consequently, if $\Phi$ is not entanglement breaking then $E_d\left(\rho_{AB}\right)>0$ and hence $K(\Phi,N_s)\ge Q_2(\Phi,N_s)>0$. Conversely, entanglement-breaking channels have vanishing two-way capacities.
\end{proof}

The crux of the above proof is that the state at the end of Step 2 is distillable if the channel is not entanglement breaking. Additionally, in the SM~\cite{Note1} we show that selecting the appropriate value for $c$ (as defined in~\eqref{state_our_protocol}) ensures that {also} the state at the end of Step 3 is %also 
distillable when the channel is not entanglement breaking, meaning that Step 3's Pauli-based twirling does not affect the distillability of the shared state.

In the SM~\cite{Note1} we provide also an alternative proof of Theorem~\ref{th1_main}, which relies on the facts that $Q_2(\Phi,N_s)\ge E_d\left(\Id_{A}\otimes\Phi(\ketbra{\Psi_{N_s}}) \right)$, where $\ket{\Psi_{N_s}}$ denotes the two-mode squeezed vacuum state (TMSV) with mean local photon number equal to $N_s$, and that any two-mode Gaussian state is distillable if and only if it is not PPT~\cite{Giedke01}.



%\begin{proof}[Proof sketch 2]First, we apply~\eqref{ineq_Q2_Ed} to deduce that for any piBGC $\Phi$ it holds that $Q_2(\Phi,N_s)\ge E_d\left(\Id_{A}\otimes\Phi(\ketbra{\Psi_{N_s}}) \right)$, where $\ket{\Psi_{N_s}}$ denotes the two-mode squeezed vacuum state (TMSV) with mean local photon number equal to $N_s$. Second, by using that any two-mode Gaussian states is separable if and only if it has positive partial transpose~\cite{Simon00, BUCCO}, we show that $\Id_{A}\otimes\Phi(\ketbra{\Psi_{N_s}})$ is entangled if and only if $\Phi$ is not entanglement breaking. Third, since any two-mode Gaussian entangled state is distillable~\cite{Giedke01}, we deduce that if $\Phi$ is not entanglement breaking then $E_d\left(\Id_{A}\otimes\Phi(\ketbra{\Psi_{N_s}})\right)>0$ and hence $K(\Phi,N_s)\ge Q_2(\Phi,N_s)>0$. On the other hand, entanglement-breaking channels have vanishing two-way capacities. \end{proof}
%\tcb{[Francesco: se vogliamo guadagnare spazio, penso che il seguente paragrafo si possa accorciare o addirittura omettere.] The proof of Theorem~\ref{th1_main} provides another entanglement distribution protocol which is mathematically guaranteed to yield a \emph{faithful} lower bound on the two-way capacities, i.e.~it vanishes if and only if the PLOB upper bound vanishes. Such a protocol is the following: Alice sends many halves of the TMSV $\ket{\Psi_{N_s}}$ to Bob across the piBGC $\Phi$, and then they run the protocol introduced by Giedke et al.~\cite{Giedke01} to distil $\Id_{A}\otimes\Phi(\ketbra{\Psi_{N_s}})$. The latter consists of two main steps: first Alice and Bob make local measurements on each pair in order to project on the span of the first $d$ Fock states where $d\in\N$ is sufficiently large, second they apply the entanglement distillation protocol introduced in~\cite{Horodecki1999}. Although the rate of such entanglement distribution protocol is faithful, we numerically observe that it is much lower than the rate of our previously discussed entanglement distribution protocol.}


%\subsection{Usare il master equation trick per dimostrare che la single-shot coherent information non è massimizzata da stati gaussiani}

%\subsection{Cenno sul die-hard quantum communication effect per il general amplifier}

\section*{Discussion}
%In this paper we provided a constructive method that improves on the best previously known lower bounds on the two-way quantum and secret-key capacities of all phase-insensitive Gaussian channels. These are the thermal attenuator $\mathcal{E}_{\lambda,\nu}$, the thermal amplifier $\Phi_{g,\nu}$, and the additive Gaussian noise $\Lambda_\xi$. We also proved that the parameter regions where the (EC) two-way capacities of piBGCs are strictly positive are precisely those where these channels are not entanglement breaking: $\lambda\in(\frac{\nu}{\nu+1},1]$ for the thermal attenuator, $g\in[1,1+\frac{1}{\nu})$ for the thermal amplifier, and $\xi\in[0,1)$ for the additive Gaussian noise. This solves the open problem posed by Pirandola et al.\ in~\cite[Section~7]{Pirandola18}, deemed `crucial' by the authors of~\cite{Pirandola18}. All of our results apply to both the energy-constrained and the unconstrained scenario.
In this paper we provided a constructive method that improves on the best previously known lower bounds on the two-way quantum and secret-key capacities of all phase-insensitive bosonic Gaussian channels (piBGCs). These are the thermal attenuator $\mathcal{E}_{\lambda,\nu}$, the thermal amplifier $\Phi_{g,\nu}$, and the additive Gaussian noise $\Lambda_\xi$. 

Except for the special case of the pure loss channel $\mathcal{E}_{\lambda,0}$ and pure amplifier channel $\Phi_{g,0}$, it was an open question whether the (reverse) coherent information lower bounds could equal the true two-way quantum capacities of piBGCs: our new lower bounds in Fig.~\ref{fig_main} provide a negative answer to this question{, showing that entanglement distribution across piBGCs is possible in a much broader region than previously known. This is likely to bear a significant impact on the design of practical entanglement distribution protocols on optical networks.}

Let $\lambda(\nu)$ (resp.~$g(\nu)$) be the upper (resp.~lower) endpoint for the $\lambda$-range (resp.~$g$-range) for which the coherent information lower bound on $Q_2(\mathcal{E}_{\lambda,\nu})$ (resp.~$Q_2(\Phi_{g,\nu})$) 
vanishes. Importantly, our lower bounds on $Q_2(\mathcal{E}_{\lambda(\nu),\nu})$ and $Q_2(\Phi_{g(\nu),\nu})$ do not vanish for $\nu>0$ and, for large $\nu$, they approach $14\%$ of the tightest known upper bound. In this sense, our results constitute significant progress over the current state of affairs, although a relatively large gap between upper and lower bounds on two-way capacities remains. 

Furthermore, our numerical analysis reveals that the optimal value of $M$ in~\eqref{state_our_protocol} is never greater than three. This indicates that our protocol uses only a modest amount of input energy per channel use. Moreover, the EC version of our lower bound outperforms the tightest known lower bound on the EC two-way capacities of any piBGC, due to Noh et al.~\cite{Noh2020}, in a large parameter region (see the SM~\cite{Note1}). 

In Theorem~\ref{th1_main} we proved that the parameter regions where the (EC) two-way capacities of piBGCs are strictly positive are precisely those where these channels are not entanglement breaking: $\lambda\in(\frac{\nu}{\nu+1},1]$ for the thermal attenuator, $g\in[1,1+\frac{1}{\nu})$ for the thermal amplifier, and $\xi\in[0,1)$ for the additive Gaussian noise. This solves the open problem posed by Pirandola et al.\ in~\cite[Section~7]{Pirandola18}, deemed `crucial' by the authors of~\cite{Pirandola18}: ``What is the maximum
excess noise that is tolerable in QKD? I.e., optimizing over all QKD protocols?"
Such maximum tolerable excess noise is defined as
\begin{equation}\label{exc}
    \epsilon(\lambda)\coloneqq \frac{1-\lambda}{\lambda}\max\{\nu\ge0\,:\,K(\mathcal{E}_{\lambda,\nu})>0\}\,,
\end{equation} 
and Theorem~\ref{th1_main} establishes that $\epsilon(\lambda)=1$ for all $\lambda\in(0,1)$. In~\cite{Pirandola18,Pirandola20} it was found that $\varepsilon(\lambda)\le1$ by applying the PLOB bound, and it was found a lower bound on $\varepsilon(\lambda)$, which was far from $1$, by applying the best known lower bound on $K(\mathcal{E}_{\lambda,\nu})$~\cite{Ottaviani_new_lower}.   

We established Theorem~\ref{th1_main} through two distinct proofs that rely on the distillability criteria discovered in~\cite{2-qubit-distillation} and~\cite{Giedke01}. By examining the proofs of these criteria~\cite{2-qubit-distillation, Giedke01}, one can develop alternative entanglement distillation protocols, and hence also alternative entanglement distribution protocols, that are mathematically guaranteed to provide a positive rate when the channel is not entanglement breaking. However, our numerical observations indicate that the ebit rate of such protocols is much lower than that of %ours.
{our protocol.}

%The two-way capacities of piBGCs pinpoint the fundamental limits of repeaterless quantum optical communication and hence provide benchmarks for quantum repeaters. Note that a proof-of-principle solution to the problem of transmitting qubits over long distances without using quantum repeaters has been recently proposed~\cite{Die-Hard-2-PRL,Die-Hard-2-PRA}.

%We remark that our lower bounds do not coincide with the known upper bounds on the (EC and unconstrained) two-way capacities of piBGCs. Hence, an interesting open problem is to close these gaps, thus calculating the capacities, or at least to improve further lower and upper bounds.  
%A way to do so would be to find an efficient entanglement distillation protocol to distil the state $\Id_{A}\otimes\Phi(\ketbra{\Psi_{N_s}})$ in the limit $N_s\rightarrow\infty$, with $\ket{\Psi_{N_s}}$ being the TMSV, as done in~\cite{Winnel} for the particular case $\nu=0$. %\tcb{ [Francesco: se vogliamo guadagnare spazio, penso che il seguente paragrafo si possa accorciare o addirittura omettere.] Another method would be to devise an efficient recurrence entanglement distillation protocol for qudits and then apply it to improve our entanglement distribution protocol. Specifically, our entanglement distribution protocol would be modified as follows. First, Alice prepares many states of the form $\sum_{i=0}^{d-1}c_i\ket{i}_A\ket{i}_{A'}$ and sends the subsystems $A'$ to Bob through $\Phi$. Second, Bob makes a measurement in order to project on the span of the first $d$ Fock states. Third, Alice and Bob run the newly devised qudit recurrence protocol.  Finally, Alice and Bob run the hashing protocol. The improved bound would be obtained by optimising the ebits rate over $d\in\N^+$ and over the probability amplitudes $\{c_i\}_{i\in\{0,1,\ldots,d-1\}}$. We actually tried this method by using the P1-or-P2 protocol for qudits~\cite{p1orp2}. However, the optimal value of $d$ is $d=2$ in the parameter region where our bound outperforms all the previously known bounds.}
%In particular, evaluating the EC two-way capacities of the pure loss channel $\mathcal{E}_{\lambda,0}$ and the pure amplifier channel $\Phi_{g,0}$ is still an open problem.% At the moment, these quantities have been determined only in the unconstrained setting~\cite{PLOB}.


%We remark that our lower bounds do not coincide with the known upper bounds on the (EC and unconstrained) two-way capacities of piBGCs. Hence, an interesting open problem is to close these gaps, thus calculating the capacities, or at least to improve further lower and upper bounds. In particular, evaluating the EC two-way capacities of the pure loss channel $\mathcal{E}_{\lambda,0}$ and the pure amplifier channel $\Phi_{g,0}$ is still an open problem.

Our lower bounds do not coincide with the known upper bounds on the two-way capacities of piBGCs. Hence, an interesting open problem is to close these gaps, thus calculating the capacities, or at least to %improve further lower and upper bounds. 
{further tighten them.} A way to do so would be to find an efficient protocol to distil $\Id_{A}\otimes\Phi(\ketbra{\Psi_{N_s}})$ for $N_s\rightarrow\infty$, with $\ket{\Psi_{N_s}}$ being the TMSV, as done in~\cite{Winnel} for the particular case of the pure loss channel. Additionally, evaluating the EC two-way capacities of the pure loss and the pure amplifier channels is still an open problem.


%Building a large-scale quantum network requires to perform quantum communication tasks over long distances, which is unfortunately prevented in any point-to-point optical link (e.g.~optical fibres or free-space links) because of thermal noise. In order to overcome this limitation, it is possible to exploit quantum repeaters along the optical link. However, since quantum repeaters will likely be expensive, it is important to calculate the fundamental limits of point-to-point quantum communication without quantum repeaters. Notice that a recently proposed solution~\cite{Die-Hard-2-PRL,Die-Hard-2-PRA,die-hard} to the problem of transmitting reliably qubits over long distances without using quantum repeaters is to exploit memory effects in optical fibres~\cite{memory-review}.
%The noise affecting a signal travelling in an optical link is usually schematised as a thermal attenuator $\mathcal{E}_{\lambda,\nu}$.

%VECCHIO ABSTRACT
%Building a large-scale quantum network requires to perform quantum communication tasks over long distances, which is unfortunately prevented in a point-to-point optical link (e.g.~optical fibres or free-space links) because of thermal noise. In order to overcome this limitation, it is possible to exploit quantum repeaters along the optical link. However, since quantum repeaters will likely be expensive, it is important to calculate the fundamental limits of point-to-point quantum communication without quantum repeaters. Notice that a recently proposed solution~\cite{Die-Hard-2-PRL,Die-Hard-2-PRA,die-hard} to the problem of transmitting reliably qubits over long distances without using quantum repeaters is to exploit memory effects in optical fibres~\cite{memory-review}.
%The noise affecting a signal travelling in an optical link is usually schematised as a thermal attenuator $\mathcal{E}_{\lambda,\nu}$. In this work, we have provided an improved lower bound on the secret-key capacity, two-way private capacity, two-way quantum capacity, and two-way entanglement distribution capacity of $\mathcal{E}_{\lambda,\nu}$, both in the energy-constrained (EC) and in the unconstrained scenario (see Theorem~\ref{th_delta}). In addition, we have found that all these (EC) two-way capacities vanish if and only if $\lambda\le \frac{\nu}{\nu+1}$ (see Theorem~\ref{th1}).  

\medskip
\noindent \textbf{Acknowledgements} --- 
FAM and VG acknowledge financial support by MUR (Ministero dell'Istruzione, dell'Universit\`a e della Ricerca) through the following projects: PNRR MUR project PE0000023-NQSTI, PRIN 2017 Taming complexity via Quantum Strategies: a Hybrid Integrated Photonic approach (QUSHIP) Id. 2017SRN-BRK, and project PRO3 Quantum Pathfinder. LL was partially supported by the Alexander von Humboldt Foundation. FAM and LL thank the Freie Universit\"{a}t Berlin for hospitality. 

\medskip
\noindent\textbf{Author contributions} --- The entanglement distribution protocol was designed and optimised by FAM. The proof of Theorem~\ref{th1_main} was found in a blackboard discussion between the three authors. FAM wrote a first complete draft of the paper, which was subsequently improved by LL and VG. 

\medskip
\noindent\textbf{Supplementary Information} is available for this paper.

\medskip
\noindent\textbf{Competing interest} --- The authors declare no competing interests.



%\end{document}

\section*{Methods}\label{sec_methods}
%\subsection{Notation and preliminaries}
%\ludo{[According to Nat Photon guidelines, `Article should be divided as follows: Introduction (without heading), Results, Discussion, Online Methods.' Hence, I'm afraid we need to get rid of this whole section. Also, the Results section has enough room for the results only, while the proof methods should go in \ludo{the Methods}. I would reshuffule things around like this:
%\begin{itemize}
 %   \item the definition of Bell states goes in \ludo{the Methods}, while only the bits that are essential to the understanding of the results are moved in the Results section;
  %  \item basic material such as the informal definition of channels and capacities or eq.~(2) (to be made inline) can go in the intro;
  %  \item it's enough to say that a quantum channel is a completely positive trace preserving map, no need to bring up trace class (you'll scare them away :-));
   % \item 
%\end{itemize}
%]}

Let $\mathfrak{S}(\HH)$ denote the space of density operators on a Hilbert space $\HH$. Let $\HH_2$ be a single-qubit Hilbert space with orthonormal basis $\{\ket{0}, \ket{1}\}$. For all $i,j\in\{0,1\}$, the state $\ket{\psi_{ij}}_{AB}\in\HH_2^{(A)}\otimes\HH_2^{(B)}$ defined as 
\bb\label{Bell_states_main}
\ket{\psi_{ij}}_{AB}\coloneqq \frac{1}{\sqrt{2}}\sum_{m=0}^1 (-1)^{im}\ket{m}_A\otimes\ket{m\oplus j}_B\, ,
\ee
where $\oplus$ denotes the modulo $2$ addition, is called a Bell state (or maximally entangled state). We will also refer to $\ket{\psi_{00}}$ as an entanglement bit, or ebit. Any Bell state can be written in terms of the ebit as
\bb
    \ket{\psi_{ij}}_{AB}=\mathbb{1}_A\otimes X^j Z^i \ket{\psi_{00}}_{AB}\,,
\ee
where $\mathbb{1}_A$ denotes the identity operator on $\HH_2^{(A)}$ and $X,Z$ denote the well-known Pauli operators on $\HH_2^{(B)}$.  The Pauli-based twirling $\mathcal{T}:\mathfrak{S}( \HH_2^{(A)}\otimes\HH_2^{(B)} )\to \mathfrak{S}( \HH_2^{(A)}\otimes\HH_2^{(B)} )$ is defined as 
\bb\label{def_twirling_map}
    \mathcal{T}(\rho_{AB})=\frac{1}{4}\sum_{i,j=0}^{1}X^j_A Z^i_A\otimes  X^j_B Z^i_B\,\rho_{AB}\,(X^j_A Z^i_A\otimes  X^j_B Z^i_B)^\dagger
\ee
%\ludo{[Qua io sono abituato a mettere uno $*$ sulla seconda unitaria, quella su Bob, ma in questo caso non cambia nulla, vero?]}
for all $\rho_{AB}\in\mathfrak{S}( \HH_2^{(A)}\otimes\HH_2^{(B)} )$ and it maps any input state in a Bell-diagonal state: 
\bb
    \mathcal{T}(\rho_{AB})=\sum_{i,j=0}^{1} \bra{\psi_{ij}}\rho_{AB}\ket{\psi_{ij}}\, \ketbra{\psi_{ij}}_{AB}.
\ee
Physically realisable transformations between two quantum systems with Hilbert spaces $\HH$ (input) and $\HH'$ (output) are modelled by quantum channels, i.e.\ completely positive and trace preserving maps $\Phi:\mathfrak{S}(\HH)\to \mathfrak{S}(\HH')$. %, %where $\mathfrak{S}(\HH)$ is the space of density operators on $\HH$, and analogously for $\HH'$.
The two-way quantum capacity $Q_2(\Phi)$ and secret-key capacity $K(\Phi)$ of a quantum channel $\Phi$ is the maximum achievable rate of qubits and secret-key bits, respectively, that can be reliably transmitted through $\Phi$ by assuming that the sender Alice and the receiver Bob have free access to a public, noiseless, two-way classical communication line. The rate of qubits (resp.\ secret-key bits) is defined as the ratio between the number of reliably transmitted qubits (resp.\ secret-key bits) and the number of uses of $\Phi$. %An ebit is a Bell state $\ket{\psi_{00}}_{AB}$ shared between Alice and Bob.
A rigorous definition of the two-way capacities can be found in~\cite[Chapters 14 and 15]{Sumeet_book}. %\ludo{[Always include a precise pointer when referencing a book. Also, it may be better to use the other textbook by Mark Wilde, since that one has already been published?]} \tcb{[Francesco: I think that in the other Mark Wilde's textbook there are not the definitions of two-way capacities]}
For any $\Phi$, the two-way capacities satisfy
\bb\label{relation_2waycap_main}
Q_2(\Phi)\le K(\Phi)\,,
\ee
since an ebit can generate a secret-key bit, thanks to the `E91' protocol~\cite{Ekert91}.

In practice, Alice has access to a limited budget $N_s$ of energy to produce each input signal, as measured by a Hamiltonian $H$ on $\HH$. Fixed $N_s>0$, the energy-constrained (EC) two-way capacities $Q_2(\Phi, N_s)$ and $K(\Phi,N_s)$ %\tcb{[Francesco: Io userei la notazione $Q_2(\Phi,N_s)$ invece di $Q_2(\Phi\ludo{; H},N_s)$ dato che in seguito usiamo sempre l'operatore numero come hamiltoniana. Se vuoi uso la notazione $Q_2(\Phi\ludo{; H},N_s)$.]} \ludo{[LL: totalmente d'accordo con la notazione semplificata.]}
are defined in the same way as the two-way capacities defined above, apart from the fact that the maximisation of the rate is restricted to the strategies such that %each input signal has energy less or equal to $N_s$
the average expected value of $H$ on all input signals is required to be at most $N_s$. %[Look at the discussion around (102) and also in Section IV.C of \href{https://arxiv.org/abs/1801.08102}{arXiv:1801.08102}]
In addition note that the generalisation of~\eqref{relation_2waycap_main} to the EC case holds, i.e.
\bb\label{relation_2waycapEC_main}
Q_2(\Phi,N_s)\le K(\Phi,N_s)\,,
\ee
that any EC capacity is upper bounded by the corresponding unconstrained capacity, and {that it} tends to it in the limit $N_s\rightarrow\infty$.  

The goal of an entanglement distillation protocol %\ludo{[Ma ci servono davvero tutti questi acronimi? Io li manterrei a un minimo... Abbiamo gi\`a piBGC, LOCC, EC, PLOB, etc. Forse almeno EDP e CI li toglierei.]}
is to turn a large number $n$ of copies of a bipartite entangled state $\rho_{AB} $ shared between Alice and Bob into a number $m$ of ebits by LOCCs (local operations and classical communication). The yield %of an EDP
is defined by the ratio $m/n$. The distillable entanglement $E_d(\rho_{AB})$ %\ludo{[Temo che questo si denoti $E_d(\rho_{AB})$ :-D]} 
of $\rho_{AB}$ is defined as the maximum yields over all the possible entanglement distillation protocols~\cite{reviewEDP_dur}~\cite[Chapter 8]{Sumeet_book}. The state $\rho_{AB}$ is said to be \emph{distillable} if $E_d(\rho_{AB})>0$. The coherent information (resp.~reverse coherent information) of $\rho_{AB}$ is defined by $I_{\text{c}}(\rho_{AB})\coloneqq S(\rho_B)-S(\rho_{AB})$ (resp.\ $I_{\text{rc}}(\rho_{AB})\coloneqq S(\rho_A)-S(\rho_{AB})$), where $\rho_B\coloneqq \Tr_A\rho_{AB}$ and analogously for $\rho_A$, and moreover $S(\sigma)\coloneqq -\Tr \sigma \log_2 \sigma$ is the von Neumann entropy. The yield $I_{\text{c}}(\rho_{AB})$ (resp.\ $I_{\text{rc}}(\rho_{AB})$) is achievable by an entanglement distillation protocol~\cite{devetak2005} that only exploits one-way forward (resp.~backward) classical communication. In particular, the following inequality, known as \emph{hashing inequality}, holds:
\begin{equation}\label{hashing_ineq_main}
    E_d(\rho_{AB})\ge \max\{I_{\text{c}}(\rho_{AB})\,, I_{\text{rc}}(\rho_{AB})\}\,.
\end{equation}
Let us briefly link the notions of distillable entanglement $E_d$ and two-way quantum capacity $Q_2(\Phi ,N_s)$. Suppose that Alice produces $n$ copies of a state $\rho_{AA'}$ such that $\Tr \rho_{A'} H_{A'}\leq N_s$. Then, she can use the channel $n$ times to send all subsystems $A'$ to Bob. Then, Alice and Bob share $n$ copies of $\Id_{A}\otimes\Phi(\rho_{AA'})$, which can now be used to generate $\approx n\, E_d\left(\Id_{A}\otimes\Phi(\rho_{AA'}) \right) $ ebits by means of a suitable entanglement distillation protocol. Consequently, it holds that
\bb\label{link_D2_D_main}
    Q_2(\Phi,N_s)\ge E_d\left(\Id_{A}\otimes\Phi(\rho_{AA'}) \right)\,.
\ee
%for all $N_s\ge0$ and all $\rho_{A'A}$ satisfying $\Tr[a^\dagger a\, \rho_{A'A}]\le N_s$, where $a$ denotes the annihilation operator on $A'$.

In the context of entanglement distillation, the goal of a \emph{recurrence protocol} is to transform a certain number of copies of the state $\rho_{AB}$ into fewer copies of another state $\rho'_{AB}$ such that $\bra{\psi_{00}}\rho'_{AB}\ket{\psi_{00}}>\bra{\psi_{00}}\rho_{AB}\ket{\psi_{00}}$~\cite{Bennett-error-correction,Bennett-distillation-mixed,reviewEDP_dur}. %One chooses the number of iterations of the recurrence protocol in order to maximise the rate of the overall distillation protocol (recurrence+hashing).
Examples of recurrence protocols for qubits can be found in~\cite{Bennett-distillation-mixed,DEJMPS,DNMV}, and their generalisations to the case of qudits in~\cite{Horodecki1999,Alber_2001,Dist-Number-Theory}. In the present paper we will exploit the recently introduced P1-or-P2 recurrence protocol~\cite{p1orp2}.   Since an infinite number of iterations of a recurrence protocol is generally needed to generate a Bell state $\ket{\psi_{00}}$, the yield of a recurrence protocol is zero. To achieve a nonzero yield, one may adopt a suitable number of iterations of a recurrence protocol and then apply the hashing or breeding protocol~\cite{Bennett-error-correction,Bennett-distillation-mixed}.
The latter protocols, which exploit only one-way classical communication, achieves the yield of the hashing inequality in~\eqref{hashing_ineq_main}.
Improvements of the hashing and breeding protocols, which exploit two-way classical communication and work on bipartite-qubit systems that are diagonal in the Bell basis, have been provided in~\cite{Improvement-Hashing}.

Let $\HH_S,\HH_E\coloneqq L^2(\mathbb{R})$ be single modes of electromagnetic radiation with definite frequency and polarisation, and let $a$ and $b$ be the corresponding annihilation operators. %, respectively.
Now, let us define the piBGCs. For all $\lambda\in[0,1]$, $g\ge1$, $\nu\ge0$, $\xi\ge0$, the thermal attenuator $\mathcal{E}_{\lambda,\nu}$, the thermal amplifier $\Phi_{g,\nu}$, and the additive Gaussian noise $\Lambda_\xi$ are quantum channels on $\HH_S$ defined by 
\bb
    \mathcal{E}_{\lambda,\nu}(\rho)&\coloneqq\Tr_E\left[U_\lambda^{SE} \big(\rho^S \otimes\tau_\nu^E \big) {U_\lambda^{SE}}^\dagger\right]\,,\\
    \Phi_{g,\nu}(\rho)&\coloneqq\Tr_E\left[U_g^{SE} \big(\rho^S\otimes\tau_\nu^E\big) {U_g^{SE}}^\dagger\right]\,,\\
    \Lambda_\xi(\rho)&\coloneqq\frac{1}{\pi\xi}\int_{\mathbb C} \mathrm{d}^2 {z}\, e^{-\frac{|z|^2}{\xi}}  D(z)\,\rho\,  D(z)^\dagger\,,
\ee
where $\tau^E_\nu\coloneqq\frac{1}{\nu+1}\sum_{n=0}^\infty \left(\frac{\nu}{\nu+1}\right)^{n}\ketbra{n}_E$ is the thermal state with $\{\ket{n}_E\}_{n\in\N}$ being the Fock states on $\HH_E$, $U_\lambda^{SE}$ is the beam splitter unitary of transmissivity $\lambda$, $U_g^{SE}$ is the two-mode squeezing unitary of gain $g$, and $D(z)$ is the displacement operator:
\bb
    U_{\lambda}^{S E}&\coloneqq\exp\left[\arccos\sqrt{\lambda}\left(a^\dagger b-a\, b^\dagger\right)\right]\,,\\
    U_{g}^{S E}&\coloneqq\exp\left[\arccosh\sqrt{g}\left(a^\dagger b^\dagger-a\, b\right)\right]\,,\\
    D(z)&\coloneqq\exp{\left[z a^\dagger-z^\ast a\right] }\,.
\ee
In a communication scenario, the piBGCs are understood to map Alice's single-mode systems $A'$ to Bob's single-mode systems $B$.
%By definition, the energy of an input signal initialised in a state $\rho$ is equal to its mean photon number $\Tr[\rho\, a^\dagger a]$.
The Hamiltonian on $\HH_S$ is the photon number operator $a^\dagger a$ and, by definition, the energy of an input signal initialised in a state $\rho$ is equal to its mean photon number $\Tr[\rho\, a^\dagger a]$.
The tightest known upper bounds on the two-way capacities of these channels, shown by Pirandola, Laurenza, Ottaviani, and Banchi (PLOB)~\cite{PLOB}, are
\bb\label{PLOB_Q2_main}
    K(\mathcal{E}_{\lambda,\nu})&\le \begin{cases}
-h(\nu)-\log_2[(1-\lambda)\lambda^\nu], & \text{if $\lambda\in(\frac{\nu}{\nu+1}, 1]$,} \\
0, & \text{otherwise}
\end{cases}\\
    K(\Phi_{g,\nu})&\le \begin{cases}
-h(\nu)+\log_2\left(\frac{g^{\nu+1}}{g-1}\right), & \text{if $g\in[1, 1+\frac{1}{\nu})$,} \\
0, & \text{otherwise}
\end{cases}\\
K(\Lambda_\xi)&\le  \begin{cases}
\frac{\xi-1}{\ln2}-\log_2(\xi), & \text{if $\xi\in[0,1)$,} \\
0, & \text{otherwise}
\end{cases}
\ee
where $h(\nu)\coloneqq(\nu+1)\log_2(\nu+1)-\nu\log_2\nu$ (see~\cite{MMMM} for a strong-converse extension of the formulas above). These upper bounds vanish if and only if the piBGCs are entanglement breaking~\cite{PLOB}. %~\cite{Ent_breaking_Gaussian, Holevo-EB}
The tightest known lower bounds (before our work) on $Q_2$ are~\cite{Pirandola2009,holwer}
\begin{align}
    Q_2(\mathcal{E}_{\lambda,\nu}) &\ge\max\{0,-h(\nu)-\log_2(1-\lambda)\}\,, \label{lowQ2_main_att} \\
    Q_2(\Phi_{g,\nu})&\ge\max\left\{0,-h(\nu)+\log_2\left(\frac{g}{g-1}\right)\right\}\,, \label{lowQ2_main_amp} \\
    Q_2(\Lambda_\xi)&\ge \max\{0,-\log_2(e\,\xi)\}\,. \label{lowQ2_main_noise}
\end{align}
These lower bounds can be proved first by applying~\eqref{link_D2_D_main} with the choice $\rho_{AA'}=\ketbra{\Psi_{N_s}}_{AA'}$, where
\bb
\ket{\Psi_{N_s}}\coloneqq \frac{1}{\sqrt{N_s+1}}\sum_{n=0}^\infty \left(\frac{N_s}{N_s+1}\right)^{n/2}\ket{n}\otimes \ket{n}
\ee
is the two-mode squeezed vacuum state (TMSV) with local mean photon number equal to $N_s$, second by applying the hashing inequality in~\eqref{hashing_ineq_main}, and finally by taking the limit $N_s\rightarrow\infty$. Specifically, the lower bound in~\eqref{lowQ2_main_att} is achieved by the reverse coherent information, while that in~\eqref{lowQ2_main_amp}--\eqref{lowQ2_main_noise} is achieved by the coherent information.
Although the right-hand sides of~\eqref{lowQ2_main_att}--\eqref{lowQ2_main_noise} also lower bound the secret-key capacity $K$, improved estimates of $K(\mathcal{E}_{\lambda,\nu})$ and $K(\Phi_{g,\nu})$ have been put forth by Ottaviani et al.~\cite{Ottaviani_new_lower} (see also~\cite[Sec.~VII]{Pirandola18}) and by Wong, Ottaviani, Guo, and Pirandola (WOGP)~\cite{Wang_Q2_amplifier}, respectively.


Lower bounds on the EC two-way capacities with energy constraint $N_s$ of piBGCs are the coherent information and the reverse coherent information evaluated on the state obtained by sending the subsystem $A'$ of $\ket{\Psi_{N_s}}_{AA'}$ through the channel. For sufficiently small values of $N_s$, improved lower bounds have been found by Noh, Pirandola, and Jiang (NPJ)~\cite{Noh2020}. The best known upper bound on the EC two-way capacity of the thermal attenuator is --- depending on the parameters $\lambda$, $\nu$, and $N_s$ --- the unconstrained PLOB bound in~\eqref{PLOB_Q2_main} or the bound found by Davis, Shirokov, and Wilde (DSW)~\cite{Davis2018} (which is equal to the bound found in~\cite{TGW} for $\nu=0$). Upper bounds on the EC two-way capacities of the thermal amplifier and additive Gaussian noise are the unconstrained PLOB bound in~\eqref{PLOB_Q2_main} and the bounds which can be obtained by exploiting the results of~\cite{Goodenough16, Davis2018}.    

\bibliographystyle{unsrt}
\bibliography{biblio}


\newpage
\clearpage

\onecolumngrid
\begin{center}
\vspace*{\baselineskip}
%{\textbf{\large Supplemental material:\\ Improved lower bound on two-way quantum capacities of Gaussian channels}}\\
%{\textbf{\large Supplemental material:\\ Improved lower bound on two-way quantum and secret-key capacity of Gaussian channels}}\\
{\textbf{\large Supplemental material:\\ Maximum tolerable excess noise in CV-QKD and improved lower bound on two-way capacities}}\\
\end{center}


\renewcommand{\theequation}{S\arabic{equation}}
\renewcommand{\thethm}{S\arabic{thm}}
\setcounter{equation}{0}
\setcounter{thm}{0}
\setcounter{figure}{1}
\setcounter{table}{0}
\setcounter{section}{0}
\setcounter{page}{1}
\makeatletter

\setcounter{secnumdepth}{2}


\section*{Notation and preliminaries}
Let $\mathfrak{S}(\HH)$ be the set of quantum states on a Hilbert space $\HH$. The trace norm of a bounded linear operator $\Theta$ is defined by $\|\Theta\|_1\coloneqq \Tr\sqrt{\Theta^\dagger\Theta}\,.$ The von Neumann entropy of a quantum state $\rho$ is denoted by $S(\rho)\coloneqq -\Tr\left[\rho\log_2\rho\right]\,$. Let $\HH_2$ be a bi-dimensional Hilbert space and let $\{\ket{0}, \ket{1}\}$ be an orthonormal basis. For all $i,j\in\{0,1\}$, the state $\ket{\psi_{ij}}_{AB}\in\HH_2^{(A)}\otimes\HH_2^{(B)}$ is defined as 
\bb \label{Bell_states}
\ket{\psi_{ij}}_{AB}\coloneqq \frac{1}{\sqrt{2}}\sum_{m=0}^1 (-1)^{im}\ket{m}_A\otimes\ket{m\oplus j}_B\,,
\ee
and is called a Bell state (or maximally entangled state), where $\oplus$ denotes the modulo $2$ addition. 
%\ludo{[All papers in quantum info/computing I have seen use $\oplus$ or simply $+$ for modular addition. Shall we stick to this convention?]}


\subsection{Gaussian quantum information}
Let us briefly review the formalism of Gaussian quantum information~\cite{BUCCO}. We consider $m$-modes of harmonic oscillators $S_1$, $S_2$, $\ldots$, $S_m$, which are associated with the Hilbert space $L^2(\mathbb R^m)$ of square integrable functions. Each of these modes represents a single-mode of electromagnetic radiation with definite frequency and polarisation.
For all $j=1,2,\ldots,m$ the annihilation operator $a_j$ of the mode $S_i$ is defined as $a_j\coloneqq \frac{\hat{x}_j+i\hat{p}_j}{\sqrt{2}}$, where $\hat{x}_j$ and $\hat{p}_j$ are the well-known position and momentum operators of $S_j$. The operator $a_j^\dagger a$ is called the photon number of the mode $S_j$. The $n$th Fock state of the mode $S_j$ is denoted by $\ket{n}_{S_j}$. By defining the so-called quadrature vector $\mathbf{\hat{R}}\coloneqq (\hat{x}_1,\hat{p}_1,...,\hat{x}_m,\hat{p}_m)^{\intercal}$, one can write the canonical commutation relations as $[\mathbf{\hat{R}},\mathbf{\hat{R}}^{\intercal}]=i\,\Omega_m$, where $\Omega_m\coloneqq\left(\begin{matrix}0&1\\-1&0\end{matrix}\right)\otimes\mathbb{1}_{m}$ and $\mathbb{1}_{m}$ is the $m\times m$ identity matrix. The characteristic function $\chi_\rho: \mathbb{R}^{2m}\to \mathbb{C}$ of a state $\rho\in\mathfrak{S}(L^2(\mathbb R^m))$ is defined as $\chi_\rho(\mathbf r)=\Tr[ \rho  D_{-\mathbf{r}} ]$, where for all $\mathbf{r}\in \mathbb{R}^{2m}$ the displacement operator $ D_{\mathbf{r}}$ is defined as 
\bb\label{def_charact_func}
D_{\mathbf{r}}\coloneqq e^{i {\mathbf{r}}^{\intercal}\Omega_m \mathbf{\hat{R}}}\,.
\ee
Any state $\rho$ can be written in terms of its characteristic function as
\bb\label{inverse_fourier_displacement}
\rho=\int_{\mathbb{R}^{2m}}\frac{\mathrm{d}^{2m}\mathbf{r}}{(2\pi)^m}\chi_\rho(\mathbf r) D_{\mathbf{r}}\,
\ee
and hence quantum states and characteristic functions are in one-to-one correspondence. The first moment and the covariance matrix of a quantum state $\rho$ are defined as
\begin{align}
	&\mathbf{m}(\rho)=\Tr\left[\mathbf{\hat{R}}\,\rho\right]\,,\\
	&V(\rho)=\Tr\left[\left\{\mathbf{(\hat{R}-m(\rho)),(\hat{R}-m(\rho))}^{\intercal}\right\}\rho\right]\, ,
\end{align}
respectively, where $\{A,B\}\coloneqq AB+BA$ is the anti-commutator. Note that the covariance matrix is defined with respect an ordering of the modes in the definition of the quadrature vector: here such an ordering is $(S_1, S_2, \ldots,S_m)$. A state $\rho$ is said to be Gaussian if there exists a $2m\times 2m$ real positive definite matrix $H_\rho$ and a vector $\mathbf{m}_\rho\in\mathbb{R}^{2m}$ such that $\rho$ can be written as a ground or a thermal state of the Hamiltonian $\frac{1}{2}(\mathbf{\hat{R}}-\mathbf{m}_\rho)^{\intercal}H(\mathbf{\hat{R}}-\mathbf{m}_\rho)$, i.e.
\bb
    \rho=\frac{e^{ -\frac{1}{2}(\mathbf{\hat{R}}-\mathbf{m}_\rho)^{\intercal}H_\rho(\mathbf{\hat{R}}-\mathbf{m}_\rho)} }{\Tr\left[ e^{ -\frac{1}{2}(\mathbf{\hat{R}}-\mathbf{m}_\rho)^{\intercal}H_\rho(\mathbf{\hat{R}}-\mathbf{m}_\rho)} \right]}\,.
\ee
It can be shown that $\mathbf{m}(\rho)=\mathbf{m}_\rho$ and $V(\rho)=V_\rho$, where $V_\rho\coloneqq \coth{\left(\frac{i\,\Omega_m H_\rho}{2}\right)}i\,\Omega_m$. The characteristic function of a Gaussian state $\rho$ is a Gaussian function in $\mathbf{r}$ which can be written in terms of $\mathbf{m}(\rho) $ and $V(\rho)$ as
\bb
\chi_{\rho}(\mathbf{r})=\exp\left( -\frac{1}{4}(\Omega_m \mathbf{r})^{\intercal}V(\rho)\Omega_m \mathbf{r}+i(\Omega_m \mathbf{r})^{\intercal}\mathbf{m}(\rho) \right)\,.
\ee
An example of Gaussian state is the thermal state $\tau_{N_s}\coloneqq \frac{1}{N_s+1}\sum_{n=0}^\infty \left(\frac{N_s}{N_s+1}\right)^{n}\ketbra{n}$, where the parameter $N_s\ge0$ is its mean photon number ($N_s=\Tr[a^\dagger a\,\tau_{N_s} ]$), which satisfies
\bb\label{moments_thermal}
\mathbf{m}(\tau_{N_s})&=(0,0)^{\intercal}\,,\\
V(\tau_{N_s})&=(2N_S+1)\mathbb{1}_2 \,.
\ee
Another example of Gaussian state is the two-mode squeezed vacuum state $\ket{\Psi_{N_s}}_{S_1 S_2}$, which for all $N_s\ge0$ it is defined as
\bb\label{two_mode_sq}
\ket{\Psi_{N_s}}_{S_1S_2}\coloneqq \frac{1}{\sqrt{N_s+1}}\sum_{n=0}^\infty \left(\frac{N_s}{N_s+1}\right)^{n/2}\ket{n}_{S_1}\ket{n}_{S_2}\,,
\ee
where $N_s$ denotes the mean photon number of the mode $S_1$ (or, equivalently, of the mode $S_2$), i.e.~$N_s=\Tr_{S_2}[a_1^\dagger a_1\,\ketbra{\Psi_{N_s}}_{S_1S_2} ]$.
The first moment and covariance matrix of $\ket{\Psi_{N_s}}_{S_1S_2}$ are
\bb\label{moments_squeezed}
\mathbf{m}(\ketbra{\Psi_{N_s}})&=(0,0,0,0)^{\text{T}}\,,\\
V(\ketbra{\Psi_{N_s}})&=\left(\begin{matrix} (2N_s+1)\mathbb{1}_2 & 2\sqrt{N_s(N_s+1)}\sigma_z \\ 2\sqrt{N_s(N_s+1)}\sigma_z  &(2N_s+1)\mathbb{1}_2\end{matrix}\right)\,,
\ee
where $\mathbb{1}_2\coloneqq \left(\begin{matrix}1&0\\0&1\end{matrix}\right)$ and $\sigma_z\coloneqq \left(\begin{matrix}1&0\\0&-1\end{matrix}\right)$. 




A quantum channel is said to be Gaussian if it maps Gaussian states into Gaussian states. Later we will focus on three important examples of Gaussian quantum channels: the thermal attenuator, the thermal amplifier, and the additive Gaussian noise.
\begin{comment}
It turns out that if $\Phi:\mathfrak{S}(L^2(\mathbb R^m))\rightarrow\mathfrak{S}(L^2(\mathbb R^m))$ is a Gaussian quantum channel then there exists a $2m\times 2m$ real matrix $X$, a $2m\times 2m$ positive semi-definite real matrix $Y$, and a vector $\delta\in \mathbb{R}^{2m}$, satisfying 
\bb
Y+i\,\Omega_m\ge i X\Omega_m X^{\text{T}}\,,
\ee
such that for all state $\rho$ it holds that
\bb
\chi_{\Phi(\rho)}(\mathbf{r})=\chi_\rho(\Omega_m^{\text{T}}X\Omega_m\mathbf{r})\exp\left( -\frac{1}{4}(\Omega_m \mathbf{r})^{\intercal}Y\Omega_m \mathbf{r}+i(\Omega_m \mathbf{r})^{\intercal}\delta \right)
\ee
Hence, for any Gaussian state $\rho$ it holds that
\bb\label{relation_gen}
&\mathbf{m}\left(\Phi(\rho)\right)= X\mathbf{m}(\rho)+\delta  \,,\\
&V\left(\Phi(\rho) \right)=X V(\rho) X^{\intercal}+Y\,.
\ee
\end{comment}
Before concluding this brief recap of Gaussian quantum information, let us state a lemma which will be useful in the following. The forthcoming Lemma~\ref{ConditionPPT_cov} provides a necessary and sufficient condition on the covariance matrix to assess whether a two-mode Gaussian state is entangled~\cite{Simon00, BUCCO}. This condition is based on the fact that a two-mode Gaussian states is separable (not entangled) if and only if it is PPT~\cite{Simon00, BUCCO}.
\begin{lemma}[\cite{Simon00, BUCCO}]\label{ConditionPPT_cov}
    Let $\rho\in\mathfrak{S}(\HH_{S_1}\otimes \HH_{S_2})$ be a two-mode Gaussian state. Let us write its covariance matrix $V(\rho)$ with respect the ordering $(S_1,S_2)$ as
	\bb
	V\left( \rho \right)= \left(\begin{matrix} V_{S_1} & V_{S_1S_2} \\ V_{S_1S_2}^{\intercal} & V_{S_2}\end{matrix}\right)\,
	\ee 
	and define the function $f:\mathfrak{S}(\HH_{S_1}\otimes \HH_{S_2})\to\mathbb{R}$ as $f(\rho)\coloneqq 1+\det(V(\rho))+2\det(V_{S_1S_2})- \det(V_{S_1})-\det(V_{S_2})$.
    The state $\rho$ is entangled if and only if $f(\rho)<0$.
\end{lemma}
The forthcoming Lemma~\ref{holevo_lemma_eb_gauss} gives necessary and sufficient condition on a Gaussian quantum channel to be entanglement breaking~\cite[Chapter 4.6]{MARK}.
\begin{lemma}\cite{Holevo-EB}\label{holevo_lemma_eb_gauss}
    Let $\Phi:\mathfrak{S}(L^2(\mathbb R^m))\to\mathfrak{S}(L^2(\mathbb R^m))$ be a Gaussian quantum channel. Let $K,\beta\in\mathbb{R}^{2m\times2m}$ and $l\in\mathbb{R}^{2m}$ such that for all $\rho\in\mathfrak{S}(L^2(\mathbb R^m))$ it holds that
\bb 
&\mathbf{m}\left(\Phi(\rho)\right)=K\, \mathbf{m}(\rho)\,,\\
&V\left(\Phi (\rho) \right)=K^{\intercal}\, V(\rho)K+\beta\,.
\ee
Then, $\Phi$ is entanglement breaking if and only if $\beta$ admits the following decomposition:
\bb
\beta=\alpha+\gamma\,,\quad\text{where }\alpha,\gamma\in\mathbb{R}^{2m\times 2m}\text{ with }\alpha\ge i\, \Omega_{m}\,\text{ and }\,\gamma\ge i K^{\intercal}\,\Omega_m K\,.
\ee
    
\end{lemma}

\subsection{Two-way capacities of a quantum channel}
The two-way quantum capacity $Q_2(\Phi)$ and the secret-key capacity $K(\Phi)$ of a quantum channel $\Phi$ are the maximum achievable rate of qubits and secret-key bits, respectively, that can be reliably transmitted through $\Phi$ by assuming that the sender Alice and the receiver Bob have free access to a public, noiseless, two-way classical communication line. The rate of qubits (resp.~secret-key bits) is defined as the ratio between the number of reliably transmitted qubits (resp.~secret-key bits) and the number of uses of $\Phi$~\cite[Chapters 14 and 15]{Sumeet_book}. An ebit is a Bell state $\ket{\psi_{00}}_{AB}$ shared between Alice and Bob. For any $\Phi$, the two-way capacities satisfy
\bb\label{relation_2waycap}
Q_2(\Phi)\le K(\Phi)\,.
\ee
Indeed, by recalling that Alice and Bob can freely send an infinite amount of bits to each other, an ebit can generate a secret-key bit, thanks to E91 protocol~\cite{Ekert91}, and hence $Q_2(\Phi)\le K(\Phi)$. The two-way quantum capacity $Q_2(\Phi)$ and the secret-key capacity $K(\Phi)$ are collectively called the \emph{two-way capacities of $\Phi$}.




In practice, Alice has access to a limited budget ($N_s$) of energy to produce each input signal. Here, by definition, the energy of a signal initialised in a state $\rho$ is equal to its mean photon number $\Tr[\rho\, a^\dagger a]$. Fixed $N_s>0$, the energy-constrained (EC) two-way capacities $Q_2(\Phi,N_s)$ and $K(\Phi,N_s)$ are defined as above but the maximisation of the rate is restricted to the strategies such that the average photon number less or equal to $N_s$. In other words, $N_s$ is the maximum allowed average photon number of the input signals to the channel $\Phi$.
In addition note that the generalisation of~\ref{relation_2waycap} to the EC case  holds, i.e.
\bb\label{relation_2waycapEC}
Q_2(\Phi,N_s) \le K(\Phi,N_s) \,,
\ee
and that any EC capacity is upper bounded by the corresponding unconstrained capacity and tends to it in the limit $N_s\rightarrow\infty$. 


\begin{comment}
The forthcoming Lemma~\ref{ent_break_implies_K_zero} establishes that entanglement-breaking channels have vanishing two-way capacities.
\begin{lemma}\label{ent_break_implies_K_zero}
    The two-way capacities of any entanglement-breaking channel vanish.
\end{lemma}
\begin{proof}
     The secret-key capacity of a quantum channel is upper bounded by its squashed entanglement~\cite{squashed_channel,squashed_channel2}. Since the squashed entanglement of any separable state is zero~\cite[Chapter 5]{Sumeet_book}, the definition of squashed entanglement for channels implies that the squashed entanglement of any entanglement-breaking channel is also zero. Consequently, any entanglement-breaking channel has vanishing secret-key capacity and therefore, by exploiting~\eqref{relation_2waycap}, it has also vanishing two-way quantum capacity.
\end{proof}
\end{comment}





\subsection{Entanglement distillation}
The goal of an entanglement distillation protocol is to turn a large number $n$ of copies of a bipartite entangled state $\rho_{AB}\in\mathfrak{S}(\HH_A\otimes\HH_{B})$ shared between Alice and Bob into a smaller number $m$ of ebits by LOCCs (local operations and classical communication). The yield of an entanglement distillation protocol is defined by the ratio $m/n$. The two-way distillable entanglement $E_d(\rho_{AB})\,$ of $\rho_{AB}$ is defined as the maximum yields over all the possible entanglement distillation protocols~\cite{reviewEDP_dur}~\cite[Chapter 8]{Sumeet_book}. The state $\rho_{AB}$ is said to be \emph{distillable} if $E_d(\rho_{AB})>0$. The coherent information of $\rho_{AB}$ is defined by 
\bb
    I_{\text{c}}(\rho_{AB})\coloneqq S(\Tr_A\rho_{AB})-S(\rho_{AB}) 
\ee
and it is a yield achievable by an entanglement distillation protocol which requires classical communication only from Alice to Bob~\cite{devetak2005}.
By exchanging the roles of Alice and Bob in such an entanglement distillation protocol, the reverse coherent information of $\rho_{AB}$, which is defined by 
\bb
    I_{\text{rc}}(\rho_{AB})\coloneqq S(\Tr_B\rho_{AB})-S(\rho_{AB})\,,
\ee
is a yield achievable by an entanglement distillation protocol which only requires classical communication only from Bob to Alice~\cite{devetak2005}. In particular, the following inequality, known as \emph{hashing inequality}, holds:
\begin{equation}\label{hashing_ineq}
    E_d(\rho_{AB})\ge \max\{I_{\text{c}}(\rho_{AB})\,, I_{\text{rc}}(\rho_{AB})\}\,.
\end{equation}
Let us briefly link the notions of distillable entanglement $E_d$ and two-way quantum capacity $Q_2(\Phi,N_s)$. Suppose that Alice produces $n$ copies of a state $\rho_{AA'}$ such that the mean photon number of the half $A'$ is less or equal to $N_s$. Then, she uses $n$ times the channel $\Phi$ to send the halves $A'$, which satisfy the energy constraint, to Bob. Hence, $n$ copies of $\Id_{A}\otimes\Phi(\rho_{AA'})$ are shared between Alice and Bob and can be used to generate ebits by means of an entanglement distillation protocol. Consequently, it holds that
\bb\label{link_D2_D}
    Q_2(\Phi,N_s)\ge E_d\left(\Id_{A}\otimes\Phi(\rho_{AA'}) \right)\,
\ee
for all $N_s\ge0$ and all $\rho_{A'A}$ satisfying $\Tr[a^\dagger a\, \rho_{A'A}]\le N_s$, where $a$ denotes the annihilation operator on $A'$.






If a bipartite state $\rho_{AB}$ is such that the hashing inequality is trivial (i.e.~the right-hand side of~\eqref{hashing_ineq} is negative), in order to obtain a non-trivial lower bound on $E_d(\rho_{AB})$, one can adopt a sufficiently large number of iterations of a \emph{recurrence protocol} on $\rho_{AB}$ prior to apply the hashing inequality. In the context of entanglement distillation, the goal of a recurrence protocol is to transform a certain number of copies of the state $\rho_{AB}$ into fewer copies of another state $\rho'_{AB}$ such that $\bra{\psi_{00}}\rho'_{AB}\ket{\psi_{00}}>\bra{\psi_{00}}\rho_{AB}\ket{\psi_{00}}$~\cite{Bennett-error-correction,Bennett-distillation-mixed,reviewEDP_dur}. %One chooses the number of iterations of the recurrence protocol in order to maximise the rate of the overall distillation protocol (recurrence+hashing).
Examples of recurrence protocols for qubits can be found in ~\cite{Bennett-distillation-mixed,DEJMPS,DNMV}, and their generalisations to the case of qudits in~\cite{Horodecki1999,Alber_2001,Dist-Number-Theory}. In the present paper we will exploit the recently introduced P1-or-P2 recurrence protocol~\cite{p1orp2}.   Since it is generally needed an infinite number of iterations of a recurrence protocol to generate Bell states $\ket{\psi_{00}}$, the yield of a recurrence protocol is zero. To achieve a nonzero yield, one may adopt a suitable number of iterations of a recurrence protocol and then apply the hashing or breeding protocol~\cite{Bennett-error-correction,Bennett-distillation-mixed}.
The latter protocols, which exploit only one-way classical communication, achieves the yield of the hashing inequality in~\eqref{hashing_ineq}.
Improvements of the hashing and breeding protocols, which exploit two-way classical communication, have been provided in~\cite{Improvement-Hashing}: the two-way distillable entanglement of a convex combination of Bell states $\rho_{AB}\coloneqq \sum_{ij=0}^1\alpha_{ij}\ketbra{\psi_{ij}}$ is lower bounded by
\bb\label{improv_hashing}
   &Y(\alpha_{00},\alpha_{01},\alpha_{10},\alpha_{11})\coloneqq \max\left(0, 1-H(\{\alpha_{ij}\})+\frac{1}{2}(\alpha_{00}+\alpha_{10})(\alpha_{11}+\alpha_{01})\left[H_2\left(\frac{\alpha_{00}}{\alpha_{00}+\alpha_{10}}\right)+H_2\left(\frac{\alpha_{11}}{\alpha_{01}+\alpha_{11}}\right)\right]\right),
\ee
with $H(\{\alpha_{ij}\})\coloneqq -\sum_{m,n=0}^{1}\alpha_{mn}\log_2\alpha_{mn}$ being the Shannon entropy and $H_2(x)\coloneqq -x\log_2x-(1-x)\log_2(1-x)$ for all $x\in[0,1]$ being the binary entropy. The yield in~\eqref{improv_hashing} is larger than the yield achieved by the hashing protocol, which is $I_{\text{c}}\left(\sum_{ij=0}^1\alpha_{ij}\ketbra{\psi_{ij}}\right)=1-H(\{\alpha_{ij}\})$. Protocols with larger yiels than~\eqref{improv_hashing} may be obtained by exploiting the numerical methods introduced in~\cite{Numerical-Improvement-Hashing}.


Now, let us briefly review the definition, the relevant properties, and the known bounds on the two-way capacities of phase-insensitive bosonic Gaussian channels, namely thermal attenuator, thermal amplifier, and additive Gaussian noise.
\subsection{Thermal attenuator}
Let $\HH_S$ and $\HH_E$ be single-mode systems and let $a$ and $b$ denote their annihilation operators, respectively.
For all $\lambda\in[0,1]$ and $\nu\ge0$, a thermal attenuator $\mathcal{E}_{\lambda,\nu}:\mathfrak{S}(\HH_S)\to\mathfrak{S}(\HH_S)$ is a quantum channel defined by 
\bb\label{def_therm}
    \mathcal{E}_{\lambda,\nu}(\rho)\coloneqq\Tr_E\left[U_\lambda^{SE}\big(\rho^S \otimes\tau_\nu^E \big) {U_\lambda^{SE}}^\dagger\right]\,,
\ee
where $U_\lambda^{SE}$ denotes the unitary operator associated with a beam splitter of transmissivity $\lambda$, i.e.
\bb
	U_{\lambda}^{S E}\coloneqq\exp\left[\arccos\sqrt{\lambda}\left(a^\dagger b-a\, b^\dagger\right)\right]\,,
\ee
and $\tau_\nu\in\mathfrak{S}(\HH_E)$ denotes the thermal state with mean photon number equal to $\nu$. {The beam splitter unitary can be expressed via the following disentangling formula~\cite[Appendix 5]{BARNETT-RADMORE}
\bb
    U_{\lambda}^{SE}=e^{-\sqrt{\frac{1-\lambda}{\lambda}}ab^\dagger}e^{ \frac{1}{2}\ln\lambda\,\left(a^\dagger a -b^\dagger b\right) }e^{\sqrt{\frac{1-\lambda}{\lambda}}a^\dagger b}\,.
\ee}
By writing the quadrature vector $\mathbf{\hat{R}}$ with respect the ordering $(S,E)$, it can be shown that 
\bb\label{transf_r}
\left(U_\lambda^{SE}\right)^\dagger \mathbf{\hat{R}}\, U_{\lambda}^{SE}=S_\lambda\, \mathbf{\hat{R}}\,,
\ee
where 
\bb
S_\lambda\coloneqq	\begin{pmatrix}
		\sqrt{\lambda}\,\mathbb{1}_2 & \sqrt{1-\lambda}\,\mathbb{1}_2 \\
		-\sqrt{1-\lambda}\,\mathbb{1}_2 &\, \sqrt{\lambda}\,\mathbb{1}_2
	\end{pmatrix}\,.
\ee
This implies that for all $\sigma_{SE}\in\mathfrak{S}(\HH_S\otimes H_E)$ it holds that
\bb\label{relation_S}
&\mathbf{m}\left(U^{SE}_\lambda\sigma_{SE} \left(U_\lambda^{SE}\right)^\dagger\right)=S_\lambda\, \mathbf{m}(\sigma_{SE})\,,\\
&V\left(U^{SE}_\lambda\sigma_{SE} \left(U_\lambda^{SE}\right)^\dagger \right)=S_\lambda\,V(\sigma_{SE})\,S_\lambda^{\intercal}\,.
\ee
{In terms of the annihilation operators $a$ and $b$, the transformation in~\eqref{transf_r} reads
\bb
    \left(U_\lambda^{SE}\right)^\dagger a\, U_{\lambda}^{SE}&=\sqrt{\lambda}\,a+\sqrt{1-\lambda}\,b\,,\\
    U_\lambda^{SE} a\, \left(U_{\lambda}^{SE}\right)^\dagger&=\sqrt{\lambda}\,a-\sqrt{1-\lambda}\,b\,,\\
    \left(U_\lambda^{SE}\right)^\dagger b\, U_{\lambda}^{SE}&=-\sqrt{1-\lambda}\,a+\sqrt{\lambda}\,b\,,\\
    U_\lambda^{SE} b\, \left(U_{\lambda}^{SE}\right)^\dagger&=\sqrt{1-\lambda}\,a+\sqrt{\lambda}\,b\,.
\ee}It can be shown that for any single-mode state $\rho$ it holds that 
\bb\label{moment_therm_att}
&\mathbf{m}\left(\mathcal{E}_{\lambda,\nu}(\rho)\right)=\sqrt{
\lambda}\, \mathbf{m}(\rho)\,,\\
&V\left(\mathcal{E}_{\lambda,\nu}(\rho) \right)=\lambda\, V(\rho)+(1-\lambda)(2\nu+1)\mathbb{1}_2\,,
\ee
and, in terms of the characteristic function, for all $\mathbf{r}\in\mathbb{R}^2$ it holds that 
\bb\label{caract_att}
\chi_{\mathcal{E}_{\lambda,\nu}(\rho)}(\mathbf{r})=\chi_\rho(\sqrt{\lambda}\mathbf{r})e^{-\frac{1}{4}(1-\lambda)(2\nu+1)|\mathbf{r}|^2}\,.
\ee
By exploiting~\eqref{caract_att} and the fact that quantum states and characteristic functions are in one-to-one correspondence, for all $\lambda_1,\lambda_2\in[0,1]$ and $\nu\ge0$ the following composition rule holds:\bb\label{composition_them}
\mathcal{E}_{\lambda_1,\nu}\circ\mathcal{E}_{\lambda_2,\nu}=\mathcal{E}_{\lambda_1\lambda_2,\nu}\,.
\ee
{In Theorem~\ref{kraus_comp_thm} we will provide a simple Kraus representation of the thermal attenuator.}

\subsubsection{Bounds on two-way capacities of the thermal attenuator}
The best known upper bound on the two-way capacities of the thermal attenuator, shown by Pirandola-Laurenza-Ottaviani-Banchi (PLOB)~\cite{PLOB}, is
\bb\label{PLOB_Q2}
    K(\mathcal{E}_{\lambda,\nu})&\le \begin{cases}
-h(\nu)-\log_2[(1-\lambda)\lambda^\nu], & \text{if $\lambda\in(\frac{\nu}{\nu+1}, 1]$,} \\
0, & \text{otherwise}
\end{cases}
\ee
where 
\bb\label{bos_ent}
h(\nu)\coloneqq(\nu+1)\log_2(\nu+1)-\nu\log_2\nu\,
\ee
is the so-called bosonic entropy. The parameter region in which such an upper bound vanishes coincides with the parameter region in which the thermal attenuator $\mathcal{E}_{\lambda,\nu}$ is entanglement breaking, i.e.~$\nu\ge0$ and $\lambda\in[0,\frac{\nu}{\nu+1}]$~\cite{Ent_breaking_Gaussian, Holevo-EB}.
The best known lower bound (before our work) on $Q_2(\mathcal{E}_{\lambda,\nu})$ is given by~\cite{Pirandola2009}
\bb\label{lowQ2}
    Q_2(\mathcal{E}_{\lambda,\nu})\ge\max\{0,-h(\nu)-\log_2(1-\lambda)\}\,.
\ee
Although this is also a lower bound on $K(\mathcal{E}_{\lambda,\nu})$, it is not the best among those currently known. Indeed, an improved lower bound on $K(\mathcal{E}_{\lambda,\nu})$ has been shown by Ottaviani et al.~\cite{Ottaviani_new_lower}. In the energy-constrained case, the best known lower bound (before our work) on the EC two-way capacities of the thermal attenuator has been found by Noh-Pirandola-Jiang (NPJ)~\cite{Noh2020}, while the best known upper bound is --- depending on the parameters $\lambda$, $\nu$, and $N_s$ --- the bound found by Davis-Shirokov-Wilde (DSW)~\cite{Davis2018} or the PLOB bound in~\eqref{PLOB_Q2}.





The lower bound in~\eqref{lowQ2} on the two-way capacities of the thermal attenuator can be proved first by applying~\eqref{link_D2_D} with the choice $\rho_{AA'}=\ketbra{\Psi_{N_s}}_{AA'}$, where $\ket{\Psi_{N_s}}$ is the two-mode squeezed vacuum states with local mean photon number equal to $N_s$ defined in~\eqref{two_mode_sq},
second by applying the hashing inequality in~\eqref{hashing_ineq}, and finally by proving that the reverse coherent information satisfies
\bb\label{proof_lower}
   & \lim\limits_{N_s\rightarrow\infty}I_{\text{rc}}\left(\Id_{A}\otimes\mathcal{E}_{\lambda,\nu}(\ketbra{\Psi_{N_s}})\right) = -h(\nu)-\log_2(1-\lambda)\,.
\ee
Analogously, the coherent information  $I_{\text{c}}\left(\Id_{A}\otimes\mathcal{E}_{\lambda,\nu}(\ketbra{\Psi_{N_s}})\right)$ and the reverse coherent information $I_{\text{rc}}\left(\Id_{A}\otimes\mathcal{E}_{\lambda,\nu}(\ketbra{\Psi_{N_s}})\right)$ are lower bounds on the EC two-way capacities of the thermal attenuator $\mathcal{E}_{\lambda,\nu}$ with energy constraint equal to $N_s$:
\bb
    Q_2(\mathcal{E}_{\lambda,\nu},N_s)&\ge  \max\left\{ I_{\text{c}}\left(\Id_{A}\otimes\mathcal{E}_{\lambda,\nu}(\ketbra{\Psi_{N_s}})\right),\, I_{\text{rc}}\left(\Id_{A}\otimes\mathcal{E}_{\lambda,\nu}(\ketbra{\Psi_{N_s}})\right) \right\}\,\\
    &= \begin{cases}
I_{\text{c}}\left(\Id_{A}\otimes\mathcal{E}_{\lambda,\nu}(\ketbra{\Psi_{N_s}})\right), & \text{if $N_s\le \nu$,} \\
I_{\text{rc}}\left(\Id_{A}\otimes\mathcal{E}_{\lambda,\nu}(\ketbra{\Psi_{N_s}})\right), & \text{otherwise.}
\end{cases}
\ee
It holds that~\cite{holwer,PLOB,Pirandola2009,Noh2020} 
\bb\label{EC_coh_therm_att}
    I_{\text{c}}\left(\Id_{A}\otimes\mathcal{E}_{\lambda,\nu}(\ketbra{\Psi_{N_s}})\right)&=h\left(\lambda N_s +(1-\lambda)\nu \right)-h\left(\frac{D+(1-\lambda)(N_s-\nu)-1}{2}\right)-h\left(\frac{D-(1-\lambda)(N_s-\nu)-1}{2}\right)\,,\\
    I_{\text{rc}}\left(\Id_{A}\otimes\mathcal{E}_{\lambda,\nu}(\ketbra{\Psi_{N_s}})\right)&=h(N_s)-h\left(\frac{D+(1-\lambda)(N_s-\nu)-1}{2}\right)-h\left(\frac{D-(1-\lambda)(N_s-\nu)-1}{2}\right)\,,
\ee
where $D\coloneqq \sqrt{\left( (1+\lambda)N_s+(1-\lambda)\nu +1 \right)^2 -4\lambda N_s(N_s+1)}$. The NPJ lower bound, proved by mixing forward (coherent information) and backward (reverse coherent information) strategies, is~\cite{Noh2020}
\bb\label{npj_bound_therm}
Q_2(\mathcal{E}_{\lambda,\nu},N_s)\ge \sup_{\substack{x\in[0,1],\,N_1,N_2\ge0\\ xN_1+(1-x)N_2=N_s}} \left[  x\,I_{\text{c}}\left(\Id_{A}\otimes\mathcal{E}_{\lambda,\nu}(\ketbra{\Psi_{N_1}})\right)+(1-x)I_{\text{rc}}\left(\Id_{A}\otimes\mathcal{E}_{\lambda,\nu}(\ketbra{\Psi_{N_2}})\right) \right]\,.
\ee
Fixed $\lambda$ and $\nu$, if the energy constraint $N_s$ is sufficiently large, the NPJ lower bound is equal to the reverse coherent information bound (i.e.~the optimal values of the supremum problem in~\ref{npj_bound_therm} are $x=0$, $N_1=0$, and $N_2=N_s$).




\subsection{Thermal amplifier}
Let $\HH_S$ and $\HH_E$ be single-mode systems and let $a$ and $b$ denote their annihilation operators, respectively.
For all $g\ge 1$ and $\nu\ge0$, a thermal amplifier $\Phi_{g,\nu}:\mathfrak{S}(\HH_S)\to\mathfrak{S}(\HH_S)$ is a quantum channel defined by 
\bb\label{def_ampl}
    \Phi_{g,\nu}(\rho)\coloneqq\Tr_E\left[U_g^{SE}\big(\rho^S\otimes\tau_\nu^E\big) {U_g^{SE}}^\dagger\right]\,,
\ee
where $U_g^{SE}$ denotes the unitary operator associated with two-mode squeezing of parameter $g$, i.e.
\bb\label{def_unitary_squeez}
	U_{g}^{S E}\coloneqq\exp\left[\arccosh\sqrt{g}\left(a^\dagger b^\dagger-a\, b\right)\right]\,.
\ee
{The two-mode squeezing unitary can be expressed via the following disentangling formula~\cite[Appendix 5]{BARNETT-RADMORE}
\bb
    U_{g}^{SE}=e^{\sqrt{\frac{g-1}{g}}a^\dagger b^\dagger}e^{ \frac{1}{2}\ln\left(\frac{1}{g}\right)\,\left(a^\dagger a -b^\dagger b+1\right) }e^{-\sqrt{\frac{g-1}{g}}a b}\,.
\ee}By writing the quadrature vector $\mathbf{\hat{R}}$ with respect the ordering $(S,E)$, it can be shown that 
\bb\label{transf_r_amp}
\left(U_g^{SE}\right)^\dagger \mathbf{\hat{R}}\, U_{g}^{SE}=S_g\, \mathbf{\hat{R}}\,,
\ee
where 
\bb
S_g\coloneqq	\begin{pmatrix}
		\sqrt{g}\,\mathbb{1}_2\, & \,\sqrt{g-1}\,\sigma_z \\
		\sqrt{g-1}\,\sigma_z\, &\, \sqrt{g}\,\mathbb{1}_2
	\end{pmatrix}\,.
\ee
This implies that for all $\sigma_{SE}\in\mathfrak{S}(\HH_S\otimes H_E)$ it holds that
\bb\label{relation_S_amp}
&\mathbf{m}\left(U^{SE}_g\sigma_{SE} \left(U_g^{SE}\right)^\dagger\right)=S_g\, \mathbf{m}(\sigma_{SE})\,,\\
&V\left(U^{SE}_g\sigma_{SE} \left(U_g^{SE}\right)^\dagger \right)=S_g\,V(\sigma_{SE})\,S_g^{\intercal}\,.
\ee
{In terms of the annihilation operators $a$ and $b$, the transformation in~\eqref{transf_r_amp} reads
\bb
    \left(U_g^{SE}\right)^\dagger a\, U_{g}^{SE}&=\sqrt{g}\,a+\sqrt{g-1}\,b^\dagger\,,\\
    U_g^{SE} a\, \left(U_{g}^{SE}\right)^\dagger&=\sqrt{g}\,a-\sqrt{g-1}\,b^\dagger\,,\\
    \left(U_g^{SE}\right)^\dagger b\, U_{g}^{SE}&=\sqrt{g-1}\,a^\dagger+\sqrt{g}\,b\,,\\
    U_g^{SE} b\, \left(U_{g}^{SE}\right)^\dagger&=-\sqrt{g-1}\,a^\dagger+\sqrt{g}\,b\,.
\ee}
It can be shown that for any single-mode state $\rho$ it holds that 
\bb\label{moment_therm_amp}
&\mathbf{m}\left(\Phi_{g,\nu}(\rho)\right)=\sqrt{
g}\, \mathbf{m}(\rho)\,,\\
&V\left(\Phi_{g,\nu}(\rho) \right)=g\, V(\rho)+(g-1)(2\nu+1)\mathbb{1}_2\,,
\ee
and, in terms of the characteristic function, for all $\mathbf{r}\in\mathbb{R}^2$ it holds that 
\bb\label{caract_amp}
\chi_{\Phi_{g,\nu}(\rho)}(\mathbf{r})=\chi_\rho(\sqrt{g}\mathbf{r})e^{-\frac{1}{4}(g-1)(2\nu+1)|\mathbf{r}|^2}\,.
\ee
By exploiting~\eqref{caract_amp} and the fact that quantum states and characteristic functions are in one-to-one correspondence, for all $g_1,g_2\ge1$ and $\nu\ge0$ the following composition rule holds:\bb\label{comp_rule_amp}
\Phi_{g_1,\nu}\circ\Phi_{g_2,\nu}=\Phi_{g_1g_2,\nu}\,.
\ee
{In Theorem~\ref{kraus_comp_thm} we will provide a simple Kraus representation of the thermal amplifier.}
\subsubsection{Bounds on two-way capacities of the thermal amplifier}
The best known upper bound on the two-way capacities of the thermal amplifier, shown by PLOB~\cite{PLOB}, is
\bb\label{PLOB_amp}
    K(\Phi_{g,\nu})&\le \begin{cases}
-h(\nu)+\log_2\left(\frac{g^{\nu+1}}{g-1}\right), & \text{if $g\in[1, 1+\frac{1}{\nu})$,} \\
0, & \text{otherwise}
\end{cases}
\ee
where $h(\nu)$ is the bosonic entropy defined in~\eqref{bos_ent}. The parameter region in which such an upper bound vanishes coincides with the parameter region in which the thermal amplifier $\Phi_{g,\nu}$ is entanglement breaking, i.e.~$\nu\ge0$ and $g\ge 1+\frac{1}{\nu}$~\cite{Ent_breaking_Gaussian, Holevo-EB}. The best known lower bound (before our work) on $Q_2(\Phi_{\lambda,\nu})$ is given by~\cite{Pirandola2009}
\bb\label{lowQ2_amp}
    Q_2(\Phi_{g,\nu})\ge\max\left\{0,-h(\nu)+\log_2\left(\frac{g}{g-1}\right)\right\}\,,
\ee
which can be proved, analogously as it has been done in~\eqref{proof_lower}, by showing that the coherent information satisfies
\bb\label{proof_lower_ampl}
   & \lim\limits_{N_s\rightarrow\infty}I_{\text{c}}\left(\Id_{A}\otimes\Phi_{g,\nu}(\ketbra{\Psi_{N_s}})\right)  = -h(\nu)+\log_2\left(\frac{g}{g-1}\right)\,.
\ee
The best known lower bound on the secret-key capacity $K(\Phi_{g,\nu})$ has been shown by Wong-Ottaviani-Guo-Pirandola (WOGP)~\cite{Wang_Q2_amplifier}. In the energy-constrained scenario, the best known lower bound is the NPJ bound~\cite{Noh2020}, which is given by
\bb\label{npj_bound_amp}
Q_2(\Phi_{g,\nu},N_s)\ge \sup_{x\in[0,1]}  x\,I_{\text{c}}\left(\Id_{A}\otimes\Phi_{g,\nu}(\ketbra{\Psi_{\frac{N_s}{x}}})\right) \,,
\ee
where~\cite{holwer,PLOB,Pirandola2009,Noh2020} 
\bb
    &I_{\text{c}}\left(\Id_{A}\otimes\Phi_{g,\nu}(\ketbra{\Psi_{N_s}})\right)\\&=h\left(g N_s +(g-1)(\nu+1) \right)-h\left(\frac{D'+(g-1)(N_s+\nu+1)-1}{2}\right)-h\left(\frac{D'-(g-1)(N_s+\nu+1)-1}{2}\right)\,,
\ee
with $D'\coloneqq \sqrt{\left( (g+1)N_s+(g-1)(\nu+1) +1 \right)^2 -4g\,N_s(N_s+1)}$. Fixed $g$ and $\nu$, if the energy constraint $N_s$ is sufficiently large, the NPJ lower bound is equal to the coherent information bound (i.e.~the optimal value of the supremum problem in~\ref{npj_bound_amp} is $x=1$).



\subsection{Additive Gaussian noise}
Let $\HH_S$ be a single-mode system and let $\{D_\mathbf{r}\}_{\mathbf{r}\in\mathbb{R}^2}$ be its dispacement operators. For all $\xi\ge0$, the additive Gaussian noise $\Lambda_\xi:\mathfrak{S}(\HH_S)\to\mathfrak{S}(\HH_S)$ is a quantum channel defined by
\begin{equation}\label{def_add}
\Lambda_\xi(\rho)\coloneqq\frac{1}{2\pi\xi}\int_{\mathbb R^{2}} \mathrm{d}^2 {\mathbf{r}}\, e^{-\frac{1}{2\xi}\mathbf{r}^{\intercal}\mathbf{r}}  D_{\mathbf{r}}\rho  D_{\mathbf{r}}^\dagger\,.  
\end{equation} 
By using that $\mathbf{\hat R}= (\hat{x},\hat{p})$, $a=\frac{\hat{x}+\hat{p}}{\sqrt{2}}$, and by defining  $\mathbf{r}\coloneqq(x,p)^{\text{T}}$, $z\coloneqq \frac{x+ip}{\sqrt{2}}$, and
\bb
    D(z)\coloneqq\exp{\left[z a^\dagger-z^\ast a\right] }=D_{-\mathbf{r}}\,,
\ee
the additive Gaussian noise can be expressed in the following equivalent form: 
\bb
    \Lambda_\xi(\rho)=\frac{1}{\pi\xi}\int_{\mathbb C} \mathrm{d}^2 {z}\, e^{-\frac{|z|^2}{\xi}}  D(z)\,\rho\,  D(z)^\dagger\,,
\ee
where we have used that $\mathrm{d}^2 {\mathbf{r}}=\mathrm{d}x\,\mathrm{d}p=\frac{\mathrm{d}\text{Re}(z)\,\mathrm{d}\text{Im}(z)}{2}=\frac{\mathrm{d}^2z}{{2}}$ and we have performed the integral variable substitution $z\rightarrow -z$.
%\bb\label{transf_displ}
  %\frac{1}{2\pi\xi}\int_{\mathbb R^{2}} \mathrm{d} {\mathbf{r}} \,e^{-\frac{1}{2\xi}\mathbf{r}^{\intercal}\mathbf{r}}  &= 1\,,
%\ee
It can be shown that for all single-mode states $\rho$ it holds that
\bb\label{relation_add}
&\mathbf{m}\left(\Lambda_\xi(\rho)\right)=\mathbf{m}(\rho)\,,\\
&V\left(\Lambda_\xi(\rho) \right)=V(\rho)+2\xi\,\mathbb{1}_2\,.
\ee
and, in terms of the characteristic function, for all $\mathbf{r}\in\mathbb{R}^2$ it holds that 
\bb\label{caract_add}
\chi_{\Lambda_{\xi}(\rho)}(\mathbf{r})=\chi_\rho(\mathbf{r})e^{-\frac{1}{2}\xi|\mathbf{r}|^2}\,.
\ee
{In Theorem~\ref{kraus_comp_thm} we will provide a simple Kraus representation of the additive Gaussian noise.}
\subsubsection{Additive Gaussian noise as the strong limit of thermal attenuator or thermal amplifier}
For completeness, let us remark that the Additive Gaussian noise $\Lambda_\xi$ is the strong limit of the thermal attenuator $\mathcal{E}_{1-\frac{\xi}{\nu},\nu}$ and thermal amplifier $\Phi_{1+\frac{\xi}{\nu},\nu}$ for $\nu\rightarrow\infty$, i.e.~it holds that \bb \lim\limits_{\nu\rightarrow\infty}\|\mathcal{E}_{1-\frac{\xi}{\nu},\nu}(\rho)-\Lambda_\xi(\rho)\|_1&=0\,,\\\lim\limits_{\nu\rightarrow\infty}\|\Phi_{1+\frac{\xi}{\nu},\nu}(\rho)-\Lambda_\xi(\rho)\|_1&=0\,,\ee
for any single-mode state $\rho$.
%By exploiting~\eqref{caract_add} and the fact that quantum states and characteristic functions are in one-to-one correspondence, for all $\xi_1,\xi_2\ge0$ the following composition rule holds:\bb\Lambda_{\xi_1}\circ\Lambda_{\xi_2}=\Lambda_{\xi_1+\xi_2}\,.\ee
Indeed,~\eqref{caract_att},~\eqref{caract_amp}, and~\eqref{caract_add} imply that for any single-mode state $\rho$ and any $\mathbf{r}\in\mathbb{R}^2$ it holds that%~\cite{giova_conj} 
\bb\lim\limits_{\nu\rightarrow\infty}\chi_{\mathcal{E}_{1-\frac{\xi}{\nu},\nu}(\rho)}(\mathbf{r})&=\chi_{\Lambda_\xi(\rho)}(\mathbf{r})\,,\\ \lim\limits_{\nu\rightarrow\infty}\chi_{\Phi_{1+\frac{\xi}{\nu},\nu}(\rho)}(\mathbf{r})&=\chi_{\Lambda_\xi(\rho)}(\mathbf{r})\,.\ee
Consequently, by exploiting the fact that a sequence of states $\{\sigma_k\}_{k\in\N}\subseteq\mathfrak{S}(L^2(\mathbb{R}))$ converges in trace norm to a quantum state $\sigma\in\mathfrak{S}(L^2(\mathbb{R}))$ if and only the sequence of characteristic functions $ \{\chi_{\sigma_k} (\mathbf{r})\}_{k\in\N}$ converges pointwise to the characteristic function $\chi_\sigma(\mathbf{r})$~\cite[Theorem 2]{Cushen1971}, %(see~\cite[Lemma 4]{G-dilatable} or~\cite[Theorem 2]{Cushen1971}),
the thermal attenuator $\mathcal{E}_{1-\frac{\xi}{\nu},\nu}$ and thermal amplifier $\Phi_{1+\frac{\xi}{\nu},\nu}$ strongly converge to the additive Gaussian noise $\Lambda_\xi$ for $\nu\rightarrow\infty$.
%Hence, since a sequence of states $\sigma_k\in\mathfrak{S}(L^2(\mathbb{R}))$ converges in trace norm to a quantum state $\sigma\in\mathfrak{S}(L^2(\mathbb{R}))$ if and only if $\sigma_k$ weakly converges to $\sigma$~\cite[Theorem 2]{Cushen1971}, %(see~\cite[Lemma 4]{G-dilatable} or~\cite[Theorem 2]{Cushen1971}), it turns out that for any single-mode state $\rho$ it holds that\bb\label{add_as_limit} \lim\limits_{\nu\rightarrow\infty}\bra{\psi}\mathcal{E}_{1-\frac{\xi}{\nu},\nu}(\rho)\ket{\phi}&=\bra{\psi}\Lambda_\xi(\rho)\ket{\phi}\,,\\  \lim\limits_{\nu\rightarrow\infty}\bra{\psi}\Phi_{1+\frac{\xi}{\nu},\nu}(\rho)\ket{\phi}&=\bra{\psi}\Lambda_\xi(\rho)\ket{\phi}\,,\ee for all $\ket{\psi}, \ket{\phi}\in L^2(\mathbb{R})$.


\subsubsection{Bounds on two-way capacities of the additive Gaussian noise}
The best known upper bound on the two-way capacities of the additive Gaussian noise, shown by PLOB~\cite{PLOB}, is
\bb\label{PLOB_add}
K(\Lambda_\xi)\le  \begin{cases}
\frac{\xi-1}{\ln2}-\log_2(\xi), & \text{if $\xi< 1$,} \\
0, & \text{otherwise}
\end{cases}
\ee
where $h(\nu)$ is the bosonic entropy defined in~\eqref{bos_ent}. The parameter region in which such an upper bound vanishes coincides with the parameter region in which the Additive Gaussian noise $\Lambda_\xi$ is entanglement breaking, i.e.~$\xi\ge 1$~\cite{Ent_breaking_Gaussian, Holevo-EB}.
The best known lower bound (before our work) on $Q_2(\Lambda_\xi)$ is given by~\cite{Pirandola2009}
\bb\label{lowQ2_add}
Q_2(\Lambda_\xi)\ge \max\{0,-\log_2(e\,\xi)\}\,,
\ee
which can be proved, analogously as it has been done in~\eqref{proof_lower}, by showing that the coherent information satisfies
\bb\label{proof_lower_additive}
   \lim\limits_{N_s\rightarrow\infty}I_{\text{c}}\left(\Id_{A}\otimes\Lambda_{\xi}(\ketbra{\Psi_{N_s}}_{AA'})\right) =\log_2(e\,\xi)\,.
\ee
In the energy-constrained scenario, the best known lower bound is the NPJ bound~\cite{Noh2020}, which is given by
\bb\label{npj_bound_add}
Q_2(\Lambda_\xi,N_s)\ge \sup_{x\in[0,1]}  x\,I_{\text{c}}\left(\Id_{A}\otimes\Lambda_\xi(\ketbra{\Psi_{\frac{N_s}{x}}})\right) \,,
\ee
where~\cite{holwer,PLOB,Pirandola2009,Noh2020} 
\bb
    &I_{\text{c}}\left(\Id_{A}\otimes\Lambda_\xi(\ketbra{\Psi_{N_s}})\right)  =h\left(N_s +\xi \right)-h\left(\frac{D''+\xi-1}{2}\right)-h\left(\frac{D''-\xi-1}{2}\right)\,,
\ee
with $D''\coloneqq \sqrt{\left( 2N_s+\xi +1 \right)^2 -4N_s(N_s+1)}$. Fixed $\xi$, if the energy constraint $N_s$ is sufficiently large, the NPJ lower bound is equal to the coherent information bound (i.e.~the optimal value of the supremum problem in~\ref{npj_bound_add} is $x=1$).



 


 

\section*{Action of phase-insensitive bosonic Gaussian channels on generic operators}
{In this section we establish properties of the channel composition between pure amplifier channel and pure loss channel.}
\begin{definition}\label{def1}
For all $\lambda\in[0,1]$ and $g\ge1$ let us define the channel $\mathcal{N}_{g,\lambda}$ as the composition between pure amplifier channel $\Phi_{g,0}$ and pure loss channel $\mathcal{E}_{\lambda,0}$, i.e.
\bb\label{def_comp}
    \mathcal{N}_{g,\lambda}\coloneqq\Phi_{g,0}\circ \mathcal{E}_{\lambda,0}\,.
\ee
\end{definition}
\begin{lemma}\label{lemma_eb_comp}
    The channel $\mathcal{N}_{g,\lambda}$ is entanglement breaking if and only if $(1-\lambda)g\ge 1$.
\end{lemma}
\begin{proof}
    First, let us determine the parameter region of $g$ and $\lambda$ where the channel $\mathcal{N}_{g,\lambda}$ is entanglement breaking. Since $\mathcal{N}_{g,\lambda}$ is a Gaussian channel, we can apply Lemma~\ref{holevo_lemma_eb_gauss}. By using~\eqref{transf_caract}, one can show that $\mathcal{N}_{g,\lambda}$ transforms the first moment and the covariance matrix as
\bb\label{relation_comp}
    &\mathbf{m}\left(\mathcal{N}_{g,\lambda}(\rho)\right)=\sqrt{g\lambda}\,\mathbf{m}(\rho)\,,\\
    &V\left(\mathcal{N}_{g,\lambda}(\rho) \right)=g\lambda\, V(\rho)+(2g-1-g\lambda)\,\mathbb{1}_2\,,
\ee
for all quantum states $\rho$. Hence, Lemma~\ref{holevo_lemma_eb_gauss} establishes that $\Phi$ is entanglement breaking if and only if there exists $\alpha,\gamma\in\mathbb{R}^{2\times 2}\text{ with }\alpha\ge i\, \Omega_{1}\,\text{ and }\,\gamma\ge i \lambda g\,\Omega_1$ such that
\bb\label{conditionEB_comp}
(2g-1-g\lambda)\,\mathbb{1}_2=\alpha+\gamma\,.
\ee
The condition in~\eqref{conditionEB_comp} is equivalent to
\bb\label{conditionEB_comp2}
    (2g-1-g\lambda)\,\mathbb{1}_2\ge i\,(1+\lambda g)\,\Omega_1\,.
\ee
Indeed, if the condition in~\eqref{conditionEB_comp} is satisfied, then $(2g-1-g\lambda)\,\mathbb{1}_2=\alpha+\gamma\ge i\,(1+\lambda g)\,\Omega_1$, i.e.~also the condition in~\eqref{conditionEB_comp2} is satisfied. Conversely, assume that the condition in~\eqref{conditionEB_comp2} is satisfied. Then, the fact that 
\bb\label{fact_diag}
x\,\mathbb{1}_2\ge i\, \Omega_1 \quad\text{if and only if}\quad x\ge1\,,
\ee
implies that $(1-\lambda)g\ge1$. Consequently, by choosing $\alpha\coloneqq\mathbb{1}_2$ and $\gamma\coloneqq (2g-2-g\lambda)\mathbb{1}_2$ and by using~\eqref{fact_diag}, it holds that the condition in~\eqref{conditionEB_comp} is satisfied with $\alpha\ge i\, \Omega_{1}\,\text{ and }\,\gamma\ge i \lambda g\,\Omega_1$. By exploiting~\eqref{fact_diag}, we deduce that $\mathcal{N}_{g,\lambda}$ is entanglement breaking if and only if $(1-\lambda)g\ge 1$. 
\end{proof}
\begin{lemma}\label{lemma_comp_bos}
    Let $\nu\ge 0$, $\lambda\in[0,1]$, $g\ge 1$, and $\xi\ge0$. The thermal attenuator $\mathcal{E}_{\lambda,\nu}$, the thermal amplifier $\Phi_{g,\nu}$, and the additive Gaussian noise $\Lambda_\xi$ can be expressed in terms of the composition between pure amplifier channel and pure loss channel as
    \bb\label{channels_as_composition}
        \mathcal{E}_{\lambda,\nu}&=\mathcal{N}_{1+(1-\lambda)\nu\,,\,\frac{\lambda}{1+(1-\lambda)\nu}}\,\,,\\
        \Phi_{g,\nu}&=\mathcal{N}_{g+(g-1)\nu\,,\,\frac{g}{g+(g-1)\nu}}\,\,,\\
        \Lambda_{\xi}&=\mathcal{N}_{1+\xi\,,\,\frac{1}{1+\xi}}\,\,.
    \ee
\end{lemma}
\begin{proof}
    Let $\rho$ be a single-mode state. The characteristic function of $\mathcal{N}_{g,\lambda}(\rho)$ is
    \bb\label{transf_caract}
        \chi_{\mathcal{N}_{g,\lambda}(\rho)}(\mathbf{r})=\chi_{\Phi_{g,0}\circ\mathcal{E}_{\lambda,0}(\rho)}(\mathbf{r})=\chi_{\mathcal{E}_{\lambda,0}(\rho)}(\sqrt{g}\mathbf{r})e^{-\frac{1}{4}(g-1)|\mathbf{r}|^2}=\chi_{\rho}(\sqrt{g\lambda}\,\mathbf{r})e^{-\frac{1}{4}\left[2g-1-g\lambda\right]|\mathbf{r}|^2}\,
    \ee
    for all $\mathbf{r}\in\mathbb{R}^2$, where we have used~\eqref{caract_att} and~\eqref{caract_amp}. Consequently, by exploiting~\eqref{caract_att},~\eqref{caract_amp}, and~\eqref{caract_add}, one can check that for all $\mathbf{r}\in\mathbb{R}^2$ it holds that
    \bb
        \chi_{\mathcal{E}_{\lambda,\nu}(\rho)}(\mathbf{r})&= \chi_{\mathcal{N}_{1+(1-\lambda)\nu\,,\,\frac{\lambda}{1+(1-\lambda)\nu}}}(\mathbf{r}) \,,\\
        \chi_{\Phi_{g,\nu}(\rho)}(\mathbf{r})&= \chi_{\mathcal{N}_{g+(g-1)\nu\,,\,\frac{g}{g+(g-1)\nu}}(\rho)}(\mathbf{r}) \,,\\
        \chi_{\Lambda_{\xi}(\rho)}(\mathbf{r})&= \chi_{\mathcal{N}_{1+\xi\,,\,\frac{1}{1+\xi}}(\rho)}(\mathbf{r}) \,.\\
    \ee
    Hence, by exploiting the fact that quantum states and characteristic functions are in one-to-one correspondence,~\eqref{channels_as_composition} is proved.
\end{proof}
The forthcoming Theorem~\ref{kraus_comp_thm} provides a simple Kraus representation of $\mathcal{N}_{g,\lambda}$ and allows one to easily calculate the output of $\mathcal{N}_{g,\lambda}$ for generic input operators.
\begin{thm}\label{kraus_comp_thm}
Let $\lambda\in[0,1]$ and $g\ge 1$. The quantum channel $\mathcal{N}_{g,\lambda}$, defined in Definition~\eqref{def1}, admits the following Kraus representation:
\bb\label{krausform_comp}
    \mathcal{N}_{g,\lambda}(\rho)=\sum_{k,m=0}^\infty M_{k,m}^{\text{(comp)}}(g,\lambda)\,\rho \left(M_{k,m}^{\text{(comp)}}(g,\lambda)\right)^\dagger\,,
\ee
where
\bb\label{kraus_expl_comp}
    M_{k,m}^{(comp)}(g,\lambda)\coloneqq M_{k}^{\text{(pure amp)}}(g)\,M_{m}^{\text{(pure loss)}}(\lambda)=\sqrt{ \frac{ (g-1)^k\, (1-\lambda)^m}{k!\,m!\, g^{k+1}  } }(a^\dagger)^k \left(\sqrt{\frac{\lambda}{g}}\right)^{  a^\dagger a }a^m
\ee
and where we have introduced the Kraus operators of pure loss channel and pure amplifier channel:
\begin{align}
    M_{k}^{\text{(pure amp)}}(g)&\coloneqq \frac{1}{\sqrt{g\,k!}}\left(\sqrt{\frac{g-1}{g}}\right)^k (a^\dagger)^k \left(\frac{1}{\sqrt{g}}\right)^{  a^\dagger a }\,,\\
    M_{m}^{\text{(pure loss)}}(\lambda)&\coloneqq\sqrt{\frac{{(1-\lambda)^m}}{m!}}  (\sqrt{\lambda})^{ a^\dagger a }\, a^m\,.
\end{align}
In particular, by letting $\ket{n}$ and $\ket{i}$ two Fock states, it holds that
\begin{align}\label{action_comp_chan}
    \mathcal{N}_{g,\lambda}(\ketbraa{n}{i})&=\sum_{l=\max(i-n,0)}^\infty f_{n,i,l}(g,\lambda)\ketbraa{l+n-i}{l}\,.
\end{align}
where
\begin{align}\label{def_f_comp}
    f_{n,i,l}(g,\lambda)&\coloneqq \sum_{m=\max(i-l,0)}^{\min(n,i)}\frac{\sqrt{n!i!l!(l+n-i)!}}{(n-m)!(i-m)!m!(l+m-i)!} \frac{(g-1)^{l+m-i}(1-\lambda)^m\lambda^{\frac{n+i-2m}{2}}}{g^{l+1+\frac{n-i}{2}} } \,.
\end{align}
\end{thm}
\begin{proof}
By using~\eqref{def_therm}, the pure loss channel can be written as
    \bb\label{kraus_pure_def}
        \mathcal{E}_{\lambda,0}(\rho)=\sum_{m=0}^\infty M_{m}^{\text{(pure loss)}}(\lambda)\,\rho \left(M_{m}^{\text{(pure loss)}}(\lambda)\right)^\dagger\,,
    \ee
where for all $m\in\N$ the Kraus operator $M_{m}^{\text{(pure loss)}}(\lambda)$ is
    \bb
        M_{m}^{\text{(pure loss)}}(\lambda)\coloneqq (-1)^m\bra{m}_E U_{\lambda}^{SE}\ket{0}_E\,.
    \ee
Hence, by using the disentangling formula for beam splitter unitary~\cite[Appendix 5]{BARNETT-RADMORE}
\bb
    U_{\lambda}^{SE}=e^{-\sqrt{\frac{1-\lambda}{\lambda}}ab^\dagger}e^{ \frac{1}{2}\ln\lambda\,\left(a^\dagger a -b^\dagger b\right) }e^{\sqrt{\frac{1-\lambda}{\lambda}}a^\dagger b}\,
\ee
and the fact that
\bb
    e^{ -\frac{1}{2}\ln\lambda\, a^\dagger a }\,a^m\, e^{ \frac{1}{2}\ln\lambda\, a^\dagger a }=\lambda^{m/2} a\,,
\ee
it holds that
\bb
    M_{m}^{\text{(pure loss)}}(\lambda)=\frac{1}{\sqrt{m!}}\left(\sqrt{\frac{1-\lambda}{\lambda}}\right)^m a^m e^{ \frac{1}{2}\ln\lambda\, a^\dagger a }=\sqrt{\frac{{(1-\lambda)^m}}{m!}}  (\sqrt{\lambda})^{ a^\dagger a }\, a^m\,.
\ee
By using~\eqref{def_ampl}, the pure amplifier channel can be written as
    \bb\label{kraus_amp_def}
        \Phi_{g,0}(\rho)=\sum_{k=0}^\infty M_{k}^{\text{(pure amp)}}(g)\,\rho \left( M_{k}^{\text{(pure amp)}}(g) \right)^\dagger\,,
    \ee
where for all $k\in\N$ the Kraus operator $M_{k}^{\text{(pure amp)}}(g)$ is
    \bb
        M_{k}^{\text{(pure amp)}}(g)\coloneqq \bra{k}_E U_{g}^{SE}\ket{0}_E\,.
    \ee
Hence, by using the disentangling formula for the two-mode squeezing unitary~\cite[Appendix 5]{BARNETT-RADMORE}
\bb
    U_{g}^{SE}=e^{\sqrt{\frac{g-1}{g}}a^\dagger b^\dagger}e^{ \frac{1}{2}\ln\left(\frac{1}{g}\right)\,\left(a^\dagger a -b^\dagger b+1\right) }e^{-\sqrt{\frac{g-1}{g}}a b}\,,
\ee
it holds that
\bb
    M_{k}^{\text{(pure amp)}}(g)=\frac{1}{\sqrt{g\,k!}}\left(\sqrt{\frac{g-1}{g}}\right)^k (a^\dagger)^k \left(\frac{1}{\sqrt{g}}\right)^{  a^\dagger a }\,.
\ee
By using~\eqref{kraus_pure_def},~\eqref{kraus_amp_def}, and the fact $\mathcal{N}_{g,\lambda}=\Phi_{g,0}\circ\mathcal{E}_{\lambda,0}$,~\eqref{krausform_comp} is proved. Now, let us calculate $\mathcal{N}_{g,\lambda}(\ketbraa{n}{i})=\sum_{m,n=0}^\infty M_{k,m}^{\text{(comp)}}(g,\lambda)\ketbraa{n}{i}( M_{k,m}^{\text{(comp)}}(g,\lambda) )^\dagger$ in order to prove~\eqref{action_comp_chan}. By exploiting the following formulae
\bb
a^m\ket{n}&=
\begin{cases}
	\sqrt{\frac{n!}{(n-m)!}}\ket{n-m}, & \text{if $n\ge m$,} \\
	0, & \text{otherwise}
\end{cases}\\
\left(a^\dagger\right)^k\ket{n-m}&=
\sqrt{\frac{(n-m+k)!}{(n-m)!}}\ket{n-m+k}\,,
\ee
 for $m>n$ it holds that $M_{k,m}\ket{n}=0$, otherwise for $m\le n$ it holds that
\bb
     M_{k,m}^{(comp)}(g,\lambda)\ket{n}&=\frac{1}{(n-m)!}\sqrt{\frac{n!(n-m+k)!}{k!m!}}  \sqrt{\frac{(g-1)^{k}(1-\lambda)^m\lambda^{n-m}}{g^{k+1+n-m}}}\ket{n-m+k}\,.
\ee
Consequently, we conclude that
\bb
    \mathcal{N}_{g,\lambda}(\ketbraa{n}{i})&=\sum_{k=0}^{\infty}\sum_{m=0}^{\min(n,i)}\frac{\sqrt{n!(n-m+k)!i!(i-m+k)!}}{(n-m)!(i-m)!k!m!}  \sqrt{\frac{(g-1)^{2k}(1-\lambda)^{2m}\lambda^{n+i-2m}}{g^{2k+2+n+i-2m}}}\ketbraa{n-m+k}{i-m+k}\\&=\sum_{l=\max(i-n,0)}^\infty f_{n,i,l}(g,\lambda)\ketbraa{l+n-i}{l}   \,.
\ee
Hence,~\eqref{action_comp_chan} is proved.
\end{proof}
Calculating the action of Gaussian channels on non-Gaussian states is cumbersome in general. The forthcoming Theorem~\ref{gen_master_eq_trick} overcomes this difficulty and allows one to easily calculate the output of all piBGCs for generic input operators. 
\begin{thm}\label{gen_master_eq_trick}
Let $\nu\ge 0$, $\lambda\in[0,1]$, $g\ge 1$, and $\xi\ge0$. The thermal attenuator $\mathcal{E}_{\lambda,\nu}$, the thermal amplifier $\Phi_{g,\nu}$, and the additive Gaussian noise $\Lambda_\xi$ admit the following Kraus representations:
\bb\label{krausform_att}
    \mathcal{E}_{\lambda,\nu}(\rho)=\sum_{k,m=0}^\infty M_{k,m}^{\text{(att)}}(\lambda,\nu)\,\rho \left(M_{k,m}^{\text{(att)}}(\lambda,\nu)\right)^\dagger\,,
\ee
\bb\label{krausform_amp}
    \Phi_{g,\nu}(\rho)=\sum_{k,m=0}^\infty M_{k,m}^{\text{(amp)}}(g,\nu)\,\rho \left(M_{k,m}^{\text{(amp)}}(g,\nu)\right)^\dagger\,,
\ee
\bb\label{krausform_add}
    \Lambda_{\xi}(\rho)=\sum_{k,m=0}^\infty M_{k,m}^{\text{(add)}}(\xi)\,\rho \left(M_{k,m}^{\text{(add)}}(\xi)\right)^\dagger\,,
\ee
where 
\begin{align}
    M_{k,m}^{\text{(att)}}(\lambda,\nu)&\coloneqq M_{k,m}^{\text{(comp)}}\left(1+(1-\lambda)\nu,\frac{\lambda}{1+(1-\lambda)\nu}\right)\,,\label{kraus_op_att}\\
    M_{k,m}^{\text{(amp)}}(g,\nu)&\coloneqq M_{k,m}^{\text{(comp)}}\left(g+(g-1)\nu,\frac{g}{g+(g-1)\nu}\right)\,,\label{kraus_op_ampl}\\
    M_{k,m}^{\text{(add)}}(\xi)&\coloneqq M_{k,m}^{\text{(comp)}}\left(1+\xi,\frac{1}{1+\xi}\right)\,,\label{kraus_op_add}
\end{align}
with $M_{k,m}^{\text{(comp)}}$ being defined in~\eqref{kraus_expl_comp}.
In particular, by letting $\ket{n}$ and $\ket{i}$ two Fock states, it holds that
\begin{align}
    \mathcal{E}_{\lambda,\nu}(\ketbraa{n}{i})&=\sum_{l=\max(i-n,0)}^\infty f_{n,i,l}\left(1+(1-\lambda)\nu,\frac{\lambda}{1+(1-\lambda)\nu}\right)\,\ketbraa{l+n-i}{l}\,,\label{action_ni_att}\\
    \Phi_{g,\nu}(\ketbraa{n}{i})&=\sum_{l=\max(i-n,0)}^\infty f_{n,i,l}\left(g+(g-1)\nu,\frac{g}{g+(g-1)\nu}\right)\,\ketbraa{l+n-i}{l}\,,\label{action_ni_amp}\\
    \Lambda_{\xi}(\ketbraa{n}{i})&=\sum_{l=\max(i-n,0)}^\infty f_{n,i,l}\left(1+\xi,\frac{1}{1+\xi}\right)\,\ketbraa{l+n-i}{l}\,,\label{action_ni_add}
\end{align}
with $f_{n,i,l}$ being defined in~\eqref{def_f_comp}.
\end{thm}
\begin{proof}
    Theorem~\ref{gen_master_eq_trick} is a direct consequence of Lemma~\ref{lemma_comp_bos} and Theorem~\ref{kraus_comp_thm}.
\end{proof}
We observe here that the Kraus representation in~\eqref{kraus_op_att} of the thermal attenuator is precisely the one obtained in~\cite{Die-Hard-2-PRA} via the ``master equation trick". 






\section*{Results}
In this section we expound our results. In subsection~\ref{sub_preliminary} first we prove preliminary results on the two-way capacities of generic quantum channels and second we apply them to the composition between pure amplifier channel and pure loss channel. In subsection~\ref{sub_res_twoway} we specialise these results to the case of piBGCs (thermal attenuator, thermal amplifier, and additive Gaussian noise) and we find the following two main results:
\begin{itemize}
    \item The parameter regions where the (EC) two-way capacities of piBGCs are strictly positive are precisely those where these channels are not entanglement breaking;
    \item We find a new lower bound on the secret-key and two-way quantum capacity of piBGCs, which constitutes a significant improvement with respect the state-of-the-art lower bounds~\cite{Ottaviani_new_lower,Pirandola2009,Pirandola18,Wang_Q2_amplifier,Noh2020} in many parameter regions.
\end{itemize}
\subsection{Preliminary results}\label{sub_preliminary}
We start by proving the following lemma.
\begin{lemma}\label{lemma_Q2_positive}
    Let $\Phi:\mathfrak{S}(L^2(\mathbb{R}))\to\mathfrak{S}(L^2(\mathbb{R}))$ be a single-mode Gaussian quantum channel and let $N_s>0$. Suppose that $f\left( \Id\otimes\Phi(\ketbra{\Psi_{N_s}})\right)<0$, where $\ket{\Psi_{N_s}}$ is the two-mode squeezed vacuum state defined in~\eqref{two_mode_sq} and $f$ is the function defined in Lemma~\ref{ConditionPPT_cov}.
    The energy-constrained two-way capacities $Q_2(\Phi,N_s)$ and $K(\Phi,N_s)$ are strictly positive. In particular, the (unconstrained) two-way capacities $Q_2(\Phi)$ and $K(\Phi)$ are strictly positive.
\end{lemma}
\begin{proof} 
Since the state $\Id \otimes\Phi(\ketbra{\Psi_{N_s}})$ is a two-mode Gaussian state, we can apply Lemma~\ref{ConditionPPT_cov} to conclude that it is entangled. Consequently, since any two-mode Gaussian entangled state is distillable~\cite{Giedke01}, then  $\Id\otimes\Phi(\ketbra{\Psi_{N_s}})$ is distillable. 
Hence, by exploiting~\eqref{link_D2_D}, we deduce that $Q_2(\Phi,N_s)>0$. In addition,~\eqref{relation_2waycapEC} implies that $K(\Phi,N_s)>0$. Finally, since the energy-constrained capacities are lower bounds on the corresponding unconstrained capacities, we conclude that the unconstrained two-way capacities of $\Phi$ are strictly positive.
\end{proof}


The forthcoming Theorem~\ref{th1} determines the parameter region of $g\ge1$ and $\lambda\in[0,1]$ where the composition $\mathcal{N}_{g,\lambda}\coloneqq\Phi_{g,0}\circ \mathcal{E}_{\lambda,0}$ between pure amplifier channel $\Phi_{g,0}$ and pure loss channel $\mathcal{E}_{\lambda,0}$ has strictly positive (EC) two-way capacities. In particular, we show that the (EC) two-way capacities of $\mathcal{N}_{g,\lambda}$ are strictly positive if and only if $\mathcal{N}_{g,\lambda}$ is not entanglement breaking. 
\begin{thm}\label{th1}
    Let $\lambda\in[0,1]$, $g\ge1$, and $N_s>0$. The energy-constrained two-way capacities $Q_2(\mathcal{N}_{g,\lambda},N_s)$ and $K(\mathcal{N}_{g,\lambda},N_s)$ are strictly positive if and only if $(1-\lambda)g< 1$, i.e.~if and only if $\mathcal{N}_{g,\lambda}$ is not entanglement breaking. In particular, the same holds for the unconstrained two-way capacities.
\end{thm}
\begin{proof}
Suppose that $(1-\lambda)g\ge 1$. Then Lemma~\ref{lemma_eb_comp} implies that $\mathcal{N}_{g,\lambda}$ is entanglement breaking and hence~\cite{Goodenough16} its two way-capacities vanish. 



Now, suppose that $(1-\lambda)g< 1$. Let us check that the hypothesis of Lemma~\ref{lemma_Q2_positive} is fulfilled, i.e.~we need to check that $f\left( \Id\otimes\mathcal{N}_{g,\lambda}(\ketbra{\Psi_{N_s}})\right)<0$, where $\ket{\Psi_{N_s}}$ is the two-mode squeezed vacuum state defined in~\eqref{two_mode_sq} and $f$ is the function defined in Lemma~\ref{ConditionPPT_cov}. Let us calculate the covariance matrix of the state 
\bb 
&\Id_{A}\otimes\mathcal{N}_{g,\lambda}(\ketbra{\Psi_{N_s}}_{AA'}) \\&=  \Tr_{E_1E_2}\left[\left(\mathbb{1}_A\otimes U_g^{A'E_1}\otimes U_\lambda^{A'E_2}\right)\, \ketbra{\Psi_{N_s}}_{AA'}\otimes\ketbra{0}_{E_1}\otimes\ketbra{0}_{E_2}\text{} \left(\mathbb{1}_A\otimes U_g^{A'E_1}\otimes U_\lambda^{A'E_2}\right)^\dagger\right]\,	
\ee
with respect the ordering $(A,A',E_1,E_2)$.
By using~\eqref{relation_S} and~\eqref{relation_S_amp}, one can show that the covariance matrix of 
\bb
\left(\mathbb{1}_A\otimes U_g^{A'E_1}\otimes U_\lambda^{A'E_2}\right)\, \ketbra{\Psi_{N_s}}_{AA'}\otimes\ketbra{0}_{E_1}\otimes\ketbra{0}_{E_2}\text{} \left(\mathbb{1}_A\otimes U_g^{A'E_1}\otimes U_\lambda^{A'E_2}\right)^\dagger
\ee
with respect the ordering $(A,A',E_1,E_2)$ is
\bb\label{cov_AA'EE_comp}
\left(\mathbb{1}_2\oplus S_g\oplus \mathbb{1}_2\right) 
\left(\mathbb{1}_2\oplus \bar{S}_\lambda\right) \left(V(\ketbra{\Psi_{N_s}}_{AA'})\oplus V(\ketbra{0})\oplus V(\ketbra{0})\right) \left(\mathbb{1}_2\oplus \bar{S}_\lambda^{\intercal}\right)\left(\mathbb{1}_2\oplus S_g^{\intercal}\oplus \mathbb{1}_2\right)\,,
\ee
where
\bb
&\bar{S}_\lambda \coloneqq \left(\begin{matrix} \sqrt{\lambda}\,\mathbb{1}_2 & 0_{2\times 2} & \sqrt{1-\lambda}\,\mathbb{1}_2 \\ 0_{2\times 2} & \mathbb{1}_{2} & 0_{2\times 2} \\
-\sqrt{1-\lambda}\,\mathbb{1}_2 & 0_{2\times 2} & \sqrt{\lambda}\,\mathbb{1}_2 \end{matrix}\right) 
\ee 
and $0_{2\times 2}\coloneqq\left(\begin{matrix}0&0\\0&0\end{matrix}\right)$. Hence, since $V\left( \Id_{A}\otimes\mathcal{N}_{g,\lambda}(\ketbra{\Psi_{N_s}}_{AA'}) \right)$ is the $4\times4$ upper-left block of the covariance matrix in~\eqref{cov_AA'EE_comp}, one can show that
\bb\label{cov_choi_comp}
&V\left( \Id_{A}\otimes\mathcal{N}_{g,\lambda}(\ketbra{\Psi_{N_s}}_{AA'}) \right) = \left(\begin{matrix} (2N_s+1)\mathbb{1}_2 & 2\sqrt{g\lambda N_s(N_s+1)}\sigma_z \\ 2\sqrt{g\lambda N_s(N_s+1)}\sigma_z  &[2g\left( 1+\lambda N_s\right)-1]\mathbb{1}_2\end{matrix}\right)\,,
\ee 
where we used~\eqref{moments_thermal} and~\eqref{moments_squeezed}.
Consequently, since 
\bb
&f\left( \Id_{A}\otimes\mathcal{N}_{g,\lambda}(\ketbra{\Psi_{N_s}}_{AA'} \right) = -16 N_s (1 + N_s) g \left(1 - (1-\lambda)g \right)\,,
\ee 
and since $(1-\lambda)g < 1$, we have that $f\left( \Id_{A}\otimes\mathcal{N}_{g,\lambda}(\ketbra{\Psi_{N_s}}_{AA'} \right)<0$, i.e.~the hypothesis of Lemma~\ref{lemma_Q2_positive} is fulfilled. Hence, Lemma~\ref{lemma_Q2_positive} implies that the energy-constrained two-way capacities of $\mathcal{N}_{g,\lambda}$ are strictly positive. {This concludes the proof of Theorem~\ref{th1}. In Remark~\ref{remark_alternative_proof} we will provide an alternative proof.}
\end{proof}

In the forthcoming Theorem~\ref{th_lower_Q2} we obtain a lower bound on the two-way capacities of a quantum channel $\Phi:\mathfrak{S}(L^2(\mathbb{R}))\to\mathfrak{S}(L^2(\mathbb{R}))$ by introducing a protocol to distribute ebits though $\Phi$. The idea of such a protocol is the following. First, Alice prepares states of the form
\bb\label{initial_state}
    \ket{\Psi_{M,c}}_{AA'}\coloneqq c\ket{0}_A\ket{0}_{A'}+\sqrt{1-c^2}\ket{M}_A\ket{M}_{A'}\,,
\ee
where $M\in\N^+$ and $c\in(0,1)$. Then, she sends the halves $A'$ to Bob trough $\Phi$, who makes a measurement on each half in order to project his half onto the span of $\{\ket{0},\ket{M}\}$. Then, Alice and Bob run $k$ times the P1-or-P2 recurrence protocol~\cite{p1orp2} on the resulting states. After this, Alice and Bob run the improved hashing protocol introduced in~\cite{Improvement-Hashing} in order to generate ebits. Let $R(\Phi,M,c,k)$ be the rate of distributed ebits of this protocol. A lower bound on $Q_2(\Phi)$ (and hence on $K(\Phi)$) can be obtained by maximising $R(\Phi,M,c,k)$ over $M\in\N^+$, $c\in(0,1)$, and $k\in\N$.



\begin{thm}\label{th_lower_Q2}
Let $\Phi:\mathfrak{S}(L^2(\mathbb{R}))\to\mathfrak{S}(L^2(\mathbb{R}))$ be a quantum channel which maps a single-mode system $A'$ into another single-mode system $B$. The EC two-way capacities $Q_2(\Phi,N_s)$ and $K(\Phi,N_s)$ satisfy the following lower bound
\bb\label{lowQ2_genEC}
	K(\Phi,N_s)&\ge Q_2(\Phi,N_s) \ge \sup_{\substack{c\in(0,1),\, M\in\N^+,\, k\in\N\\ (1-c^2)M\le N_s}} R(\Phi,M,c,k)\,,
\ee
and, in particular, the unconstrained two-way capacities satisfy
\bb\label{lowQ2_gen}
	K(\Phi)&\ge Q_2(\Phi) \ge \sup_{c\in(0,1),\, M\in\N^+,\, k\in\N} R(\Phi,M,c,k)\,,
\ee
where 
\begin{equation}\label{rate_gen}
	R(\Phi,M,c,k)\coloneqq \mathcal{C}(\Phi,c,M)  \frac{\prod_{t=0}^{k-1}P_t}{2^{k}}\mathcal{I}(  \alpha^{(k)}_{00},\alpha^{(k)}_{01},\alpha^{(k)}_{10},\alpha^{(k)}_{11})\,.
\end{equation} 
Fixed $c\in(0,1)$, $M\in\N^+$, and $k\in\N$, the quantities present in~\eqref{rate_gen} are defined as follows. $\mathcal{C}(\Phi,c,M)$ is defined as
\bb\label{formula_gen_norm}
&\mathcal{C}(\Phi,c,M)  \coloneqq   \Tr\left[\mathbb{1}_A\otimes\Pi_M\, \Id_{A}\otimes\Phi(\ketbra{\Psi_{M,c}}_{AA'}) \right]\,,
\ee
where $\Pi_M\coloneqq \ketbra{0}_{B}+\ketbra{M}_{B}$ and the state $\ket{\Psi_{M,c}}_{AA'}$ is defined in~\eqref{initial_state}.
Let us define for all $m,n\in\{0,1\}$ the coefficients $\alpha_{mn}^{(0)}$ as
    \bb \label{formula_gen_alpha0}
        &\alpha_{mn}^{(0)}\coloneqq \frac{\bra{\psi^{(M)}_{mn}}_{AB}\mathbb{1}_A\otimes\Pi_M\, \Id_{A}\otimes\Phi(\ketbra{\Psi_{M,c}})\, \mathbb{1}_A\otimes\Pi_M\ket{\psi^{(M)}_{mn}}_{AB}}{\mathcal{C}(\Phi,c,M)}\,,
    \ee    
where $\ket{\psi^{(M)}_{mn}}_{AB}$ is defined as 
\bb\label{Bell_states_delta}
\ket{\psi^{(M)}_{mn}}_{AB}\coloneqq \frac{1}{\sqrt{2}}\sum_{j=0}^1 (-1)^{mj}\ket{jM}_A\otimes\ket{(j\oplus n)M}_{B}\,.
\ee 
For all $t\in\{0,1,\ldots,k-1\}$ and all $m,n\in\{0,1\}$ the coefficients $\alpha_{mn}^{(t+1)}$ and $P_t$ are defined in the following way:
\begin{itemize}
    \item If $\alpha^{(t)}_{10}<\alpha^{(t)}_{01}$, then
    \bb\label{coeff_p1}
            \alpha^{(t+1)}_{mn}\coloneqq \frac{1}{P_t}\sum_{\substack{m_1,m_2=0\\m_1\oplus m_2=m}}^{1}\alpha_{m_1n}^{(t)}\alpha_{m_2n}^{(t)}\,,
    \ee
    where
    \bb\label{probk_p1}
        P_t \coloneqq \sum_{n=0}^{1}\left(\sum_{m=0}^{1}\alpha^{(t)}_{m n}\right)^2\,.
    \ee
    \item If $\alpha^{(t)}_{10}\ge \alpha^{(t)}_{01}$, then
    \bb\label{coeff_p2}
    \alpha^{(t+1)}_{mn}\coloneqq \frac{1}{P_t}\sum_{\substack{n_1,n_2=0\\n_1\oplus n_2=n}}^{1}\alpha_{mn_1}^{(t)}\alpha_{mn_2}^{(t)}\,,
    \ee    
    where
    \bb\label{probk_p2}
        P_t \coloneqq \sum_{m=0}^{1}\left(\sum_{n=0}^{1}\alpha^{(t)}_{m n}\right)^2\,.
    \ee
\end{itemize}
For all $\alpha_{00},\alpha_{01},\alpha_{10},\alpha_{11}\ge0$ with $\alpha_{00}+\alpha_{01}+\alpha_{10}+\alpha_{11}=1$, the quantity $\mathcal{I}(  \alpha_{00},\alpha_{01},\alpha_{10},\alpha_{11})$ is defined as 
\bb\label{yield_final_protocol}
&\mathcal{I}(  \alpha_{00},\alpha_{01},\alpha_{10},\alpha_{11})\coloneqq \max\left(\,Y(  \alpha_{00},\alpha_{01},\alpha_{10},\alpha_{11}),  Y(  \alpha_{00},\alpha_{10},\alpha_{01},\alpha_{11} ),\, Y(  \alpha_{01},\alpha_{00},\alpha_{10},\alpha_{11})\right)\,,
\ee
where the function $Y$ is defined in~\eqref{improv_hashing}.
\end{thm}
\begin{proof} We introduce a protocol to distribute ebits through the channel $\Phi$, which depends on three parameters: $M\in\N^+$, $c\in(0,1)$,  $k\in\N$.
Our lower bound on $Q_2(\Phi,N_s)$ in~\eqref{lowQ2_genEC} can be obtained by optimising over these parameters the rate of ebits of such a protocol. The lower bound on the other EC two-way capacities follows from~\eqref{relation_2waycapEC}.
The steps of the protocol are the following.




\textbf{-Step 1}: Alice prepares $n_0$ copies of the state $\ket{\Psi_{M,c}}_{AA'}$ in~\eqref{initial_state} and she sends the halves $A'$ to Bob through the channel $\Phi$. Hence, Alice and Bob share $n_0$ copies of the state $\Id_{A}\otimes\Phi(\ketbra{\Psi_{M,c}})$. 



\textbf{-Step 2}: Bob performs the local POVM $\{\Pi_M, \mathbb{1}-\Pi_M\}$ on each pair $\Id_{A}\otimes\Phi(\ketbra{\Psi_{M,c}})$, where $\Pi_M\coloneqq \ketbra{0}+\ketbra{M}$. If Bob finds the outcome which corresponds to $\Pi_M$, then Alice and Bob keep the pair, otherwise they discard it. They keep the pair with probability
    \bb
            \mathcal{C}(\Phi,c,M)&\coloneqq \Tr\left[\mathbb{1}_A\otimes\Pi_M\, \Id_{A}\otimes\Phi(\ketbra{\Psi_{M,c}}) \right]\,.
    \ee
At this point, Alice and Bob shares $\approx n_0\,\mathcal{C}(\Phi,c,M)$ pairs.
    Each of these pairs are in the state $\rho'$ given by 
\bb\label{rho_primo_distil}
        \rho'&= \frac{\mathbb{1}_A\otimes\Pi_M\, \Id_{A}\otimes\Phi(\ketbra{\Psi_{M,c}})\, \mathbb{1}_A\otimes\Pi_M}{\mathcal{C}(\Phi,c,M)} \,.
\ee
Note that the support of $\rho'$ is equal to $\text{Span}\{\ket{0}\otimes\ket{0},\ket{0}\otimes\ket{M},\ket{M}\otimes\ket{0},\ket{M}\otimes\ket{M}\}$.
For simplicity, in the following we will use the notation $\ket{1}\equiv \ket{M}$. This formally corresponds to consider the state $\rho''\coloneqq U_M\otimes U_M\, \rho'\, U_M^\dagger \otimes U_M^\dagger$, which is obtained once both Alice and Bob have applied the unitary 
\bb
U_M\coloneqq \sum_{i\ne\{0, M\}}^\infty\ketbra{i}+\ketbraa{1}{M}+\ketbraa{M}{1}
\ee
on the remaining state $\rho'$. Hence, since the support of $\rho''$ is equal to $\text{Span}\{\ket{0}\otimes\ket{0},\ket{0}\otimes\ket{1},\ket{1}\otimes\ket{0},\ket{1}\otimes\ket{1}\}$, in the following we consider transformations which act on qubit systems.




  
\textbf{-Step 3}:
    For each of the $\approx n_0\,\mathcal{C}(\Phi,c,M)$ pairs, Alice and Bob choose randomly two bits $\mu,\nu\in\{0,1\}$ and they both apply the unitary $\sigma_{\mu\nu}$ defined by 
    \bb\label{def_pauli}
        \sigma_{\mu\nu}\coloneqq\sum_{i=0}^{1} (-1)^{\mu i}\ketbraa{i\oplus \nu}{i}\,
    \ee
    (in terms of the Pauli matrices it holds that $\sigma_{00}=\mathbb{1}_2$, $\sigma_{01}=\sigma_x$,  $\sigma_{10}=\sigma_z$, and $\sigma_{11}=i\sigma_y$).
    Hence, each pair is transformed into the state $\rho_0$ defined by
    \bb
        \rho_0\coloneqq \frac{1}{4}\sum_{\mu,\nu=0}^{1} \left(\sigma_{\mu\nu}\otimes\sigma_{\mu\nu}\right)\,\rho''\,\left(\sigma_{\mu\nu}\otimes\sigma_{\mu\nu}\right)^\dagger\,.
    \ee
    By exploiting the fact that the Bell states defined in~\ref{Bell_states} form an orthonormal basis, one can show that $\rho$ is diagonal in the Bell basis:
    \bb
        \rho_0=\sum_{m,n=0}^{1}\alpha_{mn}^{(0)}\ketbra{\psi_{mn}}\,,
    \ee
    where the coefficients $\alpha_{mn}^{(0)}$ are given by 
    \bb
        \alpha_{mn}^{(0)}&= \bra{\psi_{mn}}\rho''\ket{\psi_{mn}} = \bra{\psi^{(M)}_{mn}}\rho'\ket{\psi^{(M)}_{mn}}\,,
    \ee
    with $\ket{\psi^{(M)}_{mn}}$ being defined in~\ref{Bell_states_delta}.
    
    
\textbf{-Step 4}: 
 Alice and Bob run the following sub-routine, which is a recurrence protocol dubbed \emph{P1-or-P2}~\cite{p1orp2}.
     \begin{itemize}
         \item \textbf{Step 4.0}: Let $t=0$.
         \item \textbf{Step 4.1}: 
         At this point, all the pairs are in the state $\rho_t$.  Alice and Bob collect all the pairs in groups of two pairs. Let $\rho_t^{(A_1B_1)}$ denote the first pair of each group and let $\rho_t^{(A_2B_2)}$ denote the second one. If $\alpha^{(t)}_{10}<\alpha^{(t)}_{01}$, then Alice and Bob apply the bi-local unitary $U_{1}$ defined as
         \bb
	         U_{1}\coloneqq U_{\text{CNOT}}^{(A_1A_2)}\otimes U_{\text{CNOT}}^{(B_1B_2)}\,,
         \ee
         where for all $S=A,B$ the operator $U_{\text{CNOT}}^{(S_1S_2)}$ is the CNOT gate on $S_1$ and $S_2$ with control qubit $S_1$, i.e.
        \bb
          U_{\text{CNOT}}^{(S_1S_2)}\ket{i}_{S_1}\otimes\ket{j}_{S_2}=\ket{i}_{S_1}\otimes\ket{i\oplus j}_{S_2}\,.
         \ee
          Otherwise if $\alpha^{(t)}_{10}\ge\alpha^{(t)}_{01}$, they apply the bi-local unitary $U_{2}$ defined as
          	\bb
         	U_{2}&\coloneqq(H^{(A_1)}\otimes H^{(B_1)})(U_{\text{CNOT}}^{(A_1A_2)}\otimes U_{\text{CNOT}}^{(B_1B_2)}) (H^{(A_1)}\otimes H^{(A_2)}\otimes H^{(B_1)}\otimes H^{(B_2)})\,,
         	\ee
 where for all $S=A_1,A_2,B_1,B_2$ the operator $H^{(S)}$ on $S$ is the Hadamard gate, i.e.
  \bb \label{hadamard}
 H^{(S)}=\frac{1}{\sqrt{2}}\sum_{m,n=0}^{1}(-1)^{mn}\ketbraa{n}{m}_S\,.
 \ee
At this point, the state of $A_1A_2B_1B_2$ is 
\bb
            \rho_t^{(A_1A_2B_1B_2)}\coloneqq U_p\,\rho_t^{(A_1B_2)}\otimes\rho_k^{(A_2B_2)}\, U_p^\dagger\,.
\ee
with $p=1$ if $\alpha^{(t)}_{10}<\alpha^{(t)}_{01}$, and $p=2$ otherwise.
         \item \textbf{Step 4.2}:  Alice and Bob measure the pair $A_2B_2$ of each group with respect to the local POVM $\{M_{i,j}\}_{i,j\in\{0,1\}}$ with $M_{i,j}\coloneqq\ketbra{i}_{A_2}\otimes\ketbra{j}_{B_2}$ for all $i,j\in\{0,1\}$. Then they discard the pair $A_2B_2$. 
         They discard also the pair $A_1B_1$ if the outcome of the previous measurement corresponds to $M_{i,j}$ with $i\ne j$. The probability that a pair $A_1B_1$ is not discarded is given by 
          \bb
         P_t &\coloneqq \sum_{i=0}^{1}\bra{i}_{A_2}\bra{i}_{B_2}\Tr_{A_1B_1}\left[\rho_t^{(A_1A_2B_1B_2)}\right]\ket{i}_{A_2}\ket{i}_{B_2}\,.
         \ee
         By using that for all $k_1,k_2,j_1,j_2\in\{0,1\}$ it holds that
         \bb\label{formula_bi_cnot}
         &U_{\text{CNOT}}^{(A_1A_2)}\otimes U_{\text{CNOT}}^{(B_1B_2)}\ket{\psi_{k_1j_1}}_{A_1B_1}\otimes\ket{\psi_{k_2j_2}}_{A_2B_2} =\ket{\psi_{k_1\oplus k_2,\,j_1}}_{A_1B_1}\otimes\ket{\psi_{k_2,\,j_1\oplus j_2}}_{A_2B_2}
         \ee
         and that 
          \bb\label{formula_bi_hadamard}
         H^{(A)}\otimes H^{(B)}\ket{\psi_{k_1j_1}}_{AB}=(-1)^{k_1j_1}\ket{\psi_{j_1k_1}}_{AB}\,,
         \ee
		one can show that $P_t$ can be expressed as in~\eqref{probk_p1}  if $\alpha^{(t)}_{10}<\alpha^{(t)}_{01}$, and as in~\eqref{probk_p2} otherwise.
        At this point, the number of remaining pairs is 
         \begin{equation}
         	\approx n_0\,\mathcal{C}(\Phi,c,M) \frac{1}{2^{t+1}} \prod_{m=0}^{t}P_m\,
         \end{equation}
     	and each of these is in the state $\rho_{t+1}$ given by
        \bb
\rho_{t+1}&=\frac{1}{2}\sum_{i=0}^1\frac{\bra{i}_{A_2}\bra{i}_{B_2}\rho_t^{(A_1A_2B_1B_2)}\ket{i}_{A_2}\ket{i}_{B_2}}{\Tr_{A_1B_1}\left[ \bra{i}_{A_2}\bra{i}_{B_2}\rho_t^{(A_1A_2B_1B_2)}\ket{i}_{A_2}\ket{i}_{B_2} \right]}\,.
\ee
By using~\eqref{formula_bi_cnot} and~\eqref{formula_bi_hadamard}, one can show that
\bb
\rho_{t+1}=\sum_{m,n=0}^{1}\alpha^{(t+1)}_{mn}\ketbra{\psi_{mn}}\,,
\ee
where the coefficients $\alpha^{(t+1)}_{mn}$ are given by~\eqref{coeff_p1} if $\alpha^{(t)}_{10}<\alpha^{(t)}_{01}$, and by~\eqref{coeff_p2} otherwise.

         \item \textbf{Step 4.3}: Let $t=t+1$.
         \item \textbf{Step 4.4}: If the condition $t<k$ is satisfied, then go back to Step 4.1.
     \end{itemize}
 	Before introducing Step 5, let us recall that if the improved hashing protocol of~\cite{Improvement-Hashing} is applied on states of the form $\rho= \sum_{ij=0}^1\alpha_{ij}\ketbra{\psi_{ij}}$ then it can generate ebits with a yield $Y(\alpha_{00},\alpha_{01},\alpha_{10},\alpha_{11})$ given by~\eqref{improv_hashing}.
   	Note that such a yield is not invariant under permutations of the variables $\alpha_{00},\alpha_{01},\alpha_{10},\alpha_{11}$. Hence, one may achieve a yield which is larger than $Y(\alpha_{00},\alpha_{01},\alpha_{10},\alpha_{11})$ by applying suitable bi-local unitaries, which suitably permutes the Bell states, just before running the improved hashing protocol. Since the yield function $Y(\alpha_{00},\alpha_{01},\alpha_{10},\alpha_{11})$ satisfies
   	\bb
   	Y(\alpha_{00},\alpha_{01},\alpha_{10},\alpha_{11})&=Y(\alpha_{10},\alpha_{01},\alpha_{00},\alpha_{11})\,,\\
   	Y(\alpha_{00},\alpha_{01},\alpha_{10},\alpha_{11})&=Y(\alpha_{00},\alpha_{11},\alpha_{10},\alpha_{01})\,,\\
   	Y(\alpha_{00},\alpha_{01},\alpha_{10},\alpha_{11})&=Y(\alpha_{01},\alpha_{00},\alpha_{11},\alpha_{10})\,,\\
   	\ee
   	then by permuting the four variables $\alpha_{ij}$ it is possible to obtain at most three different values of the rate function, which are: $Y(  \alpha_{00},\alpha_{01},\alpha_{10},\alpha_{11})$, $Y(  \alpha_{00},\alpha_{10},\alpha_{01},\alpha_{11})$, and $Y(  \alpha_{01},\alpha_{00},\alpha_{10},\alpha_{11})$.
   	Let us define the function $\mathcal{I}$ as
   	\bb
   		&\mathcal{I}(  \alpha_{00},\alpha_{01},\alpha_{10},\alpha_{11})\coloneqq \max\left(\,Y(  \alpha_{00},\alpha_{01},\alpha_{10},\alpha_{11}),   Y(  \alpha_{00},\alpha_{10},\alpha_{01},\alpha_{11} ),\, Y(  \alpha_{01},\alpha_{00},\alpha_{10},\alpha_{11})\right)\,.
   	\ee
Note that at the beginning of Step 5, the number of remaining pairs is
\begin{equation}
	\approx n_0\,\mathcal{C}(\Phi,c,M) \frac{1}{2^{k}} \prod_{t=0}^{k-1}P_t\,
\end{equation}
and each of these is in $\rho_{k}=\sum_{m,n=0}^{1}\alpha^{(k)}_{mn}\ketbra{\psi_{mn}}$.








 \textbf{-Step 5}:  If $\mathcal{I}(  \alpha^{(k)}_{00},\alpha^{(k)}_{01},\alpha^{(k)}_{10},\alpha^{(k)}_{11})= Y(  \alpha^{(k)}_{00},\alpha^{(k)}_{10},\alpha^{(k)}_{01},\alpha^{(k)}_{11})$, then both Alice and Bob apply the Hadamard gate defined by~\eqref{hadamard}. Therefore, in this case, the state of each of pairs becomes           \bb
\left(H\otimes H\right)\rho_{k} \left(H\otimes H\right)^\dagger&=\alpha^{(k)}_{00}\ketbra{\psi_{00}}+\alpha^{(k)}_{10}\ketbra{\psi_{01}} +\alpha^{(k)}_{01}\ketbra{\psi_{10}}+\alpha^{(k)}_{11}\ketbra{\psi_{11}}\,,
\ee
where we have exploited~\eqref{formula_bi_hadamard}.
If $\mathcal{I}(  \alpha^{(k)}_{00},\alpha^{(k)}_{01},\alpha^{(k)}_{10},\alpha^{(k)}_{11})= Y(  \alpha^{(k)}_{01},\alpha^{(k)}_{00},\alpha^{(k)}_{10},\alpha^{(k)}_{11})$, then both Alice and Bob apply $B_x\coloneqq \frac{\mathbb{1}_2-i\sigma_{01}}{\sqrt{2}}$, where $\sigma_{01}$ is defined by~\eqref{def_pauli}, and hence the state becomes
\bb
\left(B_x\otimes B_x\right)\rho_{k} \left(B_x\otimes B_x\right)^\dagger&=\alpha^{(k)}_{01}\ketbra{\psi_{00}}+\alpha^{(k)}_{00}\ketbra{\psi_{01}} +\alpha^{(k)}_{10}\ketbra{\psi_{10}}+\alpha^{(k)}_{11}\ketbra{\psi_{11}}\,.
\ee




\textbf{-Step 6}: Alice and Bob run the improved hashing protocol of~\cite{Improvement-Hashing}, which can achieve the yield $\mathcal{I}(  \alpha^{(k)}_{00},\alpha^{(k)}_{01},\alpha^{(k)}_{10},\alpha^{(k)}_{11})$. Hence, in the end, Alice and Bob can generate a number of ebits equal to 
\begin{equation}
	\approx n_0\,\mathcal{C}(\Phi,c,M)  \frac{\prod_{t=0}^{k-1}P_t}{2^{k}}\mathcal{I}(  \alpha^{(k)}_{00},\alpha^{(k)}_{01},\alpha^{(k)}_{10},\alpha^{(k)}_{11})\,.
\end{equation}
Since the channel $\Phi$ is used $n_0$ times (during Step 1) to send the $n_0$ halves of the state $\ket{\Psi_{M,c}}_{AA'}$, the rate of distributed ebits of the presented protocol is 
\begin{equation}\label{rate_gen2}
\mathcal{C}(\Phi,c,M)  \frac{\prod_{t=0}^{k-1}P_t}{2^{k}}\mathcal{I}(  \alpha^{(k)}_{00},\alpha^{(k)}_{01},\alpha^{(k)}_{10},\alpha^{(k)}_{11})\,.
\end{equation}    
Since the local mean photon number of $\ket{\Psi_{M,c}}_{AA'}$ is $(1-c^2)M$, the rate in~\eqref{rate_gen2} is a lower bound on the energy-constrained two-way quantum capacity $Q_2(\Phi,N_s)$ for all $M\in\N^+$, $c\in(0,1)$, $k\in\N$ such that $(1-c^2)M\le N_s$. The optimisation over these parameters of the rate in~\eqref{rate_gen2} leads to the lower bound on $Q_2(\Phi,N_s)$ in~\eqref{lowQ2_genEC}. In addition, since $K(\Phi,N_s)\ge Q_2(\Phi,N_s)$ thanks to~\eqref{relation_2waycapEC}, we have proved~\eqref{lowQ2_genEC}. By taking the limit $N_s\rightarrow\infty$ of~\eqref{lowQ2_genEC}, the lower bound on the unconstrained two-way capacities in~\eqref{lowQ2_gen} is also proved.
\end{proof}
%In order to apply Theorem~\ref{th_lower_Q2} to the thermal attenuator, thermal amplifier, and additive Gaussian noise, we need to calculate the quantities $\mathcal{C}(\Phi,c,M)$ in~\ref{formula_gen_norm} and $\alpha^{(0)}_{mn}$ in~\ref{formula_gen_alpha0} for each of these channels. This requires to calculate the action of these Gaussian channels on a non-Gaussian state, which is cumbersome in general. In~\cite{Die-Hard-2-PRA} we introduced a method, dubbed  ``master equation trick", which allowed us to overcome this difficulty in the case of the thermal attenuator. In the forthcoming Theorem~\ref{gen_master_eq_trick} we generalise this trick to the thermal amplifier and additive Gaussian noise.
In the forthcoming Theorem~\ref{th_comp_Q2} we apply Theorem~\ref{th_lower_Q2} to the composition $\mathcal{N}_{g,\lambda}\coloneqq\Phi_{g,0}\circ \mathcal{E}_{\lambda,0}$ between pure amplifier channel $\Phi_{g,0}$ and pure loss channel $\mathcal{E}_{\lambda,0}$.
\begin{thm}\label{th_comp_Q2}
 Let $g\ge1$, $\lambda\in[0,1]$, and $N_s\ge0$. The EC two-way capacities $Q_2(\mathcal{N}_{g,\lambda},N_s)$ and $K(\mathcal{N}_{g,\lambda},N_s)$ of the composition $\mathcal{N}_{g,\lambda}\coloneqq\Phi_{g,0}\circ \mathcal{E}_{\lambda,0}$ between pure amplifier channel and pure loss channel satisfy the following lower bound
\bb\label{lowQ2_deltaEC_comp}
	K(\mathcal{N}_{g,\lambda},N_s)& \ge  Q_2(\mathcal{N}_{g,\lambda},N_s)\ge \sup_{\substack{c\in(0,1),\, M\in\N^+,\, k\in\N\\ (1-c^2)M\le N_s}} \mathcal{R}(g,\lambda,M,c,k)\,,
\ee
and, in particular, the unconstrained two-way capacities satisfy
\bb\label{lowQ2_delta_comp}
	K(\mathcal{N}_{g,\lambda})& \ge  Q_2(\mathcal{N}_{g,\lambda})\ge \sup_{c\in(0,1),\, M\in\N^+,\, k\in\N} \mathcal{R}(g,\lambda,M,c,k)\,,
\ee
where
\bb\label{def_mathcal_R}
    \mathcal{R}(g,\lambda,M,c,k)\coloneqq R(\mathcal{N}_{g,\lambda},M,c,k)\,,
\ee
with the quantity $R(\mathcal{N}_{g,\lambda},M,c,k)$ being defined in Theorem~\ref{th_lower_Q2}. The quantities $\mathcal{C}(\mathcal{N}_{g,\lambda},c,M)$ and $\alpha_{mn}^{(0)}$, which appear in the definition of $R(\mathcal{N}_{g,\lambda},M,c,k)$ in Theorem~\ref{th_lower_Q2}, can be expressed as
\bb\label{formula_N_alpha_0_comp}
\mathcal{C}(\mathcal{N}_{g,\lambda},c,M) &\coloneqq \sum_{n,l=0}^{1} c_n^2\, f_{M n,M n,M l}(g,\lambda)\,,\\        \alpha_{mn}^{(0)}&\coloneqq\frac{1}{2\mathcal{C}(\mathcal{N}_{g,\lambda},c,M)}\sum_{x,y=0}^{1}\sum_{l=\max(y-x,0)}^{1+\min(y-x,0)} \delta_{x\oplus n,l+x-y}\,\delta_{y\oplus n,l}\,  (-1)^{m(x+y)} c_x c_y\, f_{ M x,M y,M l}(g,\lambda)\,,
    \ee
where $c_0\coloneqq c$, $c_1\coloneqq \sqrt{1-c^2}$, $f_{n,i,l}(g,\lambda)$ is defined in~\eqref{def_f_comp}, and $\delta_{x,y}$ denotes the Kronecker delta.
\end{thm}
\begin{proof} 
\eqref{lowQ2_deltaEC_comp} and~\eqref{lowQ2_delta_comp} follows by applying Theorem~\ref{th_lower_Q2} to $\mathcal{N}_{g,\lambda}$. We only need to show the expressions of $\mathcal{C}(\mathcal{N}_{g,\lambda},c,M)$ and $\alpha_{mn}^{(0)}$ in~\eqref{formula_N_alpha_0_comp}. In this proof we use the notation introduced in the statement of Theorem~\ref{th_lower_Q2}. By using~\eqref{action_comp_chan}, we deduce that
\bb\label{formula_choi}
&\Id_{A}\otimes\mathcal{N}_{g,\lambda}(\ketbra{\Psi_{M,c}})=\sum_{n,i=0}^{1}\sum_{l=M\max(i-n,0)}^\infty c_n c_i f_{M n,M i,l}(g,\lambda)\ketbraa{M n}{M i}_{A}\otimes\ketbraa{l+M(n-i)}{l}_B.
\ee
Consequently, it holds that
\bb\label{step_proof}
        &\mathbb{1}_A\otimes\Pi_M\, \Id_{A}\otimes\mathcal{N}_{g,\lambda}(\ketbra{\Psi_{M,c}})\, \mathbb{1}_A\otimes\Pi_M=\sum_{n,i=0}^{1}\sum_{l=\max(i-n,0)}^{1+\min(i-n,0)} c_n c_i  f_{M n,M i,M l}(g,\lambda)\ketbraa{M n}{M i}_{A}\otimes\ketbraa{M(l+n-i)}{M l}_B.
\ee
By inserting this into the definition of $\mathcal{C}(\mathcal{N}_{g,\lambda},c,M)$ in~\eqref{formula_gen_norm} and of $\alpha_{mn}^{(0)}$ in~\eqref{formula_gen_alpha0}, one obtains the expressions in~\eqref{formula_N_alpha_0_comp}.
\end{proof}



\subsection{Remarks}
Let us consider the entanglement distribution protocol shown in the proof of Theorem~\ref{th_lower_Q2} applied to the the composition $\mathcal{N}_{g,\lambda}\coloneqq\Phi_{g,0}\circ \mathcal{E}_{\lambda,0}$ between pure amplifier channel $\Phi_{g,0}$ and pure loss channel $\mathcal{E}_{\lambda,0}$. After completing Step 2 of this protocol, the entanglement distribution process is reduced to an entanglement distillation protocol on the two-qubit state reported in~\eqref{rho_primo_distil}. We will denote this two-qubit state as $\rho^{(g,\lambda,M,c)}_{AB}$, where $M\in\N^+$ and $c\in(0,1)$ correspond to the constants appearing in the state in~\eqref{initial_state} that Alice produces during Step 1. The natural question that arises is: "Under what conditions is $\rho^{(g,\lambda,M,c)}_{AB}$ distillable?" In Remark~\ref{remark_alternative_proof} we answer this question.
\begin{remark}\label{remark_alternative_proof}
 $\rho^{(g,\lambda,M,c)}_{AB}$ is distillable if and only if $\lambda$ and $g$ satisfy the inequality $(1-\lambda) g<1$, meaning that $\mathcal{N}_{g,\lambda}$ is not entanglement breaking. This provides an alternative proof of Theorem~\ref{th1}.
\end{remark}
\begin{proof}
    By exploiting~\eqref{step_proof}, for all $g>1,\lambda\in(0,1),M\in\N^+,c\in(0,1)$ the state in~\eqref{rho_primo_distil} can be expressed as
\bb
    \rho^{(g,\lambda,M,c)}_{AB}\coloneqq\frac{ \sum_{n,i=0}^{1}\sum_{l=\max(i-n,0)}^{1+\min(i-n,0)} c_n c_i  f_{M n,M i,M l}(g,\lambda)\ketbraa{M n}{M i}_{A}\otimes\ketbraa{M(l+n-i)}{M l}_B }{ \sum_{n,l=0}^{1} c_n^2\, f_{M n,M n,M l}(g,\lambda)}\,,
\ee
where $c_0\coloneqq c$, $c_1\coloneqq \sqrt{1-c^2}$, and $f_{n,i,l}(g,\lambda)$ is defined in~\eqref{def_f_comp}. Consequently, it holds that
\bb\label{explicit_state_f}
    \rho^{(g,\lambda,M,c)}_{AB}=\frac{ 1}{ \sum_{n,l=0}^{1} c_n^2\, f_{M n,M n,M l}(g,\lambda)}\big[&c^2f_{0,0,0}(g,\lambda)\ketbraa{0}{0}_{A}\otimes\ketbraa{0}{0}_B\\&+c^2f_{0,0,M}(g,\lambda)\ketbraa{0}{0}_{A}\otimes\ketbraa{M}{M}_B\\&+c\sqrt{1-c^2}f_{0,M,M}(g,\lambda)\ketbraa{0}{M}_{A}\otimes\ketbraa{0}{M }_B\\&+c\sqrt{1-c^2}f_{M,0,0}(g,\lambda)\ketbraa{M }{0}_{A}\otimes\ketbraa{M}{0}_B\\&+(1-c^2)f_{M,M,0}(g,\lambda)\ketbraa{M}{M}_{A}\otimes\ketbraa{0}{0}_B\\&+(1-c^2)f_{M,M,M}(g,\lambda)\ketbraa{M}{M}_{A}\otimes\ketbraa{M}{M}_B\big]\,,
\ee
Hence, the matrix associated with the partial transpose on $B$ of $\rho^{(g,\lambda,M,c)}_{AB}$, written with respect the basis $\{\ket{0}_A\otimes\ket{0}_B, \ket{0}_A\otimes\ket{M}_B, \ket{M}_A\otimes\ket{0}_B, \ket{M}_A\otimes\ket{M}_B\}$, is 
\bb\nonumber
&\frac{ 1}{ \sum_{n,l=0}^{1} c_n^2\, f_{M n,M n,M l}(g,\lambda)}\left(\begin{matrix} c^2f_{0,0,0}(g,\lambda)\quad & 0& 0& 0\\ 0 & c^2f_{0,0,M}(g,\lambda)\quad & c\sqrt{1-c^2}f_{0,M,M}(g,\lambda)\quad& 0\\
0 & c\sqrt{1-c^2}f_{M,0,0}(g,\lambda)\quad & (1-c^2)f_{M,M,0}(g,\lambda)\quad & 0\\ 0 & 0 & 0 &(1-c^2)f_{M,M,M}(g,\lambda)\quad \end{matrix}\right) \,.
\ee 
It follows that $\rho^{(g,\lambda,M,c)}_{AB}$ is not PPT if and only if 
\bb\label{condition_det}
    f_{M,0,0}(g,\lambda)\,f_{M,M,0}(g,\lambda)< f_{0,M,M}(g,\lambda)\, f_{0,0,M}(g,\lambda)\,.
\ee
The definition of $f_{\cdot,\cdot,\cdot}(g,\lambda)$ in~\eqref{def_f_comp} yields
\bb\label{formulae_f_g_lam}
    f_{0,0,M}(g,\lambda)&=\frac{(g-1)^M}{g^{1+M}}\,,\\
    f_{M,M,0}(g,\lambda)&=\frac{(1-\lambda)^M}{g}\,,\\
    f_{0,M,M}(g,\lambda)&=f_{M,0,0}(g,\lambda)=\frac{\lambda^{\frac{M}{2}}}{g^{1+\frac{M}{2}}}\,.
\ee
Consequently,~\eqref{condition_det} establishes that $\rho^{(g,\lambda,M,c)}_{AB}$ is not PPT if and only if $(1-\lambda)g<1$, independentely of $c$ and $M$. The fact that any two-qubit state is distillable if and only if it is not PPT~\cite{2-qubit-distillation} implies that $\rho^{(g,\lambda,M,c)}_{AB}$ is distillable if and only if $(1-\lambda)g<1$ for all $c\in(0,1)$ and all $M\in\N^+$. 

Let us now show that this fact constitutes an alternative proof of Theorem~\ref{th1}, i.e.~let us show that the energy-constrained two-way capacities $Q_2(\mathcal{N}_{g,\lambda},N_s)$ and $K(\mathcal{N}_{g,\lambda},N_s)$ are strictly positive if and only if $(1-\lambda)g< 1$, i.e.~if and only if $\mathcal{N}_{g,\lambda}$ is not entanglement breaking. The entanglement distribution protocol's Steps S1 and S2 imply that for all $N_s\ge 0$ it holds that $Q_2(\mathcal{N}_{g,\lambda},N_s)\ge E_d\left(  \rho^{(g,\lambda,M,c)}_{AB} \right)$, for any $c\in(0,1)$ and $M\in\N^+$ satisfying $(1-c^2)M\le N_s$. Here, $E_d(\cdot)$ denotes the distillable entanglement. As we have proved above, if $(1-\lambda)g<1$ then the state $\rho^{(g,\lambda,M,c)}_{AB}$ is distillable, i.e.~$E_d(\rho^{(g,\lambda,M,c)}_{AB})>0$. This implies that if $(1-\lambda)g<1$, then the energy-constrained two-way capacities of $\mathcal{N}_{g,\lambda}$ are strictly positive, i.e.~$K(\mathcal{N}_{g,\lambda},N_s)\ge Q_2(\mathcal{N}_{g,\lambda},N_s)>0$. 
Conversely, by exploiting Theorem~\ref{lemma_eb_comp} and the fact that any entanglement-breaking channel has vanishing two-way capacities, it follows that if $(1-\lambda)g\ge1$ then $K(\mathcal{N}_{g,\lambda})= Q_2(\mathcal{N}_{g,\lambda})=0$ and hence $K(\mathcal{N}_{g,\lambda},N_s)= Q_2(\mathcal{N}_{g,\lambda},N_s)=0$.
\end{proof}

{Remark~\ref{remark_alternative_proof} ensures that if the channel $\mathcal{N}_{g,\lambda}$ is not entanglement breaking, then the state $\rho^{(g,\lambda,M,c)}_{AB}$ obtained at the end of Step~2 is distillable for any $M\in\N^+$ and $c\in(0,1)$. We now turn our attention to the state, denoted as $\sigma^{(g,\lambda,M,c)}_{AB}$, which is obtained at the end of Step 3 through Pauli-based twirling of $\rho^{(g,\lambda,M,c)}_{AB}$. It is possible for this operation to map distillable states to undistillable states, so we ask the question: "Under what conditions is $\sigma^{(g,\lambda,M,c)}_{AB}$ distillable?" In Remark~\ref{remark_step3} we will demonstrate that for any $\lambda\in(0,1)$ and $g>1$, if $\mathcal{N}_{g,\lambda}$ is not entanglement breaking, then for all $M\in\N^+$ the state $\sigma^{(g,\lambda,M,\bar{c})}_{AB}$ is distillable, where $\bar{c}\coloneqq\frac{1}{\sqrt{1+(g-1)^M}}$. This means that Alice and Bob can choose the value of $c$ appropriately such that the Pauli-based twirling does not affect the distillability of the shared state.}
{\begin{remark}\label{remark_step3}
Let $M\in\N^+$, $\lambda\in(0,1)$, and $g>1$ with $(1-\lambda)g<1$ (meaning that $\mathcal{N}_{g,\lambda}$ is not entanglement breaking). Then, the state $\sigma^{(g,\lambda,M,\bar{c})}_{AB}$ is distillable, where $\bar{c}\coloneqq\frac{1}{\sqrt{1+(g-1)^M}}$.
    \end{remark}
    \begin{proof}
    After applying the Pauli-based twirling on the state $\rho^{(g,\lambda,M,\bar{c})}_{AB}$, the resulting state $\sigma^{(g,\lambda,M,\bar{c})}_{AB}$ is transformed into a Bell-diagonal form, that is
    \bb
        \sigma^{(g,\lambda,M,\bar{c})}_{AB}=\sum_{i,j=0}^1 p_{ij} \ketbraa{\psi^{(M)}_{ij}}{\psi^{(M)}_{ij}}_{AB}\,,
    \ee
    where $p_{ij}\coloneqq \bra{\psi^{(M)}_{ij}}\rho^{(g,\lambda,M,\bar{c})}_{AB}\ket{\psi^{(M)}_{ij}}$ and $\{\ket{\psi^{(M)}_{ij}}_{AB}\}_{i,j\in\{0,1\}}$ are the Bell states defined in~\eqref{Bell_states_delta}. In particular, it holds that
    \bb\label{prob_bell_diag_state}
        p_{00}+p_{10}&=\bra{0}_{A}\bra{0}_{B}\rho^{(g,\lambda,M,c)}_{AB}\ket{0}_{A}\ket{0}_{B}+\bra{M}_{A}\bra{M}_{B}\rho^{(g,\lambda,M,c)}_{AB}\ket{M}_{A}\ket{M}_{B}\,,\\
        p_{01}-p_{11}&=2\bra{0}_{A}\bra{0}_{B}\rho^{(g,\lambda,M,c)}_{AB}\ket{M}_{A}\ket{M}_{B}\,,\\
        p_{01}+p_{11}&=\bra{0}_{A}\bra{M}_{B}\rho^{(g,\lambda,M,c)}_{AB}\ket{0}_{A}\ket{M}_{B}+\bra{M}_{A}\bra{0}_{B}\rho^{(g,\lambda,M,c)}_{AB}\ket{M}_{A}\ket{0}_{B}\,,\\
        p_{00}-p_{10}&=2\bra{0}_{A}\bra{0}_{B}\rho^{(g,\lambda,M,c)}_{AB}\ket{M}_{A}\ket{M}_{B}\,.
    \ee
    Lemma~\ref{lemma_bell_diag} guarantees that if $p_{01}+p_{11}-|p_{00}-p_{10}|<0$ then the state $\sigma^{(g,\lambda,M,\bar{c})}_{AB}$ is distillable.  By using~\eqref{explicit_state_f} and~\eqref{prob_bell_diag_state}, the condition $p_{01}+p_{11}-|p_{00}-p_{10}|<0$ is satisfied if and only if
    \bb
        \bar{c}^2f_{0,0,M}(g,\lambda)+(1-\bar{c}^2)f_{M,M,0}(g,\lambda)-2\bar{c}\sqrt{1-\bar{c}^2}f_{M,0,0}(g,\lambda)<0\,,
    \ee
    that is
    \bb
        \bar{c}^2(g-1)^M+(1-\bar{c}^2)(1-\lambda)^Mg^M-2\bar{c}\sqrt{1-\bar{c}^2}(\lambda g)^{M/2}<0\,,
    \ee
    where we have exploited~\eqref{formulae_f_g_lam}. By hypothesis, the channel $\mathcal{N}_{g,\lambda}$ is entanglement breaking and hence $(1-\lambda)g<1$, as established by Lemma~\ref{lemma_comp_bos}. Consequently, for all $g>1$ and $\lambda\in(0,1)$ it holds that 
    \bb
        \bar{c}^2(g-1)^M+(1-\bar{c}^2)(1-\lambda)^Mg^M-2\bar{c}\sqrt{1-\bar{c}^2}(\lambda g)^{M/2}&<\bar{c}^2(g-1)^M+(1-\bar{c}^2)-2\bar{c}\sqrt{1-\bar{c}^2}(g-1)^{M/2}\\&=\left( \bar{c}(g-1)^{M/2}-\sqrt{1-\bar{c}^2} \right)^2=0\,,
    \ee
    where we have used that $\bar{c}\coloneqq\frac{1}{\sqrt{1+(g-1)^M}}$. Hence, for all $M\in \N^+$, $\lambda\in(0,1)$, and $g>1$ with $(1-\lambda)g<1$, it holds that $\sigma^{(g,\lambda,M,\bar{c})}_{AB}$ is distillable.
    \end{proof}

\begin{lemma}\label{lemma_bell_diag}
    Let $\{\ket{\psi_{ij}}\}_{i,j\in\{0,1\}}$ be the Bell states defined in~\eqref{Bell_states}. A convex combination of Bell states $\rho_{AB}=\sum_{i,j=0}^1p_{ij}\ketbraa{\psi_{ij}}{\psi_{ij}}$ is distillable if and only if $p_{00}+p_{10}<|p_{01}-p_{10}|$ or $p_{01}+p_{11}<|p_{00}-p_{10}|$.
\end{lemma}
\begin{proof}
    The matrix associated with $\rho_{AB}=\sum_{i,j=0}^1p_{ij}\ketbraa{\psi_{ij}}{\psi_{ij}}$, written with respect the basis $\{\ket{0}_A\otimes\ket{0}_B, \ket{0}_A\otimes\ket{1}_B, \ket{1}_A\otimes\ket{0}_B, \ket{1}_A\otimes\ket{1}_B\}$, is 
\bb\nonumber
&\frac{ 1}{ 2}\left(\begin{matrix} p_{00}+p_{10}\quad & 0& 0& p_{00}-p_{10}\\ 0 & p_{10}+p_{11}\quad & p_{10}-p_{11}\quad& 0\\
0 & p_{10}-p_{11}\quad & p_{10}+p_{11}\quad & 0\\ p_{00}-p_{10} & 0 & 0 &p_{00}+p_{10}\quad \end{matrix}\right) \,.
\ee 
Its partial transpose on $B$ is
\bb\nonumber
&\frac{ 1}{ 2}\left(\begin{matrix} p_{00}+p_{10}\quad & 0& 0& p_{10}-p_{11}\\ 0 & p_{10}+p_{11}\quad & p_{00}-p_{10}\quad& 0\\
0 & p_{00}-p_{10}\quad & p_{10}+p_{11}\quad & 0\\ p_{10}-p_{11} & 0 & 0 &p_{00}+p_{10}\quad \end{matrix}\right) \,.
\ee 
Hence, the state $\rho_{AB}$ is PPT if and only if $p_{00}+p_{10}\ge|p_{01}-p_{10}|$ and $p_{01}+p_{11}\ge|p_{00}-p_{10}|$. Consequently, the fact that any two-qubit state is distillable if and only if it is not PPT~\cite{2-qubit-distillation} implies the validity of the thesis.
\end{proof}


}



\section{Multi-rail strategies}\label{multiplerail_section}

In this section we introduce an additional protocol for distributing ebits across the piBGC $\mathcal{N}_{g,\lambda}$ by combining and optimising the multi-rail protocol introduced in~\cite{Winnel} and the qudit P1-or-P2 protocol introduced in~\cite{p1orp2}. To begin, we will establish some notation and we will prove a useful lemma. For any $K\in\N$ with $K\geq 2$ and any $\textbf{n}\coloneqq(n_1,\ldots,n_K)\in\N^K$, we denote as $\ket{\textbf{n}}_{A_1\ldots A_K}$ the following $K$-mode Fock state with total photon number equal to $\|\textbf{n}\|_1$:
\bb
        \ket{\textbf{n}}_{A_1\ldots A_K}\coloneqq \ket{n_1}_{A_1}\otimes\ket{n_2}_{A_2}\otimes\ldots\otimes\ket{n_K}_{A_K}\,,
    \ee
    where we have used the notation $\|\textbf{n}\|_1\coloneqq \sum_{j=1}^K n_j$. For any $N,K\in\N^+$ with $K\ge2$, let us order the set 
    \bb
    \{\ket{\textbf{n}}_{A_1\ldots A_K}:\,\textbf{n}\in\N^K\,,  \|\textbf{n}\|_1=N\}
    \ee
    according to the restricted lexicographic ordering. More formally, the relation $\preceq$ is defined as
    \bb
\ket{\textbf{n}}_{A_1\ldots A_K}\preceq\ket{\textbf{m}}_{A_1\ldots A_K}   \Longleftrightarrow \sum_{j=1}^{K} n_{j}\, (N+1)^j<\sum_{j=1}^{K} m_{j} \,(N+1)^j\,.
    \ee
     The set has $\binom{N+K-1}{N}$ elements, and for all $n=0,1,\ldots,\binom{N+K-1}{N}-1$, we define the state $\ket{\phi^{(N)}_n}_{A_1\ldots A_K}$ as the $n$th element of the ordered set. For example, if $N=2$ and $K=3$, we have that
    \bb
        \ket{\phi^{(2)}_0}_{A_1 A_2 A_3}&\coloneqq \ket{0}_{A_1}\otimes\ket{0}_{A_2}\otimes\ket{2}_{A_3}\,,\\
        \ket{\phi^{(2)}_1}_{A_1 A_2 A_3}&\coloneqq \ket{0}_{A_1}\otimes\ket{1}_{A_2}\otimes\ket{1}_{A_3}\,,\\
        \ket{\phi^{(2)}_2}_{A_1 A_2 A_3}&\coloneqq \ket{0}_{A_1}\otimes\ket{2}_{A_2}\otimes\ket{0}_{A_3}\,,\\
        \ket{\phi^{(2)}_3}_{A_1 A_2 A_3}&\coloneqq \ket{1}_{A_1}\otimes\ket{0}_{A_2}\otimes\ket{1}_{A_3}\,,\\
        \ket{\phi^{(2)}_4}_{A_1 A_2 A_3}&\coloneqq \ket{1}_{A_1}\otimes\ket{1}_{A_2}\otimes\ket{0}_{A_3}\,,\\
        \ket{\phi^{(2)}_5}_{A_1 A_2 A_3}&\coloneqq \ket{2}_{A_1}\otimes\ket{0}_{A_2}\otimes\ket{0}_{A_3}\,.
    \ee
    In addition, for all $N,K\in\N^+$ with $K\ge2$ let us define the following state of $K+K$ modes $A_1,\ldots,A_k,A_1',\ldots,A_K'$:
    \bb
        \ket{\Psi_{N,K}}_{A_1\ldots A_K, A'_1,\ldots,A'_K}&\coloneqq \frac{1}{\sqrt{\binom{N+K-1}{N}}}\sum_{n=0}^{\binom{N+K-1}{N}-1}\ket{\phi^{(N)}_n}_{A_1\ldots A_k}\otimes\ket{\phi^{(N)}_n}_{A'_1\ldots A'_k}\\&=\frac{1}{\sqrt{\binom{N+K-1}{N}}} \sum_{\substack{\textbf{n}\in\N^K\\ \|\textbf{n}\|_1=N}} \ket{\textbf{n}}_{A_1\ldots A_K}\otimes\ket{\textbf{n}}_{A'_1\ldots A'_K}\,,
    \ee
    which is a $\binom{N+K-1}{N}$-dimensional maximally entangled state 
    that corresponds to the subspace of the Hilbert space of $K$ modes with total photon number equal to $N$. Moreover, let us define for all $K,F\in\N$ the projector $\Pi^{(K)}_F$ onto the subspace of $K$ modes $A_1,\ldots,A_K$ whose total photon number equals $F$, i.e.
    \bb\label{PROJECTOR_N_photon}
        \Pi^{(K)}_F\coloneqq \sum_{\substack{\textbf{m}\in\N^K\\ \|\textbf{m}\|_1=F}} \ketbra{\textbf{m}}_{A_1\ldots A_K} \,.
    \ee
    The following lemma will be useful in order to calculate the rate of our entanglement distribution protocol.
    \begin{lemma}\label{Lemma_P_F}
        Let $\lambda\in[0,1],g\ge1, N\in\N,K\in\N^+$, and $\textbf{n}\in\N^K$ such that $\|\textbf{n}\|_1=N$. Assume that Alice transmits the $K$-mode Fock state $\ket{\textbf{n}}$ to Bob via $K$ parallel uses of the piBGC $\mathcal{N}_{g,\lambda}\coloneqq \Phi_{g,0}\circ\mathcal{E}_{\lambda,0}$ and suppose further that Bob measures the total photon number of the $K$ received modes. The probability $\mathcal{P}_F$ that Bob gets the outcome $F\in\N$ is 
        \bb\label{expr_pf_values}
            \mathcal{P}_F\coloneqq \Tr\left[\mathcal{N}_{g,\lambda}^{\otimes K}(\ketbra{\textbf{n}})\,\Pi^{(K)}_F\right]=\sum_{P=0}^{\min(F,N)}\binom{N}{P}\binom{K+F-1}{F-P}\lambda^{P}(1-\lambda)^{N-P}\frac{(g-1)^{F-P}}{g^{K+F}} \,.
        \ee
        In particular, note that $\mathcal{P}_F$ depends on $\textbf{n}$ only through the total photon number $\|\textbf{n}\|_1=N$.
        Specifically, if the communication channel is the pure loss channel $\mathcal{E}_{\lambda,0}=\mathcal{N}_{1,\lambda}$, the probability of getting the outcome $F\in\N$ is
        \bb
            \Tr\left[\mathcal{E}_{\lambda,0}^{\otimes K}(\ketbra{\textbf{n}})\,\Pi^{(K)}_F\right]&=\binom{N}{F}\lambda^{F}(1-\lambda)^{N-F} \Theta(N-F)\,,
         \ee   
         where we have introduced the Heaviside function $\Theta(x)$ defined as $\Theta(x)=1$ if $x\ge 0$, and $\Theta(x)=0$ if $x<0$.
         In addition, if the communication channel is the pure amplifier channel $\Phi_{g,0}=\mathcal{N}_{g,1}$, the probability of getting the outcome $F\in\N$ is
         \bb
            \Tr\left[\Phi_{g,0}^{\otimes K}(\ketbra{\textbf{n}})\,\Pi^{(K)}_F\right]&=\binom{K+F-1}{F-N} \frac{(g-1)^{F-N}}{g^{K+F}}\Theta(F-N)\,.
        \ee
        Therefore, the probability $\mathcal{P}_F$ in \eqref{expr_pf_values} of getting $F$ photons at the output of $K$ parallel uses of the composition between pure loss channel and pure amplifier channel can be expressed as the sum over $P\in\N$ of the conditional probability of getting $F$ photons at the output of the $K$ pure amplifier channels conditioned on the event of getting $P$ photons at the output of the $K$ pure loss channels, multiplied by the probability of the latter event.
    \end{lemma}
    \begin{proof}
        As a consequence of \eqref{action_ni_att}, for all $n\in\N$ it holds that
        \bb
            \mathcal{E}_{\lambda,0}(\ketbra{n})=\sum_{l=0}^n\binom{n}{l}\lambda^l(1-\lambda)^{n-l}\ketbra{l}
        \ee
        and hence 
        \bb
        \mathcal{E}_{\lambda,0}^{\otimes K}(\ketbra{\textbf{n}})=\sum_{\substack{\textbf{l}\in\N^K\\ \textbf{l}\le \textbf{n}}}\left(\prod_{j=1}^K\binom{n_j}{l_j}\right)\lambda^{\|\textbf{l}\|_1}(1-\lambda)^{N-\|\textbf{l}\|_1}\ketbra{\textbf{l}}\,,
        \ee
        where the inequality between vectors $\textbf{a}\ge \textbf{b}$ means that $a_j \ge b_j$ for all $j=1,\ldots,K$.
        Consequently, by using that $\sum_{P=0}^\infty \Pi^{(K)}_P=\mathbb{1}$, it holds that 
        \bb\label{calculation_P_F}
            \mathcal{P}_F&\coloneqq \Tr\left[\Phi_{g,0}^{\otimes K}\left(\mathcal{E}_{\lambda,0}^{\otimes K}(\ketbra{\textbf{n}})\right)\,\Pi^{(K)}_F\right]=\sum_{P,P'=0}^\infty\Tr\left[\Phi_{g,0}^{\otimes K}\left(\Pi^{(K)}_P\mathcal{E}_{\lambda,0}^{\otimes K}(\ketbra{\textbf{n}})\Pi^{(K)}_{P'}\right)\,\Pi^{(K)}_F\right]\\&=\sum_{P=0}^{N}\lambda^{P}(1-\lambda)^{N-P}\sum_{\substack{\textbf{l}\in\N^K\\ \textbf{l}\le \textbf{n}\\ \|\mathbf{l}\|_1=P}}\left(\prod_{j=1}^K\binom{n_j}{l_j}\right)\Tr\left[\Phi_{g,0}^{\otimes K}\left(\ketbra{\textbf{l}}\right)\,\Pi^{(K)}_F\right]\,.
        \ee
        Moreover, \eqref{action_ni_amp} implies that for all $l\in\N$ it holds that
        \bb
            \Phi_{g,0}(\ketbra{l})=\frac{1}{g^{l+1}}\sum_{m=0}^\infty \binom{l+m}{l}\left(\frac{g-1}{g}\right)^m\ketbra{m+l}
        \ee
        and hence
        \bb
            \Phi_{g,0}^{\otimes K}(\ketbra{\textbf{l}})=\frac{1}{g^{P+K}}\sum_{\textbf{m}\in\N^K} \left(\prod_{j=1}^K\binom{l_j+m_j}{l_j}\right)\left(\frac{g-1}{g}\right)^{\|\textbf{m}\|_1}\ketbra{\textbf{m}+\textbf{l}}\,.
        \ee
    Consequently, it holds that
    \bb\label{expression_with_sum}
        \Tr\left[\Phi_{g,0}^{\otimes K}\left(\ketbra{\textbf{l}}\right)\,\Pi^{(K)}_F\right]=\frac{(g-1)^{F-P}}{g^{K+F}}\sum_{\substack{\textbf{m}\in\N^K\\ \|\textbf{m}\|_1=F-P}} \prod_{j=1}^K\binom{l_j+m_j}{l_j}\,.
    \ee
    The sum 
    \bb
         \sum_{\substack{\textbf{m}\in\N^K\\ \|\textbf{m}\|_1=F-P}} \prod_{j=1}^K\binom{l_j+m_j}{l_j}\,,\qquad\quad
    \ee
    which appears in \eqref{expression_with_sum}, is the coefficient of the term $x^{F-P}$ of the power series $Q(x)$ in the variable $x\in(0,1)$ defined as
    \bb
        Q(x)\coloneqq\sum_{\substack{\textbf{m}\in\N^K}} \left(\prod_{j=1}^K\binom{l_j+m_j}{l_j}\right)x^{\|\textbf{m}\|_1}\,.
    \ee
    By exploiting that for all $l\in\N$ it holds that
    \bb\label{identity_pol}
    	\sum_{m=0}^\infty \binom{m+l}{m}x^m=\frac{1}{(1-x)^{l+1}}\,,
    \ee
    one obtains that
    \bb
        Q(x)=\sum_{\substack{\textbf{m}\in\N^K}} \left(\prod_{j=1}^K\binom{l_j+m_j}{l_j}\right)x^{\|\textbf{m}\|_1}&=\prod_{j=1}^K \left(\sum_{m=0}^\infty\binom{l_j+m}{m}x^m\right)=\prod_{j=1}^K\frac{1}{(1-x)^{l_j+1}}=\frac{1}{(1-x)^{P+K}}\\&=\sum_{m=0}^\infty\binom{P+K+m-1}{m}x^m\,.
    \ee
    It follows that
    \bb
        \sum_{\substack{\textbf{m}\in\N^K\\ \|\textbf{m}\|_1=F-P}} \prod_{j=1}^K\binom{l_j+m_j}{l_j}=\binom{K+F-1}{F-P}\Theta(F-P)
    \ee
    and hence
    \bb
        \Tr\left[\Phi_{g,0}^{\otimes K}\left(\ketbra{\textbf{l}}\right)\,\Pi^{(K)}_F\right]=\frac{(g-1)^{F-P}}{g^{K+F}}\binom{K+F-1}{F-P}\Theta(F-P)\,,
    \ee
    where we have introduced the Heaviside function $\Theta(x)$ defined as $\Theta(x)=1$ if $x\ge 0$, and $\Theta(x)=0$ if $x<0$.
    Consequently, \eqref{calculation_P_F} implies that
    \bb
        \mathcal{P}_F &= \sum_{P=0}^{\min(F,N)}\lambda^{P}(1-\lambda)^{N-P}\frac{(g-1)^{F-P}}{g^{K+F}}\binom{K+F-1}{F-P}\sum_{\substack{\textbf{l}\in\N^K\\ \textbf{l}\le \textbf{n}\\ \|\mathbf{l}\|_1=P}}\left(\prod_{j=1}^K\binom{n_j}{l_j}\right)\\&=\sum_{P=0}^{\min(F,N)}\binom{N}{P}\lambda^{P}(1-\lambda)^{N-P}\frac{(g-1)^{F-P}}{g^{K+F}}\binom{K+F-1}{F-P} \,,
    \ee
    where in the last equality we have exploited that
    \bb\label{identity_binomial_constr}
        \sum_{\substack{\textbf{l}\in\N^K\\ \textbf{l}\le \textbf{n}\\ \|\mathbf{l}\|_1=P}}\left(\prod_{j=1}^K\binom{n_j}{l_j}\right)=\binom{N}{P}\,.
    \ee
    This follows from the fact that the sum in \eqref{identity_binomial_constr} is equal to the coefficient of the term 
    $x^{P}$ of the following polynomial in the variable $x\in\mathbb{R}$:
    \bb
        \sum_{\substack{\textbf{l}\in\N^K\\ \textbf{l}\le \textbf{n}}}\left(\prod_{j=1}^K\binom{n_j}{l_j}\right)x^{\|\textbf{l}\|_1}=\prod_{j=1}^K(1+x)^{n_j}=(1+x)^N=\sum_{l=0}^N\binom{N}{l}x^{l}\,.
    \ee
    \end{proof}
    
    



    
    \begin{remark}
        Here we present an alternative method to calculate the probability $\mathcal{P}_F$ reported in \eqref{expr_pf_values}. For all $x\in(0,1)$ let us consider the tensor product of $K$ thermal states with mean photon number $\frac{x}{1-x}$, i.e.
        \bb
            \tau_{\frac{x}{1-x}}^{\otimes K}=(1-x)^K\sum_{\textbf{l}\in\N^K}x^{\|\textbf{l}\|_1}\ketbra{\textbf{l}}\,.
        \ee
        Consequently, the quantity
        \bb
            \mathcal{P}_F= \Tr\left[\mathcal{N}_{g,\lambda}^{\otimes K}(\ketbra{\textbf{n}})\,\Pi^{(K)}_F\right]
        \ee
        is the coefficient of the term $x^{F}$ of the power series $P(x)$ in the variable $x\in(0,1)$ defined as
    \bb\label{eq_p_x}
        P(x)\coloneqq \frac{1}{(1-x)^K}\Tr\left[\mathcal{N}_{g,\lambda}^{\otimes K}(\ketbra{\textbf{n}})\,\tau_{\frac{x}{1-x}}^{\otimes K}\right]\,.
    \ee
    By using the characteristic function properties reported in~\eqref{def_charact_func}, \eqref{inverse_fourier_displacement}, \eqref{transf_caract}, and the fact that the characteristic function of a thermal state $\tau_\nu$ is $\chi_{\tau_{\nu}}(\mathbf r)=e^{ -\frac{1}{4}(2\nu+1)|\mathbf{r}|^2}$, one obtains that for any single-mode state $\rho$ it holds that
    \bb
         \Tr\left[\mathcal{N}_{g,\lambda}(\rho)\,\tau_{\frac{x}{1-x}}\right]&=\int_{\mathbb{R}^{2}}\frac{\mathrm{d}^{2}\mathbf{r}}{2\pi}\chi_{\mathcal{N}_{g,\lambda}(\rho)}(\mathbf r)\,\chi_{\tau_{\frac{x}{1-x}}}(\mathbf r)=\int_{\mathbb{R}^{2}}\frac{\mathrm{d}^{2}\mathbf{r}}{2\pi} \chi_{\rho}(\sqrt{g\lambda}\,\mathbf{r})e^{-\frac{1}{4}\left(2g-g\lambda+2\frac{x}{1-x}\right)|\mathbf{r}|^2}\\&=\frac{1}{g\lambda}\int_{\mathbb{R}^{2}}\frac{\mathrm{d}^{2}\mathbf{r}}{2\pi} \chi_{\rho}(\mathbf{r})e^{-\frac{1}{4g\lambda}\left(2g-g\lambda+2\frac{x}{1-x}\right)|\mathbf{r}|^2}=\frac{1}{g\lambda}\Tr\left[\rho\,\,\tau_{\frac{g-g\lambda+(1+g\lambda-g)x}{g\lambda(1-x)}}\right]\,.
    \ee
    Hence, by exploiting \eqref{identity_pol} and the fact that $\|\textbf{n}\|_1=N$, the power series $P(x)$ can be expressed as 
    \bb
    		P(x)&=\frac{1}{(1-x)^K (g\lambda)^K}\Tr\left[ \ketbra{\textbf{n}} \tau_{\frac{g-g\lambda+(1+g\lambda-g)x}{g\lambda(1-x)}}^{\otimes K}  \right] =\frac{\left[g(1-\lambda)+(1+g\lambda-g)x\right]^{N}}{[g-(g-1)x]^{N+K}}\\&=\sum_{P=0}^N \binom{N}{P}(1+g\lambda-g)^P(1-\lambda)^{N-P}g^{-P-K}x^P\sum_{l=0}^\infty \binom{N+K-1+l}{l}\left(\frac{g-1}{g}\right)^l x^l\,.
    \ee
    It follows that
    \bb\label{expr_pf_values2}
    	\mathcal{P}_F=\sum_{P=0}^{\min(F,N)}\binom{N}{P}\binom{N+K+F-P-1}{F-P}(1+g\lambda-g)^P(1-\lambda)^{N-P}\frac{(g-1)^{F-P}}{g^{F+K}}\,.
    \ee
      Incidentally, by comparing the two expressions of $\mathcal{P}_F$ in \eqref{expr_pf_values} and \eqref{expr_pf_values2}, one deduces the following identity:
    \bb
    \sum_{P=0}^{\min(F,N)}\binom{N}{P}\binom{K+F-1+N-P}{F-P}\left(\frac{g\lambda -(g-1)}{(1-\lambda)(g-1)}\right)^P=\sum_{P=0}^{\min(F,N)}\binom{N}{P}\binom{K+F-1}{F-P}\left(\frac{\lambda}{(1-\lambda)(g-1)}\right)^P\,.
    \ee
    \end{remark}

    
    

    Let us now introduce an additional entanglement distribution protocol to distribute ebits across any piBGC $\mathcal{N}_{g,\lambda}$. The protocol depends on two parameters, $K,N\in\N^+$ with $K\ge2$, and it is composed of five steps named S1-S5, which we now outline.

\begin{enumerate}[\bf S1:]
    \item Alice prepares the state $\ket{\Psi_{N,K}}_{A_1\ldots A_K, A'_1,\ldots,A'_K}$ of $K+K$ modes $A_1,\ldots,A_k,A_1',\ldots,A_K'$, sending the systems $A'_1,\ldots,A'_K$ to Bob through $K$ uses of the channel $\mathcal{N}_{g,\lambda}$. Now Alice and Bob share the state $\Id_{A_1\ldots A_k}\otimes\mathcal{N}_{g,\lambda}^{\otimes K}(\ketbra{\Psi_{N,K}})$. By using~\eqref{action_comp_chan}, such a state can be expressed as
    \bb
        &\Id_{A_1\ldots A_k}\otimes\mathcal{N}_{g,\lambda}^{\otimes K}(\ketbra{\Psi_{N,K}})\\&=\frac{1}{\binom{N+K-1}{N}}\sum_{\substack{\textbf{n}\in\N^K\\ \|\textbf{n}\|_1=N}}\,\sum_{\substack{\textbf{i}\in\N^K\\ \|\textbf{i}\|_1=N}}\,\sum_{\substack{\textbf{l}\in\N^K\\ \textbf{l}\ge \max(\textbf{i}-\textbf{n},\textbf{0})}}\left(\prod_{j=1}^K f_{n_j,i_j,l_j}(g,\lambda)\right)\ketbraa{\textbf{n}}{\textbf{i}}_{A_1\ldots A_K} \otimes \ketbraa{\textbf{l}+\textbf{n}-\textbf{i}}{\textbf{l}}_{B_1\ldots B_K}\,,
    \ee
    where $\textbf{0}\in\N^K$ is the zero vector and the inequality between vectors $\textbf{a}\ge \textbf{b}$ means that $a_j \ge b_j$ for all $j=1,\ldots,K$.



    \item Bob performs the local POVM $\{\Pi^{(K)}_F\}_{F\in\N}$, where $\Pi^{(K)}_F$ is the projector onto the subspace whose total photon number equals $F$ (see \eqref{PROJECTOR_N_photon}), on the $K$ modes he has received.
    The probability of getting the outcome $F$ is denoted by $\mathcal{P}_F$ and it can be calculated as
    \bb
            \mathcal{P}_F&\coloneqq \Tr\left[\left(\mathbb{1}_{A_1\ldots A_k}\otimes\Pi^{(K)}_F\right)\, \left(\Id_{A_1\ldots A_k}\otimes\mathcal{N}_{g,\lambda}^{\otimes K}(\ketbra{\Psi_{N,K}})\right) \right]= \frac{1}{\binom{N+K-1}{N}}\sum_{\substack{\textbf{n}\in\N^K\\ \|\textbf{n}\|_1=N}}\Tr\left[\Pi^{(K)}_F\,\mathcal{N}_{g,\lambda}^{\otimes K}(\ketbra{\textbf{n}})\right]\\&=\sum_{P=0}^{\min(F,N)}\binom{N}{P}\binom{K+F-1}{F-P}\lambda^{P}(1-\lambda)^{N-P}\frac{(g-1)^{F-P}}{g^{K+F}} \,,%= \frac{1}{\binom{N+K-1}{N}}\sum_{\substack{\textbf{n}\in\N^K\\ \|\textbf{n}\|_1=N}}\,\sum_{\substack{\textbf{l}\in\N^K\\ \|\textbf{l}\|_1=F}}\left(\prod_{j=1}^K f_{n_j,n_j,l_j}(g,\lambda)\right) \,.
    \ee
    where we have exploited Lemma~\ref{Lemma_P_F}. The post-measurement state $\rho_{A_1\ldots A_kB_1\ldots B_k}^{(F)}$ conditioned on the outcome $F\in\N$ is given by
\bb\label{rho_step2_multirail}
        &\rho^{(F)}_{A_1\ldots A_kB_1\ldots B_k} \\
        &\quad = \frac{1}{\mathcal{P}_F}\left(\mathbb{1}_{A_1\ldots A_k}\otimes\Pi^{(K)}_F\right)\left( \Id_{A_1\ldots A_k}\otimes\mathcal{N}_{g,\lambda}^{\otimes K}(\ketbra{\Psi_{N,K}})\right)\left( \mathbb{1}_{A_1\ldots A_k}\otimes\Pi^{(K)}_F\right)\\
        &\quad =\frac{1}{\mathcal{P}_F\binom{N+K-1}{N}}\sum_{\substack{\textbf{n}\in\N^K\\ \|\textbf{n}\|_1=N}}\,\sum_{\substack{\textbf{i}\in\N^K\\ \|\textbf{i}\|_1=N}}\,\sum_{\substack{\textbf{l}\in\N^K\\ \|\textbf{l}\|_1=F\\\textbf{l}\ge \max(\textbf{i}-\textbf{n},\textbf{0})}}\left(\prod_{j=1}^K f_{n_j,i_j,l_j}(g,\lambda)\right)\ketbraa{\textbf{n}}{\textbf{i}}_{A_1\ldots A_K} \otimes \ketbraa{\textbf{l}+\textbf{n}-\textbf{i}}{\textbf{l}}_{B_1\ldots B_K}\,\\
        &\quad =\sum_{n,i=0}^{\binom{N+K-1}{N}-1} \sum_{h,l=0}^{\binom{F+K-1}{F}-1}c_{n,i,h,l}\ketbraa{\phi^{(N)}_n}{\phi^{(N)}_i}_{A_1\ldots A_k}\otimes\ketbraa{\phi^{(F)}_h}{\phi^{(F)}_l}_{B_1\ldots B_k}\,,
\ee
where for all $n,i=0,1,\ldots, \binom{N+K-1}{N}-1$ and all $h,l=0,1,\ldots, \binom{F+K-1}{F}-1$ the coefficient $c_{n,i,h,l}$ is defined as follows. Let $\textbf{n},\textbf{i}, \textbf{h}, \textbf{l}\in\N^K$ such that $\ket{\phi^{(N)}_n}=\ket{\textbf{n}}$, $\ket{\phi^{(N)}_i}=\ket{\textbf{i}}$, $\ket{\phi^{(F)}_h}=\ket{\textbf{h}}$, and $\ket{\phi^{(F)}_l}=\ket{\textbf{l}}$. If $\textbf{l}\ge \max(\textbf{i}-\textbf{n},\textbf{0})$ and $\textbf{h}=\textbf{l}+\textbf{n}-\textbf{i}$, then 
\bb
    c_{n,i,h,l}\coloneqq\frac{\left(\prod_{j=1}^K f_{n_j,i_j,l_j}(g,\lambda)\right)}{\mathcal{P}_F\binom{N+K-1}{N}}\,,
\ee
    otherwise $c_{n,i,h,l}=0$. By setting 
    \bb
    d\coloneqq \max\left(\binom{N+K-1}{N}, \binom{F+K-1}{F}\right)\,,
    \ee
    the resulting state in~\eqref{rho_step2_multirail} can be seen as a bipartite two-qu$d$it state $\rho^{(F)}_{AB}\in\mathfrak{S}(\HH_d\otimes\HH_d)$ of the form
    \bb
        \rho^{(F)}_{AB}=\sum_{n,i,h,l=0}^{d-1} \eta_{n,i,h,l}\ketbraa{n}{i}_A\otimes\ketbraa{h}{l}_B\,,
    \ee
    where $\HH_d$ is the qu$d$it Hilbert space with $\{\ket{0},\ket{1},\ldots, \ket{d-1}\}$ as an orthonormal basis, and where the coefficients $\eta_{n,i,h,l}$ are defined as follows:
    \begin{itemize}
        \item if $n,i\le \binom{N+K-1}{N}-1$ and $h,l\le \binom{F+K-1}{F}-1$, then $\eta_{n,i,h,l}\coloneqq c_{n,i,h,l}$;
        \item otherwise, $\eta_{n,i,h,l}\coloneqq 0$.
    \end{itemize}
    Consequently, Alice and Bob have reduced the problem in distilling ebits from the two-qu$d$it state $\rho^{(F)}_{AB}$.




    \item Now Alice and Bob decide whether or not to run the reverse hashing protocol, which can distil ebits from $\rho^{(F)}_{AB}$ with a rate equal to its reverse coherent information, i.e. 
    \bb
        I_{\text{rc}}(\rho^{(F)}_{AB})=S(\Tr_B\rho^{(F)}_{AB})-S(\rho^{(F)}_{AB})\,,%\log_2\binom{N+K-1}{N}-S(\rho^{(F)}_{AB})\,.
    \ee
    where $S(\cdot)$ denotes the von Neumann entropy. By exploiting that
    \bb
        \Tr_B\rho^{(F)}_{AB}=\frac{1}{\binom{N+K-1}{N}}\sum_{n=0}^{\binom{N+K-1}{N}-1}\ketbra{n},
    \ee
    as guaranteed by \eqref{rho_step2_multirail} and Lemma~\ref{Lemma_P_F}, it follows that the reverse coherent information can be calculated as
    \bb
        I_{\text{rc}}(\rho^{(F)}_{AB})=\log_2\binom{N+K-1}{N}-S\left( \sum_{n,i,h,l=0}^{d-1} \eta_{n,i,h,l}\ketbraa{n}{i}\otimes\ketbraa{h}{l}\right)\,.
    \ee
    If Alice and Bob choose to run the reverse hashing protocol, the protocol terminates. Otherwise, they apply the qu$d$it Pauli-based twirling reported in~\cite[Eq.~(18)]{p1orp2} in order to transform their state in a Bell-diagonal state of the form 
    \bb
    \rho'^{(F)}_{AB}=\sum_{m,n=0}^d \alpha^{(F,0)}_{mn}\ketbra{\psi^{(d)}_{mn}}_{AB}\,,
    \ee
    where 
    \bb
        \ket{\psi^{(d)}_{mn}}_{AB}\coloneqq\frac{1}{\sqrt{d}}\sum_{r=0}^{d-1}e^{i\frac{2\pi m r}{d}}\ket{r}_A\otimes\ket{(r-n)\text{ mod } d}_B
    \ee
    and 
    \bb
        \alpha^{(F,0)}_{mn}\coloneqq \bra{\psi^{(d)}_{mn} }\rho^{(F)}_{AB}\ket{\psi^{(d)}_{mn}}=\frac{1}{d}\sum_{r_1,r_2=0}^{d-1} \cos\left(\frac{2\pi m (r_2-r_1)}{d}\right)\eta_{r_1,r_2,\,(r_1-n)\text{ mod } d,\, (r_2-n)\text{ mod } d}\,.
    \ee

    \item Alice and Bob run $\bar{k}$ times the P1-or-P2 sub-routine for qu$d$its~\cite{p1orp2}, where $\bar{k}$ is chosen in order to maximise the ebit rate. The goal of this step is to bring the shared state closer to the $d$-dimensional maximally-entangled state $\ket{\psi_{00}^{(d)}}$. This step is successful, i.e.~the protocol is not aborted, with a probability of success equal to $\prod_{t=0}^{\bar{k}-1}P^{(F)}_t$ and it allows Alice and Bob to transform $2^{\bar{k}}$ copies of $\rho'^{(F)}_{AB}=\sum_{m,n=0}^d \alpha^{(F,0)}_{mn}\ketbra{\psi^{(d)}_{mn}}_{AB}$ in a state of the form
    \bb
         \rho'^{(F,\bar{k})}_{AB}\coloneqq\sum_{m,n=0}^d \alpha^{(F,\bar{k})}_{mn}\ketbra{\psi^{(d)}_{mn}}_{AB}\,.
    \ee
    For all $t\in\{0,1,\ldots,\bar{k}-1\}$ and all $m,n\in\{0,1,\ldots,d-1\}$ the coefficients $\alpha_{mn}^{(F,t+1)}$ and the probabilities $P^{(F)}_t$ are recursively defined in the following way~\cite{p1orp2}:
\begin{itemize}
    \item If $\sum_{m_1=0}^{d-1}\alpha^{(F,t)}_{m_10}<\sum_{n_1=0}^{d-1}\alpha^{(F,t)}_{0n_1}$, then
    \bb\label{coeff_p1_multirail}
            \alpha^{(F,t+1)}_{mn}\coloneqq \frac{1}{P^{(F)}_t}\sum_{\substack{m_1,m_2=0\\(m_1+ m_2)\text{ mod }d = m}}^{d-1}\alpha_{m_1n}^{(F,t)}\alpha_{m_2n}^{(F,t)}\,,
    \ee
    where
    \bb\label{probk_p1_multirail}
        P^{(F)}_t \coloneqq \sum_{m_1,m_2,n=0}^{d-1}\alpha_{m_1n}^{(F,t)}\alpha_{m_2n}^{(F,t)} \,.
    \ee
    \item Otherwise,
    \bb\label{coeff_p2_multirail}
        \alpha^{(F,t+1)}_{mn}\coloneqq \frac{1}{P^{(F)}_t}\sum_{\substack{n_1,n_2=0\\(n_1+ n_2)\text{ mod }d = n}}^{d-1}\alpha_{mn_1}^{(F,t)}\alpha_{mn_2}^{(F,t)}\,,
    \ee    
    where
    \bb\label{probk_p2_multirail}
        P^{(F)}_t \coloneqq \sum_{m,n_1,n_2=0}^{d-1}\alpha_{mn_1}^{(F,t)}\alpha_{mn_2}^{(F,t)} \,.
    \ee
\end{itemize}

 
     \item Alice and Bob distil ebits from the state $\rho'^{(F,\bar{k})}_{AB}=\sum_{m,n=0}^d \alpha^{(F,\bar{k})}_{mn}\ketbra{\psi^{(d)}_{mn}}_{AB}$ with a yield denoted as $\mathcal{I}_d(\alpha^{(F,\bar{k})})$ by running the following protocol:
    \begin{itemize}
    
        \item If $d=2$, then Alice and Bob run the Step~5 and Step~6 of the entanglement distribution protocol introduced in the proof of Theorem~\ref{th_lower_Q2} in order to distil ebits from $\rho'^{(F,\bar{k})}_{AB}$ with a yield equal to 
        \bb
            \mathcal{I}_2(\alpha^{(F,\bar{k})})\coloneqq\mathcal{I}(  \alpha^{(F,\bar{k})}_{00},\alpha^{(F,\bar{k})}_{01},\alpha^{(F,\bar{k})}_{10},\alpha^{(F,\bar{k})}_{11})\,,
        \ee
    where $\mathcal{I}$ is defined in~\eqref{yield_final_protocol}. 

        \item If $d>2$, then Alice and Bob run the hashing protocol on $\rho'^{(F,\bar{k})}_{AB}$ and thus they distil ebits with a yield equal to the coherent information of $\rho'^{(F,\bar{k})}_{AB}$, i.e.
        \bb
            \mathcal{I}_d(\alpha^{(F,\bar{k})})\coloneqq I_{\text{c}}(\rho'^{(F,\bar{k})}_{AB})= \log_2 d+ \sum_{m,n=0}^{d-1}\alpha^{(F,\bar{k})}_{mn}\log_2\alpha^{(F,\bar{k})}_{mn}\,.
        \ee
    \end{itemize}

\end{enumerate}


    The ebit rate of the protocol is given by
    \bb\label{rate_multirail}
        R(g,\lambda,N,K)\coloneqq \frac{1}{K}\sum_{F=1}^\infty \mathcal{P}_F \max\left(I_{\text{rc}}(\rho^{(F)}_{AB})\,,\,\sup_{\bar{k}\in\N} \frac{\prod_{t=0}^{\bar{k}-1}P^{(F)}_t}{2^{\bar{k}}}\mathcal{I}_d(  \alpha^{(F,\bar{k})})\right)\,.
    \ee
    The term $\frac{1}{K}$ in the expression~\eqref{rate_multirail} arises from the fact that Alice uses the channel $K$ times during step S1, and the variable $F$ corresponds to the outcome of the total photon number measurement in step S2, with associated probability $\mathcal{P}_F$. The sum over $F$ equals the expected value of the yield of ebits that can be distilled from the post-measurement state $\rho^{(F)}_{AB}$ by running steps S3, S4, and S5. The maximum comes from the fact that during step S3 Alice and Bob choose whether or not to run the reverse hashing protocol, which can distil ebits with a rate equal to $I_{\text{rc}}(\rho^{(F)}_{AB})$. The supremum over $\bar{k}$ comes from the fact that Alice and Bob choose the number of iterations $\bar{k}$ of the P1-or-P2 subroutine in order to maximise the rate. The rate in~\eqref{rate_multirail} is a lower bound on the two-way quantum capacity of the piBGC $\mathcal{N}_{g,\lambda}$ for all $N,K\in\N^+$ with $K\ge 2$. Therefore, we have
    \bb
        K(\mathcal{N}_{g,\lambda}) \ge  Q_2(\mathcal{N}_{g,\lambda})\ge \sup_{\substack{N,K\in\N^+\\ K\ge2 }} R(g,\lambda,N,K)\,.
    \ee
    Let us summarise this result in the following theorem.
    \begin{thm}\label{thm_multirail}
        For all $\lambda\in[0,1]$ and $g\ge 1$ the secret-key capacity $K(\mathcal{N}_{g,\lambda})$ and the two-way quantum capacity $Q_2(\mathcal{N}_{g,\lambda})$ of the piBGC $\mathcal{N}_{g,\lambda}$ satisfy
        \bb
            K(\mathcal{N}_{g,\lambda}) \ge  Q_2(\mathcal{N}_{g,\lambda})\ge \sup_{\substack{N,K\in\N^+\\ K\ge2 }} R(g,\lambda,N,K)\,,
        \ee
        where
        \bb\label{rate_expr_multirail}
        R(g,\lambda,N,K)\coloneqq \frac{1}{K}\sum_{F=1}^\infty \mathcal{P}_F \max\left(I_{\text{rc}}^{(F)}\,,\,\sup_{\bar{k}\in\N} \frac{\prod_{t=0}^{\bar{k}-1}P^{(F)}_t}{2^{\bar{k}}}\mathcal{I}_d(  \alpha^{(F,\bar{k})})\right)\,.
        \ee
        The quantities present in~\eqref{rate_expr_multirail} are defined as follows. For all $F\in\N$ the dimension $d$ is defined as 
        \bb
        d\coloneqq \max\left(\binom{N+K-1}{N}, \binom{F+K-1}{F}\right)
        \ee
        and the probability $\mathcal{P}_F$ is defined as
        \bb
            \mathcal{P}_F&\coloneqq \sum_{P=0}^{\min(F,N)}\binom{N}{P}\binom{K+F-1}{F-P}\lambda^{P}(1-\lambda)^{N-P}\frac{(g-1)^{F-P}}{g^{K+F}}\,.
        \ee
        Moreover, the probabilities $P^{(F)}_{\bar{k}}$ and the coefficients $\{\alpha_{mn}^{(F,\bar{k})}\}_{m,n\in\{0,1,\ldots,d-1\}}$ are recursively defined as follows.
        For all $t\in\{0,1,\ldots,\bar{k}-1\}$ and all $m,n\in\{0,1,\ldots,d-1\}$ it holds that:
\begin{itemize}
    \item If $\sum_{m_1=0}^{d-1}\alpha^{(F,t)}_{m_10}<\sum_{n_1=0}^{d-1}\alpha^{(F,t)}_{0n_1}$, then
    \bb\label{coeff_p1_multirail2}
            \alpha^{(F,t+1)}_{mn}&\coloneqq \frac{1}{P^{(F)}_t}\sum_{\substack{m_1,m_2=0\\(m_1+ m_2)\text{ mod }d = m}}^{d-1}\alpha_{m_1n}^{(F,t)}\alpha_{m_2n}^{(F,t)}\,,\\
        P^{(F)}_t &\coloneqq \sum_{m_1,m_2,n=0}^{d-1}\alpha_{m_1n}^{(F,t)}\alpha_{m_2n}^{(F,t)} \,.
    \ee
    \item Otherwise,
    \bb\label{coeff_p2_multirail2}
        \alpha^{(F,t+1)}_{mn}&\coloneqq \frac{1}{P^{(F)}_t}\sum_{\substack{n_1,n_2=0\\(n_1+ n_2)\text{ mod }d = n}}^{d-1}\alpha_{mn_1}^{(F,t)}\alpha_{mn_2}^{(F,t)}\,,\\
        P^{(F)}_t &\coloneqq \sum_{m,n_1,n_2=0}^{d-1}\alpha_{mn_1}^{(F,t)}\alpha_{mn_2}^{(F,t)} \,.
    \ee
\end{itemize}
Moreover, for all $m,n\in\{0,1,\ldots,d-1\}$ the coefficient $\alpha^{(F,0)}_{mn}$ is defined as
    \bb
        \alpha^{(F,0)}_{mn}\coloneqq \frac{1}{d}\sum_{r_1,r_2=0}^{d-1} \cos\left(\frac{2\pi m (r_2-r_1)}{d}\right)\eta_{r_1,r_2,(r_1-n)\text{ mod } d, (r_2-n)\text{ mod } d}\,.
    \ee
    In addition, for all $n,i\in{0,1,\ldots,\binom{N+K-1}{N}-1}$, we define $(n_1,\ldots,n_K)$ and $(i_1,\ldots,i_K)$ as the $n$th and $i$th element of the ordered set $S_{K,N}$, where $S_{K,N}$ is defined as  \bb
    S_{K,N}\coloneqq\{(f_1,\ldots,f_K)\in\N^K:\,  \sum_{j=1}^{K} f_j=N\}
    \ee
    and it is ordered according to the relation $\preceq_{K,N}$, given by     \bb
        (f_1,\ldots,f_K)\,\preceq_{K,N}\,(g_1,\ldots,g_K) \, \Longleftrightarrow \sum_{j=1}^{K} f_{j}\, (N+1)^j<\sum_{j=1}^{K} g_{j}\, (N+1)^j\,.
    \ee
    Additionally, for all $h,l\in\{0,1,\ldots,\binom{F+K-1}{F}-1\}$, we define $(h_1,\ldots,h_K)$ and $(l_1,\ldots,l_K)$ as the $h$th and $l$th element of the set $S_{K,F}$ ordered according to the relation $\preceq_{F,N}$. Furthermore, for all $n,i,h,l\in\{0,1,\ldots,d-1\}$ the coefficients $\eta_{n,i,h,l}$ are defined as follows:
    \begin{itemize}
        \item If 
        \bb
        n,i&\le \binom{N+K-1}{N}-1\,,\\
        h,l&\le \binom{F+K-1}{F}-1\,,\\ 
        l_j&\ge \max(i_j-n_j,0)\,\quad \text{for all } j=1,2,\ldots,K\,,\\  
        h_j&=l_j+n_j-i_j\,\quad \text{for all } j=1,2,\ldots,K\,,  
    \ee
    then 
    \bb
    \eta_{n,i,h,l}\coloneqq\frac{\left(\prod_{j=1}^K f_{n_j,i_j,l_j}(g,\lambda)\right)}{\mathcal{P}_F\binom{N+K-1}{N}}\,,
    \ee
    where $f_{n,i,l}(g,\lambda)$ is defined in~\eqref{def_f_comp}.
    \item Otherwise, $\eta_{n,i,h,l}=0$. 
    \end{itemize}
    Moreover, the quantity $I_{\text{rc}}^{(F)}$ is defined as
    \bb
        I_{\text{rc}}^{(F)}\coloneqq \log_2\binom{N+K-1}{N}-S\left( \sum_{n,i,h,l=0}^{d-1} \eta_{n,i,h,l}\ketbraa{n}{i}\otimes\ketbraa{h}{l}\right)\,,%=\log_2\binom{N+K-1}{N}-S\left( \sum_{n,i,h,l=0}^{d-1} \eta_{n,i,h,l}\ketbraa{n}{i}\otimes\ketbraa{h}{l}\right)\,,
    \ee
    where $S(\cdot)$ denotes the von Neumann entropy.
    Finally, the term $\mathcal{I}_d(\alpha^{(F,\bar{k})})$ is defined differently depending on the value of $d$:
    \begin{itemize}
        \item If $d=2$, then
        \bb
            \mathcal{I}_2(\alpha^{(F,\bar{k})})\coloneqq\mathcal{I}(  \alpha^{(F,\bar{k})}_{00},\alpha^{(F,\bar{k})}_{01},\alpha^{(F,\bar{k})}_{10},\alpha^{(F,\bar{k})}_{11})\,,
        \ee
    where $\mathcal{I}$ is defined in~\eqref{yield_final_protocol}. 
        \item If $d>2$, then 
        \bb
            \mathcal{I}_d(\alpha^{(F,\bar{k})})\coloneqq \log_2 d+ \sum_{m,n=0}^{d-1}\alpha^{(F,\bar{k})}_{mn}\log_2\alpha^{(F,\bar{k})}_{mn}\,.
        \ee
    \end{itemize}
    \end{thm}


\subsection{Results on the two-way capacities of piBGCs}\label{sub_res_twoway}
In this subsection, for each of the piBGCs, first we determine the parameter region where the two-way capacities vanish, second we find a new lower bound on the two-way capacities, and finally we compare our results with the existing literature.
\subsubsection{Results on the two-way capacities of the thermal attenuator}
Let us consider the thermal attenuator $\mathcal{E}_{\lambda,\nu}$ of transmissivity $\lambda\in[0,1]$ and thermal noise $\nu\ge0$. Since the PLOB bound in~\eqref{PLOB_Q2} vanishes for $\lambda\le \frac{\nu}{\nu+1}$, it is already known that the two-way capacities of $\mathcal{E}_{\lambda,\nu}$ vanish for $\lambda<\frac{\nu}{\nu+1}$. The following theorem establishes that also the vice-versa is true.
\begin{thm}\label{th1_therm}
Let $\lambda\in[0,1]$, $\nu\ge0$, and $N_s>0$. The energy-constrained two-way capacities of the thermal attenuator $Q_2(\mathcal{E}_{\lambda,\nu},N_s)$ and $K(\mathcal{E}_{\lambda,\nu},N_s)$ vanish if and only if $\lambda\le \frac{\nu}{\nu+1}$, i.e.~if and only if $\mathcal{E}_{\lambda,\nu}$ is entanglement breaking. In particular, the same holds for the unconstrained two-way capacities.
\end{thm}
\begin{proof}
    Theorem~\ref{th1_therm} is a direct consequence of Lemma~\ref{lemma_comp_bos} and Theorem~\ref{th1}.
\end{proof}
The validity of Theorem~\ref{th1_therm} was not known before the present work. Indeed, in~\cite{Pirandola18,Pirandola20} the authors says that it is an open problem to determine the exact value of the maximum tolerable excess noise, which is defined by 
\begin{equation}\label{excess_noise}
   \epsilon(\lambda)\coloneqq \frac{1-\lambda}{\lambda}\max\{\nu\ge0\,:\,K(\mathcal{E}_{\lambda,\nu})>0\}\,.
\end{equation}
Theorem~\ref{th1} implies that $\varepsilon(\lambda)=1$ for all $\lambda\in(0,1)$. Hence, we have answered to the question, which was deemed ``crucial" in~\cite[Section 7]{Pirandola18}, ``What is the maximum
excess noise that is tolerable in QKD? I.e., optimizing over all QKD protocols?"
In~\cite{Pirandola18,Pirandola20} the authors showed, by applying the PLOB bound, the upper bound $\varepsilon(\lambda)\le1$ and provided also a lower bound on $\varepsilon(\lambda)$ which was far from $1$.
\begin{comment}
In addition, we find that all the two-way capacities are strictly larger than zero if and only if $\lambda>\frac{\nu}{\nu+1}$. This solves the open problem stated in~\cite[Section 7]{Pirandola18}, deemed ``crucial'' by the authors of~\cite{Pirandola18}, which consists in determining for all $\lambda\in(0,1)$ what is the maximum thermal parameter $\nu$ that is tolerable in QKD (quantum key distribution), i.e. by optimising over all the QKD protocols. We show that it is $\nu=\frac{\lambda}{1-\lambda}$. In~\cite{Pirandola18} the authors states the problem in terms of the \emph{excess noise}, defined by 
\begin{equation}\label{exc}
    \epsilon(\lambda)\coloneqq \frac{1-\lambda}{\lambda}\max_{\nu\ge0\text{ s.t. }K(\mathcal{E}_{\lambda,\nu})>0}\nu\,.
\end{equation}
We show that $\epsilon(\lambda)=1$ for all $\lambda\in(0,1)$.
\end{comment}
 


Except for the special case $\nu=0$, it is an open question whether the reverse coherent information lower bound in Eq.~\ref{lowQ2} equals the true two-way quantum capacity of the thermal attenuator $Q_2(\mathcal{E}_{\lambda,\nu})$: Theorem~\ref{th1_therm} provides a negative answer to this question. Indeed, although $Q_2(\mathcal{E}_{\lambda,\nu})=0$ if and only if $\lambda\le \frac{\nu}{\nu+1}$ (thanks to Theorem~\ref{th1_therm}), the reverse coherent information lower bound vanishes for all $\lambda \le 1-2^{-h(\nu)}$. Hence, since $1-2^{-h(\nu)}>\frac{\nu}{\nu+1}$ for all $\nu>0$, the reverse coherent information lower bound is not equal to $Q_2(\mathcal{E}_{\lambda,\nu})$ at least in the region $\nu>0$ and $\lambda\in (\frac{\nu}{\nu+1},1-2^{-h(\nu)}]$. In the following theorem we obtain an improved lower bound on the two-way capacities of the thermal attenuator. 
\begin{thm}\label{th_delta}
 Let $\lambda\in[0,1]$, $\nu\ge0$, and $N_s\ge0$. The EC two-way capacities $Q_2(\mathcal{E}_{\lambda,\nu},N_s)$ and $K(\mathcal{E}_{\lambda,\nu},N_s)$ of the thermal attenuator $\mathcal{E}_{\lambda,\nu}$ satisfy the following lower bound
\bb\label{lowQ2_deltaEC}
	K(\mathcal{E}_{\lambda,\nu},N_s)& \ge  Q_2(\mathcal{E}_{\lambda,\nu},N_s)\ge \sup_{\substack{c\in(0,1),\, M\in\N^+,\, k\in\N\\ (1-c^2)M\le N_s}} \mathcal{R}\left(1+(1-\lambda)\nu,\frac{\lambda}{1+(1-\lambda)\nu},M,c,k\right)\,,
\ee
and, in particular, the unconstrained two-way capacities satisfy
\bb\label{lowQ2_delta}
	K(\mathcal{E}_{\lambda,\nu})& \ge  Q_2(\mathcal{E}_{\lambda,\nu})\ge \sup_{c\in(0,1),\, M\in\N^+,\, k\in\N} \mathcal{R}\left(1+(1-\lambda)\nu,\frac{\lambda}{1+(1-\lambda)\nu},M,c,k\right)\,,
\ee
where the quantity $\mathcal{R}$ is defined in~\eqref{def_mathcal_R}. 
\end{thm}
\begin{proof} 
     Theorem~\ref{th_delta} is a direct consequence of Lemma~\ref{lemma_comp_bos} and Theorem~\ref{th_comp_Q2}.
\end{proof}
Theorem~\ref{th_delta} shows a new lower bound, reported in~\eqref{lowQ2_delta}, on the two-way capacities of the thermal attenuator $\mathcal{E}_{\lambda,\nu}$. Our new lower bound outperforms all the previous known lower bounds in a large region of the parameters $\lambda$ and $\nu$.
In Fig.~\ref{figure_nu2_delta}a and in Fig.~\ref{figure_nu2_delta}b we plot our new bound and its ratio with the PLOB bound, respectively, with respect to $\nu$ where the transmissivity is chosen to be equal to $\lambda(\nu)\coloneqq 1-2^{-h(\nu)}$, which is the upper endpoint for the $\lambda$-range for which the best known lower bound on $Q_2(\mathcal{E}_{\lambda,\nu})$ (i.e.~the reverse coherent information lower bound reported in~\eqref{lowQ2}) vanishes. From Fig.~\ref{figure_nu2_delta}a and Fig.~\ref{figure_nu2_delta}b we see that for these choices of $\nu$ and $\lambda(\nu)$, our new lower bound is now the best lower bound on $Q_2(\mathcal{E}_{\lambda,\nu})$ and it achieves the $\simeq 14\%$ of the PLOB bound for $\nu\gg 1$. For example, if $\nu=1$ and if the transmissivity is equal to $\lambda=1-2^{-h(1)}=0.75$, our new lower bound is $\simeq 0.033$, its ratio with the PLOB bound is $\simeq 0.08$, and the optimal parameters of the supremum present in the expression of our new bound in~\eqref{lowQ2_delta} are $c\simeq0.703$, $M=2$, and $k=2$. In Fig.~\ref{rate_vs_lambda1} we plot our new bound on $Q_2(\mathcal{E}_{\lambda,\nu})$ with respect to $\lambda$ for $\nu=1$ and $\nu=10$.

Our new bound can outperform also the best known lower bound (before our work) on the secret-key capacity $K(\mathcal{E}_{\lambda,\nu})$ found by Ottaviani et al.~\cite{Ottaviani_new_lower}. To demonstrate that our new bound can be strictly tighter than the Ottaviani et al.~lower bound, in Fig.~\ref{secret_key_nu} we plot the latter bound and our new bound with respect to $\lambda$ for $\nu=1$ and $\nu=10$. From Fig.~\ref{secret_key_nu}, we note that the Ottaviani et al.~lower bound vanishes for larger transmissivities than our bound. In particular, fixed $\nu>0$, we numerically observe that our new bound is strictly positive for all $\lambda>\frac{\nu}{\nu+1}$, which is the region where the two-way capacities of $\mathcal{E}_{\lambda,\nu}$ are strictly positive, as established by Theorem~\ref{th1}. As an example, for $\nu=1$, in Fig.~\ref{log1} we plot the ratio between our bound and the PLOB bound in logarithmic scale and we see that our bound is strictly positive for $\lambda\gtrsim \frac{\nu}{\nu+1}=0.5$. In addition, fixed $\nu>0$, we numerically observe that the optimal value of $k$ of the supremum present in the expression of our new bound in~\eqref{lowQ2_delta} increases as $\lambda$ decreases and tends to infinity as $\lambda$ tends to $\frac{\nu}{\nu+1}$, where we recall that $k$ represents the number of iterations of the P1-or-P2 sub-routine~\cite{p1orp2} in the entanglement distribution protocol we have introduced in the proof of Theorem~\ref{th_lower_Q2}.


%Our lower bound in~\eqref{lowQ2_delta} constitutes also a lower bound on the EC two-way capacities of the thermal attenuator $\mathcal{E}_{\lambda,\nu}$ with energy constraint equal to $(1-c^2)M$, where $c$ and $M$ are the optimal values of the supremum present in the expression of our bound. 
We numerically observe that for all $\lambda$ and $\nu$ the optimal choice of $M$ of the supremum present in the expression of our bound in~\eqref{lowQ2_delta} is always less or equal to $3$. Hence, since the mean photon number of each signal sent by Alice is $\Tr[a^\dagger a\ketbra{\Psi_{M,c}}]=(1-c^2)M$ (see~\eqref{initial_state}), the entanglement distribution protocol we have presented in the proof of Theorem~\ref{th_delta} exploits a mean photon number per channel use which is strictly lower than $3$. On the contrary, the entanglement distribution protocol which leads to the reverse coherent information lower bound in~\eqref{lowQ2} requires infinite mean photon number per channel use, as we reviewed in~\ref{proof_lower}. 




Theorem~\ref{th_delta} shows also the bound in~\eqref{lowQ2_deltaEC}, which constitutes a new lower bound on the EC two-way capacities of the thermal attenuator $\mathcal{E}_{\lambda,\nu}$. %Fixed the energy constraint $N_s>0$, it can be obtained by restricting the supremum in~\eqref{lowQ2_delta} to $c$ and $M$ satisfying $(1-c^2)M\le N_s$.
This new lower bound can outperform the NPJ lower bound~\cite{Noh2020} reported in~\eqref{npj_bound_therm}, which is the best known lower bound on the EC two-way capacities of the thermal attenuator, as we show in Fig.~\ref{ECfigures} where we plot our new bound in~\eqref{lowQ2_deltaEC} with respect to $\lambda$ for different choices of $\nu$ and of the energy constraint $N_s$.
%It can be shown that if $\lambda\le 1-2^{-h(\nu)}$ then the bound of Noh et al.~\cite{Noh2020} vanishes.

 

%Indeed, fixed $\nu>0$, the upper endpoint for the $\lambda$-range for which the bound of~\cite{Ottaviani_new_lower} vanishes is larger than $\frac{\nu}{\nu+1}$, however we new bound is nonzero for all $\lambda>\frac{\nu}{\nu+1}$, as we can see from Fig.~ x for $\nu=1$ by plotting it in logarithmic scale t with respect to $\lambda$.

By using the results of Section~\ref{multiplerail_section}, in the forthcoming Theorem~\ref{theorem_new_lower_multirail} we show an additional lower bound on the two-way quantum capacity of the thermal attenuator $\mathcal{E}_{\lambda,\nu}$.
\begin{thm}[Multi-rail lower bound]\label{theorem_new_lower_multirail}
 For all $\lambda\in[0,1]$ and $\nu\ge0$ the two-way capacities of the thermal attenuator $\mathcal{E}_{\lambda,\nu}$ satisfy
\bb\label{multiplerail_low_bound}
    K(\mathcal{E}_{\lambda,\nu}) &\ge  Q_2(\mathcal{E}_{\lambda,\nu})\ge \sup_{\substack{N,K\in\N^+\\ K\ge2 }} R\left(1+(1-\lambda)\nu,\frac{\lambda}{1+(1-\lambda)\nu},N,K\right)\,,
\ee
where the quantity $R$ is defined in~\eqref{rate_expr_multirail}. 
\end{thm}
\begin{proof}
    Theorem~\ref{theorem_new_lower_multirail} is a direct consequence of Theorem~\ref{thm_multirail} and Lemma~\ref{lemma_comp_bos}.
\end{proof}
Theorem~\ref{theorem_new_lower_multirail} shows an additional lower bound on $Q_2(\mathcal{E}_{\lambda,\nu})$, that we dub `multi-rail lower bound'. This bound is the ebit rate of the entanglement distribution protocol presented in Section~\ref{multiplerail_section}, which combines the multi-rail protocol introduced in~\cite{Winnel} and the qudit P1-or-P2 protocol introduced in~\cite{p1orp2}.
In Fig.~\ref{multiplerail_figures} we plot both the multi-rail lower bound (reported in~\eqref{multiplerail_low_bound}) and our previously discussed lower bound (reported in~\eqref{lowQ2_delta}) as a function of $\lambda$ for $\nu=0.1$, $\nu=0.5$, $\nu=1$, and $\nu=10$. Our numerical investigation shows that for $\nu\lesssim 1$, the multi-rail lower bound is tighter than the previously discussed lower bound, as confirmed by Fig.~\ref{multiplerail_figures}.



\subsubsection{Results on the two-way capacities of the thermal amplifier}
Let us consider the thermal amplifier $\Phi_{g,\nu}$ of gain $g\ge1$ and thermal noise $\nu\ge0$. Since the PLOB bound in~\eqref{PLOB_amp} vanishes for $g\ge 1+\frac{1}{\nu}$, it is already known that the two-way capacities of $\Phi_{g,\nu}$ vanish for $g\ge 1+\frac{1}{\nu}$. The following theorem establishes that also the vice-versa is true.
\begin{thm}\label{th1_amp}
Let $g\ge1$, $\nu\ge0$, and $N_s>0$. The energy-constrained two-way capacities of the thermal amplifier $Q_2(\Phi_{g,\nu},N_s)$ and $K(\Phi_{g,\nu},N_s)$ vanish if and only if $g\ge 1+\frac{1}{\nu}$, i.e.~if and only if $\Phi_{g,\nu}$ is entanglement breaking. In particular, the same holds for the unconstrained two-way capacities.
\end{thm}
\begin{proof}
    Theorem~\ref{th1_amp} is a direct consequence of Lemma~\ref{lemma_comp_bos} and Theorem~\ref{th1}.
\end{proof}
Except for the special case $\nu=0$, it is an open question whether the coherent information lower bound in Eq.~\ref{lowQ2_amp} equals the true two-way quantum capacity of the thermal amplifier $Q_2(\Phi_{g,\nu})$: Theorem~\ref{th1_amp} provides a negative answer to this question. Indeed, although $Q_2(\Phi_{g,\nu})=0$ if and only if $g > 1+\frac{1}{\nu}$ (thanks to Theorem~\ref{th1_amp}), the coherent information lower bound vanishes for all $g \ge \frac{1}{1-2^{-h(\nu)}}$. Hence, since $1+\frac{1}{\nu}>\frac{1}{1-2^{-h(\nu)}}$ for all $\nu>0$, the coherent information lower bound is not equal to $Q_2(\Phi_{g,\nu})$ at least in the region $\nu>0$ and $g\in [\frac{1}{1-2^{-h(\nu)}},1+\frac{1}{\nu})$. In the following theorem we obtain an improved lower bound on the two-way capacities of the thermal amplifier.
\begin{thm}\label{th_delta_amp}
Let $g\ge 1$, $\nu\ge0$, and $N_s\ge0$. The EC two-way capacities $Q_2(\Phi_{g,\nu},N_s)$ and $K(\Phi_{g,\nu},N_s)$ of the thermal amplifier $\Phi_{g,\nu}$ satisfy the following lower bound
\bb\label{lowQ2_deltaEC_amp}
	K(\Phi_{g,\nu},N_s)&\ge Q_2(\Phi_{g,\nu},N_s) \ge \sup_{\substack{c\in(0,1),\, M\in\N^+,\, k\in\N\\ (1-c^2)M\le N_s}} \mathcal{R}\left(g+(g-1)\nu,\frac{g}{g+(g-1)\nu},M,c,k\right)\,,
\ee
and, in particular, the unconstrained two-way capacities satisfy
\bb\label{lowQ2_delta_amp}
	K(\Phi_{g,\nu})&\ge Q_2(\Phi_{g,\nu}) \ge \sup_{c\in(0,1),\, M\in\N^+,\, k\in\N} \mathcal{R}\left(g+(g-1)\nu,\frac{g}{g+(g-1)\nu},M,c,k\right)\,,
\ee
where the quantity $\mathcal{R}$ is defined in~\eqref{def_mathcal_R}. 
\end{thm}
\begin{proof} 
     Theorem~\ref{th_delta_amp} is a direct consequence of Lemma~\ref{lemma_comp_bos} and Theorem~\ref{th_comp_Q2}.
\end{proof}
Theorem~\ref{th_delta_amp} shows a new lower bound, reported in~\eqref{lowQ2_delta_amp}, on the two-way capacities of the thermal amplifier $\Phi_{g,\nu}$. Our new lower bound outperforms all the previous known lower bounds in a large region of the parameters $g$ and $\nu$.
In Fig.~\ref{capvsnu_amp}a and in Fig.~\ref{capvsnu_amp}b we plot our new bound and its ratio with the PLOB bound, respectively, with respect to $\nu$ where the transmissivity is chosen to be equal to $g(\nu)\coloneqq \frac{1}{1-2^{-h(\nu)}}$, which is the lower endpoint for the $g$-range for which the best known lower bound on $Q_2(\Phi_{g,\nu})$ (i.e.~the coherent information lower bound reported in~\eqref{lowQ2_amp}) vanishes. From Fig.~\ref{capvsnu_amp}a and Fig.~\ref{capvsnu_amp}b we see that for these choices of $\nu$ and $g(\nu)$, our new lower bound is now the best lower bound on $Q_2(\Phi_{g,\nu})$ and it achieves the $\simeq 14\%$ of the PLOB bound for $\nu\gg 1$. In Fig.~\ref{bound_vs_nu1_amp} we plot our new bound with respect to $g$ for $\nu=1$ and $\nu=10$.



Our new bound can outperform also the WOGP-bound~\cite{Wang_Q2_amplifier}, which is the best known lower bound (before our work) on the secret-key capacity $K(\Phi_{g,\nu})$. To demonstrate that our new bound can be strictly tighter than the WOGP lower bound, in Fig.~\ref{secret_g_nu10} we plot the latter bound and our new bound with respect to $g$ for $\nu=1$ and $\nu=10$. From Fig.~\ref{secret_g_nu10} we note that the WOGP lower bound vanishes for smaller values of $g$ than our bound. In particular, fixed $\nu>0$, we numerically observe that our new bound is strictly positive for all $g<1+\frac{1}{\nu}$, which is the region where the two-way capacities of $\Phi_{g,\nu}$ are strictly positive, as established by Theorem~\ref{th1_amp}. As an example, for $\nu=1$, in Fig.~\ref{log_ratio_vs_lam_amp} we plot the ratio between our bound and the PLOB bound in logarithmic scale and we see that our bound is strictly positive for $g\lesssim 1+\frac{1}{\nu}=2$.



 

  
\subsubsection{Results on the two-way capacities of the additive Gaussian noise}
Let us consider the additive Gaussian noise $\Lambda_\xi$ of parameter $\xi\ge0$. Since the PLOB bound in~\eqref{PLOB_add} vanishes for $\xi\ge1$, it is already known that the two-way capacities of $\Lambda_\xi$ vanish for $\xi\ge1$. The following theorem establishes that also the vice-versa is true.
\begin{thm}\label{th1_adgn}
    Let $\xi\ge 0$, and $N_s>0$. The energy-constrained two-way capacities of the additive Gaussian noise $Q_2({\Lambda}_{\xi},N_s)$ and $K({\Lambda}_{\xi},N_s)$ vanish if and only if $\xi\ge1 $. In particular, the two-way capacities $Q_2({\Lambda}_{\xi})$ and $K({\Lambda}_{\xi})$ vanish if and only if $\xi\ge 1$.
\end{thm}
\begin{proof}
    Theorem~\ref{th1_adgn} is a direct consequence of Lemma~\ref{lemma_comp_bos} and Theorem~\ref{th1}.
\end{proof}
 It is an open question whether the coherent information lower bound in Eq.~\ref{lowQ2_add} equals the true two-way quantum capacity of the additive Gaussian noise $Q_2(\Lambda_{\xi})$: Theorem~\ref{th1_adgn} provides a negative answer to this question. Indeed, although $Q_2(\Lambda_\xi)=0$ if and only if $\xi\ge1$ (thanks to Theorem~\ref{th1_adgn}), the coherent information lower bound vanishes for all $\xi\ge\frac{1}{e}$. Hence, the coherent information lower bound is not equal to $Q_2(\Lambda_\xi)$ at least in the region $\xi\in [\frac{1}{e},1)$. In the following theorem we obtain an improved lower bound on the two-way capacities of the additive Gaussian noise.
\begin{thm}\label{th_delta_add}
Let $\xi\in [0,1)$ and $N_s\ge0$. The EC two-way capacities $Q_2(\Lambda_\xi,N_s)$ and $K(\Lambda_\xi,N_s)$ of the additive Gaussian noise $\Lambda_\xi$ satisfy the following lower bound
\bb\label{lowQ2_deltaEC_add}
	K(\Lambda_\xi,N_s)&\ge Q_2(\Lambda_\xi,N_s) \ge \sup_{\substack{c\in(0,1),\, M\in\N^+,\, k\in\N\\ (1-c^2)M\le N_s}} \mathcal{R}\left(1+\xi,\frac{1}{1+\xi},M,c,k\right)\,,
\ee
and, in particular, the unconstrained two-way capacities satisfy
\bb\label{lowQ2_delta_add}
	K(\Lambda_\xi)&\ge Q_2(\Lambda_\xi) \ge \sup_{c\in(0,1),\, M\in\N^+,\, k\in\N} \mathcal{R}\left(1+\xi,\frac{1}{1+\xi},M,c,k\right)\,,
\ee 
where the quantity $\mathcal{R}$ is defined in~\eqref{def_mathcal_R}. 
\end{thm}
\begin{proof} 
     Theorem~\ref{th_delta_add} is a direct consequence of Lemma~\ref{lemma_comp_bos} and Theorem~\ref{th_comp_Q2}.
\end{proof}
Theorem~\ref{th_delta_add} shows a new lower bound, reported in~\eqref{lowQ2_delta_add}, on the two-way capacities of the additive Gaussian noise $\Lambda_\xi$. Our new lower bound outperforms all the previous known lower bounds in a large region of the parameter $\xi$, as it can been seen from Fig.~\ref{add_vs_xi}.



\begin{figure}
\begin{tabular}{c}
  \includegraphics[width=0.97\linewidth]{Figures/figure_nu2_delta.pdf} \\  
(a)\\
 \includegraphics[width=0.97\linewidth]{Figures/ratio_delta.pdf} \\ 
(b) 
\end{tabular}
\caption{\textbf{(a).}~Bounds on the two-way quantum capacity of the thermal attenuator $Q_2(\mathcal{E}_{\lambda(\nu),\nu})$ plotted with respect to $\nu$, where the transmissivity is equal to the critical value $\lambda(\nu)\coloneqq 1-2^{-h(\nu)}$. The red curve is our lower bound calculated by exploiting~\eqref{lowQ2_delta}. The black curve is the best known lower bound on $Q_2(\mathcal{E}_{\lambda(\nu),\nu})$, which is the reverse coherent information lower bound reported in~\eqref{lowQ2} (which is zero since $\lambda(\nu)= 1-2^{-h(\nu)}$). The green curve is the PLOB upper bound reported in~\eqref{PLOB_Q2}. These bounds are also bounds on the secret-key capacity $K(\mathcal{E}_{\lambda(\nu),\nu})$. \textbf{(b).}~Ratio between our new lower bound on the two-way quantum and secret-key capacities in~\eqref{lowQ2_delta} and the PLOB bound in~\eqref{PLOB_Q2} as a function of $\nu$ where the transmissivity is $\lambda(\nu)\coloneqq 1-2^{-h(\nu)}$.}
\label{figure_nu2_delta}

\end{figure}
  
 

\begin{figure}[t]
	\centering
	\includegraphics[width=1\linewidth]{Figures/rate_vs_lambda1.pdf}
	\includegraphics[width=1\linewidth]{Figures/rate_vs_lambda10.pdf} 
	\caption{Bounds on the two-way quantum capacity of the thermal attenuator $Q_2(\mathcal{E}_{\lambda,\nu})$ plotted with respect to $\lambda$. The red line is our new lower bound obtained by exploiting~\eqref{lowQ2_delta}. The black line is the best known lower bound on $Q_2(\mathcal{E}_{\lambda,\nu})$, which is the reverse coherent information lower bound reported in~\eqref{lowQ2}. The green line is the PLOB upper bound reported in~\eqref{PLOB_Q2}. These bounds are also bounds on the secret-key capacity $K(\mathcal{E}_{\lambda,\nu})$.}
	\label{rate_vs_lambda1}
\end{figure}


\begin{figure}[t]
	\centering
	\includegraphics[width=1\linewidth]{Figures/secret_key_nu1.pdf}
	\includegraphics[width=1\linewidth]{Figures/secret_key_nu10.pdf} 
	\caption{Bounds on the secret-key capacity of the thermal attenuator $K(\mathcal{E}_{\lambda,\nu})$ plotted with respect to $\lambda$. The red line is our new lower bound obtained by exploiting~\eqref{lowQ2_delta}, the black line is the bound in~\eqref{lowQ2} calculated by evaluating the reverse coherent information in~\eqref{proof_lower}, the blue line is the best known lower bound discovered by~\cite{Ottaviani_new_lower}, and the green line is the PLOB upper bound reported in~\eqref{PLOB_Q2}.}
	\label{secret_key_nu}
\end{figure}


\begin{figure}[t]
	\centering
	\includegraphics[width=1\linewidth]{Figures/log1.pdf}
	\caption{Ratio between our lower bound on the two-way quantum and secret-key capacities of the thermal attenuator in~\eqref{lowQ2_delta} and the PLOB bound in~\eqref{PLOB_Q2} as a function of $\lambda$ for $\nu=1$.}
	\label{log1}
\end{figure}

 



 


\begin{figure}

\begin{tabular}{cc}
  \includegraphics[width=0.5\linewidth]{Figures/nu1Ns05.pdf} &   \includegraphics[width=0.5\linewidth]{Figures/nu10Ns05.pdf} \\
%(a) first & (b) second \\[1pt]
 \includegraphics[width=0.5\linewidth]{Figures/nu1Ns1.pdf} &   \includegraphics[width=0.5\linewidth]{Figures/nu10Ns1.pdf} \\
%(c) third & (d) fourth \\[1pt]
 \includegraphics[width=0.5\linewidth]{Figures/nu1Ns2.pdf} &  \includegraphics[width=0.5\linewidth]{Figures/nu10Ns2.pdf} \\
%(c) third & (d) fourth \\[1pt]
\end{tabular}
\caption{Bounds on the energy-constrained two-way quantum capacity  $Q_2(\mathcal{E}_{\lambda,\nu},N_s)$ and secret-key capacity $K(\mathcal{E}_{\lambda,\nu},N_s)$ of the thermal attenuator plotted with respect to $\lambda$ for different choices of $\nu$ and of the energy constraint $N_s$. The red line is our new lower bound obtained by exploiting~\eqref{lowQ2_deltaEC}, the black line is the NPJ lower bound~\cite{Noh2020} reported in~\eqref{npj_bound_therm}, the yellow line is the coherent information lower bound reported in~\eqref{EC_coh_therm_att}, the brown line is the reverse coherent information lower bound reported in~\eqref{EC_coh_therm_att}, the grey line is the DSW18 upper bound~\cite{Davis2018}, and the green line is the PLOB upper bound reported in~\eqref{PLOB_Q2}. }
\label{ECfigures}

\end{figure}






\begin{figure}

\begin{tabular}{cc}
  \includegraphics[width=0.5\linewidth]{Figures/multirail_nu01.pdf} &   \includegraphics[width=0.5\linewidth]{Figures/multirail_nu05.pdf} \\
%(a) first & (b) second \\[1pt]
 \includegraphics[width=0.5\linewidth]{Figures/multirail_nu1.pdf} &   \includegraphics[width=0.5\linewidth]{Figures/multirail_nu10.pdf} \\
%(c) third & (d) fourth \\[1pt]
%(c) third & (d) fourth \\[1pt]
\end{tabular}
\caption{Bounds on the two-way quantum capacity of the thermal attenuator $Q_2(\mathcal{E}_{\lambda,\nu})$ plotted with respect to $\lambda$. The blue line is our multi-rail lower bound obtained by exploiting~\eqref{multiplerail_low_bound}. The red line is our lower bound reported in~\eqref{lowQ2_delta}. The black line is the best known lower bound on $Q_2(\mathcal{E}_{\lambda,\nu})$, which is the reverse coherent information lower bound reported in~\eqref{lowQ2}. The green line is the PLOB upper bound reported in~\eqref{PLOB_Q2}. These bounds are also bounds on the secret-key capacity $K(\mathcal{E}_{\lambda,\nu})$.}
\label{multiplerail_figures}

\end{figure}







 

 \begin{figure}
\begin{tabular}{c}
  \includegraphics[width=1.0\linewidth]{Figures/cap_vs_nu_amp.pdf} \\  
(a)\\
 \includegraphics[width=1.0\linewidth]{Figures/ratio_vs_nu_amp.pdf} \\ 
(b) 
\end{tabular}
\caption{\textbf{(a).}~Bounds on the two-way quantum capacity of the thermal amplifier $Q_2(\Phi_{g(\nu),\nu})$ plotted with respect to $\nu$, where the gain is equal to the critical value $g(\nu)= \frac{1}{1-2^{-h(\nu)}}$. The red curve is our new lower bound calculated by exploiting~\eqref{lowQ2_delta_amp}. The black curve is the best known lower bound, i.e.~the coherent information lower bound reported in~\eqref{lowQ2_amp} (which is zero since $g(\nu)= \frac{1}{1-2^{-h(\nu)}}$). The green curve is the PLOB upper bound reported in~\eqref{PLOB_amp}. These bounds are also bounds on the secret-key capacity $K(\Phi_{g(\nu),\nu})$.  \textbf{(b).}~Ratio between our new lower bound in~\eqref{lowQ2_delta_amp} and the PLOB bound in~\eqref{PLOB_amp} as a function of $\nu$ where the gain is $g(\nu)\coloneqq \frac{1}{1-2^{-h(\nu)}}$.}
\label{capvsnu_amp}
\end{figure}

\begin{figure}[t]
	\centering
	\includegraphics[width=1\linewidth]{Figures/bound_vs_nu1_amp.pdf}
	\includegraphics[width=1\linewidth]{Figures/bound_vs_nu10_amp.pdf} 
	\caption{Bounds on the two-way quantum capacity of thermal amplifier $Q_2(\Phi_{g,\nu})$ plotted with respect to $g$. The red line is our new lower bound obtained by exploiting~\eqref{lowQ2_delta_amp}. The black line is the best known lower bound on $Q_2(\Phi_{g,\nu})$, which is the coherent information lower bound reported in~\eqref{lowQ2_amp}. The green line is the PLOB upper bound reported in~\eqref{PLOB_amp}. These bounds are also bounds on the secret-key capacity $K(\Phi_{g,\nu})$.}
	\label{bound_vs_nu1_amp}
\end{figure}

 
\begin{figure}[t]
	\centering
	\includegraphics[width=1\linewidth]{Figures/secret_g_nu1.pdf}
	\includegraphics[width=1\linewidth]{Figures/secret_g_nu10.pdf} 
	\caption{Bounds on the secret-key capacity of the thermal amplifier $K(\Phi_{g,\nu})$ plotted with respect to $g$. The red line is our new lower bound obtained by exploiting~\eqref{lowQ2_delta_amp}, the black line is the bound in~\eqref{lowQ2_amp} calculated by evaluating the coherent information in~\eqref{proof_lower_ampl}, the blue line is the WOGP lower bound~\cite{Ottaviani_new_lower}, and the green line is the PLOB upper bound reported in~\eqref{PLOB_amp}.}
	\label{secret_g_nu10}
\end{figure} 


\begin{figure}[t]
	\centering
	\includegraphics[width=1\linewidth]{Figures/log_ratio_vs_lam_amp.pdf}
	\caption{Ratio between our new lower bound on the two-way capacities of the thermal amplifier $\Phi_{g,\nu} $ in~\eqref{lowQ2_delta_amp} and the PLOB bound in~\eqref{PLOB_amp} as a function of $g$ for $\nu=1$.}
	\label{log_ratio_vs_lam_amp}
\end{figure}


 




\begin{figure}[t]
	\centering
	\includegraphics[width=1\linewidth]{Figures/add_vs_xi.pdf}
	\caption{Bounds on the two-way quantum capacity $Q_2(\Lambda_\xi)$ and secret-key capacity $K(\Lambda_\xi)$ of the additive Gaussian noise plotted with respect to $\xi$. The red line is our new lower bound obtained by exploiting~\eqref{lowQ2_delta_add}. The black line is the best known lower bound on $Q_2(\Lambda_\xi)$, which is the coherent information lower bound reported in~\eqref{lowQ2_add}. The green line is the PLOB bound reported in~\eqref{PLOB_add}.}
	\label{add_vs_xi}
\end{figure}

\end{document}


