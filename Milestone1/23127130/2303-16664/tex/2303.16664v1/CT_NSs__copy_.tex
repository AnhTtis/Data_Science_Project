\documentclass[prd,eqsecnum,tightenlines,11pt]{revtex4}
%\documentclass[12pt]{article} 
%\usepackage[T1]{fontenc}              
%\usepackage[utf8]{inputenc}    
%\usepackage{lipsum}
%\usepackage{lmodern}
%\usepackage{appendix}
%\usepackage{amsmath,amsfonts,amssymb}
%\usepackage{geometry}
%\usepackage{color}
%\usepackage{graphicx}
%\usepackage{subfig}
%\usepackage{array}
%\usepackage{fancyhdr}
%\usepackage{authblk}
%\usepackage[english]{babel}
%\geometry{left=2cm,right=2cm,top=2cm,bottom=2cm}       

\usepackage{hyperref}
\usepackage{graphicx}
\usepackage{amsmath,amssymb,amsfonts,amsthm,latexsym,stmaryrd}
\usepackage{marginnote}
\usepackage{color}
\usepackage{soul}
\renewcommand{\textfraction}{.1}






\begin{document}

\title{Axial and polar stability of neutron stars in scalar-tensor theories with disformal coupling}
\author{Hamza Boumaza}
\affiliation{\small Laboratoire de Physique des Particules et Physique Statistique (LPPPS),\\
 Ecole Normale Supérieure-Kouba, B.P. 92, Vieux Kouba, 16050 algiers, algeria}
%\affiliation{Laboratory of Theoretical Physics and Department of Physics,\\
%Faculty of Exact and Computer Sciences, University Mohamed Seddik Ben Yahia, BP 98, Ouled aissa, Jijel 18000, algeria}


\begin{abstract}
{\hskip 2em}In this present work, we study the radial and non-radial perturbation stability of neutron stars in which the matter is disformally coupled to the metric. First, we set the equations of the neutron star in a static and spherically symmetric background. Then, we calculate the second-order perturbations of the action that describes our theory for both axial and polar modes. From the resulting expressions, we derive the conditions to avoid the Laplacian and ghost instabilities at the center of the star and when the radial coordinate tends to infinity. In addition, a numerical analysis is performed to investigate the stability of a particular model with a constant disformal function denoted as $\Lambda$. We found that the chosen model is stable against the Laplacian instability in a small range of the constant $\Lambda$.
 \end{abstract}

\maketitle

\section{Introduction:}

 Although the complexity of observing black holes or neutron stars with high precision means that general relativity (GR) has not yet been tested in strong gravity, where we can expect new extensions of GR and new physical phenomena, it has been the most successful and powerful gravitational theory in describing stellar objects and gravitational waves (GWs).  The theory has also successfully explained the existence of compact objects such as neutron stars (NSs) and black holes (BHs), which are the main sources of the recently detected gravitational waves by the LIGO/Virgo Collaboration  \cite{abbott2016gw150914,Abbott_2017,armengol2017neutron,LIGOScientific:2017vwq,abbott2018prospects,abbott2018gw170817,abbott2020gw190814}. These recent measurements have opened a new window for studying the physics of dynamic space-time characteristics in strong regimes through the study of gravitational wave signals. More precise measurements will allow us to understand the structure of compact objects, particularly the nature of the matter in the core of neutron stars.  

 An other interesting issue which can be probed by the analysis of GWs signals is the  modification of GR in the region with strong gravity. This modifications, in many modified theories of gravity spatially scalar-tensor theory of gravity \cite{fujii2003scalar}, is a consequence of the presence of new degree of freedom which corresponds to a scalar field. The most general scalar-tensor theory of gravity with second order derivatives and provides partial equations of second order is Horndeski theory \cite{Horndeski:1974wa}. This theory is generalized to beyond Horndeski \cite{Gleyzes:2014qga} and then to Degenerate-Higher-Order-Scalar-Tensor (DHOST) theories \cite{Langlois:2015cwa}, where they contains higher derivatives but without surfing from Ostrogradski's instability \cite{ostrogradsky1850memoires}. Moreover, Horndeski gravity and DHOST theories can be mapped to each others through a conformal-disformal transformations \cite{BenAchour:2016cay}. In fact, the scalar field has not only played  an important role to have a better understanding on the late acceleration of the universe \cite{Boumaza:2020klg,Langlois:2018dxi,Langlois:2017dyl}, black holes \cite{BenAchour:2020wiw,Minamitsuji:2019shy,Motohashi:2019sen} and neutron stars \cite{Boumaza:2021fns,Ogawa:2019gjc,Boumaza:2022abj,Babichev:2016jom,Boumaza:2021lnp} in  these theories but it was the principle cause of  the deviation from GR. In this paper will limit our study to neutron stars.

 An interesting subfamily of scalar tensor theory has been studied intensively in the literature, in which the metric is  coupled to the matter via conformal transformation \cite{Ramazanoglu:2016kul,Yazadjiev:2016pcb,Harada:1998ge},  to describe the neutron star profile. Due to the tachyonic instability, we found that the neutron stars, with negative sacalar field mass, are spontaneously scalarized for small range of the parameter in the model \cite{Damour:1996ke,Freire:2012mg}. The phenomenological implications of spontaneous scalarization have been explored for  static neutron star in many situations, as an example: Slowly rotating NSs \cite{Sotani:2012eb,Pani:2014jra} where they reported that the scalar field can modify the relation between the mass and the momentum of inertia. In addition, the amplitude of gravitation wave, sourced by the collapse of Spherical neutron star collapse toward a black hole,  can be modified by the scalar field \cite{Novak:1997hw}. Other phenomena can be produced in this kind of theory, (see Ref.\cite{Harada:1996wt}).
 
Spontaneous scalarization can also occur when matter is coupled to the metric via disformal transformations \cite{Minamitsuji:2016hkk}, which are the most general transformations of the metric that preserve the causality principle \cite{Bekenstein:1992pj}. The disformal transformation, studied in Ref. \cite{Minamitsuji:2016hkk}, is a generalization of conformal transformations that preserves the mathematical structure of Horndeski theory \cite{Bettoni:2013diz}. Moreover, the disformal coupling of matter has gained considerable interest in recent years, where it has been studied in dark sectors \cite{Sakstein:2015jca,Zumalacarregui:2010wj}, black holes \cite{Koivisto:2015mwa,Erices:2021uyu}, and neutron stars \cite{Ikeda:2021skk}. In this paper, we propose to study the stability of relativistic stars by deriving ghost and Laplacian instabilities from the quadratic-order action for both axial and polar perturbations.

The decomposition of the metric into axial and polar modes is a powerful tool for studying the resonant frequencies and damping times of gravitational waves produced by compact objects, using the quasi-normal mode (QNM) formalism \cite{kokkotas1992w,kokkotas1999quasi}. Investigation in this subject not only helps us understand the construction of neutron stars but also provides new information about modifications to general relativity (GR). However, these modifications should not suffer from ghost and Laplacian instabilities, as such instabilities would lead to unstable solutions to the equations that describe gravity. The stability of relativistic stars has been studied in several models belonging to scalar-tensor theories, including Horndeski theories \cite{Kase:2020yjf,Kase:2021mix}, Gauss-Bonnet couplings \cite{Minamitsuji:2022tze}, and scalar-tensor theories with a nonminimal coupling \cite{Kase:2020qvz}. 

The  paper is organized as follow. After introducing, in the next section \ref{1}, we construct the theory of scalar tensor theories with two metric linked to each other via the disformal transformation, then we derive the main equations for a static and spherically symmetric configuration. We also derive the asymptotic  behavior of the metric, scalar field and the matter at the center of the NSs and at large distance. In the  section \ref{2}, we extend  our analysis by considering small perturbation around the static and spherical metric. Then, after we split the perturbed metric into odd- and even-parity mode, we determine the conditions of ghost and Laplacian instabilities. In section \ref{3}, we solve the system for two realistic equations of state, and obtain a continuum of neutron star solutions parametrized by their central energy density as well as we show the Laplacian instability. We finally give some conclusions and perspectives in the final section. 




\section{Neutron stars in scalar-tensor theories with disformal coupling:}\label{1}

 In this section, we will derive the background equations of scalar-tensor theories in which the physical metric $\tilde{g}_{\alpha\beta}$ is coupled to a geometric metric $g_{\alpha\beta}$ through the disformal transformation, 
\begin{eqnarray}\label{disformaltransformation}
\tilde{g}_{\alpha\beta}&=& \text{C}(\varphi)\;\left(g_{\alpha\beta} + \text{D}(\varphi)\partial_{\alpha}\varphi\partial_{\beta}\varphi\right),
\end{eqnarray}
where $\text{C}$ and $\text{D}$ are arbitrary functions of $\varphi$. The metric that describes our universe is denoted as $\tilde{g}_{\alpha\beta}$ and is called the metric in Jordan frame. On the other hand, $g_{\alpha\beta}$, referred to as the metric in Einstein frame, is used to describe the geometrical side of the Einstein-Hilbert action.
\subsection{The model}
The total action of the scalar-tensor theories with disformal coupling in Einstein frame is written as
\begin{eqnarray}\label{totalaction S}
S &=& \int \, d^4 x \sqrt{-\text{g}}\left(\frac{\kappa}{2}R+\frac{\text{a}}{2}X\right)+ S_m(\tilde{g}_{\alpha\beta},\Psi),
\end{eqnarray}
where $X=\text{g}^{\alpha\beta}\partial_{\alpha}\varphi\partial_{\beta}\varphi$, $\Psi$ and $R$ are  the kinetic therm, the matter field and the Ricci scalar, respectively. $\kappa$ is a constant equal to $c^4/(8\pi G)$ with $G$ is Newton constant. We note that this theory is an extension to  general relativity equations, which is recovered by setting $\text{C}=1$, $\text{a}=0$ and $\text{D}=0$, and to purely conformal transformation if we set only $\text{D}=0$.


The matter present in the neutron star is represented by the second term in the action (\ref{totalaction S}), $S_m$. We will chose action formulated by  Schutz-Sorkin \cite{schutz1970perfect,schutz1977variational,brown1993action}.
\begin{eqnarray}\label{Sm}
S_m &=& \int \, d^4 x \sqrt{-\tilde{g}} \tilde{P}\left[\tilde{\mu}\right],
\end{eqnarray}

The notations $\tilde{\mu}$ and $\tilde{P}$ correspond to the chemical potential and the pressure of the matter in Jordan frame, respectively. In the action (\ref{Sm}), we considered that the entropy per particle is constant, i.e. the fluid is at the equilibrium state, and thus the pressure depends only on the chemical potential. On one hand, this means  that the fourth vector velocity of the fluid is defined as \cite{schutz1970perfect,brown1993action}
\begin{eqnarray}\label{u}
\tilde{u}_\alpha &=& \frac{1}{\tilde{\mu}}\left(\partial_\alpha \tilde{q}+ A \partial_\alpha B\right),
\end{eqnarray}
where $l$, $A$ and $B$ are scalar fields. From the above equation, we can easily determine $\tilde{\mu}$ using the constraint $\tilde{u}_\alpha\tilde{u}_\beta \tilde{g}^{\alpha\beta}=-1$ and can derive  the following constraints, from (\ref{Sm}),
\begin{eqnarray}\label{cnstrntaB}
\tilde{u}_\alpha\partial^\alpha A=0,\quad\tilde{u}_\alpha\partial^\alpha B=0.
\end{eqnarray}
and on the other hand, using the first  law of thermodynamics is 
\begin{eqnarray}
d\tilde{P} &=& \tilde{n} \; d\tilde{\mu},
\end{eqnarray}
where $\tilde{n}$ is the number density in Jordan frame, we define also  

\begin{eqnarray}
\tilde{n} &=& \frac{\partial \tilde{P}}{\partial \tilde{\mu}}.
\end{eqnarray}
 Having defined  $\tilde{u}_\alpha$ and $\tilde{g}_{\alpha\beta}$, one can use these definitions to describe the fluid equations. We will see this in the next sections.
\subsection{Background equations}
 Now, let's suppose a static and spherical symmetric spacetime
\begin{eqnarray}\label{ds0}
ds^2 = -f(r) dt^2 + h(r) dr^2 + r^2 \left(d\theta^2 + \sin^2\theta d\phi^2\right),
\end{eqnarray}
where the metrics $f$ and $h$ are  expressed in Einstein frame. at this level, we also suppose that the scalar filed and the thermodynamics variables depend only on the radial coordinator $r$, i.e. $\varphi\equiv\varphi(r)$, $\tilde{P}\equiv\tilde{P}(r)$, $\tilde{\mu}\equiv\tilde{\mu}(r)$....ex. The fourth vector velocity in the space time (\ref{ds0}) is given by
\begin{eqnarray}
\tilde{u}_\alpha = \left\lbrace \sqrt{f},0,0,0 \right\rbrace.
\end{eqnarray}
Therefore, in a static and spherical space time the solutions of equations (\ref{cnstrntaB}) are
\begin{eqnarray}
A=A(r,\theta,\phi),\quad B=B(r,\theta,\phi),
\end{eqnarray}
which left us with a free choice of the functions $a$ and $B$. In our paper, we will chose 
\begin{eqnarray}
A=0,\quad B=0.
\end{eqnarray}
Therefore, integrating the component $t$ of the Eq.(\ref{u}) with respect to $t$, results
\begin{eqnarray}
\tilde{q}(t,r)&=& -\sqrt{\text{C}}\sqrt{f}\tilde{\mu} t.
\end{eqnarray}
Then substituting this result in the the component $r$ of the Eq.(\ref{u}), we obtain the following constraint
\begin{eqnarray}\label{muequation}
\frac{\tilde{\mu}'}{\tilde{\mu}(r)}&=& -\frac{  \text{C}_\varphi \varphi'}{2 \text{C}}-\frac{f'}{2 f},
\end{eqnarray}
where the prime denotes the radial derivative and the subscribe $\varphi$ represents the derivative with respect to $\varphi$. The matter conservation equation can deduced from (\ref{muequation}) as
\begin{eqnarray}
\frac{\tilde{P}'}{\tilde{P}+\tilde{\rho}}&=&-\frac{  \text{C}_\varphi \varphi'}{2 \text{C}}-\frac{f'}{2 f}.
\end{eqnarray}
where we have used the definition of energy density $\tilde{\rho}$ 
\begin{eqnarray}
\tilde{\rho} &=& \frac{\partial \tilde{P}}{\partial\tilde{\mu}}\tilde{\mu}-\tilde{P},
\end{eqnarray}
to eliminate $\tilde{\mu}$. In our representations, the disformal function $\text{D}$ does not appear in the equation of the matter conservation. As a result, the new therm can disappear if we express $f$ in Jordan frame by  performing the disformal transformation (\ref{disformaltransformation}).

To derive the equations of motion, we first substitute the metric (\ref{ds0}) in the action (\ref{totalaction S}) and second we add the constraint (\ref{muequation}) to the total action using the Lagrange multiplier $\lambda$. Then, we write the action new as
\begin{eqnarray}\label{s0}
S_{0}&=&\int dr \left[\frac{\sqrt{f} }{h^{3/2}}\left(\kappa( r h'+(h-1) h )-\text{a}\,h r^2 \, \varphi '^2\right)+\sqrt{f} r^2 \text{C}^{3/2} P \sqrt{h + \text{D} \varphi '^2}+ \lambda\, (\tilde{\mu}\sqrt{\text{C}}\sqrt{f})'\right].\nonumber\\
\end{eqnarray}
Varying $S_0$ with respect to $\tilde{\mu}$, we have
\begin{eqnarray}\label{lamda}
\lambda' &=& r^2  \text{C}^{3/2} \frac{\partial \tilde{P}}{\partial \tilde{\mu}} \sqrt{h+ \text{D} \varphi'^2}.
\end{eqnarray}
The Euler langrage equation corresponding to $f$ and $h$ are derived as
\begin{eqnarray}
\frac{h'}{h}&=& \frac{h-1}{r}+\frac{r \text{a}}{\kappa } \varphi'^2+\frac{r }{\kappa }h^{1/2} \text{C}^2 \sqrt{h+\text{D} \varphi'^2}\tilde{\rho} ,\label{eh}\\
\frac{f'}{f}&=&\frac{h-1}{r}+\frac{\text{a} }{\kappa }\varphi'^2+\frac{r h^{3/2}  \text{C}^2}{\kappa \sqrt{h+\text{D}\varphi'^2}}\tilde{P},\label{ef}
\end{eqnarray}
and, the equation of the scalar field in static and symmetric space time is written as 
\begin{eqnarray}
&&\varphi '' \left(1-\frac{h^{3/2}  \text{C}^2 \text{D}}{2 \text{a} \left(h+\text{D} \varphi '^2\right)^{3/2}}\tilde{P}\right)+h' \left(\frac{\sqrt{h}  \text{C}^2 \text{D} \varphi '}{4 \text{a} \left(h+\text{D} \varphi '^2\right)^{3/2}}\tilde{P}-\frac{\varphi '}{2 h}\right)+f' \left(\frac{\sqrt{h}  \text{C}^2 \text{D} \varphi '}{4 f \text{a} \sqrt{h+\text{D} \varphi '^2}}\tilde{\rho} +\frac{\varphi '}{2 f}\right)\nonumber\\
&&+\frac{2 \varphi '}{r}-\frac{\sqrt{h} \text{C}^2 \text{D} \varphi '}{r \text{a} \sqrt{h+\text{D} \varphi '^2}}\tilde{P} +\frac{h^{3/2}  (3 \tilde{P}-\tilde{\rho} ) \text{C}_\varphi}{4 \text{a} \sqrt{h+\text{D} \varphi '^2}}-\frac{h^{3/2} \text{C}^2 \varphi '^2 \text{D}_\varphi}{4 \text{a} \left(h+\text{D} \varphi '^2\right)^{3/2}} \tilde{P}=0.\label{ephi}
\end{eqnarray}
Note that the radial derivatives of functions $f$ and $h$ are eliminated from the above equation by using Eqs. (\ref{eh}) and (\ref{ef}). Finally, we complete our differential system of equations with the equation of state
\begin{eqnarray}
\tilde{P}=\tilde{P}(\tilde{\rho}).
\end{eqnarray}
Hence, for a particular expression of the functions $\text{C}$ and $\text{D}$, one can integrate the equations (\ref{eh}), (\ref{ef}) and (\ref{ephi}).

\subsection{asymptotic behaviors:}

 It is difficult to find analytic solution of the  Eqs (\ref{eh}), (\ref{ef}) and (\ref{ephi}), but one can  have the asymptotic solutions at the center of the stars and at large distance.

\subsubsection{at large distance:}\label{larger}
Since there is no matter at large distance, the energy density and the pressure are vanished and the metric in Jordan frame approaches a flat space time. In fact, the asymptotic behaviors of the scalar field and the metric in Einstein frame are developed as follow
\begin{eqnarray}
f =1+\sum_{i=1}^{i=n} \frac{f_{i}^{\infty}}{r^i},\qquad h =\sum_{i=0}^{i=n} \frac{h_{i}^{\infty}}{r^i}\quad\text{and}\quad \varphi =\sum_{i=0}^{i=n} \frac{\varphi_{i}^{\infty}}{r^i}.\label{dev1}
\end{eqnarray}
Note that we have chosen $f_{0}^{\infty}=1$. Up to fourth order, the developments (\ref{dev1}) are derived as follow
\begin{eqnarray}
f  &\sim & 1 +\frac{2\text{M}}{r}-\frac{4\text{M}\text{Q}^2 \text{a}}{3 \kappa r^3} + \frac{8\text{M}^2\text{Q}^2 \text{a}}{3 \kappa  r^4},\\
h  &\sim & 1 -\frac{2\text{M}}{r}+\frac{\frac{20\text{M}\text{Q}^2 \text{a}}{\kappa }-8\text{M}^3}{r^3}+\frac{4\text{M}^2-\frac{4\text{Q}^2 \text{a}}{\kappa }}{r^2}+\frac{16 \left(3 \kappa ^2\text{M}^4-13 \kappa\text{M}^2\text{Q}^2 \text{a} +3\text{Q}^4 \text{a}^2\right)}{3 \kappa^2 r^4},\\
\varphi &\sim & \varphi_0^\infty +\frac{2\text{Q}}{r}-\frac{2\text{Q}\text{M}}{r^2}+\frac{\frac{8\text{M}^2\text{Q}}{3}-\frac{4\text{Q}^3 \text{a}}{3 \kappa}}{r^3}+\frac{\frac{16\text{M}\text{Q}^3 \text{a}}{3 \kappa }-4\text{M}^3\text{Q}}{r^4},
\end{eqnarray}
where $\text{M}$ and $\text{Q}$ are constants of integrations and they correspond to the mass of the star in Einstein frame and to the scalar field charge, respectively. In Jordan frame, the function $\tilde{h}=\text{C}h +\text{D} \varphi'^2$ is 
\begin{eqnarray}
\tilde{h} &\sim & \text{C}^{\infty}+\frac{2\text{Q} \text{C}^{\infty}_{\varphi}-2\text{M} \text{C}^{\infty}}{r}
+\frac{2\text{Q}^2 \text{C}^{\infty}_{\varphi\varphi}-6\text{M}\text{Q} \text{C}^{\infty}_{\varphi}+\text{C}^{\infty} \left(4\text{M}^2-\frac{4\text{Q}^2 \text{a}}{\kappa}\right)}{r^2}\nonumber\\
&&+\frac{\frac{4}{3}\text{Q}^3 \text{C}_{\varphi\varphi\varphi}^\infty-8\text{M}\text{Q}^2 \text{C}^{\infty}_{\varphi\varphi}+\frac{4}{3}\text{Q} \text{C}^{\infty}_{\varphi} \left(11\text{M}^2-\frac{7\text{Q}^2 \text{a}}{\kappa}\right)+\text{C}^{\infty} \left(\frac{20\text{M}\text{Q}^2 \text{a}}{\kappa}-8\text{M}^3\right)}{r^3}\nonumber\\
&&+\left(\frac{2}{3}\text{Q}^4 \text{C}_{\varphi\varphi\varphi\varphi}^\infty-\frac{20}{3}\text{M}\text{Q}^3 \text{C}_{\varphi\varphi\varphi}^\infty+\text{C}^{\infty}_{\varphi\varphi} \left(\frac{70\text{M}^2\text{Q}^2}{3}-\frac{32\text{Q}^4 \text{a}}{3 \kappa}\right)+\text{C}^{\infty}_{\varphi} \left(\frac{56\text{M}\text{Q}^3 \text{a}}{\kappa}-\frac{100\text{M}^3\text{Q}}{3}\right) \right.\nonumber\\
&&\left. +\text{C}^{\infty} \left(4\text{Q}^2 \text{D}^\infty+\frac{16 \left(3 \kappa^2\text{M}^4-13 \kappa\text{M}^2\text{Q}^2 \text{a}+3\text{Q}^4 \text{a}^2\right)}{3 \kappa^2}\right)\right)/r^4,
\end{eqnarray}
where $\text{C}^{\infty}=\text{C}(\varphi(\infty))$. The disformal functions appears at the fourth order which means that  its contribution is negligible at infinity unlike the conformal function appears at all orders. If we wish to have $\tilde{h}\sim 1$ at large distance, we must impose a function $\text{C}^{\infty}\sim 1$ when $r\rightarrow\infty$. In this case the physical mass of the neutron star is
\begin{eqnarray}
\tilde{M}= \lim_{r\rightarrow\infty}   \frac{ r(\tilde{h}-1)}{2} =\text{M}-\text{Q} \text{C}^{\infty}_{\varphi}.
\end{eqnarray}
In purely disformal transformation $\text{C}=1$, we find that the mass of the star in the two frames are identical.

\subsubsection{at the center of the star:}
In order to, show the physical boundary conditions of the neutron star, we need to derive the behavior of $f$, $h$, $\tilde{P}$ and $\varphi$ when $r$ is close to $0$. At the center of star, we must have a regular behavior. In other words, $f$, $h$, $\tilde{P}$ and $\varphi$ must be  in the form

\begin{eqnarray}
f =f_{0}^{c}+\sum_{i=2}^{i=n} f_{i}^{c}r^i,\qquad h =h_{0}^{c}+\sum_{i=2}^{i=n} h_{i}^{c}r^i,\qquad \tilde{P} =\tilde{P}_{0}^{c}+\sum_{i=2}^{i=n} \tilde{P}_{i}^{c}r^i\quad\text{and}\quad \varphi =\varphi_{0}^{c}+\sum_{i=2}^{i=n} \varphi_{i}^{c}r^i.\label{dev2}
\end{eqnarray}
We can see that the first order derivative of these function vanished at the center. We evaluate these polynomial to the second order as
\begin{eqnarray}
f &\sim & f_{0}^{c}+\frac{f_{0}^{c} (3 \tilde{P}^c_0+\tilde{\rho}^c_0)}{6 \kappa } (\text{C}^{c}r)^2,\\
h & \sim & 1+ \frac{ (\tilde{\rho}^c_0)}{3 \kappa } (\text{C}^{c}r)^2,\\
\varphi &\sim & \varphi_{0}^{c}+\frac{(\tilde{\rho}^c_0-3 \tilde{P}^c_0) (\text{C}^{c}_\varphi/\text{C}^c )}{24 \text{a}-12 \tilde{P}^c_0 (\text{C}^{c})^2 \text{D}^c} (\text{C}^{c}r)^2,\label{varphic}\\
\tilde{P} &\sim & \tilde{P}_{0}^{c}-\frac{1}{24} (\tilde{P}^c_0+\tilde{\rho}^c_0) \left(\frac{(\tilde{\rho}^c_0-3 \tilde{P}^c_0) (\text{C}^{c}_\varphi/\text{C}^{c} )^2}{2 \text{a}-\tilde{P}^c_0 (\text{C}^c)^2 \text{D}^{c}}+\frac{2  (3 \tilde{P}^c_0+\tilde{\rho}^{c}_{0})}{\kappa }\right)(\text{C}^{c}r)^2.\label{Pc}
\end{eqnarray}
In the last two approximations, the Taylor development breaks down for high pressure which is due to the singularity that appears for high pressure and disappears definitely for $\text{D}=0$. However, we can over come this problem by considering  low values of the functions $\text{D}$ such us $\lvert 1- \tilde{P}^c_0 (\text{C}^{c})^2 \text{D}^c/2\rvert \ll 1 $. Note that same singularity appear in Jordan frame
\begin{eqnarray}
\tilde{h} &\sim & \text{C}^c+\left(\frac{4 \text{C}^{c} \text{D}^{c} (\tilde{\rho}^{c}_0-3 \tilde{P}^{c}_0)^2 (\text{C}^{c}_\varphi)^2}{\left(24 \text{a}-12 \tilde{P}^{c}_0 (\text{C}^{c})^2 \text{D}^{c}\right)^2}+\frac{ (\tilde{\rho}^{c}_0-3 \tilde{P}^{c}_0) (\text{C}^{c}_\varphi/\text{C}^{c})^2}{24 \text{a}-12 \tilde{P}^{c}_0 (\text{C}^{c})^2 \text{D}^{c}}+\frac{\tilde{\rho}^{c}_0(\text{C}^{c})}{3 \kappa }\right)( \text{C}^c r)^2.
\end{eqnarray}
In the case of purely disformal transformation $\text{C}=1$, this approximation is reduced to
\begin{eqnarray}
\tilde{h} &\sim &1+\frac{\tilde{\rho}^{c}_0}{3 \kappa }  r^2.
\end{eqnarray}
which correspond to the GR, and thus the function $\text{D}$ does not affect the behavior of the metric at the center of the star. These results were also found in Ref.\cite{minamitsuji2016relativistic}.
\section{Ghost and Laplacian instabilities:}\label{2}
 To investigate the ghost and Laplacian instabilities, we consider metric perturbations $ds_{per}^2$ in Einstein frame around the metric (\ref{ds0}) as follow
\begin{eqnarray}
ds_{tot}^2 &=& ds^2 + ds_{per}^2,
\end{eqnarray}
where $ds_{per}^2$ is decomposed to polar and axial parts. The polar part has even parity while the axial one has odd parity by the rotation in two dimensions plan $(\theta,\phi)$. This decomposition will allow us to study  the equations of motion  at the first order separately. In this paper, for the even perturbations, we will  choose the uniform curvature gauge
\begin{eqnarray}
ds_{per,even}^2 &= f H_0^{lm} Y_{lm}dt^2+2\sqrt{fh} H_1^{lm} Y_{lm}dtdr+h H_2^{lm} Y_{lm}dr^2\nonumber\\
&+2\sqrt{h}H_5^{lm} (\partial_\theta Y_{lm} r d\theta +\partial_\phi Y_{lm} r\sin\theta d\phi)dr.
\end{eqnarray}
where $H_0^{lm}$ , $H_1^{lm}$, $H_2^{lm}$ and $H_5^{lm}$ depend on the coordinates $t$ and $r$ and $Y_l^m(\theta,\phi)$ are the spherical Harmonic functions with the integers $l$ and $m$. For the axial perturbation, we chose  the Regge-Wheeler gauge
\begin{eqnarray}
ds_{per,odd}^2 &=& 2 r^2\sin \theta \left(\partial_\theta Y_{lm}+\frac{1}{\sin\theta}\partial_\phi Y_{lm}\right) \left(h_1^{lm} dtd\theta +h_0^{lm} dtd\phi \right),
\end{eqnarray}
where $h_0^{lm}$ and $h_1^{lm}$ are functions of the coordinates $t$ and $r$. 
 In the polar perturbation, the scalar field  and $\tilde{q}$ are perturbed, in therms of the spherical harmonics $Y_{lm}$, as
\begin{eqnarray}\label{scalarper}
\varphi = \varphi(r)+Y_{lm}\,\delta\varphi^{lm},\;\;\text{and}\;\; \tilde{q}= -\sqrt{\text{C}}\sqrt{f}\tilde{\mu} t+ Y_{lm}\,\delta\tilde{q}^{lm}.
\end{eqnarray}
But, they are equal to in axial case. We leave the scalar functions $A$ and $B$ without using the spherical harmonics, i.e.,
\begin{eqnarray}\label{ABper}
A= \delta A(t,r,\theta,\phi)\;\;\text{and}\;\; B= \delta B(t,r,\theta,\phi).
\end{eqnarray}
 In addition, the  vector velocity of the fluid components are given by
\begin{eqnarray}\label{muper}
\delta\tilde{u}^\alpha =\left( \begin{matrix}\delta\tilde{u}^{t,lm}Y_{lm}\\
                                       \delta\tilde{u}^{r,lm}Y_{lm}\\
                                      \frac{\sqrt{f}}{\text{C}r^2} v_p \partial_\theta Y_{lm}+v_a	\frac{1}{\sin\theta}\partial_\phi Y_{lm}\\
                                        v_a\partial_\theta Y_{lm}+\frac{\sqrt{f}}{\text{C}r^2}v_p \frac{1}{\sin\theta}\partial_\phi Y_{lm}

             \end{matrix}\right)+\delta^2 \tilde{u}(t,r,\theta ,\phi),
\end{eqnarray}
where second order perturbation is essential to satisfy the condition
\begin{eqnarray}\label{noramalisation2}
\delta^2 \left[\tilde{u}_\alpha\tilde{u}_\beta \tilde{g}^{\alpha\beta}\right]=0.
\end{eqnarray}
 Finally, using the Eqs. (\ref{u}) and (\ref{noramalisation2}), we deduce 
 \begin{eqnarray}\label{mu2}
 \tilde{\mu} &= \tilde{\mu}(r)\left(1+\delta\tilde{\mu}^{lm}Y_{lm}+\delta^2 \tilde{\mu}(t,r,\theta,\phi)\right).
 \end{eqnarray}
 
 We will show the expression of the therm of  second order in the next subsections for the polar and axial cases. In the following, we will set $m=0$, and thus we remove the superscribe $lm$ from the perturbed metric for both modes.
\subsection{Polar perturbations:}

 In the even parity sector, using the definition (\ref{u}) and the expansions (\ref{scalarper}), (\ref{muper}) and (\ref{mu2}), we find the equations
\begin{eqnarray}
\frac{1}{\tilde{\mu}}\left(\frac{\delta\tilde{q}}{\text{C}}\right)'&=& \frac{1}{2} r^2\sqrt{\text{C}}  \text{C}_\varphi\varphi'  v_p+\frac{ \text{C} \text{D}\varphi'}{\sqrt{f}} \delta\dot{\varphi}+ \text{C}^{3/2}  \left(\text{D}\varphi'^2+h\right)\delta\tilde{u}^r+\sqrt{h}  \text{C} H_1, \label{c1}\\
\frac{1}{\tilde{\mu}\sqrt{f}}\left(\frac{\delta \dot{\tilde{q}}}{\text{C}}\right)&=& -\frac{1}{2}    \text{C}_\varphi \delta\varphi -  \text{C} \delta\tilde{\mu} +\frac{1}{2}   \text{C} H_0  ,\label{c2}\\
\frac{l(l+1)}{r^2 \tilde{\mu} }\frac{\delta\tilde{q}}{\text{C}^{5/2}} &=&  l(l+1)\,v_p,\label{c3}
\end{eqnarray}
where the prime denotation stands for the time derivative, and 
\begin{eqnarray}
\frac{\delta^2 \tilde{\mu}}{\text{C}} &=& \frac{1}{2} Y_{l0}^2 \left(\left( \text{C} H_0-\frac{1}{2} \delta\varphi \text{C}_\varphi\right) \delta\tilde{\mu}-   \text{C}_\varphi H_0  \delta\varphi+\frac{  \text{C}_\varphi^2}{4 \text{C}}\delta\varphi^2-  \text{C}^2 (\delta\tilde{u}^{r})^2 \left(\text{C} \varphi'^2+h\right)+\frac{1}{4}  \text{C} H_0^2\right)\nonumber\\
&&-\frac{1}{2} r^2 \text{C}^2 \partial_\theta Y_{l0}^2 v_p^2-\frac{\sqrt{\text{C}} }{\sqrt{f} \tilde{\mu}}\delta A \delta\dot{B}.\label{mut2p}
\end{eqnarray}
However, from the equations (\ref{cnstrntaB}), we have 
\begin{eqnarray}
\delta\dot{A}=0\;\;\text{and}\;\;\delta\dot{B}=0\;\;\Leftrightarrow\;\;\delta A=A(r,\theta,\phi)\;\;\text{and}\;\;\delta B=B(r,\theta,\phi),
\end{eqnarray}
which means that these scalar function will not modify the structure of the  equations at this stage. Combining Eq. (\ref{c1}) with Eq. (\ref{c3}) and Eq. (\ref{c1}) with Eq. (\ref{c2}), gives
\begin{eqnarray}
&& E_0\equiv v_p'-\frac{\sqrt{\text{C}}}{\sqrt{f}} \left(h+\text{D} \varphi '^2\right)\delta\tilde{u}^r -\frac{\text{D} \varphi ' \delta\dot{ \varphi}}{f}-\sqrt{h} H_1 =0,\label{c4}\\
&& E_1\equiv \dot{v}_p+\delta\tilde{\mu}+\frac{\text{C}_\varphi}{2\sqrt{\text{C}}} \delta\varphi  -\frac{1}{2} H_0  =0.\label{c5}
\end{eqnarray}
Thus, from these constraints and by combining Eq. (\ref{c1}) with Eq. (\ref{c2}), we find a third constrain of the form
\begin{eqnarray}\label{c6}
E_2\equiv (\delta\tilde{\mu}+\frac{\text{C}_\varphi}{2\text{C}}\delta\varphi)'+\frac{\sqrt{\text{C}}  \left(h+\text{D} \varphi '^2\right)}{\sqrt{f}}\dot{\delta\tilde{u}}^r+\frac{\text{D} \varphi '}{f}\ddot{ \delta \varphi}+\frac{\sqrt{h} }{\sqrt{f}}\dot{H}_1-\frac{1}{2} H_0'=0.
\end{eqnarray}
We choose to introduce the Eq. (\ref{c6}) and (\ref{c5}) as constraint in the second order perturbation of matter action ($\delta^2 S_m$) using two Lagrange multiplier $\delta\lambda_1$ and $\delta\lambda_2$. Doing so, the matter action is written as
\begin{eqnarray}
\delta^{2}S_{m} &=&\int drdt\left(\delta^2\sqrt{-\tilde{g}} \tilde{P}+\delta\sqrt{-\tilde{g}} \delta\tilde{P}+\sqrt{-\tilde{g}} \delta^2\tilde{P}\right)-\delta\lambda_2 E_2-l(l+1)\delta\lambda_1 E_1.
\end{eqnarray}
If we vary $\delta^{2}S_{m}$ with respect to functions $\delta\tilde{\mu}$, $v_p$ and $\delta\tilde{u}^r$,the same functions  can written, by solving the resulting equations, as
\begin{eqnarray}
\delta\tilde{\mu} &=& -\frac{\tilde{c}_m^2 h}{2(h+\text{D}\varphi'^2)}H_2-\frac{\tilde{c}_m^2\tilde{\mu}}{r^2\text{C}^{3/2}(\tilde{P}+\tilde{\rho})\sqrt{h+\text{D}\varphi'^2}}(l(l+1)\delta\lambda_1+ \delta\lambda_2')-\frac{\text{D}\tilde{c}_m^2 \varphi'}{h+\text{D}\varphi'^2}\delta\varphi'\nonumber\\
&&+\left(\frac{\text{C}_\varphi}{\text{C}}-\frac{1}{2}\left(\frac{3\text{C}_\varphi}{\text{C}}+\frac{\text{D}_\varphi \varphi'^2}{h+\text{D}\varphi'^2}\right)\right)\delta\varphi,\\
\delta\tilde{u}^r &=& \frac{\tilde{\mu}}{r^2\text{C}^{2}\sqrt{f}\sqrt{h+\text{D}\varphi'^2}}\dot{\delta\lambda}_2,\\
v_p               &=& -\frac{\tilde{\mu}}{\text{C}^{3/2}f\sqrt{h+\text{D}\varphi'^2}}\dot{\delta\lambda}_1,
\end{eqnarray}
where $\tilde{c}_m^2\equiv \tilde{\mu}\tilde{P}_{\tilde{\mu}\tilde{\mu}} /\tilde{P}_{\tilde{\mu}}$ is the speed sound of the fluid. Now, if we substitute these expressions into the action ($\delta^{2}S_{m}$), it will depend only on the metric and the Lagrange multipliers. By perturbing the total action (\ref{totalaction S}) up to second order and integrating by parts, it can be expressed as:

\begin{eqnarray}\label{S2polar}
\delta^{2}S^{polar} &=&\int drdt\left(H_0\left(a_1 \delta\varphi' +L a_2 H_5' +L a_3 H_2'+a_4 \delta\varphi +a_5  H_5 +a_6 H_2+ f_1 (\delta\tilde{\lambda}_2'+L \frac{\sqrt{C}}{\sqrt{f}}\delta\lambda_1)\right)\right.\nonumber\\
 &&\left. +L a_7 H_1^2+H_1(f_2 \dot{\delta\tilde{\lambda}}_2+a_8\delta\dot{\varphi} +L a_{9} \dot{H}_5 + a_{10} \dot{H}_2)+ a_{11} H_2^2+H_2(a_{12}\delta\varphi' +a_{13}\delta\varphi  \right.\nonumber\\ 
 &&\left. +L a_{14}H_5 +f_3 (\delta\tilde{\lambda}_2'+L \frac{\sqrt{C}}{\sqrt{f}}\delta\lambda_1))+ L a_{15}H_5^2 + L a_{16}\dot{H}_5^2 + L a_{17}H_5 \delta\varphi + e_1 \delta\varphi^2 + e_2 \delta\varphi'^2\right.\nonumber\\ 
 &&\left.  + e_3 \delta\dot{\varphi}^2+e_4(\delta\tilde{\lambda}_2'+L \frac{\sqrt{C}}{\sqrt{f}}\delta\lambda_1)\delta\varphi +L c_1 \dot{\delta\lambda}_1^2+c_2 \dot{\delta\tilde{\lambda}}_2^2 +c_3 \delta\tilde{\lambda}_2'^2 +L c_4 \delta\lambda_1 \delta\tilde{\lambda}_2'+L^2 c_5 \delta\lambda_1^2 \right),\nonumber\\ 
\end{eqnarray}
with
\begin{eqnarray}
\delta\tilde{\lambda}_2 &=&\frac{\text{C}^{1/2}}{f^{1/2}} \delta\lambda_2 + \frac{\text{D}\text{C}^{3/2}(\tilde{P}+\tilde{\rho})\varphi' r^2}{\tilde{\mu}\sqrt{h+\text{D}\varphi'^2}}\delta\varphi
\end{eqnarray}
where $L=l(l+1)$ and the coefficients $a_i$, $c_i$, $f_i$ and $e_i$ are given in the appendix  (\ref{app.A}). 
\subsubsection{Case $l\geq 2$:}\label{lsup2}
In order to rewrite the action (\ref{S2polar}) in the form of  a wave action, we solve the following equations
\begin{eqnarray}
&& a_1 \delta\varphi' +L a_2 H_5' + a_3 H_2'+a_4 \delta\varphi +a_5  H_5 +a_6 H_2+ f_1 (\delta\tilde{\lambda}_2'+L \frac{\sqrt{C}}{\sqrt{f}}\delta\lambda_1)=0,\\
&& 2 L a_7 H_1+f_2 \dot{\delta\tilde{\lambda}}_2+a_8\delta\dot{\varphi} +L a_{9} \dot{H}_5 + a_{10} \dot{H}_2=0,\\
&& L a_2 H_5 + a_3 H_2=\psi,
\end{eqnarray}
with respect to $H_1$, $H_2$ and $H_5$, for $l\geq 2$. In fact, the first and the second equations are obtained by varying the action (\ref{S2polar}) with respect to $H_0$ and $H_1$, respectively. And the last equation is the combination that will allow us to express the dynamic of the gravitational sector \cite{DeFelice:2011ka,Kobayashi:2014wsa,kase2020stability,minamitsuji2016relativistic}. After complex calculations, we arrive to
\begin{eqnarray}
\delta^{2}S^{polar} &=& \int drdt\;\left( \dot{\chi}^t\textbf{K}\dot{\chi}+{\chi^{t}}' \textbf{G}\chi' +\chi^t \textbf{L}\chi' +\chi^t \textbf{M}\chi\right),
\end{eqnarray} 
with
\begin{eqnarray}
{\chi}^t=\{\delta\lambda_1,\delta\tilde{\lambda}_2,\psi,\delta \varphi\},
\end{eqnarray}
and $\textbf{K}$, $\textbf{G}$ and $\textbf{M}$ are matrices $4\times 4$ where $\textbf{G}_{11}=0$ and $\textbf{M}_{22}=0$. The other components have a complicated expressions, but due to background equations the ghost instability conditions are simplified to
\begin{eqnarray}
\textbf{K}_{11}&=& \frac{L\text{C}^2\tilde{\mu}^2 }{2 f^{3/2}(P+\rho)(h+\text{D}\varphi'^2)} \geq 0,\label{L1}\\
\sum_{\{i,j\}=\{1,2\}}\epsilon^{ij}\textbf{K}_{1i}\textbf{K}_{2j}&=& 2f^2 \tilde{\mu}^4\text{C}^3\left(L r^4 \left(\text{a}^2 \left(\varphi '\right)^4+\varphi'^2 \left(2 \text{a} h^{3/2} \text{P}+\text{D} h^2 \rho  (\text{P}+\rho )\right)+h^3 \rho ^2\right)\right.\nonumber\\
&& \left. +2 \kappa  r^2 \left(\text{a} L (h L+h-3) \varphi'^2+h^{3/2} ((L-2) \text{P}-\rho  (L (h L+h-4)+2))\right)\right.\nonumber\\
& &\left.
+\kappa ^2 L (h L+h-3)^2\right)/\Delta\geq 0,\label{L2}\\
\sum_{\{i,j,n\}=\{1,2,3\}}\epsilon^{ijn}\textbf{K}_{1i}\textbf{K}_{2j}\textbf{K}_{3n}&=& L 2 f^{1/2}h^{3/2}\text{C}^{3}\tilde{\mu}^4\left(2 \kappa  (L-2)+L r^2 \varphi '^2 \left(2 \text{a}+\text{D} \sqrt{h} \rho \right)\right)/\Delta\geq 0,\label{L3}
\end{eqnarray}
where 
\begin{eqnarray}
\Delta &=& 4 f^4 L (\text{P}+\rho )^2 \left(h+\text{D} \varphi '^2\right) \left(\kappa  (h L+h-3)+\text{a} r^2 \varphi '^2+h^{3/2} \text{P} r^2\right)^2.
\end{eqnarray}
Since $\Delta\geq 0$, the  first and the third expressions are positives, as long as the  $\text{P}+\rho\geq 0$, $a\geq 0$, $\text{D}\geq 0$ and $\text{C}\geq 0$, and the second one is positives, if the numerator is positive. The last condition to avoid ghost instability is   
\begin{eqnarray}
Det[\textbf{K}]&=& \tilde{\mu}^4\text{C} h\left(2 \text{C}^2 \sqrt{h} \kappa  (L-2) L \left(2 \text{a} \sqrt{h}-\text{D}^2 \rho  \varphi'^2\right) \right.\nonumber\\
& &\left.-\rho  r^2 \left(h+\text{D} \varphi'^2\right) \times\left(\sqrt{h}\left(\text{C}^2 \text{D}  L^2 \varphi'^2 +4 \text{C}  L r \text{C}_{\varphi } \varphi ' \right)\left(2 \text{a}+\text{D} \sqrt{h} \rho \right)\right.\right.\nonumber\\
&& \left. \left. +4 \rho  r^2 \text{C}_{\varphi }^2 \left(h+\text{D} \varphi'^2\right)\right)\right)/\Delta\geq 0,
\end{eqnarray}
which is verified, when the numerator is positive. An other feature which should be studied is the Laplacian instabilities, where in order to have a good theory of gravity, the propagation speed square of the vector $\chi$ must be positive in both polar and angular directions. In fact, to derive the conditions of Laplacian instabilities, we consider the solution $\chi=\chi_0 e^{I(\omega t- k r- l\theta)}$, where $\chi_0$ is a constant vector, and $\omega$ and $k$ are the frequency and wavenumber, respectively. If we wish to see the Laplacian instability in the radial direction or in angular direction, we take the limits $\omega\rightarrow\infty$ and $k\rightarrow\infty$ or we take the limits $\omega\rightarrow\infty$ and $l\rightarrow\infty$, respectively. Then, to ensure non-vanishing solutions, we impose 
\begin{eqnarray}
Det[f\omega^2\textbf{K}+h k^2\textbf{G}]=0,
\end{eqnarray}
for the radial direction and we impose also 
\begin{eqnarray}
Det[L f\omega^2\textbf{K}+r^2\textbf{M}]=0,
\end{eqnarray}
for the angular directions. The interesting result in our calculation is that we find the radial propagation speed ($c_{r_2}=\omega/k$) of $\delta\lambda_1$ and the angular propagation speed ($c_{\Omega_1}=\omega/l$) of $\delta\tilde{\lambda}_2$ vanish, but the radial propagation speed of $\delta\tilde{\lambda}_2$ and the angular propagation speed of $\delta\lambda_1$ are calculated as
\begin{eqnarray}
c_{r_2}^2=\frac{h}{h+\text{D}\varphi'^2}\tilde{c}_m^2,\qquad c_{\Omega_1}^2=\tilde{c}_m^2.
\end{eqnarray}
The other solutions describe the propagating speed of $\delta\varphi$ and $\psi$, where the Laplacian instabilities are assured when
\begin{eqnarray}
c_{r_\pm}^2 &=&A_1\pm\sqrt{A_1^2-A_2}\geq 0,\\
c_{\Omega\pm}^2 &=&B_1\pm\sqrt{B_1^2-B_2}\geq 0.
\end{eqnarray}
where the expressions of $A_i$ and $B_i$, with $i=\{1,2\}$ are given in the Appendix (\ref{app.B}). Like in Horndeski theories, the propagating speeds of $\delta\varphi$ and $\psi$ the angular and radial are affected by the scalar field which depend on the form of the functions $\text{C}$ and $\text{D}$. Despite, the complexity of the velocities $c_{r_\pm}^2$ and $c_{\Omega\pm}^2$, one can estimate theirs behaviors and signs at the center of the star and at the exterior of the star. In fact, the speeds in both directions are reduced to the  propagation speed of light outside the stars and we calculate $c_{r_\pm}^2$ when $r$ tend to zero as
\begin{eqnarray}
c_{r_\pm}^2 = 1+O(r^2),\quad c_{\Omega\pm}^2 = 1+O(r^2),
\end{eqnarray}
which means that the Laplacian instability in the both directions are absents at the center of the star for any forms of the functions $\text{D}$ and $\text{C}$. 





\subsubsection{Case $l=0$:}


 If we impose that $l=0$ in the action (\ref{S2polar}), we must reduce the degree of freedom by choosing the gauge $H_0=0$ (or $H_1=0$). Therefore, the action (\ref{S2polar}) is reduced to
\begin{eqnarray}
\delta^{2}S^{polar} &=& \int drdt \left(H_1(f_2 \dot{\delta\tilde{\lambda}}_2+a_8\delta\dot{\varphi} + a_{10} \dot{H}_2)+ a_{11} H_2^2+H_2(a_{12}\delta\varphi' +a_{13}\delta\varphi +f_3 \delta\tilde{\lambda}_2') +  e_1 \delta\varphi^2 \right.\nonumber\\ 
 &&\left.  + e_2 \delta\varphi'^2 + e_3 \delta\dot{\varphi}^2+e_4 \delta\tilde{\lambda}_2'\delta\varphi +c_2 \dot{\delta\tilde{\lambda}}_2^2 +c_3 \delta\tilde{\lambda}_2'^2  \right). 
\end{eqnarray}
Varying with respect to $H_1$ , one can solve the obtained  equation for  $H_2$. Then, by replacing the results in the above action, we rewrite the action as 
\begin{eqnarray}
\delta^{2}S^{polar} &=& \int drdt\;\left( \dot{\chi}^t\textbf{K}\dot{\chi}+{\chi^t}' \textbf{G}\chi'+  \chi^t \textbf{L}\chi' + {\chi}^t \textbf{M}\chi\right),
\end{eqnarray} 
with
\begin{eqnarray}
{\chi}^t=\{\delta\tilde{\lambda}_2, \delta \varphi\}.
\end{eqnarray}
And the matrices $\textbf{K}$ and $\textbf{G}$ are diagonal $2\times 2$ matrices. Thus, no ghost conditions are ensured by 
\begin{eqnarray}
\textbf{K}_{11}&=& \frac{\tilde{\mu}^2 \text{C}}{2  \sqrt{f} r^2 (\text{P}+\rho )}\geq 0,\\
\textbf{K}_{22}&=& \frac{\text{a} \sqrt{h} r^2}{\sqrt{f}}-\frac{ \text{D}^2 \rho  r^2 \varphi '^2}{2 \sqrt{f} }\geq 0.
\end{eqnarray}
And the Laplacian instability by
\begin{eqnarray}
c_{r_1}^2 &=&\frac{ h}{h+\text{D} \varphi '^2}\tilde{c}_m^2,\\
c_{r_2}^2 &=& \frac{ \left(2 \text{a} \text{D} \sqrt{h} \sqrt{h+\text{D}\varphi '^2}+\text{C}^2 \text{D}^2 h \text{P}\right)\varphi '^2+2 \text{a} h^{3/2} \sqrt{h+\text{D}\varphi'^2}}{\left(h+\text{D} \varphi'^2\right) \left(2 \text{a} \sqrt{h} \sqrt{h+\text{D} \varphi '^2}-\text{C}^2 \text{D}^2 \rho  \varphi '^2\right)}
\end{eqnarray}

 We note, in the case $\text{D}=0$, that the  propagation speed of the scalar field is equal to the speed of light in the vacuum. The same result is recovered at the exterior of the star. At the center of the star, the propagating speed of the scalar field behaves as
\begin{eqnarray}
c_{r_2}^2&=& 1+O(r^2).
\end{eqnarray}
As expected, we do not find Laplacian instabilities for all forms of the functions $\text{D}$ and $\text{C}$. We observe that $\text{D}$ plays a crucial role in modifying the propagation speed of the scalar field from the speed of light. In other words, the disformal function has a greater impact on the deviation from general relativity than the conformal function $\text{C}$.

\subsubsection{Case $l=1$:}
We have seen that in the case $l\geq 0$ the propagating speed of the vector $\chi$ is not defined when $l=1$. This is due to the presence of extra  gauge degree of freedom \cite{kase2020stability}. In this paper, we fix the gauge by setting $\delta\varphi=0$. Following the same steps in the subsection \ref{lsup2}, we derive the Laplacian instabelity conditions as 
\begin{eqnarray}
c_{r_1}^2&=& 0,\\
c_{r_2}^2&=& \frac{ h}{h+\text{D} \varphi '^2}\tilde{c}_m^2,\\
c_{r}^2&=& \frac{2 \text{a} \text{D} \varphi '^2+2 \text{a} h-\text{D} h^{3/2} \text{P}}{\left(2 \text{a}+\text{D} \sqrt{h} \rho \right) \left(h+\text{D} \varphi '^2\right)},
\end{eqnarray}
where the propagation speed of $\psi$ is reduced to the speed of light at the exterior of the star and when we set $\text{D}=0$. At $r=0$, the value of $c_{r_3}^2$ is equivalent to
\begin{eqnarray}\label{cr3l1}
c_{r}^2&=&\frac{2\text{a}-\tilde{P}^c_0 (\text{C}^c)^2 \text{D}^c}{2\text{a}+\tilde{\rho}^c_0  (\text{C}^c)^2 \text{D}^c}.
\end{eqnarray}
Since the pressure and the energy density of the star are positive, the Laplacian instability at the center of the stars is translated to
\begin{eqnarray}
\tilde{P}^c_0 (\text{C}^c)^2 \text{D}^c\geq 2\text{a},\quad\text{and}\quad \tilde{\rho}^c_0 (\text{C}^c)^2 \text{D}^c\geq -2\text{a}.
\end{eqnarray}
These conditions must be taken into consideration in the numerical analysis of the background equations. Finally, the no ghost conditions in this case is recovered by taking the limit $L\rightarrow 1$ of the equations (\ref{L1}), (\ref{L2}) and (\ref{L3}).

%here

\subsection{axial perturbations:}
In the axial perturbations, we will follow the same steps in the polar perturbations, where the calculations are more simpler. If we do so, we will obtain  the constraints, from Eqs.(\ref{u}), (\ref{muper}) and (\ref{mu2}), as
\begin{eqnarray}
v_a &=& \frac{1}{\sqrt{f} \sqrt{\text{C}}}h_0,\\
\frac{\delta^2 \tilde{\mu}}{\text{C}}&=& -\frac{\tilde{\mu}\text{C}}{2 f\tilde{\mu}} \partial_\theta Y_{l0}^2 h_0^2.\end{eqnarray}
Inserting the last equation in the total action, then by perturbing it up to second order and after integrating by part. It follows that
\begin{eqnarray}\label{s2axial}
\delta^{2}S^{axial} &=& \int drdt \frac{\kappa  l(l+1) r^2 }{4 \sqrt{f} \sqrt{h}}\left(h (l(l+1) -2) h_0^2-f (l(l+1) -2) h_1^2+r^2 \left(h_0'-\dot{h}_1\right)^2\right).
\end{eqnarray}
As we found in Horndeski theories \cite{kase2020stability}, the fluid has no effect on the Lagrangian and we distinct two cases $l=1$ and $l\geq 2$.
\subsubsection{The case $l=1$:}
In this case, the action is reduced to 
\begin{eqnarray}
\delta^{2}S^{axial} &=& \int drdt \frac{\kappa  l(l+1) r^4 }{4 \sqrt{f} \sqrt{h}} \left(h_0'-\dot{h}_1\right)^2.
\end{eqnarray}
which will gives us, by using the Euler-Lagrange equations, the following  equations
\begin{eqnarray}
\ddot{h}_1-\dot{h}_0'=0,\\
\left( \frac{r^4 }{\sqrt{f} \sqrt{h}}(\dot{h}_1-h_0')\right)'=0.
\end{eqnarray}
If we fix the gauge $h_1=0$ and integrate the above equations we find
\begin{eqnarray}\label{h0-l=0}
h_0=\text{A}\int d\hat{r} \frac{\sqrt{f} \sqrt{h}}{\hat{r}^4},
\end{eqnarray}
where $\text{A}$ is a constant of integration. Note that we have eliminated the arbitrary function (it depends only on time) that results in our integration with respect to $r$, by using a specific function of gauge mode that appear in our case \cite{kase2020stability}. We observe in the result (\ref{h0-l=0}) that the axial metrics are time independent which is similar to the one we find in Horndeski theories. However, this result is modified in Jordan frame as
\begin{eqnarray}
\tilde{h}_0 &=& \text{A}\text{C}\int d\hat{r} \frac{\text{C}\sqrt{\tilde{f}} \sqrt{\tilde{h}+\text{D}(\varphi ')^2}}{\hat{r}^4}.
\end{eqnarray}
Thus, in the case of this subsection, the  momentum of inertia of the relativistic is also modified in these theories, and thus it is expected to  have a deviation of the relation between the mass and the momentum of inertia at strong regime, (see Ref.\cite{minamitsuji2016relativistic}). 
\subsubsection{The case $l\geq 2$:}
For this case, we use the Lagrange multiplier method which allow us to have the explicit form of $h_0$ and $h_1$  in terms of  $\xi=(\dot{h}_1-h_0')/\sqrt{h}\sqrt{f}$ \cite{minamitsuji2016relativistic}. Doing so, the action (\ref{s2axial}) becomes
\begin{eqnarray}
\delta^{2}S^{axial} &= \int drdt\frac{\sqrt{f}\kappa  l(l+1) r^6  }{4 (l(l+1)-2)\sqrt{h}}\left(\left(\frac{2 r^2 \text{C}^2 \left(h (\tilde{P}-\tilde{\rho} )-\tilde{\rho}  \text{D}\varphi'^2\right)}{\kappa\sqrt{h+\text{D}\varphi'^2}   }+\frac{(6-l(l+1)) \sqrt{h}}{r^2 }\right)\xi ^2+\frac{h   }{ f }\dot{\xi}^2-\xi '^2\right).
\end{eqnarray}
Similar to what  we find GR, the ghost and Laplacian instability, in the radial direction are absents in the axial modes, as long as $h/f$ is positive. From these, the radial propagating speed of the axial modes in Einstein frame is equal to the speed of light. In the angular direction, the Laplacian instability is also avoided, since as $l$ tend to infinity we have
\begin{eqnarray}
c_\Omega^2 &=& 1.
\end{eqnarray}
Therefore, we have demonstrated that the ghost and Laplacian instabilities of the axial modes are absent in all cases that we have considered in this subsection.

\section{Numerical analysis:}\label{3}



In order to confirm our analytic studies and to see if our model is stable for all values of $r$, we must solve the background equations. To do so, we set $\text{a}=1$ and we take the particular form of the the functions $\text{C}$ and $\text{D}$ as
\begin{eqnarray}\label{PC}
\text{C}=e^{p\varphi^2}\quad \text{and}\quad \text{D}=\Lambda,
\end{eqnarray}
where $\Lambda$ and $p$ are constants. This model is symmetric by scalar field reflection ($\varphi\rightarrow-\varphi$). In our numerical integration, we  use the dimensionless variables $s=\ln(r/r_0)$ and $\tilde{\varphi}=\varphi\sqrt{G}/c^2$ and the dimensionless constants $\tilde{p}=p c^4/G$ and $\tilde{\Lambda}=\Lambda c^2/\rho_0$, where

\begin{eqnarray}
r_0=\frac{c}{\sqrt{G \rho_0}}=89.664\, {\rm km}\,, \qquad \rho_0=m_{\rm n} n_0=1.6749\times 10^{14} {\rm g.cm}^{-3}\,,
 \end{eqnarray}
 where $m_{\rm n}$ is the neutron mass and $n_0=0.1\; {\rm fm}^{-3}$ is the typical number density  in neutron stars. Given an equation of state and a gravity model characterised by the choice of the parameters $\tilde{p}$ and $\tilde{\Lambda}$,  the radial integration of the background equations depends on the central energy density $\tilde{\rho}_c$ and the initial condition $\varphi_0^c$. The other quantities at $r=0$ can be expressed in terms of $\rho_c$, as discussed in the second section. In addition, we impose that at $r\rightarrow \infty$ the metric $\tilde{h}$ tend to $1$, which can be satisfied only if $\varphi^\infty=0$. This latter happens if we fix a particular value of $\varphi_0^c$.
 \begin{figure}[htb]
\centering
\includegraphics[width=0.4\textwidth, height=5cm]{prsMSLy.pdf} \hspace{1cm minus 0.25cm}
\includegraphics[width=0.4\textwidth, height=5cm]{prsMBSk21.pdf} 
\caption{\small The Mass-Radius relations in Jordan frame for the equations of states: SLy (left graph) and BSk21 (right graph).}
\label{rM}
\end{figure}
 
\begin{figure}[htb]
\centering
\includegraphics[width=0.4\textwidth, height=5cm]{proQSLy.pdf} \hspace{1cm minus 0.25cm}
\includegraphics[width=0.4\textwidth, height=5cm]{proQBSk21.pdf} 
\caption{\small The variation of the charge $Q$ as function of the central energy $\rho_0^c$ in Jordan frame for the equations of states:  SLy (left graph) and BSk21(right graph).}
\label{roQ}
\end{figure}

We wish to avoid singularities that appear in Eqs. (\ref{varphic}) and (\ref{Pc}) in our integrations and the Laplacian instabilities. The first problem is avoided by taking small or negative values of $\tilde{\Lambda}$ ($\tilde{\Lambda}\ll 1$ or $\tilde{\Lambda}\leq 0$). The second problem appears only for the mode $l=1$ which can be avoided for $\tilde{\Lambda}\geq 0$. Thus, the model is stable if we consider the case $\tilde{\Lambda}\ll 1$ by taking the value $\tilde{\Lambda} = 0.1$. This value will allow us to integrate the equations for large values of $\tilde{\rho}_c$ without facing numerical instabilities from $\tilde{\rho}_c= 2 \rho_0$ to $\tilde{\rho}_c= 20 \rho_0$.

The integration is performed from the center of star $r\sim 0$ to the radius of the star $\tilde{r}_s$, defined by $\tilde{P}(r_s)=0$ where $r_s$ is the radius of the star in Einstein Frame. The relation between the two radius is given by
\begin{eqnarray}
\tilde{r}_s=\sqrt{\text{C}r_s^2+\text{D}\varphi'(r_s)^2}.
\end{eqnarray} 

 Then we integrate from the surface of the star to infinity, taking into account the limit $\varphi^\infty=0$. In our paper, we use only two kinds of realistic equations of state, which are SLy and BSk21. The numerical integration allows us to show the relation between the radius of the star and its mass. We show this relation in Fig. \ref{rM} for three values of the parameter $\tilde{p}$. Our results are similar to those found in Ref. \cite{minamitsuji2016relativistic} since we chose the same forms of the functions $\text{C}$ and $\text{D}$. We can see that the deviation from GR is significant when we increase the absolute value of $\tilde{p}$. Depending on this parameter, the deviation starts at a particular value of $\tilde{\rho}_c$ and ends at a high value of $\tilde{\rho}_c$. These two values can be obtained from Fig. \ref{roQ}, where the charge of the scalar field is not zero only for an interval. For example, in Fig. \ref{roQ}, the value of $Q$ is not zero when $5.8\rho_0\leq\tilde{\rho}_c\leq 10.9\rho_0$ for the SLy EoS and $\tilde{p}=-120$. We show in Fig. \ref{roQ} the variation of scalar field charge as a function of the central density for the EoSs SLy and BSk21. We observe that the maximum of $Q$ is defined by the parameter $\tilde{p}$ and the kind of equation of state.
 
 
In fact, the effect of the parameter $\tilde{\Lambda}$ is not considered in our analysis because it does not have a significant effect on the relations presented in Fig. \ref{rM} and Fig. \ref{roQ}. However, if we take negative values of $\tilde{\Lambda}$, the deviation from GR and from our model will not be negligible for high negative values of $\tilde{\Lambda}$ (although we will face the Laplacian instability for the mode $l=1$). The results in our model are close to the purely conformal transformation, which is due to the small value of $\tilde{\Lambda}$. We note that in our model, the parameter $\tilde{p}$ is not constrained by binary-pulsar observations \cite{Freire:2012mg} since the disformal function does not vanish.


\begin{figure}[htb]
\centering
\includegraphics[width=0.4\textwidth, height=5cm]{pcSLy.pdf} \hspace{1cm minus 0.25cm}
\includegraphics[width=0.4\textwidth, height=5cm]{pcBSk21.pdf} 
\caption{\small The propagation speed of the scalar field and the metric as function of the radial coordinate $r$  for the equations of states:  SLy (left graph) and BSk21(right graph), using the central density $\tilde{\rho}_c= 12 \rho_0$.}
\label{pc}
\end{figure}


To confirm the stability of our model, we plot in Fig. \ref{pc} the variations of the radial and angular propagation speeds of the scalar field and the metric for the cases $l \geq 2$ (in red and blue colors) and $l=1$ (in green color) using two different kinds of equations of state (EoSs). The results are identical to our analysis, where we observe that $c_{r\pm}$ and $c_{\Omega\pm}$ are equivalent to the speed of light at the center and outside the star for $l\geq 2$. In fact, the velocities $c_{r\pm}$ and $c_{\Omega\pm}$ increase (+) or decrease (-) from $c$ at $r=0$ until they reach a maximum or minimum value, depending on the value of $\tilde{p}$ and the equation of state. Then, they tend to $c$ at the radius of the star. However, for the case $l=1$, the propagation speed of $\psi$ is different from $c$ at the center, where its value can be calculated using Eq. (\ref{cr3l1}), and it is equivalent to $c$ outside the neutron star. In Fig. \ref{pc2}, we show that our model is also free from the Laplacian instability for the case $l=0$, where we observe a small deviation from General Relativity (GR) compared to the deviation in Fig. \ref{pc}.



\begin{figure}[htb]
\centering
\includegraphics[width=0.4\textwidth, height=5cm]{pc.pdf}  
\caption{\small The propagation speed of the scalar field  as function of the radial coordinate $r$  for the mode $l=0$, using the central density $\tilde{\rho}_c= 12 \rho_0$.}
\label{pc2}
\end{figure}




\section{Conclusion:} 


In this paper, we studied the stability of neutron stars in scalar tensor theories with a geometric metric and a physical metric which is the image of the geometric one via the disformal transformation. The stability of neutron stars was studied by deriving the equation of motion in a static and spherically symmetric background. We then extended our study to the perturbed level and showed the instability conditions in all spacetime, at the center of the star, and outside the star. In addition, for a particular model described by the functions (\ref{PC}), we performed a numerical analysis aimed at showing the Laplacian instability. We used only two realistic equations of state in our numerical analysis.

We have seen that, for the case (\ref{PC}), spontaneous scalarization happens, which was the main reason for the deviation from GR in the mass-radius relation. In fact, the scalar field does not vanish inside the star for only an interval of central density that depends on the EoS and the value of $p$. The constant $\Lambda$ can also modify the interval, but it does not have a significant contribution. The small contribution of $\Lambda$ is due to our choice, where we considered a small value $\tilde{\Lambda}=0.1$ to avoid singularities at the background level and negative propagation speed of the scalar field at the perturbed level. Indeed, a negative value of $\Lambda$ might lead us to Laplacian instability for the case $l=1$ and at high central density.

Overall, our work contributes to the understanding of the behavior of neutron stars under disformal coupling, providing important insights into the stability of these astrophysical objects. Studying the stability of polar and axial perturbations is crucial because it gives us information about QNMs for odd and even modes. Therefore, a new paper about QNMs would be interesting, and we will address this in future work.


\newpage


\appendix
\section{Coefficients:}\label{app.A}
The coefficients that appears in $\delta^2 S$ are:
\begin{eqnarray}
&&a_1=\frac{1}{2} \sqrt{f} r^2 \varphi' \left(\frac{2 \text{a}}{\sqrt{h}}+\text{D}\rho \right),\quad a_2= \frac{\kappa  }{2} r \sqrt{f},\quad a_3 =- \frac{ \kappa   }{2 \sqrt{h}}\sqrt{f}   r,\nonumber\\
&& a_4=\text{C}_{\varphi } \left(-\frac{\text{D} \sqrt{f} r^2 (\text{P}+\rho ) \varphi '^2}{4 \text{C} c_m^2 \left(h+\text{D} \varphi '^2\right)}-\frac{\sqrt{f} r^2 \left(h (3 \text{P}+\rho )-2 \text{D} \rho  \varphi '^2\right)}{4 \text{C}}\right)-\frac{\text{D} r^2 f' (\text{P}+\rho ) \varphi '}{4 \sqrt{f} c_m^2 \left(h+\text{D} \varphi '^2\right)}\nonumber\\
&&-\frac{\text{D} \sqrt{f} r (\text{P}+\rho ) \varphi ' \left(r h'-4 \left(h+\text{D} \varphi '^2\right)\right)}{4 \left(h+\text{D} \varphi '^2\right)}+\frac{\sqrt{f} r^2 \text{D}_{\varphi } \left(\text{D} \rho  \left(\varphi '\right)^4+h (\text{P}+2 \rho ) \varphi '^2\right)}{4 \left(h+\text{D} \varphi '^2\right)}+\frac{\text{D} \sqrt{f} h r^2 (\text{P}+\rho ) \varphi ''}{2 \left(h+\text{D} \varphi '^2\right)},\nonumber\\
&&a_5= \kappa\sqrt{f},\quad a_6= -\frac{1}{4} \sqrt{f} \sqrt{h} \kappa  L -\frac{1}{4} \sqrt{f} \left(2 \sqrt{h} \kappa -\rho  r^2 \text{D} \varphi '^2+h r^2 (P-\rho )\right) ,\quad a_7=\frac{\kappa}{4} \sqrt{f} \sqrt{h},\nonumber\\
&&a_8=-r^2 \varphi ' \left(\sqrt{h} \rho  \text{D}+2 \text{a}\right),\quad a_9=\kappa r,\quad a_{10}=-\frac{\kappa}{2}\sqrt{h}r,\nonumber\\
&& a_{11}=\frac{1}{8} \sqrt{f} \sqrt{h} \left(2 \kappa+\frac{ \left(2 h+3 \text{D}\varphi '^2\right)}{h+\text{D} \varphi '^2}\sqrt{h} \text{P} r^2\right)-\frac{1}{8} \sqrt{f} h^2 r^2 c_m^2 (\text{P}+\rho ),\quad a_{12}=\frac{\sqrt{f} r^2 \varphi ' }{2 \sqrt{h}}\left(2 \text{a}-\frac{\text{D} h^{3/2} \text{P}}{h+\text{D} \left(\varphi '\right)^2}\right)\nonumber\\
&& a_{13}=\text{C}_{\varphi } \left(\frac{\sqrt{f} h r^2 \left(\text{D} (\text{P}-\rho ) \varphi '^2+2 h \text{P}\right)}{4 \text{C} \left(h+\text{D} \varphi '^2\right)}-\frac{3 \sqrt{f} h^2 r^2 c_m^2 (\text{P}+\rho )}{4 \text{C}}\right)+\text{D}_{\varphi } \left(\frac{\sqrt{f} h^2 r^2 c_m^2 (\text{P}+\rho ) \varphi '^2}{4 \left(h+\text{D} \varphi '^2\right)}\right.\nonumber\\
&&\left. -\frac{\sqrt{f} h \text{P} r^2 \varphi '^2}{4 \left(h+\text{D} \varphi '^2\right)}\right)+\frac{\text{D} \sqrt{f} h^2 r^2 c_m^2 (\text{P}+\rho ) \varphi ''}{2 \left(h+\text{D} \varphi '^2\right)}+\frac{\text{D} \sqrt{f} h r c_m^2 (\text{P}+\rho ) \varphi ' \left(4 \left(h+\text{D} \varphi '^2\right)-r h'\right)}{4 \left(h+\text{D} \varphi '^2\right)}\nonumber\\
&&-\frac{\text{D} h r^2 f' (\text{P}+\rho ) \varphi '}{4 \sqrt{f} \left(h+\text{D} \varphi '^2\right)},\quad a_{14} =-\frac{\kappa   \left(r f'+2 f\right)}{4 \sqrt{f}} ,\quad a_{15}=\frac{\kappa }{2} \sqrt{f} \sqrt{h},\quad a_{16}=\frac{\sqrt{h} \kappa   r^2}{4 \sqrt{f}},\nonumber\\
&& a_{17}=\sqrt{f}  r \varphi ' \left(2 \text{a}-\text{D} \sqrt{h} \text{P}\right),\quad e_2= -\frac{1}{2} \sqrt{f} r^2 \left(\frac{2 \text{a}}{\sqrt{h}}+\frac{\text{D}^2 \text{P} \varphi '^2}{h+\text{D} \varphi '^2}\right),\quad e_3 = \frac{r^2 }{2 \sqrt{f}}\left(2 \text{a} \sqrt{h}-\text{D}^2 \rho  \varphi '^2\right),\nonumber\\
&& e_4 = -\frac{\sqrt{\text{C}} \text{D} \sqrt{f} c_m^2 \varphi ' \left(\left(\varphi '\right)^2 \left(\text{a} h r^2+\text{D} \left(h^{3/2} \rho  r^2-4 \kappa \right)\right)+h^{5/2} \rho  r^2-h (h+3) \kappa \right)}{2 \kappa  r \left(h+\text{D} \varphi '^2\right)}\nonumber\\
&&-\frac{\sqrt{\text{C}} \text{D} \sqrt{f} \varphi ' \left(\text{a} r^2 \left(\varphi '\right)^2+h^{3/2} \text{P} r^2+h \kappa -\kappa \right)}{2 \kappa  r \left(h+\text{D} \varphi '^2\right)}+\frac{\sqrt{\text{C}} \sqrt{f} h c_m^2 \text{D}_{\varphi } \left(\varphi '\right)^2}{2 \left(h+\text{D} \varphi '^2\right)}+\text{C}_{\varphi } \left(\frac{\sqrt{f} h}{\sqrt{\text{C}} \left(2 \text{D} \left(\varphi '\right)^2+2 h\right)}\right.\nonumber\\
&&\left.-\frac{3 \sqrt{f} h c_m^2}{2 \sqrt{\text{C}}}\right)+\frac{\sqrt{\text{C}} \text{D} \sqrt{f} h c_m^2 \varphi ''}{h+\text{D} \varphi '^2},\quad c_1=\frac{\text{C}^2  \tilde{\mu}^2}{2 f^{3/2} (\text{P}+\rho ) \left(h+\text{D} \varphi '^2\right)},\quad c_2= \frac{\text{C} \tilde{\mu}^2}{2 \sqrt{f} r^2 (\text{P}+\rho )}=-\frac{c_3}{f c_m^2}\nonumber\\
&&c_4=-\frac{\text{C}^{3/2}  \tilde{\mu}^2 c_m^2}{r^2 (\text{P}+\rho )},\quad c_5= -\frac{\text{C}^2  \tilde{\mu}^2 c_m^2}{2 \sqrt{f} r^2 (\text{P}+\rho )},\quad f_1 =-\frac{\sqrt{f}\sqrt{h}}{2}f_2=f_3 h c_m^2= -\frac{1}{2} \sqrt{\text{C}}\tilde{\mu} \sqrt{f},
\end{eqnarray}
where 
\begin{eqnarray}
\text{P}=\frac{\text{C}^2}{\sqrt{h+\text{D} \varphi '^2}}\tilde{P},\quad \rho=\frac{\text{C}^2}{\sqrt{h+\text{D} \varphi '^2}}\tilde{\rho},\quad c_m^2=\frac{\tilde{c}_m^2}{h+\text{D} \varphi '^2}.
\end{eqnarray}
\section{The expressions of $A_i$ and $B_i$:}\label{app.B}
The expressions of $A_i$ are:
\begin{eqnarray}
\frac{\Delta_1 A_1}{h+\text{D} \varphi '^2} &=&
\kappa  (L-2) \left(4 \text{a} h \left(h+\text{D} \varphi '^2\right)+2 \text{D}^2 h^{3/2} \text{P} \varphi '^2\right)+L r^2 \varphi '^2 \left(2 \text{a} \text{D} h^{3/2} \text{P} \left(h+\text{D} \varphi '^2\right)-\text{D}^2 h^3 \text{P}^2\right),\nonumber\\
&&\\
\frac{\Delta_1 A_2}{h+\text{D} \varphi '^2} &=& \frac{\text{C}_{\varphi } }{\text{C}}\left(h+\text{D} \varphi '^2\right) \left(-8 \text{a} \sqrt{h} \rho  r^3 \varphi '-4 \text{D} h \rho ^2 r^3 \varphi '\right)+\kappa  (L-2) \left(4 \text{a} h-2 \text{D}^2 \sqrt{h} \rho  \left(\varphi '\right)^2\right)\nonumber\\
&&+\left(h+\text{D} \varphi '^2\right) \left(-\left(2 \text{a} \text{D} \sqrt{h} L \rho  r^2 \left(\varphi '\right)^2+\text{D}^2 h L \rho ^2 r^2 \left(\varphi '\right)^2\right)\right)-\frac{4 \rho ^2 r^4 \text{C}_{\varphi }^2 \left(h+\text{D} \varphi '^2\right)^2}{\text{C}^2 L}
\end{eqnarray}
with
\begin{eqnarray}
\Delta_1 &=&  \kappa (L-2)  \left(\frac{2 \text{D}^2 \sqrt{h} \left(\varphi '\right)^2 \left(\text{D} \rho  \left(\varphi '\right)^2+h (\rho -\text{P})\right)}{h+\text{D} \varphi '^2}-8 \text{a} h \right)+\frac{\text{C}_{\varphi }}{\text{C}} \left(8 \text{a} \sqrt{h} \rho  r^3 \varphi ' \left(h+\text{D} \varphi '^2\right)\right.\nonumber\\
&& \left. -4 \text{D} h^2 \text{P} \rho  r^3 \varphi '\right)+2 \text{a} \text{D} \sqrt{h} L r^2 \left(\varphi '\right)^2 \left(\text{D} \rho  \left(\varphi '\right)^2+h (\rho -\text{P})\right)-2 \text{D}^2 h^2 L \text{P} \rho  r^2 \varphi '^2.
\end{eqnarray}
And the expression of $B_i$ are given by the following equations
\begin{eqnarray}
\frac{\Delta_2 B_1}{h+\text{D} \varphi '^2} &=& \frac{\kappa }{h+\text{D} \varphi '^2} \left(2 \text{D}^2 (\rho -\text{P}) \varphi '^2-8 \text{a} \sqrt{h}\right)+2 \text{a} \text{D} r^2 (\rho -\text{P}) \varphi '^2-2 \text{D}^2 \sqrt{h} \text{P} \rho  r^2 \varphi '^2\\
\frac{\Delta_2 B_2}{h+\text{D} \varphi '^2} &=& \left(h+\text{D} \varphi '^2\right) \left(2 \text{a} \text{D} \text{P} r^2 \varphi '^2-\text{D}^2 \sqrt{h} \text{P}^2 r^2 \varphi '^2\right)+\kappa  \left(4 \text{a} \sqrt{h}+2 \text{D}^2 \text{P} \varphi '^2\right)
\end{eqnarray}
where
\begin{eqnarray}
\Delta_2 &= \frac{\kappa  }{h+\text{D} \varphi '^2}\left(4 \text{a} \sqrt{h}-2 \text{D}^2 \rho  \varphi '^2\right)-2 \text{a} \text{D} \rho  r^2 \varphi '^2-\text{D}^2 \sqrt{h} \rho ^2 r^2 \varphi '^2.
\end{eqnarray}
\bibliographystyle{ieeetr}
\bibliography{bibliography}

\end{document}




