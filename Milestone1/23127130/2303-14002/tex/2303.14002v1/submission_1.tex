 \documentclass[aps,prx,onecolumn,nofootinbib,superscriptaddress,notitlepage, 10pt]{revtex4-2}
%general Hilbert space
\newcommand{\hi}{\mathcal{H}}
\newcommand{\ho}{\mathcal{H}_1}
%\newcommand{\ht}{\mathcal{H}_2}%
\newcommand{\hth}{\mathcal{H}_3}%
\newcommand{\hio}{\mathcal{H}_{\mathcal{R}_1}}
\newcommand{\hit}{\mathcal{H}_{\mathcal{R}_2}}
\newcommand{\hith}{\mathcal{H}_{\mathcal{R}_3}}
\newcommand{\his}{\mathcal{H}_{\mathcal{S}}}
\newcommand{\hir}
{\mathcal{H}_{\mathcal{R}}}
\newcommand{\hia}{\mathcal{H}_{\mathcal{A}}}
\newcommand{\hik}{\mathcal{K}} %Hilbert space K
\newcommand{\hv}{\mathcal{V}} %Hilbert space V
\newcommand{\hr}{\mathcal{H}_{\mathcal{R}}}
\newcommand{\supp}{\text{supp}}
\newcommand{\hs}{\mathcal{H}_{\mathcal{S}}}
\newcommand{\aut}{\text{aut}}
\newcommand{\ex}{\text{ex}_{1,2}}


%\newcommand{\Y}{\mathcal{Y}} % Yen or dollar map
%\newcommand{\yn}{\text{\textyen}}
\newcommand{\Y}{\yen}

\newcommand{\lh}{\mathcal{L(H)}} %bounded linear operators
\newcommand{\lhs}{\mathcal{L}(\mathcal{H}_{\mathcal{S}})} %bounded linear operators on system Hilbert space
\newcommand{\lha}{\mathcal{L}(\hia)}
\newcommand{\lhr}{\mathcal{L}(\hir)} %bounded linear operators on apparatus Hilbert space
\newcommand{\lk}{\mathcal{L(K)}} %bounded linear operators on K

\newcommand{\linv}{\lh_{\text{sa}}^{\text{inv}}}

\newcommand{\bh}{\mathbf{B}(\mathcal{H})}
\newcommand{\bk}{\mathbf{B}(\mathcal{K})}
\newcommand{\trh}{\mathcal{T(H)}} %trace class operators on H
\newcommand{\trk}{\mathcal{T(K)}} %trace class operators on K
\newcommand{\sh}{\mathcal{S(H)}} %states
\newcommand{\eh}{\mathcal{E(H)}} %effects
\newcommand{\ph}{\mathcal{P(H)}} %projections
\newcommand{\ip}[2]{\left\langle\,#1\,|\,#2\,\right\rangle} %inner product
\newcommand{\ket}[1]{|#1\rangle} %ket
\newcommand{\state}[1]{|#1\rangle}
\newcommand{\dstate}[1]{\langle#1|} 
\newcommand{\bra}[1]{\langle#1|} %bra
\newcommand{\kb}[2]{|#1\rangle\langle#2|} %ketbra
\newcommand{\no}[1]{\left\|#1\right\|} %norm
\newcommand{\nos}[1]{\left\|#1\right\|^2} 
\newcommand{\tr}[1]{\textrm{tr}\left[#1\right]} %trace
\newcommand{\ptr}[1]{\textrm{tr}_1[#1]} %partial trace_1
\newcommand{\pptr}[1]{\textrm{tr}_2[#1]} %partial trace_2
\newcommand{\trv}[1]{\textrm{tr}_\hv[#1]} %partial trace over V
\newcommand{\trvin}[1]{\textrm{tr}_{\hv_1}[#1]} %partial trace over V
\newcommand{\trvout}[1]{\textrm{tr}_{\hv_2}[#1]} %partial trace over V
\newcommand{\com}{\textrm{com}} %commutation domain
\newcommand{\lb}[1]{lb(#1)} %lower bounds
\newcommand{\ran}{\textrm{ran}\,} %range 
\newcommand{\id}{\mathbbm{1}} %identity operator
\newcommand{\nul}{O} %null operator
\newcommand{\fii}{\varphi}
%\newcommand{\E}{\mathsf{E}}
\newcommand{\Psf}{\mathsf{P}}
\newcommand{\Ecan}{\E ^{\text{can}}}
\newcommand{\Fsf}{\mathsf{F}}
\newcommand{\Zsf}{\mathsf{Z}}
\newcommand{\Gsf}{\mathsf{G}}
\newcommand{\Lsf}{\mathsf{L}}
\newcommand{\Msf}{\mathsf{M}}
\newcommand{\Asf}{\mathsf{A}}
\newcommand{\al}{\mathfrak{A}}
\newcommand{\Bsf}{\mathsf{B}}
\newcommand{\Sy}{\mathcal{S}}
\newcommand{\Ap}{\mathcal{A}}
\newcommand{\Bp}{\mathcal{B}}
\newcommand{\Smc}{\mathcal{S}}
\newcommand{\Ns}{N_\Smc}
\newcommand{\Nr}{N_\Rmc}
\newcommand{\Na}{N_\Ap}
\newcommand{\M}{\mathcal{M}}%set of linear maps
\renewcommand{\L}{\mathcal{L}}%set of linear maps
\newcommand{\R}{\mathcal{R}}
\newcommand*\colvec[3][]{
    \begin{pmatrix}\ifx\relax#1\relax\else#1\\\fi#2\\#3\end{pmatrix}
}
\renewcommand{\S}{\mathcal{S}}
\newcommand{\T}{\mathcal{T}}

%measures
\newcommand{\meo}{\Omega} %measurement outcomes
\newcommand{\salg}{\mathcal{B}(\Omega)} %sigma-algebra
\newcommand{\var}{\textrm{Var}} %variance
\newcommand{\bor}[1]{\mathcal{B}(#1)} % Borel sigma-algebra
\newcommand{\ltwo}[1]{L^2(#1)} % L^2 space
\newcommand{\fidelity}{\mathcal{F}} %fidelity

%vectors
\newcommand{\bosig}{{\boldsymbol\sigma}}

\newcommand{\va}{\mathbf{a}} %a
\newcommand{\vb}{\mathbf{b}} %b
\newcommand{\vc}{\mathbf{c}} %c
\newcommand{\ve}{\mathbf{e}} %e
\newcommand{\vf}{\mathbf{f}} %f
\newcommand{\vg}{\mathbf{g}} %g
\newcommand{\vu}{\mathbf{u}} %u
\newcommand{\vn}{\mathbf{n}} %n
\newcommand{\vnn}{\hat\vn} %n with hat
\newcommand{\vm}{\mathbf{m}} %m
\newcommand{\vk}{\mathbf{k}} %k
\newcommand{\vx}{\mathbf{x}} %x
\newcommand{\vy}{\mathbf{y}} %y
\newcommand{\vsigma}{\boldsymbol{\sigma}} %sigma
\newcommand{\gams}{ \gamma_{\theta}^{(\mathcal{S})}}
\newcommand{\gamr} {\gamma_{\theta}^{(\mathcal{R})}}
%\newcommand{\dh}{\mathcal{D}(\hi)}
\newcommand{\dhs}{\mathcal{D}(\his)}
\newcommand{\dho}{\mathcal{D}(\hio)}
\newcommand{\dht}{\mathcal{D}(\hit)}
%effects
\newcommand{\Aa}{A(1,\va)} %(1,a)
\newcommand{\Aan}{A(1,-\va)}%(1,-a)
\newcommand{\Ab}{A(1,\vb)} %(1,b)
\newcommand{\Abn}{A(1,-\vb)}%(1,-b)
\newcommand{\An}{A(1,\vn)} %(1,n)
\newcommand{\Ann}{A(1,-\vn)}%(1,-n)
\newcommand{\Aaa}{A(\alpha,\va)} %(\alpha,a)
\newcommand{\Aaan}{A(2-\alpha,-\frac{\alpha}{2-\alpha}\va)}%(\alpha,-a)
\newcommand{\Abb}{A(\beta,\vb)} %(\beta,b)
\newcommand{\Abbn}{A(1,-\vb)}%(\beta,-b)
\newcommand{\Aoa}{A(1,\alpha\va)} %(1,\alpha a)
\newcommand{\Aob}{A(1,\beta\vb)} %(1,\beta b)
%devices
\newcommand{\dev}{\mathfrak{D}}

%observables
\newcommand{\A}{\mathcal{A}}%generic observable
\newcommand{\B}{\mathcal{B}}%generic observable
\newcommand{\C}{\mathsf{C}}%generic observable
\newcommand{\E}{\mathsf{E}}%generic observable
%\newcommand{\M}{\mathsf{M}}%generic observable
\newcommand{\F}{\mathsf{F}}%generic observable
\newcommand{\G}{\mathsf{G}}%generic joint observable
\newcommand{\Q}{\mathsf{Q}}%sharp observable
\renewcommand{\P}{\mathsf{P}}%sharp observable
\renewcommand{\R}{\mathcal{R}}%quantum reference frame

\newcommand{\hirs}{\mathcal{H}_{\mathcal{R}} \otimes \mathcal{H}_{\mathcal{S}}}

%\theoremstyle{plain}

\newcommand{\Rmb}{\mathbb{R}}

%qubit effects
\newcommand{\Ea}{E^{1,\va}} %(1,a)
\newcommand{\Eb}{E^{1,\vb}} %(1,b)
\newcommand{\Ec}{E^{1,\vc}} %(1,c)
\newcommand{\Eaa}{E^{\alpha,\mathbf{a}}} %(\alpha,a)
\newcommand{\Ebb}{E^{\beta,\mathbf{b}}} %(\beta,b)



\usepackage{physics}
\usepackage{soul}
\usepackage{relsize}
\usepackage{lmodern}
\usepackage{slantsc}
\usepackage{bbm}
%\usepackage{natbib}
\usepackage{graphicx}
\usepackage{amsmath}
\usepackage{bbold}
\usepackage{amsfonts}
\usepackage{amssymb}
\usepackage{amsthm}
\usepackage{amsbsy}
\usepackage{mathrsfs}
\usepackage{varioref}
\usepackage{dsfont}
\usepackage{bm}
\usepackage{color}
\usepackage[usenames,dvipsnames]{xcolor}
\definecolor{myblue}{rgb}{0.2,0.2,0.8}
\definecolor{myblack}{rgb}{0,0,0}
\definecolor{myurl}{rgb}{0.1,0.1,0.4}
\usepackage[colorlinks=true,citecolor=myblue,linkcolor=myblack,urlcolor=myurl]{hyperref}
\usepackage{cleveref}
%\usepackage[font = footnotesize, labelfont = bf, justification = RaggedRight]{caption}
\usepackage{subfigure}
\usepackage{booktabs} % for much better looking tables
\usepackage{array} % for better arrays (eg matrices) in maths
\usepackage{paralist} % very flexible & customisable lists (eg. enumerate/itemize, etc.)
\usepackage{verbatim} % adds environment for commenting out blocks of text & for better verbatim
\edef\restoreparindent{\parindent=\the\parindent\relax}
\usepackage[parfill]{parskip}
\setlength{\parskip}{1mm}
%\restoreparindent

\usepackage{tikz-cd}
%\usepackage{ulem}
%\usepackage[square,numbers]{natbib}

\renewcommand\thesection{\arabic{section}}
\renewcommand\thesubsection{\thesection.\arabic{subsection}}


%\usepackage{amsfonts,amsmath,amsthm}
\usepackage{color}
\usepackage[ansinew]{inputenc}
\usepackage[T1]{fontenc}
%\usepackage{authblk}
\newtheorem{theorem}{Theorem}[section] % 1st argument is your name for it
\newtheorem{lemma}[theorem]{Lemma}     % 2nd argument is what is printed
\newtheorem{corollary}[theorem]{Corollary}
\newtheorem{proposition}[theorem]{Proposition}
\newtheorem{definition}[theorem]{Definition}
\newtheorem{example}[theorem]{Example}
\newtheorem{nonexample}[theorem]{Non-example}

\newtheorem{remark}[theorem]{Remark}
\newtheorem{remarks}[theorem]{Remarks}
%\newtheorem{remark}[remark]{Remark}
%\newcommand{\R}{\mathbb{R}}
%\newcommand{\C}{\math{C}}
\newcommand{\red}[1]{{ \color{Red}{#1}}}
\newcommand{\jan}[1]{{ \color{Green}{#1}}}
\newcommand{\titouan}[1]{{ \color{Purple}{#1}}}
\newcommand{\leon}[1]{{ \color{blue}{#1}}}
\newcommand{\blue}[1]{\textcolor{blue}{#1}} 
%operations
\newcommand{\U}{\mathcal{U}} %unitary channel
\newcommand{\Lu}{\Phi^\mathcal{L}} %Luders operation
\newcommand{\Ch}{\mathcal{C}} %channel

%instruments
\newcommand{\J}{\mathcal{J}}
\newcommand{\I}{\mathcal{I}}
\newcommand{\inv}{(B(\hir \otimes \his))^{\text{inv}}}
%memo
\newcommand{\memo}{\mathfrak{M}}
\newcommand{\swap}{U_{\mathsf{swap}}}


%pictures
\newcommand{\hh}{^H} %Heisenberg picture
\renewcommand{\ss}{^S} %Schroedinger picture

%\addbibresource{bib.bib}

%%%%%%%%%%%%%%%%%%%%%%%%%%
%%%%%%%%%%%%%%%%%%%%%%%%%%

\usepackage{amsmath}
\usepackage[makeroom]{cancel}

\newcommand{\heading}[1]{ \bigskip \noindent {\large \bf {#1}} \medskip }
\renewcommand{\familydefault}{\sfdefault}

\begin{document}

\title{Operational Quantum Reference Frame Transformations}

\author{Titouan Carette}\email{titouan.carette@lu.lv}
\affiliation{Centre for Quantum Computer Science, Faculty of Computing, University of Latvia, Raina 19, Riga, Latvia, LV-1586}
\author{Jan G{\l}owacki}\email{glowacki@cft.edu.pl}
\affiliation{Center for Theoretical Physics, Polish Academy of Sciences, Al. Lotnik\'ow 32/46, 02-668 Warsaw, Poland}
\author{Leon Loveridge}\email{leon.d.loveridge@usn.no}
\affiliation{Quantum Technology Group, Department of Science and Industry Systems, University of South-Eastern Norway, 3616 Kongsberg, Norway}


%\date{\today}





\begin{abstract}

 We provide a general, operational, and rigorous basis for quantum reference 
 frames and their transformations using covariant positive operator valued measures to represent frame observables. The framework holds for locally compact groups and differs from all prior proposals for frame changes, being built around the notion of operational equivalence, in which states that cannot be distinguished physically are identified. This allows for the construction of the space of (invariant) relative observables and the convex set of relative states as dual objects. By demanding a further equivalence relation on the relative states which takes into account the nature of the frames, we provide a quantum reference frame change map.
  We show that this map is invertible exactly when the initial and final frames admit states which are arbitrarily well localized. We compare the presented frame change with other constructions available in the literature, finding operational agreement on the domain of common applicability.
\end{abstract}


\maketitle


\section{Introduction}
The subject of quantum reference frames sits at the confluence of two strands of thinking: that the physical world is fundamentally quantum mechanical in nature, and that coordinate systems are abstractions of embodied physical systems. Therefore, quantum systems are to be described relative to other quantum systems. Recently, interest has gathered (e.g., \cite{giacomini2019quantum,de2020quantum,de2021perspective}) in understanding a new type of `internal frame change', in which the description of a collection of systems relative to one of them is transformed to a description relative to another. The purpose of this paper is to provide such a map which is both mathematically rigorous and operationally well-motivated, both aspects of which have been lacking thus far in the context of locally compact groups.


The study of quantum reference frames differs from other `relational' approaches to understanding quantum mechanics (see e.g. \cite{rovelli1996relational} for the original `Relational Quantum Mechanics' paper, \cite{adlam2022information} for a recent extension, and \cite{bene2002perspectival} for a `perspectival' version of the modal interpretation of quantum theory) in that, akin to special relativity and gauge theories, there is a group which dictates both what is observable/physical in the theory and also mediates transformations between frames. The various frameworks which now exist---perhaps four distinct strands are visible, e.g.\cite{Bartlett2007,giacomini2019quantum,Vanrietvelde:2018pgb,Loveridge2017a} are representatives of each---differ in both aspects -- the imposition of an invariance principle and the manner in which the group is used to change frame. For example, in \cite{barrett2007information} all states and observables are permitted, but if the relation between two frames is unknown, then only invariant states may be used for communication. By contrast, in \cite{Vanrietvelde:2018pgb}, and more transparently in \cite{de2021perspective} the states which are considered physical are described by only the unit vectors which are invariant under the given group action. In this paper, we take the perspective that the physical \emph{observables} are invariant and explore the consequences, culminating in a frame change procedure based on ideas arising in e.g. \cite{loveridge2012quantum,loveridge2017relativity,Loveridge2017a}. We comment as we go on relative merits, drawbacks and distinctions between the various approaches but leave a comprehensive analysis for the future.

The paper is organised as follows. Sec. \ref{sec:pre} begins with the provision of some general background and nomenclature, setting the relevant notions of state and observable and briefly discussing the sense of convergence that will be needed. We then introduce an abstract notion of \emph{operational equivalences} on the set of states and effects; primarily we are interested in those states which cannot be distinguished by some collection of observables. Such states are identified and the relevant quotient is constructed. An important example arises under a unitary representation of a locally compact group $G$; the observable algebra
is the fixed point algebra $B(\hi)^G$ of (gauge)-invariant operators, and the state space comprises the equivalence classes of states which cannot be distinguished by invariants. We choose the invariant operators rather than invariant Hilbert space vectors, as in the perspective-neutral approach \cite{de2021perspective}, on the grounds that the latter set is typically empty in the Hilbert space setting, and requires distributional techniques which are not known to be rigorously possible in general. Further operational equivalences are considered later, where the \emph{relative} and \emph{framed relative} observables/states are given, which are required for the frame changes. Following the discussion of operational equivalence, we introduce (covariant) positive operator valued measures (POVMs) which are to define quantum reference frames, generalizing the systems of coherent states used elsewhere (e.g. \cite{de2021perspective}) and fit our 'observables first' approach. We study a \emph{localizability} property which allows for probability measures to be highly concentrated around a point, which is crucial for for making rigorous objects such as $\ket{g}$ ($g \in G$) which are used freely in physics but require explanation mathematically, and for good agreement between relational kinematics and the standard one. This also allows us to make mathematically rigorous and conceptually clearer, on operational grounds, the idea that a frame should be prepared in the `$\ket{e}$' state prior to changing frame \cite{de2020quantum}. Finally, the covariance property of various POVMs is considered, for instance the familiar spectral measure of the unbounded position observable, the phase conjugate to numbers, and those generated by families of coherent states, thus drawing the first connection to the \emph{perspective-neutral} approach (e.g. \cite{Vanrietvelde:2018pgb,de2021perspective}).

Following this, Sec. \ref{sec:rqk} provides the definition of a quantum reference frame as a system of covariance (a covariant POVM on a homogeneous $G$-space), and various properties are considered which reflect the underlying action and the representation. In this work we consider the setting in which the value space $\Sigma_{\R}$ of the frame POVM transforms freely and transitively under the group action, i.e., $\Sigma_{\R}$ is a principal homogeneous space; the setting of general homogeneous spaces is considered for finite groups/spaces in \cite{glowacki2023quantum} and the locally compact case is work in progress.
Through a brief review of the relativization ($\Y^{\R}$) and restriction ($\Gamma_{\rho}$) procedures introduced elsewhere (e.g. \cite{Miyadera2015e,Loveridge2017a}), we construct the spaces of relative observables $B(\his)^{\R}$ and relative states sitting inside $\mathcal{T(\his)^{R}}$, finding that that $[\mathcal{T(\his)^{R}}]^* \cong B(\his)^{\R}$ - a 'relative' analogue of the usual duality between the trace class and bounded operators in Hilbert space.  We use the localizability property of the frame POVMs to generalise a theorem of \cite{Loveridge2017a}, showing that (gauge-)invariant observables of system-plus-reference are arbitrarily close (in the operationally motivated topology) to any given system observable for any $G$ which is locally compact, Hausdorff and second countable. The state (or sequence of states) which achieves this good approximation is well localized at the identity of $G$, which, as we will see, connects closely to the reference states given in \cite{de2020quantum} for their frame change prescription.

Our take on frame changes comes in Sec. \ref{sec:fcp}; this is inspired by
\cite{giacomini2019quantum,de2020quantum} but mathematically and physically distinct from all previous frame change constructions. The map we provide is rigorous and defined on the relevant convex `operational' state spaces. We conclude by showing that our map agrees with \cite{de2020quantum} exactly on `classical configurations' for finite groups, in which case it can be given as conjugation by a unitary; otherwise we find that there is no operational distinction between our map and that of \cite{de2020quantum}. The transformation rule of \cite{giacomini2019quantum,de2020quantum} formally yields an entangled state after the frame change from an initial product state (in which the second frame is a superposition of $G$-basis states) -- a feature that has been widely touted in the literature as a demonstration of the `frame-dependence of entanglement'. However, not only is this a category error since the initial and final states are describing different systems, though the entanglement is present in the final state, there does not seem to be any way to observe it whilst respecting symmetry and operational principles, as our prescription shows. We also provide a brief comparison with the perspective-neutral approach - a full comparison is a large project due to the difficulties of making the perspective-neutral framework fully rigorous, but we observe again operational agreement with our maps and that of \cite{de2021perspective} in the setting that the initial frame is ideal. After a brief conclusion, we provide a glossary of the various spaces we consider as a handy reference.

\section{Preliminaries}\label{sec:pre}

We briefly recall in this section some elementary definitions and properties of the Hilbert space framework of quantum mechanics. We then introduce the notion of operational equivalence, which allows for the identification of 
mathematically distinct entities (most commonly for us these are states) within the given formalism being physically indistinguishable. This recurs throughout the paper, allows us to define the relative states, and to make sure that we do not distinguish states which give rise to the same distributions on the frame observable.
 We close by recalling the definitions of covariance and localizability for POVMs, accompanied by a list of examples.

\subsection{Basics}

In the  Hilbert space framework of quantum theory, a quantum system has associated with it a (complex, separable) Hilbert space $\hi$, and (normal) states are identified with positive trace class operators of trace one; we write $\T(\hi)$ for the trace class and $\mathcal{S}(\hi)$ for the convex subset of states. The pure states are the extremal elements of $\mathcal{S}(\hi)$, characterised as the rank-1 projections and written $\dyad{\varphi}$ for some unit vector $\varphi \in \hi$; the collection of pure states is written $\mathcal{P}(\hi)$. The space/C$^*$/von Neumann algebra of bounded operators in $\hi$ is denoted $B(\mathcal{H})$, which is the dual of $\T(\hi)$. A channel $\Lambda:B(\hi) \to B(\mathcal{K})$ is a normal (i.e., continuous with respect to the ultraweak topologies on $B(\hi)$ and $B(\mathcal{K})$) completely positive (CP) map which preserves the unit; the normality means there is a unique CP trace-preserving \emph{predual} map $\Lambda_*:\T(\hi) \to \T(\mathcal{K})$ defined through
$\tr[T\Lambda(A)]=\tr[\Lambda_*(T)A]$ for all $T \in \T(\hi)$ and 
$A \in B(\hi)$. These maps will also be called channels. More generally we will consider abstract \emph{state spaces} to be (total) convex subsets of real vector spaces, with \emph{state space maps} given by affine maps and state space isomorphisms by invertible affine maps.

Observables are (identified with) positive operator valued measures (POVMs) $\mathcal{F} \to B(\hi)$, where $\mathcal{F}$ is a $\sigma$-algebra of subsets of some set $\Sigma$, which represents outcomes that may be obtained in a measurement of the given observable. Operators in the range of a POVM $\E$ are called \emph{effects}, that is, positive operators in the unit operator interval $[\mathbb{0}, \mathbb{1}]$. Throughout the rest of this paper, $\Sigma$ is a topological space and $\mathcal{F}$ is given as the Borel $\sigma$-algebra $\mathcal{B}(\Sigma)$. If each operator in the range of a POVM $\E$ is a projection, $\E$ is a projection-valued measure (PVM) and if defined on (subsets of) $\mathbb{R}$ the standard description of observables as self-adjoint operators in $\hi$ is recovered through the spectral theorem; we occasionally write $\mathsf{P}^A$ for the spectral measure of $A$, and write $A=\int x d\mathsf{P}^A(x)$ (interpreted weakly). POVMs which are PVMs will be called \emph{sharp} (observables), all others \emph{unsharp} (observables). In the case that an observable is sharp, we will also refer to the corresponding self-adjoint operator as an observable. 

We have already mentioned various topological notions; it is worth briefly formalizing these since they appear repeatedly. We are motivated by operational
ideas, and therefore the preferred topology on $B(\hi)$ is the \emph{topology of pointwise convergence of expectation values}, i.e., $A_n \to A \in B(\hi)$ exactly when $\tr[T A_n] \to \tr[T A]$ for all $T \in \T(\hi)$ (there is no loss of generality to restrict further from $\T(\hi)$ to $\mathcal{S}(\hi)$). This is the ultraweak (also called $\sigma$-weak, or weak-$*$ as the dual of $\T(\hi)$) topology on $B(\hi)$ (on norm bounded sets convergence here agrees with convergence in the weak operator topology arising from the family of seminorms $A \mapsto |\ip{\varphi}{A \phi}|$, which can be reconstructed from the pure state expectation values by polarization), and note that these topologies are Hausdorff \cite{murphy2014c}. On the predual $\T(\hi)$ (and by restriction on the state space), 
we again use the topology of point-wise convergence of the expectation values, thus $T_n \to T$ exactly when $\tr[T_n A] \to \tr[T A]$ for all $A \in B(\hi)$. Since the set of effects span $B(\hi)$, this can be equivalently given on $\mathcal{E}(\hi)$. We will refer to this topology as the \emph{operational topology} on the set of states, i.e., it is the weakest topology that makes all the $T \mapsto \tr[T A]$ continuous. The superscript "$^{\rm cl}$" will always refer to ultraweak closure of the subsets in operator algebras, and operational closure of the subsets in trace class operators.



\subsection{Operational equivalence}

The notion of operational equivalence captures the idea that distinct `entities' which cannot be distinguished physically/operationally should be identified. For instance, in the operational approach to quantum theory (e.g. \cite{busch1997operational}), states are defined as equivalence classes of preparation procedures that cannot be distinguished in any measurement. The idea also arises when mathematically distinct entities either correspond to the same physical `situation'. This is common in e.g. gauge theory wherein fields related by a gauge transformation are physically identical. At this point a choice can be made about whether to leave in the descriptive redundancy and work at the level of the 'ordinary' formalism, or to quotient by the equivalence relation and work at the level of the equivalence classes. In this work, we have chosen to opt for the latter.

We introduce a number of operational equivalence relations, based primarily on identifying quantum states which cannot be distinguished by a given set of observables. For example, we identify any two states which give the same statistics on all (gauge-)invariant observables. These are naturally identified also in terms of the predual of the von Neumann algebra $B(\his)^G$, which is a central object in this work and the analogue of the physical Hilbert space which plays a similar role in the perspective-neutral approach \cite{Vanrietvelde:2018pgb,de2021perspective}. The operational identification of such objects means that we do not draw any mistaken conclusions about the distinction between two states which, upon further examination, cannot be separated in principle.

\subsubsection{General Theory}
\begin{definition}
Let $\mathcal{O}$ be a collection of effects and $\mathcal{S}$ a collection of states for which $\tr[E\rho]=\tr[E' \rho ']$ for all $E,E' \in \mathcal{O}$
and $\rho, \rho ' \in \mathcal{S}$. Then:
\begin{itemize}
    \item The states in $\mathcal{S}$ are called \emph{operationally}  equivalent with respect to $\mathcal{O}$
    \item The effects in $\mathcal{O}$ are operationally equivalent with respect to $\mathcal{S}$.
\end{itemize}
\end{definition}
This definition may be adapted to POVMs, self-adjoint and trace class operators as needed;
 the following will be used throughout.

\begin{definition}
    For any subset $\mathcal{O}\subseteq B(\mathcal{H})$ the $\mathcal{O}$-operational equivalence relation on $\T(\mathcal{H})$ is defined as

\[\rho \sim_{\mathcal{O}} \rho' \Leftrightarrow \tr[A\rho]=\tr[A\rho'] \hspace{3pt} \forall A \in \mathcal{O}.\]
\end{definition}
The identification of $\mathcal{O}$-equivalent trace class operators amounts to taking the quotient $\T(\mathcal{H})/\hspace{-3pt}\sim_{\mathcal{O}}$; the structure of this space and its dual are given in the following proposition.

\begin{proposition}\label{prop:iml}
The space $\T(\mathcal{H})/\hspace{-3pt}\sim_{\mathcal{O}}$ is a Banach space and there is an isometric isomorphism between its dual and the ultraweak closure of the span of $\mathcal{O}$, i.e.,
\[
\left(\T(\mathcal{H})/\hspace{-3pt}\sim_{\mathcal{O}} \right)^* \cong {\rm span}(\mathcal{O})^{\rm cl}.
\]

\end{proposition}
\begin{proof}
For $A \in \mathcal{O}$, we write $\phi_A$ for the continuous linear functional $\rho \mapsto \tr[A\rho]$ and identify ${\rm span}(\mathcal{O})$ with the corresponding subspace in the dual space. Then:

\[\rho \sim_{\mathcal{O}} \rho' \Leftrightarrow \forall A \in \mathcal{O} \hspace{3pt} \phi_A(\rho)=\phi_A(\rho') \Leftrightarrow \forall A \in \mathcal{O} \hspace{3pt} \phi_A(\rho-\rho')=0 \Leftrightarrow \rho-\rho' \in {}^{\perp}\mathcal{O}.\]

Here, ${}^{\perp}\mathcal{O}$ is the pre-annihilator of $\mathcal{O}$, which can be defined as ${}^{\perp}\mathcal{O}=\bigcap_{\phi_A\in \mathcal{O}}\ker(\phi_A)$, which is always closed in $\T(\hi)$ as an intersection of closed sets. Furthermore, the pre-annihilator is always a subspace and thus ${}^{\perp}\mathcal{O} = {}^{\perp}{\rm span}(\mathcal{O})$. So $\T(\hi) /\hspace{-3pt}\sim_{\mathcal{O}} = \T(\hi) /{}^{\perp}\mathcal{O}$ is a Banach space with the quotient norm:

\[||\rho + {}^{\perp}\mathcal{O} ||=\inf_{\mu \in {}^{\perp}\mathcal{O}} ||\rho+\mu||.\]

We then have $\left(\T(\hi)/\hspace{-3pt}\sim_{\mathcal{O}} \right)^* = \left(\T(\hi)/{}^{\perp}\mathcal{O}\right)^* \simeq {}^{\perp}\mathcal{O}^{\perp} = \left({}^{\perp}{\rm span}(\mathcal{O})\right)^{\perp} = {\rm span}(\mathcal{O})^{\rm cl}$ (see Theorem 4.9 and 4.7 in \cite{rudin1974functional}).

\end{proof}


This proposition allows for the states to be understood as classes of indistinguishable density operators; the quotient space $\S(\mathcal{H})/\hspace{-3pt}\sim_{\mathcal{O}}$ is a state space in the real subspace of classes of self-adjoint trace class operators  $\T(\hi)^{\rm{sa}}/\hspace{-3pt}\sim_{\mathcal{O}}$, as is confirmed by the following.


\begin{proposition}\label{prop:statespace}
    The set $\S(\mathcal{H})/\hspace{-3pt}\sim_{\mathcal{O}}$ is a total convex subset of $\T(\hi)^{\rm{sa}}/\hspace{-3pt}\sim_{\mathcal{O}}$, and is thus a state space and, moreover,  is closed in the quotient operational topology.
\end{proposition}

\begin{proof}
Since the real linear structure of $\T(\hi)^{\rm{sa}}/\hspace{-3pt}\sim_{\mathcal{O}}$ comes from $\T(\hi)^{\rm{sa}}$, convexity is preserved under the quotient. In particular, writing $[\_]_\mathcal{O}$ for the $\mathcal{O}$-equivalence classes, for any $\rho,\rho' \in \S(\hi)$ and $0 \leq \lambda \leq 1$ we have
\[
\lambda[\rho]_\mathcal{O} + (1-\lambda)[\rho']_\mathcal{O} = [\lambda \rho + (1-\lambda)\rho']_\mathcal{O} \in \S(\hi)/\hspace{-3pt}\sim_{\mathcal{O}}.
\]

The bounded affine functionals on $\S(\mathcal{H})$ are given by $\rho \mapsto \tr[\rho A]$ with $A \in B(\hi)$, with the effects given by the subset $\mathcal{E}(\hi) = \{F \in B(\hi) | \mathbb{0} \leq F \leq \mathbb{1}\}$. The effects on $\S(\mathcal{H})/\hspace{-3pt}\sim_{\mathcal{O}}$ are then those that are well-defined on classes $\S(\mathcal{H})/\hspace{-3pt}\sim_{\mathcal{O}}$, and hence are given by the operators in $\mathcal{E}(\hi) \cap \rm{span}(\mathcal{O})^{\rm cl}$. Indeed, $F \in \mathcal{E}(\hi)$ is well-defined on the $\mathcal{O}$-equivalence classes of states if whenever $\rho \sim_\mathcal{O} \rho'$ we have $\tr[\rho F] = \tr[\rho'F]$, which means that $F \in \rm{span}(\mathcal{O})^{\rm cl}$. The effects then separate the elements of $\S(\mathcal{H})/\hspace{-3pt}\sim_{\mathcal{O}}$ by construction, providing total convexity.

The state space $\S(\hi)$ is operationally closed in $\T(\hi)$ since for any sequence of states $(\rho_n) \subset \S(\hi)$ such that $\lim_{n \to \infty}\tr[\rho_n A] = \tr[T A]$ for all $A \in B(\hi)$ and some $T\in \T(\hi)$, we can conclude that $T \in \S(\hi)$. Indeed, the continuity of the trace gives positivity and normalization of $T$. The operational topology on $\S(\hi)/\hspace{-3pt}\sim_{\mathcal{O}}$ is the quotient topology of the one on $\T(\hi)$ so we have
\[
\lim_{n \to \infty} [\rho_n]_\mathcal{O} = [T]_\mathcal{O} \in \S(\hi)/\hspace{-3pt}\sim_{\mathcal{O}}.
\]
\end{proof}


A state space of the form above will be called an \emph{operational state space}. Often in this paper, the set $\mathcal{O}$ is the image of a unital normal positive map. In this case, the corresponding state space admits an equivalent useful characterization. 

\begin{proposition}\label{generalst}
    If $\Lambda:B(\mathcal{K})\to B(\mathcal{H})$ is a normal, positive, unital map, there is a state space isomorphism
    \[
    \S(\mathcal{H})/\hspace{-3pt}\sim_{\Im \Lambda} \cong \Lambda_* (\S(\mathcal{K})).
    \]
\end{proposition}

\begin{proof}
    
     Since $\Lambda$ is normal, we can write ${}^\perp \Im \Lambda = \ker \Lambda_* $, and thus $\T (\mathcal{H})/\hspace{-3pt}\sim_{\Im \Lambda} = \T (\mathcal{H})/\ker \Lambda_*$. Then $\Lambda_* $ restricts to an invertible bounded linear map $\T (\mathcal{H})/\ker \Lambda_* \to \Im \Lambda_* $. Since $\Lambda$ is linear, unital and positive, the $\Lambda_*$ map restricts further to an affine bijection $\S (\mathcal{H})/\ker \Lambda_* \to \Lambda_* (\S(\mathcal{K}))$, providing the state space isomorphism.\footnote{Note that in general, this correspondence doesn't hold at the level of the ambient Banach spaces. In fact, $\Im \Lambda_* $ might not be closed and hence not a Banach space. Even considering its norm-closure, then we indeed have a bijective bounded linear map from $\T (\mathcal{H})/\hspace{-3pt}\sim_{\Im \Lambda}$ but not an isometry in general.}
\end{proof}



\subsubsection{Operational Equivalence from Symmetry}\label{subsubsec:oes}

We will now discuss an important example of operational equivalence that arises in the presence of a (strongly continuous) unitary representation $U:G \to B(H)$ of a locally compact group $G$. We will often write $g.A$ to stand for $ U(g) A U(g)^* $ with $A\in B(\hi)$ and $g.\rho$ for $U(g)^* \rho U(g)$ with $\rho \in \mathcal{S}(\hi)$. The observable algebra is $B(\hi)^G := \{A \in B(\hi)~|~g.A = A\}$, the self-adjoint elements of which correspond to the sharp observables that do not depend on an external frame, as will be explained in the sequel.

Setting $\mathcal{O}=B(\hi)^G$ yields the $G$-operational equivalence on states, written $\sim_G$. By Prop. \ref{prop:iml}, the set $\T(\hi)/\hspace{-3pt}\sim_G $ is a Banach space whose dual is $B(\hi)^G $ (which as the commutant of a unitary group is ultraweakly closed). Furthermore, the set of states $\S(\hi)_G := \S(\hi)/\hspace{-3pt}\sim_G $ is a state space by Prop. \ref{prop:statespace}. Note that in general $\S(\hi)_G$ cannot be identified with the set $\S(\hi)^G $ of invariant states, i.e. those $\rho\in \S(\hi)$ for which $g\cdot \rho =\rho $, this last set being generically empty if $G$ is not compact. Given that there is typically an abundance of invariant observables, the state space $\S(\hi)_G$ is non-trivial and justifies on mathematical grounds that invariance should be stipulated on the observables rather than states (see \cite{Loveridge2017a,Miyadera2015e,loveridge2017relativity} on which the present approach is based). We are also able to avoid the use of distributions/rigged Hilbert spaces which are needed for constructing the physical Hilbert space in the perspective-neutral approach \cite{de2021perspective}, which is defined as the space of invariant (`kinematical') Hilbert space vectors. We note immediately that by setting $B(\hi)^G$ as the collection of observables on which the equivalence of states is defined, any state on a $G$-orbit is operationally indistinguishable/equivalent to any other. This is as one would expect of gauge transformations: all states related by a gauge transformation are physically equivalent.

If $G$ is compact we can define the \emph{$\mathcal{G}$-twirl} (or \emph{incoherent group average}) $\mathcal{G}:B(\hi)\to B(\hi)$~by 
\begin{equation}
    \mathcal{G}(A)=\int_G d\mu(g) U(g)A U(g)^*,
\end{equation}
where $\mu$ is (normalised) Haar measure; the integral is understood in terms of functions $G \to B(\hi)$ in the sense of Bochner. $\mathcal{G}$ is a unital normal map with pre-dual $\mathcal{G}_* : \T(\hi) \to \T(\hi)$ taking the form
$\mathcal{G}_* (\rho)=\int_G d\mu(g) U(g)^* \rho U(g)$. Both $\mathcal{G}$ and $\mathcal{G}_* $ are idempotent, respectively on the sets $B(\hi)^G $ and $\T(\hi)^G $. The image under $\mathcal{G}$ of a state is often interpreted (though without clear operational meaning) as an invariant `version' of the given state. If $G$ is not compact the integral does not converge in general on states or observables---thus $\mathcal{G}$ is not defined---and we therefore avoid it in this setting.

By Prop. \ref{generalst}, for $G$ compact there is an isomorphism of state spaces $S (\mathcal{\hi})^G = \mathcal{G}_*(\S(\hi)) \cong \S (\mathcal{\hi})_G $, and  in this case, the correspondence lifts to the ambient Banach spaces:

\begin{proposition}
   If $G$ is compact, then the Banach spaces $\T(\hi)^G $ and $\T(\hi)_G $ are isometrically isomorphic.
\end{proposition}

\begin{proof}

We have: $ \T(\hi)_G = \T(\hi)/\ker(\mathcal{G}_* ) $ and $\T(\hi)^G = \Im \mathcal{G}_* $. $\mathcal{G}_*$ factorizes through a bijective map \linebreak $\tilde{\mathcal{G}_*}:\T(\hi) /\ker(\mathcal{G}_* ) \to \Im \mathcal{G}_* $. We show that $\tilde{\mathcal{G}_*}$ is an isometry. First, $\mathcal{G}_* $ is a contraction with respect to the trace norm (we assume that Haar measure $\mu$ has been normalised): 
\[|| \mathcal{G}_* (\rho) ||_1 = ||\int_G g\cdot \rho d\mu(g)||_1\leq \int_G || g\cdot \rho ||_1 d\mu(g) \leq \int_G ||\rho||_1 d\mu(g)= ||\rho ||_1 ;\]

it follows that for all $\sigma \in \ker(\mathcal{G}_* )$

\[|| \tilde{\mathcal{G}_* }(\rho + \ker(\mathcal{G}_* )) ||_1 = || \mathcal{G}_* (\rho) ||_1= || \mathcal{G}_* (\rho + \sigma) ||_1 \leq || \rho + \sigma ||_1. \]

Hence $|| \tilde{\mathcal{G}_* }(\rho+ \ker(\mathcal{G}_* )) ||_1\leq ||\rho + \ker(\mathcal{G}_* )||_1= \inf_{\mu \in \ker(\mathcal{G}_*)} ||\rho + \mu ||_1$. Then, since $\mathcal{G}_* $ is idempotent we also have

\[||\rho + \ker(\mathcal{G}_* )||_1= \inf_{\mu \in \ker(\mathcal{G}_*)} ||\rho + \mu ||_1 \leq ||\rho + (\mathcal{G}_* (\rho) - \rho ) ||_1 = ||\mathcal{G}_* (\rho) ||_1=||\tilde{\mathcal{G}_* }(\rho+ \ker(\mathcal{G}_* )) ||_1, \]
which thus provides an isometry between $\T(\his)_G $ and $\T(\his)^G$.
\end{proof}

Finally note that in the case of compact $G$ there is no operational difference in putting the invariance requirement on observables or states:
\begin{proposition}\label{prop:tw1}
$\tr[\mathcal{G}^*(A)\rho]=\tr[A\mathcal{G}(\rho)]=\tr[\mathcal{G}^*(A)\mathcal{G}(\rho)]$. \emph{The proof uses the invariance of $\mu$ and is straightforward.}
\end{proposition}


\subsection{Localizability}

The notion of localizability of a POVM that we are now going to recall is used in recovering the standard kinematics of quantum physics from the relational one summarized here (Th. \ref{th:con1}), and in defining frame transformations later.

The probability that a system is prepared in the state $\omega$ leads to an outcome in the set $X \in \mathcal{B}(\Sigma)$ upon a measurement of $\E$ is obtained through the Born formula:
\begin{equation}\label{eq:born}
    p_{\omega}^{\E}(X) = \tr[\omega \E (X)].
\end{equation}

For a fixed observable $\E$, a state therefore gives rise to a probability measure on $\Sigma$ through \eqref{eq:born}, often also denoted by $\mu^\E_\omega$ (for instance when used as a measure in an integral). We will refer to states as being `highly localized' in some (open) set $X$ or around some point (i.e., in an open neighbourhood $X$) in $\Sigma$, meaning that the given probability $\mu^\E_\omega(X)$ is close to unity on that $X$. If $\mathsf{P}: \mathcal{B}(\Sigma) \to B(\hi)$ is a PVM, for any $X \in \mathcal{B}(\Sigma)$ for which $\mathsf{P}(X) \neq 0$, there is a state $\omega$ for which $\tr[\omega \mathsf{P} (X)] = 1$ (set $\omega$ to be a projection onto any unit vector in the range of $\mathsf{P}(X)$), and this state, with respect to $\mathsf{P}$, is perfectly localized (with probabilistic certainty) in $X$. For example, if $\mathsf{P}=\mathsf{P}^A$, then any eigenvector of $A$ with  eigenvalue in $X$ can be understood in this way. However, since $A$ may have no eigenvalues (think of the position operator), the above characterisation is more general. The probability measures described by \eqref{eq:born} are typically not localized in any open set; there exist POVMs for which there is no state satisfying $p_{\omega}^{\E}(X)=1$ for any $X \neq \Sigma$. There are, however, POVMs which `almost' have the localizability property enjoyed by all PVMs:

\begin{definition}[Norm-$1$ property]\label{def:norm1}
A POVM $\E : \mathcal{B}(\Sigma) \to B(\hi)$ satisfies the \emph{norm-$1$ property} (see e.g. \cite{heinonen2003norm}) if for all $X \in \mathcal{B}(\Sigma)$ for which $\E(X) \neq 0$, $\no{\E (X)}=1$.  Such POVMs are called \emph{localizable}.
\end{definition}

\begin{proposition}\label{prop:norm1eq}
The following are equivalent \cite{heinonen2003norm}:
\begin{enumerate}
\item $\E$ is  localizable (i.e. satisfies the norm-$1$ property). 
    \item For every $X$ for which $\E(X)\neq 0$, there is a sequence of unit vectors $(\varphi_n)\subset \hi$
    for which $\lim_{n \to \infty}\ip{\varphi_n}{\E(X)\varphi_n}=1$. 
    \item For every $\E(X)\neq 0$ and for any $\epsilon > 0$ there exists a unit vector $\varphi_\epsilon \in \mathcal{S}(\hi)$ for which $\ip{\varphi_\epsilon}{\E(X)\varphi_\epsilon}>1-\epsilon$ (this is called the \emph{$\epsilon$-decidability property}).
\end{enumerate}
\end{proposition}

A useful consequence is the following:

\begin{proposition}\label{prop:locseqgen}
    If a POVM $\E : \mathcal{B}(\Sigma) \to B(\hi)$ satisfies the norm-1 property and $\Sigma$ is metrizable, for any $x \in \Sigma$ there exists a sequence of pure states $(\omega_n)$ such that the sequence of probability measures $(\mu^\E_{\omega_n})$ converges weakly to the Dirac measure $\delta_x $.
\end{proposition}

\begin{proof}
Fix $x\in \Sigma $ and denote by $B_n $ the open ball centred at $x$ of radius $\frac{1}{n}$. Since $\E$ satisfies the norm-$1$ property, using the $\epsilon$-decidability property of Prop. \ref{prop:norm1eq} we can choose unit vectors $\varphi_n $ such that $\braket{\varphi_n }{\E(B_n) \varphi_n }> 1 - 1/n $. Denoting by $\omega_n $ the associated pure state $\omega_n = \dyad{\varphi_n }{\varphi_n }$, we have $\tr[\omega_n E(B_n )]= \braket{\varphi_n }{\E(B_n) \varphi_n }> 1 - 1/n $, and thus $ \mu^\E_{\omega_n}(B_n ) > 1 - 1/n $.

We will show weak convergence using the portemanteau theorem \cite{billingsley2013convergence}. We must show that for each measurable set $X$ with negligible boundary (i.e., such that $\delta_x (\partial X) = 0 $ for all $x$) we have: $\lim\limits_{n\to \infty} \mu^\E_{\omega_n}(X) = \delta_x (X) $. We compute:

\[\mu^\E_{\omega_n}(X) = \mu^\E_{\omega_n}(X\setminus B_n ) + \mu^\E_{\omega_n}(X\cap B_n ) .\]

For the first term, we have $\mu^\E_{\omega_n}(X\setminus B_n ) \leq \mu^\E_{\omega_n}(\Sigma \setminus B_n ) = 1-\mu^\E_{\omega_n}(B_n ) \leq \frac{1}{n}$, which vanishes as $n$ goes to infinity. \\

For the second term we distinguish two cases:

\begin{itemize}
    \item If $ x\in X $, by hypothesis we may assume that $x\notin \partial X $ so $x\in \mathring{X} $, the interior of $X$. Since $\mathring{X}$ is open, for $n$ large enough we always have $B_n \subseteq \mathring{X} \subseteq X $, so $X\cap B_n = B_n $, and therefore $\mu^\E_{\omega_n}(X\cap B_n )= \mu^\E_{\omega_n}( B_n )> 1 - 1/n $. Thus, the second term goes to $1$ as $n$ goes to infinity.
    \item If $ x\notin X $, by hypothesis we may assume that $x\notin \partial X $ so $x\in \Sigma \setminus \overline{X} $, the complement of the adherence of $X$, which is an open set. So for $n$ large enough we always have $B_n \subseteq \Sigma \setminus \overline{X} \subseteq \Sigma \setminus X $, hence $X\cap B_n = \emptyset $, leading to $ \mu^\E_{\omega_n}(X\cap B_n )= 0 $. Thus, the second term goes to $0$ as $n$ goes to infinity.
\end{itemize}

Thus we have shown that $\lim\limits_{n\to \infty} \mu^\E_{\omega_n}(X) = \delta_x (X)$. Finally, the portemanteau theorem gives that the sequence $(\mu^\E_{\omega_n})$ converges weakly to $\delta_x$.
\end{proof}

Thus if $\E$ has the norm-$1$ property, we can approximate the Dirac delta measure centered at any $x \in \Sigma$ with measures of the form $\mu^\E_{\omega_n}(X) = \tr[\E(X)\omega_n(x)]$. We will call $\omega_n(x)$ a \emph{localizing sequence centered at $x$}. 


\subsection{Covariance}

Observables are often characterised by their covariance properties (e.g. \cite{busch1997operational}):
\begin{definition}\label{def:imp}
Let $\E:\mathcal{B}(\Sigma) \to B(\hi)$ be a POVM, $G$ a locally compact second countable topological group, and
$\alpha : G \times \Sigma \to \Sigma$ a continuous transitive action (so that $\Sigma$ is a homogeneous $G$-space) and $U:G \to B(\mathcal{H})$ a strongly continuous projective unitary representation. Then $(U,\E,\hi)$ is called a system of covariance based on $\Sigma$ if for all $X \in \mathcal{B}(\Sigma)$ and all $g \in G$, 
\begin{equation}\label{eq:covp}
    \E (\alpha (g, X))= U(g) \E (X) U(g)^*.
 \end{equation}
 $\E$ is called a covariant POVM, and if $\E$ is projection-valued, then $(U,\E\equiv \mathsf{P},\hi)$ is called a system of imprimitivity (SOI). 
\end{definition}
We will often write $g.X$ to stand for $\alpha (g, X)$, and will presume that $U$ as given above is a true unitary representation. 


Systems of imprimitivity are characterised by the Imprimitivity Theorem, 
which in a nutshell states that for a closed subgroup $H \subset G$ and $\Sigma = G/H$ with left $G$-action,  there is  (up to unitary equivalence) a one-to-one correspondence between systems of imprimitivity $(U,\mathsf{P},\hi)$ based on $\Sigma$ and unitary representations $U_{\chi}$ of $H$ (e.g. \cite{landsman2006between}). Irreducible systems of imprimitivity (those with no invariant subspaces of $(U,\mathsf{P})$) correspond to irreducible representations of $H$.\footnote{The exact correspondence is as follows. Given a representation 
$U_{\chi}$ of $H$, construct the SOI $(U^{\chi}, \mathsf{P}^{\chi},\hi^{\chi})$, with 
$\hi^{\chi}= L^2(G/H)\otimes \hi_{\chi}$ as the carrier space for
the representation $U^{\chi}$ of $G$ induced by the representation $U_{\chi}$ of $H$, and $\mathsf{P}^{\chi}$ acts as multiplication by the characteristic function on $\hi^{\chi}$. In the other direction, for any SOI $(U, \mathsf{P},\hi)$, there is a unitary representation $U_{\chi}$ of $H$ for which $(U, \mathsf{P},\hi)$ is unitarily equivalent to the one given above. We note that any space $\Sigma$ with a continuous transitive $G$-action can be written as $G/H$, where $H=H_x$ is the stabiliser of some point $x \in \Sigma$.}

For $G$ finite the canonical irreducible system of imprimitivity based on $G$ can be described very explicitly.

\begin{example}
\normalfont
Let $L^2(G)$ correspond to the set of all functions $G \to \mathbb{C}$ (sometimes denoted $\mathbb{C}[G]$), which has as an orthonormal basis the collection of indicator functions $\delta_g$.  
These define the rank-$1$ projections $P(g)$, and to make contact with more common notation in the physics literature we write $\delta_g\equiv \ket{g}$ and  $P(g)\equiv \dyad{g}$. The left regular representation is given as $U_L(g)\ket{g'} = \ket{gg'}$. Then with $P: g \mapsto \dyad{g} \in B(L^2(G))$, $(U_L,P,L^2(G))$ is a system of imprimitivity based on $G$, and up to unitary equivalence is the unique irreducible one.
\end{example}


The simple case above generalises to the following.

\begin{example}\label{ex:classical}
\normalfont
Let $G$ be a locally compact group acting transitively from the left on the measure space
$(\Sigma, \mu)$, with $\mu$ a $\sigma$-finite invariant measure on $\Sigma$,
 and take
$\mathcal{H}=L^2(\Sigma, \mu)$. Then $(U,P,\mathcal{H})$ as given below is a system of imprimitivity based on $\Sigma$.
\begin{equation}\label{eq:csi}
  \begin{aligned}
  P(X)f = \chi_X f \\
  (U(g)f)(x)=f(g^{-1}.x).
  \end{aligned}
\end{equation}
The covariance of $P$ (i.e., that $P$ satisfies \eqref{eq:covp}) follows from direct calculation. Setting $G=\Sigma$ as a topological space (so that $H=\{e\}$) yields the left-regular representation of $G$, equipped with the canonical (irreducible) system of imprimitivity based on $G$ as a left $G$-space. Further specialising to $G=\mathbb{R}$, with $\mu$ Lebesque measure and $\mathbb{R}$ understood as the configuration space of a single particle, \eqref{eq:csi} yields the standard Schr\"{o}dinger representation of wave mechanics for $P=P^Q$ ($Q$ position) and $U$ is therefore generated by momentum. The uniqueness (all up to unitary equivalence) of the irreducible representation
of $\{e\}$ corresponds to the uniqueness of the canonical commutation relation, and the imprimitivity point of view therefore constitutes a generalisation of the Stone-von Neumann theorem. The uniqueness of the shift-covariant spectral measure of position (which follows from the imprimitivity theorem)
is particular to the sharp case; there exist many shift-covariant unsharp observables which can be obtained through convolution with a Markov kernel (e.g. \cite{Busch2016a}).\footnote{It is possible to be more general in the following sense. Let $(\mathcal{A},G,\alpha)$ be a $W^*$ dynamical system, that is, $\mathcal{A}$ is a $W^*$ algebra, $G$ a locally compact group and $\alpha:G \to Aut(\mathcal{A})$ an ultraweakly continuous homomorphism to the group of automorphisms of $\mathcal{A}$. A covariant representation of $\mathcal{A}$ is a pair $(\pi,U)$, where  $\pi : \mathcal{A} \to B(\hi)$ is
a linear $*$-homomorphism $\mathcal{A} \to B(\hi)$ and $U$ is a unitary representation for which for all $g$ and $A$,
\begin{equation}
    \pi(\alpha_g (A))=U(g)\pi(A)U(g)^*.
\end{equation}
If $\mathcal{A}=L^{\infty}(G,\mu)$, a representation of $\mathcal{A}$ corresponds exactly to a PVM, and a covariant representation to a system of imprimitivity. Example \ref{ex:classical} is recovered by choosing the representation $f \mapsto M_f$ defined by $(M_f) \varphi = f \varphi$. This suggests that all the essential structure is displayed at the algebraic level.}
\end{example}

\begin{example}[Systems of coherent states]\label{ex:csspovm}
\normalfont
Systems of covariance can be constructed from certain families of generalized coherent states (see e.g. \cite{Perelomov,ali2000coherent} for coherent states, and \cite{de2021perspective} for an example of their use as quantum reference frames). For instance, set $U$ to be a unitary representation of a locally compact group $G$ in $\hi$, with a cyclic vector $\ket{\eta}$ (i.e., the span of $\{U(g) \eta\}$ is dense). Then the orbit $\{\eta_g := U(g) \eta\}$ is called a system of (Perelomov-Gilmore) coherent states, and if they satisfy a certain square integrability condition they resolve the identity in the sense that
\begin{equation}\label{eq:pgco}
    \int_G \dyad{\eta_g} d \mu (g) = \lambda \id,
\end{equation}
where as usual $\mu$ is Haar measure and $\lambda$ is some positive real number. Then
$$\E^{\eta}(X):= \frac{1}{\lambda}\int_X \dyad{\eta_g} d \mu (g)$$
is a covariant POVM, and therefore $(U,\E^{\eta},\hi)$ is a system of covariance. $\E$ does not satisfy the norm-$1$ property unless $\lambda = 1$. Note alse that sharp coherent state systems require $X$ to be discrete.

We note that this definition can be made much more general, and we refer to  \cite{ali2000coherent} for a precise treatment. A minor generalisation is to proceed as above, but set $X:=G/H$, with $H$ the stabilisier (up to a phase) of $\eta$, and define another system of coherent states $\eta_{\sigma(x)} = U(\sigma(x))\eta$, with $\sigma:X \to G$ any Borel section. Note that the measure $\mu$ in \eqref{eq:pgco} must be replaced by a quasi-invariant measure, and if this is actually invariant, $H$ is compact. This setting is explored in the perspective-neutral approach to quantum reference frames in \cite{de2021perspective} and has some interesting physical consequences, which are beyond the scope of the present work but will be addressed systematically within the framework presented in this paper in the future.

\end{example}

\begin{example}[Canonical Phase]
\normalfont
Let $\{\ket{n} \in \hi;n \in \mathbb{N}\}$ be an orthonormal basis of $\hi$, and  $N=\sum_{n \geq 0} n \dyad{n}$ a (densely defined) \emph{number observable}. Then 
$\E : \mathcal{B}((0,2 \pi]) \to B(\hi)$ is a \emph{covariant phase observable} if it 
satisfies \eqref{eq:covp} - explicitly, if $e^{iN\theta}\E(X)e^{-iN\theta} = \E(X+\theta)$, where addition is mod-$2\pi$. $U(\theta)=e^{iN\theta}$ is a strongly continuous unitary representation of the circle group, and 
$(U,\E,\hi)$ forms a system of covariance. $\E$ is used to model the phase (observable) of an electromagnetic field, with the states $\{\ket{n}\}$ corresponding to photon number. These observables are completely characterised and take the form 
\begin{equation}
    \E(X)= \sum_{n,m=1}^{\infty} c_{n,m}\int_{X}e^{i\theta(n-m)}\dyad{n}{m}\frac{d\theta}{2\pi},
\end{equation}
where $(c_{n,m})$ is a positive matrix with $c_{n,n}=1$ for all $n$. The boundedness from below of $N$ means that $\E$ is never sharp \cite{lahti1999covariant}, but the \emph{canonical phase} characterised by $c_{n,m}=1$ for all $n,m$ does satisfy the norm-1 property (\cite{lahti2000characterizations}).
\end{example}

 We note that in general there are obstacles to the existence of covariant POVMs with good localization properties, for example as given in the following proposition:
\begin{proposition}
Let $U$ be a strongly continuous unitary representation of $G$ in $\hi$, which is continuous, compact, abelian, generated by $N = \sum n \dyad{n}$, and $G$ acts on itself from the left. Moreover let $\hi$ have finite dimension $\alpha$. Then, for any 
$X \in \mathcal{B}(\Sigma)$ ($X \neq G$),
\begin{equation}
    \tr[\rho \E(X)] \leq \alpha. \mu(X),
\end{equation}
where $\mu(X):= \int_X d\mu (g)$.
\end{proposition}
The proof is straightforward: it always holds that $\tr[\rho \E(X)] \leq \tr[\E(X)]$, and since $\ip{n}{\E(X)n} = \ip{n}{\E(g.X)n}$ and therefore 
$\ip{n}{\E(X)n} = \mu(X)$ and $\tr[E(X)] = \alpha \mu(X)$ and the result follows. This means that for small $X$, a large $\alpha$ is needed to have localization probability close to $1$.

\section{Relational Quantum Kinematics}\label{sec:rqk}

At the very least a satisfactory description of quantum systems relative to quantum reference frames should
reproduce, at the probabilistic level, the predictions of orthodox quantum theory. Having built up the required mathematical machinery, we are now in position to provide a rigorous treatment of quantum reference frames and, {\it inter alia}, prove the equivalence between relative and `absolute' kinematical descriptions when the frame state is highly localized. We also provide, and analyze the structure of, relative states and observables, which occur in duality, and form the basis of the frame change map which follows. 

\subsection{Quantum Reference Frames}\label{sec:qrfs}

We are now in position to give the definition of a quantum reference frame suited to the purposes of this paper.

\begin{definition}
A quantum reference frame $\R$ for $G$ is a system of covariance $\R=(U_\R,\E_\R,\hi_\R)$ on a homogeneous (left) $G$-space $\Sigma_\R$. 
\end{definition}

When referring to a frame $\R$ we assume $G$ is given and often refer to $\E_\R$ as the frame (observable) when $U_\R$ and $\hir$ are understood, and occasionally drop the $\R$ subscript when it won't lead to confusion. We will only consider true unitary representations. A quantum reference frame is therefore a quantum physical system equipped with an observable which transforms covariantly under the action of $G$. When based on $G$, such an observable has a direct interpretation as an `orientation observable', and since covariant observables on $G/H$ can be pulled back to those on $G$, this is quite general.
 
\begin{definition}
    Two frames $\R_1 = (U_1,\E_1,\hi_1)$ and $\R_2=(U_2,\E_2,\hi_2)$ are called (unitarily) equivalent if
    there is a unitary map $U:\hi_1 \to \hi_2$ and a homeomorphism $f:\Sigma_1 \to \Sigma_2$ such that
    \[
        \E_2(X) = U\E_1(f^{-1}(X))U^*.
    \]
\end{definition}


There are various special classes of frame:

\begin{definition}
~

\begin{itemize} 
    \item A frame  $\R$ is called \emph{principal} if $\Sigma_\R$ is a principal homogeneous space, \emph{non}-principal otherwise.
    \item A frame  $\R$ is called \emph{sharp} if $\E_\R$ is sharp, unsharp otherwise.
    \item A frame  $\R$ is called \emph{ideal} if it is principal and sharp.
    \item A frame  $\R$ is called \emph{localizable} if $\E_\R$ is localizable.
    \item A frame  $\R$ is called \emph{complete} if there is no (non-trivial) subgroup $H_0 \subseteq G$ acting trivially on all the effects of $\E_\R$, and \emph{in}complete otherwise. Such an $H_0$ will be called an \emph{isotropy subgroup} for $\E_{\R}$.
    \item A frame $\R$ is called a \emph{coherent state frame} if $\E_\R$ is constructed from a system of coherent states \eqref{ex:csspovm}.
\end{itemize}

\end{definition}
Some remarks are in order. The definition of completeness is reminiscent of that appearing in other works (e.g. \cite{Bartlett2007,de2021perspective}), and agrees with that of \cite{de2021perspective} in the case of coherent state frames. This can be contrasted with the non-principal setting, which is not considered in other works, but analysed for finite homogeneous spaces in \cite{glowacki2023quantum}, in which there is a non-trivial isotropy group $H<G$ for the action of $G$ on $\Sigma$. General connections between incomplete frames and non-principal frames is the topic of current investigation.\footnote{Notice, however, that for a localizable frame and any $h \in H_0$ in the isotropy subgroup, the value space of $\E_{\R}$ as $G/H$ and $(\omega_n)$ the sequence localizing at the identity coset $eH \in G/H$, we have
\[
    \delta_{hH}(X) = \lim_{n \to \infty} \mu_{\omega_n(hH)}^{\E_\R}(X) = \lim_{n \to \infty}\tr[h^{-1}.\omega_n \E_\R] =
    \lim_{n \to \infty} \tr[\omega_n h^{-1}.\E_\R] = \lim_{n \to \infty} \tr[\omega_n \E_\R]  = \delta_{eH}(X),
\]
so that $hH=eH$ for any $h \in H_0$, and we can conclude that $H_0 < H$. In particular, localizable principal frames are complete, and in general, a localizable frame-orientation observable on $\Sigma_\R \cong G/H$ factorizes through $B(\hir)^H$, i.e. we can write
\[
    \E_\R : \mathcal{B}(G/H) \to B(\hir)^H \hookrightarrow B(\hir).
\]}
Sharp frames are described by systems of imprimitivity, which by the imprimitivity theorem, are unitarily equivalent to $L^2(G/H)$. Fixing $H=\{e\}$ and demanding that $(U,\mathsf{P},\hi)$ is irreducible yields (up to unitary equivalence) the left regular representation. The term `ideal' frames (and also `perfect'  frames \cite{Bartlett2007}) is used in the literature to mean something similar but not identical, the typical provision being that there is a collection of states in $\hi$ indexed by $g \in G$ (for example coherent state systems as in \cite{de2021perspective}), and these states must be orthogonal (or 'perfectly distinguishable') for $g \neq g'$ for the frame to be ideal. However, we avoid this usage since the notion of orthogonality is not  compatible with the Hilbert space framework within which we are working - $G$ is generally continuous, yet $\hi$ is assumed to be separable.  We note also that elsewhere, coherent states themselves are understood, through their $G$-dependence, as encoding `frame orientations'; for us, this comes probabilistically through the measure on $G$ defined by an arbitrary state through $\E_{\R}$, which we view as being more operationally motivated. There is no operational distinction between sharp frames and localizable frames based on the same space, and this is manifested clearly in the next subsection in which standard and relative observables are operationally compared.



There are various other interesting frames one can consider which we do not touch upon here, for instance \emph{informationally complete} frames, in which $\E$ is an informationally complete POVM, which is necessarily unsharp. 

\subsection{Relative States and Observables}\label{sec:relobs}

 Quantum reference frames are introduced, in analogy to classical physics, to study properties of some system relative to the given frame. Together they are described as a compound system.
The main principle put forward in \cite{Loveridge2017a,Loveridge2020a,Miyadera2015e,loveridge2017relativity} is that truly observable quantities are invariant under gauge/symmetry transformations, understood here as a diagonal action on the compound system. Here, we make a further distinction between the invariant algebra $B(\hirs)^G$, containing the `strong Dirac observables', and the observables in the image of the relativization map (defined below), which will be called \emph{relative observables}, and which are close to the relational Dirac observables as given in e.g. \cite{de2021perspective}.

\subsubsection{Relativization map}

Let $\R=(U_\R, \E_\R, \hir)$ be a principal frame and $U_\S$ be any (strongly continuous, unitary) representation of $G$ in $\his$. As has been shown in \cite{Loveridge2017a}, there is a map $\Y^{\R}: B(\his)\to B(\hirs)$ defined by
\begin{equation}\Y^\R(A_ \S) := \int_G d\E_\R(g)\otimes U_\S(g)A_\S U_\S(g)^*.
\end{equation}
This map has the following properties (not all of which are independent): $\Y^\R$ is linear, unital, adjoint-preserving, preserves effects, bounded (thus continuous), positive, completely positive, normal, multiplicative exactly when $\E_\R$ is projection valued (sharp), injective if $\E_\R$ satisfies the norm-1 property (see below); $\rm{Im}(\Y^\R)$ is a von Neumann algebra isomorphic to $B(\his)$ if $\E_\R$ is sharp and hence an embedding, and of course $\Y^\R(B(\his)) \subset B(\hirs)^{G}$ which can be verified by direct computation. Mathematically, $\Y^\R$ in the sharp case is then a faithful representation of $B(\his)$ in $\hirs$. Physically, $\Y^\R$ is understood as providing an explicit inclusion of the frame $\R = (U_\R, \E_\R, \hir)$ into the description; we view $\Y^\R(A_\S)$ as a \emph{relativization} of $A_\S$. This naturally extends to POVMs by composition. For instance, for $G=\mathbb{R}$, and fixing the appropriate frame observable, $\Y^\R$ takes the position observable $Q_\S $ to $Q_\S  \otimes \id_\R - \id_\S \otimes Q_R$ (which can be proven at the level of the respective spectral measures); it also produces unsharp relative time observables, relative angle and phase observables for $G=U(1)$  (\cite{loveridge2012quantum,Loveridge2017a,loveridge2019relative}). Note that for any $A_\S \in B(\his)^G$, and any principal frame $\R$ we have $\Y^\R(A_\S) = \mathbb{1}_\R \otimes A_\S$, and thus $\Y^\R(B(\his)^G) \cong B(\his)^G$, whatever the frame. The invariant algebra ($B(\his)^G$ or $B(\hirs)^G$) can thus be understood as a frame-independent description of the system. 

The definition of a relational Dirac observable as in \cite{de2021perspective} is recovered for $g \in G$ by $ \Y^{\R}(g.A_S)$ when $\E_\R$ is a covariant POVM associated to a coherent state system. Thus in the case of coherent state POVMs the set of relational Dirac observables and relativized observables are the same.

The $\Y$ construction \cite{Loveridge2017a} was developed as a generalization, and making rigorous, of the $\$$ map \cite{Bartlett2007}, and was used to construct invariants; the physical difference between observables in the image of $\Y^\R$ and general invariants was not appreciated at that time but will play an important role here. The twirl map arises as a special case of this construction when $\hir$ is taken to be $\mathbb{C}$, with (necessarily) trivial $G$ action. Indeed, the notion of a covariant POVM then coincides with that of a normalized invariant measure (Haar measure), and thus there is exactly one when $G$ is compact, and none otherwise. Another simple example is when $\R$ is taken to be the canonical irreducible system of imprimitivity for a finite group $G$, where $\Y^{\R}$ reads
\[
\Y^{\R}(A_\S) = \sum_{g \in G} \dyad{g} \otimes U_{\Sy}(g)A_\S U_{\Sy}(g)^*.
\]
The relativization in some sense replaces the naive `invariantization' given by the twirl and applies to the general case of locally compact groups and arbitrary (strongly continuous, unitary) representations. 
 
\subsubsection{Relative states, observables and dualities}
The relative observables have been constructed through the relativization map. For technical reasons we close the image:

\begin{definition}\label{def:Yrelobs}
An operator $A \in B(\hirs)^G$ is called $\R$-relative if it belongs to the ultraweak closure of the image of $\Y^\R$. We write $$B(\his)^\R := \Y^\R(B(\his))^{\rm cl} \subset B(\hirs)^G$$ for the set of $\R$-relative observables.
\end{definition}

Relative observables are thus invariant, but do not exhaust the invariant operators in general. By definition, $B(\his)^\R$ is ultraweakly closed, and hence also norm closed and therefore an \emph{operator space} (e.g. \cite{arveson2008operator}). 

Despite playing an essential role, the notion of `state relative to a frame' is left implicit, or perhaps taken as primary, in \cite{giacomini2019quantum}, where the `modern' notion of quantum reference frame transformation was first introduced. One objective of that work was, as we understand it, to provide frame changes without recourse to any `global' perspective. A definition of relative state is provided in \cite{de2020quantum} and the frame changes are in line with \cite{giacomini2019quantum} (though applied also in more general settings); however, no operational justification is provided, and strictly speaking the procedure applies only to finite (or perhaps discrete) $G$ since the frame state is assumed to be perfectly localized at $e \in G$. We now seek to rectify these various shortcomings, by providing an operationally motivated notion of relative state, which rigorously extends that of \cite{de2020quantum} to continuous groups.


\begin{definition}
Given a frame $R$, two trace class operators $T,T' \in \T(\hirs)$ are called \emph{$R$-equivalent}, written $T \sim_{\R} T'$, if they are operationally equivalent with respect to $\Im \Y^\R$, i.e.
$\tr[\Y^\R(A_\S)T]=\tr[\Y^\R(A_\S)T']$ for all $A_\S \in B(\his)$.
\end{definition} 
An $\R$-relative state is then a collection of density operators that cannot be distinguished by any  $\R$-relative observable.
The set of $\R$-relative states is written $\mathcal{S}(\his)^\R := \S(\hirs)/\hspace{-3pt}\sim_\R$, which by Prop. \ref{prop:statespace} is indeed a state space. $\mathcal{S}(\his)^\R $ admits an alternative characterisation:

\begin{proposition}
    Let $\R$ be any principal frame. There is a state space isomorphism
    \[
        \S(\hirs)/\hspace{-3pt}\sim_\R \cong \Y^\R_*(\S(\hirs)).
    \]
\end{proposition}

\begin{proof}
    Use Prop. \ref{generalst} for $\Lambda=\Y^\R$.
\end{proof}



This characterization will be used in the sequel, so it is worth introducing  some notation. We will write $\Omega^\R$ for $Y^\R_*(\Omega) \in \S(\his)$, and  $[\Omega]_\R$ for the corresponding equivalence class on the composite system, i.e. $[\Omega]_\R \simeq \Y^{\R}_*(\Omega) \equiv \Omega^\R$. 

As a direct consequence of the invariance of the image of $\Y^\R$, the predual $\Y^\R_*$ map factorizes through $\T(\hirs)/\hspace{-3pt}\sim_G$ and thus
\[
\S(\his)^\R \cong \Y^\R_*(\S(\hirs)/\hspace{-3pt}\sim_G).
\]
 For $G$ compact we then have $\Y^\R_*(T)=\Y^\R_*(\mathcal{G}_* (T))$; $\mathcal{G}_* (\omega \otimes \rho )$ is sometimes referred to as the 'relational encoding' of $\rho$ in the older quantum reference frames literature (c.f. \cite{palmer2014changing}). Note that $\Y^{\R}_*$ also factors through the $G$-orbits; if the action of $G$ is understood as a gauge transformation, applying $\Y_*^\R$ to the state on $B(\hirs)$ can thus be viewed as providing a relative description whilst maintaining gauge invariance. 

In standard quantum mechanics the duality $\T(\hi)^*\cong B(\hi)$ (which is an isomorphism of Banach spaces) broadly establishes the states and observables as dual objects: each is determined by the other. This duality persists as the level of relative states and observables:

\begin{proposition}\label{prop:reldual}
    There is an isomorphism of Banach spaces
    \[
        \big[\T(\his)^\R\big]^* \cong B(\his)^\R.
    \]
\end{proposition}

\begin{proof} Apply Proposition \ref{prop:iml} with $\mathcal{O}=\Im \Y^\R$.
\end{proof}

An example of relative observables arises when one frame is relativized with respect to another.


\begin{definition}\label{def:relorobs}
    For a pair of principal frames $\R_1$ and $\R_2$ define the \emph{observable of relative orientation} (of $\R_2$ with respect to $\R_1$), denoted by $\E_2*\E_1$, via the relativization map
    \begin{equation}
    \E_2 * \E_1 :=\Y^{\R_1} \circ \E_2 = \int_G d\E_1(g) \otimes g.\E_2(\cdot).\footnote{
    The notation is chosen to reflect the observation that when $\his$ and $\hir$ are both taken to be $\mathbb{C}$, the formula reproduces standard convolution of measures.}
    \end{equation}
\end{definition}

Indeed, many important examples of relative observables---position, time, phase and angle (\cite{loveridge2012quantum, Loveridge2017a})---are relative orientation observables. The name can be further justified by the following.
\begin{proposition}
For a pair of principal localizable frames $\R_1$ and $\R_2$ and corresponding localizing sequences $(\omega_n)$ and $(\rho_m)$ (centred at $e \in G$), writing $\Omega_{n,m}(h) := \omega_n \otimes h^{-1}.\rho_m$ we have
\[
\lim_{n,m\to\infty} \mu_{\Omega_{n,m}(h)}^{\E_2 * \E_1} = \delta_h.
\]
\end{proposition}

\begin{proof}
    We calculate
    \begin{align*}
    \lim_{n,m\to\infty} \tr[(\omega_n \otimes h^{-1}.\rho_m) \int_G d\E_1(g) \otimes g.\E_2(X)]
    &= \lim_{m\to\infty} \tr[h^{-1}.\rho_m \E_2(X)]\\
    &= \lim_{m\to\infty} \tr[\rho_m \E_2(h^{-1}.X)]
    = \delta_e(h^{-1}.X) = \delta_h(X),
    \end{align*}
    where we have used \ref{prop:locseqgen} twice.
\end{proof}

Thus when the system is localized at $h \in G$, and the frame at the identity, the relative orientation observable is localized at $h$, which actually encodes the relation between the system and frame. Note that since $\E_2 * \E_1$ is invariant, we could just as well evaluate it on $(h.\omega_n \otimes \rho_m)$, with the same result. Notice also that if $\E_1$ is relativized with respect to $\E_2$, the probability distribution of the relative orientation observable evaluated on $\omega_n \otimes h^{-1}.\rho_m$ is localized at $h^{-1}$ as one would expect. Indeed, a simple calculation gives
\[
\E_2 * \E_1 (X) = {\rm SWAP}_{1,2} \circ \E_1 * \E_2 (X^{-1}),
\]
where ${\rm SWAP}_{1,2}$ takes care of switching the tensor product factors as in \cite{de2020quantum}, i.e. for $A_1 \otimes A_2 \in B(\hio \otimes \hit)$ we have ${\rm SWAP}_{1,2}(A_1 \otimes A_2) = A_2 \otimes A_1 \in B(\hit \otimes \hio)$.


\subsection{Frame-conditioned Observables and States}

We now recall the restriction map (e.g. \cite{loveridge2017relativity}) that allows the conditioning of observables of the system-plus-reference with a specified state of the reference.

\begin{definition}\label{def:rest}
Let $\omega \in \S(\hir)$ be any state of the reference. The \emph{restriction map} $\Gamma_{\omega}: B(\hirs)\to B(\his)$ is given by
\begin{equation}\label{eq:res}
    \tr[\rho\Gamma_{\omega}(A)] = \tr[(\omega \otimes \rho )A],
\end{equation}
holding for all $A \in B(\hirs)$ and $\rho \in \mathcal{S}(\his)$. Equivalently, it is the continuous linear extension to $B(\hirs)$ of
\[
\Gamma_{\omega}: A_R \otimes A_\S \mapsto \tr[\omega A_R] A_\S,
\]
or the dual of the isometric embedding $\mathcal{V}_{\omega} : \rho  \mapsto \omega \otimes \rho $, i.e., $\tr[\rho \Gamma_{\omega}(A)] = 
\tr[\mathcal{V}_{\omega}(\rho)A]$.\footnote{Versions of this map have appeared elsewhere in the literature, often in a rather ill-defined form. Usually the implicit desired map is $B(\hirs) \ni A \mapsto \ip{\varphi_R \otimes \cdot }{A \varphi_R \otimes \cdot}: \his \times \his \to \mathbb{R}$ for $A=A^*$, which as a bounded real quadratic form defines uniquely a bounded self-adjoint $A_\S \in B(\his)$ which for all $\varphi_s$ satisfies $\ip{\varphi_R \otimes \varphi_\S  }{A \varphi_R \otimes \varphi_\S }= \ip{\varphi_\S }{A_\S \varphi_\S } =: \ip{\varphi_\S }{\Phi_{\varphi_R}(A) \varphi_\S }$. The generalisation of $\Phi_{\varphi_R}$ to mixtures is then given by $\Gamma_{\omega}$ defined in \eqref{eq:res}.}
\end{definition}

The restriction map $\Gamma_{\omega}$ is normal, completely positive and trace-preserving (hence trace norm continuous), and is a noncommutative conditional expectation (e.g., \cite{takesaki1972conditional}; see \cite{kadison2004non} for a comprehensive review). It is equivariant (covariant) exactly when $\omega$ is invariant. It can be understood as providing a description of the system contingent on a particular state of the frame.

The restriction map can of course be defined on the relative observables, i.e., those in the image of $\Y^{\R}$, in which case:
\begin{definition}
The map
\[
\Y^\R_\omega:= \Gamma_\omega \circ \Y^\R: B(\his) \to B(\his),
\]
is called the \emph{$\omega$-conditioned relativization map}.
\end{definition}
This map gives a typically non-invariant description in $B(\his)$ of the gauge-invariant description inside $B(\hir \otimes \his)^G$ (or $B(\his)^{\R}$).
As a composition of such maps the $\omega$-conditioned relativization map is unital, normal and completely positive. The $\omega$-conditioned relativized observables take the form:
\begin{equation}\label{eq:croi}
\Y^\R_\omega(A_\S) = \int_G d\mu^{\E_\R}_\omega(g) U_\S(g) A_\S U_\S(g)^*.
\end{equation}
This can be understood as a weighted average of the operator $A_\S$ with respect to the probability distribution of the frame-orientation observable.
Note that different $\omega$ may give rise to the same measure $\mu^{\E_\R}_\omega$ which in turn gives rise to the same $\omega$-conditioned relativization. 

\begin{definition}
    The operator space given by the ultraweak closure of ${\rm Im}\Y^\R_\omega$, i.e.,
    \[
    B(\his)^\R_\omega := (\Y^\R_\omega(B(\his)))^{\rm cl} \subseteq B(\his),
    \]
    will be called the space of \emph{$\omega$-conditioned relative observables}.
\end{definition}
We may again identify those density operators which cannot be distinguished in $B(\his)^\R_\omega$:

\begin{definition}
    The $(\R,\omega)$-equivalence relation on $\T(\his)$ is the operational equivalence relation with respect to \linebreak $\mathcal{O}_{\R,\omega} := \Im \Y^\R_\omega$, i.e., for $T,T' \in \T(\his)$ we write $T \sim_{(\R,\omega)} T'$ if $\tr[T \Y^\R_\omega(A_\S)] = \tr[T \Y^\R_\omega(A_\S)]$ for all $A_\S \in B(\his)$.
\end{definition}

Prop. \ref{prop:iml}, \ref{prop:statespace} and \ref{generalst} immediately give the following.

\begin{proposition}
    For any state $\omega \in \S(\hir)$ the $\omega$-conditioned relative observables form a Banach space and there is the following isomorphism
    \begin{equation}\label{eq:adrr}
   B(\his)^\R_\omega \cong \left[\T(\his)/\hspace{-3pt}\sim_{(\R,\omega)}\right]^* =: \T(\his)_{\omega}^{\R}.
    \end{equation}
    Moreover, the set $\S(\his)/\hspace{-3pt}\sim_{(\R,\omega)}$ of \emph{$\omega$-conditioned relative states} is a state space in $\T(\his)^{\rm{sa}}/\hspace{-3pt}\sim_{(\R,\omega)}$ and there is a state space isomorphism
    \[
       (\Y^\R_\omega)_*(\S(\his)) \cong \S(\his)/\hspace{-3pt}\sim_{(\R,\omega)} =:\Sy(\his)_{\omega}^{\R}
    \]
\end{proposition}

The $\omega$-conditioned relative observables are understood as follows. 
 If we view the reference $\R$ as initially external to the system, meaning that it is not explicitly realised in the theoretical description but nevertheless plays some role in the description of the phenomena/observable probability distributions, the relativization map $\Y^\R$ describes how some observable $A_\S \in B(\his)$ is realized as an invariant observable $\Y^\R(A_\S) \in B(\hirs)^G$ (naturally extending to POVMs) with respect to the external frame $\R$. In this sense the relativization map includes the external frame explicitly. The restriction map then provides a description of $\Sy$ contingent on the given state preparation of $\R$, after which $\R$ can be viewed again as external, arriving at 
the corresponding subset of observables of $\S$. We note (and discuss further later) that not all of $B(\his)$ is typically in the image of the $\omega$-conditioned relativization, and therefore the definition is not vacuous. Eq. \eqref{eq:adrr} captures again a duality between states and observables in the $\omega$-conditioned relativized setting.

Since $(\Y^{\R}_{\omega})_* = \Y^{\R}_* \circ \mathcal{V}_{\omega}$ (c.f. Def. \ref{def:rest}), the $\omega$-conditioned relativized states take the form
\begin{equation}
    \rho ^{(\omega)} := \Y^\R_*(\omega \otimes \rho ) = \int_G d\mu _{\omega}^{\E_\R}(g) U_\S(g)^*\rho  U_\S(g),
\end{equation}
 where the measure $\mu_{\omega}^{\E_\R}(X)$ is given as usual by $\mu_{\omega}^{\E_\R}(X):= \tr[\E_\R(X)\omega]$, and we have introduced the notation $\rho ^{(\omega)}$ to indicate the particular frame state that is used for conditioning. The conditioned relative states arise from product states on the global system; in the case of localizable frames they in some sense generalize the alignable states of \cite{krumm2021quantum} (we recall that $\Y^{\R}_*$ factors through the $G$-action). Notice again, as in Eq. \eqref{eq:croi}, that the state $\rho ^{(\omega)}$ depends only on the measure $\mu _{\omega}^{\E_\R}$ once $U_{\Sy}$ and $\rho$ are given.

\begin{proposition}\label{prop:sco}
Conditioned relative states satisfy the following symmetry condition
\begin{equation}
    \rho^{(h.\omega)} = (h^{-1}.\rho)^{(\omega)}
\end{equation}
\end{proposition}

\begin{proof}
We calculate:
\begin{displaymath}
    \rho^{(h.\omega)} = \int_G d\mu^{\E_\R}_{h.\omega}(g)g.\rho = \int_Gd\mu^{\E_\R}_{\omega}(hg)g.\rho\\
    = \int_Gd\mu^{\E_\R}_{\omega}(g')(h^{-1}g').\rho = \int_Gd\mu^{\E_\R}_{\omega}(g')g'.(h^{-1}.\rho) = (h^{-1}.\rho)^{(\omega)}.
\end{displaymath}
We have changed the integration variable $g'=hg$ and used the fact that $\mu^{\E_\R}_{h.\omega}(g) = \mu^{\E_\R}_{\omega}(hg)$ which follows directly from the covariance of $\E_\R$.
\end{proof}

Thus reorienting the frame by $h \in G$ is equivalent
to reorienting the system by $h^{-1} \in G$. This represents a type of `active-versus-passive' equivalence for quantum frame reorientation.

The next proposition demonstrates the plausible claim that invariant system states are defined without reference to an external frame or, more precisely, are independent from the chosen reference. 
\begin{proposition}\label{prop:foai}
Let $\rho \in \Sy(\his)^G$. Then $\rho ^{(\omega)} = \rho$ for any $\omega$, and any choice of frame $\R$.
\end{proposition}

\begin{proof}
    We calculate:
    \[
        \rho ^{(\omega)} = \int_G d\mu _{\omega}^{\E_\R}(g) U_\S(g)^*\rho  U_\S(g) = \int_G d\mu _{\omega}^{\E_\R}(g) \rho = \rho,
    \]
    where we only needed to use invariance of $\rho$ and normalization of $\E_\R$.
\end{proof}
We recall that the invariance of $\rho$ typically requires $G$ compact. It follows directly from Prop. \ref{prop:sco} that $\rho ^{(h.\omega)} = \rho ^{(\omega)}$ if $\rho$ is invariant, showing directly how the conditioned state description of $\Sy$ is not sensitive to frame reorientations if $\rho$ is invariant, as one would expect. We note that the frame-independence of invariant states follows directly from our framework, and does not need to be stated as an assumption. Observe that Prop. \ref{prop:foai} can be immediately generalized in order to pertain to  non-compact $G$, or more precisely, to the setting where we do not demand that $\rho$ is literally invariant under the representation of $G$, but instead sits in the class (or is identified with the class) of $G$-indistinguishable
density operators.
\begin{proposition}
    For any $\rho \in [\rho] \in \Sy(\his)_G$, $\rho^{(\omega)}=\rho$.
\end{proposition}
\begin{proof}
   Set $A\in B(\his)^G$. Then 
   \begin{equation*}
       \tr[\rho^{(\omega)}A] =\tr [ \int_G d\mu _{\omega}^{\E_\R}(g) U_\S(g)^*\rho U_\S(g) A] 
       = \int_G \tr [d\mu _{\omega}^{\E_\R}(g) \rho U_\S(g) A U_\S(g)^*] = \tr [\rho A], 
   \end{equation*} 
   using the cyclicity of the trace, the invariance of $A$ and again the normalisation of the measure. 
\end{proof}





Another plausible intuition---that a reference in an invariant state can only give rise to invariant relative states---is confirmed by the following proposition.

\begin{proposition}\label{Prop:soai}
 Suppose $\omega \in \Sy(\hir)$ is invariant. Then $\rho ^{(\omega)} = \mathcal{G}(\rho )$ for any $\rho $.
\end{proposition}

\begin{proof}
    If $\omega$ is invariant, then $\mu^{\E_\R}_{\omega}$ is Haar measure, denoted $d\mu$ as before. We then have
    \begin{equation*}
        \rho ^{(\omega)} = \int_G d\mu(g) U_\S(g)^*\rho U_\S(g) = \mathcal{G}(\rho). 
        \end{equation*} 
        \end{proof}
Thus if the reference is in an invariant state, the only relative states defined with respect to it are also invariant.

Now consider $\hir = \mathbb{C}$. As already noted, any unitary representation
on $\mathbb{C}$ is trivial, the only state is trivially invariant, and a covariant POVM $\E: \mathcal{B}(G) \to B(\mathbb{C}) \simeq \mathbb{C}$ is a measure $\mu_\E$ on $G$ satisfying $\mu_\E(g.X) = g.\mu_\E(X) = \mu_\E(X)$, and is thus the normalized Haar measure. There is a unique such covariant POVM iff $G$ is compact, and in that case, $\Y^\R_* = \mathcal{G}$.  Therefore, there will only be invariant states in the image. From this we conclude that in general not all states are relative states, and cannot be approximated by relative states.




\subsubsection{Localizing the Reference}

The structure of the $\omega$-conditioned relative observables is dictated by the frame $\R$ and the state $\omega$, as has been investigated in
e.g. \cite{Loveridge2017a,Miyadera2015e}. Those works analyzed the agreement between the `absolute/non-relational' description
of $\S$ and the relative one given by $\S + \R$, in which it was found that the localization/delocalization of $\omega$ was the key ingredient (for good/not good agreement). The main positive result in this vein (i.e., addressing the question of good agreement between relative/non-relative quantities) is Theorem 1. in \cite{Loveridge2017a}, which we now generalize:
\begin{theorem}\label{th:con1}
Let $\R = (U_\R,\E_\R,\hi_\R)$ be a localizable principal frame and $(\omega_n)$ a localizing sequence centered at $e \in G$. Then for any $A_\S \in B(\his)$ we have
\begin{equation}\label{eq:ftqf}
    \lim_{n \to \infty}(\Gamma_{\omega_n}\circ \Y^\R)(A_\S) = A_\S,
\end{equation}
where the limit is understood as usual in the ultraweak sense.
\end{theorem}

\begin{proof}
    It is enough to check the agreement of expectation values of both sides of Eq. \eqref{eq:ftqf}. Thus take $\rho \in \S(\his)$ and calculate
    \[
    \tr[\rho(\Gamma_{\omega_n}\circ \Y)(A_\S)] = \tr[\rho\int_G d\mu^{\E_\R}_{\omega_n}(g) g.A_\S ] = 
    \int_G d\mu^{\E_\R}_{\omega_n}(g) \tr[\rho (g.A_\S)] 
    \]
    
    The function $g\mapsto \tr[\rho (g.A_\S)]$ is continuous and bounded, and by Prop. \ref{prop:locseqgen} the sequence of measures $(\mu^{\E_\R}_{\omega_n})$ converges weakly to $\delta_e $, so by the portemanteau theorem we have: 
    \[\lim_{n \to \infty} \int_G d\mu^{\E_\R}_{\omega_n}(g) \tr[\rho (g.A_\S)] =  \tr[\rho A_\S] ; \]
    therefore the sequence of operators $ (\Gamma_{\omega_n}\circ \Y^\R)(A_\S) $ converges to $A_\S $ in the ultraweak topology.
\end{proof}

Thus the usual non-relational kinematics of quantum mechanics is recovered in the operational sense as a limiting procedure of localizing the reference -- any observable of $\S$ can be approximated to arbitrary precision by an observable relative to a localizable frame prepared in a highly localized state. Given that typically a large Hilbert space dimension is required for localizability, this points to a sort of classicality requirement on the frame. There is also a kind of converse to this statement proven in \cite{Miyadera2015e}: if $\R$ is not localizable, there is a lower bound on the difference between an effect $F_\S$ and $(\Gamma_{\omega} \circ \Y^{\R})(F_\S)$.

Notice that if $G$ is finite and the frame is ideal, no such limiting procedure is needed and the agreement is exact, i.e., we can choose  $\omega_{\R} = \dyad{e}$ and it holds that $(\Gamma_{\omega_R} \circ \Y^{\R}) = \mathbb{1}_\S$. Indeed, for $\E_\R(g) = P(g) = \dyad{g} \in B(L^2(G))$, any state is an $\ket{e}$-conditioned relative state since we have
\begin{equation}
\Y^\R_*(\dyad{e} \otimes \rho) = \sum_{g \in G}\tr[P(g)\dyad{e}]g.\rho = \sum_{g \in G} \delta_{ge}g.\rho_S = \rho.
\end{equation}

This is the setting considered in \cite{de2020quantum}, and states of the form $\dyad{e} \otimes \rho$ are called ``aligned" (to the identity) \cite{krumm2021quantum}. Since $\Y^\R_*$ is constant on orbits, the above calculation is not sensitive to whether the state is aligned or only alignable \cite{krumm2021quantum}. In the above sharp case we then have $\S(\his)^\R_{\ket{e}} \cong \S(\his)$. A similar thing happens for any localizable frame. Indeed, dualizing Th. \ref{th:con1} above, for a localizing sequence $(\omega_n)$ centred at $e \in G$ and an arbitrary $\rho \in \S(\his)$ we have
\[
\lim_{n\to \infty}\Y^\R_*(\omega_n \otimes \rho)
=\lim_{n\to \infty} \int_G d\mu^{\E_\R}_{\omega_n}(g) g.\rho = \rho.
\]
We summarise this as a proposition:
\begin{proposition}\label{cor:locrelst}
Let $\R$ be a localizable frame. Then $\mathcal{S}(\his)^\R$ is operationally dense in $\mathcal{S}(\his)$. 
\end{proposition}

Thus for localizable frames, any state of $\Sy$ can be arbitrarily well approximated by a sequence of relative states of the form $\Y^\R_*(\omega_n \otimes \rho)$. Keeping the localizability assumption, we can go even further with identifying the relative notions with the standard non-relative ones upon high localization of the reference. This points to how the standard quantum-mechanical framework, developed in the context of macroscopic, classical measuring apparatuses, is realized as a limiting case of the relational framework presented here.

\begin{proposition}
    For a localizable principal frame $\R$ the relativization map $\Y^\R$ is an isometric isomorphism between $B(\his)$ and $B(\his)^\R$ (which inherits the von Neumann algebra structure from $B(\his)$ via $\Y^\R$). If further $\R$ is sharp then $\Y^\R$ is multiplicative making $B(\his)^\R$ a subalgebra of $B(\hir \otimes \his)$.
\end{proposition}

\begin{proof}
We will first show that if $\R$ is localizable then $\Y^\R$ is isometric. Given $A\in B(\his)$ it is shown in \cite{Loveridge2017a} that $||\Y^{\R}(A_{\Sy})||\leq ||A_{\Sy}||$. So it just remains to be shown that $||A_{\Sy}|| \leq ||\Y^{\R}(A_{\Sy})||$. By Th. \ref{th:con1} there is a localizing sequence of states $(\omega_n)$ such that for all $\rho\in\S(\his) $:

\[|\tr[\rho A_{\Sy} ]|= \lim_{n\to \infty} |\tr[\rho(\Gamma_{\omega_n } \circ \Y^{\R}) (A_{\Sy})]|=\lim_{n\to \infty} |\tr[(\omega_{n}\otimes \rho) \Y^{\R} (A_{\Sy})]|\leq \sup_{\mu} |\tr[\mu \Y^{\R} (A_{\Sy})]|=||\Y^{\R} (A_{\Sy})||.\]

This being true for all $\rho$ it follows that $||A_{\Sy}||= \sup_{\rho}|\tr[\rho A_{\Sy} ]| \leq ||\Y^{\R} (A_{\Sy})||$. Thus if $\R$ is localizable, then $\Y^{\R}$ is injective and then we can unambiguously define an algebra structure on $B(\his)^\R $ by $\Y^{\R}(A_{\Sy})*\Y^{\R}(B_{\Sy})=\Y^{\R}(A_{\Sy}B_{\Sy})$. This algebra is a $C^* $-algebra since:
\[||\Y^{\R}(A_{\Sy})^* * \Y^{\R}(A_{\Sy}) ||=||\Y^{\R}(A_{\Sy}^* )*\Y^{\R}(A_{\Sy}) ||=||\Y^{\R}(A_{\Sy}^* A_{\Sy} )||=||A_{\Sy}^* A_{\Sy}||=||A_{\Sy}^* || ||A_{\Sy}||=||\Y^{\R}(A_{\Sy})^* || ||\Y^{\R}(A_{\Sy})||.\]
Since this algebra has a predual, $B(\his)^\R $ is a von Neumann algebra. By definition, $\Y^{\R} $ is an isometric homomorphism and then $B(\his)^\R $ is isometrically isomorphic to $B(\his)$. If furthermore $\R$ is sharp, then $\Y^{\R}$ is multiplicative and $B(\his)^\R $ is actually a von Neumann subalgebra of $B(\hirs)$.
\end{proof}



\section{Internal Quantum Reference Frame Transformations}\label{sec:fcp}

In this section, we consider how to transform the state description relative to one frame to that relative to another. The starting point is the invariant algebra $B(\hi_\T)^G$ ($\hi_{\T}$ is some given `global' Hilbert space) of frame-independent observables, which is the `universe of discourse' for the \emph{internal quantum reference frames programme} as envisioned in this work (though differs from e.g. \cite{hohn2022internal,de2021perspective} where a smaller algebra, given as the operators on the \emph{physical Hilbert space}, is chosen). This is in contradistinction to e.g. \cite{palmer2014changing}, which is concerned with recovering the state of a quantum system described relative to one frame from the state relative to another.

The idea is to choose some subsystem with respect to which the other subsystems can be described, and find some consistent means by which to change the description relative to one subsystem to the description relative to another. We then assume that $\hi_\T$ decomposes as $\hi_\T \cong \hi_1 \otimes \hi_2 \otimes \his$, on which we have as usual a strongly continuous unitary representation $U_\T: G \to B(\hi_\T)$ such that $U_\T = U_1\otimes U_2 \otimes U_{\S}$.
We consider the setting in which $\R_1$ and $\R_2$ are able to play the role of quantum reference frames so that each comes equipped with a frame orientation observable (system of covariance based on $G$), denoted $\E_1$ and $\E_2$ respectively. We will also consider situations where there are more than two subsystems that can be used as reference frames, e.g. $\hi_\T \cong \hi_1 \otimes \hi_2 \otimes \hi_3 \otimes \his$, and so on.

In this paper we only treat the quantum reference frame transformations in the case where the initial frame $\R_1$ is localizable, leaving the analysis of the general case, if possible, to future work. A central observation of the perspective-neutral approach (see e.g.  \cite{de2021perspective}) is that to transform between relative descriptions, one must pass through a global description, in which there is the full physical Hilbert space involving the system and all frames; the framework here follows this point of view, but differs in the implementation. We will also see that the frame change we provide reflects the intuition given in \cite{de2020quantum} that to pass from the relative state to the corresponding `global' state we need to  
`attach the identity state'. Whilst the states $\ket{e}$ used in \cite{de2020quantum} are not available as normal states in the case of continuous groups, we can make this intuition precise in the context of arbitrary locally compact topological groups using localizing sequences. Indeed, we have (Prop. \ref{prop:locseqgen}) that for $(\omega_n)$ a localizing sequence at $e \in G$,
    \[
        \lim_{n \to \infty}\Y^\R_*(h. \omega_n \otimes \Omega^\R) = \lim_{n \to \infty}\int_G d\mu_{\omega_n}^{\E_\R}(g) g.\Omega^\R = \Omega^\R
    \]
for any $h \in G$. Thus in the case of a localizable principal frame $\R$, any $\R$-relative state can be obtained as the limit of a sequence of $\omega_n$-conditioned relative states as above.

\subsection{Relativized Restriction and Lifting}

Consider a global (invariant) description in terms of $\hi_\T$ with two systems $\S$ and $\R$ distinguished, with $\hi_\T \cong \hirs$, for which the invariant observables may be conditioned upon a particular state of the reference system by applying the restriction map. If further, the factorization into $\hirs$ respects the $G$-action, i.e., $U_\T = U_\R \otimes U_\S$, and $\R$ is indeed a frame equipped with a covariant POVM $\E_\R: \mathcal{B}(G) \to B(\hir)$, we can relativize the restricted observables, arriving at the subset of the relative observables. Thus the following definition. %referred to as internal $\omega$-conditioned relative observables and given by $B(\his)^\omega:= \Y^\R \circ \Gamma_\omega(B(\hit \otimes \his)^G) \subseteq B(\his)^\R$.


\begin{definition}
The map
\[
\Gamma^\R_\omega := \Y^\R \circ \Gamma_\omega: B(\hirs)^G \to B(\hirs).
\]
will be called the ($\R$-)\emph{relativized $\omega$-restriction~map}.
\end{definition}
As a composition of such maps $\Gamma^\R_\omega$ is unital, normal and completely positive. It may be understood as providing a relative 'version' of a given invariant. The discrepancy between an invariant effect $E$ and its relativized $\omega$-restricted version is $||\Gamma^\R_\omega(E)-E||$
with the best case being estimated as $\inf_{\omega}||\Gamma^\R_\omega(E)-E||$. If $\R$ is localizable and $E$ is already relative, it is readily seen that this discrepancy can be made arbitrarily small.

\begin{definition}
    The predual $ \mathcal{L}^\R_\omega:=(\Gamma^\R_\omega)_*: \T(\his) \supset \T(\his)^{\R}  \to \T(\hir \otimes \his)_G$ of $\Gamma^\R_\omega$ is called the ($\R$-relativized) $\omega$-lifting map.
\end{definition}
Given that the image of this map consists of (classes of) product states,
there is a clear analogy with the disentangler of e.g. \cite{de2021perspective}; concretely we have
\[
\mathcal{L}^\R_\omega : \Omega^{\R} \mapsto [\omega \otimes \Omega^{\R}]_G \in \Sy(\hir \otimes \his)_G.
\]
The $\omega$-lift allows for the `attachment' of a frame state whilst respecting the
symmetry-induced operational equivalence class structure.

\begin{definition}
    The operator space given by the ultraweak closure of the image of the relativized $\omega$-restriction map, denoted $B(\his)_\R^\omega := (\Gamma^\R_\omega(B(\hirs)))^{\rm cl} \subseteq B(\hirs)^G$, will be called the space of \emph{relativized $\omega$-conditioned observables}.
\end{definition}

As usual, we define the corresponding operational equivalence relation.

\begin{definition}
    We define the $(\omega,\R)$-equivalence relation on $\T(\hirs)$ to be the operational equivalence relation with respect to $\mathcal{O}_{(\omega,\R)} := \Im \Gamma^\R_\omega$, i.e., for $T,T' \in \T(\hirs)$ we have $T \sim_{\omega,\R} T$ if $\tr[T \Gamma^\R_\omega(A)] = \tr[T' \Gamma^\R_\omega(A)]$ for all $A \in B(\hirs)^G$.
\end{definition}

Props. \ref{prop:iml}, \ref{prop:statespace} and \ref{generalst} immediately give the following.

\begin{proposition}
    For any state $\omega \in \S(\hir)$ the relativized $\omega$-restricted observables form a Banach space and we have the following isomorphism
    \[
[\T(\his)^{\omega}_{\R}]^* := \left[\T(\hirs)/\hspace{-3pt}\sim_{(\omega,\R)}\right]^* \cong B(\his)_\R^\omega.
    \]
    Moreover, the set
    \[
    \S(\his)^{\omega}_{\R}:=\S(\hirs)/\hspace{-3pt}\sim_{(\omega,\R)}
    \]
    of \emph{$\omega$-disentangled global states} is a state space in $\T(\hirs)^{\rm{sa}}/\hspace{-3pt}\sim_{(\omega,\R)}$ and we have a state space isomorphism
    \[
        \S(\his)^{\omega}_{\R} \cong (\Gamma^\R_\omega)_*(\S(\hirs)).
    \]
\end{proposition}

Thus the $\omega$-disentangled global states are classes of states, elements of which cannot be distinguished by the relativized $\omega$-conditioned observables.

In the case of localizable frame $\R$ using $\mathcal{L}^\R_{\omega_n}$, we can lift an arbitrary $\R$-relative state to a global state on the invariant algebra that gives back the initial relative state to arbitrary precision upon applying $\Y^\R_*$, thus providing an approximate right-inverse to $\Y^\R_*$.

\begin{proposition}
    Given a localizable frame $\R$ and a localizing sequence $(\omega_n)$ centred at $e \in G$ we have
    \[
    \lim_{n \to \infty} \Y^\R_* \circ \mathcal{L}^\R_{\omega_n} (\Omega^\R) = \lim_{n \to \infty} \Y^\R_*[\omega_n \otimes \Omega^\R]_G= \Omega^\R.
    \]
\end{proposition}


\subsection{Framed Relative States and Observables}

From an operational standpoint, it is reasonable to suggest that the input relative states in a frame change should be given as operational equivalence classes with respect to the relativized operators of the form $\E_2(X) \otimes A_\S$, taking into account the choice of the second frame observable. Thus the following definition. 

\begin{definition}
    We define the $(\R_1,\E_2)$-equivalence relation on $\T(\hio \otimes \hit \otimes \his)$, denoted $\Omega \sim_{\R_1,\E_2} \Omega'$, to be the operational equivalence with respect to the following set
    \[
        \mathcal{O}^{\R_1}_{\E_2} = \{\Y^{\R_1}(\E_2(X) \otimes A_\S) | X \in \mathcal{B}(G), A_\S \in B(\his)\}.
    \]
    Elements of the corresponding operational state space
    \[
\S(\hit \otimes \his)^{\R_1}_{\E_2} := \S(\hio \otimes \hit \otimes \his)/\hspace{-3pt}\sim_{(\R_1,\E_2)} = \S(\hit \otimes \his)^{\R_1}/\hspace{-3pt}\sim_{\E_2}
\]
will be called \emph{$\E_2$-framed $\R_1$-relative states} (or \emph{framed relative states for short, when the context gives unambiguous meaning}) and denoted as $[\Omega^{\R_1}]_{\E_2}$. The associated operator space
\[
B(\hit \otimes \his)^{\R_1}_{\E_2} := {\rm span}(\mathcal{O}^{\R_1}_{\E_2})^{\rm cl}\cong [\T(\hit \otimes \his)^{\R_1}_{\E_2}]^*
\]
will be referred to as the space of $\E_2$-framed $\R_1$-relative observables. The affine map given as a quotient projection from relative to framed-relative trace class operators will be denoted as
\[
\pi_{\E_2}: \T(\hit \otimes \his)^{\R_1} \to \T(\hit \otimes \his)^{\R_1}_{\E_2}.
\]
\end{definition}


Notice here that in the absence of the system $\S$, for $\Omega, \Omega' \in \S(\hio \otimes \hit)$ we have
    \[        \S(\hit)^{\R_1}_{\E_2} = \S(\hio \otimes \hit)/\hspace{-3pt}\sim_{\E_1 * \E_2},
    \]
where $\sim_{\E_2 * \E_1}$ denotes operational equivalence with respect to the relative orientation observable $\E_2 * \E_1 = \Y^{\R_1} \circ \E_2$. Thus in this situation, the $\E_2$-framed $\R_1$-relative states only allow for the measurement of the relative orientation observable. It is easy to see that in the general case, the framed relative states also allow one to separate states with respect to the \emph{extended relative orientation observable} $\E_2 * \E_2 \otimes \mathbb{1}_\S$. Indeed $\Y^\R(\E_2(X) \otimes \mathbb{1}_\S) = \E_2 * \E_2 (X) \otimes \mathbb{1}_\S \in \mathcal{O}^{\R_1}_{\E_2}$.




\subsection{Changing Reference}

We now turn to the frame change procedure. The quantum reference frame change map should translate the ($\E_2$-framed) $\R_1$-relative description to the ($\E_1$-framed) $\R_2$-relative one and thus it should be a map between the following state spaces
\[
\Phi_{1 \to 2}: \S(\hit \otimes \his)^{\R_1}_{\E_2} \to \S(\hio \otimes \his)^{\R_2}_{\E_1}.
\]
In analogy to the perspective-neutral approach \cite{de2021perspective}, this is achieved by passing through the `global' description which involves the system and both frames, making sure to respect the symmetry constraint, i.e., the global description is $\Sy(\hio \otimes \hit \otimes \his)_G$. This is done using the lift $\mathcal{L}_{\omega}^{\R_1}$, followed by $\Y^{\R_2}_*$ which gives a state relative to $\R_2$, all whilst respecting the $\E_1$ and $\E_2$ equivalences.

In the setting that  $\R_1$ is localizable we obtain the following; for brevity whenever we write `localizing sequence' without further qualification, this is centred at $e\in G$.

\begin{theorem}
Let $\R_1$ be a localizable principal frame. The map 
    \begin{equation}\label{eq:fcm}
        \Phi^{\rm loc}_{1 \to 2}: [\Omega^{\R_1}]_{\E_2} \mapsto \lim_{n \to \infty}[\Y^{\R_2}_* \circ \mathcal{L}^{\R_1}_{\omega_n}(\Omega^{\R_1})]_{\E_1} = \lim_{n \to \infty}[\Y^{\R_2}_* (\omega_n \otimes \Omega^{\R_1})]_{\E_1},
    \end{equation}
    where $(\omega_n)$ is any localizing sequence for $\E_1$, is a well-defined state space map and makes the following diagram commute

\begin{center}
\begin{tikzcd}
            & \mathcal{S}(\hio \otimes \hit \otimes \his)_G \arrow[ld, "\pi_{\E_2} \circ \Y^{\R_1}_*"'] \arrow[rd, "\pi_{\E_1} \circ \Y^{\R_2}_*"] &   \\
\mathcal{S}(\hit \otimes \his)^{R_1}_{\E_2}  \arrow[rr,"\Phi_{1 \to 2}^{\rm loc}"] &                                                                             &  \mathcal{S}(\hio \otimes \his)^{R_2}_{\E_1}.
\end{tikzcd}
\end{center}
If $\R_2$ is localizable the map $\Phi_{2 \to 1}^{\rm loc}$ (defined in the obvious way) provides an inverse, i.e.,
\[
    \Phi_{2 \to 1}^{\rm loc} \circ \Phi_{1 \to 2}^{\rm loc} = \text{Id}_{\mathcal{S}(\hit \otimes \his)^{R_1}_{\E_2}}.
\]
\end{theorem}

\begin{proof}
    Take $\Omega_1, \Omega_2 \in \S(\hio \otimes \hit \otimes \his)/\hspace{-3pt}\sim_G$ and write  $\Omega^{\R_i} = \Y^{\R_i}_*(\Omega)$ as usual. We need to show that for localizable $\E_1$ and any localizing sequence $(\omega_n)$ whenever $[\Omega_1^{\R_1}]_{\E_2} = [\Omega_2^{\R_1}]_{\E_2} $, i.e. whenever we have
    \[
        \tr[\Omega_1^{\R_1}(\E_2(X) \otimes A_\S)] =\tr[\Omega_2^{\R_1}(\E_2(X) \otimes A_\S)] \text{ for all } X \in \mathcal{B}(G), A_\S \in B(\his)
    \]
    we will also have $\Phi_{1\to2}^{\rm loc}(\Omega_1^{\R_1}) = \Phi_{1\to 2}^{\rm loc}(\Omega_2^{\R_1})$, i.e.
    \[   
        \lim_{n \to \infty}\tr[(\omega_n \otimes \Omega_1^{\R_1})\Y^{\R_2}(\E_1(X) \otimes A_\S)] = \lim_{n \to \infty}\tr[(\omega_n \otimes \Omega_2^{\R_1})\Y^{\R_2}(\E_1(X) \otimes A_\S)] \text{ for all } X \in \mathcal{B}(G), A_\S \in B(\his).
    \]

    We then calculate
    \begin{align*}
    \tr[\Phi_{1\to2}^{\rm loc}(\Omega_1^{R_1})\E_1(X) \otimes A_\S] &=
        \lim_{n \to \infty} \tr[(\omega_n \otimes \Omega_1^{\R_1})\Y^{\R_2}(\E_1(X) \otimes A_\S)]\\ &=
        \lim_{n \to \infty} \tr[(\omega_n \otimes \Omega_1^{\R_1})\int_G d\E_2(g) \otimes \E_1(g.X) \otimes g.A_\S]\\ &=
        \tr[\Omega_1^{\R_1} \int_G d\E_2(g) (\lim_{n \to \infty} \mu_{\omega_n}^{\E_1}(g.X))\otimes g.A_\S]\\ &=
        \tr[\Omega_1^{\R_1} \int_G d\E_2(g) \delta_e(g.X) \otimes g.A_\S] =
        \tr[\Omega_1^{\R_1} \int_G d\E_2(g) \chi_{g.X}(e) \otimes g.A_\S]\\ &= 
        \tr[\Omega_1^{\R_1} \int_G d\E_2(g) \chi_{X}(g^{-1}) \otimes g.A_\S] ,
        %= \tr[\Omega_1^{\R_1} \int_{X^{-1}} d\E_2(g) \otimes g.A_\S],
    \end{align*}
    where we have used that $\lim_{n \to \infty} \mu_{\omega_n}^{\E_1} = \delta_e$ and $\delta_e(g.X) = \chi_{g.X}(e) = \chi_X(g^{-1})$. Now we see that by hypothesis we can replace $\Omega_1^{\R_1}$ by $\Omega_2^{\R_1}$ and get the same number for any $X \in \mathcal{B}(G)$ and $A_\S \in B(\his)$. Running this calculation backwards gives the first claim, as the calculation does not depend on the choice of localizing sequence. To prove the second claim, we need to show that for arbitrary $\Omega \in \S(\hio \otimes \hit \otimes \his)/\hspace{-3pt}\sim_G$, $X \in \mathcal{B}(G)$ and $A_\S \jan{}\in B(\his)$ we have
    \[
    \tr[\Phi_{1\to2}^{\rm loc}(\Omega^{\R_1})\E_1(X) \otimes A_\S] = \tr[\Omega^{\R_2}\E_1(X) \otimes A_\S].
    \]
    We calculate
    \begin{align*}
        \tr[\Phi_{1\to2}^{\rm loc}(\Omega^{R_1})\E_1(X) \otimes A_\S] &=
        \tr[\Omega \int_G d\E_1(h) \otimes h.\int_Gd\E_2(g)\chi_X(g^{-1}) \otimes g.A_\S]\\ &=
        \tr[\Omega \int_G d\E_1(h) \otimes \int_Gd\E_2(hg)\chi_X(g^{-1}) \otimes hg.A_\S.] 
    \end{align*}
    Now performing the change of variables $l := hg$ in the second integral and exchanging the order of integration, we may write
    \begin{align*}
        \tr[\Phi_{1\to2}^{\rm loc}(\Omega^{R_1})\E_1(X) \otimes A_\S] &=
        \tr[\Omega \int_G d\E_1(h) \otimes \int_Gd\E_2(l)\chi_X(l^{-1}h) \otimes l.A_\S] \\&=
        \tr[\Omega \int_Gd\E_2(l) \otimes \int_G d\E_1(h)\chi_X(l^{-1}h) \otimes l.A_\S].
    \end{align*}
Since the $h$ variable appears only in the second tensor factor the second integral can be evaluated giving
    \[
    \int_G d\E_1(h)\chi_X(l^{-1}h) = \int_G d\E_1(h)\chi_{l.X}(h) = \E_1(l.X) = l.\E_1(X),
    \]
        
    
   and finally we arrive at
    \[
        \tr[\Phi_{1\to2}^{\rm loc}(\Omega^{R_1})\E_1(X) \otimes A_\S] =
        \tr[\Omega \int_Gd\E_2(l) \otimes l.(\E_1(X) \otimes A_\S)] =
        \tr[\Omega \Y^{\E_2}(\E_1(X) \otimes A_\S)] =
        \tr[\Omega^{\R_2}(\E_1(X) \otimes A_\S)].
    \]

To show the last claim, writing $(\eta_m)$ for a localizing sequence of $\R_2$, we calculate
\begin{align*}
    \tr[\Phi^{\rm loc}_{2 \to 1} \circ \Phi^{\rm loc}_{1 \to 2}(\Omega^{\R_1})\E_2(X) \otimes A_\S]
    &= \lim_{m \to \infty}\tr[\Y^{\R_1}_* \circ \mathcal{L}^{\R_2}_{\eta_m} \circ \Phi^{\rm loc}_{1 \to 2}(\Omega^{\R_1})\E_2(X)\otimes A_\S]\\
    &= \lim_{m \to \infty}\tr[\mathcal{L}^{\R_2}_{\eta_m} \circ \Phi^{\rm loc}_{1 \to 2}(\Omega^{\R_1}) \int_G d\E_1(g) \otimes g.(\E_2(X) \otimes A_\S)]\\
    &= \lim_{m \to \infty}\tr[\Phi^{\rm loc}_{1 \to 2}(\Omega^{\R_1}) \int_G d\E_1(g) \mu^{\E_2}_{\eta_m}(g.X) \otimes g.A_\S]\\
    &= \lim_{m \to \infty}\lim_{n \to \infty}\tr[\Y^{\R_2}_* \circ \mathcal{L}^{\R_1}_{\omega_n}(\Omega^{\R_1}) \int_G d\E_1(g) \mu^{\E_2}_{\eta_m}(g.X) \otimes g.A_\S]\\
    &= \lim_{m \to \infty}\lim_{n \to \infty}\tr[\mathcal{L}^{\R_1}_{\omega_n}(\Omega^{\R_1}) \int_G d\E_2(h) \otimes h.(\int_G d\E_1(g) \mu^{\E_2}_{\eta_m}(g.X) \otimes g.A_\S)]\\
    &= \lim_{m \to \infty}\lim_{n \to \infty}\tr[\mathcal{L}^{\R_1}_{\omega_n}(\Omega^{\R_1}) \int_G d\E_2(h) \otimes \int_G d\E_1(hg) \mu^{\E_2}_{\eta_m}(g.X) \otimes hg.A_\S]\\
    &= \lim_{m \to \infty}\lim_{n \to \infty}\tr[\Omega^{\R_1} \int_G d\E_2(h) \otimes \int_G d\mu^{\E_1}_{\omega_n}(hg) \mu^{\E_2}_{\eta_m}(g.X) hg.A_\S]\\
    &= \lim_{m \to \infty}\tr[\Omega^{\R_1} \int_G d\E_2(h) \mu^{\E_2}_{\eta_m}(h^{-1}.X) \otimes  A_\S]\\
    &= \tr[\Omega^{\R_1}\int_G d\E_2(h)\chi_X(h) \otimes A_\S] = \tr[\Omega^{\R_1}\E_2(X) \otimes A_\S],
\end{align*}
where we have used $\lim_{n \to \infty} \mu^{\E_1}_{\omega_n}(gh) = \delta_e(gh) = \delta_{g^{-1}}(h)$ and $\lim_{m \to \infty} \mu^{\E_2}_{\eta_m}(h^{-1}.X) = \chi_X(h)$. From commutativity it follows that the map $\Phi_{1 \to 2}^{\rm loc}: \S(\hit \otimes \his)^{\R_1}_{\E_2} \to \S(\hio \otimes \his)^{\R_2}_{\E_1}$ is well-defined in the sense that taking the limit $n \to \infty$ does not take the outcome out of the codomain. Since $\Y^{\R_2}_*$ and $\mathcal{L}^{\R_1}_{\omega_n}$ are all linear, we have a state space map.
\end{proof}

In the setup with three frames, i.e., with $\hi_\T \cong \hio \otimes \hit \otimes \hith \otimes \his$, the maps compose, i.e., 

\begin{theorem}
Assume $\R_1$ and $\R_2$ 
 to be localizable principal frames and let $\R_3$ be an arbitrary principal frame. Then 
 \begin{equation}\label{eq:come}
     \pi_{\E_2}\circ \Phi ^{\rm loc}_{1 \to 3} = \Phi ^{\rm loc}_{2 \to 3} \circ \Phi ^{\rm loc}_{1 \to 2}.
     \end{equation}
\end{theorem}
Note that $\pi_{\E_2}$ is required in order to account for the choice of the second frame orientation observable, which is needed for the definition of the right hand side of Eq. \eqref{eq:come}.
\begin{proof}
    Writing $(\omega_n)$ for a localizing sequence of $\R_1$ and $(\eta_m)$ for that of $\R_2$ as before, we calculate
\begin{align*}
    &\tr[\Phi^{\rm loc}_{2 \to 3} \circ \Phi^{\rm loc}_{1 \to 2}(\Omega^{\R_1})
    \E_1(X) \otimes \E_2(Y) \otimes A_\S]=\\
    &\hspace{2cm}= \lim_{m \to \infty} \tr[\Y^{\R_3}_* \circ \mathcal{L}_{\eta_m}^{\R_2} \circ \Phi^{\rm loc}_{1 \to 2}(\Omega^{\R_1})\E_1(X) \otimes \E_2(Y) \otimes A_\S]\\
    &\hspace{2cm}= \lim_{m \to \infty} \tr[\mathcal{L}_{\eta_m}^{\R_2} \circ \Phi^{\rm loc}_{1 \to 2}(\Omega^{\R_1})\int_G d\E_3(g) \otimes g.(\E_1(X) \otimes \E_2(Y) \otimes A_\S)]\\
    &\hspace{2cm}= \lim_{m \to \infty} \tr[\Phi^{\rm loc}_{1 \to 2}(\Omega^{\R_1})\int_G d\E_3(g)\mu_{\eta_m}^{\E_2}(g.Y) \otimes g.(\E_1(X) \otimes A_\S)]\\
    &\hspace{2cm}= \lim_{m \to \infty} \lim_{n \to \infty} \tr[\Y^{\R_2}_* \circ \mathcal{L}_{\omega_n}^{\R_1}(\Omega^{\R_1})\int_G d\E_3(g)\mu_{\eta_m}^{\E_2}(g.Y) \otimes g.(\E_1(X) \otimes A_\S)]\\
    &\hspace{2cm}= \lim_{m \to \infty} \lim_{n \to \infty} \tr[\mathcal{L}_{\omega_n}^{\R_1}(\Omega^{\R_1})\int_G d\E_2(h) \otimes h.(\int_G d\E_3(g)\mu_{\eta_m}^{\E_2}(g.Y) \otimes g.(\E_1(X) \otimes A_\S))]\\
    &\hspace{2cm}= \lim_{m \to \infty} \lim_{n \to \infty} \tr[\mathcal{L}_{\omega_n}^{\R_1}(\Omega^{\R_1})\int_G d\E_2(h) \otimes \int_G d\E_3(hg)\mu_{\eta_m}^{\E_2}(g.Y) \otimes hg.(\E_1(X) \otimes A_\S)].
\end{align*}
If we now change the integration variable in the second integral for $g' :=hg$ and change the order of integration we can write the operator above as
\[
    \int_G d\E_2(h) \otimes \int_G d\E_3(hg)\mu_{\eta_m}^{\E_2}(g.Y) \otimes hg.(\E_1(X) \otimes A_\S) = \int_G d\E_3(g') \otimes \int_G d\E_2(h)\mu_{\eta_m}^{\E_2}(h^{-1}g'.Y) \otimes g'.(\E_1(X) \otimes A_\S).
\]
Exchanging the order of limits and taking $m \to \infty$ the integral in the second tensor factor can then be evaluated giving
\[
    \lim_{m \to \infty}\int_G d\E_2(h)\mu_{\eta_m}^{\E_2}(h^{-1}g'.Y) = \int_G d\E_2(h)\chi_{g'.Y}(h) = \E_2(g'.Y) = g'.\E_2(Y).
\]
We then get
\begin{align*}
    \tr[\Phi^{\rm loc}_{2 \to 3} \circ \Phi^{\rm loc}_{1 \to 2}(\Omega^{\R_1})
    \E_1(X) \otimes \E_2(Y) \otimes A_\S]
    &= \lim_{m \to \infty} \tr[\mathcal{L}_{\omega_n}^{\R_1}(\Omega^{\R_1})\int_G d\E_3(g') \otimes g'.(\E_1(X) \otimes \E_2(Y) \otimes A_\S)]\\
    &= \lim_{m \to \infty} \tr[\Y^{\R_3}_* \circ \mathcal{L}_{\omega_n}^{\R_1}(\Omega^{\R_1})\E_1(X) \otimes \E_2(Y) \otimes A_\S]\\
    &= \tr[\Phi_{1\to 3}^{\rm loc}(\Omega^{\R_1})\E_1(X) \otimes \E_2(Y) \otimes A_\S].
\end{align*}
Since $X,Y \in \mathcal{B}(G)$ and $A_\S \in B(\his)$ are arbitrary this completes the proof.
\end{proof}

\subsection{Comparison to other work}
It is worth making a brief comparison to the work \cite{de2020quantum} and \cite{de2021perspective}; we start with the former. In order to make this as comprehensible as possible, we fix $G$ to be finite, which rigorously speaking is required in \cite{de2020quantum}, and we change convention for the $G$-actions/representations on the frames to be given by 
$U(g)\ket{h}:=\ket{hg^{-1}}$ (noting that on $G$ this is a left action). The frame observables are each defined by the PVM $P(g)=\dyad{g^{-1}}$, which is easily seen to be covariant.  First note that each frame observable $P_i$ generates a commutative (`classical') subalgebra through

$$\{P(g)\}'' \cong diag(M_n(\mathbb{C})) \cong C(G) \subset B(L^2(G)),$$

where $C(G)$ is the set of functions $G\to \mathbb{C}$ equipped with pointwise multiplication (noting the distinction of $C(G)$ with $L^2(G)$ which we view only as a linear space). The pure states of $C(G)$ are
 in bijection with elements of $G$ and therefore also the frame projections $P(g)$. We set $\his \cong \hi_1 \cong \hi_2$. At this level the frame change map on the classical pure states (understood as projections on the $G$-basis), using that $\Y^{\R_2}(\dyad{e}_{\R_2} \otimes \Omega^{\R_1}) =  \Omega^{\R_1}$, yields (noting that the relevant classes are trivial in  this case)
\begin{equation}
    \left( \dyad{h}_{\R_2} \otimes \dyad{g}_{\Sy}\right)^{\dyad{e}_{\R_1}} \mapsto 
    \dyad{e}_{\R_1}\otimes \dyad{h}_{\R_2}\otimes \dyad{g}_{\Sy} \mapsto
    \left(\dyad{h^{-1}}_{\R_1} \otimes \dyad{gh^{-1}}_\S\right)^{\dyad{e}_{\R_2}},
\end{equation}
which is identical to that given in \cite{de2020quantum}. This map is unitarily extended at the Hilbert space level (all phases are set to $1$) in \cite{de2020quantum} by the ``principle of coherent change of reference system" (which was assumed also implicitly in \cite{giacomini2019quantum}). To see how the construction here and that of \cite{de2020quantum} differs on the quantum level, consider $\omega_2 \in L^2 (G)_2$ and $\rho \in \S(\his)$. Writing $U_{1 \to 2}$ for the frame change unitary corresponding to that given in \cite{de2020quantum}, we find
\[
U_{1 \to 2}(\omega \otimes \rho)U^*_{1 \to 2} = \sum_{g,h} \bra{g} \omega \ket{h} \dyad{g^{-1}}{h^{-1}} \otimes U_\S (g) \rho U^*_\S (h).
\]
On the other hand, the map in Eq. \eqref{eq:fcm}, adapted to this finite $G$ setting, gives
\[
    \Phi_{1 \to 2}^{\rm loc} (\omega \otimes \rho) = \sum_{g} \bra{g} \omega \ket{g} \dyad{g^{-1}} \otimes U_\S (g) \rho U^*_\S (g).
\]

Perhaps surprisingly it turns out that the two states resulting from these procedures are $\E_1$-equivalent.

\begin{proposition}\label{prop:compACT}
    Consider a finite group $G$ and a pair of ideal frames $\R_1$, $\R_2$. Then:
    \[
        \Phi_{1 \to 2}^{\rm loc} = \pi_{\E_1} \circ U_{1 \to 2}(\_)U^*_{1 \to 2}: \S(\hio \otimes \his)_{\E_2}^{\R_1} \to  \S(\hit \otimes \his)^{\R_2}_{\E_1}.
    \]
\end{proposition}

\begin{proof}
    We calculate
\begin{align*}
    &\tr[U_{1 \to 2}\Omega^{\R_1}U^*_{1 \to 2}\dyad{l^{-1}}\otimes F_\S] \\
    &=\tr[\sum_{g,h}\dyad{g^{-1}}{g} \otimes U_\S(g)\Omega^{\R_1}\dyad{h}{h^{-1}} \otimes U^*_\S(h)\dyad{l^{-1}} \otimes F_\S] \\
    &=\tr[\sum_g\dyad{g^{-1}}{g} \otimes U_\S(g)\Omega^{\R_1}\dyad{l}{l^{-1}} \otimes U^*_\S(g)F_\S] \\
    &=\tr[\sum_g\Omega^{\R_1}\dyad{l}{l^{-1}} \otimes U^*_\S(g)F_\S\dyad{g^{-1}}{g} \otimes U_\S(g)] \\
    &=\tr[\Omega^{\R_1}\dyad{l} \otimes U^*_\S(l)F_\S U_\S(l)] =\tr[\Omega^{\R_1}\dyad{l} \otimes l^{-1}.F_\S] \\
\end{align*}
By contrast,
\begin{align*}
    &\tr[\Phi^{\rm loc}_{1 \to 2}(\Omega^{\R_1})\dyad{l^{-1}}\otimes F_\S] \\
    &=\tr[\dyad{e} \otimes \Omega^{\R_1} \sum_g \dyad{g^{-1}} \otimes g. \left(\dyad{l^{-1}} \otimes F_\S\right)]\\
     &=\tr[\dyad{e} \otimes \Omega^{\R_1} \sum_g \dyad{g^{-1}} \otimes \dyad{l^{-1}g^{-1}} \otimes g.F_\S]\\
     &=\tr[\Omega^{\R_1} \sum_g \dyad{g^{-1}} \delta(l,g^{-1}) \otimes g.F_\S]=\tr[\Omega^{\R_1} \dyad{l} \otimes l^{-1}.F_\S].
\end{align*}
\end{proof}
In particular, this shows that if frame $\R_2$ is prepared in a superposed state, e.g. the input state of the frame change~is
\[
    \ket{\psi}=\left((\alpha \ket{h_1}_{\R_2}+\beta \ket{h_2}_{\R_2}) \otimes \ket{g}_\S\right)^{\ket{e}_{\R_1}},
\]
at an operational level there is no difference between the transformed state as given in \cite{de2020quantum}
\begin{equation}\label{eq:fecc1}
     \left( \alpha \ket{h_1^{-1}}_{\R_1} \otimes \ket{gh_1^{-1}}_\S  +\beta \ket{h_2^{-1}}_{\R_1} \otimes \ket{gh_2^{-1}}_\S \right)^{\ket{e}_{\R_2}},
\end{equation}
and that arising from Eq. \eqref{eq:fcm} on a representative of the $\E_2$-equivalence class of $\dyad{\psi}$, which reads
\begin{equation}\label{eq:fecc}
    \left(|\alpha|^2 \dyad{h_1^{-1}}_{\R_1} \otimes \dyad{gh_1^{-1}}_\S 
 +|\beta|^2 \dyad{h_2^{-1}}_{\R_1} \otimes \dyad{gh_2^{-1}}_\S + \right)^{\dyad{e}_{\R_2}},
\end{equation}
as they occupy the same $\E_1$-equivalence class in $\S(\hi_1 \otimes \his)^{\R_2}$. Eq.  \eqref{eq:fecc} is the L\"{u}ders mixture corresponding to Eq. \eqref{eq:fecc1}, and is \emph{not entangled}. It may be tempting from Eq. \eqref{eq:fecc1} (and \cite{giacomini2019quantum}) to draw strong physical conclusions that "superposition and entanglement are frame-dependent". Whilst we do not necessarily disagree with the broad understanding of quantum properties being depending on frame choices---indeed we have seen such behaviour here---the precise notion that a superposition state for $\R_2$ (tensored with any state of $\Sy$) is transformed into an entangled state of $\Sy$ and $\R_1$, we believe deserves further scrutiny; as we can see is not as innocuous as one might think from an operational perspective.





We finish with a mention of the perspective-neutral framework (e.g. \cite{Vanrietvelde:2018pgb,ahmad2022quantum,krumm2021quantum,de2021perspective}) and how it relates to the construction presented here. This plays out on the \emph{physical Hilbert space} $\hi_{\rm phys}$, which is defined directly as the space of invariant vectors in some ambient `kinematical' Hilbert space if $G$ is compact, or via a `rigging' construction \cite{giulini1999generality,giulini1999uniqueness} if $G$ is non-compact; in either case the procedure is defined by a group averaging map, whose image must be interpreted distributionally in case $G$ is not compact (see also \cite{de2021perspective}). The perspective-neutral approach therefore leaves the Hilbert space setting, and the scope of the rigging techniques employed is not fully understood. The frame change map can be constructed informally; at the level of rigour of that work the map is unitary even for non-ideal frames and agrees with, and therefore in a sense subsumes \cite{giacomini2019quantum,de2020quantum} for ideal frames. 

The ``relational Schr{\"o}dinger picture'' frame change map of the perspective-neutral framework can be written (setting $g_i=g_j=e$) in the following form (Thm. 4 on pg. 40 of \cite{de2021perspective})
\[
    V_{1\to 2} = \int_G \dyad{\phi(g)}{\psi(g)} \otimes U_\S(g)d\mu(g),
\]
where $\{\ket{\phi(g)}\}_{g \in G} \subset \hi_1$ and $\{\ket{\psi(g)}\}_{g \in G} \subset \hi_2$ are systems of coherent states of the first and second frame, respectively, understood in the sense of \eqref{ex:csspovm}, and $\dyad{\phi(g)}{\psi(g)}$ as a map from $\hi_2$ to $\hi_1$. Within the Hilbert space framework of quantum mechanics (i.e., without considering rigged Hilbert spaces and distributions), ideal coherent state frames may only be defined on discrete groups. In this setting, for $\R_1$ ideal, we again find that there is no operational distinction between the frame change map we have provided and that of the perspective-neutral approach. Further work is needed for a full comparison in the non-ideal setting, which will require effort also on 
understanding the full scope of the perspective-neutral approach as a fully rigorous mathematical theory.




\section{Conclusions and Outlook}
In this work we have provided a rigorous, general, and operationally motivated definition of a quantum reference frame as a system of covariance based on a $G$-space, and used this to construct relative observables and relative states. We constructed a frame change map transforming one relative state space to another, which is invertible if both frames are localizable, and showed that this map is indistinsguishable from that of \cite{de2020quantum} and \cite{de2021perspective} if evaluated in an operationally motivated way and on the domain on which both frame changes can be applied. In particular, this suggests that there is no `fact of the matter' about whether the frame change is coherent, or entangling, or not, and therefore strong physical claims made on such a basis should be treated with caution. There remains, as ever, work to be done; the setting of non-principal homogeneous spaces has been solved for finite $G$(-spaces) in \cite{glowacki2023quantum}, though the frame change maps have not been constructed in that setting, and the locally compact case is forthcoming. Addressing the topic of `relativity of subsystems' from our perspective is a matter of some urgency and a precise comparison with 
 \cite{ahmad2022quantum,castro2021relative} should be carried out, particularly since the latter is not undertaken in the perspective-neutral framework. The convex flavour of the framework we have provided is well suited to general probabilistic theories, and the obvious proximity of many of these ideas to the $C^*$ and $W^*$ suggests applications in e.g. algebraic quantum field theories, in which no preferred representation is given.


\section*{Acknowledgements}
Thanks are due to Chris Fewster, James Waldron, Anne-Catherine de la Hamette, Stefan Ludescher, Philipp H\"{o}hn, Markus M\"{u}ller, Tom Galley, Sebastiano Nicolussi Golo, Isha Kotecha, Josh Kirklin, Fabio Mele, Jaros{\l}aw Korbicz, Marek Ku{\'s} and Hamed Mohammady for helpful interactions. LL would like to thank the Theoretical Visiting Sciences Programme at the Okinawa Institute of Science and Technology (OIST) for enabling his visit, and for the generous hospitality and excellent working conditions during his time there, which significantly aided the development and completion of this work. JG acknowledges the funding received via NCN through the OPUS grant nr.~$2017/27/$B/ST$2/02959$.




\bibliographystyle{apsrev4-2}
\bibliography{bib.bib}


\appendix

\section{Glossary of Spaces}
We collect below some of the spaces which appear and do not have standard meanings, for ease of reference. All closures are ultraweak.
\begin{itemize}
    \item $B(\hi)^G$: Subspace (von Neumann subalgebra) of $B(\hi)$ which is invariant under the given unitary representation of $G$.
    \item $\mathcal{T}(\hi)^G$ ($\mathcal{S}(\hi)^G$): Subspace (subset) of trace class operators (states) which are invariant under the given unitary representation of $G$.
    \item $\mathcal{T}(\hi)_G$ ($\mathcal{S}(\hi)_G$):  Equivalence classes of trace class operators (states) which cannot be distinguished by any element of $B(H)^G$. We have $\left(\mathcal{T}(\hi)_G\right)^* \cong B(\hi)^G$ in general, and $\mathcal{T}(\hi)^G\cong \mathcal{T}(\hi)_G$ if $G$ is compact.
    \item $B(\his)^{\R} \subset B(\hir \otimes \his)^G$: space of $\R$-relative observables given as the closure of the image of $\Y^{\R}$. Von Neumann algebra if $\R$ is localizable, subalgebra of $B(\hirs)$ if $\R$ is sharp.
    \item $\mathcal{S}(\hi)^{\R}$: Set of $\R$-relative states, given as equivalence classes of states which cannot be distinguished by $\R$-relative observables. Also have $S(\hi)^{\R} \cong \Y_*^{\R}(\mathcal{S}(\hir \otimes \his)) = \Y_*^{\R}(\mathcal{S}(\hir \otimes \his)/{\sim G})$ and $[\mathcal{T}(\hi)^{\R}]^* \cong B(\his)^{\R}$.
    \item $B(\his)^{\R}_{\omega}$: Space of $\omega$-conditioned relative observables, given as the closure of the image of $\Y^{\R}_{\omega}= \Gamma_{\omega} \circ \Y^{\R}$.
    \item $\mathcal{T}(\his)^{\R}_{\omega}$ ($\mathcal{S}(\his)^{\R}_{\omega}$): Space (set) of $\omega$-conditioned relative states, given as equivalence classes of trace class operators (states) which cannot be distinguished by $\omega$-conditioned relative observables. We have $\mathcal{S}(\his)^{\R}_{\omega} \cong (\Y^{\R}_{\omega})_*(\mathcal{S}(\his))$ and $
    (\mathcal{T}(\his)^{\R}_{\omega})^* \cong B(\his)^\R_\omega.$
    \item $B(\his)_{\R}^{\omega}$: Space of relativized $\omega$-conditioned  observables, given as the closure of the image of $\Gamma^{\R}_{\omega}= \Y^{\R} \circ \Gamma_{\omega}$.
    \item  $\mathcal{T}(\his)_{\R}^{\omega}$ ($\mathcal{S}(\his)_{\R}^{\omega}$): Space (set) of $\omega$-disentangled global states, given as equivalence classes of trace class operators (states) which cannot be distinguished by relativized $\omega$-conditioned observables. We have $[\mathcal{T}(\his)_{\R}^{\omega}]^* \cong B(\his)_{\R}^{\omega}$.
    \item $B(\hit \otimes \his)^{\R_1}_{\E_2}$: The space of ($\E_2$-)framed ($\R_1$-)relative observables, given as ${\rm span}(\mathcal{O}^{\R_1}_{\E_2})^{\rm cl}$ with $\mathcal{O}^{\R_1}_{\E_2}$ the set of all observables of the form $\Y^{\R_1}(\E_2(X)\otimes A_{\S})$.
    \item $\T(\hit \otimes \his)^{\R_1}_{\E_1}$ ($\S(\hit \otimes \his)^{\R_1}_{\E_1}$): Space (set) of ($\E_2$-)framed ($\R_1$-)relative states, given through the quotient of those trace class/density operators that cannot be distinguished by the framed relative observables. We have $[\T(\hit \otimes \his)^{\R_1}_{\E_1}]^* \cong B(\hit \otimes \his)^{\R_1}_{\E_2}$.
\end{itemize}

\end{document}