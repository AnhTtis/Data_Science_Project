%!TEX root = ../../main.tex
\newpage
\section{Proof of Theorem~\ref{th:impos}}
\label{app:impossibility}

\begin{figure}[H]
	\centering
	\begin{subfigure}[b]{.33\textwidth}

		\begin{adjustbox}{max width=\textwidth}
			\begin{tabular}{c||c}
				\begin{lstlisting}[xleftmargin=5mm,basicstyle=\ttfamily\footnotesize,escapeinside={(*}{*)}, tabsize=1]
begin;
a = read((*$x$*));
write((*$z$*),1);
write((*$y$*),1);
(*$\ldots$*)
commit
				\end{lstlisting} &
				\begin{lstlisting}[xleftmargin=5mm,basicstyle=\ttfamily\footnotesize,escapeinside={(*}{*)}, tabsize=1]
begin;
b = read((*$y$*));
write((*$z$*),2);
write((*$x$*),2);
(*$\ldots$*)
commit
				\end{lstlisting} 
	
			\end{tabular} 
		\end{adjustbox}
%		\resizebox{\textwidth}{!}{
%			\begin{tikzpicture}[->,>=stealth',shorten >=1pt,auto,node distance=3cm,
%				semithick, transform shape]
%				\node[draw, rounded corners=2mm,outer sep=0] (t1) at (-1.5, -0.25) {\begin{tabular}{l} $\init$ \end{tabular}};
%				\node[draw, rounded corners=2mm,outer sep=0, label={[font=\small]130:$T_1$}] (t2) at (-3, -2) {\begin{tabular}{l} 
%						$\wrt{z}{1}$ \\ $a \gets \rd{x}$ \\ $\wrt{y}{1}$ \\ \ldots
%				\end{tabular}};
%				\node[draw, rounded corners=2mm,outer sep=0,  label={[font=\small]50:$T_2$}] (t2) at (0, -2) {\begin{tabular}{l} 
%						$\wrt{z}{2}$ \\ $b \gets \rd{y}$ \\ $\wrt{x}{2}$ \\ \ldots
%				\end{tabular}};		
%				
%				\path (t1.south west) -- (t1.south) coordinate[pos=0.67] (t1x);
%				\path (t2.north west) -- (t2.north) coordinate[pos=0.67] (t2x);
%				\path (t3.north west) -- (t3.west) coordinate[pos=0.67] (t3x);
%				
%				%\path (t3x) edge [above] node[yshift=8,xshift=0] {$\wro_x$} (t2.north east);
%				%\path (t1x) edge [right] node {$\wro_y$} (t2x);
%			\end{tikzpicture}  
%		}

		\caption{Program \nver{(2 sessions)}.}
		\label{fig:ser-non-optimal:prog}
	\end{subfigure}
	\hspace{.2cm}
	\centering
	\begin{subfigure}[b]{.25\textwidth}
		\resizebox{\textwidth}{!}{
			\begin{tikzpicture}[->,>=stealth',shorten >=1pt,auto,node distance=3cm,
				semithick, transform shape]
				\node[draw, rounded corners=2mm,outer sep=0] (t1) at (-1.5, -0.25) {\begin{tabular}{l} $\init$ \end{tabular}};
				\node[draw, rounded corners=2mm,outer sep=0] (t2) at (-3, -2) {\begin{tabular}{l} 
					$\rd{x}$ \\$\wrt{z}{1}$ \\  $\wrt{y}{1}$ \\ \ldots
				\end{tabular}};
				\node[draw, rounded corners=2mm,outer sep=0] (t3) at (0, -2) {\begin{tabular}{l} 
					$\rd{y}$ \\ $\wrt{z}{2}$ \\ $\wrt{x}{2}$ \\ \ldots
				\end{tabular}};		
				
				\path (t1.south west) -- (t1.south) coordinate[pos=0.67] (t1sw);
				\path (t1.south east) -- (t1.south) coordinate[pos=0.67] (t1se);
				\path (t2.north east) -- (t2.north) coordinate[pos=0.67] (t2x);
				\path (t3.north west) -- (t3.north) coordinate[pos=0.67] (t3x);
				
				%\path (t3x) edge [above] node[yshift=8,xshift=0] {$\wro_x$} (t2.north east);
				\path (t1sw) edge [left] node {$\wro_x$} (t2x);
				\path (t1se) edge [right] node {$\wro_y$} (t3x);
			\end{tikzpicture}  
			
		}
		\caption{History $\hist$.}
		\label{fig:ser-non-optimal:h}
	\end{subfigure}
	\hspace{.2cm}
	\centering
	\begin{subfigure}[b]{.25\textwidth}
		\resizebox{\textwidth}{!}{
			\begin{tikzpicture}[->,>=stealth',shorten >=1pt,auto,node distance=3cm,
				semithick, transform shape]
				\node[draw, rounded corners=2mm,outer sep=0] (t1) at (-1.5, -0.25) {\begin{tabular}{l} $\init$ \end{tabular}};
				\node[draw, rounded corners=2mm,outer sep=0] (t2) at (-3, -2) {\begin{tabular}{l} 
					$\rd{x}$ \\  \pgfsetfillopacity{0.3}$\wrt{z}{1}$ \\ \pgfsetfillopacity{0.3}$\wrt{y}{1}$  \\ \ldots
				\end{tabular}};
				\node[draw, rounded corners=2mm,outer sep=0] (t3) at (0, -2) {\begin{tabular}{l} 
					\pgfsetfillopacity{0.3}$\rd{y}$ \\ 
					\pgfsetfillopacity{0.3}$\wrt{z}{2}$ \\ \pgfsetfillopacity{0.3}$\wrt{x}{2}$ \\ \ldots
				\end{tabular}};		
				
				\path (t1.south west) -- (t1.south) coordinate[pos=0.67] (t1sw);
				\path (t1.south east) -- (t1.south) coordinate[pos=0.67] (t1se);
				\path (t2.north east) -- (t2.north) coordinate[pos=0.67] (t2x);
				\path (t3.north west) -- (t3.north) coordinate[pos=0.67] (t3x);
				
				%\path (t3x) edge [above] node[yshift=8,xshift=0] {$\wro_x$} (t2.north east);
				\path (t1sw) edge [left] node {$\wro_x$} (t2x);
%				\path (t1se) edge [right] node {$\wro_y$} (t3x);
			\end{tikzpicture}  
		}

			\vspace{-1mm}
			\begin{tikzpicture}[shorten >=1pt,auto,node distance=3cm,
				semithick, transform shape]
				%\node (n1) at (0,0);
				%\node (n2) at (0,-4);
				\path (-1.5,0) edge[dashed] (2,0);
			\end{tikzpicture}
			\vspace{-2mm}

		\resizebox{\textwidth}{!}{
			\begin{tikzpicture}[->,>=stealth',shorten >=1pt,auto,node distance=3cm,
				semithick, transform shape]
				\node[draw, rounded corners=2mm,outer sep=0] (t1) at (-1.5, -0.25) {\begin{tabular}{l} $\init$ \end{tabular}};
				\node[draw, rounded corners=2mm,outer sep=0] (t2) at (-3, -2) {\begin{tabular}{l} 
					\pgfsetfillopacity{0.3}$\rd{x}$ \\ 
					\pgfsetfillopacity{0.3}$\wrt{z}{1}$ \\ \pgfsetfillopacity{0.3}$\wrt{y}{1}$  \\ \pgfsetfillopacity{0.3}\ldots
				\end{tabular}};
				\node[draw, rounded corners=2mm,outer sep=0] (t3) at (0, -2) {\begin{tabular}{l} 
					$\rd{y}$ \\  \pgfsetfillopacity{0.3}$\wrt{z}{2}$ \\ \pgfsetfillopacity{0.3}$\wrt{x}{2}$ \\ \pgfsetfillopacity{0.3}\ldots
				\end{tabular}};		
				
				\path (t1.south west) -- (t1.south) coordinate[pos=0.67] (t1sw);
				\path (t1.south east) -- (t1.south) coordinate[pos=0.67] (t1se);
				\path (t2.north east) -- (t2.north) coordinate[pos=0.67] (t2x);
				\path (t3.north west) -- (t3.north) coordinate[pos=0.67] (t3x);
				
				%\path (t3x) edge [above] node[yshift=8,xshift=0] {$\wro_x$} (t2.north east);
				\path (t1se) edge [right] node {$\wro_y$} (t3x);
%				\path (t1se) edge [right] node {$\wro_y$} (t3x);
			\end{tikzpicture}  
		}
		\caption{Two histories. The top history is called $\hist_1$.}
		\label{fig:ser-non-optimal:1}
	\end{subfigure}
	\centering
	%% line
	\begin{subfigure}[b]{.25\textwidth}
	\resizebox{\textwidth}{!}{
		\begin{tikzpicture}[->,>=stealth',shorten >=1pt,auto,node distance=3cm,
			semithick, transform shape]
			\node[draw, rounded corners=2mm,outer sep=0] (t1) at (-1.5, -0.25) {\begin{tabular}{l} $\init$ \end{tabular}};
			\node[draw, rounded corners=2mm,outer sep=0] (t2) at (-3, -2) {\begin{tabular}{l} 
				$\rd{x}$ \\ $\wrt{z}{1}$ \\ \pgfsetfillopacity{0.3}$\wrt{y}{1}$ \\ \ldots
			\end{tabular}};
			\node[draw, rounded corners=2mm,outer sep=0] (t3) at (0, -2) {\begin{tabular}{l} 
				\pgfsetfillopacity{0.3}$\rd{y}$ \\ \pgfsetfillopacity{0.3}$\wrt{z}{2}$ \\ \pgfsetfillopacity{0.3}$\wrt{x}{2}$ \\ \ldots
			\end{tabular}};		
			
			\path (t1.south west) -- (t1.south) coordinate[pos=0.67] (t1sw);
			\path (t1.south east) -- (t1.south) coordinate[pos=0.67] (t1se);
			\path (t2.north east) -- (t2.north) coordinate[pos=0.67] (t2x);
			\path (t3.north west) -- (t3.north) coordinate[pos=0.67] (t3x);
			
			%\path (t3x) edge [above] node[yshift=8,xshift=0] {$\wro_x$} (t2.north east);
			\path (t1sw) edge [left] node {$\wro_x$} (t2x);
			%\path (t1se) edge [right] node {$\wro_y$} (t3x);
		\end{tikzpicture}  			
	}
	\caption{History $\hist_{2}$.}
	\label{fig:ser-non-optimal:2}
	\end{subfigure}
	\hspace{.2cm}
	\centering
	\begin{subfigure}[b]{.25\textwidth}
		\resizebox{\textwidth}{!}{
			\begin{tikzpicture}[->,>=stealth',shorten >=1pt,auto,node distance=3cm,
				semithick, transform shape]
				\node[draw, rounded corners=2mm,outer sep=0] (t1) at (-1.5, -0.25) {\begin{tabular}{l} $\init$ \end{tabular}};
				\node[draw, rounded corners=2mm,outer sep=0] (t2) at (-3, -2) {\begin{tabular}{l} 
						$\rd{x}$ \\ $\wrt{z}{1}$ \\ $\wrt{y}{1}$ \\ \ldots
				\end{tabular}};
				\node[draw, rounded corners=2mm,outer sep=0] (t3) at (0, -2) {\begin{tabular}{l} 
					\pgfsetfillopacity{0.3}$\rd{y}$ \\ \pgfsetfillopacity{0.3}$\wrt{z}{2}$ \\  \pgfsetfillopacity{0.3}$\wrt{x}{2}$ \\ \ldots
				\end{tabular}};		
				
				\path (t1.south west) -- (t1.south) coordinate[pos=0.67] (t1sw);
				\path (t1.south east) -- (t1.south) coordinate[pos=0.67] (t1se);
				\path (t2.north east) -- (t2.north) coordinate[pos=0.67] (t2x);
				\path (t3.north west) -- (t3.north) coordinate[pos=0.67] (t3x);
				
				%\path (t3x) edge [above] node[yshift=8,xshift=0] {$\wro_x$} (t2.north east);
				\path (t1sw) edge [left] node {$\wro_x$} (t2x);
				%\path (t1se) edge [right] node {$\wro_y$} (t3x);
			\end{tikzpicture}  
		}
		\caption{History $\hist_{3}$.}
		\label{fig:ser-non-optimal:3}
	\end{subfigure}
	\hspace{.2cm}
	\begin{subfigure}[b]{.25\textwidth}
		\resizebox{\textwidth}{!}{
			\begin{tikzpicture}[->,>=stealth',shorten >=1pt,auto,node distance=3cm,
				semithick, transform shape]
				\node[draw, rounded corners=2mm,outer sep=0] (t1) at (-1.5, -0.25) {\begin{tabular}{l} $\init$ \end{tabular}};
				\node[draw, rounded corners=2mm,outer sep=0] (t2) at (-3, -2) {\begin{tabular}{l} 
					$\rd{x}$ \\ \pgfsetfillopacity{0.3}$\wrt{z}{1}$ \\  \pgfsetfillopacity{0.3}$\wrt{y}{1}$ \\ \ldots
				\end{tabular}};
				\node[draw, rounded corners=2mm,outer sep=0] (t3) at (0, -2) {\begin{tabular}{l} 
					$\rd{y}$ \\ \pgfsetfillopacity{0.3}$\wrt{z}{2}$ \\ \pgfsetfillopacity{0.3}$\wrt{x}{2}$ \\ \ldots
				\end{tabular}};		
				
				\path (t1.south west) -- (t1.south) coordinate[pos=0.67] (t1sw);
				\path (t1.south east) -- (t1.south) coordinate[pos=0.67] (t1se);
				\path (t2.north east) -- (t2.north) coordinate[pos=0.67] (t2x);
				\path (t3.north west) -- (t3.north) coordinate[pos=0.67] (t3x);
				
				%\path (t3x) edge [above] node[yshift=8,xshift=0] {$\wro_x$} (t2.north east);
				\path (t1sw) edge [left] node {$\wro_x$} (t2x);
				\path (t1se) edge [right] node {$\wro_y$} (t3x);
			\end{tikzpicture}  
			
		}
		\caption{History $\hist_{11}$.}
		\label{fig:ser-non-optimal:11}
	\end{subfigure}
	\centering
	%% line
	\begin{subfigure}[b]{.25\textwidth}
	\resizebox{\textwidth}{!}{
		\begin{tikzpicture}[->,>=stealth',shorten >=1pt,auto,node distance=3cm,
			semithick, transform shape]
			\node[draw, rounded corners=2mm,outer sep=0] (t1) at (-1.5, -0.25) {\begin{tabular}{l} $\init$ \end{tabular}};
			\node[draw, rounded corners=2mm,outer sep=0] (t2) at (-3, -2) {\begin{tabular}{l} 
				$\rd{x}$ \\ $\wrt{z}{1}$ \\ \pgfsetfillopacity{0.3}$\wrt{y}{1}$ \\ \ldots
			\end{tabular}};
			\node[draw, rounded corners=2mm,outer sep=0] (t3) at (0, -2) {\begin{tabular}{l} 
				$\rd{y}$ \\ \pgfsetfillopacity{0.3}$\wrt{z}{2}$ \\ \pgfsetfillopacity{0.3}$\wrt{x}{2}$ \\ \ldots
			\end{tabular}};		
			
			\path (t1.south west) -- (t1.south) coordinate[pos=0.67] (t1sw);
			\path (t1.south east) -- (t1.south) coordinate[pos=0.67] (t1se);
			\path (t2.north east) -- (t2.north) coordinate[pos=0.67] (t2x);
			\path (t3.north west) -- (t3.north) coordinate[pos=0.67] (t3x);
			
			%\path (t3x) edge [above] node[yshift=8,xshift=0] {$\wro_x$} (t2.north east);
			\path (t1sw) edge [left] node {$\wro_x$} (t2x);
			\path (t1se) edge [right] node {$\wro_y$} (t3x);
		\end{tikzpicture}  			
	}
	\caption{History $\hist_{21}$.}
	\label{fig:ser-non-optimal:21}
	\end{subfigure}
	\hspace{.2cm}
	\centering
	\begin{subfigure}[b]{.25\textwidth}
		\resizebox{\textwidth}{!}{
			\begin{tikzpicture}[->,>=stealth',shorten >=1pt,auto,node distance=3cm,
				semithick, transform shape]
				\node[draw, rounded corners=2mm,outer sep=0] (t1) at (-1.5, -0.25) {\begin{tabular}{l} $\init$ \end{tabular}};
				\node[draw, rounded corners=2mm,outer sep=0] (t2) at (-3, -2) {\begin{tabular}{l} 
						$\rd{x}$ \\ $\wrt{z}{1}$ \\ $\wrt{y}{1}$ \\ \ldots
				\end{tabular}};
				\node[draw, rounded corners=2mm,outer sep=0] (t3) at (0, -2) {\begin{tabular}{l} 
						$\rd{y}$ \\ \pgfsetfillopacity{0.3}$\wrt{z}{2}$ \\  \pgfsetfillopacity{0.3}$\wrt{x}{2}$ \\ \ldots
				\end{tabular}};		
				
				\path (t1.south west) -- (t1.south) coordinate[pos=0.67] (t1sw);
				\path (t1.south east) -- (t1.south) coordinate[pos=0.67] (t1se);
				\path (t2.north east) -- (t2.north) coordinate[pos=0.67] (t2x);
				\path (t3.north west) -- (t3.north) coordinate[pos=0.67] (t3x);
				
				%\path (t3x) edge [above] node[yshift=8,xshift=0] {$\wro_x$} (t2.north east);
				\path (t1sw) edge [left] node {$\wro_x$} (t2x);
				\path (t1se) edge [right] node {$\wro_y$} (t3x);
			\end{tikzpicture}  
		}
		\caption{History $\hist_{31}$.}
		\label{fig:ser-non-optimal:31}
	\end{subfigure}
	\hspace{.3cm}
	\centering
	\begin{subfigure}[b]{.14\textwidth}
	\resizebox{.75\textwidth}{!}{
		\begin{tikzpicture}[->,>=stealth',shorten >=1pt,auto,node distance=3cm,
			semithick, transform shape]
			\node[draw, rounded corners=2mm,outer sep=0] (t1) at (0, 0) {\begin{tabular}{l} $\init$ \end{tabular}};
			\node[draw, rounded corners=2mm,outer sep=0] (t2) at (0, -1.8) {\begin{tabular}{l} 
					$\rd{x}$ \\ $\wrt{z}{1}$ \\ $\wrt{y}{1}$ \\ \ldots
			\end{tabular}};
			\node[draw, rounded corners=2mm,outer sep=0] (t3) at (0, -4.5) {\begin{tabular}{l} 
					$\rd{y}$ \\ \pgfsetfillopacity{0.3}$\wrt{z}{2}$ \\ \pgfsetfillopacity{0.3}$\wrt{x}{2}$ \\ \ldots
			\end{tabular}};		
			
			\path (t1.south west) -- (t1.south) coordinate[pos=0.67] (t1sw);
			\path (t1.south east) -- (t1.south) coordinate[pos=0.67] (t1se);
			\path (t2.north east) -- (t2.north) coordinate[pos=0.67] (t2x);
			\path (t3.north west) -- (t3.north) coordinate[pos=0.67] (t3x);
			
			%\path (t3x) edge [above] node[yshift=8,xshift=0] {$\wro_x$} (t2.north east);
			\path (t1.south) edge [left] node {$\wro_x$} (t2.north);
			\path (t2.south) edge [left] node {$\wro_y$} (t3.north);
		\end{tikzpicture}  
			
	}
	\caption{History $\hist_{32}$.}
	\label{fig:ser-non-optimal:32}
	\end{subfigure}
	\begin{subfigure}[b]{.14\textwidth}
		\resizebox{.75\textwidth}{!}{
			\begin{tikzpicture}[->,>=stealth',shorten >=1pt,auto,node distance=3cm,
				semithick, transform shape]
				\node[draw, rounded corners=2mm,outer sep=0] (t1) at (0, 0) {\begin{tabular}{l} $\init$ \end{tabular}};
				\node[draw, rounded corners=2mm,outer sep=0] (t3) at (0, -1.8) {\begin{tabular}{l} 
					$\rd{y}$ \\ $\wrt{z}{2}$ \\ $\wrt{x}{2}$ \\ \ldots
				\end{tabular}};	
				\node[draw, rounded corners=2mm,outer sep=0] (t2) at (0, -4.5) {\begin{tabular}{l} 
					$\rd{x}$ \\ $\wrt{z}{1}$ \\ $\wrt{y}{1}$ \\ \ldots
				\end{tabular}};
					
				
				\path (t1.south west) -- (t1.south) coordinate[pos=0.67] (t1sw);
				\path (t1.south east) -- (t1.south) coordinate[pos=0.67] (t1se);
				\path (t2.north east) -- (t2.north) coordinate[pos=0.67] (t2x);
				\path (t3.north west) -- (t3.north) coordinate[pos=0.67] (t3x);
				
				%\path (t3x) edge [above] node[yshift=8,xshift=0] {$\wro_x$} (t2.north east);
				\path (t1.south) edge [left] node {$\wro_y$} (t3.north);
				\path (t3.south) edge [left] node {$\wro_x$} (t2.north);
			\end{tikzpicture}  
				
		}
		\caption{History $\hat{\hist}$.}
		\label{fig:ser-non-optimal:32}
		\end{subfigure}
\vspace{-2mm}
	\caption{A program and some partial histories. Events in grey are not yet added to the history. For $\hist_3$, $\hist_{31}$ and $\hist_{32}$, the number of events that follow $\ewrt{y,1}$ and $\ewrt{x,2}$ is not important (we use black $\ldots$  to signify that).}
	\label{fig:ser-non-optimal}
\end{figure}

\vspace{-0.5cm}

\begin{figure}[H]
	\resizebox{.5\textwidth}{!}{
		\begin{tikzpicture}[->,>=stealth',shorten >=1pt,auto,node distance=3cm,
			semithick, transform shape]
			\node[draw, rounded corners=2mm,outer sep=0] (h0) at (-3, -1.5) {$\emptyset$};
			\node[draw, rounded corners=2mm,outer sep=0] (h1) at (-1.5, -1.5) { $h_1$};
			\node[draw, rounded corners=2mm,outer sep=0] (h2) at (0, -1.5) {$h_2$};
			\node[draw, rounded corners=2mm,outer sep=0] (h3) at (1.5, -1.5) {$h_3$};
			\node[draw, rounded corners=2mm,outer sep=0] (h11) at (-1.5, -3) {$h_{11}$};
			\node[draw, rounded corners=2mm,outer sep=0] (h21) at (0, -3) { $h_{21}$};
			\node[draw, rounded corners=2mm,outer sep=0] (h31) at (1.5, -3) {$h_{31}$};
			\node[draw, rounded corners=2mm,outer sep=0] (h32) at (3, -3) {$h_{32}$};	
			\node[draw, red, rounded corners=2mm,outer sep=0] (h) at (0, -4.5) {$h$};
			\node[draw, rounded corners=2mm,outer sep=0] (h') at (3, -4.5) {$\hat{h}$};	
			\node [red] (x) at (3, -3.65) {$\bigtimes$};	
	

			
			\path (h0.east) edge [below] node[right] {} (h1.west);
			\path (h1.east) edge [below] node[right] {} (h2.west);			
			\path (h2.east) edge [below] node[right] {} (h3.west);		
			\path (h1.south) edge [below] node[right] {} (h11.north);				
			\path (h2.south) edge [below] node[right] {} (h21.north);			
			\path (h3.south) edge [below] node[right] {} (h31.north);				
			\path (h3.south) edge [below] node[right] {} (h32.north);			
			\path (h11.south) edge [below, dashed] node[right] {} (h.north);			
			\path (h21.south) edge [below, dashed] node[right] {} (h.north);			
			\path (h31.south) edge [below, dashed] node[right] {} (h.north);
			\path (h32.south) edge [below, dashed] node[right] {} (h'.north);			
			
			% \path (t1.south west) -- (t1.south) coordinate[pos=0.67] (t1x);
			% \path (t1.south west) -- (t1.south) coordinate[pos=0.67] (t1x);
			% \path (t2.north west) -- (t2.north) coordinate[pos=0.67] (t2x);
			% \path (t3.north west) -- (t3.west) coordinate[pos=0.67] (t3x);
			
			% \path (t3.west) edge [below] node[right] {$\wro_x$} (t2.north east);
			% \path (t1.south) edge [above] node[left] {$\so \cap \wro_y$} (t2.north);
			% \path (t1.east) edge [above] node[above] {$\so$} (t3.west);
			% \path (t3.south) edge node[right] {$\wro_x$} (t4.north);
			% \path (t1.south east) edge node[below] {$\so$} (t4.north west);
		\end{tikzpicture}
		
	}
	\caption{Summary of all possible execution paths from $\textsc{explore}$. Black arrows represent alternative explored options depending on $\genericNext$ while dashed arrows are mandatory visited histories from such state.}
	\label{fig:execution-ser-si-impossibility}
\end{figure}


\impossibility*

\begin{proof}
\label{proof:impossibility}
	We consider the program in Figure~\ref{fig:ser-non-optimal:prog}, and show that any concrete instance of the $\textsc{explore}$ function in Algorithm~\ref{algorithm:algo-class} \emph{can not be both} $I$-complete and strongly optimal. This program contains two transactions, where only the first three instructions in each transaction are important.
	We show that if $\textsc{explore}$ is $I$-complete, then it will necessarily be called recursively on a history $\hist$ like in Figure~\ref{fig:ser-non-optimal:h} which does not satisfy $I$, thereby violating strong optimality. In the history $\hist$, both \textit{Snapshot Isolation} and \textit{Serializability} forbid the two reads reading initial values while the writes following them are also executed (committed). A diagram of the proof can be seen in Figure~\ref{fig:execution-ser-si-impossibility}.

	Assuming that the function $\genericNext$ is not itself blocking (which would violate strong optimality), \nver{the $\textsc{explore}$ will be called recursively on \emph{exactly one} of the two histories in Figure~\ref{fig:ser-non-optimal:1}, depending on which of the two reads is returned first by $\genericNext$. 
	We will continue our discussion with the history $\hist_1$ on the top of Figure~\ref{fig:ser-non-optimal:1}. The other case is similar (symmetric).}
	% \emph{on only one of the two histories} $\hist_1$ in Figure~\ref{fig:ser-non-optimal:1} and $\hist_2$ in Figure~\ref{fig:ser-non-optimal:2}, depending on the order defined by $\genericNext$ between $\ewrt{y,1}$ and $\ewrt{z,2}$ ($\ewrt{z,2}$ is returned by $\genericNext$ before $\ewrt{y,1}$ in $\hist_1$ and vice-versa in $\hist_2$).
	%We are actually assuming that $\genericNext$ prioritizes the transaction on the left in Figure~\ref{fig:ser-non-optimal:prog}, but this is without loss of generality because the program is symmetric. 
	
	\nver{From $\hist_1$, depending on order defined by $\genericNext$ between $\ewrt{z,1}$ and $\erd{y}$, $\textsc{explore}$ can be called recursively either on $\hist_{11}$ in Figure~\ref{fig:ser-non-optimal:11} or on $\hist_{2}$ in Figure~\ref{fig:ser-non-optimal:2}. Analogously, from $\hist_2$ two alternatives arise depending on the order defined by $\genericNext$ between $\erd{y}$ and the rest of events in the left transaction: exploring $\hist_{21}$ in Figure~\ref{fig:ser-non-optimal:21} if $\erd{y}$ is added before $\ewrt{y,1}$ or $\hist_3$ in Figure~\ref{fig:ser-non-optimal:3} otherwise. Thus, from $\hist_3$ two alternatives arise when added $\erd{y}$ depending on where it reads from: $\hist_{31}$ in Figure~\ref{fig:ser-non-optimal:31} if it reads from $\init$ and $\hist_{32}$ in~Figure~\ref{fig:ser-non-optimal:32} if it reads from the left transaction.
	}
	%(if there is exactly one event from the left transaction ($\ewrt{x,1}$) between $\erd{x}$ and $\erd{y}$), or on $\hist_{3}$ in Figure~\ref{fig:ser-non-optimal:3} (otherwise, where $\ewrt{y,1}$ is added before $\erd{y}$).

	%\nver{From $\hist_1$, depending on order defined by $\genericNext$ between $\erd{x}$ and $\erd{y}$, $\textsc{explore}$ can be called recursively either on $\hist_1$ (if $\erd{x}$ comes immediately before $\erd{y}$ in $\genericNext$ 's order), on $\hist_{2}$ in Figure~\ref{fig:ser-non-optimal:2} (if there is exactly one event from the left transaction ($\ewrt{x,1}$) between $\erd{x}$ and $\erd{y}$), or on $\hist_{3}$ in Figure~\ref{fig:ser-non-optimal:3} (otherwise, where $\ewrt{y,1}$ is added before $\erd{y}$).} 

	%\nver{From $\hist_1$ and $\hist_2$, $\textsc{explore}$ explores $\hist_{11}$ in Figure~\ref{fig:ser-non-optimal:11} and $\hist_{21}$ in Figure~\ref{fig:ser-non-optimal:21} respectively; while from $\hist_3$ two alternative histories may be explored: $\hist_{31}$ and $\hist_{32}$ in Figure~\ref{fig:ser-non-optimal:31} and Figure~\ref{fig:ser-non-optimal:32} respectively.} 

	\nver{However, from histories $\hist_{11}$, $\hist_{21}$ or $\hist_{31}$ $\textsc{explore}$ will necessarily be called recursively on a history $\hist$ like in Figure~\ref{fig:ser-non-optimal:h} which does not satisfy $I$, thereby violating strong optimality: $\textsc{explore}$ always explore branches that enlarge the current history. Thus, any $\textsc{explore}$ implementation that is strong optimal should only explore $\hist_{32}$. In such case, by the restrictions on the $\genericSwap$ function (defined in Section~\ref{sec:algs}), any extension of $\hist_{32}$ does not allow to explore the history $\hat{h}$ in Figure~\ref{fig:ser-non-optimal:3} where $\erd{x}$ reads from $\ewrt{x,2}$: any outcome of a re-ordering between two contiguous subsequences $\alpha$ and $\beta$ must be prefix of such extension when the events in $\alpha$ are taken out. In particular, for any extension $\hist'$ of $\hist_{32}$ and pair of contiguous sequences $\alpha, \beta$ such that $\hist' \setminus \alpha$ is a prefix of $\hist'$, if an event from the second transaction belongs to $\beta$, $\erd{y}$ must also be in $\beta$. Therefore, $\ewrt{x, 2}$ must be in $\beta$ as it is $\wro^{-1}(\erd{y})$. Hence, $\erd{x}$ must also be in $\beta$. Analogously, if $\erd{x}$ belongs to $\beta$, $\init$ belongs to it. Altogether, if $\beta$ contains any element, then $\alpha$ must be empty; so no swaps can be produced from $h_{32}$. To conclude, in this case $\textsc{explore}$ violates $I$-completeness.}

	%The histories $\hist_{12}$ and $\hist_{12}'$ differ in the read-from associated to $\erd{y}$, and exploring at least $\hist_{12}'$ is the best scenario towards ensuring $I$-completeness. 
	% the latter histories differ in the read-from associated to $\erd{y}$). 
	%Being called recursively on both $\hist_{12}$ and $\hist_{12}'$ is the best case scenario towards ensuring $I$-completeness. 
	%If $\textsc{explore}$ is called recursively only on $\hist_{12}$, then $I$-completeness is violated because $\hist_{12}$ and any extension does not enable any re-ordering, and the history where $\erd{x}$ reads from $\ewrt{x,2}$ will never be explored. Thus, let $\hist_e$ be an extension of $h_{12}$. By the restrictions on the $\genericSwap$ function (defined in Section~\ref{sec:algs}), any outcome of a re-ordering between two subsequences $\alpha$ and $\beta$ must be prefix of $\hist_e$ when the events in $\alpha$ are taken out. Since the bottom transaction $t_2$ in $h_e$  reads from the top transaction $t_1$, removing any event different from $t_2$'s last event (the read of $y$) from $\hist_e$ will make the history a non-prefix. Then, having $\alpha$ be this last event is not possible since there is no other event after it to create a $\beta$ (as mentioned in Section~\ref{sec:algs}, the events in $\alpha$ should occur before those in $\beta$).
		
	%Indeed, the two transactions in $\hist_{12}$ are related by $\wro$ and events can be re-ordered earlier only together with their $(\so\cup\wro)^*$ predecessors. 
	%From histories $\hist_{11}$ or $\hist_{12}'$, $\textsc{explore}$ will necessarily be called recursively on a history $\hist$ like in Figure~\ref{fig:ser-non-optimal:221} 
	%(the events that follow $\ewrt{y,1}$ and $\ewrt{x,2}$ are not important) 
	%which does not satisfy $I$, thereby violating strong optimality.
	
	%From $\hist_2$, $\textsc{explore}$ can be called recursively on $\hist_{21}$ in Figure~\ref{fig:ser-non-optimal:12} and $\hist_{21}'$ in Figure~\ref{fig:ser-non-optimal:22}. As explained above for $\hist_{12}=\hist_{21}$, being called recursively only on $\hist_{21}$ violates $I$-completeness, while being called recursively on $\hist_{21}' = \hist_{12}'$ leads to an inconsistent history, thereby violating strong optimality.
\end{proof}

    \oldver{We consider the program in Figure~\ref{fig:ser-non-optimal:prog}, and show that any concrete instance of the $\textsc{explore}$ function in Algorithm~\ref{algorithm:algo-class} \emph{can not be both} $I$-complete and strongly optimal. This program contains two transactions, where only the first three instructions in each transaction are important.
    We show that if $\textsc{explore}$ is $I$-complete, then it will necessarily be called recursively on a history $\hist$ like in Figure~\ref{fig:ser-non-optimal:221} which does not satisfy $I$, thereby violating strong optimality. In the history $\hist$, both \textit{Snapshot Isolation} and \textit{Serializability} forbid the two reads reading initial values while the writes following them are also executed (committed).}
    
    \oldver{Assuming that the function $\genericNext$ is not itself blocking (which would violate strong optimality), the $\textsc{explore}$ will be called recursively on \emph{exactly one} of the four histories in Figure~\ref{fig:ser-non-optimal:1} and Figure~\ref{fig:ser-non-optimal:2}, depending on which of the two reads is returned first by $\genericNext$ and the order defined by $\genericNext$ between the writes. 
    We will continue our discussion with the history $\hist_1$ on the top of Figure~\ref{fig:ser-non-optimal:1} and the history $\hist_2$ on the left of Figure~\ref{fig:ser-non-optimal:2}. The other cases are similar (symmetric).} 
    % \emph{on only one of the two histories} $\hist_1$ in Figure~\ref{fig:ser-non-optimal:1} and $\hist_2$ in Figure~\ref{fig:ser-non-optimal:2}, depending on the order defined by $\genericNext$ between $\ewrt{y,1}$ and $\ewrt{z,2}$ ($\ewrt{z,2}$ is returned by $\genericNext$ before $\ewrt{y,1}$ in $\hist_1$ and vice-versa in $\hist_2$).
    %We are actually assuming that $\genericNext$ prioritizes the transaction on the left in Figure~\ref{fig:ser-non-optimal:prog}, but this is without loss of generality because the program is symmetric. 
    
    \oldver{From $\hist_1$, $\textsc{explore}$ can be called recursively either on $\hist_{11}$ in Figure~\ref{fig:ser-non-optimal:11}, or on $\hist_{12}$ and $\hist_{12}'$ in Figure~\ref{fig:ser-non-optimal:12} and Figure~\ref{fig:ser-non-optimal:22}, depending on the order defined by $\genericNext$ between $\erd{y}$ and $\ewrt{y,1}$ ($\erd{y}$ is returned by $\genericNext$ before $\ewrt{y,1}$ in $\hist_{11}$ and vice-versa in $\hist_{12}$ and $\hist_{12}'$).} 
    
    \oldver{The histories $\hist_{12}$ and $\hist_{12}'$ differ in the read-from associated to $\erd{y}$, and exploring at least $\hist_{12}'$ is the best scenario towards ensuring $I$-completeness. 
    % the latter histories differ in the read-from associated to $\erd{y}$). 
    %Being called recursively on both $\hist_{12}$ and $\hist_{12}'$ is the best case scenario towards ensuring $I$-completeness. 
    If $\textsc{explore}$ is called recursively only on $\hist_{12}$, then $I$-completeness is violated because $\hist_{12}$ and any extension does not enable any re-ordering, and the history where $\erd{x}$ reads from $\ewrt{x,2}$ will never be explored. Thus, let $\hist_e$ be an extension of $h_{12}$. By the restrictions on the $\genericSwap$ function (defined in Section~\ref{sec:algs}), any outcome of a re-ordering between two subsequences $\alpha$ and $\beta$ must be prefix of $\hist_e$ when the events in $\alpha$ are taken out. Since the bottom transaction $t_2$ in $h_e$  reads from the top transaction $t_1$, removing any event different from $t_2$'s last event (the read of $y$) from $\hist_e$ will make the history a non-prefix. Then, having $\alpha$ be this last event is not possible since there is no other event after it to create a $\beta$ (as mentioned in Section~\ref{sec:algs}, the events in $\alpha$ should occur before those in $\beta$).}
        
    %Indeed, the two transactions in $\hist_{12}$ are related by $\wro$ and events can be re-ordered earlier only together with their $(\so\cup\wro)^*$ predecessors. 
    \oldver{From histories $\hist_{11}$ or $\hist_{12}'$, $\textsc{explore}$ will necessarily be called recursively on a history $\hist$ like in Figure~\ref{fig:ser-non-optimal:221} 
    %(the events that follow $\ewrt{y,1}$ and $\ewrt{x,2}$ are not important) 
    which does not satisfy $I$, thereby violating strong optimality.}
    
    \oldver{From $\hist_2$, $\textsc{explore}$ can be called recursively on $\hist_{21}$ in Figure~\ref{fig:ser-non-optimal:12} and $\hist_{21}'$ in Figure~\ref{fig:ser-non-optimal:22}. As explained above for $\hist_{12}=\hist_{21}$, being called recursively only on $\hist_{21}$ violates $I$-completeness, while being called recursively on $\hist_{21}' = \hist_{12}'$ leads to an inconsistent history, thereby violating strong optimality.}
