%!TEX root = main.tex
\section{Prefix-Closed and Causally-Extensible Isolation Levels}\label{sec:props}

We define two properties of isolation levels, prefix-closure and causal extensibility, which enable efficient DPOR algorithms (as shown in Section~\ref{sec:CC-algorithm}).

%Besides models presented in figure \ref{fig:consistency_defs}, others isolation levels exists in literature and real life applications \textcolor{red}{cite Constantin's papers + Twitter, shoppingcart...}. However, our algorithm can not be analyzed under an arbitrary model. We characterize in this section the ones that can be employed by our algorithm.
%not all of them can verified with the algorithm

\subsection{Prefix Closure}

For a relation $R\subseteq A\times A$, the restriction of $R$ to $A'\times A'$, denoted by $R\downarrow A'\times A'$, is defined by $\{(a,b): (a,b)\in R, a,b\in A'\}$. Also, a set $A'$ is called $R$-downward closed when it contains $a\in A$ every time it contains some $b\in A$ with $(a,b)\in R$.


\begin{figure}[t]
%		\centering
%	\begin{subfigure}[b]{.25\textwidth}
%		\begin{adjustbox}{max width=\textwidth}
%			\begin{tabular}{c||c||c}
%				\begin{lstlisting}[xleftmargin=5mm,basicstyle=\ttfamily\scriptsize,escapeinside={(*}{*)}, tabsize=1]
%begin;
%a = read((*x*));
%b = read((*y*));
%commit
%				\end{lstlisting} &
%				\begin{lstlisting}[xleftmargin=5mm,basicstyle=\ttfamily\scriptsize,escapeinside={(*}{*)}, tabsize=1]
%begin;
%write((*$x$*),2);
%commit
%				\end{lstlisting} &
%			\begin{lstlisting}[xleftmargin=5mm,basicstyle=\ttfamily\scriptsize,escapeinside={(*}{*)}, tabsize=1]
%begin;
%read((*$x$*));
%commit
%			\end{lstlisting} 
%			\end{tabular} 
%		\end{adjustbox}
%		
%		\caption{Program.}
%		\label{fig:prefix:prog}
%	\end{subfigure}
%	\hspace{.15cm}
	\centering
	\begin{subfigure}[b]{.30\textwidth}
		\resizebox{\textwidth}{!}{
			\begin{tikzpicture}[->,>=stealth',shorten >=1pt,auto,node distance=3cm,
				semithick, transform shape]
				\node[draw, rounded corners=2mm,outer sep=0] (t1) at (-3.25, 0) {\begin{tabular}{l} $\init$ \end{tabular}};
				\node[draw, rounded corners=2mm,outer sep=0] (t2) at (-3.25, -1.6) {\begin{tabular}{l} 
						$\rd{x}$ \\ $\rd{y}$
				\end{tabular}};
				\node[draw, rounded corners=2mm,outer sep=0] (t3) at (0, 0) {\begin{tabular}{l} 
						$\wrt{x}{2}$
				\end{tabular}};	
				\node[draw, rounded corners=2mm,outer sep=0] (t4) at (0, -1.6)
				{\begin{tabular}{l} 
						$\rd{x}$
				\end{tabular}};			
				
				\path (t1.south west) -- (t1.south) coordinate[pos=0.67] (t1x);
				\path (t2.north west) -- (t2.north) coordinate[pos=0.67] (t2x);
				\path (t3.north west) -- (t3.west) coordinate[pos=0.67] (t3x);
				
				\path (t3.west) edge [below] node[right] {$\wro_x$} (t2.north east);
				\path (t1.south) edge [above] node[left] {$\so \cap \wro_y$} (t2.north);
				\path (t1.east) edge [above] node[above] {$\so$} (t3.west);
				\path (t3.south) edge node[right] {$\wro_x$} (t4.north);
				\path (t1.south east) edge node[below] {$\so$} (t4.north west);
			\end{tikzpicture}
			
		}
		\caption{A history.}
		\label{fig:prefix:a}
	\end{subfigure}
	\hspace{.15cm}
	\centering
	\begin{subfigure}[b]{.30\textwidth}
		\resizebox{\textwidth}{!}{
			\begin{tikzpicture}[->,>=stealth',shorten >=1pt,auto,node distance=3cm,
				semithick, transform shape]
				\node[draw, rounded corners=2mm,outer sep=0] (t1) at (-3.25, 0) {\begin{tabular}{l} $\init$ \end{tabular}};
				\node[draw, rounded corners=2mm,outer sep=0] (t2) at (-3.25, -1.6) {\begin{tabular}{l} 
						$\rd{x}$ \\ $\rd{y}$
				\end{tabular}};
				\node[draw, rounded corners=2mm,outer sep=0] (t3) at (0, 0) {\begin{tabular}{l} 
						$\wrt{x}{2}$
				\end{tabular}};			
				
				\path (t1.south west) -- (t1.south) coordinate[pos=0.67] (t1x);
				\path (t2.north west) -- (t2.north) coordinate[pos=0.67] (t2x);
				\path (t3.north west) -- (t3.west) coordinate[pos=0.67] (t3x);
				
				\path (t3.west) edge [below] node[right] {$\wro_x$} (t2.north east);
				\path (t1.south) edge [above] node[left] {$\so \cap \wro_y$} (t2.north);
				\path (t1.east) edge [above] node[above] {$\so$} (t3.west);

			\end{tikzpicture}
		}
		\caption{A prefix.}
		\label{fig:prefix:b}
	\end{subfigure}
	\hspace{.15cm}
	\centering
	\begin{subfigure}[b]{.30\textwidth}
		\resizebox{\textwidth}{!}{
			\begin{tikzpicture}[->,>=stealth',shorten >=1pt,auto,node distance=3cm,
				semithick, transform shape]
				\node[draw, rounded corners=2mm,outer sep=0] (t1) at (-3.25, 0) {\begin{tabular}{l} $\init$ \end{tabular}};
				\node[draw, rounded corners=2mm,outer sep=0, ] (t2) at (-3.25, -1.6) {\begin{tabular}{l} 
						$\rd{x}$ \\ $\rd{y}$
				\end{tabular}};
%				\node[draw, rounded corners=2mm,outer sep=0, white] (t3) at (0, 0) {\begin{tabular}{l} 
%					$\wrt{x}{2}$
%				\end{tabular}};	
				\node[draw, rounded corners=2mm,outer sep=0] (t4) at (0, -1.6)
				{\begin{tabular}{l} 
						$\rd{x}$
				\end{tabular}};			
				
				\path (t1.south west) -- (t1.south) coordinate[pos=0.67] (t1x);
				\path (t2.north west) -- (t2.north) coordinate[pos=0.67] (t2x);
				\path (t3.north west) -- (t3.west) coordinate[pos=0.67] (t3x);
				
				\path (t1.south east) edge node[below] {$\so$} (t4.north west);
%				\path (t3.west) edge [below] node[right] {$\wro_x$} (t2.north east);
				\path (t1.south) edge [above] node[left] {$\so \cap \wro_y$} (t2.north);
%				\path (t3.south) edge node[right] {$\wro_x$} (t4.north);
			\end{tikzpicture}
			
		}
		\caption{Not a prefix.}
		\label{fig:prefix:c}
	\end{subfigure}
	\vspace{-2mm}
	\caption{Explaining the notion of prefix of a history. $\init$ denotes the transaction log writing initial values. Boxes group events from the same transaction.}
	\vspace{-5mm}
\end{figure}
%where $T'$ contains prefixes of a subset of transaction logs $T''\subseteq T$ such that $T''$ is $(\so \cup \wro)^*$-downward closed.  

A \emph{prefix} of a transaction log $\tup{t,E, \po_t}$ is a transaction log $\tup{t,E', \po_t \downarrow E'\times E'}$ such that $E'$ is $\po_t$-downward closed. 
A \emph{prefix} of a history $\hist=\tup{T, \so, \wro}$ is a history $\hist'=\tup{T',\so\downarrow T'\times T',\wro\downarrow T'\times T'}$ such that every transaction log in $T'$ is a prefix of a different transaction log in $T$ but carrying the same id, $\events{\hist'}\subseteq\events{\hist}$, and $\events{\hist'}$ is $(\po\cup \so \cup \wro)^*$-downward closed.
For example, the history pictured in Fig.~\ref{fig:prefix:b} is a prefix of the one in Fig.~\ref{fig:prefix:a} while the history in Fig.~\ref{fig:prefix:c} is not. The transactions on the bottom of Fig.~\ref{fig:prefix:c} have a $\wro$ predecessor in Fig.~\ref{fig:prefix:a} which is not included.

\begin{definition}
An isolation level $I$ is called \emph{prefix-closed} when every prefix of an $I$-consistent history is also $I$-consistent.
\end{definition}

%For be denominated as \callout{STMC-model}, $\mathcal{M}$ has to satisfies the following three axioms for every program $\mathcal{P}$:

%Comparing to the model requirements described in \textcolor{red}{cite viktor's algorithm}, it is causal-extensibility property, a slightly stricter version of \textcolor{red}{Viktor}'s maximal-extensibility, the biggest difference. However, this weak formulation still forbids some axiomatic models such as Serializability \textcolor{red}{cite constantin's paper (SER)} \textcolor{red}{Appendix cite}.
%Indeed, some isolation levels cannot guarantee that reading from the last added $\iwrite$ event will maintain its consistency status.
Every isolation level $I$ discussed above is prefix-closed because if a history $\hist$ is $I$-consistent with a commit order $\co$, then the restriction of $\co$ to the transactions that occur in a prefix $\hist'$ of $\hist$ satisfies the corresponding axiom(s) when interpreted over $\hist'$. %Therefore, all these levels are prefix-clos
%When the isolation level is defined with an equation such as \ref{eq:axioms}, the restriction of $\co$ to the transactions that occur in a prefix $\hist'$ of a history $\hist$ satisfies the corresponding axiom(s) when interpreted over $\hist'$.


\begin{theorem}
\textit{Read Committed}, \textit{Read Atomic}, \textit{Causal Consistency}, \textit{Snapshot Isolation}, and \textit{Serializability} are prefix closed.
\end{theorem}
\begin{comment}
\begin{proof}(Sketch)
Let $\hist$ be a history that satisfies one of these isolation levels, and let $\co$ be a commit order of $\hist$ that satisfies the corresponding axiom(s). The restriction of $\co$ to the transactions that occur in a prefix $\hist'$ of $\hist$ satisfies the corresponding axiom(s) when interpreted over $\hist'$.
%As any $\so \cup \wro$-prefix-closed sub-history $h'$ of $h$ is a sub-graph of it, and there is a commit order $\co$ for $h$, it suffices to restrict $\co$ to $h'$ for obtaining a commit order for $h'$.
\end{proof}
\end{comment}

\subsection{Causal Extensibility}\label{ssec:causal_ext}

\begin{figure}[t]
	\begin{minipage}[b]{0.66\textwidth}
	
		\centering
		\begin{subfigure}[b]{.46\textwidth}
			\resizebox{\textwidth}{!}{
				\begin{tikzpicture}[->,>=stealth',shorten >=1pt,auto,node distance=3cm,
					semithick, transform shape]
					\node[draw, rounded corners=2mm,outer sep=0] (t1) at (-3.25, 0) {\begin{tabular}{l} $\init$ \end{tabular}};
					\node[draw, rounded corners=2mm,outer sep=0] (t2) at (-3.25, -1.6) {\begin{tabular}{l} 
							$\rd{x}$ \\ \textbf{\textcolor{blue}{$\rd{y}$}} %\pgfsetfillopacity{0.3}
					\end{tabular}};
					\node[draw, rounded corners=2mm,outer sep=0] (t3) at (0, 0) {\begin{tabular}{l} 
							$\wrt{x}{2}$%\\
							%\pgfsetfillopacity{0.3}$\wrt{y}{2}$
					\end{tabular}};			
					
					\path (t1.south west) -- (t1.south) coordinate[pos=0.67] (t1x);
					\path (t2.north west) -- (t2.north) coordinate[pos=0.67] (t2x);
					\path (t3.north west) -- (t3.west) coordinate[pos=0.67] (t3x);
					
					\path (t3.west) edge [below] node[right] {$\wro_x$} (t2.north east);
					\path (t1.south) edge [above] node[left] {$\so$} (t2.north);
					\path (t1.east) edge [above] node[above] {$\so$} (t3.west);
					%\path (t1x) edge [right] node {$\wro_y$} (t2x);
				\end{tikzpicture}  
				
			}
			\caption{Extensible history.}
			\label{fig:maxclosed:a}
		\end{subfigure}
		\hspace{.1cm}
		\centering
		\begin{subfigure}[b]{.5\textwidth}
			\resizebox{\textwidth}{!}{
				\begin{tikzpicture}[->,>=stealth',shorten >=1pt,auto,node distance=3cm,
					semithick, transform shape]
					\node[draw, rounded corners=2mm,outer sep=0] (t1) at (-3.25, 0) {\begin{tabular}{l} $\init$ \end{tabular}};
					\node[draw, rounded corners=2mm,outer sep=0] (t2) at (-3.25, -1.6) {\begin{tabular}{l} 
							$\rd{x}$ \\ $\rd{y}$
					\end{tabular}};
					\node[draw, rounded corners=2mm,outer sep=0] (t3) at (0, 0) {\begin{tabular}{l} 
							$\wrt{x}{2}$\\
							\textbf{\textcolor{blue}{$\wrt{y}{2}$}}
					\end{tabular}};			
					
					\path (t1.south west) -- (t1.south) coordinate[pos=0.67] (t1x);
					\path (t2.north west) -- (t2.north) coordinate[pos=0.67] (t2x);
					\path (t3.north west) -- (t3.west) coordinate[pos=0.67] (t3x);
					
					\path (t3.west) edge [below] node[right] {$\wro_x$} (t2.north east);
					\path (t1.south) edge [above] node[left] {$\so \cap \wro_y$} (t2.north);
					\path (t1.east) edge [above] node[above] {$\so$} (t3.west);
				\end{tikzpicture}
				
			}
			\caption{Non-extensible history.}
			\label{fig:maxclosed:b}
		\end{subfigure}
		\centering
		\vspace{-2mm}
		\caption{Explaining causal extensibility. $\init$ denotes the transaction log writing initial values. Boxes group events from the same transaction.
		%We write $\so \cap \wro_x$ when there is a $\so$ and a $\wro_x$ edge sharing both source and target.
		}
		\vspace{-2mm}

		\label{fig:maxclosed}
		%\vspace{-3mm}

	\end{minipage}
	\hfill
	\begin{minipage}[b]{0.32\textwidth}
		\resizebox{\textwidth}{!}{
			\begin{tikzpicture}[->,>=stealth',shorten >=1pt,auto,node distance=3cm,
				semithick, transform shape]
				\node[draw, rounded corners=2mm,outer sep=0] (t1) at (-1.5, -0.25) {\begin{tabular}{l} $\init$ \end{tabular}};
				\node[draw, rounded corners=2mm,outer sep=0] (t2) at (-3, -2) {\begin{tabular}{l} 
						$\wrt{z}{1}$ \\ $\rd{x}$ \\ $\wrt{y}{1}$ 
				\end{tabular}};
				\node[draw, rounded corners=2mm,outer sep=0] (t3) at (0, -2) {\begin{tabular}{l} 
						$\wrt{z}{2}$ \\ $\rd{y}$ \\ 	\textbf{\textcolor{blue}{$\wrt{x}{2}$}}
				\end{tabular}};		
				
				\path (t1.south west) -- (t1.south) coordinate[pos=0.67] (t1sw);
				\path (t1.south east) -- (t1.south) coordinate[pos=0.67] (t1se);
				\path (t2.north east) -- (t2.north) coordinate[pos=0.67] (t2x);
				\path (t3.north west) -- (t3.north) coordinate[pos=0.67] (t3x);
				
				%\path (t3x) edge [above] node[yshift=8,xshift=0] {$\wro_x$} (t2.north east);
				\path (t1sw) edge [left] node {$\wro_x$} (t2x);
				\path (t1se) edge [right] node {$\wro_y$} (t3x);
			\end{tikzpicture}  
		}

	\vspace{-2mm}

	\caption{A counter-example to causal extensibility for $\SI$ and $\SER$. 
	%A bi-threaded program and a $\SER$/ $\SI$-consistent history. Events in gray are not yet added to the history. 
	The $\so$-edges from $\init$ to the other transactions are omitted for legibility.}
	\vspace{-2mm}

	\label{fig:non-causally-extensible}

	\end{minipage}
\end{figure}


We start with an example to explain causal extensibility. Let us consider the histories $h_1$ and $h_2$ in Figures~\ref{fig:maxclosed:a} and \ref{fig:maxclosed:b}, respectively, \emph{without} the events $\erd{y}$ and $\ewrt{y,2}$ written in blue bold font. These histories satisfy Read Atomic. 
%respectively under RA; isolation level under which both are consistent. 
The history $h_1$ can be extended by adding the event $\rd{y}$ and the $\wro$ dependency $\wro(\init,\rd{y})$ while still satisfying Read Atomic.
%$w_1 \ [\wro] \ r_1$, where $w_1 = \wrt{x}{0}$. 
On the other hand, the history $h_2$ \emph{can not} be extended with the event $\wrt{y}{2}$ while still satisfying Read Atomic. Intuitively, if the reading transaction on the bottom reads $x$ from the transaction on the right, then it should read $y$ from the same transaction because this is more ``recent'' than $\init$ w.r.t. session order. The essential difference between these two extensions is that the first concerns a transaction which is maximal in $(\so\cup\wro)^+$ while the second no. The extension of $\hist_2$ concerns the transaction on the right in Figure~\ref{fig:maxclosed:b} which is a $\wro$ predecessor of the reading transaction. Causal extensibility will require that at least the $(\so\cup\wro)^+$ maximal (pending) transactions can always be extended with any event while still preserving consistency. The restriction to $(\so\cup\wro)^+$ maximal transactions is intuitively related to the fact that transactions should not read from non-committed (pending) transactions, e.g., the reading transaction in $\hist_2$ should not read from the still pending transaction that writes $x$ and later $y$.

%this is not the case of $h_2$: the only event that could be added in it is $w_2 = \wrt{y}{2}$. If $w_2$ would be added in $h_2$, any relation extending $\so \cup \wro$ and satisfying RA would be cyclic, so it wouldn't be a commit order. The essential difference between these two histories is the following: in $h_1$, $\tr(r)$ is $\so \cup \wro$-maximal while in $h_2$ $\tr(w)$ is not. As real database executions forbid transactions reading from non-committed ones, it is reasonable to allow those transactions $\so \cup \wro$-maximal to be executed completed without hindering the previous committed transactions. 
Formally, let  $\hist=\tup{T, \so, \wro}$ be a history. A transaction $t$ is called $(\so \cup \wro)^+$-maximal in $h$ if $h$ does not contain any transaction $t'$ such that $(t,t')\in (\so \cup \wro)^+$. We define a \emph{causal extension} of a pending transaction $t$ in $h$ with an event $e$ as \nver{a history $h'$} such that:
\vspace{-1mm}
\begin{itemize}
\item $e$ is added to $t$ as a maximal element of $\po_t$,
\item if $e$ is a read event and $t$ \emph{does not} contain a write to $\mathit{var}(e)$, then $\wro$ is extended with some tuple $(t',e)$ such that $(t', t) \in  (\so \cup \wro)^+$ \nver{in $h$} \nver{(if $e$ is a read event and $t$ \emph{does} contain a write to $\mathit{var}(e)$, then the value returned by $e$ is the value written by the latest write on $\mathit{var}(e)$ before $e$ in $t$; the definition of the return value in this case is unique and does not involve $\wro$ dependencies),}
\item the other elements of $\hist$ remain unchanged \nver{in $h'$}.
\vspace{-1mm}
\end{itemize}

\begin{figure}[t]
%	\centering
%	\begin{subfigure}[b]{.25\textwidth}
%		\begin{adjustbox}{max width=\textwidth}
%			\begin{tabular}{c||c}
%				\begin{lstlisting}[xleftmargin=5mm,basicstyle=\ttfamily\scriptsize,escapeinside={(*}{*)}, tabsize=1]
%begin;
%write((*$y$*),1);
%write((*$x$*),1);
%commit
%begin;
%write((*$x$*),2);
%commit
%				\end{lstlisting} &
%				\begin{lstlisting}[xleftmargin=5mm,basicstyle=\ttfamily\scriptsize,escapeinside={(*}{*)}, tabsize=1]
%begin;
%write((*$x$*),3);
%commit
%begin;
%a = read((*$y$*));
%b = read((*$x$*));
%commit
%				\end{lstlisting}
%			\end{tabular} 
%		\end{adjustbox}
%		
%		\caption{Program.\\$ $}
%		\label{fig:extension:prog}
%	\end{subfigure}
%	\hspace{.15cm}
%	\centering
	\begin{subfigure}[t]{.28\textwidth}
		\resizebox{\textwidth}{!}{
			\begin{tikzpicture}[->,>=stealth',shorten >=1pt,auto,node distance=3cm,
				semithick, transform shape]
				\node[draw, rounded corners=2mm,outer sep=0] (t0) at (-1.5, 0) {\begin{tabular}{l} $\init$ \end{tabular}};
				\node[draw, rounded corners=2mm,outer sep=0, label={[font=\small]100:$t_1$}] (t1) at (-3, -1.6) {\begin{tabular}{l} 
					$\wrt{x}{1}$ \\ $\wrt{y}{1}$
				\end{tabular}};
				\node[draw, rounded corners=2mm,outer sep=0, label={[font=\small]50:$t_2$}] (t2) at (-3, -3.2) {\begin{tabular}{l} 
					$\wrt{x}{2}$
				\end{tabular}};
				\node[draw, rounded corners=2mm,outer sep=0, label={[font=\small]50:$t_3$}] (t3) at (0, -1.6) {\begin{tabular}{l} 
						$\wrt{x}{3}$
				\end{tabular}};	
				\node[draw, rounded corners=2mm,outer sep=0, label={[font=\small]50:$t_4$}] (t4) at (0, -3.2)
				{\begin{tabular}{l} 
						$\rd{y}$ \\ $\cdots$%$\rd{x}$}
				\end{tabular}};			
				
				\path (t1.south west) -- (t1.south) coordinate[pos=0.67] (t1x);
				\path (t2.north west) -- (t2.north) coordinate[pos=0.67] (t2x);
				\path (t3.north west) -- (t3.west) coordinate[pos=0.67] (t3x);
				
				%\path (t3.west) edge [below] node[right] {$\wro_x$} (t2.north east);
				%\path (t1.south) edge [above] node[left] {$\so \cap \wro_y$} (t2.north);
				\path (t3.south) edge node[right] {$\so$} (t4.north);
				\path (t1.south) edge node[left] {$\so$} (t2.north);
				\path (t1.south east) edge node[above] {$\wro_y$} (t4.north west);
				\path (t0.south) edge node[above] {$\so$} (t1.north);
				\path (t0.south) edge node[above] {$\so$} (t3.north);
			\end{tikzpicture}
			
		}
		\vspace{-5.5mm}
		\caption{History $\hist$.}
		\label{fig:extension:a}
	\end{subfigure}
	\hspace{.15cm}
	\centering
	\begin{subfigure}[t]{.28\textwidth}
		\resizebox{\textwidth}{!}{
			\begin{tikzpicture}[->,>=stealth',shorten >=1pt,auto,node distance=3cm,
				semithick, transform shape]
				\node[draw, rounded corners=2mm,outer sep=0] (t0) at (-1.5, 0) {\begin{tabular}{l} $\init$ \end{tabular}};
				\node[draw, rounded corners=2mm,outer sep=0, label={[font=\small]100:$t_1$}] (t1) at (-3, -1.6) {\begin{tabular}{l} 
						$\wrt{x}{1}$ \\ $\wrt{y}{1}$
				\end{tabular}};
				\node[draw, rounded corners=2mm,outer sep=0, label={[font=\small]50:$t_2$}] (t2) at (-3, -3.2) {\begin{tabular}{l} 
						$\wrt{x}{2}$
				\end{tabular}};
				\node[draw, rounded corners=2mm,outer sep=0, label={[font=\small]50:$t_3$}] (t3) at (0, -1.6) {\begin{tabular}{l} 
						$\wrt{x}{3}$
				\end{tabular}};	
				\node[draw, rounded corners=2mm,outer sep=0, label={[font=\small]50:$t_4$}] (t4) at (0, -3.2)
				{\begin{tabular}{l} 
						$\rd{y}$ \\ \textbf{\textcolor{blue}{$\rd{x}$}}
				\end{tabular}};			
				
				\path (t1.south west) -- (t1.south) coordinate[pos=0.67] (t1x);
				\path (t2.north west) -- (t2.north) coordinate[pos=0.67] (t2x);
				\path (t3.north west) -- (t3.west) coordinate[pos=0.67] (t3x);
				
				%\path (t3.west) edge [below] node[right] {$\wro_x$} (t2.north east);
				%\path (t1.south) edge [above] node[left] {$\so \cap \wro_y$} (t2.north);
				\path (t3.south) edge node[right] {$\so$} (t4.north);
				\path (t1.south) edge node[left] {$\so$} (t2.north);
				\path (t1.south east) edge node[above, xshift = 4mm,yshift=-1mm] {$\wro_y$, $\wro_x$} (t4.north west);
				\path (t0.south) edge node[above] {$\so$} (t1.north);
				\path (t0.south) edge node[above] {$\so$} (t3.north);
			\end{tikzpicture}
		}
		\vspace{-5.5mm}
		\caption{$t_4$ reads $x$ and $y$ from $t_1$.}
		\label{fig:extension:b}
	\end{subfigure}
	\hspace{.15cm}
	\centering
	\begin{subfigure}[t]{.28\textwidth}
		\resizebox{1.1\textwidth}{!}{
			\begin{tikzpicture}[->,>=stealth',shorten >=1pt,auto,node distance=3cm,
				semithick, transform shape]
				\node[draw, rounded corners=2mm,outer sep=0] (t0) at (-1.5, 0) {\begin{tabular}{l} $\init$ \end{tabular}};
				\node[draw, rounded corners=2mm,outer sep=0, label={[font=\small]100:$t_1$}] (t1) at (-3, -1.6) {\begin{tabular}{l} 
						$\wrt{x}{1}$ \\ $\wrt{y}{1}$
				\end{tabular}};
				\node[draw, rounded corners=2mm,outer sep=0, label={[font=\small]50:$t_2$}] (t2) at (-3, -3.2) {\begin{tabular}{l} 
						$\wrt{x}{2}$
				\end{tabular}};
				\node[draw, rounded corners=2mm,outer sep=0, label={[font=\small]50:$t_3$}] (t3) at (0, -1.6) {\begin{tabular}{l} 
						$\wrt{x}{3}$
				\end{tabular}};	
				\node[draw, rounded corners=2mm,outer sep=0, label={[font=\small]10:$t_4$}] (t4) at (0, -3.2)
				{\begin{tabular}{l} 
						$\rd{y}$ \\ \textbf{\textcolor{blue}{$\rd{x}$}}
				\end{tabular}};			
				
				\path (t1.south west) -- (t1.south) coordinate[pos=0.67] (t1x);
				\path (t2.north west) -- (t2.north) coordinate[pos=0.67] (t2x);
				\path (t3.north west) -- (t3.west) coordinate[pos=0.67] (t3x);
				
				%\path (t3.west) edge [below] node[right] {$\wro_x$} (t2.north east);
				%\path (t1.south) edge [above] node[left] {$\so \cap \wro_y$} (t2.north);
				\path (t1.south east) edge node[above] {$\wro_y$} (t4.north west);
				\path (t3.south) edge node[right] {$\so \cap \wro_x$} (t4.north);
				\path (t1.south) edge node[left] {$\so$} (t2.north);
				\path (t0.south) edge node[above] {$\so$} (t1.north);
				\path (t0.south) edge node[above] {$\so$} (t3.north);
			\end{tikzpicture}
			
		}
		\vspace{-5.5mm}
		\caption{$t_4$ reads $x$ from $t_3$, $y$ from $t_1$.}
		\label{fig:extension:c}
	\end{subfigure}
	\vspace{-4mm}
	\caption{Two causal extensions of the history $h$ on the left with the $\erd{x}$ event written in blue.}
	\vspace{-3.5mm}
\end{figure}

For example, Figure~\ref{fig:extension:b} and~\ref{fig:extension:c} present two causal extensions with a $\erd{x}$ event of the transaction $t_4$ in the history $\hist$ in Figure~\ref{fig:extension:a}. The new read event reads from transaction $t_1$ or $t_3$ which were already related by $(\so \cup \wro)^+$ to $t_4$.
An extension of $h$ where the new read event reads from $t_2$ is \emph{not} a causal extension because $(t_2, t_4) \not\in (\so \cup \wro)^+$.
%and $y$ from $t_1$ is not included as it is not a causal extension ($(t_2, t_4) \not\in \so \cup \wro$)

\vspace{-1.5mm}
\begin{definition}
\label{def:causally-extensible}
An isolation level $I$ is called \emph{causally-extensible} if for every $I$-consistent history $h$, every $(\so \cup \wro)^+$-maximal pending transaction $t$ in $h$, and every event $e$, there exists a causal extension $\hist'$ of $t$ with $e$ that is $I$-consistent.
%	can be extended with an event from a $\so \cup \wro$-maximal pending transaction $T$ in $h$. Moreover, if that event $r$ is a $\iread$, it can always read from a $\iwrite$ event $w$ s.t. in $h$ $\tr(w) \ (\so \cup \wro)^* \ \tr(r)$.
\vspace{-1.5mm}
\end{definition}

\appendixver{The proof of the following theorem can be found in Appendix~\ref{app:causalExtensible}.}

\vspace{-1.5mm}
\begin{restatable}{theorem}{causalExtensibleModels}
\label{theorem:causalExtensibleModels-CC-RA-RC}
Causal Consistency, Read Atomic, and Read Committed are causally-extensible.
\vspace{-1.5mm}
\end{restatable}

%\vspace{-3mm}

Snapshot Isolation and Serializability are \emph{not} causally extensible. 
Figure~\ref{fig:non-causally-extensible} presents a counter-example to causal extensibility: the causal extension of the history $\hist$ that does \emph{not} contain the $\ewrt{x,2}$ written in blue bold font with this event does not satisfy neither Snapshot Isolation nor Serializability although $\hist$ does. Note that the causal extension with a write event is unique. \nver{(Note that both $h$ and this causal extension satisfy Causal Consistency and therefore, as expected, this counter-example does not apply to isolation levels weaker than Causal Consistency.)}
%we exhibit why $\SI$ and $\SER$ are not causally extensible. For the program in Figure~\ref{fig:non-causally-extensible:prog-ser}, the history presented in Figure~\ref{fig:non-causally-extensible:ser} is $\SER$-consistent (and therefore $\SI$-consistent) but whose only extension (Figure~\ref{fig:non-causally-extensible:ser-cont}) is $\SI$-inconsistent (and therefore $\SER$-inconsistent).




%\input{proofs:causal-extensibility}
