\begin{table}[t]
    \centering
    \scalebox{0.95}{
    \begin{tabular}{|r|c|c|}\hline
        Linguistic features & PREC & REC \\\hline
        POS & \textbf{0.8823} & \textbf{0.8782}\\
        RST & 0.6281 & 0.5170 \\
        CLUE & 0.6813 & 0.6411 \\
        CLIP \cite{Radford2021LearningTV} & 0.7215 & 0.7170
        \\ \hline
        POS+RST & 0.8838 & 0.8787
        \\
        POS+CLUE & 0.8911  & 0.8879 \\
        POS+RST+CLUE & 0.8916 & 0.8884
        \\
CLIP+POS & 0.8925 & 0.8893
        \\
        CLIP+RST & 0.7360 & 0.7312 
        \\
        CLIP+CLUE & 0.7523 & 0.7480
        \\
        CLIP+POS+RST+CLUE &\textbf{ 0.8982} & \textbf{0.8960}\\\hline
        BERT & \textbf{0.9570} & \textbf{0.9560}
        \\\hline % prec dropped from 0.968 -> 0.957 (previously evaluated on a subset)
    \end{tabular}}
    \caption{We evaluate precision/recall of DII/SIS classifiers on a VIST holdout.}
    %taking in combinations of linguistic features 
    %We see that part of speech tags are quite discriminative, however using all features: RST, CLIP*, and CLUE increase the precision and recall of the classifier. 
    %\cite{Radford2021LearningTV}. 
    \label{tab:dii_sis_classifier}
\end{table}