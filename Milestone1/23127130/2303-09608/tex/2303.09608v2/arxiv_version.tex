%File: anonymous-submission-latex-2024.tex
\documentclass[letterpaper]{article} % DO NOT CHANGE THIS

\usepackage[submission]{aaai24}  % DO NOT CHANGE THIS

\usepackage{times}  % DO NOT CHANGE THIS
\usepackage{helvet}  % DO NOT CHANGE THIS
\usepackage{courier}  % DO NOT CHANGE THIS
\usepackage[hyphens]{url}  % DO NOT CHANGE THIS
\usepackage{graphicx} % DO NOT CHANGE THIS
\urlstyle{rm} % DO NOT CHANGE THIS
\def\UrlFont{\rm}  % DO NOT CHANGE THIS
\usepackage{natbib}  % DO NOT CHANGE THIS AND DO NOT ADD ANY OPTIONS TO IT
\usepackage{caption} % DO NOT CHANGE THIS AND DO NOT ADD ANY OPTIONS TO IT
\frenchspacing  % DO NOT CHANGE THIS
\setlength{\pdfpagewidth}{8.5in} % DO NOT CHANGE THIS
\setlength{\pdfpageheight}{11in} % DO NOT CHANGE THIS
%
% These are recommended to typeset algorithms but not required. See the subsubsection on algorithms. Remove them if you don't have algorithms in your paper.
\usepackage{algorithm}
\usepackage{algorithmic}
\usepackage{amssymb}% http://ctan.org/pkg/amssymb
\usepackage{pifont}% http://ctan.org/pkg/pifont
\usepackage{multirow}

%
% These are are recommended to typeset listings but not required. See the subsubsection on listing. Remove this block if you don't have listings in your paper.
\usepackage{newfloat}
\usepackage{listings}
\DeclareCaptionStyle{ruled}{labelfont=normalfont,labelsep=colon,strut=off} % DO NOT CHANGE THIS
\lstset{%
	basicstyle={\footnotesize\ttfamily},% footnotesize acceptable for monospace
	numbers=left,numberstyle=\footnotesize,xleftmargin=2em,% show line numbers, remove this entire line if you don't want the numbers.
	aboveskip=0pt,belowskip=0pt,%
	showstringspaces=false,tabsize=2,breaklines=true}
\floatstyle{ruled}
\newfloat{listing}{tb}{lst}{}
\floatname{listing}{Listing}
%
% Keep the \pdfinfo as shown here. There's no need
% for you to add the /Title and /Author tags.
\pdfinfo{}

% \usepackage{epsfig}
% \usepackage{graphicx}
\usepackage{amsmath}
% \usepackage{amssymb}
% \usepackage{xcolor}
% \usepackage{multirow}
% \usepackage{array}
% \usepackage{makecell}
% \usepackage{tabu, booktabs}
\usepackage{xcolor,colortbl}
% \usepackage{newtxtext,newtxmath}
% \usepackage{fontspec}
% \usepackage{emoji}
\usepackage{enumitem}
% \usepackage{subfigure}
% \usepackage{bbm}

\definecolor{lightgreen}{rgb}{0.60, 0.78, 0.75}
\definecolor{lightorange}{rgb}{1.0, 0.949, 0.80}
\definecolor{verylightorange}{rgb}{1.0, 0.875, 0.502}
\definecolor{verylightgreen}{rgb}{0.796, 0.886, 0.871}
	


\begin{document}

%%%%%%%%% TITLE
\title{VEIL: Vetting Extracted Image Labels from In-the-Wild Captions for Weakly-Supervised Object Detection}

\author{Arushi Rai\\
University of Pittsburgh\\
%Institution1 address\\
{\tt\small arr159@pitt.edu}
% For a paper whose authors are all at the same institution,
% omit the following lines up until thess closing ``}''.
% Additional authors and addresses can be added with ``\and'',
% just like the second author.
% To save space, use either the email address or home page, not both
\and
Adriana Kovashka\\
University of Pittsburgh\\
%First line of institution2 address\\
{\tt\small kovashka@cs.pitt.edu}
}

\date{\vspace{-5ex}}

{

\maketitle
}
% Remove page # from the first page of camera-ready.


%%%%%%%%% ABSTRACT
\begin{abstract}
The use of large-scale vision-language datasets is limited for object detection due to the negative impact of label noise on localization. Prior methods have shown how such large-scale datasets can be used for pretraining, which can provide initial signal for localization, but is insufficient without clean bounding-box data for at least some categories. We propose a technique to ``vet'' labels extracted from noisy captions, and use them for weakly-supervised object detection (WSOD). We conduct analysis of the types of label noise in captions, and train a classifier that predicts if an extracted label is actually present in the image or not. Our classifier generalizes across dataset boundaries and across categories. We compare the classifier to eleven baselines on five datasets, and demonstrate that it can improve WSOD without label vetting by 30\% (31.2 to 40.5 mAP when evaluated on PASCAL VOC). 
\end{abstract}
\section{Introduction}
\label{sec:introduction}
% \begin{itemize}
%     % Diffusion of FL
%     \item {\st{Diffusion of FL}}
%     % Security threats to FL
%     \item {\st{Security threats to FL with particular focus on model poisoning}}
%     % Limitations of existing countermeasures
%     \item {\st{Current countermeasures (e.g., KRUM) and their limitations}}
%     % Proposed method and its advantages
%     \item {\st{Intuitive description of the proposed method and its difference (i.e., advantages) w.r.t. state of the art}}
%     % Main contributions
%     \item {\st{Summary of the main contributions of this work}}
%     % Paper's structure and organization
%     \item {\st{Paper's structure and organization}}
% \end{itemize}

% Diffusion of FL
Recently, {\em federated learning} (FL) has emerged as the leading paradigm for training distributed, large-scale, and privacy-preserving machine learning (ML) systems~\cite{mcmahan2017googleai,mcmahan2017aistats}. 
The core idea of FL is to allow multiple edge clients to collaboratively train a shared, global model without disclosing their local private training data.
%Specifically, an FL system consists of a central server and many edge clients; 
A typical FL round involves the following steps: {\em(i)} the server randomly picks some clients and sends them the current, global model; {\em(ii)} each selected client locally trains its model with its own private data; then, it sends the resulting local model to the server;\footnote{Whenever we refer to global/local model, we mean global/local model {\em parameters}.} {\em(iii)} the server updates the global model by computing an \emph{aggregation function}, usually the average (FedAvg), on the local models received from clients.
% \begin{enumerate}
%     \item[{\em(i)}] the server sends the current, global model to the clients and appoints some of them for training;
%     \item[{\em(ii)}] each selected client locally trains its copy of the global model with its own private data; then, it sends the resulting local model back to the server;\footnote{Whenever we refer to global/local model, we mean global/local model {\em parameters}.}
%     \item[{\em(iii)}] the server updates the global model by computing an \emph{aggregation function} on the local models received from clients (by default, the average, also referred to as FedAvg~\cite{mcmahan2017aistats}).
% \end{enumerate}
This process goes on until the global model converges. %(e.g., after a certain number of rounds or other similar stopping criteria).
%\\
% The advantages of FL over the traditional, centralized learning paradigm are undoubtedly clear in terms of flexibility/scalability (clients can join/disconnect from the FL network dynamically), network communications (only model weights\footnote{We will use \textit{parameters} and \textit{weights} interchangeably.} are exchanged between clients and server), and privacy (each client's private training data is kept local at the client's end and not uploaded to the server).
\\
% Security threats to FL
%However, the growing adoption of FL also raises security concerns~\cite{costa2022covert}, particularly about its confidentiality, integrity, and availability.
Although its advantages over standard ML, FL also raises security concerns~\cite{costa2022covert}. %, particularly about its confidentiality, integrity, and availability~\cite{costa2022covert}.
% OLD, LONG VERSION
% Indeed, some work deals with privacy leakage that may expose the local data of some clients~\cite{melis2019sp}. 
% A large body of work, instead, investigates attacks that usually aim to detriment the predictive accuracy of the learned global model. For instance, \emph{data poisoning} attacks achieve this goal by letting an adversary pollute the training set of some corrupt FL clients with maliciously crafted examples~\cite{jagielski2018sp}.
% Similarly, in \emph{model poisoning} the attacker attempts to tweak the global model weights~\cite{bhagoji2019pmlr} by directly perturbing the local model's weights of some infected FL clients before these are sent to the central server for aggregation, usually via so-called Byzantine attacks. 
% It turns out that Byzantine model poisoning attacks severely impact standard FedAvg; therefore, more robust aggregation functions must be designed to make FL systems secure.
Here, we focus on \emph{untargeted model poisoning} attacks~\cite{bhagoji2019pmlr}, where an adversary attempts to tweak the global model weights %\footnote{We will use the terms \textit{parameters} and \textit{weights} interchangeably.} 
by directly perturbing the local model's parameters of some infected clients before these are sent to the central server for aggregation.
In doing so, the adversary aims to jeopardize the global model \textit{indiscriminately} at inference time.
Such model poisoning attacks severely impact standard FedAvg; therefore, more robust aggregation functions must be designed to secure FL systems.
\\
% In this paper, we focus on designing a novel robust aggregation scheme at the server's end to contrast the effect of Byzantine model poisoning attacks.
%
% Current countermeasures and their limitations
%Several countermeasures have been proposed in the literature to combat model poisoning attacks on FL systems.
% Some methods use simple statistics more robust than plain average to smooth the impact of malicious updates (e.g., Trimmed Mean and FedMedian~\cite{yin2018icml}). 
% Other defenses implement outlier detection techniques to discard malicious updates from the aggregation performed at the server's end. Those are either based on heuristics (e.g., Krum/Multi-Krum~\cite{blanchard2017nips} and Bulyan~\cite{mhamdi2018pmlr}) or data-driven approaches (e.g., K-means clustering~\cite{shen2016acm} or DnC via spectral analysis~\cite{shejwalkar2021ndss}). 
% Finally, some strategies rely on a centralized ``source of trust'' to spot potential malicious updates (e.g., FLTrust~\cite{cao2020fltrust}).
% Several countermeasures have been proposed in the literature to combat model poisoning attacks on FL systems, i.e., to discard possible malicious local updates from the aggregation performed at the server's end. 
% These techniques range from simple statistics more robust than plain average (e.g., Trimmed Mean and FedMedian~\cite{yin2018icml}) to outlier detection heuristics (e.g., Krum/Multi-Krum~\cite{blanchard2017nips} and Bulyan~\cite{mhamdi2018pmlr}) or data-driven approaches (e.g., spectral analysis via K-means clustering~\cite{shen2016acm} or spectral analysis), or methods based on ``source of trust'' (e.g., FLTrust~\cite{cao2020fltrust}).
% OLD, LONG VERSION
%Several countermeasures have been proposed in the literature to combat Byzantine model poisoning attacks on FL systems.
% Descriptive statistics
% For example, Trimmed Mean and FedMedian aggregate local model updates using more robust statistics than standard average~\cite{yin2018icml}.
%
% % Heuristics for outlier detection
% Many existing Byzantine-resilient strategies implement some outlier detection heuristics to discard the model updates sent by potentially malicious clients from the input of the aggregation function.
% One of the most popular heuristics is Krum~\cite{blanchard2017nips}.
% This strategy tries to mitigate the impact of Byzantine attacks by selecting as a global model the local model with the smallest sum of Euclidean distances to {\em all} the other local models.
% Although powerful, Krum requires the server to know (or, at least, estimate) the number of malicious FL clients upfront, which is generally impossible in a realistic attack scenario. %
% Moreover, Krum may become ineffective for complex, high-dimensional model parameter spaces due to the curse of dimensionality.
% Bulyan~\cite{mhamdi2018pmlr} tries to overcome this issue by combining Krum with a variant of Trimmed Mean.
% % Data-driven outlier detection
% Other strategies use data-driven outlier detection techniques -- e.g., via K-means clustering~\cite{shen2016acm} -- to spot potential malicious local model updates. 
% %For instance, Shen et al. propose to cluster local model updates with K-means and thus identify outliers.
%
% % Other techniques
% As far as the server is concerned, any local model received can be from a potential malicious client. 
% FLTrust~\cite{cao2020fltrust} assumes the server acts as a client, i.e., trains a local model on an additional {\em trustworthy} dataset at the server's end and compares it against all the local models from other clients. 
% This way, the server can rely on some ``source of trust'' when discarding potentially malicious clients.
%\\
% Limitations of existing Byzantine-resilient strategies
Unfortunately, existing defense mechanisms either rely on simple heuristics (e.g., Trimmed Mean and FedMedian by~\cite{yin2018icml}) or need strong and unrealistic assumptions to work effectively (e.g., foreknowledge or estimation of the number of malicious clients in the FL system, as for Krum/Multi-Krum~\cite{blanchard2017nips} and Bulyan~\cite{mhamdi2018pmlr}, which, however, cannot exceed a fixed threshold).
Furthermore, outlier detection methods using K-means clustering~\cite{shen2016acm} or spectral analysis like DnC~\cite{shejwalkar2021ndss} do not directly consider the temporal evolution of local model updates received.
Finally, strategies like FLTrust~\cite{cao2020fltrust} require the server to collect its own dataset and act as a proper client, thereby altering the standard FL protocol.
\\
% OLD, LONG VERSION
% Overall, existing Byzantine-resilient strategies are either simple heuristics (e.g., FedMedian) or, if they are more complex, they rely on strong and unrealistic assumptions to work effectively (e.g., knowing the number of malicious clients in the FL system in advance, as for Krum and alike).
% Furthermore, data-driven outlier detection methods do not consider the temporary evolution of local model updates received (e.g., K-means clustering). 
% Finally, strategies like FLTrust requires the server to collect its own dataset and act as a proper client, thereby altering the standard FL protocol.
%
% Description of the proposed method
This work introduces a novel pre-aggregation \textit{filter} robust to untargeted model poisoning attacks. Notably, this filter $(i)$ operates without requiring prior knowledge or constraints on the number of malicious clients and $(ii)$ inherently integrates temporal dependencies. 
The FL server can employ this filter as a preprocessing step before applying \textit{any} aggregation function, be it standard like FedAvg or robust like Krum or Bulyan.
Specifically, we formulate the problem of identifying corrupted updates as a multidimensional (i.e., matrix-valued) time series anomaly detection task. 
The key idea is that legitimate local updates, resulting from well-calibrated iterative procedures like stochastic gradient descent (SGD) with an appropriate learning rate, show \textit{higher predictability} compared to malicious updates. This hypothesis stems from the fact that the sequence of gradients (thus, model parameters) observed during legitimate training exhibit regular patterns, as validated in Section~\ref{subsec:intuition}. %until convergence. 
%This regularity may be more pronounced for smooth convex loss functions, but it can still be captured within an appropriate time window, even for more complex and convoluted loss surfaces. 
%We provide evidence of this claim in Appendix~B, where we show that the average mutual information (i.e., ``predictability''), calculated over pairs of legitimate model updates sent at different FL rounds, is significantly higher than the corresponding computation for a malicious client.
\\
Inspired by the matrix autoregressive (MAR) framework for multidimensional time series forecasting~\cite{chen2021je}, we propose the FLANDERS ({\em \textbf{F}ederated \textbf{L}earning meets \textbf{AN}omaly \textbf{DE}tection for a \textbf{R}obust and \textbf{S}ecure}) filter.
The main advantages of FLANDERS over existing strategies like FLDetector~\cite{zhao2020multivariate} are its resilience to large-scale attacks, where $50\%$ or more FL participants are hostile, and the capability of working under realistic non-iid scenarios.
We attribute such a capability to two key factors: $(i)$ FLANDERS works without knowing a priori the ratio of corrupted clients, and $(ii)$ it embodies temporal dependencies between intra- and inter-client updates, quickly recognizing local model drifts caused by evil players. Below, we summarize our main contributions:

\begin{itemize}
\item[{\em(i)}]
We provide empirical evidence that the sequence of models sent by legitimate clients is more predictable than those of malicious participants performing untargeted model poisoning attacks.
\\
\item[{\em(ii)}] 
We introduce FLANDERS, the first pre-aggregation filter for FL robust to untargeted model poisoning based on multidimensional time series anomaly detection.
\\
\item[{\em(iii)}] 
We integrate FLANDERS into Flower,\footnote{\scriptsize{\url{https://flower.dev/}}} a popular FL simulation framework for reproducibility.
\\
\item[{\em(iv)}] 
We show that FLANDERS improves the robustness of the existing aggregation methods under multiple settings: different datasets, client's data distribution (non-iid), models, and attack scenarios.
\\
\item[{\em(v)}] 
We publicly release all the implementation code of FLANDERS along with our experiments.\footnote{\scriptsize{\url{https://anonymous.4open.science/r/flanders_exp-7EEB}}}
\end{itemize}

% Paper's structure and organization
The remainder of the paper is structured as follows. %some related work and the current state-of-the-art solutions to security issues that FL entails. 
Section~\ref{sec:background} covers background and preliminaries. 
In Section~\ref{sec:related}, we discuss related work.
Section~\ref{sec:problem} and Section~\ref{sec:method} describe the problem formulation and the method proposed. % to tackle it. 
Section~\ref{sec:experiments} gathers experimental results. %, and Section~\ref{sec:limitations} discusses some limitations of this work.
Finally, we conclude in Section~\ref{sec:conclusion}.
 %discusses the limitations of this work and draws future research directions.
%reports conclusions and draws perspectives for future research directions.

%%%%%%% OLD %%%%%%%
%to overcome the resilience of Byzantine failures in distributed Stochastic Gradient Descent computations. 
% The strength of Krum is its time complexity, which is linear in the gradient dimension. 
% However, the robustness of the approach is guaranteed for gradient-based learning applications only when the majority of the clients are not compromised. 
% Besides, the aggregation mechanism of Krum, as well as that of similar methods, is robust from a coarse-grained perspective and does not provide solutions to errors and perturbations that may occur at inference time.
%A related approach to~\cite{blanchard2017nips} is the work of Su et al.~\cite{su2016dc}. Here, the authors propose an iterated approximate agreement to tackle a multi-layer scenario attacked by Byzantine agents. 
%However, the method works efficiently on the sole discrete context and it is inapplicable to continuous state environments.
%\gabri{Maybe, we should just talk about the main limitations of existing countermeasures without digging into their details (or, we can just mention Krum as this is the most popular one). I will move the description of all these methods to the Related Work section.}
\setlength{\tabcolsep}{1.6mm}{
\renewcommand\arraystretch{1.1}
\begin{table}[ht]
  \centering
  \scalebox{0.9}{
  \begin{tabular}{llcccc}
    \toprule
    &\multirow{2}*{Methods} & \multirow{2}*{Sal.} &   \multicolumn{2}{c}{VOC} & MS~COCO \\
    \cmidrule(r){4-5}\cmidrule(r){6-6}
    &&&\texttt{val}&\texttt{test}&\texttt{val}\\
    \hline
    \multirow{13}*{\rotatebox{90}{ResNet-50}}
    &IRN~\cite{irn}          \tiny{CVPR'19}     &              & 63.5       & 64.8          & 42.0  \\
    &LayerCAM~\cite{layercam}\tiny{TIP'21}      &              & 63.0       & 64.5          & -     \\
    &AdvCAM~\cite{advcam}    \tiny{CVPR'21}     &              & 68.1       & 68.0          & 44.2  \\
    &RIB~\cite{rib}          \tiny{NeurIPS'21}  &              & 68.3       & 68.6          & 44.2  \\
    &ReCAM~\cite{recam}      \tiny{CVPR'22}     &              & 68.5       & 68.4          & 42.9  \\
    % \rowcolor{Gray}
    &\cellcolor{Gray}IRN+\texttt{LPCAM}    &\cellcolor{Gray} & \cellcolor{Gray}68.6    & \cellcolor{Gray}68.7      & \cellcolor{Gray}44.5  \\
    &SIPE~\cite{sipe}        \tiny{CVPR'22}     &              & 68.8       & 69.7          & 40.6  \\
    &OOD~\cite{ood}+Adv      \tiny{CVPR'22}     &              & 69.8       & 69.9          & -     \\
    &AMN~\cite{amn}          \tiny{CVPR'22}     &              & 69.5       & 69.6          & 44.7  \\
    &\cellcolor{Gray}AMN+\texttt{LPCAM}    &\cellcolor{Gray} & \cellcolor{Gray}70.1    &\cellcolor{Gray} 70.4      & \cellcolor{Gray}45.5  \\ 
    &ESOL~\cite{esol}        \tiny{NeurIPS'22}  &              & 69.9$^*$   & 69.3$^*$      & 42.6  \\
    &CLIMS~\cite{clims}      \tiny{CVPR'22}     &              & 70.4$^*$   & 70.0$^*$      & -     \\
    &EDAM~\cite{edam}        \tiny{CVPR'21}     &\checkmark    & 70.9$^*$   & 71.8$^*$      & -     \\
    &\cellcolor{Gray}EDAM+\texttt{LPCAM}  &\cellcolor{Gray}\checkmark & \cellcolor{Gray}71.8$^*$ &\cellcolor{Gray} 72.1$^*$& \cellcolor{Gray}42.1\\
    \hline
    \multirow{9}*{\rotatebox{90}{WideResNet-38}}
    &Spatial-BCE~\cite{sbce} \tiny{ECCV'22}     &              & 70.0       & 71.3      & 35.2  \\
    &BDM~\cite{bdm}          \tiny{ACMMM'22}    &\checkmark    & 71.0       & 71.0      & 36.7  \\ 
    &RCA~\cite{rca}+OOA      \tiny{CVPR'22}     &\checkmark    & 71.1       & 71.6      & 35.7  \\
    &RCA~\cite{rca}+EPS      \tiny{CVPR'22}     &\checkmark    & 72.2       & 72.8      & 36.8  \\
    &HGNN~\cite{hgnn}        \tiny{ACMMM'22}    &\checkmark         & 70.5$^*$   & 71.0$^*$  & 34.5  \\ 
    &EPS~\cite{eps}          \tiny{CVPR'21}     &\checkmark         & 70.9$^*$   & 70.8$^*$  & -     \\
    &RPIM~\cite{rpim}        \tiny{ACMMM'22}    &\checkmark         & 71.4$^*$   & 71.4$^*$  & -     \\ 
    &L2G~\cite{l2g}          \tiny{CVPR'22}     &\checkmark         & 72.1$^*$   & 71.7$^*$  & 44.2  \\
    \hline
    \multirow{2}*{\rotatebox{90}{\small{DeiT-S}}}
    &MCTformer~\cite{mctformer}    \tiny{CVPR'22}     &                 & 71.9$^{\dag}$  & 71.6$^{\dag}$   & 42.0  \\
    &\cellcolor{Gray}MCTformer+\texttt{LPCAM}      &\cellcolor{Gray} & \cellcolor{Gray}72.6$^{\dag}$  & \cellcolor{Gray}72.4$^{\dag}$  &\cellcolor{Gray} 42.8 \\
    \bottomrule
  \end{tabular}}
  \vspace{-2mm}
  \caption{The mIoU results (\%) based on DeepLabV2 on VOC and MS~COCO. The side column shows three backbones of multi-label classification model. ``Sal.'' denotes using saliency maps. * denotes the segmentation model is pre-trained on MS~COCO. $^\dag$ denotes the segmentation model is pre-trained on VOC.
  }
  \vspace{-6mm}
  \label{table_related}
\end{table}
}





\section{Label Noise Analysis and Dataset}%Caption Label Noise (CLaN) Dataset}
%\section{Analysis: Label and Visual Noise Dataset}
\label{sec:analysis}

We analyze what makes large-scale in-the-wild datasets a challenging source of labels for object detection methods. 

% ANALYSIS TABLE MOVED UP TO RELATED WORK SO IT CAN BE AT THE TOP OF THE PAGE

\textbf{Datasets analysed.}
%\label{sec:datasets}
Conceptual Captions (CC) \cite{Sharma2018ConceptualCA}, RedCaps  \cite{Desai2021RedCapsWI}, and SBUCaps \cite{Ordonez2011Im2TextDI}, are collected from in-the-wild data sources. \textbf{CC} contains 3 million image-alt-text pairs after heavy post-processing; named entities in captions were hypernymized and image-text pairs were accepted if there was an overlap between Google Cloud Vision API class predictions and the caption.
\textbf{RedCaps} %is the largest dataset used in our paper, 
consists of 12M image-text pairs collected from Reddit by crawling a manually curated list of subreddits with heavy visual content.
\textbf{SBUCaps} consists of 1 million Flickr photos with text descriptions written by their owners.
Only captions with at least one prepositional phrase and at least 2 matches with a predefined vocabulary were accepted.
These in-the-wild datasets exhibit very low precision of the extracted labels, ranging from 0.463 for SBUCaps, 0.596 for RedCaps, to 0.737 for CC, all much lower than the 0.948 for COCO (shown in supp). 
%This motivates our exploration of VAEL noise.

\textbf{Extracted object labels.} Given a vocabulary of object classes, we extract a label for an image if there is exact match between the object name and the corresponding caption ignoring punctuation, as in \cite{Ye_Zhang_Kovashka_Li_Qin_Berent_2019,Fang2022DataDD}.

\textbf{Gold standard object labels.} We use pseudo-ground-truth predictions from a pretrained image recognition model to estimate visual presence \textit{gold standard} labels because these in-the-wild datasets do not have image-level object annotations. 
We use an object recognition ensemble with the X152-C4 object-attribute model \cite{zhang2021vinvl} 
% (trained on four public object detection datasets) 
and the Ultralytic  YOLOv5-XL \cite{yolov5}.
This ensemble achieves strong accuracy, %on visual presence detection, e.g.
82.2\% on SBUCaps, 85.6\% on RedCaps, and 86.8\% on CC.
%We observed higher visual presence accuracy on a small annotated set compared to each model alone (see supp). 
We extract VAELs by selecting images where extracted and gold-standard labels disagree. 
%Note in some cases a VAEL will correspond to a present object which is however missed by the recognition ensemble.

\textbf{Noise annotations collected.}
We select 100 VAEL examples per dataset (RedCaps, SBUCaps, CC).
We annotate four types of information for these examples: 
\begin{itemize}[nolistsep,noitemsep]
    \item (Q1: Label Noise) How much of the VAEL object is present (\underline{vis}ible, \underline{part}ially visible, completely \underline{abs}ent); 
    \item (Q2: Similar Context) If the VAEL object is completely missing, whether a traditionally co-occurring context (``boat" and ``water"), or semantically similar object (e.g. ``cake" and ``bread", ``car" and ``truck") is present; 
    \item (Q3: Visual Defects) If visible/partially visible, whether the VAEL object is occluded, has key parts missing, or atypical appearance (e.g. knitted animal); and
    \item (Q4: Linguistic Indicators) What linguistic cues, if any, explain why the VAEL object is mentioned but absent, e.g. the caption discusses events or information beyond what the image shows (``beyond'' in Table \ref{tab:stats}), describes the past (``past''), the extracted label is part of a prepositional phrase and likely to describe the setting and not objects (``on a train''), is a noun modifying another noun, is used in a non-literal way, has a different word sense (e.g. ``bed'' vs ``river bed''), or is part of a named entity.
\end{itemize}

Two annotators (authors) provide the annotations, with high agreement: 0.76 for Q1, 0.33 for Q2, 0.45 for Q3, and 0.58 for Q4. We calculate Cohen's Kappa for each option and aggregate agreement through a weighted average for each question, with weights derived from average option counts between the two annotators across the three datasets. We label the dataset Caption Label Noise, or CLaN.

In Table \ref{tab:stats}, we show what fraction of samples fall into each annotated category, excluding ``Other'', ``Unclear'' and uncommon categories. We average the distribution between the two annotators.

\textbf{Statistics: Label noise.}
We first characterize the visibility of objects flagged as VAELs by the recognition ensemble. We find that SBUCaps has the highest rate of completely absent images (58.5\%), followed closely by RedCaps. 
CC has the highest full visibility (32.8\%),
%followed by RedCaps (29.2\%) and then SBUCaps (21.5\%) where full visibility is 
defined as the object from a given viewpoint having 75\% or more visibility. 
SBUCaps also has the highest rate of partially visible objects (20\%). 
The high rate of absent and partially-visible objects justifies the use of pseudo-ground-truth labels from the recognition ensemble; these both constitute poor training data for WSOD. 
We investigate the fully-visible objects flagged as VAELs shortly, through our visual defect annotations. 
%These significant rates of visibility motivates investigating any visual defects in partially or fully visible objects in Q3 which may explain why the recognition ensemble flagged these as absent and could be considered a difficult example \cite{Bengio2009CurriculumL}. Secondly, it also motivates marking linguistic indicators in Q4 which can be used to predict both visual defects and completely absent objects in Q4.

\textbf{Statistics: Similar context.}
%One issue with solely focusing on object visibility is that 
Certain images with absent objects may be more harmful than others. Prior work has shown that models exploit co-occurrences between an object and its context which helps overall recognition accuracy, but can hurt performance when that context is absent \cite{Singh_Mahajan_Grauman_Lee_Feiszli_Ghadiyaram_2020}. We hypothesize the inclusion of images with this context bias without the actual object present could affect localization especially when supervising detection \textit{implicitly}, and semantically similar context may blur decision boundaries.
%hurting classification. 
%This motivates annotating if the image contains co-occurring context or if semantically similar objects are present instead of the VAEL. The question is subjective as 
Different annotators may have different references for similarity or co-occurrence frequency, but our annotators achieve fair agreement ($\kappa=0.33$). In Table \ref{tab:stats}, we find high rates of co-occurring contexts in samples with completely absent VAELs for SBUCaps (42.5\%) and CC (30.9\%).
%, while this is rarer in RedCaps (4\%). 
Across all datasets, we see a similar rate, 12\%-15\%, of similar context being present instead of the VAEL. 
% For example, if the image contains co-occurring context (e.g. "bus stop with buildings") while the object (e.g. "bus") is absent, then the inclusion of this example could harm localization.

\textbf{Statistics: Visual defects.}
%VAELs include a number of fully visible samples.
%or partially  
We hypothesize there may be visual defects which caused the recognition ensemble to miss fully-visible objects. 
%Summing over the columns in Visual defects in Table \ref{tab:stats}, we observe that indeed most visible samples have defects.
Over the fully or partially visible subset, in CC 79\% of fully or partially visible objects have a visual defect, 87\% for SBUCaps, and 69\% for RedCaps. 
The most common defect for RedCaps and CC is atypical (49\% and 57.3\%); we argue these atypical examples constitute poor training data for WSOD.
%, resp.) and occlusion for SBUCaps (26.5\%) in Table \ref{tab:stats}. This also shows that using the recognition ensemble to estimate visual presence is also a decent proxy for visual defects, so the recognition ensemble's missed predictions are useful. Upon further examination of captions, 
We find that the caption context (e.g. ``acrylic illustration of the funny mouse") may indicate the possibility of a visual defect, which further motivates the VEIL design. 

\textbf{Statistics: Linguistic indicators.} Noun modifier is one of the most frequently occurring linguistic indicator. Prepositional phrase is also significant in SBUCaps (40.5\%) and CC (31.3\%).
%, compared to RedCaps (5.7\%)
%and non-literal use is common in SBUCaps and RedCaps.
%have almost double the amount of non-literal linguistic indicator compared to CC. 
All datasets contain many VAELs which are mentioned in contexts going beyond the image; for example:
%the VAEL, ``boat” is mentioned in text that goes beyond the image: 
``just got back from the river. friend \textbf{sank his truck pulling his \underline{boat} out}. long story short, rip this beast” (RedCaps). %These linguistic indicators motivate rule-based methods as our baselines, but 
%Given that VAELs can be explained by a number of these indicators, rule-based method will miss out on VAELs. 
%We believe this relationship between linguistic indicators and VAELs can be learned implicitly if a transformer-based model is provided with a loose proxy for visual presence/absence labels, motivating VEIL.
%\subsection{Broader Label Noise Patterns}
\label{sec:taxonomy}

\begin{figure}[t] %[h]   
    \includegraphics[width=1\linewidth]{iccv2023AuthorKit/figures/images/label_tax_font_larger2.png}
    %\includegraphics[scale=0.70] {iccv2023AuthorKit/figures/images/label_noise_taxonomy3.png}
    \caption{Label noise taxonomy outlining common reasons for visual absence in extracted labels from in-the-wild captions.}
    \label{fig:label_noise_taxonomy}
\end{figure}

%Objects mentioned in human-annotated captions are usually present in the image (see precision on COCO and VIST-DII in supp.); however, real world captions contain a range of expressions that mention objects not shown in the corresponding image. 
We have identified a taxonomy of label noise found in in-the-wild captions, illustrated in Figure \ref{fig:label_noise_taxonomy}. We focus on 
%with reasons as children nodes for each broad category of noise: 
visually absent, extracted labels (VAELs).
%and visually present, extracted labels (VPEL), with the former being our focus.
%Mentions that are completely missing will now be referred to as visually absent extracted labels (VPELs) and are the main focus of this paper. 
We find that both descriptive and narrative captions mention objects that are semantically unrelated to the category label. 
They can be part of a larger named entity e.g. ``Super \textbf{Bowl} Party" or part of a noun phrase which can either be related to the category e.g. ``\textbf{bottle} opener", or unrelated e.g. ``hot \textbf{dog}" or ``\textbf{orange}'' the color instead of the fruit. Other VAELs are narrative artifacts providing personal context for the image and representing the colorful ways people express themselves. These include non-literal mentions (metaphors and similies), prepositional phrases e.g. ``Left a lighter in my \textbf{car}", describing the past e.g. ``My chest hurts and I can't stop crying. Hugo was run over by a \textbf{car} yesterday", or rarer, negations e.g. ``We did not spot a \textbf{tiger} in our safari". We believe these VAELs can be learned implicitly if a transformer-based model is provided with a loose proxy for visual presence or absence labels, motivating VEIL.
%Not all VAEL examples are equally obvious to recognize without an image reference; this is due to ambiguous language that indicates that some object mentions may be missing from the image but its not obvious ("Sometimes there's a \textbf{dog} under the trucks. I'm on the hunt to photograph him and a \textbf{monkey} or too"). In addition, not all objects in an image are recognizable without an caption prior; people tend to post pictures of their pets in costumes which obfuscates what animal it really is. While we don't explore bad examples of objects in this paper, an emergent behaviour of our model is rejecting atypical examples. We leave exploring this to future work.

% --- OLD TEXT

% Hard examples
% While this is not tackled by this paper, captions can also indicate that the extracted labels would correspond to a hard example: deformed objects (cut up/deteriorated/etc), mostly occluded (most key parts are missing from the image), atypical instances (cat dressed up as something else), out of real world domain (like clipart) – hard where key parts of the object are missing, small objects or objects far off in a distance, close ups where key parts are missing. Other examples might be edge cases: food trucks → can decrease the difference between cars and trucks. Although vans are considered a truck with an enclosed cargo space (WordNet).
% Tractor -> a truck, but with only a cab
% So a truck consists of two parts, a cab and/or a cargo, but even if you have only one part then it is a specific instance of a truck
% Old example
% Antique trucks - [“old”, “antique”, “year mentioned”]

% Chat GPT Rewrite
% Objects mentioned in human-annotated captions are usually present in the image; however, real-world captions can describe a wide range of expressions that may mention objects that are not shown in the corresponding image. In this paper, we focus on a type of label noise called visually absent extracted labels (VPELs), which refers to mentions of objects that are completely missing from the image. There are several reasons why an mentioned object might be visually absent, such as if the object is semantically unrelated to the image or is absent due to narrative artifacts. These semantically unrelated object mentions can be part of a larger named entity, such as "super bowl party," or part of a noun phrase, such as "bottle opener." Other VPELs can be caused by narrative artifacts, such as non-literal mentions (e.g. metaphors and similies), prepositional phrases (e.g. "left a lighter in my car"), descriptions of the past (e.g. "my chest hurts and I can't stop crying. Hugo was run over by a car yesterday"), or negations (e.g. "no tiger"). Additionally, there may be instances that are unrecognizable without the caption, such as objects that are deformed, occluded, atypical, or out of the real-world domain (e.g. clipart).

% It is worth noting that there are also cases where the extracted labels correspond to hard examples, such as objects that are small or far off in the distance, or close-ups where key parts are missing. There may also be edge cases, such as food trucks, which can blur the line between cars and trucks, or tractors, which are considered a specific type of truck. Similarly, antique trucks may be described with words like "old" or "antique" and may include a mention of the year.





\section{Method}
\label{sec:veil}

% MOVED EXTRACTING LABELS TO ANALYSIS SECTION

\begin{figure}[t]
    \centering
    \includegraphics[scale=0.4]{iccv2023AuthorKit/figures/images/horizontal_veil_model.png}
    %scale=0.55
    % \vspace{-2.8cm}
    \caption{VEIL model architecture. A pretrained tokenizer parses the input caption into tokens which is passed to a transformer-based language model. Finally, the vetting and masking layer only predict visual presence of tokens corresponding to a label.}
    %In this example, only "dog" is an extracted label and it fails the vetting process.}
    \label{fig:elavet_arch}
\end{figure}

\textbf{Vetting labels (VEIL).} The extracted label vetting (ELV) task uses visual presence targets that are assigned based on predictions from a pretrained object recognition model for \textit{each} extracted label from the caption.
The method is overviewed in Fig.~\ref{fig:elavet_arch}.
Our model takes in a sequence of $C$ word token-level caption embeddings. WordPiece \cite{wu2016google} tokenization breaks captions into subwords, and each subword is mapped to a corresponding embedding, resulting in $e \in \mathbb{R}^{d\times C}$. These embeddings are passed through a language model, $h$, which include multiple layers of multi-head self-attention over tokens in the caption to compute token-level output embeddings $v\in \mathbb{R}^{d\times C}$.
These embeddings are then passed to an MLP and the model outputs a sequence of visual presence predictions per token, $r\in [0,1]^{C}$. 
\begin{gather}
    v = h(e) \\
    r = \sigma(W_2(\tanh(W_1v))
\end{gather}
where $W_1\in \mathbb{R}^{d\times d}$ and $W_2\in \mathbb{R}^{1\times d}$. 

Not all predictions in $r$ correspond to an extracted label, so we use a mask, $M\in [0,1]^{C}$, such that only the predictions associated with the extracted labels are used in binary cross entropy loss.  
To train this network, the pseudo-label targets are present, $y_i = 1$, if a pretrained object detector also predicts the same category as the extracted label. 
\begin{gather}
    L = \frac{1}{M^{T}M}\sum_{i=1}^{C}M_i\Big[y_i \log r_i + (1 - y_i) \log(1-r_i)\Big]
\end{gather}
During \emph{inference}, if an extracted label 
%has been tokenized into 
was mapped to multiple tokens (e.g. ``teddy bear"), the predictions are averaged.

\textbf{Special token.} We test VEIL$_{\text{ST}}$ which inserts a special token {\tt [EM\_LABEL]} before each extracted label in the caption to reduce the model's reliance on category-specific cues and improve generalization to other datasets. We find that it helps only the latter.
%\textbf{VEIL variants.} We test out two methods that can reduce the model's reliance on category specific predictions and improve generalization to other datasets: (1) VEIL$_{\text{ST}}$ which inserts a special token {\tt [EM\_LABEL] }right before each extracted label, and (2) randomly masking the tokens associated with the extracted labels with probability $p_m$.
%By inserting a special token right before each extracted label in the caption, we are explicitly indicating to the model which tokens correspond to the extracted labels. This can help the model to better learn the relationship between the surrounding context and the extracted labels, and to focus more on the relevant parts of the input. We have some success with randomly masking the input extracted label tokens to encourage the model to extract information from the surrounding context instead of specific categories. However, we only apply this variant in the category generalization experiments. 



\textbf{Weakly-supervised object detection.}
To test the ability of extracted label filtering or correction methods for weakly-supervised object detection, we train MIST \cite{ren2020instance}. MIST extends WSDDN \cite{bilen2016weakly} and OICR \cite{Tang2017MultipleID} to mine pseudo-ground truth boxes prior to iterative refinement such that multiple instances are not grouped as one.
VEIL uses training data from the in-the-wild datasets to train the vetting model, and we want to see how its ability to vet labels generalizes to unseen data. Thus, we use the test splits of the in-the-wild datasets to train MIST, as they are unseen by all vetting methods. We do not evaluate the WSOD model on these in-the-wild datasets, but rather on disjoint datasets, VOC-07 \cite{Everingham2010ThePV} and COCO 2014 \cite{Lin2014MicrosoftCC}. 

%\textbf{Weighted sampling.}
% \begin{gather}    
% \end{gather}

\textbf{Implementation details.}
VEIL is implemented in PyTorch \cite{pytorch} and uses a pretrained BERT encoder \cite{BERT} prior to the per-token visual presence classification layer. 
%We use MIST \cite{ren2020instance} to learn an object detection model using weak supervision from image-level labels extracted from captions. 
We simulate a batch size of 8 for all experiments unless specified otherwise. To address class imbalance during WSOD, 
%caused by in-the-wild datasets' long-tail distributions
we use the complement of the sub-sampling probability introduced in Word2Vec \cite{Mikolov2013DistributedRO} as weights. 
%We train under different GPU settings due to resource constraints, and use gradient accumulation for some experiments. 
We used 4 RTX A5000 GPUs and trained for 50k iterations with a batch size of 8, or 100k iterations on 4 Quadro RTX 5000 GPUs with a batch size of 4 and gradient accumulation (parameters updated every two iterations to simulate a batch size of 8).
\section{Experiments}
\label{sec:expts}

\begin{table*}[h]
\begin{center}
\small
\begin{tabular}{>{\centering\arraybackslash}m{0.5in}|p{1.3in}|p{0.4in}|p{0.4in}|p{0.4in}|p{0.4in}|p{0.4in}|p{0.4in}|p{0.4in}|p{0.4in}}
\hline
\cellcolor{verylightorange} & \cellcolor{lightgreen} \textbf{Method} & \cellcolor{lightgreen}\textbf{SBUCaps} & \cellcolor{lightgreen}\textbf{RedCaps} & \cellcolor{lightgreen}\textbf{CC} & \cellcolor{lightgreen} \textbf{VIST} & \cellcolor{lightgreen}\textbf{VIST-DII} & \cellcolor{lightgreen}\textbf{VIST-SIS}& \cellcolor{lightgreen}\textbf{COCO} & \cellcolor{lightgreen}\textbf{AVG}\\\cline{2-10}

\cellcolor{verylightorange}&\cellcolor{lightorange} No Vetting & 0.633 & 0.747 &  \underline{0.849} & 0.853 & \underline{0.876} & \underline{0.820} & \textbf{0.973} & 0.822 \\
\hline
\cellcolor{verylightorange}&\cellcolor{lightorange}Global CLIP \cite{Radford2021LearningTV} &  0.604&	0.583	&0.569	&0.668	&0.625&	0.683&	0.662& 0.628\\
\multirow{-2}{*}{\makecell{Image \\+ Lang.}}\cellcolor{verylightorange}&\cellcolor{lightorange} Global CLIP - E \cite{Radford2021LearningTV} &  0.594 &	0.569&	0.534&	0.654&	0.613&	0.660&	0.640 & 0.609\\
 \hline
\cellcolor{verylightorange}&   \cellcolor{lightorange}Local CLIP \cite{Radford2021LearningTV} &   0.347 & 0.651 & 0.363 & 0.427	&0.476	&0.418	&0.464 &0.449 \\
\cellcolor{verylightorange}&\cellcolor{lightorange} Local CLIP - E \cite{Radford2021LearningTV} &\underline{0.760} & \underline{0.840} & 0.597 & 0.759	&0.695&	0.812&	0.788 &0.750 \\
\multirow{-3}{*}{\makecell{ Image}}\cellcolor{verylightorange}&\cellcolor{lightorange} Reject Large Loss  \cite{Kim2022LargeLM}
 & 0.667 & 0.790 & 0.831&0.782	&0.794	&0.743&	0.896 &0.786 \\
  \hline
\cellcolor{verylightorange}& \cellcolor{lightorange}Accept Descriptive & 0.491 & 0.413 & 0.740 & 0.687&	0.844&	0.264&	0.935 & 0.625\\
\cellcolor{verylightorange}& \cellcolor{lightorange}Accept Narrative & 0.470 & 0.645 & 0.383 & 0.487 &	0.154&	0.757	&0.143 & 0.434 \\
\cellcolor{verylightorange}&\cellcolor{lightorange}Reject Noun Mod. (Adj) & 0.618 & 0.703 & 0.814& 0.823	&0.847	&0.788 &	0.906 & 0.786 \\
\cellcolor{verylightorange}&\cellcolor{lightorange} Reject Noun Mod. (Any)
 & 0.616 & 0.689 & 0.812 & 0.821	&0.842	&0.782 &	0.900&  0.780\\
\cellcolor{verylightorange} &\cellcolor{lightorange}  Cap2Det \cite{Ye_Zhang_Kovashka_Li_Qin_Berent_2019}
 & 0.639 & 0.758 & 0.846& 0.826&	0.854	&0.774	& \underline{0.964}& 0.809\\\cline{2-10}
\cellcolor{verylightorange}&\cellcolor{lightorange} VEIL-Same Dataset & \textbf{0.809} & \textbf{0.890} & \textbf{0.909} & \underline{0.871}	&\textbf{0.892}	& 0.816 &	\textbf{0.973} & \textbf{0.884}\\
\multirow{-7}{*}{\makecell{Lang.}}\cellcolor{verylightorange}&\cellcolor{lightorange} VEIL-Cross Dataset
 & 0.716 & 0.793 & 0.828 & \textbf{0.875} &	\textbf{0.892}	&\textbf{0.830} &	0.958 & \underline{0.842}\\\hline
\end{tabular}
\end{center}
\caption{Extracted Label Vetting F1 Performance. Visual presence ground truth is estimated by an object detection ensemble, X152-C4 \cite{zhang2021vinvl} and YOLOv5-XL \cite{yolov5}, on all datasets except for COCO, where we use existing annotations. \textbf{Bold} indicates the best performance in each column, and \underline{underlined} denotes the second-best performance.}
%Precision/recall values are shown in our supplementary materials.}
\label{tab:direct_eval}
\end{table*}

% \begin{table*}[h]
\begin{center}
\begin{tabular}{|c|c|c|}
\hline
\textbf{Method (Training Dataset Size)} &{\textbf{PASCAL VOC 2007 test}} &\textbf{COCO 2014 Val} \\
\hline
Using Ground Truth For Vetting (17K) & 0.0824 & 0.0122 \\
No Vetting  (19K) & 0.1604 & 0.0185 \\
Large Loss Matters \cite{Kim2022LargeLM} (19K) & 0.1809 &  0.0165 \\
LocalCLIP-E \cite{Radford2021LearningTV} (18K) & 0.1095 & 0.0136 \\
VEIL-Redcaps (18K) & \underline{0.2681} & \underline{0.0290} \\
VEIL-SBUCaps (16K) & \textbf{0.2910} & \textbf{0.0314} \\
\hline
\end{tabular}
\end{center}
\caption{WSOD Evaluation on PASCAL VOC 2007 and COCO 2014 Val.}
\label{tab:wsod_eval}
\end{table*}

%

\begin{table}[]
    \centering
    \small
    \begin{tabular}{c|c|c}\hline
Method  & Prec/Rec & F1 \\\hline
No Vetting & 0.323 / 1.000 & 0.488 \\\hline
ID & 0.651 / 0.656  & 0.654 \\\hline
OOD & 0.585 / 0.556 & 0.570 \\\hline
    \end{tabular}
    \caption{VEIL category generalization on SBUCaps-ID. }
    \label{tab:cross_category}
\end{table}

We show the ability of VEIL %and other language conditioned vetting methods 
to match and exceed language-agnostic filtering and image-based filtering methods in extracted label vetting (ELV). Next, we highlight the promising ability of VEIL to vet noisy extracted labels prior to weakly-supervised object detection training and %a critical ability to 
remove structured noise.
%; this is important because structured noise decreases localization ability in WSOD. 
Lastly, we benchmark the generalization ability of VEIL in cross-dataset and cross-category settings.

\subsection{Experiment Details}

%\textbf{Datasets.}  
%As described in Sec.~\ref{sec:analysis}, 
We use three in-the-wild image-caption datasets (SBUCaps \cite{Ordonez2011Im2TextDI}, RedCaps \cite{Desai2021RedCapsWI}, Conceptual Captions \cite{Sharma2018ConceptualCA}) and three human annotated datasets that fall into descriptive (COCO \cite{Lin2014MicrosoftCC}, VIST-DII \cite{huang2016visual}) and narrative (VIST-SIS \cite{huang2016visual}). 
Each in-the-wild dataset and VIST are first reduced to a subset of image-caption pairs where a COCO category is explicitly mentioned in the caption. This subset is split into 80-20 train-test split. 
The WSOD models are trained on SBUCaps with labels vetted by different methods, and evaluated on PASCAL VOC 2007 test \cite{Everingham2010ThePV} and COCO val 2014 \cite{Lin2014MicrosoftCC}. The mAP metric uses $IOU=0.5$.
%evaluates both classification and localization ability of the WSOD models at 
%unless specified otherwise. 
However, when contrasting the image-level classification performance and detection performance we refer to them as Recognition maP and Detection mAP, respectively.

\subsection{Methods Compared}
For VEIL, we use the convention VEIL-DatasetX to signify that VEIL is trained on the train-split of DatasetX. Next, we describe the methods we compare against. 
We group these into language-based, image-based, and image-language methods. They are category-agnostic, except for Cap2Det \cite{Ye_Zhang_Kovashka_Li_Qin_Berent_2019} and LLM \cite{Kim2022LargeLM} which must be applied on closed vocabularies. 
\\\textbf{No Vetting.} Accept all extracted labels (has \emph{perfect recall}).
\\\textbf{Global CLIP and CLIP-E.} We use the ViT-B/32 pretrained CLIP \cite{Radford2021LearningTV} model. To enhance alignment \cite{Hessel2021CLIPScoreAR}, we add the prompt ``A photo depicts" to the caption and calculate the cosine similarity between the image and text embeddings generated by CLIP. Since the cosine similarity distribution varies per dataset, we train a Gaussian Mixture Model with two components on the train datasets and select image-text pairs predicted to the component with higher visual-caption alignment. 
% Global CLIP filters based on the whole image and caption, even if there are visually present extracted labels. 
For the ensemble variant (CLIP-E), we prepend multiple prompts to the caption,
% or extracted label, %and calculate the cosine similarity between image-text CLIP embeddings, and lastly 
and use the score from the highest-scoring prompt.
\\\textbf{Local CLIP and CLIP-E} follow a similar process but use cosine similarity between the image and the prompt ``this is a photo of a" followed by the extracted label. Extracted labels are filtered by Local CLIP, not entire captions, making this image-conditioned, not image-language conditioned vetting like Global CLIP. Local CLIP-E ensembles prompts.
\\\textbf{Reject Large Loss.} LLM \cite{Kim2022LargeLM} is language-agnostic adaptive noise rejection and correction method. To test its ELV ability, we simulate five epochs of WSOD training \cite{bilen2016weakly} and consider label targets with a loss exceeding the large loss threshold as ``predicted to be visually absent" after the first epoch. The large loss threshold uses a relative delta hyperparameter controlling the rejection rate (set as 0.002 in \cite{Kim2022LargeLM}).
\\\textbf{Accept Descriptive / Narrative.} We train a logistic regression model to predict whether a VIST \cite{huang2016visual} caption comes from the DII (descriptive) or SIS (narrative) split. The input vector to this logistic regression model contains binary variables corresponding to the presence of a part of speech vector (e.g. proper noun, adjective, verb - past tense, etc) in the caption. We accept extracted labels from a caption if it yields a score higher than 0.5 for descriptiveness or narrativeness, respectively. 
%We use the part-of-speech (POS) based classifier trained on VIST-DII (descriptive proxy) and VIST-SIS (narrative proxy).
\\\textbf{Reject Noun Mod. (Adj/Any).} Since an extracted label could be modifying another noun (``\underline{car} park"), a very simple baseline would reject such labels. We try two variants, the first noun modifier rule rejects an extracted label if the POS label is an adjective or is followed by a noun. The second rule rejects if the extracted label is not a noun.
\\\textbf{Cap2Det.} We reject a label if it is not predicted by the Cap2Det \cite{Ye_Zhang_Kovashka_Li_Qin_Berent_2019} classifier.


\begin{table*}[]
\begin{center}
\small 
\resizebox{\linewidth}{!}{%
    \begin{tabular}{c|l|c|c|c|c|c|c|c|c|c|c|c|c|c}
    \hline
        Data & Vetting Method & \multicolumn{2}{c|}{Label noise} & \multicolumn{2}{c|}{Similar context} & \multicolumn{3}{c|}{Visual defects} & \multicolumn{6}{c}{Linguistic indicators} \\ \hline
        &  & {\scriptsize \%Part} & {\scriptsize \%Abs} & {\scriptsize \%Co-occ} & {\scriptsize \%Sim} & {\scriptsize \%Occl} & {\scriptsize \%Parts} & {\scriptsize \%Atyp}  & {\scriptsize \%Mod} & {\scriptsize \%Prep}  & {\scriptsize \%Non-lit} & {\scriptsize \%Sense} & {\scriptsize \%Named} & {\scriptsize \%Beyond} \\ \hline
        %\multicolumn{14}{|c|}{SBUCaps (S)}\\\hline
        \multirow{2}{1.2cm}{SBUCaps} & VEIL-Same Dataset &  \textbf{85.0} & \textbf{94.7} & \textbf{87.0} & \textbf{80.0} & \textbf{81.1} & \textbf{90.6} & \textbf{87.2} & \textbf{95.2} & \textbf{93.9} & \textbf{90.6} & \textbf{100.0} & \textbf{100.0} & \textbf{88.8} \\ 
        % VEIL-Cross Dataset & 45.2 & 72.3 & 67.5 & 63.7 & 55 & 60.4 & 65.6 & 47.8 & 76.6 & 77.5 & 75.6 & 91.6 & 70.8 & 52.8 \\ \hline
        & LocalCLIP-E  & 51.5 & 80.7 & 71.3 & 70.0 & 52.7 & 52.1 & 65.6 & 63.8 & 70.6 & 82.9 & 96.2 & 62.5 & 82.4 \\ \hline
        %\multicolumn{14}{|c|}{RedCaps (R)}\\\hline
        \multirow{2}{1.2cm}{RedCaps} & VEIL-Same Dataset  & \textbf{91.7} & 74.1 & \textbf{71.4} & \textbf{85.7} & \textbf{83.3} & \textbf{89.0} & \textbf{68.3} & \textbf{74.8} & \textbf{90.0} & 66.7 & \textbf{88.9} & 80.9 & 76.3 \\ 
        % VEIL-Cross Dataset & 54.1 & 68.3 & 61.5 & 50 & 45.2 & 58.3 & 74.2 & 50 & 69.2 & 63.3 & 33.3 & 72.2 & 71.4 & 60.4 \\ \hline
        & LocalCLIP-E  & 52.8 & \textbf{78.4} & 40.0 & 38.1 & 47.0 & 45.0 & 23.2 & 68.4 & 63.3 & \textbf{70.8} & 70.6 & \textbf{90.0} & \textbf{76.7} \\ \hline
        %\multicolumn{14}{|c|}{Conceptual Captions (CC)}\\\hline
        \multirow{2}{1.2cm}{CC} & VEIL-Same Dataset  & \textbf{60.6} & 83.0 & \textbf{81.2} & 55.0 & \textbf{54.9} & \textbf{53.6} & \textbf{56.3} & 64.2 & \textbf{73.7} & 81.7 & \textbf{100.0} & - & 77.4 \\ 
        % VEIL-Cross Dataset & 61.6 & 42.8 & 64.1 & 67.4 & 45 & 42.2 & 50 & 65.6 & 76.7 & 54 & 63.3 & 87.5 & - & 45.4 \\ \hline
        & LocalCLIP-E  & 45.0 & \textbf{89.1} & 74.9 & \textbf{57.5} & 49.9 & 50.0 & 24.1 & \textbf{73.3} & 63.9 & \textbf{91.7} & \textbf{100.0} & - & \textbf{86.8} \\ \hline
    \end{tabular}
    }
    \end{center}
    \caption{
    %Vetting 
    VAEL recall on CLaN. Bold indicates best performance per column/dataset. We omit named entity results for CC as it substitutes them with predefined categories (e.g. person, org.).}
    %Top = SBUCaps, middle = RedCaps, bottom = CC.}
    \label{tab:vetting_analysis}
\end{table*}


\subsection{Extracted Label Vetting Evaluation}

Table \ref{tab:direct_eval} shows the F1 score which combines the precision and recall of their vetting (shown separately in supp).  
Most language-based methods improve or maintain the F1 score of No Vetting, even though it has perfect recall, except for Accept Descriptive/Narrative.

Rule-based methods and Cap2Det perform  strongly, but are outperformed by both VEIL-Same Dataset (trained and tested on the same dataset, without a special token) and VEIL-Cross Dataset (trained on a different dataset than that shown in the column; we show the best cross-dataset result). 

VEIL-Cross Dataset outperforms language-based approaches,
%in precision (in supp) and F1, 
showing VEIL's generalization potential, except COCO where Cap2Det does slightly better.

Image-and-language-conditioned approaches (Global CLIP/CLIP-E) make label decisions based on the overall caption, so if it contains certain language, it can affect the alignment even if the object is actually visually present. Table \ref{tab:direct_eval} shows these methods obtain low F1 scores.
%; while they improve precision compared to No Vetting (see supp), recall suffers terribly.

Among image-based approaches for label vetting, we observe that Local CLIP benefits significantly from using an ensemble of prompts compared to Global CLIP; ensembling is well documented in improving zero-shot image recognition in prior work \cite{Radford2021LearningTV}. 
Reject Large Loss has the strongest F1 score among the image-based methods, and in supp we show it has strong recall but limited precision improvement over No Vetting, indicating the presence of false positives that do not lead to a large loss. 



\textbf{Vetting performance on CLaN.} We hypothesize that LocalCLIP-E would do well at vetting VAELs explained by linguistic cues like non-literal and beyond the image as they are likely to have little to no visual similarity with the extracted category representation.
%, which would be hard to ground. 
We also hypothesize that VEIL would do better than LocalCLIP-E at vetting VAELs that are noun modifiers or in prepositional phrases.
%, and by extension share similar objects/contexts with the extracted but absent category. These
Further, similar context can sometimes be explained by linguistic cues like noun modifiers and prepositional phrases which can be easily picked up from the caption, but LocalCLIP-E may be oblivious to them differing from the true VAEL category. 
We evaluate these hypotheses on the CLaN dataset in Table \ref{tab:vetting_analysis}. We omit ``visible'' VAEL samples as there are pseudo-label errors, and the ``past'' linguistic indicator due to too few samples. 
We find that VEIL vets truly absent objects for SBUCaps much better than LocalCLIP-E, and comparably for RedCaps or CC. It vets partially visible objects better than LocalCLIP-E by a significant margin; these can be harmful in WSOD which is already prone to part domination \cite{ren2020instance}.
VEIL also recognizes that similar context to, rather than the actual VAEL category, are present.
%Since VEIL performs better at vetting VAELs from partially visible objects, it 
VEIL performs better at vetting visible objects that have visual defects
%occluded or have missing key parts, 
which can be mentioned in caption context 
%. VEIL's ability to vet atypical objects can be explained from (1) caption context indicating an atypical object 
(``acryllic illustration of \underline{dog}"), and
CLIP being trained on 400M diverse images with atypical objects and other defects. 
%where objects would have high visual appearance diversity, including atypical instances.
As expected, we find that for all datasets, VEIL vets VAELs from prepositional phrases better than LocalCLIP-E, and noun modifiers for SBUCaps and RedCaps. LocalCLIP-E does better on ``beyond the image" and non-literal VAELs, except on SBUCaps, where VEIL still excels.

\begin{table}[]
    \centering
    \small
    \begin{tabular}{p{60pt}|p{52pt}|c|c}
    \hline
    Method & Train Dataset & Prec/Rec & F1 \\
    \hline
    No Vetting & - & 0.463 / 1.000 & 0.633 \\ 
    VEIL & SBUCaps & 0.828 / 0.791 & 0.809 \\ \hline
    VEIL & RedCaps (R) & 0.668 / 0.759 & 0.710 \\
    VEIL & CC & 0.585 / 0.846 & 0.692 \\
    VEIL & R, CC & 0.689 / 0.722 & 0.705 \\
    VEIL$_{\text{ST}}$ & R, CC & 0.649 / 0.797 & 0.716 \\ \hline
    % \hline
    % Alignment Model + CLIP Ensemble (max) & RedCaps, WIT \cite{Radford2021LearningTV} & 0.617 / 0.928 & 0.741 \\
    % \hline
    %  Alignment Model + CLIP Ensemble (min) & RedCaps, WIT \cite{Radford2021LearningTV} & 0.814 / 0.676 & 0.738 \\
    % \hline
    LCLIP-E & WIT  & 0.708 / 0.820 & 0.760 \\
    VEIL+LCLIP-E & R,CC,WIT& 0.733 / 0.848 & 0.786 \\ \hline
    \end{tabular}

    \caption{Source generalization of VEIL; vetting on SBUCaps. LCLIP-E is LocalCLIP-E. CLIP is trained on WIT.} %\cite{Radford2021LearningTV}.}
    % We can improve performance of alignment model by using a special token for visual presence prediction. Furthermore, ensembling with CLIP improves both precision and recall leading to a 2.6 pt increase in the F1 score; the ensemble averages the GMM alignment probability and visual presence probability from the VEIL model. compared to CLIP. Incorporating more datasets improves precision at the cost of recall, however training on multiple datasets with a special token improves recall (*) data used for pretrained CLIP model.}
    \label{tab:cross_dataset_direct_eval}
\end{table}




\begin{table}[]
    \centering
    \small
    \begin{tabular}{c|c|c}\hline
Method  & Prec/Rec & F1 \\\hline
No Vetting & 0.323 / 1.000 & 0.488 \\\hline
ID & 0.651 / 0.656  & 0.654 \\\hline
OOD & 0.585 / 0.556 & 0.570 \\\hline
    \end{tabular}
    \caption{VEIL category generalization on SBUCaps-ID. }
    \label{tab:cross_category}
\end{table}

\begin{table}[t] %[h]
\begin{center}
\small
\begin{tabular}{p{1.15in}|p{0.60in}|p{0.60in}|p{0.3in}}
\hline
\textbf{Method (Train Data Size in Thousands)} &\textbf{VOC Det. mAP} ($\Delta$) &\textbf{VOC Rec. mP} ($\Delta$) & \textbf{COCO 0.5:0.95 mAP} \\
\hline
No Vetting  (19) & 16.0 \textcolor{red}{(-45\%)} & 58.0 \textcolor{red}{(-15\%)} & 1.9 \\
GT* (17) & 8.2 \textcolor{red}{(-72\%)} & \underline{77.8} \textcolor{green}{(13\%)} & 1.2\\
Large Loss \cite{Kim2022LargeLM} (19) & 18.1 \textcolor{red}{(-38\%)} &  63.2 \textcolor{red}{(-8\%)} & 1.7 \\
LocalCLIP-E \cite{Radford2021LearningTV} (18) & 11.0 \textcolor{red}{(-62\%)} & \textbf{81.9} \textcolor{green}{(19\%)} & 1.4\\
VEIL$_{\text{ST}}$-R,CC (18) & \underline{26.8} \textcolor{red}{(-8\%)} & 61.2 \textcolor{red}{(-11\%)} & \underline{2.9} \\
VEIL-SBUCaps (16) & \textbf{29.1} (--) & 68.3 (--) & \textbf{3.1} \\
\hline
\end{tabular}
\end{center}
\caption{Impact of vetting on WSOD performance on VOC-07 and COCO-14 datasets. There is a significant difference in detection and recognition on VOC-07 illustrated by $\Delta$, relative performance change w.r.t. VEIL-SBUCaps on the same column. This highlights that VEIL variants filter out labels harmful to localization. (GT*) directly vets labels using the pretrained object detectors which were used to train VEIL.} 
\label{tab:det_v_recognition}
\end{table}


%\subsection{VEIL Generalizability}
%In addition to bootstrapping pseudo-labeled cleanliness data on the same source, we also test VEIL's ability to generalize to unseen sources and categories. Table \ref{tab:direct_eval} shows it can generalize between datasets. Across the language-conditioned approaches, VEIL-Cross Dataset exceeds F1 on most datasets except for CC and COCO, where it is still competitive. In supp, we show increased precision over all approaches. 

\textbf{Cross dataset source generalization.}
We train VEIL on one dataset (or multiple) and evaluate on an unseen target. We find that combining multiple sources improves precision in Table \ref{tab:cross_dataset_direct_eval}. To better utilize caption context, we test VEIL$_{\text{ST}}$ which predicts visual presence using a special token {\tt [EM\_LABEL]}. We find that this improves F1 performance. Lastly, we try ensembling by averaging predictions between LocalCLIP-E and VEIL-Cross Dataset, and find that its precision and recall is highest among the VEIL variants and LocalCLIP-E. This means that VEIL and LocalCLIP-E can be used together. There is still a significant gap between VEIL-Same Dataset and even the ensembled model in terms of precision and F1. We leave improving source generalizability to future research.

\textbf{Cross-category generalization.} We define an in-domain category set (ID) of 20 randomly picked categories from COCO \cite{Lin2014MicrosoftCC}, and an out-of-domain category set (OOD) consisting of the 60 remaining categories. We restrict the labels using these limited category sets and create two train subsets, ID and OOD from SBUCaps \textit{train} and one ID test subset from SBUCaps \textit{test}. %We use the VEIL variants %as described in Section \ref{sec:veil}, 
We find that transferring VEIL-OOD to unseen categories improves F1 score compared to no vetting as shown in Table \ref{tab:cross_category}. %When we trained with the special token variant, however we found little improvement so we left it out.
We hypothesize training on more categories could improve category generalization, but leave further experiments to future research.


\subsection{Impact on Weakly Supervised Object Detection}

%\textbf{Comparison between High Performing Methods in Extracted Label Vetting}
We select the most promising vetting methods from the previous section and use them to vet labels from the SBUCaps \textit{test} split since CLIP-based and VEIL-based methods use the train set for threshold and model training respectively. 
We show two different VEIL methods, VEIL-SBUCaps and VEIL$_{\text{ST}}$-RedCaps,CC. Both vet labels from SBUCaps and use them to train WSOD, but the vetting method is trained on either (1) SBUCaps or (2) RedCaps and CC, using the special token. We show both methods to demonstrate the generalizability of the vetting model.
Note that Large Loss Matters \cite{Kim2022LargeLM} has been relaxed to also \textit{correct} visually absent extracted labels, instead of only unmentioned but present objects (false negatives).
After vetting, we remove any images without labels and since category distribution follows a long-tail distribution, we apply weighted sampling.
We train MIST \cite{ren2020instance} for 50K iterations with a batch size of 8 for each method. 

We find that VEIL-SBUCaps performs the best as shown in Table \ref{tab:det_v_recognition}.
In particular, it boosts the detection performance of No Vetting by 9.3\% absolute and 29.8\% relative gain (40.5/31.2\% mAP) on VOC-07 and by 35\% relative gain (10.4/7.7\% mAP) on COCO.
Interestingly, VEIL-SBUCaps and VEIL$_{\text{ST}}$-Redcaps,CC both seem to have a similar performance improvement, despite VEIL$_{\text{ST}}$-Redcaps,CC (best VEIL cross-dataset result on SBUCaps) having poorer performance than Local CLIP-E in Table \ref{tab:cross_dataset_direct_eval}. Additionally, directly using the pretrained object recognition model (used to produce visual presence targets for VEIL) predictions to vet (GT* method in the table) performs worse than VEIL in both detection and recognition. This suggests VEIL \textbf{generalizes from its bootstrapped data}. % to perform significantly better on weakly supervised object detection. 

Using the CLaN dataset, we observe
%a number of examples with structured noise. 
one type of structured noise found from extracting labels from prepositional phrases, specifically where images were taken inside vehicles. We hypothesize such structured noise would have significant impact on localization for the vehicle objects. We use CorLoc to estimate the localization ability for vehicles in VOC-07 (``aeroplane", `bicycle", ``boat", ``car", ``bus", ``motorbike", ``train"). We observe a CorLoc of 60.2\% and 54.1\% for VEIL-SBUCaps and LocalCLIP-E, respectively. 
%For another supercategory, animal, we observe a smaller improvement (57.9 vs 56.8) from LocalCLIP-E and VEIL-SBUCaps. 
This shows structured noise can have strong impact on localization.
\begin{figure}[t]
    \centering
    \includegraphics[width=.7\linewidth]{pics/scale.pdf}
    \caption[]{\fix{Distributions of MIG scores and reconstruction errors for low-dimensional space (blue) and high-dimensional space (green). The points in the bottom right have a better balance of disentanglement and reconstruction.}}
    \label{fig:scale}
\end{figure}

\textbf{Impact of scale on WSOD.} To assess the impact of scaling the noisy training set, we progressively sampled the held-out RedCaps dataset in increments of 50K samples up to a total of 200K samples. For each scale, we train two WSOD models with weighted sampling using the unfiltered samples and those vetted with VEIL-SBUCaps,CC. The 200K-sample model trained for 120K iterations, and subsets trained for iterations proportional to their sample size. Figure \ref{fig:scale} shows consistent improvement with vetting as the sample size scales. The non-vetted model's performance declines after 150K samples; more unvetted samples has diminishing or worse, negative returns. This indicates that vetting can adapt to scale better even when VEIL is trained on other datasets. The trend suggests that vetting will continue outperforming no-vetting even when dataset sizes increase.
% anything about why the performance is lower? Perhaps our label and visual noise dataset might show that the VAELs in Redcaps are worse. However, that says nothing about the samples that are included...

\begin{table}[]
    \centering
    \small
    \begin{tabular}{c|c|c|c|c}\hline
        Clean Labels & Noisy Labels & WS & Vetting & $\text{mAP}_{50}$\\\hline
         % & \checkmark &  &  & 16.67 \\\hline
        \checkmark &  &  & n/a & 43.48 \\\hline
        $\checkmark$ & \checkmark &  &  & 42.06 \\\hline
        $\checkmark$ & $\checkmark$ &  & $\checkmark$ & 51.31 \\\hline
        $\checkmark$ & $\checkmark$ & $\checkmark$  &$ \checkmark$ & 54.76 \\\hline

    \end{tabular}
    \caption{Mixed supervision from clean (VOC-07 trainval) and noisy labels (SBUCaps).
    Eval on VOC-07 test.}
    %WS stands for weighted sampling.} %Note vetting can only be applied on labels extracted from captions, hence are applied only to the noisy labels column.}
    \label{tab:mixed_supervision}
\end{table}

\textbf{Effect of mixing clean and noisy samples for WSOD.} We study how vetting impacts a setting where labels are drawn from both annotated image-level labels from 5K VOC-07 train-val \cite{Everingham2010ThePV} (clean) and 50K in-the-wild SBUCaps \cite{Ordonez2011Im2TextDI} captions (noisy). In Table \ref{tab:mixed_supervision} we observe that adding noisy supervision to clean supervision actually hurts performance  %(-3.2\%) 
compared to only using clean supervision. After vetting the labels extracted from SBUCaps  \cite{Ordonez2011Im2TextDI} using VEIL-SBUCaps, we observe that the model sees a 17.9\% relative improvement (51.31/43.48\% mAP) to using only clean supervision from VOC-07. We see further improvements when applying weighted sampling to the added, class imbalanced data (54.76/51.31\% mAP).

% % \begin{figure}
%     \centering   
%     %\includegraphics[scale=0.5]{iccv2023AuthorKit/figures/images/prec_train_subset.png}  
%     \includegraphics[scale=0.3]{iccv2023AuthorKit/figures/images/f1_train_subset.png}
%     \includegraphics[scale=0.3]{iccv2023AuthorKit/figures/images/cross_cat_black.png}
%     \caption{This figure shows the impact of pseudo-labeled data size for VEIL-SBUCaps on its performance on SBUCaps-Test VPEL Detection. VEIL only needs 10\% of the data to beat No Vetting,
%     %(XX vs 0.633 F1), 
%     and 50\% of the labeled data (80K samples) to beat CLIP.} %This indicates that we only need a small amount of pseodo-labeled data to beat CLIP in terms of precision and slightly more to improve recall and therefore F1.}
%     \label{fig:train_subset}
% \end{figure}

\begin{figure}
  \centering
  %\subfigure[This figure shows the impact of pseudo-labeled data size for VEIL-SBUCaps on its performance on SBUCaps-Test VPEL Detection. VEIL only needs 10\% of the data to beat No Vetting,
    %(XX vs 0.633 F1),     and 50\% of the labeled data (80K samples) to beat CLIP.]{
    \includegraphics[width=0.2\textwidth]{iccv2023AuthorKit/figures/images/limited_train_data_bigger_2.png}
    %} \subfigure[Caption for subfigure 2]{
    \includegraphics[width=0.2\textwidth]{}
    %}
  \caption{(Left) The impact of pseudo-labeled data size for VEIL-SBUCaps on its performance on SBUCaps-Test vetting. 
  %VEIL only needs 10\% of the data to beat No Vetting, and 50\% of the labeled data (80K samples) to beat CLIP.}
  (Right) Cross-category generalization, comparing of WSOD performance on VOC-07 mAP (over 7 overlapping ID categories) when vetting train data, SBUCaps-Test ID, using VEIL-ID and VEIL-OOD. Dashed line indicates mean AP and solid line is median AP.}
\label{fig:train_subset}
\end{figure}

\textbf{Conclusion.} We showed visually absent extracted labels are common in in-the-wild datasets.
%than datasets commonly used for detection pretraining. 
We proposed VEIL which uses language context to infer whether mentioned objects are visually present. It outperforms other vetting strategies, generalizes across datasets and categories, and its benefits persist when adding noisy to clean data. 
%We posit that VEIL can be used to effectively sample localization critical samples and help WSOD use noisier in-the-wild image-caption data.
% \textbf{Ethics note.}




% \bibliographystyle{ieee_fullname}
\bibliography{aaai24, egbib}

Below we first briefly describe the selected models and then their implementation details during pre-training.

% Traditional convolutional action recognition networks before 2017 are mostly built to process single frame or multiple consecutive frames; however, such simple structures overlook the importance of long-range temporal context in action recognition, which somehow underestimates the intrinsic temporal information within videos. 
Temporal segment networks (TSN) proposes segment-based sampling to learn temporal information across frames. 
Specifically, in TSN, a video is evenly divided into several temporal segments, which one random frame is sampled from. 
Then the output from each segment will be aggregated via pooling to obtain the final prediction. 
Temporal Shift Module (TSM) shifts feature channels along the temporal axis, which facilitates information exchanged among neighboring frames. 
It can be plug-and-played in 2D networks to enable stronger temporal modeling at zero computation and zero parameters.
Thus, TSM can achieve the performance of heavy 3D CNNs while maintaining the efficiency of 2D CNNs.
% TSM introduces stronger temporal learning capacity to 2D networks while maintaining light-weight. 

Inflated 3D ConvNet (I3D) is designed to bootstrap from the corresponding 2D network since (1) the architecture of 2D network is well designed and (2) the  weights of 2D network is well pre-trained, e.g., Inception~\cite{inception} $\rightarrow$ Inception-I3D~\cite{carreira2017quo}. 
% utilize pre-trained weights from the corresponding 2D network since these 2D weights have been well-designed and trained to perceive visual concepts.
I3D initializes its 3D kernels by duplicating the 2D ones along the temporal dimension, which helps the convergence of 3D CNNs. 
Inspired by~\cite{vaswani2017attention}, non-local networks (NL) adapts the non-local operation (i.e., self-attention~\cite{vaswani2017attention}) in its building block to model long-range dependency.
For video action recognition, its goal is to relate the same object, or person-object interaction within a distant time interval in videos.
Similar to TSM, non-local block is compatible to most convolutional networks.


TimeSformer is a pure transformer-based model, which is an extension of ViT~\cite{dosovitskiy2020image} to the spatiotemporal space. 
Given the quadratic complexity of self-attention, TimeSformer compares several attention strategies when considering temporal dimention in videos.
Finally, TimeSformer introduces the divided space-time attention to greatly reduce the computation burden but achieves promising results.
% on most video action recognition datasets. 
% This structure shows both effectiveness and efficiency in their reported results. 
Continuing this modeling shift from CNNs to Transformers, VideoSwin extends Swin Transformer~\cite{liu2021swin} by adding the inductive bias of locality in video transformers. 
Simply speaking, it adapts the idea of 2D shifted window self-attention to 3D space, which results in better speed-accuracy trade-off compared to previous approaches~\cite{bertasius2021space,arnab2021vivit}.
% Similarly, VideoSwin is an extension of Swin Transformer~\cite{liu2021swin}, by adapting the 2D shifted window self-attention to 3D.
% And shifted window ensure the connection across distant regions in the spatiotemporal tensors.


\begin{figure}[t]
\centering
    \includegraphics[width=8cm]{figures/radar_new.pdf}
    \caption{The rank of the averaged performance within different data domains for the 6 models in different settings. The most outside in these radar images means the highest performance. For each domain, we average the top-1 accuracy as the scores in finetuning and average the top-1 accuracy of 16-shot results in few-shot learning. Complete results are shown in Table~\ref{tab:finetune} and Figure~\ref{fewshot}.}
    \label{radar}
\end{figure}
%Below we first briefly describe the selected models and then their implementation details during pre-training.

% Traditional convolutional action recognition networks before 2017 are mostly built to process single frame or multiple consecutive frames; however, such simple structures overlook the importance of long-range temporal context in action recognition, which somehow underestimates the intrinsic temporal information within videos. 
Temporal segment networks (TSN) proposes segment-based sampling to learn temporal information across frames. 
Specifically, in TSN, a video is evenly divided into several temporal segments, which one random frame is sampled from. 
Then the output from each segment will be aggregated via pooling to obtain the final prediction. 
Temporal Shift Module (TSM) shifts feature channels along the temporal axis, which facilitates information exchanged among neighboring frames. 
It can be plug-and-played in 2D networks to enable stronger temporal modeling at zero computation and zero parameters.
Thus, TSM can achieve the performance of heavy 3D CNNs while maintaining the efficiency of 2D CNNs.
% TSM introduces stronger temporal learning capacity to 2D networks while maintaining light-weight. 

Inflated 3D ConvNet (I3D) is designed to bootstrap from the corresponding 2D network since (1) the architecture of 2D network is well designed and (2) the  weights of 2D network is well pre-trained, e.g., Inception~\cite{inception} $\rightarrow$ Inception-I3D~\cite{carreira2017quo}. 
% utilize pre-trained weights from the corresponding 2D network since these 2D weights have been well-designed and trained to perceive visual concepts.
I3D initializes its 3D kernels by duplicating the 2D ones along the temporal dimension, which helps the convergence of 3D CNNs. 
Inspired by~\cite{vaswani2017attention}, non-local networks (NL) adapts the non-local operation (i.e., self-attention~\cite{vaswani2017attention}) in its building block to model long-range dependency.
For video action recognition, its goal is to relate the same object, or person-object interaction within a distant time interval in videos.
Similar to TSM, non-local block is compatible to most convolutional networks.


TimeSformer is a pure transformer-based model, which is an extension of ViT~\cite{dosovitskiy2020image} to the spatiotemporal space. 
Given the quadratic complexity of self-attention, TimeSformer compares several attention strategies when considering temporal dimention in videos.
Finally, TimeSformer introduces the divided space-time attention to greatly reduce the computation burden but achieves promising results.
% on most video action recognition datasets. 
% This structure shows both effectiveness and efficiency in their reported results. 
Continuing this modeling shift from CNNs to Transformers, VideoSwin extends Swin Transformer~\cite{liu2021swin} by adding the inductive bias of locality in video transformers. 
Simply speaking, it adapts the idea of 2D shifted window self-attention to 3D space, which results in better speed-accuracy trade-off compared to previous approaches~\cite{bertasius2021space,arnab2021vivit}.
% Similarly, VideoSwin is an extension of Swin Transformer~\cite{liu2021swin}, by adapting the 2D shifted window self-attention to 3D.
% And shifted window ensure the connection across distant regions in the spatiotemporal tensors.


\begin{figure}[t]
\centering
    \includegraphics[width=8cm]{figures/radar_new.pdf}
    \caption{The rank of the averaged performance within different data domains for the 6 models in different settings. The most outside in these radar images means the highest performance. For each domain, we average the top-1 accuracy as the scores in finetuning and average the top-1 accuracy of 16-shot results in few-shot learning. Complete results are shown in Table~\ref{tab:finetune} and Figure~\ref{fewshot}.}
    \label{radar}
\end{figure}
\end{document}
