% \begin{table}[]
%     \centering
%     \begin{tabular}{|c|c|c|} \hline
%         Dataset & Narrative (\%) \\\hline
%         COCO & 5 \\\hline
%         CC & 33 \\\hline
%         SBUCaps & 37 \\\hline
%         RedCaps & 62  \\\hline
%     \end{tabular}
%     \caption{From sample of 100 captions, we annotate 100 with descriptive or narrative labels. We find that COCO is dominated by descriptive captions and the in-the-wild datasets are more diverse, containing narrative-like captions.}
%     \label{tab:descriptiveness_in_wild_annot}
% \end{table}

\begin{table}[]
    \centering
    \begin{tabular}{|c|c|c|c|}
    \hline
        Metric & DMR & NMR & Accuracy \\ \hline
        COCO & 0.06 & 0.60 & 0.91 \\ \hline
        CC & 0.15 & 0.64 & 0.69 \\ \hline
        SBUCaps & 0.24 & 0.60 & 0.63 \\ \hline
        RedCaps & 0.53 & 0.32 & 0.60 \\ \hline
    \end{tabular}
    \caption{Using the annotated sets, we evaluate the ability of the descriptiveness classifier to generalize to in-the-wild captions. Except for RedCaps, the part-of-speech based descriptiveness classifier is unable generalize outside of the VIST descriptive/narrative domain. We report misclassification rate for the descriptive captions (DMR) and the narrative captions (NMR), defined in the text.}
    \label{tab:desc_classifier_score_in_wild_annot}
\end{table}