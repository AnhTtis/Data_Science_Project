\subsection{Broader Label Noise Patterns}
\label{sec:taxonomy}

\begin{figure}[t] %[h]   
    \includegraphics[width=1\linewidth]{iccv2023AuthorKit/figures/images/label_tax_font_larger2.png}
    %\includegraphics[scale=0.70] {iccv2023AuthorKit/figures/images/label_noise_taxonomy3.png}
    \caption{Label noise taxonomy outlining common reasons for visual absence in extracted labels from in-the-wild captions.}
    \label{fig:label_noise_taxonomy}
\end{figure}

%Objects mentioned in human-annotated captions are usually present in the image (see precision on COCO and VIST-DII in supp.); however, real world captions contain a range of expressions that mention objects not shown in the corresponding image. 
We have identified a taxonomy of label noise found in in-the-wild captions, illustrated in Figure \ref{fig:label_noise_taxonomy}. We focus on 
%with reasons as children nodes for each broad category of noise: 
visually absent, extracted labels (VAELs).
%and visually present, extracted labels (VPEL), with the former being our focus.
%Mentions that are completely missing will now be referred to as visually absent extracted labels (VPELs) and are the main focus of this paper. 
We find that both descriptive and narrative captions mention objects that are semantically unrelated to the category label. 
They can be part of a larger named entity e.g. ``Super \textbf{Bowl} Party" or part of a noun phrase which can either be related to the category e.g. ``\textbf{bottle} opener", or unrelated e.g. ``hot \textbf{dog}" or ``\textbf{orange}'' the color instead of the fruit. Other VAELs are narrative artifacts providing personal context for the image and representing the colorful ways people express themselves. These include non-literal mentions (metaphors and similies), prepositional phrases e.g. ``Left a lighter in my \textbf{car}", describing the past e.g. ``My chest hurts and I can't stop crying. Hugo was run over by a \textbf{car} yesterday", or rarer, negations e.g. ``We did not spot a \textbf{tiger} in our safari". We believe these VAELs can be learned implicitly if a transformer-based model is provided with a loose proxy for visual presence or absence labels, motivating VEIL.
%Not all VAEL examples are equally obvious to recognize without an image reference; this is due to ambiguous language that indicates that some object mentions may be missing from the image but its not obvious ("Sometimes there's a \textbf{dog} under the trucks. I'm on the hunt to photograph him and a \textbf{monkey} or too"). In addition, not all objects in an image are recognizable without an caption prior; people tend to post pictures of their pets in costumes which obfuscates what animal it really is. While we don't explore bad examples of objects in this paper, an emergent behaviour of our model is rejecting atypical examples. We leave exploring this to future work.

% --- OLD TEXT

% Hard examples
% While this is not tackled by this paper, captions can also indicate that the extracted labels would correspond to a hard example: deformed objects (cut up/deteriorated/etc), mostly occluded (most key parts are missing from the image), atypical instances (cat dressed up as something else), out of real world domain (like clipart) – hard where key parts of the object are missing, small objects or objects far off in a distance, close ups where key parts are missing. Other examples might be edge cases: food trucks → can decrease the difference between cars and trucks. Although vans are considered a truck with an enclosed cargo space (WordNet).
% Tractor -> a truck, but with only a cab
% So a truck consists of two parts, a cab and/or a cargo, but even if you have only one part then it is a specific instance of a truck
% Old example
% Antique trucks - [“old”, “antique”, “year mentioned”]

% Chat GPT Rewrite
% Objects mentioned in human-annotated captions are usually present in the image; however, real-world captions can describe a wide range of expressions that may mention objects that are not shown in the corresponding image. In this paper, we focus on a type of label noise called visually absent extracted labels (VPELs), which refers to mentions of objects that are completely missing from the image. There are several reasons why an mentioned object might be visually absent, such as if the object is semantically unrelated to the image or is absent due to narrative artifacts. These semantically unrelated object mentions can be part of a larger named entity, such as "super bowl party," or part of a noun phrase, such as "bottle opener." Other VPELs can be caused by narrative artifacts, such as non-literal mentions (e.g. metaphors and similies), prepositional phrases (e.g. "left a lighter in my car"), descriptions of the past (e.g. "my chest hurts and I can't stop crying. Hugo was run over by a car yesterday"), or negations (e.g. "no tiger"). Additionally, there may be instances that are unrecognizable without the caption, such as objects that are deformed, occluded, atypical, or out of the real-world domain (e.g. clipart).

% It is worth noting that there are also cases where the extracted labels correspond to hard examples, such as objects that are small or far off in the distance, or close-ups where key parts are missing. There may also be edge cases, such as food trucks, which can blur the line between cars and trucks, or tractors, which are considered a specific type of truck. Similarly, antique trucks may be described with words like "old" or "antique" and may include a mention of the year.




