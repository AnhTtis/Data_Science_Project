%%
%% This is file `sample-acmtog.tex',
%% generated with the docstrip utility.
%%
%% The original source files were:
%%
%% samples.dtx  (with options: `acmtog')
%% 
%% IMPORTANT NOTICE:
%% 
%% For the copyright see the source file.
%% 
%% Any modified versions of this file must be renamed
%% with new filenames distinct from sample-acmtog.tex.
%% 
%% For distribution of the original source see the terms
%% for copying and modification in the file samples.dtx.
%% 
%% This generated file may be distributed as long as the
%% original source files, as listed above, are part of the
%% same distribution. (The sources need not necessarily be
%% in the same archive or directory.)
%%
%% Commands for TeXCount
%TC:macro \cite [option:text,text]
%TC:macro \citep [option:text,text]
%TC:macro \citet [option:text,text]
%TC:envir table 0 1
%TC:envir table* 0 1
%TC:envir tabular [ignore] word
%TC:envir displaymath 0 word
%TC:envir math 0 word
%TC:envir comment 0 0
%%
%%
%% The first command in your LaTeX source must be the \documentclass command.
\documentclass[5p,sort&compress,hidelinks]{article}
%\documentclass[sigchi]{acmart}
%% NOTE that a single column version is required for 
%% submission and peer review. This can be done by changing
%% the \doucmentclass[...]{acmart} in this template to 
%% \documentclass[manuscript,screen]{acmart}
%% 
%% To ensure 100% compatibility, please check the white list of
%% approved LaTeX packages to be used with the Master Article Template at
%% https://www.acm.org/publications/taps/whitelist-of-latex-packages 
%% before creating your document. The white list page provides 
%% information on how to submit additional LaTeX packages for 
%% review and adoption.
%% Fonts used in the template cannot be substituted; margin 
%% adjustments are not allowed.

% packages accepted according to https://authors.acm.org/proceedings/production-information/accepted-latex-packages
\usepackage{subfig}
\usepackage{hyperref}
\usepackage{listings}
\hyphenation{asyn-chron-ously}

%\usepackage{todonotes}
\usepackage{arxiv}

\usepackage{graphicx}
\usepackage{booktabs}
%%
%% \BibTeX command to typeset BibTeX logo in the docs
\AtBeginDocument{%
  \providecommand\BibTeX{{%
    Bib\TeX}}}

%% Rights management information.  This information is sent to you
%% when you complete the rights form.  These commands have SAMPLE
%% values in them; it is your responsibility as an author to replace
%% the commands and values with those provided to you when you
%% complete the rights form.
%\setcopyright{acmcopyright}
%\copyrightyear{2023}
%\acmYear{2023}
%\acmDOI{XXXXXXX.XXXXXXX}

%\copyrightyear{2023}
%\acmYear{2023}
%\setcopyright{acmlicensed}\acmConference[IWOCL '23]{International Workshop on
%OpenCL}{April 18--20, 2023}{Cambridge, United Kingdom}
%\acmBooktitle{International Workshop on OpenCL (IWOCL '23), April 18--20, 2023,
%Cambridge, United Kingdom}
%\acmPrice{15.00}
%\acmDOI{10.1145/3585341.3585354}
%\acmISBN{979-8-4007-0745-2/23/04}

%%
%% Submission ID.
%% Use this when submitting an article to a sponsored event. You'll
%% receive a unique submission ID from the organizers
%% of the event, and this ID should be used as the parameter to this command.
%%\acmSubmissionID{123-A56-BU3}
%% \acmConference[IWOCL '23]{11th International Workshop on OpenCL}{April 18--20, 2023}{Cambridge, UK}
%%
%% For managing citations, it is recommended to use bibliography
%% files in BibTeX format.
%%
%% You can then either use BibTeX with the ACM-Reference-Format style,
%% or BibLaTeX with the acmnumeric or acmauthoryear sytles, that include
%% support for advanced citation of software artefact from the
%% biblatex-software package, also separately available on CTAN.
%%
%% Look at the sample-*-biblatex.tex files for templates showcasing
%% the biblatex styles.
%%

%%
%% The majority of ACM publications use numbered citations and
%% references.  The command \citestyle{authoryear} switches to the
%% "author year" style.
%%
%% If you are preparing content for an event
%% sponsored by ACM SIGGRAPH, you must use the "author year" style of
%% citations and references.
%\citestyle{acmauthoryear}

\usepackage{listings}

\lstset{
  literate={\_}{}{0\discretionary{\_}{}{\_}}%
}

\usepackage{textcomp}
\usepackage{multirow}

\usepackage{xcolor} % Required for specifying colors by name
\definecolor{azure}{rgb}{0.0, 0.5, 1.0}
\definecolor{darkgreen}{rgb}{0.0, 0.5, 0.0}
\definecolor{amaranth}{rgb}{0.9, 0.17, 0.31}
\definecolor{cadetgrey}{rgb}{0.57, 0.64, 0.69}
\definecolor{aureolin}{rgb}{0.99, 0.93, 0.0}
\definecolor{amber}{rgb}{1.0, 0.49, 0.0}

\newcommand{\orcid}[1]{\href{https://orcid.org/#1}{\includegraphics[height=10pt]{orcid}}}

%%
%% end of the preamble, start of the body of the document source.
\begin{document}

%%
%% The "title" command has an optional parameter,
%% allowing the author to define a "short title" to be used in page headers.
%\title{SYCL, HPX, Kokkos and Stellar Mergers: Reaching for the Stars with SYCL and Tasks}
%\title{No Time to Wait on Results: \\ Integrating Aynchronous SYCL Kernels into the HPX Taskgraph}
%\title{No Time to Wait for Completion: \\ Integrating Aynchronous SYCL Kernels into the HPX Taskgraph}
%\title{Polling over Waiting: Integrating Aynchronous SYCL Kernels into the HPX Taskgraph for Stellar Mergers}

%\title{No Time to Wait on Results: \\ Integrating SYCL with HPX for Stellar Mergers using Kokkos }
%\title{Integrating Task Graphs: Enabling Stellar Mergers with HPX Kokkos and SYCL }
\title{Stellar Mergers with HPX-Kokkos and SYCL: Methods of using \\ an Asynchronous Many-Task Runtime System with SYCL }

% Integrating SYCL with an Asynchronous-Many-Task System (HPX)
% Stellar Mergers as a Benchmark: Integrating SYCL with an Asynchronous-Many-Task Runtime System its Kokkos Integration
% Stellar Mergers with HPX Kokkos and SYCL: Methods of using an Asynchronous-Many-Task Runtime System with SYCL

%%
%% The "author" command and its associated commands are used to define
%% the authors and their affiliations.
%% Of note is the shared affiliation of the first two authors, and the
%% "authornote" and "authornotemark" commands
%% used to denote shared contribution to the research.



\author{Gregor Dai\ss\orcid{0000-0002-0989-5985} \\
University of Stuttgart \\
\AND
        Patrick Diehl\orcid{0000-0003-3922-8419}, Hartmut Kaiser\orcid{0000-0002-8712-2806}\\
        Louisiana State University
        \AND
        Dirk Pfl\"uger\orcid{0000-0002-4360-0212} \\
        University of Stuttgart
}









%%
%% By default, the full list of authors will be used in the page
%% headers. Often, this list is too long, and will overlap
%% other information printed in the page headers. This command allows
%% the author to define a more concise list
%% of authors' names for this purpose.
%\renewcommand{\shortauthors}{Dai\ss~ et al.}

%%
%% The abstract is a short summary of the work to be presented in the
%% article.



%%
%% The code below is generated by the tool at http://dl.acm.org/ccs.cfm.
%% Please copy and paste the code instead of the example below.
%%


%%
%% Keywords. The author(s) should pick words that accurately describe
%% the work being presented. Separate the keywords with commas.


%\received{20 February 2007}
%\received[revised]{12 March 2009}
%\received[accepted]{5 June 2009}

%%
%% This command processes the author and affiliation and title
%% information and builds the first part of the formatted document.
\maketitle



Over the past few years, there has been a significant amount of research focused on studying the ReLU activation function, with the aim of achieving neural network convergence through over-parametrization. However, recent developments in the field of Large Language Models (LLMs) have sparked interest in the use of exponential activation functions, specifically in the attention mechanism.

Mathematically, we define the neural function $F: \R^{d \times m} \times  \mathbb{R}^d \rightarrow \mathbb{R}$ using an exponential activation function. Given a set of data points with labels $\{(x_1, y_1), (x_2, y_2), \dots, (x_n, y_n)\} \subset \mathbb{R}^d \times \mathbb{R}$ where $n$ denotes the number of the data. Here $F(W(t),x)$ can be expressed as $F(W(t),x) := \sum_{r=1}^m a_r \exp(\langle w_r, x \rangle)$, where $m$ represents the number of neurons, and $w_r(t)$ are weights at time $t$. It's standard in literature that $a_r$ are the fixed weights and it's never changed during the training. We initialize the weights $W(0) \in \mathbb{R}^{d \times m}$ with random Gaussian distributions, such that $w_r(0) \sim \mathcal{N}(0, I_d)$ and initialize $a_r$ from random sign distribution for each $r \in [m]$.

Using the gradient descent algorithm, we can find a weight $W(T)$ such that $\| F(W(T), X) - y \|_2 \leq \epsilon$ holds with probability $1-\delta$, where $\epsilon \in (0,0.1)$ and $m = \Omega(n^{2+o(1)}\log(n/\delta))$. To optimize the over-parametrization bound $m$, we employ several tight analysis techniques from previous studies [Song and Yang arXiv 2019, Munteanu, Omlor, Song and Woodruff ICML 2022]. 

 


\keywords{SYCL, Kokkos, HPX, AMT, GPU, CUDA, HIP, SIMD}

\cleardoublepage
\newpage
\twocolumn

\section{Introduction}
\label{sec:introduction}
% \begin{itemize}
%     % Diffusion of FL
%     \item {\st{Diffusion of FL}}
%     % Security threats to FL
%     \item {\st{Security threats to FL with particular focus on model poisoning}}
%     % Limitations of existing countermeasures
%     \item {\st{Current countermeasures (e.g., KRUM) and their limitations}}
%     % Proposed method and its advantages
%     \item {\st{Intuitive description of the proposed method and its difference (i.e., advantages) w.r.t. state of the art}}
%     % Main contributions
%     \item {\st{Summary of the main contributions of this work}}
%     % Paper's structure and organization
%     \item {\st{Paper's structure and organization}}
% \end{itemize}

% Diffusion of FL
Recently, {\em federated learning} (FL) has emerged as the leading paradigm for training distributed, large-scale, and privacy-preserving machine learning (ML) systems~\cite{mcmahan2017googleai,mcmahan2017aistats}. 
The core idea of FL is to allow multiple edge clients to collaboratively train a shared, global model without disclosing their local private training data.
%Specifically, an FL system consists of a central server and many edge clients; 
A typical FL round involves the following steps: {\em(i)} the server randomly picks some clients and sends them the current, global model; {\em(ii)} each selected client locally trains its model with its own private data; then, it sends the resulting local model to the server;\footnote{Whenever we refer to global/local model, we mean global/local model {\em parameters}.} {\em(iii)} the server updates the global model by computing an \emph{aggregation function}, usually the average (FedAvg), on the local models received from clients.
% \begin{enumerate}
%     \item[{\em(i)}] the server sends the current, global model to the clients and appoints some of them for training;
%     \item[{\em(ii)}] each selected client locally trains its copy of the global model with its own private data; then, it sends the resulting local model back to the server;\footnote{Whenever we refer to global/local model, we mean global/local model {\em parameters}.}
%     \item[{\em(iii)}] the server updates the global model by computing an \emph{aggregation function} on the local models received from clients (by default, the average, also referred to as FedAvg~\cite{mcmahan2017aistats}).
% \end{enumerate}
This process goes on until the global model converges. %(e.g., after a certain number of rounds or other similar stopping criteria).
%\\
% The advantages of FL over the traditional, centralized learning paradigm are undoubtedly clear in terms of flexibility/scalability (clients can join/disconnect from the FL network dynamically), network communications (only model weights\footnote{We will use \textit{parameters} and \textit{weights} interchangeably.} are exchanged between clients and server), and privacy (each client's private training data is kept local at the client's end and not uploaded to the server).
\\
% Security threats to FL
%However, the growing adoption of FL also raises security concerns~\cite{costa2022covert}, particularly about its confidentiality, integrity, and availability.
Although its advantages over standard ML, FL also raises security concerns~\cite{costa2022covert}. %, particularly about its confidentiality, integrity, and availability~\cite{costa2022covert}.
% OLD, LONG VERSION
% Indeed, some work deals with privacy leakage that may expose the local data of some clients~\cite{melis2019sp}. 
% A large body of work, instead, investigates attacks that usually aim to detriment the predictive accuracy of the learned global model. For instance, \emph{data poisoning} attacks achieve this goal by letting an adversary pollute the training set of some corrupt FL clients with maliciously crafted examples~\cite{jagielski2018sp}.
% Similarly, in \emph{model poisoning} the attacker attempts to tweak the global model weights~\cite{bhagoji2019pmlr} by directly perturbing the local model's weights of some infected FL clients before these are sent to the central server for aggregation, usually via so-called Byzantine attacks. 
% It turns out that Byzantine model poisoning attacks severely impact standard FedAvg; therefore, more robust aggregation functions must be designed to make FL systems secure.
Here, we focus on \emph{untargeted model poisoning} attacks~\cite{bhagoji2019pmlr}, where an adversary attempts to tweak the global model weights %\footnote{We will use the terms \textit{parameters} and \textit{weights} interchangeably.} 
by directly perturbing the local model's parameters of some infected clients before these are sent to the central server for aggregation.
In doing so, the adversary aims to jeopardize the global model \textit{indiscriminately} at inference time.
Such model poisoning attacks severely impact standard FedAvg; therefore, more robust aggregation functions must be designed to secure FL systems.
\\
% In this paper, we focus on designing a novel robust aggregation scheme at the server's end to contrast the effect of Byzantine model poisoning attacks.
%
% Current countermeasures and their limitations
%Several countermeasures have been proposed in the literature to combat model poisoning attacks on FL systems.
% Some methods use simple statistics more robust than plain average to smooth the impact of malicious updates (e.g., Trimmed Mean and FedMedian~\cite{yin2018icml}). 
% Other defenses implement outlier detection techniques to discard malicious updates from the aggregation performed at the server's end. Those are either based on heuristics (e.g., Krum/Multi-Krum~\cite{blanchard2017nips} and Bulyan~\cite{mhamdi2018pmlr}) or data-driven approaches (e.g., K-means clustering~\cite{shen2016acm} or DnC via spectral analysis~\cite{shejwalkar2021ndss}). 
% Finally, some strategies rely on a centralized ``source of trust'' to spot potential malicious updates (e.g., FLTrust~\cite{cao2020fltrust}).
% Several countermeasures have been proposed in the literature to combat model poisoning attacks on FL systems, i.e., to discard possible malicious local updates from the aggregation performed at the server's end. 
% These techniques range from simple statistics more robust than plain average (e.g., Trimmed Mean and FedMedian~\cite{yin2018icml}) to outlier detection heuristics (e.g., Krum/Multi-Krum~\cite{blanchard2017nips} and Bulyan~\cite{mhamdi2018pmlr}) or data-driven approaches (e.g., spectral analysis via K-means clustering~\cite{shen2016acm} or spectral analysis), or methods based on ``source of trust'' (e.g., FLTrust~\cite{cao2020fltrust}).
% OLD, LONG VERSION
%Several countermeasures have been proposed in the literature to combat Byzantine model poisoning attacks on FL systems.
% Descriptive statistics
% For example, Trimmed Mean and FedMedian aggregate local model updates using more robust statistics than standard average~\cite{yin2018icml}.
%
% % Heuristics for outlier detection
% Many existing Byzantine-resilient strategies implement some outlier detection heuristics to discard the model updates sent by potentially malicious clients from the input of the aggregation function.
% One of the most popular heuristics is Krum~\cite{blanchard2017nips}.
% This strategy tries to mitigate the impact of Byzantine attacks by selecting as a global model the local model with the smallest sum of Euclidean distances to {\em all} the other local models.
% Although powerful, Krum requires the server to know (or, at least, estimate) the number of malicious FL clients upfront, which is generally impossible in a realistic attack scenario. %
% Moreover, Krum may become ineffective for complex, high-dimensional model parameter spaces due to the curse of dimensionality.
% Bulyan~\cite{mhamdi2018pmlr} tries to overcome this issue by combining Krum with a variant of Trimmed Mean.
% % Data-driven outlier detection
% Other strategies use data-driven outlier detection techniques -- e.g., via K-means clustering~\cite{shen2016acm} -- to spot potential malicious local model updates. 
% %For instance, Shen et al. propose to cluster local model updates with K-means and thus identify outliers.
%
% % Other techniques
% As far as the server is concerned, any local model received can be from a potential malicious client. 
% FLTrust~\cite{cao2020fltrust} assumes the server acts as a client, i.e., trains a local model on an additional {\em trustworthy} dataset at the server's end and compares it against all the local models from other clients. 
% This way, the server can rely on some ``source of trust'' when discarding potentially malicious clients.
%\\
% Limitations of existing Byzantine-resilient strategies
Unfortunately, existing defense mechanisms either rely on simple heuristics (e.g., Trimmed Mean and FedMedian by~\cite{yin2018icml}) or need strong and unrealistic assumptions to work effectively (e.g., foreknowledge or estimation of the number of malicious clients in the FL system, as for Krum/Multi-Krum~\cite{blanchard2017nips} and Bulyan~\cite{mhamdi2018pmlr}, which, however, cannot exceed a fixed threshold).
Furthermore, outlier detection methods using K-means clustering~\cite{shen2016acm} or spectral analysis like DnC~\cite{shejwalkar2021ndss} do not directly consider the temporal evolution of local model updates received.
Finally, strategies like FLTrust~\cite{cao2020fltrust} require the server to collect its own dataset and act as a proper client, thereby altering the standard FL protocol.
\\
% OLD, LONG VERSION
% Overall, existing Byzantine-resilient strategies are either simple heuristics (e.g., FedMedian) or, if they are more complex, they rely on strong and unrealistic assumptions to work effectively (e.g., knowing the number of malicious clients in the FL system in advance, as for Krum and alike).
% Furthermore, data-driven outlier detection methods do not consider the temporary evolution of local model updates received (e.g., K-means clustering). 
% Finally, strategies like FLTrust requires the server to collect its own dataset and act as a proper client, thereby altering the standard FL protocol.
%
% Description of the proposed method
This work introduces a novel pre-aggregation \textit{filter} robust to untargeted model poisoning attacks. Notably, this filter $(i)$ operates without requiring prior knowledge or constraints on the number of malicious clients and $(ii)$ inherently integrates temporal dependencies. 
The FL server can employ this filter as a preprocessing step before applying \textit{any} aggregation function, be it standard like FedAvg or robust like Krum or Bulyan.
Specifically, we formulate the problem of identifying corrupted updates as a multidimensional (i.e., matrix-valued) time series anomaly detection task. 
The key idea is that legitimate local updates, resulting from well-calibrated iterative procedures like stochastic gradient descent (SGD) with an appropriate learning rate, show \textit{higher predictability} compared to malicious updates. This hypothesis stems from the fact that the sequence of gradients (thus, model parameters) observed during legitimate training exhibit regular patterns, as validated in Section~\ref{subsec:intuition}. %until convergence. 
%This regularity may be more pronounced for smooth convex loss functions, but it can still be captured within an appropriate time window, even for more complex and convoluted loss surfaces. 
%We provide evidence of this claim in Appendix~B, where we show that the average mutual information (i.e., ``predictability''), calculated over pairs of legitimate model updates sent at different FL rounds, is significantly higher than the corresponding computation for a malicious client.
\\
Inspired by the matrix autoregressive (MAR) framework for multidimensional time series forecasting~\cite{chen2021je}, we propose the FLANDERS ({\em \textbf{F}ederated \textbf{L}earning meets \textbf{AN}omaly \textbf{DE}tection for a \textbf{R}obust and \textbf{S}ecure}) filter.
The main advantages of FLANDERS over existing strategies like FLDetector~\cite{zhao2020multivariate} are its resilience to large-scale attacks, where $50\%$ or more FL participants are hostile, and the capability of working under realistic non-iid scenarios.
We attribute such a capability to two key factors: $(i)$ FLANDERS works without knowing a priori the ratio of corrupted clients, and $(ii)$ it embodies temporal dependencies between intra- and inter-client updates, quickly recognizing local model drifts caused by evil players. Below, we summarize our main contributions:

\begin{itemize}
\item[{\em(i)}]
We provide empirical evidence that the sequence of models sent by legitimate clients is more predictable than those of malicious participants performing untargeted model poisoning attacks.
\\
\item[{\em(ii)}] 
We introduce FLANDERS, the first pre-aggregation filter for FL robust to untargeted model poisoning based on multidimensional time series anomaly detection.
\\
\item[{\em(iii)}] 
We integrate FLANDERS into Flower,\footnote{\scriptsize{\url{https://flower.dev/}}} a popular FL simulation framework for reproducibility.
\\
\item[{\em(iv)}] 
We show that FLANDERS improves the robustness of the existing aggregation methods under multiple settings: different datasets, client's data distribution (non-iid), models, and attack scenarios.
\\
\item[{\em(v)}] 
We publicly release all the implementation code of FLANDERS along with our experiments.\footnote{\scriptsize{\url{https://anonymous.4open.science/r/flanders_exp-7EEB}}}
\end{itemize}

% Paper's structure and organization
The remainder of the paper is structured as follows. %some related work and the current state-of-the-art solutions to security issues that FL entails. 
Section~\ref{sec:background} covers background and preliminaries. 
In Section~\ref{sec:related}, we discuss related work.
Section~\ref{sec:problem} and Section~\ref{sec:method} describe the problem formulation and the method proposed. % to tackle it. 
Section~\ref{sec:experiments} gathers experimental results. %, and Section~\ref{sec:limitations} discusses some limitations of this work.
Finally, we conclude in Section~\ref{sec:conclusion}.
 %discusses the limitations of this work and draws future research directions.
%reports conclusions and draws perspectives for future research directions.

%%%%%%% OLD %%%%%%%
%to overcome the resilience of Byzantine failures in distributed Stochastic Gradient Descent computations. 
% The strength of Krum is its time complexity, which is linear in the gradient dimension. 
% However, the robustness of the approach is guaranteed for gradient-based learning applications only when the majority of the clients are not compromised. 
% Besides, the aggregation mechanism of Krum, as well as that of similar methods, is robust from a coarse-grained perspective and does not provide solutions to errors and perturbations that may occur at inference time.
%A related approach to~\cite{blanchard2017nips} is the work of Su et al.~\cite{su2016dc}. Here, the authors propose an iterated approximate agreement to tackle a multi-layer scenario attacked by Byzantine agents. 
%However, the method works efficiently on the sole discrete context and it is inapplicable to continuous state environments.
%\gabri{Maybe, we should just talk about the main limitations of existing countermeasures without digging into their details (or, we can just mention Krum as this is the most popular one). I will move the description of all these methods to the Related Work section.}

\begin{figure}[t]
    \centering
    \includegraphics[width=1.00\linewidth,trim={0 1cm 0 1cm},clip]{octotiger_sample.png}
    \caption{Snapshot of a double white dwarf merger with Octo-Tiger. Between the two stars, an aggregation belt is formed and mass from the smaller right star is transferred to the larger star on the left. The color shows the velocity magnitude of the stream, with red being high velocities.}
    \label{fig:octo_output}
\end{figure}
\section{Real-World Application as a Benchmark: Octo-Tiger}
\label{sec:octotiger}
The application driving our development efforts in this work is Octo-Tiger, as
we aim to run it on the Kokkos SYCL execution space. Octo-Tiger, itself, benefits
from the work shown here by gaining an additional execution space, allowing it
to target a wider range of platforms in the future.

Octo-Tiger is also an ideal benchmark to test integrations like the HPX-SYCL
one planned here, due to the tight interleaving of GPU and CPU work. Even in
small, single-node scenarios, we launch thousands of GPU compute kernels each
time step, tightly interleaved with CPU pre- and post-processing. 
Furthermore, it still contains CUDA and HIP
kernels for the main solvers, allowing us to compare performance using
multiple backends.
 
In the following, we introduce Octo-Tiger and briefly outline its solvers,
data structure and required software dependencies. Afterward, we describe its
execution model, based on all the required software, in more detail.

\subsection{Octo-Tiger: Scientific Application, Data-Structure and Solvers}
Octo-Tiger is a C\texttt{++} astrophysics code, used to study binary star
systems and their eventual outcomes, such as stellar
mergers~\cite{marcello2021octo}. These star systems contain two stars,
bound together by gravity. When close enough, they interact by slowly
exchanging mass. This can either be a stable transfer, occurring over millions
of years, or on unstable one which will disrupt one of the stars. Depending on
the system, this can lead to various outcomes, ranging from a Type Ia
supernovae, to the formation of another star. Figure~\ref{fig:octo_output}
shows a snapshot of a binary-star simulation done with Octo-Tiger at a point
where there's visible mass transfer occurring. In the past, Octo-Tiger was
already used to investigate R Coronae Borealis
stars~\cite{staff2018role} and the merger of bipolytropic
stars~\cite{kadam2018numerical}.
%stars~\cite{lauer2019evolving,staff2018role} and the merger of bipolytropic
%stars~\cite{kadam2018numerical}.

The stars are modeled as self-gravitating astrophysical fluids, using the
inviscid Navier-Stokes equation and Newtonian gravity. Hence, Octo-Tiger
contains two coupled solvers: A hydrodynamics solver using finite volumes and a
gravity solver~\cite{marcello2017very} using the fast-multipole-method (FMM).
These solvers operate on a 3D grid. Octo-Tiger uses adaptive mesh refinement,
maximizing the refinement for the area of interest, which usually is the
atmosphere between the stars. The data structure utilized for this is an
adaptive octree. For efficiency, instead of having just one cell per tree-node,
we use an entire sub-grid of cells per tree-node. The size of these sub-grids
is configurable at compile time, but the default we use is $8 \times 8 \times
8$ giving us $512$ per sub-grid (as larger sub-grids have adverse effects
regarding adaptivity, data-distribution, and FMM performance as the FMM uses the
tree-structure for approximations).

Highlighting its portability, Octo-Tiger was previously used on supercomputers
such as NERSC's Cori~\cite{heller2019harnessing}, Piz
Daint~\cite{10.1145/3295500.3356221} and ORNL's Summit~\cite{diehl2021octo}.
Recently, we began to target NERSC's Perlmutter and Riken's Supercomputer\
Fugaku.
It achieves this portability by using a mixture of HPX, Kokkos and other
frameworks as will be outlined in the next section.
%\footnote{\url{https://www.hpci-office.jp/output/hp210311/outcome.pdf?1668567633}}.

\subsection{Octo-Tiger's Software-Stack}

\subsubsection{Tasks and Distributed Computing with HPX}
HPX is an Asynchronous Many-Task Runtime system (AMT)~\cite{kaiser2020hpx}. It is 
itself implemented in C\texttt{++}, and also implements all C\texttt{++}
20 APIs regarding parallelism and concurrency (including for example
functionality like \lstinline[language=c++]{hpx::mutex}). With HPX, we can
build an explicit task graph using HPX futures. Asynchronous operations return
futures, which can be chained together using continuations to form such a
graph. This way, subsequent work and communication is triggered automatically
whenever a task finishes. HPX also works in distributed scenarios, allowing us
to call functions on remote components asynchronously, getting futures in
return. 
Once their respective task dependencies are fulfilled, the HPX tasks within this task graph are executed by a pool of HPX worker
threads (each task by a single worker), with us usually using one worker thread per available CPU core. This means, while we may have
a lot of tasks available, we just have a few worker threads executing them over time.

Octo-Tiger is built entirely upon HPX, using the task-based programming framework
for efficient tree-traversals and for distributed computing. To this end, each
of the aforementioned sub-grids in the octree is a HPX component. This component contains
all required data for its sub-grid and communicates with other sub-grids using
remote function calls (via HPX) and HPX channels to fill its ghost layers and
coordinate computing. This makes each sub-grid as self-contained unit for
computing, easing distribution.

Consequently, all compute kernels only operate on one sub-grid (and its ghost
layers) at a time. However, we can usually invoke the compute kernels
concurrently for different sub-grids.

\subsubsection{Portable Compute Kernel with Kokkos and HPX-Kokkos} 
Octo-Tiger's compute kernels themselves were originally written for single-core
execution, with multiple kernels usually running at the same time for different
sub-grids. Over time, they were ported to support
Vc~\cite{https://doi.org/10.1002/spe.1149, Pfander18acceleratingFull}, CUDA and
HIP. To unify these separate kernels, we finally settled on Kokkos. Kokkos
implements a programming model for developing portable kernels~\cite{9485033}.
It contains multiple execution and memory spaces, allowing us to run kernels on
various devices (such as AMD and NVIDIA GPUs). Kokkos also already includes an
experimental SYCL execution and memory space, allowing its users to target Intel
GPUs. 
%, especially since it allows using explicit SIMD via C\texttt{++} types (similar to Vc), yet also supports GPU execution for the same kernel (using scalar instantiation for the SIMD types).

Moreover, Kokkos is tightly integrated with HPX: Both an HPX execution space
(using HPX worker threads to execute a Kokkos kernel) and an HPX-Kokkos
integration layer exist, the latter allows returning futures for Kokkos kernels
running on supported execution spaces (the
CUDA, HIP and HPX execution
space)~\cite{daiss2021beyond}.  With the HPX-SYCL integration introduced in
this work, the list of supported execution spaces now also includes the SYCL
execution space (as HPX-Kokkos directly utilizes the \lstinline{get_future}
functionality developed in this work).
%execution space (as HPX-Kokkos directly utilizes the \lstinline{get_future}
%functionality for the respective underlying API of each execution space that we
%implement here).

\subsubsection{SIMD - Kokkos SIMD and \protect\lstinline[language=c++]{std::experimental::simd}} 
Kokkos kernels allow using explicit SIMD vectorization with C\texttt{++}
types~\cite{sahasrabudhe2019portable}. In our own kernels, we use this with the
Kokkos SIMD types\footnote{\url{https://github.com/kokkos/simd-math}},
enabling us to instantiate the types with, for example, AVX512 types when
compiling for CPUs and scalar double types when compiling for GPUs. The same
implementation also works using \lstinline[language=c++]{std::experimental::simd} on various CPUs. In previous
work, we investigated the performance difference between these two type libraries and
added SVE types~\cite{daiss2022simd} for Fujitsu A64FX\textsuperscript{\texttrademark} CPUs.



\subsubsection{Work Aggregation and Memory Optimizations - CPPuddle} 
Given the small, default size of the sub-grids with $512$ cells each, and the
fact that the compute kernels only work on one sub-grid at a time, the workload
per compute kernel is of essential importance for our GPU performance. 
Invoking just one such kernel is too little work to even come close to fully
utilize a GPU. As new sub-grids might be created over the duration of the
simulation and old ones may be deleted or migrated (all depending on the
adaptive mesh refinement as the simulation evolves), a static work aggregation
approach (defining sets of sub-grids to always be executed as one kernel) would
not work well for Octo-Tiger either.

Short of increasing the size of the sub-grids itself (negatively impacting
refinement), there are two ways of dealing with this: We can either rely on
executing enough GPU kernels concurrently (using multiple GPU executors) or
dynamically aggregate kernels as they are scheduled, depending on the load of
the GPU. For dynamic aggregation, we introduced work aggregation executors in
Octo-Tiger~\cite{daiss2022aggregation}. As these can be used in other HPX
applications, we extracted them into another software dependency, CPPuddle.
This library also contains memory pools for device-side buffers, which proved
to be handy when running the same task repeatedly for different sub-grids. If
the underlying GPU executor is currently busy, these aggregation executors will
start to bunch up compatible kernels (usually the same kernel but running on a
different sub-grid) as they are being scheduled. These will be launched as one
larger kernel when either an user-defined maximum number of aggregated kernels
is reached or the underlying GPU executor becomes idle and can work on it
immediately. For more details we refer to~\cite{daiss2022aggregation}.

\subsubsection{Other Dependencies}
Octo-Tiger uses some other dependencies, such as HDF5 and Silo, for the input
and output files. Dependencies like HWLOC and Boost are also indirectly
included in HPX.  Also, there are some older parts of Octo-Tiger that still use
Vc, however, we are in the process of removing those since we deprecated the Vc
support within Octo-Tiger in favor of using Kokkos kernels with explicit SIMD
types.

\subsection{Octo-Tiger's Execution Model}
\begin{figure}[t]
\centering
\includegraphics[width=.48\textwidth]{figures/model_sycl_2.pdf}  
\caption{Octo-Tiger's execution model, adapted from~\cite{daiss2022simd} for the SYCL integration and associated execution space.}
\label{fig:model_sycl}
\end{figure}
During solver iterations, for instance with the gravity solver, Octo-Tiger traverses the
oct-tree that contains its grid-structure. To do so, we simply call the
methods executing the solver of neighboring (or child) tree-nodes. Thanks to
HPX, it does not matter whether they are actually located on the current
compute node or a remote node, as each tree-node is an HPX component.
Calling these methods via HPX returns futures, which we can use to chain
dependent tasks together. For example, we can first collect the data from
neighboring tree-nodes to fill the ghost layer of the current node, using the
associated future to chain an asynchronous Kokkos kernel launch.
%operating on both the local tree-node's data and those ghost layers.

In turn, the compute kernels work on the sub-grid of the current tree-node and
its ghost layer. The Kokkos kernel itself is launched asynchronously using
HPX-Kokkos, giving us yet another future. Each kernel is executed on
either a CPU execution space, usually the Kokkos HPX execution space which
splits the kernel into HPX tasks that will be executed by one or more of the
worker threads, or a GPU execution space, such as the Kokkos CUDA execution
space. Depending on the execution location, we instantiate the appropriate SIMD
types within the kernel, either with scalar types on GPUs or types using the
appropriate SIMD instructions on CPUs to make use of SIMD operations and masks.
We outline this execution model, and where to plug in the HPX-SYCL integration,
in Figure~\ref{fig:model_sycl}.
Of course, we could already use the Kokkos SYCL execution space without any further integration, albeit
only synchronously. However, to make the whole machinery described above work with
SYCL as it does with the supported execution spaces, we need HPX-Kokkos to be
able to return a future for the underlying SYCL kernel, hence we need the HPX-SYCL integration. Otherwise, the
asynchronous creation of the task graph using the futures would become
synchronous, impacting the overall performance. 

Lastly, also part of Octo-Tiger's execution model but not shown in this figure:
We usually use a pre-allocated pool of GPU executors, each being able to work
on GPU data-transfers/kernels independently. We can either use
only one of those (disabling concurrent GPU kernels altogether) or up to $128$
(theoretically more are possible but yield no further benefits). As working on
one sub-grid at a time might not be enough work to fully utilize some devices,
even given concurrent GPU kernels, we further employ the aforementioned dynamic
work aggregation by using the aggregation executors that can dynamically aggregate kernel
launches of the same kernel on different sub-grids. As shown in
\cite{daiss2022aggregation}, this is beneficial on
NVIDIA A100 GPUs and crucial on AMD MI100
GPUs.



\section{Integrating Alternative StateDBs}\label{sec3}

%\subsection{Components requiring changes to the HLF}

In the open source HLF code (commit \cite{l15}), the main components responsible for HLF interaction with StateDB have been identified. These are depicted in Fig. \ref{pic1}. In order to add new StateDBs, it was decided to create an analogue of the marked components for each database. So the main code upgrade was to add three code files for each DB: stateNameOfDB.go, NameOfDB\_provider.go, NameOfDB\_helper.go. Functionality of goleveldb has been fully preserved. To select StateDB you need to specify in core.yaml file the name of selected DB before running the container.

In addition to the modules described, test files covering the added code have also been created and some minor changes have been implemented in some other HLF modules.


\begin{figure}
  \includegraphics[width=9cm,height=1.3cm]{figures/StateDBMainParts.png}
  \caption{HLF components responsible for interacting with StateDB (highlighted in color).}
\label{pic1}
\end{figure}



\section{Results}
\label{results}

\begin{figure*}[ht]
    \centering
    \includegraphics[scale=0.15,trim={0 2.5cm 0 5cm},clip]{images/aoi-single_burst}
    \caption{The time average peak Age of Information with burst and \gls{soa} loss values against the dynamic reliability logic for different network topologies.}
    \label{fig:aoi_burst}\vspace{-0.4cm}
\end{figure*}


This paper focuses on both transport layer and application layer metrics to determine the feasibility of dynamic reliability. For this, we have selected the session packet volume, as transmitted, retransmitted, lost and backlogged packets as \glspl{kpi} for the transport layer; while focusing on the \gls{aoi} for the application layer. The \gls{aoi} was chosen as a crucial indicator for the freshness of packets in real-time applications. More specifically, this work adopts the time average peak \gls{aoi} equation \cite{aoi_equation} depicted in Eq. \ref{aoi}, where $\Delta(r_{i+1})$ is the $i$th update at the time it was received at the server, for a session time period of $\tau$.

\begin{equation}
    \label{aoi}
    \gls{aoi}_\tau = \frac{1}{n-1}\sum_{i=1}^{n-1} \Delta(r_{i+1})
\end{equation}

We include a comparison between the vanilla QUIC implementation which does not enjoy the dynamic reliability extension, with a number of dynamic reliability policies. The tests were run a number of times for statistical significance, with the mean value of vanilla implementation used as a baseline for comparison. The topology utilised both random loss and bursty loss to explore the bounds of dynamic reliability. The \gls{soa} loss in the figures correspond to the loss values presented in Table. \ref{tab:path_char}, for ease of comparison between bursty and random loss scenarios.

\subsection{Transport-Layer KPIs}

To analyse the performance gain at the transport layer due to dynamic reliability, the volume of transmitted and backlogged packets is examined. The figures are in the form of boxplots, which take the vanilla implementation as a benchmark, depicted as the red dashed line.

As seen in Fig. \ref{fig:sent_burst}, the loss plays a crucial role in the performance of the reliability policies. The policies under random loss did incredibly well for the networks with a larger capacity, namely \gls{mmwave} and Sub-6~GHz, whereas for burst loss, the lower network capacities had a larger packet reduction. With the increase in burst loss, the behaviour of the set split reliable policies became unpredictable, if a reliable assignment happened to coincide with a burst loss, the number of transmitted packets increases, and vice versa. On the other hand, in smarter policies, such as Loss-Aware, the performance lightly matched the vanilla baseline, as the reliable assignment dominated the session to compensate for a higher burst loss. Not only that but, the burst loss also impacted the variance of the transmitted packets for the policies.

Unsurprisingly, the unreliable focused policy, 80-20 split, outperformed other policies for all topologies in random and bursty loss scenarios, with an approximate reduction of 80\%. That being said, the majority of the policies reduced the transmitted packets on the link by approximately 70\% for random loss, while the reduction started at $\approx 15\%$ and decreased as the loss increased for the burst loss scenario.

The retransmitted and lost packets, not shown due to space limitations, followed the same trend as the transmitted packets for the random loss scenarios. However, for the burst loss scenarios, the larger capacity networks had a lower reduction in the retransmitted and lost packets. This can be seen as a favorable outcome since the lower capacity networks are scarce on resources. It is important to note that the Loss-Aware policy mimicked the vanilla approach as the burst loss increased, signifying the overwhelming appointment of reliable packets in adapting to the harsh burst loss conditions.
 
Alternatively, Fig. \ref{fig:backlog_burst} clearly shows a stark comparison between the policies and loss scenario in the reduction of the backlogged packets. The Loss-Aware policy for random loss scenario reduced the backlogged packets by up to 50\%, beating all other policies by approximately 30\%. Furthermore, it is clear that the unreliability focused policies resulted in the lowest backlog for the session. In comparison, we notice that the burst loss and the backlogged frequency have a positive correlation, where the maximum reduction of the backlogged packets for the policies is at most 20\%. Much like the transmitted packets, the probability of a burst loss occurrence plays a vital role in the number of retransmissions sent and by extension the number of backlogged packets. Thus, we can conclude that the stress placed on the buffer is a result of the reliable packets which is tightly coupled with the congestion on the session. Whereas, unreliable focused policies did not encounter such a phenomenon regardless if it was experiencing a burst loss.


\subsection{Application-Layer KPIs}

The feasibility of dynamic reliability for real-time applications can be determined by the \gls{aoi}, with comparison across different topologies and policies. If we take a strict approach and consider anything below $10$~ms is real-time \cite{real-time}, then all the reliability policies passed that requirement, which is attractive for real-time applications, as shown in Fig. \ref{fig:aoi_burst}. Utilising the median as an estimate of the runs, the policies in the WLAN and Sub-6~GHz topology with random loss floated around $4-5$~ms with negligible difference, while the \gls{aoi} for \gls{mmwave} was $\approx 2-3$~ms. It is clear that the \gls{aoi} and the network capacity have a negative correlation, as the network capacity decreases, the \gls{aoi} increases. The same correlation is extended to the bursty loss scenarios, where \gls{mmwave} dominated the other topologies. That being said, it is crucial to note that the \gls{aoi} for the reliability policies is often slightly better than or equal to the \gls{aoi} of the vanilla implementation, proving that dynamic reliability reduces the congestion of the session at no cost to the \gls{aoi}.



\section{Related Work}\label{sec:related-work}



Over the last few years, several benchmarks for stream processing frameworks have been proposed and stream processing benchmarking studies have been conducted. The differentiation between benchmarks and experimental studies applying them is sometimes blurry. Many publications that present benchmarks perform also an experimental study with them. On the other hand, many experimental studies utilize existing benchmarks, but modify them.
Nevertheless, we structure this section into two parts: First, we give an overview of stream processing benchmarks to justify our benchmark selection for this study. Second, we discuss related stream processing benchmarking studies.

\subsection{Related Work on Stream Processing Benchmarks}

Besides the Theodolite benchmarks for event-driven microservices used in this study, several other benchmarks for stream processing frameworks have been proposed.
\cref{tab:related-benchmarks} summarizes characteristics of the discussed benchmarks. 


\begin{table*}
	\begin{threeparttable}[b]
		\caption{Overview of the characteristics and implementations of stream processing benchmarks.}
		\label{tab:related-benchmarks}
		\footnotesize
		\newcommand{\cmark}{\ding{51}}%
		\newcommand{\xmark}{\ding{55}}%
		\newcommand{\qmark}{\makebox[0pt][l]{\textbf{\textit{?}}}\phantom{\cmark}}%
		
		\newcommand{\txnote}[1]{\makebox[0pt][l]{\tnote{#1}}}
		
		\newcommand\undefcolumntype[1]{\expandafter\let\csname NC@find@#1\endcsname\relax}
		\newcommand\forcenewcolumntype[1]{\undefcolumntype{#1}\newcolumntype{#1}}
		
		\newcommand*\rot{\rotatebox{90}}
		\newcolumntype{L}{>{\raggedright\arraybackslash}X}
		\newcolumntype{R}{>{\raggedleft\arraybackslash}X}
		\newcolumntype{C}{>{\centering\arraybackslash}X}
		\newcolumntype{o}{p{0pt}}
		\renewcommand{\arraystretch}{1.2}
		\newcommand{\fnoptional}{a}
		\newcommand{\fnbeam}{b}
		\newcommand{\fnriottasksamples}{d}
		\newcommand{\fnbeamnexmark}{c}
		\begin{tabularx}{\textwidth}{ll o C o C o CCC o CCCCCCC o C o CCC}
			\toprule
			&&& && && \multicolumn{3}{c}{Messaging} && \multicolumn{7}{c}{Stream processing framework} && && \multicolumn{3}{c}{Cloud-native} \\
			\cmidrule{8-10} \cmidrule{12-18} \cmidrule{22-24}
			Benchmark & Published && \rot{Task samples} && \rot{Open source} && \rot{Kafka} & \rot{Others} & \rot{None} && \rot{Flink} & \rot{Spark} & \rot{Storm} & \rot{Samza} & \rot{Kafka Streams} & \rot{Hazelcast Jet} & \rot{Others} && \rot{Database} && \rot{Containers} & \rot{Kubernetes} & \rot{Others} \\
			\midrule
			Theodolite \cite{BDR2021} & \citeyear{BDR2021}
			& %
			& 4
			& %
			& \cmark %
			& %
			& \cmark %
			& %
			& %
			& %
			& \cmark %
			& %
			& %
			& \cmark\txnote{\fnbeam} %
			& \cmark %
			& \cmark %
			& \cmark\txnote{\fnbeam} %
			& %
			& \phantom{\cmark}\txnote{\fnoptional} %
			& %
			& \cmark %
			& \cmark %
			& %
			\\
			Beam Nexmark \cite{BeamNexmark2022} & \citeyear{BeamNexmark2022}\txnote{\fnbeamnexmark}
			& %
			& 13
			& %
			& \cmark %
			& %
			& \cmark %
			& \cmark %
			& %
			& %
			& \cmark\txnote{\fnbeam} %
			& \cmark\txnote{\fnbeam} %
			& %
			& \qmark\txnote{\fnbeam} %
			& %
			& \qmark\txnote{\fnbeam} %
			& \cmark\txnote{\fnbeam} %
			& %
			& %
			& %
			& %
			& %
			& %
			\\
			ESPBench \cite{Hesse2021} & \citeyear{Hesse2021}
			& %
			& 5
			& %
			& \cmark %
			& %
			& \cmark %
			& %
			& %
			& %
			& \cmark\txnote{\fnbeam} %
			& \cmark\txnote{\fnbeam} %
			& %
			& \qmark\txnote{\fnbeam} %
			& %
			& \cmark\txnote{\fnbeam} %
			& \qmark\txnote{\fnbeam} %
			& %
			& \cmark %
			& %
			& %
			& %
			& %
			\\
			OSPBench \cite{vanDongen2020} & \citeyear{vanDongen2020}
			& %
			& 5
			& %
			& \cmark %
			& %
			& \cmark %
			& %
			& %
			& %
			& \cmark %
			& \cmark %
			& %
			& %
			& \cmark %
			& %
			& %
			& %
			& %
			& %
			& \cmark %
			& %
			& \cmark %
			\\
			DSPBench \cite{Bordin2020} & \citeyear{Bordin2020}
			& %
			& 5
			& %
			& \cmark %
			& %
			& \cmark %
			& %
			& %
			& %
			& %
			& \cmark %
			& \cmark %
			& %
			& %
			& %
			& %
			& %
			& \cmark %
			& %
			& %
			& %
			& %
			\\
			\citet{Shahverdi2019} & \citeyear{Shahverdi2019}
			& %
			& 1
			& %
			& \cmark %
			& %
			& \cmark %
			& %
			& %
			& %
			& \cmark %
			& \cmark %
			& \cmark %
			& %
			& \cmark %
			& \cmark %
			& %
			& %
			& \cmark %
			& %
			& %
			& %
			& %
			\\
			\citet{Karimov2018} & \citeyear{Karimov2018}
			& %
			& 2
			& %
			& %
			& %
			& %
			& %
			& \cmark %
			& %
			& \cmark %
			& \cmark %
			& \cmark %
			& %
			& %
			& %
			& %
			& %
			& %
			& %
			& %
			& %
			& %
			\\
			RIoTBench \cite{Shukla2017} & \citeyear{Shukla2017}
			& %
			& 4\txnote{\fnriottasksamples} %
			& %
			& \cmark %
			& %
			& %
			& \cmark %
			& %
			& %
			& %
			& %
			& \cmark %
			& %
			& %
			& %
			& %
			& %
			& \cmark %
			& %
			& %
			& %
			& %
			\\
			YSB \cite{Chintapalli2016} & \citeyear{Chintapalli2016}
			& %
			& 1
			& %
			& \cmark %
			& %
			& \cmark %
			& %
			& %
			& %
			& \cmark %
			& \cmark %
			& \cmark %
			& %
			& %
			& %
			& %
			& %
			& \cmark %
			& %
			& %
			& %
			& %
			\\
			SparkBench \cite{Li2015} & \citeyear{Li2015}
			& %
			& 10
			& %
			& \cmark %
			& %
			& %
			& %
			& \cmark %
			& %
			& %
			& \cmark %
			& %
			& %
			& %
			& %
			& %
			& %
			& %
			& %
			& %
			& %
			& %
			\\
			StreamBench \cite{Lu2014} & \citeyear{Lu2014}
			& %
			& 7
			& %
			& %
			& %
			& \cmark %
			& %
			& %
			& %
			& %
			& \cmark %
			& \cmark %
			& %
			& %
			& %
			& %
			& %
			& %
			& %
			& %
			& %
			& %
			\\
			Linear Road \cite{Arasu2004} & \citeyear{Arasu2004}
			& %
			& 5
			& %
			& %
			& %
			& %
			& %
			& \cmark %
			& %
			& %
			& %
			& %
			& %
			& %
			& %
			& \cmark %
			& %
			& %
			& %
			& %
			& %
			& %
			\\
			\bottomrule
		\end{tabularx}
		\begin{tablenotes}\footnotesize
			\item[\fnoptional] optional
			\item[\fnbeam] using Apache Beam
			\item[\fnbeamnexmark] the Beam Nexmark benchmarks are based on the Nexmark paper \cite{Tucker2010} published in \citeyear{Tucker2010}
			\item[\fnriottasksamples] RIoTBench's 4 application benchmarks are composed of 27 microbenchmarks
		\end{tablenotes}
	\end{threeparttable}
\end{table*}



StreamBench~\cite{Lu2014} is one of the earliest benchmarks for modern stream processing frameworks. While originally only implemented for Spark and Storm, it has later been used to benchmark Apache Apex, Beam, Flink, and Samza as well \cite{Hesse2019, Qian2016}.
As its name suggests, SparkBench~\cite{Li2015} is a benchmark tailored to Apache Spark.
The Yahoo Streaming Benchmark (YSB) \cite{Chintapalli2016} is frequently used and adapted in research \cite{Lopez2016, Yang2017, Karakaya2017, Nasiri2019, Zeuch2019, Chu2020, vanDongen2020}.
Worth highlighting is the work of \citet{Shahverdi2019}, who extend YSB with implementations for the frameworks Kafka Streams and Hazelcast Jet. As discussed in \cref{sec:frameworks}, these frameworks are particularly suited for building event-driven microservices.
RIoTBench \cite{Shukla2017} provides four application benchmarks for Storm composed of 27~small task samples. \citet{Nasiri2019} adopt RIoTBench for Flink and Spark.
\citet{Karimov2018} present a benchmark with two task samples, derived from a real industrial context, yet without providing open-source implementations.

More recently, DSPBench \cite{Bordin2020}, OSPBench~\cite{vanDongen2020}, and ESPBench \cite{Hesse2021} have been proposed.
DSPBench contains 15~benchmarks, which resample typical stream processing applications, derived from reviewing the literature.
OSPBench provides benchmarks for analyzing traffic sensor data. Besides evaluations of latency, throughput, and resource usage, \citeauthor{vanDongen2020} used OSPBench to also evaluate scalability~\cite{vanDongen2021b} and fault recovery~\cite{vanDongen2021a}.
In contrast to most other benchmarks, OSPBench provides implementations for the rather new framework Kafka Streams, which is also evaluated in this study.
The Enterprise Stream Processing Benchmark (ESPBench) builds upon the Senska benchmark \cite{Hesse2018}.
It is special in the sense that it integrates a relational database management system.
In contrast to most other benchmarks, ESPBench's task samples are implemented with Apache Beam. While \citet{Hesse2021} only perform evaluations with Spark, Flink, and Hazelcast Jet, we expect that also other Beam runners can be used to run the benchmark.

The Nexmark benchmark \cite{Tucker2010} has originally been proposed as the \textit{Niagara Extension to the XMark benchmark} addressed to first-generation stream processing systems (see the survey of \citet{Fragkoulis2023} for a discussion of first and second-generation stream processing systems).
The Apache Beam community adapted and extended Nexmark with implementations for Beam to benchmark the performance of different runners~\cite{BeamNexmark2022}.
Documentation and benchmark results are provided for the direct runner as well as for the Flink, the Spark, and the Google Cloud Dataflow runners.
However, running the benchmark with other runners should be possible as well.
Recently, there seems to be an effort to implement the Nexmark task samples with other frameworks in an open-source project.\footnote{\url{https://github.com/nexmark/nexmark}}
However, currently this project only provides implementations for Apache Flink.
Moreover, \citet{Gencer2021} implemented the Nexmark benchmark for their performance evaluation of Hazelcast Jet.

Worth mentioning is also the Linear Road benchmark presented by \citet{Arasu2004}. Although published years before all modern stream processing frameworks considered in this work have been released, it is still used in research \cite{Zhang2017,Zeuch2019,Sax2020} and compared to newer benchmarks \cite{Bordin2020,Hesse2021}.
\citet{Pagliari2020} and \citet{Garcia2022a, Garcia2022b} present approaches to generate benchmarks.






From \cref{tab:related-benchmarks}, we can see that a lot of open-source benchmarks have been proposed. Apart from the Theodolite benchmarks, none of these benchmarks is particularly addressed to scalability.
Often originating in data management research, many benchmarks are defined as ``queries'' over data streams~\cite{Tucker2010,Karimov2018,Hesse2021}.
Most benchmarks include a messaging system as a middleware component between workload generation and stream processing framework. In the vast majority of cases, this is Apache Kafka.
\citet{Karimov2018} exclude such a system to not let it become the benchmark's bottleneck. Our Theodolite benchmarks purposely include Kafka to represent more realistic event-driven microservice deployments~\cite{BDR2021}.
Flink, Spark, and Storm are by far the most supported frameworks. Only a few benchmarks exist for Samza, Kafka Streams, and Hazelcast Jet, which are frameworks particularly suited for implementing event-driven microservice. Our Theodolite benchmarks are the only ones providing implementations for all of them.
While some benchmarks include an interaction with a database in their setup, others do not.
With the Theodolite benchmarks, a database can optionally be used as we did in a previous study~\cite{IC2E2022FaaSStreaming}.
Besides our Theodolite benchmarks, there is only one other benchmark (OSPBench) that is provided as container images to be used in a cloud-native setting. No other benchmark provides Kubernetes manifests.





\subsection{Related Work on Stream Processing Benchmarking}


\begin{table*}
	\begin{threeparttable}[b]
		\caption{Overview of employed benchmarks, frameworks, and experimental setup of stream processing benchmarking studies.}
		\label{tab:related-experiments}
		\footnotesize
		\newcommand{\cmark}{\ding{51}}%
		\newcommand{\xmark}{\ding{55}}%
		\newcommand{\qmark}{\makebox[0pt][l]{\textbf{\textit{?}}}\phantom{\cmark}}%
		
		\newcommand{\txnote}[1]{\makebox[0pt][l]{\tnote{#1}}}
		
		\newcommand\undefcolumntype[1]{\expandafter\let\csname NC@find@#1\endcsname\relax}
		\newcommand\forcenewcolumntype[1]{\undefcolumntype{#1}\newcolumntype{#1}}
		
		\newcommand*\rot{\rotatebox{90}}
		\newcolumntype{L}{>{\raggedright\arraybackslash}X}
		\newcolumntype{R}{>{\raggedleft\arraybackslash}X}
		\newcolumntype{C}{>{\centering\arraybackslash}X}
		\newcolumntype{o}{p{0pt}}
		\renewcommand{\arraystretch}{1.2}
		\newcommand{\fnvandenpoel}{a}
		\newcommand{\fnbeam}{b}
		\begin{tabularx}{\textwidth}{ll o CCCCCCCCCCCCC o CCCCCCC o CCCCCC}
			\toprule
			&&& \multicolumn{13}{c}{Benchmark} && \multicolumn{7}{c}{Framework} && \multicolumn{6}{c}{Execution} \\
			\cmidrule{4-16} \cmidrule{18-24} \cmidrule{26-31}
			Publication & Year &&
			\rot{Theodolite \cite{BDR2021}} &
			\rot{Beam Nexmark \cite{BeamNexmark2022}} &
			\rot{ESPBench \cite{Hesse2021}} &
			\rot{OSPBench \cite{vanDongen2020}} &
			\rot{DSPBench \cite{Bordin2020}} &
			\rot{\citet{Shahverdi2019}} &
			\rot{\citet{Karimov2018}} &
			\rot{RIoTBench \cite{Shukla2017}} &
			\rot{YSB \cite{Chintapalli2016}} &
			\rot{SparkBench \cite{Li2015}} &
			\rot{StreamBench \cite{Lu2014}} &
			\rot{Linear Road \cite{Arasu2004}} &
			\rot{Others}
			&&
			\rot{Flink} &
			\rot{Spark} &
			\rot{Storm} &
			\rot{Samza} &
			\rot{Kafka Streams} &
			\rot{Hazelcast Jet} &
			\rot{Others}
			&&
			\rot{Cloud environment} &
			\rot{Distributed} &
			\rot{Different resource amounts} &
			\rot{\dots in isolated experiments} &
			\rot{Different load intensities} &
			\rot{\dots in isolated experiments}
			\\
			\midrule
			This work &
				& %
				& \cmark %
				& %
				& %
				& %
				& %
				& %
				& %
				& %
				& %
				& %
				& %
				& %
				& %
				& %
				& \cmark %
				& %
				& %
				& \cmark\txnote{\fnbeam} %
				& \cmark %
				& \cmark %
				& %
				& %
				& \cmark %
				& \cmark %
				& \cmark %
				& \cmark %
				& \cmark %
				& \cmark %
			\\
			\citet{IC2E2022FaaSStreaming} & \citeyear{IC2E2022FaaSStreaming}
				& %
				& \cmark %
				& %
				& %
				& %
				& %
				& %
				& %
				& %
				& %
				& %
				& %
				& %
				& %
				& %
				& \cmark\txnote{\fnbeam} %
				& %
				& %
				& \cmark\txnote{\fnbeam} %
				& %
				& %
				& \cmark\txnote{\fnbeam} %
				& %
				& \cmark %
				& \cmark %
				& \cmark %
				& \cmark %
				& \cmark %
				& \cmark %
			\\
			\citet{Hesse2021} & \citeyear{Hesse2021}
				& %
				& %
				& %
				& \cmark %
				& %
				& %
				& %
				& %
				& %
				& %
				& %
				& %
				& %
				& %
				& %
				& \cmark\txnote{\fnbeam} %
				& \cmark\txnote{\fnbeam} %
				& %
				& %
				& %
				& \cmark\txnote{\fnbeam} %
				& %
				& %
				& %
				& \cmark %
				& \cmark %
				& \cmark %
				& %
				& %
			\\
			van Dongen\tnote{\fnvandenpoel} \cite{vanDongen2021b} & \citeyear{vanDongen2021b}
				& %
				& %
				& %
				& %
				& \cmark %
				& %
				& %
				& %
				& %
				& %
				& %
				& %
				& %
				& %
				& %
				& \cmark %
				& \cmark %
				& %
				& %
				& \cmark %
				& %
				& %
				& %
				& \cmark %
				& \cmark %
				& \cmark %
				& %
				& \cmark %
				& \cmark %
			\\
			van Dongen\tnote{\fnvandenpoel} \cite{vanDongen2021a} & \citeyear{vanDongen2021a}
				& %
				& %
				& %
				& %
				& \cmark %
				& %
				& %
				& %
				& %
				& %
				& %
				& %
				& %
				& %
				& %
				& \cmark %
				& \cmark %
				& %
				& %
				& \cmark %
				& %
				& %
				& %
				& \cmark %
				& \cmark %
				& %
				& %
				& \cmark %
				& %
			\\
			\citet{Bordin2020} & \citeyear{Bordin2020}
				& %
				& %
				& %
				& %
				& %
				& \cmark %
				& %
				& %
				& %
				& %
				& %
				& %
				& %
				& %
				& %
				& %
				& \cmark %
				& \cmark %
				& %
				& %
				& %
				& %
				& %
				& \cmark %
				& \cmark %
				& %
				& %
				& \cmark %
				& \cmark %
			\\
			\citet{Chu2020} & \citeyear{Chu2020}
				& %
				& %
				& %
				& %
				& %
				& %
				& %
				& %
				& %
				& \cmark %
				& %
				& %
				& %
				& \cmark %
				& %
				& \cmark %
				& %
				& \cmark %
				& %
				& %
				& %
				& \cmark %
				& %
				& %
				& \cmark %
				& \cmark %
				& %
				& %
				& %
			\\
			\citet{Vikash2020} & \citeyear{Vikash2020}
				& %
				& %
				& %
				& %
				& %
				& %
				& %
				& %
				& %
				& %
				& %
				& %
				& %
				& \cmark %
				& %
				& \cmark %
				& \cmark %
				& \cmark %
				& %
				& %
				& %
				& \cmark %
				& %
				& %
				& \cmark %
				& %
				& %
				& \cmark %
				& \cmark %
			\\
			van Dongen\tnote{\fnvandenpoel} \cite{vanDongen2020} & \citeyear{vanDongen2020}
				& %
				& %
				& %
				& %
				& \cmark %
				& %
				& %
				& %
				& %
				& %
				& %
				& %
				& %
				& %
				& %
				& \cmark %
				& \cmark %
				& %
				& %
				& \cmark %
				& %
				& %
				& %
				& \cmark %
				& \cmark %
				& \cmark %
				& %
				& %
				& %
			\\
			\citet{Nasiri2019} & \citeyear{Nasiri2019}
				& %
				& %
				& %
				& %
				& %
				& %
				& %
				& %
				& \cmark %
				& \cmark %
				& %
				& %
				& %
				& %
				& %
				& \cmark %
				& \cmark %
				& \cmark %
				& %
				& %
				& %
				& %
				& %
				& %
				& \cmark %
				& \cmark %
				& \cmark %
				& \cmark %
				& \cmark %
			\\
			\citet{Shahverdi2019} & \citeyear{Shahverdi2019}
				& %
				& %
				& %
				& %
				& %
				& %
				& \cmark %
				& %
				& %
				& %
				& %
				& %
				& %
				& %
				& %
				& \cmark %
				& \cmark %
				& \cmark %
				& %
				& \cmark %
				& \cmark %
				& %
				& %
				& \cmark %
				& \cmark %
				& \cmark %
				& \cmark %
				& %
				& %
			\\
			\citet{Zeuch2019} & \citeyear{Zeuch2019}
				& %
				& %
				& %
				& %
				& %
				& %
				& %
				& %
				& %
				& \cmark %
				& %
				& %
				& \cmark %
				& \cmark %
				& %
				& \cmark %
				& \cmark %
				& \cmark %
				& %
				& %
				& %
				& \cmark %
				& %
				& %
				& \cmark %
				& %
				& %
				& \cmark %
				& \cmark %
			\\
			\citet{Karimov2018} & \citeyear{Karimov2018}
				& %
				& %
				& %
				& %
				& %
				& %
				& %
				& \cmark %
				& %
				& %
				& %
				& %
				& %
				& %
				& %
				& \cmark %
				& \cmark %
				& \cmark %
				& %
				& %
				& %
				& %
				& %
				& %
				& \cmark %
				& \cmark %
				& %
				& \cmark %
				& \cmark %
			\\
			\citet{Truong2018} & \citeyear{Truong2018}
				& %
				& %
				& %
				& %
				& %
				& %
				& %
				& %
				& %
				& %
				& %
				& %
				& %
				& \cmark %
				& %
				& %
				& %
				& %
				& %
				& %
				& %
				& \cmark %
				& %
				& \cmark %
				& \cmark %
				& %
				& %
				& \cmark %
				& \cmark %
			\\
			\citet{Karakaya2017} & \citeyear{Karakaya2017}
				& %
				& %
				& %
				& %
				& %
				& %
				& %
				& %
				& %
				& \cmark %
				& %
				& %
				& %
				& %
				& %
				& \cmark %
				& \cmark %
				& \cmark %
				& %
				& %
				& %
				& %
				& %
				& %
				& \cmark %
				& %
				& %
				& \cmark %
				& \cmark %
			\\
			\citet{Shukla2017} & \citeyear{Shukla2017}
				& %
				& %
				& %
				& %
				& %
				& %
				& %
				& %
				& \cmark %
				& %
				& %
				& %
				& %
				& %
				& %
				& %
				& %
				& \cmark %
				& %
				& %
				& %
				& %
				& %
				& \cmark %
				& \cmark %
				& \cmark %
				& %
				& %
				& %
			\\
			\citet{Yang2017} & \citeyear{Yang2017}
				& %
				& %
				& %
				& %
				& %
				& %
				& %
				& %
				& %
				& \cmark %
				& %
				& %
				& %
				& \cmark %
				& %
				& \cmark %
				& \cmark %
				& \cmark %
				& %
				& %
				& %
				& %
				& %
				& \cmark %
				& \cmark %
				& %
				& %
				& %
				& %
			\\
			\citet{Chintapalli2016} & \citeyear{Chintapalli2016}
				& %
				& %
				& %
				& %
				& %
				& %
				& %
				& %
				& %
				& \cmark %
				& %
				& %
				& %
				& %
				& %
				& \cmark %
				& \cmark %
				& \cmark %
				& %
				& %
				& %
				& %
				& %
				& %
				& \cmark %
				& \cmark %
				& \cmark %
				& %
				& %
			\\
			\citet{Lopez2016} & \citeyear{Lopez2016}
				& %
				& %
				& %
				& %
				& %
				& %
				& %
				& %
				& %
				& %
				& %
				& %
				& %
				& \cmark %
				& %
				& \cmark %
				& \cmark %
				& \cmark %
				& %
				& %
				& %
				& %
				& %
				& %
				& \cmark %
				& %
				& %
				& \cmark %
				& \cmark %
			\\
			\citet{Qian2016} & \citeyear{Qian2016}
				& %
				& %
				& %
				& %
				& %
				& %
				& %
				& %
				& %
				& %
				& %
				& \cmark %
				& %
				& %
				& %
				& %
				& \cmark %
				& \cmark %
				& \cmark %
				& %
				& %
				& %
				& %
				& %
				& \cmark %
				& \cmark %
				& \cmark %
				& %
				& %
			\\
			\citet{Lu2014} & \citeyear{Lu2014}
				& %
				& %
				& %
				& %
				& %
				& %
				& %
				& %
				& %
				& %
				& %
				& \cmark %
				& %
				& %
				& %
				& %
				& \cmark %
				& \cmark %
				& %
				& %
				& %
				& %
				& %
				& %
				& \cmark %
				& \cmark %
				& \cmark %
				& %
				& %
			\\
			\bottomrule
		\end{tabularx}
		\begin{tablenotes}\footnotesize
			\item[\fnvandenpoel] and van den Poel
			\item[\fnbeam] using Apache Beam
		\end{tablenotes}
	\end{threeparttable}
\end{table*}

\cref{tab:related-experiments} provides an overview of experimental performance evaluation and benchmarking studies. It indicates the applied benchmark, the evaluated stream processing, and information regarding the experiment setup and method. The latter includes whether the respective study was performed in a cloud environment, in a distributed fashion with multiple instances of the framework deployed. Moreover, it shows whether the benchmarks have been executed with different resource amounts and different load intensities and whether different resource amounts and load intensities are evaluated in isolated experiments. In previous work, we argued that scalability should be evaluated with isolated experiments for different combinations of load and resources~\cite{LTB2021,EMSE2022}.

We can observe that there is no established stream processing benchmark. Only YSB is used in several studies. However, YSB can be considered a micro-benchmark~\cite{Bermbach2017} and, hence, is less suited to benchmark entire microservices.
Except for the preliminary evaluation of our Theodolite benchmarks~\cite{BDR2021}, there is no benchmarking study addressed to stream processing frameworks employed within microservice architectures.

Flink, Spark, and Strom are by far the most frequently benchmarked frameworks. Kafka Stream, Hazelcast Jet, and Samza, which are particularly suited for implementing event-driven microservices, are only benchmarked in a few studies and there is no study benchmarking all of them.

9 out of 20 studies report on experiments in public or private clouds.
Except for this and our previous study~\cite{IC2E2022FaaSStreaming}, there are no evaluations in Kubernetes.
Likewise, there are no further studies evaluating scalability with a systematic approach as we do in this study. \citet{Vikash2020}, \citet{Nasiri2019}, \citet{Karakaya2017}, and \citet{vanDongen2021b} explicitly evaluate scalability, however, without testing different load intensities against different resource amounts in isolated experiments. \citet{Nasiri2019} conduct independent evaluations of scaling load and computing resources and, thus, address another aspect than our study.
Our previous study~\cite{IC2E2022FaaSStreaming} applies our Theodolite method as well, but benchmarks scalability with respect to costs and is addressed to comparing stream processing deployments against Function-as-a-Service offerings.




\section{Conclusion}\label{sec:conclusion}
In this work, we focus on addressing the fundamental challenge of OOD detection tasks, which is how to fully understand the semantic discrepancy between the ID/OOD samples. We reveal that the key to success in the realistic SCOOD task is to allocate as many ID samples in the unlabeled set correctly as possible. To this end, we propose a novel uncertainty-aware optimal transport scheme that introduces class-specific energy scores as guidance for effective label assignment. Experimental results show that our method achieves better performance than previous state-of-the-art methods on SCOOD benchmarks.

\textbf{Limitations.} In addition to temperature scaling, other techniques such as feature clipping applied in ReAct~\cite{sun2021react} also enhance the performance of energy score, so how to obtain an OOD score that best fits the SCOOD task can be further explored. Moreover, a setting highly related to SCOOD has been proposed in \cite{katz2022training} and formulated as a constrained optimization problem. We will also theoretically analyze these practical OOD settings in our feature work.

% \section*{Acknowledgments}
\textbf{Acknowledgments.} 
This work is supported by National Key R\&D Program of China under Grant 2020AAA0105701, National Natural Science Foundation of China (NSFC) under Grants 61872327, Major Special Science and Technology Project of Anhui, National Natural Science Foundation of China (62033012) and Ant Group through Ant Research Intern Program.


%%
%% The next two lines define the bibliography style to be used, and
%% the bibliography file.
\bibliographystyle{plain}
\bibliography{main}

%%
%% If your work has an appendix, this is the place to put it.
\appendix

\section{Supplementary materials}
The Octo-Tiger build scripts are available on GitHub\footnote{\url{https://github.com/STEllAR-GROUP/OctoTigerBuildChain}}. For the runs on the A100 node we used the git branch \textit{sycl\_toolchain} of this repository, for the ones on the MI100 node \textit{sycl\_toolchain\_hip}.

\end{document}
\endinput
%%
%% End of file `sample-acmtog.tex'.
