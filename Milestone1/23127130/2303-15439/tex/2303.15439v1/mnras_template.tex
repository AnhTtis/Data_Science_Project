% mnras_template.tex 
%
% LaTeX template for creating an MNRAS paper
%
% v3.0 released 14 May 2015
% (version numbers match those of mnras.cls)
%
% Copyright (C) Royal Astronomical Society 2015
% Authors:
% Keith T. Smith (Royal Astronomical Society)

% Change log
%
% v3.0 May 2015
%    Renamed to match the new package name
%    Version number matches mnras.cls
%    A few minor tweaks to wording
% v1.0 September 2013
%    Beta testing only - never publicly released
%    First version: a simple (ish) template for creating an MNRAS paper

%%%%%%%%%%%%%%%%%%%%%%%%%%%%%%%%%%%%%%%%%%%%%%%%%%
% Basic setup. Most papers should leave these options alone.
\documentclass[fleqn,usenatbib,useAMS]{mnras}

% MNRAS is set in Times font. If you don't have this installed (most LaTeX
% installations will be fine) or prefer the old Computer Modern fonts, comment
% out the following line
\usepackage{newtxtext,newtxmath}
% Depending on your LaTeX fonts installation, you might get better results with one of these:
%\usepackage{mathptmx}
%\usepackage{txfonts}

% Use vector fonts, so it zooms properly in on-screen viewing software
% Don't change these lines unless you know what you are doing
\usepackage[T1]{fontenc}

% Allow "Thomas van Noord" and "Simon de Laguarde" and alike to be sorted by "N" and "L" etc. in the bibliography.
% Write the name in the bibliography as "\VAN{Noord}{Van}{van} Noord, Thomas"
\DeclareRobustCommand{\VAN}[3]{#2}
\let\VANthebibliography\thebibliography
\def\thebibliography{\DeclareRobustCommand{\VAN}[3]{##3}\VANthebibliography}


%%%%% AUTHORS - PLACE YOUR OWN PACKAGES HERE %%%%%

% Only include extra packages if you really need them. Common packages are:
%\usepackage{txfonts}
\usepackage{hyperref}
\usepackage{subcaption}
%\usepackage[utf8]{inputenc}
\usepackage{graphicx}	% Including figure files
\usepackage{amsmath}	% Advanced maths commands
%\usepackage{amssymb}	% Extra maths symbols
\usepackage{tablefootnote}

%%%%%%%%%%%%%%%%%%%%%%%%%%%%%%%%%%%%%%%%%%%%%%%%%%

%%%%% AUTHORS - PLACE YOUR OWN COMMANDS HERE %%%%%
\def\mass{$M_{\rm{\odot}}$}
\def\kms{km~s$^{-1}$}
\def\siii{\ion{Si}{ii}~$\lambda$6\,355}
\def\mgiin{\ion{Mg}{ii}~$\lambda$9\,227}
\def\mgiit{\ion{Mg}{ii}~$\lambda$10\,952}
\def\feiit{\ion{Fe}{ii}~$\lambda$6\,247}
\def\feiif{\ion{Fe}{ii}~$\lambda$6\,456}
\def\farcs{$^{\prime\prime}$}
\def\alerce{ALeRCE}
\def\rein{$\theta_{\rm{Ein}}$}


\def\phd{\phantom{.}}  
\def\phs{\phantom{$-$}}   
\def\phm{\hbox{$\phantom{1}$}} 
\def\phn{\phantom{0}}
\def\phnneg{\phantom{$-$}}
\def\nodata{\phs$\cdots$}
\def\h0{H$_{0}$}

% Please keep new commands to a minimum, and use \newcommand not \def to avoid
% overwriting existing commands. Example:
%\newcommand{\pcm}{\,cm$^{-2}$}	% per cm-squared

%%%%%%%%%%%%%%%%%%%%%%%%%%%%%%%%%%%%%%%%%%%%%%%%%%

%%%%%%%%%%%%%%%%%%% TITLE PAGE %%%%%%%%%%%%%%%%%%%

% Title of the paper, and the short title which is used in the headers.
% Keep the title short and informative.
\title[Lensed SNe in ZTF]{Identifying gravitationally lensed supernovae within the Zwicky Transient Facility public survey}

% The list of authors, and the short list which is used in the headers.
% If you need two or more lines of authors, add an extra line using \newauthor
\author[M. R. Magee et al.]{
M. R. Magee$^{1,2}$\thanks{E-mail: mrmagee.astro@gmail.com}, A. Sainz de Murieta$^{1}$, T. E. Collett$^{1}$, W. Enzi$^{1}$
\\
$^{1}$Institute of Cosmology and Gravitation, University of Portsmouth, Burnaby Road, Portsmouth, PO1 3FX, UK \\ 
$^{2}$Department of Physics, University of Warwick, Gibbet Hill Road, Coventry CV4 7AL, UK \\ 
}


% These dates will be filled out by the publisher
\date{}

% Enter the current year, for the copyright statements etc.
\pubyear{2023}

% Don't change these lines
\begin{document}
\label{firstpage}
\pagerange{\pageref{firstpage}--\pageref{lastpage}}
\maketitle

% Abstract of the paper
\begin{abstract}
Strong gravitational lensing of supernovae is exceedingly rare. To date, only a handful of lensed supernovae are known. Despite their rarity, lensed supernovae have emerged as one of the most promising methods for measuring the current expansion rate of the Universe and breaking the Hubble tension. We present an extensive search for gravitationally lensed supernovae within the Zwicky Transient Facility (ZTF) public survey, covering 12\,524 transients with good light curves discovered during four years of observations. We crossmatch a catalogue of known and candidate lens galaxies with our transient sample and find only one coincident source, which was due to chance alignment. To search for supernovae magnified by unknown lens galaxies, we test multiple methods that have been suggested in the literature, for the first time on real data. This includes selecting objects with extremely red colours and those that appear inconsistent with the host galaxy redshift. In both cases, we find a few hundred candidates, most of which are due to contamination from activate galactic nuclei, bogus detections, or unlensed supernovae. The false positive rate from these methods presents significant challenges for future surveys. In total, 65 unique transients were identified across all of our selection methods that required detailed manual rejection, which would be infeasible for larger samples. Overall, we do not find any compelling candidates for lensed supernovae, which is broadly consistent with previous estimates for the rate of lensed supernovae in the ZTF public survey and the number expected to pass the selection cuts we apply.
\end{abstract}

% Select between one and six entries from the list of approved keywords.
% Don't make up new ones.
\begin{keywords}
	supernovae: general --- radiative transfer 
\end{keywords}

%%%%%%%%%%%%%%%%%%%%%%%%%%%%%%%%%%%%%%%%%%%%%%%%%%

%%%%%%%%%%%%%%%%% BODY OF PAPER %%%%%%%%%%%%%%%%%%

\section{Introduction}
\label{sect:intro}


Strong gravitational lensing occurs when a sufficiently massive foreground galaxy or cluster distorts space-time to such a degree that multiple images of any well-aligned background source can be formed (see e.g. \citealt{treu--10}). Depending on the gravitational potential of the galaxy and overall geometry of the lens-source system these images can appear highly magnified, offering a unique look at high redshift objects in the Universe that may be otherwise unobservable. In addition, time delays between the lensed images, arising from the different paths travelled through the Universe, are sensitive to cosmological distances and hence the rate of expansion or \h0 \citep{refsdal--64}. Time delays are independent of both the cosmic microwave background (CMB; \citealt{planck18--20}) and the local distance ladder \citep{riess--22}, and have therefore been identified as a promising tool for confronting the 5$\sigma$ difference between \h0 measured from the early- and late-Universe.


\par

Measurements of time delays require that the lensed sources are time-varying and observed over sufficiently long timescales (i.e. longer than the time delay itself; \citealt{treu--16}). To date, most time delay measurements have been made using gravitationally lensed active galactic nuclei (AGN; e.g. \citealt{kochanek--06, eulaers--13, bonvin--17, million--20, wong--20}). While significant progress has been made in this area, the stochastic nature of AGN light curves and long observing times required present challenges for precision measurements of \h0. Alternatively, supernovae (SNe) are transient phenomena much more suited to time delay measurements \citep{oguri--19}. The light curves of SNe are generally much simpler than AGN and vary on shorter timescales, requiring less observational overhead. In addition, light from the SN will eventually fade away after a few weeks or months, allowing for detailed follow up and reconstruction of the lensed host galaxy. This is very challenging for the hosts of lensed AGN however as lensed AGN typically outshine the host galaxy.


\par

Gravitationally lensed SNe (glSNe) offer many advantages over lensed AGN, but are currently much rarer. Whilst glSNe were first proposed as a tool to measure \h0 \citep{refsdal--64}, only a small handful of SNe with multiple resolved images have been confirmed. \cite{kelly--15} report the discovery of `SN~Refsdal', a SN at $z = 1.49$ lensed by a massive elliptical at $z = 0.54$ and producing a distinct Einstein cross configuration. Subsequent spectroscopic observations revealed that SN~Refsdal was likely a peculiar type II SN (SN~II; \citealt{kelly--16}). \cite{goobar--17} demonstrate that iPTF16geu was a multiply-imaged type Ia SN (SN~Ia) that occurred at $z = 0.41$ and was magnified in luminosity by a factor of $\gtrsim$50, although the initial discovery images from the intermediate Palomar Transient Facility (iPTF; \citealt{law--09}) were unresolved. \cite{rodney--21} present observations of `SN~Requiem', a glSN at $z = 1.95$ that they argue is consistent with being a SN~Ia. Most recently, \cite{goobar--22} present observations of `SN~Zwicky' a glSN~Ia at $z = 0.35$ discovered by the Zwicky Transient Facility (ZTF; \citealt{ztf, graham--19, masci--19, dekany--20}). A handful of other unresolved glSNe have also been proposed, in addition to some targeted searches proving unsuccessful \citep{amanullah--11, quimby--13, patel--14, rodney--15, petrushevska--16, rubin--18, craig--21}. Aside from the difficulty in observing glSNe, the small sample size of objects may also be affected by our ability to correctly identify them. Even for those objects with spectroscopic observations, recognising glSNe from unresolved images can be challenging \citep{chornock--13, quimby--13, quimby--14}. Therefore the possibility remains that many more glSNe have been observed, but simply were not recognised as such.


\par

The simplest method of identifying glSNe is to monitor known lenses for new SNe (e.g. \citealt{craig--21}), however only a few thousand lens systems have been confirmed. \cite{quimby--14} propose selecting candidate glSNe from unresolved images based on their colours and magnitudes during the rising phase. As glSNe will be observed at higher redshifts than non-lensed SNe, the peak of the spectral energy distribution (SED) will be shifted out of bluer bands in the observer frame, producing redder observed colours. Hence, \cite{quimby--14} suggest searching for SNe that appear significantly redder than expected for a given apparent magnitude. Indeed, they show that the glSN~Ia PS1-10afx was redder around maximum light than non-lensed SNe~Ia by $\gtrsim$1.2~mag in $r - i$. \cite{goldstein--18--lens} discuss an alternative strategy for identifying glSNe.
Almost all SNe that occur in elliptical galaxies are SNe~Ia \citep{li--2011}. In addition, elliptical galaxies dominate the strong lensing cross-section in the Universe (\citealt{auger--09}). Therefore \cite{goldstein--18--lens} suggest that if a SN occurs near an elliptical galaxy, but is inconsistent with being a SN~Ia at the redshift of the elliptical, it could be a glSN at a higher redshift. Using this technique, \cite{goldstein--19} place constraints on the number of glSNe that can be discovered by ZTF and the upcoming Legacy Survey of Space and Time (LSST; \citealt{ivezic--19}), based on simulations of both surveys and populations of glSNe of all types. \cite{goldstein--19} estimate that the depth and coverage of LSST should allow $\gtrsim$340 glSNe to be discovered each year. For ZTF the observational depth is significantly shallower than LSST, however \cite{goldstein--19} estimate $\sim$1 -- 9 glSNe should be discovered per year, depending on which ZTF data is used for the search (whether this includes $i$-band, high-cadence, or only publicly available observations). 

\par

To date, only one glSN has been confirmed within ZTF, SN~Zwicky. As part of the Bright Transient Survey (BTS; \citealt{fremling--20}), ZTF aims to spectroscopically classify all SNe brighter than $m_{g,r} \sim 19$. SN~Zwicky became sufficiently bright to trigger automatic spectroscopic classification, which indicated that it was a SN~Ia at a redshift of $z = 0.35$. The light curve however, was clearly significantly brighter than expected for a SN~Ia at this redshift. Identification of glSNe may therefore also be possible by selecting SNe that appear to be significantly brighter than expected. 


\par

Here we report on a systematic search for lensed SNe within the ZTF public survey using previously suggested identification methods. The construction of our initial sample is described in Sect.~\ref{sect:sample}. We first crossmatch our sample of transients against a list of confirmed or candidate lens systems in Sect.~\ref{sect:known}. In Sect.~\ref{sect:colour_method}, we apply the colour-based identification method presented by \cite{quimby--14} to our sample, while Sect.~\ref{sect:outlier_method} discusses the outlier-based method presented by \cite{goldstein--19}. Section~\ref{sect:luminosity_method} presents a luminosity-based selection method, following the discovery of SN~Zwicky. Finally, we discuss our results in Sect.~\ref{sect:discuss} and conclusions in Sect.~\ref{sect:conclusions}.

%
%______________________________________________________________
%___________________________________________________



\section{The ZTF public sample}
\label{sect:sample}
Since beginning operations in 2018, ZTF observing time has been divided into a series of public and private surveys. The majority of the public time is dedicated to an all-sky survey, reaching typical limiting magnitudes of $\sim$20.5~mag in the $g$- and $r$-bands. During the first phase of operations, ZTF~I, the public survey operated with a typical cadence of 3\,d. Since the end of 2020, the public survey cadence has increased to 2\,d. Further details of the surveys conducted by ZTF are given by \cite{bellm--19}. 

\par

Our initial sample consists of all transients that were observed by ZTF and announced publicly on the Transient Name Server (TNS)\footnote{https://www.wis-tns.org/} from the beginning of the survey through to 2022 March 16, equalling approximately four years of survey operations. In total this resulted in 29\,242 transients, of which only 4\,820 were spectroscopically classified. We compared against the Million Quasars (Milliquas) Catalog v7.5\footnote{https://quasars.org/milliquas.htm} \citep{flesch--21} and the Open Cataclysmic Variable Catalog\footnote{https://depts.washington.edu/catvar/} \citep{guillochon--17} to remove all transients crossmatched within a 1.5\farcs{} radius, leaving 26\,575 transients.


\par

To ensure that our sample contains light curves of sufficient quality, such that they are scientifically useful and can be reliably fit with various models, we only selected objects for which there is sufficient data in both the ZTF $g$- and $r$-bands. For each object in our sample, we queried the \alerce{}\footnote{https://alerce.readthedocs.io/en/latest/index.html} \citep{forster--21} database to retrieve the detections. We selected only those objects with detections in both the $g$- and $r$-bands over at least four separate nights and spanning a period of at least one week. In addition, we also select objects for which the majority of difference image flux detections are positive. This reduced the number of transients to approximately half, leaving 12\,524 objects in our sample. 


\begin{table*}
\begin{center}
\caption{Summary of the cuts applied to the TNS public sample.}
\label{tab:cuts}
\resizebox{\textwidth}{!}{
\begin{tabular}{lccc}
\hline
\textbf{Condition} & \textbf{No. of objects} & \textbf{Fraction} & \textbf{Total fraction} \\
& \textbf{remaining} & \textbf{removed} & \textbf{remaining} \\
\hline
\hline
Starting Sample                              & 29\,242 & -            & 100.000\%                \\
Not within 1.5\farcs{} of known galactic or stellar variable     & 26\,575 & \phn{}9.12\% & \phn{}90.880\%           \\
At least four nights of $g$- and $r$-detections over at least one week & 12\,524 & 52.87\% & \phn{}42.829\% \\
\hline
\multicolumn{4}{c}{Very red transients} \tabularnewline
\hline
Initial sample                                  & 12\,524               & 52.87\%   & \phn{}42.829\% \\
Red colours                                     & \phn{}\phn{}238       & 98.10\%   & \phn{}0.814\% \\
Pre-maximum                                     & \phn{}\phn{}\phn{}32  & 86.55\%   & \phn{}0.109\% \\
\hline
\multicolumn{4}{c}{Light curve inconsistent with SN~Ia at photometric redshift of nearby elliptical galaxy} \tabularnewline
\hline
Initial sample                                          & 12\,524                 & 52.87\%         & \phn{}42.829\% \\
Elliptical within 10\farcs{}                            & \phn{}1\,233            & 90.15\%         & \phn{}\phn{}4.217\% \\
Photo-$z$ for elliptical                                & \phn{}1\,204            & \phn{}2.35\%    & \phn{}\phn{}4.117\% \\
$\geq$1 5$\sigma$ outlier                               & \phn{}\phn{}222         & 81.56\%   & \phn{}\phn{}0.759\% \\
No clear AGN- or stellar-like variability               & \phn{}\phn{}\phn{}82    & 63.06\%   & \phn{}\phn{}0.280\% \\
Outlier not due to spurious detection                   & \phn{}\phn{}\phn{}64    & 21.95\%   & \phn{}\phn{}0.219\% \\
Outlier not due to phases beyond SALT2 template range   & \phn{}\phn{}\phn{}39    & 39.06\%   & \phn{}\phn{}0.133\% \\
No clear spiral arms in archival SDSS or PS1 imaging    & \phn{}\phn{}\phn{}32    & 17.95\%   & \phn{}\phn{}0.109\% \\
\hline
\multicolumn{4}{c}{Intrinsically luminous assuming host spectroscopic redshift} \tabularnewline
\hline
Initial sample                                  & 12\,524               & 52.87\%   & \phn{}42.829\% \\
SDSS spec-$z$ for host                          & \phn{}1\,916                & 84.70\%   & \phn{}\phn{}6.552\% \\
$M_g \leq -21$                                & \phn{}\phn{}\phn{}\phn{}2                     & 99.90\%   &  \phn{}\phn{}0.007\%\\
\hline
\hline
\end{tabular}
}
\end{center}
\end{table*}


%
%______________________________________________________________
%___________________________________________________


\section{Known lenses}
\label{sect:known}

We begin our search for lensed SNe by first identifying transients coincident with confirmed or candidate lens systems. We construct a catalogue of lens systems based on a compilation of literature sources covering a variety of surveys and identification methods \citep{bolton--08, more--12, sonnenfeld--13, gavazzi--14, cao--15b, more--16, diehl--17, shu--17, sonnenfeld--18,  wong--18, jacobs--19a, jacobs--19b, petrillo--19, canameras--20, cao--20, chan--20, huang--20, jaelani--20, li--20, sonnenfeld--20, wong--20, huang--21, li--21, rojas--21, savary--21, shu--22}. We also include all objects for which the convolutional neural networks (CNNs) presented by \cite{rojas--21} and \cite{savary--21} have assigned a $\geq0.5$ probability of being a lens ($\sim$133\,000 and $\sim$9\,200, respectively). In total, after removing duplicates and those outside the ZTF footprint, our compilation includes $\sim$60\,000 systems. 

\par

\begin{figure}
\centering
\includegraphics[width=\columnwidth]{Images/sn2020yfo.pdf}
\caption{Archival Pan-STARRS image surrounding the location of SN~2020yfo/ZTF20aclkyjg, which is marked by a red circle. The crossmatched candidate lens system is given by a blue circle, while the closest potential host is shown in green. Other sources in the Pan-STARRS catalogue are marked by white circles. }
\label{fig:sn2020yfo}
\centering
\end{figure}

Crossmatching against our sample of 12\,524 transients with a generous 10\farcs{} radius, we find only one transient coincident with a candidate lens. The field of ZTF20aclkyjg is shown in Fig.~\ref{fig:sn2020yfo} along with the positions of detected galaxies in the Pan-STARRS (PS1) photometric catalogue DR2 (\citealt{chambers--ps1, flewelling--20}; white circles). The candidate lens is marked in Fig.~\ref{fig:sn2020yfo} with a blue circle and was classified by the CNN presented in \cite{rojas--21} as having a 0.74 probability of being a lens. Using the photometric redshift code presented by \cite{tarrio--20}, we estimate the redshifts of all nearby galaxies. For the candidate lens, we find $z = 0.56\pm0.04$. Assuming a peak magnitude of $m_{{g}} = 18.99\pm0.09$ (and no host extinction), correcting for this distance and Milky Way extinction (A$_{{V}} = 0.055$) this would imply a peak magnitude of $M_{{g}} \sim -23.57\pm0.21$. Alternatively, the galaxy highlighted in green in Fig.~\ref{fig:sn2020yfo} is slightly closer, (at $\sim$8\farcs{} compared to $\sim$9\farcs{} for the candidate lens), and has a photometric redshift of $z = 0.15\pm0.05$. This would imply a peak magnitude of $M_{\rm{g}} = -20.26\pm0.95$. In both cases the uncertainty is dominated by the photometric redshift. ZTF20aclkyjg was also observed spectroscopically and classified as a SN~Ia at $z = 0.126$, which is consistent with the photometric redshift we find for the closest galaxy. Therefore the coincidence of ZTF20aclkyjg and a candidate lens system is likely simply due to chance alignment. 


\par

As surveys push to deeper limits and catalogues of candidate lens systems become more complete, the probability for chance alignment increases. Assuming a radius of 10\farcs{} around each of our candidate lenses, this results in an effective sky area of $\sim$1.5~deg$^2$. The ZTF public survey has a total footprint of $\sim$23\,675~deg$^2$ \citep{bellm--19}, therefore our lens catalogue covers 0.006\% of the survey. Assuming each of the lenses in our catalogue and each of the 12\,524 transients in our sample are randomly distributed, this would indicate that we should find $\sim$1 crossmatched transients, which is consistent with our result.

\par

We also note that another transient, ZTF22abdyjqu, was discovered coincident with a candidate lens after the cut-off date for our sample. ZTF22abdyjqu was separated by $\sim$1.6\farcs{} from DESI-311.4249-10.6762, a B-grade candidate lens with a spectroscopic redshift of $z = 0.6334$ presented by \cite{huang--21}. Follow up observations for classification were triggered, but not completed due to weather. A classification spectrum was obtained by ePESSTO+ \citep{pessto} that confirmed ZTF22abdyjqu as an unrelated foreground SN~Ia at $z = 0.108$ \citep{ZTF22abdyjqu--class}.



%
%______________________________________________________________
%___________________________________________________

\begin{figure*}
\centering
\includegraphics[width=\textwidth]{Images/quimby_plot_ztf_2.png}
\caption{\textit{Panel a: }Colour magnitude diagram for simulated non-lensed transients in the ZTF $g$- and $r$-bands. Shaded regions show a 2D histogram of the number of expected transients of a given type on a log scale. The thick black line shows the upper limit on the 4$\sigma$ probability distribution. A linear fit to this limit is shown as a dashed line. The arrow shows the change in colour and magnitude expected for an extinction of A$_{V} = 1$. \textit{Panel b: }Distribution of colours and magnitudes from our observed sample of public ZTF transients. Observations of our final candidate sample are shown as purple diamonds. We note that only the first and last observation during the rise are shown here. SN~Zwicky is shown as a red star.
}
\label{fig:colour_method}
\centering
\end{figure*}

\section{Very red transients}
\label{sect:colour_method}

Our first selection method is based on the transient colour and follows from \cite{quimby--14}, motivated by the discovery of PS1-10afx. PS1-10afx was initially classified as a superluminous SN (SLSN), due to its high luminosity ($M_{u} \sim -22.3$) and spectral similarities to other SLSNe and SNe~Ic. In contrast, \cite{quimby--13} instead argue PS1-10afx is consistent with being an extremely high redshift ($z = 1.388$) SN~Ia that was magnified by a factor of $\sim$30. Subsequent observations of the host galaxy after the SN had faded revealed the presence of the lens galaxy -- indicating that PS1-10afx was indeed likely a gravitationally lensed SN~Ia \citep{quimby--14}.

\par

Although the initial classification of PS1-10afx was unclear, it immediately stood out from other SNe due to its extremely red colours, as the high redshift had caused most of the rest-frame optical flux to shift out of the observer-frame optical bands. Around maximum light, PS1-10afx showed an $r-i$ colour of $\sim$1.7~mag, compared to $r-i \textless 0.5$ for non-lensed SNe \citep{quimby--14}. Based on this observation, \cite{quimby--14} suggest a colour-based selection method for identifying glSNe. Using Monte-Carlo simulations, they estimate the expected colours for lensed and non-lensed SNe in PS1. As a function of apparent magnitude, they calculate the upper limit on the $r-i$ colour for non-lensed SNe as they rise towards maximum light and suggest selecting candidate glSNe if they are redder than this limit (see their fig.~4). Following \cite{quimby--14}, we apply this method to our search for glSNe in the ZTF filters.



\par

\subsection{Selection method}
To calculate the distribution of colours for non-lensed transients in the ZTF filters, we used the PLAsTiCC templates \citep{kessler--19} and include SNe~Ia, 91bg-likes, SNe~Iax, SNe~II, SNe~Ibc, SLSNe, Ca-rich, and tidal disruption events (TDEs). Following the methods outlined in \cite{kessler--19}, we generate random distributions of each type of transient based on the provided templates, up to a redshift of $z = 0.15$, and calculate light curves in the ZTF $g$- and $r$-bands during the rising phase. Rates for each type of transient were also taken from \cite{kessler--19} and increased by a factor of four. This ensures the relative numbers of transients are consistent with observations, while also better sampling the underlying distributions. Milky Way extinction is not included in our simulations as we assume this can easily be corrected for, while host galaxy extinction is also not included to limit the number of potential glSNe that are discarded. For our selected candidates, we investigate whether host galaxy extinction could have produced the observed red colours (Sect.~\ref{sect:colour_method_candidates}).

\par

Figure~\ref{fig:colour_method}(a) shows the distribution of colours and magnitudes for our simulated non-lensed SNe during the rising phase. As expected, different transient types show different colour ranges. The thermonuclear group (SNe~Ia, 91bg-likes, and SNe~Iax) shows a somewhat bimodal distribution, with SNe~Ia and SNe~Iax typically being relatively blue during their rise ($g-r \lesssim 0.2$). Conversely 91bg-like SNe are generally fainter and redder, with $g-r \gtrsim 0.5$ during the rising phase. Core-collapse SNe (SNe~II and SNe~Ibc) show a broad range of colours from $-0.5 \lesssim g-r \lesssim 1.0$. Finally, for other types of transients (SLSNe, TDEs, and Ca-rich), we also find a broad range of colours. This is mostly due to the Ca-rich transients as SLSNe and TDEs are typically blue during the rise. From these simulations, we define a contour in this colour-magnitude space that encompasses the 4$\sigma$ range of the simulated light curves. This is shown in Fig.~\ref{fig:colour_method} as a thick black line. An approximate linear fit to the upper limit of this distribution is shown by a dashed black line and given by the following functional form:
\begin{equation}
    g-r = 0.3245 \times m_{r} - 4.773.
\end{equation}

\par

For each of the observed 12\,524 transients in our sample, we fit their light curves using Gaussian Processes (GP) to estimate the colour evolution during the rising phase. GPs are commonly used to interpolate transient light curves (e.g. \citealt{inserra--13a,dhawan--18,narayan--18}). Here we used a combination of three squared-exponential  kernels to account for variations of different timescales across transient populations. Colours for the transients in our sample are shown in Fig.~\ref{fig:colour_method}(b) as grey points. We find 238 transients with $g-r$ colours and $r$-band magnitudes outside the 4$\sigma$ range predicted by our simulations, after correcting the colours and apparent magnitudes for Milky Way extinction. Visually inspecting all of the colour outlier candidates we find a large number were flagged as outliers based on showing rising features in the light cure, despite being clearly post-maximum. This includes some SNe~Ia in which the rise around the secondary maximum was identified. In addition some SNe with multiple, minor fluctuations in the light curve caused multiple portions of the light curve to appear to rise and hence were flagged. Finally, some transients clearly showed repeating and/or stochastic variability similar to variable stars or AGN. Removing all of these cases and selecting only those transients that were flagged during an unambiguous rising phase leaves 32 in our sample.




\par


\subsection{Candidates}
\label{sect:colour_method_candidates}
We identify 32 candidates for glSNe within our sample of 12\,524 transients, based on their red colours during the rising phase of the light curve. For each of these candidates, we query NED to identify potential host galaxies nearby with redshift estimates and fit the light curve with various SN templates. This allows us to estimate whether the red colours could plausibly be explained by unlensed transients that are intrinsically red or experience varying degrees of extinction. The majority of these candidates (29) have been spectroscopically classified as supernovae, therefore we also investigate whether these spectroscopic classifications are consistent with the observed light curves. A full list of our identified candidates is given in the appendix in Table~~\ref{tab:colour_outliers}. 

\par

Based on our template fits, we find that four transients were initially flagged due to poorly sampled light curves resulting in the GP fits incorrectly estimating the flux and therefore the colour. An example of this is shown in Fig.~\ref{fig:colour_method:bad_gp_fit} for the case of ZTF18abmxahs, which was spectroscopically classified as a SN~Ia at $z = 0.019$. Figure~\ref{fig:colour_method:bad_gp_fit} shows that ZTF18abmxahs was initially observed shortly after explosion in both the $g$- and $r$-bands, but was not observed again in the $g$-band until around maximum light. Figure~\ref{fig:colour_method:bad_gp_fit} also shows a SALT2 model \citep{guy--07} fit to the light curve of ZTF18abmxahs, which indicates that the poorly sampled $g$-band light curve resulted in our GP fit underestimating the $g$-band flux and hence over-estimating the red colour. Based on our GP fit, we find a $g-r$ colour of $\sim$0.4 -- 1.1 for apparent $r$-band magnitudes of $\sim$15.3 -- 16.8, whereas our SALT2 fit indicates the colour was instead $\sim-0.1$ during the same period. In addition, from our SALT2 fit we find $c = -0.019\pm0.03$ for ZTF18abmxahs, which is typical of normal SNe~Ia. 


\begin{figure}
\centering
\includegraphics[width=\columnwidth]{Images/colour_candidates_ZTF18abmxahs.pdf}
\caption{Light curve of ZTF18abmxahs, which was identified as a colour outlier in our sample. The ZTF $g$- and $r$-band observations are shown as green and red points, respectively. The 1$\sigma$ ranges in the $g$- and $r$-band light curves predicted by our Gaussian Processes fit are shown as shaded regions. A SALT2 model fit is shown as a dashed line and for which $c = -0.019$. }
\label{fig:colour_method:bad_gp_fit}
\centering
\end{figure}

\par

Three transients in our sample were slightly brighter than the 4$\sigma$ range of observed magnitudes covered by our templates and hence were flagged as outliers for this reason. For example, ZTF21abiuvdk occurred at a redshift of $z = 0.0035$ and reached a peak apparent magnitude of $m_{{g, r}} \sim 12.5$.

\par

For the remaining candidates, we find that their light curves are consistent with unlensed transients that are either intrinsically red or experience relatively large amounts of extinction. Figure~\ref{fig:colour_method:colours} shows examples of some of these transients along with the corresponding template fits. Here, we briefly describe our process ruling out strong lensing for each of these SNe and follow a similar process for other candidates not discussed in detail. 

\par




\subsubsection{Spectroscopically classified candidates}


\begin{figure*}
\begin{tabular}{cc}
  \includegraphics[width=\columnwidth]{Images/colour_candidates_ZTF19acnzkph.pdf} &   \includegraphics[width=\columnwidth]{Images/colour_candidates_ZTF21aatyplr.pdf} \\
  \includegraphics[width=\columnwidth]{Images/colour_candidates_ZTF21aaaubig.pdf} &   \includegraphics[width=\columnwidth]{Images/colour_candidates_ZTF21aceehxt.pdf} 
\end{tabular}
\caption{Light curves of various transients that were identified as colour outliers in our sample. The ZTF $g$- and $r$-band observations are shown as green and red points, respectively. The 1$\sigma$ ranges in the $g$- and $r$-band light curves predicted by our Gaussian Processes fit are shown as shaded regions. Template light curve fits are shown as dahsed lines. \textit{Panel a:} ZTF19acnzkph is shown compared to a SALT2 model fit with $c = 0.67$. \textit{Panel b:} ZTF21aatyplr is shown compared to a SALT2 model fit with $c = 1.26$. \textit{Panel c:} ZTF21aaaubig is shown compared to a template fit of SN~2004fe, a SN~Ic, with a host extinction of A$_{{V}} = 0.94$~mag applied. \textit{Panel d:} ZTF21aceehxt is shown compared to a template fit of SN~2004gq, a SN~Ib, with a host extinction of A$_{{V}} = 2.3$~mag applied.}
\label{fig:colour_method:colours}
\end{figure*}

Both ZTF19acnzkph (Fig.~\ref{fig:colour_method:colours}(a)) and ZTF21aatyplr (Fig.~\ref{fig:colour_method:colours}(b)) were spectroscopically classified as SNe~Ia \citep{ZTF19acnzkph--class, ZTF21aatyplr--class}. Around maximum light, ZTF19acnzkph showed a $g-r$ colour of $\sim$0.65 and an apparent magnitude of $m_r \sim 16.4$. ZTF21aatyplr reached a similar peak apparent magnitude, $m_r \sim 16.5$, but showed a relatively flat $g-r$ colour of $\sim$1.35 during its rise. Hence both SNe are outside our colour limit shown in Fig.~\ref{fig:colour_method}, despite being classified as SNe~Ia. For ZTF19acnzkph, \cite{springob--05} provide a spectroscopic redshift of $z = 0.018$ for the nearby host galaxy, which is consistent with the SN spectrum. Assuming this redshift, we find the light curve can be well-fit by the SALT2 model with $x_1 = -0.05\pm0.15$ and $c = 0.67\pm0.04$. Similarly, for ZTF21aatyplr, the nearby host galaxy was also observed spectroscopically by \cite{springob--05}, finding a redshift of $z = 0.008$. Here, we find the light curve is in good agreement with a SALT2 model fit and $x_1 = -1.84\pm0.13$ and $c = 1.26\pm0.04$. Publicly available spectra of both SNe on the TNS also show evidence for strong Na~I~D absorption, indicating they are heavily extincted by dust \citep{poznanski--12}. From our SALT2 fits, and correcting for distance modulus and Milky Way extinction only, we find peak absolute magnitude of $M_B \sim -17.33$ and $-14.48$ for ZTF19acnzkph and ZTF21aatyplr respectively, which are significantly fainter than expected for normal SNe~Ia. Despite both SNe showing large $c$ values, they are well-fit by the SALT2 model and give a standardised luminosity (following the Tripp equation with $\alpha = 0.148$ and $\beta = 3.112$; \citealt{scolnic--22}) generally within the range expected for SNe~Ia, $M_B \sim -19.43$ and $-18.66$ for ZTF19acnzkph and ZTF21aatyplr, respectively. Therefore we find no evidence to suggest that either SN has been strongly lensed and instead both are most likely red, but otherwise relatively normal SNe~Ia.


\par

After reaching a peak magnitude of $m_g \sim 17.2$, ZTF21aaaubig continued to rise in the $r$-band before peaking at $m_r \sim 16.5$. At this point, we find that ZTF21aaaubig had a $g-r$ colour of $\gtrsim 0.6$. Using the SN templates from \cite{vincenzi--19} and the redshift of the apparent host galaxy ($z = 0.009$; \citealt{tully--08}), we find ZTF21aaaubig is consistent with SNe~Ic with $\sim$0.9 -- 1~mag of extinction applied, which also agrees with the spectroscopic classification \citep{ZTF21aaaubig--class}. In Fig.~\ref{fig:colour_method:colours}(c) we show a template fit comparing ZTF21aaaubig to the SN~Ic SN~2004fe. Assuming 1~mag of extinction and a peak $g$-band magnitude of $m_g \sim 17.2$, we find an absolute peak magnitude of $M_g \sim -17.04$, which is within the range expected for SNe~Ic \citep{kessler--19}. The high resolution classification spectrum taken approximately one week before $g$-band maximum also showed evidence for strong Na~I~D absorption \citep{ZTF21aaaubig--class}. Correcting the light curve and colour of ZTF21aaaubig for an extinction of A$_V$ = 1~mag results in values consistent with our simulations shown in Fig.~\ref{fig:colour_method:colours}. 

\par

Finally, we show the light curve of ZTF21aceehxt in Fig.~\ref{fig:colour_method:colours}(d). The colour evolution of ZTF21aceehxt was relatively flat towards peak, but eventually was flagged as an outlier as it became brighter. At this point ZTF21aceehxt had an apparent magnitude of $m_r \sim 17.25$ and a $g-r$ colour of $\gtrsim0.9$. Again using the SN templates provided by \cite{vincenzi--19} and the redshift of the apparent host ($z = 0.013$; \citealt{huchra--12}), we find ZTF21aceehxt is consistent with SNe~Ib light curves, agreeing with the spectroscopic classification \citep{ZTF21aceehxt--class}. In this case our model fits indicate a large amount of extinction is required relative to the templates ($\sim$1.8 -- 2.4~mag). The publicly available classification spectrum of ZTF21aceehxt also shows evidence of strong Na~I~D absorption \citep{ZTF21aceehxt--class}. For ZTF21aceehxt, assuming a peak $g$-band magnitude of $m_g \sim18.25$ this results in a peak absolute magnitude of $M_g \sim -15.52$. Correcting for an extinction of A$_V$ = 2~mag produces a peak magnitude of $M_g \sim -17.94$. With this correction, and including the uncertainty on the colour, we find that ZTF21aceehx is within the range predicted by our simulations shown in Fig.~\ref{fig:colour_method}.












\par
\subsubsection{Unclassified candidates}

\begin{figure*}
\begin{tabular}{cc}
  \includegraphics[width=\columnwidth]{Images/colour_candidates_ZTF18aacsudg.pdf} &   \includegraphics[width=\columnwidth]{Images/colour_candidates_ZTF20abuovvw.pdf}
\end{tabular}
\caption{\textit{Panel a: }Light curve of ZTF18aacsudg compared against a template fit of SN~1987A with $A_V = 0.56$ (dashed lines). \textit{Panel b: }Light curve of ZTF20abuovvw compared against a template fit of SN~2011bm with $A_V = 0.95$ (dahsed lines). In both panels the 1$\sigma$ ranes in the $g$- and $r$-band light curves predicted by our Gaussian Processes fits are shown as shaded regions.}
\label{fig:colour_method:class}
\end{figure*}


Three of the candidates in our sample were not spectroscopically classified: ZTF18aacsudg, ZTF19abguibf, and ZTF20abuovvw. For ZTF19abguibf, the light curve shows a relatively long rise time of more than one month. Crossmatching against SIMBAD\footnote{http://simbad.cds.unistra.fr/simbad/}, we find ZTF19abguibf is coincident with a previously identified Mira variable star. 

\par

ZTF18aacsudg shows a $g-r$ colour of $\sim$1.34 around peak, $m_r \sim 18.00$. A nearby galaxy (offset by only 0.35\farcs{}) observed spectroscopically by SDSS provides a redshift of $z = 0.025$ \citep{sdss--dr13}. Assuming this redshift and again using templates from \cite{vincenzi--19}, we find ZTF18aacsudg to be consistent with the light curve of SN~1987A and a host extinction of A$_{{V}} \sim 0.56$ (Fig.~\ref{fig:colour_method:class}(a)). Even correcting for this extinction ZTF18aacsudg remains an outlier relative to our simulations, but we note that PLAsTiCC did not include any 87A-like tempaltes specifically and therefore our simulations also do not include 87A-like SNe. Assuming a host extinction of A$_{{V}} = 0.56$ and a peak apparent magnitude of $m_g = 19.30$ we find a peak absolute magnitude of $M_g = -15.89$, which is comparable to other 87A-like SNe \citep{taddia--12}. 

\par 

Finally, our colour-based method also identified ZTF20abuovvw as an outlier around $r$-band maximum ($m_r \sim 19.0$) due to a $g-r$ colour of $\sim$1.3. ZTF20abuovvw is situated close to a nearby galaxy ($\sim$0.43\farcs{}) with an SDSS photometric redshift of $z = 0.061\pm0.018$ \citep{sdss--dr13}. Using this redshift, we find ZTF20abuovvw to be consistent with templates of core-collapse SNe (SNe~II and Ic) and $\sim$0.9 -- 2~mag of extinction, depending on the SN type. In addition, the early light curve of ZTF20abuovvw shows possible signs of early shock-breakout, which may indicate a massive star origin (Fig.~\ref{fig:colour_method:class}(b)). Based on the template fit of SN~2011bm shown in Fig.~\ref{fig:colour_method:class}(b), we find a best-fitting model with $z = 0.070$, $A_V = 0.95$, and peak $m_g = 19.99$. Therefore our model fit indicates a peak absolute magnitude of $M_g = -18.85$, which is relatively high for core collapse SNe, but within the range of SNe~Ibc templates from \cite{kessler--19}. 



\subsection{Summary}
Using simulations across a range of redshifts, we estimated the $g-r$ colours and $r$-band apparent magnitudes  for different classes of non-lensed SNe. Starting from an initial sample of 12\,524 transients observed by the ZTF public survey, we identified any transients with colours and observed magnitudes outside the range predicted by our simulations, finding 238. After removing those objects that were not clearly pre-maximum, we find 32 candidates for glSNe. Fitting each one with various SN templates, we find all of our candidates are consistent with non-lensed transients. In Sect.~\ref{sect:limits} we discuss limitations of this method. 

%
%______________________________________________________________
%___________________________________________________


\section{Light curve inconsistent with SN~Ia at photometric redshift of nearby elliptical galaxy}
\label{sect:outlier_method}


\cite{goldstein--18--lens} suggest a method of identifying candidates based on model fits to their light curves and selecting those that are incompatible with being SNe~Ia at the apparent host galaxy redshift. They argue that most strong gravitational lenses in the Universe are massive elliptical galaxies and that most SNe that occur in elliptical galaxies are SNe~Ia. Given that most glSNe will be discovered in systems for which there are no lens arcs easily observed, the glSN will therefore appear to be hosted by the lens galaxy. Hence if a SN appears to be hosted by an elliptical galaxy, but is not a SN~Ia at that redshift, it could be a glSN.

\par

Using a series of SN templates, \cite{goldstein--19} generate ZTF light curves for glSNe. For each glSN they fit the light curve using the standard SALT2 model at the redshift of the apparent host galaxy, in other words the strong gravitational lens in their simulations. They show that this results in significant outliers relative to the SALT2 model, due to the inconsistent redshift, and therefore they can reliably select candidate glSNe. We apply this same selection method to our sample of 12\,524 ZTF transients.


\subsection{Selection method}
\label{sect:outlier_method:select}
The method used by \cite{goldstein--19} requires selecting SNe that appear to be hosted by elliptical galaxies. \cite{goldstein--19} assume a complete catalogue of elliptical galaxies and hence all glSNe in their simulations pass this initial cut. In the present work, we do not have complete catalogue and therefore require some additional selection cuts. To identify elliptical galaxies, we use the selection criteria outlined by \cite{irani--22}, which were used to select core-collapse SNe observed by ZTF that occurred in elliptical galaxies. This method is based on photometry from the GALEX Data Release 8/9 (\citealt{martin--05}; FUV and NUV bands) and the ALLWISE catalogue (\citealt{wright--10}; W2 and W3 bands), in addition to the PS1 photometric catalogue DR2 ($r$-band). If a galaxy satisfies either of the criteria given below, it is selected as likely being elliptical.
\begin{itemize}
    \item If photometry is available in all of the $W2$-, $W3$-, $NUV$-, and $r$-bands: $W2 - W3 \leq 0.5$ and $NUV -$ r $\geq$ 3.
    \item If photometry is available in only the $W2$ and $W3$ bands: $W2 - W3 \leq 0.3$.
\end{itemize}
For the objects in our sample, we query the required catalogues and find 1\,233 are within 10\farcs{} of an elliptical galaxy. To reduce the number of contaminants from non-ellipticals, in cases where only upper limits on the colour are available we include only those galaxies for which the upper limit is within the criteria outlined above. For 95\% of the objects selected here, the elliptical galaxy is the closest galaxy within 10\farcs{}. In addition, we find that 1.5\% of the objects in our sample are within 10\farcs{} of two elliptical galaxies, while the rest are coincident with one. 




\par

Having identified the objects in our sample which could be hosted by elliptical galaxies, we now look to establish whether they are consistent with a SN~Ia at the required redshift. During their analysis, \cite{goldstein--19} also assume that each catalogued elliptical galaxy has a secure photometric redshift. For this work, we again use the photometric redshift code presented by \cite{tarrio--20} and photometry from the PS1 DR2 to estimate redshifts for each of the elliptical galaxies in our sample. Due to non-detections in multiple PS1 bands, we were unable to estimate redshifts for 29 potential hosts in our sample.


\par

For the remaining 1\,204 objects in our sample, we follow \cite{goldstein--18--lens} and fit the light curves of each object with the SALT2 template and the redshift of the potential elliptical host galaxy (or galaxies in the case of objects with more than one nearby elliptical galaxy). Unlike \cite{goldstein--18--lens}, who assumed a precise redshift was known, here we allow the redshift to vary across the 3$\sigma$ range predicted by the \cite{tarrio--20} photometric redshift code. As in \cite{goldstein--18--lens}, we also require $|x_1| \leq 1$ and $|c| \leq 0.2$. Using this method we find 222 objects in our sample have at least one 5$\sigma$ outlier from the best-fit SALT2 model across the 3$\sigma$ redshift range.

\par 

Following visual inspection of our model outlier candidates, we again find multiple objects with long, repeating, and/or stochastic variability consistent with stellar variables or AGN. Removing each of these objects leaves 82 in our sample. We also find a number of candidates were flagged as outliers due to likely spurious or bogus detections. We note that each detection from \alerce{} also includes a real/bogus score to identify spurious detections, however we found a number of cases where clearly bogus detections were flagged as real and vice-versa, and therefore we choose not to implement a cut based on this score. We therefore rely on visual inspection to remove objects that were identified as candidates based on spurious outliers. This leaves 64 in our sample. For 25 candidates, we find their light curves are consistent with normal, but well-observed SNe~Ia. In these cases, the observations extend to phases beyond the temporal coverage of the SALT2 template (i.e. $\textgreater+50\,d$) and therefore were flagged as outliers. Removing these observations and fitting only the light curve within $-15$\,d $\leq t_0 \leq 50$\,d we find no outliers for any candidate and therefore remove them, leaving 39 in our sample. Finally, we visually inspect the host galaxies of each of the remaining candidates and remove any objects for which spiral arms are clearly identifiable in archival PS1 or SDSS imaging, leaving 32 candidates. 



\iffalse

\begin{figure*}
\begin{tabular}{cc}
  \includegraphics[width=\columnwidth]{Images/outlier_candidates_ZTF18aavsilo.pdf} &   
  \includegraphics[width=\columnwidth]{Images/outlier_candidates_ZTF21aavqphe.pdf}
\end{tabular}
\caption{\textit{Panel a: }Light curve of ZTF18aavsilo compared against our best-fitting SALT2 model (dashed lines). A (likely bogus) detection at MJD = 58969.31 resulted in ZTF18aavsilo being flagged as a significant outlier. 
\textit{Panel b: }Light curve of ZTF21aavqphe compared against our best-fitting SALT2 model (dashed lines). Observations of ZTF21aavqphe extend far beyond the phase range of the SALT2 model, resulting in it being flagged as an outlier}
\label{fig:outlier_method:junk}
\end{figure*}
\fi

\subsection{Candidates}
We identify 32 candidates for glSNe within our sample of 12\,524 transients, based on having at least one 5$\sigma$ outlier relative to a SALT2 fit at the redshift of the nearby elliptical galaxy. We follow a similar method as in Sect.~\ref{sect:colour_method_candidates} and fit the light curve of each transient with various SN templates to determine whether they are consistent with being another SN type, rather than a SN~Ia. Again, we find that the majority (22) of our candidates have been spectroscopically classified. A full list of our identified candidates is given in the appendix in Table~~\ref{tab:elliptical_outliers}. 

\par

\subsubsection{Spectroscopically classified candidates}
Of those candidates that have been classified, 15 were classified as SNe~Ia. For these SNe, we extend the boundaries used in our SALT2 fits to $|x_1| \leq 3$ and $|c| \leq 0.3$, which are within the limits of typical cosmological samples  (e.g. \citealt{dhawan--22}). With these looser cuts, we find 7 out of these 15 spectroscopically classified SNe~Ia can be removed as outliers. Two SNe (ZTF21aaoekcj and ZTF21acciklh) are well-matched by 91bg-like templates. The remaining six outliers were flagged for multiple reasons. 

\par

\begin{figure}
\centering
\includegraphics[width=\columnwidth]{Images/outlier_candidates_ZTF18acbvgqw.pdf}
\caption{Light curve of ZTF18acbvgqw, which shows late-time fluctuations in the $r$-band light curve causing it to be flagged as an outlier. }
\label{fig:outlier_method:ZTF18acbvgqw}
\centering
\end{figure}

ZTF18acbvgqw shows some late-time (likely bogus) fluctuations in the $r$-band light curve (see Fig.~\ref{fig:outlier_method:ZTF18acbvgqw}). For ZTF18acslpba there is a single $\sim$5$\sigma$ outlier around the secondary maximum, but otherwise the SN is well-matched by the extended SALT2 boundaries. We found no suitable matches to the light curve of ZTF20acqntkr, however the spectrum taken shortly after discovery is consistent with a SN~Ia a few months after peak. Therefore ZTF20acqntkr is likely the late-time tail of a SN~Ia and hence outside the phase range of the SALT2 model. ZTF20acvziuf is not well matched with either extended SALT2 boundaries or 91bg-like templates, shows a weak secondary maximum, and a peak absolute magnitude of $M_g = -18.54$. We therefore speculate that this is a transitional SN~Ia.

 \par


\begin{figure*}
\begin{tabular}{cc}
  \includegraphics[height=6.5cm]{Images/outlier_candidates_ZTF21abfxibf.pdf} &   
  \includegraphics[height=6.5cm]{Images/outlier_candidates_ZTF21abfxibf_coords.pdf}
\end{tabular}
\caption{\textit{Panel a: }Light curve of ZTF21abfxibf, which was spectroscopically classified as a SN~Ia, showing an initial decline in the $r$-band approximately 150\,d before maximum light. 
\textit{Panel b: }Coordinates for each detection (circles) of ZTF21abfxibf given relative to the elliptical host galaxy (black star). Median coordinates for the early and late sections of the light curve are given by filled squares. }
\label{fig:outlier_method:ZTF21abfxibf}
\end{figure*}

For ZTF21abfxibf, the light curve shows evidence of two peaks (Fig.~\ref{fig:outlier_method:ZTF21abfxibf}). The main peak occurred around MJD = 59\,541.71 and is well-fit by the SALT2 model, however additional detections $\sim150$~days earlier show a clear decline in the $r$-band from $m_r$ = 18.55 -- 19.29 over $\sim$20\,d.  Taking the median coordinates of both the early and late detections, we find that they are offset by 0.41\farcs{}, which is comparable to the offset between the reported coordinates of ZTF21abfxibf and the host centre (0.39\farcs{}; Fig.~\ref{fig:outlier_method:ZTF21abfxibf}). Forced photometry at the location of ZTF21abfxibf in the ZTF DR5 also shows the early detections are generally consistent with a relatively flat light curve before the onset of the SN. Therefore, the outlier detections for ZTF21abfxibf are likely due to unrelated nuclear activity. Finally, the light curve of ZTF21acfabut is generally consistent with a SALT2 fit, but shows some strong deviations at early times (ie. more than two weeks before maximum light). These deviations could be indicative of an early flux excess (e.g. \citealt{magee--20}), however the poorly sampled nature of the light curve around this time makes a definitive statement difficult. 

\par

For the remaining seven spectroscopically classified transients identified as outliers, four were classified as TDEs (ZTF18aabdajx, ZTF20abfcszi, ZTF20acitpfz, and, ZTF21abcgnqn). In Fig.~\ref{fig:outlier_method:TDEs}, we show the light curves of each object along with fits to the decline, assuming a $t^{-5/3}$ power-law \citep{rees--88}. In general, we find good agreement with the fits. We also note that all of these objects show relatively blue colours throughout their evolution, which is unlike most normal SNe and not expected for glSNe at high redshift. 

\begin{figure}
\centering
\includegraphics[width=\columnwidth]{Images/outlier_candidates_tde.pdf}
\caption{Light curves of spectroscopically confirmed TDEs flagged as outliers in arbitrary flux units. Dashed lines show fits to the decline assuming a $t^{-5/3}$ powerlaw. Unfilled points are not included in the fit. }
\label{fig:outlier_method:TDEs}
\centering
\end{figure}

\par


\begin{figure*}
\begin{tabular}{cc}
  \includegraphics[width=\columnwidth]{Images/outlier_candidates_ZTF21abptxfk.pdf} &   
  \includegraphics[width=\columnwidth]{Images/outlier_candidates_ZTF21accwovq.pdf}
\end{tabular}
\caption{\textit{Panel a: }Light curve of ZTF21abptxfk compared against a template fit of the SN~II SN~2005gi. 
\textit{Panel b: }Light curve of ZTF21accwovq compared against a template fit of the SN~IIn SN~2011ht. }
\label{fig:outlier_method:classified_outliers}
\end{figure*}

\begin{figure*}
\begin{tabular}{cc}
  \includegraphics[width=\columnwidth]{Images/outlier_candidates_ZTF20abxyims.pdf} &   
  \includegraphics[width=\columnwidth]{Images/outlier_candidates_ZTF19aayvyeo.pdf}
\end{tabular}
\caption{\textit{Panel a: }Light curve of ZTF20abxyims, for which we are unable to find a suitable template match. Given the long duration and inferred high peak luminosity, ZTF20abxyims may be an unclassified SLSN. The relatively blue colour of the light curve indicates it is likely to result from a glSN.
\textit{Panel b: }Light curve of ZTF19aayvyeo. Circles show (corrected) alert photometry while forced photometry from ZTF DR15 are shown as diamonds. }
\label{fig:outlier_method:unclassified_outliers}
\end{figure*}

Two spectroscopically classified SNe~II were flagged as outliers in our sample. For ZTF20abzcefc, this most likely arises from a mis-identification of the host galaxy. To build our sample, we conservatively selected all objects within 10\farcs{} of an elliptical regardless of whether or not it was the closest galaxy. In the case of ZTF20abzcefc, the nearby elliptical was separated by $\sim$7.7\farcs{}, however a closer and lower redshift galaxy at $\sim$4.1\farcs{} is likely the true host, and would not have passed our colour selection criteria. Our photometric redshift for this galaxy is also consistent with the spectroscopic redshift of the SN ($z = 0.109\pm0.049$ compared to $z = 0.056$). ZTF21abptxfk was also identified as a SN~II in our sample close to an elliptical galaxy. In this case however, the potential elliptical is the closest galaxy and well within the colour cuts defined in Sect.~\ref{sect:outlier_method:select} ($W2 - W3 = -0.875$). The light curve shows strong similarities to SNe~II, which is consistent with the spectroscopic classification. In Fig.~\ref{fig:outlier_method:classified_outliers}(a) we show a comparison against the template of the SN~II SN~2005gi \citep{kessler--10}. No clearly identifiable spiral arms are visible in either PS1 or SDSS archival imaging, however we note that the host is classified as a star in the SDSS catalogue. Given the similarities to SNe~II light curves, ZTF20abzcefc may be one of the few SNe~II host in elliptical galaxies \citep{irani--22}. 

\par

Finally, ZTF21accwovq was spectroscopically classified as a SLSN. Although we are unable to find a perfect match using the \cite{vincenzi--19} templates, in Fig.~\ref{fig:outlier_method:classified_outliers}(b) we compare to SN~2011ht and find somewhat reasonable agreement. ZTF21accwovq shows a relatively long rise time, increasing in brightness by $\sim$1~mag over $\sim$50\,d. The next two weeks show a sharp decline of $\sim$1~mag that is not present in the templates. In addition, the host colour shows a relatively large uncertainty ($W2 - W3 = -0.01\pm0.37$) that is also consistent with falling outside our cuts.

\par

\subsubsection{Unclassified candidates}



The remaining ten objects flagged as outliers were not spectroscopically classified. ZTF18abzrsuh is within $\sim$2.2\farcs{} of a blazar identified in the Milliquas catalog (our previous crossmatching cut removed any objects within 1.5\farcs{}). From fitting templates to the light curves of ZTF18aasvknh and ZTF18acqywlx, we are unable to find satisfactory matches. Based on their long and relatively flat light curves however, we identify these candidates as also being likely AGN. In addition, while all three objects passed our initial colour cuts to identify elliptical galaxies, the relatively large uncertainties on the colours of ZTF18abzrsuh and ZTF18acqywlx mean they are also consistent with falling outside our cuts. 

\par

The light curve of ZTF18aavxiih is reasonably well matched by SNe~II (Fig.~\ref{fig:outlier_method:ZTF18aavxiih}). Including the uncertainty on the host galaxy colour, it is also consistent with being excluded by our colour cuts and therefore may not be a true elliptical. ZTF20acwjnux (Fig.~\ref{fig:outlier_method:ZTF20acwjnux}) and ZTF21aaanpyy (Fig.~\ref{fig:outlier_method:ZTF21aaanpyy}) also show good agreement with the light curves of SNe~II, however in both cases the host galaxy colours are well within our limits. ZTF20abxyims is offset by $\sim$0.6\farcs{} from the nearby host, for which we find a photometric redshift of $z = 0.208\pm0.034$. The colour of the host is marginally consistent with passing our cuts ($W2 - W3 = 0.30$). Here, we are unable to find suitable template matches. The light curve shows indications of a peak $M_g \lesssim -20.70$ around MJD = 59\,126 followed by a long decline lasting $\gtrsim$120\,d (Fig.~\ref{fig:outlier_method:unclassified_outliers}(a)). Given the relatively blue colour, ZTF20abxyims is unlikely to be a high redshift SN that has been lensed and instead we speculate that this is an unclassified SLSN. For ZTF19aayvyeo we are also unable to find suitable template matches (Fig.~\ref{fig:outlier_method:unclassified_outliers}(b)). The alert light curve shows a relatively flat plateau $m_{g,r} \sim 19.4$, before beginning to decline around MJD = 58\,770. Forced photometry at the location of ZTF19aayvyeo from ZTF DR15\footnote{\url{https://www.ztf.caltech.edu/ztf-public-releases.html}} shows historic variability, therefore we classify it as a likely AGN. We also note ZTF19aayvyeo is coincident with a source in the Gaia DR3 Part 4 catalogue \citep{gaia_dr3_4_var} classified as an AGN. 





\par

\begin{figure*}
\begin{tabular}{cc}
  \includegraphics[width=\columnwidth]{Images/outlier_candidates_ZTF18abdgwvs.pdf} &   
  \includegraphics[width=\columnwidth]{Images/outlier_candidates_ZTF19aawhagd.pdf}
\end{tabular}
\caption{\textit{Panel a: }Light curve of ZTF18abdgwvs. Two peaks, separated by more than two years, are clearly apparent, causing it to be flagged as an outlier. Dashed lines show SALT2 fits to each individual light curve.
\textit{Panel b: }Light curve of ZTF19aawhagd. As in \textit{Panel a}.}
\label{fig:outlier_method:siblings?}
\end{figure*}

For the remaining two unclassified objects, ZTF18abdgwvs and ZTF19aawhagd, we find their light curves show evidence of multiple peaks, which could be indicative of sibling pairs (i.e. multiple SNe in the same galaxy; \citealt{graham--22}) or additional AGN activity. We note however that the photometric uncertainties for observations of both objects are relatively large, making a definite identification difficult. In the case of ZTF18abdgwvs, we find a photometric redshift for the elliptical of $z = 0.122\pm0.027$. Fitting each light curve with the SALT2 model at this redshift, we find both are consistent with SNe~Ia although the uncertainty on fitted parameters are relatively large (Fig.~\ref{fig:outlier_method:siblings?}(a)). The first light curve peaks at MJD = 58\,303.46$\pm$1.65 with $x_1 = 1.74\pm0.91$ and $c = 0.03\pm0.02$, while the second peaks at MJD = 59\,126.12$\pm$0.70 with $x_1 = 1.92\pm0.71$ and $c = 0.05\pm0.06$. While we cannot specifically rule out AGN activity, the median coordinates for both light curves are also separated by $\sim$0.89\farcs{}, indicating these may indeed be two separate events. We also note that given the large uncertainties, both light curves are generally consistent with having the same SALT2 parameters and therefore are qualitatively consistent with expectations for glSNe -- the same SN reappearing with some time delay. As both light curves can be fit with the likely host galaxy redshift and show relatively blue colours, this is unlikely to be the case. In addition, such a significant time delay (more than 2 years) would likely only result from very high-mass lensing systems and archival imaging shows no indication of any lensing features, which may be expected in this case.

\par

Our final candidate, ZTF19aawhagd also shows evidence of multiple light curve peaks (Fig.~\ref{fig:outlier_method:siblings?}(b)). NED provides a spectroscopic redshift of $z = 0.0116$ for the nearby galaxy. The first light curve is consistent with SALT2 at this redshift and peak MJD = $58\,645.02\pm0.53$, $x_1 = 1.33\pm0.72$, and $c =  0.14\pm0.07$. For the second light curve, we find best-fitting SALT2 parameters of peak MJD = $59\,017.01\pm0.96$, $x_1 = 1.25\pm1.05$, and $c = 0.16\pm0.04$. In this case, the median coordinates of the two light curves are separated by only $\sim$0.19\farcs{}, indicating it could be one event, but the $\sim$1\farcs{} from the host galaxy centre makes AGN activity somewhat more unlikely. Again, the large uncertainties for best-fitting parameters mean that ZTF19aawhagd is also generally consistent with expectations for the same SN reappearing with a $\sim$1 year time-delay. As with ZTF18abdgwvs however, based on the colour and long delay time required we find a glSN origin to be unlikely. We note that \cite{graham--22} searched for sibling pairs within the ZTF Bright Transient Survey (BTS; \citealt{fremling--20}). Neither ZTF18abdgwvs nor ZTF19aawhagd reached the required threshold for automatic spectroscopic classification and hence were not included in BTS, however their identification shows that ZTF may contain additional previously unidentified sibling pairs due to being categorised as a single transient.



\subsection{Summary}
We identify transients observed by ZTF with nearby elliptical galaxies, following the criteria outlined by \cite{irani--22}. From our initial sample of 12\,524 transients, we find 1\,204 close to elliptical galaxies with photometric redshifts. Fitting each one with a SN~Ia template at the redshift of the elliptical, we find 222 with significant outliers relative to the model fits. The majority of these resulted from AGN, spurious detections, or SNe~Ia extending beyond the SALT2 phase range. Two transients show evidence of multiple light curves, which are qualitatively consistent with expectations for glSNe, however they are reasonably well-matched by SNe~Ia at the redshift of the host galaxy or AGN and are unlikely to be lensed.

%
%______________________________________________________________
%___________________________________________________


\section{Intrinsically luminous assuming host spectroscopic redshift}
\label{sect:luminosity_method}
Our final method of searching for glSNe is based on the discovery of iPTF16geu \citep{goobar--17} and SN~Zwicky \citep{goobar--22} from unresolved, lensed images. Both SNe became sufficiently bright to trigger spectroscopic classification, which showed them to be high redshift SNe that had been highly magnified. Spectra of iPTF16geu showed it to be consistent with a SN~Ia at $z = 0.41$, however the peak apparent magnitude of $m_B = 19.12$ implied an absolute magnitude of $M_B \lesssim -22.48$. For SN~Zwicky, spectra showed a SN~Ia at $z = 0.35$ and peak apparent magnitude of $m_B = 18.49$, implying an absolute magnitude of $M_B \lesssim -22.78$. Such high peak luminosities are not expected for SNe~Ia and therefore indicated that both objects had been significantly magnified. Here, we look for similar objects with inferred high peak luminosities among our sample of 12\,524 transients.


\par

\subsection{Selection method}
To identify transients that show evidence of high peak luminosities we first apply the directional light radius method \citep{sullivan--06, gupta--16} to select the most likely host galaxy from PS1 for each object in our sample. We again use the photometric redshift code from \cite{tarrio--20} and photometry from the PS1 DR2 to estimate redshifts. Assuming the 1$\sigma$ lower-bound on the photometric redshift of the host galaxy, we select any transient with a peak magnitude $M_g \leq -21$ (typically used as a cut-off to identify SLSNe; \citealt{gal-yam--12}). In total, this gives 778 candidates. The majority of these photometric redshifts have large uncertainties (median $\Delta z = 0.14$) and most were likely flagged due to over-estimated redshifts. Visually inspecting and classifying each of these candidates is beyond the scope of this work, but we note that this highlights one of the challenges involved with using photometric redshifts for identifying glSNe. Photometric redshifts are generally reliable for elliptical galaxies, due to the 4\,000~\AA\, break, but are more challenging for spiral galaxies. Simply selecting any object for which the photometric redshift indicates a high peak luminosity will likely result in an over-whelming number of false-positives, particularly in the era of LSST when the number of transients discovered will significantly increase.

\par

To reduce the number of false-positive contaminants, we repeat the previous search instead using SDSS spectroscopic redshifts for the most likely host galaxy. From our sample of 12\,524 transients we find only 1\,916 have spectroscopic redshifts for their host galaxies and of these we find just 2 with inferred peak magnitudes $M_g \leq -21$.


\subsection{Candidates}
Based on the spectroscopic redshift of the host galaxy, we identify two candidate glSNe in our sample due to the inferred high peak luminosity ($M_g \leq -21$). Our first candidate, ZTF20aaaweke, is offset by $\sim$1\farcs{} from the centre of an emission-line galaxy at $z = 0.108$. Given a peak apparent magnitude of $m_g \sim 17.26$ and correcting for MW extinction, this gives a peak absolute magnitude of $M_g \lesssim -21.35$. ZTF20aaaweke was spectroscopically classified as a SN~IIn at the redshift of the host, indicating it is not a glSN \citep{ZTF20aaaweke--class}. 

\par


\begin{figure*}
\begin{tabular}{cc}
  \includegraphics[width=\columnwidth]{Images/outlier_candidates_ZTF21aaxxdpa.pdf} &   
  \includegraphics[width=\columnwidth]{Images/outlier_candidates_ZTF21aaxxdpa_2.pdf}
\end{tabular}
\caption{\textit{Panel a: }Light curve of ZTF21aaxxdpa compared to a SALT2 model fit at $z = 0.565$. We find best-fitting SALT2 parameters of $x_1 = 1.30\pm1.40$ and $c = -0.39 \pm 0.05$, with $\chi^2 = 1.660$.
\textit{Panel b: }As in \textit{Panel a} for a redshift of $z = 0.1$. Here, we find best-fitting SALT2 parameters of $x_1 = 0.65\pm3.18$ and $c = -0.007\pm0.080$, with $\chi^2 = 1.521$.}
\label{fig:ZTF21aaxxdpa}
\end{figure*}

Our final candidate, ZTF21aaxxdpa, is offset by $\sim$12.71\farcs{} from a relatively red galaxy at $z = 0.565$. Assuming this is the correct host galaxy and an apparent peak magnitude of $m_g \sim 18.88$, correcting for MW extinction we find $M_g \lesssim -23.81$. In Fig.~\ref{fig:ZTF21aaxxdpa}(a) we show the light curve of ZTF21aaxxdpa with a SALT2 model fit, assuming $z = 0.565$. As shown by Fig.~\ref{fig:ZTF21aaxxdpa}(a), to match the observed colour of ZTF21aaxxdpa with a SN~Ia at $z = 0.565$, we require an intrinsically very blue SN with $c = -0.39 \pm 0.05$ (and $x_1 = 1.30\pm1.40$). Alternatively, a closer galaxy offset by $\sim$2.5\farcs{} may instead be the correct host. This source does not appear in the PS1 catalogue, hence the potential mis-identification, but SDSS provides a photometric redshift of $z = 0.378\pm0.154$. If ZTF21aaxxdpa is a SN~Ia, a redshift of $z \sim 0.1$ would produce an absolute magnitude consistent with the range observed among SNe~Ia, and is within 1.8$\sigma$ of the photometric redshift predicted for the closer SDSS galaxy. A SALT2 fit assuming this redshift is shown in Fig.~\ref{fig:ZTF21aaxxdpa}(b), for which we find $x_1 = 0.65\pm3.18$ and $c = -0.007\pm0.080$. Again, the poorly sampled light curve and large uncertainties make a definitive classification difficult, however given the blue colour and satisfactory fits at lower redshift, which are consistent with the SDSS photometric redshift for a closer galaxy, we find ZTF21aaxxdpa is unlikely to be a glSN at $z = 0.565$ and instead likely results from chance alignment.


%
%______________________________________________________________
%___________________________________________________



























%
%______________________________________________________________
%___________________________________________________

\section{Discussion}
\label{sect:discuss}


\subsection{Combined methods}
\label{sect:combined}
Each of the search methods outlined in previous sections were applied to our sample independently. Combining them however may be able to identify the most promising candidates for glSNe -- those that are close to elliptical galaxies and show red colours with high inferred peak luminosities. Using SDSS spectroscopic redshifts, only two candidates were flagged as potentially having high peak luminosities, ZTF20aaaweke and ZTF21aaxxdpa. Neither object was selected by either of our other two selection methods. Therefore combining all three selection methods does not produce any candidates. 

\par

Combining the 238 candidates from our colour-based selection method with the 222 candidates from our outlier-based method, we find 69 are present in both samples. As expected however, most of these candidates were later rejected due to showing AGN or stellar-like variability. Only one candidate, ZTF18acbvgqw, passed the full set of selection criteria for both methods, but was a spectroscopically classified as a SN~Ia. As shown in Fig.~\ref{fig:outlier_method:ZTF18acbvgqw} it was flagged by the outlier-based method due to late-time fluctuations in the $r$-band light curve. ZTF18acbvgqw also peaked slightly brighter than our template simulations ($m_g = 13.91$) and hence was flagged as a candidate by the colour-based method. 

\par

In total we find that 65 unique transients were identified by the final set of selection criteria in any of our methods. Combining all three methods however does not produce any candidate glSNe.





\subsection{Limitations}
\label{sect:limits}
Here we discuss the limitations of the methods used during this work to identify candidate glSNe. Although discovered after the cut-off date for our initial sample, SN~Zwicky provides the perfect opportunity of testing the different identification methods used here and whether they would have been able to successfully identify it.


\subsubsection{Very red transients}
With a relatively low redshift ($z = 0.35$), SN~Zwicky did not show particularly red colours during the rising phase. Using the publicly available alert photometry from \alerce{}, and correcting for Milky Way extinction, SN~Zwicky increased from $m_r \sim$19.0 -- 18.2 following discovery. Throughout this period, the $g-r$ colour became redder, from $\sim$0.1 -- 0.3. As shown by Fig.~\ref{fig:colour_method}, SN~Zwicky would therefore not have passed our colour cuts and would be excluded. Including only SNe~Ia templates in our colour calculation however, SN~Zwicky would have been selected around maximum, but this would also lead to significantly increased contamination from other SN types.

\par

The primary limitation of the colour-based method applied here is that, for relatively shallow surveys such as ZTF, extreme magnifications are required. Therefore there may be many glSNe that simply do not reach the required apparent magnitude to be recognised as outliers. To estimate the fraction of glSNe that would be sufficiently magnified to pass our colour-based selection criteria, we simulate a sample of lensed and unlensed SNe~Ia. 

\par

The rate of glSNe that will be detectable by ZTF is given by
\begin{equation}
    \frac{dN_{SL}}{d{z_s}}(z_s) = \frac{dN_s(z_s)}{dz_s}\tau(z_s)B(z_s).
\end{equation}
Here, $\frac{dN_s}{dz_s}$ corresponds to the SN~Ia rate as a function of redshift. As in Sect.~\ref{sect:colour_method}, we use the rate given by \cite{kessler--19}. The factor $\tau(z_s)$ is the lensing optical depth and represents the probability of a source at redshift $z_s$ being strongly lensed. Finally, $B(z_s)$ accounts for the magnification bias, which is the fact that glSNe will be drawn from a fainter source population than unlensed SNe due to their higher redshifts.

\par

 Assuming the mass profile of lens galaxies is given by a singular isothermal sphere (SIS) \citep{schneider-book}, the lensing optical depth is defined as
\begin{equation}
\tau (z_s) = \int_0^{z_s} dz_l \frac{d^2 V}{dz_l d\Omega } \int_{\sigma_{min}}^{\sigma_{max}} d\sigma \frac{dn}{ d\sigma } \pi [\theta_{\rm{Ein}} ( z_s,z_l,\sigma )]^2,
\end{equation}
where $V$ is the comoving volume element, $\frac{dn}{ d\sigma }$ is the number density of lenses, and $\pi$ [\rein $(z_s,z_l,\sigma )$]$^2$ is the lensing cross-section. We take the velocity dispersion function derived by \cite{bernardi--10} from SDSS DR6 for all galaxy types,
\begin{equation}
dn=\phi_{*} \left(\frac{\sigma}{\sigma_*}\right)^{\alpha}exp\left[ -\frac{\sigma}{\sigma_*}\right]^{\beta} \frac{\beta}{\Gamma(\alpha/\beta})\frac{d\sigma}{\sigma},
\end{equation}
where $\phi_{*}=2.099\times10^{-2}$~$(h/0.7)^3$~Mpc$^{-3}$, $\sigma_*=113.78$~km~s$^{-1}$, $\alpha=0.94$, $\beta=1.85$. Following from \cite{Collett_2015}, we assume a population of lenses with velocity dispersions $\sigma$ \textgreater 100~km~s$^{-1}$.

\par

The magnification bias, $B(z_s)$, is defined as
\begin{equation}
B(z_s)=\int_{\mu_{\text{Q}}(z_s)}^{\infty} d\mu P(\mu)W(f(\mu),f_{\text{lim}}).
\end{equation}
The minimum magnification required for a glSN to be detected by our colour-based method (i.e. to equal the black line in Fig.~\ref{fig:colour_method}) is given by $\mu_{\rm{Q}}(z_s)$. The magnification distribution for the brightest image from the lens equation for our SIS is given by $P(\mu)$,  while $W(f(\mu),f_{\rm{lim}})$ gives the probability of a background source at $z_s$ being sufficiently magnified such that it would be detectable above the limiting flux of ZTF, $f_{\rm{lim}}$. Following \cite{collett--12}, $W$ is given by
\begin{equation}
W(f(\mu),f_{\rm{lim}})~=~ 
  \begin{cases}
   1  & \text{if } f(\mu) \geq f_{\text{lim}}/2 \\
   \left({2f(\mu)\over f_{\text{lim}}}\right)^2      & \text{if } f(\mu) < f_{\text{lim}}/2,
  \end{cases}
\label{eqn:w}
\end{equation}
where $f(\mu)$ is the magnified flux of the source. We assume a characteristic detection depth of $m_r = 20.5$ for ZTF \citep{bellm--19}. 

\par

To compute $B(z_s)$ we use the \cite{hsiao--07} spectroscopic template to simulate $10^4$ SNe~Ia light curves between $-14$ -- 0\,d relative to $B$-band maximum in a series of redshift bins up to $z = 2.5$. We chose the \cite{hsiao--07} spectroscopic template for this purpose as it extends to shorter wavelengths than other templates, which is necessary for simulating high redshift SNe. The peak $B$-band absolute magnitude of each simulated SN~Ia is drawn from a Gaussian distribution, $\mathcal{N}(-19.3,\,0.5)$. As in \cite{feindt--19}, host extinction is drawn from an exponential distribution with a rate of $\lambda = 0.11$. Within each redshift bin, we take the average of the $g-r$ colours and $r$-band magnitudes across all simulated SNe~Ia to compute the minimum magnification, $\mu_{\rm{Q}}(z_s)$.

\par


\begin{figure}
\centering
\includegraphics[width=\columnwidth]{Images/quimby_plot_ztf_models_2.png}
\caption{As in Fig.~\ref{fig:colour_method}, colour-magnitude diagram for a sample of our simulated non-lensed and lensed SNe~Ia in the ZTF $g$- and $r$-bands. Shaded regions show a 2D histogram of the number of expected transients of a given type on a log scale. A significant majority of glSNe are not detectable via the colour-based method and fall below the thick black line.  }
\label{fig:colour_method_lensed}
\centering
\end{figure}
 
Figure~\ref{fig:colour_method_lensed} shows a sample of our simulated non-lensed SNe~Ia. For each non-lensed SN, we also simulate a glSN calculated using a randomly sampled magnification. As shown by Fig.~\ref{fig:colour_method_lensed}, the vast majority of glSNe will not be sufficiently magnified to be detected via this method. This was also observed by \cite{quimby--14} in their simulations (see their figure 4), although they observed a greater proportion of detectable glSNe. This is likely due to a combination of the increased wavelength coverage and depth of PS1 compared to the ZTF public survey used here.

\par

 The total probability that a glSN could be detected with this method, as a function of redshift, is given by the product of the magnification bias and lensing optical depth for each redshift bin and is shown by Fig.~\ref{fig:lensedprobability}. Figure~\ref{fig:lensedprobability} highlights that this selection method is only sensitive to glSNe within the redshift range $0.6 \lesssim z_s \lesssim 1.2$ and with an overall low probability of $\sim$$10^{-7}$ -- $10^{-6}$.



\begin{figure}
\centering
\includegraphics[width=\columnwidth]{Images/colour_probs.pdf}
\caption{Magnification bias, $B(z_s)$ (left axis), and lensing optical depth, $\tau(z_s)$ (right axis), as a function source redshift, $z_s$. The product $\tau(z_s) \times B(z_s)$ gives the total probability that any individual SN is strongly lensed and magnified such that it would be detectable by our colour-based method and above the flux limit of ZTF. We note that $\tau(z_s) \times B(z_s)$ is shown scaled up by a factor of $1\,000\times$.  }
\label{fig:lensedprobability}
\centering
\end{figure}

\par

While the overall probability of detecting glSNe is low, to accurately evaluate the efficiency of this method we must compare to the number of non-lensed SNe~Ia that are also detectable. In other words, what is the rate of glSNe~Ia identified relative to non-lensed SNe~Ia that ZTF could actually discover? The flux limit of ZTF will only be sensitive to SNe~Ia with $z_s \lesssim 0.15$. Based on our simulations and the SN~Ia rate from \cite{kessler--19}, we find $\sim$0.06\% of SNe~Ia up to $z_s = 2.5$ would reach the flux limit of ZTF. Therefore, comparing the probability of being strongly lensed and passing our selection cuts ($\sim10^{-7}$ -- $10^{-6}$) to the probability of any given SN~Ia over the same redshift range meeting the flux limit of ZTF ($\sim$10$^{-3}$), we find an overall expected rate for detectable glSNe~Ia of $\sim$10$^{-4}$ per detectable unlensed SN~Ia.


\par

Based on our expected rate of glSNe~Ia ($\sim$10$^{-4}$), we can estimate the absolute number that should be detectable by ZTF. Among our sample of 12\,524 transients, using the \alerce{} light curve classifier \citep{sanchez-saez--21} we find that $\sim$10\% of objects are classified as not having `transient' light curves and therefore could represent contamination due to AGN or stochastic variability. We note however that in Sect.~\ref{sect:colour_method} \& \ref{sect:outlier_method}, the contamination rate was significantly higher for our identified candidates. Assuming $\sim$10 -- 50\% contamination from non-SNe results in a sample of 6\,262 -- 11\,271 potential SNe. In addition, from our sample of 12\,524 transients, 3\,843 were spectroscopically classified with 2\,684 being SNe~Ia. This fraction of SNe~Ia relative to all SNe ($\sim70\%$) is typical for a magnitude-limited sample \citep{li--2011}. Therefore assuming a similar fraction across our whole sample would result in $\sim$4\,397 -- 7\,890 potential SNe~Ia and imply we expect $\sim$0.4 -- 0.8 glSNe~Ia to have been detected by this method. We note that the simulations discussed here include only SNe~Ia. We expect that the fainter peak magnitudes of core-collapse SNe would likely result in an even lower probability of passing our selection method for this class, due to the larger magnifications required to reach the flux limit. Although the contamination rate is highly uncertain, our simulations show that we naively expect $\lesssim$1 glSN to be detected by this method. Therefore it is unsurprising that our search yielded no positive detections of glSNe, despite including four years of observations.

\par


\subsubsection{Light curve inconsistent with SN~Ia at photometric redshift of nearby elliptical galaxy}
\label{sect:outlier_method_limits}
SN~Zwicky is offset by 1.25\farcs{} from the host galaxy, which is present in the ALLWISE catalogue with a colour upper limit of $W2 - W3 \leq 1.89$. SN~Zwicky therefore would not have passed our selection criteria outlined in Sect.~\ref{sect:outlier_method:select}, however we note that we cannot conclusively rule out the colour as being consistent based on an upper limit. 

\par

Even allowing the galaxy to pass this initial selection cut, on the basis of the colour being an upper limit, SN~Zwicky would still not have been flagged as a candidate given that the host galaxy is visible and the lens galaxy is not. In other words, the nearby elliptical galaxy is the host, not the lens. Therefore fitting a SALT2 model to SN~Zwicky with the redshift of the elliptical would not have produced any significant outliers given that the redshift is correct. Indeed, we find a photometric redshift of the host of $z = 0.41\pm0.11$, which is consistent with the spectroscopic redshift of the SN, and no significant outliers when fitting the light curve with SALT2. Including this information on the host redshift and therefore luminosity, as in Sect.~\ref{sect:luminosity_method}, could have allowed SN~Zwicky to be recognised earlier, before the classification spectrum was observed. As discussed in Sect.~\ref{sect:luminosity_method} however, the use of photometric redshifts significantly increases the number of contaminants. No prior spectroscopic redshift of the host was available. 




\par

As discussed in Sect.~\ref{sect:outlier_method}, \cite{goldstein--19} assume a complete catalogue of elliptical galaxies. For this work, as we do not have a complete catalogue, we apply the selection cuts outlined in \cite{irani--22} to find elliptical galaxies coincident with transients in our sample. As discussed by \cite{irani--22}, these cuts are not 100\% complete, but are satisfied by 75\% of elliptical galaxies in Galaxy Zoo \citep{lintott--11}. Therefore, our selection of transients coincident with elliptical galaxies is likely at most 75\% complete and some glSNe could have been discarded on this basis.



\par

\cite{goldstein--19} also assume that the redshifts of elliptical galaxies in their catalogue are known exactly, which vastly reduces the degeneracy between SALT2 model parameters when fitting the observed light curves, particularly when fitting only two bands. For this work, we again do not have precise redshifts for each of the elliptical galaxies in our sample and therefore rely on photometric redshifts. In some cases, the uncertainty on the photometric redshift of the elliptical galaxy (i.e. the potential lens galaxy) may also be sufficiently large that the 3$\sigma$ range would also cover the redshift of any potential background glSN. Therefore in those cases no significant outliers would be flagged because the range of redshifts covered during the fit also includes the true value.


\par

Using this outlier-based detection method, \cite{goldstein--19} estimate that ZTF should discover 8.60 glSNe per year. From their simulations however, only $\sim$10 -- 16\% of these are detectable with data from the public survey alone. Therefore, across the four years of observations covered by our sample, we may expect to find 3.44 -- 5.50 glSNe. Assuming again that 25\% of our sample would not pass our elliptical selection criteria, this further reduces the number of glSNe to 2.58 -- 4.13. Given that we did not find any glSNe, this is within 1.6 -- 2.0$\sigma$ of the expected number from \cite{goldstein--19}. It is unclear however, how many glSNe would be discarded due to being fit with a redshift that does not produce any significant outliers, either due to degeneracies with other parameters (such as $x_1$ and $c$) or because the 3$\sigma$ range on the redshift of the lens galaxy would also include the redshift of the source. Therefore we consider these to be upper limits on the number of glSNe that should have been detected with this method and hence it is unsurprising that we were not able to confirm any positive detections.




\subsection{Estimating Einstein radii}


The methods for identifying glSNe used throughout this work have all resulted in significant numbers of contaminants, for which visually inspecting and performing classification is non-trivial. In the era of LSST, the number of SNe discovered will increase dramatically, making visual inspection of all candidates infeasible. Additional methods are therefore required to reduce the number of contaminants. One possibility is through estimating the Einstein radius of potential lens galaxies.

\par

The Einstein radius ($\theta_{\rm{Ein}}$) represents the separation between images of the lensed background source and the centre of the lens, and depends on the mass of the lens galaxy (or velocity dispersion) and distances between the observer, lens, and source. If a transient source is separated from nearby galaxies by significantly more than their Einstein radii, it is unlikely to have been strongly lensed and therefore may be rejected as a candidate glSN. For each of the 12\,524 transients in our sample, we estimate the Einstein radius of nearby galaxies and discard objects that are sufficiently separated, $\Delta \textgreater 3\theta_{\rm{Ein}}$.


\par

We estimate the most likely Einstein radius for all galaxies within 30\farcs{} of the transients in our sample. Again using the photometric redshift code presented by \cite{tarrio--20} and photometry from PS1 DR2, we calculate the absolute magnitude $M_{{r}}$ of each galaxy. Following from \cite{hyde--09}, we estimate the velocity dispersion, $\sigma$, as
\begin{equation}
    \log_{10} \sigma = \frac{-3 M_{{r}}^2 - 185M_{{r}} -1485}{500}.
    \label{eqn:rate}
\end{equation}
In addition to the velocity dispersion, the Einstein radius also depends on the distance to the source. We therefore assume any lensed background source could have a redshift up to $z = 2$ and calculate the angular diameter distances between the host and source, and observer and source in a series of redshift bins. The Einstein radius for each redshift bin $i$ is therefore given as
\begin{equation}
    \theta_{\rm{Ein},i} = 4 \pi \left(\frac{\sigma}{c}\right)^2 \frac{D_{\rm{ls},i}}{D_{\rm{s},i}},
    \label{eqn:rate}
\end{equation}
where $D_{\rm{s},i}$ is the angular diameter distance for a source in the current redshift bin and $D_{\rm{ls},i}$ is the angular diameter distance between the source and host galaxy. To calculate the most likely Einstein radius, we weight each one by the probability of strong lensing, $\theta_{\rm{Ein}}^2 W$. Here, $W$ is defined as in Eqn.~\ref{eqn:w} using the unmagnified peak magnitude of the background source (in this case $M_B = -19.3$ for an unlensed SN~Ia) and again the detection limit of ZTF.


\par

Calculating the most likely Einstein radii for galaxies coincident with each of our 12\,524 transients, we find 3\,109 are within 3\rein of any nearby galaxy. Applying this selection cut to our sample of 238 candidates identified by our colour-based method would have reduced this to only 60 candidates. Taking only those that also passed our pre-maximum cut would have reduced the sample from 32 to 5. Likewise, for our outlier-based method, the initial candidate sample would be reduced from 222 to 76, while our final sample would have been reduced from 32 to 12. While ultimately all of our candidates were excluded from being glSNe, applying a cut based on the most likely Einstein radius could be an effective way of removing $\gtrsim$50\% of contaminants. An additional cut on the minimum separation could also prove effective at removing AGN, which are the dominant source of contamination.




%
%______________________________________________________________
%___________________________________________________


\section{Conclusions}
\label{sect:conclusions}
To date, only a handful of gravitationally lensed supernovae (glSNe) are known. We presented an extensive search for glSNe that may have previously been unclassified within four years of observations from the Zwicky Transient Facility (ZTF) public survey.


\par

We conducted an initial search by crossmatching our transient sample against a catalogue of $\sim$154\,000 known or candidate lens galaxies, compiled from various literature sources, and found a single source coincident within 10\farcs{}. This source was spectroscopically classified as a SN~Ia at $z = 0.126$ and therefore was simply in chance alignment with the candidate gravitational lens system.

\par

Using methods for finding glSNe that have been suggested in the literature (based on simulations), we also conducted a search for transients that were magnified by unknown lenses. Following the colour-based method outlined by \cite{quimby--14}, we performed simulations of unlensed SNe to estimate the range of colours produced during their rising phases.  Transients redder than this limit may be at high redshift and magnified to appear brighter, and are therefore candidate glSNe. Applying this method to our ZTF sample, we found 238 candidates. Visually inspecting each of these candidates, we rejected any that were clearly not pre-maximum (i.e. rising), leaving 32. Fitting each of these light curves, we found they were all consistent with existing templates for unlensed SNe. We therefore found no compelling evidence for candidate glSNe based on extremely red colours and bright magnitudes. 


\par

The second detection method we implemented was outlined by \cite{goldstein--18--lens}. We identified all transients in our sample within 10\farcs{} of an elliptical galaxy and tested whether they were consistent with being SNe~Ia at the redshift of the elliptical. We found 222 candidates with at least one significant outlier relative to a SALT2 fit. Analysing each of these candidates, we found the main sources of contamination were due to bogus detections, well-observed SNe~Ia extending to phases beyond the SALT2 model, or AGN- or stellar-like variability. In addition, we also found two possible candidates for sibling SNe -- multiple SNe hosted by the same galaxy that are tagged as one object in survey data. Removing these contaminants, we again fit each light curve with existing SN templates and found no compelling evidence for glSNe. Finally, we also implemented a luminosity-based method to identify any transients with absolute magnitudes brighter than expected for normal SNe. Using spectroscopic redshifts from SDSS, we identified two candidates. One of these candidates was a likely SLSN while the other was likely due to a misidentified host galaxy.


\par

To date, SN~Zwicky is the only confirmed glSN detected within ZTF. Although not included in our initial sample, we tested whether any of the methods used within this work would have successfully identified SN~Zwicky and found it would have been excluded in all cases. We also estimate the numbers of glSNe expected to be detected via these methods and find they are consistent with the null detections reported here. 

\par



In the coming years, LSST will discover thousands of SNe and should be more sensitive to glSNe given the increased observing depth and broader wavelength coverage \citep{goldstein--19}. The sheer volume of transients discovered by LSST mean that even with the relatively low efficiency rate of the methods used here, some glSNe should be discovered. Visual inspection or spectroscopic classification (as was the case for SN~Zwicky) for all candidate glSNe however will not be feasible. Therefore, given the false positive rate, the methods applied here to ZTF will not scale well to LSST.



\par


Alternatively, we suggest three complimentary approaches that would substantially improve the prospects of discovering lensed SNe in LSST. As most of the sky is not strongly lensed, an estimate of the likely Einstein radii for all elliptical galaxies would allow us to exclude most of the false positives. Watchlist functionality is already included in transient brokers and a simple watchlist of all elliptical galaxies, with specific association radii determined from the likely Einstein radii would significantly reduce contamination. One of the primary limitations of this work is the use of photometric redshifts which produce very uncertain Einstein radius estimates. A larger spectroscopic redshift catalogue compared to SDSS and a precise photometric redshift catalogue would allow us to more accurately find SNe that are much brighter than expected given the redshift of their host. Surveys and facilities such as DESI and 4MOST will provide spectroscopic redshifts for millions of galaxies and a more complete catalogue across a wider area. Finally, searches for and confirmation of galaxy-galaxy lenses should continue. A larger list of known lenses would likely provide the highest purity sample of candidates.


\par

This work has shown that pushing fainter than the very bright iPTF16geu and SN Zwicky-like lensed SNe is not trivial, even with full archival light curves. More work is needed to improve selection methods if real-time searches for cosmologically useful lensed SNe are to avoid being swamped by false positives.


\appendix
\section{Additional information for colour-based method candidates}

Full list and details of candidates identified due to red colours and relatively bright magnitudes.

\begin{table*}
\begin{center}
\caption{Final candidate glSNe selected by the colour-based method.}
\label{tab:colour_outliers}
\resizebox{\textwidth}{!}{
\begin{tabular}{llllcrll}
\hline
\textbf{ZTF name}	 &	\textbf{Classification}$^{a}$	& \textbf{Redshift}	&			 &	\textbf{First flagged detection}$^{b}$ &				&	\textbf{Estimated colour excess} & \textbf{Notes} \\

                     &	                       &                   &	\textbf{MJD}		 &	\textbf{$r$-band magnitude}     &	\textbf{$g-r$ colour} &	 &  \\
\hline
\hline
ZTF18aacsudg &	\textit{SN 1987A-like}	& 0.025$^{\textrm{†}}$	&	58745.51 & 18.03$\pm$0.04 		& 1.34$\pm$0.11 	& 	A$_V$	= \phn{}$0.56\pm0.05$		&					\\
ZTF18abgmcmv &	Ia 91T-like		& 0.019		&	58326.23 & 16.43$\pm$0.02 		& 0.53$\pm$0.03 	& 	\phn{}\phn{}c  	 	= \phn{}$0.61\pm0.03$		&										\\
ZTF18abmxahs &	Ia 				& 0.015		&	58350.18 & 16.85$\pm$0.04 		& 1.15$\pm$0.40 	& 	\phn{}\phn{}c		= $-0.02\pm0.03$	&										\\
ZTF18acbvgqw &	Ia				& 0.009		&	58437.30 & 13.44$\pm$0.42 		& 0.03$\pm$0.42 	& 	\phn{}\phn{}c		= $-0.03\pm0.03$	&										\\
ZTF19aadttht &	Ic				& 0.006		&	58509.52 & 15.97$\pm$0.03 		& 0.63$\pm$0.07 	& 	A$_V$   = \phn{}0.06 -- 0.248		&										\\
ZTF19aadyppr &	ILRT			& 0.002		&	58522.54 & 16.81$\pm$0.07 		& 0.89$\pm$0.08 	& 	-							&	No suitable template				\\
ZTF19aafncsv &	Ia				& 0.037		&	58526.55 & 17.13$\pm$0.05 		& 1.23$\pm$0.51 	& 	\phn{}\phn{}c		= \phn{}$0.13\pm0.06$	&										\\
ZTF19aarnqzw &	Ia				& 0.028		&	58606.38 & 16.31$\pm$0.12 		& 0.51$\pm$0.26 	& 	\phn{}\phn{}c		= $-0.12\pm0.04$	&										\\
ZTF19abguibf &	\textit{Mira variable}	& 0.000$^{\textrm{†}}$ &	58703.49 & 19.26$\pm$0.04 		& 1.33$\pm$0.10 	& 	-							&					\\
ZTF19abucwzt &	Ib				& 0.017		&	58909.16 & 18.99$\pm$0.03 		& 1.27$\pm$0.07 	& 	-							&	No suitable template; \cite{sollerman--20} \\ 
ZTF19abxqppy &	IIb				& 0.014		&	58737.18 & 18.09$\pm$0.07 		& 1.16$\pm$0.18 	& 	-							&	No suitable template				\\
ZTF19achaspq &	Ia				& 0.016		&	58790.11 & 16.26$\pm$0.38 		& 0.51$\pm$0.39 	& 	\phn{}\phn{}c		= \phn{}$0.47\pm0.05$		&										\\
ZTF19acnzkph &	Ia				& 0.018		&	58801.22 & 16.52$\pm$0.02 		& 0.67$\pm$0.04 	& 	\phn{}\phn{}c		= \phn{}$0.67\pm0.04$		&										\\
ZTF20aaelulu &	Ic				& 0.005		&	58859.57 & 14.71$\pm$0.03 		& 0.13$\pm$0.04 	& 	A$_V$   = \phn{}$1.09\pm0.07$		&										\\
ZTF20abefbpl &	Ic				& 0.042		&	59019.20 & 18.24$\pm$0.03 		& 1.24$\pm$0.09 	& 	A$_V$   = \phn{}$2.20\pm0.34$		&										\\
ZTF20abpmqnr &	IIn				& 0.022		&	59075.46 & 16.09$\pm$0.02 		& 0.44$\pm$0.03 	& 	A$_V$   = \phn{}1.12 -- 1.84 		&										\\
ZTF20abuovvw &	\textit{II/Ibc}			& 0.061$^{\textrm{†,ø}}$ &	59107.34 & 19.13$\pm$0.04 		& 1.32$\pm$0.11 	& 	A$_V$   = \phn{}0.90 -- 2.07		&					\\
ZTF20acdqjeq &	Iax				& 0.017		&	59136.36 & 16.14$\pm$0.02 		& 0.44$\pm$0.03 	& 	-							&	No suitable template				\\
ZTF20acrzwvx &	II				& 0.010		&	59188.45 & 16.44$\pm$0.08 		& 0.45$\pm$0.09 	& 	A$_V$   = \phn{}$0.57\pm0.04$		&										\\
ZTF20acynjjo &	Ia				& 0.015		&	59217.10 & 15.72$\pm$0.03 		& 0.24$\pm$0.04 	& 	\phn{}\phn{}c		= \phn{}$0.28\pm0.04$		&										\\
ZTF21aaabwfu &	IIb				& 0.011		&	59229.55 & 18.21$\pm$0.05 		& 1.12$\pm$0.11 	& 	A$_V$   = \phn{}$2.39\pm0.43$		&										\\
ZTF21aaaubig &	Ic				& 0.009		&	59224.46 & 16.68$\pm$0.02 		& 0.49$\pm$0.03 	& 	A$_V$   = \phn{}$0.94\pm0.12$		&										\\
ZTF21aamwqim &	II				& 0.026		&	59280.29 & 18.33$\pm$0.03 		& \phn{}1.19$\pm$0.48 	& 	A$_V$   = \phn{}0.01 -- 0.73	&										\\
ZTF21aaqytjr &	Ia				& 0.003		&	59311.25 & 13.95$\pm$0.02 		& $-$0.06$\pm$0.07	& 	\phn{}\phn{}c		= \phn{}$0.05\pm0.03$		&										\\
ZTF21aatyplr &	Ia				& 0.008		&	59317.30 & 18.13$\pm$0.04 		& 1.34$\pm$0.07 	& 	\phn{}\phn{}c		= \phn{}$1.26\pm0.04$		&										\\
ZTF21abfmbix &	Ia           	& 0.009		&	59386.20 & 15.08$\pm$0.26 		& 0.26$\pm$0.27 	& 	A$_V$	= \phn{}$0.01\pm0.01$		&	Best-matched by 91bg-like template	\\
ZTF21abiuvdk &	Ia				& 0.004		&	59400.45 & 13.80$\pm$0.02 		& 0.26$\pm$0.03 	& 	\phn{}\phn{}c		= \phn{}$0.08\pm0.04$		&										\\
ZTF21abjyiiw &	IIb				& 0.005		&	59408.40 & 17.63$\pm$0.03 		& 0.81$\pm$0.06 	& 	A$_V$   = \phn{}$1.83\pm0.12$		&										\\
ZTF21acdontl &	II				& 0.010		&	59500.51 & 16.86$\pm$0.03 		& 0.87$\pm$0.06 	& 	A$_V$   = \phn{}0.12 -- 1.05		&										\\
ZTF21aceehxt &	Ib				& 0.013		&	59496.38 & 17.48$\pm$0.03 		& 0.97$\pm$0.07 	& 	A$_V$	= \phn{}1.80 -- 2.37		&										\\
ZTF21acenkuf &	Ia				& 0.012		&	59497.37 & 17.10$\pm$0.10 		& 0.91$\pm$0.11 	& 	\phn{}\phn{}c		= \phn{}$0.88\pm0.04$		&										\\
ZTF21aclyyfm &	II 				& 0.005		&	59550.49 & 16.31$\pm$0.04 		& 0.76$\pm$0.06 	&	A$_V$   = \phn{}$1.84\pm0.05$		&										\\
\hline   
\hline

\multicolumn{8}{l}{$^{a}$ Classifications given in italics are based on photometric templates or cross-matching. All other classifications are based on spectroscopic observations.} \tabularnewline
\multicolumn{8}{l}{$^{b}$ Magnitudes and colours are given corrected for Milky Way extinction only.} \tabularnewline
\multicolumn{8}{l}{$^{\textrm{†}}$ denote redshifts of the likely host galaxy while $^{\textrm{ø}}$ give photometric redshifts. All other redshifts are based on spectroscopic observations.} \tabularnewline


\end{tabular}
}
\end{center}
\end{table*}


%______________________________________________________________

\section{Additional information for outlier-based method candidates}

Additional information and figures related to candidates identified due to outliers relative to a SALT2 model fit.

\begin{table*}
\begin{center}
\caption{Final candidate glSNe selected by the outlier-based method.}
\label{tab:elliptical_outliers}
%\resizebox{\textwidth}{!}{
\begin{tabular}{llllll}
\hline
\textbf{ZTF name}  	&	\textbf{Classification}$^{a}$ 	&    &                        \textbf{Elliptical galaxy}$^{b}$ 	&                       &                     \\
                	&                                  	& \textbf{Photometric $z$} &	\textbf{Separation (\farcs{})}        	&	\textbf{$W2 - W3$} 	& \textbf{$NUV - r$}	       \\
\hline
\hline
ZTF18aabdajx 		& TDE 								&  $0.04\pm0.02$ 				& 0.27\farcs{} 					&  $\textless-0.57$ 	             &  				   \\ 
ZTF18aasvknh 		& \textit{AGN?}						&  $0.11\pm0.03$ 				& 0.11\farcs{} 					&  \phn{}\phn{}$-0.03\pm0.36$ 		 &  				   \\ 
ZTF18aavxiih 		& \textit{II}						&  $0.12\pm0.03$ 				& 1.76\farcs{} 					&  \phn{}\phn{}\phn{}\,$0.26\pm0.21$ &  				   \\ 
ZTF18abavruc 		& Ia 								&  $0.05\pm0.02$ 				& 0.11\farcs{} 					&  \phn{}\phn{}$-0.86\pm0.46$	 	 &  				   \\ 
ZTF18abdgwvs 		& \textit{Siblings/AGN?}			&  $0.12\pm0.03$ 				& 0.40\farcs{} 					&  \phn{}\phn{}\phn{}\,$0.25\pm0.26$ &  				   \\ 
ZTF18abhhxcp 		& Ia 								&  $0.07\pm0.01$ 				& 1.91\farcs{} 					&  \phn{}\phn{}$-1.01\pm0.13$		 &  				   \\ 
ZTF18abzrsuh 		& \textit{Blazar}					&  $0.11\pm0.04$ 				& 0.33\farcs{} 					&  \phn{}\phn{}$-0.08\pm0.21$ 		 &  $3.01\pm0.15$ 	   \\ 
ZTF18acbvgqw 		& Ia 								&  $0.01\pm0.01$ 				& 1.92\farcs{} 					&  \phn{}\phn{}$-1.31\pm0.03$ 		 &  $5.39\pm0.04$ 	 	\\ 
ZTF18acqywlx 		& \textit{AGN?}						&  $0.13\pm0.03$ 				& 0.17\farcs{} 					&  \phn{}\phn{}\phn{}\,$0.28\pm0.49$ &  				  	\\ 
ZTF18acslpba 		& Ia 								&  $0.03\pm0.01$ 				& 3.67\farcs{} 					&  \phn{}\phn{}$-0.76\pm0.05$ 		 &  $6.20\pm0.12$ 	 	\\ 
ZTF19aawhagd 		& \textit{Siblings/AGN?}			&  $0.14\pm0.05$ 				& 1.22\farcs{} 					&  \phn{}\phn{}$-0.23\pm0.50$ 		 &  				 	\\ 
ZTF19aayvyeo 		& \textit{AGN?}						&  $0.28\pm0.17$ 				& 0.14\farcs{} 					&  \phn{}\phn{}\phn{}\,$0.25\pm0.05$ &  				   \\ 
ZTF20aalmeaj 		& Ia								&  $0.03\pm0.01$ 				& 5.49\farcs{} 					&  \phn{}\phn{}\phn{}\,$0.17\pm0.07$ &  $3.80\pm0.07$ 	 	\\ 
ZTF20aammhli 		& Ia 								&  $0.08\pm0.02$ 				& 7.46\farcs{} 					&  \phn{}\phn{}$-0.52\pm0.38$ 		 &  				  \\ 
ZTF20aaodkvl 		& Ia 								&  $0.04\pm0.01$ 				& 9.30\farcs{} 					&  \phn{}\phn{}$-0.63\pm0.05$ 		 &  				  \\ 
ZTF20abfcszi 		& TDE 								&  $0.09\pm0.03$ 				& 0.10\farcs{} 					&  $\textless-0.14$ 	             &  				  \\ 
ZTF20abqbzuv 		& Ia 								&  $0.03\pm0.01$ 				& 1.86\farcs{} 					&  \phn{}\phn{}$-1.40\pm0.11$ 		 &  $6.20\pm0.21$ 	 \\ 
ZTF20abxyims 		& \textit{SLSN?}					&  $0.21\pm0.03$ 				& 0.62\farcs{} 					&  $\textless\phn{}0.30$ 		     &  				 \\ 
ZTF20abzcefc 		& II 								&  $0.32\pm0.04$ 				& 7.73\farcs{} 					&  $\textless-0.21$ 	             &  				  \\ 
ZTF20acitpfz 		& TDE 								&  $0.07\pm0.03$ 				& 0.19\farcs{} 					&  $\textless-0.07$ 	             &  $4.73\pm0.17$ 	 \\ 
ZTF20acqntkr 		& Ia 								&  $0.02\pm0.01$ 				& 8.36\farcs{} 					&  \phn{}\phn{}$-1.13\pm0.06$ 		 &  				 \\ 
ZTF20acquetr 		& Ia 								&  $0.04\pm0.01$ 				& 6.12\farcs{} 					&  \phn{}\phn{}$-1.08\pm0.11$ 		 &   				  \\ 
ZTF20acvziuf 		& Ia 								&  $0.02\pm0.01$ 				& 7.34\farcs{} 					&  \phn{}\phn{}$-1.34\pm0.13$ 		 &  				  \\ 
ZTF20acwjnux 		& \textit{II}      					&  $0.11\pm0.02$ 				& 2.61\farcs{} 					&  \phn{}\phn{}\phn{}\,$0.05\pm0.32$ &  $3.69\pm0.17$ 	 \\ 
ZTF21aaanpyy 		& \textit{II}    					&  $0.07\pm0.03$ 				& 4.53\farcs{} 					&  \phn{}\phn{}\phn{}\,$0.39\pm0.12$ &  $3.70\pm0.15$ 	 \\ 
ZTF21aaoekcj 		& Ia 								&  $0.05\pm0.01$ 				& 4.36\farcs{} 					&  \phn{}\phn{}$-0.47\pm0.13$	 	 &  				  \\ 
ZTF21abcgnqn 		& TDE 								&  $0.10\pm0.03$ 				& 0.15\farcs{} 					&  \phn{}\phn{}$-0.63\pm0.32$	 	 &  				  \\ 
ZTF21abfxibf 		& Ia 								&  $0.06\pm0.02$ 				& 0.02\farcs{} 					&  $\textless-0.71$ 	             &  				  \\ 
ZTF21abptxfk 		& II 								&  $0.10\pm0.03$ 				& 3.09\farcs{} 					&  $\textless-0.87$ 	             &  				  \\ 
ZTF21acciklh 		& Ia 								&  $0.04\pm0.03$ 				& 9.68\farcs{} 					&  \phn{}\phn{}$-1.22\pm0.16$ 		 &  				  \\ 
ZTF21accwovq 		& SLSN 								&  $0.09\pm0.04$ 				& 7.75\farcs{} 					&  \phn{}\phn{}$-0.01\pm0.37$ 		 &  				  \\ 
ZTF21acfabut 		& Ia 								&  $0.07\pm0.02$ 				& 2.86\farcs{} 					&  $\textless-0.84$ 	             &  				  \\ 
\hline   
\hline

\multicolumn{6}{l}{$^{a}$ Classifications given in italics are based on photometric templates or cross-matching. }\tabularnewline
\multicolumn{6}{l}{$^{\phn{}}$ All other classifications are based on spectroscopic observations.} \tabularnewline
\multicolumn{6}{l}{$^{b}$ Colours are given corrected for Milky Way extinction only.} \tabularnewline
\end{tabular}
%}
\end{center}
\end{table*}


\begin{figure}
\centering
\includegraphics[width=\columnwidth]{Images/outlier_candidates_ZTF18aavxiih.pdf}
\caption{Light curve of ZTF18aavxiih compared to the SN~II SN~2007lx. }
\label{fig:outlier_method:ZTF18aavxiih}
\centering
\end{figure}


\begin{figure}
\centering
\includegraphics[width=\columnwidth]{Images/outlier_candidates_ZTF20acwjnux.pdf}
\caption{Light curve of ZTF20acwjnux compared to the SN~II SN~2006ez. }
\label{fig:outlier_method:ZTF20acwjnux}
\centering
\end{figure}

\begin{figure}
\centering
\includegraphics[width=\columnwidth]{Images/outlier_candidates_ZTF21aaanpyy.pdf}
\caption{Light curve of ZTF21aaanpyy compared to the SN~II SN~2006jl. }
\label{fig:outlier_method:ZTF21aaanpyy}
\centering
\end{figure}

%
%______________________________________________________________


\section*{Acknowledgements}

We thank R. Quimby for providing data on simulated glSNe.

This work has received funding from the European Research Council
(ERC) under the European Union’s Horizon 2020 research and in-
novation programme (LensEra: grant agreement No 945536). MRM acknowledges a Warwick Astrophysics prize post-doctoral fellowship made possible thanks to a generous philanthropic donation. TEC is supported by a Royal Society University Research Fellowship.

Based on observations obtained with the Samuel Oschin 48-inch Telescope at the Palomar Observatory as part of the Zwicky
Transient Facility project. ZTF is supported by the National Science Foundation under Grant No. AST-1440341 and a
collaboration including Caltech, IPAC, the Weizmann Institute for Science, the Oskar Klein Center at Stockholm University, the
University of Maryland, the University of Washington, Deutsches Elektronen-Synchrotron and Humboldt University, Los Alamos
National Laboratories, the TANGO Consortium of Taiwan, the University of Wisconsin at Milwaukee, and Lawrence Berkeley
National Laboratories. Operations are conducted by COO, IPAC, and UW. Based on observations obtained with the Samuel Oschin Telescope 48-inch and the 60-inch Telescope at the Palomar
Observatory as part of the Zwicky Transient Facility project. ZTF is supported by the National Science Foundation under Grants
No. AST-1440341 and AST-2034437 and a collaboration including current partners Caltech, IPAC, the Weizmann Institute for
Science, the Oskar Klein Center at Stockholm University, the University of Maryland, Deutsches Elektronen-Synchrotron and
Humboldt University, the TANGO Consortium of Taiwan, the University of Wisconsin at Milwaukee, Trinity College Dublin,
Lawrence Livermore National Laboratories, IN2P3, University of Warwick, Ruhr University Bochum, Northwestern University and
former partners the University of Washington, Los Alamos National Laboratories, and Lawrence Berkeley National Laboratories.
Operations are conducted by COO, IPAC, and UW.

This research has made use of the NASA/IPAC Extragalactic Database, which is funded by the National Aeronautics and Space Administration and operated by the California Institute of Technology.





%%%%%%%%%%%%%%%%%%%%%%%%%%%%%%%%%%%%%%%%%%%%%%%%%%
\section*{Data Availability}






%%%%%%%%%%%%%%%%%%%% REFERENCES %%%%%%%%%%%%%%%%%%

% The best way to enter references is to use BibTeX:

\bibliographystyle{mnras}
\bibliography{mnras_template}


% Alternatively you could enter them by hand, like this:
% This method is tedious and prone to error if you have lots of references
%\begin{thebibliography}{99}
%\bibitem[\protect\citeauthoryear{Author}{2012}]{Author2012}
%Author A.~N., 2013, Journal of Improbable Astronomy, 1, 1
%\bibitem[\protect\citeauthoryear{Others}{2013}]{Others2013}
%Others S., 2012, Journal of Interesting Stuff, 17, 198
%\end{thebibliography}

%%%%%%%%%%%%%%%%%%%%%%%%%%%%%%%%%%%%%%%%%%%%%%%%%%

%%%%%%%%%%%%%%%%% APPENDICES %%%%%%%%%%%%%%%%%%%%%

\appendix

% \onecolumn

%%%%%%%%%%%%%%%%%%%%%%%%%%%%%%%%%%%%%%%%%%%%%%%%%%


% Don't change these lines
\bsp	% typesetting comment
\label{lastpage}
\end{document}

% End of mnras_template.tex
