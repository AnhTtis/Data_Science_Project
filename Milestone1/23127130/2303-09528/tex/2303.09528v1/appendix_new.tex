\section{Example of end-components} \label{app:EC}
We show an example of a CTMDP in Figure~\ref{fig:example} which has three end-components $\widehat{A}, \widehat{B}, \text{ and } \widehat{C}$.

\begin{figure}[h]
	\centering
	{
		\scalebox{1}{
			\begin{tikzpicture}[auto]
				\node[state] (s1) at (-3,0.5) {$s_1$};
				\node[state] (s6) at (-0.1,-4) {$s_6$};
				\node[state] (s5) at (-0.1,-2) {$s_5$};
				\node[state] (s3) at (-3.1,-2) {$s_3$};
				\node[state] (s4) at (-3.1,-4) {$s_4$};
				\node[state] (s2) at (-7.5,-2) {$s_2$};
				% \coordinate[at= ($(state_0) + (45:0.5)$),label={0:a,$0$}] (state_0_a);
				% \draw[-] (s0) -- (state_0_a);
				% \node[] at (-2.3,0.3) {$a$};
				% \node[] at ($(s4) + (0.8,0.5)$) {$a$};
				% \node[] at ($(s6) + (0.8,0.5)$) {$a$};
				% \node[] at ($(s3) + (-0.9,-0.7)$) {$a$};
				% \node[] at ($(s5) + (-0.9,-0.7)$) {$a$};
			
				\path[->]
					(s1) edge node[right]{$a,2$} (s5)
					(s1) edge node[left]{$a,9$} (s3)
					(s1) edge node[above] {$b,1$} (s2)
					(s5) edge[out=225,in=135] node[below right]{$a,5$} (s6)
					(s5) edge[out=225,in=155,looseness=8] coordinate (e5loop) node[above left]{$a,5$} (s5)
					(s6) edge[out=45,in=-35,looseness=4] coordinate (e6loop) node[below right] {$a,4$} (s6)
					(s6) edge[out=45,in=-45] node[above right] {$a,3$} (s5)
					(s3) edge[out=225,in=135] node[below right]{$a,4$} (s4)
					(s3) edge[out=225,in=155,looseness=8] coordinate (e3loop) node[above left]{$a,5$} (s3)
					(s4) edge[out=45,in=-35,looseness=4] coordinate (e4loop) node[below right] {$a,2$} (s4)
					(s4) edge[out=45,in=-45] node[above right] {$a,5$} (s3)
					(s2) edge[loop left] coordinate (e2loop) node[left, inner sep=7pt]{$a,3$} (s2)
					(s2) edge[out=-80,in=180,bend right] node[below left] {$b,2$} (s4);
				
				\node[rectangle,rounded corners=3pt,draw=none,fill=black,fill opacity=0.1,fit=(s2) (e2loop)] (rectA) {};
				\node[rectangle,rounded corners=3pt,draw=none,fill=black,fill opacity=0.1,fit=(s3) (s4) (e3loop) (e4loop)] (rectB) {};
				\node[rectangle,rounded corners=3pt,draw=none,fill=black,fill opacity=0.1,fit=(s5) (s6) (e5loop) (e6loop)] (rectC) {};
				\node[] at ($(rectA) + (0,-0.8)$) {$\widehat{A}$};
				\node[] at ($(rectB) + (0,-1.7)$) {$\widehat{B}$};
				\node[] at ($(rectC) + (0,-1.7)$) {$\widehat{C}$};
			\end{tikzpicture}
		}
	}
	\caption{A CTMDP with three end-components.}\label{fig:example}
\end{figure}


\section{Example of Uniformization}
We give an example of a non-uniform CTMDP and its uniformized version.
\label{uniform}
\begin{figure}[h]
    \centering
    \begin{subfigure}[b]{0.48\linewidth}
\begin{tikzpicture}[shorten >=1pt, node distance=3 cm, on grid, auto,thick,initial text=]
\begin{scope}
\node (l0) [state,fill=safecellcolor]  {$q_0$};
\node (l1) [state, fill=safecellcolor, right = of l0,xshift = -0.75cm]   {$q_1$};
\end{scope}
 \begin{scope}
\path [->]
    (l0) edge [bend left]  node [above] {$a_1, 3$}   (l1)
    (l0) edge [bend right]  node [below] {$a_2,6$}   (l1)
    (l1) edge [loop above] node [above] {$a_3,2$}   ()
    ;
\end{scope}
\end{tikzpicture}
\caption{Non-uniform CTMDP where the exit-rates are different for various state action pairs. } \label{fig:PM1}
\end{subfigure}
\hspace{0.5em}
\begin{subfigure}[b]{0.45\linewidth}
\begin{tikzpicture}[shorten >=1pt, node distance=3 cm, on grid, auto,thick,initial text=]
\begin{scope}
\node (l1) [state, fill=safecellcolor]  {$q_1$};
\node (l0) [state, fill=safecellcolor, left = of l1,xshift = 0.75cm]  {$q_0$};
\end{scope}
\begin{scope}
\path [->]
     (l0) edge [loop above] node [above] {$a_1,3$}
    (l0) edge [bend left]  node [above] {$a_1,3$}   (l1)
    (l0) edge [bend right]  node [below] {$a_2,6$}   (l1)
    (l1) edge [loop above] node [above] {$a_3,6$}   ()
    ;
\end{scope}
\end{tikzpicture}
\caption{A Uniform CTMDP where the exit-rate for every state-action pair is $6$. } \label{fig:PM2}
\end{subfigure}
\caption{Uniformization of a CTMDP}
\label{fig:P1}
\end{figure}


\section{Good-for-CTMDP B\"uchi Automata}
\label{app:gfm}
In this section, we provide a formal definition of good-for-CTMDP automata.
\paragraph{Product CTMDP.}
 Given a \emph{labelled} CTMDP $(\Mdp, \atomicprop, \labelling)$ where $\atomicprop$ is a set of atomic propositions, and $\labelling: \states \rightarrow 2^\atomicprop$ is a labelling function and a   B{\"u}chi automaton $\oautomata = (2^{\atomicprop}, \ostate, \oinitstate, \otrans, \acceptingc)$, the product CTMDP is defined as $\Mdp \times \oautomata = ((\states \times \ostate),(\initstate,\oinitstate),\actions,\transR^{\times},\acceptingc^{\times})$ where the rates are $\transR^{\times} : (\states \times \ostate) \times \actions \times (\states \times \ostate) \rightarrow \Reals_{\geq 0}$ such that $\transR^{\times} ((s,q),a,(s',q') = \transR(s,a,s')$ if $\transR(s,a,s')>0$ and $\otrans (q,\labelling(s)) = \{q'\}$.
If $F$ is the set of accepting 
% transitions 
states
in $\oautomata$, then the accepting condition is a set 
% transitions $F^\times$ where $((s,q),a,(s',q')) \in F^\times$ iff $(q,\labelling(s),q') \in F$ 
$F^\times$ of states where $(s,q) \in F^\times$ iff $q \in F$. An example of a product CTMDP is given in Figure~\ref{fig:grid-world}.  

Given an MDP $\Mdp$, a B\"uchi automaton  $\oautomata$, and product $\Mdp \times \oautomata$, we define the following two  problems:
\begin{enumerate}
    \item {\bf Satisfaction Semantics.} 
    Compute a schedule of $\Mdp$ that maximizes the probability of visiting accepting states $F$ of $\oautomata$ infinitely often.
    We define the satisfaction probability $\PSat^{\Mdp \times \oautomata}(s, \sigma)$ of a schedule $\sigma$ from starting state $s$ as: 
\begin{equation*}
\Pr{}^{\Mdp \times \oautomata}_{\sigma}(s) \set{\forall_i  \exists_{j{\geq} i} [X_j \in F^\times] }.
\end{equation*}
The optimal satisfaction probability
$\PSemSat^{\Mdp \times \oautomata}_{\oautomata}(s)$ for specification $\oautomata$ 
is defined as $\sup_{\sigma \in \Sigma_{\Mdp \times \oautomata}} \Pr^{\Mdp \times \oautomata}_{\sigma}(s, \sigma)$ and we say
that $\sigma$ is an optimal schedule for $\oautomata$ if
$\PSemSat^{\Mdp \times \oautomata}_{\oautomata}(s, \sigma) (s) = \PSemSat^{\Mdp \times \oautomata}_{\oautomata}$.

\item {\bf Expectation Semantics.} Compute a schedule of $\Mdp$ that maximize the long-run expected average time spent in the accepting states of $\oautomata$.
We define the expected satisfaction time $\ESat^{\Mdp \times \oautomata}_{\oautomata}(s, \sigma)$ 
of $\sigma$ from starting state $s$ as: 
\begin{equation*}
 \mathbb{E}^{\Mdp \times \oautomata}_{\sigma}(s) \set{ \liminf_{n \rightarrow \infty} \frac{\sum_{i=1}^{n} [X_i {\in} F] D_i}{T_n}}.
\end{equation*}
The optimal expected satisfaction time
$\ESat^{\Mdp \times \oautomata}_{\oautomata}(s)$ for $\oautomata$ is defined as $\sup_{\sigma \in \Sigma_{\Mdp}} \ESat^{\Mdp \times \oautomata}_{\oautomata}(s, \sigma)$ and we say that $\sigma \in \Sigma_\Mdp$ is an optimal expected-satisfaction schedule for $\oautomata$ if
$\ESat^{\Mdp \times \oautomata}_{\oautomata}(s, \sigma) = \ESat^{\Mdp \times \oautomata}_{\oautomata} (s)$.
\end{enumerate}

We call a B{\"u}chi automaton $\oautomata = (2^{\atomicprop}, \ostate, \oinitstate, \otrans, \acceptingc)$ good-for-CTMDP if for every \emph{labelled} CTMDP $(\Mdp, \atomicprop, \labelling)$ where $\atomicprop$ is a set of atomic propositions, we have that 
\begin{eqnarray*}
    \PSemSat^{\Mdp}_{\oautomata}(s, \sigma) &=&\PSat^{\Mdp \times \oautomata}_{\oautomata}(s, \sigma)~\text{ and }\\
    \ESemSat{}^{\Mdp}_{\oautomata}(s, \sigma) &=&
    \ESat^{\Mdp \times \oautomata}_{\oautomata}(s, \sigma).
\end{eqnarray*}
A good-for-CTMDP automaton allows the computation of the optimal schedule by solving the corresponding problem on the product CTMDP. 
If a B\"uchi automaton is good-for-MDP, then one can show via uniformization that it is also good-for-CTMDPs.
There exists several syntactic characterizations of good-for-MDP automata including suitable limit-deterministic B\"uchi automata (SLDBA)~\cite{sickert2016limit} and slim automata~\cite{HPSS20}.
Moreover, every LTL specification can be effectively converted into a GFM B\"uchi automata.

% \begin{theorem}[Good-for-MDP B\"uchi Automata]
% \end{theorem}
\section{Need For Memory for Optimal Schedules}
\label{exmp:memory}
\begin{figure}[h]
    \centering
     \begin{tikzpicture}[shorten >=1pt, node distance=2.3 cm, on grid, auto,thick,initial text=]
\begin{scope}[every node/.style={scale=1}]
\node (l0) [state, fill = safecellcolor] {$q_0$};
\node (l1) [state,right = of l0, fill = goodcellcolor]   {$q_1$};
\node (l2) [state,left = of l0,fill = badcellcolor]   {$q_2$};
\end{scope}
 \begin{scope}
\path [-stealth, thick]
    (l0) edge [bend left]  node [above] {$a_1,r_1$}   (l1)
    (l0) edge [bend right]   node [above] {$a_2,r_2$}   (l2)
    (l1) edge [bend left]  node [below] {$d_1,r_3$}   (l0)
    (l1) edge  [loop above] node [above] {$d_2,r_4$}   ()
    (l2) edge  [loop above] node [above] {$c_1,r_5$}   ()
    (l2) edge [bend right] node [below] {$c_2, r_6$} (l0)
    ;
\end{scope}
\end{tikzpicture}
  \label{fig:exmp2}
  \end{figure}
  
\begin{example}[Why memory is required to satisfy $\omega$-regular properties.]
\label{ex:mem}
Consider the CTMDP given above, let the atomic propositions be $\mathtt{b}$ and $\mathtt{g}$ representing blue and green respectively.
Thus the labels on states $q_1$ and $q_2$ are defined as, $\labelling(q_1) = \neg \mathtt{b} \land \mathtt{g}$ and $\labelling(q_2) = \mathtt{b} \land \neg \mathtt{g}$.   
Consider the $\omega$-regular property to be satisfied be $\phi = \always \eventually(\mathtt{b}) \land \always \eventually(\mathtt{g})$. 

We can observe that for a schedule to satisfy this property, both $q_1$ and $q_2$ have to be seen infinitely often. 
As there are no transitions between $q_1$ and $q_2$, both the states can be visited infinitely often only via $q_0$. 
Therefore, the schedule cannot be memoryless as choosing any one action from $q_0$ would not satisfy $\phi$. 



\end{example}

% \begin{example}[$\omega$-regular objectives and memory]
% \label{ex:mem}
% This example demonstrate the need of memoryful schedules for $\omega$-regular objectives. Consider an MDP with single state and two self-loops with actions $a$ and $b$. 
% For an specification $\nextt a \wedge \nextt \nextt b$, it is clear that we need memory to satisfy the objective.
% \end{example} 


\section{Blackwell Optimality In CTMDP}
\label{sec:blackwell_optimality}
% \subsection{Blackwell Optimality In CTMDP}
% \track{Though both of our semantics use different reward machines, for learning schedules for both objectives, we reduce the problem to maximising the expected average reward.
% We use off-the-shelf RL algorithms for learning the schedulers in both settings.}
% Standard RL algorithms try to optimise discounted payoff objective while 
% % as shown above, for the expectation semantics, 
% we need to optimise the expected average reward. Blackwell optimality allows us to use a schedule that optimises the expected discounted payoff with a high discount factor and such a schedule also optimises the expected average payoff.

% \vspace{0.5em}\noindent\textbf{Existence of Blackwell Optimal Schedules in CTMDP.} 
% We give a simple uniformization based proof on the existence of a Blackwell optimal pure schedule in a CTMDP.
In this section, we provide a uniformization based proof of Theorem~\ref{corollary:1}.
Consider a CTMDP $\Mdp$, a pure schedule $\sigma$, and a continuous-time discounting with parameter $\dfactor > 0$. 
% The one-step expected reward obtained from taking action $a \in \actions{}(s)$ from state $s \in \states$ is given by $\rho(s,a) = rew(s,a) + \frac{rew(s)}{\dfactor + \lambda(s,a)}$ (~$\!\!$\cite{puterman2014markov}~Eq 11.5.3). 
We define a function $\ctmdprate : \states \times \actions \rightarrow [0,1) $ where $\ctmdprate(s,a) = \frac{\lambda(s,a)}{\lambda(s,a) + \dfactor}$ where $\dfactor > 0$.
We call $\ctmdprate(s,a)$ the \emph{discount rate} of the state-action pair $(s,a)$ in $\Mdp$.
So, the expected discounted reward (also known as value of $s$) $\discobjective^{\Mdp[\sigma]} (\alpha)(s)$ is given by
% For a pure schedule $\sigma$, the value of a state $s$, denoted by $\valuesigma_{\ctmdprate(s,\sigma(s))}$,
% \track{CHANGE NOTATION} 
% is given by 
\begin{equation}\label{eq:7.1}
\begin{split}
     \rho(s,a_{\sigma}) + 
     \ctmdprate(s,a_{\sigma})
     \sum_{s' \in s} \pmtrx(s,a_\sigma,s') 
     \discobjective_{\sigma}^{\Mdp}(\alpha)(s')
\end{split}
\end{equation}
Consider a DTMDP $\mathcal{N}$, a schedule $\sigma$, and a discount rate $0\leq \drate < 1$. Let $\valuesigma_{\drate}(\mathcal{N},s)$ denote the total discounted value from state $s$ in $\mathcal{N}$ under schedule $\sigma$.

A pure schedule $\bschedule$ is \emph{Blackwell optimal} in $\mathcal{N}$ if there exists a threshold discount rate $0 \leq \thrate < 1$ such that for any discount rate $\thrate \leq \drate < 1$, we have $\valuestar_{\drate}(\mathcal{N},s) \geq \valuesigma_{\drate}(\mathcal{N},s)$ for all $\sigma \in \Sigma_{\mathcal{N}}$. It is known that a Blackwell optimal schedule maximises both discounted and average reward objectives in DTMDPs.
% From \cite{puterman2014markov}~Thm 10.1.4, we have,
% \begin{theorem}\label{puterman1}
From \cite{puterman2014markov}~(Thm 10.1.4), we have that for every DTMDP $\mathcal{N}$, there exists a Blackwell optimal \emph{pure} schedule $\bschedule$, and $\bschedule$ also maximises the average reward in $\mathcal{N}$.
% \end{theorem}
Now, given a CTMDP $\Mdp$, let $C$ be a constant such that $C \geq \lambda(s,a)$ for all state-action pairs in $\Mdp$. 
Let $\uMdp$ be the uniformized CTMDP of $\Mdp$ with constant exit rate $C$, and let $\upmtrx$ be the probability matrix of $\uMdp$. As the exit rate $\lambda(s,a) = C$ for all $s\in \states$ and $a \in \actions{}(s)$, we have that $\ctmdprate(s,a) = \frac{C}{C+\dfactor}$ for all state-action pairs. We denote this discount rate by $\udrate$. 
The value of a state $s$ under a schedule $\sigma$ in $\uMdp$, 
% \track{under schedule $\sigma$}
denoted $\discobjective^{\uMdp^{\sigma}}(\alpha)(s)$
% $\valuesigma_{\udrate}(\uMdp,s)$ 
is given by, 
\begin{equation}\label{eq:7.2}
     \Bar{\rho}(s,a_\sigma) + \udrate \sum_{s' \in s} \upmtrx(s,a_\sigma,s') \discobjective^{\uMdp^{\sigma}}(\alpha)(s')
\end{equation}
where $\Bar{\rho}(s,a) = \rho(s,a)\cdot \frac{\dfactor + \lambda(s,a)}{\dfactor + C} $. 
We extend the above result of existence of Blackwell optimal schedules in DTMDPs to uniform CTMDPs.
\begin{lemma} \label{lemma:1}
For a uniform CTMDP $\uMdp$, there exists a Blackwell optimal schedule $\bschedule$. Further, $\bschedule$ also maximises the expected average reward in $\uMdp$.
\end{lemma}
\begin{proof}
Consider a DTMDP $\mathcal{N}$ with the same set of states as that of $\uMdp$, one step reward function $\Bar{r}$ and probability matrix $P_{\mathcal{N}} = \upmtrx$. For a pure schedule $\sigma$ and a discount rate $0 \leq \drate < 1$, the value of a state $s$ in $\mathcal{N}$ is 
\begin{equation}\label{eq:7.3}
    \valuesigma_{\drate}(\mathcal{N},s) = \Bar{r}(s,\sigma(s)) + \drate \sum_{s' \in s} \upmtrx(s,\sigma(s),s') \valuesigma_{\drate}(\mathcal{N},s')
\end{equation}
We observe that equation \ref{eq:7.3} is identical to equation \ref{eq:7.2} when $\udrate = \drate$. Therefore, the set of equations defining the values of states in $\mathcal{N}$ and $\uMdp$ are identical. Let this set be denoted by $E^\sigma$. From \cite{puterman2014markov}~Thm 6.1.1, we know that for each stationary schedule $\sigma$, there exists a unique solution for $E^\sigma$. The set of pure schedules in $\uMdp$ and $\mathcal{N}$ are equal as the set $S$ of states and the set $\av$ of available actions from each state are the same for both.\\*
Therefore, for a pure schedule $\sigma$, and discount rates $0 \leq \drate = \udrate < 1$, we have
\begin{equation}\label{eq:7.4}
    \valuesigma_{\udrate}(\uMdp,s) = \valuesigma_{\drate}(\mathcal{N},s) \text{\: for $\udrate = \drate$, for all states $s$}
\end{equation}
From Theorem~10.1.4 in \cite{puterman2014markov}, we know that in $\mathcal{N}$, there exist a Blackwell optimal pure schedule $\sigma^*$, and a threshold discount rate $\thrate$ such that 
\begin{align}\label{eq:7.5}
 \valuestar_{\drate}(\mathcal{N},s) \geq \valuesigma_{\drate}(\mathcal{N},s)   
\end{align}
for all $\sigma \in \Sigma_{N}$ and $\thrate \leq \drate < 1$.\\
From equations \ref{eq:7.4} and \ref{eq:7.5} we can conclude that 
\begin{equation}\label{eq:7.60}
    \valuestar_{\udrate}(\uMdp,s) \geq \valuesigma_{\udrate}(\uMdp,s)   
\end{equation}
for all $\sigma \in \Sigma_{\uMdp}^{pure}$ and $\thrate \leq \udrate < 1$.\\
From \cite{puterman2014markov}~Thm 11.5.2(d), we know that there exists an optimal pure schedule maximising the discounted reward in a CTMDP. Therefore,
\begin{equation}\label{eq:7.}
    \valuestar_{\udrate}(\uMdp,s) \geq \valuesigma_{\udrate}(\uMdp,s)   
\end{equation}
for all $\sigma \in \Sigma_{\uMdp}$ and $\thrate \leq \udrate < 1$.\\
A similar argument can be made to show that $\sigma^*$ also maximises the expected average reward in $\uMdp$.
\end{proof}

The above lemma proves the existence of a Blackwell optimal schedule in uniform CTMDPs. 
We further extend this result to general CTMDPs which is the main result of this section.

\begin{lemma}\label{lemma:2}
If $\sigma^*$ is a Blackwell optimal pure schedule in $\uMdp$, then it is also Blackwell optimal in $\Mdp$.
\end{lemma}
\begin{proof}
Since $\bstrategy$ is a Blackwell optimal pure schedule in $\uMdp$, there exists a threshold discount rate $\uthrate$ such that for all $\uthrate \leq \udrate < 1$, we have that 
\begin{equation}\label{eq:7.7}
    \valuestar_{\udrate}(\uMdp,s) \geq \valuesigma_{\udrate}(\uMdp,s)   
\end{equation}
for all $\sigma \in \Sigma_{\uMdp}$.\\
The set of pure schedules in $\Mdp$ and $\uMdp$ are the same. From \cite{puterman2014markov}~Thm 11.5.2(d), we know that there exists an optimal pure schedule maximising the discounted reward in $\Mdp$. \cite{puterman2014markov}~Prop 11.5.1, states that for every pure schedule $\sigma$ and a state $s$, we have that  
\begin{equation}\label{eq:7.8}
    \valuesigma_{\ctmdprate(s,\sigma(s))}(\Mdp,s) = \valuesigma_{\udrate}(\uMdp,s)   
\end{equation}
If $\uthrate = \frac{C}{C+\dfactor}$ is the threshold discount rate in $\uMdp$, then the corresponding threshold discount rate for a state $s$ in $\Mdp$ is given by 
$\ctmdpthrate(s,a) = \frac{\lambda(s,a)}{\lambda(s,a)+\dfactor_{o}}$.\\
From equations \ref{eq:7.7} and \ref{eq:7.8}, we can conclude that for each state $s$ in $\Mdp$, there exist a pure schedule $\bstrategy$, and a threshold discount rate $\ctmdpthrate(s,\bstrategy(s)) $ such that for all $\ctmdpthrate(s,\bstrategy(s)) \leq \ctmdprate(s,\bstrategy(s)) < 1$, we have that
\begin{equation}\label{eq:7.9}
    \valuestar_{\ctmdprate(s,\bstrategy(s))}(\Mdp,s) \geq \valuesigma_{\ctmdprate(s,\sigma(s))}(\Mdp,s)   
\end{equation}
for all $\sigma \in \Sigma_{\Mdp}$.
As the set of states is finite in $\Mdp$, the threshold discount rate for $\Mdp$ is given by
    $\ctmdpthrate^{\Mdp} = \max_{(s,a) \in \states \times \actions} \ctmdpthrate(s,a)$.
\end{proof}
Thus, any Blackwell optimal schedule $\bstrategy$ in $\uMdp$ is also Blackwell optimal in $\Mdp$. The following lemmas show that $\sigma^*$ also maximises the expected average reward.
\begin{lemma}\label{lemma:3}
An optimal schedule maximising the expected average reward in $\uMdp$ also maximises the expected average reward in $\Mdp$.
\end{lemma}
\begin{proof}
For a pure schedule $\sigma$ , the expected average reward in $\Mdp$ is denoted by $\avgrew^{\sigma}(\Mdp)$. From \cite{puterman2014markov}~Chap 11.5.3, we observe that for a pure schedule $\sigma$, 
\begin{equation}\label{eq:7.10}
    \avgrew^{\sigma}(\Mdp) = \avgrew^{\sigma}(\uMdp)\cdot C 
\end{equation}
Let $\sigma'$ be a pure schedule maximising the expected average reward in $\uMdp$, i.e, 
\begin{equation}\label{eq:7.11}
 \avgrew^{\sigma'}(\uMdp) = \sup_{\sigma \in \Sigma_{\uMdp}} \avgrew^{\sigma}(\uMdp)
\end{equation}
The set of pure schedules in $\Mdp$ and $\uMdp$ are equal and we know that there exists a pure schedule maximising the average reward in a CTMDP (\cite{puterman2014markov}~Thm 11.4.6(d)). Therefore, from equations \ref{eq:7.10} and \ref{eq:7.11} we can conclude that, 
\begin{equation}\label{eq:7.12}
    \avgrew^{\sigma'}(\Mdp) = \sup_{\sigma \in \Sigma_{\Mdp}} \avgrew^{\sigma}(\Mdp)
\end{equation}
Therefore, $\sigma'$ is an optimal schedule maximising the average reward in $\Mdp$.
\end{proof}
\begin{lemma}\label{lemma:4}
A Blackwell optimal schedule $\bstrategy$ in $\Mdp$ also maximises the average reward in $\Mdp$.
\end{lemma}
\begin{proof}
From Lemma \ref{lemma:1}, we know that $\bstrategy$ is an optimal schedule maximising the expected average reward in $\uMdp$. Lemma \ref{lemma:3} shows that if $\bstrategy$ is an optimal schedule maximising the expected average reward in $\uMdp$ then $\bstrategy$ also maximises the expected average reward in $\Mdp$. Therefore, we can conclude that $\bstrategy$ is an optimal schedule maximising the expected average reward in $\Mdp$.
\end{proof}
Lemma~\ref{lemma:2} and Lemma~\ref{lemma:4} gives us the following.
\paragraph{Theorem~\ref{corollary:1}.}For a CTMDP $\Mdp$, there exists a Blackwell optimal pure schedule $\bschedule$ and a threshold $0 \leq \ctmdpthrate^{\Mdp} < 1$ such that :
\begin{inparaenum}[(1).]
    \item For any discount-rate function $\ctmdprate$ where $\ctmdprate(s,a) \geq \ctmdpthrate^{\Mdp}$ for all valid state-action pairs $(s,a)$, the schedule $\bschedule$ is an optimal schedule maximising the expected discounted reward.
    \item The schedule $\bschedule$ also maximises the expected average reward.
\end{inparaenum}
% \paragraph{Theorem ~\ref{corollary:1}}
% For a CTMDP $\Mdp$, there exists a Blackwell optimal pure schedule $\bschedule$ and a threshold $0 \leq \ctmdpthrate^{\Mdp} < 1$ such that :
% \begin{inparaenum}[(1).]
%     \item For any discount-rate function $\ctmdprate$ where $\ctmdprate(s,a) \geq \ctmdpthrate^{\Mdp}$ for all valid state-action pairs $(s,a)$, the schedule $\bschedule$ is an optimal schedule maximising the expected discounted reward.
%     \item The schedule $\bschedule$ also maximises the expected average reward.
% \end{inparaenum}

% \begin{proof}[Proof sketch]
% % \begin{lemma}\label{lemma:2}
% Note that the set of pure schedules are the same in $\Mdp$ and $\uMdp$.
% First we prove that if $\bschedule$ is a Blackwell optimal pure schedule in $\uMdp$, then it is also Blackwell optimal in $\Mdp$. 
% This is done using the fact that the value of each state in $\uMdp$ and $\Mdp$ are equal (Prop 11.5.1 in \cite{puterman2014markov}) i.e, for every pure schedule $\sigma$ and a state $s$, we have that
% \[
% \discobjective^{\ctmc} (\alpha)(s) = \discobjective^{\uMdp^{\sigma}}(\alpha)(s)
% \] 
% % \track{ELABORATE}
% From this equation, we can get the threshold discount rate satisfying the Blackwell optimality condition for $\Mdp$.
% % \end{lemma}
% Thus, we conclude that any Blackwell optimal schedule $\sigma^*$ in $\uMdp$ is also Blackwell optimal in $\Mdp$. 
% % The following lemmas show that $\sigma^*$ also maximises the expected average reward in $\Mdp$.

% Next we show that 
% % \begin{lemma}\label{lemma:3}
% an optimal schedule maximising the expected average reward in $\uMdp$ also maximises the expected average reward in $\Mdp$. 
% For a given pure schedule $\sigma$, we denote the expected average reward in $\Mdp$ and $\uMdp$ from a state $s$ by $\avgobjective^{\Mdp^{\sigma}}(s)$ and $\avgobjective^{\uMdp^{\sigma}}(s)$
% % \track{WE ALREADY HAVE NOTATIONS FOR EXPECTED AVERAGE REWARD} 
% respectively.
% From Section~11.5.3 in \cite{puterman2014markov}, we get the following result:
% \[
% \avgobjective^{\Mdp^{\sigma}}(s) = \avgobjective^{\uMdp^{\sigma}}(s)\cdot C \] 
% % This is due to the fact that 
% % the expected average reward obtained by a schedule in $\uMdp$ is directly proportional to the expected average obtained by the same schedule in $\Mdp$, . \track{ELABORATE}
% From this equation, we have that if a pure schedule maximises the expected average reward in $\uMdp$, it also maximises the expected average reward in $\Mdp$.
% % Therefore, optimal pure schedules maximising the expected average reward in $\Mdp$ and $\uMdp$ are the same. 
% % Now we prove the following lemma.

% In Lemma~\ref{lemma:1}, we have shown the existence of a pure Blackwell optimal schedule $\bschedule$ in $\uMdp$ and it also maximises the expected average reward. 
% From the above results we obtain that $\bschedule$ also is Blackwell optimal and maximises the expected average reward in $\Mdp$.
% % Finally, using Lemma~\ref{lemma:1}, we show that a Blackwell optimal schedule $\bschedule$ in $\Mdp$ also maximises the average reward in $\Mdp$. \track{ELABORATE}
% % We can get this result directly from Lemma~\ref{lemma:1}.
% % \end{lemma}
% % Lemma \ref{lemma:2} and Lemma \ref{lemma:4} give us the following.
% % \begin{corollary}\label{corollary:1}
% % For a CTMDP $\Mdp$, there exists a Blackwell optimal pure schedule $\bschedule$ and a threshold $0 \leq \ctmdpthrate^{\Mdp} < 1$ such that :
% % \begin{inparaenum}[(1).]
% %     \item For any discount-rate function $\ctmdprate$ where $\ctmdprate(s,a) \geq \ctmdpthrate^{\Mdp}$ for all valid state-action pairs $(s,a)$, the schedule $\bschedule$ is an optimal schedule maximising the expected discounted reward.
% %     \item The schedule $\bschedule$ also maximises the expected average reward.
% % \end{inparaenum}
% % \end{corollary}
% \end{proof}

% \track{ In the following sections, we design reward functions for specific objectives such that a schedule maximising the expected average reward will also maximise the given objective. As standard RL algorithms optimise discounted reward objectives, we use the result from Theorem~\ref{corollary:1} to conclude that for a large enough discount factor, the RL algoirithm will synthesise the optimal schedule. }

\begin{comment}
\vspace{0.5em}\noindent\textbf{Algorithm for Expectation Semantics.} We show the procedure to obtain an optimal schedule $\sigma$ for expectation semantics in Algorithm \ref{algo:expt}.  Line 1 initialises the Q-function for reinforcement learning. Here, the Q-function is defined on the states of the product CTMDP i.e, $Q_f: (\states \times Q) \times \actions \rightarrow \mathbb{R}$ where $\states$ is the set of states of the CTMDP $\Mdp$ and $Q$ is the set of states of the \textbf{GFM} $A$.
% Variable $i$ keeps track of the number of episodes completed and variable $j$ keeps track of the length of an episode.
The number of episodes to be conducted and the length of the episode is defined by the user and is stored in $k$ and $eplen$ respectively.
The variables $s$ and $q$ represents the current state of $\Mdp$ and $A$ respectively and is initialised to their initial states. The action $a$ from the current state of CTMDP and the transition $t$ from the state of GFM is picked according to the RL policy used (eg. $\epsilon$-greedy). We observe the next state $(s',q')$ and the time spent $\tau$ in the current state. The reward function $R'$ is defined based on $r'$ defined previously. For a given state $(s,q)$ and action $a$, if $\tau$ is the observed time spent in $(s,q)$ then,
$$
 R'((s,q),a,\tau) = \begin{cases}
 \tau & \text{if ($s,q$) is an accepting state}\\
 0 & \text{otherwise}
 \end{cases}
 $$
Variable $r$ stores the reward obtained in each iteration of the episode. The Q-function is updated according to the Q-learning rule defined in section \ref{prelims}. An episode ends when the length of the episode reaches $eplen$. After the completion of $k$ episodes, we obtain the optimal schedule $\sigma$ by choosing the action that gives the highest Q-value from each state.

\begin{algorithm}[t]
\caption{Algorithm for expectation semantics}\label{algo:expt}
\hspace*{\algorithmicindent} \textbf{Input:}  \text{Initial state $s_{init}$, GFM $A$, discount factor $\gamma$,}\\
\hspace*{\algorithmicindent} \hspace{11mm}\text{reward function $R'$, number of episodes $k$,} \\
\hspace*{\algorithmicindent}\hspace{11mm} \text{learning rate $\alpha$, episode length $eplen$}\\
\hspace*{\algorithmicindent} \textbf{Output:}  \text{Optimal schedule $\sigma$}
\begin{algorithmic}[1]
\STATE Initialise $Q_f$ to all zeroes
% \STATE $i \leftarrow 0$
\FOR{$k$ episodes}
\STATE Initialise $s$ and $q$ to $s_0$ and $q_0$ respectively
% \STATE $q \leftarrow q_0$
% \STATE $r \leftarrow 0$
% \STATE $j \leftarrow 0$
\FOR{each episode}
\STATE Choose action $a$ using policy derived from $Q$ 
\STATE Take action $a$, observe next state $s'$, and time $\tau$
\STATE Choose non-deterministic transition $t$ in $A$ using the derived policy
\STATE Take transition $t$ in $A$, observe next state $q'$
\STATE $r \leftarrow R'(s,q,a,\tau)$
\STATE $V(s',q') \leftarrow \max\limits_{a' \in \actions} Q_f(s',q',a')$
\STATE $Q_f(s,q,a) {\leftarrow} (1-\alpha) Q_f(s,q,a)  {+}\alpha \bigl(r {+} e^{-\gamma\tau} V(s',q') \bigr)$
\STATE $s \leftarrow s'$; $q \leftarrow q'$
% \STATE $q \leftarrow q'$
% \STATE $j \leftarrow j+1$
\ENDFOR
% \STATE $i \leftarrow i+1$
\ENDFOR
\FOR{Each state (s,q)}
\STATE $\sigma(s,q) = \max_{a \in \actions}Q(s,q,a)$
\ENDFOR
\end{algorithmic}
\end{algorithm}
\end{comment}


% \begin{center}
%     {\Large Appendix}
% \end{center}
% \section{Appendix}
% \section{Appendix for Proofs}

\paragraph{Proof of Theorem \ref{thm:main}.}

\begin{proof}
\label{proof:main}
Our proof has two steps. In Step 1, we will show that SimCLR is equivalent to minimizing the cross entropy loss defined in Eqn.~(\ref{eqn:cross-entropy}). 
In Step 2, we will show  that minimizing the cross-entropy loss 
is equivalent to spectral clustering on $\bfpi$. 
Combining the two steps together, we have proved our theorem. 

\textbf{Step 1: } SimCLR is equivalent to minimizing the cross entropy loss.

The cross-entropy loss takes expectation over 
$\bfW_\bfX\sim \mathbb{P}(\cdot ; \bfpi)$, 
which means $\bfW_\bfX$ has exactly one non-zero entry in each row $i$. By Lemma~\ref{lem:multinomial}, we know every row $i$ of $\bfW_\bfX$ is independent of other rows. Moreover, 
$\bfW_{\bfX,i}\sim \mathcal{M}(1, \bfpi_i/\sum_j \bfpi_{i,j})=\mathcal{M}(1, \bfpi_i)$, because $\bfpi_i$ itself is a probability distribution.
Similarly, we know $\bfW_\bfZ$ also has the row-independent property by sampling over $\mathbb{P}(\cdot;\bfK_\bfZ)$.
Therefore, by Lemma~\ref{lem:cross_split}, we know Eqn.~(\ref{eqn:cross-entropy}) is equivalent to:
\[
 -\sum_{i=1}^n \mathbb{E}_{\bfW_{\bfX,i}}[\log \mathbb{P}(\bfW_{\bfZ,i}=\bfW_{\bfX,i};\bfK_\bfZ)],
\]

This expression takes expectation over $\bfW_{\bfX,i}$ for the given row $i$. Notice that 
$\bfW_{\bfX,i}$ has exactly one non-zero entry, which equals $1$ (same for $\bfW_{\bfZ,i}$). 
As a result
we expand the above expression to be:
\begin{equation}
 -\sum_{i=1}^n \sum_{j\neq i} \Pr(\bfW_{\bfX,i,j}=1)\log \Pr(\bfW_{\bfZ,i,j}=1).
\label{eqn:detailed-expansion}    
\end{equation}


By Lemma~\ref{lem:multinomial}, $\Pr(\bfW_{\bfZ,i,j}=1)=\bfK_{\bfZ,i,j}/\|\bfK_{\bfZ,i}\|_1$ for $j\neq i$. Recall that $\bfK_\bfZ=(k(\bfZ_i-\bfZ_j))_{(i,j)\in[n]^2}$, which means 
$\bfK_{\bfZ,i,j}/\|\bfK_{\bfZ,i}\|_1=\frac{\exp(-\|\bfZ_i-\bfZ_j\|^2/{2\tau})}{\sum_{k\neq i}
\exp(-\|\bfZ_i-\bfZ_k\|^2/{2\tau})
}$ for $j\neq i$, when $k$ is the Gaussian kernel with variance $\tau$. 

Notice that $\bfZ_i=f(\bfX_i)$, so we know
\begin{equation}
-\log \Pr(\bfW_{\bfZ,i,j}=1)=
-\log \frac{\exp(-\|f(\bfX_i)-f(\bfX_j)\|^2/{2\tau})}{\sum_{k\neq i}
\exp(-\|f(\bfX_i)-f(\bfX_k)\|^2/{2\tau}),
}
\label{eqn:infonce-equivalence}    
\end{equation}


The right hand side is exactly the InfoNCE loss defined in Eqn.~(\ref{eqn:infonce}).
Inserting Eqn.~(\ref{eqn:infonce-equivalence}) into Eqn.~(\ref{eqn:detailed-expansion}), we get the SimCLR algorithm, which first samples augmentation pairs $(i,j)$ with $\Pr(\bfW_{\bfX,i,j}=1)$ for each row $i$, and then optimize the InfoNCE loss. 

\textbf{Step 2: } minimizing the cross entropy loss 
is equivalent to spectral clustering on $\bfpi$.


By Lemma~\ref{lem:convert_to_spectral}, we may further convert the loss to 
\begin{equation}
\label{eqn:main-theorem-repul-attr}
\min_{\bfZ}
-\sum_{(i,j)\in [n]^2} \mathbf{P}_{i,j}
\log k (\bfZ_i-\bfZ_j)+\log \mathbf{R}(\bfZ).
\end{equation}
Since $k$ is the Gaussian kernel, this reduces to \[
\min_\bfZ \mathrm{tr}(\bfZ^\top \mathbf{L}(\bfpi) \bfZ)
+\log \mathbf{R}(\bfZ),
\]

where we use the fact that $\mathbb{E}_{\bfW_\bfX\sim \mathbb{P}(\cdot; \bfpi)}[\mathbf{L}(\bfW_\bfX)]
=\mathbf{L}(\bfpi)
$, because the Laplacian operator is linear and $
\mathbb{E}_{\bfW_\bfX\sim \mathbb{P}(\cdot; \bfpi)}(\bfW_\bfX)=\bfpi
$.
\end{proof}

\paragraph{Proof of Theorem \ref{thm:clip}.}
\begin{proof}
Since $\bfW_\bfX\sim \mathbb{P}(\cdot;\bfpi_{\mathbf{A}, \mathbf{B}})$, we know 
$\bfW_\bfX$ has exactly one non-zero entry in each row, denoting the pair that got sampled. 
A notable difference compared to the previous proof is we now have $n_\mathcal{A}+n_\mathcal{B}$ objects in our graph. CLIP deals with this by taking a mini-batch of size $2N$, 
such that $n_\mathcal{A}=n_\mathcal{B}=N$, and adding the $2N$ InfoNCE losses together. We label the objects in $\mathcal{A}$ as $[n_\mathcal{A}]$, and the objects in $\mathcal{B}$ as $\{n_\mathcal{A}+1, \cdots, n_\mathcal{A}+n_\mathcal{B}\}$. 

Notice that $\bfpi_{\mathbf{A}, \mathbf{B}}$ is a bipartite graph, so the edges of objects in $\mathcal{A}$ will only connect to object in $\mathcal{B}$ and vice versa. We can define the similarity matrix in $\cZ$ as $\bfK_\bfZ$, 
where $\bfK_\bfZ(i, j+n_\mathcal{A})=\bfK_\bfZ(j+n_\mathcal{A},i)= k(\bfZ_i-\bfZ_j)$ for $i\in [n_\mathcal{A}], j\in [n_\mathcal{B}]$, and otherwise we set $\bfK_\bfZ(i,j)=0$. 
The rest is same as the previous proof. 
\end{proof}

\paragraph{Proof of Theorem \ref{thm:exponential}.}

\begin{proof}
\label{proof:exponential}
Since the objective function consists of a linear term combined with an entropy regularization, which is a strongly concave function, the maximization problem is a convex optimization problem. Owing to the implicit constraints provided by the entropy function, the problem is equivalent to having only the equality constraint. We then introduce the Lagrangian multiplier $\lambda$ and obtain the following relaxed problem:

$$
\widetilde{E}(\boldsymbol{\alpha})=\psi_{1}-\sum_{i=1}^n \alpha_{i} \psi_{i}+\tau \sum_{i=1}^n \alpha_{i}\log \alpha_{i}+\lambda\left(\boldsymbol{\alpha}^{\top} \mathbf{1}_n-1\right).
$$

As the relaxed problem is unconstrained, taking the derivative with respect to $\alpha_{i}$ yields

$$
\frac{\partial \widetilde{E}(\boldsymbol{\alpha})}{\partial \alpha_{i}}=-\psi_{i}+\tau\left(\log \alpha_{i}+\alpha_{i} \frac{1}{\alpha_{i}}\right)+\lambda=0.
$$

Solving the above equation implies that $\alpha_{i}$ takes the form
$
\alpha_{i}=\exp \left(\frac{1}{\tau} \psi_{i}\right) \exp \left(\frac{-\lambda}{\tau}-1\right).
$ Since $\alpha_{i}$ lies on the probability simplex, the optimal $\alpha_{i}$ is explicitly given by
$
\alpha^{*}_{i}=\frac{\exp \left(\frac{1}{\tau} \psi_{i}\right)}{\sum_{i^{\prime}=1}^n \exp \left(\frac{1}{\tau} \psi_{i^{\prime}}\right)} .
$ Substituting the optimal point into the objective function, we obtain
$$
\begin{aligned}
E\left(\boldsymbol{\alpha}^*\right)  &=\psi_1-\sum_{i=1}^n \frac{\exp \left(\frac{1}{\tau} \psi_{i}\right)}{\sum_{i^{\prime}=1}^n \exp \left(\frac{1}{\tau} \psi_{i^{\prime}}\right)} \psi_{i}+\tau \sum_{i=1}^n \frac{\exp \left(\frac{1}{\tau} \psi_{i}\right)}{\sum_{i^{\prime}=1}^n \exp \left(\frac{1}{\tau} \psi_{i^{\prime}}\right)}\log \frac{\exp \left(\frac{1}{\tau} \psi_{i}\right)}{\sum_{i^{\prime}=1}^n \exp \left(\frac{1}{\tau} \psi_{i^{\prime}}\right)} \\
& =\psi_1 - \tau \log \left(\sum_{i=1}^n \exp \left(\frac{1}{\tau} \psi_{i}\right)\right).
\end{aligned}
$$
Thus, the Lagrangian dual function is given by
\begin{equation*}
-E\left(\boldsymbol{\alpha}^*\right)= -\tau \log \frac{\exp \left(\frac{1}{\tau} \psi_{1}\right)}{\sum_{i=1}^n \exp \left(\frac{1}{\tau} \psi_{i}\right)}.\qedhere
\end{equation*}
\end{proof}



\section{More on Experiments} \label{section: experiment_details}

\paragraph{CIFAR-10 and CIFAR-100} CIFAR-10 ~\citep{krizhevsky2009learning} and CIFAR-100 ~\citep{krizhevsky2009learning} are well-known classic image classification datasets. Both CIFAR-10 and CIFAR-100 contain a total of 60k $32 \times 32$ labeled images of different classes, with 50k for training and 10k for testing. CIFAR-10 is similar to CIFAR-100, except there are 10 different classes in CIFAR-10 and 100 classes in CIFAR-100.

\paragraph{TinyImageNet} TinyImageNet ~\citep{le2015tiny} is a subset of ImageNet ~\citep{deng2009imagenet}. There are 200 different object classes in TinyImageNet, with 500 training images, 50 validation images, and 50 test images for each class. All the images in TinyImageNet are colored and labeled with a size of $64 \times 64$.

\textbf{Pseudo-code.} Algorithm \ref{alg:Training Procedure} presents the pseudo-code for our empirical training procedure.

\begin{algorithm}[!htbp]
\caption{Training Procedure}
\label{alg:Training Procedure}
\begin{algorithmic}[1]
\REQUIRE trainable encoder network $f$, batch size $N$, augmentation strategy \textit{aug}, loss function $L$ with hyperparameters \textit{args}
\FOR {sampled minibatch ${x_i}_{i=1}^N$}
\FORALL{$i \in { 1, ..., N }$}
\STATE draw two augmentations $t_i = \textit{aug}\left(x_i\right) $, $t_i' = \textit{aug}\left(x_i\right) $
\STATE $z_i = f\left(t_i\right)$, $z_i' = f\left(t_i'\right)$
\ENDFOR
\STATE compute loss $\mathcal{L} = L(N, z, z', \textit{args})$
\STATE update encoder network $f$ to minimize $\mathcal{L}$
\ENDFOR
\STATE \textbf{Return} encoder network $f$
\end{algorithmic}
\end{algorithm}

We also provide the pseudo-code for our core loss function used in the training procedure in Algorithm \ref{alg:Core loss}. The pseudo-code is almost identical to SimCLR's loss function, with the exception of an extra parameter $\gamma$.

\begin{algorithm}[!htbp]
\caption{Core loss function $\mathcal{C}$}
\label{alg:Core loss}
\begin{algorithmic}[1]
\REQUIRE batch size $N$, two encoded minibatches $z_1, z_2$, $\gamma$, temperature $\tau$
\STATE $z = \textit{concat}\left(z_1, z_2\right)$
\FOR {$i \in {1, ..., 2N }, j \in {1, ..., 2N}$ }
\STATE $s_{i,j} = \Vert z_i - z_j \Vert_2^{\gamma}$
\ENDFOR
\STATE \textbf{define} $l(i, j)$ \textbf{as} $l(i, j) = - \log \frac{exp\left(s_{i,j}/\tau \right)}{\sum_{k=1}^{2N} \mathbf{1}{[k \ne i]} exp\left(s{i, j} / \tau \right)} $
\STATE \textbf{Return} $\frac{1}{2N} \sum_{k=1}^N\left[l(i, i+N) + l(i+N, i)\right]$
\end{algorithmic}
\end{algorithm}

Utilizing the core loss function $\mathcal{C}$, we can define all kernel loss functions used in our experiments in Table \ref{table: loss definition}. For all $z_i \in z$ with even dimensions $n$, we define $z_{L_i} = z_i\left[0:n/2\right]$ and $z_{R_i} = z_i\left[n/2:n\right]$.

\begin{table}[ht]
\centering
\begin{tabular}{{@{}l|l@{}}}
Kernel  &  Loss function \\ \midrule
Laplacian & $\mathcal{C}\left(N, z, z', \gamma=1, \tau\right)$\\ \midrule
Sum       & $\lambda * \mathcal{C}\left(N, z, z', \gamma=1, \tau_1\right) + (1-\lambda) * \mathcal{C}\left(N, z, z', \gamma=2, \tau_2\right)$  \\ \midrule
Concatenation Sum&$\lambda * \mathcal{C}\left(N, z_L, z'_L, \gamma=1, \tau_1\right) + (1-\lambda) * \mathcal{C}\left(N, z_R, z'_R, \gamma=2, \tau_2\right)$\\ \midrule
$\gamma = 0.5$ & $\mathcal{C}\left(N, z, z', \gamma=0.5, \tau\right)$          \\ 

\end{tabular}

\caption{Definition of kernel loss functions in our experiments}
\label {table: loss definition}
\end{table}

\textbf{Baselines.} We reproduce the SimCLR algorithm using PyTorch Lightning~\citep{PytorchLightning}.

\textbf{Encoder details.}
The encoder $f$ consists of a backbone network and a projection network. We employ ResNet50~\citep{ResNet} as the backbone and a 2-layer MLP (connected by a batch normalization~\citep{ioffe2015batch} layer and a ReLU \cite{nair2010rectified} layer) with hidden dimensions 2048 and output dimensions 128 (or 256 in the concatenation kernel case).

\textbf{Encoder hyperparameter tuning.}
For each encoder training case, we randomly sample 500 hyperparameter groups (sample details are shown in Table \ref{table: Hyperparameter sample}) and train these samples simultaneously using Ray Tune ~\citep{RayTune}, with the ASHA scheduler~\citep{li2018massively}. Ultimately, the hyperparameter group that maximizes the online validation accuracy (integrated in PyTorch Lightning) within 5000 validation steps is chosen for the given encoder training case.

\begin{table}[ht]
\centering

\begin{tabular}{@{}l|l|l@{}}
\midrule
Hyperparameter  & Sample Range & Sample Strategy \\ \midrule
start learning rate & $\left[10^{-2}, 10\right]$ & log uniform \\ \midrule
$\lambda$       & $\left[0, 1\right]$ & uniform \\ \midrule
$\tau$, $\tau_1$, $\tau_2$ & $\left[0, 1\right]$ & log uniform \\ \midrule
\end{tabular}

\caption{Hyperparameters sample strategy}
\label {table: Hyperparameter sample}
\end{table}

\textbf{Encoder training.} 
We train each encoder using the LARS optimizer~\citep{LARSOptimizer}, LambdaLR Scheduler in PyTorch, momentum 0.9, weight decay $10^{-6}$, batch size 256, and the aforementioned hyperparameters for 400 epochs on a single A-100 GPU.

\textbf{Image transformation.} The image transformation strategy, including augmentation, is identical to the default transformation strategy provided by PyTorch Lightning.

\textbf{Linear evaluation.}
The linear head is trained using the SGD optimizer with a cosine learning rate scheduler, batch size 64, and weight decay $10^{-6}$ for 100 epochs. The learning rate starts at $0.3$ and ends at $0$.

\textbf{Moco Experiments.} We also tested our method based on MoCo~\citep{he2019moco}. The results are summarized in Table \ref{tab:results-moco}. Here we choose ResNet18~\citep{ResNet} as the backbone and set a temperature of $0.1$ as default. For our simple sum kernel, we set $\lambda=0.8$. The results show that our method outperforms the original MoCo method.

\begin{table}[thb]
\centering
\caption{MoCo Experiment Results on CIFAR-10 and CIFAR-100.}
\label{tab:results-moco}
\resizebox{\textwidth}{!}{%
\begin{tabular}{@{}c|ccc|ccc@{}}
\toprule
\multirow{3}{*}{Method} & \multicolumn{3}{c|}{CIFAR-10} & \multicolumn{3}{c}{CIFAR-100} \\ \cmidrule(lr){2-4} \cmidrule(lr){5-7} 
                        & 200 epochs & 400 epochs    & 1000 epochs   & 200 epochs & 400 epochs & 1000 epochs         \\ \midrule
MoCo (repro.)         & $76.41 \pm 0.12$    & $80.01 \pm 0.15$          & $84.45 \pm 0.08$    & $\mathbf{47.02 \pm 0.11}$ & $52.50 \pm 0.07$ & $57.62 \pm 0.15$            \\
\midrule
Laplacian Kernel        & ${78.09 \pm 0.10}$    & $\mathbf{83.85 \pm 0.09}$          & $\mathbf{88.34 \pm 0.16}$    & $46.12 \pm 0.22$   & $53.44 \pm 0.17$ & $59.10 \pm 0.14$        \\
Simple Sum Kernel & $\mathbf{78.12 \pm 0.15}$   & $83.23 \pm 0.18$ & $87.50 \pm 0.20$ & $46.65 \pm 0.06$ & $\mathbf{53.62 \pm 0.19}$ & $\mathbf{59.83 \pm 0.12}$\\
\bottomrule
\end{tabular}
}
\end{table}



\section{More Experiments on Synthetic Data}


Consider a scenario with $n$ clusters, each containing $k$ vertices. Let the probability of vertices $u$ and $v$ from the same cluster belonging to $\bfpi$ be $p$. Conversely, for vertices $u$ and $v$ from different clusters, let the probability of belonging to $\pi$ be $q$. We generate the graph $\bfpi$ randomly, based on $p$ and $q$. We experiment with values of $k=100$ and $n=6$ for ease of visualization, embedding all points in a two-dimensional space. Each vertex's initial position originates from a normal distribution. In each iteration, we sample a subgraph of $\bfpi$ uniformly, ensuring each vertex has an out-degree of $1$. We then optimize the corresponding vectors using InfoNCE loss with an SGD optimizer and iterate until convergence. Our experimental setup consists of an SGD learning rate of $1$, an InfoNCE loss temperature of $0.5$, and a batch size of $50$. We evaluate two scenarios with different $p$ and $q$ values: $p=1$, $q=0$, and $p=0.75$, $q=0.2$. The results of these experiments are visualized in Figure \ref{fig:vis-spectral-cluster}. The obtained embeddings exhibit the hallmark pattern of spectral clustering of graph $\bfpi$.

\begin{figure}[!tb]
\centering
\subfigure{
\includegraphics[width=1\textwidth]{Figures/cluster_pi.png}
\label{fig:vis-cluster}
}
\subfigure{
\includegraphics[width=1\textwidth]{Figures/noised_cluster_pi.png}
\label{fig:vis-noised-cluster}
}
\caption{Visualizations of the optimization process using InfoNCE Loss on the vectors corresponding to $\bfpi$. Points of identical color belong to the same cluster within $\bfpi$. To showcase the internal structure of $\bfpi$, we randomly select 10 vertices from each cluster to display the edge distribution of $\bfpi$.}
\label{fig:vis-spectral-cluster}
\end{figure}


\section{Proofs from Section ~\ref{sec:theorems&algo}}
\label{app:sat}
In this section, we give a detailed proof of Theorem~\ref{theorem:4.1}.
\paragraph{Theorem~\ref{theorem:4.1}.}There exists a threshold $\zeta' \in (0,1)$ such that for all $\zeta > \zeta'$, and for every state $s$, a schedule maximising the expected average reward in $t$ is 
\begin{inparaenum}[(1)]
\item an optimal schedule in the product CTMDP $\mathcal{M} \times \mathcal{A}$ from $s$ for satisfying the $\omega$-regular objective $\phi$.
Further, since $\oautomata$ is a GFM, we have that \item $\sigma$ induces an optimal schedule for the CTMDP $\mathcal{M}$ from $s$ with objective $\phi$.
\end{inparaenum}
\begin{proof}
For a given CTMDP $\Mdp$, an embedded MDP $\embeddedMdp$ of $\mathcal{M}$ is a discrete-time MDP of the form $\embeddedMdp = (\states, \initstate, \actions{}, \transP_{\Mdp})$, that is, the transition function of $\embeddedMdp$ is derived from the probability matrix of $\Mdp$.
% Given a CTMDP $\mathcal{M}$, we denote the embedded DTMDP by $\emdp$. 
As the set of states and enabled actions from each state are the same in $\Mdp$ and $\emdp$, the set of pure schedules in them are also same. 
% Let $\phi$ be an $\omega$-regular objective, the following lemma gives us the relation between an optimal schedule satisfying $\phi$ in $\pmdp$ and one for the embedded product DTMDP $\emdp \times \oautomata$.
%  

Time does not play a role in the definition of pure schedules. Therefore, the probability of reaching a state $s$ from the initial state in a CTMDP $\Mdp$ under a pure schedule $\sigma$ is dependant only on the transition function of the embedded MDP $\emdp$. 
Now recall that for $\omega$-regular objectives, given a B\"{u}chi GFM, the states of the accepting condition of the product CTMDP need to be visited infinitely often.
As there exist optimal pure schedules for reaching such states, we get the following lemma for an $\omega$-regular objective $\phi$.

\begin{lemma} \label{lem:embedded}
There exists a pure schedule that maximises the probability of satisfying $\phi$ in $\mathcal{M} \times \mathcal{A}$ which also maximises the probability of satisfying $\phi$ in the embedded product DTMDP $\mathcal{M}_\mathcal{E} \times \mathcal{A}$.
\end{lemma}
\begin{proof}
In order to prove this lemma, we consider the semantics of a CTMC.
In a CTMC, every state $s$ has an exit rate $\lambda_s$ such that after reaching state $s$, some time $t_s$ is spent in $s$ where $t_s$ is exponentially distributed with parameter $\lambda_s$, and an outgoing transition to a state $s'$ is taken according to the probability $\transP(s,s')$ of the underlying discrete Markov chain.
Note that given a state $s$, the probability of reaching $s$ from the initial state $\initstate$ of the CTMC thus solely depends on the underlying discrete Markov chain, and not on the exit rates of the states.
Now, for every pure schedule $\sigma$, the CTMC generated is $(\Mdp \times \oautomata)^{[\sigma]}$ and the underlying discrete Markov chain is $(\Mdp_\mathcal{E} \times \oautomata)^{[\sigma]}$.

Thus, the probability of reaching the accepting end-components are the same in both $(\Mdp \times \oautomata)^{[\sigma]}$ and $(\Mdp_\mathcal{E} \times \oautomata)^{[\sigma]}$.
The lemma follows since optimal pure schedules exist for a reachability objective, in particular, there exists pure schedules maximising the probability of reaching accepting end-components, and the set of pure schedules are the same in both $(\Mdp \times \oautomata)^{[\sigma]}$ and $(\Mdp_\mathcal{E} \times \oautomata)^{[\sigma]}$.
Note that time does not play a role in the definition of stationary schedules, that is, timed stationary schedules and time-abstract stationary schedules coincide.
\end{proof}
We have shown that there exists an optimal pure schedule maximising $\phi$ in both $\pmdp$ and $\Mdp_{\mathcal{E}} \times \oautomata$. With a similar argument it also follows that an optimal pure schedule maximising the probability to reach the sink $t$ in the CTMDP $\augmdp$ also maximises the probability to reach $t$ in the DTMDP $\augmdp_{\mathcal{E}}$.

Theorem 3 from \cite{HahnPSSTW19} gives us the existence of a threshold $\zeta' \in (0,1)$ and that for any $\zeta > \zeta'$, an optimal schedule maximising the probability of reaching the sink state $t$ from a state $s$ in the DTMDP $\augmdp_{\mathcal{E}}$ also maximises the probability of satisfying $\phi$ in $\Mdp_{\mathcal{E}} \times \oautomata$.

This leads to the following statement.
There exists a threshold $\zeta' \in (0,1)$ such that for all $\zeta > \zeta'$, and for every state $s$, a schedule $\sigma$ maximising the probability $p_s(\zeta)$ of reaching the sink in $\mathcal{M}_{\mathcal{E}}^{\zeta}$ is
\begin{inparaenum}[(1)]
\item an optimal schedule in the product CTMDP $\mathcal{M} \times \mathcal{A}$ from $s$ for satisfying the $\omega$-regular objective $\phi$, and
\item induces an optimal schedule for the CTMDP $\mathcal{M}$ from $s$ with objective $\phi$.
\end{inparaenum}

The reward machine defined gives a positive reward for each time unit spent in $t$. Therefore, we can conclude that any schedule that maximises the probability of reaching $t$ also maximises the expected average reward.

\end{proof}

\section{Proofs from Section~\ref{sec:d_time}}
\label{app:expt}
In this section, we provide a proof of Lemma~\ref{lemma:6.2}.
\paragraph{Lemma~\ref{lemma:6.2}.}For a product CTMDP $\Mdp \times \oautomata$ where $\oautomata$ is a \textbf{GFM} for an $\omega$-regular objective and for a schedule $\sigma$, the expected average reward obtained w.r.t. the reward function $r'$ is equal to the expected satisfaction time in $~{(\pmdp)}$ and there exist a pure schedule that maximises this.
\begin{proof}
Consider a run $r_{inf} = (s_1,t_1,a_1,s_2,t_2,a_2...)$ in $\Mdp \times \oautomata$ under a schedule $\sigma$ . The satisfaction time is defined as 
$$
    Sat^{\Mdp{\times}\oautomata}_{\oautomata}(s,\sigma) = \liminf_{n \rightarrow \infty} \frac{\sum_{i=1,s_i \in T}^{n} t_i}{\sum_{j=1}^{n}t_j}
$$.
The reward obtained in each state $s_i$ in $r_{inf}$ is $r'(s_i)\cdot t_i$ which is $t_i$ if $s_i \in T$, and 0 otherwise.
Therefore, the average reward obtained from $r_{inf}$ is 
$$  
        \avgreward(s) = \liminf_{n \rightarrow \infty} \frac{\sum_{i=1,s_i \in T}^{n} t_i}{\sum_{j=1}^{n}t_j}
    $$.
Thus, the average reward obtained is equal to the satisfaction time for every run in $\Mdp \times \oautomata$ and therefore, the expected average reward obtained w.r.t. the reward function $r'$ is equal to the expected satisfaction time.

For proving the second part of the lemma, we use the fact that there exists an optimal pure schedule that maximises the expected average reward for any rewardful CTMDP \cite{puterman2014markov}. Therefore, there exists a pure schedule that maximises the expected residence time of $T$ in $\Mdp \times \oautomata$.
\end{proof}


\section{Pseudocode of Algorithms}
\label{algo}
In this section, we provide the pseudocode of the RL algorithm for satisfaction and expectation semantics.
\subsection*{Satisfaction Semantics}
\begin{algorithm}[H]
\caption{Algorithm for satisfaction semantics}\label{algo:sat}
\hspace*{\algorithmicindent}\hspace{2mm}\textbf{Input:}  \text{Initial state $s_{0}$, GFM $A$, discount factor $\gamma$, reward function $rew$, number of episodes (ep-n) $k$, learning rate $\beta$}\\
\hspace*{\algorithmicindent}\hspace{2mm}\textbf{Output:}\text{ Schedule $\sigma$ converging to an optimal one}
\begin{algorithmic}[1]
\STATE Initialise $\Qf$ to all zeroes
% \STATE $i \leftarrow 0$
\FOR{k episodes}
\STATE Initialise $s$ and $q$ to $s_0$ and $q_0$ respectively
\STATE Initialise $r$ to 0
% \STATE $q \leftarrow q_0$
% \STATE $r \leftarrow 0$
\WHILE{r = 0}
\STATE Choose action $a$ using schedule derived from $\Qf$
\STATE Take action $a$, observe next state $s'$ and time $\tau$
\STATE Choose non-deterministic transition $t$ in $A$ using the derived schedule ($\epsilon$-greedy)
\STATE Take transition $t$ in $A$, observe next state $q'$
\STATE $r \leftarrow rew'(s,q,a)$
\STATE $V(s',q') \leftarrow \max\limits_{a' \in \actions} \Qf(s',q',a')$
\STATE $\Qf(s,q,a) \leftarrow (1-\beta) \Qf(s,q,a)  + \beta \bigl(r {+} e^{-\gamma\tau} V(s',q') \bigr)$
\STATE $s \leftarrow s'$
\STATE $q \leftarrow q'$
\ENDWHILE
% \STATE $i \leftarrow i+1$
\ENDFOR
\STATE Initialise $\sigma$
\FOR{each state $(s,q)$}
\STATE $\sigma(s,q) = \max\limits_{a \in \actions}\Qf(s,q,a)$
\ENDFOR
\end{algorithmic}
\end{algorithm}
Recall that the reward function $rew'$ is defined as:
\[
rew'((s,q),a) = \begin{cases}
1 \text{\quad with probability $1-\zeta$ if $(s,q)$ is} \\
\text{\quad \quad accepting} \\
0  \text{\quad otherwise}
\end{cases}
\]

\subsection*{Expectation Semantics}
\begin{algorithm}[H]
\caption{Algorithm for expectation semantics}\label{algo:expt}
\hspace*{\algorithmicindent}\hspace{2mm}\textbf{Input:}  \text{Initial state $s_0$, GFM $A$, discount factor $\gamma$, reward function $rew$, number of episodes (ep-n) $k$,} \\
\hspace*{\algorithmicindent}\hspace{12mm} \text{learning rate $\beta$, episode length (ep-l) $eplen$}\\
\hspace*{\algorithmicindent}\hspace{2mm}\textbf{Output:}\text{ Schedule $\sigma$ converging to an optimal one}
% \hspace*{\algorithmicindent} \textbf{Output:}  \text{Schedule $\sigma$ converging to an optimal one}
\begin{algorithmic}[1]
\STATE Initialise $\Qf$ to all zeroes
% \STATE $i \leftarrow 0$
\FOR{$k$ episodes}
\STATE Initialise $s$ and $q$ to $s_0$ and $q_0$ respectively
% \STATE $q \leftarrow q_0$
% \STATE $r \leftarrow 0$
% \STATE $j \leftarrow 0$
\FOR{each $eplen$-length episode}
\STATE Choose action $a$ using schedule derived from $\Qf$ 
\STATE Take action $a$, observe next state $s'$ and time $\tau$
\STATE Choose non-deterministic transition $t$ in $A$ using the derived schedule ($\epsilon$-greedy)
\STATE Take transition $t$ in $A$ and observe the next state $q'$
\STATE $r \leftarrow rew((s,q),a,\tau)$
\STATE $V(s',q') \leftarrow \max\limits_{a' \in \actions} \Qf(s',q',a')$
\STATE $\Qf(s,q,a)  \leftarrow {(1-\beta)} \Qf(s,q,a)  + \beta \bigl(r {+} e^{-\gamma\tau} V(s',q') \bigr)$
\STATE $s \leftarrow s'$
\STATE $q \leftarrow q'$
% \STATE $q \leftarrow q'$
% \STATE $j \leftarrow j+1$
\ENDFOR
% \STATE $i \leftarrow i+1$
\ENDFOR
\FOR{each state (s,q)}
\STATE $\sigma(s,q) = \max_{a \in \actions}\Qf(s,q,a)$
\ENDFOR
\end{algorithmic}
\end{algorithm}
Recall that he reward function $rew$ is defined based on $r'$, i.e,
$$
 rew((s,q),a,\tau) = \begin{cases}
 \tau & \text{if ($s,q$) is an accepting state}\\
 0 & \text{otherwise}
 \end{cases}
 $$

% \section{Description of benchmarks}
% \label{app:bench}
% A brief description of the benchmarks in Section~\ref{sec:expt} are given below:
% \begin{itemize} 
% \item 
% The {\bf Dynamic Power Management}\cite{dynpow2001} system models a service provider that processes tasks of $N$ different types. The tasks that are to be processed are stored in queues of maximum size $C$ for each type. The service provider can be in three different states: {\tt idle}, {\tt busy} and {\tt standby}. 
% The key parameters here are $N$, $C$ and the time bound. The properties that are checked are with respect to the probability of getting a queue full and the expected time to get the queues full.
% \item 
% The {\bf Fault Tolerant Workstation Cluster}\cite{ftwc2000} benchmark models two sets of workstations (left and right) which are connected by a backbone. 
% A workstation in a set can freely communicate with other workstations on its own set but requires the backbone to communicate with the other set. Each workstation is connected to the backbone via a switch and the size of each set is $N$. 
% The workstation, switch and the backbone can fail and may require repair. The repair unit can repair only one component at a time.
% The property we check is the minimum probability that all the left workstations are down within a time bound.
% The parameters here are $N$ and the time bound. 
% \item 
% The {\bf Polling System}\cite{polling2013} consists of $j$ stations and $1$ server. Here, the incoming requests of $j$ types are buffered in queues of size $k$ each, until they are processed by the server and delivered to their station. The system starts in a state with all the queues being nearly full. The parameters here are $j$, $k$ and a time bound. We consider two properties: (i) all the queues are empty and (ii) one of the queues is empty.
% \item 
% The {\bf Erlang Stages}\cite{erlang2010} benchmark models have two different paths to reach the goal state: a fast but risky path or a slow but sure path. The slow path is an Erlang chain of length $k$ and rate $r$. The objective is to check which path gives faster reachability to a safe target state in an expected sense.
% \item 
% The {\bf Stochastic Job Scheduling} benchmark models a multi-processor system with $k$ processors and $n$ jobs which need to be serviced. 
% The service times are exponentially distributed.
% The jobs can be scheduled pre-emptively and multiple processors can be used to service a single job. The parameters here are $n$ and $k$. The objective is to schedule the jobs efficiently, that is, to minimise the expected time required to service all the jobs. 
% \end{itemize}



