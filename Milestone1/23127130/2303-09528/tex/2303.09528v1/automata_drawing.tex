\begin{figure*}[t]
\noindent\resizebox{0.3\textwidth}{!}{
 \begin{minipage}{0.33\textwidth}
\begin{tikzpicture}[shorten >=1pt]
\begin{scope}
\node (l0) [state,ellipse,initial above, fill = safecellcolor]   {$s_0$};
\node (l1) [state, ellipse, fill = badcellcolor] at (-2.1,0)   {$s_1$};
\node (l2) [state, ellipse, fill = goodcellcolor] at (2.1,0)   {$s_2$};
\node (l3) [state,ellipse, fill = goodcellcolor] at (0,-2.1)   {$s_3$};
% \end{scope}
% \begin{scope}[every node/.style={scale=0.8}]

% \draw[rounded corners,->, thick] (l3) -- (-4.5,-4.3) -- node [below] {$b,(\lambda(0,b){-}r)$}  (l5);
\path [->, thick,above=.5cm,align=center]
    (l0) edge node [above] {$b$} node [below, font = \scriptsize]{$\lambda(0,b)-r$}   (l1)
    (l0) edge [bend right]  node [below] {$b,r$}   (l2)
    (l2) edge [bend right]  node [above] {$f$}   (l0)
    (l0) edge [bend right]  node [left] {$a$}   (l3)
    (l3) edge [bend right]  node [right] {$c$}   (l0)
    (l1) edge [loop above]  node [above] {$d$} ()
    (l2) edge [loop above]  node [above] {$e$} ()
    ;
\end{scope}
\end{tikzpicture}     
  \end{minipage}
  }
  \noindent\resizebox{0.3\textwidth}{!}{
  \begin{minipage}{0.33\textwidth}
   \begin{tikzpicture}
    \begin{scope}
    \node (l2) [state] {$q_2$};
    \node (l0) [state,initial] at (-2,2)   {$q_0$};
    \node (l1) [state,accepting] at (2,2)   {$q_1$};
    
\draw[rounded corners,->, thick] (l0) -- (-1, 3) -- node [above] {$(g \land \neg p)$} (1, 3) -- (l1);
\draw[->, thick] (l1) --  node [above] {$(\neg g \land \neg p)$} (l0);
\draw[rounded corners,->, thick] (l0) --  node [left] {$p$} (-2, 0) -- (l2);
\draw[rounded corners,->, thick] (l1) --  node [left] {$p$} (2, 0) -- (l2);
\path [->,thick]
    (l1) edge [loop above] node [above] {$(g \land \neg p)$} () 
    (l2) edge [loop below] node [below] {$\top$} ()
    (l0) edge [loop above] node [above] {$(\neg g \land \neg p)$}   ();
\end{scope}
\end{tikzpicture}
 \end{minipage}
 }
 \noindent\resizebox{0.3\textwidth}{!}{
\begin{minipage} {0.33\textwidth}
\centering
\begin{tikzpicture}[shorten >=1pt]
\begin{scope}
\node (l0) [state,ellipse,initial above]   {$s_0,q_0$};
\node (l1) [state, ellipse] at (-2,-1.5)   {$s_2,q_0$};
\node (l2) [state, ellipse] at (-4.5,0)   {$s_3,q_0$};
\node (l3) [state,ellipse,accepting] at (-4.5,-3.5)   {$s_0,q_1$};
\node (l4) [state,ellipse,accepting] at (-2,-3.5)   {$s_2,q_1$};
\node (l5) [state, ellipse] at (0.1,-4.3)   {$s_1,q_0$};
\node (l6) [state,ellipse] at (2,-4.3)   {$*,q_2$};
\draw[rounded corners,->, thick] (l3) -- (-4.5,-4.3) -- node [below] {$b,(\lambda(0,b){-}r)$}  (l5);
\path [->, thick,above=.5cm,align=center]
    (l0) edge node [above, sloped] {$b,r$}   (l1)
    (l0) edge  node [above] {$a$}   (l2)
    (l0) edge node [above,sloped] {$b,(\lambda(0,b){-}r)$}   (l5)
    (l2) edge [bend right] node [left] {$c$} (l3)
    (l3) edge  node [right] {$a$} (l2)
    (l3) edge node [above, sloped] {$b,r$} (l1)
    (l1) edge [bend right] node [above] {$f$} (l3)
    (l1) edge  node [left] {$e$} (l4)
    (l4) edge node [above] {$f$} (l3)
    % (l3) edge node [below] {$b,(\lambda(0,b){-}r)$} (l5)
    (l5) edge  node [above] {$d$} (l6)
    (l6) edge [loop above]  node [above] {$\top$} ()
    (l4) edge [loop right] node [right] {$e$} ()
    ;
\end{scope}
\end{tikzpicture}

  \end{minipage}
  }
% }
  \caption{The mars surveillance example where a CTMDP (left) can be in four different states where states $s_0$ has the label $\{\neg \mathtt{p} , \neg \mathtt{g}\}$, state $s_2$ and $s_3$ have label $\{\neg \mathtt{p}, \mathtt{g} \}$, and state $s_1$ have the label $\{\mathtt{p} ,\neg \mathtt{g}\}$. 
The rates of each transition (if not $1$) is written in the figure. 
  A deterministic B{\"u}chi automata for the $\omega$-regular objective $\varphi = (\always \neg \mathtt{p}) \wedge (\always\eventually \mathtt{g})$ (center). Product CTMDP (right) where each zone has two components for denoting the CTMDP and the B{\"u}chi automaton parts. All the zones whose second component is $q_2$ is combined as one. 
%   The accepting zones are coloured in green.
  The exit rate of an action from a zone $(s,q_i)$ in the product CTMDP is same as the exit rate of the action from $s$ in the original CTMDP. 
%   For example, the rate of action $b$ from zone $(0,q_0)$ is $\lambda(0,b)$.
}
 \label{fig:grid-world}
\end{figure*}

% \begin{tikzpicture}[shorten >=1pt]
% \begin{scope}
% \node (l0) [state,ellipse,initial above, fill = safecellcolor]   {$q_0$};
% \node (l1) [state, ellipse, fill = badcellcolor] at (-2.1,0)   {$q_1$};
% \node (l2) [state, ellipse, fill = goodcellcolor] at (2.1,0)   {$q_2$};
% \node (l3) [state,ellipse, fill = goodcellcolor] at (0,-2.1)   {$q_3$};
% % \end{scope}
% % \begin{scope}[every node/.style={scale=0.8}]

% % \draw[rounded corners,->, thick] (l3) -- (-4.5,-4.3) -- node [below] {$b,(\lambda(0,b){-}r)$}  (l5);
% \path [->, thick,above=.5cm,align=center]
%     (l0) edge node [above] {$b$} node [below, font = \scriptsize]{$\lambda(0,b)-r$}   (l1)
%     (l0) edge [bend right]  node [below] {$b,r$}   (l2)
%     (l2) edge [bend right]  node [above] {$f$}   (l0)
%     (l0) edge [bend right]  node [left] {$a$}   (l3)
%     (l3) edge [bend right]  node [right] {$c$}   (l0)
%     (l1) edge [loop left]  node [left] {$d$} ()
%     (l2) edge [loop right]  node [right] {$e$} ()
%     ;
% \end{scope}
% \end{tikzpicture}
 % \begin{tikzpicture}[scale=1.3,action/.style={->,very thick},
 %          cell name/.style={circle,fill=black,text=white,inner sep=1pt}]
 %        \coordinate (t1) at (0,0);
 %        \coordinate (t2) at ($ (t1) + (2,0) $);
 %        \coordinate (t3) at ($ (t1) + (60:2) $);
 %        \coordinate (c0) at ($ (t1) + (2,{2/sqrt(3)}) $);
 %        \foreach \x in {1, 2, 3} {
 %          \pgfmathparse{\x == 1 ? "badcellcolor" : "goodcellcolor"}
 %          \edef\cellcolor{\pgfmathresult}
 %          \path[fill=\cellcolor] (t\x) -- ++(2,0) -- ++(120:2) --cycle;
 %          \coordinate (c\x) at ($ (t\x) + (1,{1/sqrt(3)}) $);
 %        }
 %        % Cell contours.
 %        \draw[thick] (0,0) -- ++(4,0) -- ++(120:4) --cycle;
 %        \draw[fill=safecellcolor] (2,0) -- ++(60:2) -- ++(-2,0) --cycle;
 %        % Cell names.
 %        \node[cell name] at ($ (t1) + (2,3) $) {$3$};
 %        \node[cell name] at ($ (t2) + (0,1.2) $) {$0$};
 %        \node[cell name] at ($ (t1) + (1,1.2) $) {$1$};
 %        \node[cell name] at ($ (t2) + (1,1.2) $) {$2$};
 %        % Actions.
 %        \draw[action] ($ (c0) + (0.1,0.3) $) -- node[pos=0.15,right] {$a$} +(0,0.9);
 %        \draw[action] ($ (c0) - (0,0.2) $) -- node[pos=0.4,right]
 %             {$b$} ++(0,-0.5) -- node[at end,below,black] {$\lambda(0,b){-}r~$} ++(-0.7,0);
 %             \draw[action] ($ (c0) - (0,0.2) $) ++(0,-0.5) --
 %             node[at end,below,black] {$r~~~$} ++(0.7,0);
 %             \draw[action] ($ (c3) - (0.1,0) $) -- node[pos=0.25,left] {$c$} +(0,-0.9);
 %            %  \draw[action] (c1) -- node[pos=0.1,above] {$d$} ++(30:0.9);
 %             \draw[action] ($ (c1) + (215:0.5) $)            arc[radius=0.4cm,start angle=390,end angle=45] node[at start,left] {$~d$};
 %             \draw[action] ($ (c2) + (-45:0.5) $)
 %             arc[radius=0.4cm,start angle=150,end angle=490] node[at start,right] {$e$};
 %             \draw[action] (c2) -- node[pos=0.1,above] {$f$} ++(150:0.9);
 %      \end{tikzpicture}