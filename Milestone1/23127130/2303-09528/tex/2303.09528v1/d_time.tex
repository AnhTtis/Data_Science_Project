\setlength{\tabcolsep}{2.5pt}
\begin{table*}[t]
\small
\centering
\caption{Experimental results of proposed method.}
\begin{tabular}{lcccccccccccccccclccccc}
\hline
                              &  &               &              &  & \multicolumn{7}{c}{Total   Power Consumption {[}mW{]}}                &  & \multicolumn{2}{c}{}          &  &                                                                           &  & \multicolumn{1}{l}{}                                                            \\ \cline{6-12}
                              &  & \multicolumn{2}{c}{Accuracy} &  &
			      \multicolumn{3}{c}{Standard HW} &  &
			      \multicolumn{3}{c}{Optimized HW} &  &
			      \multicolumn{2}{c}{\#Selected} &  &
			      \multirow{2}{*}{\begin{tabular}[c]{@{}c@{}}Max
				Delay\\ Red.\end{tabular}} &  &
				\multirow{2}{*}{\begin{tabular}[c]{@{}c@{}}Voltage
				  Scaling\\ Factor\end{tabular}} &
				  \multirow{2}{*}{\begin{tabular}[c]{@{}c@{}}
				    \\ V\_SHW\end{tabular}} &
				    \multirow{2}{*}{\begin{tabular}[c]{@{}c@{}}\\
				    V\_OHW\end{tabular}}\\ \cline{3-4} \cline{6-8} \cline{10-12} \cline{14-15}
Network-Dataset               &  & Orig.         & Prop.        &  & Orig.   & Prop.     & Red.      &  & Orig.   & Prop.     & Red.      &  & Wei.            & Act.          &  &                                                                           &  &                                                                                 \\ \hline
LeNet-5-CIFAR-10              &  & 80.6\%        & 78.5\%       &  & 375.5   & 149.6   & 60.2\%  &  & 360.7    & 78.3    & 78.3\%  &  & 35            & 210           &  & 40 ps                                                                    &  & 0.71/0.8  & 10.1\%  & 5.4\%                                                                       \\
ResNet-20-CIFAR-10            &  & 91.9\%        & 89.6\%       &  & 718.9   & 361.0   & 49.8\%  &  & 663.9    & 288.3   & 56.6\%  &  & 35            & 210           &  & 40 ps                                                                    &  & 0.71/0.8    & 13.2\%  & 11.5\%                                                                     \\
ResNet-50-CIFAR-100           &  & 79.9\%        & 78.5\%       &  & 708.7   & 293.8   & 58.5\%  &  & 701.8    & 157.1   & 77.6\%  &  & 41            & 223           &  & 30 ps                                                                    &  & 0.73/0.8  & 8.1\%  & 4.2\%                                                                      \\
EfficientNet-B0-Lite-ImageNet &  & 73.8\%        & 69.7\%       &  & 21.2    & 19.3    & 9.0\%   &  & 2.4      & 1.9     & 20.8\%  &  & 50            & 236           &  & 20 ps                                                                    &  & 0.75/0.8  & 6.4\%  & 6.5\%                                                                      \\ \hline
\end{tabular}
\label{tab:results}
\end{table*}





We study the expectation semantics of $\omega$-regular objective and show that the problem can be reduced to maximising the expected average reward problem in CTMDPs.
Using Theorem~\ref{corollary:1}, this reduces to maximising expected discounted reward for a large discount factor.
We then describe the corresponding reward machine to maximize the \emph{expected satisfaction time} in the good states.

\paragraph{Reduction to Average Reward.} For an $\omega$-regular objective $\phi$, let $\oautomata$ be a \textbf{GFM} corresponding to $\phi$ with a set $F$ of B{\"u}chi accepting states. Let $\Mdp$ be a CTMDP and $\Mdp \times \oautomata$ be the product CTMDP of $\Mdp$ and $\oautomata$.
For a state $s$ in $\Mdp \times \oautomata$, we define the \emph{expected satisfaction time} of a schedule $\sigma$ from starting state $s$ as:
% For a given finite run $r_f = (s_1,a_1,t_1,s_2,a_2,t_2...a_{n-1},t_{n-1},s_n)$, we define the residence time of $s$ in $r_f$ as 
% $\dtime(s)(r_f) = \frac{\sum_{s_i = s,i=1}^{n-1} t_i}{\sum_{j=1}^{n-1} t_j}$.
% Similarly for an infinite run $r_{inf}=(s_1,t_1,a_1,s_2,t_2,a_2...)$, we define the residence time of $s$ in $r_{inf}$ as $\dtime(s)(r_{inf}) = \liminf_{n \rightarrow \infty} \frac{\sum_{i=1,s_i = s}^{n} t_i}{\sum_{j=1}^{n}t_j}$. Note that we consider $\liminf$ since the limit may not exist in general.
 %\todo{Should we use equation environment}
\[
\ESat^{\Mdp {\times} \oautomata}_{\sigma}(s) {=} \mathbb{E}^{\Mdp \times \oautomata}_{\sigma}(s) \set{\liminf_{n \rightarrow \infty} \frac{\sum_{i=1}^{n} [X_i \in F^{\times}] {\cdot} D_i}{T_n}}.
\]
It gives the long-run expected average time spent in the accepting states. 
The reward rate function $r' : S \rightarrow \{0,1\}$ for $~{\Mdp \times \oautomata}$ is defined such that $r'(s) = 1$ if $s \in F^{\times}$, and $r'(s) = 0$, otherwise. Thus the reward is $r'(s)\cdot t = t$ for $s \in F^{\times}$ if $t$ time is spent in $s$.

% The following lemma gives an equivalence between the residence time and the average reward obtained for every run in $(\Mdp \times \oautomata)^{\sigma}$ for every schedule $\sigma$.
% \begin{lemma} \label{lemma:6.1}
% For a product CTMDP $\Mdp \times \oautomata$ with $\oautomata$ being a \textbf{GFM} for an $\omega$-regular objective, and a set $T$ of target states, for a schedule $\sigma$, for all runs in $(\Mdp \times \oautomata)^{\sigma}$, the average reward obtained by the reward function $r'$ is equal to the residence time in $(\Mdp \times \oautomata)^{\sigma}$.
% \end{lemma}

% For a schedule $\sigma \in \Sigma_{\Mdp \times \oautomata}$, we denote the expected residence time in $T$ under $\sigma$ by $\edtime(T)$, and the expected average reward obtained under $\sigma$ for the reward function $r'$ by $\eavgreward(T)$.
% From Lemma~\ref{lemma:6.1} it follows that for any schedule $\sigma$, the expected average reward obtained by $r'$ and the expected residence time in $T$ are equivalent.
The following lemma (proof in Appendix~\ref{app:expt}) gives an equivalence between the expected satisfaction time and expected average reward obtained in $\pmdp$.
\begin{lemma} \label{lemma:6.2}
For a product CTMDP $\Mdp \times \oautomata$ where $\oautomata$ is a \textbf{GFM} for an $\omega$-regular objective and for a schedule $\sigma$, the expected average reward obtained w.r.t. the reward function $r'$ is equal to the expected satisfaction time in $~{(\pmdp)}$ and there exists a pure schedule that maximizes this.
\end{lemma}
% For the product CTMDP $\Mdp \times \oautomata$ with accepting states $T$ corresponding to the B{\"u}chi acceptance, 
% Our objective is to find an optimal schedule $\sigma^*$ that maximises the expected residence time in $T$, i.e, $\edstime(T) = \sup_{\sigma \in \Sigma_{\Mdp \times \oautomata}}\edtime(T)$. We show that such a schedule exists in $\pmdp$, and there exists an optimal schedule that is pure.
% From \cite{puterman2014markov}, we know that there exist optimal pure schedules for maximising the expected average reward and from this result we get the following lemma.

% \begin{lemma}\label{lemma:6.3}
% For a product CTMDP $\Mdp \times \oautomata$ there exists a pure schedule that maximises the expected satisfaction time.
% \end{lemma}
Using the results from Lemma~\ref{lemma:6.2} and Theorem~\ref{corollary:1}, we can conclude that a schedule maximising the discounted reward objective for a large discount factor in $\pmdp$ with reward function $r'$ also maximizes the expected satisfaction time.
% Lemma \ref{lemma:6.3} follows from Lemma~\ref{lemma:6.2} since it is known that an optimal pure schedule exists for maximising $\eavgreward(T)$ over all $\sigma \in \Sigma_{\Mdp}$~\cite{puterman2014markov}.
% we can conclude that there exists an optimal pure schedule $\sigma^*$ for the expectation semantics. 

\paragraph{Algorithm for Expectation Semantics.}
Here, we provide a brief description of the algorithm.  
% We show the procedure to obtain an optimal schedule $\sigma$ for expectation semantics in Algorithm \ref{algo:expt}. 
% Initially, the Q-function for reinforcement learning is initialised to zeroes. 
% Line 1 initialises the Q-function for reinforcement learning.
The Q-function is defined on the states of the product CTMDP, i.e, $\mathcal{Q}_f: (\states \times Q) \times \actions \rightarrow \mathbb{R}$ where $\states$ is the set of states of the CTMDP $\Mdp$ and $Q$ is the set of states of the \textbf{GFM} $A$.
Initially, the state space is unknown to the agent and the agent will have information only on the initial state. 
States seen are stored in a Q-table where the Q-value of the state is stored. 
The initial value of a state in the Q-table is zero.
% Variable $i$ keeps track of the number of episodes completed and variable $j$ keeps track of the length of an episode.
The number of episodes to be conducted and the length of each episode are defined by the user, let these be denoted by $k$ and $eplen$ respectively.
In each episode, the RL agent picks an action from its current state in the CTMDP according to the RL schedule and observes the next state and the time spent in the current state. 
It also picks the transition in the GFM based on the observed state in the CTMDP. 
For each transition taken, the reward obtained is based on the reward function $r'$.
% \todo{Where is $r'$ defined?}
% The variables $s$ and $q$ represent the current state of $\Mdp$ and $A$ respectively and are initialised to the respective initial states. 
% The action $a$ from the current state of CTMDP and the transition $t$ from the state of GFM are picked according to the RL policy used (eg. $\epsilon$-greedy). 
% The RL agent observes the next state $(s',q')$ and the time spent $\tau$ in the current state. 
% The reward function $rew$ is defined based on $r'$ defined previously. 
% For a given state $(s,q)$ and action $a$, if $\tau$ is the observed time spent in $(s,q)$ then,
% $$
%  rew((s,q),a,\tau) = \begin{cases}
%  \tau & \text{if ($s,q$) is an accepting state}\\
%  0 & \text{otherwise}
%  \end{cases}
%  $$
% Variable $r$ stores the reward obtained in each iteration of the episode. 
The Q-function is updated according to the Q-learning rule defined in Section~\ref{prelims}. 
An episode ends when the length of the episode reaches $eplen$. 
After the completion of $k$ episodes, we obtain a schedule $\sigma$ by choosing the action that gives the highest Q-value from each state.
The schedule learnt by the Q-learning algorithm converges to an optimal schedule as the number of training episodes tend to infinity.
We provide a pseudocode of the algorithm 
% for the expectation semantics 
in Appendix~\ref{algo}. 
% \begin{algorithm}
% \caption{Algorithm for expectation semantics}\label{algo:expt}
% \hspace*{\algorithmicindent} \textbf{Input:}  \text{Initial state $s_{init}$, SLDBW $A$, discount factor $\gamma$,}\\
% \hspace*{\algorithmicindent} \hspace{11mm}\text{reward function $R'$, number of episodes $k$,} \\
% \hspace*{\algorithmicindent}\hspace{11mm} \text{learning rate $\alpha$, episode length $eplen$}\\
% \hspace*{\algorithmicindent} \textbf{Output:}  \text{Optimal strategy $\sigma$}
% \begin{algorithmic}[1]
% \STATE Initialise $Q_f$ to all zeroes
% \STATE $i \leftarrow 0$
% \WHILE{$i < k$}
% \STATE $s \leftarrow \initstate$
% \STATE $q \leftarrow q_0$
% \STATE $r \leftarrow 0$
% \STATE $j \leftarrow 0$
% \WHILE{$j < eplen$}
% \STATE Choose action $a$ according to the RL policy
% \STATE Take action $a$, observe next state $s'$, and time $\tau$
% \STATE Choose transition $t$ in SLDBW according to the RL policy
% \STATE Take transition $t$ in SLDBW, observe next state $q'$
% \STATE $r \leftarrow R'(s,q,a,\tau)$
% \STATE $Q_f(s,q,a) \leftarrow Q_f(s,q,a) +\alpha \bigl[ r + e^{-\gamma\tau} \max_{a' \in \actions} Q_f(s',q',a')- Q_f(s,q,a) \bigr]$
% \STATE $s \leftarrow s'$
% \STATE $q \leftarrow q'$
% \ENDWHILE
% \STATE $i \leftarrow i+1$
% \ENDWHILE
% \end{algorithmic}
% \end{algorithm}
% \subsection{Blackwell Optimality In CTMDP}
% % \subsection{Blackwell Optimality In CTMDP}
% \track{Though both of our semantics use different reward machines, for learning schedules for both objectives, we reduce the problem to maximising the expected average reward.
% We use off-the-shelf RL algorithms for learning the schedulers in both settings.}
% Standard RL algorithms try to optimise discounted payoff objective while 
% % as shown above, for the expectation semantics, 
% we need to optimise the expected average reward. Blackwell optimality allows us to use a schedule that optimises the expected discounted payoff with a high discount factor and such a schedule also optimises the expected average payoff.

% \vspace{0.5em}\noindent\textbf{Existence of Blackwell Optimal Schedules in CTMDP.} 
% We give a simple uniformization based proof on the existence of a Blackwell optimal pure schedule in a CTMDP.
In this section, we provide a uniformization based proof of Theorem~\ref{corollary:1}.
Consider a CTMDP $\Mdp$, a pure schedule $\sigma$, and a continuous-time discounting with parameter $\dfactor > 0$. 
% The one-step expected reward obtained from taking action $a \in \actions{}(s)$ from state $s \in \states$ is given by $\rho(s,a) = rew(s,a) + \frac{rew(s)}{\dfactor + \lambda(s,a)}$ (~$\!\!$\cite{puterman2014markov}~Eq 11.5.3). 
We define a function $\ctmdprate : \states \times \actions \rightarrow [0,1) $ where $\ctmdprate(s,a) = \frac{\lambda(s,a)}{\lambda(s,a) + \dfactor}$ where $\dfactor > 0$.
We call $\ctmdprate(s,a)$ the \emph{discount rate} of the state-action pair $(s,a)$ in $\Mdp$.
So, the expected discounted reward (also known as value of $s$) $\discobjective^{\Mdp[\sigma]} (\alpha)(s)$ is given by
% For a pure schedule $\sigma$, the value of a state $s$, denoted by $\valuesigma_{\ctmdprate(s,\sigma(s))}$,
% \track{CHANGE NOTATION} 
% is given by 
\begin{equation}\label{eq:7.1}
\begin{split}
     \rho(s,a_{\sigma}) + 
     \ctmdprate(s,a_{\sigma})
     \sum_{s' \in s} \pmtrx(s,a_\sigma,s') 
     \discobjective_{\sigma}^{\Mdp}(\alpha)(s')
\end{split}
\end{equation}
Consider a DTMDP $\mathcal{N}$, a schedule $\sigma$, and a discount rate $0\leq \drate < 1$. Let $\valuesigma_{\drate}(\mathcal{N},s)$ denote the total discounted value from state $s$ in $\mathcal{N}$ under schedule $\sigma$.

A pure schedule $\bschedule$ is \emph{Blackwell optimal} in $\mathcal{N}$ if there exists a threshold discount rate $0 \leq \thrate < 1$ such that for any discount rate $\thrate \leq \drate < 1$, we have $\valuestar_{\drate}(\mathcal{N},s) \geq \valuesigma_{\drate}(\mathcal{N},s)$ for all $\sigma \in \Sigma_{\mathcal{N}}$. It is known that a Blackwell optimal schedule maximises both discounted and average reward objectives in DTMDPs.
% From \cite{puterman2014markov}~Thm 10.1.4, we have,
% \begin{theorem}\label{puterman1}
From \cite{puterman2014markov}~(Thm 10.1.4), we have that for every DTMDP $\mathcal{N}$, there exists a Blackwell optimal \emph{pure} schedule $\bschedule$, and $\bschedule$ also maximises the average reward in $\mathcal{N}$.
% \end{theorem}
Now, given a CTMDP $\Mdp$, let $C$ be a constant such that $C \geq \lambda(s,a)$ for all state-action pairs in $\Mdp$. 
Let $\uMdp$ be the uniformized CTMDP of $\Mdp$ with constant exit rate $C$, and let $\upmtrx$ be the probability matrix of $\uMdp$. As the exit rate $\lambda(s,a) = C$ for all $s\in \states$ and $a \in \actions{}(s)$, we have that $\ctmdprate(s,a) = \frac{C}{C+\dfactor}$ for all state-action pairs. We denote this discount rate by $\udrate$. 
The value of a state $s$ under a schedule $\sigma$ in $\uMdp$, 
% \track{under schedule $\sigma$}
denoted $\discobjective^{\uMdp^{\sigma}}(\alpha)(s)$
% $\valuesigma_{\udrate}(\uMdp,s)$ 
is given by, 
\begin{equation}\label{eq:7.2}
     \Bar{\rho}(s,a_\sigma) + \udrate \sum_{s' \in s} \upmtrx(s,a_\sigma,s') \discobjective^{\uMdp^{\sigma}}(\alpha)(s')
\end{equation}
where $\Bar{\rho}(s,a) = \rho(s,a)\cdot \frac{\dfactor + \lambda(s,a)}{\dfactor + C} $. 
We extend the above result of existence of Blackwell optimal schedules in DTMDPs to uniform CTMDPs.
\begin{lemma} \label{lemma:1}
For a uniform CTMDP $\uMdp$, there exists a Blackwell optimal schedule $\bschedule$. Further, $\bschedule$ also maximises the expected average reward in $\uMdp$.
\end{lemma}
\begin{proof}
Consider a DTMDP $\mathcal{N}$ with the same set of states as that of $\uMdp$, one step reward function $\Bar{r}$ and probability matrix $P_{\mathcal{N}} = \upmtrx$. For a pure schedule $\sigma$ and a discount rate $0 \leq \drate < 1$, the value of a state $s$ in $\mathcal{N}$ is 
\begin{equation}\label{eq:7.3}
    \valuesigma_{\drate}(\mathcal{N},s) = \Bar{r}(s,\sigma(s)) + \drate \sum_{s' \in s} \upmtrx(s,\sigma(s),s') \valuesigma_{\drate}(\mathcal{N},s')
\end{equation}
We observe that equation \ref{eq:7.3} is identical to equation \ref{eq:7.2} when $\udrate = \drate$. Therefore, the set of equations defining the values of states in $\mathcal{N}$ and $\uMdp$ are identical. Let this set be denoted by $E^\sigma$. From \cite{puterman2014markov}~Thm 6.1.1, we know that for each stationary schedule $\sigma$, there exists a unique solution for $E^\sigma$. The set of pure schedules in $\uMdp$ and $\mathcal{N}$ are equal as the set $S$ of states and the set $\av$ of available actions from each state are the same for both.\\*
Therefore, for a pure schedule $\sigma$, and discount rates $0 \leq \drate = \udrate < 1$, we have
\begin{equation}\label{eq:7.4}
    \valuesigma_{\udrate}(\uMdp,s) = \valuesigma_{\drate}(\mathcal{N},s) \text{\: for $\udrate = \drate$, for all states $s$}
\end{equation}
From Theorem~10.1.4 in \cite{puterman2014markov}, we know that in $\mathcal{N}$, there exist a Blackwell optimal pure schedule $\sigma^*$, and a threshold discount rate $\thrate$ such that 
\begin{align}\label{eq:7.5}
 \valuestar_{\drate}(\mathcal{N},s) \geq \valuesigma_{\drate}(\mathcal{N},s)   
\end{align}
for all $\sigma \in \Sigma_{N}$ and $\thrate \leq \drate < 1$.\\
From equations \ref{eq:7.4} and \ref{eq:7.5} we can conclude that 
\begin{equation}\label{eq:7.60}
    \valuestar_{\udrate}(\uMdp,s) \geq \valuesigma_{\udrate}(\uMdp,s)   
\end{equation}
for all $\sigma \in \Sigma_{\uMdp}^{pure}$ and $\thrate \leq \udrate < 1$.\\
From \cite{puterman2014markov}~Thm 11.5.2(d), we know that there exists an optimal pure schedule maximising the discounted reward in a CTMDP. Therefore,
\begin{equation}\label{eq:7.}
    \valuestar_{\udrate}(\uMdp,s) \geq \valuesigma_{\udrate}(\uMdp,s)   
\end{equation}
for all $\sigma \in \Sigma_{\uMdp}$ and $\thrate \leq \udrate < 1$.\\
A similar argument can be made to show that $\sigma^*$ also maximises the expected average reward in $\uMdp$.
\end{proof}

The above lemma proves the existence of a Blackwell optimal schedule in uniform CTMDPs. 
We further extend this result to general CTMDPs which is the main result of this section.

\begin{lemma}\label{lemma:2}
If $\sigma^*$ is a Blackwell optimal pure schedule in $\uMdp$, then it is also Blackwell optimal in $\Mdp$.
\end{lemma}
\begin{proof}
Since $\bstrategy$ is a Blackwell optimal pure schedule in $\uMdp$, there exists a threshold discount rate $\uthrate$ such that for all $\uthrate \leq \udrate < 1$, we have that 
\begin{equation}\label{eq:7.7}
    \valuestar_{\udrate}(\uMdp,s) \geq \valuesigma_{\udrate}(\uMdp,s)   
\end{equation}
for all $\sigma \in \Sigma_{\uMdp}$.\\
The set of pure schedules in $\Mdp$ and $\uMdp$ are the same. From \cite{puterman2014markov}~Thm 11.5.2(d), we know that there exists an optimal pure schedule maximising the discounted reward in $\Mdp$. \cite{puterman2014markov}~Prop 11.5.1, states that for every pure schedule $\sigma$ and a state $s$, we have that  
\begin{equation}\label{eq:7.8}
    \valuesigma_{\ctmdprate(s,\sigma(s))}(\Mdp,s) = \valuesigma_{\udrate}(\uMdp,s)   
\end{equation}
If $\uthrate = \frac{C}{C+\dfactor}$ is the threshold discount rate in $\uMdp$, then the corresponding threshold discount rate for a state $s$ in $\Mdp$ is given by 
$\ctmdpthrate(s,a) = \frac{\lambda(s,a)}{\lambda(s,a)+\dfactor_{o}}$.\\
From equations \ref{eq:7.7} and \ref{eq:7.8}, we can conclude that for each state $s$ in $\Mdp$, there exist a pure schedule $\bstrategy$, and a threshold discount rate $\ctmdpthrate(s,\bstrategy(s)) $ such that for all $\ctmdpthrate(s,\bstrategy(s)) \leq \ctmdprate(s,\bstrategy(s)) < 1$, we have that
\begin{equation}\label{eq:7.9}
    \valuestar_{\ctmdprate(s,\bstrategy(s))}(\Mdp,s) \geq \valuesigma_{\ctmdprate(s,\sigma(s))}(\Mdp,s)   
\end{equation}
for all $\sigma \in \Sigma_{\Mdp}$.
As the set of states is finite in $\Mdp$, the threshold discount rate for $\Mdp$ is given by
    $\ctmdpthrate^{\Mdp} = \max_{(s,a) \in \states \times \actions} \ctmdpthrate(s,a)$.
\end{proof}
Thus, any Blackwell optimal schedule $\bstrategy$ in $\uMdp$ is also Blackwell optimal in $\Mdp$. The following lemmas show that $\sigma^*$ also maximises the expected average reward.
\begin{lemma}\label{lemma:3}
An optimal schedule maximising the expected average reward in $\uMdp$ also maximises the expected average reward in $\Mdp$.
\end{lemma}
\begin{proof}
For a pure schedule $\sigma$ , the expected average reward in $\Mdp$ is denoted by $\avgrew^{\sigma}(\Mdp)$. From \cite{puterman2014markov}~Chap 11.5.3, we observe that for a pure schedule $\sigma$, 
\begin{equation}\label{eq:7.10}
    \avgrew^{\sigma}(\Mdp) = \avgrew^{\sigma}(\uMdp)\cdot C 
\end{equation}
Let $\sigma'$ be a pure schedule maximising the expected average reward in $\uMdp$, i.e, 
\begin{equation}\label{eq:7.11}
 \avgrew^{\sigma'}(\uMdp) = \sup_{\sigma \in \Sigma_{\uMdp}} \avgrew^{\sigma}(\uMdp)
\end{equation}
The set of pure schedules in $\Mdp$ and $\uMdp$ are equal and we know that there exists a pure schedule maximising the average reward in a CTMDP (\cite{puterman2014markov}~Thm 11.4.6(d)). Therefore, from equations \ref{eq:7.10} and \ref{eq:7.11} we can conclude that, 
\begin{equation}\label{eq:7.12}
    \avgrew^{\sigma'}(\Mdp) = \sup_{\sigma \in \Sigma_{\Mdp}} \avgrew^{\sigma}(\Mdp)
\end{equation}
Therefore, $\sigma'$ is an optimal schedule maximising the average reward in $\Mdp$.
\end{proof}
\begin{lemma}\label{lemma:4}
A Blackwell optimal schedule $\bstrategy$ in $\Mdp$ also maximises the average reward in $\Mdp$.
\end{lemma}
\begin{proof}
From Lemma \ref{lemma:1}, we know that $\bstrategy$ is an optimal schedule maximising the expected average reward in $\uMdp$. Lemma \ref{lemma:3} shows that if $\bstrategy$ is an optimal schedule maximising the expected average reward in $\uMdp$ then $\bstrategy$ also maximises the expected average reward in $\Mdp$. Therefore, we can conclude that $\bstrategy$ is an optimal schedule maximising the expected average reward in $\Mdp$.
\end{proof}
Lemma~\ref{lemma:2} and Lemma~\ref{lemma:4} gives us the following.
\paragraph{Theorem~\ref{corollary:1}.}For a CTMDP $\Mdp$, there exists a Blackwell optimal pure schedule $\bschedule$ and a threshold $0 \leq \ctmdpthrate^{\Mdp} < 1$ such that :
\begin{inparaenum}[(1).]
    \item For any discount-rate function $\ctmdprate$ where $\ctmdprate(s,a) \geq \ctmdpthrate^{\Mdp}$ for all valid state-action pairs $(s,a)$, the schedule $\bschedule$ is an optimal schedule maximising the expected discounted reward.
    \item The schedule $\bschedule$ also maximises the expected average reward.
\end{inparaenum}
% \paragraph{Theorem ~\ref{corollary:1}}
% For a CTMDP $\Mdp$, there exists a Blackwell optimal pure schedule $\bschedule$ and a threshold $0 \leq \ctmdpthrate^{\Mdp} < 1$ such that :
% \begin{inparaenum}[(1).]
%     \item For any discount-rate function $\ctmdprate$ where $\ctmdprate(s,a) \geq \ctmdpthrate^{\Mdp}$ for all valid state-action pairs $(s,a)$, the schedule $\bschedule$ is an optimal schedule maximising the expected discounted reward.
%     \item The schedule $\bschedule$ also maximises the expected average reward.
% \end{inparaenum}

% \begin{proof}[Proof sketch]
% % \begin{lemma}\label{lemma:2}
% Note that the set of pure schedules are the same in $\Mdp$ and $\uMdp$.
% First we prove that if $\bschedule$ is a Blackwell optimal pure schedule in $\uMdp$, then it is also Blackwell optimal in $\Mdp$. 
% This is done using the fact that the value of each state in $\uMdp$ and $\Mdp$ are equal (Prop 11.5.1 in \cite{puterman2014markov}) i.e, for every pure schedule $\sigma$ and a state $s$, we have that
% \[
% \discobjective^{\ctmc} (\alpha)(s) = \discobjective^{\uMdp^{\sigma}}(\alpha)(s)
% \] 
% % \track{ELABORATE}
% From this equation, we can get the threshold discount rate satisfying the Blackwell optimality condition for $\Mdp$.
% % \end{lemma}
% Thus, we conclude that any Blackwell optimal schedule $\sigma^*$ in $\uMdp$ is also Blackwell optimal in $\Mdp$. 
% % The following lemmas show that $\sigma^*$ also maximises the expected average reward in $\Mdp$.

% Next we show that 
% % \begin{lemma}\label{lemma:3}
% an optimal schedule maximising the expected average reward in $\uMdp$ also maximises the expected average reward in $\Mdp$. 
% For a given pure schedule $\sigma$, we denote the expected average reward in $\Mdp$ and $\uMdp$ from a state $s$ by $\avgobjective^{\Mdp^{\sigma}}(s)$ and $\avgobjective^{\uMdp^{\sigma}}(s)$
% % \track{WE ALREADY HAVE NOTATIONS FOR EXPECTED AVERAGE REWARD} 
% respectively.
% From Section~11.5.3 in \cite{puterman2014markov}, we get the following result:
% \[
% \avgobjective^{\Mdp^{\sigma}}(s) = \avgobjective^{\uMdp^{\sigma}}(s)\cdot C \] 
% % This is due to the fact that 
% % the expected average reward obtained by a schedule in $\uMdp$ is directly proportional to the expected average obtained by the same schedule in $\Mdp$, . \track{ELABORATE}
% From this equation, we have that if a pure schedule maximises the expected average reward in $\uMdp$, it also maximises the expected average reward in $\Mdp$.
% % Therefore, optimal pure schedules maximising the expected average reward in $\Mdp$ and $\uMdp$ are the same. 
% % Now we prove the following lemma.

% In Lemma~\ref{lemma:1}, we have shown the existence of a pure Blackwell optimal schedule $\bschedule$ in $\uMdp$ and it also maximises the expected average reward. 
% From the above results we obtain that $\bschedule$ also is Blackwell optimal and maximises the expected average reward in $\Mdp$.
% % Finally, using Lemma~\ref{lemma:1}, we show that a Blackwell optimal schedule $\bschedule$ in $\Mdp$ also maximises the average reward in $\Mdp$. \track{ELABORATE}
% % We can get this result directly from Lemma~\ref{lemma:1}.
% % \end{lemma}
% % Lemma \ref{lemma:2} and Lemma \ref{lemma:4} give us the following.
% % \begin{corollary}\label{corollary:1}
% % For a CTMDP $\Mdp$, there exists a Blackwell optimal pure schedule $\bschedule$ and a threshold $0 \leq \ctmdpthrate^{\Mdp} < 1$ such that :
% % \begin{inparaenum}[(1).]
% %     \item For any discount-rate function $\ctmdprate$ where $\ctmdprate(s,a) \geq \ctmdpthrate^{\Mdp}$ for all valid state-action pairs $(s,a)$, the schedule $\bschedule$ is an optimal schedule maximising the expected discounted reward.
% %     \item The schedule $\bschedule$ also maximises the expected average reward.
% % \end{inparaenum}
% % \end{corollary}
% \end{proof}

% \track{ In the following sections, we design reward functions for specific objectives such that a schedule maximising the expected average reward will also maximise the given objective. As standard RL algorithms optimise discounted reward objectives, we use the result from Theorem~\ref{corollary:1} to conclude that for a large enough discount factor, the RL algoirithm will synthesise the optimal schedule. }

\begin{comment}
\vspace{0.5em}\noindent\textbf{Algorithm for Expectation Semantics.} We show the procedure to obtain an optimal schedule $\sigma$ for expectation semantics in Algorithm \ref{algo:expt}.  Line 1 initialises the Q-function for reinforcement learning. Here, the Q-function is defined on the states of the product CTMDP i.e, $Q_f: (\states \times Q) \times \actions \rightarrow \mathbb{R}$ where $\states$ is the set of states of the CTMDP $\Mdp$ and $Q$ is the set of states of the \textbf{GFM} $A$.
% Variable $i$ keeps track of the number of episodes completed and variable $j$ keeps track of the length of an episode.
The number of episodes to be conducted and the length of the episode is defined by the user and is stored in $k$ and $eplen$ respectively.
The variables $s$ and $q$ represents the current state of $\Mdp$ and $A$ respectively and is initialised to their initial states. The action $a$ from the current state of CTMDP and the transition $t$ from the state of GFM is picked according to the RL policy used (eg. $\epsilon$-greedy). We observe the next state $(s',q')$ and the time spent $\tau$ in the current state. The reward function $R'$ is defined based on $r'$ defined previously. For a given state $(s,q)$ and action $a$, if $\tau$ is the observed time spent in $(s,q)$ then,
$$
 R'((s,q),a,\tau) = \begin{cases}
 \tau & \text{if ($s,q$) is an accepting state}\\
 0 & \text{otherwise}
 \end{cases}
 $$
Variable $r$ stores the reward obtained in each iteration of the episode. The Q-function is updated according to the Q-learning rule defined in section \ref{prelims}. An episode ends when the length of the episode reaches $eplen$. After the completion of $k$ episodes, we obtain the optimal schedule $\sigma$ by choosing the action that gives the highest Q-value from each state.

\begin{algorithm}[t]
\caption{Algorithm for expectation semantics}\label{algo:expt}
\hspace*{\algorithmicindent} \textbf{Input:}  \text{Initial state $s_{init}$, GFM $A$, discount factor $\gamma$,}\\
\hspace*{\algorithmicindent} \hspace{11mm}\text{reward function $R'$, number of episodes $k$,} \\
\hspace*{\algorithmicindent}\hspace{11mm} \text{learning rate $\alpha$, episode length $eplen$}\\
\hspace*{\algorithmicindent} \textbf{Output:}  \text{Optimal schedule $\sigma$}
\begin{algorithmic}[1]
\STATE Initialise $Q_f$ to all zeroes
% \STATE $i \leftarrow 0$
\FOR{$k$ episodes}
\STATE Initialise $s$ and $q$ to $s_0$ and $q_0$ respectively
% \STATE $q \leftarrow q_0$
% \STATE $r \leftarrow 0$
% \STATE $j \leftarrow 0$
\FOR{each episode}
\STATE Choose action $a$ using policy derived from $Q$ 
\STATE Take action $a$, observe next state $s'$, and time $\tau$
\STATE Choose non-deterministic transition $t$ in $A$ using the derived policy
\STATE Take transition $t$ in $A$, observe next state $q'$
\STATE $r \leftarrow R'(s,q,a,\tau)$
\STATE $V(s',q') \leftarrow \max\limits_{a' \in \actions} Q_f(s',q',a')$
\STATE $Q_f(s,q,a) {\leftarrow} (1-\alpha) Q_f(s,q,a)  {+}\alpha \bigl(r {+} e^{-\gamma\tau} V(s',q') \bigr)$
\STATE $s \leftarrow s'$; $q \leftarrow q'$
% \STATE $q \leftarrow q'$
% \STATE $j \leftarrow j+1$
\ENDFOR
% \STATE $i \leftarrow i+1$
\ENDFOR
\FOR{Each state (s,q)}
\STATE $\sigma(s,q) = \max_{a \in \actions}Q(s,q,a)$
\ENDFOR
\end{algorithmic}
\end{algorithm}
\end{comment}
% \setlength{\tabcolsep}{2.5pt}
\begin{table*}[t]
\small
\centering
\caption{Experimental results of proposed method.}
\begin{tabular}{lcccccccccccccccclccccc}
\hline
                              &  &               &              &  & \multicolumn{7}{c}{Total   Power Consumption {[}mW{]}}                &  & \multicolumn{2}{c}{}          &  &                                                                           &  & \multicolumn{1}{l}{}                                                            \\ \cline{6-12}
                              &  & \multicolumn{2}{c}{Accuracy} &  &
			      \multicolumn{3}{c}{Standard HW} &  &
			      \multicolumn{3}{c}{Optimized HW} &  &
			      \multicolumn{2}{c}{\#Selected} &  &
			      \multirow{2}{*}{\begin{tabular}[c]{@{}c@{}}Max
				Delay\\ Red.\end{tabular}} &  &
				\multirow{2}{*}{\begin{tabular}[c]{@{}c@{}}Voltage
				  Scaling\\ Factor\end{tabular}} &
				  \multirow{2}{*}{\begin{tabular}[c]{@{}c@{}}
				    \\ V\_SHW\end{tabular}} &
				    \multirow{2}{*}{\begin{tabular}[c]{@{}c@{}}\\
				    V\_OHW\end{tabular}}\\ \cline{3-4} \cline{6-8} \cline{10-12} \cline{14-15}
Network-Dataset               &  & Orig.         & Prop.        &  & Orig.   & Prop.     & Red.      &  & Orig.   & Prop.     & Red.      &  & Wei.            & Act.          &  &                                                                           &  &                                                                                 \\ \hline
LeNet-5-CIFAR-10              &  & 80.6\%        & 78.5\%       &  & 375.5   & 149.6   & 60.2\%  &  & 360.7    & 78.3    & 78.3\%  &  & 35            & 210           &  & 40 ps                                                                    &  & 0.71/0.8  & 10.1\%  & 5.4\%                                                                       \\
ResNet-20-CIFAR-10            &  & 91.9\%        & 89.6\%       &  & 718.9   & 361.0   & 49.8\%  &  & 663.9    & 288.3   & 56.6\%  &  & 35            & 210           &  & 40 ps                                                                    &  & 0.71/0.8    & 13.2\%  & 11.5\%                                                                     \\
ResNet-50-CIFAR-100           &  & 79.9\%        & 78.5\%       &  & 708.7   & 293.8   & 58.5\%  &  & 701.8    & 157.1   & 77.6\%  &  & 41            & 223           &  & 30 ps                                                                    &  & 0.73/0.8  & 8.1\%  & 4.2\%                                                                      \\
EfficientNet-B0-Lite-ImageNet &  & 73.8\%        & 69.7\%       &  & 21.2    & 19.3    & 9.0\%   &  & 2.4      & 1.9     & 20.8\%  &  & 50            & 236           &  & 20 ps                                                                    &  & 0.75/0.8  & 6.4\%  & 6.5\%                                                                      \\ \hline
\end{tabular}
\label{tab:results}
\end{table*}





% % \subsection{Limiting Distribution of a CTMC}
% Limiting distribution of a CTMC gives the probability of being in a state in the `long run' or when time is infinity at the limit. Consider a CTMC $H$ where $\states$ is the set of states and $\initstate$ is the initial state. We define a random variable $X(t)$ for $t \in \realpositives$ to be the  state at which the CTMC $H$ is in at time $t$. We define the limiting distribution of $H$ as $\pi^{H} = [\pi_{s_1}^{H}, \pi_{s_2}^H... \pi_{s_n}^H]$ where $s_1,s_2...s_n \in \states$ and for each state $s_i \in \states$
% \begin{center}
%     $\pi^H_{s_i} = \lim_{t \rightarrow \infty} P(X(t) = s_i | X(0) = \initstate)$
% \end{center}
% Also $\sum_{s_i \in S} \pi^{H}_{s_i} = 1$. We omit $X(0) = \initstate$ when the initial state is clear from the context. For a set $T$ of states, we define the limiting probability of $T$ as $\pi_{T}^{H} = \sum_{s \in T} \pi^{H}_{s}$. For a given CTMDP $\Mdp$, a pure strategy $\sigma$, and set $T$ of states, we show the relationship between the expected average reward with respect to the reward function $r'$ and the limiting probability of $T$ in the CTMC $\Mdp^{[\sigma]}$.

% \begin{lemma}\label{csl1}
% For a given CTMDP $\Mdp$, a pure strategy $\sigma$, and a set $T$ of states,
% \[
% Ear^{\Mdp^{[\sigma]}}(T) = \pi_{T}^{\Mdp^{[\sigma]}}
% \]
% where $\pi_{T}^{\Mdp^{[\sigma]}}$ is the limiting probability of $T$.
% \end{lemma}
% \begin{proof}
%  We define the expected average reward from a CTMC $\ctmc$ as,
% \begin{equation*}
%     \begin{split}
%  Ear^{\ctmc} = \lim_{n \rightarrow \infty} \frac{\mathbb{E}_{\sigma}^{\Mdp} \{\sum^{n}_{i=0} r(s_i,a_i)\}}{\mathbb{E}_{\sigma}^{\Mdp} \{\sum_{i=0}^{n} \tau_i\}}
%  \end{split}
% \end{equation*}
% From the definition of reward function r', it gives a reward of $1$ for one time unit spent in a state in $T$ and $0$ otherwise. Therefore, the expected average reward becomes,
% \[
% Ear^{\ctmc} = \lim_{n \rightarrow \infty} \frac{\mathbb{E}_{\sigma}^{\Mdp} \{\sum_{i=0,s_i \in T}^{n}\tau_{s_{i}}\}}{\mathbb{E}_{\sigma}^{\Mdp} \{\sum_{i=0}^{n} \tau_i\}}
% \]
% where $\sum_{i=0,s_i \in T}^{n}\tau_{s_{i}}$ gives the expected time spent in states of $T$ and $\sum_{i=0}^{n} \tau_i$ is the total time. This gives the fraction of time spent in states of $T$ in the long run (i.e when $t \rightarrow \infty$) i.e,
% \[
% \sum_{s_i \in T}\lim_{t \rightarrow \infty} P(X(t) = s_i) = \lim_{n \rightarrow \infty} \frac{\mathbb{E}_{\sigma}^{\Mdp} \{\sum_{i=0,s_i \in T}^{n}\tau_{s_{i}}\}}{\mathbb{E}_{\sigma}^{\Mdp} \{\sum_{i=0}^{n} \tau_i\}}
% \]
 
%  From the definition of $\pi_{T}^{\Mdp^{[\sigma]}}$, we can conclude that 
% \[
% Ear^{\ctmc} = \pi_{T}^{\ctmc}
% \]
% \end{proof}
% From Lemma \ref{csl1}, it is clear that for a given set $T$ of states, a reward function $r'$, and a strategy $\sigma$, the expected average reward and the limiting probability of $T$ in the resulting CTMC are the same.

% \begin{theorem}\label{csl2}
% For a CTMDP $M$, and a set $T$ of states, an optimal pure strategy $\sigma^*$ maximising the expected average reward with respect to $r'$ results in a CTMC $M^{[\sigma^*]}$ which also maximises the limiting probability of $T$.
% \end{theorem}
% \begin{proof}
% From Lemma \ref{csl1}, for each pure strategy $\sigma$,
% \[
% Ear^{\Mdp^{[\sigma]}}(T) = \pi_{T}^{\Mdp^{[\sigma]}}
% \]
% Therefore, if $\sigma^*$ is the strategy maximising the expected average reward then we can conclude that it also maximises the limiting probability of $T$.
% \end{proof}
% From Theorem \ref{csl2}, it is clear that the reward function $r'$ gives the optimal pure strategy that maximises the limiting probability of $T$. From Theorem \ref{csl2}, and  Corollary \ref{corollary:1}, we obtain the following.
% \begin{corollary}\label{corollary2}
% For a CTMDP $\Mdp$, a set $T$ of states and a reward function $r'$, there exists a Blackwell optimal pure strategy $\sigma^*$ such that it maximises \begin{inparaenum}
% \item the expected average reward, 
% \item and the limiting probability of $T$ is maximised in $\Mdp^{[\sigma^*]}$.
% \end{inparaenum}
% \end{corollary}
% From Corollary \ref{corollary2}, it is clear that given a set $T$ of states, the problem of obtaining an optimal strategy maximising the expected average reward and obtaining a strategy which maximises the limiting probability of $T$ are the same.

% Finally, we describe the relation between continuous time stochastic logic(CSL) \cite{baier1999CSL} which is a stochastic branching time temporal logic that is used to model-check CTMCs against properties expressed in it and our expectation semantics.
% This logic contains a steady state operator $\mathcal{S}$ and maximizing the probability of satisfaction of this $\mathcal{S}$ for a formula that is true in the good states of a CTMDP corresponds to maximising the expected time of stay in these sttaes.

\begin{comment}
{\color{red} What did we decide regarding CSLs?}
Continuous time stochastic logic(CSL) \cite{baier1999CSL} is a stochastic branching time temporal logic and is used to model-check CTMCs against properties expressed in it. We show the relationship between expectation semantics and CSL.
\subsection{CSL}
Consider a CTMC $H$, the state formulas of CSL are interpreted over the states of $H$. Let $a \in \atomicprop$, and $p \in [0,1]$, the syntax of a state formula is defined by:
\begin{center}
    $\phi := True | a | \phi_1 \land \phi_2 | \lnot \phi | \mathcal{S}_{\bowtie p}(\phi) | \mathcal{P}_{\bowtie p}(\psi)$
\end{center}
Let $I \subseteq \realpositives $ be an interval. A path formula is defined by:
\begin{center}
    $\psi := \nextt \phi | \phi_1 \until \phi_2 | \phi_1 \until^{I} \phi_2$
\end{center}
Where $\bowtie \in \{\geq,\leq\}$.
Here the meaning of $\nextt$ and $\until$ is the same as LTL. The operator $\until^I$ is a timed variant of until operator used in LTL. For a path $\rho$ in a CTMC, we denote the state of $\rho$ at time $t$ by $\rho[t]$. For a time interval $I$, a formula $\phi_1 \until^I \phi_2$ is satisfied if there exists a path $\rho$ such that there exists a time $t \in I$ where $\rho[t]$ satisfies $\phi_2$ and for all $u \in [0,t)$ $\rho[u]$ satisfies $\phi_1$. The operator $\mathcal{S}_{\bowtie p}(\phi)$ is satisfied if the limiting probability of $sat (\phi)$ (the set of states satisfying $\phi$) is $q$ and $q\bowtie p$. The operator $\mathcal{P}_{\bowtie p}(\phi)$ is satisfied if the probability of paths from a state s satisfying $\phi$ is $q$ and $q\bowtie p$. We concentrate on the $S$ operator and relate it to the expectation semantics.

The semantics of state and path formulas are the same as that of LTL, and we denote the semantics of $\mathcal{S}$ operator as
\begin{center}
    $\initstate \models \mathcal{S}_{\bowtie p}(\phi)$ iff $\pi^{H}_{Sat(\phi)} \bowtie p$,
\end{center}
where $sat(\phi)$ is the set of states satisfying the formula $\phi$. 

For a CTMDP $\Mdp$, we say that it satisfies a CSL formula $\phi$ if there exists a strategy $\sigma$ such that $\Mdp^{[\sigma]}$ satisfies $\phi$. We look at formulas of the form $\mathcal{S}_{\geq p}(\phi)$ that are called steady-state formulas, where $\phi$ is an LTL formula. We show that the reward function $r'$ can be used to check if there exists a strategy that satisfies a steady-state formula. We provide a sketch of this procedure.


Given an $\omega$-regular property $\phi$, Theorem \ref{GFM} states that a corresponding \textbf{GFM} $A_\phi$ exists. For a given CTMDP $M$ and a \textbf{GFM} $A_\phi$, let $(\Mdp \times A_\phi)$ be the product CTMDP as defined in Section~\ref{prelims}. Let $Acc$ be the set of accepting states in $(\Mdp \times \caut)$. We can construct a reward function $r'$ with respect to $Acc$ as shown previously. From Theorem~\ref{csl2}, a strategy maximising the expected average reward also maximises the limiting probability for $Acc$. Let $\sigma^*$ be an optimal pure strategy in $(\Mdp \times \caut)$ maximising the expected average reward. Therefore, $\sigma^*$ also maximises the limiting probability for the states in $Acc$ in $(\Mdp \times A_{\phi})^{[\sigma^*]}$ and let this limiting probability be denoted by $q_{Acc}$, i.e, 
\begin{center}
$q_{Acc} = \sum_{s \in Acc} \pi^{(\Mdp \times A_{\phi})^{[\sigma^*]}}_{s}(\phi)$    
\end{center}


From the semantics of CSL, a formula $\mathcal{S}_{\geq p}  (\phi)$ is satisfied in $(\Mdp \times \caut)$ if there exists a strategy $\sigma$ such that 
\begin{center}
    $\pi^{(\Mdp \times \caut)^{[\sigma]}}_{sat(\phi)} \geq p$
\end{center}
From the definitions, we know that $sat(\phi) = Acc$ and as $\sigma^*$ is an optimal pure strategy maximising the limiting probability of $Acc$, we can conclude that $(\Mdp \times A_\phi)$ satisfies $\mathcal{S}_{\geq p}(\phi)$ iff $q_{Acc} \geq p$.
\end{comment}




