


\section{Applications}\label{sec:applications}
In this section, we show how our technique is a powerful tool for analytically realizing desired operations. This technique succeeds both for Hamiltonian simulation problems and general control problems. In particular, we show how the aforementioned physical intuition for a desired transformation is often sufficient to produce an approach to create desired operations.

We include applications in both phase and Fock space:
\begin{enumerate}
    \item \textbf{Phase-space} techniques are demonstrated to be useful in the case where displacements ($e^{(\alpha a^\dag + \alpha^* a)} = e^{i \alpha \hat{x}}$ for $\alpha$ real or $e^{\alpha \hat{p}}$ for $\alpha$ imaginary) are the only experimentally available gates. We 
    produce the controlled parity operator
    $e^{it \sigma^z a^{\dagger}a}$ (\cref{apps-nondestructive-meas}); the beamsplitter $e^{-it \sigma^z (a^{\dagger}b + ab^{\dagger})}$ (\cref{subsec:Conditional-(Controlled-Phase)-Beam-Splitter}); gates for two encodings of universal control of the restricted $\text{span}\{\left|0\right>,\left|1\right>\}$ Hilbert space (\cref{subsec:Universal-Control}); and gates for simulation of Fermi-Hubbard lattice dynamics using the two lowest Fock states of the cavity (\cref{sec:Fermi-Hubbard-Lattice-Dynamics}), including same-site, hopping, controlled-beamsplitter (\cref{subsec:controlled-phase}), and FSWAP gates.
    \item \textbf{Fock space} techniques are shown to be useful assuming compilation to $\mathcal{S}_1$ (\cref{defn:SX}) 
    and single-qubit operations to produce polynomials of annihilation and creation operators, (namely $a^p {a^\dagger}^q$ for integer $p, q$). We demonstrate how polynomials of these operators can be used in Hamiltonian simulation (e.g. with $\chi^{(3)}$ nonlinear materials, \cref{apndx:error-jc}) and state preparation (\cref{subsec:state-prep}).
\end{enumerate}




\subsection{Nonlinear Hamiltonian simulation}
As a simple application, let us consider the case of simulating a $\chi^{(3)}$ nonlinear material. These interactions commonly occur in nonlinear optics and appear when the index of refraction for a material varies linearly with the intensity of the electromagnetic field.  Such interactions can be modeled for a single \qumode~using the expression
\begin{equation}\label{eq:desired-ham}
    H = \omega a^\dagger a + \frac{\kappa}{2}(a^\dagger)^2 a^2.
\end{equation}
Our goal here is to examine the cost of a simulation of such a Hamiltonian in our model for time $t$ and error tolerance $\epsilon$ and to determine the parameter regimes within which a hybrid simulation using our techniques could provide an advantage with respect to a conventional qubit-based simulation of the Hamiltonian. 

Each of the terms can be approximated using formulas from \cref{tab:all-formulas}. The $\omega a^\dagger a$ term requires an embedding of Hermitian $a^\dagger a$, so that
\begin{align}
    \textrm{BCH} \left(S \cdot  i \tau_1 \mathcal{B}_{a^\dagger} \cdot S^\dagger, X \cdot  i \tau_1 \mathcal{B}_{a} \cdot X \right) = 
    2 i \tau_1^2 \begin{bmatrix}
         a^\dagger   a  & 0 \\
        0 & - a a^\dagger  
    \end{bmatrix}.
\end{align}
The second order term is treated in the same way, noting that $\mathcal{B}_{(a^\dagger)^2}$ can be produced via \cref{tab:all-formulas}, so that
\begin{align}
    \textrm{BCH} \left(S \cdot  i \tau_1 \mathcal{B}_{(a^\dagger)^2} \cdot S^\dagger, X \cdot  i \tau_1 \mathcal{B}_{a^2} \cdot X \right) = 
    2 i \tau_2^2 \begin{bmatrix}
        (a^\dagger)^2 a^2 & 0 \\
        0 & - a^2 (a^\dagger)^2
    \end{bmatrix}.
\end{align}
Thus, via the BCH formula, we can block-encode the two Hamiltonian terms. Trotterizing allows us to block-encode the entire Hamiltonian into the upper-left quadrant. Thus, by setting the qubit to $\ket{0}$, we can approximate the Hamiltonian. The error scaling is as follows and is proven in \cref{apndx:error-jc}:

\begin{restatable*}[Generating non-linear Hamiltonians]{theorem}{resultjc}\label{app:jaynes-cummings}
Let $H$ be the following non-linear Hamiltonian:
 \begin{align}
     H = \omega a^\dagger a + \frac{\kappa}{2}(a^\dagger)^2 a^2,
 \end{align}
(i.e. a Hamiltonian with a Kerr non-linearity). Let $t $ be the evolution time and $\epsilon$ be the target error tolerance.
For any positive integer $q$ we can approximate an exponential of the block-encoded Hamiltonian with error at most $\epsilon$ in the operator norm using $r e^{\mathcal{O}(q)}$ $\mathcal{S}_1$ operations where $r \in \Omega\left( \frac{(\Lambda^{4} t)^{1 + 1 / (q - \frac{3}{4})} }{\epsilon^{1 / (q - \frac{3}{4})}} \right)$. 
\end{restatable*}




This shows that we can perform a simulation of the dynamics within error $\epsilon$ using a number of operations within our instruction set that scales near-linearly with the evolution time and subpolynomially with $\epsilon$.  Further, this approach requires no ancillary memory and can be done with a single \qumode~and a qubit. In contrast, a qubit-only device would require a polylogarithmic number of qubits in $\Lambda$.

It is worth noting that in this case the ancillary qubit is not being used directly in the model.  Instead it is being used to control the dynamics and generate the appropriate nonlinear interaction between the photons present in the model.

\subsection{Nondestructive measurement of the qumode}\label{apps-nondestructive-meas}

We now demonstrate how the approach can extend beyond problems in Hamiltonian simulation. We begin with an example of the technique for control: In particular, we seek to perform a nondestructive measurement of the qumode in which we project the information into the qubit~\cite{fockreadout_wang_2020, Fockshotresolved_curtis_2021}. 


To construct such a nondestructive measurement, we seek to implement  $e^{i t \hat{n} \sigma^z}$ where $\hat{n} = a^\dagger a$ is the number operator. If we could implement this gate for arbitrary $t$, we could perform phase estimation on the qubit to nondestructively project the qumode into a fixed number of bosons. This could be done by setting $t$ sufficiently small so that $t \Lambda \leq 2 \pi$ is calculable with phase estimation. Alternatively, for $t = \pi$, this operation checks the parity of the qumode and applies an RZ gate for odd parities. We employ the instruction set in the phase-space representation to synthesize the infinitesimal conditional rotation gate
\begin{equation}
U_{\text{rot},k}=e^{i\lambda^{2}\hat{n}\sigma^{k}}
\end{equation}
for $k=x,y,z$. We rewrite $\hat{n}$ in terms of the phase-space operators by recognizing:
\begin{align}
\hat{n} & =\hat{a}^{\dagger}\hat{a}\\
 & =\hat{x}^{2}+\hat{p}^{2}-\frac{1}{2}\label{eq:NumbertoPhaseSpace}.
\end{align}
Applying Eqs.~\ref{eq:AnnihilationOperatortoPhaseSpace},~\ref{eq:CreationOperatortoPhaseSpace}, and~\ref{eq:NumbertoPhaseSpace}
yields
\begin{equation}
i\lambda^{2}\hat{n}\sigma^{k}=i\left(\hat{x}^{2}+\hat{p}^{2}-\frac{1}{2}\right)\sigma^{k}\lambda^{2},
\end{equation}
such that the gate is expressed via the Trotter decomposition as the product of $\exp\left(\left[A_{1},B_{1}\right]\right)=\exp\left(i\lambda^{2}\hat{x}^{2}\sigma^{k}\right)$,
$\exp\left(\left[A_{2},B_{2}\right]\right)=\exp\left(i\lambda^{2}\hat{p}^{2}\sigma^{k}\right)$,
and conditional displacement $\exp\left(-i\lambda^{2}\sigma^{k}/2\right)$.
Given the Pauli commutator relation, the
first commutator is 
\begin{align}
\left[A_{1},B_{1}\right] & =i\hat{x}^{2}\sigma^{k}\\
 & =i\hat{x}^{2}\left(-\frac{i}{2}\left[\sigma^{i},\sigma^{j}\right]\right)\\
 & =\left[\frac{1}{\sqrt{2}}\hat{x}\sigma^{i},\frac{1}{\sqrt{2}}\hat{x}\sigma^{j}\right],
\end{align}
and the second commutator is 
\begin{align}
\left[A_{2},B_{2}\right] & =i\hat{p}^{2}\sigma^{k}\\
 & =i\hat{p}^{2}\left(-\frac{i}{2}\left[\sigma^{i},\sigma^{j}\right]\right)\\
 & =\left[\frac{1}{\sqrt{2}}\hat{p}\sigma^{i},\frac{1}{\sqrt{2}}\hat{p}\sigma^{j}\right],
\end{align}
such that both terms are amenable to BCH decomposition,
and the infinitesimal conditional rotation is composed with a gate-depth lower bound of nine. To perform an error analysis, we may directly apply the error scaling of BCH and Trotter to find:

\begin{restatable*}{theorem}{resultmeasurement}
Suppose we can implement $\text{e}^{ it \hat{x} \sigma^i}, \text{e}^{ it \hat{p} \sigma^i}$ without error. Then, we may approximate $\exp it \mathcal{B}_{\hat{x}^2 + \hat{p}^2}$ with arbitrary error scaling $p$ as
\begin{align}
    \norm{ \exp it \widetilde{\mathcal{B}}_{\hat{x}^2 + \hat{p}^2} - \exp it \mathcal{B}_{\hat{x}^2 + \hat{p}^2} } \in \mathcal{O}( (C t)^{p + 1/ 2}),
\end{align}
where $C = \max( \norm{ \hat{x}^2 + \hat{p}^2 }, \norm{\hat{x}}^2, \norm{\hat{p}}^2)$ and using no more than $4 \cdot 5^{ \frac{p}{2} - \frac{1}{4} }$ exponentials.
\end{restatable*}

We then provide numerics in \cref{fig:ConditionalPhase}. As expected, the wavefunction
initialized in the second excited state of the cavity and the ground
state of the associated qubit has an autocorrelation function that
oscillates with phase $\exp(2it)$. Dynamics are well reproduced with
$2000$ time steps for a final time of $20$ with
cutoff $\Lambda=14$. Note the units are arbitrary in the absence
of definition of the cavity frequency $\omega$, with the only units
defined by setting the reduced Planck constant to unity $\hbar=1\text{ arb. units}$.
The close agreement between the BCH-synthesized and exact gates is
supported by the error scaling after a single gate application computed for time step $t$, which features a power law scaling in agreement with the predicted error scaling
for both BCH and Trotter decompositions.\begin{figure}[!ht]
\begin{centering}
\includegraphics[width=0.5\columnwidth]{pics/ConditionalRotation/bch.png}\includegraphics[width=0.5\columnwidth]{pics/ConditionalRotation/error.png}
\par\end{centering}
\caption{The following plots characterize the performance of using phase-space operators to synthesize $U_{\text{rot},k} = e^{i \hat{n} \sigma^z t}$. (a) The BCH-synthesized conditional rotation gate 
successfully approximates the exact dynamics for a wavefunction initialized
in the ground state of the qubit and the second excited state of
the cavity.  (b) Error of the real part of the autocorrelation
function $\text{Re}\left(\left<\psi(0)\middle|\psi(t)\right>\right)$ for the BCH-synthesized gate after a single time step of length $t=0.01$.\label{fig:ConditionalPhase}}
\end{figure}



We can obtain a similar decomposition with Fock-space operators. Observe that the $\textrm{MULT}$ subroutine applied to the $\exp it \mathcal{B}_a$, $\exp it \mathcal{B}_{a^\dagger}$ using \cref{lem:multiplication-alg} can also yield 
\begin{align}
	\norm{\textrm{MULT}( it \mathcal{B}_a, it \mathcal{B}_{a^\dagger}) - \exp i t \begin{bmatrix}
		a^\dagger a & 0 \\
		0 & - a a^\dagger
	\end{bmatrix} } \in \mathcal{O}((C^2 t)^{p + 1 / 2}),
\end{align}
when $\mathcal{B}_{a}$'s implementation is error-free. 
Note that $a a^\dagger = a^\dagger a + \identity$,  so the block encoding that is actually applied is
\begin{align}
\exp i t\left(\begin{bmatrix}
	a^\dagger a & 0 \\
	0 & - a^\dagger a 
\end{bmatrix} + \begin{bmatrix}
	0 & 0 \\
	0 & - \identity
\end{bmatrix}\right) = \exp it ( \sigma^z \hat{n}  -  (\identity - \sigma^z) \identity_{\Lambda + 1} ).
\end{align}
Thus, our Fock-space methods would also achieve the same transformation, albeit requiring a phase and RZ correction. 


\subsection{State preparation from the vacuum}
Consider the case where we seek to prepare Fock state $\ket{k}_b$ on the qumode. On qubit devices, preparing integer states is trivial, assuming the qubits represent logic in a binary fashion. However, hybrid boson-qubit devices natively implement exponentials of the phase-space or Fock-space operators. Thus, preparing $\ket{k}_b$ directly can often be challenging. Existing work \cite{law_arbitrary_1996} enables the preparation of states, but are less granular in their control of the \qq~state. We present a method that only swaps the $\ket{1} \ket{0} \leftrightarrow \ket{0} \ket{k}$ states for integer $k$, otherwise leaving the initial state intact. 

To begin, we aim to implement $(a^\dagger)^k$ on the vacuum. It is sufficient to approximate
\begin{align}
    \mathcal{T}_k(t) \coloneqq \exp i t\begin{bmatrix}
        0 & (a^\dagger)^k \\
        a^k & 0
    \end{bmatrix},
\end{align}
which we call the `unprotected' state preparation operator. 
Selecting appropriate $t$ yields precisely the desired behavior, which gives the following result:
\begin{restatable*}[Unprotected state preparation]{theorem}{statepreptime}\label{thm:statepreptime}
    For $k \leq \Lambda$, we can take $t = (2n + 1) \frac{\pi}{2 \sqrt{k!}}$ for any $n \in \mathbb{N}$ so that
    \begin{align*}    
    \mathcal{T}_k(t) \ket{1} \kron \ket{0} = \ket{0} \kron \ket{k}. 
    \end{align*}
\end{restatable*}



\newcommand{\diag}{{\rm{diag}}}
\newcommand{\stateprep}{\mathcal{P}_k}
\newcommand{\stateprepapprox}{\widetilde{\mathcal{P}}_{k, p}}

While $\mathcal{T}_k (t)$ performs the desired transformation, it may incur unwanted side effects if the starting state is of the form $\ket{1} \kron \ket{b}$ for $b > 0$. We can use our same approach to produce the following operation:

\begin{restatable*}[Protected state preparation]{theorem}{resultstateprep}\label{lem:fock-prep-unitary}
Consider the Fock preparation unitary $\stateprep$ with the form
\begin{align*}
    \exp \left( i t \begin{bmatrix}
        0 & ( a^\dagger )^k \ketbra{0}{0} \\
        \ketbra{0}{0} ( a )^k & 0
    \end{bmatrix} \right).
\end{align*}
When $t = (2n + 1) \frac{\pi}{4 \sqrt{k!}} $, we have that $\stateprep$ performs our desired state preparation
\begin{align*}
    \exp \left( i t \begin{bmatrix}
        0 & ( a^\dagger )^k \ketbra{0}{0} \\
        \ketbra{0}{0} ( a )^k & 0
    \end{bmatrix} \right) \ket{1} \kron \ket{b}  = \begin{cases}
    \ket{0} \kron  \ket{k} & b = 0 \\
    \ket{1} \kron  \ket{b} & b \neq 0
    \end{cases}.
\end{align*}
We claim that we can approximate this unitary with $\stateprepapprox$ where
\begin{align*}
    \norm{\stateprepapprox - \stateprep} \in \mathcal{O}((\Lambda^{k/2} t)^p),
\end{align*}
using no more than $4 \cdot 5^{q -1}$ $\widetilde{\mathcal{T}}_{k,p}$ subroutines.

\end{restatable*}
The proofs of \cref{thm:statepreptime}, \cref{lem:fock-prep-unitary} are provided in \cref{state_prep_proof}. Though this subroutine appears expensive, numerical results suggest it is far more implementable than theory would suggest. In the following simulations, we apply the above technique but always use a second-order symmetrized BCH formula and second-order (symmetrized) Trotter formula. This amounts to 480 exponentials for the unprotected case and 960 exponentials for the protected case. The synthesized gates are provided in \cref{fig:unprotectedt2} and \cref{fig:protectedt2}.  

\change{As can be seen in \cref{fig:both-errors}, the protected gate has higher compilation errors than the unprotected gate. This is likely due to the fact that the protected formula requires an additional application of the Trotter product formula. This will further incur commutator error over the unprotected gate.}

\begin{figure}[!ht]
    \centering
    \includegraphics[scale=0.6]{pics/stateprep/T2-Unprotected.png}
    \caption{Unprotected $T_2$ Gate: plots visualize the unitary's elements. Brighter colors represent higher absolute values / magnitudes of the matrix entries. \change{\textit{Left}} is the exact form of the $T_2$ gate with selected $t$ and \change{\textit{right}} is the BCH-synthesized form. The state $\ket{j}_q \ket{k}_m$ has index $j \cdot \Lambda + k$ where $\Lambda$ is the cutoff. By observation, the analytically realized form is accurate.}
    \label{fig:unprotectedt2}
\end{figure}
\begin{figure}[!ht]
    \centering
    \includegraphics[scale=0.6]{pics/stateprep/T2-protected.png}
    \caption{Protected $T_2$ Gate: \change{\textit{left}} is the exact form of the protected $T_2$ gate with selected $t$ and \change{\textit{right}} is the BCH-synthesized form.   The state $\ket{j}_q \ket{k}_m$ has index $j \cdot \Lambda + k$ where $\Lambda$ is the cutoff. By observation, the analytically realized form is still accurate, albeit with more incurred Trotter error. 
    }
    \label{fig:protectedt2}
\end{figure}

\begin{figure}
    \centering
    \includegraphics[width=0.8\linewidth]{pics/error_both_protected_unprotected.png}
    \caption{\change{Absolute element-wise error when compiling $T_2$ gate via scheme: \textit{left} is when the compilation target is the unprotected gate, while \textit{right} is when the target is the protected gate. The protected gate has higher synthesis errors.}}
    \label{fig:both-errors}
\end{figure}



We also analyze the error scaling as the order of the BCH formulas used increases. \cref{fig:protectedt2-error} describes the error-resource tradeoff as the Trotter step within each BCH formula increases. Observe that the error decays as higher order formulas are used or Trotter step size is reduced, as expected. While modest depths have relatively high infidelity, in theory our compilations can achieve arbitrary accuracy at the level of both entire gates and matrix individual elements. %
\begin{figure}[!ht]
    \centering
    \mbox{
    \subfigure[]{\includegraphics[scale=0.42]{pics/stateprep/errorfidT2norm.png}}
    \subfigure[]{\includegraphics[scale=0.42]{pics/stateprep/errorT2norm.png}}
    }
    \caption{Error scaling of protected $T_2$ gate with respect to BCH circuit depth according to (a) the state-fidelity metric and (b) the trace norm. Prime denotes that the formula has been symmetrized. %
    }
    \label{fig:protectedt2-error}
\end{figure}

\subsection{Hong-Ou-Mandel effect/conditional (controlled-phase) beam splitter gate}\label{subsec:Conditional-(Controlled-Phase)-Beam-Splitter}
The operations we seek to realize need not act on a single \qumode; in fact, our techniques are extensible to hybrid setups with multiple \qumodes~or qubits. Consider the conditional (controlled-phase) beam splitter
\begin{align}
U_{\text{beam split}} & =e^{-i\lambda^{2}\left(\hat{a}_{1}^{\dagger}\hat{a}_{2}+\hat{a}_{1}\hat{a}_{2}^{\dagger}\right)\sigma^{z}}.
\end{align}
This gate naturally pertains
to certain lattice gauge theories \cite{C2QA_LGT} and gives rise to exponential SWAP
(eSWAP) \cite{gao2019entanglement} and controlled-SWAP (cSWAP)
gates for state purification and SWAP tests, when paired with an ordinary (uncontrolled)
beam splitter \cite{pietikainen2022controlled}. 
The argument in phase-space representation 
is 
\begin{align}
-i\lambda^{2}\left(\hat{a}_{1}^{\dagger}\hat{a}_{2}+\hat{a}_{1}\hat{a}_{2}^{\dagger}\right)\sigma^{z} & =-2i\lambda^{2}\left(\hat{x}_{1}\hat{x}_{2}+\hat{p}_{1}\hat{p}_{2}\right)\sigma^{z},
\end{align}
such that the gate is decomposed in terms of a Trotter expansion
as the product of two exponential
terms $\exp\left(\left[A_{1},B_{1}\right]\lambda^{2}\right)=\exp\left(-2i\lambda^{2}\hat{x}_{1}\hat{x}_{2}\sigma^{z}\right)$
and $\exp\left(\left[A_{2},B_{2}\right]\lambda^{2}\right)=\exp\left(-2i\lambda^{2}\hat{x}_{1}\hat{x}_{2}\sigma^{z}\right)$.
According to the Pauli commutation relation, 
the first commutator is
\begin{align}
\left[A_{1},B_{1}\right] & =-2i\hat{x}_{1}\hat{x}_{2}\sigma^{z}\\
 & =-2i\hat{x}_{1}\hat{x}_{2}\left(-\frac{i}{2}\left[\sigma^{x},\sigma^{y}\right]\right)\\
 & =\left[i\hat{x}_{1}\sigma^{x},i\hat{x}_{2}\sigma^{y}\right],
\end{align}
and the second is
\begin{align}
\left[A_{2},B_{2}\right] & =\left[i\hat{p}_{1}\sigma^{x},i\hat{p}_{2}\sigma^{y}\right],
\end{align}
with the following error scaling:

\begin{restatable*}{theorem}{resultbeamsplitter}
Assume we may implement $\text{e}^{ i t \hat{x}_m \sigma^j}, \text{e}^{ i t \hat{p}_m \sigma^j}$ for $m \in \{ 1, 2 \}$; i.e., we may implement the qubit-conditional position shifts and momentum boosts on either \qumode~without error. Then, we may approximate $\mathcal{B}_{\hat{x}_1 \hat{x}_2 + \hat{p}_1 \hat{p}_2}$ with arbitrary error scaling $p$ as
\begin{align*}
    \norm{\exp it \widetilde{\mathcal{B}}_{\hat{x}_1 \hat{x}_2 + \hat{p}_1 \hat{p}_2} - \exp it \mathcal{B}_{\hat{x}_1 \hat{x}_2 + \hat{p}_1 \hat{p}_2} } \in \mathcal{O}((Ct)^{p + \frac{1}{2}}),
\end{align*}
where $C = \max( \norm{ \hat{x}_1 \hat{x}_2 + \hat{p}_1 \hat{p}_2}, \norm{\hat{x}_1}^2, \norm{\hat{x}_2}^2, \norm{\hat{p}_1}^2, \norm{\hat{p}_1}^2)$ and using no more than $4 \cdot 5^{ \frac{p}{2} - \frac{1}{4} }$ exponentials.
\end{restatable*}


The two exponential terms are decomposed via the BCH formula 
for a lower-bound gate depth of eight. Results are shown in~\cref{fig:ConditionalBeamSplitter} for
$15$ states per cavity with a shared qubit over a final time of
$\pi/2$ with $200$ equal time
steps, where the system is initially in the first excited state of
each cavity and the ground state of the shared qubit $\left|11g\right>$.
As expected for the conditional beam splitter, the gate exhibits the
Hong-Ou-Mandel effect, in which the occupation of cavity 1 oscillates
between the first excited mode and a superposition of the ground and
the second excited states of the cavity. The BCH-synthesized results
closely agree with that of the original gate, with no visible leakage
beyond the physical states (the lowest three states of the cavity)
into the working space under the time duration studied. As for the
conditional rotation gate, the relative error of the BCH-synthesized
gate computed for a single time step of length $t$ was found to scale according to a power law with the time step,
in accordance with the analytic result for Trotterization and BCH
decomposition.

\begin{figure}[!ht]
\begin{centering}
\includegraphics[width=0.5\columnwidth]{pics/BeamSplitter/probabilityexact}\includegraphics[width=0.5\columnwidth]{pics/BeamSplitter/probabilitybch}
\par\end{centering}
\begin{centering}
\includegraphics[width=0.5\columnwidth]{pics/BeamSplitter/probabilityworkbch}\includegraphics[width=0.5\columnwidth]{pics/BeamSplitter/error}
\par\end{centering}
\caption{Hong-Ou-Mandel effect simulated with (a) exact and (b) BCH-synthesized
conditional beam splitters, illustrated as probability cavity 1 is found in states $\left|0\right>$, $\left|1\right>$, or $\left|2\right>$. \change{Observe that the probabilities of $\ket{0}, \ket{2}$ coincide}; (c) probability of leakage into higher cavity modes; and (d) error of the real part of the autocorrelation
after a single application of a BCH-synthesized gate for time step $t$ relative to the application of the exact gate for the same time step.\label{fig:ConditionalBeamSplitter}}%
\end{figure}





{






















































}


