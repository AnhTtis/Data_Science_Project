\section{Introduction}



Today, many quantum computing architectures are homogeneous, with the same type of qubit used throughout the device. From devices made of superconducting qubits \cite{arute2019quantum,bravyi2022future,reagor2018demonstration} to ion trap qubits \cite{wright2019benchmarking}, prior work largely focuses on linking qubits of the same type together in fault-tolerant ways. However, there is emerging work \cite{C2QA_ISA, crane2024hybrid, chakram2021seamless, stavenger2022bosonic,stein2023microarchitectures} 
studying the use of heterogeneous quantum computers that leverage two or more types of quantum architectures (e.g., qubits and oscillator modes). Heterogeneous devices hold promise because they can be tailored for specific physical simulation problems, which would be especially useful in applications like material discovery~\cite{Holstein_Knorzer_2022}, molecular simulation~\cite{Wang2020FCFs,WangConicalIntersection}, topological models~\cite{PhysRevB.98.174505} or lattice gauge theories~\cite{C2QA_LGT}.


In particular, hybrid \qq~models \cite{blais2021circuit} hold some advantages: for example, microwave qumodes have long lifetimes and large accessible Hilbert spaces, making them
attractive targets for quantum error correction~\cite{blais2021circuit}. 
Introducing \qumodes~also enables new physical gates, 
such as the M{\o}lmer-S{\o}rensen gate \cite{molmer-sorensen-gate} while at the same time enabling new forms of transduction between qubits and qumodes \cite{Boissonneault_Dispersive_2009}.
Oscillator interactions have unique features, like nonlinearities, which are challenging to simulate even with homogeneous quantum architectures~\cite{stavenger2022bosonic}.




Efficiently compiling logical operations to physical pulses is a critical, but computationally expensive task. 
In theory, pulse design techniques like optimal control theory (OCT)~\cite{,werschnik2007quantum} 
can produce pulses that implement arbitrary quantum transformations on a hybrid \qq~system. These techniques have been applied to a variety of physical systems, including NMR~\cite{khaneja2005optimal}, superconducting transmon qubits~\cite{Koch-OCT-2017}, and \qumodes~\cite{ozguler_numerical_2022,anders_petersson_optimal_2022,ma_quantum_2021}. In practice, OCT is computationally intensive and inflexible, requiring pulses to be recompiled on a case-by-case basis. Furthermore, OCT is almost always uninterpretable, yielding only a pulse which performs the desired operation without providing any physical intuition. 
Experimentally, these limitations prevent the high-fidelity realization of quantum algorithms; theoretically, the complexity of our quantum circuits often inappropriately ignores the classical cost of required compilation. 


Inspired by recent experimental progress \cite{SNAP-PhysRevLett.115.137002,eickbusch2021fast}, we introduce an extensible control scheme for a universal, hybrid \qq~quantum computer (\cref{tab:all-formulas}).  
Specifically, we show how block-encoded operators can be manipulated 
using the Lie--Trotter--Suzuki and Baker--Campbell--Hausdorff matrix product formulas. We thus enable the creation of instruction set architectures (ISAs) that 
can be analytically compiled to experimentally available gate sets.  Prior art, namely \textcite{jacobs_engineering_2007}, has studied similar techniques to compile operations; we generalize these techniques to work in a variety of domains, including in settings with multiple \qumodes~and more exotic operators, and prove concrete error bounds on these techniques.  

We develop two parallel approaches, one which primarily uses the creation and annihilation operators which we refer to as based on `Fock methods,' and the other primarily relying on position and momentum operators which we refer to as based on `phase-space methods.' We demonstrate that both methods can be used to generate an ISA for \qq~devices. \change{Because our methods operate in both Fock and phase-space, we can achieve transformations that are natively described in either picture; for example, {exponentials of polynomials of annihilation and creation operators for Hamiltonian simulation and state preparation in Fock space and controlled parity and beam splitter operators in phase space}}. Our methods obtain almost-linear asymptotic scaling. 


Furthermore, we use the previously mentioned formulas to realize a number of operations of interest, including polynomials of annihilation and creation operators, namely $a^p {a^\dagger}^q$ for integer $p, q$. These block-encoded operations are crucial for quantum signal processing (QSP) and certain problems in quantum simulation. Finally, we give examples for the Hamiltonian of a nonlinear material and applications to key unsolved problems in quantum simulation such as the Fermi-Hubbard model. 
While these approaches are expensive in terms of raw gate counts, because they are analytic they provide intuition into synthesizing the gate robustly. Furthermore, these gate sequences can be used as a starting point for optimal control methods, helping to avoid cold start issues.
