


\section{Applications}\label{sec:applications}
In this section, we show how our technique is a powerful tool for analytically realizing desired operations. This technique succeeds both for Hamiltonian simulation problems and general control problems. In particular, we show how the aforementioned physical intuition for a desired transformation is often sufficient to produce an approach to create desired operations.

% \how{move to four applications}
We include applications in both phase and Fock space:
\begin{enumerate}
    \item \textbf{Phase-space} techniques are demonstrated to be useful in the case where displacements ($e^{(\alpha a^\dag + \alpha^* a)} = e^{i \alpha \hat{x}}$ for $\alpha$ real or $e^{\alpha \hat{p}}$ for $\alpha$ imaginary) are the only experimentally available gates. We 
    % combine these position $\hat{x}$ and momentum $\hat{p}$ operators with single-qubit rotations to 
    produce the controlled parity operator
    % , earlier referred to as the controlled parity operator, 
    $e^{it \sigma^z a^{\dagger}a}$ (\cref{apps-nondestructive-meas}); the beamsplitter $e^{-it \sigma^z (a^{\dagger}b + ab^{\dagger})}$ (\cref{subsec:Conditional-(Controlled-Phase)-Beam-Splitter}); gates for two encodings of universal control of the restricted $\text{span}\{\left|0\right>,\left|1\right>\}$ Hilbert space (\cref{subsec:Universal-Control}); and gates for simulation of Fermi-Hubbard lattice dynamics using the two lowest Fock states of the cavity (\cref{sec:Fermi-Hubbard-Lattice-Dynamics}), including same-site, hopping, controlled-beamsplitter (\cref{subsec:controlled-phase}), and FSWAP gates.
    \item \textbf{Fock space} techniques are shown to be useful assuming compilation to $\mathcal{S}_1$ (\cref{defn:SX}) 
    % we use qubit-controlled displacements $e^{\sigma^z(\alpha^* a + \alpha a^{\dagger})}$, rotation operator $e^{\sigma^z a^{\dagger}a}$ 
    and single-qubit operations to produce polynomials of annihilation and creation operators, (namely $a^p {a^\dagger}^q$ for integer $p, q$). We demonstrate how polynomials of these operators can be used in Hamiltonian simulation (e.g. with $\chi^{(3)}$ nonlinear materials, \cref{apndx:error-jc}) and state preparation (\cref{subsec:state-prep}).
\end{enumerate}


% For Hamiltonian simulation, we demonstrate a variety of potential systems which are simulatable,  
% We also contribute a variety of control techniques which are realizable, including the conditional rotation gate (\cref{subsec:cond-rot}), controlled-phase beam splitter gate (\cref{subsec:controlled-phase}) state preparation of arbitrary Fock states (\cref{subsec:state-prep}), and the embedding of an effective qubit in the bosonic mode (\cref{subsec:Universal-Control}).

% \how{ck: need to link into the missing appendices, describe the type of ham sim / control being done and link it}

\subsection{Nonlinear Hamiltonian simulation}
As a simple application, let us consider the case of simulating a $\chi^{(3)}$ nonlinear material. These interactions commonly occur in nonlinear optics and appear when the index of refraction for a material varies linearly with the intensity of the electromagnetic field.  Such interactions can be modeled for a single \qumode~using the expression
\begin{equation}\label{eq:desired-ham}
    H = \omega a^\dagger a + \frac{\kappa}{2}(a^\dagger)^2 a^2.
\end{equation}
Our goal here is to examine the cost of a simulation of such a Hamiltonian in our model for time $t$ and error tolerance $\epsilon$ and to determine the parameter regimes within which a hybrid simulation using our techniques could provide an advantage with respect to a conventional qubit-based simulation of the Hamiltonian. 

Each of the terms can be approximated using formulas from \cref{tab:all-formulas}. The $\omega a^\dagger a$ term requires an embedding of Hermitian $a^\dagger a$, so that
\begin{align}
    % \left[ i \tau_1 \begin{bmatrix}
    %     0 & -i  a^\dagger  \\
    %     i  a  & 0
    % \end{bmatrix},
    % i \tau_1 \begin{bmatrix}
    %     0 &  a^\dagger  \\
    %      a  & 0
    % \end{bmatrix}  \right] = 
    \textrm{BCH} \left(S \cdot  i \tau_1 \mathcal{B}_{a^\dagger} \cdot S^\dagger, X \cdot  i \tau_1 \mathcal{B}_{a} \cdot X \right) = 
    2 i \tau_1^2 \begin{bmatrix}
         a^\dagger   a  & 0 \\
        0 & - a a^\dagger  
    \end{bmatrix}.
\end{align}
The second order term is treated in the same way, noting that $\mathcal{B}_{(a^\dagger)^2}$ can be produced via \cref{tab:all-formulas}, so that
\begin{align}
    \textrm{BCH} \left(S \cdot  i \tau_1 \mathcal{B}_{(a^\dagger)^2} \cdot S^\dagger, X \cdot  i \tau_1 \mathcal{B}_{a^2} \cdot X \right) = 
    % \left[ i \tau_2 \begin{bmatrix}
    %     0 & -i (a^\dagger)^2 \\
    %     i a^2 & 0
    % \end{bmatrix}, i \tau_2 \begin{bmatrix}
    %     0 & (a^\dagger)^2 \\
    %     a^2 & 0
    % \end{bmatrix} \right] = 
    2 i \tau_2^2 \begin{bmatrix}
        (a^\dagger)^2 a^2 & 0 \\
        0 & - a^2 (a^\dagger)^2
    \end{bmatrix}.
\end{align}
Thus, via the BCH formula, we can block-encode the two Hamiltonian terms. Trotterizing allows us to block-encode the entire Hamiltonian into the upper-left quadrant. Thus, by setting the qubit to $\ket{0}$, we can approximate the Hamiltonian. The error scaling is as follows and is proven in \cref{apndx:error-jc}:

\begin{restatable*}[Generating non-linear Hamiltonians]{theorem}{resultjc}\label{app:jaynes-cummings}
Let $H$ be the following non-linear Hamiltonian:
 \begin{align}
     H = \omega a^\dagger a + \frac{\kappa}{2}(a^\dagger)^2 a^2,
 \end{align}
(i.e. a Hamiltonian with a Kerr non-linearity). Let $t $ be the evolution time and $\epsilon$ be the target error tolerance.
For any positive integer $q$ we can approximate an exponential of the block-encoded Hamiltonian with error at most $\epsilon$ in the operator norm using $r e^{\mathcal{O}(q)}$ $\mathcal{S}_1$ operations where $r \in \Omega\left( \frac{(\Lambda^{4} t)^{1 + 1 / (q - \frac{3}{4})} }{\epsilon^{1 / (q - \frac{3}{4})}} \right)$. 
% \how{is it reOq}
\end{restatable*}


%\how{nathan: what numerics do we want here?}

% get plots from \href{https://colab.research.google.com/drive/1yxOpabROaSCA-vOkNKu6QlYNxOuFtylm#scrollTo=Ff4t89vuDJEt}{here}

This shows that we can perform a simulation of the dynamics within error $\epsilon$ using a number of operations within our instruction set that scales near-linearly with the evolution time and subpolynomially with $\epsilon$.  Further, this approach requires no ancillary memory and can be done with a single \qumode~and a qubit. In contrast, a qubit-only device would require a polylogarithmic number of qubits in $\Lambda$.
% This is a dramatic memory reduction relative to the quantum case, which requires a polylogarithmic number of qubits in $\Lambda$.

It is worth noting that in this case the ancillary qubit is not being used directly in the model.  Instead it is being used to control the dynamics and generate the appropriate nonlinear interaction between the photons present in the model.

\subsection{Nondestructive measurement of the qumode}\label{apps-nondestructive-meas}
% \how{Notes for reviewer:}
% \begin{enumerate}
%     \item Where do we describe the full ISA?
%     \item 
% \end{enumerate}

% ---
We now demonstrate how the approach can extend beyond problems in Hamiltonian simulation. We begin with an example of the technique for control: In particular, we seek to perform a nondestructive measurement of the qumode in which we project the information into the qubit~\cite{fockreadout_wang_2020, Fockshotresolved_curtis_2021}. 
%Natively, measurements are destructive, i.e., counting the number of photons also consumes the photons. However, by using the qubit, we can instead project the information into the qubit~\cite{fockreadout_wang_2020, Fockshotresolved_curtis_2021}. 

% \how{is this actually a valid application}

To construct such a nondestructive measurement, we seek to implement  $e^{i t \hat{n} \sigma^z}$ where $\hat{n} = a^\dagger a$ is the number operator. If we could implement this gate for arbitrary $t$, we could perform phase estimation on the qubit to nondestructively project the qumode into a fixed number of bosons. This could be done by setting $t$ sufficiently small so that $t \Lambda \leq 2 \pi$ is calculable with phase estimation. Alternatively, for $t = \pi$, this operation checks the parity of the qumode and applies an RZ gate for odd parities. We employ the instruction set in the phase-space representation to synthesize the infinitesimal conditional rotation gate
\begin{equation}
U_{\text{rot},k}=e^{i\lambda^{2}\hat{n}\sigma^{k}}
\end{equation}
for $k=x,y,z$. We rewrite $\hat{n}$ in terms of the phase-space operators by recognizing:
\begin{align}
\hat{n} & =\hat{a}^{\dagger}\hat{a}\\
 & =\hat{x}^{2}+\hat{p}^{2}-\frac{1}{2}\label{eq:NumbertoPhaseSpace}.
\end{align}
Applying Eqs.~\ref{eq:AnnihilationOperatortoPhaseSpace},~\ref{eq:CreationOperatortoPhaseSpace}, and~\ref{eq:NumbertoPhaseSpace}
yields
\begin{equation}
i\lambda^{2}\hat{n}\sigma^{k}=i\left(\hat{x}^{2}+\hat{p}^{2}-\frac{1}{2}\right)\sigma^{k}\lambda^{2},
\end{equation}
such that the gate is expressed via the Trotter decomposition as the product of $\exp\left(\left[A_{1},B_{1}\right]\right)=\exp\left(i\lambda^{2}\hat{x}^{2}\sigma^{k}\right)$,
$\exp\left(\left[A_{2},B_{2}\right]\right)=\exp\left(i\lambda^{2}\hat{p}^{2}\sigma^{k}\right)$,
and conditional displacement $\exp\left(-i\lambda^{2}\sigma^{k}/2\right)$.
Given the Pauli commutator relation, the
first commutator is 
\begin{align}
\left[A_{1},B_{1}\right] & =i\hat{x}^{2}\sigma^{k}\\
 & =i\hat{x}^{2}\left(-\frac{i}{2}\left[\sigma^{i},\sigma^{j}\right]\right)\\
 & =\left[\frac{1}{\sqrt{2}}\hat{x}\sigma^{i},\frac{1}{\sqrt{2}}\hat{x}\sigma^{j}\right],
\end{align}
and the second commutator is 
\begin{align}
\left[A_{2},B_{2}\right] & =i\hat{p}^{2}\sigma^{k}\\
 & =i\hat{p}^{2}\left(-\frac{i}{2}\left[\sigma^{i},\sigma^{j}\right]\right)\\
 & =\left[\frac{1}{\sqrt{2}}\hat{p}\sigma^{i},\frac{1}{\sqrt{2}}\hat{p}\sigma^{j}\right],
\end{align}
such that both terms are amenable to BCH decomposition,
and the infinitesimal conditional rotation is composed with a gate-depth lower bound of nine. To perform an error analysis, we may directly apply the error scaling of BCH and Trotter to find:

\begin{restatable*}{theorem}{resultmeasurement}
Suppose we can implement $\text{e}^{ it \hat{x} \sigma^i}, \text{e}^{ it \hat{p} \sigma^i}$ without error. Then, we may approximate $\exp it \mathcal{B}_{\hat{x}^2 + \hat{p}^2}$ with arbitrary error scaling $p$ as
\begin{align}
    \norm{ \exp it \widetilde{\mathcal{B}}_{\hat{x}^2 + \hat{p}^2} - \exp it \mathcal{B}_{\hat{x}^2 + \hat{p}^2} } \in \mathcal{O}( (C t)^{p + 1/ 2}),
\end{align}
where $C = \max( \norm{ \hat{x}^2 + \hat{p}^2 }, \norm{\hat{x}}^2, \norm{\hat{p}}^2)$ and using no more than $4 \cdot 5^{ \frac{p}{2} - \frac{1}{4} }$ exponentials.
\end{restatable*}
% Where the Pauli rotation can be applied without implementation error or Trotter error. 

We then provide numerics in \cref{fig:ConditionalPhase}. As expected, the wavefunction
initialized in the second excited state of the cavity and the ground
state of the associated qubit has an autocorrelation function that
oscillates with phase $\exp(2it)$. Dynamics are well reproduced with
$2000$ time steps for a final time of $20$ with
cutoff $\Lambda=14$. Note the units are arbitrary in the absence
of definition of the cavity frequency $\omega$, with the only units
defined by setting the reduced Planck constant to unity $\hbar=1\text{ arb. units}$.
The close agreement between the BCH-synthesized and exact gates is
supported by the error scaling after a single gate application computed for time step $t$, which features a power law scaling in agreement with the predicted error scaling
for both BCH and Trotter decompositions.\begin{figure}[!ht]
\begin{centering}
\includegraphics[width=0.5\columnwidth]{pics/ConditionalRotation/bch.png}\includegraphics[width=0.5\columnwidth]{pics/ConditionalRotation/error.png}
\par\end{centering}
\caption{The following plots characterize the performance of using phase-space operators to synthesize $U_{\text{rot},k} = e^{i \hat{n} \sigma^z t}$. (a) The BCH-synthesized conditional rotation gate 
% $e^{i\hat{n}\sigma^{z}t}$
successfully approximates the exact dynamics for a wavefunction initialized
in the ground state of the qubit and the second excited state of
the cavity.  (b) Error of the real part of the autocorrelation
function $\text{Re}\left(\left<\psi(0)\middle|\psi(t)\right>\right)$ for the BCH-synthesized gate after a single time step of length $t=0.01$.\label{fig:ConditionalPhase}}
\end{figure}


% \how{will need to show how this can also be done via the fock-space operators}

We can obtain a similar decomposition with Fock-space operators. Observe that the $\textrm{MULT}$ subroutine applied to the $\exp it \mathcal{B}_a$, $\exp it \mathcal{B}_{a^\dagger}$ using \cref{lem:multiplication-alg} can also yield 
\begin{align}
	\norm{\textrm{MULT}( it \mathcal{B}_a, it \mathcal{B}_{a^\dagger}) - \exp i t \begin{bmatrix}
		a^\dagger a & 0 \\
		0 & - a a^\dagger
	\end{bmatrix} } \in \mathcal{O}((C^2 t)^{p + 1 / 2}),
\end{align}
when $\mathcal{B}_{a}$'s implementation is error-free. 
Note that $a a^\dagger = a^\dagger a + \identity$,  so the block encoding that is actually applied is
\begin{align}
\exp i t\left(\begin{bmatrix}
	a^\dagger a & 0 \\
	0 & - a^\dagger a 
\end{bmatrix} + \begin{bmatrix}
	0 & 0 \\
	0 & - \identity
\end{bmatrix}\right) = \exp it ( \sigma^z \hat{n}  -  (\identity - \sigma^z) \identity_{\Lambda + 1} ).
\end{align}
Thus, our Fock-space methods would also achieve the same transformation, albeit requiring a phase and RZ correction. 
% \how{error scaling}


\subsection{State preparation from the vacuum}
Consider the case where we seek to prepare Fock state $\ket{k}_b$ on the qumode. On qubit devices, preparing integer states is trivial, assuming the qubits represent logic in a binary fashion. However, hybrid boson-qubit devices natively implement exponentials of the phase-space or Fock-space operators. Thus, preparing $\ket{k}_b$ directly can often be challenging. Existing work \cite{law_arbitrary_1996} enables the preparation of states, but are less granular in their control of the \qq~state. We present a method that only swaps the $\ket{1} \ket{0} \leftrightarrow \ket{0} \ket{k}$ states for integer $k$, otherwise leaving the initial state intact. 

To begin, we aim to implement $(a^\dagger)^k$ on the vacuum. It is sufficient to approximate
\begin{align}
    \mathcal{T}_k(t) \coloneqq \exp i t\begin{bmatrix}
        0 & (a^\dagger)^k \\
        a^k & 0
    \end{bmatrix},
\end{align}
which we call the `unprotected' state preparation operator. 
Selecting appropriate $t$ yields precisely the desired behavior, which gives the following result:
\begin{restatable*}[Unprotected state preparation]{theorem}{statepreptime}\label{thm:statepreptime}
    For $k \leq \Lambda$, we can take $t = (2n + 1) \frac{\pi}{2 \sqrt{k!}}$ for any $n \in \mathbb{N}$ so that
    \begin{align*}    
    \mathcal{T}_k(t) \ket{1} \kron \ket{0} = \ket{0} \kron \ket{k}. 
    \end{align*}
\end{restatable*}
% Note here that we can implement such a mapping using the Baker-Campbell-Hausdorff formula (BCH).  



\newcommand{\diag}{{\rm{diag}}}
\newcommand{\stateprep}{\mathcal{P}_k}
\newcommand{\stateprepapprox}{\widetilde{\mathcal{P}}_{k, p}}

While $\mathcal{T}_k (t)$ performs the desired transformation, it may incur unwanted side effects if the starting state is of the form $\ket{1} \kron \ket{b}$ for $b > 0$. We can use our same approach to produce the following operation:

\begin{restatable*}[Protected state preparation]{theorem}{resultstateprep}\label{lem:fock-prep-unitary}
Consider the Fock preparation unitary $\stateprep$ with the form
\begin{align*}
    \exp \left( i t \begin{bmatrix}
        0 & ( a^\dagger )^k \ketbra{0}{0} \\
        \ketbra{0}{0} ( a )^k & 0
    \end{bmatrix} \right).
\end{align*}
% 
% \begin{align}
%     \exp \left( it \begin{bmatrix}
%         0 & p_{k, 1} \\
%         p_{1, k} & 0 
%     \end{bmatrix} \right)
% \end{align}
% Where $p_{1, k}$ refers to a $(\Lambda + 1) \times (\Lambda + 1)$ matrix where the only element is in the $1$st row and $k$th column with value $\sqrt{k!}$ ($p_{k, 1}$,  respectively). 
%
When $t = (2n + 1) \frac{\pi}{4 \sqrt{k!}} $, we have that $\stateprep$ performs our desired state preparation
\begin{align*}
    \exp \left( i t \begin{bmatrix}
        0 & ( a^\dagger )^k \ketbra{0}{0} \\
        \ketbra{0}{0} ( a )^k & 0
    \end{bmatrix} \right) \ket{1} \kron \ket{b}  = \begin{cases}
    \ket{0} \kron  \ket{k} & b = 0 \\
    \ket{1} \kron  \ket{b} & b \neq 0
    \end{cases}.
\end{align*}
We claim that we can approximate this unitary with $\stateprepapprox$ where
\begin{align*}
    \norm{\stateprepapprox - \stateprep} \in \mathcal{O}((\Lambda^{k/2} t)^p),
\end{align*}
using no more than $4 \cdot 5^{q -1}$ $\widetilde{\mathcal{T}}_{k,p}$ subroutines.

% \begin{align}
%     \exp \left( it \begin{bmatrix}
%         0 & p_{0, k} \\
%         p_{k, 0} & 0 
%     \end{bmatrix} \right) \ket{1} \ket{b} \mapsto \begin{cases}
%     \ket{0} \ket{k} & b = 0 \\
%     \ket{1} \ket{b} & b \neq 0
%     \end{cases}
% \end{align}
\end{restatable*}
The proofs of \cref{thm:statepreptime}, \cref{lem:fock-prep-unitary} are provided in \cref{state_prep_proof}. Though this subroutine appears expensive, numerical results suggest it is far more implementable than theory would suggest. In the following simulations, we apply the above technique but always use a second-order symmetrized BCH formula and second-order (symmetrized) Trotter formula. This amounts to 480 exponentials for the unprotected case and 960 exponentials for the protected case. The synthesized gates are provided in \cref{fig:unprotectedt2} and \cref{fig:protectedt2}.  
\begin{figure}[!ht]
    \centering
    \includegraphics[scale=0.6]{pics/stateprep/T2-Unprotected.png}
    % \includegraphics[scale=0.6]{pics/stateprep/originalT2gate.png}
    \caption{Unprotected $T_2$ Gate: plots visualize the unitary's elements. Brighter colors represent higher absolute values / magnitudes of the matrix entries. (a) is the exact form of the $T_2$ gate with selected $t$ and (b) is the BCH-synthesized form. The state $\ket{j}_q \ket{k}_m$ has index $j \cdot \Lambda + k$ where $\Lambda$ is the cutoff. By observation, the analytically realized form is accurate.}
    \label{fig:unprotectedt2}
\end{figure}
\begin{figure}[!ht]
    \centering
    % \includegraphics[scale=0.6]{pics/stateprep/fullT2gate.png}
    \includegraphics[scale=0.6]{pics/stateprep/T2-protected.png}
    \caption{Protected $T_2$ Gate: (a) is the exact form of the protected $T_2$ gate with selected $t$ and (b) is the BCH-synthesized form.   The state $\ket{j}_q \ket{k}_m$ has index $j \cdot \Lambda + k$ where $\Lambda$ is the cutoff. By observation, the analytically realized form is still accurate, albeit with more incurred Trotter error. 
    %\ck{completed: pasted over labels}
    }
    \label{fig:protectedt2}
\end{figure}

% \includegraphics[scale=0.6]{pics/draft_t2_unprotected.png}

% \includegraphics[scale=0.6]{pics/draft_t2_protected.png}

We also analyze the error scaling as the order of the BCH formulas used increases. \cref{fig:protectedt2-error} describes the error-resource tradeoff as the Trotter step within each BCH formula increases. Observe that the error decays as higher order formulas are used or Trotter step size is reduced, as expected. While modest depths have relatively high infidelity, in theory our compilations can achieve arbitrary accuracy at the level of both entire gates and matrix individual elements. %\ck{fidelity discussion....}
\begin{figure}[!ht]
    \centering
    \mbox{
    \subfigure[]{\includegraphics[scale=0.42]{pics/stateprep/errorfidT2norm.png}}
    \subfigure[]{\includegraphics[scale=0.42]{pics/stateprep/errorT2norm.png}}
    }
    \caption{Error scaling of protected $T_2$ gate with respect to BCH circuit depth according to (a) the state-fidelity metric and (b) the trace norm. Prime denotes that the formula has been symmetrized. %\ck{What does the dashed grey line mean?}
    }
    \label{fig:protectedt2-error}
\end{figure}

\subsection{Hong-Ou-Mandel effect/conditional (controlled-phase) beam splitter gate}\label{subsec:Conditional-(Controlled-Phase)-Beam-Splitter}
% \how{should be rewritten}
The operations we seek to realize need not act on a single \qumode; in fact, our techniques are extensible to hybrid setups with multiple \qumodes~or qubits. Consider the conditional (controlled-phase) beam splitter
\begin{align}
U_{\text{beam split}} & =e^{-i\lambda^{2}\left(\hat{a}_{1}^{\dagger}\hat{a}_{2}+\hat{a}_{1}\hat{a}_{2}^{\dagger}\right)\sigma^{z}}.
\end{align}
This gate naturally pertains
to certain lattice gauge theories \cite{C2QA_LGT} and gives rise to exponential SWAP
(eSWAP) \cite{gao2019entanglement} and controlled-SWAP (cSWAP)
gates for state purification and SWAP tests, when paired with an ordinary (uncontrolled)
beam splitter \cite{pietikainen2022controlled}. 
The argument in phase-space representation 
% Eq.~\ref{eq:CreationOperatortoPhaseSpace} and Eq.~\ref{eq:AnnihilationOperatortoPhaseSpace} 
is 
\begin{align}
-i\lambda^{2}\left(\hat{a}_{1}^{\dagger}\hat{a}_{2}+\hat{a}_{1}\hat{a}_{2}^{\dagger}\right)\sigma^{z} & =-2i\lambda^{2}\left(\hat{x}_{1}\hat{x}_{2}+\hat{p}_{1}\hat{p}_{2}\right)\sigma^{z},
\end{align}
such that the gate is decomposed in terms of a Trotter expansion
% Eq.~\ref{eq:TrotterFirstOrder} 
as the product of two exponential
terms $\exp\left(\left[A_{1},B_{1}\right]\lambda^{2}\right)=\exp\left(-2i\lambda^{2}\hat{x}_{1}\hat{x}_{2}\sigma^{z}\right)$
and $\exp\left(\left[A_{2},B_{2}\right]\lambda^{2}\right)=\exp\left(-2i\lambda^{2}\hat{x}_{1}\hat{x}_{2}\sigma^{z}\right)$.
According to the Pauli commutation relation, 
the first commutator is
\begin{align}
\left[A_{1},B_{1}\right] & =-2i\hat{x}_{1}\hat{x}_{2}\sigma^{z}\\
 & =-2i\hat{x}_{1}\hat{x}_{2}\left(-\frac{i}{2}\left[\sigma^{x},\sigma^{y}\right]\right)\\
 & =\left[i\hat{x}_{1}\sigma^{x},i\hat{x}_{2}\sigma^{y}\right],
\end{align}
and the second is
\begin{align}
\left[A_{2},B_{2}\right] & =\left[i\hat{p}_{1}\sigma^{x},i\hat{p}_{2}\sigma^{y}\right],
\end{align}
with the following error scaling:

% \how{error bounds ...}
\begin{restatable*}{theorem}{resultbeamsplitter}
Assume we may implement $\text{e}^{ i t \hat{x}_m \sigma^j}, \text{e}^{ i t \hat{p}_m \sigma^j}$ for $m \in \{ 1, 2 \}$; i.e., we may implement the qubit-conditional position shifts and momentum boosts on either \qumode~without error. Then, we may approximate $\mathcal{B}_{\hat{x}_1 \hat{x}_2 + \hat{p}_1 \hat{p}_2}$ with arbitrary error scaling $p$ as
\begin{align*}
    \norm{\exp it \widetilde{\mathcal{B}}_{\hat{x}_1 \hat{x}_2 + \hat{p}_1 \hat{p}_2} - \exp it \mathcal{B}_{\hat{x}_1 \hat{x}_2 + \hat{p}_1 \hat{p}_2} } \in \mathcal{O}((Ct)^{p + \frac{1}{2}}),
\end{align*}
where $C = \max( \norm{ \hat{x}_1 \hat{x}_2 + \hat{p}_1 \hat{p}_2}, \norm{\hat{x}_1}^2, \norm{\hat{x}_2}^2, \norm{\hat{p}_1}^2, \norm{\hat{p}_1}^2)$ and using no more than $4 \cdot 5^{ \frac{p}{2} - \frac{1}{4} }$ exponentials.
\end{restatable*}


The two exponential terms are decomposed via the BCH formula 
% Eq.~\ref{eq:BCHFormula}
for a lower-bound gate depth of eight. Results are shown in~\cref{fig:ConditionalBeamSplitter} for
$15$ states per cavity with a shared qubit over a final time of
$\pi/2$ with $200$ equal time
steps, where the system is initially in the first excited state of
each cavity and the ground state of the shared qubit $\left|11g\right>$.
As expected for the conditional beam splitter, the gate exhibits the
Hong-Ou-Mandel effect, in which the occupation of cavity 1 oscillates
between the first excited mode and a superposition of the ground and
the second excited states of the cavity. The BCH-synthesized results
closely agree with that of the original gate, with no visible leakage
beyond the physical states (the lowest three states of the cavity)
into the working space under the time duration studied. As for the
conditional rotation gate, the relative error of the BCH-synthesized
gate computed for a single time step of length $t$ was found to scale according to a power law with the time step,
in accordance with the analytic result for Trotterization and BCH
decomposition.

\begin{figure}[!ht]
\begin{centering}
\includegraphics[width=0.5\columnwidth]{pics/BeamSplitter/probabilityexact}\includegraphics[width=0.5\columnwidth]{pics/BeamSplitter/probabilitybch}
\par\end{centering}
\begin{centering}
\includegraphics[width=0.5\columnwidth]{pics/BeamSplitter/probabilityworkbch}\includegraphics[width=0.5\columnwidth]{pics/BeamSplitter/error}
\par\end{centering}
\caption{Hong-Ou-Mandel effect simulated with (a) exact and (b) BCH-synthesized
conditional beam splitters, illustrated as probability cavity 1 is found in states $\left|0\right>$, $\left|1\right>$, or $\left|2\right>$; (c) probability of leakage into higher cavity modes; and (d) error of the real part of the autocorrelation
after a single application of a BCH-synthesized gate for time step $t$ relative to the application of the exact gate for the same time step.\label{fig:ConditionalBeamSplitter}}%The notation with the single Fock states and two-mode Fock states is not explained.
\end{figure}



% \how{numerics}


{
% \section{Qubit-Conditional Cavity Gates\label{sec:Qubit-Conditional-Cavity-Gates}}

% We begin by employing the analytic ISA to synthesize qubit-conditional
% gates commonly required in bosonic quantum computing applications.

% \subsection{SUM Gate}

% Consider the two-cavity infinitesimal conditional SUM gate
% \begin{equation}
% U_{\text{SUM}}=e^{-i\lambda^{2}\hat{x}_{1}\hat{p}_{2}\sigma^{z}}
% \end{equation}
% where $\hat{x}_{1}$ is the position operator of cavity $1$, $\hat{p}_{2}$
% is the momentum operator of cavity $2$, and the Pauli-Z gate $\sigma^{z}$
% acts on a qubit coupled to both cavities. In Gottesman-Preskill-Knill
% (GKP) codes, the gate enables stabilizer measurements and CNOT operations
% \cite{royer2022encoding} and provides an $x_{1}$-position-dependent
% momentum boost $\hat{p}_{2}$ and equivalently a $p_{2}$-momentum-dependent
% position displacement $\hat{x}_{1}$.

% To decompose the exponential gate, its argument is expressed in terms
% of a commutator via the Pauli commutation relation Eq.~(\ref{eq:PauliCommutatorz})
% as follows:
% \begin{align}
% \left[A,B\right]\lambda^{2} & =-i\hat{x}_{1}\hat{p}_{2}\sigma^{z}\lambda^{2}\\
%  & =-\hat{x}_{1}\hat{p}_{2}\left(-\frac{i}{2}\left[\sigma^{x},\sigma^{y}\right]\right)\lambda^{2}\\
%  & =\left[\frac{i}{\sqrt{2}}\hat{x}_{1}\sigma^{x},\frac{1}{\sqrt{2}}\hat{p}_{2}\sigma^{y}\right]\lambda^{2}
% \end{align}
% \how{phat sigma y is not anti-hermitian}
% The gate is then expressed according to the BCH decomposition Eq.~(\ref{eq:BCHFormula})
% with gate depth lower bound of four.

% \subsection{Conditional Single-Cavity Squeezing Gate}

% The infinitesimal squeezing gate assumes the form

% \begin{equation}
% U_{\text{one-mode squeeze}}\approx e^{\lambda^{2}\left(\hat{a}^{\dagger2}-\hat{a}^{2}\right)\sigma^{z}}
% \end{equation}
% and aids generation of GKP states \cite{hastrup2021measurement,hastrup2021unconditional}.
% Given the relationship between the ladder and phase-space operators
% Eq.~\ref{eq:AnnihilationOperatortoPhaseSpace} and \ref{eq:CreationOperatortoPhaseSpace},
% the argument is expressed as
% \begin{align}
% \lambda^{2}\left(\hat{a}^{\dagger2}-\hat{a}^{2}\right)\sigma^{z} & =-2i\lambda^{2}\left\{ \hat{x},\hat{p}\right\} \sigma^{z}
% \end{align}
% which is in turn expressed as a commutator according to the Pauli
% anticommutator-commutator relation Eq.~\ref{eq:AnticommutatortoCommutator}
% \begin{align}
% \left\{ \hat{x},\hat{p}\right\} \sigma^{z}\lambda^{2} & =2i\left[i\hat{x}\sigma^{x},i\hat{p}\sigma^{y}\right]\lambda^{2}
% \end{align}
% to yield the exponential commutator
% \begin{equation}
% \left[A,B\right]\lambda^{2}=\left[\sqrt{2}i\hat{x}\sigma^{x},\sqrt{2}i\hat{p}\sigma^{y}\right]\lambda^{2}
% \end{equation}
% The infinitesimal squeezing gate is therefore decomposed according
% to the BCH formula Eq.~\ref{eq:BCHFormula} with lower-bound gate
% depth of four.

% \subsection{Conditional (Controlled-Phase) Beam Splitter Gate\label{subsec:Conditional-(Controlled-Phase)-Beam-Splitter}}

% Consider the conditional (controlled-phase) beam splitter
% \begin{align}
% U_{\text{beam split}} & =e^{-i\lambda^{2}\left(\hat{a}_{1}^{\dagger}\hat{a}_{2}+\hat{a}_{1}\hat{a}_{2}^{\dagger}\right)\sigma^{z}}
% \end{align}
% The argument in phase-space representation Eq.~\ref{eq:CreationOperatortoPhaseSpace}
% and Eq.~\ref{eq:AnnihilationOperatortoPhaseSpace} is 
% \begin{align}
% -i\lambda^{2}\left(\hat{a}_{1}^{\dagger}\hat{a}_{2}+\hat{a}_{1}\hat{a}_{2}^{\dagger}\right)\sigma^{z} & =-2i\lambda^{2}\left(\hat{x}_{1}\hat{x}_{2}+\hat{p}_{1}\hat{p}_{2}\right)\sigma^{z}
% \end{align}
% such that the gate is decomposed in terms of a Trotter expansion
% Eq.~\ref{eq:TrotterFirstOrder} as the product of two exponential
% terms $\exp\left(\left[A_{1},B_{1}\right]\lambda^{2}\right)=\exp\left(-2i\lambda^{2}\hat{x}_{1}\hat{x}_{2}\sigma^{z}\right)$
% and $\exp\left(\left[A_{2},B_{2}\right]\lambda^{2}\right)=\exp\left(-2i\lambda^{2}\hat{x}_{1}\hat{x}_{2}\sigma^{z}\right)$.
% According to the Pauli commutation relation Eq.~(\ref{eq:PauliCommutatorz}),
% the first commutator is
% \begin{align}
% \left[A_{1},B_{1}\right] & =-2i\hat{x}_{1}\hat{x}_{2}\sigma^{z}\\
%  & =-2i\hat{x}_{1}\hat{x}_{2}\left(-\frac{i}{2}\left[\sigma^{x},\sigma^{y}\right]\right)\\
%  & =\left[i\hat{x}_{1}\sigma^{x},i\hat{x}_{2}\sigma^{y}\right]
% \end{align}
% and the second is
% \begin{align}
% \left[A_{2},B_{2}\right] & =\left[i\hat{p}_{1}\sigma^{x},i\hat{p}_{2}\sigma^{y}\right]
% \end{align}
% The two exponential terms are in decomposed via the BCH formula Eq.~\ref{eq:BCHFormula}
% for a lower-bound gate depth of eight.

% \subsection{Conditional Beam Squeezer Gate}

% The infinitesimal conditional beam squeezer
% \begin{equation}
% U_{\text{two-mode squeeze}}=e^{-i\lambda^{2}\left(\hat{a}_{1}^{\dagger}\hat{a}_{2}^{\dagger}+\hat{a}_{1}\hat{a}_{2}\right)\sigma^{z}}
% \end{equation}
% follows analogously from the conditional beam splitter. The argument
% in phase-space representation is
% \begin{align*}
% -i\lambda^{2}\left(\hat{a}_{1}^{\dagger}\hat{a}_{2}^{\dagger}+\hat{a}_{1}\hat{a}_{2}\right)\sigma^{z} & =-2i\lambda^{2}\left(\hat{x}_{1}\hat{x}_{2}-\hat{p}_{1}\hat{p}_{2}\right)\sigma^{z}
% \end{align*}
% such that the gate is the Trotter decomposed Eq.~\ref{eq:TrotterFirstOrder}
% product of exponential terms $\exp\left(\left[A_{1},B_{1}\right]\lambda^{2}\right)=\exp\left(-2i\lambda^{2}\hat{x}_{1}\hat{x}_{2}\sigma^{z}\right)$
% and $\exp\left(\left[A_{2},B_{2}\right]\lambda^{2}\right)=\exp\left(2i\lambda^{2}\hat{p}_{1}\hat{p}_{2}\sigma^{z}\right)$.
% The Pauli commutation relation Eq.~\ref{eq:PauliCommutatorz}, yields
% \begin{align}
% \left[A_{1},B_{1}\right] & =\left[i\hat{x}_{1}\sigma^{x},i\hat{x}_{2}\sigma^{y}\right]\\
% \left[A_{2},B_{2}\right] & =\left[\hat{p}_{1}\sigma^{x},\hat{p}_{2}\sigma^{y}\right]
% \end{align}
% such that the BCH formula Eq.~\ref{eq:BCHFormula} of the two exponential
% terms yields the infinitesimal conditional beam squeezer with a lower
% bound gate depth of eight.

% \subsection{Conditional Rotations\label{subsec:Conditional-Rotations}}

% We employ the displacement analytic ISA to synthesize the infinitesimal
% conditional rotation gate
% \begin{equation}
% U_{\text{rot},k}=e^{i\lambda^{2}\hat{n}\sigma^{k}}
% \end{equation}
% for $k=x,y,z$, which naturally belongs to the analytic displacement
% and rotation ISA. Expression of the exponential argument in terms
% of phase-space operators according to Eq.~\ref{eq:NumbertoPhaseSpace},
% Eq.~\ref{eq:CreationOperatortoPhaseSpace}, and Eq.~\ref{eq:AnnihilationOperatortoPhaseSpace}
% yields
% \begin{equation}
% i\lambda^{2}\hat{n}\sigma^{k}=i\left(\hat{x}^{2}+\hat{p}^{2}-\frac{1}{2}\right)\sigma^{k}\lambda^{2}
% \end{equation}
% such that the gate is expressed via the Trotter decomposition
% Eq.~\ref{eq:TrotterFirstOrder}as the product of $\exp\left(\left[A_{1},B_{1}\right]\right)=\exp\left(i\lambda^{2}\hat{x}^{2}\sigma^{k}\right)$,
% $\exp\left(\left[A_{2},B_{2}\right]\right)=\exp\left(i\lambda^{2}\hat{p}^{2}\sigma^{k}\right)$,
% and conditional displacement $\exp\left(-i\lambda^{2}\sigma^{k}/2\right)$.
% Given the Pauli commutator relation Eq.~\ref{eq:BCHFormula}, the
% first commutator is 
% \begin{align}
% \left[A_{1},B_{1}\right] & =i\hat{x}^{2}\sigma^{k}\\
%  & =i\hat{x}^{2}\left(-\frac{i}{2}\left[\sigma^{i},\sigma^{j}\right]\right)\\
%  & =\left[\frac{1}{\sqrt{2}}\hat{x}\sigma^{i},\frac{1}{\sqrt{2}}\hat{x}\sigma^{j}\right]
% \end{align}
% and the second commutator is 
% \begin{align}
% \left[A_{2},B_{2}\right] & =i\hat{p}^{2}\sigma^{k}\\
%  & =i\hat{p}^{2}\left(-\frac{i}{2}\left[\sigma^{i},\sigma^{j}\right]\right)\\
%  & =\left[\frac{1}{\sqrt{2}}\hat{p}\sigma^{i},\frac{1}{\sqrt{2}}\hat{p}\sigma^{j}\right]
% \end{align}
% such that both terms are amenable to BCH decomposition Eq.~\ref{eq:BCHFormula}
% and the infinitesimal conditional rotation is composed in the displacement
% ISA with gate depth lower-bound of  nine.

% \section{Universal Control of the Span $\left\{ \left|0\right\rangle ,\left|1\right\rangle \right\} $
% Fock Space\label{subsec:Universal-Control}}

% To demonstrate the efficacy of the analytic ISA, we demonstrate the
% use of the approach to encode a qubit in a cavity either via generation
% of effective Pauli gates in Section~\ref{subsec:Effective-Pauli-Gate}
% or imposition of an effective Hubbard interaction in the Jaynes-Cumming
% Hamiltonian in Section~\ref{subsec:Effective-Hubbard-Lattice}.

% \subsection{Effective Pauli Gate Approach\label{subsec:Effective-Pauli-Gate}}

% For universal control is the restricted $\text{span}\left\{ \left|0\right\rangle ,\left|1\right\rangle \right\} $
% Hilbert space, we generate three effective Pauli operators $\sigma_{\text{eff}}^{x}$,
% $\sigma_{\text{eff}}^{y}$, and $\sigma_{\text{eff}}^{z}$ that produce
% Pauli rotations in the lowest two modes of the cavity minimal leakage
% to higher energy states. The form of the effective Pauli operators
% is determined by expressing the standard Pauli operators
% \begin{align}
% \sigma^{x} & =\left(\begin{array}{cc}
% 0 & 1\\
% 1 & 0
% \end{array}\right)\\
% \sigma^{y} & =\left(\begin{array}{cc}
% 0 & -i\\
% i & 0
% \end{array}\right)\\
% \sigma^{z} & =\left(\begin{array}{cc}
% 1 & 0\\
% 0 & -1
% \end{array}\right)
% \end{align}
% in terms of creation and annihilation operators truncated to the first
% two Fock states
% \begin{align}
% \hat{a}_{\text{eff}}^{\dagger} & =\left(\begin{array}{cc}
% 0 & 0\\
% 1 & 0
% \end{array}\right)\\
% \hat{a}_{\text{eff}} & =\left(\begin{array}{cc}
% 0 & 1\\
% 0 & 0
% \end{array}\right)\\
% \hat{n}_{\text{eff}} & =\hat{a}^{\dagger}\hat{a}_{\text{eff}}=\left(\begin{array}{cc}
% 0 & 0\\
% 0 & 1
% \end{array}\right)
% \end{align}
% which yields
% \begin{align}
% \sigma_{\text{eff}}^{x} & =\hat{a}_{\text{eff}}^{\dagger}+\hat{a}_{\text{eff}}\\
% \sigma_{\text{eff}}^{y} & =i\left(\hat{a}_{\text{eff}}^{\dagger}-\hat{a}_{\text{eff}}\right)\\
% \sigma_{\text{eff}}^{z} & =I-2\hat{a}_{\text{eff}}^{\dagger}\hat{a}_{\text{eff}}
% \end{align}
% To reduce leakage into higher energy states, we ensure the creation
% operator $\hat{a}_{\text{eff}}^{\dagger}$ only acts on the ground
% state $\left|0\right>$ and the annihilation operator $\hat{a}_{\text{eff}}$
% only acts on the first excited state $\left|1\right>$ with the projector
% \begin{align}
% \hat{P}_{0} & \approx I-\hat{n}\\
%  & =\begin{cases}
% 0 & n=1\\
% 1 & n=0
% \end{cases}
% \end{align}
% where $n$ is the number of photons in the cavity and where only the
% $\text{span}\left\{ \left|0\right\rangle ,\left|1\right\rangle \right\} $
% states are populated. Since the operator is a projector, it obeys
% the relation
% \begin{equation}
% \hat{P}_{0}^{2}=\hat{P}_{0}
% \end{equation}
% such that the effective Pauli gates are
% \begin{align}
% \sigma_{\text{eff}}^{x} & =\hat{a}_{\text{eff}}^{\dagger}\hat{P}_{0}+\hat{P}_{0}\hat{a}_{\text{eff}}\\
%  & \approx\hat{a}^{\dagger}\left(I-\hat{n}\right)+\left(I-\hat{n}\right)\hat{a}\\
% \sigma_{\text{eff}}^{y} & =i\left(\hat{a}_{\text{eff}}^{\dagger}\hat{P}_{0}-\hat{P}_{0}\hat{a}_{\text{eff}}\right)\\
%  & \approx i\left(\hat{a}^{\dagger}\left(I-\hat{n}\right)-\left(I-\hat{n}\right)\hat{a}\right)\\
% \sigma_{\text{eff}}^{z} & =I-2\hat{a}_{\text{eff}}^{\dagger}\hat{P}_{0}^{2}\hat{a}_{\text{eff}}\\
%  & \approx I-2\hat{a}^{\dagger}\left(I-\hat{n}\right)\hat{a}
% \end{align}


% \subsubsection{Pauli X Gate}

% Consider the infinitesimal $\sigma_{x}$-rotation gate in the $\text{span}\left\{ \left|0\right\rangle ,\left|1\right\rangle \right\} $
% Fock space 
% \begin{align}
% U_{\text{span}\left\{ 0,1\right\} ,x} & =e^{i\lambda^{2}\sigma_{\text{eff}}^{x}\sigma^{z}}\\
%  & =e^{i\lambda^{2}\left(\hat{a}^{\dagger}\left(1-\hat{n}\right)+\left(1-\hat{n}\right)\hat{a}\right)\sigma^{z}}
% \end{align}
% Expression of the exponent in terms of phase-space operators Eq.~\ref{eq:NumbertoPhaseSpace},
% Eq.~\ref{eq:CreationOperatortoPhaseSpace}, and Eq.~\ref{eq:AnnihilationOperatortoPhaseSpace}
% gives 
% \begin{gather}
% i\lambda^{2}\left(\hat{a}^{\dagger}\left(I-\hat{n}\right)+\left(I-\hat{n}\right)\hat{a}\right)\sigma^{z}\nonumber \\
% =i\lambda^{2}\left(2\hat{x}-\left\{ \hat{x},\hat{n}\right\} +i\left[\hat{p},\hat{n}\right]\right)\sigma^{z}
% \end{gather}
% The gate is therefore given by a Trotter decomposition Eq.~\ref{eq:TrotterFirstOrder}
% of three terms: $\exp\left(\left[A_{1},B_{1}\right]\lambda^{2}\right)=\exp\left(-i\lambda^{2}\left\{ \hat{x},\hat{n}\right\} \sigma^{z}\right)$,
% $\exp\left(\left[A_{2},B_{2}\right]\lambda^{2}\right)=\exp\left(-\lambda^{2}\left[\hat{p},\hat{n}\right]\sigma^{z}\right)$,
% and $\exp\left(2i\lambda^{2}\hat{x}\sigma^{z}\right)$. 

% The first and second exponential terms are decomposed according to
% the BCH decomposition Eq.~(\ref{eq:BCHFormula}). The first commutator
% is given by the Pauli anticommutation-commutation relation Eq.~(\ref{eq:AnticommutatortoCommutator})
% \begin{align}
% -i\left\{ \hat{x},\hat{n}\right\} \sigma^{z} & =-i\left(i\left[i\hat{x}\sigma^{x},i\hat{n}\sigma^{y}\right]\right)\\
%  & =\left[i\hat{x}\sigma^{x},i\hat{n}\sigma^{y}\right]\\
%  & =\left[A_{1},B_{1}\right]
% \end{align}
% $A_{1}$ corresponds to a position displacement and $B_{1}$
% corresponds to the $y$-conditional rotation gate. The argument of
% the second term is already in the form of a commutator, such that
% \begin{align}
% \left[A_{2},B_{2}\right] & =-\left[\hat{p},\hat{n}\right]\sigma^{z}\\
%  & =\left[i\hat{p},i\hat{n}\sigma^{z}\right]
% \end{align}
% $A_{2}$ corresponds to an \emph{unconditional} momentum boost,
% and $B_{2}$ corresponds to the $z$-conditional rotation gate.
% Lastly, the third term already belongs to the instruction set architecture
% and needs no further decomposition. 

% The infinitesimal $\sigma_{x}$-rotation gate in the $\text{span}\left\{ \left|0\right\rangle ,\left|1\right\rangle \right\} $
% Fock space is therefore composed of a product of nine gates in the
% displacement and rotation gate ISA and 21 gates in the displacement
% ISA.

% \subsubsection{Pauli Y Gate}

% The infinitesimal $\sigma_{y}$-rotation gate in the $\text{span}\left\{ \left|0\right\rangle ,\left|1\right\rangle \right\} $
% Fock space is determined analogously 
% \begin{align}
% U_{\text{span}\left\{ 0,1\right\} ,y} & =e^{i\lambda^{2}\sigma_{\text{eff}}^{y}\sigma^{z}}\\
%  & =e^{-\lambda^{2}\left(\hat{a}^{\dagger}\left(I-\hat{n}\right)+\left(I-\hat{n}\right)\hat{a}\right)\sigma^{z}}
% \end{align}
% Expression of the argument of the exponent in terms of phase-space
% variables Eq.~\ref{eq:CreationOperatortoPhaseSpace} and Eq.~\ref{eq:AnnihilationOperatortoPhaseSpace}
% yields 
% \begin{gather}
% -\lambda^{2}\left(\hat{a}^{\dagger}\left(I-\hat{n}\right)-\left(I-\hat{n}\right)\hat{a}\right)\sigma^{z}\nonumber \\
% =-\lambda^{2}\left(-2i\hat{p}+\left[\hat{n},\hat{x}\right]+i\left\{ \hat{n},\hat{p}\right\} \right)\sigma^{z}
% \end{gather}
% such that the gate is a Trotter decomposition Eq.~\ref{eq:TrotterFirstOrder}
% of $\exp\left(\left[A_{1},B_{1}\right]\lambda^{2}\right)=\exp\left(-\lambda^{2}\left[n,x\right]\sigma^{z}\right)$,
% $\exp\left(\left[A_{2},B_{2}\lambda^{2}\right]\right)=\exp\left(-i\lambda^{2}\left\{ p,n\right\} \sigma^{z}\right)$,
% and $\exp\left(2i\lambda^{2}p\sigma^{z}\right)$. 

% Again, the first two exponential terms are decomposed via the BCH
% formula Eq.~\ref{eq:BCHFormula}. The first commutator is 
% \begin{align}
% \left[A_{1},B_{1}\right] & =\left[\hat{n},\hat{x}\right]\sigma^{z}\\
%  & =\left[\hat{n}\sigma^{z},\hat{x}\right]
% \end{align}
% where the exponent of $A_{1}$ is a $z$-conditional rotation
% gate and the exponent of $B_{1}$ is an unconditional position
% displacement. The second commutator is given by the Pauli anticommutation-commutation
% relation Eq.~(\ref{eq:AnticommutatortoCommutator}) 
% \begin{align}
% i\left\{ p,n\right\} \sigma^{z} & =i\left(i\left[ip\sigma^{x},in\sigma^{y}\right]\right)\\
%  & =\left[ip\sigma^{x},in\sigma^{y}\right]\\
%  & =\left[A_{2},B_{2}\right]
% \end{align}
% where the exponent of $A_{2}$ corresponds to a conditional
% momentum shift and the exponent of $B_{2}$ is a $y$-conditional
% rotation gate.

% The infinitesimal $\sigma_{y}$-rotation gate in the $\text{span}\left\{ \left|0\right\rangle ,\left|1\right\rangle \right\} $
% Fock space therefore has a lower-bound gate depth of nine in the displacements-only
% analytic ISA and 21 in the displacement and rotation analytic ISA.

% \subsubsection{Pauli Z Gate}

% The infinitesimal $\sigma_{z}$-rotation gate in the $\text{span}\left\{ \left|0\right\rangle ,\left|1\right\rangle \right\} $
% Fock space is 
% \begin{align}
% U_{\text{span}\left\{ 0,1\right\} ,z} & =e^{i\lambda^{2}\sigma_{\text{eff}}^{z}\sigma^{z}}\\
%  & =e^{-\lambda^{2}\left(I-2\hat{a}^{\dagger}\left(I-\hat{n}\right)\hat{a}\right)\sigma^{z}}
% \end{align}
% whose argument in terms of ladder operators is
% \begin{gather}
% -\lambda^{2}\left(I-2\hat{a}^{\dagger}\left(I-\hat{n}\right)\hat{a}\right)\sigma^{z}\nonumber \\
% =-\lambda^{2}\left(I-2\hat{a}^{\dagger}a+2\hat{a}^{\dagger}\hat{a}^{\dagger}\hat{a}\hat{a}\right)\sigma^{z}
% \end{gather}
% Given the ladder operator commutator Eq.~\ref{eq:CommutatorCreationAnnihilation},
% \begin{equation}
% \hat{a}^{\dagger}\hat{a}=\hat{a}\hat{a}^{\dagger}-I
% \end{equation}
% the relationship between the fourth-order ladder operator term and
% the number operator is
% \begin{align}
% \hat{a}^{\dagger}\hat{a}^{\dagger}\hat{a}\hat{a} & =\hat{a}^{\dagger}\left(\hat{a}\hat{a}^{\dagger}-I\right)\hat{a}\\
%  & =\hat{a}^{\dagger}\hat{a}\hat{a}^{\dagger}\hat{a}-\hat{a}^{\dagger}\hat{a}\\
%  & =\hat{n}^{2}-\hat{n}
% \end{align}
% The argument of the exponential in terms of number operators is then
% \begin{multline}
% -\lambda^{2}\left(I-2\hat{a}^{\dagger}\left(I-\hat{n}\right)\hat{a}\right)\sigma^{z}\\
% =-\lambda^{2}\left(I-4\hat{n}+2\hat{n}^{2}\right)\sigma^{z}
% \end{multline}
% The argument is further simplified given that the state is restricted
% to the first two cavity modes, as for $n=0$ and $n=1$ the quantity
% $\hat{n}^{2}-\hat{n}$ is zero, as follows:
% \begin{gather}
% -\lambda^{2}\left(I-4\hat{n}+2\hat{n}^{2}\right)\sigma^{z}\nonumber \\
% =-\lambda^{2}\left(I-2n\right)\sigma^{z}
% \end{gather}
% The gate is therefore directly synthesized as the product of the qubit
% rotation gate $\exp\left(-\lambda^{2}\sigma^{z}\right)$ and the $z$-conditional
% rotation gate $\exp\left(2\lambda^{2}\hat{n}\sigma^{z}\right)$ for
% a lower-bound gate depth of two.

% \subsection{Effective Hubbard Lattice Interaction Approach\label{subsec:Effective-Hubbard-Lattice}}

% An alternative scheme to encode a qubit in a cavity in the analytic
% ISA scheme is to map the three-dimensional quantum electrodynamics
% (3D cQED) system to a qubit by imposing an $\hat{n}\left(\hat{n}-1\right)$
% anharmonicity into the Jaynes-Cummings Hamiltonian that describes
% the system. The anharmonicity term increases the energy gap between
% higher levels of the oscillator to effectively restrict propagation
% to the $\text{span}\left\{ \left|0\right\rangle ,\left|1\right\rangle \right\} $
% Fock space for universal control.

% Consider the standard Jaynes-Cummings Hamiltonian 
% \begin{equation}
% \hat{H}_{\text{JC}}=\omega_{R}\hat{a}^{\dagger}\hat{a}+\frac{\omega_{Q}}{2}\sigma^{z}+g\left(\hat{a}\sigma^{+}+\hat{a}^{\dagger}\sigma^{-}\right)
% \end{equation}
% where $\omega_{R}$ is the cavity frequency, $\omega_{Q}$ is the
% qubit frequency, and $g$ is the coupling parameter. Inclusion of
% the simulated $\hat{n}\left(\hat{n}-1\right)$ anharmonicity of strength
% $\Gamma$ yields 
% \begin{equation}
% \hat{H}_{\text{an}}=\omega_{R}\hat{a}^{\dagger}\hat{a}+\Gamma\hat{n}(\hat{n}-1)+\frac{\omega_{Q}}{2}\sigma^{z}+g(\hat{a}\sigma^{+}+\hat{a}^{\dagger}\sigma^{-})
% \end{equation}
% and system is switched between states $\left|0\right\rangle $ and
% $\left|1\right\rangle $ with a weak time-$t$-dependent drive of
% strength $\Omega$ at the resonance frequency $\omega_{R}$, as follows:
% \begin{equation}
% \hat{H}_{\text{drive}}\left(t\right)=\Omega e^{i\omega_{R}t}\hat{a}^{\dagger}+\Omega^{\star}e^{-i\omega_{R}t}\hat{a}
% \end{equation}
% Synthesis of a propagator of the form $\exp\left(i\lambda^{2}\hat{n}\left(\hat{n}-1\right)\right)$
% is then sufficient to employ the native 3D cQED system as a qubit.
% Note the choice of $\lambda$ for practical implementation must take
% into account both the time step and the fact the BCH decomposition
% yields a square root in the exponential argument. The required propagator
% is a Trotter decomposition Eq.~(\ref{eq:TrotterFirstOrder})
% of $\exp\left(\left[A,B\right]\lambda^{2}\right)=\exp(i\lambda^{2}\hat{n}^{2}\sigma^{z})$
% and $\exp(-i\lambda^{2}\hat{n}\sigma^{z})$. 

% The first term is synthesized according to the BCH formula Eq.~(\ref{eq:BCHFormula})
% with a commutator determined by the Pauli commutation relation Eq.~(\ref{eq:PauliCommutatorz})
% as follows:
% \begin{align}
% \left[A,B\right] & =i\hat{n}^{2}\sigma^{z}\\
%  & =i\hat{n}^{2}\left(-\frac{i}{2}\left[\sigma^{x},\sigma^{y}\right]\right)\\
%  & =\left[\frac{1}{\sqrt{2}}\hat{n}\sigma^{x},\frac{1}{\sqrt{2}}\hat{n}\sigma^{y}\right]
% \end{align}
% where $A$ and $B$ correspond to $x$-conditional and
% $y$-conditional rotations, respectively. The second term is a $z$-conditional
% rotation gate.

% The resulting anharmonicity gate therefore has a gate depth of lower
% bound five in the displacement and rotation analytic ISA and 45 in
% the displacement ISA.

% \section{Fermi-Hubbard Lattice Dynamics\label{sec:Fermi-Hubbard-Lattice-Dynamics}}

% To further demonstrate the power of the analytic ISA, we employ the
% approach to simulate fermionic dynamics on bosonic 3D cQED systems.

% We consider the Fermi-Hubbard lattice Hamiltonian

% \begin{align}
% \hat{H}_{\text{FH}} & =\hat{T}_{\text{FH}}+\hat{V}_{\text{FH}}\label{eq:FermiHubbardHamiltonian}\\
% \hat{T}_{\text{FH}} & =-J\sum_{i,\sigma}\hat{c}_{i,\sigma}^{\dagger}\hat{c}_{i+1,\sigma}+\hat{c}_{i+1,\sigma}^{\dagger}\hat{c}_{i,\sigma}\\
% \hat{V}_{\text{FH}} & =U\sum_{i}\hat{n}_{i,\uparrow}\hat{n}_{i,\downarrow}
% \end{align}
% The kinetic energy term $\hat{T}_{\text{FH}}$ describes the nearest-neighbor
% interaction for hopping of a single spin between two sites with hopping
% parameter $J$ and spin $\sigma$ given annihilation operators $\left\{ \hat{c}_{j,\sigma}\right\} $
% and creation operators $\left\{ \hat{c}_{j,\sigma}^{\dagger}\right\} $
% for sites $\left\{ j\right\} $. The potential energy term $\hat{V}_{\text{FH}}$
% describes the same-site interaction, which gives the energetic unfavorability
% of a spin up $\uparrow$ and spin down $\downarrow$ coexisting on
% the same site $i$, where $\hat{n}_{j,\sigma}$ gives the number of
% spin $\sigma$ particles on site $j$. According to fermion statistics,
% no more than a single particle of a given spin can exist on a single
% site.

% Each cavity of the 3D cQED system represents either a spin up or spin
% down particle on a single lattice site, for direct comparison to the
% qubit-based schemes of refs.~\cite{Kivlichan.2018.110501,arute2020observation,Cade.2020.235122}.
% Each cavity is connected to the cavity that represents the same site
% of opposite spin to facilitate computation of the potential energy
% $\hat{V}_{\text{FH}}$ and cavities of the same spin on neighboring
% sites to facilitate computation of the kinetic energy $\hat{T}_{\text{FH}}$.
% Cavities are also connected along Jordan-Wigner strings to take into
% account fermionic statistics. 

% The $\text{\ensuremath{\left|0\right\rangle }}$ cavity state represents
% absence of a spin and the $\text{\ensuremath{\left|1\right\rangle }}$
% state represents presence of a spin. Within each cavity, only the
% states in $\text{span}\left\{ \text{\ensuremath{\left|0\right\rangle }},\text{\ensuremath{\left|1\right\rangle }}\right\} $
% are considered as in Section~(\ref{subsec:Universal-Control}), which
% prevents leakage into unphysical high-energy cavity states. At the
% end of each operation, the cavity state must be in either the $\text{\ensuremath{\left|0\right\rangle }}$
% or $\text{\ensuremath{\left|1\right\rangle }}$ state and the qubit
% state must also be in the ground state $\text{\ensuremath{\left|g\right\rangle }}$,
% which provides an error syndrome and therefore a degree of error correction
% not employed in qubit-based representations of the Fermi-Hubbard lattice.

% Propagation of any combination of up spins and down spins is simulated
% with three two-cavity gates. The first two gates -- the same-site
% and hopping gates -- are defined as the propagator of the same-site
% and hopping Hamiltonians, respectively. The same-site term of the
% Hamiltonian for site $i$ is 
% \begin{equation}
% \hat{H}_{\text{same}}=U\hat{n}_{i,\uparrow}\hat{n}_{i,\downarrow}
% \end{equation}
% This term is zero if only one spin is on a site and $U$ if both spins
% are on the same site, which gives the diagonal Hamiltonian in the
% reduced $4\times4$ Hilbert space
% \begin{equation}
% \hat{H}_{\text{same}}=\left[\begin{array}{cccc}
% 0 & 0 & 0 & 0\\
% 0 & 0 & 0 & 0\\
% 0 & 0 & 0 & 0\\
% 0 & 0 & 0 & U
% \end{array}\right]
% \end{equation}
% and the diagonal propagator $U_{\text{same}}=\text{e}^{-\text{i}\hat{H}_{\text{same}}\tau}$
% \begin{equation}
% U_{\text{same}}=\left[\begin{array}{cccc}
% 1 & 0 & 0 & 0\\
% 0 & 1 & 0 & 0\\
% 0 & 0 & 1 & 0\\
% 0 & 0 & 0 & \text{e}^{-\text{i}U\tau}
% \end{array}\right]
% \end{equation}
% This gate is recognized as the conditional cross-Kerr interaction
% of 3D cQED systems and equivalently a controlled-phase (CPHASE) gate
% in the reduced subspace $\text{span}\left\{ \left|0\right\rangle ,\left|1\right\rangle \right\} $.
% The hopping term of the Hamiltonian for each $\sigma$ spin in sites
% $i,\left(i+1\right)$ is 
% \begin{align}
% H_{\text{hop}} & =-J\left(\hat{c}_{i,\sigma}^{\dagger}\hat{c}_{i+1,\sigma}+\hat{c}_{i+1,\sigma}^{\dagger}\hat{c}_{i,\sigma}\right)\\
%  & =-J\left(\hat{c}_{i,\sigma}^{\dagger}\hat{c}_{i+1,\sigma}-\hat{c}_{i,\sigma}\hat{c}_{i+1,\sigma}^{\dagger}\right)
% \end{align}
% where the latter expression employs the commutator relationship of
% the annihilation and creation operators. The hopping Hamiltonian for
% the specified mapping is then the off-diagonal matrix 
% \begin{equation}
% H_{\text{hop}}=\left[\begin{array}{cccc}
% 0 & 0 & 0 & 0\\
% 0 & 0 & -t & 0\\
% 0 & -t & 0 & 0\\
% 0 & 0 & 0 & 0
% \end{array}\right]
% \end{equation}
% which gives the hopping propagator $U_{\text{hop}}=\text{e}^{-\text{i}H_{\text{hop}}\tau}$
% \begin{align}
% U_{\text{hop}} & =\left[\begin{array}{cccc}
% 1 & 0 & 0 & 0\\
% 0 & \cos\left(t\tau\right) & i\sin\left(t\tau\right) & 0\\
% 0 & i\sin\left(t\tau\right) & \cos\left(t\tau\right) & 0\\
% 0 & 0 & 0 & 1
% \end{array}\right]
% \end{align}
% which is recognized as a conditional controlled-phase beam splitter
% restricted to $\text{span}\left\{ \left|0\right\rangle ,\left|1\right\rangle \right\} $
% in bosonic systems and a Givens or iSWAP-like gate in the reduced
% $\text{span}\left\{ \left|0\right\rangle ,\left|1\right\rangle \right\} $
% subspace \cite{Cade.2020.235122,arute2020observation}. The final
% gate of the three-gate set incorporates the fermionic statistics of
% the spins via the fermionic SWAP (FSWAP) gate FSWAP gate \cite{Kivlichan.2018.110501,Cade.2020.235122}.
% The content of each cavity is swapped with one of its neighbors with
% inclusion of a phase where both spins are present in neighboring cavities
% as follows
% \begin{equation}
% U_{\text{FSWAP}}=\left[\begin{array}{cccc}
% 1 & 0 & 0 & 0\\
% 0 & 0 & 1 & 0\\
% 0 & 1 & 0 & 0\\
% 0 & 0 & 0 & -1
% \end{array}\right]
% \end{equation}
% which is recognized as the product of a conditional rotation gate
% and a beam-splitter on 3D cQED systems.

% Finally, initial states are prepared by the universal set of gates
% in $\text{span}\left\{ \left|0\right\rangle ,\left|1\right\rangle \right\} $
% detailed in Section~\ref{subsec:Universal-Control}.

% \subsection{Conditional Cross-Kerr (CPHASE) Gate}

% We consider the infinitesimal conditional cross-Kerr gate
% \begin{equation}
% U_{\text{cross-Kerr}}=e^{i\lambda^{2}\hat{n}_{1}\hat{n}_{2}\sigma_{z}}
% \end{equation}
% which is also employed in GKP codes encoded in 3D cQED systems \cite{royer2022encoding}. 

% The argument is expressed in terms of a commutator according to the
% Pauli commutation relation Eq.~\ref{eq:PauliCommutatorz}, as follows:
% \begin{align}
% \left[A,B\right] & \lambda^{2}=i\lambda^{2}\hat{n}_{1}\hat{n}_{2}\sigma_{z}\\
%  & =i\lambda^{2}\hat{n}_{1}\hat{n}_{2}\left(-\frac{i}{2}\left[\sigma^{x},\sigma^{y}\right]\right)\\
%  & =\left[\frac{1}{\sqrt{2}}\hat{n}_{1}\sigma^{x},\frac{1}{\sqrt{2}}\hat{n}_{2}\sigma^{y}\right]\lambda^{2}
% \end{align}
% where $A$ corresponds to an $x$-conditional rotation gate
% and $B$ corresponds to a $y$-conditional rotation gate.

% The resulting gate features a lower-bound gate depth of four in the
% displacement and rotation analytic ISA and 16 in the displacement
% ISA.

% \subsection{$\text{Span}\left\{ \left|0\right\rangle ,\left|1\right\rangle \right\} $
% Conditional Beam Splitter Gate}

% In order to generate a $\text{span}\left\{ \left|0\right\rangle ,\left|1\right\rangle \right\} $
% that operates only when $\hat{n}_{1}\hat{n}_{2}\ne1$ (\emph{i.e.},
% $1-\hat{n}_{1}\hat{n}_{2}=0$), we formulate the infinitesimal conditional
% (controlled-phase) beam-splitter gate
% \begin{equation}
% U_{\text{cond. beam}}=e^{-i\lambda^{2}\left(\hat{a}_{1}^{\dagger}\hat{a}_{2}+\hat{a}_{1}\hat{a}_{2}^{\dagger}\right)\left(1-\hat{n}_{1}\hat{n}_{2}\right)\sigma^{z}}
% \end{equation}
% which is decomposed via the Trotter decomposition Eq.~\ref{eq:TrotterFirstOrder}
% in terms of $\exp\left(-i\lambda^{2}\left(\hat{a}_{1}^{\dagger}\hat{a}_{2}+\hat{a}_{1}\hat{a}_{2}^{\dagger}\right)\sigma^{z}\right)$
% and $\exp\left(i\lambda^{2}\left(\hat{a}_{1}^{\dagger}\hat{a}_{2}+\hat{a}_{1}\hat{a}_{2}^{\dagger}\right)\left(\hat{n}_{1}\hat{n}_{2}\right)\sigma^{z}\right)$.
% The first term is the conditional beam splitter $U_{\text{beam split.}}$
% Eq.~\ref{subsec:Conditional-(Controlled-Phase)-Beam-Splitter} and
% the second term is decomposed via BCH Eq.~\ref{eq:BCHFormula} as
% follows:

% Given the expression of the number operator in terms of the phase-space
% operators Eq.~\ref{eq:NumbertoPhaseSpace}, the argument of the second
% exponential operator is 
% \begin{gather}
% i\lambda^{2}\left(\hat{a}_{1}^{\dagger}\hat{a}_{2}+\hat{a}_{1}\hat{a}_{2}^{\dagger}\right)\hat{n}_{1}\hat{n}_{2}\sigma^{z}\nonumber \\
% =i\lambda^{2}\left(2\left(\hat{x}_{1}\hat{x}_{2}+\hat{p}_{1}\hat{p}_{2}\right)\right)\hat{n}_{1}\hat{n}_{2}\sigma^{z}
% \end{gather}
% The term is then expressed as a Trotter decomposition of $\exp\left(\left[A_{1},B_{1}\right]\lambda^{2}\right)=\exp\left(2i\lambda^{2}\hat{x}_{1}\hat{x}_{2}\hat{n}_{1}\hat{n}_{2}\sigma^{z}\right)$
% and $\exp\left(\left[A_{2},B_{2}\right]\lambda^{2}\right)=\exp\left(2i\lambda^{2}\hat{p}_{1}\hat{p}_{2}\hat{n}_{1}\hat{n}_{2}\sigma^{z}\right)$. 

% The first commutator is given by the Pauli commutation relation Eq.~\ref{eq:PauliCommutatorz}
% \begin{align}
% \left[A_{1},B_{1}\right] & =2i\hat{x}_{1}\hat{x}_{2}\hat{n}_{1}\hat{n}_{2}\sigma^{z}\\
%  & =2i\hat{x}_{1}\hat{x}_{2}\hat{n}_{1}\hat{n}_{2}\left(-\frac{i}{2}\left[\sigma^{x},\sigma^{y}\right]\right)\\
%  & =\left[\hat{x}_{1}\hat{n}_{1}\sigma^{x},\hat{x}_{2}\hat{n}_{2}\sigma^{y}\right]
% \end{align}

% The $A_{1}$ term is determined by a Trotter decomposition according
% to the product of operator formula Eq.~\ref{eq:ProductOperators}
% \begin{align}
% A_{1} & =\frac{1}{2}\left\{ \hat{x}_{1},\hat{n}_{1}\right\} \sigma^{x}+\frac{1}{2}\left[\hat{x}_{1},\hat{n}_{1}\right]\sigma^{x}\\
%  & =A_{1a}+A_{1b}
% \end{align}
% where according to the anticommutator to commutator relation $A_{1a}$
% is given by the BCH formula with 
% \begin{align}
% \left[A_{1a^{\prime}},B_{1a^{\prime}}\right] & =\frac{1}{2}\left\{ \hat{x}_{1},\hat{n}_{1}\right\} \sigma^{x}\\
%  & =\frac{i}{2}\left[i\hat{x}_{1}\sigma^{y},i\hat{n}_{1}\sigma^{z}\right]\\
%  & =\left[-\frac{1}{\sqrt{2}}\hat{x}_{1}\sigma^{y},-\frac{1}{\sqrt{2}}\hat{n}_{1}\sigma^{z}\right]
% \end{align}
% where $B_{1a^{\prime}}$ is a $z$-conditional rotation gate.
% Distribution of terms yields $A_{1b}$ as 
% \begin{equation}
% \left[A_{1b^{\prime}},B_{1b^{\prime}}\right]=\left[\frac{1}{\sqrt{2}}\hat{x}_{1},\frac{1}{\sqrt{2}}\hat{n}_{1}\sigma^{x}\right]
% \end{equation}
% where $B_{1b^{\prime}}$ is an $x$-conditional rotation gate. 

% According to the same procedure,
% \begin{align}
% B_{1} & =\frac{1}{2}\left\{ \hat{x}_{2},\hat{n}_{2}\right\} \sigma^{y}+\frac{1}{2}\left[\hat{x}_{2},\hat{n}_{2}\right]\sigma^{y}\\
%  & =B_{1a}+B_{1b}
% \end{align}
% where $B_{1a}$ is given by
% \begin{align}
% \left[A_{1a^{\prime\prime}},B_{1a^{\prime\prime}}\right] & =\frac{1}{2}\left\{ \hat{x}_{2},\hat{n}_{2}\right\} \sigma^{y}\\
%  & =\frac{i}{2}\left[i\hat{x}_{2}\sigma^{z},i\hat{n}_{1}\sigma^{x}\right]\\
%  & =\left[-\frac{1}{\sqrt{2}}\hat{x}_{2}\sigma^{z},-\frac{1}{\sqrt{2}}\hat{n}_{2}\sigma^{x}\right]
% \end{align}
% with $B_{1a^{\prime\prime}}$ an $x$-conditional rotation, and $B_{1b}$
% is given by 
% \begin{equation}
% \left[A_{1b^{\prime\prime}},B_{1b^{\prime\prime}}\right]=\left[\frac{1}{\sqrt{2}}\hat{x}_{2},\frac{1}{\sqrt{2}}\hat{n}_{2}\sigma^{y}\right]
% \end{equation}
% where $B_{1b^{\prime\prime}}$ is a $y$-conditional rotation gate.

% The second term follows analogously with the position $x$ replaced
% by the momentum $p$.

% \subsection{Conditional FSWAP Gate.}

% In analytic ISA, the FSWAP gate follows immediately from the conditional
% cross-Kerr gate detailed above and a complete beam-splitter gate (or
% conditional beam splitter gate detailed above) as 
% \begin{equation}
% U_{\text{FSWAP}}=U_{\text{cond. Kerr}}U_{\text{cond. beam}}
% \end{equation}
}


