
\section{Discussion}
%% what are the pts we should think about for the audience. 
%% may be easier to do this from the beginning

Our main contribution in this paper is a systematic approach to synthesizing unitary dynamics on a hybrid quantum computer that has access to both qubit and \qumode~operations.  Such gate sets naturally model systems such as cavity quantum electrodynamics systems and ion trap-based quantum computers.  Our main innovation here is the development of high-order analytic formulas that can be used to place bounds on the complexity of implementing arbitrary unitary operations on such a hybrid device.  Specifically, we see that these methods are capable of achieving subpolynomial scaling with the inverse error tolerance ($1/\epsilon$) and allow us to implement arbitrary nonlinearities in the field operators in the generator of the unitary that we wish to implement at low cost asymptotically.  In particular, we focus on using a construct known as a block-encoded creation operator as our fundamental construct and show numerically highly accurate approximations to the exponential of a block encoding of the square of the creation operator.  Further, we study the Hong-Ou-Mandel effect and observe that the synthesized operations used in our construction can have negligible error with respect to our target precision.
%\ck{
Beyond compilation, our work thereby provides intuition when compiling a variety of gates. For example, our methods give mathematical understanding for the recent successes of echoed conditional  displacement (ECD) gate sets for quantum computing \cite{eickbusch2021fast}.
%}
%\ck{....?}



% While this work constitutes a significant step forward in our understanding of how to control and manipulate such quantum systems, there remain many open questions.  
While this work enables better analytic control of qubit-qumode systems, there remain many open questions.  
In particular, lowering the resource cost of these compilations, especially in noisy environments, is key to practical implementations.
% in practical implementations of our synthesized operations within this gate set
We note that for both the Hong-Ou-Mandel effect and the block encoding of $(a^\dagger)^2$ that thousands of gate operations are needed to achieve infidelities of $10^{-3}$ or smaller.  This makes such sequences impractical for near-term applications where the gate infidelities are on the order of $1\%$. Ideally, these analytic sequences can be used as an initial sequence which can be further optimized via existing optimal control techniques. This would reduce warmstart challenges and potentially improve convergence. 


%\ck{
Another concern arises from the nature of the qubit system, namely that creating qubit-qumode interactions requires truncating the qubit system to a two-level system. For example, transmon qubits are typically truncated to two levels~\cite{wendin2017quantum,kjaergaard_superconducting_2020,blais2021circuit};  in reality, these transmons can experience leakage to higher states, meaning that the qubit may itself exhibit qumode-like properties when poorly calibrated.
%} 
While leakage is a practical concern, it can be somewhat avoided via well-defined pulse shaping methods (e.g., DRAG~\cite{theis_counteracting_nodate} pulses). These methods allow one to approach the control speed limit set by the anharmonicity of the level structure.
    

% Thus, directly lifting our results to devices is not currently practical. 


Several avenues of approach exist that could be used to improve upon these results: first, note that our technique is connected to ideas from quantum signal processing (QSP) \cite{martyn_grand_2021, gilyen_quantum_2019}. Thus, QSP could be applied to approximate polynomials of qumode operators; this could potentially improve the scaling with respect to the error tolerance from the subpolynomial scaling currently demonstrated to polylogarithmic scaling. Alternatively, our approach could seed gradient-descent optimization procedures for control such as GRAPE~\cite{khaneja2005optimal} at the pulse level or numerical optimization of parameterized gates at the SNAP~\cite{fosel2020efficient} or ECD \cite{eickbusch2021fast} instruction level.  These locally optimized sequences may then prove to be either better, or more understandable, than existing gradient-optimized pulse sequences for control of such systems. %\ck{
Designing pulses is, in practice, sensitive to device properties and noise environments,
and presents a rich area for further research.
%so we leave this exploration to future work.
%}

%Another important question is to ask from a bigger picture perspective what role Trotter derived synthesis methods like this one may play in controlling quantum systems.