\section{Introduction}
%\ck{Image one: new pathway to block encoding synthesis}

% \begin{figure}[hbt]
%     \centering
%     \includesvg[width=\textwidth]{ImagesAndPrograms/1-pipeline-for-alg-synthesis.svg}
%     \caption{Overview of our strategy: we compile an exotic, nonlinear block encoding acting upon both a qubit and qumode. Our compilation requires only single-qubit gates (small blue squares) and simple qubit-qumode block encodings (blue and purple rectangles).}
%     \label{fig:hero}
% \end{figure}

%\how{synergy between oct and this technique} what can be made with oct, how can it be extended with our technique

Today, many quantum computing architectures are homogeneous, with the same type of qubit used throughout the device. From devices made of superconducting qubits \cite{arute2019quantum,dial2022moving,reagor2018demonstration} to ion trap qubits \cite{wright2019benchmarking}, prior work largely focuses on linking qubits of the same type together in fault-tolerant ways. However, there is emerging work \cite{C2QA_ISA, C2QA_LGT, chakram2021seamless, stavenger2022bosonic,stein2023microarchitectures} 
% teoh2023dual,sivak2023real, duckering_virtualized_2020
studying the use of heterogeneous quantum computers that leverage two or more types of quantum architectures (e.g., qubits and oscillator modes). Heterogeneous devices hold promise because they can be tailored for specific physical simulation problems, which would be especially useful in applications like material discovery~\cite{Holstein_Knorzer_2022}, molecular simulation~\cite{Wang2020FCFs,WangConicalIntersection}, topological models~\cite{PhysRevB.98.174505} or lattice gauge theories~\cite{C2QA_LGT}.

% material discovery or molecular simulation \cite{ Wang2020FCFs,WangConicalIntersection}, or quantum simulation of lattice models~\cite{PhysRevB.98.174505}.

In particular, hybrid \qq~models \cite{blais2021circuit} hold some advantages: for example, microwave qumodes have long lifetimes and large accessible Hilbert spaces, making them
% afforded to microwave photons in superconducting resonators have made 
% and are 
attractive targets for quantum error correction~\cite{blais2021circuit}. 
% Additionally, qumodes have larger accessible Hilbert spaces than qubits. 
Introducing \qumodes~also enables new physical gates, 
% there is exists a larger set of operations that can be performed by coupling an oscillator and a qubit open up the potential for multi-qubit interactions, 
such as the M{\o}lmer-S{\o}rensen gate \cite{molmer-sorensen-gate} while at the same time enabling new forms of transduction between qubits and qumodes \cite{Boissonneault_Dispersive_2009}.
% qubits and flying qubits such as photons~\cite{lauk2020perspectives,basilewitsch2022engineering}. 
Oscillator interactions have unique features, like nonlinearities, which are challenging to simulate even with homogeneous quantum architectures~\cite{stavenger2022bosonic}.


%Spin-boson instruction set architectures (ISA) enable the general control of native bosonic hardware which could have advantages over purely spin-1/2 hardware in quantum signal processing (QSP), quantum simulation, and the preparation and use of quantum states requiring a large Hilbert space.  


Efficiently compiling logical operations to physical pulses is a critical, but computationally expensive task. 
% However, whereas many useful applications have already been demonstrated with the experimentally available gate sets in ion traps and circuit QED, often problems of interest require more complex operations, and these must be compiled from the various experimental regimes and pulse sequences available. 
In theory, pulse design techniques like optimal control theory (OCT)~\cite{,werschnik2007quantum} 
% provide ways to design at a pulse level a sequence of controls that can be set in order to enact an 
can produce pulses that implement arbitrary quantum transformations on a hybrid \qq~system. These techniques have been applied to a variety of physical systems, including NMR~\cite{khaneja2005optimal}, superconducting transmon qubits~\cite{Koch-OCT-2017}, and \qumodes~\cite{ozguler_numerical_2022,anders_petersson_optimal_2022,ma_quantum_2021}. In practice, OCT is computationally intensive and inflexible, requiring pulses to be recompiled on a case-by-case basis. Furthermore, OCT is almost always uninterpretable, yielding only a pulse which performs the desired operation without providing any physical intuition. 
% Furthermore, it is computationally intensive to produce and due to the lack of theoretical understanding behind the pulse sequences it must be carried out on a case-by-case basis.
Experimentally, these limitations prevent the high-fidelity realization of quantum algorithms; theoretically, the complexity of our quantum circuits often inappropriately ignores the classical cost of required compilation. 
% and also confuses our understanding of circuit complexities ... \ck{reword}.

% These shortcomings further make models such as this difficult to program and to analyze the circuit complexity, as the procedure and in turn the cost for constructing an arbitrary unitary transformation can be difficult to bound.

Inspired by recent experimental progress \cite{SNAP-PhysRevLett.115.137002,eickbusch2021fast}, we introduce an extensible control scheme for a universal, hybrid \qq~quantum computer (\cref{tab:all-formulas}).  
Specifically, we show how block-encoded operators can be manipulated 
% consider the potential of 
using the Lie-Trotter-Suzuki and Baker-Campbell-Hausdorff matrix product formulas. We thus enable the creation of instruction set architectures (ISAs) that 
can be analytically compiled to experimentally available gate sets.  Prior art, namely \textcite{jacobs_engineering_2007}, has studied similar techniques to compile operations; we generalize these techniques to work in a variety of domains, including in settings with multiple \qumodes~and more exotic operators, and prove concrete error bounds on these techniques.  

We develop two parallel approaches, one which primarily uses the creation and destruction operators which we refer to as based on `Fock methods,' and the other primarily relying on position and momentum operators which we refer to as based on `phase-space methods.' We demonstrate that both methods can be used to generate an ISA for \qq~devices. Our methods obtain almost-linear asymptotic scaling. 
%\ck{what more to say here}
% and provide upper bounds on the maximum number of operations required to implement the compiled gate. 


Furthermore, we use the previously mentioned formulas to realize a number of operations of interest, including polynomials of annihilation and creation operators, namely $a^p {a^\dagger}^q$ for integer $p, q$. These block-encoded operations are crucial for quantum signal processing (QSP) and certain problems in quantum simulation. Finally, we give examples for the Hamiltonian of a nonlinear material and applications to key unsolved problems in quantum simulation such as the Fermi-Hubbard model. %\ck{
While these approaches are expensive in terms of raw gate counts, because they are analytic they provide intuition into synthesizing the gate robustly. Furthermore, these gate sequences can be used as a starting point for optimal control methods, helping to avoid cold start issues.
%}

% In this paper, we introduce a novel control scheme for a hybrid boson-qubit quantum computer that enables simulation of the Jaynes-Cummings model \cite{JaynesCummings1963}. \added{SMG: Not sure this should be the main focus since you don't explicitly discuss it and there are a number of examples you include and more could be included.}  


% Problem, not knowing how to make gates, no physical insight

% We introduce an analytic approach

% Here is our method ((x, p) and (a, adag))

%% here's how we're connzecting our physical intuition to the realization of actual gates

%% CK: See how to connect the prior overview with the overall language. 

%% here's how you obtain a decomposition in phase-space / displacement operators

%% here's the algorithmic justification that this works

%% not sure about the subsection orderings