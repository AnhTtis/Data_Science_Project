\documentclass[11pt]{article}
\usepackage[margin=1in]{geometry}

\usepackage{amsthm}
\usepackage{thm-restate}
\usepackage{lscape}
\usepackage{tabularx}
  
\usepackage{thmtools} 

\usepackage{thm-restate}
\usepackage{amsmath, amsfonts, mathtools}
\usepackage{braket}
\usepackage{graphicx}
\usepackage{amssymb}
\usepackage{dsfont}

\usepackage{color,soul}
\usepackage{makecell}


% \hypersetup{colorlinks=true,linkcolor=blue, linktocpage}

\usepackage{subfigure}
\usepackage{qcircuit}

% \usetheme[progressbar=frametitle]{metropolis}
% \usepackage{appendixnumberbeamer}

\usepackage{booktabs}
\usepackage[scale=2]{ccicons}


\usepackage[dvipsnames]{xcolor}
% \usepackage[colorlinks,allcolors=blue]{hyperref}
\usepackage{hyperref}
\hypersetup{
    colorlinks = true,
    linkcolor = blue
    }
\usepackage[noabbrev,capitalize,nameinlink]{cleveref}

\usepackage{pgfplots}
\pgfplotsset{compat=1.17}
\usepgfplotslibrary{dateplot}

\usepackage[style=trad-unsrt,sorting=none,doi=false,url=false,giveninits=true,hyperref]{biblatex}
\addbibresource{references.bib}
%\bibliographystyle{unsrt}

\usepackage{xspace}
\usepackage{algorithm}
\usepackage{algpseudocode}
\newcommand{\how}[1]{\colorbox{BurntOrange}{\textbf{#1}}}
\newcommand{\themename}{\textbf{\textsc{metropolis}}\xspace}
\newcommand{\kron}{\otimes}
% \newcommand{\identity}{\mathds{1}}
\newcommand{\identity}{{\boldsymbol 1}}
\newcommand{\norm}[1]{\left\lVert#1\right\rVert}
\newcommand{\ketbra}[2]{\ket{#1}\!\bra{#1}}
\DeclarePairedDelimiter\ceil{\lceil}{\rceil}
\DeclarePairedDelimiter\floor{\lfloor}{\rfloor}

\newtheorem{theorem}{Theorem}[section]
\newtheorem{corollary}{Corollary}[theorem]
\newtheorem{lemma}{Lemma}[section]
\newtheorem{fact}[theorem]{Fact}

\newtheorem{define}[theorem]{Definition}
\newtheorem{claim}{Claim}
\newtheorem{problem}{Problem}
\newtheorem{corr}{Corollary}[section]


\usepackage[draft]{changes} %%%%---annotations are visible

\title{Leveraging Hamiltonian Simulation Techniques \\to Compile Operations on Bosonic Devices}
% \subtitle{dah dah dah}
% \date{\today}
\usepackage{authblk}

\date{\today}
% \author{Christopher Kang \\ Nathan Wiebe}

\author[1, 2, 3]{Christopher Kang$^*$}%\thanks{These two authors contributed equally}}
\author[4,5,6,7]{Micheline B.~Soley\thanks{These two authors contributed equally}}

\author[8]{Ella Crane}
\author[6]{S. M. Girvin}
\author[2, 9, 10]{Nathan Wiebe}

% \how{co-first author}

% \affiliation{School of Computer Science, University of Washington, Seattle, WA 98195}
% \affiliation{Pacific Northwest National Laboratory, Richland, WA 99354}
% \affiliation{Department of Computer Science, University of Chicago, Chicago, IL 60637, USA}

\affil[1]{School of Computer Science, University of Washington, Seattle, WA 98195 USA}
\affil[2]{Pacific Northwest National Laboratory, Richland, WA 99354 USA}
\affil[3]{Department of Computer Science, University of Chicago, Chicago, IL 60637 USA}
\affil[4]{Department of Chemistry, University of Wisconsin-Madison, 1101 University Ave., Madison, WI 53706 USA}
\affil[5]{Department of Physics, University of Wisconsin-Madison, 1150 University Ave., Madison, WI 53706 USA}
\affil[6]{Yale Quantum Institute, 17 Hillhouse Ave.,
PO Box 208 334, New Haven, CT 06520-8263 USA 
}
\affil[7]{Department of Chemistry, Yale University, 225 Prospect St., New Haven, CT 06511 USA}
\affil[8]{Joint Quantum Institute \& Joint Center for Quantum Information and Computer Science, NIST/University of Maryland, College Park, Maryland, 20742 USA}
\affil[9]{Department of Computer Science, University of Toronto, ON M5G 1Z7 Canada}
\affil[10]{Canadian Institute For Advanced Research / Institut Canadien de Recherches Avanc\`ees, ON M5G 1Z7 Canada}
% \floatname{algorithm}{Procedure}
\renewcommand{\algorithmicrequire}{\textbf{Input:}}
\renewcommand{\algorithmicensure}{\textbf{Output:}}

% \declaretheorem[name=Theorem,numberwithin=section]{thm}


\begin{document}
\date{}

\maketitle

% \input{weeks/210430}
% \input{qsp}
% \input{qsp2}
% %\documentclass[10pt,twocolumn,twoside]{IEEEtran}
%\documentclass[11pt,onecolumn,twoside,draftcls]{IEEEtran}
%\documentclass[5p]{elsarticle}
%\documentclass[preprint,11pt]{elsarticle}
\documentclass[1p,11pt]{elsarticle}
%\documentclass[journal]{IEEEtran}
%\documentclass[conference]{IEEEtran}
%\IEEEoverridecommandlockouts

\usepackage{amsmath}
\usepackage{amssymb}
\usepackage{amsthm}
\usepackage{graphicx}
\usepackage{epstopdf}
\usepackage{bm}
\usepackage{color}
%\usepackage{cite}
\usepackage{subcaption}
\usepackage{dsfont}
\usepackage{algorithm}
\usepackage{algpseudocode}
\usepackage{enumitem}

\biboptions{sort&compress}

\newtheorem{theorem}{Theorem}
\newtheorem{lemma}[theorem]{Lemma}
\newtheorem{corollary}[theorem]{Corollary}
\newtheorem{definition}{Definition}
%\newtheorem{proposition}{Proposition}
\newtheorem{assumption}{Assumption}
\newtheorem{remark}[theorem]{Remark}
\newtheorem{example}[theorem]{Example}

\allowdisplaybreaks

%\journal{Signal Processing}

\begin{document}

\begin{frontmatter}

\title{Estimation of Scalar Field Distribution in the \\Fourier Domain} 

\author[1]{Alex S. Leong\corref{cor1}%
%\fnref{fn1}
}
\ead{alex.leong@defence.gov.au}
\author[2]{Alexei T. Skvortsov%\fnref{fn2}
}
\ead{alexei.skvortsov@defence.gov.au}

\cortext[cor1]{Corresponding author}

\affiliation{organization={Defence Science and Technology Group},
%addressline={506 Lorimer St},
city={Fishermans Bend},
postcode={Vic. 3207},
country={Australia}} 

%\maketitle

\begin{abstract}
In this paper we consider the problem of estimation of scalar field distribution collected from noisy measurements. The field is modelled as a sum of Fourier components/modes, where the number of modes retained and estimated determines in a natural way the approximation quality. An algorithm for estimating the modes using an online optimization approach is presented, under the assumption that the noisy measurements are quantized. The algorithm can estimate time-varying fields through the introduction of a forgetting factor. Simulation studies demonstrate the effectiveness of the proposed approach. 
\end{abstract}

\end{frontmatter}

\section{Introduction}

Estimation of scalar field distribution from a set of point measurements is an important problem often emerging in ecology, geophysics, and many technological applications. Examples include concentration of pollutant, radiation, temperature in urban areas, carbon dioxide emission, methane sources, and many others, see 
\cite{HutchinsonOh,HutchinsonLiu,NeumannBennetts_advanced_robotics,ThomsonHirst,RisticMorelandeGunatilaka,Selvaratnam_CDC,EslingerMendez,NewazJeong,WeidmannHirst,LiChen,MartinPayton,LaSheng,LaShengChen,MorelandeSkvortsov,RazakSukumarChung_journal,LeongZamani_SP,LeongZamaniShames,TranGarratt} and references therein. This approach is often used for indirect inference of scalar fields (pressure, temperature, radiation)  in inaccessible locations where the direct measurements are prohibited due to some geometrical or physical constraints (blocking obstacles, high temperature, or exposure to hazards). The methods of source localisation \cite{HutchinsonOh,HutchinsonLiu,NeumannBennetts_advanced_robotics,ThomsonHirst,RisticMorelandeGunatilaka,Selvaratnam_CDC,EslingerMendez,NewazJeong,WeidmannHirst,LiChen} and mapping \cite{MartinPayton,LaSheng,LaShengChen,MorelandeSkvortsov,RazakSukumarChung_journal,LeongZamani_SP,LeongZamaniShames,TranGarratt} employing remote (and noisy) measurements have attracted increasing attention in recent years due to tremendous progress in instrumentation for aerial and remote sensing using unmanned aerial vehicles (UAVs) and  unmanned ground vehicles (UGVs). This technological advancement necessitates the development and evaluation of some statistical methods and algorithms that can be applied for the timely estimation of the structure (map) of the scalar field in the environment from an ever-increasing set of noisy measurements acquired in a  sequential or concurrent manner (e.g.,  sensing signals from UAVs and UGVs operating over the hazardous area, the intermittent concentration of methane leaked from the ocean floor, oil surface concentration due to androgenic spill, trigger signals from meteorological stations,  etc). These algorithms may become critical for backtracking and characterization of the main sources of the scalar field in the environment which is important for the  remediation effectiveness and retrospective forensic analysis.   This was the main motivation for the present study.

Conventionally, in work on estimation of scalar fields, the field is modelled as a sum of radial basis functions (RBFs) or Gaussian mixture models, see, e.g., \cite{LaSheng,LaShengChen,MorelandeSkvortsov,RazakSukumarChung_journal,LeongZamani_SP,LeongZamaniShames,TranGarratt}. Field estimation then reduces to a problem of estimating the parameters of these models. In the current work, we assume the field to be an arbitrary 2D function which can be viewed in the Fourier domain using, e.g., the discrete Fourier transform (DFT) or the discrete cosine transform (DCT) \cite{BritanikYipRao}. For intuition of this approach, suppose we  regard the plot of the field as an image. From image processing, it is well-known that the most important parts of an image are concentrated in the lowest (spatial) frequency components/modes. Our approach to field estimation is then to estimate the low frequency Fourier components.\footnote{We will use the terms Fourier component and DCT component interchangeably in this paper.} One of the advantages for using this Fourier component approach compared to the RBF approach is that it offers a perhaps more natural way to control the accuracy of the approximation, e.g., by controlling the number of Fourier modes used/retained. Furthermore, if one wants to refine the field estimate by estimating more modes, existing estimates of the lower order modes can be reused. 

The main contributions of this paper are:
\begin{itemize}
    \item Rather than the use of radial basis function field models, we model the 2D scalar field in the Fourier domain as a sum of Fourier components. 
    \item A numerical comparison of the approximation capabilities of the Fourier components and RBF field models is carried out.
    \item We show that the RBF field model and Fourier component field model have similar forms, which allows one to leverage existing algorithms for field estimation developed for the RBF field model to estimate the Fourier components under various different measurement models.
    \item For the quantized measurements model, we present in detail how Fourier component estimation can be carried out using an online optimization approach similar to \cite{LeongZamaniShames}. We further extend the approach of \cite{LeongZamaniShames} from binary measurements to multi-level quantized measurements, and from static to time-varying fields. 
\end{itemize}

The organization of the paper is as follows: Section \ref{sec:preliminaries} gives preliminaries on the DCT and motivation for its use in field modelling. Section \ref{sec:system_model} presents our field model, as well as various different sensor measurement models. Section~\ref{sec:DCT_RBF_comparison} compares our Fourier component field model with the RBF field model in terms of approximation performance. Section \ref{sec:DCT_estimation} first relates our field model to the RBF field models considered in previous works, and then considers in detail the estimation of Fourier components using  quantized measurements. Numerical studies  are presented in Section \ref{sec:numerical}. 

\section{Preliminaries}
\label{sec:preliminaries}

Consider a region of interest $\mathcal{S} = [X_{\textnormal{min}}, X_{\textnormal{max}}] \times [Y_{\textnormal{min}}, Y_{\textnormal{max}}]$. Discretize $[X_{\textnormal{min}}, X_{\textnormal{max}}]$ into $N_x$ points and $ [Y_{\textnormal{min}}, Y_{\textnormal{max}}]$ into $N_y$ points as
\begin{align*}
\mathcal{X}_d \triangleq \left\{X_{\textnormal{min}} + \Big(\frac{1}{2} + I_x \Big) \Delta_x:   I_x \in \{0, \dots, N_x - 1\} \right\} \\
\mathcal{Y}_d \triangleq \left\{Y_{\textnormal{min}} + \Big(\frac{1}{2} + I_y \Big) \Delta_y:  I_y \in \{0, \dots, N_y - 1\} \right\},
\end{align*}
where 
$$\Delta_x \triangleq \frac{X_{\textnormal{max}} - X_{\textnormal{min}}}{N_x}, \quad \Delta_y \triangleq \frac{Y_{\textnormal{max}} - Y_{\textnormal{min}}}{N_y}. $$


Our aim is the estimation of 2D distribution of scalar field $\phi(x,y)$, $(x,y) \in \mathcal{S}$,  which is assumed either static or slowly varying. We define
$$\phi_d(I_x, I_y) \triangleq \phi\Big(X_{\textnormal{min}} + \left(1/2 + I_x \right) \Delta_x, Y_{\textnormal{min}} + \left(1/2 + I_y \right) \Delta_y \Big)$$
as the field value at the discretized position $\big(X_{\textnormal{min}} + \left(1/2 + I_x \right) \Delta_x, Y_{\textnormal{min}} + \left(1/2 + I_y \right) \Delta_y \big) \in \mathcal{X}_d \times \mathcal{Y}_d$. 
Recall the (Type-II) discrete cosine transform (DCT), see, e.g., \cite{BritanikYipRao,Strang_DCT}: 
\begin{align*}
C(u,v) &= \sum_{I_x=0}^{N_x-1} \sum_{I_y=0}^{N_y-1} \alpha_x(u) \alpha_y(v) \phi_d(I_x,I_y) \cos \left(\frac{(2I_x+1)\pi u}{2 N_x} \right) \cos \left(\frac{(2I_y+1)\pi v}{2 N_y} \right), \\ &\quad\quad u=0,\dots,N_x-1, \quad v=0,\dots,N_y-1,
\end{align*}
where 
$$ \alpha_x(u) \triangleq \left\{\begin{array}{ll} \sqrt{\frac{1}{N_x}}, & u = 0 \\ 
\sqrt{\frac{2}{N_x}}, & u \neq 0
\end{array} \right., \quad 
\alpha_y(v) \triangleq \left\{\begin{array}{ll} \sqrt{\frac{1}{N_y}}, & v = 0 \\
\sqrt{\frac{2}{N_y}}, & v \neq 0.
\end{array} \right. $$

The inverse DCT is given by:
\begin{align}
\phi_d(I_x,I_y) &= \sum_{u=0}^{N_x-1} \sum_{v=0}^{N_y-1} \alpha_x(u) \alpha_y(v) C(u,v) \cos \left(\frac{(2I_x+1)\pi u}{2 N_x} \right) \cos \left(\frac{(2I_y+1)\pi v}{2 N_y} \right), \label{eqn:inverse_DCT} \\ &\quad\quad I_x=0,\dots,N_x-1, \quad I_y=0,\dots,N_y-1. \nonumber
\end{align}
It will be convenient for our purposes to rewrite \eqref{eqn:inverse_DCT} as
\begin{align}
\phi_d(I_x,I_y) &= \sum_{(u,v) \in \mathcal{U}}  \alpha_x(u) \alpha_y(v) C(u,v) \cos \left(\frac{(2I_x+1)\pi u}{2 N_x} \right) \cos \left(\frac{(2I_y+1)\pi v}{2 N_y} \right), \label{eqn:inverse_DCT_single_summation} \\ &\quad\quad I_x=0,\dots,N_x-1, \quad I_y=0,\dots,N_y-1, \nonumber
\end{align}
where 
$$\mathcal{U} \triangleq \{(u,v): u \in \{0, \dots, N_x-1\}, v \in \{0, \dots, N_y-1\} \}.$$

The most important information about the field distribution is concentrated in the low order modes, i.e. the components corresponding to $\cos \left(\frac{(2I_x+1)\pi u}{2 N_x} \right) \cos \left(\frac{(2I_y+1)\pi v}{2 N_y} \right)$ with $u$ and $v$ small, while higher order modes define the fine structure of the field distribution. See Figs. \ref{fig:field_seed341_DCT} and \ref{fig:field_seed343_DCT} for examples of how retaining different numbers of modes affects the quality of the approximation to the true field. 


\section{System Model}
\label{sec:system_model}

\subsection{Field Model}
Motivated by the above discussion, we propose to approximate \eqref{eqn:inverse_DCT_single_summation} by
\begin{equation}
\label{field_model} 
\begin{split}
\phi_d(I_x,I_y) & \approx \sum_{(u,v)\in \tilde{\mathcal{U}}} \alpha_x(u) \alpha_y(v) C(u,v) \cos \left(\frac{(2I_x+1)\pi u}{2 N_x} \right) \cos \left(\frac{(2I_y+1)\pi v}{2 N_y} \right), \\ & \quad\quad I_x=0,\dots,N_x-1, \quad I_y=0,\dots,N_y-1 \\ & \triangleq \tilde{\phi}_d(I_x,I_y),
\end{split}
\end{equation}
where $\tilde{\mathcal{U}} \subseteq \mathcal{U}$ is the subset of low order modes that we wish to retain.\footnote{In general one could use in \eqref{field_model} coefficients $\tilde{C}(u,v)$ which are not necessarily equal to $C(u,v)$. One reason for taking the coefficients to be equal to $C(u,v)$ is given in Lemma \ref{lemma:optimal_C_DCT}.}

For example, we could  retain the first $\tilde{N}_x \times \tilde{N}_y$ modes, with $\tilde{N}_x \leq N_x, \tilde{N}_y \leq N_y$, so that 
\begin{equation}
\label{eqn:U_tilde_rect}
 \tilde{\mathcal{U}} = \{(u,v): u \in \{0, \dots, \tilde{N}_x-1\}, v \in \{0, \dots, \tilde{N}_y-1\} \}.
 \end{equation}
The total number  of modes retained $\tilde{N} $ is thus equal to $\tilde{N} = \tilde{N}_x  \tilde{N}_y$.

Another possibility is the following:
\begin{equation}
\label{eqn:U_tilde_largest}
\tilde{\mathcal{U}} = \{ \tilde{N} \textnormal{ pairs } (u,v) \textnormal{ with smallest values of } (u+1)^2 + (v+1)^2 \}
\end{equation}
which tries to retain the $\tilde{N}$ ``largest'' (in magnitude) modes.\footnote{This is of course an approximation, as exactly determining the $\tilde{N}$ largest modes depends on and requires knowledge of the very field that we are trying to estimate.} The motivation for \eqref{eqn:U_tilde_largest} comes from a result that the DCT coefficients $C(u,v)$ decay as $O \big(\frac{1}{(u+1)^2 + (v+1)^2} \big)$ for $u,v \rightarrow \infty$ \cite{YamataniSaito}. Thus the larger components will usually have smaller values of  $(u+1)^2 + (v+1)^2$, leading to the choice \eqref{eqn:U_tilde_largest}. In numerical simulations, we have found \eqref{eqn:U_tilde_largest} to give better approximations than \eqref{eqn:U_tilde_rect} (for the same number of retained modes $\tilde{N}$) in many, though not all, cases. 


\subsection{Measurement Models}
At position $\big(X_{\textnormal{min}} + \left(1/2 + I_x \right) \Delta_x, Y_{\textnormal{min}} + \left(1/2 + I_y \right) \Delta_y \big)$, we have noisy measurements of the field
$$z(I_x,I_y) = h(\phi_d(I_x,I_y), n(I_x, I_y)),$$
where $h(\bm{\cdot},\bm{\cdot})$ is a (in general non-linear) function, and  $n(\bm{\cdot},\bm{\cdot})$ is random noise.

For example, we could have additive noise
\begin{equation}
\label{additive_noise_model}
z(I_x,I_y) = \phi_d(I_x,I_y) + n(I_x, I_y),
\end{equation}
similar to \cite{LaSheng,LaShengChen}. 

One could also further quantize \eqref{additive_noise_model}
\begin{equation}
\label{quantized_measurement_model}
z(I_x,I_y) = q(\phi_d(I_x,I_y) + n(I_x, I_y))
\end{equation}
where $q(\bm{\cdot})$ is a quantizer of $L$ levels, say $\{0, 1, \dots, L-1\}$. The quantizer can be expressed in the form 
\begin{equation}
\label{eqn:quantizer}
q(x) = \left\{\begin{array}{cc} 0, & x < \tau_0 \\ 1, & \tau_0 \leq x < \tau_1 \\ \vdots & \vdots \\ L-2, & \tau_{L-3} \leq x < \tau_{L-2} \\ L-1, & x \geq \tau_{L-2}    \end{array} \right. 
\end{equation}
where the quantizer thresholds $\{\tau_0,\dots,\tau_{L-2}\}$ satisfy $\tau_0 \leq \tau_1 \leq \dots \leq \tau_{L-2}$.


The special case of \eqref{quantized_measurement_model}-\eqref{eqn:quantizer} corresponding to a 1-bit quantizer, or binary measurements, is considered in \cite{LeongZamani_SP,LeongZamaniShames,TranGarratt}. It can be expressed as
\begin{equation}
\label{binary_measurement_model}
z(I_x,I_y) = \mathds{1env} \big(\phi_d(I_x,I_y) + n(I_x, I_y) > \tau \big),
\end{equation}
where $\tau$ is the quantizer threshold, and $\mathds{1}(\bm{\cdot})$ is the indicator function that returns 1 if its argument is true and 0 otherwise. 

Another measurement model which has been considered are Poisson measurements \cite{MorelandeSkvortsov}. Define $\mathbf{x} \triangleq (x,y)$. Then in this model
$$z(\mathbf{x}) \sim \texttt{Poisson}(\lambda(\textbf{x})),$$
where 
$$\lambda(\textbf{x}) = \int k(\mathbf{x}' - \mathbf{x}) \phi(\mathbf{x}') d\mathbf{x}'$$
and 
$$k (\mathbf{x}) = \left\{ \begin{array}{cc} \frac{1}{R^2}, & ||\mathbf{x}|| \leq R \\  \frac{1}{||\mathbf{x}||^2}, & ||\mathbf{x}|| \geq R \end{array} \right.$$
for some constant $R$.

\subsection{Problem Statement}
\label{sec:problem_statement}
The problem we wish to consider in this paper is to estimate the coefficients\footnote{When we refer to \emph{estimation of components/modes} in this paper, we specifically mean estimation of the coefficients $C(u,v)$.}
$$C(u,v), \,\,(u,v) \in \tilde{\mathcal{U}}$$
from noisy measurements $\{z(I_x,I_y)\}$ of the field $\phi_d(I_x,I_y)$. The estimation should be done in an online manner such that the estimates are continually updated as new measurements are collected.


\section{Comparison with RBF Field Model}
\label{sec:DCT_RBF_comparison}
Before we consider the problem of estimating the coefficients $C(u,v)$ (which will be studied in Section \ref{sec:DCT_estimation}), we will in this section compare the use of our Fourier component model
\eqref{field_model}
with the radial basis function model considered in \cite{RazakSukumarChung_journal, LeongZamani_SP,LeongZamaniShames,TranGarratt} (see also \cite{LaSheng,LaShengChen,MorelandeSkvortsov} for similar models), in terms of how well they can approximate a field for a given number of modes (for the Fourier component model) or basis functions (for the RBF model). 

\subsection{Fourier Component Field Model}
Define the mean squared error (MSE):
$$ \textnormal{MSE} \triangleq \frac{1}{N_x N_y} \sum_{I_x=0}^{N_x-1} \sum_{I_y=0}^{N_y-1} \Big( \phi_d (I_x, I_y) - \tilde{\phi}_d (I_x, I_y)  \Big)^2,$$
where 
%$$\phi_d(I_x,I_y) = \sum_{(u,v) \in \mathcal{U}} \alpha_x(u) \alpha_y(v) C(u,v) \cos \left(\frac{(2I_x+1)\pi u}{2 N_x} \right) \cos \left(\frac{(2I_y+1)\pi v}{2 N_y} \right)$$
 $\phi_d(I_x,I_y)$ is the (discretized) true field given by \eqref{eqn:inverse_DCT_single_summation} and
$$\tilde{\phi}_d (I_x, I_y)   \triangleq \sum_{(u,v) \in \tilde{\mathcal{U}} } \alpha_x(u) \alpha_y(v) \tilde{C}(u,v) \cos \left(\frac{(2I_x+1)\pi u}{2 N_x} \right) \cos \left(\frac{(2I_y+1)\pi v}{2 N_y} \right)$$
is the approximation of the true field using a subset of modes $\tilde{U}$ and coefficients $\tilde{C}(u,v)$. The expression for $\tilde{\phi}_d (I_x, I_y) $ is the same as \eqref{field_model} except that the coefficients $\tilde{C}(u,v)$ may be different from $C(u,v)$. However, it turns out that setting $\tilde{C}(u,v)$ to be equal to $C(u,v)$ will minimize the MSE. 

\begin{lemma}
\label{lemma:optimal_C_DCT}
Given a subset of modes $\tilde{U}$, the optimal values of $\tilde{C}(u,v)$ that minimize the MSE satisfy
$$ \tilde{C}^*(u,v) = C(u,v), \, \forall (u,v) \in \tilde{U}.$$
\end{lemma}
\begin{proof}
By definition, 
%\begin{equation}
%\label{eqn:MSE_derivation_DCT}
%\begin{split}
\begin{align}
\textnormal{MSE} &= \frac{1}{N_x N_y} \sum_{I_x=0}^{N_x-1} \sum_{I_y=0}^{N_y-1} \Big( \phi_d (I_x, I_y) - \tilde{\phi}_d (I_x, I_y)  \Big)^2  \nonumber \\
& = \frac{1}{N_x N_y} \sum_{I_x=0}^{N_x-1} \sum_{I_y=0}^{N_y-1}  \Bigg( \sum_{(u,v) \in \mathcal{U}} \alpha_x(u) \alpha_y(v) C(u,v) \cos \left(\frac{(2I_x+1)\pi u}{2 N_x} \right) \cos \left(\frac{(2I_y+1)\pi v}{2 N_y} \right) \nonumber \\
& \quad - \sum_{(u,v) \in \tilde{\mathcal{U}} } \alpha_x(u) \alpha_y(v) \tilde{C}(u,v) \cos \left(\frac{(2I_x+1)\pi u}{2 N_x} \right) \cos \left(\frac{(2I_y+1)\pi v}{2 N_y} \right) \Bigg) ^2 \nonumber \\
&= \frac{1}{N_x N_y} \sum_{I_x=0}^{N_x-1} \sum_{I_y=0}^{N_y-1}  \Bigg( \sum_{(u,v) \in \tilde{\mathcal{U}}} \alpha_x(u) \alpha_y(v) \big(C(u,v) - \tilde{C}(u,v) \big) \nonumber  \\ 
& \quad \quad \times \cos \left(\frac{(2I_x+1)\pi u}{2 N_x} \right) \cos \left(\frac{(2I_y+1)\pi v}{2 N_y} \right) \nonumber \\
& \quad + \sum_{(u,v) \in \mathcal{U} \setminus \tilde{\mathcal{U}} } \alpha_x(u) \alpha_y(v) C(u,v) \cos \left(\frac{(2I_x+1)\pi u}{2 N_x} \right) \cos \left(\frac{(2I_y+1)\pi v}{2 N_y} \right) \Bigg) ^2 \nonumber \\
& = \frac{1}{N_x N_y} \sum_{I_x=0}^{N_x-1} \sum_{I_y=0}^{N_y-1}  \Bigg[ \Bigg( \sum_{(u,v) \in \tilde{\mathcal{U}}} \alpha_x(u) \alpha_y(v) \big(C(u,v) - \tilde{C}(u,v) \big) \nonumber \\ 
& \quad \quad \times \cos \left(\frac{(2I_x+1)\pi u}{2 N_x} \right) \cos \left(\frac{(2I_y+1)\pi v}{2 N_y} \right) \Bigg)^2 \nonumber \\
& \quad + 2 \Bigg( \sum_{(u,v) \in \tilde{\mathcal{U}}} \alpha_x(u) \alpha_y(v) \big(C(u,v) - \tilde{C}(u,v) \big) \cos \left(\frac{(2I_x+1)\pi u}{2 N_x} \right) \cos \left(\frac{(2I_y+1)\pi v}{2 N_y} \right) \Bigg) \nonumber \\
& \quad \quad \times \Bigg( \sum_{(u,v) \in \mathcal{U} \setminus \tilde{\mathcal{U}} } \alpha_x(u) \alpha_y(v) C(u,v) \cos \left(\frac{(2I_x+1)\pi u}{2 N_x} \right) \cos \left(\frac{(2I_y+1)\pi v}{2 N_y} \right) \Bigg) \nonumber \\
& \quad + \Bigg(\sum_{(u,v) \in \mathcal{U} \setminus \tilde{\mathcal{U}} } \alpha_x(u) \alpha_y(v) C(u,v) \cos \left(\frac{(2I_x+1)\pi u}{2 N_x} \right) \cos \left(\frac{(2I_y+1)\pi v}{2 N_y} \right) \Bigg)^2 \Bigg] \nonumber \\
& = \frac{1}{N_x N_y} \sum_{I_x=0}^{N_x-1} \sum_{I_y=0}^{N_y-1}  \Bigg[ \Bigg( \sum_{(u,v) \in \tilde{\mathcal{U}}} \alpha_x(u) \alpha_y(v) \big(C(u,v) - \tilde{C}(u,v) \big) \nonumber \\ 
& \quad \quad \times \cos \left(\frac{(2I_x+1)\pi u}{2 N_x} \right) \cos \left(\frac{(2I_y+1)\pi v}{2 N_y} \right) \Bigg)^2 \nonumber \\
& \quad + \Bigg(\sum_{(u,v) \in \mathcal{U} \setminus \tilde{\mathcal{U}} } \alpha_x(u) \alpha_y(v) C(u,v) \cos \left(\frac{(2I_x+1)\pi u}{2 N_x} \right) \cos \left(\frac{(2I_y+1)\pi v}{2 N_y} \right) \Bigg)^2 \Bigg]  \label{eqn:MSE_derivation_DCT}
\end{align}
%\end{split}
%\end{equation}
The last equality follows since 
\begin{align*}
\sum_{I_x=0}^{N_x-1} \sum_{I_y=0}^{N_y-1} & \alpha_x(u) \alpha_y(v) \big(C(u,v) - \tilde{C}(u,v) \big) \cos \left(\frac{(2I_x+1)\pi u}{2 N_x} \right) \cos \left(\frac{(2I_y+1)\pi v}{2 N_y} \right) \\
& \quad \times \alpha_x(u') \alpha_y(v') C(u',v') \cos \left(\frac{(2I_x+1)\pi u'}{2 N_x} \right) \cos \left(\frac{(2I_y+1)\pi v'}{2 N_y} \right)  
\end{align*}
is equal to zero for all $(u,v) \in \mathcal{U}$ and $ (u',v') \in \mathcal{U} \setminus \tilde{\mathcal{U}}$, by orthogonality of the DCT basis vectors \cite{AhmedNatarajanRao,Strang_DCT}. To conclude the proof, we note that the expression for the MSE given in the last equality of \eqref{eqn:MSE_derivation_DCT} is  clearly minimized when $ \tilde{C}(u,v) = C(u,v), \, \forall (u,v) \in \tilde{U}.$
\end{proof}


\subsection{RBF Field Model}
The following RBF field model is used in \cite{RazakSukumarChung_journal, LeongZamani_SP,LeongZamaniShames,TranGarratt}:
\begin{equation}
\label{field_model_RBF}
\phi(\mathbf{x}) \approx \sum_{j=1}^J \beta_j K_j(\mathbf{x}),
\end{equation}
where $\mathbf{x} \triangleq (x,y)$ and $K_j(\mathbf{x}), j=1,\dots,J$ are radial basis functions. In particular, we consider the choice
 \begin{equation}
 \label{eqn:Gaussian_RBF}
 K_j(\mathbf{x}) = \exp \left(- \frac{\|\mathbf{c}_j-\mathbf{x}\|^2}{\sigma_j^2}\right), \quad j=1,\dots,J,
 \end{equation}
which results in a Gaussian mixture model \cite{MorelandeSkvortsov}.
For a given number of basis functions $J$, we assume that the $\mathbf{c}_j$'s and $\sigma_j$'s are chosen,\footnote{The case where the $\mathbf{c}_j$'s and $\sigma_j$'s are also estimated has been considered, but was found to suffer from identifiability issues and sometimes give very unreliable results \cite{LeongZamani_SP}.} while the $\beta_j$'s are free parameters. Algorithms for estimating the $\beta_j$'s are studied in, e.g., \cite{RazakSukumarChung_journal, LeongZamani_SP,LeongZamaniShames,TranGarratt}. Here we consider instead the problem of finding the optimal  $\beta_j$'s in order to minimize the mean squared error, to see how good the RBF model can be when approximating a field for a given set of basis functions. Define
$$ \textnormal{MSE}_{RBF} \triangleq \frac{1}{|\mathcal{S}_d|} \sum_{\mathbf{x} \in \mathcal{S}_d} \Big( \phi (\mathbf{x}) - \sum_{j=1}^J \beta_j K_j(\mathbf{x}) \Big)^2,$$
where $\phi (\mathbf{x}) $ is the true field value at position~$\mathbf{x}$, $\mathcal{S}_d$ is a discretized set of points in the search region $\mathcal{S}$, and $|\mathcal{S}_d|$ is the cardinality of $\mathcal{S}_d$. 

\begin{lemma}
\label{lemma:optimal_beta_RBF}
Given a set of radial basis functions $\{K_1(.), \dots, K_j(.)\}$ and an ordering $\{\mathbf{x}_1, \dots, \mathbf{x}_{|\mathcal{S}_d|} \}$ of the elements in $\mathcal{S}_d$, the optimal values of $(\beta_1, \dots, \beta_J)$ that minimize $ \textnormal{MSE}_{RBF}$ satisfy
$$ \bm{\beta}^* = \left( \mathcal{K}^T \mathcal{K} \right)^{-1} \mathcal{K}^T \bm{\phi},$$
where $\bm{\beta} = \begin{bmatrix} \beta_1 & \dots & \beta_J \end{bmatrix}^T$, $\bm{\phi} = \begin{bmatrix} \phi(\mathbf{x}_1), \dots, \phi(\mathbf{x}_{|\mathcal{S}_d|}) \end{bmatrix}^T$, and 
$$\mathcal{K} = \begin{bmatrix}
K_1(\mathbf{x}_1) & \dots & K_J(\mathbf{x}_1) \\
\vdots & \ddots & \vdots \\
K_1(\mathbf{x}_{|\mathcal{S}_d|}) & \dots & K_J(\mathbf{x}_{|\mathcal{S}_d|})
\end{bmatrix}.$$
\end{lemma}

\begin{proof}
This is a standard application of the optimal solution to a linear least squares / linear regression problem \cite{CalafioreElGhaoui,Murphy_book1}.    
\end{proof}

\subsection{Numerical Experiments}

\begin{figure}[t!]
\centering 
\includegraphics[scale=0.35]{field_seed341_DCT.pdf} 
\caption{True field and approximations obtained by retaining different numbers of modes}
\label{fig:field_seed341_DCT}
\end{figure} 

\begin{figure}[t!]
\centering 
\includegraphics[scale=0.35]{field_seed341_RBF.pdf} 
\caption{True field and approximations obtained by using different numbers of basis functions}
\label{fig:field_seed341_RBF}
\end{figure} 

\begin{figure}[t!]
\centering 
\includegraphics[scale=0.35]{field_seed343_DCT.pdf} 
\caption{True field and approximations obtained by retaining different numbers of modes}
\label{fig:field_seed343_DCT}
\end{figure} 

\begin{figure}[t!]
\centering 
\includegraphics[scale=0.35]{field_seed343_RBF.pdf} 
\caption{True field and approximations obtained by using different numbers of basis functions}
\label{fig:field_seed343_RBF}
\end{figure} 

In Figs.\ref{fig:field_seed341_DCT}-\ref{fig:field_seed343_RBF}  we show two example fields, and the field approximations that are obtained when various different numbers of modes (for Fourier component model) or radial basis functions (for RBF model) are used. The discretization in the true fields is set as $N_x = N_y = 100$ (so that there are $100^2 = 10000$ modes in total). For the RBF model we set $\mathcal{S}_d = \mathcal{X}_d \times \mathcal{Y}_d$, so that the discretized set of points are the same in the MSE calculations. 
For the Fourier component model, we choose $\tilde{\mathcal{U}}$ as in \eqref{eqn:U_tilde_largest} to retain the $\tilde{N}$ ``largest'' modes. For the RBF model, we use $J = J_x \times J_y$ radial basis functions with $\mathbf{c}_j$'s in \eqref{eqn:Gaussian_RBF} placed uniformly on a grid at locations $\mathcal{X}_{RBF} \times \mathcal{Y}_{RBF}$, where
\begin{align*}
\mathcal{X}_{RBF} \triangleq \left\{X_{\textnormal{min}} + \Big(\frac{1}{2} + i_x \Big) \delta_x:   i_x \in \{0, \dots, J_x - 1\} \right\} \\
\mathcal{Y}_{RBF} \triangleq \left\{Y_{\textnormal{min}} + \Big(\frac{1}{2} + i_y \Big) \delta_y:  i_y \in \{0, \dots, J_y - 1\} \right\}
\end{align*}
and 
$$\delta_x \triangleq \frac{X_{\textnormal{max}} - X_{\textnormal{min}}}{J_x}, \quad \delta_y \triangleq \frac{Y_{\textnormal{max}} - Y_{\textnormal{min}}}{J_y}. $$
The $\sigma_j$'s in \eqref{eqn:Gaussian_RBF} are chosen to be equal to $\sigma_j = \max(\delta_x,\delta_y), \forall j$. The $\beta_j$'s used in \eqref{field_model_RBF} are the optimal values computed according to Lemma \ref{lemma:optimal_beta_RBF}. 

In the figures we show two performance measures, 1) the MSE, and 2) the structural similarity (SSIM) index, which originated in \cite{WangBovikSheikhSimoncelli} and has been widely adopted in the image processing community. The structural similarity index is a measure of the similarity between two images. In our case, we can regard $\Phi = \{\phi_d(I_x, I_y): I_x = 0,\dots, N_x - 1, I_y = 0, \dots, N_y - 1 \}$ and $\tilde{\Phi} = \{\tilde{\phi}_d(I_x, I_y): I_x = 0,\dots, N_x - 1, I_y = 0, \dots, N_y - 1 \}$ as the image representations of the true and approximated fields respectively, and compute the SSIM between these two images. The SSIM gives a scalar value between 0 and 1, with $\textnormal{SSIM} = 1$ if the two images to be compared are identical. 
We refer to \cite{WangBovikSheikhSimoncelli,WangBovik_MSE} for the specific equations used to compute the SSIM. 

We see from Figs.\ref{fig:field_seed341_DCT}-\ref{fig:field_seed343_RBF} that as more modes (for Fourier component model) or basis functions (for RBF model) are used, the approximations to the true field improves.  When using a smaller number of modes / basis functions the RBF model seems to give better approximations than the Fourier component model, while for larger numbers of modes / basis functions the two approaches perform similarly. We also observe that for these two examples, using a relatively small number of modes (when compared to the total number of modes of 10000) or basis functions will still result in a qualitatively reasonable approximation to the true field.

Although the Fourier component model does not seem to offer a significant advantage in terms of approximation quality, there are other reasons where one may consider its use. One advantage of the Fourier model is that it provides a natural way to control the number of model parameters (the coefficients $C(u,v)$) that need to be estimated, in that we simply choose however many modes we wish to retain, whereas with the RBF model one would need to also choose the locations $\mathbf{c}_j$ to place the basis functions and what the values of $\sigma_j$ should be. Additionally, if we want to refine our field model with finer structure by including more model parameters, in the Fourier component model we can reuse any previous estimates  (and further improve them) of the lower order modes, since these remain the same in a model with more modes, whereas in the RBF model one would likely need  to recalculate the estimates of all the parameter values when more basis functions are utilized.

\section{Estimation of Fourier Components}
\label{sec:DCT_estimation}
We now return to the problem of estimating the coefficients $C(u,v), \, (u,v) \in \tilde{\mathcal{U}}$ stated in Section~\ref{sec:problem_statement}. 
Given a set of modes to be retained $\tilde{\mathcal{U}}$, of cardinality $\tilde{N}$, define an ordering on $\tilde{\mathcal{U}}$ indexed by $j \in \{0, \dots, \tilde{N}-1\}$. For instance, the elements of $\tilde{\mathcal{U}}$ could be sorted in lexicographic order.  
Denote the $j$-th element under this ordering  by $(u_j, v_j)$, and define 
$$ C_j \triangleq C(u_j,v_j).$$

%Given $u \in \{0,\dots,\tilde{N}_x-1\}$ and $v \in \{0, \dots, \tilde{N}_y-1\}$, define Alternatively, one can define $j \triangleq v + u \tilde{N}_y$, and conversely $u_j  \triangleq j \textnormal{ mod } \tilde{N}_y $, $v_j \triangleq \lfloor j/\tilde{N}_y \rfloor$. 
%$$j \triangleq u + v \tilde{N}_x$$
%and note that $j \in \{0,\dots, \tilde{N}_x  \tilde{N}_y-1\}$. Conversely, given  $j \in \{0,\dots, \tilde{N}_x  \tilde{N}_y-1\}$, define
%$$ u_j \triangleq  \lfloor j/\tilde{N}_x \rfloor, \quad v_j \triangleq j \textnormal{ mod } \tilde{N}_x$$ 
%Now define
%$$ C_j \triangleq C(u_j,v_j).$$
%Forming $C_j = C(u_j, v_j), j = 0, \dots, \tilde{N}_x \tilde{N}_y - 1$ then corresponds to applying the $\mathtt{vec}$ operation \cite{HornJohnson2} on the matrix with entries $C(u,v)$.

Then we can express
\begin{align*}
\tilde{\phi}_d(I_x,I_y) & = \sum_{(u,v) \in \tilde{\mathcal{U}}} \alpha_x(u) \alpha_y(v) C(u,v) \cos \left(\frac{(2I_x+1)\pi u}{2 N_x} \right) \cos \left(\frac{(2I_y+1)\pi v}{2 N_y} \right)
\end{align*}
in the alternative form 
\begin{equation}
\label{field_model_vector}
 \tilde{\phi}_d(I_x,I_y)  = \sum_{j=0}^{\tilde{N}-1} \alpha_x(u_j) \alpha_y(v_j) C_j \cos \left(\frac{(2I_x+1)\pi u_j}{2 N_x} \right) \cos \left(\frac{(2I_y+1)\pi v_j}{2 N_y} \right),
\end{equation}
which is a linear function of $(C_0, \dots, C_{\tilde{N} -1})$.

%In \cite{LeongZamani_SP,LeongZamaniShames,TranGarratt} (see also \cite{LaSheng,LaShengChen,MorelandeSkvortsov,RazakSukumarChung_journal} for similar models), the field is modelled as \eqref{field_model_RBF}.  
Comparing \eqref{field_model_vector} with the RBF field model \eqref{field_model_RBF}, we see that they are both linear functions of the parameters that are to be estimated. Thus the algorithms developed in e.g. \cite{LaSheng,LaShengChen,RazakSukumarChung_journal, LeongZamani_SP,LeongZamaniShames,TranGarratt,MorelandeSkvortsov}  for estimating fields can in principle also be adapted to work for our field model \eqref{field_model_vector}, under their various assumed measurement models.

\begin{remark}
\label{remark:param_magnitudes}
The DCT coefficients which we are trying to estimate can be of substantially different orders of magnitude, with the higher order components being much smaller in magnitude than the ``DC'' component corresponding to  $u=v=0$, due to the result that the DCT coefficients decay as $O \big(\frac{1}{(u+1)^2 + (v+1)^2} \big)$ \cite{YamataniSaito}.
%For instance, the field in Fig. \ref{fig:field_1_modes} has $C(0,0) = 56.31$, $C(1,1) = 7.749$, $C(2,2) = -2.065$, $C(3,3) = 1.040$, $C(4,4) = 0.544$, $C(5,5) = 0.0156$, $C(6,6) = 0.0284$, $C(7,7) = 0.0075$, \dots. 
In order to estimate parameters with such large differences in magnitude, it is desirable to appropriately scale the parameters that are to be estimated, see \eqref{eqn:param_scaling} below. 
\end{remark}

\subsection{Estimation of Fourier Components Using Quantized Measurements}
\label{sec:DCT_estimation_quantized_measurements}
In this subsection we describe an approach to estimating the parameters $C(u,v), \, u=0, \dots, \tilde{N} - 1$, which assumes the quantized measurement model \eqref{quantized_measurement_model}-\eqref{eqn:quantizer}, with the parameters estimated recursively. 
%A sequential Monte Carlo (SMC) approach is presented in Section \ref{sec:SMC_approach}, while an online optimization approach  is presented in Section \ref{sec:online_optim_approach}. The algorithms are similar to those of \cite{LeongZamani_SP} and \cite{LeongZamaniShames} respectively,  
The algorithm uses an online optimization approach similar to \cite{LeongZamaniShames},
however in this paper we will generalize \cite{LeongZamaniShames} from binary measurements to multi-level quantized measurements, and also extend the approach to handle time-varying fields. 

%\subsection{Sequential Monte Carlo approach}
%\label{sec:SMC_approach}
%In the measurement model \eqref{quantized_measurement_model}-\eqref{eqn:quantizer}, assume the noise has Gaussian distribution
%$n(.,.) \sim \mathcal{N}(0, \sigma_n^2)$. The noise variance $\sigma_n^2$ is assumed unknown as the algorithm can also estimate $\sigma_n^2$. Recalling the observation in Remark \ref{remark:param_magnitudes}, we consider the following scaling of the DCT coefficients:
%\begin{equation}
%\label{eqn:param_scaling}
%\theta_j \triangleq \big((u_j+1)^2 + (v_j+1)^2 \big) C_j.
%\end{equation}
%We then define 
%$$\bm{\theta} \triangleq (\theta_0, \dots, \theta_{\tilde{N}_x \tilde{N}_y-1}, \log \sigma_n)$$
%as the vector of parameters that are estimated. 

%Let $z_k$ denote the measurement,  and $(I_{x,k}, I_{y,k})$ the position index, at time/iteration $k$. For notational compactness we also denote
%\begin{equation}
%\label{eqn:I_x_vector}
%\bm{I}_{\textbf{x},k} \triangleq (I_{x,k}, I_{y,k})
%\end{equation}
%and  
%\begin{equation}
%\label{eqn:K_vector}
%\mathbf{K}(\bm{I}_{\textbf{x},k} ) \triangleq \left[\begin{array}{cccc} K_0 (\bm{I}_{\textbf{x},k} ) & K_1 (\bm{I}_{\textbf{x},k} ) &  \dots & K_{\tilde{N}_x \tilde{N}_y-1}(\bm{I}_{\textbf{x},k} ) \end{array} \right]^T,
%\end{equation}
%where
%\begin{equation}
%\label{eqn:K_vector_components}
%K_j (\bm{I}_{\textbf{x},k} ) \triangleq \frac{\alpha_x(u_j) \alpha_y(v_j) }{(u_j\!+\!1)^2 \!+\! (v_j\!+\!1)^2} \cos \Big(\frac{(2I_{x,k}+1)\pi u_j}{2 N_x} \Big) \cos \Big(\frac{(2I_{y,k}+1)\pi v_j}{2 N_y} \Big).
%\end{equation}
%Denote 
%\begin{equation}
%\label{eqn:z_all}
%z_{1:k} \triangleq \{z_1, \dots, z_k\}    
%\end{equation}
%as the set of measurements collected up to time $k$, with corresponding position indices
%\begin{equation}
%\label{eqn:I_x_all}
%\bm{I}_{\textbf{x},1:k} \triangleq \{(I_{x,1:k}, I_{y,1:k})\}.
%\end{equation}
 
%\begin{algorithm}
%\caption{Estimation of Fourier components using SMC approach}
%\label{alg:DCT_SMC_time_varying}
%\begin{algorithmic}[1]
%\State \textbf{Algorithm Parameters}: $N \in \mathbb{N}$, $a \in (0,1)$, $h = \sqrt{1-a^2}$, prior pdf $p_0(\bm{\theta})$, $b \in [0,1]$, distribution of new parameter values $p_{\bm{\theta}_{k-1}}(.)$
%\State \textbf{Inputs}: Initial location index $\bm{I}_{\textbf{x},1}$
%\State \textbf{Outputs}: Particles $\{\bm{\theta}_k^{(i)}\}$ and weights $\{ \bm{w}_k^{(i)} \}$

%\State Sample particles $\bm{\theta}_0^{(i)}, i=1,\dots,N$ from $p_0(\bm{\theta})$, and assign weights $ \bm{w}_0^{(i)} = \frac{1}{N}, i=1,\dots,N$
%\For{$k=1,2,\dots,$}
%	\State Observe $z_k(\bm{I}_{\textbf{x},k})$ at location index $\bm{I}_{\textbf{x},k}$
%	\For{$i=1,\dots,N$}  \label{line_loop_start:Nemeth}
%		\State Compute $ \textbf{m}_{k-1}^{(i)} = a \bm{\theta}_{k-1}^{(i)}  + (1-a) \bm\bar{\bm{\theta}}_{k-1}$ where $ \bar{\bm{\theta}}_{k-1}  = \sum_{i=1}^N \bm{w}_{k-1}^{(i)} \bm{\theta}_{k-1}^{(i)}$
		 
%		\State Assign $\bm{w}_{1,k}^{(i)} \propto  p(z_k(\bm{I}_{\textbf{x},k})|\textbf{m}_{k-1}^{(i)}; \bm{I}_{\textbf{x},k}) \bm{w}_{k-1}^{(i)}$ \label{line:Nemeth:w_1}
%		\State Sample $\bm{\gamma}_k^{(i)} \sim p_{\bm{\theta}_{k-1}^{(i)}} (.) $
%		\State Assign $\bm{w}_{2,k}^{(i)} \propto  p(z_k(\bm{I}_{\textbf{x},k})| \bm{\gamma}_k^{(i)}; \bm{I}_{\textbf{x},k}) \bm{w}_{k-1}^{(i)}$ \label{line:Nemeth:w_2}
%\EndFor
%	\State Normalize $\{\bm{w}_{1,k}^{(i)} \}$ and $\{\bm{w}_{2,k}^{(i)} \}$ such that $\sum_{i=1}^N \bm{w}_{1,k}^{(i)} = 1$ and $\sum_{i=1}^N \bm{w}_{2,k}^{(i)} = 1$ 
%	\State Sample $N$ times with replacement a set of indices $\{i^-: i=1,\dots,N\} $, from a distribution with probabilities $$\mathbb{P}(i^- = j) = \left\{ \begin{array}{cl} (1-b) \bm{w}_{1,k}^{(j)}, & j \in \{1, \dots,N\} \\  b \bm{w}_{2,k}^{(j-N)}, & j \in \{N+1, \dots,2N\}\end{array} \right.$$
%	\For{$i^- \in \{1,\dots,N\}$}
%		\State Sample  $\bm{\theta}_k^{(i)} \sim \mathcal{N}(\textbf{m}_{k-1}^{(i^-)}, h^{2} \textbf{V}_{k-1})$, where 		$\textbf{V}_{k-1}  = \sum_{i=1}^N \bm{w}_{k-1}^{(i)} (\bm{\theta}_{k-1}^{(i)} - \bar{\bm{\theta}}_{k-1}) (\bm{\theta}_{k-1}^{(i)} - \bar{\bm{\theta}}_{k-1})^T$  
%		\State Assign weights $ \bm{w}_k^{(i)} \propto \frac{p(z_k(\bm{I}_{\textbf{x},k})|\bm{\theta}_k^{(i)}; \bm{I}_{\textbf{x},k})}{p(z_k(\bm{I}_{\textbf{x},k})|\textbf{m}_{k-1}^{(i^-)}; \bm{I}_{\textbf{x},k})}$	\label{line:Nemeth_weights1}
%	\EndFor
%	\For{$i^- \in \{N+1,\dots,2N\}$}
%		\State Set  $\bm{\theta}_k^{(i)} = \bm{\gamma}_k^{(i^- - N)}$
%		\State Assign weights $ \bm{w}_k^{(i)} \propto \frac{p(z_k(\bm{I}_{\textbf{x},k}))|\bm{\theta}_k^{(i)}; \bm{I}_{\textbf{x},k})}{p(z_k(\bm{I}_{\textbf{x},k})|\bm{\gamma}_{k}^{(i^- -N)}; \bm{I}_{\textbf{x},k})}$	\label{line:Nemeth_weights2}
%	\EndFor
%	\State Normalize $\{ \bm{w}_k^{(i)} \}$ such that $\sum_{i=1}^N \bm{w}_{k}^{(i)} = 1$   \label{line_loop_end:Nemeth}
%	\State Determine  $\bm{I}_{\textbf{x},k+1} = \texttt{ActiveSensingSMC}(\bm{I}_{\textbf{x},k}, \{\bm{\theta}_k^{(i)}\})$  using Algorithm \ref{alg:active_sensing_SMC} 
%\EndFor
%\end{algorithmic}
%\end{algorithm} 

%We present in Algorithm \ref{alg:DCT_SMC_time_varying} a sequential Monte Carlo (SMC) \cite{Doucet_book} based algorithm that approximates the posterior pdf $ p(\bm{\theta} | z_{1:k}; \bm{I}_{\textbf{x},1:k})$ by a set of particles $\bm{\theta}_k^{(i)}, i=1,\dots,N$, and associated weights $\bm{w}_k^{(i)}, i=1,\dots,N$. 
%Denoting $\bm{\theta}_k^{(i)} \triangleq (\theta_{0,k}^{(i)}, \dots, \theta_{\tilde{N}_x \tilde{N}_y,k}^{(i)})$,  conditional mean estimates at iteration $k$ can be computed from the particle approximation as:
% \begin{align*}
% \hat{C}_{j,k} &= \sum_{i=1}^N \frac{ \bm{w}_{k}^{(i)} \theta_{j,k}^{(i)} }{(u_j+1)^2 + (v_j+1)^2}, \quad j = 0,\dots,\tilde{N}_x \tilde{N}_y - 1,   \quad \quad \hat{\sigma}_{n,k}^2 = \sum_{i=1}^N \bm{w}_k^{(i)} \exp(2 \theta_{\tilde{N}_x \tilde{N}_y,k}^{(i)}). \end{align*}
%In lines \ref{line:Nemeth_weights1} and \ref{line:Nemeth_weights2} of Algorithm \ref{alg:DCT_SMC_time_varying}, the likelihood functions are computed as:
%\begin{align} 
%p(z_k(\bm{I}_{\textbf{x},k}) = 0|\bm{\theta}_k^{(i)}; \bm{I}_{\textbf{x},k} )  &  =  \Phi \bigg( \frac{\tau_0 - \bm{\theta}_k^{(i)T} \mathbf{K}(\bm{I}_{\textbf{x},k} )}{\exp(\theta_{\tilde{N}_x \tilde{N}_y, k}^{(i)})}\bigg)\nonumber \\
% p(z_k(\bm{I}_{\textbf{x},k}) = l|\bm{\theta}_k^{(i)}; \bm{I}_{\textbf{x},k} ) & =  \Phi \bigg( \frac{\tau_l - \bm{\theta}_k^{(i)T} \mathbf{K}(\bm{I}_{\textbf{x},k} )}{\exp(\theta_{\tilde{N}_x \tilde{N}_y, k}^{(i)})}\bigg) -  \Phi \bigg( \frac{\tau_{l-1} - \bm{\theta}_k^{(i)T} \mathbf{K}(\bm{I}_{\textbf{x},k} )}{\exp(\theta_{\tilde{N}_x \tilde{N}_y, k}^{(i)})}\bigg), \quad l = 1, \dots, L-2 \nonumber 
%\\
% p(z_k(\bm{I}_{\textbf{x},k}) = L-1 | \bm{\theta}_k^{(i)} ; \bm{I}_{\textbf{x},k})  & = 1 -  \Phi \bigg( \frac{\tau_{L-2} - \bm{\theta}_k^{(i)T} \mathbf{K}(\bm{I}_{\textbf{x},k} )}{\exp(\theta_{\tilde{N}_x \tilde{N}_y, k}^{(i)})}\bigg), \label{likelihood_fn_multilevel}
%\end{align}
%and similarly for $p (z_k(\bm{I}_{\textbf{x},k}) | \textbf{m}_k^{(i)} ; \bm{I}_{\textbf{x},k}) $ and $p (z_k(\bm{I}_{\textbf{x},k}) | \bm{\gamma}_k^{(i)} ; \bm{I}_{\textbf{x},k}) $, where $\Phi(x) \triangleq \int_{-\infty}^{x} \frac{1}{\sqrt{2\pi}} \exp\left( - \frac{t^2}{2} \right) dt$ is the cdf of the standard normal distribution $\mathcal{N}(0,1)$. Expression \eqref{likelihood_fn_multilevel} generalizes the likelihood functions for the case of binary measurements considered in \cite{LeongZamani_SP}.
%Finally, we point out that Algorithm \ref{alg:DCT_SMC_time_varying} can be used to estimate slowly varying fields or fields with occasional abrupt changes in parameters \cite{NemethFearnheadMihaylova,LeongZamani_SP}.

%\subsubsection{Measurement location selection using active sensing}

%For choosing the locations in which to make measurements, ``active sensing'' algorithms can be used. Given position $\mathbf{x} \in \mathcal{S}$, denote $\bm{I}_{\textnormal{closest}} (\mathbf{x})$ as the closest location index $(I_x, I_y)$ to $\mathbf{x}$. In Algorithm \ref{alg:active_sensing_SMC} we present the active sensing mechanism from \cite{LeongZamani_SP}, which is an information maximization based approach \cite{RisticSkvortsovGunatilaka}, suitably generalized to the multi-level quantized measurement model \eqref{quantized_measurement_model}-\eqref{eqn:quantizer}. It will return the next location index $\bm{I}_{\textbf{x},k+1}$ to travel to, given the current location index $\bm{I}_{\textbf{x},k}$ and current set of particles $\{\bm{\theta}_k^{(i)}\}$. 

%\begin{algorithm}[t]
%\caption{Active sensing algorithm for SMC approach: $\bm{I}_{\textbf{x},k+1} = \texttt{ActiveSensingSMC}(\bm{I}_{\textbf{x},k}, \{\bm{\theta}_k^{(i)}\})$}
%\label{alg:active_sensing_SMC}
%\begin{algorithmic}[1]
%\State \textbf{Algorithm Parameters}: $\varepsilon \geq 0$, $\alpha \in [0,\infty) \backslash \{1\}$, $\rho_0 \geq 0$, $N_\rho \in \mathbb{N}$, $N_d \in \mathbb{N}$, search region $\mathcal{S}$
%\State \textbf{Inputs}:    $\bm{I}_{\textbf{x},k}$, $\{\bm{\theta}_k^{(i)}\}$
%\State \textbf{Output}: Next measurement location index $\bm{I}_{\textbf{x},k+1}$
%\State Set $\mathbf{x}_k = \big(X_{\textnormal{min}} + \left(1/2 + I_{x,k} \right) \Delta_x, Y_{\textnormal{min}} + \left(1/2 + I_{y,k} \right) \Delta_y \big)$
%	\State With probability $\varepsilon$ set $\bm{I}_{\textbf{x},k+1} $ to a random location index in $\{0,\dots,\tilde{N}_x-1\} \times \{0,\dots,\tilde{N}_y-1\}$, otherwise set $$\bm{I}_{\textbf{x},k+1} = \textrm{arg} \max\limits_{\bm{I}_{\textbf{x}'} \in \mathcal{I}_k} \frac{1}{\alpha-1} \sum_{z_{k+1}=0}^{L-1} \gamma_1(z_{k+1}|\bm{I}_{\textbf{x}'}) \ln  \frac{\gamma_\alpha(z_{k+1}|\bm{I}_{\textbf{x}'})}{(\gamma_1(z_{k+1}|\bm{I}_{\textbf{x}'}))^\alpha}$$ 
%	where
%	\begin{align*}
%	&\mathcal{X}_k = \bigg\{ \mathbf{x}_k + \Big(n \rho_0 \cos\Big( \frac{2\pi\ell}{N_d}\Big), n \rho_0 \sin \Big( \frac{2\pi\ell}{N_d} \Big) \Big): %\\ &\quad\quad\quad\quad\quad 
%  n \in\{ 0,\dots,N_\rho\}, \ell \in \{ 0, 1, \dots, N_d - 1\} \bigg\} \cap \mathcal{S} \\  & \mathcal{I}_k = \{\bm{I}_{\textnormal{closest}} (\mathbf{x}):  \mathbf{x} \in \mathcal{X}_k  \}
% \\	&\gamma_\alpha(z_{k+1} | \bm{I}_{\textbf{x}'}) = \frac{1}{N} \sum_{i=1}^N p(z_{k+1}|\bm{\theta}_k^{(i)}; \bm{I}_{\textbf{x}'})^\alpha
%	\end{align*}
%\end{algorithmic}
%\end{algorithm} 

%\subsection{Online optimization approach}
%\label{sec:online_optim_approach}
For the measurement model \eqref{quantized_measurement_model}-\eqref{eqn:quantizer}, $n(\bm{\cdot},\bm{\cdot})$ is taken as zero mean noise (not necessarily Gaussian). 
Recalling the observation in Remark \ref{remark:param_magnitudes}, we consider the following scaling of the DCT coefficients:
\begin{equation}
\label{eqn:param_scaling}
\beta_j \triangleq \big((u_j+1)^2 + (v_j+1)^2 \big) C_j,
\end{equation}
and define 
$$\bm{\beta} \triangleq (\beta_0, \dots, \beta_{\tilde{N}-1})$$
as the vector of parameters that are to be estimated. 

We first introduce some notation. 
Let $z_k$ denote the measurement,  and $(I_{x,k}, I_{y,k})$ the position index, at time/iteration $k$. For notational compactness we also denote
\begin{equation}
\label{eqn:I_x_vector}
\bm{I}_{\textbf{x},k} \triangleq (I_{x,k}, I_{y,k})
\end{equation}
and  
\begin{equation}
\label{eqn:K_vector}
\mathbf{K}(\bm{I}_{\textbf{x},k} ) \triangleq \left[\begin{array}{cccc} K_0 (\bm{I}_{\textbf{x},k} ) & K_1 (\bm{I}_{\textbf{x},k} ) &  \dots & K_{\tilde{N}-1}(\bm{I}_{\textbf{x},k} ) \end{array} \right]^T,
\end{equation}
where
\begin{equation}
\label{eqn:K_vector_components}
K_j (\bm{I}_{\textbf{x},k} ) \triangleq \frac{\alpha_x(u_j) \alpha_y(v_j) }{(u_j\!+\!1)^2 \!+\! (v_j\!+\!1)^2} \cos \Big(\frac{(2I_{x,k}+1)\pi u_j}{2 N_x} \Big) \cos \Big(\frac{(2I_{y,k}+1)\pi v_j}{2 N_y} \Big).
\end{equation}
Denote 
\begin{equation}
\label{eqn:z_all}
z_{1:k} \triangleq \{z_1, \dots, z_k\}    
\end{equation}
as the set of measurements collected up to time $k$, with corresponding position indices
\begin{equation}
\label{eqn:I_x_all}
\bm{I}_{\textbf{x},1:k} \triangleq \{(I_{x,1:k}, I_{y,1:k})\}.
\end{equation}

The idea is to recursively estimate $\bm{\beta}$ by trying to minimize a cost function 
$$J_k (\bm{\beta}; \bm{I}_{\textbf{x},1:k}, z_{1:k}) = \sum_{t=0}^k g_t (\bm{\beta}; \bm{I}_{\textbf{x},t}, z_t)$$ using online optimization techniques \cite{LesageLandryTaylorShames}.
For binary measurements \eqref{binary_measurement_model}, the following per stage cost function from \cite{LeongZamaniShames} can be used:
\begin{equation}
\label{eqn:per_stage_cost_binary}
g_t (\bm{\beta}; \bm{I}_{\textbf{x}}, z) = \left\{\begin{array}{cc} \log(1+\exp(\eta (\bm{\beta}^T \mathbf{K}(\bm{I}_{\textbf{x}}) - \tau))), & z = 0 \\ 
\log(1+\exp(-\eta (\bm{\beta}^T \mathbf{K}(\bm{I}_{\textbf{x}}) - \tau))), & z = 1 
\end{array}
\right.
\end{equation}
where $\eta>0$ is a parameter in the logistic function $\ell(x) \triangleq 1/(1+\exp(\eta x))$, where larger values of $\eta$ will more closely approximate the function $\mathds{1env}(x > 0)$. The cost function \eqref{eqn:per_stage_cost_binary} is similar to cost functions used in binary logistic regression problems \cite[p.516]{CalafioreElGhaoui}. In the current work, we wish to define a cost suitable for multi-level quantized measurements. Note that there are cost functions used in multinomial logistic regression problems \cite{Murphy_book1}, however they are unsuitable for our problem as they usually involve multiple sets of parameters for each possible output $z$, whereas here we just have a single set of parameters $\bm{\beta}$. 

To motivate our cost function, let us look more closely at the binary measurements cost function \eqref{eqn:per_stage_cost_binary}. In the case where the measurement  $z_t$ at time $t$ and position index $\bm{I}_{\textbf{x},t}$ is equal to 0, the cost $g_t (\bm{\beta}; \bm{I}_{\textbf{x},t}, z_t)$ will be small if  $\bm{\beta}^T \mathbf{K}(\bm{I}_{\textbf{x},t}) $ is less than the quantizer threshold $\tau$, and large otherwise. Similarly, when  $z_t=1$,  $g_t (\bm{\beta}; \bm{I}_{\textbf{x},t}, z_t)$ will be small if  $\bm{\beta}^T \mathbf{K}(\bm{I}_{\textbf{x},t}) $ is greater than $\tau$, and large otherwise. For the case of multi-level quantized measurements  with $L$ levels  given by \eqref{eqn:quantizer}, we would like to have a cost function such that 1) when $z_t=0$, $g_t (\bm{\beta}; \bm{I}_{\textbf{x},t}, z_t)$ is small for $\bm{\beta}^T \mathbf{K}(\bm{I}_{\textbf{x},t}) < \tau_0$, and large otherwise, 2)  when $z_t=l, \, l \in \{1,\dots,L-2\}$, $g_t (\bm{\beta}; \bm{I}_{\textbf{x},t}, z_t)$ is small for $\tau_{l-1} \leq \bm{\beta}^T \mathbf{K}(\bm{I}_{\textbf{x},t}) < \tau_l$, and large otherwise, and 3) when $z_t=L-1$,  $g_t (\bm{\beta}; \bm{I}_{\textbf{x},t}, z_t)$ is small for $\bm{\beta}^T \mathbf{K}(\bm{I}_{\textbf{x},t}) > \tau_{L-2}$, and large otherwise. In this paper we will choose the following per stage cost function, which can be easily checked to satisfy these three requirements:
\begin{equation}
\label{eqn:per_stage_cost_multilevel}
g_t (\bm{\beta}; \bm{I}_{\textbf{x}}, z) \triangleq \left\{\begin{array}{ll} \log(1+\exp(\eta (\bm{\beta}^T \mathbf{K}(\bm{I}_{\textbf{x}}) - \tau_0))), & z = 0 \\ 
\log(1+\exp(-\eta (\bm{\beta}^T \mathbf{K}(\bm{I}_{\textbf{x}}) - \tau_{l-1}))) & \\
 \quad + \log(1+\exp(\eta (\bm{\beta}^T \mathbf{K}(\bm{I}_{\textbf{x}}) - \tau_{l}))), & z = l \in \{1,\dots,L-2\}\\
\log(1+\exp(-\eta (\bm{\beta}^T \mathbf{K}(\bm{I}_{\textbf{x}}) - \tau_{L-2}))), & z = L-1. 
\end{array}
\right.
\end{equation}
We remark that \eqref{eqn:per_stage_cost_multilevel} reduces to \eqref{eqn:per_stage_cost_binary} when the measurements are binary. 

Now that the per stage cost \eqref{eqn:per_stage_cost_multilevel} has been defined, we will present the online estimation algorithm. First, the gradient of $g_t(\bm{\cdot};\bm{\cdot},\bm{\cdot})$ can be derived as 
\begin{equation}
\label{eqn:per_stage_gradient}
\nabla g_t (\bm{\beta}; \bm{I}_{\textbf{x}}, z) = \left\{\begin{array}{ll} 
\frac{\eta}{1+\exp(-\eta (\bm{\beta}^T \mathbf{K}(\bm{I}_{\textbf{x}}) - \tau_0))} \mathbf{K}(\bm{I}_{\textbf{x}}), & z = 0 \\
\Big( \frac{-\eta}{1+\exp(\eta (\bm{\beta}^T \mathbf{K}(\bm{I}_{\textbf{x}}) - \tau_{l-1}))} & \\ \quad \quad + \frac{\eta}{1+\exp(-\eta (\bm{\beta}^T \mathbf{K}(\bm{I}_{\textbf{x}}) - \tau_l))} \Big) \mathbf{K}(\bm{I}_{\textbf{x}}), & z = l \in \{1,\dots,L-2\}\\
 \frac{-\eta}{1+\exp(\eta (\bm{\beta}^T \mathbf{K}(\bm{I}_{\textbf{x}}) - \tau_{L-2}))} \mathbf{K}(\bm{I}_{\textbf{x}}), & z = L-1, 
\end{array}
\right.
\end{equation}
while the Hessian of $g_t(\bm{\cdot};\bm{\cdot},\bm{\cdot})$ can be derived as 
\begin{align}
& \nabla^2 g_t (\bm{\beta}; \bm{I}_{\textbf{x}}, z) \nonumber \\& = \left\{\begin{array}{ll} \frac{\eta^2 \exp(-\eta (\bm{\beta}^T \mathbf{K}(\bm{I}_{\textbf{x}}) - \tau_0))}{(1+\exp(-\eta (\bm{\beta}^T \mathbf{K}(\bm{I}_{\textbf{x}}) - \tau_0)))^2} \mathbf{K}(\bm{I}_{\textbf{x}}) \mathbf{K}(\bm{I}_{\textbf{x}})^T, & z = 0 \\
\Big( \frac{\eta^2 \exp(\eta (\bm{\beta}^T \mathbf{K}(\bm{I}_{\textbf{x}}) - \tau_{l-1}))}{(1+\exp(\eta (\bm{\beta}^T \mathbf{K}(\bm{I}_{\textbf{x}}) - \tau_{l-1})))^2} & \\ \quad \quad + \frac{\eta^2 \exp(-\eta (\bm{\beta}^T \mathbf{K}(\bm{I}_{\textbf{x}}) - \tau_l))}{(1+\exp(-\eta (\bm{\beta}^T \mathbf{K}(\bm{I}_{\textbf{x}}) - \tau_l)))^2} \Big)  \mathbf{K}(\bm{I}_{\textbf{x}}) \mathbf{K}(\bm{I}_{\textbf{x}})^T, & z = l \in \{1,\dots,L-2\}\\
 \frac{\eta^2 \exp(\eta (\bm{\beta}^T \mathbf{K}(\bm{I}_{\textbf{x}}) - \tau_{L-2}))}{(1+\exp(\eta (\bm{\beta}^T \mathbf{K}(\bm{I}_{\textbf{x}}) - \tau_{L-2})))^2} \mathbf{K}(\bm{I}_{\textbf{x}}) \mathbf{K}(\bm{I}_{\textbf{x}})^T, & z = L-1. 
\end{array}
\right. \nonumber \\
& = \left\{\begin{array}{ll} 
\frac{\eta^2 \exp(\eta (\bm{\beta}^T \mathbf{K}(\bm{I}_{\textbf{x}}) - \tau_0))}{(1+\exp(\eta (\bm{\beta}^T \mathbf{K}(\bm{I}_{\textbf{x}}) - \tau_0)))^2} \mathbf{K}(\bm{I}_{\textbf{x}}) \mathbf{K}(\bm{I}_{\textbf{x}})^T, & z = 0 \\
\Big( \frac{\eta^2 \exp(\eta (\bm{\beta}^T \mathbf{K}(\bm{I}_{\textbf{x}}) - \tau_{l-1}))}{(1+\exp(\eta (\bm{\beta}^T \mathbf{K}(\bm{I}_{\textbf{x}}) - \tau_{l-1})))^2} & \\ \quad \quad + \frac{\eta^2 \exp(\eta (\bm{\beta}^T \mathbf{K}(\bm{I}_{\textbf{x}}) - \tau_l))}{(1+\exp(\eta (\bm{\beta}^T \mathbf{K}(\bm{I}_{\textbf{x}}) - \tau_l)))^2} \Big)  \mathbf{K}(\bm{I}_{\textbf{x}}) \mathbf{K}(\bm{I}_{\textbf{x}})^T, & z = l \in \{1,\dots,L-2\}\\
 \frac{\eta^2 \exp(\eta (\bm{\beta}^T \mathbf{K}(\bm{I}_{\textbf{x}}) - \tau_{L-2}))}{(1+\exp(\eta (\bm{\beta}^T \mathbf{K}(\bm{I}_{\textbf{x}}) - \tau_{L-2})))^2} \mathbf{K}(\bm{I}_{\textbf{x}}) \mathbf{K}(\bm{I}_{\textbf{x}})^T, & z = L-1. 
\end{array} \right.  \label{eqn:per_stage_Hessian}
\end{align}
An approximate online Newton method \cite{LeongZamaniShames} for estimating the parameters $\bm{\beta}$ is now given by:
\begin{equation}
\label{eqn:approx_ONM1}
 \hat{\bm{\beta}}_{k+1} = \hat{\bm{\beta}}_k - \left( H_k (\hat{\bm{\beta}}_k; \bm{I}_{\textbf{x},1:k}, z_{1:k})  \right)^{-1} G_k (\hat{\bm{\beta}}_k; \bm{I}_{\textbf{x},k}, z_k), 
\end{equation}
where
\begin{align}
G_k (\hat{\bm{\beta}}_k; \bm{I}_{\textbf{x},k}, z_k) & = \nabla g_k (\hat{\bm{\beta}}_k; \bm{I}_{\textbf{x},k}, z_k)  \nonumber \\
H_k(\hat{\bm{\beta}}_{k}; \bm{I}_{\textbf{x},1:k}, z_{1:k}) & =  H_{k-1}(\hat{\bm{\beta}}_{k-1}; \bm{I}_{\textbf{x},1:k-1}, z_{1:k-1}) + \nabla^2 g_k (\hat{\bm{\beta}}_k; \bm{I}_{\textbf{x},k}, z_k) \nonumber \\
H_0 (\hat{\bm{\beta}}_{0}) & = \varsigma I. \label{eqn:approx_ONM2}
\end{align}
The initialization $H_0 (\hat{\bm{\beta}}_{0}) = \varsigma I$  is a Levenberg-Marquardt type modification \cite{ChongZak} to ensure that the matrices $\{H_k\}$ are always non-singular.\footnote{In \cite{LeongZamaniShames} this is equivalently expressed as a full rank initialization  on $\left(H_0(\hat{\bm{\beta}}_{0})\right)^{-1}$.}

In the case where the field (and hence the parameters $\bm{\beta})$ is time-varying, the algorithm \eqref{eqn:approx_ONM1}-\eqref{eqn:approx_ONM2} may not be able to respond quickly to changes in $\bm{\beta}$, due to all past Hessians (including Hessians from old fields) being used in the computation of $H_k(\hat{\bm{\beta}}_{k}; \bm{I}_{\textbf{x},1:k}, z_{1:k}) $ in \eqref{eqn:approx_ONM2}. To overcome this problem, we  will introduce a \emph{forgetting factor} \cite{ManolakisIngleKogon} into the algorithm, 
where the forgetting factor $\delta$ satisfies $0 < \delta \leq 1$, and typically chosen to be close to one. The final estimation procedure is summarized as Algorithm~\ref{alg:DCT_optim_time_varying}. Compared to \eqref{eqn:approx_ONM2}, we note that the Levenberg-Marquardt modification in Algorithm \ref{alg:DCT_optim_time_varying} is done at every time step by adding $\varsigma I$ to $\tilde{H}_k$, as we found that only doing it once at the beginning can lead to algorithm instability due to exponential decay of initial conditions when using a forgetting factor. We also remark that Algorithm \ref{alg:DCT_optim_time_varying} reduces to \eqref{eqn:approx_ONM1}-\eqref{eqn:approx_ONM2} when the forgetting factor $\delta = 1$. 

\begin{algorithm}
\caption{Estimation of Fourier components using online optimization approach}
\label{alg:DCT_optim_time_varying}
\begin{algorithmic}[1]
\State \textbf{Algorithm Parameters}:  Logistic function parameter $\eta > 0$, Levenberg-Marquardt parameter $\varsigma > 0$, forgetting factor $\delta \in (0,1]$
\State \textbf{Inputs}: Initial position index $\bm{I}_{\textbf{x},0}$
\State \textbf{Outputs}: Parameter estimates $\{ \hat{\bm{\beta}}_k \}$
\State Initialize $\tilde{H}_0(\hat{\bm{\beta}}_{0}) = \mathbf{0}$
\For{$k=0,1,2,\dots,$}
	\State Update estimates 
 \begin{align*}
\hat{\bm{\beta}}_{k+1} &= \hat{\bm{\beta}}_k - \left( H_k (\hat{\bm{\beta}}_k; \bm{I}_{\textbf{x},1:k}, z_{1:k}) \right)^{-1} G_k (\hat{\bm{\beta}}_k; \bm{I}_{\textbf{x},k}, z_k) \nonumber \\
G_k (\hat{\bm{\beta}}_k; \bm{I}_{\textbf{x},k}, z_k) & = \nabla g_k (\hat{\bm{\beta}}_k; \bm{I}_{\textbf{x},k}, z_k) \nonumber \\
\tilde{H}_k(\hat{\bm{\beta}}_{k}; \bm{I}_{\textbf{x},1:k}, z_{1:k}) & =  \delta \tilde{H}_{k-1}(\hat{\bm{\beta}}_{k-1}; \bm{I}_{\textbf{x},1:k-1}, z_{1:k-1}) + \nabla^2 g_k (\hat{\bm{\beta}}_k; \bm{I}_{\textbf{x},k}, z_k) \nonumber \\
H_k(\hat{\bm{\beta}}_{k}; \bm{I}_{\textbf{x},1:k}, z_{1:k}) & = \tilde{H}_k(\hat{\bm{\beta}}_{k}; \bm{I}_{\textbf{x},1:k}, z_{1:k})  + \varsigma I, \label{eqn:approx_ONM_time_varying}
\end{align*}
\,\,\,\,\,\,\, where $\nabla g_k (\bm{\cdot}; \bm{\cdot}, \bm{\cdot})$ and $ \nabla^2 g_k(\bm{\cdot}; \bm{\cdot}, \bm{\cdot})$ are computed using \eqref{eqn:per_stage_gradient}-\eqref{eqn:per_stage_Hessian}	
	
 \State Determine  $\bm{I}_{\textbf{x},k+1} =\texttt{ActiveSensing}(\bm{I}_{\textbf{x},k}, \hat{\bm{\beta}}_{k+1})$  using Algorithm \ref{alg:active_sensing_online_opt}
\EndFor
\end{algorithmic}
\end{algorithm} 

\subsection{Measurement Location Selection Using Active Sensing}
For choosing the positions in which to take measurements, an ``active sensing'' approach \cite{Kreucher_active_sensing,LaShengChen,RisticSkvortsovGunatilaka} can be used, which aims to cleverly choose the next position given information collected so far, in order to more quickly obtain a good estimate of the field. 

In the case of binary measurements, a method for choosing the next measurement location is proposed in \cite{LeongZamaniShames}, that tries to maximize the minimum eigenvalue of an ``expected Hessian'' term $H^+(\bm{I}_{\textbf{x}'})$ over candidate future position indices $\bm{I}_{\textbf{x}'}$. Formally, the problem is:
$$\bm{I}_{\textbf{x},k+1} = \textnormal{arg} \max\limits_{\bm{I}_{\textbf{x}'} \in \mathcal{I}_{k+1}} \lambda_{\min} (H^+(\bm{I}_{\textbf{x}'})),$$
where $\lambda_{\min} (H^+(\bm{I}_{\textbf{x}'}))$ is the minimum eigenvalue of  $H^+(\bm{I}_{\textbf{x}'})$, $\mathcal{I}_{k+1}$ is the set of possible future position indices\footnote{The set $ \mathcal{I}_{k+1}$ may, e.g., capture the set of reachable positions from the current state of the mobile sensor platform.}
and
\begin{equation}
\label{eqn:expected_Hessian_binary}
\begin{split}
 H^+(\bm{I}_{\textbf{x}'}) & \triangleq H_k(\bm{\hat{\beta}}_{k}; \bm{I}_{\textbf{x},1:k}, z_{1:k})  +  \frac{ \eta^2 \exp(\eta (\bm{\hat{\beta}}_{k+1}^T \mathbf{K}(\bm{I}_{\textbf{x}'}) -  \tau)) }{\big(1+\exp(\eta (\bm{\hat{\beta}}_{k+1}^T \mathbf{K}(\bm{I}_{\textbf{x}'}) -  \tau)) \big)^2}  \mathbf{K}(\bm{I}_{\textbf{x}'}) \mathbf{K}(\bm{I}_{\textbf{x}'})^T \mathbb{P}(z' = 0) \\
 & \quad +  \frac{ \eta^2 \exp(\eta (\bm{\hat{\beta}}_{k+1}^T \mathbf{K}(\bm{I}_{\textbf{x}'}) -  \tau)) }{\big(1+\exp(\eta (\bm{\hat{\beta}}_{k+1}^T \mathbf{K}(\bm{I}_{\textbf{x}'}) -  \tau)) \big)^2}  \mathbf{K}(\bm{I}_{\textbf{x}'}) \mathbf{K}(\bm{I}_{\textbf{x}'})^T \mathbb{P}(z' = 1) \\
 & = H_k  (\bm{\hat{\beta}}_{k}; \bm{I}_{\textbf{x},1:k}, z_{1:k}) +  \frac{ \eta^2 \exp(\eta (\bm{\hat{\beta}}_{k+1}^T \mathbf{K}(\bm{I}_{\textbf{x}'}) -  \tau)) }{\big(1+\exp(\eta (\bm{\hat{\beta}}_{k+1}^T \mathbf{K}(\bm{I}_{\textbf{x}'}) -  \tau)) \big)^2}  \mathbf{K}(\bm{I}_{\textbf{x}'}) \mathbf{K}(\bm{I}_{\textbf{x}'})^T, 
\end{split}
\end{equation}
The last line of \eqref{eqn:expected_Hessian_binary} holds since $ \mathbb{P}(z' = 0)  +  \mathbb{P}(z' = 1)  = 1$, irrespective of the distribution of the noise $n(\bm{\cdot},\bm{\cdot})$. 

If we attempt to generalize \eqref{eqn:expected_Hessian_binary} to multi-level measurements, we find that there will be terms $ \mathbb{P}(z' = 0), \mathbb{P}(z' = 1), \dots,  \mathbb{P}(z' = L-1)  $ which cannot all be cancelled, and we will need to specify a noise distribution in order to compute these terms. Since exact knowledge of the noise distribution is usually unavailable in practice, we will instead consider a slightly different objective to optimize, namely a ``predicted Hessian''
\begin{equation}
\label{eqn:predicted_Hessian}
\begin{split}
 \hat{H}(\bm{I}_{\textbf{x}'}) & \triangleq H_k(\bm{\hat{\beta}}_{k}; \bm{I}_{\textbf{x},1:k}, z_{1:k})  +  \nabla^2 g_{k+1}(\bm{\hat{\beta}}_{k+1}; \bm{I}_{\textbf{x}'}, \hat{z}')
\end{split}
\end{equation}
where 
$\hat{z}' \triangleq q\big(\bm{\hat{\beta}}_{k+1}^T \mathbf{K}(\bm{I}_{\textbf{x}'})\big)$ is the predicted future measurement, with the quantizer $q(\bm{\cdot})$ given by \eqref{eqn:quantizer}. Note that \eqref{eqn:predicted_Hessian} reduces to \eqref{eqn:expected_Hessian_binary} in the case of binary measurements. We then maximize the minimum eigenvalue of the predicted Hessian to determine the next measurement location target:
\begin{equation}
\label{prob:maxmin_eig}
\bm{I}_{\textbf{x}}^{\textnormal{target}} = \textnormal{arg} \max\limits_{\bm{I}_{\textbf{x}'} \in \mathcal{I}_{k+1}} \lambda_{\min} (\hat{H}(\bm{I}_{\textbf{x}'})).
\end{equation}

For the set of candidate future position indices  $ \mathcal{I}_{k+1}$, one possible choice could be positions distributed uniformly on a grid within the search region $\mathcal{S}$. 
Once a new location target $\bm{I}_{\textbf{x}}^{\textnormal{target}} $ has been determined, we head in that direction. We will collect measurements and update $\hat{\bm{\beta}}$ along the way, where we collect a new measurement after every $\rho_0$ in distance has been travelled until $ \bm{I}_{\textbf{x}}^{\textnormal{target}}$ is reached, at which time a new location target is determined. 
%$ \bm{I}_{\tilde{k}}^{\textnormal{target}}$ is determined. The index $\tilde{k}$ is the new iteration index, which may be larger than $k+2$ if intermediate measurements have been collected on the way to $ \bm{I}_{k+1}^{\textnormal{target}}$.
The procedure is summarized in Algorithm \ref{alg:active_sensing_online_opt},  where $\bm{I}_{\textnormal{closest}} (\mathbf{x})$ denotes the closest position index $(I_x, I_y)$ to $\mathbf{x} \in \mathcal{S}$.
The condition in line \ref{line:at_target} of Algorithm \ref{alg:active_sensing_online_opt} means the location target has been reached, so that a new location target is determined and a location index $\bm{I}_{\textbf{x},k+1}$ in the direction of the new target is returned. Some random exploration is also included in the algorithm, such that the new location target is random with probability $\varepsilon$, similar to $\varepsilon$-greedy algorithms used in reinforcement learning \cite{SuttonBarto}.  The condition in line \ref{line:within_range_target} means that the agent is within $\rho_0$ of the target, which will be reached at the next time step, while the condition in line \ref{line:outside_range_target} means  the agent will continue heading towards the target and collect measurements along the way. 

%An alternative method is used in \cite{LeongZamaniShames}, where a convex combination of the old and new directions is then determined, and a distance of $\rho_0$ is travelled when a new measurement is collected, $\hat{\bm{\beta}}$ is updated, and problem \eqref{prob:maxmin_eig} solved again. 

\begin{algorithm}[t]
\caption{Active sensing algorithm for online optimization approach: $\bm{I}_{\textbf{x},k+1} = \texttt{ActiveSensing}(\bm{I}_{\textbf{x},k}, \hat{\bm{\beta}}_{k+1})$}
\label{alg:active_sensing_online_opt}
\begin{algorithmic}[1]
\State \textbf{Algorithm Parameters}: Distance $\rho_0 \geq 0$, candidate position indices $\mathcal{I}_{k+1}$, search region $\mathcal{S}$, exploration probability $\varepsilon$
\State \textbf{Inputs}:    $\bm{I}_{\textbf{x},k}$, $\hat{\bm{\beta}}_{k+1}$
\State \textbf{Output}: Next position index $\bm{I}_{\textbf{x},k+1}$
\If{$k=0$}
    \State Initialize $\bm{I}_{\textbf{x}}^{\textnormal{target}}  = \bm{I}_{\textbf{x},0}$
\EndIf
\If{$\bm{I}_{\textbf{x},k} = \bm{I}_{\textbf{x}}^{\textnormal{target}}$} \label{line:at_target}
    \State With probability $\varepsilon$, set new $\bm{I}_{\textbf{x}}^{\textnormal{target}}$ to a random location index in $\{0, \dots, N_x - 1\} \times \{0, \dots, N_y-1\}$, otherwise compute new $\bm{I}_{\textbf{x}}^{\textnormal{target}} = \textnormal{arg} \max\limits_{\bm{I}_{\textbf{x}'} \in \mathcal{I}_{k+1}} \lambda_{\min} (\hat{H}(\bm{I}_{\textbf{x}'})),$ where $\hat{H}(\bm{I}_{\textbf{x}'})$ is given by \eqref{eqn:predicted_Hessian}  \label{line:random_exploration}
    \State Set $\mathbf{x}_{k+1} = \mathbf{x}_k + \rho_0 (\mathbf{x}^\textnormal{target} - \mathbf{x}_k)/||\mathbf{x}^{\textnormal{target}} - \mathbf{x}_k||$ and return $\bm{I}_{\textbf{x},k+1} = \bm{I}_{\textnormal{closest}}(\mathbf{x}_{k+1})$
\ElsIf{$||\mathbf{x}_k - \mathbf{x}^{\textnormal{target}}|| < \rho_0$} \label{line:within_range_target}
    \State Set $\mathbf{x}_{k+1} = \mathbf{x}^{\textnormal{target}}$ and return $\bm{I}_{\textbf{x},k+1} =\bm{I}_{\textbf{x}}^{\textnormal{target}}$
\Else \label{line:outside_range_target}
    \State Set $\mathbf{x}_{k+1} = \mathbf{x}_k + \rho_0 (\mathbf{x}^\textnormal{target} - \mathbf{x}_k)/||\mathbf{x}^{\textnormal{target}} - \mathbf{x}_k||$ and return $\bm{I}_{\textbf{x},k+1} = \bm{I}_{\textnormal{closest}}(\mathbf{x}_{k+1})$
\EndIf

\end{algorithmic}
\end{algorithm} 

\section{Numerical Studies}
\label{sec:numerical}
For performance evaluation of the field estimation algorithms, we will consider two performance measures, the mean squared error (MSE) and structural similarity index (SSIM). These are defined similar to Section \ref{sec:DCT_RBF_comparison}, except that we replace the approximated field with the estimated field 
$$\hat{\phi}_d (I_x, I_y)   \triangleq \sum_{(u,v) \in \tilde{\mathcal{U}} } \alpha_x(u) \alpha_y(v) \hat{C}(u,v) \cos \left(\frac{(2I_x+1)\pi u}{2 N_x} \right) \cos \left(\frac{(2I_y+1)\pi v}{2 N_y} \right).$$


\subsection{Static Fields}
We consider estimation of the (true) field shown in Fig. \ref{fig:true_field_seed355}, with search region $\mathcal{S} = [0,100] \times [0,100]$. The field is discretized using $N_x = 100$ and $N_y = 100$. We use \eqref{eqn:U_tilde_largest} to select the largest modes that we wish to retain and estimate. 
\begin{figure}[t!]
\centering 
\includegraphics[scale=0.6]{true_field_seed355.pdf} 
\caption{Static field}
\label{fig:true_field_seed355}
\end{figure} 

We use Algorithm \ref{alg:DCT_optim_time_varying} with $\eta=5$ and $\varsigma = 1/5000$. As the field is assumed static, the forgetting factor is set to $\delta = 1$. The initial position index is set to $\bm{I}_{\textbf{x},0} = (50,50)$, close to the center of the search region~$\mathcal{S}$. 
A four level quantizer is used with quantizer thresholds $\tau_0=1, \tau_1=2, \tau_2 = 3$. The measurement noise $n(\bm{\cdot},\bm{\cdot})$ is i.i.d. Gaussian with zero mean and variance equal to 0.1. For choosing the measurement locations, we use Algorithm \ref{alg:active_sensing_online_opt} with $\rho_0 = 10$. The candidate position indices $\mathcal{I}_{k+1}$ are chosen to correspond to 36 points placed uniformly on a grid within the search region $\mathcal{S}$. The exploration probability is chosen as $\varepsilon = 0.1$.

\begin{figure}[t!]
\centering 
\includegraphics[scale=0.6]{MSE_time_stepped_seed355.pdf} 
\caption{Static field: MSE vs. $k$}
\label{fig:MSE_time_stepped_seed355}
\end{figure} 

\begin{figure}[t!]
\centering 
\includegraphics[scale=0.6]{SSIM_time_stepped_seed355.pdf} 
\caption{Static field: SSIM vs. $k$}
\label{fig:SSIM_time_stepped_seed355}
\end{figure} 

Fig. \ref{fig:MSE_time_stepped_seed355}  shows the MSE vs $k$ (corresponding to the number of measurements collected), when various numbers of modes are estimated. Fig.  \ref{fig:SSIM_time_stepped_seed355} shows the SSIM vs $k$. Each point in Figs. \ref{fig:MSE_time_stepped_seed355} and \ref{fig:SSIM_time_stepped_seed355} is obtained by averaging over 10 runs. We see from the figures that there is a trade-off between the estimation quality, number of modes/parameters that need to be estimated, and number of measurements collected. If a lot of measurements can be collected, then estimating more modes will allow for a better estimate of the field.\footnote{For example, if multiple mobile agents can be utilized \cite{LeongZamani_SP} or one has a sensor network, then more measurements can be collected in a limited amount of time.} On the other hand, if fewer measurements are available, estimating fewer modes more accurately may give a better field estimate than estimating lots of modes inaccurately.  
In Fig. \ref{fig:estimated_field_seed355} we show a sample plot of the estimated field when 80 modes are estimated, after 2000 measurements have been collected. 
\begin{figure}[t!]
\centering 
\includegraphics[scale=0.6]{estimated_field_seed355.pdf} 
\caption{Static field: Estimated field using 2000 measurements}
\label{fig:estimated_field_seed355}
\end{figure} 

\subsection{Time-varying Fields}
We now consider an example with time-varying fields. Suppose the true field is the same of that of Fig. \ref{fig:field_seed341_DCT} for the first 1000 iterations, but then switches to the true field in Fig. \ref{fig:field_seed343_DCT} for the next 1000 iterations. We will use Algorithms \ref{alg:DCT_optim_time_varying} and \ref{alg:active_sensing_online_opt}  with forgetting factor $\delta = 0.995$, with other parameters the same as in the previous example. 

Figs. \ref{fig:MSE_time_varying_seed341_343} and \ref{fig:SSIM_time_varying_seed341_343} show respectively the MSE and SSIM vs. $k$, when the 60 largest modes are estimated. 
We see that after the field changes at $k=1000$ the accuracy of the field estimate drops, but Algorithm~\ref{alg:DCT_optim_time_varying} is able to recover  and estimate the new field as more measurements are collected. 

For comparison, the MSE and SSIM obtained using forgetting factor $\delta = 1$ are also shown. In this case, as there is no forgetting of old information, the field estimates will take much longer to adjust to the new field. 

\begin{figure}[t!]
\centering 
\includegraphics[scale=0.6]{MSE_time_varying_seed341_343.pdf} 
\caption{Time varying field: MSE vs. $k$}
\label{fig:MSE_time_varying_seed341_343}
\end{figure} 

\begin{figure}[t!]
\centering 
\includegraphics[scale=0.6]{SSIM_time_varying_seed341_343.pdf} 
\caption{Time varying field: SSIM vs. $k$}
\label{fig:SSIM_time_varying_seed341_343}
\end{figure} 

\section{Conclusion}
This paper has studied the estimation of scalar fields, where a field is viewed in the Fourier domain. An algorithm has been presented for estimating the lower order modes of the field under the assumption of noisy quantized measurements. Our approach assumed an agent or agents travelling around a region in order to collect measurements. Future work will consider the use of a sensor network for field estimation, with algorithms constrained by local communication and distributed computation. 


\section*{Acknowledgment}
The authors thank Mr. Shintaro Umeki for suggesting the use of structural similarity as a performance measure while working at DST Group as a summer vacation student. 

\bibliography{IEEEabrv,source_localization}
\bibliographystyle{IEEEtran} 




\end{document}
 
% \section{Appendix for Proofs}

\paragraph{Proof of Theorem \ref{thm:main}.}

\begin{proof}
\label{proof:main}
Our proof has two steps. In Step 1, we will show that SimCLR is equivalent to minimizing the cross entropy loss defined in Eqn.~(\ref{eqn:cross-entropy}). 
In Step 2, we will show  that minimizing the cross-entropy loss 
is equivalent to spectral clustering on $\bfpi$. 
Combining the two steps together, we have proved our theorem. 

\textbf{Step 1: } SimCLR is equivalent to minimizing the cross entropy loss.

The cross-entropy loss takes expectation over 
$\bfW_\bfX\sim \mathbb{P}(\cdot ; \bfpi)$, 
which means $\bfW_\bfX$ has exactly one non-zero entry in each row $i$. By Lemma~\ref{lem:multinomial}, we know every row $i$ of $\bfW_\bfX$ is independent of other rows. Moreover, 
$\bfW_{\bfX,i}\sim \mathcal{M}(1, \bfpi_i/\sum_j \bfpi_{i,j})=\mathcal{M}(1, \bfpi_i)$, because $\bfpi_i$ itself is a probability distribution.
Similarly, we know $\bfW_\bfZ$ also has the row-independent property by sampling over $\mathbb{P}(\cdot;\bfK_\bfZ)$.
Therefore, by Lemma~\ref{lem:cross_split}, we know Eqn.~(\ref{eqn:cross-entropy}) is equivalent to:
\[
 -\sum_{i=1}^n \mathbb{E}_{\bfW_{\bfX,i}}[\log \mathbb{P}(\bfW_{\bfZ,i}=\bfW_{\bfX,i};\bfK_\bfZ)],
\]

This expression takes expectation over $\bfW_{\bfX,i}$ for the given row $i$. Notice that 
$\bfW_{\bfX,i}$ has exactly one non-zero entry, which equals $1$ (same for $\bfW_{\bfZ,i}$). 
As a result
we expand the above expression to be:
\begin{equation}
 -\sum_{i=1}^n \sum_{j\neq i} \Pr(\bfW_{\bfX,i,j}=1)\log \Pr(\bfW_{\bfZ,i,j}=1).
\label{eqn:detailed-expansion}    
\end{equation}


By Lemma~\ref{lem:multinomial}, $\Pr(\bfW_{\bfZ,i,j}=1)=\bfK_{\bfZ,i,j}/\|\bfK_{\bfZ,i}\|_1$ for $j\neq i$. Recall that $\bfK_\bfZ=(k(\bfZ_i-\bfZ_j))_{(i,j)\in[n]^2}$, which means 
$\bfK_{\bfZ,i,j}/\|\bfK_{\bfZ,i}\|_1=\frac{\exp(-\|\bfZ_i-\bfZ_j\|^2/{2\tau})}{\sum_{k\neq i}
\exp(-\|\bfZ_i-\bfZ_k\|^2/{2\tau})
}$ for $j\neq i$, when $k$ is the Gaussian kernel with variance $\tau$. 

Notice that $\bfZ_i=f(\bfX_i)$, so we know
\begin{equation}
-\log \Pr(\bfW_{\bfZ,i,j}=1)=
-\log \frac{\exp(-\|f(\bfX_i)-f(\bfX_j)\|^2/{2\tau})}{\sum_{k\neq i}
\exp(-\|f(\bfX_i)-f(\bfX_k)\|^2/{2\tau}),
}
\label{eqn:infonce-equivalence}    
\end{equation}


The right hand side is exactly the InfoNCE loss defined in Eqn.~(\ref{eqn:infonce}).
Inserting Eqn.~(\ref{eqn:infonce-equivalence}) into Eqn.~(\ref{eqn:detailed-expansion}), we get the SimCLR algorithm, which first samples augmentation pairs $(i,j)$ with $\Pr(\bfW_{\bfX,i,j}=1)$ for each row $i$, and then optimize the InfoNCE loss. 

\textbf{Step 2: } minimizing the cross entropy loss 
is equivalent to spectral clustering on $\bfpi$.


By Lemma~\ref{lem:convert_to_spectral}, we may further convert the loss to 
\begin{equation}
\label{eqn:main-theorem-repul-attr}
\min_{\bfZ}
-\sum_{(i,j)\in [n]^2} \mathbf{P}_{i,j}
\log k (\bfZ_i-\bfZ_j)+\log \mathbf{R}(\bfZ).
\end{equation}
Since $k$ is the Gaussian kernel, this reduces to \[
\min_\bfZ \mathrm{tr}(\bfZ^\top \mathbf{L}(\bfpi) \bfZ)
+\log \mathbf{R}(\bfZ),
\]

where we use the fact that $\mathbb{E}_{\bfW_\bfX\sim \mathbb{P}(\cdot; \bfpi)}[\mathbf{L}(\bfW_\bfX)]
=\mathbf{L}(\bfpi)
$, because the Laplacian operator is linear and $
\mathbb{E}_{\bfW_\bfX\sim \mathbb{P}(\cdot; \bfpi)}(\bfW_\bfX)=\bfpi
$.
\end{proof}

\paragraph{Proof of Theorem \ref{thm:clip}.}
\begin{proof}
Since $\bfW_\bfX\sim \mathbb{P}(\cdot;\bfpi_{\mathbf{A}, \mathbf{B}})$, we know 
$\bfW_\bfX$ has exactly one non-zero entry in each row, denoting the pair that got sampled. 
A notable difference compared to the previous proof is we now have $n_\mathcal{A}+n_\mathcal{B}$ objects in our graph. CLIP deals with this by taking a mini-batch of size $2N$, 
such that $n_\mathcal{A}=n_\mathcal{B}=N$, and adding the $2N$ InfoNCE losses together. We label the objects in $\mathcal{A}$ as $[n_\mathcal{A}]$, and the objects in $\mathcal{B}$ as $\{n_\mathcal{A}+1, \cdots, n_\mathcal{A}+n_\mathcal{B}\}$. 

Notice that $\bfpi_{\mathbf{A}, \mathbf{B}}$ is a bipartite graph, so the edges of objects in $\mathcal{A}$ will only connect to object in $\mathcal{B}$ and vice versa. We can define the similarity matrix in $\cZ$ as $\bfK_\bfZ$, 
where $\bfK_\bfZ(i, j+n_\mathcal{A})=\bfK_\bfZ(j+n_\mathcal{A},i)= k(\bfZ_i-\bfZ_j)$ for $i\in [n_\mathcal{A}], j\in [n_\mathcal{B}]$, and otherwise we set $\bfK_\bfZ(i,j)=0$. 
The rest is same as the previous proof. 
\end{proof}

\paragraph{Proof of Theorem \ref{thm:exponential}.}

\begin{proof}
\label{proof:exponential}
Since the objective function consists of a linear term combined with an entropy regularization, which is a strongly concave function, the maximization problem is a convex optimization problem. Owing to the implicit constraints provided by the entropy function, the problem is equivalent to having only the equality constraint. We then introduce the Lagrangian multiplier $\lambda$ and obtain the following relaxed problem:

$$
\widetilde{E}(\boldsymbol{\alpha})=\psi_{1}-\sum_{i=1}^n \alpha_{i} \psi_{i}+\tau \sum_{i=1}^n \alpha_{i}\log \alpha_{i}+\lambda\left(\boldsymbol{\alpha}^{\top} \mathbf{1}_n-1\right).
$$

As the relaxed problem is unconstrained, taking the derivative with respect to $\alpha_{i}$ yields

$$
\frac{\partial \widetilde{E}(\boldsymbol{\alpha})}{\partial \alpha_{i}}=-\psi_{i}+\tau\left(\log \alpha_{i}+\alpha_{i} \frac{1}{\alpha_{i}}\right)+\lambda=0.
$$

Solving the above equation implies that $\alpha_{i}$ takes the form
$
\alpha_{i}=\exp \left(\frac{1}{\tau} \psi_{i}\right) \exp \left(\frac{-\lambda}{\tau}-1\right).
$ Since $\alpha_{i}$ lies on the probability simplex, the optimal $\alpha_{i}$ is explicitly given by
$
\alpha^{*}_{i}=\frac{\exp \left(\frac{1}{\tau} \psi_{i}\right)}{\sum_{i^{\prime}=1}^n \exp \left(\frac{1}{\tau} \psi_{i^{\prime}}\right)} .
$ Substituting the optimal point into the objective function, we obtain
$$
\begin{aligned}
E\left(\boldsymbol{\alpha}^*\right)  &=\psi_1-\sum_{i=1}^n \frac{\exp \left(\frac{1}{\tau} \psi_{i}\right)}{\sum_{i^{\prime}=1}^n \exp \left(\frac{1}{\tau} \psi_{i^{\prime}}\right)} \psi_{i}+\tau \sum_{i=1}^n \frac{\exp \left(\frac{1}{\tau} \psi_{i}\right)}{\sum_{i^{\prime}=1}^n \exp \left(\frac{1}{\tau} \psi_{i^{\prime}}\right)}\log \frac{\exp \left(\frac{1}{\tau} \psi_{i}\right)}{\sum_{i^{\prime}=1}^n \exp \left(\frac{1}{\tau} \psi_{i^{\prime}}\right)} \\
& =\psi_1 - \tau \log \left(\sum_{i=1}^n \exp \left(\frac{1}{\tau} \psi_{i}\right)\right).
\end{aligned}
$$
Thus, the Lagrangian dual function is given by
\begin{equation*}
-E\left(\boldsymbol{\alpha}^*\right)= -\tau \log \frac{\exp \left(\frac{1}{\tau} \psi_{1}\right)}{\sum_{i=1}^n \exp \left(\frac{1}{\tau} \psi_{i}\right)}.\qedhere
\end{equation*}
\end{proof}



\section{More on Experiments} \label{section: experiment_details}

\paragraph{CIFAR-10 and CIFAR-100} CIFAR-10 ~\citep{krizhevsky2009learning} and CIFAR-100 ~\citep{krizhevsky2009learning} are well-known classic image classification datasets. Both CIFAR-10 and CIFAR-100 contain a total of 60k $32 \times 32$ labeled images of different classes, with 50k for training and 10k for testing. CIFAR-10 is similar to CIFAR-100, except there are 10 different classes in CIFAR-10 and 100 classes in CIFAR-100.

\paragraph{TinyImageNet} TinyImageNet ~\citep{le2015tiny} is a subset of ImageNet ~\citep{deng2009imagenet}. There are 200 different object classes in TinyImageNet, with 500 training images, 50 validation images, and 50 test images for each class. All the images in TinyImageNet are colored and labeled with a size of $64 \times 64$.

\textbf{Pseudo-code.} Algorithm \ref{alg:Training Procedure} presents the pseudo-code for our empirical training procedure.

\begin{algorithm}[!htbp]
\caption{Training Procedure}
\label{alg:Training Procedure}
\begin{algorithmic}[1]
\REQUIRE trainable encoder network $f$, batch size $N$, augmentation strategy \textit{aug}, loss function $L$ with hyperparameters \textit{args}
\FOR {sampled minibatch ${x_i}_{i=1}^N$}
\FORALL{$i \in { 1, ..., N }$}
\STATE draw two augmentations $t_i = \textit{aug}\left(x_i\right) $, $t_i' = \textit{aug}\left(x_i\right) $
\STATE $z_i = f\left(t_i\right)$, $z_i' = f\left(t_i'\right)$
\ENDFOR
\STATE compute loss $\mathcal{L} = L(N, z, z', \textit{args})$
\STATE update encoder network $f$ to minimize $\mathcal{L}$
\ENDFOR
\STATE \textbf{Return} encoder network $f$
\end{algorithmic}
\end{algorithm}

We also provide the pseudo-code for our core loss function used in the training procedure in Algorithm \ref{alg:Core loss}. The pseudo-code is almost identical to SimCLR's loss function, with the exception of an extra parameter $\gamma$.

\begin{algorithm}[!htbp]
\caption{Core loss function $\mathcal{C}$}
\label{alg:Core loss}
\begin{algorithmic}[1]
\REQUIRE batch size $N$, two encoded minibatches $z_1, z_2$, $\gamma$, temperature $\tau$
\STATE $z = \textit{concat}\left(z_1, z_2\right)$
\FOR {$i \in {1, ..., 2N }, j \in {1, ..., 2N}$ }
\STATE $s_{i,j} = \Vert z_i - z_j \Vert_2^{\gamma}$
\ENDFOR
\STATE \textbf{define} $l(i, j)$ \textbf{as} $l(i, j) = - \log \frac{exp\left(s_{i,j}/\tau \right)}{\sum_{k=1}^{2N} \mathbf{1}{[k \ne i]} exp\left(s{i, j} / \tau \right)} $
\STATE \textbf{Return} $\frac{1}{2N} \sum_{k=1}^N\left[l(i, i+N) + l(i+N, i)\right]$
\end{algorithmic}
\end{algorithm}

Utilizing the core loss function $\mathcal{C}$, we can define all kernel loss functions used in our experiments in Table \ref{table: loss definition}. For all $z_i \in z$ with even dimensions $n$, we define $z_{L_i} = z_i\left[0:n/2\right]$ and $z_{R_i} = z_i\left[n/2:n\right]$.

\begin{table}[ht]
\centering
\begin{tabular}{{@{}l|l@{}}}
Kernel  &  Loss function \\ \midrule
Laplacian & $\mathcal{C}\left(N, z, z', \gamma=1, \tau\right)$\\ \midrule
Sum       & $\lambda * \mathcal{C}\left(N, z, z', \gamma=1, \tau_1\right) + (1-\lambda) * \mathcal{C}\left(N, z, z', \gamma=2, \tau_2\right)$  \\ \midrule
Concatenation Sum&$\lambda * \mathcal{C}\left(N, z_L, z'_L, \gamma=1, \tau_1\right) + (1-\lambda) * \mathcal{C}\left(N, z_R, z'_R, \gamma=2, \tau_2\right)$\\ \midrule
$\gamma = 0.5$ & $\mathcal{C}\left(N, z, z', \gamma=0.5, \tau\right)$          \\ 

\end{tabular}

\caption{Definition of kernel loss functions in our experiments}
\label {table: loss definition}
\end{table}

\textbf{Baselines.} We reproduce the SimCLR algorithm using PyTorch Lightning~\citep{PytorchLightning}.

\textbf{Encoder details.}
The encoder $f$ consists of a backbone network and a projection network. We employ ResNet50~\citep{ResNet} as the backbone and a 2-layer MLP (connected by a batch normalization~\citep{ioffe2015batch} layer and a ReLU \cite{nair2010rectified} layer) with hidden dimensions 2048 and output dimensions 128 (or 256 in the concatenation kernel case).

\textbf{Encoder hyperparameter tuning.}
For each encoder training case, we randomly sample 500 hyperparameter groups (sample details are shown in Table \ref{table: Hyperparameter sample}) and train these samples simultaneously using Ray Tune ~\citep{RayTune}, with the ASHA scheduler~\citep{li2018massively}. Ultimately, the hyperparameter group that maximizes the online validation accuracy (integrated in PyTorch Lightning) within 5000 validation steps is chosen for the given encoder training case.

\begin{table}[ht]
\centering

\begin{tabular}{@{}l|l|l@{}}
\midrule
Hyperparameter  & Sample Range & Sample Strategy \\ \midrule
start learning rate & $\left[10^{-2}, 10\right]$ & log uniform \\ \midrule
$\lambda$       & $\left[0, 1\right]$ & uniform \\ \midrule
$\tau$, $\tau_1$, $\tau_2$ & $\left[0, 1\right]$ & log uniform \\ \midrule
\end{tabular}

\caption{Hyperparameters sample strategy}
\label {table: Hyperparameter sample}
\end{table}

\textbf{Encoder training.} 
We train each encoder using the LARS optimizer~\citep{LARSOptimizer}, LambdaLR Scheduler in PyTorch, momentum 0.9, weight decay $10^{-6}$, batch size 256, and the aforementioned hyperparameters for 400 epochs on a single A-100 GPU.

\textbf{Image transformation.} The image transformation strategy, including augmentation, is identical to the default transformation strategy provided by PyTorch Lightning.

\textbf{Linear evaluation.}
The linear head is trained using the SGD optimizer with a cosine learning rate scheduler, batch size 64, and weight decay $10^{-6}$ for 100 epochs. The learning rate starts at $0.3$ and ends at $0$.

\textbf{Moco Experiments.} We also tested our method based on MoCo~\citep{he2019moco}. The results are summarized in Table \ref{tab:results-moco}. Here we choose ResNet18~\citep{ResNet} as the backbone and set a temperature of $0.1$ as default. For our simple sum kernel, we set $\lambda=0.8$. The results show that our method outperforms the original MoCo method.

\begin{table}[thb]
\centering
\caption{MoCo Experiment Results on CIFAR-10 and CIFAR-100.}
\label{tab:results-moco}
\resizebox{\textwidth}{!}{%
\begin{tabular}{@{}c|ccc|ccc@{}}
\toprule
\multirow{3}{*}{Method} & \multicolumn{3}{c|}{CIFAR-10} & \multicolumn{3}{c}{CIFAR-100} \\ \cmidrule(lr){2-4} \cmidrule(lr){5-7} 
                        & 200 epochs & 400 epochs    & 1000 epochs   & 200 epochs & 400 epochs & 1000 epochs         \\ \midrule
MoCo (repro.)         & $76.41 \pm 0.12$    & $80.01 \pm 0.15$          & $84.45 \pm 0.08$    & $\mathbf{47.02 \pm 0.11}$ & $52.50 \pm 0.07$ & $57.62 \pm 0.15$            \\
\midrule
Laplacian Kernel        & ${78.09 \pm 0.10}$    & $\mathbf{83.85 \pm 0.09}$          & $\mathbf{88.34 \pm 0.16}$    & $46.12 \pm 0.22$   & $53.44 \pm 0.17$ & $59.10 \pm 0.14$        \\
Simple Sum Kernel & $\mathbf{78.12 \pm 0.15}$   & $83.23 \pm 0.18$ & $87.50 \pm 0.20$ & $46.65 \pm 0.06$ & $\mathbf{53.62 \pm 0.19}$ & $\mathbf{59.83 \pm 0.12}$\\
\bottomrule
\end{tabular}
}
\end{table}



\section{More Experiments on Synthetic Data}


Consider a scenario with $n$ clusters, each containing $k$ vertices. Let the probability of vertices $u$ and $v$ from the same cluster belonging to $\bfpi$ be $p$. Conversely, for vertices $u$ and $v$ from different clusters, let the probability of belonging to $\pi$ be $q$. We generate the graph $\bfpi$ randomly, based on $p$ and $q$. We experiment with values of $k=100$ and $n=6$ for ease of visualization, embedding all points in a two-dimensional space. Each vertex's initial position originates from a normal distribution. In each iteration, we sample a subgraph of $\bfpi$ uniformly, ensuring each vertex has an out-degree of $1$. We then optimize the corresponding vectors using InfoNCE loss with an SGD optimizer and iterate until convergence. Our experimental setup consists of an SGD learning rate of $1$, an InfoNCE loss temperature of $0.5$, and a batch size of $50$. We evaluate two scenarios with different $p$ and $q$ values: $p=1$, $q=0$, and $p=0.75$, $q=0.2$. The results of these experiments are visualized in Figure \ref{fig:vis-spectral-cluster}. The obtained embeddings exhibit the hallmark pattern of spectral clustering of graph $\bfpi$.

\begin{figure}[!tb]
\centering
\subfigure{
\includegraphics[width=1\textwidth]{Figures/cluster_pi.png}
\label{fig:vis-cluster}
}
\subfigure{
\includegraphics[width=1\textwidth]{Figures/noised_cluster_pi.png}
\label{fig:vis-noised-cluster}
}
\caption{Visualizations of the optimization process using InfoNCE Loss on the vectors corresponding to $\bfpi$. Points of identical color belong to the same cluster within $\bfpi$. To showcase the internal structure of $\bfpi$, we randomly select 10 vertices from each cluster to display the edge distribution of $\bfpi$.}
\label{fig:vis-spectral-cluster}
\end{figure}




\newcommand{\ck}[1]{{\color{red} CK: #1}}

\newcommand{\bch}{\mathrm{BCH}}
\newcommand{\trotter}{\mathrm{Trotter}}


\begin{abstract}
    Circuit QED enables the combined use of qubits and oscillator modes. Despite a variety of available gate sets, many hybrid qubit-boson (i.e., oscillator) operations are realizable only through optimal control theory (OCT) which is oftentimes intractable and uninterpretable. We introduce an analytic approach with rigorously proven error bounds for realizing specific classes of operations via two matrix product formulas commonly used in Hamiltonian simulation, the Lie--Trotter and Baker--Campbell--Hausdorff product formulas. We show how this technique can be used to realize a number of operations of interest, including polynomials of annihilation and creation operators, i.e., $a^p {a^\dagger}^q$ for integer $p, q$.  We show examples of this paradigm including: obtaining universal control within a subspace of the entire Fock space of an oscillator, state preparation of a fixed photon number in the cavity, simulation of the Jaynes--Cummings Hamiltonian, simulation of the Hong-Ou-Mandel effect and more.  This work demonstrates how techniques from Hamiltonian simulation can be applied to better control hybrid boson-qubit devices.
\end{abstract}

%\newpage
%\tableofcontents

% \how{current overview}
% In section II, we describe the hybrid-boson device and introduce the two Hamiltonian simulation product formulas which will be used. In section III, we describe the addition technique and state our results about the error scaling and operation count. In section IV, we demonstrate the technique on a number of applications and include numerics to observe the scaling. We summarize our observations in section V.


%\added{Search for string SMG to find queries from Steve.  References are out of order.  You do not refer in the text to Appendices C, E, or F; please insert text referring to those.  Equation numbers are messed up by theorem environment.}

% May want to cite https://arxiv.org/pdf/2111.12177.pdf


\section{Introduction}
%\how{synergy between oct and this technique} what can be made with oct, how can it be extended with our technique

Today, many quantum computing architectures are homogeneous -- i.e. the same type of qubit is employed throughout the device. From devices made of superconducting qubits \cite{arute2019quantum,dial2022moving,reagor2018demonstration} to ion trap qubits \cite{wright2019benchmarking}, prior work largely focuses on linking qubits of the same type together in fault-tolerant ways. However, there is a nascent body of work that studies the potential for hybrid quantum computers that leverage two or more types of quantum architectures (e.g., qubits and oscillator modes). These devices hold promise because they can be tailored for specific physical simulation problems, which would be especially useful in applications like material discovery or molecular simulation \cite{Wang2020FCFs,WangConicalIntersection} or quantum simulation of lattice models~\cite{PhysRevB.98.174505}.

In particular, the hybrid qubit-boson (i.e., oscillator) models \cite{blais2021circuit} hold some advantages.  Specifically, the long lifetimes afforded to microwave photons in superconducting resonators have made them attractive targets for quantum error correction~\cite{blais2021circuit}.  In addition, the Hilbert space accessible to a mode is much larger than that of qubits.  Furthermore, the larger set of operations that can be performed by coupling an oscillator and a qubit open up the potential for multi-qubit interactions, such as the M{\o}lmer-S{\o}rensen gate \cite{molmer-sorensen-gate} while at the same time enabling new forms of transduction between qubits and flying qubits such as photons~\cite{lauk2020perspectives,basilewitsch2022engineering}. Oscillator interactions may also have unique features, like nonlinearities, which are challenging to simulate even with homogeneous quantum architectures~\cite{stavenger2022bosonic}.


%Spin-boson instruction set architectures (ISA) enable the general control of native bosonic hardware which could have advantages over purely spin-1/2 hardware in quantum signal processing (QSP), quantum simulation, and the preparation and use of quantum states requiring a large Hilbert space.  

However, whereas many useful applications have already been demonstrated with the experimentally available gate sets in ion traps and circuit QED, often problems of interest require more complex operations, and these must be compiled from the various experimental regimes and pulse sequences available.   Specifically, techniques such as optimal control theory (OCT)~\cite{khaneja2005optimal,werschnik2007quantum,Koch-OCT-2017} provide ways to design at a pulse level a sequence of controls that can be set in order to enact an arbitrary quantum transformation on a hybrid-boson qubit system.  Although optimal control theory  does provide a method of compiling nonnative operations into native gates, it is oftentimes intractable and almost always uninterpretable, yielding only a pulse which performs the desired operation without providing any physical intuition. Furthermore, it is computationally intensive to produce and due to the lack of theoretical understanding behind the pulse sequences it must be carried out on a case-by-case basis.  These shortcomings further make models such as this difficult to program and to analyze the circuit complexity, as the procedure and in turn the cost for constructing an arbitrary unitary transformation can be difficult to bound.

Inspired by recent experimental progress \cite{SNAP-PhysRevLett.115.137002,eickbusch2021fast}, we introduce in this paper an extensible control scheme for a hybrid boson-qubit quantum computer that is universal.  
Specifically, we consider the potential of instruction set architectures (ISAs) compiled from experimentally available gate sets using the Baker-Campbell-Hausdorff, Trotter-Suzuki and Lie-Trotter formulas. We develop two parallel approaches, one which primarily uses the creation and destruction operators which we refer to as based on `Fock methods', and the other primarily relying on position and momentum operators which we refer to as based on `phase-space methods'. We demonstrate that both methods can be used to generate an ISA for bosonic devices. Furthermore, we use the previously mentioned formulas to realize a number of operations of interest, including polynomials of annihilation and creation operators, i.e., $a^p {a^\dagger}^q$ for integer $p, q$. These block-encoded operations are crucial for QSP and certain problems in quantum simulation. Furthermore, we demonstrate that our compilation scheme obtains near-linear scaling and provide upper bounds on the maximum number of operations required to implement the compiled gate. Finally, we give examples for the Hamiltonian of a nonlinear material and applications to common unsolved problems in quantum simulation such as the Fermi-Hubbard model.

% In this paper, we introduce a novel control scheme for a hybrid boson-qubit quantum computer that enables simulation of the Jaynes-Cummings model \cite{JaynesCummings1963}. \added{SMG: Not sure this should be the main focus since you don't explicitly discuss it and there are a number of examples you include and more could be included.}  


% Problem, not knowing how to make gates, no physical insight

% We introduce an analytic approach

% Here is our method ((x, p) and (a, adag))

%% here's how we're connzecting our physical intuition to the realization of actual gates

%% CK: See how to connect the prior overview with the overall language. 

%% here's how you obtain a decomposition in phase-space / displacement operators

%% here's the algorithmic justification that this works

%% not sure about the subsection orderings
\newpage
\section{Preliminaries}
{In this section, we introduce the hybrid boson-qubit architecture we are operating on and the matrix product formulas we will use.}
The central goal of our work is to provide a generic toolbox that can be used to build unitary transformations on hybrid quantum devices that have both qubits as well as bosonic modes.  Such devices are common in quantum computing, spanning circuit quantum electrodynamics (superconducting qubits coupled to microwave photons) to ion-trap quantum computing (for which the mechanical modes of oscillation are coupled to atomic qubits).  The challenge though with these approaches is that fundamentally different insights are needed here to compile unitaries than the binary-based approaches that have been successful for the case of qubits.

Here we review the mathematical properties of bosonic quantum mechanics which are needed to understand the basic operations considered for the ISA architecture that we consider.  
Specifically, we present an analytic instruction set architecture (ISA) based on the Trotter-Suzuki and Baker-Campbell-Hausdorff (BCH) decompositions for decomposition of gates of the form $U=e^{i\hat{H}\sigma^{j}}$, where $\hat{H}$ is a Hermitian
operator composed of phase-space operators and Pauli gates. Before jumping into the specific details of our gate operations, we need to review the basics of bosonic quantum mechanics as well as the mathematical results needed to use these bosonic operations to compile a given unitary transformation.

\subsection{A Hybrid Boson-Qubit Device}
The central object that we need in this extension is the concept of a qumode, which stores the state of the bosons. Photons (energy quanta of the electromagnetic field) and phonons (quanta of mechanical vibrations) are examples of bosons present in current quantum computing experiments. Photons are the focus of photonic devices, cavity QED and hybrid circuit QED, whereas phonons are used to couple ion-spin qubits in ion traps. In the case of photons, the qumode stores the configuration of the electromagnetic field, and in the case of phonons, the qumode stores the vibrational state. We first produce a mathematical description of a qumode, then describe operations which can be performed on the qumode.

\subsubsection{Representing the qumode}
There are two different bases that are commonly used to describe the state of the qumode: 
\begin{enumerate}
    \item \textit{Phase-space representation}, where operators are written in terms of position ($\hat{x}$) and momentum ($\hat{p}$) operators
    \item \textit{Fock-space representation}, where operators are written in terms of bosonic creation ($a^\dagger$) and annihilation ($a$) operators.
\end{enumerate}

In the phase-space representation, the computational basis corresponds to the strength of the electric field in the case of photons (or equivalently the position of a mechanical oscillator for vibrational systems). We refer to this with the operator $\hat{x}$ and we have that for any $x\in \mathbb{R}$, $\hat{x} \ket{x} = x \ket{x}$.  It is often common to speak of the momentum operator $\hat{p}$, which can be found through the Fourier transform of $\hat{x}$.  This describes the magnetic field for a photonic system.  In practice, cutoffs are imposed on the values of the field and further discretization error on the gates and the outputs prevent arbitrary precision readout; however, for simplicity we ignore the latter issue in order to provide a simpler (if unrealistic) computational model and ignore the issue that even when cutoffs are imposed the vector space does not strictly form a Hilbert space without also including spatial discretization.

In the Fock-space representation, we track the number of bosons (number of photons or the energy level of the harmonic oscillator for the vibrational case) in the computational basis.  In this representation the computational basis is defined to be an eigenvector of the boson number operator $\hat{n} \ket{n} = n \ket{n}$, where $\hat{n} = a^\dagger a$ is the number operator and $a$ and $a^\dagger$ add and remove a boson from the system, respectively.  Formally this spectrum is countably infinite, but after truncation it forms a finite-dimensional Hilbert space and thus can be thought of as a qudit.  For example, assuming a cutoff $\Lambda = 3$:
\begin{align}
    P_3 a^\dagger P_3 = \begin{bmatrix}
    0 & 0 & 0 & 0 \\
    1 & 0 & 0 & 0 \\
    0 & \sqrt{2} & 0 & 0 \\
    0 & 0 & \sqrt{3} & 0 
    \end{bmatrix}, P_3 a P_3 = \begin{bmatrix}
    0 & 1 & 0 & 0\\
    0 & 0 & \sqrt{2} & 0 \\
    0 & 0 & 0 & \sqrt{3} \\
    0 & 0 & 0 & 0
    \end{bmatrix}.
\end{align}
%\textbf{Above we should use a projection to discuss the truncation}
Here $P_\Lambda$ is the projector onto the subspace of the cavity containing at most $\Lambda$ photons, i.e.:
\begin{align}
    P_\Lambda : P_\Lambda \ket{n} = \begin{cases}
    \ket{n} & n \leq \Lambda \\
    0 & \textrm{otherwise}
    \end{cases}.
\end{align}

In this work, if unbounded operators are present, the remainder terms in the expansions that we consider become undefined and some form of a truncation is needed to ensure that the mathematics is well defined.  Provided that an appropriate cutoff is picked for the system, the discrepancies between the truncated and untruncated systems will often be negligible. For notational clarity, we assume a cutoff of $\Lambda$ for all further equations and assume the annihilation and creation operators implicitly have the projectors $P_\Lambda$.


%$The relationship between these operators and the position operator forms the translation guide needed to move between these two equivalent descriptions of the qumodes.

% \how{we need a discussion about why we can use cutoffs and the reduced notation for it}

% \how{ie need to consolidate $\Tilde{a} \coloneqq P_\Lambda a P_\Lambda $}

The final aspect that we need to talk about in our description of the state is the qubits.  In our model of computing we assume that the vector space is a tensor product of a qubit Hilbert space and the vector space of qumodes, i.e., the state space is $ \mathcal{H}_2 \kron \mathcal{H}_{\Lambda}$. This means that, for example, a computational basis state will be of the form $\ket{q} \kron \ket{m}_b $ where the state $\ket{q}$ here can be thought of as the union of the qubit registers in the system, and $\ket{m}_b$ represents a qumode state where the system is either in position $x=m$ for the phase-space encoding or has $m$ photons if the Fock-space encoding is used. Our work will assume only a single qubit register with a qumode in the Fock-space encoding.



\subsubsection{Valid operations}
Now that we have described the vector spaces that our operators act in, we will discuss the bosonic operations that act on the qumodes for the system. In order to introduce this, we need to review some notation surrounding bosonic creation and annihilation operators (otherwise known as raising and lowering or Fock operators). 
In the following, we demonstrate how some operations can be analytically decomposed, regardless of whether they are defined with Fock operators ($a, a^\dagger$) or phase-space operators ($\hat{x}, \hat{p}$), as these have the following equivalences:
\begin{align}
    \hat{x} = \frac{1}{2} (a + a^\dagger) \qquad &\Leftrightarrow \qquad a = \hat{x} + i \hat{p} \label{eq:AnnihilationOperatortoPhaseSpace},\\
    \hat{p} = -\frac{i}{2} (a - a^\dagger) \qquad &\Leftrightarrow \qquad a^\dagger = \hat{x} - i \hat{p}, \label{eq:CreationOperatortoPhaseSpace}
\end{align}
and commutation relations:
\begin{align}
    [\hat{x}, \hat{p}] &= \frac{i}{2} \label{eq:Commutatorxp}, \\
    [a, a^\dagger] &= 1 \label{eq:CommutatorCreationAnnihilation}.
\end{align}

There are many gate operations that can be considered as part of an instruction set. Within our ISA, operations can be qubit or qumode-exclusive or entangle across the qubit and qumode. For example, we can still perform single-qubit operations like $H$ and $R_z$.

Additionally, we can assume that linear optical operations (which are at most quadratic in the field operators) can be performed on the modes, such as the displacement operations $e^{ \alpha a^\dagger + \alpha^* a }$ with $\alpha$ a complex number, phase delays (or phase-space rotations) $e^{-i\alpha  a^\dagger a }$, squeezing operations $e^{\alpha a^2 -\alpha^{*}a^2}$, and beamsplitter operations $e^{(\alpha a^\dagger b - \alpha^{*}a b^{\dagger})}$ where $b$ is the creation operator acting on a different mode (although in this paper we are mainly interested in the single-mode case).  We further assume that the qubit can be measured directly, but the oscillator can only be measured by entangling it with a qubit.

Finally, there are several entangling operations between oscillator and qubit that widely appear in the circuit QED literature.  Two common ones are the controlled displacement operation \cite{eickbusch2021fast} $e^{-i(\alpha a^\dagger + \alpha a) \otimes \sigma^z}$ and the Selective Number-dependent Arbitrary Phase (SNAP) gate \cite{SNAP-PhysRevLett.115.137002} $e^{-i\sum_n\alpha_n \hat P_n \otimes \sigma^z}$, where $\hat P_n=|n\rangle\langle n|$ is the projector onto the $n$th Fock state. (In our paper, we refer to hybrid gates with $\sigma^i$ notation, while single-qubit gates like $S, X, H$ are capitalized).
%SMG: The SNAP gate puts a different phase $\theta_n$ on each Fock state.
Here, for convenience, we focus on a different gate, which we name $\mathcal{S}_1$, which we assume we can implement without error. While this gate can be approximated with arbitrary fidelity using conditional displacements (see \cref{obtaining_s1}), in practice it can be implemented straightforwardly with OCT \cite{PhysRevX.4.041010,PhysRevA.91.043846,Rosenblum2018,Rosenblum2023}. 
\begin{define}\label{defn:SX}
For any $t>0$ and any positive integer cutoff $\Lambda$, we define $\mathcal{S}_1$ to be the unitary acting on the Hilbert space $\mathcal{H}_2 \otimes \mathcal{H}_{\Lambda}$ that has the following representation as a block-matrix
\begin{align}
    \mathcal{S}_1 = \exp \left(it \begin{bmatrix}
    0 & a^{\dagger} \\
    a & 0
    \end{bmatrix} \right).
\end{align}
\end{define}

Notice that this gate operates simultaneously on the boson and qubit modes (producing, in the language of ion-trap quantum computers, a `red-sideband' transition). In particular, the block-encoded matrix is equivalent to the following cavity-qubit coupling operator in the Jaynes-Cummings model:
\begin{align}\label{eq:JC}
    \begin{bmatrix}
    0 &  a^\dagger  \\
     a & 0
    \end{bmatrix} = \ket{0}\bra{1} \kron a^\dagger  + \ket{1}\bra{0} \kron  a.
\end{align}

The Jaynes-Cummings Hamiltonian is the natural one in circuit QED, therefore the coupling $\mathcal{S}_1$ is naturally present. However, one can choose an experimental parameter regime in which the coupling becomes negligible. This happens when the cavity and qubit are strongly detuned, and they obey the dispersive coupling Hamiltonian. This is a useful regime, which may be chosen, for example, when needing to use the controlled displacement gate. The $S_1$ block encoding can be synthesized in this regime, and this is demonstrated in Appendix~\ref{obtaining_s1}.

The reason why we focus on this incredibly simple example is to provide a case that is likely to have an advantage in our hybrid model.  Specifically, the further we are away from the natural interactions that are present in our quantum simulator the more expensive we anticipate a simulation to be, as we will have to construct the desired interaction from a potentially long sequence of fundamental interactions.  In this sense, our approach borrows inspiration from analog simulation.  Unlike analog simulation, the interactions here are universal in the sense that we can emulate \emph{any Jaynes-Cummings model} using our approach.

% EC: This commented out version is what was there before my edits - just leaving this here in case you'd rather go back to it.
{
% Integer occupancy levels of the oscillator are called \textit{Fock} states and denoted $\ket{n}$, where $n$ is the number of bosons. While there could theoretically be an arbitrary number of bosons in the oscillator, in practice, we consider the oscillator to be bounded by some constant $\Lambda$. 

% Our oscillator has two primary operators which act upon it: annihilation and creation operators. These operate similar to fermionic operators, in that they increment / decrement the number of bosons present, and thus transform Fock state to a different Fock state. However, unlike fermionic operators, bosonic operators do not have finite matrix representations; again, we must assume some $\Lambda$ in order to produce a finite matrix form. For example, assuming $\Lambda = 3$:
% \begin{align}
%     P_3 a^\dagger P_3 = \begin{bmatrix}
%     0 & 0 & 0 & 0 \\
%     1 & 0 & 0 & 0 \\
%     0 & \sqrt{2} & 0 & 0 \\
%     0 & 0 & \sqrt{3} & 0 
%     \end{bmatrix}, P_3 a P_3 = \begin{bmatrix}
%     0 & 1 & 0 & 0\\
%     0 & 0 & \sqrt{2} & 0 \\
%     0 & 0 & 0 & \sqrt{3} \\
%     0 & 0 & 0 & 0
%     \end{bmatrix}
% \end{align}
% %\textbf{Above we should use a projection to discuss the truncation}
% Here $P_\Lambda$ is the projector onto the subspace of the cavity containing at most $\Lambda$ photons, i.e.:
% \begin{align}
%     P_\Lambda : P_\Lambda \ket{n} = \begin{cases}
%     \ket{n} & n \leq \Lambda \\
%     0 & \textrm{Otherwise}
%     \end{cases}
% \end{align}
% Note that, in practice, it is common to ignore the truncation onto a finite-dimensional space for the bosonic operations.  We avoid doing so here, except where essential, because the remainder terms in the \ck{ec comment} expansions that we consider become undefined if unbounded operators are present.  Provided that an appropriate cutoff is picked for the system, the discrepancies between the truncated and untruncated systems will often be negligible.

%Integer occupancy levels of the mode are called \textit{Fock} states and denoted $\ket{n}$, where $n$ is the number of bosons. Our mode has two primary operators which act upon it: annihilation and creation operators. These operate similar to fermionic operators, in that they increment / decrement the number of bosons present, and thus transform Fock state to a different Fock state. 
}




\subsection{Lie product formulas}
Lie product formulas describe the behavior when matrix exponentials are multiplied, i.e., how $e^A e^B$ behaves. These formulas are well-known in Hamiltonian simulation~\cite{childs2021theory,berry2007efficient,su2021fault} and are used to approximate a discretized version of the time evolution operator $e^{-iHt}$ using the Trotter formula. However, in our application, we demonstrate how these formulas can be used as a control scheme for our device.

We introduce two product formulas: the Baker-Campbell-Hausdorff (BCH) formula and the Lie-Trotter formula. The BCH formula can be used to create a commutator (or anticommutator) of operators. The Trotter formula can be used to add these commutators/anticommutators together. We introduce the informal theorems below:
\begin{theorem}[Informal BCH theorem from \cite{Childs_2013}]\label{thm:BCH}
    Suppose we can implement the operators $e^{A \lambda}, e^{B \lambda}$ for $\lambda \in \mathbb{R}$. Then, a BCH formula of order $p$ has the following error scaling:
    \begin{align}
        \bch_p(A\lambda, B \lambda) &= e^{A \lambda} e^{B \lambda} e^{-A \lambda} e^{-B \lambda}\\&= e^{[A, B] \lambda^2} + \mathcal{O}((\max (\norm{A}, \norm{B}) \lambda)^{2p + 1}), 
    \end{align}
    requiring no more than $8 \cdot 6^{ p -1}$ exponentials.
\end{theorem}

\begin{theorem}[Informal Trotter theorem from \cite{berry2007efficient}]
    Suppose we may implement $e^{A \lambda}$ and $e^{B \lambda}$ for arbitrary $\lambda \in \mathbb{R}$. Then, a $p^\text{th}$ order Trotter formula has the following error scaling:
    \begin{align}
        \trotter_{2p}(A\lambda, B \lambda) & = e^{(A+B)\lambda} + \mathcal{O}((\norm{A + B} \lambda)^{2p + 1}),
    \end{align}
    requiring no more than $4 \cdot 5^{ p -1}$ exponentials.
\end{theorem}

Though the BCH formula is defined only to produce commutators, we can use our qubit to produce anticommutators of bosonic operations. Informally, we exploit commutators of Pauli operations to create a phase change, as $\sigma^z = -\frac{i}{2} [\sigma^x, \sigma^y]$. Thus, BCH can be used to create an anticommutation relation on the mode operators
\begin{align}
    i \left[ i A \sigma^x, i B \sigma^{y }\right] = \{ A, B \} \sigma^z.\label{eq:sigmazterm}
\end{align}
We later show how these commutators and anticommutators can be combined to directly implement products of operators. Observing that $\frac{1}{2}[A, B] + \frac{1}{2} \{A , B\} = AB$, we can use the Trotter formula to produce:
\begin{align}
    \trotter \left(\frac{1}{2} [A, B], \frac{1}{2} \{ A, B \} \right) \approx \exp (AB).
\end{align}
 

% \begin{theorem}[Informal Trotter theorem]
%     Suppose we have $H = \sum_i H_i$ and we may implement all $e^{H_i \lambda}$ for arbitrary $\lambda \in \mathbb{R}$. Then, a $p$th order Trotter formula has the following error scaling:
%     \begin{align}
%         \trotter_{2p}(\{ H_i \} , \lambda) & = e^{\lambda \sum_i H_i} \\&= \left(\prod_i e^{(H_i \lambda / p)}\right)^p + \mathcal{O}(\lambda^p)\\&= e^{\sum H_i \lambda} + \mathcal{O}((\norm{H}\lambda)^{2p + 1})
%     \end{align}
% \end{theorem}



\section{Producing Anticommutators and Exponential Products}
In this section, we formalize our technique and present methods for both hermitian operators, such as phase-space operators $\hat{x}$ and $\hat{p}$, and non-hermitian operators, such as Fock space operators $a, a^\dagger$.  We examine the task of constructing polynomials in phase space as well as Fock space operators using qubit controlled operations.  We first show how the additional qubit of our hybrid boson-qubit architecture allows for the synthesis of anticommutators of hermitian operators and, by proxy, matrix products of phase-space operators. We then use similar techniques with non-hermitian operators such as Fock-space operators to manipulate block encodings of matrices. Finally, we contextualize these methods with asymptotic error bounds, providing theoretical analyses of our proposed techniques.

We denote block-encoded matrices with the following notation:
\begin{align}\label{eq:blockencodedmatrix}
    \mathcal{B}_A = \exp it \begin{bmatrix}
        0 & A \\
        A^\dagger & 0
    \end{bmatrix},
\end{align}
where the subscript is the upper right block and the lower left block is the transpose and complex conjugate to preserve hermiticity. This block encoded gate corresponds to $\exp{itA\sigma^x}$ only if $A$ is hermitian, otherwise it corresponds to $\exp{it( \ket{0}\bra{1} \kron A^\dagger  + \ket{1}\bra{0} \kron  A)}$, the synthesis of which we show in Appendix~\ref{obtaining_s1}.




% \begin{landscape}

\begin{table}
    \small
    \centering
    \begin{tabularx}{\textwidth}{ccll}
        \hline
        \textbf{Formula} & \textbf{Target} &\textbf{Preconditions} & \textbf{Reference} \\
        \hline
        $\textrm{BCH}(\exp it B \sigma^i, \exp it A \sigma^i)$ & $\exp( t^2 [A, B]) \identity$ &$A, B$ Hermitian &\cref{thm:BCH}  \\
        %%%%
        $\textrm{BCH}(\exp it A \sigma^j, \exp it B \sigma^k) $ & $\exp (- i t^2 \{ A, B \} \sigma^i)$ &$A, B$ Hermitian &\cref{eq:sigmazterm} \\
        %%%%%
                \hline
         $\textrm{BCH}(X  \mathcal{B}_B(t) X, \mathcal{B}_A(t) ) $ & $\exp (t^2 (AB - (AB)^\dagger) \sigma^z)$ & $[A,B]=0$ &\cref{eq:C2}\\
        %%%%%%
         $\textrm{BCH}(S \mathcal{B}_A(t) S^\dagger, X \mathcal{B}_B(t) X) $ & $\exp (i t^2 (AB + (AB)^\dagger) \sigma^z) $ & $[A,B]=0$ & \cref{eq:AC2}\\
        %%%%
         $\textrm{BCH}(S \mathcal{B}_A(t) S^\dagger, X \mathcal{B}_B(t) X) $ & $\exp (2 i t^2 AB \sigma^z )$ & \makecell[l]{$[A,B]=0$, \\$AB = (AB)^\dagger $} &\cref{eq:Product} \\
        %%%%
         $\textrm{Trotter}\Big(\exp t (AB - (AB)^\dagger) \sigma^y, $ & $\exp \Big(2 it \begin{bmatrix}
            0 & BA \\
            AB & 0
        \end{bmatrix} \Big)$ & $[A,B]=0$ &\cref{thm:general-adder-error}\\
     \qquad$ \exp it (AB + (AB)^\dagger) \sigma^x\Big)$ & \\
         $\textrm{BCH}(S \mathcal{B}_B(t) S^\dagger, X \mathcal{B}_A(t) X)$ & $\exp \left(it \begin{bmatrix}
            2 AB & 0 \\
            0 & -BA - (BA)^\dagger
        \end{bmatrix}\right)$ & $AB = (AB)^\dagger$ &\cref{lem:multiplication-alg}
        %%%%
        % $\mathcal{B}_A(t), \mathcal{B}_B(t)$ & $\textrm{Trotter}(\exp t [A, B] \sigma^y, \exp it \{ A, B \} \sigma^x) $ & $\exp 2 it \begin{bmatrix}
        %     0 & BA \\
        %     AB & 0
        % \end{bmatrix}$ 
    \end{tabularx}
    \caption{Overview of techniques for synthesizing particular unitary transformations and the quantum gates needed in order to build the constructs proposed in this paper. Each row contains the formula used, the target to approximate, the preconditions, and a reference to the location of the precise statement of the performance of the method. The formula provided denotes hybrid gates with $\sigma^i$ terms, while single-qubit gates are capitalized (e.g., $S, X, H$). The bounds on the number of gates depend on the accuracy required of the approximation and are given in the corresponding theorems. 
    % \how{ sigmai is xyz; single-qubits are capital, sigmas are part of exponentials. be definition repetitive} 
    %\how{move info to preconditions column}
    }
    \label{tab:my_label}
\end{table}
% \how{the table has a mistake on eq 46 and theorem 3.2 -- those are the same formulas and should be aligned}

% \begin{table}[]
%     \centering
%     \begin{tabularx}{\textwidth}{Xlc}
%         \hline
%         \textbf{Available gates} & \textbf{Formula} & \textbf{Target} \\
%         \hline
%         $\exp it A \sigma^j, \exp it B \sigma^j$ & $\textrm{BCH}(\exp it B \sigma^i, \exp it A \sigma^i)$ & $\exp t^2 [A, B] \identity$ \how{?} \\
%         %%%%
%         $\exp it A \sigma^j, \exp it B \sigma^j $ & $\textrm{BCH}(\exp it A \sigma^j, \exp it B \sigma^k) $ & $\exp t^2 \{ A, B \} \sigma^i$ \\
%         %%%%%
%         $\mathcal{B}_A (t), \mathcal{B}_B(t)$ & $\textrm{BCH}(\sigma^x \mathcal{B}_B(t) \sigma^x, \mathcal{B}_A(t) ) $ & $\exp (t^2 (AB - (AB)^\dagger) \sigma^z)$ \\
%         %%%%%%
%         $\mathcal{B}_A(t), \mathcal{B}_B(t)$ & $\textrm{BCH}(S \mathcal{B}_A(t) S^\dagger, \sigma^x \mathcal{B}_B(t) \sigma^x) $ & $\exp (i t^2 (AB + (AB)^\dagger) \sigma^z) $ \\
%         %%%%
%         $\mathcal{B}_A(t), \mathcal{B}_B(t)$ & $\textrm{BCH}(S \mathcal{B}_A(t) S^\dagger, \sigma^x \mathcal{B}_B(t) \sigma^x) $ & $\exp 2 i t^2 AB \sigma^z $ \footnote{When $AB$ Hermitian, i.e., $AB = (AB)^\dagger$. This occurs, for example, when $A, B$ are annihilation / creation operators of the same order.} \\
%         %%%%
%         $\mathcal{B}_A(t), \mathcal{B}_B(t)$ & $\textrm{Trotter}(\exp t (AB - (AB)^\dagger) \sigma^y, \exp it (AB + (AB)^\dagger) \sigma^x) $ & $\exp 2 it \begin{bmatrix}
%             0 & BA \\
%             AB & 0
%         \end{bmatrix}$  \\
%         %%%%
%         $\mathcal{B}_A(t), \mathcal{B}_B(t)$ & $\textrm{Trotter}(\exp t [A, B] \sigma^y, \exp it \{ A, B \} \sigma^x) $ & $\exp 2 it \begin{bmatrix}
%             0 & BA \\
%             AB & 0
%         \end{bmatrix}$ 
%     \end{tabularx}
%     \caption{Overview of contributed control techniques}
%     \label{tab:my_label}
% \end{table}

% \end{landscape}
% \how{update formula for second to last row}

% \how{should go in introduction (?)}

% \how{eq 29 has the def of mathcal B (linked)} 


\subsection{Phase-space methods: anti-commutators of Hermitian operators}
Here, we demonstrate how to synthesize anti-commutation relations between Hermitian operators, as is the case for the phase space operators $\hat{x}$ and $\hat{p}$ (but not Fock space operators). To do this, we use qubit gates to introduce the phase change necessary for obtaining an anticommutator. These methods can be used to  generate beamsplitters and the Hong-Ou Mandel effect, as is explored in~\cref{sec:applications}. 

% Our native gates on the hybrid device can be defined in terms of phase-space operators; our hope is to apply BCH on these native gates, thereby yielding commutators of $\hat{x}, \hat{p}$. It remains to show that we can also produce $\{ \hat{x}, \hat{p}\}$. 

Our technique applies the BCH formula on a hybrid boson-qubit operation. Thus, we need to study how the commutator affects both bosonic and qubit operators. In the bosonic case, observe:
\begin{align}
    [\hat{x}, \hat{p}] = \frac{i}{2}.
\end{align}
In the qubit case, observe that:
\begin{align}
    \sigma^{i} & =\epsilon_{ijk}\frac{i}{2}\left[\sigma^{j},\sigma^{k}\right],\label{eq:PauliCommutator}\\
    % \sigma^{x} & =-\frac{i}{2}\left[\sigma^{y},\sigma^{z}\right],\label{eq:PauliCommutatorx}\\
    % \sigma^{y} & =-\frac{i}{2}\left[\sigma^{z},\sigma^{x}\right],\label{eq:PauliCommutatory}\\
    % \sigma^{z} & =-\frac{i}{2}\left[\sigma^{x},\sigma^{y}\right]\label{eq:PauliCommutatorz}.
\end{align}
where $\epsilon_{ijk}$ is the Levi-Civita symbol, and $i,j,k \in \{x,y,z\}$.

We use these relations, as well as the Pauli product identity 
\begin{align}
\sigma^{i} & =\epsilon_{ijk}i\sigma^{j}\sigma^{k}.\label{eq:PauliProduct}
% \sigma^{x} & =-i\sigma^{y}\sigma^{z}\label{eq:PauliProductx},\\
% \sigma^{y} & =-i\sigma^{z}\sigma^{x}\label{eq:PauliProducty},\\
% \sigma^{z} & =-i\sigma^{x}\sigma^{y}\label{eq:PauliProductz},
\end{align}
to enable the expression of the product of a single Pauli operator and an anticommutator\emph{
}$\left\{ A,B\right\} \sigma_{i}$ in terms
of the product of modes and single Pauli operators in the form of a commutator, as follows:
\begin{align}
\left\{ A,B\right\} \sigma^{i} & = AB\left(i\sigma^{j}\sigma^{k}\right)+BA\left(i\sigma^{k}\sigma^{j}\right)\\
 & =i\left(-\left(A\sigma^{j}\right)\left(B\sigma^{k}\right)+BA\left(\sigma^{k}\sigma^{j}\right)\right)\\
 & =i\left(\left(iA\sigma^{j}\right)\left(iB\sigma^{k}\right)-\left(iB\sigma^{k}\right)\left(iA\sigma^{j}\right)\right)\\
 & =i\left[iA\sigma^{j},iB\sigma^{k}\right]\label{eq:AnticommutatortoCommutator-1}.
\end{align} 
% \begin{align}
% \left\{ A,B\right\} \sigma^{i} & =-AB\left(-i\sigma^{j}\sigma^{k}\right)+BA\left(i\sigma^{k}\sigma^{j}\right)\\
%  & =i\left(-\left(A\sigma^{j}\right)\left(B\sigma^{k}\right)+BA\left(\sigma^{k}\sigma^{j}\right)\right)\\
%  & =i\left(\left(iA\sigma^{j}\right)\left(iB\sigma^{k}\right)-\left(iB\sigma^{k}\right)\left(iA\sigma^{j}\right)\right)\\
%  & =i\left[iA\sigma^{j},iB\sigma^{k}\right]\label{eq:AnticommutatortoCommutator-1}
% \end{align} 
assuming $A$ and $B$ are Hermitian, and commute with $\sigma^{i}$ as is the case when $A, B$ are mode-only operators. Thus, by using a hybrid qubit-cavity operation of the form $\exp i A \sigma^j, \exp i B \sigma^k$, with $A$ and $B$ Hermitian, the BCH formula can convert commutators into anticommutators. 
% Intuitively, we have exploited the qubit to introduce a phase change.
% These relations are sufficient to demonstrate the following equality:
% \begin{align}
% \left\{ A,B\right\} \sigma^{i} & =i\left[iA\sigma^{j},iB\sigma^{k}\right]\label{eq:AnticommutatortoCommutator}
% \end{align}

% Finally, an argument consisting of a product of operators is expressed
% as
% \begin{align}
% AB\sigma^{j} & =\frac{1}{2}\left\{ A,B\right\} \sigma^{j}+\frac{1}{2}\left[A,B\right]\sigma^{j}\label{eq:ProductOperators}\\
% \left\{ A,B\right\}  & =AB+BA\\
% \left[A,B\right] & =AB-BA
% \end{align}
% which is further decomposed in terms of the Pauli anticommutator-commutator
% relation Eq.~(\ref{eq:AnticommutatortoCommutator}).

%%% the below section is omitted because it is not true
%%% [A, B] \sigma^j, \{ A, B \} \sigma^j cannot simultaneously be antihermitian unless one (or both) is zero.
% Finally, our results will require the creation of both commutators and anticommutators. We observe that, by block-encoding the matrices, anticommutators naturally follow from BCH. For example, given commuting operators $\Tilde{A}, \Tilde{B}$,
% \begin{align}
%     \sigma^x \left\{ \Tilde{A}, \Tilde{B} \right\}  = i \left[ i \sigma^y \Tilde{A} , i \sigma^z \Tilde{B}  \right]
% \end{align}
% Where $\sigma^i \Tilde{A}$ represents a hybrid qubit-boson operation. Thus, by taking $\Tilde{A}, \Tilde{B}$ to be readily implementable operations, many of which are defined in terms of $a, a^\dagger$ or $\hat{x}, \hat{p}$, it seems intuitive that we could implement sums of commutators / anticommutators of these operators.

% \subsection{Fock methods}

\subsection{Fock methods - non-linear terms via block-encodings}
In this section, we show how to achieve $A^q$ for an arbitrary bosonic operator $A$. This extends the prior techniques, which required that $A$ be Hermitian. When $A$ is a Fock space operator, this corresponds to realizing arbitrary powers of $a$ and $a^{\dagger}$, which is known to generate a universal set of operations on the bosonic modes. This is useful for the simulation of non-linear materials which naturally lead to terms that are polynomial in the creation and annihilation operators, as well as quantum signal processing~\cite{martyn2021grand}, for example. 

Our method again uses the qubit coupling to induce a phase in the compound qubit-boson system, similar to the previous section. We begin with block-encodings as described in Eq.~\ref{eq:blockencodedmatrix}. First, we show how to create auxiliary block encodings via qubit-only gates, a technique known as conjugation. Second, we show how to generate squeezing and non-linear terms using the BCH formula. Third, we note that qubit rotations lead to quadratic block encodings. Fourth, we show that these methods may be extended to multiple Hilbert spaces. Finally, we present a formal algorithm and associated error bound to attain these block-encodings.


% To achieve a similar effect with Fock operators $a, a^\dagger$, we can also use the qubit coupling to induce phase changes. In particular, we instead observe that $A \sigma^x$ is comparable to the following block encoding: \how{explain what a block encoding is}
% \begin{align}
%     \mathcal{B}_A(t) = \exp it \begin{bmatrix}
%         0 & A \\
%         A^\dagger & 0
%     \end{bmatrix},
% \end{align}
% where we require the lower block-encoding to be $A^\dagger$ to enforce hermiticity and thus unitarity of the operation. Observe that when we take $A = \hat{x}$ that $\mathcal{B}(t)$ is precisely $\exp i t A \sigma^x$. Suppose we took $A = a^\dagger$ so that\textcolor{blue}{Refer to equation (7)}:
% \begin{align}
%     \exp it \begin{bmatrix}
%         0 & a^\dagger \\
%         a & 0
%     \end{bmatrix}.
% \end{align}
% \how{what are we doing? why do we care about the X s x terms}

% While the phase-space block encoding could be expressed as a product between a qubit and bosonic operator due to the hermitian nature of the phase-space operator, the Fock-space block encoding is not as easily separable. Thus, we must be more careful about the intermixing of terms. 

% Luckily, we still can implement $\exp i t \{ A, B \} \sigma^j $, provided that $A, B$ commute. We also demonstrate an implementation of $\exp t [A, B] $. Together, these two constituents suggest that we may also implement $\exp it AB \sigma^j$ via the following relation:

To manipulate the block-encodings, begin by recognizing that qubit-only operations can modify the exponential via ``conjugation":
\begin{align}
    U e^A U^\dagger = e^{U A U^\dagger}.
\end{align}
Thus, given any block-encoding, we can also create the following auxiliaries:
\begin{align}
    X \exp it \begin{bmatrix}
        0 & A \\
        A^\dagger & 0
    \end{bmatrix} X &= \exp it \begin{bmatrix}
        0 & A^\dagger \\
        A & 0
    \end{bmatrix}, \\
    S \exp it \begin{bmatrix}
        0 & A \\
        A^\dagger & 0
    \end{bmatrix} S^\dagger &= \exp it \begin{bmatrix}
        0 & -i A \\
        i A^\dagger & 0
    \end{bmatrix},
\end{align}
where $S$ is a qubit phase gate. Applying BCH yields the following commutators:
\begin{align}
    \left[X \mathcal{B}(t) X, \mathcal{B}(t) \right] = \left[ it \begin{bmatrix}
        0 & A^\dagger \\
        A & 0
    \end{bmatrix} , it \begin{bmatrix}
        0 & A \\
        A^\dagger & 0
    \end{bmatrix} \right] &= - t^2 \left( \begin{bmatrix}
        (A^\dagger)^2 & 0 \\
        0 & A^2
    \end{bmatrix} - \begin{bmatrix}
        A^2 & 0 \\
        0 & (A^\dagger)^2
    \end{bmatrix} \right) \\
    &= t^2 \sigma^z \kron (A^2 - (A^\dagger)^2), \\
    \left[S \mathcal{B}(t) S^\dagger, X \mathcal{B}(t) X \right] = \left[ it \begin{bmatrix}
        0 & -i A \\
        i A^\dagger & 0
    \end{bmatrix}, it \begin{bmatrix}
        0 & A^\dagger \\
        A & 0
    \end{bmatrix} \right] &= -t^2 \left( \begin{bmatrix}
        - i A^2 & 0 \\
        0 & i (A^\dagger)^2
    \end{bmatrix} - \begin{bmatrix}
        i (A^\dagger)^2 & 0 \\
        0 & - iA^2
    \end{bmatrix} \right) \\
    &= i t^2 \sigma^z \kron (A + (A^\dagger)^2).
\end{align}
These commutators themselves can be conjugated. Recall that $H Z H = X$ and $SH Z HS^\dagger = Y$:
\begin{align}
    SH \left[X \mathcal{B}(t) X, \mathcal{B}(t) \right] HS^\dagger &=t^2 \sigma^y \kron (A^2 - (A^\dagger)^2), \\
    H \left[S \mathcal{B}(t) S^\dagger, X \mathcal{B}(t) X \right] H &= i t^2 \sigma^x \kron (A + (A^\dagger)^2),
\end{align}
so that:
\begin{align}
    i t^2 \sigma^x \kron (A + (A^\dagger)^2) + t^2 \sigma^y \kron (A^2 - (A^\dagger)^2) = 2 it^2 \begin{bmatrix}
        0 & (A^\dagger)^2 \\
        A^2 & 0
    \end{bmatrix}.
\end{align}
This approach can also be extended to cases where the constituent operators are of different types, e.g. when the synthesized unitary operates on two different modes, as in the conditional beamsplitter which is a gate that acts as a beamsplitter controlled on an ancillary qubit. For example, consider the following commutator:
\begin{align}
    \left[ i\tau \begin{bmatrix}
        0 & B^\dagger \\
        B & 0
    \end{bmatrix}, i\tau \begin{bmatrix}
        0 & A \\
        A^\dagger & 0
    \end{bmatrix} \right] &= -\tau^2 \left( \begin{bmatrix}
        B^\dagger A^\dagger & 0 \\
        0 & BA 
    \end{bmatrix} - \begin{bmatrix}
        AB & 0 \\
        0 & A^\dagger B^\dagger
    \end{bmatrix} \right) \\
    &= \tau^2 \begin{bmatrix}
        AB - (AB)^\dagger & 0 \\
        0 & (BA)^\dagger - BA
    \end{bmatrix}, \\
    %%%%%%
    \left[ i\tau \begin{bmatrix}
        0 & -iA \\
        iA^\dagger & 0
    \end{bmatrix}, i\tau \begin{bmatrix}
        0 & B^\dagger \\
        B & 0
    \end{bmatrix}\right] &= -\tau^2 \left( \begin{bmatrix}
        - iAB & 0 \\
        0 & i A^\dagger B^\dagger 
    \end{bmatrix} - \begin{bmatrix}
        i B^\dagger A^\dagger & 0 \\
        0 & -iBA 
    \end{bmatrix} \right) \\
    &= i\tau^2 \begin{bmatrix}
        AB + (AB)^\dagger & 0 \\
        0 & -BA - (BA)^\dagger
    \end{bmatrix}.
\end{align}

Observe that when $[A, B] = 0$, as is the case when $A, B$ are both annihilation/creation operators or when they act on different modes,  we can rewrite the commutators in the following forms:
\begin{align}
    \left[ i\tau \begin{bmatrix}
        0 & B^\dagger \\
        B & 0
    \end{bmatrix}, i\tau \begin{bmatrix}
        0 & A \\
        A^\dagger & 0
    \end{bmatrix} \right] = -\tau^2 \begin{bmatrix}
        AB - (AB)^\dagger & 0 \\
        0 & (AB)^\dagger - AB
    \end{bmatrix} &= \tau^2 (AB - (AB)^\dagger) \sigma^z, \\
    \left[ i\tau \begin{bmatrix}
        0 & -iA \\
        iA^\dagger & 0
    \end{bmatrix}, i\tau \begin{bmatrix}
        0 & B^\dagger \\
        B & 0
    \end{bmatrix}\right] = 
    i\tau^2 \begin{bmatrix}
        AB + (AB)^\dagger & 0 \\
        0 & - AB - (AB)^\dagger
    \end{bmatrix} &= i\tau^2 (AB + (AB)^\dagger) \sigma^z,
\end{align}
so that, again:
\begin{align}
    SH \tau^2 (AB - (AB)^\dagger) \sigma^z HS^\dagger &= \tau^2 (AB - (AB)^\dagger) \sigma^y \label{eq:C2},\\
    H i\tau^2 (AB + (AB)^\dagger) \sigma^z H &= i\tau^2 (AB + (AB)^\dagger) \sigma^x\label{eq:AC2}.
\end{align}
Via Trotter, we can finally implement the sum:
\begin{align}
    \tau^2 (AB - (AB)^\dagger) \sigma^y  + i\tau^2 (AB + (AB)^\dagger) \sigma^x   = 2 i \tau^2 \begin{bmatrix}
        0 & (AB)^\dagger \\
        AB & 0
    \end{bmatrix}\label{eq:Product}.
\end{align}
Finally, we select $\tau = \sqrt{\frac{t}{2}}$ to obtain the desired time and conjugate by $\sigma^x$ to produce the desired matrix. Observe that \cref{alg:adder} thus yields $\mathcal{B}_{AB}$, provided that $[A, B] = 0$. This process can be repeated iteratively, assuming $AB$ commutes with $B$; for example, if $A = B = a$, then this process can be used to produce higher powers $a^k, (a^\dagger)^k$ of the annihilation / creation operators.

\cref{alg:adder} is an extension of the prior commutator approaches in phase space because the $\sigma^i = - \frac{i}{2} [\sigma^j, \sigma^k]$ relation is natively expressed in the algorithm; i.e., if we have $\mathcal{B}_{A} = \mathcal{B}_B =  \mathcal{B}_{\hat{x}} = \exp it \hat{x} \sigma^x$, the ``$\textrm{Left}$" term vanishes and the``$\textrm{Right}$" term is the commutator we would apply.

% \how{...}

Finally, in \cref{alg:mult}, we demonstrate how to implement $AB$ if $AB = (AB)^\dagger$. This process cannot be performed recursively, contrary to \cref{alg:adder}, because it places the terms in the upper-left block encoding. However, this actually may be more useful, as it allows the precise simulation of $e^{i AB t}$, assuming the qubit begins in the $\ket{0}$ state:
\begin{align}
    \left[ i \tau \begin{bmatrix}
        0 & -i B^\dagger \\
        i B & 0
    \end{bmatrix}, i \tau \begin{bmatrix}
        0 & A \\
        A^\dagger & 0
    \end{bmatrix} \right] &= \tau^2 \begin{bmatrix}
        i B^\dagger A^\dagger + i A B  & 0 \\
        0 &  - i B A -i A^\dagger B^\dagger
    \end{bmatrix} \\
    &= i \tau^2 \begin{bmatrix}
        2 AB & 0 \\
        0 & -BA - (BA)^\dagger
    \end{bmatrix}.
\end{align}
% \how{is the bottom right block equality true? I don't think so}

%%%
% The product of two Hermitian matrices is only Hermitian when [A, B] = 0 because (AB)dag = Bdag Adag = B A
 
% Is it that:
% \begin{align}
%     AB = (AB)^\dagger \implies BA = (BA)^\dagger
% \end{align}
% \begin{align}
%     BA = BA - AB + AB = AB - [A, B]
% \end{align}
% So that:
% \begin{align}
%     (AB - [A, B])^\dagger \implies [A, B] = ([A, B])^\dagger
% \end{align}
% When is it that the commutator is also Hermitian?

\subsection{Error analysis}
The prior description of our algorithm assumes error-less product formulas. However, the BCH and Trotter formulas actually introduce errors which must be accounted for, especially when applying our algorithm recursively. In this section, we cite the error scaling of the general addition algorithm described in \cref{alg:adder} and the multiplication algorithm described in \cref{alg:mult}. The proofs and full results are included in \cref{apndx:error-analysis}.

\begin{algorithm}[t]
\caption{ADD($\mathcal{B}_A (t), \mathcal{B}_B(t), p_l, p_r, t$) }\label{alg:adder}
\begin{algorithmic}
\Require $[A, B] = 0$, $\norm{\mathcal{B}_A - \exp it \begin{bmatrix}
    0 & A \\
    A^\dagger & 0
\end{bmatrix} } \in \mathcal{O}((c_A t)^{p_A})$, $\norm{\mathcal{B}_B - \exp it \begin{bmatrix}
    0 & B \\
    B^\dagger & 0
\end{bmatrix} } \in \mathcal{O}((c_B t)^{p_B})$, $p_A, p_B \geq 1$, $t > 0$
\Ensure $\mathcal{B}_{AB}$ where $\norm{\mathcal{B}_{AB}(t) - \exp it \begin{bmatrix}
    0 & AB \\
    (AB)^\dagger & 0
\end{bmatrix}} \in \mathcal{O}((Ct)^{\min (p_A, p_B) / 2})$ for constant $C$
\State $q \coloneqq \max ( \ceil{\frac{1}{2}(\min(p_l, p_r) - 1)}, 1 )$
\State $s \coloneqq \max ( \ceil{\frac{1}{2} ( \min(p_l, p_r) - 1)}, 1 )$
\State $\tau \coloneqq \sqrt{t / 2}$
%%%%%%
\State Left $\coloneqq \bch_{q, 1}(X \cdot \mathcal{B}_B(t) \cdot X, \mathcal{B}_A(t))$
\State Right $\coloneqq \bch_{q, 1}(S\cdot \mathcal{B}_A(t) \cdot S^\dagger, X \cdot \mathcal{B}_B(t) \cdot X)$
\State Left' $\coloneqq SH \cdot \textrm{Left} \cdot HS^\dagger$
\State Right' $\coloneqq H \cdot \textrm{Right} \cdot H$ 
%%%%
\State \Return $X \cdot \trotter_s(\textrm{Left'}, \textrm{Right'}) \cdot X$
\end{algorithmic}
\end{algorithm}

% \how{coloneqq instead of left arrow for gets?}

\begin{algorithm}[t]
\caption{MULT($\mathcal{B}_A (t), \mathcal{B}_B(t), p_l, p_r, t$) }\label{alg:mult}
\begin{algorithmic}
\Require $AB = (AB)^\dagger$, $p_l, p_r \geq 1$, $t > 0$
\Ensure An upper-left block encoding $\mathcal{M}_{AB}$ where $\norm{\mathcal{M}_{AB}(t) - \exp it \begin{bmatrix}
    AB & 0 \\
    0 & \frac{1}{2} ( - BA - (BA)^\dagger)
\end{bmatrix}} \in \mathcal{O}( (C^2 t)^{\min (p_A, p_B) / 2} )$ for constant $C$
\State $q \coloneqq \max ( \ceil{\frac{1}{2}(\min(p_l, p_r) - 1)}, 1 )$
\State $\tau \coloneqq \sqrt{t / 2}$
\State \Return $\textrm{BCH}_q(S \mathcal{B}_{B}(\tau) S^\dagger, X \mathcal{B}_{A}(\tau) X) $
\end{algorithmic}
\end{algorithm}

% \begin{theorem}[Analysis of \cref{alg:adder}]\label{lem:adder}
% Assume we can implement the following $k_l, k_r$th order approximations of $S$ with error scaling $p_l, p_r$:
% \begin{align}
%     \norm{- \mathcal{S}_{k_l}(t)} &\in \mathcal{O}((c t)^{p_l}) \\
%     \norm{ - \mathcal{S}_{k_r}(t)} &\in \mathcal{O}((c t)^{p_r})
% \end{align}
% With $c \geq \Lambda^{\max(k_l, k_r) / 2}$. Then, we can implement higher order operators with comparable $t$ scaling:
% % \textbf{The notation below could use a definition maybe separately, and unification with the SX notation used before.  S seems to be the same concept but just with a different time argument.}
% \begin{align}
%     \norm{
%         \exp \left( i t \begin{bmatrix}
%     0 & (P_\Lambda a^\dagger P_\Lambda)^{k_l + k_r} \\
%     (P_\Lambda a P_\Lambda)^{k_l + k_r} & 0
%     \end{bmatrix}
%     \right) - \widetilde{\mathcal{S}}_{k_l + k_r, \min(p_l, p_r)}(t)
%     } \in \mathcal{O}((c^2 t)^{\min(p_l, p_r) / 2})
% \end{align}
% Using no more than $1.07 \cdot 30^q$ $\mathcal{S}_{k_l},  \mathcal{S}_{k_r}$ operators.
% \end{theorem}

% \how{COPY / PASTE UPDATED PROOF} FROM \cref{thm:general-adder-error}


% We apply the above results to produce the error analysis of \cref{alg:adder}:
\begin{restatable}{theorem}{algproduct}\label{thm:general-adder-error}
    Suppose we have approximations $\widetilde{\mathcal{B}}_A(t), \widetilde{\mathcal{B}}_B(t)$ with the following error scaling:
    \begin{align}
        \norm{\widetilde{\mathcal{B}}_A(t) - \mathcal{B}_A(t)} &\in \mathcal{O}( (ct)^{p_A} ), \\
        \norm{\widetilde{\mathcal{B}}_B(t) - \mathcal{B}_B(t)} &\in \mathcal{O} ((ct)^{p_B} ), 
    \end{align}
    for some constant $c$ and order $p_A, p_B \geq 1$ where $[A,B]=0$. Then, the application of \cref{alg:adder} will yield the following scaling:
    \begin{align}
        \norm{\textrm{ADD}(\widetilde{\mathcal{B}}_A(t), \widetilde{\mathcal{B}}_B(t) ) - \exp it \begin{bmatrix}
            0 & AB \\
            BA & 0 
        \end{bmatrix}} \in \mathcal{O} \left((C_{TOTAL} t)^{\min(p_A, p_B) / 2}\right),
    \end{align}
    with $C_{TOTAL} = \max( \norm{AB}, \norm{BA}, C_{BCH}^2)$ and $C_{BCH} = \max( \norm{A}, \norm{B}, c)$,
    using no more than $ 1.07 \cdot 30^q $ exponentials, where $q = \max (\ceil{\frac{\min(p_1, p_2) - 1}{2}}, 1) $. 
\end{restatable}


% ..

\begin{restatable}{theorem}{algmult}\label{lem:multiplication-alg}
    Suppose we have some approximate block encodings $\widetilde{\mathcal{B}}_A, \widetilde{\mathcal{B}}_B$  with the following error:
    \begin{align}
        \norm{\widetilde{\mathcal{B}}_A(t) - \exp it \begin{bmatrix}
            0 & A \\
            A^\dagger & 0
        \end{bmatrix}} &\in \mathcal{O}( (ct)^{p_A}), \\
        \norm{\widetilde{\mathcal{B}}_B(t) - \exp it \begin{bmatrix}
            0 & B \\
            B^\dagger & 0
        \end{bmatrix}} &\in \mathcal{O} ((ct)^{p_B}),
    \end{align}
    for constant $c$ and $p_A, p_B \geq 1$ where $AB = (AB)^\dagger$.
    Then, \cref{alg:mult} has the following error:
    \begin{align}
        \norm{ \textrm{MULT}(\widetilde{\mathcal{B}}_A(t),  \widetilde{\mathcal{B}}_B(t)) - \exp it \begin{bmatrix}
            AB & 0 \\
            0 & \frac{1}{2}( - BA - (BA)^\dagger)
        \end{bmatrix}} \in \mathcal{O}\left((C^2 t)^{\min (p_A, p_B) / 2}\right),
    \end{align}
    % \begin{align}
    %     \norm{ \textrm{BCH}_q(S \widetilde{\mathcal{B}}_B(\tau) S^\dagger, \widetilde{\mathcal{B}}_A(\tau) ) - \exp it \begin{bmatrix}
    %         AB & 0 \\
    %         0 & - BA - (BA)^\dagger
    %     \end{bmatrix}} \in \mathcal{O}((C^2 t)^{\min (p_A, p_B) / 2})
    % \end{align}
    with $C = \max(\norm{A}, \norm{B}, c)$, using no more than $8 \cdot 6^{q - 1}$ exponentials where $q = \max ( \ceil{\frac{\min(p_A, p_B) - 1}{2}}, 1) $.
\end{restatable}

While the asymptotic error analysis suggests that the cost of this method is onerous, we note that the product formulas often have overly pessimistic error scaling and operation counts \cite{zhao2021hamiltonian}. In the applications below, we provide numerical simulations which suggest our technique is more readily implementable than theory suggests.



% \subsection{Parallel between Fock methods and phase-space methods}


%% Chris' section 
% Do we want/can we make an error analysis about BCH error in Micheline's solutions?

%% Thm Corrollary are scary! May need to remove 
%% also need to spell out more of the intermediate steps for the derivations

\section{Applications}\label{sec:applications}
In this section, we show how our technique is a powerful tool for analytically realizing desired operations. This technique not only works for Hamiltonian simulation problems, but also for general control problems. In particular, we show how the physical intuition for a desired transformation is often sufficient to produce an approach to create desired operations.

% \how{move to four applications}
We introduce applications in both phase- and Fock-space. Phase-space techniques may be useful in the case where, for example, displacements ($e^{(\alpha a^\dag + \alpha^* a)} = e^{i \alpha \vec{x}}$ for $\alpha$ real or $e^{i \alpha \vec{p}}$ for $\alpha$ imaginary) are the only experimentally available gates. We combine these position $\vec{x}$ and momentum $\vec{p}$ operators with single-qubit rotations $\vec{\sigma}$ to produce the phase-space rotation operator, earlier referred to as the controlled parity operator, $e^{i a^{\dagger}a \sigma^z t}$; the beamsplitter $e^{-i\sigma^z (a^{\dagger}b + ab^{\dagger}) t}$; gates for two encodings of universal control of the restricted $\text{span}\{\left|0\right>,\left|1\right>\}$ Hilbert space (\cref{subsec:Universal-Control}); and gates for simulation of Fermi-Hubbard lattice dynamics using the vacuum and $1^\text{st}$ Fock states of the cavity (\cref{sec:Fermi-Hubbard-Lattice-Dynamics}), including same-site, hopping, controlled-beamsplitter (\cref{subsec:controlled-phase}), and FSWAP gates. For applications in Fock space, we use qubit-controlled displacements $e^{\sigma^z(\alpha^* a + \alpha a^{\dagger})}$, controlled parity maps $e^{\sigma^z a^{\dagger}a}$ and single-qubit operations $\vec{\sigma}$ to produce polynomials of annihilation and creation operators, i.e., $a^p {a^\dagger}^q$ for integer $p, q$. We then demonstrate how polynomials of these operators can be used in Hamiltonian simulation (e.g. with $\chi^{(3)}$ nonlinear materials, \cref{apndx:error-jc}) and state preparation (\cref{subsec:state-prep}).


% For Hamiltonian simulation, we demonstrate a variety of potential systems which are simulatable,  
% We also contribute a variety of control techniques which are realizable, including the conditional rotation gate (\cref{subsec:cond-rot}), controlled-phase beam splitter gate (\cref{subsec:controlled-phase}) state preparation of arbitrary Fock states (\cref{subsec:state-prep}), and the embedding of an effective qubit in the bosonic mode (\cref{subsec:Universal-Control}).

% \how{ck: need to link into the missing appendices, describe the type of ham sim / control being done and link it}

\subsection{Nonlinear Hamiltonian simulation}
As a simple application, let us consider the case of simulating a $\chi^{(3)}$ nonlinear material. These interactions commonly occur in nonlinear optics and appear when the index of refraction for a material varies linearly with the intensity of the electromagnetic field.  Such interactions can be modeled for a single mode using the following expression
\begin{equation}\label{eq:desired-ham}
    H = \omega a^\dagger a + \frac{\kappa}{2}(a^\dagger)^2 a^2.
\end{equation}
Our goal here is to examine the cost of a simulation of such a Hamiltonian in our model for time $t$ and error tolerance $\epsilon$ and determine the parameter regimes within which a hybrid simulation using our techniques could provide an advantage with respect to a conventional qubit-based simulation of the Hamiltonian. 

Each of the terms can be approximated by using the BCH formula. For the $\omega a^\dagger a$ term, observe that we can approximate this term with a single application of the BCH formula. Namely,
\begin{align}
    \left[ i \tau_1 \begin{bmatrix}
        0 & -i  a^\dagger  \\
        i  a  & 0
    \end{bmatrix},
    i \tau_1 \begin{bmatrix}
        0 &  a^\dagger  \\
         a  & 0
    \end{bmatrix}  \right] = 2 i \tau_1^2 \begin{bmatrix}
         a^\dagger   a  & 0 \\
        0 & - a a^\dagger  
    \end{bmatrix},
\end{align}
where the first matrix corresponds to $\mathcal{S}_1^y = S \cdot \mathcal{S}_1 \cdot S^\dagger $ operator and the second matrix corresponds to $\mathcal{S}_1$ operator. Similarly, observe that:
\begin{align}
    \left[ i \tau_2 \begin{bmatrix}
        0 & -i (a^\dagger)^2 \\
        i a^2 & 0
    \end{bmatrix}, i \tau_2 \begin{bmatrix}
        0 & (a^\dagger)^2 \\
        a^2 & 0
    \end{bmatrix} \right] = 2 i \tau_2^2 \begin{bmatrix}
        (a^\dagger)^2 a^2 & 0 \\
        0 & - a^2 (a^\dagger)^2
    \end{bmatrix}.
\end{align}
Thus, via BCH formula, we can block-encode the two Hamiltonian terms. Applying Trotterization allows us to block-encode the entire Hamiltonian into the upper-left quadrant. Thus, by setting the qubit to $\ket{0}$, we can approximate the Hamiltonian. The error scaling is as follows and is proven in \cref{apndx:error-jc}

\begin{restatable*}{theorem}{resultjc}\label{app:jaynes-cummings}
 Let
$
    H = \omega a^\dagger a + \frac{\kappa}{2}(a^\dagger)^2 a^2
$, $t$ be an evolution time and $\epsilon$ be an error tolerance.
For any positive integer $q$ we can approximate an exponential of the block-encoded Hamiltonian with error at most $\epsilon$ in the operator norm using $r e^{\mathcal{O}(q)}$ $\mathcal{S}_1$ operations where $r \in \Omega\left( \frac{(\Lambda^{4} t)^{1 + 1 / (q - \frac{3}{4})} }{\epsilon^{1 / (q - \frac{3}{4})}} \right)$. 
% \how{is it reOq}
\end{restatable*}


%\how{nathan: what numerics do we want here?}

% get plots from \href{https://colab.research.google.com/drive/1yxOpabROaSCA-vOkNKu6QlYNxOuFtylm#scrollTo=Ff4t89vuDJEt}{here}

This shows that we can perform a simulation of the dynamics within error $\epsilon$ using a number of operations within our instruction set that scales near-linearly with the evolution time and subpolynomially with $\epsilon$.  Further, this approach requires no ancillary memory and can be done with a single oscillator and a qubit.  This is a dramatic memory reduction relative to the quantum case, which requires a polylogarithmic number of qubits in $\Lambda$.

It is worth noting that in this case the ancillary qubit is not being used directly in the model.  Instead it is being used to control the dynamics and generate the appropriate nonlinear interaction between the photons present in the model.

\subsection{Nondestructive measurement of the qumode}
% \how{Notes for reviewer:}
% \begin{enumerate}
%     \item Where do we describe the full ISA?
%     \item 
% \end{enumerate}

% ---
We now demonstrate how the approach can extend beyond problems in Hamiltonian simulation. We begin with an example of the technique for control. In particular, we seek to perform a nondestructive measurement of the qumode. Natively, measurements are destructive, i.e., counting the number of photons also consumes the photons. However, by using the qubit, we can instead project the information into the qubit~\cite{fockreadout_wang_2020, Fockshotresolved_curtis_2021}. 

% \how{is this actually a valid application}

Intuitively, we seek to implement  $e^{i t \hat{n} \sigma^z}$ where $\hat{n} = a^\dagger a$ is the number operator. If we could implement this gate for arbitrary $t$, we could perform phase estimation on the qubit to nondestructively project the qumode into a fixed number of bosons. This could be done by setting $t$ sufficiently small so that $t \Lambda \leq 2 \pi$ is calculable with phase estimation. Alternatively, for $t = \pi$, this operation checks the parity of the qumode and applies an RZ gate for odd parities. We employ the instruction set in the phase-space representation to synthesize the infinitesimal conditional rotation gate
\begin{equation}
U_{\text{rot},k}=e^{i\lambda^{2}\hat{n}\sigma^{k}}
\end{equation}
for $k=x,y,z$. We rewrite $\hat{n}$ in terms of the phase-space operators by recognizing:
\begin{align}
\hat{n} & =\hat{a}^{\dagger}\hat{a}\\
 & =\hat{x}^{2}+\hat{p}^{2}-\frac{1}{2}\label{eq:NumbertoPhaseSpace}.
\end{align}
Thus, by applying Eq.~\ref{eq:NumbertoPhaseSpace},
Eq.~\ref{eq:CreationOperatortoPhaseSpace}, and Eq.~\ref{eq:AnnihilationOperatortoPhaseSpace}
yields
\begin{equation}
i\lambda^{2}\hat{n}\sigma^{k}=i\left(\hat{x}^{2}+\hat{p}^{2}-\frac{1}{2}\right)\sigma^{k}\lambda^{2},
\end{equation}
such that the gate is expressed via the Trotter-Suzuki decomposition as the product of $\exp\left(\left[A_{1},B_{1}\right]\right)=\exp\left(i\lambda^{2}\hat{x}^{2}\sigma^{k}\right)$,
$\exp\left(\left[A_{2},B_{2}\right]\right)=\exp\left(i\lambda^{2}\hat{p}^{2}\sigma^{k}\right)$,
and conditional displacement $\exp\left(-i\lambda^{2}\sigma^{k}/2\right)$.
Given the Pauli commutator relation, the
first commutator is 
\begin{align}
\left[A_{1},B_{1}\right] & =i\hat{x}^{2}\sigma^{k}\\
 & =i\hat{x}^{2}\left(-\frac{i}{2}\left[\sigma^{i},\sigma^{j}\right]\right)\\
 & =\left[\frac{1}{\sqrt{2}}\hat{x}\sigma^{i},\frac{1}{\sqrt{2}}\hat{x}\sigma^{j}\right],
\end{align}
and the second commutator is 
\begin{align}
\left[A_{2},B_{2}\right] & =i\hat{p}^{2}\sigma^{k}\\
 & =i\hat{p}^{2}\left(-\frac{i}{2}\left[\sigma^{i},\sigma^{j}\right]\right)\\
 & =\left[\frac{1}{\sqrt{2}}\hat{p}\sigma^{i},\frac{1}{\sqrt{2}}\hat{p}\sigma^{j}\right],
\end{align}
such that both terms are amenable to BCH decomposition,
and the infinitesimal conditional rotation is composed with a gate-depth lower bound of nine. To perform an error analysis, we may directly apply the error scaling of BCH and Trotter to find:

\begin{restatable*}{theorem}{resultmeasurement}
Suppose we can implement $\text{e}^{ it \hat{x} \sigma^i}, \text{e}^{ it \hat{p} \sigma^i}$ without error. Then, we may approximate $\mathcal{B}_{\hat{x}^2 + \hat{p}^2}$ with arbitrary error scaling $p$:
\begin{align}
    \norm{ \widetilde{\mathcal{B}}_{\hat{x}^2 + \hat{p}^2} - \mathcal{B}_{\hat{x}^2 + \hat{p}^2} } \in \mathcal{O}( (C t)^{p + 1/ 2}),
\end{align}
where $C = \max( \norm{ \hat{x}^2 + \hat{p}^2 }, \norm{\hat{x}}^2, \norm{\hat{p}}^2)$ and using no more than $4 \cdot 5^{ \frac{p}{2} - \frac{1}{4} }$ exponentials.
\end{restatable*}
% Where the Pauli rotation can be applied without implementation error or Trotter error. 

We then provide numerics in \cref{fig:ConditionalPhase}. As expected, the wavefunction
initialized in the second excited state of the cavity and the ground
state of the associated transmon has an autocorrelation function that
oscillates with phase $\exp(2it)$. Dynamics are well reproduced with
$2000$ time steps for a final time of $20$ with
$15$ states in the cavity. Note the units are arbitrary in the absence
of definition of the cavity frequency $\omega$, with the only units
defined by setting the reduced Planck constant to unity $\hbar=1\text{ arb. units}$.
The close agreement between the BCH-synthesized and exact gates is
supported by the error scaling after a single gate application computed for time step $t$, which features a power law scaling in agreement with the predicted error scaling
for both BCH and Trotter decompositions.\begin{figure}[!ht]
\begin{centering}
\includegraphics[width=0.5\columnwidth]{pics/ConditionalRotation/bch.png}\includegraphics[width=0.5\columnwidth]{pics/ConditionalRotation/error.png}
\par\end{centering}
\caption{The following plots characterize the performance of using phase-space operators to synthesize $e^{i \hat{n} \sigma^z t}$. (a) The BCH-synthesized conditional rotation gate $e^{i\hat{n}\sigma^{z}t}$
successfully reproduces the exact dynamics for a wavefunction initialized
in the ground state of the transmon and the second excited state of
the cavity.  (b) Error of the real part of the autocorrelation
function for the BCH-synthesized gate after a single time step of length $t$.\label{fig:ConditionalPhase}}
\end{figure}


% \how{will need to show how this can also be done via the fock-space operators}

We can obtain a similar decomposition with Fock-space operators. Observe that the $\textrm{MULT}$ subroutine applied to the $\mathcal{B}_a$, $\mathcal{B}$ would yield the following operators:
\begin{align}
	\norm{\textrm{MULT}(\mathcal{B}_a, \mathcal{B}_{a^\dagger}) - \exp i t \begin{bmatrix}
		a^\dagger a & 0 \\
		0 & - a a^\dagger
	\end{bmatrix} }.
\end{align}
Note that $a a^\dagger = a^\dagger a + \identity$ provided we understand this operator to be acting on vectors that have no support on the singularity,  so the block encoding that is actually applied is actually the following:
\begin{align}
\exp i t\left\{\begin{bmatrix}
	a^\dagger a & 0 \\
	0 & - a^\dagger a 
\end{bmatrix} + \begin{bmatrix}
	0 & 0 \\
	0 & - \identity
\end{bmatrix}\right\} = \exp it ( \hat{n} \sigma^z - \identity_\gamma (\identity - \sigma^z) ).
\end{align}
Thus, our Fock-space methods would also achieve the same transformation, albeit requiring a phase and RZ correction. 
% \how{error scaling}


\subsection{State preparation from the vacuum}
Consider the case where we seek to prepare $\ket{k}_b$ on the qumode. On qubit devices, this sort of state prep is trivial, assuming the qubits represent logic in a binary fashion. However, hybrid boson-qubit devices natively implement exponentials of the phase-space or Fock-space operators. Thus, preparing $\ket{k}_b$ directly can often be challenging. 

To begin, we recognize that we intuitively aim to implement $(a^\dagger)^k$ on the vacuum. It is sufficient to approximate:
\begin{align}
    \mathcal{T}_k(t) \coloneqq \exp i t\begin{bmatrix}
        0 & (a^\dagger)^k \\
        a^k & 0
    \end{bmatrix}.
\end{align}
Because selecting appropriate $t$ yields precisely the desired behavior, we have the following result:
\begin{restatable*}{theorem}{statepreptime}
    For $k \leq \Lambda$, we can take $t = (2n + 1) \frac{\pi}{2 \sqrt{k!}}$ for any $n \in \mathbb{N}$ so that:
    \begin{align*}    
    \mathcal{T}_k(t) \ket{1} \kron \ket{0} = \ket{0} \kron \ket{k}. 
    \end{align*}
\end{restatable*}
Note here that we can implement such a mapping using the Baker-Campbell-Hausdorff formula (BCH).  



\newcommand{\diag}{{\rm{diag}}}
\newcommand{\stateprep}{\mathcal{P}_k}
\newcommand{\stateprepapprox}{\widetilde{\mathcal{P}}_{k, p}}

While $\mathcal{T}_k (t)$ performs the desired transformation, it may incur unwanted side effects if the starting state is of the form $\ket{1} \kron \ket{b}$ for $b > 0$. We can use our same approach to produce the following operation:

\begin{restatable*}{theorem}{resultstateprep}\label{lem:fock-prep-unitary}
Consider the Fock preparation unitary $\stateprep$ with the following form:
\begin{align*}
    \exp \left( i t \begin{bmatrix}
        0 & ( a^\dagger )^k \ketbra{0}{0} \\
        \ketbra{0}{0} ( a )^k & 0
    \end{bmatrix} \right).
\end{align*}
% 
% \begin{align}
%     \exp \left( it \begin{bmatrix}
%         0 & p_{k, 1} \\
%         p_{1, k} & 0 
%     \end{bmatrix} \right)
% \end{align}
% Where $p_{1, k}$ refers to a $(\Lambda + 1) \times (\Lambda + 1)$ matrix where the only element is in the $1$st row and $k$th column with value $\sqrt{k!}$ ($p_{k, 1}$,  respectively). 
%
When $t = (2n + 1) \frac{\pi}{4 \sqrt{k!}} $, we have that $\stateprep$ performs our desired state preparation:
\begin{align*}
    \exp \left( i t \begin{bmatrix}
        0 & ( a^\dagger )^k \ketbra{0}{0} \\
        \ketbra{0}{0} ( a )^k & 0
    \end{bmatrix} \right) \ket{1} \kron \ket{b}  = \begin{cases}
    \ket{0} \kron  \ket{k} & b = 0 \\
    \ket{1} \kron  \ket{b} & b \neq 0
    \end{cases}.
\end{align*}
We claim that we can approximate this unitary with $\stateprepapprox$ where:
\begin{align*}
    \norm{\stateprepapprox - \stateprep} \in \mathcal{O}((\Lambda^{k/2} t)^p),
\end{align*}
using no more than $4 \cdot 5^{q -1}$ $\widetilde{\mathcal{T}}_{k,p}$ subroutines.

% \begin{align}
%     \exp \left( it \begin{bmatrix}
%         0 & p_{0, k} \\
%         p_{k, 0} & 0 
%     \end{bmatrix} \right) \ket{1} \ket{b} \mapsto \begin{cases}
%     \ket{0} \ket{k} & b = 0 \\
%     \ket{1} \ket{b} & b \neq 0
%     \end{cases}
% \end{align}
\end{restatable*}
Where the proofs are provided in \cref{state_prep_proof}. Though this subroutine appears expensive, numerical results suggest it is far more implementable than theory would suggest. In the following simulations, we apply the above technique but always use a second-order symmetrized BCH formula and second-order (symmetrized) Trotter formula. This amounts to 480 exponentials for the unprotected case and 960 exponentials for the protected case. The error plot is provided in \cref{fig:unprotectedt2} and \cref{fig:protectedt2}.  
\begin{figure}[!ht]
    \centering
    \includegraphics[scale=0.6]{pics/stateprep/originalT2gate.png}
    \caption{Unprotected $T_2$ Gate: (a) is the exact form of the $T_2$ gate with selected $t$ and (b) is the BCH-synthesized form. The state $\ket{j}_q \ket{k}_m$ has index $j \cdot \Lambda + k$ where $\Lambda$ is the cutoff. By observation, the analytically realized form is accurate.}
    \label{fig:unprotectedt2}
\end{figure}
\begin{figure}[!ht]
    \centering
    \includegraphics[scale=0.6]{pics/stateprep/fullT2gate.png}
    \caption{Protected $T_2$ Gate: (a) is the exact form of the protected $T_2$ gate with selected $t$ and (b) is the BCH-synthesized form.   The state $\ket{j}_q \ket{k}_m$ has index $j \cdot \Lambda + k$ where $\Lambda$ is the cutoff. By observation, the analytically realized form is still accurate, albeit with more incurred Trotter error.}
    \label{fig:protectedt2}
\end{figure}

% \includegraphics[scale=0.6]{pics/draft_t2_unprotected.png}

% \includegraphics[scale=0.6]{pics/draft_t2_protected.png}

We also analyze the error scaling as the order of the BCH formulas used increases. \cref{fig:protectedt2-error} describes the error-resource tradeoff as the Trotter step within each BCH formula increases.
\begin{figure}[!ht]
    \centering
    \mbox{
    \subfigure[]{\includegraphics[scale=0.5]{pics/stateprep/errorfidT2norm.png}}\quad
    \subfigure[]{\includegraphics[scale=0.5]{pics/stateprep/errorT2norm.png}}
    }
    \caption{Error scaling of protected $T_2$ gate with respect to BCH circuit depth. Prime denotes that the formula has been symmetrized. (a) uses the state-fidelity metric while (b) uses the matrix norm.}
    \label{fig:protectedt2-error}
\end{figure}

\subsection{Hong-Ou-Mandel effect/conditional (controlled-phase) beam splitter gate}\label{subsec:Conditional-(Controlled-Phase)-Beam-Splitter}
% \how{should be rewritten}
The operations we seek to realize need not act on a single mode; in fact, our techniques are extensible to hybrid setups with multiple modes or qubits. Consider the conditional (controlled-phase) beam splitter
\begin{align}
U_{\text{beam split}} & =e^{-i\lambda^{2}\left(\hat{a}_{1}^{\dagger}\hat{a}_{2}+\hat{a}_{1}\hat{a}_{2}^{\dagger}\right)\sigma^{z}}.
\end{align}
This gate naturally pertains
to certain lattice gauge theories and gives rise to exponential SWAP
(eSWAP) \cite{gao2019entanglement} and controlled-SWAP (cSWAP)
gates for state purification and SWAP tests, when paired with an uncontrolled
beam splitter \cite{pietikainen2022controlled}. 
The argument in phase-space representation 
% Eq.~\ref{eq:CreationOperatortoPhaseSpace} and Eq.~\ref{eq:AnnihilationOperatortoPhaseSpace} 
is 
\begin{align}
-i\lambda^{2}\left(\hat{a}_{1}^{\dagger}\hat{a}_{2}+\hat{a}_{1}\hat{a}_{2}^{\dagger}\right)\sigma^{z} & =-2i\lambda^{2}\left(\hat{x}_{1}\hat{x}_{2}+\hat{p}_{1}\hat{p}_{2}\right)\sigma^{z},
\end{align}
such that the gate is decomposed in terms of a Trotter-Suzuki expansion
% Eq.~\ref{eq:TrotterFirstOrder} 
as the product of two exponential
terms $\exp\left(\left[A_{1},B_{1}\right]\lambda^{2}\right)=\exp\left(-2i\lambda^{2}\hat{x}_{1}\hat{x}_{2}\sigma^{z}\right)$
and $\exp\left(\left[A_{2},B_{2}\right]\lambda^{2}\right)=\exp\left(-2i\lambda^{2}\hat{x}_{1}\hat{x}_{2}\sigma^{z}\right)$.
According to the Pauli commutation relation 
the first commutator is
\begin{align}
\left[A_{1},B_{1}\right] & =-2i\hat{x}_{1}\hat{x}_{2}\sigma^{z}\\
 & =-2i\hat{x}_{1}\hat{x}_{2}\left(-\frac{i}{2}\left[\sigma^{x},\sigma^{y}\right]\right)\\
 & =\left[i\hat{x}_{1}\sigma^{x},i\hat{x}_{2}\sigma^{y}\right],
\end{align}
and the second is
\begin{align}
\left[A_{2},B_{2}\right] & =\left[i\hat{p}_{1}\sigma^{x},i\hat{p}_{2}\sigma^{y}\right],
\end{align}
with the following error scaling:

% \how{error bounds ...}
\begin{restatable*}{theorem}{resultbeamsplitter}
Assume we may implement $\text{e}^{ i t \hat{x}_m \sigma^j}, \text{e}^{ i t \hat{p}_m \sigma^j}$ for $m \in \{ 1, 2 \}$; i.e., we may implement the qubit-conditional position shifts and momentum boosts on either mode without error. Then, we may approximate $\mathcal{B}_{\hat{x}_1 \hat{x}_2 + \hat{p}_1 \hat{p}_2}$ with arbitrary error scaling $p$:
\begin{align*}\norm{\widetilde{\mathcal{B}}_{\hat{x}_1 \hat{x}_2 + \hat{p}_1 \hat{p}_2} - \mathcal{B}_{\hat{x}_1 \hat{x}_2 + \hat{p}_1 \hat{p}_2} } \in \mathcal{O}((Ct)^{p + \frac{1}{2}}),
\end{align*}
where $C = \max( \norm{ \hat{x}_1 \hat{x}_2 + \hat{p}_1 \hat{p}_2}, \norm{\hat{x}_1}^2, \norm{\hat{x}_2}^2, \norm{\hat{p}_1}^2, \norm{\hat{p}_1}^2)$ and using no more than $4 \cdot 5^{ \frac{p}{2} - \frac{1}{4} }$ exponentials.
\end{restatable*}


The two exponential terms are in decomposed via the BCH formula 
% Eq.~\ref{eq:BCHFormula}
for a lower-bound gate depth of eight. Results are shown in Fig.~\ref{fig:ConditionalBeamSplitter} for
$15$ states per cavity with a shared transmon over a final time of
$\pi/2$ with $200$ equal time
steps, where the system is initially in the first excited state of
each cavity and the ground state of the shared transmon $\left|11g\right>$.
As expected for the conditional beam splitter, the gate exhibits the
Hong-Ou-Mandel effect, in which the occupation of cavity 1 oscillates
between the first excited mode and a superposition of the ground and
the second excited states of the cavity. The BCH-synthesized results
closely agree with that of the original gate, with no visible leakage
beyond the physical states (the lowest three states of the cavity)
into the working space under the time duration studied. As for the
conditional rotation gate, the relative error of the BCH-synthesized
gate computed for a single time step of length $t$ was found to scale according to a power law with the time step,
in accordance with the analytic result for Trotterization and BCH
decomposition.

\begin{figure}[!ht]
\begin{centering}
\includegraphics[width=0.5\columnwidth]{pics/BeamSplitter/probabilityexact}\includegraphics[width=0.5\columnwidth]{pics/BeamSplitter/probabilitybch}
\par\end{centering}
\begin{centering}
\includegraphics[width=0.5\columnwidth]{pics/BeamSplitter/probabilityworkbch}\includegraphics[width=0.5\columnwidth]{pics/BeamSplitter/error}
\par\end{centering}
\caption{Hong-Ou-Mandel effect simulated with (a) exact and (b) BCH-synthesized
conditional beam splitters, illustrated as probability cavity 1 is found in states $\left|0\right>$, $\left|1\right>$, or $\left|2\right>$; (c) probability of leakage into higher cavity modes; and (d) error of the real part of the autocorrelation
after a single application of a BCH-synthesized gate for time step $t$ relative to the application of the exact gate for the same time step.\label{fig:ConditionalBeamSplitter}}%The notation with the single Fock states and two-mode Fock states is not explained.
\end{figure}



% \how{numerics}


{
% \section{Qubit-Conditional Cavity Gates\label{sec:Qubit-Conditional-Cavity-Gates}}

% We begin by employing the analytic ISA to synthesize qubit-conditional
% gates commonly required in bosonic quantum computing applications.

% \subsection{SUM Gate}

% Consider the two-cavity infinitesimal conditional SUM gate
% \begin{equation}
% U_{\text{SUM}}=e^{-i\lambda^{2}\hat{x}_{1}\hat{p}_{2}\sigma^{z}}
% \end{equation}
% where $\hat{x}_{1}$ is the position operator of cavity $1$, $\hat{p}_{2}$
% is the momentum operator of cavity $2$, and the Pauli-Z gate $\sigma^{z}$
% acts on a transmon coupled to both cavities. In Gottesman-Preskill-Knill
% (GKP) codes, the gate enables stabilizer measurements and CNOT operations
% \cite{royer2022encoding} and provides an $x_{1}$-position-dependent
% momentum boost $\hat{p}_{2}$ and equivalently a $p_{2}$-momentum-dependent
% position displacement $\hat{x}_{1}$.

% To decompose the exponential gate, its argument is expressed in terms
% of a commutator via the Pauli commutation relation Eq.~(\ref{eq:PauliCommutatorz})
% as follows:
% \begin{align}
% \left[A,B\right]\lambda^{2} & =-i\hat{x}_{1}\hat{p}_{2}\sigma^{z}\lambda^{2}\\
%  & =-\hat{x}_{1}\hat{p}_{2}\left(-\frac{i}{2}\left[\sigma^{x},\sigma^{y}\right]\right)\lambda^{2}\\
%  & =\left[\frac{i}{\sqrt{2}}\hat{x}_{1}\sigma^{x},\frac{1}{\sqrt{2}}\hat{p}_{2}\sigma^{y}\right]\lambda^{2}
% \end{align}
% \how{phat sigma y is not anti-hermitian}
% The gate is then expressed according to the BCH decomposition Eq.~(\ref{eq:BCHFormula})
% with gate depth lower bound of four.

% \subsection{Conditional Single-Cavity Squeezing Gate}

% The infinitesimal squeezing gate assumes the form

% \begin{equation}
% U_{\text{one-mode squeeze}}\approx e^{\lambda^{2}\left(\hat{a}^{\dagger2}-\hat{a}^{2}\right)\sigma^{z}}
% \end{equation}
% and aids generation of GKP states \cite{hastrup2021measurement,hastrup2021unconditional}.
% Given the relationship between the ladder and phase-space operators
% Eq.~\ref{eq:AnnihilationOperatortoPhaseSpace} and \ref{eq:CreationOperatortoPhaseSpace},
% the argument is expressed as
% \begin{align}
% \lambda^{2}\left(\hat{a}^{\dagger2}-\hat{a}^{2}\right)\sigma^{z} & =-2i\lambda^{2}\left\{ \hat{x},\hat{p}\right\} \sigma^{z}
% \end{align}
% which is in turn expressed as a commutator according to the Pauli
% anticommutator-commutator relation Eq.~\ref{eq:AnticommutatortoCommutator}
% \begin{align}
% \left\{ \hat{x},\hat{p}\right\} \sigma^{z}\lambda^{2} & =2i\left[i\hat{x}\sigma^{x},i\hat{p}\sigma^{y}\right]\lambda^{2}
% \end{align}
% to yield the exponential commutator
% \begin{equation}
% \left[A,B\right]\lambda^{2}=\left[\sqrt{2}i\hat{x}\sigma^{x},\sqrt{2}i\hat{p}\sigma^{y}\right]\lambda^{2}
% \end{equation}
% The infinitesimal squeezing gate is therefore decomposed according
% to the BCH formula Eq.~\ref{eq:BCHFormula} with lower-bound gate
% depth of four.

% \subsection{Conditional (Controlled-Phase) Beam Splitter Gate\label{subsec:Conditional-(Controlled-Phase)-Beam-Splitter}}

% Consider the conditional (controlled-phase) beam splitter
% \begin{align}
% U_{\text{beam split}} & =e^{-i\lambda^{2}\left(\hat{a}_{1}^{\dagger}\hat{a}_{2}+\hat{a}_{1}\hat{a}_{2}^{\dagger}\right)\sigma^{z}}
% \end{align}
% The argument in phase-space representation Eq.~\ref{eq:CreationOperatortoPhaseSpace}
% and Eq.~\ref{eq:AnnihilationOperatortoPhaseSpace} is 
% \begin{align}
% -i\lambda^{2}\left(\hat{a}_{1}^{\dagger}\hat{a}_{2}+\hat{a}_{1}\hat{a}_{2}^{\dagger}\right)\sigma^{z} & =-2i\lambda^{2}\left(\hat{x}_{1}\hat{x}_{2}+\hat{p}_{1}\hat{p}_{2}\right)\sigma^{z}
% \end{align}
% such that the gate is decomposed in terms of a Trotter-Suzuki expansion
% Eq.~\ref{eq:TrotterFirstOrder} as the product of two exponential
% terms $\exp\left(\left[A_{1},B_{1}\right]\lambda^{2}\right)=\exp\left(-2i\lambda^{2}\hat{x}_{1}\hat{x}_{2}\sigma^{z}\right)$
% and $\exp\left(\left[A_{2},B_{2}\right]\lambda^{2}\right)=\exp\left(-2i\lambda^{2}\hat{x}_{1}\hat{x}_{2}\sigma^{z}\right)$.
% According to the Pauli commutation relation Eq.~(\ref{eq:PauliCommutatorz}),
% the first commutator is
% \begin{align}
% \left[A_{1},B_{1}\right] & =-2i\hat{x}_{1}\hat{x}_{2}\sigma^{z}\\
%  & =-2i\hat{x}_{1}\hat{x}_{2}\left(-\frac{i}{2}\left[\sigma^{x},\sigma^{y}\right]\right)\\
%  & =\left[i\hat{x}_{1}\sigma^{x},i\hat{x}_{2}\sigma^{y}\right]
% \end{align}
% and the second is
% \begin{align}
% \left[A_{2},B_{2}\right] & =\left[i\hat{p}_{1}\sigma^{x},i\hat{p}_{2}\sigma^{y}\right]
% \end{align}
% The two exponential terms are in decomposed via the BCH formula Eq.~\ref{eq:BCHFormula}
% for a lower-bound gate depth of eight.

% \subsection{Conditional Beam Squeezer Gate}

% The infinitesimal conditional beam squeezer
% \begin{equation}
% U_{\text{two-mode squeeze}}=e^{-i\lambda^{2}\left(\hat{a}_{1}^{\dagger}\hat{a}_{2}^{\dagger}+\hat{a}_{1}\hat{a}_{2}\right)\sigma^{z}}
% \end{equation}
% follows analogously from the conditional beam splitter. The argument
% in phase-space representation is
% \begin{align*}
% -i\lambda^{2}\left(\hat{a}_{1}^{\dagger}\hat{a}_{2}^{\dagger}+\hat{a}_{1}\hat{a}_{2}\right)\sigma^{z} & =-2i\lambda^{2}\left(\hat{x}_{1}\hat{x}_{2}-\hat{p}_{1}\hat{p}_{2}\right)\sigma^{z}
% \end{align*}
% such that the gate is the Trotter-Suzuki decomposed Eq.~\ref{eq:TrotterFirstOrder}
% product of exponential terms $\exp\left(\left[A_{1},B_{1}\right]\lambda^{2}\right)=\exp\left(-2i\lambda^{2}\hat{x}_{1}\hat{x}_{2}\sigma^{z}\right)$
% and $\exp\left(\left[A_{2},B_{2}\right]\lambda^{2}\right)=\exp\left(2i\lambda^{2}\hat{p}_{1}\hat{p}_{2}\sigma^{z}\right)$.
% The Pauli commutation relation Eq.~\ref{eq:PauliCommutatorz}, yields
% \begin{align}
% \left[A_{1},B_{1}\right] & =\left[i\hat{x}_{1}\sigma^{x},i\hat{x}_{2}\sigma^{y}\right]\\
% \left[A_{2},B_{2}\right] & =\left[\hat{p}_{1}\sigma^{x},\hat{p}_{2}\sigma^{y}\right]
% \end{align}
% such that the BCH formula Eq.~\ref{eq:BCHFormula} of the two exponential
% terms yields the infinitesimal conditional beam squeezer with a lower
% bound gate depth of eight.

% \subsection{Conditional Rotations\label{subsec:Conditional-Rotations}}

% We employ the displacement analytic ISA to synthesize the infinitesimal
% conditional rotation gate
% \begin{equation}
% U_{\text{rot},k}=e^{i\lambda^{2}\hat{n}\sigma^{k}}
% \end{equation}
% for $k=x,y,z$, which naturally belongs to the analytic displacement
% and rotation ISA. Expression of the exponential argument in terms
% of phase-space operators according to Eq.~\ref{eq:NumbertoPhaseSpace},
% Eq.~\ref{eq:CreationOperatortoPhaseSpace}, and Eq.~\ref{eq:AnnihilationOperatortoPhaseSpace}
% yields
% \begin{equation}
% i\lambda^{2}\hat{n}\sigma^{k}=i\left(\hat{x}^{2}+\hat{p}^{2}-\frac{1}{2}\right)\sigma^{k}\lambda^{2}
% \end{equation}
% such that the gate is expressed via the Trotter-Suzuki decomposition
% Eq.~\ref{eq:TrotterFirstOrder}as the product of $\exp\left(\left[A_{1},B_{1}\right]\right)=\exp\left(i\lambda^{2}\hat{x}^{2}\sigma^{k}\right)$,
% $\exp\left(\left[A_{2},B_{2}\right]\right)=\exp\left(i\lambda^{2}\hat{p}^{2}\sigma^{k}\right)$,
% and conditional displacement $\exp\left(-i\lambda^{2}\sigma^{k}/2\right)$.
% Given the Pauli commutator relation Eq.~\ref{eq:BCHFormula}, the
% first commutator is 
% \begin{align}
% \left[A_{1},B_{1}\right] & =i\hat{x}^{2}\sigma^{k}\\
%  & =i\hat{x}^{2}\left(-\frac{i}{2}\left[\sigma^{i},\sigma^{j}\right]\right)\\
%  & =\left[\frac{1}{\sqrt{2}}\hat{x}\sigma^{i},\frac{1}{\sqrt{2}}\hat{x}\sigma^{j}\right]
% \end{align}
% and the second commutator is 
% \begin{align}
% \left[A_{2},B_{2}\right] & =i\hat{p}^{2}\sigma^{k}\\
%  & =i\hat{p}^{2}\left(-\frac{i}{2}\left[\sigma^{i},\sigma^{j}\right]\right)\\
%  & =\left[\frac{1}{\sqrt{2}}\hat{p}\sigma^{i},\frac{1}{\sqrt{2}}\hat{p}\sigma^{j}\right]
% \end{align}
% such that both terms are amenable to BCH decomposition Eq.~\ref{eq:BCHFormula}
% and the infinitesimal conditional rotation is composed in the displacement
% ISA with gate depth lower-bound of  nine.

% \section{Universal Control of the Span $\left\{ \left|0\right\rangle ,\left|1\right\rangle \right\} $
% Fock Space\label{subsec:Universal-Control}}

% To demonstrate the efficacy of the analytic ISA, we demonstrate the
% use of the approach to encode a qubit in a cavity either via generation
% of effective Pauli gates in Section~\ref{subsec:Effective-Pauli-Gate}
% or imposition of an effective Hubbard interaction in the Jaynes-Cumming
% Hamiltonian in Section~\ref{subsec:Effective-Hubbard-Lattice}.

% \subsection{Effective Pauli Gate Approach\label{subsec:Effective-Pauli-Gate}}

% For universal control is the restricted $\text{span}\left\{ \left|0\right\rangle ,\left|1\right\rangle \right\} $
% Hilbert space, we generate three effective Pauli operators $\sigma_{\text{eff}}^{x}$,
% $\sigma_{\text{eff}}^{y}$, and $\sigma_{\text{eff}}^{z}$ that produce
% Pauli rotations in the lowest two modes of the cavity minimal leakage
% to higher energy states. The form of the effective Pauli operators
% is determined by expressing the standard Pauli operators
% \begin{align}
% \sigma^{x} & =\left(\begin{array}{cc}
% 0 & 1\\
% 1 & 0
% \end{array}\right)\\
% \sigma^{y} & =\left(\begin{array}{cc}
% 0 & -i\\
% i & 0
% \end{array}\right)\\
% \sigma^{z} & =\left(\begin{array}{cc}
% 1 & 0\\
% 0 & -1
% \end{array}\right)
% \end{align}
% in terms of creation and annihilation operators truncated to the first
% two Fock states
% \begin{align}
% \hat{a}_{\text{eff}}^{\dagger} & =\left(\begin{array}{cc}
% 0 & 0\\
% 1 & 0
% \end{array}\right)\\
% \hat{a}_{\text{eff}} & =\left(\begin{array}{cc}
% 0 & 1\\
% 0 & 0
% \end{array}\right)\\
% \hat{n}_{\text{eff}} & =\hat{a}^{\dagger}\hat{a}_{\text{eff}}=\left(\begin{array}{cc}
% 0 & 0\\
% 0 & 1
% \end{array}\right)
% \end{align}
% which yields
% \begin{align}
% \sigma_{\text{eff}}^{x} & =\hat{a}_{\text{eff}}^{\dagger}+\hat{a}_{\text{eff}}\\
% \sigma_{\text{eff}}^{y} & =i\left(\hat{a}_{\text{eff}}^{\dagger}-\hat{a}_{\text{eff}}\right)\\
% \sigma_{\text{eff}}^{z} & =I-2\hat{a}_{\text{eff}}^{\dagger}\hat{a}_{\text{eff}}
% \end{align}
% To reduce leakage into higher energy states, we ensure the creation
% operator $\hat{a}_{\text{eff}}^{\dagger}$ only acts on the ground
% state $\left|0\right>$ and the annihilation operator $\hat{a}_{\text{eff}}$
% only acts on the first excited state $\left|1\right>$ with the projector
% \begin{align}
% \hat{P}_{0} & \approx I-\hat{n}\\
%  & =\begin{cases}
% 0 & n=1\\
% 1 & n=0
% \end{cases}
% \end{align}
% where $n$ is the number of photons in the cavity and where only the
% $\text{span}\left\{ \left|0\right\rangle ,\left|1\right\rangle \right\} $
% states are populated. Since the operator is a projector, it obeys
% the relation
% \begin{equation}
% \hat{P}_{0}^{2}=\hat{P}_{0}
% \end{equation}
% such that the effective Pauli gates are
% \begin{align}
% \sigma_{\text{eff}}^{x} & =\hat{a}_{\text{eff}}^{\dagger}\hat{P}_{0}+\hat{P}_{0}\hat{a}_{\text{eff}}\\
%  & \approx\hat{a}^{\dagger}\left(I-\hat{n}\right)+\left(I-\hat{n}\right)\hat{a}\\
% \sigma_{\text{eff}}^{y} & =i\left(\hat{a}_{\text{eff}}^{\dagger}\hat{P}_{0}-\hat{P}_{0}\hat{a}_{\text{eff}}\right)\\
%  & \approx i\left(\hat{a}^{\dagger}\left(I-\hat{n}\right)-\left(I-\hat{n}\right)\hat{a}\right)\\
% \sigma_{\text{eff}}^{z} & =I-2\hat{a}_{\text{eff}}^{\dagger}\hat{P}_{0}^{2}\hat{a}_{\text{eff}}\\
%  & \approx I-2\hat{a}^{\dagger}\left(I-\hat{n}\right)\hat{a}
% \end{align}


% \subsubsection{Pauli X Gate}

% Consider the infinitesimal $\sigma_{x}$-rotation gate in the $\text{span}\left\{ \left|0\right\rangle ,\left|1\right\rangle \right\} $
% Fock space 
% \begin{align}
% U_{\text{span}\left\{ 0,1\right\} ,x} & =e^{i\lambda^{2}\sigma_{\text{eff}}^{x}\sigma^{z}}\\
%  & =e^{i\lambda^{2}\left(\hat{a}^{\dagger}\left(1-\hat{n}\right)+\left(1-\hat{n}\right)\hat{a}\right)\sigma^{z}}
% \end{align}
% Expression of the exponent in terms of phase-space operators Eq.~\ref{eq:NumbertoPhaseSpace},
% Eq.~\ref{eq:CreationOperatortoPhaseSpace}, and Eq.~\ref{eq:AnnihilationOperatortoPhaseSpace}
% gives 
% \begin{gather}
% i\lambda^{2}\left(\hat{a}^{\dagger}\left(I-\hat{n}\right)+\left(I-\hat{n}\right)\hat{a}\right)\sigma^{z}\nonumber \\
% =i\lambda^{2}\left(2\hat{x}-\left\{ \hat{x},\hat{n}\right\} +i\left[\hat{p},\hat{n}\right]\right)\sigma^{z}
% \end{gather}
% The gate is therefore given by a Trotter-Suzuki decomposition Eq.~\ref{eq:TrotterFirstOrder}
% of three terms: $\exp\left(\left[A_{1},B_{1}\right]\lambda^{2}\right)=\exp\left(-i\lambda^{2}\left\{ \hat{x},\hat{n}\right\} \sigma^{z}\right)$,
% $\exp\left(\left[A_{2},B_{2}\right]\lambda^{2}\right)=\exp\left(-\lambda^{2}\left[\hat{p},\hat{n}\right]\sigma^{z}\right)$,
% and $\exp\left(2i\lambda^{2}\hat{x}\sigma^{z}\right)$. 

% The first and second exponential terms are decomposed according to
% the BCH decomposition Eq.~(\ref{eq:BCHFormula}). The first commutator
% is given by the Pauli anticommutation-commutation relation Eq.~(\ref{eq:AnticommutatortoCommutator})
% \begin{align}
% -i\left\{ \hat{x},\hat{n}\right\} \sigma^{z} & =-i\left(i\left[i\hat{x}\sigma^{x},i\hat{n}\sigma^{y}\right]\right)\\
%  & =\left[i\hat{x}\sigma^{x},i\hat{n}\sigma^{y}\right]\\
%  & =\left[A_{1},B_{1}\right]
% \end{align}
% $A_{1}$ corresponds to a position displacement and $B_{1}$
% corresponds to the $y$-conditional rotation gate. The argument of
% the second term is already in the form of a commutator, such that
% \begin{align}
% \left[A_{2},B_{2}\right] & =-\left[\hat{p},\hat{n}\right]\sigma^{z}\\
%  & =\left[i\hat{p},i\hat{n}\sigma^{z}\right]
% \end{align}
% $A_{2}$ corresponds to an \emph{unconditional} momentum boost,
% and $B_{2}$ corresponds to the $z$-conditional rotation gate.
% Lastly, the third term already belongs to the instruction set architecture
% and needs no further decomposition. 

% The infinitesimal $\sigma_{x}$-rotation gate in the $\text{span}\left\{ \left|0\right\rangle ,\left|1\right\rangle \right\} $
% Fock space is therefore composed of a product of nine gates in the
% displacement and rotation gate ISA and 21 gates in the displacement
% ISA.

% \subsubsection{Pauli Y Gate}

% The infinitesimal $\sigma_{y}$-rotation gate in the $\text{span}\left\{ \left|0\right\rangle ,\left|1\right\rangle \right\} $
% Fock space is determined analogously 
% \begin{align}
% U_{\text{span}\left\{ 0,1\right\} ,y} & =e^{i\lambda^{2}\sigma_{\text{eff}}^{y}\sigma^{z}}\\
%  & =e^{-\lambda^{2}\left(\hat{a}^{\dagger}\left(I-\hat{n}\right)+\left(I-\hat{n}\right)\hat{a}\right)\sigma^{z}}
% \end{align}
% Expression of the argument of the exponent in terms of phase-space
% variables Eq.~\ref{eq:CreationOperatortoPhaseSpace} and Eq.~\ref{eq:AnnihilationOperatortoPhaseSpace}
% yields 
% \begin{gather}
% -\lambda^{2}\left(\hat{a}^{\dagger}\left(I-\hat{n}\right)-\left(I-\hat{n}\right)\hat{a}\right)\sigma^{z}\nonumber \\
% =-\lambda^{2}\left(-2i\hat{p}+\left[\hat{n},\hat{x}\right]+i\left\{ \hat{n},\hat{p}\right\} \right)\sigma^{z}
% \end{gather}
% such that the gate is a Trotter-Suzuki decomposition Eq.~\ref{eq:TrotterFirstOrder}
% of $\exp\left(\left[A_{1},B_{1}\right]\lambda^{2}\right)=\exp\left(-\lambda^{2}\left[n,x\right]\sigma^{z}\right)$,
% $\exp\left(\left[A_{2},B_{2}\lambda^{2}\right]\right)=\exp\left(-i\lambda^{2}\left\{ p,n\right\} \sigma^{z}\right)$,
% and $\exp\left(2i\lambda^{2}p\sigma^{z}\right)$. 

% Again, the first two exponential terms are decomposed via the BCH
% formula Eq.~\ref{eq:BCHFormula}. The first commutator is 
% \begin{align}
% \left[A_{1},B_{1}\right] & =\left[\hat{n},\hat{x}\right]\sigma^{z}\\
%  & =\left[\hat{n}\sigma^{z},\hat{x}\right]
% \end{align}
% where the exponent of $A_{1}$ is a $z$-conditional rotation
% gate and the exponent of $B_{1}$ is an unconditional position
% displacement. The second commutator is given by the Pauli anticommutation-commutation
% relation Eq.~(\ref{eq:AnticommutatortoCommutator}) 
% \begin{align}
% i\left\{ p,n\right\} \sigma^{z} & =i\left(i\left[ip\sigma^{x},in\sigma^{y}\right]\right)\\
%  & =\left[ip\sigma^{x},in\sigma^{y}\right]\\
%  & =\left[A_{2},B_{2}\right]
% \end{align}
% where the exponent of $A_{2}$ corresponds to a conditional
% momentum shift and the exponent of $B_{2}$ is a $y$-conditional
% rotation gate.

% The infinitesimal $\sigma_{y}$-rotation gate in the $\text{span}\left\{ \left|0\right\rangle ,\left|1\right\rangle \right\} $
% Fock space therefore has a lower-bound gate depth of nine in the displacements-only
% analytic ISA and 21 in the displacement and rotation analytic ISA.

% \subsubsection{Pauli Z Gate}

% The infinitesimal $\sigma_{z}$-rotation gate in the $\text{span}\left\{ \left|0\right\rangle ,\left|1\right\rangle \right\} $
% Fock space is 
% \begin{align}
% U_{\text{span}\left\{ 0,1\right\} ,z} & =e^{i\lambda^{2}\sigma_{\text{eff}}^{z}\sigma^{z}}\\
%  & =e^{-\lambda^{2}\left(I-2\hat{a}^{\dagger}\left(I-\hat{n}\right)\hat{a}\right)\sigma^{z}}
% \end{align}
% whose argument in terms of ladder operators is
% \begin{gather}
% -\lambda^{2}\left(I-2\hat{a}^{\dagger}\left(I-\hat{n}\right)\hat{a}\right)\sigma^{z}\nonumber \\
% =-\lambda^{2}\left(I-2\hat{a}^{\dagger}a+2\hat{a}^{\dagger}\hat{a}^{\dagger}\hat{a}\hat{a}\right)\sigma^{z}
% \end{gather}
% Given the ladder operator commutator Eq.~\ref{eq:CommutatorCreationAnnihilation},
% \begin{equation}
% \hat{a}^{\dagger}\hat{a}=\hat{a}\hat{a}^{\dagger}-I
% \end{equation}
% the relationship between the fourth-order ladder operator term and
% the number operator is
% \begin{align}
% \hat{a}^{\dagger}\hat{a}^{\dagger}\hat{a}\hat{a} & =\hat{a}^{\dagger}\left(\hat{a}\hat{a}^{\dagger}-I\right)\hat{a}\\
%  & =\hat{a}^{\dagger}\hat{a}\hat{a}^{\dagger}\hat{a}-\hat{a}^{\dagger}\hat{a}\\
%  & =\hat{n}^{2}-\hat{n}
% \end{align}
% The argument of the exponential in terms of number operators is then
% \begin{multline}
% -\lambda^{2}\left(I-2\hat{a}^{\dagger}\left(I-\hat{n}\right)\hat{a}\right)\sigma^{z}\\
% =-\lambda^{2}\left(I-4\hat{n}+2\hat{n}^{2}\right)\sigma^{z}
% \end{multline}
% The argument is further simplified given that the state is restricted
% to the first two cavity modes, as for $n=0$ and $n=1$ the quantity
% $\hat{n}^{2}-\hat{n}$ is zero, as follows:
% \begin{gather}
% -\lambda^{2}\left(I-4\hat{n}+2\hat{n}^{2}\right)\sigma^{z}\nonumber \\
% =-\lambda^{2}\left(I-2n\right)\sigma^{z}
% \end{gather}
% The gate is therefore directly synthesized as the product of the qubit
% rotation gate $\exp\left(-\lambda^{2}\sigma^{z}\right)$ and the $z$-conditional
% rotation gate $\exp\left(2\lambda^{2}\hat{n}\sigma^{z}\right)$ for
% a lower-bound gate depth of two.

% \subsection{Effective Hubbard Lattice Interaction Approach\label{subsec:Effective-Hubbard-Lattice}}

% An alternative scheme to encode a qubit in a cavity in the analytic
% ISA scheme is to map the three-dimensional quantum electrodynamics
% (3D cQED) system to a qubit by imposing an $\hat{n}\left(\hat{n}-1\right)$
% anharmonicity into the Jaynes-Cummings Hamiltonian that describes
% the system. The anharmonicity term increases the energy gap between
% higher levels of the oscillator to effectively restrict propagation
% to the $\text{span}\left\{ \left|0\right\rangle ,\left|1\right\rangle \right\} $
% Fock space for universal control.

% Consider the standard Jaynes-Cummings Hamiltonian 
% \begin{equation}
% \hat{H}_{\text{JC}}=\omega_{R}\hat{a}^{\dagger}\hat{a}+\frac{\omega_{Q}}{2}\sigma^{z}+g\left(\hat{a}\sigma^{+}+\hat{a}^{\dagger}\sigma^{-}\right)
% \end{equation}
% where $\omega_{R}$ is the cavity frequency, $\omega_{Q}$ is the
% qubit frequency, and $g$ is the coupling parameter. Inclusion of
% the simulated $\hat{n}\left(\hat{n}-1\right)$ anharmonicity of strength
% $\Gamma$ yields 
% \begin{equation}
% \hat{H}_{\text{an}}=\omega_{R}\hat{a}^{\dagger}\hat{a}+\Gamma\hat{n}(\hat{n}-1)+\frac{\omega_{Q}}{2}\sigma^{z}+g(\hat{a}\sigma^{+}+\hat{a}^{\dagger}\sigma^{-})
% \end{equation}
% and system is switched between states $\left|0\right\rangle $ and
% $\left|1\right\rangle $ with a weak time-$t$-dependent drive of
% strength $\Omega$ at the resonance frequency $\omega_{R}$, as follows:
% \begin{equation}
% \hat{H}_{\text{drive}}\left(t\right)=\Omega e^{i\omega_{R}t}\hat{a}^{\dagger}+\Omega^{\star}e^{-i\omega_{R}t}\hat{a}
% \end{equation}
% Synthesis of a propagator of the form $\exp\left(i\lambda^{2}\hat{n}\left(\hat{n}-1\right)\right)$
% is then sufficient to employ the native 3D cQED system as a qubit.
% Note the choice of $\lambda$ for practical implementation must take
% into account both the time step and the fact the BCH decomposition
% yields a square root in the exponential argument. The required propagator
% is a Trotter-Suzuki decomposition Eq.~(\ref{eq:TrotterFirstOrder})
% of $\exp\left(\left[A,B\right]\lambda^{2}\right)=\exp(i\lambda^{2}\hat{n}^{2}\sigma^{z})$
% and $\exp(-i\lambda^{2}\hat{n}\sigma^{z})$. 

% The first term is synthesized according to the BCH formula Eq.~(\ref{eq:BCHFormula})
% with a commutator determined by the Pauli commutation relation Eq.~(\ref{eq:PauliCommutatorz})
% as follows:
% \begin{align}
% \left[A,B\right] & =i\hat{n}^{2}\sigma^{z}\\
%  & =i\hat{n}^{2}\left(-\frac{i}{2}\left[\sigma^{x},\sigma^{y}\right]\right)\\
%  & =\left[\frac{1}{\sqrt{2}}\hat{n}\sigma^{x},\frac{1}{\sqrt{2}}\hat{n}\sigma^{y}\right]
% \end{align}
% where $A$ and $B$ correspond to $x$-conditional and
% $y$-conditional rotations, respectively. The second term is a $z$-conditional
% rotation gate.

% The resulting anharmonicity gate therefore has a gate depth of lower
% bound five in the displacement and rotation analytic ISA and 45 in
% the displacement ISA.

% \section{Fermi-Hubbard Lattice Dynamics\label{sec:Fermi-Hubbard-Lattice-Dynamics}}

% To further demonstrate the power of the analytic ISA, we employ the
% approach to simulate fermionic dynamics on bosonic 3D cQED systems.

% We consider the Fermi-Hubbard lattice Hamiltonian

% \begin{align}
% \hat{H}_{\text{FH}} & =\hat{T}_{\text{FH}}+\hat{V}_{\text{FH}}\label{eq:FermiHubbardHamiltonian}\\
% \hat{T}_{\text{FH}} & =-J\sum_{i,\sigma}\hat{c}_{i,\sigma}^{\dagger}\hat{c}_{i+1,\sigma}+\hat{c}_{i+1,\sigma}^{\dagger}\hat{c}_{i,\sigma}\\
% \hat{V}_{\text{FH}} & =U\sum_{i}\hat{n}_{i,\uparrow}\hat{n}_{i,\downarrow}
% \end{align}
% The kinetic energy term $\hat{T}_{\text{FH}}$ describes the nearest-neighbor
% interaction for hopping of a single spin between two sites with hopping
% parameter $J$ and spin $\sigma$ given annihilation operators $\left\{ \hat{c}_{j,\sigma}\right\} $
% and creation operators $\left\{ \hat{c}_{j,\sigma}^{\dagger}\right\} $
% for sites $\left\{ j\right\} $. The potential energy term $\hat{V}_{\text{FH}}$
% describes the same-site interaction, which gives the energetic unfavorability
% of a spin up $\uparrow$ and spin down $\downarrow$ coexisting on
% the same site $i$, where $\hat{n}_{j,\sigma}$ gives the number of
% spin $\sigma$ particles on site $j$. According to fermion statistics,
% no more than a single particle of a given spin can exist on a single
% site.

% Each cavity of the 3D cQED system represents either a spin up or spin
% down particle on a single lattice site, for direct comparison to the
% qubit-based schemes of refs.~\cite{Kivlichan.2018.110501,arute2020observation,Cade.2020.235122}.
% Each cavity is connected to the cavity that represents the same site
% of opposite spin to facilitate computation of the potential energy
% $\hat{V}_{\text{FH}}$ and cavities of the same spin on neighboring
% sites to facilitate computation of the kinetic energy $\hat{T}_{\text{FH}}$.
% Cavities are also connected along Jordan-Wigner strings to take into
% account fermionic statistics. 

% The $\text{\ensuremath{\left|0\right\rangle }}$ cavity state represents
% absence of a spin and the $\text{\ensuremath{\left|1\right\rangle }}$
% state represents presence of a spin. Within each cavity, only the
% states in $\text{span}\left\{ \text{\ensuremath{\left|0\right\rangle }},\text{\ensuremath{\left|1\right\rangle }}\right\} $
% are considered as in Section~(\ref{subsec:Universal-Control}), which
% prevents leakage into unphysical high-energy cavity states. At the
% end of each operation, the cavity state must be in either the $\text{\ensuremath{\left|0\right\rangle }}$
% or $\text{\ensuremath{\left|1\right\rangle }}$ state and the transmon
% state must also be in the ground state $\text{\ensuremath{\left|g\right\rangle }}$,
% which provides an error syndrome and therefore a degree of error correction
% not employed in qubit-based representations of the Fermi-Hubbard lattice.

% Propagation of any combination of up spins and down spins is simulated
% with three two-cavity gates. The first two gates -- the same-site
% and hopping gates -- are defined as the propagator of the same-site
% and hopping Hamiltonians, respectively. The same-site term of the
% Hamiltonian for site $i$ is 
% \begin{equation}
% \hat{H}_{\text{same}}=U\hat{n}_{i,\uparrow}\hat{n}_{i,\downarrow}
% \end{equation}
% This term is zero if only one spin is on a site and $U$ if both spins
% are on the same site, which gives the diagonal Hamiltonian in the
% reduced $4\times4$ Hilbert space
% \begin{equation}
% \hat{H}_{\text{same}}=\left[\begin{array}{cccc}
% 0 & 0 & 0 & 0\\
% 0 & 0 & 0 & 0\\
% 0 & 0 & 0 & 0\\
% 0 & 0 & 0 & U
% \end{array}\right]
% \end{equation}
% and the diagonal propagator $U_{\text{same}}=\text{e}^{-\text{i}\hat{H}_{\text{same}}\tau}$
% \begin{equation}
% U_{\text{same}}=\left[\begin{array}{cccc}
% 1 & 0 & 0 & 0\\
% 0 & 1 & 0 & 0\\
% 0 & 0 & 1 & 0\\
% 0 & 0 & 0 & \text{e}^{-\text{i}U\tau}
% \end{array}\right]
% \end{equation}
% This gate is recognized as the conditional cross-Kerr interaction
% of 3D cQED systems and equivalently a controlled-phase (CPHASE) gate
% in the reduced subspace $\text{span}\left\{ \left|0\right\rangle ,\left|1\right\rangle \right\} $.
% The hopping term of the Hamiltonian for each $\sigma$ spin in sites
% $i,\left(i+1\right)$ is 
% \begin{align}
% H_{\text{hop}} & =-J\left(\hat{c}_{i,\sigma}^{\dagger}\hat{c}_{i+1,\sigma}+\hat{c}_{i+1,\sigma}^{\dagger}\hat{c}_{i,\sigma}\right)\\
%  & =-J\left(\hat{c}_{i,\sigma}^{\dagger}\hat{c}_{i+1,\sigma}-\hat{c}_{i,\sigma}\hat{c}_{i+1,\sigma}^{\dagger}\right)
% \end{align}
% where the latter expression employs the commutator relationship of
% the annihilation and creation operators. The hopping Hamiltonian for
% the specified mapping is then the off-diagonal matrix 
% \begin{equation}
% H_{\text{hop}}=\left[\begin{array}{cccc}
% 0 & 0 & 0 & 0\\
% 0 & 0 & -t & 0\\
% 0 & -t & 0 & 0\\
% 0 & 0 & 0 & 0
% \end{array}\right]
% \end{equation}
% which gives the hopping propagator $U_{\text{hop}}=\text{e}^{-\text{i}H_{\text{hop}}\tau}$
% \begin{align}
% U_{\text{hop}} & =\left[\begin{array}{cccc}
% 1 & 0 & 0 & 0\\
% 0 & \cos\left(t\tau\right) & i\sin\left(t\tau\right) & 0\\
% 0 & i\sin\left(t\tau\right) & \cos\left(t\tau\right) & 0\\
% 0 & 0 & 0 & 1
% \end{array}\right]
% \end{align}
% which is recognized as a conditional controlled-phase beam splitter
% restricted to $\text{span}\left\{ \left|0\right\rangle ,\left|1\right\rangle \right\} $
% in bosonic systems and a Givens or iSWAP-like gate in the reduced
% $\text{span}\left\{ \left|0\right\rangle ,\left|1\right\rangle \right\} $
% subspace \cite{Cade.2020.235122,arute2020observation}. The final
% gate of the three-gate set incorporates the fermionic statistics of
% the spins via the fermionic SWAP (FSWAP) gate FSWAP gate \cite{Kivlichan.2018.110501,Cade.2020.235122}.
% The content of each cavity is swapped with one of its neighbors with
% inclusion of a phase where both spins are present in neighboring cavities
% as follows
% \begin{equation}
% U_{\text{FSWAP}}=\left[\begin{array}{cccc}
% 1 & 0 & 0 & 0\\
% 0 & 0 & 1 & 0\\
% 0 & 1 & 0 & 0\\
% 0 & 0 & 0 & -1
% \end{array}\right]
% \end{equation}
% which is recognized as the product of a conditional rotation gate
% and a beam-splitter on 3D cQED systems.

% Finally, initial states are prepared by the universal set of gates
% in $\text{span}\left\{ \left|0\right\rangle ,\left|1\right\rangle \right\} $
% detailed in Section~\ref{subsec:Universal-Control}.

% \subsection{Conditional Cross-Kerr (CPHASE) Gate}

% We consider the infinitesimal conditional cross-Kerr gate
% \begin{equation}
% U_{\text{cross-Kerr}}=e^{i\lambda^{2}\hat{n}_{1}\hat{n}_{2}\sigma_{z}}
% \end{equation}
% which is also employed in GKP codes encoded in 3D cQED systems \cite{royer2022encoding}. 

% The argument is expressed in terms of a commutator according to the
% Pauli commutation relation Eq.~\ref{eq:PauliCommutatorz}, as follows:
% \begin{align}
% \left[A,B\right] & \lambda^{2}=i\lambda^{2}\hat{n}_{1}\hat{n}_{2}\sigma_{z}\\
%  & =i\lambda^{2}\hat{n}_{1}\hat{n}_{2}\left(-\frac{i}{2}\left[\sigma^{x},\sigma^{y}\right]\right)\\
%  & =\left[\frac{1}{\sqrt{2}}\hat{n}_{1}\sigma^{x},\frac{1}{\sqrt{2}}\hat{n}_{2}\sigma^{y}\right]\lambda^{2}
% \end{align}
% where $A$ corresponds to an $x$-conditional rotation gate
% and $B$ corresponds to a $y$-conditional rotation gate.

% The resulting gate features a lower-bound gate depth of four in the
% displacement and rotation analytic ISA and 16 in the displacement
% ISA.

% \subsection{$\text{Span}\left\{ \left|0\right\rangle ,\left|1\right\rangle \right\} $
% Conditional Beam Splitter Gate}

% In order to generate a $\text{span}\left\{ \left|0\right\rangle ,\left|1\right\rangle \right\} $
% that operates only when $\hat{n}_{1}\hat{n}_{2}\ne1$ (\emph{i.e.},
% $1-\hat{n}_{1}\hat{n}_{2}=0$), we formulate the infinitesimal conditional
% (controlled-phase) beam-splitter gate
% \begin{equation}
% U_{\text{cond. beam}}=e^{-i\lambda^{2}\left(\hat{a}_{1}^{\dagger}\hat{a}_{2}+\hat{a}_{1}\hat{a}_{2}^{\dagger}\right)\left(1-\hat{n}_{1}\hat{n}_{2}\right)\sigma^{z}}
% \end{equation}
% which is decomposed via the Trotter-Suzuki decomposition Eq.~\ref{eq:TrotterFirstOrder}
% in terms of $\exp\left(-i\lambda^{2}\left(\hat{a}_{1}^{\dagger}\hat{a}_{2}+\hat{a}_{1}\hat{a}_{2}^{\dagger}\right)\sigma^{z}\right)$
% and $\exp\left(i\lambda^{2}\left(\hat{a}_{1}^{\dagger}\hat{a}_{2}+\hat{a}_{1}\hat{a}_{2}^{\dagger}\right)\left(\hat{n}_{1}\hat{n}_{2}\right)\sigma^{z}\right)$.
% The first term is the conditional beam splitter $U_{\text{beam split.}}$
% Eq.~\ref{subsec:Conditional-(Controlled-Phase)-Beam-Splitter} and
% the second term is decomposed via BCH Eq.~\ref{eq:BCHFormula} as
% follows:

% Given the expression of the number operator in terms of the phase-space
% operators Eq.~\ref{eq:NumbertoPhaseSpace}, the argument of the second
% exponential operator is 
% \begin{gather}
% i\lambda^{2}\left(\hat{a}_{1}^{\dagger}\hat{a}_{2}+\hat{a}_{1}\hat{a}_{2}^{\dagger}\right)\hat{n}_{1}\hat{n}_{2}\sigma^{z}\nonumber \\
% =i\lambda^{2}\left(2\left(\hat{x}_{1}\hat{x}_{2}+\hat{p}_{1}\hat{p}_{2}\right)\right)\hat{n}_{1}\hat{n}_{2}\sigma^{z}
% \end{gather}
% The term is then expressed as a Trotter decomposition of $\exp\left(\left[A_{1},B_{1}\right]\lambda^{2}\right)=\exp\left(2i\lambda^{2}\hat{x}_{1}\hat{x}_{2}\hat{n}_{1}\hat{n}_{2}\sigma^{z}\right)$
% and $\exp\left(\left[A_{2},B_{2}\right]\lambda^{2}\right)=\exp\left(2i\lambda^{2}\hat{p}_{1}\hat{p}_{2}\hat{n}_{1}\hat{n}_{2}\sigma^{z}\right)$. 

% The first commutator is given by the Pauli commutation relation Eq.~\ref{eq:PauliCommutatorz}
% \begin{align}
% \left[A_{1},B_{1}\right] & =2i\hat{x}_{1}\hat{x}_{2}\hat{n}_{1}\hat{n}_{2}\sigma^{z}\\
%  & =2i\hat{x}_{1}\hat{x}_{2}\hat{n}_{1}\hat{n}_{2}\left(-\frac{i}{2}\left[\sigma^{x},\sigma^{y}\right]\right)\\
%  & =\left[\hat{x}_{1}\hat{n}_{1}\sigma^{x},\hat{x}_{2}\hat{n}_{2}\sigma^{y}\right]
% \end{align}

% The $A_{1}$ term is determined by a Trotter decomposition according
% to the product of operator formula Eq.~\ref{eq:ProductOperators}
% \begin{align}
% A_{1} & =\frac{1}{2}\left\{ \hat{x}_{1},\hat{n}_{1}\right\} \sigma^{x}+\frac{1}{2}\left[\hat{x}_{1},\hat{n}_{1}\right]\sigma^{x}\\
%  & =A_{1a}+A_{1b}
% \end{align}
% where according to the anticommutator to commutator relation $A_{1a}$
% is given by the BCH formula with 
% \begin{align}
% \left[A_{1a^{\prime}},B_{1a^{\prime}}\right] & =\frac{1}{2}\left\{ \hat{x}_{1},\hat{n}_{1}\right\} \sigma^{x}\\
%  & =\frac{i}{2}\left[i\hat{x}_{1}\sigma^{y},i\hat{n}_{1}\sigma^{z}\right]\\
%  & =\left[-\frac{1}{\sqrt{2}}\hat{x}_{1}\sigma^{y},-\frac{1}{\sqrt{2}}\hat{n}_{1}\sigma^{z}\right]
% \end{align}
% where $B_{1a^{\prime}}$ is a $z$-conditional rotation gate.
% Distribution of terms yields $A_{1b}$ as 
% \begin{equation}
% \left[A_{1b^{\prime}},B_{1b^{\prime}}\right]=\left[\frac{1}{\sqrt{2}}\hat{x}_{1},\frac{1}{\sqrt{2}}\hat{n}_{1}\sigma^{x}\right]
% \end{equation}
% where $B_{1b^{\prime}}$ is an $x$-conditional rotation gate. 

% According to the same procedure,
% \begin{align}
% B_{1} & =\frac{1}{2}\left\{ \hat{x}_{2},\hat{n}_{2}\right\} \sigma^{y}+\frac{1}{2}\left[\hat{x}_{2},\hat{n}_{2}\right]\sigma^{y}\\
%  & =B_{1a}+B_{1b}
% \end{align}
% where $B_{1a}$ is given by
% \begin{align}
% \left[A_{1a^{\prime\prime}},B_{1a^{\prime\prime}}\right] & =\frac{1}{2}\left\{ \hat{x}_{2},\hat{n}_{2}\right\} \sigma^{y}\\
%  & =\frac{i}{2}\left[i\hat{x}_{2}\sigma^{z},i\hat{n}_{1}\sigma^{x}\right]\\
%  & =\left[-\frac{1}{\sqrt{2}}\hat{x}_{2}\sigma^{z},-\frac{1}{\sqrt{2}}\hat{n}_{2}\sigma^{x}\right]
% \end{align}
% with $B_{1a^{\prime\prime}}$ an $x$-conditional rotation, and $B_{1b}$
% is given by 
% \begin{equation}
% \left[A_{1b^{\prime\prime}},B_{1b^{\prime\prime}}\right]=\left[\frac{1}{\sqrt{2}}\hat{x}_{2},\frac{1}{\sqrt{2}}\hat{n}_{2}\sigma^{y}\right]
% \end{equation}
% where $B_{1b^{\prime\prime}}$ is a $y$-conditional rotation gate.

% The second term follows analogously with the position $x$ replaced
% by the momentum $p$.

% \subsection{Conditional FSWAP Gate.}

% In analytic ISA, the FSWAP gate follows immediately from the conditional
% cross-Kerr gate detailed above and a complete beam-splitter gate (or
% conditional beam splitter gate detailed above) as 
% \begin{equation}
% U_{\text{FSWAP}}=U_{\text{cond. Kerr}}U_{\text{cond. beam}}
% \end{equation}
}



\section{Discussion}
%% what are the pts we should think about for the audience. 
%% may be easier to do this from the beginning

Our main contribution in this paper is a systematic approach to synthesizing unitary dynamics on a hybrid quantum computer that has access to both qubit as well as bosonic operations.  Such gatesets naturally model systems such as cavity quantum electrodynamics systems as well as ion trap-based quantum computers.  Our main innovation here is the development of high-order analytic formulas that can be used to place bounds on the complexity of implementing arbitrary unitary operations on such a hybrid device.  Specifically, we see that these methods are capable of achieving subpolynomial scaling with the inverse error tolerance ($1/\epsilon$) and allows us to implement arbitrary nonlinearities in the field operators in the generator of the unitary that we wish to implement at low cost asymptotically.  In particular, we focus on using a construct known as a block-encoded creation operator as our fundamental construct and show numerically highly accurate approximations to the exponential of a block-encoding of the square of the creation operator.  Further, we study the Hong-Ou-Mandel effect and observe that the synthesized operations used in our construction can have negligible error with respect to our target precision.

While this work constitutes a significant step forward in our understanding of how to control and manipulate such quantum systems, there remain many open questions.  The first issue involves the large constant-factor overheads observed in practical implementations of our synthesized operations within this gate set.  Specifically, we note that for both the Hong-Ou-Mandel effect and the block-encoding of $(a^\dagger)^2$ that thousands of gate operations are needed to achieve infidelities of $10^{-3}$ or smaller.  This makes such sequences impractical for near-term applications where the gate infidelities are on the order of $1\%$.  Several avenues of approach exist that could be used to improve upon these results.  The first approach would be to use ideas related to quantum signal processing to implement functions of creation operators; this could potentially improve the scaling with respect to the error tolerance from the subpolynomial scaling currently demonstrated to polylogarithmic scaling.  The second approach would be to use sequences designed here as seeds for gradient-descent optimization procedures for control such as GRAPE~\cite{khaneja2005optimal} at the pulse level or numerical optimization of parameterized gates at the SNAP \cite{fosel2020efficient} or Controlled Displacement \cite{eickbusch2021fast} instruction level.  These locally optimized sequences may then prove to be either better, or more understandable, than existing gradient-optimized pulse sequences for control of such systems.

%Another important question is to ask from a bigger picture perspective what role Trotter-Suzuki derived synthesis methods like this one may play in controlling quantum systems.

\section*{Acknowledgements}
This project was primarily supported by the U.S. Department of
Energy, Office of Science, National Quantum Information
Science Research Centers, Co-design Center for Quantum
Advantage under contract number DE-SC0012704, (Basic Energy Sciences, PNNL FWP 76274).
Eleanor Crane was supported by UCL Faculty of Engineering Sciences and the Yale-UCL exchange scholarship from
RIGE (Research, Innovation and Global Engagement), and by
the Princeton MURI award SUB0000082, the DoE QSA, NSF
QLCI (award No.OMA-2120757), DoE ASCR Accelerated
Research in Quantum Computing program (award No.DESC0020312), NSF PFCQC program. Micheline B.~Soley was supported by the Yale Quantum Institute Postdoctoral Fellowship.

%\newpage
\appendix
% \begin{appendices}

% \section{Homeless}

% \begin{equation}
% \left[\hat{x},\hat{p}\right]=\frac{i}{2}\label{eq:Commutatorxp}
% \end{equation}
% according to the relations
% \begin{align}
% \hat{a} & =\hat{x}+i\hat{p}\label{eq:AnnihilationOperatortoPhaseSpace}\\
% \hat{a}^{\dagger} & =\hat{x}-i\hat{p}\label{eq:CreationOperatortoPhaseSpace}
% \end{align}
% where 
% \begin{equation}
% \left[\hat{a},\hat{a}^{\dagger}\right]=1\label{eq:CommutatorCreationAnnihilation}
% \end{equation}
% The number operator is further
% expressed as
% \begin{align}
% n & =\hat{a}^{\dagger}\hat{a}\\
%  & =\hat{x}^{2}+\hat{p}^{2}-\frac{1}{2}\label{eq:NumbertoPhaseSpace}
% \end{align}

% Where $\hat{H}$ is given by the sum of operators $\hat{H}=\sum_{i=1}^{N}\hat{H}_{i}$,
% the exponential propagator is synthesized according to the Trotter-Suzuki
% decomposition, where the first-order Trotter-Suzuki decomposition
% of $U$ is \cite{Suzuki.1992.387,childs2021theory} 
% \begin{equation}
% \mathcal{\mathcal{T}}_{1}=e^{-i\hat{H}_{N}\lambda}e^{-i\hat{H}_{N-1}\lambda}\cdots e^{-i\hat{H}_{1}\lambda}+\mathcal{O}\left(\lambda^{2}\right)\label{eq:TrotterFirstOrder}
% \end{equation}
% for small parameter $\lambda$, the symmetrized second-order Trotter-Suzuki
% decomposition is 
% \begin{equation}
% \mathcal{T}_{2}=e^{-i\hat{H}_{1}\lambda/2}e^{-i\hat{H}_{2}\lambda/2}\cdots e^{-i\hat{H}_{N}\lambda/2}e^{-i\hat{H}_{N}\lambda/2}\cdots e^{-i\hat{H}_{2}\lambda/2}e^{-i\hat{H}_{1}\lambda/2}+\mathcal{O}\left(\lambda^{3}\right)\label{eq:TrotterSecondOrder}
% \end{equation}
% and higher-order Trotter-Suzuki decompositions are generated as 
% \begin{equation}
% \mathcal{T}_{2p}=\mathcal{T}_{2p-2}\left(u_{p}\lambda\right)^{2}\mathcal{T}_{2p-2}\left(\left(1-4u_{p}\right)\lambda\right)\mathcal{T}_{2p-2}\left(u_{p}\lambda\right)^{2}
% \end{equation}
% for 
% \begin{equation}
% u_{p}=\frac{1}{4-4^{1/\left(2p-1\right)}}
% \end{equation}

% Where $\hat{H}$ is given by the operator product $\hat{H}=\prod_{i=1}^{M}\hat{H}_{i}$,
% the argument of the exponential is expressed as the commutator of
% two operators $\left[\hat{A},\hat{B}\right]$ and the exponential is decomposed
% according to the BCH decomposition \cite{childs2013product}, where
% the lowest-order BCH decomposition of $U$ for small parameter $\lambda$
% is (see Appendix~\ref{sec:expansion})
% \begin{equation}
% V_{1}\left(\hat{A}\lambda,\hat{B}\lambda\right)=e^{\left[\hat{A},\hat{B}\right]\lambda^{2}}\approx e^{\hat{A}\lambda}e^{\hat{B}\lambda}e^{-\hat{A}\lambda}e^{-\hat{B}\lambda}+\mathcal{O}\left(\lambda^{3}\right)\label{eq:BCHFormula}
% \end{equation}
% and higher-order BCH decompositions are determined recursively as
% \begin{align}
% V_{p+1}\left(\hat{A}\lambda,B\lambda\right) & =V_{p}\left(\hat{A}\gamma_{p}\lambda,\hat{B}\gamma_{p}\lambda\right)V_{p}\left(-\hat{A}\gamma_{p}\lambda,-\hat{B}\gamma_{p}\lambda\right)\nonumber \\
%  & \times V_{p}\left(\hat{A}\beta_{p}\lambda,\hat{B}\beta_{p}\lambda\right)^{-1}V_{p}\left(-\hat{A}\beta_{p}\lambda,-\hat{B}\beta_{p}\lambda\right)^{-1}\nonumber \\
%  & \times V_{p}\left(\hat{A}\gamma_{p}\lambda,\hat{B}\gamma_{p}\lambda\right)V_{p}\left(-\hat{A}\gamma_{p}\lambda,-\hat{B}\gamma_{p}\lambda\right)\nonumber \\
%  & +O\left(\lambda^{2p+3}\right)\label{eq:RecursiveBCH}
% \end{align}
% where 
% \begin{align}
% V_{p}\left(\hat{C},\hat{D}\right) & =\begin{cases}
% V_{p}\left(\hat{D},\hat{C}\right)^{-1} & p\text{ is odd}\\
% V_{p}\left(-\hat{D},-\hat{C}\right)^{-1} & p\text{ is even}
% \end{cases}\\
% r_{p} & =\frac{2^{\frac{1}{p+1}}}{4\left(2-2^{\frac{1}{p+1}}\right)}\\
% \beta_{p} & =\left(2r_{p}\right)^{1/2}\\
% \gamma_{p} & =\left(\frac{1}{4}+r_{p}\right)^{1/2}
% \end{align}
% Higher-order expansions are also generated by symmetrization, as detailed in Appendix~\ref{sec:symmetrization}.  When $\hat{A}$  or $\hat{B}$ itself is a commutator ({\em i.e.}, nested commutators), the BCH
% expansion is reapplied until only elementary gates remain.

% Three strategies are then used to express the argument of the desired
% exponential gate in terms of a commutator. An argument of a qubit-conditional
% gate is expressed in terms of a commutator according to the Pauli
% commutator relations

% % \begin{align}
% % \sigma^{x} & =-\frac{i}{2}\left[\sigma^{y},\sigma^{z}\right]\label{eq:PauliCommutatorx}\\
% % \sigma^{y} & =-\frac{i}{2}\left[\sigma^{z},\sigma^{x}\right]\label{eq:PauliCommutatory}\\
% % \sigma^{z} & =-\frac{i}{2}\left[\sigma^{x},\sigma^{y}\right]\label{eq:PauliCommutatorz}
% % \end{align}
% and an argument consisting of a product of an anticommutator and a
% Pauli gate is expressed in terms of the Pauli anticommutator-commutator
% relation derived in Appendix~\ref{subsec:Anti-Commutator-to-Commutator}

% \begin{align}
% \left\{ \hat{A},\hat{B}\right\} \sigma^{i} & =i\left[i\hat{A}\sigma^{j},i\hat{B}\sigma^{k}\right]\label{eq:AnticommutatortoCommutator}
% \end{align}
% Finally, an argument consisting of a product of operators is expressed
% as
% \begin{align}
% \hat{A}\hat{B}\sigma^{j} & =\frac{1}{2}\left\{ \hat{A},\hat{B}\right\} \sigma^{j}+\frac{1}{2}\left[\hat{A},\hat{B}\right]\sigma^{j}\label{eq:ProductOperators}\\
% \left\{ \hat{A},\hat{B}\right\}  & =\hat{A}\hat{B}+\hat{B}\hat{A}\\
% \left[\hat{A},\hat{B}\right] & =\hat{A}\hat{B}-\hat{B}\hat{A}
% \end{align}
% which is further decomposed in terms of the Pauli anticommutator-commutator
% relation Eq.~(\ref{eq:AnticommutatortoCommutator}).

% This systematic approach is used to generate a wide array of qubit-conditional
% gates (Section~\ref{sec:Qubit-Conditional-Cavity-Gates}), universal
% control of a qubit encoded in a cavity (Section~\ref{subsec:Universal-Control}),
% and fermionic Fermi-Hubbard lattice dynamics on bosonic quantum computers
% (Section~\ref{sec:Fermi-Hubbard-Lattice-Dynamics}). All figures can be reproduced with the code contained in the same image folder.  Time is reported in unitless quantity via multiplication by the inverse time step $1/\tau$ to provide a sense of the required gate depth that is equally applicable to differing quantum computing architectures.

\newpage\leftline{\bf APPENDICES}

\section{Obtaining \texorpdfstring{$S_1$}{}}\label{obtaining_s1}
Here, we demonstrate how to obtain the $\mathcal{S}_1$ operator, with the native gates present in the dispersive coupling regime.

% If the $S_1$ gate is not present in a given instruction set, it can be in principle compiled from simpler gates. 
% As an example of the construction of a block-encoded operator, we concatenate oscillator phase-delays with displacements of the oscillator controlled on the qubit state. 

The controlled  displacement operator with magnitude $\alpha$ is written:
\begin{equation}
    U_{d}(\alpha) = e^{i  (\alpha a^\dagger + \alpha^* a) \otimes \sigma^z }.
\end{equation}
For a Fock state $\ket{n}$, $a^\dagger a \ket{n} = n \ket{n}$; therefore $e^{i a^\dagger a \theta} a^\dagger e^{-i a^\dagger a \theta}=e^{i\theta}a^\dagger$. Taking $\alpha=\alpha^*$, we have
\begin{equation}
    e^{i(\pi/2) a^\dagger a}e^{i(\alpha(a^\dagger + a))\otimes \sigma^y} e^{-i(\pi/2) a^\dagger a} = \exp\left(\alpha \begin{bmatrix}0 & (ia^\dagger - ia)\\ (-ia^\dagger + ia) &0 \end{bmatrix} \right).
\end{equation}
This operation can be built using single-qubit operations on a controlled displacement gate with additional phase delays on the oscillator.  Both are linear optical operations or single-qubit operations, which we expect to be inexpensive in our computational model.

Next, note that 
\begin{equation}
    e^{i(\alpha(a^\dagger + a))\otimes \sigma^x} = \exp\left(\alpha\begin{bmatrix} 0 & i(a^\dagger + a) \\ i(a^\dagger +a) &0 \end{bmatrix} \right).
\end{equation}
Thus to $O(\alpha^2)$ the block-encoded creation operation can be constructed using single-qubit operations, two controlled displacement operations and linear optical operations through

\begin{equation}
    \mathcal{S}_1 \approx e^{i(\pi/2) a^\dagger a}e^{i(\alpha(a^\dagger + a))\otimes \sigma^y} e^{-i(\pi/2) a^\dagger a}e^{i(\alpha(a^\dagger + a))\otimes \sigma^x}.
\end{equation}
This provides an intuition for our strategy for compiling gates. A major goal in this work is to optimize such formulas to minimize the number of operations needed to closely approximate evolution.  To this end, symmetric and higher-order approximations of this form will be vital to achieving the optimal scaling for such formulas and in turn get a deeper understanding of the costs of hybrid boson-qubit computations.

\[
\Qcircuit @C=1em @R=.7em {
\lstick{\textrm{Cavity } \ket{0}} & \qw & \gate{\exp i \frac{\pi}{2} \hat{n}} & \gate{CD[\alpha]} & \gate{\exp - i \frac{\pi}{2} \hat{n}} & \qw & \qw\\
%%%%
\lstick{\textrm{Qubit } \ket{0}} & \gate{S} & \gate{H} & \ctrl{-1} & \gate{H} & \gate{S^\dagger} & \qw
}
\]

\section{Error analysis}\label{apndx:error-analysis}
To assess the error scaling of the addition algorithm, we must consider three sources of error: the underlying implementation error from using approximations of $\mathcal{B}_A(t)$, the error from BCH, and the error from Trotter. In \cref{apndx:implementation-error-bch-trotter}, we show that the BCH and Trotter formulas can still be applied on exponentials that have error. Then, we use these formulas in \cref{apndx:general-addition-error} to produce the error bounds for addition. Finally, in \cref{apndx:multiplication}, we produce the error bounds for multiplication. 

% approximation of operations

\subsection{Product formulas with implementation error}\label{apndx:implementation-error-bch-trotter}
We begin by formally stating the Trotter and BCH formulas when there is no implementation error:
\begin{theorem}[BCH Product Formula (Theorem 2 from \cite{Childs_2013})]\label{fact:bch}
Let $A$ and $B$ be bounded complex-valued matrices and assume without loss of generality $t \in \mathbb{R}^+$ is assumed for the purposes of asymptotic analysis to be in $o(1)$.  We then define
\begin{align}
    \bch_{1, k} (At, Bt^k) \coloneqq e^{At} e^{Bt^k} e^{-At} e^{-Bt^k}.
\end{align}
We further define $\bch_{p,k}$ recursively for $p \geq 2$ and odd $ k \geq 1$:
\begin{align}
    \bch_{p + 1, k}(At, Bt^k) &\coloneqq \bch_{p, k} (A \gamma_p t, B (\gamma_p t)^k ) \bch_{p, k} (-A\gamma_p t, - B (\gamma_p t)^k) \\
    &\times \bch_{p, k} (A \beta_p t, B (\beta_p t)^k )^{-1} \bch_{p, k} (-A \beta_p t, - B (\beta_p t)^k )^{-1} \\
    &\times \bch_{p, k} (A \gamma_p t, B (\gamma_p t)^k ) \bch_{p, k} (-A \gamma_p t, - B (\gamma_p t)^k ),
\end{align}
with the following constants:
\begin{align}
    \beta_p \coloneqq (2r_p )^{1/(k + 1)}, \gamma_p \coloneqq (1/4 + r_p )^{1 / (k + 1)}, r_p \coloneqq \frac{2^{\frac{(k + 1)}{2p + k + 1}}}{4 \left(2 - 2^{\frac{k + 1}{2p + k + 1}} \right)}.
\end{align}
This recursive formula has the following error scaling, where $\gamma = \max(\norm{A}, \norm{B}^{1/k})$~\cite{Childs_2013}: 
\begin{align}
    \bch_{p, k}(At, Bt^k) = e^{[A, B]t^{k + 1} + \mathcal{O}((\gamma t)^{2p + k})}
\end{align}
and uses $8 \cdot 6^{p - 1}$ exponentials when $k = 1$ and $4 \cdot 6^{p-1}$ exponentials otherwise. 
\end{theorem}

\begin{theorem}[Trotter Formula (Lemma 1 from \cite{berry2007efficient})]\label{fact:trotter}
Let $\{H_j: j = 1\ldots m \}$ be a set of $M$ bounded Hermitian operators acting on a Hilbert space of dimension $2^n$ and assume without loss of generality that $t\ge 0$. For $H= \sum H_j$, the error in the Trotter-Suzuki formulas of order $k$ and timestep $r$ obeys the following error bound: 
\begin{align}
    \norm{\exp\left( -it \sum_{j = 1}^m H_j \right) - \trotter_{ 2k}(\{ H_j \}, t / r)^r  } \leq 5 (2 \times 5^{k - 1} m \tau)^{2k + 1} / r^{2k},
\end{align}
% \how{see notes}
where $\tau = \norm{H} t$ and
\begin{align}\label{req:trotter-constraints}
    4 m 5^{k - 1} \tau / r &\leq 1 , \\
    (16/3) ( 2 \times 5^{k - 1} m \tau)^{2k + 1} / r^{2k } &\leq 1 ,
\end{align}
using no more than $2 m 5^{k - 1} r$ exponentials. We define the $k^\text{th}$ order Trotter formula as:
\begin{align*}
    \trotter_{2k} (\lambda) := [\trotter_{2k - 2} (p_k \lambda)]^2 \trotter_{2k - 2} ((1 - 4p_k)\lambda) [\trotter_{2k - 2} (p_k \lambda)]^2,
\end{align*}
where $p_k = (4 - 4^{1/(2k - 1)})^{-1}$ and $k > 1$. The relation has the following base case:
\begin{align*}
    \trotter_2(\lambda) = \prod_{j = 1}^m e^{H_j \lambda / 2} \prod_{j' = m}^1 e^{H_{j'} \lambda / 2}, 
\end{align*}
\end{theorem}
which implies the following corollary: 
\begin{corr}\label{corr:trotter-r}
If $r = 1$, i.e. there is no time stepping, the Trotter formula exhibits the following error scaling:
\begin{align}
    \norm{\exp \left(-it \sum_{j = 1}^m H_j \right) - \trotter_{2k}(\{H_j\}, t)} \in \mathcal{O}((\norm{H} t)^{2k + 1}),
\end{align}
using no more than $2 m 5^{k - 1}$ exponentials.
\end{corr}
% \begin{proof}
% The bound and operator count follows immediately by substituting $r = 1$.
% \end{proof}

\newcommand{\uone}{U_1(t)}
\newcommand{\utwo}{U_2(t)}
\newcommand{\uonetilde}{\widetilde{U}_{1, p_1}(t)}
\newcommand{\utwotilde}{\widetilde{U}_{2, p_2}(t)}

Both of these formulas, however, assume that our implementation of exponentials occurs without error. However, if we would like to apply our technique recursively, our matrix product formulas must account for implementation error, i.e. be able to use primitives that themselves may have error. Thus, we restate both Trotter and BCH when the operators have asymptotic error:
{
\begin{lemma}[BCH under implementation error]\label{lem:bch-with-approx}
Suppose there are some ideal operators $\uone, \utwo$ which are exponentials of some anti-Hermitian matrix, i.e.: 
\begin{align}
    \uone &= \exp t A_1, \\
    \utwo &= \exp t A_2.
\end{align}
We seek to build $\exp t^2 [A_1, A_2]$, the exponential of the commutator of the matrices. Also suppose that may approximate $\uone, \utwo$ with $\uonetilde, \utwotilde$ with the following error scaling:
\begin{align}
    \norm{\uonetilde - \uone} &\in \mathcal{O}((c t)^{p_1}), \\
    \norm{\utwotilde - \utwo} &\in \mathcal{O}((c t)^{p_2}),
\end{align}
for some $p_1, p_2 \geq 1$. Then, applying a $q^\text{th}$-ordered BCH formula, where $q = \max(\ceil{\frac{\min(p_l, p_2) - 1}{2}}, 1 )$ on the implementable $\uonetilde, \utwotilde$ can still approximate the commutator  exponential by applying \cref{fact:bch}:
\begin{align}
    \norm{\exp t^2 [A_1, A_2] - \bch_{q, r}(\uonetilde, \utwotilde)} \in \mathcal{O}((C t)^{\min (p_1, p_2)}),
\end{align}
where $C = \max(\norm{A_1}, \norm{A_2}, c)$. This procedure uses $8 \cdot 6^{q - 1}$ total exponentials.

% \how{talk about how gamma will exceed c1, c2 norms, energy scaling of system.}

\end{lemma}
\begin{proof}
Recognize that we may decompose the error involved in implementing the commutator exponential into two sources: the error incurred from the BCH formula intrinsically and the implementation error from the realizable terms. Thus, by the triangle inequality:
\begin{align}
    &\norm{\exp t^2 [A_1, A_2] - \bch_q(\uonetilde, \utwotilde)} \leq \nonumber\\
    &\; \norm{\exp t^2 [A_1, A_2] - \bch_q(\uone, \utwo)} + \norm{\bch_q(\uone, \utwo) - \bch_q(\uonetilde, \utwotilde)} .
\end{align}
We begin with the LHS term. By \cref{fact:bch}, 
\begin{align}
    \norm{\exp t^2 [A_1, A_2] - \bch_q(\uone, \utwo)} \in \mathcal{O}((C_{\textrm{BCH}} t)^{2q + 1}),
\end{align}
where $C_{\textrm{BCH}} = \max(\norm{A_1}, \norm{A_2})$. 
For the RHS, recall by Box 4.1 of Nielsen and Chuang \cite{nielsen_chuang_2010} that implementation errors accumulate at most linearly; thus, we can sum over the $8 \cdot 6^{q- 1}$ operations used by BCH. By symmetry of the BCH formula, we apply the $\uonetilde, \utwotilde$ exponentials precisely $4 \cdot 6^{q- 1}$ times. Thus:
\begin{align}
    \norm{\bch_q(\uone, \utwo) - \bch_q(\uonetilde, \utwotilde)} &\leq \sum_{j = 1}^{4 \cdot 6^{q - 1}} \mathcal{O}((c t)^{p_1}) + \mathcal{O}((c t)^{p_2}) \\ 
    &\in \mathcal{O}((c t)^{\min(p_l, p_r)}).
\end{align}
Setting $q = \max\{\ceil{\frac{\min(p_l, p_r) - 1}{2}}, 1 \} $ and recalling $\gamma \geq 1$, observe:
\begin{align}
    \norm{\exp t^2 [A_1, A_2] - \bch_q(\uonetilde, \utwotilde)} \in \mathcal{O}((C t)^{\min(p_l, p_r)}),
\end{align}
as desired.
\end{proof}
\begin{lemma}[Trotter under implementation error]\label{lem:trotter-with-approx}
Given~\cref{lem:bch-with-approx}'s assumptions, \cref{fact:trotter}'s assumptions, and assuming $\norm{A_1 + A_2} \geq 1$, an operator can be constructed by applying a $q^\text{th}$-ordered Trotter formula $\trotter_{2q}$ by setting $q = \max(\ceil{\frac{\min(p_l, p_2) - 1}{2}}, 1 )$ so that: 
% \how{this sounds weird}
\begin{align}
    \norm{\exp t (A_1 + A_2) - \trotter_q (\uonetilde, \utwotilde)} \in \mathcal{O}((C t)^{\min(p_1, p_2)}),
\end{align}
where $C = \max(\norm{A_1 + A_2}, c)$ and using no more than $4 \cdot 5^{q -1} $ operator exponentials. 
% \leq 2^{ + 1 / q} 5^{2q} \gamma^{1 + 1/2q}
% where $\gamma = \max(\norm{A_1}, \norm{A_2})$.
\end{lemma}
\begin{proof}
Similarly, we may apply the triangle inequality in order to determine a bound by separating the error accrued into the intrinsic Trotter error and the implementation error:
\begin{align}
    &\norm{\exp t (A_1 + A_2) - \trotter_{2q} (\uonetilde, \utwotilde) } \nonumber\\
    &\leq \norm{\exp t(A_1 + A_2) - \trotter_{2q} (\uone, \utwo)}\nonumber \\
    &\qquad+ \norm{\trotter_{2q} (\uone, \utwo) - \trotter_{2q} (\uonetilde, \utwotilde)}.
\end{align}
To analyze the LHS, which represents the Trotter error, \cref{corr:trotter-r} provides a bound:
\begin{align}
    \norm{\exp t(A_1 + A_2) - \trotter_{2q} (\uone, \utwo)} \in \mathcal{O}((\norm{A_1 + A_2}t)^{2q + 1}).
\end{align}
For the RHS, which represents the implementation error, recall by box 4.1 of Nielsen and Chuang \cite{nielsen_chuang_2010} that the error accrues linearly. Furthermore, the number of operations in Trotter is no more than $4 \cdot 5^{q - 1} $ operations in total. Therefore, we only apply each constituent operation at most $2 \cdot 5^{q - 1}$ times. Thus, a loose upper bound can be written as:
\begin{align}
    \norm{\trotter_{2q} (\uone, \utwo) - \trotter_{2q} (\uonetilde, \utwotilde)} &\leq \sum_{j = 1}^{4 \cdot 5^{q -1 }} \mathcal{O}((c t)^{p_1}) + \mathcal{O}((c t)^{p_2}) \\
    &\in \mathcal{O}((c t)^{\min(p_1, p_2)}).
\end{align}
It is sufficient to set $q = \max(\ceil{\frac{\min(p_1, p_2) - 1}{2}}, 1)$ so that:
\begin{align}
    &\norm{\exp t (A_1 + A_2) - \trotter_{2q} (\uonetilde, \utwotilde) } \\
    &\qquad \in \mathcal{O}((\norm{A_1 + A_2} t)^{2q + 1}) + \mathcal{O}(( c t)^{\min(p_1, p_2)}) =  \mathcal{O}((Ct)^{\min(p_1, p_2)}),
\end{align}
as desired.
\end{proof}
}


\subsection{Scaling of the addition algorithm}\label{apndx:general-addition-error}
We apply the above results to produce the error analysis of \cref{alg:adder}:
\algproduct*
% \begin{theorem}\label{thm:general-adder-error}
%     Suppose we have approximations $\tilde{\mathcal{B}}_A(t), \tilde{\mathcal{B}}_B(t)$ with the following error:
%     \begin{align}
%         \norm{\tilde{\mathcal{B}}_A(t) - \mathcal{B}_A(t)} &\in \mathcal{O}( (ct)^{p_A} ) \\
%         \norm{\Tilde{\mathcal{B}}_B(t) - \mathcal{B}_B(t)} &\in \mathcal{O} ((ct)^{p_B} ) 
%     \end{align}
%     For some constant $c$ and order $p_A, p_B \geq 1$. Then, the application of \cref{alg:adder} will yield the following scaling:
%     \begin{align}
%         \norm{\textrm{ADD}(\Tilde{\mathcal{B}}_A(t), \Tilde{\mathcal{B}}_B(t) ) - \exp it \begin{bmatrix}
%             0 & AB \\
%             (AB)^\dagger & 0 
%         \end{bmatrix}} \in \mathcal{O} ((C_{TOTAL} t)^{\min(p_A, p_B) / 2})
%     \end{align}
%     With $C_{TOTAL} = \max( \norm{AB}, \norm{(AB)^\dagger}, C_{BCH}^2)$ and $C_{BCH} = \max( \norm{A}, \norm{B}, c)$,
%     using no more than $ 1.07 \cdot 30^q $ exponentials. 
% \end{theorem}
\begin{proof}
Our proof proceeds by applying the above theorems upon our operations. By setting $q = \max(\ceil{\frac{\min(p_1, p_2) - 1}{2}}, 1)$, \cref{lem:bch-with-approx} implies that:
\begin{align}
    \norm{\textrm{BCH}_q ( X \widetilde{\mathcal{B}}_B(\tau) X, \widetilde{\mathcal{B}}_A(\tau)) -  \exp \tau^2 [A, B] \sigma^z} &\in \mathcal{O}((C_{BCH}\tau)^{p_{BCH}}), \label{eq:C}\\
    \norm{\textrm{BCH}_q (S \widetilde{\mathcal{B}}_A(\tau) S^\dagger,  X \widetilde{\mathcal{B}}_B(\tau) X) -  \exp i \tau^2 \{A, B \} \sigma^z} &\in \mathcal{O}((C_{BCH} \tau)^{p_{BCH}}), \label{eq:AC}
\end{align}
where $C_{BCH} = \max( \norm{A}, \norm{B}, c)$ and $p_{BCH} = \min (p_A, p_B) $. We'll need to set $\tau = \sqrt{\frac{t}{2}}$ to achieve the desired time evolution. Additionally, Pauli conjugation has no impact on the error scaling. Call these formulas ``Left" and ``Right":
\begin{align}
    \textrm{LEFT} &\coloneqq SH \cdot \textrm{BCH}_q ( X \widetilde{\mathcal{B}}_B(\tau) X, \widetilde{\mathcal{B}}_A(\tau)) \cdot HS^\dagger, \\
    \textrm{RIGHT} &\coloneqq H \textrm{BCH}_q (S \widetilde{\mathcal{B}}_A(\tau) S^\dagger, X\widetilde{\mathcal{B}}_B(\tau) X ) H,
\end{align}
where:
\begin{align}
    \norm{\textrm{LEFT} - \exp \tau^2 (AB - (AB)^\dagger) \sigma^y} &\in \mathcal{O}((C^2 t)^{p_{BCH} / 2}), \\
    \norm{\textrm{RIGHT} - \exp i \tau^2 (AB + (AB)^\dagger)  \sigma^x } &\in \mathcal{O}((C^2 t)^{p_{BCH} / 2}).
\end{align}
Finally, we use \cref{lem:trotter-with-approx}, implying that a Trotter formula with order $q$ has the following error scaling:
\begin{align}
    \norm{\textrm{Trotter}_s (\textrm{LEFT}, \textrm{RIGHT})  - \exp it \begin{bmatrix}
        0 & (AB)^\dagger \\
        AB & 0 
    \end{bmatrix} } \in \mathcal{O}((C_{TOTAL} t)^{p_{BCH} / 2}),
\end{align}
where: 
% \how{does this analysis work because it depends on scaling on earlier levels which depends on tau?}
\begin{align}
    C_{TOTAL} = \max \left( \norm{\begin{bmatrix}
        0 & (AB)^\dagger \\
        AB & 0
    \end{bmatrix}}, C_{BCH}^2 \right) = \max \left(\norm{AB}, \norm{(AB)^\dagger}, C_{BCH}^2 \right).
\end{align}




To bound the number of operations used, recognize that the Trotter formula requires at most $4 \cdot 5^{q - 1}$ commutators, each of which requires $8 \cdot 6^{q - 1}$ constituent operators. Thus, the total number of operators required is bounded as follows:
\begin{align}
    4 \cdot 5^{q - 1} \cdot 8 \cdot 6^{q - 1} \leq 1.07 \cdot 30^q ,
\end{align}
\end{proof}


\newcommand{\leftarbop}{\widetilde{\mathcal{S}}_{k_l, p_l}(t)}
\newcommand{\leftarbopt}{\widetilde{\mathcal{S}}_{k_l, p_l}(\tau)}
\newcommand{\leftarbopx}{\widetilde{\mathcal{S}}^X_{k_l, p_l}(\tau)}
\newcommand{\leftarbopy}{\widetilde{\mathcal{S}}^Y_{k_l, p_l}(\tau)}

\newcommand{\rightarbop}{\widetilde{\mathcal{S}}_{k_r, p_r}(t)}
\newcommand{\rightarbopt}{\widetilde{\mathcal{S}}_{k_r, p_r}(\tau)}
\newcommand{\rightarbopx}{\widetilde{\mathcal{S}}^X_{k_r, p_r}(\tau)}
\newcommand{\rightarbopxr}{\widetilde{\mathcal{S}}^X_{k_r, p_r}(\tau / r_{\textrm{BCH})}}
\newcommand{\rightarbopy}{\widetilde{\mathcal{S}}^Y_{k_r, p_r}(\tau)}
\newcommand{\rightarbopyr}{\widetilde{\mathcal{S}}^Y_{k_r, p_r}(\tau / r_{\textrm{BCH}})}
which implies the following corollary for the annihilation/creation operators:
\begin{corr}[\cref{alg:adder} applied to polynomials of annihilation/creation operators]\label{lem:adder}
Assume we can implement the following $k_l, k_r^\text{th}$ order approximations of $S$ with error scaling $p_l, p_r$:
\begin{align}
    \norm{\leftarbop - \mathcal{S}_{k_l}(t)} &\in \mathcal{O}((c t)^{p_l}), \\
    \norm{\rightarbop - \mathcal{S}_{k_r}(t)} &\in \mathcal{O}((c t)^{p_r}),
\end{align}
with $c \geq \Lambda^{\max(k_l, k_r) / 2}$. Then, we can implement higher order operators with comparable $t$ scaling:
\begin{align}
    \norm{
    \widetilde{\mathcal{S}}_{k_l + k_r, \min(p_l, p_r)}(t)  - 
    \exp \left(i t \begin{bmatrix}
    0 & ( a^\dagger )^{k_l + k_r} \\
    a^{k_l + k_r} & 0
    \end{bmatrix}
    \right)
    } \in \mathcal{O}((c^2 t)^{\min(p_l, p_r) / 2}),
\end{align}
using no more than $1.07 \cdot 30^q$ $\mathcal{S}_{k_l},  \mathcal{S}_{k_r}$ operators.
\end{corr}
\begin{proof}
To synthesize the block-encoding of higher-order annihilation/creation operators, we will directly apply 
 \cref{thm:general-adder-error}. We quantify the error by bounding the block-encoding norm. Note that:
    \begin{fact}\label{fact:norm-blockencodedxt}
A $k^\text{th}$ order block-encoded operator has a bounded norm:
\begin{align}
    \norm{\begin{bmatrix}
        0 & ( a^\dagger )^k \\
        ( a )^k & 0
    \end{bmatrix}}  \leq \Lambda^{k / 2}.
\end{align}
\end{fact}
Thus, the constant $C_\textrm{BCH} $ is bounded, as $C_\textrm{BCH} \leq \max(\Lambda^{k_l / 2}, \Lambda^{k_r / 2}, c) \leq \max( \Lambda^{\max(k_l, k_r) / 2}, c) = c$ by hypothesis. Next, observe that $\norm{AB}, \norm{(AB)^\dagger} \leq \Lambda^{(k_l + k_r) / 2} $. Thus, $C_{TOTAL} \leq \max( \Lambda^{(k_l + k_r) / 2}, c^2) = c^2$. Therefore, the final error scaling will be upper bounded by $\mathcal{O}((c^2 t )^{\min(p_l, p_r) / 2})$.
\end{proof}

% And directly apply \cref{thm:general-adder-error}. 


% \how{is there a relationship between norm M and norm Mdag?}

\subsection{Scaling of the multiplication algorithm}\label{apndx:multiplication}
\cref{alg:mult}'s error scaling follows directly from the BCH formula:
\algmult*
\begin{proof}
    We can directly apply \cref{lem:bch-with-approx} using the $\widetilde{\mathcal{B}}_A, \widetilde{\mathcal{B}}_B$ operators. When applied, we find that:
    \begin{align}
        \norm{ \textrm{BCH}_q( S \widetilde{\mathcal{B}}_A(\tau) S^\dagger, \widetilde{\mathcal{B}}_B(\tau)) - \exp 2 i \tau^2 \begin{bmatrix}
            AB & 0 \\
            0 & - BA - (BA)^\dagger
        \end{bmatrix}} \in \mathcal{O}((C \tau)^{\min(p_A, p_B)}),
    \end{align}
    where $C = \max (\norm{A}, \norm{B}, c)$ and $q = \max ( \ceil{\frac{\min(p_A, p_B) - 1}{2}}, 1) $. Then, by taking $\tau = \sqrt{\frac{t}{2}}$, we yield:
    \begin{align}
        \norm{\textrm{BCH}_q(S \widetilde{\mathcal{B}}_B(\tau) S^\dagger, \widetilde{\mathcal{B}}_A(\tau) ) - \exp i t \begin{bmatrix}
            AB & 0 \\
            0 & -BA
        \end{bmatrix}} \in \mathcal{O}((C \tau)^{\min(p_A, p_B)}) = \mathcal{O}((C^2 t)^{\min(p_A, p_B) / 2}).
    \end{align}
    By counting the number of exponentials in the result via \cref{lem:bch-with-approx} we finally find that the number of exponentials needed is at most $8 \cdot 6^{q - 1}$. 
\end{proof}


%%%%%%%%%%%%%%
\section{Phase-Space Applications}
In the following section, we derive error bounds for the two phase-space applications described: the conditional rotation gate and the controlled-phase beam splitter. However, the typical cutoff approach we employ to bound $\norm{a}, \norm{a^\dagger}$ is more complex for position and momentum operators.

To obtain error bounds, we leave all expressions in terms of $\norm{\hat{x}}, \norm{\hat{p}}$. We leave a more concrete bound --- which can be found by applying a cutoff upon both $\hat{x}, \hat{p}$ simultaneously --- to future work. 

\subsection{Conditional rotation gate}\label{subsec:cond-rot}

\resultmeasurement
\begin{proof}
    Begin by directly applying \cref{fact:bch} to identify the error scaling. This implies that:
    \begin{align}
        \norm{\textrm{BCH}_p(\exp i \tau \hat{x} \sigma^i, \exp i \tau \hat{x} \sigma^j) - \exp \tau^2 \hat{x}^2 [\sigma^i, \sigma^j] } &\in \mathcal{O}((\norm{\hat{x}} \tau)^{2p + 1}), \\
        \norm{\textrm{BCH}_p(\exp i \tau \hat{p} \sigma^i, \exp i \tau \hat{p} \sigma^j) - \exp \tau^2 \hat{p}^2 [\sigma^i, \sigma^j] } &\in \mathcal{O}((\norm{\hat{p}} \tau)^{2p + 1}).
    \end{align}
    Without loss of generality (WLOG), select $\sigma^i = \sigma^y$ and $\sigma^j = \sigma^z$ so that $[\sigma^i, \sigma^j] = 2i \sigma^x$. Then, by selecting $\tau = \sqrt{\frac{t}{2}}$, the BCH formula is an approximation of  $\mathcal{B}_{\hat{x}^2} (t)$. Thus,
    \begin{align}
        \norm{\widetilde{ \mathcal{B}}_{\hat{x}^2} - \exp i t \hat{x}^2 \sigma^x} &\in \mathcal{O}((\norm{\hat{x}}^2 t)^{p + \frac{1}{2}}).
    \end{align}
    Similarly, for $\hat{p}^2$:
    \begin{align}
        \norm{\widetilde{ \mathcal{B}}_{\hat{p}^2} - \exp i t \hat{p}^2 \sigma^x} &\in \mathcal{O}((\norm{\hat{p}}^2 t)^{p + \frac{1}{2}}).
    \end{align}
    We apply \cref{lem:trotter-with-approx} and observe:
    \begin{align}
        \norm{\textrm{Trotter}_q( \widetilde{\mathcal{B}}_{\hat{x}^2}, \widetilde{\mathcal{B}}_{\hat{p}^2}) - \exp i t (\hat{x}^2 + \hat{p}^2)\sigma^x } \in \mathcal{O}( (C t)^{p + \frac{1}{2}}),
    \end{align}
    where $C = \max( \norm{ \hat{x}^2 + \hat{p}^2 }, \norm{\hat{x}}^2, \norm{\hat{p}}^2)$ and $q = \ceil{\frac{p}{2} - \frac{1}{4}}$. This requires no more than $4 \cdot 5^{q - 1}$ operator exponentials, thus implying:
    \begin{align}
        4 \cdot 5^{q - 1} \leq 4 \cdot 5^{ \frac{p}{2} - \frac{1}{4} }.
    \end{align}
\end{proof}

% \how{new asymptotic analysis on the maintext formulas}




\subsection{Controlled-phase beam splitter gate}\label{subsec:controlled-phase}
\resultbeamsplitter
\begin{proof}
We may take a similar approach as above. Apply \cref{fact:bch}:
\begin{align}
    \norm{\textrm{BCH}_p(\exp i \tau \hat{x}_1 \sigma^i, \exp i \tau \hat{x}_2 \sigma^j) - \exp \tau^2 \hat{x}_1 \hat{x}_2 [\sigma^i, \sigma^j] } &\in \mathcal{O}((\norm{\hat{x}} \tau)^{2p + 1}), \\
    \norm{\textrm{BCH}_p(\exp i \tau \hat{p}_1 \sigma^i, \exp i \tau \hat{p}_2 \sigma^j) - \exp \tau^2 \hat{p}_1 \hat{p}_2 [\sigma^i, \sigma^j]  } &\in \mathcal{O}((\norm{\hat{p}} \tau)^{2p + 1}),
\end{align}
where we set $\norm{\hat{x}} = \max( \norm{\hat{x}_1}, \norm{\hat{x}_2})$ and $\norm{\hat{p}} = \max(\norm{\hat{p}_1}, \norm{\hat{p}_2})$. We again take $\tau = \sqrt{\frac{t}{2}}$ so that:
\begin{align}
    \norm{\widetilde{ \mathcal{B}}_{\hat{x}_1 \hat{x}_2} - \exp i t \hat{x}_1 \hat{x}_2 \sigma^x} &\in \mathcal{O}((\norm{\hat{x}}^2 t)^{p + \frac{1}{2}}), \\
    \norm{\widetilde{ \mathcal{B}}_{\hat{p}_1 \hat{p}_2} - \exp i t \hat{p}_1 \hat{p}_2 \sigma^x} &\in \mathcal{O}((\norm{\hat{p}}^2 t)^{p + \frac{1}{2}}) .
\end{align}
Applying \cref{lem:trotter-with-approx}:
\begin{align}
    \norm{\textrm{Trotter}_q( \widetilde{\mathcal{B}}_{\hat{x}_1 \hat{x}_2}, \widetilde{\mathcal{B}}_{\hat{p}_1 \hat{p}_2}) - \exp i t (\hat{x}_1 \hat{x}_2 + \hat{p}_1 \hat{p}_2)\sigma^x } \in \mathcal{O}( (C t)^{p + \frac{1}{2}}),
\end{align}
where $C = \max( \norm{ \hat{x}_1 \hat{x}_2 + \hat{p}_1 \hat{p}_2}, \norm{\hat{x}}^2, \norm{\hat{p}}^2)$ and $q = \ceil{\frac{p}{2} - \frac{1}{4}}$. This requires no more than $4 \cdot 5^{q - 1}$ operator exponentials, thus implying:
    \begin{align}
        4 \cdot 5^{q - 1} \leq 4 \cdot 5^{ \frac{p}{2} - \frac{1}{4} }.
    \end{align}
\end{proof}


\section{Fock-Space Applications}
We now introduce a series of techniques that allow us to realize polynomials of Fock-space operators. We first begin in \cref{subsec:arbitrary_power} by identifying the error scaling of an arbitrary order Fock-space block encoding, i.e. $\mathcal{B}_{a^k}$. In \cref{apndx:error-jc}, we show how the techniques can be used to simulate the Jaynes-Cummings Hamiltonian, which itself is a polynomial of Fock space operators. In \cref{subsec:state-prep}, we demonstrate how this technique can be extended beyond simulation into realizing more general operators, like a state-prep unitary.

\subsection{Realizing block encodings of arbitrary order}\label{subsec:arbitrary_power}
We seek to demonstrate the following result:
\begin{theorem}\label{thm:main}
For positive integer $k \geq 1$ and timestep $t \in\mathbb{R}$, we seek to implement the target block encoding $\mathcal{T}_k(t)$ defined as:
\begin{align}
    \mathcal{T}_{k}(t) = \exp \left( it \begin{bmatrix}
    0 &  ( a^{\dagger})^{k}  \\
    ( a )^{k} & 0
    \end{bmatrix} \right).
\end{align}
For any $\epsilon>0$ and $p > 1$ there exists an implementable unitary operation $\widetilde{\mathcal{T}}_{k, p}$ of order $p$ such that:
\begin{align}
    \norm{\mathcal{T}_k - \widetilde{\mathcal{T}}_{k, p}}\leq \epsilon,
\end{align}
and the number of applications of $\mathcal{S}_1(t)$ needed to implement the operation scales in:
\begin{align}
    r \cdot n^{1.6} 30^{np} 420^{n^2 p / 2} 6^{\log_2 n + 1} ,
\end{align}
where $r \in \Theta \left( \frac{(\Lambda^{k/2}t)^{1 + 1/(p - 1)}}{\epsilon^{1 / (p - 1)}} \right)$.
\end{theorem}

Because we can add two lower-order block encodings via \cref{alg:adder}, we can exploit a binary expansion to achieve arbitrary orders (e.g.~$(a^\dagger)^9 = (a^{\dagger})^{2^3} a^\dagger$). Thus, our first task is to demonstrate the implementation of these block encodings with orders that are a power of two. This is achievable through the recursive $\textrm{POWER}$ algorithm:
\begin{algorithm}[H]
\caption{POWER($k, t, p$) }\label{alg:power_of_two}
% \how{ algorithm ensure }
\begin{algorithmic}
\Require $k = 2^\ell$ for nonnegative integer $\ell$, timestep $t > 0$, order $p > 1$
\Ensure $\widetilde{\mathcal{T}}_k$ with $\norm{\widetilde{\mathcal{T}}_k  - \exp it \begin{bmatrix}
    0 & (a^\dagger)^k \\
    a^k & 0
\end{bmatrix}} \in \mathcal{O}((\Lambda^{k / 2} t)^p)$
\If{$k = 1$}
    \State \Return $\mathcal{S}_1(t)$
\Else
    \State $p' \coloneqq 2p$
    \State $\textrm{HalfOp} \coloneqq \textrm{POWER}(k/2, \sqrt{t/2}, p')$
    \State \Return $\textrm{ADD}(\textrm{HalfOp}, \textrm{HalfOp}, p', p', \sqrt{t / 2})$
\EndIf
\end{algorithmic}
\end{algorithm}

Building to the following result:
\begin{theorem}\label{lem:key-scaling}
For any $t \geq 0$, $p \geq 1$, and fixed $k = 2^\ell$ for some $\ell \geq 1$ we have that the unitary implemented by~\Cref{alg:power_of_two}, ${\rm POWER}$ acting on $\mathcal{H}_2\otimes \mathcal{H}_\Lambda$, satisfies:
\begin{align}
    \norm{ {\rm POWER}(k,t,p) - \exp \left( i t \begin{bmatrix}
    0 & (a^\dagger)^{k} \\
    ( a )^{k}  & 0
    \end{bmatrix} \right) } \in \mathcal{O}((\Lambda^{k/2} t)^{p}),
\end{align}
using no more than $ 6^{\log_2 k} \cdot 420^{kp / 2}$ unitary $\mathcal{S}_1$ operators.
% , which are defined in
\end{theorem}

To bound the error of this algorithm, we can begin by identifying the implementation error of the second-order formula, i.e. $\widetilde{\mathcal{S}}_2$, the first operator with implementation error:
\begin{lemma}[Implementing second-order block encodings]\label{corr:second_order}
Suppose we can implement the following operation without error (as defined in~\Cref{defn:SX} and subject to a bosonic cutoff):
\begin{align}
\mathcal{S}_1(t) = \exp \left( it 
\begin{bmatrix}
    0 & a^\dagger
    \\
    a & 0
\end{bmatrix} \right).
\end{align}
Then, we can approximate $\mathcal{S}_2(t) $ to the $p^\text{th}$ order, i.e. implement $\widetilde{\mathcal{S}}_2$ such that:
\begin{align}
    \norm{\widetilde{\mathcal{S}}_{2, p}(t) - \mathcal{S}_2(t)} \in \mathcal{O}((\Lambda t)^{p + \frac{1}{2}}),
\end{align}
using no more than $6 \cdot 14^p$ $\mathcal{S}_1(t)$ operations.
\end{lemma}
\begin{proof}
Note that, if $\mathcal{S}_1(t)$ is errorless, then we only need to account for error incurred by the BCH and Trotter formulas. By employing a $p^\text{th}$ order BCH formula we can produce commutator exponentials with error $\mathcal{O}((\Lambda^{1/2} \tau)^{2p + 1})$ by \cref{fact:bch} and \cref{fact:norm-blockencodedxt}.

% \how{230227 reviewed up to here}

We again set $\tau = \sqrt{\frac{t}{2}}$ so that the error scales in at worst $\mathcal{O}((\Lambda t)^{p + \frac{1}{2}})$. Then, we apply a Trotter formula \cref{lem:trotter-with-approx} of order $\ceil{\frac{p}{2}}$ so that:
\begin{align}
    \norm{\exp \left(i t \begin{bmatrix}
        0 & ( a^\dagger)^2 \\
        ( a )^2 & 0
    \end{bmatrix} \right) - \widetilde{\mathcal{S}}_2(t)} \in \mathcal{O}((Ct)^{p + \frac{1}{2}}),
\end{align}
with $C \leq \max( \Lambda, \Lambda^{2/2}) = \Lambda $ so that our worst case error scaling is $\mathcal{O}((\Lambda t)^{p + \frac{1}{2}})$. This requires no more than $2 \cdot 2 \cdot 5^{\ceil{ \frac{p}{2}} - 1}$ of the commutators, each of which required $8 \cdot 6^{p - 1}$ first order operations. Thus, the cost scales in no more than:
\begin{align}
    4 \cdot 5^{p / 2} \cdot 8 \cdot 6^{p - 1} \leq 6 \cdot 14^p
\end{align}
total number of $\mathcal{S}_1$ operations.
\end{proof}

This base case allows us to analyze the performance of \cref{alg:power_of_two}:

\begin{proof}[Proof of \cref{lem:key-scaling}]
We demonstrate the bounds inductively. The base case ($\ell = 1$) holds via \cref{corr:second_order}. For the inductive hypothesis, we assume that, for any $p' \geq 1$ and $k = 2^\ell$, we may implement $\textrm{POWER}(k, t, p')$:
\begin{align}
    \norm{ {\rm POWER}(k,t, p') - \exp \left( i t \begin{bmatrix}
    0 & ( a^\dagger )^{k} \\
    ( a )^{k}  & 0
    \end{bmatrix} \right) } \in \mathcal{O}((\Lambda^{k / 2} t)^{p'}).
\end{align}
To demonstrate the inductive step, we seek to apply \cref{lem:adder} directly to the implementable operators from the inductive hypothesis. Thus, we set $p' = 2p$ so that:
\begin{align}
    \norm{ {\rm POWER}(2k,t, p) - \exp \left( i t \begin{bmatrix}
    0 & (a^\dagger)^{2k} \\
    ( a )^{2k}  & 0
    \end{bmatrix} \right) } \in \mathcal{O}((\Lambda^{2k / 2} t)^{p' / 2}) = \mathcal{O}((\Lambda^{2k/2} t)^{p}),
\end{align}
our desired error scaling. By \cref{lem:adder}, we require an adder of order $\max(\ceil{\frac{p' - 1}{2}}, 1) \leq p + \frac{1}{2}$. Thus, the adder requires requires $1.07 \cdot 30^{p + 1/2} \leq 6 \cdot 30^p$ of the $\textrm{POWER}(k, t, p')$ operations, i.e.:
\begin{align}
    \textrm{COST}(2k, p) &\leq 6 \cdot 30^p \cdot \textrm{COST}(k, 2p)  \\
    &\leq 6 \cdot 30^p \cdot 6 \cdot 30^{2p} \cdot \textrm{COST}(k / 2, 4p) \\
    &\leq \prod_{j = 1}^{n} 6 \cdot 30^{2^{j - 1}p} \cdot \textrm{COST}(2k / 2^n, 2^n p) \\
    &\leq 6^\ell \cdot 30^{kp} \cdot \textrm{COST}(2, k p).
\end{align}
Since $\textrm{COST}(2, kp) \leq 6 \cdot 14^{kp}$ by \cref{corr:second_order}, the number of $\mathcal{S}_1$ operations is upper bounded by:
\begin{align}
    \textrm{COST}(2k, p) \leq  6^{\log_2 k + 1} \cdot 420^{k p} \implies \textrm{COST}(k, p) \leq 6^{\log_2 k} \cdot 420^{kp / 2}.
\end{align}
\end{proof}

%% this is the theorem



% \subsection{Achieving Arbitrary Orders}\label{subsec:arb_order}
% Via the BCH formulas, we can also build operators of arbitrary order. 

Together, the POWER and ADD algorithms allow us to approximate arbitrary orders. We describe a recursive algorithm below to construct any order $k \geq 1$:
\begin{algorithm}[H]
\caption{ARB\_POWER($k, t, p, l, r$) that produces $\mathcal{T}_k(t)$ for any $k \geq 1$}\label{alg:arb_power}
\begin{algorithmic}
\Require $k > 0$ and has the binary representation $k = k_n k_{n-1} ... k_1$.

\If{$r - l = 0$}
    \If{$k_r = 1$}
        \State \Return $\textrm{POWER}(H, 2^r, t, p)$
    \Else
        \State \Return $RX(t) \kron \identity$
    \EndIf
\Else
    \State \Return $\textrm{ADD}(\textrm{ARB\_POWER}(k, \sqrt{t/2}, p, l, \floor{\frac{r - l}{2}} + l), \textrm{ARB\_POWER}(k, \sqrt{t/2}, p, l + \floor{\frac{r - l}{2}} + 1, r) $
\EndIf
\end{algorithmic}
\end{algorithm}


% \how{this algorithm is extra - we use RX instead of just iterating}.
% \textbf{The theorem statement below is unclear, specifically the target operator isn't obvious to an external reader also it may be useful to reduce the number of $S_X$ for the statement of the theorem into raw number of exponentials needed since those are more likely to be the drivers of the cost.}
% \how{under the assumptions of blah blah blah this holds }

\begin{theorem}\label{thm:arb_calc}
Assuming that the $\mathcal{S}_1$ operator can be implemented without error, the \cref{alg:arb_power} produces a series of gates $\{ \mathcal{S}_1(t_i(t)) \}$ such that:
\begin{align}
    \norm{\prod_i \mathcal{S}_1(t_i(t)) - \mathcal{T}_k(t)} \in \mathcal{O}((\Lambda^{k/2} t)^{p}),
\end{align}
where $\mathcal{T}_k$ is our target operator and we have order $k > 0$. The number of $\mathcal{S}_1$ gates required is no more than $n^{1.6} 30^{np} 420^{n^2 p / 2} 6^{\log_2 n + 1}$.
\end{theorem}
\begin{proof}
We demonstrate this constructively on the worst case scenario where $k = k_n k_{n - 1} ... k_1$ and $k_n, k_{n-1}..., k_1 = 1$. WLOG, we assume $n = 2^{\ell}$ for some integer $\ell$. This is because, if $n$ not a power of two, we can simply pad the leading digits with zeros to achieve a balanced binary tree.

The proof is as follows: we first identify the error scaling necessary for each leaf node of the binary tree so that the overall formula has our desired order, then we perform the cost accounting and estimate the number of $\mathcal{S}_1$ operations required.

% \how{key idea: count the number of leaf nodes, then add the number of sx requried for each}

To achieve an error scaling of $\mathcal{O}((\Lambda^{2^n / 2} t)^p)$, the two terms being `added' below must have error scaling of order at worst $\mathcal{O}((\Lambda^{2^n / 4} t)^{2p})$ by \cref{lem:adder}, and so on for each subsequent layer. Thus, each of the leaf $\textrm{POWER}$ terms must have order at least $\mathcal{O}((\Lambda^{2^n / 2^{1 + \log_2 n} } t)^{2^{\log_2 n} p }) = \mathcal{O}(( \Lambda^{2^n / 2n} t)^{np})$.

Now, we compute the cost incurred by the formula. We begin by counting the number of times $\textrm{POWER}$ is used, then accounting for the number of $\mathcal{S}_1$ required to implement each $\textrm{POWER}$. Note that each $\textrm{POWER}$ term will be used proportionally to the number of $\textrm{ADD}$ operations necessary, so we can compute the cost of implementing each $\textrm{POWER}$ operation for the $i^\text{th}$ digit, i.e. the $\textrm{POWER}$ operation has degree $j = 2^i$. We thus bound the number of $\mathcal{S}_1$ operations required to implement this $np^\text{th}$-ordered operator: 
\begin{align}
    6^{\log_2 j} \cdot 420^{j n p / 2} = 6^{i} \cdot 420^{2^i n p / 2}.
\end{align}
Finally, we seek to bound the number of times each $\textrm{POWER}$ operator is used through the $\textrm{ADD}$ algorithm. Recall from \cref{lem:adder} that each $\textrm{ADD}$ operation requires at most $1.07 \cdot 30^q$ of the constituent operators, where $q = \max ( \ceil{\frac{\min(p_l, p_r) - 1}{2}}, 1)$ and $p_l, p_r$ are the orders of the underlying operators. By assuming symmetry of the Trotter formula for $\textrm{ADD}$, each addition requires $\frac{1}{2} 1.07 \cdot 30^q$ of the underlying operator. Thus, we can obtain a bound on the number of applications required of each fundamental $\textrm{POWER}$ operator:
\begin{align}
    \prod_{s = 1}^{\log_2 n} \frac{1}{2} 1.07 \cdot 30^{ 2^{s - 1}p + 1/2} \leq 3^{\log_2 n} 30^{np} \leq n^{1.6} 30^{np}
\end{align}
% \how{i used s here because it shows that we're adding over add alg, not the digit position}
because the $s^\text{th}$ layer of $\textrm{ADD}$ requires constituent operators of order $2^s p$, so $q \leq 2^{s - 1}p + \frac{1}{2}$. 

Finally, consider the total cost by adding up the cost of the individual $\textrm{POWER}$ operators multiplied by the number of applications required:
\begin{align}
    \sum_{i = 1}^{\log_2 n} n^{1.6} 30^{np} \cdot 6^{i} \cdot 420^{2^{i} np / 2} &\leq n^{1.6} 30^{np} 420^{n^2 p / 2} \sum_{i = 1}^{\log_2 n} 6^i \\
    &\leq n^{1.6} 30^{np} 420^{n^2 p / 2} 6^{\log_2 n + 1},
\end{align}
as desired.
\end{proof}

% \how{take some time to ensure all the different layers of t/r hold vs tau eq 16}

Now, we seek to finalize the number of ops required in terms of $\epsilon$. Recognize that we may use timeslicing to reduce the error arbitrarily. Note that:
\begin{lemma}\label{r-scaling}
Suppose we may implement $\widetilde{\mathcal{T}}_{k, p}(t)$, an approximation of $\mathcal{T}_k(t)$ with $p > 1$ such that:
\begin{align}
    \norm{\mathcal{T}_k(t) - \widetilde{\mathcal{T}}_{k, p}(t)} \in \mathcal{O}((\Lambda^{k / 2} t)^p).
\end{align}
Then, by timeslicing the approximation, we can produce $\widetilde{\mathcal{T}}_{k, p}^r(t)$ where:
\begin{align}
    \norm{\mathcal{T}_k(t) - \widetilde{\mathcal{T}}_{k, p}^r(t)} \leq \epsilon,
\end{align}
where $\widetilde{\mathcal{T}}_k^r(t)$ requires $r \in \Theta \left( \frac{(\Lambda^{k/2}t)^{1 + 1/(p - 1)}}{\epsilon^{1 / (p - 1)}} \right)$ applications of the $\widetilde{\mathcal{T}}_k(t)$ operator.
\end{lemma}
\begin{proof}
% $\widetilde{\mathcal{T}}_k(t)$ is implementable by \cref{thm:arb_calc}. 
% Recall by \cref{thm:arb_calc} we can implement the following operation:
% \begin{align}
%     \widetilde{\mathcal{T}}_k(t)= e^{i A_k t} + \Delta(t) t^{p + \frac{1}{2}}
% \end{align}
% With matrix error term $\norm{\Delta(t)} \in \mathcal{O}(1)$. 
We define the timeslicing of $\widetilde{T}_k(t)$ as applying $\widetilde{\mathcal{T}}_k(t/r)$ operator $r$ times:
\begin{align}
    \widetilde{\mathcal{T}}_{k, p}^r(t) = \widetilde{\mathcal{T}}_{k, p}(t / r)^r.
\end{align}
To find a Taylor expansion for $\widetilde{\mathcal{T}}_{k, p}(t / r)^r$, note the explicit form for $\widetilde{\mathcal{T}}_{k, p}(t)$:
\begin{align}
    \widetilde{\mathcal{T}}_{k, p}(t) = e^{i A_k t} + \Delta(t) (\Lambda^{k /2} t)^p,
\end{align}
where $\norm{\Delta(t)} \in \mathcal{O}(1)$.This allows us to express the refined operator as follows:
\begin{align}
    \widetilde{\mathcal{T}}_k(t/r)^r &= \left( e^{i A_k t / r} + \Delta\left( \frac{t}{r}\right) \left( \frac{t^{p}}{r^{p}} \right) \right)^r \\
    %%%%%
    &= e^{i A_k t} + \left[ \sum_{j = 1}^{r - 1} (e^{i A_k t / r})^{j} \Delta\left( \frac{t}{r} \right) (e^{i A_k t / r})^{r - 1 - j} \right] \left( \frac{(\Lambda^{k/2}t)^{p}}{r^{p}} \right) + \mathcal{O} \left(\left( \frac{\Lambda^{k/2} t}{r} \right)^{p + 1} \right),
\end{align}
Thus, when we analyze the error:
\begin{align}
    \norm{\mathcal{T}_k(t/r)^r - e^{i A_k t}} &\leq \mathcal{O} \left( r \norm{(e^{i A_k t /r})^{r - 1}} \norm{\Delta\left( \frac{t}{r} \right)} \left(  \frac{(\Lambda^{k/2}t)^{p}}{r^{p}} \right) \right) \nonumber\\
    &\subset \mathcal{O} \left( \frac{(\Lambda^{k/2}t)^{p}}{r^{p - 1}} \right).
\end{align}
To bound the implementation error by $\epsilon$, i.e. $\epsilon \in \mathcal{O} \left(  \frac{(\Lambda^{k/2}t)^{p}}{r^{p - 1}} \right)$, we should select $r$ as follows:
\begin{align}
    r \in \Theta \left( \frac{(\Lambda^{k/2}t)^{p / (p - 1)}}{\epsilon^{1 / (p - 1)}} \right) = \Theta \left( \frac{(\Lambda^{k/2}t)^{1 + 1/(p - 1)}}{\epsilon^{1 / (p - 1)}} \right),
\end{align}
as desired.
\end{proof}

% \how{restate cleanly the thm (2.1)}

% \how{fix below proof}

Finally, we can demonstrate our original theorem statement, allowing us to create a bound on the number of $\mathcal{S}_1$ operations necessary to achieve an arbitrarily ordered operator:
\begin{proof}[Proof of \cref{thm:main}]
By \cref{thm:arb_calc}, we can perform a single Trotter step of timestep $\frac{t}{r}$ using $n^{1.6} 30^{np} 420^{n^2 p / 2} 6^{\log_2 n + 1}$ $\mathcal{S}_1$ operations. Thus, the total number of $\mathcal{S}_1$ operations required scales in:
\begin{align}
    r \cdot n^{1.6} 30^{np} 420^{n^2 p / 2} 6^{\log_2 n + 1} ,
\end{align}
where, by \cref{r-scaling}, it is sufficient to set $r \in \Theta \left( \frac{(\Lambda^{k/2}t)^{1 + 1/(p - 1)}}{\epsilon^{1 / (p - 1)}} \right)$. 

\end{proof}

\subsection{Generation of nonlinear Hamiltonians}\label{apndx:error-jc}
% \ck{mention scaling with relation to max (kappa, omega)}

%\subsubsection{Combining the two terms via Trotter}
% We can combine the above results to implement our desired hamiltonian:

% Finally, by the timestepping analysis \cref{r-scaling}, we can obtain an epsilon bound:

% Thus, we can construct an approximation of arbitrary order. We conclude by proving an upper bound on the number of $\mathcal{S}_1$ exponentials required:
% \how{also need to verify tau}
% \how{clarify H}

% We can approximate an exponential of the following block-encoded Hamiltonian $H$:

% $\exp i t \begin{bmatrix}
%     H & \cdot \\
%     \cdot & \cdot
% \end{bmatrix}$
% \how{``H" the hamiltonian of 75}

% abstract: 
% \how{near linear simulation time}

% \how{compilation technique from gateset to hamiltonian simulation problem}

\resultjc
%\how{yes, this is the chi 3 model -- micheline, can you reword appropriately?}

\begin{proof}
We first show that the two Hamiltonian terms are implementable separately. Then, via Trotter, we combine them and perform an error analysis. In particular, we hope to embed the Hamiltonian such that we approximate the following operator:
\begin{align}
    \exp it \begin{bmatrix}
        H & 0 \\
        0 & \cdot
    \end{bmatrix},
\end{align}
i.e., where the Hamiltonian is embedded in the upper left hand block. Thus, when applied to a system with the $\ket{0}$ qubit, this amounts to implementing $\exp i t H $ on the mode.  

We begin by embedding the $a^\dagger a$ term. Notice that $a^\dagger a$ Hermitian; thus, we can apply \cref{alg:mult} on the $a, a^\dagger$ block encodings. By the error analysis in \cref{fact:bch} and the bound on norm from \cref{fact:norm-blockencodedxt}:
% \begin{align}
%     \norm{ \textrm{MULT}(\mathcal{B}_{a}(\tau), \mathcal{B}_{a^\dagger}(\tau)) - \exp i t \begin{bmatrix}
%         a^\dagger a & 0 \\
%         0 & a a^\dagger
%     \end{bmatrix} } \in \mathcal{O}( (C^2 t)^{\min (p_A, p_B) / 2}
% \end{align}
% Where $C = \max (\norm{}$
% \cref{fact:bch} 
\begin{align}
    \norm{\bch_q \left(\mathcal{S}_1^Y(\tau), \mathcal{S}_1(\tau) \right) - \exp 2 i \tau^2 \begin{bmatrix}
         a^\dagger   a  & 0 \\
        0 &  a    a^\dagger  
    \end{bmatrix} } \in \mathcal{O}((\Lambda^{1/2} \tau)^{2q + 1}),
\end{align}
so that, by setting $\tau = \sqrt{\frac{\omega t}{2}}$, we have: 
% \how{take omega not as constant}
\begin{align}
    \norm{\bch_q \left(\mathcal{S}_1^Y \left(\sqrt{\frac{\omega t}{2}} \right), \mathcal{S}_1\left(\sqrt{\frac{\omega t}{2}} \right) \right) - \exp i \omega t \begin{bmatrix}
         a^\dagger   a  & 0 \\
        0 &  a    a^\dagger  
    \end{bmatrix} } \in \mathcal{O}((\Lambda \omega t)^{q + \frac{1}{2}}),
\end{align}
using $8 \cdot 6^{q - 1}$ total $\mathcal{S}_1$ operations.

Recall that we can easily block-encode $( a^\dagger )^2$ and $ ( a )^2$ via \cref{alg:power_of_two}. We then apply \cref{alg:mult} to $(a^\dagger)^2, a^2$ to yield the desired upper-left block encoding. Namely, because we can implement $\mathcal{S}_{2, p}$ with the following error scaling:
% via the formulas described above. Thus, we can repeat a similar approach as above and instead use \cref{alg:mult} on the approximated operators to produce a block-encoded $( a^\dagger )^2 ( a )^2$. By \cref{corr:second_order} and \cref{lem:multiplication-alg},
\begin{align}
    \norm{\mathcal{S}_{2, p}(\tau) - \exp i \tau \begin{bmatrix}
        0 & ( a^\dagger)^2 \\
        a ^2  & 0
    \end{bmatrix}} \in \mathcal{O}((\Lambda \tau)^{p + \frac{1}{2}}),
\end{align}
using no more than $6 \cdot 14^p$ $\mathcal{S}_1$ operations, we can apply \cref{alg:mult} to find:
% We then define $\mathcal{S}_{2, p}^Y(\tau) = (S \kron \identity_\gamma) \mathcal{S}_{2, p}^Y(\tau) (S^\dagger \kron \identity_\gamma)$ and apply \cref{lem:bch-with-approx}:
\begin{align}
    \norm{\textrm{MULT} (\mathcal{S}_{2, p}(\tau), X \mathcal{S}_{2, p}(\tau) X ) - \exp 2 i \tau^2 \begin{bmatrix}
        ( a^\dagger )^2 ( a )^2 & 0 \\
        0 & ( a )^2 ( a^\dagger )^2
    \end{bmatrix} } \in \mathcal{O}((\Lambda^2 \tau)^{p + \frac{1}{2}}),
\end{align}
% \how{i don't like u}
% \how{we need to check the error of this formula because now the constant factor is larger, i believe}
by setting $\ell = \ceil{\frac{p - \frac{1}{2}}{2}} \leq \frac{p}{2} + 1 $ and thus using $8 \cdot 6^{\ell - 1}$ exponentials. When $\tau = \sqrt{\frac{\kappa t}{4}}$: 
% \how{take kappa not as constant}
\begin{align}
    \norm{\bch_\ell \left(\mathcal{S}_{2, p}^Y\left(\sqrt{\frac{\kappa t}{4}} \right), \mathcal{S}_{2, p}\left(\sqrt{\frac{\kappa t}{4}} \right) \right) - \exp i \frac{\kappa}{2} t  \begin{bmatrix}
        ( a^\dagger )^2 ( a )^2 & 0 \\
        0 & ( a )^2 ( a^\dagger )^2
    \end{bmatrix} } \in \mathcal{O}( (\Lambda^4 \kappa t)^{\frac{p}{2} + \frac{1}{4}}),
\end{align}
using no more than $8 \cdot 6^{\ell - 1} \cdot 6 \cdot 14^p \leq 48 \cdot 6^{p / 2} \cdot 14^{p} \leq 48 \cdot 35^p$ total $\mathcal{S}_1$ operators. Then, we may set $p = 2q$ so that, given no more than $48 \cdot 35^{2q}$ total $\mathcal{S}_1$ operations, we can implement the BCH formula with error scaling $\mathcal{O}((\Lambda^4 \kappa t)^{q + \frac{1}{4}})$.

Finally, we apply the Trotter formula to the two subterms via \cref{lem:trotter-with-approx}, finding:
\begin{align}
    & \Biggl\lVert
    \trotter_{2s} 
        \Biggl(\bch_q 
            \Biggl(\mathcal{S}_X^Y
                \Biggl(\sqrt{\frac{\omega t}{2}}\Biggr), 
            \mathcal{S}_X
                \left(\sqrt{\frac{\omega t}{2}})\right) 
            \Biggr) , 
    \bch_u 
            \left(\mathcal{S}_{2, p}^Y
                \left(\sqrt{\frac{\kappa t}{4}}\right), 
            \mathcal{S}_{2, p}
                \left(\sqrt{\frac{\kappa t}{4}})\right) 
            \right) 
        \Biggr) \nonumber \\
        &\qquad - \exp i t \begin{bmatrix}
        H & 0 \\
        0 & \cdot
    \end{bmatrix} 
    \Biggr\rVert  \\
    &\qquad \in \mathcal{O}((\Lambda^4 \max( \omega, \kappa) t)^{q + \frac{1}{4}}),
\end{align}
% \how{recheck}
where the constant factor can be obtained by observing the Hamiltonian norm is bounded via the triangle inequality. Now, by setting $s = \ceil{\frac{1}{2} (q - \frac{3}{4})} \leq \frac{q}{2} + \frac{5}{8}$, we can obtain the desired error scaling. This formula requires no more than $4 \cdot 5^{s - 1} $ total operations; by symmetry, we can assume each of the BCH formulas only must be applied $2 \cdot 5^{s - 1}$ times. Thus, the total number of $\mathcal{S}_1$ operations is no more than:
\begin{align}
    2 \cdot 5^{s - 1} (8 \cdot 6^{q - 1} + 48 \cdot 35^{2q}) \leq 4 \cdot 5^{\frac{q}{2}} \cdot 48 \cdot 35^{2q} \leq 192 \cdot 2900^q.
\end{align}
% \how{is this good enough lol}

% \how{move 82, 83 above eq into proof of thm 3.1}
% Begin first by assessing the number of $\mathcal{S}_1$ operations necessary to implement the Trotterized operator. From \cref{lem:trotter-with-approx}, this requires at most $5^{s - 1} + 1$ of each of the exponentials. Thus, the total cost is:
% \begin{align}
%     (5^{s -1 } + 1)(8 \cdot 6^{q - 1} + 8 \cdot 6^{p/2 + 1} \mathcal{O}(e^{2.597p})) \subset \mathcal{O}(5^{q/ 2} \cdot 6^q e^{2.597 \cdot 2 q}) \subset \mathcal{O}(e^{7.8 q})
% \end{align}

To produce an $\epsilon$ scaling, we apply \cref{r-scaling} to the Trotterized operator, implying that we require the following $r$ scaling for fixed $q$:
\begin{align}
    r \in \Omega\left( \frac{(\Lambda^{4} t)^{1 + 1 / (q - \frac{3}{4})} }{\epsilon^{1 / (q - \frac{3}{4})}} \right),
\end{align}
where the total number of $\mathcal{S}_1$ operations is no more than:
\begin{align}
    r \cdot 192 \cdot 2900^q \subset re^{\mathcal{O}(q)}.
\end{align}

\end{proof}

\subsection{Application to state preparation}\label{subsec:state-prep}
We first need to demonstrate the connection between block-encoded powers of annihilation/creation operators and state preparation. First, observe that the ideal block encoding would allow for initialization from the vacuum:
\statepreptime
\begin{proof}\label{state_prep_proof}
    The proof is algebraic; begin by producing the Taylor series expansion of the operator:
    \begin{align}
        \mathcal{T}_{k}(t) &= \exp \left( it \begin{bmatrix}
            0 & ( a^\dagger )^k \\
            ( a )^k & 0
        \end{bmatrix} \right) \\
        &\qquad= \sum_{j = 0}^\infty \frac{(it)^j}{j!} \begin{bmatrix}
            0 & ( a^\dagger )^k \\
            ( a )^k & 0
        \end{bmatrix}^j \\ 
        &\qquad= \sum_{j = 0}^\infty \frac{(it)^{2j}}{(2j)!} \begin{bmatrix}
            0 & ( a^\dagger )^k \\
            ( a )^k & 0
        \end{bmatrix}^{2j}
        + \sum_{j = 0}^\infty \frac{(it)^{2j + 1}}{(2j + 1)!} \begin{bmatrix}
            0 & ( a^\dagger )^k \\
            ( a )^k & 0
        \end{bmatrix}^{2j + 1},
    \end{align}
where the matrix products have well-defined forms:
    \begin{align}
        % \begin{bmatrix}
        %     0 & ( a^\dagger )^k \\
        %     ( a )^k & 0
        % \end{bmatrix}^{2} &= \begin{bmatrix}
        %     ( a^\dagger )^k ( a )^k & 0 \\
        %     0 & ( a )^k ( a^\dagger )^k 
        % \end{bmatrix} \\
        \begin{bmatrix}
            0 & ( a^\dagger )^k \\
            ( a )^k & 0
        \end{bmatrix}^{2j} &= \begin{bmatrix}
            (( a^\dagger )^k ( a )^k)^j & 0 \\
            0 & (( a )^k ( a^\dagger )^k)^j 
        \end{bmatrix}, \\
        \begin{bmatrix}
            0 & ( a^\dagger )^k \\
            ( a )^k & 0
        \end{bmatrix}^{2j + 1} &= \begin{bmatrix}
             0 & ( a^\dagger )^k (( a )^k ( a^\dagger )^k)^j \\
            ( a )^k (( a^\dagger )^k ( a )^k)^j & 0 
        \end{bmatrix},
    \end{align}
so that:
\begin{align}
    &\exp \left( it \begin{bmatrix}
            0 & ( a^\dagger )^k \\
            ( a )^k & 0
        \end{bmatrix} \right) \ket{1} \kron \ket{0} \\
        &\qquad = \sum_{j = 0}^\infty \frac{(it)^{2j}}{(2j)!} \sqrt{k!}^{2j} \ket{1} \kron \ket{0} + \sum_{j = 0}^\infty \frac{(it)^{2j + 1}}{(2j + 1)!} \sqrt{k!}^{2j + 1}  \ket{0} \kron \ket{k} \\
        &\qquad= \cos (t \sqrt{k!}) \ket{1} \kron \ket{0} + i \sin (t \sqrt{k!}) \ket{0} \kron \ket{k}.
\end{align}
When $t \sqrt{k!} = (2n + 1) \frac{\pi}{2}$ for $n \in \mathbb{N}$, the $\ket{1} \kron \ket{0}$ term vanishes and we are left with the $\ket{0} \kron \ket{k}$ Fock state, as desired.
\end{proof}

% This technique can then be extended to precisely perform $\ket{1} \ket{0} \mapsto \ket{0} \ket{k}$ while leaving all initial states intact (i.e., $\ket{1} \ket{k} \mapsto \ket{1} \ket{k} $ for $k \neq 1$. We claim:


While this result allows us to prepare the $\ket{k}$ Fock state, it also will incur unwanted transformations on starting states other than the vacuum ($\ket{1} \kron \ket{n}, n \neq 1$). By applying the BCH formula, we can isolate this operation so that it only operates on the $\ket{1} \kron \ket{0}$ term. In particular, we argue:
\resultstateprep
\begin{proof}
The general construction of the operator emerges from the use of a Trotter formula in conjunction with a phase rotation gate. Begin by defining the rotation operator:
\begin{define}
Call $R_{Z0}$ the phase-flip operator acting on some set of modes $B$ to be:
\begin{align}
    R_{Z0} := \identity \otimes (\identity - 2 \ket{0}\!\bra{0}),
\end{align}
i.e., only flip the phase for the vacuum. This operator is implementable using a 0-controlled cavity-conditioned qubit rotation gate. 

    % Define the phase-flip operator acting on some set of modes $B$ to be:
    % \begin{align}
    %     R_Z^B \ket{q} \ket{b} \mapsto (-1)^{\mathbb{I}[b \in B]} \ket{q} \ket{b}
    % \end{align}
    % I.e. only flip the phase for the modes in $S$. This has the following matrix form:
    % \begin{align}
    %     \exp \left(i \frac{\pi}{2} \begin{bmatrix}
    %     \diag(\mathcal{B}) & 0 \\
    %     0 & \diag(\mathcal{B})
    %     \end{bmatrix} \right)
    % \end{align}
    % Where $\mathcal{B} = [x \in B  : x \in [0..\Lambda]]$. \how{why is this realizable}
\end{define}
Then, because $R_{Z0}$ is self-adjoint, we can conjugate $\mathcal{T}_k(-t)$ without error as follows: 
\begin{align}
    R_{Z0} \mathcal{T}_k (-t) R_{Z0} &= R_{Z0} \exp \left( it' \begin{bmatrix}
            0 & ( a^\dagger )^k \\
            ( a )^k & 0
        \end{bmatrix} \right) R_{Z0} \\
        &= \exp \left( - it R_{Z0} \begin{bmatrix}
            0 & ( a^\dagger )^k \\
            ( a )^k & 0
        \end{bmatrix} R_{Z0} \right) \\
        &= \exp \left( it  \begin{bmatrix}
            0 & ( a^\dagger )^k (2 \ket{0}\!\bra{0} - \identity) \\
            (2 \ket{0}\!\bra{0} - \identity) ( a )^k  & 0
        \end{bmatrix} \right),
\end{align}
% Finally, label the operator $\check{\mathcal{T}_k}(t)$ set $t' = -t$ so that:
% \begin{align}
%      \check{\mathcal{T}_k}(t) &= R_{Z0} \exp \left( - it \begin{bmatrix}
%             0 & ( a^\dagger )^k \\
%             ( a )^k & 0
%         \end{bmatrix} \right) R_{Z0} \\
%         &= \exp \left( it \begin{bmatrix}
%             0 & -( a^\dagger )^k R_{Z0} \\
%             -R_{Z0} ( a )^k & 0
%         \end{bmatrix} \right)
% \end{align}
where specific left- and right-hand $(2 \ket{0}\!\bra{0} - \identity)$ terms vanish given the annihilation/creation operators. We then apply the Trotter formula upon $\mathcal{T}_k(t/2), R_{Z0} \mathcal{T}_k(-t/2) R_{Z0}$, yielding:
\begin{align}
    \exp \left( i t \begin{bmatrix}
        0 & ( a^\dagger )^k \ketbra{0}{0} \\
        \ketbra{0}{0} ( a )^k & 0
    \end{bmatrix} \right),
\end{align}
as desired.

% \begin{align}
%      \exp \left( it \begin{bmatrix}
%         0 & p_{k, 1} \\
%         p_{1, k} & 0 
%     \end{bmatrix} \right) = \exp \left( i \frac{t}{2} \begin{bmatrix}
%         0 & ( a^\dagger )^k (\identity - R_{Z0} ) \\
%         (\identity - R_{Z0}) ( a )^k & 0
%     \end{bmatrix} \right) 
% \end{align}

To compute the error scaling, recall our result from \cref{thm:arb_calc} which states the error scaling of $\widetilde{\mathcal{T}}_{k, p}$ (and, respectively, $R_{Z0} \widetilde{\mathcal{T}}_{k, p} R_{Z0}$):
\begin{align}
    \norm{\widetilde{\mathcal{T}}_{k, p}(t) - \mathcal{T}_k (t)} \in \mathcal{O}((\Lambda^{k/2} t)^p) ,
\end{align}
so that, by $\cref{lem:trotter-with-approx}$, 
\begin{align}
    \norm{\trotter_{2q}(\widetilde{\mathcal{T}}_{k, p}(t/2), R_{Z0} \widetilde{\mathcal{T}}_{k, p}(-t/2) R_{Z0}) - \exp \left( i t \begin{bmatrix}
        0 & ( a^\dagger )^k \ketbra{0}{0} \\
        \ketbra{0}{0} ( a )^k & 0
    \end{bmatrix} \right) } \in \mathcal{O}((\Lambda^{k/2} t)^{p}),
\end{align}
when $q = \max(\ceil{\frac{p - 1}{2}}, 1 ) $ and using no more than $4 \cdot 5^{q-1}$ operator exponentials. Thus, we can set $q = \frac{p + 1}{2} \geq \max(\ceil{\frac{p - 1}{2}}, 1 ) $ so that we use no more than $ 4 \cdot 5^{\frac{p - 1}{2}} \leq 2 \cdot 5^{p/2}$ $\widetilde{\mathcal{T}}_{k, p}$ terms.
% Where the even-$j$ terms converge to:
% \begin{align}
%     \begin{bmatrix}
%         0 & ( a^\dagger )^k (\identity - R_{Z0} ) \\
%         (\identity - R_{Z0}) ( a )^k & 0
%     \end{bmatrix}^{2j} = \\ \begin{bmatrix}
%         ( a^\dagger )^k (\identity - R_{Z0} )^2 ( a )^k & 0 \\
%         0 & (\identity - R_{Z0}) ( a )^k ( a^\dagger )^k (\identity - R_{Z0} ) 
%     \end{bmatrix}^j
% \end{align}

\end{proof}

% Observe that the commutator of the block-encoding and $R_Z^B$ takes the following form: 
% \begin{fact}
%     \begin{align}
%         &\left[ it \begin{bmatrix}
%             0 & ( a^\dagger )^k \\
%             ( a )^k & 0 
%         \end{bmatrix}, i \frac{\pi}{2} \begin{bmatrix}
%         \diag(\mathcal{B}) & 0 \\
%         0 & \diag(\mathcal{B})
%         \end{bmatrix} \right] \\
%         &\qquad= - t \frac{\pi}{2} \begin{bmatrix}
%             0 & ( a^\dagger )^k \diag(\mathcal{B}) - \diag(\mathcal{B}) ( a^\dagger )^k \\
%             ( a )^k \diag(\mathcal{B}) - \diag(\mathcal{B})( a )^k & 0
%         \end{bmatrix}
%     \end{align}
% \end{fact}

% \how{include surgical lemma}

% \ck{Givens rotations?}

Thus, our approximate operators can be applied to yield the same result with high probability:
\begin{theorem}
We can prepare the $\ket{0} \kron \ket{k}$ with probability at least $ 1- \delta$ using no more than $r \in \Theta \left( \frac{(\Lambda^{k/2}t)^{1 + 1/(p - 1)}}{(\delta / 2)^{1 / (p - 1)}} \right)$ $\mathcal{F}_{k, p}$ operators or at most
\begin{align}
    r \cdot 2 \cdot 5^{p/2} \cdot n^{1.6} 30^{np} 420^{n^2 p / 2} 6^{\log_2 n + 1} 
\end{align}
$\mathcal{S}_1$ operators.

\end{theorem}

\begin{proof}
Begin by identifying the $\epsilon$ precision necessary to yield a failure probability less than $\delta$. A sufficient condition would be that:
\begin{align}
    \left| \norm{\ketbra{0}{0} \kron \ketbra{k}{k} \stateprepapprox \ket{1} \kron \ket{0}}^2 - \norm{\ketbra{0}{0} \kron \ketbra{k}{k} \stateprep \ket{1} \kron \ket{0}}^2 \right| \leq \delta.
\end{align}
% \begin{align}
%     | \bra{1} \kron \bra{0} \mathcal{F}_{k, p}^\dagger \ketbra{0}{0} \kron \ketbra{k}{k} \mathcal{F}_{k, p} \ket{1} \kron \ket{0} - \bra{1} \kron \bra{0} \stateprep^\dagger \ketbra{0}{0} \kron \ketbra{k}{k} \stateprep \ket{1} \kron \ket{0} | \leq \delta 
% \end{align}
Observe that our idealized operator has a success probability; thus, we seek to demonstrate that:
\begin{align}
    \left|\norm{\ketbra{0}{0} \kron \ketbra{k}{k} \stateprepapprox \ket{1} \kron \ket{0}}^2 - 1\right| \leq \delta .
\end{align}
Because the probability of measuring $\ket{0} \kron \ket{k}$ lies in $[0, 1]$, the above inequality holds when:
\begin{align}
    \norm{\ketbra{0}{0} \kron \ketbra{k}{k} \stateprepapprox \ket{1} \kron \ket{0}}^2 \geq 1 - \delta.
\end{align}
% \begin{align}
%     \bra{1} \kron \bra{0} \mathcal{F}_{k, p}^\dagger \ketbra{0}{0} \kron \ketbra{k}{k} \mathcal{F}_{k, p} \ket{1} \kron \ket{0} \geq 1 - \delta
% \end{align}
Recognize that we can lower bound the norm:
\begin{align}
    &\norm{\ketbra{0}{0} \kron \ketbra{k}{k} \stateprepapprox \ket{1} \kron \ket{0}}\\
    &\qquad= \norm{\ketbra{0}{0} \kron \ketbra{k}{k} (\stateprep - (\stateprep - \stateprepapprox)) \ket{1} \kron \ket{0}} \\
    &\qquad\geq \left| \norm{\ketbra{0}{0} \kron \ketbra{k}{k} \mathcal{F}_{k} \ket{1} \kron \ket{0}} - \norm{\ketbra{0}{0} \kron \ketbra{k}{k} (\stateprep - \stateprepapprox) \ket{1} \kron \ket{0}} \right| \\
    &\qquad\geq 1 - \norm{\stateprep - \stateprepapprox}_\infty.
\end{align}
By requiring $\norm{\stateprep - \stateprepapprox}_\infty \leq 1$. This allows us to produce a lower bound on the original LHS:
\begin{align}
    \norm{\ketbra{0}{0} \kron \ketbra{k}{k} \stateprepapprox \ket{1} \kron \ket{0}}^2 \geq 1 - 2 \norm{\stateprep - \stateprepapprox}_\infty.
\end{align}
Thus, it is sufficient for the following to hold:
\begin{align}
    1 - 2 \norm{\stateprep - \stateprepapprox}_\infty \geq 1 - \delta \iff \norm{\stateprep - \stateprepapprox}_\infty \leq \frac{\delta}{2}.
\end{align}
Apply \cref{r-scaling} to \cref{lem:fock-prep-unitary} so that the time-sliced $\stateprepapprox^r$ has:
\begin{align}
    \norm{\stateprepapprox^r - \stateprep} \leq \frac{\delta}{2},
\end{align}
by using $r \in \Theta \left( \frac{(\Lambda^{k/2}t)^{1 + 1/(p - 1)}}{(\delta / 2)^{1 / (p - 1)}} \right)$ applications of $\stateprepapprox(t/r)$. The $\mathcal{S}_1$ bound follows from a similar analysis to \cref{thm:main} applied to the result from \cref{lem:fock-prep-unitary}.

\end{proof}




\section{Universal Control of the \texorpdfstring{Span $\left\{ \left|0\right\rangle ,\left|1\right\rangle \right\} $}{}
Fock Space\label{subsec:Universal-Control}}

To demonstrate the efficacy of the instruction set, we demonstrate the
use of the approach to encode a qubit in a cavity either via generation
of effective Pauli gates 
or imposition of an effective Hubbard interaction in the Jaynes-Cumming
Hamiltonian.  In this sense, the techniques presented here are analogous to those in~\cite{liu2021constructing}, in that we  use our results to effectively truncate the quantum information to a two-dimensional subspace despite the fact that the natural dynamics of the systems causes the quantum information to leak from this space into the larger Hilbert space of the cavity.

%\subsection{Effective Pauli Gate Approach\label{subsec:Effective-Pauli-Gate}}

For universal control in the restricted $\text{span}\left\{ \left|0\right\rangle ,\left|1\right\rangle \right\} $
Hilbert space, we generate three effective Pauli operators $\sigma_{\text{eff}}^{x}$,
$\sigma_{\text{eff}}^{y}$, and $\sigma_{\text{eff}}^{z}$ that produce
Pauli rotations in the lowest two modes of the cavity, with minimal leakage
to higher energy states. The form of the effective Pauli operators
is determined by expressing the standard Pauli operators
\begin{align}
\sigma^{x} & =\left(\begin{array}{cc}
0 & 1\\
1 & 0
\end{array}\right),\\
\sigma^{y} & =\left(\begin{array}{cc}
0 & -i\\
i & 0
\end{array}\right),\\
\sigma^{z} & =\left(\begin{array}{cc}
1 & 0\\
0 & -1
\end{array}\right),
\end{align}
in terms of creation and annihilation operators truncated to the first
two Fock states
\begin{align}
\hat{a}_{\text{eff}}^{\dagger} & =\left(\begin{array}{cc}
0 & 0\\
1 & 0
\end{array}\right),\\
\hat{a}_{\text{eff}} & =\left(\begin{array}{cc}
0 & 1\\
0 & 0
\end{array}\right),\\
\hat{n}_{\text{eff}} & =\hat{a}^{\dagger}\hat{a}_{\text{eff}}=\left(\begin{array}{cc}
0 & 0\\
0 & 1
\end{array}\right),
\end{align}
which yields
\begin{align}
\sigma_{\text{eff}}^{x} & =\hat{a}_{\text{eff}}^{\dagger}+\hat{a}_{\text{eff}},\\
\sigma_{\text{eff}}^{y} & =i\left(\hat{a}_{\text{eff}}^{\dagger}-\hat{a}_{\text{eff}}\right),\\
\sigma_{\text{eff}}^{z} & =I-2\hat{a}_{\text{eff}}^{\dagger}\hat{a}_{\text{eff}}.
\end{align}
To reduce leakage into higher energy states, we ensure the creation
operator $\hat{a}_{\text{eff}}^{\dagger}$ only acts on the ground
state $\left|0\right>$ and the annihilation operator $\hat{a}_{\text{eff}}$
only acts on the first excited state $\left|1\right>$ with the projector
\begin{align}
\hat{P}_{0} & \approx I-\hat{n}\\
 & =\begin{cases}
0 & n=1\\
1 & n=0
\end{cases},
\end{align}
where $n$ is the number of photons in the cavity and where only the
$\text{span}\left\{ \left|0\right\rangle ,\left|1\right\rangle \right\} $
states are populated. Since the operator is a projector, it obeys
the relation
\begin{equation}
\hat{P}_{0}^{2}=\hat{P}_{0},
\end{equation}
such that the effective Pauli gates are
\begin{align}
\sigma_{\text{eff}}^{x} & =\hat{a}_{\text{eff}}^{\dagger}\hat{P}_{0}+\hat{P}_{0}\hat{a}_{\text{eff}}\\
 & \approx\hat{a}^{\dagger}\left(I-\hat{n}\right)+\left(I-\hat{n}\right)\hat{a},\\
\sigma_{\text{eff}}^{y} & =i\left(\hat{a}_{\text{eff}}^{\dagger}\hat{P}_{0}-\hat{P}_{0}\hat{a}_{\text{eff}}\right)\\
 & \approx i\left(\hat{a}^{\dagger}\left(I-\hat{n}\right)-\left(I-\hat{n}\right)\hat{a}\right),\\
\sigma_{\text{eff}}^{z} & =I-2\hat{a}_{\text{eff}}^{\dagger}\hat{P}_{0}^{2}\hat{a}_{\text{eff}}\\
 & \approx I-2\hat{a}^{\dagger}\left(I-\hat{n}\right)\hat{a}.
\end{align}


\subsubsection*{Pauli X Gate}

Consider the infinitesimal $\sigma_{x}$-rotation gate in the $\text{span}\left\{ \left|0\right\rangle ,\left|1\right\rangle \right\} $
Fock space 
\begin{align}
U_{\text{span}\left\{ 0,1\right\} ,x} & =e^{i\lambda^{2}\sigma_{\text{eff}}^{x}\sigma^{z}}\\
 & =e^{i\lambda^{2}\left(\hat{a}^{\dagger}\left(1-\hat{n}\right)+\left(1-\hat{n}\right)\hat{a}\right)\sigma^{z}}.
\end{align}
Expression of the exponent in terms of phase-space operators Eq.~\ref{eq:NumbertoPhaseSpace},
Eq.~\ref{eq:CreationOperatortoPhaseSpace}, and Eq.~\ref{eq:AnnihilationOperatortoPhaseSpace}
gives 
\begin{gather}
i\lambda^{2}\left(\hat{a}^{\dagger}\left(I-\hat{n}\right)+\left(I-\hat{n}\right)\hat{a}\right)\sigma^{z}\nonumber \\
=i\lambda^{2}\left(2\hat{x}-\left\{ \hat{x},\hat{n}\right\} +i\left[\hat{p},\hat{n}\right]\right)\sigma^{z}.
\end{gather}
The gate is therefore given by a Trotter-Suzuki decomposition 
% Eq.~\ref{eq:TrotterFirstOrder}
of three terms: $\exp\left(\left[\hat{A}_{1},\hat{B}_{1}\right]\lambda^{2}\right)=\exp\left(-i\lambda^{2}\left\{ \hat{x},\hat{n}\right\} \sigma^{z}\right)$,
$\exp\left(\left[\hat{A}_{2},\hat{B}_{2}\right]\lambda^{2}\right)=\exp\left(-\lambda^{2}\left[\hat{p},\hat{n}\right]\sigma^{z}\right)$,
and $\exp\left(2i\lambda^{2}\hat{x}\sigma^{z}\right)$. 

The terms consisting of exponentials of commutators are decomposed via BCH. The relationship between the commutator and anticommutator required for the first term is given by the Pauli anticommutation-commutation relation
\begin{align}
-i\left\{ \hat{x},\hat{n}\right\} \sigma^{z} & =-i\left(i\left[i\hat{x}\sigma^{x},i\hat{n}\sigma^{y}\right]\right)\\
 & =\left[i\hat{x}\sigma^{x},i\hat{n}\sigma^{y}\right]\\
 & =\left[\hat{A}_{1},\hat{B}_{1}\right],
\end{align}
where $\hat{A}_{1}$ corresponds to a position displacement and $\hat{B}_{1}$
corresponds to the $y$-conditional rotation gate. The argument of
the second term is already in the form of a commutator, such that
\begin{align}
\left[\hat{A}_{2},\hat{B}_{2}\right] & =-\left[\hat{p},\hat{n}\right]\sigma^{z}\\
 & =\left[i\hat{p},i\hat{n}\sigma^{z}\right],
\end{align}
where $\hat{A}_{2}$ corresponds to an \emph{unconditional} momentum boost,
and $\hat{B}_{2}$ corresponds to the $z$-conditional rotation gate.
Lastly, the third term already belongs to the instruction set architecture
and needs no further decomposition. 

The infinitesimal $\sigma_{x}$-rotation gate in the $\text{span}\left\{ \left|0\right\rangle ,\left|1\right\rangle \right\} $
Fock space is therefore composed of a product of nine rotation and displacement gates or 21 displacement gates.

\subsubsection*{Pauli Y Gate}

The infinitesimal $\sigma_{y}$-rotation gate in the $\text{span}\left\{ \left|0\right\rangle ,\left|1\right\rangle \right\} $
Fock space is determined analogously 
\begin{align}
U_{\text{span}\left\{ 0,1\right\} ,y} & =e^{i\lambda^{2}\sigma_{\text{eff}}^{y}\sigma^{z}}\\
 & =e^{-\lambda^{2}\left(\hat{a}^{\dagger}\left(I-\hat{n}\right)+\left(I-\hat{n}\right)\hat{a}\right)\sigma^{z}}.
\end{align}
Expression of the argument of the exponent in terms of phase-space
variables Eq.~\ref{eq:CreationOperatortoPhaseSpace} and Eq.~\ref{eq:AnnihilationOperatortoPhaseSpace}
yields 
\begin{gather}
-\lambda^{2}\left(\hat{a}^{\dagger}\left(I-\hat{n}\right)-\left(I-\hat{n}\right)\hat{a}\right)\sigma^{z}\nonumber \\
=-\lambda^{2}\left(-2i\hat{p}+\left[\hat{n},\hat{x}\right]+i\left\{ \hat{n},\hat{p}\right\} \right)\sigma^{z},
\end{gather}
such that the gate is a Trotter-Suzuki decomposition 
% Eq.~\ref{eq:TrotterFirstOrder}
of $\exp\left(\left[\hat{A}_{1},\hat{B}_{1}\right]\lambda^{2}\right)=\exp\left(-\lambda^{2}\left[n,x\right]\sigma^{z}\right)$,
$\exp\left(\left[\hat{A}_{2},\hat{B}_{2}\lambda^{2}\right]\right)=\exp\left(-i\lambda^{2}\left\{ p,n\right\} \sigma^{z}\right)$,
and $\exp\left(2i\lambda^{2}p\sigma^{z}\right)$. 

Again, the first two exponential terms are decomposed via the BCH
formula 
% Eq.~\ref{eq:BCHFormula}
. The first commutator is 
\begin{align}
\left[\hat{A}_{1},\hat{B}_{1}\right] & =\left[\hat{n},\hat{x}\right]\sigma^{z}\\
 & =\left[\hat{n}\sigma^{z},\hat{x}\right],
\end{align}
where the exponent of $\hat{A}_{1}$ is a $z$-conditional rotation
gate and the exponent of $\hat{B}_{1}$ is an unconditional position
displacement. The second commutator is given by the Pauli anticommutation-commutation
relation:
\begin{align}
i\left\{ p,n\right\} \sigma^{z} & =i\left(i\left[ip\sigma^{x},in\sigma^{y}\right]\right)\\
 & =\left[ip\sigma^{x},in\sigma^{y}\right]\\
 & =\left[\hat{A}_{2},\hat{B}_{2}\right],
\end{align}
where the exponent of $\hat{A}_{2}$ corresponds to a conditional
momentum shift and the exponent of $\hat{B}_{2}$ is a $y$-conditional
rotation gate.

The infinitesimal $\sigma_{y}$-rotation gate in the $\text{span}\left\{ \left|0\right\rangle ,\left|1\right\rangle \right\} $
Fock space therefore has a lower bound gate depth of nine displacement and rotation gates or 21 in displacement gates.

\subsubsection*{Pauli Z Gate}

The infinitesimal $\sigma_{z}$-rotation gate in the $\text{span}\left\{ \left|0\right\rangle ,\left|1\right\rangle \right\} $
Fock space is 
\begin{align}
U_{\text{span}\left\{ 0,1\right\} ,z} & =e^{i\lambda^{2}\sigma_{\text{eff}}^{z}\sigma^{z}}\\
 & =e^{-\lambda^{2}\left(I-2\hat{a}^{\dagger}\left(I-\hat{n}\right)\hat{a}\right)\sigma^{z}},
\end{align}
whose argument in terms of ladder operators is
\begin{gather}
-\lambda^{2}\left(I-2\hat{a}^{\dagger}\left(I-\hat{n}\right)\hat{a}\right)\sigma^{z}\nonumber \\
=-\lambda^{2}\left(I-2\hat{a}^{\dagger}a+2\hat{a}^{\dagger}\hat{a}^{\dagger}\hat{a}\hat{a}\right)\sigma^{z},
\end{gather}
Given the ladder operator commutator Eq.~\ref{eq:CommutatorCreationAnnihilation},
\begin{equation}
\hat{a}^{\dagger}\hat{a}=\hat{a}\hat{a}^{\dagger}-I,
\end{equation}
the relationship between the fourth-order ladder operator term and
the number operator is
\begin{align}
\hat{a}^{\dagger}\hat{a}^{\dagger}\hat{a}\hat{a} & =\hat{a}^{\dagger}\left(\hat{a}\hat{a}^{\dagger}-I\right)\hat{a}\\
 & =\hat{a}^{\dagger}\hat{a}\hat{a}^{\dagger}\hat{a}-\hat{a}^{\dagger}\hat{a}\\
 & =\hat{n}^{2}-\hat{n}.
\end{align}
The argument of the exponential in terms of number operators is then
\begin{equation}
-\lambda^{2}\left(I-2\hat{a}^{\dagger}\left(I-\hat{n}\right)\hat{a}\right)\sigma^{z}=-\lambda^{2}\left(I-4\hat{n}+2\hat{n}^{2}\right)\sigma^{z}.
\end{equation}
The argument is further simplified given that the state is restricted
to the first two cavity modes, as for $n=0$ and $n=1$ the quantity
$\hat{n}^{2}-\hat{n}$ is zero, as follows:
\begin{gather}
-\lambda^{2}\left(I-4\hat{n}+2\hat{n}^{2}\right)\sigma^{z}\nonumber \\
=-\lambda^{2}\left(I-2n\right)\sigma^{z}.
\end{gather}
The gate is therefore directly synthesized as the product of the qubit
rotation gate $\exp\left(-\lambda^{2}\sigma^{z}\right)$ and the $z$-conditional
rotation gate $\exp\left(2\lambda^{2}\hat{n}\sigma^{z}\right)$ for
a lower bound gate depth of two.

\subsection{Effective Hubbard-lattice interaction approach\label{subsec:Effective-Hubbard-Lattice}}

An alternative scheme to encode a qubit in a cavity with the instruction set is to map the three-dimensional quantum electrodynamics
(3D cQED) system to a qubit by imposing an $\hat{n}\left(\hat{n}-1\right)$
anharmonicity into the Jaynes-Cummings Hamiltonian that describes
the system. The anharmonicity term increases the energy gap between
higher levels of the oscillator to effectively restrict propagation
to the $\text{span}\left\{ \left|0\right\rangle ,\left|1\right\rangle \right\} $
Fock space in which there is universal control.

Consider the standard Jaynes-Cummings Hamiltonian 
\begin{equation}
\hat{H}_{\text{JC}}=\omega_{R}\hat{a}^{\dagger}\hat{a}+\frac{\omega_{Q}}{2}\sigma^{z}+g\left(\hat{a}\sigma^{+}+\hat{a}^{\dagger}\sigma^{-}\right),
\end{equation}
where $\omega_{R}$ is the cavity frequency, $\omega_{Q}$ is the
qubit frequency, and $g$ is the coupling parameter. Inclusion of
the simulated $\hat{n}\left(\hat{n}-1\right)$ anharmonicity of strength
$\Gamma$ yields 
\begin{equation}
\hat{H}_{\text{an}}=\omega_{R}\hat{a}^{\dagger}\hat{a}+\Gamma\hat{n}(\hat{n}-1)+\frac{\omega_{Q}}{2}\sigma^{z}+g(\hat{a}\sigma^{+}+\hat{a}^{\dagger}\sigma^{-}),
\end{equation}
and the system is switched between states $\left|0\right\rangle $ and
$\left|1\right\rangle $ with a weak time-$t$-dependent drive of
strength $\Omega$ at the resonance frequency $\omega_{R}$, as follows:
\begin{equation}
\hat{H}_{\text{drive}}\left(t\right)=\Omega e^{i\omega_{R}t}\hat{a}^{\dagger}+\Omega^{\star}e^{-i\omega_{R}t}\hat{a}.
\end{equation}
Synthesis of a propagator of the form $\exp\left(i\lambda^{2}\hat{n}\left(\hat{n}-1\right)\right)$
is then sufficient to employ the native 3D cQED system as a qubit.
Note the choice of $\lambda$ for practical implementation must take
into account both the time step and the fact the BCH decomposition
yields a square root in the exponential argument. The required propagator
is a Trotter-Suzuki decomposition 
% Eq.~(\ref{eq:TrotterFirstOrder})
of $\exp\left(\left[\hat{A},\hat{B}\right]\lambda^{2}\right)=\exp(i\lambda^{2}\hat{n}^{2}\sigma^{z})$
and $\exp(-i\lambda^{2}\hat{n}\sigma^{z})$. 

The first term is synthesized according to the BCH formula 
% Eq.~ (\ref{eq:BCHFormula})
with a commutator determined by the Pauli commutation relation Eq.~(\ref{eq:PauliCommutator})
as follows:
\begin{align}
\left[\hat{A},\hat{B}\right] & =i\hat{n}^{2}\sigma^{z}\\
 & =i\hat{n}^{2}\left(-\frac{i}{2}\left[\sigma^{x},\sigma^{y}\right]\right)\\
 & =\left[\frac{1}{\sqrt{2}}\hat{n}\sigma^{x},\frac{1}{\sqrt{2}}\hat{n}\sigma^{y}\right],
\end{align}
where $\hat{A}$ and $\hat{B}$ correspond to $x$-conditional and
$y$-conditional rotations, respectively. The second term is a $z$-conditional
rotation gate.

The resulting anharmonicity gate therefore has a gate depth of lower
bound five displacement and rotation gates or 45 displacement gates.

\section{Fermi-Hubbard Lattice Dynamics\label{sec:Fermi-Hubbard-Lattice-Dynamics}}

To further demonstrate the power of the  ISA, we employ the
approach to simulate fermionic dynamics on bosonic 3D cQED systems.
We consider the Fermi-Hubbard lattice Hamiltonian

\begin{align}
\hat{H}_{\text{FH}} & =\hat{T}_{\text{FH}}+\hat{V}_{\text{FH}},\label{eq:FermiHubbardHamiltonian}\\
\hat{T}_{\text{FH}} & =-J\sum_{i,\sigma}\hat{c}_{i,\sigma}^{\dagger}\hat{c}_{i+1,\sigma}+\hat{c}_{i+1,\sigma}^{\dagger}\hat{c}_{i,\sigma},\\
\hat{V}_{\text{FH}} & =U\sum_{i}\hat{n}_{i,\uparrow}\hat{n}_{i,\downarrow}.
\end{align}
The kinetic energy term $\hat{T}_{\text{FH}}$ describes the nearest-neighbor
interaction for hopping of a single spin between two sites with hopping
parameter $J$ and spin $\sigma$ given annihilation operators $\left\{ \hat{c}_{j,\sigma}\right\} $
and creation operators $\left\{ \hat{c}_{j,\sigma}^{\dagger}\right\} $
for sites $\left\{ j\right\} $. The potential energy term $\hat{V}_{\text{FH}}$
describes the same-site interaction, which gives the energetic unfavorability
of a spin up $\uparrow$ and spin down $\downarrow$ coexisting on
the same site $i$, where $\hat{n}_{j,\sigma}$ gives the number of
spin $\sigma$ particles on site $j$. According to fermion statistics,
no more than a single particle of a given spin can exist on a single
site.

Each cavity of the 3D cQED system represents either a spin up or spin
down particle on a single lattice site, for direct comparison to the
qubit-based schemes of refs.~\cite{Kivlichan.2018.110501,arute2020observation,Cade.2020.235122}.
Each cavity is connected to the cavity that represents the same site
of opposite spin to facilitate computation of the potential energy
$\hat{V}_{\text{FH}}$, as well as to cavities of the same spin on neighboring
sites to facilitate computation of the kinetic energy $\hat{T}_{\text{FH}}$.
Cavities are also connected along Jordan-Wigner strings to take into
account fermionic statistics. 

The $\text{\ensuremath{\left|0\right\rangle }}$ cavity state represents
absence of a spin and the $\text{\ensuremath{\left|1\right\rangle }}$
state represents presence of a spin. Within each cavity, only the
states in $\text{span}\left\{ \text{\ensuremath{\left|0\right\rangle }},\text{\ensuremath{\left|1\right\rangle }}\right\} $
are considered, as in Section~(\ref{subsec:Universal-Control}), which
prevents leakage into unphysical high-energy cavity states. At the
end of each operation, the cavity state must be in either the $\text{\ensuremath{\left|0\right\rangle }}$
or $\text{\ensuremath{\left|1\right\rangle }}$ state and the transmon
state must also be in the ground state $\text{\ensuremath{\left|g\right\rangle }}$,
which provides an error syndrome and therefore a degree of error detection 
not employed in qubit-based representations of the Fermi-Hubbard lattice.

Propagation of any combination of up spins and down spins is simulated
with three two-cavity gates. The first two gates -- the same-site
and hopping gates -- are defined as the propagator of the same-site
and hopping Hamiltonians, respectively. The same-site term of the
Hamiltonian for site $i$ is 
\begin{equation}
\hat{H}_{\text{same}}=U\hat{n}_{i,\uparrow}\hat{n}_{i,\downarrow}.
\end{equation}
This term is zero if only one spin is on a site and $U$ if both spins
are on the same site, which gives the diagonal Hamiltonian in the
reduced $4\times4$ Hilbert space
\begin{equation}
\hat{H}_{\text{same}}=\left[\begin{array}{cccc}
0 & 0 & 0 & 0\\
0 & 0 & 0 & 0\\
0 & 0 & 0 & 0\\
0 & 0 & 0 & U
\end{array}\right]
\end{equation}
and the diagonal propagator $U_{\text{same}}=\text{e}^{-\text{i}\hat{H}_{\text{same}}\tau}$
\begin{equation}
U_{\text{same}}=\left[\begin{array}{cccc}
1 & 0 & 0 & 0\\
0 & 1 & 0 & 0\\
0 & 0 & 1 & 0\\
0 & 0 & 0 & \text{e}^{-\text{i}U\tau}
\end{array}\right].
\end{equation}
This gate is recognized as the conditional cross-Kerr interaction
of 3D cQED systems and equivalently a controlled-phase (CPHASE) gate
in the reduced subspace $\text{span}\left\{ \left|0\right\rangle ,\left|1\right\rangle \right\} $.
The hopping term of the Hamiltonian for each $\sigma$ spin in sites
$i,\left(i+1\right)$ is 
\begin{align}
H_{\text{hop}} & =-J\left(\hat{c}_{i,\sigma}^{\dagger}\hat{c}_{i+1,\sigma}+\hat{c}_{i+1,\sigma}^{\dagger}\hat{c}_{i,\sigma}\right)\\
 & =-J\left(\hat{c}_{i,\sigma}^{\dagger}\hat{c}_{i+1,\sigma}-\hat{c}_{i,\sigma}\hat{c}_{i+1,\sigma}^{\dagger}\right),
\end{align}
where the latter expression employs the commutator relationship of
the annihilation and creation operators. The hopping Hamiltonian for
the specified mapping is then the off-diagonal matrix 
\begin{equation}
H_{\text{hop}}=\left[\begin{array}{cccc}
0 & 0 & 0 & 0\\
0 & 0 & -t & 0\\
0 & -t & 0 & 0\\
0 & 0 & 0 & 0
\end{array}\right],
\end{equation}
which gives the hopping propagator $U_{\text{hop}}=\text{e}^{-\text{i}H_{\text{hop}}\tau}$
\begin{align}
U_{\text{hop}} & =\left[\begin{array}{cccc}
1 & 0 & 0 & 0\\
0 & \cos\left(t\tau\right) & i\sin\left(t\tau\right) & 0\\
0 & i\sin\left(t\tau\right) & \cos\left(t\tau\right) & 0\\
0 & 0 & 0 & 1
\end{array}\right],
\end{align}
which is recognized as a conditional controlled-phase beam splitter
restricted to $\text{span}\left\{ \left|0\right\rangle ,\left|1\right\rangle \right\} $
in bosonic systems and a Givens or iSWAP-like gate in the reduced
$\text{span}\left\{ \left|0\right\rangle ,\left|1\right\rangle \right\} $
subspace \cite{Cade.2020.235122,arute2020observation}. The final
gate of the three-gate set incorporates the fermionic statistics of
the spins via the fermionic SWAP (FSWAP) gate \cite{Kivlichan.2018.110501,Cade.2020.235122}.
The content of each cavity is swapped with one of its neighbors with
inclusion of a phase where both spins are present in neighboring cavities
as follows
\begin{equation}
U_{\text{FSWAP}}=\left[\begin{array}{cccc}
1 & 0 & 0 & 0\\
0 & 0 & 1 & 0\\
0 & 1 & 0 & 0\\
0 & 0 & 0 & -1
\end{array}\right],
\end{equation}
which is recognized as the product of a conditional rotation gate
and a beam-splitter on 3D cQED systems.

Finally, initial states are prepared by the universal set of gates
in $\text{span}\left\{ \left|0\right\rangle ,\left|1\right\rangle \right\} $
detailed in Section~\ref{subsec:Universal-Control}.

\subsection{Conditional cross-Kerr (CPHASE) gate}

We consider the infinitesimal conditional cross-Kerr gate
\begin{equation}
U_{\text{cross-Kerr}}=e^{i\lambda^{2}\hat{n}_{1}\hat{n}_{2}\sigma_{z}},
\end{equation}
which is also employed in GKP codes encoded in 3D cQED systems \cite{royer2022encoding}. 

The argument is expressed in terms of a commutator according to the
Pauli commutation relation Eq.~\ref{eq:PauliCommutator}, as follows:
\begin{align}
\left[A,B\right] & \lambda^{2}=i\lambda^{2}\hat{n}_{1}\hat{n}_{2}\sigma_{z}\\
 & =i\lambda^{2}\hat{n}_{1}\hat{n}_{2}\left(-\frac{i}{2}\left[\sigma^{x},\sigma^{y}\right]\right)\\
 & =\left[\frac{1}{\sqrt{2}}\hat{n}_{1}\sigma^{x},\frac{1}{\sqrt{2}}\hat{n}_{2}\sigma^{y}\right]\lambda^{2},
\end{align}
where $\hat{A}$ corresponds to an $x$-conditional rotation gate
and $B$ corresponds to a $y$-conditional rotation gate.

The resulting gate features a lower bound gate depth of four displacement and rotation gates or 16 displacement gates.

\subsection{ \texorpdfstring{$\text{Span}\left\{ \left|0\right\rangle ,\left|1\right\rangle \right\} $}{}
conditional beam splitter gate}

In order to generate a $\text{span}\left\{ \left|0\right\rangle ,\left|1\right\rangle \right\} $
that operates only when $\hat{n}_{1}\hat{n}_{2}\ne1$ (\emph{i.e.},
$1-\hat{n}_{1}\hat{n}_{2}=0$), we formulate the infinitesimal conditional
(controlled-phase) beam-splitter gate
\begin{equation}
U_{\text{cond. beam}}=e^{-i\lambda^{2}\left(\hat{a}_{1}^{\dagger}\hat{a}_{2}+\hat{a}_{1}\hat{a}_{2}^{\dagger}\right)\left(1-\hat{n}_{1}\hat{n}_{2}\right)\sigma^{z}},
\end{equation}
which is decomposed via the Trotter-Suzuki decomposition 
% Eq.~\ref{eq:TrotterFirstOrder}
in terms of $\exp\left(-i\lambda^{2}\left(\hat{a}_{1}^{\dagger}\hat{a}_{2}+\hat{a}_{1}\hat{a}_{2}^{\dagger}\right)\sigma^{z}\right)$
and $\exp\left(i\lambda^{2}\left(\hat{a}_{1}^{\dagger}\hat{a}_{2}+\hat{a}_{1}\hat{a}_{2}^{\dagger}\right)\left(\hat{n}_{1}\hat{n}_{2}\right)\sigma^{z}\right)$.
The first term is the conditional beam splitter $U_{\text{beam split.}}$
Eq.~\ref{subsec:Conditional-(Controlled-Phase)-Beam-Splitter} and
the second term is decomposed via BCH 
% Eq.~\ref{eq:BCHFormula}
as
follows:

Given the expression of the number operator in terms of the phase-space
operators Eq.~\ref{eq:NumbertoPhaseSpace}, the argument of the second
exponential operator is 
\begin{gather}
i\lambda^{2}\left(\hat{a}_{1}^{\dagger}\hat{a}_{2}+\hat{a}_{1}\hat{a}_{2}^{\dagger}\right)\hat{n}_{1}\hat{n}_{2}\sigma^{z}\nonumber \\
=i\lambda^{2}\left(2\left(\hat{x}_{1}\hat{x}_{2}+\hat{p}_{1}\hat{p}_{2}\right)\right)\hat{n}_{1}\hat{n}_{2}\sigma^{z}.
\end{gather}
The term is then expressed as a Trotter decomposition of $\exp\left(\left[\hat{A}_{1},\hat{B}_{1}\right]\lambda^{2}\right)=\exp\left(2i\lambda^{2}\hat{x}_{1}\hat{x}_{2}\hat{n}_{1}\hat{n}_{2}\sigma^{z}\right)$
and $\exp\left(\left[\hat{A}_{2},\hat{B}_{2}\right]\lambda^{2}\right)=\exp\left(2i\lambda^{2}\hat{p}_{1}\hat{p}_{2}\hat{n}_{1}\hat{n}_{2}\sigma^{z}\right)$. 

The first commutator is given by the Pauli commutation relation Eq.~\ref{eq:PauliCommutator}
\begin{align}
\left[\hat{A}_{1},\hat{B}_{1}\right] & =2i\hat{x}_{1}\hat{x}_{2}\hat{n}_{1}\hat{n}_{2}\sigma^{z}\\
 & =2i\hat{x}_{1}\hat{x}_{2}\hat{n}_{1}\hat{n}_{2}\left(-\frac{i}{2}\left[\sigma^{x},\sigma^{y}\right]\right)\\
 & =\left[\hat{x}_{1}\hat{n}_{1}\sigma^{x},\hat{x}_{2}\hat{n}_{2}\sigma^{y}\right].
\end{align}

The $\hat{A}_{1}$ term is determined by a Trotter decomposition such that
\begin{align}
\hat{A}_{1} & =\frac{1}{2}\left\{ \hat{x}_{1},\hat{n}_{1}\right\} \sigma^{x}+\frac{1}{2}\left[\hat{x}_{1},\hat{n}_{1}\right]\sigma^{x}\\
 & =\hat{A}_{1a}+\hat{A}_{1b},
\end{align}
where according to the anticommutator to commutator relation $\hat{A}_{1a}$
is given by the BCH formula with 
\begin{align}
\left[\hat{A}_{1a^{\prime}},\hat{B}_{1a^{\prime}}\right] & =\frac{1}{2}\left\{ \hat{x}_{1},\hat{n}_{1}\right\} \sigma^{x}\\
 & =\frac{i}{2}\left[i\hat{x}_{1}\sigma^{y},i\hat{n}_{1}\sigma^{z}\right]\\
 & =\left[-\frac{1}{\sqrt{2}}\hat{x}_{1}\sigma^{y},-\frac{1}{\sqrt{2}}\hat{n}_{1}\sigma^{z}\right],
\end{align}
where $\hat{B}_{1a^{\prime}}$ is a $z$-conditional rotation gate.
Distribution of terms yields $\hat{A}_{1b}$ as 
\begin{equation}
\left[\hat{A}_{1b^{\prime}},\hat{B}_{1b^{\prime}}\right]=\left[\frac{1}{\sqrt{2}}\hat{x}_{1},\frac{1}{\sqrt{2}}\hat{n}_{1}\sigma^{x}\right],
\end{equation}
where $B_{1b^{\prime}}$ is an $x$-conditional rotation gate. 

According to the same procedure,
\begin{align}
\hat{B}_{1} & =\frac{1}{2}\left\{ \hat{x}_{2},\hat{n}_{2}\right\} \sigma^{y}+\frac{1}{2}\left[\hat{x}_{2},\hat{n}_{2}\right]\sigma^{y}\\
 & =\hat{B}_{1a}+\hat{B}_{1b},
\end{align}
where $\hat{B}_{1a}$ is given by
\begin{align}
\left[\hat{A}_{1a^{\prime\prime}},\hat{B}_{1a^{\prime\prime}}\right] & =\frac{1}{2}\left\{ \hat{x}_{2},\hat{n}_{2}\right\} \sigma^{y}\\
 & =\frac{i}{2}\left[i\hat{x}_{2}\sigma^{z},i\hat{n}_{1}\sigma^{x}\right]\\
 & =\left[-\frac{1}{\sqrt{2}}\hat{x}_{2}\sigma^{z},-\frac{1}{\sqrt{2}}\hat{n}_{2}\sigma^{x}\right],
\end{align}
with $B_{1a^{\prime\prime}}$ an $x$-conditional rotation, and $B_{1b}$
is given by 
\begin{equation}
\left[\hat{A}_{1b^{\prime\prime}},\hat{B}_{1b^{\prime\prime}}\right]=\left[\frac{1}{\sqrt{2}}\hat{x}_{2},\frac{1}{\sqrt{2}}\hat{n}_{2}\sigma^{y}\right],
\end{equation}
where $B_{1b^{\prime\prime}}$ is a $y$-conditional rotation gate.
The second term follows analogously with the position $x$ replaced
by the momentum $p$.

\subsection{Conditional FSWAP gate}

The FSWAP gate follows immediately from the conditional
cross-Kerr gate detailed above and a complete beam-splitter gate (or
conditional beam-splitter gate detailed above) as 
\begin{equation}
U_{\text{FSWAP}}=U_{\text{cond. Kerr}}U_{\text{cond. beam}}.
\end{equation}



% \end{appendices}

% \input{poster_formula.tex}

%\newpage
\printbibliography
\end{document}
% ------------------------------------------- ORPHANED STUFF ---------------------------------

\begin{lemma}\label{conjugation}
Suppose we may implement $\exp (O \kron A)$ for some arbitrary operator $O$ and self-adjoint gate operating on a qubit $A$. Furthermore, suppose we have qubit operators $M, M^\dagger$. Then,
\begin{align}
    M \exp (O \kron A) M^\dagger = \exp (O \kron M A M^\dagger)
\end{align}
\end{lemma}
\begin{proof}
This is easily verifiable algebraically:
\begin{align}
    M \exp (O \kron A) M^\dagger &= M \sum_{k = 0}^\infty \frac{(O \kron A)^k}{k!} M^\dagger \\
    &= M \Big[ \sum_{j = 0}^\infty \frac{(O \kron A)^{2j}}{(2j)!} + \sum_{j = 0}^\infty \frac{(O \kron A)^{2j + 1}}{(2j + 1)!} \Big] M^{\dagger} \\
    &= \sum_{j = 0}^\infty \frac{O^{2j} \kron \identity}{(2j)!} + \sum_{j = 0}^\infty \frac{O^{2j + 1} \kron M A^{2j + 1}M^{\dagger}}{(2j + 1)!} \\
    &= \sum_{k = 0}^\infty \frac{(O \kron M A M^{\dagger})^{k}}{k!}
\end{align}

As desired.
\end{proof}

\begin{lemma}\label{lem:exp-error}
An exponential of a polynomial error is still polynomial. For $k \geq 1$:
\begin{align}
    e^{\mathcal{O}(x^k)} = \sum_{j = 0} \frac{x^{k \cdot j}}{j!} = 1 + x^k + \sum_{j = 2} \frac{x^{k\cdot j}}{j!} \in \mathcal{O}(x^k)
\end{align}
\end{lemma}

\begin{fact}[Box 4.1 from \cite{nielsen2002quantum}]\label{fact:41-nc}
Suppose we have some gates $V_1, V_2, ... V_n$ intending to approximate some other gates $U_1, U_2, ..., U_n$. Then, the error incurred is most linear, i.e.:
\begin{align}
    \norm{V_1 V_2 ... V_n - U_1 U_2 ... U_n} \leq \sum_{k = 1}^n \norm{V_k - U_k}
\end{align}
\end{fact}

\subsection{Psuedocode \& Bounds}