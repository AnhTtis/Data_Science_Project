\newcommand{\ck}[1]{{\color{red} CK: #1}}

\newcommand{\bch}{\mathrm{BCH}}
\newcommand{\trotter}{\mathrm{Trotter}}


\begin{abstract}
    Circuit QED enables the combined use of qubits and oscillator modes. Despite a variety of available gate sets, many hybrid qubit-boson (i.e., oscillator) operations are realizable only through optimal control theory (OCT) which is oftentimes intractable and uninterpretable. We introduce an analytic approach with rigorously proven error bounds for realizing specific classes of operations via two matrix product formulas commonly used in Hamiltonian simulation, the Lie--Trotter and Baker--Campbell--Hausdorff product formulas. We show how this technique can be used to realize a number of operations of interest, including polynomials of annihilation and creation operators, i.e., $a^p {a^\dagger}^q$ for integer $p, q$.  We show examples of this paradigm including: obtaining universal control within a subspace of the entire Fock space of an oscillator, state preparation of a fixed photon number in the cavity, simulation of the Jaynes--Cummings Hamiltonian, simulation of the Hong-Ou-Mandel effect and more.  This work demonstrates how techniques from Hamiltonian simulation can be applied to better control hybrid boson-qubit devices.
\end{abstract}

%\newpage
%\tableofcontents

% \how{current overview}
% In section II, we describe the hybrid-boson device and introduce the two Hamiltonian simulation product formulas which will be used. In section III, we describe the addition technique and state our results about the error scaling and operation count. In section IV, we demonstrate the technique on a number of applications and include numerics to observe the scaling. We summarize our observations in section V.


%\added{Search for string SMG to find queries from Steve.  References are out of order.  You do not refer in the text to Appendices C, E, or F; please insert text referring to those.  Equation numbers are messed up by theorem environment.}

% May want to cite https://arxiv.org/pdf/2111.12177.pdf


\section{Introduction}
%\how{synergy between oct and this technique} what can be made with oct, how can it be extended with our technique

Today, many quantum computing architectures are homogeneous -- i.e. the same type of qubit is employed throughout the device. From devices made of superconducting qubits \cite{arute2019quantum,dial2022moving,reagor2018demonstration} to ion trap qubits \cite{wright2019benchmarking}, prior work largely focuses on linking qubits of the same type together in fault-tolerant ways. However, there is a nascent body of work that studies the potential for hybrid quantum computers that leverage two or more types of quantum architectures (e.g., qubits and oscillator modes). These devices hold promise because they can be tailored for specific physical simulation problems, which would be especially useful in applications like material discovery or molecular simulation \cite{Wang2020FCFs,WangConicalIntersection} or quantum simulation of lattice models~\cite{PhysRevB.98.174505}.

In particular, the hybrid qubit-boson (i.e., oscillator) models \cite{blais2021circuit} hold some advantages.  Specifically, the long lifetimes afforded to microwave photons in superconducting resonators have made them attractive targets for quantum error correction~\cite{blais2021circuit}.  In addition, the Hilbert space accessible to a mode is much larger than that of qubits.  Furthermore, the larger set of operations that can be performed by coupling an oscillator and a qubit open up the potential for multi-qubit interactions, such as the M{\o}lmer-S{\o}rensen gate \cite{molmer-sorensen-gate} while at the same time enabling new forms of transduction between qubits and flying qubits such as photons~\cite{lauk2020perspectives,basilewitsch2022engineering}. Oscillator interactions may also have unique features, like nonlinearities, which are challenging to simulate even with homogeneous quantum architectures~\cite{stavenger2022bosonic}.


%Spin-boson instruction set architectures (ISA) enable the general control of native bosonic hardware which could have advantages over purely spin-1/2 hardware in quantum signal processing (QSP), quantum simulation, and the preparation and use of quantum states requiring a large Hilbert space.  

However, whereas many useful applications have already been demonstrated with the experimentally available gate sets in ion traps and circuit QED, often problems of interest require more complex operations, and these must be compiled from the various experimental regimes and pulse sequences available.   Specifically, techniques such as optimal control theory (OCT)~\cite{khaneja2005optimal,werschnik2007quantum,Koch-OCT-2017} provide ways to design at a pulse level a sequence of controls that can be set in order to enact an arbitrary quantum transformation on a hybrid-boson qubit system.  Although optimal control theory  does provide a method of compiling nonnative operations into native gates, it is oftentimes intractable and almost always uninterpretable, yielding only a pulse which performs the desired operation without providing any physical intuition. Furthermore, it is computationally intensive to produce and due to the lack of theoretical understanding behind the pulse sequences it must be carried out on a case-by-case basis.  These shortcomings further make models such as this difficult to program and to analyze the circuit complexity, as the procedure and in turn the cost for constructing an arbitrary unitary transformation can be difficult to bound.

Inspired by recent experimental progress \cite{SNAP-PhysRevLett.115.137002,eickbusch2021fast}, we introduce in this paper an extensible control scheme for a hybrid boson-qubit quantum computer that is universal.  
Specifically, we consider the potential of instruction set architectures (ISAs) compiled from experimentally available gate sets using the Baker-Campbell-Hausdorff, Trotter-Suzuki and Lie-Trotter formulas. We develop two parallel approaches, one which primarily uses the creation and destruction operators which we refer to as based on `Fock methods', and the other primarily relying on position and momentum operators which we refer to as based on `phase-space methods'. We demonstrate that both methods can be used to generate an ISA for bosonic devices. Furthermore, we use the previously mentioned formulas to realize a number of operations of interest, including polynomials of annihilation and creation operators, i.e., $a^p {a^\dagger}^q$ for integer $p, q$. These block-encoded operations are crucial for QSP and certain problems in quantum simulation. Furthermore, we demonstrate that our compilation scheme obtains near-linear scaling and provide upper bounds on the maximum number of operations required to implement the compiled gate. Finally, we give examples for the Hamiltonian of a nonlinear material and applications to common unsolved problems in quantum simulation such as the Fermi-Hubbard model.

% In this paper, we introduce a novel control scheme for a hybrid boson-qubit quantum computer that enables simulation of the Jaynes-Cummings model \cite{JaynesCummings1963}. \added{SMG: Not sure this should be the main focus since you don't explicitly discuss it and there are a number of examples you include and more could be included.}  


% Problem, not knowing how to make gates, no physical insight

% We introduce an analytic approach

% Here is our method ((x, p) and (a, adag))

%% here's how we're connzecting our physical intuition to the realization of actual gates

%% CK: See how to connect the prior overview with the overall language. 

%% here's how you obtain a decomposition in phase-space / displacement operators

%% here's the algorithmic justification that this works

%% not sure about the subsection orderings
\newpage
\section{Preliminaries}
{In this section, we introduce the hybrid boson-qubit architecture we are operating on and the matrix product formulas we will use.}
The central goal of our work is to provide a generic toolbox that can be used to build unitary transformations on hybrid quantum devices that have both qubits as well as bosonic modes.  Such devices are common in quantum computing, spanning circuit quantum electrodynamics (superconducting qubits coupled to microwave photons) to ion-trap quantum computing (for which the mechanical modes of oscillation are coupled to atomic qubits).  The challenge though with these approaches is that fundamentally different insights are needed here to compile unitaries than the binary-based approaches that have been successful for the case of qubits.

Here we review the mathematical properties of bosonic quantum mechanics which are needed to understand the basic operations considered for the ISA architecture that we consider.  
Specifically, we present an analytic instruction set architecture (ISA) based on the Trotter-Suzuki and Baker-Campbell-Hausdorff (BCH) decompositions for decomposition of gates of the form $U=e^{i\hat{H}\sigma^{j}}$, where $\hat{H}$ is a Hermitian
operator composed of phase-space operators and Pauli gates. Before jumping into the specific details of our gate operations, we need to review the basics of bosonic quantum mechanics as well as the mathematical results needed to use these bosonic operations to compile a given unitary transformation.

\subsection{A Hybrid Boson-Qubit Device}
The central object that we need in this extension is the concept of a qumode, which stores the state of the bosons. Photons (energy quanta of the electromagnetic field) and phonons (quanta of mechanical vibrations) are examples of bosons present in current quantum computing experiments. Photons are the focus of photonic devices, cavity QED and hybrid circuit QED, whereas phonons are used to couple ion-spin qubits in ion traps. In the case of photons, the qumode stores the configuration of the electromagnetic field, and in the case of phonons, the qumode stores the vibrational state. We first produce a mathematical description of a qumode, then describe operations which can be performed on the qumode.

\subsubsection{Representing the qumode}
There are two different bases that are commonly used to describe the state of the qumode: 
\begin{enumerate}
    \item \textit{Phase-space representation}, where operators are written in terms of position ($\hat{x}$) and momentum ($\hat{p}$) operators
    \item \textit{Fock-space representation}, where operators are written in terms of bosonic creation ($a^\dagger$) and annihilation ($a$) operators.
\end{enumerate}

In the phase-space representation, the computational basis corresponds to the strength of the electric field in the case of photons (or equivalently the position of a mechanical oscillator for vibrational systems). We refer to this with the operator $\hat{x}$ and we have that for any $x\in \mathbb{R}$, $\hat{x} \ket{x} = x \ket{x}$.  It is often common to speak of the momentum operator $\hat{p}$, which can be found through the Fourier transform of $\hat{x}$.  This describes the magnetic field for a photonic system.  In practice, cutoffs are imposed on the values of the field and further discretization error on the gates and the outputs prevent arbitrary precision readout; however, for simplicity we ignore the latter issue in order to provide a simpler (if unrealistic) computational model and ignore the issue that even when cutoffs are imposed the vector space does not strictly form a Hilbert space without also including spatial discretization.

In the Fock-space representation, we track the number of bosons (number of photons or the energy level of the harmonic oscillator for the vibrational case) in the computational basis.  In this representation the computational basis is defined to be an eigenvector of the boson number operator $\hat{n} \ket{n} = n \ket{n}$, where $\hat{n} = a^\dagger a$ is the number operator and $a$ and $a^\dagger$ add and remove a boson from the system, respectively.  Formally this spectrum is countably infinite, but after truncation it forms a finite-dimensional Hilbert space and thus can be thought of as a qudit.  For example, assuming a cutoff $\Lambda = 3$:
\begin{align}
    P_3 a^\dagger P_3 = \begin{bmatrix}
    0 & 0 & 0 & 0 \\
    1 & 0 & 0 & 0 \\
    0 & \sqrt{2} & 0 & 0 \\
    0 & 0 & \sqrt{3} & 0 
    \end{bmatrix}, P_3 a P_3 = \begin{bmatrix}
    0 & 1 & 0 & 0\\
    0 & 0 & \sqrt{2} & 0 \\
    0 & 0 & 0 & \sqrt{3} \\
    0 & 0 & 0 & 0
    \end{bmatrix}.
\end{align}
%\textbf{Above we should use a projection to discuss the truncation}
Here $P_\Lambda$ is the projector onto the subspace of the cavity containing at most $\Lambda$ photons, i.e.:
\begin{align}
    P_\Lambda : P_\Lambda \ket{n} = \begin{cases}
    \ket{n} & n \leq \Lambda \\
    0 & \textrm{otherwise}
    \end{cases}.
\end{align}

In this work, if unbounded operators are present, the remainder terms in the expansions that we consider become undefined and some form of a truncation is needed to ensure that the mathematics is well defined.  Provided that an appropriate cutoff is picked for the system, the discrepancies between the truncated and untruncated systems will often be negligible. For notational clarity, we assume a cutoff of $\Lambda$ for all further equations and assume the annihilation and creation operators implicitly have the projectors $P_\Lambda$.


%$The relationship between these operators and the position operator forms the translation guide needed to move between these two equivalent descriptions of the qumodes.

% \how{we need a discussion about why we can use cutoffs and the reduced notation for it}

% \how{ie need to consolidate $\Tilde{a} \coloneqq P_\Lambda a P_\Lambda $}

The final aspect that we need to talk about in our description of the state is the qubits.  In our model of computing we assume that the vector space is a tensor product of a qubit Hilbert space and the vector space of qumodes, i.e., the state space is $ \mathcal{H}_2 \kron \mathcal{H}_{\Lambda}$. This means that, for example, a computational basis state will be of the form $\ket{q} \kron \ket{m}_b $ where the state $\ket{q}$ here can be thought of as the union of the qubit registers in the system, and $\ket{m}_b$ represents a qumode state where the system is either in position $x=m$ for the phase-space encoding or has $m$ photons if the Fock-space encoding is used. Our work will assume only a single qubit register with a qumode in the Fock-space encoding.



\subsubsection{Valid operations}
Now that we have described the vector spaces that our operators act in, we will discuss the bosonic operations that act on the qumodes for the system. In order to introduce this, we need to review some notation surrounding bosonic creation and annihilation operators (otherwise known as raising and lowering or Fock operators). 
In the following, we demonstrate how some operations can be analytically decomposed, regardless of whether they are defined with Fock operators ($a, a^\dagger$) or phase-space operators ($\hat{x}, \hat{p}$), as these have the following equivalences:
\begin{align}
    \hat{x} = \frac{1}{2} (a + a^\dagger) \qquad &\Leftrightarrow \qquad a = \hat{x} + i \hat{p} \label{eq:AnnihilationOperatortoPhaseSpace},\\
    \hat{p} = -\frac{i}{2} (a - a^\dagger) \qquad &\Leftrightarrow \qquad a^\dagger = \hat{x} - i \hat{p}, \label{eq:CreationOperatortoPhaseSpace}
\end{align}
and commutation relations:
\begin{align}
    [\hat{x}, \hat{p}] &= \frac{i}{2} \label{eq:Commutatorxp}, \\
    [a, a^\dagger] &= 1 \label{eq:CommutatorCreationAnnihilation}.
\end{align}

There are many gate operations that can be considered as part of an instruction set. Within our ISA, operations can be qubit or qumode-exclusive or entangle across the qubit and qumode. For example, we can still perform single-qubit operations like $H$ and $R_z$.

Additionally, we can assume that linear optical operations (which are at most quadratic in the field operators) can be performed on the modes, such as the displacement operations $e^{ \alpha a^\dagger + \alpha^* a }$ with $\alpha$ a complex number, phase delays (or phase-space rotations) $e^{-i\alpha  a^\dagger a }$, squeezing operations $e^{\alpha a^2 -\alpha^{*}a^2}$, and beamsplitter operations $e^{(\alpha a^\dagger b - \alpha^{*}a b^{\dagger})}$ where $b$ is the creation operator acting on a different mode (although in this paper we are mainly interested in the single-mode case).  We further assume that the qubit can be measured directly, but the oscillator can only be measured by entangling it with a qubit.

Finally, there are several entangling operations between oscillator and qubit that widely appear in the circuit QED literature.  Two common ones are the controlled displacement operation \cite{eickbusch2021fast} $e^{-i(\alpha a^\dagger + \alpha a) \otimes \sigma^z}$ and the Selective Number-dependent Arbitrary Phase (SNAP) gate \cite{SNAP-PhysRevLett.115.137002} $e^{-i\sum_n\alpha_n \hat P_n \otimes \sigma^z}$, where $\hat P_n=|n\rangle\langle n|$ is the projector onto the $n$th Fock state. (In our paper, we refer to hybrid gates with $\sigma^i$ notation, while single-qubit gates like $S, X, H$ are capitalized).
%SMG: The SNAP gate puts a different phase $\theta_n$ on each Fock state.
Here, for convenience, we focus on a different gate, which we name $\mathcal{S}_1$, which we assume we can implement without error. While this gate can be approximated with arbitrary fidelity using conditional displacements (see \cref{obtaining_s1}), in practice it can be implemented straightforwardly with OCT \cite{PhysRevX.4.041010,PhysRevA.91.043846,Rosenblum2018,Rosenblum2023}. 
\begin{define}\label{defn:SX}
For any $t>0$ and any positive integer cutoff $\Lambda$, we define $\mathcal{S}_1$ to be the unitary acting on the Hilbert space $\mathcal{H}_2 \otimes \mathcal{H}_{\Lambda}$ that has the following representation as a block-matrix
\begin{align}
    \mathcal{S}_1 = \exp \left(it \begin{bmatrix}
    0 & a^{\dagger} \\
    a & 0
    \end{bmatrix} \right).
\end{align}
\end{define}

Notice that this gate operates simultaneously on the boson and qubit modes (producing, in the language of ion-trap quantum computers, a `red-sideband' transition). In particular, the block-encoded matrix is equivalent to the following cavity-qubit coupling operator in the Jaynes-Cummings model:
\begin{align}\label{eq:JC}
    \begin{bmatrix}
    0 &  a^\dagger  \\
     a & 0
    \end{bmatrix} = \ket{0}\bra{1} \kron a^\dagger  + \ket{1}\bra{0} \kron  a.
\end{align}

The Jaynes-Cummings Hamiltonian is the natural one in circuit QED, therefore the coupling $\mathcal{S}_1$ is naturally present. However, one can choose an experimental parameter regime in which the coupling becomes negligible. This happens when the cavity and qubit are strongly detuned, and they obey the dispersive coupling Hamiltonian. This is a useful regime, which may be chosen, for example, when needing to use the controlled displacement gate. The $S_1$ block encoding can be synthesized in this regime, and this is demonstrated in Appendix~\ref{obtaining_s1}.

The reason why we focus on this incredibly simple example is to provide a case that is likely to have an advantage in our hybrid model.  Specifically, the further we are away from the natural interactions that are present in our quantum simulator the more expensive we anticipate a simulation to be, as we will have to construct the desired interaction from a potentially long sequence of fundamental interactions.  In this sense, our approach borrows inspiration from analog simulation.  Unlike analog simulation, the interactions here are universal in the sense that we can emulate \emph{any Jaynes-Cummings model} using our approach.

% EC: This commented out version is what was there before my edits - just leaving this here in case you'd rather go back to it.
{
% Integer occupancy levels of the oscillator are called \textit{Fock} states and denoted $\ket{n}$, where $n$ is the number of bosons. While there could theoretically be an arbitrary number of bosons in the oscillator, in practice, we consider the oscillator to be bounded by some constant $\Lambda$. 

% Our oscillator has two primary operators which act upon it: annihilation and creation operators. These operate similar to fermionic operators, in that they increment / decrement the number of bosons present, and thus transform Fock state to a different Fock state. However, unlike fermionic operators, bosonic operators do not have finite matrix representations; again, we must assume some $\Lambda$ in order to produce a finite matrix form. For example, assuming $\Lambda = 3$:
% \begin{align}
%     P_3 a^\dagger P_3 = \begin{bmatrix}
%     0 & 0 & 0 & 0 \\
%     1 & 0 & 0 & 0 \\
%     0 & \sqrt{2} & 0 & 0 \\
%     0 & 0 & \sqrt{3} & 0 
%     \end{bmatrix}, P_3 a P_3 = \begin{bmatrix}
%     0 & 1 & 0 & 0\\
%     0 & 0 & \sqrt{2} & 0 \\
%     0 & 0 & 0 & \sqrt{3} \\
%     0 & 0 & 0 & 0
%     \end{bmatrix}
% \end{align}
% %\textbf{Above we should use a projection to discuss the truncation}
% Here $P_\Lambda$ is the projector onto the subspace of the cavity containing at most $\Lambda$ photons, i.e.:
% \begin{align}
%     P_\Lambda : P_\Lambda \ket{n} = \begin{cases}
%     \ket{n} & n \leq \Lambda \\
%     0 & \textrm{Otherwise}
%     \end{cases}
% \end{align}
% Note that, in practice, it is common to ignore the truncation onto a finite-dimensional space for the bosonic operations.  We avoid doing so here, except where essential, because the remainder terms in the \ck{ec comment} expansions that we consider become undefined if unbounded operators are present.  Provided that an appropriate cutoff is picked for the system, the discrepancies between the truncated and untruncated systems will often be negligible.

%Integer occupancy levels of the mode are called \textit{Fock} states and denoted $\ket{n}$, where $n$ is the number of bosons. Our mode has two primary operators which act upon it: annihilation and creation operators. These operate similar to fermionic operators, in that they increment / decrement the number of bosons present, and thus transform Fock state to a different Fock state. 
}




\subsection{Lie product formulas}
Lie product formulas describe the behavior when matrix exponentials are multiplied, i.e., how $e^A e^B$ behaves. These formulas are well-known in Hamiltonian simulation~\cite{childs2021theory,berry2007efficient,su2021fault} and are used to approximate a discretized version of the time evolution operator $e^{-iHt}$ using the Trotter formula. However, in our application, we demonstrate how these formulas can be used as a control scheme for our device.

We introduce two product formulas: the Baker-Campbell-Hausdorff (BCH) formula and the Lie-Trotter formula. The BCH formula can be used to create a commutator (or anticommutator) of operators. The Trotter formula can be used to add these commutators/anticommutators together. We introduce the informal theorems below:
\begin{theorem}[Informal BCH theorem from \cite{Childs_2013}]\label{thm:BCH}
    Suppose we can implement the operators $e^{A \lambda}, e^{B \lambda}$ for $\lambda \in \mathbb{R}$. Then, a BCH formula of order $p$ has the following error scaling:
    \begin{align}
        \bch_p(A\lambda, B \lambda) &= e^{A \lambda} e^{B \lambda} e^{-A \lambda} e^{-B \lambda}\\&= e^{[A, B] \lambda^2} + \mathcal{O}((\max (\norm{A}, \norm{B}) \lambda)^{2p + 1}), 
    \end{align}
    requiring no more than $8 \cdot 6^{ p -1}$ exponentials.
\end{theorem}

\begin{theorem}[Informal Trotter theorem from \cite{berry2007efficient}]
    Suppose we may implement $e^{A \lambda}$ and $e^{B \lambda}$ for arbitrary $\lambda \in \mathbb{R}$. Then, a $p^\text{th}$ order Trotter formula has the following error scaling:
    \begin{align}
        \trotter_{2p}(A\lambda, B \lambda) & = e^{(A+B)\lambda} + \mathcal{O}((\norm{A + B} \lambda)^{2p + 1}),
    \end{align}
    requiring no more than $4 \cdot 5^{ p -1}$ exponentials.
\end{theorem}

Though the BCH formula is defined only to produce commutators, we can use our qubit to produce anticommutators of bosonic operations. Informally, we exploit commutators of Pauli operations to create a phase change, as $\sigma^z = -\frac{i}{2} [\sigma^x, \sigma^y]$. Thus, BCH can be used to create an anticommutation relation on the mode operators
\begin{align}
    i \left[ i A \sigma^x, i B \sigma^{y }\right] = \{ A, B \} \sigma^z.\label{eq:sigmazterm}
\end{align}
We later show how these commutators and anticommutators can be combined to directly implement products of operators. Observing that $\frac{1}{2}[A, B] + \frac{1}{2} \{A , B\} = AB$, we can use the Trotter formula to produce:
\begin{align}
    \trotter \left(\frac{1}{2} [A, B], \frac{1}{2} \{ A, B \} \right) \approx \exp (AB).
\end{align}
 

% \begin{theorem}[Informal Trotter theorem]
%     Suppose we have $H = \sum_i H_i$ and we may implement all $e^{H_i \lambda}$ for arbitrary $\lambda \in \mathbb{R}$. Then, a $p$th order Trotter formula has the following error scaling:
%     \begin{align}
%         \trotter_{2p}(\{ H_i \} , \lambda) & = e^{\lambda \sum_i H_i} \\&= \left(\prod_i e^{(H_i \lambda / p)}\right)^p + \mathcal{O}(\lambda^p)\\&= e^{\sum H_i \lambda} + \mathcal{O}((\norm{H}\lambda)^{2p + 1})
%     \end{align}
% \end{theorem}



\section{Producing Anticommutators and Exponential Products}
In this section, we formalize our technique and present methods for both hermitian operators, such as phase-space operators $\hat{x}$ and $\hat{p}$, and non-hermitian operators, such as Fock space operators $a, a^\dagger$.  We examine the task of constructing polynomials in phase space as well as Fock space operators using qubit controlled operations.  We first show how the additional qubit of our hybrid boson-qubit architecture allows for the synthesis of anticommutators of hermitian operators and, by proxy, matrix products of phase-space operators. We then use similar techniques with non-hermitian operators such as Fock-space operators to manipulate block encodings of matrices. Finally, we contextualize these methods with asymptotic error bounds, providing theoretical analyses of our proposed techniques.

We denote block-encoded matrices with the following notation:
\begin{align}\label{eq:blockencodedmatrix}
    \mathcal{B}_A = \exp it \begin{bmatrix}
        0 & A \\
        A^\dagger & 0
    \end{bmatrix},
\end{align}
where the subscript is the upper right block and the lower left block is the transpose and complex conjugate to preserve hermiticity. This block encoded gate corresponds to $\exp{itA\sigma^x}$ only if $A$ is hermitian, otherwise it corresponds to $\exp{it( \ket{0}\bra{1} \kron A^\dagger  + \ket{1}\bra{0} \kron  A)}$, the synthesis of which we show in Appendix~\ref{obtaining_s1}.




% \begin{landscape}

\begin{table}
    \small
    \centering
    \begin{tabularx}{\textwidth}{ccll}
        \hline
        \textbf{Formula} & \textbf{Target} &\textbf{Preconditions} & \textbf{Reference} \\
        \hline
        $\textrm{BCH}(\exp it B \sigma^i, \exp it A \sigma^i)$ & $\exp( t^2 [A, B]) \identity$ &$A, B$ Hermitian &\cref{thm:BCH}  \\
        %%%%
        $\textrm{BCH}(\exp it A \sigma^j, \exp it B \sigma^k) $ & $\exp (- i t^2 \{ A, B \} \sigma^i)$ &$A, B$ Hermitian &\cref{eq:sigmazterm} \\
        %%%%%
                \hline
         $\textrm{BCH}(X  \mathcal{B}_B(t) X, \mathcal{B}_A(t) ) $ & $\exp (t^2 (AB - (AB)^\dagger) \sigma^z)$ & $[A,B]=0$ &\cref{eq:C2}\\
        %%%%%%
         $\textrm{BCH}(S \mathcal{B}_A(t) S^\dagger, X \mathcal{B}_B(t) X) $ & $\exp (i t^2 (AB + (AB)^\dagger) \sigma^z) $ & $[A,B]=0$ & \cref{eq:AC2}\\
        %%%%
         $\textrm{BCH}(S \mathcal{B}_A(t) S^\dagger, X \mathcal{B}_B(t) X) $ & $\exp (2 i t^2 AB \sigma^z )$ & \makecell[l]{$[A,B]=0$, \\$AB = (AB)^\dagger $} &\cref{eq:Product} \\
        %%%%
         $\textrm{Trotter}\Big(\exp t (AB - (AB)^\dagger) \sigma^y, $ & $\exp \Big(2 it \begin{bmatrix}
            0 & BA \\
            AB & 0
        \end{bmatrix} \Big)$ & $[A,B]=0$ &\cref{thm:general-adder-error}\\
     \qquad$ \exp it (AB + (AB)^\dagger) \sigma^x\Big)$ & \\
         $\textrm{BCH}(S \mathcal{B}_B(t) S^\dagger, X \mathcal{B}_A(t) X)$ & $\exp \left(it \begin{bmatrix}
            2 AB & 0 \\
            0 & -BA - (BA)^\dagger
        \end{bmatrix}\right)$ & $AB = (AB)^\dagger$ &\cref{lem:multiplication-alg}
        %%%%
        % $\mathcal{B}_A(t), \mathcal{B}_B(t)$ & $\textrm{Trotter}(\exp t [A, B] \sigma^y, \exp it \{ A, B \} \sigma^x) $ & $\exp 2 it \begin{bmatrix}
        %     0 & BA \\
        %     AB & 0
        % \end{bmatrix}$ 
    \end{tabularx}
    \caption{Overview of techniques for synthesizing particular unitary transformations and the quantum gates needed in order to build the constructs proposed in this paper. Each row contains the formula used, the target to approximate, the preconditions, and a reference to the location of the precise statement of the performance of the method. The formula provided denotes hybrid gates with $\sigma^i$ terms, while single-qubit gates are capitalized (e.g., $S, X, H$). The bounds on the number of gates depend on the accuracy required of the approximation and are given in the corresponding theorems. 
    % \how{ sigmai is xyz; single-qubits are capital, sigmas are part of exponentials. be definition repetitive} 
    %\how{move info to preconditions column}
    }
    \label{tab:my_label}
\end{table}
% \how{the table has a mistake on eq 46 and theorem 3.2 -- those are the same formulas and should be aligned}

% \begin{table}[]
%     \centering
%     \begin{tabularx}{\textwidth}{Xlc}
%         \hline
%         \textbf{Available gates} & \textbf{Formula} & \textbf{Target} \\
%         \hline
%         $\exp it A \sigma^j, \exp it B \sigma^j$ & $\textrm{BCH}(\exp it B \sigma^i, \exp it A \sigma^i)$ & $\exp t^2 [A, B] \identity$ \how{?} \\
%         %%%%
%         $\exp it A \sigma^j, \exp it B \sigma^j $ & $\textrm{BCH}(\exp it A \sigma^j, \exp it B \sigma^k) $ & $\exp t^2 \{ A, B \} \sigma^i$ \\
%         %%%%%
%         $\mathcal{B}_A (t), \mathcal{B}_B(t)$ & $\textrm{BCH}(\sigma^x \mathcal{B}_B(t) \sigma^x, \mathcal{B}_A(t) ) $ & $\exp (t^2 (AB - (AB)^\dagger) \sigma^z)$ \\
%         %%%%%%
%         $\mathcal{B}_A(t), \mathcal{B}_B(t)$ & $\textrm{BCH}(S \mathcal{B}_A(t) S^\dagger, \sigma^x \mathcal{B}_B(t) \sigma^x) $ & $\exp (i t^2 (AB + (AB)^\dagger) \sigma^z) $ \\
%         %%%%
%         $\mathcal{B}_A(t), \mathcal{B}_B(t)$ & $\textrm{BCH}(S \mathcal{B}_A(t) S^\dagger, \sigma^x \mathcal{B}_B(t) \sigma^x) $ & $\exp 2 i t^2 AB \sigma^z $ \footnote{When $AB$ Hermitian, i.e., $AB = (AB)^\dagger$. This occurs, for example, when $A, B$ are annihilation / creation operators of the same order.} \\
%         %%%%
%         $\mathcal{B}_A(t), \mathcal{B}_B(t)$ & $\textrm{Trotter}(\exp t (AB - (AB)^\dagger) \sigma^y, \exp it (AB + (AB)^\dagger) \sigma^x) $ & $\exp 2 it \begin{bmatrix}
%             0 & BA \\
%             AB & 0
%         \end{bmatrix}$  \\
%         %%%%
%         $\mathcal{B}_A(t), \mathcal{B}_B(t)$ & $\textrm{Trotter}(\exp t [A, B] \sigma^y, \exp it \{ A, B \} \sigma^x) $ & $\exp 2 it \begin{bmatrix}
%             0 & BA \\
%             AB & 0
%         \end{bmatrix}$ 
%     \end{tabularx}
%     \caption{Overview of contributed control techniques}
%     \label{tab:my_label}
% \end{table}

% \end{landscape}
% \how{update formula for second to last row}

% \how{should go in introduction (?)}

% \how{eq 29 has the def of mathcal B (linked)} 


\subsection{Phase-space methods: anti-commutators of Hermitian operators}
Here, we demonstrate how to synthesize anti-commutation relations between Hermitian operators, as is the case for the phase space operators $\hat{x}$ and $\hat{p}$ (but not Fock space operators). To do this, we use qubit gates to introduce the phase change necessary for obtaining an anticommutator. These methods can be used to  generate beamsplitters and the Hong-Ou Mandel effect, as is explored in~\cref{sec:applications}. 

% Our native gates on the hybrid device can be defined in terms of phase-space operators; our hope is to apply BCH on these native gates, thereby yielding commutators of $\hat{x}, \hat{p}$. It remains to show that we can also produce $\{ \hat{x}, \hat{p}\}$. 

Our technique applies the BCH formula on a hybrid boson-qubit operation. Thus, we need to study how the commutator affects both bosonic and qubit operators. In the bosonic case, observe:
\begin{align}
    [\hat{x}, \hat{p}] = \frac{i}{2}.
\end{align}
In the qubit case, observe that:
\begin{align}
    \sigma^{i} & =\epsilon_{ijk}\frac{i}{2}\left[\sigma^{j},\sigma^{k}\right],\label{eq:PauliCommutator}\\
    % \sigma^{x} & =-\frac{i}{2}\left[\sigma^{y},\sigma^{z}\right],\label{eq:PauliCommutatorx}\\
    % \sigma^{y} & =-\frac{i}{2}\left[\sigma^{z},\sigma^{x}\right],\label{eq:PauliCommutatory}\\
    % \sigma^{z} & =-\frac{i}{2}\left[\sigma^{x},\sigma^{y}\right]\label{eq:PauliCommutatorz}.
\end{align}
where $\epsilon_{ijk}$ is the Levi-Civita symbol, and $i,j,k \in \{x,y,z\}$.

We use these relations, as well as the Pauli product identity 
\begin{align}
\sigma^{i} & =\epsilon_{ijk}i\sigma^{j}\sigma^{k}.\label{eq:PauliProduct}
% \sigma^{x} & =-i\sigma^{y}\sigma^{z}\label{eq:PauliProductx},\\
% \sigma^{y} & =-i\sigma^{z}\sigma^{x}\label{eq:PauliProducty},\\
% \sigma^{z} & =-i\sigma^{x}\sigma^{y}\label{eq:PauliProductz},
\end{align}
to enable the expression of the product of a single Pauli operator and an anticommutator\emph{
}$\left\{ A,B\right\} \sigma_{i}$ in terms
of the product of modes and single Pauli operators in the form of a commutator, as follows:
\begin{align}
\left\{ A,B\right\} \sigma^{i} & = AB\left(i\sigma^{j}\sigma^{k}\right)+BA\left(i\sigma^{k}\sigma^{j}\right)\\
 & =i\left(-\left(A\sigma^{j}\right)\left(B\sigma^{k}\right)+BA\left(\sigma^{k}\sigma^{j}\right)\right)\\
 & =i\left(\left(iA\sigma^{j}\right)\left(iB\sigma^{k}\right)-\left(iB\sigma^{k}\right)\left(iA\sigma^{j}\right)\right)\\
 & =i\left[iA\sigma^{j},iB\sigma^{k}\right]\label{eq:AnticommutatortoCommutator-1}.
\end{align} 
% \begin{align}
% \left\{ A,B\right\} \sigma^{i} & =-AB\left(-i\sigma^{j}\sigma^{k}\right)+BA\left(i\sigma^{k}\sigma^{j}\right)\\
%  & =i\left(-\left(A\sigma^{j}\right)\left(B\sigma^{k}\right)+BA\left(\sigma^{k}\sigma^{j}\right)\right)\\
%  & =i\left(\left(iA\sigma^{j}\right)\left(iB\sigma^{k}\right)-\left(iB\sigma^{k}\right)\left(iA\sigma^{j}\right)\right)\\
%  & =i\left[iA\sigma^{j},iB\sigma^{k}\right]\label{eq:AnticommutatortoCommutator-1}
% \end{align} 
assuming $A$ and $B$ are Hermitian, and commute with $\sigma^{i}$ as is the case when $A, B$ are mode-only operators. Thus, by using a hybrid qubit-cavity operation of the form $\exp i A \sigma^j, \exp i B \sigma^k$, with $A$ and $B$ Hermitian, the BCH formula can convert commutators into anticommutators. 
% Intuitively, we have exploited the qubit to introduce a phase change.
% These relations are sufficient to demonstrate the following equality:
% \begin{align}
% \left\{ A,B\right\} \sigma^{i} & =i\left[iA\sigma^{j},iB\sigma^{k}\right]\label{eq:AnticommutatortoCommutator}
% \end{align}

% Finally, an argument consisting of a product of operators is expressed
% as
% \begin{align}
% AB\sigma^{j} & =\frac{1}{2}\left\{ A,B\right\} \sigma^{j}+\frac{1}{2}\left[A,B\right]\sigma^{j}\label{eq:ProductOperators}\\
% \left\{ A,B\right\}  & =AB+BA\\
% \left[A,B\right] & =AB-BA
% \end{align}
% which is further decomposed in terms of the Pauli anticommutator-commutator
% relation Eq.~(\ref{eq:AnticommutatortoCommutator}).

%%% the below section is omitted because it is not true
%%% [A, B] \sigma^j, \{ A, B \} \sigma^j cannot simultaneously be antihermitian unless one (or both) is zero.
% Finally, our results will require the creation of both commutators and anticommutators. We observe that, by block-encoding the matrices, anticommutators naturally follow from BCH. For example, given commuting operators $\Tilde{A}, \Tilde{B}$,
% \begin{align}
%     \sigma^x \left\{ \Tilde{A}, \Tilde{B} \right\}  = i \left[ i \sigma^y \Tilde{A} , i \sigma^z \Tilde{B}  \right]
% \end{align}
% Where $\sigma^i \Tilde{A}$ represents a hybrid qubit-boson operation. Thus, by taking $\Tilde{A}, \Tilde{B}$ to be readily implementable operations, many of which are defined in terms of $a, a^\dagger$ or $\hat{x}, \hat{p}$, it seems intuitive that we could implement sums of commutators / anticommutators of these operators.

% \subsection{Fock methods}

\subsection{Fock methods - non-linear terms via block-encodings}
In this section, we show how to achieve $A^q$ for an arbitrary bosonic operator $A$. This extends the prior techniques, which required that $A$ be Hermitian. When $A$ is a Fock space operator, this corresponds to realizing arbitrary powers of $a$ and $a^{\dagger}$, which is known to generate a universal set of operations on the bosonic modes. This is useful for the simulation of non-linear materials which naturally lead to terms that are polynomial in the creation and annihilation operators, as well as quantum signal processing~\cite{martyn2021grand}, for example. 

Our method again uses the qubit coupling to induce a phase in the compound qubit-boson system, similar to the previous section. We begin with block-encodings as described in Eq.~\ref{eq:blockencodedmatrix}. First, we show how to create auxiliary block encodings via qubit-only gates, a technique known as conjugation. Second, we show how to generate squeezing and non-linear terms using the BCH formula. Third, we note that qubit rotations lead to quadratic block encodings. Fourth, we show that these methods may be extended to multiple Hilbert spaces. Finally, we present a formal algorithm and associated error bound to attain these block-encodings.


% To achieve a similar effect with Fock operators $a, a^\dagger$, we can also use the qubit coupling to induce phase changes. In particular, we instead observe that $A \sigma^x$ is comparable to the following block encoding: \how{explain what a block encoding is}
% \begin{align}
%     \mathcal{B}_A(t) = \exp it \begin{bmatrix}
%         0 & A \\
%         A^\dagger & 0
%     \end{bmatrix},
% \end{align}
% where we require the lower block-encoding to be $A^\dagger$ to enforce hermiticity and thus unitarity of the operation. Observe that when we take $A = \hat{x}$ that $\mathcal{B}(t)$ is precisely $\exp i t A \sigma^x$. Suppose we took $A = a^\dagger$ so that\textcolor{blue}{Refer to equation (7)}:
% \begin{align}
%     \exp it \begin{bmatrix}
%         0 & a^\dagger \\
%         a & 0
%     \end{bmatrix}.
% \end{align}
% \how{what are we doing? why do we care about the X s x terms}

% While the phase-space block encoding could be expressed as a product between a qubit and bosonic operator due to the hermitian nature of the phase-space operator, the Fock-space block encoding is not as easily separable. Thus, we must be more careful about the intermixing of terms. 

% Luckily, we still can implement $\exp i t \{ A, B \} \sigma^j $, provided that $A, B$ commute. We also demonstrate an implementation of $\exp t [A, B] $. Together, these two constituents suggest that we may also implement $\exp it AB \sigma^j$ via the following relation:

To manipulate the block-encodings, begin by recognizing that qubit-only operations can modify the exponential via ``conjugation":
\begin{align}
    U e^A U^\dagger = e^{U A U^\dagger}.
\end{align}
Thus, given any block-encoding, we can also create the following auxiliaries:
\begin{align}
    X \exp it \begin{bmatrix}
        0 & A \\
        A^\dagger & 0
    \end{bmatrix} X &= \exp it \begin{bmatrix}
        0 & A^\dagger \\
        A & 0
    \end{bmatrix}, \\
    S \exp it \begin{bmatrix}
        0 & A \\
        A^\dagger & 0
    \end{bmatrix} S^\dagger &= \exp it \begin{bmatrix}
        0 & -i A \\
        i A^\dagger & 0
    \end{bmatrix},
\end{align}
where $S$ is a qubit phase gate. Applying BCH yields the following commutators:
\begin{align}
    \left[X \mathcal{B}(t) X, \mathcal{B}(t) \right] = \left[ it \begin{bmatrix}
        0 & A^\dagger \\
        A & 0
    \end{bmatrix} , it \begin{bmatrix}
        0 & A \\
        A^\dagger & 0
    \end{bmatrix} \right] &= - t^2 \left( \begin{bmatrix}
        (A^\dagger)^2 & 0 \\
        0 & A^2
    \end{bmatrix} - \begin{bmatrix}
        A^2 & 0 \\
        0 & (A^\dagger)^2
    \end{bmatrix} \right) \\
    &= t^2 \sigma^z \kron (A^2 - (A^\dagger)^2), \\
    \left[S \mathcal{B}(t) S^\dagger, X \mathcal{B}(t) X \right] = \left[ it \begin{bmatrix}
        0 & -i A \\
        i A^\dagger & 0
    \end{bmatrix}, it \begin{bmatrix}
        0 & A^\dagger \\
        A & 0
    \end{bmatrix} \right] &= -t^2 \left( \begin{bmatrix}
        - i A^2 & 0 \\
        0 & i (A^\dagger)^2
    \end{bmatrix} - \begin{bmatrix}
        i (A^\dagger)^2 & 0 \\
        0 & - iA^2
    \end{bmatrix} \right) \\
    &= i t^2 \sigma^z \kron (A + (A^\dagger)^2).
\end{align}
These commutators themselves can be conjugated. Recall that $H Z H = X$ and $SH Z HS^\dagger = Y$:
\begin{align}
    SH \left[X \mathcal{B}(t) X, \mathcal{B}(t) \right] HS^\dagger &=t^2 \sigma^y \kron (A^2 - (A^\dagger)^2), \\
    H \left[S \mathcal{B}(t) S^\dagger, X \mathcal{B}(t) X \right] H &= i t^2 \sigma^x \kron (A + (A^\dagger)^2),
\end{align}
so that:
\begin{align}
    i t^2 \sigma^x \kron (A + (A^\dagger)^2) + t^2 \sigma^y \kron (A^2 - (A^\dagger)^2) = 2 it^2 \begin{bmatrix}
        0 & (A^\dagger)^2 \\
        A^2 & 0
    \end{bmatrix}.
\end{align}
This approach can also be extended to cases where the constituent operators are of different types, e.g. when the synthesized unitary operates on two different modes, as in the conditional beamsplitter which is a gate that acts as a beamsplitter controlled on an ancillary qubit. For example, consider the following commutator:
\begin{align}
    \left[ i\tau \begin{bmatrix}
        0 & B^\dagger \\
        B & 0
    \end{bmatrix}, i\tau \begin{bmatrix}
        0 & A \\
        A^\dagger & 0
    \end{bmatrix} \right] &= -\tau^2 \left( \begin{bmatrix}
        B^\dagger A^\dagger & 0 \\
        0 & BA 
    \end{bmatrix} - \begin{bmatrix}
        AB & 0 \\
        0 & A^\dagger B^\dagger
    \end{bmatrix} \right) \\
    &= \tau^2 \begin{bmatrix}
        AB - (AB)^\dagger & 0 \\
        0 & (BA)^\dagger - BA
    \end{bmatrix}, \\
    %%%%%%
    \left[ i\tau \begin{bmatrix}
        0 & -iA \\
        iA^\dagger & 0
    \end{bmatrix}, i\tau \begin{bmatrix}
        0 & B^\dagger \\
        B & 0
    \end{bmatrix}\right] &= -\tau^2 \left( \begin{bmatrix}
        - iAB & 0 \\
        0 & i A^\dagger B^\dagger 
    \end{bmatrix} - \begin{bmatrix}
        i B^\dagger A^\dagger & 0 \\
        0 & -iBA 
    \end{bmatrix} \right) \\
    &= i\tau^2 \begin{bmatrix}
        AB + (AB)^\dagger & 0 \\
        0 & -BA - (BA)^\dagger
    \end{bmatrix}.
\end{align}

Observe that when $[A, B] = 0$, as is the case when $A, B$ are both annihilation/creation operators or when they act on different modes,  we can rewrite the commutators in the following forms:
\begin{align}
    \left[ i\tau \begin{bmatrix}
        0 & B^\dagger \\
        B & 0
    \end{bmatrix}, i\tau \begin{bmatrix}
        0 & A \\
        A^\dagger & 0
    \end{bmatrix} \right] = -\tau^2 \begin{bmatrix}
        AB - (AB)^\dagger & 0 \\
        0 & (AB)^\dagger - AB
    \end{bmatrix} &= \tau^2 (AB - (AB)^\dagger) \sigma^z, \\
    \left[ i\tau \begin{bmatrix}
        0 & -iA \\
        iA^\dagger & 0
    \end{bmatrix}, i\tau \begin{bmatrix}
        0 & B^\dagger \\
        B & 0
    \end{bmatrix}\right] = 
    i\tau^2 \begin{bmatrix}
        AB + (AB)^\dagger & 0 \\
        0 & - AB - (AB)^\dagger
    \end{bmatrix} &= i\tau^2 (AB + (AB)^\dagger) \sigma^z,
\end{align}
so that, again:
\begin{align}
    SH \tau^2 (AB - (AB)^\dagger) \sigma^z HS^\dagger &= \tau^2 (AB - (AB)^\dagger) \sigma^y \label{eq:C2},\\
    H i\tau^2 (AB + (AB)^\dagger) \sigma^z H &= i\tau^2 (AB + (AB)^\dagger) \sigma^x\label{eq:AC2}.
\end{align}
Via Trotter, we can finally implement the sum:
\begin{align}
    \tau^2 (AB - (AB)^\dagger) \sigma^y  + i\tau^2 (AB + (AB)^\dagger) \sigma^x   = 2 i \tau^2 \begin{bmatrix}
        0 & (AB)^\dagger \\
        AB & 0
    \end{bmatrix}\label{eq:Product}.
\end{align}
Finally, we select $\tau = \sqrt{\frac{t}{2}}$ to obtain the desired time and conjugate by $\sigma^x$ to produce the desired matrix. Observe that \cref{alg:adder} thus yields $\mathcal{B}_{AB}$, provided that $[A, B] = 0$. This process can be repeated iteratively, assuming $AB$ commutes with $B$; for example, if $A = B = a$, then this process can be used to produce higher powers $a^k, (a^\dagger)^k$ of the annihilation / creation operators.

\cref{alg:adder} is an extension of the prior commutator approaches in phase space because the $\sigma^i = - \frac{i}{2} [\sigma^j, \sigma^k]$ relation is natively expressed in the algorithm; i.e., if we have $\mathcal{B}_{A} = \mathcal{B}_B =  \mathcal{B}_{\hat{x}} = \exp it \hat{x} \sigma^x$, the ``$\textrm{Left}$" term vanishes and the``$\textrm{Right}$" term is the commutator we would apply.

% \how{...}

Finally, in \cref{alg:mult}, we demonstrate how to implement $AB$ if $AB = (AB)^\dagger$. This process cannot be performed recursively, contrary to \cref{alg:adder}, because it places the terms in the upper-left block encoding. However, this actually may be more useful, as it allows the precise simulation of $e^{i AB t}$, assuming the qubit begins in the $\ket{0}$ state:
\begin{align}
    \left[ i \tau \begin{bmatrix}
        0 & -i B^\dagger \\
        i B & 0
    \end{bmatrix}, i \tau \begin{bmatrix}
        0 & A \\
        A^\dagger & 0
    \end{bmatrix} \right] &= \tau^2 \begin{bmatrix}
        i B^\dagger A^\dagger + i A B  & 0 \\
        0 &  - i B A -i A^\dagger B^\dagger
    \end{bmatrix} \\
    &= i \tau^2 \begin{bmatrix}
        2 AB & 0 \\
        0 & -BA - (BA)^\dagger
    \end{bmatrix}.
\end{align}
% \how{is the bottom right block equality true? I don't think so}

%%%
% The product of two Hermitian matrices is only Hermitian when [A, B] = 0 because (AB)dag = Bdag Adag = B A
 
% Is it that:
% \begin{align}
%     AB = (AB)^\dagger \implies BA = (BA)^\dagger
% \end{align}
% \begin{align}
%     BA = BA - AB + AB = AB - [A, B]
% \end{align}
% So that:
% \begin{align}
%     (AB - [A, B])^\dagger \implies [A, B] = ([A, B])^\dagger
% \end{align}
% When is it that the commutator is also Hermitian?

\subsection{Error analysis}
The prior description of our algorithm assumes error-less product formulas. However, the BCH and Trotter formulas actually introduce errors which must be accounted for, especially when applying our algorithm recursively. In this section, we cite the error scaling of the general addition algorithm described in \cref{alg:adder} and the multiplication algorithm described in \cref{alg:mult}. The proofs and full results are included in \cref{apndx:error-analysis}.

\begin{algorithm}[t]
\caption{ADD($\mathcal{B}_A (t), \mathcal{B}_B(t), p_l, p_r, t$) }\label{alg:adder}
\begin{algorithmic}
\Require $[A, B] = 0$, $\norm{\mathcal{B}_A - \exp it \begin{bmatrix}
    0 & A \\
    A^\dagger & 0
\end{bmatrix} } \in \mathcal{O}((c_A t)^{p_A})$, $\norm{\mathcal{B}_B - \exp it \begin{bmatrix}
    0 & B \\
    B^\dagger & 0
\end{bmatrix} } \in \mathcal{O}((c_B t)^{p_B})$, $p_A, p_B \geq 1$, $t > 0$
\Ensure $\mathcal{B}_{AB}$ where $\norm{\mathcal{B}_{AB}(t) - \exp it \begin{bmatrix}
    0 & AB \\
    (AB)^\dagger & 0
\end{bmatrix}} \in \mathcal{O}((Ct)^{\min (p_A, p_B) / 2})$ for constant $C$
\State $q \coloneqq \max ( \ceil{\frac{1}{2}(\min(p_l, p_r) - 1)}, 1 )$
\State $s \coloneqq \max ( \ceil{\frac{1}{2} ( \min(p_l, p_r) - 1)}, 1 )$
\State $\tau \coloneqq \sqrt{t / 2}$
%%%%%%
\State Left $\coloneqq \bch_{q, 1}(X \cdot \mathcal{B}_B(t) \cdot X, \mathcal{B}_A(t))$
\State Right $\coloneqq \bch_{q, 1}(S\cdot \mathcal{B}_A(t) \cdot S^\dagger, X \cdot \mathcal{B}_B(t) \cdot X)$
\State Left' $\coloneqq SH \cdot \textrm{Left} \cdot HS^\dagger$
\State Right' $\coloneqq H \cdot \textrm{Right} \cdot H$ 
%%%%
\State \Return $X \cdot \trotter_s(\textrm{Left'}, \textrm{Right'}) \cdot X$
\end{algorithmic}
\end{algorithm}

% \how{coloneqq instead of left arrow for gets?}

\begin{algorithm}[t]
\caption{MULT($\mathcal{B}_A (t), \mathcal{B}_B(t), p_l, p_r, t$) }\label{alg:mult}
\begin{algorithmic}
\Require $AB = (AB)^\dagger$, $p_l, p_r \geq 1$, $t > 0$
\Ensure An upper-left block encoding $\mathcal{M}_{AB}$ where $\norm{\mathcal{M}_{AB}(t) - \exp it \begin{bmatrix}
    AB & 0 \\
    0 & \frac{1}{2} ( - BA - (BA)^\dagger)
\end{bmatrix}} \in \mathcal{O}( (C^2 t)^{\min (p_A, p_B) / 2} )$ for constant $C$
\State $q \coloneqq \max ( \ceil{\frac{1}{2}(\min(p_l, p_r) - 1)}, 1 )$
\State $\tau \coloneqq \sqrt{t / 2}$
\State \Return $\textrm{BCH}_q(S \mathcal{B}_{B}(\tau) S^\dagger, X \mathcal{B}_{A}(\tau) X) $
\end{algorithmic}
\end{algorithm}

% \begin{theorem}[Analysis of \cref{alg:adder}]\label{lem:adder}
% Assume we can implement the following $k_l, k_r$th order approximations of $S$ with error scaling $p_l, p_r$:
% \begin{align}
%     \norm{- \mathcal{S}_{k_l}(t)} &\in \mathcal{O}((c t)^{p_l}) \\
%     \norm{ - \mathcal{S}_{k_r}(t)} &\in \mathcal{O}((c t)^{p_r})
% \end{align}
% With $c \geq \Lambda^{\max(k_l, k_r) / 2}$. Then, we can implement higher order operators with comparable $t$ scaling:
% % \textbf{The notation below could use a definition maybe separately, and unification with the SX notation used before.  S seems to be the same concept but just with a different time argument.}
% \begin{align}
%     \norm{
%         \exp \left( i t \begin{bmatrix}
%     0 & (P_\Lambda a^\dagger P_\Lambda)^{k_l + k_r} \\
%     (P_\Lambda a P_\Lambda)^{k_l + k_r} & 0
%     \end{bmatrix}
%     \right) - \widetilde{\mathcal{S}}_{k_l + k_r, \min(p_l, p_r)}(t)
%     } \in \mathcal{O}((c^2 t)^{\min(p_l, p_r) / 2})
% \end{align}
% Using no more than $1.07 \cdot 30^q$ $\mathcal{S}_{k_l},  \mathcal{S}_{k_r}$ operators.
% \end{theorem}

% \how{COPY / PASTE UPDATED PROOF} FROM \cref{thm:general-adder-error}


% We apply the above results to produce the error analysis of \cref{alg:adder}:
\begin{restatable}{theorem}{algproduct}\label{thm:general-adder-error}
    Suppose we have approximations $\widetilde{\mathcal{B}}_A(t), \widetilde{\mathcal{B}}_B(t)$ with the following error scaling:
    \begin{align}
        \norm{\widetilde{\mathcal{B}}_A(t) - \mathcal{B}_A(t)} &\in \mathcal{O}( (ct)^{p_A} ), \\
        \norm{\widetilde{\mathcal{B}}_B(t) - \mathcal{B}_B(t)} &\in \mathcal{O} ((ct)^{p_B} ), 
    \end{align}
    for some constant $c$ and order $p_A, p_B \geq 1$ where $[A,B]=0$. Then, the application of \cref{alg:adder} will yield the following scaling:
    \begin{align}
        \norm{\textrm{ADD}(\widetilde{\mathcal{B}}_A(t), \widetilde{\mathcal{B}}_B(t) ) - \exp it \begin{bmatrix}
            0 & AB \\
            BA & 0 
        \end{bmatrix}} \in \mathcal{O} \left((C_{TOTAL} t)^{\min(p_A, p_B) / 2}\right),
    \end{align}
    with $C_{TOTAL} = \max( \norm{AB}, \norm{BA}, C_{BCH}^2)$ and $C_{BCH} = \max( \norm{A}, \norm{B}, c)$,
    using no more than $ 1.07 \cdot 30^q $ exponentials, where $q = \max (\ceil{\frac{\min(p_1, p_2) - 1}{2}}, 1) $. 
\end{restatable}


% ..

\begin{restatable}{theorem}{algmult}\label{lem:multiplication-alg}
    Suppose we have some approximate block encodings $\widetilde{\mathcal{B}}_A, \widetilde{\mathcal{B}}_B$  with the following error:
    \begin{align}
        \norm{\widetilde{\mathcal{B}}_A(t) - \exp it \begin{bmatrix}
            0 & A \\
            A^\dagger & 0
        \end{bmatrix}} &\in \mathcal{O}( (ct)^{p_A}), \\
        \norm{\widetilde{\mathcal{B}}_B(t) - \exp it \begin{bmatrix}
            0 & B \\
            B^\dagger & 0
        \end{bmatrix}} &\in \mathcal{O} ((ct)^{p_B}),
    \end{align}
    for constant $c$ and $p_A, p_B \geq 1$ where $AB = (AB)^\dagger$.
    Then, \cref{alg:mult} has the following error:
    \begin{align}
        \norm{ \textrm{MULT}(\widetilde{\mathcal{B}}_A(t),  \widetilde{\mathcal{B}}_B(t)) - \exp it \begin{bmatrix}
            AB & 0 \\
            0 & \frac{1}{2}( - BA - (BA)^\dagger)
        \end{bmatrix}} \in \mathcal{O}\left((C^2 t)^{\min (p_A, p_B) / 2}\right),
    \end{align}
    % \begin{align}
    %     \norm{ \textrm{BCH}_q(S \widetilde{\mathcal{B}}_B(\tau) S^\dagger, \widetilde{\mathcal{B}}_A(\tau) ) - \exp it \begin{bmatrix}
    %         AB & 0 \\
    %         0 & - BA - (BA)^\dagger
    %     \end{bmatrix}} \in \mathcal{O}((C^2 t)^{\min (p_A, p_B) / 2})
    % \end{align}
    with $C = \max(\norm{A}, \norm{B}, c)$, using no more than $8 \cdot 6^{q - 1}$ exponentials where $q = \max ( \ceil{\frac{\min(p_A, p_B) - 1}{2}}, 1) $.
\end{restatable}

While the asymptotic error analysis suggests that the cost of this method is onerous, we note that the product formulas often have overly pessimistic error scaling and operation counts \cite{zhao2021hamiltonian}. In the applications below, we provide numerical simulations which suggest our technique is more readily implementable than theory suggests.



% \subsection{Parallel between Fock methods and phase-space methods}


%% Chris' section 
% Do we want/can we make an error analysis about BCH error in Micheline's solutions?

%% Thm Corrollary are scary! May need to remove 
%% also need to spell out more of the intermediate steps for the derivations

\section{Applications}\label{sec:applications}
In this section, we show how our technique is a powerful tool for analytically realizing desired operations. This technique not only works for Hamiltonian simulation problems, but also for general control problems. In particular, we show how the physical intuition for a desired transformation is often sufficient to produce an approach to create desired operations.

% \how{move to four applications}
We introduce applications in both phase- and Fock-space. Phase-space techniques may be useful in the case where, for example, displacements ($e^{(\alpha a^\dag + \alpha^* a)} = e^{i \alpha \vec{x}}$ for $\alpha$ real or $e^{i \alpha \vec{p}}$ for $\alpha$ imaginary) are the only experimentally available gates. We combine these position $\vec{x}$ and momentum $\vec{p}$ operators with single-qubit rotations $\vec{\sigma}$ to produce the phase-space rotation operator, earlier referred to as the controlled parity operator, $e^{i a^{\dagger}a \sigma^z t}$; the beamsplitter $e^{-i\sigma^z (a^{\dagger}b + ab^{\dagger}) t}$; gates for two encodings of universal control of the restricted $\text{span}\{\left|0\right>,\left|1\right>\}$ Hilbert space (\cref{subsec:Universal-Control}); and gates for simulation of Fermi-Hubbard lattice dynamics using the vacuum and $1^\text{st}$ Fock states of the cavity (\cref{sec:Fermi-Hubbard-Lattice-Dynamics}), including same-site, hopping, controlled-beamsplitter (\cref{subsec:controlled-phase}), and FSWAP gates. For applications in Fock space, we use qubit-controlled displacements $e^{\sigma^z(\alpha^* a + \alpha a^{\dagger})}$, controlled parity maps $e^{\sigma^z a^{\dagger}a}$ and single-qubit operations $\vec{\sigma}$ to produce polynomials of annihilation and creation operators, i.e., $a^p {a^\dagger}^q$ for integer $p, q$. We then demonstrate how polynomials of these operators can be used in Hamiltonian simulation (e.g. with $\chi^{(3)}$ nonlinear materials, \cref{apndx:error-jc}) and state preparation (\cref{subsec:state-prep}).


% For Hamiltonian simulation, we demonstrate a variety of potential systems which are simulatable,  
% We also contribute a variety of control techniques which are realizable, including the conditional rotation gate (\cref{subsec:cond-rot}), controlled-phase beam splitter gate (\cref{subsec:controlled-phase}) state preparation of arbitrary Fock states (\cref{subsec:state-prep}), and the embedding of an effective qubit in the bosonic mode (\cref{subsec:Universal-Control}).

% \how{ck: need to link into the missing appendices, describe the type of ham sim / control being done and link it}

\subsection{Nonlinear Hamiltonian simulation}
As a simple application, let us consider the case of simulating a $\chi^{(3)}$ nonlinear material. These interactions commonly occur in nonlinear optics and appear when the index of refraction for a material varies linearly with the intensity of the electromagnetic field.  Such interactions can be modeled for a single mode using the following expression
\begin{equation}\label{eq:desired-ham}
    H = \omega a^\dagger a + \frac{\kappa}{2}(a^\dagger)^2 a^2.
\end{equation}
Our goal here is to examine the cost of a simulation of such a Hamiltonian in our model for time $t$ and error tolerance $\epsilon$ and determine the parameter regimes within which a hybrid simulation using our techniques could provide an advantage with respect to a conventional qubit-based simulation of the Hamiltonian. 

Each of the terms can be approximated by using the BCH formula. For the $\omega a^\dagger a$ term, observe that we can approximate this term with a single application of the BCH formula. Namely,
\begin{align}
    \left[ i \tau_1 \begin{bmatrix}
        0 & -i  a^\dagger  \\
        i  a  & 0
    \end{bmatrix},
    i \tau_1 \begin{bmatrix}
        0 &  a^\dagger  \\
         a  & 0
    \end{bmatrix}  \right] = 2 i \tau_1^2 \begin{bmatrix}
         a^\dagger   a  & 0 \\
        0 & - a a^\dagger  
    \end{bmatrix},
\end{align}
where the first matrix corresponds to $\mathcal{S}_1^y = S \cdot \mathcal{S}_1 \cdot S^\dagger $ operator and the second matrix corresponds to $\mathcal{S}_1$ operator. Similarly, observe that:
\begin{align}
    \left[ i \tau_2 \begin{bmatrix}
        0 & -i (a^\dagger)^2 \\
        i a^2 & 0
    \end{bmatrix}, i \tau_2 \begin{bmatrix}
        0 & (a^\dagger)^2 \\
        a^2 & 0
    \end{bmatrix} \right] = 2 i \tau_2^2 \begin{bmatrix}
        (a^\dagger)^2 a^2 & 0 \\
        0 & - a^2 (a^\dagger)^2
    \end{bmatrix}.
\end{align}
Thus, via BCH formula, we can block-encode the two Hamiltonian terms. Applying Trotterization allows us to block-encode the entire Hamiltonian into the upper-left quadrant. Thus, by setting the qubit to $\ket{0}$, we can approximate the Hamiltonian. The error scaling is as follows and is proven in \cref{apndx:error-jc}

\begin{restatable*}{theorem}{resultjc}\label{app:jaynes-cummings}
 Let
$
    H = \omega a^\dagger a + \frac{\kappa}{2}(a^\dagger)^2 a^2
$, $t$ be an evolution time and $\epsilon$ be an error tolerance.
For any positive integer $q$ we can approximate an exponential of the block-encoded Hamiltonian with error at most $\epsilon$ in the operator norm using $r e^{\mathcal{O}(q)}$ $\mathcal{S}_1$ operations where $r \in \Omega\left( \frac{(\Lambda^{4} t)^{1 + 1 / (q - \frac{3}{4})} }{\epsilon^{1 / (q - \frac{3}{4})}} \right)$. 
% \how{is it reOq}
\end{restatable*}


%\how{nathan: what numerics do we want here?}

% get plots from \href{https://colab.research.google.com/drive/1yxOpabROaSCA-vOkNKu6QlYNxOuFtylm#scrollTo=Ff4t89vuDJEt}{here}

This shows that we can perform a simulation of the dynamics within error $\epsilon$ using a number of operations within our instruction set that scales near-linearly with the evolution time and subpolynomially with $\epsilon$.  Further, this approach requires no ancillary memory and can be done with a single oscillator and a qubit.  This is a dramatic memory reduction relative to the quantum case, which requires a polylogarithmic number of qubits in $\Lambda$.

It is worth noting that in this case the ancillary qubit is not being used directly in the model.  Instead it is being used to control the dynamics and generate the appropriate nonlinear interaction between the photons present in the model.

\subsection{Nondestructive measurement of the qumode}
% \how{Notes for reviewer:}
% \begin{enumerate}
%     \item Where do we describe the full ISA?
%     \item 
% \end{enumerate}

% ---
We now demonstrate how the approach can extend beyond problems in Hamiltonian simulation. We begin with an example of the technique for control. In particular, we seek to perform a nondestructive measurement of the qumode. Natively, measurements are destructive, i.e., counting the number of photons also consumes the photons. However, by using the qubit, we can instead project the information into the qubit~\cite{fockreadout_wang_2020, Fockshotresolved_curtis_2021}. 

% \how{is this actually a valid application}

Intuitively, we seek to implement  $e^{i t \hat{n} \sigma^z}$ where $\hat{n} = a^\dagger a$ is the number operator. If we could implement this gate for arbitrary $t$, we could perform phase estimation on the qubit to nondestructively project the qumode into a fixed number of bosons. This could be done by setting $t$ sufficiently small so that $t \Lambda \leq 2 \pi$ is calculable with phase estimation. Alternatively, for $t = \pi$, this operation checks the parity of the qumode and applies an RZ gate for odd parities. We employ the instruction set in the phase-space representation to synthesize the infinitesimal conditional rotation gate
\begin{equation}
U_{\text{rot},k}=e^{i\lambda^{2}\hat{n}\sigma^{k}}
\end{equation}
for $k=x,y,z$. We rewrite $\hat{n}$ in terms of the phase-space operators by recognizing:
\begin{align}
\hat{n} & =\hat{a}^{\dagger}\hat{a}\\
 & =\hat{x}^{2}+\hat{p}^{2}-\frac{1}{2}\label{eq:NumbertoPhaseSpace}.
\end{align}
Thus, by applying Eq.~\ref{eq:NumbertoPhaseSpace},
Eq.~\ref{eq:CreationOperatortoPhaseSpace}, and Eq.~\ref{eq:AnnihilationOperatortoPhaseSpace}
yields
\begin{equation}
i\lambda^{2}\hat{n}\sigma^{k}=i\left(\hat{x}^{2}+\hat{p}^{2}-\frac{1}{2}\right)\sigma^{k}\lambda^{2},
\end{equation}
such that the gate is expressed via the Trotter-Suzuki decomposition as the product of $\exp\left(\left[A_{1},B_{1}\right]\right)=\exp\left(i\lambda^{2}\hat{x}^{2}\sigma^{k}\right)$,
$\exp\left(\left[A_{2},B_{2}\right]\right)=\exp\left(i\lambda^{2}\hat{p}^{2}\sigma^{k}\right)$,
and conditional displacement $\exp\left(-i\lambda^{2}\sigma^{k}/2\right)$.
Given the Pauli commutator relation, the
first commutator is 
\begin{align}
\left[A_{1},B_{1}\right] & =i\hat{x}^{2}\sigma^{k}\\
 & =i\hat{x}^{2}\left(-\frac{i}{2}\left[\sigma^{i},\sigma^{j}\right]\right)\\
 & =\left[\frac{1}{\sqrt{2}}\hat{x}\sigma^{i},\frac{1}{\sqrt{2}}\hat{x}\sigma^{j}\right],
\end{align}
and the second commutator is 
\begin{align}
\left[A_{2},B_{2}\right] & =i\hat{p}^{2}\sigma^{k}\\
 & =i\hat{p}^{2}\left(-\frac{i}{2}\left[\sigma^{i},\sigma^{j}\right]\right)\\
 & =\left[\frac{1}{\sqrt{2}}\hat{p}\sigma^{i},\frac{1}{\sqrt{2}}\hat{p}\sigma^{j}\right],
\end{align}
such that both terms are amenable to BCH decomposition,
and the infinitesimal conditional rotation is composed with a gate-depth lower bound of nine. To perform an error analysis, we may directly apply the error scaling of BCH and Trotter to find:

\begin{restatable*}{theorem}{resultmeasurement}
Suppose we can implement $\text{e}^{ it \hat{x} \sigma^i}, \text{e}^{ it \hat{p} \sigma^i}$ without error. Then, we may approximate $\mathcal{B}_{\hat{x}^2 + \hat{p}^2}$ with arbitrary error scaling $p$:
\begin{align}
    \norm{ \widetilde{\mathcal{B}}_{\hat{x}^2 + \hat{p}^2} - \mathcal{B}_{\hat{x}^2 + \hat{p}^2} } \in \mathcal{O}( (C t)^{p + 1/ 2}),
\end{align}
where $C = \max( \norm{ \hat{x}^2 + \hat{p}^2 }, \norm{\hat{x}}^2, \norm{\hat{p}}^2)$ and using no more than $4 \cdot 5^{ \frac{p}{2} - \frac{1}{4} }$ exponentials.
\end{restatable*}
% Where the Pauli rotation can be applied without implementation error or Trotter error. 

We then provide numerics in \cref{fig:ConditionalPhase}. As expected, the wavefunction
initialized in the second excited state of the cavity and the ground
state of the associated transmon has an autocorrelation function that
oscillates with phase $\exp(2it)$. Dynamics are well reproduced with
$2000$ time steps for a final time of $20$ with
$15$ states in the cavity. Note the units are arbitrary in the absence
of definition of the cavity frequency $\omega$, with the only units
defined by setting the reduced Planck constant to unity $\hbar=1\text{ arb. units}$.
The close agreement between the BCH-synthesized and exact gates is
supported by the error scaling after a single gate application computed for time step $t$, which features a power law scaling in agreement with the predicted error scaling
for both BCH and Trotter decompositions.\begin{figure}[!ht]
\begin{centering}
\includegraphics[width=0.5\columnwidth]{pics/ConditionalRotation/bch.png}\includegraphics[width=0.5\columnwidth]{pics/ConditionalRotation/error.png}
\par\end{centering}
\caption{The following plots characterize the performance of using phase-space operators to synthesize $e^{i \hat{n} \sigma^z t}$. (a) The BCH-synthesized conditional rotation gate $e^{i\hat{n}\sigma^{z}t}$
successfully reproduces the exact dynamics for a wavefunction initialized
in the ground state of the transmon and the second excited state of
the cavity.  (b) Error of the real part of the autocorrelation
function for the BCH-synthesized gate after a single time step of length $t$.\label{fig:ConditionalPhase}}
\end{figure}


% \how{will need to show how this can also be done via the fock-space operators}

We can obtain a similar decomposition with Fock-space operators. Observe that the $\textrm{MULT}$ subroutine applied to the $\mathcal{B}_a$, $\mathcal{B}$ would yield the following operators:
\begin{align}
	\norm{\textrm{MULT}(\mathcal{B}_a, \mathcal{B}_{a^\dagger}) - \exp i t \begin{bmatrix}
		a^\dagger a & 0 \\
		0 & - a a^\dagger
	\end{bmatrix} }.
\end{align}
Note that $a a^\dagger = a^\dagger a + \identity$ provided we understand this operator to be acting on vectors that have no support on the singularity,  so the block encoding that is actually applied is actually the following:
\begin{align}
\exp i t\left\{\begin{bmatrix}
	a^\dagger a & 0 \\
	0 & - a^\dagger a 
\end{bmatrix} + \begin{bmatrix}
	0 & 0 \\
	0 & - \identity
\end{bmatrix}\right\} = \exp it ( \hat{n} \sigma^z - \identity_\gamma (\identity - \sigma^z) ).
\end{align}
Thus, our Fock-space methods would also achieve the same transformation, albeit requiring a phase and RZ correction. 
% \how{error scaling}


\subsection{State preparation from the vacuum}
Consider the case where we seek to prepare $\ket{k}_b$ on the qumode. On qubit devices, this sort of state prep is trivial, assuming the qubits represent logic in a binary fashion. However, hybrid boson-qubit devices natively implement exponentials of the phase-space or Fock-space operators. Thus, preparing $\ket{k}_b$ directly can often be challenging. 

To begin, we recognize that we intuitively aim to implement $(a^\dagger)^k$ on the vacuum. It is sufficient to approximate:
\begin{align}
    \mathcal{T}_k(t) \coloneqq \exp i t\begin{bmatrix}
        0 & (a^\dagger)^k \\
        a^k & 0
    \end{bmatrix}.
\end{align}
Because selecting appropriate $t$ yields precisely the desired behavior, we have the following result:
\begin{restatable*}{theorem}{statepreptime}
    For $k \leq \Lambda$, we can take $t = (2n + 1) \frac{\pi}{2 \sqrt{k!}}$ for any $n \in \mathbb{N}$ so that:
    \begin{align*}    
    \mathcal{T}_k(t) \ket{1} \kron \ket{0} = \ket{0} \kron \ket{k}. 
    \end{align*}
\end{restatable*}
Note here that we can implement such a mapping using the Baker-Campbell-Hausdorff formula (BCH).  



\newcommand{\diag}{{\rm{diag}}}
\newcommand{\stateprep}{\mathcal{P}_k}
\newcommand{\stateprepapprox}{\widetilde{\mathcal{P}}_{k, p}}

While $\mathcal{T}_k (t)$ performs the desired transformation, it may incur unwanted side effects if the starting state is of the form $\ket{1} \kron \ket{b}$ for $b > 0$. We can use our same approach to produce the following operation:

\begin{restatable*}{theorem}{resultstateprep}\label{lem:fock-prep-unitary}
Consider the Fock preparation unitary $\stateprep$ with the following form:
\begin{align*}
    \exp \left( i t \begin{bmatrix}
        0 & ( a^\dagger )^k \ketbra{0}{0} \\
        \ketbra{0}{0} ( a )^k & 0
    \end{bmatrix} \right).
\end{align*}
% 
% \begin{align}
%     \exp \left( it \begin{bmatrix}
%         0 & p_{k, 1} \\
%         p_{1, k} & 0 
%     \end{bmatrix} \right)
% \end{align}
% Where $p_{1, k}$ refers to a $(\Lambda + 1) \times (\Lambda + 1)$ matrix where the only element is in the $1$st row and $k$th column with value $\sqrt{k!}$ ($p_{k, 1}$,  respectively). 
%
When $t = (2n + 1) \frac{\pi}{4 \sqrt{k!}} $, we have that $\stateprep$ performs our desired state preparation:
\begin{align*}
    \exp \left( i t \begin{bmatrix}
        0 & ( a^\dagger )^k \ketbra{0}{0} \\
        \ketbra{0}{0} ( a )^k & 0
    \end{bmatrix} \right) \ket{1} \kron \ket{b}  = \begin{cases}
    \ket{0} \kron  \ket{k} & b = 0 \\
    \ket{1} \kron  \ket{b} & b \neq 0
    \end{cases}.
\end{align*}
We claim that we can approximate this unitary with $\stateprepapprox$ where:
\begin{align*}
    \norm{\stateprepapprox - \stateprep} \in \mathcal{O}((\Lambda^{k/2} t)^p),
\end{align*}
using no more than $4 \cdot 5^{q -1}$ $\widetilde{\mathcal{T}}_{k,p}$ subroutines.

% \begin{align}
%     \exp \left( it \begin{bmatrix}
%         0 & p_{0, k} \\
%         p_{k, 0} & 0 
%     \end{bmatrix} \right) \ket{1} \ket{b} \mapsto \begin{cases}
%     \ket{0} \ket{k} & b = 0 \\
%     \ket{1} \ket{b} & b \neq 0
%     \end{cases}
% \end{align}
\end{restatable*}
Where the proofs are provided in \cref{state_prep_proof}. Though this subroutine appears expensive, numerical results suggest it is far more implementable than theory would suggest. In the following simulations, we apply the above technique but always use a second-order symmetrized BCH formula and second-order (symmetrized) Trotter formula. This amounts to 480 exponentials for the unprotected case and 960 exponentials for the protected case. The error plot is provided in \cref{fig:unprotectedt2} and \cref{fig:protectedt2}.  
\begin{figure}[!ht]
    \centering
    \includegraphics[scale=0.6]{pics/stateprep/originalT2gate.png}
    \caption{Unprotected $T_2$ Gate: (a) is the exact form of the $T_2$ gate with selected $t$ and (b) is the BCH-synthesized form. The state $\ket{j}_q \ket{k}_m$ has index $j \cdot \Lambda + k$ where $\Lambda$ is the cutoff. By observation, the analytically realized form is accurate.}
    \label{fig:unprotectedt2}
\end{figure}
\begin{figure}[!ht]
    \centering
    \includegraphics[scale=0.6]{pics/stateprep/fullT2gate.png}
    \caption{Protected $T_2$ Gate: (a) is the exact form of the protected $T_2$ gate with selected $t$ and (b) is the BCH-synthesized form.   The state $\ket{j}_q \ket{k}_m$ has index $j \cdot \Lambda + k$ where $\Lambda$ is the cutoff. By observation, the analytically realized form is still accurate, albeit with more incurred Trotter error.}
    \label{fig:protectedt2}
\end{figure}

% \includegraphics[scale=0.6]{pics/draft_t2_unprotected.png}

% \includegraphics[scale=0.6]{pics/draft_t2_protected.png}

We also analyze the error scaling as the order of the BCH formulas used increases. \cref{fig:protectedt2-error} describes the error-resource tradeoff as the Trotter step within each BCH formula increases.
\begin{figure}[!ht]
    \centering
    \mbox{
    \subfigure[]{\includegraphics[scale=0.5]{pics/stateprep/errorfidT2norm.png}}\quad
    \subfigure[]{\includegraphics[scale=0.5]{pics/stateprep/errorT2norm.png}}
    }
    \caption{Error scaling of protected $T_2$ gate with respect to BCH circuit depth. Prime denotes that the formula has been symmetrized. (a) uses the state-fidelity metric while (b) uses the matrix norm.}
    \label{fig:protectedt2-error}
\end{figure}

\subsection{Hong-Ou-Mandel effect/conditional (controlled-phase) beam splitter gate}\label{subsec:Conditional-(Controlled-Phase)-Beam-Splitter}
% \how{should be rewritten}
The operations we seek to realize need not act on a single mode; in fact, our techniques are extensible to hybrid setups with multiple modes or qubits. Consider the conditional (controlled-phase) beam splitter
\begin{align}
U_{\text{beam split}} & =e^{-i\lambda^{2}\left(\hat{a}_{1}^{\dagger}\hat{a}_{2}+\hat{a}_{1}\hat{a}_{2}^{\dagger}\right)\sigma^{z}}.
\end{align}
This gate naturally pertains
to certain lattice gauge theories and gives rise to exponential SWAP
(eSWAP) \cite{gao2019entanglement} and controlled-SWAP (cSWAP)
gates for state purification and SWAP tests, when paired with an uncontrolled
beam splitter \cite{pietikainen2022controlled}. 
The argument in phase-space representation 
% Eq.~\ref{eq:CreationOperatortoPhaseSpace} and Eq.~\ref{eq:AnnihilationOperatortoPhaseSpace} 
is 
\begin{align}
-i\lambda^{2}\left(\hat{a}_{1}^{\dagger}\hat{a}_{2}+\hat{a}_{1}\hat{a}_{2}^{\dagger}\right)\sigma^{z} & =-2i\lambda^{2}\left(\hat{x}_{1}\hat{x}_{2}+\hat{p}_{1}\hat{p}_{2}\right)\sigma^{z},
\end{align}
such that the gate is decomposed in terms of a Trotter-Suzuki expansion
% Eq.~\ref{eq:TrotterFirstOrder} 
as the product of two exponential
terms $\exp\left(\left[A_{1},B_{1}\right]\lambda^{2}\right)=\exp\left(-2i\lambda^{2}\hat{x}_{1}\hat{x}_{2}\sigma^{z}\right)$
and $\exp\left(\left[A_{2},B_{2}\right]\lambda^{2}\right)=\exp\left(-2i\lambda^{2}\hat{x}_{1}\hat{x}_{2}\sigma^{z}\right)$.
According to the Pauli commutation relation 
the first commutator is
\begin{align}
\left[A_{1},B_{1}\right] & =-2i\hat{x}_{1}\hat{x}_{2}\sigma^{z}\\
 & =-2i\hat{x}_{1}\hat{x}_{2}\left(-\frac{i}{2}\left[\sigma^{x},\sigma^{y}\right]\right)\\
 & =\left[i\hat{x}_{1}\sigma^{x},i\hat{x}_{2}\sigma^{y}\right],
\end{align}
and the second is
\begin{align}
\left[A_{2},B_{2}\right] & =\left[i\hat{p}_{1}\sigma^{x},i\hat{p}_{2}\sigma^{y}\right],
\end{align}
with the following error scaling:

% \how{error bounds ...}
\begin{restatable*}{theorem}{resultbeamsplitter}
Assume we may implement $\text{e}^{ i t \hat{x}_m \sigma^j}, \text{e}^{ i t \hat{p}_m \sigma^j}$ for $m \in \{ 1, 2 \}$; i.e., we may implement the qubit-conditional position shifts and momentum boosts on either mode without error. Then, we may approximate $\mathcal{B}_{\hat{x}_1 \hat{x}_2 + \hat{p}_1 \hat{p}_2}$ with arbitrary error scaling $p$:
\begin{align*}\norm{\widetilde{\mathcal{B}}_{\hat{x}_1 \hat{x}_2 + \hat{p}_1 \hat{p}_2} - \mathcal{B}_{\hat{x}_1 \hat{x}_2 + \hat{p}_1 \hat{p}_2} } \in \mathcal{O}((Ct)^{p + \frac{1}{2}}),
\end{align*}
where $C = \max( \norm{ \hat{x}_1 \hat{x}_2 + \hat{p}_1 \hat{p}_2}, \norm{\hat{x}_1}^2, \norm{\hat{x}_2}^2, \norm{\hat{p}_1}^2, \norm{\hat{p}_1}^2)$ and using no more than $4 \cdot 5^{ \frac{p}{2} - \frac{1}{4} }$ exponentials.
\end{restatable*}


The two exponential terms are in decomposed via the BCH formula 
% Eq.~\ref{eq:BCHFormula}
for a lower-bound gate depth of eight. Results are shown in Fig.~\ref{fig:ConditionalBeamSplitter} for
$15$ states per cavity with a shared transmon over a final time of
$\pi/2$ with $200$ equal time
steps, where the system is initially in the first excited state of
each cavity and the ground state of the shared transmon $\left|11g\right>$.
As expected for the conditional beam splitter, the gate exhibits the
Hong-Ou-Mandel effect, in which the occupation of cavity 1 oscillates
between the first excited mode and a superposition of the ground and
the second excited states of the cavity. The BCH-synthesized results
closely agree with that of the original gate, with no visible leakage
beyond the physical states (the lowest three states of the cavity)
into the working space under the time duration studied. As for the
conditional rotation gate, the relative error of the BCH-synthesized
gate computed for a single time step of length $t$ was found to scale according to a power law with the time step,
in accordance with the analytic result for Trotterization and BCH
decomposition.

\begin{figure}[!ht]
\begin{centering}
\includegraphics[width=0.5\columnwidth]{pics/BeamSplitter/probabilityexact}\includegraphics[width=0.5\columnwidth]{pics/BeamSplitter/probabilitybch}
\par\end{centering}
\begin{centering}
\includegraphics[width=0.5\columnwidth]{pics/BeamSplitter/probabilityworkbch}\includegraphics[width=0.5\columnwidth]{pics/BeamSplitter/error}
\par\end{centering}
\caption{Hong-Ou-Mandel effect simulated with (a) exact and (b) BCH-synthesized
conditional beam splitters, illustrated as probability cavity 1 is found in states $\left|0\right>$, $\left|1\right>$, or $\left|2\right>$; (c) probability of leakage into higher cavity modes; and (d) error of the real part of the autocorrelation
after a single application of a BCH-synthesized gate for time step $t$ relative to the application of the exact gate for the same time step.\label{fig:ConditionalBeamSplitter}}%The notation with the single Fock states and two-mode Fock states is not explained.
\end{figure}



% \how{numerics}


{
% \section{Qubit-Conditional Cavity Gates\label{sec:Qubit-Conditional-Cavity-Gates}}

% We begin by employing the analytic ISA to synthesize qubit-conditional
% gates commonly required in bosonic quantum computing applications.

% \subsection{SUM Gate}

% Consider the two-cavity infinitesimal conditional SUM gate
% \begin{equation}
% U_{\text{SUM}}=e^{-i\lambda^{2}\hat{x}_{1}\hat{p}_{2}\sigma^{z}}
% \end{equation}
% where $\hat{x}_{1}$ is the position operator of cavity $1$, $\hat{p}_{2}$
% is the momentum operator of cavity $2$, and the Pauli-Z gate $\sigma^{z}$
% acts on a transmon coupled to both cavities. In Gottesman-Preskill-Knill
% (GKP) codes, the gate enables stabilizer measurements and CNOT operations
% \cite{royer2022encoding} and provides an $x_{1}$-position-dependent
% momentum boost $\hat{p}_{2}$ and equivalently a $p_{2}$-momentum-dependent
% position displacement $\hat{x}_{1}$.

% To decompose the exponential gate, its argument is expressed in terms
% of a commutator via the Pauli commutation relation Eq.~(\ref{eq:PauliCommutatorz})
% as follows:
% \begin{align}
% \left[A,B\right]\lambda^{2} & =-i\hat{x}_{1}\hat{p}_{2}\sigma^{z}\lambda^{2}\\
%  & =-\hat{x}_{1}\hat{p}_{2}\left(-\frac{i}{2}\left[\sigma^{x},\sigma^{y}\right]\right)\lambda^{2}\\
%  & =\left[\frac{i}{\sqrt{2}}\hat{x}_{1}\sigma^{x},\frac{1}{\sqrt{2}}\hat{p}_{2}\sigma^{y}\right]\lambda^{2}
% \end{align}
% \how{phat sigma y is not anti-hermitian}
% The gate is then expressed according to the BCH decomposition Eq.~(\ref{eq:BCHFormula})
% with gate depth lower bound of four.

% \subsection{Conditional Single-Cavity Squeezing Gate}

% The infinitesimal squeezing gate assumes the form

% \begin{equation}
% U_{\text{one-mode squeeze}}\approx e^{\lambda^{2}\left(\hat{a}^{\dagger2}-\hat{a}^{2}\right)\sigma^{z}}
% \end{equation}
% and aids generation of GKP states \cite{hastrup2021measurement,hastrup2021unconditional}.
% Given the relationship between the ladder and phase-space operators
% Eq.~\ref{eq:AnnihilationOperatortoPhaseSpace} and \ref{eq:CreationOperatortoPhaseSpace},
% the argument is expressed as
% \begin{align}
% \lambda^{2}\left(\hat{a}^{\dagger2}-\hat{a}^{2}\right)\sigma^{z} & =-2i\lambda^{2}\left\{ \hat{x},\hat{p}\right\} \sigma^{z}
% \end{align}
% which is in turn expressed as a commutator according to the Pauli
% anticommutator-commutator relation Eq.~\ref{eq:AnticommutatortoCommutator}
% \begin{align}
% \left\{ \hat{x},\hat{p}\right\} \sigma^{z}\lambda^{2} & =2i\left[i\hat{x}\sigma^{x},i\hat{p}\sigma^{y}\right]\lambda^{2}
% \end{align}
% to yield the exponential commutator
% \begin{equation}
% \left[A,B\right]\lambda^{2}=\left[\sqrt{2}i\hat{x}\sigma^{x},\sqrt{2}i\hat{p}\sigma^{y}\right]\lambda^{2}
% \end{equation}
% The infinitesimal squeezing gate is therefore decomposed according
% to the BCH formula Eq.~\ref{eq:BCHFormula} with lower-bound gate
% depth of four.

% \subsection{Conditional (Controlled-Phase) Beam Splitter Gate\label{subsec:Conditional-(Controlled-Phase)-Beam-Splitter}}

% Consider the conditional (controlled-phase) beam splitter
% \begin{align}
% U_{\text{beam split}} & =e^{-i\lambda^{2}\left(\hat{a}_{1}^{\dagger}\hat{a}_{2}+\hat{a}_{1}\hat{a}_{2}^{\dagger}\right)\sigma^{z}}
% \end{align}
% The argument in phase-space representation Eq.~\ref{eq:CreationOperatortoPhaseSpace}
% and Eq.~\ref{eq:AnnihilationOperatortoPhaseSpace} is 
% \begin{align}
% -i\lambda^{2}\left(\hat{a}_{1}^{\dagger}\hat{a}_{2}+\hat{a}_{1}\hat{a}_{2}^{\dagger}\right)\sigma^{z} & =-2i\lambda^{2}\left(\hat{x}_{1}\hat{x}_{2}+\hat{p}_{1}\hat{p}_{2}\right)\sigma^{z}
% \end{align}
% such that the gate is decomposed in terms of a Trotter-Suzuki expansion
% Eq.~\ref{eq:TrotterFirstOrder} as the product of two exponential
% terms $\exp\left(\left[A_{1},B_{1}\right]\lambda^{2}\right)=\exp\left(-2i\lambda^{2}\hat{x}_{1}\hat{x}_{2}\sigma^{z}\right)$
% and $\exp\left(\left[A_{2},B_{2}\right]\lambda^{2}\right)=\exp\left(-2i\lambda^{2}\hat{x}_{1}\hat{x}_{2}\sigma^{z}\right)$.
% According to the Pauli commutation relation Eq.~(\ref{eq:PauliCommutatorz}),
% the first commutator is
% \begin{align}
% \left[A_{1},B_{1}\right] & =-2i\hat{x}_{1}\hat{x}_{2}\sigma^{z}\\
%  & =-2i\hat{x}_{1}\hat{x}_{2}\left(-\frac{i}{2}\left[\sigma^{x},\sigma^{y}\right]\right)\\
%  & =\left[i\hat{x}_{1}\sigma^{x},i\hat{x}_{2}\sigma^{y}\right]
% \end{align}
% and the second is
% \begin{align}
% \left[A_{2},B_{2}\right] & =\left[i\hat{p}_{1}\sigma^{x},i\hat{p}_{2}\sigma^{y}\right]
% \end{align}
% The two exponential terms are in decomposed via the BCH formula Eq.~\ref{eq:BCHFormula}
% for a lower-bound gate depth of eight.

% \subsection{Conditional Beam Squeezer Gate}

% The infinitesimal conditional beam squeezer
% \begin{equation}
% U_{\text{two-mode squeeze}}=e^{-i\lambda^{2}\left(\hat{a}_{1}^{\dagger}\hat{a}_{2}^{\dagger}+\hat{a}_{1}\hat{a}_{2}\right)\sigma^{z}}
% \end{equation}
% follows analogously from the conditional beam splitter. The argument
% in phase-space representation is
% \begin{align*}
% -i\lambda^{2}\left(\hat{a}_{1}^{\dagger}\hat{a}_{2}^{\dagger}+\hat{a}_{1}\hat{a}_{2}\right)\sigma^{z} & =-2i\lambda^{2}\left(\hat{x}_{1}\hat{x}_{2}-\hat{p}_{1}\hat{p}_{2}\right)\sigma^{z}
% \end{align*}
% such that the gate is the Trotter-Suzuki decomposed Eq.~\ref{eq:TrotterFirstOrder}
% product of exponential terms $\exp\left(\left[A_{1},B_{1}\right]\lambda^{2}\right)=\exp\left(-2i\lambda^{2}\hat{x}_{1}\hat{x}_{2}\sigma^{z}\right)$
% and $\exp\left(\left[A_{2},B_{2}\right]\lambda^{2}\right)=\exp\left(2i\lambda^{2}\hat{p}_{1}\hat{p}_{2}\sigma^{z}\right)$.
% The Pauli commutation relation Eq.~\ref{eq:PauliCommutatorz}, yields
% \begin{align}
% \left[A_{1},B_{1}\right] & =\left[i\hat{x}_{1}\sigma^{x},i\hat{x}_{2}\sigma^{y}\right]\\
% \left[A_{2},B_{2}\right] & =\left[\hat{p}_{1}\sigma^{x},\hat{p}_{2}\sigma^{y}\right]
% \end{align}
% such that the BCH formula Eq.~\ref{eq:BCHFormula} of the two exponential
% terms yields the infinitesimal conditional beam squeezer with a lower
% bound gate depth of eight.

% \subsection{Conditional Rotations\label{subsec:Conditional-Rotations}}

% We employ the displacement analytic ISA to synthesize the infinitesimal
% conditional rotation gate
% \begin{equation}
% U_{\text{rot},k}=e^{i\lambda^{2}\hat{n}\sigma^{k}}
% \end{equation}
% for $k=x,y,z$, which naturally belongs to the analytic displacement
% and rotation ISA. Expression of the exponential argument in terms
% of phase-space operators according to Eq.~\ref{eq:NumbertoPhaseSpace},
% Eq.~\ref{eq:CreationOperatortoPhaseSpace}, and Eq.~\ref{eq:AnnihilationOperatortoPhaseSpace}
% yields
% \begin{equation}
% i\lambda^{2}\hat{n}\sigma^{k}=i\left(\hat{x}^{2}+\hat{p}^{2}-\frac{1}{2}\right)\sigma^{k}\lambda^{2}
% \end{equation}
% such that the gate is expressed via the Trotter-Suzuki decomposition
% Eq.~\ref{eq:TrotterFirstOrder}as the product of $\exp\left(\left[A_{1},B_{1}\right]\right)=\exp\left(i\lambda^{2}\hat{x}^{2}\sigma^{k}\right)$,
% $\exp\left(\left[A_{2},B_{2}\right]\right)=\exp\left(i\lambda^{2}\hat{p}^{2}\sigma^{k}\right)$,
% and conditional displacement $\exp\left(-i\lambda^{2}\sigma^{k}/2\right)$.
% Given the Pauli commutator relation Eq.~\ref{eq:BCHFormula}, the
% first commutator is 
% \begin{align}
% \left[A_{1},B_{1}\right] & =i\hat{x}^{2}\sigma^{k}\\
%  & =i\hat{x}^{2}\left(-\frac{i}{2}\left[\sigma^{i},\sigma^{j}\right]\right)\\
%  & =\left[\frac{1}{\sqrt{2}}\hat{x}\sigma^{i},\frac{1}{\sqrt{2}}\hat{x}\sigma^{j}\right]
% \end{align}
% and the second commutator is 
% \begin{align}
% \left[A_{2},B_{2}\right] & =i\hat{p}^{2}\sigma^{k}\\
%  & =i\hat{p}^{2}\left(-\frac{i}{2}\left[\sigma^{i},\sigma^{j}\right]\right)\\
%  & =\left[\frac{1}{\sqrt{2}}\hat{p}\sigma^{i},\frac{1}{\sqrt{2}}\hat{p}\sigma^{j}\right]
% \end{align}
% such that both terms are amenable to BCH decomposition Eq.~\ref{eq:BCHFormula}
% and the infinitesimal conditional rotation is composed in the displacement
% ISA with gate depth lower-bound of  nine.

% \section{Universal Control of the Span $\left\{ \left|0\right\rangle ,\left|1\right\rangle \right\} $
% Fock Space\label{subsec:Universal-Control}}

% To demonstrate the efficacy of the analytic ISA, we demonstrate the
% use of the approach to encode a qubit in a cavity either via generation
% of effective Pauli gates in Section~\ref{subsec:Effective-Pauli-Gate}
% or imposition of an effective Hubbard interaction in the Jaynes-Cumming
% Hamiltonian in Section~\ref{subsec:Effective-Hubbard-Lattice}.

% \subsection{Effective Pauli Gate Approach\label{subsec:Effective-Pauli-Gate}}

% For universal control is the restricted $\text{span}\left\{ \left|0\right\rangle ,\left|1\right\rangle \right\} $
% Hilbert space, we generate three effective Pauli operators $\sigma_{\text{eff}}^{x}$,
% $\sigma_{\text{eff}}^{y}$, and $\sigma_{\text{eff}}^{z}$ that produce
% Pauli rotations in the lowest two modes of the cavity minimal leakage
% to higher energy states. The form of the effective Pauli operators
% is determined by expressing the standard Pauli operators
% \begin{align}
% \sigma^{x} & =\left(\begin{array}{cc}
% 0 & 1\\
% 1 & 0
% \end{array}\right)\\
% \sigma^{y} & =\left(\begin{array}{cc}
% 0 & -i\\
% i & 0
% \end{array}\right)\\
% \sigma^{z} & =\left(\begin{array}{cc}
% 1 & 0\\
% 0 & -1
% \end{array}\right)
% \end{align}
% in terms of creation and annihilation operators truncated to the first
% two Fock states
% \begin{align}
% \hat{a}_{\text{eff}}^{\dagger} & =\left(\begin{array}{cc}
% 0 & 0\\
% 1 & 0
% \end{array}\right)\\
% \hat{a}_{\text{eff}} & =\left(\begin{array}{cc}
% 0 & 1\\
% 0 & 0
% \end{array}\right)\\
% \hat{n}_{\text{eff}} & =\hat{a}^{\dagger}\hat{a}_{\text{eff}}=\left(\begin{array}{cc}
% 0 & 0\\
% 0 & 1
% \end{array}\right)
% \end{align}
% which yields
% \begin{align}
% \sigma_{\text{eff}}^{x} & =\hat{a}_{\text{eff}}^{\dagger}+\hat{a}_{\text{eff}}\\
% \sigma_{\text{eff}}^{y} & =i\left(\hat{a}_{\text{eff}}^{\dagger}-\hat{a}_{\text{eff}}\right)\\
% \sigma_{\text{eff}}^{z} & =I-2\hat{a}_{\text{eff}}^{\dagger}\hat{a}_{\text{eff}}
% \end{align}
% To reduce leakage into higher energy states, we ensure the creation
% operator $\hat{a}_{\text{eff}}^{\dagger}$ only acts on the ground
% state $\left|0\right>$ and the annihilation operator $\hat{a}_{\text{eff}}$
% only acts on the first excited state $\left|1\right>$ with the projector
% \begin{align}
% \hat{P}_{0} & \approx I-\hat{n}\\
%  & =\begin{cases}
% 0 & n=1\\
% 1 & n=0
% \end{cases}
% \end{align}
% where $n$ is the number of photons in the cavity and where only the
% $\text{span}\left\{ \left|0\right\rangle ,\left|1\right\rangle \right\} $
% states are populated. Since the operator is a projector, it obeys
% the relation
% \begin{equation}
% \hat{P}_{0}^{2}=\hat{P}_{0}
% \end{equation}
% such that the effective Pauli gates are
% \begin{align}
% \sigma_{\text{eff}}^{x} & =\hat{a}_{\text{eff}}^{\dagger}\hat{P}_{0}+\hat{P}_{0}\hat{a}_{\text{eff}}\\
%  & \approx\hat{a}^{\dagger}\left(I-\hat{n}\right)+\left(I-\hat{n}\right)\hat{a}\\
% \sigma_{\text{eff}}^{y} & =i\left(\hat{a}_{\text{eff}}^{\dagger}\hat{P}_{0}-\hat{P}_{0}\hat{a}_{\text{eff}}\right)\\
%  & \approx i\left(\hat{a}^{\dagger}\left(I-\hat{n}\right)-\left(I-\hat{n}\right)\hat{a}\right)\\
% \sigma_{\text{eff}}^{z} & =I-2\hat{a}_{\text{eff}}^{\dagger}\hat{P}_{0}^{2}\hat{a}_{\text{eff}}\\
%  & \approx I-2\hat{a}^{\dagger}\left(I-\hat{n}\right)\hat{a}
% \end{align}


% \subsubsection{Pauli X Gate}

% Consider the infinitesimal $\sigma_{x}$-rotation gate in the $\text{span}\left\{ \left|0\right\rangle ,\left|1\right\rangle \right\} $
% Fock space 
% \begin{align}
% U_{\text{span}\left\{ 0,1\right\} ,x} & =e^{i\lambda^{2}\sigma_{\text{eff}}^{x}\sigma^{z}}\\
%  & =e^{i\lambda^{2}\left(\hat{a}^{\dagger}\left(1-\hat{n}\right)+\left(1-\hat{n}\right)\hat{a}\right)\sigma^{z}}
% \end{align}
% Expression of the exponent in terms of phase-space operators Eq.~\ref{eq:NumbertoPhaseSpace},
% Eq.~\ref{eq:CreationOperatortoPhaseSpace}, and Eq.~\ref{eq:AnnihilationOperatortoPhaseSpace}
% gives 
% \begin{gather}
% i\lambda^{2}\left(\hat{a}^{\dagger}\left(I-\hat{n}\right)+\left(I-\hat{n}\right)\hat{a}\right)\sigma^{z}\nonumber \\
% =i\lambda^{2}\left(2\hat{x}-\left\{ \hat{x},\hat{n}\right\} +i\left[\hat{p},\hat{n}\right]\right)\sigma^{z}
% \end{gather}
% The gate is therefore given by a Trotter-Suzuki decomposition Eq.~\ref{eq:TrotterFirstOrder}
% of three terms: $\exp\left(\left[A_{1},B_{1}\right]\lambda^{2}\right)=\exp\left(-i\lambda^{2}\left\{ \hat{x},\hat{n}\right\} \sigma^{z}\right)$,
% $\exp\left(\left[A_{2},B_{2}\right]\lambda^{2}\right)=\exp\left(-\lambda^{2}\left[\hat{p},\hat{n}\right]\sigma^{z}\right)$,
% and $\exp\left(2i\lambda^{2}\hat{x}\sigma^{z}\right)$. 

% The first and second exponential terms are decomposed according to
% the BCH decomposition Eq.~(\ref{eq:BCHFormula}). The first commutator
% is given by the Pauli anticommutation-commutation relation Eq.~(\ref{eq:AnticommutatortoCommutator})
% \begin{align}
% -i\left\{ \hat{x},\hat{n}\right\} \sigma^{z} & =-i\left(i\left[i\hat{x}\sigma^{x},i\hat{n}\sigma^{y}\right]\right)\\
%  & =\left[i\hat{x}\sigma^{x},i\hat{n}\sigma^{y}\right]\\
%  & =\left[A_{1},B_{1}\right]
% \end{align}
% $A_{1}$ corresponds to a position displacement and $B_{1}$
% corresponds to the $y$-conditional rotation gate. The argument of
% the second term is already in the form of a commutator, such that
% \begin{align}
% \left[A_{2},B_{2}\right] & =-\left[\hat{p},\hat{n}\right]\sigma^{z}\\
%  & =\left[i\hat{p},i\hat{n}\sigma^{z}\right]
% \end{align}
% $A_{2}$ corresponds to an \emph{unconditional} momentum boost,
% and $B_{2}$ corresponds to the $z$-conditional rotation gate.
% Lastly, the third term already belongs to the instruction set architecture
% and needs no further decomposition. 

% The infinitesimal $\sigma_{x}$-rotation gate in the $\text{span}\left\{ \left|0\right\rangle ,\left|1\right\rangle \right\} $
% Fock space is therefore composed of a product of nine gates in the
% displacement and rotation gate ISA and 21 gates in the displacement
% ISA.

% \subsubsection{Pauli Y Gate}

% The infinitesimal $\sigma_{y}$-rotation gate in the $\text{span}\left\{ \left|0\right\rangle ,\left|1\right\rangle \right\} $
% Fock space is determined analogously 
% \begin{align}
% U_{\text{span}\left\{ 0,1\right\} ,y} & =e^{i\lambda^{2}\sigma_{\text{eff}}^{y}\sigma^{z}}\\
%  & =e^{-\lambda^{2}\left(\hat{a}^{\dagger}\left(I-\hat{n}\right)+\left(I-\hat{n}\right)\hat{a}\right)\sigma^{z}}
% \end{align}
% Expression of the argument of the exponent in terms of phase-space
% variables Eq.~\ref{eq:CreationOperatortoPhaseSpace} and Eq.~\ref{eq:AnnihilationOperatortoPhaseSpace}
% yields 
% \begin{gather}
% -\lambda^{2}\left(\hat{a}^{\dagger}\left(I-\hat{n}\right)-\left(I-\hat{n}\right)\hat{a}\right)\sigma^{z}\nonumber \\
% =-\lambda^{2}\left(-2i\hat{p}+\left[\hat{n},\hat{x}\right]+i\left\{ \hat{n},\hat{p}\right\} \right)\sigma^{z}
% \end{gather}
% such that the gate is a Trotter-Suzuki decomposition Eq.~\ref{eq:TrotterFirstOrder}
% of $\exp\left(\left[A_{1},B_{1}\right]\lambda^{2}\right)=\exp\left(-\lambda^{2}\left[n,x\right]\sigma^{z}\right)$,
% $\exp\left(\left[A_{2},B_{2}\lambda^{2}\right]\right)=\exp\left(-i\lambda^{2}\left\{ p,n\right\} \sigma^{z}\right)$,
% and $\exp\left(2i\lambda^{2}p\sigma^{z}\right)$. 

% Again, the first two exponential terms are decomposed via the BCH
% formula Eq.~\ref{eq:BCHFormula}. The first commutator is 
% \begin{align}
% \left[A_{1},B_{1}\right] & =\left[\hat{n},\hat{x}\right]\sigma^{z}\\
%  & =\left[\hat{n}\sigma^{z},\hat{x}\right]
% \end{align}
% where the exponent of $A_{1}$ is a $z$-conditional rotation
% gate and the exponent of $B_{1}$ is an unconditional position
% displacement. The second commutator is given by the Pauli anticommutation-commutation
% relation Eq.~(\ref{eq:AnticommutatortoCommutator}) 
% \begin{align}
% i\left\{ p,n\right\} \sigma^{z} & =i\left(i\left[ip\sigma^{x},in\sigma^{y}\right]\right)\\
%  & =\left[ip\sigma^{x},in\sigma^{y}\right]\\
%  & =\left[A_{2},B_{2}\right]
% \end{align}
% where the exponent of $A_{2}$ corresponds to a conditional
% momentum shift and the exponent of $B_{2}$ is a $y$-conditional
% rotation gate.

% The infinitesimal $\sigma_{y}$-rotation gate in the $\text{span}\left\{ \left|0\right\rangle ,\left|1\right\rangle \right\} $
% Fock space therefore has a lower-bound gate depth of nine in the displacements-only
% analytic ISA and 21 in the displacement and rotation analytic ISA.

% \subsubsection{Pauli Z Gate}

% The infinitesimal $\sigma_{z}$-rotation gate in the $\text{span}\left\{ \left|0\right\rangle ,\left|1\right\rangle \right\} $
% Fock space is 
% \begin{align}
% U_{\text{span}\left\{ 0,1\right\} ,z} & =e^{i\lambda^{2}\sigma_{\text{eff}}^{z}\sigma^{z}}\\
%  & =e^{-\lambda^{2}\left(I-2\hat{a}^{\dagger}\left(I-\hat{n}\right)\hat{a}\right)\sigma^{z}}
% \end{align}
% whose argument in terms of ladder operators is
% \begin{gather}
% -\lambda^{2}\left(I-2\hat{a}^{\dagger}\left(I-\hat{n}\right)\hat{a}\right)\sigma^{z}\nonumber \\
% =-\lambda^{2}\left(I-2\hat{a}^{\dagger}a+2\hat{a}^{\dagger}\hat{a}^{\dagger}\hat{a}\hat{a}\right)\sigma^{z}
% \end{gather}
% Given the ladder operator commutator Eq.~\ref{eq:CommutatorCreationAnnihilation},
% \begin{equation}
% \hat{a}^{\dagger}\hat{a}=\hat{a}\hat{a}^{\dagger}-I
% \end{equation}
% the relationship between the fourth-order ladder operator term and
% the number operator is
% \begin{align}
% \hat{a}^{\dagger}\hat{a}^{\dagger}\hat{a}\hat{a} & =\hat{a}^{\dagger}\left(\hat{a}\hat{a}^{\dagger}-I\right)\hat{a}\\
%  & =\hat{a}^{\dagger}\hat{a}\hat{a}^{\dagger}\hat{a}-\hat{a}^{\dagger}\hat{a}\\
%  & =\hat{n}^{2}-\hat{n}
% \end{align}
% The argument of the exponential in terms of number operators is then
% \begin{multline}
% -\lambda^{2}\left(I-2\hat{a}^{\dagger}\left(I-\hat{n}\right)\hat{a}\right)\sigma^{z}\\
% =-\lambda^{2}\left(I-4\hat{n}+2\hat{n}^{2}\right)\sigma^{z}
% \end{multline}
% The argument is further simplified given that the state is restricted
% to the first two cavity modes, as for $n=0$ and $n=1$ the quantity
% $\hat{n}^{2}-\hat{n}$ is zero, as follows:
% \begin{gather}
% -\lambda^{2}\left(I-4\hat{n}+2\hat{n}^{2}\right)\sigma^{z}\nonumber \\
% =-\lambda^{2}\left(I-2n\right)\sigma^{z}
% \end{gather}
% The gate is therefore directly synthesized as the product of the qubit
% rotation gate $\exp\left(-\lambda^{2}\sigma^{z}\right)$ and the $z$-conditional
% rotation gate $\exp\left(2\lambda^{2}\hat{n}\sigma^{z}\right)$ for
% a lower-bound gate depth of two.

% \subsection{Effective Hubbard Lattice Interaction Approach\label{subsec:Effective-Hubbard-Lattice}}

% An alternative scheme to encode a qubit in a cavity in the analytic
% ISA scheme is to map the three-dimensional quantum electrodynamics
% (3D cQED) system to a qubit by imposing an $\hat{n}\left(\hat{n}-1\right)$
% anharmonicity into the Jaynes-Cummings Hamiltonian that describes
% the system. The anharmonicity term increases the energy gap between
% higher levels of the oscillator to effectively restrict propagation
% to the $\text{span}\left\{ \left|0\right\rangle ,\left|1\right\rangle \right\} $
% Fock space for universal control.

% Consider the standard Jaynes-Cummings Hamiltonian 
% \begin{equation}
% \hat{H}_{\text{JC}}=\omega_{R}\hat{a}^{\dagger}\hat{a}+\frac{\omega_{Q}}{2}\sigma^{z}+g\left(\hat{a}\sigma^{+}+\hat{a}^{\dagger}\sigma^{-}\right)
% \end{equation}
% where $\omega_{R}$ is the cavity frequency, $\omega_{Q}$ is the
% qubit frequency, and $g$ is the coupling parameter. Inclusion of
% the simulated $\hat{n}\left(\hat{n}-1\right)$ anharmonicity of strength
% $\Gamma$ yields 
% \begin{equation}
% \hat{H}_{\text{an}}=\omega_{R}\hat{a}^{\dagger}\hat{a}+\Gamma\hat{n}(\hat{n}-1)+\frac{\omega_{Q}}{2}\sigma^{z}+g(\hat{a}\sigma^{+}+\hat{a}^{\dagger}\sigma^{-})
% \end{equation}
% and system is switched between states $\left|0\right\rangle $ and
% $\left|1\right\rangle $ with a weak time-$t$-dependent drive of
% strength $\Omega$ at the resonance frequency $\omega_{R}$, as follows:
% \begin{equation}
% \hat{H}_{\text{drive}}\left(t\right)=\Omega e^{i\omega_{R}t}\hat{a}^{\dagger}+\Omega^{\star}e^{-i\omega_{R}t}\hat{a}
% \end{equation}
% Synthesis of a propagator of the form $\exp\left(i\lambda^{2}\hat{n}\left(\hat{n}-1\right)\right)$
% is then sufficient to employ the native 3D cQED system as a qubit.
% Note the choice of $\lambda$ for practical implementation must take
% into account both the time step and the fact the BCH decomposition
% yields a square root in the exponential argument. The required propagator
% is a Trotter-Suzuki decomposition Eq.~(\ref{eq:TrotterFirstOrder})
% of $\exp\left(\left[A,B\right]\lambda^{2}\right)=\exp(i\lambda^{2}\hat{n}^{2}\sigma^{z})$
% and $\exp(-i\lambda^{2}\hat{n}\sigma^{z})$. 

% The first term is synthesized according to the BCH formula Eq.~(\ref{eq:BCHFormula})
% with a commutator determined by the Pauli commutation relation Eq.~(\ref{eq:PauliCommutatorz})
% as follows:
% \begin{align}
% \left[A,B\right] & =i\hat{n}^{2}\sigma^{z}\\
%  & =i\hat{n}^{2}\left(-\frac{i}{2}\left[\sigma^{x},\sigma^{y}\right]\right)\\
%  & =\left[\frac{1}{\sqrt{2}}\hat{n}\sigma^{x},\frac{1}{\sqrt{2}}\hat{n}\sigma^{y}\right]
% \end{align}
% where $A$ and $B$ correspond to $x$-conditional and
% $y$-conditional rotations, respectively. The second term is a $z$-conditional
% rotation gate.

% The resulting anharmonicity gate therefore has a gate depth of lower
% bound five in the displacement and rotation analytic ISA and 45 in
% the displacement ISA.

% \section{Fermi-Hubbard Lattice Dynamics\label{sec:Fermi-Hubbard-Lattice-Dynamics}}

% To further demonstrate the power of the analytic ISA, we employ the
% approach to simulate fermionic dynamics on bosonic 3D cQED systems.

% We consider the Fermi-Hubbard lattice Hamiltonian

% \begin{align}
% \hat{H}_{\text{FH}} & =\hat{T}_{\text{FH}}+\hat{V}_{\text{FH}}\label{eq:FermiHubbardHamiltonian}\\
% \hat{T}_{\text{FH}} & =-J\sum_{i,\sigma}\hat{c}_{i,\sigma}^{\dagger}\hat{c}_{i+1,\sigma}+\hat{c}_{i+1,\sigma}^{\dagger}\hat{c}_{i,\sigma}\\
% \hat{V}_{\text{FH}} & =U\sum_{i}\hat{n}_{i,\uparrow}\hat{n}_{i,\downarrow}
% \end{align}
% The kinetic energy term $\hat{T}_{\text{FH}}$ describes the nearest-neighbor
% interaction for hopping of a single spin between two sites with hopping
% parameter $J$ and spin $\sigma$ given annihilation operators $\left\{ \hat{c}_{j,\sigma}\right\} $
% and creation operators $\left\{ \hat{c}_{j,\sigma}^{\dagger}\right\} $
% for sites $\left\{ j\right\} $. The potential energy term $\hat{V}_{\text{FH}}$
% describes the same-site interaction, which gives the energetic unfavorability
% of a spin up $\uparrow$ and spin down $\downarrow$ coexisting on
% the same site $i$, where $\hat{n}_{j,\sigma}$ gives the number of
% spin $\sigma$ particles on site $j$. According to fermion statistics,
% no more than a single particle of a given spin can exist on a single
% site.

% Each cavity of the 3D cQED system represents either a spin up or spin
% down particle on a single lattice site, for direct comparison to the
% qubit-based schemes of refs.~\cite{Kivlichan.2018.110501,arute2020observation,Cade.2020.235122}.
% Each cavity is connected to the cavity that represents the same site
% of opposite spin to facilitate computation of the potential energy
% $\hat{V}_{\text{FH}}$ and cavities of the same spin on neighboring
% sites to facilitate computation of the kinetic energy $\hat{T}_{\text{FH}}$.
% Cavities are also connected along Jordan-Wigner strings to take into
% account fermionic statistics. 

% The $\text{\ensuremath{\left|0\right\rangle }}$ cavity state represents
% absence of a spin and the $\text{\ensuremath{\left|1\right\rangle }}$
% state represents presence of a spin. Within each cavity, only the
% states in $\text{span}\left\{ \text{\ensuremath{\left|0\right\rangle }},\text{\ensuremath{\left|1\right\rangle }}\right\} $
% are considered as in Section~(\ref{subsec:Universal-Control}), which
% prevents leakage into unphysical high-energy cavity states. At the
% end of each operation, the cavity state must be in either the $\text{\ensuremath{\left|0\right\rangle }}$
% or $\text{\ensuremath{\left|1\right\rangle }}$ state and the transmon
% state must also be in the ground state $\text{\ensuremath{\left|g\right\rangle }}$,
% which provides an error syndrome and therefore a degree of error correction
% not employed in qubit-based representations of the Fermi-Hubbard lattice.

% Propagation of any combination of up spins and down spins is simulated
% with three two-cavity gates. The first two gates -- the same-site
% and hopping gates -- are defined as the propagator of the same-site
% and hopping Hamiltonians, respectively. The same-site term of the
% Hamiltonian for site $i$ is 
% \begin{equation}
% \hat{H}_{\text{same}}=U\hat{n}_{i,\uparrow}\hat{n}_{i,\downarrow}
% \end{equation}
% This term is zero if only one spin is on a site and $U$ if both spins
% are on the same site, which gives the diagonal Hamiltonian in the
% reduced $4\times4$ Hilbert space
% \begin{equation}
% \hat{H}_{\text{same}}=\left[\begin{array}{cccc}
% 0 & 0 & 0 & 0\\
% 0 & 0 & 0 & 0\\
% 0 & 0 & 0 & 0\\
% 0 & 0 & 0 & U
% \end{array}\right]
% \end{equation}
% and the diagonal propagator $U_{\text{same}}=\text{e}^{-\text{i}\hat{H}_{\text{same}}\tau}$
% \begin{equation}
% U_{\text{same}}=\left[\begin{array}{cccc}
% 1 & 0 & 0 & 0\\
% 0 & 1 & 0 & 0\\
% 0 & 0 & 1 & 0\\
% 0 & 0 & 0 & \text{e}^{-\text{i}U\tau}
% \end{array}\right]
% \end{equation}
% This gate is recognized as the conditional cross-Kerr interaction
% of 3D cQED systems and equivalently a controlled-phase (CPHASE) gate
% in the reduced subspace $\text{span}\left\{ \left|0\right\rangle ,\left|1\right\rangle \right\} $.
% The hopping term of the Hamiltonian for each $\sigma$ spin in sites
% $i,\left(i+1\right)$ is 
% \begin{align}
% H_{\text{hop}} & =-J\left(\hat{c}_{i,\sigma}^{\dagger}\hat{c}_{i+1,\sigma}+\hat{c}_{i+1,\sigma}^{\dagger}\hat{c}_{i,\sigma}\right)\\
%  & =-J\left(\hat{c}_{i,\sigma}^{\dagger}\hat{c}_{i+1,\sigma}-\hat{c}_{i,\sigma}\hat{c}_{i+1,\sigma}^{\dagger}\right)
% \end{align}
% where the latter expression employs the commutator relationship of
% the annihilation and creation operators. The hopping Hamiltonian for
% the specified mapping is then the off-diagonal matrix 
% \begin{equation}
% H_{\text{hop}}=\left[\begin{array}{cccc}
% 0 & 0 & 0 & 0\\
% 0 & 0 & -t & 0\\
% 0 & -t & 0 & 0\\
% 0 & 0 & 0 & 0
% \end{array}\right]
% \end{equation}
% which gives the hopping propagator $U_{\text{hop}}=\text{e}^{-\text{i}H_{\text{hop}}\tau}$
% \begin{align}
% U_{\text{hop}} & =\left[\begin{array}{cccc}
% 1 & 0 & 0 & 0\\
% 0 & \cos\left(t\tau\right) & i\sin\left(t\tau\right) & 0\\
% 0 & i\sin\left(t\tau\right) & \cos\left(t\tau\right) & 0\\
% 0 & 0 & 0 & 1
% \end{array}\right]
% \end{align}
% which is recognized as a conditional controlled-phase beam splitter
% restricted to $\text{span}\left\{ \left|0\right\rangle ,\left|1\right\rangle \right\} $
% in bosonic systems and a Givens or iSWAP-like gate in the reduced
% $\text{span}\left\{ \left|0\right\rangle ,\left|1\right\rangle \right\} $
% subspace \cite{Cade.2020.235122,arute2020observation}. The final
% gate of the three-gate set incorporates the fermionic statistics of
% the spins via the fermionic SWAP (FSWAP) gate FSWAP gate \cite{Kivlichan.2018.110501,Cade.2020.235122}.
% The content of each cavity is swapped with one of its neighbors with
% inclusion of a phase where both spins are present in neighboring cavities
% as follows
% \begin{equation}
% U_{\text{FSWAP}}=\left[\begin{array}{cccc}
% 1 & 0 & 0 & 0\\
% 0 & 0 & 1 & 0\\
% 0 & 1 & 0 & 0\\
% 0 & 0 & 0 & -1
% \end{array}\right]
% \end{equation}
% which is recognized as the product of a conditional rotation gate
% and a beam-splitter on 3D cQED systems.

% Finally, initial states are prepared by the universal set of gates
% in $\text{span}\left\{ \left|0\right\rangle ,\left|1\right\rangle \right\} $
% detailed in Section~\ref{subsec:Universal-Control}.

% \subsection{Conditional Cross-Kerr (CPHASE) Gate}

% We consider the infinitesimal conditional cross-Kerr gate
% \begin{equation}
% U_{\text{cross-Kerr}}=e^{i\lambda^{2}\hat{n}_{1}\hat{n}_{2}\sigma_{z}}
% \end{equation}
% which is also employed in GKP codes encoded in 3D cQED systems \cite{royer2022encoding}. 

% The argument is expressed in terms of a commutator according to the
% Pauli commutation relation Eq.~\ref{eq:PauliCommutatorz}, as follows:
% \begin{align}
% \left[A,B\right] & \lambda^{2}=i\lambda^{2}\hat{n}_{1}\hat{n}_{2}\sigma_{z}\\
%  & =i\lambda^{2}\hat{n}_{1}\hat{n}_{2}\left(-\frac{i}{2}\left[\sigma^{x},\sigma^{y}\right]\right)\\
%  & =\left[\frac{1}{\sqrt{2}}\hat{n}_{1}\sigma^{x},\frac{1}{\sqrt{2}}\hat{n}_{2}\sigma^{y}\right]\lambda^{2}
% \end{align}
% where $A$ corresponds to an $x$-conditional rotation gate
% and $B$ corresponds to a $y$-conditional rotation gate.

% The resulting gate features a lower-bound gate depth of four in the
% displacement and rotation analytic ISA and 16 in the displacement
% ISA.

% \subsection{$\text{Span}\left\{ \left|0\right\rangle ,\left|1\right\rangle \right\} $
% Conditional Beam Splitter Gate}

% In order to generate a $\text{span}\left\{ \left|0\right\rangle ,\left|1\right\rangle \right\} $
% that operates only when $\hat{n}_{1}\hat{n}_{2}\ne1$ (\emph{i.e.},
% $1-\hat{n}_{1}\hat{n}_{2}=0$), we formulate the infinitesimal conditional
% (controlled-phase) beam-splitter gate
% \begin{equation}
% U_{\text{cond. beam}}=e^{-i\lambda^{2}\left(\hat{a}_{1}^{\dagger}\hat{a}_{2}+\hat{a}_{1}\hat{a}_{2}^{\dagger}\right)\left(1-\hat{n}_{1}\hat{n}_{2}\right)\sigma^{z}}
% \end{equation}
% which is decomposed via the Trotter-Suzuki decomposition Eq.~\ref{eq:TrotterFirstOrder}
% in terms of $\exp\left(-i\lambda^{2}\left(\hat{a}_{1}^{\dagger}\hat{a}_{2}+\hat{a}_{1}\hat{a}_{2}^{\dagger}\right)\sigma^{z}\right)$
% and $\exp\left(i\lambda^{2}\left(\hat{a}_{1}^{\dagger}\hat{a}_{2}+\hat{a}_{1}\hat{a}_{2}^{\dagger}\right)\left(\hat{n}_{1}\hat{n}_{2}\right)\sigma^{z}\right)$.
% The first term is the conditional beam splitter $U_{\text{beam split.}}$
% Eq.~\ref{subsec:Conditional-(Controlled-Phase)-Beam-Splitter} and
% the second term is decomposed via BCH Eq.~\ref{eq:BCHFormula} as
% follows:

% Given the expression of the number operator in terms of the phase-space
% operators Eq.~\ref{eq:NumbertoPhaseSpace}, the argument of the second
% exponential operator is 
% \begin{gather}
% i\lambda^{2}\left(\hat{a}_{1}^{\dagger}\hat{a}_{2}+\hat{a}_{1}\hat{a}_{2}^{\dagger}\right)\hat{n}_{1}\hat{n}_{2}\sigma^{z}\nonumber \\
% =i\lambda^{2}\left(2\left(\hat{x}_{1}\hat{x}_{2}+\hat{p}_{1}\hat{p}_{2}\right)\right)\hat{n}_{1}\hat{n}_{2}\sigma^{z}
% \end{gather}
% The term is then expressed as a Trotter decomposition of $\exp\left(\left[A_{1},B_{1}\right]\lambda^{2}\right)=\exp\left(2i\lambda^{2}\hat{x}_{1}\hat{x}_{2}\hat{n}_{1}\hat{n}_{2}\sigma^{z}\right)$
% and $\exp\left(\left[A_{2},B_{2}\right]\lambda^{2}\right)=\exp\left(2i\lambda^{2}\hat{p}_{1}\hat{p}_{2}\hat{n}_{1}\hat{n}_{2}\sigma^{z}\right)$. 

% The first commutator is given by the Pauli commutation relation Eq.~\ref{eq:PauliCommutatorz}
% \begin{align}
% \left[A_{1},B_{1}\right] & =2i\hat{x}_{1}\hat{x}_{2}\hat{n}_{1}\hat{n}_{2}\sigma^{z}\\
%  & =2i\hat{x}_{1}\hat{x}_{2}\hat{n}_{1}\hat{n}_{2}\left(-\frac{i}{2}\left[\sigma^{x},\sigma^{y}\right]\right)\\
%  & =\left[\hat{x}_{1}\hat{n}_{1}\sigma^{x},\hat{x}_{2}\hat{n}_{2}\sigma^{y}\right]
% \end{align}

% The $A_{1}$ term is determined by a Trotter decomposition according
% to the product of operator formula Eq.~\ref{eq:ProductOperators}
% \begin{align}
% A_{1} & =\frac{1}{2}\left\{ \hat{x}_{1},\hat{n}_{1}\right\} \sigma^{x}+\frac{1}{2}\left[\hat{x}_{1},\hat{n}_{1}\right]\sigma^{x}\\
%  & =A_{1a}+A_{1b}
% \end{align}
% where according to the anticommutator to commutator relation $A_{1a}$
% is given by the BCH formula with 
% \begin{align}
% \left[A_{1a^{\prime}},B_{1a^{\prime}}\right] & =\frac{1}{2}\left\{ \hat{x}_{1},\hat{n}_{1}\right\} \sigma^{x}\\
%  & =\frac{i}{2}\left[i\hat{x}_{1}\sigma^{y},i\hat{n}_{1}\sigma^{z}\right]\\
%  & =\left[-\frac{1}{\sqrt{2}}\hat{x}_{1}\sigma^{y},-\frac{1}{\sqrt{2}}\hat{n}_{1}\sigma^{z}\right]
% \end{align}
% where $B_{1a^{\prime}}$ is a $z$-conditional rotation gate.
% Distribution of terms yields $A_{1b}$ as 
% \begin{equation}
% \left[A_{1b^{\prime}},B_{1b^{\prime}}\right]=\left[\frac{1}{\sqrt{2}}\hat{x}_{1},\frac{1}{\sqrt{2}}\hat{n}_{1}\sigma^{x}\right]
% \end{equation}
% where $B_{1b^{\prime}}$ is an $x$-conditional rotation gate. 

% According to the same procedure,
% \begin{align}
% B_{1} & =\frac{1}{2}\left\{ \hat{x}_{2},\hat{n}_{2}\right\} \sigma^{y}+\frac{1}{2}\left[\hat{x}_{2},\hat{n}_{2}\right]\sigma^{y}\\
%  & =B_{1a}+B_{1b}
% \end{align}
% where $B_{1a}$ is given by
% \begin{align}
% \left[A_{1a^{\prime\prime}},B_{1a^{\prime\prime}}\right] & =\frac{1}{2}\left\{ \hat{x}_{2},\hat{n}_{2}\right\} \sigma^{y}\\
%  & =\frac{i}{2}\left[i\hat{x}_{2}\sigma^{z},i\hat{n}_{1}\sigma^{x}\right]\\
%  & =\left[-\frac{1}{\sqrt{2}}\hat{x}_{2}\sigma^{z},-\frac{1}{\sqrt{2}}\hat{n}_{2}\sigma^{x}\right]
% \end{align}
% with $B_{1a^{\prime\prime}}$ an $x$-conditional rotation, and $B_{1b}$
% is given by 
% \begin{equation}
% \left[A_{1b^{\prime\prime}},B_{1b^{\prime\prime}}\right]=\left[\frac{1}{\sqrt{2}}\hat{x}_{2},\frac{1}{\sqrt{2}}\hat{n}_{2}\sigma^{y}\right]
% \end{equation}
% where $B_{1b^{\prime\prime}}$ is a $y$-conditional rotation gate.

% The second term follows analogously with the position $x$ replaced
% by the momentum $p$.

% \subsection{Conditional FSWAP Gate.}

% In analytic ISA, the FSWAP gate follows immediately from the conditional
% cross-Kerr gate detailed above and a complete beam-splitter gate (or
% conditional beam splitter gate detailed above) as 
% \begin{equation}
% U_{\text{FSWAP}}=U_{\text{cond. Kerr}}U_{\text{cond. beam}}
% \end{equation}
}



\section{Discussion}
%% what are the pts we should think about for the audience. 
%% may be easier to do this from the beginning

Our main contribution in this paper is a systematic approach to synthesizing unitary dynamics on a hybrid quantum computer that has access to both qubit as well as bosonic operations.  Such gatesets naturally model systems such as cavity quantum electrodynamics systems as well as ion trap-based quantum computers.  Our main innovation here is the development of high-order analytic formulas that can be used to place bounds on the complexity of implementing arbitrary unitary operations on such a hybrid device.  Specifically, we see that these methods are capable of achieving subpolynomial scaling with the inverse error tolerance ($1/\epsilon$) and allows us to implement arbitrary nonlinearities in the field operators in the generator of the unitary that we wish to implement at low cost asymptotically.  In particular, we focus on using a construct known as a block-encoded creation operator as our fundamental construct and show numerically highly accurate approximations to the exponential of a block-encoding of the square of the creation operator.  Further, we study the Hong-Ou-Mandel effect and observe that the synthesized operations used in our construction can have negligible error with respect to our target precision.

While this work constitutes a significant step forward in our understanding of how to control and manipulate such quantum systems, there remain many open questions.  The first issue involves the large constant-factor overheads observed in practical implementations of our synthesized operations within this gate set.  Specifically, we note that for both the Hong-Ou-Mandel effect and the block-encoding of $(a^\dagger)^2$ that thousands of gate operations are needed to achieve infidelities of $10^{-3}$ or smaller.  This makes such sequences impractical for near-term applications where the gate infidelities are on the order of $1\%$.  Several avenues of approach exist that could be used to improve upon these results.  The first approach would be to use ideas related to quantum signal processing to implement functions of creation operators; this could potentially improve the scaling with respect to the error tolerance from the subpolynomial scaling currently demonstrated to polylogarithmic scaling.  The second approach would be to use sequences designed here as seeds for gradient-descent optimization procedures for control such as GRAPE~\cite{khaneja2005optimal} at the pulse level or numerical optimization of parameterized gates at the SNAP \cite{fosel2020efficient} or Controlled Displacement \cite{eickbusch2021fast} instruction level.  These locally optimized sequences may then prove to be either better, or more understandable, than existing gradient-optimized pulse sequences for control of such systems.

%Another important question is to ask from a bigger picture perspective what role Trotter-Suzuki derived synthesis methods like this one may play in controlling quantum systems.