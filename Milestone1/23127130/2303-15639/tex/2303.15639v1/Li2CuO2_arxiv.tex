%% ****** Start of file apstemplate.tex ****** %
%%
%%
%%   This file is part of the APS files in the REVTeX 4.2 distribution.
%%   Version 4.2a of REVTeX, January, 2015
%%
%%
%%   Copyright (c) 2015 The American Physical Society.
%%
%%   See the REVTeX 4 README file for restrictions and more information.
%%
%
% This is a template for producing manuscripts for use with REVTEX 4.2
% Copy this file to another name and then work on that file.
% That way, you always have this original template file to use.
%
% Group addresses by affiliation; use superscriptaddress for long
% author lists, or if there are many overlapping affiliations.
% For Phys. Rev. appearance, change preprint to twocolumn.
% Choose pra, prb, prc, prd, pre, prl, prstab, prstper, or rmp for journal
%  Add 'draft' option to mark overfull boxes with black boxes
%  Add 'showkeys' option to make keywords appear
\documentclass[aps,prb,twocolumn,superscriptaddress,floatfix]{revtex4-2}
%\documentclass[aps,prl,preprint,superscriptaddress]{revtex4-2}
%\documentclass[aps,prl,reprint,groupedaddress]{revtex4-2}
\pdfoutput=1 
\usepackage[colorlinks=true,citecolor=blue,linkcolor=blue,breaklinks=true]{hyperref}
\usepackage{amssymb} 
\usepackage{amsmath} 
\usepackage{physics} 


\usepackage{times} 
\usepackage{setspace}
\usepackage{xfrac} 
\usepackage{placeins} 

% You should use BibTeX and apsrev.bst for references
% Choosing a journal automatically selects the correct APS
% BibTeX style file (bst file), so only uncomment the line
% below if necessary.
%\bibliographystyle{apsrev4-2}


\newcommand{\LCO}{Li$_{2}$CuO$_{2}$}
\newcommand\Tstrut{\rule{0pt}{2.6ex}}         % = `top' strut
\newcommand\Bstrut{\rule[-0.9ex]{0pt}{0pt}}   % = `bottom' strut

\begin{document}

% Use the \preprint command to place your local institutional report
% number in the upper righthand corner of the title page in preprint mode.
% Multiple \preprint commands are allowed.
% Use the 'preprintnumbers' class option to override journal defaults
% to display numbers if necessary
%\preprint{}

%Title of paper
\title{Refined Spin Wave Model and Multi-magnon Bound States in \LCO}

% repeat the \author .. \affiliation  etc. as needed
% \email, \thanks, \homepage, \altaffiliation all apply to the current
% author. Explanatory text should go in the []'s, actual e-mail
% address or url should go in the {}'s for \email and \homepage.
% Please use the appropriate macro foreach each type of information

% \affiliation command applies to all authors since the last
% \affiliation command. The \affiliation command should follow the
% other information
% \affiliation can be followed by \email, \homepage, \thanks as well.
\author{Eli Zoghlin}
\affiliation{William H. Miller III Department of Physics and Astronomy, Johns Hopkins University, Baltimore, MD 21218, USA}
\affiliation{Materials Department, University of California, Santa Barbara, California 93106, USA}

\author{Matthew B. Stone}
\affiliation{Neutron Scattering Division, Oak Ridge National Laboratory, Oak Ridge, Tennessee 37831, USA}

%\author{Leon Balents}
%\affiliation{Kavli Institute for Theoretical Physics, University of California, Santa Barbara, Santa Barbara, 93106, California, USA}

\author{Stephen D. Wilson}
\email[]{stephendwilson@ucsb.edu}
\affiliation{Materials Department, University of California, Santa Barbara, California 93106, USA}


%Collaboration name if desired (requires use of superscriptaddress
%option in \documentclass). \noaffiliation is required (may also be
%used with the \author command).
%\collaboration can be followed by \email, \homepage, \thanks as well.
%\collaboration{}
%\noaffiliation

\date{\today}

\begin{abstract}
Here we report a study of the spin dynamics in the ferromagnetic chain compound \LCO. Inelastic neutron scattering measurements allow for the spin Hamiltonian to be determined using a $J_1 - J_2$ $XXZ$-Heisenberg spin chain model with weak inter-chain interactions.  The primary exchange parameters determined from our data are qualitatively consistent with those of W. E. A. Lorentz \textit{et al.} Europhys. Lett. 88, 37002 (2009), and our data allow for the resolution of additional inter-chain exchange interactions. We also observe the formation of two- and three-magnon bound states, stabilized by the weak, Ising-like exchange anisotropy of the nearest-neighbor intra-chain interaction. While the two-magnon bound state exists only in the magnetically ordered phase of this material, the three-magnon state persists in a finite temperature regime above $T_{N}$, indicating an unconventional character. Our results establish \LCO as an experimental platform for the study of exchange anisotropy-stabilized bound states in a ferromagnetic chain.
\end{abstract}


\pacs{}
% insert suggested keywords - APS authors don't need to do this
%\keywords{}

%\maketitle must follow title, authors, abstract, and keywords
\maketitle

% body of paper here - Use proper section commands
% References should be done using the \cite, \ref, and \label commands
\section{Introduction \label{intro}}

Quasi-1-dimensional (quasi-1D) magnetic materials, or ``spin chain'' compounds, are valuable experimental platforms for testing numerous magnetic models due to the analytical simplification that dimensional confinement provides. Furthermore, the suppression of ordering in quasi-1D systems effectively enhances the effects of quantum fluctuations. Accordingly, quasi-1D materials platforms have been found to manifest a range of fascinating phase behaviors, including Haldane singlet formation \cite{schiffler1986, yamashita2000,xu1996,yokoo1995} fractionalized excitations (\textit{e.g.} spinons) \cite{nagler1991,tennant1993,tennant1995,kim2006}, quantum criticality \cite{kinross2014,morris2014,faure2018,niesen2013}, and solid state manifestations of quantum few-body phenomena \cite{dally2020,garret1997,tennant2003,bai2021,legros2021,torrance1969,hoogerbeets1984}.

The magnetic properties of the \textit{S} = $\frac{1}{2}$  spin chain compound \LCO were originally examined more than 30 years ago \cite{sreedhar1988}. The crystal structure is composed of chains of edge-sharing, square-planar coordinated Cu$^{2+}$ ions,  prompting later proposals of \LCO as a model compound for exploring magnetism in edge-sharing cuprates \cite{sapina1990, okuda1992, mizuno1998,ebisu1998}. 
The chains in this system run along the crystallographic \textit{b}-axis with each array of chains in the \textit{ac}-plane shifted by $\frac{a}{2}$ relative to the arrays above and below along the \textit{c}-axis. An early powder neutron diffraction study reported a commensurate, antiferromagnetic (C-AFM) ground state \cite{sapina1990}, which is composed of ferromagnetically (FM) aligned chains with the moment direction alternating along the extended $c$-axis (Fig. \ref{fig:Figure1}(a)). Considerable effort was then made to determine an exchange model that captures the collinear magnetic ground state \cite{yushankhai1999,weht1998,xiang2007,boehm1998,lorenz2009,giri2001}. Despite initial discrepancies in inelastic neutron scattering (INS) results \cite{boehm1998}, the leading interactions in \LCO are now known to be the frustrated FM nearest-neighbor (NN, $J_{010}$) and AFM next-nearest-neighbor (NNN, $J_{020}$) interactions along the chain axis, where the ratio $\alpha$ = $|J_{020}$/$J_{010}|$ defines the degree of frustration \cite{lorenz2009}. This is similar to models of other cuprate spin chain compounds (\textit{e.g.} LiCuVO$_{4}$ \cite{enderle2005} and PbCuSO$_{4}$(OH)$_{2}$ \cite{wolter2012,rule2017}) which are often described in the context of a Heisenberg $J_{1}$-$J_{2}$ model.

 \begin{figure*}
 \includegraphics[width = 17.4 cm]{Figure1.png}
 \caption{\label{fig:Figure1} \textbf{a)} Atomic crystal (orthorhombic) and magnetic (black arrows) structures of \LCO, with the shared unit cell shown by the thin grey line (red = O, blue = Cu). The Li atoms have been omitted for clarity. The labels show the $c$-axis coordinate of the two distinct chain subsystems. \textbf{b)} Illustration of the exchange interactions in the $ab$-plane (left) and $bc$-plane (right). The red arrows represent interactions which were included in the final exchange model, while the arrow in black represents the $J_{110}$ interaction which was not included in our final model.}
 \end{figure*}

Of particular interest in the Heisenberg $J_{1}$-$J_{2}$ model are theoretical phase diagrams \cite{hikihara2008,sudan2009} that indicate the formation of unconventional, multipolar orders in certain regimes of $\alpha$. At small values of $\alpha <$ 1/4, the degree of frustration is sufficiently small that a FM state occurs \cite{bursill1995,bader1979,dimitriev2008}. As $J_{2}$ increases relative to $J_{1}$, multipolar order parameters -- which can be thought of as Bose-Einstein-like condensates of \textit{n}-magnon bound states -- are predicted to be stabilized for $\alpha >$ 1/4, with quantum fluctuations preventing the classically predicted long-range ordered spiral state \cite{nishimoto2015, hikihara2008, sudan2009, chubukov1991, syromyatnikov2012}. Within the context of a  $J_{1}$-$J_{2}$ model, \LCO is a potential candidate for hosting these states ($\alpha \approx$ 1/3 \cite{lorenz2009}); however previous work has also emphasized the importance of inter-chain interactions (specifically the NNN interaction  $\tilde{J}_{131}$) \cite{lorenz2009,mizuno1998,xiang2007} and exchange anisotropy \cite{dimitriev2009a}. These additional terms suppress fluctuations and are believed to stabilize the observed C-AFM order over the spiral state expected for a $J_{1}$-$J_{2}$ system with $\alpha >$ 1/4. Nevertheless, the inter-chain interactions remain reasonably weak compared to the intra-chain exchange ($\tilde{J}_{131} \approx$ 0.04$J_{1}$ and 0.1$J_{2}$) \cite{lorenz2009} and, as a result, this material can be considered as an effective system of FM spin chains with the potential to host \textit{n}-magnon bound states.

The existence of multi-magnon bound states in FM spin chains, first studied by Bethe in 1931 \cite{bethe1931}, has been the subject of many theoretical studies \cite{dimitriev2009a,dimitriev2009b,haldane1982, schneider1981, southern1994, huang1990, hood1984, kecke2007, onishi2015}. These studies have predicted the existence of a rich array of bound states, with distinct physics arising due to differences in spin and the character of the exchange interactions and anisotropies. CoCl$_{2}\cdotp$2H$_{2}$O provided one of the earliest experimental manifestations of this in the solid state, with far-infrared transmission measurements showing results consistent with \textit{n} $\geq$ 2 bound states \cite{torrance1969}. Indications of a series of \textit{n} $\geq$ 2 bound states have also been seen in (C$_{6}$H$_{11}$NH$_{3}$)CuCl$_{3}$ through FM resonance measurements \cite{hoogerbeets1984}. Suggestively, as in \LCO, both of these compounds are \textit{S} = $\frac{1}{2}$, two-sublattice AFMs composed of alternating FM chains with easy-axis exchange anisotropy. In fact, the potential for \LCO to host multi-magnon bound states has been proposed explicitly in previous theoretical work \cite{dimitriev2009a, dimitriev2009b}. For the integer spin case (\textit{S} = 1), a distinct two-magnon bound state stabilized by single-ion anisotropy has been found in FeI$_{2}$ via far infrared absorption spectroscopy \cite{Fert1978} and INS \cite{petitgrand1979,bai2021}. Subsequent time-domain terahertz spectroscopy measurements have also revealed even higher-order bound states (\textit{n} = 4, 6) \cite{legros2021}. Similar measurements in NiNb$_{2}$O$_{6}$ also reported evidence of magnon-magnon interactions \cite{chauhan2020}.

Strangely, to the best of our knowledge, the canonical exchange anisotropy-driven magnon bound states of a FM spin chain have yet to be observed by INS \footnote{Note that the exchange model presented in Bai \textit{et al.} for FeI$_{2}$ does include exchange anisotropy. However, the bound state in that case is fundamentally a result of single-ion anisotropy, which provides the energy reduction upon binding. The diagonal exchange anisotropy cooperates with a finite symmetric off-diagonal exchange to hybridize the single-ion anisotropy-stabilized bound state with the one-magnon mode. This explains the visibility of the bound state in INS, as well as the disperison, but not its stability \cite{bai2021}.}. This represents a substantive gap in the available data since INS, unlike the optical or resonance-based experiments mentioned previously, yields a measure of both the energy \textit{and}  momentum-dependence (away from $|\va*{Q}|$ = 0) of low-energy magnetic excitations. As a result, INS provides additional information about the nature of the magnon-magnon interaction potential.

In this work, we report on exchange anisotropy stabilized two- and three-magnon bound states in \LCO observed via INS. Detailed measurements of the one-magnon dispersion are first presented and confirm previous results regarding the importance of the $\tilde{J}_{131}$ interaction in addition to clarifying the nature of further inter-chain interactions. Here, we also demonstrate explicitly that a uniaxially anisotropic NN intra-chain interaction alone fully captures the origin of the gap in the spectrum, \textit{i.e.} the spin gap. Next, we demonstrate the existence of additional magnetic modes, separate from the one-magnon dispersion, which we ascribe to the formation of two- and three-magnon bound states. Analysis of the momentum dependence of the two-magnon state shows a clear splitting along the $H$ and $L$ directions, which most likely stems from the influence of inter-chain interactions on the momentum dependence of the magnon-magnon interaction potential. These results establish \LCO as an accessible experimental platform for studying the physics of exchange anisotropy-driven bound states in a FM chain via INS and other spectroscopic techniques.

\section{Methods \label{methods}}

The crystal studied in this work was grown using a novel high-pressure floating-zone furnace employing a similar procedure as reported previously \cite{schmehr2019}. In this case, isotopically enriched $^{7}$LiOH$\cdotp$H$_{2}$O (Cambridge Isotopes, 99.9\%) was utilized to mitigate neutron absorption effects. The $^{7}$LiOH$\cdotp$H$_{2}$O was dried at 150 \textdegree C under a rough, dynamic vacuum for 12 hours resulting in anhydrous $^{7}$LiOH. Polycrystalline powder was then produced by grinding together a stoichiometric amount of CuO (Fisher, 99.999\%) with $^{7}$LiOH, using 5 mass \% excess of $^{7}$LiOH over the stoichiometric amount. All grinding was carried out in an inert atmosphere glove box. We note that the difference in the required excess of LiOH -- determined by the amount needed for full reaction of the CuO -- relative to our previous report \cite{schmehr2019} is due to differences in the water content. The well-mixed powder was then pressed into a pellet at 300 MPa using a cold isostatic press and fired at 750 \textdegree C in air for 24 hours. The reacted powder was then pressed into seed and feed rods ($\approx$ 4 mm diameter) at 300 MPa and fired in air for an additional 32 hours at 750 \textdegree C on a sacrificial bed of (isotopically enriched) \LCO powder. Growth was conducted under 100 bar of an Ar:O$_{2}$ = 80:20 mixture. The feed and seed rods were counter rotated at 8 and 9 rpm, respectively, and a growth rate of 10 mm/hr was used \cite{wizent2011}.

 \begin{figure}
 \includegraphics[width = 8.6 cm]{Figure2.png}
 \caption{\label{fig:Figure2}\textbf{ a)} Magnetic susceptibility ($\chi \equiv M$/$H$) at low temperature showing $T_{N}$ = 9.8 K. Note the lack of irreversibility at low-temperature ($T <$ 3 K). \textbf{b)} Field dependence of the magnetization ($M$) at low-temperature. \textit{Inset:} Low-field ($\mu_{0}H <$ 0.5 T) region.}
 \end{figure}
 
The phase purity and lattice parameters of the resulting crystal were characterized using laboratory powder X-ray diffraction (Emperyan, Panalytical). Powder diffraction data were collected on crushed crystal pieces and showed a nearly phase pure sample with the presence of a very small amount (estimated to be $<$ 1 mass \%) of an unidentified impurity. Comparison of measurements conducted on powders produced from crystalline material collected near the boule's surface versus the boule's core suggests this impurity occurs as a very thin surface layer. The refined lattice parameters were: $a$ = 3.66171(2) \r{A}, $b$ = 2.86436(7) \r{A}, and $c$ = 9.3945(2) \r{A}, consistent with literature values \cite{sapina1990}. 

Bulk magnetic properties were measured on a small crystal ($\approx$ 10 mg) cleaved from the main boule, which was mounted on a quartz paddle using GE varnish such that the applied field lay within the (101) plane. Measurements were conducted using a Quantum Design MPMS3 SQUID magnetometer in VSM mode. Field-cooled and zero-field-cooled magnetic susceptibility measurements (Fig. \ref{fig:Figure2}(a)) show $T_{N}$ = 9.8 K, with no sign of hysteresis. The field-dependence of the magnetization at $T$ = 1.8 K (Fig. \ref{fig:Figure2}(b)) is linear across the full measured range ($\mu_0 H = \pm$ 7 T), with no observable hysteresis. Previous chemical analysis on a crystal grown in a very similar manner showed Li:Cu = 1.99:1.01 \cite{schmehr2019}, and the lack of any low-temperature ferromagnetism suggests a fully occupied oxygen site \cite{shu2018}. The measured $T_{N}$ is elevated compared to previous results for crystals grown at lower pressures ($\approx$ 8 - 9 K) \cite{chung2003,lorenz2009,shu2018,kawamata2004}, likely reflecting an optimized stoichiometry.

INS measurements were conducted using the direct-geometry time-of-flight (TOF) spectrometer SEQUOIA at the Spallation Neutron Source (SNS). \LCO cleaves easily along the (101) plane and the crystal separated into pieces upon removal from the furnace. Due to the brittle nature of the fracture, the pieces could be reassembled using Al wire with reasonably good registry. The resulting sample ($\approx$ 2 g) was attached to an Al mount and aligned in the (0\textit{KL}) scattering plane using the two-axis alignment station at the High-Flux Isotope Reactor (HFIR). The sample and mount were sealed under a He environment inside an Al can to provide exchange gas for temperature control. A thermocouple was mounted to the sample can and a closed-cycle refrigerator (base temperature $\approx$ 5 K) was utilized for temperature control. The motorized incident beam mask was used to provide measurements of a ``pseudo-empty'' can for background subtractions.  For this background measurement, the mask was translated to obscure the entire sample such that only scattered neutrons from the Al sample holder and the sample environment were detected. To produce a smooth background for subtraction this data was integrated in the proper $\va*{Q}$-range and fit to an exponential decay function, $I(\Delta E)$ = $y_{0}$ + Aexp(-$\tau\Delta E$). This function yields an essentially flat background in the main $\Delta E$-range of interest where it is relevant ($\approx$ 2 to 5 meV), which was scaled to prevent over-subtraction.

Measurements were conducted by rotating the crystal around the vertical axis of the instrument in steps of 1$^\circ$ or 2$^\circ$. Measurements were conducted with incident energies $E_{i}$ = 11, 25, 50 and 185 meV to access the full bandwidth of the highly dispersive intra-chain excitations and the data were reduced using the software package Mantid \cite{arnold2014}. Where possible, the data were symmetrized by folding in order to improve the counting statistics. All error bars in intensities represent one standard deviation. 

The dispersion curves along high symmetry directions were initially extracted using Gaussian fits to constant-$\va*{Q}$, wave-vector transfer, and constant-$\Delta E$, energy transfer, cuts through the data; constant-$\Delta E$ cuts were only used for the steep portions of the dispersion along $K$ (the chain direction). Analysis of the extracted dispersion was then conducted using the SpinW software package \cite{toth2015}. For validation, the data were also directly fit using the analytical form of the dispersion relation from linear spin wave theory; both approaches gave very similar results. Magnetic structure optimization was also performed with SpinW. The optimal propagation vector was determined by the Luttinger-Tisza method and the moment direction was optimized to obtain the lowest energy configuration consistent with the propagation vector. 

\section{Spin Wave Measurements and Modeling \label{results}}

INS data were collected using multiple $E_{i}$'s in the ordered state ($T=5$ K), allowing the spin waves along almost all high-symmetry directions in the Brillouin zone to be resolved. In Fig. \ref{fig:Figure3}(a) and (b), $E_{i}$ = 11 meV data show the magnon dispersion orthogonal to the chain direction for $\Delta E <$ 6 meV. Consistent with previous measurements, both inter-chain directions are weakly dispersive relative to the intra-chain dispersion \cite{boehm1998,lorenz2009}. In Fig. \ref{fig:Figure3}(c), the much steeper intra-chain dispersion is revealed in the $E_{i}$ = 185 meV data, which approaches a zone boundary energy of $\approx$ 40 meV.  As discussed below, the dispersion could be well parameterized by linear spin wave theory. 

 \begin{figure*}[h!]
 \includegraphics[width = 17.4 cm]{Figure3.png}
 \caption{\label{fig:Figure3} Neutron scattering data collected at $T$ = 5 K along high-symmetry directions. The same scale is used in all plots. \textbf{a)} Scattering intensity along the ($H$, 0, 1) direction, collected using $E_{i}$ = 11 meV. The integration bounds are: $K$ = [-0.025, 0.025] and $L$ = [.875, 1.125]. \textbf{b)} Scattering intensity along the (0, 0, $L$) direction, collected using $E_{i}$ = 11 meV. The integration bounds are: $H$ = [-0.125, 0.125] and $K$ = [0.075, 0.075]. \textbf{c)} Scattering intensity along (0, $K$, 0) direction collected out to the zone boundary $\Delta E$ using $E_{i}$ = 185 meV. The integration bounds are: $H$ = [-2, 2] and $L$ = [-2, 2].}
 \end{figure*}
 
The spin system of \LCO has previously been described \cite{boehm1998,lorenz2009} by the following Hamiltonian:  

\begin{equation}
\hat{H}=\frac{1}{2} \sum_{\mathbf{n}, \mathbf{r}}\left[J_{\mathbf{r}}^{z} \hat{S}_{\mathbf{n}}^{z} \hat{S}_{\mathbf{n}+\mathbf{r}}^{z}+J_{\mathbf{r}} \hat{S}_{\mathbf{n}}^{+} \hat{S}_{\mathbf{n}+\mathbf{r}}^{-}\right]
\end{equation} 

Here, $J_{\mathbf{r}}^{z}$ is the component of the exchange interaction along the $a$-axis (moment direction), determined to be the magnetic easy-axis by electron spin resonance measurements \cite{ohta1993, kawamata2004}. The double sum is over the Cu-sites ($\mathbf{n}$) and the lattice vectors ($\mathbf{r}$) which connect Cu-sites. Using the convention that $J_{r} <$ 0 is FM, the dispersion relation derived via linear spin wave theory \cite{oguchi1960} for this Hamiltonian is given by the following relation:

\begin{equation}
\omega_{\mathbf{Q}}=\sqrt{\left(J_{\mathbf{Q}}-J_{\mathbf{0}}+\tilde{J}_{\mathbf{0}}-D\right)^{2}-\left(\tilde{J}_{\mathbf{Q}}\right)^{2}}
\end{equation}with $J_{Q} \equiv (1/2) \sum_{\mathbf{r}}J_{r}exp(i\va*{Q}\cdotp\va*{r})$ and  $J_{0} \equiv (1/2)\sum_{r}J_{r}$ (the same definitions apply to $\tilde{J}$). The $\va*{k}$ = [0 0 1] magnetic structure breaks the body-centered symmetry of the nuclear unit cell, creating two inequivalent Cu-sites and therefore two FM chain subsystems. The $J$ parameters represent intra- and inter-chain interactions within a subsystem, while the $\tilde{J}$ parameters represent inter-chain interactions which connect the two subsystems. In this analytical form of the dispersion, $D$ parameterizes the magnitude of the exchange anisotropy according to: $D \equiv J_{\mathbf{0}}^{z}  - J_{0} - \tilde{J}_{0}^{z} + \tilde{J}_{0}$. The spin gap, $\Delta E_{N=1}$, arises due to the finite $D$ but is also influenced by the magnitude of the inter-sublattice interactions: 

\begin{equation}
\Delta E_{N=1} = \sqrt{D(D - 2\tilde{J}_{0})} .
\end{equation} Within the SpinW implementation of linear spin wave theory the individual components of the exchange matrices are fitting parameters.

The extracted dispersion relations from the data shown in part in Fig. \ref{fig:Figure3} are plotted along select directions in Fig. \ref{fig:Figure4}. These were initially fit (blue dashed line in Fig. \ref{fig:Figure4}) using the previously reported minimal model \cite{lorenz2009} that includes only the $J_{010}$, $J_{020}$ and $\tilde{J}_{131}$ exchange interactions as well as exchange anisotropy (see Fig. \ref{fig:Figure1}(b) for an illustration of all considered exchange interactions -- note that $\tilde{J}_{131}$ is associated with $\va*{r}$ = (1/2)[1 3 1]). $J_{010}$ was set to have an easy-axis anisotropy  \cite{sapina1990} (\textit{i.e.} $J_{010}^{z}$ was fit independently of $J_{010}^{x} = J_{010}^{y} = J_{010}$) since it is by far the largest energy scale. This choice is consistent with theoretical expectations given the nature of a 90$^\circ$ Cu-O-Cu superexchange pathway (the angle in \LCO is $\approx$ 94$^\circ$) \cite{yushankhai1999}. All other exchange interactions were then set to be isotropic. As shown in Fig. \ref{fig:Figure4}, while this initial, minimal model largely captures the behavior of the dispersion within the (0, $K$, $L$) plane, including the magnitude of the spin gap, there are a number of qualitative failures in capturing the dispersion along trajectories with a finite $H$-component. 

 \begin{figure*}[h!]
 \includegraphics[width = 17.4 cm]{Figure4.png}
 \caption{\label{fig:Figure4}\textbf{a) - b)} The extracted one-magnon (N = 1) dispersion along several high symmetry directions; the $x$-axes are scaled by the magnitude of the corresponding reciprocal space basis vector ($\va*{c}$* $< \va*{a}$* $< \va*{b}$*). The solid red line shows the fit resulting from the final model described in the text. The dashed blue line shows the fit resulting from the prior minimal model without $J_{100}$, $J_{110}$ or $J_{001}$. \textbf{c)} Closer view of the (1/2 0 $L$) dispersion showing the small, but finite, bandwidth.\textbf{ d)} Closer view of the difference in the (1/2 $K$ 0) and (0 $K$ 1/2) dispersions near the $\Gamma$-point.}
 \end{figure*}

The first qualitative failure of the previous model is the prediction of a flat mode along the ($H$, 0, 1/2) direction, where the data clearly show a finite bandwidth (Fig. \ref{fig:Figure4} (a)). A second failure appears in comparing the dispersion along the (1/2, $K$, 0) and (0, $K$, 1/2) directions, as shown in Fig. \ref{fig:Figure4} (d). While the prior model gives an identical dispersion relation for these two directions, the data show a clear difference between them, and this is most apparent near the $\Gamma$-point (Fig. \ref{fig:Figure4}(d)). The third qualitative failure is more subtle. The data show a small, but finite, bandwidth to the dispersion along the (1/2, 0, $L$) direction, seen most clearly in Fig. \ref{fig:Figure4}(c), which is not captured within the prior minimal model. 

These departures from the prior minimal model motivate the inclusion of additional exchange interactions to provide a Hamiltonian which fully represents the measured dispersions. The analytical form of the dispersion shows that only an intra-sublattice exchange with a component along $L$ can induce finite dispersion along the (1/2, 0, $L$) direction. Given the relatively large $c$-axis lattice parameter ($\approx$ 9.395 \r{A}), $J_{001}$ (with the smallest associated real space distance) is the most reasonable exchange interaction to reproduce this feature. To address the departures observed in the ($H$, 0, 1/2), (0, $K$, 1/2), and (1/2, $K$, 0) dispersion relations, additional exchange terms with components along $H$ are required. This suggests the potential inclusion of two additional terms, $J_{110}$ and $J_{100}$. Fitting the dispersion with the inclusion of all three of these exchange interactions greatly improves the fits to the data. In this modified model, $J_{100}$ was found to be AFM and of a similar magnitude as $\tilde{J}_{131}$, which was found to be $\approx$ 7.7 K. $J_{110}$ and $J_{001}$ were both found to be an order of magnitude smaller and FM. 

The modified model with both $J_{110}$ and $J_{100}$ terms significantly decreased the $\chi^{2}$ value of the refinement. However, the inclusion of both terms was found to not be of equal significance. If $J_{100}$ was removed from the model the $\chi^{2}$ value increased by 37\%, while for the model without $J_{110}$ the $\chi^{2}$ value increased by only 1.5\%. The latter increase is marginal and simply due to the decrease in the total number of fit parameters (from seven to six). Therefore $J_{110}$ was set to zero in the final fits. Figure \ref{fig:Figure4} shows the resulting dispersion relations from this final model (solid red line) overplotted with the data and the corresponding parameters are presented in Table \ref{tab:Table1}. We note here that, despite the significance of the inter-chain interactions with respect to the the magnetic properties -- including the nature of the long-range order \cite{ lorenz2009, mizuno1998, giri2001,xiang2007} and the largely undiminished Cu$^{2+}$ moment \cite{sapina1990} -- the intra-chain interactions are still clearly the dominant energy scale. 

 \begin{table*}[]
 \caption{\label{tab:Table1}Comparison of the final exchange parameters (given in Kelvin) determined in this work with corresponding values given in Lorenz \textit{et al.} \cite{lorenz2009} and \cite{lorenzThesis2011}. Values in italics were fixed to zero. In our results, $J_{101}^{z}$ ( -217.4(9) K ) and $J_{010}^{y}  = J_{010}^{x}  = J_{010}$ are separate fit parameters. As a result,  $D$ is no longer a fitting parameter but is calculated as $D=J_{010}^{z} - J_{010}$. Since $D$ is a small difference between two large values, the resulting error is rather large -- direct fits to the analytical form of the dispersion gave $D$ with a lower error bar, and a value more similar to the results from Lorenz \textit{et al.} }
 \begin{ruledtabular}
\begin{tabular*}{17cm}{c|c|c|c|c|c|c|c|c} 
  & $J_{010}$ & $J_{020}$ & $\alpha =|J_{020} / J_{010}|$ & $\tilde{J}_{131}$ & $J_{100}$ & $J_{110}$ & $J_{001}$ & $D$ \\ \hline
  & & & & & & & &  \\
  This work & -213.3(9) & 67.2(3) & 0.315(2) & 7.44(2) & 5.55(4) & \textit{0} & -0.75(4) & -4(1) \\ 
  & & & & & & & &  \\
  Lorenz \textit{et al.} & -230(2) & 75(1) & 0.326(1) & 9.0(1) & \textit{0} & \textit{0} & \textit{0} & -3.3(2) \\
 \end{tabular*}
 \end{ruledtabular}
 \end{table*}

\section{Multi-magnon Bound States}

In addition to resolving the one-magnon dispersion relation, low-energy INS data also reveal additional modes near the magnetic zone-center.  As a baseline illustrating the expectations of one-magnon scattering, Fig. \ref{fig:Figure5}(a) shows a select ($E$, $K$)-slice populated with the single-magnon spectral weight and dispersion trajectories calculated by SpinW from our exchange parameters and anisotropy values in Table \ref{tab:Table1}.  INS data collected using $E_{i}$ = 11 meV at $T$ = 5 K show the scattering in this same ($E$, $K$)-slice in Fig. \ref{fig:Figure5} (b). Additional spectral weight appears in the data above the one-magnon gap at $\Delta E_{N=1}$ and, to further parameterize this, a constant momentum cut is plotted along the dashed line shown in Fig. \ref{fig:Figure5}(c). This cut along energy through the $K=0$ magnetic zone center reveals two additional features above $\Delta E_{N=1}$: a second peak appears at $\Delta E_{N=2}\approx 2\times\Delta E_{N=1}$, and a third, weaker, peak appears at $\Delta E_{N=3}\approx 3\times\Delta E_{N=1}$. The feature at $\Delta E_{N=2}$ is also clearly visible in the slices presented in Fig. \ref{fig:Figure3}(a) and (b), provided the intensity is set to a lower maximum value that saturates the one-magnon signal. 

We note that an additional, even weaker feature is visible near $\Delta E \approx$ 5.5 meV in the heat map; however, it appears near the edge of the detector coverage, making analysis less reliable. Furthermore, this region is partially polluted by an optical phonon at slightly higher energies,  making it difficult to unambiguously isolate any magnetic signal. We defer further exploration of this potential $\Delta E_{N=4}$ mode for future experiments.

 \begin{figure*}[ht]
 \includegraphics[width = 17.4 cm]{Figure5.png}
 \caption{\label{fig:Figure5}\textbf{a)} Spin wave calculation of the $T$ = 0 K, one-magnon dispersion along the (0, $K$, 1) direction  using the parameters in Table I. The calculation includes convolution with the $\Delta$E-dependent resolution function and the $\va*{Q}$-resolution was set to the Gaussian FWHM of the (001) magnetic Bragg peak. A small constant background term was added to simulate the incoherent background in the measured data.  \textbf{b)} Measured INS intensities for the same region of  ($\va*{Q}$, $\Delta E$) at $T$ = 5 K using $E_{i}$ = 11 meV. The data were integrated with the following bounds: $H$ = [-0.125, 0.125], $L$ = [0.95, 1.05]. The broad intensity below $\Delta E \approx$ 2 meV at higher $\abs{K}$ is due to the Al background, which has not been subtracted. In \textbf{a} and \textbf{b} the shared color scaling was set by first normalizing the two data sets to the value at $K$ = 0 and $\Delta E$ = $\Delta E_{1}$. To highlight the weak, higher-energy features observed at the $\Gamma$-point in \textbf{b}, the maximum of the color scaling was set to a low value. Both data sets were then normalized once more to this new maximum value. \textbf{c)} Energy cut along the dashed line in \textbf{b} (the bar at the top shows the $K$-integration region), with the Al background subtracted, showing the excitations above the one-magnon dispersion (N = 2 and 3). The data has been scaled by the same overall factor as in \textbf{b}. The blue line is a fit using Gaussians on a linear background; the solid color regions show the Gaussian components and the grey dashed line shows the background. The N = 2 and 3 modes are fit together as a sum and the N = 1 mode is fit separately using a higher $\Delta E$ point density. The black lines show the instrumental $\Delta E$ resolution (FWHM). The cut was integrated in reciprocal space with the following bounds: $H$ = [-0.075, 0.075], $K$ = [-0.05, 0.05], and $L$ = [0.95, 1.05]. \textit{Inset:} The one-magnon (N = 1) peak using the same reciprocal space integration but a higher $\Delta E$ point density.}
 \end{figure*}

A comparison of the (0, $K$, 1) slices collected at $T$ = 5 K and $T$ = 20 K is shown in Fig. \ref{fig:Figure6}(a). This comparison reveals a collapse of the one-magnon gap upon heating out of the ordered state, consistent with its origin from weak exchange anisotropy (magnitude $\approx$ 4 K). Further demonstration of this is given by the zone center energy cuts shown in the inset of Fig. \ref{fig:Figure6}(b). The analogous cut in the main panel of this figure also shows that strong paramagnon scattering remains at 20 K, as is illustrated by the slice in Fig. \ref{fig:Figure6}(a), and consistent with previous reports that \LCO possesses short-range correlations well above $T_{N}$ \cite{ebisu1998,kawamata2004}. 

These same slices and the zone center cut shown in the main panel of  Fig. \ref{fig:Figure6}(b) similarly indicate that the N = 2 peak vanishes at 20 K. However, in striking contrast, the intensity of the N = 3 peak is largely unchanged at 20 K, with only the overall background decreasing. Fitting this data (blue lines in Fig. \ref{fig:Figure6}(b)) shows that the $\Delta E$-value of the N = 3 excitation changes slightly: $\Delta E_{N=3}$ = 4.00(2) meV and 4.25(4) meV at 5 K and 20 K, respectively. The N = 3 peak vanishes by $T$ = 150 K, indicating a finite regime of stability above $T_{N}$.

 \begin{figure}[h]
 \includegraphics[width = 8.6 cm]{Figure6.png}
 \caption{\label{fig:Figure6}\textbf{a)} Temperature dependence of the INS intensities along the (0, $K$, 1) direction collected using $E_{1}$  = 11 meV. The integration ranges are: $H$ = [-0.125, 0.125], $L$ = [0.95,1.05]. The Al background has not been subtracted. \textbf{b)} Comparison of the same energy cut shown in Fig. \ref{fig:Figure5}(c) for  $T$ = 5 K ($T< T_{N}$), 20 K ($T> T_{N}$) and 150 K ($T >> T_{N}$). The data has been converted to the dynamic susceptibility using the detailed balance correction $\chi''$($\va*{Q}$, $\Delta E$) = $\pi(1 - e^{-\Delta E/k_{B}T})S$($\va*{Q}$, $\Delta E$) after subtracting the Al background. The data was then scaled using the same factor as in Fig. \ref{fig:Figure5}. The blue lines are fits using Gaussians plus a linear (solid) or exponential decay (dashed) background. The Gaussian fit for the N = 3 excitation at 20 K uses the full-width-at-half-maximum (FWHM) resulting from the fit at 5 K. \textit{Inset:} View of the one-magnon peak region at 5 K and 20 K, also plotted as $\chi''$, with a higher $\Delta E$ point density }
 \end{figure}

To study the momentum dependence of the N = 1, 2, and 3 excitations in the ordered state, constant-$\Delta E$ slices through the ($H$, $K$, 1) plane are shown in the left panels of Fig. \ref{fig:Figure7}(a), (b), and (c). At higher $\Delta E$, the one-magnon N = 1 dispersion presents as roughly parallel lines along the $H$-direction due to the much steeper dispersion along $K$, as shown in the simulated panels on the right hand side of Figure \ref{fig:Figure7}. The N = 2 and N = 3 mode intensities emerge between these lines in the data, with the strongest scattering around the $\Gamma$-point.  Taking momentum cuts along the $K$-direction, Fig. \ref{fig:Figure8}(a), right, and (c) show the presence of a single peak around $K$ = 0. Note that the stronger peaks at $K \approx \pm$ 0.2 on either side of the central peak come from the line-cuts crossing the N = 1 mode's dispersion. The persistence of signal in these regions above $T_{N}$ is, again, due to the strong paramagnon scattering. Line-cuts along the $H$-direction at $\Delta E_{N=2}$ reveal a two-peak structure around $H$ = 0, as shown in Fig. \ref{fig:Figure8}(a), left. A similar two-peak structure appears in line-cuts along the $L$-direction (not shown). Our current data set does not allow for a definitive statement on whether the same structure occurs along $H$ and $L$ at $\Delta E_{N=3}$; we aim to address this question in a future experiment. 

 \begin{figure}[h]
 \includegraphics[width = 8.6 cm]{Figure7.png}
 \caption{\label{fig:Figure7} Constant $\Delta E$ slices in the ($H$, $K$, 1) plane from $E_{i}$ = 11 meV data  (left), and corresponding spin wave calculations using the exchange constants in Table \ref{tab:Table1} (right). No Al background has been subtracted. The extent of integration in $\Delta E$ is indicated at the top of each panel. The integration in the orthogonal direction in reciprocal space is $L$ = [0.95,1.05]. The calculated values have been convolved with the instrumental $\Delta E$-dependent resolution function. For N = 1, the $\va*{Q}$-resolution of the calculated data was set to the FWHM of the (0 0 1) magnetic Bragg peak. For N = 2 and 3, a larger $\va*{Q}$ resolution was needed to approximate the measured data. A constant value has been added to the calculated intensity to approximate an overall constant incoherent background. Within each plot, the intensity has been scaled by the maximum value. The black dashed lines and bars show the directions and integration regions, respectively, of the line cuts in Fig. \ref{fig:Figure8}. \textbf{a)} N = 1 excitation. \textbf{b)} N = 2 excitation \textbf{c)} N = 3 excitation.}
 \end{figure}
 
 \begin{figure*}[ht!]
 \includegraphics[width =17.4 cm]{Figure8.png}
 \caption{\label{fig:Figure8} Cuts through reciprocal space at constant $\Delta E$ at \textit{T} = 5 K and 20 K using $E_{i}$ 11 meV data. The extent of integration in $\Delta E$, based on the fitted values from Fig. \ref{fig:Figure5}(c), is indicated at the top of the panels. All plots are on the same scale. \textbf{a)} N = 2 cuts along ($H$ 0 1) (left) and (0 $K$ 1) (right). The ($H$ 0 1) data was integrated in reciprocal space with the following bounds: $K$ = [-0.1, 0.1] and $L$ = [0.95, 1.05]. For the (0 $K$ 1) data the bounds are: $H$ = [-0.1, 0.1] and $L$ = [0.95, 1.05]. \textbf{b)} Cut along $K$ through the N = 3 excitation, with the following reciprocal space integration bounds: $H$ = [-0.1, 0.1] and $L$ = [0.95, 1.05]. }
 \end{figure*}

\section{Discussion \label{discussion}}
	
The expanded set of exchange parameters determined in fits to the one-magnon dispersion relation (summarized in Table \ref{tab:Table1}) remain consistent with the experimentally observed C-AFM structure when optimized within SpinW. These exchange parameters can be used to numerically test the conclusion from Lorenz \textit{et al.} \cite{lorenz2009} that the C-AFM state is stabilized primarily by $\tilde{J}_{131}$, with further stabilization from $D$. To do this, $\tilde{J}_{131}$ was set to 0 and $D$ was empirically modified to reproduce the measured spin gap, causing the optimized magnetic structure to transition into a spiral state. If $\tilde{J}_{131}$ is retained and the anisotropy is set to 0 (\textit{i.e.} $D$ = 0 so that $J_{010}^{x} = J_{010}^{y} = J_{010}^{z}$) then the C-AFM structure results; however the predicted moment direction shifts to align along the $c$-axis. These results indicate that, within the expanded set of exchange parameters refined in our model, the collinear magnetic order remains stabilized by the $\tilde{J}_{131}$ interaction. We note here that later, unpublished, work from the authors of W. E. Lorenz \textit{et al.} \cite{lorenz2009} also explored the role of additional exchange parameters beyond the prior minimal model \cite{lorenzThesis2011}. The presented exchange model is similar to our final model, though the need to include $J_{001}$ was not recognized nor was the relative importance of $J_{100}$ over $J_{110}$.

Our neutron scattering measurements reveal additional modes near the zone center outside of the one-magnon, N = 1, dispersion. Based on the $T$ = 5 K data shown in Fig. \ref{fig:Figure5}(c), the peak maximum of the first additional mode, $\Delta E_{N=2}$ = 2.924(5), is slightly above the value of 2$\Delta E_{N=1}$. This is suggestive of the influence of additional two-magnon continuum scattering underneath a bound state \cite{tennant2003,huberman2005}. A result for the two-magnon binding energy in a 1D FM $J_1 - J_2$ $XXZ$-Heisenberg model ($\alpha > 1/4$) was presented previously by Dmitriev and Krivnov \cite{dimitriev2009b}. Inputting our refined exchange parameters yields a theoretical binding energy of $\approx$ 0.03 meV. This calculation remains an estimate since it does not take into account the inter-chain interaction $\tilde{J}_{131}$, which is needed to stabilize the FM chains. The presence of such an interaction is expected to slightly reduce the binding energies of multimagnon bound states \cite{nishimoto2015}. Our observation of a single peak in the spectrum near 2$\Delta E_{N=1}$ is qualitatively consistent with a small binding energy of this magnitude, which is obscured by the presence of continuum scattering. Both N = 1 and N = 2 modes collapse upon warming above $T_N$, consistent with their formation via an interplay between the inter-chain coupling $\tilde{J}_{131}$ stabilizing the magnetic ground state and the Ising-like exchange anisotropy. 

The splitting observed within the N = 2 excitation along $H$ (Fig. \ref{fig:Figure8}(a), left) and $L$ is indicative of a bound state superimposed onto the continuum scattering. Naively, the intensity of a two-magnon continuum is expected to be peaked at the magnetic zone center \cite{huberman2005} with structure possible near the zone boundary \cite{barnes2003}. The appearance of a splitting around the zone center reflects additional interactions. Such a splitting can potentially arise within a bound state when the potential that binds together the two interacting magnons (and which does not act upon the delocalized two-particle states of the continuum \cite{barnes2003}) possesses a $\va*{Q}$-dependence \cite{schneider1981}. Given the existence of significant inter-chain exchange interactions with components along $H$ and $L$, we ascribe these as the likely origin.

The N = 3 mode is located at $\Delta E_{N=3}$ = 4.00(1) meV (see Fig. \ref{fig:Figure5}(c)) yielding 3$\Delta E_{N=1} - \Delta E_{N=3}$ = 0.08(1) meV, which exceeds the statistical error in the fitting procedure. However, additional systematic error in the peak position introduced by the finite reciprocal space integration window means that this value likely falls within the resolution of our measurement. Generally, a three-magnon binding energy is expected to be larger than a two-magnon one, with the simplest picture being a factor of two increase. Given the estimation of the two-magnon binding energy it is not surprising that we are unable to resolve the three-magnon value. However, we are not currently aware of any theoretical calculation to which we can compare our results. The energy width of this mode is 0.56(3) meV (Gaussian FWHM), relative to an instrumental resolution of 0.19 meV at $\Delta E=4.00$ meV, placing the mode well within an underdamped regime. The persistence of the N = 3 mode upon warming above $T_N$, where the one-magnon gap collapses, implies that its stabilization mechanism is distinct from that discussed previously for the N = 2 mode.

Models for three-magnon condensates have been proposed for the parameter space that \LCO occupies within $J_{1} - J_{2}$ ferromagnetic spin chain models \cite{kecke2007, nishimoto2011,nishimoto2015}. The survival of the N = 3 mode in a finite range above $T_N$ is potentially relevant to these models, though it would require an unconventional setting and the presence of a higher energy one-magnon mode not currently resolved in our measurements. Based on present calculations, these models also indicate that the relatively strong $\tilde{J}_{131}$ should move \LCO out of the regime of multipolar ground states \cite{nishimoto2011, nishimoto2015}. Utilization of our refined exchange parameters within these models may help resolve this question. Extension of recent numerical calculations of the dynamical structure factor at finite temperature to the specific case of \LCO might also be enlightening \cite{nayak2022}. On the experimental side, INS measurements exploring the polarization and dispersion of the N = 3 mode would be of use in identifying its origin. 

\section{Conclusions \label{conclusions}}

INS data exploring the magnetic excitations in \LCO were presented and used to refine the previously published exchange model. The new analysis confirms the importance of the $\tilde{J}_{131}$ interaction and determines the nature of further inter-chain interactions, both of which are consistent with the commensurate AFM ground state. Additional modes within the INS spectrum cannot be described in terms of the one-magnon dispersion, and we identify these as two- and three-magnon bound states with very small binding energies. The persistence of the three-magnon excitation in a finite range of $T > T_{N}$ is anomalous and motivates further investigation of \LCO as a platform for exchange anisotropy induced few-body bound states in Heisenberg ferromagnetic chains.  

\FloatBarrier

\begin{acknowledgments}
We thank Alexander Kolesnikov for assistance with the inelastic neutron scattering experiment and acknowledge insightful discussions with Leon Balents, Satoshi Nishimoto, and Clio Agrapidis.  This work was supported by the US Department of Energy (DOE), Office of Basic Energy Sciences, Division of Materials Sciences and Engineering under Grant No. DE-SC0017752.  A portion of this research used resources at the Spallation Neutron Source, a DOE Office of Science User Facility operated by the Oak Ridge National Laboratory. This work made use of the MRL Shared Experimental Facilities which are supported by the MRSEC Program of the NSF under Award No. DMR 1720256, a member of the NSF-funded Materials Research Facilities Network. It also used facilities supported via the UC Santa Barbara NSF Quantum Foundry funded via the Q-AMASE-i program under award DMR-1906325.SCGSR fellowship. This work was additionally supported by the U.S. Department of Energy, Office of Science, Office of Workforce Development for Teachers and Scientists, Office of Science Graduate Student Research (SCGSR) program. The SCGSR program is administered by the Oak Ridge Institute for Science and Education for the DOE under contract number DE‐SC0014664. E.Z. recognizes financial support from Johns Hopkins University through the Sweeney Family Postdoctoral Fellowship.
\end{acknowledgments}

% Create the bibliography  using BibTeX:
%\bibliography{Bibliography}

% Include bibliography directly in Tex file:
%apsrev4-2.bst 2019-01-14 (MD) hand-edited version of apsrev4-1.bst
%Control: key (0)
%Control: author (8) initials jnrlst
%Control: editor formatted (1) identically to author
%Control: production of article title (0) allowed
%Control: page (0) single
%Control: year (1) truncated
%Control: production of eprint (0) enabled
\begin{thebibliography}{69}%
\makeatletter
\providecommand \@ifxundefined [1]{%
 \@ifx{#1\undefined}
}%
\providecommand \@ifnum [1]{%
 \ifnum #1\expandafter \@firstoftwo
 \else \expandafter \@secondoftwo
 \fi
}%
\providecommand \@ifx [1]{%
 \ifx #1\expandafter \@firstoftwo
 \else \expandafter \@secondoftwo
 \fi
}%
\providecommand \natexlab [1]{#1}%
\providecommand \enquote  [1]{``#1''}%
\providecommand \bibnamefont  [1]{#1}%
\providecommand \bibfnamefont [1]{#1}%
\providecommand \citenamefont [1]{#1}%
\providecommand \href@noop [0]{\@secondoftwo}%
\providecommand \href [0]{\begingroup \@sanitize@url \@href}%
\providecommand \@href[1]{\@@startlink{#1}\@@href}%
\providecommand \@@href[1]{\endgroup#1\@@endlink}%
\providecommand \@sanitize@url [0]{\catcode `\\12\catcode `\$12\catcode
  `\&12\catcode `\#12\catcode `\^12\catcode `\_12\catcode `\%12\relax}%
\providecommand \@@startlink[1]{}%
\providecommand \@@endlink[0]{}%
\providecommand \url  [0]{\begingroup\@sanitize@url \@url }%
\providecommand \@url [1]{\endgroup\@href {#1}{\urlprefix }}%
\providecommand \urlprefix  [0]{URL }%
\providecommand \Eprint [0]{\href }%
\providecommand \doibase [0]{https://doi.org/}%
\providecommand \selectlanguage [0]{\@gobble}%
\providecommand \bibinfo  [0]{\@secondoftwo}%
\providecommand \bibfield  [0]{\@secondoftwo}%
\providecommand \translation [1]{[#1]}%
\providecommand \BibitemOpen [0]{}%
\providecommand \bibitemStop [0]{}%
\providecommand \bibitemNoStop [0]{.\EOS\space}%
\providecommand \EOS [0]{\spacefactor3000\relax}%
\providecommand \BibitemShut  [1]{\csname bibitem#1\endcsname}%
\let\auto@bib@innerbib\@empty
%</preamble>
\bibitem [{\citenamefont {Schiffler}\ and\ \citenamefont
  {Müller-Buschaum}(1986)}]{schiffler1986}%
  \BibitemOpen
  \bibfield  {author} {\bibinfo {author} {\bibfnamefont {S.}~\bibnamefont
  {Schiffler}}\ and\ \bibinfo {author} {\bibfnamefont {H.}~\bibnamefont
  {Müller-Buschaum}},\ }\href
  {https://doi.org/https://doi.org/10.1002/zaac.19865320103} {\bibfield
  {journal} {\bibinfo  {journal} {\textit{Z. Anorg. Allg. Chem.}}\ }\textbf
  {\bibinfo {volume} {532}},\ \bibinfo {pages} {10} (\bibinfo {year}
  {1986})}\BibitemShut {NoStop}%
\bibitem [{\citenamefont {Yamashita}\ \emph {et~al.}(2000)\citenamefont
  {Yamashita}, \citenamefont {Ishii},\ and\ \citenamefont
  {Matsuzaka}}]{yamashita2000}%
  \BibitemOpen
  \bibfield  {author} {\bibinfo {author} {\bibfnamefont {M.}~\bibnamefont
  {Yamashita}}, \bibinfo {author} {\bibfnamefont {T.}~\bibnamefont {Ishii}},\
  and\ \bibinfo {author} {\bibfnamefont {H.}~\bibnamefont {Matsuzaka}},\ }\href
  {https://doi.org/https://doi.org/10.1016/S0010-8545(99)00212-X} {\bibfield
  {journal} {\bibinfo  {journal} {\textit{Coord. Chem. Rev.}}\ }\textbf
  {\bibinfo {volume} {198}},\ \bibinfo {pages} {347} (\bibinfo {year}
  {2000})}\BibitemShut {NoStop}%
\bibitem [{\citenamefont {Xu}\ \emph {et~al.}(1996)\citenamefont {Xu},
  \citenamefont {DiTusa}, \citenamefont {Ito}, \citenamefont {Oka},
  \citenamefont {Takagi}, \citenamefont {Broholm},\ and\ \citenamefont
  {Aeppli}}]{xu1996}%
  \BibitemOpen
  \bibfield  {author} {\bibinfo {author} {\bibfnamefont {G.}~\bibnamefont
  {Xu}}, \bibinfo {author} {\bibfnamefont {J.~F.}\ \bibnamefont {DiTusa}},
  \bibinfo {author} {\bibfnamefont {T.}~\bibnamefont {Ito}}, \bibinfo {author}
  {\bibfnamefont {K.}~\bibnamefont {Oka}}, \bibinfo {author} {\bibfnamefont
  {H.}~\bibnamefont {Takagi}}, \bibinfo {author} {\bibfnamefont
  {C.}~\bibnamefont {Broholm}},\ and\ \bibinfo {author} {\bibfnamefont
  {G.}~\bibnamefont {Aeppli}},\ }\href
  {https://doi.org/10.1103/PhysRevB.54.R6827} {\bibfield  {journal} {\bibinfo
  {journal} {\textit{Phys. Rev. B}}\ }\textbf {\bibinfo {volume} {54}},\
  \bibinfo {pages} {R6827} (\bibinfo {year} {1996})}\BibitemShut {NoStop}%
\bibitem [{\citenamefont {Yokoo}\ \emph {et~al.}(1995)\citenamefont {Yokoo},
  \citenamefont {Sakaguchi}, \citenamefont {Kakurai},\ and\ \citenamefont
  {Akimitsu}}]{yokoo1995}%
  \BibitemOpen
  \bibfield  {author} {\bibinfo {author} {\bibfnamefont {T.}~\bibnamefont
  {Yokoo}}, \bibinfo {author} {\bibfnamefont {T.}~\bibnamefont {Sakaguchi}},
  \bibinfo {author} {\bibfnamefont {K.}~\bibnamefont {Kakurai}},\ and\ \bibinfo
  {author} {\bibfnamefont {J.}~\bibnamefont {Akimitsu}},\ }\href
  {https://doi.org/10.1143/JPSJ.64.3651} {\bibfield  {journal} {\bibinfo
  {journal} {\textit{J. Phys. Soc. Japan}}\ }\textbf {\bibinfo {volume} {64}},\
  \bibinfo {pages} {3651} (\bibinfo {year} {1995})}\BibitemShut {NoStop}%
\bibitem [{\citenamefont {Nagler}\ \emph {et~al.}(1991)\citenamefont {Nagler},
  \citenamefont {Tennant}, \citenamefont {Cowley}, \citenamefont {Perring},\
  and\ \citenamefont {Satija}}]{nagler1991}%
  \BibitemOpen
  \bibfield  {author} {\bibinfo {author} {\bibfnamefont {S.~E.}\ \bibnamefont
  {Nagler}}, \bibinfo {author} {\bibfnamefont {D.~A.}\ \bibnamefont {Tennant}},
  \bibinfo {author} {\bibfnamefont {R.~A.}\ \bibnamefont {Cowley}}, \bibinfo
  {author} {\bibfnamefont {T.~G.}\ \bibnamefont {Perring}},\ and\ \bibinfo
  {author} {\bibfnamefont {S.~K.}\ \bibnamefont {Satija}},\ }\href
  {https://doi.org/10.1103/PhysRevB.44.12361} {\bibfield  {journal} {\bibinfo
  {journal} {\textit{Phys. Rev. B}}\ }\textbf {\bibinfo {volume} {44}},\
  \bibinfo {pages} {12361} (\bibinfo {year} {1991})}\BibitemShut {NoStop}%
\bibitem [{\citenamefont {Tennant}\ \emph {et~al.}(1993)\citenamefont
  {Tennant}, \citenamefont {Perring}, \citenamefont {Cowley},\ and\
  \citenamefont {Nagler}}]{tennant1993}%
  \BibitemOpen
  \bibfield  {author} {\bibinfo {author} {\bibfnamefont {D.~A.}\ \bibnamefont
  {Tennant}}, \bibinfo {author} {\bibfnamefont {T.~G.}\ \bibnamefont
  {Perring}}, \bibinfo {author} {\bibfnamefont {R.~A.}\ \bibnamefont
  {Cowley}},\ and\ \bibinfo {author} {\bibfnamefont {S.~E.}\ \bibnamefont
  {Nagler}},\ }\href {https://doi.org/10.1103/PhysRevLett.70.4003} {\bibfield
  {journal} {\bibinfo  {journal} {\textit{Phys. Rev. Lett.}}\ }\textbf
  {\bibinfo {volume} {70}},\ \bibinfo {pages} {4003} (\bibinfo {year}
  {1993})}\BibitemShut {NoStop}%
\bibitem [{\citenamefont {Tennant}\ \emph {et~al.}(1995)\citenamefont
  {Tennant}, \citenamefont {Cowley}, \citenamefont {Nagler},\ and\
  \citenamefont {Tsvelik}}]{tennant1995}%
  \BibitemOpen
  \bibfield  {author} {\bibinfo {author} {\bibfnamefont {D.~A.}\ \bibnamefont
  {Tennant}}, \bibinfo {author} {\bibfnamefont {R.~A.}\ \bibnamefont {Cowley}},
  \bibinfo {author} {\bibfnamefont {S.~E.}\ \bibnamefont {Nagler}},\ and\
  \bibinfo {author} {\bibfnamefont {A.~M.}\ \bibnamefont {Tsvelik}},\ }\href
  {https://doi.org/10.1103/PhysRevB.52.13368} {\bibfield  {journal} {\bibinfo
  {journal} {\textit{Phys. Rev. B}}\ }\textbf {\bibinfo {volume} {52}},\
  \bibinfo {pages} {13368} (\bibinfo {year} {1995})}\BibitemShut {NoStop}%
\bibitem [{\citenamefont {Kim}\ \emph {et~al.}(2006)\citenamefont {Kim},
  \citenamefont {Koh}, \citenamefont {Rotenberg}, \citenamefont {Oh},
  \citenamefont {Eisaki}, \citenamefont {Motoyama}, \citenamefont {Uchida},
  \citenamefont {Tohyama}, \citenamefont {Maekawa}, \citenamefont {Shen} \emph
  {et~al.}}]{kim2006}%
  \BibitemOpen
  \bibfield  {author} {\bibinfo {author} {\bibfnamefont {B.}~\bibnamefont
  {Kim}}, \bibinfo {author} {\bibfnamefont {H.}~\bibnamefont {Koh}}, \bibinfo
  {author} {\bibfnamefont {E.}~\bibnamefont {Rotenberg}}, \bibinfo {author}
  {\bibfnamefont {S.-J.}\ \bibnamefont {Oh}}, \bibinfo {author} {\bibfnamefont
  {H.}~\bibnamefont {Eisaki}}, \bibinfo {author} {\bibfnamefont
  {N.}~\bibnamefont {Motoyama}}, \bibinfo {author} {\bibfnamefont {S.-i.}\
  \bibnamefont {Uchida}}, \bibinfo {author} {\bibfnamefont {T.}~\bibnamefont
  {Tohyama}}, \bibinfo {author} {\bibfnamefont {S.}~\bibnamefont {Maekawa}},
  \bibinfo {author} {\bibfnamefont {Z.-X.}\ \bibnamefont {Shen}}, \emph
  {et~al.},\ }\href {https://doi.org/10.1038/nphys316} {\bibfield  {journal}
  {\bibinfo  {journal} {\textit{Nature Phys.}}\ }\textbf {\bibinfo {volume}
  {2}},\ \bibinfo {pages} {397} (\bibinfo {year} {2006})}\BibitemShut {NoStop}%
\bibitem [{\citenamefont {Kinross}\ \emph {et~al.}(2014)\citenamefont
  {Kinross}, \citenamefont {Fu}, \citenamefont {Munsie}, \citenamefont
  {Dabkowska}, \citenamefont {Luke}, \citenamefont {Sachdev},\ and\
  \citenamefont {Imai}}]{kinross2014}%
  \BibitemOpen
  \bibfield  {author} {\bibinfo {author} {\bibfnamefont {A.~W.}\ \bibnamefont
  {Kinross}}, \bibinfo {author} {\bibfnamefont {M.}~\bibnamefont {Fu}},
  \bibinfo {author} {\bibfnamefont {T.~J.}\ \bibnamefont {Munsie}}, \bibinfo
  {author} {\bibfnamefont {H.~A.}\ \bibnamefont {Dabkowska}}, \bibinfo {author}
  {\bibfnamefont {G.~M.}\ \bibnamefont {Luke}}, \bibinfo {author}
  {\bibfnamefont {S.}~\bibnamefont {Sachdev}},\ and\ \bibinfo {author}
  {\bibfnamefont {T.}~\bibnamefont {Imai}},\ }\href
  {https://doi.org/10.1103/PhysRevX.4.031008} {\bibfield  {journal} {\bibinfo
  {journal} {\textit{Phys. Rev. X}}\ }\textbf {\bibinfo {volume} {4}},\
  \bibinfo {pages} {031008} (\bibinfo {year} {2014})}\BibitemShut {NoStop}%
\bibitem [{\citenamefont {Morris}\ \emph {et~al.}(2014)\citenamefont {Morris},
  \citenamefont {Vald\'es~Aguilar}, \citenamefont {Ghosh}, \citenamefont
  {Koohpayeh}, \citenamefont {Krizan}, \citenamefont {Cava}, \citenamefont
  {Tchernyshyov}, \citenamefont {McQueen},\ and\ \citenamefont
  {Armitage}}]{morris2014}%
  \BibitemOpen
  \bibfield  {author} {\bibinfo {author} {\bibfnamefont {C.~M.}\ \bibnamefont
  {Morris}}, \bibinfo {author} {\bibfnamefont {R.}~\bibnamefont
  {Vald\'es~Aguilar}}, \bibinfo {author} {\bibfnamefont {A.}~\bibnamefont
  {Ghosh}}, \bibinfo {author} {\bibfnamefont {S.~M.}\ \bibnamefont
  {Koohpayeh}}, \bibinfo {author} {\bibfnamefont {J.}~\bibnamefont {Krizan}},
  \bibinfo {author} {\bibfnamefont {R.~J.}\ \bibnamefont {Cava}}, \bibinfo
  {author} {\bibfnamefont {O.}~\bibnamefont {Tchernyshyov}}, \bibinfo {author}
  {\bibfnamefont {T.~M.}\ \bibnamefont {McQueen}},\ and\ \bibinfo {author}
  {\bibfnamefont {N.~P.}\ \bibnamefont {Armitage}},\ }\href
  {https://doi.org/10.1103/PhysRevLett.112.137403} {\bibfield  {journal}
  {\bibinfo  {journal} {\textit{Phys. Rev. Lett.}}\ }\textbf {\bibinfo {volume}
  {112}},\ \bibinfo {pages} {137403} (\bibinfo {year} {2014})}\BibitemShut
  {NoStop}%
\bibitem [{\citenamefont {Faure}\ \emph {et~al.}(2018)\citenamefont {Faure},
  \citenamefont {Takayoshi}, \citenamefont {Petit}, \citenamefont {Simonet},
  \citenamefont {Raymond}, \citenamefont {Regnault}, \citenamefont {Boehm},
  \citenamefont {White}, \citenamefont {M{\aa}nsson}, \citenamefont {R{\"u}egg}
  \emph {et~al.}}]{faure2018}%
  \BibitemOpen
  \bibfield  {author} {\bibinfo {author} {\bibfnamefont {Q.}~\bibnamefont
  {Faure}}, \bibinfo {author} {\bibfnamefont {S.}~\bibnamefont {Takayoshi}},
  \bibinfo {author} {\bibfnamefont {S.}~\bibnamefont {Petit}}, \bibinfo
  {author} {\bibfnamefont {V.}~\bibnamefont {Simonet}}, \bibinfo {author}
  {\bibfnamefont {S.}~\bibnamefont {Raymond}}, \bibinfo {author} {\bibfnamefont
  {L.-P.}\ \bibnamefont {Regnault}}, \bibinfo {author} {\bibfnamefont
  {M.}~\bibnamefont {Boehm}}, \bibinfo {author} {\bibfnamefont {J.~S.}\
  \bibnamefont {White}}, \bibinfo {author} {\bibfnamefont {M.}~\bibnamefont
  {M{\aa}nsson}}, \bibinfo {author} {\bibfnamefont {C.}~\bibnamefont
  {R{\"u}egg}}, \emph {et~al.},\ }\href
  {https://doi.org/10.1038/s41567-018-0126-8} {\bibfield  {journal} {\bibinfo
  {journal} {\textit{Nature Phys.}}\ }\textbf {\bibinfo {volume} {14}},\
  \bibinfo {pages} {716} (\bibinfo {year} {2018})}\BibitemShut {NoStop}%
\bibitem [{\citenamefont {Niesen}\ \emph {et~al.}(2013)\citenamefont {Niesen},
  \citenamefont {Kolland}, \citenamefont {Seher}, \citenamefont {Breunig},
  \citenamefont {Valldor}, \citenamefont {Braden}, \citenamefont {Grenier},\
  and\ \citenamefont {Lorenz}}]{niesen2013}%
  \BibitemOpen
  \bibfield  {author} {\bibinfo {author} {\bibfnamefont {S.~K.}\ \bibnamefont
  {Niesen}}, \bibinfo {author} {\bibfnamefont {G.}~\bibnamefont {Kolland}},
  \bibinfo {author} {\bibfnamefont {M.}~\bibnamefont {Seher}}, \bibinfo
  {author} {\bibfnamefont {O.}~\bibnamefont {Breunig}}, \bibinfo {author}
  {\bibfnamefont {M.}~\bibnamefont {Valldor}}, \bibinfo {author} {\bibfnamefont
  {M.}~\bibnamefont {Braden}}, \bibinfo {author} {\bibfnamefont
  {B.}~\bibnamefont {Grenier}},\ and\ \bibinfo {author} {\bibfnamefont
  {T.}~\bibnamefont {Lorenz}},\ }\href@noop {} {\bibfield  {journal} {\bibinfo
  {journal} {\textit{Phys. Rev. B}}\ }\textbf {\bibinfo {volume} {87}},\
  \bibinfo {pages} {224413} (\bibinfo {year} {2013})}\BibitemShut {NoStop}%
\bibitem [{\citenamefont {Dally}\ \emph {et~al.}(2020)\citenamefont {Dally},
  \citenamefont {Heng}, \citenamefont {Keselman}, \citenamefont {Bordelon},
  \citenamefont {Stone}, \citenamefont {Balents},\ and\ \citenamefont
  {Wilson}}]{dally2020}%
  \BibitemOpen
  \bibfield  {author} {\bibinfo {author} {\bibfnamefont {R.~L.}\ \bibnamefont
  {Dally}}, \bibinfo {author} {\bibfnamefont {A.~J.~R.}\ \bibnamefont {Heng}},
  \bibinfo {author} {\bibfnamefont {A.}~\bibnamefont {Keselman}}, \bibinfo
  {author} {\bibfnamefont {M.~M.}\ \bibnamefont {Bordelon}}, \bibinfo {author}
  {\bibfnamefont {M.~B.}\ \bibnamefont {Stone}}, \bibinfo {author}
  {\bibfnamefont {L.}~\bibnamefont {Balents}},\ and\ \bibinfo {author}
  {\bibfnamefont {S.~D.}\ \bibnamefont {Wilson}},\ }\href
  {https://doi.org/10.1103/PhysRevLett.124.197203} {\bibfield  {journal}
  {\bibinfo  {journal} {\textit{Phys. Rev. Lett.}}\ }\textbf {\bibinfo {volume}
  {124}},\ \bibinfo {pages} {197203} (\bibinfo {year} {2020})}\BibitemShut
  {NoStop}%
\bibitem [{\citenamefont {Garrett}\ \emph {et~al.}(1997)\citenamefont
  {Garrett}, \citenamefont {Nagler}, \citenamefont {Tennant}, \citenamefont
  {Sales},\ and\ \citenamefont {Barnes}}]{garret1997}%
  \BibitemOpen
  \bibfield  {author} {\bibinfo {author} {\bibfnamefont {A.~W.}\ \bibnamefont
  {Garrett}}, \bibinfo {author} {\bibfnamefont {S.~E.}\ \bibnamefont {Nagler}},
  \bibinfo {author} {\bibfnamefont {D.~A.}\ \bibnamefont {Tennant}}, \bibinfo
  {author} {\bibfnamefont {B.~C.}\ \bibnamefont {Sales}},\ and\ \bibinfo
  {author} {\bibfnamefont {T.}~\bibnamefont {Barnes}},\ }\href
  {https://doi.org/10.1103/PhysRevLett.79.745} {\bibfield  {journal} {\bibinfo
  {journal} {\textit{Phys. Rev. Lett.}}\ }\textbf {\bibinfo {volume} {79}},\
  \bibinfo {pages} {745} (\bibinfo {year} {1997})}\BibitemShut {NoStop}%
\bibitem [{\citenamefont {Tennant}\ \emph {et~al.}(2003)\citenamefont
  {Tennant}, \citenamefont {Broholm}, \citenamefont {Reich}, \citenamefont
  {Nagler}, \citenamefont {Granroth}, \citenamefont {Barnes}, \citenamefont
  {Damle}, \citenamefont {Xu}, \citenamefont {Chen},\ and\ \citenamefont
  {Sales}}]{tennant2003}%
  \BibitemOpen
  \bibfield  {author} {\bibinfo {author} {\bibfnamefont {D.~A.}\ \bibnamefont
  {Tennant}}, \bibinfo {author} {\bibfnamefont {C.}~\bibnamefont {Broholm}},
  \bibinfo {author} {\bibfnamefont {D.~H.}\ \bibnamefont {Reich}}, \bibinfo
  {author} {\bibfnamefont {S.~E.}\ \bibnamefont {Nagler}}, \bibinfo {author}
  {\bibfnamefont {G.~E.}\ \bibnamefont {Granroth}}, \bibinfo {author}
  {\bibfnamefont {T.}~\bibnamefont {Barnes}}, \bibinfo {author} {\bibfnamefont
  {K.}~\bibnamefont {Damle}}, \bibinfo {author} {\bibfnamefont
  {G.}~\bibnamefont {Xu}}, \bibinfo {author} {\bibfnamefont {Y.}~\bibnamefont
  {Chen}},\ and\ \bibinfo {author} {\bibfnamefont {B.~C.}\ \bibnamefont
  {Sales}},\ }\href {https://doi.org/10.1103/PhysRevB.67.054414} {\bibfield
  {journal} {\bibinfo  {journal} {\textit{Phys. Rev. B}}\ }\textbf {\bibinfo
  {volume} {67}},\ \bibinfo {pages} {054414} (\bibinfo {year}
  {2003})}\BibitemShut {NoStop}%
\bibitem [{\citenamefont {Bai}\ \emph {et~al.}(2021)\citenamefont {Bai},
  \citenamefont {Zhang}, \citenamefont {Dun}, \citenamefont {Zhang},
  \citenamefont {Huang}, \citenamefont {Zhou}, \citenamefont {Stone},
  \citenamefont {Kolesnikov}, \citenamefont {Ye}, \citenamefont {Batista} \emph
  {et~al.}}]{bai2021}%
  \BibitemOpen
  \bibfield  {author} {\bibinfo {author} {\bibfnamefont {X.}~\bibnamefont
  {Bai}}, \bibinfo {author} {\bibfnamefont {S.-S.}\ \bibnamefont {Zhang}},
  \bibinfo {author} {\bibfnamefont {Z.}~\bibnamefont {Dun}}, \bibinfo {author}
  {\bibfnamefont {H.}~\bibnamefont {Zhang}}, \bibinfo {author} {\bibfnamefont
  {Q.}~\bibnamefont {Huang}}, \bibinfo {author} {\bibfnamefont
  {H.}~\bibnamefont {Zhou}}, \bibinfo {author} {\bibfnamefont {M.~B.}\
  \bibnamefont {Stone}}, \bibinfo {author} {\bibfnamefont {A.~I.}\ \bibnamefont
  {Kolesnikov}}, \bibinfo {author} {\bibfnamefont {F.}~\bibnamefont {Ye}},
  \bibinfo {author} {\bibfnamefont {C.~D.}\ \bibnamefont {Batista}}, \emph
  {et~al.},\ }\href
  {https://doi.org/https://doi.org/10.1038/s41567-020-01110-1} {\bibfield
  {journal} {\bibinfo  {journal} {\textit{Nature Phys.}}\ }\textbf {\bibinfo
  {volume} {17}},\ \bibinfo {pages} {467} (\bibinfo {year} {2021})}\BibitemShut
  {NoStop}%
\bibitem [{\citenamefont {Legros}\ \emph {et~al.}(2021)\citenamefont {Legros},
  \citenamefont {Zhang}, \citenamefont {Bai}, \citenamefont {Zhang},
  \citenamefont {Dun}, \citenamefont {Phelan}, \citenamefont {Batista},
  \citenamefont {Mourigal},\ and\ \citenamefont {Armitage}}]{legros2021}%
  \BibitemOpen
  \bibfield  {author} {\bibinfo {author} {\bibfnamefont {A.}~\bibnamefont
  {Legros}}, \bibinfo {author} {\bibfnamefont {S.-S.}\ \bibnamefont {Zhang}},
  \bibinfo {author} {\bibfnamefont {X.}~\bibnamefont {Bai}}, \bibinfo {author}
  {\bibfnamefont {H.}~\bibnamefont {Zhang}}, \bibinfo {author} {\bibfnamefont
  {Z.}~\bibnamefont {Dun}}, \bibinfo {author} {\bibfnamefont {W.~A.}\
  \bibnamefont {Phelan}}, \bibinfo {author} {\bibfnamefont {C.~D.}\
  \bibnamefont {Batista}}, \bibinfo {author} {\bibfnamefont {M.}~\bibnamefont
  {Mourigal}},\ and\ \bibinfo {author} {\bibfnamefont {N.~P.}\ \bibnamefont
  {Armitage}},\ }\href
  {https://doi.org/https://doi.org/10.1103/PhysRevLett.127.267201} {\bibfield
  {journal} {\bibinfo  {journal} {\textit{Phys. Rev. Lett.}}\ }\textbf
  {\bibinfo {volume} {127}},\ \bibinfo {pages} {267201} (\bibinfo {year}
  {2021})}\BibitemShut {NoStop}%
\bibitem [{\citenamefont {Torrance}\ and\ \citenamefont
  {Tinkham}(1969)}]{torrance1969}%
  \BibitemOpen
  \bibfield  {author} {\bibinfo {author} {\bibfnamefont {J.~B.}\ \bibnamefont
  {Torrance}}\ and\ \bibinfo {author} {\bibfnamefont {M.}~\bibnamefont
  {Tinkham}},\ }\href {https://doi.org/10.1103/PhysRev.187.595} {\bibfield
  {journal} {\bibinfo  {journal} {\textit{Phys. Rev.}}\ }\textbf {\bibinfo
  {volume} {187}},\ \bibinfo {pages} {595} (\bibinfo {year}
  {1969})}\BibitemShut {NoStop}%
\bibitem [{\citenamefont {Hoogerbeets}\ \emph {et~al.}(1984)\citenamefont
  {Hoogerbeets}, \citenamefont {van Duyneveldt}, \citenamefont {Phaff},
  \citenamefont {Swuste},\ and\ \citenamefont {de~Jonge}}]{hoogerbeets1984}%
  \BibitemOpen
  \bibfield  {author} {\bibinfo {author} {\bibfnamefont {R.}~\bibnamefont
  {Hoogerbeets}}, \bibinfo {author} {\bibfnamefont {A.~J.}\ \bibnamefont {van
  Duyneveldt}}, \bibinfo {author} {\bibfnamefont {A.~C.}\ \bibnamefont
  {Phaff}}, \bibinfo {author} {\bibfnamefont {C.~H.~W.}\ \bibnamefont
  {Swuste}},\ and\ \bibinfo {author} {\bibfnamefont {W.~J.~M.}\ \bibnamefont
  {de~Jonge}},\ }\href {https://doi.org/10.1088/0022-3719/17/14/016} {\bibfield
   {journal} {\bibinfo  {journal} {\textit{J. Phys. C: Solid State Phys.}}\
  }\textbf {\bibinfo {volume} {17}},\ \bibinfo {pages} {2595} (\bibinfo {year}
  {1984})}\BibitemShut {NoStop}%
\bibitem [{\citenamefont {Sreedhar}\ and\ \citenamefont
  {Ganguly}(1988)}]{sreedhar1988}%
  \BibitemOpen
  \bibfield  {author} {\bibinfo {author} {\bibfnamefont {K.}~\bibnamefont
  {Sreedhar}}\ and\ \bibinfo {author} {\bibfnamefont {P.}~\bibnamefont
  {Ganguly}},\ }\href {https://doi.org/10.1021/ic00286a011} {\bibfield
  {journal} {\bibinfo  {journal} {\textit{Inorg. Chem.}}\ }\textbf {\bibinfo
  {volume} {27}},\ \bibinfo {pages} {2261} (\bibinfo {year}
  {1988})}\BibitemShut {NoStop}%
\bibitem [{\citenamefont {Sapina}\ \emph {et~al.}(1990)\citenamefont {Sapina},
  \citenamefont {Rodriguez-Carvajal}, \citenamefont {Sanchis}, \citenamefont
  {Ibanez}, \citenamefont {Beltran},\ and\ \citenamefont
  {Beltran}}]{sapina1990}%
  \BibitemOpen
  \bibfield  {author} {\bibinfo {author} {\bibfnamefont {F.}~\bibnamefont
  {Sapina}}, \bibinfo {author} {\bibfnamefont {J.}~\bibnamefont
  {Rodriguez-Carvajal}}, \bibinfo {author} {\bibfnamefont {M.}~\bibnamefont
  {Sanchis}}, \bibinfo {author} {\bibfnamefont {R.}~\bibnamefont {Ibanez}},
  \bibinfo {author} {\bibfnamefont {A.}~\bibnamefont {Beltran}},\ and\ \bibinfo
  {author} {\bibfnamefont {D.}~\bibnamefont {Beltran}},\ }\href
  {https://doi.org/10.1016/0038-1098(90)90934-4} {\bibfield  {journal}
  {\bibinfo  {journal} {\textit{Solid State Commun.}}\ }\textbf {\bibinfo
  {volume} {74}},\ \bibinfo {pages} {779} (\bibinfo {year} {1990})}\BibitemShut
  {NoStop}%
\bibitem [{\citenamefont {Okuda}\ \emph {et~al.}(1992)\citenamefont {Okuda},
  \citenamefont {Noguchi}, \citenamefont {Konishi}, \citenamefont {Deguchi},\
  and\ \citenamefont {Takeda}}]{okuda1992}%
  \BibitemOpen
  \bibfield  {author} {\bibinfo {author} {\bibfnamefont {K.}~\bibnamefont
  {Okuda}}, \bibinfo {author} {\bibfnamefont {S.}~\bibnamefont {Noguchi}},
  \bibinfo {author} {\bibfnamefont {K.}~\bibnamefont {Konishi}}, \bibinfo
  {author} {\bibfnamefont {H.}~\bibnamefont {Deguchi}},\ and\ \bibinfo {author}
  {\bibfnamefont {K.}~\bibnamefont {Takeda}},\ }\href
  {https://doi.org/https://doi.org/10.1016/0304-8853(92)90374-W} {\bibfield
  {journal} {\bibinfo  {journal} {\textit{J. Magn. Magn. Mater. }}\ }\textbf
  {\bibinfo {volume} {104-107}},\ \bibinfo {pages} {817} (\bibinfo {year}
  {1992})},\ \bibinfo {note} {proceedings of the International Conference on
  Magnetism, Part II}\BibitemShut {NoStop}%
\bibitem [{\citenamefont {Mizuno}\ \emph {et~al.}(1998)\citenamefont {Mizuno},
  \citenamefont {Tohyama}, \citenamefont {Maekawa}, \citenamefont {Osafune},
  \citenamefont {Motoyama}, \citenamefont {Eisaki},\ and\ \citenamefont
  {Uchida}}]{mizuno1998}%
  \BibitemOpen
  \bibfield  {author} {\bibinfo {author} {\bibfnamefont {Y.}~\bibnamefont
  {Mizuno}}, \bibinfo {author} {\bibfnamefont {T.}~\bibnamefont {Tohyama}},
  \bibinfo {author} {\bibfnamefont {S.}~\bibnamefont {Maekawa}}, \bibinfo
  {author} {\bibfnamefont {T.}~\bibnamefont {Osafune}}, \bibinfo {author}
  {\bibfnamefont {N.}~\bibnamefont {Motoyama}}, \bibinfo {author}
  {\bibfnamefont {H.}~\bibnamefont {Eisaki}},\ and\ \bibinfo {author}
  {\bibfnamefont {S.}~\bibnamefont {Uchida}},\ }\href
  {https://doi.org/10.1103/PhysRevB.57.5326} {\bibfield  {journal} {\bibinfo
  {journal} {\textit{Phys. Rev. B}}\ }\textbf {\bibinfo {volume} {57}},\
  \bibinfo {pages} {5326} (\bibinfo {year} {1998})}\BibitemShut {NoStop}%
\bibitem [{\citenamefont {Ebisu}\ \emph {et~al.}(1998)\citenamefont {Ebisu},
  \citenamefont {Komatsu}, \citenamefont {Wada}, \citenamefont {Hashiguchi},
  \citenamefont {Kichambare},\ and\ \citenamefont {Nagata}}]{ebisu1998}%
  \BibitemOpen
  \bibfield  {author} {\bibinfo {author} {\bibfnamefont {S.}~\bibnamefont
  {Ebisu}}, \bibinfo {author} {\bibfnamefont {T.}~\bibnamefont {Komatsu}},
  \bibinfo {author} {\bibfnamefont {N.}~\bibnamefont {Wada}}, \bibinfo {author}
  {\bibfnamefont {T.}~\bibnamefont {Hashiguchi}}, \bibinfo {author}
  {\bibfnamefont {P.}~\bibnamefont {Kichambare}},\ and\ \bibinfo {author}
  {\bibfnamefont {S.}~\bibnamefont {Nagata}},\ }\href
  {https://doi.org/https://doi.org/10.1016/S0022-3697(98)00238-8} {\bibfield
  {journal} {\bibinfo  {journal} {\textit{J. Phys. Chem. Solids}}\ }\textbf
  {\bibinfo {volume} {59}},\ \bibinfo {pages} {1407} (\bibinfo {year}
  {1998})}\BibitemShut {NoStop}%
\bibitem [{\citenamefont {Yushankhai}\ and\ \citenamefont
  {Hayn}(1999)}]{yushankhai1999}%
  \BibitemOpen
  \bibfield  {author} {\bibinfo {author} {\bibfnamefont {V.~Y.}\ \bibnamefont
  {Yushankhai}}\ and\ \bibinfo {author} {\bibfnamefont {R.}~\bibnamefont
  {Hayn}},\ }\href {https://doi.org/10.1209/epl/i1999-00360-9} {\bibfield
  {journal} {\bibinfo  {journal} {\textit{Europhys. Lett.}}\ }\textbf {\bibinfo
  {volume} {47}},\ \bibinfo {pages} {116} (\bibinfo {year} {1999})}\BibitemShut
  {NoStop}%
\bibitem [{\citenamefont {Weht}\ and\ \citenamefont
  {Pickett}(1998)}]{weht1998}%
  \BibitemOpen
  \bibfield  {author} {\bibinfo {author} {\bibfnamefont {R.}~\bibnamefont
  {Weht}}\ and\ \bibinfo {author} {\bibfnamefont {W.~E.}\ \bibnamefont
  {Pickett}},\ }\href {https://doi.org/10.1103/PhysRevLett.81.2502} {\bibfield
  {journal} {\bibinfo  {journal} {\textit{Phys. Rev. Lett.}}\ }\textbf
  {\bibinfo {volume} {81}},\ \bibinfo {pages} {2502} (\bibinfo {year}
  {1998})}\BibitemShut {NoStop}%
\bibitem [{\citenamefont {Xiang}\ \emph {et~al.}(2007)\citenamefont {Xiang},
  \citenamefont {Lee},\ and\ \citenamefont {Whangbo}}]{xiang2007}%
  \BibitemOpen
  \bibfield  {author} {\bibinfo {author} {\bibfnamefont {H.~J.}\ \bibnamefont
  {Xiang}}, \bibinfo {author} {\bibfnamefont {C.}~\bibnamefont {Lee}},\ and\
  \bibinfo {author} {\bibfnamefont {M.~H.}\ \bibnamefont {Whangbo}},\ }\href
  {https://doi.org/10.1103/PhysRevB.76.220411} {\bibfield  {journal} {\bibinfo
  {journal} {\textit{Phys. Rev. B}}\ }\textbf {\bibinfo {volume} {76}},\
  \bibinfo {pages} {220411(R)} (\bibinfo {year} {2007})}\BibitemShut {NoStop}%
\bibitem [{\citenamefont {Boehm}\ \emph {et~al.}(1998)\citenamefont {Boehm},
  \citenamefont {Coad}, \citenamefont {Roessli}, \citenamefont {Zheludev},
  \citenamefont {Zolliker}, \citenamefont {Böni}, \citenamefont {Paul},
  \citenamefont {Eisaki}, \citenamefont {Motoyama},\ and\ \citenamefont
  {Uchida}}]{boehm1998}%
  \BibitemOpen
  \bibfield  {author} {\bibinfo {author} {\bibfnamefont {M.}~\bibnamefont
  {Boehm}}, \bibinfo {author} {\bibfnamefont {S.}~\bibnamefont {Coad}},
  \bibinfo {author} {\bibfnamefont {B.}~\bibnamefont {Roessli}}, \bibinfo
  {author} {\bibfnamefont {A.}~\bibnamefont {Zheludev}}, \bibinfo {author}
  {\bibfnamefont {M.}~\bibnamefont {Zolliker}}, \bibinfo {author}
  {\bibfnamefont {P.}~\bibnamefont {Böni}}, \bibinfo {author} {\bibfnamefont
  {D.~M.}\ \bibnamefont {Paul}}, \bibinfo {author} {\bibfnamefont
  {H.}~\bibnamefont {Eisaki}}, \bibinfo {author} {\bibfnamefont
  {N.}~\bibnamefont {Motoyama}},\ and\ \bibinfo {author} {\bibfnamefont
  {S.}~\bibnamefont {Uchida}},\ }\href
  {https://doi.org/10.1209/epl/i1998-00322-3} {\bibfield  {journal} {\bibinfo
  {journal} {\textit{Europhys. Lett.}}\ }\textbf {\bibinfo {volume} {43}},\
  \bibinfo {pages} {77} (\bibinfo {year} {1998})}\BibitemShut {NoStop}%
\bibitem [{\citenamefont {Lorenz}\ \emph {et~al.}(2009)\citenamefont {Lorenz},
  \citenamefont {Kuzian}, \citenamefont {Drechsler}, \citenamefont {Stein},
  \citenamefont {Wizent}, \citenamefont {Behr}, \citenamefont {M{\'{a}}lek},
  \citenamefont {Nitzsche}, \citenamefont {Rosner}, \citenamefont {Hiess} \emph
  {et~al.}}]{lorenz2009}%
  \BibitemOpen
  \bibfield  {author} {\bibinfo {author} {\bibfnamefont {W.~E.~A.}\
  \bibnamefont {Lorenz}}, \bibinfo {author} {\bibfnamefont {R.~O.}\
  \bibnamefont {Kuzian}}, \bibinfo {author} {\bibfnamefont {S.-L.}\
  \bibnamefont {Drechsler}}, \bibinfo {author} {\bibfnamefont {W.-D.}\
  \bibnamefont {Stein}}, \bibinfo {author} {\bibfnamefont {N.}~\bibnamefont
  {Wizent}}, \bibinfo {author} {\bibfnamefont {G.}~\bibnamefont {Behr}},
  \bibinfo {author} {\bibfnamefont {J.}~\bibnamefont {M{\'{a}}lek}}, \bibinfo
  {author} {\bibfnamefont {U.}~\bibnamefont {Nitzsche}}, \bibinfo {author}
  {\bibfnamefont {H.}~\bibnamefont {Rosner}}, \bibinfo {author} {\bibfnamefont
  {A.}~\bibnamefont {Hiess}}, \emph {et~al.},\ }\href
  {https://doi.org/10.1209/0295-5075/88/37002} {\bibfield  {journal} {\bibinfo
  {journal} {\textit{Europhys. Lett.}}\ }\textbf {\bibinfo {volume} {88}},\
  \bibinfo {pages} {37002} (\bibinfo {year} {2009})}\BibitemShut {NoStop}%
\bibitem [{\citenamefont {Giri}\ \emph {et~al.}(2001)\citenamefont {Giri},
  \citenamefont {Chudo}, \citenamefont {Nakamura},\ and\ \citenamefont
  {Shiga}}]{giri2001}%
  \BibitemOpen
  \bibfield  {author} {\bibinfo {author} {\bibfnamefont {S.}~\bibnamefont
  {Giri}}, \bibinfo {author} {\bibfnamefont {H.}~\bibnamefont {Chudo}},
  \bibinfo {author} {\bibfnamefont {H.}~\bibnamefont {Nakamura}},\ and\
  \bibinfo {author} {\bibfnamefont {M.}~\bibnamefont {Shiga}},\ }\href
  {https://doi.org/https://doi.org/10.1016/S0925-8388(01)01232-4} {\bibfield
  {journal} {\bibinfo  {journal} {\textit{J. Alloys Compd.}}\ }\textbf
  {\bibinfo {volume} {326}},\ \bibinfo {pages} {61} (\bibinfo {year}
  {2001})}\BibitemShut {NoStop}%
\bibitem [{\citenamefont {Enderle}\ \emph {et~al.}(2005)\citenamefont
  {Enderle}, \citenamefont {Mukherjee}, \citenamefont {F{\aa}k}, \citenamefont
  {Kremer}, \citenamefont {Broto}, \citenamefont {Rosner}, \citenamefont
  {Drechsler}, \citenamefont {Richter}, \citenamefont {Malek}, \citenamefont
  {Prokofiev} \emph {et~al.}}]{enderle2005}%
  \BibitemOpen
  \bibfield  {author} {\bibinfo {author} {\bibfnamefont {M.}~\bibnamefont
  {Enderle}}, \bibinfo {author} {\bibfnamefont {C.}~\bibnamefont {Mukherjee}},
  \bibinfo {author} {\bibfnamefont {B.}~\bibnamefont {F{\aa}k}}, \bibinfo
  {author} {\bibfnamefont {R.~K.}\ \bibnamefont {Kremer}}, \bibinfo {author}
  {\bibfnamefont {J.-M.}\ \bibnamefont {Broto}}, \bibinfo {author}
  {\bibfnamefont {H.}~\bibnamefont {Rosner}}, \bibinfo {author} {\bibfnamefont
  {S.-L.}\ \bibnamefont {Drechsler}}, \bibinfo {author} {\bibfnamefont
  {J.}~\bibnamefont {Richter}}, \bibinfo {author} {\bibfnamefont
  {J.}~\bibnamefont {Malek}}, \bibinfo {author} {\bibfnamefont
  {A.}~\bibnamefont {Prokofiev}}, \emph {et~al.},\ }\href
  {https://doi.org/10.1209/epl/i2004-10484-x} {\bibfield  {journal} {\bibinfo
  {journal} {\textit{Europhys. Lett.}}\ }\textbf {\bibinfo {volume} {70}},\
  \bibinfo {pages} {237} (\bibinfo {year} {2005})}\BibitemShut {NoStop}%
\bibitem [{\citenamefont {Wolter}\ \emph {et~al.}(2012)\citenamefont {Wolter},
  \citenamefont {Lipps}, \citenamefont {Sch\"apers}, \citenamefont {Drechsler},
  \citenamefont {Nishimoto}, \citenamefont {Vogel}, \citenamefont {Kataev},
  \citenamefont {B\"uchner}, \citenamefont {Rosner}, \citenamefont {Schmitt}
  \emph {et~al.}}]{wolter2012}%
  \BibitemOpen
  \bibfield  {author} {\bibinfo {author} {\bibfnamefont {A.~U.~B.}\
  \bibnamefont {Wolter}}, \bibinfo {author} {\bibfnamefont {F.}~\bibnamefont
  {Lipps}}, \bibinfo {author} {\bibfnamefont {M.}~\bibnamefont {Sch\"apers}},
  \bibinfo {author} {\bibfnamefont {S.-L.}\ \bibnamefont {Drechsler}}, \bibinfo
  {author} {\bibfnamefont {S.}~\bibnamefont {Nishimoto}}, \bibinfo {author}
  {\bibfnamefont {R.}~\bibnamefont {Vogel}}, \bibinfo {author} {\bibfnamefont
  {V.}~\bibnamefont {Kataev}}, \bibinfo {author} {\bibfnamefont
  {B.}~\bibnamefont {B\"uchner}}, \bibinfo {author} {\bibfnamefont
  {H.}~\bibnamefont {Rosner}}, \bibinfo {author} {\bibfnamefont
  {M.}~\bibnamefont {Schmitt}}, \emph {et~al.},\ }\href
  {https://doi.org/10.1103/PhysRevB.85.014407} {\bibfield  {journal} {\bibinfo
  {journal} {\textit{Phys. Rev. B}}\ }\textbf {\bibinfo {volume} {85}},\
  \bibinfo {pages} {014407} (\bibinfo {year} {2012})}\BibitemShut {NoStop}%
\bibitem [{\citenamefont {Rule}\ \emph {et~al.}(2017)\citenamefont {Rule},
  \citenamefont {Willenberg}, \citenamefont {Sch\"apers}, \citenamefont
  {Wolter}, \citenamefont {B\"uchner}, \citenamefont {Drechsler}, \citenamefont
  {Ehlers}, \citenamefont {Tennant}, \citenamefont {Mole}, \citenamefont
  {Gardner}, \citenamefont {Sullow},\ and\ \citenamefont
  {Nishimoto}}]{rule2017}%
  \BibitemOpen
  \bibfield  {author} {\bibinfo {author} {\bibfnamefont {K.~C.}\ \bibnamefont
  {Rule}}, \bibinfo {author} {\bibfnamefont {B.}~\bibnamefont {Willenberg}},
  \bibinfo {author} {\bibfnamefont {M.}~\bibnamefont {Sch\"apers}}, \bibinfo
  {author} {\bibfnamefont {A.~U.~B.}\ \bibnamefont {Wolter}}, \bibinfo {author}
  {\bibfnamefont {B.}~\bibnamefont {B\"uchner}}, \bibinfo {author}
  {\bibfnamefont {S.-L.}\ \bibnamefont {Drechsler}}, \bibinfo {author}
  {\bibfnamefont {G.}~\bibnamefont {Ehlers}}, \bibinfo {author} {\bibfnamefont
  {D.~A.}\ \bibnamefont {Tennant}}, \bibinfo {author} {\bibfnamefont {R.~A.}\
  \bibnamefont {Mole}}, \bibinfo {author} {\bibfnamefont {J.~S.}\ \bibnamefont
  {Gardner}}, \bibinfo {author} {\bibfnamefont {S.}~\bibnamefont {Sullow}},\
  and\ \bibinfo {author} {\bibfnamefont {S.}~\bibnamefont {Nishimoto}},\ }\href
  {https://doi.org/10.1103/PhysRevB.95.024430} {\bibfield  {journal} {\bibinfo
  {journal} {\textit{Phys. Rev. B}}\ }\textbf {\bibinfo {volume} {95}},\
  \bibinfo {pages} {024430} (\bibinfo {year} {2017})}\BibitemShut {NoStop}%
\bibitem [{\citenamefont {Hikihara}\ \emph {et~al.}(2008)\citenamefont
  {Hikihara}, \citenamefont {Kecke}, \citenamefont {Momoi},\ and\ \citenamefont
  {Furusaki}}]{hikihara2008}%
  \BibitemOpen
  \bibfield  {author} {\bibinfo {author} {\bibfnamefont {T.}~\bibnamefont
  {Hikihara}}, \bibinfo {author} {\bibfnamefont {L.}~\bibnamefont {Kecke}},
  \bibinfo {author} {\bibfnamefont {T.}~\bibnamefont {Momoi}},\ and\ \bibinfo
  {author} {\bibfnamefont {A.}~\bibnamefont {Furusaki}},\ }\href
  {https://doi.org/10.1103/PhysRevB.78.144404} {\bibfield  {journal} {\bibinfo
  {journal} {\textit{Phys. Rev. B}}\ }\textbf {\bibinfo {volume} {78}},\
  \bibinfo {pages} {144404} (\bibinfo {year} {2008})}\BibitemShut {NoStop}%
\bibitem [{\citenamefont {Sudan}\ \emph {et~al.}(2009)\citenamefont {Sudan},
  \citenamefont {L\"uscher},\ and\ \citenamefont {L\"auchli}}]{sudan2009}%
  \BibitemOpen
  \bibfield  {author} {\bibinfo {author} {\bibfnamefont {J.}~\bibnamefont
  {Sudan}}, \bibinfo {author} {\bibfnamefont {A.}~\bibnamefont {L\"uscher}},\
  and\ \bibinfo {author} {\bibfnamefont {A.~M.}\ \bibnamefont {L\"auchli}},\
  }\href {https://doi.org/10.1103/PhysRevB.80.140402} {\bibfield  {journal}
  {\bibinfo  {journal} {\textit{Phys. Rev. B}}\ }\textbf {\bibinfo {volume}
  {80}},\ \bibinfo {pages} {140402(R)} (\bibinfo {year} {2009})}\BibitemShut
  {NoStop}%
\bibitem [{\citenamefont {Bursill}\ \emph {et~al.}(1995)\citenamefont
  {Bursill}, \citenamefont {Gehring}, \citenamefont {Farnell}, \citenamefont
  {Parkinson}, \citenamefont {Xiang},\ and\ \citenamefont
  {Zeng}}]{bursill1995}%
  \BibitemOpen
  \bibfield  {author} {\bibinfo {author} {\bibfnamefont {R.}~\bibnamefont
  {Bursill}}, \bibinfo {author} {\bibfnamefont {G.~A.}\ \bibnamefont
  {Gehring}}, \bibinfo {author} {\bibfnamefont {D.~J.~J.}\ \bibnamefont
  {Farnell}}, \bibinfo {author} {\bibfnamefont {J.~B.}\ \bibnamefont
  {Parkinson}}, \bibinfo {author} {\bibfnamefont {T.}~\bibnamefont {Xiang}},\
  and\ \bibinfo {author} {\bibfnamefont {C.}~\bibnamefont {Zeng}},\ }\href
  {https://doi.org/10.1088/0953-8984/7/45/016} {\bibfield  {journal} {\bibinfo
  {journal} {\textit{J. Phys.: Cond. Matt.}}\ }\textbf {\bibinfo {volume}
  {7}},\ \bibinfo {pages} {8605} (\bibinfo {year} {1995})}\BibitemShut
  {NoStop}%
\bibitem [{\citenamefont {Bader}\ and\ \citenamefont
  {Schilling}(1979)}]{bader1979}%
  \BibitemOpen
  \bibfield  {author} {\bibinfo {author} {\bibfnamefont {H.~P.}\ \bibnamefont
  {Bader}}\ and\ \bibinfo {author} {\bibfnamefont {R.}~\bibnamefont
  {Schilling}},\ }\href {https://doi.org/10.1103/PhysRevB.19.3556} {\bibfield
  {journal} {\bibinfo  {journal} {\textit{Phys. Rev. B}}\ }\textbf {\bibinfo
  {volume} {19}},\ \bibinfo {pages} {3556} (\bibinfo {year}
  {1979})}\BibitemShut {NoStop}%
\bibitem [{\citenamefont {Dmitriev}\ and\ \citenamefont
  {Krivnov}(2008)}]{dimitriev2008}%
  \BibitemOpen
  \bibfield  {author} {\bibinfo {author} {\bibfnamefont {D.~V.}\ \bibnamefont
  {Dmitriev}}\ and\ \bibinfo {author} {\bibfnamefont {V.~Y.}\ \bibnamefont
  {Krivnov}},\ }\href {https://doi.org/10.1103/PhysRevB.77.024401} {\bibfield
  {journal} {\bibinfo  {journal} {\textit{Phys. Rev. B}}\ }\textbf {\bibinfo
  {volume} {77}},\ \bibinfo {pages} {024401} (\bibinfo {year}
  {2008})}\BibitemShut {NoStop}%
\bibitem [{\citenamefont {Nishimoto}\ \emph {et~al.}(2015)\citenamefont
  {Nishimoto}, \citenamefont {Drechsler}, \citenamefont {Kuzian}, \citenamefont
  {Richter},\ and\ \citenamefont {van~den Brink}}]{nishimoto2015}%
  \BibitemOpen
  \bibfield  {author} {\bibinfo {author} {\bibfnamefont {S.}~\bibnamefont
  {Nishimoto}}, \bibinfo {author} {\bibfnamefont {S.-L.}\ \bibnamefont
  {Drechsler}}, \bibinfo {author} {\bibfnamefont {R.}~\bibnamefont {Kuzian}},
  \bibinfo {author} {\bibfnamefont {J.}~\bibnamefont {Richter}},\ and\ \bibinfo
  {author} {\bibfnamefont {J.}~\bibnamefont {van~den Brink}},\ }\href
  {https://doi.org/10.1103/PhysRevB.92.214415} {\bibfield  {journal} {\bibinfo
  {journal} {\textit{Phys. Rev. B}}\ }\textbf {\bibinfo {volume} {92}},\
  \bibinfo {pages} {214415} (\bibinfo {year} {2015})}\BibitemShut {NoStop}%
\bibitem [{\citenamefont {Chubukov}(1991)}]{chubukov1991}%
  \BibitemOpen
  \bibfield  {author} {\bibinfo {author} {\bibfnamefont {A.~V.}\ \bibnamefont
  {Chubukov}},\ }\href {https://doi.org/10.1103/PhysRevB.44.4693} {\bibfield
  {journal} {\bibinfo  {journal} {\textit{Phys. Rev. B}}\ }\textbf {\bibinfo
  {volume} {44}},\ \bibinfo {pages} {4693} (\bibinfo {year}
  {1991})}\BibitemShut {NoStop}%
\bibitem [{\citenamefont {Syromyatnikov}(2012)}]{syromyatnikov2012}%
  \BibitemOpen
  \bibfield  {author} {\bibinfo {author} {\bibfnamefont {A.~V.}\ \bibnamefont
  {Syromyatnikov}},\ }\href {https://doi.org/10.1103/PhysRevB.86.014423}
  {\bibfield  {journal} {\bibinfo  {journal} {\textit{Phys. Rev. B}}\ }\textbf
  {\bibinfo {volume} {86}},\ \bibinfo {pages} {014423} (\bibinfo {year}
  {2012})}\BibitemShut {NoStop}%
\bibitem [{\citenamefont {Dmitriev}\ and\ \citenamefont
  {Krivnov}(2009{\natexlab{a}})}]{dimitriev2009a}%
  \BibitemOpen
  \bibfield  {author} {\bibinfo {author} {\bibfnamefont {D.}~\bibnamefont
  {Dmitriev}}\ and\ \bibinfo {author} {\bibfnamefont {V.~Y.}\ \bibnamefont
  {Krivnov}},\ }\href {https://doi.org/10.1134/S1990793109020171} {\bibfield
  {journal} {\bibinfo  {journal} {\textit{Russ. J. Phys. Chem. B}}\ }\textbf
  {\bibinfo {volume} {3}},\ \bibinfo {pages} {280} (\bibinfo {year}
  {2009}{\natexlab{a}})}\BibitemShut {NoStop}%
\bibitem [{\citenamefont {Bethe}(1931)}]{bethe1931}%
  \BibitemOpen
  \bibfield  {author} {\bibinfo {author} {\bibfnamefont {H.}~\bibnamefont
  {Bethe}},\ }\href {https://doi.org/10.1007/BF01341708} {\bibfield  {journal}
  {\bibinfo  {journal} {\textit{Z. Phys.}}\ }\textbf {\bibinfo {volume} {71}},\
  \bibinfo {pages} {205} (\bibinfo {year} {1931})}\BibitemShut {NoStop}%
\bibitem [{\citenamefont {Dmitriev}\ and\ \citenamefont
  {Krivnov}(2009{\natexlab{b}})}]{dimitriev2009b}%
  \BibitemOpen
  \bibfield  {author} {\bibinfo {author} {\bibfnamefont {D.~V.}\ \bibnamefont
  {Dmitriev}}\ and\ \bibinfo {author} {\bibfnamefont {V.~Y.}\ \bibnamefont
  {Krivnov}},\ }\href {https://doi.org/10.1103/PhysRevB.79.054421} {\bibfield
  {journal} {\bibinfo  {journal} {\textit{Phys. Rev. B}}\ }\textbf {\bibinfo
  {volume} {79}},\ \bibinfo {pages} {054421} (\bibinfo {year}
  {2009}{\natexlab{b}})}\BibitemShut {NoStop}%
\bibitem [{\citenamefont {Haldane}(1982)}]{haldane1982}%
  \BibitemOpen
  \bibfield  {author} {\bibinfo {author} {\bibfnamefont {F.~D.~M.}\
  \bibnamefont {Haldane}},\ }\href
  {https://doi.org/10.1088/0022-3719/15/36/008} {\bibfield  {journal} {\bibinfo
   {journal} {\textit{J. Phys. C: Solid State Phys.}}\ }\textbf {\bibinfo
  {volume} {15}},\ \bibinfo {pages} {L1309} (\bibinfo {year}
  {1982})}\BibitemShut {NoStop}%
\bibitem [{\citenamefont {Schneider}(1981)}]{schneider1981}%
  \BibitemOpen
  \bibfield  {author} {\bibinfo {author} {\bibfnamefont {T.}~\bibnamefont
  {Schneider}},\ }\href {https://doi.org/10.1103/PhysRevB.24.5327} {\bibfield
  {journal} {\bibinfo  {journal} {\textit{Phys. Rev. B}}\ }\textbf {\bibinfo
  {volume} {24}},\ \bibinfo {pages} {5327} (\bibinfo {year}
  {1981})}\BibitemShut {NoStop}%
\bibitem [{\citenamefont {Southern}\ \emph {et~al.}(1994)\citenamefont
  {Southern}, \citenamefont {Lee},\ and\ \citenamefont {Lavis}}]{southern1994}%
  \BibitemOpen
  \bibfield  {author} {\bibinfo {author} {\bibfnamefont {B.~W.}\ \bibnamefont
  {Southern}}, \bibinfo {author} {\bibfnamefont {R.~J.}\ \bibnamefont {Lee}},\
  and\ \bibinfo {author} {\bibfnamefont {D.~A.}\ \bibnamefont {Lavis}},\ }\href
  {https://doi.org/10.1088/0953-8984/6/46/024} {\bibfield  {journal} {\bibinfo
  {journal} {\textit{J. Phys.: Condens. Matter}}\ }\textbf {\bibinfo {volume}
  {6}},\ \bibinfo {pages} {10075} (\bibinfo {year} {1994})}\BibitemShut
  {NoStop}%
\bibitem [{\citenamefont {Huang}\ \emph {et~al.}(1990)\citenamefont {Huang},
  \citenamefont {Shi}, \citenamefont {Dai},\ and\ \citenamefont
  {Tao}}]{huang1990}%
  \BibitemOpen
  \bibfield  {author} {\bibinfo {author} {\bibfnamefont {G.}~\bibnamefont
  {Huang}}, \bibinfo {author} {\bibfnamefont {Z.-P.}\ \bibnamefont {Shi}},
  \bibinfo {author} {\bibfnamefont {X.}~\bibnamefont {Dai}},\ and\ \bibinfo
  {author} {\bibfnamefont {R.}~\bibnamefont {Tao}},\ }\href
  {https://doi.org/10.1088/0953-8984/2/50/011} {\bibfield  {journal} {\bibinfo
  {journal} {\textit{J. Phys.: Condens. Matter}}\ }\textbf {\bibinfo {volume}
  {2}},\ \bibinfo {pages} {10059} (\bibinfo {year} {1990})}\BibitemShut
  {NoStop}%
\bibitem [{\citenamefont {Hood}(1984)}]{hood1984}%
  \BibitemOpen
  \bibfield  {author} {\bibinfo {author} {\bibfnamefont {M.}~\bibnamefont
  {Hood}},\ }\href {https://doi.org/10.1088/0022-3719/17/8/011} {\bibfield
  {journal} {\bibinfo  {journal} {\textit{J. Phys. C: Solid State Phys.}}\
  }\textbf {\bibinfo {volume} {17}},\ \bibinfo {pages} {1367} (\bibinfo {year}
  {1984})}\BibitemShut {NoStop}%
\bibitem [{\citenamefont {Kecke}\ \emph {et~al.}(2007)\citenamefont {Kecke},
  \citenamefont {Momoi},\ and\ \citenamefont {Furusaki}}]{kecke2007}%
  \BibitemOpen
  \bibfield  {author} {\bibinfo {author} {\bibfnamefont {L.}~\bibnamefont
  {Kecke}}, \bibinfo {author} {\bibfnamefont {T.}~\bibnamefont {Momoi}},\ and\
  \bibinfo {author} {\bibfnamefont {A.}~\bibnamefont {Furusaki}},\ }\href
  {https://doi.org/10.1103/PhysRevB.76.060407} {\bibfield  {journal} {\bibinfo
  {journal} {\textit{Phys. Rev. B}}\ }\textbf {\bibinfo {volume} {76}},\
  \bibinfo {pages} {060407(R)} (\bibinfo {year} {2007})}\BibitemShut {NoStop}%
\bibitem [{\citenamefont {Onishi}(2015)}]{onishi2015}%
  \BibitemOpen
  \bibfield  {author} {\bibinfo {author} {\bibfnamefont {H.}~\bibnamefont
  {Onishi}},\ }\href {https://doi.org/10.1088/1742-6596/592/1/012109}
  {\bibfield  {journal} {\bibinfo  {journal} {\textit{J. Phys.: Conf. Ser.}}\
  }\textbf {\bibinfo {volume} {592}},\ \bibinfo {pages} {012109} (\bibinfo
  {year} {2015})}\BibitemShut {NoStop}%
\bibitem [{\citenamefont {Fert}\ \emph {et~al.}(1978)\citenamefont {Fert},
  \citenamefont {Bertrand}, \citenamefont {Leotin}, \citenamefont {Ousset},
  \citenamefont {Magarino},\ and\ \citenamefont {Tuchendler}}]{Fert1978}%
  \BibitemOpen
  \bibfield  {author} {\bibinfo {author} {\bibfnamefont {A.}~\bibnamefont
  {Fert}}, \bibinfo {author} {\bibfnamefont {D.}~\bibnamefont {Bertrand}},
  \bibinfo {author} {\bibfnamefont {J.}~\bibnamefont {Leotin}}, \bibinfo
  {author} {\bibfnamefont {J.}~\bibnamefont {Ousset}}, \bibinfo {author}
  {\bibfnamefont {J.}~\bibnamefont {Magarino}},\ and\ \bibinfo {author}
  {\bibfnamefont {J.}~\bibnamefont {Tuchendler}},\ }\href
  {https://doi.org/https://doi.org/10.1016/0038-1098(78)90721-4} {\bibfield
  {journal} {\bibinfo  {journal} {\textit{Solid State Commun.}}\ }\textbf
  {\bibinfo {volume} {26}},\ \bibinfo {pages} {693} (\bibinfo {year}
  {1978})}\BibitemShut {NoStop}%
\bibitem [{\citenamefont {Petitgrand}\ \emph {et~al.}(1979)\citenamefont
  {Petitgrand}, \citenamefont {Hennion},\ and\ \citenamefont
  {Escribe}}]{petitgrand1979}%
  \BibitemOpen
  \bibfield  {author} {\bibinfo {author} {\bibfnamefont {D.}~\bibnamefont
  {Petitgrand}}, \bibinfo {author} {\bibfnamefont {B.}~\bibnamefont
  {Hennion}},\ and\ \bibinfo {author} {\bibfnamefont {C.}~\bibnamefont
  {Escribe}},\ }\href
  {https://doi.org/https://doi.org/10.1016/0304-8853(79)90138-0} {\bibfield
  {journal} {\bibinfo  {journal} {\textit{J. Magn. Magn. Mater.}}\ }\textbf
  {\bibinfo {volume} {14}},\ \bibinfo {pages} {275} (\bibinfo {year}
  {1979})}\BibitemShut {NoStop}%
\bibitem [{\citenamefont {Chauhan}\ \emph {et~al.}(2020)\citenamefont
  {Chauhan}, \citenamefont {Mahmood}, \citenamefont {Changlani}, \citenamefont
  {Koohpayeh},\ and\ \citenamefont {Armitage}}]{chauhan2020}%
  \BibitemOpen
  \bibfield  {author} {\bibinfo {author} {\bibfnamefont {P.}~\bibnamefont
  {Chauhan}}, \bibinfo {author} {\bibfnamefont {F.}~\bibnamefont {Mahmood}},
  \bibinfo {author} {\bibfnamefont {H.~J.}\ \bibnamefont {Changlani}}, \bibinfo
  {author} {\bibfnamefont {S.~M.}\ \bibnamefont {Koohpayeh}},\ and\ \bibinfo
  {author} {\bibfnamefont {N.~P.}\ \bibnamefont {Armitage}},\ }\href
  {https://doi.org/https://doi.org/10.1103/PhysRevLett.124.037203} {\bibfield
  {journal} {\bibinfo  {journal} {\textit{Phys. Rev. Lett.}}\ }\textbf
  {\bibinfo {volume} {124}},\ \bibinfo {pages} {037203} (\bibinfo {year}
  {2020})}\BibitemShut {NoStop}%
\bibitem [{Note1()}]{Note1}%
  \BibitemOpen
  \bibinfo {note} {Note that the exchange model presented in Bai \protect
  \textit {et al.} for FeI$_{2}$ does include exchange anisotropy. However, the
  bound state in that case is fundamentally a result of single-ion anisotropy,
  which provides the energy reduction upon binding. The diagonal exchange
  anisotropy cooperates with a finite symmetric off-diagonal exchange to
  hybridize the single-ion anisotropy-stabilized bound state with the
  one-magnon mode. This explains the visibility of the bound state in INS, as
  well as the disperison, but not its stability \cite {bai2021}.}\BibitemShut
  {Stop}%
\bibitem [{\citenamefont {Schmehr}\ \emph {et~al.}(2019)\citenamefont
  {Schmehr}, \citenamefont {Aling}, \citenamefont {Zoghlin},\ and\
  \citenamefont {Wilson}}]{schmehr2019}%
  \BibitemOpen
  \bibfield  {author} {\bibinfo {author} {\bibfnamefont {J.~L.}\ \bibnamefont
  {Schmehr}}, \bibinfo {author} {\bibfnamefont {M.}~\bibnamefont {Aling}},
  \bibinfo {author} {\bibfnamefont {E.}~\bibnamefont {Zoghlin}},\ and\ \bibinfo
  {author} {\bibfnamefont {S.~D.}\ \bibnamefont {Wilson}},\ }\href
  {https://doi.org/10.1063/1.5085327} {\bibfield  {journal} {\bibinfo
  {journal} {\textit{Rev. Sci. Instrum.}}\ }\textbf {\bibinfo {volume} {90}},\
  \bibinfo {pages} {043906} (\bibinfo {year} {2019})}\BibitemShut {NoStop}%
\bibitem [{\citenamefont {Wizent}\ \emph {et~al.}(2011)\citenamefont {Wizent},
  \citenamefont {Behr}, \citenamefont {Löser}, \citenamefont {Büchner},\ and\
  \citenamefont {Klingeler}}]{wizent2011}%
  \BibitemOpen
  \bibfield  {author} {\bibinfo {author} {\bibfnamefont {N.}~\bibnamefont
  {Wizent}}, \bibinfo {author} {\bibfnamefont {G.}~\bibnamefont {Behr}},
  \bibinfo {author} {\bibfnamefont {W.}~\bibnamefont {Löser}}, \bibinfo
  {author} {\bibfnamefont {B.}~\bibnamefont {Büchner}},\ and\ \bibinfo
  {author} {\bibfnamefont {R.}~\bibnamefont {Klingeler}},\ }\href
  {https://doi.org/https://doi.org/10.1016/j.jcrysgro.2010.10.029} {\bibfield
  {journal} {\bibinfo  {journal} {\textit{J. Cryst. Growth}}\ }\textbf
  {\bibinfo {volume} {318}},\ \bibinfo {pages} {995} (\bibinfo {year}
  {2011})}\BibitemShut {NoStop}%
\bibitem [{\citenamefont {Shu}\ \emph {et~al.}(2018)\citenamefont {Shu},
  \citenamefont {Tian}, \citenamefont {Lin}, \citenamefont {Hayashi},
  \citenamefont {Liou}, \citenamefont {Chen}, \citenamefont {Wong},
  \citenamefont {Liou},\ and\ \citenamefont {Chou}}]{shu2018}%
  \BibitemOpen
  \bibfield  {author} {\bibinfo {author} {\bibfnamefont {G.~J.}\ \bibnamefont
  {Shu}}, \bibinfo {author} {\bibfnamefont {J.~C.}\ \bibnamefont {Tian}},
  \bibinfo {author} {\bibfnamefont {C.~K.}\ \bibnamefont {Lin}}, \bibinfo
  {author} {\bibfnamefont {M.}~\bibnamefont {Hayashi}}, \bibinfo {author}
  {\bibfnamefont {S.~C.}\ \bibnamefont {Liou}}, \bibinfo {author}
  {\bibfnamefont {W.~T.}\ \bibnamefont {Chen}}, \bibinfo {author}
  {\bibfnamefont {D.~P.}\ \bibnamefont {Wong}}, \bibinfo {author}
  {\bibfnamefont {H.~L.}\ \bibnamefont {Liou}},\ and\ \bibinfo {author}
  {\bibfnamefont {F.~C.}\ \bibnamefont {Chou}},\ }\href
  {https://doi.org/10.1088/1367-2630/aac0a9} {\bibfield  {journal} {\bibinfo
  {journal} {\textit{New J. Phys.},}\ }\textbf {\bibinfo {volume} {20}},\
  \bibinfo {pages} {059501} (\bibinfo {year} {2018})}\BibitemShut {NoStop}%
\bibitem [{\citenamefont {Chung}\ \emph {et~al.}(2003)\citenamefont {Chung},
  \citenamefont {McIntyre}, \citenamefont {Paul}, \citenamefont
  {Balakrishnan},\ and\ \citenamefont {Lees}}]{chung2003}%
  \BibitemOpen
  \bibfield  {author} {\bibinfo {author} {\bibfnamefont {E.~M.~L.}\
  \bibnamefont {Chung}}, \bibinfo {author} {\bibfnamefont {G.~J.}\ \bibnamefont
  {McIntyre}}, \bibinfo {author} {\bibfnamefont {D.~M.}\ \bibnamefont {Paul}},
  \bibinfo {author} {\bibfnamefont {G.}~\bibnamefont {Balakrishnan}},\ and\
  \bibinfo {author} {\bibfnamefont {M.~R.}\ \bibnamefont {Lees}},\ }\href
  {https://doi.org/https://doi.org/10.1103/PhysRevB.68.144410} {\bibfield
  {journal} {\bibinfo  {journal} {\textit{Phys. Rev. B}}\ }\textbf {\bibinfo
  {volume} {68}},\ \bibinfo {pages} {144410} (\bibinfo {year}
  {2003})}\BibitemShut {NoStop}%
\bibitem [{\citenamefont {Kawamata}\ \emph {et~al.}(2004)\citenamefont
  {Kawamata}, \citenamefont {Okuda},\ and\ \citenamefont
  {Kindo}}]{kawamata2004}%
  \BibitemOpen
  \bibfield  {author} {\bibinfo {author} {\bibfnamefont {S.}~\bibnamefont
  {Kawamata}}, \bibinfo {author} {\bibfnamefont {K.}~\bibnamefont {Okuda}},\
  and\ \bibinfo {author} {\bibfnamefont {K.}~\bibnamefont {Kindo}},\ }\href
  {https://doi.org/https://doi.org/10.1016/j.jmmm.2003.12.579} {\bibfield
  {journal} {\bibinfo  {journal} {\textit{J. Magn. Magn. Mater. }}\ }\textbf
  {\bibinfo {volume} {272-276}},\ \bibinfo {pages} {939} (\bibinfo {year}
  {2004})}\BibitemShut {NoStop}%
\bibitem [{\citenamefont {Arnold}\ \emph {et~al.}(2014)\citenamefont {Arnold},
  \citenamefont {Bilheux}, \citenamefont {Borreguero}, \citenamefont {Buts},
  \citenamefont {Campbell}, \citenamefont {Chapon}, \citenamefont {Doucet},
  \citenamefont {Draper}, \citenamefont {{Ferraz Leal}}, \citenamefont {Gigg}
  \emph {et~al.}}]{arnold2014}%
  \BibitemOpen
  \bibfield  {author} {\bibinfo {author} {\bibfnamefont {O.}~\bibnamefont
  {Arnold}}, \bibinfo {author} {\bibfnamefont {J.}~\bibnamefont {Bilheux}},
  \bibinfo {author} {\bibfnamefont {J.}~\bibnamefont {Borreguero}}, \bibinfo
  {author} {\bibfnamefont {A.}~\bibnamefont {Buts}}, \bibinfo {author}
  {\bibfnamefont {S.}~\bibnamefont {Campbell}}, \bibinfo {author}
  {\bibfnamefont {L.}~\bibnamefont {Chapon}}, \bibinfo {author} {\bibfnamefont
  {M.}~\bibnamefont {Doucet}}, \bibinfo {author} {\bibfnamefont
  {N.}~\bibnamefont {Draper}}, \bibinfo {author} {\bibfnamefont
  {R.}~\bibnamefont {{Ferraz Leal}}}, \bibinfo {author} {\bibfnamefont
  {M.}~\bibnamefont {Gigg}}, \emph {et~al.},\ }\href
  {https://doi.org/https://doi.org/10.1016/j.nima.2014.07.029} {\bibfield
  {journal} {\bibinfo  {journal} {\textit{Nucl. Instrum. Meth. in Phys. Res.
  A}}\ }\textbf {\bibinfo {volume} {764}},\ \bibinfo {pages} {156} (\bibinfo
  {year} {2014})}\BibitemShut {NoStop}%
\bibitem [{\citenamefont {Toth}\ and\ \citenamefont {Lake}(2015)}]{toth2015}%
  \BibitemOpen
  \bibfield  {author} {\bibinfo {author} {\bibfnamefont {S.}~\bibnamefont
  {Toth}}\ and\ \bibinfo {author} {\bibfnamefont {B.}~\bibnamefont {Lake}},\
  }\href {https://doi.org/10.1088/0953-8984/27/16/166002} {\bibfield  {journal}
  {\bibinfo  {journal} {\textit{J. Phys.: Condens. Matter}}\ }\textbf {\bibinfo
  {volume} {27}},\ \bibinfo {pages} {166002} (\bibinfo {year}
  {2015})}\BibitemShut {NoStop}%
\bibitem [{\citenamefont {Ohta}\ \emph {et~al.}(1993)\citenamefont {Ohta},
  \citenamefont {Yamauchi}, \citenamefont {Nanba}, \citenamefont {Motokawa},
  \citenamefont {Kawamata},\ and\ \citenamefont {Okuda}}]{ohta1993}%
  \BibitemOpen
  \bibfield  {author} {\bibinfo {author} {\bibfnamefont {H.}~\bibnamefont
  {Ohta}}, \bibinfo {author} {\bibfnamefont {N.}~\bibnamefont {Yamauchi}},
  \bibinfo {author} {\bibfnamefont {T.}~\bibnamefont {Nanba}}, \bibinfo
  {author} {\bibfnamefont {M.}~\bibnamefont {Motokawa}}, \bibinfo {author}
  {\bibfnamefont {S.}~\bibnamefont {Kawamata}},\ and\ \bibinfo {author}
  {\bibfnamefont {K.}~\bibnamefont {Okuda}},\ }\href
  {https://doi.org/10.1143/JPSJ.62.785} {\bibfield  {journal} {\bibinfo
  {journal} {\textit{J. Phys. Soc. Japan}}\ }\textbf {\bibinfo {volume} {62}},\
  \bibinfo {pages} {785} (\bibinfo {year} {1993})}\BibitemShut {NoStop}%
\bibitem [{\citenamefont {Oguchi}(1960)}]{oguchi1960}%
  \BibitemOpen
  \bibfield  {author} {\bibinfo {author} {\bibfnamefont {T.}~\bibnamefont
  {Oguchi}},\ }\href {https://doi.org/10.1103/PhysRev.117.117} {\bibfield
  {journal} {\bibinfo  {journal} {\textit{Phys. Rev.}}\ }\textbf {\bibinfo
  {volume} {117}},\ \bibinfo {pages} {117} (\bibinfo {year}
  {1960})}\BibitemShut {NoStop}%
\bibitem [{\citenamefont {Lorenz}(2011)}]{lorenzThesis2011}%
  \BibitemOpen
  \bibfield  {author} {\bibinfo {author} {\bibfnamefont {W.~E.~A.}\
  \bibnamefont {Lorenz}},\ }\emph {\bibinfo {title} {On the spin-dynamics of
  the quasi-one-dimensional, frustrated quantum magnetic Li$_2$CuO$_2$: Studies
  by means of inelastic neutron scattering and thermodynamic methods}},\
  \href@noop {} {Ph.D. thesis},\ \bibinfo  {school} {Technische Universitat
  Dresden} (\bibinfo {year} {2011})\BibitemShut {NoStop}%
\bibitem [{\citenamefont {Huberman}\ \emph {et~al.}(2005)\citenamefont
  {Huberman}, \citenamefont {Coldea}, \citenamefont {Cowley}, \citenamefont
  {Tennant}, \citenamefont {Leheny}, \citenamefont {Christianson},\ and\
  \citenamefont {Frost}}]{huberman2005}%
  \BibitemOpen
  \bibfield  {author} {\bibinfo {author} {\bibfnamefont {T.}~\bibnamefont
  {Huberman}}, \bibinfo {author} {\bibfnamefont {R.}~\bibnamefont {Coldea}},
  \bibinfo {author} {\bibfnamefont {R.~A.}\ \bibnamefont {Cowley}}, \bibinfo
  {author} {\bibfnamefont {D.~A.}\ \bibnamefont {Tennant}}, \bibinfo {author}
  {\bibfnamefont {R.~L.}\ \bibnamefont {Leheny}}, \bibinfo {author}
  {\bibfnamefont {R.~J.}\ \bibnamefont {Christianson}},\ and\ \bibinfo {author}
  {\bibfnamefont {C.~D.}\ \bibnamefont {Frost}},\ }\href
  {https://doi.org/10.1103/PhysRevB.72.014413} {\bibfield  {journal} {\bibinfo
  {journal} {\textit{Phys. Rev. B}}\ }\textbf {\bibinfo {volume} {72}},\
  \bibinfo {pages} {014413} (\bibinfo {year} {2005})}\BibitemShut {NoStop}%
\bibitem [{\citenamefont {Barnes}(2003)}]{barnes2003}%
  \BibitemOpen
  \bibfield  {author} {\bibinfo {author} {\bibfnamefont {T.}~\bibnamefont
  {Barnes}},\ }\href {https://doi.org/10.1103/PhysRevB.67.024412} {\bibfield
  {journal} {\bibinfo  {journal} {\textit{Phys. Rev. B}}\ }\textbf {\bibinfo
  {volume} {67}},\ \bibinfo {pages} {024412} (\bibinfo {year}
  {2003})}\BibitemShut {NoStop}%
\bibitem [{\citenamefont {Nishimoto}\ \emph {et~al.}(2011)\citenamefont
  {Nishimoto}, \citenamefont {Drechsler}, \citenamefont {Kuzian}, \citenamefont
  {van~den Brink}, \citenamefont {Richter}, \citenamefont {Lorenz},
  \citenamefont {Skourski}, \citenamefont {Klingeler},\ and\ \citenamefont
  {B\"uchner}}]{nishimoto2011}%
  \BibitemOpen
  \bibfield  {author} {\bibinfo {author} {\bibfnamefont {S.}~\bibnamefont
  {Nishimoto}}, \bibinfo {author} {\bibfnamefont {S.-L.}\ \bibnamefont
  {Drechsler}}, \bibinfo {author} {\bibfnamefont {R.~O.}\ \bibnamefont
  {Kuzian}}, \bibinfo {author} {\bibfnamefont {J.}~\bibnamefont {van~den
  Brink}}, \bibinfo {author} {\bibfnamefont {J.}~\bibnamefont {Richter}},
  \bibinfo {author} {\bibfnamefont {W.~E.~A.}\ \bibnamefont {Lorenz}}, \bibinfo
  {author} {\bibfnamefont {Y.}~\bibnamefont {Skourski}}, \bibinfo {author}
  {\bibfnamefont {R.}~\bibnamefont {Klingeler}},\ and\ \bibinfo {author}
  {\bibfnamefont {B.}~\bibnamefont {B\"uchner}},\ }\href
  {https://doi.org/10.1103/PhysRevLett.107.097201} {\bibfield  {journal}
  {\bibinfo  {journal} {\textit{Phys. Rev. Lett.}}\ }\textbf {\bibinfo {volume}
  {107}},\ \bibinfo {pages} {097201} (\bibinfo {year} {2011})}\BibitemShut
  {NoStop}%
\bibitem [{\citenamefont {Nayak}\ and\ \citenamefont {Mila}(2022)}]{nayak2022}%
  \BibitemOpen
  \bibfield  {author} {\bibinfo {author} {\bibfnamefont {M.}~\bibnamefont
  {Nayak}}\ and\ \bibinfo {author} {\bibfnamefont {F.}~\bibnamefont {Mila}},\
  }\href {https://doi.org/10.1103/PhysRevB.105.094407} {\bibfield  {journal}
  {\bibinfo  {journal} {\textit{Phys. Rev. B}}\ }\textbf {\bibinfo {volume}
  {105}},\ \bibinfo {pages} {094407} (\bibinfo {year} {2022})}\BibitemShut
  {NoStop}%
\end{thebibliography}%



\end{document}
%
% ****** End of file apstemplate.tex ******

