\documentclass[./AATesi.tex]{subfiles}


%\definechangesauthor[name=Alice, color=blue]{ADT}

\begin{document}

 
If quantum theory is universal, and there currently is no reason to think otherwise, then the universe should be describable by a quantum state just like any other system. While its quantum properties might be hidden today they may well have played a crucial role in an early phase of its evolution. In this chapter, based on \cite{DiTucci:2019dji}, \cite{DiTucci:2019bui} and \cite{Bramberger:2019zks}, we continue our study of such semi-classical effects in the early universe defining path integrals with Robin type of boundary conditions (see section \ref{sec:boundaryterms}). We will study two different physical situations where these boundary conditions are useful: first, following \cite{DiTucci:2019dji} and \cite{DiTucci:2019bui}, we will return to the question of how to define a no-boundary path integral, which we have seen in the last chapter fails to be well-defined with Dirichlet type of boundary conditions; then, in section \ref{papersebastian}, we will study homogeneous quantum transitions during inflation \cite{Bramberger:2019zks}. There, we will see that a Robin boundary condition is needed to recover the correct semi-classical limit, since both the expansion rate and the size of the universe need to be specified to a certain degree. Part of the discussion will be reminiscent of section \ref{InitialConditions} of \ref{quantuminitial} where a similar type of Robin boundary conditions was implemented to the define suitable initial conditions for inflation.




%%%%%%%%%%%%%%%%%%%%%%%%%%%%%%%%%%%%%%%%%%

\subsection{The no boundary proposal with canonical Robin boundary conditions}


 
 

Let us consider the usual FLRW ansatz \eqref{FLRW} augmenting the Dirichlet Einstein-Hilbert action \eqref{S} with a (Special) Robin boundary term at the initial surface specified by $q_0,$
\begin{equation}
S_{tot}= S + \alpha q_0 + \frac{q_0^2}{2 \beta} + \gamma\,. \label{eq:Robin}
\end{equation}
The constant $\gamma$ plays no role in the boundary value problem and we add it to the action just to keep the discussion general. Note also that it changes the value of the action by the same factor for all the geometries so that it does not affect relative probabilities. The variation of the full action is 
\begin{equation}
\delta S_{tot} = V_3 \int_0^1 dt \Bigl[ \frac{3 \ddot{q}}{2 N} - 3 N H^2 \Bigr] \, \delta q  - \frac{3 V_3}{2 N} \dot{q}_1 \, \delta q_1 + \Bigl[  \frac{3 V_3}{2 N} \dot{q}_0  + \alpha + \frac{q_0}{\beta}  \Bigl]  \,\delta q_0\,.
\end{equation}
Thus we can see that the variational principle is well defined if we impose 
\begin{align}
 \delta q_1 &= 0 \\
 \frac{3 V_3}{2 N} \dot{q}_0 + \alpha + \frac{q_0}{\beta}  &= 0 
\end{align}
In other words we fix the value of the field at $t=1$ to be $q(t= 1) = q_1$ corresponding to a Dirichlet boundary condition, while at $t= 0$ we impose a condition on the linear combination of $q_0$ and $\dot{q}_0$. This is called a Special Robin boundary condition. (Special because in general one can set that linear combination to any constant value, here it is set to zero.) This way of implementing Robin boundary conditions as given by the boundary term in \eqref{eq:Robin} has very important implications for quantum cosmology and within the minisuperspace approximation which will be discussed below. It is however important to note that  this term is not covariant. Despite this drawback in what concerns diffeomorphisms, canonical Robin conditions have nice properties from a quantum mechanical point of view. The boundary terms can in fact be interpreted as initial and final states of (in general complexified) Gaussian form.

With these boundary conditions, the solution to the equation of motion reads
\begin{equation}
q(t)= H^2 N^2 t^2 - \frac{(\alpha \beta + q_1 - H^2 N^2)}{3 \beta V_3 - 2 N} 2 N t + \frac{(q_1 - H^2 N^2) 3 V_3 + 2 N \alpha}{3 \beta V_3 - 2 N} \beta \label{qsol}
\end{equation}
Plugging this solutions back into the action we find that the saddle points are
\begin{equation}
N_s = \frac{3 V_3 \beta}{2} + c_1 \frac{\sqrt{H^2 q_1 -1}}{H^2} + c_1 c_2 \frac{ \sqrt{9 \beta H^4 V_3^2 - 4 (1 + \alpha \beta H^2  )}}{2 H^2} \label{sadab}
\end{equation}
A crucial requirement in order to obtain an implementation of the no-boundary idea is now that at the saddle point, the geometry should be of Hawking type, and in particular it should start at zero size. From \eqref{qsol} one can see that the initial size $\bar{q}_0$ vanishes at one (or more) of the saddle points if $\alpha = \pm 3 i V_3$ or $\beta = 0 $. These are indeed the values for the initial momentum of the Hartle-Hawking and Vilenkin geometries found in eq. \ref{initialmomentum}.
For $\alpha = + 3 i V_3$, the initial geometries vanishes at the ``tunnelling'' saddle points
\begin{equation}
N_{1,2} = \frac{i}{H^2} \pm \frac{\sqrt{H^2 q_1 -1}}{H^2}
\end{equation}
These geometries are unstable; thus in the following we will consider $\alpha = - 3 i V_3$ (we will discuss the meaning of $\beta$ momentarily). This gives that $\overline{q}_0=0$ for
\begin{equation}
N_{3,4} = \frac{- i }{H^2} \pm \frac{ \sqrt{H^2 q_1 - 1}}{H^2}\,,
\end{equation}
which are precisely the Hartle-Hawking saddle points. The other two saddle points are now located at
\begin{equation}
N_{1,2} =  \frac{i }{H^2} \pm \frac{ \sqrt{H^2 q_1 - 1}}{H^2} + 3 \beta V_3 
\end{equation}
and their initial size is $\overline{q}_0 = 3 \beta V_3 (2 i + 3 \beta V_3 H^2)$. 

Therefore requiring that at least one of the saddle point geometries starts out at zero size corresponds to a specific value of $\alpha$ but leaves $\beta$ free.   The value of $\beta$ in fact determines which saddle point(s) are relevant to the path integral. 

\begin{figure}
\centering
\includegraphics[scale=0.5]{FlowLinesDirichlet2.pdf}
\caption{The minisuperspace path integral contains $4$ saddle points (orange dots) in the plane of the complex lapse $N$. Steepest descent/ascent lines of the magnitude of the action are drawn as black lines, with the arrows indicating directions of descent. Asymptotic regions of convergence are shown in green, while divergent ones are red. For the path integral with Dirichlet boundary conditions, the defining contour of positive real lapse (orange line) can be deformed to the (orange dashed) thimble flowing through saddle point $1$ only. This saddle point however admits unstable perturbations. With Robin boundary conditions, saddle points $1,2$ and the singularity of the action at $N=0$ move. Shown here is the direction of motion for negative imaginary $\beta.$} \label{fig:Robin1}
\end{figure}

For $\beta = 0 $, the relevant saddle point is the Lorentzian one $N_1 = \frac{i + \sqrt{H^2 q_1 - 1}}{H^2}$. In this limit the Robin boundary condition reduces to the Dirichlet boundary value $q_0 = 0 $. For non-zero $\beta$ the saddle points in the upper half plane move, and the singularity in the action, which originally resides at $N^\star=0,$ is also shifted to $N^\star= \frac{3 \beta V_3}{2}$. Importantly, the lower saddle points (numbers $3$ and $4$ in Fig. \ref{fig:Robin1}), which correspond to the desired Hartle-Hawking geometries, stay put. 

Our strategy will be to define the path integral on a thimble, for the simple reason that it is then manifestly convergent. This has the important consequence that the partial integrations over the scale factor, the lapse and any other fields that might be present, can be performed in any desired order without changing the end result (i.e. in the case of absolute convergence, Fubini's theorem applies). For completeness, we show in Appendix \ref{WdWproof} that the
path integral satisfies the Wheeler-deWitt equation. At vanishing $\beta$ our defining integration contour is simply the Lorentzian one, i.e. along real positive values of $N.$ This can then be deformed, using Picard-Lefschetz theory, to the thimble passing through the unstable saddle point $1,$ as shown in Fig. \ref{fig:Robin1}. As $\beta$ is turned on, the singularity of the integral at $N=0$ shifts to $N^\star,$ and thus, in order to maintain an invariant definition, we will consider as our integration contour the thimble(s) emanating from $N^\star.$ 

\begin{figure}
\centering
\includegraphics[scale=0.5]{FlowLinesRobin2.pdf}
\caption{For sufficiently large negative imaginary $\beta,$ the unstable saddle point $1$ moves below the Hartle-Hawking saddle point $4$. The thimble emanating from the singularity of the action at $N^\star$ now only passes through this stable saddle point, and the problem with instabilities is avoided. In white we indicate the locus of geometries that contain a singularity, and which we require the thimble to avoid.} \label{fig:Robin2}
\end{figure}

But which value should $\beta$ take? Roughly speaking, the unstable saddle points remain relevant until $\beta$ is large enough in magnitude so that they have moved ``out of the way''. For real negative $\beta$ this approximately means that they have to move further to the left than $N^\star,$ which marks the origin of the integration contour. But since the real part of the saddle points depends on $q_1,$ this is non-sensical from a physical point of view, as the initial conditions would have to keep being readjusted as the universe keeps expanding! On the other hand, in the imaginary $N$ direction, the unstable point (with an original location at $Im(N_{1,1})=+i/H^2$) only has to move beyond the stable one at $Im(N_{1,-1})=-i/H^2.$ This condition is independent of the final size of the universe, and thus we will only consider imaginary $\beta$. The minimal magnitude of $\beta$ is determined precisely by the condition that the unstable saddle point move below the stable one,
\begin{align}  
|\beta| > \beta_{min} = \frac{2}{3 V_3 H^2}\,.
\end{align}
For $\beta$ being negative imaginary and of magnitude larger than this minimal value, the Hartle-Hawking saddle point becomes the only relevant one -- see Fig. \ref{fig:Robin2} for an illustration. Thus we have successfully isolated the Hartle-Hawking saddle point, leading to the propagator
 \begin{equation}
 G[q_1 , 0]_{Robin} = e^{+ \frac{2 V_3}{H^2 \hbar} - i \frac{2 V_3 H}{\hbar} (q_1 - \frac{1}{H^2})^{3/2}} \label{G}
\end{equation} 
In fact, we also have the possibility of including both thimbles emanating from $N^\star,$ i.e. the ones passing through saddles numbered $3$ and $4$ in Fig. \ref{fig:Robin2}. In this case we obtain a linear combination of the Hartle-Hawking saddle point and its time reverse, and if this linear combination is taken with equal coefficients, as would for instance be the case if we defined the original integration contour to run over all real values of the lapse function, from minus infinity to plus infinity, the result will the the real no-boundary wavefunction \eqref{hh} proposed by Hartle and Hawking. As we will discuss next, there exists however one remaining criterion in order for the path integral to be well defined.







%%%%%%%%%%%%%%%%%%%%%%%%%%%%%%%%%%%%%%%%%%


\subsubsection{Avoiding singular geometries}



The Robin boundary condition implies a relationship between the initial size and the initial momentum of the geometries summed over in the path integral. It does however not necessarily avoid the appearance of singularities, in the sense that it may be possible that some of the off-shell geometries contain a region where the size of the universe passes through zero, see Fig. \ref{fig:regsing}. We would like to avoid summing over such geometries, as they are (infinitely) sensitive to the addition of higher curvature terms, thus rendering the result obtained so far potentially unreliable. We will then explore the possibility of defining a path integral that avoids such singularities along the thimble used to define the path integral. This will imply an upper bound on the magnitude of $\beta.$ Clearly, this upper bound will not be as strong as the lower bound found in the previous section: $|\beta|> \beta_{min}$ is essential for the no-boundary well defined, here we are investigating and discussing a potential problem which might arise with the inclusion of higher order curvature corrections.

\begin{figure}
\centering
\includegraphics[scale=0.5]{RegularSingular2.pdf}
\caption{On the left we have the smooth saddle point geometry of Hartle-Hawking type. By contrast, with Robin boundary conditions a typical off-shell geometry will not start at zero size (middle). Some off-shell geometries contain a recollapse to zero size, and it is these geometries that we would like to avoid summing over (right).} \label{fig:regsing}
\end{figure}

Thus we must analyse the locus of geometries containing a region of zero volume. We start by noting that the imaginary part of the scale factor vanishes, $Im(q(\tau))=0,$ at
\begin{equation}
\tau = - \frac{3 i \beta V_3}{ H^2 m } \frac{[-q_1 + 3 i \beta V_3 + H^2(n^2 + m^2 + 3 i \beta m V_3 )]}{[4 n^2 + (2 m + 3i \beta V_3)^2 ]}\,, \label{tau}
\end{equation}
where we have split the lapse into real and imaginary parts as $N = n + i \, m $.  We must then determine if the real part of the scale factor \eqref{q} may simultaneously vanish, with $0 \leq \tau \leq 1$.    There is no concise analytic expression for this function. However, we are only interested in finding the slope of such a curve of singular geometries at the relevant saddle point. It is therefore sufficient to expand to first order around $N_4$ and then solve for the curve $m(n)$. The angle $\phi$ of the ``singular curve'' at the saddle point $N_4$ is 
\begin{equation}
\tan(\phi) = \frac{(- 4 + 3 i \beta H^2 V_3) \sqrt{H^2 q_1 - 1}}{(2 q_1 + 3 i \beta  V_3)H^2 - 4}\,.
\end{equation} 
Meanwhile, at large magnitudes of the lapse, this singular curve is approximately horizontal, see Fig. \ref{fig:Robin2} for a sketch.


Defining $e^{i S_{tot}} \equiv e^{h + i s}$ with $h , s \in \mathcal{R}$, the steepest descent path at the saddle point runs in the direction of the eigenvector of the Hessian matrix for $h(n,m) = - Im (S_{tot}(n , m))$ associated with the negative eigenvalue. At the saddle point $N_4,$ the angle $\theta$ of the thimble is given by
\begin{equation}
\tan(\theta) = \frac{-2 + 3 i \beta H^2 V_3}{2 \sqrt{H^2 q_1  - 1} + H \sqrt{4 q_1 - 12 i \beta V_3 - 9 V_3^2 H^2 \beta^2}}\,, \label{thetathimble}
\end{equation}
where we have assumed that $|\beta|> \beta_{min}$. It is then important that the thimble emanates from the saddle point at an angle that is larger than that of the singular curve, as the thimble runs off to infinity in the upper right quadrant at an angle that asymptotically reaches $\pi/6$, see \cite{Feldbrugge:2017kzv} and also Fig. ref{fig:Robin2}. Only then can the thimble avoid crossing the singular line.  Thus we must ensure that the condition $\tan(\theta) > \tan(\phi)$ is satisfied. For $q_1 > \frac{2}{H^2},$ this happens for negative imaginary $\beta$ with
\begin{equation}
|\beta| < \beta_{max} = \frac{{2}}{{3 V_3 H^2}}\frac{{ 3 H^2 q_1 -4}}{{ H^2 q_1 -2}} \approx \frac{2}{V_3 H^2}\,.
\end{equation}
Thus for negative imaginary $\beta$ with magnitude between $\beta_{min}$ and $\beta_{max}$ (e.g. $\beta = -i/(V_3 H^2)$) the path integral is well defined and contains only the Hartle-Hawking saddle point(s). 


%%%%%%%%%%%%%%%%%%%%%%%%%%%%%%%%%%%%%%%%%%%%%%%


\subsubsection{Perturbations}

Let us now explicitly verify that the perturbations are indeed suppressed, and that the propagator (or wavefunction) describes a stable universe. In our model we only have gravitational wave perturbations to deal with. For a single mode with fixed polarisation, the action at quadratic order is given by (see e.g. \cite{Feldbrugge:2017fcc,Feldbrugge:2017mbc})
\begin{equation}
S^{(2)} = \frac{V_3}{2} \int_0^1 dt \, N \Bigl[q^2 \frac{\dot{\phi}^2}{N^2} - l (l + 2) \phi^2 \Bigr] 
\end{equation} 
where $\phi$ denotes the magnitude of the perturbation, which has been expanded in spherical harmonics with $l\geq 2$ being the principal quantum number. The extension to a sum over all modes is straightforward. The equation of motion for $\phi$ is thus
\begin{equation}
\ddot{\phi} + 2 \frac{\dot{q}}{q} \dot{\phi} + \frac{N^2 }{q^2} l (l + 2) \phi = 0\,,  \label{eqpert}
\end{equation}
where we must also specify boundary conditions. On the final hypersurface we will choose a Dirichlet boundary condition, $\phi(t=1)=\phi_1.$ Then, at the saddle point, regularity alone will automatically pick out the stable Bunch-Davies mode \cite{Feldbrugge:2017fcc}. However, in general we must specify a boundary condition for perturbations around the off-shell geometries too. Let us first look at the general solutions to the equations of motion. In order to do so, it is convenient to rewrite the scale factor as
\begin{equation}
q(t)= H^2 N^2 (t - \gamma)(t - \delta)\,, \label{background}
\end{equation}
where $\gamma, \delta$ can be read off from \eqref{qsol}. Plugging \eqref{background} into eq. \eqref{eqpert} we find that two linearly independent solutions for $\phi(t)$ are $f(t)/\sqrt{q}$ and $g(t)/\sqrt{q}$ with
\begin{align}
f(t) &= [\frac{t - \delta}{t - \gamma}]^{\frac{\mu}{2}} [(1 - \mu)(\gamma - \delta)+ 2 (t - \gamma)] \\
g(t) &= [\frac{t - \gamma}{t - \delta}]^{\frac{\mu}{2}} [(1 + \mu)(\gamma - \delta)+ 2 (t - \gamma)] \label{modes}
\end{align}
and
\begin{equation}
\mu^2 = 1 - \frac{4l(l+2)}{(\gamma-\delta)^2 N^2H^4}\,.
\end{equation}
At the saddle point the parameters $\gamma , \delta $ and $\mu$ reduce to
\begin{align}
\mu (N_4) = (l + 1), \quad \gamma (N_4) = 0, \quad \delta (N_4) = \frac{- 2 i}{\sqrt{H^2 q_1 -1} -i}\,.
\end{align}
It is straightforward to check that this implies that the mode $f$ blows up at $t=0$ whereas $g/\sqrt{q} |_{t=0} = 0 $. Thus the no boundary geometry automatically selects the perturbative mode $g$ and $\phi_0 = 0 $. 
The regular solution at the saddle point reads
\begin{equation}
g(t) = \Bigl[ \frac{t(- i + \sqrt{H^2 q_1 -1})}{2 i + t(- i + \sqrt{H^2 q_1 -1})} \Bigr]^{\frac{l + 1}{2}} \Bigl( \frac{2 i l + 4 i + 2t(- i + \sqrt{H^2 q_1 - 1})}{\sqrt{H^2 q_1 - 1} - i} \Bigr) \label{stablemode}
\end{equation} 
with associated action
\begin{equation}
S^{(2)} = -\phi_1^2  \frac{ l(l + 2) q_1}{2i (l + 1) + 2 \sqrt{H^2 q_1 - 1}} = i \, \phi_1^2 \frac{l (l + 1) (l +2) }{2 H^2} - \phi_1^2 \frac{l (l + 2) \sqrt{q_1}}{2 H} + \mathcal{O} \Bigl( \frac{1}{\sqrt{q_1}} \Bigr)
\end{equation}
The resulting amplitude $e^{i S^{(2)}} $ describes Gaussian distributed perturbations with a scale-invariant spectrum. We conclude therefore that the model is stable against small deviations since large values of $\phi_1$ are exponentially suppressed.  

More generally, we can specify Robin boundary conditions for the perturbations.
The full action for the perturbations is then
\begin{equation}
S_{tot}^{(2)} = \frac{V_3}{2} \int dt [q^2 \frac{\dot{\phi}^2}{ N^2} - l (l+2) \phi^2] + \alpha_{\phi} \phi_0 + \frac{\phi_0^2}{2 \beta_{\phi}}
\end{equation}
which leads to 
\begin{align}
\delta \phi_1 &= 0 \\
\delta \phi_0 [- \frac{V_3 q_0^2 \dot{\phi}_0}{N^2} + \alpha_{\phi} +  \frac{\phi_0}{\beta_\phi} ] &= 0 
\end{align}
Any Robin boundary condition with $\alpha_\phi = 0$ and arbitrary $\beta_\phi$ will then retain the stable mode at the saddle point, while also specifying initial conditions for the perturbations off-shell. Since we are only summing over non-singular geometries off-shell, this will leave the path integral well-defined, and stable. 
 



%%%%%%%%%%%%%%%%%%%%%%%%%%%%%%%%%%%%%%%%%%%





\subsubsection{Interpretation and discussion}

As already pointed out in the case of perturbations by Vilenkin and Yamada \cite{VY1,VY2}, the Robin boundary condition can also be implemented as an additional Gaussian integral 
\begin{align}
\int \, dN \, dq \, e^{iS} \int \, dq_0 \, e^{i\alpha q_0 - \frac{q_0^2}{2|\beta|}} \,,
\end{align}
where we must include an integration over the initial scale factor $q_0.$ Owing to the fact that $\beta$ has to be negative imaginary, this may then also be interpreted as an initial coherent state, albeit one with a Euclidean momentum $\alpha$. 
\begin{align}
\Psi &= \int dN \, {\cal D}q \, dq_0  \, e^{iS/\hbar} \, \Psi_0\,, \quad
\Psi_0 \propto e^{ i  \alpha q_0 -  \frac{q_0^2}{2 |\beta|}} \, . \label{robinstate}
\end{align}
The presence of a Euclidean momentum not only implements the idea of closing the geometry off in Euclidean time, but it also adds a positive weighting to the associated geometries. In this way we can obtain a final result with an enhanced weighting $e^{+2V_3/(\hbar H^2)}$, which in the implementation with Dirichlet boundary conditions was simply impossible \cite{Feldbrugge:2017mbc}. Note that in this context $\sqrt{|\beta|}$ takes on the role of the uncertainty in the initial size $q_0.$ And because we have a coherent state, the uncertainty in the initial momentum is then simply its inverse,
\begin{align}
\Delta q_0 = \sqrt{|\beta|} \sim \frac{1}{H}\,, \quad \Delta p_0 = \frac{\hbar}{\sqrt{|\beta|}} \sim \hbar H\,.
\end{align}
Thus we see that in order to have a well defined path integral, the uncertainty must be shared between the initial size and the initial momentum, with the uncertainty in the initial size being of the order of the Hubble length.\\
One can think of the original formulation of the no-boundary wavefunction as having as initial state a delta function centred at zero size $\delta(q_0)$, inevitably leading to unsuppressed fluctuations. We are dealing here instead with a Gaussian state peaked around zero, with a spread whose range is determined solely by the cosmological constant. This can be considered as a successful, minor modification of the original no-boundary wavefunction, where the HH geometry is actually dominant and singular geometries are avoided. Note however that the original state already encodes fluctuations of the spacetime geometry, i.e. the universe does not arise out of pure nothingness, but rather out of spacetime fluctuations. In some sense this is to be expected on grounds of the uncertainty principle, when applied to the spacetime geometry. 

The (canonical) Robin boundary term in the action can equally well be interpreted as arising from a different state than \eqref{robinstate}
\begin{equation}
\Phi_0 \propto e^{i  \delta q_0 -  \frac{(q_0 - q_i)^2}{2 |\beta|}} \, ,
\end{equation}
if $\delta = - \frac{q_i}{\beta} + \alpha  = - \frac{q_i}{\beta}  -  i  $.  This is possible because $\alpha$ and $ \beta $ are imaginary in this implementation of the no-boundary wavefunction, and the imaginary piece $- \frac{q_i}{\beta}$ can be absorbed into a redefinition of the momentum. 
In particular, if $ \bar{q}_i=- \delta \, \beta $ one can rewrite the state as 
\begin{equation}\label{key}
\Phi_0 \propto e^{ -  \frac{(q_0 - \bar{q}_i)^2}{2 |\beta|}} \, ,
\end{equation}
where now the mean momentum is zero, and the mean size is $\bar{q_i} =i\beta \approx \frac{2}{H^2} \neq 0.$ In this case the central values of the scale factor and momentum are real, yet we still obtain complex saddle points of the path integral because the real values chosen are classically impossible (classically, the momentum is only zero at the waist of the de Sitter hyperboloid, where $q=1/H^2,$ while here we would demand the momentum to be zero at a larger scale factor value). This rewriting reinforces the point that we can no longer interpret this result as tunneling from nothing, even if the dominant saddle points are the complex Hartle-Hawking saddles.


%%%%%%%%%%%%%%%%%%%%%%%%%%%%%%%%%%%%%%%%%%%%%%%%%%%%%%%%%%%%%%%%%%%%%%%%%%%%%%%%%%%%%%%%%%%%%%%%%%%%%%%%%%%%%%%%%%%%%%%%%%%%%%%%%%%



\subsection{The final Hubble rate as a boundary condition}\label{sec:hubblerate}

We have seen in the previous section how, by fixing an initial Robin condition for the path integral, it is possible to define a path integral peaked around the HH saddle point(s). Interestingly, the Robin boundary term can be interpreted as a coherent state. However, we also mentioned that this boundary term is not covariant: it was shown in \cite{Krishnan:2017bte} that the Robin problem for gravity in 4 dimensions is obtained from the boundary term
\begin{equation}
S_B = \frac{1}{\xi} \int_{\partial M} d^3 y \, \sqrt{h} \, ,\label{actionRob}
\end{equation}
where $\xi $ is a constant. With the ansatz~\eqref{FLRW} the Robin boundary term becomes $S_B = \frac{V_3}{\xi} q_1^{3/2}$ which clearly differs from \eqref{eq:Robin}. As a result, it is not clear how this formulation could be extended beyond the minisuperspace approximation. \\
In this section we will provide a formulation of the quantum cosmology integral which avoids this problem making use of an initial Neumann condition and a final covariant Robin condition. One of the reasons why we pick an initial Neumann and a final Robin condition is that other ways to implement covariant Robin conditions (such as initial covariant Robin or Robin conditions at both sides) lead to nonsensical results or unmanageable expressions. This however is not simply of mathematical interest, but the main advantage is in fact that such conditions are physically sensible, as they allow one to specify the Hubble rate on the final hypersurface. One might argue that this is in any case more realistic, since the flatness of the presently observed universe does not allow us to measure the size of the universe whereas we can obtain the Hubble rate rather directly from redshift measurements. The price to pay will be a redefinition of the integral over the lapse $ N $: the Euclidean momentum imposed on the initial hypersurface requires the contour of integration of the lapse function to be shifted away from the Lorentzian contour by a constant imaginary offset. The other drawback of this approach is that, while the canonical Robin boundary term could be interpreted as an initial coherent state of the universe, the covariant Robin term has less clear interpretation from a quantum mechanical point of view. \\
Equipping the Einstein-Hilbert action with a Robin boundary term evaluated on the final spatial surface $ S=S_{EH}+S_B $, the variational principle requires the boundary conditions 
\begin{align}
\frac{\dot{q}_0}{2 N} &= \pi_0\,, \\
\frac{\dot{q}_1}{N } + \frac{2 \sqrt{q}_1}{\xi} & = \alpha  \label{rob} \, ,
\end{align}
with generic $\alpha$ and $\pi_0$. The solution to the equation of motion satisfying these boundary conditions is 
\begin{equation}
\overline{q}(t) = H^2 N^2 t^2 + 2 N \pi_0 t + \frac{\xi^2}{4} (\alpha - 2 (H^2 N + \pi_0))^2 - N (H^2 N + 2 \pi_0) \, ,
\end{equation} 
and the total classical action is 
\begin{align}
\frac{S}{V_3} = &N^3 H^4 (1 - H^2 \xi^2) + N^2 (\frac{3}{2} H^4 \xi^2 (\alpha - 2 \pi_0) + 3 H^2 \pi_0)  \nonumber \\
& + N (- \frac{3}{4} H^2 \xi^2(\alpha - 2 \pi_0 )^2 + 3  \pi_0^2  + 3) + \frac{\xi^2}{8} (\alpha - 2 \pi_0)^3  \, .
\end{align}
In our coordinates \eqref{FLRW}, the Hubble rate is given by $\frac{\dot{q}}{2 N \sqrt{q}},$ and thus we can see from \eqref{rob} that if we set $\alpha=0,$ we may interpret $H_1=-\frac{1}{\xi}$ as the Hubble rate on the final hypersurface. Note that due to the closed spatial slicing of de Sitter space we should require that $H_1 \le H$ or, equivalently, $\xi^2 H^2\ge 1$. With vanishing $\alpha,$ the action can be usefully rewritten as
\begin{equation}
\frac{S}{V_3} = H^2 (1 - H^2 \xi^2) (N + \frac{\pi_0}{H^2})^3 + 3 (N + \frac{\pi_0}{H^2})  - \pi_0 \frac{3  + \pi_0^2}{H^2}\,.\label{actionoffset}
\end{equation}
It is clear from this expression that for real boundary conditions $\pi_0 \in \mathbb{R}$, the integrand $e^{i S}$ oscillates along the real $N$ line and is conditionally convergent. Note that, since $(1 - H^2 \xi^2) \le 0 $, the asymptotic regions of convergence for the lapse integral lie in the wedges between the angles $(\frac{\pi}{3},\frac{2\pi}{3})$, $(\pi,\frac{4\pi}{3})$ and $(\frac{5\pi}{3},2\pi)$. Thus the Lorentzian integral can be defined and calculated, using Picard-Lefschetz theory to deform the real $N$ line to the appropriate steepest descent contour \cite{Feldbrugge:2017kzv}. For example, with the boundary condition $\pi_0=0$ the saddle points are located on the real axis at $N_{\pm} = \pm \frac{H_1}{H^2 \sqrt{H^2-H_1^2}}$ and they describe the expansion of the universe from the waist of the de Sitter hyperboloid where $q(t=0)=1/H^2$ to a final hypersurface with Hubble rate $H_1=-1/\xi,$ according to $q(t)=H^2 N_\pm^2 (t^2-1+\frac{H^2}{H_1^2}).$

For the no-boundary wavefunction however we need an imaginary initial momentum (see eq. \ref{initialmomentum}). When $\pi_0 \in i \mathbb{R}$ the integrand is oscillatory not along the real $N$ line, but rather (asymptotically) along the line parallel to the real axis with an offset given by $-\frac{\pi_0}{H^2},$ as can be seen by inspection from the action \eqref{actionoffset}. In fact, if $\pi_0 \in i \mathbb{R}^-$, the integral along the real line is convergent whereas for $\pi_0 \in i \mathbb{R}^+$ it explicitly diverges for large real $N$. Consequently, if we were to impose the Vilenkin momentum $\pi_0 = - i$ or any classically allowed boundary condition, the Lorentzian path integral would be mathematically well defined -- see also Fig. \ref{fig:RobinFlow}. However in order to implement the no-boundary wavefunction we need $\pi_0 = + i $, for which the integral along the real line diverges. 

\begin{figure}
	\centering
	\includegraphics[width=0.45\textwidth]{NeumtoCov_new.pdf} 
\includegraphics[width=0.45\textwidth]{NeumtoCov_Vilenkin.pdf}
	\caption{Flow lines and saddle points in the complex lapse plane, for a Neumann condition on $\Sigma_0$ and a covariant Robin condition on $\Sigma_1.$ {\it Left panel:} no-boundary wavefunction, $ \pi_0=i $.
	{\it Right panel:} Tunneling wavefunction, $ \pi_0=-i $. In both cases we set $H=1, H_1=1/2.$ In solid green, the real $N$ line; in dashed green, the shifted defining contour. We show in light green the regions of descent from the saddles and in light blue the regions where $ \text{Re}(iS)<0 $. The deformed contour runs along the steepest descent lines $ J_\pm $ and is shown in dashed green superimposed on the black steepest descent contours. The lines of zeroes (in red, dashed) are always avoided by the thimbles.
	}
	\label{fig:RobinFlow}
\end{figure}

A meaningful integral can be obtained by shifting the defining contour to the line $N=-\frac{i}{H^2} + x$ with $x \in \mathbb{R}$ (or potentially shifting the defining contour even below this line). This represents a departure from the exact Lorentzian integral, which is forced upon us by requiring the integral to be well defined. In some sense the departure is quite minimal, as the integration direction is still in the Lorentzian time direction. However, it is a clear departure as the defining sum is now over complex geometries. The extent to which this might constitute a problem may be debated. If we assume this new contour of integration, then the path integral will be equivalent to a sum over the two associated Lefschetz thimbles, as shown in Fig. \ref{fig:RobinFlow}. The thimbles are peaked on the HH saddle points
\begin{align}
N_{\pm} = \frac{1}{H^2}\left(\pm \frac{H_1}{\sqrt{H^2-H_1^2}} -i\right)\,,
\end{align} 
and consequently we again recover the Hartle-Hawking wavefunction
\begin{align}
\Psi \simeq e^{iS(N_-)/\hbar} + e^{iS(N_+)/\hbar}  = e^{+\frac{2V_3}{\hbar H^2}} \cos[  \frac{2 V_3 H}{\hbar} (q_1 - \frac{1}{H^2})^{3/2}]\,.
\end{align}

In analogy with the discussion of the previous section, we would now like to know whether the thimbles intersect any geometry that contains a singularity in the form of $q(t)=0$ for some real $t$ with $0<t<1.$  The equation $q(t) =0 $ is solved for 
\begin{equation}
t_{1,2} = - i \frac{H_1 \pm i \sqrt{H^2 - H^2_1} (1 - i H^2 N)}{H^2 N H_1}
\end{equation}
The imaginary part of $t_{1,2}$ vanishes respectively on the two circles $m^2 + n^2 + \frac{m}{H^2} \pm \frac{n H_1}{H^2 \sqrt{H^2 - H_1^2}} = 0$, where $ N=n + i m $. 
Moreover, the real part of $t_{1,2}$ should vary between $0$ and $1$, which imposes the condition $0< -\frac{H_1 m+\sqrt{H^2-H_1^2} \, n}{H^2 H_1 \left(m^2+n^2\right)} <1 $. This condition selects the arcs of the circumferences that link the saddle points (where $t _{1,2}= 0 $) to the point $ (n,m) = (0 , - \frac{1}{H^2})$ (where $t_{1,2} = 1$). 
Combining  the two conditions above, it is easy to see that the line of zeroes corresponds to the lower arcs of the two circles emanating from the saddles points, see Fig.~\ref{fig:RobinFlow}. In particular, one can verify that close to the saddle points the conditions impose $ m\leq \text{Im}(N_{\pm})=-1/H^2 $. Near the saddle point the curve of zeroes is approximated by the straight line
\begin{align}
q(\delta t)\mid_{N_{\pm} + \delta N} = 0 \rightarrow \delta N = \delta t  \, \frac{H_1^2}{(H^2 - H_1^2)} ( - \frac{i}{H^2} \mp \frac{\sqrt{H^2 - H_1^2}}{H^2 H_1})\,.
\end{align}
Thus at the two saddle points the line of zeroes forms an angle of $\tan(\theta) =\pm \frac{H_1}{\sqrt{H^2 - H_1^2}}$ with the horizontal, respectively. In other worlds, $ \theta \in (\pi, 3 \pi /2) $ for $ N_+ $ and $ \theta \in (0, -\pi/2) $ for $ N_- $.

As for the thimble, it is given by the equation
\begin{align}
\text{Re}\left( \mp 3\frac{H^2}{H_1^2} \sqrt{H^2-H_1^2}(\delta N)^2\right)=0 \, .
\end{align}
Since $H_1\le H$, the flow lines again point in the directions $e^{i\pi/4},e^{i3\pi/4}, e^{i5\pi/4}, e^{i7\pi/4}.$ In this case however the steepest descent path points at $3\pi/4$ radians away from the saddle point $N_+$ (and $\pi/4$ from $N_-$). 
Thus the thimble and line of zeroes avoid each other, and numerical calculations confirm this beyond leading order for the entire trajectory traced by these lines (see Fig.~\ref{fig:RobinFlow}).

In summary, the no-boundary wavefunction can be defined with an initial Neumann condition and a final covariant Robin condition. The latter has the physical interpretation of fixing the  Hubble rate on the final slice. The thimbles associated with the HH saddles also avoid singular geometries everywhere, so that the curvature is everywhere bounded and general relativity can be trusted at every step. The price to pay is a redefinition of the lapse integral. With the standard choice, a contour coinciding with the real line, the integral would have been divergent. The integral is convergent if the contour is shifted by an imaginary offset of at least $-i/H^2$. The important feature is that the initial Neumann condition eliminates the saddle points with unstable fluctuations, leaving only the stable HH saddles. Meanwhile, the physically attractive final condition of imposing the current Hubble rate rather than the (currently unobservable) size of the universe eliminates any potential interference of singular geometries.

%%%%%%%%%%%%%%%%%%%%%%%%%%%





\end{document}