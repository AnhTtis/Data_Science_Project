\documentclass[./AATesi.tex]{subfiles} 

\definechangesauthor[name=Alice, color=blue]{ADT}

\begin{document}
We take here a break from cosmology to introduce all the necessary mathematical and conceptual tools which will be used in the next chapters in our analysis of the semi-classical properties of the early universe. Using the ADM formalism, we will introduce the canonical quantization of the gravitational field \textit{\`{a}} la Wheeler-de Witt. We will also describe the path integral approach underlining the differences between the Euclidean and the Lorentzian formulation, focusing on the notion of causality and the impact of various boundary conditions. In the rest of the work we mainly will make use of path integral techniques and only occasionally refer to the analogue results that can be derived in the canonical framework.

Let us consider a manifold $M$ equipped with the metric $g$. The couple $(M,g)$ describes a globally hyperbolic spacetime if it admits a Cauchy hypersurface i.e. a hypersurface such that the future and the past evolution of the metric is uniquely determined given the initial conditions defined on it.
If this is the case, it is possible to define a slicing of the full spacetime introducing a time-like four-vector field $\overrightarrow{\eta}$. The vectors orthogonal to $\overrightarrow{\eta}$ define a sub-manifold and $M$ is diffeomorphic to a manifold $\Sigma_t \otimes \mathbb{R} $ where $\Sigma_t$ denotes hypersurfaces of equal time \cite{Geroch:1970uw}.    \\
We introduce the four-vector $N^\mu = (N, N^i)$ which describes the deformation that connects the surface of time $t$ with that of time $t + \delta t$. The deformation vector joining the points $(t, x^i)$ and $(t + \delta t, x^i)$ has non vanishing projections on the normal vector $\overrightarrow{\eta}$, given by lapse function $N$, and on the tangent basis vectors corresponding to the components of the shift vector $N^i$. The lapse function measures the proper-time separation of surfaces of constant $t$ while the shift vector measures the deviation of the lines of constant $x^i$ from the normal to the surface $\Sigma$.
\begin{figure}[htbp]
\begin{center}
\includegraphics[width = 7 cm]{adm}
\caption{ADM slicing of a four-dimensional manifold. The deformation vector, $\stackrel{\rightarrow}{N}$ in the picture, which connects two points with same coordinates $x^i$ has non vanishing projection on the normal vector $\stackrel{\rightarrow}{\eta}$, the lapse function $N$, and on the tangent basis vectors $\stackrel{\rightarrow}{b}_i$, the components of the shift vector $N^i$.}
\label{fig:integrando}
\end{center}
\end{figure}
This defines the so-called Arnowitt-Deser-Misner (ADM) variables which will be useful derive the Hamiltonian formulation of general relativity~\cite{Arnowitt:1959ah}:
\begin{equation}
ds^2 = - (N^2 - N_i N^i) dt^2 + 2N_i dx^i dt + h_{ij} dx^i dx^j \label{aDm}
\end{equation}
The Einstein-Hilbert Lagrangian density in ADM variables \eqref{einsteinhilbert} reads
\begin{equation}
\mathcal{L}_{ADM} = \frac{c^3}{16 \pi G}  N \sqrt{h} (R^{(3)} + K_{ij}K^{ij} - K^2) \label{ladm}
\end{equation}
where $R^{(3)}$ is the Ricci scalar on the three-dimensional constant time hypersurfaces and the extrinsic curvature $K_{ij}$ reads\footnote{$D$ denotes the covariant derivative on the hypersurface $D_i A^k = \partial_i A^k + ^3\Gamma_{i r}^k A^r$ where  $^3\Gamma_{i r}^k$ are the three-dimensional Christoffel symbols.}
\begin{equation}
K_{ij}= \frac{1}{2N} \Bigl( \partial_t h_{ij} - D_i N_j - D_j N_i \Bigr)\label{curv}
\end{equation}
Let us now switch to the Hamiltonian formalism defining the conjugate momenta associated to the 10 variables $h_{ij}, N^i,N$.
The momenta conjugated to $h_{ij}$ read
\begin{equation}
\Pi^{ij} := \frac{\partial \mathcal{L}_{ADM}}{\partial( \partial_t h_{ij})}= \frac{c^3}{16 \pi G} \sqrt{h} \, \Bigl(h^{ri } h^{sj} - h^{ij} h^{rs}\Bigr) \, K_{rs}
\end{equation}
It follows from eq.(\ref{ladm}) and eq.(\ref{curv}) that the momenta conjugated to $N$ and $N^i$ vanish. Thus the dynamics satisfies the four primary first-class constraints
\begin{align}
\Pi &= \frac{\partial \mathcal{L}_{ADM}}{\partial( \partial_t N)} \approx 0  \label{cont1}\\
\Pi_{i} &= \frac{\partial \mathcal{L}_{ADM}}{\partial (\partial_t N^i)} \approx 0  \label{cont2}
\end{align} 
Here and in the following the symbol $\approx $  indicates a weak equality that is, $f$ is weakly equal to $g$ ($f \approx g$ ) if they differ for an arbitrary linear combination of the constraints.\\ 
The total Hamiltonian density can be written introducing four Lagrangian multipliers $ \lambda, \lambda_i$ 
\begin{equation}
\begin{split}
H_{ADM} &= \Pi^{ij} \partial_t h_{ij} + \lambda \Pi + \lambda^i \Pi_i - \mathcal{L}_{ADM} \\
&= N \mathcal{H} + N^i \mathcal{H}_i  +  \lambda \Pi + \lambda^i \Pi_i - 2 D_i (\Pi^{ij} N^k h_{kj})
\end{split}
\end{equation}
where the last term vanishes with suitable boundary conditions.\\
The Hamiltonian of the system, which coincides with the total Hamiltonian on the primary constraints surface, reads
\begin{equation}
H = N \mathcal{H} + N^i \mathcal{H}_i \label{hamil}
\end{equation}
where the quantities $\mathcal{H}$ and  $\mathcal{H}_i$ are known as the ``super-Hamiltonian''  and the '`super-momentum'' respectively
\begin{align}
\mathcal{H} = &\frac{16 \pi G}{2}  G_{i j k l}  \Pi^{ij} \Pi^{kl} - \frac{\sqrt{h}}{16 \pi G} \, (^{(3)}R)  \label{HH} \\
\mathcal{H}_i = &- 2  D_j \Pi_i^j \label{hi} 
\end{align}
\\
The ``supermetric'' $  G_{i j k l} $ is given by
\begin{equation}
    G_{i j k l} = \frac{1}{\sqrt{h} } (h_{ik}h_{jl} + h_{il} h_{jk} - h_{ij} h_{kl})
\end{equation}
As the primary constraints are conserved along the time evolution we get the four secondary first-class constraints 
\begin{align}
\partial_t \Pi &= \{ \Pi, H_{ADM}\} = - \mathcal{H} \approx 0 \label{conny}\\
\partial_t \Pi_i &= \{ \Pi_i, H_{ADM}\} = - \mathcal{H}_i \approx 0 \label{conny1}
\end{align}
As a consequence, the total Hamiltonian of General Relativity is a linear combination of constraints and weakly vanishes too
\begin{equation}
H_{ADM} \approx 0
\end{equation}
The evolution of $N$ and $N^i$ is completely arbitrary as it follows from the equations of motion
\begin{align}
\partial_t N &= \{N, H_{ADM}\} = \lambda \\
\partial_t N^i &= \{N^i, H_{ADM}\} = \lambda^i
\end{align}
All the eight constraints (\ref{cont1}, \ref{cont2}, \ref{conny}, \ref{conny1}) are first-class, since the Poisson brackets of anyone of them with any other weakly vanishes, and thus are generators of gauge transformations.

%%%%%%%%%%%%%%%%%%%%%%%%%%%%%%%%%%%%%%%%%%%%%%%%%%%

\subsection{Canonical Quantization} \label{canonical}

The canonical quantization \textit{\`{a}} la Wheeler-de Witt applies Dirac's quantization procedure \cite{Diracqm} to the gravitational field \cite{Dirac:1958sc, DeWitt:1967yk, Wheeler:1988zr}.
The configurations variables and momenta are promoted to quantum operators acting on some Hilbert space. Observables quantities are represented by Hermitian operators acting on this space. The commutator of two operators is an operator corresponding to the Poisson brackets of the two observables.\\
Ehrenfest's theorem states that quantum expectation values behave almost as classical space phase functions\footnote{We use Dirac's notation: $F= \langle \Psi | \hat{F} | \Psi \rangle$ is the expectation value of the quantum operator $\hat{F}$ on the state $|\Psi \rangle$ }
\begin{equation}
 \langle \dot{F} \rangle = \langle \Psi | \{ \widehat{F, H} \} | \Psi \rangle = \frac{1}{i \hbar} \langle \Psi | [\widehat{F}, \widehat{H}] | \Psi \rangle + O(\hbar) = \frac{\partial }{\partial t} \langle \widehat{F}\rangle + O(\hbar)
\end{equation}
This should in particular holds for the constraints.\\
Denoting with $\phi_a$ ($a=1,..,8$) all the constraints of the theory the following relation must hold for every value of $a$ and any element $|\Psi \rangle$ of the Hilbert space
\begin{equation}
\langle \Psi | \widehat{\phi_a}|\Psi \rangle = 0
\end{equation}
We interpret this quantum constraint as a restriction to be imposed on the state $|\Psi \rangle$. That is, we define the ``physical state space'' as the linear subspace of the representation space of states that are annihilated by the constraints. Since first class constraints are generators of gauge transformations, physical states are gauge invariant quantities. Note that the physical state space is not a Hilbert space yet. Once the constraints are solved at the quantum level one has to define a proper scalar product on the space of the solutions to make sense of the quantum theory. This is however a not well understood issue in the canonical quantization program for general relativity.\\ 
Let us apply now Dirac's program to Einstein theory of gravity.\\
The space of states is that of proper functionals of configuration variables, the ``wave functionals"
 \begin{equation}
\Psi = \Psi [N,N^i, h_{ij}]
\end{equation}
In the standard representation, the configuration variables and the conjugate momenta act as multiplicative and derivative operators respectively:
\begin{align}
\widehat{h}_{ij}(x) \Psi &= h_{ij}(x) \Psi \; \; \; \; \widehat{\Pi}^{ij}(x) \Psi= - i \hbar \frac{\delta \Psi}{\delta h_{ij}(x)}\\
\widehat{N}(x) \Psi &= N(x) \Psi \; \;  \; \; \;\widehat{\Pi}(x) \Psi=- i \hbar \frac{\delta \Psi}{\delta N(x)} \\
\widehat{N^i}(x) \Psi &= N^i(x) \Psi \; \;  \; \; \;\widehat{\Pi}_i(x) \Psi=- i \hbar \frac{\delta \Psi}{\delta N^i(x)} 
\end{align} 
The implementation of the constraints (\ref{cont1},\ref{cont2}, \ref{conny}, \ref{conny1}) on a quantum level leads to the following set of equations
\begin{align}
- i\hbar \frac{\delta}{\delta N} \Psi[N,N^i, h_{ij}] &= 0 \label{c1} \\
- i\hbar \frac{\delta}{\delta N^i} \Psi[N,N^i, h_{ij}] &= 0 \label{c2}\\
\widehat{\mathcal{H}_i} \Psi[N,N^i, h_{ij}] &=  0  \label{c3}\\
\widehat{\mathcal{H}} \Psi[N,N^i, h_{ij}]  &= 0 \label{c4}  
\end{align}
Eq.(\ref{c1}) and eq.(\ref{c2}) imply that the physical states do not depend on $N$ and $N^i$ i.e. they do not depend on the slicing of the spacetime. Thus the wave functional $\Psi$, which describes the quantum state of the universe, is a function on the infinite-dimensional manifold $W$ of all three-metrics $h_{ij}$. \\
The super-momentum constraint is satisfied by wave functionals which depend on three-geometries $ \{ h_{ij} \}$ rather than any specific representation in a given coordinate system.
This can be seen for example considering the variation of $\Psi$ under an infinitesimal spatial translation $x^i \rightarrow x^i + \xi^i(x^l)$ is
\begin{equation}
\delta \Psi = -2 \int d^3x  \, \frac{\delta \Psi}{\delta h_{ij}(x)} D_j \xi_i = 2 \int d^3x \, D_j \Bigl[ \frac{\delta \Psi}{\delta h_{ij}(x)} \Bigr]\xi_i
\end{equation}
Thus requiring that $\delta \Psi =0$ implies that
\begin{equation}
\widehat{\mathcal{H}_i} \Psi[N,N^i, h_{ij}] = - 2 i \hbar D_j \Bigl[ \frac{\delta \Psi}{\delta h_{ij} (x)}\Bigr] = 0 \label{hii}
\end{equation}
where we used the expression definition \eqref{hi} of $ \widehat{\mathcal{H}_i}$\footnote{Note that the last equality holds for compact manifolds. For manifold with a boundary the same result is achieved introducing suitable boundary terms.}. \\
The physical space is thus made up of equivalence classes of metrics connected by a spatial coordinate transformation i.e. geometries $\{ h_{ij} \}$. This space is called ``superspace''. Note that the super-momentum constraint is trivially satisfied in cosmology where the universe is described with a homogeneous and isotropic model of spacetime.
Eq.(\ref{c4}) is called Wheeler-deWitt (WdW) equation and is a functional differential equation. It follows from expression (\ref{HH}) that $\widehat{H}$ contains products of operators evaluated at the same point and it is general not tractable or ill-defined. In this work we will only consider highly symmetric classes of metrics $h_{ij}$ ( in most cases homogeneous and isotropic metrics $h_{ij} = h_{ij}(t)$). This means that the wave functional will not take values in the entire infinite dimensional superspace but in the smaller space known as ``minisuperspace'', with a finite number of degrees of freedom. Notice that this restriction corresponds to a symmetry reduction performed at the classical level, before the canonical quantization, which will allow us to deal with solvable models. This procedure is strictly speaking in tension with the Heisenberg's uncertainty principle: for example the quantization of a geometry of FLRW type corresponds to setting to zero at the quantum level both all of the in-homogeneous quantum degrees of freedom and their conjugate momentum \cite{Kuchar:1989tj}. We will come back to this issue in chapter \ref{chapterneumann} where we will discuss a possible path to verify the validity of this approximation, or possibly disprove it, making use of holography.\\


%WdW equation can be derived from a path integral using Einstein-Hilbert action for gravity \cite{kiefer}, \cite{hal}. In the following we will focus on the Path-Integral formulation of Quantum Cosmology and relate our results with those of the canonical approach showing under which assumptions the wave function constructed via path integral satisfies the super-Hamiltonian constraint (see Section \ref{wheldewitt}). \\
%As mentioned before, the Hamiltonian Eq.(\ref{hamil}) is a linear combination of constraints. Therefore it also annihilates physical states
%\begin{equation}
%\widehat{H} \Psi = 0 \label{wdwequation}
%\end{equation}  
%This equation expresses what is known as the problem of time in quantum gravity \cite{rovelli}, \cite{thiemann}. It can be seen as a Schroedinger equation for a quantum state which does not evolve in time giving rise to what is called the \textit{frozen formalism}.
%The wave function $\Psi$ does not depend explicitly on the time $t$ because $t$ is just a coordinate which can be given arbitrary values by different choices of the Lagrangian multipliers $N$ and $N^i$. \\



\subsection{Path Integral} \label{sectionpathintegral}
In the path integral formulation of quantum mechanics one defines the quantum amplitude for a transition of a particle from a position x at time t to a position x' at time t' as a sum over all possible paths linking the initial and the final points
\begin{equation}
\langle x', t' | x, t \rangle = \int \delta x(t) \, e^{i S[x(t)]/\hbar} \label{pathi}
\end{equation}
where all the path in the summation are continuous but might be non differentiable.\\
By construction, this formalism is useful in defining the classical limit of the theory. As $\hbar \rightarrow 0$, two close histories associated with two close values of the action $S$ correspond to huge changes in the phase $e^{iS/\hbar}$. As a consequence most of the paths interfere destructively. The path integral will then be peaked on the classical histories which by definition render the action stationary $\delta S = 0$.\\ 
The quantum mechanics of the gravitational field can also be formulated through path integral methods and deals with the transition amplitude from one three-dimensional geometry to another.
These three-dimensional geometries are defined on spatial hypersurfaces labelled by the time coordinate and separated by a local proper time interval, as given by the ADM slicing (\ref{aDm}). One must consider the class of all the four-geometries in which these two spacelike surfaces occur but which are in general different off the surfaces. The path integral is defined as a sum over all such four-geometries joining the two boundary geometries. Notice that the local proper time separation between the surfaces is not specified as this quantity is be different for every four-geometry in the sum and thus the path integral includes also a sum over all the possible proper time separation between the boundary hypersurfaces.\\
The quantum gravitational path integral for a transition from a three-geometry $g_0$ to another $g_1$ \cite{Misner:1957wq} is then defined to be
\begin{equation}
G[g_1;g_0]=\int \delta g_{\mu \nu}(t,x)\, e^{i S [g_{\mu \nu}(t,x)/ \hbar} \label{path}
\end{equation}
where $S$ is the action functional for gravity with the Gibbons-Hawking-York boundary term \eqref{einsteinhilbert}. As we will see in section \ref{sec:boundaryterms}, the GHY boundary term is required to fix Dirichlet type of boundary conditions i.e. if we wish to fix the two boundary geometries $g_1$ and $g_0$. When other quantities than the induced three-geometry are fixed at the boundaries, different boundary terms must be introduced accordingly. \\
The action \eqref{einsteinhilbert} is invariant under diffeomorphisms i.e. the gauge transformations generated by the first class constraint $\mathcal{H},\mathcal{H}_i$ . When computing the path integral one should sum only over physically different histories  and hence keep only one representative of the class of equivalence of geometries related to each other by these transformations. This means that the domain of integration is a gauge fixing hypersuperface in the configuration space and not the whole space. Hence we should fix the gauge in the summation and introduce a Faddeev-Popov type of ghost contribution according to the usual procedure for systems with gauge freedom. The whole treatment of Batalin-Fradkin-Vilkovisky (BFV) ghost is showed in  \cite{Teitelboim:1983fk, Teitelboim:1981ua, Teitelboim:1983fh, Teitelboim:1983fi, Halliwell:1988wc} and here we will only make some comments on the  key points useful for our purposes.\\
The path integral with the ghost contribution reads
\begin{equation}
G[g_1;g_0]=\int \delta h_{ij} \, \delta \Pi^{ij}  \delta N \delta N^i \, \delta c \, \delta p \, f(N,N^i)e^{i (S + S_{ghost})/ \hbar} \label{pathintegral}
\end{equation}
where $f(N,N^i)$ is gauge fixing condition and $c$ and $p$ are the ghosts and their conjugate momenta.\\
As explained in \cite{Teitelboim:1983fk, Teitelboim:1981ua}, the simplest gauge condition we can impose is the so called ``proper time gauge'' condition
\begin{align}
 \dot{N}(x,t) = 0 \\
 N^i(x,t) = 0
\end{align}
Importantly, with this choice of gauge fixing, all of the ghosts decouple in minisupersace \cite{Teitelboim:1981ua}.\\
The first condition implies that pointwise the proper time separation between two neighbour surfaces does not depend on the position of the surfaces. From a practical point of view this gauge choice implies that the functional integral over the lapse reduces to a ordinary one.\\
The latter condition means that the vector joining a point on the spatial hypersurface at $t$ to the point with the same spatial coordinates at $t + dt$ is normal to $\Sigma_t$. \\
If $\langle g_1, T | g_0 , 0\rangle$ is the usual propagator in quantum mechanics, the path integral is an object of the form
\begin{equation}
\int d T \langle g_1, T | g_0, 0 \rangle = G (g_1;g_0 | E)|_{E = 0}
\end{equation}
where the energy Green function is defined to be
\begin{equation}
G (g_1,g_0| E) = \int d T \, e^{i E T} \langle g_1, T | g_0, 0 \rangle
\end{equation}
The integration over the whole four-metric, including the lapse, implies then that the quantum-gravitation path integral resembles more a energy Green function than the propagator of ordinary quantum mechanics.

We discussed in the previous section how the constraints \eqref{c1}-\eqref{c4} have a functional nature and are therefore difficult if not impossible to solve. The same problem arises trying to define quantum transitions for gravity via path integral. The full integral is in general not manageable, when not ill-defined.  
It is useful in this sense to formulate the problem performing a symmetry reduction. This allows us to deal with systems with a  finite number of degrees of freedom or more tractable problems with still infinitely many of them. The former case includes homogeneous models for which the configuration space is finite-dimensional and we reduce the superspace to a minisuperspace. The term midisuperspace refers to the latter case, which includes for example systems with spherical symmetry, which we will discuss in the last chapter. In this work we are going to consider symmetry reduced path integrals with FLRW, Kantowski-Sachs and Bianchi IX ans\"{a}tze. We discussed in section \ref{canonical} how the minisuperspace approximation is in tension with the uncertainty principle in the canonical quantization. The symmetry reduction of the path integral is clearly equally problematic. The questions we ask here are: how much of the physical information is lost by excluding all of the in-homogeneous histories from the path integral? Is the final result heavily dependent on this restriction? Does the minisuperspace path integral already encode the key elements to describe the system? We are going to keep these questions in mind throughout the work. We will see that, for example, the no boundary proposal implemented with Dirichlet boundary conditions is ill-defined because it leads to unstable inhomogeneous perturbations. This instance is clearly signaling that the minisuperspace approximation is not enough in this case since the inhomogeneous degrees of freedom are not suppressed and one runs into inconsistencies when trying to ignore them.
\\


\subsubsection{Comments on causality}\label{sec:causality}
In quantum field theory Feynman's path integral defines a causal propagator given that there is causal structure in the histories summed over.
Here, once the gravitational field is quantized, there is no more notion of time and apparently we cannot save any clear concept of causality. As pointed out in \cite{ Teitelboim:1983fh}, we can impose a causal structure to the histories that form the path integral partially breaking the gauge invariance of the classical theory. 
In order to describe a transition from $g_0$ to $g_1$, one calculates the quantum amplitude for having a three-geometry $g_0$ on a given spacelike hypersurface and the three-geometry $g_1$ on another one when the proper time separation between two points with spatial coordinate $x$ on the two hypersurfaces is $N(x)(t_1 - t_0)$. Then one integrates over all possible values of the lapse. To recover a causal structure of the propagator, we can choose to allow a summation over only those histories for which $g_1$ lies in the future of $g_0$. This is implemented by the request that $N(x)>0 $ $\forall x$.
Recall that the timelike diffeomorphisms, through the Lie derivative, push the initial hypersurface backward and forward. Thus, with such restriction in the integration over the diffeomorphism group elements, one averages the amplitude only over half of the space of the possible normal deformations of the initial surface. As a consequence the path integral is not invariant under the action of the generator of the time translations i.e.
\begin{equation}
\hat{H}(x)\, G_{+}[g_1,g_0] = - i \delta(g_1 - g_0) \neq 0
\end{equation}
Thus the path integrals gives a Green function of the WdW operator i.e. a propagator.\\
If one allows $N$ to run over the whole real axis one recovers indeed that
\begin{equation}
\hat{H}(x)\, G[g_1,g_0]  = 0
\end{equation}
but the amplitude is now a-causal. \\
Thus if one considers ($ - \infty < N < + \infty$) the path integral is a real solution to Wheeler-DeWitt equation that is, a wave function, with no reference to any underling causal structure.\\
In this work we will consider both propagators and wave functions and will always specify when unclear what type of object the path integrals represents.


\subsubsection{Euclidean vs Lorentzian formulation}\label{sec:conffactor}
In quantum field theory it is often appropriate to perform a rotation from the original formulation in spacetime to the four-dimensional Euclidean speace. This procedure is known as ``Wick rotation" and is implemented by just sending the time coordinate $t \rightarrow - i \tau$. Among other advantages, the Euclidean formulation of the theory improves the convergence properties of the path integral as the integrand changes as follows
\begin{equation}
e^{i S/ \hbar} \rightarrow e^{- S_E}
\end{equation}
and the Euclidean action $S_E$ is bounded from below.\\ 
In analogy with the Wick-rotated quantum field theory, the gravitational path integral is also often formally defined over positive definite Euclidean metrics. Euclidean path integrals are typically advocated in quantum cosmology and the no boundary proposal, on which we will focus later, was originally formulated in this fashion (see \cite{Hartle:1983ai}). 
However in a diffeomorphism invariant theory the analytic continuation of time has not clear interpretation as the 'time' coordinate is not uniquely determined.
A general Lorentzian metric will not have a sector in the complexified spacetime manifold where it is real and positive definite. That is, it is in general not possible to change a metric with a Lorentzian signature (-,+,+,+) to an euclidean signature (+,+,+,+) just sending $t \rightarrow -i \tau$. 
Moreover in the case of gravity, unlike the cases of scalar or Yang-Mills fields, if one takes the Euclidean nature to be fundamental ab initio, the action is unbounded from below because of the so called ``conformal factor problem'' \cite{Gibbons:1978ac}.
The Euclidean action can indeed be written as
\begin{equation}
S_{E} = - \frac{c^4}{16 \pi G} \int d^4 x \sqrt{g} R - \frac{c^4}{8 \pi G} \int d^3 y \, \sqrt{h} K 
\end{equation}
for a positive definite metric $g_{\mu \nu}$.
Considering a conformal decomposition
\begin{equation}
g_{\mu \nu} = e^{2 \phi(x)}\overline{g}_{\mu \nu}
\end{equation}
the gravity action becomes
\begin{equation}
S_E = - \frac{1}{16 \pi G} \int d^4 x e^{2 \phi} \sqrt{\overline{g}} \Bigl[\overline{R} + 6 (\nabla \phi )^2 \Bigr] - \frac{1}{8 \pi G} \int d^3 y \, \sqrt{h} e^{2 \phi}K 
\end{equation}
Since the kinetic term of the conformal mode is positive definite the action can assume arbitrarily  large negative values for rapidly varying $\phi$. As a consequence, the path integral over metrics spanning a given fixed boundary is in general ill-defined and one has to specify a complex contour of integration to get a finite result. There is of course some arbitrariness in the choice of such a contour which affects the final results. In particular this might influence which saddle points are relevant to the path integral. \\
For these reasons we consider misleading the passage to the Euclidean theory and will be focusing in this work on Lorentzian path integrals for quantum cosmology. The Lorentzian formulation is in our opinion much more natural and of more direct physical interpretation. In the following, we will evaluate the corresponding oscillating integral using the mathematical tools of ``Picard Lefschetz theory'' (see section \ref{picard}).  We will show in particular that in the case of a positive cosmological constant, which is relevant for cosmology, minisuperspace Lorentzian path integrals indeed converge. This will let us derive precise and un-ambiguous results for the no boundary proposal. 
We will see in chapter \ref{chapterneumann} that the Lorentzian integral fails to converge if the cosmological constant is taken to be negative and we will be forced to introduce suitable complex integration contours. In this case the path integral gives the partition function for the dual
Euclidean quantum field theory and it is not clear a priori what is the natural set of geometries one should sum over. In this work we adopt the point of view that path integrals for cosmology, where the cosmological is positive, which shall give a semi-classical description of our world and where the lapse function has a clear interpretation in terms of proper time distance between event, should by Lorentzian. We remain agnostic regarding signature requirements for other cases including in particular asymptotically AdS spacetimes.


\subsubsection{Boundary conditions}\label{sec:boundaryterms}
The $d$ dimensional solutions to the Einstein's equations \eqref{einsteinequations} are critical points of the Einstein-Hilbert action
\begin{equation}
    S = \frac{1}{16 \pi G } \int_M d^d x \sqrt{-g} [ R - 2 \Lambda]  +  S_B
\end{equation}
 The boundary contributions is necessary to obtain a consistent variational problem $\delta S=0$ upon variation with respect to the metric.
Fixing different quantities requires different boundary terms. Dirichlet and Neumann are widely known and adequate conditions under most circumstances. While less common, Robin conditions have already proven useful for some gravitational problems, e.g. the formal definition of perturbation theory in Euclidean gravity~\cite{Witten:2018lgb}. The following are the usual options:
\begin{itemize}
\item \textit{Dirichlet} boundary conditions: the required boundary term is the well-known Gibbons-Hawking-York (GHY) term \cite{York:1972sj, Gibbons:1976ue}
\begin{equation}
     S_B = \frac{1}{8 \pi G} \int_{\partial M} d^{(d-1)}y \, \epsilon \sqrt{|h|}  K 
\end{equation}
with $K$ the trace of the intrinsic curvature and $h$ the determinant of the metric induced on the boundary.\\
The variation of the action yields
\begin{equation}
\begin{split}
    \delta S =   &+\frac{1}{16 \pi G } \int_M d^d x \sqrt{-g} \Bigl( R_{\mu \nu } - \frac{1}{2} g_{\mu \nu } R + \Lambda g_{\mu \nu} \Bigr) \, \delta g^{\mu \nu} + \\
    &-   \frac{1}{16 \pi G } \int_{\partial M} d^{(d-1)}y \, \epsilon \sqrt{h} \Bigl(K^{ij} - K h^{ij} \Bigr) \delta h_{ij} 
    \end{split}
\end{equation}
Recalling that the conjugate momentum is given by 
\begin{equation}
    \Pi^{ij}:= \frac{\delta \mathcal{L}}{\delta \dot{g}_{ij}}  = -   \frac{1}{16 \pi G }  \epsilon \sqrt{h} \Bigl(K^{ij} - K h^{ij} \Bigr)
\end{equation}
we can write that variational principle as
\begin{equation}
    \delta S =   
    \frac{1}{16 \pi G } \int_M d^d x \sqrt{-g} \Bigl( R_{\mu \nu } - \frac{1}{2} g_{\mu \nu } R + \Lambda g_{\mu \nu} \Bigr) \, \delta g^{\mu \nu} +\int_{\partial M} d^{(d-1)}y \, \Pi^{ij} \, \delta h_{ij} =0 \label{Dirichlet}
\end{equation}
Thus we find that the principle of least action is compatible with vanishing variation of the metric induced on the boundary $\delta h_{ij} = 0$ or, in other words, that the boundary metric  $h_{ij}$ can be fixed to an arbitrary value. 
\item \emph{Neumann} boundary conditions demand the boundary term~\cite{Krishnan:2016mcj}
\begin{equation}
    S_B= \frac{4 - d}{8 \pi G} \int_{\partial M} d^{(d-1)}y \, \epsilon \sqrt{|h|}  K  
    \end{equation}
The variation of the action is 
\begin{equation}
    \delta S =   
    \frac{1}{16 \pi G } \int_M d^d x \sqrt{-g} \Bigl( R_{\mu \nu } - \frac{1}{2} g_{\mu \nu } R + \Lambda g_{\mu \nu} \Bigr) \, \delta g^{\mu \nu} +\int_{\partial M} d^{(d-1)}y \, h_{ij} \, \delta \Pi^{ij} \label{Neumann}
\end{equation}

implying that we can set the momentum $\Pi^{ij}$ to any desired value at the end points, without fixing the boundary metric itself. Note that for $d=4$ the boundary term vanishes so that in this case Neumann boundary conditions can be imposed adding no boundary term to the Einstein-Hilbert action. 

\item For \emph{Robin} boundary conditions \cite{Krishnan:2017bte}

\begin{equation}
  S_B= \frac{4 - d}{8 \pi G} \int_{\partial M} d^{(d-1)}y \, \epsilon \sqrt{h} K     - \xi (d-3) \int_{\partial M} d^{(d-1)}y \, \epsilon \sqrt{|h|} 
\end{equation}
In this case the variation of the boundary action yields a condition on a combination of both the field value and the momentum, since the variation gives
\begin{equation}
     \delta S =   
    \frac{1}{16 \pi G } \int_M d^d x \sqrt{-g} \Bigl( R_{\mu \nu } - \frac{1}{2} g_{\mu \nu } R + \Lambda g_{\mu \nu} \Bigr) \, \delta g^{\mu \nu} - \int_{\partial M} d^{(d-1)}y \, \delta \Bigl( \Pi^{ij}+ \xi \sqrt{|h|}  h^{ij} \Bigr)
\end{equation}


so that the combination $\Pi^{ij}+ \xi \sqrt{|h|}  h^{ij}$ can be set to any desired value on the boundary. Note that a Robin boundary condition is in no way exotic: when we specify the Hubble rate on a given hypersurface, we are effectively imposing a Robin condition: say we specify $H = \frac{a_{,t_{p}}}{a} \equiv {\cal H},$ then we can rewrite this condition in Robin form as $a_{,t_{p}} - {\cal H} a = 0,$ where $t_{p}$ denotes the physical time. 

\item \emph{Mixed} boundary conditions are obtained introducing different types of boundary terms and thus fixing different quantities on disjoint parts of the boundary. 


\item \emph{Special} boundary conditions arise when the prefactor of the variation is set to zero. For instance, we may obtain a Special Neumann condition if in \eqref{Dirichlet}, instead of setting $ \delta h_{ij}=0,$ we set $\Pi^{ij}  =0$ at the boundary. The variational problem is then also well-defined, but note that this only works for the special case where the momentum is set to zero. If one wants to set the momentum to a non-zero value, one must use the Neumann condition \eqref{Neumann} instead, and this requires a different boundary term. In a similar vein, one may set $h_{ij}=0$ on the boundary thereby converting \eqref{Neumann} into a Special Dirichlet condition. We will encounter slightly more general examples of such Special boundary conditions in section \ref{robincosmology}, where we will implement a Special Robin condition. \\


\end{itemize}



\subsubsection{Picard-Lefschetz theory} \label{picard}

Picard-Lefschetz theory can be seen as a systematic way of evaluating conditionally convergent integrals which we will use for computing the saddle point approximation of oscillatory integrals.
We will briefly review the salient features in this section where part of the text is based on \cite{Bramberger:2019zks}. \\
We are interested in integrals of the form
\begin{equation}
    \int_{\mathcal{C}} dz \, e^{i S(z)/\hbar}
\end{equation}
with $\mathcal{C}$ a real infinite domain. Evidently, this integral does not converge absolutely. The main idea of Picard-Lefschetz theory is to complexify the integral of interest and then deform the original contour of integration in such a way as to render the resulting integral manifestly convergent. It may be useful to consider a simple example, say $\tilde{I}=\int_{\mathbb{R}} dx e^{ix^2}.$ Along the defining contour, namely the real line, this is a highly oscillating integral. But now we can deform the contour by defining $x=e^{i\pi/4}y,$ such that $\tilde{I}=e^{i\pi/4}\int dy e^{-y^2}.$ Along the new contour, the integral has stopped oscillating, and in fact the magnitude of the integrand decreases as rapidly as possible. The integral is now manifestly convergent, and one may check that the arcs at infinity linking the original contour to the new one yield zero contribution. Note that along the steepest descent path, there is an overall constant phase factor (here $e^{i\pi/4}$) -- this is a general feature of such paths.\\
More formally, we can write the exponent $iS[z]/\hbar$ and its argument, taken to be $z$ here, in terms of their real and imaginary parts, $iS/\hbar=h+i H$ and $z=x + i y$ --  see Fig.~\ref{fig:thimble} for an illustration of the concepts. The complex function $iS[z]/\hbar$ is holomorphic if its Wirtinger derivative with respect to the complex conjugate of $z $ vanishes or, equivalently, if its real and imaginary part are solutions to the Cauchy-Riemann equations
\begin{equation}
\begin{split}
\frac{\partial \mathcal{S}}{\partial \overline{z}} = 0 \; \Longleftrightarrow \; &\frac{\partial h}{\partial x} = +  \frac{\partial H}{\partial y} \\
&\frac{\partial h}{\partial y} = - \frac{\partial H}{\partial x} 
\end{split} \label{cr}
\end{equation}
Then the critical points $z_\sigma$ of the action are also critical points of the real part of the exponent $h = \operatorname{Re}(\mathcal{S}) $, the so called the Morse function
\begin{equation}
\frac{i}{\hbar}\frac{\partial S }{\partial z}  = 2 \frac{\partial h}{\partial z}
= 0  \Longleftrightarrow \frac{\partial h}{\partial x} = 0, \; \frac{\partial h}{\partial y}=0
\end{equation}
The critical points $z_\sigma$  are in fact saddle points of $h$ since its Hessian matrix is indefinite
\begin{equation}
     {\mathcal H} = det  \left( \begin{array}{cc} \frac{\partial^2 h}{\partial x^2} & \frac{\partial^2 h}{\partial x \partial y} \\  
\frac{\partial^2 h}{\partial x \partial y} &  \frac{\partial^2 h}{\partial y^2}
\end{array} \right) \Bigr|_{z_{\sigma} }=0 
\end{equation}
Therefore, the solutions to the classical equation of motion, which are stationary points $z_{\sigma}$ of $S$, are saddle points for $h$, from which a steepest descents and a steepest ascent paths start. Downward flow of the magnitude of the integrand is then defined by 
\begin{equation}
\frac{\mathrm{d}u^i}{\mathrm{d}\lambda} = -g^{ij}\frac{\partial h}{\partial u^j}\,,
\label{eq:dw}
\end{equation}
with $u^i=(x,y)$ and $\lambda$ denoting a parameter (along the flow) and $g_{ij}$ denoting a metric on the complexified plane of the original variable $z$ (here we can take this metric to be the trivial one, $ds^2 = d|u|^2$). The Morse function decreases along the flow, since  $\frac{\mathrm{d}h}{\mathrm{d} \lambda} = \sum_i\frac{\partial h}{\partial u^i}\frac{\mathrm{d}u^i}{\mathrm{d}\lambda} = -\sum_i\left(\frac{\partial h}{\partial u^i}\right)^2<0.$ The downward flow Eq. (\ref{eq:dw}) can be rewritten as 
\begin{equation}
\frac{\mathrm{d}u}{\mathrm{d}\lambda} = - \frac{\partial {\bar{\cal I}}}{\partial \bar{u}}, \quad \frac{\mathrm{d}\bar{u}}{\mathrm{d}\lambda} = - \frac{\partial {{\cal I}}}{\partial {u}}\,,
\end{equation} 
and this form of the equations is useful in that it straightforwardly implies that the phase of the integrand,  $H = \text{Im}[iS/\hbar],$ is conserved along a flow,
\begin{equation}
\label{eq:imh}
\frac{\mathrm{d} H}{\mathrm{d}\lambda} = \frac{1}{2i}\frac{\mathrm{d}({\cal I} - \bar{\cal I})}{\mathrm{d}\lambda} = \frac{1}{2i}\left( \frac{\partial {\cal I}}{\partial u}\frac{\mathrm{d}u}{\mathrm{d}\lambda} - \frac{\partial \bar{\cal I}}{\partial \bar{u}}\frac{\mathrm{d}\bar{u}}{\mathrm{d}\lambda}\right) = 0\,.
\end{equation}
Thus, along a flow the integrand does not oscillate, rather its amplitude decreases as fast as possible. Such a downwards flow emanating from a saddle point $z_\sigma$ is denoted by $\mathcal{J}_\sigma$ and is often called a ``Lefschetz thimble''. They define convergent contours for the integral $\int dz \, e^{i S/\hbar}$ under very general assumptions.

\begin{figure}[h] 
		\includegraphics[width=0.85\textwidth]{fig-PLbasic.pdf}
	\caption{Picard-Lefschetz theory instructs us how to deform a contour of integration such that an oscillating integral along a contour $\mathcal{C}$ gets replaced by a steepest descent contour (or in general a sum thereof) along a Lefschetz thimble $\mathcal{J}_\sigma$ associated with a saddle point $\sigma$. Only those saddle points contribute for which the flow of steepest ascent $\mathcal{K}_\sigma$ intersects $\mathcal{C}.$. The figure is taken from \cite{Bramberger:2019zks}.}
	\protect
	\label{fig:thimble}
\end{figure} 

In much the same way one can define an upwards flow
\begin{equation}
\frac{\mathrm{d}u^i}{\mathrm{d}\lambda} = +g^{ij}\frac{\partial h}{\partial u^j}\,,
\label{eq:uw}
\end{equation}
with $H$ likewise being constant along such flows. Upwards flows are denoted by ${\cal K}_\sigma,$ and they intersect the thimbles at the saddle points. Thus we can write
\begin{equation}
{\rm Int}({\cal J}_\sigma, {\cal K}_{\sigma'})=\delta_{\sigma \sigma'}. \label{eq:intersection}
\end{equation}
Our goal then is to express the original integration contour $\mathcal{C}$ as a sum over Lefschetz thimbles, 
\begin{equation}
\label{eq:contourexp}
{\cal C} = \sum_\sigma n_\sigma {\cal J}_\sigma\,.
\end{equation}
Multiplying this equation on both sides by ${\cal K}_{\sigma}$ we obtain that $n_\sigma= {\rm Int}(\mathcal{C}, {\cal K}_{\sigma}).$ Thus a saddle point, and its associated thimble, are relevant if and only if one can reach the original integration contour via an upwards flow from the saddle point in question. Intuitively, this makes sense: we are replacing an oscillating integral, with many cancellations, by one which does not contain cancellations, and thus the amplitude along the non-oscillating path must be lower. Putting everything together, we can then re-express the conditionally convergent integral by a sum over convergent integrals, 
\begin{align}
\label{eq:contour}
\int_{\cal C} \mathrm{d} x \, e^{iS[x]/\hbar} & = \sum_\sigma n_\sigma \int_{{\cal J}_\sigma} \mathrm{d} x \, e^{iS[x]/\hbar} \\
& = \sum_\sigma n_\sigma \, e^{i \, H(x_\sigma)}\int_{{\cal J}_\sigma} e^h \mathrm{d}x \\
& \approx \sum_\sigma n_\sigma \, e^{i S(x_\sigma)/\hbar}\,.
\end{align}
The last line expresses the fact that the integral along each thimble may easily be approximated via the saddle point approximation, the leading term being the value at the saddle point itself.  If required, one can then evaluate sub-leading terms by expanding in $\hbar,$ but in the present work this will not be necessary. This concludes our mini-review of Picard-Lefschetz theory -- for a detailed discussion see \cite{Witten:2010cx} and \cite{Behtash:2015loa}, and for applications in a similar context than the present one see \cite{Feldbrugge:2017kzv,Feldbrugge:2017fcc,Feldbrugge:2018gin}.




\subsection{Classicality}\label{classicality}

The saddle point approximation of the wavefunction of the universe can be evaluated in a precise way thanks to Picard-Lefschetz theory. But under what conditions can we recover a classical universe from the wavefunction?
\cite{Vilenkin:1988yd, Kiefer:1990pt, Kiefer:1993fg, Halliwell:1984eu}\\
Let us consider a vacuum gravitational system described by a set of variables $q^a$. The wavefunction of the system satisfies the WdW equation \eqref{c4}

\begin{equation}
\Bigl[    - \frac{1}{2} G^{a b} \frac{\partial^2 }{\partial q^a \partial q^b} + V(q^a) \Bigr] \Psi = 0 
\end{equation}
where the potential $V$ is given by $V= \frac{\sqrt{h}}{2 (8 \pi G)^2} \, (^{(3)}R - 2 \Lambda) $.
We can identify the real and the oscillatory part of the wavefunction introducing the two real functions $A(q^a)$ and $P (q^a)$
\begin{equation}
    \Psi(q^a) = e^{ - A(q^a) + i P (q^a)}
\end{equation}
Separating the real and in imaginary part of the WdW equation we obtain 

\begin{align}
    \frac{1}{2} G^{ab} \Bigl[ \frac{\partial^2 A}{\partial q^{a} \partial q^b} - \frac{\partial A}{\partial q^a} \frac{\partial A}{\partial q^b}  + \frac{\partial P}{\partial q^a} \frac{\partial P}{\partial q^b}   \Bigr] + V &= 0\\
 \frac{1}{2} G^{ab} \Bigl[ \frac{\partial^2 P}{\partial q^{a} \partial q^b}  - 2\frac{\partial A}{\partial q^a} \frac{\partial P}{\partial q^b}  \Bigr]   &= 0  \label{wdwRI}
\end{align}
If the phase of the wavefunction varies much faster than its amplitude as a function of all variables $q^a$ 
\begin{equation}
    \frac{\partial A}{\partial q^a} \ll  \frac{\partial P}{\partial q^a}  \label{classicalitycondition}
\end{equation}
the first equation becomes the Hamilton-Jacobi equation for gravity 
\begin{equation}
      \frac{1}{2} G^{ab}\frac{\partial P}{\partial q^a} \frac{\partial P}{\partial q^b}   + V = 0 
\end{equation}
where $P$ can be identified as the classical action. With this identification one obtains that the canonical momentum is related to the action in the usual way
\begin{equation}
    p_a = \frac{\partial P}{\partial q^a} \label{momentumcl}
\end{equation}
Knowing $P$, one can intregrate this equation to find the classical dynamics of the system. Note that the Hamilton-Jacobi equation is equivalent to Einstein's equations and thus when the classicality condition \eqref{classicalitycondition} is matched the solution is a classical spacetime.  The WdW equation, which is equivalent to a Klein-Gordon equation, is associated with the conserved current
\begin{equation}
    J^a = - \frac{i}{2} G^{ab} \Bigl[\Psi^* \frac{\partial \Psi}{\partial q^b} - \Psi \frac{\partial \Psi^*}{\partial q^b} \Bigr]
\end{equation}
The second equation \eqref{wdwRI} enforces the conservation of this current and fixes the amplitude associated to the wavefunction.
With the identification \eqref{momentumcl} we obtain that the modulus squared of the wavefunction is roughly given by $1/p$: in ordinary quantum mechanics the amplitude gives the probability for a particle to be found in a certain space interval and it is expected that in the classical limit this probability is given by the inverse of the particle's velocity \cite{Landau:1991wop}.\\
This perturbative method to solve differential equations such as the Wheeler-deWitt and the Schr\"{o}dinger equations was developed by Gregor Wentzel, Hendrik Anthony Kramers and L\'{e}on Brillouin (WKB) and boils down to a familiar concept in quantum mechanics: the solution to the Schr\"{o}dinger equation for a particle in a potential is oscillatory in the classically allowed region and decays exponentially in the classically forbidden region. Throughout this work we will say that a system is becoming classical in a WKB sense when the phase of its wavefunction varies rapidly as compared to its amplitude.\\
In semi-classical gravity we are interested in the regime where part of the system behaves classically while the rest does not. For this case, we shall consider a set of genuinely quantum variables $v^{\alpha}$ together with the classical ones $q^a$. The WdW equation will then be

\begin{equation}
   \Bigl[    - \frac{1}{2} G^{a b} \frac{\partial^2 }{\partial q^a \partial q^b} + V(q^a) + H_v(q^a , v^{\alpha}) \Bigr] \Psi = 0 
\end{equation}
with $H_v(q^a , v^{\alpha})$ is the part of the Hamiltonian which involves the $v^{\alpha}$ variables. In the case of cosmology, we could think for example of the $q^a$ as associated with the FLRW background and the $v^\alpha$ as rescribing the quantum fluctuations.\\
The wavefunction can be written as
\begin{equation}
    \Psi(q^a, v^{\alpha}) = e^{- A(q^a) + i P(q^a)} \chi(q^a , v^{\alpha})
\end{equation}
where to introduce a separation of scales we can think of $\chi$ as given by
\begin{equation}
     \chi(q^a , v^{\alpha}) = e^{i \epsilon \, S(q^a , v^{\alpha})  }
\end{equation}
with $\epsilon$ a small parameter\footnote{This procedure is often called the Born-Oppenheimer approximation of the wavefunction of the universe and the expansion parameter is taken to be the inverse Planck mass \cite{Kiefer:1990pt}.}.\\
The WdW equation for the full system is given by the two equations \eqref{wdwRI}, which are of order $\epsilon^0$, plus the higher order equation
\begin{equation}
    \frac{1}{2}G^{ab} \Bigl[ - \frac{\partial^2 \chi}{\partial q^a \partial q^b} + 2 \frac{\partial \chi}{\partial q^a} \frac{\partial A}{\partial q^b} - 2 i \frac{\partial P}{\partial q^a} \frac{\partial \chi}{\partial q^b} \Bigr] + H_v \chi =0
\end{equation}
In the limit of $\epsilon \ll 1$ and  $ \frac{\partial A}{\partial q^a} \ll  \frac{\partial P}{\partial q^a} $, the first two terms are negligible and we obtain

\begin{equation}
   -G^{ab} i \frac{\partial P}{\partial q^a} \frac{\partial \chi}{\partial q^b}  + H_v \chi =0 \label{wkb2}
\end{equation}
Since the background part of the system, described by the $q^a$ only, is classical and $p_a = \frac{\partial P}{\partial q^a} $, we can define the WKB time as 
\begin{equation}
     \frac{\partial}{\partial \eta}=G^{ab}   \frac{\partial P}{\partial q^a} \frac{\partial }{\partial q^b}
\end{equation}
in which case the eq. \eqref{wkb2} becomes a Schr\"{o}dinger equation for the quantum part of the system
\begin{equation}
    i     \frac{\partial \chi}{\partial \eta} = H_v \chi
\end{equation}
When we quantized the primordial fluctuations in section \ref{sec:QFT} we solved precisely this Schr\"{o}dinger equation. We can in fact think of that Schr\"{o}dinger equation as emerging from solving the WdW 
equation for the full system, including background and perturbations, in the semi-classical limit, that is, when the background FLRW universe behaves classically in a WKB sense.\\
The WKB classicality is based on the saddle point approximation of the wavefunction since in this limit the wavefunction its given by $\Psi \propto e^{iP}$, with $P$ the action evaluated along a solution to the classical equations of motion. It could happen that there is more than one such saddle point, in which case the wavefunction is given by a sum of WKB type of contributions
\begin{equation}
\Psi = \sum_{k}   e^{- A_k(q^a) + i P_k(q^a)}\, ( \chi_k(q^a, v^{\alpha})) \label{stato}
\end{equation}
This superposition does not describe a classical world since interferences come into play and a notion of classical universe seems difficult to recover. Decoherence \cite{Giulini:1996nw} provides a dynamical mechanism to describe the emergence of a classical world considering a separation between relevant and irrelevant degrees of freedom in a measurement.
In cosmology, we can for example take as relevant degrees of freedom the scale factor and some homogeneous scalar fields. Irrelevant degrees of freedom will be 'all the rest of the world' such as gravitational waves and density fluctuations or gas molecules, photons, etc. \cite{Kiefer:1987ft}. 
The interaction of the inhomogeneous fields with the homogeneous ones can render the latter classical. In a sense, then, a classical space-time arises from a 'self-measurement' of the universe \cite{Giulini:1996nw}. 
In the high-dimensional configuration space which includes both homogeneous and in-homogeneous fields, states like (\ref{stato}) are highly correlated. However, most of the environmental degrees of freedom are inaccessible to a localised observer. This leads to a reduced density matrix which is obtained from (\ref{stato}) by integrating out the huge number of irrelevant degrees of freedom $\rho (q_k, q_k') =\int \, dv^\alpha \psi \, \psi^* $. 
After the integration over the irrelevant degrees of freedom, the different branches of the superposition will be correlated to orthogonal states of the 'rest of the world' and hence locally there would be no observed interference at all. The off-diagonal terms of $\rho (q_k, q_k')$ are vanishingly small due to the interaction with the environment and the system is said to decohere. If this happens, the density matrix can be considered to describe a mixture of non-interfering WKB branches.\\
Taking these notions into account, one can attempt to characterize a classical behaviour by using the following criteria \cite{Paz:1991nd}: a system can be regarded as behaving classically when the wave function predicts the existence of correlations between coordinates and momenta \cite{Halliwell:1987eu}. These correlations should be such that the classical equations of motion are satisfied. Moreover, in the classical limit the system decoheres and can be written as a statistical mixture of non-interfering branches.\\
In cosmology, the very concept of time makes sense only after decoherence has occurred. In fact, the suppression of interferences selects a unique background spacetime. Such background acts as a clock which defines the time with respect to which the other fields evolve. In fact, in this limit we recovered an approximate Schroedinger equation for quantum fields from the timeless Wheleer-deWitt equation. This is the limit where QFT in curved space time applies. The concept of decoherence is often advocated in the context of the boundary proposal (see chapter \ref{nbDirichlet}): the real Hartle-Hawking wavefunction is peaked around the two time-reversed versions of the Hartle-Hawking geometry. A single classical world with a definite time direction is said to emerge in this picture due to decoherence. In the next chapter we will see another example where the saddle point approximation of the gravitational path integral is given by two distinct background histories. There we will see that the two histories interfere in a catastrophic way and decoherence cannot fix the problem. The reason for that is that one of the two solutions is always associated with unsuppressed inhomogeneous fluctuations and this leads to a non normalizable wavefunction. In this case one cannot trace over the inhomogeneous degrees of freedom to obtain a separation of WKB branches and the limit of QFT in curved spacetime is ill-defined.







\end{document}