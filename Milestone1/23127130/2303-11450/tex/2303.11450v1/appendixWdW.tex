\documentclass[./AATesi.tex]{subfiles}


%\definechangesauthor[name=Alice, color=blue]{ADT}

\begin{document}




\section{ The path integral satisfies the WdW equation }\label{WdWproof}

We show in this section that the path integral with initial canonical Robin condition satisfies the homogeneous or inhomogeneous Wheeler--deWitt (WdW) equation depending on the defining integration contour. Let us consider the normalised initial state 
\begin{equation}
\Psi(q_0) = \sqrt{\frac{2}{ \pi i \beta}} \, e^{i \alpha q_0 + i   \frac{q_0^2}{ 2  \beta}}\,.
\end{equation}
Note that $\Psi(q_0)$ is a coherent state for $\beta = - i |\beta|$. 
The path integral with Robin boundary conditions can be written as follows 
\begin{equation}
\Psi(q_1) = \int dN \,  d q_0 \,  \delta q \, e^{ i S_D} \,  \Psi(q_0) \, ,
\end{equation} 
where $S_D$ is the appropriate action for the Dirichlet problem for gravity i.e. the Einstein-Hilbert bulk term plus the Gibbons-Hawking-York boundary term,
\begin{equation}
S_D= S_{EH}+S_{GHY} = 3 V_3 \int_0^1  \left[ - \frac{\dot{q}^2}{4 N} + N (1 - H^2 q)\right] \,.
\end{equation}
The integral can be written as follows
\begin{equation}
\Psi(q_1) = \sqrt{\frac{2}{ \pi i \beta}}  dN \,  d q_0 \, \int \delta q \, e^{i S_0}\,,
\end{equation}
where
\begin{equation}
S_0 =3 V_3  \int_0^1  [ - \frac{\dot{q}^2}{4 N} + N (1 - H^2 q)] + \alpha q_0 + \frac{q_0^2}{2 \beta}
\end{equation}
is the total action, including the initial canonical boundary term.
The functional integral over $q$ gives
\begin{equation}
\Psi(q_1)  =\sqrt{\frac{2}{ \pi i \beta_0}} \int d N \,  d q_0 \sqrt{\frac{i 3 V_3 }{4 N}} e^{i S}\,,
\end{equation}
with 
\begin{equation}
S = 3 V_3 \frac{1}{3} \left(\frac{H^4 N^3}{4} - \frac{3 (q_1 - q_0)^2}{4 N} + 3 N \Bigl( 1 - \frac{H^2}{2}(q_1 + q_0)\Bigr) \right) + \alpha q_0 + \frac{q_0^2}{2 \beta}\,.
\end{equation}
To implement the no-boundary proposal, we need to fix $\alpha = - 3 V_3  i  $. Integrating over $ q_0 $, we find
\begin{equation}
\Psi(q_1 ) =  - i \sqrt{\frac{3 V_3}{2 i }}\int \frac{d N}{\sqrt{2 N  - 3 V_3 \beta }} e^{i \overline{S}} \,.
\end{equation}
The action $ \overline{S} = S(q_1 , \overline{q}_0)$ is evaluated at the saddle point $\overline{q}_0 = \frac{3 V_3 \beta (q_1 - N(2 i + H^2 N ))}{3 V_3 \beta  - 2 N}$,
\begin{equation}
\begin{split}
\frac{\overline{S}}{3 V_3} = \frac{1}{6 \left(2 N - 3 V_3 \beta \right)} \Bigl(&H^4 N^4 - 6 V_3 \beta  H^4 N^3 + 6 N^2 (2 - H^2 ( i 3 V_3 \beta  +   q_1)) + \\
& + 18 V_3 \beta  H^2 q_1 N + 3 q_1 ( i 6  V_3 \beta  -  q_1) \Bigr) \, .
\end{split}
\end{equation}
To evaluate the WdW equation, we need to compute
\begin{align}
\frac{\partial^2 \Psi(q_1)}{\partial q_1^2} & =   - i \sqrt{\frac{3 V_3}{2 i }} \int \frac{dN }{\sqrt{2 N -  3 V_3 \beta }} \Bigl[i \frac{\partial^2 \overline{S}}{\partial q_1^2} - \Bigl( \frac{\partial \overline{S}}{\partial q_1}\Bigr)^2 \Bigr] e^{i \overline{S}} \, . %\quad \frac{\partial^2 \overline{S}}{\partial q_1^2}  = \frac{3 V_3}{\beta - 2 N }\,,
\end{align}
We find
\begin{equation}
\begin{split}
 \int \frac{dN }{\sqrt{2 N -3 V_3 \beta}} i   e^{i \overline{S}} \frac{\partial^2 \overline{S}}{\partial q_1^2}  = &- 3 i  V_3 \int \frac{d N}{[2 N -3 V_3 \beta]^{3/2}} \, e^{i \overline{S}} = \\
 =&- 3 i V_3  \Bigl[ \int \frac{d N}{\sqrt{2 N -3 V_3 \beta}} \, i \,  \frac{\partial \overline{S}}{\partial N} \, e^{i \overline{S}} - %\int dN \frac{\partial}{\partial N} 
 \Bigl( \frac{e^{i \overline{S}}}{\sqrt{2 N -3 V_3 \beta}}\Bigr)\Bigr|_{boundary} \Bigr]\,.
 \end{split}
\end{equation}
Therefore 
\begin{align}
\frac{\partial^2 \Psi(q_1)}{\partial q_1^2} =  &- i \sqrt{\frac{3 V_3}{2 i }}  \int \frac{dN}{\sqrt{2 N -3 V_3 \beta}} e^{i \overline{S}} \Bigl[ - \Bigl(\frac{\partial \overline{S}}{\partial q_1} \Bigr)^2 + 3 V_3  \frac{\partial \overline{S}}{\partial N}\Bigr] + \sqrt{\frac{(3 V_3)^3}{2 i }}  \Bigl( \frac{e^{i \overline{S}}}{\sqrt{2 N -3 V_3 \beta}}\Bigr)\Bigr|_{boundary} \nonumber \\
= & \, i \sqrt{\frac{(3 V_3)^5}{2 i }} \int \frac{dN}{\sqrt{2 N -3 V_3 \beta}}  \Bigl[(H^2 q_1 - 1) \, e^{i \overline{S}} \Bigl]+ \sqrt{\frac{(3 V_3)^3}{2 i }} \Bigl(\frac{e^{i \overline{S}}}{\sqrt{2 N -3 V_3 \beta}} \Bigr) \Bigr|_{boundary}
\end{align}
If we consider an integration contour which runs from $N \rightarrow - \infty$ to $N \rightarrow + \infty$ along the real line $\Psi (q_1 )$ satisfies the WdW equation. Indeed, in this case the boundary term vanishes,
\begin{equation}
\lim_{N \rightarrow \pm \infty} \frac{e^{i \overline{S}}}{\sqrt{2 N -  3 V_3 \beta }} = 0 \, . 
\end{equation}
We thus obtain that $\frac{\partial^2 \Psi(q_1)}{\partial q_1^2}  =   - 9 V_3^2 (H^2 q_1 - 1) \Psi(q_1)$.  

The other possibility is to take the contour to run from the singularity at $N^* = \frac{3 V_3 \beta}{2}$ to $N \rightarrow + \infty$.  In order to calculate the boundary term, notice that for $N \approx N^*$ the action diverges as
\begin{equation}
\frac{\overline{S}}{3 V_3} \approx - \frac{1}{32 \left(2 N -  3 V_3 \beta \right)} \left(4 q_1 -  \beta_0  (4 i + 3 V_3 \beta H^2)\right)^2 \,.
\end{equation}
It was shown in~\cite{DiTucci:2019dji} and in Section~\ref{sec:canonical} that the relevant case for the no-boundary proposal is when $\beta$ takes negative imaginary values, $\beta = - i |\beta|$. In this case the singularity $N^* $ lies on the negative imaginary axis and the thimble approaches it along the axis, so that $N \sim i \, n$ as $N \rightarrow N^*$. 
The boundary term at the singularity is then proportional to a Dirac delta function
\begin{equation}
\begin{split}
\lim_{N \rightarrow N^*}  \frac{e^{i \overline{S}}}{\sqrt{2 N -3 V_3 \beta }} & = \lim_{x \rightarrow 0 } \sqrt{2 \pi i } \frac{1}{\sqrt{2 \pi i  x}} \, e^{- \frac{i 3 V_3}{4 x} \left(q_1 -\frac{3 V_3 \beta}{4}(4 i +  3 V_3 \beta  H^2)\right)^2} \\ 
&= \sqrt{\frac{4 \pi}{3  i V_3 }} \, \delta ( q_1 - \frac{3 V_3 \beta}{4} (4 i + \beta_0 H^2)) \, ,
\end{split}
\end{equation} 
where $x = N - N^* $ is purely imaginary. Therefore
\begin{equation}
\frac{\partial^2 \Psi(q_1)}{\partial q_1^2}  =   - 9 V_3^2(H^2 q_1 - 1) \Psi(q_1) - 3 V_3 i  \sqrt{2 \pi}  \,  \delta ( q_1 - \frac{3 V_3 \beta}{4} (4 i + 3 V_3 \beta H^2))\,,
\end{equation}
i.e. $\Psi$ satisfies the inhomogeneous WdW equation.

\clearpage

\end{document}
