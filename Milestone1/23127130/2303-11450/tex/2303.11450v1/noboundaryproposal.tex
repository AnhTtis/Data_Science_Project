\documentclass[./AATesi.tex]{subfiles}


\definechangesauthor[name=Alice, color=blue]{ADT}

\begin{document}
 The universe is a system different from any other in the fact that it is a one-time experiment for which the initial conditions where set once and for all, with no observer being there to measure them. Thus in this case the definition of the initial conditions becomes part of the research program itself. In this sense, quantum cosmology stands as a theory of initial conditions for the universe which aims at giving a definite answer stemming first principles. The question we ask then is if our universe started out in a very special state and if so whether this was a necessary condition or a fortunate accident.
The no-boundary proposal \cite{Hartle:1983ai}, also known as Hartle-Hawking proposal (HH), provides a possible answer to this question based on the idea that among the infinitely many possible initial states of the universe there is one which plays a distinct role and as such should be identified as the correct one. It selects a specific wavefunction of the universe which shall provide suitable initial conditions for our universe as described by inflationary and the standard cosmology. Such wave function describes a (quasi-)de Sitter space being nucleated out of a geometry of zero size or, in other words, 'nothing'.\\
%In the particle analogy, there is a classically forbidden region $0<a<a_0$. A zero-energy particle in $a=0$ will be stuck there if only classical paths are allowed. However, due to quantum mechanics there is non-vanishing probability that the particle tunnels through the barrier and emerges at the classical turning point $a=a_0$, from which it evolves classically.
%The probability of the tunneling is given by $e^{- S_{E}}$ where $S_E$ is the Euclidean action associated with the corresponding instanton solution\cite{kolb} \todo{maybe $e^{- 2 Im[S]}$? o qualcosa del genere}.\todo{elaborate}\\
Let us describe in what follows the proposal in its simplest version where an inflationary universe is approximated by a portion of de Sitter space in closed slicing
\begin{align}
ds^2 = - dt_p^2 + \frac{1}{H^2}\cosh^2 (Ht_p) d\Omega_3^2\,. \label{eq.dS}
\end{align}
where $d\Omega_3^2$ is the line element of a unit three sphere $V_3=1$ and $\Lambda = 3\,H^2,$ is the cosmological constant.\\
In this case, the minimum classically allowed size of the universe corresponds to the waist of the hyporboloid de Sitter (where the scale factor $a$ takes the value $a(t_p=0) = H^{-2}$) and at the classical level it is impossible to start from 'nothing' ($a=0$) and end up in de Sitter space. We can however think about a quantum tunneling from zero size to the minimum classically allowed size that is, to the waist of the de Sitter hyperboloid. The corresponding instanton is then given by the Euclidean version of de Sitter space. Concretely, the Euclidean de Sitter space in four dimensions is a four-sphere $S^4$ and the instanton is given by half of the sphere, where the equator corresponds to the size $H^{-2} $ 
\begin{align}
ds^2 = d\eta^2 + \frac{1}{H^2} \sin^2 (H\eta) d\Omega_3^2\,, \quad t_p = - i \left(\eta - \frac{\pi}{2H}\right)\,, \quad \eta \in [0 , \frac{\pi}{2 H}] \label{eq.S4}
\end{align}
and the geometry smoothly closes of at the South Pole $\eta = 0$.\\
Importantly, this instanton solution\footnote{Here and in the following we will call, with a slight abuse of language, "instantons" the saddle points solutions around which the semi-classical wavefunction is peaked not only when these solutions are purely Euclidean but also when they are generically complex.} is not meant to describe the actual geometry of the early universe but represents solely a mathematical tool to describe a tunneling probability.\\ Classical solutions exist for values of the scale factor larger than the de Sitter waist. Thus, the nucleation from nothing of a universe of size larger than the waist $H^{-2}$ will be described by a complex geometry which can be thought of as half of the Euclidean de Sitter four-sphere, describing the classically forbidden region, glued onto a portion of the Lorentzian de Sitter hyperboloid, corresponding to the classically allowed region. This gives rise to the famous Hartle-Hawking geometry represented in Fig. \ref{fig:Wick}. The no-boundary proposal states that the wavefunction of the universe ought to be peaked around it. This wavefunction then provides,``the amplitude for the Universe to appear from nothing" \cite{Hartle:1983ai} as given by $e^{iS}$, where $S$ is the action evaluated along the HH geometry. Notice that, while the Lorentzian de Sitter space has two boundaries, one in the future and one in the past, with this construction, we obtained a geometry which has no boundary in the past, from which the name of the proposal. Importantly, for this to work it is essential that the space-like surfaces of the universe are closed which becomes a genuine prediction of the no-boundary proposal.\\ 
\begin{figure}
	\centering
	\includegraphics[width=0.4\textwidth]{NoBoundaryGeometry.pdf}
	\includegraphics[width=0.4\textwidth]{FigWick.pdf} 
	\caption{{\it Left panel:} The Hartle-Hawking/Vilenkin geometry, typically represented as a Lorentzian geometry (in green) glued onto a Euclidean geometry (in red). {\it Right panel:}  Green lines with arrowheads indicate directions of increasing real/Lorentzian time, red lines depict imaginary/Euclidean time directions. The Euclidean extensions of the geometries imply specific Wick rotations \cite{DiTucci:2019bui}.
	}
	\label{fig:Wick}
\end{figure}
While the separation into a Euclidean and a Lorentzian section is useful for grasping the physical intuition, the full geometry is really a complex smooth geometry. In order to understand that, it is useful to switch time coordinate to $t \in [0,1]$ with \cite{Halliwell:1988ik}
\begin{equation}
\begin{split}
&ds^2 = - \frac{N_{HH}^2}{q(t)} dt^2 + q(t) d \Omega_3^2, \quad q(t) = H^2 N_{HH}^2 + (R_3^2 -H^2 N_{HH}^2 )t \\ 
    &\sinh(H t_p) = H^2 N_{HH} t + i   , \quad N_{HH} = \frac{\sqrt{H^2 q_1 - 1}}{H^2} - \frac{i}{H^2} \label{eq.nbgeometry}
    \end{split}
\end{equation}
With this choice of coordinates the entire HH complex geometry is covered where $q_1$ is size of the final three-sphere $S^3$ and is the argument of the Hartle-Hawking wave function.\\
In order to understand the relation of the metric \eqref{eq.nbgeometry} to the Hartle-Hawking geometry given by the appropriate gluing of~\eqref{eq.dS} with~\eqref{eq.S4}, it is useful to measure the comoving ``distance''~$D$ traversed in the geometry~\eqref{eq.nbgeometry} as $t$ varies from~0 to~1\footnote{The next paragraph is taken from Ref. \cite{DiTucci:2020weq}.}
\begin{equation}
\label{eq.HHdistance}
D = \int_{0}^1 dr \, \sqrt{-\frac{N_{HH}^2}{q}}.
\end{equation}
If $q_1 \leq 1/H^2$, then $D$ is real and positive and one can check that the geometry~\eqref{eq.nbgeometry} is Euclidean and represents a portion of the four-sphere~\eqref{eq.S4}. In the case when $q_1 > 1/H^2$, one obtains complex~$D$. The real part of~$D$ is then always equal to $\frac{\pi}{2\,H}$, which corresponds to the Euclidean distance traversed in the half-sphere part of the Hartle-Hawking geometry~\eqref{eq.S4}. The imaginary part of~$D$, which in the absence of a real part of~$D$ would correspond to a time-like separation in the Lorentzian signature, turns out to be nothing else than the proper time elapsed in the de Sitter geometry~\eqref{eq.dS} between $t = 0$ and $t = H^{-1} \, \mathrm{arcosh}{(H\, \sqrt{q_1})}$, i.e. the time in which the three-sphere reaches proper size~$q_1$. This implies that our complex saddle-point geometry~\eqref{eq.nbgeometry} for $q_1 > 1/H^2$ interpolates ``diagonally'' in the complex metric plane between the locus where~$q = q_1 > H^{-2}$ and the locus where~$q = 0$ with $q$ in between these points being a complex function of $t$. The Hartle-Hawking geometry of~\eqref{eq.S4} and~\eqref{eq.dS} achieves the same end point for $q$ by first moving in the real direction of~$D$ (the four-sphere part) and then in the imaginary direction of~$D$ (the de Sitter part), in such a way that $q$ takes real values everywhere (see right panel of Fig. \ref{fig:Wick}). This geometry, which may be seen as a gluing of geometries with two different lapse values, is related to our saddle point geometry by a complex diffeomorphism and should be regarded as equivalent in our formalism\footnote{It is only in the case of a cosmological constant or adiabatic matter that the Hartle-Hawking geometry has a representation in which the scale factor is everywhere real \cite{Feldbrugge:2017fcc}. When more general matter is added, such as a scalar field in a non-constant potential, it is necessarily complex \cite{Lyons:1992ua}.}. 

Hartle and Hawking's idea is extremely powerful in describing the initial conditions of our universe. The classical (de Sitter) inflating universe is in fact closed off in the past in a smooth way by means of a Euclidean region: What was the Big Bang singularity in the classical picture is now replaced by the smooth sphere with the South Pole playing no special role. \\
Quoting Hawking's lecture at the Pontifical Academy in 2016
\begin{quote}
    Asking what came before the Big Bang is meaningless, according to the no-boundary proposal, because there is no notion of time available to refer to. It would be like asking what lies south of the South Pole.
    \end{quote}
At the same time, the Hartle-Hawking wavefunction of the universe $\Psi \approx e^{iS(q_1)}$ describes the quantum birth of a classical universe. The Einstein-Hilbert action \ref{einsteinhilbert} evaluated along the HH geometry is
\begin{align}
S(q_1) = \frac{2 V_3}{H^2} [- i + (H^2 q_1 -1 )^{3/2}]\,. \label{hhaction}
\end{align}
The imaginary part determines the weighting and thus the relative probability of nucleation of a universe with this value of the cosmological constant while the real part of the action is associated with the classical growth of the universe up to a radius $q_1.$ This classical growth is seen as a phase in the wavefunction. Thus the larger the final size the more oscillating is the wavefunction. The fact that this phase grows fast as the universe expands, while the weighting remains constant, is then an indication that the universe has become classical in a WKB sense. When the system is classical enough, the saddle point solution can be trusted to represent the actual trajectory of the system that is, the classical story of cosmology starts with a large universe which is inflating and the HH wavefunction effectively describes its quantum nucleation.\\
Since the weighting gets larger for smaller values of $H$, one would be tempted to conclude that the no boundary proposal favours the quantum birth of a large de Sitter universe. An analogous calculation to the one presented above \cite{Halliwell:1990uy} shows that when the inflationary phase is driven by a scalar field $\phi$ rolling down a potential $V(\phi)$, the HH saddle point gives the action \eqref{hhaction} where in this case $3 H^2(\phi)= V(\phi)$. As a consequence, a universe where the scalar field finds itself on a lower location on the potential seems to be quantum mechanically favoured, raising the question of whether the no boundary proposal is compatible with an inflationary phase which is long enough to match observations. In this sense it is fair to say that, despite the nice properties discussed above, the question of how to extract precise physical predictions from the no boundary proposal and whether it predicts the universe we live in still lacks a definite answer (see for example \cite{Hartle:2008ng, Vilenkin:1987kf, Hartle:2007gi, Lehners:2015sia, Matsui:2020tyd} for ideas and discussions). \\ 
So far we only discussed the no-boundary wavefunction of a homogeneous and isotropic universe. When perturbations are included in the picture they are forced by regularity to be in the Bunch-Davies vacuum. In the HH geometry, the Euclidean section is reached by means of a Wick-rotation in a specific direction that is, we chose the sign $t_p = - i \left(\eta - \frac{\pi}{2H}\right)$ over $t_p = + i \left(\eta - \frac{\pi}{2H}\right)$ in \eqref{eq.S4}. Upon such Wick-rotation, one of the perturbative modes in eq.\eqref{eq:BD} diverges and while the other vanishes as the universe reaches zero size, which one of the two depends on the direction of the rotation. The HH geometry is constructed in such a way that the Bunch-Davies mode is the regular one and thus perturbations are forced by regularity to start out in this state. The geometry given by the opposite Wick-rotation corresponds to the so called tunneling/Vilenkin proposal \cite{Vilenkin:1982de, Linde:1983mx} (on which we will comment further momentarily) and in our opinion does not represent a suitable choice precisely because it leads to unsuppressed perturbations.
%The direction of the Wick-rotation also effects the resulting tunneling probability giving rise to different interpretation of the semiclassical properties of the two proposals.\todo{vedi se elaborare}. \\
The Hartle-Hawking wavefunction thus predicts the nucleation out of 'nothing' of a classical inflating closed universe with gaussian distributed inhomogeneous perturbations providing the suitable initial conditions for inflation we were after~\cite{Halliwell:1984eu}.\\
Why should we pick this initial state of the universe? 
In Hawking words \cite{Hawking:1981gb}:
\begin{quote}
    There ought to be something very special about the boundary conditions of the universe, and what can be more special than the condition that there is no boundary?
\end{quote}
or put differently \cite{Hawking:1988qt}
\begin{quote}
     One could say: ``The boundary condition of
the universe is that it has no boundary." The universe would be completely self-contained and not affected by
anything outside itself. It would neither be created nor destroyed, It would just BE."
\end{quote}
This is ultimately the fundamental principle which selects the Hartle-Hawking wavefunction as the initial state of the universe and comes from an intuition provided by the HH saddle point geometry. \\
Classical inflationary spacetimes are not past-complete \cite{Borde:2001nh} and thus appropriate boundary conditions for every field in the universe, including the primordial perturbations, need to be put in by hand or explained by physics other than inflation. In chapter \ref{quantuminitial} we provided a semi-classical argument which points in the same direction: an external mechanism needs to be advocated to put gravity wave perturbations in the Bunch-Davies vacuum showing the 'quantum' incompleteness of inflation. The no-boundary proposal solves the problem of boundary conditions all together since the HH geometry has no boundary in the past.
So far, we only discussed the saddle point approximation of the Hartle-Hawking wavefunction. The full wavefunction  of the universe is formulated as a path integral according to Feymann's description of quantum mechanics, where here the sum is a sum over geometries:
\begin{equation}
\Psi[h_{ij}, \phi] =\sum_M\nu(M) \int_D \delta g \, \delta \phi \, e^{i S [g_{\mu \nu},\phi]} \label{hawk}
\end{equation}
where $\phi$ represents a generic matter degree of freedom and $S$ is the action for the matter fields and gravity including the cosmological constant. The integral is a sum over four-geometries with a compact boundary $\Sigma$ whose induced three-metric is $h_{ij}$ and all field configurations which match $\phi$ on the boundary. The path integral is often defined to include also a sum over all four-manifolds $M$ with measure $\nu(M)$. The sum \eqref{hawk} provides a path integral of a solution to the WdW equation \eqref{c4}. A specific wavefunction is single out when the class of geometries in sum $D$ is specified. Hartle-Hawking proposal is to include in the summation only compact geometries i.e. whose only boundary is $\Sigma$ with the fields regular on it. This idea extends the intuition suggested by the HH geometry that the universe has no boundary in space and time to the entire sum and shall give a wavefunction representing the amplitude for a three-geometry to arise from 'nothing' i.e. a three-geometry of zero size. In their '83 paper \cite{Hartle:1983ai}, Hartle and Hawking thought about the path integral \eqref{hawk} as a sum over compact Euclidean geometries. The reason behind that is that the oscillatory Lorentzian integral was thought to be ill-defined. The idea, very popular at the time, was that, just like in quantum field theory, a Wick rotation to Euclidean time would improve the convergence properties of the integral, as opposed to the sum over Lorentzian metrics, and thus that Euclidean quantum gravity would be a better defined theory than a hypothetical Lorentzian quantum gravity. While in general not each Lorentzian metric can be recasted into a Euclidean form by means of a Wick rotation, the idea was that the sum over all Euclidean metric should be equivalent to the sum over all Lorentzian metrics \cite{Hawking:1981gb}. However, the Euclidean action suffers of the conformal factor problem discussed in section \ref{sec:conffactor} which makes the path integral \eqref{hawk} diverge. The way out suggested by Hartle and Hawking was then to ensure the convergence of the integral with a suitable choice of complex contour of integration. However, different convergent complex contours give different wavefunctions. At the semi-classical level, each such contour can be peaked around a different saddle point geometry providing inequivalent semi-classical descriptions of our universe. Thus the prescription that the wavefunction of the universe should be given by a sum over compact geometries is incomplete as it gives different wave functions for different choices of integration contour. Various contours of integration for the no-boundary proposal have been analysed over the years in a large number of papers (see for example \cite{Louko:1988bk, Halliwell:1988ik, Halliwell:1989dy}). A breakthrough in the field happened thanks to Ref.  \cite{Feldbrugge:2017kzv} were the authors where able to demonstrate that the Lorentzian no-boundary path integral indeed converges and, surprisingly enough, does not give the desired answer. In what follows we will present the results of \cite{Feldbrugge:2017kzv} as reviewed in \cite{DiTucci:2019dji}.\\

The full no-boundary path integral \eqref{hawk} cannot be evaluated explicitly in full generality and it is normally studied within the minisuperspace approximation where the geometries considered are of the FLRW form already introduced in eq. \eqref{eq.nbgeometry}
\begin{equation}
ds^2 = - \frac{N^2 }{q(t)} dt^2 + q(t) d \Omega_3^2 \label{FLRW}
\end{equation}  
  with $ d \Omega_3^2$ representing the line element of a three sphere of volume $V_3$ and $t \in [0,1]$.  The function $q(t)$ is the squared scale factor which determines physical lengths in the universe, while $N(t)$ is the lapse function. As mentioned already in section \ref{sec:causality} the range of the lapse integration determines whether the path integral represents a propagator or a wavefunction. In \cite{Feldbrugge:2017kzv}, the authors study the propagator, corresponding to an integration over the positive real $N$ line, and so we will do here but it is important to stress that, qualitatively, the negative result we are going to present applies to both cases.
  The propagator for the no boundary proposal $G[q_1 , 0]$ describes a transition from a spatial 3-geometry of zero size, $q_0 = 0$, to another one which lies in the future of the first and where the scale factor takes the value $q_1 \in \mathcal{R}^+$. The propagator can be evaluated as a path integral, i.e. as a sum over 4-geometries which interpolate the two 3-geometries mentioned above. Each such 4-geometry enters in the sum with a weighting $e^{iS}$ given by Einstein-Hilbert action with a cosmological constant $\Lambda=3H^2$. The various steps needed to define the gravitational path integral can be found in \cite{Teitelboim:1981ua} and \cite{Halliwell:1988wc}. We report here only the final result which in constant lapse gauge $\dot{N} = 0 $ reads 
  \begin{equation}
  G[q_1 , 0 ] = \int_{0^+}^\infty dN \, \int_{q(t = 0)= 0}^{q(t = 1) = q_1} \delta q \, e^{i S / \hbar}  \label{prop}
  \end{equation}
where the action takes form (with $8 \pi G =1$)   
  \begin{equation}
  S = \int d^4 x \sqrt{-g} \left( \frac{R}{2} - \Lambda \right) = V_3 \int_0^1 dt \, [ - \frac{3 \dot{q}^2}{4 N} + 3 N( 1 - H^2 q)] \label{S}
  \end{equation}
The functional integral is evaluated over functions $q(t)$ which take the value $q_1$ at $t=1$ and $q_0 = 0$ at $t= 0 $, for a fixed value of $N$. The integral over the lapse $N$ is an ordinary one due to the gauge fixing condition $\dot{N} = 0 $.
 The lapse is in fact a real variable which takes values along the positive real line $N \in (0^+ , \infty)$. This ensures that the geometries in the sum have a Lorentzian signature. The considered domain of integration makes the path integral a propagator in the sense that it solves the inhomogenous Wheleer-deWitt equation $\hat{\mathcal{H}} G[q_1  , 0 ] = - i \delta(q_1)$ where $\hat{\mathcal{H}} $ is the quantum Hamiltonian.
 
It has been shown in \cite{Feldbrugge:2017kzv} that the result of the various integrations to leading order in the saddle point approximation is 
 \begin{equation}
 G[q_1 , 0] = e^{- \frac{2 V_3}{H^2 \hbar} - i \frac{2 V_3 H}{\hbar} (q_1 - \frac{1}{H^2})^{3/2}} \label{GGG}
\end{equation}  
where it is assumed that $q_1 > \frac{1}{H^2}$. The obtained negative weighting $ e^{- \frac{2 V_3}{H^2 \hbar} } $ is characteristic of the tunnelling proposal \cite{Vilenkin:1982de}, whereas the conjectured no-boundary result (at the same level of approximation) is \cite{Hartle:1983ai}
\begin{equation}
\Psi (q_1) =  e^{+\frac{2 V_3}{H^2 \hbar}} \cos[  \frac{2 V_3 H}{\hbar} (q_1 - \frac{1}{H^2})^{3/2}]\,. \label{hh}
\end{equation}  

The meaning of this result together with the difference between the two proposals may be elucidated by considering the saddle points of the full path integral \eqref{prop}. In fact, there are four saddle points, with geometries given by
\begin{equation}
q(t)= H^2 N^2 t (t - 1) + q_1 t \label{q}
\end{equation}
with the lapse taking the values
\begin{equation}
N_{c_1,c_2} = c_1 \frac{\sqrt{H^2 q_1  - 1} + c_2 i}{H^2} \label{saddle}
\end{equation}
for $c_1 , c_2 \in \{ -1 , 1 \}$. \\
We recognize that the HH geometry \eqref{eq.nbgeometry} corresponds to $c_1 = 1, c_2 = -1$. The geometry corresponding to $c_1 = - 1, c_2= 1$ can be interpreted as the time reversed of the HH geometry. The no-boundary wave function is often required to be peaked around both the HH geometry and its time reversed so that it becomes the real function of the three-sphere of radius $q_1$ given in eq. \eqref{hh}. The complex conjugate geometries, associated with the complex conjugate saddle points $c_1 = 1, c_2 = - 1$ and $c_1 =- 1, c_2 = 1$, give the Vilenkin geometries. Coming back to the representation of these geometries as the gluing of a Lorentzian~\eqref{eq.dS} with a Euclidean~\eqref{eq.S4} section, the Vilenkin geometry amounts to reaching Euclidean time with a opposite Wick-rotation  $t_p = + i \left(\eta - \frac{\pi}{2H}\right)$. Indeed the Hartle-Hawking ($N_{HH}$) and Vilenkin geometries ($N_V$) correspond to opposite signs of initial Euclidean momentum $p := \frac{\partial \mathcal{L}}{\partial \dot{q}}$
\begin{align}
p_{HH}^0=\frac{3 V_3 \, \dot{q}}{2N_{HH}}\mid_{t=0} = + 3 V_3 \, i \,,\qquad p_{V}^0=\frac{3 V_3 \, \dot{q}}{2N_{V}}\mid_{t=0} = - 3 V_3 \, i\,, \label{initialmomentum}
\end{align}

The action evaluated on those solutions is
\begin{equation}
S(N_{c_1 , c_2}) = c_1 \frac{2 V_3}{H^2} [ c_2 i - (H^2 q_1 - 1)^{3/2}]
\end{equation}
From this expression we see that the propagator \eqref{GGG} is given by the contribution of the saddle point with $c_1 =1$ and $c_2 = 1,$ in the upper right quadrant of the complex N plane. It is in fact possible to show, applying Picard-Lefschetz theory, that the integral along the positive real $N$ line is equivalent to the integral along the complex steepest descent path (``thimble'') running through this saddle point alone. As mentioned earlier, this saddle point  is however unstable against inhomogeneous perturbations around the FLRW background (see \cite{Feldbrugge:2017fcc} and \cite{Halliwell:1989dy}). Thus the result \eqref{G}, although mathematically correct, cannot describe our universe on physical grounds. 

The no-boundary result \eqref{hh} instead would have been obtained by considering the contribution of the two saddle points in the lower half of the complex $N$ plane corresponding to $c_1 = 1$, $c_2= -1$ and $c_1 = -1$, $c_2= 1$. There is however no convergent contour which can be deformed into a steepest descent path running through solely these two saddle points \cite{Feldbrugge:2017mbc}. In this sense, the wavefunction \eqref{hh} is not the saddle point approximation of the no boundary wavefunction for any Lorentzian path integral with these boundary conditions. With different boundary conditions, the situation may change, as we will discuss later.

To conclude this section, we would like to highlight that these results imply that the two definitions that "the no-boundary wave function is peaked around the Hartle-Hawking geometry" and "the no-boundary wavefunction is given by a sum over compact geometries" are in fact not compatible, the latter giving a wavefunction peaked around the Vilenkin geometry and thus representing the Vilenkin tunneling proposal. This wavefunction corresponds however to bad behaved perturbations. Notice that the issue with this definition is twofold. On the one hand, it predicts that larger and larger perturbations are quantum mechanically favored, in obvious disagreement with the CMB observations. On the other hand, at a more technical level, this conclusion undermines the assumptions of the computation itself. The full gravitational path integral \eqref{hawk} shall be given by the sum over all geometries which satisfy the imposed boundary conditions. Since the integral cannot be evaluated explicitly in full generality, we restricted the sum to the subset of highly symmetric geometries of FLRW type. The idea is that since our universe is indeed in first approximation homogeneous and isotropic, not much fundamental information is lost with this reduction. If however inhomogeneous perturbations turn out to be unsuppressed, the validity of the approximation itself gets into trouble. These facts lead to the claims of \cite{Feldbrugge:2017fcc, Feldbrugge:2017mbc, Feldbrugge:2018gin} that the no-boundary proposal is ill-defined and not tenable as a theory of initial conditions of the universe. In this work we will take however a different viewpoint of this issue. We will take as the definition of the Hartle-Hawking wavefunction that it must be peaked around the Hartle-Hawking geometry, with all the associated desired properties, and ask the questions: there exists a suitable path integral representation of the Hartle-Hawking wavefunction within the minisuperspace approximation? If so, what is the interpretation associated with this representation? As we will see, it is indeed possible to define the Hartle-Hawking wavefunction within the minisuperspace approximation imposing suitable (non Dirichlet) boundary conditions. These findings will however change drastically the status of the no-boundary proposal.



\end{document}