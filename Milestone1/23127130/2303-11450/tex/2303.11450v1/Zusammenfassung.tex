\documentclass[./AATesi.tex]{subfiles}


\definechangesauthor[name=Alice, color=blue]{ADT}

\begin{document}




\begin{changemargin}{0cm}{-0.48cm}
\section*{Zusammenfassung}
\noindent Das sehr fr\"{u}he Universum wird erfolgreich durch die Quantenfeldtheorie in gekr\"{u}mmter Raumzeit beschrieben, wobei die klassische Hintergrundraumzeit typischerweise eine FLRW-Kosmologie ist und die Quantenfelder, die sich darauf ausbreiten, beinhalten Gravitationswellen und Energiedichtfluktuationen. 
Dieses Regime ist jedoch aus theoretischer Sicht wenig verst\"{a}ndlich, da nur ein Teil der Gravitationsfreiheitsgrade, und nur ein Teil davon, quantisiert werden. 
In dieser Arbeit untersuchen wir diese Begrenzung durch das Zuweisen von Quanteneigenschaften, sowohl zum Hintergrunduniversum als auch zu den Fluktuationen und konzentrieren uns dann auf dem Limes, an der sich das Hintergrunduniversum fast klassisch verh\"{a}lt. 
Die Quantisierung wird im Rahmen der allgemeinen Quantenrelativit\"{a}t durch Feymanns Pfadintegralen beschrieben. 
Wir untersuchen die Sattelpunktsn\"{a}herung von Gravitationspfadintegralen in den F\"{a}llen einer positiven und einer negativen kosmologischen Konstante, unter Benutzung der Minisuperspace Ann\"{a}herung. 
Unsere ersten Ergebnisse sind zwei wichtige negative Ergebnisse in Bezug auf Pfadintegrale mit Dirichlet-Randbedingungen: Die Inflation erlaubt keine Bunch-Davies Anfangsbedingungen, wenn keine pr\"{a}inflation\"{a}re Epoche zugelassen ist und der no-boundary Vorschlag 
%(der Vorschlag ``ohne Grenzen") 
als Summe regul\"{a}rer Geometrien, die bei einer Gr\"{o}{\ss}e von Null beginnen, schlecht definiert ist.
Dies motiviert uns, die Auswirkungen auf das Pfadintegral anderer Klassen von Randbedingungen, wie Neumann- und Robin-Arten zu untersuchen.
Wir finden heraus, dass Robin Randbedingungen verwendet werden k\"{o}nnen, um die Inflation mit der Bunch-Davies Anfangsbedingung in Einklang zu bringen und auch n\"{u}tzlich sind, um gro{\ss}e homogene Skalarfeldschwankungen im Ewigen Inflationsregime zu beschreiben.
Unsere wichtigste Erkenntnis ist, dass sowohl f\"{u}r den no-boundary Vorschlag als auch f\"{u}r die Schwarzen L\"{o}cher im Euklidischen anti-De-Sitter Raum das Pfadintegral mit Neumann-Randbedingungen ben\"{o}tigt wird.
Die Neumann-Bedingung ist in der Tat 
notwendig, um eine vern\"{u}nftige Thermodynamik f\"{u}r Schwarze L\"{o}cher wiederherzustellen und um den no-boundary Vorschlag zu stabilisieren.
Gleichzeitig kann man dies in beiden F\"{a}llen als Regelm\"{a}{\ss}igkeitsanforderung an die Geometrien zusammenfassen.
Die Notwendigkeit von Neumann-Bedingungen impliziert, dass sich die Interpretation der no-boundary Wellenfunktion stark von Hartles und Hawkings urspr\"{u}nglicher Idee unterscheidet, da die anf\"{a}ngliche Expansionsrate des Universums eher als seine Gr\"{o}{\ss}e bestimmt wird.
Unsere Ergebnisse f\"{u}r Schwarze L\"{o}cher unterst\"{u}tzen diese Implementierung des no-boundary Vorschlags, bei dem Regelm\"{a}{\ss}igkeit die Hauptanforderung ist, und erm\"{o}glichen einen genauen definierten Limes der QFT im gekr\"{u}mmten Raumzeit.
Dar\"{u}ber hinaus stellen wir im Fall von Schwarzen L\"{o}chern fest, dass zus\"{a}tzliche Sattelpunkte zum Pfadintegral beitragen, wenn die asymptotische AdS Raumzeit mit einem endlichen Radius abgeschnitten wird.
Die M\"{o}glichkeit, dieses Ergebnis im dualen Bild zu testen, verleiht der Minisuperspace Ann\"{a}herung ein Element der Falsifizierbarkeit, das f\"{u}r die Zuverl\"{a}ssigkeit des gesamten Paradigmas entscheidend ist.

\end{changemargin}

\end{document}