\documentclass[./AATesi.tex]{subfiles}


%\definechangesauthor[name=Alice, color=blue]{ADT}

\begin{document}

In this thesis we discussed various implications of including semi-classical effects in the description of the early universe. We focused on the path integral implementation of semi-classical gravity where the amplitude for a transition from a three-geometry to another, or similarly the wavefunction of the universe, is given by a sum over geometries weighted with the Einstein-Hilbert action. We saw in chapter \ref{quantuminitial} how this framework allows us to check the robustness of inflation against quantum effects. In the usual treatment of inflation, the expansion history of the universe is described by the classical solution to Einstein's equations and only primordial perturbations are given quantum properties, according to QFT in curved spacetime. We saw that this description breaks down if we assume no pre-inflationary phase and thus the beginning of inflation is pushed all the way back to a zero size universe. This work helps us answering the question of how much of a classical phase inflation actually is. A single classical inflating trajectory, over which quantum fields propagate according to QFT in curved spacetime, is recovered if inflation starts when the universe is already large and classical enough. We saw in section \ref{InitialConditions} how under these circumstances
primordial perturbations can then be put in the Bunch-Davies vacuum, a condition which comes about here as an external assumption with some associated level of fine-tuning. We concluded then that inflation, rather than generating primordial fluctuations with the correct features, is a mechanism to process such perturbations in the right way. With this motivation in mind we moved to an analysis of the no boundary proposal, a theory for the wavefunction of the universe, which aims at providing the initial condition for inflation describing the quantum birth of a large, classical, inflationary universe with gaussian distributed linear perturbations. We discussed how the no boundary wavefunction fails to be formulated as a path integral with Dirichlet initial conditions in chapter \ref{nbDirichlet}: if all geometries in the sum start out at zero size, primordial perturbations are necessarily unsuppressed. This means that the no boundary proposal is not given by a sum over compact geometries, questioning the idea that the Hartle-Hawking wavefunction represents the quantum state of a universe with no boundary in space and time. Indeed this conclusion applies within our framework that is, a minisuperspace approximation of the wavefunction of the universe as given by semi-classical general relativity. One cannot exclude that this prescription might give positive results within the yet-to-be-discovered theory of everything where quantum gravity is at work in its full power. It is however worth noting that no boundary solutions seem to be stable against higher order derivative quantum effects \cite{Jonas:2020pos}.
One of the points of strength of the no boundary proposal was indeed its simplicity and minimality: a full theory of quantum gravity might not be needed to answer many open questions in cosmology if the no boundary proposal gave a complete and consistent picture of the early history of the universe with just semi-classical elements where extremely high curvature regimes are not reached at early times. The Hartle-Hawking saddle point and the saddle point approximation of the wavefunction $\Psi \approx e^{i \, S_{HH}}$ are indeed consistent in this sense. 
We thus took the point of view in this work of keeping this simple framework and looking for path integrals that do give the Hartle-Hawking wavefunction in the saddle point approximation. Our goal was to understand what is the meaning of the no boundary proposal if such path integrals exist and at the same time extend our understanding of gravitational path integrals which we believe is a matter of relevance on its own. We thus studied minisuperspace path integrals with Robin type of boundary conditions and showed how the Hartle-Hawking wavefunction can indeed be formulated in this fashion in chapter \ref{robincosmology}. In this case, rather than requiring the initial size of the universe to vanish, one fixes a linear combination of its initial size and momentum. The specification of the desired initial Euclidean momentum is essential to get rid of the unstable solutions with the related unsuppressed perturbations. First, we studied path integrals with what we called ``canonical'' Robin conditions: this Robin initial condition is specific to the minisuperspace approximation and has the nice feature that it can be interpreted as an initial coherent state with both initial size and momentum specified with a given uncertainty. In this case, the initial size of the universe, instead of being exactly zero, is associated with a gaussian peaked around zero, and one could say that, in a sense, there were already fluctuations of space and time from which our universe originated. This type of Robin initial condition could also be useful for a deeper semi-classical understanding of the regime of eternal inflation. We saw in section \ref{papersebastian} that if one wants to describe the inflaton jumping up the potential, in a eternal inflation-like homogeneous transition, one needs such type of Robin condition because the path integral 'needs to know' that the universe is already inflating (through the conjugate momentum) and where the scalar field is located on the potential (that is, the field value) when the transition happens. 
We discussed also, in section \ref{sec:hubblerate}, the possibility of a final covariant Robin condition (with an initial Neumann condition) for the Hartle-Hawking wavefunction, which corresponds to fixing the final Hubble rate of the universe. We consider this an appealing feature, being the expansion rate a measurable quantity. Both the canonical and the covariant implementations result in a well defined Hartle-Hawking wavefunction upon carefully specifying the initial Euclidean momentum. In the covariant case, however, the convergent contour does not correspond to the real lapse line but has a finite off-set with respect to it, for no boundary boundary conditions, a problem which does not arise in the canonical case. In the canonical implementation, sending the variance of the gaussian to infinity, the coherent state becomes a plane wave and the Robin condition a Neumann condition. The implementation of the no boundary path integral with Neumann boundary conditions can be given a covariant form (and correspond to no boundary term in four dimensions) and is convergent over the real $N$ line. For these reasons it represents, in our opinion, the best path integral representation of the Hartle-Hawking wavefunction. This is showed in section \ref{sec:noboundaryterm} where we illustrate how the Harte-Hawking wavefunction can be constructed using such Neumann conditions for both the closed FLRW model and the Bianchi IX model. The various well-defined implementations of the no boundary proposal presented in this work have important elements in common: they all require a departure from the idea of a sum over compact geometries and focus instead on the sign of the initial Euclidean momentum. The initial size of the geometries summed over is not fully specified and in fact geometries of all size are included in the sum. Note that, while Hartle and Hawking suggested to sum over compact regular geometries, the path integral with Dirichlet boundary conditions is full of singularities: many of the off-shell geometries shrink to zero size in an irregular way causing linear perturbations to blow up. This does not happen with Robin or Neumann conditions where only the Hartle-Hawking saddle point geometry starts at zero size and the initial momentum is picked in such a way to ensure a regular closing off. We can then think about the focus on the initial momentum as a regularity condition and thus the Hartle-Hawking wavefunction as the correct one not because it describes a universe which starts at zero size but because it makes it close off regularly. There is hence here a shift of focus in what is the crucial essence of the no-boundary construction from the 'nothing'-ness to the smoothness of the early universe. This intuition is reinforced by the study of black holes in anti-de Sitter space. There, a Neumann initial condition of the same type was needed to make sure that the geometries in the sum had no conical deficit, hence implementing regularity at the horizon. Path integrals for black holes were studied in section \ref{sec:blackholes} where we demonstrated that the Hawking-Page phase transition can be recovered in our framework using a Kantowsky-Sachs type of ansatz. In this case we imposed Dirichlet boundary conditions at the ``outer'' boundary (what was the final boundary in cosmology) and Neumann conditions in the interior (the initial condition in cosmology). The Dirichlet condition at the ``outer'' boundary has the meaning of fixing, via holography, the temperature of the corresponding canonical ensemble. We studied in section \ref{thermodynamics} the thermodynamic interpretation of the saddle points corresponding to AdS black holes finding that Neumann boundary conditions in the interior are in fact necessary in order to obtain the correct canonical partition function of the system. We interpret these results for black holes as a strong support in favour of the Neumann path integrals for the no boundary proposal. The no-boundary wavefunction and the black holes partition function with the Neumann condition are truly sums over regular geometries in the sense of that we explicitly require that there are no irregular geometries in the sum. We start seeing in fact some elements of universality of the Neumann condition in the interior as it is the only condition which gives physically meaningful results and can indeed be seen as a regularity condition in all studied cases. This intuition requires certainly further study which we leave for future work. \\


A deep understanding of the saddle point approximation of gravitational path integrals is of crucial importance in many aspects of theoretical physics. Indeed what we know about the connection between thermodynamics and event horizons is based on this construction \cite{Gibbons:1977mu, Hartle:1976tp}. We observe a renovated attention in recent times towards gravitational path integrals \cite{Almheiri:2019qdq, Penington:2019kki}. The latest breakthrough in this field is certainly the novel understanding of the role that quantum extremal surfaces and islands \cite{Engelhardt:2014gca, Almheiri:2019hni, Penington:2019kki, Almheiri:2019qdq} play in the black holes information paradox \cite{Almheiri:2019psf, Almheiri:2019hni, Hashimoto:2020cas,  Penington:2019npb} and there also similar questions come about \cite{Almheiri:2020cfm, Lewkowycz:2013nqa, Faulkner:2013ana, Dong:2017xht}. In general, the best understood realizations of the holographic principle focus on identification between the saddle point approximation of gravitational path integrals and QFT partition functions. In this sense we believe it will be illuminating to understand the role of the Neumann initial condition in other situations which have a well understood holographic dual extending our calculation for example to arbitrary dimensions or more general types of spacetimes. Another important concept in holography, which is of relevance for our work, is the understanding of specific types of CFT deformations as cutting the bulk spacetime at a finite radius where the imposed boundary conditions are deformed accordingly with respect to those at infinity \cite{McGough:2016lol, Guica:2019nzm, Balasubramanian:2012hb}. It would be nice to see this effect at work in specific examples using the techniques developed in our studies.
 See for example Ref. \cite{Caputa:2019pam} and \cite{Donnelly:2019pie} for a use of minisuperspace integrals in this context. There is also another reason why the study of path integrals for asymptotically anti-de Sitter space is of interest for us. Our work is heavily based on the use of the minisuperspace approximation. If, on the one hand, this allows us perform many calculation explicitly with a good handle on the technicalities of the path integral evaluation, on the other hand, the question remains of how much of the physical input is actually lost with such simplification. We have already mentioned how the ABJM superconformal field theory three-sphere partition function can be recovered exactly with a minisuperspace path integral \cite{Caputa:2018asc, Hirano:2019szi}. It is an interesting question to ask whether this is just a fortunate coincidence or the minisuperspace approximation does indeed already include all of the relevant information in specific physical situations.
We saw in section \ref{sec:blackholes} how when AdS spacetime is cut at a finite radius additional saddle points become relevant to the path integral. This is a genuine prediction of our minisuperspace calculation in the sense of that there is no way to make sense of black hole thermodynamics avoiding such contributions. Hence, through the study of the dual role of such saddle points, holography could in principle provide a playground for testing the soundness of the minisuperspace approximation itself. \\

Our understanding of the early universe cosmology through the mechanism of inflation relies on the assumption of Bunch-Davies initial conditions for the primordial fluctuations. From a theoretical point of view, this assumption is really natural only if inflation is infinite. %However, in order to match observations, inflation must last only around NUMBER of e-folds, making this assumption strictly speaking inconsistent. 
We know however that this cannot be the case, at the very least because %there is another relevant scale in cosmology?, namely the Planck scale, where 
%
at the Planck scale new physics is expected to kick in, making this assumption strictly speaking inconsistent.
In order to derive concrete predictions for any specific model of inflation to compare with astronomical data, one may in many cases relatively safely ignore the problem. With the opposite take on it, one could say that these types of inconsistencies put the entire framework of inflation in serious trouble~\cite{Martin:2000xs, Agrawal:2018own,  Bedroya:2019snp}.
%A question we ask in theoretical cosmology of the early universe is whether there is something of fundamental nature that is inconsistency is signaling and whether this  fact can give us useful hints for answering the question of what happened in the early universe/how we should really picture the early universe. Following this way of reasoning, if there exists a regime where quantum gravity is approximated by semi-classical general relativity, then the no boundary proposal tells us why perturbations should start out in the Bunch-Davies vacuum.
%(more to it/), (is signaling/giving)useful hints towards 
A question we ask in theoretical cosmology of the early universe is whether there is something of fundamental nature that this inconsistency is signaling and whether this fact can give us useful hints for uncovering how we should really picture the early universe. Following this way of reasoning, if there exists a regime where quantum gravity is approximated by semi-classical general relativity, then the no boundary proposal tells us why perturbations should start out in the Bunch-Davies vacuum.
If the no boundary proposal was really inconsistent has claimed in Refs.  \cite{Lehners:2018eeo, Feldbrugge:2017mbc}, then there would really be an issue with the Bunch-Davies assumption itself. Not in the sense that this would make it a wrong assumption but because, from a theoretical point of view, there would be really no justification or firm ground to support it. What we showed in this work is that the Hartle-Hawking wavefunction is in fact well-defined and consistent within the framework on semi-classical gravity. This is means that one can consistently derive the Bunch-Davies condition from quantum cosmology. At the same time, we showed that the Hartle-Hawking wavefunction represents possibly something very different from the nucleation of the universe out of nothing. The challenge is now to deeply understand what this quantum state means and whether the no boundary proposal could really explain the initial conditions of the universe, rather than simply describe them. Our intuition is that the Hartle-Hawking wavefunction might come out as the necessary answer to the gravitational path integral in the semi-classical framework because of the intrinsic properties of such integrals, in particular due to regularity requirements implemented through Neumann boundary conditions.




\clearpage




\end{document}