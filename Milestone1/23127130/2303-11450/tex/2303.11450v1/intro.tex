\documentclass[./AATesi.tex]{subfiles}


\definechangesauthor[name=Alice, color=blue]{ADT}

\begin{document}


Cosmology studies the characteristics and the history of our universe as a whole. The starting point are the observed features of the universe as provided by the Cosmic Microwave Background radiation (CMB) and detected by the COBE, WMAP and PLANCK satellites \cite{Fixsen:1996nj, Bennett:2012zja, Akrami:2018vks}. The first relevant data for us comes from the COBE's observations and is showed in Fig. \ref{Cobe}. The left panel of the figure shows the temperature of the CMB on the full sky projected onto a oval. We learn that the temperature of the CMB is in first approximation the same in every direction and thus our universe is homogeneous and isotropic. This confirms the intuition given by the cosmological principle: we occupy no special place in the universe, in fact any point is just like any other. The right panel shows that the CMB spectrum matches very precisely that of a black body at T$= (2.72548 \pm 0.00057 )$K \cite{Mather:1993ij, Fixsen11} and represents a striking piece of evidence in support of the Big Bang theory for cosmology: the early universe was a very hot black body in thermal equilibrium where the low temperature detected today is due to the fact the universe cools down as it expands. 
The central figure in the left panel shows the first degree of anisotropy observed in the CMB and is a Doppler shift due to the motion of the Solar System with respect to the CMB. The second relevant piece of data for us is showed in the bottom left oval of Fig. \ref{Cobe} and, in a refined version, in Fig. \ref{Planck}. When resolving to a level of 1 part in $10^5$, the CMB shows temperature anisotropies with a spectrum represented in the right panel of Fig. \ref{Planck}. From this spectrum we learn that the universe is filled with 5\% ordinary baryonic matter, 27\% dark matter, 68\% dark energy and is approximately flat \cite{Aghanim:2018eyx}. The inhomogeneous perturbations can be traced back via well established physics to primordial nearly scale invariant perturbations. The goal of theoretical cosmology is to provide a consistent history which explains these observed features. We will begin, in the rest of the chapter, by reviewing the well understood part of this story, namely the one corresponding to the $\Lambda \mbox{CDM}$ model (the standard Big Bang cosmology). We will then continue our journey discussing the theory inflation, which aims at describing a phase in the history of the universe prior to the $\Lambda \mbox{CDM}$ cosmology. The bulk of this thesis will deal with the problem of initial conditions of inflation and the universe, going all the way back to very early times when semi-classical aspects of gravity might have played a crucial role.
\begin{figure}
  \centering
      \includegraphics[width = 6 cm]{CobeNasaPictures}
    \includegraphics[width = 8 cm]{firas_spectrum.jpg}
    \caption{Left panel: The CMB temperature map as observed by COBE. Right panel: The CMB black body spectrum. Error bars are too small to be seen in this picture. The figures are taken from the NASA FIRAS/COBE website.}
    \label{Cobe}
\end{figure}

\subsection{The standard model of cosmology} 
The standard model of cosmology results from the application of general relativity to the entire universe and describes very successfully most of its history. We are going to review the main features in what follows with the goal of clarifying the notation and at the same time setting the scene for the core of our work by highlighting the aspects and open questions which will be relevant for us.\\  

\begin{figure}
    \centering
     \includegraphics[width = 6 cm]{2018PlanckFig}
      \includegraphics[width = 8 cm]{2018PlanckSpectrum}
    \caption{
   Left panel: The CMB temperature map as observed by PLANCK. The region in between the gray lines represents our galaxy. Right panel: The CMB temperature power spectrum.
    The figures are taken from \cite{Akrami:2018vks}.}
    \label{Planck}
\end{figure} 

Denoting with $M$ a manifold equipped with a metric $g_{\mu \nu}$ with $(-, +, + , +)$ signature, the action functional for general relativity reads\footnote{In the following the greek letters refer to the spacetime components $\mu = 0,1,2,3$. The Latin letters denote the spatial components and run from 1 to 3. Also, we use units such that $c=1$.}
\begin{equation}
    S = \frac{1}{16 \pi G } \int_M d^4 x \sqrt{-g} [ R - 2 \Lambda + \mathcal{L}_m]  +  \frac{1}{8 \pi G} \int_{\partial M} d^3y \, \epsilon \sqrt{|h|}  K   \label{einsteinhilbert}
\end{equation}
where $g= det(g_{\mu \nu})$ is the determinant of the metric,  $\Lambda$ is the cosmological constant and $\mathcal{L}_m$ is the matter Lagrangian. The Ricci scalar $R= g^{\mu \nu } R_{\mu \nu}$ is the trace of the Ricci tensor 
\begin{equation}
    R_{\mu \nu} = \Gamma^{\rho}_{\mu \nu , \rho} - \Gamma^{\rho}_{\mu \rho, \nu}+ \Gamma_{\mu \nu}^{\sigma}\Gamma_{\rho \sigma}^{\rho} - \Gamma_{\rho \mu}^{\sigma} \Gamma_{\sigma \nu}^{\rho}
\end{equation}
where the Christoffel symbols are given by
\begin{equation}
    \Gamma_{\mu \nu}^{\sigma} = \frac{1}{2} g^{\sigma \rho} (g_{\nu \rho, \mu} + g_{\rho \mu , \nu} - g_{\mu \nu , \rho})
\end{equation}
The second term in eq. \eqref{einsteinhilbert} is the Gibbons-Hawking-York boundary term \cite{York:1972sj, Gibbons:1976ue} and is necessary to properly implement the Dirichlet variational principle on manifolds with a boundary and will be discussed in more details in the next chapter. $K= h^{ij}K_{ij}$ is the trace of the extrinsic curvature 
\begin{equation}
   K_{\alpha \beta} =  h_{\alpha}^{\gamma} h^{\delta}_{\beta} \nabla_\gamma N_{\delta}
\end{equation}
where $N^{\alpha}$ is the normalized vector normal to the boundary $\partial M$ of $M$,  $h_{\alpha  \beta} = g_{\alpha  \beta} - \epsilon N_{\alpha} N_{\beta}$ is the induced metric on the boundary with determinant $h$. The constant $\epsilon= N^{\alpha}N_{\alpha}$ equals $+1$ or $-1$ depending on if the boundary is space-like or time-like.\\
The field equations of general relativity are Einstein's equations, a set of ten partial non-linear differential equations for the ten independent component of the metric tensor $g_{\mu \nu}$, which extremize the Einstein-Hilbert action \ref{einsteinhilbert}:
\begin{equation}
R_{\mu \nu} - \frac{1}{2} R \, g_{\mu \nu}+ \Lambda g_{\mu \nu} = 8 \pi G \, T_{\mu \nu}  \label{einsteinequations}
\end{equation}
where the matter stress-energy tensor $T_{\mu \nu}$ is given by 
\begin{equation}
 T_{\mu \nu} := - 2 \frac{\delta \mathcal{L}_m}{\delta g^{\mu \nu }} + g_{\mu \nu } \mathcal{L}_m
\end{equation}
 %$R_{\mu \nu} = g^{\rho \sigma} R_{\rho \mu \sigma \nu}$ and $R_{\alpha \beta \mu \nu}$ is the Riemann tensor.\\
According to the observations of the CMB radiation, our universe is on large scales mainly homogeneous and isotropic. As a consequence, the four-dimensional line element that describes the geometry of our universe can be taken to be of the FLRW (Friedmann-Lema\^{i}tre-Robertson-Walker) form
\begin{equation}
ds^2 = g_{\mu \nu} dx^\mu dx^\nu = - dt^2 + a(t)^2 \Bigl[ \frac{dr^2}{1 - k r^2} + r^2 d\theta^2 + \sin^2(\theta) d\phi^2 \Bigr] \label{frw} =  - dt^2 + a(t)^2 \Bigl[ h^{RW}_{ij} dx^i dx^j  \Bigr]
\end{equation}
with a suitable choice of coordinates $\{ t, r, \theta , \phi \}$ .\\
The constant $k$ represents the curvature of the spatial surfaces which, for an appropriate choice of units for $r$, takes the values $0, +1, -1$ corresponding to flat, positively curved and negatively curved universes, respectively. The scale factor $a(t)$ encodes how proper distances change with time and thus parameterizes the expansion of the universe.
Under this ansatz,
Einstein's equations determine the dynamics of the universe fixing the scale factor $a(t)$, for a given value of $k$. 
The Ricci tensor non-vanishing components and trace are given by
\begin{align}
    R_{00} &= - 3 \frac{\ddot{a}}{a}\\
    R_{ij} &=  \Bigl( \frac{\ddot{a}}{a} + \frac{2 \dot{a}^2}{a^2} + \frac{2 k }{a^2}\Bigr)\, h_{ij}^{RW}\\  
    R &=  6 \Bigl(\frac{\ddot{a}}{a}+ \frac{\dot{a}^2}{a^2} + \frac{k}{a^2} \Bigr) \label{Ricciscalar}
\end{align}
The macroscopic behaviour of the universe thermal bath is well-described by a sum of homogeneous and isotropic perfect fluids whose stress-energy tensor is given by
\begin{equation}
T_{\mu \nu} = (\rho + p ) u_\mu u_\nu + p g_{\mu \nu} \label{tmunu}
\end{equation}
$u_\mu$ being the fluid velocity, $\rho$ and $p$ its energy density and pressure, respectively.\\
Einstein's equations reduce in this case to the Friedmann's equations
\begin{align}
\Bigl (\frac{\dot{a}}{a}  \Bigr)^2 &= \frac{8 \pi G}{3} \rho - \frac{k}{a^2} \label{einstein1} \\
\frac{\ddot{a}}{a} &= - \frac{4 \pi G}{3} (\rho + 3 p ) \label{einstein2}
\end{align}
Using the first equation, the second can be re-written as the continuity equation 
\begin{equation}
    \dot{\rho} + 3 \frac{\dot{a}}{a} (\rho + p) = 0 \label{einstein3}
\end{equation}
The cosmological fluid can be described by an equation of state $p = \omega \rho$ where $\omega=0$ corresponds to non-relativistic matter, $w=\frac{1}{3}$ to radiation or relativistic matter, $\omega = -1 $ to dark energy.
Equation (\ref{einstein3}) gives the energy density of the fluid as function of the scale factor 
\begin{equation}
    \rho(a) \propto a^{-3(\omega + 1)} \label{energysclae}
\end{equation}
To determine the dynamics of the universe, it is useful to introduce the dimensionless density parameters
\begin{equation}
    \Omega_r = \frac{\rho_{r, 0}}{\rho_c}, \; \; \;   \Omega_m = \frac{\rho_{m, 0}}{\rho_c},  \; \; \;  \Omega_{\Lambda} = \frac{\rho_{\Lambda}}{\rho_c} , \; \; \;  \Omega_k = - \frac{k}{H_0^2}
\end{equation}
where $H := \frac{\dot{a}}{a}$ is the Hubble parameter, $\rho_c = \frac{3 H_0^2}{8 \pi G}$ is the energy density of a flat universe and quantities with subscript ``0" are measured at the current age of the universe $t_0$ with the convention that $a(t_0) = 1$.\\
Equation (\ref{einstein1}) can then be written as 
\begin{equation}
\Bigl(\frac{H}{H_0} \Bigr)^2 = \frac{\Omega_r}{a^4 } + \frac{\Omega_m}{a^3} + \frac{\Omega_k}{a^2} + \Omega_{\Lambda}
\end{equation}
from which we see that the various components dominate the expansion of the universe in different epochs since their energy density scales with different powers of the size of universe.\\
As measured by the PLANCK satellite \cite{Aghanim:2018eyx}, the universe we live in is characterized (with $68\%$ confidence region) by
\begin{equation}
\begin{split}
    \Omega_k &= - 0.001 \pm 0.002\\
    \Omega_\Lambda &= 0.680 \pm 0.013 \\
    \Omega_m &= 0.321 \pm 0.013\\
    \Omega_r &= (9.14 \pm 0.34)10^{-5}\\
    H_0 &= (66.88 \pm 0.92) \mbox{ Km} \, \mbox{s}^{-1} \mbox{Mpc}^{-1}
    \end{split}
\end{equation}
The picture of the $\Lambda\mbox{CDM} $\footnote{ $\Lambda\mbox{CDM} $ stands for a universe filled with dark energy ($\Lambda$) and cold dark matter (CDM).} for cosmology is then that of a flat universe which initially expands dominated by radiation; as the expansions proceeds the matter component comes to dominate and at late times dark energy takes over. \\
Analytic solutions to Einstein's equations for a flat universe in the phases when only one of the components is relevant are given by 
\begin{align}
    a(t) &= t^{\frac{2}{3 (1 + \omega)}} \; \; \; \; \mbox{ for  } \;  \omega \neq  - 1   \label{timescale} \\
    a(t) &=  e^{H t} \; \; \; \;  \; \; \; \, \,  \mbox{  for  }\;  \omega = - 1 
\end{align}
where for $ \omega = - 1 $, the Hubble rate $H := \frac{\dot{a}}{a} = \sqrt{\frac{8 \pi G}{3} \rho_{\Lambda}} = \sqrt{\frac{\Lambda}{3}}$ is constant. Notice that the PLANCK measurements point at a flat universe with experimental error bars which allow for closed and negatively curved cosmologies. For this reason we will discuss in the following FLRW models of all three types.\\
The radiation dominated universe hits the Big Bang singularity at $t=0$ where $a=0$ and the scalar curvature \eqref{Ricciscalar} and energy density of the fluid \eqref{energysclae} diverge.\\
As a final remark let us introduce the concept of comoving particle horizon. For this, it will be useful to rewrite the flat FLRW line element in conformal time
\begin{equation}
    ds^2 = a(\eta)^2 [ - d\eta^2 + dr^2 + r^2 (d \theta^2 + \sin^2 \theta d \phi^2)]
\end{equation}
A null geodesic corresponds to $ds^2 = 0$ and thus radial photon trajectories are given by
\begin{equation}
 dr = \pm d \eta
\end{equation}
Thus the maximum comoving distance light can travel between time $0$ and time $t$ is
\begin{equation}
    \eta = \int_0^t \frac{dt'}{a(t')} = \int_0^{a} d \ln a' \Bigl(\frac{1}{a' H}\Bigr) \label{horizonc}
\end{equation}
This is a causal horizon in the sense that two regions separated by a distance larger than $\eta$ could never have communicated with each other. Both the horizon $\eta$ and the comoving Hubble radius $(a H)^{-1}$ grow with the expansion of the universe in the $\Lambda \mbox{CDM}$ model. This means that the large scales on the CMB, which only recently entered the horizon, could have not been in causal contact with each other when the CMB was emitted or, in other words, that, when we look at the CMB sky, we look at a large number of causally disconnected regions. If we assume, for the sake of the argument and with a tremendous simplification\footnote{The number of causally disconnected regions one gets from the fully accurate calculation is of the same order of magnitude.}, that the universe was matter dominated all the way back to the Big Bang than $\eta \propto a^{1/2}$ and the number of disconnected regions, which goes as the volume, is \begin{equation}
    \Bigl( \frac{r_0}{r_{CMB}} \Bigr)^3 =  \Bigl( \frac{a_0}{a_{CMB}} \Bigr)^{3/2} \approx 35000
    \end{equation} 
where $a_{CMB} \approx 9.08 \, 10^{-4}\, a_0$ according to the PLANCK data.
How come then that these 35000 causally disconnected regions share the same temperature up to a part in $10^5$? \cite{Dodelson:2003ft, Montani:2011zz, Baumann:2009ds} One possible answer to this question is that the universe simply started out in a very special state, homogeneous and isotropic to a very large degree with only tiny primordial deviations from that. The level of fine tuning of these initial conditions required for this explanation to work is however extremely high. That is the reason why cosmologists are after a dynamical explanation, beyond the  
$\Lambda \mbox{CDM}$  model, to explain the observed features of the CMB.





\subsection{Inflation and the cosmological perturbations}

If the comoving Hubble radius $(a H)^{-1}$ was shrinking in the very early universe, the large-angle isotropy of the CMB could possibly be explained without requiring fine-tuned initial conditions: the comoving horizon \eqref{horizonc} would have received its largest contribution at early times so that regions which cannot communicate today (because they are outside each other's Hubble sphere) could have been in causal contact in the past (being inside each other's horizon).\\
It follows from the Friedmann's acceleration equation \eqref{einstein2} that the comoving Hubble radius shrinks in an accelerated universe or, equivalently, in a universe dominated by matter with negative pressure:
\begin{equation}
    \frac{d}{d t} \Bigl( \frac{1}{a H}\Bigr)<0, \; \; \; \ddot{a}>0, \; \; \; \rho + 3 p <0
\end{equation}
According to the theory of inflation \cite{Guth:1980zm,Linde:1981mu,Albrecht:1982wi,Mukhanov:1981xt,Starobinsky:1979ty,Starobinsky:1982ee, Guth:1982ec,Hawking:1982cz,Bardeen:1983qw}, our universe underwent a phase of such accelerated expansion prior to the standard $\Lambda\mbox{CDM}$ story described in the previous section. If this phase lasted long enough, it could prepare our universe to be already extremely flat, homogeneous and isotropic at the beginning of the radiation-dominated phase, potentially answering some of the questions left open by the $\Lambda\mbox{CDM}$ model. Crucially, inflation provides also a framework where the classical gaussian temperature fluctuations of the CMB are generated as primordial quantum fluctuations, as we will show in the next section. It however is important to keep in mind that while inflation can provide the suitable initial conditions for the standard Big Bang cosmology, the question arises of how peculiar the initial conditions of the universe must be for inflation to happen in the first place. If such initial conditions must be highly fine-tuned as well, the question of how our universe found itself in such a special state is simply moved from the beginning of the $\Lambda\mbox{CDM}$ cosmology to the beginning of inflation. This thesis will deal with our understanding of the initial conditions for inflation especially focusing on the semi-classical aspects of the problem. It is worth mentioning also that while accepted by most of the scientific community and in agreement with all current observations, the framework of inflation still lives in the realm of the theory and will be treated as such in this work (see for example \cite{Gibbons:2006pa,Ijjas:2013vea, East:2015ggf,Clough:2016ymm,Marsh:2018fsu} for details and discussions).\\

\subsubsection{De Sitter space}
The simplest model of inflationary universe is provided by de Sitter space which is the solution to vacuum Einstein's equation with a positive cosmological constant.\\
De Sitter space is an Einstein manifold i.e. a pseudo-Riemannian differentiable manifold whose Ricci tensor is proportional to the metric with
\begin{align}
R_{\mu \nu} &= \Lambda g_{\mu \nu}\\
R &= 4 \Lambda
\end{align}
The constant $\textit{l} := \frac{1}{H} =  \sqrt{\frac{3}{\Lambda}}$ has the units of length and is called the ``De Sitter radius''.\\
Introducing the flat five dimensional space $\mathbb{R}^{1,4}$ with metric 
\begin{equation}
ds^2 = - dx_0^2 + dx_1^2 + dx_2^2 + dx_3^2 + dx_4^2  
\end{equation}
de Sitter space can be thought of as the hyperboloid
\begin{equation}
- x_0^2 + x_1^2 + x_2^2 + x_3^2 + x_4^2 = \frac{3}{\Lambda}
\end{equation}
The de Sitter metric is then the metric induced on the hypersurface by the Lorentzian geometry of the five-dimensional Minkowski space.
The four-dimensional de Sitter metric can take the form of all three possible FLRW cosmologies with suitable choices of coordinates.
\begin{figure}[htbp]
\begin{center}
\includegraphics[width = 14 cm]{desitter.pdf}
\caption{The de Sitter hyperboloid with different choices of coordinates. The blue lines represent time-like geodesics. The black lines are hypersurfaces of constant cosmic time. The red, blue and yellow manifolds show the embedding of closed, flat and hyperbolic cosmological models respectively. The figure is taken from \cite{Moschella:2006pkh}.}
\label{fig:desitter}
\end{center}
\end{figure}
\\
Consider the following set of coordinates
\begin{align}
x_0 &= \sqrt{\frac{3}{\Lambda}} \sinh(\sqrt{\frac{\Lambda}{3}} t), \\
x_1 &= \sqrt{\frac{3}{\Lambda}} \cosh(\sqrt{\frac{\Lambda }{3}} t)\, \cos \chi,  \\
x_2 &= \sqrt{\frac{3}{\Lambda}} \cosh(\sqrt{\frac{\Lambda }{3}} t)\, \sin \chi \cos \theta,\\  \;
x_3 &= \sqrt{\frac{3}{\Lambda}} \cosh(\sqrt{\frac{\Lambda }{3}} t)\, \sin \chi \sin \theta \cos \phi, \\
x_4 &= \sqrt{\frac{3}{\Lambda}} \cosh(\sqrt{\frac{\Lambda }{3}} t)\, \sin \chi \sin \theta \sin \phi 
\end{align}
with $  - \infty < t < \infty $, $0 \leq \chi \leq \pi $, $0 \leq \theta \leq \pi$, $0 \leq \phi \leq 2 \pi$, the de Sitter line element describes a closed FLRW model
\begin{equation}
ds^2 = - dt^2 + \frac{3}{\Lambda} \cosh^2 (\sqrt{\frac{\Lambda}{3}} t) [ d \chi^2 + \sin^2 \chi (d \theta^2 + \sin^2 \theta d \phi^2)] 
\end{equation}
where the hypersurfaces of constant time are spheres $\mathbb{S}^3$ and $d \Omega_3 =  d \chi^2 + \sin^2 \chi (d \theta^2 + \sin^2 \theta d \phi^2) $ is the line element of the unit three-sphere. 
\begin{figure}[htbp]
\begin{center}
\includegraphics[width = 7 cm]{desitter33.pdf}
\includegraphics[width = 7 cm]{desitter22.pdf}
\caption{Left panel: The de Sitter space as a closed FLRW model. The hypersurfaces of constant time are spheres given by the intersection of the de Sitter hyperboloid with the hyperplanes $x_0 = const$. Right panel: The de Sitter space as a flat FLRW model. The hypersurfaces of constant time are intersections of the de Sitter hyperboloid with the hyperplanes $x_0 + x_4 = const$. The figures are taken from \cite{Moschella:2006pkh} and $r$ represents the de Sitter radius $l$.}
\label{fig:desitter2}
\end{center}
\end{figure}
\\ One can also introduce the coordinates
\begin{equation}
t = \sqrt{\frac{3}{\Lambda}} \log \Bigl( \frac{ x_0 + x_4}{\sqrt{3/\Lambda}}\Bigr), \; x = \sqrt{\frac{3}{\Lambda}} \frac{x_1}{x_0 + x_4}, \; y = \sqrt{\frac{3}{\Lambda}} \frac{x_2}{x_0 + x_4}, \; z = \sqrt{\frac{3}{\Lambda}} \frac{x_3}{x_0 + x_4}
\end{equation}
In these coordinates the constant time surfaces are copies of $\mathbb{R}^3$ and the De Sitter line element is the analogous of a flat FLRW model
\begin{equation}
ds^2 = - d t^2 + \exp \Bigl(2 \sqrt{\frac{\Lambda}{3} } t \Bigr)(dx^2 + dy^2 + dz^2)
\end{equation}
with $- \infty < t < \infty$.
Note that these coordinates cover only half of the hyperboloid since $t$ is defined only for $x_0 + x_4 = l \, e^{t/l} > 0$. \\
Most often in the literature, de Sitter space in flat slicing is written in terms of conformal time $d \tau = \frac{dt}{a} $ with $\tau \in (- \infty , 0^{-})$. 
\begin{equation}
    ds^2 = a(\tau)^2 [ - d \tau^2  + dx^2 + dy^2 + dz^2], \; \; \; \; \; a(\tau) = - \frac{1}{H \tau}
\end{equation}

In the following we will consider de Sitter space both in flat and closed slicing as toy-models for inflation. It is important to keep in mind that this realization is evidently not realistic because it does not allow inflation to come to an end and it does not generate scalar perturbations, in contradiction with the CMB observations. One can however study tensor perturbations around this background, which are exactly gaussian distributed in this case, to gain a qualitative understanding of many aspects of slow-roll inflation. The de Sitter approximation will greatly simplify the calculations without altering the qualitative analysis and provide a favorable framework for the purposes of our work.    \\



\subsubsection{Scalar field inflation}
Realistic models of inflation are realized by minimally coupling gravity with a scalar field $\phi$, the inflaton, for a suitable choice of the potential $V(\phi)$:
\begin{equation}
 S =  \int d^4 x \sqrt{-g} [ \frac{1}{16  \pi G} R - \frac{1}{2}g^{\mu \nu} \partial_\mu \phi \partial_\nu \phi - V(\phi) ]  + S_B   
\end{equation}
For a homogeneous scalar field $\phi = \phi(t)$ in a flat FLRW universe (eq. (\ref{frw})) the stress-energy tensor is that of a perfect fluid with 
 \begin{align}
     \rho_\phi &= \frac{1}{2}  \dot{\phi}^2 + V(\phi)  \\ 
     p_\phi &= \frac{1}{2}  \dot{\phi}^2 - V(\phi) 
 \end{align}
If the potential energy $V(\phi)$ dominates over the kinetic energy $\frac{1}{2}\dot{\phi}^2$ the scalar field drives an accelerated expansion ($\omega_{\phi} < - \frac{1}{3}$). 
The equations of motion for this system are 
\begin{align}
    &\ddot{\phi } + 3 H \dot{\phi} + V_{, \phi}  = 0 \label{inflation} \\
   & \frac{3}{8 \pi G} H^2 = \frac{\dot{\phi}^2}{2} + V(\phi)
\end{align}
from which it follows that
\begin{equation}
    \frac{\ddot{a}}{a} = H^2 (1 - \epsilon) \; \; \mbox{  with  } \epsilon := \frac{3}{2} (1 + \omega_{\phi}) = 8 \pi G  \frac{\dot{\phi}^2}{2 H^2}
\end{equation}
With a suitable choice of the potential it is then possible to build a dynamical system in which the expansion of the universe is accelerated as long as the slow-roll parameter $\epsilon< 1$ and where inflation comes to an end when this condition fails to be satisfied ($\epsilon(\phi_{end})=1$). \\
Note that de Sitter spacetime represents the ``no roll" limit $\epsilon= const = 0$.\\ 
For inflation to last long enough it is also required that
\begin{equation}
|\ddot{\phi}| \ll |3 H \dot{\phi}|, |V_{, \phi}|
\end{equation}
and thus that
\begin{equation}
    |\eta| := | - \frac{\ddot{\phi}}{H \dot{\phi}}| < 1
\end{equation}
The slow-roll regime is realized when 
\begin{equation}
    \epsilon \ll 1 \; \; \; \; |\eta| \ll 1
\end{equation}
in which case the slow-roll parameter is approximated by
\begin{equation}
    \epsilon \approx \frac{1}{16 \pi G} \Bigl( \frac{V_{, \phi}}{V} \Bigr)^2
\end{equation}
We then learn that slow-roll inflation can be realized if the inflaton potential is sufficiently flat.

%%%%%%%%%%%%%%%%%%%%%%%%%%%%%%%%%%%%%%%%%%%%%%%%%%

\subsubsection{Cosmological perturbation theory}\label{sec:QFT}

The Universe we observe today is homogeneous and isotropic on scales larger than about 300 Mly \cite{Scrimgeour:2012wt}. As a first approximation, many phenomena of physical interest can be described by FLRW type of metrics (\ref{frw}). However, the CMB tells us that this is not the whole story. At the time of the decoupling our universe was nearly homogeneous and isotropic with small inhomogeneities.
Indeed the testability of cosmological models relies mostly on the prediction of the correct amount of these inhomogeneities. The biggest success of the theory inflation is to provide a mechanism through which they can be generated. In the following we will describe the primordial fluctuations during inflation and their quantization within the framework of QFT in curved spacetime using the cosmological perturbations theory \cite{Mukhanov:1990me}. \\
Let us consider small generic perturbations $ \delta g_{\mu \nu}(t , \mathbf{x})$ around a homogeneous and isotropic background $g_{\mu \nu}^0(t)$
\begin{equation}
g_{\mu \nu} (t, \mathbf{x}) = g_{\mu \nu}^0 (t) + \delta g_{\mu \nu}(t , \mathbf{x})
\end{equation}
associated with the line element 
\begin{equation}
ds^2 = g_{\mu \nu} dx^\mu dx^\nu = - N^2(1 + 2A) dt^2 + 2 a B_i dt dx^i + a^2 [(1 - \psi) h_{ij} + E_{ij}]dx^i dx^j \label{metri}
\end{equation}
In a spacetime filled with other homogeneous fields $\phi(t)$ they get perturbed too $\phi(t, \mathbf{x}) = \phi(t) + \delta \phi(t, \mathbf{x}) $ as a consequence to Einstein's equations. \\
%We will not investigate any further these aspects of theory since we are studying pure (empty) De Sitter space. \\
We will deal with first order perturbations whose dynamics is given by Einstein's equation linearized around the background where all the terms of second order in perturbations are neglected. Their action is given by an expansion to the second order of the Einstein-Hilbert action for gravity. In this case, the treatment of perturbations is rather simplified by the properties of symmetry of the homogeneous and isotropic background. The translation invariance of the linear equations of motion for the perturbations allows one to work in Fourier space rather than in real space (see \cite{Baumann:2009ds} for further details). The Fourier modes of the perturbations do not interact and can be treated independently. Moreover, the metric and stress-energy perturbations can be the decomposed in their scalar, vector and tensor components (SVT decomposition), according to their helicity. Given a wave vector $\mathbf{k}$, a rotation by a angle $\theta $ around it effects a perturbation of helicity \textit{m} by a dilatation of his amplitude by a factor $e^{i m \theta}$. 
Helicity scalars are then defined to have $m=0$, vectors correspond to $m = \pm 1$, tensors to $m = \pm 2$. The rotational invariance of the FLRW background implies that the three different sectors evolve independently and thus we can treat them separately. \\
The SVT decomposition applies to the elements $E_{ij}$ and $B_i$ of the metric (\ref{metri}) and allows us write them as follows
\begin{equation}
B_i \equiv \partial_i B  - S_i \, \, \mbox{where} \, \, \partial^i S_i = 0
\end{equation}
\begin{equation}
E_{ij} \equiv 2 \partial_{ij} E + 2 \partial_{(i} F_{j)} + \eta_{ij} \, \, \mbox{where} \, \, \partial^i F_{i} =0 ,  \,\, \eta^i_i= \partial^i \eta_{ij} = 0  \label{svt}
\end{equation}
The scalar sector of the perturbations is described by the four scalars A, $\psi$, B, E and is related to the temperature fluctuations of the CMB. The tensor components ($\eta_{ij}$) can be observed as gravitational waves. The vector fluctuations, given by $S_i$ and $F_i$, always decay during the expansion of the universe and are usually ignored in cosmology.\\
While tensor perturbations are gauge invariant, the four functions A,$\phi$, B,E are affected by changes of coordinates.\\
Under the transformation
\begin{align}
t &\rightarrow t + \alpha \\
x^i &\rightarrow x^i + \delta^{ij} \beta_{, j } 
\end{align}
the metric scalar perturbations transform as
\begin{align}
A &\rightarrow A - \dot{\alpha}\\
B &\rightarrow B + a^{-1} \alpha - a \dot{\beta} \\
E &\rightarrow E - \beta \\
\psi &\rightarrow \psi + \frac{\dot{a}}{a} \alpha
\end{align}
and the scalar field perturbation transforms as $\delta\phi \to \delta\phi - \dot\phi \alpha.$
The two functions $\alpha$ and $\beta$ can be chosen in such a way that two scalar degrees of freedom vanish, say $E$ and $\psi$. At linear order the constraints, which can be thought of as the $00$ and $0i$ Einstein's equations, are given by (see e.g. \cite{Koehn:2015vvy})
\begin{align}
A=& \frac{\dot\phi}{2 H}\,\delta\phi  = \sqrt{\frac{\epsilon}{2}} \delta \phi \label{eq:alpha}\\
\partial^i \partial_i B =& -\frac{1}{2 H}(V_{,\phi}+\frac{\dot\phi}{H}V)\delta\phi-\frac{\dot\phi}{2 H} \dot{\delta\phi} = -\epsilon \frac{\mathrm{d}}{\mathrm{d}t}\left( \frac{\delta\phi}{\sqrt{2\epsilon}}\right),
\end{align} 
where in the constraint for $B$ we have already used \eqref{eq:alpha} to replace $A.$  In pure de Sitter space, where $\epsilon=0$, scalar perturbations are thus forced to vanish. Said differently, there are no scalar degrees of freedom in an empty de Sitter spacetime and thus no scalar perturbations can be generated. Gravitational waves can instead always be produced and will be the subject of our study.


%%%%%%%%%%%%%%%%%%%%%%%%%%%%%%%%

\subsubsection{Tensor perturbations} 
For tensor perturbations in a flat de Sitter universe, it is conventional to define the following Fourier expansion
\begin{equation}
\eta_{ij} (\eta, \mathbf{x})=\int \frac{d^3 k}{(2 \pi)^3} \sum_{s =+, \times} \epsilon_{ij}^s (\mathbf{k}) \phi_{\mathbf{k}}^s(\eta) e^{i \mathbf{k \cdot x}} 
\end{equation}
where the polarization tensor satisfies $\epsilon_{ii} = k^i \epsilon_{ij} = 0$ and $\epsilon^s_{ij}(\mathbf{k}) \epsilon^{s'}_{ij}(\mathbf{k}) = 2 \, \delta_{s s'}$.\\
Given that different Fourier modes do not interact with each other at linear order, the study of tensor perturbations reduces to the study of a single scalar $\phi_{\mathbf{k}}^s(\eta)$ of wave number $k$ and polarization $s$, where the overall sum over all modes will be kept implicit\footnote{For simplicity of notation, we will mostly use the notation $\phi(\eta)$ rather than $\phi_k^s(\eta)$}. It will also be useful to introduce the canonically normalized field 
\begin{equation}
v_{\mathbf{k}}^s = \frac{a}{2 \sqrt{8 \pi G}} \; \phi_{\mathbf{k}}^s \label{defcanvar}
\end{equation}
With this construction the second variation of the Einstein-Hilbert action (\ref{einsteinhilbert}) can be written in conformal time as follows
\begin{equation}
\delta_2 S =\frac{1}{2} \sum_s \int d \eta \, d^3 k \, \Bigl[(v_{\mathbf{k}}^{'s} )^2 - \Bigl(k^2 - \frac{a''}{a}\Bigr) (v_{\mathbf{k}}^s)^2 \Bigr]
\end{equation}
Therefore the action for a single mode of wave number $k$ and a given polarization is
\begin{equation}
S^{(2)} = \frac{1}{2}  \int d \eta \, \Bigl[(v_k^{'s} )^2 - \Bigl(k^2 - \frac{a''}{a}\Bigr) (v_k^s)^2 \Bigr] \label{actionperturbations}
\end{equation}
where $\frac{a''}{a} = \frac{2}{\eta^2}$ in de Sitter space.\\
When dealing with the closed FLRW model, we consider the harmonic expansion in terms of the normalised eigenfunction $Q_{nlm }$ of the Laplacian operator
\begin{align}
\Delta Q_{n l m } &= - n(n +2) Q_{nlm}\\
\int d^3x \sqrt{\eta} \, Q_{nlm}(x) Q_{n'l'm'} (x) &= \delta_{n n'} \, \delta_{l l'} \, \delta_{m m'}
\end{align}
where $n$ is an integer $n\geq 2$.\\
The expansion in modes reads
\begin{equation}
\phi(\eta, \mathbf{x}) = \sum_{nlm} \phi_{nlm}(\eta) Q_{nlm}(\mathbf{x})  
\end{equation}
In this case the action for a single canonically normalized mode takes the form
\begin{equation}
S^{(2)} =\frac{1}{2} \int d \eta \, \Bigl[(\phi_k^{'} )^2 - \Bigl( n(n+2) - \frac{a''}{a}\Bigr) \phi_n^2 \Bigr] \label{pertk1}
\end{equation}		
where we dropped the $l$ and $m$ subscripts.\\
We can recognize in eq. \eqref{actionperturbations} and \eqref{pertk1} the action of a harmonic oscillator with a time-dependent frequency. The quantization of cosmological perturbations is thus simply the quantization of a series of independent harmonic oscillators.

%%%%%%%%%%%%%%%%%%%%%%%%%%%%%%%%%%%%%%%%%%%%%%%%%%%%%%%%%%%%%%%

\subsubsection{Quantization and the Bunch-Davies vacuum}\label{sec:QFT}
Perturbations during inflation are assumed to start out in their vacuum state. The question is: which vacuum? 
In quantum field theory in Minkowski spacetime there is a unique vacuum state defined as state over which the expectation value of Hamiltonian is minimized. If the spacetime evolves with time, this definition gives a vacuum state which depends on the time at which 
the expectation value is calculated. Hence the ground state at time $t_0$ might not be the state of lowest energy at the time $t_1$. Said differently, for a quantum field theory in de Sitter space there exists an entire class of quantum states which are invariant under de Sitter isometries known as $\alpha$-vacua. This leads to an ambiguity in the choice of the vacuum state which is usually solved by identifying the Bunch-Davies vacuum \cite{Bunch:1978yq} as the preferred one.
In what follows we are going to review the quantization of cosmological perturbations in flat space and introduce the Bunch-Davies vacuum in the Schr\"{o}dinger picture \cite{Martin:2012pea, Brizuela:2015tzl, Lehners:2020pem}. \\
We start by promoting the perturbative modes $v_{\mathbf{k}}$ and their conjugate momenta $\pi_{\mathbf{k}} = v_{\mathbf{k}}^{'} $ to quantum operators  $\hat{v}_{\mathbf{k}}$, $ \hat{\pi}_{\mathbf{k}} $ with 
\begin{align}
    [\hat{v}_{\mathbf{k}}, \hat{\pi}_{\mathbf{p}}] &= i \, \delta^{(3)} (\mathbf{k} - \mathbf{p}) \\
    \hat{v}_{\mathbf{k}} \Psi = v_{\mathbf{k}} \Psi,&  \; \; \; \; \;   \hat{\pi}_{\mathbf{p}} \Psi = - i \frac{\partial \Psi}{\partial v_{\mathbf{k}}}
\end{align}
The wavefunction $\Psi_{\mathbf{k}} (\eta, v_{\mathbf{k}}) $ satisfies the Schr\"{o}dinger equation 
\begin{equation}
    i \frac{\partial \Psi_{\mathbf{k}}}{\partial \eta} = \mbox{H}_{\mathbf{k}} \Psi_{\mathbf{k}} \label{schroe}
\end{equation}
where the quantum Hamiltonian which follows from the action (\ref{actionperturbations}) is
\begin{equation}
    \mbox{H}_{\mathbf{k}} = - \frac{1}{2} \frac{\partial^2}{\partial v_{\mathbf{k}}^2} + \frac{1}{2} \left(k^2 - \frac{2}{\eta^2} \right)  v_{\mathbf{k}}^2
\end{equation}
for each mode.\\
Since we are after the ground state of a harmonic oscillator, we can solve the Schr\"{o}dinger equation using a gaussian ansatz
\begin{equation}
    \Psi_{\mathbf{k}} = N_{\mathbf{k}} (\eta) \, e^{- \frac{1}{2} \Omega_{\mathbf{k}} (\eta) v_{\mathbf{k}}^2 } \label{answave}
\end{equation}
Plugging this ansatz into eq. \eqref{schroe} we obtain the two equations
\begin{align}
    i\, \dot{N}_{\mathbf{k}} &= \frac{1}{2} N_{\mathbf{k}} \Omega_{\mathbf{k}} \\
    i \, \dot{\Omega}_{\mathbf{k}} &= \Omega_{\mathbf{k}}^2 -  \left(k^2 - \frac{2}{\eta^2} \right) \label{eqpert2}
\end{align}
where the first simply fixes the normalization factor. To solve the second equation we change variable to 
\begin{equation}
    \Omega_{\mathbf{k}} = - i \frac{f^{*'}_{\mathbf{k}}}{f_{\mathbf{k}}^{*}} \label{ansatzpert}
\end{equation}
The $f_{\mathbf{k}}^{*}$ are simply the complex conjugate of the mode functions one encounters in the more standard quantization of cosmological quantization in the Heisenberg picture. In fact, with this ansatz, eq. \eqref{eqpert2}
becomes the familiar Mukhanov-Sasaki equation
\begin{equation}
f_k^{\prime\prime} + \left(k^2 - \frac{2}{\eta^2} \right) f_k = 0\,,
\end{equation}
which admits two linearly independent solutions, one of positive and one of negative frequency
\begin{equation}
f_k = c_1\, e^{-ik\eta}\left(1 - \frac{i}{k\eta} \right) + c_2\, e^{+ik\eta}\left(1 + \frac{i}{k\eta} \right)\,, \label{eq:BD} 
\end{equation}
The ambiguity in the definition of the vacuum in de Sitter lies in the freedom in the choice of the coefficients $c_{1}$ and $c_2$. The standard choice is to select the Bunch-Davies vacuum noticing that in the far past, \textit{i.e.} in the limit $\eta \rightarrow - \infty,$ the equation of motion becomes that of a fluctuation in Minkowski spacetime. Since the stable positive frequency solution to the wave equation in Minkowski spacetime is of the form $v_k \propto e^{-ik\eta}$, it is usually argued that the mode functions should satisfy
\begin{equation}
\lim_{\eta \to -\infty} f_k(\eta) = \sqrt{\frac{1}{2k}} e^{-ik\eta}
\end{equation}
which leads one to set $c_2=0$ so that one obtains the Bunch-Davies vacuum \cite{Bunch:1978yq}\footnote{A number of cosmologists have pointed out the dangers of this assumption in the past, see in particular the description of the trans-Planckian problem in \cite{Martin:2000xs}.}
\begin{equation}
f_k = \sqrt{\frac{1}{2k}} e^{-ik\eta}\left(1 - \frac{i}{k\eta} \right) \,.
\end{equation}
Physically this is related to the fact that in the far past all the modes of astrophysical interest today had a physical wavelength smaller than the Hubble radius. This allows one to impose the initial condition when
$\eta \rightarrow - \infty$ or $|k\eta| \gg 1$ or $k \gg a \mbox{H}$ and the curvature of the spacetime is not felt and thus it becomes a physical requirement that the mode functions should limit to the Minkowski vacuum solutions. \\
With this choice we obtain that $\Omega_{\mathbf{k}}$ is given by
\begin{equation}
    \Omega_{\mathbf{k}} =+ \frac{k^3 \eta^2}{1 + k^2 \eta^2} + \frac{i}{\eta (1 + k^2 \eta^2)}
\end{equation}
At early times, $ |k \eta| \gg 1$, the wavefunction \eqref{answave} resembles the gaussian ground state of an ordinary of oscillator in Minskowsi spacetime with $\Omega_{\mathbf{k}} \approx k$. Had we chosen the complex conjugate mode we would have obtained a minus sign in front of the real part of  $\Omega_{\mathbf{k}}$ resulting in a nonsensical inverse gaussian distribution for the wavefunction. While at early times the wavefunction is real, at late times, when $ |k \eta| \ll 1$, it becomes increasingly oscillatory with the real part of $\Omega_{\mathbf{k}}$ shrinks to zero. The transition happens at the horizon exit, when $|k \eta| = 1$. We will see in section \ref{classicality} that a system can be said to behave classically in the WKB sense when the phase of its wave function varies rapidly as compared to its amplitude. We thus learn that primordial perturbations become classical at the horizon exit.\\    
The power spectrum of the $v_{\mathbf{k}}$ is then given by \cite{Martin:2012pea} 
\begin{equation}
    P(k) := \frac{k^3}{2 \pi^2} \frac{1}{2 \mathfrak{Re}(\Omega_{\mathbf{k}})}
\end{equation}
with $\mathfrak{Re}(\Omega_{\mathbf{k}})$ evaluated at late times $|k \eta| \ll 1$ . Plugging in the definition of the canonical variable $v_{\mathbf{k}}$ \eqref{defcanvar}, we obtain the standard result for the power spectrum of tensor perturbations \cite{Peter:2013avv}
\begin{equation}
    P_T := 2 \Bigl( \frac{32 \pi G}{a^2}\Bigr)  \frac{k^3}{2 \pi^2} \frac{1}{2 k^3 \eta^2} = \frac{16 G H^2}{\pi}
\end{equation}
where the factor of 2 in front comes from the fact that each tensor mode has two polarizations and we used $a \, \eta = - H^{-1}$.\\
The Bunch-Davies vacuum quantum fluctuations are amplified into a Gaussian distribution of late-time fluctuations $v_k/a$ that reach a constant value  on super-horizon scales and exhibit a scale-invariant spectrum ($|v_k/a|^2 \propto \hbar \,H^2/k^3$). Hence, according to this argument, primordial perturbations with the correct features are naturally produced in inflation and thus a universe which starts out in an initial (quasi-) de Sitter inflationary phase nicely matches current observations. It is important to note that within this framework of QFT in curved spacetime, the early universe is treated as a set of quantum harmonic oscillators on a classical spacetime. We will challenge this framework in chapter \ref{quantuminitial} allowing for quantum properties of the background spacetime and we will see that the Bunch-Davies vacuum is not recovered unless extra ingredients are introduced because the background quantum effects force the choice of the bad behaved mode, complex conjugate to the Bunch-Davies.

%%%%%%%%%%%%%%%%%%%%%%%%%%%%%%%%%%%%%%%%%%%%%%%%%%%%%%%%%%%%%%%%%%%%%%%%%%%%%%%%%%%%



\subsubsection{Eternal inflation}\label{eternalintro}

The usual description of inflationary fluctuations uses the framework of quantum field theory (QFT) in curved spacetime described above, in which quantum fluctuations are superimposed on a classical background spacetime. Even for large fluctuations, such as those envisioned during a regime of eternal inflation~\cite{Vilenkin:1983xq, Guth:2000ka, Linde:1994gy}\footnote{Our discussion of eternal inflation here and in section \ref{papersebastian} will be largely based on \cite{Bramberger:2019zks}. To keep the same notation of the reference in this section we use units such that $8 \pi G =1$.}, this framework is frequently used \cite{Steinhardt:1982kg,Vilenkin:1983xq}. 
Within this regime the quantum fluctuations of the inflaton become comparable to the field displacement due to the classical evolution. In this case, in certain patches of the universe the inflaton might be effectively jumping up its potential, instead of classically rolling down, giving new fuel to the inflationary evolution. If this regime was reached in the early universe, inflation might have ended in the part of the universe we live in but would be still happening somewhere else becoming, at the global level, eternal.\\
 The Einstein's constraint equations (\ref{eq:alpha}) show that when the slow-roll parameter is very small, $\epsilon \ll 1,$ the metric perturbations are negligible compared to the scalar field fluctuations $\delta\phi$ since they are suppressed by factors of $\sqrt{\epsilon}.$ This is the basis for the standard intuition that in slow-roll inflation one may think of the background spacetime as being constant, with only the scalar field fluctuating.

This picture is reinforced by the fact that at cubic order in interactions, up to a numerical factor of order one the leading contribution in the Lagrangian is a term of the form $\sqrt{\epsilon}(\dot{\delta\phi})^2\delta\phi,$ which is also small in the slow-roll limit. Hence, in the presence of a very flat potential, the system is perturbative. In other words, to a first approximation the system is described by free scalar field fluctuations in a fixed geometry. 

In flat gauge the comoving curvature perturbation is given by $\mathcal{R} = \psi - \frac{H}{\dot \phi}\delta  \phi = - \frac{H}{\dot \phi}\delta  \phi \approx -\frac{1}{\sqrt{2\epsilon}} \delta\phi.$ An analogous calculation to the one presented in the previous section shows that inflation amplifies scalar quantum fluctuations and induces a variance of the curvature perturbation which on super-Hubble scales and in the slow-roll limit is given by \cite{Mukhanov:1981xt,Guth:1982ec,Hawking:1982cz,Bardeen:1983qw}
\begin{align}
\Delta_{\mathcal{R}}^2 = \frac{H^2}{8\pi^2 \epsilon}\,.
\end{align} 
The relation between the curvature perturbation and the scalar field perturbation then implies that the variance of the scalar field is given by 
\begin{equation}
\Delta\phi_{qu} \equiv \langle (\delta \phi)^2 \rangle^{1/2} = \frac{H}{2\pi}\,.
\end{equation}
This is the typical quantum induced change in the scalar field value during one Hubble time. By comparison, the classical rolling of the scalar field during the same time interval induces a change
\begin{equation}
\Delta\phi_{cl} \equiv \frac{|\dot\phi|}{H} 
\end{equation}
Note that the quantum change dominates over the classical rolling when 
\begin{equation}
\Delta\phi_{qu} > \Delta\phi_{cl} \quad\leftrightarrow \quad\frac{H^2}{2\pi |\dot\phi|} \approx \frac{H}{\sqrt{8\pi^2 \epsilon}}>1 \quad\leftrightarrow\quad \Delta_{\mathcal{R}}^2 > 1\,,
\end{equation}
i.e. precisely when the variance of the curvature perturbation is larger than one, and when perturbation theory becomes questionable. In this regime inflation is thought to be eternal, and this leads to severe paradoxes in its interpretation \cite{Ijjas:2014nta, Guth:2000ka, Linde:1994gy}. The entire framework of inflation might in fact get in trouble because the eternal possibility: if an inflating region of spacetime enters the eternal regime, pocket universes, which expand extremely fast, are created as the scalar field jumps up the potential. Within each such pocket universe, new pocket universes are created in a process which goes on forever. We thus end up with a picture of infinitely many disconnected universes, each with possibly different physics and predictions for cosmology. Since the eternal regime is not yet well understood, inflation is in danger of loosing its predictive power. One motivation of this work is to verify the intuitions from QFT in curved spacetime: does quantum cosmology, where the scale factor of the universe is also quantized, support the view that the scalar field fluctuations evolve in a fixed background spacetime? Does this picture become better or worse as the potential becomes flatter? Is there a qualitative difference between the eternal and non-eternal regimes?





%%%%%%%%%%%%%%%%%%%%%%%%%%%%%%%%%%%%%%%%%%%%%%%%%%%%%%%%%%%%%%%%%%%%%%%%%%%%%%%%%%%%%%%%%%%%%%%%%%%%%%%%%%%%%%%%%%%%%%%%%%%%%%%%%%%%



\end{document}