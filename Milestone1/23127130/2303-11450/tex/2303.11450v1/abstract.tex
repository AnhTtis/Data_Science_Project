\documentclass[./AATesi.tex]{subfiles}


\definechangesauthor[name=Alice, color=blue]{ADT}

\begin{document}

\section*{Abstract}

%1 page abstract in English ad German

%the early universe cosmology qft in curved space time. this deicate limit not well understood. we calculate explicitly the saddle point approximation of integrals in the minisuperspace approximation. directly for cosmology, no boundary proposal. Somewhat indirectly studying negative lambda to connect with holography. Our main finding is that neumann works for everyone and is a regularity condition.

The very early universe is successfully described by quantum field theory in curved spacetime where the classical background spacetime is typically an FLRW cosmology and the quantum fields which propagate on it include gravitational waves and energy density fluctuations.
This regime is however little understood from a theoretical point of view because part of the gravitational degrees of freedom, and only part of them, are quantized. In this work we study this limit by assigning quantum properties both to the background universe and the fluctuations and then focusing on the limit where the background universe behaves nearly classically. The quantization is realized in the framework of quantum general relativity through Feymann's path integrals.
We study the saddle point approximation of gravitational path integrals in the cases of a positive and a negative cosmological constant making use of the minisuperspace approximation.
Our first findings are two important negative results concerning path integrals with Dirichlet boundary conditions: inflation does not allow for Bunch-Davies initial conditions if no pre-inflationary phase is admitted and the no boundary proposal is ill-defined as a sum of regular geometries which start at zero size. This motivates us to study the impact on the path integral of other classes of boundary conditions such as those of Neumann and Robin types. 
We find that Robin types of boundary conditions can be used to reconciliate inflation with the Bunch-Davies initial condition and are also useful to describe large homogeneous scalar field fluctuations in the eternal inflation regime.
Our main finding is that, for both the no boundary proposal and black holes in Euclidean anti-de Sitter space, the path integral needs to be defined with Neumann initial conditions. The Neumann condition is in fact necessary to recover sensible black holes thermodynamics and to stabilize the no boundary proposal. At the same time, it can be seen, in both cases, as a regularity requirement on the geometries entering the sum. The need for Neumann conditions implies that the interpretation of the no boundary wavefunction is very different from Hartle and Hawking's original intuition, since the initial expansion rate of the universe is fixed rather than its size. Our results for black holes stands in support of this implementation of the no boundary proposal, where regularity is the primary requirement, and allows for a well-defined QFT in curved spacetime limit. Moreover, in the case of black holes, we find that when the asymptotic AdS spacetime is cut off at a finite radius additional saddle points contribute to the path integral. The possibility of testing this result in the dual picture gives an element of falsifiability to the minisuperspace approximation, crucial for the reliability of the entire paradigm.



\end{document}