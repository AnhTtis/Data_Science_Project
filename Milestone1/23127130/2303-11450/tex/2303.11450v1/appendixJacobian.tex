\documentclass[./AATesi.tex]{subfiles}


%\definechangesauthor[name=Alice, color=blue]{ADT}

\begin{document}



\section{Fluctuation determinant for mixed Neumann-Dirichlet boundary conditions} \label{sec:determinant}

In evaluating our path integrals, we could make use of the fact that the actions were quadratic in the scale factors, thus allowing a decomposition of the path into a classical solution $\bar{q}$ and a fluctuation $Q$, i.e. $q(r) = \bar{q}(r) + Q(r),$ with the resulting path integral over $Q$ being of Gaussian form,
\begin{align}
F(N) = \int_{\dot{Q}(0)=0}^{Q(1)=0} D[Q] e^{i\int_0^1 dr \frac{\dot{Q}^2}{N}}\,,    
\end{align}
where we have neglected an unimportant numerical factor in the exponent. To ensure that the total scale factor $q$ satisfies the mixed Neumann-Dirichlet boundary conditions, the fluctuation must satisfy $\dot{Q}(0)=0$ and $Q(1)=0.$ Here we would like to determine the dependence of the above integral on the lapse $N.$ To do so, we will use a re-scaled coordinate $\tilde{r}=rN,$ with range $0 \leq \tilde{r} \leq N.$ The integral then becomes 
\begin{align}
F(N) &= \int_{Q_{,\tilde{r}}(0)=0}^{Q(1)=0} D[Q] e^{i\int_0^N d\tilde{r} {Q}_{,\tilde{r}}^2} \nonumber \\ &= \int_{Q_{,\tilde{r}}(0)=0}^{Q(1)=0} D[Q] e^{- i\int_0^N {Q}_{,\tilde{r}} \frac{d^2}{d\tilde{r}^2} {Q}_{,\tilde{r}}} = \sqrt{\frac{2}{\pi i}}\left[\text{det}\left(-\frac{d^2}{d\tilde{r}^2}\right)\right]^{-1/2}\,.
\end{align}
With the assumed boundary conditions, the operator $-\frac{d^2}{d\tilde{r}^2}$ satisfies the eigenvalue equation $-\frac{d^2}{d\tilde{r}^2} x_n = \lambda_n x_n$ with eigenfunctions $x_n$ and eigenvalues $\lambda_n,$
\begin{align}
    x_n = a_n \cos\left[\frac{(2n+1)\pi}{2N}\tilde{r} \right]\,, \quad \lambda_n = \left[\frac{(2n+1)\pi}{2N}\right]^2\,, \quad n \in \mathbb{N}\,.
\end{align}
The determinant is given by the product of all eigenvalues. We can evaluate it using zeta function regularisation (see e.g. Ref.~\cite{Grosche:1998yu}). Thus in analogy with the zeta function $\zeta(s)=\sum_{n \in \mathbb{N}}n^{-s}$ we define
\begin{align}
    \zeta_\lambda(s) \equiv \sum_{n \in \mathbb{N}} \lambda_n^{-s} = \left(\frac{2N}{\pi} \right)^{2s}\sum_{n\in \mathbb{N}} \frac{1}{(2n+1)^{2s}}\,.
\end{align}
The last term corresponds to the zeta function where one would sum only over odd terms. We can obtain this sum by subtracting the even terms,
\begin{align}
    1+\frac{1}{3^{2s}}+\frac{1}{5^{2s}}+\cdots & = 1+\frac{1}{2^{2s}}+\frac{1}{3^{2s}}+\cdots - [\frac{1}{2^{2s}}+\frac{1}{4^{2s}}+\cdots] \nonumber \\ & = 1+\frac{1}{2^{2s}}+\frac{1}{3^{2s}}+\cdots - \frac{1}{2^{2s}} [1+ \frac{1}{2^{2s}}+\frac{1}{3^{2s}}+\cdots]\,.
\end{align}
Hence we obtain 
\begin{align}
    \zeta_\lambda(s) = \left(\frac{2N}{\pi} \right)^{2s} \left( 1 - 2^{-2s}\right)\zeta(s)\,.
\end{align}
The zeta function can be analytically continued to $s=0,$ where the derivative $\zeta_\lambda^\prime(0)$ is related to the product of all $\lambda_n$ such that
\begin{align}
    \left[\text{det}\left(-\frac{d^2}{d\tilde{r}^2}\right)\right] = e^{-\zeta_\lambda^\prime(0)}=2\,,
\end{align}
where we have made use of $\zeta(0)=-\frac{1}{2}.$ In the end we find the remarkably simple result that 
\begin{align}
    F(N) = \frac{1}{\sqrt{\pi \, i}}\,.
\end{align}
In particular, note that the fluctuation determinant for the Neumann-Dirichlet problem does not contain any dependence on the lapse $N,$ unlike in the well known pure Dirichlet case where the determinant is proportional to $N^{-1/2}$ \cite{Grosche:1998yu}.


\clearpage


\end{document}