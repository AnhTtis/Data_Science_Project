\documentclass[./AATesi.tex]{subfiles}

%\usepackage{changes}
%\definechangesauthor[name=Alice, color=blue]{ADT}

\begin{document}
\setcounter{page}{1}
In a classical world, there is only one real trajectory which links two events: the one that minimizes the action. In the quantum world, the probability for the transition is associated with the sum over all possible paths linking the two boundary configurations. 
%Each such path is weighted with a factor of $e^{i S/\hbar}$ which depends on the action $S$ evaluated along the path and the probability of the transition is given by the modulus of the path integral.
In the semi-classical limit the full sum is well approximated by a single trajectory, a solution to the classical equations of motion. If the boundary conditions are classically forbidden, the solution will be complex giving a quantum weighting to the transition.  What happens if we apply these concepts to the entire universe? Very large progress in understanding the semi-classical properties of gravity has been done by looking at the contribution of the dominant saddle point of gravitational path integrals. We talk of semi-classical gravity as the regime where part of the system is well described by the saddle point approximation of the path integral while the rest is not. The classical part of the system provides a notion of classical spacetime where the quantum part of the system lives and quantum field theory (QFT) in curved spacetime applies. 
%In particular we talk of a semi-classical limit when part of the system behaves classically while the rest doesn't. In this case the rest of the world could be use to make the background classical? cite Kiefer?. This is a very important limit?, because it is the one where .. and the quantumness of the system is there because the solution to the classical equations of motion can be complex. ?\todo{levare sta frase?}
This is the case for example of the no boundary \cite{Hawking:1981gb, Hartle:1983ai, Hawking:1983hj, Hartle:2008ng} and tunneling \cite{Vilenkin:1982de, Vilenkin:1983xq, Vilenkin:1984wp, Vilenkin:1986cy} proposals in cosmology and the Hawking-Page phase transition with the associated thermodynamic properties for what concerns black holes \cite{Hawking:1982dh, Witten:1998zw, Hawking:1974rv, Hawking:1974sw, Bardeen:1973gs, Hartle:1976tp, Bousso:1996au}. Another notable framework where these concepts are truly foundational is that of the AdS/CFT correspondence~\cite{Maldacena:1997re,Gubser:1998bc,Witten:1998qj}, where the saddle point approximation of the gravity path integral gives the partition function of the dual quantum field theory, in the appropriate limit. The holographic principle is by now widely applied in cosmology too \cite{Hertog:2004rz, Hertog:2005hu, Harlow:2010az, Maldacena:2010un, McFadden:2009fg, Casalderrey-Solana:2020vls, Hertog:2011ky}.  In all these cases there is an open question how the full integral is defined and under what conditions the sum over all paths is well approximated by a specific saddle point contribution. In this thesis we investigate such questions with particular focus on fundamental issues which find their setting in cosmology.
This type of questions are in fact crucial in cosmology since our current understanding of the very early universe is based on the treatment of background and perturbative gravitational degrees of freedom on a different footing within the semi-classical framework of QFT in curved space time~\cite{Mukhanov:1981xt,Starobinsky:1979ty,Starobinsky:1982ee, Guth:1982ec,Hawking:1982cz,Bardeen:1983qw, Lehners:2007ac, Khoury:2001wf, Khoury:2001zk, Gratton:2003pe, Boyle:2004gv}. The idea of our work is to test the validity of this assumption allowing for quantum properties of the background universe in the most conservative manner. We include in the path integral a sum over background geometries with a weight which depends on the action of general relativity only. We are after a systematic study of gravity path integrals within the minisuperspace approximation where we can handle the calculations in full detail~\cite{Halliwell:1988wc, Halliwell:1988ik, Halliwell:1989vu, Halliwell:1990tu, Kiefer:1990ms, Feldbrugge:2017kzv}. We focus on characterizing well-defined convergent integrals and look for their saddle point approximation to verify that in the semi-classical limit we indeed recover the standard description. As we will see, the outcome is in many cases somewhat unexpected. 
We will focus in particular on the impact of various boundary conditions on the path integral. In the case of AdS/CFT it is well known what the condition on the boundary of AdS means: with a Dirichlet condition we fix the geometry on which the dual QFT lives, while with different types of boundary conditions one can allow for this geometry to be dynamical \cite{Papadimitriou:2005ii, Compere:2008us}. To solve second order equations of motion two conditions are needed. In the case of AdS/CFT it is however obvious what the other condition should be: AdS spacetime has only one boundary and the second requirement in holographic calculations is that fields behave regularly in the interior. The situation is rather different in cosmology. Here, we have a future space-like boundary where the wavefunction of the universe lives, which could be for example the surface where inflation ends. Then any predictions of the model under study will depend on what condition one imposes on the past space-like boundary. We can think of cutting the spacetime in the past at some finite radius and fine tune the desired initial conditions there. Then the question arises of how these conditions where generated. One would need to consider another space-like surface where to fine tune the right conditions in order to get the desired initial conditions and the repeat the procedure for the new surface. If we think of the case of de Sitter space, it truly has two disjoint (past and future) boundaries and there is no way out of this argument. The idea of Hartle and Hawking is to cut out the problem of initial conditions all together by closing off de Sitter space in such a way that it has only one boundary, the future boundary where the wavefunction of the universe takes values (see the bottom right panel of Fig. \ref{ideafig}). This is the background saddle point geometry which shall approximate the wavefunction of the universe in the semi-classical limit. Then a regularity condition is naturally imposed on the fields living on this geometry in resemblance to the case of AdS. The question we ask in this work is how to construct path integrals which admit such saddle point approximation.  We will make use of an ADM splitting of spacetime~\cite{Arnowitt:1959ah} which allows us to identify two surfaces and evaluate the sum over histories which interpolated between the induced quantities fixed on the two surfaces. The idea is sketched in Fig. \ref{ideafig} where the parameter $t$ used to slice the spacetime is a time-like variable in the case of a positive cosmological constant and can be thought of as a radial variable if the cosmological constant is negative. One surface will be the future boundary for the wavefunction of the universe or the ``outer'' boundary for a partition function in asymptotically AdS spacetime. We called this surface $\Sigma_{t_1}$ in Fig. \ref{ideafig} and the wavefunction or partition function are functionals of the three-geometry $h_1$ induced on this surface. We will study boundary conditions on other surface, $\Sigma_{t_0}$ in the figure, where in both cases we need to enforce the requirement that the saddle point geometry does not have any boundary other that $\Sigma_{t_1}$. We will discuss in particular Dirichlet, Robin and Neumann type of boundary conditions fixed on this past or ``inner'' boundary surface, for a positive and negative cosmological constant respectively, corresponding to fixing the three-metric $h_0$ induced on $\Sigma_{t_0}$, its conjugate momentum $p_0$ or a linear combination of the two. The question is what type of condition gives the desired saddle point approximation of the path integral. As we will see this will not correspond to the requirement that all of the geometries in the sum have no boundary. 

\begin{figure}
    \centering
    \includegraphics[width = 6.9 cm]{myfig11}
    \hspace{1.1em}
        \includegraphics[width = 7.5 cm]{myfig22}\\
    \vspace{1cm}
          \includegraphics[width = 7 cm]{myfig33}
           \hspace{1em}
           \includegraphics[width = 6.7 cm]{myfig44}
    \caption{The wavefunction of the unvierse $\Psi[h_1]$ and partition function $Z[h_1]$ are functionals of the three-geometry $h_1$. The path integral is given by the sum over four-geometries which interpolate between $h_1$ and the quantity fixed on $\Sigma_{t_0}$ which can be taken to be the induced metric $h_1$ (top left panel), the conjugate momentum $p_0$ (bottom left panel) or a linear combination of the two (top right panel where $\alpha$ and $\beta$ are constants.). The pictures are intended to clarify our setting but do not capture some of the features of the path integral: the sum over four-geometries includes also a sum over proper time separations between the two surfaces and in the cases of Neumann and Robin conditions contains geometries of different initial sizes. The saddle point geometry shall have no boundary other that $\Sigma_{t_1}$ (bottom right panel).}.
\label{ideafig}
\end{figure}



We will start, in chapter \ref{introchapter}, by reviewing well-understood aspects of cosmology which will serve as a motivation for our work. We discuss the standard model of cosmology and highlight some of the open questions related to it. Then we introduce inflation as a possible answer to such questions. In chapter \ref{sec:quantum} we review the methods of canonical and path integral quantization of the gravitational field which we will use throughout the thesis. In chapter \ref{quantuminitial} we see our method at work for the first time. We show how inflation is in fact not robust against quantum corrections since two background solutions contribute in general to the path integral leading to a breakdown to the QFT in curved spacetime picture and an instability of fluctuations. One way out of this problem is to provide a semi-classical explanation for the initial conditions of inflation which allows inflation to start and develop already well within the realm of QFT in curved spacetime, with perturbations in their Bunch-Davis vacuum. This explanation is 
possibly provided by the no boundary proposal, which we introduce in chapter \ref{nbDirichlet}. In chapters \ref{nbDirichlet}, \ref{robincosmology} and \ref{chapterneumann} we study minisuperspace path integrals with Dirichlet, Robin and Neumann boundary conditions, respectively. In chapter \ref{nbDirichlet}, we show that the Hartle-Hawking wavefunction cannot be represented as a path integral with Dirichlet boundary conditions. We provide possible path integral representations of the Hartle-Hawking wavefunction using Robin boundary conditions in chapter \ref{robincosmology}. In this chapter we show also how Robin boundary conditions can be used to describe large quantum fluctuations of the inflaton in the eternal inflation regime. In chapter \ref{chapterneumann} we propose what we believe is the best implementation of the no boundary proposal using Neumann boundary conditions for FLRW and the Bianchi IX model. In this case, the initial expansion rate of the universe is fixed rather than its size. Given that momentum and position are conjugate variables in quantum mechanics, fixing the momentum corresponds effectively to a sum over all possible initial sizes. We thus learn that a radical change in our understanding of the no boundary proposal is needed when the gravitational path integral is carefully analyzed. This motivates us to study similar Neumann path integrals for the case of a negative cosmological constant in the second part of chapter \ref{chapterneumann}, where we focus in particular on the Hawking-Page phase transition. As we will see, an analogous Neumann condition is needed to recover the usual thermodynamic description of black holes in anti-de Sitter space. We thus find a nice agreement on the requirements for a well-defined path integral for AdS/CFT and the no boundary proposal. We summarize our findings and discuss possible future research directions and extensions of our studies in chapter \ref{chapconclusion}.

Throughout this work we make use of the minisuperspace approximation to evaluate the saddle point approximation of gravitational path integrals.
Our main message is that the the saddle point approximation of a path integral is not merely obtained by taking one's favorite solution to Einstein's equation and evaluating $e^{i S/\hbar}$ using the action $S$ along the trajectory, as the approximation to some not better defined  
or definable sum over geometries. The point is that a single solution to the classical equations of motion can be a saddle point of different types of integrals and the quantum state of a system is really defined only by the full path integral. Focusing on the saddle point geometry only can in fact be misleading: in chapter \ref{chapterneumann} we will be interested in saddle points which correspond to black holes in Euclidean anti-de Sitter space. One might be tempted to think that the sum shall be given by a sum over Euclidean geometries but we will see this is not the case and the Euclidean integral is explicitly divergent. Similarly, the no boundary saddle point instanton has no boundary in the past but the integral cannot be thought of as a sum over geometries with no boundary. Can then one still think of the Hartle-Hawking quantum state as describing the nucleation of the universe out of nothing, as the saddle point geometry suggests?\\
We describe a number of minisuperspace path integrals with positive and negative cosmological constant specifying boundary conditions and integration contours and in each case we discuss the meaning of the sum and the shortcomings of the implementation. By the end of the thesis, the reader shall be able to associate, for the cases studied, a specific saddle point to an explicitly characterized integral. In this sense, we achieve what we think is a systematization of minisuperspace path integrals.



\end{document}