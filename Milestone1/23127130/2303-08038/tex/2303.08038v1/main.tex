%% 
%% Copyright 2007-2020 Elsevier Ltd
%% 
%% This file is part of the 'Elsarticle Bundle'.
%% ---------------------------------------------
%% 
%% It may be distributed under the conditions of the LaTeX Project Public
%% License, either version 1.2 of this license or (at your option) any
%% later version.  The latest version of this license is in
%%    http://www.latex-project.org/lppl.txt
%% and version 1.2 or later is part of all distributions of LaTeX
%% version 1999/12/01 or later.
%% 
%% The list of all files belonging to the 'Elsarticle Bundle' is
%% given in the file `manifest.txt'.
%% 

%% Template article for Elsevier's document class `elsarticle'
%% with numbered style bibliographic references
%% SP 2008/03/01
%%
%% 
%%
%% $Id: elsarticle-template-num.tex 190 2020-11-23 11:12:32Z rishi $
%%
%%
\documentclass[preprint,12pt]{elsarticle}


%% The amssymb package provides various useful mathematical symbols
\usepackage{amssymb}
%% The amsthm package provides extended theorem environments
%% \usepackage{amsthm}
\usepackage{amssymb,amsmath}
\usepackage{wrapfig}
\usepackage{graphicx,grffile}
\usepackage{lscape}
%\usepackage{lipsum} % for dummy text only REMOVE
\usepackage{natbib}
\usepackage{adjustbox}
\usepackage{multirow}
\usepackage[T1]{fontenc} 
\usepackage{booktabs}
\usepackage{xurl}
\usepackage{xcolor}
\usepackage{rotating}

%% The lineno packages adds line numbers. Start line numbering with
%% \begin{linenumbers}, end it with \end{linenumbers}. Or switch it on
%% for the whole article with \linenumbers.
%% \usepackage{lineno}

\journal{Journal of Biomedical Informatics}

\begin{document}

\begin{frontmatter}

%% Title, authors and addresses

%% use the tnoteref command within \title for footnotes;
%% use the tnotetext command for theassociated footnote;
%% use the fnref command within \author or \address for footnotes;
%% use the fntext command for theassociated footnote;
%% use the corref command within \author for corresponding author footnotes;
%% use the cortext command for theassociated footnote;
%% use the ead command for the email address,
%% and the form \ead[url] for the home page:
%% \title{Title\tnoteref{label1}}
%% \tnotetext[label1]{}
%% \author{Name\corref{cor1}\fnref{label2}}
%% \ead{email address}
%% \ead[url]{home page}
%% \fntext[label2]{}
%% \cortext[cor1]{}
%% \affiliation{organization={},
%%             addressline={},
%%             city={},
%%             postcode={},
%%             state={},
%%             country={}}
%% \fntext[label3]{}

\title{Progress Note Understanding - Assessment and Plan Reasoning: Overview of the 2022 N2C2 Track 3 Shared Task}

%% use optional labels to link authors explicitly to addresses:
%% \author[label1,label2]{}
%% \affiliation[label1]{organization={},
%%             addressline={},
%%             city={},
%%             postcode={},
%%             state={},
%%             country={}}
%%
%% \affiliation[label2]{organization={},
%%             addressline={},
%%             city={},
%%             postcode={},
%%             state={},
%%             country={}}

\author[inst1]{Yanjun Gao, PhD \corref{cor1}%
\fnref{fn1}}
\author[inst2]{Dmitriy Dligach, PhD}
\author[inst3]{Timothy Miller, PhD}
\author[inst1]{Matthew M. Churpek MD, MPH, PhD}
\author[inst4]{Ozlem Uzuner, PhD} 
\author[inst1]{Majid Afshar, MD, MSCR }

\address[inst1]{ICU Data Science Lab, Department of Medicine, University of Wisconsin Madison}
\address[inst2]{Department of Computer Science, Loyola University Chicago}
\address[inst3]{Boston Children’s Hospital, Harvard University}
\address[inst4]{Department of Information Sciences and Technology, George Mason University}

\cortext[cor1]{Corresponding author}
\fntext[fn1]{Email author: ygao@medicine.wisc.edu}


% \affiliation[inst1]{organization={ICU Data Science Lab, Department of Medicine, University of Wisconsin Madison},%Department and Organization
%             addressline={1685 Highland Ave}, 
%             city={Madison},
%             postcode={53792}, 
%             state={WI},
%             country={USA}}
            
% \affiliation[inst2]{organization={Department of Computer Science, Loyola University Chicago},%Department and Organization
%             addressline={1032 W Sheridan Rd}, 
%             city={Chicago,},
%             postcode={60660}, 
%             state={IL},
%             country={USA}}
            
% \affiliation[inst3]{organization={Boston Children’s Hospital, Harvard University},%Department and Organization
%             addressline={300 Longwood Ave}, 
%             city={ Boston,},
%             postcode={02115}, 
%             state={MA},
%             country={USA}}
            
% \affiliation[inst4]{organization={Department of Psychiatry and Behavioral Sciences, Rush University Medical Center},%Department and Organization
%             addressline={1620 W Harrison St}, 
%             city={Chicago},
%             postcode={60612}, 
%             state={IL},
%             country={USA}}

\begin{abstract}
%% Text of abstract
Daily progress notes are a common note type in the electronic health record (EHR) where healthcare providers document the patient's daily progress and treatment plans. The EHR is designed to document all the care provided to patients, but it also enables note bloat with extraneous information that distracts from the diagnoses and treatment plans. Applications of natural language processing (NLP) in the EHR is a growing field with the majority of methods in information extraction. Few tasks use NLP methods for downstream diagnostic decision support. We introduced the 2022 National NLP Clinical Challenge (N2C2) Track 3: Progress Note Understanding - Assessment and Plan Reasoning as one steps towards a new suite of tasks. The Assessment and Plan Reasoning task focuses on the most critical components of progress notes, Assessment and Plan subsections where health problems and diagnoses are contained. The goal of the task was to develop and evaluate NLP systems that automatically predict causal relations between the overall status of the patient contained in the Assessment section and its relation to each component of the Plan section which contains the diagnoses and treatment plans. The goal of the task was to identify and prioritize diagnoses as the first steps in diagnostic decision support to find the most relevant information in long documents like daily progress notes. We present the results of the 2022 n2c2 Track 3 and provide a description of the data, evaluation, participation and system performance.          
\end{abstract}

%%Graphical abstract
%\begin{graphicalabstract}
%\includegraphics{grabs}
%\end{graphicalabstract}

%%Research highlights
%\begin{highlights}
% \section*{Statement of Significance}
% \begin{itemize}
%     \item \textbf{Problem} 
%     \item \textbf{What is Already Known} 
%     \item \textbf{What this Paper adds}   
% \end{itemize}

\begin{keyword}
%% keywords here, in the form: keyword \sep keyword
Natural language processing \sep clinical reasoning \sep clinical diagnostic decision support \sep national nlp clinical challenge
%% PACS codes here, in the form: \PACS code \sep code
%\PACS 0000 \sep 1111
%% MSC codes here, in the form: \MSC code \sep code
%% or \MSC[2008] code \sep code (2000 is the default)
%\MSC 0000 \sep 1111
\end{keyword}

\end{frontmatter}

%% \linenumbers

%% main text

\section{Introduction}\label{sec1}
Healthcare providers generate notes in the electronic health record (EHR) to update diagnoses and treatment plans, and to document changes in the patient’s health status using the daily progress note type. The progress note is one of the most frequent note types that carry the most relevant and viewed documentation of the patient's care~\cite{brown2014physicians}. The Subjective, Objective, Assessment and Plan (SOAP) format is the framework currently taught in medical schools for generating a daily progress note and it remains the most ubiquitous format for daily care note taking in the EHR~\cite{weed1964medical}. The Assessment and Plan sections of the progress notes are the free-text fields where healthcare providers identify patients’ problems/diseases and treatment plans. Specifically, the Assessment section summarizes the patients’ active health problems or diseases for that day. The Plan section consists of multiple subsections, each addressing a specific health issue or diagnosis followed by a detailed treatment plan (e.g., Respiratory failure with MRSA pneumonia: continue seven days of vancomycin, continue mechanical ventilation with lung-protective strategy, wean oxygen). Patients typically have a main diagnosis or problem with associated conditions or downstream health effects. Healthcare providers can implicitly understand what parts of a treatment plan directly target the primary concerns of a series of health problems based on their background knowledge and medical reasoning. Still, this information is not easily available for downstream analysis. Automatic extraction of this information can potentially be useful for downstream use cases such as problem list generation.

However, EHRs suffer from issues such as note bloat (copying and pasting), information overload (automatic inclusion of data and administrative documentation), and disorganized notes, which can lead to burnout among providers and hinder efficient care~\cite{shoolin2013association}. Methods in Natural Language Processing (NLP) hold promise to help overcome the negative consequences of large-scale EHRs. More shared tasks are needed to focus on clinical decision support to guide providers through the breadth and depth of EHR data to reduce diagnostic errors and provide more efficient documentation. To date, tasks proposed for clinical NLP (cNLP) included named entity recognition (NER), information extraction (IE), relation extraction, document classification, sentence classification and others. In a scoping review on publicly available clinical NLP tasks from English sources between 2006 and 2021 showed that half of the tasks were for IE and NER, and over sixty percent of the tasks were non-specific secondary use in clinical settings~\cite{gao2022scoping}. While tasks and models for clinical IE are important, few tasks were motivated by providers' needs in daily practice and framed to assist the providers' decision-making~\cite{lederman2022tasks}. A shift towards more tasks designed for clinical decision support  (CDS) are needed~\cite{lederman2022tasks, gao2022scoping}. 




To facilitate this paradigm shift and research focus of cNLP tasks for clinical decision-making, in particular, diagnostic decision-making, we prepared Track 3 of the 2022 National NLP Clinical Challenges (N2C2), a shared task named as \textsc{Progress Note Understanding - Assessment and Plan Reasoning}. Since 2006, the N2C2 has been a key contributor to the advancement of cNLP through shared tasks that focused on developing and evaluating NLP algorithms for extracting and organizing clinical data, including the EHR. We formulated the providers’ reasoning process as a relation prediction task between Assessment and every Plan Subsection within a single daily progress note. Given each pair of Assessment and Plan subsection as input, the goal was to develop a model that accurately predicted one of the following relations: \textsc{(1) Direct}, \textsc{ (2) Indirect}, \textsc{ (3) Neither} and \textsc{ (4) Not Relevant}. The four relations corresponded to the provders' judgment on whether a diagnosis presented in the Plan Subsection was the main reason for hospitalization (\textsc{Direct}), the secondary health problem/diagnosis to the main problem/diagnosis (\textsc{Indirect}), an issue that was not documented (\textsc{Neither}), and not a diagnosis or problem (\textsc{Not Relevant}). A model developed on this task could assist in prioritizing health concerns and allow guidance on the most critical and potentially life-threatening issues in a patient's medical chart. 

Predicting the relation between the Assessment and Plan sections helps to evaluate if the NLP model could use medical knowledge to perform reasoning. The aim of this paper was to present an overview of the shared task including the data preparation, evaluation, and submitted system performance. 

\section{Methods}
\subsection{Data Preparation}

Our task was based on the daily progress notes sampled from the Medical Information Mart for Intensive Care (MIMIC-III) dataset. MIMIC-III is a large, single-center EHR database with de-identified patient records in critical care units (ICU). It is publicly available through PhysioNet with a Data Usage Agreement. We randomly sampled a subset of 5000 physician-written progress notes over 84 note types that represent daily progress notes. The sampling method was multi-disciplinary and included notes from the Trauma ICU, Medical ICU, Surgical ICU, Cardiovascular ICU, Cardiothoracic ICU, and Neurological ICU. An initial screening was performed to exclude the progress notes that did not contain any diagnoses in the Assessment and Plan sections. 


The task of \textsc{Assessment and Plan Reasoning} was the second task in a new suite of clinical NLP tasks, \textsc{Progress Note Understanding}~\cite{gao-etal-2022-hierarchical}. The suite aims at training and evaluating future NLP models for clinical text understanding, clinical knowledge representation, inference and summarization. To build this suite, we used an annotation guideline developed by two physicians with board certifications in critical care medicine and clinical informatics. The guidelines specified three sequential stages of the annotation process: \textsc{SOAP Section Tagging} labeling all sections of the progress note into a SOAP category; \textsc{Assessment and Plan Relation Labeling} specifying the relations between symptoms and problems
mentioned in the Assessment and diagnoses covered in
each Plan Subsection; \textsc{Problem List Identification}
highlighting the final diagnoses and treatment plans. The data for our N2C2 task comes from the second stage of the annotation \textsc{Assessment and Plan Relation Labeling}. The annotation guideline that described the rules for labelling the Plan subsections is presented in Figure~\ref{fig:guidelines}. 

\begin{figure}[ht]
    \centering
    \includegraphics[width=\textwidth]{Figs/inception_screenshot.png}
    \caption{An example of INCEpTION interface for \textsc{Assessment and Plan Relation Labeling}. Linebreaks are naturally preserved from MIMIC-III raw notes. The pop-up window shows an link between the Plan Subsection ``Hypoxic respiratory failure: Pt ... 3 days of chemontherapy'' and the Assessment ``RENAL FAILURE, ACUTE 
 ...''. The relation label is \textsc{Direct}. }
    \label{fig:inception}
\end{figure}


We used INCEpTION~\cite{tubiblio106270} as the annotation platform. Figure~\ref{fig:inception} shows a screenshot of INCEpTION interface for \textsc{Assessment and Plan Relation Labeling} task. Linebreaks are naturally preserved from MIMIC-III raw data. On every line of text that belongs to Assessment and Plan sections (text spans marked with orange tags \textsc{A/P}), we further categorized them as \textsc{Assessment} or \textsc{Plan Subsection} (the blue color box above tag \textsc{A/P}). Then we linked the two text spans and marked the relation type (the tag \textsc{Direct} and blue dotted lines connecting the two text spans). A pop-up window would appear to show the linkage between the Plan Subsection and the Assessment.      

Two medical students were recruited as annotators. The medical students had completed their first year of medical school and were trained in the SOAP note format. They completed basic training in pathology, anatomy, pharmacology, and pathophysiology and also trained for an additional three weeks with the physician-scientists supervising the task. The inter-annotator agreement using Cohen's kappa score was measured between each annotator and the physician-scientist supervisor until each annotator met a threshold of 0.80 Kappa score with the supervisor. The supervisor audited the annotations periodically to re-assess the agreement would re-train if the kappa score fell below 0.80. During annotation, annotators augmented their knowledge with subscription and access to a large medical reference library, UpToDate®\footnote{\url{https://www.wolterskluwer.com/en/solutions/uptodate}}. The final inter-annotator agreement between the two medical student annotators was reported as 0.76, which was acceptable given the complexity of the task. 



\begin{figure*}
\small
\centering
    \begin{tabular}{l|l} \hline 
    Criterion & Label \\ 
    \hline 
       Assessment section includes a primary diagnosis/problem and  &  \textsc{direct} \\
       it is mentioned in the Plan subsection.  \\ \midrule
       Progress note includes a primary diagnosis/problem for   & \textsc{direct} \\
       hospitalization and it is mentioned in the Plan subsection. &  \\ \midrule
       Plan subsection contains a problem/diagnosis related to  & \textsc{direct} \\
       the primary signs/symptoms in the Assessment section. &  \\ \midrule 
       Plan subsection contains complications/subsequent events   & \textsc{indirect} \\ 
       or organ failure related to the primary diagnosis/problem  & \\
       from the Assessment section. & \\ \midrule 
       Plan subsection contains other listed diagnoses/problems & \textsc{indirect}  \\ 
       from the overall Progress Note or in the Assessment & \\
       section that are not part of the primary diagnosis/problem. &  \\  \midrule 
       Plan subsection contains a diagnosis/problem that is not & \textsc{indirect} \\
       previously mentioned but closely related (i.e., same organ &  \\
        system) to the primary diagnoses/problems mentioned in   &  \\
       the overall Progress Note or Assessment section. &  \\  \midrule 
       None of the criteria for Directly Related or Indirectly & \textsc{neither} \\
       Related are met but a diagnosis/problem or other signs/ & \\ 
        symptoms are mentioned. &  \\ \midrule 
       Plan subsection does not include a diagnosis/problems OR & \textsc{not rel} \\
       signs/symptoms.  \\ 
       \hline
    \end{tabular}

    \caption{Guidelines for annotating the four relations (\textsc{direct, indirect, neither, not relevant}) between Assessment and each subsection of Plan}
    \label{fig:guidelines}
\end{figure*}

The final annotated corpus contained 768 progress notes and 5934 labels for the four relations. Specifically, we annotated 1404, 1599, 1913, and 1018 relations for \textsc{Direct, Indirect, Neither} and \textsc{Not Relevant}, respectively. 

\subsection{Task setup, timeline and evaluation }

\begin{figure}[]
    \centering
    \begin{tabular}{l|l} 
    \toprule 
         Assessment & Mr. [**Known lastname 512**] is a 60 y/o male with  \\
         & HTN who presents with a food impaction after ingestion  \\ 
         & of roast beef at the RedSox game.  \\ \midrule
         Plan Subsection & Opacity on CXR. Concerning for aspiration \\
         & pneumonitis. \\
         &   - Continue clindamycin for now \\ 
         &   - Will monitor patient for new O2 requirement, fever. \\  \midrule
         Relation & Indirect \\ 
         \bottomrule 
    \end{tabular}
    \caption{An example pair of Assessment and Plan Subsection with \textsc{Indirect} as the ground truth labels. ``Aspiration pneumonitis'' is the diagnosis mentioned in the Plan Subsection. }
    \label{fig:error_example}
\end{figure}

Figure~\ref{fig:error_example} presents an input example with the ground truth label as \textsc{Indirect}. To correctly predict the relation, a model needs to understand the medical concepts in both Assessment and Plan Subsections, and establish the connections between concepts given the input context. Each sample was organized as a pair of Assessment and Plan Subsection with the gold relation labels into csv files. As a result, we obtained 4670, 594, and 677 samples in training, development, and test set, respectively. Using a BERT tokenizer~\cite{devlin2019bert}, the average length of the Assessment section was 77.43 tokens, and the average length of each Plan Subsection was 86.84.  Along with the sample, we released the Hospital Admission ID (\textsc{HADMID}) indicating the particular admission the progress note was generated from. The N2C2 participants had the opportunity to use the entire progress note using \textsc{HADMID} as well as other note types, and we asked the participants to report if they used other note types as input to the model when they submitted their systems.  

The training set was released at the beginning of May, 2022 and the test set was released in mid-July 2022. Participants had two months to develop models. Each team was allowed to submit at most 9 runs, and we picked the best-performing system as the final performance for the teams. Given that the relation classes were distributed evenly, the Macro F1 score and the 95\% confidence intervals (CIs) were reported as the evaluation metric used for ranking systems.

During the final submissions of systems, each team was asked to complete a survey about system development. The survey addressed questions about the methods, external datasets, expert resources, input data, long document usage with the entire progress note, and multi-modality EHR usage.

\section{Results}
\subsection{Participation}
\begin{table}[ht]
\small 
    \centering
    \begin{tabular}{l|l|l|r} \toprule
       Team  & Institutions & Country & \# Runs  \\ \toprule 
      Carnegie Mellon University & Carnegie Mellon University  & USA & 9 \\ 
      (CMU)  & National Library of Medicine &  & \\ \midrule 
       \multirow{1}{*}{ University of Florida (UFL)} & University of Florida & USA & 4 \\ \midrule 
       \multirow{2}{*}{ Yale University (Yale)} & Yale University, &USA &  1\\ 
       & University College Dublin & Ireland \\ \midrule 
       University of Massachusetts& UMass-Amherst, & USA & 5\\ 
       - Amherst (UMass) & UMass-Lowell  & \\  \midrule 
       John Hopkins University (JHU) &  Harvard University, & USA & 3 \\  & John Hopkins University, &  \\ 
       & University of Pittsburgh & 
       \\ \midrule
       University of Wisconsin (UWisc) & University of Wisconsin & USA & 3\\ \midrule 
       University of Duisburg-Essen  & University of Duisburg-Essen & Germany & 3\\ 
       (UDuisburg) & & \\  \midrule 
       Dalian Minzu University  & Dalian Minzu University & China & 1  \\ 
       (Dalian-Minzu) & Dalian University of Technology & & \\ 
       \bottomrule 
    \end{tabular}
    \caption{Overview of participated teams: the institutions the teams represented, countries represented, and number of submitted runs. }
    \label{tab:participation}
\end{table}

Participating teams were required to sign a data use agreement to get access to the shared task. During the registration period, we had 53 participants registered for the shared task. Twenty-six participants across eight teams finished the model development and made successful submissions during the system evaluation period. Table~\ref{tab:participation} presents the teams, countries represented, and the number of submitted runs. 

\subsection{Team performance}

  
\begin{table}[ht]
\small 
    \centering
    \begin{tabular}{l|l|l} \toprule
       Rank  & Teams & Macro F1  \\ \toprule 
        1 & CMU & 0.8212  \\ 
2 & Yale &  0.8133   \\ 
3 &  UWisc & 0.8119  \\ 
4 & UDuisburg &  0.8034  \\
5 & JHU & 0.7994 \\ 
6 & UMass & 0.7949 \\ 
7 & UFL &  0.7947 \\ 
8 & Dalian-Minzu & 0.7454 \\
       \bottomrule 
    \end{tabular}
    \caption{Team performance ranking based on the best system performance.}
    \label{tab:best_ranking}
\end{table}

\begin{figure}
    \centering
    \includegraphics[scale=0.5]{Figs/CIs.png}
    \caption{Macro F1 with 95\% confidence intervals from all teams. }
    \label{fig:CI_best}
\end{figure}

\begin{table}[ht]
\small 
    \centering
    \begin{tabular}{l|l|l|l|l|l} \toprule
       Teams  & Mean F1 & Median F1 & Min F1 & Max F1 & SD F1  \\ \toprule 
       CMU &  0.8064 &	0.8090	& 0.7920 &	0.8212 & 0.0099  \\ 
       Yale & 0.8133 & 0.8133 &	0.8133 & 0.8133	 &0.0000 \\
       UDuisburg & 0.8006 &	0.7998 &	0.7987	& 0.8034 & 0.0025 \\ 
       UWisc & 0.7990 &	0.7979 &	0.7872 &	0.8119 &	0.0124 \\
       UFL & 0.7915	& 0.7937 &	0.7839 &	0.7947 & 0.0051 \\ 
       JHU & 0.7851 &	0.7896 & 0.7663 & 0.7994 &	0.0170 \\ 
       UMass & 0.7833&	0.7785	& 0.7771 &	0.7949 & 0.0082 \\
       Dalian-Minzu & 0.7454	& 0.7454 &	0.7454 &	0.7454 &	0.0000 \\ \midrule
       All & 0.7949 &	0.7949 & 0.7454 & 0.8212 &	0.0160 \\ 
       \bottomrule 
    \end{tabular}
    \caption{Statistics over all submitted runs (scores reported in Macro F1).}
    \label{tab:average_ranking}
\end{table}

We received 29 submissions of system output and ran an evaluation script to generate the final scores. In this section, we presented the system ranking and average performance across all submitted systems. 

Table~\ref{tab:best_ranking} presents the ranking based on the best systems submitted by each team. The best optimal results were achieved by CMU with a Macro-F1 score of 0.8212, followed by the team Yale with an F1 score of 0.8133, and the system from UWisc followed with an F1 score of 0.8119. 

   

\begin{table}[ht]
\small 
    \centering
    \begin{tabular}{l|l|l|l|l} \toprule
       Teams  & Direct & Indirect & Neither & Not Relevant  \\ \toprule 
       CMU &  0.8306 &	0.6917	& 0.8254 &	0.9372   \\ 
       Yale & 0.8328 & 0.6764 & 0.8158 & 0.9307 \\
       UWisc & 0.8101 &	0.7005 &	0.8227 &	0.9143 \\
       UDuisburg & 0.8038 &	0.6792 &	0.7999	& 0.9306  \\ 
       UFL & 0.8202	& 0.6593 &	0.8115 & 0.8878  \\ 
       JHU & 0.8115 &	0.6772 & 0.7972 & 0.9117 \\ 
       UMass & 0.8011 &	0.6667	& 0.8054 &	0.9064 \\
       Dalian-Minzu & 0.7391	& 0.6196 &	0.7229 &	0.9000 \\ \midrule
       All &  0.8062 & 0.6713 & 0.8001 & 0.9154 \\ 
       \bottomrule 
    \end{tabular}
    \caption{Macro F1 on the four label classes across the eight teams' best systems.}
    \label{tab:label_f1}
\end{table}

We ran an evaluation on best system prediction using bootstrapping on 10000 samples and plotted the 95\% CI in Figure~\ref{fig:CI_best}. The 95\% CI from the top 7 teams overlapped, indicating that there might not be a significant difference in the team performances. In Table~\ref{tab:average_ranking}, we presented the overall statistics on Macro F1 across all submitted systems. The team Yale achieved the highest average F1 score at 0.8133, followed by team CMU at 0.8064 and UDuisburg at 0.8006. The average performance across 29 submitted systems was 0.7949, with a standard deviation of 0.0160.

Table~\ref{tab:label_f1} presents the broken-down Macro F1 scores on the four labels across each team's best system. On the four labels, all teams achieved the highest performance on predicting \textsc{Not Relevant} with average scores as 0.9154, and most teams had \textsc{Direct} labels predicted correctly with average scores at 0.8062. \textsc{Indirect} is the hardest label to predict and reported as 0.6713 average macro F1. Finally, the average score on \textsc{Neither} is 0.8001.   



\subsection{Methods overview}

\begin{table}[ht]
\small 
    \centering
    \begin{tabular}{l|l|r} \toprule
       Fields  & Explanations  & \# Yes\\ \toprule 
       method-used & Describe the methods. & - \\ 
        other-dataset & Did you use any other datasets besides MIMIC?  & 0  \\ 
        external-resource & Did you use any external medical resource? & 3 \\ 
        md-involved & Did you use medical doctors' expertise?  & 4\\ 
        additional-data & Did you use additional data as input?  & 4  \\ 
        entire-progress-note & Did you use the entire progress notes ? & 4\\ 
        other-parts & Did you use other parts of the notes? & 7 \\ 
        multi-modal & Did you use multi-modality?  & 0 \\  
       \bottomrule 
    \end{tabular}
    \caption{Fields in the submission form that describe the method development and the collected number of ``Yes'' over 29 submissions.}
    \label{tab:system_spec}
\end{table}
 

\begin{figure}
    \centering
    \includegraphics[width=\textwidth]{Figs/all_keywords_frequency.png}
    \caption{Top 10 most frequent keywords occurred in the submission descriptions. }
    \label{fig:most_freq_keywords}
\end{figure}

\begin{table}[]
    \centering
    \begin{tabular}{c|c}
    \toprule
      Transformer Type &	Count \\ \midrule
      BERT &	12 \\ 
    ClinicalLongformer &	6 \\ 
    GatorTron	& 4 \\ 
    Longformer	& 3\\ 
    EntityBERT	& 3 \\ 
    BioClinicalBERT & 3 \\ 
    T5	& 1 \\ 
    RoBERTa	& 1 \\ 
    BioClinicalLongformer	&1 \\  
    \bottomrule
    \end{tabular}
    \caption{Frequency of various Transformer models in submission. }
    \label{tab:transformers}
\end{table}


 
% 95 CIs
% Methods overview

\begin{sidewaystable}[]
\adjustbox{max width=\textwidth}{%
\small 
    \centering
           \begin{tabular}{l|l|l|l|l|l|l} \toprule 
       Team  & Methods & external-resources & md-involved & additional-data & entire-progress-notes & other-parts  \\ \midrule 
       %  & & -resources & & -data &  -notes \\ 
         CMU & The team used ClinicalBERT ensemble without data shuffling.    & N & N & Y & N & Discharge Summaries \\
         & They applied Bayesian inference to combine posterior probabilities   &  & & & & \\ 
         & of single model output.  &  & & & & \\ 
         \midrule
         Yale & The team used RoBERTa-Large as base model and a Human-in-the-loop  & N & Y & N & N & N \\
         & approach  where clinicians annotated clinical notes for primary and   &  & & & &\\
         &  secondary problems. They then trained an NER tagger on the       &  & & & & \\ 
         & augmented data and applied the tagger with the base model for final  &  & & & &    \\ 
         & prediction. &  & & & &    \\  \midrule
         UWisc & They encoded concept features using International Classification of   & UMLS, MetaMap & N & N & N &N   \\ 
         & Diseases (ICD) codes provided by MetaMap. The final input included     & & & & & \\
         & concept features and text from Assessment and Plan Subsection and     & & & & & \\
         & was fed into an ensemble model with Enity-BERT and LightGBM. & & & & & \\ \midrule
         UDuisburg & They tested with general domain BERT model and ClinialBERT. & NER tagger, Stanza & N & N & N & N \\
         & For ClinicalBERT model, they utilized plan ordering feature and  & & & & &\\
         & Stanza NER tagger. The best system is an ensemble method over   & & & & &\\
         & CRF,  general domain BERT and ClinicalBERT.  &  &  &  &  \\ \midrule
         JHU & The team uses ClincalBERT and fine-tuned different clinicalBERT   &  N & N & N & N & N  \\ 
         & models by formatting the input Assessment and Plan Subsection as   & & & & &\\  
         & next sentence prediction task. Then they built ensemble model to  & & & & &\\
         &  make the final prediction using majority voting.  & & & & &\\ \midrule 
         UMass & The team utilized UMLS semantic type features, and built a ensemble & UMLS, MedCAT & Y & N & N & N \\ 
         & method over BioClinicalLongformer, ClinicalLongformer and LSTM. & & & & &\\
         & The system made prediction over sequence of input paragraphs. & & & & &\\ \midrule 
         UFL & The team utilized a pretrained transformer model – GatorTron, & N & N & N & N & N \\
         & which was developed using over 80 billion words of clinical narratives & & & & &\\ 
         & at UF Health using the BERT architecture. & & & & &\\ \midrule  
         Dalian-Minzu & The team reformulated the task as a masked language text & N & N & N & Y & N  \\ 
         & generation task by constructing cloze-style prompt
templates, & & & & &\\ 
& and fed the prompt and input text to general domain T5 models. & & & & &\\  
         \bottomrule
    \end{tabular} 
 }
    \caption{Best system and configurations from each team. }
    \footnote{The column headers of the table corresponding to the survey we collected from shared task submissions. In particular, \textit{external-resources} indicates if the team uses external resources (such as UMLS, MetaMap); \textit{md-involved} indicates if the team has a medical expert involving in system development and data analysis; \textit{additional-data} indicates if the team uses any data besides MIMIC;  \textit{entire-progress} indicates if the team the entire progress notes during pre-training or continuous training; \textit{other-parts} indicates if the team uses other parts of the notes or outside the notes for training. }
    \footnote{Abbreviations in methods: NER: Named Entity Recognition; UMLS: Unified Medical Library System.  }
    
    \label{tab:best_system_config}
\end{sidewaystable}

We received 29 submissions for the survey with the response rate for each question shown in Table~\ref{tab:system_spec}. None of the submissions used other datasets besides MIMIC or involved multi-modal models. Four teams answered ``Yes'' to the question of whether the entire progress note was being used or other parts of the notes were used; however, the additional sections or notes were only utilized in pre-training and continuous training. Four systems used discharge summaries as input, and four consulted medical doctors' expertise during system development. There were three systems that used external medical resources such as the National Library of Medicine's Unified Medical Language System (UMLS). 

We identified the keywords used in the ``method-used'' and plotted the top 10 most frequent words used in the submitted abstract methods (Figure~\ref{fig:most_freq_keywords}). We found ``Ensemble'' and ``BERT'' was mentioned 12 times across all submissions. All teams used transformer-based pre-trained language models~\cite{vaswani2017attention}, and concatenated each input Assessment and Plan Subsection with special tokens like ``[CLS]'' and ``[SEP]''. A summary of the full list of language models used is in Table~\ref{tab:transformers}. The original BERT model pre-trained on the general domain was used most frequently in ensemble methods with other BERT variants and machine learning algorithms (Bayes rules, Conditional Random Fields, etc.). BERT-based models that were pre-trained or continuously trained on the biomedical domain were also common in this shared task. For instance, GatorTron was trained on a corpus with over 90 billion words from the University of Florida Health System's EHR, PubMed articles, Wikipedia and MIMIC ~\cite{yang2022gatortron}. EntityBERT was continuously trained on PubMed and MIMIC dataset using medical entity masking strategy~\cite{lin2021entitybert}. Besides BERT, ten submitted system outputs were produced from Longformer~\cite{Beltagy2020Longformer} and its biomedical domain variants (``ClinicalLongformer''~\cite{li2022clinical}, ``BioClinicalLongformer''). Longformer is a BERT-variant with longer token limits and was designed to process document-level information, and it was a popular choice among all teams. 

Table~\ref{tab:best_system_config} summarized each team's best system design and configuration. Three of eight teams utilized the Unified Medical Language System (UMLS), a large medical vocabulary for medical concepts~\cite{bodenreider2004unified}.  The best-performing system developed by team CMU designed an ensembled BERT method and trained the models on both progress notes and discharge summaries. They utilized the Bayes Inference on different models. The second-best system developed by team Yale trained a NER tagger based on the training and development set, and a Human-in-the-loop approach with the help of two physicians to generate augmented data for training. Specifically, they have clinical experts labeled primary and secondary problems/symptoms, and complications from in-house ICU SOAP notes. The UWsic constructed a pipeline that first encoded hand-crafted features and disease and symptoms categories identified from a concept extractor (MetaMap~\cite{aronson2006metamap}), then fed these features into Entity-BERT models, running a LightBGM algorithm~\cite{ke2017lightgbm} to select the final prediction. Like team UWsic, team Duisburg employed both hand-crafted features and machine learning algorithms, specifically utilizing a Conditional Random Field (CRF) classifier and features from Stanza i2b2 clinical NER tagger~\cite{zhang2021biomedical} in BioClinicalBERT~\cite{alsentzer2019publicly} models. They took majority vote over ensemble BERT models as the final prediction, and yielded competitive performance. Team JHU also employed an ensemble method, specifically utilizing a majority vote over BioClinicalBERT models. Team UMass utilized a set of Longformer models including ClinicalLongformer~\cite{li2022clinical} and BioClinicalLongformer. Their best-performing system applied MedCAT~\cite{Kraljevic2021-ln} to identify the medical concepts as part of the input and a layer of Long short-term memory (LSTM)~\cite{hochreiter1997long} for final relation prediction. Team UFL ran GatorTron, a transformer-based model pre-trained on over 90 billion words including clinical and general domains. Finally, team Dalian-Minzu was the only team that used generative language models for this task. Specifically, they used T5~\cite{raffel2020exploring} with cloze-style prompt templates to generate the relations.    

We also found that two teams used Plan Subsection ordering information: team UWisc and team UDuisburg, but the effects of incorporating such information was not significant.  

\section{Discussion}
 
For Track 3 of the n2c2 shared task, we attracted international participation and eight teams with 27 participants contributed to solving the task. The best system achieved 0.8212 Macro-F1 score, and the average Macro F1 score across all submitted teams was 0.7949. All the submitted systems were developed based on BERT and Transformer methods, demonstrating the impact that pre-trained language models have had on the field of clinical NLP. We also observed that the most common errors occurred on \textsc{Indirect} and \textsc{Neither} predictions possibly because the useful information to determine these two relations were contained in other sections of the progress note and required complex medical reasoning. None of the tasks used the full progress note and the different modalities of data from the SOAP sections (overnight events, vital signs, physical exam findings, laboratory results, iamging results), so future research may include working on long document processing and  multi-modal cNLP models. 

Most ensemble methods performed better than single models, and combining hand-crafted features from knowledge sources with observations of the data helped the models predict relationships with greater accuracy than without. The major medical knowledge source for pre-trained models included EHR data, PubMed articles, and Wikipedia. We observed that models trained on EHR data showed a subtle advantage over the other pre-trained models, as two of the top 3 systems were trained on MIMIC dataset. In particular, the top-performing system showed that utilizing other EHR note types (i.e., discharge summaries) to augment training data and finding posterior estimates with Bayes Inference over ensemble BERT models helped to augment the models' performance. We also observed that applying NER tagging on the input text could help the model better capture the relations between concepts. In particular, team Yale demonstrated that a model benefited from the human-in-the-loop approach where clinical experts would identify the primary and secondary diagnoses from the input data and inject the domain knowledge into pre-trained language models. Their approach achieved competitive results by being the second-top-performing system.  

Several teams observed from the training data that the \textsc{Direct} relations often occurred at the beginning of the Plan sections. They developed models that incorporated the ordering information of Plan Subsections but the improvements were not significant. The primary diagnosis may be placed before other diagnoses in some physicians' writing, but this was not always the case and the ordering of that Plan section proved not to be helpful.     

Longformer was a popular model selection among all teams, but the improvements over BERT models with smaller token limits did not provide much performance gain, possibly due to few samples exceeding the token limit. If participants had used the entire progress note then Longformer would have been more applicable to ingest the long document of daily care notes. During annotation, full access to the progress note was provided and the full document was used in reasoning for the labels. Modeling the behavior of the annotators in the NLP models was not an approach taken by any participants. Important details were available in the Subjective sections and Objective sections that were missed in all model inputs. For instance, the diagnosis of ``leukocytosis'' may be reflected in blood test results of the Objective section of the SOAP progress note. Understanding the blood test results would also require the system's capacity in understanding the structured text. Further, some diagnoses required complex medical reasoning. Figure~\ref{fig:error_example} presents an example Assessment and Plan Subsection with the ground truth relation as \textit{Indirect} where all systems predicted incorrect labels. To diagnose a patient with aspirations pneumonitis, typically a healthcare provider would perform imaging and other diagnostic tests. Although aspirations pneumonitis is not directly related to hypertension (``HTN'' in the Assessment section in Figure~\ref{fig:error_example}), hypertension was a mentioned comorbidity that may confer a higher risk for a poor health outcome from the primary diagnosis of pneumonitis. Further, our task surfaces the potential role for a multi-modal NLP model with advanced medical reasoning that could improve the performance by understanding and interpreting the clinical evidence from other parts of the progress note that include vital signs, physical exam findings, laboratory data, and imaging results to infer logical relations on the concepts that were distantly related.   

Several limitations occurred in the shared task. First, no formal sample size calculations were performed to identify the number of labels and notes needed for a well-powered task.  The small differences in F1 scores between participants may have been attributable to similar use of pre-trained language models but also that the sample size in the test set was underpowered to detect a difference. Future directions in shared tasks should include sample size calculations a priori.  Further, our sampling was performed from a single health system and was limited to ICU progress notes. While the SOAP format is ubiquitous, variations in notetaking style by discipline and templated language from diferent EHR vendors likely provide more heterogeneity in progress notes than was available in  MIMIC. Multi-center sampling across more disciplines is another future direction to further challenge model performance.

        

\section{Conclusion} 
Overall, the top-performing models achieved high F1 scores but room for improvement remains. This task addressed a specific clinical use that helps providers to prioritize their time on the most urgent diagnoses and health problems. Yet, our single task is not enough to transform the focus of the cNLP field to promote model developments for clinical applications that assist clinical decision-making. For an NLP-driven clinical decision support system that could be run at the bedside, the basic design decisions will require NLP models that are capable of utilizing medical knowledge and understanding and synthesizing information. In the suite of cNLP tasks where this shared task originated from, two additional tasks are publicly available to facilitate model development and evaluation for clinical diagnostic reasoning~\cite{gao2022tasks, goldberger2000physiobank}. The two other tasks are a SOAP Section Labeling task and a Problem Summarization task, and both tasks are widely recognized as useful tasks for note-writing practice and diagnostic decision-making~\cite{gao-etal-2022-hierarchical}. We thus encourage the community to create, follow and conduct research on clinical diagnostic reasoning and advance the field of cNLP.  

 
     


%% The Appendices part is started with the command \appendix;
%% appendix sections are then done as normal sections
\appendix
\section*{Data Availability}
 ``MIMIC-III' is available at PhysioNet (\url{https://physionet.org/content/mimiciii-demo/1.4/}).
 The data used in this N2C2 challenges is available at N2C2 website. (\url{https://n2c2.dbmi.hms.harvard.edu/}). 


\section*{Competing Interest}
No competing interest is declared. 

\section*{Declarations}
The research data used in this work is only available through PhysioNet and the original publication. Data Use Agreement (DUA) is required for MIMIC-III based dataset. We do not claim authorship over the dataset.  

\section*{Funding}
The work was supported by NIH/NIDA grant number R01DA051464 (to MA), NIH/NIGM grant number R01HL157262 (to MMC), NIH/NLM grant numbers R01LM012793 (to TIM), NIH/NLM grant number R01LM010090 (to DD), NIH/NLM grant number R13LM013127 (to OU). The content is solely the responsibility of the authors and does not necessarily represent the official views of the National Library Of Medicine or the National Institutes of Health.

 \bibliographystyle{elsarticle-num} 
 \bibliography{bibs}
 
%%%% -*-BibTeX-*-
%%% Do NOT edit. File created by BibTeX with style
%%% ACM-Reference-Format-Journals [18-Jan-2012].

\begin{thebibliography}{52}

%%% ====================================================================
%%% NOTE TO THE USER: you can override these defaults by providing
%%% customized versions of any of these macros before the \bibliography
%%% command.  Each of them MUST provide its own final punctuation,
%%% except for \shownote{}, \showDOI{}, and \showURL{}.  The latter two
%%% do not use final punctuation, in order to avoid confusing it with
%%% the Web address.
%%%
%%% To suppress output of a particular field, define its macro to expand
%%% to an empty string, or better, \unskip, like this:
%%%
%%% \newcommand{\showDOI}[1]{\unskip}   % LaTeX syntax
%%%
%%% \def \showDOI #1{\unskip}           % plain TeX syntax
%%%
%%% ====================================================================

\ifx \showCODEN    \undefined \def \showCODEN     #1{\unskip}     \fi
\ifx \showDOI      \undefined \def \showDOI       #1{#1}\fi
\ifx \showISBNx    \undefined \def \showISBNx     #1{\unskip}     \fi
\ifx \showISBNxiii \undefined \def \showISBNxiii  #1{\unskip}     \fi
\ifx \showISSN     \undefined \def \showISSN      #1{\unskip}     \fi
\ifx \showLCCN     \undefined \def \showLCCN      #1{\unskip}     \fi
\ifx \shownote     \undefined \def \shownote      #1{#1}          \fi
\ifx \showarticletitle \undefined \def \showarticletitle #1{#1}   \fi
\ifx \showURL      \undefined \def \showURL       {\relax}        \fi
% The following commands are used for tagged output and should be
% invisible to TeX
\providecommand\bibfield[2]{#2}
\providecommand\bibinfo[2]{#2}
\providecommand\natexlab[1]{#1}
\providecommand\showeprint[2][]{arXiv:#2}

\bibitem[\protect\citeauthoryear{Albrecht and Stone}{Albrecht and
  Stone}{2017}]%
        {Albrecht2017ReasoningAH}
\bibfield{author}{\bibinfo{person}{Stefano~V. Albrecht} {and}
  \bibinfo{person}{P. Stone}.} \bibinfo{year}{2017}\natexlab{}.
\newblock \showarticletitle{Reasoning about Hypothetical Agent Behaviours and
  their Parameters}. In \bibinfo{booktitle}{\emph{AAMAS}}.
\newblock


\bibitem[\protect\citeauthoryear{Andrejczuk, Berger, Rodriguez-Aguilar, Sierra,
  and Mar{\'\i}n-Puchades}{Andrejczuk et~al\mbox{.}}{2018}]%
        {andrejczuk2018composition}
\bibfield{author}{\bibinfo{person}{Ewa Andrejczuk}, \bibinfo{person}{Rita
  Berger}, \bibinfo{person}{Juan~A Rodriguez-Aguilar}, \bibinfo{person}{Carles
  Sierra}, {and} \bibinfo{person}{V{\'\i}ctor Mar{\'\i}n-Puchades}.}
  \bibinfo{year}{2018}\natexlab{}.
\newblock \showarticletitle{The composition and formation of effective teams:
  computer science meets organizational psychology}.
\newblock \bibinfo{journal}{\emph{The Knowledge Engineering Review}}
  \bibinfo{volume}{33} (\bibinfo{year}{2018}), \bibinfo{pages}{e17}.
\newblock


\bibitem[\protect\citeauthoryear{Arjona-Medina, Gillhofer, Widrich,
  Unterthiner, Brandstetter, and Hochreiter}{Arjona-Medina
  et~al\mbox{.}}{2019}]%
        {arjona2019rudder}
\bibfield{author}{\bibinfo{person}{Jose~A Arjona-Medina},
  \bibinfo{person}{Michael Gillhofer}, \bibinfo{person}{Michael Widrich},
  \bibinfo{person}{Thomas Unterthiner}, \bibinfo{person}{Johannes
  Brandstetter}, {and} \bibinfo{person}{Sepp Hochreiter}.}
  \bibinfo{year}{2019}\natexlab{}.
\newblock \showarticletitle{Rudder: Return decomposition for delayed rewards}.
\newblock \bibinfo{journal}{\emph{NeurIPS}}  \bibinfo{volume}{32}
  (\bibinfo{year}{2019}).
\newblock


\bibitem[\protect\citeauthoryear{Beal, Changder, Norman, and Ramchurn}{Beal
  et~al\mbox{.}}{2020}]%
        {beal2020learning}
\bibfield{author}{\bibinfo{person}{Ryan Beal}, \bibinfo{person}{Narayan
  Changder}, \bibinfo{person}{Timothy Norman}, {and} \bibinfo{person}{Sarvapali
  Ramchurn}.} \bibinfo{year}{2020}\natexlab{}.
\newblock \showarticletitle{Learning the value of teamwork to form efficient
  teams}. In \bibinfo{booktitle}{\emph{Proceedings of the AAAI Conference on
  Artificial Intelligence}}, Vol.~\bibinfo{volume}{34}.
  \bibinfo{pages}{7063--7070}.
\newblock


\bibitem[\protect\citeauthoryear{Beetz, Hoyningen-Huene, Bandouch,
  Kirchlechner, Gedikli, and Maldonado}{Beetz et~al\mbox{.}}{2006}]%
        {beetz2006camera}
\bibfield{author}{\bibinfo{person}{Michael Beetz}, \bibinfo{person}{Nico~v
  Hoyningen-Huene}, \bibinfo{person}{Jan Bandouch}, \bibinfo{person}{Bernhard
  Kirchlechner}, \bibinfo{person}{Suat Gedikli}, {and} \bibinfo{person}{Alexis
  Maldonado}.} \bibinfo{year}{2006}\natexlab{}.
\newblock \showarticletitle{Camera-based observation of football games for
  analyzing multi-agent activities}. In \bibinfo{booktitle}{\emph{Proceedings
  of the fifth international joint conference on Autonomous agents and
  multiagent systems}}. \bibinfo{pages}{42--49}.
\newblock


\bibitem[\protect\citeauthoryear{Bialkowski, Lucey, Carr, Yue, Sridharan, and
  Matthews}{Bialkowski et~al\mbox{.}}{2014}]%
        {bialkowski2014large}
\bibfield{author}{\bibinfo{person}{Alina Bialkowski}, \bibinfo{person}{Patrick
  Lucey}, \bibinfo{person}{Peter Carr}, \bibinfo{person}{Yisong Yue},
  \bibinfo{person}{Sridha Sridharan}, {and} \bibinfo{person}{Iain Matthews}.}
  \bibinfo{year}{2014}\natexlab{}.
\newblock \showarticletitle{Large-scale analysis of soccer matches using
  spatiotemporal tracking data}. In \bibinfo{booktitle}{\emph{2014 IEEE
  international conference on data mining}}. IEEE, \bibinfo{pages}{725--730}.
\newblock


\bibitem[\protect\citeauthoryear{Bouveret and Lang}{Bouveret and Lang}{2014}]%
        {bouveret2014manipulating}
\bibfield{author}{\bibinfo{person}{Sylvain Bouveret} {and}
  \bibinfo{person}{J{\'e}r{\^o}me Lang}.} \bibinfo{year}{2014}\natexlab{}.
\newblock \showarticletitle{Manipulating picking sequences.}. In
  \bibinfo{booktitle}{\emph{ECAI}}, Vol.~\bibinfo{volume}{14}.
  \bibinfo{pages}{141--146}.
\newblock


\bibitem[\protect\citeauthoryear{Brams and Straffin~Jr}{Brams and
  Straffin~Jr}{1979}]%
        {brams1979prisoners}
\bibfield{author}{\bibinfo{person}{Steven~J Brams} {and}
  \bibinfo{person}{Philip~D Straffin~Jr}.} \bibinfo{year}{1979}\natexlab{}.
\newblock \showarticletitle{Prisoners' dilemma and professional sports drafts}.
\newblock \bibinfo{journal}{\emph{The American Mathematical Monthly}}
  \bibinfo{volume}{86}, \bibinfo{number}{2} (\bibinfo{year}{1979}),
  \bibinfo{pages}{80--88}.
\newblock


\bibitem[\protect\citeauthoryear{Bransen and Van~Haaren}{Bransen and
  Van~Haaren}{2020}]%
        {bransen2020player}
\bibfield{author}{\bibinfo{person}{Lotte Bransen} {and} \bibinfo{person}{Jan
  Van~Haaren}.} \bibinfo{year}{2020}\natexlab{}.
\newblock \showarticletitle{Player chemistry: Striving for a perfectly balanced
  soccer team}.
\newblock \bibinfo{journal}{\emph{Sports Analytics Conference}}
  (\bibinfo{year}{2020}).
\newblock


\bibitem[\protect\citeauthoryear{Dafoe, Bachrach, Hadfield, Horvitz, Larson,
  and Graepel}{Dafoe et~al\mbox{.}}{2021}]%
        {DafoeNature2021}
\bibfield{author}{\bibinfo{person}{Allan Dafoe}, \bibinfo{person}{Yoram
  Bachrach}, \bibinfo{person}{Gillian Hadfield}, \bibinfo{person}{Eric
  Horvitz}, \bibinfo{person}{Kate Larson}, {and} \bibinfo{person}{Thore
  Graepel}.} \bibinfo{year}{2021}\natexlab{}.
\newblock \showarticletitle{Cooperative {AI}: machines must learn to find
  common ground}.
\newblock \bibinfo{journal}{\emph{Nature}}  \bibinfo{volume}{593}
  (\bibinfo{year}{2021}), \bibinfo{pages}{33--36}.
\newblock


\bibitem[\protect\citeauthoryear{Derks and Peters}{Derks and Peters}{1993}]%
        {derks1993shapley}
\bibfield{author}{\bibinfo{person}{Jean Derks} {and} \bibinfo{person}{Hans
  Peters}.} \bibinfo{year}{1993}\natexlab{}.
\newblock \showarticletitle{A Shapley value for games with restricted
  coalitions}.
\newblock \bibinfo{journal}{\emph{International Journal of Game Theory}}
  \bibinfo{volume}{21}, \bibinfo{number}{4} (\bibinfo{year}{1993}),
  \bibinfo{pages}{351--360}.
\newblock


\bibitem[\protect\citeauthoryear{Durugkar, Liebman, and Stone}{Durugkar
  et~al\mbox{.}}{2020}]%
        {Durugkar2020BalancingIP}
\bibfield{author}{\bibinfo{person}{Ishan Durugkar}, \bibinfo{person}{E.
  Liebman}, {and} \bibinfo{person}{P. Stone}.} \bibinfo{year}{2020}\natexlab{}.
\newblock \showarticletitle{Balancing Individual Preferences and Shared
  Objectives in Multiagent Reinforcement Learning}. In
  \bibinfo{booktitle}{\emph{IJCAI}}.
\newblock


\bibitem[\protect\citeauthoryear{Elitzur}{Elitzur}{2020}]%
        {elitzur2020data}
\bibfield{author}{\bibinfo{person}{Ramy Elitzur}.}
  \bibinfo{year}{2020}\natexlab{}.
\newblock \showarticletitle{Data analytics effects in major league baseball}.
\newblock \bibinfo{journal}{\emph{Omega}}  \bibinfo{volume}{90}
  (\bibinfo{year}{2020}), \bibinfo{pages}{102001}.
\newblock


\bibitem[\protect\citeauthoryear{Ellis}{Ellis}{1983}]%
        {ellis1983similarities}
\bibfield{author}{\bibinfo{person}{M Ellis}.} \bibinfo{year}{1983}\natexlab{}.
\newblock \showarticletitle{Similarities and differences in games: A system for
  classification}. In \bibinfo{booktitle}{\emph{International association for
  physical education in higher education Conference}}.
\newblock


\bibitem[\protect\citeauthoryear{Fern{\'a}ndez, Bornn, and
  Cervone}{Fern{\'a}ndez et~al\mbox{.}}{2021}]%
        {fernandez2021framework}
\bibfield{author}{\bibinfo{person}{Javier Fern{\'a}ndez}, \bibinfo{person}{Luke
  Bornn}, {and} \bibinfo{person}{Daniel Cervone}.}
  \bibinfo{year}{2021}\natexlab{}.
\newblock \showarticletitle{A framework for the fine-grained evaluation of the
  instantaneous expected value of soccer possessions}.
\newblock \bibinfo{journal}{\emph{Machine Learning}} \bibinfo{volume}{110},
  \bibinfo{number}{6} (\bibinfo{year}{2021}), \bibinfo{pages}{1389--1427}.
\newblock


\bibitem[\protect\citeauthoryear{Fisac, Bronstein, Stefansson, Sadigh, Sastry,
  and Dragan}{Fisac et~al\mbox{.}}{2019}]%
        {fisac2019hierarchical}
\bibfield{author}{\bibinfo{person}{Jaime~F Fisac}, \bibinfo{person}{Eli
  Bronstein}, \bibinfo{person}{Elis Stefansson}, \bibinfo{person}{Dorsa
  Sadigh}, \bibinfo{person}{S~Shankar Sastry}, {and} \bibinfo{person}{Anca~D
  Dragan}.} \bibinfo{year}{2019}\natexlab{}.
\newblock \showarticletitle{Hierarchical game-theoretic planning for autonomous
  vehicles}. In \bibinfo{booktitle}{\emph{ICRA}}. IEEE,
  \bibinfo{pages}{9590--9596}.
\newblock


\bibitem[\protect\citeauthoryear{Garner, Humphrey, and Simkins}{Garner
  et~al\mbox{.}}{2016}]%
        {garner2016business}
\bibfield{author}{\bibinfo{person}{Jacqueline Garner},
  \bibinfo{person}{Phillip~R Humphrey}, {and} \bibinfo{person}{Betty Simkins}.}
  \bibinfo{year}{2016}\natexlab{}.
\newblock \showarticletitle{The business of sport and the sport of business: A
  review of the compensation literature in finance and sports}.
\newblock \bibinfo{journal}{\emph{International Review of Financial Analysis}}
  \bibinfo{volume}{47} (\bibinfo{year}{2016}), \bibinfo{pages}{197--204}.
\newblock


\bibitem[\protect\citeauthoryear{Goes, Kempe, Meerhoff, and Lemmink}{Goes
  et~al\mbox{.}}{2019}]%
        {goes2019not}
\bibfield{author}{\bibinfo{person}{Floris~R Goes}, \bibinfo{person}{Matthias
  Kempe}, \bibinfo{person}{Laurentius~A Meerhoff}, {and}
  \bibinfo{person}{Koen~APM Lemmink}.} \bibinfo{year}{2019}\natexlab{}.
\newblock \showarticletitle{Not every pass can be an assist: a data-driven
  model to measure pass effectiveness in professional soccer matches}.
\newblock \bibinfo{journal}{\emph{Big data}} \bibinfo{volume}{7},
  \bibinfo{number}{1} (\bibinfo{year}{2019}), \bibinfo{pages}{57--70}.
\newblock


\bibitem[\protect\citeauthoryear{Hu, Xie, Liang, and Chang}{Hu
  et~al\mbox{.}}{2022}]%
        {hu2022policy}
\bibfield{author}{\bibinfo{person}{Siyi Hu}, \bibinfo{person}{Chuanlong Xie},
  \bibinfo{person}{Xiaodan Liang}, {and} \bibinfo{person}{Xiaojun Chang}.}
  \bibinfo{year}{2022}\natexlab{}.
\newblock \showarticletitle{Policy diagnosis via measuring role diversity in
  cooperative multi-agent {RL}}. In \bibinfo{booktitle}{\emph{ICML}}.
  \bibinfo{pages}{9041--9071}.
\newblock


\bibitem[\protect\citeauthoryear{Le, Yue, Carr, and Lucey}{Le
  et~al\mbox{.}}{2017}]%
        {le2017coordinated}
\bibfield{author}{\bibinfo{person}{Hoang~M Le}, \bibinfo{person}{Yisong Yue},
  \bibinfo{person}{Peter Carr}, {and} \bibinfo{person}{Patrick Lucey}.}
  \bibinfo{year}{2017}\natexlab{}.
\newblock \showarticletitle{Coordinated multi-agent imitation learning}. In
  \bibinfo{booktitle}{\emph{International Conference on Machine Learning}}.
  PMLR, \bibinfo{pages}{1995--2003}.
\newblock


\bibitem[\protect\citeauthoryear{Ledezma, Aler, Sanchis, and Borrajo}{Ledezma
  et~al\mbox{.}}{2009}]%
        {ledezma2009ombo}
\bibfield{author}{\bibinfo{person}{Agapito Ledezma}, \bibinfo{person}{Ricardo
  Aler}, \bibinfo{person}{Araceli Sanchis}, {and} \bibinfo{person}{Daniel
  Borrajo}.} \bibinfo{year}{2009}\natexlab{}.
\newblock \showarticletitle{OMBO: An opponent modeling approach}.
\newblock \bibinfo{journal}{\emph{{AI} Communications}} \bibinfo{volume}{22},
  \bibinfo{number}{1} (\bibinfo{year}{2009}), \bibinfo{pages}{21--35}.
\newblock


\bibitem[\protect\citeauthoryear{Lewis}{Lewis}{2004}]%
        {lewis2004moneyball}
\bibfield{author}{\bibinfo{person}{Michael Lewis}.}
  \bibinfo{year}{2004}\natexlab{}.
\newblock \bibinfo{booktitle}{\emph{Moneyball: The art of winning an unfair
  game}}.
\newblock \bibinfo{publisher}{WW Norton \& Company}.
\newblock


\bibitem[\protect\citeauthoryear{Liemhetcharat and Luo}{Liemhetcharat and
  Luo}{2015}]%
        {liemhetcharat2015applying}
\bibfield{author}{\bibinfo{person}{Somchaya Liemhetcharat} {and}
  \bibinfo{person}{Yicheng Luo}.} \bibinfo{year}{2015}\natexlab{}.
\newblock \showarticletitle{Applying the Synergy Graph Model to Human
  Basketball.}. In \bibinfo{booktitle}{\emph{AAMAS}}.
  \bibinfo{pages}{1695--1696}.
\newblock


\bibitem[\protect\citeauthoryear{Liu, Schulte, Poupart, Rudd, and Javan}{Liu
  et~al\mbox{.}}{2020}]%
        {liu2020learning}
\bibfield{author}{\bibinfo{person}{Guiliang Liu}, \bibinfo{person}{Oliver
  Schulte}, \bibinfo{person}{Pascal Poupart}, \bibinfo{person}{Mike Rudd},
  {and} \bibinfo{person}{Mehrsan Javan}.} \bibinfo{year}{2020}\natexlab{}.
\newblock \showarticletitle{Learning agent representations for ice hockey}.
\newblock \bibinfo{journal}{\emph{Advances in Neural Information Processing
  Systems}}  \bibinfo{volume}{33} (\bibinfo{year}{2020}),
  \bibinfo{pages}{18704--18715}.
\newblock


\bibitem[\protect\citeauthoryear{Ljung, Carlsson, and Lambrix}{Ljung
  et~al\mbox{.}}{2018}]%
        {Ljung2018PlayerPV}
\bibfield{author}{\bibinfo{person}{Dennis Ljung}, \bibinfo{person}{Niklas
  Carlsson}, {and} \bibinfo{person}{P. Lambrix}.}
  \bibinfo{year}{2018}\natexlab{}.
\newblock \showarticletitle{Player Pairs Valuation in Ice Hockey}. In
  \bibinfo{booktitle}{\emph{MLSA@PKDD/ECML}}.
\newblock


\bibitem[\protect\citeauthoryear{Lucey, Bialkowski, Carr, Foote, and
  Matthews}{Lucey et~al\mbox{.}}{2012}]%
        {lucey2012characterizing}
\bibfield{author}{\bibinfo{person}{Patrick Lucey}, \bibinfo{person}{Alina
  Bialkowski}, \bibinfo{person}{Peter Carr}, \bibinfo{person}{Eric Foote},
  {and} \bibinfo{person}{Iain Matthews}.} \bibinfo{year}{2012}\natexlab{}.
\newblock \showarticletitle{Characterizing multi-agent team behavior from
  partial team tracings: Evidence from the english premier league}. In
  \bibinfo{booktitle}{\emph{Proceedings of the AAAI Conference on Artificial
  Intelligence}}, Vol.~\bibinfo{volume}{26}. \bibinfo{pages}{1387--1393}.
\newblock


\bibitem[\protect\citeauthoryear{Pourmehr and Dadkhah}{Pourmehr and
  Dadkhah}{2011}]%
        {pourmehr2011overview}
\bibfield{author}{\bibinfo{person}{Shokoofeh Pourmehr} {and}
  \bibinfo{person}{Chitra Dadkhah}.} \bibinfo{year}{2011}\natexlab{}.
\newblock \showarticletitle{An overview on opponent modeling in RoboCup soccer
  simulation 2D}.
\newblock \bibinfo{journal}{\emph{Robot Soccer World Cup}}
  (\bibinfo{year}{2011}), \bibinfo{pages}{402--414}.
\newblock


\bibitem[\protect\citeauthoryear{Raabe, Nabben, and Memmert}{Raabe
  et~al\mbox{.}}{2022}]%
        {raabe2022graph}
\bibfield{author}{\bibinfo{person}{Dominik Raabe}, \bibinfo{person}{Reinhard
  Nabben}, {and} \bibinfo{person}{Daniel Memmert}.}
  \bibinfo{year}{2022}\natexlab{}.
\newblock \showarticletitle{Graph representations for the analysis of
  multi-agent spatiotemporal sports data}.
\newblock \bibinfo{journal}{\emph{Applied Intelligence}}
  (\bibinfo{year}{2022}), \bibinfo{pages}{1--21}.
\newblock


\bibitem[\protect\citeauthoryear{Radke, Brecht, and Radke}{Radke
  et~al\mbox{.}}{2022a}]%
        {radke2022identifying}
\bibfield{author}{\bibinfo{person}{David Radke}, \bibinfo{person}{Tim Brecht},
  {and} \bibinfo{person}{Daniel Radke}.} \bibinfo{year}{2022}\natexlab{a}.
\newblock \showarticletitle{Identifying Completed Pass Types and Improving
  Passing Lane Models}. In \bibinfo{booktitle}{\emph{Link{\"o}ping Hockey
  Analytics Conference}}. \bibinfo{pages}{71--86}.
\newblock


\bibitem[\protect\citeauthoryear{Radke, Larson, and Brecht}{Radke
  et~al\mbox{.}}{2022b}]%
        {Radke2022Exploring}
\bibfield{author}{\bibinfo{person}{David Radke}, \bibinfo{person}{Kate Larson},
  {and} \bibinfo{person}{Tim Brecht}.} \bibinfo{year}{2022}\natexlab{b}.
\newblock \showarticletitle{Exploring the Benefits of Teams in Multiagent
  Learning}. In \bibinfo{booktitle}{\emph{IJCAI}}.
\newblock


\bibitem[\protect\citeauthoryear{Radke, Larson, and Brecht}{Radke
  et~al\mbox{.}}{2022c}]%
        {radke2022importance}
\bibfield{author}{\bibinfo{person}{David Radke}, \bibinfo{person}{Kate Larson},
  {and} \bibinfo{person}{Tim Brecht}.} \bibinfo{year}{2022}\natexlab{c}.
\newblock \showarticletitle{The Importance of Credo in Multiagent Learning}.
\newblock \bibinfo{journal}{\emph{ALA Workshop at AAMAS}}
  (\bibinfo{year}{2022}).
\newblock


\bibitem[\protect\citeauthoryear{Radke, Radke, Brecht, and Pawelczyk}{Radke
  et~al\mbox{.}}{2021}]%
        {Radke2021Passing}
\bibfield{author}{\bibinfo{person}{D.~T. Radke}, \bibinfo{person}{D.~L. Radke},
  \bibinfo{person}{T. Brecht}, {and} \bibinfo{person}{A. Pawelczyk}.}
  \bibinfo{year}{2021}\natexlab{}.
\newblock \showarticletitle{Passing and Pressure Metrics in Ice Hockey}.
\newblock \bibinfo{journal}{\emph{Workshop of AI for Sports Analytics}}
  (\bibinfo{year}{2021}).
\newblock


\bibitem[\protect\citeauthoryear{Rahimian and Toka}{Rahimian and Toka}{2022}]%
        {rahimian2022optical}
\bibfield{author}{\bibinfo{person}{Pegah Rahimian} {and}
  \bibinfo{person}{Laszlo Toka}.} \bibinfo{year}{2022}\natexlab{}.
\newblock \showarticletitle{Optical tracking in team sports}.
\newblock \bibinfo{journal}{\emph{Journal of Quantitative Analysis in Sports}}
  \bibinfo{volume}{18}, \bibinfo{number}{1} (\bibinfo{year}{2022}),
  \bibinfo{pages}{35--57}.
\newblock


\bibitem[\protect\citeauthoryear{Rahwan, Michalak, Wooldridge, and
  Jennings}{Rahwan et~al\mbox{.}}{2015}]%
        {rahwan2015coalition}
\bibfield{author}{\bibinfo{person}{Talal Rahwan}, \bibinfo{person}{Tomasz~P
  Michalak}, \bibinfo{person}{Michael Wooldridge}, {and}
  \bibinfo{person}{Nicholas~R Jennings}.} \bibinfo{year}{2015}\natexlab{}.
\newblock \showarticletitle{Coalition structure generation: A survey}.
\newblock \bibinfo{journal}{\emph{Artificial Intelligence}}
  \bibinfo{volume}{229} (\bibinfo{year}{2015}), \bibinfo{pages}{139--174}.
\newblock


\bibitem[\protect\citeauthoryear{Rashid, Samvelyan, Schroeder, Farquhar,
  Foerster, and Whiteson}{Rashid et~al\mbox{.}}{2018}]%
        {rashid2018qmix}
\bibfield{author}{\bibinfo{person}{Tabish Rashid}, \bibinfo{person}{Mikayel
  Samvelyan}, \bibinfo{person}{Christian Schroeder}, \bibinfo{person}{Gregory
  Farquhar}, \bibinfo{person}{Jakob Foerster}, {and} \bibinfo{person}{Shimon
  Whiteson}.} \bibinfo{year}{2018}\natexlab{}.
\newblock \showarticletitle{Qmix: Monotonic value function factorisation for
  deep multi-agent reinforcement learning}. In
  \bibinfo{booktitle}{\emph{ICML}}. \bibinfo{pages}{4295--4304}.
\newblock


\bibitem[\protect\citeauthoryear{Rein and Memmert}{Rein and Memmert}{2016}]%
        {rein2016big}
\bibfield{author}{\bibinfo{person}{Robert Rein} {and} \bibinfo{person}{Daniel
  Memmert}.} \bibinfo{year}{2016}\natexlab{}.
\newblock \showarticletitle{Big data and tactical analysis in elite soccer:
  future challenges and opportunities for sports science}.
\newblock \bibinfo{journal}{\emph{SpringerPlus}} \bibinfo{volume}{5},
  \bibinfo{number}{1} (\bibinfo{year}{2016}), \bibinfo{pages}{1--13}.
\newblock


\bibitem[\protect\citeauthoryear{Ritchie, Harell, and Shreeves}{Ritchie
  et~al\mbox{.}}{2022}]%
        {ritchie2022pass}
\bibfield{author}{\bibinfo{person}{Robyn Ritchie}, \bibinfo{person}{Alon
  Harell}, {and} \bibinfo{person}{Phillip Shreeves}.}
  \bibinfo{year}{2022}\natexlab{}.
\newblock \showarticletitle{Pass Evaluation in Women's Olympic Ice Hockey}. In
  \bibinfo{booktitle}{\emph{Proceedings of the 5th International ACM Workshop
  on Multimedia Content Analysis in Sports}}. \bibinfo{pages}{65--73}.
\newblock


\bibitem[\protect\citeauthoryear{Sampaio, McGarry, Calleja-Gonz{\'a}lez,
  Jim{\'e}nez~S{\'a}iz, Schelling i~del Alc{\'a}zar, and Balciunas}{Sampaio
  et~al\mbox{.}}{2015}]%
        {sampaio2015exploring}
\bibfield{author}{\bibinfo{person}{Jaime Sampaio}, \bibinfo{person}{Tim
  McGarry}, \bibinfo{person}{Julio Calleja-Gonz{\'a}lez},
  \bibinfo{person}{Sergio Jim{\'e}nez~S{\'a}iz}, \bibinfo{person}{Xavi
  Schelling i~del Alc{\'a}zar}, {and} \bibinfo{person}{Mindaugas Balciunas}.}
  \bibinfo{year}{2015}\natexlab{}.
\newblock \showarticletitle{Exploring game performance in the National
  Basketball Association using player tracking data}.
\newblock \bibinfo{journal}{\emph{PloS one}} \bibinfo{volume}{10},
  \bibinfo{number}{7} (\bibinfo{year}{2015}), \bibinfo{pages}{e0132894}.
\newblock


\bibitem[\protect\citeauthoryear{Santos, Santos, Pacheco, and Levin}{Santos
  et~al\mbox{.}}{2021}]%
        {Santos2021SocialNI}
\bibfield{author}{\bibinfo{person}{F. Santos}, \bibinfo{person}{F.~C. Santos},
  \bibinfo{person}{J. Pacheco}, {and} \bibinfo{person}{S. Levin}.}
  \bibinfo{year}{2021}\natexlab{}.
\newblock \showarticletitle{Social Network Interventions to Prevent
  Reciprocity-driven Polarization}. In \bibinfo{booktitle}{\emph{AAMAS}}.
\newblock


\bibitem[\protect\citeauthoryear{Schr{\"o}der, Hoey, and Rogers}{Schr{\"o}der
  et~al\mbox{.}}{2016}]%
        {schroder2016modeling}
\bibfield{author}{\bibinfo{person}{Tobias Schr{\"o}der}, \bibinfo{person}{Jesse
  Hoey}, {and} \bibinfo{person}{Kimberly~B Rogers}.}
  \bibinfo{year}{2016}\natexlab{}.
\newblock \showarticletitle{Modeling dynamic identities and uncertainty in
  social interactions: Bayesian affect control theory}.
\newblock \bibinfo{journal}{\emph{American Sociological Review}}
  \bibinfo{volume}{81}, \bibinfo{number}{4} (\bibinfo{year}{2016}),
  \bibinfo{pages}{828--855}.
\newblock


\bibitem[\protect\citeauthoryear{Schuckers}{Schuckers}{2011}]%
        {schuckers2011s}
\bibfield{author}{\bibinfo{person}{Michael~E Schuckers}.}
  \bibinfo{year}{2011}\natexlab{}.
\newblock \showarticletitle{What's An NHL Draft Pick Worth? A Value Pick Chart
  for the National Hockey League}.
\newblock \bibinfo{journal}{\emph{Statistical Sports Consulting}}
  (\bibinfo{year}{2011}).
\newblock


\bibitem[\protect\citeauthoryear{Schulte, Khademi, Gholami, Zhao, Javan, and
  Desaulniers}{Schulte et~al\mbox{.}}{2017}]%
        {schulte2017markov}
\bibfield{author}{\bibinfo{person}{Oliver Schulte}, \bibinfo{person}{Mahmoud
  Khademi}, \bibinfo{person}{Sajjad Gholami}, \bibinfo{person}{Zeyu Zhao},
  \bibinfo{person}{Mehrsan Javan}, {and} \bibinfo{person}{Philippe
  Desaulniers}.} \bibinfo{year}{2017}\natexlab{}.
\newblock \showarticletitle{A Markov Game model for valuing actions, locations,
  and team performance in ice hockey}.
\newblock \bibinfo{journal}{\emph{Data Mining and Knowledge Discovery}}
  \bibinfo{volume}{31}, \bibinfo{number}{6} (\bibinfo{year}{2017}),
  \bibinfo{pages}{1735--1757}.
\newblock


\bibitem[\protect\citeauthoryear{Schwind, Demirovic, Inoue, and
  Lagniez}{Schwind et~al\mbox{.}}{2021}]%
        {schwind2021partial}
\bibfield{author}{\bibinfo{person}{Nicolas Schwind}, \bibinfo{person}{Emir
  Demirovic}, \bibinfo{person}{Katsumi Inoue}, {and}
  \bibinfo{person}{Jean-Marie Lagniez}.} \bibinfo{year}{2021}\natexlab{}.
\newblock \showarticletitle{Partial Robustness in Team Formation: Bridging the
  Gap between Robustness and Resilience.}. In
  \bibinfo{booktitle}{\emph{AAMAS}}, Vol.~\bibinfo{volume}{21}.
  \bibinfo{pages}{20th}.
\newblock


\bibitem[\protect\citeauthoryear{Simon}{Simon}{1990}]%
        {simon1990bounded}
\bibfield{author}{\bibinfo{person}{Herbert~A Simon}.}
  \bibinfo{year}{1990}\natexlab{}.
\newblock \showarticletitle{Bounded rationality}.
\newblock In \bibinfo{booktitle}{\emph{Utility and probability}}.
  \bibinfo{publisher}{Springer}, \bibinfo{pages}{15--18}.
\newblock


\bibitem[\protect\citeauthoryear{Spearman}{Spearman}{2018}]%
        {spearman2018beyond}
\bibfield{author}{\bibinfo{person}{William Spearman}.}
  \bibinfo{year}{2018}\natexlab{}.
\newblock \showarticletitle{Beyond expected goals}. In
  \bibinfo{booktitle}{\emph{Proceedings of the 12th MIT sloan sports analytics
  conference}}. \bibinfo{pages}{1--17}.
\newblock


\bibitem[\protect\citeauthoryear{Stone, Riley, and Veloso}{Stone
  et~al\mbox{.}}{2000}]%
        {stone2000defining}
\bibfield{author}{\bibinfo{person}{Peter Stone}, \bibinfo{person}{Patrick
  Riley}, {and} \bibinfo{person}{Manuela Veloso}.}
  \bibinfo{year}{2000}\natexlab{}.
\newblock \showarticletitle{Defining and using ideal teammate and opponent
  agent models}. In \bibinfo{booktitle}{\emph{AAAI/IAAI}}.
  \bibinfo{pages}{1040--1045}.
\newblock


\bibitem[\protect\citeauthoryear{Tuyls, Omidshafiei, Muller, Wang, Connor,
  Hennes, Graham, Spearman, Waskett, Steel, et~al\mbox{.}}{Tuyls
  et~al\mbox{.}}{2021}]%
        {tuyls2021game}
\bibfield{author}{\bibinfo{person}{Karl Tuyls}, \bibinfo{person}{Shayegan
  Omidshafiei}, \bibinfo{person}{Paul Muller}, \bibinfo{person}{Zhe Wang},
  \bibinfo{person}{Jerome Connor}, \bibinfo{person}{Daniel Hennes},
  \bibinfo{person}{Ian Graham}, \bibinfo{person}{William Spearman},
  \bibinfo{person}{Tim Waskett}, \bibinfo{person}{Dafydd Steel},
  {et~al\mbox{.}}} \bibinfo{year}{2021}\natexlab{}.
\newblock \showarticletitle{Game Plan: What {AI} can do for Football, and What
  Football can do for AI}.
\newblock \bibinfo{journal}{\emph{Journal of Artificial Intelligence Research}}
   \bibinfo{volume}{71} (\bibinfo{year}{2021}), \bibinfo{pages}{41--88}.
\newblock


\bibitem[\protect\citeauthoryear{Van Der~Hoek, Jamroga, and Wooldridge}{Van
  Der~Hoek et~al\mbox{.}}{2005}]%
        {van2005logic}
\bibfield{author}{\bibinfo{person}{Wiebe Van Der~Hoek},
  \bibinfo{person}{Wojciech Jamroga}, {and} \bibinfo{person}{Michael
  Wooldridge}.} \bibinfo{year}{2005}\natexlab{}.
\newblock \showarticletitle{A logic for strategic reasoning}. In
  \bibinfo{booktitle}{\emph{Proceedings of the fourth international joint
  conference on Autonomous agents and multiagent systems}}.
  \bibinfo{pages}{157--164}.
\newblock


\bibitem[\protect\citeauthoryear{Vats, Fani, Clausi, and Zelek}{Vats
  et~al\mbox{.}}{2022}]%
        {vats2022evaluating}
\bibfield{author}{\bibinfo{person}{Kanav Vats}, \bibinfo{person}{Mehrnaz Fani},
  \bibinfo{person}{David~A Clausi}, {and} \bibinfo{person}{John~S Zelek}.}
  \bibinfo{year}{2022}\natexlab{}.
\newblock \showarticletitle{Evaluating deep tracking models for player tracking
  in broadcast ice hockey video}.
\newblock \bibinfo{journal}{\emph{arXiv preprint arXiv:2205.10949}}
  (\bibinfo{year}{2022}).
\newblock


\bibitem[\protect\citeauthoryear{Visser, Dr{\"u}cker, H{\"u}bner, Schmidt, and
  Weland}{Visser et~al\mbox{.}}{2000}]%
        {visser2000recognizing}
\bibfield{author}{\bibinfo{person}{Ubbo Visser}, \bibinfo{person}{Christian
  Dr{\"u}cker}, \bibinfo{person}{Sebastian H{\"u}bner}, \bibinfo{person}{Esko
  Schmidt}, {and} \bibinfo{person}{Hans-Georg Weland}.}
  \bibinfo{year}{2000}\natexlab{}.
\newblock \showarticletitle{Recognizing formations in opponent teams}. In
  \bibinfo{booktitle}{\emph{Robot Soccer World Cup}}. Springer,
  \bibinfo{pages}{391--396}.
\newblock


\bibitem[\protect\citeauthoryear{Williamson and Cox}{Williamson and
  Cox}{2014}]%
        {williamson2014distributed}
\bibfield{author}{\bibinfo{person}{Kellie Williamson} {and}
  \bibinfo{person}{Rochelle Cox}.} \bibinfo{year}{2014}\natexlab{}.
\newblock \showarticletitle{Distributed cognition in sports teams: Explaining
  successful and expert performance}.
\newblock \bibinfo{journal}{\emph{Educational Philosophy and Theory}}
  \bibinfo{volume}{46}, \bibinfo{number}{6} (\bibinfo{year}{2014}),
  \bibinfo{pages}{640--654}.
\newblock


\bibitem[\protect\citeauthoryear{Yan, Kroer, and Peysakhovich}{Yan
  et~al\mbox{.}}{2020}]%
        {Yan2020EvaluatingAR}
\bibfield{author}{\bibinfo{person}{Tom Yan}, \bibinfo{person}{Christian Kroer},
  {and} \bibinfo{person}{A. Peysakhovich}.} \bibinfo{year}{2020}\natexlab{}.
\newblock \showarticletitle{Evaluating and Rewarding Teamwork Using Cooperative
  Game Abstractions}.
\newblock \bibinfo{journal}{\emph{NeurIPS}} (\bibinfo{year}{2020}).
\newblock


\end{thebibliography}

%% else use the following coding to input the bibitems directly in the
%% TeX file.

\end{document}
\endinput
%%
%% End of file `elsarticle-template-num.tex'.
