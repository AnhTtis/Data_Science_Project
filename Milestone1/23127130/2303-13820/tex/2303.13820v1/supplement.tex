\documentclass[%
 reprint,
superscriptaddress,
%groupedaddress,
%unsortedaddress,
%runinaddress,
%frontmatterverbose, 
%preprint,
%preprintnumbers,
%nofootinbib,
%nobibnotes,
%bibnotes,
 amsmath,amssymb,
 aps,
%pra,
%prb,
%rmp,
%prstab,
%prstper,
%floatfix,
]{revtex4-2}

\usepackage{graphicx}% Include figure files
\usepackage{dcolumn}% Align table columns on decimal point
\usepackage{bm}% bold math
\usepackage{hyperref}% add hypertext capabilities
\usepackage{chngcntr}
\usepackage{color}


\bibliographystyle{apsrev4-2}


%\author{} %leave this blank
%% DO NOT ADD AUTHOR INFORMATION HERE; IT WILL BE ADDED DURING PRODUCTION

\begin{document}
\title{Supplementary Material for Observation of Colossal Terahertz Magnetoresistance and Magnetocapacitance in a Perovskite Manganite}

\author{Fuyang~Tay}
\author{Swati~Chaudhary}
\author{Jiaming~He}
\author{Nicolas~Marquez~Peraca}
\author{Andrey~Baydin}
\author{Gregory~A.~Fiete}
\author{Jianshi~Zhou}
\author{Junichiro~Kono}

\maketitle

\section{DC resistivity measurement}
The temperature dependence of DC resistivity in Fig.~\ref{fig:DCrhovsT} indicated a gap of 19.6\,meV (or 4.7\,THz) near 30\,K.
\begin{figure}[htbp]
\centering\includegraphics[width=0.4\textwidth]{myfigS1.png}
\caption{Temperature dependence of DC resistivity for a different La$_{0.875}$Sr$_{0.125}$MnO$_3$ crystal with a small current.}
\label{fig:DCrhovsT}
\end{figure}

\section{Polarization dependence of terahertz transmission}
\label{sec:anisotropy}
We investigated THz transmittance spectra when the THz electric field polarization was parallel to the $a$ and $b$ axes. Figure~\ref{fig:anisotropy} shows transmittance spectra at 4.2\,K. The transmittance of a THz wave with the electric field polarized along the $a$ axis was much higher than that of a THz wave polarized along the $b$ axis. The anisotropy existed at all temperatures below 250\,K. Such behavior has been previously reported~\cite{PimenovEtAl1999PRB}.

\begin{figure}[htbp]
\centering\includegraphics[width=0.25\textwidth]{myfigS2.pdf}
\caption{Transmittance spectra when the THz electric field was polarized along the $a$ and $b$ axes, represented by red and blue lines, respectively.}
\label{fig:anisotropy}
\end{figure}

\section{Power-law fittings}
\label{sec:powerlaw}
The frequency dependence of the complex permittivity (or dielectric constant) of dielectric systems, in particular amorphous and composite materials, have been found to follow a universal behavior~\cite{Jonscher1977N,Jonscher1999JPDAP}. In general, one can express $\sigma_1(\omega)$ and $\varepsilon_1(\omega)$ as
\begin{align}
    \sigma_1(\omega) &= \sigma_{\text{DC}} + A\omega^s,\label{eqn:powerlaw_sigma}\\
    \varepsilon_1(\omega) &= A\omega^{s-1}\varepsilon_0^{-1}\tan{\left (\frac{s\pi}{2} \right )},
    \label{eqn:powerlaw_eps}
\end{align}
where $\sigma_{\text{DC}}$ is the DC conductivity, $\varepsilon_0$ is the vacuum permittivity, and $A$ and $s$ are fitting parameters for both $\sigma_1(\omega)$ and $\varepsilon_1(\omega)$. However, we found that the $\sigma_1(\omega)$ and $\varepsilon_1(\omega)$ obtained from our experiments cannot be fit well simultaneously by Eqns.~(\ref{eqn:powerlaw_sigma}-\ref{eqn:powerlaw_eps}). Instead, good agreement was obtained when $\sigma_1(\omega)$ and $\varepsilon_1(\omega)$ were fit individually, i.e., different sets of optimum values of $A$ and $s$ were obtained for $\sigma_1$ and $\varepsilon_1$ spectra. This may be caused by other mechanisms that are not considered in the dielectric relaxation model, such as the contribution from phonons~\cite{PimenovEtAl1999PRB}. Furthermore, the $\sigma_\text{DC}$ obtained from the DC measurements did not account for the anisotropy discussed in Section~\ref{sec:anisotropy}.


% Bibliography
\bibliography{supplement}


\end{document}