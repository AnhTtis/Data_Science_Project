% ****** Start of file apssamp.tex ******
%
%   This file is part of the APS files in the REVTeX 4.1 distribution.
%   Version 4.1r of REVTeX, August 2010
%
%   Copyright (c) 2009, 2010 The American Physical Society.
%
%   See the REVTeX 4 README file for restrictions and more information.
%
% TeX'ing this file requires that you have AMS-LaTeX 2.0 installed
% as well as the rest of the prerequisites for REVTeX 4.1
%
% See the REVTeX 4 README file
% It also requires running BibTeX. The commands are as follows:
%
%  1)  latex apssamp.tex
%  2)  bibtex apssamp
%  3)  latex apssamp.tex
%  4)  latex apssamp.tex
%
\documentclass[%
 reprint,
superscriptaddress,
%groupedaddress,
%unsortedaddress,
%runinaddress,
%frontmatterverbose,
%preprint,
%showpacs,preprintnumbers,
%nofootinbib,
%nobibnotes,
%bibnotes,mathtools,
 amsmath,amssymb,
 aps,pre,hyphens,floatfix
%pra,
%prb,
%rmp,
%prstab,
%prstper,
%floatfix,
]{revtex4-2}

\usepackage[]{amsmath}
\usepackage{mathrsfs}
\usepackage{mathtools} 
\usepackage{enumerate}
\usepackage[usenames,dvipsnames,svgnames,table]{xcolor}
\usepackage{IEEEtrantools}
\usepackage{graphicx}% Include figure files
\usepackage{array}
\usepackage{multirow}
\usepackage{verbatim}
\usepackage[normalem]{ulem}
\usepackage{tabularx}
\usepackage{booktabs,siunitx}
\usepackage[T1]{fontenc}
%\PassOptionsToPackage{hyphens}{url}
\usepackage{hyperref} 
\usepackage[hyphenbreaks]{breakurl}
%\usepackage[]{longtable}

\newcommand{\blue}[1]{\textcolor{blue}{#1}}
\newcommand{\mage}[1]{\textcolor{magenta}{#1}}
\newcommand{\red}[1]{\textcolor{red}{#1}}
\newcommand{\cyan}[1]{\textcolor{cyan}{#1}}
\newcommand*{\tabbox}[2][t]{%
    \vspace{-8pt}\parbox[#1][12\baselineskip]{0.8cm}{\strut#2\strut}}
\newcommand{\mycomment}[1]{}

\newcommand{\BA}{\blue{B\'alint}}
\newcommand{\SF}{\blue{Shengfeng}}
\newcommand{\GE}{\blue{G\'eza}}
\newcommand{\JF}{\blue{Jeffrey}}
%\AtBeginDocument{\usepackage{booktabs}}
\begin{document}

\preprint{APS/123-QED}

\title{Revisiting power-law distributions in empirical outage data of power systems}
% Force line breaks with \\ 
% \thanks{A footnote to the article title}%

\author{B\'{a}lint Hartmann}
\email[]{hartmann.balint@ek-cer.hu, ORCID: 0000-0001-5271-2681}
\affiliation{Institute of Energy Security and Environmental Safety, ELKH Center for Energy Research, P.O. Box 49, H-1525 Budapest, Hungary}
%Lines break automatically or can be forced with \\
\author{Shengfeng Deng}
\email[]{shengfeng.deng@ek-cer.hu, ORCID: 0000-0002-1952-8599}
\affiliation{Institute of Technical Physics and Materials Science, ELKH Center for Energy Research, P.O. Box 49, H-1525 Budapest, Hungary}

\author{G\'{e}za \'{O}dor}
%\email[]{odor@mfa.kfki.hu}
\affiliation{Institute of Technical Physics and Materials Science, ELKH Center for Energy Research, P.O. Box 49, H-1525 Budapest, Hungary}

\author{Jeffrey Kelling}
\affiliation{Faculty of Natural Sciences, Chemnitz University of Technology, \\ Stra{\ss}e der Nationen 62,  09111 Chemnitz, Germany}
\affiliation{Department of Information Services and Computing,
Helmholtz-Zentrum Dresden-Rossendorf, P.O.Box 51 01 19, 01314 Dresden, Germany
}

\date{\today}
% It is always \today, today, but any date may be explicitly specified

\begin{abstract}
The size distribution of planned and forced outages in power systems have been studied for almost two decades and has drawn great interest as they display heavy tails. Understanding of this phenomenon has been done by various threshold models, which are self-tuned at their critical points, but as many papers pointed out, explanations are intuitive, and more empirical data is needed to support hypotheses. In this paper, the authors analyze outage data collected from various public sources to calculate the outage energy and outage duration exponents of possible power-law fits. Temporal thresholds are applied to identify crossovers from initial short-time behavior to power-law tails. We revisit and add to the possible explanations of the uniformness of these exponents. By performing power spectral analyses on the outage event time series and the outage duration time series, it is found that, on the one hand, while being overwhelmed by white noise, outage events show traits of self-organized criticality (SOC), which may be modeled by a crossover from random percolation to directed percolation branching process with dissipation. On the other hand, in responses to outages, the heavy tails in outage duration distributions could be a consequence of the highly optimized tolerance (HOT) mechanism, based on the optimized allocation of maintenance resources.
\end{abstract}

\maketitle

\section{Introduction\label{sec:1}}

As the power sector undergoes an unprecedented transition, understanding the vulnerability of these systems receives more attention from the research community. Replacing conventional, dominantly fossil-fueled power plants with ones relying on variable sources such as solar and wind poses a number of challenges, mainly due to the appearing correlated spatio-temporal fluctuations. In case of adverse conditions and also taking into consideration the planned and forced outages of the power system, these fluctuations can lead to blackouts of various sizes. The size distribution of such events has been the focus of research for almost two decades, largely because it displays fat tails; a suitable candidate for fitting power laws. It is known that the severity of catastrophic events exhibits such behavior, even after removing extreme outliers \cite{hardle,chernobai}. The probability distribution of the number of people killed in natural disasters and man-made situations also shows power-law decay, with very similar exponent values. As presented in \cite{chatterjee}, the size distribution of deaths in man-made events (wars, battles, conflicts, terrorist attacks) decay with power-law tails, with exponents around 1.6. Similarly, the magnitudes of natural disasters (earthquakes, storms, wildfires, etc.) show exponents between 1.8 and 1.5. If such extreme events do follow a power law, they are not completely unexpected anymore; though they remain unpredictable. (These events were coined by Taleb as `gray swans' \cite{Taleb2007-qr}, in contrast to ‘black swans’, which can still occur due to the possible dynamics of the upper cutoff \cite{PhysRevResearch.4.033079}.)

The research community is actively working on the improvement of our understanding on how power outages and blackouts can be forecasted  \cite{ABEDI2019153} and more specifically whether the risk of failure in power systems represents a specific case of the risk of system-wide breakdown in threshold activated disordered systems. Self-organized criticality (SOC), explained first by the Bak-Tang-Wieselfeld model \cite{bak1987}, is widely used for the modeling of such phenomena \cite{carreras2004}. In this regard, SOC is expected as the consequence of self-tuning to a critical point, which is determined by the competition of power consumption (behavioral properties) and available transmission capabilities (infrastructural properties) of the examined power system. It was also shown in \cite{Hohensee2011PowerLB} that the highly optimized tolerance (HOT) mechanism can also replicate empirical statistics of power outages. (HOT was coined by Carlson and Doyle, claiming that the model takes into consideration the fact that designs are developed and biological systems evolve in a manner that rewards successful strategies subject to a specific form of external stimulus \cite{PhysRevE.60.1412}.) An interesting point raised by \cite{Hohensee2011PowerLB} is that while both the SOC and HOT models can replicate the size distributions of outages, the main difference might be seen in the temporal correlation of the events, with which SOC could manifest a $1/f^\alpha$ noise \cite{jensen19891,kertesz1990}. However, previously studied data still cover a too short time span, restricting one to assert, in typical power systems, if it is SOC or HOT in play or an interplay of both.
 
By studying more outage data from various sources and time spans, this latter argument served as one of the main motivations for our paper, as we revisit the empirical studies carried out in the last two decades, and perform an in-depth comparison with our recent findings. The first widely analyzed dataset was the probability distribution of unserved energy in relation to North American blackouts between 1984 and 1998 \cite{dobson2007c}, which showed a power-law-like tail \cite{carreras2004,weron2006} with a decay exponent between 1.3 and 2.0. Other datasets for the dimension of energy include Sweden \cite{holmgren2006} (exponent $\approx$ 1.6), New Zealand \cite{ancell2005} (1.6) and China \cite{Weng2006} (1.8), while for the dimension of power, data for North America \cite{926277,weron2006} (2.0) and Norway \cite{refId0} (1.7) were analyzed earlier. More recently, substantially longer datasets were analyzed in \cite{ROSASCASALS2011805} and \cite{kancherla2018}. The first study, comprised of 7 years of pan-European data shows power-law fits with exponents of 1.7 and 2.1 for energy and power dimensions, respectively. The second study analyzes 14-year restoration time data of the Bonneville Power Administration, and a power-law fit (1.84) is shown to be the most suitable among other distributions, but empirical data has a bit heavier tail than the fit. 

Those previous data had mainly focused on more noticeable blackout events that usually resulted from cascade failures. In this paper, the collected data record all types of forced outage events, be they almost negligible or affecting vast regions. The natural question is then if heavy tails could still be observed for accessible statistics, such as for the distributions of unavailable energy and unavailable duration. And if so, what are the implications of these heavy tails? In what follows, we should be poised to answer these questions.

The remainder of the paper is organized as follows. Section \ref{sec:2}. presents the methods and data used for the research. Analytical results are detailed in Section \ref{sec:3}. Discussion and explanation of the results are presented in Section \ref{sec:4}.

\section{Methods and data \label{sec:2}}
We collected outage data from various online public sources of Transmission System Operators (TSOs), including the European Network of Transmission System Operators for Electricity (ENTSO-E) online data for the unavailability of production and generation units \cite{entsoeprodgen} (for ``Control Areas''  \footnote{According to \href{https://eepublicdownloads.entsoe.eu/clean-documents/pre2015/resources/Transparency/MoP_Ref_02_-_Detailed_Data_Descriptions_v1r2.pdf}{the ENTSO-E data descriptions}, a generation unit is ``a single electricity generator belonging to a production unit'', while a production unit is understood as ``a facility for generation of electricity made up of a single generation unit or of an aggregation of generation units''; also see ``{C}ommission {R}egulation ({EU}) {N}o 543/2013 of 14 {J}une 2013 on submission and publication of data in electricity markets'' for the definition of a ``Control Area''.}) and for the unavailability of transmission grid \cite{entsotransm} (between ``Control Areas''), the California Independent System Operator (CAISO) generation outage data \cite{caliiso}, the Alberta Electric System Operator (AESO) historical transmission outage data \cite{aesohist}, the Bonneville Power Administration (BPA) outage data (for customer, transformer, and transmission outages) \cite{BPAout}, the Hungarian Transmission System Operator (Magyar Villamosenergia-ipari \'{A}tviteli Rendszerir\'{a}ny\'{i}t\'{o}, MAVIR) data on generation outages \cite{huMAVIR}, and the production outage data from the Italian Power Exchange (Gestore dei mercati energetici, GME) Inside Information Platform \cite{gmeitaly}. Most of these data sources do not provide dataset dumps but rather data access APIs. Hence, we have employed automatic data-scraping methods to assemble as much data as possible for further analyses. Note that automatic data collection methods are particularly useful for data sources that are limited by single-time query quota restrictions. For a detailed account of data collection, see the descriptions in the respective references \cite{entsotransm,caliiso,aesohist,BPAout,huMAVIR,gmeitaly} and the raw data repository \cite{datarepo}.

All the data are formatted in terms of many outage entries containing several columns that detail the information of an outage event, such as the ID of the plant, the starting time and the ending time, the outage type, the unavailable power (or the installed power and the available power), etc. In this paper, we only focus on the statistics of unexpected outages that TSOs usually do not have precautions for prevention, which are usually due to natural causes, such as storms and lightning strikes, or failures of generator components and transmission facilities. To this end, we filtered out entries for any planned outages for maintenance. This amounts to selecting the entries labeled with ``Unplanned outage'' (or ``Forced outage'') for the ENTSO-E data, the entries labeled with the ``FORCED'' outage type for the CAISO data, the entries labeled with the ``Auto'' outage type for the BPA data, and the entries labeled with "UNPLANNED" unavailability type for the Italy GME data, while the Hungary MAVIR data are already for  unplanned outages. The AESO data does not contain such a field for filtering, and we simply selected entries designated with ``Significant Outage''.

Previously, it has been shown that the sizes of empirical outages, measured by various quantities (energy, power, affected customers, etc.)  follow power laws characterized by exponents ranging from $\sim1.3$ to $\sim2.0$ \cite{carreras2000,dobson2007c}. However, as remarked in the introduction, although several possible explanations, for example, SOC for the direct current (AC) model \cite{dobson2007c}, cascade failures in the second-order Kuramoto model for alternating currents \cite{odor2020,odor2022}, and power-law city sizes \cite{nesti2020}, have been proposed to understand the emergence of these power laws, a conclusive, full account for the origin of these power laws is not yet reached in the literature. Especially, due to the complexity of power-grid systems and their interaction with the environment, with our everyday life and production activities, and with the plant operation and maintenance staff, then, natural events, population distributions, power-grid infrastructure failures, human societal behaviors, responses of the maintenance staff to outage events, and so on could all contribute to the heavy tails of various outage measures to a certain extent.  To further probe the possible origins of these power laws, we first reproduce similar power-law observations with the collected data, with outage sizes measured by the unavailable duration ($T_u$) and the unavailable energy ($E_u=T_u\times P_u$, with the notation $P_u=\text{Unavailable Power}$). As will be shown later, empirical unavailable duration distributions could also manifest power-law tails. Since many outage quantities can take the form of $T_u \times \text{certain measure}$, we hence speculate that the understanding of the manifested power law for the unavailable duration constitutes a crucial ingredient for the understanding of general power-law behavior in outage distributions.

Suppose the studied measure of outage sizes is denoted by $x$, representing $E_u$ or $T_u$. One can assume that $x$ is drawn from a \textit{continuous} probability distribution $\mathrm{Prob}(x)$. In particular, empirical data usually display a power-law tail starting from a certain minimum value $x_\mathrm{min}$ \cite{clauset2009} \footnote{From Eq.~\eqref{eqs:plpdf}, the probability density diverges as $x\to 0$, so a lower bound is also mandatory mathematically.}. One can further assume the tail part of the distribution is drawn from a \textit{continuous} power-law distribution with the probability distribution function (PDF)
\begin{equation}
    p(x)=\frac{\tau-1}{x_\mathrm{min}}\left(\frac{x}{x_\mathrm{min}}\right)^{-\tau}\,,
    \label{eqs:plpdf}
\end{equation}
and the cumulative probability function (CDF)
\begin{equation}
    P(x)=\int_x^\infty p(x')\mathrm{d}\!x' = \left(\frac{x}{x_\mathrm{min}}\right)^{-\tau+1}\,.
    \label{eqs:plcdf}
\end{equation}
In what follows, when concrete data are studied, we will associate $\tau$ and $\tau_t$ with the distributions of $E_u$ and $T_u$, respectively, for distinction.

For an outage dataset of $N$ entries, assuming the power law starts to hold from $x_\mathrm{min}$ so that the remaining $n$ entries whose $x \ge x_\mathrm{min}$ fit into a power law, the exponent $\tau$ can then be estimated by maximizing the logarithmic value of the likelihood
\begin{equation}
    p(x|\tau)=\prod_{i=1}^{n} \frac{\tau-1}{x_\mathrm{min}} \left(\frac{x_i}{x_\mathrm{min}}\right)\,, \quad \forall x_i\ge x_\mathrm{min}\,,
\end{equation}
giving rise to the estimated exponent \cite{clauset2009}
\begin{equation}
    \hat{\tau}=1+ n \left[\sum_{i=1}^n \ln \frac{x_i}{x_\mathrm{min}}\right]\,.
    \label{eqs:tau}
\end{equation}
However, since $x_\mathrm{min}$ is usually not known \textit{a priori}, one can select any  value $x=X$ as $x_\mathrm{min}$ and obtain an estimation $\hat{\tau}(x_\mathrm{min}=X)$. In order to find the optimal estimation $\hat{x}_\mathrm{min}$, we resort to minimizing the Kolmogorov-Smirnov statistic distance \cite{clauset2009}
\begin{equation}
    D(x_\mathrm{min})=\max_{x\ge x_\mathrm{min}}|S(x)-P(x)|\,
\end{equation}
that quantifies the distance between the empirical CDF $S(x)$ of the data for observations $x\ge x_\mathrm{min}$ and the CDF Eq.~\eqref{eqs:plcdf} for the best power-law fit of the data [with $\tau=\hat{\tau}$ estimated by Eq.~\eqref{eqs:tau}] in the region $x\ge x_\mathrm{min}$.

For the unavailable duration $T_u$, to explore the crossover from short-time behavior due to quick routine maintenance to power-law behavior in the much longer maintenance regime, we separate the events with respect to a temporal threshold and perform power-law fittings for $T_u\le 24\ \text{hours}$ and $T_u> 24\ \text{hours}$, respectively. The threshold was determined based on consultations with experts from the fields of power system maintenance and reliability of electric machinery. According to operational experience, forced outages rarely end in permanent failures, as they are caused by random events and the majority of such failures can be corrected in a range of hours. On the other hand, major faults typically require specialized equipment and/or personnel, which is available to a limited extent on-site. Service level agreements for operation and maintenance typically consider repairs to be started within 24 hours, but as the correction of these major faults usually requires mechanical works and possible disassembly of the equipment, registered outage times are exceeding 24 hours.

\section{Results \label{sec:3}}
We categorize the outage statistics into outages in generation/production units and outages in transmissions (see Note~\cite{Note1} for the difference between generation and production units.). The BPA data also provide outage duration monitored in customer services and in transformers \cite{BPAout}.

\begin{figure*}[!htbp]
        \centering
        \includegraphics[width=0.99\textwidth]{unen.eps}
        \caption{Probability distributions (black dots) of generation outages measured in terms of the unavailable energy. For the ENTSO-E data, we show the generation outage data for the control areas ``DE\_AMPRION'', ``GB'', and ``FR'', as well as the generation and production outage data from all control areas. The fitted power laws and their corresponding $x_\mathrm{min}$s are marked by solid red lines and vertical black lines, respectively.}
        \label{fig:alltsoen}
\end{figure*}

\begin{figure*}[!htbp]
        \centering
        \includegraphics[width=0.99\textwidth]{undu.eps}
        \caption{Probability distributions (black dots) of generation outages measured in terms of the unavailable duration. For the ENTSO-E data, we show the generation outage data for the control areas ``DE\_AMPRION'', ``GB'', and ``FR'', as well as the generation and production outage data from all control areas. The fitted power laws and their corresponding $x_\mathrm{min}$s are marked by solid red lines and vertical black lines, respectively. For the BPA data, we use units of minutes directly from the raw data, similarly for the subsequent figures.}
        \label{fig:alltsodu}
\end{figure*}

Figs.~\ref{fig:alltsoen} and \ref{fig:alltsodu} show the probability distributions of outages in generation units and production units from various data sources, measured in terms of unavailable energy and unavailable duration, respectively. For the ENTSO-E data, we show results for all generation units and all production units, as well as results for several large control areas, including ``DE\_AMPRION'', ``GB'', and ``FR'' \cite{Note1}. The distributions were obtained by binning the data logarithmically with base 1.08 first and then plotted in a log-log scale.  These data distributions clearly display power-law tails. We employed the fitting method detailed in Sec.~\ref{sec:2} to determine $x_\mathrm{min}$ and the power-law exponent $\tau$/$\tau_t$; also c.f. Table \ref{tab:exps}. For outages measured by the unavailable energy, $\tau$ ranges from $\sim1.3$ to $\sim2.0$, and for generation/production outages measured in terms of the unavailable duration, $\tau_t$ ranges from $\sim1.8$ to $\sim2.1$. In addition to the statistics for generation/production units, Fig.~\ref{fig:alltsodu} also shows the unavailable duration distributions for BPA customer services and BPA transformers, with $\tau_t\simeq 1.83$ and $\tau_t\simeq 1.09$, respectively. The observations for energy outage are thus in good agreement with the literature \cite{carreras2000,carreras2004,weron2006,holmgren2006,Weng2006,ROSASCASALS2011805,dobson2007c}. What is more, the listed unavailable duration distributions further exemplify the findings for the transmission line restoration duration in Ref.~\cite{kancherla2018}, (there a power-law tail was identified with $\tau_t\simeq 1.84$), however, here for the restoration of outages in generation facilities and in BPA customer services and transformers).

Now as in Ref.~\cite{kancherla2018}, the examples in Fig.~\ref{fig:dutrans} demonstrate that transmission outages measured in terms of the unavailable duration (restoration time) indeed follow power laws, with $\tau_t$ ranging from $\sim1.07$ to $\sim2.4$. Note that the transmission outages from the ENTSO-E data are transmission outages between pairs of control areas and those from the AESO data and the BPA data are for outages in transmission lines.

As remarked at the end of the previous section, the 24-hour threshold typically marks a separation for short- and long-time maintenance behaviors. One may then expect to observe quite different statistics for $T_u\le 24$ hours and for $T_u > 24$ hours. This is immediately justified by Figs.~\ref{fig:duth24} and \ref{fig:dutransth24}, upon applying the 24-hour threshold on the duration of generation outages and the duration of transmission outages. For generation outages, both the distributions for $T_u\le 24$ hours and $T_u > 24$ hours display power-law tails, hinting that maintenance activities are governed by different universal behaviors in short- and long-time maintenance. A similar conclusion can be drawn for transmission outages as well, only that for $T_u\le 24$ hours, the statistics are not justifiable for power laws. In this case, it may be that transmission lines usually locate in remote places, many of which could only be fixed with a longer time so that only longer-time maintenance activities follow a universal behavior.

\begin{figure*}[!htbp]
        \centering
        \includegraphics[width=0.99\textwidth]{undutransmt.eps}
        \caption{Probability distributions (black dots) of transmission outages measured in terms of the unavailable duration. For the ENTSO-E data, transmission outages between pairs of control areas are given (``DE\_50HZ-PL\_CZ'', ``PL\_CZ-DE\_50HZ'', ``NO-SE'', ``SE-NO'', and between all pairs of control areas); while the BPA data and the AESO data measure the transmission line outages. The fitted power laws and their corresponding $x_\mathrm{min}$s are marked by solid red lines and vertical black lines, respectively.}
        \label{fig:dutrans}
\end{figure*}

\begin{figure*}[!htbp]
    \centering
    \includegraphics[width=0.99\textwidth]{unduth24.eps}
    \caption{Probability distributions (black dots) of generation outages measured in terms of the unavailable duration, with the duration separated by a threshold at $24$ hours (1440 minutes). The fitted power laws and their corresponding $x_\mathrm{min}$s are marked by solid red lines and vertical black lines, respectively.}
    \label{fig:duth24}
\end{figure*}

\begin{figure*}[!htbp]
        \centering
        \includegraphics[width=0.99\textwidth]{undutransmth24.eps}
        \caption{Probability distributions (black dots) of generation outages measured in terms of the unavailable duration, with the duration separated by a threshold at $24$ hours (1440 minutes). For $T_u>24$ hours, the fitted power laws and their corresponding $x_\mathrm{min}$s are marked by solid red lines and vertical black lines, respectively. For $T_u\leq 24$, unlike in Fig.~\ref{fig:duth24}, the statistics are not justifiable for power-law fits.}
        \label{fig:dutransth24}
\end{figure*}

\begin{table*}[!htbp]
%\footnotesize
\begin{center}
        %\setlength{\tabcolsep}{0.125em}
        \caption{Summary of various exponents obtained for energy outages ($\tau$) and for outage duration ($\tau_t$), with available $\tau_t$ for $T_u\le 24$ hours (denoted as $\tau_{t_{\le 24}}$) and $T_u>24$ hours (denoted as $\tau_{t_{> 24}}$) also displayed.\label{tab:exps}} 
        \begin{tabular}{lllllllll}
                \toprule
                Generation & $\,\tau\,$ & $\,\tau_t\,$ & $\,\tau_{t_{\leqslant 24}}\,$ & $\,\tau_{t_{> 24}}\,$ & \hspace{0.8em} & Transmission & $\,\tau_t\,$ & $\,\tau_{t_{> 24}}\,$ \\
                \hline
                ENTSO-E All Gen. & 1.86  & 1.85  & 1.43 & 1.86  & &  ENTSO-E All & 1.54  & 1.74 \\
                ENTSO-E DE\_AMPRION $\,\,\,$ & 1.57 & 1.83 & 1.40 & 1.90 & & ENTSO-E DE\_50HZ-PL\_CZ $\,\,\,$ & 1.07 & 1.01 \\
                ENTSO-E GB & 1.54 & 2.09 & 1.49 & 2.02 & & ENTSO-E PL\_CZ-DE\_50HZ & 1.06 & 1.01 \\
                ENTSO-E FR & 1.62 & 1.84 & 1.50 & 1.84 & & ENTSO-E NO-SE & 1.12 & 1.25 \\
                USA CAISO & 2.01 & 2.03 &  1.47 & 2.20 & & ENTSO-E SE-NO & 1.09 & 1.79 \\
                Hungary MAVIR & 1.34 & 2.15 & 1.08 & 2.14 & & AESO & 2.37 & 2.37 \\
                GME Italy & 1.86 & 1.85 & 1.54 & 1.85 & & BPA Auto & 1.85 & 1.72 \\
                Literature  & 1.3--2.0 \cite{carreras2004,weron2006,holmgren2006,Weng2006,ROSASCASALS2011805} & 1.84 \cite{kancherla2018} & $-$ & $-$ &&&& \\
                \hline
                BPA Customer$^*$ & $-$ & 1.83 & 1.83 & 1.65 & & \multicolumn{3}{l}{$^*$Note: these BPA data are not for generation} \\
                BPA Transformer$^*$ & $-$ & 1.09 & 1.47 & 1.27 & & \multicolumn{3}{l}{$\,\,\,$outages.} \\
                \bottomrule
        \end{tabular}
\end{center}
\end{table*}

\section{Discussion \label{sec:4}}
We summarize the obtained exponents in Table \ref{tab:exps}. As mentioned earlier, the exponent $\tau$ for the unavailable energy shows good agreement with the literature. For outage duration in the generation sector, we conclude that a 24-hour threshold typically renders $\tau_{t_{\le 24}}\sim 1.0-1.5$ and $\tau_{t_{> 24}}\sim 1.80-2.20 \sim \tau_t>\tau_{t_{\le 24}}$. For transmission outage duration, even though the statistics are not good enough for power-law fits when $T_u\le 24$, we still see that $\tau_{t_{> 24}}\sim \tau_t$. Hence, the observed power-law tail for $T_u$ for each dataset is essentially unaffected if outage events with $T_u\le 24$ are all excluded. These results suggest that typical outages not yet restored beyond 24 hours are governed by a universal mechanism that gives rise to heavy tails.

\subsection{Potential explanations for heavy tails}
These observations for unavailable duration distributions thus provide another perspective to understand the ubiquitous existence of heavy tails in power-grid systems. In the following, potential explanations of the underlying phenomena are presented.

(i) Empirical data of faulty electrical (and many other types) components are expected to show exponential repair time distributions \cite{duffey2019}. Thus, the heavy-tailed outage durations we observe in the databases mean the repair events should not always be independent. A possible explanation for the correlations among the repairs can be related to the limited capacity or availability of the maintenance staff/equipment within a region. Thus, the observed power laws can have a similar origin as those of the cascading blackout events themselves: self-organized criticality (SOC) \cite{bak1987,bak2013}, tuned by the competition of  supply and demand, to the edge of a critical point \cite{carreras2000,dobson2007c}, which is optimal for the function of the whole economy \cite{stanley2002}. In other words, keeping too large a maintenance capacity is economically inefficient, while too small is dangerous for the function of the whole power grid, therefore the maintenance staff/resources of the system tune them to a SOC state as in the case of power production. One can naturally view the heavy tails in outage duration distributions as a direct result of the responses of maintenance resources to outages of different scales driven by SOC.
    
To understand the critical exponents and how the crossover behavior can emerge, one can map the outages onto simple non-equilibrium reaction models with phase transitions. Since there is no direct analogy to quantify outage duration by the number of particles, without loss of generality, let us measure outage sizes in terms of failed units instead. To convert an absorbing-state phase transition system to a SOC system, all one needs is to provide the general driving (particle adding) and dissipation (particle removing) mechanism, in which the drive typically is extremely slow \cite{dickman1998,dickman2000}.  Denoting functioning units by $0$-s, faulty ones by $A$-s, and the repairing teams by $B$-s we can set up the following reaction scheme
\begin{equation}
    0 \to A, \ \ A + 0\to A+A, \ \ A+B \to 0+B
\label{eqs:react}
\end{equation}
where $B$-s can be considered as conserved agents that move in the system and their role is only to sustain the system function in a SOC way, as we proposed above. The outage size is then measured by the number of $A$ particles. This simple system can be considered as the combination of spontaneous isotropic percolation (IP) and a branching process, describing a possible failure cascade, which implies the criticality of the directed percolation coupled with a conserved density (DP-C), as $B$ can be regarded as a background and the total number $N_A+N_0$ is conserved \cite{van1998}. The process takes place on a network of electrical units, which is presumably small-world like. For high-voltate nodes, we showed \cite{odor2020,odor2022} that the power grids have exponential degree distribution and graph dimensions are between 2 and 3, but we don't know the network of the components of the database, because the elements can be sub-units or electrical control circuits. On the other hand, when there is no particle number conservation in the system, we know that for regular networks the DP fixed point is stable asymptotically as compared to the IP \cite{IP-DP-cross-FT,IP-DP-cross}, so a crossover may happen between IP to DP. Such crossover between critical points has also been studied in the IP + DP model in the case of brain models~\cite{PRX-Kor}, on different complex networks. Then similarly, with reaction \eqref{eqs:react}, we should expect a IP to DP-C crossover. However, if the spontaneous failure probability is small, it does not cause critical IP percolation, but a slightly off-critical DP-C, which cannot be distinguished from wandering around DP-C criticality as in the case of SOC-like models \footnote{Numerical analysis showed, that in the presence of such composite reactions, the cascade size and duration distribution exponents crossover to larger values on complex networks~\cite{IP-small,IP-PL} for larger avalanche sizes and durations. This agrees and may explain our power-law fitting results for the failure durations, where we see a crossover around $t_c \simeq 24$ hours from smaller to bigger $\tau_t$ exponents. Note, that in the databases the outage times are rounded to 1-min or 1-hour slots, so shorter-time results may not be reliable}.

In case of the critical DP-C branching process, we can expect the occurrence of avalanches triggered by a failure of a single unit with a power-law size distribution. As for the outage duration distribution, the outage times of single units can be related to the auto-correlation function, which gives the probability of an outage $A$ still not yet being restored after $T_u$, and which exhibits the asymptotic scaling: $C_{AA}(T_u)\propto T_u^{-\lambda/Z}$, where $\lambda$ is the auto-correlation exponent, and $Z$ is the dynamical exponent of the critical process~\cite{odorbook}. Put in another way, this probability can as well be expressed in terms of the outage duration distribution $P_T(T_u)$ as follows
\begin{equation}
    \int_{T_u}^\infty P_T(T) \mathrm{d}T=C_{AA}(T_u)\,,
\end{equation}
giving rise to $P_T(T_u)\sim T_u^{-\tau_t}=T_u^{-\lambda/Z-1}$.

Since the inert 0-s are intrinsically not moving, the scaling properties of reactions \eqref{eqs:react} then belongs to the Manna universality class \cite{henkel2008}, which also encompasses the conserved threshold transfer process (CTTP) in $d\ge 2$ dimensions \cite{henkel2008,lubeck2003}. By utilizing the scaling relation $\lambda=d-\Theta Z$ \cite{henkel2010}, where $\Theta$ is the initial slipping exponent, we obtain
\begin{equation}
    \tau_t=d/Z-\Theta+1\,.
\end{equation}
Inserting the numerical values for $\Theta$ and $Z$ from Ref.~\cite{henkel2008} gives $\tau_t\approx 1.37, 1.99$, and $2.51$ in 1, 2, and 3 dimensions. The exponent value in 2 dimensions seems to agree with our database analysis for generation outages that have resulted in $1.8 \le \tau_t \le 2.1$, depending on the region of the data. The exponent values for transmission outages span a wider range, but still seem to be close to the model values in 1 to 3 dimensions. These exponents suggest a fairly universal asymptotic behavior. Differences from the theoretical values can be the consequence of SOC quasi-critical behavior by the spontaneous failures, under-sampling, different networks with different dimensions, or even Griffiths effects ~\cite{griffiths1969,munoz2010}, which occur in case of quasi-static heterogeneity of the system.
    
We don't have data for the failure cluster size distributions, but for the distribution of lost energy $E_u$, which is related to the product of outage duration $T_u$ and the unavailable power capacity $P_u$ of the nodes, which$\tau$ is 3/2 in the mean field also shows power-law distributions for certain databases (for example in the CAISO data), the power-law tails of $E_u$ follow immediately from that of $T_u$.

(ii) The above-simplified model unavoidably leaves out many complications. First, outages do not necessarily occur in a SOC cascading manner, then the heavy tails in outage duration distributions may not be fully accounted for by the responses of the limited maintenance resources to the SOC cascading failures, as we assumed above. Second, the \textit{ad hoc} $B$ agents themselves, albeit being considered conserved for simplicity, should be organized as a result of further hidden mechanisms for achieving economically efficient responses to outages, so they are not strictly conserved in a long-time span. What is more, {the above simple model only regards $B$-s as a background so that their spatial distribution as well as how they will move in responses to the generations of $A$-s are totally discarded, whereas, in reality, some generators in the network may be considered more crucial than others or have to provide higher availability or meet contractual maximum outage criteria. Operators keep extra resources for such facilities to recover more quickly from outage types which would normally cause a longer outage time, thus skewing the outage time distribution towards smaller outage times.

Since man-made complex systems like power grid systems are highly structured and dominated by optimization designs, the above considerations point to another mechanism for the observed heavy tails: the HOT mechanism \cite{PhysRevE.60.1412,doyle2000}. For forest fires and sandpiles as exemplified in Ref.~\cite{PhysRevE.60.1412}, the HOT mechanism introduces barriers (the resources) to minimize the expected size of the event. These barriers are concentrated in the regions which are expected to be most vulnerable, leaving open the possibility of large events in less probable zones. So in this way, power-law size distribution can emerge by minimizing the average event size via the variational principle with respect to the resource distribution function, subjecting to the restriction of the resources which is used to construct the Lagrange multiplier. Similar barriers are indeed introduced in power systems. The size of an outage is physically limited both on high-voltage and medium-voltage networks. As high-voltage networks tend to be looped, the removal of a line due to an outage should not affect consumers. In the case of medium voltage lines, the radial topology is constructed in an ``arc-ring'' topology, that in case of a fault, circuit-breakers and switches are operated to separate the location and to minimize the number of affected consumers. Hence, even for outage events, although the SOC mechanism can play a role regionally, we can not exclude that the HOT mechanism could partially contribute to the heavy tails in outage size distributions globally.

To be more specific, consider that \textit{independent} outage events can be indexed by $1\le i\le N$. In the original HOT mechanism, the resources $r_i$ are allocated to suppress an event of size $l_i$ from happening. By minimizing the expected outage size \cite{doyle2000}
\begin{equation}
J=\left\{\sum p_i l_i | l_i=f(r_i), \quad \sum r_i\le R \right\}\,,
\label{eqs:hotL}
\end{equation}
subject to the limited resource $R$, it was shown that the size distribution follows a power law $P(l)\sim l^{-\alpha}$. In the above expression, $p_i$ denotes the probability of event $i$ during some time span of observation. The general idea is of course to allocate the resources to places where a large-size event is more likely to happen. However, in the context of outage duration, the focus becomes utilizing the limited resources $R$, say the available maintenance manpower in hours, to fix outages of various sizes $l$ during the observation time span. Now an event should be directly considered as an outage of a generator or a transmission line. These events of course do not occur independently, but for a long enough time span, each of these outage events should associate with a probability $p_i$ and an expected restoration time (outage duration) $T_{ui}$. By extending the HOT mechanism \eqref{eqs:hotL}, the observed heavy tails in outage duration distributions may be attained by minimizing the following objective cost
\begin{equation}
J=\left\{\sum p_i T_{ui} | T_{ui}=g(r_i), \quad \sum r_i\le R \right\}\,.
\label{eqs:hotT}
\end{equation}
For the above purpose, the maintenance manpower are naturally concentrated in regions where outages happen more frequently, leaving outage events in remote places needing longer time to fix, even if it is just a trivial one. Hence, heavy tails $P_T(T_u)\sim T_u^{-\tau}$ ensue in outage duration distributions in a similar manner as the original HOT for outage sizes. Nevertheless, at this stage, the HOT explanation can only be a hypothesis as we still lack the knowledge of $p_i$ and the functional form of $g(r)$.
    
(iii) Our empirical results show similarities with experiences of the power sector as well. Outage restoration times show the most significant correlation with the cause of the outage (and thus the faulty equipment), by far exceeding correlations to other parameters such as weather or temporality \cite{517530,1425608,MALISZEWSKI2012668,en13112736,9410392}. It is known that preventive maintenance highly affects the availability of the equipment, but it comes at a cost (monetary and human resources) \cite{RePEc:ecl:harjfk:rwp05-027,en13143571}. A recent study \cite{Wu2022} also suggests that restoration time of outages also depends on the number of nearby outages, which once again indicates relation to how maintenance teams and their response areas are located.

By considering all these intertwined factors, we see that there are various aspects for the emerged heavy tails in outage statistics. On the one hand, some intervention measures, which are usually not optimized, are introduced to the infrastructure, so that outage events are not entirely occurring in a spontaneous SOC manner, but SOC processes could still dominate in many parts of the system. On the other hand, in response to outage events, maintenance resources enjoy bigger flexibility and may be optimally deployed as needed. In summary, our above discussion favored a SOC explanation for the occurrences of outage events, but we don't entirely exclude HOT, whereas, for outage duration, it is more plausible to ascribe the observed heavy tails to an extended HOT mechanism. For outages, since the HOT mechanism restricts more probable large-size events from happening, it is suggested that the $1/f$ spectra may not be observed, contrary to SOC processes \cite{Hohensee2011PowerLB}. To partially unravel if it is SOC or HOT in play for outage events and their duration, in the next section, we try to detect if there is any traits of $1/f$ noise by performing power spectral analyses on outage and restoration time series.

\subsection{Power spectral analysis}
For major outage events, where large-scale blackout could occur, the entire outage duration distribution (and its power-law tail) may be accounted for by the inverse Weibull distribution ($\alpha x^{-\beta-1}e^{-\lambda x^{-\beta}}\sim \alpha x^{-\beta-1}$ for $x\gg 1$) due to the symmetry of failures and restorations \cite{harvey2004}. However, since our data also include many intermittently occurred outage events, we observe that contrary to Ref.~\cite{Wu2022}, the inverse Weibull distribution could not fit the data to the whole range. 
\begin{figure*}[!htbp]
    \centering
    \includegraphics[width=0.85\textwidth]{powspect.eps}
    \caption{Power spectra of the time series of (a) the number of outage events $S(t)$, (b) the intervals between successive outage events $I(t)$, (c) the Brownian noise $N(t)$, with and without the presence of a white noise background, and (d) the outage duration $D(t)$. In panels (a), (b), and (d), we show the respective power spectra for the ENTSO-E generation outage data from all control areas, the Hungarian MAVIR generation outage data, the GME Italian generation outage data, and the BPA transmission outage data.}
    \label{fig:powsp}
\end{figure*}

To demonstrate that the studied outage data are indeed comprised of random outage events, characterized by white noise signals, as well as correlated events potentially resulting from cascading outages, we perform power spectral analyses \cite{jensen19891,kertesz1990} on the time series $S(t)$, which, similar to the number of ``topplings'' in the sandpile model \cite{bak1987}, is the number of outage events at time $t$, and on the time series $I(t)$ for the $t$th interval between successive outage events. In addition, the duration $D(t)$ for the $t$th outage event can also be regarded as a time series and should be analyzed. For $S(t)$, we have used the accessible time resolution, either one hour or one minute, as the time unit, while for $I(t)$ and $D(t)$, the time $t$ bears the meaning of the index of an event interval and the index of an event, respectively \footnote{For both $S(t)$ and $I(t)$, we had first excluded events recorded at 1/4, 1/2, 3/4 hours and whole hours to eliminate any artifacts in outage bookkeeping time, if many events were recorded at such time points.}. As an intuitive example for showing the effect of white noise on correlated signals, we also performed power spectral analyses on the noise signal $N(t)$ obtained by superposing a Brownian noise with white noise of different intensities
\begin{equation}
        N(t)=\sum_{i=1}^{t} X_i + Y\,, \quad t=1, 2, \dots, T\,,
\end{equation}
where $X_i$ and $Y$ are random variables drawn from the normal distributions $\mathcal{N}(0,1)$ and $\mathcal{N}(0,\sigma)$, respectively, with $\sigma=0, 1,$ and $10$. The power spectrum $P_F(f)$ of a signal $F(t)$ [$=S(t)$, $I(t)$, $D(t)$, or $N(t)$] is then computed as the absolute square of the discrete Fourier transform of $F(t)$ (via the fast Fourier transform) \cite{brown1997}
\begin{IEEEeqnarray}{rCl}
        H(f&=&\frac{t}{T}) = \frac{1}{\sqrt{T}}\sum_{k=1}^{T}F(k) e^{2\pi i (k-1) (t-1)/T}\,,\\
        P_F(f) &=& |H(f)|^2+|H(1-f)|^2\,, \,\,0 < f\le \frac{1}{2}\,.
\end{IEEEeqnarray}

As shown in Figs.~\ref{fig:powsp} (a) and (b), our collected outage data typically display a $1/f^\alpha$ power law, a signature of correlation in the outage events, in the lower-frequency range, while the higher-frequency range is overwhelmed by white noise, which is characterized by a constant power spectrum. This is immediately evident if we compare Figs.~\ref{fig:powsp} (a) and (b) to how the $1/f^2$ power spectrum of a Brownian noise is affected by white noise of different intensities, as illustrated in Fig.~\ref{fig:powsp} (c). Hence, even though the exponent values for $P_S(f)$, ranging from $\sim 1.91$ to $\sim 3.49$, are not entirely in accord with $\alpha<2$  for the Bak–Tang–Wiesenfeld and Manna sandpile models with dimensionality $d<d_c=4$ \cite{laurson2005}, we do not exclude that the topology of the system may come to play, or that there could exist a crossover in the $1/f^\alpha$ law, as the correlations manifested in the higher-frequency range are largely masked by white noise. 

For outage events, the observed $1/f$ noises strengthen the complex IP + DP-C SOC hypothesis advanced before, calling for a dominating branching process for long times and a branching process overwhelmed by random events for short times and excluding the HOT explanation for outage sizes. Yet, the vastness of those random, singular events may partially be a result of the aforementioned barriers introduced to strengthen the power grids, which prevent failures from spreading further. For the $\alpha < 2$ inter-event cases, non-ergodicity, aging, and consequently an age-dependent scaling can also occur~\cite{KALASHYAN2009895}, related to so-called `crucial events'.
    
Since the majority of the outage events seem to occur out of quite random causes, the above observations suggest that the manifested power laws in outage duration distributions can not be attributed to the responses to cascade blackout events alone. In Fig.~\ref{fig:powsp} (d), we show that the outage duration time series $D(t)$ barely show evident traits of $1/f$ noises, except for the Italy data, in which the events were recorded hourly instead of in every minute. Since many outage events can be fixed in a matter of a few hours, we suspect that the $1/f$ noise in the high-frequency part (short time) of the Italy data merely reflects a superficial correlation of outage durations due to its larger time resolution. Based on these arguments, the pervasive heavy tails in outage duration distributions may then be more deeply rooted in the modified HOT mechanism conjectured in \eqref{eqs:hotT}, featuring the optimized responses of the limited maintenance resources to both random outage events as well as cascade outage events resulting from SOC processes.

%   Multiple causes of power laws. Crucial events, Hurst exponent. Cannot directly infer avalanches, so we resort to power spectra analysis. Lack of constant drives? To probe the correlations within the events...

\section{Conclusions}
In the present paper, the authors revisited the topic of the size distribution of forced outages in power systems to formulate possible theoretical explanations of the uniformness of these distributions. To address a shortcoming of previous studies, long-term outage data of various power systems were collected and analyzed. First, exponents of power-law fits were extracted to cross-check the results with related literature, then this step was repeated after setting a threshold, speculating that the understanding of the manifested power law for the unavailable duration constitutes a crucial ingredient for the understanding of general power-law behavior in outage distributions. Based on the numerical results, potential explanations were presented, and a power spectral analysis was performed to demonstrate that the studied outage data are comprised of many random events as well as some correlated events characterized by the $1/f$ noise. This hints that SOC processes could take place in outage events. Therefore, for outage events, we consider the system under study as the combination of spontaneous isotropic percolation and a branching process, where an integer is describing a possible failure cascade, implying directed percolation criticality. Although a SOC explanation based on the DP-C criticality for the heavy tails in outage duration distributions is tempting, given that the majority of the outage events occurred out of random causes, the manifested power laws in outage duration can not be attributed to the responses to SOC cascading failures alone. The power spectra of the outage duration time series further indicate a lack of $1/f$ noise, leading us to conjecture an extended HOT explanation for the heavy tails in outage duration distributions. This can be quite sensible, as on the one hand, power-grid infrastructures are built more or less in a self-organized manner to meet customers' demands, so they are more rigid to give rise to SOC processes, despite some measures being introduced to confine the spread of outages; on the other hand, the needed maintenance resources in responses to outage events can be more fluidly distributed and allocated, permitting a greater extent of optimization for economical efficiency.

In the future, it would be interesting and valuable to examine in-depth more structured data that allows a proper separation of cascading events and random outages so that one can compare how these two types of events are restored differently. As power-grid systems consist of many coupled subsystems and are constantly subjected to various kinds of drive and dissipation, understanding the pervasive heavy tails in various measures is always challenging. To gain a better understanding, it will also be interesting to set up simple hybrid models with the SOC mechanism to account for the outage events and the HOT mechanism for optimizing the restoration processes.

\begin{acknowledgments}
Support from the Hungarian National Research, Development and Innovation 
Office NKFIH (K128989) and from the ELKH grant SA-44/2021 is acknowledged.
\end{acknowledgments}


%\Urlmuskip=0mu plus 1mu\relax
%\bibliographystyle{apsrev4-2}
%\bibliography{outage}% Produces the bibliography via BibTeX.
\begin{thebibliography}{68}%
\makeatletter
\providecommand \@ifxundefined [1]{%
 \@ifx{#1\undefined}
}%
\providecommand \@ifnum [1]{%
 \ifnum #1\expandafter \@firstoftwo
 \else \expandafter \@secondoftwo
 \fi
}%
\providecommand \@ifx [1]{%
 \ifx #1\expandafter \@firstoftwo
 \else \expandafter \@secondoftwo
 \fi
}%
\providecommand \natexlab [1]{#1}%
\providecommand \enquote  [1]{``#1''}%
\providecommand \bibnamefont  [1]{#1}%
\providecommand \bibfnamefont [1]{#1}%
\providecommand \citenamefont [1]{#1}%
\providecommand \href@noop [0]{\@secondoftwo}%
\providecommand \href [0]{\begingroup \@sanitize@url \@href}%
\providecommand \@href[1]{\@@startlink{#1}\@@href}%
\providecommand \@@href[1]{\endgroup#1\@@endlink}%
\providecommand \@sanitize@url [0]{\catcode `\\12\catcode `\$12\catcode
  `\&12\catcode `\#12\catcode `\^12\catcode `\_12\catcode `\%12\relax}%
\providecommand \@@startlink[1]{}%
\providecommand \@@endlink[0]{}%
\providecommand \url  [0]{\begingroup\@sanitize@url \@url }%
\providecommand \@url [1]{\endgroup\@href {#1}{\urlprefix }}%
\providecommand \urlprefix  [0]{URL }%
\providecommand \Eprint [0]{\href }%
\providecommand \doibase [0]{https://doi.org/}%
\providecommand \selectlanguage [0]{\@gobble}%
\providecommand \bibinfo  [0]{\@secondoftwo}%
\providecommand \bibfield  [0]{\@secondoftwo}%
\providecommand \translation [1]{[#1]}%
\providecommand \BibitemOpen [0]{}%
\providecommand \bibitemStop [0]{}%
\providecommand \bibitemNoStop [0]{.\EOS\space}%
\providecommand \EOS [0]{\spacefactor3000\relax}%
\providecommand \BibitemShut  [1]{\csname bibitem#1\endcsname}%
\let\auto@bib@innerbib\@empty
%</preamble>
\bibitem [{\citenamefont {Härdle}\ and\ \citenamefont
  {Cabrera}(2010)}]{hardle}%
  \BibitemOpen
  \bibfield  {author} {\bibinfo {author} {\bibfnamefont {W.~K.}\ \bibnamefont
  {Härdle}}\ and\ \bibinfo {author} {\bibfnamefont {B.~L.}\ \bibnamefont
  {Cabrera}},\ }\href@noop {} {\bibfield  {journal} {\bibinfo  {journal}
  {Journal of Risk and Insurance}\ }\textbf {\bibinfo {volume} {77}},\ \bibinfo
  {pages} {625} (\bibinfo {year} {2010})}\BibitemShut {NoStop}%
\bibitem [{\citenamefont {Chernobai}\ \emph {et~al.}(2006)\citenamefont
  {Chernobai}, \citenamefont {Burnecki}, \citenamefont {Rachev}, \citenamefont
  {Tr{\"u}ck},\ and\ \citenamefont {Weron}}]{chernobai}%
  \BibitemOpen
  \bibfield  {author} {\bibinfo {author} {\bibfnamefont {A.}~\bibnamefont
  {Chernobai}}, \bibinfo {author} {\bibfnamefont {K.}~\bibnamefont {Burnecki}},
  \bibinfo {author} {\bibfnamefont {S.}~\bibnamefont {Rachev}}, \bibinfo
  {author} {\bibfnamefont {S.}~\bibnamefont {Tr{\"u}ck}},\ and\ \bibinfo
  {author} {\bibfnamefont {R.}~\bibnamefont {Weron}},\ }\href@noop {}
  {\bibfield  {journal} {\bibinfo  {journal} {Computational Statistics}\
  }\textbf {\bibinfo {volume} {21}},\ \bibinfo {pages} {537} (\bibinfo {year}
  {2006})}\BibitemShut {NoStop}%
\bibitem [{\citenamefont {Chatterjee}\ and\ \citenamefont
  {Chakrabarti}(2017)}]{chatterjee}%
  \BibitemOpen
  \bibfield  {author} {\bibinfo {author} {\bibfnamefont {A.}~\bibnamefont
  {Chatterjee}}\ and\ \bibinfo {author} {\bibfnamefont {B.~K.}\ \bibnamefont
  {Chakrabarti}},\ }\href@noop {} {\bibfield  {journal} {\bibinfo  {journal}
  {Reports in Advances of Physical Sciences}\ }\textbf {\bibinfo {volume}
  {01}},\ \bibinfo {pages} {1740007} (\bibinfo {year} {2017})}\BibitemShut
  {NoStop}%
\bibitem [{\citenamefont {Taleb}(2007)}]{Taleb2007-qr}%
  \BibitemOpen
  \bibfield  {author} {\bibinfo {author} {\bibfnamefont {N.~N.}\ \bibnamefont
  {Taleb}},\ }\href@noop {} {\emph {\bibinfo {title} {The black swan}}}\
  (\bibinfo  {publisher} {Random House},\ \bibinfo {address} {New York, NY},\
  \bibinfo {year} {2007})\BibitemShut {NoStop}%
\bibitem [{\citenamefont {De~Marzo}\ \emph {et~al.}(2022)\citenamefont
  {De~Marzo}, \citenamefont {Gabrielli}, \citenamefont {Zaccaria},\ and\
  \citenamefont {Pietronero}}]{PhysRevResearch.4.033079}%
  \BibitemOpen
  \bibfield  {author} {\bibinfo {author} {\bibfnamefont {G.}~\bibnamefont
  {De~Marzo}}, \bibinfo {author} {\bibfnamefont {A.}~\bibnamefont {Gabrielli}},
  \bibinfo {author} {\bibfnamefont {A.}~\bibnamefont {Zaccaria}},\ and\
  \bibinfo {author} {\bibfnamefont {L.}~\bibnamefont {Pietronero}},\
  }\href@noop {} {\bibfield  {journal} {\bibinfo  {journal} {Phys. Rev. Res.}\
  }\textbf {\bibinfo {volume} {4}},\ \bibinfo {pages} {033079} (\bibinfo {year}
  {2022})}\BibitemShut {NoStop}%
\bibitem [{\citenamefont {Abedi}\ \emph {et~al.}(2019)\citenamefont {Abedi},
  \citenamefont {Gaudard},\ and\ \citenamefont {Romerio}}]{ABEDI2019153}%
  \BibitemOpen
  \bibfield  {author} {\bibinfo {author} {\bibfnamefont {A.}~\bibnamefont
  {Abedi}}, \bibinfo {author} {\bibfnamefont {L.}~\bibnamefont {Gaudard}},\
  and\ \bibinfo {author} {\bibfnamefont {F.}~\bibnamefont {Romerio}},\
  }\href@noop {} {\bibfield  {journal} {\bibinfo  {journal} {Reliability
  Engineering \& System Safety}\ }\textbf {\bibinfo {volume} {183}},\ \bibinfo
  {pages} {153} (\bibinfo {year} {2019})}\BibitemShut {NoStop}%
\bibitem [{\citenamefont {Bak}\ \emph {et~al.}(1987)\citenamefont {Bak},
  \citenamefont {Tang},\ and\ \citenamefont {Wiesenfeld}}]{bak1987}%
  \BibitemOpen
  \bibfield  {author} {\bibinfo {author} {\bibfnamefont {P.}~\bibnamefont
  {Bak}}, \bibinfo {author} {\bibfnamefont {C.}~\bibnamefont {Tang}},\ and\
  \bibinfo {author} {\bibfnamefont {K.}~\bibnamefont {Wiesenfeld}},\
  }\href@noop {} {\bibfield  {journal} {\bibinfo  {journal} {Physical review
  letters}\ }\textbf {\bibinfo {volume} {59}},\ \bibinfo {pages} {381}
  (\bibinfo {year} {1987})}\BibitemShut {NoStop}%
\bibitem [{\citenamefont {Carreras}\ \emph {et~al.}(2004)\citenamefont
  {Carreras}, \citenamefont {Newman}, \citenamefont {Dobson},\ and\
  \citenamefont {Poole}}]{carreras2004}%
  \BibitemOpen
  \bibfield  {author} {\bibinfo {author} {\bibfnamefont {B.}~\bibnamefont
  {Carreras}}, \bibinfo {author} {\bibfnamefont {D.}~\bibnamefont {Newman}},
  \bibinfo {author} {\bibfnamefont {I.}~\bibnamefont {Dobson}},\ and\ \bibinfo
  {author} {\bibfnamefont {A.}~\bibnamefont {Poole}},\ }\href@noop {}
  {\bibfield  {journal} {\bibinfo  {journal} {IEEE Transactions on Circuits and
  Systems}\ }\textbf {\bibinfo {volume} {51}},\ \bibinfo {pages} {1733}
  (\bibinfo {year} {2004})}\BibitemShut {NoStop}%
\bibitem [{\citenamefont {Hohensee}(2011)}]{Hohensee2011PowerLB}%
  \BibitemOpen
  \bibfield  {author} {\bibinfo {author} {\bibfnamefont {G.~T.}\ \bibnamefont
  {Hohensee}}\ }(\bibinfo {year} {2011})\BibitemShut {NoStop}%
\bibitem [{\citenamefont {Carlson}\ and\ \citenamefont
  {Doyle}(1999)}]{PhysRevE.60.1412}%
  \BibitemOpen
  \bibfield  {author} {\bibinfo {author} {\bibfnamefont {J.~M.}\ \bibnamefont
  {Carlson}}\ and\ \bibinfo {author} {\bibfnamefont {J.}~\bibnamefont
  {Doyle}},\ }\href@noop {} {\bibfield  {journal} {\bibinfo  {journal} {Phys.
  Rev. E}\ }\textbf {\bibinfo {volume} {60}},\ \bibinfo {pages} {1412}
  (\bibinfo {year} {1999})}\BibitemShut {NoStop}%
\bibitem [{\citenamefont {Jensen}\ \emph {et~al.}(1989)\citenamefont {Jensen},
  \citenamefont {Christensen},\ and\ \citenamefont {Fogedby}}]{jensen19891}%
  \BibitemOpen
  \bibfield  {author} {\bibinfo {author} {\bibfnamefont {H.~J.}\ \bibnamefont
  {Jensen}}, \bibinfo {author} {\bibfnamefont {K.}~\bibnamefont
  {Christensen}},\ and\ \bibinfo {author} {\bibfnamefont {H.~C.}\ \bibnamefont
  {Fogedby}},\ }\href@noop {} {\bibfield  {journal} {\bibinfo  {journal}
  {Physical Review B}\ }\textbf {\bibinfo {volume} {40}},\ \bibinfo {pages}
  {7425} (\bibinfo {year} {1989})}\BibitemShut {NoStop}%
\bibitem [{\citenamefont {Kert{\'e}sz}\ and\ \citenamefont
  {Kiss}(1990)}]{kertesz1990}%
  \BibitemOpen
  \bibfield  {author} {\bibinfo {author} {\bibfnamefont {J.}~\bibnamefont
  {Kert{\'e}sz}}\ and\ \bibinfo {author} {\bibfnamefont {L.}~\bibnamefont
  {Kiss}},\ }\href@noop {} {\bibfield  {journal} {\bibinfo  {journal} {Journal
  of Physics A: Mathematical and General}\ }\textbf {\bibinfo {volume} {23}},\
  \bibinfo {pages} {L433} (\bibinfo {year} {1990})}\BibitemShut {NoStop}%
\bibitem [{\citenamefont {Dobson}\ \emph {et~al.}(2007)\citenamefont {Dobson},
  \citenamefont {Carreras}, \citenamefont {Lynch},\ and\ \citenamefont
  {Newman}}]{dobson2007c}%
  \BibitemOpen
  \bibfield  {author} {\bibinfo {author} {\bibfnamefont {I.}~\bibnamefont
  {Dobson}}, \bibinfo {author} {\bibfnamefont {B.~A.}\ \bibnamefont
  {Carreras}}, \bibinfo {author} {\bibfnamefont {V.~E.}\ \bibnamefont
  {Lynch}},\ and\ \bibinfo {author} {\bibfnamefont {D.~E.}\ \bibnamefont
  {Newman}},\ }\href@noop {} {\bibfield  {journal} {\bibinfo  {journal}
  {Chaos}\ }\textbf {\bibinfo {volume} {17}},\ \bibinfo {pages} {026103}
  (\bibinfo {year} {2007})}\BibitemShut {NoStop}%
\bibitem [{\citenamefont {Weron}\ and\ \citenamefont
  {Simonsen}(2006)}]{weron2006}%
  \BibitemOpen
  \bibfield  {author} {\bibinfo {author} {\bibfnamefont {R.}~\bibnamefont
  {Weron}}\ and\ \bibinfo {author} {\bibfnamefont {I.}~\bibnamefont
  {Simonsen}},\ }in\ \href@noop {} {\emph {\bibinfo {booktitle} {Practical
  Fruits of Econophysics}}},\ \bibinfo {editor} {edited by\ \bibinfo {editor}
  {\bibfnamefont {H.}~\bibnamefont {Takayasu}}}\ (\bibinfo  {publisher}
  {Springer Tokyo},\ \bibinfo {address} {Tokyo},\ \bibinfo {year} {2006})\ pp.\
  \bibinfo {pages} {215--219}\BibitemShut {NoStop}%
\bibitem [{\citenamefont {Åke J.~Holmgren}\ and\ \citenamefont
  {Molin}(2006)}]{holmgren2006}%
  \BibitemOpen
  \bibfield  {author} {\bibinfo {author} {\bibnamefont {Åke J.~Holmgren}}\
  and\ \bibinfo {author} {\bibfnamefont {S.}~\bibnamefont {Molin}},\
  }\href@noop {} {\bibfield  {journal} {\bibinfo  {journal} {Journal of
  Infrastructure Systems}\ }\textbf {\bibinfo {volume} {12}},\ \bibinfo {pages}
  {243} (\bibinfo {year} {2006})}\BibitemShut {NoStop}%
\bibitem [{\citenamefont {Ancell}\ \emph {et~al.}(2005)\citenamefont {Ancell},
  \citenamefont {Edwards},\ and\ \citenamefont {Krichtal}}]{ancell2005}%
  \BibitemOpen
  \bibfield  {author} {\bibinfo {author} {\bibfnamefont {G.}~\bibnamefont
  {Ancell}}, \bibinfo {author} {\bibfnamefont {C.}~\bibnamefont {Edwards}},\
  and\ \bibinfo {author} {\bibfnamefont {V.}~\bibnamefont {Krichtal}},\ }in\
  \href@noop {} {\emph {\bibinfo {booktitle} {Electricity Engineers Association
  2005 Conference “Implementing New Zealands Energy Options”, Aukland, New
  Zealand}}}\ (\bibinfo {year} {2005})\BibitemShut {NoStop}%
\bibitem [{\citenamefont {Weng}\ \emph {et~al.}(2006)\citenamefont {Weng},
  \citenamefont {Hong}, \citenamefont {Xue},\ and\ \citenamefont
  {Mei}}]{Weng2006}%
  \BibitemOpen
  \bibfield  {author} {\bibinfo {author} {\bibfnamefont {X.}~\bibnamefont
  {Weng}}, \bibinfo {author} {\bibfnamefont {Y.}~\bibnamefont {Hong}}, \bibinfo
  {author} {\bibfnamefont {A.}~\bibnamefont {Xue}},\ and\ \bibinfo {author}
  {\bibfnamefont {S.}~\bibnamefont {Mei}},\ }\href@noop {} {\bibfield
  {journal} {\bibinfo  {journal} {Journal of Control Theory and Applications}\
  }\textbf {\bibinfo {volume} {4}},\ \bibinfo {pages} {235} (\bibinfo {year}
  {2006})}\BibitemShut {NoStop}%
\bibitem [{\citenamefont {Chen}\ \emph {et~al.}(2001)\citenamefont {Chen},
  \citenamefont {Thorp},\ and\ \citenamefont {Parashar}}]{926277}%
  \BibitemOpen
  \bibfield  {author} {\bibinfo {author} {\bibfnamefont {J.}~\bibnamefont
  {Chen}}, \bibinfo {author} {\bibfnamefont {J.}~\bibnamefont {Thorp}},\ and\
  \bibinfo {author} {\bibfnamefont {M.}~\bibnamefont {Parashar}},\ }in\
  \href@noop {} {\emph {\bibinfo {booktitle} {Proceedings of the 34th Annual
  Hawaii International Conference on System Sciences}}}\ (\bibinfo {year}
  {2001})\ pp.\ \bibinfo {pages} {738--744}\BibitemShut {NoStop}%
\bibitem [{\citenamefont {{Bakke, J. \O{}. H.}}\ \emph
  {et~al.}(2006)\citenamefont {{Bakke, J. \O{}. H.}}, \citenamefont {{Hansen,
  A.}},\ and\ \citenamefont {{Kert\'esz, J.}}}]{refId0}%
  \BibitemOpen
  \bibfield  {author} {\bibinfo {author} {\bibnamefont {{Bakke, J. \O{}. H.}}},
  \bibinfo {author} {\bibnamefont {{Hansen, A.}}},\ and\ \bibinfo {author}
  {\bibnamefont {{Kert\'esz, J.}}},\ }\href@noop {} {\bibfield  {journal}
  {\bibinfo  {journal} {Europhys. Lett.}\ }\textbf {\bibinfo {volume} {76}},\
  \bibinfo {pages} {717} (\bibinfo {year} {2006})}\BibitemShut {NoStop}%
\bibitem [{\citenamefont {Rosas-Casals}\ and\ \citenamefont
  {Solé}(2011)}]{ROSASCASALS2011805}%
  \BibitemOpen
  \bibfield  {author} {\bibinfo {author} {\bibfnamefont {M.}~\bibnamefont
  {Rosas-Casals}}\ and\ \bibinfo {author} {\bibfnamefont {R.}~\bibnamefont
  {Solé}},\ }\href@noop {} {\bibfield  {journal} {\bibinfo  {journal}
  {International Journal of Electrical Power \& Energy Systems}\ }\textbf
  {\bibinfo {volume} {33}},\ \bibinfo {pages} {805} (\bibinfo {year}
  {2011})}\BibitemShut {NoStop}%
\bibitem [{\citenamefont {Kancherla}\ and\ \citenamefont
  {Dobson}(2018)}]{kancherla2018}%
  \BibitemOpen
  \bibfield  {author} {\bibinfo {author} {\bibfnamefont {S.}~\bibnamefont
  {Kancherla}}\ and\ \bibinfo {author} {\bibfnamefont {I.}~\bibnamefont
  {Dobson}},\ }\href@noop {} {\bibfield  {journal} {\bibinfo  {journal} {IEEE
  Transactions on Power Systems}\ }\textbf {\bibinfo {volume} {33}},\ \bibinfo
  {pages} {1145} (\bibinfo {year} {2018})}\BibitemShut {NoStop}%
\bibitem [{ent(2022{\natexlab{a}})}]{entsoeprodgen}%
  \BibitemOpen
  \href@noop {} {\bibinfo {title} {{ENTSO-E}, {U}navailability of {P}roduction
  and {G}eneration {U}nits}},\ \bibinfo {howpublished}
  {\url{https://transparency.entsoe.eu/outage-domain/r2/unavailabilityOfProductionAndGenerationUnits/show}}
  (\bibinfo {year} {Dates span from December 11, 2014 to July 10,
  2022}{\natexlab{a}}),\ \bibinfo {note} {data downloaded through the provided
  API:
  \url{https://transparency.entsoe.eu/content/static_content/Static\%20content/web\%20api/Guide.html},
  which can be conveniently accessed via the ``entsoe-py'' client:
  \url{https://github.com/EnergieID/entsoe-py}. The relevant columns of each
  entry are those named `avail\_qty' (available power, in unit of
  $\mathrm{MW}$), `end' (outage ending time), `nominal\_power' (installed
  power, in unit of $\mathrm{MW}$), and `start' (outage starting time). Hence,
  $P_u=\text{nominal\_power}-\text{avail\_qty}$ and
  $T_u=\text{end}-\text{start}$ (in unit of hour).}\BibitemShut {Stop}%
\bibitem [{Note1()}]{Note1}%
  \BibitemOpen
  \bibinfo {note} {According to \protect \href
  {https://eepublicdownloads.entsoe.eu/clean-documents/pre2015/resources/Transparency/MoP_Ref_02_-_Detailed_Data_Descriptions_v1r2.pdf}{the
  ENTSO-E data descriptions}, a generation unit is ``a single electricity
  generator belonging to a production unit'', while a production unit is
  understood as ``a facility for generation of electricity made up of a single
  generation unit or of an aggregation of generation units''; also see
  ``{C}ommission {R}egulation ({EU}) {N}o 543/2013 of 14 {J}une 2013 on
  submission and publication of data in electricity markets'' for the
  definition of a ``Control Area''.}\BibitemShut {Stop}%
\bibitem [{ent(2022{\natexlab{b}})}]{entsotransm}%
  \BibitemOpen
  \href@noop {} {\bibinfo {title} {{ENTSO-E}, {U}navailability of
  {T}ransmission {G}rid}},\ \bibinfo {howpublished}
  {\url{https://transparency.entsoe.eu/outage-domain/r2/unavailabilityInTransmissionGrid/show}}
  (\bibinfo {year} {Dates span from December 11, 2014 to July 10,
  2022}{\natexlab{b}}),\ \bibinfo {note} {see \cite{entsoeprodgen} for data
  collection details.}\BibitemShut {Stop}%
\bibitem [{cal(2022)}]{caliiso}%
  \BibitemOpen
  \href@noop {} {\bibinfo {title} {California {ISO} - {C}urtailed and
  {N}on-{O}perational {G}enerators}},\ \bibinfo {howpublished}
  {\url{http://www.caiso.com/market/Pages/OutageManagement/CurtailedandNonOperationalGenerators.aspx}}
  (\bibinfo {year} {Dates span from June 17, 2021 to July 25, 2022}),\ \bibinfo
  {note} {we use prior trade date reports with fields for both the unavailable
  duration and unavailable power included. The relevant columns are those named
  `CURTAILMENT START DATE TIME', `CURTAILMENT END DATE TIME', and `CURTAILMENT
  MW': $T_u=\text{`CURTAILMENT END DATE TIME'}-\text{`CURTAILMENT START DATE
  TIME'}$ (in unit of $\text{hour}$), $P_u=\text{`CURTAILMENT
  MW'}$}\BibitemShut {NoStop}%
\bibitem [{aes(2019)}]{aesohist}%
  \BibitemOpen
  \href@noop {} {\bibinfo {title} {{AESO} {H}istorical {T}ransmission {O}utages
  {D}ata 2013-2019}},\ \bibinfo {howpublished}
  {\url{https://www.aeso.ca/market/market-and-system-reporting/data-requests/historical-transmission-outages-data/}}
  (\bibinfo {year} {Dates span from April 28, 2013 to November 4, 2019}),\
  \bibinfo {note} {the relevant columns are `Outage Start' and `Planned End',
  giving $T_u=\text{`Planned End'}-\text{`Outage Start'}$ in unit of
  hour.}\BibitemShut {Stop}%
\bibitem [{BPA(2022)}]{BPAout}%
  \BibitemOpen
  \href@noop {} {\bibinfo {title} {Bonneville {P}ower {A}dministration {O}utage
  \& {R}eliability {R}eports}},\ \bibinfo {howpublished}
  {\url{https://transmission.bpa.gov/Business/Operations/Outages/}} (\bibinfo
  {year} {(Dates span from January 1, 1988 to April 6, 2022)}),\ \bibinfo
  {note} {outages are monitored for Customer Service Interruptions,
  Transmission Line Interruptions, and Transformer Interruptions. The relevant
  column is $T_u=\text{`Duration'}$ in unit of minute.}\BibitemShut {Stop}%
\bibitem [{huM(2022)}]{huMAVIR}%
  \BibitemOpen
  \href@noop {} {\bibinfo {title} {Expost {I}nformation {O}n {U}nplanned
  {U}navailability of {G}eneration {U}nits - {MAVIR} - {M}agyar
  {V}illamosenergia-ipari {Á}tviteli {R}endszerir\'{a}ny\'{i}t\'{o} {Z}rt.}},\
  \bibinfo {howpublished}
  {\url{https://www.mavir.hu/web/mavir-en/expost-information-on-unplanned-unavailability-
  of-generation-units}} (\bibinfo {year} {Dates span from September 8, 2010 to
  June 29, 2022}),\ \bibinfo {note} {the relevant columns are those named
  `Start of the outage', `End of the outage', and `Unavailable capacity' (in
  unit of $\mathrm{MW}$): $T_u=\text{`End of the outage'}-\text{`Start of the
  outage'}$ (in unit of hour), $P_u=\text{`Unavailable
  capacity'}$.}\BibitemShut {Stop}%
\bibitem [{gme(2017)}]{gmeitaly}%
  \BibitemOpen
  \href@noop {} {\bibinfo {title} {{GME} {I}nside {I}nformation {P}latform,
  {P}roduction {U}navailability}},\ \bibinfo {howpublished}
  {\url{https://pip.ipex.it/PipWa/Front/\#/PowerUmms}} (\bibinfo {year} {Dates
  from December 31, 2015 to July 7, 2017}),\ \bibinfo {note} {the relevant
  columns are those named `EventStart', `EventStop', and `UnavailableCapacity':
  $T_u=\text{EventStop}-\text{EventStart}$ (in unit of hour) and
  $P_u=\text{UnavailableCapacity}$ (in unit of $\mathrm{MW}$).}\BibitemShut
  {Stop}%
\bibitem [{\citenamefont {B\'alint}\ \emph {et~al.}(2023)\citenamefont
  {B\'alint}, \citenamefont {Deng}, \citenamefont {\'Odor},\ and\ \citenamefont
  {Kelling}}]{datarepo}%
  \BibitemOpen
  \bibfield  {author} {\bibinfo {author} {\bibfnamefont {H.}~\bibnamefont
  {B\'alint}}, \bibinfo {author} {\bibfnamefont {S.}~\bibnamefont {Deng}},
  \bibinfo {author} {\bibfnamefont {G.}~\bibnamefont {\'Odor}},\ and\ \bibinfo
  {author} {\bibfnamefont {J.}~\bibnamefont {Kelling}},\ }\href@noop {}
  {\bibinfo {title} {Raw outage data for this paper}},\ \bibinfo {howpublished}
  {\url{https://github.com/gitstevendeng/power-grid-outage-data}} (\bibinfo
  {year} {2023}),\ \bibinfo {note} {(Accessed on 01/30/2023)}\BibitemShut
  {NoStop}%
\bibitem [{\citenamefont {Carreras}\ \emph {et~al.}(2000)\citenamefont
  {Carreras}, \citenamefont {Newman}, \citenamefont {Dobson},\ and\
  \citenamefont {Poole}}]{carreras2000}%
  \BibitemOpen
  \bibfield  {author} {\bibinfo {author} {\bibfnamefont {B.~A.}\ \bibnamefont
  {Carreras}}, \bibinfo {author} {\bibfnamefont {D.~E.}\ \bibnamefont
  {Newman}}, \bibinfo {author} {\bibfnamefont {I.}~\bibnamefont {Dobson}},\
  and\ \bibinfo {author} {\bibfnamefont {A.}~\bibnamefont {Poole}},\ }in\
  \href@noop {} {\emph {\bibinfo {booktitle} {Proceedings of the 33rd annual
  Hawaii international conference on system sciences}}}\ (\bibinfo
  {organization} {IEEE},\ \bibinfo {year} {2000})\ pp.\ \bibinfo {pages}
  {6--pp}\BibitemShut {NoStop}%
\bibitem [{\citenamefont {{\'O}dor}\ and\ \citenamefont
  {Hartmann}(2020)}]{odor2020}%
  \BibitemOpen
  \bibfield  {author} {\bibinfo {author} {\bibfnamefont {G.}~\bibnamefont
  {{\'O}dor}}\ and\ \bibinfo {author} {\bibfnamefont {B.}~\bibnamefont
  {Hartmann}},\ }\href@noop {} {\bibfield  {journal} {\bibinfo  {journal}
  {Entropy}\ }\textbf {\bibinfo {volume} {22}},\ \bibinfo {pages} {666}
  (\bibinfo {year} {2020})}\BibitemShut {NoStop}%
\bibitem [{\citenamefont {{\'O}dor}\ \emph {et~al.}(2022)\citenamefont
  {{\'O}dor}, \citenamefont {Deng}, \citenamefont {Hartmann},\ and\
  \citenamefont {Kelling}}]{odor2022}%
  \BibitemOpen
  \bibfield  {author} {\bibinfo {author} {\bibfnamefont {G.}~\bibnamefont
  {{\'O}dor}}, \bibinfo {author} {\bibfnamefont {S.}~\bibnamefont {Deng}},
  \bibinfo {author} {\bibfnamefont {B.}~\bibnamefont {Hartmann}},\ and\
  \bibinfo {author} {\bibfnamefont {J.}~\bibnamefont {Kelling}},\ }\href@noop
  {} {\bibfield  {journal} {\bibinfo  {journal} {Physical Review E}\ }\textbf
  {\bibinfo {volume} {106}},\ \bibinfo {pages} {034311} (\bibinfo {year}
  {2022})}\BibitemShut {NoStop}%
\bibitem [{\citenamefont {Nesti}\ \emph {et~al.}(2020)\citenamefont {Nesti},
  \citenamefont {Sloothaak},\ and\ \citenamefont {Zwart}}]{nesti2020}%
  \BibitemOpen
  \bibfield  {author} {\bibinfo {author} {\bibfnamefont {T.}~\bibnamefont
  {Nesti}}, \bibinfo {author} {\bibfnamefont {F.}~\bibnamefont {Sloothaak}},\
  and\ \bibinfo {author} {\bibfnamefont {B.}~\bibnamefont {Zwart}},\
  }\href@noop {} {\bibfield  {journal} {\bibinfo  {journal} {Physical Review
  Letters}\ }\textbf {\bibinfo {volume} {125}},\ \bibinfo {pages} {058301}
  (\bibinfo {year} {2020})}\BibitemShut {NoStop}%
\bibitem [{\citenamefont {Clauset}\ \emph {et~al.}(2009)\citenamefont
  {Clauset}, \citenamefont {Shalizi},\ and\ \citenamefont
  {Newman}}]{clauset2009}%
  \BibitemOpen
  \bibfield  {author} {\bibinfo {author} {\bibfnamefont {A.}~\bibnamefont
  {Clauset}}, \bibinfo {author} {\bibfnamefont {C.~R.}\ \bibnamefont
  {Shalizi}},\ and\ \bibinfo {author} {\bibfnamefont {M.~E.}\ \bibnamefont
  {Newman}},\ }\href@noop {} {\bibfield  {journal} {\bibinfo  {journal} {SIAM
  review}\ }\textbf {\bibinfo {volume} {51}},\ \bibinfo {pages} {661} (\bibinfo
  {year} {2009})}\BibitemShut {NoStop}%
\bibitem [{Note2()}]{Note2}%
  \BibitemOpen
  \bibinfo {note} {From Eq.~\protect \textup {\hbox {\mathsurround \z@ \protect
  \normalfont (\ignorespaces \ref {eqs:plpdf}\unskip \@@italiccorr )}}, the
  probability density diverges as $x\to 0$, so a lower bound is also mandatory
  mathematically.}\BibitemShut {Stop}%
\bibitem [{\citenamefont {Duffey}(2019)}]{duffey2019}%
  \BibitemOpen
  \bibfield  {author} {\bibinfo {author} {\bibfnamefont {R.~B.}\ \bibnamefont
  {Duffey}},\ }\href@noop {} {\bibfield  {journal} {\bibinfo  {journal}
  {International Journal of Disaster Risk Science}\ }\textbf {\bibinfo {volume}
  {10}},\ \bibinfo {pages} {134} (\bibinfo {year} {2019})}\BibitemShut
  {NoStop}%
\bibitem [{\citenamefont {Bak}(2013)}]{bak2013}%
  \BibitemOpen
  \bibfield  {author} {\bibinfo {author} {\bibfnamefont {P.}~\bibnamefont
  {Bak}},\ }\href@noop {} {\emph {\bibinfo {title} {How nature works: the
  science of self-organized criticality}}}\ (\bibinfo  {publisher} {Springer
  Science \& Business Media},\ \bibinfo {year} {2013})\BibitemShut {NoStop}%
\bibitem [{\citenamefont {Stanley}\ \emph {et~al.}(2002)\citenamefont
  {Stanley}, \citenamefont {Amaral}, \citenamefont {Buldyrev}, \citenamefont
  {Gopikrishnan}, \citenamefont {Plerou},\ and\ \citenamefont
  {Salinger}}]{stanley2002}%
  \BibitemOpen
  \bibfield  {author} {\bibinfo {author} {\bibfnamefont {H.}~\bibnamefont
  {Stanley}}, \bibinfo {author} {\bibfnamefont {L.}~\bibnamefont {Amaral}},
  \bibinfo {author} {\bibfnamefont {S.~V.}\ \bibnamefont {Buldyrev}}, \bibinfo
  {author} {\bibfnamefont {P.}~\bibnamefont {Gopikrishnan}}, \bibinfo {author}
  {\bibfnamefont {V.}~\bibnamefont {Plerou}},\ and\ \bibinfo {author}
  {\bibfnamefont {M.}~\bibnamefont {Salinger}},\ }\href@noop {} {\bibfield
  {journal} {\bibinfo  {journal} {Proceedings of the National Academy of
  Sciences}\ }\textbf {\bibinfo {volume} {99}},\ \bibinfo {pages} {2561}
  (\bibinfo {year} {2002})}\BibitemShut {NoStop}%
\bibitem [{\citenamefont {Dickman}\ \emph {et~al.}(1998)\citenamefont
  {Dickman}, \citenamefont {Vespignani},\ and\ \citenamefont
  {Zapperi}}]{dickman1998}%
  \BibitemOpen
  \bibfield  {author} {\bibinfo {author} {\bibfnamefont {R.}~\bibnamefont
  {Dickman}}, \bibinfo {author} {\bibfnamefont {A.}~\bibnamefont
  {Vespignani}},\ and\ \bibinfo {author} {\bibfnamefont {S.}~\bibnamefont
  {Zapperi}},\ }\href@noop {} {\bibfield  {journal} {\bibinfo  {journal}
  {Physical Review E}\ }\textbf {\bibinfo {volume} {57}},\ \bibinfo {pages}
  {5095} (\bibinfo {year} {1998})}\BibitemShut {NoStop}%
\bibitem [{\citenamefont {Dickman}\ \emph {et~al.}(2000)\citenamefont
  {Dickman}, \citenamefont {Mu{\~n}oz}, \citenamefont {Vespignani},\ and\
  \citenamefont {Zapperi}}]{dickman2000}%
  \BibitemOpen
  \bibfield  {author} {\bibinfo {author} {\bibfnamefont {R.}~\bibnamefont
  {Dickman}}, \bibinfo {author} {\bibfnamefont {M.~A.}\ \bibnamefont
  {Mu{\~n}oz}}, \bibinfo {author} {\bibfnamefont {A.}~\bibnamefont
  {Vespignani}},\ and\ \bibinfo {author} {\bibfnamefont {S.}~\bibnamefont
  {Zapperi}},\ }\href@noop {} {\bibfield  {journal} {\bibinfo  {journal}
  {Brazilian Journal of Physics}\ }\textbf {\bibinfo {volume} {30}},\ \bibinfo
  {pages} {27} (\bibinfo {year} {2000})}\BibitemShut {NoStop}%
\bibitem [{\citenamefont {Van~Wijland}\ \emph {et~al.}(1998)\citenamefont
  {Van~Wijland}, \citenamefont {Oerding},\ and\ \citenamefont
  {Hilhorst}}]{van1998}%
  \BibitemOpen
  \bibfield  {author} {\bibinfo {author} {\bibfnamefont {F.}~\bibnamefont
  {Van~Wijland}}, \bibinfo {author} {\bibfnamefont {K.}~\bibnamefont
  {Oerding}},\ and\ \bibinfo {author} {\bibfnamefont {H.}~\bibnamefont
  {Hilhorst}},\ }\href@noop {} {\bibfield  {journal} {\bibinfo  {journal}
  {Physica A: Statistical Mechanics and its Applications}\ }\textbf {\bibinfo
  {volume} {251}},\ \bibinfo {pages} {179} (\bibinfo {year}
  {1998})}\BibitemShut {NoStop}%
\bibitem [{\citenamefont {Frey}\ \emph {et~al.}(1994)\citenamefont {Frey},
  \citenamefont {T\"auber},\ and\ \citenamefont {Schwabl}}]{IP-DP-cross-FT}%
  \BibitemOpen
  \bibfield  {author} {\bibinfo {author} {\bibfnamefont {E.}~\bibnamefont
  {Frey}}, \bibinfo {author} {\bibfnamefont {U.~C.}\ \bibnamefont {T\"auber}},\
  and\ \bibinfo {author} {\bibfnamefont {F.}~\bibnamefont {Schwabl}},\
  }\href@noop {} {\bibfield  {journal} {\bibinfo  {journal} {Phys. Rev. E}\
  }\textbf {\bibinfo {volume} {49}},\ \bibinfo {pages} {5058} (\bibinfo {year}
  {1994})}\BibitemShut {NoStop}%
\bibitem [{\citenamefont {Zhou}\ \emph {et~al.}(2012)\citenamefont {Zhou},
  \citenamefont {Yang}, \citenamefont {Ziff},\ and\ \citenamefont
  {Deng}}]{IP-DP-cross}%
  \BibitemOpen
  \bibfield  {author} {\bibinfo {author} {\bibfnamefont {Z.}~\bibnamefont
  {Zhou}}, \bibinfo {author} {\bibfnamefont {J.}~\bibnamefont {Yang}}, \bibinfo
  {author} {\bibfnamefont {R.~M.}\ \bibnamefont {Ziff}},\ and\ \bibinfo
  {author} {\bibfnamefont {Y.}~\bibnamefont {Deng}},\ }\href@noop {} {\bibfield
   {journal} {\bibinfo  {journal} {Phys. Rev. E}\ }\textbf {\bibinfo {volume}
  {86}},\ \bibinfo {pages} {021102} (\bibinfo {year} {2012})}\BibitemShut
  {NoStop}%
\bibitem [{\citenamefont {Korchinski}\ \emph {et~al.}(2021)\citenamefont
  {Korchinski}, \citenamefont {Orlandi}, \citenamefont {Son},\ and\
  \citenamefont {Davidsen}}]{PRX-Kor}%
  \BibitemOpen
  \bibfield  {author} {\bibinfo {author} {\bibfnamefont {D.~J.}\ \bibnamefont
  {Korchinski}}, \bibinfo {author} {\bibfnamefont {J.~G.}\ \bibnamefont
  {Orlandi}}, \bibinfo {author} {\bibfnamefont {S.-W.}\ \bibnamefont {Son}},\
  and\ \bibinfo {author} {\bibfnamefont {J.}~\bibnamefont {Davidsen}},\
  }\href@noop {} {\bibfield  {journal} {\bibinfo  {journal} {Physical Review
  X}\ }\textbf {\bibinfo {volume} {11}} (\bibinfo {year} {2021})}\BibitemShut
  {NoStop}%
\bibitem [{Note3()}]{Note3}%
  \BibitemOpen
  \bibinfo {note} {Numerical analysis showed, that in the presence of such
  composite reactions, the cascade size and duration distribution exponents
  crossover to larger values on complex networks~\cite {IP-small,IP-PL} for
  larger avalanche sizes and durations. This agrees and may explain our
  power-law fitting results for the failure durations, where we see a crossover
  around $t_c \simeq 24$ hours from smaller to bigger $\tau _t$ exponents.
  Note, that in the databases the outage times are rounded to 1-min or 1-hour
  slots, so shorter-time results may not be reliable}\BibitemShut {NoStop}%
\bibitem [{\citenamefont {{\'O}dor}(2008)}]{odorbook}%
  \BibitemOpen
  \bibfield  {author} {\bibinfo {author} {\bibfnamefont {G.}~\bibnamefont
  {{\'O}dor}},\ }\href@noop {} {\emph {\bibinfo {title} {Universality in
  nonequilibrium lattice systems: theoretical foundations}}}\ (\bibinfo
  {publisher} {World Scientific},\ \bibinfo {year} {2008})\BibitemShut
  {NoStop}%
\bibitem [{\citenamefont {Henkel}\ \emph {et~al.}(2008)\citenamefont {Henkel},
  \citenamefont {Hinrichsen},\ and\ \citenamefont {L{\"u}beck}}]{henkel2008}%
  \BibitemOpen
  \bibfield  {author} {\bibinfo {author} {\bibfnamefont {M.}~\bibnamefont
  {Henkel}}, \bibinfo {author} {\bibfnamefont {H.}~\bibnamefont {Hinrichsen}},\
  and\ \bibinfo {author} {\bibfnamefont {S.}~\bibnamefont {L{\"u}beck}},\
  }\href@noop {} {\emph {\bibinfo {title} {Non-equilibrium phase
  transitions}}},\ Vol.~\bibinfo {volume} {1}\ (\bibinfo  {publisher}
  {Springer},\ \bibinfo {year} {2008})\BibitemShut {NoStop}%
\bibitem [{\citenamefont {L{\"u}beck}(2003)}]{lubeck2003}%
  \BibitemOpen
  \bibfield  {author} {\bibinfo {author} {\bibfnamefont {S.}~\bibnamefont
  {L{\"u}beck}},\ }\href@noop {} {\bibfield  {journal} {\bibinfo  {journal}
  {Physical review letters}\ }\textbf {\bibinfo {volume} {90}},\ \bibinfo
  {pages} {210601} (\bibinfo {year} {2003})}\BibitemShut {NoStop}%
\bibitem [{\citenamefont {Henkel}\ and\ \citenamefont
  {Pleimling}(2010)}]{henkel2010}%
  \BibitemOpen
  \bibfield  {author} {\bibinfo {author} {\bibfnamefont {M.}~\bibnamefont
  {Henkel}}\ and\ \bibinfo {author} {\bibfnamefont {M.}~\bibnamefont
  {Pleimling}},\ }\href@noop {} {\emph {\bibinfo {title} {Non-equilibrium phase
  transitions}}},\ Vol.~\bibinfo {volume} {2}\ (\bibinfo  {publisher}
  {Springer},\ \bibinfo {year} {2010})\BibitemShut {NoStop}%
\bibitem [{\citenamefont {Griffiths}(1969)}]{griffiths1969}%
  \BibitemOpen
  \bibfield  {author} {\bibinfo {author} {\bibfnamefont {R.~B.}\ \bibnamefont
  {Griffiths}},\ }\href@noop {} {\bibfield  {journal} {\bibinfo  {journal}
  {Physical Review Letters}\ }\textbf {\bibinfo {volume} {23}},\ \bibinfo
  {pages} {17} (\bibinfo {year} {1969})}\BibitemShut {NoStop}%
\bibitem [{\citenamefont {Munoz}\ \emph {et~al.}(2010)\citenamefont {Munoz},
  \citenamefont {Juh{\'a}sz}, \citenamefont {Castellano},\ and\ \citenamefont
  {{\'O}dor}}]{munoz2010}%
  \BibitemOpen
  \bibfield  {author} {\bibinfo {author} {\bibfnamefont {M.~A.}\ \bibnamefont
  {Munoz}}, \bibinfo {author} {\bibfnamefont {R.}~\bibnamefont {Juh{\'a}sz}},
  \bibinfo {author} {\bibfnamefont {C.}~\bibnamefont {Castellano}},\ and\
  \bibinfo {author} {\bibfnamefont {G.}~\bibnamefont {{\'O}dor}},\ }\href@noop
  {} {\bibfield  {journal} {\bibinfo  {journal} {Physical review letters}\
  }\textbf {\bibinfo {volume} {105}},\ \bibinfo {pages} {128701} (\bibinfo
  {year} {2010})}\BibitemShut {NoStop}%
\bibitem [{\citenamefont {Doyle}\ and\ \citenamefont
  {Carlson}(2000)}]{doyle2000}%
  \BibitemOpen
  \bibfield  {author} {\bibinfo {author} {\bibfnamefont {J.}~\bibnamefont
  {Doyle}}\ and\ \bibinfo {author} {\bibfnamefont {J.~M.}\ \bibnamefont
  {Carlson}},\ }\href@noop {} {\bibfield  {journal} {\bibinfo  {journal}
  {Physical Review Letters}\ }\textbf {\bibinfo {volume} {84}},\ \bibinfo
  {pages} {5656} (\bibinfo {year} {2000})}\BibitemShut {NoStop}%
\bibitem [{\citenamefont {Chow}\ \emph {et~al.}(1996)\citenamefont {Chow},
  \citenamefont {Taylor},\ and\ \citenamefont {Chow}}]{517530}%
  \BibitemOpen
  \bibfield  {author} {\bibinfo {author} {\bibfnamefont {M.-Y.}\ \bibnamefont
  {Chow}}, \bibinfo {author} {\bibfnamefont {L.}~\bibnamefont {Taylor}},\ and\
  \bibinfo {author} {\bibfnamefont {M.-S.}\ \bibnamefont {Chow}},\ }\href@noop
  {} {\bibfield  {journal} {\bibinfo  {journal} {IEEE Transactions on Power
  Delivery}\ }\textbf {\bibinfo {volume} {11}},\ \bibinfo {pages} {1652}
  (\bibinfo {year} {1996})}\BibitemShut {NoStop}%
\bibitem [{\citenamefont {Rodriguez}\ and\ \citenamefont
  {Vargas}(2005)}]{1425608}%
  \BibitemOpen
  \bibfield  {author} {\bibinfo {author} {\bibfnamefont {J.}~\bibnamefont
  {Rodriguez}}\ and\ \bibinfo {author} {\bibfnamefont {A.}~\bibnamefont
  {Vargas}},\ }\href@noop {} {\bibfield  {journal} {\bibinfo  {journal} {IEEE
  Transactions on Power Systems}\ }\textbf {\bibinfo {volume} {20}},\ \bibinfo
  {pages} {1095} (\bibinfo {year} {2005})}\BibitemShut {NoStop}%
\bibitem [{\citenamefont {Maliszewski}\ and\ \citenamefont
  {Perrings}(2012)}]{MALISZEWSKI2012668}%
  \BibitemOpen
  \bibfield  {author} {\bibinfo {author} {\bibfnamefont {P.~J.}\ \bibnamefont
  {Maliszewski}}\ and\ \bibinfo {author} {\bibfnamefont {C.}~\bibnamefont
  {Perrings}},\ }\href@noop {} {\bibfield  {journal} {\bibinfo  {journal}
  {Applied Geography}\ }\textbf {\bibinfo {volume} {32}},\ \bibinfo {pages}
  {668} (\bibinfo {year} {2012})}\BibitemShut {NoStop}%
\bibitem [{\citenamefont {Vinogradov}\ \emph {et~al.}(2020)\citenamefont
  {Vinogradov}, \citenamefont {Bolshev}, \citenamefont {Vinogradova},
  \citenamefont {Jasiński}, \citenamefont {Sikorski}, \citenamefont
  {Leonowicz}, \citenamefont {Goňo},\ and\ \citenamefont
  {Jasińska}}]{en13112736}%
  \BibitemOpen
  \bibfield  {author} {\bibinfo {author} {\bibfnamefont {A.}~\bibnamefont
  {Vinogradov}}, \bibinfo {author} {\bibfnamefont {V.}~\bibnamefont {Bolshev}},
  \bibinfo {author} {\bibfnamefont {A.}~\bibnamefont {Vinogradova}}, \bibinfo
  {author} {\bibfnamefont {M.}~\bibnamefont {Jasiński}}, \bibinfo {author}
  {\bibfnamefont {T.}~\bibnamefont {Sikorski}}, \bibinfo {author}
  {\bibfnamefont {Z.}~\bibnamefont {Leonowicz}}, \bibinfo {author}
  {\bibfnamefont {R.}~\bibnamefont {Goňo}},\ and\ \bibinfo {author}
  {\bibfnamefont {E.}~\bibnamefont {Jasińska}},\ }\href@noop {} {\bibfield
  {journal} {\bibinfo  {journal} {Energies}\ }\textbf {\bibinfo {volume} {13}}
  (\bibinfo {year} {2020})}\BibitemShut {NoStop}%
\bibitem [{\citenamefont {Carrington}\ \emph {et~al.}(2021)\citenamefont
  {Carrington}, \citenamefont {Dobson},\ and\ \citenamefont {Wang}}]{9410392}%
  \BibitemOpen
  \bibfield  {author} {\bibinfo {author} {\bibfnamefont {N.~K.}\ \bibnamefont
  {Carrington}}, \bibinfo {author} {\bibfnamefont {I.}~\bibnamefont {Dobson}},\
  and\ \bibinfo {author} {\bibfnamefont {Z.}~\bibnamefont {Wang}},\ }\href
  {https://doi.org/10.1109/TPWRS.2021.3074898} {\bibfield  {journal} {\bibinfo
  {journal} {IEEE Transactions on Power Systems}\ }\textbf {\bibinfo {volume}
  {36}},\ \bibinfo {pages} {5814} (\bibinfo {year} {2021})}\BibitemShut
  {NoStop}%
\bibitem [{\citenamefont {Harvey}\ \emph {et~al.}(2005)\citenamefont {Harvey},
  \citenamefont {Hogan},\ and\ \citenamefont
  {Schatzki}}]{RePEc:ecl:harjfk:rwp05-027}%
  \BibitemOpen
  \bibfield  {author} {\bibinfo {author} {\bibfnamefont {S.~M.}\ \bibnamefont
  {Harvey}}, \bibinfo {author} {\bibfnamefont {W.~W.}\ \bibnamefont {Hogan}},\
  and\ \bibinfo {author} {\bibfnamefont {T.}~\bibnamefont {Schatzki}},\
  }\href@noop {} {\emph {\bibinfo {title} {{A Hazard Rate Analysis of
  Mirant’s Generating Plant Outages in California (Jan-04)}}}},\ \bibinfo
  {type} {Working Paper Series}\ \bibinfo {number} {rwp05-027}\ (\bibinfo
  {institution} {Harvard University, John F. Kennedy School of Government},\
  \bibinfo {year} {2005})\BibitemShut {NoStop}%
\bibitem [{\citenamefont {Kim}\ \emph {et~al.}(2020)\citenamefont {Kim},
  \citenamefont {Chang}, \citenamefont {Kim},\ and\ \citenamefont
  {Kim}}]{en13143571}%
  \BibitemOpen
  \bibfield  {author} {\bibinfo {author} {\bibfnamefont {T.-W.}\ \bibnamefont
  {Kim}}, \bibinfo {author} {\bibfnamefont {Y.}~\bibnamefont {Chang}}, \bibinfo
  {author} {\bibfnamefont {D.-W.}\ \bibnamefont {Kim}},\ and\ \bibinfo {author}
  {\bibfnamefont {M.-K.}\ \bibnamefont {Kim}},\ }\href@noop {} {\bibfield
  {journal} {\bibinfo  {journal} {Energies}\ }\textbf {\bibinfo {volume} {13}}
  (\bibinfo {year} {2020})}\BibitemShut {NoStop}%
\bibitem [{\citenamefont {Wu}\ \emph {et~al.}(2022)\citenamefont {Wu},
  \citenamefont {Meng}, \citenamefont {Danziger}, \citenamefont {Cornelius},
  \citenamefont {Tian},\ and\ \citenamefont {Barab{\'a}si}}]{Wu2022}%
  \BibitemOpen
  \bibfield  {author} {\bibinfo {author} {\bibfnamefont {H.}~\bibnamefont
  {Wu}}, \bibinfo {author} {\bibfnamefont {X.}~\bibnamefont {Meng}}, \bibinfo
  {author} {\bibfnamefont {M.~M.}\ \bibnamefont {Danziger}}, \bibinfo {author}
  {\bibfnamefont {S.~P.}\ \bibnamefont {Cornelius}}, \bibinfo {author}
  {\bibfnamefont {H.}~\bibnamefont {Tian}},\ and\ \bibinfo {author}
  {\bibfnamefont {A.-L.}\ \bibnamefont {Barab{\'a}si}},\ }\href
  {https://doi.org/10.1038/s41467-022-35104-9} {\bibfield  {journal} {\bibinfo
  {journal} {Nature Communications}\ }\textbf {\bibinfo {volume} {13}},\
  \bibinfo {pages} {7372} (\bibinfo {year} {2022})}\BibitemShut {NoStop}%
\bibitem [{\citenamefont {Harvey}\ \emph {et~al.}(2004)\citenamefont {Harvey},
  \citenamefont {Hogan},\ and\ \citenamefont {Schatzki}}]{harvey2004}%
  \BibitemOpen
  \bibfield  {author} {\bibinfo {author} {\bibfnamefont {S.~M.}\ \bibnamefont
  {Harvey}}, \bibinfo {author} {\bibfnamefont {W.~W.}\ \bibnamefont {Hogan}},\
  and\ \bibinfo {author} {\bibfnamefont {T.}~\bibnamefont {Schatzki}},\ }in\
  \href@noop {} {\emph {\bibinfo {booktitle} {Toulouse Conference paper}}}\
  (\bibinfo {year} {2004})\BibitemShut {NoStop}%
\bibitem [{Note4()}]{Note4}%
  \BibitemOpen
  \bibinfo {note} {For both $S(t)$ and $I(t)$, we had first excluded events
  recorded at 1/4, 1/2, 3/4 hours and whole hours to eliminate any artifacts in
  outage bookkeeping time, if many events were recorded at such time
  points.}\BibitemShut {Stop}%
\bibitem [{\citenamefont {Brown}\ and\ \citenamefont
  {Hwang}(1997)}]{brown1997}%
  \BibitemOpen
  \bibfield  {author} {\bibinfo {author} {\bibfnamefont {R.~G.}\ \bibnamefont
  {Brown}}\ and\ \bibinfo {author} {\bibfnamefont {P.~Y.}\ \bibnamefont
  {Hwang}},\ }\href@noop {} {\bibfield  {journal} {\bibinfo  {journal}
  {Introduction to random signals and applied Kalman filtering: with MATLAB
  exercises and solutions}\ } (\bibinfo {year} {1997})}\BibitemShut {NoStop}%
\bibitem [{\citenamefont {Laurson}\ \emph {et~al.}(2005)\citenamefont
  {Laurson}, \citenamefont {Alava},\ and\ \citenamefont
  {Zapperi}}]{laurson2005}%
  \BibitemOpen
  \bibfield  {author} {\bibinfo {author} {\bibfnamefont {L.}~\bibnamefont
  {Laurson}}, \bibinfo {author} {\bibfnamefont {M.~J.}\ \bibnamefont {Alava}},\
  and\ \bibinfo {author} {\bibfnamefont {S.}~\bibnamefont {Zapperi}},\
  }\href@noop {} {\bibfield  {journal} {\bibinfo  {journal} {Journal of
  Statistical Mechanics: Theory and Experiment}\ }\textbf {\bibinfo {volume}
  {2005}},\ \bibinfo {pages} {L11001} (\bibinfo {year} {2005})}\BibitemShut
  {NoStop}%
\bibitem [{KAL(2009)}]{KALASHYAN2009895}%
  \BibitemOpen
  \href@noop {} {\bibfield  {journal} {\bibinfo  {journal} {Chaos, Solitons \&
  Fractals}\ }\textbf {\bibinfo {volume} {39}},\ \bibinfo {pages} {895}
  (\bibinfo {year} {2009})}\BibitemShut {NoStop}%
\bibitem [{\citenamefont {Moore}\ and\ \citenamefont
  {Newman}(2000)}]{IP-small}%
  \BibitemOpen
  \bibfield  {author} {\bibinfo {author} {\bibfnamefont {C.}~\bibnamefont
  {Moore}}\ and\ \bibinfo {author} {\bibfnamefont {M.~E.~J.}\ \bibnamefont
  {Newman}},\ }\href@noop {} {\bibfield  {journal} {\bibinfo  {journal} {Phys.
  Rev. E}\ }\textbf {\bibinfo {volume} {62}},\ \bibinfo {pages} {7059}
  (\bibinfo {year} {2000})}\BibitemShut {NoStop}%
\bibitem [{\citenamefont {Cohen}\ \emph {et~al.}(2002)\citenamefont {Cohen},
  \citenamefont {ben Avraham},\ and\ \citenamefont {Havlin}}]{IP-PL}%
  \BibitemOpen
  \bibfield  {author} {\bibinfo {author} {\bibfnamefont {R.}~\bibnamefont
  {Cohen}}, \bibinfo {author} {\bibfnamefont {D.}~\bibnamefont {ben Avraham}},\
  and\ \bibinfo {author} {\bibfnamefont {S.}~\bibnamefont {Havlin}},\
  }\href@noop {} {\bibfield  {journal} {\bibinfo  {journal} {Phys. Rev. E}\
  }\textbf {\bibinfo {volume} {66}},\ \bibinfo {pages} {036113} (\bibinfo
  {year} {2002})}\BibitemShut {NoStop}%
\end{thebibliography}%
\end{document}
%
% ****** End of file apssamp.tex ******
