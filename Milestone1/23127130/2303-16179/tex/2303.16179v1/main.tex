\documentclass[10pt]{amsart}

\usepackage[top=3cm,bottom=3cm,left=3.5cm,right=3.5cm,marginparsep=0.3cm,marginparwidth=3cm,includefoot]{geometry}
\usepackage[foot]{amsaddr}
\usepackage{amssymb}
\usepackage{mathabx}
\usepackage[backend = biber]{biblatex}
\addbibresource{biblio_latin.bib}
%\usepackage{natbib}
\usepackage{bbm}
\usepackage[english]{babel}
\usepackage[latin1]{inputenc}
%\usepackage[sorting = nty, bibstyle=alphabetic]{biblatex}
%\addbibresource{biblio.bib}
\usepackage[T1]{fontenc}
%\usepackage{cleveref}
\usepackage{hyperref}
\hypersetup{
    colorlinks = true,
    linkcolor = purple,
    citecolor = brown,
    urlcolor = blue
    }
\usepackage{ifpdf}
\ifpdf
  \usepackage[pdftex]{graphicx}
  \usepackage{epstopdf}
\else
  \usepackage[dvips]{graphicx}
\fi
%\usepackage{subfigure}
%\usepackage{sidecap}
%\usepackage{pdfpages}
%\usepackage{algorithm}
%\usepackage{algorithmic}
%\usepackage[draft]{showlabels} % *** switch to final to remove
\usepackage{graphicx}
\usepackage{xcolor}
\usepackage{cleveref}
\setcounter{tocdepth}{1}
\usepackage[all]{xy}
\usepackage{csquotes}
\usepackage{dynkin-diagrams}

% **************************************************************************

%\makeatletter
%\renewcommand\theequation%
%{\thesection.\arabic{equation}}
%\@addtoreset{equation}{section}
\makeatother

\theoremstyle{plain}
\newtheorem{theorem}{Theorem}[section]
\newtheorem*{freetheorem}{Conjecture}
\newtheorem{lemma}[theorem]{Lemma}
\newtheorem{proposition}[theorem]{Proposition}
\newtheorem{corollary}[theorem]{Corollary}
\newtheorem{question}[theorem]{Question}
\newtheorem{conjecture}[theorem]{Conjecture}
\newtheorem{principle}[theorem]{Principle}
%\newtheorem{question}[theorem]{Question}
%\newtheorem{algorithm}[theorem]{Algorithm}
%\newtheorem{criterion}[theorem]{Criterion}

\theoremstyle{definition}
\newtheorem{definition}[theorem]{Definition}
\newtheorem{property}[theorem]{Property}
\newtheorem{convention}[theorem]{Convention}
\newtheorem{notation}[theorem]{Notation}
\newtheorem{remark}[theorem]{Remark}
%\newtheorem{problem}[definition]{Problem}

\theoremstyle{remark}
\newtheorem{example}[theorem]{Example}
\newtheorem*{note}{Note}

%\newcommand{\todo}[1]{\textcolor{blue}{Référence : #1}}
%\newcommand{\question}[1]{\textcolor{purple}{Question : #1}}
%\newcommand{\warning}[1]{\textcolor{red}{Attention : #1}}
%\newcommand{\change}[1]{\textcolor{green}{À changer : #1}}
%\newcommand{\comment}[1]{\textcolor{brown}{Commentaire : #1}}

\newcommand{\N}{\mathbb N}
\newcommand{\Z}{\mathbb Z}
\newcommand{\Q}{\mathbb Q}
\newcommand{\C}{\mathbb C}
\newcommand{\R}{\mathbb R}
\newcommand{\D}{\mathbb D}
\newcommand{\V}{\mathbb V}
\newcommand{\U}{\mathbb U}
\newcommand{\p}{\mathbb P}
\newcommand{\s}{\mathbb S}
\newcommand{\sm}{\mathrm{sm}}
\newcommand{\zar}{\mathrm{Zar}}
\newcommand{\bi}{\mathrm{bi}}
\newcommand{\ppabel}{\mathcal{A}}
\newcommand{\cur}{\mathcal{M}}
\newcommand{\torelli}{\mathcal{T}}
\newcommand{\halfplane}{\mathbb H}
\newcommand{\fibre}{\mathcal{H}}
\newcommand{\mult}{\mathbb G}
\newcommand{\G}{\mathbf G}
\newcommand{\mon}{\mathbf H}
\newcommand{\M}{\mathbf M}
\newcommand{\LL}{\mathbf L}
\newcommand{\MT}{\mathbf{MT}}
\newcommand{\torus}{\mathbf T}
\newcommand{\HL}{\mathrm{HL}(S, \V^\otimes)}
\newcommand{\HLM}{\mathrm{HL}(S, \V^\otimes, \M)}
\newcommand{\HLMI}{\mathrm{HL}(S, \V_I^\otimes, \M')}
\newcommand{\e}{\ast}
\newcommand{\bul}{\bullet}
\newcommand{\la}{\lambda}
\newcommand{\im}{\mathrm{im} \hspace{0.05cm}}
\newcommand{\id}{\mathrm{id}}
\newcommand{\pr}{\mathrm{pr}}
\newcommand{\an}{\mathrm{an}}
\newcommand{\hod}{\mathrm{Hdg}}
\newcommand{\vol}{\mathrm{vol}}
\newcommand{\Hom}{\mathrm{Hom}}
\newcommand{\hol}{\mathcal{O}}
\newcommand{\del}{\partial}
\newcommand{\suniv}{\widetilde{S^\an}}
\newcommand{\delbar}{\overline{\partial}}
\newcommand{\h}{\mathfrak{h}}
\newcommand{\quo}{\backslash}
\newcommand{\codim}{\mathrm{codim}}
\newcommand{\sh}{\Q\mathsf{SH}}
\newcommand{\ab}{\Q-\mathsf{ev}}
\newcommand{\rd}{\mathsf{RD}}
\newcommand{\cl}{\mathrm{cl}}
\newcommand{\card}{\mathrm{card}}
\newcommand{\gl}{\mathbf{GL}}
\newcommand{\spl}{\mathbf{SL}}
\newcommand{\gsp}{\mathbf{GSp}}
\newcommand{\ssp}{\mathbf{Sp}}
\newcommand{\gal}{\mathrm{Gal}}
\newcommand{\res}{\mathrm{Res}}
\newcommand{\conj}{\overline}
\newcommand{\spa}{\hspace{0.1cm}}
\newcommand{\hp}{\tilde{h}}
\newcommand{\aut}{\mathrm{Aut}}
\newcommand{\en}{\mathrm{End}}
\newcommand{\sym}{\mathrm{Sym}}
\newcommand{\g}{\mathfrak{g}}
\newcommand{\car}{\mathfrak{t}}
\newcommand{\ph}{\varphi}
\newcommand{\tph}{\tilde{\varphi}}
\newcommand{\rg}{\mathrm{rg} \hspace{0.05cm}}
\newcommand{\bor}{\mathfrak{b}}
\newcommand{\para}{\mathfrak{p}}
\newcommand{\comp}{\mathfrak{k}}
\newcommand{\ncomp}{\mathfrak{q}}
\newcommand{\mt}{\mathfrak{m}}
\newcommand{\ort}{\mathfrak{h}}
\newcommand{\abel}{\mathfrak{a}}
\newcommand{\su}{\mathfrak{su}}
\newcommand{\sll}{\mathfrak{sl}}
\newcommand{\Ad}{\mathrm{Ad}}
\newcommand{\ad}{\mathrm{ad}}
\newcommand{\gra}{\mathtt{T}}
\newcommand{\ci}{\mathtt{L}}
\newcommand{\supp}{\mathrm{supp}}
\newcommand{\period}{\mathcal{D}}
\newcommand{\eps}{\varepsilon}
\newcommand{\niv}{\mathrm{niv}}
\newcommand{\typ}{\mathrm{typ}}
\newcommand{\atyp}{\mathrm{atyp}}
\newcommand{\pos}{\mathrm{pos}}
\newcommand{\der}{\mathrm{der}}
\newcommand{\ch}[1]{\widecheck{{#1}}}
% **************************************************************************

\title{Existence and Density of typical Hodge Loci}
\author{Nazim Khelifa}
\author{David Urbanik}
\date{\today}

% **************************************************************************

\begin{document}
%\bibliography{memoire}
%\bibliographystyle{amsplain}
\maketitle
\begin{abstract}
Motivated by a question of Baldi-Klingler-Ullmo, we provide a general sufficient criterion for the existence and analytic density of typical Hodge loci associated to a polarizable $\Z$-variation of Hodge structures $\V$. Our criterion reproves the existing results in the literature on density of Noether-Lefschetz loci. It also applies to understand Hodge loci of subvarieties of $\mathcal{A}_{g}$. For instance, we prove that for $g \geqslant 4$, if a subvariety $S$ of $\mathcal{A}_{g}$ has dimension at least $g$ then it has an analytically dense typical Hodge locus. This applies for example to the Torelli locus of $\ppabel_g$. 
\end{abstract}
\begin{center}
\tableofcontents
\end{center}
\section{Introduction}
\subsection{Context} The observation that Hodge loci of polarizable $\Z$-variations of Hodge structures ($\Z$-VHS) can be realized as intersection loci has recently been fruitfully applied to obtain refinements of the celebrated algebraicity result of Cattani-Deligne-Kaplan \cite{cdk, bkt}. For instance, if $\V$ is a $\Z$-VHS on a smooth, irreducible and quasi-projective variety $S$ over $\C$, one would like to understand criteria for density, in both the Zariski and analytic topology, for families of components of the Hodge locus $\HL$. With this goal in mind Klingler introduced a dichotomy, later refined by Baldi-Klingler-Ullmo, between the part of the Hodge locus corresponding to typical intersections and the part corresponding to atypical ones (\cite[Def. 2.2, 2.4]{bku}):
\[
\HL = \HL_\typ \cup \HL_\atyp.
\]
These two parts are expected to behave very differently. On the one hand, the atypical part is conjectured to be algebraic:
\begin{conjecture}[{\cite[Conj. 2.5]{bku}}]
\label{zilberpink}
Let $\V$ be a polarizable $\Z$-VHS on a smooth, irreducible and quasi-projective variety $S$ over $\C$. The atypical Hodge locus $\HL_\atyp$ of $S$ for $\V$ is an algebraic suvariety of $S$.
\end{conjecture}
\noindent Indeed, Conjecture \ref{zilberpink} can be seen as an extension of the classical Zilber-Pink conjecture and appears, for atypical points, to be entirely out of reach. Understanding the typical Hodge locus, however, appears more accessible, and towards this end Baldi-Klingler-Ullmo conjecture the following:
\begin{conjecture}[{\cite[Conj. 2.7]{bku}}]
\label{typdenseconj}
Let $\V$ be a polarizable $\Z$-VHS on a smooth, irreducible and quasi-projective variety $S$ over $\C$. If $\HL_\typ$ is non-empty, $\HL_\typ$ is analytically (hence Zariski) dense in $S$.
\end{conjecture}
Conjecture \ref{typdenseconj} naturally raises the following question asked by Baldi-Klingler-Ullmo:
\begin{question}[{\cite[Quest. 2.9]{bku}}]
\label{criterionquestion}
Is there a simple criterion on the generic Hodge datum of a polarizable $\Z$-VHS to decide if its typical Hodge locus is non-empty?
\end{question}
Towards this question, they prove that if the $\Z$-VHS $\V$ has level greater or equal to $3$, its typical Hodge locus is empty, leaving open the case of levels $1$ and $2$.
\subsection{Our Results} The main contribution of this work will be to provide a sufficient criterion for the density of (typical) Hodge loci. As we will explain, this gives almost a complete answer to Question \ref{criterionquestion}.
\subsubsection{Density of typical Hodge loci} Let $\V$ be a polarizable $\Z$-VHS on a smooth, irreducible and quasi-projective variety $S$ over $\C$, let $(\G,D)$ be its generic Hodge datum, and $\Phi : S^\an \rightarrow \Gamma \quo D$ be an associated period map. Given a strict Hodge sub-datum $(\M, D_M) \subsetneq (\G,D)$ we define the \textit{Hodge locus of type $\M$} as
\[
\HLM := \{s \in S^\an \hspace{0.1cm} : \hspace{0.1cm} \exists g \in \G(\Q)^+, \MT(\V_s) \subseteq g \M g^{-1} \},
\]
where for $s \in S^\an$, $\MT(\V_s)$ denotes the Mumford-Tate group of the Hodge structure on $\V_s$. The \textit{typical Hodge locus of type $\M$} is defined as the union $\HLM_\typ$ of those components of $\HLM$ which are typical with respect to their generic Hodge datum; see Definition \ref{typdef}.
\begin{definition}
Let $X,Y$ be closed analytic subvarieties of some complex analytic variety $Z$. Let $U$ be an analytic irreducible component of the intersection $X \cap Y$. We say that \textit{$X$ and $Y$ intersect in $Z$ with bigger dimension than expected along $U$} if
\[
\dim U > \dim X + \dim Y - \dim Z.
\]
Otherwise, we say that they \textit{intersect in $Z$ with expected dimension along $U$}.
\end{definition}

In this terminology, a special subvariety $Z$ of $S$ is typical if it corresponds to a component of the intersection between $\Gamma_{\G_Z} \quo D_{G_Z}$ and $\Phi(S^\an)$ along which these two intersect with expected dimension inside $\Gamma \quo D$. Then, a first trivial remark is that for $\HLM_\typ$ to be non-empty, the Hodge datum $(\M, D_M)$ has to be of the following type:
\begin{definition}
A strict Hodge sub-datum $(\M,D_M) \subsetneq (\G,D)$ is said \textit{$\V$-admissible} if it satisfies the inequality of dimensions
\[ \dim \Phi(S^\an) + \dim D_M - \dim D \geq 0. \]
We say that $(\M, D_M)$ is \textit{strongly $\V$-admissible} if the above inequality is strict.
\end{definition}
Our general result (see Theorem \ref{main}) is somewhat technical, but simplifies in the case where the algebraic monodromy group $\mon$ of $\mathbb{V}$ is $\mathbb{Q}$-simple and equal to $\G^\der$, saying that (strong) $\V$-admissibility of a sub-datum is enough to get density of the associated (typical) Hodge locus:
\begin{theorem}\label{mainqsimple}
Let $\V$ be a polarizable $\Z$-VHS on a smooth, irreducible and quasi-projective variety $S$ over $\C$ with generic Hodge datum $(\G,D)$. Assume that its algebraic monodromy group is $\mon = \G^\der$ and is $\Q$-simple. Let $(\M, D_M)\subsetneq (\G, D)$ be a strict Hodge sub-datum.
\begin{itemize}
\item[(i)] If $(\M, D_M)$ is $\V$-admissible, then $\HLM$ is analytically dense in $S^\an$. 
\item[(ii)] If $(\M, D_M)$ is strongly $\V$-admissible, then $\HLM_\typ$ is analytically dense in $S^\an$.
\end{itemize}
\end{theorem}
\begin{remark}
As pointed out in \cite[Rmk. 10.2]{bku}, Baldi-Klingler-Ullmo's geometric Zilber-Pink theorem gives (because we are in the $\Q$-simple monodromy case) that (i) implies (ii). We give an independent argument for (ii) that still works in the factor case.
\end{remark}
In terms of the full Hodge locus we thus have immediately:
\begin{corollary}\label{qsimplemon}
Let $\mathbb{V}$ be as in Theorem \ref{mainqsimple}. If $(\G,D)$ admits a (resp. strongly) $\V$-admissible strict Hodge sub-datum, the Hodge locus $\HL$ (resp. the typical Hodge locus $\HL_{\typ}$) of $S$ for $\V$ is analytically dense in $S^\an$. 
\end{corollary}
\begin{remark}\label{constfactorrem}
The condition that $\mon = \G^\der$ will be verified in most geometric applications (e.g. hypersurfaces variations, Hodge-generic subvarieties of $\ppabel_g$). It is however needed to exclude the eventuality that the datum $(\M,D_M)$ is $\V$-admissible but not big enough on the factor of $D$ on which $\Phi$ is constant to generically intersect the period image. The most general condition that one can hope for (see Theorem \ref{main}) is that, writing $\G^\der = \mon \cdot \LL$ as an almost direct product, $\M$ contains $\LL$. Otherwise, the statement would contradict the Zilber-Pink conjecture for an auxiliary $\Z$-VHS (see Example \ref{factorwisexample}).
\end{remark}
Working with not necessarily $\Q$-simple algebraic monodromy groups leads to complications. The following example illustrates the fact that in this general setting we cannot hope for Theorem \ref{mainqsimple} to remain true and that we need to impose similar numerical conditions on each factor.
\begin{example}\label{factorwisexample}
Let $C \subset \ppabel_3$ be a Hodge generic curve, where $\ppabel_3$ is the Shimura variety parametrising $3$-dimensional abelian varieties with level-$n$-structure (for some fixed integer $n \geqslant 3$). Let $\V$ be the $\Z$-VHS with period map the inclusion $\ppabel_3 \times C \hookrightarrow \ppabel_3 \times \ppabel_3$. Let $(\M,D_M)$ be a strict Hodge sub-datum with $\Gamma_\M \quo D_M = \ppabel_3 \times (\ppabel_1 \times \ppabel_2)$. It is strongly $\V$-admissible. If $\HLM_\typ$ was dense, one would find that $C \subset \mathcal{A}_{3}$ intersects Hecke translates of $\mathcal{A}_{1} \times \mathcal{A}_{2}$ in a dense subset. This contradicts the Zilber-Pink conjecture \ref{zilberpink} for the variation on $C$ with period map $C \hookrightarrow \mathcal{A}_{3}$: $C$ is of dimension $1$ while $\ppabel_1 \times \ppabel_2$ has codimension $2$ in $\ppabel_3$, so the intersections of $C$ with Hecke translates of $\ppabel_1 \times \ppabel_2$ are atypical. So Theorem \ref{mainqsimple} should fail in the non-$\Q$-simple monodromy case.
\end{example}
Our criterion is nearly a complete answer to \ref{criterionquestion} in the sense that the only missing ingredient for our methods to give a full characterization of the density of typical Hodge loci is a better understanding of the distribution of typical special points in $S$. More precisely, using the methods of this work, we would need the following statement, which is a consequence of the Zilber-Pink conjecture \ref{zilberpink}, to get a full answer to \ref{criterionquestion}:
\begin{conjecture}\label{qsimpleconj}
Let $\V$ be as in Theorem \ref{mainqsimple}, and $(\M, D_M) \subsetneq (\G,D)$ be a strict Hodge sub-datum which satisfies
\[ \dim \Phi(S^\an) + \dim D_M = \dim D. \]
Then in any neighbourhood $B \subset S^\an$, one has a strict containment
\[ \HLM_\atyp \cap B \subsetneq \HLM \cap B . \]
\end{conjecture}
\noindent Indeed, the reason for distinguishing between admissibility and strong admissibility in Theorem \ref{mainqsimple}(ii) ultimately arises from the inability to handle typicality questions in the situation where one expects $\HLM$ to consist of typical special points. 
\begin{remark}
Combining recent work of Tayou-Tholozan and the density of $\HL_\typ$ in $S$ that we prove here, one should be able to obtain that this locus is even equidistributed in $S$ for a natural measure. We refer the interested reader to their paper \cite{TT}.
\end{remark}
\subsection{Applications}
\subsubsection{Classical Noether-Lefschetz loci} We now explain how to apply our results to reprove classical theorems on the density of Noether-Lefschetz loci. We consider in particular the classical result of Ciliberto-Harris-Miranda \cite{zbMATH04097545} on the density of the Noether-Lefschetz locus of degree $d$ hypersurfaces in $\mathbb{P}^{3}$ as soon as $d \geq 5$. (The case $d = 4$ requires extra analysis, which we omit.)

In this case one considers the variation $\mathbb{V} = (R^{2} f_{*} \underline{\mathbb{Z}})_{\textrm{prim}}$ on primitive cohomology associated to the universal family $f$ of such hypersurfaces, and the polarization $Q : \mathbb{V} \otimes \mathbb{V} \to \underline{\mathbb{Z}}$ induced by cup-product. Fix some $s \in S^\an$. The result of Beauville \cite{zbMATH03973031} guarantees that the algebraic monodromy group $\mon$ of $\mathbb{V}$ is the full orthogonal group $\textrm{Aut}(\mathbb{V}_{s}, Q_{s})$ stabilizing the polarizing form. As $Q_{s}$ is a Hodge class in $(\mathbb{V}_{s} \otimes \mathbb{V}_{s})^{\vee}$ and the algebraic monodromy group lies inside the generic Mumford-Tate group, this means that the generic Hodge datum for $\mathbb{V}$ takes the form $(\G, D)$, with $\G = \textrm{GAut}(\mathbb{V}_{s}, Q_{s})$ and with $D$ the space of all polarized Hodge structures with the same Hodge numbers $(h^{2,0}, h^{1,1}, h^{0,2})$ as $\mathbb{V}$. Given any subdatum $(\M, D_{M}) \subset (\G, D)$ corresponding to a single fixed Hodge vector, one computes that $\dim D - \dim D_{M} = h^{2,0}$. On the other hand it is classical that $h^{2,0} = {d-1 \choose 3}$, hence one has for $d \geq 5$ that 
\[ \dim D - \dim D_{M} = {d-1 \choose 3} < {d+3 \choose 3} - 16 = \dim \Phi(S^\an) . \]
Here we have computed the period dimension by observing that the projective space of degree $d$ homogeneous polynomials in $4$ variables is ${d+3 \choose 3} - 1$ dimensional, and the period map becomes generically injective modulo the natural action of $\textrm{SL}_{4}$ on the moduli space as a consequence of Donagi's generic Torelli theorem \cite{zbMATH03964006}, which holds assuming $d \geq 5$. Thus the datum $(\M, D_M)$ is strongly $\mathbb{V}$-admissible, and Theorem \ref{mainqsimple} lets us conclude.

We expect that the above argument, which requires essentially only a verification of Zariski dense monodromy and a generic Torelli theorem, can be repeated to reprove numerous other analytic density results in the literature.

\subsubsection{Special subvarieties of the Torelli locus}
Let $g \geqslant 4$ be an integer, and fix a level structure $n \geqslant 3$. Let $\ppabel_g$ be the Shimura variety parametrising principally polarized abelian varieties of dimension $g$ (with the given level structure) whose associated Shimura datum is $(\gsp_{2g}, \halfplane_g)$, where $\halfplane_g$ is the Siegel upper half-plane. Recall that $\ppabel_g$ is a quasi-projective complex variety of dimension $\frac{g(g+1)}{2}$ and that the largest dimension obtained by any of its strict special subvarieties is $\frac{g(g-1)}{2} + 1$, realised for example by $\ppabel_{g-1} \times \ppabel_1$. As suggested in \cite[Rmk 3.15]{bku}, one expects in this setting that:
\begin{conjecture}
Let $S \subset \ppabel_g$ be a closed Hodge generic subvariety of dimension $q$, and $\V$ the induced polarizable $\Z$-VHS on $S$. The typical Hodge locus of $S$ for $\V$ is analytically dense if and only if $q \geqslant g-1$. In that case, the Hecke translates of $\ppabel_{g-1} \times \ppabel_{1}$ in $\ppabel_g$ intersect $S$ in an analytically dense subset.
\end{conjecture}
Unconditionally, Theorem \ref{mainqsimple} gives the following:
\begin{corollary}\label{Ag}
Let $S \subset \ppabel_g$ be a closed Hodge generic subvariety of dimension $q$, and $\V$ the induced polarizable $\Z$-VHS on $S$. If $q \geqslant g$, then the typical Hodge locus of $S$ for $\V$ is analytically dense. In that case, the Hecke translates of $\ppabel_{g-1} \times \ppabel_{1}$ in $\ppabel_g$ intersect $S$ in an analytically (positive dimensional) dense subset.
\end{corollary}
\begin{proof}
$\V$ has generic Mumford-Tate group $\G = \gsp_{2g}$ whose derived subgroup $\G^\der = \ssp_{2g}$ is $\Q$-simple. In particular, since $\V$ is non-constant and the algebraic monodromy group $\mon$ is a $\mathbb{Q}$-normal subgroup of the derived Mumford-Tate group, one has $\mon = \G^\der = \ssp_{2g}$. To conclude, one simply remarks that the inequality $q \geqslant g$ is equivalent to the strong $\V$-admissiblity of the Hodge sub-datum corresponding to $\mathcal{A}_{g-1} \times \mathcal{A}_{1}$.
\end{proof}
Let $\cur_g$ be the moduli space of smooth projective curves of genus $g$ with $n$ marked points, let $j : \cur_g \hookrightarrow \ppabel_g$ the Torelli morphism, 
and $\torelli^0_g = j(\cur_g) \subset \ppabel_g$ the open Torelli locus. In this setting, Corollary \ref{Ag} gives :
\begin{corollary}\label{torelli}
The typical Hodge locus of $\cur_g$ for the polarizable $\Z$-VHS induced by $j$ is analytically dense in $\cur_g$.
\end{corollary}
\begin{proof}
$\torelli^0_g$ is known to be of dimension $3g - 3$ which is bigger than $g$ in our setting ($g \geqslant 4$). Therefore, applying Corollary \ref{Ag} to the closure of $\torelli^0_g$ gives the density of the Hodge locus in $\cur_g$ for the analytic topology.
\end{proof}
\begin{remark}
Similar questions had already been studied in \cite{izadi, colombopirola} for $\ppabel_g$ and \cite{chai} in the more general setting of Shimura varieties. Chai had developed a numerology to give a sufficient condition on the codimension of a subvariety of a Shimura variety to have analytically dense Hodge locus of given type $\M$. 
These conditions weren't however optimal, and Theorem \ref{mainqsimple} (and its generalisation Theorem \ref{main}) can be seen as a refinement of these works in three directions. First, we give numerical conditions in a general Hodge theoretic setting while previous authors only worked in the Shimura setting. Furthermore, our numerical conditions are sharper than previous ones, even in the Shimura setting. For example, they only knew that subvarieties of $\ppabel_g$ had analytically dense Hodge locus if they had codimension less than $g$. We prove the same thing for subvarieties of dimension greater than $g$. Finally, the notion of typicality doesn't appear in the aforementionned works, so proving density of the typical part of the Hodge locus is an additional improvement.
\end{remark}
\subsection{Recent related work}
Recently, S. Eterovi{\'c} and T. Scanlon have obtained similar results in a general framework of so-called definable arithmetic quotients. In particular, they also give \cite[Theorem 3.4]{es} a criterion for Hodge loci to be dense. Our work can be seen as refining these results in the Hodge-theoretic setting, taking into account the Hodge-theoretic (a)typicality of the resulting loci. In particular, in light of Remarks \ref{ominrem} and \ref{ominrem2}, our proofs of Theorem \ref{mainqsimple}(i) and Theorem \ref{main}(i) do not require o-minimality, and we prove in Theorem \ref{mainqsimple}(ii) and Theorem \ref{main}(ii) that the components of $\HLM$ that we construct have the expected Mumford-Tate groups, whereas a conclusion of this type does not appear in the main theorem of \cite{es}.
\subsection{Acknowledgements} We thank Emmanuel Ullmo and Gregorio Baldi for helpful feedback on a draft of this manuscript, and for many useful conversations during the completion of this work. The second author would also like to thank Thomas Scanlon for explaining to him some aspects of his manuscript \cite{es}.
\section{Recollections}
We now review notions from variational Hodge theory; we refer to \cite{voi} and \cite{ggk} for a detailed account of the theory. The first two subsections recall the basic notions, and the subsequent two subsections will review more recent results which we will need for our proofs.
\begin{notation}
For an algebraic group $\G$ over $\Q$, we denote by $\G^\ad$ its adjoint group and $\G^\der$ its derived subgroup. $\G(\R)^+$ will denote the preimage by $\G \rightarrow \G^\ad$ of $\G^\ad(\R)^+$ (where the superscript $^+$ denotes the identity component for the real analytic topology). Finally, $\G(\Q)^+$ is defined to be $\G(\R)^+ \cap \G(\Q)$.
\end{notation}
\subsection{Hodge data and special subvarieties}
Recall that the Mumford-Tate group of a $\Q$-Hodge structure $x : \s \rightarrow \gl(V_\R)$ on a finite dimensional $\Q$-vector space $V$ (where $\s = \res_{\C/\R} \mult_m$ is the Deligne torus) is the smallest $\Q$-subgroup $\MT(x)$ of $\gl(V)$ such that the morphism of $\R$-algebraic groups $x$ factors through $\MT(x)_\R$. Equivalently, it is the fixator in $\gl(V)$ of all Hodge tensors of $x$. It is a connected $\Q$-algebraic group, which is reductive as soon as $x$ is polarizable, which will always be assumed in the sequel. The associated Mumford-Tate domain is the orbit of $x$ under the identity connected component $\MT(x)(\R)^+$ of the real Lie group $\MT(x)(\R)$ in $\Hom(\s, \MT(x)_\R)$. Both notions can be summarized in the following, now usual, definition:
\begin{definition}
~
\begin{itemize}
\item[(i)] A \textit{Hodge datum} is a pair $(\G,D)$ where $\G$ is the Mumford-Tate group of some Hodge structure, and $D$ is the associated Mumford-Tate domain. 
\item[(ii)] A \textit{morphism of Hodge data} $(\G,D) \rightarrow (\G',D')$ is a morphism of $\Q$-groups $\G \rightarrow \G'$ sending $D$ to $D'$. 
\item[(iii)] A \textit{Hodge sub-datum} of $(\G, D)$ is a Hodge datum $(\G', D')$ such that $\G'$ is a $\Q$-subgroup of $\G$ and the inclusion $\G' \hookrightarrow \G$ induces a morphism of Hodge data $(\G',D') \rightarrow (\G,D)$.
\item[(iv)] A \textit{Hodge variety} is a quotient variety of the form $\Gamma \quo D$ for some Hodge datum $(\G,D)$ and some torsion-free arithmetic lattice $\Gamma \subset \G(\Q)^+$.
\end{itemize}
\end{definition}
\begin{remark}
Note that what we call here a \textit{Hodge datum} is usually referred to as a \textit{connected Hodge datum} in the literature. Since we will always be considering connected Hodge data, we ommit the "connected" in the sequel.
\end{remark}
Let $\V$ be a polarizable $\Z$-VHS on a smooth, irreducible and quasi-projective algebraic variety $S$ over $\C$. There exists a reductive $\Q$-algebraic group $\G$ and a countable union $\HL$ of irreducible algebraic subvarieties of $S$ such that for each point $s \in S^\an$, the polarized Hodge structure carried by $\V_s$ has Mumford-Tate group isomorphic (by parallel transport) to a subgroup of $\G$ which is strictly contained in $\G$ if and only if $s \in \HL$ (\cite{cdk}, \cite{bkt}). The group $\G$ is called the \textit{generic Mumford-Tate group of $\V$}, $\HL$ is the \textit{Hodge locus of $S$ for $\V$}, irreducible algebraic subvarieties of $S$ contained in $\HL$ are called \textit{special subvarieties of $S$ for $\V$}, 
and points $s \in S^\an - \HL$ are said to be \textit{Hodge generic}. The orbit in $\Hom(\s, \G_\R)$ of the Hodge structure carried by $\V_s$ under $\G(\R)^+$ doesn't depend on the choice of $s \in S^\an$ and is denoted $D$. We will refer to the Hodge datum $(\G,D)$ as the \textit{generic Hodge datum of $\V$}. Furthermore, the usual Griffiths period map (whose target is a period domain) factors, up to replacing $S^\an$ by a finite �tale cover, as
\[
\Phi : S^\an \rightarrow \Gamma \quo D
\]
where $\Gamma \subset \G(\Q)^+$ is a torsion free arithmetic lattice containing the image of the monodromy representation associated to $\V$ as a finite index subgroup. We will always refer to this factorization as the \textit{period map associated to $\V$} in the sequel.

If $Y$ is an algebraic subvariety of $S$, we can apply the previous paragraph to $\V \vert_Y$ to define the \textit{generic Hodge datum $(\G_Y, D_{G_Y})$ of $Y$ for $\V$}. We see it as a Hodge sub-datum of the generic Hodge datum $(\G,D)$ of $\V$. The following characterisation of special subvarieties is straightforward in view of definitions:
\begin{proposition}\label{caracspec}
An irreducible algebraic subvariety $Y \subset S$ is special for $\V$ if and only if it is maximal for the inclusion among irreducible algebraic subvarieties of $S$ whose generic Mumford-Tate group is $\G_Y$.
\end{proposition}

Special subvarieties of $S$ for $\V$ can also be characterized in term of the generic Hodge datum. A \textit{special subvariety of the Hodge variety $\Gamma \quo D$} is defined as a subvariety of the form
\[
\Gamma' \quo D' \subset \Gamma \quo D 
\]
where $(\G',D')$ is a Hodge sub-datum of $(\G,D)$, and $\Gamma' = \Gamma \cap \G'(\Q)^+$. Then we have that:
\begin{proposition}[{\cite[Lem. 4.7]{bku}}]
An irreducible algebraic subvariety $Y \subset S$ is special for $\V$ if and only if $Y$ is an irreducible component of the preimage by $\Phi$ of a special subvariety of the Hodge variety $\Gamma \quo D$.
\end{proposition}
This characterisation, which realises special subvarieties as preimages by the period map $\Phi$ of intersections between $\Phi(S^\an)$ and some analytic subvarieties of $\Gamma \quo D$, naturally suggests the dichotomy presented in introduction between typical and atypical special subvarieties. More precisely, one defines:
\begin{definition}\label{typdef}
Let $Y$ be a special subvariety of $S$ for $\V$ with generic Hodge datum $(\G_Y, D_{\G_Y})$. It is said to be \textit{atypical} if $\Phi(S^\an)$ and $\Gamma_{\G_Y} \quo D_{G_Y}$ intersect with bigger dimension than expected along $\Phi(Z^\an)$, i.e.
\[
\codim_{\Gamma \quo D} \Phi(Z^\an) < \codim_{\Gamma \quo D}\Phi(S^\an) + \codim_{\Gamma \quo D}\Gamma_{\G_Y} \quo D_{G_Y}
\]
Otherwise, it is said to be \textit{typical}. %Here, the dimension is taken in the sense of definable subsets in the o-minimal structure $\R_{\an,\exp}$
\end{definition}
\begin{remark}
Note the difference with \cite[Def. 2.2]{bku}: here we allow typical intersections to be contained in the singular locus of $\Phi(S^\an)$. This was a technical assumption in their paper, due to the fact that they needed reasoning on tangent spaces. 
\end{remark}

Then, the Hodge locus splits in two parts 
\[
\HL = \HL_\typ \cup \HL_\atyp
\]
where $\HL_\typ$ (resp. $\HL_\atyp$) is defined as the union of the strict typical (resp. atypical) special subvarieties of $S$ for $\V$, and is referred to as the \textit{typical (resp. atypical) Hodge locus}.
\subsection{Monodromy and weakly special subvarieties}
\label{monowspsec}
Let $\V$ be a $\Z$-VHS on a smooth, irreducible and quasi-projective algebraic variety $S$ over $\C$, let $(\G,D)$ be its generic Hodge datum, and fix $s_0 \in S^\an$ a Hodge generic point such that the Mumford-Tate group of the Hodge structure carried by $\V_{s_0}$ is equal (and not only isomorphic) to $\G$. The local system $\V_\Q := \V \otimes_\Z \Q$ corresponds to a representation of the topological fundamental group of $S^\an$ with base-point $s_0$:
\[
\rho : \pi_1(S^\an,s_0) \rightarrow \gl(\V_{\Q,s_0})
\]
We define the \textit{algebraic monodromy group $\mon$ of $\V$} as the identity connected component of the $\Q$-Zariski closure of $\rho(\pi_1(S^\an, s_0))$ in $\gl(\V_{\Q,s_0})$. More generally, if $Y \subset S$ is an irreducible algebraic subvariety, we define the \textit{algebraic monodromy group $\mon_Y$ of $Y$ for $\V$} as the algebraic monodromy group of the restricted $\Z$-VHS $\V \vert_{Y^\sm}$ on the smooth locus $Y^\sm$ of $Y$. Then, recall the following celebrated result of Andr� which follows from the theorem of the fixed part and the semi-simplicity theorem of Deligne \cite{hodge2} (in the geometric setting) and Schmid \cite{schmid} (for general $\Z$-VHS).
\begin{theorem}[{\cite[Thm. 1]{andre}}]\label{andre}
Let $Y \subset S$ be an irreducible algebraic subvariety. Then the monodromy group $\mon_Y$ of $Y$ for $\V$ is a normal subgroup of the derived subgroup $\G_Y^\der$ of the generic Mumford-Tate group of $Y$ for $\V$.
\end{theorem}
In analogy with Proposition \ref{caracspec} one defines:
\begin{definition}
An irreducible algebraic subvariety $Y \subset S$ is said \textit{weakly special for $\V$} if it is maximal for the inclusion among irreducible algebraic subvarieties of $S$ which have algebraic monodromy group $\mon_Y$.
\end{definition}
As in the case of special subvarieties, there is a nice group theoretic description of weakly special subvarieties. A \textit{weakly special subvariety of the Hodge variety $\Gamma \quo D$} is defined to be either a special subvariety, or the image of a subvariety of the form
\[
\Gamma_\M \quo D_M \times \{t\} \subset \Gamma_\M \quo D_M \times \Gamma_\LL \quo D_L
\]
by the morphism of Hodge varieties $\Gamma_\M \quo D_M \times \Gamma_\LL \quo D_L \rightarrow \Gamma \quo D$, where $(\M \times \LL, D_M \times D_L)$ is a Hodge subdatum of the adjoint Hodge datum $(\G^\ad, D)$, and $t \in D_L$ is a Hodge generic point in $\Gamma_\LL \quo D_L$. Now we have the following:
\begin{proposition}[{\cite[Cor. 4.14]{ko}}]
Let $Y \subset S$ be an irreducible algebraic subvariety. $Y$ is a weakly special subvariety of $S$ for $\V$ if and only if it is the preimage by $\Phi$ of a weakly special subvariety of $\Gamma \quo D$.
\end{proposition}
\begin{remark}
Note that by Andr�'s Theorem \ref{andre}, the derived subgroup of $\G$ factors as $\G^\der = \mon \cdot \LL$, and one shows \cite[Lem. 4.12]{ko} that the period map $\Phi$ is constant equal to some Hodge generic $t_L \in \Gamma_\LL \quo D_L$ when projected to the factor corresponding to $\LL$. In the sequel, we will omit this constant factor and simply write $\Phi : S^\an \rightarrow \Gamma_\mon \quo D_H$ for the period map to signify $\Phi : S^\an \rightarrow \Gamma_\mon \quo D_H \times \{t_L\} \subset \Gamma \quo D$.
\end{remark}
\subsection{Algebraicity of period maps} We will make use of the following result on algebraicity of period maps, recently proven by Bakker-Brunebarbe-Tsimerman.
\begin{theorem}[{\cite[Thm. 1.1]{bbt}}] \label{algpermap}
Let $\V$ be a polarizable $\Z$-VHS on a smooth and quasi-projective algebraic variety $S$ over $\C$, and denote by $\Phi : S^\an \rightarrow \Gamma \quo D$ the associated period map. $\Phi$ factors uniquely as $\Phi = \iota \circ f^\an$ where $f : S \rightarrow T$ is a dominant regular map to an algebraic variety $T$, and $\iota : T^\an \hookrightarrow \Gamma \quo D$ is a closed analytic immersion.
\end{theorem}
\subsection{Functional transcendance}
Our arguments heavily rely on a functional transcendence result, Bakker and Tsimerman's Ax-Schanuel theorem, which will allow us to force some intersections to be typical. Recall that the Mumford-Tate domain $D_H$ embeds in its compact dual $\check{D}_H$ which is a projective algebraic variety over $\C$. This embedding allows to define an irreducible algebraic subvariety of $D_H$ as an analytic irreducible component of the intersection $D_H \cap V$ of $D_H$ with an algebraic subvariety $V$ of $\check{D}_H$. Bakker-Tsimerman prove:
\begin{theorem}[{\cite[Thm. 1.1]{baktsi}}] \label{axschan}
Let $W \subset S \times D_H$ be an algebraic subvariety. Let $U$ be an irreducible complex analytic component of $W \cap (S \times_{\Gamma_\mon \quo D_H} D_H)$ such that
\[
\codim_{S \times D_H} U < \codim_{S \times D_H} W + \codim_{S \times D_H} \big(S \times_{\Gamma_\mon \quo D_H} D_H\big).
\]
Then the projection of $U$ to $S$ is contained in a strict weakly special subvariety of $S$ for $\V$.
\end{theorem}
\begin{remark}
\label{ominrem}
The proof of Bakker-Tsimerman's Ax-Schanuel crucially relies on the definability of period maps in the o-minimal structure $\R_{\an,\exp}$. It has been recently reproven in a more general setting using an o-minimal free approach (\cite{diffaxschan}).
\end{remark}
\section{Density of typical special subvarieties}
\subsection{Statement of the result in the general setting}
Let $\V$ be a polarizable $\Z$-VHS on a smooth, irreducible and quasi-projective variety $S$ over $\C$. Let $(\G,D)$ be its generic Hodge datum, $\Gamma \subset \G(\Q)^+$ some torsion-free arithmetic lattice containing the image of the monodromy representation of $\V$ as a finite index subgroup, $\mon$ the algebraic monodromy group and $\Phi : S^\an \rightarrow \Gamma_\mon \quo D_H \subset \Gamma \quo D$ be the associated period map (up to replacing $S$ by an adequate finite �tale cover). The monodromy group $\mon$ decomposes as an almost direct product of its $\Q$-simple factors:
\[
\mon = \mon_1  \cdots \mon_n.
\]
Up to replacing $S$ by a finite �tale cover, the above decomposition induces a factorization of the period map
\[
\Phi : S^\an \rightarrow \Gamma_\mon \quo D_H = \Gamma_1 \quo D_1 \times \cdots \times \Gamma_n \quo D_n,
\]
where for each $i$, we denote by $D_i$ the Mumford-Tate domain associated to $\G_i := \mon_i^\ad$ and $\Gamma_i = \Gamma \cap \G_i(\Q)^+$. For each $I \subset \{1, \cdots, n\}$, we write $D_I := \prod_{i \in I} D_i$ which is a Mumford-Tate domain for the group $\G_I := \prod_{i \in I} \G_i$, and we write
\[ p_I : \Gamma \quo D \rightarrow \Gamma_I \quo D_I := \prod_{i \in I} \Gamma_i \quo D_i\]
for the projection, $\Phi_I := p_I \circ \Phi$ and $\V_I$ the associated $\Z$-VHS on $S$. Finally, we denote by $\pi : D \rightarrow \Gamma \quo D$ the quotient map, and $\pi_I : D \rightarrow \Gamma_I \quo D_I$ the composition $p_I \circ \pi$.
When trying to give a sufficient condition generalizing that of Theorem \ref{mainqsimple} to the not necessarily $\Q$-simple monodromy case, Example \ref{factorwisexample} suggests that one should at least impose conditions like $\V$-admissibility on each factor. Our main theorem states that we don't need more.
\begin{definition}\label{nf}
Let $(\M, D_M) \subsetneq (\G, D)$ be a strict Hodge sub-datum. It is said:
\begin{itemize}
\item[(i)] \textit{factorwise $\V$-admissible} if for every non-empty set of indexes $I \subseteq \{1, \cdots, n\}$ the inequality
\[
\dim \pi_I(D_M) + \dim \Phi_I(S^\an) - \dim D_I \geq 0
\]
holds.
\item[(ii)] \textit{factorwise strongly $\V$-admissible} if for every non-empty set of indexes $I \subseteq \{1, \cdots, n\}$, the strict inequality
\[
\dim \pi_I(D_M) + \dim \Phi_I(S^\an) - \dim D_I > 0
\]
holds.
\end{itemize}
\end{definition}
As explained in Remark \ref{constfactorrem}, on also should add some condition on the constant factor:
\begin{definition}
A strict Hodge sub-datum $(\M,D_M) \subsetneq (\G,D)$ is said to be \textit{full on the $\LL$-factor} if, writing $\G^\der = \mon \cdot \LL$ as an almost-direct product, the normal subgoup $\LL$ of $\G^\der$ is contained in $\M$.
\end{definition}
We can now state the main result of this paper:
\begin{theorem}\label{main}
Let $\V$ be a polarizable $\Z$-VHS on a smooth, irreducible and quasi-projective variety $S$ over $\C$, with generic Hodge datum $(\G,D)$. Let $(\M, D_M) \subsetneq (\G,D)$ be a strict Hodge sub-datum.
\begin{itemize}
    \item[(i)] If $(\M,D_M)$ is full on the $\LL$-factor and factorwise $\V$-admissible, $\HLM$ is analytically dense in $S^\an$.
    \item[(ii)] If $\mon = \G^\der$ and $(\M,D_M)$ is factorwise strongly $\V$-admissible, $\HLM_\typ$ is analytically dense in $S^\an$.
\end{itemize}
\end{theorem}
\subsection{Reformulation of factorwise $\V$-admissibility}
To prove this result, we will need to reformulate the factorwise $\V$-admissibility condition in a way which is better suited to our proof, although more complicated to define. We will need the:
\begin{lemma}\label{indepcap}
Let $(\M,D_M) \subsetneq (\G,D)$ be a strict Hodge sub-datum, and $I \subseteq \{1, \cdots, n\}$ be a non-empty set of indexes. Let $g \in \G(\R)^+$ and $t \in D_{I^c} \times D_\LL$ a Hodge generic point such that 
\[ 
\Phi(S^\an) \cap \pi(g \cdot D_M) \cap \pi(D_I \times \{t\}) \neq \emptyset.
\] 
Up to replacing $S$ by some non-empty Zariski open subset, the quantity
\[
d_I(\M, D_M) := \dim\big((g \cdot D_M) \cap (D_{I} \times \{t\})\big) + \dim\big(\Phi(S^\an) \cap \pi(D_{I} \times \{t\})\big) - \dim D_I
\]
only depends on the set of indexes $I$ and not on $g$ and $t$.
\end{lemma}
\begin{remark}\label{zariskiopen}
In our main theorems we are only interested in proving analytic density results. Since $S$ is irreducible, any Zariski open is analytically dense and we can freely replace $S$ by a Zariski open, as in the above lemma, without changing the conclusion that some locus is analytically dense in the base.
\end{remark}
\begin{convention}\label{conv}
Let $(\M,D_M) \subsetneq (\G,D)$ be a strict Hodge sub-datum. By convention, we set $d_\emptyset(\M,D_M) = 0$. 
\end{convention}
We then have the following reformulation, where we write $d_H(\M,D_M)$ for $d_{\{1, \cdots,n\}}(\M,D_M)$:
\begin{lemma}\label{reform}
Let $(\M,D_M)$ be a strict Hodge sub-datum. It is:
\begin{itemize}
    \item[(i)] factorwise $\V$-admissible if and only if for every strict set of indexes $I \subsetneq \{1, \cdots n\}$, the inequality
    \[
    d_H(\M,D_M) \geq d_I(\M,D_M).
    \]
    holds.
    \item[(ii)] factorwise strongly $\V$-admissible if and only if for every strict set of indexes $I \subsetneq \{1, \cdots n\}$, the strict inequality
    \[
    d_H(\M,D_M) > d_I(\M,D_M)
    \]
    holds.
\end{itemize}
\end{lemma}
\subsection{Proofs} We start by proving the two lemmas above:
\begin{proof}[Proof of Lemma \ref{indepcap}]
Take any other choice of $g' \in \G(\R)^+$ and $t'$. Let us first prove that
\[
\dim\big((g \cdot D_M) \cap (D_{I} \times \{t\})\big) = \dim\big((g' \cdot D_M) \cap (D_{I} \times \{t'\})\big).
\]
Take $x \in (g \cdot D_M) \cap (D_{I} \times \{t\})$ and $x' \in (g' \cdot D_M) \cap (D_{I} \times \{t'\})$. Since $g' M g'^{-1}$ acts transitively on $g' \cdot D_M$ (where we set $M = \M(\R)^+$), there is some $m \in g' M g'^{-1}$ sending $(g'g^{-1})x$ to $x'$. We then have
\[
x' \in (mg'g^{-1})((g \cdot D_M) \cap (D_{I} \times \{t\})) = (g' \cdot D_M) \cap (mg'g^{-1})(D_{I} \times \{t\})
\]
and necessarily, $(mg'g^{-1})(D_{I} \times \{t\}) = D_I \times \{t'\}$ because they are translates of each other both containing $x'$. This proves the first wanted equality of dimension, and already that $d_I(\M,D_M)$ doesn't depend on the choice of $g$.

We are left to prove that up to replacing $S$ by some Zariski open subset, we have
\[
\dim\big(\Phi(S^\an) \cap \pi(D_{I} \times \{t\})\big) = \dim\big(\Phi(S^\an) \cap \pi(D_{I} \times \{t'\})\big).
\]
Let $M$ the minimal dimension of $\Phi(S^\an) \cap \pi(D_I \times \{t\})$ for $t$ varying. These intersections lie in a definable family, so that the locus in $\Phi(S^\an)$ where $\Phi(S^\an) \cap \pi(D_I \times \{t\})$ has dimension strictly bigger than $M$ is a (closed analytic) definable subset of $\Phi(S^\an)$. Now, by \ref{algpermap}, $\Phi(S^\an)$ is algebraic, so that the above locus is in fact algebraic. The preimage of the complement $\Omega$ of this locus by $\Phi$ is Zariski open, and replacing $S$ by $\Phi^{-1}(\Omega)$, the wanted equality of dimension holds for any couple $(t,t')$ such that $\Phi(S^\an) \cap \pi(g \cdot D_M) \cap \pi(D_I \times \{t\})$ and $\Phi(S^\an) \cap \pi(g \cdot D_M) \cap \pi(D_I \times \{t'\})$ are non-empty.
\end{proof}
\begin{remark}\label{ominrem2}
One can avoid the use of o-minimality which is implicit in citing Theorem \ref{algpermap} by using \cite[Prop. 4]{sommese} instead (which is all we need since we allow ourselves to replace $S$ by some non-empty Zariski open subset following Remark \ref{zariskiopen}). The algebraicity of the complement of $\Omega$ can also be argued directly using the algebraicity of the Gauss-Manin connection.
\end{remark}
\begin{proof}[Proof of Lemma \ref{reform}]
Let $I \subseteq \{1, \cdots n\}$ be a non-empty set of indexes, $g \in \G(\R)^+$ and $t' \in D_I$ such that 
\[
\pi(g \cdot D_M) \cap \Phi(S^\an) \cap \pi(D_{I^c}\times \{t'\}) \neq \emptyset.
\]
Then both (i) and (ii) follow from the equality
\begin{eqnarray*}
\dim \pi_I(g \cdot D_M) + \dim \phi_I(S^\an) - \dim D_I & = &\dim \pi(g \cdot D_M) - \dim\pi(g \cdot D_M) \cap \pi(D_{I^c} \times \{t'\})\\ 
&& + \dim \Phi(S^\an) - \dim\Phi(S^\an \cap \pi(D_{I^c} \times \{t'\}) \\
&& - \dim D_H + \dim D_{I^c} \\
& = & d_H(\M,D_M) - d_{I^c}(\M,D_M).
\end{eqnarray*}
where we denoted by $I^c$ the complement of $I$.
\end{proof}
We can now turn to the proof of our main results. First note that our general criterion implies the one announced in the introduction for the $\Q$-simple case as follows:
\begin{proof}[Proof of Theorem \ref{mainqsimple}]
Let $(\M,D_M)$ be a strict Hodge sub-datum of $(\G,D)$. It is immediate from the definitions and Convention \ref{conv} that under the $\Q$-simplicity assumption on $\mon$ along with the assumption that $\mon = \G^\der$, factorwise (resp. strong) $\V$-admissibility is equivalent to (resp. strong) $\V$-admissibility. Besides, under the assumption $\mon = \G^\der$, the datum $(\M,D_M)$ is automatically full on the $\LL$-factor. Hence, Theorem \ref{mainqsimple} is simply a reformulation of Theorem \ref{main} in the $\Q$-simple monodromy case.
\end{proof}
\begin{proof}[Proof of Theorem \ref{main}]
First note that by Theorem \ref{algpermap}, the period image $\Phi(S^\an)$ is algebraic so that its singular locus is a Zariski closed subset. Therefore, up to replacing $S$ by some Zariski open subset (which doesn't change the result, see Remark \ref{zariskiopen}) we can assume that $\Phi(S^\an)$ is smooth, which we do in the sequel. In particular, a special subvariety $Z \subset S$ for $\V$ is typical if and only if the equality of dimension
\[
\dim \Phi(Z^\an) = \dim \Phi(S^\an) + \dim D_{G_Z} - \dim D
\]
holds. We will insist on this in the sequel by saying that $\Gamma_{\G_Z} \quo D_{G_Z}$ and $\Phi(S^\an)$ intersect inside $\Gamma \quo D$ with \textit{the} expected dimension along $\Phi(Z^\an)$.

Let $(\M, D_M)$ a strict Hodge sub-datum of $(\G,D)$. Let $s \in S^\an - \HL$ be a Hodge generic point. Fix some $z \in D$ such that $\pi(z) = \Phi(s) =: x$. As $G := \G(\R)^+$ acts transitively on $D$, there exists an element $g \in G$ such that $z \in g \cdot D_M$. Let $U$ be an irreducible complex analytic component of the intersection $\pi(g \cdot D_M) \cap \Phi(S^\an)$ containing $x$. Assume for now that $U$ has the expected dimension $d = \dim D_M + \dim \Phi(S^\an) - \dim D$ for a component of the intersection of $\pi(g \cdot D_M)$ and $\Phi(S^\an)$ inside $\Gamma \quo D$. This condition is open on $g \in G := \G(\R)^+$, and by \cite[Theorem 18.2]{lagbor}, $\G$ is unirational as an algebraic variety over $\Q$ so $\G(\Q)^+$ is dense in $G$. We conclude that there must exist an open neighbourhood $\mathcal{V}$ of $g$ in $G$ such that for each $g_{0} \in \G(\Q)^+ \cap \mathcal{V}$ the projection $\pi(g_{0} \cdot D_M)$ and $\Phi(S^\an)$ intersect in $\Gamma \quo D$ with the expected dimension $d$ along some complex analytic irreducible component. In this way, we may construct $\G(\Q)^+$-translates of $D_M$ that intersect $\Phi(S^\an)$ with the expected dimension $d$ in the analytic neighborhood of any Hodge generic point of $\Phi(S^\an)$, and this gives the desired property of density of the Hodge locus $\HLM$ of type $\M$. Let us emphasize at this point that concluding the density of the typical Hodge locus of type $\M$ as in (ii) requires more work, as the components we constructed above could in principle all be atypical with respect to their generic Mumford-Tate group.

Let us now prove that under our assumptions, $\pi(g\cdot D_M)$ and $\Phi(S^\an)$ intersect inside $\Gamma \quo D$ with the expected dimension along $U$. Assume for the sake of contradiction that they do not. Because $(\M,D_M)$ is full on the $\LL$-factor, we have that $\dim D - \dim D_M = \dim D_H - \dim (D_M \cap D_H)$. Therefore, $\pi((g \cdot D_M) \cap D_H)$ and $\Phi(S^\an)$ intersect inside the monodromy orbit $\Gamma_\mon \quo D_H$ with bigger dimension than expected along $U$ i.e.
\begin{equation}\label{atyp}
\codim_{\Gamma_\mon \quo D_H} U < \codim_{\Gamma_\mon \quo D_H} \Phi(S^\an) + \codim_{\Gamma_\mon \quo D_H} \pi((g \cdot D_M) \cap D_H).
\end{equation}
Consider the algebraic subvariety $W = S \times ((g \cdot D_M) \cap D_H)$ of $S \times D_H$, let $\tilde{U}$ be the complex analytic component of $W \cap (S \times_{\Gamma_\mon \quo D_H} D_H)$ containing $(s,z)$ such that $\Phi(\pr_S(\tilde{U})) = U$. Then, the condition (\ref{atyp}) gives the inequality required in Theorem \ref{axschan}, which then ensures that the projection $S' := \pr_S(\tilde{U})$ of $\tilde{U}$ to $S$ is contained in a strict weakly special subvariety $Y$ of $S$ for $\V$. We take $Y$ to be minimal for the inclusion among those weakly special subvarieties of $S$ whose image by $\Phi$ contains $U$. Since $Y$ contains $s$ which is Hodge generic, it has generic Mumford-Tate group $\G$, so that, by Andr�'s Theorem \ref{andre}, the algebraic monodromy group $\mon_Y$ of $\V \vert_{Y}$ is a strict normal subgroup of $\mon$. Let us write $\mon_Y = \prod_{i \in I} \mon_i$, so that $Y^\an$ is an analytic irreducible component of the preimage by $\Phi$ of some sub-domain of $\Gamma \quo D$ of the form $\pi(D_I \times \{t\})$. 

(i) Assume that $(\M,D_M)$ is factorwise $\V$-admissible. By construction, $U$ is a component of the intersection of $\pi((g\cdot D_M) \cap (D_I \times \{t\}))$ and $\Phi(Y^\an)$. We claim that these two intersect in $\pi(D_I \times \{t\})$ with the expected dimension along $U$. Indeed, assuming that they don't, we can apply the Ax-Schanuel theorem for $\V \vert_Y$ to this intersection to conclude that $U$ must be contained in the image of some strict weakly special subvariety $Y'$ of $Y$ for $\V\vert_Y$. $Y'$ is therefore the pullback by $\Phi$ of some translate of $\Gamma_{\mon_{Y'}} \quo D_{H_{Y'}}$ that is a weakly special subvariety of $S$ for $\V$ by definition. By construction, $\Phi(Y'^\an)$ contains $U$, and $Y'$ is strictly contained in $Y$. This contradicts the minimality of $Y$, proving our above claim.

We then get the following chain of (in)equalities:
\begin{eqnarray*}
0 & = & \dim U - \dim\big(U \cap \pi(D_{I} \times \{t\})\big) \\ & > & d_H(\M,D_M) - d_I(\M,D_M) \\
& \geqslant & 0,
\end{eqnarray*}
where the first line follows from the fact that $U$ lies in $\pi(D_I \times \{t\})$, the second follows from the assumption that $U$ has bigger dimension than expected and the fact that $U\cap \pi(D_{I} \times \{t\})$ has the expected dimension, and the third one is factorwise $\V$-admissibility. This is a contradiction, so the inequality (\ref{atyp}) must fail and the reasoning presented at the beginning of the proof gives the desired density of $\HLM$.

(ii) Assume that $(\M,D_M)$ is factorwise strongly $\V$-admissible. (i) gives an open neighbourhood $\mathcal{V}$ of $g$ in $G$ such that for each $g_{0} \in \G(\Q)^+ \cap \mathcal{V}$ the projection $\pi(g_{0} \cdot D_M)$ and $\Phi(S^\an)$ intersect along some complex analytic irreducible component with the expected dimension $d$. As explained above, this gives a dense set of components of $\HLM$ with the expected dimension. However these components might, in principle, not belong to $\HLM_\typ$ on account of their Mumford-Tate groups being properly contained in some conjugate of $\M$. 

To rule this out we show that after removing a proper closed subset of $\mathcal{V}$ of smaller definable dimension the components of $\Phi^{-1}(\pi(g_{0} \cdot D_{M}))$ will have the desired property. To construct this closed subset we recall that Mumford-Tate domains in $D$ lie inside finitely many real-analytic definable families $f_{k} : \mathcal{F}_{k} \to G$ for $1 \leq k \leq \ell$, where for each $k$ there is a Mumford-Tate domain $D^{(k)}$ such that for each $g' \in G$, $\mathcal{F}_{k,g'} := f_k^{-1}(g') = g' \cdot D^{(k)}$. We consider only those families for which some fibre of $f_{k}$ is properly contained in $D_M$, and we can assume without loss of generality that for these families we have $D^{(k)} \subsetneq D_M$. For each such $k$, construct the locus
\[ \mathcal{P}_{k} = \{ (g', x) : x \in g' \cdot D^{(k)} \cap \mathcal{I}, \hspace{0.5em} \dim_{x} (g' \cdot D^{(k)} \cap \mathcal{I}) = d \} \subset G \times D \]
where $\mathcal{I} \subset D$ is the intersection of $\pi^{-1}(\Phi(S^\an))$ with some definable fundamental domain for the projection $\pi$. Without loss of generality we may assume that $\mathcal{I}$ is chosen such that $g' \cdot D_{M}$ intersects $\mathcal{I}$ for some $g' \in \mathcal{V}$. 

We claim that the projection $\mathcal{G}_{k}$ of $\mathcal{P}_{k}$ to $G$ intersects $\mathcal{V}$ in a set whose closure has smaller definable dimension. Using definable cell decomposition, it suffices to show that $\mathcal{G}_{k} \cap \mathcal{V}$ does not contain an open neighbourhood $\mathcal{B}$. Supposing it did, we could pick $g' \in \mathcal{B}$ such that some irreducible component $U' \subseteq (g' \cdot D^{(k)}) \cap \mathcal{I}$ of the intersection contains a Hodge generic point and has dimension $d$. Because $D^{(k)}$ has strictly smaller dimension than $D_{M}$ and $U'$ has dimension $d$, $g' \cdot D^{(k)}$ and $\mathcal{I}$ intersect in $D$ with bigger dimension than expected along $U'$. Because we assumed $\mon = \G^\der$, this implies that they intersect in $D_H$ with bigger dimension than expected along $U'$, and the Ax-Schanuel Theorem \ref{axschan} gives a strict weakly special subvariety $Y' \subset S$ such that $\pi(U') \subset \Phi(Y'^\an)$. Choose $Y'$ to be minimal for the inclusion among the weakly special subvarieties of $S$ for $\V$ whose image by $\Phi$ contains $U'$. By Andr�'s theorem \ref{andre}, since $Y'$ contains a Hodge generic point, $\mon_{Y'}$ is a normal subgroup of $\G^\der$, hence of $\mon$. Write $\mon_{Y'} = \prod_{i \in I} \mon_i$. Because $g \in \mathcal{G}_k \cap \mathcal{V}$, $U'$ is also a component of $(g' \cdot D_M) \cap \mathcal{I}$ whose projection to $\Gamma \quo D$ is contained in the image of some strict weakly special of $S$. We will get a contradiction by a similar calculation as in (i), replacing the fact that $U'$ corresponds to an intersection between $\mathcal{I}$ and $g' \cdot D_M$ inside $D$ with bigger dimension than expected (which is not true anymore by construction) by the fact that $(\M,D_M)$ is factorwise strongly $\V$-admissible, and not only factorwise $\V$-admissible. First, applying the Ax-Schanuel theorem for $\V\vert_{Y'}$ to the intersection of $(g' \cdot D^{(k)}) \cap (D_I \times \{t\})$ and $\mathcal{I}$ in $D_I \times \{t\}$ along $U'$, one proves as in (i) that this intersection has expected dimension.

Again, we get the following chain of (in)equalities:
\begin{eqnarray*}
0 & = & \dim U' - \dim\big(U' \cap (D_{I} \times \{t\})\big) \\ & = & d_H(\M,D_M) - d_I(\M,D_M) \\
& > & 0,
\end{eqnarray*}
where the first line follows from the fact that $U'$ lies in $\pi(D_I \times \{t\})$, and the third is factorwise strong $\V$-admissibility. This is a contradiction, and we have eventually proven that $\mathcal{G}_k$ intersects $\mathcal{V}$ in a set whose closure has strictly smaller definable dimension.

Now, by density of $\G(\Q)^+$ in $G$, we can pick some $g_1 \in \G(\Q)^+ \cap (\mathcal{V}-\mathcal{G})$ where $\mathcal{G} = \bigcup_k \mathcal{G}_k$. By construction, pulling back the intersection between $\pi(g_1 \cdot D_M)$ and $\Phi(S^\an)$ to $S$ gives rise to components of the typical Hodge locus of type $\M$. Since we can take $g_1$ arbitrarily close to the $g$ we fixed at the beginning of the proof, we therefore can construct components of $\HLM_\typ$ intersecting any neighbourhood of any Hodge generic point of $S$. This finishes the proof.
\end{proof}

\printbibliography
\end{document}