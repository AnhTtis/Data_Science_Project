\documentclass[12pt]{article}
\usepackage[UKenglish]{babel}
\usepackage[T1]{fontenc}
\usepackage{lmodern,amsmath,amsthm,amsfonts,amssymb,graphicx,float,microtype,thmtools,underscore,mathtools,thm-restate}
\usepackage[shortlabels]{enumitem}
\setlist[itemize]{topsep=0ex,itemsep=0ex,parsep=0ex}
\setlist[enumerate]{topsep=0ex,itemsep=0ex,parsep=0ex}
\usepackage[usenames,dvipsnames,svgnames,table]{xcolor}
%%%%
\usepackage[unicode=true]{hyperref}
\hypersetup{ 
colorlinks,
linkcolor={blue!60!black},
citecolor={black},
urlcolor={blue!60!black},
pdftitle={The Excluded Tree Minor Theorem Revisited}}
%%%%%%%%%%%%
\usepackage[capitalise, compress, nameinlink, noabbrev]{cleveref}
\crefname{lem}{Lemma}{Lemmas}
\crefname{thm}{Theorem}{Theorems}
\crefname{cor}{Corollary}{Corollaries}
\crefname{enumi}{Item}{Items}
\crefformat{equation}{(#2#1#3)}
\Crefformat{equation}{Equation #2(#1)#3}
\crefformat{enumi}{#2#1#3}
\Crefformat{enumi}{Item (#2#1#3)}
%%%%%%%%%%%%
\newcommand{\defn}[1]{\textcolor{Maroon}{\emph{#1}}}
%%%%%%%%%%%%%%%%%%%%%%%%%%%%%%%%%%%%%%%%%%
\usepackage[longnamesfirst,numbers,sort&compress]{natbib}
\makeatletter
\def\NAT@spacechar{~}
\makeatother
\setlength{\bibsep}{0.4ex plus 0.2ex minus 0.2ex}
%%%%%%%%%%%%%%%%%%
\usepackage[margin=28mm]{geometry}
\renewcommand{\baselinestretch}{1.1}
\setlength{\footnotesep}{\baselinestretch\footnotesep}
\setlength{\parindent}{0cm}
\setlength{\parskip}{1.2ex}
\allowdisplaybreaks
%%%%%%%%%%
\newcommand{\half}{\ensuremath{\protect\tfrac{1}{2}}}
\DeclarePairedDelimiter{\floor}{\lfloor}{\rfloor}
\DeclarePairedDelimiter{\ceil}{\lceil}{\rceil}
\DeclarePairedDelimiter{\abs}{\lvert}{\rvert}
\DeclarePairedDelimiter{\set}{\{}{\}} 
%%%% Commands
% A version of \subseteq with a ~ on top
\newcommand\subsetcong{\mathrel{\text{%
    \setbox0\hbox{$\subseteq$}%
    \rlap{\hbox to \wd0{\hss\hss\hss\raisebox{1.5\height}{$\sim$}\hss}}\box0
}}}

\renewcommand{\epsilon}{\varepsilon}
\renewcommand{\emptyset}{\varnothing}
\renewcommand{\ge}{\geqslant}
\renewcommand{\le}{\leqslant}
\renewcommand{\geq}{\geqslant}
\renewcommand{\leq}{\leqslant}
%%%
\DeclareMathOperator{\scol}{scol}
\DeclareMathOperator{\wcol}{wcol}
\DeclareMathOperator{\dist}{dist}
\DeclareMathOperator{\tw}{tw}
\DeclareMathOperator{\bn}{bn}
\DeclareMathOperator{\pw}{pw}
\DeclareMathOperator{\td}{td}
\DeclareMathOperator{\sep}{sep}
\DeclareMathOperator{\stw}{stw}
\DeclareMathOperator{\ltw}{ltw}
\DeclareMathOperator{\rtw}{rtw}
%%
\newcommand{\RR}{\mathbb{R}}
\newcommand{\JJ}{\mathcal{J}}
\newcommand{\PP}{\mathcal{P}}
\newcommand{\DD}{\mathcal{D}}
\newcommand{\BB}{\mathcal{B}}
\newcommand{\FF}{\mathcal{F}}
\newcommand{\GG}{\mathcal{G}}
\newcommand{\HH}{\mathcal{H}}
\newcommand{\LL}{\mathcal{L}}
\newcommand{\NN}{\mathbb{N}}
\newcommand{\OO}{\mathcal{O}}
\newcommand{\WW}{\mathcal{W}}
\newcommand{\scr}[1]{\mathcal{#1}}
\newcommand{\ds}[1]{\mathbb{#1}}
\newcommand{\undefined}{\mathrm{undefined}}
%%%
\newcommand{\david}[1]{{\color{orange} DW: #1}}
\newcommand{\vida}[1]{{\color{DarkGreen} V: #1}}
\newcommand{\robert}[1]{\textcolor{red}{RH: #1}}
\newcommand{\piotr}[1]{\textcolor{brown}{PM: #1}}
\newcommand{\pat}[1]{\textcolor{DarkGrey}{PaMo: #1}}
\newcommand{\gwen}[1]{\textcolor{cyan}{GJ: #1}}
%%%
%\newcommand{\torso}[2]{#1\langle #2\rangle}
%%%%
\renewcommand{\thefootnote}{\fnsymbol{footnote}}
%%%
\theoremstyle{plain}
\newtheorem{thm}{Theorem}
\newtheorem{lem}[thm]{Lemma}
\newtheorem{cor}[thm]{Corollary}
\newtheorem{ques}[thm]{Question}
\newtheorem{prop}[thm]{Proposition}
\newtheorem{obs}[thm]{Observation}
\newtheorem*{claim}{Claim}
\crefname{obs}{Observation}{Observations}
\newtheorem*{lem*}{Lemma}
\theoremstyle{definition}
\newtheorem{conj}[thm]{Conjecture}
\newtheorem*{conj*}{Conjecture}
%%%%%%%%%%%%%%%%%%%%%%%%%%%%%%%%%%%%%%%%%%%%%%

\begin{document}
\title{\bf\boldmath
\fontsize{18pt}{18pt}\selectfont
The Excluded Tree Minor Theorem Revisited}

\author{
\fontsize{16pt}{16pt}\selectfont
Vida Dujmovi{\'c}\,\footnotemark[6]\qquad
Robert~Hickingbotham\,\footnotemark[2] \qquad
Gwena\"el Joret\,\footnotemark[4] \\
Piotr Micek\,\footnotemark[5] \qquad
Pat Morin\,\footnotemark[3] \qquad
David~R.~Wood\,\footnotemark[2]
}

\maketitle

\begin{abstract}

%We prove that for every planar graph $X$ of tree-depth $h$, every $X$-minor-free graph is contained in $H\boxtimes K_{f(X)}$ for some graph $H$ with tree-depth at most $3h+??$. This is a qualitative strengthening of the Grid Minor Theorem of Robertson and Seymour (GM III). 

We prove that for every tree $T$ of radius $h$, there is an integer $c$ such that every $T$-minor-free graph is contained in $H\boxtimes K_c$ for some graph $H$ with pathwidth at most $2h-1$. This is a qualitative strengthening of the Excluded Tree Minor Theorem of Robertson and Seymour (GM I). We show that radius is the right parameter to consider in this setting, and $2h-1$ is the best possible bound. 
\end{abstract}

\footnotetext[2]{School of Mathematics, Monash University, Melbourne, Australia (\texttt{\{robert.hickingbotham, david.wood\}@monash.edu}). Research of Wood supported by the Australian Research Council. Research of Hickingbotham supported by Australian Government Research Training Program Scholarships.}

\footnotetext[6]{School of Computer Science and Electrical Engineering, University of Ottawa, Ottawa, Canada (\texttt{vida.dujmovic@uottawa.ca}). Research supported by NSERC, and a Gordon Preston Fellowship from the School of Mathematics at Monash University.}

\footnotetext[4]{D\'epartement d'Informatique, Universit\'e libre de Bruxelles, Belgium (\texttt{gwenael.joret@ulb.be}). Supported by the Australian Research Council, and by a CDR grant and a PDR from the National Fund for Scientific Research (FNRS).}

\footnotetext[3]{School of Computer Science, Carleton University, Ottawa, Canada (\texttt{morin@scs.carleton.ca}). Research  supported by NSERC.}

\footnotetext[5]{Faculty of Mathematics and Computer Science, Jagiellonian University, Kraków, Poland (\texttt{piotr.micek@uj.edu.pl}). }

\section{\Large Introduction}
\label{Introduction}

%\subsection{Pathwidth}

\citet{RS-I} proved that for every tree $T$ there is an integer $c$ such that every $T$-minor-free graph has pathwidth at most $c$. \citet{BRST91} and \citet{Diestel95} showed the same result with $c=|V(T)|-2$, which is best possible, since the complete graph on $|V(T)|-1$ vertices is $T$-minor-free and has pathwidth $|V(T)|-2$. Graph product structure theory describes graphs in complicated classes as subgraphs of products of simpler graphs \citep{DJMMUW20,ISW,UTW}. Inspired by this viewpoint, we prove the following result, where $H\boxtimes K_c$ is the graph obtained from $H$ by replacing each vertex of $H$ by a copy of $K_c$ and replacing each edge of $H$ by the join between the corresponding copies of $K_c$.

%\david{should we use `inflation' language?}\robert{What is `inflation'?} \david{I have found papers that refer to $H\boxtimes K_c$ as the $c$-inflation of $H$, sometimes denoted by $H[K_c]$. Given that `blow-up' most often means the blow-up by an independent set, it makes sense to use the word `inflation'.} \piotr{I prefer to stay within our current language.}

\begin{thm}
\label{ExcludedTree}
For every tree $T$ of radius $h$, there exists  $c\in\NN$ such that every $T$-minor-free graph $G$ is contained in $H\boxtimes K_c$ for some graph $H$ with pathwidth at most $2h-1$. 
\end{thm}

\cref{ExcludedTree} is a qualitative strengthening of the above-mentioned result of \citet{RS-I} since $\pw(G)\leq \pw(H\boxtimes K_c) \leq c(\pw(H)+1)-1 \leq 2ch-1$. Note that the proof of \cref{ExcludedTree} depends on the above-mentioned result of \citet{RS-I}. The point of \cref{ExcludedTree} is that $\pw(H)$ only depends on the radius of $T$, not on $|V(T)|$ which may be much greater than the radius. Moreover, radius is the right parameter of $T$ to consider here, as we now show. 

%Our proof of \cref{ExcludedTree} uses the above-mentioned results \cite{RS-I,BRST91,Diestel95}.

For a tree $T$, let $g(T)$ be the minimum $k\in\NN$ such that for some $c\in\NN$ every $T$-minor-free graph $G$ is contained in $H\boxtimes K_c$ where $\pw(H)\leq k$. \cref{ExcludedTree} shows that if $T$ has radius $h$, then $g(T)\leq 2h-1$. 
Now we show a lower bound. The following lemma by \citet{UTW} is useful, where $T_{h,d}$ is the complete $d$-ary tree of radius $h$.

\begin{lem}[{\protect\citep[v1,~Proposition~56]{UTW}}]
\label{CompleteTreePathwidthH}
For any $h,c\in\NN$, there exists $d\in\NN$ such that for every graph $H$, if $T_{h,d}$ is contained in $H\boxtimes K_c$, then $\pw(H)\geq h$. 
\end{lem}

Let $T$ be any tree with radius $h$. Thus $T$ contains a path on $2h$ vertices, and $T_{h-1,d}$ contains no $T$-minor, as otherwise $T_{h-1,d}$ would contain a path on $2h$ vertices. 
By \cref{CompleteTreePathwidthH}, if $T_{h-1,d}$ is contained in $H\boxtimes K_c$, then $\pw(H)\geq h-1$. Hence 
\begin{equation}
\label{Summary}
h-1 \leq g(T) \leq 2h-1.
\end{equation}
This says that the radius of $T$ is the right parameter to consider in \cref{ExcludedTree}. 

Moreover, both the lower and upper bounds in \cref{Summary} can be achieved, as we now explain. The upper bound in \cref{Summary} is achieved when $T$ is a complete ternary tree, as shown by the  following result.

\begin{restatable}{prop}{TreeLowerBound}
\label{TreeLowerBound}
For all $h,c\in \NN$, there is a $T_{h,3}$-minor-free graph $G$, such that for every graph $H$, if $G$ is contained in $H\boxtimes K_c$, then $H$ has a clique of size $2h$, implying $\pw(H)\geq\tw(H)\geq 2h-1$.
\end{restatable}

The next result improves \cref{ExcludedTree} for an excluded path. It shows that the lower bound in \cref{Summary} is achieved when $T$ is a path, since $P_{2h+1}$ has radius $h$, and a graph has no path on $2h+1$ vertices if and only if it is $P_{2h+1}$-minor-free. 

\begin{prop}
\label{ExcludedPath}
For any $h\in\NN$, every graph $G$ with no path on $2h+1$ vertices is contained in $H\boxtimes K_{4h}$ for some graph $H$ with $\pw(H) \leq h-1$. 
\end{prop}

%\robert{Delete the following paragraph?}
%After presenting background material in \cref{Background}, we prove \cref{ExcludedTree,ExcludedPath}  in \cref{Proofs}, and prove  \cref{TreeLowerBound} in \cref{LowerBounds}.

%\david{Given that radius of $T$ is the right parameter to consider in \cref{ExcludedTree}, I am less convinced that tree-depth is the right parameter to consider for excluded planar minors. Any thoughts?} \robert{I'm still confident that tree-depth is the right parameter here. In particular, this is true for paths (see `Excluding a Subgraph' in \cite{UTW}).} \david{good point}

%%%%%%%%%%%%%%%%%%%%%
\section{\Large Background}
\label{Background}

We consider simple, finite, undirected graphs~$G$ with vertex-set~${V(G)}$ and edge-set~${E(G)}$. See \citep{Diestel5} for graph-theoretic definitions not given here. 
%A \defn{clique} in a graph is a set of pairwise adjacent vertices. %Let~${\NN \coloneqq \{1,2,\dots\}}$ and~${\NN_0 \coloneqq \{0,1,\dots\}}$. 
For $m,n \in \mathbb{Z}$ with $m \leq n$, let $[m,n]:=\{m,m+1,\dots,n\}$ and $[n]:=[1,n]$. 

A graph $H$ is a \defn{minor} of a graph $G$ if $H$ is isomorphic to a graph that can be obtained from a subgraph of $G$ by contracting edges. A graph~$G$ is \defn{$H$-minor-free} if~$H$ is not a minor of~$G$. 
An \defn{$H$-model} in a graph $G$ consists of pairwise-disjoint vertex subsets  $(W_x \subseteq V(G) :x\in V(H))$ (called \defn{branch sets})  such that each subset induces a connected subgraph of $G$, and 
for each edge $xy\in V(H)$ there is an edge in $G$ joining $W_x$ and $W_y$. Clearly $H$ is a minor of $G$ if and only if $G$ contains an $H$-model.

A \defn{tree-decomposition} of a graph $G$ is a collection $(B_x :x\in V(T))$ of subsets of $V(G)$ (called \defn{bags}) indexed by the vertices of a tree $T$, such that (a) for every edge $uv\in E(G)$, some bag $B_x$ contains both $u$ and $v$, and (b) for every vertex $v\in V(G)$, the set $\{x\in V(T):v\in B_x\}$ induces a non-empty (connected) subtree of $T$. 
%The \defn{adhesion} of $(B_x:x\in V(T))$ is $\max\{|B_x\cap B_y| \colon xy\in E(T)\}$. 
The \defn{width} of $(B_x:x\in V(T))$ is $\max\{|B_x| \colon x\in V(T)\}-1$. The \defn{treewidth} of a graph $G$, denoted by \defn{$\tw(G)$}, is the minimum width of a tree-decomposition of $G$. A \defn{path-decomposition} is a tree-decomposition in which the underlying tree is a path, simply denoted by the sequence of bags $(B_1,\dots,B_n)$. The \defn{pathwidth} of a graph $G$, denoted by \defn{$\pw(G)$}, is the minimum width of a path-decomposition of $G$. 

%A tree-decomposition $(B_x:x\in V(T))$ is \defn{rooted} if $T$ is rooted. Then, for each $x\in V(T)$, a clique $C$ in the torso $\torso{G}{B_x}$ is a \defn{child-adhesion clique} if there is a child $y$ of $x$ such that $C\subseteq B_x\cap B_y$. 

%For graphs $F$ and $G$, an \defn{$F$-decomposition} of $G$ is a collection $(B_x :x\in V(F))$ of subsets of $V(G)$ (called \defn{bags}) indexed by the vertices of $F$, such that (a) for every edge $uv\in E(G)$, some bag $B_x$ contains both $u$ and $v$, and (b) for every vertex $v\in V(G)$, the set $\{x\in V(F):v\in B_x\}$ induces a non-empty connected subgraph of $F$. The \defn{adhesion} of $(B_x:x\in V(F))$ is $\max\{|B_x\cap B_y| \colon xy\in E(F)\}$. The \defn{width} of $(B_x:x\in V(F))$ is $\max\{|B_x| \colon x\in V(F)\}-1$. 
%The \defn{torso} of a bag $B_x$ (with respect to $(B_x:x\in V(F))$), denoted by \defn{$\torso{G}{B_x}$}, is the graph obtained from the induced subgraph $G[B_x]$ by adding edges so that $B_x\cap B_y$ is a clique for each edge $xy\in E(F)$. A \defn{tree-decomposition} is a $T$-decomposition for any tree $T$. The \defn{treewidth} of a graph $G$, denoted by \defn{$\tw(G)$}, is the minimum width of a tree-decomposition of $G$. 
%A $T$-decomposition $(B_x:x\in V(T))$ is \defn{rooted} if $T$ is rooted. Then, for each $x\in V(T)$, a clique $C$ in the torso $\torso{G}{B_x}$ is a \defn{child-adhesion clique} if there is a child $y$ of $x$ such that $C\subseteq B_x\cap B_y$. 

The following lemma is folklore (see \citep{ISW} for a proof). 

\begin{lem}
\label{HittingSet}
For every graph $G$, for every tree-decomposition $\DD$ of $G$, for every collection $\FF$ of connected subgraphs of $G$, and for every $\ell\in\NN$, either\textnormal{:}
\begin{enumerate}[\textnormal{(}a\textnormal{)}]
    \item there are $\ell$ vertex-disjoint subgraphs in $\FF$, or
    \item there is a set $S\subseteq V(G)$ consisting of at most $\ell-1$ bags of $\DD$ such that $S \cap V(F) \neq \emptyset$ for all $F \in \FF$.
\end{enumerate}
\end{lem}



%Let $G$ be a graph. We say that two subgraphs $X,Y\subseteq G$ \defn{touch} if $(V(X) \cup N_G(X)) \cap V(Y) \neq \emptyset$. A \defn{bramble}, $\mathcal{B}$, is a set of connected subgraphs in $G$ such that any two elements of $\mathcal{B}$ touch. A set $S \subseteq V(G)$ is a \defn{cover} of $\mathcal{B}$ if $S \cap B \neq \emptyset$ for all $B \in \mathcal{B}$. The \defn{order} of $\mathcal{B}$ is the minimum size of a cover of $\mathcal{B}$. The \defn{bramble number} of a graph $G$, $\bn(G)$, is the maximum order of a bramble in $G$. \citet{ST93} proved the following Treewidth Duality Theorem.

%\begin{thm}[{\protect\citep{ST93}}]\label{TreewidthDuality}For every graph $G$, $\bn(G)=\tw(G)+1$.\end{thm}

%A graph $H$ is a \defn{minor} of a graph $G$ if $H$ is isomorphic to a graph that can be obtained from a subgraph of $G$ by contracting edges. A graph~$G$ is \defn{$H$-minor-free} if~$H$ is not a minor of~$G$. 
%The graph minor structure theorem of \citet{RS-XVI} shows that $K_t$-minor-free graphs has a tree-decomposition where each torso can be constructed using three ingredients: graphs on surfaces, vortices, and apex vertices. To describe this formally, we need the following definitions. 

% The \defn{Euler genus} of a surface with~$h$ handles and~$c$ cross-caps is~${2h+c}$. The \defn{Euler genus} of a graph~$G$ is the minimum integer $g\geq 0$ such that there is an embedding of~$G$ in a surface of Euler genus~$g$; see \cite{MoharThom} for more about graph embeddings in surfaces.

% Let $G_0$ be a graph embedded in a surface $\Sigma$. Let $F$ be a facial cycle of $G_0$ (thought of as a subgraph of $G_0$). An \defn{$F$-vortex} is an $F$-decomposition $(B_x:x\in V(F))$ of a graph $H$ such that $x\in B_x$ for each $x\in V(F)$, and $V(G_0\cap H)=V(F)$. For $g,p,a\geq0$ and $k\geq1$, a graph $G$ is \defn{$(g,p,k,a)$-almost-embeddable} if for some set $A\subseteq V(G)$ with $|A|\leq a$, there are graphs $G_0,G_1,\dots,G_s$ for some $s\in\{0,\dots,p\}$ such that:
% \begin{itemize}
% 	\item $G-A = G_{0} \cup G_{1} \cup \cdots \cup G_s$,
% 	\item $G_{1}, \dots, G_s$ are pairwise vertex-disjoint,
% 	\item $G_{0}$ is embedded in a surface of Euler genus at most $g$,
% 	\item there are $s$ pairwise vertex-disjoint facial cycles $F_1,\dots,F_s$ of $G_0$, and
% 	\item for $i\in\{1,\dots,s\}$, there is an $F_i$-vortex $(B_x:x\in V(F_i))$ of $G_i$ of width at most $k$.
% \end{itemize}
% The vertices in $A$ are called \defn{apex} vertices---they can be adjacent to any vertex in $G$. A graph is \defn{$k$-almost-embeddable} if it is $(k,k,k,k)$-almost-embeddable.

% We use the following version of the graph minor structure theorem, which is implied by a result of \citet[Theorem~4]{DKMW12}.

% \begin{thm}[\citep{DKMW12}]
% \label{GMSTimproved}
% For every integer $t\geq 1$ there exists an integer $k\geq 1$ such that every $K_t$-minor-free graph $G$ has a rooted tree decomposition $(B_x\colon x\in V(T))$ such that for every node $x\in V(T)$, the torso $\torso{G}{B_x}$ is $k$-almost-embeddable and if $A_x$ is the apex-set of $\torso{G}{B_x}$, then for every child-adhesion clique $C$ of $\torso{G}{B_x}$, either $C\setminus A_x$ is contained in a bag of a vortex of $\torso{G}{B_x}$, or $|C\setminus A_x|\leq 3$.
% \end{thm}

The \defn{strong product} of graphs~$A$ and~$B$, denoted by~${A \boxtimes B}$, is the graph with vertex-set~${V(A) \times V(B)}$, where distinct vertices ${(v,x),(w,y) \in V(A) \times V(B)}$ are adjacent if
	${v=w}$ and ${xy \in E(B)}$, or
	${x=y}$ and ${vw \in E(A)}$, or
	${vw \in E(A)}$ and~${xy \in E(B)}$.

Let $G$ be a graph. A \defn{partition} of $G$ is a set $\PP$ of sets of vertices in $G$ such that each vertex of $G$ is in exactly one element of $\PP$. Each element of $\PP$ is called a \defn{part}. The \defn{width} of $\PP$ is the maximum number of vertices in a part. The \defn{quotient} of $\PP$ (with respect to $G$) is the graph, denoted by \defn{$G/\PP$}, with vertex set $\PP$ where distinct parts $A,B\in \PP$ are adjacent in $G/\PP$ if and only if some vertex in $A$ is adjacent in $G$ to some vertex in $B$. An \defn{$H$-partition} of $G$ is a partition $\PP$ of $G$ such that $G/\PP$ is contained in $H$. The following observation connects partitions and products.

\begin{obs}[\citep{DJMMUW20}]
\label{ObsPartitionProduct}
For all graphs $G$ and $H$ and any $p\in\NN$, $G$ is contained in $H\boxtimes K_p$ if and only if $G$ has an $H$-partition with width at most $p$.
\end{obs}

%A \defn{layering} of a graph $G$ is a partition $\PP$ of $G$, whose parts are ordered $\PP=(V_0,V_1,\dots)$ such that for each edge $vw\in E(G)$, if $v\in V_i$ and $w\in V_j$ then $|i-j|\leq 1$. Equivalently, a layering is a $P$-partition for some path $P$. Consider a connected graph $G$. Let $r\in V(G)$ and let $V_i:=\{v\in V(G):\dist_G(v,r)=i\}$ for each $i\geq 0$. Then $(V_0,V_1,\dots)$ is a \defn{\textsc{bfs}-layering} of $G$ rooted at $r$. Let $T$ be a spanning tree of $G$, where for each non-root vertex $v\in V_i$ there is a unique edge $vw$ in $T$ for some $w\in V_{i-1}$. Then $T$ is called a \defn{\textsc{bfs}-spanning tree} of $G$. (These trees are a superset of the trees that can be generated by the breadth-first search algorithm.)\ 

%If $T$ is a tree rooted at a vertex $r$, then a non-empty path $P$ in $T$ is \defn{vertical} if the vertex of $P$ closest to $r$ in $T$ is an end-vertex of $P$. 

% Many recent results show that certain graphs can be described as subgraphs of the strong product of a graph with bounded treewidth and a path \citep{DJMMUW20,DHHW22,DMW,HW21b,DEMWW22,HJMW,UWY22}. For example, \citet{DHHW22} proved the following result (building on the work of \citet{DJMMUW20}).

% \begin{lem}[\citep{DHHW22}]
% \label{GenusPartition}
% Every connected graph $G$ of Euler genus at most $g$ is contained in $H\boxtimes P \boxtimes K_{\max\{2g,3\}}$ for some planar graph $H$ with treewidth 3, and for some path $P$. In particular, for every rooted spanning tree $T$ of $G$, there is a partition $\PP$ of $G$ such that $G/\PP$ is planar with treewidth at most $3$ and each part of $\PP$ is a subset of the union of at most $\max\{2g,3\}$ vertical paths in $T$.
% \end{lem}

%\section{Proof of \cref{DwidthTheorem}}

%%%%%%%%%%%%%%%%%%%%%%%%%%%%%%%%%%%%%%%%%%%
\section{\Large Proofs}
\label{Proofs}

We prove the following quantitative version of \cref{ExcludedTree}.

\begin{thm}
\label{ExcludedTreePrecise}
Let $T$ be a tree with $t$ vertices, radius $h$, and maximum degree $d$. Then every $T$-minor-free graph $G$ is contained in $H\boxtimes K_{(d+h-2)(t-1)}$ for some graph $H$ with pathwidth at most $2h-1$.
\end{thm}

Recall that $T_{h,d}$ is the complete $d$-ary tree of radius $h$. \cref{ObsPartitionProduct} and the next lemma imply \cref{ExcludedTreePrecise}, since  the tree $T$ in \cref{ExcludedTreePrecise} is a subtree of $T_{h,d}$, and every $T$-minor-free graph $G$ satisfies $\tw(G)\leq\pw(G)\leq t-2$ by the result of \citet{BRST91} mentioned in \cref{Introduction}.


\begin{lem}
For any $h,d\in\NN$ with $d+h\geq 3$, for every $T_{h,d}$-minor-free graph $G$, for every tree-decomposition $\DD$ of $G$, and for every vertex $r$ of $G$, the graph $G$ has a partition $\PP$ such that: 
\begin{itemize}
\item each part of $\PP$ is a subset of the union of at most $d+h-2$ bags of $\DD$, 
\item $\{r\}\in\PP$, and
\item $G/\PP$ has a path-decomposition of width at most $2h-1$ in which the first bag contains $\{r\}$. 
\end{itemize}
\end{lem}

\begin{proof} 
We proceed by induction on pairs $(h,|V(G)|)$ in a lexicographic order. 
Fix $h$, $d$, $G$, $\DD$, and $r$ as in the statement. 
We may assume that $G$ is connected. The statement is trivial if $|V(G)|\leq 1$. Now assume that $|V(G)|\geq 2$. 

For the base case, suppose that $h=1$. For $i\geq 0$, let $V_i:=\{v\in V(G):\dist_G(v,r)=i\}$. So $V_0=\{r\}$. If $|V_i|\geq d$ for some $i\geq 1$, then contracting $G[V_0\cup\dots\cup V_{i-1}]$ into a single vertex gives a $T_{1,d}$ minor. So $|V_i|\leq d-1=d+h-2$ for each $i\geq 0$. 
Thus $\PP:=(V_i:i\geq 0)$ is a partition of $G$, and each part of $\PP$ is a subset of the union of at most $d+h-2$ bags of $\DD$. Moreover, the quotient $G/\PP$ is a path, which has a path-decomposition of width 1, in which the first bag  contains $\{r\}$. 

Now assume that $h\geq 2$ and the result holds for $h-1$. 
Let $R$ be the neighbourhood of $r$ in $G$.
Let $\mathcal{F}$ be the set of all connected subgraphs of $G-r$ that contain a vertex from $R$ and contain a $T_{h-1,d+1}$ minor. If there are $d$ pairwise vertex-disjoint subgraphs $S_1,\ldots,S_d$ in $\mathcal{F}$, 
then we claim that $G$ contains a $T_{h,d}$ minor. 
Indeed, for each $i\in [d]$ consider a $T_{h-1,d+1}$-model $(W^i_x: x\in V(T_{h-1,d+1}))$ in $S_i$. Since $S_i$ is connected, we may assume that all vertices of $S_i$ are in the model. 
For each $i\in [d]$, let $y_i$ be a node of $T_{h-1,d+1}$ such that $W^i_{y_i}$ contains a vertex from $R$, and let $Y^i$ be the union of $W^i_x$ for all ancestors $x$ of $y_i$ in $T_{h-1,d+1}$. 
Observe that there is a $T_{h-1,d}$-model in $S_i$ such that the root of $T_{h-1,d}$ is mapped to the set $Y^i$. Therefore $G-r$ contains $d$ pairwise disjoint models of $T_{h-1,d}$ such that each root branch set contains a vertex from $R$. So $G$ contains a model of $T_{h,d}$, as claimed. 


So $\mathcal{F}$ contains no $d$ pairwise vertex-disjoint elements. By \cref{HittingSet}, there is a minimal set $X\subseteq V(G-r)$, such that $X$ is a subset of the union of $d-1\leq d+h-2$ bags of $\DD$, and $G-r-X$ contains no element of $\mathcal{F}$. 

Let $G_1,\dots,G_p$ be the components of $G-r-X$ that contain a vertex from $R$. 
%Let $r_i$ be a neighbour of $r$ in $G_i$. 
%If $G_i$ contains a $T_{h-1,d+1}$ minor, then $G_i\in\mathcal{F}$, which is a contradiction. 
By construction of $X$, the graph $G_i$ contains no $T_{h-1,d+1}$ minor. By induction, 
$G_i$ has a partition $\PP_i$ such that: 
\begin{itemize}
\item each part of $\PP_i$ is a subset of the union of at most $(d+1)+(h-1)-2=d+h-2$ bags of $\DD$, and %\item $\{r_i\}\in\PP$, and
\item $G_i/\PP_i$ has a path-decomposition $\mathcal{B}_i$ of width at most $2h-3$.
%in which the first bag contains $\{r_i\}$. 
\end{itemize}

Let $Z:= V(G-r-X)\setminus V(G_1\cup\dots\cup G_p)$; that is, $Z$ is the set of vertices of all components of $G-r-X$ that have no vertex in $R$.

Consider a vertex $v\in X$. By the minimality of $X$, the graph $G-r-(X\setminus\{v\})$ contains a connected subgraph $Y_v$ that contains $v$ and a vertex $r_v\in R$ (and contains a $T_{h-1,d+1}$ minor). Let $P_v$ be a path from $v$ to $r_v$ in $Y_v$ plus the edge $r_vr$. So $P_v-\{v,r\}$ is contained in some $G_i$, and thus $P_v$ avoids $Z$. So 
$\cup\{P_v:v\in X\}$ is a connected subgraph in $G-Z$. Let $G'$ be obtained from $G$ by contracting $\cup\{P_v:v\in X\}$ into a vertex $r'$, and deleting any remaining vertices not in $Z$. So $V(G')=\{r'\}\cup Z$. Since $G'$ is a minor of $G$, the graph $G'$ is  $T_{h,d}$-minor-free. Let $\DD'$ be the tree-decomposition of $G'$ obtained from $\DD$ by replacing each instance of each vertex in $\cup\{P_v:v\in X\}$ by $r'$ then removing the other vertices in $V(G)\setminus V(G')$.  
Observe that for every bag $B$ in $\mathcal{D'}$, we have $B-\{r'\}$ contained in some bag of $\DD$. 
By induction, $G'$ has a partition $\PP'$ such that: 
\begin{itemize}
\item each part of $\PP'$ is a subset of the union of at most $d+h-2$ bags of $\DD'$, 
\item $\{r'\}\in\PP'$, and
\item $G'/\PP'$ has a path-decomposition $\BB'$ of width at most $2h-1$ in which the first bag contains $\{r'\}$. 
\end{itemize}

Let $\PP:=\{\{r\}\}\cup\{X\}\cup\PP_1\cup\dots\cup\PP_p \cup (\PP'\setminus\{\{r'\}\})$. Then $\PP$ is a partition of $G$ such that each part is a subset of the union of at most $d+h-2$ bags of $\DD$. 
Let $\mathcal{B}$ be a sequence of subsets of vertices of $G/\PP$ obtained from the concatenation of $\mathcal{B}_1,\dots,\mathcal{B}_p,$ and $\BB'$ by adding $\{r\}$ and $X$ to every bag that comes from $\mathcal{B}_1,\dots,\mathcal{B}_p$ and replacing $\{r'\}$ by $X$. 
Now we argue that $\mathcal{B}$ is a path-decomposition of $G/\PP$. 
Indeed, each part of $\PP$ is contained in consecutive bags of $\BB$, specifically $\{r\}$ and $X$ are added to all bags across $\BB_1,\ldots,\BB_p$, and $X$ is in the first bag of $\BB'$. 
Since $G_1,\ldots,G_p$ are components of $G-r-X$,
the neighbourhood in $G/\PP$ of a part in $\PP_i$ is contained in $\PP_i\cup\{\{r\}, X\}$.
Note also that the neigbourhood of $\{r\}$ in $G/\PP$ is contained in $\PP_1\cup\cdots \cup \PP_p\cup \{X\}$. 
It follows that $\BB$ is a path-decomposition of $G/\PP$. By construction, the width of $\BB$ is at most $2h-1$ and the first bag contains $\{r\}$, as required. 
\end{proof}



% \begin{conj}
% There is a function $f$ such that for every tree $T$ of tree-depth $h$, every $T$-minor-free graph is contained in $H\boxtimes K_{f(T)}$ for some graph $H$ with pathwidth at most $O(h)$.
% \end{conj}

% The first case to consider here is when $T=P_k$ (the $k$-vertex path), which has tree-depth $\Theta(\log k)$. 

% \begin{conj}
% \label{ExcludedPathTreeDepth}
% There is a function $f$ such that every $P_k$-free graph is contained in $H\boxtimes K_{f(k)}$ for some graph $H$ with pathwidth at most $O(\log k)$.
% \end{conj}

%Here is a rough approach: show that the block-cut tree of $G$ has small pathwidth (or perhaps consider the SPQR-tree), then apply some of the tricks in \citep{MRW17} about cycles in 3-connected graphs (with circumference $k$) to reduce to a graph $G$ containing no cycle on $k/2$ vertices. One issue is that we actually want to reduce to a graph $G$ containing no path on $k/2$ vertices (so that we can apply induction). Does every 2- or 3-connected $P_k$-free graph $G$ have a path $Q$ such that $G-Q$ is $P_{k/2}$-free (or something like that)? \citet{BJMMSS} may be relevant too. 

%I'm beginning to wonder if \cref{ExcludedPathTreeDepth} is false. What if $G$ is a complete $n$-ary tree of height $k$.

% \begin{lem}
% Let $T_{n,k}$ be the complete $n$-ary tree of height $k$.
% Assume that $T_{n,k}$ has an $H$-partition of width $c$, where $H$ is a tree and $n\gg c,k$. Then $H$ contains a complete ternary tree of height $k$ rooted at the bag containing the root of $T_{n,k}$ (and $\pw(H)\geq k$).
% \end{lem}

% \begin{proof}
% Let $(B_x:x\in V(H))$ be an $H$-partition of $T_{n,k}$ of width $c$, Let $r$ be the root of $T_{n,k}$. 
% Say $r\in B_x$. 
% Let $A$ be the set of $n$ children of $r$ in $T_{n,k}$. 
% For each $v\in A$, let $T_v$ be the copy of $T_{n,k-1}$ rooted at $v$. 
% Since at most $c$ such copies intersect $B_x$, 
% at least $n-c$ copies of $T_{n,k-1}$ avoid $B_x$. 
% Let $v_1$ be a child of $r$ such that $T_{v_1}$ avoids $B_x$. 
% Say $v_1\in B_{y_1}$. So $x\neq y_1$. 
% At least $n-2c$ copies of $T_{n,k-1}$ avoid $B_x\cup B_{y_1}$. 
% Let $v_2$ be a child of $r$ such that $T_{v_2}$ avoids $B_x\cup B_{y_1}$. 
% Say $v_2\in B_{y_2}$. So $x, y_1,y_2$ are distinct. 
% At least $n-3c$ copies of $T_{n,k-1}$ avoid $B_x\cup B_{y_1} \cup B_{y_2}$. 
% Let $v_3$ be a child of $r$ such that $T_{v_3}$ avoids $B_x\cup B_{y_1\cup B_{y_2}}$. 
% Say $v_3\in B_{y_3}$. So $x, y_1,y_2,y_3$ are distinct. 
% Consider the three copies of $T_{n,k-1}$ rooted at $v_1$, $v_2$ and $v_3$. 
% Since each such copy is connected and disjoint from $B_x$, no bag of the $H$-partition contains vertices in two of the copies. That is, the three copies map to pairwise disjoint subtrees $H_1,H_2,H_3$ of $H$ rooted at $y_1,y_2,y_3$. By induction, each $H_i$ contains a complete ternary tree of height $k-1$ rooted at $y_i$. By construction, $H$ contains a complete ternary tree of height $k$ rooted at $x$. 
% \end{proof}

% The above proof generalises to force a $T_{n/c,k}$ in $H$ (or thereabouts).

% What if $H$ is not a tree? 

% \begin{proof}[Proof of \cref{CompleteTreePathwidthH}]
% We proceed by induction on  $k$. Consider the base case $k=1$. Consider any $H$-partition of $T_{n,k}$ of width $c$. Since $T_{n,k}$ has $n+1$ vertices and is connected, taking $n\geq c$, we have $|E(H)|\geq 1$, implying $\pw(H)\geq 1$. 

% Now assume that $k\geq 2$ and the result holds for $k-1$. 
% Let $r$ be the root of $T_{n,k}$. 
% For each child $v$ of $r$ in $T_{n,k}$, 
% let $T_v$ be the copy of $T_{n,k-1}$ rooted at $v$. 
% Let $(B_x:x\in V(H))$ be an $H$-partition of $T_{n,k}$ of width $c$. 
% Assume for the sake of contradiction that $H$ has a path-decomposition $(D_1,\dots,D_m)$ of width at most $k-1$, so $|D_i|\leq k$ for each $i$. 
% Say $r\in B_x$.
% Let $D_i,\dots,D_j$ be the bags containing $x$. 
% For each child $v$ of $r$, let $y_v$ be the vertex of $H$ such that $v\in B_{y_v}$; so $x=y_v$ or $xy_v\in E(H)$. 
% Thus $y_v$ is in $D_i\cup \dots\cup D_j$. 
% At most $c(2k+1)$ copies of $T_{n,k-1}$ (rooted at the children of $r$) intersect 
% $B_x\cup (\bigcup_{y\in D_i}B_y) \cup (\bigcup_{y\in D_j}B_y)$. Let $T'$ be obtained from $T_{n,k}$ by deleting these at most $c(2k+1)$ copies of $T_{n,k-1}$. 
% Assume $n-c(2k+1)\geq 1$. 
% Thus, for some child $v$ of $r$ in $T'$, the copy $T_v$ avoids 
% $B_x\cup (\bigcup_{y\in D_i}B_y) \cup (\bigcup_{y\in D_j}B_y)$. So $y_v \in D_{i+1}\cup\dots\cup D_{j-1}$. 
% Let $H'$ be the subgraph of $H$ induced by those vertices $z\in V(H)$ such that $B_z$ contains a vertex in $T_v$. 
% By construction, $T_v$ has an $H'$-partition of width $c$. Since $T_v$ avoids $B_x$, we have $x\not\in V(H')$. 
% Now, $y_v \in D_{i+1}\cup\dots\cup D_{j-1}$, and $T_v$ is connected and avoids
% $(\bigcup_{y\in D_i}B_y) \cup (\bigcup_{y\in D_j}B_y)$. Thus $D_{i+1}\setminus\{x\},\dots, D_{j-1}\setminus\{x\}$ is a path-decomposition of $H'$ of width at most $k-2$, which contradicts the inductive assumption. 
% \end{proof}

% This result disproves \cref{ExcludedPathTreeDepth} since $T_{n,k}$ contains no $P_{2k+1}$. 

% \robert{\cref{CompleteTreePathwidthH} is in \citep[v1]{UTW}.}

We now turn to the proof of \cref{ExcludedPath}. We in fact prove a stronger result in terms of tree-depth. A forest is \defn{rooted} if each component has a root vertex (which defines the ancestor relation). The \defn{vertex-height} of a rooted forest $F$ is the maximum number of vertices in a root--leaf path in $F$. The \defn{closure} of a rooted forest $F$ is the graph $G$ with $V(G):=V(F)$  with $vw\in E(G)$ if and only if $v$ is an ancestor of $w$ (or vice versa). The \defn{tree-depth} of a graph $G$ is the minimum vertex-height of a rooted forest $F$ such that $G$ is a subgraph of the closure of $F$. It is well-known and easily seen that $\pw(G)\leq\td(G)-1$ for every graph $G$. Thus, the following lemma implies \cref{ExcludedPath} since every $P_{2h+1}$-minor-free graph $G$ has $\tw(G)\leq\pw(G)\leq 2h-1$ by the result of \citet{BRST91} mentioned in \cref{Introduction}.

\begin{lem}
For any $h,k\in\NN$, for every graph $G$ with no path on $2h+1$ vertices, for every tree-decomposition $\DD$ of $G$, the graph $G$ has a partition $\PP$ such that $\td(G/\PP) \leq h$ and each part of $\PP$ is a subset of at most two bags of $\DD$. 
\end{lem}

\begin{proof}
We proceed by induction on $h$. For $h=1$, $G$ is the disjoint union of copies of $K_1$ and $K_2$. Let $\PP$ be the partition of $G$ where the vertex-set of each component of $G$ is a part of $\PP$. Thus $E(G/\PP)=\emptyset$ and $\td(G/\PP)=1$. Each part is a subset of one bag of $\DD$. 

Now assume $h \geq 2$ and the claim holds for $h-1$. We may assume that $G$ is connected. 
%If $G$ is $P_{2h-1}$-free, then we are done by induction. So we may assume that $G$ contains $P_{2h-1}$.
Suppose $G$ contains three vertex-disjoint paths, $P^{(1)}$, $P^{(2)}$ and $P^{(3)}$, each with $2h-1$ vertices. Let $G'$ be the graph obtained by contracting each path  $P^{(i)}$ into a vertex $v_i$. Since $G'$ is connected, there is a $(v_i,v_j)$-path of length at least $2$ in $G'$ for some distinct $i,j\in \{1,2,3\}$. Without loss of generality, $i=1$ and $j=2$. So there exist vertices $u\in V(P^{(1)})$ and $v\in V(P^{(2)})$ together with a $(u,v)$-path $Q$ of length at least $2$ in $G$ that internally avoids $P^{(1)} \cup P^{(2)}$. Let $x$ be the endpoint of $P^{(1)}$ that is furthest from $u$ (on $P^{(1)}$) and let $y$ be the endpoint of $P^{(2)}$ that is furthest from $v$ (on $P^{(2)}$). Then $(xP^{(1)}uQvP^{(2)}y)$ is a path with at least $2h+1$ vertices, a contradiction.

Now assume that $G$ contains no three vertex-disjoint paths with $2h-1$ vertices. By \cref{HittingSet}, there is a set $S\subseteq V(G)$ consisting of at most two bags of $\DD$ such that $G-S$ is $P_{2h-1}$-free. By induction, $G-S$ has a partition $\PP'$ such that $\td((G-S)/\PP') \leq h-1$ and each part of $\PP'$ is a subset of at most two bags of $\DD$. Let $\PP:=\PP'\cup \{S\}$. Then $\PP$ is the desired partition of $G$ since $\td(G/\PP)\leq 
\td( (G-S)/\PP') +1 \leq h$.
\end{proof}


We turn to the proof of Proposition~\ref{TreeLowerBound}. 
It is a strengthening of a similar result by \citet[Lemma~13]{NSSW19}.

\TreeLowerBound*

\begin{proof}
We proceed by induction on $h\geq 1$. First consider the base case $h=1$. Let $G$ be a path on $n= c+1$ vertices. Thus $G$ is $T_{1,3}$-minor-free. Suppose that $G$ is contained in $H\boxtimes K_c$. Since $n>c$ and $G$ is connected, $|E(H)|\geq 1$ and $H$ has a clique of size 2, as desired. 

Now assume $h\geq 2$ and the result holds for $h-1$. Let $t_0:=|V(T_{h-1,3})|$. By induction,  there is a $T_{h-1,3}$-minor-free graph $G_0$, such that for every graph $H$, if $G_0$ is contained in $H\boxtimes K_{c}$, then $H$ has a clique of size $2h-2$. Let $G$ be obtained from a path $P$ of length $c+1$ as follows: for each edge $vw$ of $P$, add $2c$ copies of $G_0$ complete to $\{v,w\}$. 

Suppose for the sake of contradiction that $G$ contains a $T_{h,3}$-model. Let $X$ be the branch set corresponding to the root of $T_{h,3}$. So $G-X$ contains 
three pairwise disjoint subgraphs $Y_1,Y_2,Y_3$, each containing a $T_{h-1,3}$-minor. Each $Y_i$ intersects $P$, otherwise $Y_i$ is contained in some component of $G-P$ which is a copy of $G_0$. By the construction of $G$, each $Y_i$ intersects $P$ in a subpath $P_i$. Without loss of generality, $P_1,P_2,P_3$ appear in this order in $P$. Since each component of $G-P$ is only adjacent to an edge of $P$, no component of $G-P_2$ is adjacent to both $Y_1$ and $Y_3$. In particular, $X$ is not adjacent to both $Y_1$ and $Y_3$, which is a contradiction. Thus $G$ is $T_{h,3}$-minor-free.

Now suppose that $G$ is contained in $H\boxtimes K_c$. Let $\PP$ be the corresponding $H$-partition of $G$. Since $|V(P)|>c$ there is an edge $v_1v_2$ of $P$ with $v_i\in Q_i$ for some distinct parts $Q_1,Q_2\in\PP$. At most $c-1$ of the copies of $G_0$ attached to $v_1v_2$ intersect $Q_1$, and at most $c-1$ of the copies of $G_0$ attached to $v_1v_2$ intersect $Q_2$. Thus some copy of $G_0$ attached to $v_1v_2$ avoids $Q_1\cup Q_2$. Let $H_0$ be the subgraph of $H$ induced by those parts that intersect this copy of $G_0$. So neither $Q_1$ nor $Q_2$ is in $H_0$. By induction, $H_0$ has a clique $C_0$ of size $2(h-1)$. Since $G_0$ is complete to $v_1v_2$, we have that $C_0\cup\{Q_1,Q_2\}$ is a clique of size $2h$ in $H$, as desired. 
\end{proof}

{\fontsize{10pt}{11pt}\selectfont
\bibliographystyle{DavidNatbibStyle}
\bibliography{DavidBibliography}}

\end{document}




\appendix

\section{Proof of \cref{ExcludedTree} (Work in Progress)}

When $v$ is a vertex of in a rooted tree $T$, 
the subtree of $T$ rooted at $v$ is denoted $T_v$. 
A $2$-\emph{blow-up} of a rooted tree $T$ is a rooted tree $S$ such that 
if the root of $T$ has $x$ children $v_1,\ldots,v_x$ 
then the root of $S$ has $2x$ children $v_{1,1},v_{1,2},\ldots,v_{x,1},v_{x,2}$ and 
the subtrees $S_{v_{i,1}}$, $S_{v_{i,2}}$ of $S$ are both isomorphic to the subtree $T_{v_i}$ of $T$, for each $i\in[x]$.

\begin{lem}
There exists a function $f(\cdot,\cdot)$ such that for every rooted tree $T$ of vertex-height at most $h$ on $t$ vertices and every graph $G$ excluding $T$ as a minor, there exists a graph $H$ with $\pw(H)\leq 2h-1$ such that $G\subsetcong H\boxtimes K_{f(h,t)}$.
\end{lem}

\begin{proof}
Let $T$ and $G$ be as in the lemma statement. 
First, we argue that we can assume that $G$ is connected. 
Indeed, if each $C$ component of $G$ 
is isomorphic to a subgraph of $H_{C}\boxtimes K_{f(h,t)}$ 
for some graph $H_C$ with $\pw(H_C)\leq 2h-1$, 
then $G$ is isomorphic to a subgraph of $\left(\bigcup H_C\right)\boxtimes K_{f(h,t)}$, 
where $\bigcup H_C$ is a disjoint union of all $H_C$. 
Since $\pw\left(\bigcup H_C\right)\leq 2h-1$, we conclude the statement of the lemma for $G$.
Therefore, from now on we assume that $G$ is connected.

Since $G$ has no $T$-minor, by Theorem ? \gwen{Add reference to theorem}, 
we get $\pw(G)\leq t-2$. 
Let $(A_1,\ldots,A_{q})$ be a path decomposition of $G$ with 
$|A_i|\leq t-1$.
We prove the lemma by induction on $h$.

\gwen{Base case $h=1$ to be written}

\piotr{$f(0,t)=0$, $H$ is empty, $\pw(H)=-1$ but where is $r_H$?}

We proceed with the inductive case, $h\geq 2$. 
Let $r_T$ be the root of $T$. 
A subtree $T_0$ of $T$ rooted at $r_T$ is \emph{complete} if 
there exists a subset $V_0$ of children of $r_T$ such that 
$T_0$ is induced by $\set{r_T}\cup\bigcup_{v\in V_0} V(T_v)$.
Note that we allow $V_0$ to be empty, 
in which case $T_0$ is a single-vertex tree containing $r_T$. 
Note also that $T$ is a complete subtree of itself.

Let $X$ be a nonempty vertex subset of $G$.  
Let $\partial X$ be the set of vertices in $X$ that have neighbors in $V(G)-X$.
Let $g(\cdot,\cdot)=?$ and $g_0(\cdot,\cdot,\cdot)=?$. \gwen{fill in bounds}
Note that $g_0(h,t,t_0)\leq g(h,t)$ for all $t_0\in[t]$.
We say that $X$ is \emph{good} if the following holds
\begin{enumerate}
\item $G[X]$ is connected;
\item there exists $(H,r_{H},s_{H})$ where $H$ is a graph with a distinguished vertex $r_{H}$, called the \emph{root vertex} of $H$, and $s_{H}$ is either another distinguished vertex of $H$ called the \emph{stash vertex} of $H$, or $s_H$ is left undefined;
\item there exists $\mathcal{Z}=\set{Z_y}_{y\in V(H)}$ an $H$-partition of $G[X]$ such that 
\[
\partial X \subseteq Z_{r_{H}}\cup Z_{s_{H}}\quad\quad\textrm{(let $Z_{s_{H}}$ be empty if $s_{H}$ is undefined);}
\]
\item there exists $\mathcal{B}=(B_1,\ldots,B_m)$ a path decomposition of $H$  such that
\begin{align*}
|B_i| &\leq 2h\quad\textrm{for all $i\in[m]$,}\\
r_H&\in B_m\quad\textrm{and}\quad s_H\in B_m\ \textrm{if $s_H$ is defined}.
\end{align*}
\item Let $C$ be a component of $G-X$. 
We say that $C$ is \emph{active} if there is an edge in $G$ 
between a vertex in $C$ and a vertex in $Z_{r_H}$. 
Otherwise, $C$ is called \emph{non-active}.


There exist a complete subtree $T_0$ of $T$ and 
a model $\mathcal{Y}=\{Y_s\}_{s\in V(T_0)}$ of $T_0$ in $G\left[X\cup\bigcup_{\substack{C\in\mathcal{C}\\\textrm{$C$ non-active}}} V(C)\right]$ such that 
\begin{align*}
Z_{r_{H}}&\subseteq Y_{r_T},\\
|Z_y| &\leq g(h,t)&&\textrm{for each $y\in V(H)$,}\\
|Z_{s_{H}}|&\leq g_0(h,t,|V(T_0)|)&&\textrm{if $s_{H}$ is defined}.
\end{align*}
\end{enumerate}
When $X$ is a good set, we say that 
$((H,r_{H},s_{H}),\mathcal{Z},\mathcal{B},T_0,\mathcal{Y})$
is a \emph{witness} for $X$.

\begin{claim} 
Let $X$ be a good subset of $V(G)$. 
Then either $X=V(G)$, or there exists $X'$ a subset $V(G)$ 
such that $X\subsetneq X'$ and $X'$ is good.
\end{claim}
\begin{proof}[Proof of the Claim]
If $X=|V(G)|$ then there is nothing to prove. 
Therefore, we assume that $G-X$ is non-empty. 
Since $G$ is connected every component $C$ of 
$G-X$ is adjacent to $X$. 
Let $((H,r_{H},s_{H}),\mathcal{Z},\mathcal{B},T_0,\mathcal{Y})$
be a witness for $X$.
We split into two cases depending on whether there is an active component of $G-X$.

\emph{Case 1:} 
There is a component of $G-X$ adjacent to $Z_{r_{H}}$.

Fix such an active component $C$. 
Since $G$ has no $T$-minor we conclude that $T_0$ is missing at least one child tree of $T$. 
Let $v$ be a child of $r_T$ in $T$ that is not in $T_0$. 
A model $\{M_u\}_{u\in V(T_v)}$ of $T_v$ in $C$ is $Z_{r_{H}}$-\emph{rooted} 
if there exists an edge in $G$ 
between a vertex in $Z_{r_{H}}$ and a vertex in $M_{r_{T_v}}$.
Now, we split into two cases depending on whether 
$C$ contains a $Z_{r_{H}}$-rooted model of $T_v$. 

\emph{Case 1.1:} 
$C$ contains a $Z_{r_{H}}$-rooted model of $T_v$. 

Consider the ambient path decomposition $\set{A_i}$ of $G$ and 
let $q'$ be the maximum integer such that 
$C\cap\bigcup_{q'\leq i\leq q} A_i$ contains a $Z_{r_{H}}$-rooted model of $T_v$. 
Fix a model $\{N_u\}_{u\in V(T_v)}$ of $T_v$ in $C\cap\bigcup_{q'\leq i\leq q} A_i$. 

Let $A'$ be a minimal subset of $V(C)\cap A_{q'}$ such that 
$(C\cap\bigcup_{q'\leq i\leq q} A_i) - A'$ contains no $Z_{r_{H}}$-rooted model of $T_v$. 
Note that $A'$ is well-defined as $V(C)\cap A_{q'}$ is a valid choice for $A'$. 
Note also that $A'$ is non-empty.

Let $\mathcal{D}$ be the components of $(C\cap\bigcup_{q'\leq i\leq q} A_i) - A'$. 
We claim that for each $D$ in $\mathcal{D}$ 
either $D$ has no vertex adjacent to $Z_{r_{H}}$ or 
$D$ has no model of the $2$-blow up of $T_v$. 
Let $T''$ be a $2$-blow up of $T_v$. 
Fix $D$ in $\mathcal{D}$. 
Suppose to the contrary that there is a vertex $d\in D$ adjacent with a vertex in $Z_{r_{H}}$ and $D$ contains a model $\{M''_u\}_{u\in V(T'')}$ of $T''$. 
Let $P$ be a shortest path from $d$ to a vertex in $\bigcup M''_u$ in $D$; 
it exists as $D$ is connected. 
Let $d'$ be the end of $P$ that is in $\bigcup M''_u$, say $d'\in M''_{u'}$ 
where $u'\in V(T'')$. 
Let $P'$ be a path in $T$ from $u'$ to the root $r_{T''}$ of $T''$. 
Since $T''$ is a $2$-blow up of $T_v$, 
there is a copy of $T_v$ in $T''$ that is rooted in $r_{T''}$ and 
disjoint from $P'$, 
except the vertex $r_{T''}$. 
We identify the vertices of the copy with the corresponding vertices of $T_v$. 
Consider the family $\{M_u\}$ of subsets indexed by vertices of $T_v$: 
\begin{align*}
M_{r_{T_v}} &= \bigcup_{u\in V(P')} M''_u\cup V(P),\\
M_u &= M''_u&&\textrm{for all $u \in V(T_v)-\{r_{T_v}\}$.}
\end{align*}
Note that $\{M_u\}_{u\in V(T_v)}$ is a model of $T_v$ in $D$. 
Moreover, this model is $Z_{r_H}$-rooted as $d \in M_{r_{T_v}}$. 
Since $D\subseteq (C\cap\bigcup_{q'\leq i\leq q} A_i) - A'$, 
this contradicts the choice of $A'$. 
This way we conclude that
either $D$ has no vertex adjacent to $Z_{r_{H}}$ or 
$D$ has no model of the $2$-blow up of $T_v$, as desired.


Let $\mathcal{D}_1$ consists of $D\in\mathcal{D}$ such that 
$D$ has no model of a $2$-blow up of $T_v$. 
Since the $2$-blow up of $T_v$ has vertex-height at most $h-1$ (and at most $2t$ vertices), 
we can apply inductive hypothesis to $D$ and conclude that 
$D\subsetcong H_D\boxtimes K_{f(h-1,2t)}$ where $H_D$ is a graph with $\pw(H_D)\leq 2h-3$. 
Let $\mathcal{Z}_D=\{Z_{D,y}\}$ be an $H_D$-partition of $D$ 
with $|Z_{D,y}|\leq f(h-1,2t)$ for each $y\in V(H_D)$.
Let $(B_{D,1},\ldots,B_{D,q_D})$ be a path decomposition of $H_D$ 
with $|B_{D,i}|\leq 2h-2$ for all $i\in[q_D]$.

Let $X'=X\cup A'\cup\bigcup_{D\in \mathcal{D}_1} V(D)$. 
Since $A'$ is non-empty, we have $X\subsetneq X'$.
We are going to prove that $X'$ is good.

\piotr{Argue that $G[X']$ is connected.}

Let $H'$ be a graph with vertex-set $V(H')$ defined as
the union of $V(H)$, $\bigcup_{D\in\mathcal{D}_1} V(H_D)$, 
and additionally if $s_{H}$ is undefined, $H'$ contains one more extra vertex $s$. 
Let
\[
r_{H'}=r_{H}\quad \textrm{and}\quad s_{H'}=\begin{cases}s_H&\textrm{if $s_H$ is defined,}\\s&\textrm{otherwise.}\end{cases}
\]
The edge-set of $H'$ is defined as 
\[
E(H')\ =\ E(H)\ \cup\ \bigcup_{D\in\mathcal{D}_1}E(H_D)\ \cup\ \{(r_{H'},d),(s_{H'},d)\mid\ \textrm{for all $D\in\mathcal{D}_1$ and $d\in V(D)$}\}.
\]

Let $\mathcal{Z}'=\{Z'_y\}_{y\in V(H')}$ be a family of subsets of $V(G)$ defined as
\[
Z'_y = \begin{cases}
Z_y&\textrm{for all $y\in V(H)-\set{s_H}$},\\
Z_{D,y}&\textrm{for all $D\in\mathcal{D}_1$ and $y\in V(H_D)$},\\
Z_{s_{H}}\cup A'&\textrm{if $y=s_{H'}$ (let $Z_{s_H}$ be empty if $s_{H}$ is undefined).} 
\end{cases}
\]
We claim that $\mathcal{Z}'$ is an $H'$-partition of $G[X']$.

\piotr{To be continued.}

Let $D_1,\ldots,D_{|\mathcal{D}_1|}$ be elements of $\mathcal{D}_1$ in arbitrary order. 
Let $m'=m+\sum_{i\in[|\mathcal{D}_1|]}$ and let $\mathcal{B'}=\{B'_i\}_{i\in[m']}$ be defined as 
\[
B'_i = \begin{cases}
B_i&\textrm{for all $i\in[m]$,}\\
B_{D_j,k}\cup\{r_{H'},s_{H'}\}&\textrm{when $i=m+\left(\sum_{1\leq j'< j}q_{D_{j'}}\right)+k$.}
\end{cases}
\]

\piotr{To be continued.}

Let $T'_0$ be the complete subtree of $T$ induced by the vertices of $T_0$ and $T_v$. 
Let $\mathcal{Y'}=\{Y'_s\}_{s\in V(T'_0)}$ be a family of subsets of $V(G)$ defined as 
\[
Y'_s = \begin{cases}
Y_s&\textrm{for all $s\in V(T_0)$,}\\
N_s&\textrm{for all $s\in V(T_v)$.}
\end{cases}
\]

\piotr{To be continued.}


Let $\mathcal{D}_2=\mathcal{D}-\mathcal{D}_1$.
As we argued, 
$D$ has no vertex adjacent to $Z_{r_{H}}$ for all $D\in\mathcal{D}_2$.


\end{proof}
This concludes the proof of the claim and of the lemma. 
\end{proof}



\section{Proof of \cref{ExcludedTree} (Old)}

Let $T_{h,d}$ be the complete $d$-ary tree of height $d$. Note that $T_{1,d}$ is the $d$-leaf star. Let $T_{h,d,d'}$ be the rooted tree in which the root $r$ has $d'$ children, and for each child $v$ of $r$, the subtree rooted at $v$ is isomorphic to $T_{h-1,d}$. 

The next lemma implies \cref{UpperBound}, since for any tree $T$ of height $h$ and with $t$ vertices, $T$ is a subtree of $T_{h,d}$ for some $d\leq t$, and every $T$-minor-free graph $G$ satisfies $\tw(G)\leq\pw(G)\leq t-2$ \citep{BRST91,Diestel95}. Define $f(1,d):=d-1$, and for $h\geq 2$ let 
\[f(h,d):=.\]

\begin{lem}
Let $G$ be a $T_{h,d}$-minor-free graph. Let $\DD$ be a tree-decomposition of $G$. Let $r\in V(G)$ and $S\subseteq V(G-r)$ such that $G[S\cup\{r\}]$ is connected, $S$ is a subset of at most $d-d'$ bags in $\DD$, and $G-S$ has no $T_{h,d,d'}$ minor rooted at $r$. Then $G$ has a partition $\PP$ such that $\{r\}\in\PP$,  $S\subseteq S'$ for some $S'\in\PP$, each part of $\PP$ is a subset of $f(h,d,d')$ bags of $\DD$, and $G/\PP$ has a path-decomposition of width at most $2h-1$ in which $\{r\}$ and $S'$ are in the first bag.
\end{lem}

\begin{proof} We proceed by induction on $|V(G)|+h$. 

Case 1. There is a component $X$ of $G-r-S$ for which  $T_{h-1,d+1}$ is a minor of $X$ and some neighbour of $r$ is in $X$: 
..... there is a partition $A,B,C$ of $V(G-r-S)$, such that:
$B$ is a subset of a bag of $\DD$, 
$G[A]$ contains no $T_{h-1,d+1}$ minor,
$G[A\cup B\cup\{r\}]$ contains a $T_{h,d,1}$ minor rooted at $r$,
and $G[A\cup B]$ is connected.

Let $G_1:=G[\{r\}\cup S\cup A]$. So $G_1$ is a $T_{h,d}$-minor-free graph. By induction (applied with the tree-decomposition of $G_1$ induced by $\DD_1$)

Let $\DD$ be a tree-decomposition of $G$. Let $r\in V(G)$ and $S\subseteq V(G-r)$ such that $G[S\cup\{r\}]$ is connected, $S$ is a subset of at most $d-d'$ bags in $\DD$, and $G-S$ has no $T_{h,d,d'}$ minor rooted at $r$ 

Then $G$ has a partition $\PP$ such that $\{r\}\in\PP$,  $S\subseteq S'$ for some $S'\in\PP$, each part of $\PP$ is a subset of $f(h,d,d')$ bags of $\DD$, and $G/\PP$ has a path-decomposition of width at most $2h-1$ in which $\{r\}$ and $S'$ are in the first bag.





.........

Case 2. There is a component $X$ of $G-r-S$ with no neighbour of $r$ in $X$: .........

Case 3. For each component $X$ of $G-r-S$, $T_{h-1,d+1}$ is not a minor of $X$: .........

\end{proof}

\section{Old Intro}

Treewidth, pathwidth and tree-depth are fundamental parameters in graph structure theory. It is known that $\tw(G)\leq\pw(G)\leq\td(G)-1$ for every graph $G$. The Grid Minor Theorem of \citet{RS-III} says that a minor-closed class of graphs has bounded treewidth if and only if some planar graph is not in the class; see \citep{??} for qualitative improvements. 

Graph product structure theorems describe graphs in complicated classes as subgraphs of products of simpler graphs. Inspired by this viewpoint, we prove the following qualitative strengthening of Robertson and Seymour's result.  

\begin{thm}
\label{ExcludedPlanar}
For every planar graph $X$ there is an integer $c$ such that every $X$-minor-free graph $G$ is contained in $H\boxtimes K_c$ for some graph $H$ with treewidth at most $3 \td(X)-??$.
\end{thm}

Here a graph $G$ is \defn{contained} in a graph $G'$ if $G$ is isomorphic to a subgraph of $G'$. And $H\boxtimes K_c$ is the strong product of $H$ and a complete graph $K_c$, which is the \mbox{\defn{$c$-complete-blowup}} of $H$; that is, the graph obtained by replacing each vertex of $H$ by a copy of $K_c$ and replacing each edge of $H$ by a copy of $K_{c,c}$.

The point of \cref{ExcludedPlanar} is that the underlying treewidth only depends on the treedepth of $X$, not the number of vertices. \cref{ExcludedPlanar} implies $G$ has bounded treewidth (for fixed $X$) since $\tw(G)\leq (\tw(H)+1)c-1$. Tree-depth is the right function to consider here since ....... 

In fact, we prove the following stronger result:

\begin{thm}
\label{ExcludedX}
There is a function $f$ such that for any graph $X$, every $X$-minor-free graph $G$ is contained in $H\boxtimes K_{f(X,\tw(G)}$ for some graph $H$ with treewidth at most $3 \td(X)-??$. 
\end{thm}

In the language of \citep{UTW}, \cref{ExcludedX} says that the underlying treewidth of the class of $X$-minor-free graphs is at most $3\td(X)-??$. It is known that the underlying treewidth of the class of $X$-minor-free graphs is at least $\td(X)$. 

In fact, we prove the following stronger result:

\begin{thm}
\label{DwidthTheorem}
For any graph $X$ there is an integer $c$, such that for any $X$-minor-free graph $G$ and any tree-decomposition $\DD$ of $G$, there is a partition $\PP$ of $G$ with $\tw(G/\PP)\leq3\td(X)-??$, where each part of $\PP$ has $\DD$-width at most $f(X)$. 
\end{thm}

\cref{DwidthTheorem} builds on results of \citet{ISW} who showed analogous results when $X$ is a complete graph or a complete bipartite graph. Here we consider arbitrary graphs $X$. Again, the treedepth of $X$ is the right parameter to consider. 

By taking $\DD$ to be a tree-decomposition of $G$ with minimum width, \cref{DwidthTheorem} immediately implies \cref{ExcludedX}. The advantage of \cref{ExcludedX} is that it applies to tree-decompositions that have other properties besides bounded bag size. See 



\section{Apex-Minor-Free Graphs}

A graph $X$ is \defn{apex} if $X-v$ is planar for some vertex $v\in V(X)$. 

\begin{thm}[\citep{DMW17,DJMMUW20}]
The following are equivalent for a graph $X$:
\begin{enumerate}[(a)]
\item the class of $X$-minor-free graphs has bounded layered treewidth;
\item there are integers $\alpha,\beta$ such that every $X$-minor-free graph is contained in $H\boxtimes K_\beta$ for some graph $H$ with $\tw(H)\leq\alpha$;
\item $X$ is apex. 
\end{enumerate}
\end{thm}

Given this theorem, for a given apex graph $X$, it is natural to ask for the minimum $\alpha$ such that for some $\beta$, every $X$-minor-free graph is contained in $H\boxtimes K_\beta$ for some graph $H$ with $\tw(H)\leq\alpha$. The original bound on $\alpha(X)$ in \citep{DJMMUW20} was huge, depending on constants from the Graph Minor Structure Theorem. \citet{ISW} proved the much improved bound, $\alpha(X)\leq\tau(X)$, where $\tau(X)$ is the \defn{vertex-cover number} of $X$ (the size of a smallest set $S\subseteq V(X)$ such that every edge of $X$ has at least one end-vertex in $S$). We improve this result as follows:

%By definition, $G$ is a subgraph of every graph in $\JJ_{\tau(G), \abs{V(G)}-\tau(G)}$. Thus, the next result follows from \cref{KstMinorFreeLayered}.

\begin{thm}\label{Apex}
For every apex graph $X$ there exists $c \in \NN$, such that every $X$-minor-free graph is contained in $H \boxtimes P \boxtimes K_c$, where $P$ is a path and $\tw(H) \leq 3 \td(X)-???$.
\end{thm}

\begin{proof}

\end{proof}

\section{$p$-Centred Colouring}

Our results have applications in $p$-centred colouring, as we now explain. For $p \in \NN$, a vertex colouring of a graph $G$ is \defn{$p$-centred }if for every connected subgraph $X$ of $G$, $X$ receives more than $p$ colours or some vertex in $X$ receives a unique colour. The \defn{$p$-centred chromatic number $\chi_p(G)$} is the minimum number of colours in a $p$-centred colouring of $G$. Centred colourings are important within graph sparsity theory as they characterise graph classes with bounded expansion~\cite{Sparsity}. 
\citet{PS21} proved that for every graph $X$ there exists $c$ such that every $X$-minor-free graph has $p$-centred chromatic number $\OO(p^c)$. However, the known bounds on $c$ are huge (depending on the Graph Minor Structure Theorem). If $X$ is apex, much improved bounds are known: \citet{ISW} showed that for every apex graph $X$ there exists $c \in \NN$ such that  $\chi_p(G)\leq c (p +1)^{\tau(X)+1}$ for every $X$-minor-free graph $G$. We make the following further improvement:

\begin{thm}\label{pCentred}
For every apex graph $X$ there exists $c \in \NN$ such that for every $X$-minor-free graph $G$,
\begin{equation*}
    \chi_p(G)\leq c (p +1)^{3\td(X)+???}.
\end{equation*}
\end{thm}

\begin{proof}
A result of \citet*[Lem.~8]{DFMS21} implies that $\chi_p(H \boxtimes P \boxtimes K_m) \leq m(p + 1) \chi_p(H)$ for every graph $H$. \Citet[Lem.~15]{PS21} proved that every graph of treewidth at most $t$ has $p$-centred chromatic number at most $\tbinom{p + t}{t} \leq (p + 1)^t$. In particular, \cref{Apex} implies.....
\end{proof}


%As an example, since $K^{\ast}_{3,t}$ is apex with $\tau(K^{\ast}_{3,t})\leq 3$, \cref{pcentred} implies there exists $m = m(t)$ such that  $\chi_p(G)\leq m (p + 1)^4$ for every $K^{\ast}_{3,t}$-minor-free graph $G$.

\section{Standard Examples}

%\david{If $T$ is a tree with no degree-2 vertices and radius $h$, does $\td(T)=h+1$?}


\section{Pat: It's back on, maybe}

{\color{DarkGrey}Let's try to combine our previous plan with the incorrect proof.   To begin with, let's consider the vertex height $2$ case so we assume that there is no $U_{d,d}$-minor.  As we did all last week, we have $G$, our input graph, $X\subseteq V(G)$, our processed graph, and a partition $\mathcal{P}_X$ of $X$ such that $H_X:=G[X]/\mathcal{P}_X$ has $\mathcal{D}$-width at most $f(h)$, and we want to grow size of the processed set $X$.  For each component $C$ of $G-X$, we have a pair of parts $R_X,S_X\in\mathcal{P}_X$.

Here is where things are different: $M_X$ is a model of $U_{1,d',d}$ for some $d'<d$.  (This is just a model $U_{d',d}$ with two copies of the "root" vertex.  Also different: The root $R_X$ is partitioned into two parts $R_{X,1}$ and $R_{X,2}$ so that $G[R_{X,i}]$ is connected for each $i\in\{1,2\}$ and there exists an edge $vw$ in $G$ with $v\in R_{X,1}$ and $w\in R_{X,2}$.}

\david{let's talk tomorrow about this}
\pat{Na, it's contains the same mistake I've been repeatedly making}

Here are couple of observations that might turn out to be useful.

\begin{lem}
  Let $M$ be a minimal model of $U_{1^sd^t}$ in $G$ and let $v$ and $w$ be any two vertices of $M$.  Then there exists a model $M'\subseteq M$ of $U_{1^{s-1}(d-1)^t}$ in which $v$ and $w$ are in different branch sets.
\end{lem}


\begin{lem}\label{s_merge_model}
  Let $M_1$ and $M_2$ be two disjoint minimal models of $U_{1^sd^t}$ in $G$ such that $G$ contains $a$ vertex disjoint paths $P_1,\ldots,P_a$ between $V(M_1)$ and $V(M_2)$.  Then there exists a model $M'\subseteq M_1\cup M_2\cup P_1\cup\cdots\cup P_a$ of $U_{1^a21^{s-a-2}d^t}$.
\end{lem}

I was wondering if we could use Lemma~\ref{s_merge_model}, along with the fact that $G$ (and therefore the graph $G_0$ obtained by contracting the models of $U_{1^sd^t}$ or the graph $G_1$ obtained by contracting the branch sets of these models) has treewidth at most $f(h,k)$.

\end{document}