% Recommended, but optional, packages for figures and better typesetting:
\usepackage[utf8]{inputenc} % allow utf-8 input
\usepackage[T1]{fontenc}    % use 8-bit T1 fonts
\usepackage{hyperref}       % hyperlinks
\usepackage{url}            % simple URL typesetting
\usepackage{booktabs}       % professional-quality tables
\usepackage{amsfonts}       % blackboard math symbols
\usepackage{nicefrac}       % compact symbols for 1/2, etc.
\usepackage{xcolor}         % colors

\usepackage{microtype}
\usepackage{graphicx}
\usepackage{subfigure}
\usepackage{amsmath}
\usepackage{amsthm}
\usepackage{mathtools}
\usepackage{tabularx}
\usepackage[numbers,sectionbib]{natbib}
\usepackage[nottoc]{tocbibind}
%\usepackage{comment}
\usepackage[textsize=tiny]{todonotes}
\usepackage[capitalize,noabbrev]{cleveref}
\usepackage{algorithm}
\usepackage{algpseudocode}
\usepackage{tcolorbox}
\usepackage{bm}
\usepackage[shortlabels]{enumitem}
\usepackage{wrapfig}
\usepackage{amsmath}


%%%%%%%%%%%%%%%%%%%%%%
\usepackage{tikz}
\usepackage{pgfplots}
\usepackage{environ}
\usepackage{tikz-cd}
\usetikzlibrary{decorations.pathreplacing}


\pgfplotsset{compat=newest} 


\AtEndEnvironment{example}{\hfill$\blacksquare$}%
\AtEndEnvironment{definition}{\hfill$\blacksquare$}%



%tikz stuff
\pgfkeys{/pgfplots/scale/.style={
  x post scale=#1,
  y post scale=#1,
  z post scale=#1}
}

\pgfplotsset{every tick label/.append style={font=\tiny}}

\usetikzlibrary{arrows,backgrounds,decorations.markings,hobby}
\usepgflibrary{shapes.multipart}
\usetikzlibrary{patterns}
\makeatletter
\newcommand{\pgfplotsdrawaxis}{\pgfplots@draw@axis}
\makeatother
\def\axisdefaultwidth{3cm}
\def\axisdefaultheight{2cm}

\pgfplotsset{only axis on top/.style={axis on top=false, after end axis/.code={
             \pgfplotsset{axis line style=opaque, ticklabel style=opaque, tick style={thick,opaque},
                          grid=none}\pgfplotsdrawaxis}}}
\pgfplotsset{every axis/.style={scale only axis}}
\tikzset{>=latex} % arrows

\tikzstyle{oper}=[rounded corners, draw=black, thick, minimum size = 4mm]
\tikzstyle{input}=[rounded corners, draw=white, thick, minimum size = 4mm]
\tikzstyle{output}=[rounded corners, thick, draw=white, minimum size = 4mm]
\tikzstyle{empty}=[circle, draw=white, minimum size = 4mm]
%%%%%%%%%%%%
\makeatletter
\newsavebox{\measure@tikzpicture}
\NewEnviron{scaletikzpicturetowidth}[1]{%
  \def\tikz@width{#1}%
  \def\tikzscale{1}\begin{lrbox}{\measure@tikzpicture}%
  \BODY
  \end{lrbox}%
  \pgfmathparse{#1/\wd\measure@tikzpicture}%
  \edef\tikzscale{\pgfmathresult}%
  \BODY
}
\makeatother
%%%%%%%%%%%%%%%%%%%%%
% hyperref makes hyperlinks in the resulting PDF.
% If your build breaks (sometimes temporarily if a hyperlink spans a page)
% please comment out the following usepackage line and replace
% \usepackage{icml2021} with \usepackage[nohyperref]{icml2021} above.
\usepackage{hyperref}


% Attempt to make hyperref and algorithmic work together better:
%\rnewcommand{\theHalgorithm}{\arabic{algorithm}}

% Use the following line for the initial blind version submitted for review:
%\usepackage{icml2022}

% If accepted, instead use the following line for the camera-ready submission:
%\usepackage[accepted]{icml2021}

% The \icmltitle you define below is probably too long as a header.
% Therefore, a short form for the running title is supplied here:
%\icmltitlerunning{GroLIP: An SDP Method to Estimate the Lipschitz Constant of Neural Networks}

%%Above taken from ICML directly
\renewcommand{\bibname}{References}
\makeatletter
\renewcommand{\@biblabel}[1]{#1.}
\makeatother

%%%Taken From Complexity style
\newcommand{\Pclass}{\mathsf{P}}
\newcommand{\SAT}{\mathsf{SAT}}
\newcommand{\coNP}{\mathsf{coNP}}
\newcommand{\NP}{\mathsf{NP}}
\newcommand{\p}{\mathsf{P}}
\newcommand{\MAXSNP}{\mathsf{MAX SNP}}
\newcommand{\CNF}{\mathsf{CNF}}
\newcommand{\DNF}{\mathsf{DNF}}

\newcommand{\R}{\mathbb{R}}
\newcommand{\Z}{\mathbb{Z}}
\newcommand{\bx}{\mathbf{x}}
\newcommand{\by}{\mathbf{y}}
\newcommand{\bz}{\mathbf{z}}
\newcommand{\bool}{\{0,1\}}
\newcommand{\cut}{\text{cut}}
\newcommand{\SDP}{\text{SDP}}
\newcommand{\OPT}{\text{OPT}}
\newcommand{\MAXCUT}{\text{Max-Cut}}
\newcommand{\norm}[1]{||#1||}
\newcommand\inprod[1]{\langle #1 \rangle}
\newcommand{\func}{\circ}
\newcommand{\ACTL}{\text{diag}}
\newcommand{\act}{\sigma}
\newcommand{\expt}{\mathbb{E}}
\newcommand{\sign}{\text{sign}}
\newcommand{\wtensor}{\mathcal{W}}
\newcommand{\xtensor}{\mathcal{X}}
\newcommand{\dualv}{\eta}
\newcommand{\dualvec}{\Lambda}
\newcommand{\trace}{\text{tr}}
\newcommand{\lagr}{\mathcal{L}}
\newcommand{\relu}{\text{ReLU}}
\newcommand{\obju}{\zeta}
\newcommand{\eig}{\lambda}
\newcommand{\class}{\mathcal{C}}


\DeclareMathOperator*{\argmax}{arg\,max}
%\renewcommand{\cite}{\citep}

\theoremstyle{plain}
\newtheorem{theorem}{Theorem}[section]
\newtheorem{proposition}[theorem]{Proposition}
\newtheorem{lemma}[theorem]{Lemma}
\newtheorem{corollary}[theorem]{Corollary}
\theoremstyle{definition}
\newtheorem{definition}[theorem]{Definition}
\newtheorem{assumption}[theorem]{Assumption}
\newtheorem{example}[theorem]{Example}
\theoremstyle{remark}
\newtheorem{remark}[theorem]{Remark}
%\theoremstyle{example}

%\newcommand{\zw}[1]{{\color{red}{\noindent\textsf{\textbf{Zi}---#1}}}}

\renewcommand{\paragraph}[1]{\vspace{.05in}\noindent\textbf{{#1}.~~}}

