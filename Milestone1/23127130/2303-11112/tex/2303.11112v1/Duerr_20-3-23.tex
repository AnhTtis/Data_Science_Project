
\documentclass[a4paper,11pt]{article}
\usepackage{titlesec}
\titleformat{\paragraph}[runin]{\normalfont\itshape}{\theparagraph.}{.3em}{}[.]\titlespacing{\paragraph}{0pt}{1ex plus .1ex minus .2ex}{.5em}
\usepackage{amsmath,amssymb,bm}
\usepackage{mathabx}
\usepackage[T1]{fontenc}
\usepackage[utf8]{inputenc}
\usepackage{lmodern}
\usepackage{mathtools}
\usepackage{dsfont}
\pdfoutput=1
\usepackage[english]{babel}
\usepackage[letterpaper, hmargin=1in, top=1in, bottom=1.2in, footskip=0.6in]{geometry}
\titleformat{\paragraph}[runin]{\normalfont\itshape}{\theparagraph.}{.3em}{}[.]\titlespacing{\paragraph}{0pt}{1ex plus .1ex minus .2ex}{.5em}
\usepackage[letterpaper, hmargin=1in, top=1in, bottom=1.2in, footskip=0.6in]{geometry}
\usepackage[utf8]{inputenc}
\usepackage{lmodern}
\usepackage{mathtools}
\usepackage{dsfont}
\pdfoutput=1
\usepackage[T1]{fontenc}
\usepackage[english]{babel}
% This package is required, and is provided along with the template
\usepackage{graphicx} % Allows including images
\usepackage{booktabs} % Allows the use of \toprule, \midrule and \bottomrule in tables
\usepackage{color}
\definecolor{aquamarine}{rgb}{0.5, 1.0, 0.83}
\definecolor{ao(english)}{rgb}{0.0, 0.5, 0.0}
\definecolor{armygreen}{rgb}{0.29, 0.33, 0.13}
\definecolor{awesome}{rgb}{1.0, 0.13, 0.32}
\definecolor{ballblue}{rgb}{0.13, 0.67, 0.8}
\definecolor{bittersweet}{rgb}{1.0, 0.44, 0.37}
\definecolor{blue}{rgb}{0.0, 0.0, 1.0}
\definecolor{brinkpink}{rgb}{0.98, 0.38, 0.5}
\definecolor{ballblue}{rgb}{0.13, 0.67, 0.8}
\definecolor{brightturquoise}{rgb}{0.03, 0.91, 0.87}
\definecolor{blue-green}{rgb}{0.0, 0.87, 0.87}
\definecolor{caribbeangreen}{rgb}{0.0, 0.8, 0.6}
\definecolor{cyan}{rgb}{0.0, 1.0, 1.0}
\definecolor{amber(sae/ece)}{rgb}{1.0, 0.49, 0.0}
\graphicspath{ {images/} }
\usepackage[utf8]{inputenc}
\usepackage{lmodern}

\author{J\"{u}rg Fr\"{o}hlich$^{1}$ \quad Zhou Gang$^{2}$ \quad Alessandro Pizzo$^{3}$ \vspace{0.6cm}\\
\textit{Dedicated to the memory of Detlef D\"urr}}

\title{A Tentative Completion of Quantum Mechanics\footnote{expanded version}}

\begin{document}

\maketitle

\begin{abstract}
We review a proposal of how to complete non-relativistic Quantum Mechanics to a physically 
meaningful, mathematically precise and logically coherent theory. This proposal has been dubbed 
\textit{ETH - Approach} to Quantum Mechanics, ``$E$'' standing for ``Events,'' ``$T$'' for ``Trees,'' 
and ``$H$'' for ``Histories.'' The $ETH$ - Approach supplies the last one of  three pillars Quantum Mechanics 
can be constructed upon in such a way that its foundations are solid and stable. Two of these pillars
are well known. The third one has been proposed quite recently; it implies a general non-linear 
stochastic \textit{law} for the time-evolution of states of \textit{individual} physical systems.

We illustrate the general ideas and results by sketching an application to the quantum theory of \textit{fluorescence} 
of an atom coupled to the radiation field (in a limit where the velocity of light tends to $\infty$).
\end{abstract}

\section{What is missing in text-book quantum mechanics?}\label{Intro}
\textit{``It seems clear that the present quantum mechanics is not in its final form.''} (Paul Adrien Maurice Dirac)

Our main aim in this paper is to make a modest contribution towards removing some of the 
enormous jumble befuddling many people who attempt to work on the foundations of quantum mechanics 
($QM$). The material contained in Sects.~1 and 2 is introductory; it is intended to show that there is a need 
to complete the present (text-book) $QM$ in such a way that it can be used to describe the behavior of 
individual physical systems without invoking any \textit{ad-hoc} mechanisms, such as various ``measurement
postulates.'' In Sects.~3 and 4 we review the main ideas and results underlying the so-called 
$ETH$ - Approach to $QM$, which is a specific proposal of a completion of $QM$. Most results in these
sections have appeared in papers already published. Various applications of this approach to concrete 
phenomena, such as the radioactive decay of nuclei or the fluroescence of atoms coupled to the quantized 
electromagnetic field, have been or will be presented elsewhere; but a short sketch of recent results on 
fluorescence is included in Sect.~5. in Sect.~6 we describe some conclusions.

We sadly miss the clear insights and useful comments our colleague and friend
\textit{Detlef D\"urr} would have contributed to the endeavor pursued in this paper. He thought about 
fundamental problems of quantum mechanics more deeply than most people and over many years 
\cite{Durr}.\footnote{We suspect, though, that our views of how to complete $QM$ are likely to 
differ somewhat from what we think were his.}

\subsection{Text-book quantum mechanics}
We start our review by explaining some of the shortcomings of text-book $QM$ and the \textit{``Copenhagen
interpretation''}, with the purpose to highlight the need for a \textit{completion} of the theory -- as \textit{Dirac} 
had anticipated. Text-book $QM$ is a theory -- alas, \textit{incomplete} -- of (\textit{ensemble averages} 
over many identical) physical systems and of the time evolution of ensemble-averaged states
based on the following two pillars:
\begin{enumerate}
\item[(i)]{A physical system, $S$, is characterized by a list
$$\mathcal{O}_{S}= \big\{\widehat{X}_{\iota}= \widehat{X}^{*}_{\iota} \big| \iota \in \mathfrak{I}_{S}\big\}$$
of abstract bounded self-adjoint operators, where $\mathfrak{I}_{S}$ is a (continuous) index set. Every
operator $\widehat{X}\in \mathcal{O}_{S}$ represents a (bounded function of a) \textit{physical quantity} 
characteristic of $S$, such as the electromagnetic field in a bounded region of space-time, or the 
total momentum, energy or spin of all particles (e.g., atoms) in $S$ localized in some bounded 
domain of physical space and interacting with the electromagnetic field. Of course, different 
operators in $\mathcal{O}_S$ do in general \textit{not} commute with one another. One assumes that if 
$\widehat{X}\in \mathcal{O}_{S}$ and $F$ is a real-valued, bounded continuous function on $\mathbb{R}$ then 
$F(\widehat{X})\in \mathcal{O}_{S}$, too. In general $\mathcal{O}_{S}$ does not have 
any additional structure (it is usually not a real linear space, let alone an algebra).

At every time $t$, there is a representation of $\mathcal{O}_{S}$ by bounded self-adjoint operators
acting on a separable Hilbert space $\mathcal{H}$
\begin{equation}\label{1}
\mathcal{O}_{S} \ni \widehat{X} \mapsto X(t)=X(t)^{*} \in B(\mathcal{H})\,,
\end{equation}
where $B(\mathcal{H})$ is the algebra of all bounded operators on $\mathcal{H}$.

\textit{\underline{Heisenberg picture time evolution}:} If $S$ is an \textbf{isolated} system, 
i.e., one whose interactions with the rest of the Universe are negligibly weak, then the operators 
$X(t)$ and $X(t')$ representing a physical quantity $\widehat{X}\in \mathcal{O}_{S}$ 
at two times, $t$ and $t'$, are unitarily conjugated to one another. 
In an \textit{autonomous} system,
\begin{equation}\label{2}
X(t')= e^{i(t'-t)H_S/\hbar} \,X(t)\,,e^{-i(t'-t)H_S/\hbar}\,,
\end{equation}
where $H_S$ is the (time-independent) Hamiltonian of $S$. For simplicity, we will henceforth 
assume that $S$ is autonomous.}
\item[(ii)]{``States,'' $\omega$, of $S$ are assumed to be given by density matrices, $\Omega$, i.e., 
by non-negative trace-class operators on $\mathcal{H}$ of trace one. The expectation at time $t$ 
of an operator $\widehat{X}\in \mathcal{O}_{S}$ in the ``state'' $\omega$ of $S$ is given by
$$\omega\big(X(t)\big):=\text{Tr}\big(\Omega\,X(t)\big).$$
The state $\omega$ given by a density matrix $\Omega$ is \textit{pure} iff $\Omega$ is a rank-1 
orthogonal projection $P=P^{*}=P^{2}$; otherwise it is a mixed state.}
\end{enumerate}
In text-book $QM$, it is usually assumed, following \textit{Schr\"odinger,} that, in the \textit{\underline{Heisenberg picture}}, 
``states'' of an isolated physical system $S$ are \textit{independent} of time $t$, and, hence, that the Heisenberg picture 
is equivalent to the \textit{\underline{Schr\"odinger picture}}; namely
$$\omega(X(t))= \text{Tr}\big(\Omega\, X(t)\big)= \text{Tr}\big(\Omega(t)\, X\big), \quad X:=X(t_0),\, \Omega: =\Omega(t_0),$$
where $t_0$ is an (arbitrarily chosen) initial time. In the Schr\"odinger picture, 
the Schr\"odinger (-von Neumann) equation
\begin{equation}\label{S-L}
\dot{\Omega}(t) = -\frac{i}{\hbar}\big[H_S, \Omega(t)\big]\,, \quad t\in \mathbb{R}.
\end{equation}
describes the time evolution of \textit{states} of $S$, while physical quantities of $S$ 
are represented by \textit{time-independent} bounded operators $X$ on $\mathcal{H}$. 

More generally, the time-dependence of states of a system $S$ interacting with some environment 
is described by \textit{linear, deterministic, trace-preserving, completely-positive maps,} 
$\big\{\Gamma(t, t')\big| t\geq t' \big\}$, 
\begin{equation}\label{Kraus}
\Omega(t)= \Gamma(t,t') \big[\Omega(t')\big]\,, \qquad \forall t'\geq t\,,
\end{equation}
where the operators $\Gamma(t,t')$ are defined on the linear space of trace-class operators on $\mathcal{H}$,
and $\Gamma(t,t')= \Gamma(t,t'')\cdot\Gamma(t'',t'), t\geq t'' \geq t',$ with $\Gamma(t,t)= \mathbf{1}$; see
\cite{Kraus, Lindblad}.

\subsection{The shortcomings of text-book quantum mechanics}
In text-book $QM$, the \textit{time evolution of states} in the Schr\"odinger picture 
(see Eqs.~\eqref{S-L}, \eqref{Kraus}) is \textit{linear} and \textit{deterministic}.
Of course, this \textit{cannot be the full story}! As already recognized by \textit{Einstein} in 1916 in his
paper on spontaneous and induced emission and absorbtion of light by atoms, which he described in
probabilistic terms (introducing his $A$- and $B$-coefficients), $QM$ is a \textit{fundamentally probabilistic
theory.} To anticipate an important fact about $QM$, we claim that the \textit{linear deterministic evolution 
equations \eqref{S-L} and \eqref{Kraus} only describe the evolution of \textbf{ensemble averages} 
of very many identical systems;} but that the time evolution of an \textit{\textbf{individual} system} 
is \textit{non-linear} and \textit{stochastic}. 

Thus, the fundamental problem arises to introduce an appropriate notion of \textit{states of \textbf{individual} systems}
and to formulate a general \textit{\textbf{law}} that correctly determines their non-linear stochastic evolution. In
other words, our task is to find the right \textit{``ontology''} underlying $QM$.

According to the \textit{Copenhagen interpretation} of $QM$, the deterministic evolution of the ``state'' of 
an individual system identical to $S$ is \textit{``interrupted''} at all times $t$ when an \textit{``event''} 
happens, such as the emission or absorption of a photon by an atom, or when a measurement 
of the value of a physical quantity $\widehat{X}\in \mathcal{O}_{S}$ is carried out.
In this latter case, the ``state'' of $S$ is claimed to make a \textit{``quantum jump''} to some ``state'' in the range of the 
spectral projection of $X(t)$ corresponding to the value of $\widehat{X}$ measured at time $t$, i.e., corresponding 
to the eigenvalue of $X(t)$ associated with the measured value of $\widehat{X}$. $QM$ is claimed to 
predict the \textit{probabilities} or \textit{frequencies} of ``quantum jumps'' to eigenstates corresponding 
to \textit{different possible values} of $\widehat{X}$ when measurements of the value of $\widehat{X}$ 
are repeated many times for identical, identically prepared systems. These frequencies are supposed 
to be given by the \textit{Born Rule} applied to the state of $S$ at the time when the measurement of 
$\widehat{X}$ begins. -- This is, in words, the contents of \textit{L\"uders' measurement postulate} \cite{Luders}.

If the equipment used to measure the value of $\widehat{X}$ is \textit{included} in what constitutes the \textit{total} 
system $S$ (now assumed to be \textit{isolated}) one might expect -- \textit{erroneously} -- that the event corresponding 
to a measurement of the value of $\widehat{X}$ for an \textit{individual} system could be viewed as the result of the 
\textit{Schr\"odinger-von Neumann evolution of the state of the \textbf{total} system}. This would imply that $QM$ is a
deterministic theory -- \textit{which it obviously isn't}, as already noticed by Einstein; (see
\cite{Wigner, FFS} for more recent observations in this direction). So, what is going on?

In order to clarify what the problems are that we have to address, it may be useful to take a brief look 
at Stern-Gerlach experiments used to measure the vertical component of the spin of a silver atom 
by exploiting what used to be called \textit{``Richtungsquantelung''}. 
\begin{center}
\includegraphics[width=8cm]{2-S-G-1}\\
{\small{Figure 1: \textit{Illustration of Schr\"odinger-von Neumann evolution}}}
\end{center}
Figure 1 is supposed to illustrate the \textit{false} prediction derived from the Schr\"odinger-von Neumann equation 
for the ``state'' of the \textit{total} system, \textit{including} the detectors, that the orbital wave function of a 
silver atom always evolves to split into two pieces, one hitting the upper detector and the other one hitting 
the lower detector, which may then both fire or remain mute.

However, in every such experiment only \textbf{one} detector is obserevd to fire, as illustrated in \mbox{Figure 2.}
\begin{center}
\includegraphics[width=8cm]{2-S-G-2}\\
{\small{Figure 2: \textit{Illustration of what people find in experiments}}}
\end{center}
The frequencies for the upper or the lower detector to fire when the same experiment
is repeated very many times are given by the usual Born Rule. -- Well, this means that, apparently, 
the evolution of an \textit{individual} system is described by some \textit{stochastic evolution equation,} 
\textit{not} by a Schr\"odinger equation; the latter only describing the evolution of an \textit{average} 
of states of very many identical, identically prepared systems.

It is of interest to further imagine an experiment where \textit{two} silver atoms are initially localized in a central 
region and prepared in a spin-singlet state and in an orbital wave function with the properties that one atom
propagates to the East, while the other one propagates to the West, both of them passing through essentially
vertical magnetic fields. It is assumed that the vertical components of the spins of both atoms will be 
measured in pairs of detectors located in the East of the central region and in the West of it, respectively. 
(It may be difficult to carry out exactly this experiment in practice; but related ones can surely be carried out.) 
If this experimental set-up, \textit{including the four detectors} in the East and the West, could be described 
completely by Schr\"odinger-von Neumann evolution of the total-system wave function then there would be 
\textbf{no} correlation between the measurement results in the East and those in the West, as proven 
in \cite{FFS} by exploiting cluster properties. However, it is claimed that, at the end of \textit{every} 
successful experiment, one will only see correlated spin states, namely \textit{either} 
$\big| \uparrow \big>_{1} \otimes \big| \downarrow\big>_{2}$ \textit{or} 
$\big| \downarrow \big>_{1} \otimes \big| \uparrow\big>_{2}$. 
This is a manifestation of what people (somewhat unfortunately) call the ``non-locality'' of 
quantum mechanics. It proves that Schr\"odinger-von Neumann evolution does \textit{not} describe 
the evolution of the states of \textit{individual} systems; (for further details see \cite{FFS}).

Another interesting experiment concerns the \textit{fluorescence of} (i.e., spontaneous emission of photons 
from) \textit{a static, two-level atom} coupled to the quantized electromagnetic field, which is put into a suitable 
laser beam that causes it to exhibit Rabi oscillations and, every once in a while, to spontaneously emit a photon. 
The two energy levels of the atom correspond to a ground-state and an excited state. It turns out 
that the spontaneous emission of photons is a \textit{stochastic process}, which is accompanied by a 
``quantum jump'' of the atom to its ground-state, the times of emission of the photons being \textit{random variables}
whose law one would like to determine. One would like to derive the stochastic process describing the fluorescence 
of such an atom from a \textit{completion of} $QM$. A closely related example will be considered and treated 
fully quantum-mechanically in Sect.~5.

\vspace{0.2cm} \textit{\underline{Critique of text-book $QM$}.}
\begin{enumerate}
\item{The notion of a \textit{``measurement''} or \textit{``observation''} appearing in the Copenhagen interpretation of
$QM$ is extremely vague. What is the difference between a period in the evolution of the state of a system \textit{without}
``measurement'' and a period of evolution when a ``measurement'' is carried out?
}
\item{If one takes L\"uders' measurement postulate literally one is tempted to conclude that $QM$
only makes useful predictions if it is known beforehand which measurements are planned by ``observers'' 
to be carried out, as well as what the times of their interventions are. One might then be misled to believe 
that the \textit{free will} of ``observers'' plays a central role in $QM$.}
\item{The hypotheses implicit in the ``Copenhagen interpretation'' that one can freely choose the time
when a measurement begins and that there are measurements that only take an arbitrarily small amount
of time (which would actually imply that there are infinitely large energy fluctuations associated with such 
measurements) strike us as totally absurd.}
\item{There are quantum phenomena, such as the radioactive decays of certain nuclei, as well as the precise 
decay times, or the fluorescence of atoms, that are intrinsically \textit{random} and involve ``quantum jumps.'' 
They are to be described by appropriate \textit{stochastic} processes. But there are no ``observers'' involved 
to trigger them. So, where does the randomness of such phenomena originate from?}
\end{enumerate}

\vspace{0.1cm} \textit{``\underline{Fake cures'' of text-book $QM$}.}
\begin{itemize}
\item{We think it is a mistake to imagine that the problems and paradoxes of text-book $QM$ 
can be cured by some sort of \textit{``interpretation''} of $QM$, such as ``Relational QM,''
``QBism,'' ``Consistent Histories,'' ``Many-Worlds Interpretation'' \cite{Everett}, ``Information ontologies,'' etc.; 
see \cite{Wiki} and references given there.\\
As \textit{David Mermin} put it: \textit{New interpretations appear every year. None ever disappear.}}
\item{We expect it to be equally unlikely that these problems and paradoxes can be eliminated by
supplementing text-book $QM$ with some \textit{``ad-hoc mechanisms,''} such as ones based 
on decoherence \cite{Griffiths, G-M-H}, spontaneous wave-function collapse \cite{GRW} (which may 
remind one of electromagnetic or mechanical mechanisms used to explain \textit{Lorentz contraction} before
the advent of the theory of special relativity), or by attempting to reproduce the predictions of 
quantum mechanics by using cellular automata 
\cite{Hooft}, etc.}
\end{itemize}

\textit{\underline{Remark}:} \textit{Bohmian mechanics is} a logically coherent completion of 
(non-relativistic) quantum mechanics \cite{Durr}. But it reminds one of ``completing'' classical electrodynamics
by introducing a mechanical medium, the ether, thought to be the carrier of electromagnetic waves. 
The Bohmian particles are apparently as ``unobservable'' as the ether, the most likely reason being 
that they are point-particles without any physical properties, such as electric charge or spin. -- 
We do not expect that Dirac would have accepted this theory as a completion of $QM$.

\vspace{0.1cm} In the following we attempt to convince the reader that the fundamental problem 
to solve in order to ``complete'' $QM$ is to find a \textit{universal quantum-mechanical \textbf{law}} 
that determines the non-linear stochastic time evolution of states of \textit{\textbf{individual} systems,} 
with the properties that it correctly describes what is seen in experiments and that it reproduces the 
linear deterministic Schr\"odinger-von Neumann evolution of states when averaged over an ensemble of 
very many identical isolated systems.

\section{An analogy with the theory of diffusion processes}\label{BM}
An \textit{analogous problem} in classical physics that may guide our thought process towards the 
right law is found in the theory of \textit{diffusion} and \textit{Brownian motion.} Consider a system consisting 
of a drop, $\mathfrak{E}$, of ink (e.g., eosin) in water (assumed to be in thermal equilibrium at some
temperature $T>0$). The \textit{``state''} of $\mathfrak{E}$ at time $t$ is 
given by its \textit{density} $\rho_t$, which is a non-negative function on physical space $\mathbb{E}^{3}$. 
We normalize it such that $\int_{\mathbb{E}^{3}} d^{3}x \rho_t(x) = 1$. The time dependence of $\rho_t$ is
governed by the \textit{diffusion equation}, viz. by a \textit{deterministic linear law of evolution.}
\begin{equation}\label{diffusion}
\dot{\rho}_{t}(x)= D\, (\Delta \rho_t)(x), \qquad D: \text{diffusion constant.}
\end{equation}
The well known solution of this equation is given by
\begin{equation*}
\rho_{t}(x)= \int_{\mathbb{E}^3} d^{3}x' \,\Gamma_{t-t'}(x-x') \rho_{t'}(x'), \quad \Gamma_{t}(x) := 
(2\pi Dt)^{-\frac{3}{2}} e^{-\frac{|x|^2}{2Dt}}\,,
\end{equation*}
and the heat kernels $\Gamma_t$ satisfy the \textit{Chapman-Kolmogorov} equation. We will see 
that it is this property that distinguishes this classical model of a physical system from all 
quantum-mechanical models of physical systems.

According to the atomistic view of matter, $\mathfrak{E}$ really consists of very many eosin molecules, 
which, in an idealized description, can be viewed as point-like particles (far separated from one another, 
so that interactions among these particles can be neglected). The state of an individual particle is 
its \textit{position} in physical space. The ``state'' of $\mathfrak{E}$, given by its density $\rho$, should 
then be interpreted as an \textit{ensemble average} over the states of the particles constituting the 
ensemble $\mathfrak{E}$. An \textit{individual system} in this ensemble consists of a \textit{single particle}. 
According to \textit{Einstein} and \textit{Smoluchowski} (1905), the particles in $\mathfrak{E}$ exhibit 
\textit{Brownian motion} arising from random collisions with lumps of water molecules. (From this they 
derived, for example, a formula for the diffusion constant, namely $D=\frac{k_B T}{6\pi \eta r}$.)
We have learned from Einstein, Smoluchowski and \textit{Wiener} that Brownian motion ``unravels''
the diffusion equation, with the follwing \textit{\textbf{ontology}}. 
\begin{enumerate}
\item[(i)]{ At every time $t$, the particle is located in some point $x_{\xi}(t)\in \mathbb{E}^{3}$.}
\item[(ii)]{Its trajectory $\xi:=\big\{x_{\xi}(t)\big\}_{t\geq t_0}$ is a random continuous curve -- a Brownian path -- in 
physical space $\mathbb{E}^{3}$; but the velocity of the particle is ill-defined at all times.}
\item[(iii)]{As shown by \textit{Wiener}, there exists a \textit{\textbf{probability measure},} $dW_{x_0}(\xi)$, 
on the space, $\Xi$, of particle trajectories, 
$\xi:= \big\{x_{\xi}(t)\in \mathbb{E}^{3}\,\big|\, t \geq t_0, \, x_{\xi}(t_0)=x_0 \big\}$, 
starting from $x_0$ at time $t_0$; this measure is supported on trajectories $\xi$ that are 
\textit{H\"older continuous of index} $\frac{1}{2}$, etc.}
\item[(iv)]{An ``event'' at time $t$ is the manifestation of the position, $x_{\xi}(t)$, of the particle. 
The trajectory $\xi$ can thus be viewed as a \textit{``history of events,''} a \textit{random object}, 
and $\Xi$ is the \textit{``space of histories.''} The manifestation of the position of a particle at some time
does not affect its future trajectory after that time. In this respect, $QM$ is different.}
\end{enumerate}
Wiener measure $dW_{x_0}(\xi)$ allows us to predict probabilities of measurable sets of histories;
for example,
\begin{align}\label{LSW}
\begin{split}
\text{prob}\big\{\xi\in \Xi\,\big| x_{\xi}(t_i)\in &\mathcal{O}_i, \,i=1,2,\dots,n, \,t_0<t_1<\cdots<t_n\big\}\\
&=\int_{\Xi} dW_{x_0}(\xi) \prod_{i=1}^{n} \chi_{\{x_{\xi}(t_i)\in \mathcal{O}_i\}}\big(\xi\big)\,,
\end{split}
\end{align}
where $\chi_{\Delta}$ is the characteristic function of the set $\Delta \subset \Xi$.

The Chapman-Kolmogorov equation satisfied by the heat kernels implies that if regions 
$\mathcal{O}_{i}^{(\alpha)},$ $\alpha = 1,\dots, N$, for some $N$, are chosen such that
$\bigcup_{\alpha=1}^{N} \mathcal{O}_{i}^{(\alpha)}= \mathbb{E}^{3}$ then
\begin{align}\label{Markov}
\sum_{\alpha=1}^{N} \text{prob}\big\{\xi\,\big|& x_{\xi}(t_1)\in \mathcal{O}_1, \dots, x_{\xi}(t_i)\in \mathcal{O}_{i}^{(\alpha)},
\dots x_{\xi}(t_n)\in \mathcal{O}_n\big\}\nonumber\\
=&\text{prob}\big\{\xi\,\big|\, x_{\xi}(t_1)\in \mathcal{O}_1, \dots, x_{\xi}(t_{i-1})\in \mathcal{O}_{i-1}, x_{\xi}(t_{i+1})\in \mathcal{O}_{i+1}, \dots x_{\xi}(t_n)\in \mathcal{O}_n\big\}\,.
\end{align}
This property implies that if the position of a particle were measured at some intermediate time $t_i$ and 
then a sum over all possible outcomes of this measurement were taken one would obtain the \textit{same} predictions for
the outcomes of measurements of the particle positions at times earlier than $t_i$ and at times later than
$t_i$ as if no measurement had been made at time $t_i$. This means that the retrieval of 
information about the position of a particle does not affect its evolution. $QM$ yields a totally different 
picture of reality (actually a more accurate one): A measurement \textit{always} affects predictions 
on the evolution of a system \textit{even} if a sum over all possible outcomes of the
measurement were taken.

Using Wiener measure to take an average over the ensemble $\mathfrak{E}$ of very many identical 
particles, one recovers the \textit{deterministic law} in Eq.~\eqref{diffusion} for the evolution 
of the ``state'' $\rho_t$,
\begin{align}\label{average}
\begin{split}
\int_{\mathcal{O}}d^{3}x\,\rho_{t}(x)= &\int_{\mathcal{O}}d^{3}x\,\int d^{3}x_0 \,\Gamma_{t-t_0}(x-x_0) \rho_{t_0}(x_0)\\
=& \int d^{3}x_0\, \rho_{t_0}(x_0) \int_{\Xi} dW_{x_0}(\xi)\, \chi_{\{x_{\xi}(t)\in \mathcal{O}\}}\big(\xi\big),
\end{split}
\end{align}
for an arbitrary open subset $\mathcal{O}\subset \mathbb{E}^{3}$.
We note that the Chapman-Kolmogorov equation for the heat kernels implies the 
Markov property for the Wiener measure $dW_{x_0}$, i.e., that a measurement 
of the particle position at some time $t$ wipes out all memory of its trajectory at times earlier
than $t$. In contrast, in quantum mechanics there usually are memory effects.

One might say that the Wiener measure \textit{``unravels''} the diffusion equation \eqref{diffusion}.
In the next section, we describe an ``unraveling'' of the linear, deterministic Schr\"odinger-von Neumann evolution 
of ensemble-averages of states of identical systems by a \textit{non-linear, stochastic evolution} of states of 
\textit{\textbf{individual}} systems inspired by the observations concerning diffusion and Brownian motion just sketched. 
This will yield a \textit{completion of $QM$} and equip it with a plausible ``ontology.''

\section{``Unraveling'' the Schr\"odinger-von Neumann equation}\label{ETH approach}
\textit{The atoms or elementary particles themselves are not ``real;'' they form a world of potentialities 
or possibilities rather than one of things or facts.} (Werner Heisenberg)

\vspace{0.2cm}In this section we describe the \textit{\textbf{third pillar}} to be added to the two conventional pillars 
of text-book quantum mechanics described in Sect.~1, in order to arrive at a complete theory. 
The \textit{\textbf{ontology}} of our completion of $QM$ will be found in \textit{``random histories of events,''} 
defined appropriately; in analogy to histories of positions (Brownian paths) occupied by a point-like particle 
exhibiting Brownian motion. In $QM$ one would like to equip the (non-commutative) space of histories 
of events with a \textit{``quantum probability measure''}; in analogy with the Wiener measure of Brownian motion. 
Our task is to \textit{find} this probability measure, or, more precisely, \textit{to find an appropriate 
notion of states of physical systems in quantum mechanics and to describe their non-linear 
stochastic time evolution.}

The \textit{ETH - Approach to QM}, developed during the past decade (see \cite{FS, BFS, Froh, FP}), 
accomplishes this task. Since this completion of $QM$ may not be very widely known and appreciated, yet,
we have to briefly sketch it again (in this paper for non-relativistic $QM$; but there also exists a \textit{relativistic version} 
\cite{Fr}). We follow the presentation in \cite{FZ}.\footnote{It
really does not make much sense to present this approach to $QM$ in a new way each time it has to be recalled, 
because people have chosen not to take notice of it.}

\subsection{Fundamental ingredients of the \textit{ETH} - Approach to quantum mechanics}
In this section, we make use of the Heisenberg picture; and we consider \textit{isolated} systems, 
i.e., systems, $S$, that have negligibly weak interactions with the rest of the Universe. 
For, only for isolated systems, the time-evolution of operators representing physical 
quantities of $S$ has a conceptually clear description in the form of the \textit{Heisenberg equations of motion}. 
The main ingredients of the $ETH$-Approach to the quantum theory of isolated systems
are the following ones.
\begin{enumerate}
\item[I.]{We define $\mathcal{E}_{\geq t}$ to be the (weakly closed) algebra\footnote{i.e., a von Neumann algebra} 
generated by all the operators
\begin{equation}\label{algebra}
\big\{\,X(t')\,\big|\, t'\geq t,\,\, \widehat{X} \in \mathcal{O}_S\,\big\}\,,
\end{equation}
Evidently, 
$$\mathcal{E}_{\geq t'}\, \subseteqq\,\, \mathcal{E}_{\geq t}, \,\text{ for }\,\, t'>t\,.$$

For an \textit{autonomous system} one has that
\begin{align}\label{inclusion}
\mathcal{E}_{\geq t'}\, =\,\, e^{i(t'-t)H_S/\hbar}\, \mathcal{E}_{\geq t}\, e^{-i(t'-t)H_S/\hbar},\,\, \text{ for }\,\, t, t' \,\text{ in } \, \mathbb{R}\,.
\end{align}
}
\item[II.]{An \textit{isolated \textbf{open} physical system}, $S$, (i.e., an isolated system releasing \textit{``events''}) 
is described by a ``co-filtration,'' $\big\{\mathcal{E}_{\geq t}\,\vert\, t\in \mathbb{R}\big\}$, of von Neumann 
algebras (contained in the algebra, $B(\mathcal{H})$, of all bounded operators on $\mathcal{H}$) 
that satisfy the following

\vspace{0.1cm}\noindent \textbf{Principle of Diminishing Potentialities ($PDP$):} \textit{In an isolated \textbf{open}
system $S$ featuring events the following strict inclusions hold}
\begin{equation}\label{PDP}
\hspace{2.4cm}\mathcal{E}_{\geq t}\,\, \supsetneqq\,\, \mathcal{E}_{\geq t'}\,, \,\text{ for arbitrary }\,\, t'>t\,.
\end{equation}

People tend to be perplexed when hearing about $PDP$, because they find it hard to believe that
$PDP$ is compatible with the unitary Heisenberg dynamics of operators described in Eqs.~\eqref{2} and \eqref{inclusion}.
However, in a relativistic local quantum (field) theory over an even-dimensional, flat space-time 
containing a massless ``radiation field,'' such as quantum electrodynamics, and for an 
appropriate choice of the algebras $\mathcal{E}_{\geq t}, t\in \mathbb{R}$, 
$PDP$ can be shown to be a consequence of \textit{Huygens' Principle}, 
as formulated and proven in \cite{Buchholz} in the context of algebraic quantum field theory. 
In \cite{FP}, some concrete models, including models arising when the velocity of light tends to $\infty$, 
are shown to satisfy $PDP$.}
\item[III.]{The notion of \textit{\textbf{``events''}}\footnote{in the sense the late \textit{Rudolf Haag} used this terminology; see \cite{Haag}}
plays a central role in the $ETH$-Approach: A \textit{\textbf{potential event}} in $S$ setting in at time $t$ is described 
by a partition of unity,
\begin{equation}\label{p-o-u}
\mathfrak{P}:=\big\{\pi_{\xi}\,\big|\, \xi \in \mathfrak{X}\big\} \subset \mathcal{E}_{\geq t}, 
\end{equation}
by orthogonal, mutually disjoint projections, $\pi_{\xi}$, with the properties that
\begin{equation}\label{projections}
\pi_{\xi}= \pi_{\xi}^{*}, \quad \pi_{\xi}\cdot \pi_{\eta}= \delta_{\xi \eta} \pi_{\xi},\,\,\, \forall\, \xi, \eta \in \mathfrak{X}, 
\quad \sum_{\xi \in \mathfrak{X}} \pi_{\xi}= \mathbf{1}\,,
\end{equation}
where $\mathfrak{X}$ is a finite or countably infinite set of labels called the \textit{spectrum} of the potential
event $\mathfrak{P}$ and denoted by $\frak{X}= \text{spec}(\mathfrak{P}$).
}
\item[IV.]{A \textit{state} of an isolated system $S$ at time $t$ is given by a \textit{quantum probability measure} 
on the lattice of orthogonal projections in $\mathcal{E}_{\geq t}$, i.e., by a functional, $\omega_t$, 
with the properties that
\begin{enumerate}
\item[(i)]{$\omega_t$ assigns to every orthogonal projection $\pi \in \mathcal{E}_{\geq t}$ a non-negative number 
$\omega_{t}(\pi) \in [0,1]$, \hspace{0.4cm} with $\omega_t(0)=0,$ and $\omega_{t}(\mathbf{1}) =1$; and}
\item[(ii)]{$\omega_{t}$ is \textit{additive}, i.e.,
\begin{equation} \label{additivity}
\sum_{\pi \in \mathfrak{P}} \omega_{t}(\pi) = 1, \quad \forall\, \text{ potential events }\, 
\mathfrak{P} \subset \mathcal{E}_{\geq t}\,.
\end{equation}
A generalization of \textit{Gleason's theorem} due to \textit{Maeda} \cite{GM} implies that states, $\omega_t$, of 
$S$ at time $t$, as defined above, are \textit{positive, normal, normalized linear functionals} on 
$\mathcal{E}_{\geq t}$, i.e., \textit{states} on $\mathcal{E}_{\geq t}$ in the usual sense of this notion 
employed in the mathematical literature. (Ignoring some mathematical subtleties) we henceforth identify 
$\omega_t$ with a density matrix on $\mathcal{H}$ denoted by $\Omega_t$.}
\end{enumerate}
}
\end{enumerate}

\subsection{Consequences of the Principle of Diminishing Potentialities}
The Principle of Diminishing Potentialities, when combined with the phenomenon of \textit{entanglement,}
implies that even if the state $\omega_t$ of $S$ at time $t$ were a ``pure'' state on the algebra 
$\mathcal{E}_{\geq t}$ its restriction to the algebra $\mathcal{E}_{\geq t'}$ must be expected to 
be \textit{``mixed''} if $t'>t$. This observation opens the possibility to introduce the notion of 
``events \textit{actualizing} at some time.'' 

In accordance with the \textit{``Copenhagen interpretation''} of $QM$, one might expect that a 
\textit{potential event} 
$\mathfrak{P}=\big\{ \pi_{\xi}\, \big|\, \xi \in \text{spec}(\mathfrak{P}) \big\} \subset \mathcal{E}_{\geq t}$, 
becomes actual (manifest) at some time $\geq t$ iff
\begin{equation}\label{incoherent sp}
\text{tr}(\Omega_{t}\,A) = \sum_{\xi \in \mathfrak{X}} \text{tr}(\pi_{\xi}\, \Omega_{t}\,\pi_{\xi} \, A),\quad 
\forall A \in \mathcal{E}_{\geq t}\,.
\end{equation}
where $\Omega_t$ is the density matrix representing the state, $\omega_t$, of $S$ at time $t$.
Notice that, off-diagonal elements do \textit{not} appear on the right side of \eqref{incoherent sp}, 
which thus describes an \textit{incoherent} superposition of states in the images of disjoint 
orthogonal projections, i.e., a \textit{``mixture.''} 

This expectation is made precise as follows. Given a state $\omega_t$ on $\mathcal{E}_{\geq t}$, 
we define $\mathcal{C}(\omega_t)$ to be the subalgebra of $\mathcal{E}_{\geq t}$ 
generated by \textit{all} projections belonging to \textit{all} potential events 
$\mathfrak{P}\subset \mathcal{E}_{\geq t}$ for which Eq.~\eqref{incoherent sp} holds. 
Further, $\mathfrak{P}(\omega_t)$ is the \textit{finest potential event} contained in $\mathcal{C}(\omega_t)$ 
with the property that \textit{all} its elements commute with \textit{all} operators in $\mathcal{C}(\omega_t)$.\footnote{In 
more technical jargon, $\mathfrak{P}(\omega_t)$ generates the \textit{center} of the centralizer $\mathcal{C}(\omega_t)$
of $\omega_t$.}
We then say that the potential event $\mathfrak{P}(\omega_t)$ \textit{actualizes} at some time 
$\geq t$ iff $\mathfrak{P}(\omega_t)$ contains \textit{at least} two non-zero orthogonal projections, 
$\pi^{(1)}, \pi^{(2)}$, which are disjoint, i.e., $\pi^{(1)}\cdot \pi^{(2)} =0$, and have non-vanishing 
Born probabilities, i.e.,
$$0< \omega_{t}(\pi^{(i)}) = \text{tr}\big(\Omega_t\,\pi^{(i)}\big) < 1\,, \quad \text{ for  }\, i=1,2\,.$$
Equation \eqref{incoherent sp} then holds true for $\mathfrak{P}=\mathfrak{P}(\omega_t)$, and the sum on the
right side of \eqref{incoherent sp} contains at least two distinct non-vanishing terms.

\subsection{The state-reduction postulate and the stochastic evolution of states}
The \textit{\textbf{law}} describing the non-linear stochastic time evolution of states of an individual isolated 
open system $S$ unraveling the linear deterministic evolution of ensemble averages of states
is derived from a \textit{\textbf{state-reduction postulate}} described next. This postulate makes precise 
mathematical sense as long as \textit{time} is \textit{discrete}.

Let $\omega_t$ be the state of $S$ at time $t$. Let $dt$ denote a time step; ($dt$ is strictly positive 
if time is discrete; otherwise one attempts to let $dt$ tend to 0 at the end of the following constructions). 
We define a state $\overline{\omega}_{t+dt}$ on the algebra 
$\mathcal{E}_{\geq t+ dt} \,\,(\subsetneqq \mathcal{E}_{\geq t})$ by restriction of $\omega_t$ to the algebra 
$\mathcal{E}_{\geq t+dt}$,
\[\overline{\omega}_{t+dt}:= \omega_{t}\big|_{\mathcal{E}_{\geq t+dt}}\,.\]
As a manifestation of $PDP$ and \textit{entanglement,} the algebra $\mathcal{C}(\overline{\omega}_{t+dt})$
can be expected to be non-trivial (i.e., $\not= \mathbb{C}\cdot \mathbf{1}$) in general. This does, of
course, \textit{not} imply that the potential event $\mathfrak{P}(\overline{\omega}_{t+dt})$ 
actualizing at some time $\geq t+dt$ is non-trivial, too, i.e., $\not= \mathbf{1}$. But it is plausible 
that it will in general be non-trivial. (This is shown to be the case in a family of models studied in \cite{FP}.)\\

\noindent {\textbf{Axiom CP:}} \,\,\textit{Let} 
$$\mathfrak{P}(\overline{\omega}_{t+dt})=\big\{\pi_{\xi}\,|\,\xi\in 
\text{spec}\big(\mathfrak{P}(\overline{\omega}_{t+dt})\big)\big\}$$ 
\textit{be the potential event actualizing at some time $\geq t+dt$, given the state $\overline{\omega}_{t+dt}$ 
on $\mathcal{E}_{\geq t+dt}$. Then \textit{`Nature'} 
replaces the state $\overline{\omega}_{t+dt}$ on 
$\mathcal{E}_{\geq t+dt}$ by a state $\omega_{t+dt} \equiv \omega_{t+dt, \pi}$ represented by the density
matrix}
\begin{equation}\label{state red}
\Omega_{t+dt, \pi}:=
\text{tr}(\overline{\Omega}_{t+dt}\,\pi)^{-1} \cdot \pi\,\,\overline{\Omega}_{t+dt}\,\,\pi\,, \,\,\text{for some }\,\,\pi \in 
\mathfrak{P}(\overline{\omega}_{t+dt}),
\end{equation}
\textit{with} $\text{tr}(\overline{\Omega}_{t+dt}\,\pi)\not= 0$.
\textit{The probability,} $\text{prob}_{t+dt}(\pi)$, \textit{for the state 
$\omega_{t+dt,\pi}, \pi \in \mathfrak{P}(\overline{\omega}_{t+dt}),$ to be selected 
by \textit{`Nature'} as the state of $S$ at time $t+dt$ is given by Born's Rule}
\begin{equation}\label{Born Rule}
\hspace{4cm}\text{prob}_{t+dt}(\pi)= \text{tr}(\overline{\Omega}_{t+dt}\,\,\pi)\,. \hspace{3.5cm}\square
\end{equation}

The projection $\pi(t+dt):=\pi \in \mathfrak{P}(\overline{\omega}_{t+dt})$ appearing in \eqref{state red}
and \eqref{Born Rule} is called \textit{\textbf{actual event},} or \textit{``actuality,''} at time $t+ dt$.

The analogue of the initial position, $x_0$, of a Brownian path at time $t_0$ is the initial state 
$\omega_0$ on $\mathcal{E}_{\geq t_0}$; the analogue of the Brownian trajectory 
$\xi=\big\{x_{\xi}(t)\,\big|\,t\geq t_0\big\}$ is given by a \textit{\textbf{history},} 
$\mathfrak{h}:=\big\{\pi(t_0+dt), \pi(t_0 + 2dt), \dots, \pi(t)\big\}$, of \textit{\textbf{actual events}} originating
from the initial state $\omega_0$ of $S$ at time $t_0$. With a history $\mathfrak{h}$ we 
associate a \textit{``history operator''} defined by
$$H_{\mathfrak{h}}(t_0, t):= \prod_{t'\in \mathbb{Z}_{dt},\, t_0 < t' \leq t} \pi(t') \,.$$
In quantum mechanics, the role of the Wiener measure, $dW_{x_0}$, of Brownian motion is played
by the probabilities
\begin{align}\label{prob history}
\text{prob}_{\omega_{0}}\big[ \mathfrak{h}=&\big\{\pi(t_0+ dt), \pi(t_0 + 2dt),\dots, \pi(t)\big\}\big]:=\nonumber\\
&=\omega_{0}\big(H_{\mathfrak{h}}(t_0,t)\cdot H_{\mathfrak{h}}(t_0,t)^{*}\big)
=\text{tr}\big[H_{\mathfrak{h}}(t_0,t)^{*}\cdot \Omega_{0}\cdot H_{\mathfrak{h}}(t_0,t)\big]
\end{align}
of histories of events, where $\Omega_o$ is the density matrix representing the initial
state $\omega_0$ on the algebra $\mathcal{E}_{\geq t_0}$.

It follows from our discusssion that the time-evolution of the \textit{state} of an \textit{\textbf{individual}} physical 
system $S$ is described by a \textit{stochastic branching process}, called \textit{``quantum Poisson process''}, 
whose ``state space'' is referred to as the \textit{non-commutative spectrum}, $\mathfrak{Z}_{S}$, of $S$ 
and is defined as follows. By equation \eqref{inclusion}, all the algebras $\mathcal{E}_{\geq t}$ are isomorphic 
to one specific (universal) von Neumann algebra, which we denote by $\mathcal{M}$. The non-commutative 
spectrum, $\mathfrak{Z}_{S}$, of $S$ is defined by
\begin{equation}\label{NCspect}
\mathfrak{Z}_{S}:= \bigcup_{\omega} \Big(\,\omega\,, \mathfrak{P}(\omega)\Big)\,, 
\end{equation}
where the union over $\omega$ is a disjoint union, and $\omega$ ranges over \textit{all} states on $\mathcal{M}$ 
of physical interest. (``States of physical interest'' are normal states on $\mathcal{M}$ a concrete system 
can actually be prepared in.) 
The branching rules of a quantum Poisson process on $\mathfrak{Z}_S$ are uniquely determined by \textbf{Axiom CP}.

\vspace{0.2cm}\textit{\underline{Comments}.}
\begin{itemize}
\item{One may expect -- and this can be verified in concrete models (see \cite{FP} 
for further details) -- that, most of the time, the actual event, $\pi \in \mathfrak{P}(\overline{\omega}_{t+dt})$, 
which, according to the Born Rule, has the largest probability to happen, and hence is most likely 
to be chosen by `Nature' (see \eqref{state red}), has the property that
\begin{equation}\label{trivial evol}
\omega_{t+ dt}\equiv \omega_{t+dt, \pi} \approx \overline{\omega}_{t+ dt} = \omega_{t}\big|_{\mathcal{E}_{\geq t+dt}}\,.
\end{equation}
This would imply that, most of the time, the evolution of the state is close to being trivial (as assumed
in text-book $QM$ in the absence of ``measurements''). But, every once in a while, the 
state of the system makes a \textit{``quantum jump''} corresponding to an actual event $\pi$ in
\eqref{state red} that is very unlikely to materialize. Such ``quantum jumps'' happen for purely \textit{entropic} 
reasons at \textit{random times}.}
\item{One may check that the non-linear stochastic evolution of states outlined above has the 
desirable feature that it reproduces the usual Schr\"odinger-von Neumann evolution when an ensemble-average 
over all possible histories of very many identical systems is taken.}
\item{Our construction of the non-linear stochastic time evolution of individual systems is meaningful, 
mathematically, as long as $dt >0$; but, for the time being, the limiting theory, as $dt \searrow 0$, 
is only understood precisely in examples (see Sect.~5).}
\end{itemize}

\section{The Principle of Diminishing Potentialities as a consequence of Huygens' Principle}
\textit{If speculative ideas cannot be tested, they're not science; they don't even rise to the level of being wrong.} 
(Wolfgang Pauli)

In this section we sketch a physical mechanism implying the validity of the Principle of 
Diminishing Potentialities ($PDP$) in realistic models: It is claimed that \textit{Huygens' Principle} 
in local relativistic quantum field theory on even-dimensional Minkowski space-time with
a \textit{``radiation field''} that describes massless modes, such as the electromagnetic field or 
the gravitational field, implies that $PDP$ holds for \textit{isolated physical systems with degrees of 
freedom coupled to the radiation field.} Rather than reviewing the general theory of the Huygens 
Principle originally developed in the framework of algebraic quantum field theory by \textit{D. Buchholz} 
in \cite{Buchholz} we consider an example.\\

\noindent\textit{\underline{Huygens' Principle in an idealized system}:}
Let $S$ be an isolated system consisting of a \textit{static atom} (located near $\mathbf{x}=0$)
with an electric dipole moment that couples to the \textit{quantized electromagnetic field.} We assume that
\begin{itemize}
\item{the atom has $M$ energy levels; hence its Hilbert space of state vectors is given by
$\mathfrak{h}_A\simeq \mathbb{C}^{M}$;}
\item{the Hilbert space of the free electromagnetic  field is the usual \textit{Fock space}, $\mathfrak{F}$, of photons.
The quantized electromagnetic field is described by its field tensor, $F_{\mu \nu}(t, \mathbf{x})$,  ($x=(t, \mathbf{x})$ 
a point in Minkowski space-time), which is an operator-valued distribution with the property that, for real-valued 
test functions $\big\{h^{\mu \nu}\big\}$ on space-time, 
\[ F(h):= \int_{\mathbb{R}\times \mathbb{R}^{3}} dt\, d\mathbf{x} \, F_{\mu \nu}(t, \mathbf{x})\,h^{\mu\nu}(t, \mathbf{x})\]
is a self-adjoint operator on $\mathfrak{F}$ that satisfies \textit{locality} in the form of ``Einstein causality.''}
\end{itemize}
\begin{center}
\includegraphics[height=7.8cm]{4-Huygens}\\
\hspace{0.6cm} {\small{space-time diamonds \hspace{3.5cm} 
time slices in NR limit, \,$c\rightarrow \infty$ \quad\\
Figure 4: \textit{An illustration of Huygens Principle}}}
\end{center}
The free electromagnetic field satisfies the \textit{spectrum condition,} i.e., the energy of all its physical states 
is non-negative. Its Hamiltonian is denoted by $H_f$; it satisfies $H_f\geq 0$. 

Since the atom is located near $\mathbf{x}=0$ and is static, it is useful to introduce the space-time diamonds 
$$D_{[t,t']}:= V^{+}_{t} \cap V^{-}_{t'}, \,\,\, t'>t,$$ 
centered on the time axis ($\mathbf{x}=0$), with $V^{\pm}_{t}$ the forward or backward light cone, respectively, 
with apex in the point $(t, \mathbf{x}=0)$ on the time axis. We will also consider models arising when the speed
of light, $c$, tends to $\infty$. These models will serve to illustrate $PDP$ in the context of non-relativistic 
quantum mechanics (see also \cite{FP}).

\subsection{A (not so) simple model}
The Hilbert space of the system $S$ is chosen to be
$$\mathcal{H}:=\mathfrak{h}_{A}\otimes \mathfrak{F}\,.$$
Bounded functions of the field operators $F(h)$, with $h^{\mu\nu}$ real-valued and supported in $D_{[t,t']}$, 
for all $\mu, \nu,$ generate a von Neumann algebra $\mathcal{A}_{I=[t,t']}$. We define the von Neumann 
algebras
\begin{align}\label{algebras}
\begin{split}
\mathcal{D}_{I}^{(0)} := \mathbf{1}\big|_{\mathfrak{h}_A}\otimes \mathcal{A}_I\,, & \qquad
\mathcal{E}_{I}^{(0)}:= B(\mathfrak{h}_{A})\otimes \mathcal{A}_{I}\,,\\
\mathcal{E}_{\geq t}^{(0)}:= &\overline{\bigvee_{I \subset [t,\infty)} \mathcal{E}_{I}^{(0)}}\,,
\end{split}
\end{align}
where the closure is taken in the weak$^{*}$ topology. 

We first convince ourselves that 
$PDP$ holds for this system before the atom is coupled to the electromagnetic field. One has that
\begin{equation}\label{rel commutants}
\big[\mathcal{E}_{\geq t'}^{(0)}\big]' \cap \mathcal{E}_{\geq t}^{(0)} = \mathcal{D}_{[t,t']}^{(0)}\quad
(\text{an }\infty-\text{dim. algebra})\,,
\end{equation} 
which is a strong form of $PDP$.\\
\textit{\underline{Remark}:} Property \eqref{rel commutants} follows from \textit{``Huygens' Principle''}, namely from
\begin{equation}\label{loc comm}
[F_{\mu\nu}(x), F_{\rho, \sigma}(y)]=0, \,\,\forall\, \mu, \nu, \rho, \sigma, \,\, \text{ unless }\,\,x-y \text{ is \textbf{light-like}};
\end{equation}
(see Figure 4, and \cite{FP} for details).

From now on, we make use of an ultraviolet regularization of quantum electrodynamics arising 
from discretizing time, $t_{n} := n\,\tau,\,\, n \in \mathbb{Z}, \,\,\tau>0$ denotes the time step, \,($\tau \equiv dt$, 
in the notation of Sect.~3).
To describe interactions, we pick a unitary operator $U \in \mathcal{E}_{[0,\tau]}^{(0)}$ and define
$$U_{k}:= e^{i(k-1)\tau H_f} \,U\, e^{-i(k-1)\tau H_f}, \quad k=1,2,\dots, \quad U(n):=\prod_{k=1}^{n} U_{k}\,.$$
\begin{equation}\label{propagator}
\Gamma:= e^{-i\tau H_f} U \,\, \Rightarrow\,\, \Gamma^{n}=e^{-in\tau H_f} U(n), \,\, (\Gamma^{n})^{*}=: \Gamma^{-n}, \,\, n=0,1,2,\dots,
\end{equation}
with $\Gamma^{0}=\mathbf{1}$. The operators $\big\{\Gamma^{n}\big\}_{n\in \mathbb{Z}}$ represent the 
propagator of an interacting system with discrete time.

To study the dynamics of this model it suffices to consider the time evolution for times $t \geq t_0:=0$. We define
\begin{equation}\label{interacting algebra}
\mathcal{E}:=\mathcal{E}_{\geq 0}^{(0)}, \quad \mathcal{E}_{\geq n}:= \big\{ \Gamma^{-n}\,X\,\Gamma^{n}\,\big|\,X\in 
\mathcal{E}\big\}\,.
\end{equation}

\textit{\underline{$PDP$ for the interacting model}:} It is not difficult to show, using \eqref{propagator} and 
\eqref{interacting algebra}, that
\begin{equation}\label{PDP interacting}
\big[\mathcal{E}_{\geq n'}\big]' \cap \mathcal{E}_{\geq n} \simeq \mathcal{D}_{[n,n']}\,, \,\,\,\text{for }\,\,n'>n,
\end{equation}
where \,\,$\mathcal{D}_{[n,n']}:= \big\{U(n')^{*}\, X\, U(n')\,\big|\, X\in \mathcal{D}_{[n\tau,n'\tau]}^{(0)}\big\}$.\\

Preparing the interacting system $S$ in an initial state $\omega_0$ at time $n=0$ (e.g., one where the electromagnetic
field is in the vacuum state and the atom is in an excited state), one may determine the stochastic time evolution 
of the state of $S$ (featuring spontaneous emission of photons by the atom at random times) according to the 
$ETH$ - Approach, as prescribed in \textbf{Axiom CP} of Subsect.~3.3.
It is rather difficult to come up with explicit results, because, for a finite velocity of light, $c$, the electromagnetic 
field gives rise to memory effects related to the fact that expectations of field operators localized in compact 
regions belonging to different time slices do not factorize in states of physical interest. Memory effects 
are related to the possibility that virtual photons emitted by the atom can be reabsorbed by it. 
To avoid this difficulty we will pass to a non-relativistic description of the system $S$ 
emerging in the limit where $c$ tends to $\infty$. 

\section{Fluorescence of two-level atoms coupled to the quantized radiation field}
\textit{``One is} [thus] \textit{led to conclude that the description of reality as given by} [the Schr\"odinger 
evolution of] \textit{a wave function is not complete.''} (A.~Einstein, B.~Podolsky, N.~Rosen)\\
In this section, we study \textbf{fluorescence} of very heavy two-level atoms coupled to the 
radiation field; (see, e.g., \cite{Pomeau} and refs.~given there). In order to be able to reach explicit results, we study this phenomenon in the ``non-relativistic''
limit where the speed of light $c\rightarrow \infty$, which drastically simplifies our analysis. The space-time 
diamonds $D_{[t',t'']}$ introduced in the last section then open up to time slices, 
\mbox{$\big\{(t, \mathbf{x})\,\big|\, t'\leq t < t''\big\}$,} 
and functionals of the ``radiation field'' localized in different time slices \textbf{commute}. The field Hamiltonian 
$H_f$ gets replaced by the generator, $\mathcal{P}$, of translations in the direction of the time axis, and the 
algebras $\mathcal{D}_{[t,t+T]}^{(0)}$ ``collapse'' to full matrix algebras $\simeq B(\mathcal{H}_{T}),$ \,
where\, $\mathcal{H}_T$ is a separable Hilbert space (described below), for arbitrary $T>0$. If the initial 
state of the radiation field is chosen to be a ``product state'' factorizing over different time slices, namely 
the vacuum $\big|\emptyset \big>$ introduced below, then the time evolution of ensemble-averaged 
states of $S$ becomes \textit{``Markovian.''} In this case very explicit results can be obtained; see \cite{FP} --
it may help the reader to first consider the case where time is discretized, as in Sects.~3 and 4.

\subsection{An explicit model of fluorescence}

We imagine that, every $T$ seconds, an atom source releases an atom prepared in a superposition of 
a ground state, $|\downarrow\rangle$, and an excited state, $|\uparrow\rangle$. In less than $T$ seconds, 
such an atom propagates to an atom-detector where, e.g., the ``observable'' 
$X:=\begin{pmatrix} 1&0\\0&-1 \end{pmatrix}$, acting on the Hilbert space, $\mathfrak{h}_A=\mathbb{C}^{2}$, 
of internal states of the atom, is measured. During the trip from source to detector, the atom may 
jump from $|\uparrow\rangle$ to $|\downarrow\rangle$ and emit a ``photon'',\footnote{We put the expression ``photon''
in quotation marks when we consider the model in the limit $c\rightarrow \infty$} $\gamma$, as first 
studied by \textit{Einstein} in 1916. Different atoms are treated as statistically independent -- there are 
no correlations between the states they are prepared in.\vspace{0.15cm}\\
We consider two different experiments.
\begin{enumerate}
\item{A``photon'' possibly emitted by an atom on its trip from source to detector escapes the experimental 
setup and is \textit{not} detected before any entanglement between its state and the state
of the atom is wiped out.}
\item{A ``photon'' possibly emitted by such an atom ``immediately'' hits a photo-multiplier that clicks when hit
by the ``photon'' before the atom ends its trip to its own detector, where the ``observable'' $X$ is measured; 
entanglement between the state of the photon and the state of the atom is preserved until the two measurements
occur.}
\end{enumerate}
The Hilbert space of the total system is given by 
$$\mathcal{H}:=\mathfrak{h}_{A}\otimes \mathfrak{F} \otimes \mathfrak{H}_{\gamma}\,,$$
where $\mathfrak{h}_A=\mathbb{C}^{2}$ is the Hilbert space of a two-level atom, and $\mathfrak{F}$ is the 
Fock space of the ``radiation field'' (in the limit where $c\rightarrow \infty$), which is defined below. 
Moreover, $\mathfrak{H}_{\gamma}$ is the Hilbert space of the photo-multiplier; it will not enter 
the following considerations explicitly. In experiment 1, the photo-multiplier is turned off.\\

It is convenient to parametrize the \textit{states of an atom} by density matrices, $\rho(\vec{n})$, on 
$\mathfrak{h}_A$ given by
\begin{equation}\label{Bloch param}
\rho(\vec{n}):= \frac{1}{2}\big(\mathbf{1}_2 + \vec{n}\cdot \vec{\sigma}\big),\quad \vec{n}\in \mathbb{R}^{3}, 
\text{ with } |\vec{n}\, | \leq 1\,,
\end{equation}
where $\vec{\sigma}:= \big(\sigma_1, \sigma_2, \sigma_3\big)$ is the vector of Pauli matrices. We recall that
the state $\rho(\vec{n})$ is \textit{pure}, i.e., $\rho(\vec{n})$ is a rank-1 orthogonal projector, iff $|\vec{n}\,|=1$
(i.e., $\vec{n}$ belongs to the ``Bloch sphere'').
Moreover,
$\rho(\vec{n}) + \rho(-\vec{n})=\mathbf{1}$, and $\text{tr}\big( \vec{\sigma} \cdot \rho(\vec{n})\big) = \vec{n}.$
The matrix $\rho(\vec{n})$ has eigenvalues $\frac{1+|\vec{n}|}{2}$ and $\frac{1-|\vec{n}|}{2}$, 
with eigenspaces given by the ranges of the projections $\rho(\pm \frac{\vec{n}}{|\vec{n}|})$, respectively.

The Hamiltonian, $H_A$, of an atom decoupled from the radiation field is given by
\begin{align}\label{Hamiltonian}
H_A:= (1/2)\vec{\omega}\cdot \vec{\sigma},\,\, \text{ with }\,\vec{\omega}= (0,0, \Omega).
\end{align}
One then has that
\begin{align}
e^{itH_A}\rho(\vec{n}_0)e^{-itH_A}=& \rho\big(\vec{n}(t)\big), \,\, \text{ where }\,\,
\vec{n}(t)= (\text{sin}\theta_0 \, \text{cos}\varphi(t), \text{sin}\theta_0 \, \text{sin}\varphi(t), \text{cos}\theta_0),
\end{align}
with $\varphi(t)= \varphi_0 + \Omega\cdot t$ and   $\vec{n}_0 = \vec{n}(t=0)=
(\text{sin}\theta_0 \, \text{cos}\varphi_0, \text{sin}\theta_0 \, \text{sin}\varphi_0, \text{cos}\theta_0)$; ($(\theta, \varphi)$ 
are the usual polar angles).\\

The \textit{Fock space,} $\mathfrak{F}$, of the radiation field is defined as follows. We introduce creation- 
and annihilation operators $a^{*}(t, X)$ and $a(t, X), \text{ with }\, t \in \mathbb{R},\, X \in \mathcal{X}$, 
where $\mathcal{X}$ represents ``physical space'' (which, for simplicity, we may suppose to be a finite set of points) and
the possible polarizations of a ``photon.'' The operators $a^{*}(t, X)$ and $a(t, X)$ satisfy the canonical commutation relations
\begin{align}
[a(t, X), a^{*}(t', X')]= \delta(t-t')\cdot C_{X\,X'} \qquad 
[a^{\#}(t, X), a^{\#}(t', X')] = 0\,,
\end{align}
for all $t, t'$ in $\mathbb{R}$ and all $X, X'$ in $\mathcal{X}$, where $a^{\#} = a \text{ or } a^{*}$,
and $\big\{C_{X\,X'}\big| X, X' \text{ in } \mathcal{X}\big\}$ are the matrix elements of a norm-bounded, 
positive-definite quadratic form C. We introduce a ``vacuum vector,'' $\big|\emptyset\big>,$ with the properties that 
$\big< \emptyset \big| \emptyset\big> =1$, and
\begin{equation}\label{vacuum}
a(t, X)\big|\emptyset \big> = 0, \quad \forall t\in \mathbb{R},\,\, \forall X \in \mathcal{X}\,.
\end{equation}
Fock space $\mathfrak{F}$ is defined to be the completion in the norm induced by the scalar product 
$\big<\cdot \big| \cdot\big>$ of the linear space of vectors obtained by applying arbitrary polynomials in creation 
operators, $a^{*}(\cdot, X),\, X\in \mathcal{X},$ smeared out with test functions in the time variable $t\in \mathbb{R}$. 
The Hamiltonian, $\mathcal{P}$, of the radiation field has the property that
\begin{equation}\label{field Hamiltonian}
e^{it \mathcal{P}}\, a^{\#}(s, X)\, e^{-it \mathcal{P}} = a^{\#}\big(t+s, \phi_{t}(X)\big), \quad \forall\,\, t, s \text{ in } \mathbb{R},\,\, 
\forall \,\,X \in \mathfrak{X}\,,
\end{equation}
where $\mathcal{X}\ni X\mapsto \phi_{t}(X)\in \mathcal{X}, \,t \in \mathbb{R},$ is some deterministic 
dynamics defined on $\mathcal{X}$ that preserves the quadratic form $C$; the choice of $\phi_t$ is 
irrelevant in the following discussion. The spectrum of $\mathcal{P}$ covers the entire real line and 
is absolutely continuous.

By $\big|\gamma\big>$ we denote a state of $\geq 1$ photons; any such state is orthogonal to the vacuum, 
i.e., $\big< \gamma \big| \emptyset\big> =0$.
General states of the radiation field are density matrices on $\mathfrak{F}$.

\textit{\underline{Remark}:} Simpler models of the radiation field with \textit{discrete} time have been 
considered in \cite{FP}. The purpose of introducing the model presented above is just to make
clear that, in the limit where the velocity of light tends to $\infty$, we can actually accommodate models 
with continuous time.\\

\textit{States of the photomultiplier} won't appear explicitly in what follows. The only important feature is
that the state of the ``dormant'' photo-multiplier is \textit{orthogonal} to all its states right after being hit by 
some photons. A photo-multiplier is a system with infinitely many degrees of freedom with the property that 
states occupied by the photo-multiplier right after being hit by some ``photons,'' when
evaluated on ``quasi-local observables,'' relax back to the state of the ``dormant'' photo-multiplier 
within a short relaxation time. Simple models of such systems have been studied; (see, e.g., \cite{FGS, DeR-K}).\\

Disregarding the photo-multiplier, the Hamiltonian, $H_S$, of the system to be studied is given by
\begin{equation}\label{total Ham}
H_S= H_A \otimes \mathbf{1} + \mathbf{1}_2 \otimes \mathcal{P} + H_I\,,
\end{equation}
where $H_I$ is an \textit{interaction Hamiltonian} describing emission and absorption of ``photons'' by the atom. For example,
$$H_I= g\,\big[ \sigma_{-}\otimes a^{*}(0, X_0) + \sigma_{+} \otimes a(0, X_0)\big]\,,$$
where $g$ is a real coupling constant assumed to be small, $\alpha:= g^{2}\ll 1$, 
$\sigma_{-}=\begin{pmatrix} 0&0\\1&0 \end{pmatrix}$ is the usual lowering operator on 
$\mathfrak{h}_A$, $\sigma_{+} =\begin{pmatrix} 0&1\\0&0 \end{pmatrix}$ is the 
raising operator, and $X_0\in \mathcal{X}$ is the ``position of the atom.'' The details of how $\mathcal{P}$ and
$H_I$ are chosen do not matter in the following discussion; they are indicated here just for concreteness
and in order to help the readers' intuition.

\subsection{Effective time evolution of an atom coupled to the radiation field}
In the following we describe the main results of our analysis; (details can be inferred from \cite{FP} and will
be reported in more detail in a separate paper).
\begin{enumerate}
\item{\underline{Photo-multiplier turned off}.

The initial state of the radiation field is chosen to be the vacuum $\big|\emptyset\big>$, which does not entangle 
``photons'' at different times. The effective time evolution of an atom is then ``Markovian'' and can be determined 
explicitly. In the parametrization of atomic states introduced in \eqref{Bloch param}, the effective dynamics
of an atomic state can be described in terms of the dynamics of a vector $\vec{n}(t)$ in the unit ball 
$\big\{\vec{n}\in \mathbb{R}^{3}\big|\, \vert \vec{n}\vert \leq 1\big\}$. In notations inspired by 
those used in Sect.~2, with $\overline{\omega}_{t+\tau} \mapsto \overline{\vec{n}}(t+dt)$, 
we find that
\begin{align}
\overline{\vec{n}}(t+dt)= \,\vec{n}(t) + d\vec{n}(t)\,, \quad \text{with }\quad
d\vec{n}(t) =\,\vec{\omega}\times \vec{n}(t) \,dt + dK\big[\vec{n}(t)\big]\,.
\end{align}
Here $\vec{\omega}:= (0,0,\Omega)$, and $dK$ is a linear ``dissipative'' map proportional in size to $dt$ 
(and known explicitly) which has the effect that the length of $\big|\vec{n}(t)\big|$ shrinks,
$$|\overline{\vec{n}}(t+dt)| = 1-\mathcal{O}(\alpha)\,dt<1, \quad \text{except if }\,\vec{n}(t)=-\vec{e}_3.$$
Applying \textbf{Axiom CP} of Sect.~2, the evolution of the state of an atom is found to be given by a 
\textit{Poisson jump process} on the Bloch sphere:
\begin{align}\label{jumps}
{+}) \quad \vec{n}(t) \mapsto &\vec{n}(t+dt):=  \frac{\overline{\vec{n}}(t+dt)}{|\overline{\vec{n}}(t+dt)|} \quad 
\text{with probability } 1-\mathcal{O}(\alpha)\,dt\,,\nonumber \\
{-}) \quad\vec{n}(t) \mapsto &\vec{n}(t+dt):= \mathbf{-}\,\frac{\overline{\vec{n}}(t+dt)}{|\overline{\vec{n}}(t+dt)|} 
\quad \text{with probability } \mathcal{O}(\alpha)\,dt\,.
\end{align}
The rate of this jump process is proportional to $\alpha$, i.e., the number of jumps from
$\vec{n}(t)$ to its antipode during the atom's trip from source to detector is proportional to 
$\alpha \,T$. The times when jumps occur are \textit{random variables} whose law can be determined explicitly. 
(In verifying these claims, it may be helpful to first imagine that time is discrete, as in Sects.~2 and 3, with $dt=\tau>0$, 
and let $dt$ approach 0 at the end of the calculations.)

When entering the atom detector the state of the atom is given by
\begin{equation}\label{out state}
\rho(\vec{n}_{out}), \quad\text{with } \,\, \vec{n}_{out}\approx 
(\text{sin}\theta_{out}\, \text{cos}\varphi_{out}, \text{sin}\theta_{out}\, \text{sin}\varphi_{out}, \text{cos}\theta_{out})\,,
\end{equation}
where $|\theta_{out} - \theta_0| = \mathcal{O}(\alpha)$ (no jump, mod. 2), $|\theta_{out} -\pi+ \theta_0| = \mathcal{O}(\alpha)$ 
(1 jump, mod. 2).}
\item{\underline{Photo-multiplier turned on}.

We begin with the observation that, in our model, the dynamics of the photo-multiplier really only enters 
in so far as the state of the ``dormant'' photo-multiplier,  i.e, the state of the photo-multiplier at the moment when 
an atom leaves the atom source, is orthogonal to its state right after being hit by a ``photon'' emitted 
by an atom on its journey from source to detector. For this reason, states of the photo-multiplier 
do not appear explicitly in our formulae and are therefore not indicated in the following.
 
 According to the $ETH$ - Approach, the time evolution of the initial state, 
 $\Psi_{in}:=\rho(\vec{n}_0)\otimes \big|\emptyset\big> \big< \emptyset\big|$, of the atom and the ``radiation field''
 to the final state when the atom enters the atom detector and a ``photon'' may have been emitted by it is given by 
\begin{equation}\label{full evol}
\Psi_{in} \mapsto \Psi_{out}^{(0)}= \rho(\vec{n}_{out})\otimes \big|\emptyset\big>\big< \emptyset \big|, \quad 
\text{with probability }\quad 1-\mathcal{O}(\alpha),
\end{equation}
where $\vec{n}_{out} \approx (\text{sin}\theta_0\, \text{cos}\varphi_{out}, \text{sin}\theta_0\, \text{sin}\varphi_{out}, 
\text{cos}\theta_0)$; and
\begin{equation}\tag{19}
\Psi_{in}\mapsto \Psi_{out}^{(1)}= \rho(-\vec{e}_3) \otimes \big| \gamma \big>\big< \gamma \big|, \quad 
\text{with probability }\quad \mathcal{O}(\alpha),
\end{equation}
where $\big| \gamma \big>$ is orthogonal to $\big| \emptyset \big>,$ and 
$\rho(-\vec{e}_3)= |\downarrow\rangle \langle \downarrow|$.}
\end{enumerate}

If $\alpha$ is not very small the difference between the atomic out-states in the \textit{absence} of the photo-multiplier 
and in its \textit{presence}, respectively, can be detected by measurements of suitable 
atomic ``observables'' in the atom detector. These measurements can also be described within 
the $ETH$- Approach (see \cite{Fr, FP}).

We think that, when compared to earlier treatments of fluorescence (see, e.g., \cite{Pomeau}), 
the analysis sketched here represents progress.

\section{Concluding remarks}
In this section we offer some conclusions reached from the analysis presented in this paper.
\begin{enumerate}
\item{The \textit{ETH-Approach} to Quantum Mechanics represents a \textit{completion of \textit{QM}} that provides
a logically coherent description of the \textit{stochastic time evolution of states} of \textit{\textbf{individual} 
systems} in $QM$ (unraveling Schr\"odinger-von Neumann evolution) and of \textit{events} and their recordings 
(see \cite{BFS, Froh, Fr}). It has resemblences (albeit rather vague ones) with Everett's ``Many Worlds'' 
formalism \cite{Everett} and spontaneous collapse models \`a la ``GRW'' \cite{GRW}. 
But it supersedes these ad-hoc formalisms by a precise and more natural one. And it describes only 
\textbf{One World}: hopefully ours! Of course, it will have to stand the test of experiments.}
\item{To quote \textit{Pauli} once more: \textit{If speculative ideas cannot be tested, they're not science; 
they don't even rise to the level of being wrong.}
We thus should ask whether the \textit{Principle of Diminishing Potentialities} ($PDP$), which is a 
corner stone of the $ETH$ - Approach to $QM$, is more than a speculative idea and whether 
it can be tested. It is clear that this principle can only be established in quantum theories of systems 
with infinitely many degrees of freedom. It has the status of a \textit{theorem} in local relativistic 
quantum theory with massless particles on even-dimensional space-times; e.g., in 4D 
quantum electrodynamics (QED) \cite{Buchholz}, and in simple models of QED 
regularized at high energies by discretizing time that we sketched in Sect.~4; see \cite{FP} for further 
details. It also holds in models emerging in the limit of the velocity of light tending to $\infty$; see Sect.~5
and \cite{FP}. However, in this limit, the Hamiltonian is not bounded from below; i.e., the spectrum condition 
($\nexists$ negative-energy states) is violated. 

We thus have strong reasons to expect that a completion of $QM$ satisfying the spectrum
condition and solving the \textit{``measurement problem''} will succeed \textit{only} in the guise of
\textit{local relativistic quantum theory} on even-dimensional space-times featuring massless bosons, 
photons and gravitons; (so that ``Huygens Principle'' \cite{Buchholz} holds).\footnote{Besides Huygens' 
Principle there may, however, be further mechanisms implying $PDP$. For example, certain theories 
with extra dimensions may exhibit certain mechanisms implying $PDP$.}}

\item{The $ETH$ - Approach to quantum mechanics sketched above \textit{does have} an extension 
to \textit{local relativistic quantum theory}; (see \cite{Fr} for a preliminary account).
}
\item{A quantum-mechanical analogue of the magic formula \eqref{LSW} for Brownian motion
(see Sect.~1.2) has been proposed by \textit{L\"uders, Schwinger} and \textit{Wigner} 
(see \cite{LSW}). However, when applied to time-ordered series of measurements, 
their formula fails to satisfy an analogue of Eq.~\eqref{Markov}, because the non-commutativity 
of different potential events actualizing at different times leads to \textit{interference effects}. 
Not surprizingly, this has been noticed by many people, who thought of various ways to rescue 
the \textit{L\"uders-Schwinger-Wigner} formula. One formalism seemingly enabling one to come 
up with meaningful predictions that has become quite popular is known under the name of 
\textit{``consistent histories''} \cite{Griffiths, G-M-H}. However, in our opinion, this formalism 
does \textbf{not} represent an acceptable completion of $QM$, because it talks about unpredictable 
and instantaneous interventions by ``observers,'' a feature that extinguishes much of the predictive power 
of $QM$. For, according to standard wisdom in $QM$, a measurement of a physical quantity of a 
system \textit{always} affects the future evolution of its state, \textit{even} if no record of the outcome 
of the measurement has been taken.} 
\item{The $ETH $- Approach to $QM$, in particular $PDP$, introduces a fundamental distinction 
between past and future into the theory: The past consists of \textit{facts}, namely histories of 
\textit{``actualities'',} while the future consists of \textit{``potentialities''} (much in the sense in which
\textit{Aristotle} originally conceived these notions).}
\end{enumerate}

\textit{\underline{Acknowledgements}:} One of us (J.~F.) thanks his collaborators, in particular 
\textit{Baptiste Schubnel,} 
in earlier work on related problems for the pleasure of cooperation, and \textit{Carlo Albert, 
Philippe Blanchard, Shelly Goldstein} and \textit{Erhard Seiler} for their very encouraging 
interest in our efforts.

\begin{thebibliography}{}

\bibitem{Durr} D.~D\"urr and S.~Teufel, \textit{Bohmian Mechanics}, Springer-Verlag, Berlin and Heidelberg, 2009

\bibitem{Kraus} K.~Kraus, \textit{States, Effects, and Operations,} Lecture Notes in Physics, vol. \textbf{190}, 
Springer-Verlag, Berlin, 1983

\bibitem{Lindblad} V.~Gorini, A.~Kossakowski and E.C.G.~Sudarshan, \textit{Completely positive semigroups 
of N-level systems,} J. Math. Phys. \textbf{17}(5), 821 (1976);\\
G.~Lindblad, \textit{On the generators of quantum dynamical semigroups,} Commun. Math. Phys. \textbf{48}, 
119-130 (1976)

\bibitem{Luders} G.~L\"uders, \textit{Über die Zustandsänderung durch den Messprozess,} Ann. Phys. (Leipzig) 
\textbf{443} (5–8), 322-328 (1950)

\bibitem{Wigner} E.P.~Wigner, \textit{Remarks on the mind-body question,} in: ``Symmetries and Reflections,'' 
pp. 171-184, Indiana University Press, Bloomington, 1967

\bibitem{FFS} J.~Faupin, J.~Fr\"ohlich and B.~Schubnel, \textit{On the probabilistic nature of
quantum mechanics and the notion of closed systems.,} Ann. H. Poincaré \textbf{17}, 689-731 (2016)

\bibitem{Everett} H.~D.~Everett III, \textit{Relative state formulation of quantum mechanics,} 
Rev.~Mod.~Phys.~\textbf{29}, 454-462 (1957)
 
\bibitem{Wiki} In: \textit{Interpretations of quantum mechanics}, Wikipedia; see\\
https://en.wikipedia.org/wiki/Interpretations\{hyphen\}of\{hyphen\}quantum\{hyphen\}mechanics

\bibitem{Griffiths} R.~B.~Griffiths, \textit{Consistent histories and the interpretation of quantum mechanics,} 
J.~Stat.~Phys.~\textbf{36} (1), 219–272 (1984)

\bibitem{G-M-H} M.~Gell-Mann and J.~B.~Hartle, \textit{Classical equations for quantum systems,} 
Phys.~Rev.~D \textbf{47} (8), 3345–3382 (1993)

\bibitem{GRW} G.~C.~Ghirardi, A.~Rimini and T.~Weber, \textit{Unified dynamics for 
microscopic and macroscopic systems,} Phys.~Rev.~D \textbf{34}, 470-491 (1986)

\bibitem{Hooft} G.~'tHooft, \textit{The Cellular Automaton Interpretation of Quantum Mechanics}, vol.~\textbf{185} of
``Fundamental Theories of Physics,'' Henk van Beijeren et al.~(eds.), Springer-Verlag, Cham, Heidelberg, New York,
2016

\bibitem{FS} J.~Fr\"ohlich and B.~Schubnel, \textit{Quantum probability theory and the foundations of 
quantum mechanics,} in: Ph.~Blanchard and J.~Fr\"ohlich (eds.), ``The Message of Quantum Science,''
Springer-Verlag, Berlin, 2015

\bibitem{BFS} Ph.~Blanchard, J.~Fr\"ohlich and B.~Schubnel, \textit{A ``Garden of forking paths'' --
the quantum mechanics of histories of events,} Nucl. Phys. B \textbf{912}, 463-484 (2016)

\bibitem{Froh} J.~Fr\"ohlich, \textit{A Brief Review of the ``ETH - Approach to Quantum Mechanics,''} in: N.~Anantharaman, A.~Nikeghbali and M.~Rassias (eds.), ``Frontiers in Analysis and Probability,'' Springer-Verlag, Cham, 2020

\bibitem{FP} J.~Fr\"ohlich and A.~Pizzo, \textit{The Time-Evolution of States in Quantum Mechanics 
according to the ETH-Approach,} Commun.~Math.~Phys. \textbf{389}, 1673-1715 (2022)

\bibitem{Fr} J.~Fr\"ohlich, \textit{Relativistic quantum theory,} in: V.~Allori, A.~Bassi, D.~Dürr and N.~Zanghi (eds.),
``Do Wave Functions Jump? Perspectives of the Work of GianCarlo Ghirardi,'' Fundamental Theories of Physics, 
Springer-Verlag, Cham, 2020; and paper in preparartion

\bibitem{FZ} J.~Fr\"ohlich and Gang Zhou, \textit{On the Evolution of States in a Quantum-Mechanical 
Model of Experiments,} arXiv:2212.02599, Ann.~H.~Poincar\'e (online) (online) 
https://doi.org/10.1007/s00023-023-01292-3

\bibitem{Haag} R.~Haag, \textit{Fundamental irreversibility and the concept of events,} 
Commun. Math. Phys. \textbf{132}, 245-251 (1990); see also:
R.~Haag, \textit{On quantum theory,} Int.~J.Quantum Inf.~\textbf{17},1950037 (2019)

\bibitem{GM} A.~M.~Gleason, \textit{Measures on the closed subspaces of a Hilbert space,}
J.~Math.~Mech.~\textbf{6}, 885-893 (1957);\\
S.~Maeda, \textit{Probability measures on projections in von Neumann algebras,} 
Rev.~Math.~Phys.~\textbf{1}, 235-290 (1989)

\bibitem{Buchholz} D.~Buchholz, \textit{Collision theory for massless particles,} 
Commun.~Math.~Phys.~\textbf{52}, 147-173 (1977); see also: D.~Buchholz and J.~Roberts, 
\textit{New light on infrared problems: sectors, statistics, symmetries and spectrum,}
Commun. Math. Phys. \textbf{330}, 935-972 (2014)

\bibitem{Pomeau} Y.~Pomeau, M.~Le Berre and J.~Ginibre, \textit{Ultimate statistical physics: 
fluorescence of a single atom,} J.~Stat.~Mech.: Theory and Experiment, 104002 (2016) -- Special Issue on
Statphys \textbf{26}

\bibitem{FGS} J.~Fr\"ohlich, M.~Griesemer and B.~Schlein, \textit{Asymptotic completeness for Rayleigh scattering,} 
Ann.~H.~Poincar\'e \textbf{3}, 107-170 (2002)

\bibitem{DeR-K} W.~De Roeck and A.~Kupiainen, \textit{Approach to Ground State and 
Time-Independent Photon Bound for Massless Spin-Boson Models,} Ann.~H.~Poincar\'e \textbf{14}, 253-311 (2013)

\bibitem{LSW} G.~L\"uders, see \cite{Luders};\\
J.~Schwinger, \textit{The algebra of microscopic measurement,} Proc.~Natl.~Acad.~Sci.~(USA)
\textbf{45}(10), 1542-1553 (1959);\\
E.~P.~Wigner, in: ``The Collected Works of Eugene Paul Wigner,'' Springer-Verlag, Berlin, 1993

\end{thebibliography}

\noindent
1 Institut f\"ur theoretische Physik, ETH-Zurich, 8093 Zurich, Switzerland. Email: juerg@phys.ethz.ch\\
2 Binghamton University, Department of Mathematics and Statistics. Email: gangzhou@binghamton.edu \\
3 Dipartimento di Matematica, Università di Roma ``TorVergata,'' Rome, Italy. Email: pizzo@mat.uniroma2.it


\end{document}

