%%
%% Beginning of file 'sample62.tex'
%%
%% Modified 2018 January
%%
%% This is a sample manuscript marked up using the
%% AASTeX v6.2 LaTeX 2e macros.
%%
%% AASTeX is now based on Alexey Vikhlinin's emulateapj.cls 
%% (Copyright 2000-2015).  See the classfile for details.

%% AASTeX requires revtex4-1.cls (http://publish.aps.org/revtex4/) and
%% other external packages (latexsym, graphicx, amssymb, longtable, and epsf).
%% All of these external packages should already be present in the modern TeX 
%% distributions.  If not they can also be obtained at www.ctan.org.

%% The first piece of markup in an AASTeX v6.x document is the \documentclass
%% command. LaTeX will ignore any data that comes before this command. The 
%% documentclass can take an optional argument to modify the output style.
%% The command below calls the preprint style  which will produce a tightly 
%% typeset, one-column, single-spaced document.  It is the default and thus
%% does not need to be explicitly stated.
%%
%%
%% using aastex version 6.2
\documentclass[twocolumn]{aastex62}

%% The default is a single spaced, 10 point font, single spaced article.
%% There are 5 other style options available via an optional argument. They
%% can be envoked like this:
%%
%% \documentclass[argument]{aastex62}
%% 
%% where the layout options are:
%%
%%  twocolumn   : two text columns, 10 point font, single spaced article.
%%                This is the most compact and represent the final published
%%                derived PDF copy of the accepted manuscript from the publisher
%%  manuscript  : onezw text column, 12 point font, double spaced article.
%%  preprint    : one text column, 12 point font, single spaced article.  
%%  preprint2   : two text columns, 12 point font, single spaced article.
%%  modern      : a stylish, single text column, 12 point font, article with
%% 		  wider left and right margins. This uses the Daniel
%% 		  Foreman-Mackey and David Hogg design.
%%  RNAAS       : Preferred style for Research Notes which are by design 
%%                lacking an abstract and brief. DO NOT use \begin{abstract}
%%                and \end{abstract} with this style.
%%
%% Note that you can submit to the AAS Journals in any of these 6 styles.
%%
%% There are other optional arguments one can envoke to allow other stylistic
%% actions. The available options are:
%%
%%  astrosymb    : Loads Astrosymb font and define \astrocommands. 
%%  tighten      : Makes baselineskip slightly smaller, only works with 
%%                 the twocolumn substyle.
%%  times        : uses times font instead of the default
%%  linenumbers  : turn on lineno package.
%%  trackchanges : required to see the revision mark up and print its output
%%  longauthor   : Do not use the more compressed footnote style (default) for 
%%                 the author/collaboration/affiliations. Instead print all
%%                 affiliation information after each name. Creates a much
%%                 long author list but may be desirable for short author papers
%%
%% these can be used in any combination, e.g.
%%
%% \documentclass[twocolumn,linenumbers,trackchanges]{aastex62}
%%
%% AASTeX v6.* now includes \hyperref support. While we have built in specific
%% defaults into the classfile you can manually override them with the
%% \hypersetup command. For example,
%%
%%\hypersetup{linkcolor=red,citecolor=green,filecolor=cyan,urlcolor=magenta}
%%
%% will change the color of the internal links to red, the links to the
%% bibliography to green, the file links to cyan, and the external links to
%% magenta. Additional information on \hyperref options can be found here:
%% https://www.tug.org/applications/hyperref/manual.html#x1-40003
%%
%% If you want to create your own macros, you can do so
%% using \newcommand. Your macros should appear before
%% the \begin{document} command.
%%
\newcommand{\vdag}{(v)^\dagger}
\newcommand\aastex{AAS\TeX}
\newcommand\latex{La\TeX}

%%%%%%%%%%%%%%%%%%%%%%%%%%%%%%%%%%%%%%%%%%%%%%%%%%%%%%%%%%%%%
%
% Add new commands
%
%%%%%%%%%%%%%%%%%%%%%%%%%%%%%%%%%%%%%%%%%%%%%%%%%%%%%%%%%%%%%

\newcommand{\Msun}{M_{\odot}} 
\newcommand{\Rsun}{R_{\odot}} 
\newcommand{\Lsun}{L_{\odot}}
\newcommand{\cmc}{\mathrm{cm}^{-3}}
\newcommand{\mdotsunyr}{M_\odot \mathrm{yr}^{-1}}

%% Tells LaTeX to search for image files in the 
%% current directory as well as in the figures/ folder.
\graphicspath{{./}{figures/}}


%% Reintroduced the \received and \accepted commands from AASTeX v5.2
%\received{January 1, 2018}
%\revised{January 7, 2018}
%\accepted{\today}
\received{XXX}
\revised{YYY}
\accepted{ZZZ}

%% Command to document which AAS Journal the manuscript was submitted to.
%% Adds "Submitted to " the arguement.
\submitjournal{ApJL}

%% Mark up commands to limit the number of authors on the front page.
%% Note that in AASTeX v6.2 a \collaboration call (see below) counts as
%% an author in this case.
%
%\AuthorCollaborationLimit=3
%
%% Will only show Schwarz, Muench and "the AAS Journals Data Scientist 
%% collaboration" on the front page of this example manuscript.
%%
%% Note that all of the author will be shown in the published article.
%% This feature is meant to be used prior to acceptance to make the
%% front end of a long author article more manageable. Please do not use
%% this functionality for manuscripts with less than 20 authors. Conversely,
%% please do use this when the number of authors exceeds 40.
%%
%% Use \allauthors at the manuscript end to show the full author list.
%% This command should only be used with \AuthorCollaborationLimit is used.

%% The following command can be used to set the latex table counters.  It
%% is needed in this document because it uses a mix of latex tabular and
%% AASTeX deluxetables.  In general it should not be needed.
%\setcounter{table}{1}

%%%%%%%%%%%%%%%%%%%%%%%%%%%%%%%%%%%%%%%%%%%%%%%%%%%%%%%%%%%%%%%%%%%%%%%%%%%%%%%%
%%
%% The following section outlines numerous optional output that
%% can be displayed in the front matter or as running meta-data.
%%
%% If you wish, you may supply running head information, although
%% this information may be modified by the editorial offices.
\shorttitle{3D RHD simulations resolving protostellar interior}
\shortauthors{K. Kimura et al.}
%%
%% You can add a light gray and diagonal water-mark to the first page 
%% with this command:
% \watermark{text}
%% where "text", e.g. DRAFT, is the text to appear.  If the text is 
%% long you can control the water-mark size with:
%  \setwatermarkfontsize{dimension}
%% where dimension is any recognized LaTeX dimension, e.g. pt, in, etc.
%%
%%%%%%%%%%%%%%%%%%%%%%%%%%%%%%%%%%%%%%%%%%%%%%%%%%%%%%%%%%%%%%%%%%%%%%%%%%%%%%%%

%% This is the end of the preamble.  Indicate the beginning of the
%% manuscript itself with \begin{document}.

\begin{document}

\title{3D Radiation-Hydrodynamic Simulations Resolving Interior of Rapidly Accreting Primordial Protostar}

%% LaTeX will automatically break titles if they run longer than
%% one line. However, you may use \\ to force a line break if
%% you desire. In v6.2 you can include a footnote in the title.

%% A significant change from earlier AASTEX versions is in the structure for 
%% calling author and affilations. The change was necessary to implement 
%% autoindexing of affilations which prior was a manual process that could 
%% easily be tedious in large author manuscripts.
%%
%% The \author command is the same as before except it now takes an optional
%% arguement which is the 16 digit ORCID. The syntax is:
%% \author[xxxx-xxxx-xxxx-xxxx]{Author Name}
%%
%% This will hyperlink the author name to the author's ORCID page. Note that
%% during compilation, LaTeX will do some limited checking of the format of
%% the ID to make sure it is valid.
%%
%% Use \affiliation for affiliation information. The old \affil is now aliased
%% to \affiliation. AASTeX v6.2 will automatically index these in the header.
%% When a duplicate is found its index will be the same as its previous entry.
%%
%% Note that \altaffilmark and \altaffiltext have been removed and thus 
%% can not be used to document secondary affiliations. If they are used latex
%% will issue a specific error message and quit. Please use multiple 
%% \affiliation calls for to document more than one affiliation.
%%
%% The new \altaffiliation can be used to indicate some secondary information
%% such as fellowships. This command produces a non-numeric footnote that is
%% set away from the numeric \affiliation footnotes.  NOTE that if an
%% \altaffiliation command is used it must come BEFORE the \affiliation call,
%% right after the \author command, in order to place the footnotes in
%% the proper location.
%%
%% Use \email to set provide email addresses. Each \email will appear on its
%% own line so you can put multiple email address in one \email call. A new
%% \correspondingauthor command is available in V6.2 to identify the
%% corresponding author of the manuscript. It is the author's responsibility
%% to make sure this name is also in the author list.
%%
%% While authors can be grouped inside the same \author and \affiliation
%% commands it is better to have a single author for each. This allows for
%% one to exploit all the new benefits and should make book-keeping easier.
%%
%% If done correctly the peer review system will be able to
%% automatically put the author and affiliation information from the manuscript
%% and save the corresponding author the trouble of entering it by hand.

\correspondingauthor{Kazutaka Kimura}
\email{kazutaka.kimura@yukawa.kyoto-u.ac.jp}

\author[0000-0001-8382-3966]{Kazutaka Kimura}
\affil{Center for Gravitational Physics and Quantum Information, \\
Yukawa Institute for Theoretical Physics, Kyoto University, Kyoto, 606-8502, Japan}

\author[0000-0003-3127-5982]{Takashi Hosokawa}
\affiliation{Department of Physics, Graduate School of Science, Kyoto University, Sakyo, Kyoto 606-8502, Japan}
\author[0000-0001-7842-5488]{Kazuyuki Sugimura}
\affiliation{Department of Physics, Graduate School of Science, Kyoto University, Sakyo, Kyoto 606-8502, Japan}
\affiliation{The Hakubi Center for Advanced Research, Kyoto University, Sakyo, Kyoto 606-8501, Japan}
\affiliation{Faculty of Science, Hokkaido University, Sapporo, Hokkaido 060-0810, Japan}

\author[0000-0002-0547-3208]{Hajime Fukushima}
\affiliation{Center for Computational Sciences, University of Tsukuba, Ten-nodai, 1-1-1 Tsukuba, Ibaraki 305-8577, Japan}

%% Note that the \and command from previous versions of AASTeX is now
%% depreciated in this version as it is no longer necessary. AASTeX 
%% automatically takes care of all commas and "and"s between authors names.

%% AASTeX 6.2 has the new \collaboration and \nocollaboration commands to
%% provide the collaboration status of a group of authors. These commands 
%% can be used either before or after the list of corresponding authors. The
%% argument for \collaboration is the collaboration identifier. Authors are
%% encouraged to surround collaboration identifiers with ()s. The 
%% \nocollaboration command takes no argument and exists to indicate that
%% the nearby authors are not part of surrounding collaborations.

%%%%%%%%%%%%%%%%%%%%%%%%%%%%%%%%%%%%%%%%%%%%%%%%%%%%%%%%%%%%%%%
%
% Abstract
%
%%%%%%%%%%%%%%%%%%%%%%%%%%%%%%%%%%%%%%%%%%%%%%%%%%%%%%%%%%%%%%%
%% Mark off the abstract in the ``abstract'' environment. 
\begin{abstract}
Direct collapse of supermassive stars is a possible pathway to form supermassive black hole seeds at high redshifts. Whereas previous three-dimensional (3D) simulations demonstrate that supermassive stars form via rapid mass accretion, those resolving the stellar interior have been limited. We here report 3D radiation-hydrodynamic (RHD) simulations following the evolution of rapidly accreting protostars resolving the stellar interior. We use an adaptive mesh refinement code with our newly developed RHD solver employing an explicit M1 closure method. 
We follow the early evolution until the stellar mass reaches $\sim 10~\Msun$ from two different initial configurations of spherical and turbulent clouds. 
We demonstrate that, in both the cases, a swollen protostar whose radius is $100-1000~\Rsun$ appears, as predicted by the stellar evolution calculations. Its effective temperature remains a few thousand Kelvin, and the radiative feedback by ionizing photons is too weak to disturb the accretion flow up to the epoch examined in this work. In the turbulent case, the protostar rotates rapidly at more than 0.4 times the Keplerian velocity owing to the angular momentum provided by the initial turbulence. 
The protostar approximates an oblate spheroid, and its equatorial radius is more than twice the polar radius. Our results suggest that we need to consider the rapid stellar rotation to elucidate the realistic 3D protostellar evolution in the supermassive star formation.
\end{abstract}

%% Keywords should appear after the \end{abstract} command. 
%% See the online documentation for the full list of available subject
%% keywords and the rules for their use.
\keywords{early universe --- stars: first stars, Population III --- stars: formation --- accretion, accretion disks}

%% From the front matter, we move on to the body of the paper.
%% Sections are demarcated by \section and \subsection, respectively.
%% Observe the use of the LaTeX \label
%% command after the \subsection to give a symbolic KEY to the
%% subsection for cross-referencing in a \ref command.
%% You can use LaTeX's \ref and \label commands to keep track of
%% cross-references to sections, equations, tables, and figures.
%% That way, if you change the order of any elements, LaTeX will
%% automatically renumber them.
%%
%% We recommend that authors also use the natbib \citep
%% and \citet commands to identify citations.  The citations are
%% tied to the reference list via symbolic KEYs. The KEY corresponds
%% to the KEY in the \bibitem in the reference list below. 


%%%%%%%%%%%%%%%%%%%%%%%%%%%%%%%%%%%%%%%%%%%%%%%%%%%%%%%%%%%%%%%
%
% Introduction
%
%%%%%%%%%%%%%%%%%%%%%%%%%%%%%%%%%%%%%%%%%%%%%%%%%%%%%%%%%%%%%%%
\section{Introduction} \label{sec:intro}
\par
Many supermassive black holes (SMBHs) exceeding $10^9~\Msun$ exist in the high-redshift universe ($z\ge6$), and their possible origins have been intensively studied \citep[see][for reviews]
{Volonteri_2010,Woods_et_al_2019,Inayoshi_et_al_2020}.
%\par
One possible pathway to form such SMBHs is the direct collapse (DC) scenario, which starts with the primordial gas cloud collapsing via H atomic cooling \citep{Bromm_and_Loeb_2003}.
In this case, the cloud maintains high temperatures ($\gtrsim 3000$~K), and protostars grow in mass with high accretion rates of $\sim 0.1\mathrm{-}1~\mdotsunyr$ \citep[e.g.][]{Omukai_2001,Latif_et_al_2013,Chon_et_al_2018}. If such rapid accretion continues for $\sim 1$~Myr, protostars evolve into supermassive stars (SMSs) with masses of $\sim 10^5\mathrm{-}10^6~\Msun$.
Finally, they collapse into BHs with the same masses due to general relativistic instability \citep{Zeldovich_and_Novikov_1971,Shapiro_and_Teukolsky_1983,Shibata_and_Shapiro_2002}. Such BH seeds can grow into the high-redshift SMBHs even with sub-Eddington accretion rates.
\par
Evolution of the protostellar structure is crucial in this scenario. If a protostar emits a copious amount of ionizing photons,
the strong radiative feedback may hinder the mass accretion \citep[e.g.][]{McKee_and_Tan_2008,Hosokawa_et_al_2011,Sugimura_et_al_2020} before the formation of a SMS. Stellar evolution calculations predict that rapidly accreting massive protostars should have very large radii and low effective temperatures \citep{Hosokawa_et_al_2012, Hosokawa_et_al_2013,Schleicher_et_al_2013,Umeda_et_al_2016,Haemmerle_et_al_2018a}. As a result, their ionizing emissivities remain very low, even if the luminosities approach the Eddington limit.
\par
One-dimensional (1D) stellar evolution calculations are a major methodology in studying the protostellar evolution. 
However, 1D stellar models suffer from limitations. For instance, it normally assumes the hydrostatic balance for the stellar materials and steady accretion flow for the surroundings. Furthermore, realistic three-dimensional (3D) structure, such as stellar rotation provided by angular momentum carried by the accretion flow, is difficult to be modeled. The protostellar rotation should affect the evolution as it induces efficient chemical mixing in the interior. With fast rotation, moreover, the centrifugal force may limit mass accretion through a surrounding disk via the so-called $\Omega\Gamma$-limit \citep{Lee+Yoon2016,Takahashi_and_Omukai_2017,Haemmerle_et_al_2018b}. The actual protostellar evolution and resulting feedback may differ from predictions by previous 1D calculations. 
\par
To elucidate the realistic 3D protostellar evolution, we need to perform three-dimensional radiation hydrodynamic (RHD) simulations.
\citet{Luo_et_al_2018} report 3D RHD simulations following an early evolution in the SMS formation with the flux-limited diffusion approximation, resolving the stellar interior structure. They show that a protostar is dissolved by the growing pressure, which differs from the previous 1D calculations. We need further study to address how generally such evolution occurs and what 3D RHD effects may cause it.
\par
In this work, we perform 3D simulations using a newly developed RHD solver with an explicit M1 closure scheme, resolving the protostellar interior. We present simulations starting from two initial configurations: spherical and turbulent clouds. We study the idealized spherical case for comparisons to the 1D calculations and the turbulent case for revealing realistic protostellar evolution.
We follow the early evolution of the SMS formation until a protostar accretes the gas of $\sim 10~\Msun$.

%%%%%%%%%%%%%%%%%%%%%%%%%%%%%%%%%%%%%%%%%%%%%%%%%%%%%%%%%%%%%%%
%
% Method
%
%%%%%%%%%%%%%%%%%%%%%%%%%%%%%%%%%%%%%%%%%%%%%%%%%%%%%%%%%%%%%%%
\section{Numerical Method}
\subsection{Radiation Hydrodynamics Code}
\par
We use the self-gravitational magneto-hydrodynamics code with adaptive mesh refinement, SFUMATO-RT \citep{Matsumoto_2007, Matsumoto_et_ak_2015, Sugimura_et_al_2020}.
SFUMATO-RT includes the primordial gas chemistry module, which applies only to a regime where the density is lower than $10^{13}~\cmc$. We add chemical reactions that are relevant for a high-density medium \citep[see also][]{Sadanari_et_al_2021}. 
Moreover, we implement the newly developed radiation solver with an explicit M1 closure scheme applicable even for the optically thick regime \citep{Pomraning_1969,Kershaw_1976,Levermore_1984,Rosdahl_and_Teyssier_2015,Fukushima_and_Yajima_2021}.
We do not use the ray-tracing radiation solver implemented by \citet{Sugimura_et_al_2020} in this work. Our radiation solver will be separately described in detail in a forthcoming paper (K. Kimura et al., in preparation).
\par
As for the chemistry network, we solve the non-equilibrium reactions among six components, $\mathrm{H}$, $\mathrm{H}_2$, $\mathrm{H}^+$, $\mathrm{H}^-$, $\mathrm{H}_2^+$, and $\mathrm{e}$ as in \citet{Sugimura_et_al_2020}. We have implemented chemical reactions effective in the dense medium with $n_\mathrm{H} \gtrsim 10^{13}~\cmc$ following \citet{Omukai_2001}. To follow the coupled evolution of the gas and radiation in the optically thick medium, we incorporate the photon emission and absorption processes into our chemical network. We simultaneously update the gas temperature, chemical abundances, and radiation fields using the implicit method, to solve their evolution consistently.
\par
We develop a new explicit M1 closure scheme based on \citet{Rosdahl_and_Teyssier_2015}.
Their method is applicable to both the optically thin and thick regions, particularly for cases where the optical depth over a single grid cell greatly exceeds unity. We have modified their scheme to handle a situation where the optical thickness abruptly changes across a few cells. This occurs near the protostellar surface in our simulations.
\par
We consider two frequency bins in the radiation solver: low energy ($h\nu<13.6~\mathrm{eV}$) and extreme-ultraviolet (EUV; $h\nu>13.6~\mathrm{eV}$).
In the SMS formation, the emission of the low-energy radiation is crucial for the cooling processes of the collapsing gas.
Meanwhile, the protostar emits a large amount of EUV photons, which ionize the surrounding gas and suppress the gas accretion, if the protostellar effective temperature rises to $\sim10^5$~K.
Taking these two frequency bins is thus necessary to study the SMS formation.
Furthermore, for each frequency bin, we separately solve the photon energy and number densities, whose ratio provides the photon mean energy. We locally reconstruct a consistent spectral distribution within each bin, assuming the Planck distribution with the temperature corresponding to the mean energy. This method improves our approximate treatment with only two bins, allowing us to follow the radiation physics such as heating by absorption accurately. Note that the luminosity shown in Section~\ref{sec:Results} is the total of low-energy and EUV components.
\subsection{Initial Conditions and Settings}
\par
One initial condition we consider is the ``spherical'' case, where we start with an isothermal Bonner-Ebert sphere \citep{Bonnar_1956} with the central density $10^9~\cmc$ and temperature $4.8 \times 10^3~\mathrm{K}$. These values are realized on the SMS-evolution path given by the one-zone model of the cloud collapse \citep{Omukai_2001}. We increase the density profile by a factor of $1.6$ to ensure the collapse.
\par
The other initial condition is the ``turbulent'' case, where we start from the same cloud as above but with additional turbulent velocity fields. We assume the power spectrum of the turbulence obeying Larson's law, $P(k) \propto k^{-4} $ \citep{Larson_1981}. We set its Mach number as unity, considering recent studies showing that the turbulence grows up to this level during the collapse stage \citep{Federrath_et_al_2011, Higashi_et_al_2021}.
\par
We set the side length of the computational domain as $0.09\,\mathrm{pc}$, 3.3 times larger than the radius of the Bonner-Ebert sphere.
The cell number of the base grid is 40 in each direction, and we adaptively refine the grids to resolve the Jeans length with 8 cells. 
The corresponding minimum cell size is $3.4 \times 10^{-3}\,\mathrm{AU}$ in the spherical case and $6.9 \times 10^{-2}\,\mathrm{AU}$ in the turbulent case.
We prohibit the grid de-refinement because it artificially increases the entropy in the stellar interior.
\par
In our explicit implementation of the M1 closure scheme, we adopt the reduced speed of light approximation to increase time steps restricted by the Courant-Friedrichs-Lewy condition \citep{Gnedin_and_Abel_2001,Skinner_and_Ostriker_2013}. We set the reduced light speed $\tilde{c}$ as $10^{-3}c$, where $c$ is the speed of light.
\par
The appearance of a standing accretion shock marks the epoch of the protostellar birth \citep{Omukai_and_Nishi_1998}. We define the stellar radius as the position at which the radial velocity falls below the sound speed and the photospheric radius as where the optical depth measured from the outside exceeds unity. The photospheric radius often becomes larger than the stellar radius, and the region between them is called a radiative precursor \citep{Stahler_et_al_1980}. We regard the total mass inside the stellar surface as the protostellar mass.
\par
In both the cases, we follow the evolution for $\sim 10$~yr after the birth of an embryonic protostar. The protostar accretes the gas of $\sim 10~\Msun$ by the end of the simulations with the mean accretion rate of $\sim 1~\mdotsunyr$.

%%%%%%%%%%%%%%%%%%%%%%%%%%%%%%%%%%%%%%%%%%%%%%%%%%%%%%%%%%%%%%%
%
% Result
%
%%%%%%%%%%%%%%%%%%%%%%%%%%%%%%%%%%%%%%%%%%%%%%%%%%%%%%%%%%%%%%%
\section{Results} \label{sec:Results}
\par
Figure~\ref{fig:Snapshot} shows the final snapshots for the spherical and turbulent cases. An accreting protostar with $M_* \sim 10~\Msun$ appears for both cases, and it has a large radius of $\sim 10^3~\Rsun$. Nonetheless, there are significant differences between these cases. 
The turbulent case especially shows that the protostar rotates and has a flattened shape. We describe the protostellar structure and evolution for each case in detail below.
%%% FIG.1 %%%
\begin{figure*}
\begin{center}
  \includegraphics[width=\linewidth]{Snapshot.png}
  \caption{Snapshots at the end of the simulation. 
           The left column corresponds to the spherical case. The middle and right columns show the face-on and edge-on slices in the turbulent case.
           The top, middle, and bottom panels display the distributions of the density, temperature, and isotropic luminosity $L_{\rm iso} \equiv 4\pi r^2 F_r$, where $F_r$ is the local outward radiation flux. 
           The isotropic luminosity represents the flux intensity corrected for the geometrical dilution. In the top panel, arrows represent velocity distributions. 
           In the bottom panel, arrows only represent directions of the radiation flux.
           The white ellipses approximately delineate the protostellar surfaces. 
           }
  \label{fig:Snapshot}
\end{center}
\end{figure*}
\subsection{Spherical Case}
\par
In this case, the spherical symmetry is kept throughout the evolution as suggested by Figure~\ref{fig:Snapshot}. Figure~\ref{fig:Sphe_StellarEvol} shows the protostellar evolution, where a protostar first appears at $M_* \sim 0.1~\Msun$. The top panel shows at this epoch the stellar radius is $\simeq 10~\Rsun$, much smaller than the photospheric radius $\simeq 300~\Rsun$. The protostellar radius is comparable to the Jeans length when the collapsing gas becomes adiabatic at $n_\mathrm{H} \sim 10^{20}~\cmc$ \citep{Omukai_2001}, and the photospheric radius is determined by the envelope structure created during the collapse.
As the accretion proceeds, only the protostellar radius grows and approaches the photospheric radius. These radii increase in tandem for $M_* \gtrsim 3~\Msun$, after the stellar radius exceeds $\sim 100~\Rsun$. The protostellar and photospheric radii at this stage agree with the 1D stellar evolution calculation taken from \citet{Hosokawa_et_al_2012}, for which the steady spherical accretion at a constant rate $1~\mdotsunyr$ was assumed.
Note that the 1D results are not available for $M_* \lesssim 2.5~\Msun$, because of numerical difficulties in constructing models. \citet{Hosokawa_et_al_2012} adopt an arbitrary $M_* \simeq 2.5~\Msun$ initial model with the radius $R_* \simeq 298.4~\Rsun$ instead. In contrast, our 3D simulation consistently follows the early evolution from the birth of the protostar. 
%%% FIG.2 %%%
\begin{figure}
\begin{center}
  \includegraphics[width=\linewidth]{Sphe_StellarEvol.png}
  \caption{Stellar evolution in the spherical case. Panels show the evolution of (a) protostellar radius $R_*$ and photospheric radius $R_\mathrm{phs}$, (b) photospheric luminosity $L_\mathrm{phs}$, 
and (c) effective temperature $T_\mathrm{eff} \equiv (L_\mathrm{phs}/(4\pi R_\mathrm{phs}^2 \sigma_\mathrm{SB}))^{1/4}$, where $\sigma_\mathrm{SB}$ is the Stefan-Boltzmann constant, against the stellar mass $M_*$.
In all the panels, the blue and orange colors represent our 3D simulation and the 1D stellar evolution calculation at the constant accretion rate $1~\mdotsunyr$ taken from \citet{Hosokawa_et_al_2012}.
The open and filled circles denote the actual values evaluated from the simulation data. In panel (a), the solid and dashed lines represent the protostellar and photospheric radii, respectively. The shaded regions correspond to the radiative precursor. In panel (b), the dashed line indicates the Eddington luminosity. The vertical cyan and magenta lines mark the epochs whose radial profiles are presented in Figure \ref{fig:Sphe_profile}.
           }
  \label{fig:Sphe_StellarEvol}
\end{center}
\end{figure}
%
\par
The middle panel in Figure~\ref{fig:Sphe_StellarEvol} shows the evolution of the luminosity at the photosphere.
In the early stage for $M_* \lesssim 2~\Msun$, the radiative precursor near the photosphere has the same structure as that created during the collapse. As a result, the luminosity remains almost constant at $\simeq 3 \times 10^3~\Lsun$.
Later for $M_* \gtrsim 2~\Msun$, the photosphere expands, and the luminosity increases to reach the Eddington values. However, the radiative feedback is ineffective for the accreting gas because it is neutral and the opacity is lower than that of Thomson scattering.
The lower panel shows the evolution of the effective temperature $T_\mathrm{eff}$. We see that $T_\mathrm{eff}$ takes low values of a few $\times 10^3$~K in spite of the large luminosities because of the large radii. These features agree with the 1D calculation. 
%%% FIG.3 %%%
\begin{figure}
\begin{center}
  \includegraphics[width=\linewidth]{Sphe_profile.png}
  \caption{Spherically averaged radial profiles of physical quantities in the spherical case. The colors represent different epochs when the stellar mass is $0.2~\Msun$ (cyan) and $10.0~\Msun$ (magenta). The filled and open circles denote the protostellar and photospheric radii (see the top panel in Fig.~\ref{fig:Sphe_StellarEvol}). Panels show the distributions of (a) gas number density, (b) temperature, (c) radial velocity, and (d) luminosity against the distance from the stellar center. In panels (c) and (d), the solid and dashed lines correspond to the outward and inward flux, respectively.
}
\label{fig:Sphe_profile}
\end{center}
\end{figure}
\par
Figure~\ref{fig:Sphe_profile} shows the spherically averaged radial profiles of physical quantities when the stellar mass is $0.2~\Msun$ and $10.0~\Msun$. Panels (a)--(c) show the presence of an accretion shock at the stellar surface, where the infall velocity drops and both the density and temperature abruptly rise instead. 
Throughout the evolution, the stellar interior is fully radiative, or convectively stable, as expected by the 1D calculation for the early ``adiabatic accretion'' stage \citep[e.g.][]{Hosokawa_and_Omukai_2009}. 
\par
Panel (d) shows that the luminosity in the deep stellar interior is small due to slow diffusion. Each profile has the local maximum near the surface owing to the large temperature gradient created by the shock heating. 
When $M_*=0.2~\Msun$, the luminosity increases outward in the radiative precursor since the temperature decreases and the $\mathrm{H}^-$ opacity consequently decreases outward. Moreover, at this time, the luminosity slightly increases further outward from the photosphere.
This is because the $\mathrm{H}^-$ free-bound emission, the dominant cooling process for the envelope gas, produces low energy ($h\nu < 13.6$~eV) radiation and contributes to the luminosity.
\subsection{Turbulent Case}
%%% FIG.4 %%%
\begin{figure}
\begin{center}
  \includegraphics[width=\linewidth]{Turb_StellarEvol.png}
  \caption{Stellar evolution in the turbulent case. Panels (a) and (c) show the same quantities as in Figure~\ref{fig:Sphe_StellarEvol}, and Panel (b) presents the isotropic luminosity.
  The blue and green circles show the quantities in the equatorial and polar directions. 
  In panel (a), the equatorial radius (blue) indicates that averaged over the equatorial plane. The polar radius (green) is the minimum value on the polar axis, i.e., whether in the positive or negative directions. We use this to avoid a temporal dense structure created by the turbulence and resulting spurious large radius. Note that these equatorial and polar radii define the ellipses shown in Figure~\ref{fig:Snapshot}. In panels (b) and (c), the equatorial and polar values correspond to those averaged over the region within $\pi/4$ radians from the equatorial plane and polar axis.
  The vertical dashed line indicates the epoch whose snapshots are shown in Figures \ref{fig:Snapshot} and \ref{fig:Turb_profile}.
  }
  \label{fig:Turb_StellarEvol}
\end{center}
\end{figure}
%%% FIG.5 %%%
\begin{figure}
\begin{center}
  \includegraphics[width=\linewidth]{Turb_profile.png}  
  \caption{Radial profiles of physical quantities at the end of the turbulent case. Panels (a)--(d) show the same quantities as in Figure \ref{fig:Sphe_profile}. The colors represent the averaged radial profiles over the region within $\pi$/4 radians from the equatorial plane (blue) and polar axis (green). The filled and open circles denote the protostellar and photospheric radii. Panel (e) shows the spherically averaged angular velocities compared to the Keplerian values.
}
  \label{fig:Turb_profile}
\end{center}
\end{figure}
\par
We show the protostellar evolution in Figure \ref{fig:Turb_StellarEvol}. 
As shown in the middle and right columns of Figure~\ref{fig:Snapshot}, the protostar has a complicated shape deviated from the spherical symmetry. Here, we show the evolution in equatorial and polar directions separately. Panel (a) illustrates that the stellar and photospheric radii increase with the increasing stellar mass. The stellar and photospheric radii are of the same order of magnitude as the 1D calculation for $M_* \gtrsim 3~\Msun$. 
Meanwhile, the equatorial radii are more than twice as large as the polar ones because of the rotation. The protostar rotates so fast that the centrifugal force is comparable to the pressure gradient force near the surface in the equatorial direction, affecting the equilibrium shape, as shown below.
\par
Panel (b) presents that, at the end of the simulation, the luminosity is about the same as the 1D calculation and near the Eddington value.
At this time, the radiative feedback is ineffective because the surrounding gas is neutral as in the spherical case. Whereas the radiative flux is anisotropic depending on the accreting gas structure (Figure~\ref{fig:Snapshot}), the averaged luminosity over the equatorial and polar regions is almost the same.
Panel (c) shows that the effective temperature is always a few thousand Kelvin and agrees with the 1D calculation. The resulting UV feedback is too weak to disturb the rapid accretion onto the protostar.
\par
The middle and right columns of Figure~\ref{fig:Snapshot} show the snapshots at the end of the simulation when $M_*=11.8~\Msun$.
While the gas structure around the protostar is complicated due to the turbulence, the protostar rotates and has a flattened shape. This is because the accreting gas brings into the protostar the angular momentum of the initial turbulence.
A circumstellar disk is about to emerge at this time. 
In our current definition of the stellar surface, which compares the radial velocity to the sound velocity, the white ellipse approximates a protostar as shown in Figure~\ref{fig:Snapshot}.
Nevertheless, even inside this ellipse, it is the centrifugal force rather than the pressure gradient that is balanced by gravity in an outer part near the equatorial plane. From this point of view, such a part can be interpreted as a circumstellar disk.
The middle panels show the stellar interior is much hotter than the surrounding gas, which is nearly constant at several thousands of Kelvin with some fluctuations.
Moreover, as seen in the top and bottom panels, $L_\mathrm{iso}$ is anti-correlated with the density, indicating that the stellar radiation predominantly escapes through low-density regions.
\par
Figure \ref{fig:Turb_profile} shows the radial profiles of physical quantities at the end of the simulation. 
As seen in panel (a), the equatorial density is basically larger than the polar one because of the rotation.
Panels (b) and (c) show that the radial infall velocity abruptly decreases and the temperature rises after the shock front. 
\par
Panel (d) shows similar $L_\mathrm{iso}$ profiles, regardless of the equatorial and polar directions. The luminosity is very low owing to the high optical depth in the deep stellar interior. 
Outside the protostar, the luminosity is sufficiently large that, unlike the case with $M_*=0.2~\Msun$ in Figure~\ref{fig:Sphe_profile}, the emissivity of the infalling gas hardly enhances the luminosity.
\par
Panel (e) shows the spherically averaged angular velocity distribution. For comparison, the dashed line also shows the Keplerian angular velocity $\Omega_\mathrm{Kep}=\sqrt{GM_r/r}$, where $M_r$ is the enclosed mass within the radius $r$.
The angular velocity is almost constant in the stellar interior, which means rigid rotation. The protostar rotates rapidly at more than 0.4 times the Keplerian velocity at any given radius. Especially near the surface at $r \simeq 5$~AU, the rotational velocity is very close to the Keplerian value. 

%%%%%%%%%%%%%%%%%%%%%%%%%%%%%%%%%%%%%%%%%%%%%%%%%%%%%%%%%%%%%%%
%
% Discussion
%
%%%%%%%%%%%%%%%%%%%%%%%%%%%%%%%%%%%%%%%%%%%%%%%%%%%%%%%%%%%%%%%
\section{Discussions} \label{sec:discussion}
In this letter, we have studied the protostellar evolution under very rapid accretion supposed for the DC, performing 3D RHD simulations resolving the stellar interior structure.
We have used the AMR code SFUMATO-RT, in which we have implemented a newly-developed RHD solver employing an explicit M1 closure method. We follow the evolution for about 10~yr after the protostar formation, during which the stellar mass increases to $\sim 10~\Msun$.
\par
In one case starting from a spherical cloud, a swollen protostar forms as expected by previous 1D stellar evolution calculations. 
At the end of the simulation when $M_* \sim 10~\Msun$, the stellar radius is $\simeq 1000~\Rsun$, the luminosity is $\sim10^5~\Lsun$, and the effective temperature is $\simeq 4000$~K.
Such protostars with the cool atmosphere emit a negligible amount of ionizing photons, and the resulting radiative feedback is too weak to disturb the accretion.
\par
In the other case starting from a turbulent cloud,  
a rapidly-rotating swollen protostar appears with some different properties from the counterpart in the spherical case. 
The protostar rotates at more than about $0.4$ times the Keplerian velocity everywhere in the interior due to the angular momentum brought by the accreting gas.
As a result, the equatorial radius is more than twice as large as the polar one. Also in this case, the effective temperature remains several thousand Kelvin and the radiative feedback is ineffective.
\par
Our simulations show a swollen protostar steadily grows in mass by accretion, which differs from the previous result by \citet{Luo_et_al_2018}. This may be attributed to differences in the RHD solver or initial conditions, but the main reason is currently uncertain and open for further studies.
Regarding the protostellar rotation, our results show the rigid rotation inside the protostar and nearly Keplerian motion near the surface. This is the same trend as in \citet{Greif_et_al_2012} and \citet{Stacy_et_al_2013}, who study the ordinary Pop III star formation without solving the radiative transfer. They found that protostars forming in various mini-halos generally have high spins.
Their and our results suggest that, if there are no processes efficiently extracting the angular momentum, the primordial stars should be rapid rotators. Their structure and evolution should differ from the non-rotating case.
\par
Although we have terminated the simulations when $M_* \sim 10~\Msun$, we can follow a longer-term evolution %over a longer period 
with our code, if we implement nuclear fusion processes effective in the later stage ($M\gtrsim50~\Msun$).
This will reveal directly how the protostar evolves under the strong rotation and whether it grows into a SMS as predicted by 1D stellar evolution calculations. The stellar spin affects the evolution in various ways. 
First, it potentially triggers some instabilities which generate turbulence. The turbulence induces chemical mixing, and it may change the nuclear fusion rate and the overall stellar structure. Second, the rotation stabilizes the SMSs and raises their maximum masses by an order of magnitude compared to the non-rotating cases \citep{Haemmerle_2021}.
The masses of the seed BHs can be consequently elevated, which assists the formation of the SMBHs observed at high redshifts. 
\par
A circumstellar disk forms at the end of our simulation, and it should continue to grow in mass and size afterward. The disk accretion process onto a rapidly rotating SMS is still under debate.
For instance, \citet{Lee+Yoon2016} argue that the $\Omega\Gamma$ limit regulates the gas accretion. \citet{Takahashi_and_Omukai_2017} show that the steady disk accretion solution for an arbitrary angular momentum flux exists, concluding that protostars can grow even under the $\Omega\Gamma$ limit. Whereas these authors only rely on 1D models, our 3D RHD simulations can reveal the realistic disk structure and the accretion process.
The circumstellar disk is also expected to become gravitationally unstable and fragment \citep{Becerra_et_al_2015, Kimura_et_al_2021}. If some of the emerging fragments fall onto the central rotating protostar, it may alter the protostellar evolution with resulting violently variable accretion rates \citep{Sakurai16}.
Further 3D RHD simulations studying these effects are necessary for elucidating the evolution of the star-disk system.
\par
We have followed the protostellar birth and subsequent evolution for $\sim 10$~yr expected in the DC. This is only the beginning of the evolution until the seed BH formation after $\sim$~Myr. Our 3D RHD simulations have a potential to reveal the realistic protostellar evolution in the later stage, for which only 1D modeling has been applied so far.

%%%%%%%%%%%%%%%%%%%%%%%%%%%%%%%%%%%%%%%%%%%%%%%%%%%%%%%%%%%%%%%
%
% Acknowledgments
%
%%%%%%%%%%%%%%%%%%%%%%%%%%%%%%%%%%%%%%%%%%%%%%%%%%%%%%%%%%%%%%%
%% If you wish to include an acknowledgments section in your paper,
%% separate it off from the body of the text using the \acknowledgments
%% command.
\acknowledgments
We are grateful to Kazuyuki Omukai, Kunihito Ioka, Tomoaki Matsumoto, Shinsuke Takasao, Kengo Tomida and Takahiro Tanaka for fruitful discussions and comments. 
The numerical simulations were carried out on XC50  {\tt Aterui II} at the Center for Computational Astrophysics (CfCA) of the National Astronomical Observatory of Japan, and Yukawa-21 at Yukawa Institute for Theoretical Physics of Kyoto University. 
This research could never be accomplished without the support by Grants-in-Aid for Scientific Research (TH:19H01934, 21H00041; KS: 21K20373) from the Japan Society for the Promotion of Science. This work is also supported by JST SPRING, Grant Number JPMJSP2110 (KK), the ANRI Fellowship (KK), and the Hakubi Project Funding of Kyoto University (KS).




%% To help institutions obtain information on the effectiveness of their 
%% telescopes the AAS Journals has created a group of keywords for telescope 
%% facilities.
%
%% Following the acknowledgments section, use the following syntax and the
%% \facility{} or \facilities{} macros to list the keywords of facilities used 
%% in the research for the paper.  Each keyword is check against the master 
%% list during copy editing.  Individual instruments can be provided in 
%% parentheses, after the keyword, but they are not verified.

%\vspace{5mm}
%\facilities{HST(STIS), Swift(XRT and UVOT), AAVSO, CTIO:1.3m,
%CTIO:1.5m,CXO}

%% Similar to \facility{}, there is the optional \software command to allow 
%% authors a place to specify which programs were used during the creation of 
%% the manusscript. Authors should list each code and include either a
%% citation or url to the code inside ()s when available.

%\software{astropy \citep{Astropy13},  
%          Cloudy \citep{2013RMxAA..49..137F}, 
%          SExtractor \citep{1996A&AS..117..393B}
%          }


%%%%%%%%%%%%%%%%%%%%%%%%%%%%%%%%%%%%%%%%%%%%%%%%%%%%%%%%%%%%%%%
%
% Appendix
%
%%%%%%%%%%%%%%%%%%%%%%%%%%%%%%%%%%%%%%%%%%%%%%%%%%%%%%%%%%%%%%%
%% Appendix material should be preceded with a single \appendix command.
%% There should be a \section command for each appendix. Mark appendix
%% subsections with the same markup you use in the main body of the paper.

%% Each Appendix (indicated with \section) will be lettered A, B, C, etc.
%% The equation counter will reset when it encounters the \appendix
%% command and will number appendix equations (A1), (A2), etc. The
%% Figure and Table counter will not reset.




%%%%%%%%%%%%%%%%%%%%%%%%%%%%%%%%%%%%%%%%%%%%%%%%%%%%%%%%%%%%%%%%
%
% References
%
%%%%%%%%%%%%%%%%%%%%%%%%%%%%%%%%%%%%%%%%%%%%%%%%%%%%%%%%%%%%%%%%
%% The reference list follows the main body and any appendices.
%% Use LaTeX's thebibliography environment to mark up your reference list.
%% Note \begin{thebibliography} is followed by an empty set of
%% curly braces.  If you forget this, LaTeX will generate the error
%% "Perhaps a missing \item?".
%%
%% thebibliography produces citations in the text using \bibitem-\cite
%% cross-referencing. Each reference is preceded by a
%% \bibitem command that defines in curly braces the KEY that corresponds
%% to the KEY in the \cite commands (see the first section above).
%% Make sure that you provide a unique KEY for every \bibitem or else the
%% paper will not LaTeX. The square brackets should contain
%% the citation text that LaTeX will insert in
%% place of the \cite commands.

%% We have used macros to produce journal name abbreviations.
%% \aastex provides a number of these for the more frequently-cited journals.
%% See the Author Guide for a list of them.

%% Note that the style of the \bibitem labels (in []) is slightly
%% different from previous examples.  The natbib system solves a host
%% of citation expression problems, but it is necessary to clearly
%% delimit the year from the author name used in the citation.
%% See the natbib documentation for more details and options.



%\begin{thebibliography}{}

%\bibitem[Astropy Collaboration et al.(2013)]{2013A&A...558A..33A} Astropy Collaboration, Robitaille, T.~P., Tollerud, E.~J., et al.\ 2013, \aap, 558, A33 
%\bibitem[Bertin \& Arnouts(1996)]{1996A&AS..117..393B} Bertin, E., \& Arnouts, S.\ 1996, \aaps, 117, 393 
%\bibitem[Corrales(2015)]{2015ApJ...805...23C} Corrales, L.\ 2015, \apj, 805, 23
%\bibitem[Ferland et al.(2013)]{2013RMxAA..49..137F} Ferland, G.~J., Porter, R.~L., van Hoof, P.~A.~M., et al.\ 2013, \rmxaa, 49, 137
%\bibitem[Hanisch \& Biemesderfer(1989)]{1989BAAS...21..780H} Hanisch, R.~J., \& Biemesderfer, C.~D.\ 1989, \baas, 21, 780 
%\bibitem[Lamport(1994)]{lamport94} Lamport, L. 1994, LaTeX: A Document Preparation System, 2nd Edition (Boston, Addison-Wesley Professional)
%\bibitem[Schwarz et al.(2011)]{2011ApJS..197...31S} Schwarz, G.~J., Ness, J.-U., Osborne, J.~P., et al.\ 2011, \apjs, 197, 31  
%\bibitem[Vogt et al.(2014)]{2014ApJ...793..127V} Vogt, F.~P.~A., Dopita, M.~A., Kewley, L.~J., et al.\ 2014, \apj, 793, 127  

%\end{thebibliography}

\bibliography{bib} % if your bibtex file is called example.bib


%% This command is needed to show the entire author+affilation list when
%% the collaboration and author truncation commands are used.  It has to
%% go at the end of the manuscript.
%\allauthors

%% Include this line if you are using the \added, \replaced, \deleted
%% commands to see a summary list of all changes at the end of the article.
%\listofchanges

\end{document}

% End of file `sample62.tex'.