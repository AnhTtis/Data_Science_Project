\section{Related Work}
We briefly review some work on full posterior inference for SLAM with a focus on non-Gaussian representations of the posterior. Note that the non-Gaussian representation can be parametric models (e.g., sum of Gaussians) or non-parametric (i.e., samples).

The framework of RBPF has been leveraged in a big class of SLAM algorithms (e.g., FastSLAM) \cite{montemerlo2003fastslam, blanco2008pure, blanco2008efficient}, where the posterior of robot poses is represented by a set of particles, and each particle is attached with parametric models \cite{montemerlo2003fastslam, blanco2008efficient} or samples \cite{blanco2008pure} of the conditional of the map. Although our work exploits the same conditional independence relation as FastSLAM, i.e., landmarks are conditionally independent given robot poses, our work swaps representations of the robot path and map in RBPF, resulting in a parametric (Gaussian) model to describe the belief about robot poses and a set of particle filters to independently draw samples from the posterior of each landmark. Thus the issue of losing diversity in robot path particles no longer exists, warranting the scalability of our work. We will compare to \cite{blanco2008efficient} in our experiments section. mm-iSAM \cite{fourie2016nonparametric} and NF-iSAM \cite{huang2021online} are recent non-Gaussian inference algorithms that leverage the Bayes tree algorithm \cite{kaess2012isam2} to exploit more conditional independence relations. As solvers for general factor graphs, they can tackle both nonlinear measurement models and multi-modal data association, while our work is only focused on nonlinear measurement models. Another class of approaches purely relies on Monte Carlo techniques to directly draw samples from the joint posterior \cite{torma2010markov,shariff15aistats, huang2021reference}. These methods, in general, do not suit real-time online applications due to the exorbitant computational cost but can serve as reference solutions to full posterior inference.

%Range-only and bearing-only SLAM are typical examples that incur highly non-Gaussian posteriors since a single range or bearing cannot pinpoint a landmark.
Our work is inspired by some works \cite{davison2003real,blanco2008efficient,blanco2008pure} dedicated to bearing-only (ponit-object-based) or range-only SLAM. Existing approaches for these two problems can be categorized as batch optimizations \cite{newman2003pure}, delayed initializations of landmarks \cite{lemaire2005practical,davison2003real}, and undelayed initializations of landmarks \cite{kwok02iros,sola05iros,blanco2008efficient,blanco2008pure}. Our work falls into the undelayed category. Our experiments show that we can solve both of these problems in a unified framework and demonstrate the evolution of landmark posteriors since time step zero. Furthermore, we clarify that, while our object-based SLAM system requires similar input (RGB images and object detections) as the systems in \cite{nicholson2018quadricslam, yang2019cubeslam, ok2019robust, rubino20173d}, we do not estimate occupied areas of objects and focus sololy on inferring how distributions of object locations evolve. The ellipsoids in the object-based SLAM section are confidence intervals of object locations rather than models often seen in those works for indicating the 3D occupancy of objects in a scene.