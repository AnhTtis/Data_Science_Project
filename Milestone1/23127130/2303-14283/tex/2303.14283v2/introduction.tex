\section{Introduction}
Simultaneous localization and mapping (SLAM) is a foundational technique for autonomous driving, AR/VR, and robotic navigation. While the primary goal of SLAM is to find a point estimate of robot paths and maps, inferring the posterior density encountered in SLAM is also essential. This is because full posterior inference reveals how uncertain the robot path and map could be, and supports the robot in planning how to reduce the uncertainty for safe navigation \cite{Rosen2021Advances}.

Standard \emph{real-time} methods for full posterior inference in SLAM research either compute a Gaussian approximation, centered on the point estimate, to the posterior \cite{dellaert2012factor}, or represent the posterior using samples in the framework of Rao-Blackwellized particle filtering (RBPF) \cite{montemerlo2003fastslam}. Constructing the Gaussian approximation enjoys great advantages in scalability for tackling high-dimensional posteriors, owing to efficient nonlinear optimization solvers for SLAM \cite{kaess2012isam2, kummerle2011g}. However, the Gaussian approximation is inherently incapable of describing highly non-Gaussian/multi-modal posteriors, which often appear in realworld SLAM problems due to nonlinear measurement models. While the RBPF can capture non-Gaussian features of the posterior via samples, it suffers from scalability issues incurred by particle degeneracy and depletion \cite{arulampalam2002tutorial}.

\begin{figure}[t]
    \centering
    \includegraphics[width=\linewidth]{figs/illustration.pdf}
    \caption{Illustration of our method for inferring robot pose and landmark distributions: (a) a SLAM example, where the robot moves along poses in green and makes measurements to landmarks in red, and (b) our method, which blends Gaussain approximation in yellow and particle filters in pink. The Gaussian approximation, centered on the maximum a posteriori (MAP) estimate, provides distributions of robot poses on which the particle filters are conditioned to draw samples that represent landmark distributions. If a sample attains a higher probability than the MAP estimate when evaluating the posterior, landmarks in the Gaussian solver will be re-initialized by that sample.}
    \label{fig:my_label}
	\vspace{-15pt}
\end{figure}

By blending advantages of the \textbf{G}aussian \textbf{A}pproximation and \textbf{P}article filters on scalability and non-Gaussian estimation, respectively, we present a novel algorithm, \textbf{GAP}SLAM, to infer marginal posteriors encountered in SLAM. The contributions of the paper include:
\begin{enumerate}
\item An adaptive modeling strategy whereby only marginals with great uncertainty are sampled by particle filters, while others are represented by the Gaussian approximation.
\item An uncertainty-aware re-initialization technique that leverages particle filters to reset linearization points in the nonlinear optimization solver.
\item Range-only and object-based bearing-only SLAM experiments that demonstrate the scalability, generalizability, and accuracy of GAPSLAM, as well as its ability to precisely describe the evolution of non-Gaussian posteriors in real-time.
\end{enumerate}
