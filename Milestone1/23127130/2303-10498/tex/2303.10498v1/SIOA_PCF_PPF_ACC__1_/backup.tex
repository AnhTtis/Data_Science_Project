
\section{Proof of Proposition \ref{pro:scaling_matrix}}
\begin{proof}
This can be shown by $\boldsymbol{D}_R\widetilde{\mathbfsbilow{u}}_{\Xi,c} =\widetilde{\mathbfsbilow{u}}_{\Xi,c} \boldsymbol{D}_L$ and and the property of determinant:
\begin{subequations}
\begin{align}
    &\text{det}(\mathsfbi{I}-\mathbfsbilow{\widetilde{H}}_{\nabla}\mathbfsbilow{\widetilde{u}}_{\Xi,c})\\
    = &\text{det}(\mathsfbi{I}-\mathbfsbilow{\widetilde{H}}_{\nabla}\boldsymbol{D}_R^{-1}\mathbfsbilow{\widetilde{u}}_{\Xi,c}\boldsymbol{D}_L)\\
    =&\text{det}(\boldsymbol{D}_L^{-1}-\mathbfsbilow{\widetilde{H}}_{\nabla}\boldsymbol{D}_L^{-1}\mathbfsbilow{\widetilde{u}}_{\Xi,c})\text{det}(\boldsymbol{D}_L)\\
    =&\text{det}(\boldsymbol{D}_L)\text{det}(\boldsymbol{D}_L^{-1}-\mathbfsbilow{\widetilde{H}}_{\nabla}\boldsymbol{D}_R^{-1}\mathbfsbilow{\widetilde{u}}_{\Xi,c})\\
    =&\text{det}(\mathsfbi{I}-\boldsymbol{D}_L\mathbfsbilow{\widetilde{H}}_{\nabla}\boldsymbol{D}_R^{-1}\mathbfsbilow{\widetilde{u}}_{\Xi,c})
\end{align}
\end{subequations}
Using the definition of structured singular value in \eqref{eq:mu}, we have $\mu_{\widetilde{\mathbfsbilow{U}}_{\Xi,c}}[\mathbfsbilow{\widetilde{H}}_{\nabla}]=\mu_{\widetilde{\mathbfsbilow{U}}_{\Xi,c}}(\boldsymbol{D}_L \mathbfsbilow{\widetilde{H}}_{\nabla} \boldsymbol{D}_R^{-1})$. 
\end{proof}




We firstly provide the relation between structured singular value and the large singular value of a scaled spatio-temporal frequency response operator based on \cite[theorem 3.8]{packard1993complex}. 

\begin{pro}
\label{pro:scaling_matrix}
Given $(k_x,k_z,\omega)$ pair and $\mathbfsbilow{\widetilde{H}}_{\nabla}\in \mathbb{C}^{9N_y\times 9N_y}$, uncertainty set $\widetilde{\mathbfsbilow{U}}_{\Xi,c}$, and scaling positive definite matrix set $\underline{\boldsymbol{D}}_L$ ($\boldsymbol{D}_L^*=\boldsymbol{D}_L\succ 0$) and $\underline{\boldsymbol{D}}_R$ ($\boldsymbol{D}_R^*=\boldsymbol{D}_R\succ 0$) such that for $\boldsymbol{D}_L\in \underline{\boldsymbol{D}}_L$  $\boldsymbol{D}_R\in \underline{\boldsymbol{D}}_R$ and $\widetilde{\mathbfsbilow{u}}_{\Xi,c} \in \widetilde{\mathbfsbilow{U}}_{\Xi,c}$, they satisfy
\begin{align}
\boldsymbol{D}_R\widetilde{\mathbfsbilow{u}}_{\Xi,c} =\widetilde{\mathbfsbilow{u}}_{\Xi,c} \boldsymbol{D}_L.
\label{eq:D_L_D_R_relation}
\end{align}
Then, 
\begin{align}
\mu_{\widetilde{\mathbfsbilow{U}}_{\Xi,c}}[\mathbfsbilow{\widetilde{H}}_{\nabla}]=\mu_{\widetilde{\mathbfsbilow{U}}_{\Xi,c}}(\boldsymbol{D}_L \mathbfsbilow{\widetilde{H}}_{\nabla} \boldsymbol{D}_R^{-1}).
\end{align}
\label{thm:chapter2_mu_lmi}
\end{pro}
However, scaling the matrix $\mathbfsbilow{\widetilde{H}}_{\nabla}$ as $\boldsymbol{D}_L \mathbfsbilow{\widetilde{H}}_{\nabla} \boldsymbol{D}_R^{-1}$ may change the largest singular value, which gives the following inequality to compute the upper bound of structured singular value \cite[equation (11.14)]{zhou1996robust}:
\begin{align}
    \mu_{\widetilde{\mathbfsbilow{U}}_{\Xi,c}}[ \mathbfsbilow{\widetilde{H}}_{\nabla}]=&\mu_{\widetilde{\mathbfsbilow{U}}_{\Xi,c}}[\boldsymbol{D}_L \mathbfsbilow{\widetilde{H}}_{\nabla} \boldsymbol{D}_R^{-1}]\\
    \leq& \underset{\boldsymbol{D}_L\in \underline{\boldsymbol{D}}_L,\boldsymbol{D}_R\in \underline{\boldsymbol{D}}_R }{\text{inf}}\bar{\sigma}[ \boldsymbol{D}_L \mathbfsbilow{\widetilde{H}}_{\nabla} \boldsymbol{D}_R^{-1}].
    \label{eq:chapter2_mu_DMD}
\end{align}
\begin{remark}
The scaling matrix sets $\underline{\boldsymbol{D}}_L$ and $\underline{\boldsymbol{D}}_R$ associated with $\widetilde{\mathbfsbilow{U}}_{\Xi,c}$ in \eqref{eq:uncertain_set} satisfying \eqref{eq:D_L_D_R_relation} can be chosen as:
\begin{align}
    \underline{\boldsymbol{D}}_L=\underline{\boldsymbol{D}}_L=\{&\text{diag}[d_1 \mathsfbi{I}_{N_y\times N_y},d_2\mathsfbi{I}_{N_y\times N_y},... ,d_9\mathsfbi{I}_{N_y\times N_y}]\nonumber\\
    &:d_1, d_2, ...,d_9\in \mathbb{R}\}
\end{align}
\end{remark}

We then use \texttt{mussvextract} command in MATLAB to obtain corresponding $\boldsymbol{D}_L$ and $\boldsymbol{D}_R$ scaling matrices and  

\textbf{Mention the relation of rank-1}



\begin{align}
    \boldsymbol{P}:=&\text{diag}\left(\mathcal{I}_{1\times 3},\mathcal{I}_{1\times 3},\mathcal{I}_{1\times 3}\right),\label{eq:P}\\ \boldsymbol{\widehat{u}}_{\Xi,c}:=&\mathcal{I}_{3\times 3}\otimes\text{diag}\left(-\widehat{u}_\xi,-\widehat{v}_\xi,-\widehat{w}_\xi \right),
    \label{eq:uncertainty_u_Xi_c}
\end{align}

\iffalse
\begin{figure}
    \centering
    \includegraphics[width=0.49\textwidth]{feedback_interconnection_detail_tilde.png}
    \caption{Illustration of structured input--output analysis: (a) a componentwise description,  where blocks inside of ({\color{blue}$\dashed$}, blue) represent the modeled forcing in equation \eqref{eq:feedback_structured_uncertainty} with $\widehat{\boldsymbol{u}}_{\Xi,c}$ being relaxed as $\widetilde{\boldsymbol{u}}_{\Xi,c}$ in \eqref{eq:uncertainty_u_Xi_c_non_repeated}.}
    \label{fig:feedback_detail}
\end{figure}

\begin{figure}
    \centering
    \includegraphics[width=0.1\textwidth]{feedback_interconnection_abstract_tilde.png}
    \caption{ Panel (b) redraws panel (a) after discretization with the top block corresponding to the combination of the three top blocks in panel (a) and the bottom block corresponding to the bottom block of panel (a).}
    \label{fig:feedback_discretized}
\end{figure}

\fi




%In the present work we first modify the originally proposed SIOA feedback interconnection in order to  We then demonstrate that the structural features associated with these correlations are consistent with results of nonlinear optimal perturbation analysis and the secondary stability analysis associated with streamwise streaks. These results provide further evidence that behavior associated with nonlinear effects can be captured through using SIOA based approaches. 
%The structured uncertainty is then the most destabilizing perturbations associated with the given feedback interconnection, which we interpret as being associated with flow structures most likely to be amplified or predominant in the transitional flow.


% \dennice{we could say secondary stability analysis or something here but the term secondary analysis is unclear, even secondary stability analysis is a bit strange for a controls audience but it is fine because people who work on this problem will know what we mean}


%Recently proposed structured input-output analysis (SIOA) employs a feedback interconnection between a spatio-temporal response operator and structured uncertainty that allows us to include a model of the nonlinearity to identify the  characteristic features of transitional wall-bounded shear flows. In the present work we first slightly modify the originally proposed SIOA feedback interconnection in order to decompose the structured uncertainty into three components associated with the streamwise, spanwise and wall-normal velocity correlations. We then demonstrate that the structural features associated with velocity correlations are consistent with results of nonlinear optimal perturbation analysis and the secondary stability analysis associated with streamwise streaks. These results provide further evidence that behavior associated with nonlinear effects can be captured through using SIOA based approaches. 


%Recently, structured input-output analysis (SIOA) has been introduced into transitional wall-bounded shear flows to identify characteristic scales in translational invariant directions. This work aims to further develop SIOA to provide insight into inhomogeneous (wall-normal) directions. We introduce a feedback interconnection between a modified spatio-temporal response operator and structured uncertainty to model the nonlinearity, where structured uncertainty isolates destabilizing velocity vectors into each component. This work then interprets the destabilizing structured uncertainty as destabilizing velocity correlation. The most destabilizing velocity correlation shows absolute value maximized near the channel center consistent with secondary instability of streamwise streaks and shows real part reversing sign near the channel center consistent with nonlinear optimal perturbations. 



%\cliu{It is also observed that nonlinear mechanisms disadvantage the growth of streamwise elongated streaks \cite{duguet2013minimal,Brandt2014}. }

%demonstrated that the input mode is associated with cross-stream forcing resembling streamwise vortices while the output mode corresponds to streamwise streaks \cite{Jovanovic2005,schmid2007nonmodal,jovanovic2020bypass}.


%is able to capture the non-normality of the linearized operator and associated transient growth that is inherent in wall-bounded shear flows. The streamwise elongated structures are identified as the most amplified flow structures, where the input mode is associated with cross-stream forcing resembling streamwise vortices while the output mode corresponds to streamwise streaks \cite{Jovanovic2005,schmid2007nonmodal,jovanovic2020bypass}. This observation captures the important lift-up mechanism that cross-stream forcing redistributes background mean shear across channel height to form streamwise streaks \cite{Brandt2014,jovanovic2020bypass}. However, input-output analysis obscures streamwise dependent structures, which are also known to be important from observations in experiments \cite{prigent2003long}, direct numerical simulation (DNS) \cite{reddy1998stability} and nonlinear optimal perturbations (NLOP) \cite{Rabin2012}. 

%\dennice{I am not sure the point here, singular values are known to provide this type of information and this is a controls paper so people know the difference between eigenvalues and singular values. I think you want to start with input output analysis and then mention that traditional analysis emphasizes the streamwise elongated while nonlinear shows importance of other structures to lead into why we are using SIOA} \cliu{I have edited this paragraph starting from input-output analysis.}




%\dennice{To me the idea here is to further explore the nature of the structure not just the wall normal extent. Before we did not look at the actual forcing so I think that needs to be the focus here. }  \cliu{Chang: Actually this work does not look at the forcing but the structured uncertainty. This structured uncertainty maps the output from $\widetilde{\mathcal{H}}_\nabla$ to become the forcing. Edited this paragraph as below. }\dennice{Forcing in the sense that we say these are optimal perturbations in a sense}

%Previous SIOA based studies have focused primarily on the streamwise and spanwise characteristics of the flow \cite{liu2021structuredJournal,liu2021feedback,liu2022structured}. 


%the wall-normal variation of destabilizing structured uncertainty, which is important within feedback interconnection to generate the structured input forcing. \dennice{Again I do not see the wall normal as the focus here as we are digging into what the singular value looks like and not really providing a wall normal analog of what we had before but we can discuss} 

%\dennice{I think we can say these have been useful but is that really what motivates this interpretation?}



% \dennice{from figure 3 it looks like this is true only for the autocorrelation} \dennice{I added auto to correlation as the profile is $y=y'$} \cliu{Thanks, autocorrelation looks good to me. These statement here is also true for correlation matrix with $y\neq y'$, although we only show one component.  Plotting $y=y'$ is mainly trying to save space otherwise it is not easy to show totally nine components} \dennice{I am a bit confused here maybe we should have  brief talk once you are up. Do you mean the full contour plots look like this? I changed the statement a bit to reflect this}


% The nonlinear term in \eqref{eq:NSDecompf1} is then written as \dennice{formatting makes this hard to read as it looks like one long equation rather than a matrix. I am also not sure why this equation is necessary.  I would just say we model the nonlinearity $u\cdot\nabla u $ as and move right to (3)} \cliu{I agree}
% \begin{subequations}
%      \label{eq:f_nonlinear}
% \begin{align}
% \boldsymbol{f}:=&-\boldsymbol{u}\!\cdot\! \boldsymbol{\nabla }\boldsymbol{u}\\
% =&\begin{bmatrix}-u\partial_x u-v\partial_y u-w\partial_z u\\ -u\partial_x v-v\partial_y v-w\partial_z v\\
% -u\partial_x w-v\partial_y w-w\partial_z w\end{bmatrix}=:\begin{bmatrix}f_x\\
% f_y\\
% f_z\end{bmatrix}. 
% \end{align}
% \end{subequations}
% This expression of the nonlinearity as forcing terms makes \eqref{eq:NS_All} into a set of forced linear evolution equations. 

%\\=:&\begin{bmatrix}f_{xu}+f_{xv}+f_{xw}\\ f_{yu}+f_{yv}+f_{yw}\\f_{zu}+f_{zv}+f_{zw}\end{bmatrix}


%The linear form of \eqref{eq:f_uncertain_model}


% \dennice{I think this sentence belosongs with equation (6) because the is where you are defining the forcing and the structure but we can discuss} \cliu{I agree} \sout{The structured uncertainty $\widetilde{\boldsymbol{u}}_{\Xi,c}$ in \eqref{eq:uncertainty_u_Xi_c_non_repeated} has a block-diagonal structure such that the resulting feedback interconnection leads to a forcing model that retains the componentwise structure of the nonlinearity, which is demonstrated to be important to uncover streamwise dependent flow structures \cite[\S 3.3]{liu2021structuredJournal}.}


%\dennice{I would move this paragraph up before discusing the modification of the formulation. There are a few options which we can discuss}


    % 0, \,\text{if}\;\;\forall \mathbfsbilow{\widetilde{u}}_{\Xi,c}\in \mathbfsbilow{\widetilde{U}}_{\Xi,c}, \text{det}(\mathsfbi{I}-\mathbfsbilow{\widetilde{H}}_{\nabla}\mathbfsbilow{\widetilde{u}}_{\Xi,c})\neq 0
    % \end{cases}.
    
    
    
% \end{widetext}
% \noindent where $N_y$ denotes the number of grid points in $y$. 


% \dennice{This needs a bit more discussion. }

% In order to directly use the \texttt{mussv} command in the Robust Control Toolbox of MATLAB, we relax $\boldsymbol{\widehat{u}}_{\Xi,c}$ in \eqref{eq:uncertainty_u_Xi_c} as:
% \begin{align}
%     \boldsymbol{\widetilde{u}}_{\Xi,c}:=&\text{diag}(-\widehat{u}_{\xi,1}-\widehat{v}_{\xi,1},-\widehat{w}_{\xi,1},\nonumber\\
%     &-\widehat{u}_{\xi,2}-\widehat{v}_{\xi,2},-\widehat{w}_{\xi,2},-\widehat{u}_{\xi,3}-\widehat{v}_{\xi,3},-\widehat{w}_{\xi,3}),
%     \label{eq:uncertainty_u_Xi_c_non_repeated}
% \end{align}
% where $\widehat{u}_{\xi,j}$, $\widehat{v}_{\xi,j}$, $\widehat{w}_{\xi,j}$ ($j=1,2,3$) are not necessarily repeated for three different indices $j=1,2,3$. This decomposition of the forcing function in \eqref{eq:feedback_structured_uncertainty} with relaxation in \eqref{eq:uncertainty_u_Xi_c_non_repeated} is illustrated in the three blocks inside the blue dashed line ({\color{blue}$\dashed$}) in Fig. \ref{fig:feedback_detail}(a).

% \dennice{This should be a different section as it is really results}

% \dennice{Please add number of points, I do not see this elsewhere} \cliu{add the number here.}



%\dennice{there was some repetition here so I edited, please check}


% The upper bound of structured singular value in definition \ref{def:mu} can be obtained by the largest singular value \cite{packard1993complex,zhou1996robust}. We define $\|\widetilde{\mathcal{H}}_{\nabla}\|_\infty(k_x,k_z):=\underset{\omega \in \mathbb{R}}{\text{sup}}\;\bar{\sigma}\left[\mathbfsbilow{\widetilde{H}}_{\nabla}(k_x,k_z,\omega)\right]$, which similarly provides $\|\widetilde{\mathcal{H}}_{\nabla}\|_{\mu,c}(k_x,k_z)\leq \|\widetilde{\mathcal{H}}_{\nabla}\|_\infty(k_x,k_z)$.



% \[
% \mathcal{H}_{pqj}=\mathcal{\hat{C}}_p(i\omega\mathcal{I}-\hat{A})^{-1}\mathcal{\hat{B}}_q,\quad j=1,2,3
% \]
% where 
% BLAH BLAH

%\dennice{I would define this here since we have a bit of space since you are using the notation below and it is not defined you can use the version in (3.3) of the original paper with the modification here. } \cliu{I move the definition from theorem to above here. }

% Quantities similar to a velocity correlation have been widely analyzed in related analysis; see e.g., \cite[Fig. 9]{Zare2017}, \cite[Fig. 8]{liu2020input}, \cite[Fig. 14]{towne2020resolvent} and \cite[Fig. 6]{nogueira2021forcing}.We note that $\widetilde{\mathbfsbilow{u}}_{\Xi,c}(y,y';k_x, k_z, \omega)$ is in spectrum space so in general it can be a complex function.


% \dennice{This notation does not match equation (12) it should be a capitol u throughout this part right?}

%\dennice{I removed the note about it being complex.  You can move that to after equation (12) but it does not make sense here sine (12) says they are complex objects}  

%\dennice{I do not see another place where number of grid points is indicated}


%that imaginary parts $\widehat{u}_{\xi,2}$, $\widehat{v}_{\xi,2}$ and $\widehat{w}_{\xi,2}$ are closely matching the absolute values over wall-normal direction $y$, a behavior different from that associated with $x$ momentum equation shown in Fig. \ref{fig:u_xi_abs_real_imag_diag_1_cou}. For the  $\widehat{u}_{\xi,3}$, $\widehat{v}_{\xi,3}$ and $\widehat{w}_{\xi,3}$, their shape is similar to  $\widehat{u}_{\xi,1}$, $\widehat{v}_{\xi,1}$ and $\widehat{w}_{\xi,1}$ as shown in Fig. \ref{fig:u_xi_abs_real_imag_diag_1_cou}, while their absolute values on $\widehat{u}_{\xi,3}$ and $\widehat{w}_{\xi,3}$ is smaller than their counterpart by $\widehat{u}_{\xi,1}$ an $\widehat{w}_{\xi,1}$. This suggests that different components of structured uncertainty that destabilizes the feedback interconnection are showing different behavior associated with each momentum equation within the current formulation. 



%We also find that the absolute values of $\widehat{u}_{\xi,1}$ and $\widehat{w}_{\xi,1}$ have a similar order of magnitude and both of them are larger than $\widehat{v}_{\xi,1}$. Furthermore, the $\widehat{u}_{\xi,1}$ and $\widehat{w}_{\xi,1}$ show almost the same shape over $(y,y')$. 



\iffalse
\begin{widetext}
\begin{figure}
    \centering
    %u1
 (a) $|10^3\,\widehat{u}_{\xi,1}(y,y')|$ \hspace{0.12\textwidth} (b) $\mathcal{R}e[10^3\,\widehat{u}_{\xi,1}(y,y')]$ \hspace{0.09\textwidth} (c) $\mathcal{I}m[10^3\,\widehat{u}_{\xi,1}(y,y')]$ \hspace{0.04\textwidth} (d) $10^3\,\widehat{u}_{\xi,1}(y,y'=y)$
    
    \includegraphics[width=0.27\textwidth]{figure/rotating_laminar_cou_Re_358rotation_0_ssvd_20220804_N122u1_xiNy=120_abs.png}
    \includegraphics[width=0.27\textwidth]{figure/rotating_laminar_cou_Re_358rotation_0_ssvd_20220804_N122u1_xiNy=120_real.png}
    \includegraphics[width=0.27\textwidth]{figure/rotating_laminar_cou_Re_358rotation_0_ssvd_20220804_N122u1_xiNy=120_imag.png}
    \includegraphics[width=0.135\textwidth,trim=-0 -0.7in 0 0]{figure/rotating_laminar_cou_Re_358rotation_0_ssvd_20220804_N122u1_xiNy=120_diag_abs_real_imag.png}

    %v1
     (e) $|10^3\,\widehat{v}_{\xi,1}(y,y')|$ \hspace{0.12\textwidth} (f) $\mathcal{R}e[10^3\,\widehat{v}_{\xi,1}(y,y')]$ \hspace{0.09\textwidth} (g) $\mathcal{I}m[10^3\,\widehat{v}_{\xi,1}(y,y')]$ \hspace{0.04\textwidth} (h) $10^3\,\widehat{v}_{\xi,1}(y,y'=y)$
     
    \includegraphics[width=0.27\textwidth]{figure/rotating_laminar_cou_Re_358rotation_0_ssvd_20220804_N122v1_xiNy=120_abs.png}
    \includegraphics[width=0.27\textwidth]{figure/rotating_laminar_cou_Re_358rotation_0_ssvd_20220804_N122v1_xiNy=120_real.png}
    \includegraphics[width=0.27\textwidth]{figure/rotating_laminar_cou_Re_358rotation_0_ssvd_20220804_N122v1_xiNy=120_imag.png}
    \includegraphics[width=0.135\textwidth,trim=-0 -0.7in 0 0]{figure/rotating_laminar_cou_Re_358rotation_0_ssvd_20220804_N122v1_xiNy=120_diag_abs_real_imag.png}
    
    %w1
     (i) $|10^3\,\widehat{w}_{\xi,1}(y,y')|$ \hspace{0.11\textwidth} (j) $\mathcal{R}e[10^3\,\widehat{w}_{\xi,1}(y,y')]$ \hspace{0.08\textwidth} (k) $\mathcal{I}m[10^3\,\widehat{w}_{\xi,1}(y,y')]$ \hspace{0.03\textwidth} (l) $10^3\,\widehat{w}_{\xi,1}(y,y'=y)$
     
    \includegraphics[width=0.27\textwidth]{figure/rotating_laminar_cou_Re_358rotation_0_ssvd_20220804_N122w1_xiNy=120_abs.png}
    \includegraphics[width=0.27\textwidth]{figure/rotating_laminar_cou_Re_358rotation_0_ssvd_20220804_N122w1_xiNy=120_real.png}
    \includegraphics[width=0.27\textwidth]{figure/rotating_laminar_cou_Re_358rotation_0_ssvd_20220804_N122w1_xiNy=120_imag.png}
    \includegraphics[width=0.135\textwidth,trim=-0 -0.7in 0 0]{figure/rotating_laminar_cou_Re_358rotation_0_ssvd_20220804_N122w1_xiNy=120_diag_abs_real_imag.png}

    % % %u2
    %  (a) $|10^3\,\widehat{u}_{\xi,2}(y,y')|$ \hspace{0.12\textwidth} (b) $\mathcal{R}e[10^3\,\widehat{u}_{\xi,2}(y,y')]$ \hspace{0.09\textwidth} (c) $\mathcal{I}m[10^3\,\widehat{u}_{\xi,2}(y,y')]$ \hspace{0.04\textwidth} (d) $10^3\,\widehat{u}_{\xi,2}(y,y'=y)$

    % \includegraphics[width=0.27\textwidth]{figure/rotating_laminar_cou_Re_358rotation_0_ssvd_20220804_N122u2_xiNy=120_abs.png}
    % \includegraphics[width=0.27\textwidth]{figure/rotating_laminar_cou_Re_358rotation_0_ssvd_20220804_N122u2_xiNy=120_real.png}
    % \includegraphics[width=0.27\textwidth]{figure/rotating_laminar_cou_Re_358rotation_0_ssvd_20220804_N122u2_xiNy=120_imag.png}
    % \includegraphics[width=0.135\textwidth,trim=-0 -0.7in 0 0]{figure/rotating_laminar_cou_Re_358rotation_0_ssvd_20220804_N122u2_xiNy=120_diag_abs_real_imag.png}
    
    % %v2
    % (e) $|10^3\,\widehat{v}_{\xi,2}(y,y')|$ \hspace{0.12\textwidth} (f) $\mathcal{R}e[10^3\,\widehat{v}_{\xi,2}(y,y')]$ \hspace{0.09\textwidth} (g) $\mathcal{I}m[10^3\,\widehat{v}_{\xi,2}(y,y')]$ \hspace{0.04\textwidth} (h) $10^3\,\widehat{v}_{\xi,2}(y,y'=y)$

    % \includegraphics[width=0.27\textwidth]{figure/rotating_laminar_cou_Re_358rotation_0_ssvd_20220804_N122v2_xiNy=120_abs.png}
    % \includegraphics[width=0.27\textwidth]{figure/rotating_laminar_cou_Re_358rotation_0_ssvd_20220804_N122v2_xiNy=120_real.png}
    % \includegraphics[width=0.27\textwidth]{figure/rotating_laminar_cou_Re_358rotation_0_ssvd_20220804_N122v2_xiNy=120_imag.png}
    % \includegraphics[width=0.135\textwidth,trim=-0 -0.7in 0 0]{figure/rotating_laminar_cou_Re_358rotation_0_ssvd_20220804_N122v2_xiNy=120_diag_abs_real_imag.png}
    
    % %w2
    % (i) $|10^3\,\widehat{w}_{\xi,2}(y,y')|$ \hspace{0.11\textwidth} (j) $\mathcal{R}e[10^3\,\widehat{w}_{\xi,2}(y,y')]$ \hspace{0.08\textwidth} (k) $\mathcal{I}m[10^3\,\widehat{w}_{\xi,2}(y,y')]$ \hspace{0.03\textwidth} (l) $10^3\,\widehat{w}_{\xi,2}(y,y'=y)$

    % \includegraphics[width=0.27\textwidth]{figure/rotating_laminar_cou_Re_358rotation_0_ssvd_20220804_N122w2_xiNy=120_abs.png}
    % \includegraphics[width=0.27\textwidth]{figure/rotating_laminar_cou_Re_358rotation_0_ssvd_20220804_N122w2_xiNy=120_real.png}
    % \includegraphics[width=0.27\textwidth]{figure/rotating_laminar_cou_Re_358rotation_0_ssvd_20220804_N122w2_xiNy=120_imag.png}
    % \includegraphics[width=0.135\textwidth,trim=-0 -0.7in 0 0]{figure/rotating_laminar_cou_Re_358rotation_0_ssvd_20220804_N122w2_xiNy=120_diag_abs_real_imag.png}

    % %u3
    % (a) $|10^3\,\widehat{u}_{\xi,3}(y,y')|$ \hspace{0.12\textwidth} (b) $\mathcal{R}e[10^3\,\widehat{u}_{\xi,3}(y,y')]$ \hspace{0.09\textwidth} (c) $\mathcal{I}m[10^3\,\widehat{u}_{\xi,3}(y,y')]$ \hspace{0.04\textwidth} (d) $10^3\,\widehat{u}_{\xi,3}(y,y'=y)$

    % \includegraphics[width=0.27\textwidth]{figure/rotating_laminar_cou_Re_358rotation_0_ssvd_20220804_N122u3_xiNy=120_abs.png}
    % \includegraphics[width=0.27\textwidth]{figure/rotating_laminar_cou_Re_358rotation_0_ssvd_20220804_N122u3_xiNy=120_real.png}
    % \includegraphics[width=0.27\textwidth]{figure/rotating_laminar_cou_Re_358rotation_0_ssvd_20220804_N122u3_xiNy=120_imag.png}
    % \includegraphics[width=0.135\textwidth,trim=-0 -0.7in 0 0]{figure/rotating_laminar_cou_Re_358rotation_0_ssvd_20220804_N122u3_xiNy=120_diag_abs_real_imag.png}
    
    % %v3
    % (e) $|10^3\,\widehat{v}_{\xi,3}(y,y')|$ \hspace{0.12\textwidth} (f) $\mathcal{R}e[10^3\,\widehat{v}_{\xi,3}(y,y')]$ \hspace{0.09\textwidth} (g) $\mathcal{I}m[10^3\,\widehat{v}_{\xi,3}(y,y')]$ \hspace{0.04\textwidth} (h) $10^3\,\widehat{v}_{\xi,3}(y,y'=y)$

    % \includegraphics[width=0.27\textwidth]{figure/rotating_laminar_cou_Re_358rotation_0_ssvd_20220804_N122v3_xiNy=120_abs.png}
    % \includegraphics[width=0.27\textwidth]{figure/rotating_laminar_cou_Re_358rotation_0_ssvd_20220804_N122v3_xiNy=120_real.png}
    % \includegraphics[width=0.27\textwidth]{figure/rotating_laminar_cou_Re_358rotation_0_ssvd_20220804_N122v3_xiNy=120_imag.png}
    % \includegraphics[width=0.135\textwidth,trim=-0 -0.7in 0 0]{figure/rotating_laminar_cou_Re_358rotation_0_ssvd_20220804_N122v3_xiNy=120_diag_abs_real_imag.png}
    
    % %w3
    % (i) $|10^3\,\widehat{w}_{\xi,3}(y,y')|$ \hspace{0.11\textwidth} (j) $\mathcal{R}e[10^3\,\widehat{w}_{\xi,3}(y,y')]$ \hspace{0.08\textwidth} (k) $\mathcal{I}m[10^3\,\widehat{w}_{\xi,3}(y,y')]$ \hspace{0.03\textwidth} (l) $10^3\,\widehat{w}_{\xi,3}(y,y'=y)$

    % \includegraphics[width=0.27\textwidth]{figure/rotating_laminar_cou_Re_358rotation_0_ssvd_20220804_N122w3_xiNy=120_abs.png}
    % \includegraphics[width=0.27\textwidth]{figure/rotating_laminar_cou_Re_358rotation_0_ssvd_20220804_N122w3_xiNy=120_real.png}
    % \includegraphics[width=0.27\textwidth]{figure/rotating_laminar_cou_Re_358rotation_0_ssvd_20220804_N122w3_xiNy=120_imag.png}
    % \includegraphics[width=0.135\textwidth,trim=-0 -0.7in 0 0]{figure/rotating_laminar_cou_Re_358rotation_0_ssvd_20220804_N122w3_xiNy=120_diag_abs_real_imag.png}
    
    \caption{$\widehat{u}_{\xi,1}$ (panels (a)-(d)), $\widehat{v}_{\xi,1}$ (panels (e)-(h)) and $\widehat{w}_{\xi,1}$ (panels (i)-(l)) of plane Couette flow at $Re=358$, $k_x=0.22$, $k_z=0.67$ and $\omega=0$. }
    \label{fig:u_xi_abs_real_imag_diag_1}
\end{figure}
\end{widetext}
\fi

%{\color{blue}shouldn't the symbols be bold in the bottom loop? this should mirror (11) and (12) right?}




% \widetilde{\mathcal{H}}_{\nabla,i}:=&\text{diag}\left(\widehat{\boldsymbol{\nabla}},\widehat{\boldsymbol{\nabla}},\widehat{\boldsymbol{\nabla}}\right)\mathcal{H}\boldsymbol{P}_i,\;\;i=1,2,3 \;\;\text{with}\\
%     \boldsymbol{P}_1=&\text{diag}(\mathcal{I}_{1\times 3}, \boldsymbol{0}_{1\times 3},\boldsymbol {0}_{1,\times 3}),\\
%     \boldsymbol{P}_2=&\text{diag}(\boldsymbol{0}_{1\times 3}, \mathcal{I}_{1\times 3}, \boldsymbol{0}_{1,\times 3}),\\
%     \boldsymbol{P}_3=&\text{diag}(\boldsymbol{0}_{1\times 3}, \boldsymbol{0}_{1,\times 3}, \mathcal{I}_{1\times 3}),




\begin{figure}
    \centering
    (a) $10^3\,\widehat{\mathbfsbilow{u}}_{\xi,1}$\hspace{0.05\textwidth} (b) $10^3\,\widehat{\mathbfsbilow{v}}_{\xi,1}$ \hspace{0.05\textwidth} (c) $10^3\,\widehat{\mathbfsbilow{w}}_{\xi,1}$
    
    \includegraphics[width=0.135\textwidth]{figure/rotating_laminar_poi_Re_690rotation_0_ssvd_20220804_N122u1_xiNy=120_diag_abs_real_imag.png}
    \includegraphics[width=0.135\textwidth]{figure/rotating_laminar_poi_Re_690rotation_0_ssvd_20220804_N122v1_xiNy=120_diag_abs_real_imag.png}
    \includegraphics[width=0.135\textwidth]{figure/rotating_laminar_poi_Re_690rotation_0_ssvd_20220804_N122w1_xiNy=120_diag_abs_real_imag.png}
    
    % (d) $10^3\,\widehat{\mathbfsbilow{u}}_{\xi,2}$\hspace{0.05\textwidth} (e) $10^3\,\widehat{\mathbfsbilow{v}}_{\xi,2}$ \hspace{0.05\textwidth} (f) $10^3\,\widehat{\mathbfsbilow{w}}_{\xi,2}$
    
    % \includegraphics[width=0.135\textwidth]{figure/rotating_laminar_poi_Re_690rotation_0_ssvd_20220804_N122u2_xiNy=120_diag_abs_real_imag.png}
    % \includegraphics[width=0.135\textwidth]{figure/rotating_laminar_poi_Re_690rotation_0_ssvd_20220804_N122v2_xiNy=120_diag_abs_real_imag.png}
    % \includegraphics[width=0.135\textwidth]{figure/rotating_laminar_poi_Re_690rotation_0_ssvd_20220804_N122w2_xiNy=120_diag_abs_real_imag.png}

    % (g) $10^3\,\widehat{\mathbfsbilow{u}}_{\xi,3}$\hspace{0.05\textwidth} (h) $10^3\,\widehat{\mathbfsbilow{v}}_{\xi,3}$ \hspace{0.05\textwidth} (i) $10^3\,\widehat{\mathbfsbilow{w}}_{\xi,3}$

    % \includegraphics[width=0.135\textwidth]{figure/rotating_laminar_poi_Re_690rotation_0_ssvd_20220804_N122u3_xiNy=120_diag_abs_real_imag.png}
    % \includegraphics[width=0.135\textwidth]{figure/rotating_laminar_poi_Re_690rotation_0_ssvd_20220804_N122v3_xiNy=120_diag_abs_real_imag.png}
    % \includegraphics[width=0.135\textwidth]{figure/rotating_laminar_poi_Re_690rotation_0_ssvd_20220804_N122w3_xiNy=120_diag_abs_real_imag.png}

    \caption{First three components of $\widetilde{\mathbfsbilow{u}}_{\Xi,c}$; i.e.,  $10^3\,\widehat{\mathbfsbilow{u}}_{\xi,1}$, $10^3\,\widehat{\mathbfsbilow{v}}_{\xi,1}$, and $10^3\,\widehat{\mathbfsbilow{w}}_{\xi,1}$ for plane Poiseuille flow at $Re=690$, $k_x=0.65$, $k_z=1.56$ and $c=-\omega/k_x=0.53$. 
    %Each component of $\widetilde{\mathbfsbilow{u}}_{\Xi,c}$ for plane Poiseuille flow at $Re=690$, $k_x=0.65$, $k_z=1.56$ and $c=-\omega/k_x=0.53$. 
    }
    \label{fig:u_xi_diag_2_3_poi}
\end{figure}