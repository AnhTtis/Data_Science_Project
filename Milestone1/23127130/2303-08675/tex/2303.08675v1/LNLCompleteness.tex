\documentclass[12pt]{amsart}
\usepackage{graphicx}
\title[Completeness of trajectories]{Completeness of trajectories associated to Apell hypergeometric functions}
\author[L. Boulton]{Lyonell Boulton$^\beta$} 
\address{$^\beta$Department of Mathematics and Maxwell Institute for Mathematical Sciences, Heriot-Watt University, Edinburgh, EH14 4AS.}
\email{L.Boulton@hw.ac.uk}
\date{15th March 2023}
\newcommand{\sgn}{\operatorname{sgn}}
\newcommand{\sn}{\operatorname{sn}}

\newtheorem{Lemma}{Lemma}
\newtheorem{Theorem}{Theorem}
\newtheorem{Corollary}{Corollary}


\begin{document}
\maketitle

\begin{abstract}
We examine the linear completeness of trajectories of eigenfunctions associated to non-linear eigenvalue problems, subject to Dirichlet boundary conditions on a segment. We pursue two specific goals. On the one hand, we establish that linear completeness persists for the non-linear Schr{\"o}dinger equation, even when the trajectories lie far from those of the linear problem. On the other hand, we show that this is also the case for a new fully non-linear eigenvalue equation which is naturally associated with Apell hypergeometric functions.  Both models shed new light on a framework for completeness in the non-linear setting, considered by L.E.~Fraenkel over 40 years ago, that may have significant potential but which does not seem to have received much attention. 
\end{abstract}

\vfill

\tableofcontents

\vfill


\newpage

\section{Introduction}
In two consecutive papers \cite{F1980a,F1980b} published at the end of the 1970s which appear to have been largely overlooked, L.E.~Fraenkel considered the questions of linear and non-linear completeness for a family of trajectories on a Hilbert space. Taking as one of two models for his investigations\footnote{The other model was the eigenvalue problem with a plus rather than a minus sign in front of the non-linearity.} the semi-linear eigenvalue problem
\begin{equation} \label{nl-schrod}
\begin{aligned} 
&u''-u^3+\lambda u=0 \\
&u(0)=u(1)=0,
\end{aligned}
\end{equation}
he established among several other remarkable results, that a collection of eigenfunctions $\{u_n\}_{n=1}^\infty\equiv \{u_n\}$ of \eqref{nl-schrod} is linearly complete in $L^2(0,1)\equiv L^2$, when subject to a control on the growth of the norms of the $u_n$.  

In terms of the Jacobi elliptic functions, $\sn(y,\mu)$, this family is given explicitly by 
\begin{equation} \label{nl-schrod_efu}
u(x)=u_n(x,\mu)=2^{\frac32}n \mu K(\mu) \sn(2K(\mu)nx,\mu) 
\end{equation} with associated eigenvalues \[ \lambda=\lambda_n=4n^2(1+\mu^2)K(\mu)^2.\] 
Here the modulus $\mu$ lies in $(0,1)$ and it is a free parameter, while the 1/4 period $K(\mu)$ is the complete elliptic integral. According to \cite[Theorem~3.3(i)]{F1980a}, if $\{\mu_n\}_{n=1}^\infty\subset (0,1)$  is a sequence for which 
\begin{equation} \label{growthEF}
    \{\|u_n(\cdot,\mu_n)\|/n\}\in \ell^4(\mathbb{N}),
\end{equation}
then $\{u_n\}$ is a basis of $L^2$. 

A key component in the proof of this result, is the fact that \eqref{growthEF} is equivalent to
\[
      \sum_{n=1}^\infty \left\|\frac{u_n(\cdot,\mu_n)}{n \gamma_n}-e_n\right\|^2<\infty
\]
for $e_n(x)=2^{\frac12} \sin(n\pi x)$ and $\gamma_n$ the $n$-th sine Fourier coefficient of $u_n/n$. 
This is known to be sufficient, but not necessary, for $\{u_n\}$ to become a basis of $L^2$. Moreover, although it allows $\|u_n(\cdot,\mu_n)\|$ to be $O(n^c)$ as $n\to\infty$ for all $c<3/4$, \eqref{growthEF} holds if and only if $\{\mu_n\}\in \ell^4(\mathbb{N})$. Therefore, it requires $\mu\to 0$. In this regime, the equation \eqref{nl-schrod} bifurcates from the linear eigenvalue equation and it is natural to expect that $\{u_n\}$ is close enough to an orthonormal basis of $L^2$. 
 
The present paper is devoted to two specific goals in the context of Fraenkel's original idea of asking questions about linear and non-linear completeness for trajectories of eigenfunctions. On the one hand, we show that  $\{u_n\}$ is also a basis for any $\{\mu_n\}\subset (0,1)$ such that $\sup \mu_n \leq \mu_0$, where $\mu_0<1$ is a constant very close to 1. Concretely, we show that $\{u_n/n\}$ is a Riesz basis in the sense that it is equivalent to the orthonormal basis $\{e_n\}$. On the other hand, we examine linear completeness on a fully non-linear version of \eqref{nl-schrod}. The eigenvalue equation {\small \begin{equation} \label{p-nl-schrod}
\begin{aligned} 
&(\sgn(\phi')|\phi'|^{p-1})'-(p-1)\sgn(\phi)|\phi|^{2p-1}+\lambda (p-1)\sgn( \phi)|\phi|^{p-1}=0 \\
&\phi(0)=\phi(1)=0,
\end{aligned}
\end{equation}}for $p>1$, in which we have replaced the Laplacian term by the non-linear $p$-Laplacian.

This new equation is neither artificial nor it is a purposeless generalisation of \eqref{nl-schrod}. Rather, it provides a link between the framework of the papers \cite{F1980a,F1980b} and the one developed recently for the $p$-Laplacian and other families of dilated periodic functions arising in the context of Sobolev embeddings on the segment. A systematic treatment of the latter can be found in  the book \cite{EL2011} and the state-of-the-art on the basis question for the $p$-Laplacian and other related non-linear operators can be found in the papers \cite{BE2012} and \cite{BL2015}. For a full list of updated references on the subject and a more general perspective, see also the paper \cite{BM2018}. 

We thank Domenic Petzinna for his useful comments and attentive reading of an earlier version of this work. The background concepts and terminology employed below can be found in  the books \cite{H2011} and \cite{WW1920}.

\section{The eigenvalue problem}

Let $p>1$. We examine completeness of the solutions of the eigenvalue equation \eqref{p-nl-schrod}. The case $p=2$ corresponds to equation \eqref{nl-schrod}. Our interest is bases properties near and far from bifurcations occurring at the eigenvalue problem 
\begin{equation} \label{p-schrod}
\begin{aligned} 
&(\sgn(\phi')|\phi'|^{p-1})'+\lambda (p-1)\sgn( \phi)|\phi|^{p-1}=0 \\
&\phi(0)=\phi(1)=0.
\end{aligned}
\end{equation}
We know that the eigenfunctions of \eqref{p-schrod} become a Riesz basis of $L^2$ for all $p>p_0$, where $p_0>1$ is a constant close to 1. Specific information about this constant can be found in \cite[Theorem~4.5]{BE2012} and \cite[Theorem~6.5]{BL2015}.

The solutions to \eqref{p-nl-schrod} are given in terms of inverse Apell hypergeometric functions, as follows. For (general) modulus $\mu\in(0,1)$, let 
\[
    K_p(\mu)=\int_0^1 (1-s^p)^{-\frac{1}{p}}(1-\mu^p s^p)^{-\frac{1}{p}}\, \mathrm{d}s
\]
and let $\sn_p(y,\mu)$ be the odd $4K_p(\mu)$-periodic continuous extension of the inverse function of
\[
      w_p(z)=\int_{0}^z (1-s^p)^{-\frac{1}{p}}(1-\mu^p s^p)^{-\frac{1}{p}}\, \mathrm{d}s.
\]
Then, $\sn_p(y,\mu)$ is a positive increasing function for $y\in\big(0,K_p(\mu)\big)$. The re-scaled function $\sn_p(2K_p(\mu)x,\mu)$ is a 2-periodic odd differentiable function. It is positive on $(0,1)$ with maximum value equal to 1 at $x=\frac12$. It is also even with respect to that point.



\begin{Theorem} \label{Theorem1}
Let $p>1$. A non-zero differentiable function $\phi:\mathbb{R}\longrightarrow \mathbb{R}$ satisfies the equation \eqref{p-nl-schrod} for some $\lambda>0$ if and only if the following holds true. For a unique pair $(\mu,n)\in (0,1)\times \mathbb{N}$,
\[\phi(x)=\pm 2^{\frac{p+1}{p}}n \mu K_p(\mu) \sn_p(2nK_p(\mu)x,\mu) \] and \[ \lambda=2^{p}n^p(1+\mu^p)K_p(\mu)^p.\]
\end{Theorem}

Note that  $n$ determines the number of zeros of the eigenfunction in the segment $(0,1)$ and, alongside with $\mu$ and $p$, it determines its norm.
 
\begin{proof}
Consider a positive solution $\phi(x)$ satisfying \eqref{p-nl-schrod} for some $\lambda>0$, such that $\phi(x)>0$ for all $x\in (0,1)$.  Then, 
\[
     \begin{aligned}
&[(\phi')^{p-1}]'-(p-1)\phi^{2p-1}+\lambda (p-1) \phi^{p-1}=0, \\
&\phi(0)=\phi(1)=0.
\end{aligned}
\] 
Multiply each summing term of the equation by $\frac{p}{p-1}\phi'$, to get
\[
     \frac{p}{p-1}\phi'[(\phi')^{p-1}]'=[(\phi')^{p}]', \qquad \frac{p}{p-1}\phi'(p-1)\phi^{2p-1}=\frac12 (\phi^{2p})' 
\]
and
\[
     \frac{p}{p-1}\phi'\lambda (p-1)\phi^{p-1}=\lambda (\phi^p)'.
\]
Integrating the new equation, gives
\[
    (\phi')^p=\frac12 \phi^{2p} -\lambda \phi^p+c^p=\frac12 (\alpha-\phi^p)(\beta-\phi^p)
\]
for $c=\phi'(0)>0$ and 
\[
     \alpha,\,\beta=\lambda \pm \sqrt{\lambda^2-2c^p}.
\]
Here $\alpha$ picks the `$+$' sign and $\beta$ the `$-$' sign.
Then,
\begin{equation} \label{ch-equ-int}
    \phi'=2^{-\frac{1}{p}} \left[(\alpha-\phi^p)(\beta-\phi^p)\right]^{\frac{1}{p}}.
\end{equation}

Since $\phi(0)=\phi(1)=0$, there exists $x_0\in(0,1)$ such that $\phi'(x_0)=0$. As $\phi$ is positive by our assumption, it is then increasing from $x=0$. That is,  $\phi'(x)>0$ for all $x\in [0,x_0)$ and $0<\beta \leq \alpha$ (both roots should be positive and real). Also,
\[
     \phi^p(x_0)=\beta >0 \qquad \text{and} \qquad \lambda^2\geq 2c^p >0.
\]  
Let $a^p=\alpha$ and $b^p=\beta$. Integrating \eqref{ch-equ-int}, gives
\[
       x=\int_{0}^\phi (a^p-t^p)^{-\frac{1}{p}}(b^p-t^p)^{-\frac{1}{p}}\,\mathrm{d}t
\]
for $0\leq x(\phi)\leq x_0$ and $0\leq \phi(x)\leq b$, both increasing. Writing
\[
       x=\frac{2^{\frac{1}{p}}}{ab}\int_0^\phi \left(1-\left(\frac{t}{a}\right)^{p}\right)^{-\frac{1}{p}}
\left(1-\left(\frac{t}{b}\right)^{p}\right)^{-\frac{1}{p}}\,\mathrm{d}t
\] 
and changing variables to $s=\frac{t}{b}$, then calling $\mu=\frac{b}{a}$, gives
\[
      x=\frac{2^{\frac{1}{p}}}{a}
\int_{0}^{\frac{\phi}{b}} (1-\mu^ps^p)^{-\frac{1}{p}} (1-s^p)^{-\frac{1}{p}}\,\mathrm{d}s.
\]
Hence, with $w_p(z)$ as above,
\[
     w_p\left(\frac{\phi(x)}{b}   \right)=\frac{ax}{2^{\frac{1}{p}}}.
\]
The function $\phi(x)$ attains its maximum at $x=x_0$ and $\phi(x_0)=b$.
With the definition of $K_p(\mu)$ as above,
this amounts to
\[
      K_p(\mu)=\frac{ax_0}{2^{\frac{1}{p}}}.
\]

By Corollary~\ref{Corollary6}, formulated in the final section of this paper, we know that $\phi(x)$ should be $4x_0$-periodic and $\phi(2x_0)=0$. Moreover, $\phi$ should be odd with respect to $2kx_0$ and even with respect to $(2k+1)x_0$ for all $k\in\mathbb{Z}$. Hence, necessarily,
\[
    \phi(2x_0)=0.
\]  

Now, because $2x_0$ is the first zero of $\phi$, then $x_0=\frac12$. Thus
\[
     a=2^{\frac{p+1}{p}}K_p(\mu).
\]
Therefore, recalling that $b=a\mu$, 
\[
    \phi(x)=b\sn_p\left(\frac{ax}{2^{\frac{1}{p}}},\mu\right)=
2^{\frac{p+1}{p}}K_p(\mu)\mu \sn_p\left(2K_p(\mu)x,\mu\right)
\]
where $\sn_p(y,\mu)$ is the inverse function of $w_p(z)$ with maximal segment of definition and periodicity as stated above. This determines the expression for an eigenfunction $\phi(x)$ satisfying the Dirichlet boundary conditions with no zeros in $(0,1)$. It also shows that this eigenfunction is unique. 

Since
\[
     \lambda=\frac{a^p+b^p}{2}=2^p (1+\mu^p)K_p(\mu)^p
\]
we also obtain the expression for the eigenvalue in terms of $p$ and the parameter $\mu$ for the case $n=1$. So, we know that the eigenpair $(u,\lambda)$ is determined uniquely from the pair $(\mu,1)\in(0,1)\times \mathbb{N}$.

Finally, for $n>1$, evaluating directly the equation \eqref{p-nl-schrod} in $\phi(nx)$, which also satisfies the Dirichlet boundary conditions at $x=0$ and $x=1$, give the general expression for the eigenpairs $(\phi,\lambda)$ in terms of $(\mu,n)$, as stated in the theorem. Corollary~\ref{Corollary6} ensures that this solution is unique too, for $\phi'(0)>0$. The two signs choice in the conclusion is a consequence of the fact that if $\phi(x)$ is an eigenfunction, then also $-\phi(x)$ is.
\end{proof}


Note that
\[
     K_p(\mu)=\int_0^1 \frac{x^{\frac{1}{p}-1}}{p}(1-x)^{-\frac{1}{p}}(1-\mu^p x)^{-\frac{1}{p}} \,\mathrm{d}x=\frac{\operatorname{B}\left(\frac{1}{p},\frac{1}{p'}\right)}{p} \ _2\!\operatorname{F}_1\left(\frac{1}{p},\frac{1}{p};1,\mu^p    \right)
\]
and
\[
    w_p(z)=zF_1\left(\frac{1}{p},\frac{1}{p},\frac{1}{p},\frac{p+1}{p};z^p,\mu^pz^p    \right).
\]
Therefore, $w_p(z)$ is associated with the Appell hypergeometric functions \cite{E1950}.

Now, turning back to the function $\sn_p(y,\mu)$, let us describe some of its structural  properties. Since
\[
    \frac{\mathrm{d}(\sn_p(y,\mu))}{\mathrm{d}y}=(1-[\sn_p(y,\mu)]^p)^{\frac{1}{p}} (1-\mu^p [\sn_p(y,\mu)]^p)^{\frac{1}{p}}
\]
for all $y\in[0,K_p(\mu)]$, then $\sn_p(y,\mu)$ and its periodic extension are continuously differentiable on $\mathbb{R}$. Moreover, 
\[
     \frac{\mathrm{d^2}(\sn_p(y,\mu))}{\mathrm{d}y^2}=h_p(\sn_p(y,\mu))
\]
where
\[
       h_p(z)=z^{p-1}(1-z^p)^{\frac{2}{p}-1}(1-\mu^pz^p)^{\frac{2}{p}-1}
       ((\mu^p+1)z^p-2)<0
\]
for all $z\in(0,1)$. Then, for the periodic extension (and $p\not=2$), $\sn_p''(y,\mu)<0$ for $y\in[0,K_p(\mu))$ and $\sn_p''(y,\mu)>0$ whenever $y\in (K_p(\mu),2K_p(\mu)]$. At the 1/4-period $y=K_p(\mu)$ and at its odd integer multiples, the second derivative is continuous for all $1<p\leq 2$ but it has a singularity for all $p>2$. However, this second derivative is always locally $L^1$, because 
\[
     \int_{K_p(\mu)-\epsilon}^{K_p(\mu)+\epsilon} \!|\sn_p''(y,\mu)|\,\mathrm{d}y=-\sn_p'(y,\mu)\Big|_{y=K_p(\mu)-\epsilon}^{y=K_p(\mu)}\!\!\!+ \sn_p'(y,\mu)\Big|_{y=K_p(\mu)}^{y=K_p(\mu)+\epsilon}\!\!\!<\infty.
\]
That is,
\begin{equation}\label{diffdiff_in_L1loc}
\sn_p''(\cdot,\mu)\in \begin{cases} C(\mathbb{R}) & 1<p\leq 2 \\
L^1_{\mathrm{loc}}(\mathbb{R}) & p>2.\end{cases}\end{equation}
All these basic properties will be invoked in the proof of our main theorem below. 

The next lemma can be regarded as a version of the classical Jordan Inequality. The case corresponding to $\mu=0$ was established in \cite[Proposition~2.3]{BE2012}. The proof mimicks the latter.

\begin{Lemma} \label{ineq_snp_lin}
For all $y\in\left(0,K_p(\mu)\right)$,
\[
     \frac{1}{K_p(\mu)} \leq \frac{\sn_p(y,\mu)}{y} \leq 1.
\]
\end{Lemma}
\begin{proof}
    Changing variables to $s=zr$ in the integral defining $w_p(z)$, 
\[
    w_p(z)=z\int_0^1 (1-y^pr^p)^{-\frac{1}{p}}(1-\mu^pz^pr^p)^{-\frac{1}{p}}\,\mathrm{d}r.
\]
Then, substituting $y=w_p(z)$, gives
\[
     y=\sn_p(y,\mu)\int_0^1 (1-\sn_p^p(y,\mu)r^p)^{-\frac{1}{p}}(1-\mu^p\sn_p^p(y,\mu)r^p)^{-\frac{1}{p}} \, \mathrm{d}r.
\]
Denote the integral on the right hand side by $A$. Since $0<\sn_p(y,\mu)<1$, 
\[
     1\leq A\leq \int_{0}^1(1-r^p)^{-\frac{1}{p}}(1-\mu^p r^p)^{-\frac{1}{p}} \, \mathrm{d}r=K_p(\mu).
\]
\end{proof}


For $\underline{\mu}=\{\mu_n\}_{n=1}^\infty\in (0,1)^{\infty}$, we write
\[
    f_n(x)\equiv f_{n,\underline{\mu}}(x)=\sn_p(2K_p(\mu_n)nx,\mu_n).
\]
Whenever it is sufficiently clear from the context, we leave implicit the dependence of $\{f_n\}$ on $\underline{\mu}$. Then,
\[
     \phi_n(x)=2^{\frac{p+1}{p}}\mu_n nK_p(\mu_n) f_n(x) 
\]
form a collection of eigenfunctions of \eqref{p-nl-schrod}, for $n\in\mathbb{N}$. Our next statement is the first main contributions of this paper. It gives sufficient conditions on $\underline{\mu}$ for $\{f_{n,\underline{\mu}}\}$ to be a Riesz basis of $L^2(0,1)$. 

\begin{figure}[t]
\includegraphics[width=60mm]{Theorem1_reg}
\caption{\label{Figure1}In the shaded region, $K_{p}(\mu)<\frac{8}{\pi^2-8}$. The numerical approximations employed might not be too accurate for $\mu$ near 1. But away from that region, the graph gives a sharp approximation illustrating the interplay between the two parameters for the conclusion of Theorem~\ref{fn_basis} to hold true.}
\end{figure}

\begin{Theorem} \label{fn_basis}
 Let $p>1$ and $\underline{\mu} \in (0,1)^{\infty}$. If
\begin{equation} \label{firstcond}
      \sup_{n\in\mathbb{N}}K_p(\mu_n)<\frac{8}{\pi^2-8},
\end{equation}
then $\{f_{n,\underline{\mu}}\}$ is a Riesz basis of $L^2(0,1)$.
\end{Theorem}
\begin{proof}
We split the argument into five different steps. 

\underline{Step~1}. We describe explicitly the linear operator $A:e_n\longmapsto f_{n,\underline{\mu}}$ in terms of isometries and diagonal operators of $L^2$. 

Let
\[
    \tau_{k}(n)=\sqrt{2}\int_0^1 \sn_p(2K_p(\mu_n)x,\mu_n) \sin(k\pi x)\,\mathrm{d}x
\]
be the $k$-th sine Fourier coefficient of the function $\sn_p(2K_p(\mu_n)\cdot,\mu_n)$. Since the latter is even with respect to $x=\frac12$, then $\tau_j(n)=0$ for all even index $j$ and all $n\in\mathbb{N}$. Let the isommetries $M_k:L^2\longrightarrow L^2$ be given by $M_{k}e_n(x)=e_{kn}(x)$. Let the diagonal operators $A_k:L^2\longrightarrow L^2$ be given by $A_k=\operatorname{diag}[\tau_k(n):n\in\mathbb{N}]$. That is, we fix the index $k$ of the Fourier coefficient and move the index $n$ of the entries of $\underline{\mu}$.

Since
\[
     f_{n,\underline{\mu}}(x)=\sum_{k=1}^\infty \tau_k(n) M_k e_n(x)=
 \sum_{k=1}^\infty  M_k A_k e_n(x),
\]
and both families of operators are bounded, then
\[
     A=\sum_{k=1}^\infty M_kA_k.
\]
Here the series is absolutely covergent in operator norm. This follows from the arguments given in Step~4 below, as these arguments show that
\[
    \sum_{k=1}^\infty \|M_k A_k\|=\sum_{k=1}^\infty \|A_k\|<\infty.
\]

\underline{Step~2}. We claim that
\[
       \tau_1(n)\geq \frac{4\sqrt{2}}{\pi^2} \qquad \qquad \forall n\in\mathbb{N}
\]
irrespective of the choice of $\underline{\mu}$. Indeed, by substituting $y=2xK_p(\mu_n)$ in Lemma~\ref{ineq_snp_lin}, follows that
\begin{align*}
   \tau_1(n)& = 2\sqrt{2}\int_0^{\frac12} \sn_p(2 K_p(\mu_n)x,\mu_n) \sin(\pi x) \,\mathrm{d}x \\
&\geq 4\sqrt{2}  \int_0^{\frac12} x  \sin(\pi x)\,\mathrm{d}x =\frac{4\sqrt{2}}{\pi^2}.
\end{align*}

\underline{Step~3}. We next show that
\[
    \sum_{\substack{k=3\\ k\equiv_2 1}} \sup_{n\in \mathbb{N}}|\tau_k(n)| \leq \frac{4 \sqrt{2}}{\pi^2} \left(\frac{\pi^2}{8}-1 \right)\sup_{n\in \mathbb{N}}K_p(\mu_n).
\]
For this purpose, let $g(x)=\sn_p(2K_p(\mu_n)x,\mu_n)$ and recall the regularity properties of $\sn_p(y,\mu)$ given in \eqref{diffdiff_in_L1loc}. For $k\geq 3$ odd, integrating by parts twice and noting that $g'(\frac12)=0$, gives 
\[
    \tau_k(n)=\frac{2 \sqrt{2}}{n^2 \pi^2} \int_0^{\frac12} g''(x) \sin(k\pi x)\,\mathrm{d}x.
\] 
Hence, since $g''(x)<0$ for all $x\in (0,\frac12)$, 
\begin{align*}
|\tau_k(n)|&\leq  \frac{2 \sqrt{2}}{k^2 \pi^2} \int_0^{\frac12} |g''(x)| \,\mathrm{d}x =
\frac{2 \sqrt{2}}{k^2 \pi^2} \left(g'(0)-g'\Big(\frac12\Big)\right) \\
&= \frac{4 \sqrt{2}K_p(\mu_n)}{k^2 \pi^2}
[\sn_p'(0,\mu_n)-\sn_p'(K_p(\mu_n),\mu_n)]=
\frac{4 \sqrt{2}K_p(\mu_n)}{k^2 \pi^2}. 
\end{align*}
By taking the suprema in $n$ and then the summation in the index $k$, this yields the claim made above.

\underline{Step~4}. If
\begin{equation} \label{hypo_invert}
     \sum_{\substack{k=3\\ k\equiv_2 1}} \sup_{n\in \mathbb{N}} |\tau_k(n)|<\inf_{n\in \mathbb{N}}\tau_1(n),
\end{equation}
then $A$ is an invertible operator. Note that the right hand side of this inequality is always positive according to the step~2. 

Assume that \eqref{hypo_invert} holds true. To show that $A$ is invertible, firstly note that the left hand side of the inequality above equals 
\[
      \sum_{\substack{k=3\\ k\equiv_2 1}} \|M_kA_k\|.
\]
Indeed, 
\[
    \|A_k\|=\sup_{n\in \mathbb{N}} |\tau_k(n)|. 
\]
And, since $M_k$ are isommetries, for all $\varepsilon>0$ there is $\tilde{v}\in L^2$, such that $\|\tilde{v}\|=1$ and
\[
    \|M_kA_k\tilde{v}\|=\|A_k\tilde{v}\|> \|A_k\|-\varepsilon.
\]
Then $\|A_k\|\geq \|M_kA_k\|\geq \|A_k\|-\varepsilon$. Taking $\varepsilon\to 0$ gives 
\begin{equation}  \label{normlhs}
    \sum_{\substack{k=3\\ k\equiv_2 1}} \sup_{n\in \mathbb{N}} |\tau_k(n)|=
\sum_{\substack{k=3\\ k\equiv_2 1}}^\infty \|M_kA_k\|.
\end{equation}

Now, since $\inf_{n\in \mathbb{N}}\tau_1(n)>0$ and
\[
     \|A_1^{-1}\|= \frac{1}{\inf_{n\in \mathbb{N}}\tau_1(n)},
\]
then $A_1$ is invertible. Note also that $M_1=I$, the identity operator. Then, 
\[
   A=M_1A_1+\sum_{\substack{k=3\\ k\equiv_2 1}}^\infty M_kA_k=
   A_1\Big(I+A_1^{-1}\sum_{\substack{k=3\\ k\equiv_2 1}}^\infty M_kA_k\Big).
\]
Moreover, from \eqref{normlhs} and the hypothesis \eqref{hypo_invert}, we have
\[
    \Big\|A_1^{-1}\sum_{\substack{k=3\\ k\equiv_2 1}}^\infty M_kA_k)\Big\|\leq \|A_1^{-1}\|\sum_{\substack{k=3\\ k\equiv_2 1}}^\infty\|M_kA_k\|<1.
\]
Hence, indeed $A$ is invertible. 

\underline{Step~5}. According to step~3, \eqref{firstcond} implies that  
\[
    \sum_{\substack{k=3\\ k\equiv_2 1}} \sup_{n\in \mathbb{N}} |\tau_k(n)|<\frac{4\sqrt{2}}{\pi^2}.
\]
But from step~2, we know that \eqref{hypo_invert} holds true. Therefore, as $A$ is invertible and $A:e_{n}\longmapsto f_n$, we have $\{f_n\}$ equivalent to the orthonormal basis $\{e_n\}$.
\end{proof}

Note that the condition \eqref{firstcond} holds for $p\approx 2$ and $\sup \mu_n$ small enough, not necessarily approaching zero as $n\to\infty$. Indeed $K_p(\mu)\approx \pi/2$ under these conditions. Figure~\ref{Figure1} shows the interplay between the parameters $\mu$ and $1/p$ for the condition of the theorem to be verified from a computer approximation of $K_p(\mu)$.

The proof of the next statement follows in a straightforward manner from Theorem~\ref{fn_basis} by  changing a finite number of terms in the sequence $\mu_n$ and re-scaling $f_{n,\underline{\mu}}$.  

\begin{Corollary} \label{corollary1}
If $\{\mu_n\}_{n=1}^\infty\subset (0,1)$ is such that 
\[
    \limsup_{n\to \infty} K_p(\mu_n)<\frac{8}{\pi^2-8},    
\] 
then the family $\{
\phi_n(\cdot,\mu_n)\}$ of eigenfunctions of \eqref{p-nl-schrod} form a basis of $L^2(0,1)$.
\end{Corollary}


%%%%%%%%%%%%%%%%%%%%%%%%%%%%%%%%%

\section{The semi-linear case}

We now focus on the case $p=2$. Lemma~3.2(i) and Theorem~3.3(i) of \cite{F1980a}, establish conditions for the eigenfunctions of \eqref{nl-schrod} to be a basis of $L^2$ by means of a different criterion than the one invoked in the proof of Theorem~\ref{fn_basis} above. These conditions are given in terms of a parameter, 
\[
     s=\pm\frac{4\pi q^{\frac12}}{1-q}\in\mathbb{R}
\]
where $q$ is the nome and the sign convention matches that of the eigenfunction. See \cite[(2.9)]{F1980a}.
Concretely, for the sequence $s=r_n$,  we know that
\begin{equation}  \label{eq3}
      \sum_{n=1}^\infty \left\|\frac{u_n}{n \langle u_n,e_n\rangle}-e_n\right\|^2<\infty
\end{equation}
 if and only if $\{r_n\}\in\ell^4$. This, alongside with $\omega$-linear independence, ensures that the family $\{\frac{u_n}{n \langle u_n,e_n\rangle}\}_{n=1}^\infty$ is a Riesz basis of $L^2$. The latter can, for example, be derived directly from the proof of \cite[Theorem~2.20, p.265]{K1980}. 

In \cite{F1980a}, the choice of the alternative parameter $s$ was convenient so to confirm the validity of \eqref{eq3}. In terms of the modulus $\mu\in(0,1)$,  \cite[p486]{WW1920} we know that  $\mu\sim q^{\frac12}$ as $q\to 0$. Hence, \eqref{eq3} is equivalent to the choice of $\mu=\mu_n$ in each of the bifurcation curves to be $\{\mu_n\}\in \ell^4$ also. As we can deduce from Theorem~\ref{fn_basis}, in the case $p=2$, the latter is sufficient but not necessary, for the family $\big\{\frac{u_n}{n \langle u_n,e_n\rangle}\big\}$ to become a Riesz basis. Our main goal now is to show that this family is in fact a Riesz basis for any choice of $\mu_n\in(0,1)$ such that $\sup \mu_n< \mu_0$ where $\mu_0$ is substantially closer to $1$. 


The Jacobi elliptic function has Fourier expansion
\[
    \operatorname{sn}\big(2K(\mu)x\big)=\frac{2\pi q^{\frac12}}{K(\mu)\mu}\sum_{j=0}^\infty \frac{q^j}{1-q^{2j+1}} \sin\big((2j+1)\pi x\big).
\]
Let
\begin{equation} \label{eq4}
     g(x,\mu)\equiv g(x)=(1-q)\sum_{j=0}^\infty \frac{q^j}{1-q^{2j+1}}e_{2j+1}(x).
\end{equation}
Then, the eigenfunctions \eqref{nl-schrod_efu} of \eqref{nl-schrod} are
\[
     u_n(x)=\frac{2^{\frac52}\pi n q^{\frac12}}{1-q}g(nx,\mu).
\]
In order to establish an improvement to the statement of Theorem~\ref{fn_basis} for this specific case, we first characterise the summation of the Fourier coefficients of $g(x)$. 

For $0<\beta<1$, let  the Lambert series
\[
   L(\beta)=\sum_{n=1}^\infty \frac{\beta^n}{1-\beta^n}.
\]
Consider the $q$-digamma function
\[
     \psi_q(x)=\frac{\mathrm{d}}{\mathrm{d}x}\log \Gamma_q(x)=\frac{\Gamma_q'(x)}{\Gamma_q(x)}
\]
where 
\[
    \Gamma_q(x)=(1-q)^{1-x}\prod_{n=0}^{\infty} \frac{1-q^{n+1}}{1-q^{n+x}}
\]
is the $q$-gamma function. Then 
\[
     \psi_q(x)=-\log(1-q)+(\log q)\sum_{n=1}^\infty \frac{q^{nx}}{1-q^n},
\]
for all $x>0$ and $q\in(0,1)$. See \cite[(1.5)]{AG2007}. Hence,
\[
      L(\beta)=\frac{\psi_\beta(1)+\log(1-\beta)}{\log(\beta)}.
\]

We claim that 
\[ 
    \sum_{n=0}^\infty \frac{\beta^{2n+1}}{1-\beta^{4n+2}}=L(\beta)-2L(\beta^2)+L(\beta^4).
\]
Indeed, 
\[
    C=L(\beta)-L(\beta^2)=\sum_{n=0}^\infty \frac{\beta^{4n+1}}{1-\beta^{4n+1}}+\frac{\beta^{4n+3}}{1-\beta^{4n+3}}
\]
and
\[
    D=L(\beta^4)-L(\beta^2)=-\sum_{n=0}^\infty \frac{\beta^{4n+2}}{1-\beta^{4n+2}}
\]
add up to
\[
   C+D=\sum_{n=0}^\infty \frac{(1+\beta^{2n+1})\beta^{2n+1}-\beta^{4n+1}}{1-\beta^{4n+2}}
\]
which is equal to the left hand side of the above claim. From it, we then gather that
\begin{equation}   \label{SumLambert}
     \sum_{n=1}^\infty \frac{q^n}{1-q^{2n+1}}=\frac{L(\sqrt{q})-2L(q)+L(q^2)}{\sqrt{q}}-\frac{1}{1-q}.
\end{equation}
This identity will now be crucial in the proof of the next theorem, which is the other main contribution of this paper.

\begin{Theorem} \label{casep2}
Let $q_0\in(0,1)$ be such that
\begin{equation} \label{sharp}
      \frac{L(\sqrt{q_0})-2L(q_0)+L(q_0^2)}{\sqrt{q_0}}=\frac{2}{1-q_0}
\end{equation}
and let
\[
    \mu_0=\frac{\vartheta_2^2(0,q_0)}{\vartheta_3^2(0,q_0)}.
\] 
Let $\{\mu_n\}_{n=1}^\infty\subset (0,1)$. If $\sup \mu_n< \mu_0$, then $\{g(n\cdot,\mu_n)\}_{n=1}^\infty$ is a Riesz basis of $L^2$.
\end{Theorem}
\begin{proof}
We proceed as in the proof of Theorem~\ref{fn_basis}. Let
\[
    \rho_k(n)=\langle g(\cdot,\mu_n),e_k\rangle.
\]  
That is, the $j$-th Fourier coefficient of the function $g(x,\mu_n)$. Let
\[
     B_k=\operatorname{diag}[\rho_k(n)\,:\,n\in\mathbb{N}].
\]
Then, carrying over the notation from the step~1 of the proof of Theorem~\ref{fn_basis}, we have that
$
       g(nx,\mu_n)=Be_n(x)
$
for all $n\in\mathbb{N}$,
where
\[
     B=\sum_{k=1}^\infty M_kB_k.
\]
According to \eqref{eq4}, $\rho_1(\mu_n)=1$ for all $n\in \mathbb{N}$. Then,
$B_1=I$. The proof of the present theorem reduces to showing that $B$ is an invertible bounded operator acting on $L^2$.

Arguing as in step~4 of the proof of Theorem~\ref{fn_basis}, if
\begin{equation} \label{trianglecasep2}
      \sum_{\substack{k=3 \\ k\equiv_2 1}} \sup_{n\in\mathbb{N}} \rho_k(n)<1,
\end{equation}
then $B$ is invertible. We now confirm this inequality.

Let $\mu_0$ be as in the hypothesis. Then \cite[p.486]{WW1920}, the nome associated to $\mu_0$ is $q_0$ satisfying \eqref{sharp}. For each fixed $j\in \mathbb{N}$, by differentiating with respect to $q$, it is straightforward to see that the function
\begin{equation} \label{monoto}
      q\longmapsto \frac{(1-q)q^j}{1-q^{2j+1}}=\frac{1}{q^{-j}+\cdots+q^{-1}+1+q^1+\cdots+q^j}
\end{equation}
is increasing as $q$ increases.  Then,
\[
     \sup_{n\in\mathbb{N}} \rho_{2j+1}(n)< \frac{(1-q_0)q_0^j}{1-q_0^{2j+1}},
\]
for all $j\in \mathbb{N}$. Now, let
\[
     S=\sum_{j=1}^\infty \frac{(1-q_0)q_0^j}{1-q_0^{2j+1}}.
\]
According to \eqref{SumLambert} and clearing from \eqref{sharp}, $S=1$. Hence, indeed
 \eqref{trianglecasep2} holds true and the theorem is valid.
\end{proof}

This theorem implies that,  whenever $p=2$, the conclusion of Corollary~\ref{corollary1} holds true for $\{\mu_n\}_{n=1}^\infty\subset(0,1)$ such that $\limsup \mu_n< \mu_0$.  Two comments are now in place. 

Firstly, note that the condition \eqref{sharp} is optimal in the following precise sense. Due to the monotonicity in $q$ of the terms \eqref{monoto}, the inequality \eqref{trianglecasep2} reverses for $\mu>\mu_0$ and the argument leading to the invertibility of $B$ is no longer valid.

Secondly, the condition \eqref{firstcond} for $p=2$ holds true, only for $q\in(0,0.315323)$ which corresponds to $\mu\in(0,0.996912)$. By contrast, from numerical estimations of the $q$-digamma function and substitution, \eqref{sharp} holds true for $q_0\approx 0.768062$. This corresponds to
$1-\mu_0<10^{-7}$. Therefore, Theorem~\ref{casep2} significantly improves the general Theorem~\ref{fn_basis} for $p=2$. 

%%%%%%%%%%%%%%%%%%%%%%%%%%%%%%%%%%%%%%%%%%%%%%%%%%%%%%

\section{Uniqueness and symmetries of periodic solutions}
This final section settles the uniqueness of periodic solutions associated to \eqref{p-nl-schrod}. Corollary~\ref{Corollary6} below completes the claim made in the proof of Theorem~\ref{Theorem1}.

Let $c>0$ and $\lambda\geq \sqrt{2}c^{\frac{p}{2}}$. Consider the initial value problem
\begin{equation} \label{1}
   \begin{aligned} &\phi'(x)=\left[\frac12 \phi^{2p}(x)-\lambda \phi^p(x)+c^p\right]^{\frac{1}{p}} \\ &\phi(0)=0, \end{aligned}
\end{equation}
which led to \eqref{ch-equ-int}. 

\begin{Lemma} \label{Lemma1}
The equation \eqref{1} has a unique solution $\phi:[0,\epsilon)\longrightarrow [0,\infty)$ for sufficiently small $\epsilon>0$. This solution is increasing and it can be extended uniquely to an increasing solution  $\phi:[0,x_0)\longrightarrow [0,\infty)$ for maximal $x_0>0$. If $x_0<\infty$, then
\[
    \lim_{x\to x_0}\phi'(x)=0.
\]
\end{Lemma}
\begin{proof}
Let
\[F(x,y)=\Big(\frac12 y^{2p}-\lambda y^p+c^p\Big)^{\frac{1}{p}}=2^{-\frac{1}{p}} \big((\alpha-y^p)(\beta-y^p)\big)^{\frac{1}{p}}.\]
where $\alpha,\,\beta=\lambda \pm \sqrt{\lambda^2-2c^p}$ and the choice of signs is as above so $\beta$ is the first root. Then, $F:\mathbb{R}\times (-\infty,\beta)\longrightarrow (0,\infty)$ is $C^\infty$ and hence it is a Lipschitz function. Therefore, according to the Cauchy-Peano Theorem, there exists a unique solution $\phi:[0,\epsilon)\longrightarrow [0,\infty)$ for sufficiently small $\epsilon>0$. Moreover, this solution is increasing, as $\phi'(x)>0$ for all $x\in[0,\epsilon)$. 

Since $\phi''(x)$ is negative, then $\phi(\epsilon),\,\phi'(\epsilon)<\infty$. If  $\phi'(\epsilon)>0$, we can extend $\phi$, uniquely, to a solution in an open neighbourhood of $\epsilon$, say $\phi:[0,\epsilon+\tilde{\epsilon})\longrightarrow [0,\infty)$. We can repeat this argument, as long as $\phi'(\epsilon+\tilde{\epsilon})>0$. Let
\[
      x_0=\sup\{x>0\,:\, \phi(x)<\beta^{\frac{1}{p}}\}.
\]
Either $x_0=\infty$ or $x_0<\infty$ and $\lim_{x\to x_0} \phi(x_0)=\beta^{\frac{1}{p}}$. This dychotomy follows by continuity, extending the solution $\phi$, and from the fact that $y=\beta$ is the first zero of $F(x,y)$. Then, $\lim_{x\to x_0}\phi'(x)=0$ for $x_0<\infty$.
\end{proof}

Here we do not rule out that $x_0=\infty$. However, in the context of the equation \eqref{p-nl-schrod}, we will only need the case $x_0<\infty$. Therefore, from now on we asume that this is the case. 

Consider the solution $\phi:[0,x_0]\longrightarrow [0,\beta]$, which is increasing, $\phi(x_0)=\beta^{\frac{1}{p}}$ and $\phi'(x_0)=0$. Let the extension $\tilde{\phi}:[-x_0,x_0]\longrightarrow [-\beta,\beta]$, given by
\[
     \tilde{\phi}(x)=\begin{cases}-\phi(-x) & x\in [-x_0,0] \\
     \phi(x) & x\in[0,x_0]. \end{cases}
\] 
Then, $\tilde{\phi}\in C^1(-x_0,x_0)$. Indeed, $\tilde{\phi}$ satisfies \eqref{1}, it is odd and its derivative is continuous at $x=0$, because $\tilde{\phi}^p$ is continuous. Also, \[(\tilde{\phi}')^{p-1}=\left[\frac12 \tilde{\phi}^{2p}(x)-\lambda \tilde{\phi}^p(x)+c^p\right]^{1-\frac{1}{p}},\] is differentiable. Indeed, $1-\frac{1}{p}>0$ and $\tilde{\phi}'(x)>0$ for all $x\in (-x_0,x_0)$. 

By differentiating \eqref{1} after taking power $p-1$ on the left hand side for $x\geq 0$ and by changing variables to $-x$ for $x<0$, we find that 
{\small \begin{equation} \label{2}
  ([\tilde{\phi}'(x)]^{p-1})'-(p-1)\sgn(x)|\tilde{\phi}(x)|^{2p-1}+
  \lambda (p-1) \sgn(x) |\tilde{\phi}(x)|^{p-1}=0
\end{equation}}for all $x\in[-x_0,x_0]$. According to Lemma~\ref{Lemma1}, the solution to this is unique in $[0,x_0]$. By changing variables, the extended solution is also unique, as we shall see next.

\begin{Lemma} \label{Lemma2}
A function $\tilde{\phi}\in C^1(-x_0,x_0)$ such that $(\tilde{\phi}')^{p-1}$ is differentiable on $(-x_0,x_0)$, satisfying \eqref{2} together with $\tilde{\phi}(0)=0$ and $\tilde{\phi}'(0)=c>0$, is unique.
\end{Lemma}
\begin{proof}
Let $\hat{\phi}:[-\hat{x}_0,\hat{x}_0]\longrightarrow \mathbb{R}$ be another solution to \eqref{2} as in the hypotheses. Then, $\hat{x}_0=x_0$ and $\hat{\phi}(x)=\tilde{\phi}(x)$ for all $x\in[0,x_0]$, by lemma~\ref{Lemma1}. Let $\overline{\phi}(x)=-\hat{\phi}(-x)$. Then, by repeating the previous argument, also $\overline{\phi}(x)=\hat{\phi}(x)=\tilde{\phi}(x)$ for $x\geq 0$. Thus, $\hat{\phi}(x)=\tilde{\phi}(x)$ also for $x\in[-x_0,0]$.
\end{proof}

Now, let $\phi^*:\mathbb{R}\longrightarrow [-\beta,\beta]$ be given by
\[
     \phi^*(x)=(-1)^k \tilde{\phi}(x-2k x_0)
\]
whenever $x\in [(2k-1)x_0,(2k+1)x_0]$ for $k\in \mathbb{Z}$. That is, $\phi^*$ is the $4x_0$-periodic extension of $\tilde{\phi}$, which is even with respect to $x_0$. Then, we have that $\phi^*\in C^1(\mathbb{R})$ and
$[|(\phi^*)'|]^{p-1}$ is differentiable on $\mathbb{R}$.
By changing variables from $x$ to $\pm(x-2kx_0)$, it then follows that
$\phi=\phi^*$ is such that
{\small \begin{equation} \label{3}
\big(\sgn(\phi')|\phi'|^{p-1}\big)'-(p-1)\sgn(\phi)|\phi|^{2p-1}+\lambda (p-1)\sgn( \phi)|\phi|^{p-1}=0.
\end{equation}}

\begin{Lemma} \label{Lemma3}
The $4x_0$-periodic extension $\phi^*:\mathbb{R}\longrightarrow [-\beta,\beta]$ of the solution $\phi$ to \eqref{1} described above, is the unique solution to \eqref{3} for the conditions $\phi(0)=0$ and $\phi'(0)=c>0$.
\end{Lemma}
\begin{proof}
By Lemma~\ref{Lemma2} the solution is unique in $[-x_0,x_0]$. Then, apply the change of variable $x$ to $\pm(x-2kx_0)$ and the same technique as in the proof of that lemma, to show uniqueness in $[(2k-1)x_0,(2k+1)x_0]$.
\end{proof}


Now we will establish monotonicity of the solutions in the parameter $c$. Let $c_2>c_1>0$ be such that $\lambda\geq \sqrt{2}c_2^{\frac{p}{2}}$. Consider the corresponding solutions to \eqref{1}, $\phi_1(x)$ and $\phi_2(x)$ with $c=c_j$, respectively. By Lemma~\ref{Lemma1} these solutions are each unique. Let $x_0^{[j]}$ denote the maximal points where $\phi_j(x)$ is increasing and exists uniquely in $[0,x_0^{[j]}]$.

\begin{Lemma} \label{Lemma4}
In the notation of the previous paragraph, $x^{[2]}_0>x^{[1]}_0$ and $\phi_2(x)>\phi_1(x)$ for all $x\in (0,x_0^{[1]})$.
\end{Lemma}
\begin{proof}
The corresponding equations \eqref{1} can be re-written as
\[
     \phi'(x)=r_j(\phi^p(x))^{\frac{1}{p}}
\] 
where 
\[ r_j(y)=\frac12 y^2-\lambda y+c_j^p=\frac12 (\alpha_j-y)(\beta_j-y). \] We know that $r_1(y)<r_2(y)$ for all $y\in\mathbb{R}$, $\beta_1<\beta_2$ and $\alpha_2<\alpha_1$. From monotonicity of the right hand side of the initial value problem, it follows that there exists $\epsilon>0$ such that $\phi_2(x)>\phi_1(x)$ for all $x\in(0,\epsilon)$. By Lemma~\ref{Lemma1}, these solutions can be continued to corresponding segments $[0,x^{[j]}_0]$. If there is a point $\tilde{x}$ in the intersection of these segments, such that $\phi_2(\tilde{x})=\phi_1(\tilde{x})$, assuming that $\tilde{x}$ is the first such point, we necessarily have $r_2(\tilde{y})<r_1(\tilde{y})$ for some $\tilde{y}\in (0,\beta_1)$. This would be the case, because we would need $\phi_2'(x)<\phi_1'(x)$ for $x<\tilde{x}$ close enough. But, this is a contradiction to the layout of the $r_j(y)$.
The conclusion of the lemma follows.   
\end{proof}

As a consequence of this lemma we are in the position now to determine the symmetries of the solutions to \eqref{p-nl-schrod}. Let $x_1>0$. Let the equation
\begin{equation} \label{4}
 \begin{aligned} &\big((\phi')^{p-1}\big)'-(p-1)\phi^{2p-1}+\lambda (p-1) \phi^{p-1}=0 \\
 &\phi(0)=\phi(x_1)=0 \end{aligned}
\end{equation}
posed for non-negative solutions.

\begin{Corollary} \label{Corollary5}
If the equation \eqref{4} has a solution which is non-zero in the segment $(0,x_1)$, then this solution is unique, increasing in $(0,\frac{x_1}{2})$, $\phi'(\frac{x_1}{2})=0$ and $\phi(x_1-x)=\phi(x)$ for all $x\in (\frac{x_1}{2},x_1)$. Also, \[\phi'(0)\leq \frac{\lambda^{\frac{2}{p}}}{2^{\frac{1}{p}}}.\]\end{Corollary}
\begin{proof}
Write \eqref{4} in the form \eqref{1} for $\phi'(0)=c$, as in the proof of Theorem~\ref{Theorem1}. Then, there exists $x_2\in (0,x_1)$ such that $\phi'(x_2)=0$.  This implies that $\lambda\geq \sqrt{2}\phi'(0)^{\frac{p}{2}}$. Also that $\phi(x)$ in increasing in $(0,x_2)$, decreasing in $(x_2,x_1)$ and 
$\max \phi(x)=\phi(x_2)$. The rest of the argument splits into two possibilities.

Suppose $x_2\geq \frac{x_1}{2}$. Let $\hat{\phi}(x)=\phi(x_1-x)$ for all $x\in[0,x_1]$. We show that $\hat{\phi}'(0)=c$. Then, according to Lemma~\ref{Lemma1}, $\hat{\phi}(x)=\phi(x)$, $x_2=\frac{x_1}{2}$ and the claimed conclusion will follow. We proceed by contradiction.  If $\hat{\phi}'(0)<c$, by virtue of Lemma~\ref{Lemma4}, then $\hat{\phi}(x)<\phi(x)$ for all $x\in\big(0,x_1-x_2\big)$. Thus,
\[
    \hat{\phi}(x_1-x_2)\leq \phi (x_1-x_2) <\phi(x_2).
\] 
But, this is a contradiction, as we know that $\hat{\phi}(x_1-x_2)=\phi(x_2)$. If, on the other hand, $\hat{\phi}'(0)>c$, by Lemma~\ref{Lemma4} then $\hat{\phi}(x)>\phi(x)$ for all $x\in (0,x_2]$. Hence,  \[\hat{\phi}(x_2)>\phi(x_2)= \max_{x\in (0,x_1)}\phi(x)>\phi(x_1-x_2)=\hat{\phi}(x_2).\] So again we arrive at a contradiction. Therefore, necessarily $\hat{\phi}'(0)=c$ and the conclusion follows.

Suppose $x_2<\frac{x_1}{2}$. The same conclusion is attained by applying the previous argument to the ``flip'' solution $\hat{\phi}(x)=\phi(x_1-x)$.
\end{proof}

In conjunction with Lemma~\ref{Lemma3}, this corollary ensures the validity of the following statement, which is the principal result of this section. Let the equation
{\small \begin{equation} \label{5}
\begin{aligned} 
&(\sgn(\phi')|\phi'|^{p-1})'-(p-1)\sgn(\phi)|\phi|^{2p-1}+\lambda (p-1)\sgn( \phi)|\phi|^{p-1}=0, \\
&\phi(0)=\phi(2x_0)=0,
\end{aligned}
\end{equation}}for some $x_0>0$ and fixed $\lambda>0$.

\begin{Corollary} \label{Corollary6}
If the equation \eqref{5} has a solution $\phi$, such that $\phi$ has exactly $n+2$ zeros in $[0,2x_0]$ and $\phi'(0)>0$, then this solution is unique. Moreover, the zeros of $\phi$ are located at $x=\frac{2x_0}{n+1}j$ for $j=0,\ldots,n+1$. The $\frac{4 x_0}{n+1}$-periodic extension $\phi^*$ is the unique solution to \eqref{5} on $\mathbb{R}$.
\end{Corollary}
\begin{proof}
By Corollary~\ref{Corollary5} applied on $[0,x_1]$, where $x_1\leq 2x_0$ is the first zero of $\phi$, we know that $\phi$ is symmetric in this sub-segment. Namely, 
$ \phi'(\frac{x_1}{2})=0$ and $\phi'(x_1)=-\phi'(0)$. Then, by considering the translated solution $\tilde{\phi}(x)=-\phi(x-x_1)$, and using Corollary~\ref{Corollary5} again, we know that $\phi(2x_1)=0$ and that $\phi$ extends periodically to $\mathbb{R}$ with $\phi'(2x_1)=\phi'(0)$. This periodic extension is the unique solution to \eqref{3} on $\mathbb{R}$ by Lemma~\ref{Lemma3}. The latter shows that the zeros of $\phi$ must be exactly located as claimed. Then $2x_0=x_1$.  
\end{proof}

The method we employed in this section to derive Corollary~\ref{Corollary6} is often called a shooting method. 

\begin{thebibliography}{5}
\bibitem{AG2007} \textsc{H.~Alzer and A.~Grinshpan,} Inequalities for the gamma and $q$-gamma functions. \emph{J. Approx. Theo.} \textbf{144} (2007) 67-83.
\bibitem{BL2015} \textsc{L.~Boulton and G.~Lord,} Basis properties of the $p,q$-sine functions. \emph{Proc. R. Soc. A} \textbf{471} (2015) 20140642.
\bibitem{BM2018} \textsc{L.~Boulton and H.~Melkonian,} A multi-term basis criterion for families of dilated periodic functions. \emph{Z. fur Anal. ihre Anwend.} \textbf{38} (2018) 107--124. 
\bibitem{BE2012} \textsc{P.J.~Bushell and D.~Edmunds,} Eigenvalue embeddings and generalised trigonometric functions. \emph{Rocky Mt. J. Math.} \textbf{42} (2012) 25--57.
\bibitem{E1950} \textsc{A.~Eld{'e}lyi}, Hypergeometric functions of two variables. \emph{Acta Mathematica} \textbf{83} (1950) 131--164.
\bibitem{EL2011} \textsc{D.E.~Edmunds and J.~Lang,} \emph{Eigenvalues, Embeddings and Generalised Trigonometric Functions.} (Berlin, Springer, 2011).
\bibitem{F1980a} \textsc{L.E.~Fraenkel,} Completeness properties in $L_2$ of the eigenfunctions of two semi-linear differential operators. \emph{Math. Proc. Cambridge Philos. Soc.}  \textbf{88} (1980), 451--468.
\bibitem{F1980b} \textsc{L.E.~Fraenkel,} A numerical sequence and a family of polynomials arising from a question of completeness. \emph{Math. Proc. Cambridge Philos. Soc.} \textbf{88} (1980), 469--481.
\bibitem{H2011} \textsc{C.~Heil,} \emph{A Basis Theory Primer.} (Berlin, Birkh{\"a}user, 2011).
\bibitem{K1980} \textsc{T.~Kato,} \emph{Perturbation Theory of Linear Operators.} (Berlin, Springer-Verlag, 1980).
%\bibitem{OT2001} \textsc{M.~Oliver and T.~Edriss,} On the domain of analyticity for solutions of second order analytic nonlinear differential equations. \emph{J. Differential Equations} \textbf{174} (2001), 55--74. 
\bibitem{WW1920} \textsc{E.T.~Whittaker and G.N.~Watson,} \emph{A Course of Modern Analysis} (New York, Dover, 2020). Reprint of the CUP edition 1920. 
\end{thebibliography}

\end{document}
