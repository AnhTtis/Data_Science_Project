% \documentclass{article}
\documentclass[a4paper,english,cleveref]{lipics-v2021}
\usepackage[utf8]{inputenc}

% \usepackage{mathtools}
\usepackage{xspace}
\usepackage{csquotes}
% \usepackage{complexity}

% \usepackage{amsmath,amsthm,amssymb}
% \usepackage[hidelinks]{hyperref}
\usepackage[table]{xcolor}
% \usepackage[noabbrev,capitalise]{cleveref}

% \newtheorem{theorem}{Theorem}
% \newtheorem{lemma}{Lemma}
% \newtheorem{conj}{Conjecture}
% \newtheorem{proposition}{Proposition}
% \newtheorem{observation}{Observation}
% \newtheorem{definition}{Definition}

% \usepackage{enumitem}
% \usepackage{graphicx}

% For tabularx package to have X columns that are centered
\newcolumntype{W}{>{\columncolor{gray!10}\centering\arraybackslash}X}
\newcolumntype{Y}{>{\columncolor{gray!20}\centering\arraybackslash}X}
\newcolumntype{Z}{>{\columncolor{gray!30}\centering\arraybackslash}X}



% Reference commands that show more than just the number and do the links right.
\newcommand{\invref}[1]{\hyperref[#1]{(I\ref*{#1})}\xspace}
% \newcommand{\propref}[1]{\hyperref[#1]{(P\ref*{#1})}\xspace}


\newcommand{\tw}{\operatorname{tw}}
\newcommand{\dist}{\operatorname{dist}}

\title{Clustered independence and bounded treewidth}

\author{Kolja Knauer}{Universtitat de Barcelona}{kolja.knauer@ub.edu}{https://orcid.org/0000-0002-8151-2184}{RYC-2017-22701 and  PID2019-104844GB-I00 of Ministerio de Econom\'ia, Industria y Competitividad}

\author{Torsten Ueckerdt}{Karlsruhe Institute of Technology}{torsten.ueckerdt@kit.edu}{https://orcid.org/0000-0002-0645-9715}{}

\authorrunning{K. Knauer and T. Ueckerdt}

\Copyright{K. Knauer and T. Ueckerdt}

%%% DOUBLE-CHECK THIS CLASSIFICATION
\ccsdesc[100]{Mathematics of computing $\rightarrow$ Discrete mathematics $\rightarrow$ Combinatorics $\rightarrow$ Combinatoric problems}

\keywords{treewidth, clustered sets}

\hideLIPIcs
\nolinenumbers

\begin{document}

\maketitle

\begin{abstract}
    A set $S\subseteq V$ of vertices of a graph $G$ is a \emph{$c$-clustered set} if it induces a subgraph with components of order at most $c$ each, and $\alpha_c(G)$ denotes the size of a largest $c$-clustered set. 
    For any graph $G$ on $n$ vertices and treewidth $k$, we show that $\alpha_c(G) \geq \frac{c}{c+k+1}n$, which improves a result of Wood [arXiv:2208.10074, August 2022], while we construct $n$-vertex graphs $G$ of treewidth~$k$ with $\alpha_c(G)\leq \frac{c}{c+k}n$.
    In the case $c\leq 2$ or $k=1$ we prove the better lower bound $\alpha_c(G) \geq \frac{c}{c+k}n$, which settles a conjecture of Chappell and Pelsmajer [Electron.\ J.\ Comb., 2013] and is best-possible.
    Finally, in the case $c=3$ and $k=2$, we show $\alpha_c(G) \geq \frac{5}{9}n$ and which is best-possible. 
\end{abstract}



\section{Introduction}

Let $G = (V,E)$ be a graph and $c$ a positive integer.
We call a subset $S \subseteq V$ of vertices of $G$ a \emph{$c$-clustered set} if every connected component of the subgraph $G[S]$ of $G$ induced by $S$ has at most $c$ vertices.
We define the \emph{$c$-clustered independence number} $\alpha_c(G)$ of $G$ as the maximum size of a $c$-clustered set in $G$.
In particular, the $1$-clustered sets of $G$ are exactly the independent sets of $G$ and $\alpha_1(G)$ equals the independence number $\alpha(G)$.
On the other hand, for each $2$-clustered set $S$ of $G$ the subgraph $G[S]$ is a collection of vertices and edges in $G$ with no edge of $G$ between these components, and $\alpha_2(G)$ is the largest number of vertices in $G$ inducing only isolated vertices and edges.

\begin{figure}
    \centering
    \includegraphics{figures/example-clustered-sets.pdf}
    \caption{Examples of a $2$-clustered set (left), a $3$-clustered set (middle), and a $4$-clustered set (right).}
    \label{fig:example-clustered-sets}
\end{figure}

In the literature, $c$-clustered sets appear primarily as the color classes of $c$-clustered colorings.
A $t$-coloring $\phi \colon V \to [t]$ of the vertices of a graph $G=(V,E)$ is a \emph{$c$-clustered coloring} if there is no monochromatic connected subgraph on more than $c$ vertices in $G$.
In other words, each color class $\phi^{-1}(i)$, $i=1,\ldots,t$, is a $c$-clustered set.
The \emph{$c$-clustered chromatic number} $\chi_c(G)$ is then the minimum $t$ for which $G$ admits a $c$-clustered $t$-coloring.
So, for example, $\chi_1(G)$ is equal to the classical chromatic number $\chi(G)$, while $\chi_2(G)$ is the smallest number of $2$-clustered sets into which the vertex set $V(G)$ can be partitioned.
Clearly, for any $c \geq 1$ and any graph $G$ we have
\begin{equation}
    \alpha_c(G) \geq \frac{|V(G)|}{\chi_c(G)}, \quad \text{ or equivalently } \quad \frac{\alpha_c(G)}{|V(G)|} \geq \frac{1}{\chi_c(G)}.\label{eq:trivial-LB}
\end{equation}

In this paper, we focus on the quantity $\alpha_c(G) / |V(G)|$, i.e., the proportion of vertices of $G$ that we can put into a $c$-clustered set, and seek to find (for graphs of a particular class) universal lower bounds that significantly improve on the $1/\chi_c(G)$ in~\cref{eq:trivial-LB}.
We shall focus on graphs of treewidth $k$, as introduced by Robertson and Seymour~\cite{RS86}, while we also briefly discuss other graph classes at the very end.




%%%%%%%%%%%%%%%%%%%%%%%
%%    our results    %%
%%%%%%%%%%%%%%%%%%%%%%%
\subparagraph*{Our Results.}

We are interested in the largest $c$-clustered number guaranteed in each $n$-vertex graph $G$ with $\tw(G) = k$, i.e., graph of treewidth $k$.
To this end, let us define
\begin{equation}
    x_{k,c} = \liminf\{ \frac{\alpha_c(G)}{|V(G)|} \colon \tw(G) = k\}.\label{eq:definition-x-kc}
\end{equation}
In fact, for lower bounds on $x_{k,c}$ we will show the slightly stronger statement that every graph $G$ of treewidth $k$ satisfies $\alpha_c(G) \geq x_{k,c}\cdot |V(G)|$.
Similarly, for our upper bounds on $x_{k,c}$ we shall construct an infinite set of graphs $G$ of treewidth $k$ with $\alpha_c(G) \leq \lceil x_{k,c}\cdot |V(G)|\rceil$ for each such $G$.
% In fact, there will be no lower-order terms in our upper and lower bounds on $x_{k,c}$.
Our results on $x_{k,c}$ are summarized in the following theorem and illustrated in~\cref{fig:all-bounds}.

\begin{theorem}\label{thm:main}
 Each of the following holds.
 \begin{enumerate}%[label = (\roman*)]
     \item $\frac{c}{k+c+1} \, \leq \, x_{k,c} \, \leq \, \frac{c}{k+c}$ for every $c,k \geq 1$.\label{enum:general-c-k}
     
     \item $x_{1,c} = \frac{c}{1+c}$ for every $c \geq 1$.\label{enum:k1}
     
     \item $x_{k,1} = \frac{1}{k+1}$ and $x_{k,2} = \frac{2}{k+2}$ for every $k \geq 1$.\label{enum:c1-and-c2}
     
     \item $x_{2,3} = \frac{5}{9}$.\label{enum:c3-and-k2}
 \end{enumerate}
% 
 % For every positive integers $c,k \geq 1$ we have
 % \begin{equation}
 % \frac{c}{k+c+1} \, \leq \, x_{k,c} \, \leq \, \frac{c}{k+c}.\label{eq:general-bounds}
 % \end{equation}
 % Moreover,
 % \begin{enumerate}[label = (\roman*)]
 %     \item the upper bound in \cref{eq:general-bounds} is tight for all $c \geq 1$ and $k=1$,
 %     \item the upper bound in \cref{eq:general-bounds} is tight for $c \in \{1,2\}$ and all $k \geq 1$, and
 %     \item $\inf\{ \frac{\alpha_3(G)}{|V(G)|} \colon \tw(G) = 2\} = \frac{5}{9}$.\\
 %     I.e., neither the upper nor lower bound in \cref{eq:general-bounds} is tight for $c=3$ and $k=2$.
 % \end{enumerate}
\end{theorem}

\begin{figure}
    \centering
    \includegraphics{figures/all-bounds.pdf}
    \caption{Illustration of the bounds on $x_{k,c} = \liminf\{\alpha_c(G)/|V(G)| \colon \tw(G) = k\}$ in \cref{thm:main}.}
    \label{fig:all-bounds}
\end{figure}

That is, we show a general upper bound of $x_{k,c} \leq \frac{c}{k+c}$ and a general lower bound of $x_{k,c} \geq \frac{c}{k+c+1}$ that is just an additive $1$ in the denominator away, cf.~\cref{enum:general-c-k}.
We show our upper bound to be tight if $k = 1$ or $c \leq 2$, cf.~\crefrange{enum:k1}{enum:c1-and-c2}.
But somewhat surprisingly, in the smallest case $k=2$ and $c=3$ not covered by this, we show that $x_{2,3} = \frac59$, cf.~\cref{enum:c3-and-k2}, which matches neither the general upper nor the general lower bound in~\cref{enum:general-c-k}.
This leaves us with no obvious good candidate for the true value of $x_{k,c}$ in general. We wonder if there are further infinite families of parameters for which simple formulas for $x_{k,c}$ exist. 



%%%%%%%%%%%%%%%%%%%%%%%
%%   organisation    %%
%%%%%%%%%%%%%%%%%%%%%%%
\subparagraph*{Organisation of the paper.}
After discussing some related work and previously known bounds on $x_{k,c}$, we start in \cref{sec:treewidth-intro} with a short introduction of the model for treewidth-$k$ graphs that we primarily use in our proofs.
In \cref{sec:treewidth-intro} we also explain that \cref{eq:trivial-LB} does not yield interesting lower bounds on $x_{k,c}$ if $c \neq 1$.

We then prove \cref{enum:general-c-k} of \cref{thm:main} in \cref{sec:general-case}, \cref{enum:k1} in \cref{sec:k1-case}, \cref{enum:c1-and-c2} in \cref{sec:c2-case}, and \cref{enum:c3-and-k2} in \cref{sec:c3-k2-case}.
Finally, we give a brief discussion of $c$-clustered independence in other classes of graphs in \cref{sec:conclusions}.



%%%%%%%%%%%%%%%%%%%%%%%
%%   related work    %%
%%%%%%%%%%%%%%%%%%%%%%%
\subparagraph*{Related Work.}

While the $c$-clustered independence number $\alpha_c(G)$ has (to the best of our knowledge) not been explicitly considered before, there is a number of equivalent or closely related concepts in the literature.
Most relevant to us are the following results of Wood~\cite{Wo}, and Chappell and Pelsmajer~\cite{ChPe}.

\begin{theorem}[{Wood~\cite[Theorem 22]{Wo}}]\label{wood}{\ \\}
    Let $G=(V,E)$ be a graph on $n$ vertices and treewidth at most $k$.
    If $n\leq \lfloor\frac{p}{k+1}\rfloor(c+1)+k+c-1$ and $k+1\leq p$, then there is a set $S\subseteq V$ of size $p$, such that all connected components of $G\setminus S$ have order at most $c$. 
\end{theorem}

Rearranging terms, \cref{wood} proves that
\[
    \frac{c-k}{c+1} \leq \inf\{ \frac{\alpha_c(G)}{|V(G)|} \colon \tw(G) = k\},
\]
i.e., gives the lower bound $x_{k,c} \geq \frac{c-k}{c+1}$.
As $\frac{c-k}{c+1} < \frac{c}{k+c+1}$ for all $k,c \geq 1$, the lower bound $x_{k,c} \geq \frac{c}{k+c+1}$ in~\cref{enum:general-c-k} of \cref{thm:main} supersedes \cref{wood}.

Chappell and Pelsmajer~\cite{ChPe} investigated for $d \geq 0$ and $k \geq 1$ the largest set $S$ of vertices in any $n$-vertex graph $G$ of treewidth $k$, such that the induced subgraph $G[S]$ is a forest of maximum degree at most $d$.
Equivalently, for $d=0$, $S$ is an independent set, and for $d=1$, $S$ is a $2$-clustered set. (For $d \geq 2$ there is no equivalent correspondence in $c$-clustered sets.)

\begin{theorem}[Chappell and Pelsmajer~\cite{ChPe}]{\ \\}
    Let $G=(V,E)$ be a graph on $n$ vertices and treewidth at most $k$.
    Then there is a subset $S \subseteq V$ such that $G[S]$ has maximum degree at most $1$ and
    \[
      |S| \geq \frac{2n+2}{2k+3} \;\text{ if }k \geq 4 \qquad \text{ respectively } \qquad |S| \geq \frac{2}{k+2}n \;\text{ if }k \leq 3\text{.}
    \]
 % \begin{itemize}
     % \item For $k=1$ and every $d \geq 0$, every graph $G = (V,E)$ of treewidth $k$ admits an induced subgraph on at least $\frac{d+1}{d+2}|V|$ vertices maximum degree at most $d$.
% 
     % \item For every $k$ and $d = 1$, every graph $G = (V,E)$ of treewidth $k$ admits an induced subgraph on at least $\frac{2|V|+2}{2k+3}$ vertices maximum degree at most $d$.
% 
     % \item For every $k \leq 3$ and $d = 1$, every graph $G = (V,E)$ of treewidth $k$ admits an induced subgraph on at least $\frac{2}{k+2}|V|$ vertices maximum degree at most $d$.
% 
     % \item For every $k,d \geq 2$, every graph $G = (V,E)$ of treewidth $k$ admits an induced subgraph on at least $\frac{2d|V|+2}{kd+d+1}$ vertices maximum degree at most $d$, except when $k=d=2$ and $G \in \{K_{2,3}, K_{1,1,3}\}$.
 % \end{itemize}
\end{theorem}

They also conjecture that their (better) bound for $k\leq 3$ should hold for all $k$.

\begin{conjecture}[{Chappell and Pelsmajer~\cite[Conjecture 13]{ChPe}}]\label{conj:Chappell-Pelsmajer}{\ \\}
    For $k \geq 0$, if $G$ is a graph of order $n$ and treewidth at most $k$, then $G$ admits an induced subgraph $G[S]$ of maximum degree at most $1$ and $|S| \geq \lceil \frac{2n}{k+2} \rceil$.
\end{conjecture}

We confirm \cref{conj:Chappell-Pelsmajer} in \cref{prop:lower-bound-c2} below.
In particular, \cref{enum:c1-and-c2} in \cref{thm:main} indeed states that $x_{k,2} \geq \frac{2}{k+2}$.

\medskip

Let us also briefly mention some further notions that are related to the $c$-clustered sets of a graph $G=(V,E)$.
Clearly, $S \subseteq V$ is $1$-clustered (i.e., an independent set) if and only if its complement $V-S$ is a vertex cover.
Along these lines, complements of $c$-clustered sets are also known as $c$-vertex separators~\cite{Lee19}, or $c$-component order connected sets~\cite{KL17}, and for the special case of $c=2$ as $3$-path vertex covers~\cite{BKKS11}.
Meanwhile, $2$-clustered sets appeared under the name of dissociation sets~\cite{Yan81}.


%%%%%%%%%%%%%%%%%%%%%%%
%%  treewidth intro  %%
%%%%%%%%%%%%%%%%%%%%%%%
\section{Graphs of treewidth $k$ and a first observation}
\label{sec:treewidth-intro}

All graphs considered here are finite, simple, and undirected.
For a graph $G$ and a vertex $v \in V(G)$, we denote the neighborhood of $v$ in $G$ by $N_G(v) = \{u \in V(G) \colon uv \in E(G)\}$.

For our arguments it will be convenient to rely on the definition of the treewidth of a graph in terms of $k$-tree models below.
In a rooted tree $T$ with root $r$, a vertex $u$ is an \emph{ancestor} of vertex $v$ (and $v$ is a \emph{descendant} of $u$) if $u$ lies on the unique path in $T$ from $v$ to $r$.
We denote the distance between two vertices $u$ and $v$ by $\dist(u,v)$ and call vertices in $T$ \emph{lower} if they have a larger distance to the root.
The \emph{height} of $T$ is the largest distance of any vertex in $T$ to the root plus $1$.

\begin{definition}
 A \emph{$k$-tree model} of a graph $G$ is a rooted tree $T$ with vertex set $V(T) = V(G)$, together with a labeling $L \colon V(T) \to [k+1]$ such that for every edge $uv \in E(G)$ we have
 \begin{itemize}
  \item $L(u) \neq L(v)$ and
  \item $u$ is the lowest ancestor of $v$ with label $L(u)$, or vice versa.
 \end{itemize}
 The \emph{treewidth} of $G$, denoted by $\tw(G)$, is the smallest $k \in \mathbb{N}$ for which there exists a $k$-tree model $(T,L)$ of $G$.
\end{definition}

See \cref{fig:full-k-ary-tree,fig:path-on-k-and-c} for examples of graphs $G$ with $\tw(G) = k$ and some corresponding $k$-tree models $(T,L)$.
Establishing some notation, for a fixed $k$-tree model $(T,L)$ of $G$ and a vertex $v$ of $G$, the \emph{parents} of $v$ are those neighbors $u \in N_G(v)$ of $v$ in $G$ that are ancestors of $v$ in $T$.
Similarly, whenever $u$ is a parent of $v$, then $v$ is a \emph{child} of $u$.
Note that in any $k$-tree model $(T,L)$, the parents of $v$ have pairwise distinct labels and distinct from $L(v)$.
Thus, $v$ has at most $k$ parents, while $v$ may have arbitrarily many children.

If $(T,L)$ is a $k$-tree model of $G$, then in particular $L$ is a proper vertex coloring of $G$ with $k+1$ colors.
Hence, $\chi_1(G) = \chi(G) \leq \tw(G)+1$ for every graph $G$, which implies with \cref{eq:trivial-LB} that
\[
    \frac{\alpha_1(G)}{|V(G)|} \geq \frac{1}{k+1} \text{ if } \tw(G) = k \quad \text{ and thus } \quad x_{k,1} \geq \frac{1}{k+1}.
\]
It is easy to see that in fact $x_{k,1} = \frac{1}{k+1}$, cf.~\cref{thm:main}~\cref{enum:c1-and-c2}.
However, \cref{eq:trivial-LB} does not give anything better for $c > 1$, due to the following.

\begin{observation}\label{obs:c-clustered-coloring}
    For any $k,c$, let $T$ be the full $c$-ary tree with root $r$ and height $k+1$, and $L \colon V(T) \to [k+1]$ the labeling given by $L(v) = \dist(v,r) + 1$.
    See \cref{fig:full-k-ary-tree} for an example.
    Then the edge-maximal graph $G$ with $k$-tree model $(T,L)$ satisfies $\chi_c(G) = k+1$.

    \begin{figure}
        \centering
        \includegraphics{figures/full-k-ary-tree.pdf}
        \caption{
            A graph $G$ (right) with a $k$-tree model $(T,L)$ (left) with $\chi_c(G) = k+1$ for $c = 4$, $k = 2$.
        }
        \label{fig:full-k-ary-tree}
    \end{figure}

    In fact, if $\phi$ is any $c$-clustered coloring of $G$ and the root $r$ receives color $i$, then at least one of the $c$ subtrees below $r$ contains no vertex of color $i$, and it follows by induction on $k$ that there are at least $k+1$ distinct colors.
\end{observation}

\cref{obs:c-clustered-coloring} shows that for $c \neq 1$ we do not get anything better than the obvious lower bound $x_{k,c} \geq x_{k,1} \geq \frac{1}{k+1}$ from considering $c$-clustered chromatic numbers and the simple averaging argument in~\cref{eq:trivial-LB}.


%%%%%%%%%%%%%%%%%
%%  general k  %%
%%  general c  %%
%%%%%%%%%%%%%%%%%
% \section{$c$-clustered sets in graphs of treewidth $k$}
\section{General bounds for all $k$ and $c$}
\label{sec:general-case}

Here we consider the case of any integers $k,c \geq 1$, i.e., we prove \cref{thm:main}~\cref{enum:general-c-k}, starting with the lower bound.

\begin{proposition}\label{prop:lower-bound}
 For every $k,c \geq 1$ and every graph $G$ of treewidth $k$ we have $\alpha_c(G) \geq \frac{c}{k+c+1}|V(G)|$, i.e., $x_{k,c} \geq \frac{c}{k+c+1}$.
\end{proposition}

\begin{proof}
 Fix $G = (V,E)$ to be any graph of treewidth $k$.
 We proceed by induction on $n = |V|$ and find a $c$-clustered set $S$ in $G$ of size $|S| \geq \frac{c}{k+c+1}n$.
 For the induction base, the case $n \leq c$, it is enough to take $S = V$.
 So assume that $n > c$.
 Let $(T,L)$ be a $k$-tree model of $G$.
 Let $v$ be a lowest vertex in $T$ that has at least $c$ descendants.
 Let $A$ be the set of descendants of $v$ and $B = N_G(v) - A$.
Then $|A| \geq c$, $A$ is a $c$-clustered set in $G$ (by the minimality in the choice of $v$), $|B| \leq k$, and no vertex in $A$ is adjacent to any vertex in $G' = G- (A \cup B \cup \{v\})$.

 By induction on $G'$, there is a $c$-clustered set $S'$ of at least $\frac{c}{k+c+1}(n - |A| - |B|-1)$ vertices in $G'$.
 Then $S = S' \cup A$ is a $c$-clustered set of size at least
 \begin{multline*}
   \frac{c}{k+c+1}(n - |A| - |B|-1) + |A| \\
   = \frac{c}{k+c+1}n + \frac{k+1}{k+c+1}|A| - \frac{c}{k+c+1}(|B|-1) \\
   \geq \frac{c}{k+c+1}n + \frac{(k+1)c}{k+c+1} - \frac{c(k+1)}{k+c+1} = \frac{c}{k+c+1}n.\qedhere
 \end{multline*}
\end{proof}

The upper bound on $x_{k,c}$ is a simple construction.

\begin{observation}\label{obs:upper-bound}
    For any $k,c$, let $T = [v_1,\ldots,v_{k+c}]$ be a path on $k+c$ vertices rooted at $v_1$, and $L\colon V(T) \to [k+1]$ the labeling given by $L(v_i) = i$ for $i = 1,\ldots,k$ and $L(v_i) = k+1$ for $i = k+1,\ldots,k+c$.
    See \cref{fig:path-on-k-and-c} for an example.
    Then the edge-maximal graph $G$ with $k$-tree model $(T,L)$ satisfies $\alpha_c(G) = c$ and $|V(G)| = k+c$.

    \begin{figure}
        \centering
        \includegraphics{figures/path-on-k-and-c.pdf}
        \caption{A graph $G$ (left) with two different $k$-tree models $(T,L)$ (left) with $\alpha_c(G) = c$ and $|V(G)| = k+c$, i.e., $\alpha_c(G) = \frac{c}{k+c}|V(G)|$, for $k=3$, $c=4$.}
        \label{fig:path-on-k-and-c}
    \end{figure}

    In fact, $v_1,\ldots,v_k$ are universal vertices in $G$ and hence any set of $c+1$ vertices in $G$ induces a connected subgraph of size $c+1$, i.e., is not $c$-clustered.
\end{observation}

Taking arbitrary vertex-disjoint unions of the graph $G$ in \cref{obs:upper-bound}, it follows that $x_{k,c} \leq \frac{c}{k+c}$.
So \cref{prop:lower-bound} and \cref{obs:upper-bound} together prove \cref{thm:main}~\cref{enum:general-c-k}.


% general situation

% set of $(k+1)$-cliques --> vertex set $A$

% set of simplicial vertices $S$
% $|S| = s$

% every vertex of $A$ has $b$ neighbors in $S$

% every vertex of $S$ has $k$ neighbors in pairwise distinct cliques

% ==> $sk = |A|b$ ==> $|A| = sk/b$
% $s = 3i$
% $|A| = 6i$

% ==> $n = s + |A| = s(1 + k/b) = (sb + sk)/b = s(b+k)/b$
% ==> $s/n = b/(b+k)$
% $n = 9i$

% % $(b+1) \ell > c$
% % ==> $\ell > c/(b+1)$

% $c+1 = (b+1)(\ell+1)$
% $\ell = ((c+1)-(b+1))/(b+1)$

% $\alpha_c \leq s + \ell \cdot |A|/(k+1) = s + \frac{c sk }{(b+1)b(k+1)} = s \frac{ (b+1)b(k+1) + ck }{ (b+1)b(k+1) }$

% $\alpha_c \leq s + \ell \cdot |A|/(k+1) = s + \frac{ ((c+1) - (b+1))sk }{ (b+1)(k+1)b } = s \frac{ (b+1)( b(k + 1) - k) + (c+1) }{ (b+1)(k+1)b } 
% = s(1 + \frac{ c+1 - k(b+1) }{ (b+1)(k+1)b } )$
% $3/2$
% $s + \frac{c sk }{(b+1)b(k+1)} = 3i + 3i$
% $\alpha_3 = 5i$

% ==> $\alpha_c / n \leq s( 1 + \frac{ c+1 - k(b+1) }{ (b+1)(k+1)b } ) / n
% = \frac{b}{b+k} ( 1 + \frac{ c+1 - k(b+1) }{ (b+1)(k+1)b } ) $
%    ----> wir wollen das mit c/(k+c) und c/(k+c+1) vergleichen

% % ==> $\alpha_c / n \leq \frac{ b c s k }{ s(b+k)(b+1)b(k+1) } = \frac{ c k }{ (b+k)(b+1)(k+1) } \leq \frac{ c }{ (b+k)(b+1) } $

% ==> better than $c/(k+c)$ if 
% $ (k+1)(b+k)(b+1) > (k+c)k $

% ==> should be at least $c/(k+c+1)$, i.e.,
% $ (k+c+1)k \geq (b+k)(b+1)(k+1) > (k+c)k $

% ==> working example $k=2, c=3, b=1$
% $ 12 \geq 3 2 3 $




%%%%%%%%%%%%%%%%%
%%  general c  %%
%%    k = 1    %%
%%%%%%%%%%%%%%%%%
\section{The case of graphs of treewidth $1$}
\label{sec:k1-case}

We shall prove \cref{thm:main}~\cref{enum:k1}.
For this we shall argue that, for the case of treewidth $k=1$, we can improve the lower bound in \cref{prop:lower-bound} by just slightly changing its proof.
To this end, we remark that any graph $G$ with at least one edge admits a $k$-tree model $(T,L)$ with $k = \tw(G)$ in which the two endpoints of any edge in $T$ have distinct labels in $L$.
In fact, if $v$ is the immediate ancestor of $u$ in $T$ and $L(u) = L(v)$, we can change the tree (keeping all labels) by hanging the subtree $T_u$ rooted at $u$ under the lowest ancestor $w$ of $u$ with $L(w) \neq L(u)$.
If there is no such ancestor, then the vertices in $T_u$ are not connected to the remaining graph, and we can permute the labeling in $T_u$ to give $u$ a label distinct from $L(v)$.
In either case, the result is still a $k$-tree model of $G$ with fewer problematic edges.

\begin{proposition}\label{prop:lower-bound-k1}
 For every $c \geq 1$ and every graph $G$ of treewidth $1$ we have $\alpha_c(G) \geq \frac{c}{1+c}|V(G)|$, i.e., $x_{1,c} \geq \frac{c}{1+c}$.
\end{proposition}

\begin{proof}
 We proceed by induction on $n = |V|$ and find a $c$-clustered set $S$ in $G$ of size $|S| \geq \frac{c}{1+c}n$.
 For the induction base, the case $n \leq c$, it is enough to take $S = V$.
 So assume that $n > c$.
 Let $(T,L)$ be a $1$-tree model of $G$ with the property that any edge of $T$ connects two vertices of distinct label in $L$.
 As in the proof of \cref{prop:lower-bound}, let $v$ be a lowest vertex in $T$ with a set $A$ of at least $c$ descendants.
 Then $A$ is a $c$-clustered set of size $|A| \geq c$, and no vertex in $A$ is adjacent to any vertex in $G' = G - (A \cup \{v\})$.
 In fact take any $w \in A$ and assume by symmetry that $L(v) = 1$.
 If $L(w) = 2$, then $v$ is the only ancestor of $w$ in $N_G(w)$.
 And if $L(w) = 1$, the ancestor of $w$ in $N_G(w)$ has label $2$ and is the immediate ancestor of $w$ in $T$ by our additional assumption on the $1$-tree model.

 Now by induction on $G'$, there is a $c$-clustered set $S'$ of at least $\frac{c}{1+c}(n-|A|-1)$ vertices in $G'$.
 Then $S = S' \cup A$ is a $c$-clustered set of size at least
 \begin{multline*}
   \frac{c}{1+c}(n-|A|-1) + |A| = \frac{c}{1+c}n + \frac{1}{1+c}|A| - \frac{c}{1+c}\\
   \geq \frac{c}{1+c}n + \frac{c}{1+c} - \frac{c}{1+c} = \frac{c}{1+c}n.\qedhere
 \end{multline*}
\end{proof}





%%%%%%%%%%%%%%%%%
%%    c = 2    %%
%%  general k  %%
%%%%%%%%%%%%%%%%%
\section{The case of $1$-clustered and $2$-clustered sets}
\label{sec:c2-case}

We shall prove \cref{thm:main}~\cref{enum:c1-and-c2}.
In fact, we already have the upper bound $x_{k,c} \leq \frac{c}{k+c}$ and need to prove a matching lower bound when $c \leq 2$.
For $c=1$, this is already given by~\cref{eq:trivial-LB} and it remains to consider the case $c=2$ here.

\begin{proposition}\label{prop:lower-bound-c2}
 For every $k \geq 1$ and every graph $G$ of treewidth $k$ we have $\alpha_2(G) \geq \frac{2}{k+2}|V(G)|$, i.e., $x_{k,2} \geq \frac{2}{k+2}$.
\end{proposition}

\begin{proof}
 Let $G = (V,E)$ be any graph of treewidth $k$.
 We proceed by induction on $n = |V(G)|$ and find a $2$-clustered set $S$ in $G$ of size $|S| \geq \frac{2}{k+2}n$.
 For the induction base, the case $n \leq k+2$, it is enough to let $S$ be any pair of vertices.
 So assume that $n \geq k+3$ and let $(T,L)$ be a $k$-tree model of $G$.
 Without loss of generality assume that $G$ is edge-maximal with model $(T,L)$.
 In particular, the parents of each vertex form a clique. 
 % For $v \in V$ let $T_v$ denote the subtree in $T$ corresponding to $v$, and $r_v$ be the root of $T_v$.
 % Recall that each node of $\mathcal{T}$ is the root of exactly one vertex in $G$.
 % In particular, $|V(\mathcal{T})| = n \geq k+3$.

 We describe a procedure that gradually determines which vertices of $G$ to \emph{take}, i.e., include it in the desired set $S$, and which to \emph{discard}, i.e., include it in another set $D$.
 All such decisions will be irrevocable, and eventually $S$ and $D$ will partition $V$.
 During the course of the procedure, vertices in $V - (S \cup D)$ are called \emph{undecided}.
 In order to ensure that $S$ is a $2$-clustered set we maintain the following invariant for every vertex $v$.

 \begin{enumerate}[({I}1)]
     \item If $v \in S$, then $v$ has no undecided children and at most one child in $S$. If $v$ has a child in $S$, then all parents of $v$ are in $D$. \label{inv:2-clustered}
 \end{enumerate}

 Note that this indeed ensures $S$ to be a $2$-clustered set.
 In order to bound the size $|S|$ of $S$ in terms of $n$ the number of vertices in $G$, we use \emph{tokens} placed on vertices. Initially, there are no tokens.
 Taking an undecided vertex $v$ grants $k$ tokens, which we can distribute on the remaining undecided vertices $V-(S \cup D \cup \{v\})$.
 Discarding an undecided vertex $v$ costs $2$ tokens, which we can remove from $v$ or other undecided vertices.
 Thus, as soon as there are no more undecided vertices, i.e., $V = S \cup D$, we can conclude that
 \[
  k|S| \geq \#\text{tokens granted by taking vertices} \geq \#\text{tokens spend by discarding vertices} \geq 2|D|
 \]
 and thus
 \[
  \frac{k}{2} |S| \geq |D| = n-|S| \quad \Rightarrow \quad \frac{k+2}{k}|S| \geq n \quad \Rightarrow \quad |S| \geq \frac{2}{k+2}n,
 \]
 as desired.
 
 At intermediate states we allow a negative token count at undecided vertices.
 But still, in order to discard a vertex $v$, we first must redistribute tokens so that the token count at $v$ is at least $2$, and in order to take vertex $v$ the token count at $v$ must be at least $0$.
 
 % The \emph{parents} of a vertex $v \in V - (S \cup D)$ are the neighbors $u$ of $v$ in $G$ with $r_u$ being above $r_v$ in $\mathcal{T}$.
 % If $u$ is a parent of $v$, then we also say that $v$ is a \emph{children} of $u$.
 % All vertices (except the topmost $k$ in $\mathcal{T}$) have initially exactly $k$ parents, while there can be any number of children.

 Initially $S = D = \emptyset$, i.e., all vertices of $G$ are undecided.
 Throughout we maintain a $k$-tree model $(T,L)$ for $G' = G - D$, i.e., the induced subgraph of $G$ on all taken and undecided vertices.
 By discarding a vertex $v$, we remove $v$ from the current $k$-tree model of $G'$ by contracting $v$ into its immediate ancestor in $T$, keeping the labels at all vertices (except the removed $v$).
 Note that this indeed results in a $k$-tree model of $G' - v$ with the set of parents of each vertex forming a clique.
 For convenience we shall denote the new $k$-tree model again by $(T,L)$.
 Recall that a parent of $v$ is a vertex $u \in N_G(v)$ that is an ancestor of $v$ in $T$.
 Each vertex has at most $k$ parents but by discarding vertices, we may reduce the number of parents of other undecided (or taken) vertices.

 For any undecided vertex $v$, let $t_v$ denote the number of tokens at $v$, $p_v$ the number of parents of $v$, and $s_v$ the number of children of $v$ that are in $S$.
 We maintain the following invariants for every undecided vertex $v$:

 \begin{enumerate}[({I}1)]%[label = (I\arabic*)]
    \setcounter{enumi}{1}
     \item If $v$ has undecided children or $s_v \geq 2$, then $t_v \geq s_v$.\label{inv:inner-vertex}
    
     % \item %If no children of $v$ is in $S$, there are $0$ tokens at $v$. I.e., 
     % If $s_v = 0$, then $t_v = 0$. \label{inv:no-child-taken}
     
     % \item %If $v$ has $s_v \geq 1$ children in $S$, $t_v$ tokens and $p_v$ parents,
     % If $s_v \geq 1$, then $t_v \geq s_v + p_v-k$. \label{inv:token-parents}
     
     % \item %If some children of $v$ is in $S$ and $v$ has $t_v \leq 0$ tokens
     % If $s_v \geq 1$ and $t_v \leq 0$, then $v$ has no undecided children. \label{inv:non-leaf-all-parents}

     \item If $v$ has no undecided children, then $t_v \geq s_v + p_v - k$. \label{inv:leaf-vertex}
 \end{enumerate}

 Note that these invariants hold initially when $S = D = \emptyset$.
 
 Let $v$ be a lowest undecided vertex in $T$. %that is in $V - (S \cup D)$.
 I.e., all children of $v$ (if any) are in $S$.

 \begin{enumerate}[{Case} 1]%[label = Case \Alph*:]
    \item No children of $v$ are in $S$.

    We have $s_v = 0$ and by \invref{inv:leaf-vertex} there are $t_v \geq s_v + p_v - k = p_v - k$ tokens at $v$.
    % So $s_v = 0$ and by \invref{inv:no-child-taken} there are $t_v = 0$ tokens at $v$.
    We take $v$, i.e., add $v$ to the set $S$, which grants $k$ tokens, pay $k-p_v$ of these tokens to have the token count at $v$ at $0$, and spend the remaining $p_v$ tokens by putting $1$ token on each of the $p_v$ parents of $v$.
    This maintains the invariants.

    \item At least two children of $v$ are in $S$.

    We have $s_v \geq 2$ and by \invref{inv:inner-vertex} there are $t_v \geq s_v \geq 2$ tokens at $v$.
    We spend $2$ tokens from $v$ to discard $v$, i.e., add $v$ to the set $D$.
    The invariants are again maintained.
 \end{enumerate}

 For the remainder we may assume that each undecided vertex either has an undecided child or exactly one child in $S$.
 Let $v$ be again a lowest undecided vertex in $T$, i.e., we have $s_v = 1$, and let $w$ denote the lowest parent of $v$ in $T$.

 \begin{enumerate}[{Case} 1]%[label = Case \Alph*:, start = 2]
    \setcounter{enumi}{2}
    \item $w$ has $t_w \geq 1$ tokens.

    We take one token from $v$ and one token from $w$ and use these $2$ tokens to discard $w$.
    This maintains \invref{inv:leaf-vertex} as $v$ loses one token but also one parent.
    % It might result in a negative token count at $v$, but as $v$ has no undecided children, \invref{inv:non-leaf-all-parents} is satisfied.
    
    \item $w$ has another undecided child $u$ different from $v$.

    We have that $u$ is a lowest undecided vertex in $T$, i.e., $u$ has no undecided children.
    By \invref{inv:inner-vertex} $w$ has $t_w \geq s_w \geq 0$ tokens.
    We take one token from $u$ and one from $v$ and use these $2$ tokens to discard $w$.
    This maintains \invref{inv:leaf-vertex} as $u$ and $v$ each lose one token but also one parent.
    % It might result in a negative token count at $u$ and or $v$, but as none of them has undecided children, \invref{inv:leaf-vertex} is satisfied.
 \end{enumerate}

 For the remainder we may assume that $v$ is the only undecided child of $w$ and that $w$ has $t_w \leq 0$ tokens.
 By \invref{inv:inner-vertex} we have $t_w \geq s_w \geq 0$, i.e., $t_w = s_w = 0$ and $w$ has no children in $S$.
 Thus we are left with the following case.

 \begin{enumerate}[{Case} 1]%[label = Case \Alph*:, start = 4]
    \setcounter{enumi}{4}
    \item $w$ has only $v$ as undecided child and no children in $S$.

    In this case we rely on induction.
    We perform a local modification on the $k$-tree model $(T,L)$, replacing $v$ and $w$ by a single new vertex $u$.
    To this end, let $P$ denote the set of all parents of $v$.
    Contract $v$ and $w$ into the immediate ancestor $z$ of $w$ in $T$.
    Add a new vertex $u$ as a leaf to $z$, make $P - \{w\}$ the parents of $u$ by giving $u$ the label of $v$, put $u$ into $S$ and put $1$ additional token on each vertex in $P - \{w\}$.
    This modification costs us $|P|-1 = p_v - 1$ tokens and provides us with $t_v+t_w$ tokens from $v$ and $w$, causing the total cost
    \[
     t_v + t_w - (p_v - 1) \overset{\invref{inv:leaf-vertex}}{\geq} s_v + p_v - k + 0 - p_v + 1 = 2 - k,
    \]    
    which we will balance out by taking one out of $v,w$ and discarding the other. Our choice will be determined by induction. 
    For now, observe that the new situation satisfies our invariants and has one vertex less.
    By induction we get the partition $(S',D')$ of the set $V' = V - (S \cup D \cup \{v,w\})$ of all remaining undecided vertices.
    
    If $S' \cap P = \emptyset$, then we discard $w$.
    Now all parents of $v$ are in $D \cup D'$.
    Thus we can take $v$ which has only $s_v = 1$ child in $S$.
    Taking $v$ grants us $k$ tokens and discarding $w$ costs us $2$ tokens, balancing the $2-k$ deficit from the modification.

    If $S' \cap P \neq \emptyset$, then let $a$ be a vertex in $S' \cap P$.
    We want to discard $v$ and take $w$.
    To see that this is possible, let $A$ denote the set of all parents of $w$.
    Then $P - \{w\} \subseteq A$.
    Recall that $P-\{w\}$ is the set of parents of the artificial vertex $u$.
    As $S' \cup S$ is $2$-clustered, $u,a \in S \cup S'$, and $A$ forms a clique in $G$, it follows that $G[S' \cup S - \{u\}]$ has a component of size $1$ only consisting of vertex $a$.
    As no child of $w$ is in $S \cup S'$, we can indeed take $w$ and discard $v$.
    Again, this grants us $k$ tokens and costs us $2$ tokens, balancing the $2-k$ deficit from the modification.
 \end{enumerate}

 Observe that by our invariants, this complete case distinction concludes the proof.
\end{proof}






%%%%%%%%%%%%%%%%%
%%    c = 3    %%
%%    k = 2    %%
%%%%%%%%%%%%%%%%%
% \section{$3$-clustered sets in graphs of treewidth $2$}
\section{The case of $3$-clustered sets in graphs of treewidth~$2$}
\label{sec:c3-k2-case}

We shall show in this section that for $k=2$ and $c=3$, the smallest $c$-clustered number of $n$-vertex graphs of treewidth $k$ is $\lceil \frac{5}{9}n \rceil$, i.e., $x_{2,3} = \frac59$.
Note that for $k=2$ and $c=3$ we have
\[
 \frac{c}{k+c+1} = \frac{1}{2} < \frac{5}{9} < \frac{3}{5} = \frac{c}{k+c},
\]
i.e., this value lies strictly between the general lower and upper bound in \cref{enum:general-c-k} of \cref{thm:main}.

We start with an explicit construction for the upper bound.
Let $G_1$ be the $10$-vertex graph shown in the left of \cref{fig:k2_c_3_counterexample}.
For an integer $i \geq 2$ let $G_i$ be the graph obtained from $i$ copies of $G_1$ by identifying vertex $v_6$ of each copy (except the last) with the vertex $v_1$ of the previous copy, and adding an edge between vertex $v_5$ of each copy (except the last) with vertex $v_2$ of the previous copy.
See the right of \cref{fig:k2_c_3_counterexample} for an illustration.
Note that $G_i$ has $9i+1$ vertices and treewidth $2$.
In fact, each $G_i$ is outerplanar as shown in \cref{fig:k2_c_3_counterexample}.

\begin{figure}
    \centering
    \includegraphics{figures/k2_c3_counterexample.pdf}
    \caption{Illustration of the graph $G_i$, $i \geq 1$, with $9i+1$ vertices and no $3$-clustered set of size more than $5i+2$.}
    \label{fig:k2_c_3_counterexample}
\end{figure}

\begin{proposition}\label{prop:x23-upper-bound}
 For each $i \geq 1$ we have $\alpha_3(G_i) \leq 5i+1 = \lceil \frac{5}{9}|V(G_i)| \rceil$.
\end{proposition}

\begin{proof}
 Let $S \subseteq V$ be a maximum $3$-clustered set in $G_i$.
 % Let $a$ be vertex $v_1$ of the first copy of $G_1[3,2]$ and $b$ be vertex $v_6$ of the last copy of $G_1[3,2]$.
 Let $A$ be the set of all vertices $v_1,v_3,v_4,v_6$ from all copies of $G_1$.
 This is, $A$ is the set of $3i+1$ vertices shown in white in \cref{fig:k2_c_3_counterexample}.
 First, we claim that without loss of generality we may assume that $A \subseteq S$.
 So assume that $v \in A$ with $v \notin S$.
 By maximality of $S$, there is a neighbor $u \in N_G(v)$ with $u \in S$.
 Note that $u \notin A$, as $A$ is an independent set in $G_i$.
 The vertices in $A$ are the simplicial vertices of $G_i$, giving that $N_G(v) - u \subseteq N_G(u) - v$.
 Hence $S' = S - u \cup \{v\}$ is also a $3$-clustered set of the same size with $|A - S'| < |A - S|$.
 
 Second, the vertices in $V - A$ are partitioned into $2i$ vertex-disjoint triangles, highlighted in gray in \cref{fig:k2_c_3_counterexample}.
 Given that $A \subseteq S$ and $S$ is $3$-clustered, observe that from each such triangle there is at most one vertex in $S$.
 Thus $|S| \leq |A| + 2i = 5i+1$, as desired.
\end{proof}

By \cref{prop:x23-upper-bound} we have $x_{2,3} \leq \frac59$.
For the proof of the lower bound $x_{2,3} \geq \frac{5}{9}$ we shall show that every graph $G$ of treewidth~$2$ admits a $3$-clustered set of size at least $\frac59 |V(G)|$.
We present this proof without using $k$-tree models.
Instead, it will be more convenient to work with a \emph{$2$-tree $G$ rooted at some edge $e_0$}, i.e., a graph that can be constructed starting with $e_0$ by iteratively adding a new vertex to the endpoints of an already constructed edge. We consider \emph{cut pairs} in $G$, i.e., edges $e = uv$ such that $G-u-v$ is disconnected.
There is a connected component of $G-u-v$ for each $w \in N(u) \cap N(v)$ where those in a different component than $u_0,v_0$ are called the children of $u$ and $v$.
In terms of a $2$-tree model $(T,L)$ of $G$ rooted at $u_0$, the children of $e=uv$, say with $v$ below $u$ in $T$, are the highest vertices $w$ in the subtree below $v$ with the label $L(w) \notin \{L(u),L(v)\}$.

\begin{proposition}
 For every graph $G=(V,E)$ of treewidth at most $2$, we have $\alpha_3(G) \geq \frac{5}{9}|V|$.
\end{proposition}

\begin{proof}
 Since every graph of treewidth $2$ is subgraph of a $2$-tree, we can assume without loss of generality that $G$ is a $2$-tree.
 We proceed by induction on $n = |V|$.
 If $n = 3$, then clearly $\alpha_3(G) = 3 \geq \frac{5}{9}\cdot 3$.
 So assume for the remainder that $n \geq 4$.

 We root $G$ at an arbitrary edge $e_0 = u_0v_0$, and for any edge $e = uv$ in $G$ call the common neighbors of $u$ and $v$ that are not in the component of $G - u - v$ that contains $u_0$ or $v_0$ the \emph{children} of $e$.
 Every vertex $w \neq u_0,v_0$ is the child of exactly one edge $e = uv$ and we call $u$ and $v$ the \emph{parents} of $w$.
 Note that for every edge one of its endpoints is a parent of the other endpoint (where by convention $u_0$ is the parent of $v_0$).
 Throughout this proof for each edge we always list the parent first, i.e., for any edge $uv$ vertex $u$ is a parent of vertex $v$.

 For each edge $e = uv$ and each subset $W$ of children of $e$, let $G_W$ denote the subgraph of $G$ induced by $W \cup \{u,v\}$ and all vertices that have an ancestor in $W$.
 % For simplicity, let us also write simply $G_w$ for $G_{\{w\}}$.
 Note that $G_W$ is a $2$-tree and we consider it rooted at edge $uv$.
 See \cref{fig:2-tree-illustration} for a schematic illustration.

 \begin{figure}
     \centering
     \includegraphics{figures/2-tree-illustration.pdf}
     \caption{A $2$-tree $G$ rooted at edge $e_0 = u_0v_0$, an edge $e = uv$ in $G$, its children $w_1,w_2,w_3$, and the $2$-tree $G_W$ for $W = \{w_2,w_3\}$ (blue).}
     \label{fig:2-tree-illustration}
 \end{figure}

 Our goal is to find an edge $e = uv$ and a subset $W$ of children of $e$ for which $G_W$ admits a $3$-clustered set $S_W$ with $S_W \cap \{u,v\} = \emptyset$ and $|S_W| \geq \frac{5}{9}|V(G_W)|$.
 Let us call such a set $W$ a \emph{good} set. For simplicity, if $|W|=1$, i.e., $W = \{w\}$, we also write $S_w$ and $G_w$ instead of $S_{\{w\}}$ and $G_{\{w\}}$.
 Once we have a good set $W$, the result easily follows by induction on $G' = G - V(G_W)$.
 In fact, let $S'$ be a $3$-clustered set in $G'$ of size $|S'| \geq \frac59 |V(G')|$, then $S = S' \cup S_W$ is a $3$-clustered set in $G$ of size
 \[
    |S| = |S'| + |S_W| \geq \frac59 |V(G')| + \frac59 |V(G_W)| = \frac59 |V(G)|.
 \]
 To find a good set, we consider edges in bottom-up order, that is, descendants before ancestors, starting with the edges that have no children.
 For each considered edge $e = uv$ with $W$ being a subset of all children of $e$, we store a $3$-clustered set $S_W$ of $G_W$ with $S_W \cap \{u,v\} = \emptyset$.
 We shall ensure that $|S_W| \geq \frac{5}{9}|V(G_W - \{u,v\})| = \frac{5}{9}(|V(G_W)| - 2)$.
 With this in mind, we define
 \[
    \mathrm{sp}(W) = 9|S_W| - 5(|V(G_W)| - 2)
 \]
 as the \emph{surplus} of $W$.
 Thus we shall ensure that $\mathrm{sp}(W) \geq 0$, while if $\mathrm{sp}(W) \geq 10$, then
 \[
    9|S_W| \geq 5(|V(G_W)|-2) + 10 = 5|V(G_W)| \quad \text{and thus} \quad |S_W| \geq \frac59 |V(G_W)|.
 \]
 In other words, if $\mathrm{sp}(W) \geq 10$, then $W$ is good.
 % Instead of finding one good vertex, we shall sometimes find a pair $w_1,w_2$ of vertices with the same parent edge $uv$, that are individually not good but are sufficient when considered together.
 % To this end, $w_1,w_2$ with the same parent edge $uv$ is called a \emph{good pair} if $|S_{w_1}\cup S_{w_2}| \geq \frac{5}{9}|V(G_{w_1}\cup G_{w_2})|$.
 % Observe that $S_{w_1}$ and $S_{w_2}$ are disjoint and there is no edge in $G$ between these sets.
 % I.e., $S_{w_1} \cup S_{w_2}$ is a $3$-clustered set of size $|S_{w_1}| + |S_{w_2}|$.
 % Further note that $w_1,w_2$ is a good pair if $\mathrm{sp}(w_1) + \mathrm{sp}(w_2) \geq 10$.

 Looking for good sets, let us assume again that we currently consider edge $e=uv$ with a subset $W$ of children of $e$.
 Besides determining $S_W$ and thereby $\mathrm{sp}(W)$, we also define the \emph{threat} at $u$ from $W$, denoted by $\mathrm{th}_W(u)$ as the total size of all components of $G[S_W]$ that contain a neighbor of $u$, i.e.,
 \[
    \mathrm{th}_W(u) = \# \{ x \in S_W \colon x \in C, C \text{ a component of } G[S_W], N(u) \cap C \neq \emptyset\}.
 \] 
 In other words, $S_W \cup \{u\}$ is a $3$-clustered set if and only if $\mathrm{th}_W(u) \leq 2$, i.e., the threat at $u$ from $W$ is at most $2$.
 The threat $\mathrm{th}_W(v)$ at the other endpoint $v$ of $e$ is defined analogously.
 For brevity, let us combine surplus and threats and simply say that
 \[
    \text{the \emph{type} of $W$ is $_\alpha(s)_\beta$ with $\alpha = \mathrm{th}_W(u)$, $s = \mathrm{sp}(W)$, $\beta = \mathrm{th}_W(v)$.}
 \] 
 For example, if $W = \{w\}$ and $w$ has no children, then we set $S_W = \{w\}$ and thus $\mathrm{sp}(W) = 4$, $\mathrm{th}_W(u) = \mathrm{th}_W(v) = 1$, which gives $W$ the type $_1(4)_1$.
 Note that $_\alpha(s)_\beta \neq {}_\beta(s)_\alpha$ for $\alpha \neq \beta$ because we assume that $u$ is a parent of $v$ and not vice versa.

 With these definitions in place, a good set $W$ is one whose type $_\alpha(s)_\beta$ satisfies $s \geq 10$.
 % And a good pair $w_1,w_2$ are two vertices with the same parent edge whose types $_{\alpha_1}(s_1)_{\beta_1}$ and $_{\alpha_2}(s_2)_{\beta_2}$ satisfy $s_1 + s_2 \geq 10$.
 In the entire argument below, we shall compute the types of sets in a bottom-up approach and either find a good set or encounter and store one of the following nine different types:
 \begin{equation}
    % _1(4)_1, \quad _1(6)_1, \quad _1(8)_1, \quad _1(7)_2, \quad _2(7)_1, \quad _1(9)_2, \quad _2(9)_1, \quad _2(8)_2, \quad _0(0)_0\label{eq:all-types}
    _0(0)_0 \quad _1(4)_1, \quad _2(8)_2, \quad _1(7)_2, \quad _2(7)_1, \quad _1(6)_1, \quad _1(9)_2, \quad _2(9)_1, \quad _1(8)_1\label{eq:all-types}
 \end{equation}
 The first type $_0(0)_0$ is used only for $W = \emptyset$ with the corresponding $3$-clustered set $S_W = \emptyset$ which has a surplus of $0$ and a threat of $0$ at each parent.
 To start the process, we set this type at every edge $e$ of $G$ that has \emph{no} children, i.e., for the set $W = \emptyset$ of all children of $e$.

 We proceed to consider a single vertex $w$ with parent edge $uv$, and assume that we already determined the type $_{\alpha_1}(s_1)_{\beta_1}$ of the set $X$ of all children of $uw$ and the type $_{\alpha_2}(s_2)_{\beta_2}$ of the set $Y$ of all children of $vw$, and that both these types are among the nine types in \eqref{eq:all-types}.
 % , each of whose children already has one of the above types. %, we distinguish three cases. 
 % Recall that $u$ and $v$ denote the parents of $w$.
 % Finally, we proceed by a complete case distinction and shall see that in all cases we find a good vertex or good pair.
 % Recall that $w$ has parent edge $uv$ and we already have processed all children of $w$.
 % Observe that each children of $w$ has either $uw$ or $vw$ as its parent edge.
 % By using the type $_0(0)_0$ if $a=0$ or $b=0$, we may assume that there are exactly $\max(1,a)$ types at $uw$ and exactly $\max(1,b)$ types at $vw$.
 % , i.e., there is at least one type at edge $uw$ and at least one type at edge $vw$.
 % Thus, the following also covers the situation that neither $uw$ nor $vw$ has children.
 In each of the cases below, we shall find a good set or define a $3$-clustered set $S_w$ such that the type of $W = \{w\}$ is again one of the nine types in \eqref{eq:all-types}.
 Note that the total threat at $w$ from the types of $X$ and $Y$ is $\beta_1 + \beta_2$.

 \begin{enumerate}[{Case} 1]
 % \begin{description}
    \item $\beta_1 + \beta_2 \leq 2$.{\ \\}
        We set $S_w = S_X \cup S_Y \cup \{w\}$.
        Since $\beta_1 + \beta_2 \leq 2$, the set $S_w$ is indeed a $3$-clustered set.
        The corresponding surplus is calculated by
        \begin{multline*}
            \mathrm{sp}(w) = 9|S_w|-5(|V(G_w)|-2) = 9(|S_X|+|S_Y|+1) - 5(|V(G_X)|-2+|V(G_Y)|-2+1)\\
            = 9|S_X|-5(|V(G_X)|-2) + 9|S_Y| - 5(|V(G_Y)|-2) + 9 - 5 = s_1+s_2 + 4,
        \end{multline*}
         i.e., the sum of the two surpluses plus $4$.
        Observe from the list \eqref{eq:all-types} of all types, that $\mathrm{sp}(w) \geq 10$ and thus $\{w\}$ a good set, unless we are in one of the following cases.
        \begin{itemize}
            \smallskip
            
            \item If $_{\alpha_1}(s_1)_{\beta_1} = {}_0(0)_0$ and $_{\alpha_2}(s_2)_{\beta_2} = {}_0(0)_0$, then the type of $\{w\}$ is $_1(4)_1$.
            
            \item If $_{\alpha_1}(s_1)_{\beta_1} = {}_0(0)_0$ and $_{\alpha_2} (s_2)_{\beta_2} = {}_1(4)_1$, then the type of $\{w\}$ is $_2(8)_2$.

            \item Symmetrically, if $_{\alpha_1}(s_1)_{\beta_1} = {}_1(4)_1$ and $_{\alpha_2}(s_2)_{\beta_2} = {}_0(0)_0$, the type of $\{w\}$ is $_2(8)_2$.

            \smallskip
        \end{itemize}
        In each case the type of $\{w\}$ is again on the list \eqref{eq:all-types}.
        % \textbf{a)} $s_1 = s_2 = 0$, \textbf{b)} $s_1 = 0$, $s_2 = 4$, or \textbf{c)} $s_1 = 4$, $s_2 = 0$.
        % But in the first case the type of $w$ is $_1(4)_1$, while in the latter two cases the type of $w$ is $_2(8)_2$, which are both on the list \eqref{eq:all-types}.

    \item $\beta_1 + \beta_2 \geq 3$.{\ \\}
        In this case we set $S_w = S_X \cup S_Y$ and calculate the surplus 
        \begin{multline*}
            \mathrm{sp}(w) = 9|S_w|-5(|V(G_w)|-2) = 9(|S_X|+|S_Y|)-5(|V(G_X)|-2+|V(G_Y)|-2+1)\\
            = 9|S_X|-5(|V(G_X)|-2) + 9|S_Y| - 5(|V(G_Y)|-2) - 5 = s_1+s_2 - 5,
        \end{multline*}
        i.e., the sum of the two surpluses minus $5$.
        For the threats at $u$ and $v$, we have $\mathrm{th}_w(u) = \alpha_1$ and $\mathrm{th}_w(v) = \alpha_2$.
        Again, observe from the list \eqref{eq:all-types} of all types, that $\mathrm{sp}(w) \geq 10$ and thus $\{w\}$ is a good set, unless we are in one of the following cases.
        \begin{itemize}
            \smallskip
            
            \item If $_{\alpha_1}(s_1)_{\beta_1} = {}_1(4)_1$ and $_{\alpha_2}(s_2)_{\beta_2} = {}_1(7)_2$, then the type of $\{w\}$ is $_1(6)_1$.

            \item If $_{\alpha_1}(s_1)_{\beta_1} = {}_1(4)_1$ and $_{\alpha_2}(s_2)_{\beta_2} = {}_1(9)_2$, then the type of $\{w\}$ is $_1(8)_1$.

            \item If $_{\alpha_1}(s_1)_{\beta_1} = {}_1(4)_1$ and $_{\alpha_2}(s_2)_{\beta_2} = {}_2(8)_2$, then the type of $\{w\}$ is $_1(7)_2$.

            \smallskip

            \item If $_{\alpha_1}(s_1)_{\beta_1} = {}_1(6)_1$ and $_{\alpha_2}(s_2)_{\beta_2} = {}_1(7)_2$, then the type of $\{w\}$ is $_1(8)_1$.
            
            \item If $_{\alpha_1}(s_1)_{\beta_1} = {}_1(6)_1$ and $_{\alpha_2}(s_2)_{\beta_2} = {}_2(8)_2$, then the type of $\{w\}$ is $_1(9)_2$.

            \smallskip
        \end{itemize}
        Here we omitted the symmetric cases, such as if $_{\alpha_1}(s_1)_{\beta_1} = {}_2(8)_2$ and $_{\alpha_2}(s_2)_{\beta_2} = {}_1(4)_1$, then the type of $\{w\}$ is $_2(7)_1$.
        As before, in each case the type of $\{w\}$ is again on the list \eqref{eq:all-types}.
 \end{enumerate}
 % \end{description}

Having computed the type of each one-element subset of children of $uv$, we proceed to combine these to obtain, again, either a good set or a type for the set $W$ of \emph{all} children of $uv$ to be one of the nine in \eqref{eq:all-types}.
To this end, assume that we already determined the type $_{\alpha_1}(s_1)_{\beta_1}$ of a non-empty subset $X$ of children of $uv$, and the type $_{\alpha_2}(s_2)_{\beta_2}$ of another non-empty subset $Y$ of children of $uv$ with $X \cap Y = \emptyset$, and that both these types are among the nine types in \eqref{eq:all-types}.
We shall consider the set $W = X \cup Y$ now.

We set $S_W = S_X \cup S_Y$ and calculate the surplus
\begin{multline*}
    \mathrm{sp}(w) = 9|S_w|-5(|V(G_w)|-2) = 9(|S_X|+|S_Y|)-5(|V(G_X)|-2+|V(G_Y)|-2)\\
    = 9|S_X|-5(|V(G_X)|-2) + 9|S_Y| - 5(|V(G_Y)|-2) = s_1+s_2,
\end{multline*}
i.e., the sum of the two surpluses.
For the threats at $u$ and $v$, we have $\mathrm{th}_W(u) = \alpha_1 + \alpha_2$ and $\mathrm{th}_W(v) = \beta_1 + \beta_2$.
Again, observe from the list \eqref{eq:all-types} of all types, that $\mathrm{sp}(W) \geq 10$ and thus $W$ is a good set, unless both, $X$ and $Y$ have type $_1(4)_1$.
But in this case $W$ has type $_2(8)_2$, which is again on the list \eqref{eq:all-types}.

\medskip

 Thus by the above, we either find a good set, or determine a $3$-clustered set $S_W$ for the set $W$ of \emph{all} children of $uv$, such that the corresponding type of $W$ is on the list \eqref{eq:all-types}.
 Finally, we should argue that we will eventually find a good set.
 To this end, observe that in each of the above cases, whenever we determine a type $_\alpha(s)_\beta$ of some set $W$ based on two already determined types $_{\alpha_1}(s_1)_{\beta_1}$ and $_{\alpha_2}(s_2)_{\beta_2}$, then the new type $_\alpha(s)_\beta$ is further right in the list \eqref{eq:all-types} than both $_{\alpha_1}(s_1)_{\beta_1}$ and $_{\alpha_2}(s_2)_{\beta_2}$.
 Thus, this procedure will eventually find a good set or we have determined the type of the set $W_0$ of all children of the base edge $u_0v_0$ and it is on the list \eqref{eq:all-types}.
 In this case $S = S_{W_0} \cup \{u_0\}$ is a $3$-clustered set and we have
 \[
    9|S| = 9|S_{W_0}|+9 = \mathrm{sp}(W_0) + 5(|V|-2) + 9 \geq 4 + 5|V| - 10 + 9 = 5|V|+5. 
 \]
 In particular $|S| \geq \frac59 |V|$, as desired.
\end{proof}


\subsection{Extension to larger $c$ and $k$}

Let us remark that the proof strategy for \cref{prop:x23-upper-bound} with types, surplus, and threats can be adjusted to each fixed $c$ and $k$, to certify lower bounds of the form $x_{k,c} \geq \frac{p}{q}$, and possibly also find matching upper bound examples.

For $k=2$, fixed $c$ and test threshold $p/q$, we determine (as in the proof above) for each edge $uv$ and set $W$ of children of $uv$ a corresponding $c$-clustered set $S_W$.
This determines a type, which includes the surplus $\mathrm{sp}(W) = q|S_W|-p(|V(G_W)|-2)$ and in general three (not just two) threats.
The threat at $u$ is the total size of components of $G[S_W]$ that contain a neighbor of $u$ \emph{but no} neighbor of $v$, the threat at $v$ is symmetrical, and the common threat at $u,v$ is the total size of components of $G[S_W]$ that contain a neighbor of $u$ \emph{and} a neighbor of $v$ (hence a vertex of $W$).
        
We start with the type corresponding to $W = \emptyset$, which has surplus~$0$ and each threat~$0$.
Then, we exhaustively combine two known types to a single type by \textbf{(1)} knowing the types for all children of $uw$ and all children of $vw$ and combining these to the type for the single children $W = \{w\}$ of $uv$, and \textbf{(2)} knowing the types of two disjoint subsets $X,Y$ of children of $uv$ and combining these to the type of $X \cup Y$.
If this, as in the above proof, generates only a \emph{finite} list of types (which implies that each surplus is non-negative) without cyclic dependencies, this proves that indeed $x_{2,c} \geq \frac{p}{q}$.
On the other hand, if we encounter a type with negative surplus, then tracing back the combinations, we obtain a particular $2$-tree $G$.
Linearly arranging copies of $G$, this could\footnote{it did, in all cases we considered} lead to a family of $2$-trees certifying that $x_{2,c} < \frac{p}{q}$.

We have implemented this strategy for $k=2$ and small $c$.
The obtained results suggest that the true value of $x_{2,c}$ depends on the parity of $c$ modulo $3$:
\[
    \renewcommand{\arraystretch}{1.3}
    \begin{tabularx}{\linewidth}{r||Z|W|Y|Z|W|Y|Z|W|Y|Z|W|Y|Z|W}        
        $c$ & $2$ & $3$ & $4$ & $5$ & $6$ & $7$ & $8$ & $9$ & $10$ & $11$ & $12$ & $13$ & $14$ & $15$ \\
        \hline
        $x_{2,c}$ &
        $\frac{1}{2}$ &
        $\frac{5}{9}$ &
        $\frac{8}{13}$ &
        $\frac{2}{3}$ &
        $\frac{9}{13}$ &
        $\frac{13}{18}$ &
        $\frac{3}{4}$ &
        $\frac{13}{17}$ &
        $\frac{18}{23}$ &
        $\frac{4}{5}$ &
        $\frac{17}{21}$ &
        $\frac{23}{28}$ &
        $\frac{5}{6}$ &
        $\frac{21}{25}$ \\[2pt]
    \end{tabularx}
\]

Finally, the same approach can work for larger $k$, but the computational effort increases.
In rooted $k$-trees we would have parent $k$-cliques $K$, a type would store a threat for each non-empty subset of $K$, and to determine the type of a single children $W = \{w\}$ of $K$, we would combine the known types of all children of the $k$-subsets of $K \cup \{w\}$ containing $w$.
As this approach determines $x_{k,c}$ only for singular values of $k$ and $c$, we did not embark upon this path.


% \section{Comparison to Wood's result}

% The following is due to Wood; see Theorem 22 in~\cite{Wo}.

% \begin{theorem}[Wood~\cite{Wo}]\label{wood}{\ \\}
%     Let $G=(V,E)$ be a graph on $n$ vertices and treewidth at most $k$.
%     If $n\leq \lfloor\frac{p}{k+1}\rfloor(c+1)+k+c-1$ and $k+1\leq p$, then there is a set $S\subseteq V$ of size $p$, such that all connected components of $G\setminus S$ have order at most $c$. 
% \end{theorem}

% Rearranging terms, the above theorem shows that
% \[
%     \inf\{ \frac{\alpha_c(G)}{|V(G)|} \colon \tw(G) = k\} \geq \frac{c-k}{c+1}.
% \]
% every $n$-vertex graph $G$ of treewidth $k$ admits a $c$-clustered set of size at least 

% \begin{proposition}
% Our Theorem (with offset one) strengthens~\Cref{wood} for $n\geq\frac{(k+c)^2-1}{k}$.
% \end{proposition}
% \begin{proof}
% Suppose that a strengthening of~\Cref{wood} holds, namely, only require $n\leq \frac{p}{k+1}(c+1)+k+c-1$. This implies $p\geq \frac{k+1}{c+1}n-\frac{(k+c-1)(k+1)}{c+1}$, and hence there exist a set $S$ of the latter size, such that all connected components of $G\setminus S$ have order at most $c$. In other words, there exists an induced subgraph on at least $n-(\frac{k+1}{c+1}n-\frac{(k+c-1)(k+1)}{c+1})=\frac{c-k}{c+1}n+\frac{(k+c-1)(k+1)}{c+1}$ vertices, such that all components have size at most $c$. 

% Now, we compare with the $c$-cluster (is this a good word?) guaranteed by our theorem.

% \begin{align*}
%     ~&\frac{c-k}{c+1}n+\frac{(k+c-1)(k+1)}{c+1}&\leq& \frac{c}{k+c+1}n\\
%     \iff & \frac{(c-k)(k+c+1)-c(c+1)}{(c+1)(k+c+1)}n&\leq& -\frac{(k+c-1)(k+1)}{c+1}\\
%     \iff & \frac{k(k+1)}{(k+c+1)}n&\geq& (k+c-1)(k+1)\\
%     \iff & n&\geq& \frac{(k+c)^2-1}{k}.
% \end{align*} 
% \end{proof}




\section{Other graph classes}
\label{sec:conclusions}

Let us briefly discuss other classes besides the class of treewidth-$k$ graphs.
In accordance with \eqref{eq:definition-x-kc}, for any graph class $\mathcal{G}$ let us define
\[
    x_{\mathcal{G},c} = \liminf \{\frac{\alpha_c(G)}{|V(G)|} \colon G \in \mathcal{G}\}\text{.}
\]

\begin{lemma}\label{lem:lb}
    If a graph class $\mathcal{G}$ is closed under vertex-disjoint unions and $G \in \mathcal{G}$ is $k$-connected on $k+c$ vertices, then $x_{\mathcal{G},c} \leq \frac{c}{k+c}$.
\end{lemma}
\begin{proof}
 Take a $k$-connected $G\in \mathcal{G}$ on $k+c$ vertices.
 Whenever we remove $k-1$ vertices, i.e., look at $\frac{c+1}{k+c}$ vertices, we get a connected subgraph on $c+1$ vertices.
 Hence, $\alpha_c(G) \leq c = \frac{c}{k+c}|V(G)|$, and  taking vertex-disjoint unions of $G$, we conclude $x_{\mathcal{G},c} \leq \frac{c}{k+c}$.
\end{proof}

In fact, \cref{obs:upper-bound} giving $x_{k,c} \leq \frac{c}{k+c}$ is just a special case of \cref{lem:lb} applied to the class of all graphs of treewidth~$k$.

Looking at the class $\mathcal{P}$ of all planar graphs, we can also apply~\cref{lem:lb}.
For $c\in \{2,\ldots,6\}$ we can take a $4$-connected planar graph on $c+4$ vertices, which gives an upper bound of $x_{\mathcal{P},c} \leq \frac{c}{c+4}$.
From $c\geq 7$ on one can even choose a $5$-connected planar graph and get an upper bound of $x_{\mathcal{P},c} \leq \frac{c}{c+5}$.
These bounds are not tight in general.
For instance for the icosahedron graph $G$ (shown in the middle of \cref{fig:example-clustered-sets}) we have $\alpha_5(G) = 6$ and $\alpha_6(G) = 7$, and taking vertex-disjoint unions of it yields the better bounds $\alpha_{\mathcal{P},5} \leq \frac{5}{12}$ and $\alpha_{\mathcal{P},6} \leq \frac{1}{2}$.
For $c=2$, \cref{lem:lb} yields $\alpha_{\mathcal{P},2} \leq \frac{1}{3}$ by taking the octahedron (shown on the left of \cref{fig:example-clustered-sets}), which we conjecture to be best-possible.

\begin{conjecture}\label{conj:1/3}
    In every planar graph there is a set $S$ on at least a third of the vertices, such that each vertex in $S$ is adjacent to at most one other vertex in $S$, i.e., $\alpha_{\mathcal{P},2}(G)=\frac{1}{3}$. 
\end{conjecture}

% We have verified \cref{conj:1/3} on all planar graphs up to $18$ vertices by computer.
We remark that \cref{conj:1/3} is implied by the Albertson-Berman Conjecture that every planar graph $G$ admits an induced forest on at least half of the vertices~\cite{AB79}.
In fact, any such forest would contain a $2$-clustered set on at least $\frac23$ of its vertices (hence $\frac13$ of the vertices of $G$) by \cref{enum:k1} of \cref{thm:main}.
Along the same lines, we get the best known lower bound by the acyclic $5$-coloring of Borodin~\cite{Bor79}, which implies the existence of an induced forest on at least $\frac25$ of the vertices. Hence, $\alpha_{\mathcal{P},2}(G) \geq \frac25 \cdot \frac23 = \frac{4}{15}$.

\medskip

Concerning $x_{\mathcal{G},c}$ asymptotically, recall that for any graph $G$ we have 
$\alpha_1(G) \leq \alpha_2(G) \leq \cdots$ and hence for any graph class $\mathcal{G}$ we have $x_{\mathcal{G},1} \leq x_{\mathcal{G},2} \leq \cdots \leq 1$.
Edwards and McDiarmid~\cite{EM94} define a graph class $\mathcal{G}$ to be \emph{fragmentable} if for every $\varepsilon > 0$ there exist integers $c,n_0$ such that each graph $G \in \mathcal{G}$ with $n \geq n_0$ vertices admits a $c$-clustered set of size at least $(1-\varepsilon)n$.
In other words, $\mathcal{G}$ is fragmentable if $x_{\mathcal{G},c} \to 1$ as $c \to \infty$.
Edwards and McDiarmid prove that any class with \emph{strongly sublinear separators}\footnote{There exists a fixed $\varepsilon < 1$ such that every $n$-vertex $G \in \mathcal{G}$ has a $S \subseteq V(G)$ of size  $O(n^{\varepsilon})$ such that each component of $G-S$ has at most $\frac{n}{2}$ vertices.} is fragmentable~\cite{EM94}.
% Edwards and McDiarmid prove that if there exist $0 < \alpha <1$ and $0 \leq \lambda < 1$ such that every $n$-vertex $G \in \mathcal{G}$ admits an $\alpha$-balanced separator of size $O(n^\lambda)$, then $\mathcal{G}$ is fragmentable~\cite{EM94}.
This includes planar graphs~\cite{LT79}, graphs of bounded orientable genus $g$~\cite{GHT84}, proper minor-closed graph classes~\cite{AST90}, $k$-planar graphs~\cite{GB07}, and touching graphs of $d$-dimensional balls~\cite{MTTV97}.

\begin{observation}
    A graph class $\mathcal{G}$ is fragmentable if and only if $\lim_{c \to \infty} x_{\mathcal{G},c} = 1$.
\end{observation}

%% ZDENEK told me that there are fragmentable classes w/o strongly sublinear separators
% It is worth noting that we are not aware of any graph class $\mathcal{G}$ that is fragmentable (equivalently $\lim_{c \to \infty} x_{\mathcal{G},c} = 1$) but that does not admit strongly sublinear separators.

While we might be able to derive an explicit lower bound on $\alpha_c(G)$ for any $c$ and any $G \in \mathcal{G}$ from a proof that $\mathcal{G}$ is fragmentable, our results suggest finding large induced subgraphs of bounded treewidth in $G$.
For example, for any proper minor-closed class $\mathcal{G}$ there exists a constant $k$ such that any graph $G = (V,E)$ with $G \in \mathcal{G}$ admits a $2$-coloring of $V$ for which each monochromatic induced subgraph has treewidth at most $k$~\cite{DDOSRSV04}.
This gives $\alpha_c(G)/|V| \geq \frac12 \cdot x_{k,c}$ for every $c \geq 0$.
%For example, for the class $\mathcal{P}$ of planar graphs, the constant is $k=2$.


% KONKRETE RECHNUNG fuer \alpha_c bei Graphen mit Euler genus g

% \begin{theorem}\label{large_c}
% For every $n$ vertex graph $G$ of orientable genus $g$, we have $$\alpha_c(G) \geq (1-\frac{2\sqrt{2(2g+3)}}{(\sqrt{2}-1)\sqrt{c}})n.$$
% \end{theorem}
% \begin{proof}
% We call a subset $X\subseteq V$ of an $n$ vertex graph $G=(V,E)$ a \emph{balanced} separator if every component of $G\setminus X$ is of size at most $\frac{n}{2}$. By~\cite[Theorem 14]{DMW17} every $n$-vertex graph of genus $g$ has treewidth at most $2\sqrt{2g+3}n-1$ and by~\cite[2.5]{RS86} every graph of treewidth $w$ has a balanced separator of order at most $w$. 
% Now, in~\cite[Lemma 14]{Woo18} it is shown that if every subgraph $G'\subset G$ has a balanced separator of size $\alpha|V(G')|^{\beta}$, then for all $c>1$ there exists $X \subseteq V$ of size at most $\frac{\alpha 2^\beta n}{(2^\beta-1)c^\beta}$ such that each component of $G-S$ has at most $c$ vertices. By the previous arguments for $G$ of genus $g$ we can set $\alpha=2\sqrt{2g+3}$ and $\beta=\frac{1}{2}$. This yields the result.
% \end{proof}

%EVENTUELL KANN MAN DIE GANZE VARIANTEN-DISKUSSION WEGLASSEN...

%\subsection{Maximum degree}
%Similarly to the clustering number one can consider the largest set of vertices inducing a subgraph of maximum degree $d$. Note that the $c$-clustering number is a lower bound for this parameter with respect to $d=c-1$. In particular note that for $c=2$ the parameter coincides with the case $d=1$. We do not know much about this problem except for a result of Chappell and Pelsmajer~\cite[Proposition 11]{ChPe}, saying the in trees this parameter is of order $\frac{d+1}{d+2}$. For planar graphs, note that the case $d=1$ is covered by Conjecture~\cref{conj:1/3}. For other values we have the following upper bounds
%$d=2$ $\frac{1}{2}$ Full stack of $K_4$, treewidth $3$
%$d=3$ $\frac{4}{7}$ Full stack of Octahedron (aka Small Triakis Octahedral Graph,) treewidth $4$
%$d=4$ $\frac{2}{3}$ Icosahedron, but also Figure~\cref{fig:example} of treewidth $3$.

%\begin{figure}[htp]
%    \centering
%    \includegraphics[width=.3\textwidth]{figures/n12tw3maxd4_8.pdf}
%    \caption{A planar triangulation on $12$ vertices and treewidth $3$ such that the largest induced subgraph of maximum degree $4$ has order $8$.}\label{fig:example}
%\end{figure}

%When considering treewidth $k$ graphs, one can take as lower bound on the $d$-number(WIE SOLL ES HEISSEN?) the maximum of out lower bound for the $c+1$-clustering number and the $d$-number, where addtioonally the indiced graph has to be a forest, given by~\cite{ChPe}

%\subsection{Degeneracy}

%\begin{observation}
   % For every $k,d$ with $d \leq k$ the following hold:
   % \begin{itemize}
        %\item There is a graph $G$ with $\tw(G) = k$ such that for every $d$-degenerate induced subgraph $H$ of $G$ we have $|V_H| \leq \frac{d+1}{k+1}|V_G|$.

        %\item For every graph $G$ with $\tw(G) = k$ there exists a $d$-degenerate induced subgraph $H$ of $G$ with $|V_H| \geq \frac{d+1}{k+1}|V_G|$.
    %\end{itemize}
%\end{observation}

%Indeed, for the particular example graph $G$ it is enough to consider (vertex-disjoint unions of) $K_{k+1}$.
%And if $G$ is any graph with $\tw(G) = k$, it is well-known (see for example~\cite{??}) that $G$ admits a $(k+1)$-coloring of its vertices such that every $(d+1)$-subset of colors induces a subgraph $H$ with $\tw(H) \leq d$.
%I.e., taking the $d+1$ largest color classes, we have $|V_H| \geq \frac{d+1}{k+1}|V_G|$, as desired.

\subsection*{Acknowledgments}

The authors thank David Wood for pointing us to~\cite{Wo} and Thomas Bl\"asius for mentioning the connection to $k$-vertex separators and $\ell$-component order connected sets.


\bibliographystyle{plainurl}
\bibliography{lit}


\end{document}
