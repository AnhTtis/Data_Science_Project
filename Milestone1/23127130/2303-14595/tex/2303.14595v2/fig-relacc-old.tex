\begin{figure}[!t]
    \centering
    \includegraphics[width=\linewidth]{figures/cifar100-svd.pdf}
    \includegraphics[width=\linewidth]{figures/cifar100-svd-rel.pdf}
    \includegraphics[width=\linewidth]{figures/cifar100-acc.pdf}
    \includegraphics[width=\linewidth]{figures/cifar100-acc-rel.pdf}
    \caption{
    Feature distribution and importance for classification using an oracle CL method (JT) and a naive one (FT). Upper plot: feature variance (singular values) along each principal direction. Lower plot: classification accuracies using features that are projected on a subspace formed by $k$ principal vectors (note that the right endpoint of all curves already, saturates, achieving more than 99\% of the full accuracy). Different curves are obtained at the end of different tasks during continual learning, and the legend shows how many classes have been seen so far. As the model is trained on more classes, more feature dimensions are used for classification, but still remain a small portion compared to the full feature dimension (512). 
    \qiao{Are second and fourth plot better? - Make it a 2x2 grid of plots. }
    }
    \label{fig:relacc}
\end{figure}
