% CVPR 2023 Paper Template
% based on the CVPR template provided by Ming-Ming Cheng (https://github.com/MCG-NKU/CVPR_Template)
% modified and extended by Stefan Roth (stefan.roth@NOSPAMtu-darmstadt.de)

\documentclass[10pt,twocolumn,letterpaper]{article}

%%%%%%%%% PAPER TYPE  - PLEASE UPDATE FOR FINAL VERSION
% \usepackage[review]{cvpr}      % To produce the REVIEW version
% \usepackage{cvpr}              % To produce the CAMERA-READY version
\usepackage[pagenumbers]{cvpr} % To force page numbers, e.g. for an arXiv version

\makeatletter
\@namedef{ver@everyshi.sty}{}
\makeatother

% Include other packages here, before hyperref.
\usepackage{graphicx}
\usepackage{amsmath}
\usepackage{amssymb}
\usepackage{booktabs}
\usepackage{multirow}
\usepackage{xcolor}
\usepackage{algorithmic}
\usepackage{algorithm}
\usepackage{multicol}

\usepackage{twoopt}
\usepackage{tikz}
\usetikzlibrary{fit}
\newcommand\tikzmark[1]{%
  \tikz[remember picture,overlay]\node[inner xsep=0pt] (#1) {};}

\newcommandtwoopt\TextboxReservoir[5][2.5cm][2cm]{%
\begin{tikzpicture}[remember picture,overlay]
  \coordinate (aux) at ([xshift=#1]#4);
  \node[inner ysep=5pt,yshift=0.6ex,draw=black,thick,
    fit=(#3) (aux),baseline] 
    (box) {};
\end{tikzpicture}%
}
\newcommandtwoopt\TextboxBalancedReservoir[5][2.5cm][2cm]{%
\begin{tikzpicture}[remember picture,overlay]
  \coordinate (aux) at ([xshift=#1]#4);
  \node[inner ysep=17pt,yshift=0.5ex,draw=black,thick,
    fit=(#3) (aux),baseline] 
    (box) {};
\end{tikzpicture}%
}

% It is strongly recommended to use hyperref, especially for the review version.
% hyperref with option pagebackref eases the reviewers' job.
% Please disable hyperref *only* if you encounter grave issues, e.g. with the
% file validation for the camera-ready version.
%
% If you comment hyperref and then uncomment it, you should delete
% ReviewTempalte.aux before re-running LaTeX.
% (Or just hit 'q' on the first LaTeX run, let it finish, and you
%  should be clear).
\usepackage[pagebackref,breaklinks,colorlinks]{hyperref}


% Support for easy cross-referencing
\usepackage[capitalize]{cleveref}
\crefname{section}{Sec.}{Secs.}
\Crefname{section}{Section}{Sections}
\Crefname{table}{Table}{Tables}
\crefname{table}{Tab.}{Tabs.}

\usepackage[accsupp]{axessibility}  % Improves PDF readability for those with disabilities.

\newcommand{\qiao}[1]{\textcolor{cyan}{\textbf{Qiao:} #1}}
\newcommand{\qiaocr}[1]{\textcolor{red}{\textbf{Qiao:} #1}}
% \newcommand{\crchange}[1]{\textcolor{red}{#1}}
\newcommand{\crchange}[1]{#1}

%%%%% GENERAL MATH COMMANDS
% Reals
\newcommand{\R}{{\mathbb R}}
% Integers
\newcommand{\Z}{{\mathbb Z}}
% Naturals
\newcommand{\N}{{\mathbb N}}
% Expectation
\DeclareMathOperator*{\E}{\mathbb{E}}
% ^th notation
\newcommand{\tth}{^{\text{th}}}
% Small dots for integer range [a .. b]
\newcommand{\sdots}{\,..\,}
% Vectorized version of matrix
\newcommand{\matvec}{\mbox{vec}}

% := sign
\newcommand{\defeq}{\vcentcolon=}
% Zero function
\newcommand{\zf}{\mathbf{0}}
% Vector of ones
\newcommand{\ones}{\mathbf{1}}

% Argmin and argmax definitions
\DeclareMathOperator*{\argmax}{arg\,max}
\DeclareMathOperator*{\argmin}{arg\,min}


%%%%% PROBLEM STATEMENT NOTATION 
% \newcommandtwoopt{\St}[2][t][]{{S_{#1}^{#2}}} % State
\newcommand{\task}[1][i]{{\mathcal{T}_{#1}}} % Task, optionally takes index
\newcommand{\tasks}{\{ \task \}_{i=1}^N}
\newcommand{\losst}[1][i]{{l_{#1}}}
\newcommand{\lossv}[1][i]{{l_{#1}^{\textrm{val}}}}
\newcommand{\tasktarget}{{\mathcal{T}_{\textrm{target}}}}
\newcommand{\lossttarget}{l_{\textrm{target}}}
\newcommand{\lossvtarget}{l_{\textrm{target}}^{\textrm{val}}}
\newcommand{\lossttargetit}{l_{\textrm{target}}^{(k)}}
\newcommand{\losstotal}{l^{\textrm{total}}}
\newcommand{\lossopt}{l^*}

\newcommand{\thetait}[2]{\theta_{#1}^{(#2)}}
\newcommand{\phit}[1]{\phi^{(#1)}}
\newcommand{\hist}[2]{S_{#1}^{(#2)}}
\newcommand{\grad}[2]{G_{#1}^{(#2)}}

\newcommand{\Alg}{\textup{\textbf{Opt}}}
\newcommand{\MetaAlg}{\textup{\textbf{MetaOpt}}}

%%%%% Theorems
\newtheoremstyle{mytheoremstyle} % name
    {\topsep}                    % Space above
    {\topsep}                    % Space below
    {\itshape}                   % Body font
    {}                           % Indent amount
    {\scshape}                   % Theorem head font
    {.}                          % Punctuation after theorem head
    {.5em}                       % Space after theorem head
    {}  % Theorem head spec (can be left empty, meaning ‘normal’)
\theoremstyle{mytheoremstyle}
\theoremstyle{plain}
\newtheorem{theorem}{Theorem}
\newtheorem{proposition}{Proposition}
\newtheorem{assumption}{Assumption}
\newtheorem{definition}{Definition}
\newtheorem{lemma}{Lemma}
\theoremstyle{remark}
\newtheorem{remark}{Remark}


%%%%%%%%% PAPER ID  - PLEASE UPDATE
\def\cvprPaperID{8341} % *** Enter the CVPR Paper ID here
\def\confName{CVPR}
\def\confYear{2023}


\begin{document}

%%%%%%%%% TITLE - PLEASE UPDATE
\title{
% Reducing Catastrophic Forgetting by Learning Backward Feature Projection 
Preserving Linear Separability in Continual Learning \\by Backward Feature Projection
% Trading off Feature Space Plasticity and Stability in Continual Learning by Backward Feature Projection
% \qiao{Better Title? Better Method Name?}
}

\author{Qiao Gu\\
University of Toronto\\
% Institution1 address\\
{\tt\small qgu@cs.toronto.edu}
% For a paper whose authors are all at the same institution,
% omit the following lines up until the closing ``}''.
% Additional authors and addresses can be added with ``\and'',
% just like the second author.
% To save space, use either the email address or home page, not both
\and
Dongsub Shim\\
LG AI Research\\
{\tt\small dongsub.shim@lgresearch.ai}
\and
Florian Shkurti\\
University of Toronto\\
{\tt\small florian@cs.toronto.edu}
}
\maketitle

%%%%%%%%% ABSTRACT
\begin{abstract}
    Catastrophic forgetting has been a major challenge in continual learning, where the model needs to learn new tasks with limited or no access to data from previously seen tasks. To tackle this challenge, methods based on knowledge distillation in feature space have been proposed and shown to reduce forgetting~\cite{douillard2020podnet, dhar2019learning, kang2022class}. However, most feature distillation methods directly constrain the new features to match the old ones, overlooking the need for plasticity. To achieve a better stability-plasticity trade-off, we propose Backward Feature Projection (BFP), a method for continual learning that allows the new features to change up to a learnable linear transformation of the old features. BFP preserves the linear separability of the old classes while allowing the emergence of new feature directions to accommodate new classes. BFP can be integrated with existing experience replay methods and boost performance by a significant margin. We also demonstrate that BFP helps learn a better representation space, in which linear separability is well preserved during continual learning and linear probing achieves high classification accuracy. The code can be found at \href{https://github.com/rvl-lab-utoronto/BFP}{https://github.com/rvl-lab-utoronto/BFP}. 
    \vspace{-1em}
\end{abstract}

%%%%%%%%% BODY TEXT

\section{Introduction}
\IEEEPARstart{T}{he} method Neural Radiance Fields (NeRF)~\cite{mildenhall2020nerf} is proposed for photorealistic novel view synthesis. Given many views of the scene, it creates implicit multi-view geometry and learns for view synthesis. However, it has poor generalizations to new scenes and requires retraining or fine-tuning on each scene. 
 
 Recent work~\cite{Yu_2021_CVPR,Trevithick_2021_ICCV} has explored the ways of using a single image to train NeRF. They introduce a convolutional feature encoder to learn the image representation which gives it some limited generalization abilities to unseen scenes.  But, without fine-tuning, these methods produce many floats and artifacts in rendering novel views. 
 
  Multi-Plane Images (MPI) representation that learns multiple RGB images from a single image is also used in \cite{Wu_2021_ICCV,Tucker_2020_CVPR,wu2022remote} for  novel view synthesis. However, MPI heavily relies on the qualities of the planar images and needs plenty of image planes to avoid blurs. There is no strong 3D geometry constraint and it fails in many complex scenes.
  
  MINE~\cite{Li_2021_ICCV2} introduces the volume rendering of NeRF into the MPI. It runs faster and produces better depth rendering quality compared with single-view NeRFs~\cite{Yu_2021_CVPR,Trevithick_2021_ICCV}. However, the rendering quality heavily relies on the number of image planes. It needs high-resolution 4D volumes to store the 4-channel  (RGB and volume density) image planes that cost a large amount of GPU memory in both training and 
 prediction.  
 

 
 \begin{figure}[t]
\setlength{\abovecaptionskip}{7pt}
\setlength{\belowcaptionskip}{0pt}
	\centering
% 	\subfigure[MINE (PSNR:14.9)]{  % for AAAI
	\subfloat[MINE (PSNR:14.9)]{
%			\centering
			\includegraphics[width=0.23\textwidth]{figure/intro/DJI_20200223_163206_598_0_MINE.png}
%			\label{subfig:pixelnerf}
	}\subfloat[MINE (depth)]{
%			\centering
			\includegraphics[width=0.23\textwidth]{figure/intro/MINE_disp.png}
%			\label{subfig:mpi}
	}
	\\[-3mm]
	\subfloat[Ours (PSNR:17.0)]{
%			\centering
			\includegraphics[width=0.23\textwidth]{figure/intro/DJI_20200223_163206_598_0_ours.png} 
	}\subfloat[Ours (depth)]{
%			\centering
			\includegraphics[width=0.23\textwidth]{figure/intro/ours_disp.png}
	}
	\caption{Comparison with state-of-the-art methods. (a-b) RGB and depth rendering results of  \cite{Li_2021_ICCV2}. It produces many blurs and floats in the occluded regions and at the object/depth edges. 
	(c-d) Our method employs a joint rendering mechanism that preserves more image details and predicts sharp depth edges.}
	\label{fig:performance_illustration}
\end{figure}
 
 In this paper, we propose a joint rendering mechanism that takes the MPI strategy for coarse sampling proposals and the MLP\&volume-based rendering~\cite{mildenhall2020nerf} for fine sampling and rendering. Then, both the coarse point samples and the fine samples are combined according to their geometry distribution to realize a more accurate joint rendering. More importantly, we introduce a depth teacher net that serves as the guidance for the joint rendering. The monocular depth teacher predicts dense pseudo depth maps that assist the consistent 3D geometry learning between the MPI, the fine volume, and the joint rendering. It also boosts the multi-view geometry consistency between the source view and the target novel views that 
helps handle the occlusions, reduce the blurs and floats, and render accurate depths. 
 
In the experiments,  we verify the effectiveness of our method on three challenging real-scene datasets (RealEstate10K~\cite{zhou2018stereo}, NYU~\cite{silberman2012indoor} and  NeRF-LLFF~\cite{mildenhall2020nerf}) for novel view synthesis or depth estimation. Given a single image as input, our method is shown able to produce higher qualities in both the RGB image rendering and depth map prediction. It far outperforms state-of-the-art methods~\cite{Li_2021_ICCV2,Yu_2021_CVPR} with improvements of 5$\sim$20\% in PSNR and SSIM for the RGB rendering and reduces 20$\sim$50\% of the errors for the depth prediction.

\vspace{-4mm}
\section{Related Works}
\noindent\textbf{Sign Language Recognition.} Sign language recognition (SLR) is a fundamental task in the field of sign language understanding.
Feature extraction plays a key role in an SLR model.
% 
Most recent SLR works \cite{jiang2021sign, jiang2021skeleton, hu2021signbert, hu2021hand, li2020transferring, li2020word, joze2019ms, hu2021global, stmc, zuo22_interspeech, vac} adopt CNN-based architectures, \eg, I3D \cite{I3D} and R3D \cite{qiu2017learning}, to extract vision features from RGB videos.
In this work, we adopt S3D \cite{xie2018rethinking} as the backbone of our VKNet due to its excellent accuracy-speed trade-off.

However, RGB-based SLR models may suffer from the large variation of video backgrounds. 
As a complement, some SLR works \cite{jiang2021skeleton, jiang2021sign, hu2021hand, hu2021signbert, chentwo} explore to jointly model RGB videos and keypoints.
For example, SAM-SLR \cite{jiang2021skeleton} uses graph convolutional networks (GCNs) to model pre-extracted keypoints.
HMA \cite{hu2021hand} and SignBERT \cite{hu2021signbert} propose to decode 3D hand keypoints from RGB videos.
A common deficiency of these works is that they need a dedicated network to model keypoints.
In this work, we represent keypoints as a sequence of heatmaps~\cite{duan2022revisiting, chentwo} so that the keypoint encoder of our VKNet can share the identical architecture with the video encoder.

To enable mini-batch training, previous works \cite{jiang2021sign, jiang2021skeleton, hu2021signbert, hu2021hand, li2020transferring, li2020word} crop fixed-length clips from raw videos as model inputs.
However, the model may overfit to the training videos of fixed temporal receptive fields.
In contrast, our VKNet is trained on videos with varied temporal receptive fields to improve its generalization capability.



\noindent\textbf{Word Representation Learning.}
Word2vec \cite{word2vec} and GloVe \cite{glove} are two classical word representation learning frameworks in the field of NLP.
Based on word2vec, fastText \cite{mikolov2018advances} improves word representations with several modifications including the use of sub-word information \cite{bojanowski2017enriching} and position independent features \cite{mnih2013learning}.
Although some advanced language models, \eg, BERT \cite{kenton2019bert}, can also be used to extract word representations, they are computationally intensive and are not dedicated to word representation learning.
In this paper, we adopt the lightweight but effective fastText, which is also used in a recent sign language translation work \cite{yin2021simulslt}, to pre-compute gloss (word) representations.


\noindent\textbf{Vision-Language Models.}
Recently, a majority of vision-language models \cite{clip, align, yao2022filip, gu2022wukong} learn visual representations on large-scale image-text pairs.
Among them, CLIP \cite{clip} is the pioneer to jointly optimize an image encoder and a text encoder through a contrastive loss. 
% 
Besides, the pre-trained CLIP can be generalized to various downstream tasks, \eg, semantic segmentation \cite{xu2022groupvit, li2021language, xu2021simple}, object detection \cite{du2022learning, rao2022denseclip}, image classification~\cite{zhou2022learning,huang2022unsupervised}, and style transfer \cite{patashnik2021styleclip, kwon2022clipstyler}.
In this work, we exploit the implicit knowledge included in glosses (sign labels), which is distinct from previous works on vision-language modeling.


\noindent\textbf{Multi-label Classification.} Real-world objects may have multiple semantic meanings, which motivates research on multi-label classification \cite{ridnik2021asymmetric, ke2022hyperspherical, zhang2013review, rajeswar2022multi, kim2022large} requiring models to map inputs to multiple possible labels.
Although the VISigns may be associated with the multi-label classification problem, most widely-adopted SLR datasets \cite{li2020word, joze2019ms, hu2021global} are singly labeled.
In this work, we deal with the VISigns by incorporating language information included in glosses.


\vspace{-0.3em}
\section{Method}
\vspace{-0.3em}

Our sensitivity-aware visual parameter-efficient fine-tuning consists of two stages. In the first stage, SPT measures the task-specific sensitivity for the pre-trained parameters (Section~\ref{subsec:sensitivity}). Based on the parameter sensitivity and a given parameter budget, SPT then adaptively allocates trainable parameters to task-specific important positions (Section~\ref{subsec:SPT}).

\vspace{-0.3em}
\subsection{Task-specific Parameter Sensitivity}
\label{subsec:sensitivity}
\vspace{-0.3em}

Recent research has observed that pre-trained backbone parameters exhibit varying feature patterns~\cite{raghu2021vision,naseer2021intriguing} and criticality~\cite{zhang2019all,chatterji2019intriguing} at distinct positions. 
Moreover, when transferred to downstream tasks, their efficacy varies depending on how much pre-trained features are reused and how well they adapt to the specific domain gap~\cite{yosinski2014transferable,kumar2022finetuning,neyshabur2020being}. Motivated by these observations, we argue that not all parameters contribute equally to the performance across different tasks in PEFT and propose a new criterion to measure the sensitivity of the parameters in the pre-trained backbone for a given task.

Specifically, given the training dataset $\gD_t$ for the $t$-th task and the pre-trained model weights $\vw=\left\{w_1, w_2, \ldots, w_N\right\}\in \sR^N$ where $N$ is the total number of parameters, the objective for the task is to minimize the empirical risk: $\min_{\vw} E(\gD_t, \vw)$.
We denote the parameter sensitivity \bohan{set} as $\gS=\{s_1, \ldots, s_N\}$ and the sensitivity $s_n$ for parameter $w_n$ is measured by the empirical risk difference when tuning it:
\begin{equation}
\vspace{-0.3em}
    \begin{aligned}
        s_n = E(\gD_t, \vw)-E(\gD_t, \vw\mid w_n=w_n^*),
    \end{aligned}
\label{eq:sensitivity}
\end{equation}
where $w_n^*=\underset{w_n}{\rm argmin}(E(\gD_t, \vw))$. We can reparameterize the tuned parameters as  $w_n^*=w_n+\Delta_{w_n}$, where $\Delta_{w_n}$ denotes the update for $w_n$ after tuning. Here we individually measure the sensitivity of each parameter, which is reasonable given that most of the parameters are frozen during fine-tuning in PEFT. However, it is still computationally intensive to compute Eq.~(\ref{eq:sensitivity}) for two reasons. Firstly, getting the empirical risk for $N$ parameters requires forwarding the entire network $N$ times, which is time-consuming. Secondly, it is challenging to derive $\Delta_{w_n}$, as we have to tune each individual $w_n$ until convergence.

{\begin{algorithm}[t!]
\caption{\label{alg:tps} Computing task-specific parameter sensitivities}
\begin{algorithmic}
    \STATE \textbf{Input:} Pre-trained model with network parameters $\vw$, training set $\gD_t$ for the $t$-th task, and number of training samples $C$ used to calculate the parameter sensitivities
    \STATE \textbf{Output:} Sensitivity set $\gS=\{s_1, \ldots, s_N\}$
    \STATE Initialize $\gS=\{0\}^N$
    \FOR{$i\in\{1,\ldots,C\}$}
        \STATE Get the $i$-th training sample of $\gD_t$
	    \STATE Compute loss $E$
		\STATE Compute gradients $\vg$
		\FOR{$n\in\{1,\ldots,N\}$}
                \STATE Update sensitivity for the $n$-th parameter: $s_{n} = s_{n} + g_n^2$
		    \ENDFOR
    \ENDFOR
\end{algorithmic}
\end{algorithm}}


\begin{figure*}[t]
\begin{center}
    \includegraphics[width=\linewidth]{main_figure.pdf}
\end{center}\vspace{-2em}
\caption{Overview of our trainable parameter allocation strategy. With the parameter sensitivity \bohan{set} $\gS$, we first get the top-$\tau$ sensitive parameters. Instead of directly tuning these sensitive parameters, we also boost the representational capability by replacing unstructured tuning with structured tuning at sensitive weight matrices that have a large number of sensitive parameters, which can be implemented by an existing structured tuning method, \eg, LoRA~\cite{hu2022lora} and Adapter~\cite{houlsby2019parameter}. Red lines and blocks represent trainable parameters and modules, while blue lines represent frozen parameters.}
\label{fig:main}
\vspace{-1.5em}
\end{figure*}


To overcome the first barrier, we simplify the empirical loss by approximating $s_n$ in the vicinity of $\vw$ by its first-order Taylor expansion
\vspace{-0.3em}
\begin{equation}
\vspace{-0.5em}
    \begin{aligned}
        s_n^{(1)} = -g_n\Delta_{w_n},
    \end{aligned}
\label{eq:first-order}
\end{equation}
where the gradients $\vg=\partial E/\partial\vw$, and $g_n$ is the gradient of the $n$-th element of $\vg$. 
To address the second barrier, following~\cite{liu2018darts,cai2018proxylessnas}, we take the one-step unrolled weight as the surrogate for $w_n^*$ and approximate $\Delta_{w_n}$ in Eq.~(\ref{eq:first-order}) with a single step of gradient descent. We can accordingly get $s_n^{(1)} \approx g_n^2\epsilon$,
where $\epsilon$ is the learning rate. Since $\epsilon$ is the same for all parameters, we can eliminate it when comparing the sensitivity with the other parameters and finally get 
\vspace{-0.5em}
\begin{equation}
\vspace{-0.3em}
    \begin{aligned}
        s_n^{(1)} \approx g_n^2.
    \end{aligned}
\label{eq:first-order-simp}
\end{equation}
Therefore, the sensitivity of a parameter can be efficiently measured by its potential to reduce the loss on the target domain. Note that although our criterion draws inspiration from pruning work~\cite{molchanov2019importance}, it is distinct from it. \cite{molchanov2019importance} measures the parameter importance by the squared change in loss when removing them, \ie, $\left( E(\gD_t, \vw)-E(\gD_t, \vw\mid w_n=0) \right)^2$ and finally derives the parameter importance by $\left( g_n w_n \right)^2$, which is different from our formulations in Eqs.~(\ref{eq:sensitivity}) and~(\ref{eq:first-order-simp}).

In practice, we accumulate $\gS$ from a total number of $C$ training samples ahead of fine-tuning to generate accurate sensitivity as shown in Algorithm~\ref{alg:tps}, where $C$ is a pre-defined hyper-parameter. In Section~\ref{subsec:abl}, we show that employing only 400 training samples is sufficient for getting reasonable parameter sensitivity, which requires only 5.5 seconds with a single GPU for any VTAB-1k dataset with ViT-B/16 backbone~\cite{vit}.

\vspace{-0.3em}
\subsection{Adaptive Trainable Parameters Allocation}
\label{subsec:SPT}
\vspace{-0.2em}

Our next step is to allocate trainable parameters based on the obtained parameter sensitivity set $\gS$ and a desired parameter budget $\tau$. A straightforward solution is to directly tune the top-$\tau$ most sensitive unstructured connections (parameters) \rev{while keeping the rest frozen}, which we name unstructured tuning. Specifically, we select the top-$\tau$ most sensitive weight connections in $\gS$ to form the sensitive weight connection set $\gT$. Then, for \rev{a} weight matrix $\mW\in \sR^{d_{\rm in}\times d_{\rm out}}$, we can get a binary mask $\mM\in \sR^{d_{\rm in}\times d_{\rm out}}$ computed by
\vspace{-0.5em}
\begin{equation}
\vspace{-0.5em}
    {\begin{array}{ll}
    \small
    \begin{aligned}
    \mM^j =
    \left\{\begin{array}{ll} 
    1 ~~~~~ \mW^j \in \gT \\
    0 ~~~~~ \mW^j \notin \gT
    \end{array}\right.
    \end{aligned},
    \small
    \end{array}}
\label{eq:mask}
\end{equation}
where $\mW^j$ and $\mM^j$ are the $j$-th element in $\mW$ and $\mM$, respectively. Accordingly, we can train the sensitive parameters by gradient descent and the updated weight matrix can be formulated as $\mW'\leftarrow \mW - \epsilon\vg_{\mW}\odot\mM$, where $\vg_{\mW}$ is the gradient for $\mW$.

However, considering PEFT approaches generally limit the proportion of trainable parameters to less than 1\%, tuning only a small number of unstructured weight connections might not have enough representational capability to handle the downstream datasets with large domain gaps from the source pre-training data. Therefore, to improve the representational capability, we propose to replace unstructured tuning with structured tuning at the sensitive weight matrices that have a high number of sensitive parameters. To preserve the parameter budget, we can implement structured tuning with an existing efficient structured tuning PEFT method~\cite{hu2022lora,chen2022adaptformer,houlsby2019parameter,jie2022convolutional} that learns to directly adjust \rev{all hidden dimensions at once}. We depict an overview of our trainable parameter allocation strategy in Figure~\ref{fig:main}. For example, we can employ the low-rank reparameterization trick LoRA~\cite{hu2022lora} to the sensitive weight matrices \rev{and the one-step update for $\mW$ can be formulated as}
\vspace{-0.4em}
\begin{equation}
\vspace{-0.4em}
    {\begin{array}{ll}
    \small
    \begin{aligned}
    \mW' = \left\{\begin{array}{ll} 
    \mW + \mW_{\rm down}\mW_{\rm up} & ~~ \text { if } ~~ \sum_{j=0}^{d_{\rm in}\times d_{\rm out}} \mM^j \geq \sigma_{\rm opt} \\
    \mW - \epsilon\vg_{\mW}\odot\mM & ~~ {\rm otherwise}
    \end{array}\right.
    \end{aligned},
    \small
    \end{array}}
\label{eq:weight_updat}
\end{equation}
where $\mW_{\rm down}\in \sR^{d_{\rm in}\times r}$ and $\mW_{\rm up}\in \sR^{r\times d_{\rm out}}$ are two learnable low-rank matrices to approximate the update of $\mW$ and rank $r$ is a hyper-parameter where $r \ll {\rm min}(d_{\rm in},d_{\rm out})$. In this way, we perform structured tuning on $\mW$ when its number of sensitive parameters exceeds $\sigma_{\rm opt}$, whose value depends on the pre-defined type of structured tuning method. For example, since implementing structured tuning with LoRA requires $2\times d_{\rm in} \times d_{\rm out} \times r$ trainable parameters for each sensitive weight matrix, we set $\sigma_{\rm LoRA} \leftarrow 2\times d_{\rm in} \times d_{\rm out} \times r$ to ensure that the number of trainable parameters introduced by structured tuning is always equal to or lower than the number of sensitive parameters.

In this way, our SPT adaptively incorporates both structured and unstructured tuning granularities to enable higher flexibility and stronger representational power, simultaneously. In Section~\ref{subsec:abl}, we show that structured tuning is important for the downstream tasks with larger domain gaps and both unstructured and structured tuning contribute clearly to the superior performance of our SPT.

%%%%%%%%%%%%%%%%%%%%%%%%%%%%%%%%%%%%%%%%%%%%%%%%%

\begin{table*}[t!]
\centering
\caption{{Main Results on OV-COCO and OV-LVIS:} We evaluate box AP with IoU threshold 0.5 ($\mathrm{mAP^{50}}$) on OV-COCO, and box AP ($\mathrm{mAP^{box}}$) and mask AP ($\mathrm{mAP^{mask}}$) on OV-LVIS. Note that $\mathrm{mAP_{novel}}$ and $\mathrm{mAP}$ indicate the performance of zero-shot and the entire of categories, respectively. Lastly, latency implies inference time per image in seconds for OV-LVIS.}
 \vspace*{-0.25cm}
\label{tab:main}
\resizebox{1.0\linewidth}{!}{%
\begin{tabular}{@{}lrrrrrrrrrrr@{}}
\toprule 
& & & & \multicolumn{3}{c}{OV-COCO}  & \multicolumn{5}{c}{OV-LVIS}  \\
\cmidrule(lr){5-7} \cmidrule(lr){8-12}  
{Methods}& Backbone & CLIP & Res. & $\mathrm{mAP^{50}_{novel}}$ & $\mathrm{mAP^{50}_{base}}$ & $\mathrm{mAP^{50}}$ & $\mathrm{mAP^{box}_{novel}}$ & $\mathrm{mAP^{box}}$ & $\mathrm{mAP^{mask}_{novel}}$ & $\mathrm{mAP^{mask}}$ & Latency\,(s)\\ 
\midrule
\multicolumn{1}{r}{\normalsize {\sffamily DETR-based}} \vspace*{0.1cm} \\
% OWL-ViT & ViT-L/14 & ViT-L/14 & 25.6 & 34.7 & - & - \\ 
OV-DETR~\cite{zang2022open} & RN50 & ViT-B/32 & 1333  & 29.4 & 61.0 & 52.7 & 18.0 & 27.4 & 17.4 & 26.6 & 12.28 \\
\textbf{Prompt-OVD} & ViT-B/16 & ViT-L/14 & 840 & \textbf{30.6} & \textbf{63.5} & \textbf{54.9} & \textbf{29.4} & \textbf{33.0} & \textbf{23.1} & 24.2 & 0.58 \\
\cmidrule(lr){1-12}
% & %(memory update / robust learning) &
\multicolumn{1}{r}{\normalsize {\sffamily RCNN-based}}&  \multicolumn{4}{r}{\normalsize \!\!\!\!\!\!\!\!{\sffamily(Latency Range: 0.40 -- 0.70 seconds)}}\vspace*{0.1cm} \\ 
% ViLD-text & 10.1 & 24.9 & 5.9 & 49.3 \\
Detic~\cite{zhou2022detic}     & RN50 & ViT-B/32 & 1333 & 27.8 & 51.0 & 45.0 & 23.6 & 30.4 & 21.4 & \textbf{26.9} & 0.47 \\
ViLD~\cite{guopen2022vild}      & RN50 & ViT-B/32 & 1333 & 27.6 & 59.6 & 51.3& 16.7 & 27.8 & 16.6 & 25.5 & 0.48  \\
F-VLM~\cite{kuoopen2023fvlm}     & RN50 & RN50 & 1024 & 28.0 & 43.7 & 39.6& 20.3 & 27.8 & 18.6 & 24.2 & 0.50 \\
DetPro~\cite{du2022detpro}    & RN50 & ViT-B/32 & 1333  & - & -& - & 20.8 & 28.4 & 19.8 & 25.9 & 0.67\\
% F-VLM~\cite{kuoopen2023fvlm}    & RN50x4 & RN50x4 & 1024 & - & - & 26.3 & 28.5 & 0.72\\ 
% \cmidrule(lr){1-9} 
\bottomrule
\end{tabular}%
}
\vspace*{-0.1cm}
\end{table*}

\begin{table*}[t!]
% \begin{wraptable*}{r}{3cm}

\parbox{0.3\linewidth}{
\centering
\caption{Latency change by modifying OV-DETR to \algname{}.}
\vspace*{-0.25cm}
\label{tab:inference_study}
\resizebox{1.0\linewidth}{!}{
\begin{tabular}{@{}llr@{}}
\toprule 
& {Modification}& Latency (s)\\ 
\midrule
 & OV-DETR & 12.28\\
% \cmidrule{1-3}
\,\,\,(1)\!\! & ResNet $\xrightarrow{}$ ViT & 12.36\\ 
\,\,\,(2)\!\! & ViT $\xrightarrow{}$ ViTDet & 8.75\\ 
\,\,\,(3)\!\! & Prompt-based Decoding & 2.89\\
\,\,\,(4)\!\! & Ensemble with CLIP & 3.03\\
\,\,\,(5)\!\! & RoI Pruning ($\epsilon=0.3)$ & 0.58\\
\bottomrule
\label{table:modification}
\vspace*{-0.4cm}
\end{tabular}%
}}
{\color{white} \,}
\hfill
\parbox{0.33\linewidth}{
\centering
\caption{Performance with varying $\alpha$ when fixing $\beta=0.4$.}
\vspace*{-0.3cm}
\label{tab:alpha}
\resizebox{1.0\linewidth}{!}{%
\begin{tabular}{@{}lrrrr@{}}
\toprule 
{$\alpha$}& $\mathrm{mAP^{box}_{novel}}$ & $\mathrm{mAP^{box}}$ & $\mathrm{mAP^{mask}_{novel}}$ & $\mathrm{mAP^{mask}}$ \\ 
\midrule
% & %(memory update / robust learning) &
% \multicolumn{1}{r}{\footnotesize {\sffamily RCNN-based}} \\
% ViLD-text & 10.1 & 24.9 & 5.9 & 49.3 \\
0.0 & 28.1 & 30.8 & 21.9 & 22.4 \\
0.1 & 28.7 & 32.3 & 22.6 & 23.6\\
% \rowcolor{LightCyan}
\textbf{0.2} & \textbf{29.4} & \textbf{33.0} & \textbf{23.1} & \textbf{24.2}  \\
0.3 & 29.5 & 32.2 & 23.2 & 23.7 \\
0.4 & 29.7 & 30.6 & 23.3 & 22.5 \\
0.5 & 29.9 & 28.8 & 23.5 & 21.2\\
1.0 & 30.5 & 14.5 & 24.0 & 10.8 \\
\bottomrule
\label{table:alpha_search}
\vspace*{-0.4cm}
\end{tabular}%
}}
{\color{white} \,}
\hfill
\parbox{0.33\linewidth}{
\centering
\caption{Performance with varying $\beta$ when fixing $\alpha=0.2$.}
\vspace*{-0.3cm}
\label{tab:beta}
\resizebox{1.0\linewidth}{!}{
\begin{tabular}{@{}lrrrr@{}}
\toprule 
{$\beta$}& $\mathrm{mAP^{box}_{novel}}$ & $\mathrm{mAP^{box}}$ & $\mathrm{mAP^{mask}_{novel}}$ & $\mathrm{mAP^{mask}}$ \\ 
\midrule
% & %(memory update / robust learning) &
% \multicolumn{1}{r}{\footnotesize {\sffamily RCNN-based}} \\
% ViLD-text & 10.1 & 24.9 & 5.9 & 49.3 \\
0.0 & 15.9 & 30.0 & 12.1 & 21.7\\
0.1 & 22.8 & 31.4 & 17.9 & 22.9 \\
0.2 & 29.0 & 32.7 & 22.5 & 23.9 \\
0.3 & 29.3 & 33.0 & 23.0 & 24.1 \\
\textbf{0.4} & \textbf{29.3} & \textbf{33.0} & \textbf{23.1} & \textbf{24.2} \\
0.5 & 28.6 & 33.0 & 22.4 & 24.1 \\
1.0 & 19.6 & 31.5 & 15.3 & 22.9 \\
\bottomrule
\label{table:beta_search}
\vspace*{-0.4cm}
\end{tabular}%
}
}


\vspace*{-0.35cm}
\end{table*}

\section{Evaluation} 

\noindent\textbf{Datasets.} We evaluate our approach on two popularly used benchmark datasets, namely OV-COCO and OV-LVIS, each of which is modified from MS-COCO~\cite{lin2014microsoft} and LVIS\,(v1)~\cite{gupta2019lvis}; OV-COCO has 121K images with 64 classes while OV-LVIS has 100K images with 1,203 classes.
Following OV-DETR~\cite{zang2022open}, COCO is split into 17 novel classes and 48 base classes. LVIS is split into three categories: 337 novel classes, and 866 common or base classes based on the number of training images. Note that we refer to the two datasets as OV-COCO and OV-LVIS, respectively, and only base classes are used for training. 


\smallskip\smallskip
\noindent\textbf{Algorithms.} We compare \algname{} with an end-to-end OVD detection model named OV-DETR\,\cite{zang2022open} (baseline) and four two-stage OVD models. However, it should be noted that the two-stage OVD models are based on Mask-RCNN or allow the use of external large-scale data, which makes a fair comparison with the end-to-end Transformer-based detectors difficult. Therefore, to ensure a fair comparison, we follow two criteria: (1) the results should be obtained by only using base categories in training, i.e., the restricted OVD setup and (2) the models' inference speed should be in the range of 0.4~--~0.7 seconds/image, which is similar to that of our proposed framework. 

%To validate the effectiveness of our method, we chose OV-DETR and four Mask-RCNN-based methods as baselines. Since Mask-RCNN-based approaches have a completely different model architecture from ours, we chose these methods as baselines based on two criteria for a fair comparison: (i) using only the dataset for base categories during training (not allowing to use the additional dataset), and (ii) having inference times between 0.4 and 0.7 seconds, which are similar to Prompt-OVD.

\smallskip\smallskip
\noindent\textbf{Implementation.} The proposed \algname{} builds upon Deformable DETR\,\cite{zhu2020deformable}, similar to OV-DETR. However, we merge the independent ResNet backbone and Transformer encoder into a single ViT encoder using ViTDet~\cite{li2022exploring}. As a result, our architecture is a purely Transformer encoder-decoder structure, following recent fully Transformer detection pipeline\,\cite{litransformer, songvidt}. 

For training, we initialize the backbone weights with a plain ViT backbone that has been pre-trained as Masked Autoencoders on ImageNet-1K. The entire model is then trained end-to-end for 50 epochs with a batch size of 32, a weight decay of 1e-4, and an AdamW optimizer. We set the initial learning rate to 2e-4 while using an image size of 840$\times$840. We implement and test all algorithms using PyTorch on eight NVIDIA V100 GPUs. 

For inference, there are three hyperparameters: the weights $\alpha$ and $\beta$ for ensembling in Eq.\,\eqref{eq:ensemble}; the threshold $\epsilon$ for RoI pruning in Eq.\,\eqref{eq:pruning}. The former weights are set to be $(0.2, 0.35)$ and $(0.2, 0.4)$ for OV-COCO and OV-LVIS, while the latter pruning threshold is set to 0.125 and 0.3 for OV-COCO and OV-LVIS. As for the ensemble for classification, we leverage the CLIP model that uses ViT-L/14 as its image encoder with a image size of 336$\times$336, which adds very little computational overhead with our RoI-based masked attention and RoI pruning. The detailed analysis of the hyperparameters and additional overhead due to using CLIP is provided in Section \ref{sec:abl_study}. 

In addition, we need to incorporate the mask head into our model for the evaluation on OV-LVIS, as RPN-based two-stage methods have reported both box and mask APs. Following the recent literature\,\cite{dong2021solq, song2022extendable}, we extend our DETR-based detector using SOLQ\,\cite{dong2021solq}, which can perform a joint training of object detection and instance segmentation by simply adding a unified query representation module. The mask vector size is set to be 1,024 while keeping remaining hyperparameters to be the same as SOLQ.

\smallskip\smallskip
\noindent\textbf{Evaluation Metrics.} We evaluate the detection accuracy of our method following exactly the same metrics used in prior OVD studies\,\cite{guopen2022vild, zang2022open, zhong2022regionclip, minderer2022owlvit}. Specifically, for OV-COCO, we use $\mathrm{mAP^{50}}$ which is a measure of the box average precision\,(AP) with an IoU threshold of 0.5. On the other hand, for OV-LVIS, we use both box mAP ($\mathrm{mAP^{box}}$) and mask mAP ($\mathrm{mAP^{mask}}$) obtained by the joint learning of object detection and instance segmentation, respectively. 

Inference time is also a crucial metric for practical applications. To compare the efficiency of different models, we compute the inference time of all methods using the same hardware environment, consisting of a single NVIDIA V100 GPU and six Intel(R) Xeon(R) Gold 5120 CPUs. To ensure an accurate inference time measurement, we calculate the average time of 100 iterations after five initial iterations, using a batch size of 1.

%\noindent\textbf{Implementation.} To enhance the performance of our model, we opt for the DETR~\cite{carion2020end} architecture based on OV-DETR~\cite{zang2022open}, and we replace the backbone and encoder with ViT-DET~\cite{li2022exploring}. The initial weights of the backbone start from the MAE pre-trained model. Our model is trained for 50 epochs, with a 32 batch size, a weight decay of 1e-4, an optimizer AdamW~\cite{loshchilov2017decoupled}, and an initial learning rate of 2e-4, while using an image size of 840x840. It is worth noting that the training process is carried out using 8 NVIDIA A100 GPUs.

%When performing inference, we use the values of ($\alpha$, $\beta) = (0.2, 0.35)$ and (0.2, 0.4) for OV-COCO and OV-LVIS, respectively, to ensemble with CLIP. In addition, we set the values of $\epsilon$ to 0.125 and 0.3 for OV-COCO and OV-LVIS, respectively, for RoI pruning. We limit the number of detections per image to a maximum of 300 and 1500, and the temperature is set to 0.065 and 0.01 for OV-COCO and OV-LVIS, respectively. Finally, we measure the inference time of all methods using the same hardware environment consisting of a NVIDIA V100 GPU and 6 Intel(R) Xeon(R) Gold 5120 CPUs. To ensure accurate inference time measurements, we calculate the average time of 100 iterations after five initial iterations using a batch size of 1.


%\smallskip\smallskip
%\noindent\textbf{Instance Segmentation.} In order to calculate the mask mAP for OV-LVIS, we need to incorporate the mask head into our model. Since the mask head in DETR is a FPN-style network, and our backbone cannot combine with it, we adopt the SOLQ method~\cite{dong2021solq}, which segments objects by jointly learning the unified query representation for three tasks (classification, localization, and segmentation). We utilize the 1024 dimension of the mask vector, while keeping all other parameters the same as SOLQ.

%\smallskip\smallskip
%\noindent\textbf{Evaluation Metrics.} We follow the same metrics as earlier open vocabulary studies. Specifically, for OV-COCO, they use a metric called $\mathrm{mAP^{50}}$ which measures the box AP with an IoU threshold of 0.5. For OV-LVIS, the metrics use both mask mAP ($\mathrm{mAP^{mask}}$) and box mAP ($\mathrm{mAP^{box}}$). 


\subsection{Main Experiment}

We present a comprehensive comparison of \algname{} with other five OVD methods in terms of detection accuracy and speed. To ensure a fair comparison, we only include the results of RCNN-based methods that can operate at a similar inference speed as ours. Table \ref{tab:main} summarizes the results of \algname{} and other five OVD methods.

In general, \algname{} outperforms the previous end-to-end OVD method, OV-DETR, on both datasets. Notably, \algname{}'s inference speed is {$21.2$ times} faster than OV-DETR, thanks to its prompt-based decoding approach. Refer to Section \ref{sec:inference_speed} for an in-depth comparison of efficiency with OV-DETR. %
%
Moreover, Prompt-OVD exhibits superior performance in terms of box mAP, even compared to RCNN-based OVD methods. These results support the effectiveness of our design that utilizes ViT-based CLIP with RoI-based mask attention and RoI pruning, improving the overall performance. Further investigation of the two techniques can be found in Section~\ref{sec:masked_attention} and ~\ref{sec:pruning}.

Although we observe a larger gap between box mAP and mask mAP (29.4 $\mathrm{mAP^{box}_{novel}} \rightarrow 23.1 \mathrm{mAP^{mask}_{novel}})$, this is a result of inheriting the limitation of SOLQ\,\cite{dong2021solq}. Specifically, the vector encoding of 2D segmentation masks using discrete cosine transformation loses object details compared to the conventional FPN-style mask head\,\cite{song2022extendable}.  
%
Furthermore, despite using a larger ViT-B/16 backbone than ResNet-50, \algname{} exhibits comparable inference time to RCNN-based methods, thanks to its simple encoding-decoding pipeline. Therefore, our results demonstrate that \algname{} shows a potential of the end-to-end Transformer-based framework for OVD. 

In Appendices B and C, we discuss potential enhancements to our method and present the results of our experiments on using image queries other than text queries for open-vocabulary object detection, respectively.


%Prompt-OVD performs better than both Mask-RCNN-based and DETR-based models for all metrics on OV-LVIS. Although Prompt-OVD significantly outperforms the other models in terms of box mAP, F-VLM and Detic achieve the highest scores for mask mAP in novel and entire categories, respectively. We believe that SOLQ has limitations in instance segmentation because it only uses 1024 mask vectors that differ from the FPN-style mask head. Consequently, Prompt-OVD has a larger gap between box mAP and mask mAP than models that utilize the FPN-style mask head.

%Also, Table~\ref{tab:main} shows that Prompt-OVD outperforms all other baselines on OV-COCO. It is noteworthy that Prompt-OVD surpasses the others in terms of box $\mathrm{mAP^{50}}$ for both novel and overall categories. We believe that the RoI pruning and score fusion with CLIP may be crucial in improving the overall performance, and we investigate the effects of these techniques in the Section~\ref{sec:abl_study}.


%\smallskip\smallskip
%\noindent\textbf{Inference Time.} Table~\ref{tab:main} also shows the inference time per image on OV-LVIS. Surprisingly, despite using a larger backbone compared to Mask-RCNN-based baselines, there is little difference in inference time. Furthermore, Prompt-OVD reduces the inference time for OV-DETR, which is the method based on DETR, by up to $\mathsf{24x}$.


% \begin{table*}[t!]
% \centering
% \caption{Main Results on OV-LVIS}
% \vspace*{-0.2cm}
% \label{tab:main_lvis}
% \resizebox{1.0\linewidth}{!}{%
% \begin{tabular}{@{}lrrrrrrrr@{}}
% \toprule 
% % & & & \multicolumn{2}{c}{OV-LVIS} & \multicolumn{2}{c}{OV-COCO} \\
% % \cmidrule(lr){4-5} \cmidrule(lr){6-7}  
% {Methods}& Backbone & CLIP & Res. & $\mathrm{mAP^{box}_{novel}}$ & $\mathrm{mAP^{box}}$ & $\mathrm{mAP^{mask}_{novel}}$ & $\mathrm{mAP^{mask}}$ & s/img\\ 
% \midrule
% \multicolumn{1}{r}{\footnotesize {\sffamily DETR-based}} \\
% % OWL-ViT & ViT-L/14 & ViT-L/14 & 25.6 & 34.7 & - & - \\ 
% OV-DETR~\cite{zang2022open} & RN50 & ViT-B/32 & 1024 & 18.0 & 27.4 & 17.4 & 26.6 & 12.28\\
% \textbf{Prompt-OVD} & ViT-B/16 & ViT-L/14 & 840 & \textbf{29.4} & \textbf{33.0} & \textbf{23.1} & 24.2 & 0.58\\
% \cmidrule(lr){1-9}

% % & %(memory update / robust learning) &
% \multicolumn{1}{r}{\footnotesize {\sffamily RCNN-based}} \\
% % ViLD-text & 10.1 & 24.9 & 5.9 & 49.3 \\
% Detic~\cite{zhou2022detic}     & CenterNet2 & ViT-B/32 & 1333 & 23.6 & 30.4 & 21.4 & \textbf{26.9} & 0.47 \\
% ViLD~\cite{guopen2022vild}      & RN50 & ViT-B/32 & 1333 & 16.7 & 27.8 & 16.6 & 25.5 & 0.48 \\
% F-VLM~\cite{kuoopen2023fvlm}     & RN50 & RN50 & 1024 & 20.3 & 27.8 & 18.6 & 24.2 & 0.50 \\

% DetPro~\cite{du2022detpro}    & RN50 & ViT-B/32 & 1333 & 20.8 & 28.4 & 19.8 & 25.9 & 0.67 \\

% % F-VLM~\cite{kuoopen2023fvlm}    & RN50x4 & RN50x4 & 1024 & - & - & 26.3 & 28.5 & 0.72\\ 


% % \cmidrule(lr){1-9} 

% \bottomrule
% \end{tabular}%
% }
% % \vspace*{-0.5em}
% \end{table*}



% \begin{table}[t!]
% \centering
% \caption{Main Results on OV-COCO}
% \vspace*{-0.2cm}
% \label{tab:main_coco}
% \resizebox{0.9\linewidth}{!}{%
% \begin{tabular}{@{}lrrr@{}}
% \toprule 
% % & & & \multicolumn{2}{c}{OV-LVIS} & \multicolumn{2}{c}{OV-COCO} \\
% % \cmidrule(lr){4-5} \cmidrule(lr){6-7}  
% {Methods}& $\mathrm{mAP^{novel}_{50}}$ & $\mathrm{mAP^{base}_{50}}$ & $\mathrm{mAP_{50}}$ \\ 
% \midrule
% % & %(memory update / robust learning) &
% \multicolumn{1}{r}{\footnotesize {\sffamily DETR-based}} \\
% % OWL-ViT & ViT-L/14 & ViT-L/14 & 25.6 & 34.7 & - & - \\ 
% OV-DETR~\cite{zang2022open} & 29.4 & 61.0 & 52.7 \\
% \textbf{Prompt-OVD} & \textbf{30.6}& \textbf{63.5} &\textbf{54.9} \\
% \cmidrule(lr){1-4}

% \multicolumn{1}{r}{\footnotesize {\sffamily RCNN-based}} \\
% % ViLD-text & 10.1 & 24.9 & 5.9 & 49.3 \\
% Detic~\cite{zhou2022detic}      &  27.8 & 51.0 & 45.0 \\
% ViLD~\cite{guopen2022vild}      &  27.6 & 59.6 & 51.3 \\
% F-VLM~\cite{kuoopen2023fvlm}     &  28.0 & 43.7 & 39.6 \\

% \bottomrule
% \end{tabular}%
% }
% \end{table}
\begin{figure*}[t]
\begin{center}
\includegraphics[width=16.7cm]{figures/ablation_study.pdf}
\end{center}
\vspace*{-0.5cm}
\begin{subfigure}{0.3\textwidth}
\label{fig:gt}
\caption{GT}
\end{subfigure}
\begin{subfigure}{0.15\textwidth}
\label{fig:rpn}
\caption{RPN}
\end{subfigure}
\begin{subfigure}{0.31\textwidth}
\caption{No pruning ($\epsilon = 0$)}
\label{fig:noprune}
\end{subfigure}
\begin{subfigure}{0.16\textwidth}
\caption{Pruning ($\epsilon =0.3$)}
\label{fig:prune}
\end{subfigure}
\vspace*{-0.4em}
\caption{Box predictions: (a) ground-truth boxes, (b) boxes estimated by RPN from DetPro~\cite{du2022detpro}, (c)--(d) boxes estimated by \algname{} without and with RoI pruning. Due to numerous predicted boxes in (b) and (c), we limit the number of boxes to 40 for better visualization. Red and green boxes are the ground-truth of novel and base classes, while blue and white ones represent predicted boxes that are either in close proximity or not in close proximity to the ground-truth, respectively.}
\label{fig:pruning_abl}
\vspace*{-0.4cm}
\end{figure*}


\subsection{Main Ablation Study}
\label{sec:abl_study}

\subsubsection{Inference Speed Up from OV-DETR} 
\label{sec:inference_speed} 
Table \ref{table:modification} summarizes the change in inference speed when replacing each design component of OV-DETR with our proposed ones on OV-LVIS. (1) Despite having more parameters, using ViT-B/16 incurs little additional latency, as it rather reduces the computational burden for the multi-scale deformable attention of the DETR encoder. (2) The use of local attention and the replacement of the DETR encoder with a simple feature pyramid network, as suggested by \cite{li2022exploring, song2022extendable}, result in a meaningful reduction in latency. (3) The primary speedup comes from replacing the decoding of OV-DETR with prompt-based decoding. (4) The ensemble with ViT-based CLIP only adds very little latency thanks to our efficient RoI-based masked attention. (6) RoI pruning also significantly contributes to speeding up by reducing the number of RoI candidates for detection and segmentation, without sacrificing detection accuracy. Overall, \algname{} speeds up inference by $21.2$ times over OV-DETR.

\begin{table}[t!]
\centering

\caption{Performance between RoI Align and RoI-based Masked Attention (RMA) on OV-LVIS.}% We set $\epsilon$ as 0.3 while pruning.} % If RoI pruning is employed, $\epsilon$ is assigned a value of 0.3; otherwise, it should be set to 0.0.
\vspace*{-0.3cm}
\label{tab:roi_pool}
\resizebox{1.0\linewidth}{!}{%
\begin{tabular}{@{}lrrrrrr@{}}
\toprule 
{RoI Proc.}& $\mathrm{mAP^{box}_{novel}}$ & $\mathrm{mAP^{box}}$ & $\mathrm{mAP^{mask}_{novel}}$ & $\mathrm{mAP^{mask}}$ & Latency\\ 
\midrule
% & %(memory update / robust learning) &
% \multicolumn{1}{r}{\footnotesize {\sffamily RCNN-based}} \\
% ViLD-text & 10.1 & 24.9 & 5.9 & 49.3 \\
Naive & 12.8 & 28.3 & 9.9 & 20.3 & 13.51 \\
\hspace{3mm}+Pruning & 13.7 & 28.3 & 10.2 & 20.4 & 1.85\\
Align~\cite{he2017mask} & 24.2 & 31.7 & 19.1 & 23.0 & 3.06\\
\hspace{3mm}+Pruning & 26.2 & 32.0 & 20.8 & 23.4 & 0.60 \\
RMA (ours) & 26.6 & 32.5 & 21.0 & 23.8 & 3.03 \\
\hspace{3mm}+Pruning & \textbf{29.4} & \textbf{33.0} & \textbf{23.1} & \textbf{24.2} & 0.58 \\


\bottomrule
\end{tabular}%
}
\vspace*{-0.3cm}
\label{tab:rma_analysis}
\end{table}

%\vspace*{-0.18cm}
\subsubsection{Ensemble Coefficient}
%\label{sec:hyperparams}
We investigate the influence of the ensembling weights $\alpha$ and $\beta$ for base and novel classes. We vary the value of each hyperparameter while keeping the other constant, as summarized in Tables \ref{tab:alpha} and \ref{tab:beta}, with a fixed RoI pruning threshold $\epsilon$ of 0.3. In general, the overall performance increases and then reaches at their maximum values when $\alpha=0.2$ and $\beta=0.4$. Using extreme values 0.0 or 1.0 for either coefficient results in significantly worse performance than using a more balanced ensemble. Moreover, the results show that the ensemble is more effective for novel classes than base classes, as evidenced by the significant impact of $\beta$ on the results of $\mathrm{mAP_{novel}}$. This suggests that the knowledge from CLIP has a greater positive impact on novel classes than on base classes. We set the values of $\alpha$ and $\beta$ to 0.2 and 0.4 for all experiments.
%

%Also, for getting optimal probability weights $\alpha$ and $\beta$, we report the performance change as $\alpha$ and $\beta$ increase in Table~\ref{tab:alpha} and ~\ref{tab:beta}, respectively. Table~\ref{tab:alpha} illustrates that while the zero-shot performance ($\mathrm{mAP_{novel}}$) increases monotonically, the overall performance ($\mathrm{mAP}$) increases and peaks at $\alpha=0.2$, then decreases. Interestingly, Table~\ref{tab:beta} demonstrates that both zero-shot and overall performances increase and reach a maximum value when $\beta=0.4$. Therefore, we anticipate that the knowledge from CLIP have a greater positive impact on new classes than the base classes, as the optimal value of $\beta$ is larger than that of $\alpha$.
%Note that we set the values of $\alpha$ and $\beta$ to 0.2 and 0.4, respectively, for all the experiments.



\subsubsection{RoI-based Masked Attention} 
\label{sec:masked_attention}
We conduct a comparison between our RoI-based masked attention method with both the naive approach and the commonly used RoI Align method, as summarized in Table \ref{tab:rma_analysis}. The naive approach implies that CLIP infers all the cropped images of RoIs according to Eq.~\eqref{eq:iter_infer}. To apply the RoI Align method to CLIP's ViT encoder, we reconstruct its patch tokens into a 2D feature map prior to the final Transformer layer. Compared to the naive approach, our RoI-based masked attention method has a significantly smaller computational overhead. Additionally, the efficiency and effectiveness of our method can be further improved by utilizing RoI pruning, which removes background RoIs. In contrast, RoI Align shows substantially lower $\mathrm{mAP^{box}_{novel}}$ and $\mathrm{mAP^{box}}$ than our method, as it is not optimized for the Transformer structure. Surprisingly, the naive crop method did not perform well, likely due to resizing a small object to be too large. Therefore, using masked attention is a more appropriate approach for Transformers than others. 

%To validate our proposed RoI-based masked attention, we compare it with RoI Align, which is utilized by Mask-RCNN in Table~\ref{tab:roi_pool}. Masked attention overcomes RoI Align for all the metrics with a large margin. Moreover, we analyze the optimal usage of the attention layer number in Table~\ref{tab:blk_num}, and using the last layer has the best results compared to the others for all the metrics. Through those emperical studies, therefore, we select RoI-based masked attention with the last attention layer instead of RoI align. 
\label{sec:eval_RMA}
\begin{table}[t!]
\centering
\caption{Performance trade-off with varying $\epsilon$ on OV-LVIS.}
\vspace*{-0.2cm}
\label{tab:roi_thres}
\resizebox{1.0\linewidth}{!}{%
\begin{tabular}{@{}lrrrrr@{}}
\toprule 
{$\epsilon$}& $\mathrm{mAP^{box}_{novel}}$ & $\mathrm{mAP^{box}}$ & $\mathrm{mAP^{mask}_{novel}}$ & $\mathrm{mAP^{mask}}$ & Latency (s)\\ 
\midrule
% & %(memory update / robust learning) &
% \multicolumn{1}{r}{\footnotesize {\sffamily RCNN-based}} \\
% ViLD-text & 10.1 & 24.9 & 5.9 & 49.3 \\
0.0 & 26.6 & 32.5 & 21.0 & 23.8 & 3.03\\
0.1 & 27.7 & 32.8 & 21.9 & 24.0 & 1.38\\
0.2 & 28.3 & 33.0 & 22.3 & 24.1 & 0.86\\
% \rowcolor{LightCyan}
\textbf{0.3} & \textbf{29.4} & \textbf{33.0} & \textbf{23.1} & \textbf{24.2} & 0.58\\
0.4 & 28.3 & 31.9 & 22.5 & 23.4 & 0.43\\
0.5 & 25.3 & 28.1 & 19.8 & 20.8 & 0.37\\
\bottomrule
\end{tabular}%
}
\vspace*{-0.2cm}
\end{table}



\subsubsection{Box Regression over RPN}
%\label{sec:eval_prune}
% tab2에서 masked attention과 align과의 비교. 
% block number 마지막으로 선택한 이유. 
%\vspace*{-0.1cm}
We validate the effectiveness of \algname{} in terms of box regression compared with the existing RPN-based method. Figure \ref{fig:pruning_abl} compares their estimated bounding boxes based on the ground-truth ones. %Specifically, Figure \ref{fig:pruning_abl}(a) shows the base and novel objects with their ground-truth bounding boxes. 
Figure \ref{fig:pruning_abl}(b) is an example of the scenario where the RPN method fails to accurately localize objects, i.e., a  missing box for carriage (novel class) and two deviated boxes from the ground-truth for the bread (novel class) and plate (base class). In contrast, \algname{} successfully localizes all base and novel objects with high recall using the prompt-guided decoding, as shown in Figure \ref{fig:pruning_abl}(c). Despite the presence of background or inaccurate boxes in the box candidates, \algname{} successfully covers all the ground-truth boxes with its predictions. Furthermore, as seen in Figure \ref{fig:pruning_abl}(d), RoI pruning effectively excludes such irrelevant boxes from the detection process.


%In order to validate the effectiveness of RoI pruning, we compare the proposals before and after pruning in Fig.~\ref{fig:pruning_abl}. Prior to pruning, there are numerous false positive boxes that did not match the ground truth, and those might negatively affect the performance due to score fusion with CLIP. However, pruning significantly reduces the number of non-matching boxes. In addition, since object detection consumes their time for handling lots of proposals, RoI pruning improves both performance and inference speed. We believe that RoI pruning narrow the gap in inference time between our method and the baseline despite using a larger backbone. 
%\vspace*{-0.3cm}
\subsubsection{RoI Pruning Threshold}
\label{sec:pruning}
%\vspace*{-0.1cm}
We investigate the trade-off between detection performance and computational efficiency by varying the pruning threshold $\epsilon$, as summarized in Table \ref{tab:roi_thres}. As the threshold increases, fewer bounding boxes are retained, as the number of boxes with object scores greater than the threshold decreases. For instance, when the threshold is set to be 0.0, which means RoI pruning is not applied, the detection accuracy deteriorates due to the inclusion of background boxes, also resulting in a high latency. On the contrary, the detection accuracy improves as $\epsilon$ increases within the reasonable range of 0.0~--~0.3, but deteriorates with a larger threshold of 0.4~--~0.5. That is, as $\epsilon$ increases, more false positive boxes begins to be excluded, leading to improved performance. However, further increasing the threshold leads to the removal of true positive boxes, causing performance degradation. Therefore, we set the value of $\epsilon$ to 0.3 for all experiments.

%, we analyze the trend of performance and inference time as $\epsilon$ increases and report the results in Table~\ref{tab:roi_thres}. It shows that latency decreases as $\epsilon$ increases, since the number of boxes of which score is bigger than $\epsilon$ decreases. Especially, the inference time is drastically higher when not using RoI prunning ($\epsilon=0.0$). On the contrary, the performance increases as $\epsilon \in [0.0, 0.3]$, and decreases as $\epsilon \in [0.3, 0.5]$. We expect that, as $\epsilon$ increases, it drops more false positive boxes to improve the performance at $\epsilon \in [0.0, 0.3]$, then it causes degradation of performance by dropping true positive boxes at $\epsilon \in [0.3, 0.5]$. Hence, we select $\epsilon$ as 0.3, which has the best performance with valid inference time. 


\begin{table}[t!]
\centering
\caption{Performance when using different CLIP models for ensembling on OV-LVIS.}
\vspace*{-0.25cm}
\label{tab:clip_arch}
\resizebox{1.0\linewidth}{!}{%
\begin{tabular}{@{}lrrrrr@{}}
\toprule 
{CLIP}& $\mathrm{mAP^{box}_{novel}}$ & $\mathrm{mAP^{box}}$ & $\mathrm{mAP^{mask}_{novel}}$ & $\mathrm{mAP^{mask}}$ & Latency (s)\\ 
\midrule
% & %(memory update / robust learning) &
% \multicolumn{1}{r}{\footnotesize {\sffamily RCNN-based}} \\
% ViLD-text & 10.1 & 24.9 & 5.9 & 49.3 \\
None     & 15.4 & 28.1 & 11.7 & 20.3 & 0.54 \\
ViT-B/32 & 20.5 & 30.1 & 16.4 & 21.9 & 0.56 \\
ViT-B/16 & 22.3 & 31.1 & 17.5 & 22.6 & 0.57\\
\textbf{ViT-L/14} & \textbf{29.4} & \textbf{33.0} & \textbf{23.1} & \textbf{24.2} & 0.58 \\


\bottomrule
\end{tabular}%
}
\label{tab:clip_size}
\vspace{-1.2em}
\end{table}


\smallskip\smallskip
\subsection{Additional Design Choice}
%\vspace*{-0.1cm}
\noindent We explore two supplementary design choices to utilize the ViT-based CLIP model in a manner that achieves the best balance between OVD performance and inference speed. %Following this, we further examine the impact of utilizing different initial weights for the backbone. %on the overall performance of \algname{}.
We provide more supplementary analysis on applying RoI-based masked attention to different attention layers and using different pre-trained ViT backbones with \algname{} in Appendix D.

%\vspace*{-0.3cm}
\subsubsection{CLIP Model Size} 
%\vspace*{-0.1cm}
% We conduct experiments to examine the impact of classification ensemble using CLIP with respect to its model size. 
Table \ref{tab:clip_size} summarizes the performance obtained after the ensemble with three different sizes of ViT-based CLIP models, including the scenario where CLIP is not used at all. We observe that without using the ensemble technique, the zero-shot performance is very poor compared to when CLIP is employed. However, as the size of the CLIP model increases, both zero-shot and overall performance improve, albeit with slightly higher inference speed. The difference in inference time is negligible due to our proposed efficient techniques; RoI-based masked attention and RoI pruning. Therefore, we conclude that the benefits obtained from using a larger CLIP encoder with ViT-L/14 outweigh the minimal increase in inference time.

%Since we expect that the zero-shot results ($\mathrm{mAP_{novel}}$) could be influenced by the utilization of the CLIP model for ensembling, we report the performance on OV-LVIS in Table~\ref{tab:clip_arch} as the CLIP model is varied. Specifically, the zero-shot performance is noticeably worse when not utilizing CLIP for ensembling compared to when CLIP is used. As the size of the CLIP model increases, both zero-shot and overall performance increase, and the inference time also increases slightly. However, the difference in inference time is negligible in comparison to the performance benefits gained. As such, we can conclude that boxes from DETR architecture contains not only base classes but also novel classes, and CLIP can more contribute the classification for unseen classes as the size increases without increasing latency much.

%\vspace*{-0.3cm}
\subsubsection{CLIP Input Resolution} 
%$\vspace*{-0.1cm}
Another design consideration is the resolution of input to the ViT-based CLIP. To evaluate the performance trade-off between detection accuracy and latency of CLIP, we vary the input image size and report the results in Table \ref{tab:image_size}. Similar to the model size, the latency change by input resolution is negligible thanks to our efficient methodological design. This indicates that we can use a variety of image resolutions without sacrificing the latency of \algname{}. However, the best image resolution for the ViT-L/14 encoder is $336 \times 336$, which is the original input size used to train the CLIP model, while the detection accuracy for base and novel classes drops with a larger $672\times672$. The $336 \times 336$  provides a good balance between detection accuracy and inference time, so it is the recommended input resolution. %for the proposed \algname{} framework. 

%Here, we evaluate the performance trade-off between detection accuracy and latency with of CLIP by varying the input image size, and report the results in Table~\ref{tab:clip_img_res}. Although there is a significant difference in performance, the latency is negligible when using image resolutions of 168 and 336. However, when compared to image resolutions of 336 and 672, the performance difference is minimal, but there is a significant gap in inference time. As a result, we choose an image resolution of 336 when feeding images into the CLIP model. 



%e analyze the performance is affected by using different initial weights for the backbone, specifically ImageNet and MAE. Table~\ref{tab:pretrained_model} displays the performance results when training the backbone using these two pretrained models. As a result, starting from the MAE pretrained model yields better performance than starting from the ImageNet pretrained model. Therefore, we use the trained model that starts from the MAE pretrained model for all experiments.







\begin{table}[t!]
\centering
\caption{Performance when using different input image resolution for ViT-based CLIP on OV-LVIS.}
\vspace*{-0.25cm}
\label{tab:clip_img_res}
\resizebox{1.0\linewidth}{!}{%
\begin{tabular}{@{}lrrrrr@{}}
\toprule 
{img. res.}& $\mathrm{mAP^{box}_{novel}}$ & $\mathrm{mAP^{box}}$ & $\mathrm{mAP^{mask}_{novel}}$ & $\mathrm{mAP^{mask}}$ & Latency \\ 
\midrule
% & %(memory update / robust learning) &
% \multicolumn{1}{r}{\footnotesize {\sffamily RCNN-based}} \\
% ViLD-text & 10.1 & 24.9 & 5.9 & 49.3 \\

168$\times$168 & 28.5 & 30.1 & 20.9 & 22.0 & 0.57 \\ 
\textbf{336$\times$336} & \textbf{29.4} & \textbf{33.0} & \textbf{23.1} & \textbf{24.2} & 0.58 \\ 
672$\times$672 & 29.2 & 33.0 & 23.0 & 24.2 & 0.83 \\


\bottomrule
\end{tabular}%
}
\vspace*{-0.35cm}
\label{tab:image_size}
\end{table}


% \begin{table}[t!]
% \centering
% \caption{Latency change as modifying OV-DETR to Prompt-OVD on OV-LVIS.}
% \label{tab:inference_study}
% \resizebox{0.7\linewidth}{!}{%
% \begin{tabular}{@{}llr@{}}
% \toprule 
% & {Modification}& Latency (s)\\ 
% \midrule
%  & OV-DETR & 12.28\\
% % \cmidrule{1-3}
% (1) & ResNet $\xrightarrow{}$ ViT & \\ 
% (2) & ViT $\xrightarrow{}$ ViTDET & \\ 
% (3) & Encoder $\xrightarrow{}$ FPN & \\
% (4) & prompt decode & \\
% (5) & + Ensemble with CLIP & \\


% \bottomrule
% \end{tabular}%
% }

% \end{table}



%%%%%%%%%%%%%%%%%%%%%%%%%%%%%%%%%%%%%%%%%%%%%%%%%

\section{Conclusion} 
We introduce \algname{}, a novel object detection method that achieves highly efficient inference speed while also improving zero-shot generalization compared with existing methods. The prompt-based decoding approach reduces the computational burden of object queries. The RoI-based masked attention and RoI pruning techniques allow us to efficiently leverage a large ViT-based CLIP model, enhancing detection performance through classification prediction ensembling. Comprehensive experiments show that \algname{} is $21.2$ times faster than OV-DETR while achieving comparable or higher APs on base and novel classes compared to two-stage OVD methods. %We believe that our work will inspire future work to explore the benefits of using Transformers.


%\paragraph{Ethics Statement.} 
%The focus of this paper is on open-vocabulary object detection. Our approach involves the integration of Transformer-based object detector and CLIP. We have not identified any foreseeable negative social impact associated with our work to share our findings with the scientific community. Nonetheless, we will continue to monitor and consider any potential concerns that may arise. 



%%%%%%%%% REFERENCES
{\small
\bibliographystyle{ieee_fullname}
\bibliography{main}
}

\newpage
\appendix
\newpage
\onecolumn
\appendix

\section{Experimental details: Simulated Quadrupeds}

\subsection{Data generation}

\paragraph{Simulation details.} We record a total of 5182 trajectories. 2756 were generated for robots of type ANYmal B and 2426 sequences were generated for robots of type ANYmal C. These are quadruped robots, which means that they have four legs. Each leg has 3 degrees of freedoms - hip, shank and thigh. The position and velocities of these degrees of freedom for all 4 legs were recorded. This results in 24 features  for each robot.
Robots are generated while traversing an procedurally generated environment with different terrain types and traversal difficulty, as show in Figure~\ref{fig:robsim}. We only keep trajectories that correspond to a successful traversal.  

\begin{figure*}[h!]
\centering
   \includegraphics[width=0.95\textwidth]{figures/robots_sim.png}
   \caption{\footnotesize{\em Procedurally generated environments}. (A) Screenshots from the simulator showing a robot walking down some stairs, and a view of the terrain lanscape. (B) Visualization of the different terrain sections, characterized by a terrain type and different levels of difficulty. Terrain are made more difficult to traverse by either making them more rough or have steeper slopes.}  \label{fig:robsim}
   \vspace{-3mm}
\end{figure*}


\vspace{-2mm}
\paragraph{Tasks.}
To evaluate the representation quality of our model, we use multi-task probes that correspond to different long-term and short-term behavioral factors.
\begin{itemize}
\setlength\itemsep{0.2em}
    \item Robot type: the robot can either be of type "ANYmal B" or "ANYmal C". These robots have the same degrees of freedom and tracked joints but differ by their morphology. This is a sequence-level task.
    \item Linear velocity: the command of the robot is a constant velocity vector. The amplitude of the velocity dictates how fast the robot is commanded to traverse the environment. A higher velocity would translate into more clumpsy and more risk-taking behavior. This is a sequence-level task.
    \item Terrain type: the environment is generated with multiple segments of five terrain types that are categorized as: flat surfaces, pits, hills, ascending and descending stairs. This is a frame-level task.
    \item Terrain slope: the slope of the surface the robot is walking on. This is a frame-level task.
    \item Terrain difficulty: the different terrain segments have different difficulty levels based on terrain roughness or steepness of the surface. This is a frame-level task.
\end{itemize}

\paragraph{Why this dataset.}
Simulation-based data collection enables access to information that is generally inaccessible or hard to acquire in a real-world setting. Unlike noisy measurements coming from the camera-based feature extractor in the case of the mouse dataset, physics engines do not suffer from the problem of noise. Instead, they provide accurate ground-truth information about the creature and the world state free of charge. Access to such information is at times critical for scrutinizing the capabilities of the learning algorithms.



\subsection{Visualizing differences between short-term and long-term embeddings}

In Figure~\ref{fig:robosmooth}, we visualize how the short-term and long-term embeddings evolve over time, for a single sample sequence. We note a clear difference in the smoothness in the two timescales. In the short-term embeddings, we note a clear block structure corresponding to different blocks of behavior that span a few seconds, while in the long-term embeddings the representation is more stable over time. This suggests that, as expected, the bootstrapping objectives are forming representations with different levels of granularity. 

\begin{figure*}[h!]
\centering
   \includegraphics[width=0.95\textwidth]{figures/smooth_robo.png}
   \caption{\footnotesize{\em Visualization of the short-term and long-term embeddings.} We visualize for a single sequence how the short-term and long-term embeddings evolve over time.}  \label{fig:robosmooth}
   \vspace{-3mm}
\end{figure*}


\section{Experimental details: Mouse Triplet}

\subsection{Feature extraction}
Each mouse in the arena is tracked using 12 anatomically defined keypoints. We process these keypoints to extract 36 different features characterizing each mouse individually, similar to \cite{segalin2021mouse}. We separate the keypoints into two different areas, the head and the body, for each we extract different measures of displacement, that we express in the frame of the mouse, i.e. these features are invariant to the pose of the mouse relative to the arena. These features include:
\begin{itemize}
    \setlength\itemsep{0.1em}
    \item Head linear velocity vector that we express using polar coordinates. 
    \item Head angular velocity denoting the change in the heading direction in the arena.
    \item Body linear velocity vector that we express using polar coordinates.
    \item Body angular velocity denoting the change in the direction of the body with respect to the arena.
    \item Angular and linear velocities of the fore paws and the hind paws.
    \item Spine length change, depicting the expansion and contraction of the mouse's body. 
    \item Angles formed by the tail with respect to the body.
\end{itemize}

We normalize all features before training. We also use cosine and sinus of the angles instead of the angles. During training, we did not use any form of augmentation. 

\vspace{-2mm}
\paragraph{Noise in the data.}
Because of errors in pose estimation and tracking, there are sometimes errors in the tracking data, notably some identity swap issues \cite{sun2022mabe22}. To address this, we simply zero out all of the corresponding features and flag the frame as invalid. A binary feature is also add to the input features indicating whether or not the frame is valid. When predicting future actions, we only compute the error over windows in which at least 80\% of the frames are valid.


\subsection{Difference between Histogram of Actions and previous objectives}

Our novel objective consists in predicting the future histogram of actions instead of predicting the future sequence of actions. In Figure~\ref{fig:hoa_example}, we show what the target is for a sample from the MABe Mouse Triplet dataset. Note that the time dimension is collapsed, blurring the exact unrolling of the future events, but preserving the set of values that these actions will sweep. Note that the loss (EMD) is applied for each action feature.

\begin{figure}[!h]
    \centering
    \includegraphics[width=0.7\textwidth]{figures/bams_vs_previous_work.pdf}
    \caption{Prediction target for a sample of the MABe Mouse Triplet dataset.}
    \label{fig:hoa_example}
\end{figure}

\subsection{Training details}

\paragraph{Architecture.}
We use two TCNs \cite{bai2018tcn}. Each TCN is built using multiple residual blocks, each residual block is composed of two convolutional layers, and use PReLU activation, dropout and weight normalization. All convolutions are dilated with a rate $r$, that increases after each residual block. The formula is $r^i$ where $i$ is the index of the residual block.
The first TCN is the short-term encoder, which uses 4 blocks with output sizes $[64, 64, 32, 32]$ and a dilation rate $r=2$. The second TCN is the long-term encoder, which uses 5 blocks with output size $[64, 64, 64, 32, 32]$ and a dilation rate $r=4$. The output of both encoders are concatenated to form a $64$d embedding. The predictor is a multi-layer perceptron (MLP) that has 4 hidden layers. 

\vspace{-2mm}
\paragraph{Training.}
We train the model for 500 epochs using the Adam optimizer with a learning rate of $10^{-3}$ and weight decay $4 \cdot 10^{-5}$, we decrease the learning rate to $10^{-4}$ after 100 epochs. We use a batch size of 96, and compute the future histogram of action prediction error for each timestep $t$ starting at 5 seconds after the start of each sequence, in order to allow the model to aggregate enough context. We set the learning rate of the predictors used for bootstrapping to be $10$ times higher than the learning rate used for the rest of the weights.

\vspace{-2mm}
\paragraph{Evaluation.}
During the development of the model, we test our model on the public test splits, and only look at the performance on the private set after finishing any hyperparameter tuning. We repeat the training and evaluation 5 times and report the average performance.

\begin{figure}[!h]
    \centering
    \includegraphics[width=0.85\textwidth]{figures/bams_linear_evaluation.pdf}
    \caption{Linear evaluation protocol. The model is frozen, and for each task, a single linear layer is trained to predict the corresponding labels.}
    \label{fig:lineareval}
\end{figure}


\subsection{BAMS in the inductive setting} \label{app:inductive}
The mouse triplet dataset (5336 sequences) has three different sets, a training set (1800 sequences), a private test set and a public test set. During training of the representation learning model, we can either pre-train on all of the available data (transductive setting) or on the training set only (inductive setting). During linear evaluation, the different linear layers are trained using labels from the training set and the performance is reported on the public test set (during the challenge) and then on the private test set (to rank models). 

We train BAMS in the inductive setting and report the performance in Table~\ref{tab:mouseaccsupp}. We find that even when BAMS is trained with approximately one third of the data, the drop in performance is modest. More importantly, BAMS preserves its ranking compared to other methods, and still is the state-of-the-art.

\input{tables/mabe_tables_supp.tex}

\subsection{Notes on TS2Vec experiments}

TS2Vec \cite{yue2022ts2vec} employs two types of contrastive losses to learn representations. The first of these losses is an instance contrastive loss which contrasts a sequence with all other sequences in a batch which are treated as negative examples, while two subsequences extracted from the same sequence  are treated as positive examples. The second loss is a temporal contrastive loss which acts along a single time series.  Temporal representations of nearby time points are taken as positive examples,  while the rest of the time points within the same sequence are taken as negative examples. 
The results for the three versions of TS2vec, namely TS2Vec-I, which uses only instance contrastive loss, TS2Vec-T, which uses only temporal contrastive loss,and TS2Vec IT, which uses both instance and temporal contrastive losses, are listed in table \ref{tab:TS2vecmouseacc}.
Our TS2Vec experiments on the mouse dataset showed that using temporal contrastive loss resulted in worse performance across all tasks as compared to only using instance contrastive loss. For this reason, we only report results for TS2vec that only employs instance contrastive loss.

We note that for both TS2Vec and TS2Vec-IT, we ran into out-of-memory errors when creating instance-level or global contrast. Contrastive learning methods usually incur high computational costs, we find that our method, which doesn't rely on negative examples, can scale better when working with longer sequences and larger datasets. 

\begin{table*}[h!]
\centering
\caption{{\em TS2Vec Linear readouts of mouse behavior.}}
\vspace{0.1in}
\resizebox{0.99\textwidth}{!}{
\vspace{-2mm}
\begin{tabular}{l|cccc|ccccccccc}
\hline
 & \multicolumn{4}{c|}{\textit{Sequence-level subtasks}} & \multicolumn{9}{c}{\textit{Frame-level subtasks}}\\
%Model & F1-score & MSE & T1\tnote{*} & T2\tnote{*} & T3 & T13 & T4 & T5 & T6  & T7  & T8 & T9  & T10  & T11 & T12 \\   
\textit{Model} & Day ($\downarrow$) & Time ($\downarrow$) & Strain & Lights & Approach & Chase & Close  & Contact  & Huddle & O/E & O/G  & O/O & Watching \\   
\hline
TS2Vec-I & 0.09380 & 0.09422 & 57.12 & 65.60 & 1.29  & 0.66 &  59.53 & 46.13 & 24.74 & 0.35 &  1.09 &  0.74 & 12.37\\
TS2Vec-T & 0.09882 & 1.0252 & 45.82 & 46.69 & 0.72 & 0.14 & 45.19 &  34.93 & 9.38 & 0.186 & 0.38 & 0.38 & 05.31\\
TS2Vec-IT & 0.09846 & 1.01646 & 46.67 & 44.28 & 0.67 & 0.13 & 44.56 &  33.87 & 9.79 & 0.178 & 0.42 & 0.42 & 04.58\\
 \hline
\end{tabular}
}
\label{tab:TS2vecmouseacc}
\end{table*}




\end{document}
