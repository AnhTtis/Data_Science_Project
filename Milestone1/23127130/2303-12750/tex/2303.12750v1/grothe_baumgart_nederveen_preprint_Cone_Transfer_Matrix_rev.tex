\documentclass[a4paper, 10pt]{article}
\usepackage[utf8]{inputenc} 
\usepackage{amsmath}
\usepackage{graphicx}
\usepackage[small,bf]{caption2}  
\usepackage{float}
\usepackage{hyperref}

\begin{document}

\date{March 19, 2023} 
\title{A Transfer Matrix for the Input Impedance of weakly tapered Cones as of Wind Instruments}
\author{
Timo Grothe$^1$, Johannes Baumgart$^2$, Cornelis J.Nederveen$^3$\\
$^1$ \emph{\small Erich-Thienhaus Institut, Hochschule f\"ur Musik Detmold, 32756 Detmold}
\\
$^2$ \emph{\small Independent Researcher, Dresden, Germany}  \\
$^3$ \emph{\small Independent Researcher, Pijnacker, The Netherlands}  
}
\maketitle



\begin{abstract}

A formula for the local acoustical admittance in a conical waveguide with viscous and thermal losses given by Nederveen in \emph{Acoustical Aspects of Woodwind Instruments} (1969) is rewritten as an impedance transmission matrix.
Based on a self-consistent approximation for the cone, it differs from other one-dimensional transmission matrices used in musical acoustics, which implicitly include the loss model of a cylinder.
The resonance frequencies of air columns calculated with this new transmission matrix are in better agreement with more comprehensive models. 
Even for long cones with a slight taper, there is no need to discretize along the axis.

\end{abstract}

\maketitle

\section{\label{sec:1} Introduction}
The first mention of a transfer matrix to describe sound propagation in a conical waveguide has been given by Benade~\cite{Benade1988}. 
For the derivation in his original publication, he argued with network analogy. 
However, the derivation by Fletcher~\cite{Fletcher1988} from the Webster Horn equation leads to the same expression, which is an analytical solution for the input impedance of a conical waveguide without losses~\cite{Chaigne2016}.\\
To include the effect of wall losses in this impedance description, it is a common practice to locally apply the thermo-viscous theory of wave propagation in a cylindrical duct. 
This leads to an integral mean value of the propagation constant~\cite{Walstijn2002, Kulik2007} or, equivalently, to an effective radius of the cone accounting for visco-thermal losses at the wall in a single value.
The accuracy required in musical acoustics has recently inspired some studies which reconsider this approach~\cite{Grothe2013, Chabassier2019, Ernoult2020, Thibault2023}.\\
In this letter we recall one of the first approaches to treat visco-thermal losses in a cone by Nederveen~\cite{Nederveen1969}.
It is rewritten here in today's common nomenclature, brought to the form of a transfer matrix and applied to long slim cones with typical dimensions of woodwind air columns. 
Results of other transfer matrices and an up-to-date finite-element model are compared. 
\section{Equations}
Nederveen's description %(Eq.(23.30)) 
of the local admittance in a conical waveguide with viscous and thermal losses reads~\cite{Nederveen1969}
\begin{equation}
Y(\omega,x)	= \frac{1}{\mathrm{j}\,Z_{c}}\left[\frac{1-\alpha_g}{k\,x}-\frac{1+\alpha_f}{\tan{(k\,x(1+\alpha\,\xi) + \Psi)}}\right]
\label{eq:Y_loss}
\end{equation}
where $\omega$ is the angular frequency, j $= \sqrt{-1}$, $x$ is the axial distance to the cone apex, $k=\omega/c$ is the wavenumber, and $Z_c$ is the characteristic impedance of the fluid. $\Psi$ and $\xi$ are variables related to the phase of the propagating wave components. The visco-thermal losses in the fluid at the wall are included in the form of three dimensionless, complex-valued parameters related to the apical distance $x$ as
\begin{subequations}
\begin{eqnarray}
\alpha(\omega,x) &=&C_v \left(\frac{\gamma-1}{\sqrt{Pr}}+1\right)\label{eq:alpha}
\\
\alpha_f(\omega,x) &=&C_v\left(2-2\,f\,k\,x \left(\frac{\gamma-1}{\sqrt{Pr}}+1\right)\right)\label{eq:af_cone}
\\
\alpha_g(\omega,x) &=&C_v\left(1+2\,g\,(k\,x)^2 \left(\frac{\gamma-1}{\sqrt{Pr}}+1\right)\right)\label{eq:ag_cone}
\\
\textrm{with}& & C_v = \frac{1}{2}\sqrt{2}\frac{\delta_v}{r(x)}(1-\mathrm{j})\, \notag
\end{eqnarray}
\end{subequations}
where $r(x)$ is the local radius, $\delta_v = \sqrt{\eta/(\omega\rho)}$ is the viscous boundary layer thickness, $\eta$ is the dynamic viscosity, $\rho$ is the density, $\gamma$ is the ratio of specific heats, and $Pr$ is the Prandtl number. 
The functions $f$ and $g$ are evaluations of sine and cosine integrals
\begin{subequations}
\begin{eqnarray}
f(\omega,x)&=&  \mathrm{Ci}(2\,k\,x)\sin(2\,k\,x) - \left(\mathrm{si}(2\,k\,x)-\frac{\pi}{2}\right)\cos(2\,k\,x), \label{eq:f}
\\
g(\omega,x)&=& \mathrm{-Ci}(2\,k\,x)\cos(2\,k\,x) - \left(\mathrm{si}(2\,k\,x)-\frac{\pi}{2}\right)\sin(2\,k\,x). \label{eq:g}
\end{eqnarray}
\label{eq:fg}
\end{subequations}
Assuming that $\Psi$ is constant along $x$ it eliminates when writing an admittance transmission equation
%\begin{widetext}
\begin{equation}
Y_1(\omega) = \frac{1}{\mathrm{j}\,Z_{c1}}\,\left[\frac{1-{\alpha}_{g1}}{k\,x_1}- \frac{\left(1+{\alpha}_{f1}\right)\left(\frac{1-{\alpha}_{g2}}{k\,x_2}-\mathrm{j}\,\frac{Z_{c2}}{Z_2}\right)+\left(1+{\alpha}_{f1}\right)
\left(1+{\alpha}_{f2}\right)\tan(\sigma)}
{\left(1+{\alpha}_{f2}\right)-\left(\frac{1-{\alpha}_{g2}}{k\,x_2}-\mathrm{j}\,\frac{Z_{c2}}{Z_2}\right)\tan(\sigma)}\right].
\label{eq:A_Nederveen}
\end{equation}
%\end{widetext}
with
\begin{equation}
\sigma = k\left( x_2(1+(1-\mathrm{j})\alpha_2\,\xi_2)-x_1(1+(1-\mathrm{j})\alpha_1\,\xi_1) \right)
\label{eq:sigma}
\end{equation}
where $\xi_i=\ln(k\,x_i)+g_i$, and $g_i = g(\omega,x_i)$. Subscripts $(\cdot)_{i}$ with $i = 1,2$ denote input and output end of the cone, respectively.
The input impedance $Z_1$ is the inverse of Eq.~(\ref{eq:A_Nederveen}), which written as \begin{equation}
Z_1 = \frac{Z_2\,A+B}{Z_2\,C+D}
\label{eq:impedance}
\end{equation}
describes the impedance transformation between between output and input end in form of a transfer matrix $[A,B;C,D]$ with these elements:
\begin{equation}
\begin{array}{lclll} 
A & = & 		           &\frac{r_2}{r_1}     &\cos(\sigma)(1+\alpha_{f2})-\frac{1}{k\,x_1} \sin(\sigma)(1+\alpha_{g2})\\
B & = &Z_{c1}          &\frac{r_1}{r_2}\,\mathrm{j}  &\sin(\sigma)\\
C & = &\frac{1}{Z_{c1}}& \frac{r_2}{r_1}\,\mathrm{j} &\sin(\sigma)(1+\alpha_{f1})(1+\alpha_{f2})+ ...\\
  &  &                 & &\mathrm{j}\ \frac{1}{k\,x_1}\left(\frac{1}{k\,x_1}\sin(\sigma)(1+\alpha_{g1})(1+\alpha_{g2}) + \right. ...\\ 
  &  &                 & &\hspace{4em} \left. \cos(\sigma)(1+\alpha_{f1})(1+\alpha_{g2})- \right. ...\\
	&  &                 & &\hspace{4em} \left. \frac{r_2}{r_1}\cos(\sigma)(1+\alpha_{f2})(1+\alpha_{g1})\ \right)\\
D & = &                &\frac{r_1}{r_2} \left(\vphantom{\frac{1}{k\,x_1}} \right.&\left.\cos(\sigma)(1+\alpha_{f1})+\frac{1}{k\,x_1}\sin(\sigma)(1+\alpha_{g1})\right)
\end{array}
\label{eq:TM_Nederveen}
\end{equation}
With
\begin{equation}
\alpha^\prime= \frac{1}{2}\sqrt{2}\,\delta_v\left(\frac{\gamma-1}{\sqrt{Pr}}+1\right)
\end{equation}
being a boundary layer thickness including both viscous and thermal losses, Eq.~(\ref{eq:sigma}) becomes
\begin{equation}
\sigma = k\,L(1+\frac{\alpha^\prime}{r_{\mathrm{eff}}}(1-\mathrm{j})), \label{eq:eqa} %[-]
\end{equation}
where
\begin{equation}
r_{\mathrm{eff}}      = \frac{r_2-r_1}{g_2-g_1+\ln\frac{r_2}{r_1}}. \label{eq:reff} 
\end{equation}
can be interpreted as an \emph{effective radius} of the conical frustum.\\
The transfer matrix Eq.~(\ref{eq:TM_Nederveen}), written here conformal with Benade's original publication~\cite{Benade1988} also found in musical acoustics textbooks~\cite{Fletcher2005a, Chaigne2016}, is the main result of this letter.
 
\section{Discussion}
The reorganization of Nederveen's admittance formula following the notation of standard transfer matrices can conveniently be compared to the literature.
The contribution of visco-thermal losses is scaled by a dimensionless number which relates the corresponding boundary layer thickness to a characteristic length.
The characteristic length in a duct is the quotient of area and perimeter~\cite{Pierce1989}, the hydraulic radius. 
The radius change in a cone motivates the definition for an effective radius $r_{\mathrm{eff}}$, which captures the entire losses within the cone in a single value. Several definitions have been suggested in the past:
The simplest approximation is based on the arithmetic mean $r_{\mathrm{eff}} = (r_2+r_1)/2$.
Another option is to use the radius of a cylinder with the same volume-to-surface ratio as the cone, resulting in $r_{\mathrm{eff}}=(r_2^3-r_1^3)/(3r_1(r_2+r_1))$, which is similar to an empirically useful value~\cite{Chabassier2019} $r_{\mathrm{eff}}= (2\,r_1+r_2)/3$.
A different approach is to calculate an averaged propagation constant for the cone as the integral mean of the local propagation constant~\cite{Walstijn2002, Kulik2007} from cylinder theory 
\begin{equation}
\begin{array}{lcl} 
\bar{\sigma}_{con.} &=& \left(\frac{1}{x_2-x_1}\int_{x_1}^{x_2}{k_{cyl.}(x)\,dx}\right) L \\
                    &=&k\,L\left(1+\alpha^\prime(1-\mathrm{j})\frac{1}{x_2-x_1}\int_{x_1}^{x_2}\frac{1}{r(x)} dx\right)\\
\end{array}
\label{eq:sigmacon}
\end{equation}
which yields $r_{\mathrm{eff}} = (r_2-r_1)/\ln(r_2/r_1)$.
Taking into account, that the hydraulic radius in a conical duct refers to a spherical cap area rather than to a planar cross-sectional area, leads to a correction factor~\cite{Thibault2023} $c_m= 1+(1/m^2(1-\sqrt{m^2+1})^2)$ to $r_{\mathrm{eff}}$, depending on the taper $m = (r_2-r_1)/L$ of the cone. 
The effective radius Eq.~(\ref{eq:reff}) resulting from Nederveen's approach resembles the one from averaging with Eq.~(\ref{eq:sigmacon}), but with a correction depending on cone length and frequency, according to Eq.~(\ref{eq:g}) evaluated at both ends of the cone. \\
Aside from this correction which affects the loss parameter $\alpha^\prime/r_{\mathrm{eff}}$, Nederveen's approach introduces two additional dimensionless, complex numbers $\alpha_f$ and $\alpha_g$, which appear in scaling factors to the propagation terms in $\sin(\sigma)$ and $\cos(\sigma)$.\\ 
In the limit of negligible losses ($\eta \rightarrow 0$) Eq.~(\ref{eq:TM_Nederveen}) converges to the transfer matrix of a loss-free cone ($\alpha = \alpha_f = \alpha_g = 0$), including the limit case of a loss-free cylinder ($r_2\rightarrow r_1$ and $1/(k\, x_i)\rightarrow 0$).
For vanishing taper, however, the limits $\lim_{r_2\to r_1} \alpha_{f,g}$ are small non-zero values in the order of $\alpha$ which prevent convergence towards the transfer matrix of a cylinder with wall losses.
To achieve this convergence in practical applications it is sufficient to force $\lim_{r_2\to r_1} \alpha_{f} = 0$, e.g. by sigmoid regularization of Eq.~(\ref{eq:af_cone}).\\ 
Kulik~\cite{Kulik2007} has discussed the self-consistency of a cone transfer matrix extended by the loss model Eq.~(\ref{eq:sigmacon}), by showing that $A_{0}/A_{n} = B_{0}/B_{n} = C_{0}/C_{n} = D_{0}/D_{n} = 1$ holds regardless of $n$, where the subscripts $(.)_j$ with $j = 0,n$ denote the number of subdivisions the cone.
Similarly, with $\tilde{Z}_i = Z_i/Z_{ci}$ and $Z_{ci} = \rho\,c/(\pi r_i^2)$ inserted in Eq.~(\ref{eq:impedance}), the matrix elements in Eq.~(\ref{eq:TM_Nederveen}) become non-dimensional and numerical evaluation shows, that $A_{0}/A_{n} = B_{0}/B_{n} = C_{0}/C_{n} = D_{0}/D_{n} = C(\omega)$. 
This means, that with Nederveen's cone model a discretization into shorter conical segments does not affect the impedance result.\\
\section{Application}
We demonstrate the effect of different models of the acoustics of the cone by calculating the fundamental resonance frequency $f_{R1}$, which is at the first maximum of the input impedance magnitude. 
It is an essential property for the sound generation in wind instruments, where the excitation mechanism is non-linearly coupled to the resonating air column. 
The high pitch sensitivity of humans demands a high frequency accuracy in determining peak frequencies.
\\
The comparison of Nederveen's model to others is done within the same computation scheme, by modifying the parameters in Eq.~(\ref{eq:TM_Nederveen}) accordingly:
The model of Kulik~\cite{Kulik2007} is obtained from  Eq.~(\ref{eq:TM_Nederveen}) by setting $\alpha_{fi} = \alpha_{gi}=f_i = g_i= 0$ and replacing $k\,x_i$ by $\omega/c\,(1+\alpha^\prime/r_i(1-\mathrm{j}))\,x_i$ with $i = 1,2$.
The approximation of the conical geometry by a chain of $n$ discrete cylinders represents another model, called \emph{cylindrical-slices model} hereafter. Its transfer matrix is the product of $n$ transfer matrices according to Eq.~(\ref{eq:TM_Nederveen}) in which, additional to the above changes, $r_{1,j} = r_{2,j} = r_{\mathrm{eff},j}$ are replaced by the discrete radii $r_j$ ($j =1\ldots n$) approximating $r(x)$.
The single-cone models do not require such a discretization and are potentially more efficient.
This motivates their comparison, especially for long cones.
As in typical geometries of woodwind instruments the taper is small; with area ratios $c_m$ in the order of 10$^{-4}$, the assumption of planar wavefronts is applicable.
Consequently, the cylindrical-slices model is regarded as a reference in Figure~\ref{fig:FIG1} which shows the fundamental $f_{R1,con}$ of the two single-cone models relative to the fundamental $f_{R1,cyl}$.
The cone length $l$ is varied from both ends while the taper $m$ remains constant. 
The maximum lengths correspond to straight cones as of simplified geometries of typical woodwinds, according to Table~\ref{tab:woodwinds}.
While both single-cone models of Kulik and Nederveen converge towards the cylindrical-slices model for small $l$, the deviation becomes significant for long slim cones.
The minor but systematic differences between the models raise the question of a reliable reference solution.
The theory of Zwikker and Kosten~\cite{Zwikker1949} captures well the physical problem but needs a discretization along the length~\cite{Thibault2023}. 
To compare the transmission matrix models to this reference, we use as test case a very long slim conical frustum, with the taper and input radius of a bassoon, and three meters length closed at its far end. \\
In Figure~\ref{fig:FIG2}, the results of the two single-cone models by Kulik~\cite{Kulik2007} and Nederveen (Eq.~(\ref{eq:TM_Nederveen})) are compared with the cylindrical-slices model, and the Zwikker and Kosten solution computed using one-dimensional finite elements with the software OpenWind~\cite{Openwind2022}. 

\begin{table}[htbp]
\centering
\begin{tabular}{lllll}
			& bassoon & oboe & tenor sax & soprano sax  \\ \hline
 $R_1$ [mm]& 2.0 & 1.5 & 4.1 &3.5\\
 $L$  [mm]& 2501 & 566  & 1369 & 650\\
 $m$  [mm/m]& 7 & 12.4 & 26.5 & 35.5
\end{tabular}
\caption{\label{tab:woodwinds}Conical frustum geometries (input radius $R_1$, length $L$ and, taper $m$) related to typical woodwinds~\cite{Nederveen1969}}
\end{table}

\begin{figure}[t]
\includegraphics[width=1\textwidth]{FIG1}
\caption{\label{fig:FIG1}{Deviation of impedance peak frequency $f_{R1}$ of weakly conical frustums as calculated from different single-cone transfer matrices: Kulik~\cite{Kulik2007} (dashed), and Nederveen~(Eq.~(\ref{eq:TM_Nederveen})) (straight), referenced to the result 
of a cylindrical-slices model. Markers indicate typical cone geometries of woodwind instruments (see Table~\ref{tab:woodwinds}). Along each curve, the taper $m$ is constant and the input radius $r_1$ varies with $l$ as $r_{1} = R_1+m(L-l)/2$}}
\raggedright
\end{figure}

\begin{figure}[t]
\includegraphics[width=1\textwidth]{FIG2}
\caption{\label{fig:FIG2}{Impedance magnitude of a bassoon-like conical frustum ($r_1 = 2.1$~mm, $r_2= 23.5$~mm, $L = 3$~m) closed at the far end. Calculations with three different transfer matrix methods (cylindrical slices, Kulik~\cite{Kulik2007}, and Nederveen~(Eq.~(\ref{eq:TM_Nederveen}))) in comparison to a numerical solution (1D FEM) computed with Openwind~\cite{Openwind2022}. All calculations refer to air at 20~$^\circ$C with physical properties given by Chaigne and Kergomard~\cite{Chaigne2016}}}
\raggedright
\end{figure}

\section{Conclusion}
Accounting for wall losses in a cone by integrating the local propagation constant of a cylinder along the center axis overestimates the dissipation.
This is not due to the bulging of the wavefront, as it also appears for weakly tapered cones with almost perfectly planar wavefronts for which a series of short cylindrical slices is a valid approximation.
The deviation to such a model is generally very small but increases with length, taper, and decreasing input radius and can become significant for long slim cones (Fig.~\ref{fig:FIG1}).
The theoretical physical background has been elaborated recently by Thibault et al.~\cite{Thibault2023}, who point out that no exact closed-form solution for the governing equations to this problem exists.
This is the reason for us to highlight the approximation suggested by Nederveen~\cite{Nederveen1969} already in the year 1969, which probably was the first attempt to include visco-thermal losses in an analytical description of the acoustic input impedance in a cone:
%Already in 1969, an approximate solution had been suggested by Nederveen~\cite{Nederveen1969}, which probably was the first attempt to include visco-thermal losses in an analytical description of the acoustic wave propagation in a cone.
This pioneering work agrees remarkably well with an up-to-date finite element model for a long slim cone (Fig.~\ref{fig:FIG2}), without the need for discretization along the axis at the expense of evaluating two sine and cosine integral expressions~Eq.~(\ref{eq:fg}). 
The here presented transfer matrix Eq.~(\ref{eq:TM_Nederveen}) can be directly included in existing impedance calculation frameworks.\\
Future work could generalize this transfer matrix by modification of the perturbation functions to ensure a smooth transition towards the cylindrical geometry with zero taper.
 
 \section*{Acknowledgments}
For valuable discussions and support, the authors thank Alexis Thibault, Augustin Ernoult, Juliette Chabassier, Malte Kob, and Peter K\"oltzsch. Augustin Ernoult kindly provided the results of the finite-element calculations.


\begin{thebibliography}{10}

\bibitem{Benade1988}
A.~H. Benade.
\newblock Equivalent circuits for conical waveguides.
\newblock {\em The Journal of the Acoustical Society of America},
  83(5):1764--1769, 1988.

\bibitem{Chabassier2019}
Juliette Chabassier and Robin Tournemenne.
\newblock About the transfert matrix method in the context of acoustical wave
  propagation in wind instruments ({R}esearch {R}eport {RR}-9254).
\newblock Technical report, INRIA Bordeaux, 2019.

\bibitem{Chaigne2016}
Antoine Chaigne and Jean Kergomard.
\newblock {\em Acoustics of {Musical} {Instruments}}.
\newblock Springer, 2016.

\bibitem{Ernoult2020}
Augustin Ernoult and Jean Kergomard.
\newblock Transfer matrix of a truncated cone with viscothermal losses:
  application of the {WKB} method.
\newblock {\em Acta Acustica}, 4(2):7, 2020.
\newblock Number: 2 Publisher: EDP Sciences.

\bibitem{Fletcher1988}
N.~H. Fletcher and Suszanne Thwaites.
\newblock Obliquely truncated simple horns: Idealized models for vertebrate
  pinnae.
\newblock {\em Acta Acustica united with Acustica}, 65(4):194--204, 1988.

\bibitem{Fletcher2005a}
Neville~H. Fletcher and Thomas~D. Rossing.
\newblock {\em The Physics of Musical Instruments}.
\newblock Springer, 2005.

\bibitem{Grothe2013}
Timo Grothe.
\newblock {\em Experimental investigations of Bassoon Acoustics}.
\newblock PhD thesis, Technische Universit\"{a}t Dresden, 2013.

\bibitem{Kulik2007}
Yakov Kulik.
\newblock Transfer matrix of conical waveguides with any geometric parameters
  for increased precision in computer modeling.
\newblock {\em The Journal of the Acoustical Society of America},
  122(5):EL179--EL184, 2007.

\bibitem{Nederveen1969}
C.J. Nederveen.
\newblock {\em Acoustical Aspects of Woodwind Instruments}.
\newblock Frits Knuf. Reprint Northern Illinois University Press, De Kalb 1998,
  1969.

\bibitem{Openwind2022}
Openwind.
\newblock Open {Wind} {INstrument} {Design}, February 2022.

\bibitem{Pierce1989}
Allan~D. Pierce.
\newblock {\em Acoustics}.
\newblock The Acoustical Society of America, 1989.

\bibitem{Thibault2023}
Alexis Thibault, Juliette Chabassier, Henri Boutin, and Thomas H\'{e}lie.
\newblock Transmission line coefficients for viscothermal acoustics in conical
  tubes.
\newblock {\em Journal of Sound and Vibration}, 543:117355, January 2023.

\bibitem{Walstijn2002}
Maarten van Walstijn.
\newblock {\em Discrete-Time Modelling of Brass and Reed Woowind Instruments
  with Application to Musical Sound Synthesis}.
\newblock PhD thesis, The University of Edinburgh, 2002.

\bibitem{Zwikker1949}
C.~Zwikker and Cornelis~Willem Kosten.
\newblock {\em Sound absorbing materials}.
\newblock Elsevier Pub. Co., New York, 1949.

\end{thebibliography}



\end{document}
