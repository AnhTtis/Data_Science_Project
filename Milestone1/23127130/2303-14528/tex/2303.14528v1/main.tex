%%%%%%%%%%%%%%%%%%%%%%%%%%%%%%%%%%%
%This is the LaTeX ARTICLE template for RSC journals
%Copyright The Royal Society of Chemistry 2016
%%%%%%%%%%%%%%%%%%%%%%%%%%%%%%%%%%%

\documentclass[twoside,twocolumn,9pt]{article}
\usepackage{extsizes}
\usepackage[super,sort&compress,comma]{natbib} 
\usepackage[version=3]{mhchem}
\usepackage[left=1.5cm, right=1.5cm, top=1.785cm, bottom=2.0cm]{geometry}
\usepackage{balance}
\usepackage{mathptmx}
\usepackage{sectsty}
\usepackage{graphicx} 
\usepackage{lastpage}
\usepackage[format=plain,justification=justified,singlelinecheck=false,font={stretch=1.125,small,sf},labelfont=bf,labelsep=space]{caption}
\usepackage{float}
\usepackage{fancyhdr}
\usepackage{fnpos}
\usepackage[english]{babel}
\addto{\captionsenglish}{%
  \renewcommand{\refname}{Notes and references}
}
\usepackage{array}
\usepackage{droidsans}
\usepackage{charter}
\usepackage[T1]{fontenc}
\usepackage[usenames,dvipsnames]{xcolor}
\usepackage{setspace}
\usepackage[compact]{titlesec}
\usepackage{hyperref}
%%%Please don't disable any packages in the preamble, as this may cause the template to display incorrectly.%%%

\usepackage{epstopdf}%This line makes .eps figures into .pdf - please comment out if not required.

\usepackage{physics}

\definecolor{cream}{RGB}{222,217,201}

\begin{document}

\pagestyle{fancy}
\thispagestyle{plain}
\fancypagestyle{plain}{
%%%HEADER%%%
\renewcommand{\headrulewidth}{0pt}
}
%%%END OF HEADER%%%

%%%PAGE SETUP - Please do not change any commands within this section%%%
\makeFNbottom
\makeatletter
\renewcommand\LARGE{\@setfontsize\LARGE{15pt}{17}}
\renewcommand\Large{\@setfontsize\Large{12pt}{14}}
\renewcommand\large{\@setfontsize\large{10pt}{12}}
\renewcommand\footnotesize{\@setfontsize\footnotesize{7pt}{10}}
\makeatother

\renewcommand{\thefootnote}{\fnsymbol{footnote}}
\renewcommand\footnoterule{\vspace*{1pt}% 
\color{cream}\hrule width 3.5in height 0.4pt \color{black}\vspace*{5pt}} 
\setcounter{secnumdepth}{5}

\makeatletter 
\renewcommand\@biblabel[1]{#1}            
\renewcommand\@makefntext[1]% 
{\noindent\makebox[0pt][r]{\@thefnmark\,}#1}
\makeatother 
\renewcommand{\figurename}{\small{Fig.}~}
\sectionfont{\sffamily\Large}
\subsectionfont{\normalsize}
\subsubsectionfont{\bf}
\setstretch{1.125} %In particular, please do not alter this line.
\setlength{\skip\footins}{0.8cm}
\setlength{\footnotesep}{0.25cm}
\setlength{\jot}{10pt}
\titlespacing*{\section}{0pt}{4pt}{4pt}
\titlespacing*{\subsection}{0pt}{15pt}{1pt}
%%%END OF PAGE SETUP%%%

%%%FOOTER%%%
\fancyfoot{}
\fancyfoot[LO,RE]{\vspace{-7.1pt}\includegraphics[height=9pt]{head_foot/LF}}
\fancyfoot[CO]{\vspace{-7.1pt}\hspace{13.2cm}\includegraphics{head_foot/RF}}
\fancyfoot[CE]{\vspace{-7.2pt}\hspace{-14.2cm}\includegraphics{head_foot/RF}}
\fancyfoot[RO]{\footnotesize{\sffamily{1--\pageref{LastPage} ~\textbar  \hspace{2pt}\thepage}}}
\fancyfoot[LE]{\footnotesize{\sffamily{\thepage~\textbar\hspace{3.45cm} 1--\pageref{LastPage}}}}
\fancyhead{}
\renewcommand{\headrulewidth}{0pt} 
\renewcommand{\footrulewidth}{0pt}
\setlength{\arrayrulewidth}{1pt}
\setlength{\columnsep}{6.5mm}
\setlength\bibsep{1pt}
%%%END OF FOOTER%%%

%%%FIGURE SETUP - please do not change any commands within this section%%%
\makeatletter 
\newlength{\figrulesep} 
\setlength{\figrulesep}{0.5\textfloatsep} 

\newcommand{\topfigrule}{\vspace*{-1pt}% 
\noindent{\color{cream}\rule[-\figrulesep]{\columnwidth}{1.5pt}} }

\newcommand{\botfigrule}{\vspace*{-2pt}% 
\noindent{\color{cream}\rule[\figrulesep]{\columnwidth}{1.5pt}} }

\newcommand{\dblfigrule}{\vspace*{-1pt}% 
\noindent{\color{cream}\rule[-\figrulesep]{\textwidth}{1.5pt}} }

\makeatother
%%%END OF FIGURE SETUP%%%

%%%TITLE, AUTHORS AND ABSTRACT%%%
\twocolumn[
  \begin{@twocolumnfalse}
{\includegraphics[height=30pt]{head_foot/SM}\hfill\raisebox{0pt}[0pt][0pt]{\includegraphics[height=55pt]{head_foot/RSC_LOGO_CMYK}}\\[1ex]
\includegraphics[width=18.5cm]{head_foot/header_bar}}\par
\vspace{1em}
\sffamily
\begin{tabular}{m{4.5cm} p{13.5cm} }

\includegraphics{head_foot/DOI} & \noindent\LARGE{\textbf{Splay and polar order in a system of hard wedges: numerical confrontation with the Density Functional Theory}} \\%Article title goes here instead of the text "This is the title"
\vspace{0.3cm} & \vspace{0.3cm} \\

 & \noindent\large{Piotr Kubala,$^{\ast}$\textit{$^{a}$} and Micha\l{} Cie\'sla\textit{$^{b}$}} \\%Author names go here instead of "Full name", etc.

\includegraphics{head_foot/dates} & \noindent\normalsize{Recent experimental discoveries of novel nematic types with polar order, including ferroelectric nematic and splay nematic have brought the resurgence of the interest in  polar and modulated phases with a non-standard symmetry. One of the most important factors that is widely believed to be crucial for the formation of the new phases is the wedge (pearlike) shape of the mesogenic molecules. Such shapes were treated using second-virial density functional theory in [De Gregorio, P \textit{et al.}, \textit{Soft Matter}, 2016, \textbf{12(23)}, 5188-5198], where the authors showed that the $K_{11}$ splay elastic constant can become negative due to solely entropic reasons leading to long-range splay and polar correlations. To verify whether the predictions are correct, we performed Monte Carlo simulations of the same hard-core particles used in the DFT study. As our results suggest, no nonstandard liquid crystalline phases emerge; polar and splay correlations are at most short-range or completely absent. On the other hand, a polar ferroelectric crystal was observed. The system was analyzed in terms of nematic, smectic, and hexatic bond order parameters, as well as correlation functions. We also propose the possible reasons for failed predictions of DFT computations, where the most probable one is the absence of the density modulation in the theoretical model.} \\%The abstrast goes here instead of the text "The abstract should be..."

\end{tabular}

 \end{@twocolumnfalse} \vspace{0.6cm}

  ]
%%%END OF TITLE, AUTHORS AND ABSTRACT%%%

%%%FONT SETUP - please do not change any commands within this section
\renewcommand*\rmdefault{bch}\normalfont\upshape
\rmfamily
\section*{}
\vspace{-1cm}


%%%FOOTNOTES%%%

\footnotetext{\textit{$^{a}$~Institute of Theoretical Physics, Jagiellonian University in Krak\'ow, \L{}ojasiewicza 11, 30-348 Krak\'ow, Poland. Email: piotr.kubala@doctoral.uj.edu.pl}}

\footnotetext{\textit{$^{b}$~Institute of Theoretical Physics, Jagiellonian University in Krak\'ow, \L{}ojasiewicza 11, 30-348 Krak\'ow, Poland. Email: michal.ciesla@uj.edu.pl}}

%%%END OF FOOTNOTES%%%


%%%MAIN TEXT%%%%

\section{Introduction}

Since their discovery in 1888 \cite{Reinitzer1888}, liquid crystals have become one of the most fruitful areas of study for various phases characterized by unexpected properties resulting from their internal structure based on the orientational order of the anisotropic molecules that built them \cite{Stephen1974, Chandrasekhar1992, deGennes1993}. The internal structure of liquid crystals leads to their anisotropic macroscopic properties, which are the basis for numerous applications, the most popular of which are liquid crystal displays (LCD-s) \cite{Uchida2022}. 

The most common and simplest liquid crystalline phases are nematics and smectics, where the director $\vu{n}$ -- the direction along which molecules tend to align -- is the same in the whole system. Apart from these structures, modulated phases, where the direction of the ordering changes, are of particular interest. Such phases are typically induced by the breaking of specific symmetry at the molecular level, and therefore, instead of parallel alignment of neighboring molecules, a slight tilt in their orientations is preferred; thus, the director is no longer spatially constant. In general, the deformations of a uniform director field are usually described in terms of the Oseen-Zocher-Frank free energy\cite{Oseen1933,Zocher1933,Frank1958}.
%
\begin{equation}
    \mathcal{F}_\text{OZF} = \frac{1}{2} K_{11} [\vu{n} (\div{\vu{n}})]^2 + \frac{1}{2} K_{22} [\vu{n} \vdot (\curl{\vu{n}})]^2 + \frac{1}{2} K_{33}[\vu{n} \cp (\curl{\vu{n}})]^2.
\end{equation}
%
The subsequent terms correspond to, respectively, splay, twist, and bend deformation modes. $K_{ii}$, called the elastic constants, determine the energetic cost of deviation from a uniform director field. In most scenarios, they are all positive with $K_{22} < K_{11} < K_{33}$\cite{Demus2008,Luckhurst2001}, thus any deformations are opposed by restoring torques. There are, however, experimental, theoretical, and numerical cases, where one of them is anomalously low, or even negative \cite{Meyer1976,Dozov2001,Cestari2011,Shamid2013,Borshch2013,Chen2013,Greco2015,Allesandro2017,Chiappini2021,Kubala2022}. In the latter case, the uniform nematic ceases to be the stable structure in favor of modulated phases. These include cholesterics \cite{Harris1999}, blue phases \cite{Wright1989}, twist-bend nematic $\text{N}_\text{TB}$\cite{Borshch2013,Chen2013,Greco2014,Chiappini2021,Kubala2022}, splay nematic $\text{N}_\text{S}$\cite{Dhakal2010,Mertelj2018,Rosseto2020,Sebastian2020}, splay-bend nematic $\text{N}_\text{SB}$\cite{Archbold2015,Pajak2018,Chaturvedi2019,Fernandez2020} and smectic $\text{Sm}_\text{SB}$\cite{Chiappini2021,Kubala2022} as well as splay-twist-bend smectic $\text{Sm}_\text{STB}$\cite{Chiappini2021}. Of particular interest has recently been splay nematic $\text{N}_\text{S}$ phase, as it is closely related to the ferroelectric nematic $\text{N}_\text{F}$\cite{Mandle2017,Mertelj2018,Chen2020,Sebastian2022} with global polarization, which has been gaining a lot of traction due to its scientific and practical significance\cite{Chen2020}.

Among the most important factors responsible for the softening of the elastic constants is the shape effect. It was repeatedly proven using theoretical models\cite{Greco2015,Gregorio2016,Dussi2016} as well as molecular dynamics (MD) and Monte Carlo (MC) simulations\cite{Memmer2002,Greco2015,Chiappini2021,Kubala2022}, that purely repulsive bent-core (banana-shaped) particles spontaneously form $\text{N}_\text{TB}$ phase with broken mirror symmetry. One of these works is that of De Gregorio \emph{et al.}\cite{Gregorio2016}, which, by means of density functional theory (DFT) calculations, predicts the existence of the $\text{N}_\text{TB}$ phase for banana-shaped particles, as well as the splay nematic $\text{N}_\text{S}$ phase for wedge-shaped (pearlike) particles. Moreover, the onset of spontaneous director field modulation was accompanied by the polar order; in the former case, the bend was coupled to the transversal polarization of the particle, while in the latter case, it was coupled to the longitudinal polarization. To our knowledge, the $\text{N}_\text{S}$ phase was not observed in an MD or MC study with purely entropic interactions.

\begin{figure}[htbp]
    \centering
    \includegraphics[width=0.8\linewidth]{gfx/wedge}
    \caption{Illustration of a wedge-shaped molecule used in the study. It consists of eleven co-linear tangent balls with diameters increasing linearly from 0.78 to 1.22. Long molecular axis $\vu{a}$ points from the smallest to the largest ball. The same model was used in Ref.~\cite{Gregorio2016}.}
    \label{fig:wedge}
\end{figure}

In this manuscript, we attempt to recreate the splay nematic or the splay smectic phase using MC simulations of the hard-core wedge model (see Fig.~\ref{fig:wedge}). The particles are built of eleven co-linear tangent hard spheres with diameters linearly increasing from 0.78 to 1.22. The same model was used in Ref.~\cite{Gregorio2016}, where the negative $K_{11}$ splay constant was reported. A similar shape, with six beads instead of eleven, was the topic of our previous work\cite{Kubala2022phases}. As our results suggest, long-range polar and splay order appear only in the crystalline phase, whereas in the nematic and smectic phase, they are present at most locally. The paper is structured as follows. In Sec.~\ref{sec:mc} we describe the numerical methods that we were using. Then, in Secs.~\ref{sec:order_params} and \ref{sec:corr} we describe the order parameters and correlation functions. Sec.~\ref{sec:phases} walks through all phases, from the isotropic liquid to the solid state. Finally, Sec.~\ref{sec:polar_splay} discusses the presence and range of polar and splay correlations in the system, while Sec.~\ref{sec:conclusions} sums up the findings and outlines the potential further directions of studies.


\section{Methods}

\subsection{Monte Carlo simulations} \label{sec:mc}

To establish the phase sequence for the particle model under consideration, we performed NpT Monte Carlo simulations\cite{Wood1968,Allen2017,Allen2019} of $N = 4320$ particles using our RAMPACK software\footnote{\url{https://github.com/PKua007/rampack}}. As the interactions are purely hard-core, the only independent thermodynamic parameter is the reduced pressure and temperature ratio $p^*/T^* = p V/k_B T$, where $V$ is the volume of a single particle and $k_B$ is the Boltzmann constant. We investigated $p^*/T^*$ from the range $[1.0, 11.0]$ corresponding to packing fractions $\eta \in [0.15, 0.50]$, which covered the entire phase sequence, from the isotropic liquid to the solid state. The simulations consisted of the thermalization run and the production run, where ensemble averages of order parameters and correlation functions were gathered. Both simulation phases lasted around $10^8$ Monte Carlo cycles. Each cycle consisted of $N$ rototranslation moves, $N/10$ flip moves, and a single box move. In the rototranslation move, a molecule was chosen at random and its position and rotation were perturbed randomly. During the flip move, a random shape was rotated 180 degrees so that the sense of the molecular axis $\vu{a}_i$ changed ($\vu{a}_i \rightarrow -\vu{a}_i$), while the direction remained unchanged. The flip move facilitated polar order relaxation, especially for high packing fractions. For both types of moves, perturbations were accepted only if no overlap was introduced. During the box move, box vectors $\vb{b}_1, \vb{b}_2, \vb{b}_3$ were randomly perturbed. Overlapping configurations were rejected immediately, while non-overlapping ones were accepted with a probability given by the Metropolis criterion
%
\begin{equation}
    P = \min\qty{1, \exp(N \log \frac{V_1}{V_0} - \frac{p \Delta V}{T})},
\end{equation}
%
where $V_0 = \abs{\vb{b}_1 \vdot (\vb{b}_2 \cp \vb{b}_3)}$ is the box volume before the move, $V_1$ -- after the move, and $\Delta V = V_1 - V_0$. For $p^*/T^* < 9.0$ ($\eta < 0.45$), the box was tetragonal and its dimensions were logarithmically scaled, with the $z$ axis perturbed independently of the $x$ and $y$ axes (making the scaling anisotropic). For $p^*/T^* \ge 9.0$ ($\eta \ge 0.45$), the box was triclinic and the box vectors were altered by adding small random vectors to them. Initial configurations for all liquid phases were slightly diluted hexagonal honeycomb layers with random up-down orientations of particles. For the crystalline phase at $p^*/T^* = 11$ ($\eta = 0.50$), the initial configuration was the final snapshot of the $p^*/T^* = 8$ ($\eta = 0.44$) smectic A liquid. In all the cases periodic boundary conditions were used.

\subsection{Order parameters} \label{sec:order_params}

To quantitatively characterize the observed phases, we ensemble averaged three order parameters: nematic order $\expval{P_2}$, smectic order $\expval{\tau}$ and hexatic bond order $\expval{\psi_6}$. The nematic order parameter $P_2$ is defined as\cite{Eppenga1984}
%
\begin{equation} \label{eq:p2}
    P_2 = \frac{3}{2}\qty(\vu{a} \cdot \vu{n} - \frac{1}{3}),
\end{equation}
%
where $\vu{a}$ is a molecular axis and $\vu{n}$ is the director. The mean $P_2$ in a single snapshot can be conveniently computed using the $\vb{Q}$-tensor\cite{Eppenga1984}:
%
\begin{equation} \label{eq:Q}
    \vb{Q} = \frac{1}{N} \sum_{i=1}^{N} \frac{3}{2}\qty(\vu{a}_i \otimes \vu{a}_i - \frac{1}{3} \vb{I}),
\end{equation}
%
where the summation goes over all particles in the system. In this formulation, $P_2$ is the eigenvalue of $\vb{Q}$ with the highest magnitude, and $\vu{n}$ is the corresponding eigenvector. Then $\expval{P_2}$ is calculated by averaging it over uncorrelated system snapshots.

The smectic order parameter can be defined as\cite{Gennes1993}
%
\begin{equation}
    \expval{\tau} = \expval{\frac{1}{N}\abs{\sum_{i=1}^{N}\exp(\imath \vb{k} \vdot \vb{r}_i)}},
\end{equation}
%
where $\vb{k}$ is the smectic wavevector, $\vb{r}_i$ is the position of the $i$-th shape, $\expval{\dots}$ is ensemble averaging, and $\imath$ is imaginary unit. In a finite system, $\vb{k}$ must be compatible with periodic boundary conditions. In general, one can use the following formula to enumerate all possibilities:
%
\begin{equation}
    \vb{k} = h \vb{g}_1 + k \vb{g}_2 + l \vb{g}_3.
\end{equation}
%
Here, $\vb{g}_i$ are reciprocal box vectors\cite{Kittel2018} and $h, k, l$ are integers (the Miller indices). In this study, all initial states had four layers stacked along the $z$-axis, therefore, we assumed $hkl = 004$.

To quantify local hexatic order, one can use the hexatic bond order parameter $\expval{\psi_6}$\cite{Nelson2012}. It is essentially a two-dimensional parameter; thus, its computation has to be restricted to a single plane in a three-dimensional system. A natural choice is to compute it for each layer separately and average it over all layers. For a single layer $l$ it is defined as
%
\begin{equation}
    \psi_6^l = \frac{1}{N_l} \sum_{i=1}^{N_l}\frac{1}{6} \abs{\sum_{j \in \text{6NN}(i)} \exp(6 \imath \theta_{ij})},
\end{equation}
%
where $N_l$ is the number of particles in the $l$-th layer, $\text{6NN}(i)$ is a list of the six nearest neighbors of the $i$-th particle and $\theta_{ij}$ is an angle between the projection of vector joining $i$-th and $j$-th particle onto the layer and an arbitrary direction within the layer. Then, $\psi_6^l$ is averaged over all four layers and uncorrelated system snapshots
%
\begin{equation}
    \expval{\psi_6} = \expval{\frac{1}{4} \sum_{l=1}^{4} \psi_6^l}.
\end{equation}

\subsection{Correlations} \label{sec:corr}

As postulated in Ref.~\cite{Gregorio2016}, splay deformation mode is coupled to a longitudinal polarization of the molecule. Thus, we should be looking for a long-range order in the polarization field and splay correlations. To probe the polarizations, we used the transversal $S_\perp^{110}(r_\perp)$ correlation function\cite{Stone1978}. It is defined layer-wise, similar to $\psi_6$:
%
\begin{equation} \label{eq:s110}
    S_\perp^{110}(r_\perp) = \expval{\frac{1}{4} \sum_{l=1}^{4} \expval{\vu{a}_{i_l} \cdot \vu{a}_{j_l}}_{i_l j_l}},
\end{equation}
%
where $l$ is the layer number, $\expval{\dots}_{i_l j_l}$ denotes averaging over all pairs $(i_l, j_l)$ of particles, whose distance calculated along the layer lies in the range $[r_\perp - \dd{r}, r_\perp + \dd{r}]$ and $2\dd{r}$ is the numerical bin size.

Quantifying splay deformation is significantly more difficult. The reason is that, contrary to all the observables introduced earlier, the splay term is a derivative of the director field itself. Thus, to compute it numerically, a sufficiently smooth vector field estimation is needed. This requires ensemble averaging of the field prior to computation. If a long-range order is not present, instantaneous short-range correlations evolve with time, which, with the help of translational Goldstone mode, average out to a uniform, nematic-like director field. In order to probe local splay correlations, we propose a different scheme. The director field with only the splay deformation and a single singularity in the origin is
%
\begin{equation} \label{eq:n_splay}
    \vu{n}(\vb{r}) = \frac{\vb{r}}{r},
\end{equation}
%
which can be described as a hedgehog-like structure. If we choose a single point $\vu{n} = \vu{n}(\vb{r})$ and move in a transverse direction by $\vb{r}_\perp$ to $\vu{n}' = \vu{n}(\vb{r} + \vb{r}_\perp)$, the angle between $\vu{n}$ and $\vu{n}'$ should be approximately a linear function of $\norm{\vb{r}_\perp}$, as long as $\norm{\vb{r}_\perp} \ll \norm{\vb{r}}$. Based on this, we define $P(\theta|r_\perp)$ as a conditional probability distribution of finding two particles $i$ and $j$ with a transversal distance $r_\perp$, whose molecular axes form angle $\theta = \cos^{-1}\abs{\vu{a}_i \cdot \vu{a}_j}$ (which respects $\vu{n} \leftrightarrow -\vu{n}$ director symmetry), normalized as
%
\begin{equation}
    \int_{0}^{\pi} P(\theta|r_\perp) \dd{\theta} = 1 \qquad \forall r_\perp.
\end{equation}
%
Then, the local splay correlation should manifest itself as a set of maxima of $P(\theta|r_\perp)$ as a function of both $\theta$ and $r_\perp$, for which $\theta$ and $r_\perp$ are linearly dependent.


\section{Results}

\subsection{Phase sequence} \label{sec:phases}

\begin{figure}[htbp]
    \centering
    \includegraphics[width=0.9\linewidth]{gfx/pd}
    \caption{Phase sequence as a function of packing fraction $\eta$. Colored areas represent the ranges of subsequent phases, as labeled above the diagram: Iso (isotropic), N (nematic), SmA (smectic A), $\text{Cr}_\text{DS}\text{P}_\text{F}$ (ferroelectric double-splay crystal). Lines are the order parameters' dependence on $\eta$: nematic order $\expval{P_2}$ (black solid), smectic order $\expval{\tau}$ (red dashed), hexatic bond order $\expval{\psi_6}$ (blue dotted). Hatched regions indicate the vicinity of phase transitions.}
    \label{fig:pd}
\end{figure}

\begin{figure*}[htbp]
    \centering
    \includegraphics[width=0.75\linewidth]{gfx/pack}
    \caption{System snapshots for all observed phases apart from isotropic liquid: (a) N ($\eta = 0.32$), (b) SmA ($\eta = 0.45$), (c) $\text{Cr}_\text{DS}\text{P}_\text{F}$ ($\eta = 0.50$). Each row depicts different aspects of a single snapshot. The first column is the top view of the simulation box, the second one -- the front view, and the last one shows the geometric centers (black dots) of wedges overlain on top of the side view (the second column). Particles in the first two columns are color coded according to $P_1 = \vu{a}_i \vdot \vu{z}$.}
    \label{fig:pack}
\end{figure*}

Fig.~\ref{fig:pd} shows the phase sequence as well as the order parameters as a function of the packing fraction $\eta$ obtained from the Monte Carlo simulations. A rigorous determination of the orders of phase transitions is beyond the scope of this work, so the regions in the vicinity of transition points are shown with a hatch filling. The following phases were observed: isotropic (Iso), nematic (N), smectic A (SmA), and ferroelectric double-splay crystal ($\text{Cr}_\text{DS}\text{P}_\text{F}$). System snapshots for all phases, except Iso, are shown in Fig.~\ref{fig:pack}.

At the lowest densities, below $\eta = 0.2$, a disordered isotropic liquid is formed. It then undergoes a phase transition to the nematic phase [see Fig.~\ref{fig:pack}(a)], as indicated by the sharp jump in both the $\eta$ and the nematic order $\expval{P_2}$ parameter (black solid line in Fig.~\ref{fig:pd}). It may suggest that the transition is first order. No smectic or hexatic order is observed, as indicated by the smectic $\expval{\tau}$ (red dashed line) and hexatic bond $\expval{\psi_6}$ (blue dotted line) order parameters. The former is close to zero in the entire range of the nematic phase. Similarly, the latter retains its disorder value around $\expval{\psi_6} \approx 0.37$. The nematic order increases monotonically from $\expval{P_2} = 0.8$ for $\eta = 0.23$ to $\expval{P_2} = 0.95$ for $\eta = 0.32$. Such high $\expval{P_2}$ values are above the typical range $\expval{P_2} \in [0.3, 0.7]$ seen in experiments\cite{Chandrasekhar1980}, but are not uncommon for hard-core systems treated numerically\cite{Vega2001,Lansac2003,Kubala2022,Kubala2022phases}. In particular, as seen in Fig.~\ref{fig:pack}(a) and discussed in detail in the next section, no long-range splay or polar order is present in the system.

Around $\eta = 0.33$, an N-SmA phase transition occurs. The snapshot of the SmA phase can be seen in Fig.~\ref{fig:pack}(b). Nematic order $\expval{P_2}$ remains high and a rather quick ascent in $\expval{\tau}$ value to $\expval{\tau} = 0.8$ can be clearly seen. In experimental setups, the N-SmA phase transition can be both of the first and the second order \cite{Singh2000}, however, for idealized hard-core interactions the former one is usually observed\cite{McGrother1996,Polson1997,Lansac2003}. In our simulation data, $\eta$ did not experience a sudden jump and $\expval{\tau}$ rose fast, however smoothly, which would suggest a second-order phase transition. The smectic order $\expval{\tau}$ increases up to $\expval{\tau} = 0.94$ near the SmA-$\text{Cr}_\text{DS}\text{P}_\text{F}$ phase transition. At the same time, the hexatic bond order $\expval{\psi_6}$ rises almost linearly from $\expval{\psi_6} = 0.4$ at the lower phase boundary to $\expval{\psi_6} = 0.55$ at the top. A higher local hexatic order facilitates a more optimal packing, allowing the system to achieve a higher packing fraction $\eta$. On the other hand, $\expval{\psi_6} = 0.55$ is still lower than in systems with long-range hexatic order, where the values $\expval{\psi_6} > 0.7$ are observed\cite{Kubala2022phases}. Contrary to the nematic phase, the SmA system snapshot Fig.~\ref{fig:pack}(b) reveals some amounts of splay and polar order; however, it appears to be rather short-ranged.

Above $\eta = 0.46$, the system freezes and the ferroelectric double-splay crystal ($\text{Cr}_\text{DS}\text{P}_\text{F}$) phase emerges [cf. Fig.~\ref{fig:pack}(c)] with a sharp jump in $\eta$ and $\expval{\psi}$ and, simultaneously, a slight decrease of $\expval{P_2}$ and $\expval{\tau}$. The phase was first observed and discussed in detail in Ref.~\cite{Kubala2022phases}, where a similar system of wedges was considered; however, the particles were built of six instead of eleven beads. Here, we only give a brief summary. The phase is built up of layers. Within each layer, we observe square clusters of particles with the same polarization, either ``up'' [$P_1 > 0$, colored red in Fig.~\ref{fig:pack}(c)] or ``down'' ($P_1 < 0$, colored blue). These clusters are arranged in a chessboard-like long-range pattern. The particles within each cluster form an ordered hexatic crystal with a high $\expval{\psi_6} = 0.7$ value. Long-ranged splay deformation is also clearly visible -- the particles orient in a hedgehog-like pattern discussed in Sec.~\ref{sec:corr}. The presence of splay deformation weakens the global nematic and smectic order, as confirmed by a lowered $\expval{P_2} = 0.9$ and $\expval{\tau} = 0.92$ values, as compared to $\expval{P_2} = 0.97$ and $\expval{\tau} = 0.95$ in the dense SmA phase. Between the clusters, there are sharp domain walls, where the signs of both polarization and splay vector $\vu{n} (\div{\vu{n}})$ change abruptly. The ``chessboard'' layers are stacked on each other, matching the signs of polarization and splay deformation between them\footnote{In contrast to antiferroelectric splay ($\text{Cr}_\text{S}\text{P}_\text{A}$) and double splay ($\text{Cr}_\text{DS}\text{P}_\text{A}$) crystal phases, see Ref.~\cite{Kubala2022phases}}. However, it is important to note that more defects were observed for this system compared to wedges built of six balls in Ref.~\cite{Kubala2022phases}.

\subsection{Polar order and splay deformation} \label{sec:polar_splay}

\begin{figure}[htbp]
    \centering
    \includegraphics[width=0.9\linewidth]{gfx/pack_splay}
    \caption{Hypothetical (a) single splay and (b) double splay nematic structures proposed in Ref.~\cite{Rosseto2020}. Layout and color coding correspond to the first two columns in Fig.~\ref{fig:pack}.}
    \label{fig:pack_splay}
\end{figure}

\begin{figure}[htbp]
    \centering
    \includegraphics[width=0.9\linewidth]{gfx/S110}
    \caption{Layer-wise transversal correlation $S_\perp^{110}(r_\perp)$ function for nematic (black solid line), low-density smectic A (red dashed line), high-density smectic A (blue dotted line) and ferroelectric double-splay crystal (orange dot-dashed line) as a function of transversal distance $r_\perp$. $\bar{d} = 1$ is the average diameter of balls building the wedge.}
    \label{fig:S110}
\end{figure}

\begin{figure}[htbp]
    \centering
    \includegraphics[width=0.65\linewidth]{gfx/angle_N}
    \includegraphics[width=0.65\linewidth]{gfx/angle_lowSm}
    \includegraphics[width=0.65\linewidth]{gfx/angle_highSm}
    \includegraphics[width=0.65\linewidth]{gfx/angle_Cr}
    \caption{The evolution of the distribution of angles between the molecules with the transversal distance $r_\perp$ between them. For each section with a fixed value of $r_\perp$, the maps show a conditional probability density $P(\theta|r_\perp)$ of finding the two particles with the angle given by the $y$-axis value. Subsequent panels correspond to different phases and packing fractions: (a) N ($\eta = 0.32$), (b) SmA ($\eta = 0.37$), (c) SmA ($\eta = 0.45$) and (d) $\text{Cr}_\text{DS}\text{P}_\text{F}$ ($\eta = 0.50$).}
    \label{fig:angle_hist}
\end{figure}

As noted in the Introduction, the authors in Ref.~\cite{Gregorio2016} predicted the softening of $K_{11}$ splay constant, when coupling with longitudinal polarization was taken into account. Thus, according to second-order DFT calculations, long-range splay and polar order should emerge. As there does not exist a non-singular director field with only splay deformation [see Eq.~\eqref{eq:n_splay} and Ref.~\cite{Mertelj2018}], it has to be accompanied by other deformations (twist, bend), or the singularities must be allowed. Rosseto \emph{et al.}\cite{Rosseto2020} discussed two types of splay nematic $\text{N}_\text{S}$\footnote{This structure has actually both splay and bend modes, however the name \emph{splay-bend} nematic $\text{N}_\text{SB}$ is widely used to describe a different director profile\cite{Pajak2018}.}, which can be observed. These structures are presented in Fig.~\ref{fig:pack_splay}. The first one, already known in the literature\cite{Chaturvedi2019}, was called \emph{single splay nematic} [Fig.~\ref{fig:pack_splay}(a)], where stripes with opposite splay and polarization are stacked in an alternating manner. In the lowest order, polarization and director fields are assumed to be
%
\begin{align}
    \vu{n}(x) &= (\sin[\Theta(x)], 0, \cos[\Theta(x)]), \qquad \Theta(x) = \Theta_0 \sin(\kappa x), \\
    \vb{P}(x) &= P_0 \cos(\kappa x) \vu{n}(x),
\end{align}
%
where $\Theta_0$ determines the strength of the deformation, $\kappa$ is the structure wavevector, and $P_0$ is the maximal local polarization. These profiles have been proven to work well near the critical point; however, for lower temperatures, sharper changes are observed\cite{Rosseto2020}. The second one was named \emph{double splay nematic} [Fig.~\ref{fig:pack_splay}(b)]. There, columns with opposite splay and polarization form a checkerboard pattern with the fields in lowest order given by
%
\begin{align}
    \vu{n}(x, y) &= \frac{(\Theta_0\sin(\kappa x)\cos(\kappa y), \Theta_0\cos(\kappa x)\sin(\kappa y), 1)}{\sqrt{\Theta_0^2\sin(\kappa x)^2\cos(\kappa y)^2 + \Theta_0^2\cos(\kappa x)^2\sin(\kappa y)^2 + 1}}, \\
    \vb{P}(x, y) &= 2P_0 \cos(\kappa x) \cos(\kappa y) \vu{n}(x).
\end{align}
%
As DFT calculations from Ref.~\cite{Gregorio2016} suggest, $K_{11}$ becomes negative for the number density $\rho \approx 0.050$ (see Fig.~3 therein), which corresponds to $\eta_S \approx 0.30$. In our simulations, it is the nematic phase near the N-SmA phase boundary. Thus, long-range polar and splay correlations should be observed for a high-density nematic and in the whole range of the smectic phase. As stated in Sec.~\ref{sec:phases} and visible in system snapshots, this is not the case. However, local correlations, especially in the smectic phase, seem to be present.

Polar order correlations can be quantified using $S_\perp^{110}(r_\perp)$. The dependence for selected packing fractions is shown in Fig.~\ref{fig:S110}. For a high-density nematic just below N-SmA transition point ($\eta = 0.32$, black solid line), lying above $\eta_S$, polar correlations are marginal -- they are non-zero only for nearest neighbors ($r_\perp \approx \bar{d}$, where $\bar{d} = 1$ is mean ball diameter) with $S_\perp^{110} \approx 0.15$. Slightly higher correlations are observed for a low-density smectic A ($\eta = 0.37$, red dashed line). Here, the maximum is $S_\perp^{110}(\bar{d}) \approx 0.3$ and the correlations have a slightly longer range, reaching $r_\perp \approx 2\bar{d}$. For high-density smectic A, the correlations are more prominent, with maximum $S_\perp^{110}(\bar{d}) \approx 0.7$, but still short-ranged, reaching $r_\perp \approx 4\bar{d}$. Local maxima lie near integer multiples of $\bar{d}$, which suggests that they correspond to nearest, next-nearest, next-next-nearest, etc. neighbors. Long-range correlations appear only after crystallization. For the $\text{Cr}_\text{DS}\text{P}_\text{A}$ phase at $\eta = 0.50$, positive correlations reach $S_\perp^{110} \approx 0.87$ and extend as far as $r_\perp \approx 9.5\bar{d}$. After that point, the polarizations are anticorrelated, which is in line with the checkerboard pattern visible in Fig.~\ref{fig:pack}(c). Interestingly, the phase realizes the double splay nematic structure proposed in Ref.~\cite{Rosseto2020}, but as a crystalline phase.

An insight into the range of splay order can be given by the relative probability density $P(\theta|r_\perp)$ of the angles $\theta$ between molecules at a given distance between them $r_\perp$. Histograms are shown in Fig.~\ref{fig:angle_hist}. In the nematic phase [Fig.~\ref{fig:angle_hist}(a)], $\theta \approx 10^\circ$ is the most probable angle for most $r_\perp$ with high spread. The highest maximum is for nearest neighbors ($r_\perp \approx \bar{d}$). The oscillations in the maxima positions stretch up to $r_\perp \approx 4\bar{d}$ and are spaced by $r_\perp \approx \bar{d}/2$. They can be divided into two series, each with the spacing $\Delta r_\perp \approx \bar{d}$, and shifted with respect to each other by $r_\perp \approx \bar{d}/2$. One of the two series probably corresponds to particles with the same polarizations, which prefer larger angles $\theta$ due to packing reasons. On the contrary, the second series of maxima corresponds to anticorrelated polarizations, which enable particles to move closer to each other (hence the shift in relation to the former series), while at the same time, the angles between them are smaller. It should be noted that the range of angle correlations is higher than for the $S_\perp^{110}$ correlation function. For a low-density smectic A [Fig.~\ref{fig:angle_hist}(b)], the histogram is qualitatively similar, with only a quantitative difference. The preferred angle is lower -- $\theta \approx 6^\circ$ with a narrower spread. Local maxima oscillations are also present with a similar range of $r_\perp \approx 4\bar{d}$. The situation changes slightly for a high-density smectic A [Fig.~\ref{fig:angle_hist}(c)]. Apart from a smaller preferred angle $\theta \approx 5^\circ$ with even less spread than for a low-density smectic, local maxima are no longer oscillatory and their positions $(r_\perp, \theta)$ are close to a linear relation -- the signature we would expect from splay clusters. Unfortunately, the correlations are short-ranged, spanning up to $r_\perp \approx 5\bar{d}$. It changes drastically for the crystalline phase [Fig.~\ref{fig:angle_hist}(d)]. Maxima are clearly visible as far as $r_\perp \approx 15\bar{d}$ and they form an almost perfect linear relation given by $\theta = 0.53^\circ + 1.97^\circ r_\perp/\bar{d}$. The maxima become weaker with increasing $r_\perp$, which is expected since correlations between clusters with opposite polarization are also taken into account. These correlations are also the source of the wide local maximum around $\theta \approx 10^\circ$ for $r_\perp > 10\bar{d}$. Between the main maxima, one can observe long vertical lines. They likely also originate from off-lattice correlations between adjacent clusters.

As both the visual inspection of system snapshots and the quantitative correlation functions clearly show that both long-range polar order and splay deformation are missing, while at the same time, the DFT predicts the N-$\text{N}_\text{S}$ transition, the question arises of why there is a mismatch between DFT and MC simulations. The first possible reason, which also applies to most theoretical frameworks, is that DFT computations use a second-virial expansion with a correcting Parson-Lee factor\cite{Parsons1979,Lee1987}, while MC simulations do not restrict interactions to the two-particle term. Another reason, arguably more probable, is that the authors did not include density modulation in their calculations, while the alleged N-$\text{N}_\text{S}$ transition point is $\eta \approx 0.30$, which is close to the N-SmA transition point $\eta \approx 0.33$ in our simulations. As second-virial theories often give qualitatively correct results, however, with slightly wrong quantitative predictions\footnote{See for example Ref.~\cite{Greco2015} of the same research group, where theoretical and numerical N-$\text{N}_\text{TB}$ transition points are for number densities, respectively, $\rho \approx 0.056$ and $\rho \approx 0.050$.}, $\text{N}_\text{S}$ may have lower free energy than $N$ only over the N-SmA point, where non-modulated smectic A may be favorable over both modulated nematic and smectic phases. 


\section{Conclusions} \label{sec:conclusions}

We have performed Monte Carlo simulations of hard wedge-shaped particles built of eleven tangent balls. We scanned a wide range of packing fractions covering the whole phase sequence: isotropic (Iso), nematic (N), smectic A (SmA), and ferroelectric double-splay crystal ($\text{Cr}_\text{DS}\text{P}_\text{F}$). The phases were classified using visual inspection of simulation snapshots, order parameters, and correlation functions. For the shape under consideration, second-order DFT calculations~\cite{Gregorio2016} suggested the existence of a long-range polar and splay order for packing fractions $\eta > 0.30$, which cover high-density nematics and the entire smectic range. Despite the DFT predictions, no long-range polar or splay order was observed until the system crystallized. Only short-ranged correlations were present in the smectic phase, and in the nematic phase, they were even less prominent. The possible reason why the DFT predictions were not met may be that the theoretical N-$\text{N}_\text{S}$ transition point is close to the N-SmA point in our study, while at the same time, the smectic order was not included in the original manuscript. Another possibility is that the theory is only second-virial. However, since a negative value of the $K_{11}$ splay constant was proved to be theoretically possible, it is worth further exploring other variants of the wedge model, for example with a different length or a smooth, convex surface.


\section*{Data availability}
The datasets generated during and/or analyzed during the current study are available from P.K. upon reasonable request.


\section*{Code availability}

The source code of an original simulation package used to perform Monte Carlo sampling is available at \url{https://github.com/PKua007/rampack}. 


\section*{Author Contributions}

P.K.: conceptualization, data curation, formal analysis, funding acquisition, investigation, software, visualization, writing. M.C.: conceptualization, writing.


\section*{Conflicts of interest}

There are no conflicts to declare.


\section*{Acknowledgements}

P.K. acknowledges the support of Ministry of Science and Higher Education (Poland) grant no. 0108/DIA/2020/49. M.C. acknowledges the support of National Science Center in Poland grant no. 2021/43/B/ST3/03135. The authors are grateful to Prof. Lech Longa for inspiring discussions.
%Part of the numerical simulations were carried out with the support of the Interdisciplinary Center for Mathematical and Computational Modelling (ICM) at the University of Warsaw under grant no. GB76-1.


%%%REFERENCES%%%
\bibliography{main.bib} %You need to replace "rsc" on this line with the name of your .bib file
\bibliographystyle{rsc} %the RSC's .bst file

\end{document}
