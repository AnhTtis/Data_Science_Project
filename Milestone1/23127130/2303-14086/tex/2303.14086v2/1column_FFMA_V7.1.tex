\documentclass[11pt,draftclsnofoot, onecolumn]{IEEEtran}
\hyphenation{op-tical net-works semi-conduc-tor}
\usepackage{graphicx,cite,epsfig,amssymb,amsmath,subfigure,url,stfloats,latexsym}
\usepackage{array}
\usepackage{amsfonts}
\usepackage{pgfplots}
\usepackage{algorithm}
\usepackage[noend]{algpseudocode}
%\usepackage{tikz}
%\usepackage{tkz-orm}
\usepackage{epstopdf}
\usepackage{amsfonts,amsthm}
\usepackage{multirow}
\usepackage{mathrsfs}
\usepackage{subfigure}
%\usepackage{subcaption}


%\usepackage{algorithmic}
\usepackage{algorithm}
\usepackage{algpseudocode}


\newtheorem{theorem}{Theorem}
\newtheorem{lemma}{Lemma}
\newtheorem{corollary}{Corollary}
\newtheorem{property}{Property}
\newtheorem{definition}{Definition}
\newtheorem{proposition}{Proposition}
\newtheorem{remark}{Remark}
\newtheorem{conjecture}{Conjecture}
\newtheorem{example}{Example}[section]


\newenvironment{varalgorithm}[1]
  {\algorithm\renewcommand{\thealgorithm}{#1}}
  {\endalgorithm}

\graphicspath{{./}{icons/}}
\tikzset{My Line Style/.style={samples=400}}
\graphicspath{{figures/}}

\usepackage[flushleft]{threeparttable}

\begin{document}

\markboth{IEEE TRANSACTIONS ON COMMUNICATIONS, Vol. XX, No. Y, Month 2022} { \ldots}
\title{\mbox{}\vspace{0.5cm}\\
\textsc{\huge Finite Field Multiple Access} \vspace{1.5cm}}

\vspace{1.2cm}
\author{\normalsize
Qi-Yue~Yu, {\it IEEE Senior Member},
Jiang-Xuan~Li,
and Shu~Lin, {\it IEEE Life Fellow} 
%\thanks{Q.-Y.~Yu (email: yuqiyue@hit.edu.cn) and J.-X.~Li (email: 21S005095@stu.hit.edu.cn) are with the Communication Research Center, Harbin Institute of Technology, China. S. Lin (email: shulin@ucdavis.edu) is in University of California, Davis, U.S.}
%\thanks{The work presented in this paper was supported by the National Natural Science Foundation of China under Grand No. 62071148.}
%\thanks{The paper was submitted on Feb. 12, 2017, and revised on \today.}\\
}



\date{\today}
\renewcommand{\baselinestretch}{1.2}
\thispagestyle{empty} \maketitle \thispagestyle{empty}

%\newpage
%\setcounter{page}{1}

\vspace{0.1in}
\begin{abstract}
In the past several decades, various multiple-access (MA) techniques have been developed and used. 
These MA techniques are carried out in complex-field domain to separate the outputs of the users. 
It becomes problematic to find new resources from the physical world. 
It is desirable to find new resources, physical or virtual, to confront the fast development of MA systems.
In this paper, an algebraic virtual resource is proposed to support multiuser transmission. 
For binary transmission systems, the algebraic virtual resource is based on assigning each user an element pair (EP) from a finite field GF($p^m$). 
The output bit from each user is mapped into an element in its assigned EP, called the output symbol. 
For a downlink MA system, the output symbols from the users are jointly multiplexed into a unique symbol in the same field GF($p^m$) for further physical-layer transmission. 
The EPs assigned to the users are said to form a multiuser algebraic uniquely decodable (UD) code. 
Using EPs over a finite field, a network, a downlink, and an uplink orthogonal/non-orthogonal MA systems are proposed, which are called finite-field MA (FFMA) systems. 
Methods for constructing algebraic UD codes for FFMA systems are presented. An FFMA system can be designed in conjunction with the classical complex-field MA techniques to provide more flexibility and varieties. 




\vspace{0.1in}
{\emph{Index Terms} --- 
Multiple access, finite field, binary transmission system, element pair, unique decodable code, finite-field multi-access, network multiple access, downlink multiple access, uplink multiple access, semantic communications, AWGN channel.
} \noindent
\end{abstract}


%\newpage
%\setcounter{page}{1}
\section{Introduction}
Multiple access (MA) is one of the most important techniques for wireless communications. 
During the past several decades, various MA techniques have been developed for mobile communications
to support various users and services \cite{FAdachi1, ZDing2017_survey, YChen_2018}. 
MA is a technique for multiple users to access the same terminal simultaneously, i.e., a base station (BS); or one terminal simultaneously transmits multiple signals to different users. 
There are various types of MA techniques, from orthogonal MA (OMA) to non-orthogonal MA (NOMA) techniques. 
The classical OMA techniques are frequency division multiple access (FDMA), time division multiple access (TDMA), code division multiple access (CDMA), and orthogonal frequency division multiple access (OFDMA). The OMA technique is to assign orthogonal resources to different users, thus, there is no interference among users. However, the spectrum efficiency (SE) of OMA is generally equal to one. 

 
To improve the SE, NOMA has been paid much attention lately, 
since it can provide a higher SE via overloading information using the same time and/or frequency resources to support more users \cite{QWang_2018,QY_ISJ_2019,LDai2015,HNiko2013}. 
In \cite{QWang_2018}, a unified transceiver was proposed, including both transmit and received signal expressions. 
In \cite{QY_ISJ_2019}, a unified multiuser coding framework for multiuser transmission was proposed, and  uniquely-decodable mapping (UDM) was introduced for separating users at the receiving end(s) without ambiguity.



Besides extensive work on MA techniques, 
work on constructions of uniquely decodable codes (UDCs) for MA channels was investigated in \cite{Liao1972,Kasami1976,Chang1976, Kasami1978,Peterson1979,Chevillat1981,Kasami1983,Van1983,Vanroose1992,Jevtic1992,Fan1995,Khachatrian1998,Bross1998,Cheng2001,Kiviluoto2007,Yu_P2P},
but mainly in integer domain for multiple-access adder channels. 
The essential idea behind UDCs is to extract data from the superimposed signals without ambiguity. 
%The achievable data rate of a multiuser UDC is generally higher than any single user data rate, and some UDCs can even correct errors in noisy channels \cite{Kasami1978}. 
How to construct multiuser UDCs is a complicated issue.
Besides UDCs, some researchers also consider lattice codes for supporting multiuser transmission \cite{R1, R2}. In \cite{R1}, the authors utilize lattice codes for supporting compute-forward multiple access. 
The authors of \cite{R2} further present a lattice-partition-based downlink non-orthogonal multiple access framework. 


   
Most of the current MA techniques are combinations of coding, modulation, interleaving, and orthogonal resources jointly to provide beneficial performances. Because of the superposition of signals from multiple users, the present MA techniques are all processed in complex-field by signal processing, which consume physical resources, i.e., time, frequency, code, and space. It becomes problematic to find new resources from the physical world. Consequently, the MA technique seems to get into a bottleneck. 
Therefore, it is a challenge to researchers and system designers in communications to find new resources to break through the bottleneck.



In this paper, an \textit{algebraic virtual resource} is proposed to support multiuser transmission. 
Since binary transmission systems are widely used, this paper designs the algebraic virtual resource based on assigning each user an \textit{additive inverse element pair (AIEP)} from a finite field GF($p^m$). 
The output bit from each user is mapped into an element in its assigned AIEP, called the output symbol. 
For downlink transmission, the output symbols from the users can be multiplexed into a sum-pattern symbol in the same field GF($p^m$) which is then modulated and transmitted. 
When the transmitted signal is received and demodulated at the receiving end, 
the receiver can separate transmitted bits from received superimposed element without ambiguity. 
The AIEPs assigned to the users are said to form a set of \textit{uniquely decodable AIEPs (UDAIEPs)}, 
called a \textit{multiuser algebraic UD code}. 
Using UDAIEPs (or AIEPs) over a finite-field, network MA, downlink MA and uplink MA systems are all called \textit{finite-field MA (FFMA)} systems. 
The downlink FFMA system can achieve full diversity gain over fading channels, thus improves the bit error rate (BER) performance. 
The uplink FFMA system can support \textit{many-to-one mapping} between the received superposition signals and the transmit multiuser bits in a Gaussian multiple-access channel (GMAC), which may provide inspiration of designing novel multiuser transceivers.
The \textit{major difference} between the complex-field and finite-field MA techniques is that
the former one is based on complex-field signal processing, and the latter one is operated in finite-field. 
More importantly, we can jointly design FFMA and complex-field MA (CFMA) systems, and obtain a \textit{fusion MA} system, where the FFMA is operated in network layer and the CFMA works in physical layer.




The rest of this paper is organized as follows.  
In Section II, two methods for constructing multiuser UD codes over finite fields for MA are presented, one based on prime fields and the other based on extensions of prime fields. 
Section III gives an overview of FFMA systems.
In Section IV, a downlink FFMA scheme using a multiuser UD code for frequency-selective fading channels is presented. 
Sections V and VI respectively investigate an orthogonal FFMA (O-FFMA) system by using an \textit{orthogonal} multiuser UD code constructed over GF($2^m$) and a non-orthogonal FFMA (NO-FFMA) system over GF($3^2$) for an uplink GMAC.
In Section VII, simulations of the error performances of the proposed FFMA schemes are given. Section VIII concludes the paper with some remarks.


In this paper, we use $x$, $\mathbf{x}$ and $\mathbf{X}$ as a variable, a vector, and a matrix, respectively. 
The symbol $\mathbb {C}$ is used to denote the complex-field. 
The notions $\bigoplus$, $\bigotimes$, $\sum$ and $\prod$ are used to denote modulo-$q$ addition, 
modulo-$q$ multiplication, complex addition, and complex product, respectively.
The symbols $\lceil x \rceil$ and $\lfloor x \rfloor$ denote the smallest integer that is equal to or larger than $x$ and the largest integer that is equal to or smaller than $x$, respectively.
The notation $(a)_q$ stands for modulo-$q$, and/or an element in GF($q$).



%\vspace{0.2in}
\section{Uniquely Decodable Codes over Finite Fields for Multiple Access Channels}

In this section, we present \textit{algebraic unique decodable (UD) codes} constructed based on finite fields. These algebraic UD codes provide an additional resource for MA communications. 
The first part of this section presents a method for constructing UD codes based on \textit{prime fields} and the second part presents a class of \textit{orthogonal} algebraic UD codes which are constructed based on \textit{extension fields} of prime fields. 


\subsection{Construction of UD Codes Based on Prime Fields}

Let GF($p$) = $\{0, 1, \ldots, p-1\}$ be a prime field with $p > 2$.  
Partition the $p - 1$ nonzero elements GF($p$) into $(p - 1)/2$ 
\textit{mutually disjoint element-pairs} (\textit{EPs}), 
each EP consisting of a nonzero element $k$ in $\text{GF}(p)\backslash 0$ and 
its additive inverse $p - k$ (or simply $- k$). 
We call each such pair $(k, p - k)$ an \textit{additive inverse EP (AIEP)}. 
The partition of $\text{GF}(p)\backslash \{0\}$ into AIEPs, denoted by $\mathcal P$, is referred to as 
\textit{AIEP-partition}.


Let $J$ be a positive integer less than or equal to $(p - 1)/2$, 
i.e., $1 \le J \le (p - 1)/2$, and let $C_1, C_2, \ldots, C_{J}$ be $J$ AIEPs in $\mathcal P$. 
For $1 \le j \le J$, let $(j, p - j)$ denote the AIEP in $C_j$, i.e., $C_j = (j, p - j)$.
Denote $C_j$'s \textit{reverse order AIEP (R-AIEP)} by $C_j^{\rm R} = C_{p-j} = (p - j, j)$,
where the subscript ``R'' of $C_j^{\rm R}$ stands for ``reverse order''.
The $J$ AIEPs $C_1, C_2, \ldots, C_J$ form a partition of a subset of $2J$ elements in GF$(p) \backslash \{0\}$, a sub-partition of $\mathcal P$. 
Let $\mathcal C$ denote the set $\{C_1, C_2, \ldots, C_J\}$, 
i.e., ${\mathcal C} = \{C_1, C_2, \ldots, C_J \}$.  
The $J$ R-AIEPs $C_1^{\rm R}, C_2^{\rm R}, \ldots, C_J^{\rm R}$ of $C_1, C_2, \ldots, C_J$ form a \textit{reverse order set} of $\mathcal C$, denoted by ${\mathcal C}^{\rm R} = \{C_1^{\rm R}, C_2^{\rm R}, \ldots, C_J^{\rm R} \}$ 
(or ${\mathcal C}^{\rm R} = \{C_{p-1}, C_{p-2}, \ldots, C_{p-J}\}$).




Let $(u_1, u_2, \ldots, u_J)$  be a $J$-tuple over GF($p$) in which the $j$-th component $u_j$ for $1 \le j \le J$, is an element from $C_j$. The $J$-tuple $(u_1, u_2, \ldots, u_J)$ is an element in the \textit{Cartesian product} $C_1 \times C_2 \times \ldots \times C_J$ of the AIEPs in $\mathcal C$. 
The modulo-$p$ sum $\tau = \bigoplus_{j=1}^{J} u_j$ of the $J$ components in $(u_1, u_2, \ldots, u_{J})$ 
is called the \textit{sum-pattern} of the $J$-tuple $(u_1, u_2, \ldots, u_{J})$ 
which is an element in GF($p$). 
The $J$-tuple $(p-u_1, p-u_2, \ldots, p-u_{J})$ is also an element in $C_1  \times C_2 \times \ldots \times C_{J}$. 
The sum-pattern of $(p-u_1, p-u_2, \ldots, p-u_{J})$ is 
$p - \tau = p -  \bigoplus_{j=1}^{J} u_j$. 
If $\tau = 0, p - \tau = 0$ (modulo $p$), i.e., if the \textit{sum-pattern} of $(u_1, u_2, \ldots, u_{J})$ is the zero element $0$ in GF($p$), 
the sum-pattern of $(p-u_1, p-u_2, \ldots, p-u_{J})$ is also the zero element $0$ in GF($p$).
Let $(u_1, u_2, \ldots, u_{J})$ and $(u_1', u_2', \ldots, u_{J}')$ be \textit{any two} $J$-tuples in $C_1  \times C_2 \times \ldots \times C_{J}$. 
If $\bigoplus_{j=1}^{J} u_j \neq \bigoplus_{j=1}^{J} u_j'$, 
then a sum-pattern uniquely specifies a $J$-tuple in $C_1  \times C_2 \times \ldots \times C_{J}$. 
That is to say that the mapping 
\begin{equation} \label{e.UDmap}
(u_1, u_2, \ldots, u_{J}) \longleftrightarrow \bigoplus_{j=1}^{J} u_j
\end{equation} 
is one-to-one mapping. 

In this case, given the sum-pattern $\tau = \bigoplus_{j=1}^{J} u_j$, 
we can uniquely recover the $J$-tuple $(u_1, u_2, \ldots, u_{J})$ \textit{without ambiguity}. 
We say that the Cartesian product $C_1 \times C_2 \times \ldots \times C_{J}$  
(or ${\mathcal C} = \{C_1, C_2, \ldots, C_{J}$) has a \textit{unique sum-pattern mapping (USPM) structural property}.


A set ${\mathcal C} = \{C_1, C_2, \ldots, C_{J}\}$ of $J$ AIEPs with the USPM structural property can be used as a type of \textit{finite field resource} for MA communications. 
Suppose in a MA communication system, 
the $J$ AIEPs $C_1, C_2, \ldots, C_{J}$ in ${\mathcal C}$ are assigned to $J$ users for bit mapping. 
If each user transmits a symbol from its assigned AIEP, 
the transmitter combines the $J$ transmitted symbols, 
forms the sum-pattern $\tau=\bigoplus_{j=1}^{J} u_j$, 
and transmits the sum-pattern over an MAC. 
Assume that the channel is noiseless.  
When the receiver(s) receives the sum-pattern $\tau=\bigoplus_{j=1}^{J} u_j$, 
the $J$ transmitted symbols in $(u_1, u_2, \ldots, u_{J})$ can be uniquely recovered from the sum-pattern $\tau$ without ambiguity.
Note that each sum-pattern contains information from $J$ users.



If we view each $J$-tuple $(u_1, u_2, \ldots, u_{J})$ in $C_1 \times C_2 \times \ldots \times C_{J}$ with the USPM structural property as a \textit{$J$-user codeword}, then ${\mathcal C} = C_1 \times C_2 \times \ldots \times C_{J}$ forms a $J$-user code over GF($p$) with $2^J$ codewords.  
We call ${\mathcal C} = C_1 \times C_2 \times \ldots \times C_{J}$ a $J$-user \textit{uniquely decodable (UD) AIEP code} over GF($p$), simply a $J$-user UDAIEP code. 
When this code is used for an MA communication system with $J$ users, 
the $j$-th component $u_j$ in a codeword $(u_1, u_2, \ldots, u_{J})$ is the symbol to be transmitted by the $j$-th user. 


Note that, for a $J$-user UDAIEP code ${\mathcal C}$, we can replace any AIEP $C_j$ in ${\mathcal C}$ by its R-AIEP $C_j^{\rm R}$, and still obtain a $J$-user UDAIEP code,
i.e., ${\mathcal C} = C_1 \times C_2 \times \ldots \times C_{j}^{\rm R} \times \ldots \times C_J$.
Applications of UDAIEP codes to MA communications will be presented in later sections of this paper. 


\begin{figure}[t]
  \centering
  \includegraphics[width=0.55\textwidth]{GF5.pdf} 
  \caption{A diagram in table-form of Example 1. 
  (a) The sum-patterns of $C_1 = (1, 4)$ and $C_2 = (2, 3)$ in GF($5$); 
  (b) the sum-patterns of $C_1 = (1, 4)$ and $C_2^{\rm R} = (3, 2)$ in GF($5$).
  }
  \vspace{-0.2in}
\end{figure}



\textbf{Example 1:} 
Consider the prime field GF($5$). Using this prime field, two AIEPs $C_1 = (1, 4)$ and $C_2 = (2, 3)$ can be constructed. 
The R-AIEPs of $C_1$ and $C_2$ are $C_1^{\rm R} = (4, 1)$ and $C_2^{\rm R} = (3, 2)$, respectively.
The Cartesian product $C_1 \times C_2$ of $C_1$ and $C_2$ satisfies Eq. (\ref{e.UDmap}) as shown in Fig. 1 (a). Hence, $C_1 \times C_2$ is a 2-user UDAIEP code over GF($5$).
When we replace $C_2$ by $C_2^{\rm R}$, the Cartesian product $C_1 \times C_2^{\rm R}$ of $C_1$ and $C_2^{\rm R}$ still satisfies Eq. (\ref{e.UDmap}), 
thus $C_1 \times C_2^{\rm R}$ also forms a 2-user UDAIEP code over GF($5$) as shown in Fig. 1 (b).
$\blacktriangle \blacktriangle$




For a $J$-user UDAIEP code $\mathcal C$ over GF($p$), 
no codeword can have sum-pattern equal to the zero element $0$ of GF($p$). 
Since the sum-patterns of the $2^J$ codewords in $\mathcal C$ must be distinct elements in GF($p$), 
$2^{J}$ must be less than or equal to $p - 1$, 
i.e., $2^J \le p - 1$. 
Hence, the number $J$ of users for an UDAIEP code over GF($p$) is upper bounded as follows:
\begin{equation} \label{e2.1}
  J \le \log_2 (p-1).
\end{equation}
Summarizing the results developed above, we have the following theorem.

\begin{theorem}
The Cartesian product of $J$ AIEPs over a prime field GF($p$) with $p > 2$ is a $J$-user UDAIEP code 
if and only if the sum-patterns of all its $2^J$ codewords are different nonzero elements in GF($p$) 
with $J$ upper bounded by $\log_2 (p-1)$.
\end{theorem}


Note that, there are totally $(p-1)/2$ AIEPs in GF($p$), 
and $\log_2 (p-1)$ of them can form an UDAIEP code.
Owing to the feature of finite-field,
we can totally have
$\eta_p = \lfloor \frac{p-1}{2\log_2 (p-1)}\rfloor$ UDAIEP codes over GF($p$) for $p>3$, 
forming a UDAIEP code set $\Xi$, i.e.,
\begin{equation*}
  \Xi = \{ {\mathcal C}(1), {\mathcal C}(2), \ldots, 
  {\mathcal C}(\eta_p)\},
\end{equation*}
where ${\mathcal C}(1), {\mathcal C}(2), \ldots, {\mathcal C}(\eta_p)$ are $\log_2 (p-1)$-user UDAIEP codes.
%We ignore the orders of AIEPs in ${\mathcal C}(l)$ for $1 \le l \le \eta_p$, since an AIEP and its R-AIEP have the same elements.




\begin{figure}[t]
  \centering
 % \subfigure[UDAIEP code ${\mathcal C}(1)$ over GF(17)] {
    \includegraphics[width=0.7\textwidth]{GF17.pdf}
 % \subfigure[UDAIEP code ${\mathcal C}(2)$ over GF(17)] {
 %   \includegraphics[width=0.7\textwidth]{GF17_2.pdf}}
  \caption{A diagram in table-form of Example 2. 
  (a) The $4$-user UDAIEP code ${\mathcal C}(1)$ over GF($17$) with $C_1 = (1,16), C_2 =(2,15), C_4 = (4,13), C_8=(8,9)$; and
  (b) the $4$-user UDAIEP code ${\mathcal C}(2)$ over GF($17$) with $C_3 = (3,14), C_5 =(5,12), C_6 = (6,11), C_7=(7,10)$.}
  \vspace{-0.2in}
\end{figure}




\textbf{Example 2:} 
Suppose we use the prime field GF($17$) for UDAIEP codes construction. 
Eight AIEPs over GF($17$) can be constructed, which are 
$C_1 = (1, 16), C_2 = (2, 15), C_3 = (3, 14), C_4 = (4, 13), 
C_5 = (5, 12), C_6 = (6, 11), C_7 = (7, 10), C_8 = (8, 9)$. 
Since $\log_2 (17-1) = 4$, there are at most $4$ AIEPs whose Cartesian product satisfies the necessary and sufficient condition given by Theorem 1. 



We can find two $4$-user UDAIEP codes, denoted by ${\mathcal C}(1)$ and ${\mathcal C}(2)$, whose Cartesian products satisfy the necessary and sufficient condition given by Theorem 1.
One of these two $4$-user UDAIEP code is 
${\mathcal C}(1) =\{C_1 = (1, 16), C_2 = (2, 15), C_4 = (4, 13), C_8 = (8, 9)\}$ as shown in Fig. 2 (a); 
and the other $4$-user UDAIEP code is 
${\mathcal C}(2) =\{C_3 = (3, 14), C_5 = (5, 12), C_6 = (6, 11), C_7 = (7, 10)\}$ as shown in Fig. 2 (b).
The Cartesian products ${\mathcal C}(1) = C_1 \times C_2 \times C_4 \times C_8$ 
and ${\mathcal C}(2) = C_3 \times C_5 \times C_6 \times C_7$ give two 4-user UDAIEP codes over GF(17). 
All the AIEPs are different between ${\mathcal C}(1)$ and ${\mathcal C}(2)$.
$\blacktriangle  \blacktriangle$




\subsection{Construction of UD Codes based on Extension Fields of Prime Fields}

Let $m$ be a positive integer and GF($p^m$) be the extension field of the prime field GF($p$). 
The extension field GF($p^m$) is constructed based on a primitive polynomial 
${\bf g}(X) = g_0 + g_1 X + g_2 X^2 + \ldots + g_m X^m$
of degree $m$ with coefficients from GF($p$) which consists of $p^m$ elements and contains GF($p$) as a subfield \cite{Shu2009}.



Let $\alpha$ be a primitive element in GF($p^m$). 
Then, the powers of $\alpha$, namely $\alpha^{-1} = 0, \alpha^0 = 1, \alpha, \alpha^2, \ldots, \\
\alpha^{(p^m - 2)}$, give all the $p^m$ elements of GF($p^m$). 
Each element $\alpha^j$, with $j = -1, 0, \ldots, p^m - 2$, in GF($p^m$) can be expressed as a linear sum of $\alpha^0 = 1, \alpha, \alpha^2, \ldots, \alpha^{(m - 1)}$ with coefficients from GF($p$) as follows:
\begin{equation} \label{e2.2}
    \alpha^j = a_{j,0} + a_{j,1} \alpha + a_{j,2} \alpha^2 + \ldots + a_{j,m-1} \alpha^{(m-1)}.      
\end{equation}
From (\ref{e2.2}), we see that the element $\alpha^j$ can be uniquely represented by the $m$-tuple 
$(a_{j,0}, a_{j,1}, a_{j,2},\ldots, a_{j,m-1})$ over GF($p$). 
Hence, an element in GF($p^m$) can be expressed in three forms, 
namely \textit{power, polynomial and $m$-tuple forms}.


The sum of two elements 
$\alpha^j = a_{j,0} + a_{j,1} \alpha + a_{j,2} \alpha^2 + \ldots + a_{j,m-1} \alpha^{(m - 1)}$ and 
$\alpha^k = a_{k,0} + a_{k,1} \alpha + a_{k,2} \alpha^2 + \ldots + a_{k,m-1} \alpha^{(m - 1)}$ 
is equal to
\begin{equation}
   \alpha^j + \alpha^k = (a_{j,0} + a_{k,0}) + (a_{j,1} + a_{k,1}) \alpha + \ldots 
   + (a_{j,m-1} + a_{k,m-1}) \alpha^{(m - 1)}.    
\end{equation}


The $m$-tuple representation of the sum $\alpha^j + \alpha^k$ is
\begin{equation}
    \left((a_{j,0} + a_{k,0}), (a_{j,1} + a_{k,1}), \ldots, (a_{j,m-1} + a_{k,m-1})\right). 
\end{equation}
For $0 \le i < m$, if $(a_{j,i}, a_{k,i})$ is an additive inverse pair over GF($p$) or a pair of zero elements $(0, 0)$, then $a_{j,i} + a_{k,i} = 0$ and $\alpha^j + \alpha^k  = 0$, 
i.e., $\alpha^j$ and $\alpha^k$ are additive inverse to each other. 
In this case, $(\alpha^j, \alpha^k)$ forms an AIEP over GF($p^m$). 
If $(a_{j,i}, a_{k,i})$ is a nonzero AIEP over GF($p$), then 
\begin{equation*}
\alpha^i  (a_{j,i}, a_{k,i}) \triangleq (a_{j,i}  \alpha^i, a_{k,i} \alpha^i),
\end{equation*}
is an AIEP over GF($p^m$).


Let ${\mathcal C} = \{C_1, C_2, \ldots, C_{J}\}$ be a set of $J$ AIEPs over GF($p$) 
with the \textit{USPM} structural property. 
For $1 \le j \le J$ and $0 \le i < m$, 
let $C_j = (j, p - j)$ with $j \in {\rm GF}(p) \backslash{0}$ and 
let $\psi_i(j, p - j)$ denote the AIEP $\alpha^i C_j = (j \alpha^i, (p - j) \alpha^i)$ over GF($p^m$), i.e., $\psi_i(j, p - j) = \alpha^i C_j = (j \alpha^i, (p - j) \alpha^i)$. 
Then, for $0 \le i < m$,
\begin{equation} \label{e2.5}
\Psi_i = \{\psi_i(1, p-1), \psi_i(2, p-2), \ldots, \psi_i(J, p-J) \}
\end{equation}
is a set of $J$ AIEPs over GF($p^m$) with the USPM structural property.
Hence, the Cartesian product
\begin{equation} \label{e2.6}
\Psi_i \triangleq \psi_i(1, p-1) \times \psi_i(2, p-2) \times \ldots \times \psi_i(J, p-J)
\end{equation}
of the $J$ AIEPs in $\Psi_i$ forms a $J$-user UDAIEP code over GF($p^m$) with $2^J$ codewords,
each consisting of $2^J$ nonzero elements in GF($p^m$).
With $i = 0, 1, \ldots, m-1$, we can form $m$ $J$-user UDAIEP codes over GF($p^m$),
$\Psi_0, \Psi_1, \ldots, \Psi_{m-1}$, which are \textit{mutually disjoint}, 
i.e., $\Psi_k \bigcap \Psi_i = \emptyset$ for $k \neq i$ and $0 \le k, i < m$.


If we represent an element in GF($p^m$) as an $m$-tuple over GF($p$), 
the two additive inverse elements in the pair $\psi_i(j, p - j) = (j \alpha^i, (p - j) \alpha^i)$ is a pair of two $m$-tuples with nonzero components, $j$ and $p - j$, 
in the $i$-th location, respectively, and $0$s in all the other locations, 
i.e., $(0, 0, \ldots, j, 0, \ldots, 0)$ and $(0, 0,\ldots, p - j, 0, \ldots, 0)$. 
Hence, in $m$-tuple form, all the $J$ AIEPs in $\Psi_i$ have either $j$ and $p - j$ in the $i$-th locations and $0$'s elsewhere with $1 \le j \le J$, i.e., 
%\begin{small}
\begin{equation*}
  \begin{aligned}
%\psi_i(t_1,p-t_1) &= \{(0, 0, \ldots, t_1, 0, \ldots, 0), (0, 0, \ldots, p-t_1, 0, \ldots, 0) \},\\
\psi_i(j, p-j) &= \{(0, 0, \ldots, j, 0, \ldots, 0), (0, 0, \ldots, p-j, 0, \ldots, 0) \},\\
%&\vdots\\
%\psi_i(t_J,p-t_J) &= \{(0, 0, \ldots, t_J, 0, \ldots, 0), (0, 0, \ldots, p-t_J, 0, \ldots, 0) \}.\\
  \end{aligned}
\end{equation*}
%\end{small}


From $m$-tuple (vector) point of view, $\Psi_0, \Psi_1,\ldots, \Psi_{m-1}$ are \textit{orthogonal} to each other, and they form $m$ orthogonal sets of AIEPs over GF($p$). 
Hence, $\Psi_0, \Psi_1, \ldots, \Psi_{m-1}$ give $m$ orthogonal $J$-user UDAIEP codes over GF($p^m$) (or over GF($p$) in $m$-tuple form). 
In $m$-tuple form, each codeword in $\Psi_i$ consists of $J$ $m$-tuples over GF($p$), 
each consisting of a single nonzero element from GF($p$) in the same location.


The union $\Psi \triangleq \Psi_0 \bigcup \Psi_1 \bigcup \ldots \bigcup \Psi_{m-1}$ forms a $Jm$-user orthogonal UDAIEP code over GF($p^m$) with $2^{Jm}$ codewords over GF($p^m$) (or over GF($p$) in $m$-tuple form). The code $\Psi$ has a diagonal structure as shown below:
\begin{equation} \label{e2.7}
{\Psi} = {\rm diag}({\Psi}_0, {\Psi}_1, \ldots, {\Psi}_{m-1}),
\end{equation}
which is an $m \times m$ diagonal array with ${\Psi}_0, {\Psi}_1, \ldots, {\Psi}_{m-1}$ lying on its main diagonal and zeros elsewhere. 


From (\ref{e2.7}), $\Psi$ can be viewed as a \textit{cascaded} UDAIEP code obtained by cascading the $m$ $J$-user UDAIEP codes ${\Psi}_0, {\Psi}_1, \ldots, {\Psi}_{m-1}$. 
We call ${\Psi}_0, {\Psi}_1, \ldots, {\Psi}_{m-1}$ the \textit{constituent codes} of $\Psi$. 
The UDAIEP code $\Psi$ can serve $J m$ users of a FFMA system in conjunction with TDMA which will be presented in a later section.


%The number of users in an orthogonal $J m$-user UDAIEP code over GF($p^m$) is upper bounded by $m\log_2(p-1)$. In the following, we give an example in constructing an orthogonal multiple-user UDAIEP code over an extension field of a prime field. 


\textbf{Example 3:} 
For $p = 5$ and $m = 2$, consider the extension field GF($5^2$) of the prime field GF($5$). 
As shown in Example 1, using the prime field GF($5$), two AIEPs $C_1 = (1, 4)$ and $C_2 = (2, 3)$ can be constructed whose Cartesian product $C_1 \times C_2$ is 2-user UDAIEP code over GF($5$). 
Based on this code, four UDAIEP codes over GF($5^2$) can be formed. 
They form 2 orthogonal groups,


\begin{small}
\begin{equation*}
  \begin{aligned}
    (\psi_0(C_1), \psi_0(C_2)) = (\psi_0(1, 4), \psi_0(2, 3)),
    (\psi_1(C_1), \psi_1(C_2)) = (\psi_1(1, 4), \psi_1(2, 3))\\
  \end{aligned}
\end{equation*}
\end{small}
The Cartesian products of these 2 groups give 2 orthogonal 2-user UDAIEP codes ${\Psi}_0, {\Psi}_1$ over GF($5^2$) (or over GF($5$) in $2$-tuple from). 
Their union gives an orthogonal 4-user UDAIEP code over GF($5^2$) with $2^{4}$ = 16 codewords. 
$\blacktriangle  \blacktriangle$





\subsection{Orthogonal Encoding of an Error-Correcting Code}
In a latter section, UDAIEP codes over finite fields will be used in conjunction with nonbinary error-correcting codes for error control in FFMA communication systems. 
In the following, we present an encoding of an error-correcting code over GF($p^m$) in a form to match UD encoding of multiple users.


If we represent each element $\alpha^j = a_{j,0} + a_{j,1} \alpha + \ldots + a_{j,m-1} \alpha^{(m - 1)}$ 
in GF($p^m$) by an $m$-tuple $(a_{j,0}, a_{j,1}, \ldots, a_{j,m-1})$ over GF($p$), 
the field GF($p^m$) is a vector space ${\bf V}_p(m)$ over GF($p$) of dimension $m$. 
Each vector in ${\bf V}_p(m)$ is $m$-tuple over GF($p$). 
For $0 \le i < m$, let $\epsilon_i = (0, 0, \ldots, 1, 0,\ldots, 0)$ be an $m$-tuple with a 1-component in the $i$-th location and 0s elsewhere. 
The $m$ $m$-tuples $\epsilon_0, \epsilon_1, \epsilon_2, \ldots, \epsilon_{m-1}$ form an \textit{orthogonal (or normal) basis} of ${\bf V}_p(m)$. 
Every $m$-tuple $(a_{j,0}, a_{j,1}, \ldots, a_{j,m-1})$ in ${\bf V}_p(m)$ is a linear combination of 
$\epsilon_0, \epsilon_1, \epsilon_2, \ldots, \epsilon_{m-1}$,
\begin{equation}
   (a_{j,0}, a_{j,1}, \ldots, a_{j,m-1}) = a_{j,0} \epsilon_0 + a_{j,1} \epsilon_1 + \ldots 
   + a_{j,m-1} \epsilon_{m-1},            
\end{equation}
denoted by $\bigoplus_{i=0}^{m-1} a_{j,i} \epsilon_i$, i.e.,
\begin{equation}
    \bigoplus_{i=0}^{m-1} a_{j,i} \epsilon_i = a_{j,0} \epsilon_0 + a_{j,1} \epsilon_1 + \ldots + a_{j,m-1} \epsilon_{m-1}.
\end{equation}
Hence, the $m$-tuple representation of the element 
$\alpha^j = a_{j,0} + a_{j,1} \alpha + \ldots + a_{j,m-1} \alpha^{(m - 1)}$ in GF($p^m$) is 
$a_{j,0} \epsilon_0 + a_{j,1} \epsilon_1 + \ldots + a_{j,m-1} \epsilon_{m-1}$.


For a positive integer $K$, 
let ${\bf u} = (u_0, u_1, \ldots, u_k, \ldots, u_{K-1})$ be a $K$-tuple over GF($p^m$). 
For $0 \le k < K$, let $(a_{k,0}, a_{k,1},\ldots, a_{k,m-1})$ be the $m$-tuple representation of the $k$-th component $u_k$ of ${\bf u}$. 
For $0 \le i < m$, we form the following $K$-tuple over GF($p$)
\begin{equation} \label{e2.10}
    {\bf a}_i = (a_{0,i}, a_{1,i}, \ldots, a_{k,i},\ldots, a_{K-1,i}),                           
\end{equation}
where $a_{k,i}$ is the $i$-th component of the $m$-tuple $(a_{k,0}, a_{k,1}, \ldots, a_{k,m-1})$ representation of the $k$-th component $u_k$ of $\bf u$. 
Define the following $K$ $m$-tuples over GF($p$)
\begin{equation} \label{e2.11}
    {\bf a}_i \bullet \epsilon_i \triangleq (a_{0,i} \epsilon_i, a_{1,i} \epsilon_i, \ldots, a_{k,i} \epsilon_i, \ldots, a_{K-1,i} \epsilon_i).   
\end{equation}


Then, the $K$-tuple $\bf u$ over GF($p^m$) can be decomposed into the following ordered sequence of $K$ 
$m$-tuples over GF($p$),
\begin{equation} \label{e2.12}
    [{\bf u}]_m \triangleq  
    {\bf a}_0 \bullet \epsilon_0 \oplus {\bf a}_1 \bullet \epsilon_1 \oplus \ldots \oplus 
    {\bf a}_i \bullet \epsilon_i \oplus \ldots \oplus {\bf a}_{m-1} \bullet \epsilon_{m-1},    
\end{equation}
where the $k$-th $m$-tuple is 
$a_{k,0} \epsilon_0 + a_{k,1} \epsilon_1 + \ldots + a_{k,m-1} \epsilon_{m-1}$. 
We call $[{\bf u}]_m$ the \textit{orthogonal $m$-tuple decomposition of $\bf u$}.


A sequence $\bf u$ of $K$ codewords in the $Jm$-user orthogonal UDAIEP code $\Psi$ over GF($p^m$) can be decomposed into an ordered sequence $[{\bf u}]_m$ of $J$ $m$-tuples over GF($p$) in the orthogonal form of (\ref{e2.12}). 
This orthogonal form will be used in presentation an uplink FFMA scheme in a later section.


Let $\bf G$ be the generator matrix of a $p^m$-ary $(N, K)$ linear block code $\bf W$ over GF($p^m$) with $m \ge 2$. Let ${\bf g}_0, {\bf g}_1, \ldots, {\bf g}_{K-1}$ be the $K$ rows of $\bf G$, each an $N$-tuple over GF($p^m$). 
Let ${\bf u} = (u_0, u_1, \ldots, u_k, \ldots, u_{K-1})$ be a message over GF($p^m$) whose orthogonal decomposition $[{\bf u}]_m$ is given by (\ref{e2.12}). 
If we encode this message $\bf u$ into a codeword $\bf v$ in $\bf W$ using the generator $\bf G$, 
we have
\begin{equation*} 
  {\bf v} = (v_0, v_1, v_2, \ldots, v_{N-1}) = u_0 {\bf g}_0 \oplus u_1 {\bf g}_1 \oplus \ldots 
  \oplus u_{K-1} {\bf g}_{K-1}.
\end{equation*}
The orthogonal $m$-tuple decomposition of $\bf v$ is 
\begin{small}
\begin{equation} \label{e2.13}
  \begin{aligned}
\left[{\bf v}\right]_m =& 
              \left(a_{0,0}  \epsilon_0 \oplus a_{0,1} \epsilon_1 \oplus \ldots \oplus 
                     a_{0,m-1} \epsilon_{m-1} \right) {\bf g}_0 \oplus
               \left(a_{1,0}  \epsilon_0 \oplus a_{1,1} \epsilon_1 \oplus \ldots \oplus 
                     a_{1,m-1} \epsilon_{m-1} \right) {\bf g}_1 \oplus \ldots \oplus\\
               &\left(a_{K-1,0} \epsilon_0 \oplus a_{K-1,1} \epsilon_1 \oplus \ldots \oplus 
                     a_{K-1,m-1} \epsilon_{m-1} \right) {\bf g}_{K-1} \\
            =& \left(a_{0,0} \epsilon_0 {\bf g}_0 \oplus a_{1,0} \epsilon_0 {\bf g}_1 \oplus \ldots 
                    \oplus a_{K-1,0} \epsilon_0 {\bf g}_{K-1}\right) \oplus
               \left(a_{0,1} \epsilon_1 {\bf g}_0 \oplus a_{1,1} \epsilon_1 {\bf g}_1 \oplus \ldots 
                    \oplus a_{K-1,1} \epsilon_1 {\bf g}_{K-1}\right) \oplus \ldots \oplus\\
              &\left(a_{0,m-1} \epsilon_{m-1} {\bf g}_0 \oplus a_{1,m-1} \epsilon_{m-1} {\bf g}_1 \oplus \ldots \oplus a_{K-1,m-1} \epsilon_{m-1} {\bf g}_{K-1}\right) \\
            =& ({\bf a}_0 {\bf G}) \epsilon_0 \oplus ({\bf a}_1 {\bf G}) \epsilon_1 \oplus \ldots \oplus
               ({\bf a}_{m-1} {\bf G}) \epsilon_{m-1} \\
            =& {\bf v}_0 \epsilon_0 \oplus {\bf v}_1 \epsilon_1 \oplus \ldots \oplus 
               {\bf v}_{m-1} \epsilon_{m-1},     
  \end{aligned}
\end{equation}
\end{small}
where ${\bf v}_i = {\bf a}_i {\bf G}$ is the codeword of ${\bf a}_i$ for $0 \le i < m$.
The vector $[{\bf v}]_m$ is called the codeword $\bf v$ of the message $\bf u$ in orthogonal form.
The above encoding is referred to as orthogonal encoding.





%\vspace{0.2in}
\section{An overview of FFMA}

This section gives an introduction of FFMA, which distinguishes users by using AIEPs over finite-fields.
In general, AIEPs can be viewed as virtual resources of FFMA systems, which consist of integers, elements from a prime field GF($p$) or powers of a primitive element from an extension field GF($p^m$) of a prime field GF($p$).
An FFMA system not only can be applied to the physical layer but can also be applied to the network layer, as \textit{network multiple access}.
Additionally, there are two types of FFMA, one is \textit{orthogonal FFMA (O-FFMA)}, and the other is \textit{non-orthogonal FFMA (NO-FFMA)} that can be applied to both finite-fields and complex-fields.




\subsection{Complex-Field and Finite-Field Multiple-Access Systems} 

To begin with, we compare the differences between a complex-field (CF) MA system (CFMA) and an FFMA system. 
%Suppose $J$ users are supported in each system.


In general, a CFMA system can distinguish users by using different \textit{physical resource blocks (RBs)}, e.g., time, frequency, or other types, and different users are assigned different RBs.
The basic components of a CFMA system are consisted of a source encoder, 
a physical-layer channel encoder, a modulation module, and a complex-field (CF) multiplex module determined by the assigned physical RBs. 
%The subscripts ``F'' and ``C'' indicate ``finite-field'' and ``complex-field'', respectively. 

%In an orthogonal CFMA (O-CFMA) system, an information bit, $(0)_2$ or $(1)_2$, at the output of each user is first modulated into a complex symbol and assigned to a specific orthogonal physical resource, such as a time slot, frequency, or other types, where the subscript ``$2$'' of $(0)_2$ or $(1)_2$ stands for binary. Then, the $J$ modulated orthogonal signals are multiplexed and transmitted. At the receiving end, the receiver separates the $J$ transmitted symbols and recovers the bits transmitted from the $J$ users, via the assigned physical resources.


Then, we investigate a type of FFMA system, and take \textit{binary transmission system} as an example, in which the transmit bit is either $(0)_2$ or $(1)_2$.
Two different elements in GF($q$) are required to express the bit information, 
i.e., $(0)_2 \to \alpha^{k_1}$ and $(1)_2 \to \alpha^{k_2}$, 
where $\alpha^{k_1}, \alpha^{k_2} \in$ GF($q$) and $k_1 \neq k_2$.
In Sect. II, we have constructed AIEPs (or algebraic UD code), and each AIEP has two different elements in GF($q$).
Therefore, for an FFMA system, each user is assigned an AIEP (or multiple AIEPs) as a \textit{virtual resource(s)}.
When a $J$-user UDAIEP code $\mathcal C$ is used for an FFMA system, the $j$-th component $u_j$ in a codeword $(u_1, u_2, \ldots, u_j, \ldots, u_{J})$ of $\mathcal C$ is the output symbol of the $j$-th user for $1 \le j \le J$. 
Hence, the AIEP (or algebraic UD) encoder plays a role as a type of mapping function, 
denoted by ${\rm F}_{{\rm B}2q}$, which transforms binary bit(s) into finite-field GF($q$) symbol(s). 
After ${\rm F}_{{\rm B}2q}$ transform, each bit of a user is uniquely mapped to an element in GF($q$).
After the B2$q$ mapping, we obtain $J$ elements in GF($q$), which are then passed to a \textit{finite-field (FF) multiplex module}, denoted by ${\bf G}_{\rm M}$, where ``M'' stands for ``multiplexing''.
For a given finite-field GF($q$), we can design different finite-field multiplex modules to support different number of users.


\begin{figure}[t]
  \centering
  \includegraphics[width=0.99\textwidth]{Fig_Net.pdf}
  \caption{Diagram of a downlink MA system, including both finite-field and complex-field multiplexing modules. At this moment, the FFMA is operated in network layer, and the CFMA works in physical layer.}
\end{figure}


In fact, we can jointly design FFMA and CFMA systems, and obtain a \textit{fusion MA} system.
A CFMA system and an FFMA system are not independent systems.
Fig. 3 shows a downlink MA system, including both the finite-field and complex-field multiplex modules.
At this moment, the FFMA is operated in network layer (or called network FFMA), and the CFMA works in physical layer.
The basic components of a network FFMA system consist of a source encoder, a network-layer channel encoder, an UD encoder ${\rm F}_{{\rm B}2q}$, and a finite-field multiplex module ${\bf G}_{\rm M}$.
Then, the multiplex module ${\bf G}_{\rm M}$ combines the $J$ transmitted symbols, i.e., $u_1, u_2, \ldots, u_j, \ldots, u_{J}$, and forms finite-field sum-pattern symbol(s) which contain(s) the information from $J$ users. The finite-field sum-pattern symbol(s) is (or are) then passed to the physical-layer transmission system. 


Observe the fusion MA system, it is found that channel coding can be performed in different locations.
For one hand, we can do network-layer channel coding in the finite-field GF($q$), which may be jointly designed with network coding. For another hand, a physical-layer channel code constructed in the finite-field GF($Q$) can be carried out before or after complex-field multiplexing module, where GF($Q$) may be different from GF($q$).
It is a general case to do channel coding before multiplexing.
For the uplink FFMA systems over GF($p^m$), we can also perform channel encoding after multiplexing. For this scenario, we must multiplex the users' information at special locations of the extension field, which has been discussed in Sect. II-C. In fact, this scenario is not a general manner for a MA system. 



For an FFMA system, the core issue is to establish transforms (or called mapping functions) between different fields, especially, transform from \textit{finite-field symbol(s) to complex-field signal(s)} denoted by ${\rm F}_{{\rm F2C}}$, and \textit{transform from complex-field signal(s) to finite-field symbol(s)} denoted by ${\rm F}_{{\rm C2F}}$.
If we can find suitable mapping functions ${\rm F}_{{\rm F2C}}$ and ${\rm F}_{{\rm C2F}}$, the MA design issue can be moved to the finite-field, and the physical-layer plays as a ``physical road''.
Therefore, the fusion MA is appealing for \textit{semantic communications}.
In addition, if the network FFMA system is viewed as an unified module, the transmitter and receiver of the network FFMA system can be called as \textit{equivalent source} and \textit{equivalent destination}, respectively.




%Then, the sum of these nonbinary elements is modulated into one or several signals, which is determined by the modulation order. At the receiver, after demodulation, a nonbinary element $\alpha^k$ in GF($q$) is obtained. We call $\alpha^k$ the received element. From a uniquely decodable sum-pattern table and the inverse function ${\rm F}_{q2{\rm B}}$ of ${\rm F}_{{\rm B}2q}$, the transmitted bits from the $J$ users are uniquely recovered from the received element in GF($q$). The subscript ``$q$2B'' of ${\rm F}_{q2{\rm B}}$ stands for \textit{GF($q$) to binary transform}.



%Assume that the wireless channel is noiseless.  When the receiver(s) receives the sum-pattern $\tau=\bigoplus_{j=1}^{J} u_j$, the $J$ transmitted symbols of $(u_1, u_2, \ldots, u_{J})$ can be uniquely recovered from the sum-pattern $\tau$ without ambiguity.








\subsection{Finite-field multiplex module}


This subsection investigates the finite-field multiplex module of a given finite-field GF($p$), 
where $p > 3$.
Suppose there are $J$ users.
Denote a $J$-user AIEP set by $\{C_1 = (1, p-1), C_2 =(2, p-2), \ldots, C_J = (J, p-J) \}$.
We view each $J$-tuple ${\bf u} = (u_1, u_2, \ldots, u_j, \ldots, u_J)$ in
$C_1 \times C_2 \times \ldots \times C_J$ as a $J$-user AIEP codeword, 
where ${\mathcal C} =C_1 \times C_2 \times \ldots \times C_J$ forms a $J$-user AIEP code over GF($p$) with 
$2^J$ codewords.
First, each user is assigned an AIEP, e.g., the AIEP of $C_j$ is assigned to the $j$-th user.
Then, we can obtain a $J$-tuple ${\bf u} = (u_1, u_2, \ldots, u_j, \ldots, u_J)$,
by operating ${\bf F}_{{\rm B}2q}$ to the $J$ users.
Let ${\rm G}_{{\rm M}}$ be a multiplex module, which is a $T \times J$ binary matrix, 
i.e., ${\rm G}_{{\rm M}} \in {\mathbb B}^{T \times J}$, given by
\begin{equation}
  {\rm G}_{{\rm M}} = \left[
  \begin{matrix}
    g_{1,1} & g_{1,2} & \ldots & g_{1,J}\\
    g_{2,1} & g_{2,2} & \ldots & g_{2,J}\\
    \vdots  & \vdots  & \ddots & \vdots \\
    g_{T,1} & g_{T,2} & \ldots & g_{T,J}\\
  \end{matrix}
  \right],
\end{equation}
where $1 \le T < J$. 
Then, the output multiplexed codeword ${\bf \tau}$ is
\begin{equation*}
  {\tau}^{\rm T} = {\rm G}_{{\rm M}} \cdot {\bf u}^{\rm T},
\end{equation*}
which is a $1 \times T$ symbol vector over GF($p$).
%and the subscript ``$q$'' in ${\bf u}_q$ and ${\bf v}_q$ stands for the operation in GF($q$) where $q=p$.
If each codeword ${\bf u}$ has a uniquely mapping codeword ${\bf \tau}$, 
i.e., ${\bf u} \leftrightarrow {\bf \tau}$, it is derived that $2^J \le p^T$ (or $J \le T \cdot \log_2 p$). 

When we observe ${\tau}$, there are two scenarios.
If $T=1$ and $J \le \log_2 (p-1)$, we call such a system an orthogonal FFMA (O-FFMA) system;
otherwise, if $T>1$ and $\log_2 (p-1) < J \le (p-1)/2 $, we call it a non-orthogonal FFMA (NO-FFMA) system.
The data rate of an FFMA system over finite-field GF($p$) is defined by
%for supporting $J = \frac{p-1}{2}$ users is defined by
\begin{equation}
R_p = \frac{J}{T} \le \log_2 p,
\end{equation} 
which is upper bounded by $\log_2 p$, because of $J \le T \cdot \log_2 p$.
%Because of the assumption of $J > T$, then $R_q$ is ranged in $1 < R_q \le \log_2 p$.
It is noted that the data rate of an O-FFMA system $R_p$ is equal to ${\log_2 (p-1)}$. 
Then, the data rate of the proposed NO-FFMA can be approximately upper bounded by an O-FFMA system.
Although the proposed NO-FFMA system cannot improve the spectrum efficiency (SE), it can support more users than an O-FFMA system for a given finite-field GF($p$). 
Now, we investigate these two scenarios one-by-one.


\subsubsection{Orthogonal FFMA}


An FFMA system of GF($p$) is said to be orthogonal if each user in the system is assigned to a unique AIEP in a UDAIEP-collection over GF($p$) and the number of user served by the system is equal to or smaller than $\log_2 (p-1)$, i.e., $J \le \log_2 (p-1)$.



To explain the ${\rm F}_{{\rm B}2q}$ transform and orthogonal multiplex processing, 
let us consider the 2-user UDAIEP code over GF(5) given in Example 1 which consists of two AIEPs $C_1 = (1, 4)$ and $C_2 = (2, 3)$ whose sum-patterns are given in Fig. 1 (a). 
In a 2-user O-FFMA, we assign $C_1$ and $C_2$ to users 1 and 2, respectively. 
The two output bits $(0)_2$ and $(1)_2$ of 1-user are mapped into two elements $(1)_5$ and $(4)_5$ in GF(5), respectively, where the subscript ``5'' of $(1)_5$ or $(4)_5$ stands for element in GF($5$),
i.e., $(1)_5$ and $(4)_5$ are elements 1 and 4 in GF($5$) which are additive inverse to each other.
For user-2, the two output bits $(0)_2$ and $(1)_2$ are mapped into two elements $(2)_5$ and $(3)_5$ in GF($5$), respectively, which are additive inverse to each other. 
Then, the multiplex module ${\bf G}_{\rm M}$ is simply the addition operation, 
i.e., $\tau = u_1 \oplus u_2$,
which is sent to the channel.
Based on the uniquely decodable sum-pattern table given by Fig. 1 (a), we can recover the bit information of the two users from a sum-pattern appearing in the sum-pattern table. 
For example, if the received sum-pattern of the two-users in GF(5) is $(3)_5$, 
the nonbinary elements of users 1 and 2 are $(1)_5$ and $(2)_5$, respectively, which indicate the bits from users 1 and 2 are $(0)_2$ and $(0)_2$, respectively.


Consider the FFMA system based on the most popular field GF($2^m$), 
an extension field of the binary field GF($2$). 
In this case, there is only one EP over GF($2$), not an additive inverse pair, defined by $C_0 =(0,1)$.  
Hence, the number of orthogonal UDAIEPs over GF($2^m$) is equal to $m$. 
For $0 \le i < m$, let $\psi_i(0,1)$ denote the EP $C = (0, 1)$ assigned to the $i$-th location of an $m$-tuple. If we assign $\psi_i(0, 1)$ to the $i$-th user of an $m$-user FFMA, then this FFMA is a type of TDMA in finite-field, in which the outputs of the $m$ users completely occupy $m$ locations in an $m$-tuple (similarly to the $m$ time slots).


\subsubsection{Non-orthogonal FFMA}


As aforementioned, there are totally $(p-1)/{2}$ AIEPs in a finite-field GF($p$).
Nevertheless, only $\log_2 (p-1)$ AIEPs can form UDAIEPs.
If $p > 3$, we have $\frac{p-1}{2} > \log_2 (p-1)$, 
indicating $\frac{p-1}{2} - \log_2 (p-1)$ AIEPs are unused.
In fact, the remainder $\frac{p-1}{2} - \log_2 (p-1)$ AIEPs can also be assigned to the users,
which can be viewed as a type of \textit{non-orthogonal FFMA (NO-FFMA)} system.
The NO-FFMA systems are realized by designing the \textit{multiplex module} ${\rm G}_{{\rm M}} \in {\mathbb B}^{T \times J}$, where $T > 1$ and $J \le (p-1)/2$.


There are various methods to construct the multiplex module ${\rm G}_{{\rm M}}$. 
In this paper, we present a \textit{multiple UDAIEP codes (Multi-UDC)} method, which consists of two phases:
  \begin{enumerate}
    \item
    Phase one is to find all the UDAIEP codes in the finite-field GF($p$), i.e.,
    \begin{equation*} 
     \Xi = \{ {\mathcal C}(1), {\mathcal C}(2), \ldots, {\mathcal C}(\eta_p)\},
    \end{equation*}
    where ${\mathcal C}(1), {\mathcal C}(2), \ldots, {\mathcal C}(\eta_p)$ are $\log_2 (p-1)$-user UDAIEP codes, with $\eta_p = \lfloor \frac{p-1}{2\log_2(p-1)} \rfloor$.
    \item
    Phase two is to make each output bit $\tau_t$ of $\tau$ for $1 \le t \le T$ belong to a UDAIEP code in $\Xi$,
    i.e., $\tau_t \in {\mathcal C}(t)$,
    where $\tau = (\tau_1, \tau_2,\ldots, \tau_t, \ldots, \tau_T)$ and $T \le \eta_p$.  
  \end{enumerate}


  



\textbf {Example 4:} For the finite-field GF(17), its eight AIEPs have been given in Example 2,
which provide two UDAIEP codes ${\mathcal C}(1)$ and ${\mathcal C}(2)$.
If there are $J = 8$ users, the O-FFMA (or UDAIEP code) cannot support $8$ users.
We can design a $2 \times 8$ binary multiplex module matrix ${\rm G}_{{\rm M}}$ as
\begin{equation*}
  {\rm G}_{{\rm M}} = \left[
  \begin{matrix}
    1 & 1 & 0 & 1 & 0 & 0 & 0 & 1\\
    0 & 0 & 1 & 0 & 1 & 1 & 1 & 0\\
  \end{matrix}
  \right].
\end{equation*}
Then ${\tau} = {\rm G}_{{\rm M}} \cdot {\bf u}_p^{\rm T}$ which is a $1\times 2$ vector in GF($17$).
The two components of ${\tau}$ are:
\begin{equation*}
  \begin{aligned}
  \tau_1 = u_1 \oplus u_2 \oplus u_4 \oplus u_8 \in {\mathcal C}(1), \\
  \tau_2 = u_3 \oplus u_5 \oplus u_6 \oplus u_7 \in {\mathcal C}(2), \\
  \end{aligned}
\end{equation*}
in which ${\mathcal C}(1)$ and ${\mathcal C}(2)$ have been proved to be UDAIEP codes.
Hence, the data rate of ${\bf G}_{\rm M}$ is ${R_p} = 4$, 
which is also equal to the data rate of an O-FFMA system over GF(17).
This shows that the designed ${\rm G}_{{\rm M}}$ can be used for realizing NO-FFMA, 
without data rate loss. 
For example, if ${\tau} = (\tau_1, \tau_2) = (15, 4)_{17}$, following from the Fig. 2 (a) and (b),
we find that $u_1 = 1, u_2 = 2, u_4 = 4, u_8 = 8$, and $u_3 = 3, u_5 = 5, u_6 = 6, u_7 = 7$.
$\blacktriangle \blacktriangle$


%Due to the page limitation, we only consider the NO-FFMA over GF($p$).




%\vspace{-0.3in}
\subsection{Network FFMA}  
This subsection introduces \textit{network FFMA}, which is applied FFMA into network layer.
In the following, we compare network FFMA with network coding.


\begin{figure}[t]
  \centering
 % \label{Fig3} 
  \includegraphics[width=0.75\textwidth]{3D_BN.pdf}
  \caption{Butterfly networks. (a) 2-dimensional butterfly network; 
  (b) 3-dimensional butterfly network; and (c) sum-patterns of $3$-user AIEP code over GF($7$), 
  where $C_1 =(1,6), C_2 = (2,5)$ and $C_3 = (3,4)$.}
  \vspace{-0.2in}
\end{figure}


Network coding is one of the most important networking technique, which can increase the throughput and reduce delay significantly \cite{R3}. The transmit multiple bit sequences can be encoded at the relay nodes, and the destination nodes can recover all the transmit bit sequences via algebraic decoding. 
Fig. 4 (a) shows the famous ``butterfly network'' which transmits 2-bit message from the source node to the two destination nodes.



Similarly, we construct a ``3-dimensional (3D) butterfly network'', in which the source wants to send 3-bit message through the network, as shown in Fig. 4 (b).
We apply an FFC-FFMA system over GF($7$) to the 3D butterfly network.
The AIEPs over GF($7$) are $C_1 = (1, 6), C_2 = (2, 5), C_3 = (3, 4)$, 
which can form a $3$-user AIEP code by Cartesian product, i.e., ${\mathcal C} = C_1 \times C_2 \times C_3$.
Because of $\log_2 (7-1) < 3$, it is a NO-FFMA system for supporting $3$-user over GF($7$).
Let $(u_1, u_2, u_3)$ belong to the AIEP code ${\mathcal C}$, i.e., $(u_1, u_2, u_3) \in {\mathcal C}$.
Now, we analyze the network multiple access processing.
\begin{enumerate}
  \item
  The source transmits 3-bit message to the network. 
  The $j$th transmit node, denoted by $A_j$, firstly maps 1-bit message by the function ${\rm F}_{{\rm B}2q}$ and obtains $u_j$, where $u_j \in C_j$ and $1 \le j \le 3$. 
  For example, suppose the 3-bit message is $(0, 0, 1)_2$, 
  then it can obtain $u_1 = (1)_7, u_2 = (2)_7$, and $u_3 = (4)_7$. 
  \item
  When $u_j$ for $1 \le j \le 3$ arrives at the relay node $R_1$, we can obtain the sum-pattern 
  $\tau = \bigoplus_{j=1}^3 u_j$, which is then transmitted to the relay node $R_2$. 
  Consider the above example, it is derived that $\tau = (1 + 2 + 4)_7 = 0$.
  According to Theorem 1, the sum-pattern $\tau=0$ may result in ambiguity, thus it is a NO-FFMA system.
  \item
  At the destination node $D_j$, based on the sum-pattern $\tau$ and information $u_j$, 
  it can recover the transmit 3-bit from the source node. 
  Take the $1^{st}$ destination node $D_1$ as an example, $D_1$ receives $u_1 = (1)_7$ and $\tau=0$.
  By the sum-pattern table, it is found that both $(u_1, u_2, u_3) = (1, 2, 4)_7$ and $(u_1, u_2, u_3) = (6, 5, 3)_7$ can make the sum-pattern $\tau$ equal to $0$. 
  Regarding as $u_1 = (1)_7$, it can know $(u_1, u_2, u_3) = (1, 2, 4)_7$, 
  thus the transmit 3-bit message is $(0, 0, 1)_2$. 
\end{enumerate}
By using the proposed FFMA systems, we can extend the current theory of network coding.
Due to the page limitation, more contents will be introduced in our future paper.
%It is interesting to investigate the FFMA systems for network layer.





%\vspace{0.2in}
\section{A downlink O-FFMA over GF($p$) system for Frequency-Selective Fading channels}

This section presents a downlink O-FFMA system of a prime field GF($p$), $p > 2$,
for frequency-selective fading channels. 
Suppose the system is to support $J$ users with $J \le \lfloor \log_2(p-1) \rfloor$. 
For such an O-FFMA system, we choose a $J$-user UDAIEP code $\mathcal C$ over GF($p$) with $J$ constituent AIEP codes $C_1, C_2, \ldots, C_j, \ldots, C_{J}$ as the multiple access resources. 
The $j$-th AIEP $C_j$ is assigned to the $j$-th user. 
A block diagram for such a downlink $J$-user FFMA system is shown in Fig. 5.



\vspace{-0.1in}
\begin{figure}[t]
  \centering
  \label{Fig_DL}
  \includegraphics[width=0.99\textwidth]{Fig_DL.pdf}
  \caption{System model of a downlink O-FFMA system for frequency-selective fading channels, 
  where ${\rm F}_{{\rm B}2q}$ and ${\rm F}_{q2{\rm B}}$ stand for ``binary to finite-field GF($q$) transform'' and ``finite-field GF($q$) to binary transform''; 
  ${\rm F}_{q2Q}$ and ${\rm F}_{Q2q}$ are ``finite-field GF($q$) to finite-field GF($Q$) transform'' and ``finite-field GF($Q$) to finite-field GF($q$) transform''; 
  $\rm F_{F2C}$ and $\rm F_{C2F}$ stand for ``finite-field to complex-field transform'' and ``complex-field to finite-field transform''.}
  \vspace{-0.2in}
\end{figure}


\subsection{Transmitter}
 
Let $K$ be a positive integer and ${\bf b}_j = (b_{j,0}, b_{j,1},\ldots, b_{j, k}, \ldots, b_{j, K})$ be the bit-sequence at the output of user-$j$, where $b_{j,k} \in {\rm GF}(2)$ and $0 \le k < K$. 
In transmission, each bit $b_{j,k}$ in ${\bf b}_j$ is mapped into an element $u_{j,k}$ in the $j$-th constituent UDAIEP $C_j$ assigned to the $j$-th user by 
a binary to GF($q$) transform function (B$2q$) 
${\rm F}_{{\rm B}2q}$, i.e., $u_{j,k} = {\rm F}_{{\rm B}2q}(b_{j,k})$. 
The B2$q$ transform function maps the output bit-sequences of the $J$ users into $J$ sequences, 
i.e., ${\bf u}_j = (u_{j,0}, u_{j,1},\ldots, u_{j, k}, \ldots, u_{j, K-1})$ over GF($p$) 
where $1 \le j \le J$. 
For $0 \le k < K$, the $J$-tuple $(u_{1,k}, u_{2,k}, \ldots, u_{J,k})$ is a codeword in the $J$-user UDAIEP code ${\mathcal C} = C_1 \times C_2 \times \ldots \times C_{J}$.



Next, we multiplex the $J$ sequences, 
${\bf u}_1, {\bf u}_2, \ldots, {\bf u}_{J}$ 
into a sequence ${\bf \tau} = (\tau_0, \tau_1, \ldots, \tau_k, \ldots, \tau_{K-1})$ over GF($p$) by addition operation, 
where $\tau_k = \bigoplus_{j=1}^{J} u_{j, k}$ and $\tau_k \in {\rm GF}(p) \backslash \{0\}$. 
The multiplexed sequence $\tau$ is a sequence of sum-patterns of the $J$-user UDAIEP code $\mathcal C$ over GF($p)$. 
We call $\tau$ as the sum-pattern sequence of ${\bf u}_1, {\bf u}_2, \ldots, {\bf u}_{J}$. 
Since $\mathcal C$ is a uniquely decodable code, 
the $J$ sequences ${\bf u}_1, {\bf u}_2, \ldots, {\bf u}_{J}$ can be uniquely separated from $\bf u$ symbol by symbol without ambiguity. 
Through the inverse function, denoted by ${\rm F}_{q2{\rm B}}$, of ${\rm F}_{{\rm B}2q}$, 
we can recover the output bit-sequences ${\bf b}_1, {\bf b}_2, \ldots, {\bf b}_{J}$ of the $J$ users from ${\bf u}_1, {\bf u}_2, \ldots, {\bf u}_J$ without ambiguity.



Suppose an $(N, K)$ linear block code $\bf W$ over GF($Q$) of length $N$ and dimension $K$ with generator matrix $\bf G$ is used for error control, where $Q$ may or may not equal to $p$. 
Each component $\tau_k$ in the sequence ${\tau} = (\tau_0, \tau_1, \ldots, \tau_k, \ldots, \tau_{K-1})$ is uniquely mapped into an element $w_k$ in GF($Q$) by a GF($p$) to GF($Q$) transform function ${\rm F}_{q2Q}$, i.e.,  $w_k = {\rm F}_{q2Q}(\tau_k)$. 
%The mapping results in a sequence ${\bf w} = (w_0, w_1, \ldots, w_k, \ldots, w_{K-1})$ over GF($Q$).
%Next, ${\bf w}$ is encoded into a codeword ${\bf v} = {\bf w} {\bf G}$ in $\bf W$. 
%Then, $\bf v$ is modulated and transmitted as shown in Fig. 4 (a). 
%The encoding allows all $J$ users to achieve the same frequency diversity gain.

In general, the $q2Q$ transform function ${\rm F}_{q2Q}$ is determined by the relationship between GF($p$) and GF($Q$). One case is given as follows. 
Let $m = \lceil \log_2(p-1) \rceil$ and $Q = 2^m$. 
Next, we construct the extension field GF($2^m$) of the binary field GF(2). 
Find a $p$ to $2^m$ transform function ${\rm F}_{p2Q}$ which maps each non-zero element $\tau_k$ in GF($p$) into an element $w_k$ in GF($2^m$) whose binary $m$-tuple representation is $(w_{k,0}, w_{k,1}, \ldots, w_{k,m-1})$. 
With this mapping, the UDAIEP coded sequence ${\tau} = (\tau_0, \tau_1, \ldots, \tau_k, \ldots, \tau_{K-1})$ over GF($p$) for the $J$ users, with $\tau_k = \bigoplus_{j=1}^{J} u_{j,k}$, 
is transformed into a sequence ${\bf w} = (w_0, w_1, \ldots, w_k, \ldots, w_{K-1})$ over GF($2^m$) in which the $m$-tuple representation of the $k$-th component $w_k$ is $(w_{k,0}, w_{k,1}, \ldots, w_{k,m-1})$. 
%Decompose the sequence $\bf w$ into $m$ constituent sequence ${\bf w}_0, {\bf w}_1, \ldots, {\bf w}_{m-1}$, with ${\bf w}_i = (w_{0,i}, w_{1,i}, \ldots,\\ w_{k,i}, \ldots, w_{K-1,i})$ over GF(2) for $0 \le i < m$. 
%Let $\bf G$ be the generator matrix of an $(N, K)$ linear code $\bf W$ over GF($2^m$). 
Then, we encode the sequence $\bf w$ into a codeword
%\vspace{-0.1in}
\begin{equation*}
  {\bf v} = (v_0, v_1, \ldots, v_n, \ldots, v_{N - 1}) = {\bf w} {\bf G}.
\end{equation*}
%in $\bf W$ in orthogonal form given by (\ref{e2.12}) and (\ref{e2.13}).


If $p - 1 = 2^m$, every nonzero element in GF($p$) can be uniquely mapped into an element in GF($2^m$) $(Q = 2^m)$ in a simple way. Let $\alpha$ be a primitive element in GF($2^m$) and $\tau_k$ be a nonzero sum-pattern. Then, $\tau_k$ can be mapped into an element in GF($2^m$) as follows:
\begin{equation} \label{e4.1}
w_k = {\rm F}_{q2Q}(\tau_k) = \alpha^{\tau_k-2},
\end{equation}
with $\alpha^{-1}$ denoting the zero element $0$ of GF($2^m$), i.e., ${\alpha}^{-1} = 0$.



\textbf{Example 5:} 
Consider the prime field GF($17$). 
As shown in Example 1, 
based on this field, a 4-user UDAIEP code $\mathcal C(1)$ (or $\mathcal C(2)$) can be constructed. 
This code can be used in a downlink FFMA to support $4$ users. 
Set $m = \log_2 (17 - 1) = 4$ and $Q = 2^4$. 
Construct the field GF($2^4$) based on the primitive polynomial ${\bf g}(X) = 1 + X + X^4$. 
Let $\alpha$ be a primitive element in GF($2^4$). 
Then, the mappings of $16$ nonzero sum-patterns in $\mathcal C$ to the 16 elements in GF($2^4$) are: 
${\rm F}_{q2Q}(j) = \alpha^{j - 2}$, with $1 \le j \le 16$.


Note that $257$ is a prime. 
Using the prime field GF($257$), we can construct an $8$-user UDAIEP code over GF($257$) which can be used in a downlink O-FFMA system to support $8$ users. 
Set $m = 8$. 
Then, the transform function ${\rm F}_{q2Q}(j) = \alpha^{j - 2}$ for $1 \le j \le 256$, 
uniquely maps a nonzero element in GF($257$) into an element in GF($2^8$).
$\blacktriangle  \blacktriangle$


To process the codeword ${\bf v} = (v_0, v_1,\ldots, v_n, \ldots, v_{N - 1})$  at the output of the encoder for transmission, we first expand each symbol $v_n$ for $0 \le n < N$ in $\bf v$ into an $m$-tuple  
$(v_{n,0}, v_{n,1}, \ldots, v_{n, m-1})$ over GF(2). 
This expansion results in a codeword ${\bf v}_{exp}$ over GF(2) which consists of $N$ $m$-tuples, 
a sequence of $mN$ bits. 
Then, ${\bf v}_{exp}$ is sequentially (bit by bit) modulated with BPSK into a complex-field vector ${\bf s} \in {\mathbb C}^{1 \times L}$, where $L = mN$. 
Since each code symbol $v_n$ in the codeword $\bf v$ is expanded into an $m$-tuple and then modulated, 
the modulation can be regarded as a kind of transform $\rm F_{F2C}$ from finite-field GF($2^m$) to complex-field $\mathbb C$.


Next, the modulated sequence ${\bf s}$ is arranged into an $N_c \times M$ matrix $\bf S$ in complex-field frequency-domain, where $N_c$ is the number of \textit{sub-carrier} and $M = \lceil \frac{L}{N_c} \rceil$.
The $m_c$-th column, $1 \le m_c \le M$, of ${\bf S}$ is defined by ${\bf S}_{m_c} \in {\mathbb C}^{N_c \times 1}$. Each column of ${\bf S}$ consists of $N_c$ symbols which form a \textit{block}. 
The $M$ blocks of $\bf S$ will be processed and transmitted block-by-block. 
The transformation from $\bf s$ to $\bf S$ is referred to as serial to parallel (S/P) transformation.


Performing the \textit{inverse discrete Fourier transform (IDFT)} of each block ${\bf S}_{m_c}$ in ${\bf S}$, we obtain an $N_c \times M$ matrix $\bf X$ in complex-field time-domain whose $m_c$-th block (or column) is given as ${\bf X}_{m_c} = {\rm IDFT}[{\bf S}_{m_c}]$ for $1 \le m_c \le M$. 
After adding a \textit{cyclic prefix (CP)} to each block ${\bf X}_{m_c}$ of $\bf X$, 
the signals in $\bf X$ are then transmitted through different frequency-selective fading channels to $J$ users via multiple paths with different delays. 
The length $N_g$ of the inserted CP should be larger than the maximum time delay of the fading channels. 
The time-domain (TD) channel impulse response between the BS and the $j$-th user, 
$1 \le j \le J$, over ${\mathcal L}$ $({\mathcal L} < N_c)$ multi-paths is defined as
\begin{equation}
{\bf h}_{{\rm TD},j}  = \sum_{\ell = 1}^{\mathcal L} h_{j,\ell} \cdot \delta(t- \tau_{j,\ell}),
\end{equation}
where $h_{j,\ell}$ and $\tau_{j,\ell}$ are the channel gain and time-delay of the $\ell$th path to 
the $j$-th user, respectively. 
The channel impulse response ${\bf h}_{{\rm TD}, j}$ can be expressed by a $1 \times N_c$ vector, 
where the $\tau_{j,\ell}$-th location has value $h_{j,\ell}$ and the other locations are all zeros.
If $\tau_{j,\ell} = \ell T_s$, then the time-domain channel vector of ${\bf h}_{TD, j}$ is
%\begin{equation}
${\bf h}_{{\rm TD},j} = (h_{j,1}, h_{j,2},\ldots, h_{j,\ell}, \ldots, h_{j,\mathcal L}, 0, \ldots, 0) 
              \in {\mathbb C}^{1 \times N_c}$,
%\end{equation}
where $T_s$ is the symbol duration. 
The frequency-domain channel vector of ${\bf h}_{{\rm TD}, j}$ is 
%\begin{equation}
${\bf H}_{{\rm FD},j} = {\rm DFT}[{\bf h}_{{\rm TD},j}] = (H_{j,1}, H_{j,2},\ldots, H_{j, n_c}, \ldots, H_{j, N_c}) \in {\mathbb C}^{1 \times N_c}$,
%\end{equation}
where the subscript ``FD'' of ${\bf H}_{{\rm FD},j}$ stands for ``frequency-domain''.



\subsection{Receiver}

At the receiver of the $j$-user for $1 \le j \le J$ (see Fig. 4 (b)), 
we first remove the inserted CP from each block, 
and let the received sequence of $M$ blocks of signals in complex-field frequency-domain be
%\begin{equation} \label{e4.5}
 $ {\bf Y}_j = ({\bf Y}_{j,1}, {\bf Y}_{j,2}, \ldots, {\bf Y}_{j,m_c},\ldots, {\bf Y}_{j,M})$
%\end{equation}
with
\begin{equation}
{\bf Y}_{j,m_c} = {\bf H}_{{\rm FD},j}^{\rm T} \circ {\bf X}_{m_c} + {\bf Z}_{j,m_c},
\end{equation}
where $\circ$ is the \textit{Hadamard product}, 
and ${\bf Y}_{j,m_c} \in \mathbb{C}^{N_c \times 1}$ for $1 \le m_c \le M$. 
${\bf Z}_{j,m_c} \in \mathbb{C}^{N_c \times 1}$ is the additive white Gaussian noise (AWGN) 
with distribution ${\mathcal N}(0, N_0/2)$.


Then, each received block ${\bf Y}_{j,m_c}$ in ${\bf Y}_j$ is processed by a \textit{frequency Fourier transform (FFT)} unit and a \textit{frequency-domain equalization (FDE)} unit to produce a block of signals,
\begin{equation}
\hat {\bf Y}_{j,m_c} = {\bf w}_{{\rm FD},j,m_c} \circ ({\bf H}_{{\rm FD},j}^{\rm T} \circ {\bf X}) 
+ {\bf Z}_{j,m_c},
\end{equation}
where ${\bf w}_{{\rm FD},j,m_c}$ is the FDE weight,
which can be computed by the \textit{minimum mean square error (MMSE)} criterion.
The sequence of processed blocks 
%\begin{equation*}
$\hat {\bf Y}_j = (\hat{\bf Y}_{j,0}, \hat{\bf Y}_{j,1}, \ldots, \hat{\bf Y}_{j,m_c}, \ldots, \hat{\bf Y}_{j,N-1})$,
%\end{equation*}
is then transformed into a vector ${\bf y}_j \in {\mathbb C}^{1 \times L}$ which consists of $N$ $m$-tuples over the complex field ${\mathbb C}$, i.e., 
${\bf y}_j = (y_{j,0}, y_{j,1}, \ldots, y_{j,n}, \ldots, y_{j,N-1})$ for $0 \le n < N$,
with $y_{j,n}$ as an $m$-tuple 
$y_{j,n} = (y_{j,n,0}, y_{j,n,1}, \ldots, y_{j,n,i},\ldots,\\ y_{j,n,m-1})$.
Then, decoding can be performed based on ${\bf y}_j$ and the parity-check matrix $\bf H$ of the error control code $\bf W$. 



If $\bf W$ is a $q$-ary LDPC code, i.e., $q = 2^m$, 
we can decode ${\bf y}_j$ iteratively with a Fast-Fourier-Transform $q$-ary sum-product algorithm (FFT-QSPA) based on the probability mass function (pmf).
Hence, the obtained bit-wise probabilities, 
i.e., ${\rm Pr}(w_{j,n,i}=0|y_{j,n,i})$ and ${\rm Pr}(w_{j,n,i}=1|y_{j,n,i})$, are then converted to symbol-wise probabilities by computing appropriate products.


After FFT-QSPA operation, we can obtain the decoded vector 
$\hat {\bf w} =(\hat w_{0}, \hat w_{1}, \ldots, \hat w_{k}, \ldots, \hat w_{K-1})$.
Through ${\rm F}_{Q2q}$, the corresponding non-zero element $\hat \tau_{k}$ can be obtained as
\begin{equation*}
  \hat \tau_{k} = {\rm F}_{Q2q}(\hat w_{k})  = \log_{\alpha}{\hat w_{k}} + 2,
\end{equation*}
where $\hat w_{k}$ is expressed as the power of $\alpha$ and $\hat \tau_{k} \in \text{GF}(p)\backslash 0$.
Then, the decoded sequence of sum-patterns is given as
%\begin{equation*}
$\hat \tau = (\hat {\tau}_{0}, \hat {\tau}_{1}, \ldots, \hat {\tau}_{k}, \ldots, \hat {\tau}_{K-1})$
%\end{equation*}
over GF($p$). 
%of the $J$-user UDAIEP code $\mathcal C$. 
According to the sum-pattern table and $\hat \tau$, we can uniquely extract the output symbol sequence $\hat {\bf u}_j$ of the $j$-th user, and then obtain the bit sequence $\hat {\bf b}_j$.
If decoding $\hat {\bf y}_j$ is successful, $\hat {\bf b}_j$ is error-free.



%\vspace{-0.3in}
\section{An uplink O-FFMA over GF($2^m$) System for a GMAC}


This section presents an uplink O-FFMA system for a GMAC based on the extension field GF($2^m$) of the binary field GF(2). Since the base field is GF(2), it has only one EP $C_0 =(0, 1)$ referred as the base EP. 
Using this EP, we can construct a set of $m$ orthogonal UDAIEPs over GF($2^m$), 
\begin{equation*}
  \Psi = \{\psi_0(0,1), \psi_1(0,1), \ldots, \psi_i(0,1), \ldots, \psi_{m-1}(0,1) \}.
\end{equation*}


Suppose the system is designed to support $J$ users with $1 \le J \le m$. 
The UDAIEP $\psi_{j-1}(0,1)$ is assigned to the $j$-th user. 
A block diagram of the designed uplink FFMA system is shown in Fig. 5. 
%In the system, BPSK is used for modulation.

\begin{figure}[t]
  \centering
  \includegraphics[width=0.95\textwidth]{Fig_UL.pdf}
  \caption{System model of an uplink FFMA system of GF($2^m$) in a GMAC,
  where ${\rm F}_{{\rm B}2q}$ and ${\rm F}_{q2{\rm B}}$ stand for ``binary to finite-field GF($q$) transform'' and ``finite-field GF($q$) to binary transform''; 
  ${\rm F}_{q2Q}$ and ${\rm F}_{Q2q}$ are ``finite-field GF($q$) to finite-field GF($Q$) transform'' and ``finite-field GF($Q$) to finite-field GF($q$) transform''; 
  $\rm F_{F2C}$ and $\rm F_{C2F}$ stand for ``finite-field to complex-field transform'' and ``complex-field to finite-field transform''.}
  \vspace{-0.2in}
\end{figure}

\vspace{-0.1in}
\subsection{Transmitter}
Let $K$ be a positive integer and ${\bf b}_j = (b_{j,0}, b_{j,1},\ldots, b_{j,k}, \ldots, b_{j,K-1})$ be the bit-sequence at the output of the $j$-th user. 
The transmitter first maps the bit-sequence ${\bf b}_j$ uniquely into a sequence 
${\bf u}_j = (u_{j,0},u_{j,1},\ldots, u_{j,k},\ldots, u_{j,K-1})$ over GF($2^m$) by the transform function ${\rm F}_{{\rm B}2q}$, 
where $u_{j,k} = {\rm F}_{{\rm B}2q}(b_{j,k})$ which is determined by the EP $\psi_{j-1}(0,1)$ that assigned to the $j$-th user. 
For $0 \le k < K$, express the $k$-th component of ${\bf u}_j$ into an $m$-tuple 
$u_{j,k} =(u_{j,k,0}, u_{j,k,1},\ldots, u_{j,k,i},\ldots, u_{j,k,m-1})$. 
Since the $j$th user utilizes the UDAIEP $\psi_{j-1}(0,1)$, i.e.,
  ${u}_{j,k} = b_{j,k} \cdot \psi_{j-1}(0,1) = (0,\ldots, 0, b_{j,k}, 0,\ldots, 0)$, 
the $i$-th symbol $u_{j,k,i}$ of $u_{j,k}$ is 
\begin{equation} \label{e.u_j}
u_{j,k,i} =
\left\{
  \begin{matrix}
    b_{j,k}, & i = j-1\\
    0,     & i \neq j-1 \\
  \end{matrix}. \right. 
\end{equation}
In the next step, the sequence ${\bf u}_j$ is encoded into a codeword ${\bf v}_j$ of an $(N, K)$ linear block code $\bf W$ over GF($2^m$) specified by a $K \times N$ generator matrix $\bf G$. 
The codeword ${\bf v}_j$ for ${\bf u}_j$ is
\begin{equation} \label{e5.3}
   {\bf v}_j = {\bf u}_j {\bf G} = (v_{j,0}, v_{j,1},\ldots, v_{j,n}, \ldots, v_{j,N-1}).      
\end{equation}
Express each code symbol $v_{j,n}$ for $0 \le n < N$ in ${\bf v}_j$ into an $m$-tuple over GF(2),
%\begin{equation*}
    $v_{j,n} = (v_{j,n,0}, v_{j,n,1},\ldots,\\ v_{j,n,i}, \ldots, v_{j,n,m-1})$.
%\end{equation*}
With BPSK signaling, ${\bf v}_j$ is modulated and mapped to a complex vector 
${\bf x}_j \in {\mathbb C}^{1 \times (N \cdot m)}$, 
where ${\bf x}_j = (x_{j,0}, x_{j,1}, \ldots, x_{j,n}, \ldots, x_{j,N-1})$ with $x_{j,n} = (x_{j,n,0}, x_{j,n,1},\ldots, x_{j,n,i}, \ldots, x_{j,n,m-1})$. 
For $0 \le i < m$, the $i$-th component of $x_{j,n}$ is given by
\begin{equation}  \label{e.x_j}
{x}_{j,n,i} = 2 {v}_{j,n,i} - 1,
\end{equation}
where $x_{j,n,i} \in {\mathbb C}$. 
The mapping from ${\bf v}_j$ to ${\bf x}_j$ is regarded as 
\textit{finite-field to complex-field transform}, 
denoted by $\rm F_{F2C}$. Then ${\bf x}_j$ is sent to a GMAC. 



Note that if the channel code $\bf W$ is a code over GF($Q$) which is different from GF($2^m$), 
a transform function ${\rm F}_{q2Q}$ is needed to match the two finite fields GF($2^m$) and GF($Q$).

\vspace{-0.1in}
\subsection{Receiver}
For presenting the receiver and its process, 
we define ${\bf \tau} = \bigoplus_{j=1}^J {\bf u}_j$ and ${\bf v} = \bigoplus_{j=1}^J {\bf v}_j$ as an information sum-pattern vector and a codeword sum-pattern vector over GF($2^m$) of the $J$ users 
where ${\tau} = (\tau_0, \tau_1, \ldots, \tau_{k}, \ldots, \tau_{K-1})$ with $\tau_k = \bigoplus_{j=1}^J u_{j,k}$ 
and ${\bf v} = (v_0, v_1,\ldots, v_n,\ldots, v_{N-1})$ with $v_n = \bigoplus_{j=1}^J v_{j,n}$. 
It follows from (\ref{e2.12}) and (\ref{e2.13}) that $\bf v = \tau \cdot G$ in orthogonal form.



At the receiving end, the received vector ${\bf y} \in {\mathbb C}^{1 \times (m \cdot N)}$ is the complex-field sum-pattern of the outputs of the $J$ users plus noise which is given by
\begin{equation} 
{\bf y} = \sum_{j=1}^{J} {\bf x}_j + {\bf z} = {\bf r} + {\bf z},
\end{equation}
where ${\bf z} \in \mathbb{C}^{1 \times (m \cdot N)}$ is an AWGN vector
with ${\mathcal N}(0, N_0/2)$,
and ${\bf r} \in \mathbb{C}^{1 \times (m \cdot N)}$ is a complex-field sum-pattern of the modulated sequences ${\bf x}_1, {\bf x}_2,\ldots, {\bf x}_J$.
The vector ${\bf r} = (r_0, r_1, \ldots, r_n, \ldots, r_{N-1})$ consists of $N$ $m$-tuples with $r_n = (r_{n,0}, r_{n,1}, \ldots, r_{n,i}, \ldots, r_{n,m-1})$ and
\begin{equation} \label{e5.7}
   r_{n,i} = \sum_{j=1}^J x_{j,n,i} = 2 \sum_{j=1}^J v_{j,n,i} - J.              
\end{equation}
The first step of the decoding process is to transform the received vector $\bf y$ into a vector 
\begin{equation} \label{e5.8}
  \hat{{\bf y}} = \rm F_{C2F}({\bf r}) + {\bf z},
\end{equation}
over GF($2^m$) by a \textit{complex-field to finite-field (C2F)} transform function $\rm F_{C2F}$, 
where ${\rm F_{C2F}}({\bf r}) = {\bf v} = (v_0, v_1,\ldots, v_n, \ldots, v_{N-1})$ 
which consists of $N$ $m$-tuples. 
The $n$-th component $v_n$ in $\bf v$ is an $m$-tuple, i.e., 
$v_n = (v_{n,0}, v_{n,1},\ldots, v_{n,i}, \ldots, v_{n,m-1})$, 
given by
\begin{equation}  \label{e5.9}
  v_{n,i} = {\rm F_{C2F}}(r_{n,i}) = \bigoplus_{j=1}^J v_{j,n,i}.  
\end{equation}                    
Note that it is important to find the function $\rm F_{C2F}$, otherwise, the system is ineffective.
%The decoding is based on the relationship between the complex-field sum-pattern $r_{n,i}$ and the finite-field sum-pattern $v_{n,i}$, the transform functions and unique decidability of the orthogonal UDAIEPs assigned to the users. 
Fig. 7 shows the relationship between the complex-field sum-pattern $r_{n,i}$ and finite-field sum-pattern $v_{n,i}$, where each user utilizes BPSK, and the numbers of users are set to be $J = 2, 3, 4$. 
From (\ref{e5.7}) and Fig. 7, we find the following facts:
\begin{enumerate}
\item
The value of $r_{n,i}$ is determined by the number of users who send ``+1'' and the number of users who send ``-1''. Thus, the maximum and minimum values of $r_{n,i}$ are respectively $J$ and $-J$. The set of $r_{n,i}$'s values in ascendant order is $\Omega = \{-J, -J+2, \ldots, J-2, J\}$, 
in which the difference between two adjacent values is $2$. The total number of $\Omega$ is equal to $|\Omega| = J+1$. 
\item
Since $v_{n,i} \in {\rm GF}(2)$, the possible values of $v_{n,i}$ are $(0)_2$ and $(1)_2$. 
Then, $v_{n,i}$ is uniquely determined by the number of $(1)_2$ bits coming from the $J$ users, 
i.e., $v_{1,n,i}, v_{2,n,i}, \ldots, v_{J,n,i}$. 
If there are odd number of bits $(1)_2$ in the set $\{v_{1,n,i}, v_{2,n,i}, \ldots, v_{J,n,i}\}$, 
then $v_{n,i} = (1)_2$; otherwise, $v_{n,i} = (0)_2$. 
Since the values of $r_{n,i}$ are arranged in ascendant order, 
the corresponding binary set $\Omega_v$ of $\Omega$ is $\{0, 1, 0, 1, \ldots\}$, 
in which $(0)_2$ and $(1)_2$ appear alternatively. 
The number of $|\Omega_v|$ is also equal to $J+1$, i.e., $|\Omega_v| = |\Omega|$.
\item
Let $C_J^\iota$ denote the number of users which send ``+1''. 
The values of $\iota$ are from $0$ to $J$. 
When $\iota = 0$, it indicates that all the $J$ users send $(0)_2$, 
i.e., $v_{j,n,i} = (0)_2$ and $x_{j,n,i} = -1$ for all $1 \le j \le J$, 
thus, $v_{n,i} = \bigoplus_{j=1}^J v_{j,n,i} = (0)_2$. 
If $\iota$ increases by one, the number of $(1)_2$ bits coming from $J$ users increases by one accordingly. Therefore, the difference between two adjacent values is $2$, and the bits $(0)_2$ and $(1)_2$ appear alternatively.
\end{enumerate}


\vspace{-0.1in}
\begin{figure}[t] 
  \centering
  \label{Fig7}
  \includegraphics[width=0.55\textwidth]{Fig5_GF2.pdf}
 \caption{The relationship between complex-field sum-pattern $r_{n,i}$ and finite-field sum-pattern $v_{n,i}$ of an uplink FFMA system of GF($2^m$), where each user uses BPSK and the numbers of users are set to be $J=2, 3, 4$.}
 \vspace{-0.1in}
\end{figure}

Based on the above facts, the transform function $\rm F_{C2F}$ maps each received symbol $r_{n,i}$ to a uniquely sum-pattern element $v_{n,i}$, i.e., ${\rm F_{C2F}}: r_{n,i} \to v_{n,i}$. 
Since $r_{n,i} \in \{-J, -J+2, \ldots, J-2, J\}$ and $v_{n,i} \in \{0,1\}$, 
$\rm F_{C2F}$ is a many-to-one mapping function.


\textbf{Example 6:}
If $J=2$, we have $\Omega =\{-2, 0, 2\}$, and $\Omega_v = \{0, 1, 0\}$.
If $J=3$, then $\Omega = \{-3, -1, 1, 3\}$, and $\Omega_v = \{0, 1, 0, 1\}$.
If $J=4$, we have $\Omega =\{-4, -2, 0, 2, 4\}$, and $\Omega_v = \{0, 1, 0, 1, 0\}$.
$\blacktriangle \blacktriangle$


Now, we consider the probabilities used for decoding $\hat {\bf y} = {\rm F_{C2F}}({\bf r}) + {\bf z}$. 
Once $\hat {\bf y}$ is formed, 
it is decoded based on a chosen decoding method for the channel code $\bf W$. 
%If decoding is successful, $\hat {\bf y}$ is decoded into the transmitted vector $\bf v$. 
Then, the transforms ${\rm F}_{Q2q}$ and ${\rm F}_{q2{\rm B}}$ 
(the inverse transforms of ${\rm F}_{q2Q}$ and ${\rm F}_{{\rm B}2q}$ performed in the transmitting end), 
are applied to $\bf v$ to recover the bit sequences 
${\bf b}_1, {\bf b}_2,\ldots, {\bf b}_J$ 
transmitted by the $J$ users based on the orthogonal UDAIEPs assigned to the $J$ users.


Assume that $v_{n,i} = (0)_2$ and $v_{n,i} = (1)_2$ are equally likely, 
i.e., ${\rm Pr}(v_{n,i}=0) = {\rm Pr}(v_{n,i} = 1) = 0.5$. 
Then, the probabilities of the elements in $\Omega$ are
%\vspace{-0.1in}
%\begin{equation} \label{e5.10}
   ${\mathcal P}_r = \left\{ {C_J^0}/{2^J}, {C_J^{1}}/{2^J},\ldots, {C_J^{J-1}}/{2^J}, {C_J^J}/{2^J} \right\}$.
%\end{equation}


Based on the relationship between $r_{n,i}$ and $v_{n,i}$, 
we can calculate the posterior probabilities of the received signals. 
The conditional probability $v_{n,i}$ given by $y_{n,i}$ is

\vspace{-0.1in}
\begin{small}
\begin{equation} \label{e5.11}
  {\rm Pr}(v_{n,i}|y_{n,i}) = \frac{{\rm Pr}(v_{n,i})p(y_{n,i}|v_{n,i})}{p(y_{n,i})} ,                   
\end{equation}
\end{small}
where $p(y_{n,i})$ is the probability of $y_{n,i}$.

Since $y_{n,i}$ is determined by $r_{n,i}$ that is selected from the set $\Omega$, thus,

\begin{small}
\begin{equation} \label{e_MAP}
  \begin{aligned}
  p(y_{n,i}) %&= \sum_{\iota=0}^{J} \text{Pr}(c_{\iota}) p(c_{\iota}),\\
         &= \sum_{\iota=0}^{J} {\mathcal P}_r(\iota) \cdot 
         \frac{1}{\sqrt{\pi N_0}} \exp \left\{
         - \frac{\left[y_{n,i} - \Omega(\iota) \right]^2}{N_0} 
         \right\},  \\
  \end{aligned}
\end{equation}
\end{small}
where ${\mathcal P}_r(\iota)$ and $\Omega(\iota)$ stand for the $\iota$th element of the sets ${\mathcal P}_r$ and $\Omega$, respectively. 

When $v_{n,i} = (0)_2$, the corresponding $r_{n,i}$ equals to $\{-J, -J+4, -J+8, \ldots\}$. 
Thus, the posteriori probability of $v_{n,i} = (0)_2$ is

\begin{small}
\begin{equation}
  {\rm Pr}(v_{n,i}=0|y_{n,i}) = \frac{1}{{p(y_{n,i})}} 
  \sum_{\iota=0,\iota+2}^{\iota \le J} {\mathcal P}_r(\iota)\cdot \frac{1}{\sqrt{\pi N_0}}
  \exp \left\{
  -\frac{\left[y_{n,i} - \Omega(\iota) \right]^2}{N_0}
  \right\},
\end{equation}
\end{small}

\vspace{-0.2in}
Similarly, when $v_{n,i} = (1)_2$, it indicates $r_{n,i}$ belongs to $\{-J+2, -J+6, -J+10,\ldots\}$, and
\begin{equation}
  {\rm Pr}(v_{n,i}=1|y_{n,i}) = \frac{1}{{p(y_{n,i})}} 
  \sum_{\iota=1,\iota+2}^{\iota \le J} {\mathcal P}_r(\iota)\cdot \frac{1}{\sqrt{\pi N_0}}
  \exp \left\{
  -\frac{\left[y_{n,i} - \Omega(\iota) \right]^2}{N_0}
  \right\}.
\end{equation}

Then, ${\rm Pr}(v_{n,i} = 0|y_{n,i})$ and ${\rm Pr}(v_{n,i} = 1|y_{n,i})$ are used for decoding $\bf y$. 
If a binary LDPC code is utilized, we can directly calculate LLRs based on ${\rm Pr}(v_{n,i} = 0|y_{n,i})$ and ${\rm Pr}(v_{n,i} = 1|y_{n,i})$. If an NB-LDPC code is utilized, the pmf can be computed based on ${\rm Pr}(v_{n,i} = 0|y_{n,i})$ and ${\rm Pr}(v_{n,i} = 1|y_{n,i})$.
Finally, we can separate the detected vector $\hat{\bf v}$ into $J$ different bit sequences as 
$\hat{\bf b}_1, \hat{\bf b}_2, \ldots, \hat{\bf b}_J$ 
by using an inverse function ${\rm F}_{q2{\rm B}}$ (or decoding table).



%If channel code is not used, then $\bf v = u$. 
%In this case, we can directly estimate the value of $v_{n,i}$ by comparing 
%${\rm Pr}(v_{n,i} = 0|y_{n,i})$ and ${\rm Pr}(v_{n,i} = 1|y_{n,i})$. 
%If ${\rm Pr}(v_{n,i} = 0|y_{n,i}) > {\rm Pr}(v_{n,i} = 1|y_{n,i})$, 
%the detected element $\hat{v}_{n,i} = (0)_2$; 
%otherwise, $\hat{v}_{n,i} = (1)_2$. 


%\vspace{-0.3in}
\subsection{An O-FFMA system over GF($2^m$) without channel code}
Now, we consider an FFMA system over GF($2^m$) without channel code.
For convenience, we can set $K = N = 1$, 
remove the subscript $k$ from $b_{j,k}$ and $u_{j,k}$ which are then $b_j$ and $u_j$, 
and remove the subscript $n$ from $x_{j,n,i}$ which is then rewritten as $x_{j,i}$. 
When the $j$-th user transmits $b_j$, according to (\ref{e.u_j}) and (\ref{e.x_j}),
we have
\begin{equation}
x_{j,i} = \left\{
  \begin{matrix}
    2 b_{j} - 1, & i = j-1\\
    -1,          & i \neq j-1 \\
  \end{matrix} \right. ,
\end{equation}
where $0 \le i < m$.
Thereafter, the received signal is
\begin{equation*}
y_{i} = \sum_{j=1}^{J} x_{j,i} + z_{i} = (2 b_{j-1} - 1) + \sum_{j' \neq j-1}^{J} (-1) + z_i \\
      = (-J + 2 b_{j-1})+ z_i,
\end{equation*}
where $r_i = -J + 2 b_{j-1}$.
If $b_{j-1} = (0)_2$, we have $r_i = -J$; 
otherwise $b_{j-1} = (1)_2$, and we have $r_i = -J+2$.
Then, the complex-field sum-pattern is $\Omega = \{-J, -J+2\}$, 
with equiprobability ${\mathcal P}_r = \{0.5, 0.5\}$.


The Euclidean distance between $-J$ and $-J+2$ is the same as that of BPSK modulation.
Therefore, we can directly utilize BPSK demodulation schemes to detect the proposed FFMA systems.
%In addition, the BER of the proposed system is given as $P_e = Q(\sqrt{2 \gamma})$, where $\gamma$ is the average bit energy per noise power density $E_b/N_0$.
Now, we give an example to illustrate the transmission process.



\textbf{Example 7}:
Given a finite-field GF($2^4$), assume there are $4$ users, i.e., $J = 4$, without considering the effect of noise.
Let ${b}_1 = (1)_2$, ${b}_2 = (0)_2$, ${b}_3 = (1)_2$, and ${b}_4 = (1)_2$. 
Thus, we have

\begin{small}
\begin{equation} \label{e.Ex6}
\begin{aligned}
{b}_1 = (1)_2 \to & {\bf u}_1 = (\textcolor{blue}{1}, 0, 0, 0)_2 \to & {\bf x}_1 = (+1, -1, -1, -1), \\
{b}_2 = (0)_2 \to & {\bf u}_2 = (0, \textcolor{blue}{0}, 0, 0)_2 \to & {\bf x}_2 = (-1, -1, -1, -1), \\
{b}_3 = (1)_2 \to & {\bf u}_3 = (0, 0, \textcolor{blue}{1}, 0)_2 \to & {\bf x}_3 = (-1, -1, +1, -1), \\
{b}_4 = (1)_2 \to & {\bf u}_4 = (0, 0, 0, \textcolor{blue}{1})_2 \to & {\bf x}_4 = (-1, -1, -1, +1), \\
\end{aligned}
\end{equation}
\end{small}
and the received complex-field sum-pattern is
${\bf r} = \sum_{j=1}^{4} {\bf x}_j = (-2, -4, -2, -2).$



Since $J = 4$, we have $\Omega = \{-4, -2, 0, +2, +4\}$ and $\Omega_v = \{0, 1, 0, 1, 0\}$,
indicating that ${\rm F_{C2F}}(-4) = (0)_2$, ${\rm F_{C2F}}(-2) = (1)_2$, ${\rm F_{C2F}}(0) = (0)_2$,
${\rm F_{C2F}}(+2) = (1)_2$, and ${\rm F_{C2F}}(+4) = (0)_2$.
Therefore,
${\bf u} = {\rm F_{C2F}}({\bf r}) = (1, 0, 1, 1)_2$.
Through ${\rm F}_{q2{\rm B}}$, 
we can obtain that ${b}_1 = (1)_2$, ${b}_2 = (0)_2$, ${b}_3 = (1)_2$, 
and ${b}_1 = (1)_2$. 
Additionally, the practical complex-field sum-pattern set of ${\bf r}$ becomes as $\Omega = \{-4, -2\}$ with probability ${\mathcal P}_r =  \{0.5, 0.5\}$, which are accord with the above analyses.
$\blacktriangle \blacktriangle$




From Example 7, we find some interesting phenomenon.
Through ${\rm F}_{{\rm B}2q}$ transform, 
the transmit bit $b_j$ is transformed into the $m$-tuple ${\bf u}_j$, 
and ${\bf u}_1, {\bf u}_2, \ldots, {\bf u}_J$ together form a $J$-user \textit{finite-field TDMA} system.
After ${\rm F_{F2C}}$ transform from ${\bf u}_j$ to ${\bf x}_j$, 
the transmit signal ${\bf x}_j$ can be viewed as a type of \textit{spreading} of bit $b_j$.
Suppose a $4$-user CDMA system utilizes Walsh codes to separate users, 
i.e., ${\mathcal C}_{wal} = \{(+1, +1, +1, +1), \\
(+1, -1, +1, -1), (+1, +1, -1, -1), (+1, -1, -1, +1)\}$, 
which are used to compare with the transmit signals 
${\bf x}_1, {\bf x}_2, {\bf x}_3, {\bf x}_4$ of Example 7.
It can be seen that both have the same transmit power and the same BER performance.
Consequently, the proposed uplink FFMA system of GF($2^m$) in a GMAC is likewise a type of CDMA system. 
Nevertheless, there are some differences between them.
%The major features are two folds.
\begin{enumerate}
  \item
  The transmit signals are different. For an FFMA system over GF($2^m$), 
  the $j$-th user only transmits at the $(j-1)$-th location of the $m$-tuple, and the other locations are all zeros.
  So that, only $x_{j,j-1}$ in ${\bf x}_j$ is determined by the $j$-th user.
  On contrast, 
  the spreading signals of the $j$-th user is determined by the Walsh code ${\mathcal C}_{wal}$.
  Hence, an FFMA system is more flexible in designing transmit signals.
  \item
  The detection algorithms are different. For an FFMA system over GF($2^m$) without channel code,
  we can directly use BPSK demodulation schemes, which are quite easy. 
  While, for a CDMA system, we must use de-spreading algorithms to recover the signals from different users. Moreover, an uplink CDMA transmission system is generally strict with synchronization.
\end{enumerate}
%It is interesting to further investigate the uplink FFMA over GF($2^m$), which is left as our future work.



%\vspace{-0.2in}

\section{An uplink NO-FFMA System over GF($p^m$) for a GMAC}


In this section, we first introduce a constraint on \textit{multiuser complex-field sum-pattern to finite-field sum-pattern (C2F) mapping}, and then investigate a NO-FFMA system over GF($3^2$) for a GMAC. 

\vspace{0.1in}
\subsection{C2F-constraint and Spectrum Efficiency}


Consider a $J$-user uplink FFMA system,
let $\Omega$ and $\Theta$ be the complex-field sum-pattern set and the transmitted bit set of the $J$ users, respectively. For example, if each user utilizes the same BPSK modulation, 
it is known that $|\Omega| = J+1$ (as aforementioned in Sect. V) and $|\Theta| = 2^J$.
To recover an uplink FFMA system over the finite-field GF($p^m$) without ambiguity,
we have the following proposition.


\begin{proposition}
(C2F-constraint for BPSK): Suppose there are $J$ users where $J \ge 2$, and each user utilizes the same BPSK modulation. Denote the complex-field sum-pattern by $\Omega$ with $|\Omega| = J+1$. 
Let $m$ denote the number of independent resources, which is the extension factor of the extension field GF($p^m$) of the prime field GF($p$).
To recover the transmitted bits without ambiguity, $\rm F_{C2F}$ should be a many-to-one or one-to-one mapping function, and the number of served users $J$ satisfies the constraint:
\begin{equation} \label{C2F-con}
    2^J \le (J+1)^m,
\end{equation}
which is referred to as the multiuser complex-field sum-pattern to finite-field sum-pattern mapping constraint (C2F-constraint) for BPSK modulation.
\end{proposition}


According to (\ref{C2F-con}), when $J = 3$, it can be readily verified that $m$ satisfies $m \ge 2$.
Note that, when $m \ge 2$, a user may not send the same information during all $m$ resources.
Hence, at the $i$-th location for $0 \le i < m$, the number of arrival users may be equal to or smaller than $J$.
It indicates that the proposed C2F-constraint is a loose upper bound, e.g., the gap between $|\Theta| =2^3$ and $|\Omega| = 4^2$ is $8$.
For a practical system, we can find more tighter bounds.
For example, if $J = 3$, $m = 2$, and the numbers of arrival users at $i = 0$ and $i = 1$ are respectively $4$ and $3$, the gap between $|\Theta| = 2^3$ and $|\Omega| = 4 \times 3$ is only $4$.
Nevertheless, the C2F-constraint can help us quickly decide the required resource factor $m$.



This paper only investigates the scenario that each user utilizes the same BPSK modulation, which is the fundamental scenario for an uplink MA system. When each user utilizes the same high-order modulations or different modulations, the C2F-constraint may alter correspondingly. 



Under the C2F-constraint for BPSK, the SE $\lambda$ of the proposed FFMA system of GF($p^m$) is defined by
\begin{equation}
   \lambda = \frac{J \cdot R_c}{m}, 
\end{equation}
where $J$ is the number of users and $R_c$ is the data rate. 
When $J = m$ and $R_c = 1$, $\lambda$ equals to 1, i.e., $\lambda = 1$. Thus, the SE of the uplink FFMA system of GF($2^m$) is equal to one, which is an O-FFMA system.
When $J > m$ and $R_c = 1$, it is a NO-FFMA system. 
It is an open issue to design complex-field NO-FFMA. 
In the following, we introduce one of the possible structures of a complex-field NO-FFMA using a small finite-field.


%\vspace{-0.2in}
\subsection{A NO-FFMA System over GF($3^2$)}


In the following, we investigate a NO-FFMA system over GF($3^2$) without channel codes in a GMAC.
Since the base field is GF(3), it has only one EP $C_1 = (1, 2)$, whose reverse order EP is $C_2 = (2,1)$. 
Suppose the number of users is $J = 3$. Then, the SE is equal to $\lambda = J/m=1.5$, which is a NO-FFMA system.



Without considering the effect of channel codes, refer to the definition in Sect. V-C,
the $j$-th user firstly maps the bit information $b_j \in {\mathbb B}$ into a $2$-tuple $u_j = (u_{j,0}, u_{j,1})_3 \in {\rm GF}(3^2)$ by the mapping function ${\rm F}_{{\rm B}2q}$.
Let the bits of $3$-user be ${\bf b} = (b_1, b_2, b_3)_2 \in \Delta$, where $|\Delta|=2^3=8$.
To support a NO-FFMA system with $J$ users over GF($p^m$), ${\rm F}_{{\rm B}2q}$ is a $J \times m$ matrix.
For $J=3$ and $m=2$, we set 
\begin{equation}
{\rm F}_{{\rm B}2q} = 
\left[
  \begin{matrix}
    \psi_0(C_{t_{10}}) & \psi_1(C_{t_{11}})\\
    \psi_0(C_{t_{20}}) & \psi_1(C_{t_{21}})\\
    \psi_0(C_{t_{30}}) & \psi_1(C_{t_{31}})\\
  \end{matrix}
\right],
\end{equation}
where $t_{ji}$ belongs to an element over GF($3$), for $1 \le j \le J$ and $0 \le i < m$. 
When $t_{ji} = 0$, we set $C_{t_{ji}}=0$.
The $j$-th user selects the $j$-th row of ${\rm F}_{{\rm B}2q}$, 
i.e., ${\rm F}_{{\rm B}2q,j}$ for $1 \le j \le J$, as its mapping function,
such as $u_j = {\rm F}_{{\rm B}2q,j}(b_j)$. 
If the $i$-th location EP of the $j$-th user is 
$\psi_i(C_{t_{ji}}) = \psi_i(t_{ji}, p-t_{ji})$, then,
\begin{equation}
u_{j,i} = 
\left\{
  \begin{matrix}
  t_{ji},   & b_j = 0 \\
  p-t_{ji}, & b_j = 1\\
  \end{matrix}.
\right.
\end{equation}
The Cartesian product of the mapping function ${\rm F}_{{\rm B}2q, j}$ for $1 \le j \le J$ is defined as
\begin{equation}
  \Psi \triangleq \left(\psi_0(C_{t_{10}}), \psi_1(C_{t_{11}})\right) 
  \times \left(\psi_0(C_{t_{20}}), \psi_1(C_{t_{21}})\right) 
  \times \left(\psi_0(C_{t_{30}}), \psi_1(C_{t_{31}})\right),
\end{equation}
which forms a $2$-tuple AIEP code over GF($3^2$), and ${\bf u} = (u_1, u_2, u_3)$ is a codeword in $\Psi$.
Let the finite-field sum-pattern $\tau$ of the $3$-user be $\tau = \bigoplus_{j=1}^{3} u_j$,
which is a $2$-tuple over GF($3^2$).
Each codeword ${\bf u}$ in $\Psi$ has a finite-field sum-pattern $\tau$, i.e., ${\bf u} \to \tau$.
Assume all the sum-patterns $\tau$ belong to the finite-field sum-pattern set $\Gamma$, 
where $\tau \in \Gamma$ and $\Gamma \subseteq {\rm GF}(3^2)$.



Suppose each user utilizes the same BPSK modulation, and the mapping function $\rm F_{F2C}$ between the transmit signal $x_{j,i}$ and $u_{j,i}$ is $\rm F_{F2C}$ and
\begin{equation}
x_{j,i} = 
\left\{
  \begin{matrix}
  -1,   & u_{j,i} = (1)_3 \\
  +1,   & u_{j,i} = (2)_3 \\
  \end{matrix}.
\right.
\end{equation}
Then, the $j$-th user transmits $x_j$ for $1\le j \le J$ to the GMAC. 
At the receiver, the received signal is given by
\begin{equation}
y = \sum_{j}^{J} x_{j} + z = r + z,
\end{equation}
where $r = \sum_{j}^{J} x_{j} = (r_{0}, r_{1}) \in {\Omega}$, and $z$ is an AWGN noise with ${\mathcal N}(0, N_0/2)$.
To recover the transmit bits of $3$ users, 
it is required to build a uniquely mapping ${\rm F}_{{\rm C2F}}$ between the complex-field sum-pattern $r$ and transmit bits ${\bf b}$. 
In the proposed system, ${\rm F}_{{\rm C2F}}(r_i)$ is set to be $\tau_i = {\rm F}_{{\rm C2F}}(r_i)$ for $0 \le i < m$.



In Sect. V-B, we showed that the complex-field sum-pattern set $\Omega$ of $3$-user case is $\Omega = \{-3, -1, +1, +3\}$.
When $r_i = -3$ and/or $r_i = +3$, it can be uniquely mapped to ${\bf b} = (000)_2$ and/or ${\bf b} = (\textcolor{red}{111})_2$, indicating that the users can be separated without ambiguity. 
Thus, let ${\Omega}_0 = \{-3, +3\}$ be the \textit{solved subset} of the complex-field sum-pattern set,
and the other elements in ${\Omega}\backslash {\Omega}_0$ be the \textit{unsolved subset} of the complex-field sum-pattern set.
%When $r = -1$ and/or $r = +1$, we cannot recover the users' information bits, because of the existing multiple combinations who own the same number of $(0)_2$ and/or $(1)_2$.
Afterwards, the complex-field sum-pattern $r$ is determined by two aspects:
one is the numbers of $(0)_2$ and/or $(1)_2$ in ${\bf b}$;
and the other is the location distributions of $(0)_2$ and/or $(1)_2$.


\begin{figure}[t]
  \centering
  \includegraphics[width=0.7\textwidth]{Fig_NOMA.pdf}
  \caption{Diagram of the proposed $3$-user NO-FFMA system over GF($3^2$) in a GMAC.
  (a) $3$-user with 2PSK in GF($3^2$); (b) Sum-patterns of $C_1$ and $C_1^{\mathrm R}$ in GF($3$);
  and (c) Mapping function between $\Theta$ and $\Omega$.
  }
  \vspace{-0.2in}
\end{figure}


Thereby, for the $3$-user uplink MA system, we can divide $\Delta$ into three subsets, i.e., 
$\Delta_0 = \{(000)_2, (\textcolor{red}{111})_2\}$,
$\Delta_1 = \{(\textcolor{red}{1}00)_2, (0\textcolor{red}{1}0)_2, (00\textcolor{red}{1})_2\}$, and 
$\Delta_2 = \{(0\textcolor{red}{11})_2, (\textcolor{red}{1}0\textcolor{red}{1})_2, (\textcolor{red}{11}0)_2\}$, where $\Theta_k$ for $k=1, 2$ indicates there are $k$ $(1)_2$ in the subset $\Delta_k$.
In the subset $\Delta_0$, all the elements are the same, and its corresponding complex-field sum-patterns are in the solved subset ${\Omega}_0$.
In the following discussion, we mainly focus on the unsolved subset ${\Omega}\backslash {\Omega}_0$.
Observe $\Theta_k$ for $k=1, 2$, it is found that it has cyclic structure.
Hence, we bring finite-field to record the location changes of $0$ (or $1$) in $\Delta_k$ for $k=1, 2$, since the cyclic structure can be well-mannered exhibited by the feature of finite-field.



Denote $\Gamma_k$ for $k=1, 2$ as the corresponding finite-field sum-pattern subset of $\Theta_k$,
where $(\tau_0, \tau_1)_3 \in \Gamma_k$ and $\Gamma_k \subset$ GF($3^2$).
We set $\tau_0 = k$ for $k = 1, 2$, to record the number of $(1)_2$ in $\Theta_k$;
and $\tau_1$ as a cyclic mode, e.g., $0 \to 1 \to 2 \to 0$ (or $0 \to 2 \to 1 \to 0$), 
to record the cyclic location changes of $(1)_2$ and/or $(0)_2$ in $\Theta_k$, 
as shown in Fig. 8 and Eq. (\ref{e.3UE}).
\begin{equation} \label{e.3UE}
  \begin{array} {ll}
  \Delta_1  \to \Gamma_1 \to \Omega_1 &
  \Delta_2  \to \Gamma_2 \to \Omega_2 \\
 % (000)_2                  \to (00)_3 \to (-3, 0)  &
 % (\textcolor{red}{111})_2 \to (00)_3 \to (+3, 0)\\
  (\textcolor{red}{1}00)_2 \to (11)_3 \to (-1, +2) &
  (0\textcolor{red}{11})_2 \to (22)_3 \to (+1, -2) \\
  (0\textcolor{red}{1}0)_2 \to (12)_3 \to (-1, -2) &
  (\textcolor{red}{1}0\textcolor{red}{1})_2 \to (21)_3 \to (+1, +2)\\
  (00\textcolor{red}{1})_2 \to (10)_3 \to (-1, 0) &
  (\textcolor{red}{11}0)_2 \to (20)_3 \to (+1, 0)\\
  \end{array}
\end{equation}


Then, each $\tau = (\tau_0, \tau_1)$ in $\Gamma$ is assigned a uniquely complex-field sum-pattern $r = (r_0, r_1)$.
%In general, $r_0$ is suggested to travel all the $J+1$ values; and $r_i$ for $0 < i < m$ can be designed according to the requirement. 
For example, at the $i=0$ location, we can set $(\tau_0 = (1)_3) = {\rm F}_{{\rm C2F}}(r_0 = -1)$ and 
$(\tau_0 = (2)_3) = {\rm F}_{{\rm C2F}}(r_0 = +1)$.
Consider $|\Theta_k|$ for $k=1, 2$ is equal to $3$, at the $i=1$ location, 
we require three values to distinct the cyclic locations.
In our system, we set $(\tau_1 = (0)_3) = {\rm F}_{{\rm C2F}}(r_1 = 0)$,
$(\tau_1 = (1)_3) = {\rm F}_{{\rm C2F}}(r_1 = +2)$ and $(\tau_1 = (2)_3) = {\rm F}_{{\rm C2F}}(r_1 = -2)$.



Based on the designed $\Gamma$ and finite-field calculations, the mapping function ${\rm F}_{{\rm B}2q}$ can be given as
\begin{equation}
{\rm F}_{{\rm B}2q} = 
\left[
  \begin{matrix}
    \psi_0(C_1) & \psi_1(C_1)\\
    \psi_0(C_1) & \psi_1(C_1^{\rm R})\\
    \psi_0(C_1) & 0 \\
  \end{matrix}
\right],
\end{equation}
whose Cartesian product $\Psi$'s finite-field sum-pattern set is accord with our designed $\Gamma$. 
In fact, there exist different combinations of ${\rm F}_{{\rm B}2q}$ and ${\rm F}_{{\rm C2F}}$ to satisfy the mapping relationship between $\Gamma$, $\Psi$ and $\Omega$, which can be designed flexibly.



In general, we can design a NO-FFMA system over GF($p^m$) based on a \textit{0-1 distribution} approach. 
For an uplink $J$-user MA system, the major stages are:
\begin{enumerate}
  \item
  Divide $\Theta$ into $J$ subsets, defined by $\Theta=\{\Theta_0, \Theta_1, \ldots, \Theta_k, \ldots, \Theta_{J-1}\}$.
  $\Theta_0$ only has two elements, i.e., $\Theta_0 = \{(0\ldots0)_2, (1\ldots1)_2\}$, 
  in which all the users transmit the same bit. 
  The complex-field sum-pattern of $\Theta_0$ is corresponding to the solved subset $\Omega_0$.
  $\Theta_k$ for $1 \le k \le J-1$ indicates that 
  there are $k$ $(1)_2$ (or $(0)_2$) in the subset $\Theta_k$.
  \item
  For the subset $\Theta_k$, we record the number of $1$s (or $0$s), and their cyclic locations by using finite-field GF($p^m$). 
  Therefore, each $\Theta_k$ has a corresponding designed sum-pattern subset $\Gamma_k$, 
  in which $\Gamma=\{\Gamma_1, \ldots, \Gamma_k,\ldots, \Gamma_{J-1}\}$ with $1 \le k \le J-1$. 
  Each element ${\bf b}$ in $\Theta_k$ should have a uniquely sum-pattern $\tau$ in $\Gamma_k$,
  thus, $|\Theta_k| = |\Gamma_k|$.
  \item
  According to the designed $\Gamma$, 
  we further construct $\Gamma$'s corresponding $\Psi$ and $\Omega$. 
  For a selected finite-field GF($p^m$), 
  each finite-field sum-pattern $\tau_i$ for $0 \le i < m$ should be uniquely mapped to a complex-field sum-pattern $r_i$, i.e., $\tau_i \leftrightarrow r_i$, so that the superposition signals can be recovered without ambiguity. 
\end{enumerate}
Actually, NO-FFMA systems can be applied to \textit{multi-node relay communication network}, which may reduce the transmit slots and improve the throughput. For this scenario, our proposed 0-1 distribution approach can play an important role. 



\vspace{-0.1in}
\section{Simulation results}

In this section, we simulate the BER performances of the proposed downlink and uplink FFMA systems, 
in which a $16$-ary $(864,576)$ LDPC code with data rate $R_c = 0.6667$ is used for error control. 
%Both systems are designed to support maximum $J=4$ users.


First, we consider the downlink FFMA system of GF($17$) which supports $J = 4$ users for transmission over frequency-selective fading channels. We evaluate its error performance for two cases. 
In the first case, the number of subcarriers is set to $N_c = 64$; 
and for the second case, the number of subcarriers is set to $N_c = 256$. 
For both cases, we assume the transmissions are over either ${\mathcal L} = 4$ paths or ${\mathcal L} = 16$ paths, and the length of CP is $N_g = 16$. 
For comparison, we also evaluate the performance of an equivalent OFDMA system which supports $J = 4$ users. In this system, each user is assigned with $N_c/J$ carriers, i.e., either $64/4 = 16$ subcarriers or $256/4 = 64$ subcarriers.
   

The BER performances of the downlink FFMA and the OFDMA systems with the above design parameters are shown in Fig. 9. From Fig. 9, we see that for a given $N_c$ and $\mathcal L$, the proposed downlink FFMA system provides a slight better BER performance than that of the classical downlink OFDMA system. 
Since the proposed FFMA can exploit all the sub-carriers to transmit signals, thus it can achieve all frequency-domain diversity gain. 
Moreover, it is found that the BER of $N_c=64$ is generally better than the case of $N_c=256$, because of the utilization of LDPC code. For the given LDPC code $C_q$, a smaller block, i.e., $N_c = 64$, indicates a larger interleaving in frequency-domain, which may improve the BER performance. In addition, the BER performance of the case with a larger number of independent paths, i.e., $\mathcal L = 16$, 
provides a slightly better BER performance than the case with a small number of independent paths, i.e., $\mathcal L=4$. Since a larger $\mathcal L$ may result in a serious frequency-selective fading, we can achieve frequency diversity gain by using a channel code.


\begin{figure}[t]
  \centering
  \includegraphics[width=0.4\textwidth]{Graph_DL.pdf}
  \caption{BER performance of the proposed downlink FFMA system of GF($17$) over frequency-selective fading channels.
  %The BER performance of the code over an AWGN channel decoded with 50 iterations of the FFT-QSPA as shown in Fig. 8 with $J = 1$.
  }
  \vspace{-0.2in}
\end{figure}


Fig. 10 shows the BER performances of the O-FFMA systems of GF($2^4$) which can support $J = 1, 2, 3, 4$ users and a $3$-user NO-FFMA system of GF($3^2$) over GMACs. 
Without channel coding, the BER performance of the O-FFMA system of GF($2^4$) is exactly in accord with the theoretical BER performance of BPSK modulation.
Moreover, the BER performance of the proposed NO-FFMA system is worse than that of BPSK modulation, because of the improved SE.



Consider the uplink O-FFMA system using a $16$-ary $(864, 576)$ NB-LDPC code $C_q$ for error control.
Decoding is carried out with 50 iterations of the FFT-QSPA \cite{LinBook3}. 
For $J = 1$, the curve is the BER performance of the $16$-ary $(864,576)$ NB-LDPC code $C_q$ over an AWGN channel for a single user.
When a channel code is utilized for an uplink O-FFMA system,
the BER performance very much depends on the probabilities of the elements in $\Omega$. 
Two different sets of probabilities of the elements in $\Omega$ may results in a big gap in decoding of the error control code $C_q$. 
To see this, we consider two cases with $J = 4$. In Case 1, we directly exploit the given $\Omega$ and ${\mathcal P}_r$, i.e.,
$\Omega = \{-4,-2,0,+2,+4\}$ and ${\mathcal P}_r=\{0.0625,0.25,0.375,0.25,0.0625\}$. 
Case 2 is based on the systematic form of an encoded codeword. 
If the information vector ${\bf u}_j$ is encoded into a codeword ${\bf v}_j$ in systematic form, 
i.e., ${\bf v}_j = ({\bf u}_j, {\bf v}_{j,red})$, 
where ${\bf v}_{j,red}$ is the inserted redundancy vector. 
In this case, we can set  $\Omega = \{-4,-2\}$ and ${\mathcal P}_r = \{0.5,0.5\}$ to the first $K$ received information symbols; 
and $\Omega = \{-4,-2,0,+2,+4\}$ and ${\mathcal P}_r=\{0.0625,0.25,0.375,0.25,0.0625\}$ 
to the next $N-K$ received parity symbols. 
From Fig. 9, we see that the system in Case 2 performs better than the system in Case 1, 
because more accurate probabilities of the elements $\Omega$ are used. 
In addition, the BER performance of an uplink FFMA system becomes slightly worse as the number of user increases. However, the small performance degradation is acceptable.




\begin{figure}[t]
  \centering
  \includegraphics[width=0.4\textwidth]{Graph_UL.pdf}
  \caption{BER performance of the proposed uplink FFMA system of GF($2^4$) over an AWGN MAC, where $J = 1, 2, 3, 4$ and $16$-ary $(864, 576)$ NB-LDPC code $C_q$ is used for error control.}
  \vspace{-0.2in}
\end{figure}




\vspace{-0.1in}
\section{Conclusion and Remarks}

In this paper, we presented an FFMA technique to support multiuser transmission in algebraic domain using finite fields. Unlike the classical complex-field MA techniques, an FFMA system is operated and processed in finite-fields based on their algebra structures.

We first presented multiuser UD codes over finite fields. 
A multiuser UD code consists of a collection of EPs over a finite field with unique mapping structural property. Each EP in a multiuser UD code is a pair of additive inverses in a chosen finite field. 
In the paper, two classes of multiuser UD codes were presented. The first class of multiuser UD codes is constructed based on prime fields. The second class of multiuser UD codes is constructed based on extension fields of prime fields. 
The multiuser UD codes in the second class have orthogonal structure. 
%Bounds on the numbers of users that both types of multiuser UD codes can support were derived.


Based on the algebraic structure of multiuser UD codes, we presented FFMA systems, including network MA, downlink MA and uplink MA systems. 
An FFMA system can be jointly designed with a CFMA system, and obtained a fusion MA.
The downlink system is designed based on a prime field GF($p$) for multiuser transmission over frequency-selective fading channels.
In the process of transmission, we also introduced a transform between two different fields for error control coding. We showed that the proposed downlink FFMA system can achieve full diversity gain.


The uplink O-FFMA system is designed based an extension field GF($2^m$) of the binary field GF(2) for multiuser transmission over a GMAC. 
For such an uplink FFMA, a collection of $m$ orthogonal UDAIEPs is used to separate the bit-sequences sent by the users from a received multiplexed sequence at the receiving end. 
In the detection process, a complex-field to finite-field constraint was introduced for unique recovering the output bit-sequences of the users. 
Then, a $3$-user NO-FFMA system was proposed, whose SE is equal to 1.5.



%In construction of multiuser UD codes over a finite field, each EP is a pair of additive elements. Multiuser UD codes in which each EP is a pair of multiplicative inverse elements can also be constructed. Using UDAIEPs, each bit from a user is mapped into an element in an assigned AIEP. 


In the future, we can do researches on FFMA systems from three aspects.
First, some fundamental theory on FFMA systems should be investigated.
For example, it is interesting to investigate the case in which each block of $k$ bits from a user is mapped in a unique $r$-tuple over a finite field such the sets of $r$-tuples assigned to the users form a uniquely decodable code. 
Secondly, we can design different FFMA systems based on different finite-fields, e.g., we will consider design downlink and uplink FFMA systems based on finite fields GF($p^m$) with $p>2$ and $m>1$. 
Thirdly, it is important to apply FFMA systems into different scenarios to achieve well-behaved performances.
For example, we intend to look into design an FFMA system in conjunction with classical complex-field MA to provide more flexibility and variety. It is appealing to investigate FFMA systems for random access channels \cite{R4, Yu_UDAS}. Moreover, we can also apply FFMA system into multi-node relay communications, which may reduce the transmit time slots and improve the throughput.
It is also interesting to jointly consider FFMA with semantic communications.




%\vspace{0.1in}
\begin{thebibliography}{99}
%%%%%%%%%%%%%%%%%%%%%%%%%%%%%%%%%%%%%%
%%Following is references on Power & data rate & channel.





\bibitem{FAdachi1}
F. Adachi, D. Garg, S. Takaoka, and K. Takeda, ``Broadband CDMA Techniques," \textit{IEEE Wireless Communications}, vol. 12, no. 2, pp. 8-18, April 2005.


\bibitem{ZDing2017_survey}
Z. Ding, X. Lei, G. K. Karagiannidis, R. Schober, J. Yuan, and V. K. Bhargava, ``A Survey on Non-Orthogonal Multiple Access for 5G Networks: Research Challenges and Future Trends," \textit{IEEE Journal on Selected Areas in Communications}, vol. 35, no. 10, pp. 2181 - 2195, July 2017.


\bibitem{YChen_2018}
Y. Chen et al., ``Toward the Standardization of Non-Orthogonal Multiple Access for Next Generation Wireless Networks," \textit{IEEE Communications Magazine}, vol. 56, no. 3, pp. 19-27, March 2018.



\bibitem{QWang_2018}
Q. Wang, R. Zhang, L. Yang, and L. Hanzo, "Non-Orthogonal Multiple Access: A Unified Perspective," \textit{IEEE Wireless Communications}, vol. 25, no. 2, pp. 10-16, April 2018.


\bibitem{QY_ISJ_2019}
Q. Yu, H. Chen and W. Meng, ``A Unified Multiuser Coding Framework for Multiple Access Technologies,'' \textit{IEEE Systems Journal}, vol. 13, no. 4, pp. 3781-3792, 2019. 


\bibitem{LDai2015}
L. Dai, B. Wang, Y. Yuan, S. Han, C. I, and Z. Wang, ``Nonorthogonal multiple access for 5G: Solutions, challenges, opportunities, and future research trends," \textit{IEEE Communication Magazine}, vol. 53, no. 9, pp.74-81, Sept. 2015.


\bibitem{HNiko2013}
H. Nikopour and H. Baligh, ``Sparse code multiple access," \textit{in Proc. IEEE 24th International Symposium on Personal Indoor and Mobile Radio Communications (PIMRC)}, pp. 332-336, Nov. 2013.


%\bibitem{HNiko2014}
%H. Nikopour, E. Yi, A. Bayesteh, K. Au, M. Hawryluck, H. Baligh, and J. Ma, ``SCMA for downlink multiple access of 5G wireless networks," \textit{in Proc. IEEE GLOBECOM}, pp. 3940-3945, Dec. 2014.





%%%%% Below is about uniquely-decodable codes
\bibitem{Liao1972}
H. H. J. Liao, ``Multiple access channels," Ph.D dissertation, Dept. Electrical Engineering, Univ. Hawaii, Honolulu, HI, 1972.

\bibitem{Kasami1976}
T. Kasami, and Shu Lin, ``Coding for a Multiple-Access Channel," \textit{IEEE Trans. Information Theory}, vol. IT-22, no. 2, pp. 129-137, March 1976.

\bibitem{Chang1976}
S. C. Chang and E. J. Weldon, ``Coding for a T-user Multiple Access Channel," \textit{IEEE Trans. Information Theory}, Vol.IT-25, No.5, pp.684-691, Sept.1979.



\bibitem{Kasami1978}
T. Kasami, and Shu Lin, ``Bounds on the Achievable Rates of Block Coding for a Memoryless Multiple-Access Channel," \textit{IEEE Trans. Information theory}, vol. IT-24, no. 2, pp. 187-197, March 1978.


\bibitem{Peterson1979}
Peterson R, Costello D., ``Binary convolutional codes for a multiple-access channel,'' \textit{IEEE Trans. Information Theory}, vol. 25, no. 1, pp. 101-105, 1979.

\bibitem{Chevillat1981}
P. Chevillat, `` N-user trellis coding for a class of multiple-access channels,'' \textit{IEEE Trans. Information Theory}, vol. 27, no. 1, pp. 114-120, 1981.


\bibitem{Kasami1983}
T. Kasami, Shu Lin, Victor K. Wei, and Saburo Yamamura, ``Graph Theoretic Approaches to the Code Construction for the Two-User Multiple-Access Binary Adder Channel," \textit{IEEE Trans. Information Theory}, vol. IT-29, no. 1, pp. 114-130, 1983.

\bibitem{Van1983}
V. Tilborg H., ``Upper bounds on $|C_2|$ for a uniquely decodable code pair $(C_1,C_2)$  for a two-access binary adder channel," \textit{IEEE Trans. on Information Theory}, vol. 29, no. 3, pp. 386-389, May 1983.


\bibitem{Vanroose1992}
Vanroose, P., Van Der Meulen, E. C., ``Uniquely decodable codes for deterministic relay channels," \textit{IEEE Trans. on Information Theory}, vol.38, no.4, pp. 1203-1212, July 1992.

\bibitem{Jevtic1992}
Jevtic, D. B., ``Disjoint uniquely decodable codebooks for noiseless synchronized multiple-access adder channels generated by integer sets," \textit{IEEE Trans. on Information Theory}, vol.38, no.3, pp. 1142-1146, May 1992.


\bibitem{Fan1995}
P. Fan, M. Darnell, and B. Honary, ``Superimposed codes for the multi-access binary adder channel'', \textit{IEEE Trans. Information theory}, vol. 41, no. 4, pp. 1178-1182, 1995.

\bibitem{Khachatrian1998}
Khachatrian G H, Martirossian S S. ``Code construction for the T-user noiseless adder channel,'' \textit{IEEE Trans. Information Theory}, vol. 44, no. 5, pp. 1953-1957, 1998.

\bibitem{Bross1998}
Bross, S. I., Blake, I.F., ``Upper bound for uniquely decodable codes in a binary input N-user adder channel," \textit{IEEE Trans. on Information Theory}, vol. 44, no. 1, pp. 334-340, Jan 1998.


\bibitem{Cheng2001}
Cheng J, Watanabe Y, ``A multiuser $k$-ary code for the noisy multiple-access adder channel,'' \textit{IEEE Trans. Information Theory}, vol. 47, no. 6, pp. 2603-2607, 2001.


\bibitem{Kiviluoto2007}
Kiviluoto, L., Ostergard, P. R. J., ``New Uniquely Decodable Codes for the T-User Binary Adder Channel With $3 \le T \le 5$," \textit{IEEE Trans. on Information Theory}, vol. 53, no. 3, pp. 1219-1220, March 2007.

\bibitem{Yu_P2P}
Q. Yu, W. Meng, and S. Lin, ``Packet Loss Recovery Scheme with Uniquely-Decodable Codes for Streaming Multimedia over P2P Networks," \textit{IEEE Journal on Selected Areas in Communications}, vol.31, no.9, pp. 142-154, Aug. 2013.


\bibitem{R1}
J. Zhu and M. Gastpar, ``Gaussian Multiple Access via Compute-and-Forward,'' IEEE Transactions on Information Theory, vol. 63, no. 5, pp. 2678-2695, May 2017.

\bibitem{R2}
M. Qiu, Y. -C. Huang, S. -L. Shieh and J. Yuan, ``A Lattice-Partition Framework of Downlink Non-Orthogonal Multiple Access Without SIC,'' \textit{IEEE Transactions on Communications}, vol. 66, no. 6, pp. 2532-2546, June 2018.


\bibitem{R3}
R. Ahlswede, Ning Cai, S. Li and R. W. Yeung, ``Network information flow,'' \textit{IEEE Transactions on Information Theory}, vol. 46, no. 4, pp. 1204-1216, July 2000.





%%%%%% LDPC 
%\bibitem{Gallager}
%R. G. Gallager, ``Low-Density Parity-Check Codes, " \textit{IRE Trans. Information Theory}, vol. 8, no. 1, pp. 21- 28, Jan. 1962.

%\bibitem{Tanner}
%R. M. Tanner, ``A recursive approach to low complexity codes, " \textit{IEEE Trans. Information Theory}, vol. 27, no. 9, pp. 533-547, Sept. 1981.

%\bibitem{Polar2009}
%Erdal Arikan, ``Channel Polarization: A Method for Constructing Capacity-Achieving Codes for Symmetric Binary-Input Memoryless Channels,'' \textit{IEEE Trans. Information Theory}, vol. 55, no. 7, pp. 3051-3073, July 2009.

%\bibitem{JDai2016}
%J. Dai, K. Niu, Z. Si, and J. Lin, ``Polar coded non-orthogonal multiple access," \textit{IEEE International Symposium on Information Theory (ISIT)}, Barcelona, pp. 988-992, June 2016.


\bibitem{GDF1989_1}
G. D. Forney Jr., and L. Wei, ``Multidimensional constellations. Part.I: Introduction, figures of merit, and generalized cross constellations," \textit{IEEE Journal on Selected Areas in Communications}, vol. 7, no. 6, pp. 877-892, Aug. 1989.


\bibitem{GDF1989_2}
G. D. Forney Jr., ``Multidimensional constellations. Part. II: Voronio Constellations," \textit{IEEE Journal on Selected Areas in Communications}, vol. 7, no. 6, pp.941-958, Aug. 1989.


\bibitem{R4}
R. C. Yavas, V. Kostina and M. Effros, ``Gaussian Multiple and Random Access Channels: Finite-Blocklength Analysis,'' \textit{IEEE Transactions on Information Theory}, vol. 67, no. 11, pp. 6983-7009, Nov. 2021.


\bibitem{Yu_UDAS}
Q. Yu, and K. Song, ``Uniquely Decodable Multi-Amplitude Sequence for Grant-Free Multiple-Access Adder Channels,'' \textit{IEEE Transactions on Wireless communications}, 2023, Early Access.



\bibitem{Shu2009}
William E. Ryan and Shu Lin, Channel Codes classical and Modern, Cambridge University Press, 2009.

%\bibitem{John2009}
%John G. Proakis, Digital Communications, Fifth Edition, Beijing, Publishing House of Electronics Industry, 2009.

\bibitem{LinBook3}
J. Li, S. Lin, K. Abdel-Ghaffar, W. E. Ryan, and D. J. Costello, ``LDPC Code Designs, Constructions, and Unification," Cambridge University Press, 2017.

%\bibitem{Thomas}
%Thomas M. Cover, and Joy A. Thomas, ``Elements of Information Theory,'' Tsinghua University Press, 2010.

\bibitem{Yi-Jing}
https://ctext.org/book-of-changes/yi-jing 


\end{thebibliography}


\vfill
\end{document}

