\documentclass[11pt,draftclsnofoot, onecolumn]{IEEEtran}
\hyphenation{op-tical net-works semi-conduc-tor}
\usepackage{graphicx,cite,epsfig,amssymb,amsmath,subfigure,url,stfloats,latexsym}
\usepackage{array}
\usepackage{amsfonts}
\usepackage{pgfplots}
\usepackage{algorithm}
\usepackage[noend]{algpseudocode}
%\usepackage{tikz}
%\usepackage{tkz-orm}
\usepackage{epstopdf}
\usepackage{amsfonts,amsthm}
\usepackage{multirow}
\usepackage{mathrsfs}
\usepackage{subfigure}
%\usepackage{subcaption}


%\usepackage{algorithmic}
\usepackage{algorithm}
\usepackage{algpseudocode}


\newtheorem{theorem}{Theorem}
\newtheorem{lemma}{Lemma}
\newtheorem{corollary}{Corollary}
\newtheorem{property}{Property}
\newtheorem{definition}{Definition}
\newtheorem{proposition}{Proposition}
\newtheorem{remark}{Remark}
\newtheorem{conjecture}{Conjecture}
\newtheorem{example}{Example}[section]


\newenvironment{varalgorithm}[1]
  {\algorithm\renewcommand{\thealgorithm}{#1}}
  {\endalgorithm}

\graphicspath{{./}{icons/}}
\tikzset{My Line Style/.style={samples=400}}
\graphicspath{{figures/}}

\usepackage[flushleft]{threeparttable}

\begin{document}

\markboth{IEEE TRANSACTIONS ON COMMUNICATIONS, Vol. XX, No. Y, Month 2022} { \ldots}
\title{\mbox{}\vspace{2.5cm}\\
\textsc{\huge Finite Field Multiple Access} \vspace{1.5cm}}

\vspace{1.5cm}
\author{\normalsize
Qi-Yue~Yu, {\it IEEE Senior Member},
Jiang-Xuan~Li,
and Shu~Lin, {\it IEEE Life Fellow} 
\thanks{Q.-Y.~Yu (email: yuqiyue@hit.edu.cn) and J.-X.~Li (email: 21S005095@stu.hit.edu.cn) are with the Communication Research Center, Harbin Institute of Technology, China. S. Lin (email: shulin@ucdavis.edu) is in University of California, Davis, U.S.}
\thanks{The work presented in this paper was supported by the National Natural Science Foundation of China under Grand No. 62071148.}
%\thanks{The paper was submitted on Feb. 12, 2017, and revised on \today.}\\
}



\date{\today}
\renewcommand{\baselinestretch}{1.2}
\thispagestyle{empty} \maketitle \thispagestyle{empty}

%\newpage
%\setcounter{page}{1}

\vspace{0.1in}
\begin{abstract}
%In the past several decades, various multiple-access (MA) techniques have been developed and used. These MA techniques are carried out in complex-field domain by signal processing which consumes physical resources to separate the outputs of the users. It becomes problematic to find new resources from the physical world. In this paper, an algebraic resource is proposed to support multiuser transmission. This algebraic resource is based on assigning each user an element pair (EP) from a finite field GF($p^m$). The output bit from each user is mapped into an element in its assigned EP, called the output symbol. Then, the output symbols from the users are jointly mapped into a unique symbol in the same field GF($p^m$) for modulation and transmission. When the transmitted symbol (after demodulation) is received at the receiving end, the receiver can separate the transmitted bits from the users without ambiguity. The EPs assigned to the users are said to form a set of uniquely decodable EPs (UDEPs), called a multiuser UD code. Using UDEPs over a finite field, a downlink and an uplink MA systems are proposed, which are called finite-field MA (FFMA) systems. The performances of both FFMA systems are analyzed. The downlink FFMA system can achieve full diversity gain over fading channels. An FFMA system can be designed in conjunction with the classical complex-field MA techniques to provide more flexibility and varieties. Methods for constructing finite-field UD codes for FFMA systems are provided.

In the past several decades, various multiple-access (MA) techniques have been developed and used. 
These MA techniques are carried out in complex-field domain by signal processing which consumes physical resources to separate the outputs of the users. It becomes problematic to find new resources from the physical world. 
In this paper, an algebraic resource is proposed to support multiuser transmission. This algebraic resource is based on assigning each user an element pair (EP) from a finite field GF($p^m$). The output bit from each user is mapped into an element in its assigned EP, called the output symbol. Then, the output symbols from the users are jointly mapped into a unique symbol in the same field GF($p^m$) for modulation and transmission. The EPs assigned to the users are said to form a set of uniquely decodable EPs (UDEPs), called a multiuser UD code. Using UDEPs over a finite field, a downlink and an uplink MA systems are proposed, which are called finite-field MA (FFMA) systems. Methods for constructing finite-field UD codes for FFMA systems are provided. An FFMA system can be designed in conjunction with the classical complex-field MA techniques to provide more flexibility and varieties. 




\vspace{0.1in}
{\emph{Index Terms} --- 
Multiple access, AWGN channel, frequency-selective fading channel, finite field, 
element pair, unique decodable code, finite-field multi-access, downlink multi-access, 
uplink multi-access, LDPD codes.} \noindent
\end{abstract}


\newpage
\setcounter{page}{1}
\section{Introduction}
Multiple access (MA) is one of the most important techniques for wireless communications. 
During the past several decades, various MA techniques have been developed for mobile communications
to support various users and services \cite{FAdachi1, ZDing2017_survey, R1}. 
MA is a technique for multiple users to access the same terminal simultaneously, i.e., a base station (BS); or one terminal simultaneously transmit multiple signals to different users. 
There are various types of MA techniques, from orthogonal MA (OMA) to non-orthogonal MA (NOMA) techniques. 
The classical OMA techniques are frequency division multiple access (FDMA), time division multiple access (TDMA), code division multiple access (CDMA), and orthogonal frequency division multiple access (OFDMA). The OMA technique is to assign orthogonal resources to different users, thus, there is no interference among users. However, the spectral efficiency (SE) of OAM is generally equal to one. 


To improve the SE, NOMA has been paid much attention lately, 
since it can provide a higher SE via overloading information using the same time and/or frequency resources to support more users \cite{R2,QY_ISJ_2019,LDai2015,HNiko2013,HNiko2014}. 
In \cite{R2}, the authors proposed a unified transceiver, including transmit and received signal expressions. 
In \cite{QY_ISJ_2019}, a unified multiuser coding framework for multiuser transmission was proposed, and  uniquely-decodable mapping (UDM) was introduced for separating the users at the receiving end(s) without ambiguity.



Besides extensive work on MA techniques, 
work on constructions of uniquely decodable codes (UDCs) for MA channels was investigated in \cite{Liao1972,Kasami1976,Chang1976, Kasami1978,Peterson1979,Chevillat1981,Kasami1983,Van1983,Vanroose1992,Jevtic1992,Fan1995,Khachatrian1998,Bross1998,Cheng2001,Kiviluoto2007,Yu_P2P},
but mainly in integer domain for multiple-access adder channels. 
The essential idea behind UDCs is to extract data from the superimposed signals without ambiguity. 
The achievable data rate of a multiuser UDC is generally higher than any single user data rate, and some UDCs can even correct errors in noisy channels \cite{Kasami1978}. 
How to construct multiuser UDCs is a complicated issue.

   
Most of the current MA techniques are combinations of coding, modulation, interleaving, and orthogonal resources jointly to provide beneficial performances. Because of the superposition of signals from multiple users, the present MA techniques are all processed in complex-field by signal processing, which consume physical resources, i.e., time, frequency, code, and space. It becomes problematic to find new resources from the physical world. Consequently, the MA technique seems to get into a bottleneck. New resources are needed.


In this paper, an \textit{algebraic resource} is proposed to support multiuser transmission. 
This algebraic resource is based on assigning each user an \textit{element pair (EP)} from a finite field GF($p^m$). 
The output bit from each user is mapped into an element in its assigned EP, called the output symbol. 
Then, the output symbols from the users are jointly mapped into a unique element in the same field GF($p^m$) which is then modulated and transmitted. 
When the transmitted signal is received and demodulated at the receiving end, 
the receiver can separate transmitted bits from received superimposed element without ambiguity. 
The EPs assigned to the users are said to form a set of \textit{uniquely decodable EPs (UDEPs)}, 
called a multiuser UD code. Using UDEPs over a finite field, a downlink and an uplink MA systems are presented which are called \textit{finite-field MA (FFMA)} systems. 
The performances of both FFMA systems are analyzed. 
The downlink FFMA system can achieve full diversity gain over fading channels.
The \textit{major difference} between the complex-field and finite-field MA techniques is that
the former one is based on complex-field signal processing, and the latter one is operated in finite-field. 
FFMA can be used in conjunction with the classical complex-field MA techniques to provide more flexibility and varieties.



The remainder of this paper is organized as follows.  
In Section II, two methods for constructing multiuser UD codes over finite fields for MA are presented, one based on prime fields and the other based on extensions of prime fields. 
Section III gives an overview of FFMA systems using finite-fields as resources. 
In Section IV, a downlink FFMA scheme using a multiuser UD code for frequency-selective fading channels is presented. 
In Section V, an uplink FFMA system using an \textit{orthogonal} multiuser UD code for an AWGN multi-access channel is presented. 
In Section VI, the capacity and bit-error-rate (BER) of the proposed uplink FFMA system over an AWGN MA channel are derived. 
In Section VII, simulations of the error performances of the proposed FFMA schemes using a non-binary LDPC codes for error control are given. Section VIII concludes the paper with some remarks.


In this paper, we use $x$, $\mathbf{x}$ and $\mathbf{X}$ as a variable, a vector and a matrix, respectively. 
The symbol $\mathbb {C}$ is used to denote the complex-field. 
The notions $\bigoplus$, $\bigotimes$, $\sum$ and $\prod$ are used to denote modulo-$q$ addition, 
modulo-$q$ multiplication, complex addition, and complex product, respectively.
The symbols $\lceil x \rceil$ and $\lfloor x \rfloor$ denote the smallest integer that is equal to or larger than $x$ and the largest integer that is equal to or smaller than $x$, respectively.
The notation $(a)_q$ stands for modulo-$q$, and/or an element in GF($q$).



%\vspace{0.2in}
\section{Uniquely Decodable Codes over Finite Fields for Multiple Access Channels}

In this section, we present \textit{algebraic unique decodable (UD) codes} constructed based on finite fields. These algebraic UD codes provide an additional resource for multiple access (MA) communications. 
The first part of this section presents a method for constructing UD codes based on \textit{prime fields} and the second part presents a class of \textit{orthogonal} algebraic UD codes which are constructed based on \textit{extension fields} of prime fields. 


\subsection{Construction of UD Codes Based on Prime Fields}

Let GF($p$) = $\{0, 1, \ldots, p-1\}$ be a prime field with $p > 2$.  
Partition the $p - 1$ nonzero elements GF($p$) into $(p - 1)/2$ 
\textit{mutually disjoint element-pairs} (\textit{EPs}), 
each EP consisting of a nonzero element $k$ in $\text{GF}(p)\backslash 0$ and 
its additive inverse $p - k$ (or simply $- k$). 
We call each such pair $(k, p - k)$ an \textit{additive inverse EP (AIEP)}. 
The partition of $\text{GF}(p)\backslash \{0\}$ into AIEPs, denoted by $\mathcal P$, is referred to as 
\textit{AIEP-partition}.


Let $J$ be a positive integer less than or equal to $(p - 1)/2$, 
i.e., $1 \le J \le (p - 1)/2$, and let $C_1, C_2, \ldots, C_{J}$ be $J$ AIEPs in $\mathcal P$. 
For $1 \le j \le J$ and $1 \le t_j < p$, with $j$ as an index number of $J$ AIEPs and $t_j$ as an integer element in GF($p$), let $(t_j, p - t_j)$ denote the AIEP in $C_j$, i.e., $C_j = (t_j, p - t_j)$. 
The $J$ AIEPs $C_1, C_2, \ldots, C_J$ form a partition of a subset of $2J$ elements in GF$(p) \backslash \{0\}$, a sub-partition of $\mathcal P$. 
Let $\mathcal C$ denote the set $\{C_1, C_2, \ldots, C_J\}$, 
i.e., ${\mathcal C} = \{C_1, C_2, \ldots, C_J \}$.  



Let $(u_1, u_2, \ldots, u_J)$  be a $J$-tuple over GF($p$) in which the $j$-th component $u_j$ for $1 \le j \le J$, is an element from $C_j$. The $J$-tuple $(u_1, u_2, \ldots, u_J)$ is an element in the \textit{Cartesian product} $C_1 \times C_2 \times \ldots \times C_J$ of the AIEPs in $\mathcal C$. 
The modulo-$p$ sum $\tau = \bigoplus_{j=1}^{J} u_j$ of the $J$ components in $(u_1, u_2, \ldots, u_{J})$ 
is called the \textit{sum-pattern} of the $J$-tuple $(u_1, u_2, \ldots, u_{J})$ 
which is an element in GF($p$). 
The $J$-tuple $(p-u_1, p-u_2, \ldots, p-u_{J})$ is also an element in $C_1  \times C_2 \times \ldots \times C_{J}$. 
The sum-pattern of $(p-u_1, p-u_2, \ldots, p-u_{J})$ is 
$p - \tau = p -  \bigoplus_{j=1}^{J} u_j$. 
If $\tau = 0, p - \tau = 0$ (modulo $p$), i.e., if the \textit{sum-pattern} of $(u_1, u_2, \ldots, u_{J})$ is the zero element $0$ in GF($p$), 
the sum-pattern of $(p-u_1, p-u_2, \ldots, p-u_{J})$ is also the zero element $0$ in GF($p$).
Let $(u_1, u_2, \ldots, u_{J})$ and $(u_1', u_2', \ldots, u_{J}')$ be \textit{any two} $J$-tuples in $C_1  \times C_2 \times \ldots \times C_{J}$. 
If $\bigoplus_{j=1}^{J} u_j \neq \bigoplus_{j=1}^{J} u_j'$, 
then a sum-pattern uniquely specifies a $J$-tuple in $C_1  \times C_2 \times \ldots \times C_{J}$. 
That is to say that the mapping 
\begin{equation*}
(u_1, u_2, \ldots, u_{J}) \longleftrightarrow \bigoplus_{j=1}^{J} u_j
\end{equation*} 
is one-to-one mapping. 

In this case, given the sum-pattern $\tau = \bigoplus_{j=1}^{J} u_j$, 
we can uniquely recover the $J$-tuple $(u_1, u_2, \ldots, u_{J})$ \textit{without ambiguity}. 
We say that the Cartesian product $C_1 \times C_2 \times \ldots \times C_{J}$  
(or ${\mathcal C} = \{C_1, C_2, \ldots, C_{J}$) has a \textit{unique sum-pattern mapping (USPM) structural property}.


A set ${\mathcal C} = \{C_1, C_2, \ldots, C_{J}\}$ of $J$ AIEPs with the USPM structural property can be used as a type of \textit{finite field resource} for MA communications. 
Suppose in a MA communication system, 
the $J$ AIEPs $C_1, C_2, \ldots, C_{J}$ in ${\mathcal C}$ are assigned to $J$ users for bit mapping. 
If each user transmits a symbol from its assigned AIEP, the transmitter multiplexes the $J$ transmitted symbols, forms the sum-pattern $\tau=\bigoplus_{j=1}^{J} u_j$, 
and transmits the sum-pattern over an MA channel. 
Assume that the channel is noiseless.  
When the receiver(s) receives the sum-pattern $\tau=\bigoplus_{j=1}^{J} u_j$, 
the $J$ transmitted symbols in $(u_1, u_2, \ldots, u_{J})$ can be uniquely recovered from the sum-pattern $\tau$ without ambiguity.



If we view each $J$-tuple $(u_1, u_2, \ldots, u_{J})$ in $C_1 \times C_2 \times \ldots \times C_{J}$ with the USPM structural property as a \textit{$J$-user codeword}, then ${\mathcal C} = C_1 \times C_2 \times \ldots \times C_{J}$ forms a $J$-user code over GF($p$) with $2^J$ codewords.  
We call ${\mathcal C} = C_1 \times C_2 \times \ldots \times C_{J}$ a $J$-user \textit{uniquely decodable (UD) AIEP code} over GF($p$), simply a $J$-user UDAIEP code. 
When this code is used for an MA communication system with $J$ users, 
the $j$-th component $u_j$ in a codeword $(u_1, u_2, \ldots, u_{J})$ is the symbol to be transmitted by the $j$-th user. 
Applications of UDAIEP codes to MA communications will be presented in later sections of this paper. 



For a $J$-user UDAIEP code $\mathcal C$ over GF($p$), 
no codeword can have sum-pattern equal to the zero element $0$ of GF($p$). 
Since the sum-patterns of the $2^J$ codewords in $\mathcal C$ must be distinct elements in GF($p$), 
$2^{J}$ must be less than or equal to $p - 1$, 
i.e., $2^J \le p - 1$. 
Hence, the number $J$ of users for an UDAIEP code over GF($p$) is upper bounded as follows:
\begin{equation} \label{e2.1}
  J \le \log_2 (p-1).
\end{equation}
Summarizing the results developed above, we have the following theorem.

\begin{theorem}
The Cartesian product of $J$ AIEPs over a prime field GF($p$) with $p > 2$ is a $J$-user UDAIEP code 
if and only if the sum-patterns of all its $2^J$ codewords are different nonzero elements in GF($p$) 
with $J$ upper bounded by $\log_2 (p-1)$.
\end{theorem}


%Table 1 Two UDAIEP codes given in Example 1.
\begin{figure}[t]
  \centering
  \subfigure[AIEPs of GF(5)] {
    \includegraphics[width=0.4\textwidth]{GF5.pdf}}
  \subfigure[AIEPs of GF(17)] {
    \includegraphics[width=0.8\textwidth]{GF17.pdf}}
  \caption{A diagram in table-form of Example 1. (a) Two AIEPs $C_1 = (1, 4)$ and $C_2 = (2, 3)$ are constructed for GF(5); (b) Four AIEPs $C_1 = (1,16), C_2 =(2,15), C_3 = (4,13)$, and $C_4=(8,9)$ are constructed for GF(17).}
\end{figure}


For better understand, we give an example of UDAIEP codes.

\textbf{Example 1:} 
Consider the prime field GF($5$). Using this prime field, two AIEPs $C_1 = (1, 4)$ and $C_2 = (2, 3)$ can be constructed. The Cartesian product $C_1 \times C_2$ of $C_1$ and $C_2$ satisfies the necessary and sufficient condition given by Theorem 1 as shown in Fig. 1 (a). 
Hence, $C_1 \times C_2$ is a 2-user UDAIEP code over GF($5$).


Suppose we use the prime field GF($17$) for UD code construction. 
Eight AIEPs over GF($17$) can be constructed. 
They are $(1, 16), (2, 15), (3, 14), (4, 13), (5, 12), (6, 11), (7, 10), (8, 9)$. 
Since $\log_2 (17-1) = 4$, there are at most $4$ AIEPs whose Cartesian product satisfies the necessary and sufficient condition given by Theorem 1. 
We find $4$ AIEPs whose Cartesian product satisfies the necessary and sufficient condition given by Theorem 1. These four AIEPs are $C_1 = (1, 16), C_2 = (2, 15), C_3 = (4, 13)$, and $C_4 = (8, 9)$ as shown in Fig. 1 (b). The Cartesian product ${\mathcal C} = C_1 \times C_2 \times C_3 \times C_4$ is a 4-user UDAIEP code over GF(17). $\blacktriangle  \blacktriangle$





\subsection{Construction of UD Codes based on Extension Fields of Prime Fields}

Let $m$ be a positive integer and GF($p^m$) be the extension field of the prime field GF($p$). 
The extension field GF($p^m$) is constructed based on a primitive polynomial 
${\bf g}(X) = g_0 + g_1 X + g_2 X^2 + \ldots + g_m X^m$
of degree $m$ with coefficients from GF($p$) which consists of $p^m$ elements and contains GF($p$) as a subfield \cite{Shu2009}.



Let $\alpha$ be a primitive element in GF($p^m$). 
Then, the powers of $\alpha$, namely $\alpha^{-1} = 0, \alpha^0 = 1, \alpha, \alpha^2, \ldots, \\
\alpha^{(p^m - 2)}$, give all the $p^m$ elements of GF($p^m$). 
Each element $\alpha^j$, with $j = -1, 0, \ldots, p^m - 2$, in GF($p^m$) can be expressed as a linear sum of $\alpha^0 = 1, \alpha, \alpha^2, \ldots, \alpha^{(m - 1)}$ with coefficients from GF($p$) as follows:
\begin{equation} \label{e2.2}
    \alpha^j = a_{j,0} + a_{j,1} \alpha + a_{j,2} \alpha^2 + \ldots + a_{j,m-1} \alpha^{(m-1)}.      
\end{equation}
From (\ref{e2.2}), we see that the element $\alpha^j$ can be uniquely represented by the $m$-tuple 
$(a_{j,0}, a_{j,1}, a_{j,2},\ldots, a_{j,m-1})$ over GF($p$). 
Hence, an element in GF($p^m$) can be expressed in three forms, 
namely \textit{power, polynomial and $m$-tuple forms}.


The sum of two elements 
$\alpha^j = a_{j,0} + a_{j,1} \alpha + a_{j,2} \alpha^2 + \ldots + a_{j,m-1} \alpha^{(m - 1)}$ and 
$\alpha^k = a_{k,0} + a_{k,1} \alpha + a_{k,2} \alpha^2 + \ldots + a_{k,m-1} \alpha^{(m - 1)}$ 
is equal to
\begin{equation}
   \alpha^j + \alpha^k = (a_{j,0} + a_{k,0}) + (a_{j,1} + a_{k,1}) \alpha + \ldots 
   + (a_{j,m-1} + a_{k,m-1}) \alpha^{(m - 1)}.    
\end{equation}


The $m$-tuple representation of the sum $\alpha^j + \alpha^k$ is
\begin{equation}
    \left((a_{j,0} + a_{k,0}), (a_{j,1} + a_{k,1}), \ldots, (a_{j,m-1} + a_{k,m-1})\right). 
\end{equation}
For $0 \le i < m$, if $(a_{j,i}, a_{k,i})$ is an additive inverse pair over GF($p$) or a pair of zero elements $(0, 0)$, then $a_{j,i} + a_{k,i} = 0$ and $\alpha^j + \alpha^k  = 0$, 
i.e., $\alpha^j$ and $\alpha^k$ are additive inverse to each other. 
In this case, $(\alpha^j, \alpha^k)$ forms an AIEP over GF($p^m$). 
If $(a_{j,i}, a_{k,i})$ is a nonzero AIEP over GF($p$), then 
\begin{equation*}
\alpha^i  (a_{j,i}, a_{k,i}) \triangleq (a_{j,i}  \alpha^i, a_{k,i} \alpha^i),
\end{equation*}
is an AIEP over GF($p^m$).


Let ${\mathcal C} = \{C_1, C_2, \ldots, C_{J}\}$ be a set of $J$ AIEPs over GF($p$) 
with the \textit{USPM} structural property. 
For $1 \le j \le J$ and $0 \le i < m$, 
let $C_j = (t_j, p - t_j)$ with $t_j \in {\rm GF}(p) \backslash{0}$ and 
let $\psi_i(t_j, p - t_j)$ denote the AIEP $\alpha^i C_j = (t_j \alpha^i, (p - t_j) \alpha^i)$ over GF($p^m$), i.e., $\psi_i(t_j, p - t_j) = \alpha^i C_j = (t_j \alpha^i, (p - t_j) \alpha^i)$. 
Then, for $0 \le i < m$,
\begin{equation} \label{e2.5}
\Psi_i = \{\psi_i(t_1, p-t_1), \psi_i(t_2, p-t_2), \ldots, \psi_i(t_J, p-t_J) \}
\end{equation}
is a set of $J$ AIEPs over GF($p^m$) with the USPM structural property.
Hence, the Cartesian product
\begin{equation} \label{e2.6}
\Psi_i \triangleq \psi_i(t_1, p-t_1) \times \psi_i(t_2, p-t_2) \times \ldots \times \psi_i(t_J, p-t_J)
\end{equation}
of the $J$ AIEPs in $\Psi_i$ forms a $J$-user UDAIEP code over GF($p^m$) with $2^J$ codewords,
each consisting of $2^J$ nonzero elements in GF($p^m$).
With $i = 0, 1, \ldots, m-1$, we can form $m$ $J$-user UDAIEP codes over GF($p^m$),
$\Psi_0, \Psi_1, \ldots, \Psi_{m-1}$, which are \textit{mutually disjoint}, 
i.e., $\Psi_k \bigcap \Psi_i \neq \emptyset$ for $k \neq i$ and $0 \le k, i < m$.


If we represent an element in GF($p^m$) as an $m$-tuple over GF($p$), 
the two additive inverse elements in the pair $\psi_i(t_j, p - t_j) = (t_j \alpha^i, (p - t_j) \alpha^i)$ is a pair of two $m$-tuples with nonzero components, $t_j$ and $p - t_j$, 
in the $i$-th location, respectively, and $0$s in all the other locations, 
i.e., $(0, 0, \ldots, t_j, 0, \ldots, 0)$ and $(0, 0,\ldots, p - t_j, 0, \ldots, 0)$. 
Hence, in $m$-tuple form, all the $J$ AIEPs in $\Psi_i$ have either $t_j$ and $p - t_j$ in the $i$-th locations and $0$'s elsewhere with $1 \le j \le J$, i.e., 

\begin{small}
\begin{equation*}
  \begin{aligned}
\psi_i(t_1,p-t_1) &= \left((0, 0, \ldots, t_1, 0, \ldots, 0), (0, 0, \ldots, p-t_1, 0, \ldots, 0) \right),\\
\psi_i(t_2,p-t_2) &= \left((0, 0, \ldots, t_2, 0, \ldots, 0), (0, 0, \ldots, p-t_2, 0, \ldots, 0) \right),\\
&\vdots\\
\psi_i(t_J,p-t_J) &= \left((0, 0, \ldots, t_J, 0, \ldots, 0), (0, 0, \ldots, p-t_J, 0, \ldots, 0) \right).\\
  \end{aligned}
\end{equation*}
\end{small}

From $m$-tuple (vector) point of view, $\Psi_0, \Psi_1,\ldots, \Psi_{m-1}$ are \textit{orthogonal} to each other, and they form $m$ orthogonal sets of AIEPs over GF($p$). 
Hence, $\Psi_0, \Psi_1, \ldots, \Psi_{m-1}$ give $m$ orthogonal $J$-user UDAIEP codes over GF($p^m$) (or over GF($p$) in $m$-tuple form). 
In $m$-tuple form, each codeword in $\Psi_i$ consists of $J$ $m$-tuples over GF($p$), 
each consisting of a single nonzero element from GF($p$) in the same location.


The union $\Psi \triangleq \Psi_0 \bigcup \Psi_1 \bigcup \ldots \bigcup \Psi_{m-1}$ forms a $Jm$-user orthogonal UDAIEP code over GF($p^m$) with $2^{Jm}$ codewords over GF($p^m$) (or over GF($p$) in $m$-tuple form). The code $\Psi$ has a diagonal structure as shown below:
\begin{equation} \label{e2.7}
{\bf \Psi} = {\rm diag}({\Psi}_0, {\Psi}_1, \ldots, {\Psi}_{m-1}),
\end{equation}
which is an $m \times m$ diagonal array with ${\Psi}_0, {\Psi}_1, \ldots, {\Psi}_{m-1}$ lying on its main diagonal and zeros elsewhere. 


From (\ref{e2.7}), $\Psi$ can be viewed as a \textit{cascaded} UDAIEP code obtained by cascading the $m$ $J$-user UDAIEP codes ${\Psi}_0, {\Psi}_1, \ldots, {\Psi}_{m-1}$. 
We call ${\Psi}_0, {\Psi}_1, \ldots, {\Psi}_{m-1}$ the \textit{constituent codes} of $\Psi$. 
The UDAIEP code $\Psi$ can serve $J m$ users of a FFMA system in conjunction with TDMA which will be presented in a later section.


The number of users in an orthogonal $J m$-user UDAIEP code over GF($p^m$) is upper bounded by 
$m\log_2(p-1)$.
In the following, we give an example in constructing an orthogonal multiple-user UDAIEP code over an extension field of a prime field. 
\begin{figure}[t]
  \centering
  \label{Fig3} 
  \includegraphics[width=0.5\textwidth]{Fig3.pdf}
  \caption{Finite-field resource blocks of GF($5^6$), where the AIEPs are 
  $\psi_{0}(1,4), \psi_{0}(2,3), \psi_{1}(1,4), \psi_{1}(2,3), \psi_{2}(1,4), \psi_{2}(2,3)$,
  $\psi_{3}(1,4), \psi_{3}(2,3), \psi_{4}(1,4), \psi_{4}(2,3), \psi_{5}(1,4), \psi_{5}(2,3)$.
        }
\end{figure}

{\textbf Example 2:}
For $p = 5$ and $m = 6$, consider the extension field GF($5^6$) of the prime field GF($5$). 
As shown in Example 1, using the prime field GF($5$), two AIEPs $C_1 = (1, 4)$ and $C_2 = (2, 3)$ can be constructed whose Cartesian product $C_1 \times C_2$ is 2-user UDAIEP code over GF($5$). 
Based on this code, twelve UDAIEP codes over GF($5^6$) can be formed. 
They form 6 orthogonal groups,

\begin{small}
\begin{equation*}
  \begin{aligned}
    (\psi_0(C_1), \psi_0(C_2)) = (\psi_0(1, 4), \psi_0(2, 3)),
    (\psi_1(C_1), \psi_1(C_2)) = (\psi_1(1, 4), \psi_1(2, 3))\\
    (\psi_2(C_1), \psi_2(C_2)) = (\psi_2(1, 4), \psi_2(2, 3)),
    (\psi_3(C_1), \psi_3(C_2)) = (\psi_3(1, 4), \psi_3(2, 3))\\
    (\psi_4(C_1), \psi_4(C_2)) = (\psi_4(1, 4), \psi_4(2, 3)),
    (\psi_5(C_1), \psi_5(C_2)) = (\psi_5(1, 4), \psi_5(2, 3))\\
  \end{aligned}
\end{equation*}
\end{small}

The Cartesian products of these 6 groups give 6 orthogonal 2-user UDAIEP codes ${\Psi}_0, {\Psi}_1, {\Psi}_2, {\Psi}_3, {\Psi}_4, {\Psi}_5$ over GF($5^6$) (or over GF($5$) in $6$-tuple from). 
Their union gives an orthogonal 12-user UDAIEP code over GF($5^6$) with $2^{12}$ = 4096 codewords. 
The orthogonal structure of the code is shown in Fig. 2.
$\blacktriangle  \blacktriangle$


\subsection{Orthogonal Encoding of an Error-Correcting Code}
In a latter section, UDAIEP codes over finite fields will be used in conjunction with nonbinary error-correcting codes for error control in FFMA communication systems. 
In the following, we present an encoding of an error-correcting code over GF($p^m$) in a form to match UD encoding of multiple users.


If we represent each element $\alpha^j = a_{j,0} + a_{j,1} \alpha + \ldots + a_{j,m-1} \alpha^{(m - 1)}$ 
in GF($p^m$) by an $m$-tuple $(a_{j,0}, a_{j,1}, \ldots, a_{j,m-1})$ over GF($p$), 
the field GF($p^m$) is a vector space ${\bf V}_p(m)$ over GF($p$) of dimension $m$. 
Each vector in ${\bf V}_p(m)$ is $m$-tuple over GF($p$). 
For $0 \le i < m$, let $\epsilon_i = (0, 0, \ldots, 1, 0,\ldots, 0)$ be an $m$-tuple with a 1-component in the $i$-th location and 0s elsewhere. 
The $m$ $m$-tuples $\epsilon_0, \epsilon_1, \epsilon_2, \ldots, \epsilon_{m-1}$ form an \textit{orthogonal} (or \textit{normal basis}) of ${\bf V}_p(m)$. 
Every $m$-tuple $(a_{j,0}, a_{j,1}, \ldots, a_{j,m-1})$ in ${\bf V}_p(m)$ is a linear combination of 
$\epsilon_0, \epsilon_1, \epsilon_2, \ldots, \epsilon_{m-1}$,
\begin{equation}
   (a_{j,0}, a_{j,1}, \ldots, a_{j,m-1}) = a_{j,0} \epsilon_0 + a_{j,1} \epsilon_1 + \ldots 
   + a_{j,m-1} \epsilon_{m-1},            
\end{equation}
denoted by $\bigoplus_{i=0}^{m-1} a_{j,i} \epsilon_i$, i.e.,
\begin{equation}
    \bigoplus_{i=0}^{m-1} a_{j,i} \epsilon_i = a_{j,0} \epsilon_0 + a_{j,1} \epsilon_1 + \ldots + a_{j,m-1} \epsilon_{m-1}.
\end{equation}
Hence, the $m$-tuple representation of the element 
$\alpha^j = a_{j,0} + a_{j,1} \alpha + \ldots + a_{j,m-1} \alpha^{(m - 1)}$ in GF($p^m$) is 
$a_{j,0} \epsilon_0 + a_{j,1} \epsilon_1 + \ldots + a_{j,m-1} \epsilon_{m-1}$.


For a positive integer $K$, 
let $u = (u_0, u_1, \ldots, u_k, \ldots, u_{K-1})$ be a $K$-tuple over GF($p^m$). 
For $0 \le k < K$, let $(a_{k,0}, a_{k,1},\ldots, a_{k,m-1})$ be the $m$-tuple representation of the $k$-th component $u_k$ of ${\bf u}$. 
For $0 \le i < m$, we form the following $K$-tuple over GF($p$)
\begin{equation} \label{e2.10}
    {\bf a}_i = (a_{0,i}, a_{1,i}, \ldots, a_{k,i},\ldots, a_{K-1,i}),                           
\end{equation}
where $a_{k,i}$ is the $i$-th component of the $m$-tuple $(a_{k,0}, a_{k,1}, \ldots, a_{k,m-1})$ representation of the $k$-th component $u_k$ of $\bf u$. 
Define the following $K$ $m$-tuples over GF($p$)
\begin{equation} \label{e2.11}
    {\bf a}_i \bullet \epsilon_i \triangleq (a_{0,i} \epsilon_i, a_{1,i} \epsilon_i, \ldots, a_{k,i} \epsilon_i, \ldots, a_{K-1,i} \epsilon_i).   
\end{equation}


Then, the $K$-tuple $\bf u$ over GF($p^m$) can be decomposed into the following ordered sequence of $K$ 
$m$-tuples over GF($p$),
\begin{equation} \label{e2.12}
    [{\bf u}]_m \triangleq  
    {\bf a}_0 \bullet \epsilon_0 \oplus {\bf a}_1 \bullet \epsilon_1 \oplus \ldots \oplus 
    {\bf a}_i \bullet \epsilon_i \oplus \ldots \oplus {\bf a}_{m-1} \bullet \epsilon_{m-1},    
\end{equation}
where the $k$-th $m$-tuple is 
$a_{k,0} \epsilon_0 + a_{k,1} \epsilon_1 + \ldots + a_{k,m-1} \epsilon_{m-1}$. 
We call $[{\bf u}]_m$ the \textit{orthogonal $m$-tuple decomposition of $\bf u$}.


A sequence $\bf u$ of $K$ codewords in the $Jm$-user orthogonal UDAIEP code $\Psi$ over GF($p^m$) can be decomposed into an ordered sequence $[{\bf u}]_m$ of $J$ $m$-tuples over GF($p$) in the orthogonal form of (\ref{e2.12}). 
This orthogonal form will be used in presentation an uplink FFMA scheme in a later section.


Let $\bf G$ be the generator matrix of a $p^m$-ary $(N, K)$ linear block code $\bf W$ over GF($p^m$) with $m \ge 2$. Let ${\bf g}_0, {\bf g}_1, \ldots, {\bf g}_{K-1}$ be the $K$ rows of $\bf G$, each an $N$-tuple over GF($p^m$). 
Let ${\bf u} = (u_0, u_1, \ldots, u_k, \ldots, u_{K-1})$ be a message over GF($p^m$) whose orthogonal decomposition $[{\bf u}]_m$ is given by (\ref{e2.12}). 
If we encode this message $\bf u$ into a codeword $\bf v$ in $\bf W$ using the generator $\bf G$, 
we have
\begin{equation*} 
  {\bf v} = (v_0, v_1, v_2, \ldots, v_{N-1}) = u_0 {\bf g}_0 \oplus u_1 {\bf g}_1 \oplus \ldots 
  \oplus u_{K-1} {\bf g}_{K-1}.
\end{equation*}
The orthogonal $m$-tuple decomposition of $\bf v$ is 
\begin{small}
\begin{equation} \label{e2.13}
  \begin{aligned}
\left[{\bf v}\right]_m =& 
              \left(a_{0,0}  \epsilon_0 \oplus a_{0,1} \epsilon_1 \oplus \ldots \oplus 
                     a_{0,m-1} \epsilon_{m-1} \right) {\bf g}_0 \oplus
               \left(a_{1,0}  \epsilon_0 \oplus a_{1,1} \epsilon_1 \oplus \ldots \oplus 
                     a_{1,m-1} \epsilon_{m-1} \right) {\bf g}_1 \oplus \ldots \oplus\\
               &\left(a_{K-1,0} \epsilon_0 \oplus a_{K-1,1} \epsilon_1 \oplus \ldots \oplus 
                     a_{K-1,m-1} \epsilon_{m-1} \right) {\bf g}_{K-1} \\
            =& \left(a_{0,0} \epsilon_0 {\bf g}_0 \oplus a_{1,0} \epsilon_0 {\bf g}_1 \oplus \ldots 
                    \oplus a_{K-1,0} \epsilon_0 {\bf g}_{K-1}\right) \oplus
               \left(a_{0,1} \epsilon_1 {\bf g}_0 \oplus a_{1,1} \epsilon_1 {\bf g}_1 \oplus \ldots 
                    \oplus a_{K-1,1} \epsilon_1 {\bf g}_{K-1}\right) \oplus \ldots \oplus\\
              &\left(a_{0,m-1} \epsilon_{m-1} {\bf g}_0 \oplus a_{1,m-1} \epsilon_{m-1} {\bf g}_1 \oplus \ldots \oplus a_{K-1,m-1} \epsilon_{m-1} {\bf g}_{K-1}\right) \\
            =& ({\bf a}_0 {\bf G}) \epsilon_0 \oplus ({\bf a}_1 {\bf G}) \epsilon_1 \oplus \ldots \oplus
               ({\bf a}_{m-1} {\bf G}) \epsilon_{m-1} \\
            =& {\bf v}_0 \epsilon_0 \oplus {\bf v}_1 \epsilon_1 \oplus \ldots \oplus 
               {\bf v}_{m-1} \epsilon_{m-1},     
  \end{aligned}
\end{equation}
\end{small}
where ${\bf v}_i = {\bf a}_i {\bf G}$ is the codeword of ${\bf a}_i$ for $0 \le i < m$.

The vector $[{\bf v}]_m$ is called the codeword $\bf v$ of the message $\bf u$ in orthogonal form.
The above encoding is referred to as orthogonal encoding.



%\vspace{0.2in}
\section{An overview of FFMA}
This section gives an introduction of FFMA  
which distinguishes users by using UDAIEPs over finite-fields, i.e., using finite-fields as resources.


\subsection{Complex-Field and Finite-Field Multiple-Access Systems} 

To begin with, we compare the differences between a complex-field (CF) MA system (CFMA) and an FFMA system. Suppose $J$ users are supported in each system.


In an orthogonal CFMA (O-CFMA) system, an information bit, $(0)_2$ or $(1)_2$, at the output of each user is first modulated into a complex symbol and assigned to a specific orthogonal physical resource, such as a time slot, frequency, or other types, where the subscript ``$2$'' of $(0)_2$ or $(1)_2$ stands for binary.
Then, the $J$ modulated orthogonal signals are multiplexed and transmitted. 
At the receiving end, the receiver separates the $J$ transmitted symbols and recovers the bits transmitted from the $J$ users, via the assigned physical resources.


In an orthogonal FFMA (O-FFMA) system, the key is to establish a relationship between different fields. 
In general, there are four different forms of data: 
(1) the information bits $(0)_2$ and $(1)_2$ which are regarded as two elements in the binary field GF(2); 
(2) a finite-field GF($q$) with $q = p^m$, for fulfilling FFMA; 
(3) a channel code over a field GF($Q$) for combating noise, where GF($Q$) may be different from GF($q$); and 
(4) modulated signals over a complex-field ${\mathbb C}$, which is mainly determined by the modulation order ${\mathcal M}$. 
The entire transmission of a FFMA system is based on the transform between different fields.


In the following, we use a simple case to illustrate the information transmission of an FFMA system. Complete FFMA systems and their performance analysis will be presented in the next three sections.


Firstly, the bit to be transmitted from a user is mapped into an element of GF($q$) by a transform function $\rm F_{N2NB}$, where the subscript ``B2NB'' of $\rm F_{N2NB}$ stands for \textit{binary to nonbinary transform}. 
Since each output bit of a user is either $(0)_2$ or $(1)_2$, 
two different nonbinary elements in GF($q$) are required for realizing $\rm F_{N2NB}$, 
i.e., $(0)_2 \to \alpha^{k_1}$ and $(1)_2 \to \alpha^{k_2}$, where $\alpha^{k_1}, \alpha^{k_2} \in$ GF($q$) and $k_1 \neq k_2$.



For the proposed FFMA, $(\alpha^{k_1}, \alpha^{k_2})$ forms an AIEP and each user is assigned to a unique AIEP in a collection $\mathcal C$ of $J$ AIEPs which have the USPM structural properties as presented in Section II, i.e., we can allocate different AIEPs in $\mathcal C$ to different users. 
After $\rm F_{B2NB}$ transform, each bit of a user is uniquely mapped to a nonbinary element in GF($q$).
After the B2NB mapping, we obtain $J$ nonbinary elements in GF($q$). 
Then, the sum of these nonbinary elements is modulated into one or several signals, which is determined by the modulation order.


At the receiver, after demodulation, a nonbinary element $\alpha^k$ in GF($q$) is obtained. 
We call $\alpha^k$ the received element.
From a uniquely decodable sum-pattern table and the inverse function $\rm F_{NB2B}$ of $\rm F_{B2NB}$, 
the transmitted bits from the $J$ users are uniquely recovered from the received element in GF($q$). 
The subscript ``NB2B'' of $\rm F_{NB2B}$ stands for \textit{nonbinary to binary transform}.


To explain the $\rm F_{B2NB}$ transform processing, let us consider the 2-user UDAIEP code over GF(5) given in Example 1 which consists of two AIEPs $C_1 = (1, 4)$ and $C_2 = (2, 3)$ whose sum-patterns are given in Fig. 1 (a). 
In a 2-user FFMA, we assign $C_1$ and $C_2$ to users 1 and 2, respectively. 
The two output bits $(0)_2$ and $(1)_2$ of 1-user are mapped into two elements $(1)_5$ and $(4)_5$ in GF(5), respectively, where the subscript ``5'' of $(1)_5$ or $(4)_5$ stands for element in GF($5$),
i.e., $(1)_5$ and $(4)_5$ are elements 1 and 4 in GF($5$) which are additive inverse to each other.
For user-2, the two output bits $(0)_2$ and $(1)_2$ are mapped into two elements $(2)_5$ and $(3)_5$ in GF($5$), respectively, which are additive inverse to each other. 
Based on the uniquely decodable sum-pattern table given by Fig. 1 (a), we can recover the bit information of the two users from a sum-pattern appearing in the sum-pattern table. 
For example, if the received sum-pattern of the two-users in GF(5) is $(3)_5$, 
the nonbinary elements of users 1 and 2 are $(1)_5$ and $(2)_5$, respectively, which indicate the bits from users 1 and 2 are $(0)_2$ and $(0)_2$, respectively.


\subsection{Classification}

Unlike the CFMA, the resources of an FFMA system are AIEPs that consist of integers, elements from a prime field GF($p$) or powers of a primitive element from an extension field GF($p^m$) of a prime field GF($p$).
In general, there are two types of FFMA, orthogonal FFMA (O-FFMA) and non-orthogonal FFMA (NO-FFMA).


\subsubsection{Orthogonal FFMA}


An FFMA system of GF($p^m$) is said to be orthogonal if each user in the system is assigned to a unique AIEP in a uniquely decodable AIEP-collection over GF($p^m$) and the number of user served by the system is equal to or smaller than $m \log_2 (p-1)$. We call such a system an orthogonal FFMA (O-FFMA) system.



Consider the orthogonal UDAIEPs over GF($5^6$) constructed in Example 2. 
Suppose we assign the 3 orthogonal UDAIEPs $\psi_0(1,4)$, $\psi_0(2,3)$ and $\psi_3(1,4)$ to users 1, 2, and 3, respectively, for an O-FFMA to support 3 users. 
In this case, the 6-tuple outputs of user-1 and user-2 have nonzero elements in the same location $i = 0$ but they come from two different UDAIEPs, $C_1 = (1, 4)$ and $C_2 = (2,3)$ over GF($5$). 
Hence, the two nonzero elements can be separated from their 6-tuple sum-pattern. 
User-1 and user-3 are assigned to the same UDAIEP $C_1 = (1,4)$ over GF(5) but in two different locations, 
$0$ and $3$, in their 6-tuple outputs, thus, the output symbols of the two users can be uniquely separated from their 6-tuple sum-pattern. 
Since user-2 and user-3 are assigned to two different UDAIEPs in two different locations in their $6$-tuple outputs, obviously their output symbols can be uniquely separated from their $6$-tuple sum-pattern.



Consider the FFMA system based on the most popular field GF($2^m$), 
an extension field of the binary field GF($2$). 
In this case, there is only one EP over GF($2$), not an additive inverse pair, defined by $C=(0,1)$.  
Hence, the number of orthogonal UDEPs over GF($2^m$) is equal to $m$. 
For $0 \le i < m$, let $\psi_i(0,1)$ denote the EP $C = (0, 1)$ assigned to the $i$-th location of an $m$-tuple. If we assign $\psi_i(0, 1)$ to the $i$-th user of an $m$-user FFMA, 
then this FFMA is a type of TDMA in finite field, in which the outputs of the $m$ users completely occupy $m$ locations in an $m$-tuple (similarly to the $m$ time slots).



\subsubsection{Non-orthogonal FFMA}
For a given finite-field GF($p^m$), 
if the number of users to be served in an FFMA system is larger than $m \log_2(p-1)$, 
then the FFMA system is defined as a \textit{non-orthogonal FFMA (NO-FFMA)} system. 
For an NO-FFMA system, there exists ambiguity due to number of users greater than the upper bound on the number of UDAIEPs. Nevertheless, the spectral efficiency (SE) of an NO-FFMA system is in general higher than the O-FFMA scenario.


\subsection{Downlink and Uplink FFMA}

A multiple-access system can be classified into two transmission scenarios, downlink and uplink. 
In the downlink case, transmission is from a base station (BS) to multiple users and in the uplink case, transmission is from multiple users to a BS.


For a downlink FFMA system, 
all the users' output bit-sequences are first multiplexed into a nonbinary UD vector $\bf u$ over GF($q$). 
Then, the multiplexed vector $\bf u$ is encoded, modulated, and transmitted. 
At the receiving end, after demodulation, the receiver separates users' bit-sequences, bit by bit,
through a sum-pattern decoding table. 
If we encode the multiplexed vector $\bf u$ by a channel code, 
then all the users' output bit-sequences are interleaved into a sequence in the code/time domain. 
In this case, all the users can achieve the same time/frequency diversity gain, 
which is appealing for a fading channel. 
In general, the design of a downlink FFMA system can be easily realized. We can use various finite-field resource allocation algorithms to satisfy the quality of service (QoS) requirement.


Uplink multiuser transmission is generally much more complex than a downlink case, 
since the signals from different users are transmitted through an MA channel and then arrived at the BS. 
When users' signals are transmitted through fading channels, the detection algorithms become extremely complex.


One of the core issues in an uplink FFMA system is to find a relationship between the complex-field sum-patterns and the sum-patterns over a finite field GF($q$). 
Based on this relationship, we determine a transform function $\rm F_{C2F}$ which transforms a complex-field sum-pattern uniquely into a sum-pattern over GF($q$). 
The subscript of $\rm F_{C2F}$ stands for \textit{complex-field to finite-field transform}. 
In a later section, we present an uplink FFMA system using the field GF($2^m$) for an AWGN MA channel based on a designed $\rm F_{C2F}$ function.


An FFMA system can be designed in conjunction with classical complex-field multiple-access techniques to provide more flexibility and variety.




%\vspace{0.2in}
\section{A downlink FFMA system for Frequency-Selective Fading channels}

This section presents a downlink FFMA system of a prime field GF($p$), $p > 2$,
for frequency-selective fading channels. 
Suppose the system is to support $J$ users with $J \le \lfloor \log_2(p-1) \rfloor$. 
For such an FFMA system, we choose a $J$-user UDAIEP code $\mathcal C$ over GF($p$) with $J$ constituent AIEP codes $C_1, C_2, \ldots, C_j, \ldots, C_{J}$ as the multiple access resource. 
The $j$-th AIEP $C_j$ is assigned to the $j$-th user. 
A block diagram for such a downlink $J$-user FFMA system is shown in Fig. 3.



\vspace{-0.1in}
\begin{figure}[t]
  \centering
  \label{Fig_DL}
  \includegraphics[width=0.7\textwidth]{Fig_DL.pdf}
  \caption{System model of a downlink FFMA system for frequency-selective fading channels, 
  where $\rm F_{B2NB}$ and $\rm F_{NB2N}$ stand for ``binary to nonbinary transform'' and ``nonbinary to binary transform''; 
  ${\rm F}_{q2Q}$ and ${\rm F}_{Q2q}$ are ``finite-field GF($q$) to finite-field GF($Q$) transform'' and ``finite-field GF($Q$) to finite-field GF($q$) transform''; 
  $\rm F_{F2C}$ and $\rm F_{C2F}$ stand for ``finite-field to complex-field transform'' and ``complex-field to finite-field transform''.}
\end{figure}


\subsection{Transmitter}
 
Let $K$ be a positive integer and ${\bf b}_j = (b_{j,0}, b_{j,1},\ldots, b_{j, k}, \ldots, b_{j, K})$ be the bit sequence at the output of user-$j$, where $b_{j,k} \in {\rm GF}(2)$ and $0 \le k < K$. 
In transmission, each bit $b_{j,k}$ in ${\bf b}_j$ is mapped into an element $u_{j,k}$ in the $j$-th constituent UDAIEP $C_j$ assigned to the $j$-th user by a \textit{binary to non-binary} transform function (B2NB) ${\rm F_{B2NB}}$, i.e., $u_{j,k} = {{\rm F_{B2NB}}(b_{j,k})}$. 
The B2NB transform function maps the output bit-sequences of the $J$ users produces $J$ sequences, 
i.e., ${\bf u}_j = (u_{j,0}, u_{j,1},\ldots, u_{j, k}, \ldots, u_{j, K-1})$ over GF($p$) 
where $1 \le j \le J$. 
For $0 \le k < K$, the $J$-tuple $(u_{1,k}, u_{2,k}, \ldots, u_{J,k})$ is a codeword in the $J$-user UDAIEP code ${\mathcal C} = C_1 \times C_2 \times \ldots \times C_{J}$.



Next, we multiplex the $J$ sequences, 
${\bf u}_1, {\bf u}_2, \ldots, {\bf u}_{J}$ 
into a sequence ${\bf u} = (u_0, u_1, \ldots, u_k, \ldots, u_{K-1})$ over GF($p$) by addition operation, 
where $u_k = \bigoplus_{j=1}^{J} u_{j, k}$ and $u_k \in {\rm GF}(p) \backslash \{0\}$. 
The multiplexed sequence $\bf u$ is a sequence of sum-patterns of 
the $J$-user UDAIEP code $\mathcal C$ over GF($p)$. 
We call $\bf u$ is the sum-pattern of ${\bf u}_1, {\bf u}_2, \ldots, {\bf u}_{J}$. 
Since $\mathcal C$ is a uniquely decodable code, 
the $J$ sequences ${\bf u}_1, {\bf u}_2, \ldots, {\bf u}_{J}$ can be uniquely separated from $\bf u$ symbol by symbol without ambiguity. 
Through the inverse function, denoted by $\rm F_{NB2B}$, of $\rm F_{B2NB}$, 
we can recover the output bit-sequences ${\bf b}_1, {\bf b}_2, \ldots, {\bf b}_{J}$ of the $J$ users from ${\bf u}_1, {\bf u}_2, \ldots, {\bf u}_J$ without ambiguity.



Suppose an $(N, K)$ linear block code $\bf W$ over GF($Q$) of length $N$ and dimension $K$ with generator matrix $\bf G$ is used for error control, where $Q$ may or may not equal to $p^m$. 
Each component $u_k$ in the sequence ${\bf u} = (u_0, u_1, \ldots, u_k, \ldots, u_{K-1})$ is uniquely mapped into an element $w_k$ in GF($Q$) by a GF($p$) to GF($Q$) transform function ${\rm F}_{q2Q}$, i.e.,  $w_k = {\rm F}_{q2Q}(u_k)$. 
The mapping results in a sequence ${\bf w} = (w_0, w_1, \ldots, w_k, \ldots, w_{K-1})$ over GF($Q$).
Next, ${\bf w}$ is encoded into a codeword ${\bf v} = {\bf w} {\bf G}$ in $\bf W$ in orthogonal form as shown by (\ref{e2.12}) and (\ref{e2.13}). 
Then, $\bf v$ is modulated and transmitted as shown in Fig. 3 (a). 
The orthogonal encoding allows all $J$ users to achieve the same frequency diversity gain.



In general, the $q2Q$ transform function ${\rm F}_{q2Q}$ is determined by the relationship between GF($p$) and GF($Q$). One case is given as follows. 
Let $m = \lceil \log_2(p-1) \rceil$ and $Q = 2^m$. 
Next, we construct the extension field GF($2^m$) of the binary field GF(2). 
Find a $p$ to $2^m$ transform function ${\rm F}_{p2Q}$ which maps each non-zero element $u_k$ in GF($p$) into an element $w_k$ in GF($2^m$) whose binary $m$-tuple representation is $(w_{k,0}, w_{k,1}, \ldots, w_{k,m-1})$. 
With this mapping, the UDAIEP coded sequence ${\bf u} = (u_0, u_1, \ldots, u_k, \ldots, u_{K-1})$ over GF($p$) for the $J$ users, with $u_k = \bigoplus_{j=1}^{J} u_{j,k}$, 
is transformed into a sequence ${\bf w} = (w_0, w_1, \ldots, w_k, \ldots, w_{K-1})$ over GF($2^m$) in which the $m$-tuple representation of the $k$-th component $w_k$ is $(w_{k,0}, w_{k,1}, \ldots, w_{k,m-1})$. 
Decompose the sequence $\bf w$ into $m$ constituent sequence ${\bf w}_0, {\bf w}_1, \ldots, {\bf w}_{m-1}$, with ${\bf w}_i = (w_{0,i}, w_{1,i}, \ldots,\\ w_{k,i}, \ldots, w_{K-1,i})$ over GF(2) for $0 \le i < m$. 
Let $\bf G$ be the generator matrix of an $(N, K)$ linear code $\bf W$ over GF($2^m$). 
Then, we encode the sequence $\bf w$ into a codeword
%\vspace{-0.1in}
\begin{equation*}
  {\bf v} = (v_0, v_1, \ldots, v_n, \ldots, v_{N - 1}) = {\bf w} {\bf G}
\end{equation*}
in $\bf W$ in orthogonal form given by (\ref{e2.12}) and (\ref{e2.13}).


If $p - 1 = 2^m$, every nonzero element in GF($p$) can be uniquely mapped into an element in GF($2^m$) $(Q = 2^m)$ in a simple way. Let $\alpha$ be a primitive element in GF($2^m$) and $u_k$ be a nonzero sum-pattern in an $m$-user UDAIEP code $\bf C$. Then, $u_k$ can be mapped into an element in GF($2^m$) as follows:
\begin{equation} \label{e4.1}
w_k = {\rm F}_{q2Q}(u_k) = \alpha^{u_k-2},
\end{equation}
with $\alpha^{-1}$ denoting the zero element $0$ of GF($2^m$), i.e., ${\alpha}^{-1} = 0$.



\textbf{Example 3:} 
Consider the prime field GF($17$). As shown in Example 1 (Fig. 1 (b)), 
based on this field, a 4-user UDAIEP code $\mathcal C$ can be constructed. 
This code can be used in a downlink FFMA to support $4$ users. 
Set $m = \log_2 (17 - 1) = 4$ and $Q = 2^4$. 
Construct the field GF($2^4$) based on the primitive polynomial ${\bf g}(X) = 1 + X + X^4$. 
Let $\alpha$ be a primitive element in GF($2^4$). 
Then, the mappings of $16$ nonzero sum-patterns in $\mathcal C$ to the 16 elements in GF($2^4$) are: 
${\rm F}_{q2Q}(j) = \alpha^{j - 2}$, with $1 \le j \le 16$.


Note that $257$ is a prime. 
Using the prime field GF($257$), we can construct an $8$-user UDAIEP code over GF($257$) which can be used in a downlink FFMA system to support $8$ users. 
Set $m = 8$. 
Then, the transform function ${\rm F}_{q2Q}(j) = \alpha^{j - 2}$ for $1 \le j \le 256$, 
uniquely maps a nonzero element in GF($257$) into an element in GF($2^8$).
$\blacktriangle  \blacktriangle$


To process the codeword ${\bf v} = (v_0, v_1,\ldots, v_n, \ldots, v_{N - 1})$  at the output of the encoder for transmission, we first expand each symbol $v_n$, $0 \le n < N$, in $\bf v$ into an $m$-tuple  
$(v_{n,0}, v_{n,1}, \ldots, v_{n, m-1})$ over GF(2). 
This expansion results in a codeword ${\bf v}_{exp}$ over GF(2) which consists of $N$ $m$-tuples, 
a sequence of $mN$ bits. 
Then, ${\bf v}_{exp}$ is sequentially (bit by bit) modulated with BPSK into a complex-field vector ${\bf s} \in {\mathbb C}^{1 \times L}$, where $L = mN$. 
Since each code symbol $v_n$ in the codeword $\bf v$ is expanded into an $m$-tuple and then modulated, 
the modulation can be regarded as a kind of transform $\rm F_{F2C}$ from finite-field GF($2^m$) to complex-field $\mathbb C$.


Next, the modulated sequence ${\bf s}$ is arranged into an $N_c \times M$ matrix $\bf S$ in complex-field frequency-domain, where $N_c$ is the number of \textit{sub-carrier} and $M = \lceil \frac{L}{N_c} \rceil$.
The $m_c$-th column, $1 \le m_c \le M$, of ${\bf S}$ is defined by ${\bf S}_{m_c} \in {\mathbb C}^{N_c \times 1}$. Each column of ${\bf S}$ consists of $N_c$ symbols which form a \textit{block}. 
The $M$ blocks of $\bf S$ will be processed and transmitted block-by-block. 
The transformation from $\bf s$ to $\bf S$ is referred to as serial to parallel (S/P) transformation.


Performing the \textit{inverse discrete Fourier transform (IDFT)} of each block ${\bf S}_{m_c}$ in ${\bf S}$, we obtain an $N_c \times M$ matrix $\bf X$ in complex-field time-domain whose $m_c$-th block (or column) is given as ${\bf X}_{m_c} = {\rm IDFT}[{\bf S}_{m_c}]$ for $1 \le m_c \le M$. 
After adding a \textit{cyclic prefix (CP)} to each block ${\bf X}_{m_c}$ of $\bf X$, 
the signals in $\bf X$ are then transmitted through different frequency-selective fading channels to $J$ users via multiple paths with different delays. 
The length $N_g$ of the inserted CP should be larger than the maximum time delay of the fading channels. 



The time-domain (TD) channel impulse response between the base station (BS) and the $j$-th user, 
$1 \le j \le J$, over ${\mathcal L} ({\mathcal L} < N_c)$ multi-paths is defined as
\begin{equation}
{\bf h}_{{\rm TD},j}  = \sum_{\ell = 1}^{\mathcal L} h_{j,\ell} \cdot \delta(t- \tau_{j,\ell}),
\end{equation}
where $h_{j,\ell}$ and $\tau_{j,\ell}$ are the channel gain and time-delay of the $\ell$th path to 
the $j$-th user, respectively. 
The channel impulse response ${\bf h}_{{\rm TD}, j}$ can be expressed by a $1 \times N_c$ vector, 
where the $\tau_{j,\ell}$-th location has value $h_{j,\ell}$ and the other locations are all zeros.
If $\tau_{j,\ell} = \ell T_s$, then the time-domain channel vector of ${\bf h}_{TD, j}$ is
%\begin{equation}
${\bf h}_{{\rm TD},j} = (h_{j,1}, h_{j,2},\ldots, h_{j,\ell}, \ldots, h_{j,\mathcal L}, 0, \ldots, 0) 
              \in {\mathbb C}^{1 \times N_c}$,
%\end{equation}
where $T_s$ is the symbol duration. 
The frequency-domain channel vector of ${\bf h}_{{\rm TD}, j}$ is 
%\begin{equation}
${\bf H}_{{\rm FD},j} = {\rm DFT}[{\bf h}_{{\rm TD},j}] = (H_{j,1}, H_{j,2},\ldots, H_{j, n_c}, \ldots, H_{j, N_c}) \in {\mathbb C}^{1 \times N_c}$,
%\end{equation}
where the subscript ``FD'' of ${\bf H}_{{\rm FD},j}$ stands for ``frequency-domain''.



\subsection{Receiver}

At the receiver of the $j$-user for $1 \le j \le J$ (see Fig. 3(b)), 
we first remove the inserted CP from each block, 
and let the received sequence of $M$ blocks of signals in complex-field frequency-domain be
%\begin{equation} \label{e4.5}
 $ {\bf Y}_j = ({\bf Y}_{j,1}, {\bf Y}_{j,2}, \ldots, {\bf Y}_{j,m_c},\ldots, {\bf Y}_{j,M})$
%\end{equation}
with
\begin{equation}
{\bf Y}_{j,m_c} = {\bf H}_{{\rm FD},j}^{\rm T} \circ {\bf X}_{m_c} + {\bf Z}_{j,m_c},
\end{equation}
where $\circ$ is the \textit{Hadamard product}, 
and ${\bf Y}_{j,m_c} \in \mathbb{C}^{N_c \times 1}$ for $1 \le m_c \le M$. 
${\bf Z}_{j,m_c} \in \mathbb{C}^{N_c \times 1}$ is the additive white Gaussian noise (AWGN) 
with distribution ${\mathcal N}(0, N_0/2)$.


Then, each received block ${\bf Y}_{j,m_c}$ in ${\bf Y}_j$ is processed by a \textit{frequency Fourier transform (FFT)} unit and a \textit{frequency-domain equalization (FDE)} unit to produce a block of signals
\begin{equation}
\hat {\bf Y}_{j,m_c} = {\bf w}_{{\rm FD},j,m_c} \circ ({\bf H}_{{\rm FD},j}^{\rm T} \circ {\bf X}) 
+ {\bf Z}_{j,m_c},
\end{equation}
where ${\bf w}_{{\rm FD},j,m_c}$ is the FDE weight,
which can be computed by the \textit{minimum mean square error (MMSE)} criterion.
The sequence of processed blocks 
%\begin{equation*}
$\hat {\bf Y}_j = (\hat{\bf Y}_{j,0}, \hat{\bf Y}_{j,1}, \ldots, \hat{\bf Y}_{j,m_c}, \ldots, \hat{\bf Y}_{j,N-1})$,
%\end{equation*}
is then transformed into a vector ${\bf y}_j \in {\mathbb C}^{1 \times L}$ which consists of $N$ $m$-tuples over the complex field ${\mathbb C}$, i.e., 
${\bf y}_j = (y_{j,0}, y_{j,1}, \ldots, y_{j,n}, \ldots, y_{j,N-1})$ for $0 \le n < N$,
with $y_{j,n}$ as an $m$-tuple 
$y_{j,n} = (y_{j,n,0}, y_{j,n,1}, \ldots, y_{j,n,i},\ldots,\\ y_{j,n,m-1})$.
Then, decoding can be performed based on ${\bf y}_j$ and the parity-check matrix $\bf H$ of the error control code $\bf W$. 



If $\bf W$ is a $q$-ary LDPC code, i.e., $q = 2^m$, 
we can decode ${\bf y}_j$ iteratively with a Fast-Fourier-Transform (FFT) $q$-ary sum-product algorithm (FFT-QSPA) based on the probability mass function (pmf).
Hence, the obtained bit-wise probabilities, 
i.e., ${\rm Pr}(w_{j,n,i}=0|y_{j,n,i})$ and ${\rm Pr}(w_{j,n,i}=1|y_{j,n,i})$, are then converted to symbol-wise probabilities by computing appropriate products.


For example, the field GF($2^4$) is constructed based on the primitive polynomial 
${\bf g}(X) = 1 + X + X^4$. 
Let $\alpha$ be a primitive element in GF($2^4$). 
It is known that $\alpha^{4} = (1100)$ and its probability is calculated as
\begin{equation*}
\text{Pr}(\alpha^4|y_{j,n}) = \text{Pr}(1|y_{j,n,0}) \text{Pr}(1|y_{j,n,1}) \text{Pr}(0|y_{j,n,2}) \text{Pr}(0|y_{j,n,3}).
\end{equation*}
Based on these $q$ symbol-wise probabilities, the decoder obtains pmfs of the elements in GF($q$).
Then, FFT-QSPA is utilized to obtain the decoded vector 
$\hat {\bf w}_j =(\hat w_{j,0}, \hat w_{j,1}, \ldots, \hat w_{j,k}, \ldots, \hat w_{j,K-1})$.
Through ${\rm F}_{Q2q}$, the corresponding non-zero element $\hat u_{j,k}$ can be obtained as
\begin{equation*}
  \hat u_{j,k} = {\rm F}_{Q2q}(\hat w_{j,k})  = \log_{\alpha}{\hat w_{j,k}} + 2,
\end{equation*}
where $\hat w_{j,k}$ is expressed as the power of $\alpha$
and $\hat u_{j,k} \in \text{GF}(p)\backslash 0$.
Then, the decoded sequence of sum-patterns is given as
%\begin{equation*}
$\hat {\bf u}_j = (\hat {u}_{j,0}, \hat {u}_{j,1}, \ldots, \hat {u}_{j,k}, \ldots, \hat {u}_{j,K-1})$
%\end{equation*}
over GF($p$) of the $J$-user UDAIEP code $\mathcal C$. 
From this sequence of sum-patterns, 
we can uniquely extract the output bit sequence ${\bf b}_j$ of the $j$-th user. 
If decoding ${\bf y}_j$ is successful, ${\bf b}_j$ is error-free.


Note that, the proposed downlink FFMA system of GF($p$) can be extended to the GF($p^m$) scenario. 



%\vspace{0.2in}
\section{An uplink FFMA System for an AWGN Multi-Access Channel}

This section presents an uplink FFMA system for an AWGN MA channel based on the extension field GF($2^m$) of the binary field GF(2). Since the base field is GF(2), it has only one element pair $(0, 1)$ which is not an additive inverse pair. 
Using this element pair (EP), we can construct a set of $m$ orthogonal UDEPs over GF($2^m$), 
\begin{equation}
  \Psi = \{\psi_0(0,1), \psi_1(0,1), \ldots, \psi_i(0,1), \ldots, \psi_{m-1}(0,1) \}.
\end{equation}
The EP $= (0,1)$ is referred as the base EP. 


Suppose the system is designed to support $J$ users with $1 \le J \le m$. 
The EP $\psi_{j-1}(0,1)$ is assigned to the $j$-th user. 
A block diagram of the designed uplink FFMA system is shown in Fig. 4. 
In the system, BPSK is used for modulation.

\begin{figure}[t]
  \centering
  \includegraphics[width=0.95\textwidth]{Fig_UL.pdf}
  \caption{System model of an uplink FFMA system of GF($2^m$) in an AWGN MAC,
  where $\rm F_{B2NB}$ and $\rm F_{NB2N}$ stand for ``binary to non-binary transform'' and ``non-binary to binary transform''; 
  ${\rm F}_{q2Q}$ and ${\rm F}_{Q2q}$ are ``finite-field GF($q$) to finite-field GF($Q$) transform'' and ``finite-field GF($Q$) to finite-field GF($q$) transform''; 
  $\rm F_{F2C}$ and $\rm F_{C2F}$ stand for ``finite-field to complex-field transform'' and ``complex-field to finite-field transform''.}
\end{figure}


\subsection{Transmitter}
Let $K$ be a positive integer and ${\bf b}_j = (b_{j,0}, b_{j,1},\ldots, b_{j,k}, \ldots, b_{j,K-1})$ be the bit-sequence at the output of the $j$-th user. 
The transmitter first maps the bit-sequence ${\bf b}_j$ uniquely into a sequence 
${\bf u}_j = (u_{j,0},u_{j,1},\ldots, u_{j,k},\ldots, u_{j,K-1})$ over GF($2^m$) by a binary to nonbinary transform function $\rm F_{B2NB}$, 
where $u_{j,k} = {\rm F_{B2NB}}(b_{j,k})$ which is determined by the EP $\psi_{j-1}(0,1)$ that assigned to the $j$-th user. For $0 \le k < K$, express the $k$-th component of ${\bf u}_j$ into an $m$-tuple 
$u_{j,k} =(u_{j,k,0}, u_{j,k,1},\ldots, u_{j,k,i},\ldots, u_{j,k,m-1})$. 
Since the $j$th user utilizes the EP $\psi_{j-1}(0,1)$, the $i$-th bit $u_{j,k,i}$ of $u_{j,k}$ is
\begin{equation}
u_{j,k,i} =
\left\{
  \begin{matrix}
    b_{j,k}, & i = j-1\\
    0,     & i \neq j-1 \\
  \end{matrix} \right. 
\end{equation}
In the next step, the sequence ${\bf u}_j$ is encoded into a codeword ${\bf v}_j$ of an $(N, K)$ linear block code $\bf W$ over GF($2^m$) specified by a $K \times N$ generator matrix $\bf G$. 
The codeword ${\bf v}_j$ for ${\bf u}_j$ is
\begin{equation} \label{e5.3}
   {\bf v}_j = {\bf u}_j {\bf G} = (v_{j,0}, v_{j,1},\ldots, v_{j,n}, \ldots, v_{j,N-1}).      
\end{equation}
Express each code symbol $v_{j,n}$ for $0 \le n < N$, in ${\bf v}_j$ into an $m$-tuple over GF(2),
%\begin{equation*}
    $v_{j,n} = (v_{j,n,0}, v_{j,n,1},\ldots,\\ v_{j,n,i}, \ldots, v_{j,n,m-1})$.
%\end{equation*}
With BPSK signaling, ${\bf v}_j$ is modulated and mapped to a complex vector 
${\bf x}_j \in {\mathbb C}^{1 \times (N \cdot m)}$, 
where ${\bf x}_j = (x_{j,0}, x_{j,1}, \ldots, x_{j,n}, \ldots, x_{j,N-1})$ with $x_{j,n} = (x_{j,n,0}, x_{j,n,1},\ldots, x_{j,n,i}, \ldots, x_{j,n,m-1})$. 
For $0 \le i < m$, the $i$-th component of $x_{j,n}$ is given by
\begin{equation}  
{x}_{j,n,i} = 2 {v}_{j,n,i} - 1,
\end{equation}
where $x_{j,n,i} \in {\mathbb C}$. 
The mapping from ${\bf v}_j$ to ${\bf x}_j$ is regarded as 
\textit{finite-field to complex-field transform}, 
denoted by $\rm F_{B2C}$. Then ${\bf x}_j$ is sent to an AWGN MA channel. 



Note that if the channel code $\bf W$ is a code over GF($Q$) which is different from GF($2^m$), 
a transform function ${\rm F}_{q2Q}$ is needed to match the two finite fields GF($2^m$) and GF($Q$).


\subsection{Receiver}
For presenting the receiver and its process, 
we define ${\bf u} = \bigoplus_{j=1}^J {\bf u}_j$ and ${\bf v} = \bigoplus_{j=1}^J {\bf v}_j$ as an information sum-pattern vector and a codeword sum-pattern vector over GF($2^m$) of the $J$ users 
where ${\bf u} = (u_0, u_1, \ldots, u_{k}, \ldots, u_{K-1})$ with $u_k = \bigoplus_{j=1}^J u_{j,k}$ 
and ${\bf v} = (v_0, v_1,\ldots, v_n,\ldots, v_{N-1})$ with $v_n = \bigoplus_{j=1}^J v_{j,n}$. 
It follows from (\ref{e2.12}) and (\ref{e2.13}) that $\bf v = u \cdot G$ in orthogonal form.



At the receiving end, the received vector ${\bf y} \in {\mathbb C}^{1 \times (N \cdot m)}$ is the complex-field sum-pattern of the outputs of the $J$ users plus noise which is given by
\begin{equation} 
{\bf y} = \sum_{j=1}^{J} {\bf x}_j + {\bf z} = {\bf r} + {\bf z},
\end{equation}
where ${\bf z} \in \mathbb{C}^{1 \times (N \cdot m)}$ is an AWGN vector
with ${\mathcal N}(0, N_0/2)$,
and ${\bf r} \in \mathbb{C}^{1 \times (N \cdot m)}$ is a complex-field sum-pattern of the modulated sequences ${\bf x}_1, {\bf x}_2,\ldots, {\bf x}_J$.
The vector ${\bf r} = (r_0, r_1, \ldots, r_n, \ldots, r_{N-1})$ consists of $N$ $m$-tuples with $r_n = (r_{n,0}, r_{n,1}, \ldots, r_{n,i}, \ldots, r_{n,m-1})$ and
\begin{equation} \label{e5.7}
   r_{n,i} = \sum_{j=1}^J x_{j,n,i} = 2 \sum_{j=1}^J v_{j,n,i} - J.              
\end{equation}
The first step of the decoding process is to transform the received vector $\bf y$ into a vector 
\begin{equation} \label{e5.8}
  \hat{{\bf y}} = \rm F_{C2F}({\bf r}) + {\bf z},
\end{equation}
over GF($2^m$) by a \textit{complex-field to finite-field (C2F)} transform function $\rm F_{C2F}$, 
where ${\rm F_{C2F}}({\bf r}) = {\bf v} = (v_0, v_1,\ldots, v_n, \ldots, v_{N-1})$ 
which consists of $N$ $m$-tuples. 
The $n$-th component $v_n$ in $\bf v$ is an $m$-tuple, i.e., 
$v_n = (v_{n,0}, v_{n,1},\ldots, v_{n,i}, \ldots, v_{n,m-1})$, 
given by
\begin{equation}  \label{e5.9}
  v_{n,i} = {\rm F_{C2F}}(r_{n,i}) = \bigoplus_{j=1}^J v_{j,n,i}.  
\end{equation}                    
Note that it is important to find the function $\rm F_{C2F}$, otherwise, the system is ineffective.



Once $\hat {\bf y}$ is formed, 
it is decoded based on a chosen decoding method for the channel code $\bf W$. 
If decoding is successful, $\hat {\bf y}$ is decoded into the transmitted vector $\bf v$. 
Then, the transforms ${\rm F}_{Q2q}$ and $\rm F_{NB2B}$ 
(the inverse transforms of ${\rm F}_{q2Q}$ and $\rm F_{B2NB}$ performed in the transmitting end), 
are applied to $\bf v$ to recover the bit sequences 
${\bf b}_1, {\bf b}_2,\ldots, {\bf b}_J$ transmitted by the $J$ users based on the orthogonal UDEPs assigned to the $J$ users.

\vspace{-0.1in}
\begin{figure}[t] 
  \centering
  \label{Fig7}
  \includegraphics[width=0.7\textwidth]{Fig5_GF2.pdf}
 \caption{The relationship between complex-field sum-pattern $r_{n,i}$ and finite-field sum-pattern $v_{n,i}$ of an uplink FFMA system of GF($2^m$), where each user uses BPSK and the numbers of users are set to be $J=2, 3, 4$.}
\end{figure}


The decoding is based on the relationship between the complex-field sum-pattern $r_{n,i}$ and the finite-field sum-pattern $v_{n,i}$, the transform functions and unique decidability of the orthogonal UDEPs assigned to the users. 
Fig. 5 shows the relationship between the complex-field sum-pattern $r_{n,i}$ and finite-field sum-pattern $v_{n,i}$, where each user utilizes BPSK, and the numbers of users are set to be $J = 2, 3, 4$. 
From (\ref{e5.7}) and Fig. 5, we find the following facts:
\begin{enumerate}
\item
The value of $r_{n,i}$ is determined by the number of users who send ``+1'' and the number of users who send ``-1''. Thus, the maximum and minimum values of $r_{n,i}$ are respectively $J$ and $-J$. The set of $r_{n,i}$'s values in ascendant order is $\Omega = \{-J, -J+2, \ldots, J-2, J\}$, 
in which the difference between two adjacent values is $2$. The total number of $\Omega$ is equal to $|\Omega| = J+1$. 
\item
Since $v_{n,i} \in {\rm GF}(2)$, the possible values of $v_{n,i}$ are $(0)_2$ and $(1)_2$. 
Then, $v_{n,i}$ is uniquely determined by the number of $(1)_2$ bits coming from the $J$ users, 
i.e., $v_{1,n,i}, v_{2,n,i}, \ldots, v_{J,n,i}$. 
If there are odd number of bits $(1)_2$ in the set $\{v_{1,n,i}, v_{2,n,i}, \ldots, v_{J,n,i}\}$, 
then $v_{n,i} = (1)_2$; otherwise, $v_{n,i} = (0)_2$. 
Since the values of $r_{n,i}$ are arranged in ascendant order, 
the corresponding binary set $\Omega_v$ of $\Omega$ is $\{0, 1, 0, 1, \ldots\}$, 
in which `0' and `1' appear alternatively. 
The number of $|\Omega_v|$ is also equal to $J+1$, i.e., $|\Omega_v| = |\Omega|$.
\item
Let $C_J^\iota$ denote the number of users who send ``+1''. 
The values of $\iota$ are from $0$ to $J$. 
When $\iota = 0$, it indicates that all the $J$ users send $(0)_2$, 
i.e., $v_{j,n,i} = (0)_2$ and $x_{j,n,i} = -1$ for all $1 \le j \le J$, 
thus, $v_{n,i} = \bigoplus_{j=1}^J v_{j,n,i} = (0)_2$. 
If $\iota$ increases by one, the number of $(1)_2$ bits coming from $J$ users increases by one accordingly. Therefore, the difference between two adjacent values is $2$, and the bits $(0)_2$ and $(1)_2$ appear alternatively.
\end{enumerate}



Based on the above development, the transform function $\rm F_{C2F}$ maps each received symbol $r_{n,i}$ to a uniquely sum-pattern element $v_{n,i}$, i.e., ${\rm F_{C2F}}: r_{n,i} \to v_{n,i}$. 
Since $r_{n,i} \in \{-J, -J+2, \ldots, J-2, J\}$ and $v_{n,i} \in \{0,1\}$, 
$\rm F_{C2F}$ is a many-to-one mapping function.


\textbf{Example 4:}
If $J=2$, we have $\Omega =\{-2, 0, 2\}$, and $\Omega_v = \{0, 1, 0\}$.
If $J=3$, then $\Omega = \{-3, -1, 1, 3\}$, and $\Omega_v = \{0, 1, 0, 1\}$.
If $J=4$, we have $\Omega =\{-4, -2, 0, 2, 4\}$, and $\Omega_v = \{0, 1, 0, 1, 0\}$.
$\blacktriangle \blacktriangle$



Now, we consider the probabilities used for decoding $\hat {\bf y} = {\rm F_{C2F}}({\bf r}) + {\bf z}$.  Assume that $v_{n,i} = (0)_2$ and $v_{n,i} = (1)_2$ are equally likely, 
i.e., ${\rm Pr}(v_{n,i}=0) = {\rm Pr}(v_{n,i} = 1) = 0.5$. 
Then, the probabilities of the elements in $\Omega$ are
%\vspace{-0.1in}
%\begin{equation} \label{e5.10}
   ${\mathcal P}_r = \left\{ {C_J^0}/{2^J}, {C_J^{1}}/{2^J},\ldots, {C_J^{J-1}}/{2^J}, {C_J^J}/{2^J} \right\}$.
%\end{equation}


Based on the relationship between $r_{n,i}$ and $v_{n,i}$, 
we can calculate the posterior probabilities of the received signals. 
The conditional probability $v_{n,i}$ given by $y_{n,i}$ is

\vspace{-0.1in}
\begin{small}
\begin{equation} \label{e5.11}
  {\rm Pr}(v_{n,i}|y_{n,i}) = \frac{{\rm Pr}(v_{n,i})p(y_{n,i}|v_{n,i})}{p(y_{n,i})} ,                   
\end{equation}
\end{small}
where $p(y_{n,i})$ is the probability of $y_{n,i}$.

Since $y_{n,i}$ is determined by $r_{n,i}$ that is selected from the set $\Omega$, thus,

\begin{small}
\begin{equation} \label{e_MAP}
  \begin{aligned}
  p(y_{n,i}) %&= \sum_{\iota=0}^{J} \text{Pr}(c_{\iota}) p(c_{\iota}),\\
         &= \sum_{\iota=0}^{J} {\mathcal P}_r(\iota) \cdot 
         \frac{1}{\sqrt{\pi N_0}} \exp \left\{
         - \frac{\left[y_{n,i} - \Omega(\iota) \right]^2}{N_0} 
         \right\},  \\
  \end{aligned}
\end{equation}
\end{small}
where ${\mathcal P}_r(\iota)$ and $\Omega(\iota)$ stand for the $\iota$th element of the sets ${\mathcal P}_r$ and $\Omega$, respectively. 

When $v_{n,i} = (0)_2$, the corresponding $r_{n,i}$ equals to $\{-J, -J+4, -J+8, \ldots\}$. 
Thus, the posteriori probability of $v_{n,i} = (0)_2$ is

\begin{small}
\begin{equation}
  {\rm Pr}(v_{n,i}=0|y_{n,i}) = \frac{1}{{p(y_{n,i})}} 
  \sum_{\iota=0,\iota+2}^{\iota \le J} {\mathcal P}_r(\iota)\cdot \frac{1}{\sqrt{\pi N_0}}
  \exp \left\{
  -\frac{\left[y_{n,i} - \Omega(\iota) \right]^2}{N_0}
  \right\},
\end{equation}
\end{small}

\vspace{-0.2in}
Similarly, when $v_{n,i} = (1)_2$, it indicates $r_{n,i}$ belongs to $\{-J+2, -J+6, -J+10,\ldots\}$, and
\begin{equation}
  {\rm Pr}(v_{n,i}=1|y_{n,i}) = \frac{1}{{p(y_{n,i})}} 
  \sum_{\iota=1,\iota+2}^{\iota \le J} {\mathcal P}_r(\iota)\cdot \frac{1}{\sqrt{\pi N_0}}
  \exp \left\{
  -\frac{\left[y_{n,i} - \Omega(\iota) \right]^2}{N_0}
  \right\}.
\end{equation}

Then, ${\rm Pr}(v_{n,i} = 0|y_{n,i})$ and ${\rm Pr}(v_{n,i} = 1|y_{n,i})$ are used for decoding $\bf y$. 
If a binary LDPC code is utilized, we can directly calculate LLRs based on ${\rm Pr}(v_{n,i} = 0|y_{n,i})$ and ${\rm Pr}(v_{n,i} = 1|y_{n,i})$. If an NB-LDPC code is utilized, the pmf can be computed based on ${\rm Pr}(v_{n,i} = 0|y_{n,i})$ and ${\rm Pr}(v_{n,i} = 1|y_{n,i})$.



If channel code is not used, then $\bf v = u$. 
In this case, we can directly estimate the value of $v_{n,i}$ by comparing 
${\rm Pr}(v_{n,i} = 0|y_{n,i})$ and ${\rm Pr}(v_{n,i} = 1|y_{n,i})$. 
If ${\rm Pr}(v_{n,i} = 0|y_{n,i}) > {\rm Pr}(v_{n,i} = 1|y_{n,i})$, 
the detected element $\hat{v}_{n,i} = (0)_2$; 
otherwise, $\hat{v}_{n,i} = (1)_2$. 
Then, we can separate the detected vector $\hat{\bf v}$ into $J$ different bit sequences as 
$\hat{\bf b}_1, \hat{\bf b}_2, \ldots, \hat{\bf b}_J$ 
by using an inverse function ${\rm F_{NB2B}}$ (or decoding table).



In the following, we use an example to illustrate the transmission process.

\textbf{Example 5}:
Given a finite-field GF($2^4$), we assume $J = 4$ and $K = 1$.
Let ${\bf b}_1 = (1)_2$, ${\bf b}_2 = (0)_2$, ${\bf b}_3 = (1)_2$, and ${\bf b}_4 = (1)_2$. 
Thus, we have

\begin{small}
\begin{equation*}
\begin{aligned}
{\bf b}_1 = (1)_2 \to & {\bf u}_1 = (\textcolor{blue}{1}, 0, 0, 0)_2 \to & {\bf x}_1 = (+1, -1, -1, -1), \\
{\bf b}_2 = (0)_2 \to & {\bf u}_2 = (0, \textcolor{blue}{0}, 0, 0)_2 \to & {\bf x}_2 = (-1, -1, -1, -1), \\
{\bf b}_3 = (1)_2 \to & {\bf u}_3 = (0, 0, \textcolor{blue}{1}, 0)_2 \to & {\bf x}_3 = (-1, -1, +1, -1), \\
{\bf b}_4 = (1)_2 \to & {\bf u}_4 = (0, 0, 0, \textcolor{blue}{1})_2 \to & {\bf x}_4 = (-1, -1, -1, +1), \\
\end{aligned}
\end{equation*}
\end{small}
and the received complex-field sum-pattern is
%\begin{small}
%\begin{equation*}
${\bf r} = \sum_{j=1}^{4} {\bf x}_j = (-2, -4, -2, -2).$
%\end{equation*}
%\end{small}

Since $J = 4$, we have $\Omega = \{-4, -2, 0, +2, +4\}$ and $\Omega_v = \{0, 1, 0, 1, 0\}$,
indicating that ${\rm F_{C2F}}(-4) = (0)_2$, ${\rm F_{C2F}}(-2) = (1)_2$, ${\rm F_{C2F}}(0) = (0)_2$,
${\rm F_{C2F}}(+2) = (1)_2$, and ${\rm F_{C2F}}(+4) = (0)_2$.
Therefore,
%\begin{equation*}
${\bf v} = {\rm F_{C2F}}({\bf r}) = (1, 0, 1, 1)_2$.
%\end{equation*}
Through ${\rm F_{NB2B}}$, we can obtain that $\hat {\bf b}_1 = (1)_2$, $\hat {\bf b}_2 = (0)_2$, $\hat {\bf b}_3 = (1)_2$, and $\hat {\bf b}_1 = (1)_2$.
$\blacktriangle \blacktriangle$


From Example 5, we find an interesting phenomenon. 
If there is no channel code, the practical complex-field sum-pattern set becomes as $\Omega = \{-4, -2\}$, with probability ${\mathcal P}_r =  \{0.5, 0.5\}$. 
In the following section, we will prove that the BER performance of the proposed system without channel code is the same as that of BPSK.





\vspace{-0.1in}
\section{Theoretical Analysis}
In this section, we first introduce a constraint on complex-field to finite-field (C2F) mapping, and then analyze the capacity and bit-error-rate (BER) performance of the proposed uplink FFMA system of GF($2^m$) over an AWGN multi-access channel. Since the proposed downlink FFMA system over GF($p$) is like the classical OFDM system, we will not analyze its performance in this paper.


\subsection{C2F-constraint and Spectral Efficiency}
In general, NOMA can simultaneously support multiuser transmission by sharing the same time and frequency resources. Hence, it improves the spectral efficiency (SE) of multiple-access systems. Well-known NOMA schemes \cite{ZDing2017_survey} can separate users by using power-domain, or multi-constellation. 
When each user utilizes the same constellation and power, the classical NOMA schemes seem unworkable, 
since the superimposed signals (or called complex-field sum-pattern) cannot be independently separated. 


\begin{figure}[t]
  \centering
  \label{Fig9} 
  \includegraphics[width=0.9\textwidth]{Fig9.pdf}
  \caption{A diagram to show the relationship between $|\Omega|$ and $|\Theta|$, where $J=2$.
  (a) shows one-to-many mapping; (b) shows one-to-one mapping; and (c) shows many-to-one mapping.}
\end{figure}


Consider the uplink FFMA system proposed in Section V which supports $J$ users. 
Let $\Omega$ and $\Theta$ be the complex-field sum-pattern set and the transmitted bit set of the $J$ users, respectively. There are three cases:
\begin{enumerate}
  \item
  $|\Omega|<|\Theta|$. When all the $J$ users utilize the same BPSK modulation, it has been shown in Section V.B that $|\Omega| = J+1$. In this case, $|\Theta| = 2^J$ which indicates that $|\Omega|<|\Theta|$ for $J \ge 2$. If we perform multiuser detection (MUD) to recover the bit information of the $J$ users, it is a \textit{one-to-many mapping} between $\Omega$ and $\Theta$. Hence, there exists ambiguity at the receiver in recovering the transmitted bits. An example is shown in Fig. 6 (a).
  \item
  $|\Omega|=|\Theta|$. In \cite{QY_ISJ_2019}, the concept of uniquely decodable mapping (UDM) was introduced. 
  When each user utilizes 2ASK modulation and the amplitudes of the $J$ users are $1, 2^1,\ldots, 2^{J-1}$, respectively, the $J$ users can be separated without ambiguity, since it is a \textit{one-to-one mapping} between $\Omega$ and $\Theta$. 
  For example, if there are two users whose transmit symbols are respectively $\{-1,+1\}$ and $\{-2,+2\}$, then the complex-field sum-pattern set is $\Omega = \{-3,-1,+1,+3\}$, where $|\Omega|=|\Theta|=2^2=4$, 
  as shown in Fig. 6 (b).
  \item
  $|\Omega|>|\Theta|$. If GF($2^m$) is utilized in an FFMA system, then $|\Omega|>2$. 
  In this case, $\rm F_{C2F}$ is a \textit{many-to-one mapping}. An example is shown in Fig. 6 (c).
\end{enumerate}
The above cases for the finite field GF($2^m$) can be extended to a general finite field GF($p^m$). 
From the above developments, we have the following proposition.

\begin{proposition}
(C2F-constraint for BPSK): Suppose there are $J$ users where $J \ge 2$, and each user utilizes BPSK modulation. Let $\Omega$ be the complex-field sum-pattern with $|\Omega| = J+1$. 
For a given finite-field GF($p^m$) where $p$ is a prime with $p > 2$, 
assume that $J \le m \cdot \log_2 (p-1)$ which means $2^{\frac{J}{m}}+1 \le p$.  
Then, we have the following three cases for the mapping function $\rm F_{C2F}$:
\begin{enumerate}
  \item
  If $p > J+1$, then $\rm F_{C2F}$ is a one-to-many mapping function.
  \item
  If $p = J+1$, then $\rm F_{C2F}$ is a one-to-one mapping function; and
  \item
  If $p < J+1$, then $\rm F_{C2F}$ is a many-to-one mapping function.
\end{enumerate}
To recover the transmitted bits without ambiguity, $\rm F_{C2F}$ should be a many-to-one or one-to-one mapping function, and the prime $p$ must satisfy the constraint:
\begin{equation}
    2^{\frac{J}{m}}+1 \le p \le J+1,
\end{equation}
which is referred to as the complex-field to finite-field mapping constraint (C2F-constraint) for BPSK modulation.
\end{proposition}



\begin{small}
\begin{table} [t]
\centering
\label{Table_BPSK}
\caption{The values of $p$ for BPSK based $J$ users transmission in GF($p^m$).}
\vspace{-0.1in}
  \begin{tabular}{c|c|c|c|c|c|c}
  \hline
  \hline
           & $m = 1$   & $m = 2$   & $m = 3$   & $m = 4$   & $m=5$    & $m=6$\\
  \hline
  $J = 2$  & $\times$  & $3$       & $3$       & $3$       & $3$      & $3$\\
  \hline
  $J = 3$  & $\times$  & $\times$  & $3$       & $3$       & $3$      & $3$\\
  \hline
  $J = 4$  & $\times$  & $5$       & $5$       & $3,5$     & $3, 5$   & $3,5$\\
  \hline 
  $J = 5$  & $\times$  & $\times$  & $5$       & $5$       & $3, 5$   & $3, 5$\\
  \hline
  $J = 6$  & $\times$  & $\times$  & $5, 7$    & $5, 7$    & $5, 7$   & $3, 5, 7$ \\ 
  \hline
  $J = 7$  & $\times$  & $\times$  & $7$       & $5, 7$    & $5, 7$   & $5, 7$ \\
  \hline
  $J = 8$  & $\times$  & $\times$  & $\times$  & $5, 7$    & $5, 7$   & $5, 7$ \\
  \hline
  $J = 9$  & $\times$  & $\times$  & $\times$  & $7$       & $5, 7$   & $5, 7$ \\
  \hline
  $J = 10$ & $\times$  & $\times$  & $\times$  & $7, 11$   & $5, 7, 11$ & $5, 7, 11$\\
  \hline
  \hline
  \end{tabular}
\end{table}
\end{small}


Note that, the finite-field GF($2^m$) is naturally satisfied the C2F-constraint, 
since $|\Omega|>2$ for $J \ge 2$. 
A list of primes that satisfy the C2F-constraint is given in Table I, 
where the output bits of $J$ users are transmitted over an AWGN multi-access channel, 
and each user uses BPSK modulation. 
As $m$ increases, there are more primes satisfy the C2F-constraint.  
For $m = 1$, no prime $p$ satisfies the C2F-constraint, because of $2^J > J$ for $J \ge 2$. 


Next, we investigate the case for C2F-constraint with QPSK modulation. 
Due to the quadrature feature, both the real and imaginary parts have the same mapping criterion. 
Thereby, the C2F-constraint for QPSK is the same as that for the BPSK case. 
Besides the BPSK and QPSK, there are many types of modulations, e.g., QAM, FSK, which will not be investigated in this paper.


Under the C2F-constraint, the spectral efficiency (SE) $\lambda$ of the proposed FFMA system of GF($p^m$) is defined by
\begin{equation}
   \lambda = \frac{J \cdot R_c \cdot \log_2 \mathcal M}{m}, 
\end{equation}
where $\mathcal M$ is the modulation order, and $R_c$ is the data rate. 
When $J = m, R_c = 1 $ and $\mathcal M=2$ (BPSK modulation), $\lambda$ equals to 1, i.e., $\lambda = 1$. Thus, the SE of the proposed uplink FFMA system of GF($2^m$) is equal to one, which is an O-FFMA system.




\subsection{Capacity of the uplink FFMA System of GF($2^m$) over an AWGN Multi-Access Channel}

Next, we investigate the capacity of the proposed uplink FFMA system of GF($2^m$) over an AWGN multi-access channel under the C2F-constraint.

\begin{theorem} 
Assume that the joint transmit random variable of the multiple users is 
${\mathcal X}=({\mathcal X}_1, {\mathcal X}_2,\ldots,{\mathcal X}_J)$, 
the received complex-field superposition-signal random variable is ${\mathcal R}$, 
and the obtained finite-field sum-pattern random variable is ${\mathcal V}$. 
Suppose $\rm F_{C2F}$ is the function that maps a complex-field sum-pattern $r_{n,i}$ to a unique finite-field sum-pattern $v_{n,i}$, i.e., ${\rm F_{C2F}}: {\mathcal R} \to {\mathcal V}$. 
Then, the capacity ${\rm I}({\mathcal X}:{\mathcal V})$ of the uplink FFMA system of GF($2^m$) over an AWGN multi-access channel is not larger than the capacity ${\rm I}({\mathcal X}:{\mathcal R})$ of the classical AWGN multi-access channel, i.e., 
\begin{equation*}
{\rm I}({\mathcal X}:{\mathcal V}) \le {\rm I}({\mathcal X}:{\mathcal R}).
\end{equation*}
\end{theorem} 


\begin{proof}
Since ${\mathcal X} \to {\mathcal R} \to {\mathcal V}$ forms a Markov chain, 
according to the data processing inequality, it is easy to prove the theorem. 
Note that the proposed $\rm F_{C2F}$ satisfies the C2F-constraint. 
For a given finite-field GF($2^m$), $\rm F_{C2F}$ is a many-to-one mapping function, 
thus,
\begin{equation*}
{\rm I}({\mathcal X}:{\mathcal R}|{\mathcal V}) \neq 0,
\end{equation*}
which implies ${\rm I}({\mathcal X}:{\mathcal V}) < {\rm I}({\mathcal X}:{\mathcal R})$. 
While, if $\rm F_{C2F}$ is a one-to-one mapping function, 
it is known that ${\rm I}({\mathcal X}:{\mathcal R}|{\mathcal V}) = 0$. 
So that, we have ${\rm I}({\mathcal X}:{\mathcal V}) = {\rm I}({\mathcal X}:{\mathcal R})$. 
\end{proof}


\subsection{BER performance}
In the following, we analyze the BER of the proposed FFMA system in GF($2^m$) over an AWGN multi-access channel, under the assumption that there is no channel code. 
Thus, we can set $K = N = 1$, 
remove the subscript $k$ from $b_{j,k}$ and $u_{j,k}$ which are then $b_j, u_j$, 
and remove the subscript $n$ from $x_{j,n,i}$ which is then rewritten as $x_{j,i}$. 
Assume there are $J$ users, and the $j$-th user is assigned the EP $\psi_{j-1}(0,1)$, 
where $1 \le j \le J$ and $J \le m$. When the $j$-th user transmits $b_j$, we have

\begin{small}
\begin{equation*}
x_{j,i} = \left\{
  \begin{matrix}
    2 u_{j} - 1, & i = j-1\\
    -1,          & i \neq j-1 \\
  \end{matrix} \right. ,
\end{equation*}
\end{small}
where $0 \le i < m$.
Thereafter, the received signal is
\begin{small}
\begin{equation*}
y_{i} = \sum_{j=1}^{J} x_{j,i} + z_{i} = (2 u_{j-1} - 1) + \sum_{j' \neq j-1}^{J} (-1) + z_i \\
      = (-J + 2 u_{j-1})+ z_i,
\end{equation*}
\end{small}
where $r_i = -J + 2 u_{j-1}$.
If $u_{j-1} = (0)_2$, we have $r_i = -J$; 
otherwise $u_{j-1} = (1)_2$, and we have $r_i = -J+2$.
Then, the complex-field sum-pattern is $\Omega = \{-J, -J+2\}$, 
with equiprobability ${\mathcal P}_r = \{0.5, 0.5\}$.
The Euclidean distance between $-J$ and $-J+2$ is 2, which is the same as the BPSK modulation.
Then, the BER is given as
%\begin{equation*}
$P_e = Q(\sqrt{2 \gamma})$,
%\end{equation*}
where $\gamma$ is the average bit energy per noise power density $E_b/N_0$.



The proposed uplink FFMA system of GF($2^m$) over an AWGN multi-access channel is likewise a type of TDMA system, where the field extension factor $m$ can be viewed as finite-field time-domain resources. 
Additionally, when each user utilizes BPSK, 
the BER of the proposed system is the same as that of an equivalent TDMA system. 
Nevertheless, there is a major difference between the proposed FFMA and TDMA. 
When we insert a channel code, the transmit bits in finite-field time domain of the proposed FFMA system can be interleaved into the entire complex-field time domain. This achieves a time-domain diversity gain.


\vspace{-0.1in}
\section{Simulation results}

In this section, we simulate the error performances of a downlink and an uplink FFMA systems proposed in Sections IV and V using a nonbinary LDPC (NB-LDPC) code for error control.

\vspace{-0.1in}
\subsection{Code Construction}

Let $\alpha$ be a primitive element of GF(73). We construct a $4 \times 12$ matrix over GF(73) as follows:
\begin{small}
\begin{equation}
  {\bf B}(4,12)=\left[
  \begin{array}{*{20}{c}}
  {\alpha}^{48}   &  0   & {\alpha}^{20}  & 0  & {\alpha}^{28}  & {\alpha}^{25}  
  & {\alpha}^{38} & {\alpha}^{54}   & {\alpha}^{21}  & 0  & {\alpha}^{64}  & 0\\
  0  &  {\alpha}^{48} & 0 & {\alpha}^{20}  & {\alpha}^{10} & {\alpha}^{28} 
  & {\alpha}^{25}  & {\alpha}^{38}  & 0 & {\alpha}^{21}   & 0 & {\alpha}^{64}\\    
    {\alpha}^{59}  & {\alpha}^{41}  & {\alpha}^{48}  & {\alpha}^{29}  & {\alpha}^{20}   & 0
    & {\alpha}^{28}  & 0  & {\alpha}^{38} & {\alpha}^{54}  & {\alpha}^{21}  & {\alpha}^{14}  \\
    {\alpha}^{32} & {\alpha}^{59} & {\alpha}^{41}  & {\alpha}^{48}  & 0   & {\alpha}^{20} 
    & 0 & {\alpha}^{28}  &  {\alpha}^{25} & {\alpha}^{38}  & {\alpha}^{54} & {\alpha}^{21} \\
  \end{array}
  \right].
\end{equation}
\end{small}
Every $2 \times 2$ submatrix of ${\bf B}(4,12)$ is nonsingular. 
The matrix ${\bf B}(4, 12)$ is called a \textit{base matrix}. 



Next, we disperse each nonzero entry $\alpha^j$ in the base matrix ${\bf B}(4, 12)$ into a $72 \times 72$ circulant permutation matrix (CPM) (with columns labeled from 0 to 71) whose generator (or the top row) is $72$-tuple with a single $1$-component in the position $j$ and $0$s elsewhere. Every $0$ entry in ${\bf B}(4, 12)$ is decomposed into a $72 \times 72$ zero matrix (ZM). The CPM/ZM dispersion of ${\bf B}(4, 12)$ results in a $4 \times 12$ array ${\bf H}(4, 12)$ of CPMs and ZMs of size $72 \times 72$ which is a $288 \times 864$ binary matrix with constant column weight $3$ and two different row weights $8$ and $10$. 
As a matrix, ${\bf H}(4, 12)$ satisfies the RC-constraint \cite{LinBook3}, 
i.e., no two rows (or two columns) in ${\bf H}(4, 12)$ have more than one position where they both have $1$-component. The RC-constraint structure of ${\bf H}(4, 12)$ ensures that the girth of the Tanner graph of ${\bf H}(4, 12)$ is at least $6$.     

   

Replace all the 1-components in a CPM of ${\bf H}(4, 12)$ by the same nonbinary element chosen at random from the field GF($2^4$). This results in a $16$-ary CPM of size $72 \times 72$. 
The random replacement of each binary CPM in ${\bf H}(4, 12)$ by a $16$-ary CPM results in a $4 \times 12$ array ${\bf H}_q(4, 12)$ of $16$-array CPMs and ZMs of size $72 \times 72$.
The array ${\bf H}_q(4, 12)$ is a $288 \times 864$ matrix over GF($2^4$).    
The subscript ``$q$'' of ${\bf H}_q(4, 12)$ stands for $2^4 = 16$. 
The null space of ${\bf H}_q(4, 12)$ gives a $16$-ary $(864,576)$ LDPC code $C_q$ of length $N=864$, dimension $K=576$ and rate $R_c = 0.6667$. The BER performance of the code over an AWGN channel decoded with 50 iterations of the FFT-QSPA as shown in Fig. 8 with $J = 1$.



\subsection{BER Performances of a Coded Downlink and a Coded Uplink FFMA Systems}  

In the following, we present the BER performances of a downlink and an uplink FFMA systems 
in which the $16$-ary $(864,576)$ LDPC code $C_q$ constructed above is used for error control. 
Both systems are designed to support maximum $J=4$ users.


First, we consider the downlink FFMA system of GF($17$) which supports $J = 4$ users for transmission over frequency-selective fading channels. We evaluate its error performance for two cases. In the first case, the number of subcarriers is set to $N_c = 64$ and the length of CP used is $N_g = 16$. For the second case, the number of subcarriers is set to $N_c = 256$ and the length of CP used is also $N_g = 16$. For both cases, we assume the transmissions are over either ${\mathcal L} = 4$ paths or ${\mathcal L} = 16$ paths. For comparison, we also evaluate the performance of an equivalent OFDMA system which supports $J = 4$ users. In this system, each user is assigned with $N_c/J$ carriers, i.e., either $64/4 = 16$ subcarriers or $256/4 = 64$ subcarriers.
   

The BER performances of the downlink FFMA and the OFDMA systems with the above design parameters are shown in Fig. 7. From Fig. 7, we see that for a given $N_c$ and $\mathcal L$, the proposed downlink FFMA system provides a slight better BER performance than that of the classical downlink OFDMA system. 
Since the proposed FFMA can exploit all the sub-carriers to transmit signals, thus it can achieve all frequency-domain diversity gain. 


Moreover, it is found that the BER of $N_c=64$ is generally better than the case of $N_c=256$, because of the utilization of LDPC code. For the given LDPC code $C_q$, a smaller block, i.e., $N_c = 64$, indicates a larger interleaving in frequency-domain, which may improve the BER performance. In addition, the BER performance of the case with a larger number of independent paths, i.e., $\mathcal L = 16$, 
provides a slightly better BER performance than the case with a small number of independent paths, i.e., $\mathcal L=4$. Since a larger $\mathcal L$ may result in a serious frequency-selective fading, we can achieve frequency diversity gain by using channel code.


\begin{figure}[t]
  \centering
  \includegraphics[width=0.5\textwidth]{Graph_DL.pdf}
  \caption{BER performance of the proposed downlink FFMA system of GF($17$) over frequency-selective fading channels.}
\end{figure}


For the uplink case, we consider the uplink FFMA system of GF($2^4$) over an AWGN MA channel in which the $16$-ary $(864, 576)$ NB-LDPC code $C_q$ is used for error control. 
This system can support $J = 1, 2, 3, 4$ users. The BER performances of the system for various users are shown in Fig. 8. 
The NB-LDPC code $C_q$ is decoded with 50 iterations of the FFT-QSPA \cite{LinBook3}. 
Also included in Fig. 8 is BER performance of the system without channel code,
which is found exactly in accord with the theoretical BER performance of BPSK modulation.
For $J = 1$, the curve is the BER performance of the $16$-ary $(864,576)$ NB-LDPC code $C_q$ over an AWGN channel for a single user.


When channel code is utilized,
the BER performance of an uplink FFMA system very much depends on the probabilities of the elements in $\Omega$. 
Two different sets of probabilities of the elements in $\Omega$ may results in a big gap in decoding of the error control code $C_q$. To see that, we consider two cases with $J = 4$. In Case 1, we directly exploit the given $\Omega$ and ${\mathcal P}_r$, i.e.,
$\Omega = \{-4,-2,0,+2,+4\}$ and ${\mathcal P}_r=\{0.0625,0.25,0.375,0.25,0.0625\}$. 
Case 2 is based on the systematic form of an encoded codeword. 
If the information vector ${\bf u}_j$ is encoded into a codeword ${\bf v}_j$ in systematic form, 
i.e., ${\bf v}_j = ({\bf u}_j, {\bf v}_{j,red})$, 
where ${\bf v}_{j,red}$ is the inserted redundancy vector. 
In this case, we can set  $\Omega = \{-4,-2\}$ and ${\mathcal P}_r = \{0.5,0.5\}$ to the first $K$ received information symbols and
$\Omega = \{-4,-2,0,+2,+4\}$ and ${\mathcal P}_r=\{0.0625,0.25,0.375,0.25,0.0625\}$. 
to the next $N-K$ received parity symbols. 
From Fig. 8, we see that the system in Case 2 performs better than the system in Case 1, 
because more accurate probabilities of the elements $\Omega$ are used. 
In addition, the BER performance of the uplink FFMA system becomes slightly worse as the number of user increases. However, the small performance degradation is acceptable.




\begin{figure}[t]
  \centering
  \includegraphics[width=0.5\textwidth]{Graph_UL.pdf}
  \caption{BER performance of the proposed uplink FFMA system of GF($2^4$) over an AWGN multi-access channel, where $J = 1, 2, 3, 4$. $16$-ary $(864, 576)$ NB-LDPC code $C_q$ is used for error control.}
\end{figure}




%\vspace{0.2in}
\section{Conclusion and Remarks}

In this paper, we presented an FFMA technique to support multiuser transmission in algebraic domain using finite fields. Unlike the classical complex-field MA techniques, an FFMA system is operated and processed in finite-fields based on their algebra structures.

We first presented multiuser UD codes over finite fields. 
A multiuser UD code consists of a collection of EPs over a finite field with unique mapping structural property. Each EP in a multiuser UD code is a pair of additive inverses in a chosen finite field. 
In the paper, two classes of multiuser UD codes were presented. The first class of multiuser UD codes is constructed based on prime fields. The second class of multiuser UD codes is constructed based on extension fields of prime fields. 
The multiuser UD codes in the second class have orthogonal structure. Bounds on the numbers of users that both types of multiuser UD codes can support were derived.


Based on the algebraic structure of multiuser UD codes, we presented a downlink and an uplink FFMA systems. The downlink system is designed based on a prime field GF($p$) for multiuser transmission over frequency-selective fading channels.
It was proved that maximum number $J$ of users that can be supported by such a downlink FFMA is upper bounded by $\log_2 (p-1)$, i.e., $J \le \log_2 (p-1)$. 
In the process of transmission, we also introduced a transform between two different fields for error control coding. We showed that the proposed downlink FFMA system can achieve full diversity gain.


The uplink FFMA system is designed based an extension field GF($2^m$) of the binary field GF(2) for multiuser transmission over an AWGN MA channel. 
For such an uplink FFMA, a collection of $m$ orthogonal UDEPs is used to separate the bit-sequences sent by the users from a received multiplexed sequence at the receiving end. 
In the decoding process, a complex-field to finite-field constraint was introduced for unique recovering the output bit-sequences of the users. 
The capacity and BER performance of such an uplink FFMA were analyzed. It was shown that, the proposed uplink FFMA designed based on GF($2^m$) is likewise a type of TDMA system, where the field extension factor $m$ can be viewed as a finite-field time-domain resource. When each user utilizes BPSK without channel code, the BER of the proposed system is the same as that of an equivalent TDMA system.


In the paper, we used a nonbinary LDPC code for error control in two example FFMA systems, a 4-user downlink, and a 4-user uplink FFMA systems. Simulation results show that the proposed systems perform well.


In construction of multiuser UD codes over a finite field, each EP is a pair of additive elements. Multiuser UD codes in which each EP is a pair of multiplicative inverse elements can also be constructed. Using UDAIEPs, each bit from a user is mapped into an element in an assigned AIEP. 
It would be interesting to investigate the case in which each block of $k$ bits from a user is mapped in a unique $r$-tuple over a finite field such the sets of $r$-tuples assigned to the users form a uniquely decodable code. In further research, we will consider design downlink and uplink FFMA systems based on finite fields GF($p^m$) with $p>2$ and $m>1$.



%\vspace{0.1in}
\begin{thebibliography}{99}
%%%%%%%%%%%%%%%%%%%%%%%%%%%%%%%%%%%%%%
%%Following is references on Power & data rate & channel.





\bibitem{FAdachi1}
F. Adachi, D. Garg, S. Takaoka, and K. Takeda, ``Broadband CDMA Techniques," \textit{IEEE Wireless Communications}, vol. 12, no. 2, pp. 8-18, April 2005.


\bibitem{ZDing2017_survey}
Z. Ding, X. Lei, G. K. Karagiannidis, R. Schober, J. Yuan, and V. K. Bhargava, ``A Survey on Non-Orthogonal Multiple Access for 5G Networks: Research Challenges and Future Trends," \textit{IEEE Journal on Selected Areas in Communications}, vol. 35, no. 10, pp. 2181 - 2195, July 2017.


\bibitem{R1}
Y. Chen et al., ``Toward the Standardization of Non-Orthogonal Multiple Access for Next Generation Wireless Networks," \textit{IEEE Communications Magazine}, vol. 56, no. 3, pp. 19-27, March 2018.



\bibitem{R2}
Q. Wang, R. Zhang, L. Yang, and L. Hanzo, "Non-Orthogonal Multiple Access: A Unified Perspective," \textit{IEEE Wireless Communications}, vol. 25, no. 2, pp. 10-16, April 2018.


\bibitem{QY_ISJ_2019}
Q. Yu, H. Chen and W. Meng, ``A Unified Multiuser Coding Framework for Multiple Access Technologies,'' \textit{IEEE Systems Journal}, vol. 13, no. 4, pp. 3781-3792, 2019. 


\bibitem{LDai2015}
L. Dai, B. Wang, Y. Yuan, S. Han, C. I, and Z. Wang, ``Nonorthogonal multiple access for 5G: Solutions, challenges, opportunities, and future research trends," \textit{IEEE Communication Magazine}, vol. 53, no. 9, pp.74-81, Sept. 2015.


\bibitem{HNiko2013}
H. Nikopour and H. Baligh, ``Sparse code multiple access," \textit{in Proc. IEEE 24th International Symposium on Personal Indoor and Mobile Radio Communications (PIMRC)}, pp. 332-336, Nov. 2013.


\bibitem{HNiko2014}
H. Nikopour, E. Yi, A. Bayesteh, K. Au, M. Hawryluck, H. Baligh, and J. Ma, ``SCMA for downlink multiple access of 5G wireless networks," \textit{in Proc. IEEE GLOBECOM}, pp. 3940-3945, Dec. 2014.





%%%%% Below is about uniquely-decodable codes
\bibitem{Liao1972}
H. H. J. Liao, ``Multiple access channels," Ph.D dissertation, Dept. Electrical Engineering, Univ. Hawaii, Honolulu, HI, 1972.

\bibitem{Kasami1976}
T. Kasami, and Shu Lin, ``Coding for a Multiple-Access Channel," \textit{IEEE Trans. Information Theory}, vol. IT-22, no. 2, pp. 129-137, March 1976.

\bibitem{Chang1976}
S. C. Chang and E. J. Weldon, ``Coding for a T-user Multiple Access Channel," \textit{IEEE Trans. Information Theory}, Vol.IT-25, No.5, pp.684-691, Sept.1979.



\bibitem{Kasami1978}
T. Kasami, and Shu Lin, ``Bounds on the Achievable Rates of Block Coding for a Memoryless Multiple-Access Channel," \textit{IEEE Trans. Information theory}, vol. IT-24, no. 2, pp. 187-197, March 1978.


\bibitem{Peterson1979}
Peterson R, Costello D., ``Binary convolutional codes for a multiple-access channel,'' \textit{IEEE Trans. Information Theory}, vol. 25, no. 1, pp. 101-105, 1979.

\bibitem{Chevillat1981}
P. Chevillat, `` N-user trellis coding for a class of multiple-access channels,'' \textit{IEEE Trans. Information Theory}, vol. 27, no. 1, pp. 114-120, 1981.


\bibitem{Kasami1983}
T. Kasami, Shu Lin, Victor K. Wei, and Saburo Yamamura, ``Graph Theoretic Approaches to the Code Construction for the Two-User Multiple-Access Binary Adder Channel," \textit{IEEE Trans. Information Theory}, vol. IT-29, no. 1, pp. 114-130, 1983.

\bibitem{Van1983}
V. Tilborg H., ``Upper bounds on $|C_2|$ for a uniquely decodable code pair $(C_1,C_2)$  for a two-access binary adder channel," \textit{IEEE Trans. on Information Theory}, vol. 29, no. 3, pp. 386-389, May 1983.



\bibitem{Vanroose1992}
Vanroose, P., Van Der Meulen, E. C., ``Uniquely decodable codes for deterministic relay channels," \textit{IEEE Trans. on Information Theory}, vol.38, no.4, pp. 1203-1212, July 1992.

\bibitem{Jevtic1992}
Jevtic, D. B., ``Disjoint uniquely decodable codebooks for noiseless synchronized multiple-access adder channels generated by integer sets," \textit{IEEE Trans. on Information Theory}, vol.38, no.3, pp. 1142-1146, May 1992.


\bibitem{Fan1995}
P. Fan, M. Darnell, B. Honary, ``Superimposed codes for the multi-access binary adder channel'', \textit{IEEE Trans. Information theory}, vol. 41, no. 4, pp. 1178-1182, 1995.

\bibitem{Khachatrian1998}
Khachatrian G H, Martirossian S S. ``Code construction for the T-user noiseless adder channel,'' \textit{IEEE Trans. Information Theory}, vol. 44, no. 5, pp. 1953-1957, 1998.

\bibitem{Bross1998}
Bross, S. I., Blake, I.F., ``Upper bound for uniquely decodable codes in a binary input N-user adder channel," \textit{IEEE Trans. on Information Theory}, vol. 44, no. 1, pp. 334-340, Jan 1998.

\bibitem{Cheng2001}
Cheng J, Watanabe Y, ``A multiuser $k$-ary code for the noisy multiple-access adder channel,'' \textit{IEEE Trans. Information Theory}, vol. 47, no. 6, pp. 2603-2607, 2001.


\bibitem{Kiviluoto2007}
Kiviluoto, L., Ostergard, P. R. J., ``New Uniquely Decodable Codes for the T-User Binary Adder Channel With $3 \le T \le 5$," \textit{IEEE Trans. on Information Theory}, vol. 53, no. 3, pp. 1219-1220, March 2007.



\bibitem{Yu_P2P}
Q. Yu, W. Meng, and S. Lin, ``Packet Loss Recovery Scheme with Uniquely-Decodable Codes for Streaming Multimedia over P2P Networks," \textit{IEEE Journal on Selected Areas in Communications}, vol.31, no.9, pp. 142-154, Aug. 2013.



%%%%%% LDPC 
%\bibitem{Gallager}
%R. G. Gallager, ``Low-Density Parity-Check Codes, " \textit{IRE Trans. Information Theory}, vol. 8, no. 1, pp. 21- 28, Jan. 1962.

%\bibitem{Tanner}
%R. M. Tanner, ``A recursive approach to low complexity codes, " \textit{IEEE Trans. Information Theory}, vol. 27, no. 9, pp. 533-547, Sept. 1981.

%\bibitem{Polar2009}
%Erdal Arikan, ``Channel Polarization: A Method for Constructing Capacity-Achieving Codes for Symmetric Binary-Input Memoryless Channels,'' \textit{IEEE Trans. Information Theory}, vol. 55, no. 7, pp. 3051-3073, July 2009.

%\bibitem{JDai2016}
%J. Dai, K. Niu, Z. Si, and J. Lin, ``Polar coded non-orthogonal multiple access," \textit{IEEE International Symposium on Information Theory (ISIT)}, Barcelona, pp. 988-992, June 2016.


\bibitem{GDF1989_1}
G. D. Forney Jr., and L. Wei, ``Multidimensional constellations. Part.I: Introduction, figures of merit, and generalized cross constellations," \textit{IEEE Journal on Selected Areas in Communications}, vol. 7, no. 6, pp. 877-892, Aug. 1989.


\bibitem{GDF1989_2}
G. D. Forney Jr., ``Multidimensional constellations. Part. II: Voronio Constellations," \textit{IEEE Journal on Selected Areas in Communications}, vol. 7, no. 6, pp.941-958, Aug. 1989.




%\bibitem{Juane2014}
%J. Li, K. Liu, S. Lin, and K. Abdel-Ghaffar, ``Algebraic quasi-cyclic LDPC codes: Construction, low error-floor, large girth and a reduced-complexity decoding scheme", \textit{IEEE Trans. Commun.}, vol. 62, no. 8, pp. 2626-2637, Aug. 2014.


%\bibitem{R3}
%A. Sanderovich, M. Peleg, and S. Shamai, ``LDPC coded MIMO multiple access with iterative joint decoding," \textit{IEEE Transactions on Information Theory}, vol. 51, no. 4, pp. 1437-1450, April 2005.

%\bibitem{R4}
%J. Zhang, L. Yang, and L. Hanzo, ``Frequency-Domain Turbo Equalisation in Coded SC-FDMA Systems: EXIT Chart Analysis and Performance," \textit{2012 IEEE Vehicular Technology Conference (VTC Fall)}, Quebec City, QC, 2012, pp. 1-5.

%\bibitem{R5}
%J. Zhang, L. Yang, L. Hanzo, and H. Gharavi, ``Advances in Cooperative Single-Carrier FDMA Communications: Beyond LTE-Advanced," \textit{IEEE Communications Surveys \& Tutorials}, vol. 17, no. 2, pp. 730-756, Second quarter 2015.

\bibitem{Shu2009}
William E. Ryan and Shu Lin, Channel Codes classical and Modern, Cambridge University Press, 2009.

\bibitem{John2009}
John G. Proakis, Digital Communications, Fifth Edition, Beijing, Publishing House of Electronics Industry, 2009.

\bibitem{LinBook3}
J. Li, S. Lin, K. Abdel-Ghaffar, W. E. Ryan, and D. J. Costello, ``LDPC Code Designs, Constructions, and Unification," Cambridge University Press, 2017.

\bibitem{Thomas}
Thomas M. Cover, and Joy A. Thomas, ``Elements of Information Theory,'' Tsinghua University Press, 2010.



\end{thebibliography}


\vfill
\end{document}

