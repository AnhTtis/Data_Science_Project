\section{Methodology}
\label{sec:methodology}
Next, we delineate our methodology for conducting this systematic review (\S\ref{conducting} and \S\ref{limitations}) and provide our positionality statement (\S\ref{positionality}).




\begin{figure}[tb!]
  \centering
  \includegraphics[width=\linewidth]{figures/methodologyNew2.pdf}
  \caption{\textbf{Illustration of the literature review methodology.} A high-level overview of the process, including the querying, manual extraction, and filtering, and one of the main results. Note that for readability purposes, we present a simplified version of our query and data extraction. Our query retrieves $\sim55\%$ of all IMWUT publications, while our eligibility assessment filtering ends up with 49 papers ($\sim9\%$ of retrieved papers). We notice that only a very small fraction of all IMWUT papers looks at fairness issues, with only a small deviation across years.  \label{fig:methodology}}
\end{figure}
\subsection{Conducting the Literature Review\label{conducting}}
We followed \rev{Kitchenaham and Charters'} protocol \rev{\cite{kitchenham2007guidelines}} for conducting systematic reviews to ensure the quality of included works and limit the initial retrieved papers. At least three authors were involved at each step to minimize the effects of bias and priming. In accordance with this protocol, we initially identified the need for a systematic review, as discussed in Section~\ref{introduction}, namely to explore where UbiComp fairness overlaps with traditional fairness definitions, \rev{in what ways and to which extend}. Figure~\ref{fig:methodology} provides a high-level overview of the process.
\smallskip 

\noindent\textbf{Paper Identification \& Screening.} 
UbiComp crosses several fields, ranging from HCI, Hardware and Software Systems, and Knowledge Discovery and Data Mining. For the scope of this review, we focused on the Proceedings of the ACM on Interactive, Mobile, Wearable and Ubiquitous Technologies (IMWUT), a premier journal in the UbiComp community, also placed among the top-3 publications in HCI.
For the search process, we utilized the ACM Digital Library, focusing on papers that were published in the last five years \rev{(2018-2022)} to capture emerging trends in fairness and UbiComp research. Apart from year filtering, for the most part, we did not limit our search to meta-data, such as titles, keywords, and abstracts, but rather we expanded it to any searchable field, including full text. That excludes the first part of the query, which tries to match terms such as wearable(s) or mobile(s) only in the papers' meta-data, as seen in Figure~\ref{fig:query}. 
\smallskip


\noindent\textbf{Query Definition.} For the definition of our query, we followed similar terminology with relevant review papers in the fairness literature \cite{le2022survey,caton2020fairness}. Additionally, according to Fjeld et al.'s analysis of prominent AI principles documents, \cite{fjeld2020principled}, ``the fairness and non-discrimination theme is the most highly represented theme in our dataset, with every document referencing at least one of its six principles: ``non-discrimination and the prevention of bias'', ``representative and high-quality data'', ``fairness'', ``equality'', ``inclusiveness in impact'', and ``inclusiveness in design'', mostly included in our query's coverage. To capture the industrial perspective, we consulted the Responsible Artificial Intelligence (RAI) white papers issued by large tech companies. Specifically, Google's\footnote{\url{https://ai.google/responsibilities/responsible-ai-practices/?category=fairness}} and Meta's\footnote{\url{https://ai.facebook.com/blog/facebooks-five-pillars-of-responsible-ai/}} RAI principles talk about ``fairness and inclusion'', Amazon's\footnote{\url{https://aws.amazon.com/machine-learning/responsible-machine-learning/}} RAI principles promote ``diversity, equity, and inclusion'' through ``detecting bias''. Similarly, Nokia's\footnote{\url{https://www.bell-labs.com/institute/blog/introducing-nokias-6-pillars-of-responsible-ai/}} RAI fairness pillar talks about ``fairness, non-discrimination, accessibility, and inclusivity''. Thus, an iterative refinement process resulted in the query shown in Figure~\ref{fig:query}. \rev{Alternative queries assessed are given in Appendix~\ref{ap:queries}.}

\begin{figure}[tb!]
  \centering
  \includegraphics[page=1,width=.75\linewidth]{figures/queries.pdf}
  \caption{\textbf{The query utilized for recovering relevant papers from the ACM Digital Library}. Terms related to UbiComp are highlighted in green, ML in orange, and fairness in purple.\label{fig:query}}
\end{figure}
\smallskip 

\noindent\textbf{Eligibility Assessment.} To further validate our query, we manually inspected all publications from the latest IMWUT proceedings (Volume 6, Issue 4, published in January 2023) ($N=56$) to identify eligible papers for inclusion (see inclusion and exclusion criteria below). In total, we identified seven relevant publications, all of which were also returned by our query. This process was irrelevant to our final paper retrieval (pictured in Figure~\ref{fig:prisma}) and served validation purposes only. To ensure the high quality and relevance of the included papers, we defined appropriate exclusion criteria that helped us determine the included papers:
\begin{enumerate}
    \item Papers that do not provide a quantitative assessment of at least one empirical or artifact contribution in \rev{mobile and wearable} computing (UBI);
    \item Papers that do not include a quantitative assessment of bias or performance discrepancy in their evaluation with regard to sensitive attributes, such as age, gender, race, disability, religion, and sexual orientation, among others (FAIR);
    \item Papers that discuss different domains, such as natural language processing or computer vision, without incorporating a ubiquitous component (DOM);
    \item Papers that refer to bias in a different context, such as the bias-variance trade-off or the bias parameter in neural networks (CON). 
\end{enumerate}

\begin{figure}[tb!]
  \centering
  \includegraphics[width=.9\linewidth]{figures/prismaNew2.pdf}
  \caption{\textbf{PRISMA flow diagram for paper inclusion}. Out of the 611 papers retrieved by our query after the screening, only 9\% ($N=49$) did not check any exclusion criterion and thus were included in the literature review.  \label{fig:prisma}}
\end{figure}
\noindent\textbf{Inclusion \& PRISMA Statement.} \rev{Figure~\ref{fig:prisma} shows the Preferred Reporting Items for Systematic Reviews and Meta-Analyses (PRISMA) \cite{moher2009preferred} flow diagram.} Specifically, the sequential execution of the steps above led to our review's included papers. Overall, we screened 523 papers after date filtering and duplicate elimination. We then excluded 474 based on our exclusion criteria \rev{(UBI: 6.5\%, FAIR: 83.1\%, DOM: <0.5\%, CON: 8.9\%, OTHER: 1.1\%)}. Hence, we included 49 papers in our review\footnote{To foster reproducibility, upon acceptance, we intend to release the review data and codebooks}. 


\subsection{Methodological Limitations}
\label{limitations}
While we have made every effort to ensure broad coverage of papers relevant to fairness in UbiComp studies, our search for literature might not be exhaustive. However, covering IMWUT as a prominent academic venue for ubiquitous computing research allowed us to capture emerging trends. Besides, we intended to provide insights and research directions to the IMWUT community; therefore, we narrowed down our research to IMWUT proceedings. \rev{Nevertheless, to generalize our findings
across venues of similar scope, we additionally analyzed recent proceedings of the ACM MobiCom, MobiSys, SenSys, IEEE Pervasive, and IEEE Trans. Mob. Comp. (\S\ref{generalizability}).} We also acknowledge that despite our best efforts to pick literature-driven keywords and manually validate the retrieved results (\rev{\S\ref{conducting}}), the output might have produced both false positives and false negative results. \rev{For instance, terms such as ``machine learning'' or ``artificial intelligence'' might not be explicitly mentioned in certain papers not covered by our manual validation.} \rev{Additionally, while our keywords (and scope) focus primarily on mobile and wearable computing, we acknowledge that UbiComp expands beyond those to Smart Environments, Internet of Things (IoT), and Augmented and Virtual Reality (AR/VR) research, among others, which we consider beyond the scope of this work. Yet, while such areas might be underrepresented due to the query choice, we do encounter instances of relevant works in the included papers \cite{10.1145/3448111,10.1145/3264899,10.1145/3432235,10.1145/3411807,wu2022g2auth}. Finally, while we consider the past 5 years of publications in an effort to capture emerging trends, we acknowledge that there might be instances of early fairness works missed.} 




\subsection{Positionality Statement\label{positionality}}
Understanding researcher positionality is essential to demystifying our lens on data collection and analysis~\cite{frluckaj2022gender, havens2020situated}. We situate this review paper in a Western country (REDACTED FOR REVIEW) in the 21\textsuperscript{st} century, writing as authors who primarily work as academic and industry researchers. We identify as two females and four males, and our shared backgrounds include HCI, ML, ubiquitous computing\rev{, and RAI}. 
