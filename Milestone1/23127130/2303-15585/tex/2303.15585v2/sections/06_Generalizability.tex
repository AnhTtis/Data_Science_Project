\section{\rev{How is fairness discussed across UbiComp venues?}}\label{generalizability}

\begin{table}[]
\caption{\textbf{To generalize these results across venues of similar scope, we analyzed recent proceedings of ACM MobiCom, MobiSys, and SenSys, and IEEE Trans. of Mobile Computing and Pervasive Computing, and found no deviation from our primary result.} Out of the 514 total papers published in 2022 in these venues, 245 were retrieved by our query, and only 12 complied with our screening eligibility criteria.}
\label{tab:venues}
\begin{tabular}{llll}
\rowcolor[HTML]{1C9E78} 
{\color[HTML]{FFFFFF} \textbf{Venue}} & {\color[HTML]{FFFFFF} \textbf{\#Published (2022)}} & {\color[HTML]{FFFFFF} \textbf{\#Retrieved (\% total)}} & {\color[HTML]{FFFFFF} \textbf{\#Included (\% total)}} \\
{ACM IMWUT}           & {154}                          & {131 (85\%)}                       & {15 (10\%)}                        \\
\rowcolor[HTML]{F3F3F3} 
MobiCom                               & 56                                                 & 25 (45\%)                                              & 0 (0\%)                                               \\
MobiSys                               & 38                                                 & 18 (47\%)                                              & 3 (8\%)                                               \\
\rowcolor[HTML]{F3F3F3} 
SenSys                                & 52                                                 & 8 (15\%)                                               & 2 (4\%)                                               \\
IEEE Pervasive                        & 49                                                 & 15 (31\%)                                              & 2 (4\%)                                               \\
\rowcolor[HTML]{F3F3F3} 
IEEE Trans. Mob. Comp.              & 319                                                & 179 (56\%)                                             & 5 (2\%)                                               \\
{\textbf{Total}}           & {\textit{668}}                & {\textit{376 (56\%)}}             & {\textit{27 (4\%)}}                                  
\end{tabular}
\end{table}
\rev{To evaluate how our results generalize across UbiComp venues, we reviewed last year's (2022) publications from relevant mobile and wearable computing conferences and journals. Specifically, we reviewed ACM SIGMOBILE-sponsored events, including the International Conference on Mobile Computing and Networking (MobiCom, $N_{2022}=56$), the International Conference on Mobile Systems, Applications, and Services (MobiSys, $N_{2022}=38$), and the ACM Conference on Embedded Networked Sensor Systems (SenSys, $N_{2022}=52$), and relevant IEEE journals, including IEEE Pervasive Computing (IEEE Pervasive, $N_{2022}=49$) and IEEE Transactions on Mobile Computing (IEEE Trans. Mob. Comp., $N_{2022}=319$). Following the methodology described in Section~\ref{sec:methodology}, out of 245 retrieved papers, 12 papers complied with our screening and eligibility criteria: none from MobiCom (0\% inclusion rate), 3 from MobiSys (8\%), 2 from SenSys (4\%), 2 from IEEE Pervasive (4\%), and 5 from IEEE Trans. Mob. Comp. (2\%). Table~\ref{tab:venues} summarizes our analysis. The papers came from domains overlapping with those identified at IMWUT: \textit{Sound, Voice, \& Hearing} (33.3\%) \cite{chatterjee2022clearbuds,jin2022earhealth,xie2021hearsmoking,xu2020leveraging}, \textit{Privacy \& Security} (25\%) \cite{wu2022g2auth,zhao2020fingertip,li2020characterising}, \textit{Health} (25\%) \cite{xu2022hearing,stasak2022breaking,lamichhane2022econet}, \textit{Motion, Gaze, Gesture \& Touch} (8.3\%) \cite{cao2022gaze}, and \textit{Mobility \& Navigation} (8.3\%) \cite{hu2020mathsf}.}

\rev{Of all reviewed papers, we found that only 2\% adhered to fairness reporting ($N_{included}=12$), slightly below half the percentage published in IMWUT, highlighting the timeliness and the need for this review (takeaway \#1). At the same time, none introduced fairness enhancement mechanisms or utilized fairness metrics (takeaways \#2 and \#4). Similarly to IMWUT, physiology \cite{chatterjee2022clearbuds,xu2022hearing,xie2021hearsmoking,xu2020leveraging,zhao2020fingertip}, age \cite{jin2022earhealth,wu2022g2auth,stasak2022breaking}, gender \cite{jin2022earhealth,wu2022g2auth}, and health conditions \cite{stasak2022breaking,lamichhane2022econet} were the most frequently considered protected attributes, whereas there were no mentions of race, nationality, or language (takeaway \#3). Nevertheless, once again, in-the-wild deployments and ablation studies were widespread amongst reviewed papers, considering user-related \cite{jin2022earhealth,cao2022gaze,xie2021hearsmoking,xu2020leveraging}, device-related \cite{wu2022g2auth,cao2022gaze,li2020characterising}, environmental \cite{chatterjee2022clearbuds,wu2022g2auth,xie2021hearsmoking,xu2020leveraging,zhao2020fingertip}, experimental \cite{xu2022hearing,cao2022gaze,xie2021hearsmoking,zhao2020fingertip}, and domain-specific components \cite{zhao2020fingertip} (takeaway \#6). Finally, following our IMWUT findings, users were recruited mainly from the USA (57\%) and China (43\%). They were generally balanced in terms of gender with a median of 10 females but predominantly young ($\mu_{age}=23.1)$. However, due to the lack of standardization in demographics reporting, data were insufficient to draw conclusions about race, education, and employment distributions (takeaway \#8).} 

\rev{Overall, while we acknowledge the limitations of considering a single publication for this review, these findings provided an indication of the generalizability of our results across UbiComp venues.} 