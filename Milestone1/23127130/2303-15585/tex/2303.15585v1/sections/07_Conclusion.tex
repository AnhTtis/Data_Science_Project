\section{Conclusion}
\label{sec:conclusion}
The field of mobile, wearable and ubiquitous computing (UbiComp) faces significant challenges in ensuring fairness in the development of ML-based UbiComp technologies. Although efforts have been made to address biases, only a small percentage of publications in the Proceedings of the ACM IMWUT journal focus on fairness reporting and enhancement mechanisms. Sensitive attributes such as race, nationality, and language are often overlooked, while it is evident that there is a need for more diverse sample recruitment to ensure that the benefits of these technologies are shared equally across all members of society. The lack of a universal fairness definition, metric, or ``fair'' threshold that applies to different applications poses a sociotechnical challenge. UbiComp researchers must be explicit and transparent about their fairness priorities, definitions, and assumptions, making trade-offs between competing priorities, ethical risks, and opportunities. Despite these challenges, the UbiComp community strives for ``fairer'' models by conducting and reporting ablation studies, in-the-wild vs. in-the-lab experiments, and personalized model development. Ultimately, the UbiComp community must continue to prioritize fairness to ensure that the development of these technologies leads to just and equitable outcomes.