% ****** Start of file apssamp.tex ******
%
%   This file is part of the APS files in the REVTeX 4.2 distribution.
%   Version 4.2a of REVTeX, December 2014
%
%   Copyright (c) 2014 The American Physical Society.
%
%   See the REVTeX 4 README file for restrictions and more information.
%
% TeX'ing this file requires that you have AMS-LaTeX 2.0 installed
% as well as the rest of the prerequisites for REVTeX 4.2
%
% See the REVTeX 4 README file
% 
\documentclass[%
 reprint,
%superscriptaddress,
%groupedaddress,
%unsortedaddress,
%runinaddress,
%frontmatterverbose, 
% preprint,
%preprintnumbers,
%nofootinbib,
%nobibnotes,
%bibnotes,
 amsmath,amssymb,
 aps,
%pra,
%prb,
%rmp,
%prstab,
%prstper,
%floatfix,
]{revtex4-2}

\usepackage{graphicx}% Include figure files
\usepackage{dcolumn}% Align table columns on decimal point
\usepackage{bm}% bold math
\PassOptionsToPackage{hyphens}{url}\usepackage{hyperref}% add hypertext capabilities
%\usepackage[mathlines]{lineno}% Enable numbering of text and display math
%\linenumbers\relax % Commence numbering lines

%\usepackage[showframe,%Uncomment any one of the following lines to test 
%%scale=0.7, marginratio={1:1, 2:3}, ignoreall,% default settings
%%text={7in,10in},centering,
%%margin=1.5in,
%%total={6.5in,8.75in}, top=1.2in, left=0.9in, includefoot,
%%height=10in,a5paper,hmargin={3cm,0.8in},
%]{geometry}

\begin{document}

\preprint{APS/123-QED}

\title{Vorticity wave interaction and exceptional points in shear flow instabilities}

\author{Cong Meng$^{1,2}$ and Zhibin Guo$^{2}$}

 \email{zbguo@pku.edu.cn}

 \affiliation{$^{1}$
 Southwestern Institute of Physics, PO Box 432, Chengdu 610041, China}
 
\affiliation{$^{2}$
 State Key Laboratory of Nuclear Physics and Technology, Fusion Simulation Center, School of Physics, Peking University, Beijing 100871, China
}

\date{\today}% It is always \today, today,
             %  but any date may be explicitly specified

\begin{abstract}
We establish a link between vorticity wave interaction and $\mathcal{PT}$-symmetry breaking in shear flow instabilities. The minimal dynamical system for coupled counterpropagating vorticity waves is shown to be a non-Hermitian system with a saddle-node exceptional point. Vorticity wave dynamics of the system are then linked to the bifurcation of fixed points through the exceptional point and the Krein collision between eigenmodes with opposite Krein signatures. In the vicinity of the exceptional point, the transition of phase dynamics from a phase-slip state to a phase-locking state corresponds to spontaneous $\mathcal{PT}$-symmetry breaking and the onset of Kelvin-Helmholtz instabilities. 
The phase-slip dynamics lead to non-modal, transient growth of perturbations, and the phase-slip frequency $\Omega \propto |k-k_c|^{1/2}$ shares the same critical exponent with the phase rigidity of system eigenvectors. The results can be readily extended to the interaction of multiple vorticity waves with multiple exceptional points and rich transient dynamics.

\end{abstract}

%\keywords{Suggested keywords}%Use showkeys class option if keyword
                              %display desired
\maketitle

%\tableofcontents

\section{\label{sec:intro}Introduction}

Interaction of vorticity waves has been widely demonstrated to be a physical interpretation of shear flow instabilities \cite{baines_1994,Heifetz_1999,heifetz_2004}, and naturally provides a non-modal approach to study optimal growth and transient dynamics in such instabilities \cite{Heifetz_2005,guha_2014,guha_2017}, as thoroughly reviewed in \cite{Carpenter_2011}. The model of counter-propagating vorticity waves can be reframed into a minimal nonlinear dynamical system \cite{Heifetz_2019_PRE,Heifetz_PRE_2022}, where the onset of shear instabilities corresponds to a bifurcation of fixed points from two neutral centers to a pair of stable and unstable nodes \cite{Heifetz_2019_PRE}.

On the other hand, it has recently come to light that shear flow instabilities are the result of spontaneous $\mathcal{PT}$-symmetry breaking \cite{Hong_Qin_2019_PoP,Fu_2020_NJP,Tomos_David_2023}. In the framework of spectral stability analysis of pseudo-Hermitian, or equivalently, G-Hamiltonian systems \cite{Yakubovich1975,Kirillov2013,Kirillov2014}, the mechanism of $\mathcal{PT}$-symmetry breaking and instability onset is illustrated as the Krein collision between eigenmodes with opposite Krein signatures \cite{Ruili_Zhang_2016,Ruili_Zhang_2020}. The results, however, have not yet been related to the dynamics of vorticity waves. 

In this work, we establish a link between dynamics of vorticity wave interaction and $\mathcal{PT}$-symmetry breaking in shear flow instabilities. We show that the minimal dynamical system that describes coupling between counterpropagating vorticity waves is a pseudo-Hermitian system with a saddle-node exceptional point. Vorticity wave dynamics of the system are then directly linked to the spectral stability analysis of the Krein collision and the bifurcation of fixed points through the exceptional point. The key parameter that leads the system to the Krein collision and $\mathcal{PT}$-symmetry breaking is the ratio between frequency detuning and coupling strength of the vorticity waves.

The most striking feature of non-Hermitian systems is the existence of exceptional points \cite{Ashida_2020,Ding_2022}. In this work, special attention is given to the critical behavior and transient dynamics that exhibit in the vicinity of the exceptional points. In particular, we show that the transition of phase dynamics from a phase-slip state to a phase-locking state corresponds to spontaneous $\mathcal{PT}$-symmetry breaking and the onset of Kelvin-Helmholtz instabilities. The phase-slip dynamics lead to non-modal, transient growth of perturbations, and the phase-slip frequency $\Omega \propto |k-k_c|^{1/2}$ shares the same critical exponent with the phase rigidity of system eigenvectors. We also extend the results to the dynamical system of multiple coupled vorticity waves. It is shown that multiple exceptional points provide boundaries for $\mathcal{PT}$-symmetry breaking and onset of shear instabilities, characterized by the transition of phase dynamics between a phase-slip state and a phase-locking state near the exceptional points.

% This work is organized as follows. In Sec. \ref{sec: dynamical system}, we review the general dynamical system that models interaction of vorticity waves in a piecewise shear layer with $N$ interfaces, the generalized dynamical system 

\section{\label{sec: dynamical system}dynamical system analysis of vorticity wave interaction}

Consider a two-dimentional, incompressible and inviscid shear flow without density stratification, where the background state is a mean flow in $y$ direction with profile $U(x)$. The velocity field is $\bm{v}=(v_x,v_y)=(-\partial \phi/\partial y, \partial \phi/\partial x)$ where $\phi$ is the stream function, and the vorticity $\bm{q}=\nabla \times \bm{v}$ of a 2D flow reduces to a scalar $ q=\nabla^2 \phi$. The vorticity equation reads $dq/dt=0$, where $d/dt=\partial/\partial t+\bm{v}\cdot\nabla$. To analyze  stability of the shear flow, denote perturbations of the background state as $v_x=\tilde{v}_x, v_y=U(x)+\tilde{v}_y, q=q_0+\tilde{q}$ and $\phi=\phi_0+\tilde{\phi}$, where the mean vorticity $q_0=\nabla^2 \phi_0=U'$, and mean vorticity gradient $q_0'=\partial q_0 /\partial x=U''$. (We use primes to denote $\partial_x$.) Then the linearized equation of perturbed vorticity is
\begin{equation}
    \frac{\partial \tilde{q}}{\partial t}+U\frac{\partial \tilde{q}}{\partial y}=U'' \frac{\partial \tilde{\phi}}{\partial y}.
     \label{linearized_vorticity}
\end{equation}

Eq. (\ref{linearized_vorticity}) expresses the generation of perturbed vorticity by the background mean vorticity gradient, in terms of the source term $U'' \partial \tilde{\phi}/\partial y$. If the perturbations has normal mode form as $\tilde{\phi}(x,y,t)=\hat{\phi}(x)e^{i(k y-\omega t)}$, then the Taylor-Goldstein equation is obtained for stability analysis of the eigenvalue problem \cite{Drazin2002,Carpenter_2011}. Alternatively, we take the non-modal approach \cite{Nonmodal_2007,Nonmodal_2018} and keep the time-dependent $\tilde{\phi}(x,y,t)=\hat{\phi}(x,t)e^{ik y}$ and $\tilde{q}(x,y,t)=\hat{q}(x,t)e^{ik y}$, with Fourier transform in $y$ direction. Then the linearized initial value problem is 
\begin{subequations}
\begin{gather}
    \frac{\partial \hat{q}}{\partial t}+i k U \hat{q}=ik U''\hat{\phi}, 
    \label{nonmodal_1}
    \\
    \hat{q}=\hat{\phi}''-k^2\hat{\phi}.
    \label{nonmodal_2}
    \end{gather}
\end{subequations}

Eq. (\ref{nonmodal_2}) is the Poisson equation and can be written as 
\begin{equation}
    \hat{\phi}(x)=\int G(x,x')\hat{q}(x')dx',
    \label{q_phi_Green}
\end{equation}
where the Green function kernel describes non-local $\hat{\phi}-\hat{q}$ coupling \cite{Mao_2022}. For boundary conditions $\hat{\phi}(\pm \infty)=0$, the kernel $G(x,x')=-e^{-k|x-x'|}/(2k)$. 

For a discretized, piecewise profile of background shear layer, neutrally stable vorticity waves are induced on each interface of vorticity jump, and the direction of intrinsic propagation is determined by the sign of the vorticity gradient \cite{Carpenter_2011}. In general, consider a piecewise shear layer with $N$ interfaces, 
\begin{equation}
    q_0'(x)  = U''(x)= \sum_{j=1}^{N}\Delta \bar{q}_j\delta(x-x_j),
\end{equation}
where $\Delta \bar{q}_j=q_0(x_j^+)-q_0(x_j^-)$ is the vorticity jump across the interface $x=x_j$.
The perturbations thus vanish at all locations except at the interfaces, \textit{i.e.}, $\hat{q}(x,t)=\sum_{j=1}^N q_j(t)\delta(x-x_j)$, and $\hat{\phi}(x,t)=\sum_{j=1}^N \phi_j(t)\delta(x-x_j)$. 
Eq. (\ref{q_phi_Green}) is then translated into 
\begin{equation}
\phi_i(t)=\sum_{j=1}^N G_{ij} q_j(t), \hspace{6mm} G_{ij}=-\frac{e^{-k|x_i-x_j|}}{2k}.
\label{discretized_poisson}
\end{equation}

Let $q_j(t)=Q_je^{i\theta_j}$, $\theta_{ij}=\theta_i-\theta_j$, 
Eq. (\ref{nonmodal_1}) can be written as
\begin{subequations}
\begin{gather}
    \frac{dQ_i}{dt}= k \Delta \bar{q}_i \sum_{j \neq i} G_{ij} Q_j \sin\theta_{ij} , 
    \label{vortex_coupling_amplitude}
\\
    \frac{d\theta_i}{dt}=-\omega_i + k  \Delta \bar{q}_i \sum_{j \neq i} G_{ij} \frac{Q_j}{Q_i} \cos\theta_{ij}.
    \label{vortex_coupling_phase}
    \end{gather}
\end{subequations}

The frequency of each individual mode
\begin{equation}
    \omega_i=k U_i + \frac{\Delta \bar{q}_i}{2}
    \label{vorticity wave frequency}
\end{equation}
includes both the intrinsic frequency $ \Delta \bar{q}_i/2$ and the Doppler shift by the \textit{in situ} shear flow $U_i$. Eq. (\ref{vorticity wave frequency}) tells that each single interface of background vorticity jump $\Delta \bar{q}_i$ induces a neutrally stable vorticity wave, and the direction of phase speed $v_{ph}=- \Delta \bar{q}_i/(2k)$, in the reference frame moving with $U_i$, is determined by the sign of vorticity jump. 

Eq. (\ref{vortex_coupling_amplitude}) and (\ref{vortex_coupling_phase}) describes a dynamical system that models vorticity wave interaction in a general shear layer with multiple interfaces. The system has a canonical Hamiltonian representation in action-angle form \cite{Heifetz_2009,heifetz_guha_2018}, with the Hamiltonian 
\begin{equation}
    H= \sum_i \omega_i A_i-\frac{1}{2} \sum_{j\neq i} G_{ij}Q_iQ_j\cos\theta_{ij} = -\sum_i A_i \dot{\theta}_i,
\end{equation}
where $A_i=Q_i^2/(2k\Delta \bar{q}_i)$ is the action of each vorticity wave. The canonical Hamiltonian equation follows as 
\begin{equation}
    \frac{\partial H}{\partial A_i}=-\dot{\theta}_i, \hspace{6mm} \frac{\partial H}{\partial \theta_i} =\dot{A}_i. 
\end{equation}

To provide concrete examples of our main results, we consider minimal dynamical systems with two or three interfaces, as shown in Fig.~\ref{fig:profile}, in the remainder of the work.

\begin{figure}[ht!]
\includegraphics[scale=0.6]{U_profile.eps}
\caption{\label{fig:profile} Profiles of the background shear layer and corresponding vorticity waves. (a) A piesewise shear layer profile that supports two counter-propagating vorticity waves, rescaled as $U(x_1)=-1$ and $U(x_2)=1$. (b) A saw-tooth jet profile with three interfaces, rescaled as $U(x_1)=U(x_3)=-1$ and $U(x_2)=1$. The arrows represent the direction of intrinsic propagation of vorticity waves.}
\end{figure}

\section{\label{sec: two waves}Interaction of two vorticity waves}

\subsection{\label{subsec: two wave dynamical system}Dynamical system analysis}

For a piecewise shear layer shown in Fig. 1(a), the interaction of two vorticity waves can be described by a minimal nonlinear dynamical system \cite{Heifetz_2019_PRE,guha_2014}, 
\begin{subequations}
\begin{gather}
    \dot{Q}_1=-\sigma Q_2 \sin \theta_{12}, \hspace{4mm} \dot{Q}_2=-\sigma Q_1 \sin \theta_{12}, 
    \label{two_vorticity_wave_system_a} 
    \\
    \dot{\theta}_1= -\omega_1-\sigma \frac{Q_2}{Q_1} \cos \theta_{12}, \hspace{4mm} \dot{\theta}_2= -\omega_2 + \sigma \frac{Q_1}{Q_2} \cos \theta_{12}
    \label{two_vorticity_wave_system_b}
\end{gather}
\end{subequations}
where $\omega_1=-k+1/2$, $\omega_2=k-1/2$,  and the coupling coefficient $\sigma=e^{-2k}/2$. The complex form of the dynamical system Eqs. (\ref{two_vorticity_wave_system_a}) and (\ref{two_vorticity_wave_system_b}) is 
\begin{equation}
    \begin{split}
        \dot{q}_1=-i\omega_1 q_1-i\sigma q_2,
        \\
        \dot{q}_2=-i \omega_2 q_2+i \sigma q_1,
        \label{complex_two_vorticity_wave_system}
    \end{split}  
\end{equation}
or, in matrix form,
\begin{equation}
    \dot{\bm{q}}=\mathbf{A}\bm{q}, \hspace{8mm} \bm{q}=\begin{pmatrix} q_1\\q_2\end{pmatrix},
\end{equation}
where 
\begin{equation}
    \mathbf{A}=-i\mathbf{H}=-i\begin{pmatrix}
    \omega_1 & \sigma \\ -\sigma &  \omega_2
        % \Delta \omega/2 & \sigma \\ -\sigma & -\Delta \omega/2
    \end{pmatrix}
\end{equation}
is a G-Hamiltonian matrix \cite{Yakubovich1975,Ruili_Zhang_2016,Ruili_Zhang_2020}. That is, $\mathbf{A}$ has the form of
$\mathbf{A}=-i \mathbf{G}^{-1}\mathbf{S}$, where $\mathbf{G}$ is a non-singular Hermitian matrix and $\mathbf{S}$ is a Hermitian matrix. Denote the frequency mismatch $\Delta \omega = \omega_1-\omega_2 = 1-2k$, then
\begin{equation}
    \mathbf{G}= \frac{1}{2k} 
    \begin{pmatrix}  1 & 0 \\ 0 & -1 \end{pmatrix},  \hspace{4mm}
    \mathbf{S}=\frac{1}{2k}  \begin{pmatrix}
         \frac{\Delta \omega}{2} & \sigma \\ \sigma & \frac{\Delta \omega}{2}
    \end{pmatrix}.
\end{equation}

The corresponding Hamiltonian is 
\begin{gather}
    H(\bm{q})=\bm{q}^{\dag}\mathbf{S}\bm{q}\\
    =\omega_1 \frac{Q_1^2}{2k}-\omega_2 \frac{Q_2^2}{2k}+\frac{\sigma}{k}Q_1Q_2\cos \theta_{12},
\end{gather}
where $\bm{q}^{\dag}$ is the conjugate transpose of $\bm{q}$, and the complex canonical Hamiltonian equation has the form
\begin{equation}
    \dot{\bm{q}}=-i\mathbf{G}^{-1}\frac{\partial H}{\partial \bm{q}^*}.
    \label{complex_canonical_equation}
\end{equation}

The system conserves two constants of motion, one is the Hamiltonian and the other is the total wave action. Define an indefinite inner product \cite{Kirillov2013} as 
\begin{equation}
    [\bm{x},\bm{y}]=(\mathbf{G}\bm{x},\bm{y})=\bm{y}^{\dag}\mathbf{G}\bm{x},
\end{equation}
then the total wave action can be written as
\begin{equation}
    A(\bm{q})=[\bm{q},\bm{q}]=\frac{Q_1^2}{2k}-\frac{Q_2^2}{2k}=A_1+A_2.
\end{equation}

The onset of Kelvin-Helmholtz instability can then be described via the dynamical system approach. Let $R_{12}=Q_1/Q_2$, and rewrite the dynamical system as
\begin{subequations}
\begin{gather}
    \dot{R}_{12}=\sigma (R_{12}^2-1)\sin \theta_{12},
    \label{dynamical_R_12}
    \\
    \dot{\theta}_{12}=-\Delta \omega -\sigma \big( R_{12}+\frac{1}{R_{12}}\big) \cos \theta_{12}.
    \label{dynamical_theta_12}
\end{gather}
\end{subequations}

The fixed points of the system require either $\sin \theta_{12}=0$, where the two vorticity waves remain neutrally stable, or
\begin{equation}
    R_{12}=1, \hspace{4mm} \cos \theta_{12}=-\frac{\Delta \omega}{2\sigma}=(2k-1)e^{2k},
\end{equation}
where the phase-locking of two vorticity waves trigger mutual growth and instability onset when $-\pi<\theta_{12}<0$, or mutual damping when $0<\theta_{12}<\pi$. The parameter space for instability onset is then $-1<(2k-1)e^{2k}<1$ and $0<k<k_c \simeq 0.64$, as shown in Fig.~\ref{fig: fixed_point_two_wave}. The onset of shear instability corresponds to a bifurcation of fixed points from two neutral centers to a pair of stable and unstable nodes \cite{Heifetz_2019_PRE}. When $k \geq k_c$, the phase detuning $\Delta \omega$ dominates over coupling $\sigma$ between two vorticity waves, and the system has two neutrally stable fixed points. When $0<k< k_c$, the $-\pi<\theta_{12}<0$ fixed point is stable, as a dynamical sink of phase trajectories (see Fig. 6 in Ref.~\cite{Heifetz_2019_PRE}), corresponding to mutual growth and instability onset; and the $0<\theta_{12}<\pi$ fixed point is unstable, as a source of phase trajectories, corresponding to unphysical, mutual damping of initial perturbations.  
\begin{figure}[ht!]
\includegraphics[scale=0.6]{fig_2}
\caption{\label{fig: fixed_point_two_wave} Parameter space of the two vorticity wave system and bifurcation of fixed points with the control parameter $k$. The black vertical dashed line is the critical $k=k_c\simeq0.64$. When $k \geq k_c$, the system has two neutrally stable fixed points, located at $\theta_{12}=0$, with unbroken $\mathcal{PT}$-symmetry. When $0<k<k_c$, the system has a pair of stable (solid red line, $-\pi<\theta_{12}<0$) and unstable (dashed red line, $0<\theta_{12}<\pi$) fixed points, and $\mathcal{PT}$-symmetry is spontaneously broken. The two vorticity waves become phase-locked and lead to Kelvin-Helmholtz instability.}
\end{figure}
\subsection{Krein collision and $\mathcal{PT}$-symmetry breaking\label{subsec: Krein collision}}
The dynamics of vorticity wave interaction provides a clear physical description and a non-modal approach to understand the Kelvin-Helmholtz instability. Here, we show that it also provides a natural and simple testbed for the rich non-Hermitian physics, and thus, can be naturally linked to the fact that Kelvin-Helmholtz instability is the result of spontaneous $\mathcal{PT}$-symmetry breaking, which has been shown by normal mode analysis in Ref.~\cite{Hong_Qin_2019_PoP,Fu_2020_NJP,Tomos_David_2023}. 

For the two-vorticity wave dynamical system, the parity operator $\mathcal{P}$ and the time reversal operator $\mathcal{T}$ acting on vorticity fluctuation variables $\{q_1(x,t),q_2(x,t)\}$ are written as
\begin{equation}
    \mathcal{P}=\begin{pmatrix} 1 & 0 \\ 0 & 1 \end{pmatrix} \hspace{4mm} \text{and} \hspace{4mm} \mathcal{T}=\begin{pmatrix} -1 & 0 \\ 0 & -1 \end{pmatrix} \mathcal{K},
\end{equation}
satisfying $\mathcal{P}^2=\mathcal{T}^2=I_{2\times 2}$, where $\mathcal{K}$ is the complex-conjugate operator \cite{Tomos_David_2023}. One can readily see that the Hamiltonian $\mathbf{H}$ is $\mathcal{PT}$-symmetric, \textit{i.e.}, $\mathcal{PT}\mathbf{H}-\mathbf{H}\mathcal{PT}=0$. Also, $\mathbf{H}=\mathbf{G}^{-1}\mathbf{H}^{\dag}\mathbf{G}$ is similar to its conjugate transpose $\mathbf{H}^{\dag}$, so that $\mathbf{H}$ is a pseudo-Hermitian Hamiltonian. 
% Therefore, according to \textit{Theorem 2} in Ref.~\cite{Ruili_Zhang_2020},
% Also, as shown in Ref.~\cite{Ruili_Zhang_2020}, for a finite-dimensional system $\dot{\bm{q}}=-i\mathbf{H}\bm{q}=\mathbf{A}\bm{q}$, 
% a $\mathcal{PT}$-symmetric Hamiltonian $\mathbf{H}$ is necessarily pseudo-Hermitian.
Note that to say $\mathbf{H}$ is pseudo-Hermitian is equivalent to say that the matrix $\mathbf{A}$ is G-Hamiltonian \cite{Ruili_Zhang_2020}, and the systematically developed theory of G-Hamiltonian systems \cite{Yakubovich1975} can be readily applied. In particular, the \textit{Krein-Gel'fand-Lidskii theorem} \cite{Yakubovich1975} proves that a G-Hamiltonian system is strongly stable (\textit{i.e.}, the stability of the system is preserved by any infinitesimal deformation of the Hamiltonian) if and only if all eigenvalues of $\mathbf{A}$ lie on the imaginary axis and are definite. The only route for the system to become unstable is through the overlap between two eigenvalues with opposite Krein signatures on the imaginary axis, known as the Krein collision \cite{Yakubovich1975,Kirillov2013,Kirillov2014,Ruili_Zhang_2016,Ruili_Zhang_2020}. The onset of shear instabilities is then linked to the Krein collision between a positive-action mode and a negative-action mode. 
To see this, consider the eigensystem of $\mathbf{A}$, $\mathbf{A}\bm{u}=\lambda\bm{u}$, and the eigensystem of $\mathbf{H}$, $\mathbf{H}\bm{u}=E\bm{u}$, satisfying $\lambda=-iE$. From $\mathbf{H}=\mathbf{G}^{-1}\mathbf{S}$ and $H(\bm{u})=\bm{u}^{\dag}\mathbf{S}\bm{u}$, we obtain
\begin{equation}
    [\bm{u},\bm{u}]=\frac{H(\bm{u})}{E}.
    \label{action_u}
\end{equation}

The Krein signature $\kappa (\lambda)$ of a purely imaginary eigenvalue $\lambda$ of a G-Hamiltonian matrix $\mathbf{A}$ is defined as
\begin{equation}
    \kappa (\lambda) =\text{sign}[\bm{u},\bm{u}].
\end{equation}
Eq.~(\ref{action_u}) shows that the physical interpretation of the Krein signature is the sign of the action of the corresponding eigenmode. In our system, the eigenvalues of $\mathbf{H}$ are 
\begin{equation}
    E_{1,2}=\pm \frac{\sqrt{\Delta \omega^2 -4 \sigma^2}}{2} 
\end{equation}
and the eigenvectors
\begin{equation}
    \bm{u}_{1,2}=\begin{pmatrix}
        \displaystyle -\frac{\Delta \omega}{2\sigma} \mp \frac{\sqrt{\Delta \omega^2 -4 \sigma^2}}{2\sigma} \vspace{2mm} \\  1
    \end{pmatrix}.
\end{equation}
Denote the discriminant of the characteristic polynomial of $\mathbf{H}$ as $D=\Delta \omega^2 -4 \sigma^2=(2k-1)^2-e^{-4k}$. For $k>k_c$, $D>0$ and both eigenvalues of $\mathbf{H}$ are real, suggesting that the system has unbroken $\mathcal{PT}$-symmetry; for $0<k<k_c$, $D<0$ and $\mathbf{H}$ has a pair of complex conjugate eigenvalues, suggesting the spontaneously breaking of $\mathcal{PT}$-symmetry of the system, as shown in Fig.~\ref{fig: fixed_point_two_wave}. The boundary of this change is marked by the spectral singularity at $k=k_c$, denoted as the exceptional point. At the exceptional point, $D=0$ and both real and imaginary parts of $E_{1,2}$ are identical. Therefore, the exceptional point is a saddle point in (Re($E$),Im($E$),$k$) space and the onset of instability is via bifurcation through the exceptional point, as shown in Fig.~\ref{fig: EP_two_wave}.
\begin{figure}[ht!]
\includegraphics[scale=0.6]{fig_braid_two_new.eps}
\caption{\label{fig: EP_two_wave} The eigenvalues $E_{1}(k)$ (red) and $E_{2}(k)$ (blue) in (Re($E$),Im($E$),$k$) space. The saddle-node exceptional point locates at $k=k_c$.}
\end{figure}

The indefinite inner products are
\begin{equation*}
    [\bm{u}_{1,2}, \bm{u}_{1,2}]= \begin{cases}
     \displaystyle \frac{-1+e^{4k}(1-2k\pm \sqrt{D})^2}{2k}, & k>k_c
    \\ 0, & 0<k<k_c
    \end{cases}
\end{equation*} 
so that when $k>k_c$, the eigenvalues $\lambda_{1,2}=-iE_{1,2}$ lie on the imaginary axis and the system is spectrally stable, the Krein signatures are $\kappa(\lambda_1)=\text{sign}[\bm{u}_1,\bm{u}_1]=-1$ and $\kappa(\lambda_2)=\text{sign}[\bm{u}_2,\bm{u}_2]=1$, corresponding to a negative-action mode and a positive-action mode, as shown in Fig.~\ref{fig: wave_action}. When $k$ crosses through $k_c$, the two eigenmodes collide and the total wave action vanishes, with $A(\bm{q})=A_1+A_2=0$.
\begin{figure}[ht!]
\includegraphics[scale=0.56]{fig_wave_action_plot.eps}
\caption{\label{fig: wave_action} The indefinite inner products (actions) of eigenmodes, $[\bm{u}_1,\bm{u}_1]$ and $[\bm{u}_2,\bm{u}_2]$. The Krein collision requires a negative-action eigenmode and a positive-action eigenmode to collide at $k=k_c$.}
\end{figure}
\begin{figure*}[ht!]
    \centering
    \includegraphics[scale=1]{graphic_blue_cut.eps}
    \caption{Graphical interpretation of the Krein signature, with two eigenvalue branches of the pencil $\mathcal{L}$ at (a) $k=0.4$, (b) $k=k_c\simeq 0.6392$ and (c) $k=1$. The zeros of the eigenvalue branches correspond to purely imaginary eigenvalues of $\mathbf{A}$.}
    \label{fig: graphic}
\end{figure*}

The Krein signatures can also be obtained by a graphical method \cite{SIAM_2014}. Consider a Hermitian linear pencil $\mathcal{L}=\mathbf{S}-E\mathbf{G}$, and solve the eigenvalue problem
\begin{equation}
    \mathcal{L}(E)\bm{u}(E)=(\mathbf{S}-E\mathbf{G})\bm{u}(E)=\nu(E)\bm{u}(E)
    \label{linear_pencil}
\end{equation}
parameterized by $E \in \mathbb{R}$, where $\nu=\nu(E)$ is called an eigenvalue branch of $\mathcal{L}$. The intersection points of $\nu(E)$ with $\nu=0$ axis then correspond to real eigenvalues of $\mathbf{H}$. 
Differentiate Eq. (\ref{linear_pencil}) at $E=E_0$ and $\nu(E_0)=0$, we obtain
\begin{equation}
    (\mathbf{S}-E_0\mathbf{G})\bm{u}'(E_0)=\nu'(E_0)\bm{u}(E_0)+\mathbf{G} \bm{u}(E_0).
\end{equation}
Taking the inner product with $\bm{u}(E_0)$ then gives
\begin{equation}
    \nu'(E_0)(\bm{u},\bm{u})=-(\mathbf{G}\bm{u},\bm{u}),
\end{equation}
so that
\begin{equation}
    \kappa(\lambda_0)=-\text{sign}\big[ \nu'(E_0)\big].
\end{equation}
Therefore, we can determine the Krein signatures of the imaginary spectrum of $\mathbf{A}$ simply by looking at the sign of the slope of the eigenvalue branch $\nu=\nu(E)$ at the intersection points. In our system, we plot
\begin{equation}
    \nu_{1,2}(E)=\frac{\Delta \omega}{4k}\pm \frac{\sqrt{E^2+\sigma^2}}{2k}
\end{equation}
for specific choices of the control parameter $k$ in Fig. \ref{fig: graphic}. When $k>k_c$, there are two intersection points with opposite signs of the slope, corresponding to two imaginary eigenvalues of $\mathbf{A}$ with opposite Krein signatures, and the system is spectrally stable. When $k\rightarrow k_c$, the pair of adjacent intersection points collide at $E=0$, leading to the Krein collision. When $k<k_c$, the eigenvalue branches $\nu=\nu_{1,2}(E)$ have no zeros, implying spectral instability of the system.
\begin{figure}[ht!]
    \centering
    \includegraphics[scale=0.4]{dynamics_new_cut.eps}
    \caption{The amplitude and phase dynamics of two coupled vorticity waves, calculated from Eq. (\ref{two_vorticity_wave_system_a}) and (\ref{two_vorticity_wave_system_b}). The initial perturbations are random with amplitude $\sim 10^{-5}$. The control parameter of the system is (a) $k=0.4$, (b) $k=0.639\sim k_c^-$, (c) $k=0.641\sim k_c^+$, and (d) $k=1$. }
    \label{fig: dynamics}
\end{figure}

In general, we have shown that the dynamics of vorticity wave interaction are directly linked to the spectral stability analysis and the theory of (graphic) Krein signatures. The key parameter that leads the system to Krein collision and spontaneously $\mathcal{PT}$-symmetry breaking is the ratio $|\frac{\Delta \omega}{2\sigma}|$, which measures the competition between vorticity wave frequency detuning and coupling strength. The parameter space of spectral instability $0<k<k_c$ corresponds to phase-locking dynamics and exponential growth of initial perturbations, as shown in Fig.~\ref{fig: dynamics}(a) and \ref{fig: dynamics}(b), when $|\frac{\Delta \omega}{2\sigma}|<1$ and the coupling dominates over phase detuning. The parameter space of a stable spectrum $k>k_c$ corresponds to phase-slip dynamics and transient growth of initial perturbations, as shown in Fig.~\ref{fig: dynamics}(c) and \ref{fig: dynamics}(d), when $|\frac{\Delta \omega}{2\sigma}|>1$ and the coupling strength is weak compared with the phase detuning. 
% Note that the vorticity waves decouple and the transient growth vanishes when $k\gg k_c$, as the coupling strength decays exponentially with increasing distance between the waves.

\subsection{Critical behavior in the vicinity of the exceptional point}

One of the most striking features of non-Hermitian systems is the existence of exceptional points -- branch singularities on complex eigenvalue manifolds where both the real and imaginary parts of eigenvalues are identical \cite{Ding_2022}. Apart from eigenvalue properties, the behaviors of eigenvectors of a non-Hermitian Hamiltonian are also nontrivial. Denote the left and right eigenvectors of $\mathbf{H}$ as $\langle u^L|$ and $|u^R\rangle$, and the right eigenvectors of $\mathbf{H}^\dag$ as $\bm{v}$, \textit{i.e.} $\mathbf{H}^\dag \bm{v} =E^*\bm{v}$, then we obtain $|u^R\rangle =\bm{u}$ and $\langle u^L|=\bm{v}^\dag$. For the two-vorticity wave system,
\begin{equation*}
    \langle u_{1,2}^L|=
     \Big(
        \displaystyle \frac{\Delta \omega}{2\sigma} \pm \frac{\sqrt{\Delta \omega^2 -4 \sigma^2}}{2\sigma} \hspace{2mm} 1
    \Big),
\end{equation*}
% \begin{equation*}
%     \langle u_{1,2}^L|=\begin{cases}
%      \Big(
%         \displaystyle \frac{\Delta \omega}{2\sigma} \pm \frac{\sqrt{\Delta \omega^2 -4 \sigma^2}}{2\sigma} \hspace{2mm} 1
%     \Big), & k>k_c
%     \vspace{4mm} \\   \Big(
%          \displaystyle \frac{\Delta \omega}{2\sigma} \pm \frac{i\sqrt{4 \sigma^2-\Delta \omega^2}}{2\sigma} \hspace{2mm} 1
%     \Big), & 0<k<k_c
%     \end{cases}
% \end{equation*}
and the eigenvectors of the non-Hermitian system are biorthonormal, with $\langle u_{1}^L|u_{2}^R\rangle=\langle u_{2}^L|u_{1}^R\rangle=0$. A quantitative measure for the biothogonality of the eigenvectors is the phase rigidity \cite{Rotter_2009,Rotter_2017_PRA,Ding_2022}, defined as
\begin{equation}
    r=\frac{\langle u^L|u^R\rangle}{\langle u^R|u^R\rangle}.
\end{equation}

In Hermitian systems, the right eigenvectors are orthogonal and the phase rigidity $r=1$. In non-Hermitian systems, the right eigenvectors of different states are skewed instead of orthogonal, and the phase rigidity is parameter dependent, with $|r|<1$. Approaching the exceptional point, the two eigenvectors become identical and $r\rightarrow 0$. For the two-vorticity wave system, 
\begin{equation}
    |r_{1,2}|=\begin{cases}
    \displaystyle \sqrt{1-\frac{4\sigma^2}{\Delta \omega^2}}, & k>k_c \vspace{4mm} \\
       \displaystyle \sqrt{1-\frac{\Delta \omega^2}{4\sigma^2}}, &0<k<k_c
    \end{cases}
\end{equation}
as plotted in Fig.~\ref{fig: phase_rigidity}. Expand the phase rigidity near the exceptional point, and we obtain $|r_{1,2}|\propto |k-k_c|^{1/2}$. In the vicinity of the exceptional point, the phase rigidity quantifies the splitting of eigenvectors \cite{Ding_2022}, which follows a square-root dependence similar to the splitting of eigenvalues, $\delta E=|E_1-E_2|\propto |k-k_c|^{1/2}$.  
\begin{figure}[ht!]
\includegraphics[scale=0.56]{fig_phase_rigidity.eps}
\caption{\label{fig: phase_rigidity} Plot of the phase rigidity $|r|$ as function of $k$. At the exceptional point, $|r|=0$ and the eigenvectors are identical. In the vicinity of the exceptional point, $|r|\propto |k-k_c|^{1/2}$. When $k\gg k_c$, the coupling is weak and $|r|\rightarrow 1$. Note that at $k=0.5$, the frequency of two vorticity waves are naturally equal regardless of coupling and $|r|\simeq 1$.}
\end{figure}

The critical exponent $s=1/2$ is also associated with critical behavior of the dynamical system near the exceptional point. Note that when the system crosses through the exceptional point from $k=k_c^+$ to $k=k_c^-$, a transition of phase dynamics occurs from a phase-slip state (Fig.~\ref{fig: dynamics}(c)) to a phase-locking state (Fig.~\ref{fig: dynamics}(b)). The phase-slip state corresponds to transient growth of initial perturbations \cite{Heifetz_2005,guha_2014,guha_2017}, which cannot be captured by normal mode analysis. The phase-slip period $T$ is then the time required for $\theta_{12}$ to jump by $2\pi$. For most of the time in the period, the dynamical phase $\theta_{12}$ is fixed around multiples of $2\pi$, predicted by the neutral fixed point $\theta_{12}=2n\pi$ when $k>k_c$. More precisely, $\theta_{12}$ is around $2n\pi^-$ in the first half of the period, which corresponds to transient growth; and around $2n\pi^+$ in the second half of the period, which corresponds to transient damping. A phase-slip period then ends with a short duration of phase jump about $2\pi$, so that the initial perturbations endure periodic oscillations of non-modal growth and damping. When $k\gg k_c$ (Fig.~\ref{fig: dynamics}(d)), the phase-slip period $T\rightarrow 2\pi/\Delta \omega$ and $\theta_{1}, \theta_{2} \propto t$, so that normal modes are recovered.

To estimate the phase-slip period $T$ around the exceptional point, we plug $R_{12}\simeq 1$ into Eq. (\ref{dynamical_theta_12}) and obtain the Adler equation \cite{Strogatz_2015}
\begin{equation}
    \dot{\theta}_{12}=-\Delta \omega -2\sigma \cos \theta_{12},
\end{equation}
then the phase-slip period 
\begin{equation}
    T=\int_0^{2\pi}\frac{d\theta_{12}}{-\Delta \omega -2 \sigma \cos \theta_{12}}=\frac{2\pi}{\sqrt{\Delta \omega^2-4\sigma^2}}, 
\end{equation}
And the phase slip frequency $\Omega=\sqrt{\Delta \omega^2-4\sigma^2}$.

Therefore, the phase-slip frequency satisfies $\Omega \propto |k-k_c|^{1/2}$ around the exceptional point, as shown in Fig.~\ref{fig: phase_slip_frequency}, and shares same critical exponent $s=1/2$ with the phase rigidity.
\begin{figure}[ht!]
\includegraphics[scale=0.56]{fig_phase_slip_frequency.eps}
\caption{\label{fig: phase_slip_frequency} The critical exponent of phase slip frequency $\Omega$ near exceptional point is $s=1/2$. The red dots are numerically calculated from Eq. (\ref{two_vorticity_wave_system_a}) and (\ref{two_vorticity_wave_system_b}), and the solid blue line is $\propto|k-k_c|^{1/2}$.}
\end{figure}

Accordingly, the transient growth of perturbations around the exceptional point
\begin{equation}
    \frac{Q(t)}{Q(0)}=\exp \left[ -\sigma \int_0^t \sin \theta_{12}(t) dt\right]
\end{equation}
can be estimated for the first half-period as
\begin{equation}
    \frac{Q(\frac{T}{2})}{Q(0)}=\exp \left[ -\sigma \int_{\cos \theta_0}^1 \frac{d \cos \theta_{12}}{\Delta \omega+2\sigma \cos \theta_{12}} \right] = \sqrt{\frac{\mu -\cos \theta_0}{\mu-1 }},
\end{equation}
where $\theta_0=\theta_{12}(0)$ and $\displaystyle \mu=-\frac{\Delta \omega}{2\sigma}$. The optimal amplification factor $G$ for transient growth \cite{Farrell_96,Heifetz_2005,smyth_carpenter_2019} is then obtained when $\cos \theta_0=-1$, as
\begin{equation}
    G=\sqrt{\frac{\mu+1}{\mu-1}}.
\end{equation}


\section{Interaction of three vorticity waves}

\subsection{Dynamical system analysis}

The results in Sec. \ref{sec: two waves} can be readily extended to the interaction of multiple vorticity waves, with multiple exceptional points and richer transient dynamics. To show this, consider a saw-tooth jet shear layer shown in Fig. 1(b). The interaction of three vorticity waves can be described by a complex nonlinear dynamical system \cite{guha_2017}, 
\begin{equation}
    \begin{split}
        \dot{q}_1&=-i\omega_1 q_1-i\sigma_1 q_2-i\sigma_2 q_3,
        \\
        \dot{q}_2&=-i \omega_2 q_2+i \sigma_1 q_1 +i \sigma_1 q_3,
        \\
        \dot{q}_3&=-i \omega_3 q_3-i \sigma_1 q_2-i \sigma_2 q_1,
        \label{complex_three_vorticity_wave_system}
    \end{split}  
\end{equation}
where $\omega_1=-k+1$, $\omega_2=k-1$, $\omega_3=-k+1$, and the coupling coefficients $\sigma_1=e^{-2k}$, $\sigma_2=e^{-4k}$. 
In matrix form, we have
\begin{equation}
    \dot{\bm{q}}=\mathbf{A}\bm{q}, \hspace{8mm} \bm{q}=\begin{pmatrix} q_1&q_2&q_3\end{pmatrix}^T,
\end{equation}
where 
\begin{equation}
    \mathbf{A}=-i\mathbf{H}=-i\begin{pmatrix}
    \omega_1 & \sigma_1 & \sigma_2 \\ -\sigma_1 &  \omega_2 &-\sigma_1 \\ \sigma_2 & \sigma_1 & \omega_3
    \end{pmatrix}
\end{equation}
is a G-Hamiltonian matrix that has the form of
$\mathbf{A}=-i \mathbf{G}^{-1}\mathbf{S}$. Denote the frequency mismatch $\Delta \omega = \omega_1-\omega_2 =\omega_3-\omega_2= 2-2k$, then the non-singular Hermitian matrix $\mathbf{G}$ and the Hermitian matrix $\mathbf{S}$ reads
\begin{equation}
    \mathbf{G}= \frac{1}{4k} 
    \begin{pmatrix}  1 & 0 & 0\\ 0 & -1 & 0 \\ 0 & 0 & 1 \end{pmatrix},  \hspace{4mm}
    \mathbf{S}=\frac{1}{4k} \begin{pmatrix}
        \frac{\Delta \omega}{2} & \sigma_1 & \sigma_2 \\ \sigma_1 & \frac{\Delta \omega}{2} & \sigma_1 \\ \sigma_2 & \sigma_1 & \frac{\Delta \omega}{2}
    \end{pmatrix}.
\end{equation}

The corresponding Hamiltonian is 
\begin{gather}
    H(\bm{q})=\bm{q}^{\dag}\mathbf{S}\bm{q}
    =\omega_1 \frac{Q_1^2}{4k}-\omega_2 \frac{Q_2^2}{4k}+\omega_3 \frac{Q_3^2}{4k}\\
    +\frac{\sigma_1}{2k}Q_1Q_2\cos \theta_{12}+\frac{\sigma_1}{2k}Q_2Q_3\cos \theta_{23}+\frac{\sigma_2}{2k}Q_3Q_1\cos \theta_{31},
\end{gather}
where $\bm{q}^{\dag}$ is the conjugate transpose of $\bm{q}$. The canonical equation still has the form of Eq.~(\ref{complex_canonical_equation}), and the total wave action 
\begin{equation}
    A(\bm{q})=[\bm{q},\bm{q}]=\frac{Q_1^2}{4k}-\frac{Q_2^2}{4k}+\frac{Q_3^2}{4k}=A_1+A_2+A_3.
\end{equation}

Let amplitude ratios $R_{ij}=Q_i/Q_j$, and rewrite the dynamical system (see Eqs. (3.7)--(3.10) in Ref.~\cite{guha_2017}) as 
\begin{widetext}
\begin{subequations}
    \begin{gather}
        \dot{R}_{21}=(1-R_{21}^2)\sin \theta_{21}e^{-2k}+\frac{R_{21}}{R_{23}}\sin \theta_{23} e^{-2k} -\frac{R_{21}^2}{R_{23}}\sin \theta_{31} e^{-4k}, \\
        \dot{R}_{23}=(1-R_{23}^2)\sin \theta_{23}e^{-2k}+\frac{R_{23}}{R_{21}}\sin \theta_{21} e^{-2k} +\frac{R_{23}^2}{R_{21}}\sin \theta_{31} e^{-4k}, \\
        \dot{\theta}_{21}=-2k+2+\big( R_{21}+\frac{1}{R_{21}}\big)\cos \theta_{21} e^{-2k} 
        +\frac{1}{R_{23}}\cos \theta_{23} e^{-2k} + \frac{R_{21}}{R_{23}}\cos \theta_{31} e^{-4k}, \\
        \dot{\theta}_{23}=-2k+2+\big( R_{23}+\frac{1}{R_{23}}\big)\cos \theta_{23} e^{-2k} 
        +\frac{1}{R_{21}}\cos \theta_{21} e^{-2k} + \frac{R_{23}}{R_{21}}\cos \theta_{31} e^{-4k}. 
    \end{gather}
\end{subequations}
\end{widetext}

Solving $\dot{R}_{21}=0$ and $\dot{R}_{23}=0$ then gives either 
\begin{equation}
    \text{Case 1:} \hspace{4mm} \frac{1}{R_{21}^2}+\frac{1}{R_{23}^2}=1
\end{equation}
or
\begin{equation}
    \text{Case 2:} \hspace{4mm}  \frac{\sin \theta_{21}}{R_{21}}+\frac{\sin \theta_{23}}{R_{23}}=0.
\end{equation}
For Case 1, $\dot{\theta}_{21}=0$ and $\dot{\theta}_{23}=0$ give the relation
\begin{equation}
    \cos \theta_{31}=\frac{R_{23}R_{21}}{2}\geq 1,
\end{equation}
so that $\theta_{31}=0$ and $R_{21}=R_{23}=\sqrt{2}$,
\begin{equation}
\cos \theta_{21}= \cos \theta_{23} =-\frac{\Delta \omega+\sigma_2}{2\sqrt{2} \sigma_1}=\frac{(2k-2)e^{2k}-e^{-2k}}{2\sqrt{2}},
\end{equation}
which requires $k_{c1}<k<k_{c2}$, where $k_{c1}\simeq 0.6545$ and $k_{c2} \simeq 1.1475$. The fixed point is stable and the phase-locking of three vorticity waves trigger shear instabilities when $0<\theta_{21}=\theta_{23}<\pi$. When $-\pi<\theta_{21}=\theta_{23}<0$, the fixed point is unstable, corresponding to decay of initial perturbations. 

For Case 2, we obtain $\sin \theta_{21}=\sin \theta_{23}=0$, so that the fixed points are neutrally stable when $k<k_{c1}$ or $k>k_{c2}$, as shown in the bifurcation diagram of fixed points in Fig. \ref{fig: fixed_point_three_wave}. (See also Fig. 2 in Ref. \cite{guha_2017}.) Note that a singular case is ignored when $(2k-1)+e^{-4k}=0$ and $k\simeq 0.3984$.   
\begin{figure}[ht!]
    \centering
    \includegraphics[scale=0.6]{fig_3.eps}
    \caption{ Parameter space of the three vorticity wave system and bifurcation of fixed points with the control parameter $k$. The black vertical dashed lines are the critical $k=k_{c1}\simeq 0.6545$ and $k=k_{c2}\simeq 1.1475$. When $k \leq k_{c1}$ or $k \geq k_{c2}$, the system has two neutrally stable fixed points, located at $\theta_{21}=\theta_{23}=\pm \pi$ or $\theta_{21}=\theta_{23}=0$, with unbroken $\mathcal{PT}$-symmetry. When $k_{c1}<k<k_{c2}$, the system has a pair of stable (solid red line, $0<\theta_{21}=\theta_{23}<\pi$) and unstable (dashed red line, $-\pi<\theta_{21}=\theta_{23}<0$) fixed points, and $\mathcal{PT}$-symmetry is spontaneously broken. The three vorticity waves become phase-locked and lead to shear instabilities.}
    \label{fig: fixed_point_three_wave}
\end{figure}

\subsection{Exceptional points and $\mathcal{PT}$-symmetry breaking}

The Hamiltonian $\mathbf{H}$ of the three vorticity wave system is naturally $\mathcal{PT}$-symmetric, with parity operator $\mathcal{P}=I_{3\times3}$ and time reversal operator $\mathcal{T}=-I_{3\times 3}\mathcal{K}$, as $\mathbf{A}$ is a G-Hamiltonian matrix. The eigenvalues of $\mathbf{H}$ are
\begin{equation}
    E_{1,2}=\frac{\sigma_2}{2} \pm \frac{\sqrt{(\Delta \omega+\sigma_2)^2-8\sigma_1^2}}{2} 
\end{equation}
and $E_3=\frac{\Delta \omega}{2}-\sigma_2$. The eigenvectors are
\begin{equation}
    \bm{u}_{1,2}=\begin{pmatrix}
        1 & \displaystyle -\frac{\Delta \omega+\sigma_2}{2\sigma_1} \pm \frac{\sqrt{(\Delta \omega+\sigma_2)^2-8\sigma_1^2}}{2\sigma_1} &1
    \end{pmatrix}^T
\end{equation}
and $\bm{u}_3=\begin{pmatrix}
        -1&0&1
    \end{pmatrix}^T.$
The discriminant of the characteristic polynomial of $\mathbf{H}$ is
\begin{equation}
    D=4(\sigma_1^2+\sigma_2^2-\Delta \omega \sigma_2)^2\big[(\Delta \omega+\sigma_2)^2-8\sigma_1^2\big].
\end{equation}

Therefore, when $0<k<k_{c1}$ or $k>k_{c2}$, $D \geq 0$ and the eigenvalues $E_{1,2,3}$ are real, the system has unbroken $\mathcal{PT}$-symmetry; when $k_{c1}<k<k_{c2}$, $D<0$ and $\mathbf{H}$ has a pair of pure imaginary eigenvalues $E_{1,2}$, implying the spontaneously breaking of $\mathcal{PT}$-symmetry. The boundaries of this change is marked by two saddle-node exceptional points at $k=k_{c1}$ and $k=k_{c2}$, as shown in Fig.~\ref{fig: EP_three_wave}.
\begin{figure}[ht!]
\includegraphics[scale=0.6]{fig_braid_three.eps}
\caption{\label{fig: EP_three_wave} The eigenvalues $E_{1}(k)$ (blue), $E_{2}(k)$ (red) and $E_{3}(k)$ (green) in (Re($E$),Im($E$),$k$) space. The two saddle-node exceptional points locate at $k=k_{c1}$ and $k=k_{c2}$.}  
\end{figure}

The eigenvalue branches $\nu(E)$ of the Hermitian pencil $\mathcal{L}=\mathbf{S}-E\mathbf{G}$ are then
% \begin{equation}
%     \nu_{1,2}(E)=\frac{\Delta \omega +\sigma_2}{8k} \pm \frac{\sqrt{(2E-\sigma_2)^2+8\sigma_1^2}}{8k}
% \end{equation}
% and $\displaystyle \nu_3(E)=\frac{\Delta \omega-2\sigma_2-2E}{8k}$. 
\begin{figure*}[ht!]
    \centering
    \includegraphics[scale=0.5]{fig_three_graphic_new_cut.eps}
    \caption{Graphical interpretation of the Krein signature, with three eigenvalue branches of the pencil $\mathcal{L}$ at (a) $k=0.36$, (b) $k=k_{c1}\simeq 0.6545$, (c) $k=1$, (d) $k=k_{c2}\simeq 1.1475$ and (e) $k=1.2$. The zeros of the eigenvalue branches correspond to purely imaginary eigenvalues of $\mathbf{A}$.}
    \label{fig: graphic_three}
\end{figure*}
plotted in Fig.~\ref{fig: graphic_three}. The pair of imaginary eigenvalues of $\mathbf{A}$ with opposite Krein signatures $\lambda_{1,2}=-iE_{1,2}$ collide at $k=k_{c1}^-$ and $k=k_{c2}^+$, and the G-Hamiltonian system becomes spectrally unstable when $k_{c1}<k<k_{c2}$. The transition from phase-slip to phase-locking dynamics is shown in Fig.~\ref{fig: dynamics_three}, with the phase-slip frequency
\begin{figure}
    \centering
    \includegraphics[scale=0.4]{fig_dynamics_three.eps}
    \caption{The amplitude and phase dynamics of three coupled vorticity waves, with random initial perturbations $\sim 10^{-5}$. The control parameter of the system is (a) $k=0.6$, (b) $k=0.66$, (c) $k=1.14$, and (d) $k=1.15$. }
    \label{fig: dynamics_three}
\end{figure}
\begin{equation}
    \Omega=\sqrt{(\Delta \omega +\sigma_2)^2-8\sigma_1^2}.
\end{equation}
The critical exponent is then $s=1/2$ as the phase-slip frequency $\Omega \propto |k-k_c|^{1/2}$ near the exceptional points.

The phase rigidity of system eigenvectors is 
\begin{equation}
    |r_{1,2}|=\begin{cases}
    \displaystyle \sqrt{1-\frac{8\sigma_1^2}{(\Delta \omega +\sigma_2)^2}}, & k<k_{c1} \text{ or } k>k_{c2} \vspace{1mm}\\
       \displaystyle \sqrt{1-\frac{(\Delta \omega +\sigma_2)^2}{8\sigma_1^2}}, & k_{c1}<k<k_{c2} 
    \end{cases}
\end{equation}
as plotted in Fig.~\ref{fig: phase_rigidity_three}, with $|r|\propto |k-k_c|^{1/2}$ near the exceptional points. 
\begin{figure}[ht!]
\includegraphics[scale=0.6]{fig_phase_rigidity_three_wave.eps}
\caption{\label{fig: phase_rigidity_three} Plot of the phase rigidity $|r|$ as function of $k$ in three vorticity wave system.}
\end{figure}
Therefore, the square-root dependence is a universal characteristic of the exceptional points in vorticity wave interaction systems.

\section{Summary and discussion}

In this work, we have established a link between the vorticity wave coupling interpretation of shear flow instabilities and the breaking of $\mathcal{PT}$-symmetry at exceptional points in non-Hermitian systems. We have shown that the minimal dynamical system that describes coupling of vorticity waves is a non-Hermitian system with saddle-node exceptional points. The bifurcation of fixed points in the dynamical system is equivalent to the Krein collision between eigenmodes with opposite Krein signatures, both through the exceptional point in $E-k$ space. The key parameter that leads the eigenmodes to collide with each other and breaks the $\mathcal{PT}$-symmetry is the ratio between frequency detuning and coupling strength of the vorticity waves.

A central result of this work is that the transition of phase dynamics from a phase-slip state to a phase-locking state exhibit in the vicinity of the exceptional point. We have shown that this transition of phase dynamics corresponds to the spontaneous $\mathcal{PT}$-symmetry breaking and onset of shear instabilities. The phase-slip dynamics then lead to non-modal, transient growth of perturbations near the exceptional point. The phase-slip frequency shares the same critical exponent $1/2$ with the phase rigidity of system eigenvectors, indicating a square-root dependence as characteristic of the exceptional point. The results have been readily extended to the dynamical system of multiple coupled vorticity waves, where multiple exceptional points provide multiple boundaries for $\mathcal{PT}$-symmetry breaking and onset of shear instabilities.

In general, we have shown that the framework of vorticity wave interaction provides a natural and intuitive testbed for rich non-Hermitian physics. Therefore, we anticipate further applications of this framework not limited to instabilities in shear flows, but also in analysis of instabilities in other continuous media, such as drift wave instabilities \cite{Qin_PRE_2021, Mao_2022} and magnetohydrodynamic instabilities \cite{heifetz_2015,YZhang_2020} in plasmas. 

\begin{acknowledgments}
We acknowledge useful discussions with J. Q. Li, J. Wang, Z. J. Mao and Y. Zhang. This work was supported by the National Natural Science Foundation of China with grant Nos. U1967206, 12275071 and 12075013.
\end{acknowledgments}

% \clearpage

\bibliography{apssamp}% Produces the bibliography via BibTeX.

\end{document}
%
% ****** End of file apssamp.tex ******
