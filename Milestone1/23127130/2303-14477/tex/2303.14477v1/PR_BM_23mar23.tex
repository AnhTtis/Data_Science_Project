\documentclass[12pt, leqno]{amsart}

\usepackage[T1]{fontenc}
%\usepackage{lmodern}
\usepackage{slantsc}
\usepackage{bold-extra}

\usepackage{amssymb}
\usepackage{enumitem}
\usepackage{geometry}
% \geometry{
% a4paper,
% total={170mm,257mm},
% left=20mm,
% top=20mm,
% }
%\geometry{bottom=3.5cm}
 
\usepackage{mathtools}
\usepackage{mathrsfs}
\usepackage{xcolor}
\usepackage[colorlinks=true, linkcolor=black, citecolor=black, urlcolor=black]{hyperref}
\usepackage{cleveref}
\usepackage{bookmark}

\usepackage{graphicx, caption, subfig}
%\renewcommand{\thefigure}{\arabic{part}.\arabic{figure}}
\captionsetup[figure]{margin=25pt, font=footnotesize, labelfont=sc, %labelsep=endash
}
\captionsetup[subfigure]{labelformat=simple, labelsep=space, font=scriptsize, labelfont=normalfont, margin=0pt}

\usepackage[section]{placeins}
\usepackage[export]{adjustbox}

%\usepackage[symbol]{footmisc}
%\renewcommand{\thefootnote}{\fnsymbol{footnote}}


\newtheorem{thm}{Theorem}[section]
\newtheorem{prop}[thm]{Proposition}
\newtheorem{lem}[thm]{Lemma}
\newtheorem{cor}[thm]{Corollary}
\crefname{thm}{Theorem}{Theorems}
\crefname{prop}{Proposition}{Propositions}
\crefname{lem}{Lemma}{Lemmas}
\crefname{cor}{Corollary}{Corollaries}

\newtheorem{claim}[thm]{Claim}
\newtheorem{step}{Step}
\crefname{claim}{Claim}{Claims}
\crefname{step}{Step}{Steps}

\theoremstyle{definition}
\newtheorem{definition}[thm]{Definition}
\newtheorem{example}[thm]{Example}
\crefname{definition}{Definition}{Definitions}
\crefname{example}{Example}{Examples}

\theoremstyle{remark}
\newtheorem{remark}[thm]{Remark}
\crefname{remark}{Remark}{Remarks}


\numberwithin{equation}{section}

\renewcommand\qedsymbol{$\blacksquare$}



\newcommand{\pair}[2]{\ensuremath\langle#1,#2\rangle}
\newcommand{\norm}[1]{\ensuremath\lVert#1\rVert}
%\newcommand{\compl}{^\mathsf{c}}
\newcommand{\compl}{^\complement}
\newcommand{\defeq}{\vcentcolon=}
\newcommand{\trinorm}[1]{\ensuremath|\!|\!|#1|\!|\!|}
\newcommand{\leqold}{\leq}
\newcommand{\geqold}{\geq}
%\newcommand{\emptyold}{\emptyset}
\renewcommand{\leq}{\leqslant}
\renewcommand{\geq}{\geqslant}
%\renewcommand{\emptyset}{\varnothing}
\newcommand{\ssubset}{%
%\subset\joinrel\subset
\Subset}
\newcommand{\phii}{\phi}
\renewcommand{\phi}{\varphi}
\newcommand{\epsi}{\epsilon}
\renewcommand{\epsilon}{\varepsilon}

\newcommand{\tildee}[1]{\widetilde{#1}}

\newcommand{\ap}{\textsuperscript}
\newcommand{\ped}{\textsubscript}

\DeclareMathOperator{\LSC}{LSC}
\DeclareMathOperator{\USC}{USC}
\DeclareMathOperator{\Lip}{Lip}
\DeclareMathOperator{\epi}{epi}
\DeclareMathOperator{\conv}{conv}
\DeclareMathOperator{\Diff}{Diff}
\DeclareMathOperator{\intr}{int}
\DeclareMathOperator{\Aff}{Aff}
\DeclareMathOperator*{\osc}{osc}
\DeclareMathOperator*{\aplimsup}{ap\,lim\,sup}
\DeclareMathOperator*{\aplim}{ap\,lim}

\newcommand{\fns}{\footnotesize}

\newcommand{\C}{\call C^+}
\newcommand{\cont}{\mathscr{C}}
\newcommand{\N}{\mathbb{N}}
\newcommand{\Pc}{\mathcal{P}}
\newcommand{\Sc}{\mathcal{S}}
\newcommand{\Z}{\mathbb{Z}}
\newcommand{\de}{\partial}
\newcommand{\di}{\mathrm{d}}
\newcommand{\R}{{\mathbb{R}}}
\newcommand{\I}{I}
\newcommand{\call}[1]{\ensuremath\mathcal{#1}}
\newcommand{\frk}[1]{\ensuremath\mathfrak{#1}}
\newcommand{\scr}[1]{\ensuremath\mathscr{#1}}
%\newcommand{\barr}[1]{\ensuremath\bar{#1}}
\newcommand{\barr}[1]{\ensuremath\overline{#1}}
\newcommand{\dto}{\ensuremath{\searrow}}
\newcommand{\uto}{\ensuremath{\nearrow}}

\newcommand{\Q}{Q}
\newcommand{\RR}{R}

\newcommand\restr[2]{{% we make the whole thing an ordinary symbol
  \left.\kern-\nulldelimiterspace % automatically resize the bar with \right
  #1 % the function
  \vphantom{|} % pretend it's a little taller at normal size
  \right|_{#2} % this is the delimiter
  }}
  

%comandi da Payne
%\setlength\marginparwidth{2cm}
\newcommand{\noticina}[1]{\marginpar{\tiny\emph{#1}}}

%numeri sezioni
\makeatletter
\@addtoreset{section}{part}
\makeatother
\renewcommand\thesection{\arabic{part}.\arabic{section}}
%

%per indentare le subsections nella toc
\makeatletter
\def\@tocline#1#2#3#4#5#6#7{\relax
  \ifnum #1>\c@tocdepth % then omit
  \else
    \par \addpenalty\@secpenalty\addvspace{#2}%
    \begingroup \hyphenpenalty\@M
    \@ifempty{#4}{%
      \@tempdima\csname r@tocindent\number#1\endcsname\relax
    }{%
      \@tempdima#4\relax
    }%
    \parindent\z@ \leftskip#3\relax \advance\leftskip\@tempdima\relax
    \rightskip\@pnumwidth plus4em \parfillskip-\@pnumwidth
    #5\leavevmode\hskip-\@tempdima
      \ifcase #1
       \or\or \hskip 1em \or \hskip 2em \else \hskip 3em \fi%
      #6\nobreak\relax
    \hfill\hbox to\@pnumwidth{\@tocpagenum{#7}}\par% <---- \dotfill -> \hfill
    \nobreak
    \endgroup
  \fi}
\makeatother
%%


\begin{document}

%toglie le note
%\renewcommand{\marginpar}[2][]{}
%%

\date{\today} \linespread{1.1}

\title[Quasi-convex functions in nonlinear potential theories]{A primer on quasi-convex functions in nonlinear potential theories}
%\author{Marco Cirant}
%\address{Dipartimento di Matematica ``T.\ Levi-Civita''\\ Università degli Studi di Padova\\ \newline Via Trieste 63\\ 35121--Padova, Italy}
%\email{cirant@math.unipd.it (Marco Cirant)}
%%\thanks{Cirant partially supported by the Fondazione CaRiPaRo Project ``Nonlinear Partial Differential Equations: Asymptotic Problems and Mean-Field Games'' and the Programme ``FIL-Quota Incentivante'' of University of Parma, co-sponsored by Fondazione Cariparma.}
\author{Kevin R.\ Payne}
\address{Dipartimento di Matematica ``F.\ Enriques''\\ Università degli Studi di Milano\\ \newline Via~C.~Saldini~50\\ 20133--Milano, Italy}
\email{kevin.payne@unimi.it (Kevin R. Payne)} %\thanks{Payne partially supported by the Gruppo Nazionale per l'Analisi Matematica, la Probabilit\`a e le loro Applicazioni (GNAMPA) of the Istituto Nazionale di Alta Matematica (INdAM) and the projects: GNAMPA 2017 ``Viscosity solution methods for fully nonlinear degenerate elliptic equations'', GNAMPA 2018 ``Costanti critiche e problemi asintotici per equazioni completamente non lineari'' e GNAMPA 2019 ``Problemi differenziali per operatori fully nonlinear fortemente degeneri''.}
\author{Davide F.\ Redaelli}
\address{Dipartimento di Matematica ``T.\ Levi-Civita''\\ Università degli Studi di Padova\\ \newline Via Trieste 63\\ 35121--Padova, Italy}
\email{redaelli@math.unipd.it (Davide F.\ Redaelli)}

\keywords{}

\subjclass[2010]{}

\begin{abstract}
We present a self-contained introduction to the fundamental role that {\em quasi-convex functions} play in general (nonlinear second order) potential theories, which concerns the study of {\em generalized subharmonics} associated to a suitable closed subset of the space of $2$-jets. Quasi-convex functions build a bridge between classical and viscosity notions of solutions of the natural Dirichlet problem in any potential theory. Moreover, following a program initiated by Harvey and Lawson in~\cite{hldir09}, a potential theoretic-approach is widely being applied for treating nonlinear partial differential equations (PDEs). This viewpoint revisits the conventional viscosity approach to nonlinear PDEs~\cite{user} under a more geometric prospective inspired by Krylov~\cite{kr} and takes much insight from classical pluripotential theory. The possibility of a symbiotic and productive relationship between general potential theories and nonlinear PDEs relies heavily on the class of quasi-convex functions, which are themselves the protagonists of a particular (and important) pure second order potential theory. 
\end{abstract}

\maketitle

\setcounter{tocdepth}{3}
\tableofcontents

\section*{Introduction}

This work is dedicated to foundational aspects of general (nonlinear) potential theory as developed by Reese Harvey and Blaine Lawson, beginning with a trio of papers, \cite{hlpotcg}, \cite{hlpluri09} and \cite{hldir09}, published in 2009. Our aim is to give a comprehensive presentation of the main tools in quasi-convex analysis which are useful in order to study general potential theories as well as to deal with nonlinear PDEs by a the potential-theoretic approach of Harvey and Lawson, which (in many cases) can recast the operator theory in a flexible and more geometric way though the use of an associated potential theory. We will also propose to the reader an introduction to their theory, highlighting the main ideas behind it and the basic tools which make it a robust method. 

We should point out, at the onset, that Harvey and Lawson's definition of \emph{quasi-convexity} is not the one of convex analysis; they use \emph{quasi-convex} as a synonym of \emph{semiconvex} in the sense of classic viscosity theory (see, e.g., \cite{user}); that is, quasi-convex functions are functions whose Hessian (in the viscosity sense) is bounded from below. Even though the term \emph{semiconvex} widely used in the viscosity literature, we will adopt their terminology, which is consistent with the use of \emph{quasi-plurisubharmonic} in several complex analysis, as noted in~\cite{hpsurvey}.

In order to place the discussion into context, we begin with a brief description of the program initiated by Harvey and Lawson.
It was inspired by the geometric approach of Krylov~\cite{kr}, who replaced the differential operator $u \mapsto F(D^2u)$ with a closed subset $\call F\subset \Sc(n)$ such that $\de \call F \subset \{ A \in \Sc(n):\ F(A) = 0 \}$, thus introducing the notions of \emph{elliptic branch} of the fully nonlinear equation $F(D^2u) = 0$ on $\Omega\ssubset \R^n$ and of \emph{elliptic set}. In this approach Harvey and Lawson formulated a suitable notion of \emph{duality} which reformulates the notion of viscosity solutions in a precise topological framework and developed a \emph{monotonicity--duality} method of studying fully nonlinear second-order equations.

After the pioneering paper of Krylov~\cite{kr} and the first systematic work of Harvey and Lawson~\cite{hldir09} in the direction he suggested, other ambitious papers followed, such as \cite{hldir}, in which Harvey and Lawson study the Dirichlet problem for fully nonlinear second-order equations on a Riemannian manifold, and the book \cite{chlp}, which gathers essentially all the results obtained so far by employing the theory in a constant-coefficient setting. Besides these works, many others should be mentioned and the reader can find some of them in the references; for instance, the works of Harvey and Lawson on potential theory in calibrated geometry~\cite{hlpotcg} and on the special Lagrangian potential equation~\cite{hlpssl}, as well as their paper on the inhomogeneous Dirichlet problem for ``natural'' operators~\cite{hlidir}, and that of M.\ Cirant and the first author~\cite{cpaux} for elliptic branches of equations of the form $F(x, D^2u(x)) = 0$, followed by their work~\cite{cpmain} on comparison principles for fully nonlinear equations independent of the gradient. 


The present work was born with the purpose of unifying and harmonizing the two articles \cite{hlae} and \cite{hlqc} of Harvey and Lawson on quasi-convex functions which were deposited in the arXiv in 2016. These two preprints are full of foundational aspects of nonsmooth analysis, some well known, others less well known and many, many new things. In particular, as will be described in the work, there are many instances of lost opportunities of synergy between general nonlinear potential theories and nonlinear PDEs, with many concepts being discovered independently in the two realms. A major aim of this work is to contribute to the many opportunities for productive interplay between these two realms. Moreover, the two unpublished manuscripts of Harvey and Lawson contain many refinements and efficient use of quasi-convex functions for unifying and simplifying many aspects of the conventional viscosity theory, and as such, merit a comprehensive and unified presentation. The authors of this work are extremely grateful to Harvey and Lawson for their encouragement to attempt this synthesis, which was the basis of the Master's Thesis of the second author, supervised by the first author.


One of the main results in \cite{hlae} is the \emph{Almost Everywhere Theorem} (AE Theorem), which, roughly speaking, states that a locally quasi-convex function which is subharmonic almost everywhere (with respect to Lesgue measure) is in fact subharmonic everywhere in the viscosity sense. We refer to \Cref{def:subh} for a precise definition of \emph{subharmonicity}. By using the AE Theorem, a quasi-convex version of the \emph{Subharmonic Addition Theorem} follows easily. Subharmonic addition theorems concern the validity ot the following implication: given a trio $\call F, \call G$ and $\call H$ of {\em subequation constraint sets} in $\call J^2$ (which determine spaces of {\em subharmonic functions} $\call F(X), \call G(X)$ and $\call H(X)$ on each open subset $X \subset \R^n$), from a purely algebraic statement on the space of $2$-jets 
$$
	\mathcal{F} + \mathcal{G} \subset \mathcal{H} \quad \text{(jet addition)}
$$
one concludes an analytic statement at the potential theory level,
$$
\mathcal{F}(X) + \mathcal{G}(X) \subset \mathcal{H}(X) \quad \text{(subharmonic addition)}.
$$
When combined with the aforementioned {\em monotonicity} and {\em duality}, one obtains a robust and elegant method for proving comparison principles. This will be discussed in Section \ref{sec:comp}.


Another important result is called in \cite{hlae} ``a stronger form of the addition theorem''; in this work it is called the \emph{Theorem on Summands} (see~\Cref{pusc:sum}), in analogy with the well-known \emph{Theorem on Sums} of Crandall, Ishii, and Lions~\cite{user}, which is a powerful tool in order to prove comparison principles in viscosity theory. In fact, the two theorems yield similar useful information on the sum of two quasi-convex functions and on the summands of such a sum. Also, a quasi-convex version of the Theorem on Sums actually follows from the Theorem on Summands.
%%comes from a partial upper semicontinuity property for the second-derivatives of quasi-convex functions, which is proved invoking two nontrivial results on quasi-convex functions, namely the Jensen--S{\l}odkowski Theorem and the celebrated Alexandrov's theorem on second-order differentiability of convex functions. These are precisely the main results in \cite{hlqc}, while 

The article \cite{hlae} also contains a discussion about constant coefficient pure second-order and gradient-free subequations that summarises certain results on subharmonic addition and comparison which can be found in other works of Harvey and Lawson, such as \cite{hldir09} or the research monograph \cite{chlp} with Cirant and the first author. A \emph{strict comparison} theorem for quasi-convex subharmonics is also included which is the nonconstant coefficient counterpart of \cite[Corollary~C.3]{hldir} with an additional quasi-convexity assumption.

We point out that such results in nonlinear potential theory are clever instruments which help one to prove maximum and comparison principles, which are undeniably useful; for instance, in order to prove existence and uniqueness of solutions of Dirichlet problems for general potential theories and for all fully nonlinear PDEs (which are {\em compatible} with the potential theory) via the Perron method. Harvey and Lawson's point of view ``encodes'' a PDE in the potential theoretic language, by substituting the differential operator with a geometric constraint (the \emph{subequation constraint set}), which is then studied with a monotonicity--duality method for proving the comparison principle, which imples uniqueness of solutions to the Dirichlet problem. This will be discussed in \Cref{sas}. Moreover, their approach has the huge advantage that the potential theory also correctly identifies the suitable geometric notion of {\em strict boundary pseudoconvexity} on domains for them to admit the required barrier functions needed for the existence part of Perron's method. We will not discuss existence here, but the reader may wish to consult \cite{hpsurvey} for a recent survey on this aspect.

Both the AE Theorem and the Theorem on Summands follow from two essential theorems for quasi-convex functions in viscosity theory: Alexandrov's theorem on second-order differentiability of convex functions~\cite{alex} and a joint formulation of two distinct results of Jensen~\cite{jensen} and S{\l}odkowski~\cite{slod} which Harvey and Lawson call the Jensen--S{\l}odkowski Theorem. These results represent the ``hard analysis'' tools for dealing with viscosity notions. They are to be found in \cite{hlqc}, along with some elementary yet useful properties of convex functions. At first glance, the formulations of the Jensen--S{\l}odkowski results appear to be quite different from the well-known result of Jensen (see \cite{jensen, slod, user}), but in fact they are all equivalent, which was a major discovery in Harvey--Lawson \cite{hlqc}. It is worth noting that the result of S{\l}odkowski~\cite{slod} was developed in the context of plurisubharmonic theory and predates that of Jensen~\cite{jensen} which was developped for nonlinear PDEs. A lack of recognition of this equivalence represents a missed opportunity in the possible interplay between potential theories and PDEs. \Cref{convediff} offers a systematic study of these pivotal theorems and a presentation of some important related results.

Almost everything one finds in the two original articles by Harvey and Lawson is included here, albeit sometimes arranged in a different manner. We slightly changed some proofs and we also included additional results in order to clarify certain passages. We also added many more results and remarks to enrich the presentation, as well as some thoughts of Harvey which help to highlight the weight the area formula (see for instance \cite[Theorem~3.2.3]{fed:geo}) has in the ecosystem we display.


\part{Quasi-convex apparatus} \label{convediff}

This first part opens with some elementary properties of the first-order subdifferential of convex functions which are going to be useful in the proof of Alexandrov's theorem and for many other results we will present. Then we introduce the Legendre transform and discuss those properties that makes it an essential instrument in our proof of Alexandrov's theorem, which concludes the first section. The proof of the other crucial tool in proving Alexandrov's theorem, namely a version of Sard's theorem for Lipschitz functions, is instead to be found in the \nameref{proofsard}; we have not included it in this first part because it is a bit technical.

After introducing the notion of \emph{upper contact jet} and its interplay with quasi-convex functions in~\Cref{qcfaj}, culminating with the two theorems on upper contact jets (\Cref{ucjt}) and on summands (\Cref{pusc:sum}), we will devote \Cref{sec:js} to our extensive study of Jensen's lemma as stated by Harvey and Lawson~\cite{hlqc} and its classic counterparts one finds in \cite{jensen,user}. We prove that they are in fact equivalent, and establish the strong and deep link with certain classic results of S{\l}odkowski~\cite{slod}. A unified reformulation of them is then presented, called by Harvey and Lawson~\cite{hlqc} \emph{the Jensen--S{\l}odkowski theorem}, and we eventually end up with two nonclassic proofs of Jensen's lemma: one coming from S{\l}odkowski's approach in proving his density estimate~\cite[Theorem~3.2]{slod} revisited in a ``paraboloidal'' key by Harvey and Lawson, another one exploiting a version of Alexandrov's maximum principle noted by Harvey~\cite{har:pc}.

Our interest in giving importance to the theorems of Alexandrov and Jensen--S{\l}odkowski is motivated by the following consideration. For a good viscosity theory, which is based on quadratic upper test functions, one has the problem to guarantee \emph{a priori} a sufficient amount of upper contact points for a semicontinuous function; those two theorems, yielding the AE Theorem (see~\Cref{aet}, in~\Cref{sas}) and the Summand Theorem, ensure that, for quasi-convex functions, \emph{almost all} (Alexandrov) points admit an upper test function, and they are \emph{enough} (AE Theorem) for viscosity-theoretic purposes, thus depicting quasi-convex functions as the ideal candidate to be the link between classical theory (smooth functions) and viscosity theory (semicontinuous functions).

In other words, thanks to the fundamental theorems of Alexandrov and Jensen--S{\l}odkowksi, if there is a way to approximate a semicontinuous function with a quasi-convex function (and there is, as we will discuss in~\Cref{sec:apqc}), then one is able to recover a sufficient amount of pieces of information about differentiability to work inside viscosity theory. However, clearly, this is not enough; one will also need to have a good ``stability'' or a good ``control'' on such an approximation process in order to close the circle and obtain results for the semicontinuous functions one has started with. This is a crucial problem, which we will address only in a few basic situations, in~\Cref{sas}. For further investigations on the possibility of applying such a quasi-convex approximation technique inside the context of potential theory, we invite the reader to see, e.g.,~\cite{cpaux,cpmain,cprdir}.

\section{Differentiability of convex functions}

We shall introduce in this section a fundamental result when one deals with convex functions in second-order PDEs: the celebrated theorem of Alexandrov stating that every locally convex function on an Euclidean space is twice differentiable in the Peano sense almost everywhere with respect to the Lebesgue measure. 

For instance, it is one of the main tools, along with Jensen's lemma, required in the proof of the Theorem on Sums~\cite[Theorem 3.2]{user}, and it will be also widely used throughout all this work. 

Our proof of Alexandrov's theorem follows the main steps of Harvey and Lawson's proof \cite{hlqc}, which are basically the same Crandall, Ishii and Lions propose in~\cite{user}: we use the Legendre transform and a Lipschitz version of Sard's theorem. A different proof exploiting Lebesgue's Differentiation Theorem and mollifications is given in the classic book of Evans and Gariepy~\cite{evansgar}.

For the benefit of the reader we first summarize and prove those properties of the subdifferential of convex functions we shall need in these pages. %Actually, another one on first-order differentiability of convex functions is missing, as we preferred to state it in a more opportune context, after introducing \emph{upper contact jets}, in \Cref{datucp}.


\subsection{First order theory}

Let us begin by recalling some well-known properties of subdifferentials of convex functions; that is, those functions whose epigraph is a convex set. We are aware that the results that we are going to mention actually hold in a more general context, but we believe that a formulation of them which better fits our purposes in this work is more appropriate. Most of the essential results come from~\cite{hlqc}.

\begin{definition} \label{def:subd}
Let $u \colon X \to \R$ be a (not necessarily convex) function defined on a subset $X\subset\R^n$. We define its \emph{subdifferential at $x\in X$} as the set
\begin{equation} \label{defsubd}
\de u(x) \defeq \{ p \in \R^n :\ u(y) \geq u(x) + \pair{p}{y-x}\ \; \forall y \in X \}.
\end{equation}
Each $p \in \de u(x)$ will be called a \emph{subgradient} of $u$ at $x$, and we will say that {\em $u$ is subdifferentiable at $x$} if $\de u(x) \neq \emptyset$.
\end{definition}

\begin{remark} \label{geointerpsubdiff}
Geometrically, this means that the hyperplane which is the graph of the affine function $u(x) + \pair{p}{\cdot - x}$ over $X$ is a supporting hyperplane from below at $x$ for the epigraph of $u$, for all $p \in \de u(x)$. Hence one may observe that $x$ is a minimum point for $u$ if and only if $0 \in \de u(x)$.
\end{remark}

\begin{remark}
We have essentially two ways to interpret the \emph{subdifferential} of $u$: as a map
\[
\de u \colon X \to \scr P(\R^n), \quad x \mapsto \de u(x)
\]
or as a subset of a trivial bundle over $X$ whose fiber over each $x \in X$ is $\de u(x)$; that is,
\[
\de u \defeq \bigsqcup_{x \in X} \de u(x) \subset X \times \R^n.
\]
Note that if one considers in the latter interpretation, the projection $\pi \colon \de u \to X$, mapping $\de u(x) \ni p \mapsto x$, then adopting the former interpretation basically means that we are calling $\pi^{-1} = \de u$.  We will be mainly adopt the former interpretation, for instance in using the notation
\[
\de u(\Omega) = \bigcup_{x \in \Omega} \de u (x) \subset \R^n, \quad \text{$\Omega\subset X$},
\]
yet we will also write
\[
(x,p) \in \de u \quad\iff\quad p \in \de u(x),
\]
thus employing the latter interpretation.
\end{remark}

We now recall the definition of a convex function and various equivalent formulations and elementary properties in order to examine the subdifferentiability of $u$ convex.

\begin{definition} \label{defn:convex}
Let $u \colon X \to \R$ be a function defined on a convex subset $X\subset \R^n$. We say that $u$ is \emph{convex} if its epigraph $\epi(u) \defeq \{ y \in \R : y \geq u(x) \}$ is convex.
%\footnote{The notation $\barr\epi$, instead of $\epi$, is coherent with the choice made in Subsection~\ref{subs:svop} of denoting by $\epi$ the \emph{open} (or \emph{strict}) epigraph.}
\end{definition}

\begin{remark}
It is immediate that one has the standard inequality formulation for convexity; that is, $u$ is convex in the sense of \Cref{defn:convex} if and only if: for each $x_1, x_2 \in X$ one has
\begin{equation}\label{Jensen_2}
u(t x_1 + (1-t)x_2) \leq t u(x_1) + (1-t) u(x_2)\quad  \text{for each} \  t \in [0,1].
 \end{equation}
Of course, this means that the restriction of $u$ to each segmment $[x_1,x_2] \subset X$ is a convex function of one variable. Moreover, the inequality \eqref{Jensen_2} is a version of {\em Jensen's inequalty} for convex functions.
\end{remark}

\begin{lem}[Jensen's inequality; finite form] \label{lem:Jensen} Let $u: X \to \R$ be convex on $X \subset \R^n$ convex. Then for each collection of points $\{x_j\}_{j=1}^N \subset X$ with $N \geq 2$ 
\begin{equation}\label{Jensen_N}
u\!\left( \sum_{j=1}^N t_j x_j \right) \leq  \sum_{j=1}^N t_j u(x_j), \quad  \text{for each}  \ \{t_j\}_{j=1}^N  \subset [0,1] \ \text{with} \ \sum_{j=1}^N t_j = 1.
\end{equation}	
	\end{lem}
The proof of Lemma \ref{lem:Jensen} is standard and typically done by induction on $N \geq 2$, where the case $N = 2$ is \eqref{Jensen_2}.
\begin{remark}
We also briefly remind the reader of the following obvious facts: affine functions are convex; nonnegative weighted sums of convex functions are convex; the (pointwise) supremum of any family of convex functions is convex.
\end{remark}

Convex functions $u$ on $X$ (open and convex) are {\em subdifferentiable} in the sense of \Cref{def:subd}. Moreover, this property characterizes the convexity of $u$.

\begin{prop}[Subdifferentiability of convex funtions]\label{prop:subd_convex}
For any function $u: X \to \R$ defined on an open and convex subset $X\subset \R^n$, one has 
\begin{equation*}
	u \ \text{is convex} \ \ \Leftrightarrow \ \ \partial u(x) \neq \emptyset, \ \forall \, x \in X;
\end{equation*}
that is, $u$ is convex on $X$ if and only if $u$ is subdifferntiable on $X$.
	\end{prop}

\begin{proof}
	If $u$ is convex, then by the Hahn--Banach Theorem for any $x \in X$ there exists a supporting hyperplane for the convex set $C \defeq \barr\epi(u)$ at the point $(x,u(x))$; that is, there exist $(q,r) \in \R^n \times \R \setminus \{(0,0)\}$ such that
	\begin{equation} \label{subdiffnonempty:hb}
	\pair{q}{y-x} + r(z-u(x)) \geq 0 \quad \text{for each $(y,z) \in C$.}
	\end{equation}
	With the choices $y=x$ and $z > u(x)$, the inequality \eqref{subdiffnonempty:hb} implies that $r \geq 0$. By choosing now $z = u(x)$, one has that
	\[
	\pair{q}{y-x} + r(u(y)-u(x)) \geq 0 \quad \forall y \in X,
	\]
	which yields $-r^{-1}q \in \de u(x)$, provided that $r \neq 0$. To show that in fact $r\neq 0$, notice that otherwise by inequality \eqref{subdiffnonempty:hb} one would have $\pair{q}{y-x} \geq 0$ for each $y \in X$, which is impossible since $X$ is open.
	
	Conversely, if $x = \lambda x_1 + (1-\lambda) x_2$ for some $\lambda \in [0,1]$, we have
	\[
	u(x_1) \geq u(x) + (1-\lambda)\pair{p}{x_1-x_2} \quad \text{and} \quad u(x_2) \geq u(x) - \lambda \pair{p}{x_1-x_2}
	\] 
	for every $p\in \de u(x)$, thus yielding $u(x) \leq \lambda u(x_1) + (1-\lambda)u(x_2)$; that is, $u$ is convex.  
	\end{proof}

Convex functions are also locally bounded.

\begin{lem}\label{lem:convex_loc_bdd} Let $u: X \to \R$ be convex on $X$ open and convex. Then 
	\begin{equation}\label{loc_lim1}
	\mbox{$u$ is bounded on every compact subset $K \subset X$,}
	\end{equation}
or equivalently,
\begin{equation}\label{loc_lim2}
\mbox{$u$ is bounded on some neighborhood of every $x \in X$.}
\end{equation}
	\end{lem}

\begin{proof}
	The equivalence of \eqref{loc_lim1} and \eqref{loc_lim2} is standard. Let $K \subset X$ be compact. By choosing any $x \in K$
	 and any $p \in \partial u(x) \neq \emptyset$, one has 
	 \[
	 u(y) \geq u(x) + \pair{p, y-x} := a_x(y), \ \ \text{for every} \ y \in X,
	 \]  
	 where $a_x$ is affine and hence bounded from below on $K$ compact. Hence $u$ is bounded from below on $K$.
	 
	 For the local boundedness from above, let $x \in X$ be arbitrary and consider any compact cube $\mathcal{C}$ centered in $x$ such that $\mathcal{C} \subset X$. We denote by $\{v_j\}_{j=1}^{2^n}$ the collection of vertices of $\mathcal{C}$. Every closed cube is the (closed) convex hull of its vertices and hence for each $y \in \mathcal{C}$ one has
 $$
 	y = \sum_{j=1}^N t_j v_j \quad \text{with} \quad  \sum_{j=1}^N t_j \ \ \text{and each} \ \ t_j \in [0,1].
 $$
By Jensen's inequality \eqref{Jensen_N} we have
$$
	u(y) = u \left( \sum_{j=1}^N t_j v_j  \right) \leq \sum_{j=1}^N t_j  u(v_j) \leq \max_{1 \leq j \leq N} v(x_j) < + \infty,
$$ 
and hence $u$ is bounded from above on every compact cube contained in $X$.
	\end{proof}

Two interesting properties concerning the subifferential of a convex function are given in the following lemma. 

\begin{lem}\label{lem:intpropsubdiff}
For each convex function $u: X \to \R$, one has:
\begin{itemize}
\item[(a)]	the fiber $\de u(x) \subset \R^n$ is closed and convex  for each $x \in X$;
\item[(b)]	for each pair of points $x$ and $y$ in $X$,
\begin{equation} \label{slope}
\pair{p}{y-x} \leq u(y) - u(x) \leq \pair{q}{y-x} \quad \forall p\in \de u(x),\ q\in \de u(y),
\end{equation}
\end{itemize}
\end{lem}

\begin{proof}
For the claims of part (a), it is easy to see that
\[
\de u(x) = \bigcap_{y \in X} \big\{ p \in \R^n: u(x) + \pair{p}{y-x} \leq u(y) \big\} 
\]
is the intersection of closed affine half-spaces and therefore it is closed and convex. This follows from the fact that intersections of convex sets are clearly convex. Notice that the above argument applies to any function $u$; the convexity ensures that $\de u(x)$ is not empty.

For the chain of inequalities of part (b), by \Cref{prop:subd_convex}, the subdifferentials $\partial u(x)$ and $\partial u(y)$ are non-empty. By the definition (\ref{defsubd}), for each fixed $x \in X$ one has
\begin{equation} \label{slope_1}
u(y) \geq u(x) + \pair{p}{y-x}  \quad \forall p\in \de u(x), \ \forall \, y \in X 
\end{equation}
and for each fixed $y \in X$ one has
\begin{equation} \label{slope_x}
u(x) \geq u(y) + \pair{q}{x-y}  \quad \forall q\in \de u(y), \ \forall \, x \in X,
\end{equation}
from which \eqref{slope} easily follows.
\end{proof}

We will see that the chain of inequalities in \eqref{slope} is estremely useful. Notice that the geometric meaning of (\ref{slope}) is that if we restrict our attention to the segment $[x,y]$ with intrinsic real coordinate such that $x\leq y$, then $\barr p \leq s \leq \barr q$, where $\barr p$, $\barr q$ are the slopes of the supporting hyperplane at $x,y$ identified by $p,q$ and $s$ is the slope of the chord connecting $(x, u(x))$ and $(y,u(y))$. This implies the following property.

\begin{prop} \label{subdmon}
The subdifferential $\de u \colon X \to \scr P(\R^n)$ is a monotone operator; that is,
\begin{equation} \label{eq:subdmon}
\pair{q-p}{y-x} \geq 0 \qquad \forall (x,p), (y,q) \in \de u.
\end{equation}
\end{prop}

This generalizes the definition of (nondecreasing) monotonicity for single-valued functions from $\R$ to $\R$, since in that case \eqref{eq:subdmon} is equivalent to $(\de u(y) - \de u(x))\cdot (y-x) \geq 0$, that is $\de u(x) \leq \de u(y)$ whenever $x\leq y$.

\begin{remark} \label{equiv:subdmonconv}
We observe that since \eqref{slope} directly follows from \Cref{def:subd}, \Cref{subdmon} tells that the subdifferential $\de u$ of any function $u \colon X \to \R$ is in fact monotone, in the sense that one knows that \eqref{eq:subdmon} holds whenever $x,y$ are such that $\de u(x) \neq \emptyset \neq \de u(y)$. This provides in general information of no use, since $\de u$ can be empty in ``too many'' (possibly all) points. However, by \Cref{lem:intpropsubdiff} one knows that if $u$ is convex on $X$ open and convex, then the situation is the exact opposite: inequality \eqref{eq:subdmon} actually means something for each pair of points $x,y \in X$, since the domain of $\de u$ is the whole $X$; that is, $\mathrm{dom}\,\de u \defeq \{ x \in X : \de u(x) \neq \emptyset \} = X$. Furthermore, for differentiable functions $u \colon \R \to \R$ one knows that convexity is equivalent to the (nondecreasing) monotonicity of $Du = u'$, which is in fact equivalent to the monotonicity of $\de u$ in the sense of \eqref{eq:subdmon} (cf.~also the upcoming \Cref{deusingv}); so one could wonder whether a characterization of convexity for generic functions exists in term of the monotonicity of their subdifferential. We have essentially just commented on the fact that the answer is no. Indeed all functions have monotone (possibly empty) subdifferential, and also adding an hypothesis like $\mathrm{dom}\,\de u \neq \emptyset$ would be to weak, as long as one does not ask for $\mathrm{dom}\,\de u = X$ (think of $u$ defined by $u \defeq f$ on $X \setminus \{x_0\}$, and $u(x_0) \defeq \alpha > f(x_0)$, for some convex function $f$ on $X$; and note that $\mathrm{dom}\,\de u = X \setminus \{ x_0 \}$, yet $u$ is not convex); but the hypothesis $\mathrm{dom}\,\de u = X$ is too strong because it is in fact itself equivalent to the convexity of $u$. Anyhow, for the sake of completeness, we mention that a characterization of convexity in terms of the monotonicity of ``a subdifferential'' actually exists, and uses the so-called \emph{proximal subdifferential} (see \cite[Theorem 4.1]{csw} and the references therein).
\end{remark}

For future purposes, we prove here a particular case of the equivalence discussed in \Cref{equiv:subdmonconv}.

\begin{prop} \label{conviffmon}
Let $u$ be differentiable on an open convex set $X\subset \R^n$; then
\[
\text{$u$ is convex} \quad \iff \quad \text{$Du$ is monotone}.\footnote{That is, $\pair{Du(x) - Du(y)}{x-y} \geq 0$ for all $x,y \in X$.}
\]
\end{prop}

\begin{proof}
The implication ``$\Longrightarrow$'' immediately follows from \Cref{subdmon} and the upcoming \Cref{deusingv}. Alternatively, one can fix $x,y \in X$ and restrict their attention to the linear segment $[x,y]$; indeed, the convexity of $u$ is equivalent to the convexity of
\[
\tilde u(t) \defeq u(x+t(y-x)), \quad t \in [0,1],
\]
for each fixed pair of points $x,y \in X$. This is in turn equivalent to
\begin{equation} \label{convonline}
\frac{\tilde u(t_2) - \tilde u(t_1)}{t_2-t_1} \leq \frac{\tilde u(t_3) - \tilde u(t_2)}{t_3-t_2} \quad \text{whenever $0 \leq t_1 < t_2 < t_3 \leq 1$}.
\end{equation}
Letting $t_2 \dto t_1 = 0$ one gets
\begin{equation} \label{monder:1}
\pair{Du(x)}{y-x} \leq \frac{u(x+t(y-x)) - u(x)}{t} \quad \forall t \in (0,1],
\end{equation}
while letting $t_2 \uto t_3 = 1$ one gets
\begin{equation} \label{monder:2}
\frac{u(y) - u(x+t(y-x))}{1-t} \leq \pair{Du(y)}{y-x} \quad \forall t \in [0,1).
\end{equation}
Hence combining \eqref{convonline}, \eqref{monder:1} and \eqref{monder:2} yields
\[
\pair{Du(x)}{y-x} \leq \pair{Du(y)}{y-x},
\]
which is the monotonicity of $Du$. For the converse implication, notice that by the Mean Value Theorem, for $j=1,2$ there exists $\xi_j \in [x+t_j(y-x), x+t_{j+1}(y-x)]$ such that $\tilde u(t_{j+1}) - \tilde u(t_j) = \pair{Du(\xi_j)}{(t_{j+1} - t_j)(y-x)}$, and thus condition \eqref{convonline} holds if and only if
\begin{equation} \label{quasimon}
\pair{Du(\xi_1)}{y-x} \leq \pair{Du(\xi_2)}{y-x},
\end{equation}
for some
\[
\xi_1 = x + s_1(y-x), \ \xi_2 = x+ s_2(y-x), \quad 0 < s_1 < s_2 < 1.
\]
Therefore we have
\[
\xi_2 - \xi_1 = (s_2 - s_1)(y-x),
\]
and we can rewrite \eqref{quasimon} as
\[
(s_2-s_1)^{-1} \pair{Du(\xi_2) - Du(\xi_1)}{\xi_2 - \xi_1} \geq 0,
\]
which is true by the monotonicity of $Du$ (since $s_2 - s_1 > 0$). This concludes the proof.
\end{proof}


Additional important consequences of the chain of inequalities \eqref{slope} concern convex functions on compact subsets.

\begin{thm}\label{thm:convex_Lip} Let $u: X \to \R$ be convex on $X \subset \R^n$ open and convex. Then for each $K \subset X$ compact
	\begin{itemize}
\item[(a)] $\de u(K) = \bigcup_{x \in K} \de u (x)$ is compact in $\R^n$;
\item[(b)] $u \in \rm{Lip}(K)$ with
\begin{equation}\label{convex_Lip}
	|u(x) - u(y)| \leq \max_{p \in \partial u(K)} \!|p| \cdot |x - y|, \ \ \forall \, x,y \in K.
\end{equation}
In particular, $u$ is continuous on $X$.
\end{itemize}
\end{thm}

\begin{proof}
First, we prove that $\de u(K)$ is bounded. Since $K$ is compact in $X$ there exists  $\delta > 0$ such that $K^{\delta}:= \{ x \in X: \ d(x,K) \leq \delta \} \subset X$, where
$K^{\delta}$ is compact. By Lemma \ref{lem:convex_loc_bdd}, $u$ is bounded on $K^{\delta}$ and by \eqref{slope} one has
\begin{equation}\label{bounded1}
\pair{p}{y-x} \leq u(y) - u(x) \leq 2 \max_{K^{\delta}}|u| < + \infty, \ \ \forall \, x \in K,\,  y \in K^{\delta},\,  p \in \partial u(x).
\end{equation}
With $y:= x + \delta p/|p|$ for $p\neq 0$ in \eqref{bounded1}, one has
$$
\frac{\delta}{|p|} \pair{p}{p} \leq 2 \max_{K^{\delta}}|u|,
$$
and hence
$$
	|p| \leq \frac{2}{\delta} \max_{K^{\delta}}|u|, \ \ \forall \, p \in \partial u(K),
$$
which gives $\de u(K)$ is bounded in $\R^n$.

Next, we prove the Lipschitz estimate \eqref{convex_Lip}. Again using the inequality chain \eqref{slope} we have
$$
\forall x,y \in K: \quad |u(x) - u(y)| \leq \max \{|p|, |q| \} \, |x-y|, \quad \forall \, p \in \partial u(x), q \in \partial u(y),
$$
which implies \eqref{convex_Lip}, and hence the continuity of $u$ on $X$.

Finally, we show that $\partial u(K) \subset \R^n$ is closed, which together with the boundedness of $\partial u(K)$ shows that $\partial u(K)$ is compact. Let $\{ p_k\}_{k \in \N}  \subset \partial u(K)$ be any convergent sequence with $p_k \to p$ in $\R^n$ for $k \to \infty$. We need to show that $p \in \partial u(K)$. For each $k \in \N$, $p_k \in \partial u(K)$ implies that there exists $x_k \in K$ such that $(x_k, p_k) \in \partial u$. Since $K$ is compact, there exist $x \in K$ and a subsequence $\{x_{k_j}\}_{j \in \N}$ such that $x_{k_j} \to x$ as $j \to \infty$. Since $p_{k_j} \in \partial u(x_{k_j})$ for each $j \in \N$, we have 
	$$
	u(y) \geq u(x_{k_j}) + \pair{ p_{k_j}}{ y - x_{k_j} }, \quad \forall \, y \in X,\ \forall \, j \in \N
	$$ 
and passing to the limit $j \to \infty$ gives $u(y) \geq u(x) + \pair{p}{ y - x}$ 	for all $y \in X$, and hence $p \in \partial u(x) \subset \partial u(K)$.
\end{proof}

\begin{remark}\label{rem:equivalences} 
In the end, for $u: X \to \R$ convex on $X$ open and convex, the following are equivalent:
\begin{itemize}
	\item[(1)] $u$ is locally Lipschitz on $X$;
	\item[(2)] $u$ is continuous on $X$;
	\item[(3)] $u$ is continuous at some point $x \in X$;
	\item[(4)] $u$ is locally bounded on $X$;
	\item[(5)] $u$ is locally bounded from above on $X$.
\end{itemize}
In fact, the implications $(1) \Rightarrow (2) \Rightarrow (3) \Rightarrow (5)$ and $(2) \Rightarrow (4) \Rightarrow (5)$ are obvious. Hence it is enough to show that $(5) \Rightarrow (1)$, which follows from the considerations above. 
\end{remark}

An important consequence on the differentiability of a convex function follows. 

\begin{cor} \label{deusingv}
For $u: X \to \R$ convex on $X$ open and convex, one has
\begin{equation}\label{equivalences}
u \ \text{is differentiable on $X$} \ \ \Leftrightarrow \ \ \de u \ \text{is single-valued on $X$}  \ \ \Leftrightarrow \ \  u\ \text{is $ C^1(X)$},
\end{equation}
with $\de u (x) = Du (x)$ for each $x \in X$.
\end{cor}

\begin{proof}
We first show that if $\de u(x) = \{p\}$ then $u$ is differentiable in $x$ with $Du(x) = p$. Consider a nested sequence of compact sets $\{K_j\}_{j \in \N}$ such that $\bigcap_{j \in \N} K_j = \{x\}$, then by compactness
\[
\{p\} = \bigcap_{j \in \mathbb{N}} \de u(K_j),
\]
and thus
\begin{equation} \label{contconv}
p = \lim_{\substack{y\, \to\, x \\ q\, \in\, \de u(y)}} q.
\end{equation}
By (\ref{slope}) we have for each $x,y \in X$ with $x \neq y$:
\[
0 \leq \frac{ u(y)-u(x) - \pair{p}{y-x} }{ |y-x| } \leq \pair{q-p}{e},
\]
where $e \defeq (y-x)/|y-x| \in \mathbb{S}^{n-1}$. Hence for $y \to x$, \eqref{contconv} shows that $u$ is differentiable in $x$ with $Du(x) = p$. This also completes the implication that $\de u$ is single valued implies that $u$ is differentiable everywhere. 

Conversely, if $u$ is differentiable at $x$, let $p \in \de u(x) \neq \emptyset$. Again by (\ref{slope}) we have $t\pair{p}{e} \leq u(x+te) -u(x)$ for all $e \in \mathbb{S}^{n-1}$ and $t$ small. This implies that $\pair pe \leq \pair{Du(x)}e$, which, by linearity, forces $p = Du(x)$. This completes the first equivalence in \eqref{equivalences}.

Finally, we show that $u$ convex is differentiable on $X \ \ \Leftrightarrow \ \ u \in C^1(X)$. The implication $(\Leftarrow)$ is obvious and the implication $(\Rightarrow)$ follows from (\ref{contconv}) since it now gives
\[
		Du(x) = \lim_{y \to x} Du(y). \qedhere
\]
\end{proof}

Since $u$ convex is locally Lipschitz by Theorem \ref{thm:convex_Lip}(b), one can deduce first-order differentiability properties of convex functions, by exploiting the following famous result of Rademacher~\cite{rademacher}.

\begin{thm}[Rademacher] \label{rade}
Let $\Omega\subset \R^n$ be open and $G\colon \Omega \to \R^m$ be Lipschitz continuous. Then $G$ is differentiable almost everywhere in $\Omega$, with respect to the Lebesgue measure.
\end{thm}

The unidimensional version of Rademacher's theorem can be proved as a corollary of the analogous result for monotone functions, which can be extended to bounded variation functions thanks to Jordan's decomposition, and thus to the subspace of Lipschitz continuous functions (see e.g.\ \cite{anr, evansgar}). The passage to higher dimensions essentially uses the Fubini--Tonelli theorem (see e.g.\ \cite[Theorem~2.2.4]{meast} or \cite[Theorem~3.1.6]{fed:geo}).

An immediate consequence of Theorem \ref{thm:convex_Lip}(b) (convex functions are locally Lipschitz) and Rademacher's \Cref{rade} is the following.

\begin{cor} \label{rade:conv}
A convex function is differentiable almost everywhere, with respect to the Lebesgue measure.
\end{cor}

This is the first-order counterpart of Alexandrov's theorem on second-order differentiability of convex functions (\Cref{aleks}), which we are going to prove in \Cref{sec:so}.

\medskip
Finally, we conclude this section by highlighting two properties about sums and differences of convex functions, when one of them is quadratic.

Let us note that it is false in general that $\de (u+v) = \de u + \de v$; nevertheless the equality holds if, for instance, $v = \phi$ is a quadratic function. 

\begin{lem} \label{lem:subdiffsum}
Let $u$ be convex on a convex open set $X \subset \R^n$ and let $\phi$ be a quadratic convex function; then
\begin{equation} \label{eq:equalitysubdiff}
\de(u + \phi) = \de u + D\phi \quad \text{on $X$}.
\end{equation}
\end{lem}

\begin{proof}
Fix $x \in X$. By the definition \eqref{defsubd} of the subdifferential we always have $\de (u+ \phi)(x) \supset \de u(x) + \de \phi(x)$, and since $\phi$ is smooth by \Cref{deusingv} we know that $\de \phi(x) = \{D\phi(x)\}$. So let
\begin{equation} \label{barpinde}
\barr p \in \de(u+\phi)(x)
\end{equation}
(which is nonempty by \Cref{prop:subd_convex}), and set $p \defeq \barr p - D\phi(x)$. We want to show that $p \in \de u(x)$. Since one can write 
$$
	\phi(\cdot) = \phi(x) + \pair{D\phi(x)}{\cdot - x} + \frac12\pair{P(\cdot-x)}{\cdot-x},
	$$
where $P\geq 0$ is such that $D^2\phi \equiv P$ on $X$, we see that \eqref{barpinde} is equivalent to
\[
u(\cdot) \geq \psi(\cdot) \defeq u(x) + \pair p{\cdot - x} - \tfrac12\pair{P (\cdot -x)}{\cdot - x}.
\]
Hence $\epi(u) \subset \epi(\psi)$. Now, any dilation $\rho_t$ by $t>0$ centered at $(x, u(x))$ preserves this inclusion; that is,
\begin{equation}\label{sum_quad1}
	\rho_t(\epi(u)) \subset \rho_t(\epi(\psi)), \quad \forall \, t > 0.
\end{equation}
By the convexity of $u$ we also have 
\begin{equation}\label{sum_quad2}
	\epi(u) \subset \rho_t\epi(u), \quad \forall \, t \geq 1. 
\end{equation}
Combining \eqref{sum_quad1} and \eqref{sum_quad2}, for all $t\geq 1$ one has $\epi(u) \subset \rho_t\epi(\psi)$ where 
$$
\rho_t\epi(\psi) = (x,u(x)) + t\,\big\{ (z-x,y-u(x)) \in (X-x) \times \R :\ y \geq \psi(z) \big\} 
$$
which can be written as
$$
 \big\{ (z',y') \in (t(X-x)+x) \times \R :\ y' \geq u(x) + \pair{p}{z'-x} - \tfrac{1}{2t}\pair{P(z'-x)}{z'-x} \big\}.
$$
Letting $t \to +\infty$ shows that the dilations of the epigraph of $\psi$ decrease to the half-space $H$ which is the closed epigraph of the affine function $u(x) + \pair{p}{\cdot-x}$. Hence $\epi(u) \subset H$, yielding $p \in \de u(x)$.
\end{proof}

\begin{remark}
One might wonder if an analogous result could hold for either $u$ or $\phi$ not necessarily convex. The answer is no, in general. For example, if $u = \psi$ is a quadratic convex function and $\phi = -2\psi$, then $\de(u+\phi)(x)$ is empty for each $x \in X$, while $\de u(x) + D\phi(x)$ is not.

On the other hand, the answer is yes if both $u$ and $\phi$ are not convex (though one gets the useless identity $\emptyset = \emptyset$). In that case the right-hand side of \eqref{eq:equalitysubdiff} is clearly empty, and so is the left-hand side. Indeed, if it were nonempty, then $u+\phi$ would be convex (by \Cref{prop:subd_convex}); since $\phi$ is supposed to be smooth and non convex near $x$, it is (strictly) concave near $x$, yielding $u = (u+\phi) + (-\phi)$ convex near $x$ (being it the sum of two convex functions), thus contradicting the assumption that $u$ is not convex.
\end{remark}

Besides the result of \Cref{lem:subdiffsum}, one has an interesting result about differences of a convex function $u$ and a quadratic convex function $\phi$ as well: suppose that $u-\phi$ has a local maximum at some point $x$, with $(u+\phi)(x) = 0$ (borrowing a nomenclature which will be used in the next section (cf.~\Cref{ucqf}), we call such a $\phi$ an \emph{upper contact quadratic function for $u$ at $x$}); then $u$ is differentiable at $x$, with $Du(x) = D\phi(x)$. This is what the following proposition states.

\begin{lem} \label{ch1:datucp}
Let $u$ be convex on $X$ and suppose there exists a quadratic function $\phi$ such that
\[
(u- \phi)(y) \leq 0 \qquad \text{$\forall y$ near $x \in X$, with equality at $x$}.
\]
Then $u$ is differentiable at $x$ and $Du(x) = D\phi(x)$.
\end{lem}

\begin{proof}
Since $\de u(x)$ is nonempty, let $q\in\R^n$ such that 
\[
u(x) + \pair q{y-x} \leq u(y) \qquad \forall y\in X.
\]
Consider now the convex function $\tilde u \defeq u - u(x) - \pair q{\cdot-x}$, so that we have 
\[
\tilde u \geq 0\ \text{on $X$} \quad \text{and} \quad  \tilde u(x) = 0.
\]
Hence $0\in\de\tilde u(x)$ and since
\[
\phi = \phi(x) + \pair{D\phi(x)}{\cdot-x} + \tfrac12\pair{D^2\phi(x)(\cdot-x)}{\cdot-x}, \qquad \phi(x) = u(x),
\]
we immediately see that 
\[
(\tilde u- \tilde \phi)(y) \leq 0 \qquad \text{$\forall y$ near $x$, with equality at $x$}
\] 
if $\tilde\phi \defeq \phi - \phi(x) - \pair q{\cdot-x}$.
It follows that
\[
%0 \leq \tilde u(y) \leq \pair{p-q}{y-x} + \tfrac12 \pair{A(y-x)}{y-x}	\qquad	\forall y\ \text{near}\ x,
0 \leq \tilde u(y) \leq \pair{D\phi(x)-q}{y-x} + o(|y-x|)	\qquad	\forall y\ \text{near}\ x,
\]
and thus $q=D\phi(x)$. Indeed, if that were not the case, by choosing $y=x+t(D\phi(x)-q)$, for sufficiently small $t\in \R$, we would get
\[
0 \leq t + o(t) \qquad \forall t\ \text{near}\ 0,
\]
which is false. 

Finally, since now 
%$0\leq \tilde u(y) \leq \frac12\pair{A(y-x)}{y-x}$
$0\leq \tilde u(y) \leq o(|y-x|)$ for all $y$ near $x$, we see that
\[
\frac{|\tilde u(y)|}{|y-x|} = o(1) 	\qquad	\text{as}\ y\to x,
\]
thus $\tilde u$ is differentiable at $x$ with $D\tilde u(x) = 0$. Hence, $Du(x) = D\phi(x)$. Alternatively, since our argument proves that $\de u(x) = \{ D\phi(x) \}$, by \Cref{deusingv} one knows that $u$ is differentiable at $x$ with $Du(x) = D\phi(x)$.
\end{proof}

\begin{remark}
Note that a consequence of \Cref{ch1:datucp} is that \emph{all} upper contact quadratic functions for a convex function at some point $x$ share the same differential at $x$. 
\end{remark}


\subsection{The Legendre transform}

In this subsection we discuss the Legendre transform which is an important tool in the study convex functions. In particular, it provides a means of passing from first-order diferentiability to second-order differentiability for convex functions. We being with its definition.
Let $f\colon X \to \R$ be a convex function on a convex open set $X\subset \R^n$, and let $\de f$ be its subdifferential.
%\footnote{Note that since we implicitly assume (by writing that $f$ is real-valued) that the domain of $f$ is the whole $X$, then $\de f(x) \neq \emptyset$ for all $x\in \Omega$ by the Hahn--Banach Theorem, as we already said. However, what follows holds indeed also if $f$ is proper and $\Omega$ is suitably substituted by $\mathrm{dom}f$.}

\begin{definition}
The {\em Legendre transform $g = \scr Lf$ of $f$} is the function $g\colon \de f(X) \to \R$ defined by the relation
\begin{equation}	\label{leg:def}
f(x) + g(y) = \pair xy \qquad \forall (x,y)\in\de f;
\end{equation}
that is, one defines
\begin{equation}	\label{leg:def2}
g(y) \defeq \pair xy - f(x) \quad \forall y \in \de f(x), \ x \in X
\end{equation}
\end{definition}

\begin{remark}[Alternate definition of the Legendre transform] \label{Lt:aboveremark}
Notice that the definition \eqref{leg:def2} is somewhat implicit in the sense in order to define $g(y) = \scr L f(y)$ for $y \in \de f(X) = {\rm dom}(g)$ one must identify a fiber $\de f(x)$ in which $y$ lives. A more explicit way to define the Legendre transform $g$ of $f$ involves an optimization procedure which takes its form from the observation that
\[
(x, y) \in \de f \quad \iff \quad f(x) + \pair y{z-x} \leq f(z), \quad \forall z\in X,
\]
or equivalently
$$
f(z) - \pair{y}{z} \geq f(x) - \pair{y}{x}, \quad \forall z\in X.
$$
Hence $(x,y) \in \de f$ if and only if the function $f - \pair y\cdot$ assumes its minimum value on $X$ in $x$. That minimum value defines $-g(y)$; that is,
\[
-g(y) \defeq \inf_{z \in X} \big(f(z) - \pair{y}{z} \big) \quad \forall y \in \de f(X),
\]
or equivalently
\begin{equation}\label{def:leg_sup}
g(y) \defeq \sup_{z \in X} \big(\pair{y}{z} - f(z) \big) \quad \forall y \in \de f(X).
\end{equation}
We will see that these equivalent formulations \eqref{leg:def2} and \eqref{def:leg_sup} are both useful.
\end{remark}

The first result we prove is that the Legendre transform is an involution that produces a convex function whose subdifferential is the inverse of the subdifferential of $f$, in the sense of multivalued maps.

\begin{prop} \label{leg:gen}
Let $f: X \to R$ be convex on $X \subset \R^n$ open and convex. Then the following properties hold.
\begin{itemize}
	\item[(a)] $y \in \de f(X)$ if and only if 
	\begin{equation} \label{leg:sup}
	\scr L f (y) = \sup_{z\in X} \left( \pair yz - f(z) \right);
	\end{equation}
	\item[(b)] $\scr Lf: \de f(X) \to \R$ is convex;
	\item[(c)] for each $x\in X$, $y\in \de f(X)$,
	\[
	(x,y) \in \de f \ \ \iff \ \ (y,x) \in \de \scr L f;
	\]
	\item[(d)] $\scr L$ is an involution; that is, $\scr L(\scr L f)\big|_X = f$.
\end{itemize}
\end{prop}

\begin{proof}
 The equivalence of part (a) has been presented in \Cref{Lt:aboveremark}. 

The convexity of $\scr Lf$ in part (b) now follows from the fact that $\scr L f$ is the supremum of the family of affine functions $\{\pair\cdot z - f(z)\}_{z\in X}$.

For the implication $(\Longrightarrow)$ of part (c), let $(x,y) \in X \times \de f(X)$. Suppose that $(x,y)\in\de f$ (that is, $y \in \de f(x)$). In order to show that $(y,x) \in \de g$ where $g = \scr L f$, one needs to show that $y \in \de f(x)$; that is,
\begin{equation}\label{dual_leg1}
g(y) + \pair{x}{w-y} \leq g(w), \quad \forall \, w \in {\rm dom}(g) = \de f(X).
\end{equation}
For arbitrary $w\in \de f(X)$ we have $w \in \de f(z)$ for some $z\in X$ and hence
\begin{equation}\label{dual_leg2}
f(z) + \pair{w}{\hat{x}-z} \leq f(\hat{x}), \quad \forall \, \hat{x} \in X.
\end{equation}
By the definition \eqref{leg:def2} of the Legendre transform we have
\begin{equation}\label{dual_leg3}
w \in \de f(z) \ \ \Rightarrow \ \ g(w):= \pair{z}{w} - f(z)
\end{equation}
and
\begin{equation}\label{dual_leg4}
y \in \de f(x) \ \ \Rightarrow \ \ g(y):= \pair{x}{y} - f(x)
\end{equation}
which imply
\begin{equation*}\label{dual_leg5}
g(y) + \pair{x}{w-y} = \pair{x}{w} - f(x) = g(w) + \pair{x-z}{w} + f(z) - f(x) \leq g(w),
\end{equation*}
where the last inequality uses \eqref{dual_leg2} in $\hat{x} = x$ and gives \eqref{dual_leg1}. Notice that we also have $X \subset \de g(\de f(X))$ since $x \in X$ is arbitrary and $y \in \de f(x) \neq \emptyset$ implies $x \in \de g(y)$.

For the involution claim of part (d), notice that we have shown that $(x,y) \in \de f$ implies that $(y,x) \in \de \scr L f = \de g$, where $g$ is convex. Hence, for each $x \in X$, by the definition \eqref{leg:def}, we have
\begin{equation*} \label{eq:Linvol}
\scr L(\scr Lf)(x) = \scr Lg(x) := \pair xy - g(y) := \pair xy - \pair yx + f(x) = f(x), 
\end{equation*}
which proves {\em(d)} since $x \in X$ is arbitrary.

Finally, we need to prove the implication $(\Longleftarrow)$ of part (c).  Let $(x,y) \in X \times \de f(X)$ and suppose that $(y,x) \in \de \scr Lf$. We now have by \eqref{leg:def2} and the involution of part (d)
$$
\pair{x}{y} =  \scr L f (y) + \scr{L}\!\scr{L}f (y) =  \scr L f (y) + f(x)
$$
and hence by \eqref{leg:sup}
\[
\pair xy = \sup_{z\in X} \left( \pair{y}{z} - f(z) \right) + f(x) \geq  \pair{y}{z} - f(z) + f(x), \quad \forall z \in X,
\]
yielding $f(z) \geq  f(x) + \pair y{z-x}$ for each $z \in X$; that is, $y \in \de f(x)$, which is equivalent to $(x,y) \in \de f$.
\end{proof}

\begin{remark}
It is false in general that
\[
(x,y) \in \de f \ \iff \ (y,x) \in \de g,
\]
without requiring \emph{a priori} that $x \in X$ for the implication ``$\Longleftarrow$''. That is to say, $(y,x) \in \de g$ does not imply that $(x,y) \in \de f$ if $x$ is only assumed to belong to $\de g(\de f(X)) \supset X$ as it should be according to the definition of the subdifferential of $g$. This happens because, in general, $\de g(\de f(X)) \supsetneqq X$, which also implies that $\scr L\!\scr L f \neq f$. For example, consider $f = |\cdot|$, defined on $X=(-1,1) \subset \R$; it is easy to see that
\[
\de f(x) = \begin{cases}
-1 & \text{if $x \in (-1,0)$} \\
[-1,1] & \text{if $x = 0$} \\
1 & \text{if $x \in (0,1)$}
\end{cases},
\]
so that, according to \eqref{leg:def}
\[
g \defeq \scr Lf \equiv 0 \quad \text{on} \quad \de f(X) = [-1,1].
\]
Therefore,
\[
\de g(y) = \begin{cases}
(-\infty, 0] & \text{if $y = -1$} \\
0 & \text{if $y \in (-1,1)$} \\
[0,+\infty) & \text{if $y = 1$}
\end{cases}
\]
and
\[
\scr L g = \scr L\!\scr L f = |\cdot| \quad \text{on} \quad \de g(\de f(X)) = \de g([-1,1]) = \R \supsetneqq (-1,1) = X.
\]
For the sake of completeness, we also mention that this example is based on the fact that the function $\tilde f$ defined on $\R$ by $\tilde f = f$ in $(-1,1)$ and $f \equiv +\infty$ in $\R \setminus (-1,1)$ is not a {\em closed} convex function.
%\footnote{Recall that function is said to be closed if its epigraph is closed. One can prove that a convex function is closed if and only if it is lower semicontinuous. Furthermore, one can prove that lower semicontinuous convex functions on open convex subsets $X \subset \R^n$ are in fact continuous.}
On the other hand, the Fenchel--Moreau Theorem tells that if $f \colon \R^n \to \R$ is defined on the whole $\R^n$ (that is, we have $f \colon \R^n \to \R$ convex), then the equality $\scr L\!\scr L f = f$ is equivalent to $f$ being closed. Since one can always think of $f \colon X \to \R$ as $\tilde f \colon \R^n \to \R$ letting $\tilde f \equiv +\infty$ on $\R^n \setminus X$, in such a way that $\de f(x) = \de \tilde f(x)$ for each $x \in X$ while $\de \tilde f(x) = \emptyset$ if $x \notin X$ in order not to alter the domain $\de f(X)$ of the Legendre transform, one sees that $\scr L\!\scr Lf = f$ if and only if $\mathrm{dom}\,\scr L\!\scr L\tilde f \defeq \{ x \in \R^n :\ \scr L\!\scr L\tilde f < +\infty \} = \mathrm{dom}\, \tilde f = X$, it and only if $\scr L\!\scr L\tilde f = \tilde f$, if and only if $\tilde f$ is closed, which forces $X = \R^n$. Therefore, we conclude that one has $\scr L\!\scr L f = f$ in \Cref{leg:gen}{\em(iv)} only in the trivial case $X = \R^n$. The reader can see for instance \cite{rock} or \cite{vesely} for further details.
\end{remark}

If we now consider the Legendre transform of a function of the form
\begin{equation} \label{fform}
f = ru + \tfrac 12 |\cdot |^2, \quad r>0
\end{equation}
 we see that something better is possible. Recall that our final goal is to prove Alexandrov's theorem and note that $u$ is twice differentiable at some point if and only if $f$ is; this will be used later.

\begin{prop} \label{leg:f}
Let $u\colon X \to \R$ be a convex function; define $f$ as in~\emph{(\ref{fform})}. Then the following hold:
\begin{itemize}
	\item[(a)] the subdifferential $\de f \colon X \to \scr P(\R^n)$ is an expansive map; i.e.,
	\begin{equation} \label{leg:espan}
	|y_2 - y_1| \geq |x_2 - x_1| \quad \forall y_j \in \de f(x_j), \ j=1,2;
	\end{equation}
	\item[(b)] the subdifferential $G\defeq \de g$ of $g = \scr L f$ is a single-valued function from $\de f(X)$ onto $X$, and it is the inverse of $\de f$; that is,
	\[
	G(\de f(x)) = x \quad \forall x \in X;
	\] 
	\item[(c)] $G$ is contractive ($1$-Lipschitz); that is,
		\begin{equation} \label{leg:contr}
	|G(y_2) - G(y_1)| \leq |y_2 - y_1| \quad \forall y_j \in \de f(x_j), \ j=1,2.
	\end{equation}	
\end{itemize}
If, in addition, $u$ is bounded with $M\defeq \sup_X |u|$, $\delta \defeq 2\sqrt{rM}$ and $X_\delta \defeq \{ x\in X: d(x,\de X) > \delta \}$, then the following hold:
\begin{itemize}
	\item[(d)]  $X_\delta\subset \de f(X)$ and $\de f(X_{\delta}) \subset X$;
	\item[(e)] $g\defeq \scr L f\in C^1(X_\delta)$, with $Dg = G$ on $X_\delta$.
\end{itemize}
\end{prop}

\begin{proof}
First of all, by \Cref{lem:subdiffsum} applied to $ru$ and $\frac12{|\cdot|^2}$ we have $\de f = r\de u + \I$. Therefore if $x\in X$, $y,p\in\R^n$ are related by $y = x + rp$, then
\begin{equation}\label{subd_legendre}
(x,y) \in \de f \iff (x,p) \in \de u.
\end{equation}

For the expansivity claim of part (a), consider  $(x_j, y_j) \in \de f$, with $j=1,2$ and $p_j \defeq (y_j - x_j)/r$. Using \eqref{subd_legendre}, we have $p_j \in \de u(x_j)$ and then by the monotonicity of $\de u$ (\Cref{subdmon}) we have
\begin{equation}\label{mono_subd}
	\pair{p_2 - p_1}{x_2 - x_1} \geq 0, \quad \forall \, p_j \in \de u(x_j), j = 1,2,
\end{equation}
from which it follows that
\begin{align*}
|y_2-y_1||x_2-x_1| &\geq \pair{y_2-y_1}{x_2-x_1} = |x_2-x_1|^2 + r\pair {p_2-p_1}{x_2-x_1}\\
		&\geq |x_2 - x_1|^2,
\end{align*}
which yields \eqref{leg:espan}. 

For the claims of part (b), let $x_1,x_2 \in G(y) \cap X$ with $y \in \de f(X)$. Then by \Cref{leg:gen}{\em(c)} one has $y \in \de f(x_1) \cap \de f(x_2)$, and thus $x_1 = x_2$ by \eqref{leg:espan}. Therefore $G(\de f(x)) \cap X = \{ x \}$ for each $x \in X$, and in fact $G(\de f(x)) = \{ x \}$ for each $x \in X$; indeed, this easily follows from the fact that $G(\de f(x))$ is convex (by \Cref{lem:intpropsubdiff}{\em(a)} and \Cref{leg:gen}{\em(b)}) and $X$ is open. This shows that $G$ is single-valued and it is the inverse of $\de f$.

The contractivity of $G$ in part (c) follows easily. Indeed, in light of part (b), one has $x_j = G(y_j)$ in \eqref{leg:contr}; that is,
\[
|G(y_2) - G(y_1)| \leq |y_2 - y_1| \quad \forall y \in \de f(X).
\]

Now assuming that $u$ convex is also bounded, we verify the claims of parts (d) and (e). 
First for $y\in X_\delta$ we will show that $y \in \de f(X)$. To that end, we first note that the continuous function defined by $h_y \defeq ru + \frac12|\cdot-\,y|^2$ satisfies
\[
\inf_X  h_y = \min_{\barr B_\delta(y)} h_y.
\]
Indeed, if $z\notin \barr B_\delta(y)$ we have (where by hypothesis $|u| \leq M$ on $X$):
\[
h_y(z) - h_y(y) = ru(z) + \tfrac12|z-y|^2 - ru(y) > -2rM + \tfrac12\delta^2 = 0,
\]
therefore  $\min_{\barr B_\delta(y)} h_y \leq h_y(y) \leq \inf_{X\setminus \barr B_\delta(y)} h_y$. Now let $x \in \barr B_\delta(y)$ be a point which realizes the minimum of $h_y$ on $X$. Hence $0\in \de h_y(x)$; that is, $0 = rp + (x-y)$ for some $p\in \de u(x)$. Hence we have that $y\in\de f(X)$ by \eqref{subd_legendre}, which proves the first inslusion in part (d).

Next, for $y\in \de f (X_{\delta})$ we will show that $x \in X$. Since $y\in \de f (X_{\delta})$ we have $y \in \de f (x)$ for some $x\in X_{\delta}$. We know that $f- \pair y\cdot $ has a minimum point on $X$ at $x$. This is equivalent to $h_y$ having a minimum point on $X$ at $x$, since by definition $h_y = f -\pair y\cdot + \frac12|y|^2$. Then, $h_y$ has a minimum point at $x$, and we have
\[
|x-y|^2 \leq 2r(u(z)-u(x)) + |z-y|^2 \quad \forall z\in X, 
\]
yielding
\[
 |x-y|^2 \leq 4rM + |z-y|^2 \quad \forall z\in X
\]
and thus, taking the infimum over $z\in X$,
\begin{equation} \label{usata:lemLf}
|x-y| \leq \sqrt{\delta^2 + d(y, X)^2} \leq \delta + d(y, X).
\end{equation}
Suppose now that $y\notin X$ and consider the linear segment $I\defeq[x,y]$; since it is connected, there exists some point $w\in\de X \cap I$, otherwise $\{X,\barr X{}\compl\}$ would be a separation of $I$. We have
\[
d(x,\de X) \leq |x-w| = |x-y| - |y-w| \leq \delta,
\]
where we used \eqref{usata:lemLf} for the last inequality, since $|y-w| \geq d(y,\barr X) = d(y,X)$. This contradicts the hypothesis that $x\in X_{\delta}$, thus $y\in X$.

Finally, the claims of part (e) follow easily. Since $G=\de g$ is single-valued, we know that $G=Dg$ on $X_\delta \subset \de f(X)$, and that $G$ is continuous; see \Cref{deusingv}.
\end{proof}


\subsection{Second order theory: Alexandrov's theorem} \label{sec:so}

We begin with the following well-known definition.

\begin{definition}
Let $X \subset \R^n$ be open. A function $u \colon X \to \R$ is \emph{twice differentiable (in the Peano sense) at $x \in X$} if there exist $p \in \R^n$ and a symmetric $n\times n$ real matrix $A \in \Sc(n)$ such that
\begin{equation} \label{2od:def}
u(y) = u(x) + \pair{p}{y-x} + Q_A(y-x) + o(|y-x|^2) \quad \text{as $y \to x$},
\end{equation}
where $Q_A \defeq \frac12 \pair{A\cdot}\cdot$ is the quadratic form associated to $\frac12 A \in \Sc(n)$ whose Hessian is $A$. We will denote by
\begin{equation}\label{Diff2}
\mathrm{Diff}^2 u := \{x \in X: \ \text{$u$ is twice differentiable in $x$} \}.
\end{equation}
\end{definition}
The following elementary lemma will be used often.

\begin{lem}\label{lem:Diff2}
Given $u: X \to \R$ such that there exist $p \in \R^n$ and $A \in \Sc(n)$ such that \eqref{2od:def} holds for $x \in X$. Then
\begin{itemize}
	\item[(a)] $u$ is differentiable in $x$ with gradient $Du(x) = p$ ($p$ is unique);
	\item[(b)] $A$ is unique as well, and we denote $D^2u(x) = A$.
\end{itemize}
\end{lem}

\begin{proof}
It is clear that twice differentiability at $x$ implies differentiability at $x$ since \eqref{2od:def} implies
\begin{equation}\label{Diff1}
u(y)  = u(x) + \pair{p}{y-x} + o(|y-x|) \quad \text{as $y \to x$}.
\end{equation}
The uniqueness of $p$ follows easily from \eqref{Diff1}. In fact, if \eqref{Diff1} holds for $p$ and $p'$ with $p \neq p'$ one has
$$
\pair{p-p'}{y-x} = o(|y-x|) \quad \text{as $y \to x$},
$$
but choosing $y:=x+t(p-p')$ with $t \neq 0$ so small that $y \in X$ one obtains
$$
t|p - p'|^2 = |p - p'| \, o(t) \quad \text{as $t \to 0$},
$$
which gives $|p - p'| = o(1)$ for $t \to 0$, which contradicts $p-p'\neq 0$.

To see that $A$ is also unique, suppose that there exist $A \neq A \in \Sc(n)$ such that \eqref{2od:def} holds with $p = Du(x)$. Hence one has
$$
\frac{1}{2} \pair{(A - A')(y-x)}{y-x} = o(|y-x|^2) \quad \text{as $y \to x$}.
$$
Choosing $y = x + te$ for some eigenvector $e$ of $A-A'$ with eigenvalue $\lambda \neq 0$ one obtains a contradiction, since $\lambda t^2 = o(t^2)$ for $t \to 0$.
\end{proof}

We now move on to proving the celebrated Alexandrov's theorem on second-order differentiability of convex functions.

\begin{thm}[Alexandrov] \label{aleks}
Let $X \subset \R^n$ be open and convex, and let $u \colon X \to \R$ be convex. Then $u$ is twice differentiable (in the Peano sense) almost everywhere in $X$, with respect to the Lebesgue measure.
\end{thm}

The proof of this theorem depends on two ingredients (\Cref{mplt} and \Cref{sard} below). The former is a crucial property of the Legendre transform, which provides a useful sufficient condition for $f$ defined by~(\ref{fform}), with $u$ convex and bounded, to be twice differentiable. It essentially states that the subdifferential $G$ of the Legendre transform $g$ of $f$ maps the complement of its critical points $k_G$ to a set of points where $f$ is twice differentiable; that is, $G(X_\delta \setminus k_G) \subset \mathrm{Diff}^2 f$, where $X_\delta$ is defined as in \Cref{leg:f}. The latter is a Lipschitz version of Sard's theorem: it helps deal with the set of ``bad'' points (namely the set of critical values of $G$, which is a Lipschitz function by \Cref{leg:f}{\em(c)}), by telling that it is null.

Let us recall what critical points and critical values are, state the aforementioned ingredients, and prove Alexandrov's \Cref{aleks} making use of them. Then, we will conclude with the proof of the former ingredient, while the reader can find a technical measure-theoretic proof of the latter in \Cref{proofsard}. 

\begin{definition}
Let $\Omega \subset \R^n$ be open and let $G\colon \Omega \to \R^n$. We define the set of \emph{critical points} of $G$ to be
\[
k_G \defeq \big\{x\in \Omega:\ \text{either $DG(x)$ does not exists or $\det DG(x) = 0$} \big\},
\]
and we call $G(k_G)$ the set of \emph{critical values} of $G$.
\end{definition}

\begin{lem}[Magic property of the Legendre transform]	\label{mplt}
Let $f$ be defined by (\ref{fform}); that is, $f := ru + \frac{1}{2} | \cdot |^2$ with $r > 0$ a real number and $u$ a convex function which is bouned on $X$. Let $G = \de g = \de \scr Lf$ and $X_\delta$ be defined as in \Cref{leg:f}. Suppose that
\begin{itemize}
\item[(i)] $G$ is differentiable at $y_0 \in X_\delta$, with $B\defeq DG(y_0)$;
\item[(ii)] $x_0 \defeq G(y_0)$ is not a critical value of $G$; that is, ${\rm det} \, DG(y_0) \neq 0$;
\item[(iii)] $f$ is differentiable at $x_0$.
\end{itemize}
Then $f$ is twice differentiable at $x_0$, with $D^2f(x_0) = B^{-1}$.
\end{lem}

\begin{remark} \label{rmk:mplt}
In short, \Cref{mplt} tells that, if $|u| \leq M$ on $X$, $f \defeq ru + \frac12|\cdot|^2$ with $r>0$, and $G = \de \scr Lf$, then
\[
G\!\left( X_{2\sqrt{rM}} \setminus k_{G} \right) \cap \mathrm{Diff}^1 f \subset \mathrm{Diff}^2 f
\]
and
\[
D^2 u \circ G = r^{-1}\!\left( (DG)^{-1} - I \right) \quad \text{on $\left( X_{2\sqrt{rM}} \setminus k_G \right) \cap G^{-1}(\mathrm{Diff}^1 f)$}.
\]
Clearly, one can also replace each $\mathrm{Diff}^k f$ by $\mathrm{Diff}^k u$, since they are the same set because $\frac12|\cdot|^2$ is smooth.
\end{remark}

\begin{thm}[Sard's theorem for Lipschitz functions] \label{sard}
The set of critical values of a Lipschitz function has (Lebesgue) measure zero.
\end{thm}

\begin{remark}
As the reader might have noticed, our choice of defining the critical points as those points at which $G$ either is not differentiable or has singular derivative (sometimes they are defined by the second condition only) implies that the proof of Sard's theorem is going to need to use Rademacher's theorem.
\end{remark}

\begin{proof}[Proof of Alexandrov's \Cref{aleks}]
Without loss of generality, we can suppose that $u$ is in fact bounded on $X$; indeed, if not, it suffices to consider an exhaustion by compact convex sets $K_j \uto X$ (for instance, one can take $K_j \defeq \big\{ x \in X :\ d(x,\de X) \geq 2^{-j}, \, |x| < 2^j \big\}$ for all $j\in\N$), prove the theorem for $u|_{\intr\! K_j}$ (which is bounded since $u$ is continuous on $X$; cf.~\Cref{thm:convex_Lip}), and then note that the set of all points in $X$ where $u$ is twice differentiable, $\mathrm{Diff}^2 u$, has full measure as well, since 
\[
X \setminus \mathrm{Diff}^2 u  \subset \bigcup_{j\in\N} \left( \intr K_j \setminus \mathrm{Diff}^2 u \right),
\]
which, by the $\sigma$-subadditivity of the Lebesgue measure, yields $|X \setminus \mathrm{Diff}^2 u| = 0$.
%Fix $\epsilon > 0$ and let $X_\epsilon \defeq \{x\in X: d(x,\de X) > \epsilon \}$.
%Since $X$ is second countable, hence Lindel\"of, there exists a countable collection of open balls $\{B^i\}_{i\in\N}$ such that
%\begin{equation} \label{alex:sub}
%{X_{\epsilon}} \subset \bigcup_{i\in\N} {B^i} \subset \bigcup_{i\in\N} \overline{B^i} \subset X \quad \text{$u$ convex on each $B^i$}\footnote{To see this, consider the open cover of $X_\epsilon$ given by $\{ B_{\rho(z)}(z) : z\in X_{\epsilon} \}$, with $\rho(z) < \epsilon$ for all $z\in X_\epsilon$. Then it has a countable subcover satisfying~(\ref{alex:sub}).  
%Indeed, suppose $y\in \barr{B^i} \setminus X$ for some $B^i$ of center $z_i$ and consider the linear segment $\mathrm I \defeq [z_i, y]$; since it is connected, there exists $w\in \mathrm I \cap \de X$ and thus $\epsilon < d(z_i, \de X) \leq d(z_i, y) \leq \rho(z_i)< \epsilon$, contradiction.}
%\end{equation}
%
%Furthermore, by choosing the $\rho(z)$'s to be sufficiently small, we may also assume that $u$ is convex on each $B^i$.
%
%Fix now $i \in \N$ and $r>0$. 

Hence let us prove the theorem for $u$ bounded on $X$. Fix $r>0$ and let $f \defeq ru + \frac12|\cdot|^2$; by \Cref{mplt} (cf.~also \Cref{rmk:mplt}), we know that
\begin{equation} \label{proofalex:incl}
G(X_\delta \setminus k_G) \cap \mathrm{Diff}^1 u \subset \mathrm{Diff}^2 u, 
\end{equation}
where (cf.~\Cref{leg:f})
\begin{equation} \label{proofalex:delta}
\delta = 2\sqrt{r\norm{u}_\infty}
\end{equation}
By \Cref{rade:conv}, $\mathrm{Diff}^1 u$ has full measure, and, since $G$ is Lipschitz by \Cref{leg:f}{\em(c)}, $G(k_G)$ has measure zero by \Cref{sard}. Hence from \eqref{proofalex:incl} we have
\begin{equation} \label{alexproof:ineq}
\big|\mathrm{Diff}^2 u \big| \geq \big|G(X_\delta)\big| \geq \big|X_{2\delta}\big|,
\end{equation}
where the latter inequality comes from the fact that $X_{2\delta} \subset G(X_\delta)$; indeed, by \Cref{leg:f}{\em(d)}, $\de f(X_{2\delta}) \subset X_\delta$, which by part {\em(b)} of the same Proposition is equivalent to $X_{2\delta} \subset G(X_\delta)$, as we claimed.

Notice now that $r > 0$ is arbitrary, thus we can let $r \dto 0$; that is, (cf.~\eqref{proofalex:delta}) we can let $\delta \dto 0$ in \eqref{alexproof:ineq}. We get
\begin{equation*} \label{leg:proof:ball}
\big|\mathrm{Diff}^2 u \big| \geq | X \big|,
\end{equation*}
yielding in fact
\[
\big|\mathrm{Diff}^2 u \big| = | X \big|
\]
since the converse inequality trivially holds. This proves that $\mathrm{Diff}^2 u$ has full measure (in $X$), which is the desired conclusion.
\end{proof}

As we promised, we conclude this section with the proof of the magical property of \Cref{mplt}. Also, we invite the reader to have a look at \Cref{proofsard} for a proof of \Cref{sard} based on Besicovitch's covering theorem.

\begin{proof}[Proof of \Cref{mplt}]
First of all, we recall that $G = \de g: \de f(X) \to \R^n$ is single-valued and contractive by \Cref{leg:f} (b) and (c). Now, with $x_0 = G(y_0)$ for $y_0 \in \de f(X)$  we have  $Df(x_0) = y_0$ since
$$
	x_0 \in \de g(y_0) = \{ G(y_0)\} \ \iff \ y_0 \in \de f(x_0) = \{ Df(x_0)\} 
$$
by Proposition~\ref{leg:gen} and the hypothesis (iii) that $f$ is differentiable in $x_0$. 
 
 Assume now, without loss of generality, that $x_0 = 0$ and that $f(0) = Df(0) = 0$. This is possible since the function
\[
\tilde f \defeq f(\cdot + x_0) -f(x_0) - \pair {y_0}\cdot,
\]
will satisfy $\tilde f (0) = 0$ and $D \tilde f(0) = Df(x_0) - y_0 = 0$ and will be twice differentiable in $0$ if and only $f$ is twice differentiable in $x_0$. 

By the hypothesis (ii) (and the differentiablity of $G$ in $y_0$), $0$ is not a critical value of $G$ and hence $B:= DG(0)$ is invertible. The proof of the lemma reduces to showing that
\begin{equation}\label{D2f(0)}
f(x) - Q_A(x) = o(|x|^2),  \quad \text{for $x \to 0$},
\end{equation}
where $A\defeq B^{-1}$ and $Q_A(x) \defeq \frac12\pair{Ax}x$.

By \Cref{leg:f} (d), $X_{\delta} \subset \de f(X)$ so given $x\in X_\delta$ there exists $y\in X$ such that $f \in \de f(x)$, but then by \Cref{leg:gen} (c)
$$
x \in \de g(y) = \{G(y)\};
$$
that is, $x=G(y)$. For this pair $(x,y) \in \de f$, by the definition of the Legendre transform one can write
\begin{equation}\label{fg_identity}
%f(x) - \tfrac12\pair{Ax}x  = \tfrac12\pair{By}y - g(y) + \tfrac12\pair{y-Ax}x + \tfrac12\pair{x-By}y.
f(x) - Q_A(x)  = Q_B(y) - g(y) + \tfrac12\pair{y-Ax}x + \tfrac12\pair{x-By}y.
\end{equation}

Since $G$ is differentiable in $y_0 = 0$ with $G(0) = 0$ and $B=DG(0)$, one has 
\begin{equation}\label{taylor1}
	x =G(y) = By + o(|y|), \quad \text{for $y \to 0$}.
\end{equation}
From \eqref{taylor1} one also has for $x = G(y)$:
\begin{equation}\label{taylor2}
x - By = o(|y|), \quad \text{for $y \to 0$};
\end{equation}
\begin{equation}\label{taylor3}
\pair{x-By}y = o(|y|^2), \quad \text{for $y \to 0$};
\end{equation}
\begin{equation}\label{taylor4}
\pair{y-Ax}y \leq ||A|| \, |x-By| \, |x| = o(|y|^2), \quad \text{for $y \to 0$}.
\end{equation}
The consequeces \eqref{taylor2}--\eqref{taylor3} are obvious, and for \eqref{taylor4} one uses $A = B^{-1}$, the contractivity of $G$ (which implies $|x| =|G(y)| \leq |y|$) and \eqref{taylor3} to find 
\begin{align*}
\pair{y-Ax}y &= \pair{-A(x-By)}{x} \leq ||A|| \, |x - By| \, |x| \leq ||A|| \, |xBy| \, |y| \\
&= o(|y|^2), \quad \text{for $y \to 0$}.
\end{align*}

We will use the asymptotic expansions \eqref{taylor1}-\eqref{taylor4} to show that the Legendre transform $g$ of $f$ is twice differentiable with 
\begin{equation}\label{leg:cl:der}
 D^2g(0) = B.
\end{equation}
By definition $g(0) = 0$ (since $f(0) = 0$ and $(0,0) \in \de f$) and by hypothesis $G(0) = Dg(0) = 0$. Hence  the claim \eqref{leg:cl:der} is equivalent to 
\begin{equation}\label{D2g(0)}
g(y) - Q_B(y) = o(|y|^2) \quad \text{for $y \to 0$.}
\end{equation}
To see that \eqref{D2g(0)} holds, one first uses the mean value theorem for each $y$ fixed near $0$ to conclude that there exists $z\in[0,y]$ such that
\begin{equation}\label{MVT}
%g(y) - \tfrac12\pair{By}y = \pair {Dg(z)- Bz}{y}.
g(y) - Q_B(y) = \pair {Dg(z)- Bz}{y}.
\end{equation}
Indeed, applying the mean value theorem to the function $h := g - Q_B$ which satisfies $h(0) = 0$ there exists $z \in [0,y]$ such that $h(y) = \pair{Dh(z)}{y}$, which is \eqref{MVT}. 

Notice that $z = z(y) \to 0$ as $y \to 0$, which we will use to estimate the right hand side of \eqref{MVT}. Using \eqref{taylor1} one has
$$
Dg(z) - Bz = G(z) - Bz = o(|z|) \quad \text{for $z \to 0$}
$$
but since $|z(y)| \leq |y|$ one also has
$$
Dg(z) - Bz = G(z) - Bz = o(|y|) \quad \text{for $y \to 0$},
$$
which together with \eqref{MVT} implies
$$
| g(y) - Q_B(y)| \leq |DG(z) - Bz| \, |y| = o(|y|^2) \quad \text{for $y \to 0$},
$$
as desired in \eqref{D2g(0)}. 

Using \eqref{leg:cl:der}, we conclude the proof of the lemma by establishing \eqref{D2f(0)}.
The identity \eqref{fg_identity} says that
$$
f(x) - Q_A(x)  = Q_B(y) - g(y) + \tfrac12\pair{y-Ax}x + \tfrac12\pair{x-By}y,
$$
for $x = G(y)$. Using \eqref{taylor2}, \eqref{taylor4} and \eqref{D2g(0)} on the terms in the right hand yields
\begin{equation}\label{D2g(0)2}
f(x) - Q_A(x)  = o(|y|^2) \quad \text{for $y \to 0$},
\end{equation}
which will imply the needed \eqref{D2f(0)} ( $f(x) - Q_A(x)  = o(|x|^2)$ for $x \to 0$) provided that we can show that $x$ and $y$ related by $x = G(y)$ satisfy
\begin{equation}\label{bigO}
|y| = O(|x|) \quad \forall \, y \ \text{with $|y|$ small.}
\end{equation}
Since $A = B^{-1}$,  one has 
\begin{equation}\label{bigO1}
|y| = |ABy| \leq || A|| \, (|x-By| + |x| ), \quad \forall x,y \in \R^n
\end{equation}
and using \eqref{taylor3} for $x$ and $y$ with $x = G(y)$ one has 
\begin{equation}\label{bigO2}
|x-By| = |G(y) - By| = o(|y|), \quad \text{for $y \to 0$} 
\end{equation}
and hence
\begin{equation}\label{bigO3}
\forall \, \varepsilon > 0 \ \exists \, \delta > 0 \quad \text{such that} \quad |y| < \delta \ \Rightarrow \ |x - By| < \varepsilon |y|.
\end{equation}
Combining \eqref{bigO1} - \eqref{bigO3} one has
$$
	|y| \leq ||A|| (\varepsilon |y| + |x|), \quad \forall \, y \in B_{\delta}(0),
$$
and choosing $\varepsilon < 1/(2||A||)$ gives
$$
	|y| \leq 2 ||A|| \, |x|,  \quad \forall \, y \in B_{\delta}(0),
$$
and hence \eqref{bigO}, as needed. Notice that since $G$ is contractive $|x| = |G(y)| \leq |y|$ so that, in fact one has $|x| = |G(y)| \asymp |y|$ for all $ y \in B_{\delta}(0)$
\end{proof}


\section{Quasi-convex functions and upper contact jets} \label{qcfaj}  

With Alexandrov's theorem now in hand, we will discuss the next important ingredient for the theory of viscosity solutions, which roughly speaking concerns the existence of a sufficient amount of upper (and lower) test functions. These test functions are used in the viscosity formulation of subsolutions (and supersolutions) for weakly elliptic PDEs and for the subharmonics (and superharmonics) in general potential theories. This will be systematically developped in Part II of this work, but it is perhaps useful to recall the viscosity formalism as motivation.

Given a continuous differential operator $F \in C(X \times \R \times \R^n \times \Sc(n))$ on an open domain $X \subset \R^n$, by definition an upper semicontinuous function $u \in \USC(X)$ is a {\em viscosity subsolution} (on $X$) of the equation
\begin{equation} \label{viscosityequation}
F[u] \defeq F(x,u(x), Du(x), D^2u(x)) = 0, \quad x\in X
\end{equation}
if and only if, for each $x\in X$ fixed, one has $F[\phi] \geq 0$ for all \emph{upper test functions} $\phi$ of class $C^2$ for $u$ in $x$; that is, if and only if, for each $x \in X$ fixed, any $\phi$ of class $C^2$ in a neighborhood of $x$ such that
\begin{equation} \label{eq:ucqf}
u \leq \phi \quad \text{near $x$, with equality at $x$,}
\end{equation}
satisfies
\begin{equation} \label{visceqtest}
F(x, \phi(x), D\phi(x), D^2\phi(x)) \geq 0.
\end{equation}

Notice that upper semicontinuous functions \emph{not always} admit an upper test function (for instance, $u = |\cdot|$ at $0$); it is not difficult to show that there exists \emph{at least one} upper test function in \emph{every neighborhood} of each point of their domain,\footnote{See for example the beginning of the proof of \Cref{p:uvm}.} nonetheless this is not sufficient in order to have a good viscosity theory based on upper test functions, since this guarantees the possibility to check for the validity of \eqref{visceqtest} only in countably many points $x$. As we pointed out in the introduction, Alexandrov's theorem guarantees that \emph{convex} functions (and also \emph{quasi-convex} functions; see \Cref{def:qc} below) admit upper test functions \emph{almost everywhere}, and it characterizes their second-order Taylor polynomial as well (cf.~\Cref{baspropucp}); furthermore, we will show in the crucial \Cref{ucjt} by Harvey and Lawson (which is \cite[Theorem 1.8]{hlqc}) that the set of upper test functions for a quasi-convex function enjoys ``nice continuity properties''. As a consequence, it will turn out (see the AE \Cref{aet}, another essential result by Harvey and Lawson; \cite[Theorem 4.1]{hlae}) that one has indeed a sufficient amount of upper test functions when one deals with \emph{quasi-convex} viscosity subsolutions.
To close the circle, we recall that we have also anticipated that it is known (see \Cref{ch:qca}) that, under certain hypotheses, upper semicontinuous functions can be approximated by quasi-convex functions, and in some cases quasi-convex approximation techniques exist which allow to get from quasi-convex to upper semicontinuous subsolutions in viscosity theory (see~\Cref{sas}).

\subsection{Quasi-convex functions}

The definition we adopt of a (locally) quasi-convex function is the following.

\begin{definition} \label{def:qc}
A function $u\colon C\to \R$ is said $\lambda$-\emph{quasi-convex} on the convex set $C \subset \R^n$ if there exists $\lambda \in \R_+$ such that the function $u + \frac{\lambda}{2}|\cdot|^2$ is convex on $C$.

We say that $u$ is \emph{locally quasi-convex} on $X$ if for every $x\in X$ there exists a ball $B\subset X$ with $x\in B$ such that $u$ is $\lambda$-quasi-convex on $B$, for some nonnegative real number $\lambda = \lambda(x)$. If $\lambda$ is constant on $X$, we say that $u$ is \emph{locally $\lambda$-quasi-convex}.
\end{definition}

In proofs involving quasi-convex functions, we will suppose that functions are in fact convex, whenever this can be done without loss of generality. For instance, this is possible for results about differentiability, since the squared norm is smooth. Amongst such results, one has Alexandrov's~\Cref{aleks} as well, which clearly extends from convex to locally quasi-convex functions; that is, the following result holds.

\begin{thm}[Alexandrov, for quasi-convex functions] \label{aleks:qc}
Let $X \subset \R^n$ be open and let $u$ be locally quasi-convex on $X$. Then $u$ is twice differentiable (in the Peano sense) almost everywhere on $X$ with respect to the Lebesgue measure.
\end{thm}

\begin{proof}
Since the Euclidean space is Lindel{\"o}f, we know that there exists a countable family of balls $\{ B_i \}_{i\in\N}$ such that $u$ is $\lambda_i$-quasi-convex on each $B_i \subset X$, for some $\lambda_i > 0$, and $X = \bigcup_{i\in\N} B_i$. Since each $v_i \defeq u + \frac{\lambda_i}2|\cdot|^2$ is convex on $B_i$, by Alexandrov's \Cref{aleks}
\[
\big| B_i \setminus \Diff^2\! v_i \big| = 0 \quad \forall i \in \N,
\]
where we also note that $\Diff^2\! v_i = \Diff^2\! u$ since the squared norm is smooth. Hence we have, by the $\sigma$-subadditivity of the Lebesgue measure,
\[
\big| X \setminus \Diff^2\! u \big| \leq \sum_{i\in\N} \big| B_i \setminus \Diff^2\! u \big| = 0;
\]
that is, $u$ is twice differentiable almost everywhere on $X$, as desired.
\end{proof}

\begin{remark} \label{rmk:ch2:datucp}
Another property that quasi-convex functions inherit from convex functions is the differentiability at upper contact points of \Cref{ch1:datucp}; indeed, note that if $\phi$ is an upper contact quadratic test function for a $\lambda$-quasi-convex function $u$ at some point $x$, then $\tilde\phi \defeq \phi + \frac\lambda2|\cdot|^2$ is an upper contact quadratic test function for the convex function $v \defeq u + \frac\lambda2|\cdot|^2$ at $x$, and one recovers the identity $Du(x) = D\phi(x)$. This property will be used in a moment in the proof of \Cref{charc11}, and it will be also paraphrased in the language of {\em upper contact jets} later on in \Cref{datucp}.
\end{remark}

\subsubsection{A quasi-convexity characterization of $C^{1,1}$}
It is well-known that a function is affine if and only if it is simultaneously convex and concave. 
%For a better understanding of the notion of quasi-convexity, we are able to characterize those function that are simultaneously quasi-convex and quasi-concave. 
An interesting fact which helps to understand the notion of quasi-convexity is that functions which are both quasi-convex and quasi-concave must be differentiable with Lipschitz gradient (that is, of class $C^{1,1}$).
Of course, we say that $u$ is quasi-concave if $-u$ is quasi-convex. 

This characterization is included in \cite{hlkry} as an appendix, and was previously to be found in~\cite{hlqc}, in an appendix beginning as follows: ``It is interesting that the condition that a function be $ C^{1,1}$ is directly related to quasi-convexity, in fact it is equivalent to the function being simultaneously quasi-convex and quasi-concave. This was probably first observed by Hiriart-Urruty and Plazanet in \cite{hiriart}. An alternate proof appeared in \cite{eber}.''

The proof exploits the fact that quasi-convexity is preserved when one convolves with a \emph{mollifier}, that is a function $\eta$ such that
\[
\eta \in  C_\text{c}^\infty(\R^n), \quad \mathrm{supp}\,\eta \subset B_1,\quad \eta \geq 0,\quad \int_{\R^n} \eta = 1.
\]
This fact is formalised in the following lemma, where we use the standard notation
\[
\eta_\epsilon \defeq \frac1{\epsilon^n}\,\eta\!\left(\frac\cdot\epsilon\right), \quad \forall \epsilon > 0.
\]

\begin{lem} \label{convqc}
Suppose $u$ is $\lambda$-quasi-convex on $\R^n$ and let $\eta$ be a mollifier; let $\{\eta_\epsilon\}_{\epsilon>0}$ the approximate identity based on $\eta$. Then $u_{\epsilon} \defeq u \ast \eta_\epsilon$ is $\lambda$-quasi-convex.
\end{lem}

\begin{proof}
Let $f\defeq u + \frac\lambda2 |\cdot |^2$, convex. It is easy to see that $f_{\epsilon} \defeq f \ast \eta_\epsilon$ is still convex; also,
\[ \begin{split}
\left(|\cdot|^2  \ast \eta_\epsilon \right)\!(x) &\defeq \int_{\R^n} |x-z|^2\eta(z/\epsilon) \epsilon^{-n}\di z \\
&=  \int_{\R^n} |x-\epsilon y|^2\, \eta(y)\, \di y = |x|^2 - 2\epsilon \pair{x}{a} + \epsilon^2 b,
\end{split} \]
where
\[
a \defeq \int_{\R^n} y\, \eta(y)\,\di y \in \R^n \quad \text{and}\quad b \defeq \int_{\R^n} |y|^2\,\eta(y)\,\di y \in \R.
\]
Therefore, $0\leq D^2 f_{\epsilon} = D^2 u_{\epsilon} + \lambda\I$ and the thesis follows.
\end{proof}

\begin{prop} \label{charc11}
$u$ is of class $\lambda$-$ C^{1,1}$ (that is, $u \in C^{1,1}$ and the Lipschitz constant of $Du$ is $\lambda$) if and only if both $\pm u$ are $\lambda$-quasi-convex.
\end{prop}

\begin{proof}
Suppose $u$ is of class $\lambda$-$ C^{1,1}$ and let 
\[
f\defeq u + \frac\lambda2|\cdot|^2.
\]
 We prove that $u$ is $\lambda$-quasi-convex; the proof for $-u$ is analogous. We have
\[
\pair{Df(x) - Df(y)}{x-y} \geq \left(\lambda|x-y| - |Du(x) - Du(y)| \right) |x-y| \geq 0, 
\]
thus proving (cf.~\Cref{conviffmon}) that $f$ is convex, i.e.\ $u$ is $\lambda$-quasi-convex.

Conversely, suppose $\pm u$ are $\lambda$-quasi-convex. We first assume that $u$ is smooth; then we have $-\lambda\I \leq D^2u(x) \leq \lambda \I$ for all $x$. By the Mean Value Theorem,
\[
\text{$Du(x) - Du(y) = D^2u(z) (x-y)$ \quad for some $z\in[x,y]$,}
\]
therefore $|Du(x) - Du(y)| \leq \lambda|x-y|$.

For the general case, note that quasi-convexity and quasi-concavity together imply $u$ being $C^1$. Indeed, both $f_\pm \defeq \pm u + \frac\lambda2 |\cdot|^2$ have a supporting hyperplane from below at every point; this means that, for every $x$, there exist $p_\pm=p_\pm(x)$ such that the affine (and thus quadratic) functions $\phi_\pm \defeq -f_\pm(x) - \pair{p_\pm}{\cdot - x}$ are upper contact functions for $-f_\pm$ at $x$ (in the sense of \Cref{ch1:datucp}). Note now that $-f_\pm$ are both $2\lambda$-quasi-convex, since both $\pm u$ are $\lambda$-quasi-convex, and thus by \Cref{ch1:datucp} (cf.~also \Cref{rmk:ch2:datucp}) we know that $p=D(-f_+)(x)$, $q=D(-f_-)(x)$, yielding $(q-p)/2 = Du(x)$. That is to say, $u$ is differentiable everywhere. By \Cref{deusingv} we conclude that $u$ is $\scr C^1$.
 
Finally, approximate $u$ by $u_{\epsilon}$ and recall that $Du_{\epsilon} \to Du$ locally uniformly, since $u$ is $C^1$ (see, for instance, \cite[Theorems 9.3 and 9.8]{wheedzyg}). Therefore, since by \Cref{convqc} and what we showed above in the smooth case we know that $Du_{\epsilon}$ is $\lambda$-Lipschitz, so $Du$ is and we are done.
 \end{proof}

\subsection{Upper contact jets}

Let now $X\subset\R^n$ be any subset, and consider a function $u \colon X \to \R$. The next definition is of crucial importance for all that follows. As far as the notation we will use, $\Sc(n)$ will denote the space of symmetric $n\times n$ real matrices, equipped with its standard partial ordering (the Loewner order), where the quadratic form associated to $A \in \Sc(n)$ will often be normalized
\[
Q_A(y) \defeq \tfrac12\pair{Ay}y, \quad \text{so that}\ D^2Q_A \equiv A.
\]

\begin{definition} \label{def:ucp}
We say that a point $x \in X$ is an \emph{upper contact point for $u$} if there exists $(p,A)\in \R^n\times \Sc(n)$ such that 
\begin{equation}	\label{ucp}
u(y) \leq u(x) + \pair p{y-x} + Q_A(y-x) \qquad \forall y \in X\ \text{near}\ x;
\end{equation}
in this case, we write $(p,A) \in J_x^{2,+}u$, and we say that $(p,A)$ is an \emph{upper contact jet} for $u$ at $x$. In addition, if \eqref{ucp} holds with strict inequality for $y\neq x$, then $x$ is called a \emph{strict upper contact point}, and $(p,A)$ is called a \emph{strict upper contact jet}, for $u$ at $x$. 
\end{definition}

\begin{remark} \label{rmk:ucqf}
Note that we always have equality in~(\ref{ucp}) for $y=x$, hence we call $x$ a \emph{contact} point. Also, notice that $(p,A) \in J^{2,+}_xu$ if and only if the unique quadratic function $\phi$ such that $J^2_x \phi \defeq (\phi, D\phi, D^2\phi)(x) = (u(x), p , A)$ satisfies \eqref{eq:ucqf}; that is, if and only if $\phi$ is a \emph{quadratic upper test function for $u$ at $x$} in the viscosity sense.
\end{remark}

\begin{remark}
Besides the notion of strictness given in \Cref{def:ucp}, one can also consider a different one, of ``uniform'' strictness, which we call \emph{$\epsilon$-strictness}: given $\epsilon > 0$ fixed, we say that a point $x \in X$ is an \emph{$\epsilon$-strict upper contact point for $u$} if there exists $(p,A) \in \R^n \times \Sc(n)$ such that
\begin{equation}	\label{esucp}
u(y) \leq u(x) + \pair p{y-x} + Q_A(y-x) - \epsilon | y - x |^2 \qquad \forall y \in X\ \text{near}\ x;
\end{equation}
and, as above, $(p,A)$ will be called an \emph{$\epsilon$-strict upper contact jet for $u$ at $x$}.

We also point out that, when it comes to viscosity theory, it is equivalent to work with upper contact jets or $\epsilon$-strict upper contact jets, in the following sense: notice that the validity of inequality \eqref{visceqtest} for any upper test function $\phi$ for $u$ at $x$ is equivalent to the validity, for $(p,A) = (D\phi(x), D^2\phi(x))$ for some such $\phi$, of
\begin{equation} \label{visceq:jet}
F(x,u(x), p, A) \geq 0.
\end{equation}
Notice (cf.~\Cref{rmk:ucqf}) that such jets are not upper contact jets in general, yet they are \emph{little-o upper contact jets}; that is,
\begin{equation}	\label{loucp}
u(y) \leq u(x) + \pair p{y-x} + Q_A(y-x) + o(|y-x|^2) \qquad \forall y \in X\ \text{near}\ x.
\end{equation}
Hence, we have five families of upper contact jets (for $u$ at $x$), namely
\begin{itemize}[leftmargin=20pt]
\item {$J_S(x,u)$, $\epsilon$-strict upper contact jets, for some $\epsilon > 0$ possibly different from jet to jet},
\item {$J_s(x,u)$, strict upper contact jets according to \Cref{def:ucp}},
\item {$J_Q(x,u)$, quadratic upper contact jets; that is, $J_Q(x,u) = J^{2,+}_x u$ is the set of all the upper contact jets for $u$ at $x$ according to \Cref{def:ucp}},
\item {$J_{ C^2}(x,u)$, upper contact jets associated to upper test functions $\phi \in C^2$; the usual ones of viscosity theory},
\item {$J_o(x,u)$, little-o upper contact jets}.
\end{itemize}
It is easy to see that, for $(x,u)$ fixed,
\begin{equation} \label{inclusionsformulations}
J_S \subset J_s \subset J_Q \subset J_{C^2} \subset J_o,
\end{equation}
and one can prove (see \cite[Lemma C.1]{chlp}) that
\[
J_o \subset \barr{J_S},
\]
so that the closures of all sets in \eqref{inclusionsformulations} in fact coincide. By the continuity of the operator $F$, this shows that, in order to prove that $u$ is a viscosity subsolution of \eqref{viscosityequation} at $x$, one can check for \eqref{visceq:jet} to hold for all upper contact jets in any of the families above.
\end{remark}

As highlighted by Harvey and Lawson \cite{hlae}, there are two extreme cases where the set of upper contact jets of a function $u$ at a point $x$ are essentially completely understood. The first case is where $u$ has no upper contact jets at $x$; for instance, this is true of $u = |\cdot|$, at $x = 0$. The opposite case is when the function is twice differentiable at $x$; by basic differential calculus one has the following.

\begin{prop} \label{baspropucp}
Suppose $u$ is twice differentiable at $x \in X$; then 
\[
(p,A) \in J^{2,+}_x u \implies (p,A) = \big(Du(x), D^2u(x) + P\big),\ P\geq 0
\]
and the converse is true if $P > 0$.
%\noticina{Che fare di questa seconda parte?} Furthermore, for each $v \in \USC(X)$ one has
%\[
%(q,B) \in J^{2,+}_x v \iff \big(q+ Du(x), B + D^2u(x)\big) \in J^{2,+}_x (v+u).
%\]
\end{prop}

\begin{proof}
Since
\begin{equation} \label{twdiff}
u(y) = u(x) + \pair{Du(x)}{y-x} + Q_{D^2u(x)}(y-x) + o(|y-x|^2) \quad \forall\, y\ \text{near}\ x,
\end{equation}
by (\ref{ucp}) one sees that $(p,A) \in J^{2,+}_x u$ if and only if
\begin{equation} \label{d-a}
\pair{Du(x)-p}{y-x} + Q_{D^2u(x)-A}(y-x) + o(|y-x|^2) \leq 0 \quad \forall\, y\ \text{near}\ x.
\end{equation}
By choosing $y = x + t(Du(x) - p)$ for $t$ small, one has
\[
t|Du(x)-p|^2 + o{(t)} \leq 0 \quad \forall t\ \text{small},
\]
which forces $Du(x) = p$. At this point, recall that the Hessian of a twice differentiable function is symmetric and let $e$ be a unit eigenvector of $D^2u(x) - A$, relative to an eigenvalue $\lambda$; then by choosing $y = x+te$ with $t$ small one has
\[
\lambda t^2 + o(t^2) \leq 0,
\]
forcing $\lambda \leq 0$, and thus all the eigenvalues of $D^2u(x) - A$ have to be nonpositive, that is $A = D^2u(x) + P$ for some $P\geq 0$. Conversely, by (\ref{d-a}) it immediately follows that $(p,A) = (D^2u(x), D^2u(x) + P)$ is an upper contact jet for $u$ at $x$ if $P> 0$.
%
%For the second part, note that if $(q,B) \in J_x^{2,+} v$, then by (\ref{ucp}) and (\ref{twdiff})
%\[
%(v+u)(y) \leq (v+u)(x) + \pair{q+Du(x)}{y-x} + Q_{B+D^2u(x)}(y-x) + o(|y-x|^2) 
%\]
%for all $y$ near $x$, yielding $(q+Du(x), B+D^2u(x)) \in J^{2,+}_x(v+u)$, since the sign is preserved without the term $o(|y-x|^2)$ for $|y-x|$ small enough. Conversely, if $(q+Du(x), B+D^2u(x)) \in J^{2,+}_x(v+u)$, then analogously
%\[
%v(y) \leq v(x) + \pair{q}{y-x} + Q_{B}(y-x) + o(|y-x|^2), 
%\]
%and thus $(q,B) \in J_x^{2,+} v$.
\end{proof}

Let us also notice that one can also define \emph{lower} contact points and jets by reversing the inequality in (\ref{def:ucp}). Hence the set $J^{2,-}_xu$ of all lower contact jets for $u$ at $x$ can be considered as a \emph{local second-order subdifferential} of $u$ at $x$; indeed, if $(p,0) \in J^{2,-}_xu$, then $p \in \de u(x)$ according to \Cref{def:subd}, provided that we restrict the domain of $u$ to a neighborhood of $x$ in which (\ref{ucp}) holds.

Hence, as we pointed out in \Cref{geointerpsubdiff}, if $x$ is an upper (resp.~lower) contact point whose associated upper (resp.~lower) contact jets have zero matrix component, then $u$ has, for each such jet, one locally supporting hyperplane from above (resp.~below) at $x$. Hence we will call such points \emph{flat} contact points. Also, it is worth noting that $(0,0) \in J^{2,+}_x u$ (resp.~$(0,0) \in J^{2,-}_xu$) is equivalent to $x$ being a local maximum (resp.~minimum) point for $u$.

\subsection{The theorems on upper contact jets and on summands}
The main result on upper contact jets we wish to present is to be found in \cite[Theorem~4.1]{hlqc}. Two thirds of it have, in a certain sense, already been proved, since they come from properties discussed in \Cref{convediff} which quasi-convex functions inherit from convex functions.

\begin{thm}[Upper Contact Jet Theorem] \label{ucjt}
Let $u$ be quasi-convex on a neighborhood of $x$ and suppose $(p,A)$ is an upper contact jet for $u$ at $x$. Then
\begin{itemize}
\item \hspace{-3pt}{\small\sf(D\;at\;UCP)} \ if $(p,A)$ is an upper contact jet for $u$ at $x$, then $u$ is differentiable at $x$ and $p=Du(x)$ is unique.
\end{itemize}
Furthermore, for every set $E$ of full measure in a neighborhood of $x$ there exist a sequence $\{x_j\}_{j\in\N} \subset E \cap \Diff^2\! u$, with $x_j \to x$ as $j\to\infty$, and $\bar A \in \Sc(n)$ such that
\begin{itemize}
\item \hspace{-3pt}{\small\sf(PC\;of\;FD)} \ $Du(x_j) \to Du(x) = p$;
\item \hspace{-3pt}{\small\sf(PUSC\;of\;SD)} \ $D^2u(x_j) \to \bar A \leq A$.
\end{itemize}
\end{thm}

We split this theorem into three lemmas, hence it will be equivalent to the combinations of Lemmas \ref{datucp}, \ref{pcfd} and \ref{pusc} here below.

As we mentioned, the first two lemmas follow directly from certain results we proved for convex functions, namely the first one is a restatement of \Cref{ch1:datucp} for quasi-convex functions with the formalism of upper contact jets, and the second one is immediately deduced from \eqref{contconv} (though we will give short proof of them for the sake of completeness).

\begin{lem}[Differentiability at upper contact points, {\small\sf D\;at\;UCP}] \label{datucp}
Let $u$ be locally quasi-convex. If $x$ is an upper contact point for $u$, then $u$ is differentiable at $x$. Moreover, if $(p, A)$ is any upper contact jet for $u$ at $x$, then $p = Du(x)$ is unique.
\end{lem}

\begin{proof}
There exists a ball $B$ about $x$ and a quadratic function $\phi$ such that $u$ is $\lambda$-quasi-convex on $B$ for some $\lambda>0$ and $u \leq \phi$ on $B$, with $u(x) = \phi(x)$ (that is, $\phi$ is an upper contact quadratic function for $u$ at $x$). Hence $\tilde\phi \defeq \phi + \frac\lambda2|\cdot|^2$ is an upper contact quadratic function for the convex function $\tilde u \defeq u + \frac\lambda2|\cdot|^2$ at $x$. Therefore by \Cref{ch1:datucp} $\tilde u$ is differentiable at $x$ with $D\tilde u(x) = D\tilde \phi(x)$; that is, $u$ is differentiable at $x$ with $Du(x) = D\phi(x)$. Finally, the same argument also proves that $p = Du(x)$ is unique for any $(p,A) \in J^{2,+}_xu$ (cf.~also \Cref{baspropucp}).
\end{proof}

\begin{lem}[Partial continuity of first derivatives, {\small\sf PC\;of\;FD}] \label{pcfd}
Let $u$ be locally quasi-convex, and $x_j \to x$. If $u$ is differentiable at each $x_j$ and at $x$, then $Du(x_j) \to Du(x)$.
\end{lem}

\begin{proof}
As above, let $B$ be a ball about $x$ on which $\tilde u \defeq u+\frac\lambda2|\cdot|^2$ is convex. Eventually, $x_j,x \in B \cap \Diff^1\!\tilde u$, hence by the continuity property \eqref{contconv} (being $\de \tilde u(x_j) = \{ D\tilde u(x_j) \}$ and $\de \tilde u(x) = \{ D\tilde u(x) \}$), $D\tilde u(x_j) \to D\tilde u(x)$; this immediately yields $Du(x_j) \to Du(x)$, as desired.
\end{proof}

\begin{remark}
\Cref{datucp} clearly gives the first point of \Cref{ucjt}, while \Cref{datucp,pcfd} together give the second point. Indeed, since $(p,A) \in J^{2,+}_x u$, by \Cref{datucp} $x \in \Diff^1\!u$, and by Rademacher's and Alexandrov's theorems $|{\Diff^k\! u \cap \call U}| = \left|\,\call U\right|$ for $k=1,2$, where we denoted by $\call U$ the neighborhood of $x$ which we consider. Also, we know that $x$ is a limit point for $E \subset \call U$ with $|E| = |\,\call U|$ (otherwise there would exists a small ball $B_\rho(x) \subset \call U \cap E\compl$, yielding $|E| < |\,\call U|$, contradiction). Analogously $x$ is a limit point for $E \cap \Diff^1\! u$ as well, hence we can conclude by invoking \Cref{pcfd}. 
\end{remark}

The third point of \Cref{ucjt} is finally contained in the following lemma, which states indeed the ``deepest'' of the three properties; we postpone its proof in \Cref{sec:proof:pusc}, as we will exploit the Jensen--S{\l}odkowski Theorem \ref{jenslod} in combination with Alexandrov's \Cref{aleks:qc}.

\begin{lem}[Partial upper semicontinuity of second derivatives, {\small\sf PUSC\;of\;SD}] \label{pusc}
Let $u$ be a locally quasi-convex function, and $E$ be a set of full measure in a neighborhood of $x$. If $(p,A)$ is an upper contact jet for $u$ at $x$, then there exists a sequence of upper contact points $\{x_j\} \subset E\cap \Diff^2(u)$ such that $D^2u(x_j) \to \bar A \in \Sc(n)$ with $\bar A \leq A$.
\end{lem}

\medskip
While the Upper Contact Jet \Cref{ucjt} collects the basic information about upper contact jets of a given locally convex function $u$, the following result shows that when one has the sum of two locally quasi-convex functions, then one obtains useful information on the upper contact jets of the summands. Interestingly, this theorem, which is \cite[Theorem 7.1]{hlae}, can be interpreted in multiple ways: it reaffirms the partial upper semicontinuity of second derivatives, it yields a version of the Theorem on Sums for quasi-convex functions, and it conceals an \emph{addition theorem} within. The second interpretation is given below in \Cref{tosqc2} while the third one will be explained later on in \Cref{rmkone}.

\begin{thm}[Summand Theorem] \label{pusc:sum}
Suppose $u$ and $v$ are locally quasi-convex and that the sum $w \defeq u+v$ has an upper contact jet $(p,A)$ at $x$. Then the following hold:
\begin{enumerate}[label=\it(\roman*)]
\item	$x$ is an upper contact point for both the summands $u$ and $v$ (which are therefore differentiable at $x$ by \Cref{datucp}), whose upper contact jets at $x$ are of the form $(Du(x), \ast)$ and $(Dv(x), \ast)$;
\item 	for every set $E$ of full measure in a neighborhood of $x$, there exist a sequence $\{x_j\}_{j\in\N}$ of upper contact points for $w$, $x_j\in E \cap \Diff^2\! u \cap \Diff^2 \! v$, $x_j \to x$, and $B,C \in \Sc(n)$, such that 
\begin{gather*}
Du(x_j) \to Du(x), \quad \text{and} \quad Dv(x_j) \to Dv(x),\\
D^2u(x_j) \to B \quad \text{and} \quad D^2v(x_j) \to C,\quad \text{with $B+C \leq A$}.
\end{gather*}
\end{enumerate}
\end{thm}

\begin{proof}
Without loss of generality, suppose $u$ and $v$ are convex. By the Hahn--Banach Theorem there exists $q\in \R^n$ such that $(-q, 0)$ is an upper contact jet for $-v$ at $x$, thus $(p-q, A)$ is an upper contact jet for $u = w-v$ at $x$. Analogously we prove that $x$ is an upper contact point for $v$ as well.
Now, by \Cref{pusc} there exist a sequence $\{x_j\}\subset E \cap \Diff^2\!u \cap \Diff^2\!v$ and a matrix $\bar A \in \Sc(n)$, $\bar A \leq A$, such that $x_j \to x$ and $D^2w(x_j) \to \bar A$ as $j\to\infty$.\footnote{Note that in order to apply the lemma we are considering $E\cap \Diff^2(u)$ as the set of full measure near $x$.} Finally, if
\begin{equation} \label{arebounded} 
\{ D^2u(x_j) \}_{j\in\N} \ \text{and} \ \{ D^2v(x_j) \}_{j\in\N} \ \text{are bounded in $\Sc(n)$},
\end{equation}
then there exist $B,C \in \Sc(n)$ such that, up to a subsequence, $D^2u(x_j) \to B$ and $D^2v(x_j) \to C$, where $B+C = \lim_{j\to\infty}\left( D^2u(x_j) + D^2v(x_j) \right) = \lim_{j\to\infty} D^2w(x_j) = \bar A \leq A$. In order to prove \eqref{arebounded}, let $\lambda >0$ be such that both $u$ and $v$ are $\lambda$-quasi-convex in a neighborhood of $x$, and note that for any $\epsilon > 0$ there exists $j_0 = j_0(\epsilon) \in \N$ such that 
\[
-\lambda I \leq D^2u(x_j) = D^2w(x_j) - D^2v(x_j) \leq \bar A + \epsilon I + \lambda I \qquad \forall j \geq j_0.
\]
By fixing for instance $\epsilon = 1$ one sees that the tail $\{ D^2u(x_j) \}_{j\geq j_0}$ is bounded, thus the whole sequence $\{ D^2u(x_j) \}_{j\in\N}$ is bounded as well. The proof for $\{D^2v(x_j)\}_{j\in\N}$ is analogous.
\end{proof}

\begin{remark} \label{rmk:puscsum}
Note that since upper contact points for $w$ are also upper contact points for both $u$ and $v$, if $(p,A) \in J^{2,+}_x w$, then (cf.\ the proof above) we have $(p-q, A) \in J^{2,+}_x u$ and $(p-q', A) \in J^{2,+}_x v$, with $q+q' = p$, because one knows by \Cref{datucp} that $(p-q)+(p-q') = Du(x) + Dv(x) = Dw(x) = p$.
\end{remark}

\begin{remark} 
We used the basic fact that each interval $[A, B] \subset (\Sc(n), \leq)$ is compact  (with respect to the subspace topology inherited from $\R^{n^2}$). To prove it, let $Y \in [A,B]$ and note that by the Courant--Fischer min-max theorem one has that $\lambda_i(A) \leq \lambda_i(Y) \leq \lambda_i(B)$ for all $i = 1,\dots,n$.  Hence, if the eigenvalues are arranged in increasing order, the entries of $Y$ are bounded by $n\left(|\lambda_1(A)| \vee |\lambda_n(B)|\right)$, thus, since it is trivial that $[A,B]$ is closed, one concludes by invoking the Heine--Borel theorem.
\end{remark}

%Note that one may suppose that by construction the sequence $\{x_j\}$ of \Cref{pusc:sum} enjoys all the properties guaranteed by \Cref{ucjt}. This makes the \emph{Summand Theorem} a powerful tool, possibly stronger than the classic Theorem on Sums.

As we said, a version of the Theorem on Sums for quasi-convex functions immediately follows if we double the variables, as it is easy to see.

\begin{cor}[Theorem on Sums for quasi-convex functions] \label{tosqc2}
Let $u$ and $v$ be $\lambda$-quasi-convex on some open subsets $X$, $Y$ of $\R^n$, respectively. Define 
\[
w(x,y) \defeq u(x) + v(y)
\]
on $X\times Y \subset \R^n \times \R^n$ and suppose that $(p,A)$ is an upper contact jet for $w$ at $z = (\hat x,\hat y) $. Then $\hat x$, $\hat y$ are contact points for $u$, $v$, respectively, and for each set $E$ of full measure near $z$ there exists a sequence $(x_j, y_j) \in \Diff^2(u) \times \Diff^2(v)$ such that 
\[\begin{split}
(x_j, u(x_j), Du(x_j), D^2u(x_j)) &\to (\hat x, u(\hat x), Du(\hat x), A_1), \\
(y_j, v(y_j), Dv(y_j), D^2v(y_j)) &\to (\hat y, u(\hat y), Du(\hat y), A_2),
\end{split}\] 
with
\[
-\lambda I \leq \left(\! \begin{array}{cc}
A_1 &  0 \\
0  & A_2 \\
\end{array}\! \right)
\leq A.
\]
\end{cor}

By induction, one may also extend this result to several variables, so that it assumes a shape which more resembles the classic Theorem on Sums for semicontinuous functions, in the equivalent formulation one finds in~\cite[Appendix, Theorem 3.2$'$]{user}. 
We leave it to the reader to write an appropriate statement.
%Let us give the following definition to streamline the upcoming statement.
%
%
%\begin{defn}
%Let $u$ be a real-valued function defined on a subset $E$ of $\R^n$ and, for $N > n$, let $\pi\colon \R^N \to \R^n$ the projection onto the first component of the product $ \R^n \times \R^{N-n}  \simeq\R^N$. We define the \emph{cylindrical extension} of $u$ to $E \times \R^{N-n} \subset \R^N$ to be $\tilde u \defeq u\circ \pi$.
%\end{defn}
%
%\begin{thm} \label{tosqc}
%Let $u_i$ be $\lambda_i$-quasi-convex in a neighborhood $V_i$ of $0\in \R^{n_i}$, for $i=1,\dots, k$, with $u_i(0) = 0$. Let $\tilde{u_i}$ be the cylindrical extension of $u_i$ to $\R^{n_1} \times \cdots \times V_i \times \cdots \times \R^{n_k}$ and set
%\[
%w \defeq \sum_{i=1}^k \tilde{u_i} \qquad \text{on $V_1  \times \cdots \times V_k$}.
%\]
%Suppose that $(0,A)$ is an upper contact jet for $w$ at $0 \in \R^N$, where $N\defeq \sum_{i=1}^k n_i$.
%
%Then $Du_i(0) = 0$ and there exist sequences $\{x_j^i\}_{j\in \mathbb{N}} \subset \Diff^2(u_i)$ converging to $0$ such that $D^2u_i(x^i_j) \to A_i \in \Sc(n_i)$ with
%\[
%-\Lambda I \leq \left( \begin{array}{ccc}
%A_1 & \cdots & 0 \\
%\vdots & \ddots & \vdots \\
%0 & \cdots & A_k \\
%\end{array} \right)
%\leq A,
%\]
%where $\Lambda \defeq \max_i \lambda_i$.
%\end{thm}

For the sake of completeness, we shall prove in the next part a version of the classic Theorem on Sums. It can be quite easily deduced from its quasi-convex counterpart because, roughly speaking, it is possible to transform a function which is bounded above into a quasi-convex function, in such a way that it produces a regular displacement of the contact points of every given upper contact jet; to be slightly more precise, the displacement is proportional to the gradient-component of the jet, as we shall see.

\section{The lemmas of Jensen and S{\l}odkowski} \label{sec:js}

In the crucial passage ``from almost everywhere to everywhere'', two classic instruments dealing with (quasi-)convex functions are S{\l}odkowski's density estimate \cite[Theorem~3.2]{slod} and Jensen's Lemma~\cite[Lemma~3.10]{jensen}. For instance, the former result yields S{\l}odkowski's \emph{Largest Eigenvalue Theorem} (LET) \cite[Corollary~3.5]{slod}, playing an essential role in Harvey and Lawson's proof of the Subaffine Theorem~\cite[Theorem~6.5]{hldir09}, while the latter yields Jensen's maximum principle \cite[Theorem~3.1]{jensen}, and is a key ingredient in the proof of the Theorem on Sums~\cite[Theorem~3.2]{user} as well.

S{\l}odkowski's original proof of his density estimate is based on two nontrivial properties~\cite[Proposition~3.3(iii) and Lemma~3.4]{slod}, concerning the generalized largest eigenvalue of the Hessian at $x$ of a convex function $u$~\cite[Definition~3.1]{slod}, and spheres supporting the graph of $u$, $\Gamma(u)$, from the above at $(x,u(x))$; that is, spheres in $\R^{n+1}$ lying above $\Gamma(u)$ and touching it only at $(x, u(x))$. Following Harvey and Lawson~\cite{hlqc}, our proof of S{\l}odkowski's estimate will be based on ``paraboloidal'', simpler, counterparts of such properties.

This approach will lead to an important consequence, which Harvey and Lawson call S{\l}odkowski's lemma, and which turns out to be equivalent to a reformulation of theirs of Jensen's lemma. This eventually allows one to merge the two lemmas into a \emph{Jensen--S{\l}odkowski's theorem}, and enlightens an interesting path to prove both Jensen's lemma and S{\l}odkowski's LET.

At the end of the section, a second nonclassic proof of Jensen's lemma will be proposed. Jensen points out that his lemma is based on C.\ Pucci's ideas \cite{pucci} and gives a proof which is relies on the area formula (see, e.g.,~\cite[Theorem~3.2.3]{fed:geo}); we will offer a different area-formula-based proof by reformulating Jensen's lemma \emph{à la} Harvey--Lawson~\cite{hlqc} and then following an argument by Harvey~\cite{har:pc} which passes also through Alexandrov's maximum principle (see, for example, \cite{cafcab}).

\subsection{Classic statements: a review}

For the convenience of the reader, we recall some classic statements of Jensen's and S{\l}odkowki's which we deal with.

In \cite{slod}, S{\l}odkowski proves the following estimate for the Lebesgue lower density of a sub level set of his function $\call K(u,x)$, generalizing the largest eigenvalue of the Hessian at $x$ of a convex function $u$. He defines
\[
\call K(u,x) \defeq \limsup_{\epsilon \to 0} 2\epsilon^{-2} \max_{{\mathbb S}^{n-1}} \! \big( u(x+\epsilon\,\cdot\,) - u(x) - \epsilon\pair{Du(x)}{\cdot\,} \big)
\]
if $x \in \Diff^1\! u$ and $\call K(u,x) \defeq +\infty$ otherwise, and proves the following.

\begin{lem}[S{\l}odkowski's density estimate; {\cite[Theorem 3.2]{slod}}] \label{slodthm}
Let $u$ be convex near $x^* \in \R^n$, and suppose that $\call K(u,x^*) = k^* < +\infty$. Then for any $k > k^*$ the set $\{ x : \ \call K(u,x) < k \}$ is Borel and its Lebesgue lower density at $x^*$ satisfies
\[
\delta^-_{x^*}(\{ \call K(u,\cdot) < k\}) \geq \left(\frac{k-k^*}{2k}\right)^n.
\]
\end{lem}
We recall that the Lebesgue lower density at $x^*$ of a measurable set $E$ is defined as
\[
\delta^-_{x^*}(E) \defeq \liminf_{\epsilon \dto 0} \frac{|E \cap B_\epsilon(x^*)|}{|B_\epsilon(x^*)|} .
\]

\begin{remark}
$\call K(u,x)$ indeed generalizes the concept of \emph{largest eigenvalue} of the Hessian of a convex function $u$ at $x$, since, if $u$ has second-order Peano derivatives at $x$, then $\call K(u,x) = \lambda_n(D^2u(x))$, the largest eigenvalue of the Hessian $D^2u(x)$:
\[
\call K(u,x) = \limsup_{\epsilon \to 0}\, 2 \epsilon^{-2} \displaystyle\max_{|h|=1}\! \Big(  \epsilon^2 Q_{D^2u(x)}(h) + o(\epsilon^2) \Big) = \Vert D^2u(x) \Vert = \lambda_n(D^2u(x)).
\]
\end{remark}

The two nontrivial properties on which S{\l}odkowski's proof is essentially based are the following.

\begin{lem}[{\cite[Proposition 3.3(iii)]{slod}}] \label{slod(iii)}
Let $U\subset \R^n$ be open and $u \colon U \to \R$ be convex. Suppose there exists a sphere $\mathbb S(c,r) \defeq \de B_r(c) \subset \R^{n+1}$ which supports $\Gamma(u)$ from the above at $(x,u(x))$, with $x \in \Diff^1\!u$; then
\[
\call K(u,x) \leq \frac{\left(1+|Du(x)|^2\right)^{\frac32}}{r}.
\]
\end{lem}

\begin{lem}[{\cite[Lemma 3.4]{slod}}] \label{slodestlem}
Let $u$ be nonnegative and convex on $B_\rho \subset \R^n$, for some $\rho > 0$, with $u(0) = 0$ and $Du(0) = 0$. Assume that there exists a ball $B_R((0,\dots,0,R)) \subset \R^{n+1}$ which intersects the graph of $u$ only at $0 \in \R^{n+1}$, and let, for $0<r<R$,
\[
X_r \defeq \big\{ x \in B_\rho :\ \text{$\exists\,c \in \R^{n+1}$\! s.t.\ $\mathbb S(c,r)$ supports $\Gamma(u)$ from the above at $(x,u(x))$} \big\}.
\]
Then the lower density of $X_r$ at $0$ satisfies
\[
\delta^-_0(X_r) \geq \left(\frac{R-r}{2r}\right)^n.
\]
\end{lem}

As a corollary of his density estimate, S{\l}odkowski obtains the following Largest Eigenvalue Theorem.

\begin{thm}[{S{\l}odkowski's LET; \cite[Corollary~3.5]{slod}}] \label{slodle}
Let $u$ be locally convex on an open set $X$ and suppose $K(u,x) \geq M$ for a.e.\ $x\in X$. Then $K(u,x) \geq M$ for all $x \in X$.
\end{thm}

We conclude recalling Jensen's lemma, which we present as one finds it in the \emph{User's guide}~\cite{user}.

\begin{lem}[{Jensen; \cite[Lemma A.3]{user}}] \label{u:jen}
 Let $w\colon \R^n \to \R$ be quasi-convex and $\hat x$ be a strict local maximum point of $w$. For $p\in\R^n$, set $w_p(x) =w(x) + \pair px$. Then, for $r,\delta>0$, the set
\[
K\defeq\{x\in \barr B_r(\hat x):\ \text{$\exists\, p\in \barr B_\delta$ for which $w_p$ has a local maximum at $x$} \}
\]
has positive Lebesgue measure. 
\end{lem}

\begin{remark}
Jensen's original statement~\cite[Lemma~3.10]{jensen} is expressed with different hypotheses, as it requires $u$ to be continuous and in $W^{1,\infty}$, with $D_\nu^2 u \geq -\lambda I$ in the sense of distribution, for all directions $\nu$. Nevertheless, one can see that this is equivalent to asking that $u$ be $\lambda$-quasi-convex (for instance, one can use \cite[Theorem~31]{dudley}).
\end{remark}


\subsection{Upper contact quadratic functions and the vertex map}

Our journey towards the reformulation of these classic results begins with the definitions of the objects which will be at the center of our study in the present section. What follows here was developed by Harvey and Lawson \cite{hlqc}, for their ``paraboloidal'' reformulation of S{\l}odkowski's approach.

\begin{definition}
We say that $\phi$ is a \emph{quadratic function of radius $r$} if $D^2\phi \equiv \frac1r\I$. In this case it is easy to see that $\phi$ can be written in a unique way as $\phi = \phi(v) + \frac{1}{2r}|\cdot - v|^2$; we will call $v$ the \emph{vertex point} of $\phi$. 
\end{definition}

\begin{definition} 	\label{ucqf}
Let $u \colon X \to \R$ be any function. We say that a quadratic function $\phi$ is an \emph{upper contact quadratic function for $u$ at $x\in X$} if condition \eqref{eq:ucqf} holds; that is, if
\[
\text{$u(y) \leq \phi(y)$ for all $y \in X$ near $x$, \quad and \quad $u(x) = \phi(x)$.}
\]
If the inequality is strict for all $y\neq x$, then we say that $\phi$ is \emph{strict} upper contact quadratic function; if it holds for all $y\in X$, we say that $\phi$ is \emph{global} upper contact quadratic function \emph{on $X$}.
\end{definition}

The idea coming from this kind of upper contact functions is to focus on upper contact jets whose matrix component $A \in \Sc(n)$ is fixed, and on their respective {\em global} contact points; that is, those upper contact points such that inequality \eqref{ucp} holds for each $y \in X$.

\begin{definition}
An upper contact jet for $u$ at $x$ is said to be of \emph{type $A$} if it is of the form $(p,A)$ for some $p \in \R^n$, and $x \in X$ will be called an \emph{upper contact point of type $A$ for $u$}. 
The set of all \emph{global} upper contact points of type $A$ for $u$ on $X$ will be denoted by $\C(u, X, A)$; that is,
\[
\C(u, X, A) \defeq \{ x \in X :\ \text{$\exists \, p \in \R^n$ such that \eqref{ucp} holds $\forall \, y \in X$} \}.
\]
\end{definition}

Note that, according to the \cref{def:ucp}, the set of the contact points for some global upper contact quadratic function of radius $r$ coincides with $\C(u, X, \lambda I)$, for $\lambda = \frac1{r}$.

\begin{remark}[Notation]
We will denote by $B_\rho(x) \subset \R^n$ the open ball of radius $\rho$ about $x$, and we will omit the center when it is the origin; that is, we will use the notation $B_\rho \defeq B_\rho(0)$.
\end{remark}

\begin{remark}
We are going to focus our attention on sets of upper contact points which are global on \emph{closed} balls; that is of the form $\C(u, \barr B_\rho, A)$, up to translations that center the ball at the origin. In particular, we will be mainly interested in the Lebesgue measure of such sets, so let us also point out that, when it comes to measures, if $u$ is continuous, then one can equivalently consider open or closed balls, since
\[
\C(u, B_\rho, A) \subset \C(u, \barr B_\rho, A) \quad\text{and}\quad \C(u, \barr B_\rho, A) \setminus \C(u, B_\rho, A) \subset \de B_\rho.
\]
\end{remark}

\begin{remark}
Since we are going to discuss, as we said, the Lebesgue measure of the set $\C(u, X, A)$, in particular in the special case when $u$ is locally quasi-convex, for the sake of completeness, let us point out that in that case it is in fact a measurable set. Indeed, if one defines 
%$h_{y,p,A} = u - u(y) + \pair p{y-\,\cdot} + \tfrac12\pair{A(y-\,\cdot)}{y-\,\cdot}$, 
$h_{y,p,A} \defeq u - u(y) + \pair p{y-\,\cdot} + Q_A(y-\,\cdot)$, then it is easy to see that
\[
\C(u, X, A) = \bigcup_{p\in \R^n} \bigcap_{y \in X} h_{y,p,A}^{-1}([0,+\infty)),
\]
where, by \Cref{datucp},
\[
\bigcap_{y \in X} h_{y,p,A}^{-1}([0,+\infty)) =\vcentcolon \C(u,X,A;\,p) \subset Du^{-1}(p);
\]
hence in fact
\[
\C(u,X,A) = \bigcap_{y \in X} \tilde h_{y,A}^{-1}([0,+\infty)) \subset \Diff^1\! u,
\]
where
\[
\tilde h_{y,A} \defeq u - u(y) + \pair {Du}{y-\cdot} + Q_A(y-\cdot) \quad \text{on} \ \Diff^1\! u,
\]
and $\Diff^1\! u$ has full measure by Rademacher's \Cref{rade} (or Alexandrov's \Cref{aleks:qc}, if one prefers). Now, clearly $\tilde h_{y,A}$ is a measurable function, so that each $\tilde h_{y,A}^{-1}([0,+\infty))$ is a measurable set; and to conclude, note that by the continuity of the map
\[
y \mapsto u(x) - u(y) + \pair{Du(x)}{y-x} + Q_A(y-x), \quad \text{with $x$ fixed},
\]
we can also write
\[
\C(u,X,A) = \bigcap_{y \in D} \tilde h_{y,A}^{-1}([0,+\infty)), \quad \text{for some $D \subset X$ countable and dense},
\]
so that one finally sees that $\C(u,X,A)$ is measurable.
\end{remark}

The following result is an interesting exercise that shows an elementary geometric property of the convex hull of two open paraboloids (namely the open epigraphs of two quadratic functions). We will denote the graph of $\phi$ by $\Gamma(\phi)$, its \emph{open} epigraph by $\epi(\phi)$, and the convex hull of a subset $S \subset \R^{n+1}$ by $\conv(S)$.

\begin{lem}[Slubbed hull property] \label{shp}
There is an open vertical slab $\frk S\subset \R^n\times\R$ written as the intersection $\call{H}_1 \cap \call{H}_2$ of two parallel vertical open half-spaces with the following property. Let $\frk C \defeq \conv(\epi(\phi_1) \cup \epi(\phi_2))$, then
\[
\Gamma(\phi_j) \cap \frk C =  \Gamma(\phi_j)\cap \call H_j \qquad \text{for}\ j=1,2.
\]
Moreover, if $v_j$ is the vertex point of $\phi_j$, for $j=1,2$, then the width of $\frk S$ is $|v_1 - v_2|$.
\end{lem}


\begin{remark}[Tangent plane to the union of two paraboloids] \label{parab:cases}
Let $z\in \de\big(\!\epi(\phi_1)\cup\epi(\phi_2)\big) \setminus \big(\Gamma(\phi_1) \cap \Gamma(\phi_2)\big) =\vcentcolon \frk B$ and let $H_z$ be the tangent hyperplane to $\epi(\phi_1)\cup\epi(\phi_2)$ at $z$. Then one of the following is verified (see \Cref{fig:parabcases}):
\begin{enumerate}[label=\textit{(\roman*)}]
\item	$H_z \cap \de(\epi(\phi_1) \cup \epi(\phi_2))$ is a singleton;
\item	$H_z \cap \de(\epi(\phi_1) \cup \epi(\phi_2))$ is a doubleton;
\item	$H_z \cap \de(\epi(\phi_1) \cup \epi(\phi_2))$ is infinite.
\end{enumerate}
In the cases \emph{(i)} and \emph{(ii)} we have that $z \in \mathrm{ext}\,\barr{\frk C}$ (the \emph{extreme points} of the closure of the convex hull defined in \Cref{shp}), and in the case \emph{(iii)} we have $z \in \frk B \setminus \frk C$. Also, if $\frk S = \call H_1 \cap \call H_2$ satisfies
\[
\de \frk S \cap \de\!\bigcup_{j=1,2}\!\epi(\phi_j) = \{ z: H_z\ \text{satisfies \emph{(ii)}} \} = \bigcup_{j=1,2}(\de \call H_j \cap \Gamma(\phi_j)),
\]
then $\call H_j \subset (\mathrm{ext}\,\barr{\frk C} \cap \Gamma(\phi_j))\compl$, for $j=1,2$, in such a way that $\de\frk C \setminus \frk S = \mathrm{ext}\,\barr{\frk C}$. We leave the details to the willing reader.
\end{remark}

\begin{figure}[ht]
\renewcommand\thesubfigure{\emph{(\roman{subfigure})}}
\centering
\subfloat[][\centering The hyperplane touches only one paraboloid.]
	{\scalebox{0.65}{\includegraphics{paraboloidi_tan1.pdf}}} \qquad\quad 
\subfloat[][\centering The hyperplane is tangent to both paraboloids.]
	{\scalebox{0.65}{\includegraphics{paraboloidi_tan2.pdf}}} \qquad\quad
\subfloat[][\centering The hyperplane cuts the other paraboloid.] 
	{\scalebox{0.65}{\includegraphics{paraboloidi_sec.pdf}}} 
\caption{The three situations described in \Cref{parab:cases}.}
\label{fig:parabcases}
\end{figure}


\begin{proof}[Proof of \Cref{shp}]
Set 
\[
e \defeq \frac{v_2 - v_1}{|v_2-v_1|} \quad \text{and} \quad m \defeq \frac{\phi_2(v_2)-\phi_1(v_1)}{|v_2-v_1|}
\]
and consider $z_1\defeq(y_1, \phi_1(y_1))\in\Gamma(\phi_1)$ and $z_2\defeq(y_2, \phi_2(y_2))\in\Gamma(\phi_2)$. We want to determine a condition on $y_1$ and $y_2$ which ensures that $z_1$ and $z_2$ have a common tangent hyperplane $H$. Equating normals $(D\phi_1(y_1), -1)$ and $(D\phi_2(y_2), -1)$ yields 
\[
y_1-v_1 = y_2-v_2.
\]
 Thus $y_1 = v_1 + w$ and $y_2=v_2 + w$ for some $w\in\R^n$, so that $(\frac w r, -1)$ is normal to $H$. Since $z_1, z_2 \in H$ we have
\[
\tfrac1r\pair{y_1}w - \phi_1(y_1) = \tfrac1r\pair{y_2}w - \phi_2(y_2), 
\]
therefore $r(\phi_2(v_2) - \phi_1(v_1)) = \pair{y_2 - y_1}w = \pair{v_2-v_1}w$. Furthermore, since in particular $|y_1- v_1| = |y_2 - v_2|$, we know that $\phi_2(v_2) - \phi_1(v_1) = \phi_2(y_2) - \phi_1(y_1)$, proving that $\pair ew = rm$. Hence, if we decompose $w\in\R^n =  \langle e \rangle \oplus \langle e \rangle^\perp$, then there exists $\barr w\in\langle e \rangle^\perp$ such that 
\[
y_j = v_j+ rme + \barr w
\]
 for $j=1,2$.
Define now $\call H_1$ to be the open half-space whose boundary hyperplane $\de\call H_1$ has interior normal $(e,0)$ and passes through $(v_1+rme, 0)$. Similarly, define $\call H_2$ to have interior normal $(-e, 0)$ and boundary $\de \call H_2$ passing through $(v_2 + rme, 0)$. Clearly $\frk S = \call H_1 \cap \call H_2$ has width $|v_1-v_2|$. Also, the above proves that the mapping $\barr w \mapsto z_j$ parametrizes $\Gamma(\phi_j) \cap \de\call H_j$, for $j=1,2$, and the conclusion follows.
\end{proof}

\begin{figure}[ht]
\centering
	\scalebox{0.96}{\includegraphics{paraboloidi_proof_wF.pdf}}
\caption{The slab constructed in the proof of \Cref{shp}.}
%\label{fig:subfig}
\end{figure}

This lemma turns out to be useful in order to prove a crucial property of the vertex map we are going to define, which is the main tool we wish to introduce.

 Note that it is trivial that the vertex point of a generic upper contact quadratic funciton $\phi$ of radius $r$ for $u$ at $x$, $\phi = u(x) + \pair p{\cdot - x} + \frac1{2r}|\cdot -\, x|^2$, is $v=x-rp$, and if $x\in \mathrm{Diff}^1(u)$, then $p=Du(x)$ is unique. This motivates the following definition.

\begin{definition}
Let $u\in \USC(X)$. The map
\[
V \colon \C(u, X, \tfrac1r\I) \cap \mathrm{Diff}^1(u) \to \R^n, \quad V\defeq \I - rDu,
\]
is called the \emph{upper vertex map} for $u$.
\end{definition}

\begin{remark} \label{rmk:vertexc}
If $u$ is convex, then the upper vertex map is well-defined on the whole $\C(u, X, \tfrac1rI)$, as $ \C(u, X, \frac1rI) \subset \mathrm{Diff}^1(u)$ by \Cref{datucp}.

In particular, if $u$ is smooth, then $D^2 V = \I - rD^2u$, and thus $0\leq D^2V \leq \I$ on $\C(u, X, \frac1r\I)$; that is, $V$ is nonexpansive (or 1-Lipschitz). This can be proved true in general; see the next proposition.
\end{remark} 

\begin{prop} \label{prop:contr}
Given a convex function $u$ defined on an open convex set $X \subset \R^n$, the vertex map $V\colon \C(u, X, \frac1r\I) \to \R^n$ is nonexpansive.
 \end{prop}
 
\begin{proof}
For $j=1,2$, given $x_j \in \C(u, X, \frac1r\I)$, let $\phi_j$ be the upper contact quadratic function of radius $r$ for $u$ at $x_j$ with vertex $v_j \defeq V(x_j)$. With the notation in \Cref{shp}, since $u$ is convex and below the $\phi_j$'s, then $\frk C \subset \epi(u)$. Recall that $\epi$ is the \emph{open} epigraph, so that $z_j\defeq(x_j, \phi_j(x_j)) = (x_j, u(x_j))\in \Gamma(\phi_j) \setminus \epi(u)$; hence $z_j\in\Gamma(\phi_j) \setminus\frk C$. Therefore, by \Cref{shp}, $z_j \notin \call H_j$. It follows that the $z_j$'s lie on opposite sides of $\frk S$, therefore $|x_1 - x_2| \geq |v_1 - v_2|$.
\end{proof}

We also have a condition under which $V(x)$ cannot be too far from $x$; it is a consequence of the following result, which guarantees a wealth of upper contact points for any upper semicontinuous function which is bounded below.

\begin{prop}	\label{p:uvm}
Let $K \subset \R^n$ be compact, $u\in\USC(K)$ bounded below and $\osc_K u \defeq \sup_K u - \inf_K u$ the \emph{oscillation of $u$ on $K$}. Set 
\[
K_\delta \defeq\{y\in K:\ d(y, \de K) > \delta\}, \quad \text{with} \ \delta \defeq \sqrt{2r\osc_K u},
\]
and let $\C(u, K, \tfrac1r\I, v)$ be the set of the points of $K$ at which $u$ has an upper contact quadratic function of radius $r$ and vertex point $v$.
Then, for each $v\in K_\delta$, $\C(u, K, \tfrac1r\I, v)$ is a nonempty and compact subset of $\barr B_\delta(v) \subset K$.
\end{prop}

\begin{proof}
Set 
\[
c\defeq \sup_{y\in X} \left( u(y) - \tfrac1{2r}|y-v|^2 \right);
\]
then $c$ is finite since $u\in\USC(K)$, and $\phi _v= \phi_v(v) + \frac1{2r}|\cdot -\, v|^2$ is an upper contact quadratic function of radius $r$ for $u$ at $x$ with vertex point $v$ if and only if $\phi_v(v) = c$ and the supremum is attained at $x$. Thus,
\[
\C(u, K, \tfrac1r\I, v) = \{x\in K: u(x) = \phi_v(x) \}
\]
is nonempty and compact.  

Since $u$ is bounded below, its oscillation is finite. Suppose $\delta$ so small that $K_\delta$ is not empty and fix $v\in K_\delta$. Clearly, if $x\in\C(u,K,\frac1r\I, v)$, then
\[
\tfrac1{2r}|v-x|^2 = \phi_v(x) - \phi_v(v) = u(x) - c \leq u(x) - u(v) \leq  \osc_K u.
\]
This proves that $\C(u, K, \frac1r\I, v) \subset \barr B_\delta(v)$.
\end{proof}

\subsection{A nonclassic proof of S{\l}odkowski's density estimate}

The tools we presented allow to prove the following ``paraboloidal'' counterpart of \Cref{slodestlem} which one finds in~\cite{hlqc}.

\begin{lem} \label{jen:den}
Let $u$ be a convex function on $\barr B_{\barr\rho}$, such that, for some $R>0$,
\[
0\leq u(y) < \frac{|y|^2}{2R} \quad \text{for $y\neq 0$}.
\]
Then
\[
\frac{|\C(u, \barr B_\rho, \tfrac1r\I)|}{|B_\rho|} > \left(1-\sqrt{\frac rR} \,\right)^n
\]
for all $r \in (0,R\,]$ and $\rho \in (0,\barr\rho\,]$.
\end{lem}

\begin{proof}
Apply \Cref{p:uvm} with $K= \barr B_\rho$, $\rho\leq\barr\rho$, and $r\leq R$. Since by our assumptions $\osc_K u = M(\rho) \defeq \sup_{B_\rho} u$, we have $\delta = \delta(\rho) = \sqrt{2rM(\rho)}$ and for every $v\in \barr B_{\rho-\delta}$, the contact set $\C(u, \barr B_\rho, \frac1r\I, v)$ is a nonempty subset of $B_\rho$. Note that, by convexity, if $x$ is a point in the contact set, then the upper vertex map $V$ is well-defined at $x$. Hence, for every $v\in \barr B_{\rho-\delta}$ we have that $v= V(x)$ for some $x\in \C(u, \barr B_\rho, \frac1r\I, v) \subset \C(u, \barr B_\rho, \frac1r\I)$; that is,
\[
\barr B_{\rho-\delta} \subset V\!\left( \C(u, \barr B_\rho, \tfrac1r\I) \right).
\]
Since $V$ is a contraction, one obtains that
\begin{equation} \label{slodineq1}
|B_{\rho-\delta}| \leq | \C(u, \barr B_\rho, \tfrac1r\I) |.
\end{equation}
One concludes by noting that Bauer's maximum principle (see~\cite{bau}) tells that $M(\rho) = \max_{\de B_\rho} u$, and thus $M(\rho) < \rho^2/(2R)$, whence $\delta < \rho\sqrt{r/R}$. Therefore
\begin{equation} \label{slodineq2}
\rho - \delta > \rho \left(1-\sqrt{\frac rR}\, \right) \geq 0
\end{equation}
and the desired conclusion easily follows.
\end{proof}

On the other hand, a ``paraboloidal'' version of \Cref{slod(iii)} is easy to write.

\begin{lem} \label{hlKb}
Let $U\subset \R^n$ be open and $u \colon U \to \R$ be convex. Suppose there exists a paraboloid of radius $r$ (i.e.~$\barr\epi(\phi)$, with $\phi$ quadratic of radius $r$), which supports $\Gamma(u)$ from the above at $(x,u(x))$. Then
\[
\call K(u,x) \leq \frac{1}{r} \,.
\]
\end{lem}

\begin{proof}
By \Cref{ucjt}({\small\sf D\;at\;UCP}) one knows that $x \in \Diff^1\! u$, and
\[
u(x+\cdot\,) \leq u(x) + \pair{Du(x)}{\cdot\,} + \frac1{2r}|\cdot|^2 \quad \text{near $0$}.
\]
Inequality $\call K(u,x) \leq \frac1r$ then immediately follows from the definition of $\call K$.
\end{proof}

These two instruments provided by Harvey and Lawson's approach are sufficient in order to prove S{\l}odkowski's density estimate following the same reasoning of S{\l}odkowski.

\begin{proof}[Proof of \Cref{slodthm}]
Without loss of generality, suppose that $x^* = 0$, $u(0) = 0$ and $Du(0) = 0$; this yields in particular $u \geq 0$ since $u$ is convex. Take $k^* < K < k$, and note that by the definition of $\call K$ there exists some $\bar\rho > 0$ such that $u < \frac K2 |\cdot|^2$ on $B_{\bar\rho}$.
Now, for any $r > 0$ such that $k^{-1} < r < K^{-1}$, by \Cref{hlKb}
\[
\C(u,\bar B_\rho, \tfrac 1r I) \subset \{ x :\ \call K(u,x) < k \} \qquad \forall \rho \in (0,\bar\rho);
\]
hence by \Cref{jen:den}, with $R = K^{-1}$,
\[
\delta^-_{0}(\{ \call K(u,\cdot) < k \}) \geq \liminf_{\rho \dto 0} \frac{|\C(u,\bar B_\rho, \frac1r I)|}{|B_\rho|} \geq \left( 1-\sqrt{\frac{r}{R}} \right)^n.
\]
Letting $K \dto k^*$ (that is, $R \uto (k^*)^{-1}$) and $r \dto k^{-1}$, this yields
\[
\delta^-_{0}(\{ \call K(u,\cdot) < k \}) \geq \bigg( 1-\sqrt{\frac{k^*}{k}} \,\bigg)^n = \left({\frac{k-k^*}{\sqrt{k}(\sqrt{k} + \sqrt{k^*})}} \right)^n \geq \left(\frac{k-k^*}{2k}\right)^n,
\]
as desired.
\end{proof}

\subsection{The Jensen--S{\l}odkowski theorem}

From \Cref{jen:den} one deduces immediately the following, which is to be found again in~\cite{hlqc}.

\begin{lem}[HL--S{\l}odkowski lemma] \label{slod}
Suppose that $u$ is a locally convex function on $X\subset \R^n$ with strict upper contact jet $(0,\lambda \I)$ at $x$. Then there exists $\barr\rho > 0$ such that
\[
|\C(u,\barr B_\rho(x),\lambda \I)| > 0 \qquad \forall \rho \in (0,\barr\rho\,].
\]
\end{lem}

\begin{proof}
It is straightforward to see that $\lambda>0$ by convexity, unless $u$ is constant near $x$, which happens if $\lambda = 0$ and for which the thesis is trivial. Hence  we may assume that $\lambda >0$ and, without loss of generality, that $x=0$ and $u(x) = 0$. Then the hypothesis of $(0, \lambda\I)$ being a strict upper contact jet for $u$ at $x=0$ is equivalent to the statement that there exists $\barr\rho >0$ such that
\[
0\leq u(y) < \frac1{2R}|y|^2 \qquad \text{for}\ 0<|y|\leq \barr\rho,
\]
where $R=1/\lambda$. Indeed, if $(0, \lambda\I)$ is a strict upper contact jet at $0$, then by \Cref{datucp} of differentiability at upper contact points, $Du(0) = 0$ and by convexity $u \geq 0$; the converse implication is trivial. The thesis now follows from \Cref{jen:den} with $r = R$.
\end{proof}


This lemma assures that, if $u$ is convex, then the set of contact points for some global upper contact quadratic function of radius $r$ on a ball $B_\rho(x)$ cannot be too small (in the sense of the Lebesgue measure), provided that $x$ is the vertex point of a strict upper contact quadratic function of radius $r$ for $u$ at $x$ itself and $\rho$ is sufficiently small. Indeed, the requirement that $x$ is both a contact point of such a quadratic function and its vertex point is equivalent to $(0, \lambda I)$ being a strict upper contact jet for $u$ at $x$.

Furthermore, note that the HL--S{\l}odkowski lemma can be paraphrased by saying that if $u$ is locally convex and its quadratic perturbation $u-\frac{\lambda}{2}|\cdot-\,x|^2$ has a strict local maximum at $x$, then the set of global upper contact points of type $\lambda I$ for $u$ on each small ball about $x$ has positive measure. Analogously, Jensen's \Cref{u:jen} considers a quasi-convex function with a strict local maximum at $\hat x$ and states that the set of the upper contact points near $\hat x$ whose associated upper contact jets are of the form $(p, 0)$, with $p \in \R^n$ small ($|p| \leq \delta$), has positive measure. 

With the intuition that a bridge between convex and quasi-convex function can relate the two lemmas, Harvey and Lawson~\cite{hlqc} prove that the HL--S{\l}odkowski lemma is equivalent to the following, which Harvey and Lawson call \emph{Jensen's lemma}, roughly stating that if $u$ is locally quasi-convex, then the set of contact points of locally supporting hyperplanes from above near a strict local maximum point cannot be too small.

\begin{lem}[HL--Jensen lemma]  \label{hl:jen}
Suppose that $w$ is a locally quasi-convex function on $X\subset \R^n$ with strict upper contact jet $(0, 0)$ at $x$. Then there exists $\barr\rho > 0$ such that
\[
|\C(w,\barr B_\rho(x),0)| > 0	\qquad  \forall \rho \in (0,\barr\rho\,].
\]
\end{lem}

\begin{claim}
The HL--S{\l}odkowski and the HL--Jensen lemmas are equivalent.
\end{claim}

\begin{proof}
Consider $u$ and $w$ related by the identity $u = w + \frac\lambda 2|\cdot -\, x|^2$. Then, on $B_\rho(x)$, $u$ is convex if and only if $w$ is $\lambda$-quasi-convex; note that by~\cref{def:qc}, the parameter of quasi-convexity $\lambda(x)$ is locally constant, therefore we may assume that the locally quasi-convex function $w$ is $\lambda$-quasi-convex on $B_\rho(x)$ if $\rho>0$ small enough. Furthermore, $(0,\lambda\I)$ is a strict upper contact jet for $u$ at $x$ if and only if $(0,0)$ is for $w$. This shows that $\C(w,\barr B_\rho(x),0) = \C(u,\barr B_\rho(x),\lambda \I)$, and completes the proof of the desired equivalence.
\end{proof}

We find interesting to clarify why the HL--Jensen lemma is indeed a reformulation of Jensen's \Cref{u:jen}, thus motivating its name. In view of our discussion above, note that one only needs to show that \emph{all} global upper contact jets of type $0 \in \Sc(n)$ for $w$ whose associated upper contact points are close enough to $x$ are of the form $(p, 0)$ with $|p|\leq \delta$.

\begin{claim} \label{cl:j-hlj}
Jensen's and the HL--Jensen lemmas are equivalent.
\end{claim}

\begin{proof}
Note that the request that $w$ has a strict local maximum at $\hat x$ is equivalent to the request that $(0,0)$ is a strict upper contact jet for $w$ at $\hat x$, which for $w$ $\lambda$-quasi-convex is equivalent to the requirement that $(0, \lambda I)$ is a strict upper contact jet for $u \defeq w + \frac\lambda2|\cdot-\,\hat x|^2$ at $\hat x$. Furthermore, $w_p$ has a local maximum at $x$ if and only if $(-p, 0)$ is an upper contact jet for $w$ at $x$; hence
\[
K = \bigcup_{|p| \leq \delta} \C(w,\barr B_r(\hat x),0\,;\, p),
\]
where the set $\C(w,\barr B_r(\hat x),0\,;\, p)$ denotes the set of the global upper contact points on $\barr B_r(\hat x)$ for $w$ associated to the jet $(p,0)$.
It is easy to see that $\C(w, \barr B_r(\hat x), 0\,;\, p)$ is the set of those points $x\in \barr B_r(\hat x)$ such that $(p + \lambda(x-\hat x), \lambda I)$ is an upper global upper contact jet for $u$ at $x$. By~\Cref{datucp}, $Du(x) = p + \lambda(x-\hat x)$ and thus
\[
V(x) = \hat x - \frac p\lambda \quad \implies\quad  |p| = \lambda\,|V(x) - \hat x|,
\]
where $V\colon \C(u, \barr B_r(\hat x), \lambda I) \to \R^n$ is the upper vertex map associated to $u$.
Since $V$ is nonexpansive (\Cref{prop:contr}) and $V(\hat x) = \hat x$,  we see that
\[
|V(x) - \hat x| \leq |V(x) - V(\hat x)| + |V(\hat x)- \hat x| \leq |x-\hat x|,
\]
whence $|p| \leq \lambda r$. Therefore  we have, for $r \leq \delta/\lambda$,
\begin{equation*} \label{fatica}
K = \bigcup_{p \in \R^n} \C(w,\barr B_r(\hat x),0\,;\, p) = \C(w,\barr B_r(\hat x), 0).\qedhere
\end{equation*}
\end{proof}

\begin{remark}
We implicitly assumed that in the statement of Jensen's \Cref{u:jen}, by \emph{local} one means \emph{global on $\barr B_r(\hat x)$}. In this case, the thesis of Jensen's lemma holds for every $\delta >0$ and sufficiently small $r>0$, and the statements of Jensen's and HL--Jensen lemmas are equivalent; if instead one requires that the thesis of Jensen's lemma must hold for all $r>0$, then, in the definition of $K$, $w_p$ having a local maximum at $x$ means that there exists $\eta=\eta(x)>0$ such that $x\in \C(w, \barr B_\eta(x), 0)$, so that the set $K$ is \emph{a priori} larger than $\C(w, \barr B_\rho(\hat x), 0)$ for any $\rho\leq r$. 
\end{remark}

\begin{remark}
The proof of \Cref{cl:j-hlj} also competes an alternate proof of Jensen's \Cref{u:jen} which exploits Harvey and Lawson's strategy of focusing on upper contact quadratic functions.
\end{remark}

One can then now merge the two equivalent Lemmas~\ref{slod} and \ref{hl:jen} into a theorem which, despite being another equivalent reformulation of those, unveils their full generality. This is also done in~\cite{hlqc}.

\begin{thm}[Jensen--S{\l}odkowski Theorem] \label{jenslod}
Suppose that $w$ is a locally quasi-convex function with strict upper contact jet $(p, A)$ at $x$. Then there exists $\barr\rho > 0$ such that
\[
|\C(w,\barr B_\rho(x),A)| > 0	\qquad  \forall \rho \in (0,\barr\rho\,].
\]
\end{thm}

\begin{proof}

Take 
\[
\phi \defeq -\pair p{\cdot - x} - Q_A(\,\cdot - x) + \frac\lambda2 |\cdot-\,x|^2,
\]
 for some $\lambda >0$ to be determined. It is easy to see that $(0, \lambda\I)$ is a strict upper contact jet for $u\defeq w+\phi$ at $x$, and $\C(w, \barr B_\rho(x), A) = \C(u, \barr B_\rho(x), \lambda\I)$. Moreover, let $\alpha, \beta >0$ such that $w$ is $\alpha$-quasi-convex and $A\leq \beta\I$; then $u$ is convex if $\lambda \geq \alpha + \beta$. The thesis now follows from \Cref{slod}.
 \end{proof}
 
 \begin{remark}
 As we proved that the HL--S{\l}odkowski lemma implies the Jensen--S{\l}odkowski theorem, while it is clear that the converse is also true, one sees that actually the two results (and the HL--Jensen lemma as well) are in fact equivalent.
 \end{remark}
 
 \subsection{Consequences} \label{sec:proof:pusc}
 
 The main consequence of the Jensen--S{\l}odkowski theorem (\Cref{jenslod}) we are interested in is, as we anticipated, the property of partial upper semicontinuity of second derivatives of a locally quasi-convex function ({\small\sf PUSC\;of\;SD}, \Cref{pusc}), which we can now prove, thus competing the proof of the Upper Contact Jet Theorem (\Cref{ucjt}).
 
\begin{proof}[Proof of \Cref{pusc}]
Given a sequence $\epsilon_k\dto 0$ of positive numbers, pick 
\[
x_k \in E \cap \Diff^2(u) \cap \C(u, \barr B_{\epsilon_k}(x), A+ \epsilon_k I).
\]
Note that this is certainly possible for every sufficiently small $\epsilon>0$ since $E \cap \Diff^2(u)$ has full measure in a neighborhood of $x$ by Alexandrov's \Cref{aleks:qc} and $(p,A+\epsilon I)$ is a strict upper contact jet for $u$ at $x$, therefore $\C(u, \barr B_{\epsilon}(x), A+ \epsilon I)$ has positive measure for $\epsilon$ small by the Jensen--S{\l}odkowski Theorem~\ref{jenslod}.
We know that since $u$ is twice differentiable at $x_k$, then $D^2u(x_k) \leq A+\epsilon_k I$. Furthermore, by quasi-convexity, $D^2u(x_k) + \lambda I \geq 0$ for some $\lambda\geq0$. By compactness, there exists a subsequence $\{x_j\}$ such that $D^2u(x_j) \to \bar A$ for some $\bar A \in [-\lambda I, A]$.
\end{proof} 

We also wish to note here that, as the HL--S{\l}odkowski and the HL--Jensen lemmas are equivalent, despite dealing with convex and quasi-convex functions, respectively, one should expect results coming from the former lemma to have a counterpart holding for quasi-convex functions as well. This is the case, for instance, of S{\l}odkowski's LET (\Cref{slodle}). In fact, if $u$ is $\lambda$-quasi-convex near $x^*$ with $\call K(u,x) = k^*$, then $\tilde u \defeq u + \frac\lambda2|\cdot|^2$ is convex with $\call K(\tilde u, x^*) = \call K(u,x^*) + \lambda = k^* + \lambda$. Then \Cref{slodthm} tells that, for $k > k^*$,
\[
\delta^-_{x^*}(\{ \call K(u, \cdot) < k \}) = \delta^-_{x^*}(\{ \call K(\tilde u, \cdot) < k + \lambda \}) \geq \left(\frac{k-k^*}{2(k+\lambda)}\right)^n.
\]
Exploiting this density estimate for quasi-convex functions one can deduce a quasi-convex S{\l}odkowski's LET. Alternatively, it is possible to prove it directly from the Jensen--S{\l}odkowski \Cref{jenslod}, as we do here.
 
 \begin{proof}[Proof of \Cref{slodle}]
Seeking for a contradiction, suppose that there exists $\hat x\in X$ such that $K(u, \hat x) < M$; say $K(u, \hat x) = M - 3\delta$ for some $\delta > 0$. Therefore there exists $\barr\epsilon >0$ such that $\barr B_{\barr\epsilon}(\hat x) \subset X$ we have
\[
2 \epsilon^{-2} \displaystyle\max_{|h|=1} \big( u(\hat x+\epsilon h) - u(\hat x) - \epsilon\pair{Du(\hat x)}{h} \big) \leq M-2\delta \qquad \forall\, \epsilon \in (0, \barr \epsilon),
\]
which is equivalent to
\[
u(\hat x + z) \leq  u(\hat x) + \pair{Du(\hat x)}{z} + \tfrac12(M-2\delta)|z|^2 \qquad \forall\, z\in B_{\barr\epsilon}(0);
\]
that is, it is equivalent to $(Du(\hat x), (M-2\delta)I)$ being an upper contact jet for $u$ at $\hat x$. Hence, $(Du(\hat x), (M-\delta)I)$ is strict and by~\Cref{jenslod} we have 
\[
|\C(u, \barr B_\rho(\hat x), (M-\delta)I)| > 0
\]
 for some $\rho > 0$. Since by definition
\[
u(y) \leq u(x) + \pair{Du(x)}{y-x} + \tfrac12(M-\delta)|y-x|^2
\]
for all $x\in \C(u, \barr B_\rho(\hat x), (M-\delta)I)$ and every $y \in \barr B_\rho(\hat x)$, we see that this implies that $\C(u, \barr B_\rho(\hat x), (M-\delta)I) \subset \{ K(u,\cdot) \geq M \}\compl$, thus contradicting the hypothesis that $K(u,\cdot) \geq M$ almost everywhere.
 \end{proof}

\subsection{An alternate proof of the Jensen--S{\l}odkowski theorem}

We find particularly interesting to conclude this section by pointing out the following fact observed by Harvey~\cite{har:pc}: the Jensen--S{\l}odkowski theorem can be seen as an immediate consequence of Alexandrov's maximum principle, a short proof of which can be in turn built around the well-known area formula (see \cite[Area Theorem~3.2.3]{fed:geo}).  Therefore, if one also considers that the Lipschitz version of Sard's theorem we used in the proof of Alexandrov's theorem can be seen as a corollary of the area formula as well (cf.~\cite[Notes on the appendix]{user}),\footnote{The proof of Sard's theorem we propose in the \nameref{proofsard}, based on Besicovitch's covering theorem, could seem at first sight more involute than just invoking the area formula, yet the reader should be aware that classical proofs of the area formula itself (see, e.g.~\cite{evansgar}) rely, among other instruments, on the Radon--Nikodym theorem, which in turn also requires Besicovitch's result.} then the Area Theorem arises as the pillar of the theory we discussed so far.

We recall the following definitions of \emph{approximate limit superior} and \emph{approximate differential} (for further details, see \cite[p.\ 212]{fed:geo}). 

\begin{definition}
Let $g$ be a measurable function in a neighborhood of a point $x$. Define
\[
\aplimsup_{y\to x} g(y) \defeq \inf \left\{ t \in \R\ :\ \lim_{r \dto 0} \frac{|g^{-1}((t,+\infty)) \cap B_r(x)|}{|B_r(x)|}= 0 \right\}.
\]
\end{definition}

\begin{definition}
Let $f\colon A \to \R^n$ with $A$ and $f$ measurable, $A\subset \R^n$. The \emph{approximate differential} of $f$ at $a \in A$ is the unique linear map $\mathrm{ap}\,Df(a) \colon \R^n \to \R^n$ such that 
\[
\aplim_{x\to a} \frac{|f(x) - f(a) - \mathrm{ap}\, Df(a) (x-a)|}{|x-a|} = 0.
\]
\end{definition}

If we let $f$ be as above and define
\[ 
S(f) \defeq \left\{ x \in A:\ \aplimsup_{y\to x} \frac{|f(y) - f(x)|}{|y-x|} < \infty \right\},
\]
then by H.~Federer's extension of W.~Stepanoff's theorem \cite[Lemma~3.1.7 and Theorem~3.1.8]{fed:geo}, $f$ has an approximate differential almost everywhere in $S(f)$; also, as noted in the beginning of~\cite[Section~3.2]{fed:geo}, the Area Theorem can be restated as follows.

\begin{thm}[Area Theorem] \label{area}
For each measurable subset $E\subset S(f)$,
\[
|f(E)| \leq \int_{\R^n} N(\restr{f}{E}, y)\, \di y = \int_E |\det \mathrm{ap}\,Df(x)|\, \di x,
\]
where $N(\restr{f}{E}, y) \equiv \#\{x\in E:\ f(x) = y\}$.
\end{thm}

Let now $X$ be an open subset of $\R^n$ and let $u\colon X \to \R$ be locally quasi-convex; consider $Du \colon \Diff^1(u) \to \R^n$. We know that both $\Diff^1(u)$ and $\Diff^2(u)$ have full measure in $X$, and the following result is immediate.

\begin{lem}
Let $u$ be as above. Then $\Diff^2(u) \subset S(Du)$.
\end{lem}

\begin{proof}
Let $f\defeq Du$. Fix $x\in \Diff^2(u)$ and set
\[
E_t \defeq \left\{ y\in \Diff^1(u)\ :\  g(y)\defeq\frac{|f(y) - f(x)|}{|y-x|} > t \right\}.
\]
Since $|Du(y) - Du(x)| \leq |D^2u(x)|\,|y-x| + o(|y-x|)$ for $y \to x$, let $t>|D^2u(x)|$ and $\epsilon > 0$ such that $g(y) \leq t$ for all $y\in B_\epsilon(x)\cap \Diff^1(u)$. Hence $|E_t \cap B_\epsilon(x)| = 0$,  yielding $\aplimsup_{y\to x} g(y) \leq |D^2u(x)|$ and the desired conclusion follows.
\end{proof}

Hence we have the following version of the Area Theorem, which is more suitable for our purposes.

\begin{thm}[Area Theorem for gradients of quasi-convex functions] \label{area:cor}
Assume that $E\subset \Diff^2(u)$ is measurable and set $\Sigma \defeq Du(E)$. Then
\[
|\Sigma| \leq \int_E |\det D^2u(x)|\, \di x.
\]
\end{thm}

We now show that the proof of a version of Alexandrov's maximum principle for quasi-convex functions amounts to finding a ball $B_\delta \subset \Sigma$, or, more generally, contained in a smaller set than $\Sigma$ in the sense of the measure. Note that if one defines
\begin{equation} \label{alex:delta}
\delta \defeq \frac{\sup_{\overline\Omega} u - \sup_{\de\Omega} u}{\mathrm{diam}(\Omega)}\,,
\end{equation}
then, for any nonnegative constant $R$,
\[
|B_\delta| \leq R \iff \sup_{\overline\Omega} u \leq \sup_{\de\Omega} u + \frac{\mathrm{diam}(\Omega)}{\omega_n^{1/n}} R^{1/n},
\]
so that if one has
\begin{equation} \label{alexball}
|B_\delta| \leq \int_{E} |\det D^2u(x) |\,\di x
\end{equation}
one gets
\begin{equation} \label{alexineq}
\sup_{\overline\Omega} u \leq \sup_{\de\Omega} u + \frac{\mathrm{diam}(\Omega)}{\omega_n^{1/n}} \Bigg(\int_{E} |\det D^2u(x)|\,\di x\Bigg)^{1/n}.
\end{equation}
Precisely, we will consider $E = \C(u, \Omega, 0) \cap \Diff^2(u)$ (that is, the set of flat global upper contact points for $u$ on $\Omega$ at which $u$ is twice differentiable), thus proving the following.

\begin{thm}[Alexandrov's maximum principle for quasi-convex functions] \label{alexmp}
Let $\Omega \subset \R^n$ be open and bounded, and let $u\in\USC(\overline\Omega)$ be locally quasi-convex. Then inequality~\eqref{alexineq} (or, equivalently, \eqref{alexball}) holds with $E = \C(u, \Omega, 0) \cap \Diff^2(u)$.
\end{thm}

This immediately follows from the argument above and the next two lemmas, where we let $\tilde E \defeq \C(u, \Omega, 0) \subset \Diff^1(u)$, the inclusion coming from property ({\small\sf D\;at\;UCP}) of the Upper Contact Jet \Cref{ucjt}.

\begin{lem}
Let $u$ be as above. Then $Du(\tilde E) \setminus Du(E)$ is null.
\end{lem}

\begin{proof}
We show that the larger set $Du(\tilde E \setminus E)$ is null. First, note that $\tilde E\setminus E$ is null by Alexandrov's \Cref{aleks:qc}. Then consider an exhaustion of $\Omega$ by compact sets  $\{K_n\}_{n\in\N}$, so that $u$ is $\lambda_n$-quasi-convex on $K_n$. Set $E^{{(n)}} \defeq \tilde E \cap K_n \subset \C(u, K_n, 0)$ and $u_n \defeq u + \frac{\lambda_n}2|\cdot|^2$. Since $\C(u, K_n, 0) = \C(u_n, K_n, \lambda_n\I)$, we know that the upper vertex map for $u_n$ is well-defined and $1$-Lipschitz on $E^{{(n)}}$ (by \Cref{rmk:vertexc} and \Cref{prop:contr}). This implies that $Du_n$ is $2\lambda_n$-Lipschitz, and thus $Du$ is $3\lambda_n$-Lipschitz, on $E^{{(n)}}$. We conclude that $Du(E^{{(n)}} \setminus E)$ is null, for every $n\in\N$, as Lipschitz functions map null sets into null sets (see, e.g., \cite[Theorem~2.8(i)]{evansgar}). The claim now follows, for instance, by the subadditivity of the Lebesgue measure.
\end{proof}

\begin{lem}
Let $\delta$ be as in \eqref{alex:delta}. Then $B_\delta \subset Du(\tilde E)$.
\end{lem}

\begin{proof}
Fix $p\in B_\delta$ and set $k_0 \defeq \inf\{k\in \R :\ k+\pair py \geq u(y)\ \forall y \in \overline\Omega \}$. Note that it is well-defined as a real number because of the hypotheses in Alexandrov's maximum principle (\Cref{alexmp}). Then, by the definition of infimum,  there exists a point $x\in\overline\Omega$ such that $k_0 + \pair px = u(x)$, thus 
\begin{equation} \label{ucp:l2alex}
u(y) \leq u(x) + \pair p{y-x} \quad \forall \, y\in\overline\Omega.
\end{equation}
Since such a point $x$ is an upper contact point for $u$ on $\Omega$, by \Cref{ucjt}({\small\sf D\;at\;UCP}) we know that $p\in Du(E)$. To see that $x \in \Omega$, suppose that $x\in\de\Omega$ and pick $y\in\overline\Omega$ with $u(y) = \sup_{\overline\Omega} u$; then from \eqref{ucp:l2alex} one has
\[
\sup_{\overline\Omega} u - \sup_{\de\Omega} u \leq \pair p{y-x} \leq |p|\, \mathrm{diam}(\overline\Omega),
\]
which cannot hold for $p\in B_\delta$ by the definition of $\delta$.
\end{proof}

A proof of Jensen--S{\l}odkowski theorem using Alexandrov's maximum principle now proceed as follows.

\begin{proof}[Alternate proof of \Cref{jenslod}]
We consider now the HL--Jensen \Cref{hl:jen} as an equivalent formulation of the Jensen--S{\l}odkowski theorem. By (\ref{alexball}) with $\Omega =  B_\rho(x)$ and $\rho$ so small that the radius $\delta$ defined by~(\ref{alex:delta}) is positive, one has
\[
0 < |B_\delta| \leq \int_{E_2} |\det D^2 u(y)|\, \di y,
\]
which forces $\C(u, \barr B_\rho(x), 0)$ to have positive measure. 
\end{proof}

The ``paraboloidal'' proof of the Jensen--S{\l}odkowski theorem passed through a sort of density estimate for a certain set of global upper contact points for a convex function (\Cref{jen:den}), likewise we can deduce a similar estimate, for the set of flat global upper contact points for a quasi-convex function on small balls around a strict local maximum point, using Alexandrov's maximum principle.

\begin{cor}
Let $u \colon X \to \R^n$ be $\lambda$-quasi-convex, with strict upper contact jet $(0,0)$ at $x$. Then there exists $\bar\rho > 0$ such that
\[
\frac{|\C(u, \barr B_\rho(x), 0)|}{|B_\rho(x)|} \geq  \left(\frac{\sup_{B_\rho(x)} u - \sup_{\de B_\rho(x)} u }{2\lambda\rho^2}\right)^n \qquad \forall \, \rho \in (0,\bar\rho\,].
\]
\end{cor}

\begin{proof}
Consider $y\in E_2$; by the quasi-convexity of $u$ we have $D^2u(y) \geq - \lambda \I$, for some $\lambda>0$, and since $y$ is a maximum point of $u + \pair p\cdot$ for some $p\in \R^n$, we have $D^2u(y) \leq 0$. Hence $|\det D^2 u| \leq \lambda^n$ on $E_2$, and, letting $\delta$ be as in \eqref{alex:delta}, \Cref{alexmp} with $\Omega = B_\rho(x)$ (along with \Cref{aleks:qc}) yields $|B_\delta| \leq \lambda^n |\tilde E|$, with $\tilde E = \C(u, \barr B_\rho(x), 0)$. The desired conclusion now is obtained by noting that
\[
 |B_\delta|= \left(\frac{\sup_{B_\rho(x)} u - \sup_{\de B_\rho(x)} u }{2\rho^2}\right)^n |B_\rho(x)|. \qedhere
\]
\end{proof}



\section{Quasi-convex approximation of USC functions} \label{sec:apqc}

\subsection{Elementary properties of semicontinuous functions}

Let us recall here the definition and some well-known and useful properties of upper semicontinuous functions. Dual definitions and properties, which we decide to omit, can be straightforwardly given for \emph{lower} semicontinuous function as well.

\begin{definition}
Let $X \subset \R^n$. We say that a function $u \colon X \to (-\infty, +\infty]$ is \emph{upper semicontinuous at $x_0 \in X$} if
\[
\text{$\{ x \in X :\ u(x) < \alpha \}$ \ is a neighborhood of $x_0$ \quad for each $\alpha > u(x_0)$};
\]
that is, if for any $\alpha \in \R \cup \{+\infty\}$ such that $u(x_0) < \alpha$ there exists \,$\call U$ open, $x_0 \in \call U$, such that $u(y) < \alpha$ for all $y \in \call U \cap X$.

We say that $u$ is upper semicontinuous \emph{on $X$}, and we write $u \in \USC(X)$, if
\[
\text{$\{ x \in X :\ u(x) < \alpha \}$ \ is open \quad $\forall \alpha \in \R$}.
\]
\end{definition}

\begin{remark} \label{usc:equivdef}
Other equivalent definitions of upper semicontinuity are the following, as it is easy to see: let $u$ be as above; then
\begin{enumerate}[label=\it(\roman*)]
\item $u$ is upper semicontinuous at $x_0$ if and only if
\[
\limsup_{X \ni x \to x_0} f(x) \leq f(x_0);\footnote{This characterization requires $x_0$ to be a limit point for $X$.}
\]
\item $u$ is upper semicontinuous on $X$ if and only if its hypograph
\[
\mathrm{hypo}(u) \defeq \{ (x,t) \in X \times \R :\ u(x) \geq t \} = \{ (x,t) \in X \times \R : \ (x,-t) \in \epi(-u) \}
\]
is closed (with respect to the subspace topology of $X\times \R$).
\end{enumerate}
\end{remark}

\begin{remark}
Each upper semicontinuous function $u \colon X \to \R$ admits a natural upper semicontinuous extension to the whole space $\R^n$, given by
\[
\tilde u(x) \defeq \begin{cases}
u(x) & \text{on $X$} \\
\sup_X u & \text{outside $X$}\,,
\end{cases}
\]
where possibly $\sup_X u = +\infty$.
\end{remark}

\begin{remark}
By the former equivalent definition recalled in \Cref{usc:equivdef}, one sees that the sum of two upper semicontinuous functions is still upper semicontinuous. Thanks to the latter definition, instead, it is immediate to deduce the following stability property of the family of all upper semicontinuous functions on $X$: let $\{ f_i \}_{i\in I} \subset \USC(X)$, then the pointwise infimum
\[
\inf_{i\in I} f_i \in \USC(X), \quad \text{no matter the cardinality of $I$},
\]
while the pointwise supremum
\[
\sup_{i \in I} f_i = \max_{i \in I} f_i \in \USC(X) \quad \text{provided that $I$ if finite}.
\]
This follows by noticing that $\mathrm{hypo}\big(\inf_{i\in I} f_i\big) = \bigcap_{i\in I} \mathrm{hypo}(f_i)$ is closed for any $I$, while one has that $\mathrm{hypo}\big(\sup_{i\in I} f_i\big) = \bigcup_{i\in I} \mathrm{hypo}(f_i)$ is closed as well, provided that $I$ is finite.
\end{remark}

It is also easy to see that, by definition, a function that is both upper and lower semicontinuous is in fact continuous. Hence one may expect upper semicontinuous function to enjoy ``half'' the property of obtaining their extreme values on compact sets guaranteed by Weierstra\ss's theorem to continuous functions. The first part of the following proposition states precisely this fact, and the second part is an immediate consequence.

\begin{prop} \label{uscweier}
Let $u \in \USC(\R^n)$. The following hold:
\begin{enumerate}[label=\it(\roman*)]
\item suppose that $K$ is compact; then the supremum $\sup_K u$ is attained; that is, there exists $\bar x \in K$ such that
\[
\sup_K u = \max_K u = u(\bar x);
\]
\item suppose that $u$ is \emph{anti-coercive}; that is, $u(x) \to - \infty$ as $|x| \to \infty$; then the supremum $\sup_{\R^n} u$ is attained; that is, there exists $\bar x \in \R^n$ such that
\[
\sup_{\R^n} u = \max_{\R^n} u = u(\bar x).
\]
\end{enumerate}
\end{prop}

\begin{proof}
\underline{\it(i)}. Let $\{ x_n \}_{n\in\N} \subset K$ be a sequence such that
\begin{equation} \label{uscmax1}
u(x_n) \to \sup_K u =\vcentcolon M \quad \text{as $n\to\infty$}.
\end{equation}
Since $K$ is sequentially compact, there exists a point $\bar x \in K$ and a subsequence $x_{n_k} \to \bar x$. By the upper semicontinuity of $u$, one has
\begin{equation} \label{uscmax2}
\limsup_{k\to\infty} u(x_{n_k}) \leq u(\bar x),
\end{equation}
thus combining \eqref{uscmax1} and \eqref{uscmax2} yields $M \leq u(\bar x)$, forcing $u(\bar x) = M$ since $\bar x \in K$ implies $u(\bar x) \leq M$.

\noindent\underline{\it(ii)}. Let $M \defeq \sup_{\R^n} u$. By the anti-coercivity of $u$, there exists an open ball $B$, centered at the origin, such that $u|_{B\compl} < M$. Therefore $\sup_{\R^n} u = \sup_{\barr B} u$, and the conclusion follows from the previous point.
\end{proof}

\subsection{Quasi-convex approximation via supconvolution} \label{ch:qca}

One point making quasi-convex functions so interesting in this theory is that an upper semicontinuous $u$ function which is bounded above can be approximated by quasi-convex functions (for instance, pointwisely or locally uniformly,  depending on the regularity of the function; for further details, see e.g.~\cite{lioham, crankoc, hldir09}); such approximating quasi-convex functions are the \emph{supconvolutions} of $u$, defined as follows.

\begin{definition} \label{def:supconv}
Let $X \subset \R^n$ be open, and let $u \in \USC(X)$ be bounded above; fix $\epsilon >0$. We call $\epsilon$-\emph{supconvolution} of $u$ the function
\begin{equation} \label{supconv}
u^\epsilon \defeq \sup_{y \in X} \Big( u(y) - \frac1{2\epsilon} |y-\,\cdot|^2 \Big), \quad \text{on} \ X.
\end{equation}
\end{definition}

Before proving the results we need, here are four noteworthy facts.

\begin{remark}
One can visualize how a function is transformed by the supconvolution operation by noticing that $u^\epsilon$ is the upper envelope of all quadratic functions of radius $-2\epsilon$ with vertex at some point of the graph of $u$.\footnote{By \emph{upper envelope} of a family of functions $\scr F$ we mean the function $g \defeq \sup_{f \in \scr F} f$. Also note that a quadratic function with negative radius is well-defined: it just opens downwards.} Also, notice that, even if we decided to ``preserve'' the domain of $u$ in \Cref{def:supconv} by defining its supconvolution only on $X$, $u^\epsilon$ is in fact well-defined by \eqref{supconv} on the whole space $\R^n$.
\end{remark}

\begin{remark}
We use the name \emph{convolution} for $u^\epsilon$ defined above because it is obtained by a transformation which is an actual convolution. Indeed, if we consider the functions to be valued in the so-called \emph{tropical semiring} $(\R \cup \{-\infty\}, \vee, +)$, we see that $u^\epsilon = u \ast (-\frac1{2\epsilon}|\cdot|^2)$.
\end{remark}

\begin{remark}
In \Cref{baspropsc} below, we prove that supconvolutions are quasi-convex functions. In the obvious manner, one can also define the \emph{inf-convolution} and prove that it is quasi-concave. Moreover, it is not difficult to see that if $\epsilon$ is sufficiently small, then the $\epsilon$-supconvolution (resp.\ $\epsilon$-inf-convolution) of a quasi-concave (resp.\ quasi-convex) function remains quasi-concave (resp.\ quasi-convex). This fact, along with the characterization given in \Cref{charc11},  proves a part of a well-known result of J.~M.~Lasry and Lions~\cite{laslio} on \emph{sup-inf-convolutions} being $C^{1,1}$.
\end{remark}

Four important properties of the supconvolution are gathered in the following theorem. We have already partially revealed the first one.

\begin{thm} \label{baspropsc}
Let $u \in \USC(X)$ be bounded above, let $u^\epsilon$ be its $\epsilon$-supconvolution. The following hold:
\begin{enumerate}[label=\it(\roman*)]
\item	$u^\epsilon$ is $\frac1\epsilon$-quasi-convex;
\item	the supremum in (\ref{supconv}) is attained; that is, for every $x \in \R^n$ there exists $\xi = \xi(\epsilon,x) \in \R^n$ such that $u^\epsilon(x) = u(\xi) - \frac1{2\epsilon}|\xi-x|^2$;
\item	$u^\epsilon$  decreases pointwise to $u$ as $\epsilon\dto 0$; also, if $x$ is a local maximum point for $u$, then $u^\epsilon(x) = u(x)$ for any $\epsilon$ sufficiently small;
\item	if $u$ is also bounded from below, and thus $|u| \leq M$ for some $M>0$, then
\begin{equation} \label{supcball}
u^\epsilon = \max_{z \in \barr B_\delta} \Big( u(\cdot-z) - \frac1{2\epsilon} |z|^2 \Big), \quad \text{where $\delta \defeq 2\sqrt{\epsilon M}$},
\end{equation}
on $X_\delta \defeq \{ x \in X :\ d(x,\de X) > \delta \}$.
\end{enumerate}
\end{thm}

\begin{proof}
\underline{\emph{(i)}}. It suffices to note that
\[
u^\epsilon + \frac1{2\epsilon}|\cdot|^2 = \sup_{y \in \R^n} \bigg( u(y) -  \frac1{2\epsilon}|y|^2 +  \frac1{\epsilon}\pair y\cdot \bigg)
\]
is the supremum of a family of affine functions, hence it is convex.

\noindent\underline{\emph{(ii)}}. Since, for each $x \in X$ fixed, the function $y \mapsto u(y) - \frac1{2\epsilon} |y-\cdot\,|^2$ is upper semicontinuous on $X$ and anti-coercive (because $u \in \USC(X)$ and bounded above). Hence the desired conclusion follows from \Cref{uscweier}{\em(ii)}.
%Fix $x \in \R^n$; define, for $\eta >0$ small,
%\[
%\call O(x) \defeq  \sup_{\R^n} u - u^\epsilon(x) + \eta \quad\text{and}\quad  \rho(x) \defeq \sqrt{2\epsilon \call O(x)}.
%\]
%Note that $\call O$ and $\rho$ depend also on $\epsilon$ and $\eta$, nevertheless we decided not to write this dependence explicitly in order to keep our notation as cleaner as possible. Note now that for every $y \notin  B_{\rho(x)}(x)$ one has
%\[
%u(y) - \frac1{2\epsilon}|y-x|^2 \leq \sup_{\,\R^n} u - \frac{\rho(x)^2}{2\epsilon} = u^\epsilon(x) - \eta.
%\]
%Hence 
%\begin{equation} \label{maxatt}
%u^\epsilon(x) = \sup_{y \in \barr B_{\rho(x)}(x)} \Big( u(y) - \frac1{2\epsilon} |y-x|^2 \Big)
%\end{equation}
%for all $\eta >0$. Since $\barr B_{\rho(x)}(x)$ is compact, the supremum is actually a maximum thanks to the upper semicontinuity of $u$, thus proving \emph{(ii)}.

\noindent\underline{\emph{(iii)}}. By definition
\begin{equation} \label{uepsdecr}
u(x)\leq u^{\epsilon'}(x) \leq u^\epsilon(x) \quad \text{whenever $\epsilon'< \epsilon$},
\end{equation}
and by point {\em(ii)} there exists a ball $B_{\rho(x)}(x)$ about $x$ such that
\begin{equation} \label{questa}
u^\epsilon (x) = \max_{\barr B_{\rho(x)}(x)} \left(u - \frac1{2\epsilon} | \cdot - \, x|^2\right) \leq \max_{\barr B_{\rho(x)}(x)} u,
\end{equation}
where we can write ``$\max$'' in the rightmost term thanks to \Cref{uscweier}{\em(i)} since $u$ is upper semicontinuous. It is easy to get convinced that $\rho(x) \dto 0$ as $\epsilon \dto 0$; indeed, if not, eventually (for small $\epsilon$'s) $\arg\max_{y \in X} \left( u - \frac1{2\epsilon} |\cdot -\, x|^2\right) \subset B_r(x)\compl$, for some $r = r(x) > 0$ independent of $\epsilon$, yielding $u^\epsilon(x) \leq \sup_{B_r(x)} u - \frac{r^2}{2\epsilon} < u(x)$ if $\epsilon$ is sufficiently small, which contradicts \eqref{uepsdecr}. Hence, the rightmost term in \eqref{questa} converges to $u(x)$ as $\epsilon \dto 0$,\footnote{The term converges to $\max\{ u(x), \limsup_{y\to x} u(y) \}$, where $ \limsup_{y\to x} u(y) \leq u(x)$ by the upper semicontinuity of $u$.} yielding $u^\epsilon(x) \dto$ $u(x)$.

By \eqref{questa} we also see that if $x$ is a local maximum point for $u$, then for $\epsilon$ so small that $x$ is a global maximum on $\barr B_{\rho(x)}(x)$ we have $u^\epsilon(x) \leq u(x)$ and hence the equality.

\noindent\underline{\emph{(iv)}}. If $|u| \leq M$, note that if $|y-x|^2 > 4\epsilon M$, then
\[
\left( u(y) - \frac{1}{2\epsilon} |y-x|^2 \right) - u(x) < M - \frac{1}{2\epsilon} 4 \epsilon M + M = 0;
\] 
that is, $\sup_{y \in X \cap B_\delta(x)\compl} \left( u(y) - \frac{1}{2\epsilon} |y-x|^2 \right) \leq u(x) \leq u^\epsilon(x)$, and thus we see that one can compute $u^\epsilon(x) = \sup_{y \in X \cap \bar B_\delta(x)} \left( u(y) - \frac{1}{2\epsilon} |y-x|^2 \right)$, which is \eqref{supcball} after the change of variables $x-y= z$, provided that $B_\delta(x) \ssubset X$ (which is true if $x \in X_\delta$).
\end{proof}

%\begin{remark}
%If we make the dependence on $\eta$ explicit in the proof above by writing $\rho_\eta$ instead of $\rho(x)$, we see that as $\eta \dto 0$ one obtains a sequence of points $\xi_k \in \barr B_{\rho_{\eta_k}}(x)$ such that $u(\xi_k) -\frac1{2\epsilon}|\xi_k-x|^2 = u^\epsilon(x)$. Since $B_{\rho_\eta}(x) \subset B_{\rho_{\eta'}}(x)$ if $\eta < \eta'$ we have that $\{\xi_\eta\}_{\eta<\eta'} \subset \barr B_{\rho_{\eta'}}(x)$, hence by compactness, up to a subsequence, $\xi_k \to \xi \in \barr B_{\rho_0}(x)$. By the semicontinuity of $u$ one has 
%\[
%u^\epsilon(x) = \limsup_{\xi_k \to \xi}\big( u(\xi_k) - \tfrac1{2\epsilon}|\xi_k-x|^2\big) \leq u(\xi) - \tfrac1{2\epsilon}|\xi-x|^2 \leq u^\epsilon(x),
%\]
%therefore (\ref{maxatt}) holds also for $\eta = 0$.
%\end{remark}

\begin{remark}
Property \emph{(iv)} is crucial for the quasi-convex approximation technique described in the next part as it ensures that the supconvolution of a subharmonic function is still subharmonic, on an adequately shrunk domain (cf.~\Cref{prop:approx}).
\end{remark}

A finer result on the correspondence between upper contact points and jets of a function and upper contact points and jets of its supconvolutions is the following, which Crandall--Ishii--Lions~\cite{user} call the \emph{magic property} of the supconvolution.

\begin{prop} \label{magprop}
Let $u \in \USC(\R^n)$ be bounded above and, for $\epsilon >0$, let $u^\epsilon$ be its $\epsilon$-supconvolution. If $(p,A)$ is an upper contact jet for $u^\epsilon$ at $x$, then $(p,A)$ is also an upper contact jet for $u$ at $x+ \epsilon p$. 

Furthermore, if $(0,A)$ is the limit of upper contact jets for $u^\epsilon$ at points converging to $0$, so it is for $u$, and $u$ is continuous at $0$ along the sequence of their contact points.
\end{prop}

\begin{proof}
Let $\xi\in \R^n$ such that
\begin{equation} \label{eq0:magprop}
u(\xi) - \frac1{2\epsilon} |\xi-x|^2 = u^\epsilon(x);
\end{equation}
note that such a point $\xi$ exists by part (b) of \Cref{baspropsc}. For every $z\in \R^n$ and every $y \in B_{\rho}(x)$, with $\rho > 0$ sufficiently small, we have
\begin{equation} \label{eq1:magprop}
u(z) - \tfrac1{2\epsilon}|y-z|^2 \leq  u^{\epsilon}(y) \leq u^{\epsilon}(x) + \pair{p}{y-x}+ Q_A(y-x),
\end{equation}
where the first inequality follows from the definition of $u^\epsilon$ and the second inequality holds since $(p,A)$ is an upper contact jet for $u^\epsilon$ at $x$. Now using 
the choice of $\xi$ in \eqref{eq0:magprop} we can rewrite \eqref{eq1:magprop} as
\begin{equation} \label{eq2:magprop}
u(z) - \tfrac1{2\epsilon}|y-z|^2 \leq  u(\xi)- \tfrac1{2\epsilon}|\xi-x|^2 + \pair{p}{y-x}+ Q_A(y-x),
\end{equation}
which holds for every $z \in \R^n$ and every  $y \in B_{\rho}(x)$.


By choosing $z=y-x+\xi$ in \eqref{eq2:magprop} and using the fact that $y \in B_\rho(x)$ is arbitrary, one has
\[
u(z) \leq u(\xi) + \pair{p}{z-\xi} + Q_A(z-\xi), \quad \forall \, z \in B_\rho(\xi);
\]
that is, $(p,A)$ is an upper contact jet for $u$ at $\xi$. 

For the first part of the proposition it remains only to show $\xi = x + \epsilon p$. One makes the choices 
\[
z = \xi \quad \text{and} \quad y=x + \frac{t}{\epsilon}(x +\epsilon p -\xi )
\]
with
\[
|t| < \frac{\rho}{\epsilon |x  + \epsilon p - \xi|} \quad \text{if $\xi \neq x+\epsilon p$}, \qquad \text{$t$ arbitrary if $\xi = x+\epsilon p$},
\]
so that $y \in B_{\rho}(x)$ and one can again apply ~\eqref{eq2:magprop} (multiplied by $\epsilon^2$) to find that as $t \to 0$,
\[
- \pair{x-\xi}{t(x + \epsilon p -\xi)} + o(t) \leq \epsilon \pair{p}{t(x+\epsilon p-\xi)} + o(t);
\]
that is,
\begin{equation} \label{abs:magprop}
t|x + \epsilon p - \xi |^2 + o(t) \geq 0.
\end{equation}
Since along a sequence $0 > t_n \uto 0^-$ the left-hand side of the above inequality~\eqref{abs:magprop} is eventually negative when $\xi \neq x+\epsilon p$, the only possibility for that to hold is that
\begin{equation} \label{xi=:magprop}
\xi = x + \epsilon p.
\end{equation}

It is now immediate to see that if $(p_j, A_j) \to (0, A)$ are upper contact jets for $u^\epsilon$ at $x_j \to 0$, then they are upper contact jets for $u$ as well, at $x_j + \epsilon p_j \to 0$. Finally, note that plugging \eqref{xi=:magprop} into \eqref{eq0:magprop}, one has $u^\epsilon(x) = u(x+\epsilon p) - \frac\epsilon2|p|^2$. Since  $u \leq u^\epsilon$, $u^\epsilon$ is continuous, and $u$ is upper semicontinuous, it follows that
\[
\begin{split}
u(0) \leq u^\epsilon(0) &= \lim_{j\to\infty} \!\big( u^\epsilon(x_j) + \tfrac\epsilon2|p_j|^2 \big) \\&=  \liminf_{j\to\infty} u(x_j + \epsilon p_j) \leq \limsup_{j\to\infty} u(x_j + \epsilon p_j) \leq u(0);
\end{split}
\]
this implies that $u(x_j + \epsilon p_j) \to u(0)$, and the proof is concluded.
\end{proof}

\subsection{The Crandall--Ishii--Lions Theorem on Sums}

We will make a strong use of \Cref{baspropsc} in the following part on subharmonics in nonlinear potential theory. Moreover, as is well known, the quasi-convex approximation of semicontinuous functions is a key ingredient in the theory of viscosity solutions for nonlinear PDEs. In particular, together with Jensen's \Cref{u:jen}, one is able to prove the important Theorem on Sums (see \cite[Appendix]{user}). As an application of what has been developed thus far, we give a short proof of that celebrated theorem in a normalised form (cf.\ \cite[Theorem 3.2$'$]{user}). The idea is to use \Cref{magprop} to extend the quasi-convex version in \Cref{tosqc2} to upper semicontinuous functions.
%In some situations the magic property (\Cref{magprop}) is not required, and the other properties are sufficient in order to develop a quasi-convex approximation technique, as we are going to show in the next part (see \Cref{bastool}). Nevertheless, sometimes such a technique cannot be applied, as one needs some finer analysis; in this direction, a natural consequence of \Cref{magprop} is the Theorem on Sums.

\begin{thm}[On Sums] \label{tos}
Let $u,v \in \USC(\R^n)$, bounded above, with $u(0) = 0 = v(0)$. Define 
\[
w(x,y) \defeq u(x) + v(y)
\]
on $\R^n \times \R^n$ and suppose that $(0,A)$ is an upper contact jet for $w$ at $0 \in \R^{2n}$. Then $0 \in \R^n$ is a contact point for both $u$ and $v$, and for every $\epsilon >0$ there exist $A_1, A_2\in \Sc(n)$ such that $(0,A_1), (0,A_2)$ are limits of upper contact jets for $u, v$, respectively, at points converging to $0$,
with
\begin{equation} \label{tosineq}
-\Big(\,\frac1\epsilon + \Vert A\Vert \Big) \,I \leq \left(\! \begin{array}{cc}
A_1 &  0 \\
0  & A_2 \\
\end{array}\! \right)
\leq A + \epsilon A^2.
\end{equation}
Also, $u$ and $v$ are continuous along the respective sequences of contact points.
\end{thm}

\begin{proof}
Fix $\epsilon >0$. Easy calculations exploiting the Cauchy--Schwarz inequality yield\footnote{For the sake of completeness, here are the calculations (recall that $A$ is symmetric):
\[\begin{split}
0 &\leq \Big(\sqrt{\epsilon} |Aw| - \frac1{\sqrt\epsilon}|z-w|\Big)^2 \\
&\leq \epsilon |Aw|^2 - 2\pair{Aw}{z-w} + \frac1\epsilon|z-w|^2 \\
&= \pair{Aw}{w} + \epsilon \pair{Aw}{Aw} + \frac1\epsilon|z-w|^2 + \pair{A(z-w)}{z-w} - \pair{Az}{z} \\
&\leq \pair{(A+\epsilon A^2)w}{w} + \Big(\frac1\epsilon + \norm{A}\Big) |z-w|^2 - \pair{Az}{z}.
\end{split}\]
%The second author wishes to thank E.~Danesi for the crucial contribution in clarifying the genesis of this inequality.
}
\[
\pair{Az}{z} \leq \pair{(A+\epsilon A^2)w}{w} + \lambda|z-w|^2, \qquad \forall z,w \in \R^{2n}, \ \lambda \defeq \frac1\epsilon + \Vert A\Vert.
\]
Therefore
\[
%w(z) - \frac\lambda2|z-w|^2 \leq \tfrac12 \pair{(A+\epsilon A^2)w}{w},
w(z) - \frac\lambda2|z-w|^2 \leq Q_{A+\epsilon A^2}(w),
\]
hence, on taking the supremum over $z\in \R^{2n}$ and writing $w=(x,y)$, we have $(0, A+\epsilon A^2)$ is an upper contact jet at $0$ for the supconvolution
\[
w^{1/\lambda}(x,y) =  u^{1/\lambda}(x) +  v^{1/\lambda}(y).
\]
Indeed, $u^{1/\lambda}(0) \geq u(0) = 0$ by definition, and the same holds for $v$, while $w^{1/\lambda}(0) \leq 0$, and thus $u^{1/\lambda}(0) = 0 = v^{1/\lambda}(0)$. Now, by \Cref{tosqc2} we know that there exist jets $(0, A_i)$ which are the limits of upper contact jets for the $\lambda$-quasi-convex functions $u^{1/\lambda}, v^{1/\lambda}$ at some points (of second-order differentiability) converging to $0$ and that the $A_i$'s satisfy~(\ref{tosineq}). The conclusion now follows from \Cref{magprop}.
\end{proof}

\begin{remark}
In view of \Cref{tosqc2} and the proof of \Cref{magprop}, we can also say that one may choose the converging sequences of upper contact jets in \Cref{tos} to be $(Du^{1/\lambda}(x_j), D^2u^{1/\lambda}(x_j))$ and $(Dv^{1/\lambda}(y_j), D^2v^{1/\lambda}(y_j))$, for some sequences $\{x_j\}, \{y_j\}$ of points at which the supconvolutions $u^{1/\lambda}, v^{1/\lambda}$ are twice differentiable.
\end{remark}

\begin{remark}
On choosing
\[
A = \frac1\epsilon \left(\! \begin{array}{cc}
I & -I \\
-I  & I \\
\end{array}\! \right)
\]
one finds the formulation in \cite[Theorem C.1]{hldir}, whose property (2) follows from \Cref{rmk:puscsum},  \Cref{magprop} and \cite[Proposition~3.7]{user}.
\end{remark}


\part{Subequations and subsolutions}\label{sas}

This second part incorporates the results in nonlinear potential theory contained in \cite{hlae}, together with some subsequent and related investigations.

We start by proposing a deduction of the minimal properties one needs a subequation to satisfy when one identifies it with a subset of the bundle of the $2$-jets of functions on an Euclidean domain.

A \emph{subharmonic addition} theorem for quasi-convex functions (\Cref{add}) is deduced as a corollary of the \emph{Almost Everywhere Theorem}~\ref{aet}, which states that, for locally quasi-convex functions, it is equivalent to be subharmonic either almost everywhere or everywhere. 
By means of the Summand \Cref{pusc:sum}, a \emph{strict comparison} theorem for quasi-convex functions (\Cref{scqc}) is proven which holds in the nonconstant coefficient case.

In \cite{hlae}, these three potential-theoretic theorems are stated in a more general form on manifolds; in this work we remain in Euclidean spaces, and a brief discussion on how one can extend such results to manifolds is included at the end of this part.

Exploiting and adapting a quasi-convex approximation technique outlined in \cite{hldir09}, we deduce a subharmonic addition theorem for constant coefficient subequations from its quasi-convex counterpart; similarly, strict comparison holds in the constant coefficient case without the quasi-convexity assumption. We mention that the first result can also be found in \cite[Theorem~7.1]{chlp}, with an analogous proof, while the latter has a Theorem-of-Sums-based proof in \cite[Corollary~C.3]{hldir}.

Basic examples of applications of the subharmonic addition theorem in deducing comparison for constant coefficient pure second-order or gradient-free subequations (see Definitions~\ref{psos} and \ref{gfs}) are also included.

Furthermore, we will show that the basic properties of dual subequations proven in \cite{hldir09, hldir} and regathered in \cite{chlp}, as well as the \emph{definitional comparison} \cite[Lemma~3.14]{chlp}, hold for nonconstant coefficient subequations, too. Consequently, we extend \emph{almost all} the elementary properties of families of subharmonic functions contained in \cite{hldir09, hldir, chlp} to nonconstant coefficient subequations; only one of them must be left behind, namely the \emph{translation property}, so that a possible extensions of potential theoretic comparison must confront themselves with the lack of this property. We will not discuss this key fact, as this work is meant to be a introductory presentation to Harvey and Lawson's theory, and we refer the interested reader to~\cite{cpaux,cpmain,cprdir}.
% \noticina{To rewrite---We don't have this part here anymore.}A brief discussion about the implications of the lack of this property in the nonconstant coefficient case in included at the beginning of the last section, where we adapt certain results of Cirant--Payne on variable coefficient gradient-free subequations~\cite{cpaux, cpmain} to variable coefficient subequations depending on the gradient too, by adding a \emph{directionality} assumption inspired by the theory of monotonicity cones in \cite{chlp}. Such results are then used in the proof of a \emph{strict comparison theorem for uniformly continuous subequations with directionality}. 
%\footnote{We decided not to discuss the \emph{vice versa}, that is the problem of associating a ``compatible'' operator to a given subequation. The reader can see~\cite[part~11]{chlp} for some results in this direction.}


\section{Subequation constraint sets} \label{sec:conset}

We start with a discussion of the axioms which should be placed on a subequation constraint set as a subset of the $2$-jet bundle in order to determine a reasonable potential theory.

Subequation constraint sets and their associated potential theory arise naturally in the study of a given PDE in the following manner.
Suppose we are given an operator $F \in C(X \times \R \times \R^n \times \Sc(n))$ on some open set $X\subset \R^n$ which determines a PDE
\begin{equation} \label{Fxududdu=0}
F(x, u(x), Du(x), D^2u(x)) = 0, \quad x \in X.
\end{equation} 
A classical \emph{subsolution} of (\ref{Fxududdu=0}) is a function $u\in C^2(X)$ which satisfies
\begin{equation} \label{Fxududdu>0}
F(x, u(x), Du(x), D^2u(x)) \geq 0 \quad \forall x \in X.
\end{equation}
The inequality $F(x, r, p, A) \geq 0$ identifies a closed subset $\call F \subset \call J^2(X) = X \times \R \times \R^n \times \Sc(n))$,\footnote{Note that in fact this Cartesian product represents for us the bundle $\call J^2(X)$ of the $2$-jets of functions on $X$; indeed, for a Riemannian $m$-dimensional manifold $M$ one has the following isomorphism for the bundle of the $2$-jets of functions on $M$:
\[
\call J^2(M) \simeq M\times \R \underset{M}{\oplus} T^*M \underset{M}{\oplus} \Sc(T^*M),
\]
where $\Sc(T^*M)$ denotes the space of all symmetric $(0,2)$-tensor fields on $M$.}
which is a natural constraint set associated to classical subharmonics where one says that $u \in C^2(X)$ is \emph{$\call F$-subharmonic on $X$} if 
\begin{equation} \label{coherence}
\call J^2_x u \defeq (u(x), Du(x), D^2u(x))\in \call F_x \qquad \forall x \in X,
\end{equation}
where
\[
\call F_x \defeq \left\{(r,p,A)\in \R \times \R^n \times \Sc(n):\ (x,r,p,A) \in \call F \right\}
\] 
is the fiber of $\call F$ over $x$. One extends the notions of subsolutions and subharmonics to upper semicontinuous functions by using a pointwise viscosity formulation. For example, a function $u \in \USC(X)$ is a \emph{viscosity} subsolution of (\ref{Fxududdu=0}) \emph{at $x$} if, for any upper contact quadratic function $\phi$ for $u$ at $x$, we have
\begin{equation} \label{subsvisc}
F(x, u(x), D\phi(x), D^2\phi(x)) \geq 0
\end{equation}
(for this and other equivalent definitions of viscosity subsolution, see for instance \cite[Lemma C.1]{chlp}). This means that in order for $u$ to be a viscosity subsolution at $x$ we need that $u$ is \emph{$\call F$-subharmonic at $x$} in the sense that 
\begin{equation} \label{subhatx}
(u(x), p, A) \in \call F_x \quad \text{for every upper contact jet $(p,A)$ for $u$ at $x$.}
\end{equation}

Note that it is fundamental to have \emph{coherence} in the sense that every classical subsolution is also a viscosity subsolution; that is, for every $u \in C^2(X)$ satisfying (\ref{coherence}) one has (\ref{subhatx}) for each $x \in X$.

By \Cref{baspropucp} one knows that for $u$ which is at least twice differentiable at $x$, if $(p,A)$ is an upper contact jet for $u$ at $x$, then $(p,A) = (Du(x), D^2u(x) + P)$ for some $P\geq 0$. Hence we will have coherence if the operator is \emph{degenerate elliptic}; that is 
\begin{equation} \label{degell}
F(x,r,p,A+P)\geq F(x,r,p,A) \qquad \forall P\geq 0
\end{equation}
or more generally if $F(x,r,p,A+P)\geq 0$ for all $P\geq 0$ if $F(x,r,p,A)\geq 0$. 

If instead we now look for \emph{viscosity supersolutions}, it is not difficult to see that we can proceed with an analogous discussion, where we reverse the inequality in (\ref{subsvisc}),  thus getting $\barr{{\call F_x}{}\compl}$ in place of $\call F_x$, and replace \emph{upper} semicontinuous functions with \emph{lower} semicontinuous functions, \emph{upper} contact jet with \emph{lower} contact jet. 

Let us now give to the following definitions.

\begin{definition} \label{def:subh}
Let $\call F \subset \call J^2(X)$.
\begin{enumerate}[label=\it(\roman*), left=0pt]
\item A function $u \in \USC(X)$ is said to be \emph{$\call F$-subharmonic} (or a \emph{viscosity subsolution of $\call F$}) \emph{at $x\in X$} if 
\[
(u(x),p,A) \in \call F_x
\]
for every $(p,A)\in J^{2,+}_x u$.

The function $u$ is said to be $\call F$-subharmonic \emph{on $X$} if  $u$ is $\call F$-subharmonic at each point $x\in X$; we denote by $\call F(X)$ the set of functions which are $\call F$-subharmonic on $X$.

\item A function $v \in \LSC(X)$ is said to be \emph{$\call F$-superharmonic} (or a \emph{viscosity supersolution of $\call F$}) \emph{at $x\in X$} if 
\[
(v(x),q,B) \notin \intr{\call F}_x
\]
for every $(q,B)\in J^{2,-}_x v$.

The function $v$ is said to be $\call F$-superharmonic \emph{on $X$} if  $v$ is $\call F$-superharmonic at each point $x\in X$; with a notation that will be clarified in \Cref{sec:duality}, we denote by $-\tildee{\call F}(X)$ the set of functions which are $\call F$-superharmonic on $X$.

\item A function $w\in \scr C(X)$ is \emph{$\call F$-harmonic} (or a \emph{viscosity solution of $\call F$}) \emph{at $x\in X$} if it is both $\call F$-subharmonic and $\call F$-superharmonic at $x$. It is $\call F$-harmonic \emph{on X} if it is both $\call F$-subharmonic and $\call F$-superharmonic on $X$.
\end{enumerate}
\end{definition}

Note that being subharmonic, or superharmonic, at some point is a local notion, since the definition of upper contact jet is.

Notice that these notions of solutions are free from any particular operator $F \in C(X \times \R \times \R^n \times \Sc(n))$, they depend only on a choice of $\call F \subset \call J^2(X)$. In order for this definitions to be useful we require the two properties we highlighted above. Hence the following definition.

\begin{definition}
A subset $\call F \subset \call J^2(X)$ is said to be a \emph{primitive subequation (constraint set)} if
\begin{equation} \tag{C} \label{(C)}
\text{$\call F$ is closed},
\end{equation}
and $\call F$ satisfies the condition of \emph{positivity} (or \emph{degenerate ellipticity}); i.e.
\begin{equation} \tag{P} \label{(P)}
(r,p,A) \in \call F_x \implies (r,p,A+P) \in \call F_x \quad \forall P> 0.
\end{equation}
\end{definition}

As we noted, the positivity condition is essential in order to have coherence (cf.\ (\ref{degell})), while the closedness condition ensures that the elementary properties of $\call F(X)$ involving limits and upper envelopes hold (see \cite[Theorem~2.6]{hldir}; the same properties are also listed, for nonconstant coefficient subequations, in \Cref{elemprop}).

The two minimal conditions \eqref{(C)} and \eqref{(P)}, however,  are not sufficient, in general, if we wish to  have some chances of proving comparison; that is,
\[
u\leq v\ \text{on $\de X$} \implies u\leq v\ \text{on $X$}
\]
whenever $u$ and $v$ are $\call F$-subharmonic and superharmonic respectively (or a sub/super-solution pair for a PDE). We will need that $\call F$ satisfies the condition of \emph{negativity} (or \emph{properness}); i.e.
\begin{equation} \tag{N} \label{(N)}
(r,p,A) \in \call F_x \implies (r+s, p, A) \in \call F_x \quad \forall s < 0.
\end{equation}

\begin{example}
Consider, for $n=1$, the ordinary differential equation $u''+u = 0$ on $\R$, that is $F(u, u'') = 0$, with $F(r,A) = A+ r$. It is elementary to see that the associated $\call F$ does not satisfy \eqref{(N)} and comparison does not hold on arbitrary bounded domains, but only on intervals of length less than $\pi$.
\end{example}

The negativity condition \eqref{(N)} for $\call F$ whose fibers are 
\[
\call F_x \defeq \left\{(r,p,A):\ F(x,r,p,A) \geq 0\right\}
\]
 is implied by the operator $F$ being \emph{proper}, that is if $F(x,r+s,p,A) \geq F(x,r,p,A)$ for all $s < 0$, and it guarantees a \emph{sliding property} for subsolutions: if $u$ is a subsolution, so is $u-m$. Indeed, if $(p,A)$ is an upper contact jet for $u$ at $x$, so it will be for $u-m$, for any $m \in \R_+$.  This is going to be a crucial property in the proofs the upcoming results, and it is included, along with the coherence property, in \Cref{elemprop}.

Hence a good definition of subequation has to take into account that we need \emph{proper elliptic operators} (or at least that $\call F$ satisfies \eqref{(P)} and \eqref{(N)}). As for \eqref{(C)}, for our purposes we actually need a strengthening of it, namely the topological condition \eqref{(T)} in the following definition.

\begin{remark}
Note that if a fiber $\call F_x$ is empty, then $\call F$ could not admit a viscosity subsolution on the whole $X$. To see this, note the same argument we used at the beginning of the proof of \Cref{p:uvm} proves that the set
\begin{equation} \label{defj2+}
\call J^{2,+}_x u \defeq \{ (x,u(x), p, A):\ x\in X,\ (p,A) \in J^{2,+}_xu \}
\end{equation}
is nonempty for some $x \in X$, since $u \in \USC(X)$ and $X$ is locally compact. Analogously, one needs to ask that $\call F_x \subsetneqq \call J^2 \defeq \R \times \R^n \times \Sc(n)$ for all $x \in X$ in order not to exclude \emph{a priori} that a supersolution exists. 
%The fact that $\call J^{2,+}_x u$ is never empty is of crucial importance: $\call J^{2,+}_x u$ would be an insurmountable obstacle for the viscosity theory.
\end{remark}

\begin{definition} \label{defnsubeq}
A primitive subequation $\call F$ with nontrivial fibers (that is, proper and nonempty fibers) is called a \emph{subequation (constraint set)} if $\call F$ satisfies condition \eqref{(N)}, and the three conditions of \emph{topological stability}; i.e.
\begin{equation} \tag{T} \label{(T)}
\call F = \barr{\intr{\call F}}, \qquad (\intr{\call F})_x = \intr(\call F_x), \qquad \call F_x = \barr{\intr{\call F}_x}.
\end{equation}
\end{definition}

This last condition has multiple implications. For instance, it yields a property of \emph{reflexivity} with respect to Dirichlet duality (see \Cref{sec:duality}), and it also guarantees \emph{local existence of strict classical subharmonics}, when we adopt the following notion of {strictness}.

\begin{definition} \label{def:strict}
We say that $u\in C^2(X)$ is a \emph{strict} subsolution at $x$ of a (primitive) subequation $\call F$ if 
\[
\call J^2_x u \in \intr{\call F}_x.
\]
If the above inclusion holds for all $x \in X$, then we say that $u$ is \emph{strictly} $\call F$-subharmonic on $X$.
\end{definition}

\begin{remark}[Local existence of strict classical subharmonics] \label{exquadsub}
One can prove that thanks to the topological condition \eqref{(T)}, for each $x \in X$  there exists a quadratic function which is strictly subharmonic on a neighborhood of $x$. Indeed, by \eqref{(T)} we know that $\intr{\call F}_x \neq \emptyset$ for all $x \in X$; hence, let $J = (r, p, A) \in \intr{\call F}_x$ and consider the quadratic function $\phi_{J}$ associated to $J$, namely
\[
%\phi_{J} = r + \pair{p}{\cdot -\, x} + \tfrac12 \pair{A(\cdot -\, x)}{\cdot -\, x},
\phi_{J} = r + \pair{p}{\cdot -\, x} + Q_A(\,\cdot - x),
\]
 so that $\call J^2_x \phi_J= J$. Since $\phi_J$ is $C^2$ and $(x, J) \in \intr{\call F}$, we have that $\call J^2_y \phi_J \in \intr{\call F}_y$ for all $y$ in some neighborhood $U$ of $x$, hence $\phi_J \in \call F(U)$.
 \end{remark}
 

We conclude this section with two interesting remarks one finds in \cite{hlae}, and with the definition of \emph{constant-coefficient} subequation.

\begin{remark}\label{subinclusion}
The notion of being subharmonic can be expressed in terms of an inclusion between subequations. Indeed,
each upper semicontinuous function $w$ on $X$ determines a minimal primitive subequation $\call G$ such that $w$ is $\call G$-subharmonic, namely $\call G=\barr{\call J^{2,+}w}$, where $\call J^{2,+}w$ is defined as in \eqref{defj2+}. Therefore $w$ is $\call F$-subharmonic if and only if $\call J^{2,+}w \subset \call F$, if and only if $\call G \subset \call F$ since $\call F$ is closed.
\end{remark}

\begin{remark} \label{ae?}
Note that we already found a subset which surely contains $\call J^{2,+}w$. Indeed, if we set 
\[
\call J(w,E) \defeq \big\{ (x, w(x), Dw(x), D^2w(x) + P):\ x \in E \cap \Diff^2(w),\ P \geq 0 \big\},
\]
then \Cref{pusc} says that if $w$ is locally quasi-convex and $E$ has full measure, then $\call J^{2,+}w \subset \barr{\call J(w,E)}$.
\end{remark}

\begin{definition}
A (primitive) subequation $\call F$ is said a \emph{constant coefficient}, or \emph{translation invariant}, (primitive) subequation if the mapping $x \mapsto \call F_x$ is constant, that is if the fibers do not depend on the point $x$.
\end{definition}

\begin{remark} \label{rmkccs}
Since our definition of subequation requires that all the fibers are nontrivial, it is important to specify the open set $X$ on which we are considering a subequation $\call F$. Nevertheless, if $\call F$ is a constant coefficient subequation on $X$, it can be naturally extended to the whole $\R^n$ by setting, for some $x_0 \in X$ fixed, $\call F_y = \call F_{x_0}$ for all $y \in \R^n$. Hence we can think that constant coefficient subequations are always defined on $\R^n$ and identify them with a nontrivial subset of $\call J^2 \defeq \R \times \R^n \times \Sc(n)$ (that satisfy the conditions \eqref{(P)}, \eqref{(N)} and \eqref{(T)}). 

Obviously, if instead we are talking about sub-- and supersolution, we must specify on which set they are $\call F$-sub-- or superharmonic even if $\call F$ is a constant coefficient subequation.
\end{remark}


\section{Duality} \label{sec:duality}

A subequation naturally gives rise to a dual subequation, which will play a crucial role in all that follows. Consider the following notion of duality, introduced by Harvey--Lawson \cite{hldir09, hldir}.

\begin{definition} \label{def:dual}
Given a proper subset $\call F \subset \call J^2(X)$, we define its \emph{Dirichlet dual}
\[
\tildee{\call F} \defeq  (-\intr{\call F})\compl = - (\intr{\call F})\compl,
\]
where the complement is relative to $\call J^2(X)$.
\end{definition}

It turns out that if $\call F$ is a subequation, then $\tildee{\call F}$ is a subequation as well; this is a consequence of the following proposition, which collects some elementary properties of the Dirichlet dual. These properties are to be found in \cite[Section 4]{hldir09} in a pure second-order context and in \cite[Section 3]{hldir} for subequations on Riemannian manifolds, as well as  in \cite[Proposition~3.2]{chlp} for constant coefficient subequations;  we essentially reproduce the proof of this last proposition, in order to point out that it works fine also if one has nonconstant coefficients. This happens because of a consequence of the topological condition \eqref{(T)}, as noted in \cite{hldir}. More precisely, if we define the dual of the fiber $\call F_x \subset \R \times \R^n \times \Sc(n)$ relative to the ambient space $\R \times \R^n \times \Sc(n)$, one can compute the dual of $\call F$ fiberwise; that is,
\[
\tildee{\call F} = \bigsqcup_{x\in X}  (-\intr{\call F}_x)\compl,
\]
where the complement is relative to $\R \times \R^n \times \Sc(n)$, so there is no ambiguity in the notation $\tildee{\call F}_x$.

\begin{prop} \label{propdual}
Let $\call F$ and $\call G$ be proper subsets of $\call J^2(X)$ and let $J \in \R \times \R^n \times \Sc(n)$. Then the following hold:
\begin{enumerate}[label={\it(\arabic*)}, topsep=7pt, itemsep=4pt, partopsep=1.1pt, parsep=1.1pt]
\item\ $\call F_x \subset \call G_x \implies \tildee{\call G}_x \subset \tildee{\call F}_x$;
\item\ $\call F_x + J \subset \call F_x \implies \tildee{\call F}_x + J \subset \tildee{\call F}_x$;
\item\ $\tildee{\call F_x + J} = \tildee{\call F}_x - J$;
\item\ $\call F$ satisfies \eqref{(P)} $\implies$ $\tildee{\call F}$ satisfies \eqref{(P)};
\item\ $\call F$ satisfies \eqref{(N)} $\implies$ $\tildee{\call F}$ satisfies \eqref{(N)};
\item\ $\call F =\barr{\intr{\call F}}$ $\iff$ $\tildee{\tildee{\call F}} = \call F$;
\item\ $\call F$ satisfies \eqref{(T)} $\implies$ $\tildee{\call F}$ satisfies \eqref{(T)};
\item\ $\call F$ is a subequation $\implies$ $\tildee{\call F}$ is a subequation.
\end{enumerate}
\end{prop}

\begin{proof}
Properties~\emph{(4)} and~\emph{(5)} follows from~\emph{(2)}, and property~\emph{(8)} follows from properties~\emph{(4)}, \emph{(5)} and~\emph{(7)}. 
Property~\emph{(1)} follows from \Cref{def:dual}. 
For property~\emph{(2)}, note that $\call F_x + J \subset \call F_x$ implies that $\intr{\call F}_x + J$ is an open subset of $\call F$, hence $\intr{\call F}_x + J \subset \intr{\call F}_x$, which yields, taking the complements, $\tildee{\call F}_x \subset \tildee{\call F}_x -J$, as desired. 
Similarly, since $\intr{\call F}_x + J$ is the interior of $\call F_x + J$, we have $\tildee{\call F_x + J} = \tildee{\call F}_x - J$, which is property \emph{(3)}.
Note now that by \Cref{def:dual}
\begin{equation} \label{bidual}
\tildee{\tildee{\call F}} = \big(\intr\big((\intr\call F)\compl\big)\big)\compl;
\end{equation}
since for any subset $\call S \subset \call J^2(X)$ one has $\intr{\call S} = {\barr{\call S\compl}}\compl$, equality~(\ref{bidual}) yields $\tildee{\tildee{\call F}} = \barr{\intr{\call F}}$.
This proves property \emph{(6)}.
Similarly, if $\call F$ satisfies \eqref{(T)}, consider the following chain of equalities:
\[
\barr{\intr{\tildee{\call F}}} = \barr{{\barr{\tildee{\call F}\,{\vphantom{I}}\compl}\,}\compl} = \barr{{\barr{ - \intr{\call F}}\,}\compl} = \barr{ - \call F\compl} = (-\intr{\call F})\compl = \tildee{\call F}.
\]
This shows the first and the last of the three conditions in \eqref{(T)}, since it also holds with $\call F_x$ instead of $\call F$. Furthermore, we also read in it that
\[
(\intr{\tildee{\call F}})_x = -( \call F\,\compl)_x =  - (\call F_x)\compl= \intr(\tildee{\call F}_x)
\]
and hence the second condition is satisfied, thus proving property {\em(7)}.
\end{proof}

\begin{remark} \label{invert}
Property \emph{(6)} is the aforementioned \emph{reflexivity} of $\call F$. Note also that if it holds, then we can reverse all the implications in \Cref{propdual}.
\end{remark}

By making use of duality, we can reformulate the notions of being $\call F$-subharmonic and $\call F$-harmonic, which we record in the following observation.

\begin{remark}[\emph{$\call F$-superharmonics by duality}] 
For $v\in \LSC(X)$, one can define $v$ to be $\call F$-superharmonic on $X$ if  $-v$ is $\tildee{\call F}$-subharmonic on $X$. This is equivalent to \Cref{def:subh}; indeed, the condition $(v(x), q, B) \notin \intr{\call F}_x$ for each $(q,B) \in J^{2,-}_x v$ is equivalent to $(-v(x), p, A) \notin -\intr{\call F}_x$ for each $(p,A) = (-q,-B) \in J^{2,+}_x (-v)$, since it is immediate to see (cf.\ \Cref{def:ucp}) that $- J^{2,-}_x v = J^{2,+}_x (-v)$. This means that $v$ is $\call F$-superharmonic at $x$ if and only if $-v$ is $\tildee{\call F}$-subharmonic at $x$, because 
\[
(-v(x), p, A) \notin -\intr{\call F}_x \iff (-v(x), p, A) \in \tildee{\call F}_x,
\]
by the definition on the dual of $\call F$.

%This is a good definition because, arguing as above, if  $F \in \scr C(X \times \R \times \R^n \times \Sc(n), \R)$ is the operator associated to $\call F$, then a classical supersolution $v$ of $F(x, v, Dv, D^2v) = 0$ on $X$ satisfies $F(x, v(x), Dv(x), D^2v(x)) \leq 0$ for all $x \in X$. This produces a subset $\call G \subset \call J^2(X)$ characterized by the inequality $F(x,r,p,A) \leq 0$ and thus $\call G = \big(\,\intr{\call F}\,\big)\compl = - \tildee{\call F}$. Using viscosity theory, it is easy to see that $v$ is a viscosity supersolution on $X$ if and only if $(v(x), q, B) \in \call G_x$ for each point $x \in X$ and every lower contact jet for $v$ at $x$. This is equivalent to $(-v(x), p, A) \in -\call G_x$ for each point $x\in X$ and every upper contact jet for $-v$ at $x$, that is $-v \in \tildee{\call F}(X)$.

Finally, this yields the following equivalent definition of viscosity solution: a function $u\in C(X)$ is \emph{$\call F$-harmonic on $X$} if  $u\in \call F(X)$ and $-u\in\tildee{\call F}(X)$, that is $u \in \de \call F(X)$; indeed, since $\call F$ is closed, $\de \call F = \call F \cap \big(\!-\!\tildee{\call F}\, \big)$.
\end{remark}


\section{Basic tools in potential theory} \label{bastool}

In this section, we introduce three results that are included in the ``basic tool kit of viscosity solution techniques'' in \cite{chlp}: the Bad Test Jet \Cref{l:btj}, the Definitional Comparison \Cref{defcompa} and the Almost Everywhere \Cref{aet}. Also, we prove via the \emph{definitional comparison} some elementary properties of the set $\call F(X)$ that are known from \cite{hldir09} and a Subharmonic Addition theorem (\Cref{add}) immediately follows from the \emph{AE Theorem}, as Harvey--Lawson have proved in \cite{hlae}. 

We begin with the first tool which is very useful when one seeks to check the validity of subharmonicity at a point by a contradiction argument. More precisely, if $u$ fails to be subharmonic at a given point, then one must have the existence of a \emph{bad test jet} at that point, as stated in the following lemma. This criterion is essentially the contrapositive of the definition of viscosity subsolution, when one takes \emph{strict} upper contact quadratic functions as upper test functions (see \cite[Lemma 2.8 and Lemma C.1]{chlp}). Nevertheless, for the benefit of the reader, we provide a brief proof based on the \cref{def:subh} we gave.

\begin{lem}[Bad Test Jet Lemma] \label{l:btj}
Given $u \in \USC(X)$, $x \in X$ and $\call F_x \neq \emptyset$, suppose $u$ is not $\call F$-subharmonic at $x$. Then there exists $\epsilon > 0$ and a $2$-jet $J \notin \call F_x$ such that the (unique) quadratic function $\phi_J$ with $\call J^2_x\phi_J = J$ is an upper test function for $u$ at $x$ in the following $\epsilon$-strict sense:
\begin{equation} \label{btj:i}
u(y) - \phi_J(y) \leq -\epsilon|y-x|^2 \quad \text{$\forall y$ near $x$ (with equality at $x$)}.
\end{equation}
\end{lem}

\begin{proof}
If $u$ fails to be $\call F$-subharmonic at $x$, then according to \Cref{def:subh} there exists $J \in \call J^{2,+}_x u \setminus \call F_x$. Since the complement of $\call F_{x}$ (with respect to $\R \times \R^n \times \Sc(n)$) is open, we have that $J'\defeq J+(0,0, 2\epsilon I) \notin \call F_{x}$ for $\epsilon$ small. Also, $J' \in \call J^{2,+}_x u$ because, by \Cref{def:ucp}, if $(p,A)$ is an upper contact jet for $u$ at $x$, so is $(p, A+\eta I)$ for any $\eta > 0$. This means that 
\[
u(y) \leq \phi_{J'}(y) \quad \text{$\forall y$ near $x$ (with equality at $x$)},
\]
but $\phi_{J'} = \phi_J - \epsilon|\cdot - \,x|^2$, hence (\ref{btj:i}) follows.
\end{proof}

The second tool is a \emph{comparison principle} whose validity characterizes the $\call F$-subharmonic functions for a given subequation $\call F$. We begin by recalling two equivalent forms of potential theoretic comparison.

Given a subequation $\call F$, we say that \emph{comparison holds for $\call F$ on a domain $\Omega \subset \R^n$} if
\begin{equation} \tag{CP}
u \leq w\ \text{on $\de\Omega$} \implies u\leq w\ \text{on $\Omega$}
\end{equation}
for all $u\in \USC(\barr \Omega)$, $w \in \LSC(\barr \Omega)$ which are $\call F$-subharmonic, $\call F$-superharmonic on $\Omega$, respectively. Making use of the duality reformulation with $v \defeq -w$, comparison (CP) is equivalent to
\begin{equation} \tag{CP$'$}
u +v\leq 0\ \text{on $\de\Omega$} \implies u+v\leq 0\ \text{on $\Omega$}
\end{equation}
for all $u,v \in \USC(\barr\Omega)$ which are, respectively, $\call F$-, $\tildee{\call F}$-subharmonic on $\Omega$.
This in turn is equivalent to the following \emph{zero maximum principle}:
\begin{equation} \tag{ZMP}
z \leq 0\ \text{on $\de\Omega$} \implies z\leq 0\ \text{on $\Omega$}
\end{equation}
for all $z \in \USC(\barr \Omega) \cap \big(\call F(\Omega) + \tildee{\call F}(\Omega)\big)$.

We are now ready for the lemma, which states that comparison holds if the function $z$ in (ZMP) is the sum of a $\call F$-subharmonic and a $C^2$-smooth and strictly $\tildee{\call F}$-subharmonic. 
It is called \emph{definitional} comparison because it relies only upon the ``good'' definitions we gave for subharmonic and for subequations, which includes the negativity condition \eqref{(N)} which is important in the proof. It was stated and proven by Cirant, Harvey, Lawson, and Payne~\cite[Lemma 3.14]{chlp} in a context of constant coefficient subequations; we reproduce here its proof to highlight that it holds in the nonconstant coefficient case, too.

\begin{lem}[Definitional Comparison] \label{defcompa}
Let $\call F$ be a subequation and $u \in \USC(X)$.
\begin{enumerate}[label=\it(\roman*)]
\item	If $u$ is $\call F$-subharmonic on $X$, then the following form of the comparison principle holds for each bounded domain $\Omega \ssubset X$:
\[
u+v \leq 0\ \text{on}\ \de \Omega \implies u+v \leq 0\ \text{on}\  \Omega
\]
whenever $v \in \USC(\barr \Omega) \cap C^2(\Omega)$ is strictly $\tildee{\call F}$-subharmonic on $\Omega$.
\item	Conversely, suppose that for each $x\in X$ there is a neighborhood $\Omega \ssubset X$ of $x$ where the above form of comparison holds. Then $u$ is $\call F$-subharmonic on $X$. 
\end{enumerate}
\end{lem}

\begin{proof}
If the form of comparison in \emph{(i)} fails for some domain $\Omega \ssubset X$ and some regular strict $\tildee{\call F}$-subharmonic function $v$, then $u+v \in \USC(\barr \Omega)$ will have a positive maximum value $m>0$ at an interior point $x_0 \in \Omega$ and hence $\phi\defeq -v+m$ is $\scr C^2$ near $x_0$ and satisfies
\[
u-\phi \leq 0 \quad \text{near $x_0$, with equality at $x_0$}.
\]
Since $u$ is $\call F$-subharmonic at $x_0$, this implies that
\[
\call J^2_{x_0}(-v+m) = -\call J^2_{x_0}v + (m, 0,0) \in \call F_{x_0},
\]
which contradicts the property \eqref{(N)} for $\call F$ since $m>0$ and 
\[
\call J^2_{x_0}v \in \intr{\tildee{\call F}}_{\!x_0} = (-\call F_{x_0})\compl.
\]

Conversely, suppose that $u$ fails to be $\call F$-subharmonic at some $x_0\in X$. By the Bad Test Jet \Cref{l:btj}, there exist $\rho, \epsilon > 0$ and an upper contact jet $(p,A) \in J^{2,+}_{x_0} u$ such that $J=(u(x_0), p, A) \notin \call F_{x_0}$ and
\begin{equation} \label{btj}
u(x) - \phi_J(x) \leq -\epsilon|x-x_0|^2 \quad \text{on $B_\rho(x_0)$, with equality at $x_0$},
\end{equation}
where $\phi_J$ is the upper contact quadratic function for $u$ at $x_0$ with $\call J^2_{x_0}\phi_J = J$. Consequently, the function $-\phi_J$ is smooth and strictly $\tildee{\call F}$-subharmonic at $x_0$ and therefore, by choosing $\rho$ sufficiently small, the function $\tilde v \defeq -\phi_J + \epsilon \rho^2$ will be smooth and strictly $\tildee{\call F}$-subharmonic on $B_\rho(x_0)$.\footnote{Notice that one can make such a choice since there exist $r,R>0$ such that
\[
\call B_R\big(\call J^2_{x}(-\phi_J)\big) \subset \tildee{\call F}_{x} \quad \forall x \in B_r(x_0),
\]
and thus $\call J^2_{x}(-\phi_J+\epsilon\rho^2) \in \intr \tildee{\call F}_x$ for each $x \in B_\rho(x_0)$, provided that $\rho < \min \big\{ r, \sqrt{R/\epsilon} \big\}$.} Since reducing $\rho$ preserves the validity of (\ref{btj}), the form of comparison in \emph{(i)} fails for $\tilde v$ on $B_\rho(x_0)$ since
\[
u+\tilde v = 0 \ \text{on $\de B_\rho(x_0)$} \quad \text{but}\quad u(x_0)+\tilde v(x_0) = \epsilon \rho^2 > 0. \qedhere
\]
\end{proof}

\begin{remark}[Applying the definitional comparison] \label{appldefcompa}
Sometimes it is useful to prove the contrapositive of the form of comparison in part {\it(i)} of \Cref{defcompa} in order to conclude subharmonicity. That is to say, in order to show by {\it(ii)} that $u$ is subharmonic on $X$ one proves that, for each $x \in X$ there is a neighborhood $\Omega \ssubset X$ of $x$ where
\begin{equation} \label{appdefcompa}
(u+v)(x_0) > 0 \ \text{for some $x_0 \in \Omega$} \ \implies \ (u+v)(y_0) > 0 \ \text{for some $y_0 \in \de\Omega$}
\end{equation}
for every $v \in \USC(\barr \Omega) \cap C^2(\Omega)$ which is strictly $\tildee{\call F}$-subharmonic on $\Omega$. Conversely, one can also infer that the implication (\ref{appdefcompa}) holds whenever one knows that $u$ is subharmonic on $X$. In situations where we are interested in proving the subharmonicity of a function which is somehow related to a given subharmonic, this helps to close the circle (for example, see the proof of upcoming \Cref{elemprop}).
\end{remark}

The next tool is a collection of elementary properties shared by functions in $\call F(X)$, the set of $\call F$-subharmonics on $X$. They are to be found in \cite[Section~4]{hldir09} for pure second-order subequations, in \cite[Theorem 2.6]{hldir} for subequations on Riemannian manifolds, in \cite[Proposition D.1]{chlp} for constant-coefficient subequations. We place it here because by invoking the definitional comparison one can perform most of the proofs along the lines of Harvey--Lawson's ones in \cite{hldir09}. Indeed, we are going to use the definitional comparison in order to make up for the lack, for arbitrary subequations, of a result like \cite[Lemma 4.6]{hldir09}.

\begin{prop}[Elementary properties of $\call F(X)$] \label{elemprop}
Let $X \subset \R^n$ be open. For any subequation $\call F$ on $X$, the following properties hold:
\begin{itemize}[align=left,
%leftmargin=*,
left=5pt,
itemsep=2.5pt
]%[label=(\roman*), itemsep=2.5pt]
\item[(i:] \!\emph{\bf local}) \	$u \in \USC(X)$ locally $\call F$-subharmonic $\iff$ $u \in \call F(X)$;
\item[(ii:] \!\emph{\bf maximum}) \	$u,v \in \call F(X)$ $\implies$ $\max\{u,v\}\in \call F(X)$;
\item[(iii:] \!\emph{\bf coherence}) \ if $u \in \USC(X)$ is twice differentiable at $x_0\in X$, then
\[
\text{$u$ $\call F$-subharmonic at $x_0$}\ \iff\ \text{$\call J^2_{x_0} u \in \call F_{x_0}$;}
\]
\item[(iv:] \!\emph{\bf sliding}) \ $u\in \call F(X)$ $\implies$ $u-m \in \call F(X)$ for any $m >0$;
\item[(v:] \!\emph{\bf decreasing sequence}) \ $\{u_k\}_{k\in \mathbb{N}} \subset \call F(X)$ decreasing $\implies$ $ \lim_{k\to\infty} u_k \in \call F(X)$;
\item[(vi:] \!\emph{\bf uniform limits}) \ $\{u_k\}_{k\in \mathbb{N}} \subset \call F(X)$, $u_k \to u$ locally uniformly $\implies$ $u \in \call F(X)$;
\item[(vii:] \!\emph{\bf families locally bounded above}) \ if $\scr F \subset \call F(X)$ is a family of functions which are locally uniformly bounded above, then the upper semicontinuous envelope $u^*$ of the Perron function $u(\,\cdot\,) \defeq \sup_{w \in \scr F} w(\,\cdot\,)$ belongs to $\call F(X)$.\footnote{Recall that the \emph{upper semicontinuous envelope} of a function $g$ is defined as the function
\[
g^\ast (x) \defeq \lim_{r\dto 0}\sup_{y \in B_r(x)} g(y).
\]
It is immediate to see that the \emph{upper semicontinuous envelope operator} ${}^* \colon g \mapsto g^*$ is the identity on the set of all upper semicontinuous functions. Also, we called \emph{Perron function} the upper envelope of the family $\scr F$, since $\scr F$ is a family of subharmonics.}
\end{itemize}
Furthermore, if $\call F$ has constant coefficients, the following property also holds:
\begin{itemize}[align=left, leftmargin=*, left=5pt, itemsep=2pt]
\item[(viii:] \!\emph{\bf translation}) \ $u \in \call F(X)$ $\iff$ $u_y\defeq u(\cdot - y) \in \call F(X + y)$, for any $y \in \R^n$.
\end{itemize}
\end{prop}

\begin{proof}
\underline{\textit{(i)}}. It follows immediately from the local nature of the notion of subharmonicity.

\noindent\underline{\textit{(ii)}}. Let $w\defeq \max\{u,v\}$ and note that if $(p,A) \in J^{2,+}_x w$, then either $(p,A) \in J^{2,+}_x u$ or  $(p,A) \in J^{2,+}_x v$ since $w(x)$ is either $u(x)$ or $v(x)$ and, in general, $u(y), v(y) \leq w(y)$. Therefore, in any case, $(p,A) \in \call F_x$.

\noindent\underline{\textit{(iii)}}. It follows from the positivity condition \eqref{(P)}, thanks to the elementary fact that, for $u$ which is at least twice differentiable at $x$, if $(p,A)$ is an upper contact jet for $u$ at $x$, then $(p,A) = (Du(x), D^2u(x) + P)$ for some $P\geq 0$.

\noindent\underline{\textit{(iv)}}. It follows immediately from the negativity condition \eqref{(N)}.

\noindent\underline{\textit{(v)}}. We are going to use the \emph{definitional comparison}. Since the decreasing limit $u$ of $\{u_k\}_{k\in \mathbb N} \subset \USC(X)$ belongs to $\USC(X)$, by \Cref{defcompa} it suffices to show that for each $\Omega \ssubset X$, one has $u+v \leq 0$ on $\Omega$ for each $v \in \USC(\barr\Omega) \cap \scr C^2(\Omega)$ which is strictly $\tildee{\call F}$-subharmonic on $\Omega$ with $u+v \leq 0$ on $\de\Omega$.
For each $\epsilon >0$ fixed, consider the sets
\[
\mathfrak E_k \defeq (u_k+v)^{-1}([\epsilon, + \infty)) \cap \de\Omega.
\]
By the upper semicontinuity of each $u_k$ (and thus of each $u_k+v$), they are closed subsets of the compact $\de \Omega$, hence they are compact. Since $\{u_k\}_{k \in \mathbb N}$ is decreasing, they are nested; that is, $\mathfrak E_{k+1} \subset \mathfrak E_k$ for all $k \in \mathbb N$. Since $u+v \leq 0$ on $\de \Omega$, they have empty intersection:
\[
\bigcap_{k\in\mathbb{N}} \mathfrak E_k = (u+v)^{-1}([\epsilon, +\infty)) \cap \de \Omega = \emptyset.
\]
Therefore, by the well-known property of nested families of compact sets, the $\mathfrak E_k$'s must be eventually empty. This means that $u_k+ v < \epsilon$ on $\de \Omega$ for $k$ large and since $v - \epsilon$ is still a smooth and strict $\tildee{\call F}$-subharmonic by the sliding property, the definitional comparison yields $u_k + v \leq \epsilon$ on $ \Omega$. Hence $u+v \leq u_k + v \leq \epsilon$ on $\Omega$, for any $\epsilon >0$, which, on letting $\epsilon \dto 0$, gives $u+v \leq 0$ on $\Omega$ and thus $u\in \call F(X)$, again by definitional comparison.

\noindent\underline{\textit{(vi)}}. If $u_k$ converges to $u$ uniformly on the compact $\barr\Omega$ and we suppose $u+v \leq 0$ on $\de \Omega$ as above, then, for any $\epsilon > 0$ fixed, $u_k + v < \epsilon$ on $\de\Omega$ for $k$ large. The definitional comparison yields $u_k + v \leq \epsilon$ on $\Omega$, hence $u+v = (u-u_k) + (u_k +v) < 2\epsilon$ on $\Omega$ for $k$ large, where one exploits the uniform convergence to say that $u-u_k < \epsilon$ on $\Omega$ if $k$ is large. On letting $\epsilon \dto 0$, we have $u+v \leq 0$ on $\Omega$ and thus, by the converse implication of the definitional comparison, $u \in \call F(X)$.

\noindent\underline{\textit{(vii)}}. If $u^\ast +v \leq 0$ on $\de \Omega$, then $w+ v \leq 0$ on $\de \Omega$ for all $w \in \scr F$. By the implication \emph{(i)} of the definitional comparison, $w+ v \leq 0$ on $\Omega$ for all $w \in \scr F$ and thus $u+v \leq 0$ on $\Omega$. Since $v$ is smooth and the upper semicontinuous envelope operator preserves inequalities we conclude that $u^* + v = (u+v)^* \leq 0$ on $\Omega$ and the thesis follows from the implication \emph{(ii)} of the definitional comparison.

\noindent\underline{\textit{(viii)}}. This property holds only in a constant coefficient context; indeed, its validity for every $y\in \R^n$ is actually equivalent to the fiber $\call F_x$ being independent of the point $x\in X$. Indeed, since by the respective definitions $u_y(x+y) = u(x)$ and  $J^{2,+}_x u= J^{2,+}_{x+y} u_y$, one has, for $z \in X+y$, that $ \call J^{2,+}_z u_y \subset \call F_{z-y}$ and thus property \emph{(viii)} holds if and only if $\call F_{z-y} = \call F_z$.
\end{proof}
 
As far as coherence is concerned, property {\it(iii)} says that if we have a twice differentiable upper semicontinuous function that satisfies the condition~(\ref{coherence}) for all $x\in X$, then it is $\call F$-subharmonic. One might hope to have the same conclusion even if~(\ref{coherence}) is satisfied almost everywhere, with respect to the Lebesgue measure. The following result, stated in \cite{hlae},  shows that this is true for quasi-convex functions and was first established in the pure second-order case in \cite[Corollary 7.5]{hldir09}.

\begin{thm}[Almost Everywhere Theorem] \label{aet}
Suppose that $\call F$ is a primitive subequation on an open set $X\subset \R^n$ and
$u \colon X \to \R$ is a locally quasi-convex function. Then
\[
\call J_x^2u \in \call F_x\ \text{for almost all $x\in X$}\quad \iff\quad u\in \call F(X).
\]
\end{thm}

\begin{proof}
By \Cref{ae?} with $E \defeq \{x \in X:\ \call J^2_xu \in \call F_x\}$, we know that $\call J^{2,+}u \subset \barr{\call J(u,E)}$. By \eqref{(P)} one has $\call J(u,E) \subset \call F$ and by \eqref{(C)} also $\barr{\call J(u,E)} \subset \call F$. Hence by \Cref{subinclusion}, $u \in \call F(X)$.

Conversely, for every $P>0$, we know that $(Du(x), D^2u(x) + P)$ is an upper contact jet for $u$ at $x \in \Diff^2(u)$. Therefore in particular $\call J^2_x u + (0,0,\epsilon I) \in \call F_x$ for all $\epsilon >0$. On letting $\epsilon \dto 0$, since $\call F$ is closed, we get $\call J_x^2u \in \call F_x$ for all $x\in \Diff^2(u)$ and the conclusion follows, invoking Alexandrov's theorem.
\end{proof}

\begin{remark} We gave a proof based on Remarks \ref{subinclusion} and \ref{ae?}, which hide the combined use of the Jensen--S{\l}odkowski \Cref{jenslod} and of Alexandrov's \Cref{aleks:qc}. For slightly more explicit proofs one can see \cite{hlae} or \cite[Lemma 2.10]{chlp}, where the Bad Test Jet Lemma is also used.
\end{remark}
%The importance of the \emph{AE Theorem}, although it holds only for quasi-convex functions, lies in the fact that one can approximate subharmonics by quasi-convex functions that are still subharmonic on a smaller domain. This means that we may hope to be able to prove a result for quasi-convex subharmonics and then to use it in order to deduce an analogous one, where we drop the quasi-convexity assumption. 
%For instance, the following Subharmonic Addition \Cref{add} for quasi-convex functions will immediately follow from the AE \Cref{aet} and by \emph{quasi-convex approximation} a general Subharmonic Addition for constant coefficient subequations (\Cref{ccsa}) is deduced from it. 

In \cite{hlae}, Harvey and Lawson prove that an immediate consequence of the AE \Cref{aet} is that if we restrict our attention to quasi-convex suharmonics, then \emph{jet addition} always implies \emph{subharmonic addition}. This can be written as
\[
\call F + \call G \subset \call H \implies (\call F(X) \cap \mathsf{qc}_\text{loc}(X)) +  (\call G(X) \cap \mathsf{qc}_\text{loc}(X)) \subset \call H(X),
\] 
where we denoted by $\mathsf{qc}_\text{loc}(X)$ the space of all locally quasi-convex functions on $X$, and it is formalized as follows.

\begin{thm}[Subharmonic addition for quasi-convex functions]\label{add}
Suppose $\call F$ and $\call G$ are primitive subequations on an open set $X\subset \R^n$ and let $\call K \defeq \barr{\call F + \call G}$; suppose $u\in \call F(X)$ and $v\in \call G(X)$ are locally quasi-convex functions on $X$. Then $u+v \in \call K(X)$.
\end{thm}

\begin{proof}
By the Almost Everywhere \Cref{aet}, $\call J^2_x u \in \call F_x$ and $\call J^2_x v \in \call G_x$ almost everywhere, hence $\call J^2_x(u+v) = \call J^2_x u + \call J^2_x v \in \call K_x$ almost everywhere. The converse implication of the same theorem now yields $u+v \in \call K(X)$. 
\end{proof}

\begin{remark}
This is sufficient in order to say that if $\call F + \call G \subset \call H$ for some primitive subequation $\call K$, then $u+v \in \call H(X)$ whenever $u\in \call F(X)$ and $v\in\call G(X)$ are locally quasi-convex, since $\call K$ is the minimal primitive subequation containing $\call F + \call G$. 
 Indeed, the (fiberwise) sum satisfies \eqref{(P)}, therefore so does its closure. 
 \end{remark}
 
 \begin{remark}
Taking the closure of the sum is necessary in order to assure that property \eqref{(C)} holds. For instance, let us consider $\call F$ and $\call G$ induced by the two equations $F(u) = 0$ and $G(u) = 0$ on $\R^n$, with 
\[
F(r) \defeq \chi_{[0, +\infty)}(r) \sin^2(\pi r^2) - \tfrac34 \quad \text{and} \quad G(r) \defeq \chi_{(-\infty, 0]}(r) \sin^2(\pi (r+\alpha)^2)  - \tfrac34,
\]
for some $\alpha \in (\frac13, \frac23)$. Then it is easy to see that $(0,0,0,0) \notin \call F + \call G$, yet it is a limit point, since for every positive integer $k$, we have that  $\sqrt{1/3 + k} - \sqrt{2/3 - \alpha + k} = O\big(1/\sqrt k\big)$ belongs to the $\R$-component of each jet in $\call F + \call G$.
\end{remark}

With respect to \Cref{add} on subharmonic addition, two observations are in order. The former contains a reformulation of the Summand Theorem \ref{pusc:sum} which provides a stronger version of \Cref{add}, the latter points out, as noted in \cite{hlae}, that the quasi-convexity assumption on the subharmonics is not needed in the constant coefficient case.

\begin{remark}[\emph{A stronger form of subharmonic addition}] \label{rmkone}
It is worth noting that the Summand Theorem \ref{pusc:sum} leads to a slightly stronger form of \Cref{add}.
Indeed, it tells us that if $(p,A)$ is an upper contact jet for the sum $w = u+v$ at $x$, then there exist two jets 
\[
(x, u(x), Du(x), B') \in \barr{\call J^{2,+}u}, \quad (x, v(x), Dv(x), C) \in \barr{\call J^{2,+}v}
\]
 which sum to $(w(x), p, A)$. To see this, note that the sequence $\{x_j\}$ can be selected in such a way that $Du(x_j) \to Du(x)$ and $Dv(x_j) \to Dv(x)$, according to~\Cref{pusc}; hence it is sufficient to take $B' = A + P$, where  $P = A - B - C \geq 0$. To write this succinctly,
\[
\call J^{2,+}u + \call J^{2,+}v \subset \call J^{2,+}(u+v) \subset \barr{\call J^{2,+}u} + \barr{\call J^{2,+}v},
\]
where the former inclusion is trivial and the latter is the thesis of the proposition.

In other words,  \Cref{pusc:sum} assures that every upper contact $2$-jet of the sum $u+v$ of two quasi-convex functions can be represented as the sum of two $2$-jets of $\call F$ and $\call G$, respectively, hence $u+v \in \call F + \call G\,(X)$. This will be used in the proof of the strict comparison for quasi-convex functions (\Cref{scqc}).
\end{remark}

\begin{remark}
Furthermore, if $\call F$ and $\call G$ are constant coefficients subequations, then the assumption that $u$ and $v$ are quasi-convex can be dropped; therefore we have 
\[
\call F(X) + \call G(X) \subset \barr{\call F + \call G}\,(X).
\]
This follows from a quasi-convex approximation technique that was introduced in \cite{hldir09} in a context of pure second-order subequations and that works fine also for general constant coefficient subequations.
%We here state another property of the supconvolution that is useful in order to deduce results for semicontinuous functions from their counterparts for quasi-convex functions, as we said. 
\end{remark}

As we anticipated in the previous part, the supconvolution is a nice way to approximate upper semicontinuous functions by a quasi-convex \emph{regularization} of them (see \Cref{baspropsc} for four basic properties of the supconvolution).
The next proposition, imitating \cite[Theorem~8.2]{hldir09}, shows that if the take the supconvolution of a bounded subharmonic, then it remains subharmonic, provided we are far enough from the boundary of the domain $X$.

\begin{prop}[Quasi-convex approximation] \label{prop:approx}
Let  $\call F$ be a constant coefficient subequation and suppose that $ u \in \call F(X)$ is bounded. For $\epsilon >0$, let $\delta \defeq 2\sqrt{\epsilon \sup_X\! |u|}$, and $X_\delta \defeq \{ y \in X:\ d(y, \de X) > \delta\}$. Then $u^\epsilon \in \call F(X_\delta)$.
\end{prop}

\begin{proof}
Since $\call F$ is constant coefficient we have that $u_z \defeq u(\cdot - z) \in \call F(X_\delta)$ for all $z \in B_\delta$; indeed it is easy to see that $\call J^{2,+}_x u_z \subset \call F_{x-z} = \call F_x$ for every $x \in X_\delta$. If we now take
\[
\scr F \defeq \bigg\{ u_z - \frac{1}{2\epsilon}|z|^2:\ z \in B_\delta \bigg\},
\]
then $\scr F \subset \call F(X_\delta)$ by the negativity condition \eqref{(N)} and, by (\ref{supcball}), its upper envelope is $u^\epsilon$, and it coincides with $(u^\epsilon)^\ast$ because it is quasi-convex, hence upper semicontinuous. Therefore we have $u^\epsilon \in \call F(X_\delta)$, by the families-locally-bounded-above property of \Cref{elemprop}.
\end{proof}

We can now show how the quasi-convex approximation technique works by deducing a Subharmonic Addition Theorem for constant coefficient subequations from its quasi-convex counterpart \Cref{add}; this result can be found in \cite[Theorem~7.1]{chlp}, with basically the same proof we propose here.

\begin{cor}[Subharmonic addition for constant coefficient subequations] \label{ccsa}
Suppose $\call F$ and $\call G$ are constant coefficient subequations. Then, for any open set $X\subset \R^n$, jet addition implies subharmonic addition; that is,
\[
\call F + \call G \subset \call H \implies \call F(X) + \call G(X) \subset \call H(X).
\]
\end{cor}

\begin{proof}
Let $u\in \call F(X)$, $v \in \call G(X)$. Since the result is local, it is sufficient to show that, for each $x\in X$, one has $u+v \in \call H( U)$, where $ U$ is any neighborhood of $x$. Hence by upper semicontinuity we may assume that $u$ and $v$ are bounded above. Also, we may assume that $u$ and $v$ are bounded below, since it suffices to consider, for any positive integer $m$, the subharmonics $u \vee (\phi -m) \defeq \max\{u, \phi - m\} \in \call F( U)$ and $v \vee (\psi -m) \defeq \max\{v, \psi- m\} \in \call G( U)$, for some bounded quadratic functions $\phi \in \call F( U)$ and $\psi \in \call G( U)$, and then take the limit $m\to \infty$. We have already highlighted in \Cref{exquadsub} that the existence of such quadratic functions is guaranteed by the topological property \eqref{(T)} and that their translations by negative constants are still subsolutions by the sliding property yielded by the properness condition \eqref{(N)}, while the \emph{truncated} functions are subharmonic thanks to the \emph{maximum property} of constant coefficient subequation. With these assumptions, by \Cref{prop:approx} there exist sequences $\{u_j\}$, $\{v_j\}$ of quasi-convex subharmonics near $x$ that converge monotonically downward to $u, v$  respectively, as $j \to \infty$. By \Cref{add} we know that $u_j + v_j \in \call H(X)$ and by the \emph{decreasing sequence property} we conclude $u+v \in \call H(X)$. 
\end{proof}

\begin{remark}
One of the limitations of this quasi-convex approximation technique is that it works fine in order to deduce results from their quasi-convex counterparts in a constant-coefficient setting. Indeed, it exploits the translation property, which we saw one loses when passing to nonconstant coefficients. An adaptation of this technique to a nonconstant coefficient context will require further assumptions on the subequation, such as a continuous dependence of the fiber on the point for gradient-free subequations (see~\cite{cpaux, cpmain}), and an additional \emph{directional monotonicity} in the gradient variable for generic subequations (see~\cite{cprdir}).
\end{remark}


\section{Comparison principles}\label{sec:comp}

We recall that before stating the definitional comparison we rewrote the comparison principle on $\Omega \ssubset X$
\begin{equation} \tag{CP}
\begin{cases}
\text{$u$ $\call F$-subharmonic, $v$ $\call F$-superharmonic on $X$} \\
u \leq v\ \text{on $\de\Omega$}
\end{cases}
\implies u\leq v\ \text{on $\Omega$}
\end{equation}
as the zero maximum principle for sums
\begin{equation} \label{eq:ZMP} \tag{ZMP}
\begin{cases}
w \in \call F(X) + \tildee{\call F}(X) \\
w \leq 0\ \text{on $\de\Omega$}
\end{cases}
\implies w\leq 0\ \text{on $\Omega$},
\end{equation}
by exploiting the fact that $v$ is $\call F$-superharmonic if and only if $-v \in \tildee{\call F}(X)$.
Therefore it appears natural that a key role in the proof of comparison could be played by a subharmonic addition theorem, so that if one knows that
\[
\call F + \tildee{\call F} \subset \call G,
\]
then
\begin{equation*} \label{subhadd:f}
\call F(X) + \tildee{\call F}(X) \subset \call G(X);
\end{equation*}
this is indeed useful if $\call G$ does not depend on the specific $\call F$ (but, for instance, only on its monotonicity properties) and if we are able to say that comparison holds for $\call G$-subharmonics (for instance, via a characterization of them).

In order to find a good $\call G$, note that if
\begin{equation} \label{1moncon}
\call F + \call M \subset \call F,
\end{equation}
then by properties \emph{(1)}, \emph{(2)} and \emph{(3)} of \Cref{propdual} one has, for each $J \in \call F_x$,
\[
J + \call M_x \subset \call F_x \implies \tildee{\call F}_x \subset \tildee{\call M}_x - J,
\]
hence
\begin{equation*}\label{jetadd:f}
\call F + \tildee{\call F} \subset \tildee{\call M}.
\end{equation*}
and one may choose $\call G = \tildee{\call M}$.
Also, one knows that such a subset $\call M$ exists; indeed, regardless of the subequation, thanks to the conditions \eqref{(P)} and \eqref{(N)} a set that satisfies (\ref{1moncon}) is
\[
\call M = \call N \times \{0\} \times \call P,
\]
where 
\begin{equation} \label{def:np}
\call N \defeq \{r \in \R:\ r\leq 0\} \quad \text{and} \quad \call P \defeq \{A \in \Sc(n):\ A \geq 0 \}.\footnote{That is, $\call P \defeq \lambda_1^{-1}(\R_{\geq 0})$, where $\lambda_1 \colon \Sc(n) \to \R$ is the \emph{minimal eigenvalue operator}.}
\end{equation}
Note that $\call M$ is a primitive subequation satisfying \eqref{(N)}, but it is not a subequation, since it has empty interior because of the factor $\{0\}$, which is due to the lack of monotonicity assumptions with respect to the gradient variable $p$.

At this point the only missing piece would be to prove that comparison holds for functions in $\tildee{\call M}(X)$ where one should use a \emph{monotonicity cone subequation $\call M$} which is as large as possible in order to facilitate the proof of the ZMP for $\tildee{\call M}(X)$, which becomes smaller as $\call M$ increases. This represents the main observation to the monotonicity--duality method of Harvey and Lawson, as discussed in \cite{chlp}. First, we will present a pair of significant situations with constant coefficients coming from \cite{hldir09, chlp} for which the method succeeds. The reader is invited to consult \cite[Theorem~7.5]{chlp} for a more general result. Then, beginning in the next section, we will discuss some results in the variable coefficient setting.

\subsection{Comparison with constant coefficients and sufficient monotonicity} \label{sec:cccc} 

Let us introduce two basic examples of classes of subequations for which one has that comparison holds, thanks to the fact that characterizations of the sets $\tildee{\call M}(X)$ we are going to consider are known \cite{hldir09,chlp} and consequently one is able to prove that $\tildee{\call M}$-subharmonics satisfy (ZMP).

%\subsubsection{Pure second-order constant coefficient subequations} 
%
%Here is a formal definition.
 
\begin{definition} \label{psos}
A \emph{pure second-order constant coefficient subequation} is a constant coefficient primitive subequation of the form $\call F = \R \times \R^n \times \call A$, for some $\call A \subset \Sc(n)$.
\end{definition}

\begin{remark} \label{identify}
We used \Cref{rmkccs} to identify $\call F$ with a subset of $\R \times \R^n \times \Sc(n)$.
Also, since the only component of a jet that is actually relevant to define a subsolution is the one living in $\Sc(n)$, we shall identify $\call F$ with $\call A\subset \Sc(n)$ that satisfies the positivity condition. Hence we can say that a pure second-order constant coefficient subequation $\call A$, which must satisfy the condition of positivity, is $\call P$-monotone; that is,
\[
\call A + \call P \subset \call A\,.
\]

\end{remark}

Given $A\in \Sc(n)$ we denote by $\lambda_1(A)$ and $\lambda_n(A)$ its minimum and its maximum eigenvalue, respectively (that is, we order the eigenvalues $\lambda_1 \leq \lambda_2 \leq \cdots \leq \lambda_n$). The most basic of all examples is the \emph{convex} subequation (cf.\ (\ref{def:np}))
\[
\call P \defeq \{ A \in \Sc(n):\ \lambda_1(A) \geq 0 \},
\]
whose dual subequation is the \emph{co-convex} or \emph{subaffine} subequation
\[
\tildee{\call P} \defeq \{ A \in \Sc(n):\ \lambda_n(A) \geq 0 \}.
\]
The name \emph{subaffine} comes from the fact, proved in~\cite[Proposition 4.5]{hldir09}, that an upper semicontinuous function on $X$ is a subsolution of $\tildee{\call P}$ if and only if it is \emph{subaffine} on $X$; that is
\begin{equation} \label{p=sa}
\tildee{\call P}(X) = \mathrm{SA}(X),
\end{equation} 
where $\mathrm{SA}(X)$ is the space of all subaffine functions on $X$, in the sense of the following definition.

\begin{definition}
A function $w\in \USC(X)$ is called \emph{subaffine on $X$}, and we write $w \in \mathrm{SA}(X)$,  if, for every open subset $\Omega \ssubset X$ and each affine function $a \in \mathrm{Aff}$,
\[
w \leq a\ \text{on $\de \Omega$} \implies  w \leq a\ \text{on $\Omega$}.
\]
\end{definition}

\begin{remark}
We wrote the space of all affine functions as $\mathrm{Aff}$, without specifying their domain because it is known that any $a$ which is affine on an open subset $\Omega$ has a unique extension to the whole space $\R^n$ (see, e.g., \cite{vesely}).
\end{remark}

%It is also known that, for every pure second-order subequation $\call F\subset \Sc(n)$, one has 
%\begin{gather*}
%\call F + \call P \subset \call F  \iff \tildee{\call F} + \call P \subset \tildee{\call F}, \\
%\call F + \tildee{\call F} \subset \tildee{\call P}.
%\end{gather*}
%This follows from~\Cref{propdual} and~\Cref{invert}, since the first line is a consequence of property \emph{(4)} and the second one is equivalent to $\call F + \call P \subset \call F$, which is in turn equivalent to the condition (P). Indeed, for every $J \in \call F$ one has
%\[
%J + \call P \subset \call F \iff \tildee{\call F} \subset \tildee{J + \call P} = \tildee{\call P} - J.
%\]

Note that $\call P$ is a monotonicity cone subequation for all pure second-order constant coefficient subequations (and it is minimal in this sense), hence a straightforward application of the Subharmonic Addition Theorem (\Cref{ccsa}) yields the following result. For a proof that uses S{\l}odkowski's Largest Eigenvalue Theorem and quasi-convex approximation, the reader can instead refer to \cite{hldir09}.

\begin{thm}[{Subaffine Theorem; \cite[Theorem 6.5]{hldir09}}] \label{t:subaff}
If $\call F$ is a pure second-order constant coefficient subequation, then for any open set $X \subset \R^n$
\[
\call F(X) + \tildee{\call F}(X) \subset \mathrm{SA}(X).
\]
\end{thm}

At this point it is evident that comparison holds, provided that one has the following strengthening of the property of subaffinity.

\begin{lem} \label{str:sub}
Let $\Omega \subset \R^n$ a domain and $w\in \USC(\barr\Omega)\cap \mathrm{SA}(\Omega)$. Then
\[
w \leq a\ \text{on $\de \Omega$} \implies w \leq a\ \text{on $ \Omega$}
\]
for all $a \in \mathrm{Aff}$.
\end{lem}

\begin{proof}
Assume without loss of generality that $a=0$; exhaust $\Omega$ by compact sets $K_1 \subset K_2 \subset \cdots$ and set $U_\delta \defeq \{ x\in \barr \Omega:\ w(x) < \sup_{\de\Omega} w + \delta\}$, for any $\delta > 0$. Since $w\in \USC(\barr\Omega)$, we know that $U_\delta$ is an open neighborhood of $\de\Omega$ in $\barr \Omega$. Therefore $\de K_j \subset U_\delta$ for all  $j$ sufficiently large and, since $w$ is subaffine, $\sup_{K_j} \leq \sup_{\de \Omega} w + \delta$.
 This proves that $\sup_{\Omega} w \leq \sup_{\de \Omega} w + \delta$; since $\delta$ is arbitrary, $w\leq 0$ in $\de\Omega$ and $w$ is continuous, we conclude that $w \leq 0$ on $\barr\Omega$, which is the desired conclusion.
\end{proof}

\begin{thm}[Comparison for pure second-order subequations]
Let $\call F$ be a pure second-order constant coefficient subequation. Then comparison holds on every domain $\Omega \subset \R^n$:
\[
u+v \leq 0\ \text{on $\de \Omega$} \implies u+v \leq 0\ \text{on $ \Omega$}
\]
if $u \in \USC(\barr \Omega) \cap \call F(\Omega)$ and $v \in \USC(\barr \Omega) \cap \tildee{\call F}(\Omega)$.
\end{thm}

\begin{proof}
It follows from~\Cref{str:sub} with $a=0$, since $u+v$ is a subaffine function by the Subaffine \Cref{t:subaff}.
\end{proof}

%\subsubsection{Gradient-free constant coefficient subequations}

The case of pure second-order constant coefficient subequations is a subset of the one we are going to discuss now, yet we considered \emph{historically significant} to introduce them separately. Again we will show that comparison holds.

\begin{definition} \label{gfs}
A \emph{gradient-free constant coefficient subequation} is a closed subset of $\R \times \R^n \times \Sc(n)$ of the form $\call F = \bigcup_{\call R, \call A} \call R \times \R^n \times \call A$, for some $\call R \subset \R$ and $\call A \subset \Sc(n)$, satisfying the conditions of positivity and negativity; that is, a gradient-free constant coefficient subequation can be identified with a closed subset $\call G \subset \R \times \Sc(n)$ that satisfies 
\[
(r,A) \in \call G \implies (r+s, A+P) \in \call G \quad \forall s< 0,\ \forall P> 0.
\]
\end{definition}

\begin{remark}
The reader could have noticed that a gradient-free constant coefficient subequation is a constant coefficient primitive subequation that also satisfies \eqref{(N)}, or, equivalently, a constant coefficient subequation that does not need to satisfy \eqref{(T)}. As said in \Cref{identify}, we used \Cref{rmkccs} to identify $\call F$ with a subset of $\R\times \R^n \times \Sc(n)$ and the fact that the gradient component is irrelevant to identity $\call F$ with a subset of $\R \times \Sc(n)$. 
\end{remark}

The most basic example is the subequation
\[
\call Q \defeq \call N \times \call P,
\] 
whose dual subequation is the \emph{subaffine-plus} subequation
\[
\tildee{\call Q} = \big(\call N \times \Sc(n)\big) \cup \big(\R \times \tildee{\call P}\big).
\]

Also in this case the name comes from a characterization of dual subharmonics: by \cite[Theorem 10.7]{chlp}, an upper semicontinuous function on $X$ is $\tildee{\call Q}$-subharmonic if and only if it is \emph{subaffine-plus} on $X$; that is,
\[
\tildee{\call Q}(X) = \mathrm{SA}^+(X),
\] 
where $\mathrm{SA}^+(X)$ is the space of all subaffine-plus functions on $X$, which are defined as follows.

\begin{definition}
For $K\subset \R^n$ compact, we denote by
\[
\mathrm{Aff}^+(K) \defeq \{ \restr{a}{K}:\ a \in \mathrm{Aff},\ a \geq 0\ \text{on $K$} \}
\]
the space of \emph{affine-plus functions on $K$}.
\end{definition}

\begin{definition}
A function $w\in \USC(X)$ is called \emph{subaffine-plus on $X$}, and we write $w \in \mathrm{SA}^+(X)$,  if, for every open subset $\Omega\ssubset X$,
\[
w \leq a_+\ \text{on $\de \Omega$} \implies  w \leq a_+\ \text{on $\Omega$} \qquad
\forall a_+\! \in \mathrm{Aff}^+(\barr\Omega).
\]
\end{definition}


Similarly to the pure second-order case, $\call Q$ is a monotonicity cone subequation for all gradient-free subequations (and it is minimal in this sense), hence it is now immediate to see that, by~\Cref{ccsa}, we have an analogous of the Subaffine Theorem.

\begin{thm}[{Subaffine-Plus Theorem; \cite[Theorem 10.8]{chlp}}]
If $\call F$ is a gradient-free constant coefficient subequation, then for any open set $X \subset \R^n$
\[
\call F(X) + \tildee{\call F}(X) \subset \mathrm{SA}^+(X).
\]
\end{thm}
 

Also, since $0\in \mathrm{A}^+(\R^n)$, we have comparison. We omit the proof, which is essentially the same as in the case of pure second-order subequations. 

\begin{thm}[Comparison for gradient-free constant coefficient subequations]
Let $\call F$ be a gradient-free constant coefficient subequation. Then comparison holds on every domain $\Omega \subset \R^n$:
\[
u+v \leq 0\ \text{on $\de \Omega$} \implies u+v \leq 0\ \text{on $ \Omega$}
\]
if $u \in \USC(\barr \Omega) \cap \call F(\Omega)$ and $v \in \USC(\barr \Omega) \cap \tildee{\call F}(\Omega)$.
\end{thm}

\subsection{Strict comparison for quasi-convex functions}

Another situation where one has sufficient monotonicity in order to prove comparison is the following. If one considers a variable coefficient subequation, then its maximal monotonicity cone is, \emph{a priori},
\[
\call N \times \{0\} \times \call P;
\]
the price to pay in order for this to be enough is to focus only on \emph{quasi-convex} subharmonics and to add a \emph{strictness} assumption; namely, we give the following notion of \emph{strong strictness}.

\begin{definition} \label{def:strictv}
We say that $u \in \USC(X)$ is \emph{strongly strictly $\call F$-subharmonic on $X$} if $u \in \call G(X)$ for some subequation $\call G \subset \intr{\call F}$. We will write $u \in \call F_{\rm strict}(X)$.
\end{definition}

Note that this notion is in general stronger than the one of \Cref{def:strict} also for classical subsolutions.

\begin{example}
For $n=1$ and $X=(0,1)$, consider $F(x,p) = x-p$. Then $u(x) = \frac{x^3}3$ is a strict smooth subsolution on $X$, according to \Cref{def:strict}, but it is not strongly strict in the viscosity sense of \Cref{def:strictv}. 
\end{example}

Harvey and Lawson~\cite{hlae} showed that one always has the \eqref{eq:ZMP} form of comparison if the subsolutions are quasi-convex and one of them is strongly strict, just like the Definitional Comparison \Cref{defcompa} tells that \eqref{eq:ZMP} holds, even without the quasi-convexity assumption, if one of the subsolutions is strict and smooth.

\begin{thm}[Strict comparison for quasi-convex functions] \label{scqc}
Let $\Omega \ssubset \R^n$ be a bounded domain and let $\call F$ be a subequation on $\Omega$.  Then strict comparison holds for quasi-convex functions:
\[
u+v \leq 0\ \text{on $\de\Omega$} \implies u+v \leq 0\ \text{on $\Omega$}.
\]
if $u \in \USC(\barr\Omega) \cap \call F_{\rm strict}(\Omega)$ and $v \in \USC(\barr\Omega) \cap \tildee{\call F}(\Omega)$ are locally quasi-convex on $\Omega$.
\end{thm}

\begin{proof}
Suppose, by contradiction, that strict comparison fails on $\Omega \ssubset \R^n$. Then $w= u+v$ must have a positive interior maximum $m\defeq w(x_0) > 0$ at $x_0 \in \Omega$. Hence $w$ has $(0,0)$ as an upper contact jet at $x_0$ and the Theorem \ref{pusc:sum} on summands yields
\[
(u(x_0), p, A) \in \call G_{x_0} \qquad \text{and} \qquad (v(x_0), q, B) \in \tildee{\call F}_{x_0}, 
\]
with $p+q = 0$ and $A+B = -P \leq 0$. This contradict $\call G_{x_0} \subset \intr{\call F}_{x_0}$, since by positivity and negativity
\[
-(v(x_0), q, B) = (u(x_0)-m, p, A+P)  \in \call G_{x_0},
\]
but $-(v(x_0), q, B) \notin \intr{\call F}_{x_0}$ by the definition of dual subequation.
\end{proof}

\begin{example}
\Cref{scqc} is indeed useful if we may assure \emph{a priori} that subsolutions and \emph{dual subsolutions} of a given equation are quasi-convex. For instance, let us consider the Monge--Ampère equation
\begin{equation}\label{maex}
\det D^2 u = f
\end{equation}
for some $f\in \LSC(\barr \Omega)$, positive and bounded above. If we look for convex solutions, we immediately see that they are strongly convex, in the sense that the minimum eigenvalue $\lambda_1({D^2u}) \geq C$ on $\Omega$, for some $C>0$; indeed by semicontinuity and compactness we know that $f \geq c$ on $\Omega$, for some $c > 0$, while $\det D^2u \leq M$ for some upper bound $M>0$ for $f$. If we therefore restrict ourselves to strongly convex sub-- and supersolutions, we trivially have that subsolutions are quasi-convex. Also, since the fibers of the subequation associated to (\ref{maex}) are
\[
\call F_x = \R \times \R^n \times \big\{ A \in \Sc(n):\ \det A - f(x) \geq 0 \big\}, 
\]
we have that
\[
\tildee{\call F}_x = \R \times \R^n \times \big\{ B \in \Sc(n):\ \det (-B) - f(x) \leq 0 \big\}
\]
and the strong convexity assumption on the supersolutions leads us to require that the matrices $B$ are negative definite with their maximum eigenvalue $\lambda_n (B ) \leq \ell$ for some $\ell < 0$ independent of $x$. Therefore we see that
\[
-\lambda_1(B)\cdot(-\ell)^{n-1} \leq \det(-B) \leq M \quad\implies\quad  \lambda_1(B) \geq -\Lambda \defeq \frac{-M}{(-\ell)^{n-1}},
\]
hence the dual subsolutions we are interested in are $\Lambda$-quasi-convex.
\end{example}

Again by quasi-convex approximation, the assumption that $u$ and $v$ are quasi-convex can be dropped if $\call F$ and $\call G$ are constant coefficient subequations.  Alternatively, this is proven in \cite[Corollary~C.3]{hldir} using the Theorem on Sums.

% The proof by quasi-convex approximation we are going to propose here is of course longer and appears more technical than Harvey--Lawson's one; nevertheless, the Theorem on Sums, which certainly is not a trivial result, hides most of the technicalities in proofs that exploits it. In this sense, our proof could be considered more elementary, as it explicitly uses only some basic properties of quasi-convex functions and constant coefficient subequations, yet we point out that it still needs all the nontrivial instruments (Alexandrov, Jensen) required in order to prove the Theorem on Sums: in our case, they are hidden in \Cref{pusc:sum}, which is used in the proof of \Cref{scqc}.
% 
\begin{cor}[Strict comparison for constant coefficient subequations] \label{sccc}
Let $\call F$ be a constant coefficient subequation.  Then the strict comparison holds on every bounded domain $\Omega \ssubset \R^n$:
\[
u+v \leq 0\ \text{on $\de\Omega$} \implies u+v \leq 0\ \text{on $\Omega$}
\]
 if $u \in \USC(\barr\Omega) \cap \call F_{\rm strict}(\Omega)$ and $v \in  \USC(\barr\Omega) \cap\tildee{\call F}(\Omega)$.
 \end{cor}


\begin{remark}
We believe that ``any'' proof of the above result trying to exploit the quasi-convex approximation technique we depicted ends up using the same instruments around which the proof of the Theorem on Sums is built. Indeed, proving \Cref{sccc} via \Cref{scqc} would hide the use of Alexandrov's theorem and of the Jensen--S{\l}odkowski Theorem, which are combined in the proof of \Cref{pusc}, yielding \Cref{pusc:sum}. 
Also, if one wishes to use \Cref{prop:approx} in order to make the subharmonics quasi-convex (and still subharmonic) in order to apply \Cref{scqc}, the problem is that an additional hypothesis of boundedness of the subsolutions could be needed. Note that one cannot simply truncate the functions from below, say by defining $\tilde u \defeq \max\{u, -m\}$, $m>0$, since in general (and this is the case) the subharmonicity is not preserved if one takes the maximum of a subharmonic and a nonsubharmonic. The standard method of showing that one can assume, without loss of generality, that subsolutions are also bounded from below is the method used in the proof of \Cref{ccsa}. More precisely, one defines $\tilde u \defeq \max\{u, \phi-m\}$, where $\phi$ is any bounded subharmonic, so that the subharmonicity of $\tilde u$ is guaranteed by the maximum property. However, for comparison this method encounters the difficulty that bounded subharmonics are in general only known to locally exist (see \Cref{exquadsub}), while the global information $u+v\leq 0$ on $\de\Omega$ should be somehow preserved.

Therefore, roughly speaking, a proof of comparison via quasi-convex approximation should show that one can assume without loss of generality that the subsolutions are bounded (for instance, by proving the existence of a bounded global subharmonic) or find a way to carry the global information of the behaviour of $u+v$ on $\de\Omega$ into a neighborhood of some given point (for instance, a point in $\Omega$ at which $u+v$ assumes its positive maximum, if one is proceeding by contradiction). 
If instead one tries to work only locally as in the proof of \Cref{scqc}, then the most immediate path probably leads to a proof with the flavour of the Theorem on Sums, since it is likely that one eventually ends up needing results like \Cref{pusc:sum} and \Cref{magprop}.

Hence, in our opinion, Harvey--Lawson's proof of \Cref{sccc} via the Theorem on Sums should be considered the most natural proof via quasi-convex approximation. An alternative proof might be more explicit, but not more elementary.
On the other hand, if one knows that the subsolutions are bounded (or that there exist a bounded $\call G$-subharmonic and a bounded $\tildee{\call F}$-subharmonic), then \Cref{magprop} is not needed and thus a slightly more elementary proof is possible.
\end{remark}

\begin{proof}[Proof of \Cref{sccc} if there exist bounded $\call G$-- and $\tildee{\call F}$-subharmonics] 
By contradiction, suppose that comparison does not hold for some $\Omega \ssubset \R^n$; hence there exists $\barr x \in \Omega$ such that $u(\barr x) + v(\barr x) = M > 0$. For $\eta \in (0, M)$ fixed,  by the sliding property, $\hat u \defeq u-\eta \in \call G(\Omega)$ and we still have $\hat u(\barr x) + v(\barr x) > 0$. 
Let $\Phi \in \call G(\Omega)$, $\Psi \in \tildee{\call F}(\Omega)$ be bounded on $\barr\Omega$, and define, for $m>0$ integer, $u_m \defeq \max\{ \hat u, \Phi - m\}$ and $v_m \defeq \max\{ v, \Psi - m\}$.
Note that $u_m$ and $v_m$ are bounded on $\barr\Omega$, $u_m \dto u$ and $v_m \dto v$ pointwise as $m \to +\infty$, and $u_m(\barr x) + v_m(\barr x) \geq \hat u(\barr x) + v( \barr x) > 0$ for all $m \in \mathbb{Z}_+$; also, $u_m \in \call G(\Omega)$ and $v_m \in \tildee{\call F}(\Omega)$, by the maximum property.
 Quasi-convex approximation (\Cref{prop:approx}) yields the supconvolutions $u_m^\epsilon \in \call G(\Omega_\delta)$ and $v_m^\epsilon \in \tildee{\call F}(\Omega_\delta)$,  where $\delta = \delta(\epsilon) \defeq 2\sqrt{\epsilon \max\{\max_{\barr{\Omega}}|u_m|, \max_{\barr{\Omega}}|v_m|\}}$; they satisfy $u_m^\epsilon(\barr x) + v_m^\epsilon(\barr x) \geq u_m(\barr x) + v_m(\barr x) > 0$ for any small $\epsilon > 0$.
Therefore by the strict comparison for quasi-convex functions (\Cref{scqc}), for any $\epsilon >0$ so small that $\barr x \in \Omega_\delta$, there exists a point $y_\epsilon$ such that 
 \begin{equation} \label{disban}
 u_m^\epsilon(y_\epsilon) + v_m^\epsilon(y_\epsilon) > 0,\quad y_\epsilon\in \de\Omega_\delta.
 \end{equation}
 Consider now an infinitesimal sequence $\epsilon_k \dto 0$. By the compactness of $\barr{\Omega}$, up to a subsequence, $y_k \defeq y_{\epsilon_k} \to \barr y \in \de \Omega$ as $k\to\infty$. Let us call $\delta_k = \delta(\epsilon_k)$ and note that by (\ref{disban}) we have
\[
u_m^{\epsilon_k}(y_k) + v_m^{\epsilon_k}(y_k) > 0, \quad y_k \in \de\Omega_{\delta_k} \qquad \forall\, k\in \mathbb{N}.
\]
Also, by (\ref{supcball})
\[
u_m^{\epsilon_k}(y_k) \leq \max_{\barr B_{\delta_k}(y_k)} u_m \qquad  \forall k\in \mathbb{N};
\]
since $\delta_k \dto 0$ and $y_k \to \barr y \in \de\Omega$, for $\rho > 0$ fixed we have $B_{\delta_k}(y_k) \subset B_\rho(\barr y)$ if $k$ is large enough, hence
\[
\limsup_{k\to\infty} u_m^{\epsilon_k}(y_k) \leq \max_{\barr B_\rho( \barr y)} u_m \qquad \forall \rho > 0
\]
and on letting $\rho \dto 0$ 
\[
\limsup_{k\to\infty} u_m^{\epsilon_k}(y_k) \leq \limsup_{z \to  \barr y} u_m(z) \leq u_m( \barr y),
\]
 the last inequality coming from the upper semicontinuity of $u_m$.
The same holds for $v_m$ and we deduce $u_m( \barr y) + v_m( \barr y) \geq 0$
and, letting $m \to +\infty$, $\hat u( \barr y) + v( \barr y) \geq 0$,
that is
\[
u( \barr y) + v( \barr y) \geq \eta > 0, \quad  \barr y \in \de \Omega,
\]
which is the contradiction we were looking for.
\end{proof}

\section{From the Euclidean space to manifolds: a brief note}

In our presentation we decided to remain in an Euclidean context. Nonetheless, many results extend to Riemannian manifolds; for instance, the AE Theorem~\ref{aet}, the Subharmonic Addition Theorem~\ref{add} and the Strict Comparison Theorem~\ref{scqc} have been given in \cite{hlae} in a more general form on manifolds. We note that, roughly speaking, \emph{all local results naturally extend from Euclidean spaces to manifolds}, via a suitable use of local coordinates. 

In this passage, one needs to extend the definition of quasi-convex function to real-valued functions defined on a manifold $M$. The usual definition of a convex function on a Riemannian manifold is that of a function which is convex \emph{along each geodesic}; that is, $f \colon M \to \R$ is convex on a geodesically convex subset $U \subset M$ if and only if $f \circ \gamma \colon (-\epsilon, \epsilon) \to \R$ is convex (in the real sense) for every geodesic segment $\gamma\colon (-\epsilon, \epsilon) \to U$.
Hence, by paraphrasing the idea that a quasi-convex function is convex modulo a quadratic perturbation, one can give, on connected manifolds, the following definition:
\begin{center}
\emph{$f$ is $2\lambda$-quasi-convex if $f + \lambda\, d(\cdot, p)^2$ is convex for all $p \in M$},
\end{center}
where $d$ is the standard metric on $M$.

This definition is equivalent to the one we have given in Euclidean spaces; alternatively, one could also ask $f+\lambda\;\! d(\cdot, p)^2$ to be convex  only for at least one $p \in M$. The two definitions are clearly equivalent in the Euclidean case, and in general on every manifold with zero sectional curvature; nevertheless, the first one is in general stronger and it is the right definition if one would like, even on manifolds, functions which are both quasi-convex and quasi-concave to be of class $C^{1,1}$, as noted in \cite{azgafeqc}.\footnote{Let us also note that on manifolds with positive curvature it could be tedious that $d(\cdot, p)^2$ is not convex; for instance, one cannot say that $f$ being $\lambda$-quasi-convex implies that $f$ is $\mu$-quasi-convex for all $\mu > \lambda$.}

On the other hand usually one is interested in \emph{local} quasi-convexity, and it is known (see, e.g., \cite{spi1}) that for every $p_0 \in M$ and $\epsilon > 0$ there exists a neighbourhood $W$ of $p_0$ such that each couple of points $q, q' \in W$ is connected by a unique minimal geodesic parametrised on $[0,1]$ and of length less than $\epsilon$, namely $\gamma_{q,q'}(t) = \exp_q(tv)$ for a unique $v = v(q,q') \in T_qM$ with $|v| < \epsilon$. Hence one may give this definition:
\[
\text{\em $f$ is $2\lambda$-quasi-convex on $W$ if} \ \ \begin{array}{l} \text{\em the map $t \mapsto f(\exp_q(t v))+ \lambda|tv|^2$ is convex on $[0,1]$} \\ \text{\em for each $q \in W$ and any $v \in B_\epsilon(0) \subset T_qM$}.
\end{array}
\]
This is a natural extension of the definition of quasi-convexity. Also, note that, in a normal chart $(U,\phi)$ about $q \in M$,\footnote{That is, a coordinate chart $(U, \phi)$ with $U = \exp_q(B_\epsilon(0))$ and $\phi = \psi \circ \exp^{-1}$, for some linear isometry $\psi \colon T_q M \to \R^{m}$, where $m =\mathrm{dim}\, M$.} one reads that $f \circ \phi^{-1}$ is $2\lambda$-quasi-convex along each small linear segment starting from $0 \in \R^m$.
This partially motivates why one can also rightfully say, without relying on the metric, that
\[
\text{\em $f$ is locally quasi-convex on $M$ if} \ \ \begin{array}{l} \text{\em $f \circ \phi^{-1}$ is quasi-convex on $\phi(U) \subset \R^m$} \\ \text{\em for at least one chart $(U, \phi)$ per $p\in M$}.
\end{array}
\]
Adopting this definition of local quasi-convexity, it is immediate to see that the aforementioned results extend to manifolds.

%\addtocontents{toc}{\vspace{15pt}}
%\section{Scheme of main results}
%\noticina{Vogliamo inserire (parte di) questo schema?}
%
%\begin{figure}
%\includegraphics[max width=\linewidth]{scheme_of_results_copy.pdf}
%\end{figure}


\begin{appendix}
\renewcommand\thesection{A}
\setcounter{thm}{0}

\part*[Appendix]{Appendix. Proof of Sard's theorem for Lipschitz functions}

\label{proofsard}

We reproduce the proof \Cref{sard} one finds in~\cite{how}; it turns out to be a consequence of a general result in measure theory. Here we denote by $m$ the Lebesgue outer measure on $\R^n$. The proof of Besicovitch's covering theorem here below is adapted from a more general one contained in~\cite{fed:geo}.

\begin{lem} \label{app:ball}
Let $\Omega\subset \R^n$ open, $G\colon \Omega \to {\R^n}$ and suppose it is differentiable at some point $a$; let $\epsilon >0$. Then there exists $\barr\rho = \barr\rho(a,\epsilon)>0$ such that 
\[
m\big(G\big(B_\rho(a)\big)\big) \leq \left(|\det DG(a)|+\epsilon\right)m(B_\rho(a)) \qquad \forall \rho\leq \barr\rho.
\]
\end{lem}

\begin{proof}
Let $\ell \defeq G(a) + \pair{DG(a)}{\cdot-a}$; as $G$ is differentiable at $a$, there exists a modulus of continuity $\omega\colon \R_+ \to \R_+$ such that $| G(x) - \ell(x) | \leq \omega(|x-a|)|x-a|$. This yields
\[
G\big(B_\rho(a)\big) \subset \ell\big(B_\rho(a)\big) + \omega(	\rho) \rho B,
\]
where $B\defeq B_1$ is the unit ball centered at the origin. On taking the measures, 
\[
m(G(B_\rho(a))) \leq m(\ell(B(a)) + \omega(	\rho)  B)\rho^n,
\]
where $B(a)\defeq B_1(a)$. Now note that
\[
\lim_{\rho \dto 0^+} m(\ell(B(a)) + \omega(	\rho)  B) = m(\ell(B(a))) = |\ell(B(a))| = |\det DG(a)|\ \alpha_n,
\]
where the latter equality comes from the change of variable formula. Therefore there exists some $\barr\rho>0$ such that for $\rho\leq \barr\rho$ we have
\[
m\big(G\big(B_\rho(a)\big)\big) \leq m(\ell(B(a)) + \omega(	\rho)  B)\rho^n \leq (|\det DG(a)| + \epsilon)\,\alpha_n\rho^n
\]
and hence the result.
\end{proof}

\begin{lem}[Besicovitch's covering theorem] \label{app:bes}
Let $A\subset \R^n$ and $\rho\,\colon A \to \R_+$ bounded. Then there exists $N\in\N$ depending only on $n$ and $N$ families 
\[
\call B_k = \big\{ B_j^k \defeq B_{\rho(a_j^k)}(a_j^k) :\ a_j^k\in A,\ j\in\N,\ B_j^k \cap B_i^k\ \forall\, i\neq j \big\}, \qquad k=1,\dots, N,
\]
such that 
\[
A \subset \bigcup_{k=1}^N \bigcup_{j\in\N} B_j^k.
\]
\end{lem}

\begin{proof}
We split the proof into four steps.

\underline{Step 1}. \emph{Let $\tau >1$. Then
\[
A \subset \bigcup_{j\in \N} B_j,
\]
where $B_j = B_{\rho_j}(a_j)$, $a_j\in A$, $\rho_j=\rho(a_j)$, satisfying for every $a_i,a_j$
\[
\text{either}\quad |a_i - a_j| > \rho_i > \rho_j/\tau \quad \text{or}\quad |a_i-a_j| > \rho_j > \rho_i/\tau.
\]
}
\indent Consider the class $\Lambda$ of subsets $\Q\subset A$ with the following properties:
%%%
\begin{equation} \tag{\sc i} \label{sc(i)}
\text{either $|a - c| > \rho(a) > \rho(C)/\tau$ or $|a-c| > \rho(C) > \rho(a)/\tau$ for all $a,c\in \Q$;}
\end{equation}
\begin{equation} \tag{\sc ii} \label{sc(ii)}
\text{whenever $b\in A$,}\quad \begin{array}{l} \text{either $|a-b|\leq \rho(a)$ for some $a\in \Q$} \\ \text{or $|a-b| > \rho(a) > \rho(b)/\tau$ for all $a\in \Q$.}\end{array}
\end{equation}
By Hausdorff's maximal principle, $\Lambda$ has a maximal member $\Q$. Then the family $\{B_{\rho(a)}(a) : a\in \Q\}$ covers $A$. Indeed, otherwise 
\[
K\defeq \{ a\in A:\ |a-c| > \rho(a)\ \text{for all}\ a\in  \Q\} \neq \emptyset
\]
and thus we may choose $c\in K$ with $\tau\rho(C) > \sup_{a\in K} \rho(a)$. This implies $\Q\subset \Q \cup \{c\} \in \Lambda$, contrary to the maximality of $ \Q$. Also, since $A$ is second countable, there exists a countable subset of $\Q$ which still covers $A$.

\underline{Step 2}. \emph{Let $P\subset A$. Then there exists $R\subset P$ such that}
\begin{equation} \tag{\sc iii} \label{sc(iii)}
\text{\emph{$|a-b| > \rho(a)+\rho(b)$ for all $a,b\in \RR$;}}
\end{equation}
\begin{equation} \tag{\sc iv} \label{sc(iv)}
\text{\emph{whenever $a\in P$, there exists $b\in R$ with $|a-b| \leq \rho(a) + \rho(b)$ and $\rho(b) > \rho(a)/\tau$.}}
\end{equation}

Consider the class $\Lambda$ of those subsets $R\subset P$ satisfying \eqref{sc(iii)} and
\begin{equation} \tag{\sc v} \label{sc(v)}
\text{whenever $a\in P$,} \quad \begin{array}{l} \text{either $|a-b| > \rho(a) + \rho(b)$ for all $b\in R$} \\ \text{or \eqref{sc(iv)} holds.} \end{array}
\end{equation}
Again by Hausdorff's maximal principle, there exists a maximal member $R$ of $\Lambda$, which satisfies \eqref{sc(iv)}; indeed, otherwise, arguing as in the previous step, property \eqref{sc(v)} would imply that there exists $a\in P\setminus R$ such that $R \cup \{ a \} \in \Lambda$, contrary to the maximality of $R$.

\underline{Step 3}. \emph{Let $\epsilon > 0$ small and $\tau \in (1,2-\epsilon)$ such that
\begin{equation} \label{app:dis}
\epsilon + \frac\tau{2-\epsilon} + \tau(\tau -1) < 1.
\end{equation}
Let $\Q\subset A$ satisfying \eqref{sc(i)} and 
\begin{equation} \tag{\sc vi} \label{sc(vi)}
\text{$|a-b| \leq \rho(a) + \rho(b)$ and $\rho(b) > \rho(a)/\tau$ for all $a,b\in \Q$.}
\end{equation}
Then there exists $N \in \N$ depending on $n$ and $\epsilon$ only such that $\# \Q \leq N$.}
%%%
\setlist[1]{topsep=3.5pt, itemsep=0pt, partopsep=1.1pt, parsep=1.1pt}

First of all, note that such a $\tau$ exists (by continuity) since the left-hand side of \eqref{app:dis} for $\tau = 1$ and $\epsilon = 0$ is $\tfrac12$. Let now $\kappa = (2-\epsilon)/\tau$ and $a\in \Q$. Set
\[
\Q_1 \defeq\Q\cap \left(B_{k\rho(a)}(a) \setminus \{a\}\right), \qquad \Q_2 \defeq\Q\cap B_{k\rho(a)}(a)\compl.
\]
Let $b,c\in \Q_j$ for some $j\in\{1,2\}$ be distinct points with $|a-b| \geq |a-c|$ and $x\in\R^n$ such that
\[
|a-x| = |a-c|, \qquad |b-x| = |a-b| - |a-c|.
\]
Note that actually such an $x$ is uniquely determined by $a,b,c$ because of the strict convexity of the Euclidean norm, namely as
\begin{equation} \label{app:sc}
x = a + \frac{|a-c|}{|a-b|}(b-a).
\end{equation}
We see that 
\[
|x-c| \geq |a-c| + |b-c| - |a-b|
\]
 and distinguish two cases. If $j=1$, since by \eqref{sc(i)}  and \eqref{sc(vi)}, $|b-c| > \min\{\rho(b), \rho(C)\} > \rho(a)/\tau$ and similarly $|a-c| > \rho(a)/\tau$, where $|a-b| \leq \kappa\rho(a) $ and $\kappa\tau - 1 = 1-\epsilon >0$, we obtain
\[
|x-c| > |a-c| - (\rho(a)/\tau)(\kappa\tau-1) > \epsilon |a-c|;
\]
if $j=2$, since $|a-b| \leq \rho(a) + \rho(b)$, $|a-c| > k\rho(a)$,
\[\begin{split}
|b-c| > \rho(b)\ \text{or}\ |b-c| > \rho{(C)} > \rho(b)/\tau \quad &\implies \quad |b-c|-\rho(b) > (1-\tau)\rho{(C)},\\
|a-c| > \rho{(C)}\ \text{or}\ |a-c| > \rho(a) > \rho{(C)}/\tau \quad &\implies \quad |a-c| > \rho{(C)}/\tau,
\end{split}\]
we obtain
\[\begin{split}
|x-c| &\geq |a-c| + |b-c| - \rho(a) -\rho(b) > |a-c| -\rho(a) - \rho{(C)}(\tau-1)\\
&= |a-c|\left( 1- \frac{\rho(a)}{|a-c|} - \frac{\rho{(C)}}{|a-c|}(\tau-1) \right) > |a-c|\left( 1-\frac1k - \tau(\tau-1) \right) \\
&> \epsilon |a-c|,
\end{split}\]
where the last inequality comes from the condition~(\ref{app:dis}).

Therefore we have 
\[
\frac{|x-c|}{|a-c|} \geq \epsilon
\]
for all $a,x,c$ as above. At this point by~(\ref{app:sc}) one sees that 
\[
\frac{|x-c|}{|a-c|} = \left| \frac{b-a}{|b-a|} - \frac{c-a}{|c-a|} \right| = |\pi(b)-\pi{(C)}|,
\]
where $\pi(b),\pi{(C)}$ denote the projections of $b,c$, respectively, onto the sphere $\de B_1(a)$. This implies that there is a one-to-one correspondence between points in $\Q_j$ and their projections and $\#\Q_j \leq \kappa$ for some $\kappa \in \N$. Indeed, we know that $\pi(\Q_j)$ is finite, because otherwise by the compactness of $\de B_1(a)$ it would have a limit point and thus there would exist points $\pi(b),\pi{(C)}$, with $|\pi(b) - \pi{(C)}| < \epsilon$. It is clear that this argument is independent of $a$ and holds for every $\epsilon$-distanced subset of the unit sphere $\mathbb{S}^{n-1}$, hence $\kappa$ depends only on $n$ and $\epsilon$.

Putting all of this together, we have $\# \Q = \#(\Q_1\cup \Q_2 \cup \{a\}) = \#\Q_1 + \#\Q_2 + 1 \leq 2\kappa + 1 =: N$.

\underline{Step 4}. \emph{Let $\tau$ be as in step 3 with $\epsilon = \frac13$ and $P \defeq \{ B_j \}_{j\in\N}$ be as in step 1. Then $P$ is the union of $N$ families of pairwise disjoint balls.}

Applying step 2, we define by induction subsets $P_k$, $\RR_k$ of $P$, starting with $P_0 \defeq P$ and $\RR_0 \defeq \emptyset$ and letting, for $k\geq 1$, $P_k \defeq P_{k-1} \setminus \RR_{k-1}$ and $\RR_k \subset P_k$ satisfying \eqref{sc(iii)} and \eqref{sc(iv)}. Note that condition \eqref{sc(iii)} implies that the family $\call B_k \defeq \{B_j^k \defeq B_{\rho_j^k}(a_j^k):\ a_j^k\in \RR_k\}$ is disjointed for each $k$ and by definition
\[
P_{N+1} = \bigcap_{k=1}^N\, (P \setminus \RR_k) = P \setminus \bigcup_{k=1}^N \RR_k.
\]
Hence we conclude the proof if we show that $P_{N+1} = \emptyset$. In fact, if $a\in P_{N+1}$, one uses property \eqref{sc(iv)} to select, for all $k=1,\dots, N$, some $b_j^k\in \RR_k$ with $|a-b_j^k| \leq \rho(a) + \rho_j^k$ and $\rho_j^k > \rho(a)/\tau$. Therefore $\Q\defeq \{a,b_1,\dots, b_N\}$ satisfies \eqref{sc(i)} and \eqref{sc(iv)}, and also $\#\Q = N+1$ since the $\RR_j$'s are disjointed by definition; but this contradicts what we proved in step 3.
\end{proof}

\begin{prop}
Let $\Omega \subset \R^n$ open, $G\colon \Omega\to {\R^n}$ and $A\subset\Omega$ with $m(A)$ finite. Suppose $G$ is differentiable at every $a\in A$ and there exists a constant $M$ such that $|\det DG|\leq M$ in $A$. Then 
\[
m(G(A)) \leq cMm(A),
\]
where the constant $c$ depends only on $n$.
\end{prop}

\begin{proof}
Fix $U\supset A$ open such that $m(U) \leq 2m(A)$; given $\epsilon >0$, by Lemma \ref{app:ball}, for each $a\in A$ there exists some ball $B_{\rho(a)}(a)$ such that $m(G(B_{\rho(a)}(a))) \leq \left(M+\epsilon\right)\cdot m(B_{\rho(a)}(a))$. Also, we may choose $\rho(a)$ in such a way that $B_{\rho(a)}(a) \subset U\cap \Omega$ for all $a\in A$, hence the map $\rho\,\colon A \to \R_+$ is bounded. Besicovitch's covering theorem (Lemma \ref{app:bes}) tells us that there exists a countable subset $\{a_i\}_{i\in\N} \subset A$ such that, letting $\rho_i \defeq \rho(a_i)$, $\{B_i\defeq B_{\rho_i}(a_i)\}_{i\in\N}$ covers $A$ and every point $x\in\R^n$ is in at most $N$ of the balls $B_{\rho_i}(a_i)$. Therefore $\sum_{i\in\N} \chi_{B_i} \leq N\chi_U$ and we have
\[
\sum_{i\in\N} m(B_i) \leq Nm(U) \leq 2N m(A)
\]
and thus
\[\begin{split}
m(G(A)) &\leq m\!\left(G\Big(\bigcup_{i\in\N} B_i \Big)\right) \leq \sum_{i\in N} m(G(B_i)) \leq (M+\epsilon)\! \sum_{i\in\N} m(B_i) \\
 &\leq 2N(M+\epsilon)m(A).
\end{split}\]
As $\epsilon$ is arbitrary, we have the thesis, with $c=2N$.
\end{proof}

\begin{cor}
Let $\Omega\subset\R^n$ open and $G \colon \Omega \to \R^n$; set 
\[
\call T\defeq \{x\in\Omega:\ \exists DG(x),\ |\det DG(x)| = 0\}.
\]
Then $G(\call T)$ has measure zero.
\end{cor}

\begin{proof}
If $m(\call T)<\infty$ it is a straightforward application of the previous proposition; otherwise let $A_k \defeq B_k \cap \call T$ and the thesis follows by subadditivity. 
\end{proof}


The proof of Sard's theorem is now an easy application of Rademacher's theorem.

\begin{proof}[Proof of~\Cref{sard}]
If $G$ is Lipschitz, then $k_G = \call N \cup \call T$, where we denote by $\call N$ the set of points at which $G$ is not differentiable. Since $G$ is Lipschitz and $\call N$ has (Lebesgue) measure zero, by Rademacher's Theorem \ref{rade}, we have that $G(\call N)$ has measure zero as well; by the above corollary, $G(\call T)$ has measure zero and we are done. \qedhere
\end{proof}

\end{appendix}


\addtocontents{toc}{\vspace{15pt}}

\begin{thebibliography}{100}
\bibitem{aa} G.\ Alberti and L.\ Ambrosio, \emph{A geometrical approach to monotone functions in $\R^n$}, Math.\ Z.\ {\bf230} (1999), 259--316. 
\bibitem{alex} A.\ D.\ Alexandrov, \emph{Almost everywhere existence of the second differential of a convex function and properties of convex surfaces connected with it} (in Russian), Lenningrad State Univ.\ Ann.\ Math.\ {\bf37} (1939), 3--35.
\bibitem{azgafeqc} D.\ Azagra and J.\ Ferrera, \emph{Regularization by sup-inf convolutions on Riemannian manifolds: An extension of Lasry--Lions theorem to manifolds of bounded curvature}, J.\ Math.\ Anal.\ Appl.\ {\bf423(2)} (2015), 994--1024.
\bibitem{bau} H.\ Bauer, \emph{Minimalstellen von Funktionen und Extremalpunkte}, Archiv der Mathematik {\bf9} (1958), 389--393.
%\bibitem[B]{bur} D.\ Burago, Y.\ Burago, S.\ Ivanov, \emph{A Course in Metric Geometry}, \AmS\ (2001).
\bibitem{cafcab} L.\ Caffarelli and X.\ Cabré, \emph{Fully nonlinear elliptic equations}, American Mathematical Society Colloquium Publications {\bf43}, American Mathematical Society, Providence, RI (1995).
\bibitem{chlp} M.\ Cirant, F.\ R.\ Harvey, H.\ B.\ Lawson, Jr., and K.\ R.\ Payne, \emph{Comparison principles by monotonicity and duality for constant coefficient nonlinear potential theory and PDEs}, Annals of Mathematics Studies {\bf 218} Princeton University Press, Princeton, NJ, to appear; xii+204 p., preprint version {\tt arXiV: 2009.01611v1} 170 pages, published online 3 Sep 2020.
\bibitem{cpaux} M.\ Cirant and K.\ R.\ Payne, \emph{O\MakeLowercase{N VISCOSITY SOLUTIONS TO THE \MakeUppercase{D}IRICHLET PROBLEM FOR ELLIPTIC BRANCHES OF NONHOMOGENEOUS FULLY NONLINEAR EQUATIONS}}, Publ.\ Mat.\ {\bf61} (2017), 529--575.
\bibitem{cpmain} \rule{30pt}{.5pt}, \emph{Comparison principles for viscosity solutions of elliptic branches of fully nonlinear equations independent of the gradient.}, Math.\ in Eng.\ {\bf 3(4)} (2021), 1--45.
\bibitem{cprdir}  M.\ Cirant, K.\ R.\ Payne, and D.\ F.\ Redaelli, {\em Comparison principles for nonlinear potential theories and PDEs with fiberegularity and sufficient monotonicity}, preprint, 2023.
\bibitem{csw} F.\ H.\ Clarke, R.\ J.\ Stern, and P.\ R.\ Wolenski, \emph{Subgradient criteria for monotonicity, the Lipschitz condition, and convexity}, Can.\ J.\ Math.\ {\bf45(6)} (1993), 1167--1183.
\bibitem{user} M.\ G.\ Crandall, H.\ Ishii, and P.-L.\ Lions, \emph{User’s Guide to Viscosity Solutions of Second Order Partial Differential Equations}, Bull.\ Amer.\ Math.\ Soc.\ {\bf27} (1992), 1--67.
\bibitem{crankoc} M.\ G.\ Crandall, M.\ Kocan, P.\ Soravia, and A.\ {\'S}wi{\k{e}}ch, \emph{On the equivalence of various weak notions of solutions of elliptic PDEs with measurable ingredients}, Pitman Res.\ Notes Math.\ Ser.\ {\bf350} (1996), 136--162.
\bibitem{dephfi} G.\ De Philippis and A.\ Figalli, \emph{The Monge-Ampère equation and its link to optimal transportation}, Bull.\ Amer.\ Math.\ Soc.\ (N.S.) {\bf51} (2014), 527--580.
\bibitem{dudley} R.\ M.\ Dudley, \emph{On second derivatives of convex functions}, Math.\ Scandinavica {\bf41(1)} (1977), 159--174.
\bibitem{eber} A.\ Eberhard, \emph{Prox-Regularity and Subjets}, in A.\ Rubinov, B.\ Glover (Ed.), Optimization and Related Topics, Applied Optimization {\bf47}, Springer (2001), 237--313.
\bibitem{evansgar} L.\ C.\ Evans and R.\ F.\ Gariepy, \emph{Measure theory and fine properties of functions}, Studies in Advanced Mathematics, CRC Press (1992).
\bibitem{fed:geo} H.\ Federer, \emph{Geometric measure theory}, Grund.\ Math.\ Wiss.\ {\bf153}, Springer-Verlag, Berlin--Heidelberg--New York (1969).
\bibitem{har:pc} F.\ R.\ Harvey, \emph{Federer knows all}, private communication (Feb.\ 2018).
\bibitem{hlpotcg} F.\ R.\ Harvey and H.\ B.\ Lawson, Jr., \emph{An introduction to potential theory in calibrated geometry}, Amer.\ J.\ Math.\ {\bf131(4)} (2009), 893--944; {\tt arXiv:math.0710.3920}.
\bibitem{hlpluri09} \rule{30pt}{.5pt}, {\em Duality of positive currents and plurisubharmonic functions in calibrated geometry}, Amer.\ J.\ Math.\ {\bf 131}\ (2009), 1211--1239.
\bibitem{hldir09} \rule{30pt}{.5pt}, \emph{Dirichlet duality and the nonlinear Dirichlet problem}, Comm.\ on Pure and Applied Math.\ {\bf62} (2009), 396--443; {\texttt{arXiv:0710.3991}}.
\bibitem{hldir}  \rule{30pt}{.5pt}, \emph{Dirichlet duality and the nonlinear Dirichlet problem on Riemannian manifolds}, J.\ Diff.\ Geom.\ {\bf88} (2011), 395--482; {\texttt{arXiv:0912.5220}}.
\bibitem{hlae} \rule{30pt}{.5pt}, \emph{The AE Theorem and addition theorems for quasi-convex functions}, {\texttt{arXiv:1309.1770v3}} (30 July 2016), 1--12.
\bibitem{hlqc} \rule{30pt}{.5pt}, \emph{Notes on the differentiation of quasi-convex functions}, {\texttt{arXiv:1309.1772v3}} (30 July 2016), 1--18.
\bibitem{hlcharsmp} \rule{30pt}{.5pt}, \emph{Characterizing the strong maximum principle for constant coefficient subequations,} Rend.\ Mat.\ Appl.\ {\bf37(7)} (2016), 63--104.
\bibitem{hllagma} \rule{30pt}{.5pt}, \emph{Lagrangian Potential theory and a Lagrangian equation of Monge--Ampère type}, Surveys in Differential Geometry {\bf22}, International Press of Boston, Inc., Sommerville (2017), 217--258.
\bibitem{hlidir} \rule{30pt}{.5pt}, \emph{The inhomogeneous Dirichlet Problem for natural operators on manifolds}, {\tt arXiv:1805.111213v1} (2018), to appear in Ann.\ Inst.\ Fourier, Grenoble.
\bibitem{hlkry} \rule{30pt}{.5pt}, \emph{A generalization of PDEs from a Krylov point of view}, Adv.\ Math.\ {\bf372} (2020), 1--39.
\bibitem{hlpssl} \rule{30pt}{.5pt}, \emph{Pseudoconvexity for the special Lagrangian potential equation}, to appear in Calc.\ Var.\ Partial Differential Equations, preprint available at {\url{arxiv.org/pdf/2001.09818.pdf}}.
\bibitem{hpsurvey} F.\,R.\ Harvey and K.\,R.\ Payne, {\em Interplay between nonlinear potential theory and fully nonlinear PDEs},  Pure Appl.\ Math.\ Q.\, to appear; preprint version online {\tt arXiv:2203.14015v1} (29 Mar 2022), 1--44.
 \bibitem{how} R.\ Howard, \emph{Alexandrov's Theorem on the second derivatives of convex functions via Rademacher's Theorem on the first derivatives of Lipschitz functions}, notes (1998) available online at {\url{http://ralphhoward.github.io/SemNotes/index.html}}.
 \bibitem{hiriart} J.-B.\ Hiriart-Urruty and Ph.\ Plazanet, \emph{Moreau's decomposition theorem revisited}, Annales de l'I.H.P. Analyse non linéaire, {\bf6} (1989), 325--338.
 \bibitem{jensen} R.\ Jensen, \emph{The maximum principle for viscosity solutions of fully nonlinear second order partial differential equations}, Arch.\ Rat.\ Mech.\ Anal.\ {\bf101} (1988), 1--27.
% \bibitem{kir}  M.\ D.\ Kirszbraun, \emph{\"Uber die zusammenziehende und Lipschitzsche Transformationen}, Fund.\ Math.\ {\bf22} (1934), 77--108.
 \bibitem{kr} N.\ V.\ Krylov, \emph{On the general notion of fully nonlinear second-order elliptic equations}, Trans.\ Amer.\ Math.\ Soc.\ {\bf347(3)} (1995), 30--34.
 \bibitem{laslio} J.-M.\ Lasry and P.-L.\ Lions, \emph{A remark on regularization in Hilbert spaces}, Israel Math.\ J.\ {\bf55}, n.\ 3 (1996), 257--266.
 \bibitem{lioham} P.-L.\ Lions, \emph{Generalized Solutions of Hamilton-Jacobi Equations}, Pitman, London (1982).
 \bibitem{pucci} C.\ Pucci, \emph{Limitazioni per soluzioni di equazioni ellittiche}, Ann.\ Mat.\ Pura Appl.\ {\bf74} (1966), 15--30.
 \bibitem{rademacher} H.\ Rademacher, \emph{{\"U}ber partielle und totale Differenzierbarkeit I.}, Math.\ Ann.\ {\bf79} (1919), 340--359.
\bibitem{anr} F.\ Riesz and B.\ Sz.-Nagy, \emph{Functional Analysis}, Ungar, New York (1955).
\bibitem{rock} R.\ T.\ Rockafellar, \emph{Convex analysis}, Princeton Mathematical Series {\bf28}, Princeton University Press, Princeton, N.J. (1970).
\bibitem{slod} Z.\ S{\l}odkowski, \emph{The Bremermann--Dirichlet problem for  $q$-plurisubharmonic functions}, Ann.\ SNS di Pisa, Cl.\ di Scienze, 4\ap{a} serie, {\bf11}, n.\ {\bf2} (1984), 303--326.
\bibitem{spi1} M.\ D.\ Spivak, \emph{A Comprehensive Introduction to Differential Geometry}, vol.\ 1, Publish or Perish, 3rd edition (1999).
\bibitem{stein} E.\ M.\ Stein and R.\ Shakarchi, \emph{Real Analysis: Measure Theory, Integration, and Hilbert Spaces}, Princeton University Press (2005).
\bibitem{meast}  T.\ Tao, \emph{An Introduction to Measure Theory}, Graduate Studies in Mathematics {\bf126}, Amer.\ Math.\ Soc.\ (2011).
\bibitem{villani} C.\ Villani, \emph{Optimal transport. Old and new}, Grundlehren der Mathematischen Wissenschaften {\bf338}, Springer-Verlag, Berlin (2009).
\bibitem{vesely} L.\ Vesely, Notes for the course in Convex Analysis, Università degli Studi di Milano (2019); available online at \url{http://users.mat.unimi.it/users/libor/2018-2019.html}.
\bibitem{wheedzyg} R.\ L.\ Wheeden and A.\ Zygmund, \emph{Measure and Integral: An Introduction to Real Analysis}, Second Edition, Chapman and Hall/CRC (2015). 
\end{thebibliography}

\end{document}