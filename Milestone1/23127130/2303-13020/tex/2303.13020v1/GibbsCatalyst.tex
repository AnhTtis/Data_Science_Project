\documentclass[
notitlepage,
floatfix,
aps,
prl,
reprint,
twocolumn,
%superscriptaddress,
%longbibliography
]{revtex4-2}
\usepackage[utf8]{inputenc}

%\usepackage{fullpage}
\usepackage[caption=false]{subfig}
\usepackage{graphicx}
\usepackage{epstopdf}
\usepackage{array}
\usepackage{verbatim}
\usepackage{amsmath,amsfonts,amssymb,amscd,mathtools}
\usepackage{amsthm}
\usepackage{tabularx}
\usepackage{stmaryrd}
\usepackage{enumerate}
\usepackage{enumitem}
\usepackage{wasysym}
\usepackage{mathrsfs}
\usepackage{microtype}
\usepackage{hyperref}
\usepackage[dvipsnames]{xcolor}
\DeclareMathAlphabet{\mathmybb}{U}{bbold}{m}{n}
\newcommand{\1}{\mathmybb{1}}
\hypersetup{
	bookmarksnumbered=true, % If Acrobat bookmarks are requested, include section numbers
	unicode=false, % non-Latin characters in Acrobat bookmarks
	pdfstartview={FitH}, % fits the width of the page to the window
	pdftitle={}, % title
	pdfauthor={}, % author
	pdfsubject={}, % subject of the document
	pdfcreator={}, % creator of the document
	pdfproducer={}, % producer of the document
	pdfkeywords={}, % list of keywords
	pdfnewwindow=true, % links in new window
	colorlinks=true, % false: boxed links; true: colored links
	linkcolor=NavyBlue, % color of internal links
	citecolor=NavyBlue, % color of links to bibliography
	filecolor=NavyBlue, % color of file links
	urlcolor=NavyBlue % color of external links
}
\usepackage{tikz}
\usepackage[lmargin=.7in,rmargin=.7in,tmargin=.7in,bmargin=1in]{geometry}
\renewcommand{\baselinestretch}{1.03}
\usepackage{newtxtext,newtxmath}
\usepackage{varwidth}

\makeatletter
\def\maketitle{
	\@author@finish
	\title@column\titleblock@produce
	\suppressfloats[t]}
\makeatother
\usepackage[normalem]{ulem}


\newcommand{\nn}[1]{{\color{purple}#1}}
\newcommand{\nng}[1]{{\color{orange}#1}}
\newcommand{\js}[1]{{\color{blue}#1}}
\newcommand{\jswip}[1]{{\color{gray}#1}}
\newcommand{\Tr}{\mathrm{Tr}}
\newcommand{\border}{\boldsymbol{\pi}}
\newcommand{\pstate}{\mathbf{p}}
\newcommand{\qstate}{\mathbf{q}}
\newcommand{\rstate}{\mathbf{r}}
\newcommand{\sstate}{\mathbf{s}}
\newcommand{\taustate}{\mathbf{\tau}}
\newcommand{\slope}{\mathbf{\vec{g}}}
\newcommand{\Mswap}{\mathcal{M}}
\newcommand{\ket}[1]{\left \lvert #1 \right \rangle}
\newcommand{\bra}[1]{\langle #1\rvert}
\newcommand{\braket}[2]{\langle #1|#2\rangle}
\newcommand{\ketbra}[2]{| #1 \rangle \langle #2 |}
\newcommand{\dm}[1]{\ketbra{#1}{#1}}

\newcommand{\mA}{\mathcal{A}}
\newcommand{\mC}{\mathcal{C}}
\newcommand{\mE}{\mathcal{E}}
\newcommand{\mN}{\mathcal{N}}
\newcommand{\mU}{\mathcal{U}}
\newcommand{\mT}{\mathcal{T}}
\newcommand{\mD}{\mathcal{D}}
\newcommand{\mI}{\mathcal{I}}
\newcommand{\mF}{\mathcal{F}}
\newcommand{\mL}{\mathcal{L}}
\newcommand{\rl}{\rangle\langle}
\newcommand{\mG}{\mathcal{G}}
\newcommand{\mH}{\mathcal{H}}
\newcommand{\mM}{\mathcal{M}}
\newcommand{\mO}{\mathcal{O}}
\newcommand{\mB}{\mathcal{B}}
\newcommand{\mK}{\mathcal{K}}
\newcommand{\mP}{\mathcal{P}}
\newcommand{\mR}{\mathcal{R}}
\newcommand{\mS}{\mathcal{S}}
\newcommand{\mX}{\mathcal{X}}
\newcommand{\mY}{\mathcal{Y}}
\newcommand{\mZ}{\mathcal{Z}}
\newcommand{\mbC}{\mathbb{C}}
\newcommand{\mbD}{\mathbb{D}}
\newcommand{\mbE}{\mathbb{E}}
\newcommand{\mbF}{\mathbb{F}}
\newcommand{\mbI}{\mathbb{I}}
\newcommand{\mbH}{\mathbb{H}}
\newcommand{\mbM}{\mathbb{M}}
\newcommand{\mbN}{\mathbb{N}}
\newcommand{\mbO}{\mathbb{O}}
\newcommand{\mbR}{\mathbb{R}}
\newcommand{\mbS}{\mathbb{S}}
\newcommand{\mbV}{\mathbb{V}}
\newcommand{\mbZ}{\mathbb{Z}}

\newcommand{\setTO}{\mR_{\rm TO}}
\newcommand{\setETO}{\mR_{\rm ETO}}
\newcommand{\setMTP}{\mR_{\rm MTP}}
\newcommand{\setCTO}{\mR_{\rm CTO}}
\newcommand{\setCETO}{\mR_{\rm CETO}}
\newcommand{\setGCETO}{\mR_{\rm GC-ETO}}
\newcommand{\setNMETO}{\mR_{\rm nM-ETO}}
\newcommand{\setGCMTP}{\mR_{\rm GC-MTP}}

\newcommand{\sigstate}{\boldsymbol{\sigma}}
\DeclareMathOperator{\sgn}{sgn}

\theoremstyle{plain}
\newtheorem{thm}{Theorem}
\newtheorem{cor}[thm]{Theorem}
\newtheorem{lem}[thm]{Lemma}
\newtheorem{pro}[thm]{Proposition}
\newtheorem{claim}[thm]{Claim}

\theoremstyle{definition}
\newtheorem{defn}[thm]{Definition}
\newtheorem{remark}[thm]{Remark}
\newtheorem{ex}[thm]{Example}
\newtheorem{question}[thm]{Question}
\newcommand{\suppl}{Supplemental Material}

\begin{document}
\title{%\nn{Catalytic collapse for a hierarchy of thermal processes}
	A hierarchy of thermal processes collapses under catalysis%When are Gibbs states useful catalysts in thermodynamics?
	%Decomposing thermal processes through non-Markovianity\\ \nn{Alternative titles: Catalytic collapse of a hierarchy of thermal processes\\When are Gibbs states useful catalysts in thermodynamics?\\Unlocking the catalytic power of Gibbs states in thermodynamics}
}
\author{Jeongrak Son}
\email{jeongrak.son@e.ntu.edu.sg}
\affiliation{School of Physical and Mathematical Sciences, Nanyang Technological University, 637371, Singapore}
%\orcid{0000-0002-0236-5017}
\author{Nelly H. Y. Ng}
\email{nelly.ng@ntu.edu.sg}
\affiliation{School of Physical and Mathematical Sciences, Nanyang Technological University, 637371, Singapore}
%\orcid{0000-0003-0007-4707}



\begin{abstract}
It is not possible to decompose generic thermal operations into combinations and concatenations of simpler thermal processes that only manipulate selected system energy levels. This creates a hindrance in providing realistically-implementable protocols to reach all thermodynamic state transitions. However, in this work we show that the recycling of thermal baths allows the decomposition of thermal operations into a series of elementary thermal operations, each involving only two system levels at a time. Such a scheme is equivalent to a catalytic version of elementary thermal operations, where the catalysts are prepared in Gibbs states and re-thermalized at a later time. Thus, the Gibbs state catalyst closes the gap between elementary thermal operations and thermal operations. Furthermore, when any catalyst can be employed, we prove that a hierarchy of different thermal processes converge to that of thermal operations.
\end{abstract}

\maketitle

\emph{Introduction.}---
Thermal operations (TOs)~\cite{Janzing00_TO} are a collection of quantum channels that capture the effects of heat exchange between a system and a thermal reservoir. Over the past decade, they have been studied extensively in quantum thermodynamics~\cite{Brandao13_TRT, Gour15_TRTreview, Ng18_Qthermo_book}. 
By leveraging their genericity, TOs have been a powerful tool in discovering how thermodynamics fundamentally deviates from the classical macroscopic picture, as one considers quantum systems~\cite{Horodecki13_fundamental, Reeb_2014, YungerHalpern16_noncommuting, Scharlau18_hornslemma, Woods19_engines}.
However, due to the same reason of genericity, it remains difficult to bridge insights from this framework to real-world implementations~\cite{YungerHalpern17_Realization}, as existing protocols~\cite{Janzing00_TO, Horodecki13_fundamental, Perry18_PRX_Crude, Shiraishi2020_construction} for interesting TOs often require demanding subroutines. Another hindrance arises from the structure of the theoretical development~\cite{Lostaglio19_review}, which tends to focus on the possibility of state transformations, without providing concrete and simple heat baths and interaction Hamiltonians underlying the process. As a result, even when a transformation is known to be possible, finding the correct bath and the energy-preserving unitary is non-trivial~\cite{Hu2019_singlemode, Ende2022_bath}, obscuring the dynamical description of the process. 

To address these challenges, a subset of TOs, dubbed elementary thermal operations (ETOs)~\cite{Lostaglio_18_ETO}, was proposed. It was originally envisioned that TOs would be decomposed into sequences of much simpler ETO channels, involving at most two system levels at any given time. Such a goal would be analogous to having a small set of two-qubit universal gates in quantum computation. Since ETOs can be generally emulated through well understood interactions~\cite{JCmodel, Buzek89_intensityJC}, such a decomposition would provide a clearer implementation pathway to achieve the full extent of TOs in practical, experimental settings. Moreover, by tracking states after each ETO, a time-resolved picture of the transformation emerges naturally, creating a room for better analysis~\cite{Son2022_CETO}.
Unfortunately, it was shown that not all TOs can be recast into a convex combination of ETOs, even when restricted to energy-incoherent transformations~\cite{Lostaglio_18_ETO}. A seemingly much easier task of decomposing TOs into series of TOs involving at most all but one system levels at a time, is also shown to be impossible~\cite{Mazurek2018_Decomp}.

Two recent results hinted at the potential of catalysis to narrow this gap between TOs and ETOs. Firstly, low-dimensional examples of recovering the full TO reachable set by adding a simple catalyst to ETOs were discovered~\cite{Son2022_CETO}. Secondly, in the more restricted subset of Markovian thermal process (MTP)~\cite{Lostaglio22_MTP1}, Gibbs states which are assumed free in thermodynamic resource theories, were found useful as catalysts for transitions beyond MTPs~\cite{Korzekwa22_MTP2}. This counter-intuitive finding reveals a key insight that Gibbs states are useful as catalysts whenever the set of thermal processes have, to some extent, an in-built Markovian behaviour, i.e. some information is inevitably lost to the environment. While MTPs are fully Markovian by design, ETOs capture some non-Markovian behaviour because they employ finite-size baths and strong couplings. Yet, the extent of non-Markovianity is limited --- after each ETO, bath states are discarded and a new Gibbs state retaining no memory is introduced for the subsequent step. 

Does the inability of ETOs to emulate TOs arise solely from their Markovianity? If this is true, by injecting non-Markovianity to ETOs, we should be able to implement any complex, multi-level unitaries of TOs with sequential two-level unitaries only, which is a significant simplification.
In this letter we show that the answer is affirmative --- any TO state transition can indeed be achieved, with an error that can be made arbitrarily small with the number of ETO sequences. More importantly, we display that the use of Gibbs catalysts remains the decisive factor in closing the gap between \emph{catalytic} versions of TOs and ETOs for energy-incoherent initial states. In other words, on the level of state transitions, ETOs using arbitrary (non-Gibbsian) catalysts are arbitrarily close to catalytic thermal operations (CTOs). This is a significant progress in the characterization of catalytic ETOs, which is a problem that could not be studied systematically before, due to the lack of efficiently computable state transition conditions in the ETO framework. 


\emph{Two equivalent descriptions of non-Markovian extensions of ETOs.} ---
\begin{table*}
	\begin{tabular}{||c | c | c||} 
		\hline
		\qquad Free operations ~~~~~~~ & With Gibbs catalysts & \qquad With any catalysts (Def.~\ref{def:exact_cat})~~~~~~~ \\ [0.5ex] 
		\hline\hline
		TO (Def.~\ref{def:TO}) & TO (Gibbs states are free) & CTO  \\ 
		\hline
		ETO (Def.~\ref{def:ETO_def}) & ~~~~~~~GC-ETO (Def.~\ref{def:NMETO_def_2}) ~=~ nM-ETO (Def.~\ref{def:NMETO_def_1}), Lemma \ref{lem:nm_vs_gc}~~~~~~~ & CETO  \\
		\hline
		MTP (Def.~\ref{def:MTP}) & GC-MTP ~=~ nM-ETO (Def.~\ref{def:NMETO_def_1})  & CMTP  \\
		\hline
	\end{tabular}
	\caption{Various choices of free operations for thermodynamic resource theories, and their relationship with one another.}
	\label{table:different_opeartions}
\end{table*}
We consider various classes of thermal processes, with different levels of non-Markovianity, and their relationship to one another. To proceed, we use the following notation: system Hilbert space $ \mH_{S} $, bath space $ \mH_{R} $, and catalyst space $ \mH_{C} $. The set of quantum states on Hilbert space $ \mH $ is denoted as $ \mS(\mH) $, and the set of linear operators as $ \mL(\mH) $.
For any Hilbert space $ \mH_{X} $, define an orthonormal set of energy eigenbasis $ \{\ket{i,g}_{X}\}_{i,g} $ given by the Hamiltonian $ H_{X} $, where the index $i $ denotes the energy eigenvalue $E_i$, while the index $g$ captures the degeneracy of that energy level. Whenever the spectrum of $X$ is non-degenerate, we drop the index $g$.
We further denote the set of all reachable states $ \{\phi\vert \rho\xrightarrow{\rm X}\phi\} $ by a class of operations X as $ \mR_{\rm X}(\rho) $, and its corresponding convex hull ${\rm Conv }(\mR_{\rm X}(\rho))$. 
Acronyms for each operation can be found in Table~\ref{table:different_opeartions}.
For the main classes of thermal processs (TOs, ETOs, and MTPs) relevant for this work, we have that, in general, ~\cite{Lostaglio_18_ETO,Lostaglio22_MTP1} 
\begin{equation}\label{eq:non-catalytic-hierarchy}
		{\rm Conv}(\mR_{\rm MTP}(\rho)),{\rm Conv}(\setETO(\rho))\subset\setTO(\rho),
\end{equation}
and $ {\rm Conv}(\mR_{\rm MTP}(\rho))\subset{\rm Conv}(\setETO(\rho)) $ for incoherent $ \rho $~\cite{Lostaglio22_MTP1}.
Let us begin with the paradigm of elementary thermal operations~\cite{Lostaglio_18_ETO}, which can be decomposed into a series of two-level thermal operations.
\begin{defn}\label{def:ETO_def} (ETO)	An ETO sequence $\Phi:\mS(\mH_{S})\rightarrow\mS(\mH_{S}) $ can be written as a series of channels,
$ 		\Phi =	\Phi_N\circ\Phi_{N-1}\cdots\circ\Phi_1, $
	such that each $\Phi_i$ can be unravelled into three steps:
	\begin{enumerate}
		\item append a Gibbs state $ \tau^\beta(H_{R_{i}}) = e^{-\beta H_{R_{i}}}/Z_{R_{i}}\in\mS(\mH_{R_{i}}) $ with a suitable Hamiltonian $ H_{R_{i}}\in\mL(\mH_{R_{i}}) $ and corresponding partition function $ Z_{R_{i}} = \Tr[e^{-\beta H_{R_{i}}}] $,
		\item apply a two-system-level energy-preserving unitary $ V_{k_{i}l_{i}}\in\mL(\mH_{S}\otimes\mH_{R_{i}}) $ to the state $ \rho\otimes\tau^{\beta}(H_{R_{i}}) $, such that each  $ V_{k_{i}l_{i}} $ acts non-trivially on at most 2 energy levels $\ket{k_i}_S,\ket{l_i}_S$. Although $V_{k_{i}l_{i}}$ involves many other energy levels of the bath, we refer to it as a two-system-level unitary due to its effect on $S$. Finally,
		\item trace out the bath part with $ \Tr_{R_{i}} $.
	\end{enumerate}
\end{defn} 	

In Def.~\ref{def:ETO_def}, each two-system-level, energy-preserving unitary can be written as $ V_{kl} = v_{kl}\oplus\1_{\setminus (k,l) } $ with $ v_{kl}\in\mL(\rm{span}\{\ket{k}_{S},\ket{l}_{S}\}\otimes\mH_{R_{i}}) $, $ \1_{\setminus (k,l)}\in \mL((\mH_{S}\setminus\rm{span}\{\ket{k}_{S},\ket{l}_{S}\})\otimes\mH_{R_{i}}) $, and $ [H_S+H_{R_{i}},v_{kl}]=0 $.
Importantly, such an ETO concatenation indicates that each $\Phi_i$ in the sequence is implemented with a \emph{fresh} bath: the final state $\sigma \in\mS(\mH_{S})$ is given by $\sigma = \Phi_N\circ\Phi_{N-1}\cdots\circ\Phi_1(\rho)$, where
	\begin{equation}
		\label{eq:ETO_def_1}
	\Phi_{i}(\varrho) = \Tr_{R_{i}}\left[V_{k_{i}l_{i}}\left(\varrho\otimes\tau^{\beta}(H_{R_{i}})\right)\left(V_{k_{i}l_{i}}\right)^{\dagger}\right]. 
\end{equation} 
Intuitively, the irreversible loss of information into bath $R_i$ in each concatenation produces an in-built Markovian behaviour.
Therefore, it is unsurprising that in general there exists a strict gap between sets $\mR_{\rm ETO}(\rho) \subsetneq \mR_{\rm TO}(\rho)$~\footnote{Conventionally, convex combinations of different ETO sequences are also allowed and will be used in the proof of Theorem~\ref{cor:CETO=CTO}. Yet, for Theorem~\ref{thm:GC-ETO_TO} a sequence of ETOs without convex combinations suffices.}. However, ETOs are not fully Markovian either, since the bath goes out of equilibrium during individual evolutions $ \Phi_i $, in contrast to MTPs~\cite{Lostaglio22_MTP1} (see \suppl~ for the formal definition of MTPs). 
We attempt to supplement ETOs with additional non-Markovianity. The first approach is to assume that the baths used in ETOs are controllable for the full duration of the entire process, instead of being traced out after each two-system-level operation.


\begin{defn}\label{def:NMETO_def_1} (nM-ETO)
	A non-Markovian elementary thermal operation is a quantum map acting on the system of interest $ \rho\in\mS(\mH_{S}) $ with Hamiltonian $ H_S = \sum_{i}E_i\ketbra{i}{i}_{S}\in\mL(\mH_{S}) $, which is induced by: 
	\begin{enumerate}
		\item appending a Gibbs state $ \tau^\beta(H_R) $ with a suitable Hamiltonian $ H_R\in\mL(\mH_{R}) $,
		\item applying a series of two-system-level energy-preserving unitaries $ \{U_{k_{1}l_{1}},\cdots,U_{k_{N}l_{N}}\}\subset\mL(\mH_{S}\otimes\mH_{R}) $ to the state $ \rho\otimes\tau^{\beta}(H_{R}) $, and 
		\item tracing out the bath $R$.
	\end{enumerate}
	More explicitly, a state $ \rho $ undergoing such an nM-ETO operation transforms into the final state
	\begin{equation}\label{eq:NMETO_def}
		\sigma  = \Tr_{R}\left[\left(\prod_{i}U_{k_il_i}\right) \left(\rho \otimes\tau^\beta(H_R)\right) \left(\prod_{i}U_{k_il_i} \right)^\dagger\right].
	\end{equation}
We refer to $N$ as the number of two-system-level unitaries in the nM-ETO sequence.
\end{defn}  
Two-system-level, energy-preserving unitaries $U_{kl}$ can be written similarly as before. 
Moreover, we allow for the case where $ k_{i}=l_{i} $, which corresponds to controlled operations $	U_{kk} = (\1_{S}-\dm{k}_{S})\otimes \1_{R} + \dm{k}_{S}\otimes U_{R}$, with unitaries $ U_{R}\in\mL(\mH_{R}) $ such that $ [U_{R},H_{R}] = 0 $. However, we show in the \suppl~ that such operations are unnecessary when the input $ \rho $ is initially energy-incoherent. 

It is easy to see that nM-ETOs are more powerful than ETOs due to the added power of retaining previous baths. Nevertheless, the characterization of nM-ETO is not obvious at this point. Additional insight into this extension comes from another attempt to incorporate non-Markovianity by considering the processes that are ETOs on the composite space of the system \emph{plus any Gibbs state catalyst}, see Figure \ref{fig:illustration} for an illustration. 


\begin{figure}[t!]
	\centering
	\includegraphics[width=0.92\columnwidth]{Fig1}
	\caption{Comparison between ETOs and GC-ETOs. The upper part depicts an ETO sequence applied to a three-dimensional system. At each step, two levels (highlighted in light blue) interact with a refreshed thermal bath (in red), which becomes athermal at the end (in grey with stripes). When two new levels are chosen, the previous bath is discarded. The lower part portrays a GC-ETO process with a catalyst starting from a Gibbs state (in red). During the process, the catalyst goes out of equilibrium (in grey with stripes) but is rethermalized by an external bath at the end via the thermalizing channel $ \Phi_{\rm Th} $.}
	\label{fig:illustration}
\end{figure}

\begin{defn} (GC-ETO)\label{def:NMETO_def_2}
		A transformation $\rho\rightarrow\sigma$ is achievable by a sequence of Gibbs-catalytic elementary thermal operations if there exists a Gibbs state  $ \tau^{\beta}(H_{C})\in\mS(\mH_{C}) $, such that 
		$ 	\rho\otimes \tau^{\beta}(H_{C}) \rightarrow\sigma\otimes \tau^{\beta}(H_{C}) $
	is possible via an	ETO sequence. In other words, $\sigma\otimes \tau^{\beta}(H_{C}) \in \mR_{\rm ETO}\left(\rho\otimes \tau^{\beta}(H_{C})\right)$. 
\end{defn}
Since Gibbs state catalysts cannot provide any advantage for TOs, GC-ETOs can never exceed TOs. Intuitively, nM-ETOs and GC-ETOs are very similar: both allow for the use of Gibbs states beyond a single two-level thermal process. Indeed, by using the fact that full thermalization of a state is achievable via ETOs, we can easily show the following equivalence.
\begin{lem}\label{lem:nm_vs_gc}
$\mR_{\rm nM-ETO} (\rho)= \mR_{\rm GC-ETO}(\rho)$ for all states $\rho$. 
\end{lem}
\begin{proof} In one direction this is simple: all GC-ETOs are clearly nM-ETOs when the catalyst is included as a bath, i.e., $\tau^{\beta}(H_{R}) = \tau^{\beta}(H_{C})\bigotimes_{i}\tau^{\beta}(H_{R_{i}}) $, where $ H_{R} = H_{C}+\sum_{i}H_{R_{i}} $. 

In the converse direction, if $\sigma =\Phi (\rho)$ where $\Phi$ is an nM-ETO sequence with bath $R$ and unitaries $\lbrace U_{k_il_i} \rbrace_i$, it is equivalent to the following GC-ETO: first, set the catalyst $C$ to be $R$, and note that $\lbrace U_{k_il_i} \rbrace_i$ are valid ETOs on the system $SC$, acting on two global energy eigenstates of $SC$ without the use of an external bath. This results in an intermediate state $\sigma_{SC}$ where the reduced state on $S$ is already the target state $\sigma$. 
To restore the catalyst, simply perform $\1_{S}\otimes \Phi_{\rm Th}$($\sigma_{SC}$), where $\Phi_{\rm Th} $ is the fully thermalizing map that brings any input state $\sigma_C\in\mS(\mH_{C})$ to its thermal state, which can be performed by ETO sequences. Furthermore, since this map brings every density matrix to a fixed point, it also destroys all external correlations with $S$, and we obtain the uncorrelated state $ \sigma $ at the end. 
\end{proof}
Also, note that from the statement of the lemma, ${\rm Conv}(\mR_{\rm nM-ETO} (\rho))={\rm Conv}( \mR_{\rm GC-ETO}(\rho))$.

Lemma~\ref{lem:nm_vs_gc} formalizes the idea that using Gibbs states catalysts enable us to overcome Markovianity arising from the sequential action of two-level thermal processes. The same logic applied to Gibbs-catalytic Markovian thermal processes (GC-MTPs) results in $ \setNMETO(\rho) \subset \setGCMTP(\rho) $; see \suppl~ for details.

\emph{Gibbs catalysts bridge CETOs and CTOs.}---
We establish our first main result, which holds for arbitrary initial states.
\begin{thm}\label{thm:GC-ETO_TO} 
	For any initial state $ \rho $, the interior of TO reachable set is a subset of GC-ETO reachable set, i.e. $ {\rm int}(\setTO(\rho)) \subset {\rm Conv}(\setGCETO(\rho)) $.
\end{thm}
\begin{proof}
The proof of this theorem relies on a more technical Lemma \ref{lem:NMETO_TO}, that any TO can be approximated by nM-ETOs with arbitrary precision. Furthermore, using the equivalence of $ \setGCETO $ and $ \setNMETO $ in Lemma \ref{lem:nm_vs_gc}, we arrive at the desired statement.
\end{proof}
Note that we allow convex combinations of GC-ETO channels as in the original definition of ETOs in~\cite{Lostaglio_18_ETO} that included convex combinations of ETO channels.

\begin{lem}\label{lem:NMETO_TO}
	Any TO channel $\Phi_{\rm TO}$ with inverse temperature $\beta$ can be approximated by a nM-ETO sequence $\Phi_{\rm nM-ETO}$ with the same inverse temperature, with an error vanishing with the number of steps $N$ in $\Phi_{\rm nM-ETO}$.
\end{lem}
\begin{proof}
	Consider $\Phi_{\rm TO}: \mS(\mH_{S}) \rightarrow\mS(\mH_{S})$ that 
	admits the dilated form $	\Phi_{\rm TO}(\rho) = \Tr_{R}\left[U\left(\rho\otimes\tau^\beta(H_R)\right)U^\dagger\right]$,
	with some bath Hamiltonian $ H_R\in\mL(\mH_{R}) $ and an energy-preserving unitary $ U\in\mL(\mH_{S}\otimes\mH_{R}) $, i.e. $ [U,H_S+H_R] = 0 $. 
	Without loss of generality, $U$ must be generated by turning on some interaction Hamiltonian $ H_{\rm int}\in\mL(\mH_{S}\otimes\mH_{R}) $, that is $ U = e^{-it(H_{S}+H_{R}+H_{\rm int})} = U_tU_{0} = U_{0}U_t $, where $U_t = e^{-itH_{\rm int}}$, and the second and third equalities follow from 
	$ [H_{\rm int},H_{S}+H_{R}] = 0 $. The unitary $ U_{0} $ indicates a natural time-evolution during the interaction, which is typically irrelevant. In particular, $ U_{0} $ has no effect for energy populations or amplitudes of off-diagonal terms. Hence, focusing on $U_t$, let us note that any Hermitian operator $ H_{\rm int}^\dagger = H_{\rm int}$ can be expanded in the system energy eigenbasis as
	\begin{align}\label{eq:Hint}
		H_{\rm int} &= \sum_{k,l;k<l}(H_{kl}+H_{lk}) + \sum_{j}H_{jj},%\\
	\end{align} 
	where $H_{kl} = \ketbra{k}{l}_{S}\otimes B_{kl}$, and bath operators $ B_{kl} = B_{lk}^{\dagger}\in\mL(\mH_{R}) $. %and system energy eigenbasis $ \{\ket{i}_{S}\} $.
	Now observe that
	\begin{equation}
		[H_{kl},H_S+H_R] = \ketbra{k}{l}_{S}\otimes \left((E_l-E_k)B_{kl} + [B_{kl},H_R]\right).
	\end{equation}
	The energy-preserving condition translates into $\sum_{j}\ketbra{j}{j}_{S}\otimes[B_{jj},H_{R}] +\sum_{k,l;k\neq l}\ketbra{k}{l}_{S}\otimes \left((E_l-E_k)B_{kl} + [B_{kl},H_R]\right)=0$,	requiring $ [H_{kl}, H_S+H_R] = 0$ for all values of $k,l$.
	This implies that each two-level interaction term $ H_{kl}+H_{lk} $, and also the control term $ H_{jj} $ commutes with the total Hamiltonian $ H_{S}+H_{R} $.
	
	Next, we invoke the Lie-Trotter formula~\cite{Trotter59_Trotter, Suzuki76_Trotter} for generally non-commuting square matrices $ A_i $:	$ e^{s\sum_iA_i} = \prod_ie^{sA_i} + \mO(s^2) $, where $ \mO(x) $ is an error of order $ x $.
	This allows us to write 
	\begin{equation}\label{key}
		e^{-i\delta_tH_{\rm int}} = \prod_{k,l;k<l}e^{-i\delta_t(H_{kl}+H_{lk})}\prod_{j}e^{-i\delta_tH_{jj}} + \mO(\delta_t^{2}).
	\end{equation}
	Thus, by fixing a number $M $ and correspondingly $ \delta_{t} = t/M $, we have that $U_t$ is well-approximated by the following expression: 
	\begin{equation}\label{eq:Trottered_TO}
	U_t= \left[\prod_{k,l;k<l}e^{-i\frac{t}{M}(H_{kl}+H_{lk})}\prod_{j}e^{-i\frac{t}{M}H_{jj}}\right]^{M} + \mO\left(\frac{1}{M}\right),
	\end{equation}
 where the first term of the RHS corresponds to an ETO with a number of steps $N \propto M$.
	Since $ H_{kl}+H_{lk} $ and $ H_{jj} $ are Hermitian and energy-preserving, unitaries $ e^{-i\frac{t}{M}(H_{kl}+H_{lk})} $ and $ e^{-i\frac{t}{M}H_{jj}} $ correspond to $ U_{kl} $ or $ U_{jj} $ in Definition~\ref{def:NMETO_def_1}. Hence, as $ N\rightarrow\infty $, an nM-ETO channel $\Phi_{\rm nM-ETO}$ can emulate $\Phi_{\rm TO}$ with vanishing error. 
	Additionally, we remark that, although in Eq.~\eqref{eq:Trottered_TO}, the conditional unitaries of the form $U_{jj}$ are interlaced with $U_{kl}$, it is possible to shuffle the controlled operations $U_{jj}$ to the beginning of the sequence through suitable transformations of other unitaries. Moreover, for initial states $ \rho $ that are diagonal, these unitaries can be ignored after the transformation; see \suppl. 
\end{proof}



To see a simple example that manifests Lemma \ref{lem:NMETO_TO}, consider a qutrit state $ \rho\in\mS(\mH_{S}) $ diagonal in the energy eigenbasis $ \{\ket{0}_{S},\ket{1}_{S},\ket{2}_{S}\} $. The system Hamiltonian is set to be $ H_{S} = E(\dm{1}_{S}+\dm{2}_{S})$, satisfying $ e^{-\beta E} = 1/2 $. The population of $\rho$ is given as $ \pstate = (0,0.5,0.5) $. A transition from $ \rho $ to an energy-diagonal state $\sigma$ having populations $ \qstate = (1,0,0) $ is feasible through TOs, because their thermo-majorization curves coincide. However, by ETOs, the maximum ground state population we can achieve is $ 0.75 $~\footnote{This is the case when levels $ \ket{0}_{S},\ket{1}_{S} $ and then levels $ \ket{0}_{S},\ket{2}_{S} $ interact maximally, for example. See~\cite{Son2022_CETO} for a complete characterization of $ \setETO(\rho) $ for any three-dimensional incoherent state $ \rho $.}. 
We construct a GC-ETO process with a Gibbs state catalyst $ \tau^{\beta}(H_{C}) $ that approximates the TO limit, $\qstate$, with arbitrarily good accuracy. 
The Gibbs catalyst Hamiltonian is chosen to be  
%\begin{equation}%\label{eq:cat_ideal_bath}
	$H_{C} = \sum_{n=0}^{D-1}\sum_{j=1}^{\delta_{n}}nE\dm{n,j}_{C}$,
%\end{equation}
where the energy spectrum is equidistant until the largest energy value $ (D-1)E $. We choose a bath with degeneracy $ \delta_{n}=2^{n} $ growing exponentially with the corresponding energy. 
Furthermore, each fixed energy subspace of the catalyst can always be divided into two subspaces of equal dimension, i.e. $\delta_n /2 = 2^{n-1}$ is always an integer for $n\geq 1$.

Now we define the elements of the interaction Hamiltonian for $SC$ such that half the terms are the interactions between $\ket{0}_S$ and $\ket{1}_S$, while the other half is between $ \ket{0}_{S} $ and $ \ket{2}_{S} $.
More concretely,
\begin{align}\label{key}
	H_{01}^{(n,k)} &= \ketbra{0}{1}_{S}\otimes\ketbra{n,k}{n-1,k}_{C},\\
	H_{02}^{(n,k)} &= \ketbra{0}{2}_{S}\otimes\ketbra{n,2^{n-1}+k}{n-1,k}_{C},\\
	H_{10}^{(n,k)} &= \left(H_{01}^{(n,k)}\right)^{\dagger},\quad H_{20}^{(n,k)} =  \left(H_{02}^{(n,k)}\right)^{\dagger},
\end{align}
for $ k = 1,\cdots,2^{n-1} $. 
The Hermitian pairs 
$ H_{01}^{(n,k)}+H_{10}^{(n,k)} $ and 
$ H_{02}^{(n,k)}+H_{20}^{(n,k)} $ are all energy-preserving, and can generate unitaries $X_{x}^{(n,k)} $
that swap $ \ket{0}_{S}\ket{n,k+(x-1)2^{n-1}}_{C} $ and $ \ket{x}_{S}\ket{n-1,k}_{C} $
for $ x = 1,2 $, while keeping all other states unchanged. 
Hence, we have now designed two-level unitaries $ X_{x}^{(n,k)} $ on $ SC $, and they commute with each other $ [X_{x}^{(n,k)},X_{x^{\prime}}^{(n^{\prime},k^{\prime})}] = 0 $, for all $ x,x^{\prime},n,n^{\prime},k,k^{\prime} $.
Starting from the initial state $ \rho\otimes\tau^{\beta}(H_{C}) $, we may obtain the final, energy-incoherent state $ \sigma = \Tr_{C}[U (\rho\otimes\tau^{\beta})U^{\dagger}] $, where
%\begin{align}\label{eq:sigma_Trotter}
	$U = \prod_{n = 1}^{D-1}\prod_{k=1}^{2^{n-1}}\prod_{x = 0,1}X_{x}^{(n,k)}$.
%\end{align} 
We label the energy population vector of $ \sigma $ as $ \qstate' $.
Except for the populations of $ \{\ket{1}_{S}\ket{D-1,j}_{C}, \ket{2}_{S}\ket{D-1,j}_{C} \}_{j=1}^{2^{D-1}} $, which remain untouched by the unitaries, all populations from $ \ket{1}_{S} $ and $ \ket{2}_{S} $ are now transferred to $ \ket{0}_{S} $. 
Finally, we evaluate $ \bra{D-1,j}\tau^{\beta}(H_{C})\ket{D-1,j} = \frac{e^{-\beta (D-1)E}}{Z_{C}} = \frac{2^{1-D}}{D} $, since $e^{-\beta E} = 1/2$ and $Z_C = \sum_{n=0}^{D-1} \delta_n e^{-\beta n E} = D$. With this, we obtain the target state population $\qstate' = (1-D^{-1},D^{-1}/2,D^{-1}/2) $, which approaches the TO limit as $ D\rightarrow\infty $.

At this point, it is apparent that although each operation involves at most two energy levels on $ SC $, the catalyst $ C $ is in principle large. While our results highlight that the gap between TOs and ETOs can always be addressed by a Gibbs catalyst, ideally, a simpler Gibbs catalyst, e.g. a smaller collection of bosonic modes, would be desirable. This problem remains open for future work.
Meanwhile, from Theorem~\ref{thm:GC-ETO_TO}, we arrive at our second main result, namely when arbitrary catalysts are allowed, the hierarchy of thermal processes previously studied --- MTPs, ETOs, and TOs --- collapses for incoherent initial states.
\begin{thm}\label{cor:CETO=CTO} 
For any incoherent initial state $\rho $,
\begin{equation}
	{\rm int}( \setCTO(\rho)) \subset {\rm Conv}(\setCETO(\rho)) = {\rm Conv}(\mR_{\rm CMTP}(\rho)).
\end{equation} 
\end{thm} 
The proof of Theorem \ref{cor:CETO=CTO} can be found in the \suppl. This result reverses the ordering between classes as shown in Eq.~\eqref{eq:non-catalytic-hierarchy} by using catalysts and thus implies that catalytically-reachable sets via MTPs, ETOs and TOs (approximately) converge to the same set for incoherent initial states $\rho$.
The set $ \setCTO(\rho) $ for incoherent $ \rho $ is characterized by the family of R\'enyi divergences~\cite{Brandao15_2ndlaws}, which can then be applied to $ \setCETO(\rho) $, keeping in mind that modifications are required to capture the characterization of the interior set of $\mR_{\rm CTO}$.

\emph{Discussions and Conclusions.} --- 
ETOs have a merit of being executable with simpler building blocks. However, the characterization of state transition conditions has been notoriously hard to study.
In this work, we show that when the catalytic versions are considered, the characterization of ETOs is simplified to that of TOs with Theorem~\ref{cor:CETO=CTO}. The main ingredient for this simplification is the counter-intuitive use of Gibbs states as catalysts, whose preparation would be easier, given access to thermalization mechanisms. The technical result of Lemma~\ref{lem:NMETO_TO} is, in its spirit, analogous to that of recent results~\cite{Marvian2022_locality} in which a catalyst enables the concatenations of local symmetric operations to reproduce all symmetric operations, which are strictly larger than the set of local symmetric operations without a catalyst. While Lemma~\ref{lem:NMETO_TO} holds on the level of channels, to establish similar claims to Theorem~\ref{cor:CETO=CTO} for energy-coherent states, one would also need to consider energy-coherent catalysts, and further analyze the interplay between two coherent states. These cases have been long-standing open problems \cite{Lostaglio15_PRX_coh,Lostaglio15_coherence,Marvian14_coherence,Cwiklinski15_PRL,Gour22_coherence} in quantum thermodynamics, and are left for future work.
		
We make a final remark on the seemingly odd possibility of using Gibbs states, which are free for TOs, as catalysts. Typically, in a resource theory characterized by free operations $X$, $\tau$ being a free state indicates that for any $\rho$, the process
\begin{equation}\label{eq:def_free_states}
	\rho\xrightarrow{X}~\rho\otimes\tau
\end{equation} is always practicable. Free states according to this notion can never be useful for activating non-trivial state transitions. More concretely, if $\rho\otimes\tau \xrightarrow{X} ~\sigma\otimes\tau$, and if tracing out is also an allowed free operation, then automatically
$	\rho\xrightarrow{X}~\rho\otimes\tau \xrightarrow{X}~ \sigma\otimes\tau\xrightarrow{X}~\sigma$,
i.e., $\rho\xrightarrow{X}~\sigma$ is achievable for free, instead of catalytically, by definition. 
To understand the utility of Gibbs states as catalysts, one has to observe that, for ETOs and MTPs, Gibbs states are \emph{not} free in the most conservative sense as in Eq.~\eqref{eq:def_free_states}. 
Indeed, while ETOs allow for the usage of arbitrary bath structures, each bath expires after a single use of a two-system-level operation, and despite allowing for convex combinations (i.e. mixing different sequences by controlling on some additional randomness) the limitation persists.

Our work therefore provides a significant step forward in characterizing state transitions for catalytic versions of simple thermal processes (ETOs and MTPs), showing that they are approximately equivalent to the much more well-studied case of thermal operations. On the other hand, these findings also pave the way ahead to a simplified approach in designing generic thermal operations using only elementary interactions.

\section{Acknowledgements}
This work was supported by the start-up grant of the Nanyang Assistant Professorship of Nanyang Technological University, Singapore.
During the process of preparing this manuscript we were made aware of the overlap with \cite{korzekwateam23}, and we thank Jakub Czartowski, Alexssandre de Oliveira Junior, and Kamil Korzekwa for insightful discussions on this topic. We also thank Seok Hyung Lie for independent and constructive discussions and Marek Gluza for the careful reading of the manuscript.

%\bibliography{NMETO.bib}
%apsrev4-2.bst 2019-01-14 (MD) hand-edited version of apsrev4-1.bst
%Control: key (0)
%Control: author (8) initials jnrlst
%Control: editor formatted (1) identically to author
%Control: production of article title (0) allowed
%Control: page (0) single
%Control: year (1) truncated
%Control: production of eprint (0) enabled
\begin{thebibliography}{51}%
	\makeatletter
	\providecommand \@ifxundefined [1]{%
		\@ifx{#1\undefined}
	}%
	\providecommand \@ifnum [1]{%
		\ifnum #1\expandafter \@firstoftwo
		\else \expandafter \@secondoftwo
		\fi
	}%
	\providecommand \@ifx [1]{%
		\ifx #1\expandafter \@firstoftwo
		\else \expandafter \@secondoftwo
		\fi
	}%
	\providecommand \natexlab [1]{#1}%
	\providecommand \enquote  [1]{``#1''}%
	\providecommand \bibnamefont  [1]{#1}%
	\providecommand \bibfnamefont [1]{#1}%
	\providecommand \citenamefont [1]{#1}%
	\providecommand \href@noop [0]{\@secondoftwo}%
	\providecommand \href [0]{\begingroup \@sanitize@url \@href}%
	\providecommand \@href[1]{\@@startlink{#1}\@@href}%
	\providecommand \@@href[1]{\endgroup#1\@@endlink}%
	\providecommand \@sanitize@url [0]{\catcode `\\12\catcode `\$12\catcode
		`\&12\catcode `\#12\catcode `\^12\catcode `\_12\catcode `\%12\relax}%
	\providecommand \@@startlink[1]{}%
	\providecommand \@@endlink[0]{}%
	\providecommand \url  [0]{\begingroup\@sanitize@url \@url }%
	\providecommand \@url [1]{\endgroup\@href {#1}{\urlprefix }}%
	\providecommand \urlprefix  [0]{URL }%
	\providecommand \Eprint [0]{\href }%
	\providecommand \doibase [0]{https://doi.org/}%
	\providecommand \selectlanguage [0]{\@gobble}%
	\providecommand \bibinfo  [0]{\@secondoftwo}%
	\providecommand \bibfield  [0]{\@secondoftwo}%
	\providecommand \translation [1]{[#1]}%
	\providecommand \BibitemOpen [0]{}%
	\providecommand \bibitemStop [0]{}%
	\providecommand \bibitemNoStop [0]{.\EOS\space}%
	\providecommand \EOS [0]{\spacefactor3000\relax}%
	\providecommand \BibitemShut  [1]{\csname bibitem#1\endcsname}%
	\let\auto@bib@innerbib\@empty
	%</preamble>
	\bibitem [{\citenamefont {Janzing}\ \emph {et~al.}(2000)\citenamefont
		{Janzing}, \citenamefont {Wocjan}, \citenamefont {Zeier}, \citenamefont
		{Geiss},\ and\ \citenamefont {Beth}}]{Janzing00_TO}%
	\BibitemOpen
	\bibfield  {author} {\bibinfo {author} {\bibfnamefont {D.}~\bibnamefont
			{Janzing}}, \bibinfo {author} {\bibfnamefont {P.}~\bibnamefont {Wocjan}},
		\bibinfo {author} {\bibfnamefont {R.}~\bibnamefont {Zeier}}, \bibinfo
		{author} {\bibfnamefont {R.}~\bibnamefont {Geiss}},\ and\ \bibinfo {author}
		{\bibfnamefont {T.}~\bibnamefont {Beth}},\ }\bibfield  {title} {\bibinfo
		{title} {Thermodynamic cost of reliability and low temperatures: Tightening
			{L}andauer's principle and the second law},\ }\href
	{https://doi.org/10.1023/A:1026422630734} {\bibfield  {journal} {\bibinfo
			{journal} {Int. J. Th. Phys.}\ }\textbf {\bibinfo {volume} {39}},\ \bibinfo
		{pages} {2717} (\bibinfo {year} {2000})}\BibitemShut {NoStop}%
	\bibitem [{\citenamefont {Brand\~ao}\ \emph {et~al.}(2013)\citenamefont
		{Brand\~ao}, \citenamefont {Horodecki}, \citenamefont {Oppenheim},
		\citenamefont {Renes},\ and\ \citenamefont {Spekkens}}]{Brandao13_TRT}%
	\BibitemOpen
	\bibfield  {author} {\bibinfo {author} {\bibfnamefont {F.~G. S.~L.}\
			\bibnamefont {Brand\~ao}}, \bibinfo {author} {\bibfnamefont {M.}~\bibnamefont
			{Horodecki}}, \bibinfo {author} {\bibfnamefont {J.}~\bibnamefont
			{Oppenheim}}, \bibinfo {author} {\bibfnamefont {J.~M.}\ \bibnamefont
			{Renes}},\ and\ \bibinfo {author} {\bibfnamefont {R.~W.}\ \bibnamefont
			{Spekkens}},\ }\bibfield  {title} {\bibinfo {title} {Resource theory of
			quantum states out of thermal equilibrium},\ }\href
	{https://doi.org/10.1103/PhysRevLett.111.250404} {\bibfield  {journal}
		{\bibinfo  {journal} {Phys. Rev. Lett.}\ }\textbf {\bibinfo {volume} {111}},\
		\bibinfo {pages} {250404} (\bibinfo {year} {2013})}\BibitemShut {NoStop}%
	\bibitem [{\citenamefont {Gour}\ \emph {et~al.}(2015)\citenamefont {Gour},
		\citenamefont {M{\"u}ller}, \citenamefont {Narasimhachar}, \citenamefont
		{Spekkens},\ and\ \citenamefont {{Yunger Halpern}}}]{Gour15_TRTreview}%
	\BibitemOpen
	\bibfield  {author} {\bibinfo {author} {\bibfnamefont {G.}~\bibnamefont
			{Gour}}, \bibinfo {author} {\bibfnamefont {M.~P.}\ \bibnamefont
			{M{\"u}ller}}, \bibinfo {author} {\bibfnamefont {V.}~\bibnamefont
			{Narasimhachar}}, \bibinfo {author} {\bibfnamefont {R.~W.}\ \bibnamefont
			{Spekkens}},\ and\ \bibinfo {author} {\bibfnamefont {N.}~\bibnamefont
			{{Yunger Halpern}}},\ }\bibfield  {title} {\bibinfo {title} {The resource
			theory of informational nonequilibrium in thermodynamics},\ }\href
	{https://doi.org/https://doi.org/10.1016/j.physrep.2015.04.003} {\bibfield
		{journal} {\bibinfo  {journal} {Physics Reports}\ }\textbf {\bibinfo {volume}
			{583}},\ \bibinfo {pages} {1} (\bibinfo {year} {2015})}\BibitemShut {NoStop}%
	\bibitem [{\citenamefont {Ng}\ and\ \citenamefont
		{Woods}(2018)}]{Ng18_Qthermo_book}%
	\BibitemOpen
	\bibfield  {author} {\bibinfo {author} {\bibfnamefont {N.~H.~Y.}\
			\bibnamefont {Ng}}\ and\ \bibinfo {author} {\bibfnamefont {M.~P.}\
			\bibnamefont {Woods}},\ }\bibinfo {title} {Resource theory of quantum
		thermodynamics: Thermal operations and second laws},\ in\ \href
	{https://doi.org/10.1007/978-3-319-99046-0_26} {\emph {\bibinfo {booktitle}
			{Thermodynamics in the Quantum Regime: Fundamental Aspects and New
				Directions}}},\ \bibinfo {editor} {edited by\ \bibinfo {editor}
		{\bibfnamefont {F.}~\bibnamefont {Binder}}, \bibinfo {editor} {\bibfnamefont
			{L.~A.}\ \bibnamefont {Correa}}, \bibinfo {editor} {\bibfnamefont
			{C.}~\bibnamefont {Gogolin}}, \bibinfo {editor} {\bibfnamefont
			{J.}~\bibnamefont {Anders}},\ and\ \bibinfo {editor} {\bibfnamefont
			{G.}~\bibnamefont {Adesso}}}\ (\bibinfo  {publisher} {Springer International
		Publishing},\ \bibinfo {address} {Cham},\ \bibinfo {year} {2018})\ pp.\
	\bibinfo {pages} {625--650}\BibitemShut {NoStop}%
	\bibitem [{\citenamefont {Horodecki}\ and\ \citenamefont
		{Oppenheim}(2013)}]{Horodecki13_fundamental}%
	\BibitemOpen
	\bibfield  {author} {\bibinfo {author} {\bibfnamefont {M.}~\bibnamefont
			{Horodecki}}\ and\ \bibinfo {author} {\bibfnamefont {J.}~\bibnamefont
			{Oppenheim}},\ }\bibfield  {title} {\bibinfo {title} {Fundamental limitations
			for quantum and nanoscale thermodynamics},\ }\href
	{https://doi.org/https://doi.org/10.1038/ncomms3059} {\bibfield  {journal}
		{\bibinfo  {journal} {Nat. Commun.}\ }\textbf {\bibinfo {volume} {4}},\
		\bibinfo {pages} {1} (\bibinfo {year} {2013})}\BibitemShut {NoStop}%
	\bibitem [{\citenamefont {Reeb}\ and\ \citenamefont {Wolf}(2014)}]{Reeb_2014}%
	\BibitemOpen
	\bibfield  {author} {\bibinfo {author} {\bibfnamefont {D.}~\bibnamefont
			{Reeb}}\ and\ \bibinfo {author} {\bibfnamefont {M.~M.}\ \bibnamefont
			{Wolf}},\ }\bibfield  {title} {\bibinfo {title} {An improved landauer
			principle with finite-size corrections},\ }\href
	{https://doi.org/10.1088/1367-2630/16/10/103011} {\bibfield  {journal}
		{\bibinfo  {journal} {New J. Phys.}\ }\textbf {\bibinfo {volume} {16}},\
		\bibinfo {pages} {103011} (\bibinfo {year} {2014})}\BibitemShut {NoStop}%
	\bibitem [{\citenamefont {Yunger~Halpern}\ \emph {et~al.}(2016)\citenamefont
		{Yunger~Halpern}, \citenamefont {Faist}, \citenamefont {Oppenheim},\ and\
		\citenamefont {Winter}}]{YungerHalpern16_noncommuting}%
	\BibitemOpen
	\bibfield  {author} {\bibinfo {author} {\bibfnamefont {N.}~\bibnamefont
			{Yunger~Halpern}}, \bibinfo {author} {\bibfnamefont {P.}~\bibnamefont
			{Faist}}, \bibinfo {author} {\bibfnamefont {J.}~\bibnamefont {Oppenheim}},\
		and\ \bibinfo {author} {\bibfnamefont {A.}~\bibnamefont {Winter}},\
	}\bibfield  {title} {\bibinfo {title} {Microcanonical and resource-theoretic
			derivations of the thermal state of a quantum system with noncommuting
			charges},\ }\href {https://doi.org/10.1038/ncomms12051} {\bibfield  {journal}
		{\bibinfo  {journal} {Nat. Commun.}\ }\textbf {\bibinfo {volume} {7}},\
		\bibinfo {pages} {12051} (\bibinfo {year} {2016})}\BibitemShut {NoStop}%
	\bibitem [{\citenamefont {Scharlau}\ and\ \citenamefont
		{Mueller}(2018)}]{Scharlau18_hornslemma}%
	\BibitemOpen
	\bibfield  {author} {\bibinfo {author} {\bibfnamefont {J.}~\bibnamefont
			{Scharlau}}\ and\ \bibinfo {author} {\bibfnamefont {M.~P.}\ \bibnamefont
			{Mueller}},\ }\bibfield  {title} {\bibinfo {title} {Quantum {H}orn's lemma,
			finite heat baths, and the third law of thermodynamics},\ }\href
	{https://doi.org/10.22331/q-2018-02-22-54} {\bibfield  {journal} {\bibinfo
			{journal} {{Quantum}}\ }\textbf {\bibinfo {volume} {2}},\ \bibinfo {pages}
		{54} (\bibinfo {year} {2018})}\BibitemShut {NoStop}%
	\bibitem [{\citenamefont {Woods}\ \emph {et~al.}(2019)\citenamefont {Woods},
		\citenamefont {Ng},\ and\ \citenamefont {Wehner}}]{Woods19_engines}%
	\BibitemOpen
	\bibfield  {author} {\bibinfo {author} {\bibfnamefont {M.~P.}\ \bibnamefont
			{Woods}}, \bibinfo {author} {\bibfnamefont {N.~H.~Y.}\ \bibnamefont {Ng}},\
		and\ \bibinfo {author} {\bibfnamefont {S.}~\bibnamefont {Wehner}},\
	}\bibfield  {title} {\bibinfo {title} {The maximum efficiency of nano heat
			engines depends on more than temperature},\ }\href
	{https://doi.org/10.22331/q-2019-08-19-177} {\bibfield  {journal} {\bibinfo
			{journal} {{Quantum}}\ }\textbf {\bibinfo {volume} {3}},\ \bibinfo {pages}
		{177} (\bibinfo {year} {2019})}\BibitemShut {NoStop}%
	\bibitem [{\citenamefont {Yunger~Halpern}(2017)}]{YungerHalpern17_Realization}%
	\BibitemOpen
	\bibfield  {author} {\bibinfo {author} {\bibfnamefont {N.}~\bibnamefont
			{Yunger~Halpern}},\ }\bibinfo {title} {Toward physical realizations of
		thermodynamic resource theories},\ in\ \href
	{https://doi.org/10.1007/978-3-319-43760-6_8} {\emph {\bibinfo {booktitle}
			{Information and Interaction: Eddington, Wheeler, and the Limits of
				Knowledge}}},\ \bibinfo {editor} {edited by\ \bibinfo {editor} {\bibfnamefont
			{I.~T.}\ \bibnamefont {Durham}}\ and\ \bibinfo {editor} {\bibfnamefont
			{D.}~\bibnamefont {Rickles}}}\ (\bibinfo  {publisher} {Springer International
		Publishing},\ \bibinfo {address} {Cham},\ \bibinfo {year} {2017})\ pp.\
	\bibinfo {pages} {135--166}\BibitemShut {NoStop}%
	\bibitem [{\citenamefont {Perry}\ \emph {et~al.}(2018)\citenamefont {Perry},
		\citenamefont {\ifmmode \acute{C}\else \'{C}\fi{}wikli\ifmmode~\acute{n}\else
			\'{n}\fi{}ski}, \citenamefont {Anders}, \citenamefont {Horodecki},\ and\
		\citenamefont {Oppenheim}}]{Perry18_PRX_Crude}%
	\BibitemOpen
	\bibfield  {author} {\bibinfo {author} {\bibfnamefont {C.}~\bibnamefont
			{Perry}}, \bibinfo {author} {\bibfnamefont {P.}~\bibnamefont {\ifmmode
				\acute{C}\else \'{C}\fi{}wikli\ifmmode~\acute{n}\else \'{n}\fi{}ski}},
		\bibinfo {author} {\bibfnamefont {J.}~\bibnamefont {Anders}}, \bibinfo
		{author} {\bibfnamefont {M.}~\bibnamefont {Horodecki}},\ and\ \bibinfo
		{author} {\bibfnamefont {J.}~\bibnamefont {Oppenheim}},\ }\bibfield  {title}
	{\bibinfo {title} {A sufficient set of experimentally implementable thermal
			operations for small systems},\ }\href
	{https://doi.org/10.1103/PhysRevX.8.041049} {\bibfield  {journal} {\bibinfo
			{journal} {Phys. Rev. X}\ }\textbf {\bibinfo {volume} {8}},\ \bibinfo {pages}
		{041049} (\bibinfo {year} {2018})}\BibitemShut {NoStop}%
	\bibitem [{\citenamefont {Shiraishi}(2020)}]{Shiraishi2020_construction}%
	\BibitemOpen
	\bibfield  {author} {\bibinfo {author} {\bibfnamefont {N.}~\bibnamefont
			{Shiraishi}},\ }\bibfield  {title} {\bibinfo {title} {Two constructive proofs
			on d-majorization and thermo-majorization},\ }\href
	{https://doi.org/10.1088/1751-8121/abb041} {\bibfield  {journal} {\bibinfo
			{journal} {Journal of Physics A: Mathematical and Theoretical}\ }\textbf
		{\bibinfo {volume} {53}},\ \bibinfo {pages} {425301} (\bibinfo {year}
		{2020})}\BibitemShut {NoStop}%
	\bibitem [{\citenamefont {Lostaglio}(2019)}]{Lostaglio19_review}%
	\BibitemOpen
	\bibfield  {author} {\bibinfo {author} {\bibfnamefont {M.}~\bibnamefont
			{Lostaglio}},\ }\bibfield  {title} {\bibinfo {title} {An introductory review
			of the resource theory approach to thermodynamics},\ }\href
	{https://doi.org/10.1088/1361-6633/ab46e5} {\bibfield  {journal} {\bibinfo
			{journal} {Rep. Prog. Phys.}\ }\textbf {\bibinfo {volume} {82}},\ \bibinfo
		{pages} {114001} (\bibinfo {year} {2019})}\BibitemShut {NoStop}%
	\bibitem [{\citenamefont {Hu}\ and\ \citenamefont
		{Ding}(2019)}]{Hu2019_singlemode}%
	\BibitemOpen
	\bibfield  {author} {\bibinfo {author} {\bibfnamefont {X.}~\bibnamefont
			{Hu}}\ and\ \bibinfo {author} {\bibfnamefont {F.}~\bibnamefont {Ding}},\
	}\bibfield  {title} {\bibinfo {title} {Thermal operations involving a
			single-mode bosonic bath},\ }\href
	{https://doi.org/10.1103/PhysRevA.99.012104} {\bibfield  {journal} {\bibinfo
			{journal} {Phys. Rev. A}\ }\textbf {\bibinfo {volume} {99}},\ \bibinfo
		{pages} {012104} (\bibinfo {year} {2019})}\BibitemShut {NoStop}%
	\bibitem [{\citenamefont {vom Ende}(2022)}]{Ende2022_bath}%
	\BibitemOpen
	\bibfield  {author} {\bibinfo {author} {\bibfnamefont {F.}~\bibnamefont {vom
				Ende}},\ }\bibfield  {title} {\bibinfo {title} {Which bath {H}amiltonians
			matter for thermal operations?},\ }\href {https://doi.org/10.1063/5.0117534}
	{\bibfield  {journal} {\bibinfo  {journal} {J. Math. Phys.}\ }\textbf
		{\bibinfo {volume} {63}},\ \bibinfo {pages} {112202} (\bibinfo {year}
		{2022})}\BibitemShut {NoStop}%
	\bibitem [{\citenamefont {Lostaglio}\ \emph {et~al.}(2018)\citenamefont
		{Lostaglio}, \citenamefont {Alhambra},\ and\ \citenamefont
		{Perry}}]{Lostaglio_18_ETO}%
	\BibitemOpen
	\bibfield  {author} {\bibinfo {author} {\bibfnamefont {M.}~\bibnamefont
			{Lostaglio}}, \bibinfo {author} {\bibfnamefont {{\'{A}}.~M.}\ \bibnamefont
			{Alhambra}},\ and\ \bibinfo {author} {\bibfnamefont {C.}~\bibnamefont
			{Perry}},\ }\bibfield  {title} {\bibinfo {title} {Elementary {T}hermal
			{O}perations},\ }\href {https://doi.org/10.22331/q-2018-02-08-52} {\bibfield
		{journal} {\bibinfo  {journal} {{Quantum}}\ }\textbf {\bibinfo {volume}
			{2}},\ \bibinfo {pages} {52} (\bibinfo {year} {2018})}\BibitemShut {NoStop}%
	\bibitem [{\citenamefont {Jaynes}\ and\ \citenamefont
		{Cummings}(1963)}]{JCmodel}%
	\BibitemOpen
	\bibfield  {author} {\bibinfo {author} {\bibfnamefont {E.}~\bibnamefont
			{Jaynes}}\ and\ \bibinfo {author} {\bibfnamefont {F.}~\bibnamefont
			{Cummings}},\ }\bibfield  {title} {\bibinfo {title} {Comparison of quantum
			and semiclassical radiation theories with application to the beam maser},\
	}\href {https://doi.org/10.1109/PROC.1963.1664} {\bibfield  {journal}
		{\bibinfo  {journal} {Proc. IEEE}\ }\textbf {\bibinfo {volume} {51}},\
		\bibinfo {pages} {89} (\bibinfo {year} {1963})}\BibitemShut {NoStop}%
	\bibitem [{\citenamefont {Bu\ifmmode~\check{z}\else
			\v{z}\fi{}ek}(1989)}]{Buzek89_intensityJC}%
	\BibitemOpen
	\bibfield  {author} {\bibinfo {author} {\bibfnamefont {V.}~\bibnamefont
			{Bu\ifmmode~\check{z}\else \v{z}\fi{}ek}},\ }\bibfield  {title} {\bibinfo
		{title} {Jaynes-{C}ummings model with intensity-dependent coupling
			interacting with holstein-primakoff su(1,1) coherent state},\ }\href
	{https://doi.org/10.1103/PhysRevA.39.3196} {\bibfield  {journal} {\bibinfo
			{journal} {Phys. Rev. A}\ }\textbf {\bibinfo {volume} {39}},\ \bibinfo
		{pages} {3196} (\bibinfo {year} {1989})}\BibitemShut {NoStop}%
	\bibitem [{\citenamefont {Son}\ and\ \citenamefont {Ng}(2022)}]{Son2022_CETO}%
	\BibitemOpen
	\bibfield  {author} {\bibinfo {author} {\bibfnamefont {J.}~\bibnamefont
			{Son}}\ and\ \bibinfo {author} {\bibfnamefont {N.~H.~Y.}\ \bibnamefont
			{Ng}},\ }\href@noop {} {\bibinfo {title} {Catalysis in action via elementary
			thermal operations}} (\bibinfo {year} {2022}),\ \Eprint
	{https://arxiv.org/abs/2209.15213} {arXiv:2209.15213} \BibitemShut {NoStop}%
	\bibitem [{\citenamefont {Mazurek}\ and\ \citenamefont
		{Horodecki}(2018)}]{Mazurek2018_Decomp}%
	\BibitemOpen
	\bibfield  {author} {\bibinfo {author} {\bibfnamefont {P.}~\bibnamefont
			{Mazurek}}\ and\ \bibinfo {author} {\bibfnamefont {M.}~\bibnamefont
			{Horodecki}},\ }\bibfield  {title} {\bibinfo {title} {Decomposability and
			convex structure of thermal processes},\ }\href
	{https://doi.org/10.1088/1367-2630/aac057} {\bibfield  {journal} {\bibinfo
			{journal} {New J. Phys.}\ }\textbf {\bibinfo {volume} {20}},\ \bibinfo
		{pages} {053040} (\bibinfo {year} {2018})}\BibitemShut {NoStop}%
	\bibitem [{\citenamefont {Lostaglio}\ and\ \citenamefont
		{Korzekwa}(2022)}]{Lostaglio22_MTP1}%
	\BibitemOpen
	\bibfield  {author} {\bibinfo {author} {\bibfnamefont {M.}~\bibnamefont
			{Lostaglio}}\ and\ \bibinfo {author} {\bibfnamefont {K.}~\bibnamefont
			{Korzekwa}},\ }\bibfield  {title} {\bibinfo {title} {Continuous
			thermomajorization and a complete set of laws for {M}arkovian thermal
			processes},\ }\href {https://doi.org/10.1103/PhysRevA.106.012426} {\bibfield
		{journal} {\bibinfo  {journal} {Phys. Rev. A}\ }\textbf {\bibinfo {volume}
			{106}},\ \bibinfo {pages} {012426} (\bibinfo {year} {2022})}\BibitemShut
	{NoStop}%
	\bibitem [{\citenamefont {Korzekwa}\ and\ \citenamefont
		{Lostaglio}(2022)}]{Korzekwa22_MTP2}%
	\BibitemOpen
	\bibfield  {author} {\bibinfo {author} {\bibfnamefont {K.}~\bibnamefont
			{Korzekwa}}\ and\ \bibinfo {author} {\bibfnamefont {M.}~\bibnamefont
			{Lostaglio}},\ }\bibfield  {title} {\bibinfo {title} {Optimizing
			thermalization},\ }\href {https://doi.org/10.1103/PhysRevLett.129.040602}
	{\bibfield  {journal} {\bibinfo  {journal} {Phys. Rev. Lett.}\ }\textbf
		{\bibinfo {volume} {129}},\ \bibinfo {pages} {040602} (\bibinfo {year}
		{2022})}\BibitemShut {NoStop}%
	\bibitem [{Note1()}]{Note1}%
	\BibitemOpen
	\bibinfo {note} {Conventionally, convex combinations of different ETO
		sequences are also allowed and will be used in the proof of Theorem~\ref
		{cor:CETO=CTO}. Yet, for Theorem~\ref {thm:GC-ETO_TO} a sequence of ETOs
		without convex combinations suffices.}\BibitemShut {Stop}%
	\bibitem [{\citenamefont {Trotter}(1959)}]{Trotter59_Trotter}%
	\BibitemOpen
	\bibfield  {author} {\bibinfo {author} {\bibfnamefont {H.~F.}\ \bibnamefont
			{Trotter}},\ }\bibfield  {title} {\bibinfo {title} {On the product of
			semi-groups of operators},\ }\href
	{https://doi.org/https://doi.org/10.1090/S0002-9939-1959-0108732-6}
	{\bibfield  {journal} {\bibinfo  {journal} {Proc. Am. Math. Soc.}\ }\textbf
		{\bibinfo {volume} {10}},\ \bibinfo {pages} {545} (\bibinfo {year}
		{1959})}\BibitemShut {NoStop}%
	\bibitem [{\citenamefont {Suzuki}(1976)}]{Suzuki76_Trotter}%
	\BibitemOpen
	\bibfield  {author} {\bibinfo {author} {\bibfnamefont {M.}~\bibnamefont
			{Suzuki}},\ }\bibfield  {title} {\bibinfo {title} {Generalized {T}rotter's
			formula and systematic approximants of exponential operators and inner
			derivations with applications to many-body problems},\ }\href
	{https://doi.org/10.1007/BF01609348} {\bibfield  {journal} {\bibinfo
			{journal} {Commun. Math. Phys.}\ }\textbf {\bibinfo {volume} {51}},\ \bibinfo
		{pages} {183} (\bibinfo {year} {1976})}\BibitemShut {NoStop}%
	\bibitem [{Note2()}]{Note2}%
	\BibitemOpen
	\bibinfo {note} {This is the case when levels $ \left \lvert 0 \right \rangle
		_{S},\left \lvert 1 \right \rangle _{S} $ and then levels $ \left \lvert 0
		\right \rangle _{S},\left \lvert 2 \right \rangle _{S} $ interact maximally,
		for example. See~\cite {Son2022_CETO} for a complete characterization of $
		\protect \mathcal {R}_{\protect \rm ETO}(\rho ) $ for any three-dimensional
		incoherent state $ \rho $.}\BibitemShut {Stop}%
	\bibitem [{\citenamefont {Brand{\~a}o}\ \emph {et~al.}(2015)\citenamefont
		{Brand{\~a}o}, \citenamefont {Horodecki}, \citenamefont {Ng}, \citenamefont
		{Oppenheim},\ and\ \citenamefont {Wehner}}]{Brandao15_2ndlaws}%
	\BibitemOpen
	\bibfield  {author} {\bibinfo {author} {\bibfnamefont {F.}~\bibnamefont
			{Brand{\~a}o}}, \bibinfo {author} {\bibfnamefont {M.}~\bibnamefont
			{Horodecki}}, \bibinfo {author} {\bibfnamefont {N.}~\bibnamefont {Ng}},
		\bibinfo {author} {\bibfnamefont {J.}~\bibnamefont {Oppenheim}},\ and\
		\bibinfo {author} {\bibfnamefont {S.}~\bibnamefont {Wehner}},\ }\bibfield
	{title} {\bibinfo {title} {The second laws of quantum thermodynamics},\
	}\href {https://doi.org/10.1073/pnas.1411728112} {\bibfield  {journal}
		{\bibinfo  {journal} {Proc. Natl. Acad. Sci. U.S.A.}\ }\textbf {\bibinfo
			{volume} {112}},\ \bibinfo {pages} {3275} (\bibinfo {year}
		{2015})}\BibitemShut {NoStop}%
	\bibitem [{\citenamefont {Marvian}(2022)}]{Marvian2022_locality}%
	\BibitemOpen
	\bibfield  {author} {\bibinfo {author} {\bibfnamefont {I.}~\bibnamefont
			{Marvian}},\ }\bibfield  {title} {\bibinfo {title} {Restrictions on
			realizable unitary operations imposed by symmetry and locality},\ }\href
	{https://doi.org/10.1038/s41567-021-01464-0} {\bibfield  {journal} {\bibinfo
			{journal} {Nat. Phys.}\ }\textbf {\bibinfo {volume} {18}},\ \bibinfo {pages}
		{283} (\bibinfo {year} {2022})}\BibitemShut {NoStop}%
	\bibitem [{\citenamefont {Lostaglio}\ \emph
		{et~al.}(2015{\natexlab{a}})\citenamefont {Lostaglio}, \citenamefont
		{Korzekwa}, \citenamefont {Jennings},\ and\ \citenamefont
		{Rudolph}}]{Lostaglio15_PRX_coh}%
	\BibitemOpen
	\bibfield  {author} {\bibinfo {author} {\bibfnamefont {M.}~\bibnamefont
			{Lostaglio}}, \bibinfo {author} {\bibfnamefont {K.}~\bibnamefont {Korzekwa}},
		\bibinfo {author} {\bibfnamefont {D.}~\bibnamefont {Jennings}},\ and\
		\bibinfo {author} {\bibfnamefont {T.}~\bibnamefont {Rudolph}},\ }\bibfield
	{title} {\bibinfo {title} {Quantum coherence, time-translation symmetry, and
			thermodynamics},\ }\href {https://doi.org/10.1103/PhysRevX.5.021001}
	{\bibfield  {journal} {\bibinfo  {journal} {Phys. Rev. X}\ }\textbf {\bibinfo
			{volume} {5}},\ \bibinfo {pages} {021001} (\bibinfo {year}
		{2015}{\natexlab{a}})}\BibitemShut {NoStop}%
	\bibitem [{\citenamefont {Lostaglio}\ \emph
		{et~al.}(2015{\natexlab{b}})\citenamefont {Lostaglio}, \citenamefont
		{Jennings},\ and\ \citenamefont {Rudolph}}]{Lostaglio15_coherence}%
	\BibitemOpen
	\bibfield  {author} {\bibinfo {author} {\bibfnamefont {M.}~\bibnamefont
			{Lostaglio}}, \bibinfo {author} {\bibfnamefont {D.}~\bibnamefont
			{Jennings}},\ and\ \bibinfo {author} {\bibfnamefont {T.}~\bibnamefont
			{Rudolph}},\ }\bibfield  {title} {\bibinfo {title} {Description of quantum
			coherence in thermodynamic processes requires constraints beyond free
			energy},\ }\href {https://doi.org/10.1038/ncomms7383} {\bibfield  {journal}
		{\bibinfo  {journal} {Nat. Commun.}\ }\textbf {\bibinfo {volume} {6}},\
		\bibinfo {pages} {6383} (\bibinfo {year} {2015}{\natexlab{b}})}\BibitemShut
	{NoStop}%
	\bibitem [{\citenamefont {Marvian}\ and\ \citenamefont
		{Spekkens}(2014)}]{Marvian14_coherence}%
	\BibitemOpen
	\bibfield  {author} {\bibinfo {author} {\bibfnamefont {I.}~\bibnamefont
			{Marvian}}\ and\ \bibinfo {author} {\bibfnamefont {R.~W.}\ \bibnamefont
			{Spekkens}},\ }\bibfield  {title} {\bibinfo {title} {Extending {N}oether's
			theorem by quantifying the asymmetry of quantum states},\ }\href
	{https://doi.org/10.1038/ncomms4821} {\bibfield  {journal} {\bibinfo
			{journal} {Nat. Commun.}\ }\textbf {\bibinfo {volume} {5}},\ \bibinfo {pages}
		{3821} (\bibinfo {year} {2014})}\BibitemShut {NoStop}%
	\bibitem [{\citenamefont {\ifmmode \acute{C}\else
			\'{C}\fi{}wikli\ifmmode~\acute{n}\else \'{n}\fi{}ski}\ \emph
		{et~al.}(2015)\citenamefont {\ifmmode \acute{C}\else
			\'{C}\fi{}wikli\ifmmode~\acute{n}\else \'{n}\fi{}ski}, \citenamefont
		{Studzi\ifmmode~\acute{n}\else \'{n}\fi{}ski}, \citenamefont {Horodecki},\
		and\ \citenamefont {Oppenheim}}]{Cwiklinski15_PRL}%
	\BibitemOpen
	\bibfield  {author} {\bibinfo {author} {\bibfnamefont {P.}~\bibnamefont
			{\ifmmode \acute{C}\else \'{C}\fi{}wikli\ifmmode~\acute{n}\else
				\'{n}\fi{}ski}}, \bibinfo {author} {\bibfnamefont {M.}~\bibnamefont
			{Studzi\ifmmode~\acute{n}\else \'{n}\fi{}ski}}, \bibinfo {author}
		{\bibfnamefont {M.}~\bibnamefont {Horodecki}},\ and\ \bibinfo {author}
		{\bibfnamefont {J.}~\bibnamefont {Oppenheim}},\ }\bibfield  {title} {\bibinfo
		{title} {Limitations on the evolution of quantum coherences: Towards fully
			quantum second laws of thermodynamics},\ }\href
	{https://doi.org/10.1103/PhysRevLett.115.210403} {\bibfield  {journal}
		{\bibinfo  {journal} {Phys. Rev. Lett.}\ }\textbf {\bibinfo {volume} {115}},\
		\bibinfo {pages} {210403} (\bibinfo {year} {2015})}\BibitemShut {NoStop}%
	\bibitem [{\citenamefont {Gour}(2022)}]{Gour22_coherence}%
	\BibitemOpen
	\bibfield  {author} {\bibinfo {author} {\bibfnamefont {G.}~\bibnamefont
			{Gour}},\ }\bibfield  {title} {\bibinfo {title} {Role of quantum coherence in
			thermodynamics},\ }\href {https://doi.org/10.1103/PRXQuantum.3.040323}
	{\bibfield  {journal} {\bibinfo  {journal} {PRX Quantum}\ }\textbf {\bibinfo
			{volume} {3}},\ \bibinfo {pages} {040323} (\bibinfo {year}
		{2022})}\BibitemShut {NoStop}%
	\bibitem [{\citenamefont {Czartowski}\ \emph {et~al.}(2023)\citenamefont
		{Czartowski}, \citenamefont {de~Oliveira~Junior},\ and\ \citenamefont
		{Korzekwa}}]{korzekwateam23}%
	\BibitemOpen
	\bibfield  {author} {\bibinfo {author} {\bibfnamefont {J.}~\bibnamefont
			{Czartowski}}, \bibinfo {author} {\bibfnamefont {A.}~\bibnamefont
			{de~Oliveira~Junior}},\ and\ \bibinfo {author} {\bibfnamefont
			{K.}~\bibnamefont {Korzekwa}},\ }\bibfield  {title} {\bibinfo {title}
		{Thermal recall: Memory-assisted {M}arkovian thermal processes},\ }\href@noop
	{} {\  (\bibinfo {year} {2023})},\ \bibinfo {note} {in
		preparation}\BibitemShut {NoStop}%
	\bibitem [{\citenamefont {Chitambar}\ and\ \citenamefont
		{Gour}(2019)}]{Chitambar19_RMP}%
	\BibitemOpen
	\bibfield  {author} {\bibinfo {author} {\bibfnamefont {E.}~\bibnamefont
			{Chitambar}}\ and\ \bibinfo {author} {\bibfnamefont {G.}~\bibnamefont
			{Gour}},\ }\bibfield  {title} {\bibinfo {title} {Quantum resource theories},\
	}\href {https://doi.org/10.1103/RevModPhys.91.025001} {\bibfield  {journal}
		{\bibinfo  {journal} {Rev. Mod. Phys.}\ }\textbf {\bibinfo {volume} {91}},\
		\bibinfo {pages} {025001} (\bibinfo {year} {2019})}\BibitemShut {NoStop}%
	\bibitem [{\citenamefont {Ruch}\ \emph {et~al.}(1978)\citenamefont {Ruch},
		\citenamefont {Schranner},\ and\ \citenamefont {Seligman}}]{Ruch78_mixing}%
	\BibitemOpen
	\bibfield  {author} {\bibinfo {author} {\bibfnamefont {E.}~\bibnamefont
			{Ruch}}, \bibinfo {author} {\bibfnamefont {R.}~\bibnamefont {Schranner}},\
		and\ \bibinfo {author} {\bibfnamefont {T.~H.}\ \bibnamefont {Seligman}},\
	}\bibfield  {title} {\bibinfo {title} {The mixing distance},\ }\href
	{https://doi.org/10.1063/1.436364} {\bibfield  {journal} {\bibinfo  {journal}
			{The Journal of Chemical Physics}\ }\textbf {\bibinfo {volume} {69}},\
		\bibinfo {pages} {386} (\bibinfo {year} {1978})}\BibitemShut {NoStop}%
	\bibitem [{\citenamefont {{vom Ende}}\ and\ \citenamefont
		{Dirr}(2022)}]{vomEnde22_dmaj}%
	\BibitemOpen
	\bibfield  {author} {\bibinfo {author} {\bibfnamefont {F.}~\bibnamefont {{vom
					Ende}}}\ and\ \bibinfo {author} {\bibfnamefont {G.}~\bibnamefont {Dirr}},\
	}\bibfield  {title} {\bibinfo {title} {The d-majorization polytope},\ }\href
	{https://doi.org/https://doi.org/10.1016/j.laa.2022.05.005} {\bibfield
		{journal} {\bibinfo  {journal} {Linear Algebra and its Applications}\
		}\textbf {\bibinfo {volume} {649}},\ \bibinfo {pages} {152} (\bibinfo {year}
		{2022})}\BibitemShut {NoStop}%
	\bibitem [{\citenamefont {Junior}\ \emph {et~al.}(2022)\citenamefont {Junior},
		\citenamefont {Czartowski}, \citenamefont {\ifmmode~\dot{Z}\else
			\.{Z}\fi{}yczkowski},\ and\ \citenamefont {Korzekwa}}]{Junior22_cones}%
	\BibitemOpen
	\bibfield  {author} {\bibinfo {author} {\bibfnamefont {A.~d.~O.}\
			\bibnamefont {Junior}}, \bibinfo {author} {\bibfnamefont {J.}~\bibnamefont
			{Czartowski}}, \bibinfo {author} {\bibfnamefont {K.}~\bibnamefont
			{\ifmmode~\dot{Z}\else \.{Z}\fi{}yczkowski}},\ and\ \bibinfo {author}
		{\bibfnamefont {K.}~\bibnamefont {Korzekwa}},\ }\bibfield  {title} {\bibinfo
		{title} {Geometric structure of thermal cones},\ }\href
	{https://doi.org/10.1103/PhysRevE.106.064109} {\bibfield  {journal} {\bibinfo
			{journal} {Phys. Rev. E}\ }\textbf {\bibinfo {volume} {106}},\ \bibinfo
		{pages} {064109} (\bibinfo {year} {2022})}\BibitemShut {NoStop}%
	\bibitem [{\citenamefont {Spaventa}\ \emph {et~al.}(2022)\citenamefont
		{Spaventa}, \citenamefont {Huelga},\ and\ \citenamefont
		{Plenio}}]{Spaventa22_MTP}%
	\BibitemOpen
	\bibfield  {author} {\bibinfo {author} {\bibfnamefont {G.}~\bibnamefont
			{Spaventa}}, \bibinfo {author} {\bibfnamefont {S.~F.}\ \bibnamefont
			{Huelga}},\ and\ \bibinfo {author} {\bibfnamefont {M.~B.}\ \bibnamefont
			{Plenio}},\ }\bibfield  {title} {\bibinfo {title} {Capacity of
			non-markovianity to boost the efficiency of molecular switches},\ }\href
	{https://doi.org/10.1103/PhysRevA.105.012420} {\bibfield  {journal} {\bibinfo
			{journal} {Phys. Rev. A}\ }\textbf {\bibinfo {volume} {105}},\ \bibinfo
		{pages} {012420} (\bibinfo {year} {2022})}\BibitemShut {NoStop}%
	\bibitem [{\citenamefont {Datta}\ \emph {et~al.}(2022)\citenamefont {Datta},
		\citenamefont {Kondra}, \citenamefont {Miller},\ and\ \citenamefont
		{Streltsov}}]{Datta22_CatReview}%
	\BibitemOpen
	\bibfield  {author} {\bibinfo {author} {\bibfnamefont {C.}~\bibnamefont
			{Datta}}, \bibinfo {author} {\bibfnamefont {T.~V.}\ \bibnamefont {Kondra}},
		\bibinfo {author} {\bibfnamefont {M.}~\bibnamefont {Miller}},\ and\ \bibinfo
		{author} {\bibfnamefont {A.}~\bibnamefont {Streltsov}},\ }\href@noop {}
	{\bibinfo {title} {Catalysis of entanglement and other quantum resources}}
	(\bibinfo {year} {2022}),\ \Eprint {https://arxiv.org/abs/2207.05694}
	{arXiv:2207.05694 [quant-ph]} \BibitemShut {NoStop}%
	\bibitem [{\citenamefont {M{\"u}ller}\ and\ \citenamefont
		{Pastena}(2016)}]{Muller16_correlating}%
	\BibitemOpen
	\bibfield  {author} {\bibinfo {author} {\bibfnamefont {M.~P.}\ \bibnamefont
			{M{\"u}ller}}\ and\ \bibinfo {author} {\bibfnamefont {M.}~\bibnamefont
			{Pastena}},\ }\bibfield  {title} {\bibinfo {title} {A generalization of
			majorization that characterizes shannon entropy},\ }\href
	{https://doi.org/10.1109/TIT.2016.2528285} {\bibfield  {journal} {\bibinfo
			{journal} {IEEE Trans. Inf. Theory}\ }\textbf {\bibinfo {volume} {62}},\
		\bibinfo {pages} {1711} (\bibinfo {year} {2016})}\BibitemShut {NoStop}%
	\bibitem [{\citenamefont {Wilming}\ \emph {et~al.}(2017)\citenamefont
		{Wilming}, \citenamefont {Gallego},\ and\ \citenamefont
		{Eisert}}]{Wilming17_conjecture}%
	\BibitemOpen
	\bibfield  {author} {\bibinfo {author} {\bibfnamefont {H.}~\bibnamefont
			{Wilming}}, \bibinfo {author} {\bibfnamefont {R.}~\bibnamefont {Gallego}},\
		and\ \bibinfo {author} {\bibfnamefont {J.}~\bibnamefont {Eisert}},\
	}\bibfield  {title} {\bibinfo {title} {Axiomatic characterization of the
			quantum relative entropy and free energy},\ }\bibfield  {journal} {\bibinfo
		{journal} {Entropy}\ }\textbf {\bibinfo {volume} {19}},\ \href
	{https://doi.org/10.3390/e19060241} {10.3390/e19060241} (\bibinfo {year}
	{2017})\BibitemShut {NoStop}%
	\bibitem [{\citenamefont {M\"uller}(2018)}]{Muller18_corr}%
	\BibitemOpen
	\bibfield  {author} {\bibinfo {author} {\bibfnamefont {M.~P.}\ \bibnamefont
			{M\"uller}},\ }\bibfield  {title} {\bibinfo {title} {Correlating thermal
			machines and the second law at the nanoscale},\ }\href
	{https://doi.org/10.1103/PhysRevX.8.041051} {\bibfield  {journal} {\bibinfo
			{journal} {Phys. Rev. X}\ }\textbf {\bibinfo {volume} {8}},\ \bibinfo {pages}
		{041051} (\bibinfo {year} {2018})}\BibitemShut {NoStop}%
	\bibitem [{\citenamefont {Boes}\ \emph {et~al.}(2018)\citenamefont {Boes},
		\citenamefont {Wilming}, \citenamefont {Gallego},\ and\ \citenamefont
		{Eisert}}]{Boes18_PRX_rand}%
	\BibitemOpen
	\bibfield  {author} {\bibinfo {author} {\bibfnamefont {P.}~\bibnamefont
			{Boes}}, \bibinfo {author} {\bibfnamefont {H.}~\bibnamefont {Wilming}},
		\bibinfo {author} {\bibfnamefont {R.}~\bibnamefont {Gallego}},\ and\ \bibinfo
		{author} {\bibfnamefont {J.}~\bibnamefont {Eisert}},\ }\bibfield  {title}
	{\bibinfo {title} {Catalytic quantum randomness},\ }\href
	{https://doi.org/10.1103/PhysRevX.8.041016} {\bibfield  {journal} {\bibinfo
			{journal} {Phys. Rev. X}\ }\textbf {\bibinfo {volume} {8}},\ \bibinfo {pages}
		{041016} (\bibinfo {year} {2018})}\BibitemShut {NoStop}%
	\bibitem [{\citenamefont {Takagi}\ and\ \citenamefont
		{Shiraishi}(2022)}]{Takagi22_catalyst}%
	\BibitemOpen
	\bibfield  {author} {\bibinfo {author} {\bibfnamefont {R.}~\bibnamefont
			{Takagi}}\ and\ \bibinfo {author} {\bibfnamefont {N.}~\bibnamefont
			{Shiraishi}},\ }\bibfield  {title} {\bibinfo {title} {Correlation in
			catalysts enables arbitrary manipulation of quantum coherence},\ }\href
	{https://doi.org/10.1103/PhysRevLett.128.240501} {\bibfield  {journal}
		{\bibinfo  {journal} {Phys. Rev. Lett.}\ }\textbf {\bibinfo {volume} {128}},\
		\bibinfo {pages} {240501} (\bibinfo {year} {2022})}\BibitemShut {NoStop}%
	\bibitem [{\citenamefont {Lie}\ and\ \citenamefont
		{Ng}(2023)}]{Lie23_correlation}%
	\BibitemOpen
	\bibfield  {author} {\bibinfo {author} {\bibfnamefont {S.~H.}\ \bibnamefont
			{Lie}}\ and\ \bibinfo {author} {\bibfnamefont {N.~H.~Y.}\ \bibnamefont
			{Ng}},\ }\href@noop {} {\bibinfo {title} {Catalysis always degrades external
			quantum correlations}} (\bibinfo {year} {2023}),\ \Eprint
	{https://arxiv.org/abs/2303.02376} {arXiv:2303.02376 [quant-ph]} \BibitemShut
	{NoStop}%
	\bibitem [{\citenamefont {Jonathan}\ and\ \citenamefont
		{Plenio}(1999)}]{Jonathan99_PRL}%
	\BibitemOpen
	\bibfield  {author} {\bibinfo {author} {\bibfnamefont {D.}~\bibnamefont
			{Jonathan}}\ and\ \bibinfo {author} {\bibfnamefont {M.~B.}\ \bibnamefont
			{Plenio}},\ }\bibfield  {title} {\bibinfo {title} {Entanglement-assisted
			local manipulation of pure quantum states},\ }\href
	{https://doi.org/10.1103/PhysRevLett.83.3566} {\bibfield  {journal} {\bibinfo
			{journal} {Phys. Rev. Lett.}\ }\textbf {\bibinfo {volume} {83}},\ \bibinfo
		{pages} {3566} (\bibinfo {year} {1999})}\BibitemShut {NoStop}%
	\bibitem [{\citenamefont {Daftuar}\ and\ \citenamefont
		{Klimesh}(2001)}]{Daftuar01_trumping}%
	\BibitemOpen
	\bibfield  {author} {\bibinfo {author} {\bibfnamefont {S.}~\bibnamefont
			{Daftuar}}\ and\ \bibinfo {author} {\bibfnamefont {M.}~\bibnamefont
			{Klimesh}},\ }\bibfield  {title} {\bibinfo {title} {Mathematical structure of
			entanglement catalysis},\ }\href {https://doi.org/10.1103/PhysRevA.64.042314}
	{\bibfield  {journal} {\bibinfo  {journal} {Phys. Rev. A}\ }\textbf {\bibinfo
			{volume} {64}},\ \bibinfo {pages} {042314} (\bibinfo {year}
		{2001})}\BibitemShut {NoStop}%
	\bibitem [{\citenamefont {Anspach}(2001)}]{Anspach01_condition}%
	\BibitemOpen
	\bibfield  {author} {\bibinfo {author} {\bibfnamefont {P.~H.}\ \bibnamefont
			{Anspach}},\ }\bibfield  {title} {\bibinfo {title} {Two-qubit catalysis in a
			four-state pure bipartite system},\ }\href@noop {} {\bibfield  {journal}
		{\bibinfo  {journal} {arXiv preprint quant-ph/0102067}\ } (\bibinfo {year}
		{2001})}\BibitemShut {NoStop}%
	\bibitem [{\citenamefont {Klimesh}(2007)}]{Klimesh07_ineqs}%
	\BibitemOpen
	\bibfield  {author} {\bibinfo {author} {\bibfnamefont {M.}~\bibnamefont
			{Klimesh}},\ }\href@noop {} {\bibinfo {title} {Inequalities that collectively
			completely characterize the catalytic majorization relation}} (\bibinfo
	{year} {2007}),\ \Eprint {https://arxiv.org/abs/0709.3680} {arXiv:0709.3680
		[quant-ph]} \BibitemShut {NoStop}%
	\bibitem [{\citenamefont {Bu}\ \emph {et~al.}(2016)\citenamefont {Bu},
		\citenamefont {Singh},\ and\ \citenamefont {Wu}}]{Bu16_coherence}%
	\BibitemOpen
	\bibfield  {author} {\bibinfo {author} {\bibfnamefont {K.}~\bibnamefont
			{Bu}}, \bibinfo {author} {\bibfnamefont {U.}~\bibnamefont {Singh}},\ and\
		\bibinfo {author} {\bibfnamefont {J.}~\bibnamefont {Wu}},\ }\bibfield
	{title} {\bibinfo {title} {Catalytic coherence transformations},\ }\href
	{https://doi.org/10.1103/PhysRevA.93.042326} {\bibfield  {journal} {\bibinfo
			{journal} {Phys. Rev. A}\ }\textbf {\bibinfo {volume} {93}},\ \bibinfo
		{pages} {042326} (\bibinfo {year} {2016})}\BibitemShut {NoStop}%
\end{thebibliography}%

\clearpage
\newpage



\title{\suppl~for \\ ``Catalytic collapse of a hierarchy of Markovian thermal processes''}

\maketitle

%\onecolumngrid

\section{Thermodynamic resource theories}

In this section, we provide as context a concise background to the various thermal processes. In particular, for the unfamiliar reader, we briefly introduce the resource-theoretic approach to quantum thermodynamics, and two special classes of thermal processes: thermal operations (TOs) which encompass a very generic heat exchange process, and on the other hand Markovian thermal processes (MTPs) representing a much smaller, physically motivated set of processes.

When any operation is allowed, state transition conditions are trivial --- all states are interconvertible and thus no state is more valuable than other.  
In the presence of restrictions on operations, more interesting structures, e.g. pre-ordering relations, emerge.
Resource theories~\cite{Chitambar19_RMP} introduce various constraints, which may arise from practical limitations or fundamental interests. Accordingly, states with certain properties become more useful than states without those properties in terms of their ability to be transformed to other states. Hence, the notion of resourcefulness in each resource theory emerges.

A particularly fruitful application of the resource theoretic framework is to impose thermodynamic restrictions~\cite{Brandao13_TRT, Gour15_TRTreview, Ng18_Qthermo_book, Lostaglio19_review}. A typical setting is to assume that an (infinitely large) environment is fully thermal with some fixed temperature $ 1/\beta $, and one can do any strict energy-preserving operations between the system and some parts of the bath. Then we can rethermalize the used baths for free, with the remainder of the environment.

More precisely, these assumptions define a set of channels called thermal operations (TOs)~\cite{Janzing00_TO, Horodecki13_fundamental}.
\begin{defn} (TO)\label{def:TO} 
	A channel $\Phi:\mS(\mH_{S}) \rightarrow\mS(\mH_{S}) $ 
	is a thermal operation if it admits the dilation 
	\begin{equation}\label{key}
		\Phi(\rho) = \Tr_{R}\left[U\left(\rho\otimes\tau^{\beta}(H_{R})\right)U^{\dagger}\right],
	\end{equation}
	for some bath Hamiltonian $ H_{R} $ and energy-preserving unitary $ U\in\mL(\mH_{S}\otimes\mH_{R}) $ such that $ [U,H_{S}+H_{R}]=0 $. Here, $ \tau^{\beta}(H_{R}) = e^{-\beta H_{R}}/Z_{R} $ is a Gibbs state with respect to Hamiltonian $ H_{R} $ and inverse temperature $ \beta $.  
\end{defn}
We can also define thermal operations to map states in $S$ to $ S'$ by tracing out some $R'$ instead of $R$. Yet, we restrict ourselves to the cases where the mapping is an endomorphism.
Although there is a large degree of freedom for choosing the bath Hamiltonian and unitary operations, a single quantifier, like the free energy of macroscopic thermodyanmics, cannot capture all the resources in the framework. Instead, there is a more stringent set of conditions for state transformations. 
For initial state $ \rho $ that is energy-incoherent, i.e., it is block-diagonal in its energy eigenbasis, the set of reachable states $ \setTO(\rho) $ via TOs is well characterized by the simple criterion of thermomajorization relation~\cite{Ruch78_mixing, Horodecki13_fundamental, vomEnde22_dmaj, Junior22_cones}, which compares thermomajorization curves of the initial and the target state populations.
\begin{defn}\label{def:thermo_curve} (Thermomajorization curve)
	Given the Hamiltonian $ H_{S} $ and inverse temperature $ \beta $, the Gibbs state population vector reads $ \taustate  = (\tau_{1},\cdots,\tau_{d}) $ with $ \tau_{i} = \bra{i}_{S}\tau^{\beta}(H_{S})\ket{i}_{S} $.
	Similarly, the system population can be written as $ \pstate = (p_{1},\cdots,p_{d}) $ with $ p_{i} = \bra{i}_{S}\rho\ket{i}_{S} $. Then the thermomajorization curve $ \Gamma[\pstate] $ of a probability vector $ \pstate $ is defined as a piecewise-linear function from $ [0,1] $ to $ [0,1] $, interpolating $ (0,0) $ and elbow points given by $ (\sum_{i=1}^{k}\tau_{\pi_{i}},\sum_{i=1}^{k}p_{\pi_{i}}) $ for $ k = 1,\cdots,d $, where permutation $ \{\pi_{i}\} $ is defined so that $ p_{\pi_{i}}/\tau_{\pi_{i}}\geq p_{\pi_{j}}/\tau_{\pi_{j}} $ whenever $ j>i $. The permutation $ \{\pi_{i}\} $ is the optimal rearrangement of levels $ \{i\} $ to maximize the thermomajorization curve, i.e., for any other permutation $ \{\pi^{\prime}_{i}\} $, the thermomajorization curve satisfies $ \Gamma[\pstate](\sum_{i=1}^{k}\tau_{\pi^{\prime}_{i}}) \geq \sum_{i=1}^{k}p_{\pi^{\prime}_{i}} $, for all $ k $.
\end{defn}
When $ \Gamma[\pstate](x)\geq\Gamma[\qstate](x) $, for all $ x\in[0,1] $, we say $ \pstate $ thermomajorizes $ \qstate $ and denote $ \pstate\succ_{\beta}\qstate $. Then $ \rho $ having the population vector $ \pstate $ can be converted into the target state $ \sigma = {\rm diag}(\qstate) $, which is energy-incoherent, by some TO.
However, for general $ \rho $ that might be coherent, finding the necessary and sufficient condition for state transitions by TOs is a long-standing open problem~\cite{Lostaglio15_coherence,Cwiklinski15_PRL,Lostaglio15_PRX_coh,Gour22_coherence}. 

Since TOs limit neither the size of baths nor the controllability of the system plus bath composite state, many of the unitaries are in principle extremely demanding. Recently, variants of free operations obtained by further restricting certain features of TOs have been proposed to ameliorate the experimental challenges. 
Elementary thermal operations~\cite{Lostaglio_18_ETO} (Def.~\ref{def:ETO_def} in the main text) constraint the number of system levels we can control at a time to be at most two. Another notable class is the set of Markovian thermal processes (MTP)~\cite{Lostaglio22_MTP1, Korzekwa22_MTP2, Spaventa22_MTP}, special cases of TOs that can be implemented using a fully Markovian bath. Suppose that an MTP is performed by a (time-dependent) energy-preserving interaction between the system and bath during time $ [0,t] $. At any time $ 0\leq t_{1}<t $, the bath is still in equilibrium from the Markovianity. Then the interaction during $ [t_{1},t_{2}] $ for some $ t_{1}<t_{2}\leq t $ should also be an MTP since the interaction is energy-preserving and the initial state starts from a system plus a thermal bath. We define MTP more rigorously following the definition of~\cite{Spaventa22_MTP}.
\begin{defn}\label{def:MTP} (MTP)
	A channel $ \Phi_{(0,t)} : \mS(\mH_{S}) \rightarrow \mS(\mH_{S}) $ is a Markovian thermal process if
	\begin{enumerate}
		\item $ \Phi_{(0,t)} $ is a TO with the form
			\begin{equation}\label{key}
				\Phi_{(0,t)} = \mT\exp\left[\int_{0}^{t}L[s]ds\right],
			\end{equation}
			for some generator $ L[s] $ and time-ordering $ \mT $ and
		\item for any $ 0\leq t_{1}<t_{2}\leq t $, the channel
			\begin{equation}\label{key}
				\Phi_{(t_{1},t_{2})} = \mT\exp\left[\int_{t_{1}}^{t_{2}}L[s]ds\right],
			\end{equation}
			is also a TO.
	\end{enumerate}
\end{defn}
In~\cite{Lostaglio22_MTP1}, for any incoherent initial state $ \rho $, the reachable states $ \setMTP(\rho) $ is fully achievable by concatenations of two-system-level MTPs, which then is a subset of $ \setETO(\rho) $. 




So far, we defined three major choices for free operations: TOs, ETOs, and MTPs. An interesting way of extending the set of feasible state transitions with a fixed set of free operations is to introduce the concept of catalysis~\cite{Datta22_CatReview}. Assume that one has an access to an auxiliary state $ \mu_{C}\in\mS(\mH_{C}) $. Instead of applying free operations only to the system of interest $ \rho $, one can start from the larger state $ \rho\otimes\mu_{C} $. If after the operation, the catalyst $ \mu_{C} $ can be retrieved, then we say that the process is catalytic. There are different notions of recovering the catalyst~\cite{Muller16_correlating, Wilming17_conjecture, Muller18_corr, Boes18_PRX_rand, Takagi22_catalyst, Lie23_correlation}.
We choose the most conservative category of exact catalysis.  
\begin{defn}\label{def:exact_cat} (Exact catalysis) Given a set of free operations $X$, we say that a state transition $\rho\rightarrow\sigma$ via catalytic-$X$ is possible if 
		\begin{equation}\label{eq:exact_catalyst_recovery}
			\Phi^{(X)}(\rho\otimes\mu_C) = \sigma\otimes\mu_C,
		\end{equation}
		where $\mu_C$ is called a catalyst state, and $\Phi^{(X)} \in X$. We also denote the set of catalytically-reachable states starting from $\rho$ as $\mR_{\rm CX}(\rho)$ using some catalyst $ \mu_{C} $, which always includes $ \mR_{\rm X}$ as a subset.
\end{defn}
Eq.~\eqref{eq:exact_catalyst_recovery} is a very strict condition, yet in many resource theories~\cite{Jonathan99_PRL, Daftuar01_trumping, Anspach01_condition, Klimesh07_ineqs, Brandao15_2ndlaws, Bu16_coherence, Marvian2022_locality, Son2022_CETO, Korzekwa22_MTP2} it has been reported to be useful for enlarging the set of reachable state $ \mR_{\rm X} $.
For TOs, when allowed to bring in any catalyst, inequalities including the entire family of Reyni divergences of the state with respect to the Gibbs state determine $ \setCTO(\rho) $ for incoherent $ \rho $~\cite{Brandao15_2ndlaws}. For ETOs, however, due to the difficulty of characterizing the reachable state set, only the simplest non-trivial cases of qutrit system and qubit catalyst has been studied~\cite{Son2022_CETO}. In the main text, we prove that given the freedom of choosing any catalyst, the same state transition criteria for CTOs can also be used for catalytic ETOs, since $ \setCTO(\rho) =\setCETO(\rho) $ for incoherent state $ \rho $. 

In Lemma \ref{lem:nm_vs_gc}, we showed the relationship between non-Markovian ETOs and Gibbs-catalytic ETOs. The same argument can be applied to show that $ \setNMETO(\rho) \subset \setGCMTP(\rho)$: suppose that $\sigma$ is achievable from $\rho$ via an nM-ETO sequence, using the bath $R$. Then by using $R$ as the relevant catalyst $C$, this corresponds to an MTP as a two-stage process: stage 1 where no other bath is used, and since the nM-ETO unitary is $SC$ energy-preserving, we have that this is a valid MTP. In stage 2, $C$ is rethermalized while destroying correlations from stage 1, noting that the fully thermalizing channel is implementable by MTPs. This also then implies that 
\begin{equation}
{\rm Conv}(\setNMETO(\rho)) \subset {\rm Conv}(\setGCMTP(\rho)),
\end{equation}
which then coupled with the result from~\cite{Lostaglio22_MTP1} that for \emph{energy-incoherent} $\rho$, 
\begin{equation}
	{\rm Conv}(\setNMETO(\rho)) = {\rm Conv}(\setGCMTP(\rho)).
\end{equation}



\section{The effect of controlled unitaries in (non-Markovian) elementary thermal operations}

Conventionally, ETOs are defined in a way that at most two system levels can vary at each step of the operation. Technically, unitaries in the form $ V_{kk} = (\1_{S}-\dm{k}_{S})\otimes\1_{R} + \dm{k}_{S}\otimes V_{R} $, which are controlled unitaries that become nontrivial only for system states $ \ket{k}_{S} $, are also allowed by the aforementioned definition. We justify this choice by showing that the effect of such controlled unitaries is merely to reduce the amplitude of off-diagonal terms of the given state.

At any step of the ETO process, we start from a system density matrix $ \rho = \sum_{j,l}\rho_{jl}\ketbra{j}{l}_{S} $ and a bath $ \tau^{\beta}(H_{R}) $ starting from the thermal state. 
Observe that 
\begin{align}\label{key}
	\Tr_{R}\left[V_{kk}\left(\ketbra{j}{l}_{S}\otimes\tau^{\beta}(H_{R})\right) V_{kk}^{\dagger}\right] &= \ketbra{j}{l}_{S}, \\
	\Tr_{R}\left[V_{kk}\left(\dm{k}_{S}\otimes\tau^{\beta}(H_{R})\right) V_{kk}^{\dagger}\right] &= \dm{k}_{S},
\end{align}
for all $ j,l\neq k $.
Then all diagonal elements and off-diagonal elements that do not include the level $ k $ are invariant under this operation.
For levels $ \ketbra{k}{l}_{S} $ with $ l\neq k $,
\begin{equation}
	\Tr_{R}\left[V_{kk}\left(\ketbra{k}{l}_{S}\otimes\tau^{\beta}(H_{R})\right) V_{kk}^{\dagger}\right] = \ketbra{k}{l}_{S}\Tr[V_{R}\tau^{\beta}(H_{R})].\label{eq:ETO_controlled_kl}
\end{equation}


To make $ V_{kk} $ energy-preserving, we require $ [V_{kk},H_{S}+H_{R}] = 0 $, which is equivalent to $ [V_{R},H_{R}] = 0 $.
Then $ V_{R} $ admits the block-diagonal decomposition
\begin{equation}\label{eq:control_U_block_diagonal}
	V_{R} = \bigoplus_{E_{R}}V_{E_{R}};\quad V_{E_{R}}\in\mL(\mH_{E_{R}}),
\end{equation}
where $ \mH_{E_{R}} $ is a subspace of $ \mH_{R} $ with energy $ E_{R} $. Recall that the thermal state $ \tau^{\beta}(H_{R}) $ can be written similarly as 
\begin{equation}\label{eq:Gibbs_state_block_diag}
	\tau^{\beta}(H_{R}) = \bigoplus_{E_{R}} \Pi_{E_{R}} \frac{e^{-\beta E_{R}}}{Z_{R}},
\end{equation}
where $ \Pi_{E_{R}} $ is the projector onto the space $ \mH_{R} $.
Then the last term of Eq.~\eqref{eq:ETO_controlled_kl} becomes
\begin{align}\label{eq:eto_coherence_reduction}
	\left\vert\Tr[V_{R}\tau^{\beta}(H_{R})]\right\vert &= \left\vert\sum_{E_{R}}\frac{\Tr[V_{R}\Pi_{E_{R}}]e^{-\beta E_{R}}}{Z_{R}}\right\vert \\&= \left\vert\sum_{E_{R}}\frac{\Tr[V_{E_{R}}]e^{-\beta E_{R}}}{Z_{R}}\right\vert\leq 1,\nonumber
\end{align}
where the equality holds only when $ V_{R} \propto\1_{R} $. As a result, the initial state $ \rho\rightarrow\rho^{\prime} $, where the only difference being
\begin{equation}\label{key}
	\rho^{\prime}_{kl} =  \sum_{E_{R}}\frac{\Tr[V_{E_{R}}]e^{-\beta E_{R}}}{Z_{R}} \rho_{kl},\quad \rho^{\prime}_{lk} = \left(\rho^{\prime}_{kl}\right)^{\dagger},
\end{equation}
for all $ l\neq k $. 

Since ETO is a subset of TO, population dynamics is decoupled from the coherence dynamics~\cite{Marvian14_coherence, Lostaglio15_coherence, Lostaglio15_PRX_coh, Cwiklinski15_PRL}. Hence, for populations of the final state, controlled unitaires are irrelevant. For off-diagonal terms, relative phases are usually not important and controlled unitaries reduce their amplitudes. 
Such dephasing is known to be difficult when the bath is restricted to be too small~\cite{Hu2019_singlemode}, but are valid thermal operations. 

Now we assess the effect of controlled unitaries $ U_{kk} = (\1_{S}-\dm{k}_{S})\otimes \1_{R} + \dm{k}_{S}\otimes U_{R} $ to nM-ETOs. First, we state a remark that helps us rearrange the controlled unitaries to either the beginning or the end of the series. 

\begin{remark}
	For any two-system-level unitary $ U_{jl} = u_{jl}\oplus\1_{\setminus(j,l)} $ that is energy-preserving, i.e., $ [U_{jl},H_{S}+H_{R}] = 0 $,
	\begin{equation}\label{key}
		U_{kk} U_{jl}U_{kk}^{\dagger} = U_{jl}^{\prime},
	\end{equation}
	where $ U_{jl}^{\prime} $ is again a nM-ETO unitary, that is, $ [U_{jl}^{\prime},H_{S}+H_{R}] = 0 $ and $ U_{jl}^{\prime} = u_{jl}^{\prime}\oplus\1_{\setminus(j,l)} $.
\end{remark}
\begin{proof}
	Two-system-level energy-preserving unitary $ U_{jl} $ can be written as 
	\begin{equation}\label{key}
		U_{jl} = e^{-it(H_{jl}+H_{lj})},
	\end{equation}
	with $H_{jl} = \ketbra{j}{l}_{S}\otimes B_{jl}$, $ H_{lj} = H_{jl}^{\dagger} $, and $ [H_{jl},H_{S}+H_{R}] = 0 $ as in the proof of Lemma~\ref{lem:NMETO_TO} of the main text. Conversely, any unitary generated by such Hamiltonian $ H_{jl}+H_{lj} $ is a two-system-level and energy-preserving unitary. 
	
	From $ U_{kk} = (\1_{S}-\dm{k}_{S})\otimes \1_{R} + \dm{k}_{S}\otimes U_{R} $, we immediately obtain $ U_{kk}H_{jl} = H_{jl} $ whenever $ j\neq k $ and $ H_{jl}U_{kk}^{\dagger} = H_{jl} $ whenever $ l\neq k $.
	When $ j,l\neq k $, the subspace $ {\rm span}\{\ket{j}_{S},\ket{l}_{S}\}\otimes\mH_{R} $ remains intact after $ U_{kk} $ and thus $ U_{kk} U_{jl}U_{kk}^{\dagger} = U_{jl} $, which obviously complies with the remark statement. 
	
	If $ j\neq k $ but $ l=k $, 
	\begin{align}\label{key}
		U_{kk} U_{jk}U_{kk}^{\dagger} &= e^{-it (U_{kk}H_{jk}U_{kk}^{\dagger} + U_{kk}H_{kj}U_{kk}^{\dagger})}\\ 
		&= e^{-it (H_{jk}U_{kk}^{\dagger} + U_{kk}H_{kj}))} = U_{jk}^{\prime}.\nonumber
	\end{align}
	$ U_{jk}^{\prime} $ is a two-system-level unitary since $ U_{kk}H_{kj} = \ketbra{k}{j}_{S}\otimes U_{R}B_{kj} $ and $ H_{jk}U_{kk}^{\dagger}  = (U_{kk}H_{kj})^{\dagger} $.
	Also, $ [U_{kk}H_{kj},H_{S}+H_{R}] = 0 $ since $ [H_{kj},H_{S}+H_{R}] = 0 $ and $ [U_{kk},H_{S}+H_{R}]=0 $, rendering $ U_{jk}^{\prime} $ to be energy-preserving.
	
	Finally, when the two-system-level unitary is $ \tilde{U}_{kk} = (\1_{S}-\dm{k}_{S})\otimes \1_{R} + \dm{k}_{S}\otimes \tilde{U}_{R} $,
	\begin{equation}\label{key}
		U_{kk}\tilde{U}_{kk}U_{kk}^{\dagger} = (\1_{S}-\dm{k}_{S})\otimes \1_{R} + \dm{k}_{S}\otimes U_{R}^{\prime} = U_{kk}^{\prime} ,
	\end{equation}
	with $ [U_{kk}^{\prime},H_{S}+H_{R}] = 0 $ from  $ [U_{kk},H_{S}+H_{R}] = 0 $ and  $ [\tilde{U}_{kk},H_{S}+H_{R}] = 0 $. In other words, $ U_{kk}^{\prime} $ also follows the statement of the remark. 
\end{proof}

In an nM-ETO sequence (Eq.~\eqref{eq:NMETO_def} in the main text), a series of two-system-level unitaries $ U = \prod_{i}U_{j_{i}l_{i}} $ is applied. By repeating the transformation $ U_{kk}U_{jl} = U_{jl}^{\prime}U_{kk} $ to rearrange all controlled unitaries $ U_{kk} $, we obtain an equivalent series of two-system-level unitaries, where all controlled unitaries are performed before all the genuinely two-system-level operations $ U^{\prime}_{jl} $ with $ j\neq l $ are performed, i.e.,
\begin{equation}\label{key}
	U = \prod_{i; j_{i}\neq l_{i}}U^{\prime}_{j_{i}l_{i}}\prod_{h}U_{k_{h}k_{h}}.
\end{equation}

We will now show that the series of controlled unitaries $ \prod_{h}U_{k_{h}k_{h}} $ does not affect the initial state at all when it is incoherent in the energy eigenbasis. 
For simplicity, let us consider the case where a single unitary $ U_{kk} $, is applied to the initial state $ \rho\otimes\tau^{\beta}(H_{R}) $, where $ \rho $ is incoherent. Note that $ U_{kk} $ also has a block-diagonal form of Eq.~\eqref{eq:control_U_block_diagonal} from the energy-preserving condition. 
For $ i\neq k $, 
\begin{equation}\label{key}
	U_{kk}\left(\dm{i}_{S}\otimes \tau^{\beta}(H_{R})\right)U_{kk}^{\dagger} = \dm{i}_{S}\otimes\tau^{\beta}(H_{R}),
\end{equation}
without any change. 
For $ i =k $,
\begin{equation}\label{key}
	U_{kk}\left(\dm{k}_{S}\otimes\tau^{\beta}(H_{R})\right)U_{kk}^{\dagger} = \dm{k}_{S}\otimes U_{R}\tau^{\beta}(H_{R})U_{R}^{\dagger}.
\end{equation}
Recalling Eqs.~\eqref{eq:control_U_block_diagonal} and~\eqref{eq:Gibbs_state_block_diag}, 
\begin{equation}\label{key}
	U_{R}\tau^{\beta}(H_{R})U_{R}^{\dagger} = \bigoplus_{E_{R}}U_{E_{R}}\Pi_{E_{R}} U_{E_{R}}^{\dagger} \frac{e^{-\beta E_{R}}}{Z_{R}} = \tau^{\beta}(H_{R}).
\end{equation}
Combining everything, we arrive at the conclusion 
\begin{equation}\label{key}
	U_{kk}\left(\rho\otimes\tau^{\beta}(H_{R})\right)U_{kk}^{\dagger} = \rho\otimes\tau^{\beta}(H_{R}),
\end{equation}
for any $ k $. Hence, the entire series $ \prod_{h}U_{k_{h}k_{h}} $ can be removed when $ \rho $ is incoherent and only genuinely two-system-level unitaries $ \prod_{i; j_{i}\neq l_{i}}U^{\prime}_{j_{i}l_{i}} $ remain. In terms of implementation cost, controlled unitaries would be as hard as genuinely two-system-level unitaries, and including those into the framework would not undermine the motivation of studying ETOs.

When we interpret nM-ETO as GC-ETO, unitaries of the form $ U_{kk} $ can be further decomposed into two-system-plus-catalyst-level unitaries. 
Suppose that the unitary $ U_{kk} = e^{-itH_{kk}} $, where the generator 
\begin{equation}\label{key}
	H_{kk} = \dm{k}_{S}\otimes \sum_{j,l}\lambda_{jl}\ketbra{j}{l}_{R} \equiv \sum_{j,l;j\neq l} h_{jl}^{(k)}+\sum_{m}h_{mm}^{(k)},
\end{equation}
comprises genuinely two-system-plus-catalyst-level operators $ h_{jl}^{(k)} $ and the rank one operator $ h_{mm}^{(k)} = \lambda_{mm}\dm{k,m}_{SR} $.
By performing another Troterrization, we obtain the decomposition
\begin{equation}\label{key}
	U_{kk} = \lim_{M\rightarrow\infty}\left[\prod_{j,l;j<l}e^{-i\frac{t}{M}(h_{jl}^{(k)}+h_{lj}^{(k)})}\prod_{m}e^{-i\frac{t}{M}h_{mm}^{(k)}}\right]^{M}.
\end{equation}
Terms of the form $ e^{-i\frac{t}{M}(h_{jl}^{(k)}+h_{lj}^{(k)})} $ are genuinely two-system-plus-catalyst-level unitaries. On the other hand, $ e^{-i\frac{t}{M}h_{mm}^{(j)}} $ are partial reflections around the energy eigenstate $ \ket{k,m}_{SR} $ and again do not affect the result of operations starting from energy incoherent initial states. 

\section{Proof of Theorem \ref{cor:CETO=CTO}}
In order to prove this theorem, we first present a few technical tools. The first is an observation regarding the structure of thermo-majorization curves, which characterizes the interior of the set $\setTO(\varrho) $. 
\begin{remark}\label{rmk:interior_thermomaj}
	Consider energy-incoherent states $\varrho\in\mS(\mH_{X})$ with Hamiltonian $ H_{X}\in\mL(\mH_{X}) $ and $\varrho^{\prime}\in\mR_{\rm TO}(\varrho) $. If the thermomajorization curve $ \Gamma[\varrho^{\prime}] $ does not overlap with $ \Gamma[\varrho] $ except at trivial points $ (0,0) $ and $ (1,1) $, then $ \varrho^{\prime} $ is in the interior of $ \setTO(\varrho) $. 
\end{remark}
\begin{proof}
	To prove this, we only need to show that whenever the conditions of the remark are fulfilled, there exists another state $ \varphi\in\setTO(\varrho) $ such that
	\begin{equation}\label{eq:phi_extrap}
		\varphi = (1+\lambda)\varrho^{\prime} - \lambda\tau^{\beta}(H_{X}) ,
	\end{equation} 
	for $ 0<\lambda<1 $. Then we can write $\varrho^{\prime} $ as a mixture of $\varphi$ and $\varrho$,
	\begin{equation}
		\varrho^{\prime} = (1-\lambda^{\prime})\varphi + \lambda^{\prime}\tau^{\beta}(H_{X}),
	\end{equation} 
with $ 0<\lambda^{\prime} = \lambda/(1+\lambda)<1 $. Since $ \varrho^{\prime} $ is obtained by a convex combination of a state in $ \setTO(\varrho) $ and an interior point of the same set, $ \tau^{\beta}(H_{X}) $, it is in the interior of $ \setTO(\varrho) $.

	To do so, suppose that $ \Gamma[\varrho^{\prime}] $ and $ \Gamma[\varrho] $ indeed do not meet except at trivial points. Denote the $ x,y $-coordinates of $ i $'th elbow points for $ \Gamma[\varrho^{\prime}] $ as $ (x_{i},y_{i}) $ and the difference $ \Gamma[\varrho](x_{i}) - y_{i} = d_{i}>0 $. Furthermore, populations $ \bra{i}\varrho^{\prime}\ket{i} = q_{i} $ are necessarily stricytly positive to avoid any overlaps before $ (1,1) $ point.
	
	First, according to Eq.~\eqref{eq:phi_extrap}, we have that $ \Tr[\varphi] = 1 $ regardless of $ \lambda $. If $ (1+\lambda)q_{i} - \lambda \tau^{\beta}(H_{X})_{i}\geq0 $, then the state $ \varphi $ is a proper density matrix. For populations $ q_{i}\geq \tau^{\beta}(H_{X})_{i} $, this is always true, hence they place no further constraints on $\lambda$. Otherwise, we require
		\begin{equation}
			\lambda\leq \frac{q_{i}}{\tau^{\beta}(H_{X})_{i}-q_{i}},
	\end{equation}
	for all populations $ q_{i}<\tau^{\beta}(H_{X})_{i} $. From $ q_{i}>0 $, the RHS is also strictly positive. 
	When these conditions are satisfied, the $ \beta $-order of $ \varphi $ is the same as $ \varrho^{\prime} $: the slopes $ g_{i} = q_{i}/\tau^{\beta}(H_{X})_{i} $ become $ (1+\lambda)g_{i} - \lambda $, preserving the order between $ g_{i} $. Then, the elbow points of $ \Gamma[\varphi] $ are given as 
	\begin{equation}
(x_{i},(1+\lambda)y_{i}-\lambda x_{i}),
	\end{equation}
where the $y$-coordinates were evaluated by noting that $\Gamma[\tau^{\beta}(H_{X})](x_i) = x_i$, i.e. the thermo-majorization curve of the thermal state is defined by the straight line connecting $(0,0)$ and $(1,1)$.

Finally, we observe that the gap between $\Gamma[\varrho]$ and $\Gamma[\varphi]$ are reduced to $ d_{i} - \lambda (y_{i}-x_{i}) $, where $ y_{i}\geq x_{i} $ due to the convexity of the curve.
Hence, as long as $ x_{i} = y_{i} $ or $ \lambda \leq d_{i}/(y_{i}-x_{i}) $, the transformed state $ \varphi\in\setTO(\varrho) $. We can always find $ \lambda>0 $ satisfying all those inequalities by taking the minimum of all given upper bounds.
\end{proof}
Next we establish a technical lemma that reveals a property about Gibbs states, namely, without loss of generality, the Gibbs state can always be coupled with a suitable catalyst $\mu_{C'}'$, such that the composite state is in the interior of $\mR_{\rm TO}(\rho\otimes\mu_{C'}')$ for any incoherent $ \rho $.

\begin{lem}\label{lem:catalyst_interior}

	Suppose that $ \rho\neq\tau^\beta(H_S) $ and $ \mu_{C} $ are incoherent in their respective energy eigenbasis. 
	If 
	\begin{eqnarray}\label{eq:notin_interior}
		\tau^{\beta}(H_{S})\otimes\mu_{C} \notin {\rm int}(\setTO(\rho\otimes\mu_{C})) ,
	\end{eqnarray}
there exists a smaller catalyst $ \mu^{\prime}_{C^{\prime}}\in\mH_{C^{\prime}} $, such that 
	\begin{enumerate}
		\item with this catalyst $C'$, $\tau^\beta(H_S)$ is brought into the interior, \begin{eqnarray}
			\tau^{\beta}(H_{S})\otimes\mu^{\prime}_{C^{\prime}}\in{\rm int}(\setTO(\rho\otimes\mu^{\prime}_{C^{\prime}})),
		\end{eqnarray}
		\item for any state $\sigma$ such that $ \sigma\otimes\mu_{C}\in\setTO(\rho\otimes\mu_{C}) $, it is also true that $ \sigma\otimes\mu^{\prime}_{C^{\prime}}\in\setTO(\rho\otimes\mu^{\prime}_{C^{\prime}}) $. 
\end{enumerate}
\end{lem}
\begin{proof}
	We will prove this lemma by constructing $\mu_{C'}'$ appropriately from $\mu_C$. Assume that the system Hilbert space $ \mH_{S} $ is $ d $-dimensional, the catalyst Hilbert space $ \mH_{C} $ is $ c $-dimensional, and that the Hamiltonian of each system is given as $ H_{S} $ and $ H_{C} $.
	Denote the population of $ \rho $ as $ p_{i} = \bra{i}_{S}\rho\ket{i}_{S} $ and that of $ \mu_{C} $ as $ r_{i} = \bra{i}_{C}\mu_{C}\ket{i}_{C} $. 
	We label indices $ i $ for the catalyst according to the $ \beta $-ordering of $ \mu_{C} $ so that $ \frac{r_{1}}{\tau^{\beta}(H_{C})_{1}}\geq\frac{r_{2}}{\tau^{\beta}(H_{C})_{2}}\geq \cdots\geq \frac{r_{c}}{\tau^{\beta}(H_{C})_{c}} $, regardless of the energy order, i.e., $ \bra{i}_{C}H_{C}\ket{i}_{C} $ is not monotonic in $ i $ in general.
	
	It is easy to see that the thermomajorization curve $ \Gamma[\tau^{\beta}(H_{S})\otimes\mu_{C}] $ overlaps fully with $ \Gamma[\mu_{C}] $. Also, 
	\begin{equation}
		 \Gamma[\rho\otimes\mu_{C}]\geq\Gamma[\tau^{\beta}(H_{S})\otimes\mu_{C}]  = \Gamma[\mu_{C}] 
	\end{equation}
at all points in $ [0,1] $ since $ \rho\succ_{\beta}\tau^{\beta}(H_{S}) $. 

	If Eq.~\eqref{eq:notin_interior} holds, it means that $ \Gamma[\rho\otimes\mu_{C}] $ and $ \Gamma[\tau^{\beta}(H_{S})\otimes\mu_{C}] $ overlap at non-trivial points from Remark~\ref{rmk:interior_thermomaj}, and these would be either the elbow points of $ \Gamma[\mu_{C}] $, or whole segments between two elbow points of $\Gamma[\mu_{C}] $. Nevertheless, the first non-trivial point that two curves meet is without loss of generality, $ n $'th elbow of $ \Gamma[\mu_{C}] $ with $ n<c $, which is the point $\mN$ described by coordinates
\begin{equation}
	\mN = \left(~ \sum_{i=1}^{n}\tau^{\beta}(H_{C})_{i}~,~ \sum_{i=1}^{n}r_{i} ~\right).
\end{equation}
	For the next step, let us define the set 
	\begin{equation}
%			S_{nd} = \left\lbrace \left\lbrace ~\ket{i}_{S}\ket{j}_{C}  ~ \right\rbrace_{i=1}^d  \right\rbrace_{j=1}^n
		S_{nd} = \left\lbrace \ket{i}_{S}\ket{j}_{C} \big\vert\ i = 1,\cdots,d,\ j = 1,\cdots,n \right\rbrace,
	\end{equation} 
	consisting of $ nd $ levels. Observe that when summing up the populations of $ \rho\otimes\mu_{C} $ and the thermal state $ \tau^{\beta}(H_{S}+H_{C}) $ for levels in $ S_{nd} $, 
	\begin{align}\label{key}
		\sum_{\ket{x}\in S_{nd}}\bra{x}(\rho\otimes\mu_{C})\ket{x} &= \sum_{j=1}^{n}\sum_{i=1}^{d}p_{i}r_{j} = \sum_{j=1}^{n}r_{j},\\
		 \sum_{\ket{x}\in S_{nd}}\bra{x}\tau^{\beta}(H_{S}+H_{C})\ket{x} &= \sum_{j=1}^{n}\sum_{i=1}^{d}\tau^{\beta}(H_{S})_{i}\tau^{\beta}(H_{C})_{j}\nonumber\\
		 &= \sum_{j=1}^{n}\tau^{\beta}(H_{C})_{j},
	\end{align}
	which coincide with $ y $- and $ x $-coordinates of the intersection point $ \mN $. Because of the fact that $\mathcal{N}$ is defined as a point where the curves $\Gamma[\rho\otimes\mu_{C}]$ and $\Gamma[\mu_{C}]$ meet, we can deduce more about the $\beta$-ordering of $\rho\otimes\mu_{C}$: recalling that the $ \beta $-order is selected to maximize the thermomajorization curve, we can conclude that the first $ nd $ levels in the $ \beta $-order of $ \rho\otimes\mu_{C} $ is necessarily some rearrangement of the set $ S_{nd} $. 
	
	Let us now consider a smaller catalyst $ \mu^{\prime}_{C^{\prime}} $ with density matrix and Hamiltonian
	\begin{align}\label{key}
		\mu^{\prime}_{C^{\prime}} &= \frac{1}{\sum_{i=1}^{n}r_{i}}\sum_{i=1}^{n}r_{i}\dm{i}_{C^{\prime}},\\ 
		H_{C^{\prime}} &= \sum_{i=1}^{n}\bra{i}_{C} H_{C}\ket{i}_{C} \dm{i}_{C^{\prime}}.
	\end{align}
	The thermomajorization curve 
	\begin{equation}
		\Gamma[\mu^{\prime}_{C^{\prime}}]  = \Gamma[\tau^{\beta}(H_{S})\otimes\mu^{\prime}_{C^{\prime}}]
	\end{equation}
can be obtained by cutting $ \Gamma[\mu_{C}] $ at $ \mN $, and a linear rescaling of the axes to transform $ \mN $ to $ (1,1) $. Similarly, $ \Gamma[\rho\otimes\mu^{\prime}_{C^{\prime}}] $ is obtained by starting from $ \Gamma[\rho\otimes\mu_{C}] $, performing the same cutting and rescaling with respect to $\mN$. By the assumption that $ \mN $ is the first non-trivial intersection point of $ \Gamma[\rho\otimes\mu_{C}] $ and $ \Gamma[\mu_{C}] $, no non-trivial intersection exists between $ \Gamma[\rho\otimes\mu^{\prime}_{C^{\prime}}] $ and $ \Gamma[\tau^{\beta}(H_{S})\otimes\mu^{\prime}_{C^{\prime}}] $ anymore, indicating that $ \tau^{\beta}(H_{S})\otimes\mu^{\prime}_{C^{\prime}}\in {\rm int}(\setTO(\rho\otimes\mu^{\prime}_{C^{\prime}})) $ using Remark~\ref{rmk:interior_thermomaj}. This proves the first statement of the lemma.
	
	With the above observation, the proof of the second statement is straightforward: suppose that 
	$ \sigma\otimes\mu_{C}\in\setTO(\rho\otimes\mu_{C}) $. Then
	\begin{equation}
		\Gamma[\rho\otimes\mu_{C}]\geq \Gamma[\sigma\otimes\mu_{C}] \geq \Gamma[\tau^{\beta}(H_{S})\otimes\mu_{C}], 
	\end{equation} 
	at all points in $ [0,1] $. Hence, $ \Gamma[\sigma\otimes\mu_{C}] $ also includes the point $ \mN $ and the above argument can again be applied -- we immediately have $ \sigma\otimes\mu^{\prime}_{C^{\prime}} \in \setTO(\rho\otimes\mu^{\prime}_{C^{\prime}}) $. 

\end{proof}

With this, we proceed with the proof of Theorem \ref{cor:CETO=CTO}.

\begin{proof}
	The proof is divided into two parts: we first show that for any $ \sigma\in\setCTO(\rho) $ and energy-incoherent $ \rho $, there exists a state $ \sigma^{\prime}\in{\rm Conv}(\setCETO(\rho)) $ that is arbitrarily close to $ \sigma $, and from there we prove the main claim, namely $ {\rm int}(\setCTO(\rho))\subset{\rm Conv}(\setCETO(\rho)) $. We only prove the statement for CETO since the case of CMTP can be similarly achieved using the equivalence of nM-ETO and GC-MTP.
	
	Suppose that $\sigma \in \setCTO(\rho)$, then there exist a TO channel and a catalyst, giving $\Phi_{\rm TO} (\rho\otimes \mu_{C})=\sigma \otimes\mu_{C} $. From Lemmas~\ref{lem:nm_vs_gc} and~\ref{lem:NMETO_TO}, we already know that there is a GC-ETO process that approximates the channel $ \Phi_{\rm TO} $ and outputs some state $ \varsigma_{SC} $ that is arbitrarily close to the desired state $ \sigma\otimes\mu_{C} $. Yet, in general, the catalyst might not be recovered exacly, i.e., $ \varsigma_{SC} \neq \sigma^{\prime}\otimes\mu_{C} $ for any $ \sigma^{\prime} $ that is arbitrarily close to $ \sigma $. We demonstrate the existence of $ \sigma^{\prime}\otimes\mu_{C}\in\setGCETO(\rho\otimes\mu_{C}) $, by proving that we can choose $ \sigma^{\prime} $ such that $ \sigma^{\prime}\otimes\mu_{C}\in{\rm int}(\setTO(\rho\otimes\mu_{C})) $ and applying 
	Theorem~\ref{thm:GC-ETO_TO}.
	
	The first step to prove $ \sigma^{\prime}\otimes\mu_{C}\in{\rm int}(\setTO(\rho\otimes\mu_{C})) $ is to make use of the assumption that $ \rho $ is energy-incoherent. For incoherent states, energy-incoherent catalysts are known to be sufficient for implementing any catalytic transformations~\cite{Brandao15_2ndlaws}; hence we choose $ \mu_{C} $ to be incoherent. In addition, Lemma~\ref{lem:catalyst_interior} implies that $ \mu_{C} $ can always be chosen so that $ \tau^{\beta}(H_{S})\otimes\mu_{C}\in{\rm int}(\setTO(\rho\otimes\mu_{C})) $.
	Then, by setting
	\begin{equation}\label{key}
		\sigma^{\prime} = (1-\epsilon)\sigma + \epsilon \tau^{\beta}(H_{S}),
	\end{equation} 
	with arbitrarily small number $ \epsilon>0 $, we have that 
	\begin{enumerate}
		\item the new state $ \sigma^{\prime} $ is arbitrarily close to $ \sigma $, and
		\item $ \sigma^{\prime}\otimes\mu_{C}\in{\rm int}(\setTO(\rho\otimes\mu_{C})) $.
	\end{enumerate}   
	Subsequently, from Theorem~\ref{thm:GC-ETO_TO}, $ \sigma^{\prime}\otimes\mu_{C} \in {\rm Conv}(\setGCETO(\rho\otimes\mu_{C})) $.
	In other words, not only the errors can be made arbitrarily small, but they can always be concentrated on the system rather than the catalyst. In conclusion, for any $ \sigma\in\setCTO(\rho) $, an approximation $ \sigma^{\prime}\in{\rm Conv}(\setCETO(\rho)) $ is always achievable. 
	
	Now we prove the second part by contradiction. Suppose that there exists a state $ \sigma\in{\rm int}(\setCTO(\rho)) $. Then, by definition of the interior of a set, we can construct an open ball of size $ \epsilon $ centred at $ \sigma $ that is completely contained in $ {\rm int}(\setCTO(\rho)) $, whenver $ \epsilon<\epsilon_{0} $ for some number $ \epsilon_{0} $. There are two cases to consider.
	\begin{enumerate}
		\item Suppose that $\sigma$ is in the exterior of $ {\rm Conv}(\setCETO(\rho)) $. Again by definition of the exterior of a set, an open ball of size $ \epsilon $ centred at $ \sigma $ is completely outside of $ {\rm Conv}(\setCETO(\rho)) $, whenver $ \epsilon<\epsilon_{1} $ for some other number $ \epsilon_{1} $. Yet, this contradicts the conclusion of the first part, namely there exists $ \sigma^{\prime} $ that is $ \epsilon < \min[\epsilon_{0},\epsilon_{1}] $ close to $ \sigma $ such that $ \sigma^{\prime}\in {\rm Conv}(\setCETO(\rho)) $.
		\item Suppose that $ \sigma $ is on the boundary, but not contained in the set $ {\rm Conv}(\setCETO(\rho)) $. By definition of the boundary of a set, for any $ \epsilon>0 $, there is at least one state inside the open ball of size $ \epsilon $ centred at $ \sigma $ that is in the exterior of $ {\rm Conv}(\setCETO(\rho)) $. By taking $ \epsilon<\epsilon_{0} $, that state in the exterior of $ {\rm Conv}(\setCETO(\rho)) $ is in the interior of $ \setCTO(\rho) $. This new state corresponds to the first case above, and we again arrive at the contradiction. 
	\end{enumerate}
These two contradictions then collectively prove that $\sigma\in {\rm int} (\mR_{\rm CTO}(\rho))$	necessarily implies that $\sigma \in {\rm Conv} (\setCETO(\rho))$.
	
\end{proof}




\end{document}
