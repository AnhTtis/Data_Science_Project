%%%%%%%%%%%%%%%%%%%%%%%%%%%%%%%%%%%%%%%%%%%%%%%%%%%%%%%%%%%%%%%%%%%%%%%%

%%% LaTeX Template for AAMAS-2023 (based on sample-sigconf.tex)
%%% Prepared by the AAMAS-2023 Program Chairs based on the version from AAMAS-2022. 

%%%%%%%%%%%%%%%%%%%%%%%%%%%%%%%%%%%%%%%%%%%%%%%%%%%%%%%%%%%%%%%%%%%%%%%%

%%% Start your document with the \documentclass command.
%%% Use the first variant below for the final paper.
%%% Use the second variant below for submission.

\documentclass[sigconf]{aamas} 
\usepackage[ruled, vlined, linesnumbered]{algorithm2e}
\usepackage{amsmath}
\usepackage{dblfloatfix}
\usepackage{caption}
\usepackage{subcaption}
\usepackage{booktabs}
\usepackage{framed}
%\usepackage{enumitem}
\usepackage{paralist}
\usepackage{booktabs}
\usepackage{multirow}
\usepackage[T1]{fontenc}
\usepackage{lmodern}

%\documentclass[sigconf,anonymous]{aamas} 

%%% Load required packages here (note that many are included already).

\usepackage{balance} % for balancing columns on the final page

%%%%%%%%%%%%%%%%%%%%%%%%%%%%%%%%%%%%%%%%%%%%%%%%%%%%%%%%%%%%%%%%%%%%%%%%

%%% AAMAS-2023 copyright block (do not change!)

\setcopyright{ifaamas}
\acmConference[AAMAS '23]{Proc.\@ of the 22nd International Conference
on Autonomous Agents and Multiagent Systems (AAMAS 2023)}{May 29 -- June 2, 2023}
{London, United Kingdom}{A.~Ricci, W.~Yeoh, N.~Agmon, B.~An (eds.)}
\copyrightyear{2023}
\acmYear{2023}
\acmDOI{}
\acmPrice{}
\acmISBN{}

%%%%%%%%%%%%%%%%%%%%%%%%%%%%%%%%%%%%%%%%%%%%%%%%%%%%%%%%%%%%%%%%%%%%%%%%

%%% Use this command to specify your EasyChair submission number.
%%% In anonymous mode, it will be printed on the first page.

\acmSubmissionID{541}

%%% Use this command to specify the title of your paper.

\title[HYDRA]{Learning to Operate in Open Worlds by \\ Adapting Planning Models}
\author{Wiktor Piotrowski}
\affiliation{
  \institution{Palo Alto Research Center}
  \city{Palo Alto}
  \country{USA}}
\email{wiktorpi@parc.com}
\author{Roni Stern}
\affiliation{
  \institution{Ben-Gurion University}
  \city{Beersheeba}
  \country{Israel}}
\email{wiktorpi@parc.com}
\author{Yoni Sher}
\affiliation{
  \institution{Palo Alto Research Center}
  \city{Palo Alto}
  \country{USA}}
\email{yoni.sher@parc.com}
\author{Jacob Le}
\affiliation{
  \institution{Palo Alto Research Center}
  \city{Palo Alto}
  \country{USA}}
\email{jale@parc.com}
\author{Matthew Klenk}
\affiliation{
  \institution{Toyota Research Institute}
  \city{Los Altos}
  \country{USA}}
\email{matt.klenk@tri.global}
\author{Johan deKleer}
\affiliation{
  \institution{Palo Alto Research Center}
  \city{Palo Alto}
  \country{USA}}
\email{dekleer@parc.com}
\author{Shiwali Mohan}
\affiliation{
  \institution{Palo Alto Research Center}
  \city{Palo Alto}
  \country{USA}}
\email{smohan@parc.com}
% \author{Wiktor Piotrowski, Roni Stern, Yoni Sher, Jacob Le, Matthew Klenk, Johan deKleer, Shiwali Mohan}
% \affiliation{
% \institution{Palo Alto Research Center}
% \city{Palo Alto}
% \country{USA}
% }
% \email{wiktorpi@parc.com, roni.stern@gmail.com, yoni.sher@gmail.com, jale@parc.com}
% \email{klenk.matt@gmail.com, dekleer@parc.com, smohan@parc.com}

%%% Provide names, affiliations, and email addresses for all authors.

%%% Use this environment to specify a short abstract for your paper.

\vspace{-1cm}
\begin{abstract}
Planning agents are ill-equipped to act in novel situations in which their domain model no longer accurately represents the world. We introduce an approach for such agents operating in open worlds that detects the presence of novelties and effectively adapts their domain models and consequent action selection. It uses observations of action execution and measures their divergence from what is expected, according to the environment model, to infer existence of a novelty. Then, it revises the model through a heuristics-guided search over model changes. We report empirical evaluations on the CartPole problem, a standard Reinforcement Learning (RL) benchmark. The results show that our approach can deal with a class of novelties very quickly and in an interpretable fashion.
\end{abstract}

%%% The code below was generated by the tool at http://dl.acm.org/ccs.cfm.
%%% Please replace this example with code appropriate for your own paper.


%%% Use this command to specify a few keywords describing your work.
%%% Keywords should be separated by commas.

%\keywords{Legends, Myths, Folktales}

%%%%%%%%%%%%%%%%%%%%%%%%%%%%%%%%%%%%%%%%%%%%%%%%%%%%%%%%%%%%%%%%%%%%%%%%

%%% Include any author-defined commands here.
         
\newcommand{\BibTeX}{\rm B\kern-.05em{\sc i\kern-.025em b}\kern-.08em\TeX}
\newcommand{\tuple}[1]{\ensuremath{\langle #1 \rangle}}


%%%%%%%%%%%%%%%%%%%%%%%%%%%%%%%%%%%%%%%%%%%%%%%%%%%%%%%%%%%%%%%%%%%%%%%%

\begin{document}

%%% The following commands remove the headers in your paper. For final 
%%% papers, these will be inserted during the pagination process.

\pagestyle{fancy}
\fancyhead{}

%%% The next command prints the information defined in the preamble.

\maketitle 

%%%%%%%%%%%%%%%%%%%%%%%%%%%%%%%%%%%%%%%%%%%%%%%%%%%%%%%%%%%%%%%%%%%%%%%%
\setlength{\textfloatsep}{-01pt}
\vspace{-0.2cm}
\section{Introduction}
% ORIGINAL VERSION
%Artificial Intelligence (AI) and Machine Learning (ML) research on sequential decision-making usually rely on the \emph{closed world} assumption. That is, all relevant characteristics of the environment are known ahead of deployment, during agent design time. For a planning agent, knowledge about environmental characteristics is encoded explicitly as a domain model (description of actions, events, processes) that govern the agent's beliefs about environment's dynamics. Encoded knowledge may be incomplete or incorrect or the environment might evolve, causing the agent to fail catastrophically. 
% Our research studies how intelligent agents can be designed such that they can robustly operate in the \emph{open world}, an environment whose characteristics can change while the agent is operational \cite{langley2020open, AI_for_openworlds_2022}. We are exploring how planning agents can handle changes --- \emph{novelties} --- in open worlds. Agents following our design use the same compositional models they use for planning to also evaluate if observed outcomes diverge from what is expected in the plan. If the divergence is si\setlength{\textfloatsep}{0pt}
Artificial intelligence and machine learning research on sequential decision-making usually relies on the \emph{closed world} assumption. That is, all relevant characteristics of the environment are known ahead of deployment, during agent design time. 
For a decision-making agent that relies on automated planning techniques, 
knowledge about environmental characteristics is encoded explicitly as a domain model (description of actions, events, processes) that govern the agent's beliefs about the environment's dynamics. 
In an \emph{open world}, however, the characteristics of the environment often change while the agent is operational \cite{langley2020open, AI_for_openworlds_2022}. 
Such changes --- \emph{novelties} --- can cause a planning agent to fail catastrophically as its knowledge of the environment may become incomplete or incorrect. We explore how planning agents can robustly handle such novelties in an open world. 
Agents following our design use the planning domain model to also evaluate if observed outcomes diverge from what is expected in the plans it generated. If the divergence is significant, the novelty is inferred and accomodated through heuristics search.
This approach is applicable to planning agents implementing various levels of PDDL. Results in this paper are from a system implemented using PDDL+\cite{fox2006modelling} for CartPole \cite{brockman2016}, a classic control problem. 


%Our results show that the proposed approach enables a planning agent to adapt \emph{quickly} with few interactions with the environment. Additionally, the adaptations produced by our approach are \emph{interpretable} by design - they are represented in terms of changes to the elements of this PDDL+ model, enabling and inspection of proposed changes. 
\vspace{-0.3cm}
\section{Approach}
\begin{figure}[t]
\vspace{-0.3cm}
    \centering
    \includegraphics[width=1\columnwidth]{novelty-reasoning.png}
    \vspace{-1cm}
    \caption{Diagram of novelty reasoning. Solid lines denote the planning process and dotted denote domain model revision.}
    \label{fig:novelty_reasoning}
\end{figure}
Figure \ref{fig:novelty_reasoning} shows the proposed agent design and the novelty reasoning process. 
The agent interacts with its environment in a sequence of episodes, where each episode is a set of actions taken by the agent to reach a terminal state. 
At some episode novelty is introduced and the environment changes, the agent is oblivious to the existence, timing, and nature of the introduced novelty. 

\begin{figure*}[hb]
    \centering
    \includegraphics[width=0.98\textwidth]{combined.pdf}
    \vspace{-0.4cm}
    \caption{Graphs showing performance of DQN-static/adaptive and planning-static/repairing agents. Episodes are on the x-axis and reward on the y-axis. The results are averaged over $5$ trials. Red line indicates the episode $7$ when novelty was introduced.}
    \label{fig:combined_results}
\end{figure*}
At an episode's beginning, the agent accepts the current state $s_t$ and creates a corresponding planning problem ($s_t, G$) which is then paired with the domain model $D$. Then, it uses a planner to solve the problem to obtain plan $\pi$ and attempts to execute in the environment. During execution, it stores the observed trajectory $\tau$ as a list of $\tuple{s_t,a,s_{t+1}}$. At the episode's end, it computes an \emph{inconsistency score} for the current model $D$ by comparing the expected state trajectory with the observed execution trace, $\tau$. 


%The environment is assumed to change in some way in an episode unknown to the agent. 
%The environment starts with the transition function $T$ and in an episode unknown to the agent switches to $T^*$. Roni: I think T is not needed in this short paper




Formally, let $S(\tau)$ be the sequence of states in observations and $S(\pi,D)$ be the expected sequence of states obtained by simulating the generated plan $\pi$ with the domain model $D$. 
Let $S(x)[i]$ denote the $i^{th}$ state in the state sequence $S(x)$. 
The inconsistency score is computed as
    $C(\pi, D, \tau) = \sum_{i} \gamma^i\cdot ||S(\tau)[i] - S(\pi,D)[i]||$
where $0<\gamma<1$ is a discount factor intended to limit the impact of sensing noise. 
If the inconsistency score exceeds a set threshold $C_{th}$, the agent infers that its domain model $D$ has become inconsistent with the novel environment characteristics. 
Then, it initiates the \emph{search-based model repair} process described in Algorithm \ref{alg:repair}
to adjust $D$ accordingly. 
Algorithm \ref{alg:repair} works by searching for a \emph{domain repair} $\varPhi$, which is a sequence of model modifications that, when applied to the agent's internal domain $D$, returns a domain $D'$ that is consistent with observations. To find such a repair, the algorithm accepts as input a set of basic \emph{Model Manipulation Operators} (MMOs), denoted $\{\varphi\} = \{\varphi_0, \varphi_1, ... , \varphi_n\}$. Each MMO $\varphi_i \in \{\varphi\}$ represents a possible change to the domain. A domain repair $\varPhi$ is a sequence of one or more basic MMO $\varphi_i \in \{\varphi\}$. An MMO example is to add an amount $\Delta\in\mathbb{R}$ to a numeric domain fluent. After this repair, the agent moves on to the next episode and uses the updated internal domain model $D'$ to solve the subsequent tasks. It may take a few repair steps to find a consistent domain model because a single trajectory may not provide enough information to find the correct repair. 
\vspace{-0.4cm}
\section{Results}
We evaluated our approach using a standard implementation of CartPole \cite{brockman2016}, where the task is to balance the pole in the upright position for $n=200$ steps by pushing the cart either left or right. The environment provides information on the velocities and positions of the cart and the pole (4-tuple). The domain's system dynamics are defined by several parameters: mass of the cart, mass of the pole, length of the pole, gravity, angle limit, cart limit, push force. % Each timestep results in a $+1$ reward. 

\begin{algorithm}
\scriptsize
\SetKwInOut{Input}{Input}\SetKwInOut{Output}{Output}
% \SetKwComment{Comment}{$\triangleright$\ }{}
\Input{$\{\varphi\}$: a set of basic MMOs; $D$: the original PDDL+ domain; $\pi$: plan generated using $D$; $\tau$: a trajectory; $C_{th}$: consistency threshold}
\Output{$\varPhi_{best}$, a domain repair for $D$}
OPEN$\gets\{\emptyset\}$; 
$C_{best}\gets\infty$;
$\varphi_{best}\gets\emptyset$\\
\While{$C_{best}\geq C_{th}$}{
    $\varPhi\gets$ pop from OPEN\\
    \ForEach{$\varphi_i\in\{\varphi\}$}{
        $\varPhi'\gets \varPhi \cup \varphi_i$ {\scriptsize \tcc*{Compose a domain repair}} 
        DoRepair($\varPhi'$, $D$)\\
        $C_{\varPhi'}\gets$ InconsistencyEstimator($\pi$, $D$, $\tau$)\\
        \If{$C_\varPhi\leq C_{best}$}{
            $C_{best}\gets C_{\varPhi'}$\\
            $\varPhi_{best}\gets\varPhi'$
        }
        Insert $\varPhi'$ to OPEN with key $f(\varPhi', C_{\varPhi'})$ \nllabel{alg:line:f}\\
        UndoRepair($\varPhi'$, $D$)
    }
}
%\Return $\varPhi_{best}$\\
\caption{PDDL+ model repair algorithm.}
\label{alg:repair}
\end{algorithm}

Figure \ref{fig:combined_results} summarizes the performance of various agents. We studied two novelties: changing the pole length to $1.1$ and the gravity to $12$; and changing the pole length to $1.1$ and the cart mass to $0.9$. 
As baselines, we implemented RL agents that use standard dynamic q-networks with experience replay\cite{mnih2013playing}. DQN-static employs the policy learned in non-novel settings in the novel environment while the dynamic version learns a new policy. 

The results show that the planning agents are, first, \emph{resilient}; the impact of novelty on their performance is not as drastic as on the DQN agents. 
It is because planning agents use models that are modular, composable and are written in a general way. 
In the novelty setting, a subset of model elements are still relevant. Second, our approach, the planning-adaptive agent learns \emph{quickly} and recovers optimal performance in around $20$ episodes. This observation supports our central thesis: model-space search enables quick adaptation in dynamic environments because it can localize the learning to specific parts of the explicit model and other parts are \emph{transferred}. In contrast, a DQN agent has to learn new network parameters afresh. Finally, the adaptations are \emph{interpretable}; they are expressed in the same language as the original model, enabling a model designer to inspect what the system has learned. Our method found the following example repairs for CartPole. Eaach element in the repair is a numeric domain fluent and the reported value is a change from its nominal value. \\
\small
\noindent\textbf{Repair 1:} \texttt{mass\_cart:0,length\_pole:0.3,mass\_pole:0,\\
 force\_mag:0,gravity:0,angle\_limit:0,x\_limit:0} \\
\textbf{Repair 2:} \texttt{mass\_cart:0,length\_pole:0,mass\_pole:0,force\_mag:0, \\
 gravity:1.0,angle\_limit:0,x\_limit:0}
% Please add the following required packages to your document preamble:
% \usepackage{booktabs}
% \usepackage{multirow}
% \begin{table}[]
% \begin{tabular}{@{}llll@{}}
% \toprule
%  Repairable var.       & Nom. val. & $\Delta$  \\ \midrule
%  mass cart             & 1.000     & 1.000  \\
%                               length pole           & 0.500     & 0.100  \\
%                               mass pole             & 0.100     & 0.100  \\
%                               force magnitude       & 10.000    & 1.000  \\
%                               gravity               & 9.810     & 1.000  \\
%                               pole angle limit      & 0.205     & 0.100  \\
%                               cart position limit   & 2.400     & 0.100  \\
%                             %   \cmidrule(l){2-4} 
%                             \bottomrule
% \end{tabular}

% \caption{Domain-specific parameters for PDDL+ model repair. Lists for each domain, the repairable state variables, their nominal value, and the $\Delta$ values used when trying to repair.\label{tab:domain-paramaters}}
% \end{table}


%\subsubsection{Planning Agents}
% \begin{table*}[h]
%     \centering
%     \small
%     \begin{tabular}{l|lcccccccc}
%     \toprule
%         \textbf{repairable fluents} &  \texttt{mass cart}& \texttt{length pole}& \texttt{mass pole}& \texttt{force magnitude}& \texttt{gravity}& \texttt{pole angle limit}& \texttt{cart position limit}\\
%         \textbf{nominal values} &1.0&0.5&0.1&10.0&9.81&0.205&2.4&\\
%         \textbf{$\Delta$} & 1.0& 0.1& 0.1& 1.0& 1.0& 0.1& 0.1\\
%     \bottomrule
%     \end{tabular}
%     \caption{Domain-specific parameters for PDDL+ model repair in CartPole.}
%     \label{tab:cartpole_params}
% \end{table*}

 \subsection*{Acknowledgements}
The work presented in this paper was supported in part by the DARPA SAIL-ON program under award number HR001120C0040. The views, opinions and/or findings expressed are those of the authors’ and should not be interpreted as representing the official views or policies of the Department of Defense or the U.S. Government. 


\bibliographystyle{ACM-Reference-Format} 
\bibliography{bibliography}

%%%%%%%%%%%%%%%%%%%%%%%%%%%%%%%%%%%%%%%%%%%%%%%%%%%%%%%%%%%%%%%%%%%%%%%%

\end{document}

%%%%%%%%%%%%%%%%%%%%%%%%%%%%%%%%%%%%%%%%%%%%%%%%%%%%%%%%%%%%%%%%%%%%%%%%

