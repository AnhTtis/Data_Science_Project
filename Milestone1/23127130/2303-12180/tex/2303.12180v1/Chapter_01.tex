\chapter{Introduction}
\label{chapter:chapter01}
\section{Motivation}
In recent years, there have been great advances in robotics, e.g., smart household assistant robots, humanoid robots, and exoskeletons for assisting in rehabilitation. 
\begin{figure}[h]
    \centering
    \includegraphics[width= 0.8\textwidth,height=1\textheight,keepaspectratio]{Figure/Figure1}
    \caption{The robot Asimo serves a delicious breakfast (a) . The next generation of Atlas lifts the box (b). The first generation of Atlas can walk on rough terrain (c). The prosthetic limb can recover and improve human strength.}
    \label{figure1}
\end{figure} 
Some humanoid robots are illustrated in Fig. \ref{figure1}. 
Honda has developed Asimo based on its similarity to humans as decipted in Fig. \ref{figure1}a. 
Asimo has the potential to perform human tasks in environments designed for human activities. 
Boston Dynamics has developed the Atlas robot that can walk on a terrain (Fig. \ref{figure1}c) and lift an object (Fig. \ref{figure1}b). 
The prosthetic limb shown in Fig. \ref{figure1}d helps the user regain balance by functioning similarly to human legs. 
These examples have shown the tremendous benefits that robots can bring to humanity
%, and I believe that humanity will maintain an efficient method of controlling robots when they eventually become part of our daily lives. 

%Considering this, my greatest desire is to build reliable, effective, and safe robots that can be easily integrated into our daily lives. 
%In this thesis, 
%as a first step towards achieving my ultimate goal, I would mainly like to develop a controller that enables stable bipedal walking motion in the sagittal plane. 
The controller is an important part of every robotic system. 
Specifically for humanoids, the balance controller plays an important role to maintain the upright posture of the robot. 
In order to develop such a controller, a reduced-order model of the corresponding complex model of the robot is often utilized. 
%and then translate these insights into a target behavior for our controller. 
For this reason, in this thesis, the Biped Trunk Spring Loaded Inverted Pendulum Model (BT-SLIP) is considered to be a reduced-order model of the articulated body robot. 
%and a controller designed for the articulated robots (the 5-link robot and the 7-link robot) will be conducted. 
This reduced-order model, also known as the template model, is designed to match the forces generated by humans during walking. 
This template model can give meaningful information to the control system to reproduce the observed human dynamics, e.g., the ground reaction force model. 
%On the other hand, we thought that in bipedal walking, the trajectories in space need not be precise. 
%Our control system is based on the current state of the robot, and I would also like to answer the question of whether it is possible to develop a simple and reliable method for bipedal walking. 
%This research started with the above question. 
%If a method of bipedal walking was so simple and reliable that it could be easily implemented by many researchers, it would be a very positive contribution to the robotics society. 
%For this reason, in this work, a simple walking method for humanoid robots is presented. 
\section{Literature review}
Controlling a bipedal robot is challenging due to the high number of degrees of freedom (DoFs) and the hybrid nature of stepping, where the continuous model changes at each stage, and can quickly become under-steered. 
Since it is difficult to directly control all of the dynamics, hierarchical control approaches are becoming more popular. 
Simple template models have been shown to be useful for capturing and analyzing animal and human locomotion behavior, and a mapping controller translates the behavior into individual actuator inputs from the articulated robot. 
\subsection{Reduced-Order Models for Biped Robots}  
\subsubsection{The Inverted Pendulum model}
The inverted Pendulum (IP) \cite{hemami1977inverted} is one of the earliest template models that has its center of mass vaults over a pivot point, constructed by a fixed mass-less leg. 
The IP model is widely applied to analyze the Passive Dynamic Walker \cite{mcgeer1990passive}, \cite{collins2001three}, and human motion \cite{kuo2005energetic}. 
Later, various simple walking robots are presented. Katoh et al. \cite{katoh1984control} built a walking robot whose controller is based on a stable limit cycle of the double IP model. 
Hurmuzlu et al. \cite{hurmuzlu1986role} stated that the ground reaction impacts were a major contributor to dynamic walking stability. 
Later, researchers extended the simplified IP model to the Linear Inverted Pendulum (LIP) model. 
%The LIP model can be used as a model for the analysis of a humanoid robot in 3D space. 
A 2D version of the LIP model was first proposed in 1991 \cite{kajita1991study} and extended to a 3D version \cite{kajita20013d}, as shown in Fig. \ref{fig:LIPM}. 
It consists of assumptions based on approximating the dynamics of the robot by an inverted pendulum in the following:
\begin{itemize}
    \item The robot is represented by a point mass m located at its center of mass (CoM).
    \item The legs of the robot are massless and freely moved in swing phases (it has no swing actuation).
    \item The height of the CoM is kept constant during the motion ($z_c = constant$). 
\end{itemize}

\begin{figure}[ht]
\centering   \includegraphics[width=0.2\textwidth,height=0.5\textheight,keepaspectratio]{Figure/1.PNG}
\caption{ The 3D model of the LIP}
\label{fig:LIPM}
\end{figure}

By constraining the CoM movement of an inverted pendulum model in a horizontal plane, the relationship between the position of the Center of Pressure (CoP) and the CoM state can be simplified
\begin{equation}
\begin{aligned}
& p_y = y_c - \dfrac{z_c}{g}\ddot{y}_c \\
& p_x = x_c - \dfrac{x_c}{g}\ddot{x}_c,
\end{aligned}
\end{equation}
where $[p_x,p_y]$ is the position of the CoP and $[x_c, y_c, z_c]^T$ is the position of the CoM. 
The LIP model is widely used by many interesting and successful humanoid dynamic walking controllers. 
Kajita et al. \cite{kajita2003biped} proposed the ZMP Preview Control, which is inspired by Model Predictive Control (MPC). 
They used Model predictive control (MPC) to generate a control based on the predicted future states using a prediction horizon. 
This method takes into account the pre-planned future ZMP reference locations and uses the jerk of the CoM as a control signal. 
Then, a dynamically stable CoM trajectory can be generated by solving an optimal control problem that minimizes the ZMP tracking error and control error while maximizing the CoM trajectory smoothness. 
Later, there are several improvements that are applied to ZMP preview control. 
To deal with uneven ground and external disturbances, Kajita et al. \cite{kajita2006biped} introduced Auxiliary ZMP Control that temporarily forces the ZMP to deviate from the reference trajectory by a certain amount, but there are some negative effects to the subsequent walking motions in an undesirable way.
\subsubsection{The Spring Loaded Inverted Pendulum (SLIP) model}
More recently, the SLIP model is the other well-known model which predicts and explains essential characteristics of human walking and running such as the gait-specific pattern of ground reaction force (GRF) and the center of mass (CoM) trajectory \cite{blickhan1989spring}, \cite{mcmahon1990mechanics}. 
In this model, a whole body weight concentrates in its pelvis position, and its leg acts as a massless spring. 
Inspired by the SLIP model,  Geyer et al. \cite{geyer2006compliant} proposed a walking template which is called the Bipedal Spring Loaded Inverted Pendulum (BSLIP) model. It has some advantages when compares with the LIP model, e.g.,  this model reproduces the CoM vertical oscillations and the double peak pattern of the ground reaction force, which are the essential characteristics in human walking. 
Moreover, the double support phases are included naturally without additional mechanisms required. 
That advantage creates the resemblance from a human-like gait to the walking motion of the BSLIP model and allows it to close of the gap between humans and humanoid robots. Nevertheless, the point-mass assumption hinders this model to address postural control whereas vertical body alignment plays a significant role in the stabilization of human locomotion \cite{maus2010upright}. 
For that reason, the SLIP model is extended to include the upper body (trunk) as in the Trunk-SLIP (TSLIP) model \cite{sharbafi2013robust}, Asymmetric SLIP (ASLIP) model \cite{poulakakis2009spring}, and the Bipedal TSLIP (BTSLIP) model \cite{sharbafi2015fmch}. 
Maus et al. \cite{maus2008stable} proposed the attractive concept for stabilizing the trunk \cite{maus2010upright}. 
The authors introduced Virtual Pendulum (VP) concept through observations in several terrestrial locomotions including humans and then proposed a method to redirect the ground reaction force (GRF) towards a virtual pivot point (VPP) on the trunk located above the CoM. 
The SLIP model with trunk requires the hip-actuated such as the VPP concept and the force modulated compliant hip method (FMCH) \cite{sharbafi2015fmch} to balance the floating body. This model takes into account the ground reaction forces and elastic behaviors similarities with humans and animals. 
\subsection{Task Space Controllers}
Task-Space Control, also known as Operational-Space Control, is a framework that provides flexible and compliant control for highly redundant robotic systems. 
The main focus is on task variables and resolving joint redundancies. 
The tasks can be formulated at velocity, acceleration, or force levels. 
In walking and running locomotion, a task-space controller is often paired with a simple model planner in a hierarchical control scheme \cite{mordatch2010robust}, \cite{wensing2013optimizing}. 
More specifically, at the higher level of control, the simple model planner optimizes the task-space objectives, e.g., the CoM trajectory and foot trajectory, usually with consideration of long-term performance and stability. 
The task-space controller then serves as a low-level control component that focuses on converting the task objectives at the current instant to the whole-body joint space. 

There are different methods to solve task-space control. 
Pratt et al. \cite{pratt2001virtual} proposed the intuitive control framework which is called Virtual Model control. 
In this structure, he used the simple Jacobian transpose to map the forces of virtual springs into joint torques. 
However, this method neglects the Coriolis forces of the system dynamics. 
Later, Park and Khatib \cite{park2006contact} used the nullspace projection methods to solve the task-space control problem. 
A task is defined by equality of reference such as $e = 0$ where $e=s-s^{*}(t)$ is an error to be regulated to 0. 
The task presents a bilateral constraint. On the contrary, the unilateral constraint is typically represented by an inequality $e_i \leq 0$. 
Thus, it is hard to direct this restriction \cite{mansard2008continuous}. 
By planning task dynamics carefully, they could bypass the feasibility issue due to frictional and unidirectional limits on GRF \cite{sentis2010compliant}. 
To avoid calculating null-space projections, task-space control is very often formulated as a multi-objective quadratic programming problem (QP) with constraints on instantaneous dynamics and contact conditions \cite{abe2007multiobjective}, \cite{de2010feature}. 

In recent years, researchers have emerged the centroidal angular momentum as an important task objective for humanoid whole-body motion control \cite{orin2013centroidal}, \cite{orin2008centroidal}. 
It has been shown that properly regulating centroidal angular momentum is crucial in dynamic balancing \cite{hofmann2009exploiting} and highly dynamic movements \cite{wensing2014development}. 
Dai et al. \cite{dai2014whole} optimized the joint trajectories without taking into account the full-body dynamics of the robot. 
The author only considered the simpler centroidal dynamics of the robot. 
This method helps to quickly generate highly-dynamic humanoid whole-body motion plans. 
%%%%%%%%%%%%%%%%%%%%%%%%%%%%%%%%%%%%%%%%%%%%%%% NEW SECTION
\subsection{Hierarchical Control Frameworks}
Bio-inspired templates have been providing insights understanding about locomotion and instructing on possible control strategies for the humanoid robot. 
Based on the viewpoint of energetic and geometric similarities of the human movement to the LIP model, there are several controllers and concepts were proposed. 
Kajita et al..\cite{kajita2006biped} introduced the controller-based zero moment point (ZMP) concept that temporarily forces the ZMP to deviate from the reference trajectory by a certain amount in case of uneven ground or external disturbances during walking. 
Later, "Capture Point" is another concept that is developed with the LIPM model \cite{pratt2001virtual}. 
The capture point is a point on the ground where the robot can step to bring itself to a balanced state. 
Later the capture point concept is extended to as the "Divergent Component Point" (DCM) in \cite{takenaka2009real}. 
Englsberger et al. \cite{englsberger2011bipedal} proposed two control strategies for DCM tracking in the DLR-Biped robot. 
More recently, Englsberger et al. \cite{englsberger2013three} extended the DCM concept to a 3D model and allowed the LIPM to walk reliably over uneven terrain. 
They found that the walking DCM control framework is more flexible than the ZMP preview control. 
So far, the DCM-based walking controller has been applied on many experimental humanoid platforms and proven effective such as Atlas robot \cite{englsberger2015three}, Thor robot \cite{hopkins2015compliant}, and M2V2 robot \cite{pratt2012capturability}. 

On the other hand, ground reaction forces and elastic behaviors are other favorite aspects addressed by another controller based on the SLIP model. 
The SLIP model has been widely used as a motion template for running and hopping (jumping) in biped and humanoid robots. 
For instance, Hutter et. al. \cite{hutter2010slip} proposed the hybrid controllers for the running robot StarETH. 
This controller is proposed to combine the motion of CoM predicted by the SLIP model and Operational Space Control as a higher lever controller. 
Mordatch et al. \cite{mordatch2010robust} approximated the SLIP model by decoupling and linearizing the inverted pendulum in the horizontal plane and adding a spring with constant stiffness in the vertical direction. 
The approximate model has closed-form dynamics, so they can run a population-based preview optimization (based on Covariance Matrix Optimization \cite{hansen2006cma}) in real time to select CoM and foot trajectories for both running and walking gaits. 
Garofalo et al. \cite{garofalo2012walking} embedded the Biped-SLIP dynamics into a five-link biped robot model, but they assumed the feet have full actuation on the ground. In terms of adjusting the stiffness in the SLIP model, Visser et al. \cite{visser2012robust} presented the feedback linearization law so that it will reliably track a precomputed trajectories states of the SLIP model in the presence of a disturbance. 
Hereid et al. \cite{} used the CoM trajectory generated by a Dual-SLIP model as an optimization cost to tune the parameters of their HZD-based walking controller. 
Recently, Rezazadeh et al. \cite{rezazadeh2015toward} also synthesized a stable walking gait in ATRIAS, a bipedal underactuated robot. 
However, they found that directly commanding an underactuated robot to follow a SLIP-produced trajectory can be problematic in the real world. 
Thus, they attempted to detect the essential stabilizing control laws in the reduced-order model (SLIP) that can also maintain their stabilizing effects on the full-order robot. 
\section{Organization and Contribution} 

This thesis contributes to the development of a hierarchical control strategy for the bipedal walking robot over rough terrain and disturbance forces. 
Inspiring by the Virtual Pendulum concept which describes a mechanical behavior rather than an exact movement trajectory, we develop the Biped Trunk-Spring Loaded Inverted Pendulum (BTSLIP) model as the template model for our control strategy to capture a set of dynamic reference features.
Then, these dynamic reference features are transferred into the articulated robot design via a hierarchical control strategy. 

Chapter \ref{chapter:chapter02} introduces the control strategy on the BTSLIP model. We present two methods to achieve robust walking motion against external disturbances. In the first method, we introduce the combination of the legs' stiffness controller and the discrete linear quadratic regulator (DLQR).  
Then, the feedback linearization controller for leg stiffness tracks both the reference vertical position and velocity of the Center of Mass (CoM) while the DLQR helps to adapt the Virtual Pivot Point by tracking the periodic solution of walking. 
In the second method, to avoid the computational burden and model-based information, we propose the second approach, called the force direction controller, to stabilize the walking motion. 
The algorithms presented in this chapter are published in \cite{vu2017control}, \cite{lee2017control}, \cite{lee2017force}. 
%We claim that the direction of the GRF on the CoM is what important factor to consider, rather than the position of the point where the GRF is redirected to, in order to approach the fascinating VP concept.

Chapter \ref{chapter:chapter03} introduces the articulated robot model with a 5-link used for walking demonstration. 
In this chapter, a state-based swing controller and the force direction controller are implemented for the stance leg. 
Additionally, the task-space controller is utilized to transform the behavior of the simple template model into a more sophisticated robot. %Operation Space Control and the classical Jacobian Transpose are presented in this chapter. 
The implemented controllers are verified with a 5-link biped robot on rough surfaces and external disturbances. This chapter is published in \cite{vu2017walking}

Chapter \ref{chapter:chapter04} extends the 5-link robot to the articulated robot model with the Flat foot. 
By adding the ankle control rules, we use the task space controller implemented in the previous chapter to generate the joint torques of the robot. 
The finite state machine is implemented to give the right decision among control phases. 
The flat foot model is tested in Open Dynamics Engine under a rough surface. Finally, a summary and future work are given in Chapter \ref{chapter:chapter06}. 
This chapter is published in \cite{vu2017finite}
 


