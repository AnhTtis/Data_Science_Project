\chapter{Control Strategies for Stabilizing the Biped Trunk Spring Loaded Inverted Pendulum}
\chaptermark{}
\label{chapter:chapter02}
\section{Introduction}
In the field of robotics and biomechanics, the interest in bipedal robotic locomotion has accelerated for many years as a result of higher demands for advanced humanoids to serve in military, exploration, industries and daily life. This advancement is based on the effort to understand human locomotion in biomechanical investigations. Several different approaches were proposed including one well-known approach wherein a simple conceptual template models \cite{full1999templates} were demonstrated to be helpful for targeting and analyzing the behavior in the locomotion of animals and humans. One of the earliest models included an inverted pendulum that has its center of mass vaults over pivot point, constructed by a massless leg (i.e. Inverted Pendulum Model- IP) \cite{hemami1977inverted}. Based on that earlier point of view, Passive Dynamic Walkers (PDWs) were analyzed and demonstrated natural walking behavior \cite{mcgeer1990passive}, \cite{collins2001three}. Researchers soon after applied simplified IP model to Linear Inverted Pendulum and combined it with proper modulation of Zero Moment Point concept \cite{sardain2004forces}. That resulted in the success of many position controlled robots like ASIMO \cite{sakagami2002intelligent} which are more complex and have considerably improved mobility.

More recently, the Spring Loaded Inverted Pendulum (SLIP) model is one of the most attractive templates for predicting and explaining essential characteristics of human walking and running such as the gait-specific pattern of ground reaction force (GRF) and the center of mass (CoM) trajectory \cite{blickhan1989spring}, \cite{geyer2006compliant}. In the SLIP model, a whole body weight concentrates in its pelvis position, and its leg acts as massless spring. However, the point-mass assumption hinders this model to address postural control whereas vertical body alignment plays a significant role in stabilization of human locomotion \cite{maus2010upright}. For that reason, the SLIP model is extended to include upper body (trunk) as in the Trunk-SLIP (TSLIP) model \cite{sharbafi2013robust}, Asymmetric SLIP (ASLIP) model \cite{poulakakis2009spring}, and the Bipedal TSLIP (BTSLIP) model \cite{sharbafi2015fmch}. Maus et al. proposed the attractive concept for stabilizing the trunk \cite{maus2010upright}. The authors introduced Virtual Pendulum (VP) concept by observations in several terrestrial locomotions including human and then proposed a method to redirect the ground reaction force (GRF) towards a virtual pivot point (VPP) on the trunk located above the CoM. 

However, with external disturbance, the VP system slowly converges to t steady motion. 
Sharbafi et al. \cite{sharbafi2013robust} presented the combination between the Discrete Linear Quadratic Regulator (DLQR) and the VPP method in hopping motion. Besides, the muscular-skeletal of the human body can adapt to different gaits and terrains by adjusting the leg stiffness. Visser et al. \cite{visser2012robust} applied this advantage to the SLIP model and showed the robustness against disturbances in walking motion. 

In this chapter, we only consider a planar simple bipedal model with massless springy legs. Firstly, we present the combined control strategy for the BTSLIP model walking under external disturbances. The control strategy is proposed by the combination of controlling leg stiffness and the VPP method coupling with the DLQR. The feedback linearization controller for leg stiffness tracks both reference vertical position and velocity of the Center of Mass (CoM) to the desired trajectory. The desired trajectory of CoM is referred from the periodic trajectory of the BTSLIP model with the VPP method. 

Secondly, without precomputed trajectories, we propose a very simple Force Direction Control for bipedal walking with trunk, inspired by the Virtual Pendulum concept. 
%We first argue that having GRF towards a single fixed point such as VPP or DP is not sufficient for maintaining upright posture, even in a simple model with mass-less legs. 
We first argue that having GRF towards a single fixed point, e.g., VPP or DP, is not sufficient for maintaining upright posture. 
Based on this analysis, we propose a new GRF-redirecting control method for hip torques; the law no longer constrains the direction of GRF towards a single point but to the direction which is always providing a restoring moment to the body. 
A dynamic simulation result demonstrates the effectiveness of the proposed method. The simulation results indicate that the proposed method is a promising method for achieving stable and robust bipedal walking.

The remainder of this chapter is arranged as follows. In Section 2.2, we provide brief descriptions of the BTSLIP model consisting of its configuration, states, and dynamics. Section 2.3 presents the integrated control strategy including the detailed control strategies and simulation results. Finally, Section 2.5 highlights the Force direction controller consisting of the proposed control and simulation results. 
We conclude the paper with final remarks and future works in section V.
\section{The Bipedal Trunk Spring-Loaded Inverted Pendulum }
This section introduces the dynamics of the BTSLIP model. The model consists of the torso and two springy legs. It is assumed that the model is planar and it walks on a flat surface. The model parameters are set to match the characteristics of the human, as given in Table 1. 
\begin{table}[h]
\caption{Model parameters for the BTSLIP model}
\label{table_example}
\begin{center}
\begin{tabular}{| m{5cm} | m{3cm}| m{3cm} |}
\hline    
Parameter & Symbol & Value[unit]  \\
\hline
Torso mass & m & 80[$kg$]  \\

Torso moment of inertia & J & 4.58[\(kgm^2\)]\\

Distance hip to torso & \(r_h\) & 0.1 [\(m\)] \\

Distance torso to VPP & $r_{VPP}$ & 0.1 [$m$] \\

Leg rest length & \(L_0\) & 1 [$m$]  \\

Gravitational acceleration & g &9.81[\(ms^{-2}\)]  \\

Angle of attack & \(\alpha_0\) & \(70.6\)[\(deg\)] \\
\hline
\end{tabular}
\end{center}
\end{table}
\begin{figure}[ht]
    \centering
    \includegraphics[width=0.6\textwidth,height=0.7\textheight,keepaspectratio]{ICMA/Figure3}
    \caption{(a) Model parameters. (b) Model variables}
    \label{fig:mesh1}
\end{figure}
\subsection{System Configuration and State Transition}
The BTSLIP model parameters are shown in Fig. 1(a). The model consists of a rigid trunk, which represents the upper body with mass $m$ and moment of inertia $J$, and massless legs. \(L_i\) presents the length of each leg, we indicate the index \(i \in [1,2]\), where 1: left leg and 2: right leg.

The configuration of the system is defined by the variables \(x\), \(y\) and $\varphi$ as the CoM horizontal, vertical positions and the trunk orientation, respectively. These variables, in combination with their corresponding velocities, are used to describe the state of system as \( Q := \{ x_s := [x,y,\varphi,\dot{x},\dot{y},\dot{\varphi}]^T | \: x_s \in o\}\). In particular, $\varphi$ is defined as positive angle in counter clock-wise direction (i.e. $\varphi = 90[\deg]$ if the trunk is vertical). 
 
In a single step, the system can be either in single support (SS) or in double support (DS) which depends on the leg contact conditions. SS begins at the take-off moment of one leg and finishes at the touchdown moment with the same leg. With the assumption of the massless leg, the swing motion of the swing leg does not change the dynamics of the model. The swing leg is assumed to touch the ground with the constant angle of attack, \( \alpha_0 \).

\begin{figure}[ht]
    \centering
    \includegraphics[width=0.35\textwidth]{ICMA/Figure2}
    \caption{State transition of dynamical system}
    \label{fig:transition}
\end{figure}

The position of hip \( [x_h, y_h]^T\) which is below CoM by \(r_h\), is computed as follows:
\begin{equation}
\label{eqn: hip position}
\begin{aligned}
x_h &= x-r_{h}\cos\varphi \\
y_h &= y-r_{h}\sin\varphi  
\end{aligned}
\end{equation}

The transition from SS to DS can be mathematically defined as: 
\begin{equation}
S_{ss \rightarrow ds} = \{ x_s \in Q \: | \: y_h - L_0 \sin \alpha_0 = 0 \}
\end{equation} 

The position of the foot is defined as  $x_{f,i}$, which is assumed neither slip nor rebound during its stance phase. The length of the stance leg, $L_i$, is calculated by \(L_i = \sqrt{(x_h-x_{fi})^2+y_h^2}\). The transition from DS to SS is written as: 
\begin{equation}
S_{ds \rightarrow ss} = \{ x_s \in Q \: | \: L_i - L_0 = 0 \}.
\end{equation}

During a single step, we define two subsets of \(Q\) accordingly to two switching surfaces above. With \(x_s(t) \: \in \: Q_{ss} \) and  \(x_s(t) \: \in \: Q_{ds}\), the system is in the single support phase and double support phase respectively. Furthermore, a valid walking gait guarantees about not falling (i.e. \( y < 0\)) and forward walking (i.e. \(\dot{x} > 0 \)) in a trajectory \(x_s(t) \: \in \: Q_{ss} \bigcup Q_{ds}\). \(Q_{ss}\) and \(Q_{ds}\) are described below:
\begin{equation}
\begin{aligned}
Q_{ss} = & \:\{ x_s \in Q \: | \: L_0 \sin \alpha_0 < y_h < L_0 \}  \\
Q_{ds} = & \:\{ x_s \in Q \: | \:  0 < y_h < L_0 \sin \alpha_0 \}  
\end{aligned}
\end{equation}

The walking model is called a hybrid dynamical system \cite{sobotka2007hybrid}. Besides the continuous dynamics of the system with a set of ordinary differential dynamics equations, the system has a set of states transition. The transition of the system is shown in Fig. \ref{fig:transition}. 
\subsection{Dynamics of the BTSLIP Model}
\begin{figure}[ht]
    \centering
    \includegraphics[width=0.8\textwidth,height=0.8\textheight,keepaspectratio]{ICMA/Figure4}
    \caption{(a) Free body diagram (b) Virtual pendulum-based posture control (VPPC)}
    \label{fig: VPPC}
\end{figure}
In two phases, \(F_{s,i} = k(L_0-L_i)\) gives the spring force along the leg axis where \(L_i, L_0, k\) are respectively the current leg length, leg rest length and the spring stiffness with \(i= 1,2\) for each leg as shown in Fig 1.b. The orientation of the leg $i$ during stance phase, $\alpha_i$, can be trigonometric computed by hip position and the foot contact point.

The dynamic behavior of the system is determined by the forces acting on the CoM. The forces applied to the hip are the sum of the forces generated by spring forces \(\textbf{F}_{s,i}\) and the reaction forces \(\textbf{F}_{t,i}\) which is created by applying the hip torques \(\tau_i\), as shown in (6). A force diagram analysis is depicted in Fig. 3 (a), in local (\(x_1,y_1\)) axis parallel to the leg we have: 

\begin{equation}
\left \{
\begin{aligned}
&\sum \tau = 0, -\tau + F_tL = 0 \\
&\sum F \hat{x}_1 = 0, F_t - F_1 = 0 \\
&\sum F \hat{y}_1 = 0, F_s - F_2 = 0 \\
\end{aligned}
\right .
\end{equation}
Solving (5) for the force applied to the torso (\(F_1\hat{x}_1 + F_2\hat{y}_1\)) as the function of \(\tau\) and \(F_s\), returning the result into the global coordinate yields: 

\begin{equation}
\left \{
\begin{aligned}
F_{x,i} &= F_{s,i} \cos \alpha_i + \dfrac{\tau_i}{L_i} \sin \alpha_i \\
F_{y,i} &= F_{s,i} \sin \alpha_i - \dfrac{\tau_i}{L_i} \cos \alpha_i 
\end{aligned}
\right .
\end{equation}

With \(i \in [1,2]\) as leg index, the dynamical system of the BTSLIP model is described by:
\begin{equation}
\left \{
\begin{aligned}
 m \ddot{x} &= \sum_{i \in \mathbb{C} } F_{x,i} \\
 m \ddot{y} &= \sum_{i \in \mathbb{C}} F_{y,i} -\ m g \\
 J \ddot{\varphi} &= \sum_{i \in \mathbb{C}}^n (\tau_i + r_{COM}(F_{x,i} \sin \varphi - F_{y,i} \cos \varphi))
\end{aligned}
\right.
\label{fig:eomfi}
\end{equation}
where \(\mathbb{C}\) denotes set of contact legs 
\subsection{Searching for Periodic Gait and Local Stability of System}
The search for the periodic solution is taken at the instant of vertical leg orientation (VLO) in SS, and the torso vertically aligns with the foot point. 
Using this definition, the system state at VLO is described with state vector: \(S = (y, \varphi, \dot{x},\dot{y},\dot{\varphi})\). 
The Poincare return map P between two successive VLO is thus defined by \(S_{k+1} = P(S_k)\). 
Based on the Poincare map analysis, we can investigate the orbital stability of a limit cycle. 
For a small perturbation \(\Delta S^*\) around the fixed point, the nonlinear mapping function \(P\) can be transformed regarding Taylor series expansion: 
\begin{equation}
P(S^* + \Delta S^*) \approx P(S^*) + (\nabla P)\Delta S^*
\end{equation}
where \(\nabla P\) is the gradient of P with respect to the states, \(\nabla P\ = \dfrac{\partial P}{\partial S}\bigg|_{S=S^*}\). The local stability of a periodic motion is investigated by computing the eigenvalues \(\lambda\) of the matrix \(\nabla P \). We consider that the periodic motion is stable if all \(\lambda\) are smaller than one \cite{guckenheimer2013nonlinear}. 
\section{The Combined Control Strategy} 
In this section, we propose the control scheme to solve two separated tasks: (a) keeping the upright trunk by the VPP method and DLQR control, (b) tracking the desired vertical position and velocity of CoM. Note that the DLQR control will be active at every VLO event. The stiffness of legs are considered to be external control inputs to the system, and we model the leg stiffness \(k_i\) as the sum of a constant stiffness \(k_0\) and a variable component \(u_i\), i.e. \(k_i = k_0 + u_i \). Hence, the force exerted by the stance leg is given by: \( F_{s,i} = F_{si_0} + F_{si_u}\)
\subsection{The VPP Method and DLQR Controller}
The key idea of the VPPC is to redirect the GRF vectors towards a point located above the CoM, as shown in the Fig.\ref{fig: VPPC} (b).  Hence, the trunk oscillates like a pendulum mounted at the point VPP. By applying a torque \(\tau\) at the hip during walking, that will lead to generate a force perpendicular to the leg axis \(F_N\) to redirect of the GRF vector. As massless leg, the hip torque $\tau_i$ is computed by using geometric relation as:
\begin{equation}
\label{eqn: VPPC}
\left \{
\begin{aligned}
\tau_i &= (F_{si_0} + F_{si_u}) L_i \tan \beta_i \\
\tan \beta_i &= \dfrac{r_{h} \sin \psi + r_{VPP} \sin(\psi-\gamma)}{L_i + r_{h} \cos \psi + r_{VPP} \cos (\psi - \gamma) } 
\end{aligned}
\right .
\end{equation}
where \(r_{h}\) and \(r_{VPP}\) as distance from CoM to hip and VPP, \(\Psi\) and \(\gamma\) are the angle between hip-CoM to leg and CoM-VPP, respectively. \(i \in [1,2]\) is leg index. Solving (\ref{eqn: VPPC}) as the function of \(F_s\) and substituting into (\ref{fig:eomfi}), the equations of motion can be written in more compact form:
\begin{equation}
\dot{x_s} = f(x_s) + \sum_{i\in\mathbb{C}} g_i(x_s) u_i.\\
\end{equation}
where \(\mathbb{C}\) denotes set of contact legs and the input is only \(u_i\).

The DLQR control allows the adaptation of VPP position once per period using the computation which is based on discrete linear quadratic regulator (DLQR). At each VLO, the new VPP position determining by \(r_{VPP}\) and \(\gamma\) is computed for the next phase. Suppose that we have identified a pair \((S^*, \delta^*\)) that will produce a periodic walking gait in the BTSLIP model with the VPPC, i.e. \(r_{VPP}\) and \(\gamma\) are considered as inputs \(\delta = [r_{VPP},\gamma]^T\). We define the change of variables \(\Delta S_n = (S_n - S^*)\) and \(\Delta \delta_n = \delta_n - \delta^*\). The index n denotes the variables in the \(n^{th}\) VLO.

To design the controller, this change of variables allows to analyze the system using the first order of the Poincare return map around the fixed point \(S^*\):
\begin{equation}
\Delta S_{n+1} \approx J_S\Delta S_n + J_\delta \Delta \delta_n 
\end{equation}
where \(J_S =  \dfrac{\partial P}{\partial S}(S^*,\delta^*)\) and \(J_\delta = \dfrac{\partial P}{\partial \delta} (S^*,\delta^*)\). Consider the quadratic cost: 
\begin{equation}
\begin{aligned}
& \min\limits_{\Delta u} \sum_{n=1}^{\infty} \Delta S_n^T Q \Delta S_n + \Delta \delta_n^T R \delta_n  \\
& s.t \:\Delta S_{n+1} = J_S\Delta S_n + J_\delta \Delta \delta_n 
\end{aligned}
\end{equation} 
If the pair (\(J_S, J_\delta\)) is controllable, then the \(\Delta S_k\) can be controlled by using the feedback law as: 
\begin{equation}
\delta_n = -K S_n + \delta^*
\end{equation}
The state feed back gain \(K\) is given by: 
\begin{equation}
K = (J_S^T P J_\delta + R)^{-1} J_\delta^T P J_S, 
\end{equation}
where P is the unique solution of the Discrete- Time Algebraic Riccati Equation (DARE) \cite{bertsekas1995dynamic}.
\subsection{Feedback Linearization Control}
%Inspired by only a certain amount of stiffness variation, adjusted by control effort, it is require to track the desired gait. Thus, once the desired gait is reached, no input force is required to keep the gait [11]. This is enforce that desired gait defines a trajectory in the total state space, specifically in both term of position and velocity. The horizontal position desired gait is identified as a monotonically increasing in time value. Hence, we can parameterize the desired BTSLIP gait by \(x\). Accordingly, desired gait can be depicted as two reference function \(\bar{y}(x)\) and \(\dot{\bar{x}}(x)\). 
%Hence, by using VPPC, which the function of \(\tau\) in each instant of walking is relevant \(u_i\) function, we can rewrite the dynamic by replacing the \(\tau_i\) to the function of \(u_i\) in dynamics system using (8). Dynamics system is equivalent written in more compact form with VPPC: 
%\begin{equation}
%\dot{x_s} = f(x_s) + \sum_{i=1}^2 g_i(x_s) u_i\\
%\end{equation}
%Because of CoM position influences touchdown and takeoff event, the convergence of CoM position becomes more important. In SS phase, system is underactuated which means it only has one control input for converging horizontal position of CoM. In DS phase, the system is fully actuated with two stiffness forces input for converging both position of CoM  and forward velocity. The output target error function is defined as:
From the previous research \cite{maus2010upright} that the system described by (\ref{fig:eomfi}) with a particular set of values for the parameters and the VPPC seems to be suitable for describing human gait and may be used as a basis for more detailed models of human locomotion, referred to as the desired gait. We control the legs stiffness by feedback linearization in order to stabilize the walking model to desired gait. 

Because the CoM position influences touchdown and takeoff event, the convergence of CoM motion to the desired gait trajectory becomes critical. The model CoM horizontal position (\(x\)) of the desired gait trajectory obtained with the VPPC is monotonically increasing with respect to time. Hence, we can parameterize the desired trajectory by $x$. Accordingly, the desired gait trajectory can be described by $\bar{y}(x)$ and $\dot{\bar{x}}(x)$, which are the CoM vertical position and the CoM horizontal velocity. In SS, the system has only one control input which we can utilize to control the vertical position. In DS, we can exploit two control inputs to control the CoM horizontal velocity. Therefore, the output target error function is defined as:

\begin{align}
\begin{bmatrix}
h_1(x_s) \\
h_2(x_s)
\end{bmatrix} 
= \begin{bmatrix}
y - \bar{y}(x) \\
\dot{x} -\dot{\bar{x}}(x)
\end{bmatrix}
\end{align}


 
\textbf{\(\ast\) For \(x_s \in Q_{ss}\)\(\:\) :} we have the output \(y_s =h_1(x_s)\) results in the second- order input-output dynamics:
\begin{equation}
\begin{aligned}
\dot{y}_s(1) &= \dfrac{\partial h}{\partial x_s} [f(x_s) + g_i(x_s) u_i] =L_f h_1 + L_g h_1 u_i \\
\ddot{y}_s(1) &= L_f^2 h_1 + L_{g_i} L_f h_1 u_i, 
\end{aligned}
\label{eqn:fb1}
\end{equation}
where \(L_g h_1 u_i = 0\), \(L_f^2 h_i, L_f h_i\), and \(L_{g_i} L_f h_i\)  denote the Lie derivatives of the output target function along the vector fields of dynamics system. The feedback linearization control for legs stiffness, 
\begin{equation}
u_i = \dfrac{1}{L_{g_i}L_f h_1} (- L_f^2 h_1 - k_1 L_f h_1 -k_2 h_1)
\end{equation}
% \(u_i\) with \(i= 1, 2\), depends on whether the leg touch the ground. 


By defining the feedback controller in (\ref{eqn:fb1}), the output function are described by second-order form as follows: 
\[
\left \{
\begin{aligned}
& \ddot{h}_1 + k_1 \dot{h}_1 + k_2 h_1 =0 \\
& h_1 = c_1 e^{\lambda_1 t} + c_2 e^{\lambda_2 t} 
\end{aligned}
\right .
\]
If \(k_1, k_2\) are positive constants, then \(Re(\lambda_1)\) and \(Re(\lambda_2)\) are both negative. Therefore, the convergence of \(h_1\) function is guaranteed on \(Q_{ss}\).


\textbf{\(\ast\) For \(x_s \in Q_{ds}\)\(\:\) :} We have fully target function with two control objectives. Because \(L_{gi} h_2 \neq 0\), we can establish the  first-order feedback form for \(h_2\). The feedback control law is defined as below: 
\begin{equation}
\begin{aligned}
\begin{bmatrix}
u_1 \\
u_2 
\end{bmatrix}
= A^{-1} 
\begin{bmatrix}
-L_f^2 h_1 -k_1 L_f h_1 - k_2 h_1 \\
-L_f h_2 - k_3 h_2
\end{bmatrix}
\end{aligned}
\end{equation}
where 
\[
\begin{aligned}
A = 
\begin{bmatrix}
L_{g_1} L_f h_1 & L_{g_2} L_f h_1\\
L_{g_1} h_2 & L_{g_2} h_2
\end{bmatrix}
\end{aligned}
\]
The Lie derivative of \(L_{g_i} L_f h_1\) is not equal to zero as long as the length \(L_i\) of the stance leg $i$ satisfies \(0 < L_i <L_0\). Unfortunately, when the leg touches down the ground, the length of swing leg at this instant is still equal to \(L_0\). That leads $A$ to be singular at a very short finite time. Hence, with this condition, the feedback control with \(L_i = L_0\) in DS phase is defined as: 

\begin{equation}
\begin{aligned}
\begin{bmatrix}
u_1 \\
u_2
\end{bmatrix}
=
A^+ 
\begin{bmatrix}
-L_f^2 h_1 - k_2 L_f h_1 - k_2 h_1
\end{bmatrix}
\end{aligned}
\end{equation}
where \(A^+ = [L_{g_1} L_f h_1 ; L_{g_2} L_f h_1]^+\) is  Moore-Penrose pseudo inverse of $A$.
During DS, using the feedback control law, we can describe the target function \(h_2\) as: 
\[
\left \{
\begin{aligned}
& \dot{h}_2 + k_3 h_2 = 0 \\
& h_2 = e^{-k_3 t} h_2
\end{aligned}
\right .
\]
If \(k_3\) is positive constant then \(h_2\) certainly converges to zero on \(Q_{ss}\). 
\begin{figure}[h!]
    \centering
    \includegraphics[width=0.4\textwidth,height=0.4\textheight,keepaspectratio]{ICMA/Figure5}
    \caption{Eigenvalues of the Poincare map linearized around the fixed point of the desired gait.}
    \label{eigen}
\end{figure}
\subsection{Simulation Result of Combined Control Strategy}
In this section, nonlinear dynamic simulation results are presented that validate the control strategy. For the simulations, the BTSLIP model described in Section II is used, together with the control structure as shown in Section III. The main analyzing method for stability of the desired gait is investigating eigenvalues of Poincare return map. Moreover, by adding the disturbance forces are employed to the model to evaluate the robustness.
\subsubsection{Local Orbital Stability}
The numeric simulation shows that the desired gait obtained with a particular set of parameters and the VP model has an asymptotically stable limit cycle in Fig. \ref{eigen}. The eigenvalues are computed as \(\lambda = 0.9992, 0.074 \pm  0.9693i, 0.6387 \pm 0.7400i \). we consider that if the \(\lambda\) have magnitudes less than one, the periodic orbit is asymptotically stable. 
\begin{figure}[h!]
    \centering
    \includegraphics[width=0.85\textwidth,height=0.85\textheight,keepaspectratio]{ICMA/Figure6}
    \caption{Phase portrait of the BTSLIP model with proposed control under the disturbances. The phases plot are changed color to green and red after the disturbances are applied at \(t = 5s\) and \(t = 10s\). Phase portrait of the vertical position of CoM (a) and phase portrait of the trunk angle (b).}
    \label{proposed}
\end{figure}
\subsubsection{System Responses Against Disturbances}
In order to demonstrate the effectiveness of the control strategies, we applied the perturbation forces \(F = [F_x,F_y]^T = (-100,300)^TN \) at \(t = 5s\) and \(t = 10s\) in \(0.2\) seconds. Both translation motion and rotation motion were affected by disturbances in the proposed method and the VPP method, as shown in Fig. \ref{proposed} and Fig. \ref{VPP}, respectively. 

Fig. \ref{proposed}(a) presented the phase portrait of the vertical CoM motion. When the disturbance is applied, jerky behaviors are observed before the system response directly returns to the desired periodic orbit by the feedback linearization controller. Fig. \ref{proposed}(b) shows the motion of trunk, which shows much variability after applying disturbance but quickly converges to the steady motion. In comparison with the proposed control strategy, Fig. \ref{VPP} indicates that the VPP method could not reject the disturbance, the BTSLIP model failed after applying the first disturbances at \(t = 5s\).
\begin{figure}[t]
    \centering
    \includegraphics[width=0.85\textwidth,height=0.85\textheight,keepaspectratio]{ICMA/Figure10}
    \caption{Phase portrait of the VPP method under the disturbances. The phases plot are changed color green after the disturbances are applied at \(t = 5s\). Phase portrait of the vertical position of CoM (a) and phase portrait of the trunk angle (b).}
    \label{VPP}
\end{figure}

Fig. \ref{error} shows the time progression of the error function \(y_s\). With feedback linearization controller, the error function \(h_1\) exponentially converges to zero in both DS and SS. \(h_2\) performs the exponentially converges to zero in DS.
\begin{figure}[ht!]
    \centering
    \includegraphics[width=0.8\textwidth,height=0.8\textheight,keepaspectratio]{ICMA/Figure7}
    \caption{Error output function. The disturbances are applied at the instance of the dashed line. In particular, \(h_1(t)\) exponential decrease to zero in both phase (a), \(h_2\) is directly controlled in DS phase (b)}
\label{error}
\end{figure}
\begin{figure}[ht!]
    \centering
    \includegraphics[width=0.8\textwidth,height=0.8\textheight,keepaspectratio]{ICMA/Figure8}
    \caption{Maximum error at VLO over number strides of walking}
    \label{VLO}
\end{figure}
Fig. \ref{VLO} presents the maximum value of the \(\Delta S\) at \(n^{th}\) stride. It obviously shows that the proposed control can reject the disturbance. We consider that the rejection is satisfied if the gait is sustained around ten steps after the disturbance has been applied.  
\begin{figure}[ht!]
    \centering
    \includegraphics[width=0.8\textwidth,height=0.85\textheight,keepaspectratio]{ICMA/Figure9}
    \caption{Phase portrait of the BTSLIP model with proposed control under rough terrain (the maximum of height \(h_{max} = 2cm\)). The phases plot are changed color when the rough terrain stop at \(t = 15s\). Phase portrait of the vertical position of CoM (a) and phase portrait of the trunk angle (b).}
    \label{height_change}
\end{figure}

To further validate the controller, the Fig. \ref{height_change} shows performance of the walking model on the rough terrain (height of ground: \(h \leq 2cm\)). The vertical position is immediately stabilized, the trunk motion oscillated with a wide range, but the model still walked. After rough terrain (i.e. $t>15s$), the trunk movement quickly converges to the steady motion. 
\section{The Force Direction Controller}
The previous combined control strategy showed the stability in term of not falling and robustness against the disturbances. However, the controller is depended on the precomputed trajectory from the BTSLIP model with the VP concept. To keep inspiring from the VP concept, we seek a method to control the GRF directions that facilitate stable gait without the need for precise computed trajectory. We first argue that having GRF towards a \textit{single fixed point} (either VPP or DP) is not emerging for maintaining upright posture than adjusting the direction of GRF in an appropriate case; a careful but simple analysis supports this argument. Based on this analysis, we propose a new GRF-redirecting control law for that virtual hip torques should aim; the law no longer constraints the direction of GRF towards a single point but to the direction which is always providing a restoring moment to the body. The method extremely increases the robustness of the system. 

In Fig. \ref{fig2: restoring and uprighting moment}, we present all possible postures of the schematic model (trunk with massless legs). 
The trunk may be upright or tilted in clockwise or counter clockwise (CW/CCW) direction. At the same time, the foot of stance leg (leg in touch with the ground) may be located right below or posterior/anterior the hip. The combination provides different postures to analyze. Mostly, as intended, the GRF pointing the VPP provides restoring moment, or at worst, zero moments. This allows the trunk be to settle down in some region without specifying the desired posture. However, in some cases, the GRF pointing the VPP provides an upsetting moment, which would cause the trunk to fall. 
\begin{figure}[tb]
    \centering
%    \framebox{\parbox{3in}{ 
     \includegraphics[width=0.8\textwidth,height=0.85\textheight,keepaspectratio]{ICAS/fig2.png}
        \caption{The VPP model with the fixed VPP is not enough for the stable upright trunk. Although usually the GRF will provide restoring moment, in some cases the strategy will provide no restoring moment or even upsetting moment to the body. The graph shows that the region corresponds to $\tilde{\phi}\cdot \tilde{\beta}\geq0$. The variables $\tilde{\phi}$ and $\tilde{\beta}$ are depicted with the model in left.}
    \label{fig2: restoring and uprighting moment}
%    }}
\end{figure}
In order to provide mathematical criterion, we introduce two variables $\tilde{\phi}$ and $\tilde{\beta}$, pitch orientation from upright posture and the angle between GRF and a virtual line connecting point of action and COM, respectively, as in Fig. \ref{fig2: restoring and uprighting moment}. Then, it can be shown that the cases which collapse the trunk can be represented by the following condition,
\begin{align}
\tilde{\phi}\cdot\tilde{\beta} \geq 0,
\end{align} 
and this can be represented in the graph. 

For example, if we compare a posture in the first quadrant and the right-most posture of the fourth quadrant of the graph in Fig. \ref{fig2: restoring and uprighting moment}, in both postures the trunk is tilted in clock-wise direction and foot is anterior to the hip. However, if the controller tends to generate GRF towards a fixed VPP above the hip, one in the fourth quadrant would provide a moment in a counter-clockwise direction, which is a righting moment, whereas that in the first quadrant in a clockwise direction, which results in an upsetting moment. From this analysis, we end up with that the VPP model is not sufficient for stable and robust upright trunk walking, although the VP concept itself is fascinating, and the important factor is the direction of GRF to be always in the direction of righting moment.  

Experimental findings say that a single intersection point is not necessary for a VP system and cannot always be found \cite{maus2010upright}. Maybe that is because the single intersection point is \textit{insufficient}.

\subsection{Proposed Control Strategy}
\begin{table}[t]
\caption{Control parameters for the simulation}
\label{table:Control parameter}
\begin{center}
\begin{tabular}{|c|c|c|c|c|}
\hline
Parameter & Meaning  & Value [unit]\\
\hline
$r_{VPP}$ & distance between VPP and CoM & 0.1 [$m$]\\
$\alpha_0$ & fixed leg touchdown angle in sagittal plane & 110 [$deg$]\\
$c$ & position-proportional gain & 10[$\cdot$]\\
$d$ & velocity-proportional gain & 1 [$s$] \\
$\mu$ & VBLA parameter & 0.5 [$\cdot$]\\
\hline
\end{tabular}
\end{center}
\end{table}
Although the fixed VPP is not enough, the general argument of the VP concept is still fascinating, in that, the direction of force generated at foot is important in upright trunk walking. We just do not think that a single intersection point is a core of the VP concept. Therefore, we keep the format of the controller such that the hip torque redirect the force generated by the springy leg. The difference is how to define the angle $\beta$. 

Drawn upon the concepts, as shown in the above, a proper GRF redirecting controller should aim $\tilde{\phi}\cdot\tilde{\beta} < 0$, the region without shade in Fig. \ref{fig3: possible controller} (a). We would like to argue that if the generated GRF by a suitable redirecting controller satisfies this condition, any form of control would work. The simplest form we can propose is a redirecting controller having linear relationship as follows. 
\begin{align}
\label{eqn: new rule}
\tilde{\beta} = - c  \tilde{\phi}, \tilde{\phi} \in  [-\frac{\pi}{2}, \frac{\pi}{2}]
\end{align}
where $c$ is a positive gain which should be properly designed. 

It is worth noting that the speed and the direction of motion of the planar robot model should also be considered in determining the direction of the GRFs, we end up with the following control form for force direction control. 
\begin{equation}
\label{eqn: new rule vel}
\tilde{\beta} = - c  \tilde{\phi} - d\dot{\tilde{\phi}}, 
\end{equation}
where d is another positive gain, and it should be properly designed. Corresponding hip torques will be computed to redirect axial leg force to the desired direction with respect to CoM. The angle between the actual leg and the virtual line from foot to CoM can be computed from geometric information as well; denoting this angle as $\angle$, $\tan \beta = \tan (\tilde{\beta}+\angle)$ can be easily calculated.
\begin{figure}[tb]
    \centering
%    \framebox{\parbox{3in}{ 
    \includegraphics[width=0.8\textwidth,height=0.8\textheight,keepaspectratio]{ICAS/fig3.png}
        \caption{(a) Based on the analysis, we draw a basic rule for the direction of the GRF. $\tilde{\beta}$ can be any function of $\tilde{\phi}$ within 'Righting moment' region (dashed), but the simplest linear relationship (solid) is presented in this figure. (b) The feasible direction of the GRF is depicted (solid arc). Region A represents that the GRF redirection is accomplished using tangential force from the hip moment. Region B represents an arbitrary region that the user would define. One possible candidate is an estimated friction cone as depicted in the figure. }
    \label{fig3: possible controller}
%    }}
\end{figure}

\subsection{Feasible Direction of Foot-Ground Interaction Force}
The relation given in (\ref{eqn: new rule}) is expressed in terms of angles, therefore additional care is required for the controller. First of all, as the hip torque generates tangential force in $\tan \beta$, $\beta \in (-\frac{\pi}{2}, \frac{\pi}{2})$. if not, the direction of tangential force will be opposite the desired one. This is represented by the region A in Fig. \ref{fig3: possible controller} (b). Region A can be represented mathematically as follows
\begin{align}
A := \{\beta \in \mathbb{R} | -\frac{\pi}{2}<\beta < \frac{\pi}{2} \}
\end{align}

 Another thing we can consider is the range of the force in absolute reference. For example, we can consider friction cone. Although exact friction cone is hard to compute and thus not must thing, it improved the simulation result a lot. One can estimate the friction coefficient $\hat{\mu}$ and restraint controller generating force beyond friction cone. In this case, region B can be represented mathematically as follows. 
\begin{align}
B:=\{ \beta \in \mathbb{R}| \frac{\pi}{2}-\arctan\hat{\mu}<\alpha - \beta < \frac{\pi}{2}+\arctan\hat{\mu} \}
\end{align}
 
Even though the friction cone is neglected, at least we should consider that the GRF should be unilateral force,  \textit{i.e.}, the ground cannot pull the robot. Therefore, at least the region B should satisfy the follows.
\begin{align}
B:=\{ \beta \in \mathbb{R}| 0<\alpha - \beta < \pi \}
\end{align}

The intersection of the region A and B represents the feasible direction that the GRF can take, and that the controller would generate, $A \cap B $. 
\subsection{Foot Placement}
We adopt the velocity-based leg adjustment (VBLA) for swing leg control in order for regulation of linear momentum, which has been shown to be effective in enhancing stability and robustness of a point-mass walking mode \cite{sharbafi2016vbla}. The VBLA determines the desired swing leg direction $\textbf{O}$ as follows,
\begin{equation}
\textbf{O} = \mu \textbf{V} + (1-\mu)\textbf{G},
\end{equation}
where $\textbf{V} = \dfrac{v}{\sqrt[]{gL_0}}$ and $\textbf{G} = \dfrac{\textbf{g}}{g}$ are non-dimensional CoM velocity and gravitational acceleration. $\mu$ is the VBLA parameter to be determined properly. Swing leg touchdown angle $\alpha_0$ is obtained from the angle of the vector $\text{O}$. 
\subsection{Simulation Result of the Force Direction Controller}
\begin{figure}[ht!]
    \centering
    \includegraphics[width=1\textwidth,height=1.8\textheight,keepaspectratio]{ICAS/ICAS_4.png}
    \caption{ Simulation result with a planar bipedal walking model with the proposed method on its sagittal plane. The response of the system with disturbance forces ((a) and (b)), and with rough terrain ((c) and (d), the maximum height of terrains \(h_{max} = 2cm\)). The phases plot (c) and (d) are changed color when the rough terrain is diminished at \(t = 10s\), while in (a), (b), the phases plot are changed color when external forces is applied at \(t=1[s]\) and $t=5[s]$.}
    \label{FDC}
\end{figure}
The dynamics of the bipedal walking model with trunk and compliant legs is simulated for 20 seconds, of which control parameters are listed in Table II. In Fig. \ref{FDC}, the phases plot of the center of mass vertical (y) motion and pitch ($\phi$) of the model are presented under disturbances forces ( (a) and (b)) and under terrain surface ((c) and (d)). In Fig. \ref{FDC} (a) and (b), external force disturbance of $F_{dist} = [30,100]^T[N]$ is applied to the right foot of the model during 0.3 seconds. The proposed method quickly stabilize the motion of pitch to its steady state motion, whereas the motion of CoM vertical oscillates in a small range. The method rejects this errors quite good in pitch motion. When the robot walks on the height map,  Fig. \ref{FDC} (c) and (d) shows the similar trend, The pitch motion is immediately stabilized, and at the same time, the translational motion is indirectly stabilized by the self-stabilizing property of a mass-spring walking system, as shown in the figure with solid red lines. When comparing with previous the combined control strategy, the force direction controller can give as much as performance without any precomputed trajectories and optimal control (DLQR). Therefore, we hope that method can open the door to build and control a real bipedal robotic platform based on the simple model and force direction controller. In the next chapter, we aim to apply the model to rigid-body articulated robotic model and develop a detailed control algorithm.
\section{Summary}
This chapter focused on control strategies of the BTSLIP model and provided two solutions for small terrain ground and disturbance forces in walking motion. 

Firstly, we proposed the combined control strategy by implementing the LQR control for adjusting the VPP location and feedback linearization control for legs stiffness. These results show that with the adaptability of the VPP position and the adjusting leg stiffness not only improves the stability but also for high robustness and fast disturbance rejection. 

Secondly, nature seems to take advantage of the attractive properties of mechanical templates and to facilitate stable gait without the need for precise trajectory planning. Therefore, without precomputed trajectories, We propose a force direction control method to regulate trunk motion while walking. We validate that three components of controls including spring leg, proposed force direction control, and proper foot placement realize robust walking of a reduced order model with respect to force disturbances. 

In the next chapter, we will create the controller for the articulated based on the BTSLIP model and second control method. 