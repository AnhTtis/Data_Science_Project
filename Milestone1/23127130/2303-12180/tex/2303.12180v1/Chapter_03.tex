\chapter{Control Strategy for the 5-link robot model based on Task-Space Control}
\chaptermark{Hierarchical Control Strategy for the robot model}
\label{chapter:chapter03}
\section{Introduction}
In Chapter 2, we validate that three components of controls including spring leg, proposed force direction control, and proper foot placement realize robust walking of a reduced order model on force disturbances. In this chapter, we will show how to use force direction controller with the simple model to ``guide'' the 5-link robot model by hierarchical control.  In general, hierarchical control strategy consists of two layers. In the first layer (higher level controller), a set of reference dynamic features (we used the leg forces profile) is generated from the simple template model based on certain control laws or optimization criteria. Then the second layer (lower level controller), these dynamic reference features are tracked by task space controller. On the other words, the 5-link robot model is controlled to have two virtual springy legs similar to the BTSLIP model. The desired GRF is computed to stabilize the trunk based on the force direction control. Besides, the proportional-derivative
The controller is applied to the virtual spring of swing leg. Then, we use the task space control to drive the 5-link robot to follow these sets of the target. 


There are different methods to solve the task-space control (Operation Space Control - OSC). Khatib \cite{khatib1987unified} first proposed the operation space formulation to compute joint space actuator torques while considering task space dynamics. Later, Sentis and Khatib improved the OSC to deal with the underactuated and constrained dynamics of a humanoid robot \cite{sentis2005control}. Park and Khatib considered the control of contact forces of humanoid robots using OSC in \cite{park2006contact}. On the other hand, Aghili \cite{aghili2005unified} presented a method to simplify the computation of both direct and inverse dynamics for constrained systems using orthogonal projection, which is exploited by Mistry and Righetti \cite{mistry2012operational} who applied the method for the control of legged robots. Pratt et al. \cite{pratt2001virtual} proposed the intuitive control framework named as a Virtual Model control. In this structure, he used the simple Jacobian transpose to map the forces of virtual component (springs) into joint torques, this method neglect the Coriolis forces of the system dynamics. In this chapter, we present two approaches to task space control. In the first proposed task space control, we use the OSC formulation from \cite{mistry2012operational} for constrained and underactuated system to compute the joint torques of the robot model. The second method of task space control is from the simple Jacobian Transpose \cite{pratt2001virtual}. In this method, we do not intend to force the dynamics of the five-link robot model to exactly follow those of the reduced order model. The goal is rather to exploit the fundamental properties verified with a reduced order model to control a system with substantially more complex dynamics. In this perspective, we show that a simple form of the controller can still show adequate performance for walking, without using model-based controls or precomputed trajectory. 

\begin{figure}[ht]
    \centering
    \includegraphics[width=0.8\textwidth,height=0.9\textheight,keepaspectratio]{Figure/block_diagram.PNG}
    \caption{Block diagram of the simple-model-guided task space control system. We create virtual springs between the hip joint and foot-end of two legs, as utilized in the simple models. As in the simple virtual model, each leg controller has swing and stance mode which correspondences to the force direction controller and swing controller in layer 1. The controller in Layer 1 proposed forces at foot-end, which is converted into joint torques via Jacobian Transpose or Operation Space Control. In the simulation, we also test the walking motion under disturbance forces and rough terrain.}
    \label{fig:blockdiagram}
\end{figure}

Fig. \ref{fig:blockdiagram} provides an overview of the ``simple-model-guided task space control framework". The details in this Figure will be discussed in more detail in the following sections.

The remainder of the chapter is organized as follows. In Section 3.2, we provide brief descriptions of the robot model. Section 3.3 presents the control strategy for the robot model. Section 4 follows with simulation results which demonstrate the effectiveness of the control strategy. Finally, we conclude the paper with final remarks section 5.
\begin{table}[h]
\normalsize
\caption{Model parameters}
\label{table_example}
\begin{center}
\begin{tabular}{| m{3cm} | m{1.1cm}| m{1.1cm} | m{1.1cm} |}
\hline    
Model parameters & Trunk (t) & Femurs (f) & Shins (s) \\
\hline
\(m_{*}\) (\(kg\)) & 12.5 & 0.7 & 0.7  \\
\(L_{*}\)(\(m\)) & 0.42 & 0.19 & 0.19\\
\(J_*\) (\(kg m^2\)) & 0.23 & 0.0045 & 0.0045 \\
\(c_*\)(\(m\)) & 0.21 & 0.13 & 0.13 \\
\hline
\end{tabular}
\end{center}
\end{table}
\section{Robot model}
The robot model in this study  consists of five links as depicted in Fig. \ref{fig:first}: a trunk and two identical lower limbs with each limb having a femur and a shin; moreover, all links have mass and are connected by revolute joints. All of the joints are considered only rotating in the sagittal plane. The walking cycle consists of two phases:  the single support (SS, the phase in which only the stance leg is touching the ground) and the double support (DS,  both legs are in contact with the ground). The multi-rigid-body contact model of \cite{hurmuzlu1994rigid} is assumed as the underactuated revolute joint with constraints of uni-laterality (no lift-off) and no slipping between leg and walking surface in SS. That collapses the double support phase to an instant time (impacts) and allows a discontinuity in the velocity of states. 

To describe the robot system,  we use the Lagrangian formulation and the assumption of rigid bodies. The model parameters are given in Table I. 
  
\begin{figure}[ht]
    \centering
    \includegraphics[width=0.8\textwidth,height=0.9\textheight,keepaspectratio]{ICMA_OSC/Figure3}
    \caption{The robot model parameters (a). The robot model coordinates follow the convention such that angles and torques are defined positive in counter-clockwise direction (b). \(q_5\) is the absolute trunk angle. The position of the floating trunk are denoted as \([x_c,y_c]\)). \(\) in Cartesian coordinates. }
    \label{fig:first}
\end{figure}
\subsection{Single Support}
In the single support, legs of the robot are marked as stance and swing. With 5-link, the dynamic model consists of seven DOF, where four actuated DOF associated with the joint coordinates, two underactuated DOF associated with the horizontal and the vertical displacements of the Center of Mass (CoM), and last one underactuated DOF associated with the orientation of the robot in a sagittal plane. Thus, the generalized coordinates of the system (\(q_e\)) can be combined by two subsets \(q\) and \(r\).
\begin{equation*}
\label{eqn: coordinates}
q_e := [q,r]^T 
\end{equation*}
where \(q:= [q_1,q_2,q_3,q_4,q_5]\) encapsulates the joint coordinates, \(q_5\) is the underactuated DOF which is associated with orientation of the robot. $q_1$ and $q_3$ describe the configuration of the swing leg, while $q_2$ and $q_4$ are for the stance leg. Besides, \(r:=[x_c,y_c] \in \mathbb{R}^2\) is the Cartesian coordinates of the CoM of robot. The second order dynamical model follows from Lagrange's equation \cite{goldstein1965classical}, \cite{spong2006robot}.
\begin{equation}
\label{eqn:EOM}
\begin{aligned}
& D_e(q_e)\ddot{q}_e + C(q_e,\dot{q_e}) + G_e(q_e) = B_e \tau + J_e^{st}(q_e)^TF^{st} \\
\end{aligned}
\end{equation}
where \(D_e(q_e) \in \mathbb{R}^{7 \times 7}\) is the symmetric and positive definite inertia matrix, \(C(q_e,\dot{q}_e) \in \mathbb{R}^7\) summarizes the centrifugal and Coriolis terms, \(G(q) \in \mathbb{R}^7\) is the vector containing gravity terms. \(B_e \in \mathbb{R}^{7 \times 7}\) is the input matrix (i.e. \(B_e = \begin{bmatrix}
    I_{4\times 4}   & 0   \\
   0 & 0      
\end{bmatrix}
\) with \(I_{4\times 4}\) is the identity matrix) and \(\tau:= [u_1,...,u_4,0,0,0]^T \in \mathbb{R}^7\) includes the joint torques applied at the joints of the robot. \(J_e^{st} := \dfrac{\partial p^{st}}{\partial q_e} \in \mathbb{R}^{2 \times 7}\), \(p^{st}\) is the Cartesian coordinates of stance foot. Also,  \(F^{st} := [F_x^{st},F_y^{st}] \in \mathbb{R}^2\) is constraint force which corresponds to  the ground reaction force at the stance leg end.

We assume that a stance foot on the ground is only allowed to perform a pure rolling motion with neither slipping nor sliding. This means we have the constraint below:
\begin{equation}
\label{eqn:constraint}
\begin{aligned}
& \dot{p}^{st} = J_e^{st}\dot{q}_e = 0 \\
& \ddot{p}^{st} = J_e^{st}\ddot{q}_e + \dot{J}_e^{st}\dot{q}_e = 0
\end{aligned}
\end{equation}
\subsection{Impact}
An impact takes place when the swing leg touches the ground as a collision between two rigid bodies \cite{hurmuzlu1994rigid}, \cite{hyun2014high}. The impact is assumed to be inelastic and instantaneous. When the impact occurs, the leg previously pinned on the ground (i.e., the stance leg) loses contact, and the roles of legs are switched. Then, the impact model which meets the standard hypotheses in \cite{grizzle2001asymptotically}, results in the impact equation which can be expressed by the following equation: 
\begin{equation}
\label{eqn:impact}
\begin{aligned}
& D_e(q_e)\dot{q}_e(t^{+}) - D_e(q_e)\dot{q_e}(t^{-}) = J_e^{sw}(q_e)^T \delta F^{sw} \\
& J_e^{sw}(q_e)\dot{q}_e(t^+) = 0
\end{aligned}
\end{equation}
where \(J_e^{sw} := \dfrac{\partial p^{sw}}{\partial q_e} \in \mathbb{R}^{2 \times 7}\), \(p^{sw}\) is the Cartesian coordinates of swing foot and  \(\delta F^{sw}:= \int_{t^-}^{t^+} F^{sw}(\tau)d\tau \) is the impulsive force at impact point. The superscript \(+\) and \(-\) denotes the the post-impact and pre-impact of system, respectively. The solution of (\ref{eqn:impact}) can be more clearly given by the following equation: 
\begin{equation}
\label{eqn:sort}
\begin{bmatrix}
    \dot{q}_e(t^+)      \\
    \delta F^{sw}      
\end{bmatrix}
= \Delta(q_e)
\begin{bmatrix}
    D_e(q_e)\dot{q}_e(t^-)      \\
    0      
\end{bmatrix} 
\end{equation}
where 
\(
\Delta(q_e)
= 
\begin{bmatrix}
    D_e(q_e) & -J_e^{sw}(q_e)^T      \\
    J_e^{sw}(q_e) & 0      
\end{bmatrix} ^{-1}
\)
. Instead of introducing additional equations of motion for the next single support with new stance leg, the transform of the coordinates of the robot is done by relabeling matrix \(R \in \mathbb{R}^{7 \times 7}\). Formally, the combination with  (\ref{eqn:sort}),  the state of the system after impact can be expressed as:
\begin{equation}
\label{eqn:finalrigid}
\begin{aligned}
& q_e(t^+) = Rq_e(t^-) \\
& \dot{q}_e(t^+) = R\Delta_{11}(q_e(t^-))\dot{q}_e(t^-),
\end{aligned}
\end{equation}
where $\Delta_{11}$ is the first \(7 \times 7\) matrix of \(\Delta(q_e)\)
\subsection{Controller Modification for the Planar Five Link Model} 
In this section, the control strategy for robot model is described. We first derive desired forces to be exerted at foot of the model for both stance and swing legs. In particular, the desired force of the stance leg is determined based on simple models; a virtual spring is created to connect hip and foot of the robot model, as in the Bipedal Trunk Spring Loaded Inverted Pendulum (BTSLIP) model. A proper force direction control rule is adopted from \cite{lee2017biped} in order for bipedal walking with upright trunk. Besides, simple PD control is implemented for swing leg. Finally, the operational space control (OSC) is used to map the desired force to joint actuators. The control parameters are presented in Table \ref{table:control}.
\subsubsection{The stance leg control} 
\begin{table}[h!]
\caption{Control parameters for the the 5-link robot model}
\label{table:control}
\begin{center}
\begin{tabular}{|c|c|c|c|c|}
\hline
Parameter & Meaning  & Value [unit]\\
\hline
$k$ & spring stiffness for virtual leg & 7500 [$N/m$]\\
$c$ & position-proportional gain & 10 [$\cdot$] \\
$c_{sw}$ & position-proportional gain & 10 [$\cdot$] \\
$d$ & velocity-proportional gain  & 1 [$s$]\\
$\mu$ & coefficient value for VBLA & 0.5 [.] \\
$L_0$ & the rest length of virtual leg & 0.37 [$m$]\\ 
$k_d$ & virtual spring damping &100 [$N.s/m$]\\
\hline
\end{tabular}
\end{center}
\end{table}
\begin{figure}[ht]
    \centering
    \includegraphics[width=0.9\textwidth,height=0.85\textheight,keepaspectratio]{ICMA_OSC/Figure5}
    \caption{The robot model is considered to have virtual compliant legs. Two virtual spring is connected from hip to each foot with the same spring stiffness, \(k\). The control variables for swing leg (a). And the control variables for stance leg (b). Velocity-based leg adjustment \cite{sharbafi2016vbla} (c).}    
\label{fig:control}
\end{figure}
Fig. \ref{fig:first} shows that the 5-link model will have virtual springs from hip to each foot. We embedded the controller of the BTSLIP model \cite{maus2010upright}, \cite{lee2017biped} in order to calculate the virtual axial force and tangential force that is necessary to maintain the upright trunk in walking for the stance leg:
\begin{equation}
\begin{aligned}
&F_r^{st} = k (L_0 - L^{st}) + k_d\dot{L}^{st}\\
&F_t^{st} = F_r \tan \beta
\end{aligned}
\end{equation}
where \(k\) and \(L_0\) are the spring stiffness and the rest length of virtual leg, respectively.  \(L^{st} = | p^{st}_f - p_h|\) is current leg length, where \(p^{st}\) and \(p_h\) are the position of stance feet and hip in global frame, respectively. $\beta = \eta + \tilde{\beta}$ is the angle between \(\textbf{F}_{GRF}\) and the virtual leg, and $\eta$ is the angle between the vector from stance foot to CoM and virtual leg. We added the damping component in computing of the axial force. Note that we set up the lower limit of \(F_r\) to zero. 

The ground reaction force vector \(\textbf{F}_{GRF}\) is the combination of the virtual spring force and reaction force : $\textbf{F}_{GRF} = \textbf{F}_r^{st} + \textbf{F}_t^{st}$. The direction of $\textbf{F}_{GRF}$ is controlled by $\tilde{\beta}$. We take into account the speed and direction of the trunk in determining the direction of \(\textbf{F}_{GRF}\) as follows: 
\begin{equation}
\begin{aligned}
&\tilde{\beta} = -c\tilde{\phi} - d\dot{\tilde{\phi}},
\end{aligned}
\end{equation}
where $c$ and $d$ are the control parameters, as shown in Table \ref{table:control}. $\tilde{\phi}$ is the pitch of the trunk.
Besides, the desired GRF vector is decomposed into horizontal \(F_x^{st}\) and vertical component \(F_y^{st}\) in the global frame using an angle \(\theta_{p}^{st}\)(\(= \alpha^{st} - \beta\)): 
\begin{equation}
\begin{bmatrix}
     F_x^{st}      \\
     F_y^{st}      
\end{bmatrix}
= \begin{bmatrix}
    F_{GRF}\sin(\theta_{p}^{st})     \\
    F_{GRF}\cos(\theta_{p}^{st})   
\end{bmatrix} ,
\end{equation} 
where $F_{GRF} = ||\textbf{F}_{GRF}||$ is the magnitude of the GRF vector.
\subsubsection{The swing leg control} 
We use the method velocity-based leg adjustment (VBLA \cite{sharbafi2016vbla}) for swing leg control, to define the touchdown angle for swing leg, as shown in Fig. \ref{fig:control}(c). The VBLA determines desired swing leg orientation as follows, 
\begin{equation}
\textbf{O} = \mu \textbf{V} + (1-\mu)\textbf{G},
\end{equation}
where \(\textbf{V} = \dfrac{\textbf{v}}{\sqrt{gL_{sw}}}\) and \(\textbf{G} = \dfrac{\textbf{g}}{|g|}\) are non-dimensionalized CoM velocity and gravitational acceleration, respectively. $\textbf{v}$ and $\textbf{g}$ are the vector of CoM velocity and gravity. The angle between the desired vector \(\textbf{O}\) and the ground is determined as the desired touch down angle \(\alpha_d^{sw}\). By defining the virtual leg length trajectory for swing leg \(L_d\), and combining with the desired touch down angle from VBLA, the simple form of proportional - derivative control can be applied for swing leg control: 
\begin{equation}
\begin{aligned}
& F_r^{sw} = k(L_d - L^{sw}) + k_d \dot{L}^{sw} \\
& \tau_{sw} = c_{sw}(\alpha_d^{sw} - \alpha^{sw}) + \dot{\theta}^{sw},
\end{aligned}
\end{equation}
where \(L_{sw} = |p^{sw}_f - p_h|\) is the length of the virtual swing leg, \(p^{sw}_f\) is the position of the swing foot in global frame, and  \(\alpha^{sw}\) is the angle created by the virtual swing leg and the horizontal axis. \(c, k_d\) and \(c_{sw}\) are the control parameters defined in Table \ref{table:control}. The virtual leg length can be retracted to prevent the scuffing of leg with the walking surface by controlling the desired swing leg length \(L_d\): 
\begin{equation}
\begin{aligned}
\left \{
\begin{array}{ll}
 L_d = \dfrac{4}{5} L_0 \: \text{if} \: -\dfrac{\pi}{9}\leq (\alpha^{sw} - \dfrac{\pi}{2}) \leq \dfrac{\pi}{18} \\
 L_d = L_0 \: \: \: \: \text{if} \:\: (\alpha^{sw} - \dfrac{\pi}{2}) > \dfrac{\pi}{18} 
 \end{array}
\right.
\end{aligned}
\end{equation}
 By applying the torque \(\tau_{sw}\), it is equivalent to create the tangential force along with the virtual swing leg \(F_t^{sw} = \dfrac{\tau_{sw}}{L_{sw}}\), as shown in  Fig. \ref{fig:control}(b). The combination of tangential force and normal force creates the total desired force at end-effector of swing leg: \(\textbf{F}_{sw} = \textbf{F}_t^{sw} + \textbf{F}_r^{sw}\). Also, the end-effector force vector of swing leg is projected in the global frame with angle \(\theta^{sw}_{p}\):
\begin{equation}
\begin{aligned}
\begin{bmatrix}
     F_x^{sw}      \\
     F_y^{sw}      
\end{bmatrix}
= \begin{bmatrix}
    F^{sw}\sin(\theta_{p}^{sw})     \\
    F^{sw}\cos(\theta_{p}^{sw})   
\end{bmatrix} 
\end{aligned},
\end{equation}
where $F^{sw} = ||\textbf{F}^{sw}||$ and \(\theta^{sw}\) is the angle between \(\textbf{F}_{sw}\) and horizontal line. 
\subsection{Mapping the desired end-effector force with the actual joint torques of robot}
\subsubsection{Operational Space Control for Underactuated and Constrained system}
We consider that \(J^{st}\) and \(J^{sw}\) are the Jacobian of the vector along with the virtual swing leg and stance leg in the global frame, respectively:
\begin{equation}
\begin{aligned}
& J^{st} = \dfrac{\partial(p_h - p^{st}_f) }{\partial q_e}; \:
& J^{sw} = \dfrac{\partial(p_h - p^{sw}_f) }{\partial q_e},
\end{aligned}
\end{equation}
where \(J^{st} \in \mathbb{R}^{2 \times 7}\) and \(J^{sw} \in \mathbb{R}^{2 \times 7}\). In more compact form, we assign \(J = [J^{st};J^{sw}]\) and \(F = [F_x^{st},F_y^{st},F_x^{sw},F_y^{sw}]^T\). 

In order to compute joint torques, we adopt the OSC introduced in \cite{mistry2012operational} which is formulated for constrained and underactuated system. This method exploits the null space motion and constraint forces of the system to accomplish the tasks in operational space while minimizing the magnitude of null-space torques. The control inputs can be computed as
\begin{equation}
\label{constrain} 
\tau = J^T F + N \tau_0,
\end{equation}
where \(N = I - J^T J^{T^\#}\), \(J^{T^\#} = (JM_cPJ^T)^{-1}M_c^{-1}P\),  and \(\tau_0\) is an arbitrary null space torque. P is an orthogonal projection operator which is readily computable from the constraint Jacobian \(P = I - (J_e^{st})^+ J_e^{st}\) \cite{aghili2005unified} (+ indicates the Moore-Penrose pseudoinverse) and \(M_c = P D_e + I -P\) is invertible.

In order for minimizing the value of \(||\tau_0||\) (Euclidean norm of vector \(\tau_0\)), the null-space torque $\tau_0$ is formulated as 
\begin{equation}
\label{tau0}
\tau_0 = -[(I - B_e)N]^+ (I-B_e)J^T F,
\end{equation}
where \(I\) is the identity matrix. By substituting (\ref{tau0}) into (\ref{constrain}), we can  write the control equation as in \cite{mistry2012operational} 
\begin{equation}
\label{20}
\tau = (I-N[(I-B_e)N]^+)J^T F.
\end{equation}
\subsubsection{Polar Jacobian Transpose method}
\begin{figure}[ht]
    \centering
    \includegraphics[width=0.6\textwidth,height=0.7\textheight,keepaspectratio]{Figure/polar.PNG}
    \caption{Polar coordinate transformation for stance leg. The similarity will be applied for the estimation in swing leg.}    
\label{fig:control}
\end{figure}
Fig. shows the transformation between generalized coordinate of stance leg to polar coordinate, the similar calculation is applied for the swing leg. The stance leg control and swing leg control measure the forces at foot end with respect to a hip as $F_{polar} = [F_r^{st},F_t^{st},F_r^{sw},F_t^{sw}]^T$. The Jacobian from hip to foot end of each leg is transformed to polar coordinate and multiplied with the set of forces at foot end to calculate torque commands for each joint as: 
\begin{equation}
\tau=J^T_{polar}F_{polar}
\end{equation}
\subsection{Simulation results}
\begin{figure*}[h]
    \centering
    \includegraphics[width=\textwidth,height=6.5cm]{ICMA_OSC/Figure6}
    \caption{Simulation result of the planar 5-link robot model walking 100 steps with proposed control method. The model begins with a random initial condition and converges to its periodic trajectory within ten steps. The phase portrait of an angle from trunk to stance leg (a) and to swing leg (b). The phase plot of stance knee angle (c) and swing knee angle (d). The phase plot of absolute trunk angle (e).} 
\label{fig:result1}
\end{figure*}
To test the validity of the proposed control method with the 5-link rigid model, we use the dynamic simulation. The model parameters and control parameters are listed in Table 3.1 and Table 3.2, respectively. The 5-link robot dynamics and the proposed controller are illustrated in Section 3.2. To be a more realistic simulation, we include a small amount of viscous friction at the joints. 
\subsubsection{The simulation results of Operational Space Control}
We use ode45 integrator in Matlab to create the dynamics of 5-links biped robot with impact assumptions in Section 3.2.2. 
To test the validity of the proposed control method with the 5-link rigid model, we use the dynamic simulation. The model parameters and control parameters are listed in Table I and Table II, respectively. The 5-link robot dynamics are illustrated in Section II, and the proposed control is presented in Section III. Besides, we include a small amount of viscous friction at the joints. 
\begin{figure}[ht!]
    \centering
    \includegraphics[width=0.8\textwidth,height=0.7\textheight,keepaspectratio]{ICMA_OSC/Figure8}
    \caption{ The control inputs at hip of stance ($u_1$) and swing leg ($u_3$), respectively in (a) and the control inputs at knee of stance ($u_2$) and swing ($u_4$) in (b) over initial few steps (2.5 seconds).}  
\label{res:control}
\end{figure}

Fig. \ref{fig:result1} presents all the phase plots of the joint coordinates \(q\) with the initial condition 
\begin{equation*}
q_e=[3.7,2.6,-0.5,-0.4,-0.08,-2.4,-0.1,-1.3,-0.9,0.5]^T, 
\end{equation*}

(units are omitted). Note that the walking motion starts with the random initial condition, thus at the beginning of the simulation when time $t<2$, it has a lot of jerky motion, but after some steps, the proposed method quickly stabilizes the model to the steady-state motion. The absolute trunk angle is maintained the upright posture with small oscillation \([-0.095:-0.085]\)[rad]. Furthermore, after approximately 5 seconds, the model converges to its periodic trajectory. 

\begin{figure}[ht]
    \centering
    \includegraphics[width=\textwidth,height=7.5cm,keepaspectratio]{Figure/Figure10}
    \caption{ The external  force disturbance is applied at the stance foot (a). The phases portrait of trunk angle (b), the horizontal position (c), and the vertical position (d) with the same initial condition with Fig. \ref{fig:result1}, but under the disturbances. The phases plot are changed color to green and red after the disturbances is applied at \(t = 5s\) and \(t = 10s\).}
\label{fig:pert}
\end{figure}
It can be seen in Fig. \ref{res:control}  that the control signals oscillate with large range at the very first step. After that, the steady state behavior of control inputs is attained with the range of hip torques control is about \([-15,11]^T\)[Nm] and \([-5,30]^T\)[Nm] for the knee torques. 

To further investigate the performance of the proposed control, the external force disturbance (\(F_{pert} = [5 \: 10]^T\)) is applied to the stance foot, as seen in Fig. \ref{fig:pert} (a). Fig. \ref{fig:pert} (b), (c), and (d) show the phase plots of absolute trunk angle ($q_5$), CoM horizontal($x_{CoM}$) and vertical position($y_{CoM}$). The force disturbances can affect the horizontal, vertical motion, and the rotation of the trunk. It is obvious that the proposed control is effective and induces gait stability in the 5-link robot model. The trunk is quickly stabilized after applying perturbation; the effects of disturbance are also rejected in horizontal and vertical motion. 
\begin{figure}[h!]
    \centering
    \includegraphics[width=0.65\textwidth,height=0.5\textheight,keepaspectratio]{ICMA_OSC/Figure9}
    \caption{The ratio of two components of actual GRF (tangential force and normal force)}  
\label{fig:compare}
\end{figure}
Fig. \ref{fig:compare} shows that the assumptions we made for the simulation is not violated. In this figure, the ratio of the horizontal and vertical components of the actual GRF (\(^aF^{st} = [^aF_x^{st},^aF_y^{st}]^T\) - the superscript `a' means actual), which is a paramount notion of slippage in walking, are plotted for a few initial steps. By assuming the Column friction model, \(|^aF^{st}_{x}/^aF^{st}_y| \leq \mu_{fric}\) will guarantee the non-slippery condition of stance leg. It is notable that the model would not slip while walking in the environment with \(\mu_{fric} \geq 0.4 \), as presented in Fig. \ref{fig:compare}. 
\subsubsection{The simulation results with Polar Jacobian Transpose}
We use the Open Dynamic Engine platform which a physics simulation engine to test the performance of proposed control. To avoid the burden of computation cost and simplify the control strategy, we use the classical method Jacobian Transpose to compute the joint torques. 

To create the rough terrain in ODE, we use height map terrains function in ODE. The height level of each point in ground terrain map will be defined by formulation: $z(cm) = 1 + 5\sin(20y+20x)$, where $x$, $y$ and $z$ are the location of each point on the terrain map. More clearly, the height of each point in terrain map will be in the range $[0:6]cm$ which correspondences to $20\%$ of model leg length. The snapshots of the robot model on rough terrain are shown in Fig. \ref{fig:snapshots} in $10[s]$ with interval $\delta t = 0.001$. 

\begin{figure*}[h!]
    \centering
    \includegraphics[width=11cm,height=7.5cm]{Figure/snapshots.jpg}
    \caption{Snapshots of several steps in approximately 10 seconds on terrain ground.} 
\label{fig:snapshots}
\end{figure*}
\begin{figure*}[h!]
    \centering
    \includegraphics[width=\textwidth,height=5cm]{Figure/phaseRT.png}
    \caption{The model begins walking on the flat ground and quickly moves to the rough terrain. The phase portrait of an angle from trunk to the thigh of the leg 1 (a) and leg 2 (b). The phase plot of stance knee angle of the leg 1 (c) and the knee angle of the leg 2 (d). The
phase plot of absolute trunk angle (e). The phases plot are changed to red color when robot is in the rough terrain surface.} 
\label{fig:roughterrain_phaseplot}
\end{figure*}
\begin{figure}[h!]
    \centering
    \includegraphics[width=\textwidth,height=0.4\textheight,keepaspectratio]{Figure/Traj_RT.png}
    \caption{ Trajectories in 30 seconds of the 5-link robot model. The vertical position of the CoM (a). The velocity of vertical position of CoM (b). The forward velocity of the robot model (c). The pitch of trunk is plotted in (d)}
\label{fig:trajectory_terrin}
\end{figure}
\begin{figure}[h!]
    \centering
    \includegraphics[width=\textwidth,height=0.35\textheight,keepaspectratio]{Figure/GRF.png}
    \caption{The measured ground reaction force profiles of the robot during walking. The forces can be very similar to the double peaks pattern of human ground reaction force which suggests that humans control their legs to act as linear springs during steady state walking and running.}
\label{fig:GRF}
\end{figure}
The graph of Fig. \ref{fig:roughterrain_phaseplot} shows the phase plots of joints angle. We have found that the controller constructs a stable limit cycle when walking on flat ground and although it loses periodicity, the robot model can maintain walking on rough ground as well. More clearly, in the graph, the jerk appears when robot walks on the rough terrain and quickly diminishes in flat ground. Fig. \ref{fig:trajectory_terrin} shows the trajectories of some states in 30 seconds. It is obvious that when the robot is on the rough surface, the springy leg tried to change the height of CoM to adapt. In the graph, the trajectories of vertical position and forward velocity oscillates while the trunk orientation is stable with a small range of oscillation. It is interesting that the algorithm does not contain any explicit speed control mechanism, yet the speed is immediately stabilized when the robot comes to the flat surface. We guess that this is due to the natural system dynamics of legs (virtual spring and damper leg). The role of VBLA algorithm in swing phase is also important; the robot can choose the appropriate foot position to land with respect to the velocity of CoM.  

It worth mentioning about the measured ground reaction forces in Fig. \ref{fig:GRF}. The robot is free to follow its own dynamics in term of peak ground reaction forces which is similar with the force pattern of human walking or the virtual BTSLIP model. This correlation is a testament to how close the robot is to have the virtual BTSLIP dynamics. 
\subsection{Summary}
In this work, we presented the control strategy for the 5-link robot model using operational space control (OSC) and Polar Jacobian Transpose method. More precisely, swing and stance leg forces are designed based on our previous BTSLIP model control \cite{lee2017biped} and these forces are mapped to joint actuators using OSC  specifically formulated for constrained and underactuated system.  Then, the controller was validated in the dynamical simulation of the model walking on the flat ground and terrain ground, as the model converges to its periodic trajectory from random initial condition within a few steps. Moreover, the performance of the controller was further shown that the walking model was robust against the external force disturbances. The measured ground reaction forces show that developing machines such that they can enforce desired dynamics allow control strategies developed on widely researched fundamental locomotion models to be applied to articulated robots. In next chapter, we will verify how we can apply this control scheme onto the flat foot robot model.