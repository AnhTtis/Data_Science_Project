\chapter{Summary and Future Work}
\chaptermark{Summary and Future Work}
\label{chapter:chapter06}
\section{Summary}
This work presents an attempt to control the 2D dynamic walking of a humanoid robot over disturbance forces and rough surfaces based on the reducing ordered model (i.e., the Biped Trunk - SLIP model). The overall control approach is divided into two levels: (1) The high-level planner focuses on the performance of the most emphasized template model for a dynamic walking motion such as the force profile. By studying the stability in terms of robustness to external disturbance forces and the rough surface of the BTSLIP model integrated with the force direction controller, the main dynamic characteristics of this combination are captured. (2) The lower-lever controller, we study two methods in the field of task space controller, the simple Jacobian transpose formulation and the operational space controller. It is worth noting that the simple control method does not require a detailed whole-body model, but relies on the force control capabilities of the robot. The performance is not significantly different from that of operational space control, while the computational cost and complexity of the overall control strategy are significantly reduced. 

In the first chapter, we explore the possible control rules that can be applied to the BTSLIP model. We draw inspiration from the concept of a Virtual Pendulum, which is proposed as an intuitive posture stabilization strategy and successfully demonstrated on the compliant leg scheme. In the first chapter, we first propose a combined control strategy consisting of a discrete linear quadratic controller to adjust the position of the virtual pivot point (VPP) and a linearization controller to change the leg stiffness. In addition, quadratic programming is used to find the periodic solution for the walking model with the VP concept. As a result, the BTSLIP model is well equipped with the combined control strategy. Later, the force direction control is proposed for the BTSLIP model because we want to reduce the computational cost burden of the first method. 
Following the concept of the virtual pendulum, we argue that the direction of the ground reaction force is more important than the location of a VPP point. Interestingly, with this simple control, the walking model can achieve a robust walking motion with no falls. 

In the next chapters, we apply this simple control to the articulated body robot (e.g., the point foot model and the flat foot model). We make some changes to the force direction control and add the state-dependent control for the swing leg. Together with the force direction control to stabilize the walking motion, the state-based control for the swing leg can help the robot avoid rubbing on the ground and adapt to rough terrain based on the speed-based leg adaptation method. Then, the desired force profile is passed to the task space controller to calculate the joint moments of the walking robot. The final result shows that the point-foot model and the flat-foot model run robustly over rough surfaces and disturbance forces. Moreover, the resulting GRFs of the point-foot model are rich in human-like features, such as the double-tip GRF pattern. 
\section{Future work}
The development in this work has brought some clues for many future research directions. First, the change in swing leg control strategy can adjust and track the speed of bipedal walking, but it does not perfectly change the speed as close as the set point. We believe that both the adjustment of foot position in swing leg control mode and the control of leg length in stance leg control mode help to accomplish the task. On the other hand, we would improve the current balance control by researching some existing methods such as Capture Point, and Divergent Component Points in 3D. Although this is likely to be a motion planning-based control, we would find a way to improve our state-based control. Finally, we are also very interested in extending the work to 3D robot models. 
