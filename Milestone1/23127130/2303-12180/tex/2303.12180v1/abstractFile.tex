\textbf{} 
Controlling dynamic walking in humanoid robots remains a challenging problem. To reduce the complexity caused by the high number of degrees of freedom (DoFs) and the inadequate control of the target legs, a simple template model is often used as an intermediate solution. Even though a template model is usually simple, it can still capture the essential features of the dynamics of a walking motion. Control strategies are then developed to regulate the behavior of the model.  

In this work, the hierarchical control strategy of template-based control for a bipedal robot is described. 
Existing work in the literature has shown that a simple mass-spring model can describe the dynamic characteristics of bipedal locomotion in terms of ground reaction force (GRF) and center of mass (CoM) profile. 
To explain the mechanics of upright trunk walking, a control method based on the concept of the virtual pendulum (VP) was previously introduced. 
In this approach, the axial force of a compliant leg is redirected to a point, called the virtual pivot point (VPP), of a 2D biped robot, which is located above the CoM of the model, to generate a restoring moment for the trunk motion. 
The resulting behavior of the model would resemble a virtual pendulum rotating around this VPP, thus aiming for an upright trunk during walking. 
However, we recognized that in some cases this method generates a flip-over moment instead of a restoring moment, which affects the performance of the controller. 
Inspired by this analysis, we propose a new force redirecting method as a controller for robot walking. 
Then, these key features of the BTSLIP model with a simple force direction controller are mapped into the overall input torques of an articulated body robot via a task space controller. 
We consider a full dynamic simulation of a 2D articulated body robot to validate the performance of the proposed method under the random initial conditions and the presence of force disturbances and moderately rough surfaces. 
Moreover, with our control strategy, the robot achieves a stable walking motion while keeping its upper body upright without using optimization methods. 
Note that the generated ground reaction forces were monitored to have a similar pattern to those generated by humans. 
We hypothesize by taking the advantage of the properties of mechanical templates, also called the reduced-order model, this could enable stable gait for the full model robot without the need for precise path planning. 
Therefore, in this study, we aim to answer the question, "How far a biped robot can walk with a controller based on a reduced-order model?"



