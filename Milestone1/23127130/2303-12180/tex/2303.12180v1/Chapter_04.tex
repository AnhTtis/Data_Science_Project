\chapter{Finite State Machine and Hierarchical Control for the 7-link robot model}
\chaptermark{Hierarchical Control Strategy for the Flat Foot model}
\label{chapter:chapter04}
\section{Introduction}
In chapter 3, we have shown how to implement the hierarchical strategy for the point foot model consisting two control layer over uneven terrain and disturbance forces. In this chapter, we extend the point foot model to the flat foot model which weighs 51 kg and has 0.65-m-long legs. The robot is provided no information regarding where the change in height occurs and by how much. 
The hierarchical control strategy will be used for this robot model. By simplify the controller, we consider the virtual BTSLIP model by attaching spring from hip to ankle of each leg and decouple the ankle control with existing controller inherited from Chapter 3.  
In this chapter, the walking cycle is implemented using a Finite State Machine (FSM). By using the local information of leg, the mode control will be conducted by FSM. The 7-link robot model is controlled to have two virtual springy legs which are similar to the design in the BTSLIP model. The stance mode control will be built with appropriate modification from \cite{lee2017biped}. The proportional-derivative (PD) controller and Velocity Based Leg Adjustment are applied to the swing mode control. Additionally, ankle push-off phase of trailing limb during the step-to-step transition in walking will be activated by the decision from FSM. 
\begin{figure}[ht!]
    \centering
    \includegraphics[width=0.8\textwidth,height=0.9\textheight,keepaspectratio]{Figure/block_diagram_foot.PNG}
    \caption{Block diagram of the simple-model-guided task space control system. We combine the hierarchical structure from chapter 3 and the ankle control. Then, the Finite State Machine is designed that manages transitions among controllers. In the simulation, we also test the walking motion under disturbance forces and rough terrain}
    \label{fig:blockdiagram_foot}
\end{figure}
Fig. \ref{fig:blockdiagram_foot} provides an overview of the ``simple-model-guided task space control framework". The details in this figure will be discussed in more detail in the following sections.


The remainder of this chapter is organized as follows. We first present brief descriptions of the robot model and the ODE integrator in Section 4.2. The control strategy for the robot model is highlighted in Section 4.2. Section IV presents the simulation results as the performance of the control strategy. Finally, we conclude the paper with final remarks in this chapter. 
\begin{table}[h]
\normalsize
\caption{Model parameters}
\label{table_example}
\begin{center}
\begin{tabular}{| m{2cm} | m{1cm}| m{1cm} | m{1cm} |m{1cm}|}
\hline    
Model parameters & Trunk (t) & Femurs (f) & Shins (s)& Foot (a) \\
\hline
\(m_{*}\) (\(kg\)) & 27.13 & 3.94 & 2.38 & 2.86 \\ 
\(L_{*}\)(\(m\)) & 0.5 & 0.28 & 0.28 & 0.125\\
\(J_*\) (\(kg m^2\)) & 0.58 & 0.014 & 0.03 & 0.01 \\
\hline
\end{tabular}
\end{center}
\end{table}
\section{Robot model}
  
\begin{figure}[h!]
    \centering
    \includegraphics[width=\textwidth,height=0.9\textheight,keepaspectratio]{ICMA_foot/Figure3}
    \caption{The robot model parameters (a). The robot model coordinates follow the convention such that angles and torques are defined positive in counter-clockwise direction (b). \(q_9\) is the absolute trunk angle. The position of the floating trunk are denoted as \([x_c,y_c]\)). \(\) in Cartesian coordinates.The size of the biped robot in Open Dynamics Engine (c). This parameters of robot's size is adopted from the Mahru robot in 2D version.}
    \label{fig:first}
\end{figure}
The robot model in this paper consists of seven links as depicted in Fig. \ref{fig:first} a trunk and two identical lower limbs with each limb having a femur, a shin, and an ankle; moreover, all links have mass and are connected by hinge joints. The robot parameters are shown in Table. I. All of the joints are considered only rotating in the sagittal plane. With 7-link, the model in sagittal plane consists of 9 Degree of Freedom (DoF). Six actuated DOF associated with the joint coordinates. Two underactuated DOFs related to the horizontal and the vertical displacements of the Center of Mass (CoM). The last one underactuated DOF associated with the orientation of the robot in a sagittal plane. Thus, the generalized coordinates of the system ($q_e$) can be combined by two subsets q and r. 
\begin{equation*}
\label{eqn: coordinates}
q_e := [q,r]^T 
\end{equation*}
where \(q:= [q_1,q_2,q_3,q_4,q_5,q_6.q_9]\) encapsulates the joint coordinates, \(q_9\) is the underactuated DOF which associated with orientation of the robot. Besides, \(r:=[x_c,y_c] \in \mathbb{R}^2\) is the Cartesian coordinates of the CoM of robot. 
\begin{table}[h]
\normalsize
\caption{Open dynamics engine parameters}
\label{table ODE}
\begin{center}
\begin{tabular}{| m{6cm} | m{3cm} |}
\hline    
Name & Value \\
\hline
Time of Step & 0.001$[s]$ \\ 
Gravity & -9.81 $[m/s^2]$\\
CFM & $10^{-5}$ [$\cdot$] \\
ERP & 0.9 [$\cdot$]\\
Contact Surface Layer & 0.001 [$\cdot$]\\
ContactMaxCorrectingVel & 100 [$\cdot$]\\
\hline
\end{tabular}
\end{center}
\end{table}
We use Open Dynamic Engine (ODE) for simulating rigid body dynamics and deactivate the motion in Z plane to force the movement only in the sagittal plane. The ODE uses Newton-Euler approach to describe the system of bodies by a full set of coordinates, and set of constraining equations. In general, the state of each body part in ODE is described
by six variable $x = [p_x \: p_y \: p_z \: \theta_x \: \theta_y \: \theta_z]^T$ containing the position and orientation of the body, and an additional function h(x) that describes the allowed motion for the body. Using Newton's second law, a system of n bodies can be described as:
\begin{equation}
\label{ICMA:eq1}
\textbf{F}_{external} + \textbf{F}_{constraint} = M\mathbf{\ddot{x}},
\end{equation}
where $\mathbf{\ddot{x}} = [\mathbf{\ddot{x}}_1,\mathbf{\ddot{x}}_2,...,\mathbf{\ddot{x}}_n]$ is the acceleration of \(n\) rigid bodies, \(M\) is body mass matrix, and force $\textbf{F}$ is decomposed into external and constraint force. ODE uses the first-order
approximation by the discretization with time step $\Delta t$: $\ddot{x} \approx \dfrac{\dot{x}^{n+1} - \dot{x}^n}{\Delta t}$. By substituting the approximation in \ref{ICMA:eq1} and rearranging yields: 
\begin{equation}
\label{ICMA:eq2}
M^{-1}\textbf{F}_{constraint} = \dfrac{\mathbf{\dot{x}}^{n+1}}{\Delta t} - (\dfrac{\mathbf{\dot{x}}^n}{\Delta t} + M^{-1}\textbf{F}_{external}).
\end{equation}
Corresponding constraint forces are necessary to ensure that
the system follows the constraints. The direction of these
forces is known and perpendicular to the allowed motion, it
can be expressed by a Jacobian
\begin{equation}
\label{ICMA:eq3}
\textbf{F}_{constraint} = J^T \boldsymbol{\lambda},
\end{equation}
where $\lambda$ are the unknown signed magnitudes of the constraint forces, called Lagrange multipliers. In ODE system, the contact constraint are designed to allow some natural penetration of two soft, bouncy object, hard constraint, and to compensate the drift position error between two objects. To allow implementation of those constraints, ODE yields the velocity constraints at the (n+1) step as below:
\begin{equation}
\label{ICMA:eq4}
\mathbf{\dot{x}}^{n+1} = J^{-1}(c-k_{cfm}\boldsymbol{\lambda})
\end{equation}
where $k_{cfm}$ is a square diagonal matrix that mixes the constraints with the corresponding constraint forces, c is the corrective term for error position. To determine the value of constrain force, we substitute the relations for the constraint forces (\ref{ICMA:eq3}) and velocity constraints (\ref{ICMA:eq4}) in (\ref{ICMA:eq2}) and rearrange the formulation as below:
\begin{equation}
\label{ICMA:eq5}
(JM^{-1}J^T + \dfrac{1}{\Delta t} K_{cfm})\boldsymbol{\lambda} = \dfrac{c}{\Delta t} - J(\dfrac{\mathbf{\dot{x}}^n}{\Delta t} + M^{-1}\textbf{F}_{external}).
\end{equation}
Furthermore, ODE has boundary on the constraint force:
$\lambda \geq 0$ and friction coin rule to prevent movement in the
tangential direction: $\lambda \leq \mu F_n$ with $F_n$ is the normal
force. With boundary condition above and (\ref{ICMA:eq5}), the constraint force can be solved by the Linear Complementary Problem (LCP), 
\begin{equation}
\label{ICMA:eq6}
\begin{aligned}
& A\boldsymbol{\lambda} = b \\ 
& \boldsymbol{\lambda}_{min} \leq \boldsymbol{\lambda} \leq \boldsymbol{\lambda}_{max}, 
\end{aligned}
\end{equation}
where $A = (JM^{-1}J^T + \dfrac{1}{\Delta t}K_{cfm})$ and $B = \dfrac{c}{\Delta t} - J(\dfrac{\mathbf{\dot{x}}^n}{\Delta t} + M^{-1}\textbf{F}_{external})$. ODE uses the projected Gauss-Seidel (PGS) with Successive Over-Relaxation (SOR). Open dynamic engine parameters are given in Table. II.
\begin{table}[h]
\caption{Control parameters for flat foot robot model}
\label{table:control}
\begin{center}
\begin{tabular}{|c|c|c|c|c|}
\hline
Parameter & Meaning  & Value [unit]\\
\hline
$k$ & spring stiffness for virtual leg & 15000 [$N/m$]\\
$k_d$ & virtual spring damping & 150 [$N\cdot s /m$]\\
$k_a$ & ankle control position-proportional gain & 15 [$N/m$]\\
$c$ & position-proportional gain & 10 [$\cdot$] \\
$d$ & velocity-proportional gain  & 1 [$s$]\\
$\mu$ & coefficient value for VBLA & 0.6 [$\cdot$] \\
$L_0$ & the rest length of virtual leg & 0.55 [$m$]\\ 
$\Delta L_d$ & the maximum of retracted length & $0.2L_0[m]$\\
$\Delta \psi$ &the deviation of $\psi$ & $\pi/3[rad]$\\
$\tilde{\psi}_d$ & the desired trunk angle & $-\pi/40[rad]$ \\
$q_a^{sw}$ & the desired ankle angle in swing phase & $\pi/6[rad]$ \\
$q_a^p$ & the desired ankle angle in push-off & $-\pi/9[rad]$\\
\hline
\end{tabular}
\end{center}
\end{table}
\section{Control Strategy} 
In this section, the control strategy for robot model is described. First of all, we introduce the state machine for walking motion. In this way, the walking state machine will determine controller mode for each leg at each walking phase. To stable the trunk, we embed the simple force direction control in our previous paper \cite{lee2017biped} with appropriate modification to the robot model. In Fig. \ref{fig:control}, the control variables associated with stance and swing phase are shown. Herein, we create a virtual spring between hip joint and ankle joint of two legs, as utilized in the BTSLIP model. The hip torque and knee torque are mapped as close as possible to the simple model while the ankle torque follows the command from walking state machine to active push-off phases and heel-strike phases. 
\subsection{A Finite State Machine}
\begin{figure*}[h!]
    \centering
    \includegraphics[width=\textwidth,height=12cm,keepaspectratio]{ICMA_foot/Figure12.PNG}
    \caption{The gait phases upon which the state machine is based. The labels correspond to the phase of the gray leg (leg 1) (a) - (d). A Finite State Machine is used for each leg to determine controller mode for walking phases (e). The leg 1 and 2 are marked with gray color and blue color. S1 (a) is the pre-swing phase. The stance mode control is applied to the leg 2, and the swing mode control is applied to the opposite leg. S2 (b) is the initial swing phase. The leg 1 and 2 are still in swing mode and stance mode control. Ankle push-off control will be applied to leg 2 if $p_{toe} \leq p_{CoM}$, where $p_{toe}$ and $p_{CoM}$ are the toe horizontal position of the leg 2 and horizontal position of CoM. S3 (c) is the mid-swing phase. The leg 2 is still in the stance mode control and push-off mode control. S4 (d) is the terminal-swing phase, heel of the leg 1 touches the ground while the heel of the leg 2 takes off from the ground. In this sate, the leg 2 quickly turns to swing mode control.} 
\label{ICMA:FSM}
\end{figure*}

It is worth to note that each leg has specific mode control in each particular phase that defines in Finite State Machine, and does not require the information of the opposite leg. Therefore, without loss of generality in Figure. \ref{ICMA:FSM}, we start the description for a waking state machine with a toe-off event at the leg 1. According to Figure. 2:

$\bullet$ \textit{State 1 (S1):} The controller for the leg 1 is in swing mode control while the controller for the leg 2 is in the stance mode control. Ankle angle of the swing leg will be controlled to a fixed angle until it touches the ground by its heel in the last state. 

$\bullet$ \textit{State 2 (S2):} The swing leg (i.e. leg index 1) will be retracted and swung forward by swing mode control. The stance mode control is still applied to the leg 2. When the horizontal position of Center of Mass exceeds the toe of the leg 2, the ankle control will be applied to activate ankle push-off \cite{lipfert2014impulsive} to the existing stance leg. 

$\bullet$ \textit{State 3 (S3):} The leg 1 is still in swing mode control. The ankle push-off control and stance mode control are applied to the leg 2. 

$\bullet$ \textit{State 4 (S4):} A touchdown event for the swing leg (i.e. the heel of leg 1 touches the ground) will lead the system to this state. The controller will apply the ankle torque to the leg 1 to dampen the impulsive impact until its foot becomes flat. Instantaneously, FSM starts again with the correspondence swing leg 2. 

Additionally, the condition of foot will be determined as follow: (1) heel-strike (i.e. $p_{heel} = 0$ and $p_{toe} > 0$), (2) foot-flat (i.e. $p_{heel} = 0$ and $p_{toe} = 0$), (3) heel-off (i.e. $p_{heel} > 0$ and $p_{toe} = 0$), (4) toe-off (i.e. $p_{heel} > 0$ and $p_{toe} > 0$), where $p_{heel}$ and $p_{toe}$ are the position of heel and toe in Cartesian coordinate
\subsection{stance mode control} 
\begin{figure}[ht]
    \centering
    \includegraphics[width=0.8\textwidth,height=0.85\textheight,keepaspectratio]{ICMA_foot/Figure5}
    \caption{The robot model is considered to have virtual compliant legs. Two virtual spring is connected from hip to each foot with the same spring stiffness, \(k\). The control variables for swing leg (a). And the control variables for stance leg (b). The Velocity based Leg Adjustment \cite{sharbafi2016vbla} (c)}    
\label{fig:control}
\end{figure}
Fig. \ref{fig:control} shows that the 7-link model will have virtual springs from hip to each foot. We inherit the controller of the BTSLIP model [17] in order to calculate the virtual axial force and tangential force that is necessary to maintain the upright trunk in walking for the stance leg. The commanded ankle-end is 
calculated as:
\begin{equation}
\label{ICMA:control}
\begin{aligned}
\textbf{F} = 
\begin{bmatrix}
F_r \\
F_t
\end{bmatrix} 
= 
\begin{bmatrix}
k (L_0 - L) + k_d\dot{L}\\
F_r \tan \beta
\end{bmatrix}
\end{aligned}
\end{equation}
where \(k\) and \(L_0\) are the spring stiffness and the rest length of virtual leg, respectively.  \(L = | p_a - p_h|\) is current leg length (i.e. Virtual leg length is calculated by the distance from ankle to hip), \(p_a\) and \(p_h\) are the position of ankle and hip in global frame, respectively. $\beta = \eta + \tilde{\beta}$ is the angle between \(\textbf{F}\) and the virtual leg, and $\eta$ is the angle between the vector from stance ankle to CoM and virtual leg. We added the damping component in computing of the axial force. Note that we set up the lower limit of \(F_r\) to zero. 

The ground reaction force vector \(\textbf{F}\) is the combination of the virtual spring force and reaction force : $\textbf{F} = \textbf{F}_r + \textbf{F}_t$. The direction of $\textbf{F}$ is controlled by $\tilde{\beta}$. We take into account the speed and direction of the trunk in determining the direction of \(\textbf{F}\) as follows: 
\begin{equation}
\begin{aligned}
&\tilde{\beta} = -c\tilde{\phi} - d\dot{\tilde{\phi}}
\end{aligned}
\end{equation}
where $c$ and $d$ are the control parameters, as shown in Table \ref{table:control}. $\tilde{\phi}$ is the angle of the trunk w.r.t to vertical line.

\subsection{swing mode control} 
We use the method velocity-based leg adjustment (VBLA \cite{sharbafi2016vbla}) for swing leg control, to define the touchdown angle for swing leg, as shown in Fig. \ref{fig:control}(c). The VBLA determines desired swing leg orientation as follows, 
\begin{equation}
\textbf{O} = \mu \textbf{V} + (1-\mu)\textbf{G},
\end{equation}
where \(\textbf{V} = \dfrac{\textbf{v}}{\sqrt{gL}}\) and \(\textbf{G} = \dfrac{\textbf{g}}{|g|}\) are non-dimensionalized CoM velocity and gravitational acceleration, respectively. $\textbf{v}$ and $\textbf{g}$ are the vector of CoM velocity and the gravity. The angle between the desired vector \(\textbf{O}\) and the ground is determined as the desired touch down angle \(\alpha_d\). By defining the virtual leg length trajectory for swing leg \(L_d\), and combining with the desired touch down angle from VBLA, the simple form of proportional - derivative control can be applied for swing leg control: 
\begin{equation}
\begin{aligned}
\textbf{F}
\begin{bmatrix}
F_r \\
F_t
\end{bmatrix}
=
\begin{bmatrix}
k(L_d - L^{sw}) + \dfrac{k}{10^2}(\dot{L}_d - \dot{L}) \\
(c(\alpha_d - \alpha) + \dot{\psi})/L
\end{bmatrix}
\end{aligned}
\end{equation}
where $\alpha$ is the angle created by the virtual swing leg and the horizontal axis. The virtual leg length can be retracted to prevent the scuffing of the leg with the walking surface by
controlling the desired swing leg length $L_d$:
\begin{equation} 
\begin{aligned}
\left \{
\begin{array}{ll}
 L_d = 0.8L_0 + \dfrac{\Delta L_d}{2} - \dfrac{\Delta L_d}{2} \cos(\dfrac{\tilde{\psi}2\pi}{\Delta\psi}) \\ \: \textbf{if} \: \tilde{\psi} \geq \dfrac{\Delta\psi}{2} \:\textbf{or}\:\textbf{if} \:\tilde{\psi} \leq \dfrac{-\Delta\psi}{2}\\
 \textbf{then} \:L_d = L_0, 
 \end{array}
\right.
\end{aligned}
\end{equation}
where $\tilde{\psi} = \psi + \tilde{\psi}_d$. $L_d$ and $\Delta \psi$  are control parameters,
as shown in Table III. The stance and swing control set the commanded forces at the ankle in axial and tangential direction. This set of forces can be turned into hip and knee torques $\textbf{u} = [u_h,u_k]^T$ via Jacobian $\textbf{J}_{local} \in \mathbb{R}^{2 \times 1}$  from hip to ankle for each leg.
\begin{equation}
\textbf{u} = \textbf{J}_{local} \textbf{F}
\end{equation}
\subsection{Ankle mode control}
Feet and ankles support many benefits to bipedal walking. It can help to reduce the fluctuations of velocity when the center of pressure on the foot can travel forward. They also help to reduce the impulsive impact then the heel strikes the ground and to inject energy at the end of the stride through toe off. According to FSM, the torque at the ankle can be controlled actively. In order to dampen impulsive velocity when heel strikes the ground, the ankle torque can be described as below: 
\begin{equation}
u_a = -\dfrac{k_a}{10}\dot{q}_i
\end{equation}
where index $i = \{3,6\}$ depends on touching ground condition of legs. When the leg satisfies the condition of push-off in the FSM, the ankle control tries to plantar flex the ankle angle. The simple PD control for the ankle in this phase is expressed as below: 
\begin{equation}
u_a = k_a q_{ankle}^p - \dfrac{k_a}{10}\dot{q}_i,
\end{equation}
and during toe off state (i.e. Swing phase), the ankle is served to a fixed angle using a PD controller: 
\begin{equation}
u_a = k_a(q_i - q_{ankle}^{sw}) - \dfrac{k_a}{10}\dot{q}_i, 
\end{equation}
where $q_{ankle}^p$ and $q_{ankle}^{sw}$ ankle are control parameter as shown in Table. III.
\begin{figure*}[h]
    \centering
    \includegraphics[width=\textwidth,height=7cm]{ICMA_foot/Figure6}
    \caption{The model begins with a random initial condition and converges steady motion. The phase portrait of an angle from trunk to the thigh of the leg 1 (a) and leg 2 (b). The phase plot of stance knee angle of the leg 1 (c) and the knee angle of the leg 2 (d). The
phase plot of absolute trunk angle (e).} 
\label{fig:result1}
\end{figure*}
\section{Simulation Result and Discussion}
\begin{figure}[ht]
    \centering
    \includegraphics[width=\textwidth,height=0.4\textheight]{ICMA_foot/Figure10}
    \caption{ Trajectories in 25 seconds of the 7-link robot model. The vertical position of the CoM (a). The velocity of vertical position of CoM (b). The forward velocity of the robot model (c). The pitch of trunk is plotted in (d)}
\label{fig:general}
\end{figure}
\begin{figure}[h!]
    \centering    \includegraphics[width=\textwidth,height=0.4\textheight,keepaspectratio]{ICMA_foot/Figure8.png}
    \caption{ The control inputs of walking motion in 30 seconds. The hip torque ($u_h$ : blue), knee torque ($u_k$ : green), and ankle torque ($u_a$: blue) are applied to leg 1 (a), and leg 2
(b).}   
\label{res:control}
\end{figure}
\begin{figure*}[h!]
    \centering
    \includegraphics[width=11cm,height=8cm]{ICMA_foot/Figure9}
    \caption{The snapshots on the above left are captured in 0.4 seconds and show one swing phase motion. The rough surface is shown in the above right side with the range of height about $[0:5]cm$. The snapshots below show several steps in approximately 9 seconds on terrain ground.} 
\label{fig:natural walking}
\end{figure*}
To test the validity of the proposed control strategy with the 7-link rigid model, we use the dynamic simulation in ODE environment. The model parameters, environment parameters, and control parameters are listed in Table I, Table II and Table III, respectively. The 7-link robot dynamics are illustrated in Section II, and the proposed control is presented in Section III.

The trajectories of $y_{CoM}$, $\dot{y}_{CoM}$, $x_{CoM}$, and $q_9$ are plotted in Fig. \ref{fig:general}. The beginning of trajectories due to the initial conditions has jerky motions, and it changes to stable motion around 1:5[s]. A simulated robot walked at a moderate speed (approximately $0.75 [m/s]$). It is worth noting that the control algorithm does not use any explicit speed control method, but the speed is stabilized. We intuitively speculate that this is because of the natural system dynamics, in the same way, that speed is naturally stabilized by the VBLA controller and springy legs mechanism. Additionally, in Fig. 6(c), the trunk angle of the robot model oscillates around the desired angle $\tilde{\phi}_d$ 

It can be seen in Fig. \ref{res:control}  that the control signals oscillate with large range at the very first steps. After that, the steady state behavior of control inputs is attained with the range of hip torques are about \([-50,20]^T\)[Nm] and \([-20,40]^T\)[Nm] for the knee torques. In the ankle push-off phase, the instantaneous ankle torques will be applied which results as spikes in Fig. \ref{res:control}.  
\begin{figure}[h!]
    \centering
    \includegraphics[width=\textwidth,height=0.4\textheight]{ICMA_foot/Figure14}
    \caption{ Trajectories in 30 seconds of the 7-link robot model on rough terrain map with the same initial conditions in Fig. \ref{fig:general}. Vertical position and velocity vertical position of CoM are drew in (a), and (b). The speed of robot and trunk orientation of the robot are in (c) and (d).  The plots are changed color to red when robot is in the terrain ground.}
\label{fig:roughmap}
\end{figure}

To further investigate the performance of control strategy, we simulate the biped robot on the rough terrains ground. We use height map terrains function in ODE, the height level of each point in ground terrain map will be defined by formulation: $z(cm) = 1 + 4\sin(20y+20x)$, where $x$, $y$ and $z$ are the location of each point on the terrain map. More clearly, the height of each point in terrain map will be in the range $[0:5]cm$. The terrain map is illustrated on the right side of Fig. \ref{fig:natural walking}. Additionally, the drawing on the left side of Fig. \ref{fig:natural walking} show completely swing phase of one leg. The robot still keeps a natural walking gait even in the rough terrain surface, as shown in the figure below in Fig. \ref{fig:natural walking}. Fig. \ref{fig:roughmap} shows the robustness of our proposed control against terrain map. The trunk is quickly turned to steady motion when robot overcomes the terrain surface; the jerky behavior in the horizontal and vertical motion of CoM is also diminished after touching to the flat surface. 

\section{Summary}
In this work, we presented the control strategy for the 7-link robot model including the force direction control and simple ankle control. More precisely, the force direction control for hip and knee torques are designed based on our previous control for the BTSLIP model \cite{lee2017biped} and these forces are mapped to joint actuators by Jacobian transpose.
Besides, the swing controller is capable of retracting the swing leg to prevent it being scuffed with the ground. In additionally, the FSM will activate the ankle push-off mode
control in appropriate phases and helps to reduce the impulsive impact by controlling the ankle torque in heel strike phases. The controller was validated in Open Dynamic Engine environment on the flat ground and terrain ground, as the model converges to its steady motion from random initial condition within a few steps. While the results of this work are limited to specific
choices of the model (mass and length, size), we believe that this controller is sufficiently practical to be a basis for future research. Besides, the control for ankle model seems
very simple and needs more research on its. As for our future work, we will focus on extending the articulated robot with more real size and improve the control strategy. Reaching the
walking controller for the frontal balance and turning in the 3-D environment are also of our desire. 
