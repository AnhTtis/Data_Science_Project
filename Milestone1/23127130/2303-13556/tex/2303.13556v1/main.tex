
% This version of CVPR template is provided by Ming-Ming Cheng.
% Please leave an issue if you found a bug:
% https://github.com/MCG-NKU/CVPR_Template.

% \documentclass[review]{cvpr}
% % \documentclass[final]{cvpr}
\documentclass[10pt,twocolumn,letterpaper]{article}

%%%%%%%%% PAPER TYPE  - PLEASE UPDATE FOR FINAL VERSION
\usepackage[final]{cvpr}      % To produce the REVIEW version
\usepackage[accsupp]{axessibility}  % Improves PDF readability for those with disabilities.
\usepackage{times}
\usepackage{epsfig}
\usepackage{graphicx}
\usepackage{amsmath}
\usepackage{amssymb}
\usepackage{algorithm}
\usepackage{algorithmic}
\usepackage{listings}
\usepackage{soul}
\usepackage{cancel}
\usepackage{dsfont}
\usepackage{pifont}
%\usepackage{bbold}  # Commented this to fix the \mathbb{R}
%\usepackage{ mathrsfs } # Commented this to fix the \mathbb{R}
\usepackage[utf8]{inputenc}
\usepackage{multirow}
\usepackage[numbers,sort,compress]{natbib}
\usepackage[usenames,dvipsnames,svgnames,table]{xcolor}
% \usepackage[ruled,vlined,linesnumbered]{algorithm2e} % added by Islam
\usepackage{tablefootnote} % added by Islam
\graphicspath{{./}{./images/}}
% \usepackage{ulem}
% Include other packages here, before hyperref.
\usepackage{subcaption}
\usepackage{xspace}
\usepackage{tabularx,ragged2e,booktabs}
\usepackage{color}
\definecolor{_fbteal3}{HTML}{CBCBCB} %#9FE2BF %e6f5f0 (fb)
\definecolor{Gray}{gray}{0.85}
\definecolor{darkgreen}{rgb}{0.0, 0.5, 0.0}
\definecolor{darkred}{rgb}{0.64, 0.0, 0.0}
\newcolumntype{a}{>{\columncolor{_fbteal3}}c}
% If you comment hyperref and then uncomment it, you should delete
% egpaper.aux before re-running latex.  (Or just hit 'q' on the first latex
% run, let it finish, and you should be clear).
\usepackage[pagebackref=true,breaklinks=true,colorlinks,bookmarks=false]{hyperref}

\def\cvprPaperID{4075} % *** Enter the CVPR Paper ID here
% \def\confYear{CVPR 2023}
\setcounter{page}{4321} % For final version only
\def\confName{CVPR}
\def\confYear{2023}


\makeatletter
\DeclareRobustCommand\onedot{\futurelet\@let@token\@onedot}
\def\@onedot{\ifx\@let@token.\else.\null\fi\xspace}

\def\eg{\textit{e.g}\onedot} \def\Eg{\emph{E.g}\onedot}
%\def\ie{\emph{i.e}\onedot} \def\Ie{\emph{I.e}\onedot}
\def\ie{\textit{i.e}\onedot} \def\Ie{\emph{I.e}\onedot}
\def\cf{\emph{c.f}\onedot} \def\Cf{\emph{C.f}\onedot}
\def\etc{\emph{etc}\onedot} \def\vs{\emph{vs}\onedot}
\def\wrt{w.r.t\onedot} \def\dof{d.o.f\onedot}
%\def\etal{\emph{et al}\onedot}
\def\etal{{et al}\onedot}
\makeatother

\def\Vec#1{{\boldsymbol{#1}}}
\def\Mat#1{{\boldsymbol{#1}}}

\def\SPD#1{\mathcal{S}_{++}^{#1}}
\def\SYM#1{\mathrm{Sym}({#1})}
\def\GRASS#1#2{\mathcal{G}({#1},{#2})}

\def\TODO#1{{\color{red}{\bf [TODO:} {\it{#1}}{\bf ]}}}
\def\NOTE#1{{\bf [NOTE:} {\it\color{blue}{#1}}{\bf ]}.}
\def\CHK#1{{\bf [CHECK:} {\it\color{red} {#1}}{\bf ]}.}
\def\RH#1{{\bf [RH:} {\it\color{Magenta} {#1}}{\bf ]}.}
\def\IN#1{{\bf [IN:} {\it\color{OliveGreen} {#1}}{\bf ]}.}
\def\SH#1{{\bf [SH:} {\it\color{Blue} {#1}}{\bf ]}.}
\def\EA#1{{\bf [EA:} {\it\color{green} {#1}}{\bf ]}.}


\newcommand{\cmark}{\ding{51}}
\newcommand{\xmark}{\ding{55}}
% \newcommand{\putouralg}{\textsc{ProtoCon}\xspace}
\newcommand{\putouralg}{\textsc{ProtoCon}\xspace}
\newcommand{\vx}{\boldsymbol{x}}
\newcommand{\vy}{\boldsymbol{y}}
\newcommand{\vz}{\boldsymbol{z}}
\newcommand{\vu}{\boldsymbol{u}}
\newcommand{\vq}{\boldsymbol{q}}
\newcommand{\vp}{\boldsymbol{p}}
\newcommand{\vm}{\boldsymbol{m}}
\newcommand{\vM}{\boldsymbol{M}}
\newcommand{\Lcal}{\mathcal{L}}
\newcommand{\Qcal}{\mathcal{Q}}
\newcommand{\Ical}{\mathcal{I}}
\newcommand{\Ocal}{\mathcal{O}}
\newcommand{\Xcal}{\mathcal{X}}
\newcommand{\Ucal}{\mathcal{U}}
\newcommand{\Acal}{\mathcal{A}}
\newcommand{\Ccal}{\mathcal{C}}
\newcommand{\Mcal}{\mathcal{M}}
\newcommand{\Vcal}{\mathcal{V}}
\newcommand{\Pcal}{\mathcal{P}}

\newcommand{\tr}{\mathop{\rm  Tr}\nolimits}
\newcommand{\expm}{\mathop{\rm  expm}\nolimits}
\newcommand{\DIAG}{\mbox{Diag\@\xspace}}
\newcommand{\subject}{\mathrm{s.t.}}
\newcommand\norm[1]{\left\lVert#1\right\rVert} % added by Islam

\DeclareMathOperator*{\argmin}{arg\,min}
\DeclareMathOperator*{\argmax}{arg\,max}

\newtheorem{definition}{Definition}
\newtheorem{remark}{Remark}

\newcommand{\hr}[1]{\textcolor{purple}{\small\textbf{[HR]} #1}}

  

\def\MH#1{{\bf [} {\it \color{red} {#1}}{\bf ]}.}



% \title{ProtoCon: Semi-supervised Learning with K-Means Label Smoothing and {\color{blue}\underline{Proto}}typical {\color{blue}\underline{Con}}sistency}
% \title{\putouralg: Boosting Semi-supervised Learning via Online K-means Label Refinement and {\color{black}\underline{Proto}}typical {\color{black}\underline{Con}}sistency.}
\title{\vspace{-1.2em} \putouralg: Pseudo-label Refinement via Online Clustering and \underline{Proto}typical \underline{Con}sistency for Efficient Semi-supervised Learning}


\author{Islam Nassar\textsuperscript{1}
% \thanks{corresponding author: islam.nassar@monash.edu} 
\quad Munawar Hayat\textsuperscript{1} 
\quad Ehsan Abbasnejad\textsuperscript{2} 
\quad Hamid Rezatofighi\textsuperscript{1} \\ 
Gholamreza Haffari\textsuperscript{1}\\
{\small \textsuperscript{1} Data Science and AI Department, Monash University, Australia – firstname.lastname@monash.edu} \\
{\small \textsuperscript{2} Australian Institute for Machine Learning, The University of Adelaide, Australia – firstname.lastname@adelaide.edu.au}
}

\date{October 2020}

\begin{document}
%\pagenumbering{gobble}  % to hide page numbers for camera-ready version
\maketitle    % uncomment once we deliver supplementary

\begin{abstract}
Confidence-based pseudo-labeling is among the dominant approaches in semi-supervised learning (SSL). It relies on including high-confidence predictions made on unlabeled data as additional targets to train the model. We propose \putouralg, a novel SSL method aimed at the less-explored label-scarce SSL where such methods usually underperform. \putouralg refines the pseudo-labels by leveraging their nearest neighbours' information. The neighbours are identified as the training proceeds using an online clustering approach operating in an embedding space trained via a prototypical loss to encourage well-formed clusters. The online nature of \putouralg allows it to utilise the label history of the entire dataset in one training cycle to refine labels in the following cycle without the need to store image embeddings. Hence, it can seamlessly scale to larger datasets at a low cost. Finally, \putouralg addresses the poor training signal in the initial phase of training (due to fewer confident predictions) by introducing an auxiliary self-supervised loss.  It delivers significant gains and faster convergence over state-of-the-art across 5 datasets, including CIFARs, ImageNet and DomainNet. 

% \EA{give an intuition first. why nearest neighbors?} 
% \EA{does it mean it is different from the embedding space of the neural net used for classification?} 
% \EA{To me there are a lot of jargon without much justification. why your approach is unique? why you made these choices?}
% \EA{you can mention the margin of improvement if it is significant} 

\end{abstract}


\section{Introduction}
\label{sec:introduction}

% Currently deployed electronic payment systems do not protect user privacy. Data such as the user's identity and the date and location of the payment are leaked. From such data, privacy-sensitive information like the user's whereabouts, her health status, or her religious or political affiliations can be inferred.

At the present moment, there is a pressing need for private electronic cash (e-cash), as shown by growing interest in blockchain-based systems and Centrally-Banked Digital Currencies (CBDCs), but no privacy-enhanced and decentralized system exists that scales to the requirements of this application scenario. To address this problem, we introduce the first decentralized offline e-cash scheme with provable security and full implementation, that can efficiently support electronic payments. In contrast to the current privacy-enhanced blockchain, systems do not require a constant online presence of the payees. The issuance of coins is non-interactive, i.e., the authorities do not need to synchronise, as we do not rely on MPC protocols.
Moreover, our scheme maps to distributed payment systems such as CBDCs, a technology that urgently requires further attention in terms of privacy.
The idea of distributing issuance among a quorum of authorities has been so far explored in the context 
of attribute-based credentials~\cite{DBLP:conf/ndss/SonninoABMD19, cryptoeprint:2022:011} and online anonymous e-cash schemes~\cite{DBLP:journals/iacr/BaudetSKD22}.
\newline\noindent \textit{\textbf{Contributions:}}
Our paper makes the following contributions:
\begin{itemize}[leftmargin=*, noitemsep,topsep=0pt]
    \item We introduce a construction $\mathrm{\Pi}_{\CEC}$ for threshold issuance offline e-cash ($\CEC$). To the best of our knowledge, this is the first offline e-cash scheme with threshold issuance. To this end, we define the system model and the security properties in the ideal-world/real-world paradigm~\cite{DBLP:conf/focs/Canetti01} by proposing an ideal functionality $\Functionality_{\CEC}$ for $\CEC$ (Sections \S\ref{sec:offlineecash}, \S\ref{sec:constructionCEC}).
    \item We propose two instantiations of  $\mathrm{\Pi}_{\CEC}$ based on compact~\cite{DBLP:conf/eurocrypt/CamenischHL05} and divisible~\cite{DBLP:conf/pkc/PointchevalST17} e-cash (Section \S\ref{sec:instantiation}). Our schemes are more efficient than~\cite{DBLP:conf/eurocrypt/CamenischHL05, DBLP:conf/pkc/PointchevalST17} thanks to the use of Pointcheval-Sanders signatures in the random oracle model and to decreasing the number of coin secrets. 
    We formally prove that our construction $\mathrm{\Pi}_{\CEC}$ realizes $\Functionality_{\CEC}$ when instantiated with the algorithms of our compact and divisible e-cash schemes (Sections \S\ref{sec:securityProofCompact},  \S\ref{sec:securityProofDivisible}). 
    \item We provide an open-source Rust implementation of both schemes and present an extensive evaluation of their performance and trade-offs (Section \S\ref{sec:efficiencycomparison}). To the best of our knowledge, this is first such practical comparison.
    % Regarding the spending phase, the computation cost of the divisible $\CEC$ scheme is $125$ ms, whereas the cost of our compact $\CEC$ scheme ranges from $51$ ms to $468$ ms depending on the size of the price range and the set of denominations considered.
    \item We conclude by outlining how our schemes can be integrated with a blockchain-based bulletin board (Section \S\ref{sec:integration}). This would allow our scheme to fulfil the requirements for a distributed privacy-enhanced CBDC~\cite{digitaleuro} and even provide scalability via offline transactions for existing blockchain systems like ZCash~\cite{zcash}. 
\end{itemize}

\section{Background and Motivation}
\label{background}
Anonymous e-cash was originally proposed by Chaum as a digital analog of regular cash, which also allows for private payments~\cite{DBLP:conf/crypto/Chaum82}. 
% Anonymous e-cash systems guarantee that no coalition of dishonest users, providers (i.e. merchants), and a bank can link the spending of a coin with its withdrawal or link multiple spendings performed by the same user. Moreover, no coalition of dishonest users and providers is able to deposit more coins than have been withdrawn. 
Unlike physical cash, e-coins are easy to duplicate, hence e-cash schemes must prevent double-spending and typically that is done by having a centralized bank~\cite{DBLP:conf/eurocrypt/CamenischHL05,DBLP:conf/pairing/BelenkiyCKL09,DBLP:conf/eurocrypt/CanardG07,DBLP:conf/acns/CanardPST15,DBLP:conf/pkc/CanardPST15,DBLP:conf/pkc/PointchevalST17,DBLP:conf/asiacrypt/BoursePS19,DBLP:conf/crypto/OkamotoO89,DBLP:conf/crypto/OkamotoO91,DBLP:conf/pkc/BaldimtsiCFK15,DBLP:conf/pkc/BauerFQ21}. Thus, there has been a revival of interest in adopting privacy to decentralized blockchain systems, in which coins are authenticated by proving in ZK that they belong to a public list of valid coins maintained on the blockchain, thus they do not require a central bank to prevent double-spending~\cite{DBLP:conf/sp/MiersG0R13}. 

Although blockchain-based privacy-enhanced systems such as ZCash and Monero have users~\cite{zcash, cryptonote,DBLP:journals/ledger/NoetherM16}, such 
decentralized e-cash systems require being online to check the status of payments, which simply does not scale to the speed of transactions needed by real-world payment systems or support the reality of transactions that need to be made without internet access, so the usage of blockchains for payments remains small in proportion to traditional payments. Also, as could happen to any other low-transaction fee blockchain that advertises high throughput, the ZCash blockchain has suffered an  attack of \emph{`spam'} transactions that have increased the blockchain size to such an extent that ZCash has suffered from what is effectively a \emph{denial-of-service} attack that has collapsed its throughput (i.e., simply downloading the blockchain becomes nearly impossible)~\cite{zcashattack1}. Similarly, `layer 2' solutions based on blockchain technologies such as zero-knowledge roll-ups, which increase transaction speed and (in some cases) privacy, are also vulnerable to these attacks~\cite{zamyatin2021sok}.  By virtue of not requiring a merchant to be online all the time but only needing eventual settlement over regular epochs, our system avoids these issues while maintaining the advantages of decentralization. Hence, it is to be expected that current privacy-enhanced blockchain systems that require an online setting will evolve into decentralized offline e-cash systems similar to the one presented in this paper. While proposals exist to enable offline payments in blockchain-based cryptocurrencies
such as payment channel networks~\cite{DBLP:conf/ccs/0001M17}, 
they typically do not offer strong privacy protection as only users who share a channel can transact with each other and so users who make a payment are not anonymous, similarly as in~\cite{10.1145/3052973.3052980}, and cashing out payments still requires online blockchain interactions. Even privacy-preserving offline payment channels effectively restrict payment transactions and the network topology of payment channels can lead participants to be effectively de-anoymized~\cite{ DBLP:conf/fc/KapposYPKDMM21, sharma2022anonymity}. 
%Therefore, offline anonymous e-cash schemes with threshold issuance offer
%an interesting alternative. 
% Recent market research on CBD released by the ECB further shows increasing interest in the offline approach to electronic payments~\cite{digitaleuro}. 

On the other side of the spectrum, centralized CBDCs have gained increasing interest in over a hundred countries and will soon be deployed in Europe and China~\cite{digitaleuro, xu2022developments}. CBDC systems typically sacrifice user transaction privacy from the settlement layer and can lead to the possibility of dystopian surveillance of user financial transactions~\cite{danezis2015centrally}. Although financial transaction data could be considered a matter of national security, MIT and the Federal Reserve of Massachusetts in the United States have begun exploring blockchain systems for its CBDC efforts without transaction unlinkability~\cite{lovejoy2022high}. In stark contrast, Switzerland's Central Bank  has put forward a centralized online privacy-preserving scheme called `eCash 2.0', but offline payments would require special hardware and are currently not supported. The European Central Bank (ECB) recently published a list of requirements for the Digital Euro, which include privacy of user transactions from the settlement layer and offline transactions~\cite{digitaleuro}. Our scheme would fulfil those criteria, and we present how it could be integrated with a blockchain.
% , which the ECB seems to think will require special hardware, i.e. “a card-based solution should be supported”~\cite{digitaleuro}. 
Furthermore, distributing the issuing power would be practical for emerging multi-nation economic proposals for joint issuance of currencies such as the South American joint reserve currency SUR in the `Banco del Sur' recently endorsed by Brazil and Argentina~\cite{marshall2009financing}. Even more importantly, practically preventing byzantine faults in centralized CBDCs requires it to be managed by a set of distributed parties, similar to Facebook's Diem project's consensus protocol based on HotStuff~\cite{yin2019hotstuff}. In this manner, our system unifies the objectives of blockchain research for decentralization while maintaining compatibility with the requirements for privacy-preserving CBDC by decentralizing e-cash.

% A user withdraws a wallet with a number of coins from the bank. 
% A typical anonymous e-cash scheme consists of three types of entities: the \emph{bank}, \emph{users} and \emph{providers}. The interaction between those parties takes place through a \emph{withdrawal}, \emph{spend} and \emph{deposit} phase.  In the withdrawal phase, the user obtains a wallet with a number of coins from the bank.
% At a later point, the user spends one or several coins from her wallet with a provider, who then deposits the received payments at the bank in exchange for funds. 

% , i.e. it must be possible to detect whether a coin is spent more than once.
% To enable double-spending detection,
\vspace{-5pt}
\paragraph{Online vs Offline Ecash}
To revisit the original e-cash proposal, the solution to double-spending is to associate each coin with a unique \emph{serial number}, which is used to detect double-spending by dishonest users and prevent dishonest providers from depositing a payment twice~\cite{DBLP:conf/crypto/Chaum82}.
% which is revealed after the coin is spent and used in the deposit phase, where the bank compares the coin's serial number with the serial numbers of previously spent coins, and accepts the coin only if there is no match.
% This detects double-spending by dishonest users, and also prevents dishonest providers from depositing a payment twice.
% The serial number is hidden from the bank during the withdrawal phase, but it is revealed after the coin is spent and can be used in the deposit phase, where the bank compares the coin's serial number with the serial numbers of previously spent coins, and accepts the coin only if there is no match. This detects double-spending by dishonest users, and also prevents dishonest providers from depositing a payment twice.
In \emph{online e-cash} schemes~\cite{DBLP:conf/crypto/Chaum82,DBLP:conf/ndss/SonninoABMD19}, providers are constantly connected to the bank
and can then check if a coin has been double-spent before accepting a payment. An alternative and more realistic solution space is given by \emph{offline e-cash} schemes~\cite{DBLP:conf/eurocrypt/CamenischHL05,DBLP:conf/pairing/BelenkiyCKL09,DBLP:conf/eurocrypt/CanardG07,DBLP:conf/acns/CanardPST15,DBLP:conf/pkc/CanardPST15,DBLP:conf/pkc/PointchevalST17,DBLP:conf/asiacrypt/BoursePS19}, which do not require a permanent online connection between a provider and the bank.
% , and instead allow for double spenders identification. 
The provider can accept payment and deposit it at a later settlement stage as there is a guarantee that users who double-spend  will be identified by the bank.

An important issue in the design of anonymous e-cash schemes is paying the exact amount as users cannot receive change,
since the providers are not anonymous towards the bank. If a user receives change, the user would in fact become a provider and lose anonymity. 
% Therefore, anonymous e-cash schemes need to allow users to pay the exact amount.
% Consider a user that possesses a coin with denomination $\$100$ and wants to make a payment of $\$75$.
In online e-cash, the user can contact the bank in order to exchange a coin for lower denominations in order to make the payment. In offline e-cash, contacting the bank is not allowed during the spending phase.
% (Otherwise, the scheme would be online.)
\emph{Transferable e-cash} schemes~\cite{DBLP:conf/crypto/OkamotoO89,DBLP:conf/eurocrypt/ChaumP92,DBLP:conf/pkc/BaldimtsiCFK15} are one solution to this problem,
allowing a user to further spent a previously received coin without interacting with the bank. Hence, providers can return the change to the users that paid.
% a provider that receives a payment that exceeds the price to be paid is able to give change to the user that payed.
% First, \emph{anonymous transferable e-cash}~\cite{DBLP:conf/crypto/OkamotoO89,DBLP:conf/eurocrypt/ChaumP92,DBLP:conf/pkc/BaldimtsiCFK15} allows a user that receives a coin to spend it again, i.e.\ the user does not have to deposit it. Thanks to that, a provider that receives a payment that exceeds the price to be paid is able to give change to the user that payed.
%Note that, in non-transferable e-cash schemes, this is not possible, because the user that receives change would have to deposit it, and then her anonymity would be lost.
Although transferable e-cash is appealing, state-of-the-art schemes~\cite{DBLP:conf/pkc/BauerFQ21} are much less efficient than non-transferable ones.

An alternative solution to preserve user unlinkability is to use coins of the smallest denomination. However, the large number of coins that may need to be spent easily yields an inefficient scheme. To solve this problem, researchers have focused on designing offline e-cash protocols whose complexity does not depend on the number of coins withdrawn or spent.
% This way, it is guaranteed that the user will be able to pay the exact amount.
% The problem is the large number of coins that may need to be spent, which could yield an inefficient scheme. In our example, paying the amount of $\$75$ would involve spending $7500$ coins. To solve this problem, researchers have focused on designing offline e-cash protocols whose complexity does not depend on the number of coins withdrawn or spent.
In \emph{anonymous compact e-cash} schemes~\cite{DBLP:conf/eurocrypt/CamenischHL05,DBLP:conf/pairing/BelenkiyCKL09}, the cost of storing a wallet of $N$ coins and the cost of withdrawing $N$ coins is independent of $N$. However, the cost of spending $n \leq N$ coins grows linearly with $n$.
\emph{Anonymous divisible e-cash} schemes~\cite{DBLP:conf/crypto/OkamotoO91,DBLP:conf/eurocrypt/CanardG07,DBLP:conf/acns/CanardPST15,DBLP:conf/pkc/CanardPST15,DBLP:conf/pkc/PointchevalST17,DBLP:conf/asiacrypt/BoursePS19} improve the efficiency of compact e-cash and allow the user to spend $n \leq N$ coins with cost independent of $n$. Therefore, in the last decade, research has focused solely on divisible e-cash. However, divisible schemes achieve constant spending cost at the expense of much more expensive deposit and identification phases, and to our knowledge, efficiency analysis of compact e-cash schemes with multiple denominations has not been conducted, as well as a fair comparison between practical implementations of divisible and compact e-cash has never been done. Therefore, it is unclear which scheme is better for the use-case of offline e-cash as required by privacy-enhanced CBDCs~\cite{digitaleuro} and blockchain-based scalability. 
Our analysis shows that compact e-cash with multiple denominations is preferable for small payments, which would naturally compose the majority of offline e-cash transactions in application scenarios such as a user-facing blockchain or CBDC where practical deployment concerns would necessitate distributed authorities.
% Although constructions in~\cite{DBLP:conf/eurocrypt/CanardG07,DBLP:conf/pkc/CanardPST15, DBLP:conf/acns/CanardPST15} allow to spend
% only a number of coins given by intervals of the form $[1+j2^k, \allowbreak (j+1)2^k] \allowbreak \in \allowbreak N$, the scheme proposed in~\cite{DBLP:conf/pkc/PointchevalST17} allows the user to spend any number of coins for any interval $[a,b] \allowbreak \in \allowbreak N$ was proposed.
% have the drawback that they do not allow to spend any number of coins $n \leq N$ with cost independent of $n$, but only a number of coins given by intervals of the form $[1+j2^k, \allowbreak (j+1)2^k] \allowbreak \in \allowbreak N$. In~\cite{DBLP:conf/pkc/PointchevalST17}, an anonymous divisible e-cash scheme that allows the user to spend any number of coins for any interval $[a,b] \allowbreak \in \allowbreak N$ was proposed.

% Although spending one coin is more efficient in compact e-cash than in divisible e-cash schemes, completing a payment requires multiple coins to be spent, and thus compact e-cash schemes would be less efficient. A solution proposed in the literature to improve the efficiency of compact e-cash schemes consists of running in parallel several instances of the compact e-cash scheme each associated with a different  denomination. This would reduce dramatically the number of coins that need to be spent in a payment. However, to our knowledge, an efficiency analysis of compact e-cash schemes with multiple denominations has not been conducted.


% Most anonymous e-cash schemes~\cite{DBLP:conf/eurocrypt/CamenischHL05,DBLP:conf/pairing/BelenkiyCKL09,DBLP:conf/eurocrypt/CanardG07,DBLP:conf/acns/CanardPST15,DBLP:conf/pkc/CanardPST15,DBLP:conf/pkc/PointchevalST17,DBLP:conf/asiacrypt/BoursePS19,DBLP:conf/crypto/OkamotoO89,DBLP:conf/crypto/OkamotoO91,DBLP:conf/pkc/BaldimtsiCFK15,DBLP:conf/pkc/BauerFQ21} are unsuitable for distributed environments and incompatible with existing blockchain systems because of the use of a central bank. 
% Another possible solution to remove the central bank is to distribute the responsibility among a quorum of authorities using threshold cryptography.
% This solution has been explored recently in the context 
% of attribute-based credentials~\cite{DBLP:conf/ndss/SonninoABMD19} and online anonymous e-cash schemes~\cite{DBLP:journals/iacr/BaudetSKD22} although without security proofs. 
% While the functionality executed by the bank could be replaced with an MPC protocol, that implies the need for the authorities to communicate with each other during coins issuance, rendering a much more inefficient scheme than when using threshold cryptography. 
In order to address the urgent scalability issues of blockchain systems and possibly dangerous centralization of CBDCs without privacy, a formal treatment of offline e-cash is needed, including a fair comparison of compact and divisible e-cash. 

% \alf{Somewhere we need to explain that the reason why we added e-cash transfers is to protect the privacy of providers in the deposit phase (and thanks to that, our schemes can offer privacy for the parties that receive payments, like Zcash or similar. At the moment, I'm unsure whether adding e-cash transfer to this paper is a good idea. One  problem could be that a reviewer could realize that, with e-cash transfers, giving change is possible, and thus a more efficient offline e-cash scheme can be designed. A second problem is that reviewers may think that, if e-cash transfers are important for our system, then they should be added to the main body, leading to a second major revision. If we remove them, the problem could be that a reviewer suggests that our schemes do not protect the privacy of payment recipients like Zcash, but they haven't realized that in the first submission, and we can always argue that Zcash and other schemes are not offline.}

%nother alternative for removing the need for a central bank is given by the concept of decentralized private e-cash~\cite{DBLP:conf/sp/MiersG0R13,DBLP:conf/ccs/DanezisFKP13,DBLP:journals/iacr/Ben-SassonCG0MTV14}, in which coins are authenticated by proving in ZK that they belong to a public list of valid coins maintained on the blockchain. Other methods can also be used to provide private decentralized payments, like those used in e.g. Monero~\cite{cryptonote,DBLP:journals/ledger/NoetherM16}, Dash~\cite{dash} or Mimblewimble~\cite{DBLP:conf/eurocrypt/FuchsbauerOS19}. However, all of those approaches have the drawback of being online, i.e.\ a payment needs to be checked to detect double-spending before it can be accepted by a provider. This requires merchants to be constantly connected with the blockchain. 
%As recently shown in the example of Zcash, the online approach might be exploited by attackers performing spam attacks~\cite{zcashattack1, zcashattack2, zcashattack3}.  
%Offline schemes are advantageous because they enable faster payments with offline providers and do not require publishing large payments on the blockchain.  

%There are proposals to enable offline payments in blockchain-based cryptocurrencies, however, none of them offers strong privacy protection. In~\cite{10.1145/3052973.3052980}, a protocol for offline payments in Bitcoin is described in which users who double spend can be revoked. We remark that, in that protocol, users who make a payment are not anonymous. The user's public key is revealed to the recipient of the payment, and later the public key can be used for revocation if double spending is found. In contrast, in an anonymous e-cash, a user's public key can only be obtained from payments if the user double spent. 
%Alternatively, payment channel networks~\cite{DBLP:conf/ccs/0001M17} allow making arbitrarily many off-chain payments between channel creators. However, only users who share a channel can transact with each other, and
%similarly as in the case  of~\cite{10.1145/3052973.3052980} users who make a payment are not anonymous~\cite{DBLP:conf/fc/KapposYPKDMM21}. Moreover, cashing out payments still requires blockchain interactions. 
%Therefore, offline anonymous e-cash schemes with threshold issuance offer
%an interesting alternative. 
% Recent market research on CBD released by the ECB further shows increasing interest in the offline approach to electronic payments~\cite{digitaleuro}. 



% \textit{\textbf{Outline}}
% In \S\ref{sec:offlineecash}, we define the system model, security properties and the ideal functionality $\Functionality_{\CEC}$   
% of an offline anonymous e-cash with threshold issuance ($\CEC$) scheme. In Section~\ref{sec:constructionCEC}, we describe the construction which realises the ideal functionality. 
% In Section~\ref{sec:instantiation}, we present two instantiations of $\Functionality_{\CEC}$ using compact and divisible e-cash schemes. 
% In \S\ref{sec:efficiencycomparison}, we analyze and compare those schemes in terms of efficiency. 
% In Section~\ref{sec:integration}, we discuss how our schemes can be further extended to provide transferability and integrated with a blockchain-based bulletin board. 
% We conclude in~\S\ref{sec:conclusion}.
% In~\S\ref{sec:securitydefinitionsbuildingblocks}, we define the security properties of the cryptographic primitives used in our $\CEC$ schemes. 
% In \S\ref{sec:securityProofCompact} and \S\ref{sec:securityProofDivisible}, we provide a formal security analysis to show that $\mathrm{\Pi}_{\CEC}$ realizes $\Functionality_{\CEC}$ when instantiated with the algorithms of our compact and divisible e-cash schemes. 
% Finally, in \S\ref{sec:compactECashRangeProof}, we describe an instantiation of our compact $\CEC$ scheme with a concrete set membership proof, and in \S\ref{sec:divisibleEcashSpend}, we describe how the NIZK arguments of our divisible $\CEC$ scheme are instantiated.




% Our compact $\CEC$ scheme is based on the compact e-cash scheme in~\cite{DBLP:conf/eurocrypt/CamenischHL05}, which we choose for its efficiency. Our divisible $\CEC$ scheme is based on the divisible e-cash scheme in~\cite{DBLP:conf/pkc/PointchevalST17}, which is appealing because it allows the user to spend any number of coins. In~\cite{DBLP:conf/asiacrypt/BoursePS19}, it is shown that there is a security problem in the standard model instantiations of previous divisible e-cash schemes, including~\cite{DBLP:conf/pkc/PointchevalST17}. However, since we aim at designing practical schemes, both our compact and our divisible $\CEC$ schemes use the random oracle model, with non-interactive zero-knowledge (ZK) arguments of knowledge computed via the Fiat-Shamir heuristic, and they avoid that security problem. In~\cite{DBLP:conf/asiacrypt/BoursePS19}, a framework for divisible e-cash schemes is proposed, which allows the design of secure schemes in both the random oracle and the standard model.

% In both~\cite{DBLP:conf/eurocrypt/CamenischHL05,DBLP:conf/pkc/PointchevalST17}, a wallet is a signature obtained by a user through a blind signing protocol run with the bank. In our schemes, the bank is replaced by the authorities $(\fcecAuthority_1, \allowbreak \ldots, \allowbreak \fcecAuthority_n)$. To enable threshold issuance, we instantiate the signature scheme used to compute wallets with Pointcheval-Sanders (PS) signatures~\cite{DBLP:conf/ctrsa/PointchevalS16}, and we leverage the threshold issuance protocol for PS signatures in~\cite{DBLP:conf/ndss/SonninoABMD19} with the ~~

% We modify the scheme in~\cite{DBLP:conf/eurocrypt/CamenischHL05} to improve its efficiency. In~\cite{DBLP:conf/eurocrypt/CamenischHL05}, a wallet is a signature on $(\ssk_{\fcecUser_j}, \allowbreak \cecsn, \allowbreak \cect)$, where $\ssk_{\fcecUser_j}$ is a user secret key and $\cecsn$ and $\cect$ are coin secrets such that, in the spending phase, $\cecsn$ is used to compute serial numbers and $\cect$ is used to compute double spending tags. In our scheme, a wallet is a signature on $(\ssk_{\fcecUser_j}, \allowbreak \cecsn)$, i.e. only one coin secret is needed. This allows us to reduce the wallet size and to improve the efficiency of the withdrawal and spend protocols in comparison to~\cite{DBLP:conf/eurocrypt/CamenischHL05}.

% In both~\cite{DBLP:conf/eurocrypt/CamenischHL05,DBLP:conf/pkc/PointchevalST17}, in the blind signature protocol, both the user and the bank contribute randomness to generate the coin secret $\cecsn$. In our schemes, the user chooses $\cecsn$, which improves the efficiency of the blind signature protocol. As also noted in~\cite{DBLP:conf/asiacrypt/BoursePS19}, this change does not compromise the security of our schemes.

% We compare our compact $\CEC$ scheme when using multiple denominations with our divisible $\CEC$ scheme. To this end, we analyze the issue of choosing the optimal set of denominations. Then, for several sets of chosen denominations and price ranges, we calculate the average number of coins that need to be spent.


% In the deposit phase, our compact $\CEC$ scheme clearly outperforms our divisible $\CEC$ scheme. The reason is twofold. First, in the compact $\CEC$ scheme, an authority only needs to compare serial numbers to detect double spending, whereas in the divisible $\CEC$ scheme, the authority, given a payment, needs to compute the serial numbers for that payment before comparing them with the serial numbers of other payments. Second, when double spending is detected, the compact $\CEC$ scheme can identify the user guilty of double spending with cost independent of the number of users, whereas in the divisible e-cash scheme the cost of identification grows linearly with the number of users.



% \ania{TODO: explain why our design is more interesting than Coconut or Zef, aka what are their limitations.}
% In this work, we propose two offline anonymous e-cash schemes with threshold issuance. First, we describe a compact e-cash scheme based on the scheme in~\cite{DBLP:conf/eurocrypt/CamenischHL05}. Second, we describe a divisible e-cash scheme based on the scheme in~\cite{DBLP:conf/pkc/PointchevalST17}. For both schemes, the protocol for threshold issuance is based on~\cite{DBLP:conf/ndss/SonninoABMD19} with the changes described in~\cite{cryptoeprint:2022:011}. To the best of our knowledge these are the first offline e-cash schemes with threshold issuance.
% To the best of our knowledge, no offline e-cash schemes with threshold issuance have yet been proposed.

% \vspace{2mm}
% \noindent \textbf{Our contributions:}
% In this paper, we make the following contributions.
% \begin{itemize}[noitemsep,topsep=0pt]
%     \item \ania{TODO: Needs updating} In~\cite{DBLP:conf/eurocrypt/CamenischHL05,DBLP:conf/pkc/PointchevalST17}, a wallet is a signature obtained by a user through a blind signing protocol run with the bank. Both our schemes remove the central bank in~\cite{DBLP:conf/eurocrypt/CamenischHL05,DBLP:conf/pkc/PointchevalST17} and consider instead $n$ authorities $(\fcecAuthority_1, \allowbreak \ldots, \allowbreak \fcecAuthority_n)$. In the setup phase, a verification key $\spk$ for wallets is generated, and each authority receives a secret key share for $\spk$. In the withdrawal phase, to obtain a wallet, a user needs to run a blind signing protocol with at least $t \allowbreak \leq \allowbreak n$ authorities. Each authority issues a signature by using its secret key share, and $t$ signatures from $t$ different authorities can be turned into a signature verifiable with $\spk$.

    % The spending and deposit phase of our compact e-cash scheme follows the scheme in~\cite{DBLP:conf/eurocrypt/CamenischHL05}. Our divisible e-cash scheme is a version in the random oracle model of the scheme in~\cite{DBLP:conf/pkc/PointchevalST17}, which is secure in the standard model. (\alf{Mention here the security problem in~\cite{DBLP:conf/pkc/PointchevalST17}}.) This change allows us to simplify the scheme in~\cite{DBLP:conf/pkc/PointchevalST17} and to improve its efficiency.

    % On an abstract level, the difference between the compact e-cash scheme and the divisible e-cash scheme is the following. In compact e-cash, serial numbers of each of the coins spent are generated by the user during the spending phase, and thus the cost of this phase grows with the number of coins spent. In divisible e-cash, the serial numbers are generated by the authority in the deposit phase, with information provided by the user during the spending phase and additional public parameters generated during the setup phase.

    % \item Next, we formally prove the security of our schemes in the ideal-world/real-world paradigm~\cite{DBLP:conf/focs/Canetti01}.
    % To this end, we define the ideal functionality for offline anonymous e-cash with threshold issuance. \ania{Is there something more you would like to highlight about the security analysis?}
    %which we briefly describe in~\S\ref{sec:securitymodel}.


%\end{itemize}



\section{Background}
\label{sec:background}

We begin by reviewing existing SSL approaches with a special focus on relevant methods in the low-label regime. 

\smallskip
\noindent \textbf{Confidence-based pseudo-labeling} is an integral component in most of recent SSL methods~\cite{Sohn_fixmatch20, li2021comatch, wang2022debiased, nassar2021all, kuo2020featmatch}. However, recent research shows that using a fixed threshold underperforms in low-data settings because the model collapses to the few easy-to-learn classes early in the training. Some researchers combat this effect by using class-~\cite{zhang2021flexmatch} or instance-based~\cite{xu2021dash} adaptive thresholds, or by aligning~\cite{Berthelot_RemixMatch19} or debiasing~\cite{wang2022debiased} the pseudo-label distribution by keeping a running average of pseudo-labels to avoid the inherent imbalance in pseudo-labels. Another direction focuses on pseudo-label refinement, whereby the classifier's predictions are adjusted by training another projection head on an auxiliary task such as weak-supervision via language semantics~\cite{nassar2021all}, instance-similarity matching~\cite{zheng2022simmatch}, or graph-based contrastive learning~\cite{li2021comatch}. Our method follows the refinement approach, where we employ online constrained clustering to leverage nearest neighbours information for refinement. Different from previous methods, our method is fully online and hence allows using the entire prediction history in one training epoch to refine pseudo-labels in the subsequent epoch with minimal memory requirements.    

\smallskip
\noindent \textbf{Consistency Regularization} combined with pseudo-labeling underpins many recent state-of-the-art SSL methods~\cite{xie2019_uda, li2019decoupled, Sajjadi_NIPS16, Berthelot_MixMatch19, Sohn_fixmatch20, li2021comatch, kuo2020featmatch}; it exploits the smoothness assumption~\cite{van2020survey} where the model is expected to produce similar pseudo-labels for minor input perturbations. The seminal FixMatch~\cite{Sohn_fixmatch20} and following work~\cite{nassar2021all, li2021comatch, wang2022debiased} leverage this idea by obtaining pseudo-labels through a weak form of augmentation and applying the loss against the model's prediction for a strong augmentation. Our method utilises a similar approach, but different from previous work, we additionally apply an instance-consistency loss in our projection embedding space.

% adding a loss to enforce prediction consistency over semantic-preserving augmentations of unlabeled images. Berthelot \etal~\cite{Berthelot_MixMatch19} average predictions across multiple augmentations whereas more recent methods~\cite{Sohn_fixmatch20, nassar2021all, li2021comatch, wang2022debiased} use a pair of weak and strong augmentations for each image. 

\smallskip
\noindent \textbf{Semi-supervision via self-supervision} is gaining recent popularity due to the incredible success of self-supervised learning for model pretraining. Two common approaches are: 1) performing self-supervised pretraining followed by supervised fine-tuning on the few labeled samples~\cite{chen2020simple, chen2020big, grill2020bootstrap, nassar2022lava, caron2021emerging}, and 2) including a self-supervised loss to the semi-supervised objective to enhance training~\cite{zhai2019s4l, lucas2022barely, wallin2022doublematch, li2021comatch, zheng2022simmatch}. However, the choice of the task is crucial: tasks such as instance discrimination~\cite{he2020momentum, chen2020simple}, which treats each image as its own class, can hurt semi-supervised image classification as it partially conflicts with it. Instead, we use an instance-consistency loss akin to that of~\cite{caron2021emerging} to boost the initial training signal by leveraging samples which are not retained for pseudo-labeling in the early phase of the training.






\section{Video-Instruction Alignment}
\label{sec:method}

\begin{figure*}
    \centering
    \includegraphics[width=0.99\linewidth]{img/method.pdf}
    \caption{Visualization of our three losses described in~\secref{sec:losses}. The intent is depicted in the first row and the batch formation in the second row. Loss (a) tries to pair video and image up across the entire dataset. Loss (b) only matches video clips and images corresponding to the same manual. And loss (c) push images from the same manual apart from each other for better feature discrimination.}
    \label{fig:losses}
\end{figure*}

We formulate the task of aligning video segments to instructional diagrams as a variant of video-to-image matching.
Here the idea is to retrieve the image from a candidate set that most closely depicts the activity occurring in the short video clips and vice versa.
Importantly, since different instructional videos will involve different numbers of steps the candidate set will necessarily have variable cardinality (unlike, say, multi-class classification tasks).

Formally, given a set of $N$ video clips $\{V_i\}_{i=1}^{N}$ and a set of $M$ instructional diagrams $\{I_j\}_{j=1}^{M}$, our goal is to train a model to predict the correspondence between diagrams and clips.
A standard approach for addressing this problem is to learn a joint embedding space for videos and diagrams such that matching video-diagram pairs map near to each other in the embedding space.
Let $\mathbf{f}_i^\text{V}$ and $\mathbf{f}_j^{\text{I}}$ denote the feature embedding for the $i$-th video clip and $j$-th instructional diagram, respectively, and let $f_{\text{sim}}$ be some similarity measure.
Then, once the embedding space is learned we can use the model to predict the index of the instructional diagram corresponding to a given video clip $V$ as
%
\begin{align}
    j^\star & = \mathop{\text{argmax}}_{j=1,\dots,M} f_{\text{sim}}(\mathbf{f}^\text{V}, \mathbf{f}_j^\text{I}).
\end{align}
%
Likewise, we can find the video segment that most closely matches a given instructional diagram $I$ as
%
\begin{align}
    i^\star & = \mathop{\text{argmax}}_{i=1,\dots,N} f_{\text{sim}}(\mathbf{f}_i^\text{V}, \mathbf{f}^\text{I}).
\end{align}
%
This can be generalized to top-$k$ retrieval.
Last, we can enforce matching constraints, such as through optimal transport or dynamic time warping if order information is available, to jointly match all clips in a video to all steps in an instruction manual.
\figref{fig:model} depicts the overall model.

In this work, we use cosine similarity for $f_{\text{sim}}$.
The embedding vectors $\mathbf{f}_i^\text{V}$ and $\mathbf{f}_j^\text{I}$ are computed using video and image encoders trained under a contrastive loss and optionally augmented with temporal features such as we now describe.

\subsection{Sinusoidal Progress Rate Feature}
\label{sec:sprf}
Instruction manuals contain an ordered sequence of steps that is typically, although not always, followed during the assembly process.
However, the time needed to perform each step varies greatly depending on complexity of the step and experience of the assembler.
This suggests a weak correlation between (proportional) timestamps in the video and progress through the assembly process.
We can make use of this prior by including temporal ordering information in the video and diagram feature representations.

Given a video clip $V$ sampled from a full video of length $t_\text{duration}$ seconds, with start time $t_\text{start}$ and end time $t_\text{end}$, we define the video progress rate $r^\text{V}$ of that video clip as
%
\begin{align}
    r^\text{V} & =(t_\text{start}+t_\text{end})/2 t_\text{duration}
\end{align}
%
and the instructional diagram progress rate $r^\text{I}$ for the $j$-th step from a manual with $M$ total steps is simply $ r^\text{I} = j / M$.
%
Because we are using a cosine similarity function $f_\text{sim}$, we map the progress feature onto a half circle so that high similarity score coincides with when they align.
The final sinusoidal progress rate feature (SPRF) is then
%
\begin{align}
    (\sin(\pi r^V), \cos(\pi r^V))
\end{align}
%
for video and similarly for the instructional diagram, which we append to the feature embeddings extracted from the respective encoders (see~\figref{fig:model}).
Before and after the concatenation, the features are L2 normalized to alleviate side-effect due to fluctuation of numerical value scale.
Two fully connected layers then project each modality feature into the same dimensional space to form representations $\mathbf{f}^I$ and $\mathbf{f}^V$ for further similarity comparison.

\subsection{Training Losses}
\label{sec:losses}

Starting with pre-trained video and image encoders we finetune our model using variants of contrastive learning, which has recently been made popular for cross-modal matching by models like CLIP~\cite{radford2021learning}.
In this setting mini-batches are constructed by sampling video clip-instructional diagram pairs $(V_i, I_i)$ to optimize an infoNCE loss~\cite{hadsell2006dimensionality,oord2018representation} where pairs $(V_i, I_i)$ are considered positive and $(V_i, I_j)$, $i \neq j$ are considered negative.
Here we sample randomly from all videos and instruction manuals in the training data.

Formally, for mini-batch containing $B$ pairs, define
%
\begin{align}
    p^{V2I}_{ij} & = \frac{\exp(f_\text{sim}(\boldf^V_i,\boldf^I_j)/\tau)}{\sum_{b=1}^B{\exp(f_\text{sim}(\boldf^V_i,\boldf^I_b)/\tau)}}
    \label{eqn:video_to_image}
    \\
    p^{I2V}_{ji} & = \frac{\exp(f_\text{sim}(\boldf^V_i,\boldf^I_j)/\tau)}{\sum_{b=1}^B{\exp(f_\text{sim}(\boldf^V_b,\boldf^I_j)/\tau)}}
    \label{eqn:image_to_video}
\end{align}
%
to be the probability of matching video $V_i$ to image $I_j$ and the probability of matching image $I_j$ to video $V_i$, respectively.
Here $\tau$ is a temperature parameter that controls the bias towards difficult examples~\cite{wang2021understanding}.
%
Standard contrastive learning then minimizes
%
\begin{align}
    \Ell_\text{infoNCE} & = -\frac{1}{2B}\left(\sum_{i=1}^B \log p^{V2I}_{ii} + \sum_{j=1}^B \log p^{I2V}_{jj}\right).
\end{align}

We note that this vanilla version of contrastive learning does not consider situations where there may be many-to-one matches between pairs.
Specifically, in our application multiple video clips may map to the same step.
We introduce a specialized loss to deal with this scenario.

\noindent\textbf{Video-Diagram Contrastive Loss (\figref{fig:losses}(a)).}
%
Contrastive learning frameworks benefit from large batch sizes~\cite{radford2021learning}.
However, as batch size increases there is a greater chance that we sample multiple videos matching to the same diagram within the batch, which violates the assumptions of the infoNCE loss.
To better handle these cases we build on the work of~\cite{wang2021actionclip} that introduces a Kullback-Leibler (KL) divergence loss between predicted and ground truth distributions, $\bp$ and $\bq$, respectively.
However, rather than KL-divergence, we prefer Jensen-Shannon (JS) divergence, which we find improves training stability.

Let $\bp^{V2I}$ and $\bp^{I2V}$ be vectors containing all video-to-diagram and diagram-to-video probabilities introduced in Eqs.
\ref{eqn:video_to_image} and \ref{eqn:image_to_video}, respectively.
Similarly, let $\bq^{V2I}$ and $\bq^{I2V}$ be the corresponding ground truth alignment distributions.
Then our video-diagram contrastive loss is defined as
%
\begin{align}
    \Ell^\text{VI} & = \frac{1}{2}\left(D_{JS}(\bp^{V2I} \| \bq^{V2I}) + D_{JS}(\bp^{I2V} \| \bq^{I2V})\right)
\end{align}
%
where $D_{JS}$ is the Jensen-Shannon divergence.

\noindent\textbf{Video-Manual Contrastive Loss (\figref{fig:losses}(b)).}
%
The above losses align video and diagram pairs globally across the entire training dataset.
However, for our task we know that a given video clip only needs to match against one of the steps in its corresponding instruction manual, not other manuals.
Hence, we can perform a more task-specific discrimination by exploiting this prior information in the model.
To do so we modify our procedure for constructing a mini-batch to first sample a video clip $V_i$ and then include all instructional diagrams $\{I_1, \ldots, I_{M_i}\}$ from the video's corresponding manual.
One of these diagrams will be the ground truth positive match for the clip.
We then employ a classification loss based on cross entropy (CE) as
%
\begin{align}
    \Ell^{VM} & = \sum_{i=1}^B\frac{M_i}{\sum_{b=1}^B M_b} CE(\bp_{i}^{V2I}, \bp_{i}^{gt})
\end{align}
%
where $M_i$ indicates the length of the manual corresponding to the $i$-th video.
Here $\bp_{i}^{V2I} \subseteq (p^{V2I}_{ij})_{j=1}^{B}$ is a subvector of probabilities for matching video $V_i$ to all diagrams $I_j$ from the corresponding manual and $\bp_i^{gt}$ is the associated one-hot ground truth encoding.
We weight each term in the loss by $\frac{M_i}{\sum_{b=1}^B M_b}$ to give more emphasis to more difficult assemblies, assumed to be the ones containing more steps.

\noindent\textbf{Intra-Manual Contrastive Loss (\figref{fig:losses}(c)).}
%
The previous losses only consider contrasting embeddings between videos and diagrams.
However, most furniture assembly tasks involve a progressive process where the visual similarity between successive steps is large.
Indeed, the main component of the assembly is often introduced early in the assembly process and dominates the instructional diagram.
This makes it challenging to distinguish between steps.
To encourage diagrams from the same manual to be spread out in embedding space, so that they are more easily distinguished, we introduce an intra-manual contrastive loss.

Similar to the video-to-diagram and diagram-to-video matching probabilities defined above, let
%
\begin{align}
    p^{I2I}_{jk} & = \frac{\exp(f_\text{sim}(\boldf^I_j,\boldf^I_k)/\tau)}{\sum_{m=1}^M{\exp(f_\text{sim}(\boldf^I_j,\boldf^I_m)/\tau)}}
\end{align}
%
be the probability of matching diagram $I_j$ to diagram $I_k$ from the same manual according to our similarity metric.
Then we define our intra-manual contrastive loss as
%
\begin{align}
    \Ell^M & = \sum_{j=1}^B\frac{M_j}{\sum_{b=1}^B M_b} D_{JS}\left(\bp_j^{I2I} \| \N(j, \theta)\right)
\end{align}
%
where $\bp_j^{I2I}$ is the softmax normalized diagram-to-diagram probability vector associated with diagram $I_j$,
and $\N(j, \theta)$ is a univariate Gaussian distribution with mean $j$, learnable variance $\theta$
and discretized on support $\{1, \ldots, M_j\}$.
This encourages distances in diagram embedding space to correspond to distances between steps in the manual.
We use a normal distribution instead of a delta distribution as a relaxation since nearby negative diagrams are still likely to share some semantics.

\subsection{Set Matching}
\label{sec:ot}
Our model is very general.
Given a single video clip we can retrieve the most likely diagram showing the assembly step and given a single diagram we can retrieve a set of best matching video clips.
To align an entire video (sequence of clips) to an entire instruction manual, we can add approximate one-to-one matching priors or temporal constraints, through optimal transport (OT)
or dynamic time warping (DTW), respectively.
As we will see in our experiments, the absence of temporal order constraints in OT slightly outperforms DTW due to occasional out-of-order execution of assembly steps or strong false matches.

To apply either method we first extract features $\boldf^V_i$ for an entire video $\{V_i\}_{i=1}^N$ and $\boldf^I_j$ for all instructional diagrams in the corresponding manual $\{I_j\}_{j=1}^M$.
Denote by $s_{ij}$ the similarity $f_\text{sim}(\boldf_i^V, \boldf_j^I)$ between video clip $V_i$ and diagram $I_j$.
Let $\overline{s} = \max_{i,j} s_{ij}$ and $\underline{s} = \min_{ij} s_{ij}$.
We then construct a cost matrix $C \in \reals^{N \times M}$ with entries
%
\begin{align}
    C_{ij} & = \frac{s_{ij}^\alpha - \underline{s}^\alpha}{\overline{s}^\alpha -  \underline{s}^\alpha}.
\end{align}
%
Here $\alpha > 1$ accentuates the similarity differences and the normalization by $\overline{s}^\alpha - \underline{s}^\alpha$ restricts the range of $C_{ij}$ to $[0, 1]$.
The optimal transportation plan $T^{\star}$ obtained by solving the entropy regularized optimal transport problem,
%
\begin{align}
    \begin{array}{rl}
        \text{minimize}   & \sum_{i=1}^{N} \sum_{j=1}^{M} T_{ij} C_{ij} - \epsilon H(T)           \\
        \text{subject to} & \sum_{i=1}^{M} T_{ij} = \frac{1}{N}, \,\text{for $j = 1, \ldots, N$}  \\
                          & \sum_{j=1}^{N} T_{ij} = \frac{1}{M}, \,\text{for $i = 1, \ldots, M$},
    \end{array}
\end{align}
%
gives the joint probability of matching videos and diagrams.
It can be found efficiently by applying the Sinkhorn-Knopp algorithm~\cite{sinkhorn1967diagonal} to the optimization problem defined above.

In a similar fashion, we can use DTW to find the optimal path through the cost matrix to give the most likely matching subject to the ordering constraint that later video clips cannot match to earlier instructional diagrams and vice versa.
More formally, if video clip $V_i$ matches to diagram $I_j$ then clip $V_{i+1}$ cannot match to diagram $I_{j'}$ with $j' < j$ and diagram $I_{j+1}$ cannot match to video clip $V_{i'}$ with $i' < i$.


\section{Experiments}
\label{experiments}

We begin by validating \putouralg's performance on multiple SSL benchmarks against state-of-the-art methods. Then, we analyse the main components of \putouralg to verify their contribution towards the overall performance, and we perform ablations on important hyperparameters.

\subsection{Experimental Settings}
\noindent \textbf{Datasets.} We evaluate \putouralg on five SSL benchmarks. Following~\cite{Sohn_fixmatch20, xie2019_uda, arazo_pseudo}, we evaluate on \textbf{CIFAR-10(100)}~\cite{cifar100} datasets, which comprises 50,000 images of 32x32 resolution of 10(100) classes; as well as the more challenging \textbf{Mini-ImageNet} dataset proposed in~\cite{mini_imagenet}, having 100 classes with 600 images per class (84x84 each). We use the same train/test split as in~\cite{label_prop} and  create splits for 4 and 10 labeled images per class to test \putouralg in the low-label regime. We also test \putouralg's performance on the \textbf{DomainNet}~\cite{peng2019moment_domainnet} dataset, which has 345 classes from six visual domains: \emph{Clipart}, \emph{Infograph}, \emph{Painting}, \emph{Quickdraw}, \emph{Real}, and \emph{Sketch}. We evaluate on the \emph{Clipart} and \emph{Sketch} domains to verify our method's efficacy in different visual domains and on imbalanced datasets.  Finally, we evaluate on \textbf{ImageNet}~\cite{russakovsky2015imagenet_large} SSL protocol as in~\cite{caron2020unsupervised, chen2020simple, caron2021emerging, assran2021semi}.
% , whereby a percentage of the labeled data is used to train the model together with all the unlabeled data. 
In all our experiments, we focus on the low-label regime. %Wwe also run one experiment in Mini-ImageNet on moderate data regime to have a more comprehensive view.

\smallskip
\noindent \textbf{Implementation Details.}
For CIFAR-10(100), we follow previous work and use WideResent-28-2(28-8)~\cite{zagoruyko2016_wideresnet} as our encoder. We use a 2-layer projection MLP with an embedding dimension $d=64$. The models are trained using SGD with a momentum of 0.9 and weight decay of 0.0005(0.001) using a batch size of 64 and $\mu=7$. We set the threshold $\tau=0.95$ and train our models for 1024 epochs for a fair comparison with the baselines. However, we note that our model needs substantially fewer epochs to converge (see Fig.~\ref{fig:analysis_plots}-b and c). We use a learning rate of 0.03 with a cosine decay schedule. We use random horizontal flips for weak augmentations and RandAugment~\cite{cubuk2020_randaugment} for strong ones. For the larger datasets: ImageNet and DomainNet, we use a Resnet-50 encoder and $d=128$, $\mu=5$ and $\tau=0.7$ and follow the same hyperparameters as in~\cite{Sohn_fixmatch20} except that we use SimCLR~\cite{chen2020simple} augmentations for the strong view. For \putouralg-specific hyperparameters, we consistently use the same parameters across all experiments: we set $n$ to 250 (corresponding to $K$=200 for CIFARs, and Mini-ImageNet, and 4800 for ImageNet), and dual learning rate $\lambda = 20$, mixing ratio $\alpha=0.8$, and temperature $T=0.1$.


\noindent \textbf{Baselines.}
Since our method bears the most resemblance with CoMatch~\cite{li2021comatch}, we compare against it in all our experiments. CoMatch uses graph contrastive learning to refine pseudo-labels but uses a memory bank to store the last n-samples embeddings to build the graph. Additionally, we compare with state-of-the-art SSL method (DebiasPL)~\cite{wang2022debiased}, which proposes a pseudo-labeling debiasing plug-in to work with various SSL methods in addition to an adaptive margin loss to account for inter-class confounding. Finally, we also compare with the seminal method FixMatch and its variant with Distribution alignment (DA). We follow Oliver~\etal~\cite{oliver_realistic} recommendations to ensure a fair comparison with the baselines, where we implement/adapt all the baselines using the same codebase to ensure using the same settings across all experiments. As for ImageNet experiments, we also compare with representation learning baselines such as SwAV~\cite{caron2020unsupervised}, DINO~\cite{caron2021emerging}, and SimCLR~\cite{chen2020simple}, where we report the results directly from the respective papers. We also include results for \putouralg and DebiasPL with additional pretraining (using MOCO~\cite{he2020momentum}) and the Exponential Moving Average Normalisation method proposed by~\cite{cai2021exponential} to match the settings used in~\cite{wang2022debiased, cai2021exponential}.

\subsection{Results and Analysis}
\noindent \textbf{Results.}
Similar to prior work, we report the results on the test sets of respective datasets by averaging the results of the last 10 epochs of training. For CIFAR and Mini-ImageNet, we report the average and standard deviation over 5 different labeled splits, whereas we report for only 1 split on larger datasets (ImageNet and DomainNet). Different from most previous work, we only focus on the very low-label regime (2, 4, and 8 samples per class, and 0.2\% for ImageNet). As shown in Tab.~\ref{tab:cifar_results} - 
\ref{tab:imagenet_results}, we observe that \putouralg outperforms baselines in almost all the cases showing a clear advantage in the low-label regime. It also exhibits less variance across the different splits (and the different runs within each split). These results suggest that besides achieving high accuracy, \putouralg shows robustness and consistency across splits in low-data regime.

% \MH{I think we can use the next para to highlight our methods strengths. Right now, it sounds like, thanks to DINO, our method works well on DomainNet. We're only using part of DINO, so, the self-supervised loss is tailord to fit with rest of our architecture. please consider revising the para below to highlight our strengths}
% Surprisingly, \putouralg performs particularly well on DomainNet. Although, we are yet to investigate further the root cause behind such specific improvement, we hypothesise that this is due to including the self-supervised loss which already learns rich robust representations without any label as shown in recent work~\cite{caron2021emerging}. This is consistent with recent findings~\cite{nassar2022lava} which show consistent performance gains in visual transfer learning when using a similar self-supervised loss for pretraining. 
Notably, our method performs particularly well on DomainNet. Unlike ImageNet and CIFARs, DomainNet is an imbalanced dataset, and prior work~\cite{tan2020class} shows that it suffers from high level of label noise. This shows that our method is also more robust to noisy labels. This can be explained in context of our co-training approach:  using the prototypical neighbourhood label to smooth the softmax label is an effective way to minimise the effect of label noise. In line with previous findings~\cite{li2021learning}, since in prototypical learning, all the instances of a given class are used to calculate a class prototype which is then used as a prediction target, it results in representations which are more robust to noisy labels.  

Finally, on ImageNet (Tab.~\ref{tab:imagenet_results}), we improve upon the closest baseline with gains of 2.2\% in the challenging 0.2\% setting; whereas we slightly fall behind PAWS~\cite{assran2021semi} in the 10\% regime, again confirming our method's usefulness in the label-scarce scenario.

\begin{figure*}[h!]
 \centering
  \scalebox{0.99}{\includegraphics[width=0.97\textwidth]{figures/analysis_plots.pdf}}
 \caption{\textbf{Analysis Plots.} {\bf (a)}: Average disagreement between cluster and classifier pseudo-labels versus ground truth accuracy of the different pseudo-labels. The accuracy gap between refined pseudo-labels (green) and the cluster's and classifier's (blue and orange) decreases with disagreement rate (dashed black) showing that refinement indeed helps. {\bf (b), (c):} Convergence plots on CIFAR10/100 show that \putouralg converges faster due to the additional self-supervised training signal. {\bf (d): } \putouralg w and w/out consistency loss. }
 \label{fig:analysis_plots}
 \vspace{-2mm}
\end{figure*}

\begin{figure}[h!]
 \centering
  \scalebox{0.99}{\includegraphics[width=0.47\textwidth]{figures/image_comparison.pdf}}
 \caption{The middle panel shows the most prototypical images of CIFAR-10 classes as identified by our model. Left (resp. right) panels show images which have more accurate classifier (resp. cluster) pseudo-labels. Cluster labels are more accurate for prototypical images while classifier labels are more accurate for images with distinctive features (\eg truck wheels) even if not so prototypical. Such diversity of views is key to the success of our co-training method.}
 \label{fig:image_comparison}
 \vspace{-2mm}
\end{figure}

\begin{table}[t]
\centering
\caption{\small SSL results on ImageNet with different percentage of labels. $\dagger$ denotes results produced by our codebase. Other results are reported as appearing in the cited work. }
\label{tab:imagenet_results}
% \vspace{-2mm}
\setlength{\tabcolsep}{2pt}
\scalebox{0.9}{
    \begin{tabular}{l r c c c c}
        % & & & \multicolumn{2}{c}{\small Top 1}\\
        \small Method & \small Pre. & \small Epochs & \small 0.2\% & \small 1\% & \small 10\% \\\toprule [0.15em]
        Supervised & \xmark & 300 & -- & 25.4 & 56.4\\ \midrule[0.15em]
        \multicolumn{4}{l}{\footnotesize\itshape Representation learning methods:}\\[1mm]
        SwAV~\cite{caron2020unsupervised} & \cmark & 800 & -- & 53.9 & 70.2 \\
        SimCLRv2++ ~\cite{chen2020big} & \cmark & 1200 & --  & 60.0 & 70.5 \\
        DINO~\cite{caron2021emerging} & \cmark & 300 & -- & 55.1 & 67.8 \\
        PAWS++ ~\cite{assran2021semi} & \cmark & 300 & --  & 66.5 & \textbf{75.5} \\\midrule[0.15em]
        \multicolumn{4}{l}{\footnotesize\itshape PL \& consistency methods:}\\[1mm]
        MPL~\cite{pham2021meta} & \xmark & 800 & -- & $65.3^\dagger$ & 73.9 \\
        CoMatch~\cite{li2021comatch} & \xmark & 400 & $44.3^\dagger$ & 66.0 & 73.6 \\
        FixMatch~\cite{Sohn_fixmatch20} & \xmark & 300 & -- & 51.2 & 71.5 \\
        FMatch + DA~\cite{Sohn_fixmatch20, Berthelot_RemixMatch19} & \xmark & 300 & $41.1^\dagger$ & 53.4 & $71.5^\dagger$ \\
        FMatch + EMAN~\cite{cai2021exponential} & \cmark & 850 & 43.6 & 60.9 & 72.6 \\
        FMatch + DB~\cite{wang2022debiased} & \xmark & 300 & $45.8^\dagger$ & $63.0^\dagger$ & $71.7^\dagger$ \\
        FMatch + DB + EMAN~\cite{wang2022debiased} & \cmark & 850 & 47.9 & 63.1 & $72.8^\dagger$ \\
        \midrule[0.15em]
        \rowcolor{_fbteal3}
        \putouralg & \xmark & 300 & 47.8 & 65.6 & 73.1 \\
        \rowcolor{_fbteal3}
        \putouralg + EMAN~\cite{cai2021exponential} & \cmark & 850 & \bf 50.1 & \bf 67.2 &  73.5 \\
        \emph{delta against best baseline} &  &  & {\color{darkgreen} +2.2} & {\color{darkgreen} +0.7} &  {\color{darkred} -2.0} \\
        \bottomrule[0.15em]
    \end{tabular}}
\vspace*{-1.5em}
\end{table} 
% \noindent \textbf{Analysis and Ablations}
\noindent \textbf{How does refinement help?}
First, we would like to investigate the role of pseudo-labeling refinement in improving SSL performance. Intuitively, since we perform refinement by combining pseudo-labels from two different sources (the classifier predictions in probability space and the cluster labels in the prototypical space), we expect that there will be disagreements between the two and hence considering both the views is the key towards the improved performance. To validate such intuition, we capture a fine-grained view of the training dynamics throughout the first 300 epochs of CIFAR-10 with 40 labeled instances scenario, including: samples' pseudo-labels before and after refinement as well as their cluster pseudo-labels in each epoch. This enables us to capture disagreements between the two pseudo-label sources up to the individual sample level.
In Fig.~\ref{fig:analysis_plots}-a, we display the average disagreement between the two sources over the initial phase of the training overlaid with the classifier, cluster and refined pseudo-label accuracy. We observe that initially, the disagreement (dashed black line) is high which corresponds to a larger gap between the accuracies of both heads. As the training proceeds, we observe that disagreement decreases leading to a respective decrease in the gap. Additionally, we witness that the refined accuracy curve (green) is almost always above the individual accuracies (orange and blue) which proves that, indeed, the synergy between the two sources improves the performance. 

On the other hand, to get a qualitative understanding of where each of the pseudo-labeling sources helps, we dig deeper to classes and individual samples level where we investigate which classes/samples are the most disagreed-upon (on average) throughout the training. In Fig.~\ref{fig:image_comparison}, we display the most prototypical examples of a given class (middle) as identified by the prototypical scores obtained in the embedding space. We also display the examples which on average are always correctly classified in the prototypical space (right) opposed to those in the classifier space (left). As expected, we find that samples which look more prototypical, albeit with less distinctive features (\eg blurry), are the ones almost always correctly classified with the prototypical head; whereas, samples which have more distinctive features but are less prototypical are those correctly classified by the discriminative classifier head. This again confirms our intuitions about how co-training based on both sources helps to refine the pseudo-label.
	
Finally, we ask: is it beneficial to use the entire dataset pseudo-label history to perform refinement or is it sufficient to just use a few samples? To answer this question, we use only a subset of the samples in each cluster (sampled uniformly at random) to calculate cluster pseudo-labels in Eqn.~\ref{eqn:clust_pl}. For CIFAR-10 with 20 and 40 labels, we find that this leads to about 1-2\% (4-5\%) average drop in performance, if we use half (quarter) of the samples in each cluster.  This reiterates the usefulness of our approach to leverage the history of all samples (at a lower cost) opposed to a limited history of samples. 

\smallskip
\noindent \textbf{Role of self-supervised loss.}
Here, we are interested to tear apart our choice of self-supervised loss and its role towards the performance. To recap, our intuition behind using that loss is to boost the learning signal in the initial phase of the training when the model is still not confident enough to retain samples for pseudo-labeling. As we see in Fig.~\ref{fig:analysis_plots}-b and c. there is a significant speed up of our model's convergence compared to baseline methods with a clear boost in the initial epochs. Additionally, to isolate the potentially confounding effect of our other ingredients, we display in Fig.~\ref{fig:analysis_plots}-d the performance of our method with and without the self-supervised loss which leads to a similar conclusion. Finally, to validate our hypothesis that instance-consistency loss is more useful than instance-discrimination, we run a version of \putouralg with an instance-discrimination loss akin to that of SimCLR. This version completely collapsed and did not converge at all. We attribute this to: 1) as verified by SimCLR authors, such methods work best with large batch sizes to ensure enough negative examples are accounted for; and 2) these methods treat each image as its own class and contrast it against every other image and hence are in direct contradiction with the image classification task; whereas instance-consistency losses only ensure that the representations learnt are invariant to common factors of variations such as: color distortions, orientation, \etc. and are hence more suitable for semi-supervised image classification tasks.

\begin{table}[h!]\scriptsize	
    \centering
    \begin{tabular}{c|c|c|c|c|c}
        \multicolumn{2}{c|}{} & Baseline & w/o normals & w/o viscosity & w/o coarea \\ \hline
        \multirow{4}{*}{Anchor}
            & $d_C$ & \textbf{0.21} & 0.61 & 0.55 & 0.72 \\
            & $d_H$ & \textbf{3.00} & 7.82 & 10.83 & 10.24 \\
            & $d_C^\too$ & 0.15 & 0.37 & 0.27 & 0.36 \\
            & $d_H^\too$ & 1.07 & 7.84 & 1.44 & 9.68 \\ \hline
        \multirow{4}{*}{Daratech}
            & $d_C$ & 0.26 & 0.24 & 0.24 & \textbf{0.23} \\
            & $d_H$ & 4.06 & 4.2 & 4.3 & \textbf{2.19} \\
            & $d_C^\too$ & 0.14 & 0.13 & 0.12 & 0.13 \\
            & $d_H^\too$ & 1.76 & 2.69 & 1.77 & 1.77 \\ \hline
        \multirow{4}{*}{DC}
            & $d_C$ & \textbf{0.15} & \textbf{0.15} & \textbf{0.15} & 0.34 \\
            & $d_H$ & \textbf{2.22} & 2.24 & 2.24 & 6.58 \\
            & $d_C^\too$ & 0.09 & 0.08 & 0.08 & 0.16 \\
            & $d_H^\too$ & 2.76 & 2.76 & 2.79 & 2.82 \\ \hline
        \multirow{4}{*}{Gargoyle}
            & $d_C$ & \textbf{0.17} & 0.58 & 0.47 & 0.59 \\
            & $d_H$ & \textbf{4.40} & 6.32 & 10.38 & 6.35 \\
            & $d_C^\too$ & 0.11 & 0.07 & 0.26 & 0.38 \\
            & $d_H^\too$ & 0.96 & 2.39 & 1.34 & 1.25 \\ \hline
        \multirow{4}{*}{Lord Quas}
            & $d_C$ & \textbf{0.12} & 0.12 & 0.12 & 0.58 \\
            & $d_H$ & 1.06 & 1.38 & \textbf{1.04} & 6.05 \\
            & $d_C^\too$ & 0.07 & 0.37 & 0.06 & 0.32 \\
            & $d_H^\too$ & 0.64 & 0.69 & 0.64 & 3.73 \\ \hline %
            
    \end{tabular} \vspace{5pt}
    \caption{Ablations study. We show the contribution of each component of VisCo Grids. Baseline is the full method. The remaining columns correspond to optimizing without normal loss, viscosity loss and coarea loss, respectively. We show results for each mesh of the benchmark \cite{williams2019deep}. The results justify the use of the different components in VisCo Grids.}
    \label{tab:ablations}
\end{table}

\smallskip
\noindent \textbf{Ablations.}
Finally, we present an ablation study about the important hyperparametes of \putouralg. Specifically, we find that $n$ (minimum samples in each cluster) and $\alpha$ (mixing ratio between classifier pseudo-label and cluster pseudo-label) are particularly important. Additionally, we find that the projection dimension needs to be sufficiently large for larger datasets (we use $d=64$ for CIFARs and 128 for all others). In Tab.~\ref{tab:ablation}, we present ablation results on CIFAR-10 with 80 labeled instances.


\section{Conclusion}

We introduced \putouralg, a novel SSL learning approach targeted at the low-label regime. Our approach combines co-training, clustering and prototypical learning to improve pseudo-labels accuracy. We demonstrate that our method leads to significant gains on multiple SSL benchmarks and better convergence properties. We hope that our work helps to commodify deep learning in domains where human annotations are expensive to obtain. 

\smallskip
\noindent \textbf{Acknowledgement.} This work was partly supported by DARPA’s Learning with Less Labeling (LwLL) program under agreement FA8750-19-2-0501. I. Nassar is supported by the Australian Government Research Training Program (RTP) Scholarship, and M. Hayat is supported by the ARC DECRA Fellowship DE200101100.


{\small
\bibliographystyle{ieee_fullname}
\bibliography{references.bib}
}
\newpage

% uncomment below two lines to show the supplementary title
% \title{All Labels Are Not Created Equal : Enhancing Semi-supervision via Label Grouping and Co-training\\~\\\large{Supplementary Material}}
% \maketitle 

\setcounter{section}{0}
\setcounter{equation}{0}
\renewcommand{\thesection}{\Alph{section}}
\newpage
\newpage
\section*{Appendix}
% \subsection{Missing Results}
% Here, we report the same result tables in the main paper but after adding CoMatch~\cite{li2021comatch} results which was mistakenly omitted from some of the main text tables during last-minutes edits before submission. Note that CoMatch results are not the state-of-the-art. However, it is similar to our label-refinement approach, albeit using graph contrastive learning. Hence, we compare with it in all our experiments. We report their ImageNet results directly from their paper (except for the 0.2\% setting). Whereas for the other datasets for which they do not report results, we report results by integrating their method into our codebase. We will adjust the tables in the main text to include these results, and we will omit this section from the appendix upon the paper decision. We apologise for such inconvenience.

% \input{tables/supplements_imagenet_results.tex}
% \input{tables/supplements_domainnet_results.tex}
% \begin{table*}[h!]
\centering
\caption{\small{Complete Tab.~\ref{tab:cifar_results} results. CIFAR and Mini-ImageNet accuracy for different amounts of labeled samples averaged over 5 different splits. All results are produced using the same codebase and same splits.}}
\label{tab:cifar_results_supp}
\scalebox{0.77}{
\begin{tabular}{llccclccclcc}
\toprule[0.15em]
 &  & \multicolumn{3}{c}{\bf CIFAR-10} &  & \multicolumn{3}{c}{\bf CIFAR-100} &  & \multicolumn{2}{c}{\bf Mini-ImageNet} \\ \cline{3-5} \cline{7-9} \cline{11-12} 
Total labeled samples &  & 20 & 40 & 80 &  & 200 & 400 & 800 &  & 400 & 1000 \\ \midrule[0.15em]
FixMatch~\cite{Sohn_fixmatch20} &  & 82.32$\pm$9.77 & 86.29$\pm$4.50 & 92.06$\pm$0.88 &  & 35.37$\pm$5.68 & 51.15$\pm$1.75 & 61.32$\pm$0.92 &  & 17.18$\pm$6.22 & 39.03$\pm$3.99 \\
FixMatch + DA~\cite{Sohn_fixmatch20, Berthelot_RemixMatch19} &  & 83.84$\pm$8.35 & 86.98$\pm$3.40 & 92.29$\pm$0.86 &  & 41.28$\pm$6.03 & 52.65$\pm$2.32 & 62.12$\pm$0.79 &  & 19.40$\pm$5.87 & 40.92$\pm$4.71\\
CoMatch~\cite{li2021comatch} &  & 87.37$\pm$8.47 & 93.09$\pm$1.39 & 93.97$\pm$0.62 &  & 47.92$\pm$4.83 & 58.17$\pm$3.52 & \bf 66.15$\pm$0.71 &  & 21.29$\pm$6.19 & 40.98$\pm$3.52\\
FixMatch + DB~\cite{wang2022debiased} &  & 89.02$\pm$6.37 & 94.60 $\pm$1.31 & 95.60 $\pm$0.12 &  & 46.36$\pm$5.05 & 57.88$\pm$3.34 & 64.84$\pm$0.85 &  & 27.37$\pm$7.01 & 41.05$\pm$3.34\\
\midrule[0.15em]
\rowcolor{_fbteal3}
\putouralg  &  & \bf 90.51$\pm$4.02 & \bf 95.20$\pm$1.8 & \bf 96.11$\pm$0.20 &  & \bf 48.25$\pm$4.87 & \bf 59.53$\pm$2.94 & 65.91$\pm$0.57 &  & \bf 29.15$\pm$6.98 & \bf 45.83$\pm$4.15\\
\emph{delta against best baseline}  &  & {\color{darkgreen} +1.49} & {\color{darkgreen} +0.60} & {\color{darkgreen} +0.51} &  & {\color{darkgreen} +0.33} & {\color{darkgreen} +1.36} & {\color{darkred} -0.24} &  & {\color{darkgreen} +1.78} & {\color{darkgreen} +4.78}\\
\bottomrule[0.15em]
\end{tabular}%
}
% \vspace{-2mm}

\end{table*}


\section{Constrained K-means Additional Details}
Qian \etal~\cite{qian2022unsupervised} proposed the online mini-batch solver for the constrained K-means objective (Eqn.~\ref{eqn:ckmeans}) proposed by~\cite{bradley2000constrained}, and used it for unsupervised representation learning. In our method, we adopted the same solver but for a different purpose; we use online clustering as an alternative to offline nearest neighbour search to identify neighbourhood of images and leverage such information to perform our label refinement procedure. To that end, due to the empirical observation that the maximal value of dual variables is well bounded, our Eqn.~\ref{eq:finalrho} is an approximation of the original dual variables update proposed by Qian \etal after each mini-batch:
\begin{eqnarray}
\label{eq:update}
\rho^t_k = \Pi_{\Delta_\delta}(\rho^{t-1}_k - \eta \frac{1}{B}\sum_{i=1}^B( \mu_{i,k}^t-\frac{\gamma}{N})),
\end{eqnarray}
where $\Pi_{\Delta_\delta}$ projects the dual variables to the domain $\Delta_\delta = \{\rho| \forall k, \rho_k\geq0, \|\rho\|_1\leq \delta\}$.

We refer the readers to the original paper for guarantees of performance complete proofs.

\begin{figure*}[h!]
 \centering
  \scalebox{0.99}{\includegraphics[width=0.97\textwidth]{figures/purity_vs_acc.pdf}}
 \caption{\textbf{Analysis Plots.} {\bf (a)}: Cluster Purity per class of CIFAR-10 vs training epochs, when trained using \putouralg with 4 images per class. {\bf (b):} Pseudo-label accuracy per class vs training epochs. Best viewed in color.}
 \label{fig:purity_vs_acc}
 \vspace{-2mm}
\end{figure*}


\smallskip
\noindent \textbf{Constrained vs unconstrained clustering.}
Our purpose in \putouralg is to use K-means as an alternative for offline nearest-neighbours retrieval, which automatically mandates that we use equi-partition clustering by constraining minimum cluster size $\gamma$ to be the number of nearest neighbours $n$. However, we relax this constraint to $\gamma = 0.9n$ to allow cluster sizes to slightly vary to capture the inherent imbalance in salient properties of different classes. Empirically, we found this to work well across the datasets we used. We also tested the setting with $\gamma = 0$ which translates to unconstrained clustering. This setting was unstable and did not lead to performance gains; where we found that clustering collapses to only a few clusters. For example in CIFAR-10 (40 labels) setting, K-means converged to only 20 clusters. The consequence is that we have only 20 cluster pseudo-labels to use for refining all the unlabeled samples in subsequent epochs which is a very general summary of neighbourhoods and hence it hurts the performance rather than help it. Please refer to Tab.~\ref{tab:ablation} for further ablations on the value of $n$.

\smallskip
\noindent \textbf{Mini-batch updates vs Epoch updates}
Another decision choice is the frequency of cluster centroids updates (Eqn.~\ref{eq:updatemulti}). Since \putouralg does not memorise image representations, centroids can be updated either every mini-batch, or by accumulating representations of images based on their cluster assignment throughout an epoch and then performing the update once at the end of the epoch. The former solution is useful in helping K-means convergence which requires multiple assignment-update iterations, however it leads to higher variance due to the stochastic nature of mini-batches. On the other hand, the latter solution is also sub-optimal as it requires long time for clusters to converge. Accordingly, we adopted a warmup period during which we use mini-batch updates to speed up convergence, henceforward, we switch to epoch updates to stabilise the centroids and exhibit less variance. We found that for smaller datasets, 20 epochs of warmup are sufficient, while for the larger datasets with more classes, we increase the warmup period to 70 epochs. 

\section{Additional Training Dynamics Analysis}
Here, to further understand \putouralg, we examine more of its training dynamics.

\begin{figure*}[h!]
 \centering
  \scalebox{0.99}{\includegraphics[width=0.97\textwidth]{figures/acc_vs_retention.pdf}}
 \caption{\textbf{Analysis Plots.} {\bf (a)}: Pseudo-label accuracy vs epochs. {\bf (b):} Retention rate vs epochs which denotes the ratio of unlabeled samples retained by each method for pseudo-labeling (\ie with maximum confidence score higher than the threshold $\tau$.)}
 \label{fig:acc_vs_retention}
 \vspace{-2mm}
\end{figure*}

\smallskip
\noindent \textbf{Clustering purity vs pseudo-label accuracy.}
First, we investigate the properties of the clusters as training proceeds. We follow a similar setup like that used to obtain Fig.~\ref{fig:image_comparison}, but this time, we use the captured statistics to calculate cluster purity for each class. Specifically, by the end of each epoch, we count the members of each cluster (\eg for CIFAR-10, we use $K=250$, so we count the number of images assigned by K-means to each of the 250 clusters), then for each cluster, we check the most dominant class among its members based on their ground truth labels. Subsequently, we calculate the purity of each cluster as the ratio between the number of images belonging to the dominant class to the total number of cluster members. Finally, to calculate purity for a given class, we average the described ratio across all clusters for which that class is the dominant one. In Fig.~\ref{fig:purity_vs_acc}, we display cluster purity per class of CIFAR-10 during the first 130 epochs of training side-by-side to the pseudo-label accuracy for each class. This is to allow us to investigate the clustering effectivness in the critical initial phase of training and how it affects the obtained pseudo-labels quality. We see that for the more distinguishable classes (\eg truck or ship), clustering purity increases significantly faster than others matching with a corresponding increase in pseudo-label accuracy. Whereas for more confusing classes (\eg horse and deer), the cluster purity suffers a slow increase accompanied with what seem to be high disagreement between cluster and classifier pseudo-labels leading to an overall slow increase of pseudo-label accuracy (note that we display the refined pseudo-label accuracy in the figure). Finally, the most confusing classes (\eg cat and dog) have the lowest cluster purity leading to a low pseudo-label accuracy at first, but we notice that once the majority of other classes are learnt (\ie have higher accuracy, the more confusing classes start to catch up (notice the cat and dog curves towards the end of Fig.~\ref{fig:purity_vs_acc}-b). This is in line with our expectation that easy classes are first learnt by the network, then it moves on to discriminate the less obvious ones.   

\smallskip
\noindent \textbf{Pseudo-label Retention Ratio.} Like the state-of-the-art SSL method (DebiasPL~\cite{wang2022debiased}), \putouralg is also a confidence-based pseudo-labeling method albeit with additional ingredients. Hence, both methods only retain high-confidence unlabeled samples for pseudo-labeling. In Fig.~\ref{fig:acc_vs_retention}, we examine the retention rate (\ie ratio of samples with maximum confidence exceeding the threshold $\tau$) for both methods as training proceeds (b), and compare it with the pseudo-labeling accuracy exhibited by each (a). We observe that even though our method outperforms DebiasPL, in terms of accuracy, throughout the training, it consistently retains almost 10\% less samples for pseudo-labeling. This finding speaks to our original motivation (see Sec.~\ref{sec:introduction}) with regards to the over-confidence problem underpinning the lower performance of SOTA methods in label-scarce regime. Compared to its counterparts, \putouralg is more conservative when it comes to admitting a sample as ``reliable'' for pseudo-labeling; primarily because the refined pseudo-labels we employ is a combination of the original classifier pseudo-label and the neighbourhood pseudo-label. As we show in Fig.~\ref{fig:analysis_plots}-a, the disagreement between the two results in a lower overall confidence in predictions. Such conservative nature of \putouralg is key to avoiding confirmation bias even when there is only a few labeled samples available.

\begin{table*}[h!]
\centering
\caption{\small{CIFAR and Mini-ImageNet accuracy in moderate-label regime for different amounts of labeled samples averaged over 3 different splits. All results are produced using the same codebase and same splits.}}
\label{tab:cifar_results_moderate}
\scalebox{0.99}{
\begin{tabular}{llcclcc}
\toprule[0.15em]
 &  & \multicolumn{2}{c}{\bf CIFAR-100} &  & \multicolumn{2}{c}{\bf Mini-ImageNet} \\ \cline{3-4} \cline{6-7} 
Total labeled samples &  & 2500 & 4000  &  & 2500 & 4000 \\ \midrule[0.15em]
FixMatch~\cite{Sohn_fixmatch20} &  & 71.71$\pm$0.35 & 74.08$\pm$0.13 &  & 44.53$\pm$0.44 & 50.21$\pm$0.09 \\
FixMatch + DB~\cite{wang2022debiased} &  & 72.44$\pm$0.15 & 74.43$\pm$0.06 &  & 46.18$\pm$0.23 & 52.00$\pm$0.04 \\
\midrule[0.15em]
\rowcolor{_fbteal3}
\putouralg  &  & \bf 73.31$\pm$0.43 & \bf 75.18$\pm$0.02 &  & \bf 48.61$\pm$0.34 & \bf 53.67$\pm$0.06 \\
\emph{delta against best baseline}  &  & {\color{darkgreen} +0.87} & {\color{darkgreen} +0.75} &  & {\color{darkgreen} +2.43} & {\color{darkgreen} +1.67} \\
\bottomrule[0.15em]
\end{tabular}%
}
% \vspace{-2mm}

\end{table*}

\begin{figure*}[h!]
 \centering
  \scalebox{1}{\includegraphics[width=0.99\textwidth]{figures/image_comparison_appendix.pdf}}
 \caption{Additional examples to complement Fig.~\ref{fig:image_comparison}.}
 \label{fig:image_comparison_appendix}
 \vspace{-2mm}
\end{figure*}


\section{\textbf{\putouralg} in Moderate-label Regime}
In this section, we examine our method performance when more than 10 images per class are available (which we call moderate-label regime). To recap, our method primarily aims to address confirmation bias in label-scarce settings. Yet, intuitively, the refinement strategy might also help moderate-label regimes. As such, we investigate this hypothesis by running additional experiments on CIFARs and Mini-ImageNet with 25, and 40 images per class. We find that for CIFAR-10, performance already saturates after 10 images per class and most of the compared methods perform similarly. As for the other two datasets with 100 classes each, we find \putouralg to still provide performance gains. However, with more labels available, we find that using less neighbouring samples to perform the refinement (\ie less $n$) works better. Specifically, we reduce $n$ by a factor of 10 (\ie $n=25$ instead of $n=250$). Additionally, since with more labels, all the compared methods exhibit significantly less variance, we report results only based on 3 runs instead of 5. Please refer to the results in Tab.~\ref{tab:cifar_results_moderate}.



\section{Additional Quantitative Examples}
Here, we detail our experimental setup for obtaining Fig.~\ref{fig:image_comparison} and we provide additional examples in Fig.~\ref{fig:image_comparison_appendix}.

\smallskip
\noindent \textbf{Experimental Setup.} As training proceeds, for each epoch, we capture per-image statistics such as: the classifier pseudo-label and its max score (\ie $\argmax \vp_w$ and $\max \vp_w$ respectively); cluster pseudo-label and its max score (\ie $\argmax \vz^a$ and $\max \vz^a$ respectively), sample prototypical score (\ie $\vq^w \cdot \Pcal_{\hat{y}}$) denoting how close a sample is to its class prototype. Subsequently, to obtain the prototypical images (in middle panel of Fig.~\ref{fig:image_comparison} and ~\ref{fig:image_comparison_appendix}), we rank images of each class based on their prototypical score averaged over the first 500 epochs of training. Additionally, we identify images for which the cluster pseudo-labels are, on average, more accurate than that of the classifier (and the other way around) by comparing the respective pseudo-labels with the ground truth label of each image. Thus, we display on the left panel images for which the classifier pseudo-label is, on average, more accurate than the cluster pseudo-label, and the opposite on the right panel.


\smallskip
\noindent \textbf{Additional Examples.} In Fig.~\ref{fig:image_comparison_appendix}, we provide more examples to complement those in Fig.~\ref{fig:image_comparison}. To reiterate, we see that the cluster pseudo-labels which capture the samples' neighbourhood in the prototypical space (trained via our prototypical loss) are usually more accurate if images are more prototypical even if they are lacking discriminative features (\eg blurry images or zoomed out images). In contrast, the pseudo-labels in the class probability space (trained via one-hot cross entropy) are usually more accurate for images with discriminative features (\eg car bumpers or deer horns) even if they lack prototypicality. The diversity of views captured via the different labels is key to \putouralg's effectiveness as it helps the classifier learns via the disagreement between the two views through the refined label.
\end{document}
