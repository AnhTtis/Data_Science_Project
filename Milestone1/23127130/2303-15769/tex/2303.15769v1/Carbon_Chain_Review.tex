%%% Notice: This file contains a large number of \verb's 
%%%         or verbatim environments in order to display command names
%%%         or examples.  But the use of \verb/verbatim is *not* recommended. 
%%% ver.7 2018/05/15 
\documentclass[onecolumn]{pasj01}
%\documentclass{pasj01}
% \bibliographystyle{plain}
% \bibliography{publication}
%\usepackage{natbib}
%\usepackage[style=authoryear-comp, maxnames=2, uniquelist=false]{biblatex}
%\documentclass{pasj01}
\Received{$\langle$reception date$\rangle$}
\Accepted{$\langle$acception date$\rangle$}
\Published{$\langle$publication date$\rangle$}
%% \SetRunningHead{Astronomical Society of Japan}{Usage of \texttt{pasj00.cls}}

%\usepackage{cite}
\usepackage{longtable}
\usepackage{deluxetable}
%\usepackage{hyperref}
\usepackage{url}
\usepackage{natbib}
\usepackage{rotating}
\usepackage{graphicx}
\usepackage{longtable}
\usepackage{lineno}


% Document starts
\begin{document}


\title{ 
%\LETTERLABEL %%% <-- uncomment for LETTER article  
\REVIEWLABEL %%% <-- uncomment for REVIEW article  
Carbon-Chain Chemistry in the Interstellar Medium}

%%% begin:list of authors
% Do NOT capitalize all letters in "textsc".
\author{Kotomi \textsc{Taniguchi}\altaffilmark{1}}
%\thanks{Example: Present Address is xxxxxxxxxx}}
\altaffiltext{1}{Division of Science, National Astronomical Observatory of Japan (NAOJ), National Institutes of Natural Sciences, 2-21-1 Osawa, Mitaka, Tokyo 181-8588, Japan}
\email{kotomi.taniguchi@nao.ac.jp}

\author{Prasanta \textsc{Gorai},\altaffilmark{2}}
\altaffiltext{2}{Department of Space, Earth and Environment, Chalmers University of Technology, SE-412 96, Gothenburg, Sweden}
\email{prasanta.astro@gmail.com}

\author{Jonathan C. \textsc{Tan}\altaffilmark{2,3}}
\altaffiltext{3}{Department of Astronomy, University of Virginia, Charlottesville, VA 22904-4325, USA}
%\email{ccccc@xxx.xxx.xx.xx}



\KeyWords{astrochemistry --- ISM: molecules --- ISM: abundances}

\maketitle

%Abstract
\begin{abstract}
The presence of carbon-chain molecules in the interstellar medium (ISM) has been known since the early 1970s and $>100$ such species have been identified to date, making up $>40\%$ of the total of detected ISM molecules. 
They are prevalent not only in star-forming regions in our Galaxy, but also in other galaxies. 
These molecules provide important information on physical conditions, gas dynamics, and evolutionary stages of star-forming regions. 
More complex species of polycyclic aromatic hydrocarbons (PAHs) and fullerenes (C$_{60}$ and C$_{70}$) have been detected in circumstellar envelopes around carbon-rich Asymptotic Giant Branch (AGB) stars and planetary nebulae, while PAHs are also known to be a widespread component of interstellar dust in most galaxies. 
Recently, two line survey projects toward the starless core Taurus Molecular Cloud-1 with large single-dish telescopes have detected many new carbon-chain species, including molecules containing benzene rings. 
These new findings raise fresh questions about carbon-bearing species in the Universe.
This article reviews various aspects of carbon-chain molecules, including observational studies, chemical simulations, quantum calculations, and laboratory experiments, and discusses open questions and how they may be answered by future facilities.
\end{abstract}

%\linenumbers

\section{Introduction}
\subsection{Brief Overview of Astrochemistry} \label{sec:1_1}

Astrochemistry is an interdisciplinary research field concerning ``study of the formation, destruction, and excitation of molecules in astronomical environments and their influence on the structure, dynamics, and evolution of astronomical objects'' \citep{dalgarno08}. Astrochemical studies can involve various approaches: astronomical observations; laboratory experiments on reactions and diagnostic spectroscopy; chemical simulations; and quantum chemical calculations. Collaborative studies among these approaches have been crucial in revealing the great variety of chemical pathways that operate in space.

%revealed the complexity of chemistry in the Universe.

Approximately 270 molecules have been discovered in the interstellar medium (ISM) or circumstellar envelopes (CSEs) to date 
(\url{https://cdms.astro.uni-koeln.de/classic/molecules}).
Technical innovations and advances in observational facilities have boosted the detection of new, rarer interstellar molecules, including isotopologues. 
These molecules have been detected in various physical conditions of the ISM; diffuse atomic H clouds ($n_{\rm {H}}\approx100$ cm$^{-3}$, $T\approx70$ K), molecular clouds ($n_{\rm {H}}\approx10^{4}$ cm$^{-3}$, $T\approx10$ K), prestellar cores\footnote{We use this term for gravitationally bound objects with central number densities above $10^{5}$ cm$^{-3}$ \citep{caselli2022}.} ($n_{\rm {H}}\approx10^{5}-10^{6}$ cm$^{-3}$, $T\approx10$ K), protostellar cores ($n_{\rm {H}}\approx10^{7}$ cm$^{-3}$, $T\approx100-300$ K), protoplanetary disks ($n_{\rm {H}}\approx10^{4}-10^{7}$ cm$^{-3}$, $T\approx10-500$ K), and envelopes of evolved stars ($n_{\rm {H}}\approx10^{10}$ cm$^{-3}$, $T\approx2000-3500$ K).
Beyond our Galaxy, about 70 molecules have been detected in extragalactic sources.

Although 98\% of the total mass of baryons consists of hydrogen (H) and helium (He), trace heavier elements such as carbon (C), oxygen (O), and nitrogen (N) are important constituent elements of interstellar molecules. These elements can make interstellar molecules complex and chemically rich. 
In particular, carbon composes the backbones of many molecules and is a prerequisite for organic chemistry.

Astrochemical studies of star-forming regions in our Galaxy have progressed rapidly in recent years, including in both nearby low-mass and more distant high-mass star-forming regions.
The interstellar molecules in these regions provide information on both macroscopic aspects and microscopic processes that help us to understand physical conditions and star formation histories. In most of the Universe, including our Galaxy, stars form from self-gravitating molecular clouds, i.e., where hydrogen exists predominantly in the form of $\rm H_2$, mediated via formation on dust grain surfaces \citep{Hollenbach1971}.
%jct - add a ref here to the classic papers of Salpeter, Hollenbach about this -> KT has done
%A molecular cloud is known as a stellar nursery where molecules start to form.
In the gas phase, ion-molecule reactions, which can proceed even at cold temperatures, synthesize many molecules.
At the same time, complex organic molecules (COMs)\footnote{Molecules consisting of more than 6 atoms \citep{herbst2009}.} begin to form mainly by hydrogenation reactions on dust grain surfaces (e.g., CH$_{3}$OH formation by successive hydrogenation reactions of CO). 
During the protostellar stage and the protoplanetary disk stage, chemical processes and chemical composition become much more complex, because of stellar feedback, such as protostellar radiative heating via dust reprocessed infrared radiation, direct impact of energetic UV and X-ray photons and relativistic cosmic ray particles, and shock heating produced by protostellar outflows and stellar winds. 

Some molecules undergo isotopic fractionation. Especially deuterium fractionation (D/H) and nitrogen fractionation ($^{14}$N/$^{15}$N) are important for helping to trace the journey of materials during star and planet formation \citep[for reviews][]{caselli2012,jorgensen2020,oberg2021}.
These are particularly important for revealing the formation of our Solar System, one of the most fundamental questions of astronomy.
%, to understand chemical composition, its heritage, and chemical processes at each evolutionary stage. 
%Furthermore, chemical composition is expected to possess information on past physical conditions, and thus astrochemical studies are essential to reveal the initial conditions of star formation and star formation histories.

This review focuses on ``carbon-chain molecules'', one of the major groups of molecules in the Universe.
They are abundant in the ISM and known to be useful tracers of current physical conditions and past evolutionary history. In particular, as we will see, they can be used to probe the kinematics of chemically young gas \citep[e.g.,][]{dobashi2018,pineda2020}.
This means that line emission from rotational transitions of carbon-chain species is unique probes of gas kinematics related to star formation.
Some carbon-chain species have been suggested to possess the potential to form complex organic molecules.
For example, cyanoacetylene (HC$_{3}$N) has been suggested to be a candidate for the precursor of Cytosine, Uracil, and Thymine \citep{choe2021}.
These aspects further motivate us to study their chemical characteristics in the Universe.

\subsection{History of Studies of Carbon-Chain Molecules}\label{sec:1_2}

Carbon-chain molecules are exotic species from the point of view of chemistry on Earth.
However, these molecules are one of the major constituents of molecules detected in the Universe (see Section \ref{sec:2_1}).
After the discovery of the first carbon-chain molecules in the ISM in the 1970s, many efforts to explain their formation routes were made by laboratory experiments and chemical simulations in the 1980s.
In the beginning, the focus was on gas-phase chemical reactions of small species \citep{prasad80I,prasad80II,graedel82}.
\cite{herbst83} was able to reproduce the observed abundance of a larger species, C$_{4}$H, in Taurus Molecular Cloud-1 (TMC-1; $d\approx140$ pc), which is one of the most carbon-chain-rich sources.
\cite{herbst83} found that ion-molecule reactions with a large amount of atomic carbon (with its abundance of $\sim 10^{-5}$) are necessary to explain the observed C$_{4}$H abundance.
\cite{suzuki83} found that reactions including C$^{+}$ can also play essential roles in carbon-chain growth.
These results suggested that carbon-chain species could efficiently form in young molecular clouds before carbon is locked into CO molecules.

In the 1990s, carbon-chain molecules were detected in many molecular clouds, beyond the previously well-studied examples, such as TMC-1.
Survey observations revealed that carbon-chain molecules are evolutionary indicators of starless and star-forming cores in low-mass star-forming regions \citep{suzuki92,benson98}.
These studies reinforced the view that these molecules are formed from ionic (C$^{+}$) or atomic (C) carbon in the early stages of molecular clouds, before CO formation,
%carbon is locked into CO molecules, 
as predicted by chemical simulations.
Figure \ref{fig:route} shows the carbon-chain growth and formation pathways of nitrogen- and sulfur-bearing species (HC$_{3}$N, HC$_{5}$N, and CCS), which have been frequently detected in low-mass starless cores in the early stages of molecular cloud cores. %estimated to have ages $t<10^{4}$ yr.
%jct - I would explain a little more how this timescale was estimates - was it from demographic studies? 
%-> KT - I would like to mean that reaction schemes in Figure 1 mainly occur t<10^{4}$ yr, and these molecules can be detected even in later stages. In later stages, carbon-chain formation and destruction change and it is difficult to show all of the stages in one figure. I then only focus on early stage.
These reaction schemes are constructed from results of the latest chemical simulations with a constant temperature of 10~K and a constant density of $n_{\rm{H}}=10^{4}$ cm$^{-3}$ in $t<10^{4}$ yr \citep{tani2019}.
Hydrocarbons can form efficiently from C$^{+}$ and C via ion-molecule reactions involving H$_{2}$ and dissociative recombination reactions leading to the formation of neutral hydrocarbons.
Such carbon-chain chemistry in cold molecular clouds is basically consistent with that proposed in the 1980s.

\begin{figure*}
\begin{center}
\includegraphics[width=0.95\textwidth]{fig_carbon_formation.pdf}
\end{center}
\caption{Carbon-chain growth (for number of C~$\leq4$) and formation processes of HC$_{5}$N and CCS in early stages of molecular clouds ($t<10^{4}$ yr). 
The dashed arrow from $l,c-$C$_{3}$H to CCH indicates that this reaction is important around $10^{5}$ yr.}
\label{fig:route}
\end{figure*}

During the later stages of starless cores, carbon-chain species are adsorbed onto dust grains or destroyed by reactions with atomic oxygen.
These processes result in the depletion of carbon-chain species by the later stages of prestellar cores and protostellar cores.
Thus, carbon-chain molecules were classically known as ``early-type species''.

Subsequent studies found that carbon-chain species exist in lukewarm regions ($T\approx25-35$ K) around low-mass protostars, and a new carbon-chain formation mechanism starting from CH$_{4}$ was proposed (see Section \ref{sec:2_2}). This was named Warm Carbon-Chain Chemistry (WCCC) and a review article about it was published ten years ago \citep{sakai2013}.
More recently, carbon-chain chemistry around massive young stellar objects (MYSOs) has been explored (Section \ref{sec:2_3}), and Hot Carbon-Chain Chemistry (HCCC) has been proposed.
Furthermore, very complex carbon-chain species, branched-chain molecules, and molecules including benzene rings, have been discovered in the ISM in the last few years. These new findings bring fresh challenges and excite our curiosity for a deeper understanding of carbon-chain chemistry in the ISM.

\subsection{Outline of This Review} \label{sec:1_3}

In this review article, we summarize results from studies of carbon-chain molecules by astronomical observations, chemical simulations, quantum chemical calculations, and laboratory experiments.
In Section \ref{sec:2}, we overview the current status of the detected carbon-chain species in the ISM and the main concepts of carbon-chain chemistry around protostars.
We review observational studies (Section \ref{sec:3}), chemical models (Section \ref{sec:4}), quantum chemical calculations (Section \ref{sec:5}), and laboratory experiments (Section \ref{sec:Experiment}). 
Finally, we list current open and key questions regarding carbon-chain species and summarise in Section \ref{sec:7}.

Here, we set a definition of ``carbon-chain molecules'' for this review article, as recent detections of new interstellar species complicates this categorization. We include linear carbon-chain species with more than two carbon atoms and cyclic species with more than three carbon atoms containing at least one double (-=-) or triple (-$\equiv$-) bond as carbon-chain molecules.
Even if molecules meet the above criteria, species containing functional groups related to organic chemistry (e.g., -OH, -NH$_2$) are excluded from carbon-chain molecules, because they are generally categorized as COMs. 
As an exception, $cyclic$-C$_{2}$Si, which consists of a cyclic structure with two carbon atoms and one Si atom, is treated as a carbon-chain species. In addition to straight linear carbon-chain species, we also treat branched carbon-chain species (e.g., iso-propyl cyanide) as carbon-chain molecules. 
Molecules containing the structure of benzene, polycyclic aromatic hydrocarbons (PAHs), and fullerenes (C$_{60}$ or C$_{70}$) are also included. 
In the following sections, we abbreviate $linear$- and $cyclic$- as $l$- and $c$-, if necessary to indicate the molecular structure (e.g., $l$-C$_{3}$H$_{2}$ and $c$-C$_{3}$H$_{2}$).
This review article summarizes literature results until the end of December 2022.


\section{Development of Carbon-Chain Chemistry} \label{sec:2}

\subsection{Detection of Carbon-Chain Species in the ISM and CSEs} \label{sec:2_0}

Cyanoacetylene (HC$_3$N), the shortest cyanopolyyne (HC$_{2n+1}$N), was the first carbon-chain molecule detected in space. 
It was found toward a high-mass star-forming complex, Sagittarius B2 \citep[Sgr B2,][]{Turner71}. 
Following this,
%the first discovery of carbon-chain species from cyanopolyynes group, 
two new carbon chains, CH$_3$CCH \citep{buhl73} and C$_2$H \citep{tuck74c2h} were detected, which belong to the hydrocarbons family. 
The next higher-order cyanopolyyne, cyanodiacetylene (HC$_5$N), was identified toward Sgr B2, i.e., the same region where HC$_3$N was first observed \citep{Avery76}. 
In the following year, C$_2$ was detected toward Cygnus OB2 No.12 via electronic spectra \citep{souz77c2}. 

Observations of carbon-chain species also expanded to other star-forming regions and HC$_{3}$N was detected toward various types of sources. For instance, it was identified toward Heiles cloud 2 (TMC-1 ridge is part of Heiles cloud)
% PG- they are the same region, now we have revised the sentence as 
%jct - I replaced "or" with "and" - but check if this is the same source
%TMC-1 
\citep{Morris76}, and HC$_{5}$N was found to be abundant in the same region \citep{Little77}.  
The detections of the longer-chain, higher-order cyanopolyynes, HC$_7$N, and HC$_9$N, were also reported toward TMC-1 \citep{Kroto78, Broten78}. 
Around the same time, C$_3$N, C$_4$H, and CH$_{3}$CCH were also found toward dark clouds, including TMC-1 \citep{Friberg80,guelin78c4h,Irvine81}. The discoveries of several carbon-chain species in the ISM brought curiosity and opened a new field to be investigated further.  Several years later, HC$_{11}$N was tentatively identified toward TMC-1 \citep{Bell85}. 
However, it took more than three decades to confirm its presence in the same source \citep{loomis2021}.

In the 1980s, several carbon-chain species were discovered in the ISM; e.g., C$_3$O \citep{Matthews84}, C$_3$N \citep{guel77c3n}, C$_3$H \citep{Thaddeus85}, C$_3$S \citep{Kaifu87,Yamamoto87}, C$_5$H \citep{Cernicharo86}, C$_6$H \citep{suzu86c6h}, $c$-C$_3$H$_2$ \citep{Thaddeus85C3H2}, $c$-C$_3$H \citep{Yamamoto87}, C$_2$S \citep{Yamamoto87}, and C$_3$S \citep{Kaifu87}. In the same period, two bare carbon chains, C$_3$ and C$_5$, were identified toward IRC+10216 \citep{hink88c3,bern89c5} via their rotation-vibration spectra.

%jct - some edits here
It was noted that while many carbon-chain molecules are found to be abundant in the starless core TMC-1, some other quiescent starless cores in the Taurus region are deficient in these species. And, more generally, carbon-chain molecules have low abundances in diffuse clouds, translucent clouds, and hot molecular cores (HMCs). 

\begin{deluxetable}{lccccccc}
% \tabletypesize{\footnotesize}
\tablewidth{0pt}
\tablecaption{Ground state, polarizability, dipole moment and present astronomical status of different carbon chain species \label{tab:quant} }
\tablehead{\colhead{Species} &\colhead{Ground State} &\colhead{Polarizability} &\colhead{Dipole moment} &\colhead{Detection Status}\\
&&\colhead{($\alpha$ in $\mathring{A}^{3}$)} & \colhead{(Debye)} & \colhead{(Detected or not)}}
\startdata
C$_2$ &singlet &5.074$^{a}$  &0.0 & Yes\\
C$_3$ &singlet&5.179$^{a}$ &0.0 &Yes\\
C$_4$ &singlet&7.512$^{a}$  &0.0&No\\
C$_5$ &singlet &11.164$^{a}$ &0.0&Yes\\
C$_6$ &singlet &14.316$^{a}$ &0.0 &No\\
C$_7$ &singlet &20.498$^{a}$ &0.0 &No\\
C$_8$ &singlet &23.959$^{a}$ &0.0 &No\\
C$_9$ &singlet &33.356$^{a}$ &0.0 &No\\ 
C$_{10}$&singlet &37.703 $^{a}$ &0.0 &No\\ 
\hline
\multicolumn{5}{c}{$\rm{C_{n}H}$}\\
\hline
C$_2$H &doublet &4.415$^{a}$ &0.81$^{a}$ & Yes\\
$l$-C$_3$H & doublet&5.359$^{a}$ &3.52$^{a}$ &Yes\\
$c$-C$_3$H & doublet&4.802$^{a}$ &2.60$^{a}$ &Yes\\
C$_3$H$^+$ &singlet& &3.00$^{b}$&Yes\\
C$_4$H &doublet & 7.151$^{a}$ &2.40$^{c}$&Yes\\
C$_4$H$^-$ &singlet & -- &5.9$^{d}$&Yes\\
C$_5$H &doublet &10.504$^{a}$ &4.84$^{a}$ & Yes\\
C$_5$H$^+$ &singlet &--&2.88$^{e}$ & Yes\\
c-C$_5$H &doublet &-- & 3.39$^{f}$ & Yes\\
C$_6$H &doublet &13.679$^{a}$ &5.6$^{a}$ &Yes\\
C$_6$H$^-$ &singlet &-- &8.2$^{d}$ &Yes\\
C$_7$H &doublet &17.372$^{a}$ &5.83$^{a}$ &Yes\\
C$_8$H &doublet &21.847$^{a}$ &6.43$^{a}$ &Yes\\
C$_8$H$^-$ &singlet &--&11.9$^{5}$&Yes\\
C$_9$H &doublet &26.381$^{a}$ &6.49$^{a}$ &No\\
C$_{10}$H &singlet &31.051$^{a}$ &7.13$^{a}$ &No\\
$l$-C$_3$H$_2$ &singlet &5.609$^{a}$ &4.10$^{a}$ &Yes\\
\hline
\multicolumn{5}{c}{$\rm{HC_{n}H}$}\\
\hline
HC$_2$H &singlet&3.378$^{a}$ &0.0 & Yes\\
HC$_3$H &triplet&2.581$^{a}$ &0.51$^{g}$ &No\\
HC$_4$H &singlet&7.048 &0.0&Yes\\
HC$_5$H &triplet & -- &--&No\\
HC$_6$H &singlet & 11.946$^{a}$ &0.0&Yes\\
HC$_7$H &triplet & -- &--&No\\
HC$_8$H &singlet& 18.588$^{a}$ &0.0&No\\
\hline
\multicolumn{5}{c}{$\rm{C_{n}O}$}\\
\hline
C$_2$O &triplet&4.087$^{a}$  &1.43$^{a}$ &Yes\\
C$_3$O &singlet &6.027$^{a}$ &2.39$^{h}$ &Yes\\
HC$_3$O$^+$ &singlet&-- &3.41$^{i}$&Yes\\
C$_4$O &triplet &9.209$^{a}$  &3.01$^{j}$&No\\
C$_5$O &singlet &--&4.06$^{k}$&Yes\\
C$_6$O &triplet &--&4.88$^{j}$&No\\
C$_7$O &singlet &--&4.67$^{l}$&No\\
C$_8$O &triplet &--&4.80$^{l}$&No\\
\hline
\multicolumn{5}{c}{$\rm{C_{n}S}$}\\
\hline
C$_2$S &triplet&6.873$^{a}$ &3.12$^{a}$&Yes\\
HC$_2$S$^+$ &triplet&&2.29$^{l1}$&Yes\\
C$_3$S &singlet &9.649$^{a}$ &3.939$^{a}$&Yes\\
HC$_3$S$^+$ &singlet &--&1.73$^{d}$&Yes\\
C$_4$S &triplet &13.697$^{a}$ &4.62$^{a}$&Yes\\
C$_5$S &singlet &--&4.65$^{m}$&Yes\\
C$_6$S &triplet &--&5.40$^{l}$&No\\
C$_7$S &singlet &--&6.17$^{l}$&No\\
C$_8$S &triplet &--&6.50$^{l}$&No\\
\hline
\multicolumn{5}{c}{$\rm{C_{n}N}$}\\
\hline
C$_2$N & doublet&4.270$^{a}$  & 0.60$^{a}$ &Yes\\
C$_3$N &doublet &5.675$^{a}$ & 2.86$^{a}$ &Yes\\
C$_3$N$^-$ &singlet&--&3.1$^{n}$&Yes\\
C$_4$N &doublet &8.749$^{a}$  &0.06$^{a}$ &No \\
C$_5$N &doublet &9.430$^{a}$ &3.33$^{a}$&Yes\\
C$_5$N$^-$ & singlet& --&5.20$^{o}$&Yes\\
C$_6$N &doublet &--& 0.21$^{p}$&No\\
C$_7$N &doublet &18.945$^{a}$ &0.87$^{a}$ &No\\
C$_8$N &doublet &--&-- &No\\
\hline
\multicolumn{5}{c}{$\rm{C_{n}P}$}\\
\hline
C$_2$P &doublet &7.518$^{a}$  &3.24$^{a}$  &Yes\\
C$_3$P &doublet &10.499$^{a}$ &3.89$^{a}$&No\\
C$_4$P &doublet &12.764$^{a}$  &4.19$^{a}$&No\\
C$_5$P &doublet &--&-- &No\\
C$_6$P &doublet &--&-- &No\\
C$_7$P &doublet &--&-- &No\\
C$_8$P &doublet &--&-- &No\\
\hline
\multicolumn{5}{c}{$\rm{HC_{2n}N}$}\\
\hline
HC$_2$N &triplet& &3.30$^{q}$ &Yes\\
HC$_4$N &triplet &8.842$^{a}$&4.30$^{a}$ &Yes\\
HC$_6$N &triplet&15.066$^{a}$  &4.89$^{a}$&No\\
HC$_8$N &triplet &22.955$^{a}$ &5.57$^{a}$ &No\\
\hline
\multicolumn{5}{c}{$\rm{HC_{2n+1}N}$}\\
\hline
HC$_3$N &singlet &5.848$^{a}$ &3.78$^{a}$ &Yes\\
HNC$_3$ &singlet &-- &6.46$^{p}$ &Yes\\
HC$_3$NH$^+$ &singlet&-- &1.87$^{r}$ &Yes\\
HC$_5$N &singlet &10.416$^{a}$ &4.41$^{a}$&Yes\\
HC$_5$NH$^+$ &singlet &10.416$^{a}$ &3.26$^{n1}$&Yes\\
HC$_7$N &singlet &16.690$^{a}$ &4.90$^{a}$&Yes\\
HC$_7$NH$^+$ &singlet &-- &6.40$^{o1}$&Yes\\
HC$_9$N &singlet &23.893$^{a}$ &5.29$^{a}$&Yes\\
HC$_{11}$N &singlet &--&5.47&Yes\\
\hline
\multicolumn{5}{c}{$\rm{HC_{n}O}$}\\
\hline
HC$_2$O &doublet&4.2$^{c}$&1.8$^{s}$ &Yes\\
HC$_3$O &doublet &5.20$^{c}$&2.74$^{s}$&Yes\\
HC$_4$O &doublet &-- &2.64$^{m1}$&No\\
HC$_5$O &doublet &--& 2.16$^{m1}$&Yes\\
HC$_6$O &doublet &--&2.11$^{m1}$ &No\\
HC$_7$O &doublet &--&2.17$^{m1}$ &Yes\\
HC$_8$O &doublet &--&2.19$^{m1}$ &No\\
\hline
\multicolumn{5}{c}{$\rm{HC_{n}S}$}\\
\hline
HC$_2$S &doublet&6.92$^{s}$&1.36$^{s}$&Yes\\
HC$_3$S &doublet &9.62$^{s}$&1.28$^{s}$ &No\\
HC$_4$S &doublet &-- &1.45$^{p1}$&Yes\\
HC$_5$S &doublet &--&1.92$^{q1}$&No\\
HC$_6$S &doublet &--&2.75$^{r1}$&No\\
HC$_7$S &doublet &--&2.10$^{q1}$&No\\
HC$_8$S &doublet &--&3.21$^{r1}$&No\\
\hline
\multicolumn{5}{c}{$\rm{Metal Containing}$}\\
\hline
MgC$_{2}$H &doublet &--&1.68$^{t}$&Yes\\
MgC$_{4}$H &doublet &--&2.12$^{u}$ &Yes\\
MgC$_{3}$N &doublet &--&6.30$^{u}$ &Yes\\
MgC$_{5}$N &doublet &--&7.30$^{v}$ &Yes\\
MgC$_{6}$H &doublet &--&2.50$^{v}$ &Yes\\
$c$-C$_2$Si &doublet &6.785$^{a}$ &2.4$^{a}$&Yes\\
$c$-C$_3$Si &singlet&11.900$^{a}$ &4.1$^{a}$ &Yes\\
C$_4$Si &singlet &--& 6.3$^{w}$&Yes\\
\hline
\multicolumn{5}{c}{cyclic-carbon-chains}\\
\hline
%$c$-C$_3$H&doublet &5.359$^{a}$  &2.40 (3.52$^{a}$) &Yes\\
$c$-C$_3$H$_2$  &singlet &4.583$^{a}$  &3.41$^{a}$ &Yes\\
$l$-C$_3$H$_2$  &singlet &5.609$^{a}$  &4.16$^{a}$ &Yes\\
$l$-C$_5$H$_2$  &singlet &11.323$^{a}$  &5.89$^{a}$ &Yes\\
$c$-C$_3$HCCH   &singlet&-- &4.93$^{x}$&Yes\\
$c$-H$_2$C$_3$O &singlet &5.2$^{s}$ &4.39$^{s}$&Yes\\
\hline
\multicolumn{5}{c}{PHAs and benzene ring related species}\\
\hline
C$_6$H$_6$&singlet &10.353$^{a}$ &0.0 &Yes\\
$\rm{C_{60}^{+}}$&doublet &&0.0 &Yes\\
C$_{60}$&singlet&79.0$^{y}$ &0.0 &Yes\\
C$_{70}$&singlet &10$^{z}$&0.0 &Yes\\
c-$\rm{C_6H_5CN}$& singlet &11.91$^{y}$&4.51$^{k1}$ &Yes\\
c-$\rm{C_9H_8}$&singlet &121.2$^{a1}$&0.87$^{b1}$&Yes\\
c-$\rm{C_5H_4CCH_2}$& singlet&--&0.69$^{c1}$ &Yes\\
c-$\rm{C_5H_6}$& singlet&--& 0.416$^{d1}$&Yes\\
1-c-$\rm{C_5H_5CN}$&singlet &--&4.42$^{e1}$ &Yes\\
2-c-$\rm{C_5H_5CN}$&singlet &--&5.13$^{e1}$ &Yes\\
1-$\rm{C_{10}H_7CN}$&singlet &--&6.6$^{f1}$ &Yes\\
2-$\rm{C_{10}H_7CN}$&singlet &--&6.1$^{f1}$ &Yes\\
1-c-$\rm{C_5H_5CCH}$&singlet &--&1.13$^{g1}$&Yes\\
2-c-$\rm{C_5H_5CCH}$&singlet &--&1.48$^{g1}$&Yes\\
o-$\rm{C_6H_4}$&singlet&--&1.38$^{h1}$ &Yes\\
$\rm{C_6H_5CCH}$& singlet&--&0.66$^{i1}$ &Yes\\
$\rm{C_9H_7CN}$&singlet &--& 5.04$^{j1}$&Yes\\
\hline
\enddata
\tablecomments{$^{a}$\cite{woon09},
$^{b}$\cite{pety12c3hp}, $^{c}$\cite{oya20},
$^{d}$\cite{blanksby2001},
$^{e}$\cite{bots91},
$^{f}$\cite{crawford1999},
$^{g}$\cite{nguyen2001},
$^{h}$\cite{brown1983},
$^{i}$\cite{cernicharo2020HC3O},
$^{j}$\cite{ewing1989},
$^{k}$\cite{bots93},
$^{l}$\cite{etim20},
$^{m}$\cite{pascoli1998},
$^{n}$\cite{Thaddeus08},
$^{o}$\cite{Cernicharo2008C5Nm},
$^{p}$\cite{kawaguchi1992},
$^{q}$\cite{hirano1989},
$^{r}$\cite{bots1987},
$^{s}$KIDA (https://kida.astrochem-tools.org/), 
$^{t}$\cite{woon1996},
$^{u}$\cite{Cernicharo2019Mg},
$^{v}$\cite{pardo2022},
$^{w}$\cite{ohishi1989},
$^{x}$\cite{trav97},
$^{y}$https://cccbdb.nist.gov/pollistx.asp,
$^{z}$\cite{compa2001},
$^{a1}$\cite{ghiasi2006},
$^{b1}$\cite{caminati1993},
$^{c1}$\cite{sazakumi1993},
$^{d1}$\cite{laurie1956},
$^{e1}$\cite{sakaizumi1987},
$^{f1}$\cite{McNaughton2018},
$^{g1}$\citep{cernicharo2021cyclopentadiene},
$^{h1}$\cite{kraka1993},
$^{i1}$\cite{cox1975},
$^{j1}$\cite{sita2022},
$^{k1}$\cite{WOHLFART2008},
$^{l1}$\cite{puzzarini2008},
$^{m1}$\cite{mohamed2005},
$^{n1}$\cite{Marcelino20},
$^{o1}$\cite{cabezas2022HC7NH},
$^{p1}$\cite{Fuentetaja2022HCCCHCCC},
$^{q1}$\cite{gordon2002},
$^{r1}$\cite{wang2009}
% c-CDMS(https://cdms.astro.uni-koeln.de/classic/) $^{\rm b}$\cite{szal96},   m-\cite{trav97}; o-\cite{bens73}, $^{f}$-\cite{bots93}, 5-Gupta et al. 2007. *-Morgen and Senent., $d$-\cite{cernicharo2021HC3S}}
}
\end{deluxetable}

In the 1990s, there were first detections reported of C$_2$O \citep{Ohishi91}, HC$_2$N \citep{Guelin91}, C$_7$H \citep{Guelin97}, C$_8$H \citep{Cernicharo96}, H$_2$C$_6$ \citep{Langer97}, C$_5$N \citep{Guelin98}, $c$-SiC$_3$ \citep{Apponi99}, and HC$_3$NH$^{+}$ \citep{Kawaguchi94}. 


In the 2000s, several complex carbon-chain molecules including linear and cyclic structures were detected: HC$_4$N \citep{Cernicharo04}; CH$_3$C$_5$N \citep{Snyder06}; $c$-H$_2$C$_3$O \citep{Hollis06}; CH$_3$C$_6$H \citep{Remijan06}. 
The first and only phosphorus-bearing chain, CCP, was discovered from IRC+10216 \citep{Halfen08}. 
The benzene ring (C$_6$H$_6$) and two polyacetylene chains, C$_4$H$_2$ and C$_6$H$_2$, were detected with the Infrared Space Observatory \citep[ISO;][]{cernicharo2001}. 
Not only neutral carbon-chain species, but also various carbon-chain anions have been discovered in space. 
The first interstellar anion is C$_6$H$^-$, which was identified in both the circumstellar envelope IRC+10216 and the dense molecular cloud TMC-1 \citep{McCarthy06}.  
Subsequently, C$_4$H$^-$ \citep{Cernicharo2007} and C$_8$H$^-$ \citep{Brunken2007,Remijan2007} were detected toward the same target. 
Two nitrogen-bearing carbon-chain anions, C$_3$N$^{-}$ \citep{Thaddeus08} and C$_5$N$^{-}$ \citep{Cernicharo2008C5Nm} were discovered in IRC+10216. 


In the 2010s, we came to know about the existence of cosmic fullerenes, C$_{60}$ and C$_{70}$, in a peculiar planetary nebula with an extremely hydrogen-poor dust formation zone \citep{Cami2010}, although ionized fullerene (C$_{60}^{+}$) was discovered before that \citep{Foing1994}. 
This brought great attention to astronomers, experimentalists, theoreticians, modelers, and other scientific communities. 
%In the 2010s, 
Several other interesting carbon-chain species were detected: $l$-$\rm{C_3H^{+}}$ \citep{pety12c3hp}; C$_{5}$S \citep{Agundez2014}; CCN \citep{anderson14c2n}; carbon-chain radicals containing an oxygen atom, HC$_2$O \citep{Agundez2015hc2o}, HC$_5$O \citep{McGuire2017} and HC$_7$O \citep{Cordiner2017}; simplest nitrogen-bearing aromatic molecule, $c$-$\rm{C_6H_5CN}$ \citep{McGuire2018}; and metal-containing carbon chains, MgC$_3$N and MgC$_4$H \citep{Cernicharo2019Mg}.

In the 2020s, two line survey programs, GOTHAM (using the 100m diameter GBT) and QUIJOTE (using the Yebes 40m telescope) focusing on TMC-1, have discovered many new carbon-chain species: $\rm{HC_5NH^{+}}$ \citep{Marcelino20}; HC$_3$O \citep{cernicharo2020HC3O}; C$_5$O \citep{Cernicharo2021C5O}; HC$_3$O$^+$ \citep{cernicharo2020HC3O}; $c$-$\rm{C_5H}$ \citep{cernicharo2022C5H}; $\rm{C_5H^{+}}$ \citep{cernicharo2022C5H}; C$_4$S \citep{Cernicharo2021S}; $\rm{HC_3S^{+}}$ \citep{cernicharo2021HC3S}; HCCS \citep{cernicharo2021HC3S}; HC$_4$S \citep{fuentetaja2022}; HCCS$^{+}$ \citep{cabezas2022}; $\rm{HC_7NH^{+}}$ \citep{cabezas2022HC7NH}; aromatic molecules, which include benzene ring(s), and their precursors. 
These newly detected species have dramatically changed the list of known interstellar molecules/carbon-chain species and brought many open questions (see Section \ref{sec:3_11}).


\subsection{Different carbon-chain families and their present status}

All of the carbon-chain species belonging to the various groups, C$_n$, C$_n$H, C$_n$H$^{-}$, C$_n$O, C$_n$N, C$_n$N$^{-}$,  C$_n$S, C$_n$P, HC$_{2n+1}$N, HC$_{2n}$N, HC$_{n}$O, HC$_{n}$S, HC$_{n}$H, MgC$_{n}$H, MgC$_{n}$N, are summarized in Table \ref{tab:quant}. Here information is given on their electronic ground state, electric polarizability, electric dipole moment, and present astronomical status.

\subsubsection{Pure carbon chains - C$_n$}

All pure linear carbon chains are indicated as C$_n$ ($n >1$). 
The electronic ground state of all these species is singlet, and they do not have a permanent dipole moment (see Table \ref{tab:quant}). Hence, they do not show rotational transitions, and so are not detectable via rotational transitions using radio observations. 
Instead, they show emission in the infrared domain through their vibration-rotation transitions, and so far three chains ($n=2, 3, 5$) are astronomically detected from this group (see Section \ref{sec:2_0}). 
In diffuse and translucent environments, C$_2$ formation starts with the reaction of $\rm{C^{+} + CH \rightarrow C_2^{+}+ H}$, followed by a series of hydrogen abstraction reactions and dissociative recombination reactions that yield C$_2$ via several channels 
\citep[][and reference therein]{welty13}.  
C$_3$ is formed via a dissociative recombination reaction of C$_3$H$^+$, though neutral–neutral reactions (e.g., C + C$_2$H$_2$ ) may also contribute (Roueff et al. 2002). 

\subsubsection{Hydrocarbons - C$_n$H}

The C$_n$H group represents the simplest hydrocarbons and carbon-chain radicals. 
All carbon chains from this group have permanent dipole moments and show strong rotational transitions. To date, eight neutral ($n=1-8$) carbon-chain species have been detected from this group. All of them have been identified towards both TMC-1 and IRC+10216. Apart from neutrals, three anions ($\rm{C_4H^{-}}$, $\rm{C_6H^{-}}$, and $\rm{C_8H^{-}}$) and two cations ($\rm{C_3H^{+}}$ and $\rm{C_5H^{+}}$) have also been identified. The anions belong to the even series ($n=2, 4, 6$), while the cations belong to the odd series ($n=3, 5$). In addition, two cyclic chains, $c$-C$_3$H, and $c$-C$_5$H, have been identified. All neutral species have a doublet ground state and show a trend of increasing dipole moment with the number of carbon atoms ($n$), especially for neutrals and anions (see Table \ref{tab:quant}). The C$_n$H family is mainly formed through the atomic reactions in the following channel, $\rm{C + C_{n-1}H_2 \rightarrow C_nH + H}$ \citep{Remijan22}. Another two channels, which involve atomic and their related anions, can also form C$_n$H family species efficiently: $\rm{C + C_{n-1}H^{-} \rightarrow C_nH + e^{-}}$ and $\rm{H + C_n^{-} \rightarrow C_nH + e^{-}}$.

%jct - check edits to above sentence

%C_{n−1}H_2 \rightarrow C_nH + H}$ .
% Apart from this, there are another two channels ($\rm{C + C_{n−1}H^{−} \rightarrow C_nH + e^{−}}$ and $\rm{H + C_n^{−} \rightarrow C_nH + e^{−}}$) through which CnH family would have formed efficiently, which involve atomic and their related anions,


%\subsection{HC$_n$H}
%acetylene, di, triacetylene- DISCUSS THIS IN THE OTHER ENVIRONMENT SECTION/NEBULA

\subsubsection{Oxygen-bearing carbon chains - C$_n$O \label{sec:2_2_3}}

To date, three oxygen (O)-bearing carbon chains, C$_n$O ($n=2, 3, 5$) have been detected in the ISM. In this series, C$_3$O was the first, detected toward TMC-1 in 1984, while C$_2$O was identified in the same source in 1991. 
It took around three decades to detect the higher-order chain, C$_5$O, in TMC-1. C$_4$O is yet to be detected. Carbon chains in this group have alternate ground states, i.e., triplet and singlet, and show a trend of increasing dipole moment with the number of carbon atoms (see Table \ref{tab:quant}). 
A protonated species, HC$_3$O$^+$ with singlet ground state, has been detected in TMC-1. 
The observed trend toward TMC-1 shows that C$_3$O is the most abundant, followed by C$_2$O and C$_5$O, with C$_5$O about 50 times less abundant than C$_2$O and about 80 times less abundant than C$_3$O \citep{Cernicharo2021C5O}. 
%The species with odd $n$ are found to be more abundant than their even $n$ counterparts. 
In addition, all these species have been identified towards the circumstellar envelope of IRC+10216. 
The formation of C$_n$O and HC$_n$O chains follows similar formation mechanisms as discussed above. The first step involves the radiative association of $\rm{C_{n-1}H^{+}}$, $\rm{C_{n-1}H_2^{+}}$, and $\rm{C_{n-1}H_3^{+}}$ ions with CO, which is then followed by dissociative electron recombination reactions \citep{adams89,Cernicharo2021C5O}. 

\subsubsection{Sulfur-bearing carbon chains - C$_n$S}

Similar to C$_n$O, several sulfur (S)-bearing carbon chains, C$_n$S ($n=2, 3, 4, 5$), have been identified in the ISM. These species have been detected toward TMC-1 and IRC+10216. Two protonated species, HCCS$^{+}$ and HC$_3$S$^+$, have only been detected towards TMC-1 so far.  
Carbon chains in this group have alternative ground states, i.e., triplet and singlet, and show a trend of increasing dipole moment with the number of carbon atoms, similar to the C$_n$O group (see Table \ref{tab:quant}). 
The abundances of C$_2$S and C$_3$S are almost three orders of magnitude higher than C$_4$S and C$_5$S toward TMC-1 \citep{Cernicharo2021S}. On the other hand, the C$_5$S column density is slightly less than those of C$_2$S and C$_3$S, with differences less than one order of magnitude, toward IRC+10216 \citep{Agundez2014}.
%This result is very different from the one obtained; the C$_5$S column density is slightly less than those of C$_2$S and C$_3$S and the differences are less than one order of magnitude toward the carbon-rich star IRC+10216 \citep{Agundez2014}. 
C$_2$S and C$_3$S are mainly produced via several ion-neutral reactions followed by electron recombination reactions and via several neutral-neutral reactions \citep{sakai2007}. 
Higher order chains of this family, such as C$_4$S and C$_5$S, are thought to be formed via reactions of S + C$_4$H and C + HC$_3$S, and C$_4$H + CS and S + C$_5$H, respectively.
However, the kinematics and product distribution of these reactions is poorly known \citep{Cernicharo2021S}.

\subsubsection{Nitrogen-bearing carbon chains - C$_n$N}

In this group, C$_3$N was the first detected species, done so tentatively toward IRC+10216 in 1977 and more robustly toward TMC-1 in 1980. The next higher order chain in this series, C$_5$N was detected toward TMC-1 and tentatively detected toward IRC+10216 in 1998. The lower order chain, CCN, was found in 2014.
%It took more than three decades to detect the lower order chain CCN from this group. 
Their anions, $\rm{C_3N^{-}}$ and $\rm{C_5N^{-}}$, were discovered in the circumstellar envelope of the carbon-rich star IRC+10216 \citep{Thaddeus08,Cernicharo2008C5Nm}. 
They have also been identified toward TMC-1 by the QUIJOTE group, including their neutral analogs (C$_3$N, C$_5$N) \citep{Cernicharo2020C3Nm}. 
They measured similar abundance ratios of $\rm{C_3N^{-}}$/C$_3$N = 140 and 194, and $\rm{C_5N^{-}}$/C$_5$N = 2 and 2.3 in TMC-1 and IRC+10216, respectively, which may indicate similar formation mechanisms in these interstellar and circumstellar environments. 
However, physical conditions are completely different for TMC-1 and IRC+10216, and it might be a coincidence that there are similar abundance ratios of anion and neutral forms of C$_{n}$N ($n = 3, 5$).
%between TMC-1 and IRC+10216. 
All carbon chains from this group have doublet ground state, and the two anionic forms have singlet state. 
The dipole moment of CCN is low compared to those of C$_3$N, C$_5$N, and their anionic forms, which helps explain the later, more challenging detection of CCN,
%might be a reason for the late detection of CCN 
even though it is of lower order in the chain (see Table \ref{tab:quant}). 
C$_4$N, C$_6$N, and C$_7$N show even smaller values of their dipole moments, which suggests that much high sensitivity observations are required for their identification. 
%jct - check this commenting out - it seems repetitive:
%Although the dipole moment of C$_4$N is very low, it could be a challenge to identify this species. 
C$_2$N is produced through the reactions of N + C$_2$ and C + CN. 
Similarly, C$_3$N is produced in reactions of N + C$_3$ and C + CCN, and C$_5$N is produced through N + C$_5$ on dust surfaces\footnote{\url{https://kida.astrochem-tools.org/}}. 
The production of $\rm{C_3N^{-}}$ mainly comes from the reaction between N atoms and bare carbon-chain anions $\rm{C_n^{-}}$ \citep{Cernicharo2020C3Nm}, whereas $\rm{C_5N^{-}}$ is produced via the electron radiative attachment to C$_5$N \citep{walsh09}. 

\subsubsection{Phosphorus-bearing carbon chains - C$_n$P}

Although phosphorus (P) has a relatively small elemental abundance, it plays a crucial role in the development of life. Among the known eight phosphorus-bearing molecules, C$_2$P (or CCP) is the only P-bearing carbon-chain species. 
It has been detected toward IRC+10216 \citep{Halfen08}. 
All carbon chains in this group have doublet ground states (Table \ref{tab:quant}). Higher order chains, C$_3$P and C$_4$P, show a higher value of dipole moments, but they are yet to be detected in the ISM or circumstellar environments. Since the overall elemental abundance of phosphorous is small, higher-order phosphorous chains are expected to have very low abundances. 
CCP may be produced by radical-radical reactions, between CP and hydrocarbons (CCH and C$_3$H), or ion-molecule chemistry involving P$^{+}$ and HCCH followed by the dissociative electron recombination reaction \citep{Halfen08}. 

\subsubsection{HC$_n$O family}

Four neutral HC$_n$O ($n=2, 3, 5, 7$) chains have been identified toward TMC-1. The detection summary of this group indicates odd $n$ chains are more abundant compared to their even $n$ counterparts. This trend is the same as in the C$_n$O family.  
All neutral chains have doublet ground states and dipole moment values are less than 3 Debye. The observed cation HC$_3$O$^+$ has a singlet ground state and a dipole moment of 3.41 Debye (see Table \ref{tab:quant}). As mentioned before, C$_n$O and HC$_n$O are linked through their formation routes (sec Sec. \ref{sec:2_2_3}).  

\subsubsection{HC$_n$S family}

This family is similar to HC$_n$O but contains sulfur instead of oxygen. Only two neutral species, HCCS and HC$_4$S have been identified toward TMC-1. Observed statistics suggest chains with even $n$ have higher abundance than odd $n$ species. 
All neutral chains of this group have doublet ground states (see Table \ref{tab:quant}). The dipole moments of neutral species are less than 2.2 Debye. HCCS is mainly formed through the reaction, C + H$_2$CS \citep{Cernicharo2021S}. 
HC$_4$S is produced through the reaction between C and $\rm{H_2C_3S}$ and by the dissociative recombination reaction of $\rm{H_2C_4S^{+}}$, which is formed via reactions of S + $\rm{C_4H_3^{+}}$ and S$^{+}$ + $\rm{C_4H_3}$ \citep{fuentetaja2022}. 
For $\rm{HC_3S^{+}}$, proton transfer to C$_3$S from HCO$^{+}$ and $\rm{H_3O^{+}}$ is the main formation route. 
The reactions of S$^{+}$ + $c,l-\rm{C_3H_2}$ and S + $c,l-\rm{C_3H_3}$ are also equally important and efficient \citep{cernicharo2021HC3S}.

% HC4S formatiom pathways can reproduce the observed abudance in ISM. See Fuentetaja et al. 2022. 

\subsubsection{Cyanopolyynes - HC$_{2n+1}$N}

Cyanopolyynes are the most important, interesting, and ubiquitous organic carbon chains ($n=1, 2, 3, 4, 5$) detected in the ISM so far. As mentioned above, HC$_3$N was the first detected carbon-chain molecule in space. In this series, five species, starting from HC$_3$N to HC$_{11}$N, have been found in TMC-1. %\citep{Morris76,Little77,Kroto78,Broten78,Bell85}. 
All these species have also been detected toward IRC+10216, except HC$_{11}$N \citep{Morris76,Winnewisser1978,Matthews1985}. 
In this series, especially HC$_3$N and HC$_5$N, have been identified in various star-forming environments (see Section \ref{sec:3_32}). 
Three cations, HC$_3$NH$^+$, HC$_5$NH$^+$, and HC$_7$NH$^+$, have also been identified toward TMC-1. All neutral cynaopolyynes have a singlet ground state and show a trend of increasing dipole moment with length of the chain (see Table \ref{tab:quant}). 
Unlike other carbon-chain species, the cyanopolyyne family could form on dust surfaces through reactions N + C$_{2n+1}$H ($n=1, 2, 3, 4$) and H + C$_{2n+1}$N ($n=1, 2, 3, 4$) (see Section \ref{sec:3_12} for more detail regarding the formation of cyanopolyynes in the gas phase). Protonated cyanopolyynes (e.g., $\rm{HC_3NH^{+}}$, $\rm{HC_5NH^{+}}$) are mainly formed via a proton donor (e.g., HCO$^+$) to cyanopolyynes (e.g., HC$_3$N, HC$_5$N).  Protonated cyanopolyynes are destroyed by dissociative electron recombination reactions \citep{Marcelino20}.
%discussed in more detail with examples.

\subsubsection{Allenic chain family - HC$_{2n}$N}

HCCN was the first member of the allenic chain family, HC$_{2n}$N, observed in space \citep{Guelin91}, and HC$_4$N was the second. These species have been identified toward IRC+10216. The allenic chain family has a triplet ground state and shows increasing dipole moment with size, similar to cyanopolyynes and other families (Table \ref{tab:quant}). HC$_4$N may form through the reactions of C$_3$N + CH$_2$ and C$_3$H + HCN. For this family, ion-molecule paths are relatively slow \citep{Cernicharo04}. %and less likely than in the case of cyanopolyynes \citep{Cernicharo04}. 
HCCN is formed by the reactions between atomic nitrogen and $\rm{H_nCCH^{+}}$.


\subsection{Statistics of Detected Species} \label{sec:2_1}

More than 270 individual molecular species
%comprised of different elements, 
have been identified in the ISM and circumstellar envelopes by astronomical observations (CDMS\footnote{\url{https://cdms.astro.uni-koeln.de/classic/molecules}}, \cite{mcgu22}). 
Most of them are observed in the gas phase via their rotational transition lines, and very few of them are observed in the solid phase. 
Carbon is the fourth most abundant element in the Galaxy 
%by mass 
and plays a pivotal role in interstellar chemistry. 
A large part of interstellar dust grains is also made of carbon. 
A variety of carbon-chain species have been identified in the space ranging from simple linear chains to cyclic and PAHs. 

Figure \ref{fig:detectionsummary} shows the cumulative plot of carbon-chain detection together with the histogram plot in each year starting from 1971, the first carbon-chain detection year, until 2022. 
The lower panel shows the carbon-chain species including molecules containing benzene rings and fullerenes, and the upper panel depicts the plot excluding these species. 
Following the definition mentioned earlier in Section \ref{sec:1_3}, 118 carbon-chain species have been discovered so far.  
This accounts for 43\% of all the known 270 molecules. 
If we exclude benzene ring and PAH-type molecules, the number goes down to 101, which means that 37\% of all known molecules in the ISM and circumstellar shells belong to the linear/bending 
%jct- do you mean linear/branched ?
and cyclic carbon-chain species. 

Figure \ref{fig:pichart} illustrates the detection statistics of all known carbon-chain species per decade starting from the 1970s. We show the summary of all detected carbon-chain species in two pie charts, one including all detected carbon-chain species (right panel) and another including all species except molecules containing benzene rings and fullerenes (left panel). 
The left chart shows detection rates of 11\% in the 1970s, 17\% in the 1980s, 12\% in the 1990s, 12\% in the 2000s, 11\% in the 2010s, and 37\% in the 2020s (only 2020-2022). 
The right chart shows detection rates of
%a slightly different detection rate in each decade; 
9\% in the 1970s, 15\% in the 1980s, 11\% in the 1990s, 11\% in the 2000s, 12\% in 2010, and 42\% in the 2020s (2020-2022).
The detection rate in the 2020s becomes larger in the right chart compared to the left chart, meaning that molecules containing benzene ring(s) have been discovered in the last two years (Section \ref{sec:3_11}).

In summary, there has been a dramatic change in discovered carbon-chain species in the 2020s. 
It shows the detection of carbon-chain species in each decade was almost similar starting from the 1970s to 2010s. 
Around 50 (38) molecular species (if we exclude molecules containing benzene rings and fullerenes) have been detected in the last three years (2020-2022), i.e., the detection rate is 42$\%$ (37\%). Most of the PAH-type species are discovered in this period. A point that we would like to emphasize is that we have just only three years and the number in this decade is still increasing. All known/detected and unknown/possible candidates for future detection carbon-chain species are summarized and noted in Table \ref{tab:quant}. 

\begin{sidewaysfigure*}
\includegraphics[width=0.85\columnwidth]{cumulative.pdf}
\caption{The number of detected carbon-chain species in each year (blue bars) and its cumulative plot (red curves and numbers). The upper panel shows those excluding molecules including benzene rings and fullerene, while the lower one shows those including them. Numbers indicated in the red font in 2022 are the total detected carbon-chain species until the end of 2022.}
\label{fig:detectionsummary}
\end{sidewaysfigure*}

\begin{figure*}
\begin{center}
\includegraphics[width=0.7\textwidth]{pi.pdf}
\end{center}
\caption{Proportion of detected carbon-chain species every 10 years and in the three years of the 2020s. The left panel shows the proportion excluding molecules that contain benzene rings and fullerene. The right panel shows the proportions when considering all carbon chain species.}
\label{fig:pichart}
\end{figure*}


\subsection{Warm Carbon-Chain Chemistry (WCCC)} \label{sec:2_2}

In Section \ref{sec:1_2}, we mentioned that carbon-chain molecules have been classically known as early-type species, because they are abundant in young starless cores and deficient in evolved star-forming cores. Against this classical picture, \cite{sakai2008} detected various carbon-chain molecules from IRAS\,04368+2557 in the low-mass star-forming region L1527.
The derived rotational temperature from the C$_{4}$H$_{2}$ lines is $12.3\pm0.8$ K, which is higher than excitation temperatures of carbon-chain species in the starless core TMC-1 ($\approx 4-8$ K).
They proposed that evaporation of CH$_{4}$ from ice mantles could be the trigger of formation of carbon-chain molecules in the lukewarm envelopes around low-mass protostars, and named such a carbon-chain formation mechanism ``Warm Carbon-Chain Chemistry (WCCC)''.
A second WCCC source, IRAS\,15398-3359 in the Lupus star-forming region, was discovered soon after \citep{sakai2009}.
This
%The discovery of the second WCCC source 
suggested that the WCCC mechanism may be a common feature around low-mass protostars.
%is not a unique chemical feature in L1527 but may be a prevalent feature around low-mass protostars.

Later studies using chemical simulations confirmed the formation mechanism of carbon-chain molecules starting from CH$_{4}$ around temperatures of 25--30 K \citep{hassel2008}.
The CH$_{4}$ molecules react with C$^{+}$ in the gas phase to produce C$_{2}$H$_{3}$$^{+}$ or C$_{2}$H$_{2}$$^{+}$.
The C$_{2}$H$_{2}$$^{+}$ ion reacts with H$_{2}$ leading to C$_{2}$H$_{4}$$^{+}$.
Then, electron recombination reactions of C$_{2}$H$_{3}$$^{+}$ and C$_{2}$H$_{4}$$^{+}$ produce C$_{2}$H$_{2}$.
Regarding WCCC, the review article by \cite{sakai2013} summarized related studies in detail.
We thus avoid duplication here.

However, an important question has been raised since the review of \cite{sakai2013}, namely, the origin(s) of WCCC sources, which is still controversial.
The focus is on how CH$_{4}$-rich ice is formed.
This means that carbon atoms need to be adsorbed onto dust grains without being locked up in CO molecules.
\cite{sakai2008} proposed one possible scenario that shorter collapse times of prestellar cores may produce conditions needed for WCCC sources.
On the other hand, \cite{spezzano2016} suggested that variations in the far ultraviolet (FUV) interstellar radiation field (ISRF) could produce carbon-chain-rich or COM-rich conditions, based on their observational results toward the prestellar core L1544. They found a spatial distribution of $c$-C$_{3}$H$_{2}$ in a region relatively exposed to the ISRF, while CH$_{3}$OH was found in a relatively shielded region. In this scenario, the FUV ISRF destroys CO, a precursor of CH$_{3}$OH, leading to formation of C and/or C$^{+}$, precursors of carbon-chain species.
\cite{spezzano2020} also found similar trends with observations of $c$-C$_{3}$H$_{2}$ and CH$_{3}$OH toward six starless cores.
They concluded that the large-scale effects have a direct impact on the chemical segregation; $c$-C$_{3}$H$_{2}$ is enhanced in the region more illuminated by the ISRF, whereas CH$_{3}$OH tends to reside in the region shielded from the ISRF.
Such chemical segregation observed in starless cores may be inherited to the protostellar stage and recognized as the different classes of WCCC protostars and COM-rich hot corinos.
Recently, several authors investigated the effects of these factors on the abundances of carbon-chain species and COMs by chemical simulations \citep{aikawa2020, kalvans2021}. We discuss their model results in detail in Section \ref{sec:4}.


\subsection{Concept of Hot Carbon-Chain Chemistry} \label{sec:2_3}

The discovery of the WCCC mechanism around low-mass protostars naturally raised a question: are carbon-chain molecules also formed around high-mass ($m_{\ast}>8 M_{\odot}$) protostars?
%massive young stellar objects (MYSOs; $m_{\ast}>8 M_{\odot}$) as well as low-mass protostars?
With such a motivation, observations and chemical simulations focusing on carbon-chain species around massive young stellar objects (MYSOs) have proceeded since the late 2010s. Here, we briefly explain the concept of ``Hot Carbon-Chain Chemistry (HCCC)'' proposed to explain the observations and chemical simulations of several MYSOs.
Details of these observational and simulation studies are summarized in Sections \ref{sec:3_3} and \ref{sec:4}, respectively.

\begin{figure*}[ht]
\begin{center}
\includegraphics[width=0.8\textwidth]{fig_HCCC.pdf}
\end{center}
\caption{Temperature dependence of carbon-chain chemistry. WCCC occurs around 25--35 K, while HCCC occurs in higher-temperature regions.}
\label{fig:HCCCconcept}
\end{figure*}

Figure \ref{fig:HCCCconcept} shows a schematic view of carbon-chain chemistry around MYSOs based on the result of the chemical simulation by \cite{tani2019model}. 
We distinguish HCCC from WCCC depending on the temperature.
The HCCC mechanism refers to carbon-chain formation in the gas phase, adsorption onto dust grains, and accumulation in ice mantles during the warm-up phase (25 K $<T<$ 100 K), followed by evaporation into the gas phase in the hot-core stage ($T>$ 100 K).
This mechanism is particularly important for cyanopolyynes (HC$_{2n+1}$N).
Cyanopolyynes are relatively stable species because of no dangling bond, and they are not destroyed by reactions with O or H$_{2}$, which are destroyers of other carbon-chain species with dangling bonds. Instead, cyanopolyynes are consumed by reactions with ions such as HCO$^{+}$ and other protonated ions in the gas phase.
Thus, the destruction rates of cyanopolyynes in the gas phase are lower than those of the other carbon-chain species, which enables cyanopolyynes to then be adsorbed onto dust grains.

During the warm-up stage, cyanopolyynes are efficiently formed by the gas-phase reactions between C$_{2n}$H$_{2}$ and CN. The formed cyanopolyynes are then adsorbed onto dust grains and accumulated in ice mantles.
When conditions reach their sublimation temperatures above 100 K, the species sublimate into the gas phase and their gas-phase abundances show peaks.

Other radical-type species, such as CCH and CCS, would not behave as cyanopolyynes do, because they are efficiently destroyed by the gas-phase reactions with O or H$_{2}$ \citep{tani2019model}. 
Their gas-phase peak abundances are reached just after WCCC starts to form them around 25 K, and decrease as the temperature increases.
Thus, we expect that radical-type carbon-chain species are abundant in the lukewarm regions and deficient in the hot-core regions, whereas the emission of cyanopolyynes is expected to show their peak at the hot-core regions, similar to the emission of COMs. If we admit these two types of carbon-chain species, then radical-type species would be recognized as ``WCCC tracers'', whereas cyanopolyynes would be regarded as ``HCCC tracers''.
The difference between WCCC and HCCC is that WCCC only refers to the carbon-chain formation starting from CH$_{4}$, which is important in lukewarm regions (25--35 K), while HCCC includes successive processes from the warm regions ($T\approx25-100$ K) to the hot-core region ($T>100$ K).

The points of this section are summarized below.
\begin{enumerate}
\item Carbon-chain molecules account for around 40\% of the interstellar molecules. These molecules have been detected in the ISM since the 1970s, and an increased number of reported detections made by recent Green Bank 100m telescope (GBT) and Yebes 40m telescope observations are astonishing.
\item WCCC refers to the carbon-chain formation mechanism in the lukewarm gas ($T\approx25-35$ K) starting from CH$_{4}$ desorbing from dust grains around 25 K. The gas-phase reaction between CH$_{4}$ and C$^{+}$ is the trigger of the WCCC mechanism.
\item HCCC refers to the gas-phase carbon-chain formation and adsorption and accumulation in ice mantles during the warm-up phase ($T< 100$ K), and their sublimation in the hot-core phase ($T>100$ K).
\end{enumerate}

\section{Observations} \label{sec:3}

\subsection{Carbon-Chain Species in Dark Clouds} \label{sec:3_1}

\subsubsection{Carbon-Chain Molecules in TMC-1} \label{sec:3_11}

\begin{sidewaysfigure*}
\includegraphics[width=1.02\columnwidth]{fig_TMC1.pdf}
\caption{Top panels: Moment 0 map of the CCS ($J_N=4_3-3_2$) line overlaid by black contours indicating moment 0 map of the HC$_{3}$N ($J=5-4$) line. These data were obtained with the Nobeyama 45m radio telescope (beam size=$37^{\prime\prime}$ at 45 GHz). The original data were provided by Dr. Fumitaka Nakamura (NAOJ) and Dr. Kazuhito Dobashi (Tokyo Gakugei University). The magenta cross shows the position of the Cyanopolyyne Peak (CP) observed by the two line survey projects, GOTHAM and QUIJOTE projects. The spectral figures are from \cite{McGuire2020summary} and \cite{cernicharo2021hydrocarboncycle}.}
\label{fig:tmc1}
\end{sidewaysfigure*}

Taurus Molecular Cloud-1 (TMC-1) is one of the most well studied dark clouds \citep[e.g.,][]{kaifu2004}.
Its ``Cyanopolyyne Peak'' (hereafter TMC-1 CP) is the famous site where carbon-chain molecules are particularly abundant (Figure \ref{fig:tmc1}).
Many carbon-chain molecules were discovered from this dark cloud by radio astronomical observations (Section \ref{sec:2_0}).

\cite{dobashi2018} identified four velocity components ($v_{\rm {LSR}}$ = 5.727, 5.901, 6.064, and 6.160 km\,s$^{-1}$) with very high velocity resolution (0.0004 km\,s$^{-1}$) spectra of the CCS and HC$_{3}$N lines in the 45 GHz band obtained by the Z45 receiver \citep{Nakamura2015PASJ} installed on the Nobeyama 45m radio telescope.
They revealed the gas kinematics of the TMC-1 filament and it was found that these velocity components are moving inward toward the center of the TMC-1 filament.
\cite{dobashi2019} identified 21 subfilaments in the TMC-1 filament using CCS ($J_{N}=4_{3}-3_{2}$; 45.379033 GHz) line data. They found that the subfilaments have line densities that are close to the critical line density for dynamical equilibrium ($\sim 17 M_{\odot}$\,pc$^{-1}$).
These results indicate that self-gravity is important in the dynamics of the subfilaments.
The CCS ($J_{N}=4_{3}-3_{2}$) line was also used for measurement of the line-of-sight magnetic field strength by its Zeeman splitting \citep{nakamura2019}.
The derived magnetic field strength is $\sim117\pm21$ $\mu$G, which leads to a conclusion that the TMC-1 filament is magnetically supercritical.
%jct - I am not sure I understand the next sentence:
%As with these studies, the chemical and physical characteristics of TMC-1 have been explored by observations of carbon-chain molecules.
%KT - TI would like to summarize the observational studies using the CCS line toward TMC-1 and mention that the carbon-chain lines are useful to investigate physical conditions of starless cores. Then, I have added the next sentence:
As these studies show, rotational-transition lines of carbon-chain species are useful to investigate physical conditions of starless cores.

Two research groups have been carrying out line survey observations toward TMC-1 CP and have reported the detection of new interstellar molecules.
Their discoveries are still ongoing during the writing of this review article.
We briefly summarized the detection of carbon-chain molecules in Section \ref{sec:2_1}, and here we mention their groundbreaking results in more detail.

One project is GOTHAM (GBT Observations of TMC-1: Hunting Aromatic Molecules\footnote{\url{https://greenbankobservatory.org/science/gbt-surveys/gotham-survey/}}) using the Green Bank 100m telescope led by Dr. Brett A. McGuire. This project is a high sensitivity (2 mK) and high velocity resolution (0.02 km\,s$^{-1}$) spectral line survey in the X, K, and Ka bands (see Fig. \ref{fig:tmc1}).
The beam sizes (FWHM) are $1.4^{\prime}$, $32^{\prime\prime}$, and $26.8^{\prime\prime}$ for the X-Band receiver, KFPA, and Ka-Band (MM-F1) receiver, respectively\footnote{\url{https://www.gb.nrao.edu/scienceDocs/GBTpg.pdf}}.
They have analyzed spectra using the spectral line stacking and matched filter methods \citep{loomis2018,loomis2021} utilizing the velocity information derived by \cite{dobashi2018}, and achieved detection of new, rare interstellar molecules.

\cite{mcguire2021} detected two polycyclic aromatic hydrocarbons (PAHs), 1- and 2-cyanonaphthalene, containing two rings of benzene via spectral matched filtering.
Their molecular structures are shown in Figure \ref{fig:tmc1}.
The nitrile bond (-CN) makes the dipole moment larger, thus aiding the detection of benzonitrile ($c$-C$_6$H$_5$CN), the first detected species with a benzene ring \citep{McGuire2018}.
Benzonitrile has also been detected in other sources: Serpens 1A, Serpens 1B, Serpens 2, and L1521F \citep{burkhardt2021natureastronomy}.
Although the detection of a pure PAH was considered to be difficult due to their small dipole moments, \cite{burkhardt2021} achieved the detection of indene ($c$-C$_9$H$_8$).
\cite{mccarthy2021} detected 1-cyano-1,3-cyclopentadiene ($c$-C$_5$H$_5$CN), a five-membered ring, and \cite{lee2021} detected 2-cyano-1,3-cyclopentadiene, which is an isomer with a little higher energy (5 kJ\,mol$^{-1}$ or 600 K).
Thus, not only molecules with benzene structure, but also molecules with five-membered rings with a nitrile bond have been detected.
\cite{loomis2021} reported the detection of HC$_{11}$N.
%, whose presence in TMC-1 was once rejected \citep{loomis2016}.
%jct - one cannot reject the presence, only place upper limits.
%KT - I agree with this point.
HC$_4$NC, an isomer of HC$_5$N, has been detected by \cite{Xue2020}, and soon after \cite{cernicharo2020HC4NC} also reported its detection using the Yebes 40m telescope.
\cite{Xue2020} ran chemical simulations with formation pathways of electron recombination reactions of HC$_5$NH$^+$ and HC$_4$NCH$^+$, and reproduced the observed abundance of HC$_4$CN.

The other project is QUIJOTE (Q-band Ultrasensitive Inspection Journey to the Obscure TMC-1 Environment) line survey using the Yebes 40m telescope led by Dr. Jose Cernicharo.
This line survey project covers 31.0--50.3 GHz with a frequency resolution of 38.15 kHz, corresponding to $\sim0.29$ km\,s$^{-1}$ at 40 GHz.
They have achieved sensitivities of 0.1 -- 0.3 mK and various molecules have been successfully detected without stacking analyses.
The beam sizes are $56^{\prime\prime}$ and $31^{\prime\prime}$ at 31 GHz and 50 GHz, respectively \citep{Fuentetaja2022HCCCHCCC}.

This group reported the detection of many pure hydrocarbons consisting of only carbon and hydrogen: e.g., 1- and 2-ethynyl-1,3-cyclopentadiene \citep[$c$-C$_5$H$_5$CCH,][]{cernicharo2021cyclopentadiene}; benzyne \citep[$ortho$-C$_6$H$_4$,][]{cernicharo2021benzyne}; $c$-C$_3$HCCH, $c$-C$_5$H$_6$, $c$-C$_9$H$_8$ \citep{cernicharo2021hydrocarboncycle}; fulvenallene \citep[$c$-C$_5$H$_4$CCH$_2$,][]{cernicharo2022fulvenallene}; and CH$_2$CCHC$_4$H \citep{fuentetaja2022}.
The detection of such pure hydrocarbons is astonishing because their dipole moments are very small.
In addition to pure hydrocarbons, the QUIJOTE project has also detected carbon-chain ions: e.g., HC$_3$O$^+$ \citep{cernicharo2020HC3O}; HC$_7$NH$^+$ \citep{cabezas2022HC7NH}; HC$_3$S$^+$ \citep{cernicharo2021HC3S}; HCCS$^+$ \citep{cabezas2022}; C$_5$H$^+$ \citep{cernicharo2022C5H}). It has also detected five cyano derivatives \citep[$trans$-CH$_3$CHCHCN, $cis$-CH$_3$CHCHCN, CH$_2$C(CH$_3$)CN, $gauche$-CH$_2$CHCH$_2$CN, $cis$-CH$_2$CHCH$_2$CN;][]{cernicharo2022cyano}.
The very high sensitivity line survey observations achieved by the QUIJOTE project reveal a wide variety of carbon-chain chemistry.
At the same time, these results raise new questions 
%about carbon-chain chemistry, 
because the abundances of some of the newly detected molecules cannot be explained by chemical models.
%Our knowledge of carbon-chain chemistry is growing up.


\subsubsection{Revealing Main Formation Pathways of Carbon-Chain Molecules by $^{13}$C Isotopic Fractionation} \label{sec:3_12}

Since carbon-chain molecules are unstable even in vacuum chambers on the Earth, it is difficult to conduct laboratory experiments about their reactivity.
Instead, the observed $^{13}$C isotopic fractionation of carbon-chain molecules and the relative differences in abundance among the $^{13}$C isotopologues was proposed to be a key method for revealing their main formation mechanisms \citep{takano1998}.
The $^{13}$C isotopic fractionation is effective only in low-temperature conditions as it depends mainly on isotope exchange reactions, whose backward reactions can proceed only at warm or high temperatures. The elemental $^{12}$C/$^{13}$C abundance ratio in the local ISM is around 60--70 \citep[e.g.,][]{milam2005}.
Hence, high-sensitivity observations are necessary to detect the $^{13}$C isotopologues of carbon-chain molecules with high enough signal-to-noise (S/N) ratios to compare their abundances (i.e., S/N$>10$). With these constraints, TMC-1 CP is the most promising target source, because of the low-temperature condition ($T\approx10$ K) and high abundances of carbon-chain molecules (Section \ref{sec:3_11}).
In fact, TMC-1 CP has the largest number of carbon-chain species investigated for $^{13}$C isotopic fractionation.
Thanks to developing observational facilities, larger molecules can be investigated within reasonable observing times.
The same method has also been applied for other molecular clouds and envelopes around YSOs.
In this subsection, we review such studies conducted in dark clouds.

Three cyanopolyynes have been investigated at TMC-1 CP; HC$_{3}$N \citep{takano1998}, HC$_{5}$N \citep{tani2016}, and HC$_{7}$N \citep{burkhardt2018}.
\cite{takano1998} observed the three $^{13}$C isotopologues of HC$_{3}$N (H$^{13}$CCCN, HC$^{13}$CCN, and HCC$^{13}$CN) using the Nobeyama 45m radio telescope.
The relative abundance ratios of the three $^{13}$C isotopologues were derived to be $1.0:1.0:1.4$ ($\pm0.2$) ($1\sigma$) for [H$^{13}$CCCN]:[HC$^{13}$CCN]:[HCC$^{13}$CN].
%Based on these results, they discussed the possible main formation mechanism of HC$_{3}$N.
Table \ref{tab:HC3N} summarizes correspondences of the possible main formation pathway of HC$_{3}$N and its expected $^{13}$C isotopic fractionation \citep{tani2017}.
Regarding the last one (the electron recombination reaction of HC$_{3}$NH$^{+}$), various formation pathways of the HC$_{3}$NH$^{+}$ ion should compete, and then clear $^{13}$C isotopic fractionation would not be seen in HC$_{3}$NH$^{+}$, as well as HC$_{3}$N.
The reaction between of ``C$_{2}$H$_{2}$ + CN $\rightarrow$ HC$_{3}$N +H'' can explain the observed $^{13}$C isotopic fractionation in TMC-1 CP.
%\cite{takano1998} proposed that the reaction between a hydrocarbon with two carbon atoms and a molecule with a $^{13}$C enriched CN group is probably most important for the HC$_{3}$N formation.
%Three possible formation processes which directly form HC$_{3}$N were considered;
%\begin{enumerate}
    %\item the electron recombination reaction of HC$_{3}$NH$^{+}$ (HC$_{3}$NH$^{+}$ + e$^{-}$ $\rightarrow$ HC$_{3}$N + H),
    %\item the neutral-neutral reaction between C$_{2}$H$_{2}$ and CN (C$_{2}$H$_{2}$ + CN $\rightarrow$ HC$_{3}$N +H), and 
    %\item the neutral-neutral reaction between CCH and HNC (CCH + HNC $\rightarrow$ HC$_{3}$N +H).
%\end{enumerate}

\begin{table*}[t]
%\tabcolsep7.5pt
\caption{Main formation mechanisms of HC$_{3}$N and expected $^{13}$C isotopic fractionation}
\label{tab:HC3N}
\begin{center}
\begin{tabular}{ccc}
\hline
Formation Route & Expected fractionation pattern & Starless cores \\
\hline
C$_{2}$H$_{2}$ + CN $\rightarrow$ HC$_{3}$N + H & $1:1:x$ & TMC-1 CP, L1521B \\
CCH + HNC $\rightarrow$ HC$_{3}$N + H & $y:1:z$  & L134N \\
HC$_{3}$NH$^{+}$ + e$^{-}$ $\rightarrow$ HC$_{3}$N + H & $\approx 1:1:1$ & ... \\
\hline
\end{tabular}
\end{center}
\begin{tabnote}
``Expected fractionation pattern'' means the [H$^{13}$CCCN]:[HC$^{13}$CCN]:[HCC$^{13}$CN] ratio. Here $x$, $y$, and $z$ are arbitrary values.
\end{tabnote}
\end{table*}

%Regarding the first one, the formation processes of the HC$_{3}$NH$^{+}$ ion are needed to be considered.
%There are several formation routes to produce the ion.
%The most probable formation route is reaction between C$_{3}$H$_{3}$$^{+}$ and N.
%The C$_{3}$H$_{3}$$^{+}$ ion has two isomers ($c$-C$_{3}$H$_{3}$$^{+}$ and $l$-C$_{3}$H$_{3}$$^{+}$).
%$c$-C$_{3}$H$_{3}$$^{+}$ is more stable than the linear form, and has three identical carbon atoms. 
%If the carbon atoms in HC$_{3}$N come from $c$-C$_{3}$H$_{3}$$^{+}$, no concentration of $^{13}$C isotopologues should occur.
%Then, this route was rejected.
%In the case of the other formation routes of HC$_{3}$NH$^{+}$, situations that $^{13}$C is concentrated in specific positions in hydrocarbons and their ions, and a nitrogen atom attack the concentrated $^{13}$C were likely difficult to occur.
%\cite{takano1998} concluded that the route (1) is not the main formation mechanism of HC$_{3}$N in TMC-1 CP.

%If $^{13}$C is more concentrated in CN than C$_{2}$H$_{2}$, HCC$^{13}$CN should be more abundant than the other two $^{13}$C isotopologues, because the triple bond between carbon atom and nitrogen atom in CN is preserved during the reaction.
%Hence, the carbon atom next to the nitrogen atom should be inherited from CN.
%The two carbon atoms in C$_{2}$H$_{2}$ are identical, and then H$^{13}$CCCN and HC$^{13}$CCN should have the same abundances, if the second route is the main formation mechanism of HC$_{3}$N.
%At that time, they did not know the $^{13}$C isotopologues of CCH, and they did not discuss in detail about the third route.
%However, this route is unlikely as we see the results of the $^{13}$C isotopic fractionation of CCH later.


At the time five $^{13}$C isotopologues of HC$_{5}$N were detected \citep{takano1990}
there was no evidence for the $^{13}$C isotopic fractionation, because S/N ratios were low.
More than 25 years later, \cite{tani2016} successfully detected the lines of the five $^{13}$C isotopologues of HC$_{5}$N  ($J=9-8$ and $16-15$ at 23 GHz and 42 GHz bands, respectively) with S/N ratios of 12--20.
The derived abundance ratios among the five $^{13}$C isotopologues of HC$_{5}$N are $1.00:0.97:1.03:1.05:1.16$ ($\pm0.19$) ($1\sigma$) for [H$^{13}$CCCCCN] : [HC$^{13}$CCCCN] : [HCC$^{13}$CCCN] : [HCCC$^{13}$CCN] : [HCCCC$^{13}$CN].
Hence, even if the S/N ratios increase, there is no clear difference in abundance among the five $^{13}$C isotopologues of HC$_{5}$N, unlike HC$_{3}$N.
\cite{tani2016} proposed that the reactions between hydrocarbon ions (C$_5$H$_m$$^{+}$; $m=3-5$) and nitrogen atoms, followed by electron recombination reactions are the most possible main formation mechanism of HC$_{5}$N at TMC-1 CP.

%\cite{tani2016} considered three possible main formation routes of cyanopolyynes including longer than HC$_{5}$N;
%\begin{enumerate}
 %   \item the reactions of C$_{2n}$H$_{2}$+CN,
 %   \item the growth of the carbon-chains of cyanopolyynes by reactions with C$_{2}$H$_{2}$$^{+}$ followed by electron recombination reactions, and
 %   \item the reactions between hydrocarbon ions (C$_{2n+1}$H$_m$$^{+}$; $m=3-5$) and nitrogen atoms followed by electron recombination reactions.
%\end{enumerate}
%If HC$_{5}$N is formed by the first route, which is the same one suggested for HC$_{3}$N, HCCCC$^{13}$CN should be clearly more abundant than the others.
%If the second route is the main formation mechanism of HC$_{5}$N, the $^{13}$C isotopic fractionation is inherited from HC$_{3}$N, and the abundance ratios would be $1.0:1.0:1.4$ for [HCC$^{13}$CCCN]:[HCCC$^{13}$CCN]:[HCCCC$^{13}$CN].
%However, these characters cannot be seen in the observed $^{13}$C isotopic fractionation of HC$_{5}$N.


\cite{burkhardt2018} detected six $^{13}$C isotopologues of HC$_{7}$N and five $^{13}$C isotopomers of HC$_{5}$N using the Green Bank 100m telescope.
H$^{13}$CC$_{6}$N could not be detected in their observations.
They found no significant difference among the $^{13}$C isotopomers of HC$_{7}$N, as similar to the case of HC$_{5}$N.
They concluded that the significant formation route for HC$_{7}$N is the reaction of hydrocarbon ions and nitrogen atoms, which is the same conclusion for HC$_{5}$N by \cite{tani2016}.

HC$_{3}$N has been investigated in two other starless cores, L1521B and L134N (L183) \citep{tani2017}.
Their observational results show different fractionation patterns between the two sources.
The $^{13}$C isotopic fractionation in L1521B is similar to that in TMC-1 CP; HCC$^{13}$CN is more abundant than the others, and the other two $^{13}$C isotopologues have similar abundances.
On the other hand, the results in L134N are different from TMC-1 CP and L1521B; the abundance ratios in L134N are $1.5$ ($\pm0.2$)$:1.0:2.1$ ($\pm0.4$) ($1\sigma$) for [H$^{13}$CCCN]:[HC$^{13}$CCN]:[HCC$^{13}$CN].
%These results mean that all of the $^{13}$C isotopologues have different abundances.
%\cite{tani2017} summarized correspondences of the main formation pathway of HC$_{3}$N and its expected $^{13}$C isotopic fractionation (\textbf{table \ref{tab:HC3N}}).
The last column of Table \ref{tab:HC3N} summarizes the names of molecular clouds that were suggested to have each main formation route.
Based on these classifications (Table \ref{tab:HC3N}), the main formation mechanisms of HC$_{3}$N are determined as the reactions of C$_{2}$H$_{2}$ + CN in L1521B and CCH + HNC in L134N.
The C$^{13}$CH/$^{13}$CCH abundance ratio was found to be $>1.4$ in L134N \citep{tani2019}, which agrees with the abundance ratio of $1.5$ ($\pm0.2$)$:1.0$ for [H$^{13}$CCCN]:[HC$^{13}$CCN].
This is further supporting evidence for the conclusion that the main formation pathway of HC$_3$N includes CCH in L134N.
The difference between L134N and TMC-1CP/L1521B is probably brought about by different HNC/CN abundance ratios (HNC/CN = 35.6 and 54.2 in TMC-1 CP and L134N, respectively).
The HNC/CN abundance ratio depends on the cloud's age, and then the main formation mechanism of cyanopolyynes likely changes over the course of the cloud's evolution.

In addition to cyanopolyynes, C$_{n}$S ($n=2,3$) and C$_{2n}$H ($n=1,2$) have been studied for $^{13}$C fractionation in TMC-1 CP \citep{sakai2007,sakai2010,sakai2013JPCA}.
\cite{sakai2007} detected the lines of $^{13}$CCS and C$^{13}$CS ($J_N=2_1-1_0$, $F=5/2-3/2$).
The abundance ratio of C$^{13}$CS/$^{13}$CCS was derived to be $4.2\pm2.3$ ($3\sigma$).
%In addition, the CCS/$^{13}$CCS abundance ratio was calculated at $230\pm130$ ($3\sigma$), indicating that the $^{13}$C species are significantly depleted.
They proposed that the reaction between CH and CS is the main formation route of CCS in TMC-1 CP.
The abundance ratio of C$^{13}$CH/$^{13}$CCH was derived to be $1.6\pm0.4$ ($3\sigma$) \citep{sakai2010}. %and significant depletion of $^{13}$C isotopomers was observed
To explain the abundance difference between the two $^{13}$C isotopomers of CCH, \cite{sakai2010} proposed that the reaction of CH + C is the main formation mechanism of CCH at TMC-1 CP.

Unlike cyanopolyynes, the $^{13}$C isotopic fractionation of CCS and CCH does not necessarily provide information on their main formation mechanisms.
\cite{furuya2011} ran chemical simulations including $^{13}$C and investigated effects of the isotopomer-exchange reactions.
They considered the following two isotopomer-exchange reactions for CCH and CCS, respectively:
\begin{equation} \label{equ:CCH}
^{13}{\rm {CCH}} + {\rm {H}} \rightleftharpoons {\rm {C}}^{13}{\rm {CH}} + {\rm {H}} + 8.1 {\rm {K}};
\end{equation}
and 
\begin{equation} \label{equ:CCS}
^{13}{\rm {CCS}} + {\rm {S}} \rightleftharpoons {\rm {C}}^{13}{\rm {CS}} + {\rm {S}} + 17.4 {\rm {K}}.
\end{equation}
They also included the following neutral-neutral exchange reaction of CCS to reproduce the observed isotopomer ratio of CCS:
\begin{equation} \label{equ:CCS2}
^{13}{\rm {CCS}} + {\rm {H}} \rightleftharpoons {\rm {C}}^{13}{\rm {CS}} + {\rm {H}} + 17.4 {\rm {K}}.
\end{equation}
This reaction is regarded as a catalytic reaction by the hydrogen atom.
At low temperature conditions ($T\approx10$ K), C$^{13}$CH and C$^{13}$CS should be more abundant than the other $^{13}$C isotopomers by Reactions (\ref{equ:CCH}) -- (\ref{equ:CCS2}).
Their model results can explain the observed abundance differences between the two $^{13}$C isotopomers of CCH and CCS in TMC-1 CP \citep{sakai2007,sakai2010}.
It was found that C$^{13}$CH is more abundant than $^{13}$CCH in the other starless cores \citep[L1521B and L134N;][]{tani2017}, and such a character may be common for cold dark clouds.
Such exchange reactions may contribute to larger species such as C$_{3}$S and C$_{4}$H \citep{sakai2013JPCA}.


\subsubsection{Dilution of the $^{13}$C species} \label{sec:3_13}

From the observations deriving the $^{13}$C isotopic fractionation of carbon-chain species in dark clouds, dilution of the $^{13}$C species is inferred, i.e., the $^{12}$C/$^{13}$C ratios of carbon-chain molecules are higher than the local elemental abundance ratio \citep[60--70;][]{milam2005}.
The variations of the $^{12}$C/$^{13}$C ratios of carbon-chain species could give another hint about carbon-chain chemistry in dark clouds.
This is caused by the following reaction \citep[e.g.,][]{langer1984}:
\begin{equation} \label{equ:13C}
^{13}{\rm {C}}^{+} + {\rm {CO}} \rightleftharpoons {\rm {C}}^{+} + {\rm{^{13}CO}} + 35 {\rm {K}}.
\end{equation}
The backward reaction is ineffective in cold-temperature conditions ($\sim10$ K), and then the abundance of $^{13}$C$^{+}$ should decrease.
Ionic and atomic carbons (C$^{+}$ and C) are the main parent species of carbon-chain molecules and the low abundance of $^{13}$C$^{+}$ results in deficient $^{13}$C isotopologues of carbon-chain molecules.  
Table \ref{tab:13C} summarizes the $^{12}$C/$^{13}$C ratios of carbon-chain molecules derived in three starless cores: TMC-1 CP; L1521B; and L134N.
From Table~\ref{tab:13C}, the following tendencies can be inferred:
\begin{enumerate}
    \item Cyanopolyynes (HC$_{2n+1}$N) have relatively lower $^{12}$C/$^{13}$C ratios compared to the other hydrocarbons. Especially, the $^{13}$C isotopomers of CCH have high values.
    \item The $^{12}$C/$^{13}$C ratios are different among the dark clouds. The ratios in L134N are relatively low compared to the others.   
\end{enumerate}
The first point may be caused by the isotopomer-exchange reactions mentioned in Section \ref{sec:3_12}.
Such isotopomer-exchange reactions are not expected for cyanopolyynes.
In addition, \cite{tani2019} proposed that the high $^{12}$C/$^{13}$C ratios of CCH seem to be caused by reactions between hydrocarbons (CCH, C$_{2}$H$_{2}$, $l,c$-C$_{3}$H) and C$^{+}$.
If $^{13}$C$^{+}$ is diluted by Reaction (\ref{equ:13C}), these reactions will produce hydrocarbons with high $^{12}$C/$^{13}$C ratios.

The second point may be related to the evolution of the starless cores; TMC-1 CP and L1521B are considered to be chemically younger than L134N \citep[e.g.,][]{dickens2000}.
Currently, the available data are limited, and such studies have been conducted mainly at TMC-1 CP.
Thus, it is difficult to reach firm conclusions.
Future high-sensitivity survey observations are needed to reveal the detailed mechanisms causing the dilution of $^{13}$C species, which would give information about the chemical relationships among carbon-chain molecules in dark clouds.

\begin{table}[t]
%\tabcolsep7.5pt
\caption{The $^{12}$C/$^{13}$C ratios of carbon-chain molecules in dark clouds}
\label{tab:13C}
\begin{center}
\begin{tabular}{lccc}
\hline
Species & TMC-1 CP & L1521B & L134N \\
\hline
%& \multicolumn{2}{c}{{\textbf{TMC-1 CP}}} & \multicolumn{2}{c}{{\textbf{L1521B}}} & \multicolumn{2}{c}{{\textbf{L134N}}} \\
%\hline
H$^{13}$CCCN & $79\pm11$$^{(a)}$ & $117\pm16$$^{(b)}$ & $61\pm9$$^{(b)}$ \\
HC$^{13}$CCN & $75\pm10$$^{(a)}$ & $115\pm16$$^{(b)}$ & $94\pm26$$^{(b)}$ \\
HCC$^{13}$CN & $55\pm7$$^{(a)}$ & $76\pm6$$^{(b)}$ & $46\pm9$$^{(b)}$ \\
H$^{13}$CCCCCN & $98\pm14$$^{(c)}$ & & \\
HC$^{13}$CCCCN & $101\pm14$$^{(c)}$ & \\
HCC$^{13}$CCCN & $95\pm12$$^{(c)}$ & \\
HCCC$^{13}$CCN & $93\pm13$$^{(c)}$ & \\
HCCCC$^{13}$CN & $85\pm11$$^{(c)}$ & \\
HC$_{7}$N & $73\pm21$$^{(d)}$ &\\
$^{13}$CCH & $>250$$^{(e)}$ & $>271$$^{(f)}$ & $>142$$^{(f)}$ \\
C$^{13}$CH & $>170$$^{(e)}$ & $252^{+77}_{-48}$$^{(f)}$ & $101^{+24}_{-16}$$^{(f)}$ \\
$^{13}$CCCCH & $141\pm15$$^{(g)}$ &  \\
C$^{13}$CCCH & $97\pm9$$^{(g)}$ & \\
CC$^{13}$CCH & $82\pm5$$^{(g)}$ & \\
CCC$^{13}$CH & $118\pm8$$^{(g)}$ & \\
$^{13}$CCS & $230\pm43$$^{(h)}$ & \\
C$^{13}$CS & $54\pm2$$^{(h)}$ & \\
$^{13}$CCCS & $>206$$^{(g)}$ & \\
C$^{13}$CCS & $48\pm5$$^{(g)}$ &  \\
CC$^{13}$CS & $30-206$$^{(g)}$ & \\
\hline
\end{tabular}
\end{center}
\begin{tabnote}
Errors indicate the standard deviation. \\
References: (a) \cite{takano1998}, (b) \cite{tani2017}, (c) \cite{tani2016}, (d) \cite{burkhardt2018} (average value), (e) \cite{sakai2010}, (f) \cite{tani2019}, (g) \cite{sakai2013JPCA}, (h) \cite{sakai2007}.
%$^{\rm a}$table footnote; $^{\rm b}$second table footnote.
\end{tabnote}
\end{table}

\subsection{Carbon-Chain Species around Low-Mass YSOs} \label{sec:3_2}

Carbon-chain chemistry around low-mass young stellar objects (YSOs), namely WCCC, has been reviewed in \cite{sakai2013}, and we do not discuss WCCC in detail.
Instead, we summarize observational results published after the review article of \citet{sakai2013}.

Several survey observations with single-dish telescopes targeting carbon-chain molecules and COMs have been conducted. 
\cite{grani2016} carried out survey observations of CH$_{3}$OH and C$_{4}$H using the IRAM 30m telescope.
A tentative correlation between the gas-phase C$_{4}$H/CH$_{3}$OH abundance ratio and the CH$_{4}$/CH$_{3}$OH abundance ratio in ice was found.
These results support the scenario of WCCC: sublimation of CH$_{4}$ is a trigger of carbon-chain formation in lukewarm gas \cite[e.g.,][]{hassel2008}.
\cite{higuchi2018} conducted survey observations of CCH, $c$-C$_{3}$H$_{2}$, and CH$_{3}$OH toward 36 Class 0/I protostars in the Perseus molecular cloud using the IRAM 30m and Nobeyama 45m radio telescopes. They found that the column density ratio of CCH/CH$_{3}$OH varies by two orders of magnitudes among the target sources, and the majority of the sources show intermediate characters between hot corino and WCCC.
In other words, hot corino and WCCC are at opposite ends of a spectrum, and most low-mass YSOs could have both characters, in which carbon-chain molecules and COMs coexist.
In addition, they found a possible trend that sources with higher CCH/CH$_{3}$OH ratios are located near cloud edges or in isolated clouds.
Similar trends were suggested by \cite{lefloch2018} with data taken in the IRAM Large Program ``Astrochemical Surveys At IRAM (ASAI)''.
The ASAI program is an unbiased line survey from 80 to 272 GHz toward 10 sources with various evolutionary stages.
\cite{lefloch2018} found a difference in environmental conditions between hot corino and WCCC sources: i.e., inside and outside dense filamentary cloud regions, respectively.

High-angular-resolution observations with interferometers, such as ALMA and NOEMA, have revealed spatial distributions of carbon-chain molecules around low-mass YSOs. 
As already mentioned, most low-mass YSOs appear to show characteristics of both hot corino and WCCC.
\cite{oya2017} detected both a carbon-chain molecule (CCH) and several COMs toward the low-mass Class 0 protostar L483.
They confirmed that both WCCC and hot corino characters coexist in this source with their spatially resolved data.

\cite{zhang2018} found that CCH emission traces the outflow cavity with signatures of rotation with respect to the outflow axis toward the NGC\,1333 IRAS 4C outflow in the Perseus molecular cloud, using ALMA.
\cite{tychoniec2021} analyzed ALMA data sets toward 16 protostars and investigated spatial distributions of each molecule.
They found that CCH and $c$-C$_{3}$H$_{2}$ trace the cavity wall.
This could be explained by the fact that the chemistry of the cavity wall is similar to 
photodissociation region (PDR)
%jct - changed notation here
%photodominated region (PDR) 
%KT - Thank you so much
chemistry. The photodissociation of molecules by UV radiation keeps high gas-phase abundances of atomic carbon (C), which is a preferable condition for the formation of hydrocarbons.

\cite{pineda2020} found a streamer-like structure toward IRAS\,03292+3039 in the Perseus star-forming region with NOEMA. Such a streamer may be well traced by carbon-chain molecules such as CCS and HC$_{3}$N, if it is considered to be chemically young.
%trace chemically young gas. 
The streamer in this source seems to bring fresh gas from outside of the dense core ($>10,500$ au) down to the central protostar where the disk forms.
Thus the properties of such streamers are potentially important for the formation and evolution of protoplanetary disks.

Taking advantage of the characteristics of carbon-chain molecules, we can trace unique features around low-mass YSOs.
Rotational-transition lines of carbon-chain molecules are now found to be useful tracers not only in early starless clouds but also around star-forming cores.
ALMA Band 1 and the next generation Very Large Array (ngVLA) will cover the 7 mm band or lower frequency bands, which are suitable for observations of carbon-chain molecules, especially longer ones.
Future observations using such facilities will offer new insights into the carbon-chain chemistry around protostars, including low-, intermediate-, and high-mass systems.

\subsection{Carbon-Chain Species in High-Mass Star-Forming Regions} \label{sec:3_3}

\subsubsection{Chemical Evolutionary Indicators} \label{sec:3_31}

As mentioned in Section \ref{sec:1_2}, carbon-chain molecules classically have been known to be abundant in young starless cores and good chemical evolutionary indicators in low-mass star-forming regions.
However, it was unclear whether carbon-chain species can be used as chemical evolutionary indicators in high-mass star-forming regions and behave similarly as in the case of low-mass regions.

Survey observations of several carbon-chain species (HC$_{3}$N, HC$_{5}$N, CCS, and $c$-C$_{3}$H$_{2}$) and N$_{2}$H$^{+}$ were carried out using the Nobeyama 45m radio telescope \citep{tani2018survey,tani2019survey}.
\cite{tani2018survey} observed the HC$_{3}$N and HC$_{5}$N lines in the 42--45 GHz band toward 17 high-mass starless cores (HMSCs) and 35 high-mass protostellar objects (HMPOs), and \cite{tani2019survey} observed HC$_{3}$N, N$_{2}$H$^{+}$, CCS, and $c$-C$_{3}$H$_{2}$ in the 81--94 GHz band toward 17 HMSCs and 28 HMPOs.
They proposed the column density ratio of $N$(N$_{2}$H$^{+}$)/$N$(HC$_{3}$N) as a chemical evolutionary indicator in high-mass star-forming regions (Figure \ref{fig:evolution}). This column density ratio decreases as cores evolve from the starless (HMSC) to protostellar (HMPOs) stage. 
Sources that were categorized as HMSCs based on the IR observations but that are associated with molecular lines of COMs (CH$_{3}$OH or CH$_{3}$CN) and/or SiO (plotted as the blue diamond in Figure \ref{fig:evolution}) tend to fall between HMSCs and HMPOs.
These sources are considered to contain early-stage protostars in the dense, dusty cores, which are not easily detected with IR observations.
Thus, these sources appear to be at an intermediate evolutionary stage between HMSCs and HMPOs.

\begin{figure}[th]
\begin{center}
\includegraphics[width=0.6\textwidth]{fig_evolution.pdf}
\end{center}
\caption{A chemical evolutionary indicator in high-mass star-forming regions \citep{tani2019survey}. Off-HMPO means that IRAS-observed positions were not at exact continuum peak positions, but the beam covered the continuum core in the beam edge. Blue diamond plots are sources that were identified as HMSCs based on the IR observations, but which are associated with molecular emission lines of COMs (CH$_{3}$OH and/or CH$_{3}$CN) or SiO.}
\label{fig:evolution}
\end{figure}

The decrease of the $N$(N$_{2}$H$^{+}$)/$N$(HC$_{3}$N) ratio means that HC$_{3}$N is efficiently formed and N$_{2}$H$^{+}$ is destroyed, as cores evolve.
It is a notable point that the tendency of this column density ratio is the opposite to that in low-mass star-forming regions \citep{suzuki92,benson98}.
This tendency could be explained by higher temperatures and extended warm regions around HMPOs compared to low-mass protostars.
N$_{2}$H$^{+}$ is destroyed by a reaction with CO, i.e., abundant in the gas phase after being desorbed from dust grains, and HC$_{3}$N can be formed by CH$_{4}$ via the WCCC mechanism or via C$_{2}$H$_{2}$ desorbed from dust grains.
The desorption of CO, CH$_{4}$, and C$_{2}$H$_{2}$ from dust grains occurs when temperatures reach around 20 K, 25 K, and 50 K, respectively.
In summary, the gas-phase chemical composition is affected by the local heating from young massive protostars, and a chemical evolutionary indicator apparently shows an opposite trend compared to that of the low-mass case. Thus, carbon-chain species likely have the potential to become chemical evolutionary indicators even for high-mass protostars.


\subsubsection{Cyanopolyynes around High-Mass Protostars} \label{sec:3_32}

The survey observations mentioned in Section \ref{sec:3_31} show a possibility of different carbon-chain chemistry between high-mass and low-mass star-forming regions.
The detection rates of HC$_{3}$N, HC$_{5}$N, $c$-C$_{3}$H$_{2}$, CCS are 93\%, 50\%, 68\%, and 46\%, respectively, in high-mass star-forming regions \citep{tani2018survey,tani2019survey}.
On the other hand, \cite{law2018} reported that the detection rates of HC$_{3}$N, HC$_{5}$N, $l$-C$_{3}$H, C$_{4}$H, CCS, and C$_{3}$S are 75\%, 31\%, 81\%, 88\%, 88\%, and 38\%, respectively.
Thus, cyanopolyynes (HC$_{3}$N and HC$_{5}$N) show higher detection rates, while CCS is relatively deficient in high-mass star-forming regions, compared to low-mass regions.
These results imply that carbon-chain chemistry around MYSOs is different from WCCC found around low-mass YSOs.

\cite{tani2017MYSO} conducted observations of the multi-transition lines of HC$_{5}$N using the Green Bank 100m and Nobeyama 45m radio telescopes toward four MYSOs.
The derived rotational temperatures are around 20 -- 25 K, which are similar to the temperature regimes of the WCCC mechanism.
However, the derived rotational temperatures are lower limits due to contamination from extended cold components covered by the single-dish telescopes.
The MYSO G28.28-0.36 shows a particular unique chemical character: carbon-chain species are rich, but COMs are deficient \citep{tani2018MYSO}.
This source may be 
%jct - check
%KT - The sentences are fine. Thank you.
analogous to
%a counterpart of 
the WCCC source L1527.
These results are suggestive of the chemical diversity around MYSOs, as similar to low-mass cases (hot corino and WCCC).

Since the above HC$_{5}$N excitation temperatures derived with single-dish data are lower limits, it could not be concluded that carbon-chain molecules exist in higher temperature regions around MYSOs compared to the WCCC sources.
\cite{tani2021carbon} derived the CCH/HC$_{5}$N abundance ratios toward three MYSOs and compared the observed ratio with the results of their chemical simulations to constrain temperature regimes where carbon-chain species exist.
The CCH/HC$_{5}$N abundance ratio is predicted to decrease as the temperature increases, because CCH shows a peak abundance in the gas phase around 30 K, while the gas-phase HC$_{5}$N abundance increases as the temperature rises up to around 100 K \citep{tani2019model}.
Details about the chemical simulations are presented in Section \ref{sec:4}.
The observed CCH/HC$_{5}$N abundance ratios toward all of the three MYSOs are $\sim15$, which is much lower than that toward the low-mass WCCC source L1527 ($625^{+3041}_{-339}$).
The observed abundance ratios around MYSOs agree with the simulations around 85 K, while the ratio in L1527 matches with the simulations around 35 K.
Therefore, carbon-chain species, at least HC$_{5}$N, around MYSOs appear to exist in higher temperature regions than the locations where the WCCC mechanism occurs.
Such results indicate that carbon-chain chemistry around MYSOs may be different from the WCCC mechanism.
%jct - I made the language more cautious here 
%KT - Thank you so much

\begin{figure*}
\includegraphics[width=0.95\textwidth]{fig_HC5Nmap.pdf}
\caption{Comparison of spatial distributions around MYSOs obtained by ALMA (color scale; continuum image, white contours; the CH$_{3}$OH line ($1_{0,1}-2_{1,2}$, $v_{t}=1$; $E_{\rm {up}}=302.9$ K), black lines; the HC$_{5}$N line ($J=35-34$; $E_{\rm {up}}=80.5$ K). This figure is a modified version of Taniguchi et al. (submitted). The contour levels are relative values of the peak intensities, and the contour levels are indicated below each panel. The ellipses at the bottom of each panel indicate the angular resolutions; open one corresponds to the continuum images, and white and black ones correspond to the moment 0 maps of CH$_{3}$OH and HC$_{5}$N, respectively.}
\label{fig:HC5Nmap}
\end{figure*}

More recently, spatial distributions of the HC$_{5}$N line ($J=35-34$; $E_{\rm {up}}=80.5$ K) around MYSOs have been revealed by ALMA Band 3 data (Taniguchi et al., submitted). 
This line has been detected from three sources among five target sources. Figure \ref{fig:HC5Nmap} shows the comparison of spatial distributions among HC$_{5}$N, CH$_{3}$OH, and continuum emission in Band 3 toward the three MYSOs.
The HC$_{5}$N emission shows single peaks associated with the continuum peaks and is consistent with the emission of the CH$_{3}$OH line ($1_{0,1}-2_{1,2}$, $v_{t}=1$; $E_{\rm {up}}=302.9$ K) which should trace hot core regions with temperatures above 100 K. These results also support the ``Hot Carbon-Chain Chemistry'' scenario.

Carbon-chain molecules are formed even around MYSOs.
However, the WCCC mechanism does not match the observational results around MYSOs, and carbon-chain species, especially cyanopolyynes, exist in hot regions with temperatures above 100 K. Figure \ref{fig:chem_div} shows a summary of chemical types found around low-mass and high-mass YSOs, respectively.
Currently, a candidate of pure HCCC sources is the MYSO G28.28-0.36, in which COMs are deficient but HC$_{5}$N is abundant \citep{tani2018MYSO}. Larger sources samples are needed to clarify the apparent chemical diversity around MYSOs.

\begin{figure*}[hbt]
\begin{center}
\includegraphics[scale=0.6]{fig_chemical_diversity.pdf}
\end{center}
\caption{Chemical types found around low-mass and high-mass YSO.}
\label{fig:chem_div}
\end{figure*}

\subsubsection{Carbon-chains around Intermediate-Mass Protostars}\label{sec:3_4}

%The carbon-chain chemistry has been studied in low-mass and in high-mass star-forming regions.
Observations have revealed WCCC and HCCC in low-mass and high-mass star-forming regions, respectively. However, the carbon-chain chemistry around intermediate-mass protostars ($2\:M_{\odot} < m_* <  8\: M_{\odot}$) has not been well studied.
It is essential to understand the carbon-chain chemistry around protostars comprehensively because large physical gaps between low-mass and high-mass protostars prevent us to compare them straightforwardly. 

Survey observations of several carbon-chain molecules in the Q band are ongoing with the Yebes 40m telescope.
Twelve target intermediate-mass YSOs were selected from the source list of the SOFIA Massive (SOMA) Star Formation Survey project \citep{debuizer2017,liu2019}. 
At the writing stage of this review article, we have data toward the five target sources.
We briefly demonstrate the initial results here.

\begin{table*}[t]
\tabcolsep7.5pt
\caption{Summary of initial results of molecules around intermediate-mass protostars}
\label{tab:IM}
\begin{center}
\begin{tabular}{@{}l c c c c c@{}}
\hline
Species & Cepheus E & L1206 & HH288 & IRAS\,00420+5530 & IRAS\,20343+4129 \\
\hline
HC$_{3}$N          & $\surd$ & $\surd$ & $\surd$ & $\surd$ & $\surd$ \\
HC$_{5}$N          & $\surd$ & $\surd$ & $\surd$ & $\surd$ & $\surd$ \\
C$_{3}$H           &         & $\surd$ & $\surd$ & $\surd$ &         \\
C$_{4}$H           & $\surd$ & $\surd$ & $\surd$ & $\surd$ & $\surd$ \\
$l$-H$_{2}$CCC     & $\surd$ & $\surd$ & $\surd$ & $\surd$ & $\surd$ \\
$c$-C$_{3}$H$_{2}$ & $\surd$ & $\surd$ & $\surd$ & $\surd$ & $\surd$ \\
CCS                & $\surd$ & $\surd$ & $\surd$ & $\surd$ & $\surd$ \\
C$_{3}$S           & $\surd$ & $\surd$ & $\surd$ &         &         \\
CH$_{3}$CCH        &         & $\surd$ & $\surd$ &         & $\surd$ \\
CH$_{3}$OH         & $\surd$ & $\surd$ & $\surd$ & $\surd$ & $\surd$ \\
CH$_{3}$CHO        & $\surd$ & $\surd$ & $\surd$ & $\surd$ & $\surd$ \\
H$_{2}$CCO         & $\surd$ & $\surd$ & $\surd$ & $\surd$ & $\surd$ \\
HNCO               & $\surd$ & $\surd$ & $\surd$ & $\surd$ & $\surd$ \\
CH$_{3}$CN         & $\surd$ & $\surd$ & $\surd$ & $\surd$ & $\surd$ \\
$^{13}$CS          & $\surd$ & $\surd$ & $\surd$ & $\surd$ & $\surd$ \\
C$^{34}$S          & $\surd$ & $\surd$ & $\surd$ & $\surd$ & $\surd$ \\
HCS$^{+}$          & $\surd$ & $\surd$ & $\surd$ & $\surd$ & $\surd$ \\
H$_{2}$CS          & $\surd$ & $\surd$ & $\surd$ & $\surd$ & $\surd$ \\
\hline
\end{tabular}
\end{center}
\begin{tabnote}
The ``$\surd$'' mark indicates detection with S/N ratios above 5. \\
\end{tabnote}
\end{table*}

Table \ref{tab:IM} summarizes the detection of molecules toward five intermediate-mass protostars.
Not only carbon-chain molecules, but also several COMs and sulfur-bearing species have been detected.
We derived the rotational temperatures using seven HC$_{5}$N lines (from $J=12-11$ to $J=18-17$; $E_{\rm{up}}=$ 9.97 K -- 21.9 K). 
The derived rotational temperatures are around 20 K, which agrees with the WCCC mechanism.
In the fitting procedure, we note that the $J=12-11$ line cannot be fitted simultaneously, with the flux of this line being too high.
%because the plots of this line are outliers.
This likely indicates that the $J=12-11$ line traces mainly outer cold gas which is a remnant of molecular clouds, whereas the other lines trace the lukewarm gas where carbon-chain molecules could efficiently form by the WCCC mechanism.
COMs and carbon-chain molecules have been found to coexist, indicating that hybrid-type sources are common also in the intermediate-mass regime.

In summary, carbon-chain formation occurs around intermediate-mass protostars, as well as low-mass and high-mass protostars. Future detail analysis combined with physical parameters and observations with interferometers will reveal whether the HCCC mechanism occurs around intermediate-mass protostars.


\subsection{Carbon-Chain Species in Disks} \label{sec:3_5}

Before the ALMA era, there were only a few reported detections of carbon-chain species in the protoplanetary disks around T Tauri stars and Herbig Ae stars.
\cite{henning2010} detected CCH from two T Tauri stars, DM Tau and LkCa\,15, with the IRAM Plateau de Bure Interferometer (PdBI). 
The first detection of HC$_{3}$N from protoplanetary disks was achieved using the IRAM 30m telescope and PdBI \citep{chapillon2012}.
They detected the HC$_{3}$N lines ($J=12-11$ and $16-15$) from protoplanetary disks around two T Tauri stars, GO Tau and LkCa\,15, and the Herbig Ae star MWC\,480. 
Studies of disk chemistry have dramatically progressed, thanks to ALMA observations.
In this subsection, we summarize studies related to carbon-chain species in protoplanetary disks.

\cite{qi2013} reported the first detection of $c$-C$_{3}$H$_{2}$ in a disk around the Herbig Ae star HD\,163296 using the ALMA Science Verification data. Its emission is consistent with the Keplerian rotating disk and traces a ring structure from an inner radius of $\sim 30$ au to an outer radius of $\sim 165$ au.
The HC$_{3}$N line ($J=27-26$; $E_{\rm {up}}=165$ K) has been detected from the protoplanetary disk of MWC\,480, which is a Herbig Ae star in the Taurus region, using ALMA \citep{oberg2015}.
Angular resolutions are $0.4^{\prime \prime}-0.6^{\prime \prime}$, corresponding to 50--70 au.
The data can spatially resolve the molecular emission, and show a velocity gradient caused by Keplerian rotation of the protoplanetary disk.
\cite{oberg2015} also detected CH$_{3}$CN and H$^{13}$CN in the same observation, and found that the abundance ratios among the three species in the protoplanetary disk of MWC\,480 are different from those in the solar-type protostellar binary system IRAS\,16298-2422.
Thus, they suggested that varying conditions among protoplanetary disks can lead to chemical diversity in terms of carbon-chain species.
%jct - check above.
%that chemical reactions at the protoplanetary disks produce unique abundance ratios.
%KT - Thank you so much

\cite{bergner2018} conducted survey observations of CH$_{3}$CN and HC$_{3}$N toward four T Tauri stars (AS\,209, IM Lup, LkCa\,15, and V4046 Sgr) and two Herbig Ae stars (MWC\,480 and HD\,163296) with ALMA.
Typical angular resolutions are from $\sim0.5^{\prime\prime}$ to $\sim1.5^{\prime\prime}$.
They detected the HC$_{3}$N ($J=27-26$) line from all of their target sources.
Besides, the $J=31-30$ and $J=32-31$ lines have been detected from MWC\,480.
The spatial distributions of HC$_{3}$N and CH$_{3}$CN show similarity; compact and typically well within the bounds of the dust continuum.
HC$_{3}$N is considered to be formed by only the gas-phase reactions: C$_{2}$H$_{2}$ + CN and CCH + HNC (see also Table \ref{tab:HC3N}). 

The Molecules with ALMA at Planet-forming Scales (MAPS) ALMA Large Program has studied disk chemistry around five target sources (IM Lup, GM Aur, AS\,209, HD\,163296, and MWC\,480) in Bands 3 and 6 \citep{oberg2021maps}.
Typical beam sizes are around $0.3^{\prime \prime}$ and $0.1^{\prime \prime}$ in Band 3 and Band 6, respectively.
\cite{ilee2021} presented the results for HC$_{3}$N, CH$_{3}$CN, and $c$-C$_{3}$H$_{2}$.
The HC$_{3}$N and $c$-C$_{3}$H$_{2}$ lines have been clearly detected from four of the target sources, with the exception being IM Lup, where only one $c$-C$_{3}$H$_{2}$ line has been tentatively detected.
The $c$-C$_{3}$H$_{2}$ emission shows clear ring-like features in AS\,209, HD\,163296, and MWC\,480, suggestive of an association with the outer dust rings.
Two HC$_{3}$N lines  ($J=11-10$ and $29-28$) show ring-like distributions in AS\,209 and HD\,163296, whereas the $J=29-28$ line appears centrally peaked in MWC\,480.
The HC$_{3}$N emission of the $J=11-10$ line is similarly extended to that of $c$-C$_{3}$H$_{2}$, but the $J=29-28$ line seems to be more compact.
CH$_{3}$CN, on the contrary, appears to have a ring-like feature only in AS\,209, while more centrally peaked structures are seen in the other sources.
\cite{ilee2021} demonstrated that the observed HC$_{3}$N emission traces upper layers ($z/r=0.1-0.4$) of the protoplanetary disks compared to that of CH$_{3}$CN ($z/r\leq0.1-0.2$).
They also found that the HC$_{3}$N/HCN and CH$_{3}$CN/HCN abundance ratios of the outer regions (50--100 au) in the target disks are consistent with the composition of cometary materials.
The warmer disks, HD\,163296 and MWC\,480, likely have comet formation zones at correspondingly larger radii.

\cite{guzman2021} presented distributions of CCH toward the five protoplanetary disks of the MAPS program.
They proposed that the CCH emission comes from relatively warmer (20--60 K) layers.
In HD\,163296, there is a decrease in the column density of CCH and HCN inside of the dust gaps near $\sim 83$ au, at which a planet has been considered to be located.
The similar spatial distributions of CCH and HCN suggest that they are produced by the same chemical processes, and photochemistry is the most probable one.

%High-mass 
ALMA observations have revealed the presence of disks around not only T Tauri and Herbig Ae stars, but also around more massive, O-/B-type stars.
\cite{csengeri2018} detected the vibrationally-excited HC$_{3}$N line ($J=38-37$, $v_{7}=1e$) around the O-type star G328.2551-0.5321 (O5--O4 type star) with ALMA.
This source is a high-mass protostar in the main accretion phase.
Their data have spatial resolution of around 400 au.
The position-velocity (PV) diagram of this HC$_{3}$N vibrationally-excited line is consistent with a Keplerian disk rotation profile, and they proposed that such HC$_{3}$N vibrationally-excited emission could be a new tracer for compact accretion disks around high-mass protostars.

\cite{tani22} detected the HC$_{3}$N vibrationally-excited lines ($J=24-23$, $v_{7}=2, l=0$ and $2e$) from the hypercompact H$_{\rm {II}}$ (HC\,H$_{\rm {II}}$) region G24.47-0.08 A1 using ALMA Band 6 data.
Their emission morphologies are largely similar to those of CH$_{3}$CN, which was suggested to trace Keplerian disk rotation around a central mass of $20\:M_{\odot}$ in the previous study of \citet{moscadelli2021}.
The column densities of HC$_{3}$N and CH$_{3}$CN were derived using lines of their $^{13}$C isotopologues, and the CH$_{3}$CN/HC$_{3}$N abundance ratios were compared with those in protoplanetary disks around the lower-mass stars obtained by the MAPS program \citep{ilee2021}.
Figure \ref{fig:diskchem} shows the comparisons of the CH$_{3}$CN/HC$_{3}$N abundance ratios in disks.
It is clear that the ratio in the disk around the G24 HC\,H$_{\rm {II}}$ region is higher than those around the lower-mass stars by more than one order of magnitude.

\begin{figure*}[ht]
\begin{center}
\includegraphics[scale=0.5]{fig_diskchem.pdf}
\end{center}
\caption{Comparison of the CH$_{3}$CN/HC$_{3}$N abundance ratios in disks around various stellar masses, which is modified from \cite{tani22}. Results of Herbig Ae and T Tauri stars are from \cite{ilee2021}.}
\label{fig:diskchem}
\end{figure*}

Such a difference in the CH$_{3}$CN/HC$_{3}$N abundance ratio was explained by the HC$_{3}$N and CH$_{3}$CN chemistry in the disk: efficient thermal sublimation of CH$_{3}$CN from ice mantles and rapid destruction of HC$_{3}$N by the UV photodissociation and/or reactions with ions (H$^{+}$, H$_{3}^{+}$, HCO$^{+}$).
In the protoplanetary disks, CH$_{3}$CN is considered to be efficiently formed by dust-surface reactions: (1) the successive hydrogenation reactions of C$_{2}$N; and (2) a radical-radical reaction between CH$_{3}$ and CN \citep{loomis2018disk}.
The derived excitation temperature of CH$_{3}$CN in the G24 HC\,H$_{\rm {II}}$ region ($T_{\rm {ex}}\approx 335$ K) is much higher than its sublimation temperature ($\sim 95$ K), which suggests that CH$_{3}$CN formed on dust surfaces efficiently sublimates by the thermal desorption mechanism.
On the other hand, its excitation temperatures around the Herbig Ae and T Tauri stars were derived to be 30--60 K, which is suggestive of the non-thermal desorption mechanisms such as photodesorption \citep{loomis2018disk,ilee2021}.
This means that CH$_{3}$CN sublimation is not efficient in disks around the Herbig Ae and T Tauri stars, leading to low gas-phase abundances of CH$_{3}$CN.
Both HC$_{3}$N and CH$_{3}$CN could be destroyed by the UV radiation, and the UV photodissociation rate of HC$_{3}$N is higher than that of CH$_{3}$CN by a factor of $\sim2.4$ \citep{legal2019}.
Thus, HC$_{3}$N could be more rapidly destroyed by UV photodissociation.
In addition, HC$_{3}$N is destroyed by reactions with ions, which are expected to be abundant in the H$_{\rm {II}}$ region.
In summary, HC$_{3}$N is likely destroyed rapidly in the G24 HC\,H$_{\rm {II}}$ region.

Until now, there is only one O-type star disk in which the CH$_{3}$CN/HC$_{3}$N abundance ratio has been derived.
We need similar data for an increased sample of sources, including T Tauri, Herbig Ae, and O/B type disks, including probing various evolutionary stages, to further test the apparent tentative trends that have been so far revealed.
%jct - check above
%KT - Thank you so much
%confirm that the above suggestion can be generally applicable.
In addition, for such studies of disk structures, especially around the more distant massive protostars, it is important to conduct unbiased line survey observations with high angular resolution.

%toward massive stars with high angular resolutions that can resolve disk structures.

\subsection{Carbon-Chain Species in Other Environments} \label{sec:3_6}

Carbon-chain molecules have been detected not only from star-forming regions in our Galaxy, but also other environments of the ISM. We briefly summarize the carbon-chain species detected in these regions.

Small hydrocarbons have been known to be present in photodissociation regions (PDRs).
\cite{cuadrado2015} observed small hydrocarbons toward the Orion Bar PDR using the IRAM 30m telescope.
They detected various small hydrocarbons (CCH, C$_{4}$H, $c$-C$_{3}$H$_{2}$, $c$-C$_{3}$H, C$^{13}$CH, $^{13}$CCH, $l$-C$_{3}$H, and $l$-H$_{2}$C$_{3}$) and the $l$-C$_{3}$H$^{+}$ ion.
They found that the spatial distributions of CCH and $c$-C$_{3}$H$_{2}$ are similar but do not follow the PAH emission, and suggested that photo-destruction of PAHs is not a necessary requirement for the observed abundances of the smallest hydrocarbons. 
Instead, the gas-phase reactions between C$^{+}$ and H$_{2}$ produce the small hydrocarbons.
\cite{guzman2015} observed small hydrocarbons (CCH, $c$-C$_{3}$H$_{2}$), $l$-C$_{3}$H$^{+}$, and DCO$^{+}$ toward the Horsehead PDR with the PdBI.
They demonstrated that top-down chemistry, in which large polyatomic molecules or small carbonaceous grains are photo-destroyed into smaller hydrocarbon molecules or precursors, works in this PDR.
Suggestions by these two studies \citep{cuadrado2015, guzman2015} seem in contradiction, but may imply that the carbon-chain chemistry in PDRs differs among regions. 
Thus, further study is needed of carbon-chain chemistry in PDRs.

The envelopes of the carbon-rich Asymptotic Giant Branched (AGB) star IRC+10216 are known as a carbon-chain-rich site.
Several carbon-chain molecules have been discovered for the first time in this source in both radio and infrared regimes (see also section \ref{sec:2_1}). 
For example, C$_{4}$H$_{2}$ has been detected in mid-infrared observations with the 3m Infrared Telescope Facility (IRTF) from this source \citep{fonfria2018}.
\cite{pardo2022} conducted deep line survey observations in the Q band with the Yebes 40m telescope and summarized the detected carbon-chain species in this source.
The rotational temperatures of the carbon-chain species are around 5--25 K, suggesting that carbon-chain species may exist in different regions.
There remain a lot of unidentified lines (U lines), and future laboratory spectroscopic experiments are necessary for line identifications.

The planetary nebula CRL\,618 is another carbon-chain-rich source, also studied by radio and infrared observations. Polyacetylenic chains (C$_{4}$H$_{2}$ and C$_{6}$H$_{2}$) and benzene (C$_{6}$H$_{6}$) have been detected here with the Infrared Space Observatory (ISO) \citep{cernicharo2001}.
The abundances of C$_{4}$H$_{2}$ and C$_{6}$H$_{2}$ are lower than that of C$_{2}$H$_{2}$ by only a factor of 2--4, while benzene is less abundant than acetylene by a factor of $\sim40$.
These authors suggested that UV photons from the hot central star and shocks associated with its high-velocity winds affect the chemistry in CRL\,618: i.e., the UV photons and shocks drive the polymerization of acetylene and the formation of benzene.
These hydrocarbons likely survive in harsh regions compared to star-forming regions. \cite{pardo2005} observed cyanopolyynes up to HC$_{7}$N, and proposed rapid transformation from small cyanide to longer cyanopolyynes in this source.

Carbon-rich AGB stars or planetary nebulae, like IRC+10216 and CRL\,618, appear to possess unique carbon-chain chemistry differing from that in star-forming regions and PDRs, and thus be important laboratories to study carbon chemistry, including PAHs and benzene.
Future observations with infrared telescopes, such as James Webb Space Telescope (JWST), may give us new insights of carbon chemistry, in particular PAHs and fullerenes.

\cite{berne2012} investigated formation process of C$_{60}$ by the infrared observations with {\it {Spitzer}} and {\it {Herschel}} toward the NGC\,7023 nebula.
They found that C$_{60}$ is efficiently formed in cold environments of an interstellar cloud irradiated by the strong UV radiation field. The most plausible formation route is the photochemical processing of large PAHs. 

ALMA observations have detected carbon-chain species in extragalactic sources.
The ALMA Comprehensive High-resolution Extragalactic Molecular Inventory (ALCHEMI) large program has conducted line survey observations from 84.2 GHz to 373.2 GHz toward the starburst galaxy NGC\,253.
Several carbon-chain species (e.g., CCH, $c$-C$_{3}$H$_{2}$, HC$_{3}$N, HC$_{5}$N, HC$_{7}$N, CCS) have been detected from this galaxy \citep{martin2021}. 
The detection of more complex carbon-chain species will be reported (Dr. Sergio Martin, ESO/JAO, private comm.).
\cite{shimonishi2020} detected CCH in a hot molecular core in the Large Magellanic Cloud (LMC), and found that the CCH emission traces outflow cavity, as also seen in the low-mass YSOs in our Galaxy (see section \ref{sec:3_2}).
Such observations toward extragalactic sources with different metallicities will be important for a comprehensive understanding of carbon chemistry in the ISM.

Summaries of this section are as follows: 
\begin{enumerate}
\item Recent line survey observations toward TMC-1 by the Green Bank 100m and Yebes 40m telescopes discovered various and unexpected carbon-chain molecules. However, abundances of some of them cannot be explained by chemical simulations, meaning that our current knowledge about carbon chemistry in the ISM lacks important processes.
\item Survey observations toward low-mass YSOs revealed that carbon-chain species and COMs generally coexist around low-mass YSOs. 
Hot corino and WCCC states are likely to be extreme ends of a continuous distribution.
\item Since carbon-chain molecules trace chemically young gas, their lines can be powerful tracers of streamers, which are important structures to understand star formation and disk evolution.
\item Carbon-chain species are formed around MYSOs. 
ALMA observations have shown that they exist in hot core regions with temperatures above 100 K. Thus, the carbon-chain chemistry is not the WCCC mechanism found around low-mass YSOs, but rather indicates the presence of ``Hot Carbon-Chain Chemistry (HCCC)''.
\item The vibrationally-excited lines of HC$_{3}$N can be used as a disk tracer around massive stars. 
The disk chemistry around massive stars may be different from that around lower-mass stars (i.e., T Tauri and Herbig Ae stars), although there is the need to increase the source samples to confirm this.
\item Infrared observations toward carbon-rich AGB stars and planetary nebulae have detected polyacetylene chains, benzene, and fullerenes in their envelopes. 
These sites are unique laboratories to study carbon chemistry which is different from that in star-forming regions.
\item Beyond our Galaxy, several carbon-chain species have also been detected in other galaxies enabled by high-sensitivity ALMA observations.
\end{enumerate}


\section{Chemical Simulations} \label{sec:4}

Modeling studies about carbon-chain chemistry in starless cores have tried to obtain good agreement with the observed abundances in the dark cloud TMC-1. Here, we review modeling studies covering various types of carbon-chain molecules in dark clouds.

\cite{loison2014} studied several carbon-chain groups (C$_{n}$, C$_{n}$H, C$_{n}$H$_{2}$, C$_{2n+1}$O, C$_{n}$N, HC$_{2n+1}$N, C$_{2n}$H$^{-}$, C$_{3}$N$^{-}$) with gas-grain chemical models including updated reaction rate constants and branching ratios assuming two different C/O ratios (0.7 and 0.95).
They added a total of 8 new species and 41 new reactions, and modified 122 rate coefficients taken from the KInetic Database for Astrochemistry (KIDA, kida.uva.2011).
Their results clearly show that some carbon-chain molecules depend on the C/O elemental abundance ratio (e.g., C$_n$H where $n=4,5,6,8$).
Their models with new parameters can obtain good agreement with the observed abundances in the dark cloud TMC-1 CP, and the models with two C/O ratios (0.7 and 0.95) obtain a similar agreement at different times.
There are two ages that show better agreement between observations and models; $10^{5}$ yr and around ($1-2$)$\times 10^{6}$ yr.
The gas-phase chemistry is dominated at the earlier phase, whereas the grain surface chemistry and gas-grain interaction become more efficient at the later stages.

These authors also compared the modeled results to the observed abundances in another starless core, L134N. Here the models with a C/O ratio of 0.7 are in better agreement with the observed abundances compared to the case with the higher C/O ratio.
Ages when the models agree with the observed abundances in L134N best are ($3-5$)$\times10^{4}$ yr and $\sim6 \times 10^{5}$ yr.
Large amounts of free C, N, and O are available in the gas phase at the first age, while strong depletion effects are predicted at the later stage.
They also suggested that experimental work for the determination of rate constants for the reactions of O + C$_{n}$H and N + C$_{n}$H, especially at low-temperature conditions, is necessary.

Most recent studies have focused on particular molecules that were newly detected at TMC-1 CP (Section \ref{sec:3_11}).
Classical models for dark clouds consider bottom-up chemistry starting from C$^{+}$, with carbon-chain molecules considered to form mainly by gas-phase reactions (Section \ref{sec:1_2}).
However, such classical views need to be revisited. 
For example, it has been revealed that both gas-phase and grain-surface formation routes are important for the reproduction of the observed abundance of H$_{2}$CCCHC$_{3}$N \citep{shingledecker2021}. 
Some molecules detected by the GOTHAM project, especially cyclic molecules, have not been explained by the chemical simulations yet \citep[e.g.,$c$-C$_{9}$H$_{8}$;][]{burkhardt2021}.
These results suggest that small rings are formed through the destruction of PAHs or other unknown processes.
Our knowledge about connections among different categories (e.g., linear, cyclic, PAHs) likely lacks important processes.
Besides, we need to reveal the initial conditions of carbon-bearing species in molecular clouds.
Further observations and chemical simulations are necessary to understand carbon-chain chemistry including the newly detected molecules.

Some recent studies with sophisticated chemical simulations focused on the origin of the chemical diversity around YSOs.
\cite{aikawa2020} demonstrated their results with two phases (the static phase and the collapse phase) and a multilayered ice mantle model, and investigated how the WCCC and hot corino chemistry depend on the physical conditions during the static phase.
They found:
\begin{enumerate}
    \item The lower temperatures during the static phase can produce the WCCC sources more efficiently. 
    \item The lower visual extinction during the static phase can form CH$_{4}$ and carbon-chain molecules become more abundant. 
    \item A longer static phase is preferable for producing the WCCC sources.  
    \item It is difficult to produce the prototypical WCCC sources, where carbon-chain species are rich but COMs are deficient. On the contrary, the hot corino sources and hybrid-type sources where both COMs and carbon-chain species are reasonably abundant could be reproduced.
\end{enumerate}
In warm conditions, grains-surface formation and freeze out of CH$_{4}$, which is a key species for WCCC, become less effective. 
Moreover, the conversion of CO to CO$_{2}$ on grain surfaces becomes important, and the gaseous CO abundance decreases. 
These lead to a low abundance of C$^{+}$, which is formed by the destruction of CO by He$^{+}$.
The C$^{+}$ ion is another key species for WCCC.
Therefore, warm conditions are not suitable for the production of WCCC sources.
In the model with a longer static phase, CH$_{4}$ accumulates during the static phase, leading to a more favorable condition for WCCC.

\cite{kalvans2021} investigated the effects of the UV radiation field and cosmic rays on the formation of WCCC sources.
They concluded that WCCC can be caused by exposure of a star-forming core to the interstellar radiation field (ISRF) or just to cosmic rays (with $\zeta \geq 10^{-16}$ s$^{-1}$).
Such a conclusion agrees with the observational results that hot corino type sources are located inside dense filamentary clouds, while the WCCC sources are located outside such structures \citep[e.g.,][]{lefloch2018}.
These two model studies show that various factors, including conditions before the onset of core collapse, are related to carbon-chain chemistry around low-mass YSOs.
These factors likely are entangled in most sources.

\cite{tani2019model} tried to reproduce the observed abundances of HC$_{5}$N around the three MYSOs (Section \ref{sec:3_32}) with chemical simulations of hot-core models with a warm-up period.
They utilized three different heating timescales ($t_{h}$); $5\times10^4$ yr, $2\times10^5$ yr, and $1\times10^6$ yr, approximating high-mass, intermediate-mass, and low-mass star-formation, respectively \citep{garrod2006}.
They found that the observed HC$_{5}$N abundances around the MYSOs can be reproduced when the temperature reaches its sublimation temperature ($\sim115$ K) or the hot-core phase ($T=200$ K).

\begin{figure*}
\begin{center}
\includegraphics[scale=0.5]{fig_HC5Nmodel.pdf}
\end{center}
\caption{The modeled HC$_{5}$N abundances in the gas phase (red), dust surface (brown), and ice mantle (blue) during the warm-up and hot-core periods \citep{tani2019model}. The black line indicates the temperature evolution. Indicated molecules mark the times of efficient sublimation from dust grains.}
\label{fig:modelHC5N}
\end{figure*}

These authors also investigated cyanopolyyne chemistry in detail during the warm-up and hot-core periods. 
Basically, formation and destruction reactions of HC$_{3}$N, HC$_{5}$N, and HC$_{7}$N are similar.
We give an explanation of HC$_{5}$N as an example. 
Figure \ref{fig:modelHC5N} shows the modeled HC$_{5}$N abundances obtained by the 3-phase (gas, dust, ice) model with a heating timescale of $1\times10^6$ yr \citep{tani2019model}.
The gas-phase HC$_{5}$N abundance (red curves) shows a drastic change with time (or temperature) evolution.
During the warm-up period, HC$_{5}$N is mainly formed by the reaction between C$_{4}$H$_{2}$ and CN.
In addition to this, the reaction between CCH and HC$_{3}$N partly contributes to the HC$_{5}$N formation.
This reaction (CCH+HC$_{3}$N) is important around $t\approx8.5\times10^5$ yr, when the gas-phase HC$_{3}$N abundance increases. At that time, the HC$_{3}$N production is enhanced by the reaction between CCH and HNC.
We indicate some molecules with green arrows in Figure \ref{fig:modelHC5N}.
These indicate that each molecule directly sublimates from dust grains at these various ages.
Methane (CH$_{4}$) sublimates from dust grains around 25 K ($t\approx7.2\times10^{5}$ yr), and carbon-chain formation starts, namely WCCC.
After that, CN and C$_{2}$H$_{2}$ sublimate from dust grain at $t=7.7\times10^5$ yr ($T\approx31$ K) and $t=9.3\times10^5$ yr ($T\approx55$ K), respectively.
The C$_{4}$H$_{2}$ species is formed by the gas-phase reaction of ``CCH + C$_{2}$H$_{2}$ $\rightarrow$ C$_{4}$H$_{2}$ + H''.
When the temperature reaches around 73 K ($t=1.0\times10^{6}$ yr), C$_{4}$H$_{2}$ directly sublimates from dust grains.
The enhancement of the gas-phase abundances of CN and C$_{4}$H$_{2}$ boosts the formation of HC$_{5}$N in the gas phase.

We can see that the HC$_{5}$N abundances in dust surface and ice mantles increase when the gas-phase HC$_{5}$N abundance decreases.
This means that the HC$_{5}$N molecules, which are formed in the gas phase, adsorb onto dust grains and accumulate in ice mantles before the temperature reaches its sublimation temperature ($T\approx115$ K, corresponding to $t=1.2\times10^6$ yr).
Their results are based on the concept of HCCC (Figure \ref{fig:HCCCconcept}).

This chemical simulation is supported by observations of the $^{13}$C isotopic fractionation (c.f., Section \ref{sec:3_12}) of HC$_{3}$N toward the carbon-chain-rich MYSO G28.28-0.36 \citep{tani2016HC3N}.
The proposed main formation pathway of HC$_{3}$N is the reaction between C$_{2}$H$_{2}$ and CN in this MYSO.
This is consistent with the formation process seen in the chemical simulations during the warm-up stage.

\cite{tani2019model} proposed that longer heating timescales of the warm-up stage could produce the carbon-chain-rich conditions by comparisons of six modeled results, focusing on cyanopolyynes.
In the HCCC mechanism, cyanopolyynes are formed in the gas phase, adsorb onto dust grains, and accumulate in the ice mantle.
Their ice-mantle abundances just before sublimation determine the gas-phase peak abundances.
Thus, longer heating timescales allow cyanopolyynes to accumulate in the ice mantle abundantly, leading to their higher gas-phase abundances in hot core regions.
The long heating timescale of the warm-up stage ($t_{h}$) does not necessarily reflect the timescale of stellar evolution or accretion.
It depends on the relationships between the size of the warm region ($R_{\rm{warm}}$) and the infall velocity ($v_{\rm{infall}}$) as suggested by \cite{aikawa2008}:
\begin{equation}
 t_{h}\propto\frac{R_{\rm{warm}}}{v_{\rm {infall}}}.
\end{equation}
If $R_{\rm{warm}}$ becomes larger or $v_{\rm{infall}}$ becomes smaller, $t_{h}$ will be longer.
The $R_{\rm{warm}}$ and $v_{\rm{infall}}$ values should be related to various physical parameters (e.g., protostellar luminosity, density structure, magnetic field strength).
The conditions of $R_{\rm{warm}}$ and $v_{\rm{infall}}$ can be investigated by observations.
Combined observations to derive chemical composition and the values of $R_{\rm{warm}}/v_{\rm {infall}}$ are needed for a more comprehensive understanding of the chemical evolution and diversity around YSOs.


\section{Theoretical Studies} \label{sec:5}

\subsection{Role of Quantum Chemical Studies}

Quantum chemistry is an indispensable tool to study structures and spectral properties of a given molecule, regardless of laboratory stability, and makes this a necessary component of astrochemical analyses. It is probably the best tool to explore and interpret chemical structures, properties, and, most importantly, detectable spectra of unusual molecules detected in space. In the 1960s and 1970s, when radio telescopes came into action with better sensitivity, many species were identified based on the laboratory data of their rotational spectra. However, there were also many unidentified signatures. Closed shell molecules can be easily handled in the laboratory and generate rotational spectra. The problem arises in the case of radical and charged species because of their highly unstable and reactive nature. Initial detection of HCO$^+$ and N$_2$H$^+$ was in the 1970s and based on quantum chemistry \citep[see][and references therein]{fort15}. Following these, Patrick Thaddeus (Columbia University and then Harvard University) was a part of teams that detected nearly three dozen new molecular species \citep{mccarthy2001}. Thaddeus' group standardized the use of quantum chemistry in astrochemical detection. 

Several carbon-chain species such C$_2$H, C$_3$N, and C$_4$H were identified based on SCF (self-consistent field) computation. The usage of quantum chemistry in astrochemistry became popular and common if species had not been synthesized in a laboratory. Several carbon-chain species, such as $l$-C$_3$H$^+$ and C$_6$H$^-$, were identified based on the insight of quantum chemistry. This has continued until recent times with the detection of the kentenyl radical (HCCO) towards various dark clouds in 2015. In another example, the detection of C$_5$N$^-$ in the circumstellar envelope of the carbon-rich star IRC+10216 was based solely on quantum chemical data.

Modern quantum computations consider a lot of improvements with a wide range of quantum mechanical  methods, basis sets, especially coupled cluster theory, at the single, double, and perturbative triples [CCSD(T)] level. The CCSD(T) method is exceptionally good in providing molecular structures and rotational constants, which give us very accurate rotational spectra. 
Over the years, many groups have made remarkable contributions to astrochemistry through quantum chemistry for astrochemical detection, understanding of the formation/destruction, and collisional excitation of different interstellar molecules 
%jct - maybe this should be alphabetical order?
(e.g., L. Allamandola, C. Bauschlicher, V. Barone, P. Botschwina, M. Biczysko, C. Puzzarini  R. C. Fortenberry, J. Gauss, T. J. Lee, K. Peterson, H. F. Schaefer, J. Stanton, D. Woon, J-C Loison, J. Kästner
, N. Balucani, A. Rimola). A comprehensive strategy for treating larger molecules with the required accuracy has been presented by \cite{barone2015}.

% add definition of astromolecules 

\subsubsection{Ground State and Stability}

The ground states of carbon-chain species are crucial because they decide their stability and eventually help us find the true state of the species either in laboratory experiments or in astrophysical environments via a spectral search. Table \ref{tab:quant} summarizes the ground state of all the carbon-chain species included in this review article. The ground state, enthalpy of formation, rotational constants, dipole moment, and polarizability of many carbon-chain species can be found in KIDA database (https://kida.astrochem-tools.org/) as well as in several works in the literature \citep[e.g.,][]{woon09,etim16,etim2017,baldea2019,etim20}. For any particular carbon-chain species, the ground state can be found with the help of quantum chemical study.  

%\subsubsection{Proton BE and Electron BE}
\subsubsection{Dipole moment and Polarizability}

The dipole moment is an important parameter that decides whether a molecule is polar or non-polar. 
Non-polar molecules do not have a permanent electric dipole moment and do not show rotational transitions.
On the other hand, polar molecules have a permanent electric dipole moment and they show rotational transitions.
A higher dipole moment value means a higher intensity of rotational transitions. 
It is crucial because one can say whether it is detectable or not through their rotational transitions. 
The dipole moment values of all the carbon-chain species are summarized in Table \ref{tab:quant}. 
These can be measured theoretically with various quantum chemical methods and basis set by inclusion of structures of molecules. 

The polarizabilities of all carbon-chain species, if available, are summarized in Table \ref{tab:quant}. 
The dipole moment and polarizability are crucial for the estimation of ion-neutral reaction rates. 
Ion-neutral reactions play a crucial role in the ISM, especially in cold dark clouds, for the formation and destruction of various species. For non-polar neutrals, the rate coefficient is given by the so-called Langevin expression:
\begin{equation}
k_{L} = 2\pi e \sqrt{\frac{\alpha_{\rm pola}}{\mu}},
\end{equation}
where $e$ is the electronic charge, $\alpha_{\rm pola}$ is the  polarizability,  $\mu$ is the reduced mass of reactants, and cgs-esu units are utilized so that the units for the rate coefficient are cm$^{3}$ s$^{-1}$. 
For polar-neutral species, trajectory scaling relation is usually used. 
The best-known formula for the ion-dipolar rate coefficient $k_{D}$ in the classical regime and for linear neutrals is the Su-Chesnavich expression. 
The Su-Chesnavich \citep{su82} formula is based on the parameter $x$, defined by
\begin{equation}
x = \frac{\mu_{D}}{\sqrt{2\alpha_{\rm pola} k_BT}},
\end{equation}
where $\mu_{D}$ is an effective dipole moment of the neutral reactant, which is generally very close to the true dipole moment, and $k_B$ is the Boltzmann constant. 
The larger value of $x$ takes, the larger the rate coefficient is.
%jct - I change to \mu_D rather than \mu_{\rm D}; also k_L; \rm is normally used for elements or deuterium

Apart from the estimation of the rate coefficient of ion-neutral reactions, the dipole moment is one of the key parameters to determine the column density of observed species.  
Accurate estimation of the dipole moments is essential to derive realistic values of column density. 
For instance, the dipole moment of C$_4$H was used $\sim$ 0.87 Debye in previous literature, which was the value of mixed states, i.e., ground and excited states. 
The recent result suggests the dipole moment value is 2.10 Debye, which is 2.4 times larger than the values used before \citep{oya20}. As a result, its column densities has been overestimated by a factor of $\sim6$.
%The new estimated dipole moment of C$_4$H has revised its column density for different sources, which was overestimated before $\sim$ 6 times \citep{oya20}.

\subsubsection{Binding Energy of Carbon-Chain Species}

A major portion of carbon-chain species is primarily formed in the gas phase. 
In addition, gas-grain exchange occurs, and many reactions occur on the grain surface.
The binding energy (BE) plays a pivotal role in interstellar chemistry, especially grain surface chemistry, which eventually enriches the gas-phase chemistry. 
Here, we describe the role of binding energy in interstellar chemical processes. 

The BE values of all carbon-chain species are provided (if available) in Table \ref{tab:Ebind}. 
Most of the BE values are mainly taken from KIDA (\url{https://kida.astrochem-tools.org/}). BE estimation of a species heavily depends on the methods and surface used for the calculations \citep[e.g.,][]{pent17,das18,ferrero2020,villadsen2022,minissale2022}. The BE values of all carbon-chain species, especially higher-order linear chains, are estimated based on the addition method that are provided in KIDA. In this method, for instance, the BE of HC$_7$N is estimated by addition of binding energies of HC$_6$N and C. 800 K is usually adopted as the BE of a carbon atom. %However, recent studies measured the BE of carbon atom (C) i.e., greater than 10000 K \citep{wake17}. %We see uncertainties in the measured value of BE of the C atom. 
However, we see a huge difference in BE of a carbon atom between the old value and newly measured values based on quantum chemical study \citep[$\ge 10000$ K;][]{wake17,minissale2022}. 
Thus, the addition method of BE may lead to large uncertainties to estimate BE of carbon-chain molecules, especially longer ones.
%There is one major question, which is uncertainties about the measured values of the C atom's BE, and following that if we assume the addition method of BE estimation is correct, then we may need to use 10000 K instead of 800 K. 
If BE of a carbon atom is $\ge$10,000 K is correct, the addition method of BE may not be valid, because long carbon-chain species will have huge BE values. To overcome this issue, dedicated quantum chemical calculations or the temperature-programmed desorption (TPD) method are required to estimate the BE values of carbon-chain species with higher accuracy. 

Gas-phase species accrete onto the grain surface depending upon their energy barriers with the surfaces. The species may bind to the grain surface via the physisorbed or chemisorbed processes. 
$E_d$ expresses the binding energy (or desorption energy)  of the chemical species. 
Species attached to grain surfaces can migrate from one site to another depending on the barrier energy of migration ($E_b$). 
This is also known as a barrier against diffusion. 
The migration timescale by thermal hopping can be expressed as,
\begin{equation}
t_{\rm hop}={\nu _0}^{-1} {\rm {exp}}\Big(\frac{E_b}{k_B T_d}\Big) \ \rm s,
\end{equation}
where $\nu _0$ is the characteristic vibration frequency of the adsorbed species, and $T_d$ is the grain surface temperature. 
The theoretical expression for $\nu_0$ is,
\begin{equation}
\nu_0=\sqrt{\frac{2n_sE_d}{\pi ^2 m}}  \rm {s^{-1}},
\end{equation}
where $m$ is the mass of the adsorbed particle, and $n_s$ is the number density of sites. 
The utilized values of $n_s$ are $\sim 2\times 10^{14}\,\rm {cm^{-2}}$ for olivine based grain and $\sim 5 \times 10^{13}\,\rm cm^{-2}$ for the amorphous carbon grain, respectively \citep{biha01}. 
The characteristic frequency is $\sim 10^{12}$ Hz.
Diffusion time $t_{\rm diff}$ required for an adsorbed particle over a number of sites is given by
\begin{equation}
 t_{\rm diff}=N t_{\rm hop}S,
\end{equation}
where $S$ is the total number of sites on the grain surface. 

Desorption is the release of species on the grain surface back into the gas phase. 
There are various desorption mechanisms that directly depend on the binding energy of the species.  
For instance, the thermal evaporation time of species on a grain surface is given by, 
\begin{equation}
t_{\rm evap}=\nu^{-1}{\rm {exp}}\Big(\frac{E_d}{k_B T_{d}}\Big) \ {\rm {s}}.
\end{equation}
Species on the grain surface attain sufficient thermal energy and evaporate from the surface after this characteristic time. 
Therefore, the thermal desorption rate can be written as 
\begin{equation}
K_{\rm evap,A}=\nu {\rm {exp}}\Big(-\frac{E_{\rm bind,A}}{k_BT}\Big),
\end{equation}
where $\nu$ is the characteristics frequency that can be written as,
\begin{equation}
\nu= \sqrt{\frac{2N_sE_{\rm bind,A}}{\pi^2m_{\rm A}}},
\end{equation}
where $N_s$ is the density of the binding site, and $m_{\rm A}$ is the mass of species A. 
The binding energy of the chemical species and the evaporation rate can be estimated from TPD experiments. 
Thermal desorption is a very efficient mechanism in hot cores, hot corinos, and other dense and high-temperature locations of star-forming regions.

\section{Experimental Studies}\label{sec:Experiment}

Experimental studies are essential to measure the rotational or vibrational spectra of molecules with unprecedented resolution and accuracy. They can provide accurate molecular rotational spectra, which are used for their precise identification in data obtained from various radio and infrared telescopes.
%(e.g., ALMA, NOEMA, SMA, VLA, IRAM 30m, GBT 100m, Yebes 40m, Nobeyama 45m, ISO-SWS, IRTF, Spitzer-IRS). 
%jct - I don't think we need to list... where is JWST?
The first carbon-chain molecule, the simplest cyanopolyyne HC$_3$N, was detected with NRAO 140-foot radio telescope in 1971 based on the laboratory rotational spectra measured by \cite{tyler1963}.
%Tyler \& Sheridan (1963).
%jct - add citation properly % PG-Done
After that, the astronomical detection of many carbon-chain species was based on their laboratory rotational spectra. 
References to laboratory experiments of rotational or vibrational spectra of different molecules can be found in papers reporting their interstellar detection.

Here we describe a few studies of carbon-chain species, mainly focusing on molecules containing a benzene ring. The first benzene ring (C$_6$H$_6$) and its cyano derivative, benzonitrile ($c$-C$_6$H$_5$CN) were detected toward CRL 618 and TMC-1, respectively, based on their laboratory infrared and rotational spectra \citep[e.g.,][]{McGuire2018}. 
The most striking and ground-breaking results, the detection of fullerenes (C$_{60}$, C$_{70}$) and its protonated form C$_{60}^+$ was also based on their laboratory data \citep[e.g.,][]{martin93, nemes94,kato91}. 
In the 2020s, several PAHs have been identified in the ISM with the aid of their laboratory spectra. 
For instance, two isomers of cyanonapthalene \citep{mcguire2021,McNaughton2018}, two isomers of ethynyl cyclopentadiene \citep{mccarthy2021,cernicharo2021cyclopentadiene,McCarthy2020}, 2-cyanoindene \citep{sita2022}, fulvenallene \citep{McCarthy2020,sakaizumi1993}, ethynyl cyclopropenylidene, cyclopentadiene, and indene \citep[][and references therein]{cernicharo2021hydrocarboncycle,burkhardt2021}. 

%Not only spectroscopy but 
Laboratory experiments to investigate reactions have also been developed.
Experimental investigations of carbon-bearing molecule formation via neutral-neutral reactions are summarized in \cite{Kaiser2002}. 
Ion-molecule reactions producing carbon-chain species were also investigated via experiments \citep{takagi99,zabka14}.
%PG-done
%jct - reference needed here
Recently, \cite{Martinez2020} investigated the growth of carbon-containing species from C and H$_2$ in conditions analogous of circumstellar envelopes around carbon-rich AGB stars. 
They found that nanometer-sized carbon particles, pure carbon clusters, and aliphatic carbon species are formed efficiently, whereas aromatics are generated at trace levels and no fullerenes are detected.
They also suggested that the formation of aromatic species must occur via other processes, such as the thermal processing of aliphatic material on the surface of dust grains.
%stardust.
%jct - check above

\section{Summary and Open Questions of This Review} \label{sec:7}

\subsection{Summary}

We have reviewed carbon-chain chemistry in the ISM, mainly focusing on recent updates. A summary of our main points is as follows.

\begin{enumerate}
\item By the end of 2022, 118 carbon-chain species have been detected in the ISM. This accounts for almost 43\% of 270 interstellar molecules detected in the ISM or circumstellar shells. These include various families of carbon-chain species, involving elements of O, N, S, P, and Mg. 

\item Two line survey projects toward TMC-1 CP (GOTHAM and QUIJOTE) have recently reported detections of many new carbon-chain species. Abundances of some of these species are not yet reproducible in chemical simulations indicating a need for improved models.

\item In addition to the cold gas conditions of early-phase molecular clouds, carbon-chain formation also occurs around low-, intermediate- and high-mass YSOs. Warm Carbon-Chain Chemistry (WCCC) was found in 2008 around low-mass YSOs, while Hot Carbon-Chain Chemistry (HCCC) has been proposed based on observations around high-mass YSOs. 

\item Recent chemical simulations aim to explain conditions forming hot corino and WCCC sources. There are several possible parameters to produce the chemical differentiation: e.g., UV radiation field and temperature during the static phase.

\item Thanks to high-angular resolution and high-sensitivity observations with ALMA, several carbon-chain species (e.g., CCH, $c$-C$_3$H$_2$, HC$_3$N) have been detected from protoplanetary disks around Herbig Ae and T Tauri stars. Vibrationally-excited lines of HC$_3$N have been found to trace disk structures around massive stars.

\item Circumstellar envelopes around carbon-rich AGB stars and planetary nebulae are unique factories of carbon chemistry. Infrared observations have revealed the presence of PAHs and fullerenes in such environments. Recent laboratory experiments investigated chemistry in such regions.

\item Carbon-chain species have been detected even in extragalactic environments, such as the starburst galaxy NGC253 via the ALCHEMI project. In the Large Magellanic Cloud (LMC), CCH emission has been found to trace outflow cavities, as seen in low-mass YSOs in our Galaxy. 

\item Theoretical and experimental studies are important for the observational detection of carbon-chain species and for obtaining an understanding of their formation and destruction processes. Developments of these techniques are important to reveal carbon-chain chemistry in various physical conditions in the ISM.
\end{enumerate}

The presence of carbon-chain species in the ISM has been known since the early 1970s, and many researchers have investigated their features through observations, chemical simulations, laboratory experiments, and quantum calculations. Recent findings raise new questions about carbon-chain chemistry, and it is an exciting time of progress.
%we live in an exciting era.
To solve the newly raised questions, 
%we need breakthrough ideas and revolutionary approaches, including 
collaborative research involving observations, laboratory experiments, and chemical simulations is crucial.

To understand the carbon-chain chemistry better by observational methods, we need more dedicated low-frequency, high-sensitivity, and high angular resolution observations towards dark clouds, low- and high-mass YSOs, and other environments. 
In the near future, ALMA Band 1, ngVLA and the Square Kilometer Array (SKA) will become available.
%jct - should SKA be mentioned earlier? 
%KT - SKA is mentioned for the first time here
Observations using these facilities will be essential to reveal links between ISM physics and carbon-chain chemistry, the origin of chemical differentiation around YSOs, and relationships between WCCC and HCCC. In addition, future observational studies combining infrared data (e.g., from JWST, Thirty Meter Telescope (TMT), European Extremely Large Telescope (E-ELT)) and radio (ALMA, ngVLA, SKA, and future single-dish) telescopes have the potential from breakthrough results. For instance, relationships between PAHs/small dust grains and common carbon-chain species, which can be observed by infrared and radio regimes, respectively, can be studied by such a combination.


\subsection{Open Key Questions}

Recent new discoveries of carbon-chain molecules in the ISM have raised new questions, and we have realized that our knowledge about carbon-chain chemistry is far from complete.
We highlight the following open questions:

\begin{enumerate}
\item How do large carbon-chain species, which have been found in TMC-1 CP, form? Is there a role for both bottom-up and top-down processes?
%Not only the bottom-up process, does the top-down process work significantly?

\item Which, if any, important formation/destruction processes of carbon-chain species are missing from current chemical models? 

\item Can we estimate more accurate branching ratios for different species (e.g., isomers) in electron recombination reactions?

\item How are PAHs and fullerenes related to other carbon-chain species?

\item How can we obtain accurate binding energy of carbon-chain species?

\end{enumerate}

Answers to these questions likely require combined efforts of observational, theoretical, and experimental study.

%Future work with new observational facilities and theoretical and experimental techniques may be able to shed light on or solve the current questions.



\begin{table*}[h]
\tabcolsep7.5pt
\caption{Binding energy of carbon-chain species}
\label{tab:Ebind}
\begin{center}
\begin{tabular}{cc|cc|cc}
\hline
Species &Binding energy (K) & Species& Binding energy (K) & Species& Binding energy (K) \\
\hline
%HC$_3$N &\\
%HC$_3$N &\\
C$_2$& 10000$^{a}$& C$_8$N&7200$^{a}$&C$_3$O&2750$^{a}$ \\
%C$_{2}$&10000$^{b}$\\
C$_3$&2500$^{a}$&C$_9$N&8000$^{a}$&C$_5$O&4350$^{a}$ \\
%C$_{3}$&2500$^{b}$&&\\
C$_{4}$&3200$^{a}$&C$_{10}$N&8800$^{a}$ & C$_7$O&5950$^{a}$  \\ 
C$_5$&4000$^{a}$&C$_2$H$_2$&2587$^{a}$ &C$_9$O&7550$^{a}$ \\
C$_6$&4800$^{a}$&C$_2$H$_4$&2500$^{a}$ & HC$_2$O&2400$^{a}$ \\
C$_7$&5600$^{a}$&C$_2$H$_5$&3100$^{a}$ &SiC$_2$&4300$^{a}$ \\
%C$_{5}$&4000$^{b}$&&\\
%C$_{6}$&4800$^{b}$&&\\
%C$_{7}$&5600$^{b}$&&\\
C$_{8}$&6400$^{a}$&C$_2$H$_6$&1600$^{a}$ &SiC$_3$&5100$^{a}$ \\
C$_{9}$&7200$^{a}$&C$_4$H$_2$&4187$^{a}$& SiC$_4$&5900$^{a}$ \\
C$_{10}$&8000$^{a}$&C$_5$H$_2$&4987$^{a}$ \\
C$_{11}$&9600$^{a}$&C$_6$H$_2$&5787$^{a}$\\
C$_2$H&3000$^{a}$&C$_7$H$_2$&6587$^{a}$\\
$l$-C$_3$H&4000$^{a}$&C$_2$P&4300$^{a}$ & \\
$c$-C$_3$H&5200$^{a}$&C$_3$P&5900$^{a}$\\
C$_4$H&3737$^{a}$&C$_4$P&7500$^{a}$\\
C$_5$H&4537$^{a}$&C$_2$S&2700$^{a}$\\
C$_6$H&5337$^{a}$&C$_3$S&3500$^{a}$\\
C$_7$H&6137$^{a}$&C$_4$S&4300$^{a}$\\
C$_8$H&6937$^{a}$&HC$_3$N&4580\\
$c$-$\rm{C_3H_2}$&5900$^{a}$&HC$_4$N&5380$^{a}$\\
C$_2$N&2400$^{a}$&HC$_5$N&6180$^{a}$\\
C$_3$N&3200$^{a}$&HC$_6$N&7780$^{a}$\\
C$_4$N&4000$^{a}$&HC$_7$N&7780$^{a}$\\
C$_5$N&4800$^{a}$&HC$_8$N&9380$^{a}$\\
C$_6$N&5600$^{a}$&HC$_9$N&9380$^{a}$\\
C$_7$N&6400$^{a}$&C$_2$O&1950$^{a}$\\

%HC$_3$O&3111$^{b}$

%C$_7$N&6400$^{a}$&H$_2$C$_3$N&3133$^{b}$\\
%C$_8$N&7200$^{b}$&\\
%C$_9$N&8000$^{b}$&\\
%C$_2$O&1950$^{b}$&&\\
%C$_3$O&4208$^{a}$&&\\
%C$_5$O&4350$^{b}$&&\\
%C$_7$O&5950$^{b}$&&\\
%C$_9$O&7550$^{b}$&&\\
% &&C$_2$S&2943$^{a}$\\
% &&C$_3$S&3500$^{b}$\\
% &&C$_4$S&4300$^{b}$\\
% &&HC$_3$N&3475$^{a}$\\
% &&HC$_4$N&5380$^{b}$\\
% &&HC$_5$N&6180$^{b}$\\
% &&HC$_6$N&7780$^{b}$\\
% &&HC$_7$N&7780(it should change)$^{b}$\\
% &&HC$_8$N&9380$^{b}$\\
% &&HC$_9$N&9380(it should change)$^{b}$\\
% &&HC$_2$O&2400$^{b}$\\
% &&HC$_3$O&3111$^{a}$\\
% &&SiC$_2$&4300$^{b}$\\
% &&SiC$_3$&5100$^{b}$\\
% &&SiC$_4$&5900$^{b}$\\
% %&&\\
\hline
\end{tabular}
\end{center}
\begin{tabnote}
References: $^{a}$KIDA (https://kida.astrochem-tools.org/), also see $^{\rm b}$\cite{wake17}, $^{\rm c}$\cite{pent17}, $^{\rm d}$\cite{das18}
% higher-order-chain values are estimated with lower order chain plus one carbon atom's BE (KIDA has used 800 K for these estimations since updated binding energy of C atom is 1300 K, thus need to update those values. HC$_{2n+1}$ show different binding energy, however, we think the BE of HC$_4$N, HC$_6$N, and HC$_8$N needs to revisit. Especially for C$_n$P group, C$_n$+P is the rule for the estimation of binding energy. Also, include BE values of HC$_n$O ($n>4$) following a similar carbon addition method. Check some estimation if possible for recently detected higher-order carbon chains.
\end{tabnote}
\end{table*}


% Acknowledgements
\begin{ack}
K.T. is grateful to Professor Eric Herbst (University of Virginia) for leading me to the astrochemical field, working with me, and giving a lot of comments on studies of carbon-chain molecules that are presented in this article. 
K.T. appreciates Professor Masao Saito (National Astronomical Observatory of Japan) for giving his advice and continuous encouragement.
K.T. is supported by JSPS KAKENHI grant No.JP20K14523.
P. G acknowledges the support from the Chalmers Initiative of Cosmic Origins Postdoctoral Fellowship.
We would like to thank Professor Fumitaka Nakamura (National Astronomical Observatory of Japan) and Professor Kazuhito Dobashi (Tokyo Gakugei University) for providing original data of mapping observations of carbon-chain species toward TMC-1 obtained by the Nobeyama 45m radio telescope. We would also like to thank Dr. Emmanuel E. Etim for his comments and suggestions. J.C.T. acknowledges support from ERC Advanced Grant MSTAR.
\end{ack}

%\clearpage

% References


\bibliography{references}
%\bibliographystyle{ar-style2}
\bibliographystyle{aa}


\end{document}

