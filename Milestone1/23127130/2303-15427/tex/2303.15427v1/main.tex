% CVPR 2023 Paper Template
% based on the CVPR template provided by Ming-Ming Cheng (https://github.com/MCG-NKU/CVPR_Template)
% modified and extended by Stefan Roth (stefan.roth@NOSPAMtu-darmstadt.de)

\documentclass[10pt,twocolumn,letterpaper]{article}

%%%%%%%%% PAPER TYPE  - PLEASE UPDATE FOR FINAL VERSION
% \usepackage[review]{cvpr}      % To produce the REVIEW version
% \usepackage{cvpr}              % To produce the CAMERA-READY version
\usepackage[pagenumbers]{cvpr} % To force page numbers, e.g. for an arXiv version

% Include other packages here, before hyperref.
\usepackage{graphicx}
\usepackage{amsmath}
\usepackage{amssymb}
\usepackage{booktabs}
\usepackage{xcolor,colortbl}
\usepackage{multirow}


\usepackage{hhline}
\usepackage{array}
\usepackage{graphicx}
\usepackage{multirow}
\usepackage{ragged2e}
\usepackage{booktabs}
\usepackage{hhline}


% It is strongly recommended to use hyperref, especially for the review version.
% hyperref with option pagebackref eases the reviewers' job.
% Please disable hyperref *only* if you encounter grave issues, e.g. with the
% file validation for the camera-ready version.
%
% If you comment hyperref and then uncomment it, you should delete
% ReviewTempalte.aux before re-running LaTeX.
% (Or just hit 'q' on the first LaTeX run, let it finish, and you
%  should be clear).
\usepackage[pagebackref,breaklinks,colorlinks]{hyperref}
\usepackage{multirow}
\usepackage{siunitx} % Required : alignement des valeurs etc.
\sisetup{
    round-mode          = places, % Rounds numbers
    round-precision     = 2, % to 2 places
}
\newcommand{\maxf}[1]{{\cellcolor[gray]{0.8}} #1}
\usepackage{numprint} 
% Support for easy cross-referencing
\usepackage[capitalize]{cleveref}
\crefname{section}{Sec.}{Secs.}
\Crefname{section}{Section}{Sections}
\Crefname{table}{Table}{Tables}
\crefname{table}{Tab.}{Tabs.}


%%%%%%%%% PAPER ID  - PLEASE UPDATE
\def\cvprPaperID{1751} % *** Enter the CVPR Paper ID here
\def\confName{CVPR}
\def\confYear{2023}


\begin{document}

%%%%%%%%% TITLE - PLEASE UPDATE
\title{JAWS: Just A Wild Shot for Cinematic Transfer in Neural Radiance Fields}
% for Cinematic Motion Transfer}
\author{Xi Wang$^1$\thanks{Equal contribution. Corresponding to jinglei.shi@nankai.edu.cn.}, Robin Courant$^{1,2}$\footnotemark[1], Jinglei Shi$^{3}$, Eric Marchand$^{1}$, Marc Christie$^{1}$\\
$^{1}$Inria, IRISA, CNRS, Univ. Rennes, $^{2}$LIX, Ecole Polytechnique, IP Paris,
$^{3}$VCIP, CS, Nankai Univ.\\
{\tt\small xi.wang@inria.fr, robin.courant@polytechnique.edu, jinglei.shi@nankai.edu.cn}\\
{\tt\small \{eric.marchand, marc.christie\}@irisa.fr}
% For a paper whose authors are all at the same institution,
% omit the following lines up until the closing ``}''.
% Additional authors and addresses can be added with ``\and'',
% just like the second author.
% To save space, use either the email address or home page, not both
% \and
% Robin Courant\\
% Inria, IRISA, CNRS, Univ Rennes\\
% First line of institution2 address\\
% {\tt\small robin.courant@inria.fr}
% \and
% Jinglei Shi\\
% Inria, IRISA, CNRS, Univ Rennes\\
% First line of institution2 address\\
% {\tt\small TBA}
% \and
% Eric Marchand\\
% Inria, IRISA, CNRS, Univ Rennes\\
% First line of institution2 address\\
% {\tt\small eric.marchand@irisa.fr}
% \and
% Marc Christie\\
% Inria, IRISA, CNRS, Univ Rennes\\
% First line of institution2 address\\
% {\tt\small marc.christie@irisa.fr}
}

% \twocolumn[{%
% \renewcommand\twocolumn[1][]{#1}%
% \maketitle
% \begin{center}
%     \centering
%     \captionsetup{type=figure}
%     \includegraphics[width=.95\textwidth]{fig/teaser_placeholding.png}
%     \captionof{figure}{In this paper, we first propose a robust technique to extract cinematic characteristics from shots in the wild, by jointly exploiting optical flow and character poses to estimate a camera trajectory, even in the presence of motion blur or focus , and then design an optimization technique to transfer these cinematic characteristics to implicit representations (\eg NeRF-based) or more traditional mesh-based representations. As a result, our system has the capacity to tthe cinematic features from a given video can easily be re-applied to any new scene.}
% \end{center}%
% }]
\maketitle

%%%%%%%%% ABSTRACT
Answering first-order logical (FOL) queries over knowledge graphs (KG) remains a challenging task mainly due to KG incompleteness. 
Query embedding approaches this problem by computing the low-dimensional vector representations of entities, relations, and logical queries. 
KGs exhibit relational patterns such as symmetry and composition and modeling the patterns can further enhance the performance of query embedding models.
However, the role of such patterns in answering FOL queries by query embedding models has not been yet studied in the literature.
In this paper, we fill in this research gap and empower FOL queries reasoning with pattern inference by introducing an inductive bias that allows for learning relation patterns. 
To this end, we develop a novel query embedding method, RoConE, that defines query regions as geometric cones and algebraic query operators by rotations in complex space. RoConE combines the advantages of Cone as a well-specified geometric representation for query embedding, and also the rotation operator as a powerful algebraic operation for pattern inference. 
%Therefore, RoConE enables inferring patterns during the multi-hop reasoning process.
Our experimental results on several benchmark datasets confirm the advantage of relational patterns for enhancing logical query answering task.

%%%%%%%%% BODY TEXT
\section{Introduction}
\label{sec:introduction}
\section{Introduction}
Deep learning~\cite{dl} has been highly successful in computer vision~\cite{sg1,od1,app-detection,zhou2024diffdet4sar,li2024predicting,yang2024saratr,LiSARATRX25}, largely due to the availability of large-scale labeled datasets. However, in many practical scenarios, obtaining such large amounts of labeled data is difficult or costly. To address this challenge, Few-shot learning (FSL) aims to enable models to learn new tasks with only a limited number of labeled samples. Consequently, this problem has garnered significant attention in both academia and industry due to its broad real-world applications. While humans can easily distinguish between objects after seeing only a few examples, machines struggle to achieve similar efficiency. In domains such as natural scene images, large datasets are readily available, but FSL is crucial in scenarios where collecting large amounts of data is difficult. Since the problem was first introduced in 2006~\cite{fsl-1}, numerous methods have been proposed to tackle the challenges of FSL~\cite{fslsurvey,fslsurvey22,fslsurvey20,fsl18,fslsurvey1}.

With the development of FSL, challenges such as limited training data, domain variations, and task modifications have led to the emergence of various FSL variants, including semi-supervised FSL~\cite{semifsl}, unsupervised FSL~\cite{ufsl1,ufsl2}, zero-shot learning (ZSL)\cite{zsl1}, and cross-domain FSL (CDFSL)~\cite{feature-wise,bscd-fsl}, among others. These variants represent distinctive cases of FSL in terms of sample availability and domain learning. This paper focuses specifically on CDFSL variants. The traditional FSL problem assumes that both prior knowledge and target tasks come from the same domain, which is often restrictive in real-world applications. CDFSL addresses this issue by overcoming the domain gap between auxiliary data (which provides prior knowledge) and the target data in FSL tasks, as show in Figure~\ref{int}. For instance, in art image recognition tasks involving scribble, cartoon, or sketch images, FSL could theoretically leverage prior knowledge from related domains like cartoons and sketches. However, such data is often scarce due to copyright restrictions and the high cost of collection. As a result, researchers have turned to data-rich domains, such as natural scene images, to address the challenges of few-shot image recognition in the field of art.
However, the significant domain gap between these domains often leads to performance degradation in FSL. CDFSL faces challenges from both transfer learning and FSL, including domain gaps, class shifts, and the scarcity of labeled samples in the target domain, making it a more complex task. Since its formal introduction in 2020~\cite{feature-wise}, CDFSL has garnered widespread attention, with numerous methods published in top venues~\cite{bscd-fsl,st,dynamic,hybrid_1,feature_reweight_6}. Figure~\ref{imaging} presents the milestones of CDFSL technologies from 2019 to the present, showcasing representative CDFSL methods and related benchmarks.
\begin{figure}%[b]
	\centering
  \vspace{-0.3cm}
 	\includegraphics[width=0.9\linewidth]{CDFSLProblem-10.pdf}
  \vspace{-0.3cm}
	\caption{\textcolor{black}{The difference of few-shot learning and cross-domain few-shot learning.}}
 \vspace{-0.4cm}
	\label{int}
\end{figure}


So far, several surveys have provided comprehensive overviews and future directions for FSL~\cite{fsl18,fslsurvey,fslsurvey1,fslsurvey20,fslsurvey22}. For example,\cite{fsl18} categorizes FSL into experiential and conceptual learning, while\cite{fslsurvey} focuses on empirical risk minimization and defines FSL by experience, task, and performance, introducing CDFSL as a branch of FSL. Both~\cite{fslsurvey1} and~\cite{fslsurvey20} highlight CDFSL as a variant of FSL, discussing meta-learning approaches and benchmarks. Lastly,~\cite{fslsurvey22} offers a taxonomy based on prior knowledge and emphasizes that current methods have yet to fully tackle cross-domain challenges. Collectively, these works point to cross-domain learning as a promising area for future FSL research. Currently, there are two elementary surveys on CDFSL~\cite{wang2023survey,deng2023survey}. \cite{wang2023survey} classifies methods into benchmark, single source, and multiple source categories, while~\cite{deng2023survey} categorizes algorithms into data augmentation and feature alignment paradigms. In contrast, to stimulate future research and help newcomers better understand this challenging problem, this paper offers the first classification grounded in theoretical analysis and provides a comprehensive review, offering deeper insights into the core principles of CDFSL. Firstly, this paper compiles and analyzes a broad range of literature on the topic. An analysis of the reference index reveals that even before the formal introduction of CDFSL, some works had already tried to solve cross-domain issues within the FSL framework~\cite{clc, rffl}. Following its formal introduction as a branch of FSL, CDFSL has garnered significant attention. Additionally, we define CDFSL using both machine learning theory~\cite{ml,erm1} and transfer learning principles~\cite{tltheory}. Secondly, our analysis highlights that the unique challenge in CDFSL lies in the unreliable nature of two-stage empirical risk minimization. The details are discussed in Section~\ref{background}. To address these challenges, the paper organizes CDFSL research into four categories: $\mathcal{D}$-Extension, $\mathcal{H}$-Constraint, $\Delta$-Adaptation, and hybrid approaches. We also compile relevant datasets and benchmarks to evaluate the methods, and analyze their performance, as discussed in Sections~\ref{methods} and~\ref{performance}. Finally, we explore future research directions for CDFSL by considering three perspectives, including problem set-ups, applications, and theories, which provide a comprehensive understanding of the field and its potential for future growth. Contributions of this survey can be summarized as follows:

\begin{itemize}
    \item We analyzed existing CDFSL papers and provided a comprehensive survey. We also defined CDFSL formally, connecting it to classic ML~\cite{ml,erm1} and transfer learning theory~\cite{tltheory}. This helps guide future research in the field.
    \item We listed relevant learning problems for CDFSL with examples, clarifying their relation and differences. This helps position CDFSL among various learning problems. We also analyzed unique issues and challenges of CDFSL, helping to explore a scientific taxonomy for CDFSL work.
    \item We conducted an extensive literature review, organizing it into a unified taxonomy based on $\mathcal{D}$-Extension, $\mathcal{H}$-Constraint, $\Delta$-Adaptation, and hybrid approaches. We introduced applicable scenarios for each taxonomy to help discuss its pros and cons. We also presented datasets and benchmarks for CDFSL, summarizing insights from performance results to improve understanding of CDFSL methods.
    \item We proposed promising future directions for CDFSL in problem set-ups, applications, and theories, based on current weaknesses and potential improvements.
\end{itemize}

\begin{figure}
	\centering
  \vspace{-0.3cm}
        \includegraphics[width=\linewidth]{response/crop_fig2.pdf}
 \vspace{-0.5cm}
	\caption{Chronological milestones of CDFSL from 2019 to the present, including representative CDFSL approaches and related benchmarks. Key events include the release of Meta-Dataset~\cite{meta-dataset} and BSCD-FSL~\cite{bscd-fsl} in 2020, the introduction of pioneering works such as~\cite{feature-wise}, and subsequent contributions like~\cite{feature_reweight_1,lscdfsl}. Later works~\cite{st,dynamic,hybrid_1,hybrid_4,hybrid_2} explored new setups, while~\cite{boosting,ata,data_target_1,feature_reweight_5,parameter_weight_2,confess,feature_reweight_9} focused on improving performance. Please see Section~\ref{methods} for details.}
 \vspace{-0.3cm}
	\label{imaging}
\end{figure}

The remainder of this survey is organized as follows: Section \ref{background} provides an overview of CDFSL, including its definition, challenges, and taxonomy. Section \ref{methods} covers approaches to CDFSL in detail, while Section \ref{performance} presents performance results and evaluates methods. Section \ref{future} explores future directions in set-ups, applications, and theories. Finally, Section \ref{conclusion} concludes the survey.

\section{Related work}
\label{sec:related_work}
\section{Related Work}

\subsection{Intent Classification}
Traditionally, benchmark datasets \cite{price1990evaluation, coucke2018snips, eric2019multiwoz} for intent identification have sufficient labeled datasets for training, and the task has been solved through the classification method. For example, \citet{goo2018slot} classified the intent and slot information using the attention mechanism, and \citet{kim2017onenet} enriched word embeddings by using semantic lexicons and adapted this strategy to the intent classification. In addition, \citet{wang2015mining} grouped the intent of tweets into six categories, used a graph embedding consisting of tweet nodes, and classified their intents. However, labeling the intent for the raw dialogue dataset requires extensive human labor, so building a new labeled intent dataset in the real world is challenging. Therefore, a robust intent induction model that can be applied to a new domain as an unsupervised method is required.


\subsection{Intent induction with unsupervised method}
The representative method of unsupervised intent induction utilizes the clustering method. \citet{liu2021open} is one example of research that enhanced the clustering algorithm. They proposed a balanced score metric to obtain similar-sized clusters in K-means clustering and found proper K-values that were more stable than naive K-means. \citet{chatterjee2020intent} also enhanced the clustering algorithm by utilizing the outlier information of the density-based clustering model, which is called ITER-DBSCAN. Their work shows greater accuracy on imbalanced intent data. 
On the other hand, there has been research that improved the dialogue representations for better clustering results. For example, \citet{perkins2019dialog} iteratively enhanced the dialogue embedding by reflecting the clustering score, and \citet{lin2019deep} proposed a BiLSTM \cite{mesnil2014using} embedding model with margin loss that is effective in detecting unknown intents. However, robust intent induction in diverse domains was not examined in previous research. Therefore, as a strategy for enhancing the embedding dialogue model, we propose the DORIC method, which robustly embeds diverse dialogue domains.

\section{Preliminary knowledge}
\label{sec:preliminarly}
\noindent \textbf{Neural rendering.} Our work strongly relies on NeRF representations of 3D scenes~\cite{mildenhall2020nerf} which describe a scene through a Multiple Layer Perceptron (MLP) network by inputting different 3D locations and view directions, and inferring color and volume density. Volumetric rendering techniques then enable the reconstruction of an image from any viewpoint by retrieving the color and density of each pixel and integrating on the line direction of its pixel-correspondent rays in the MLP. A NeRF model $\mathcal{N}_\mathbf{\Theta}$ can therefore be viewed as a mapping between a camera pose $\mathbf{T}$ and a reconstructed image $\mathbf{\hat{I}}$, where $\hat{\mathbf{I}}=\mathcal{N}(\mathbf{T})$ by abusing the term and regrouping all the rays into image pixels through a camera projective model: $\mathcal{N}_{\mathbf{\Theta}}: SE(3) \rightarrow \mathbb{R}^{H \times W \times 3} ; \mathbf{T}  \mapsto  \mathcal{N}(\mathbf{T})$.

In the training, parameters $\mathbf{\Theta}$ are updated by minimizing an on-screen image loss (eg. photometric loss) on each ray, which compares pixel-level information of the reconstructed image $\mathcal{N}(\mathbf{T})$ and its associated reference view $\mathbf{I}^*$: 
\begin{equation}
    \hat{\mathbf{\Theta}} = \arg \min_{\mathbf{\Theta}} \mathcal{L}(\mathbf{\Theta} \mid \mathcal{N}(\mathbf{T}), \mathbf{I}^*)
\label{eq: trainNeRF}
\end{equation}
Training a NeRF network can therefore be seen as optimizing MLP parameters  $\mathbf{\Theta}$ from a set of known camera poses and corresponding images. 

\noindent \textbf{Camera pose inference.}
The NeRF training process can also be applied in an inverse manner (see iNeRF~\cite{yen2021inerf}), to estimate a camera pose $\hat{\mathbf{T}} \in SE(3)$ in an already trained NeRF model, against a reference view $\mathbf{I}^{*}$. 
To ensure the estimated pose still lies in the $SE(3)$ manifold, the optimization process is performed in the canonical exponential coordinate system~\cite{ma20043dvision} based on an initial camera pose $\mathbf{T}(0)$:
\begin{equation}
\begin{array}{c}
\hat{\mathbf{T}}(\theta) = e^{[{\xi}]\theta} \ \mathbf{T}(0)\quad
\text{where} \quad e^{[{\xi}]\theta} = \begin{bmatrix}
e^{[{\omega}]\theta} & K(\theta, \omega, v) \\
0 & 1 
\end{bmatrix} \\ \\
\text{with}   \quad
 K(\theta, \omega, v) = \left\{ \begin{array}{cl}
         v\theta & \text{if} \quad \omega = 0 \\
        \frac{(I - e^{[{\omega}]\theta})[{\omega}] v+\omega\omega^Tv\theta}{\lVert \omega \rVert} & \text{otherwise}
    \end{array} \right.
\end{array}
\end{equation}

\noindent where $[{\xi}]$ is a twist matrix, $(\omega, v)$ are twist coordinates, and $\theta$ is a magnitude. The optimization problem is then changed to seek for the optimal camera parameters $(\hat{\theta}_k, \hat{\omega}_k, \hat{v}_k)$. 

\section{Method}
\label{sec:method}
\section{Proposed Modification}
\label{sec:method}

In the previous section, we provided a theoretical analysis of the conditions under which computed ``probe'' functions within the deep functional map pipeline can be used as pointwise descriptors directly, and lead to the same point-to-point maps as computed by the functional maps. In \cref{sec:application}, we provide an extensive evaluation of using learned feature functions for pointwise map computation and thus affirm the validity of \cref{thm:equivalence} in practice.

% \maks{is this paragraph necessary?} 
Our main observation is that the two approaches for point-to-point map computation are indeed often equivalent in practice, especially in ``easy'' cases, where existing state-of-the-art approaches lead to highly accurate maps. In contrast, we found that in more challenging cases, where existing methods fail, the two approaches are not equivalent and can lead to significantly different results. 

% Inspired by our analysis above, we thus propose to use the structural properties suggested in \cref{thm:equivalence} as a way to bridge this gap and, possibly improve the overall accuracy. As we demonstrate in \cref{sec:application}, our proposed modifications while being relatively simple, significantly improve the quality of the computed correspondences, especially in ``difficult'' matching scenarios.

Motivated by our analysis, we propose to use the structural properties suggested in \cref{thm:equivalence} as a way to bridge this gap and improve the overall accuracy. Our proposed modifications are relatively simple, but they significantly improve the quality of computed correspondences, especially in ``difficult'' matching scenarios, as we demonstrate in \cref{sec:application}.

% The two key assumptions in \cref{thm:equivalence} are \textit{basis-aligning} functional maps and \textit{complete} feature extractors. We thus propose to modify the functional map pipeline so that the conditions of the theorem are satisfied. As mentioned above, the basis-aligning property is closely related to \textit{properness} and thus we propose to approach it by imposing that the predicted functional map arises from some pointwise correspondence. For feature completeness, we propose a simple modification of the feature extractor so that it produces \textit{smooth features}. In what follows, we will use the same notation as \cref{sec:notation}.

The two key assumptions in \cref{thm:equivalence} are \textit{basis-aligning} functional maps and \textit{complete} feature extractors. We propose modifying the functional map pipeline to satisfy the conditions of the theorem. Since the basis-aligning property is closely related to \textit{properness}, we propose to impose that the predicted functional map to be proper, \ie arises from some pointwise correspondence. For feature completeness, we suggest modifying the feature extractor to produce \textit{smooth features}. We use the same notation as in \cref{sec:notation}.


\subsection{Enforcing Properness}
\label{sec:proper_fmap}

In this section, we propose two ways to enforce functional map properness and associated losses  for both supervised and unsupervised training.

\paragraph{The adjoint method}
Given feature functions $F_1, F_2$, produced by a feature extractor, we compute the functional map $\C_{12-pred}$ as explained in \cref{sec:background}. To compute a proper functional from it, we first convert $\C_{12-pred}$ into a p2p map $\Pi_{21-pred}$ in a differentiable way and then compute the ``differentiable'' proper functional map $\C_{21-proper} = \Phi_{2}^{\dagger} \Pi_{21-pred} \Phi_{1}$. 

To compute $\Pi_{21-pred}$, denoting $G_1 = \Phi_{1}$ and $G_2 = \Phi_{2}\C_{21-pred}$, we use:
\begin{align}
& \Pi_{21-pred}^{i, j} = \dfrac{\exp\big(\langle G_2^{i}, G_1^{j}\rangle / \tau\big)}{\sum_{k=1}^{n_1}\exp\big(\langle G_2^{i}, G_1^{k} \rangle / \tau\big)}.\label{eq:diff_p2p}%\\
%&s(\mathbf{x}, \mathbf{y}) = \mathbf{x} \cdot \mathbf{y}.\label{eq:FeatureDistanceFunc}
\end{align}
%
Here $\langle \cdot,\cdot \rangle$ is the scalar product measuring the similarity between $G_1$ and $G_2$, and $\tau$ is a temperature hyper-parameter. $\Pi_{21-pred}$ can be seen as a soft point-to-point map, formulated based on the adjoint conversion method described in \cite{Pai_2021_CVPR}, and computed in a differentiable manner, hence it can be used inside a neural network.

\paragraph{The feature-based method}
The feature-based method is similar to the adjoint method in spirit, the only difference being that $\Pi_{21-pred}$ is computed using the predicted features instead of the fmap. For this, we use \cref{eq:diff_p2p}, with $G_1 = F_1$ and $G_2 = F_2$. The modified deep functional map pipeline is illustrated in \cref{fig:fmap-pipeline}.

\begin{figure}
    \centering
    \includegraphics[width=\columnwidth]{figures/fmap_pipeline.pdf}
    \caption{An overview of our revised deep functional map pipeline. The extracted features are used to compute the functional map and the proper functional map, as explained in \cref{sec:proper_fmap}}
    \label{fig:fmap-pipeline}
    \vspace{-1em}
\end{figure}

In addition to $C_{12-pred}$, the two previous methods allow to calculate $C_{21-proper}$. We adapt the functional map losses to take into account this modification.

In the supervised case, we modify the supervised loss (see \cref{eq:sup_loss}) by simply introducing an additional term into the loss: 
\begin{align}
\mathcal{L}_{proper} = \| \C_{12-pred} - \C_{12-proper} \|_F^2 \label{eq:sup_loss_proper}.
\end{align}

The motivation behind this loss is that we want the predicted functional map to be as close as possible to the ground truth and stay within the space of proper functional maps.% a proper one , the gradient is strong enough to force the features to produce a  using \cref{eq:fmap_basic}, that is close to a proper one.

In the unsupervised setting, we simply impose the standard unsupervised losses on the differentiable proper functional map $\C_{12-proper}$ instead of $\C_{12-pred}$. Specifically, in our experiments below, we use the following unsupervised losses: 
%
\begin{align}
\nonumber
\mathcal{L}_{unsup}(\C_{12}, \C_{21}) &= \| \C_{12} \C_{21} - \mathbb{I} \|_F^2 + \| \C_{21} \C_{12} - \mathbb{I} \|_F^2 \\
& + \| \C_{12}^{\top} \C_{12} - \mathbb{I} \|_F^2 + \| \C_{21}^{\top} \C_{21} - \mathbb{I} \|_F^2
\label{eq:unsup_loss_proper}
\end{align}

%\maks{add the losses here.}


% \souhaib{how to justify that the results obtained with NN are better than fmap}












%%%%%%%%%%%%%%%%%%%%%%%%%%%%%%%%%%%%%%%%%%%%%%%%%%%%
%%%%%%%%%%%%%%%%%%%%%%%%%%%%%%%%%%%%%%%%%%%%%%%%%%%%%%%%%%%%%%%%%%%%%%%%%%%%%%%%%%%%%%%%%%%%%%%%%%%%%%%%
%%%%%%%%%%%%%%%%%%%%%%%%%%%%%%%%%%%%%%%%%%%%%%%%%%%%%%%%%%%%%%%%%%%%%%%%%%%%%%%%%%%%%%%%%%%%%%%%%%%%%%%%
%%%%%%%%%%%%%%%%%%%%%%%%%%%%%%%%%%%%%%%%%%%%%%%%%%%%
%%%%%%%%%%%%%%%%%%%%%%%%%%%%%%%%%%%%%%%%%%%%%%%%%%%%

\subsection{As Smooth As Possible Feature Extractor}
Another fundamental assumption of \cref{thm:equivalence} is the completeness of the features produced by a neural network. 

We have experimented with several ways to impose it and have found that it is not easy to satisfy it exactly in general because it would require the network to always produce features in some target  subspace, which is not explicitly specified in advance. Moreover, we have found that explicitly projecting feature functions to a small reduced subspace can also hinder learning. 

To circumvent this, we propose instead to \textit{encourage} this property by promoting the feature extractor to produce smooth features. 

The motivation for this is as follows. If $F_i$ is complete, then there exist coefficients $a_1 ... a_k$ such that $F_i = \sum_{j=1}^k a_j \Phi_i^j$, where $k$ is the size of the functional map used in \cref{eq:fmap_basic}.
However, it's known that Fourier coefficients for smooth functions decay rapidly (faster than any polynomial, if $f$ is of class $C^l$, the coefficients are $o(n^{-l})$), which means that the smoother the function is, the closer it will be to being complete for some index $k$.

Inspired by this, we propose the following simple modification to feature extractors used for deep functional maps. Since feature extractors are made of multiple layers, we propose to project the output of each layer into the Laplacian basis, diffuse it over the surface following \cite{sharp2021diffusion}, and then project it back to the ambient space before feeding it to the next layer, see \cref{fig:feat-extract-modif}. Concretely, for shape $S$, if $f_i$ is the output of layer $i$, we feed to layer $i+1$ the function $f^{'}_i$, such that $f^{'}_i = \Phi_j e^{-t \Delta} \Phi_j^{\dagger} f_i$, where $\Phi_j$ denotes the first $j$ eigenfunctions of the Laplacian, $\Delta$ is a diagonal matrix containing the first j eigenvalues, and $t$ is a learnable parameter. Please note there is no need to do this operation for the final layer, since the features will be projected into the Laplacian basis anyway, for computing the functional map. In practice, we observed that it is beneficial to set $j$ to \textit{be larger} than the size of the functional maps in \cref{eq:fmap_basic}. This allows the network to impose smoothness, while still allowing degrees of freedom to enable optimization.

%Also note, that DiffusionNet \cite{sharp2021diffusion} does this operation by construction for each layer, which can in part explain its success.
%
%\souhaib{what about the receiptive field}


%
%
% - smooth the input before feeding them to the network (doesn't work practically)
% 
% - smooth the features at the end of each layer
% 
% - for better results, increase the receiptive field of the features using diffusion 
%
%
\vspace{-1em}

\paragraph{Implementation details} we provide implementation details, for all our experiments, in the supplementary. Our code and data will be released after publication.

\begin{figure}
    \centering
    \includegraphics[width=\columnwidth]{figures/feature_extractor.pdf}
    \caption{An overview of the feature extractor modification is shown here. The features are made smooth by projecting them into the Laplacian basis at the end of each layer.}
    \label{fig:feat-extract-modif}
    % \vspace{-1.5em}
\end{figure}

\section{Experiments}
\label{sec:experiments}
\section{Experiments}
\label{sec:experiments}

In this section, we provide extensive experiments to highlight the generalization power of our local feature pre-training and receptive field size optimization methods in a suite of downstream shape analysis tasks especially involving highly deformable, organic shapes.
We consider diverse benchmarks including human and partial animal shape matching as well as molecular surface segmentation.
We reuse the same pre-trained local feature extractor $\mathcal{F}_{s, \Theta}$ and then perform our receptive field size optimization once individually on each deformable shape dataset (\cref{subsec:local_feature_transfer}).

\mypara{Implementation.}
We denote our features as \OurMethodName{} for Voxelized Alignment-based DEscriptoR.
%We implemented our \OurMethodName{} pre-training and transferring pipeline with PyTorch \cite{NEURIPS2019_9015}.
In the pre-training stage, we use the 3DMatch dataset \cite{zeng20173dmatch} and train the local feature extractor $\mathcal{F}_{s, \Theta}$ for 16K steps.
We use the Adam optimizer with a learning rate of $10^{-3}$  for network weight update.
For receptive field size optimization, we use the same learning rate for Adam, and we take $n_s = 10^4$  extracted patches from the pre-training dataset. 
%Our code and data will be released after publication.

\mypara{Baselines.}
We compare a wide spectrum of hand-crafted and pre-trained features in deformable shape tasks.
For the hand-crafted features, we consider the Heat Kernel Signature (HKS)  \cite{sun2009concise}, Wave Kernel Signature (WKS) \cite{aubry2011wave}, and SHOT descriptors \cite{tombari2010unique}, as well as the straightforward vertex positions (XYZ).
For the pre-trained features, we use the PointContrast features learned with the PointInfoNCE loss (PCN) or a hardest-contrastive loss (PCH) \cite{xie2020pointcontrast}.

%\rev{TODO: comparison of Vader with NCE vs cycle-consistency loss in some downstream task?}

\subsection{Human Shape Matching}
\label{subsec:human_matching}

%\setlength{\tabcolsep}{4pt}
\begin{table}[!t]
    \begin{center}
    \ra{1.0}
        \resizebox{0.81\columnwidth}{!}{%
            \begin{tabular}{@{} lrr @{} }
                \toprule
                \textbf{Method / Dataset}                & \textbf{FR}-\textbf{SH} & \textbf{SR}-\textbf{SH} \\
                \midrule
                SURFMNET \cite{roufosse2019unsupervised} & 30.1                    & 28.6                    \\
                Cyclic FMaps \cite{ginzburg2019cyclic}   & 36.5                    & 38.6                    \\
                WSupFMNet \cite{sharma2020weakly}        & 26.3                    & 30.2                    \\
                Deep Shells \cite{eisenberger2020deep}   & 26.3                    & 22.8                    \\
                \midrule
                % \addlinespace
                DiffusionNet - XYZ                       & 22.4                    & 23.3                    \\ % 8.1 & 23.1 &
                DiffusionNet - HKS                       & 10.4                    & 15.4                    \\ % 7.1 & 15.4 &
                DiffusionNet - WKS                       & 9.3                     & 24.0                    \\ % 3.8 & 4.4 &
                DiffusionNet - SHOT                      & 10.8                    & 21.5                    \\ % 3.8 & \textbf{4.2} &
                DiffusionNet - PCH                       & 25.8                    & 33.2                    \\ % 6.3 & 4.3 &
                DiffusionNet - PCN                       & 22.6                    & 36.2                    \\ % 8.7 & 4.8 &
                DiffusionNet - \OurMethodName{} (ours)   & \textbf{6.4}            & \textbf{6.9}            \\ % % 8.7 & 4.8 &
                %DiffusionNet - \OurMethodName{} (ours)  & \textbf{8.6}            & \textbf{8.8}            \\ % \textbf{3.8} & 4.4 &
                %\hdashline
                %\quad + ZoomOut & -- & -- & -- & -- \\
                \bottomrule
            \end{tabular}
        }
        \caption{Performance of various features for unsupervised deformable shape matching on un-aligned data. X-Y means training on X and testing on Y. Values are mean geodesic error $\times 100$ on unit-area shapes.}
        \label{tab:unaligned_unsup}
    \end{center}
    \vspace{\figmargin}
\end{table}
%\setlength{\tabcolsep}{1.4pt}


\mypara{Unsupervised matching.}
We perform unsupervised shape matching \cite{roufosse2019unsupervised,eisenberger2020deep} on the FAUST-Remeshed (FR), SCAPE-Remeshed (SR), and SHREC'19 datasets (SH) \cite{Ren2019,Bogo2014,Anguelov2005,shrec19}, consisting of 100, 71, and 44 \emph{unaligned} human shapes in different poses, respectively. 
The same train/test splits in prior works \cite{sharma2020weakly,eisenberger2020deep} are adopted.
We feed the above baseline features and our \OurMethodName{} respectively as input to a surface learning backbone DiffusionNet \cite{sharp2020diffusionnet}, which produces a functional map \cite{ovsjanikov2012functional,litany2017deep} as output for a given pair of shapes.
We leverage the unsupervised functional map losses \cite{sharma2020weakly} for training the backbone.
The evaluation metric is the mean geodesic error of predicted maps with respect to the ground truth on unit-area shapes \cite{Kim2011}. We use X-Y to denote training on dataset X and testing on dataset Y.

%\maks{we don't mention what  FR-SH stands for. Also, we should highlight that this is a very difficult problem and existing unsuprevised methods have only been used on aligned data. this explains why most of the numbers in Table 1 are so bad.}

As shown in \cref{tab:unaligned_unsup}, our approach yields the best results for unsupervised shape correspondence in both FR-SH and SR-SH settings, while the other tested features fail to achieve reasonable matching performance.
We stress that this is a challenging test case as most existing unsupervised methods rely on aligned shapes, e.g.,  \cite{sharma2020weakly,eisenberger2021neuromorph}.
Note that the PointContrast features perform worse than the hand-crafted features, indicating the overfitting to the pre-training data distribution, as discussed in \cref{sec:motivation}, and thus the limited transferability.
Our approach also outperforms several recent unsupervised approaches in \cref{tab:unaligned_unsup}, including SURFMNET \cite{roufosse2019unsupervised}, Cyclic FMaps \cite{ginzburg2019cyclic}, WSupFMNet \cite{sharma2020weakly}, and Deep Shells \cite{eisenberger2020deep}. 
The comparisons clearly demonstrate the utility of incorporating generalizable pre-trained features, which is missing in prior works. 
We provide qualitative comparisons in \cref{fig:qual_all_human} (Top), showing that only our approach leads to visually plausible results.

\begin{figure}
    \centering
    \includegraphics[width=\columnwidth]{figures/all_humans.pdf}
    \caption{Qualitative comparisons of human shape matching by texture transfer. Top: results of unsupervised matching on SH. Bottom: results of supervised matching on FQ. The best three performing competitors are shown.}
    \label{fig:qual_all_human}
    \vspace{\figmargin}
\end{figure}



\mypara{Robustness to meshing.}
We evaluate the performance of supervised shape matching \cite{donati2020deep} on the original FAUST (FO) dataset \cite{Bogo2014} and its remeshed version by quadratic error simplification (FQ) \cite{sharp2020diffusionnet} to demonstrate the robustness and generalization of our approach against significant mesh connectivity changes across datasets. 
We build a point-wise MLP network \cite{litany2017deep} (to reduce the dependence on the backbone architecture) on top of the baseline features and our \OurMethodName{} respectively, and then predict functional maps for shape pairs.
The MLP backbones are trained on FO with predictions supervised by the ground-truth maps with a simple $L_2$ loss.
We report the mean geodesic error metric on FQ.

\cref{fig:super_fo_fq} shows the correspondence quality with a varying error threshold. 
It can be seen that hand-crafted features such as SHOT degrade rapidly under remeshing. 
Although WKS and HKS are intrinsic features and do not depend on the meshing connectivity, they are not expressive enough by themselves and need to be combined with a powerful network backbone such as DiffusionNet, instead of the point-wise MLPs used in this experiment.
Differently, our approach achieves superior generalization performance in this challenging setting, showing that \OurMethodName{} is highly robust to remeshing and resampling, and effectively captures local geometric structures in deformable shapes.
\cref{fig:qual_all_human} (bottom) presents a qualitative comparison of the computed maps with the three best-performing competitors.

\begin{figure}
    \centering
    \includegraphics[width=\columnwidth]{figures/supervised_fo_fq.pdf}
    \caption{Accuracy of various features for supervised shape matching when the connectivity changes from training to test (mean errors $\times 100$ are reported in the legend).}
    \label{fig:super_fo_fq}
\end{figure}


\subsection{Molecular Surface Segmentation}
\label{subsec:molecular_segmentation}

%\setlength{\tabcolsep}{4pt}
\begin{table}[!t]
     \begin{center}
    \ra{1.0}
          \resizebox{\columnwidth}{!}{%
               \begin{tabular}{@{}lrrr@{}}
                    \cmidrule[\heavyrulewidth]{1-4}
                    \textbf{Method}                                 & \multicolumn{3}{c}{\textbf{Accuracy $\pm$ s.d}}                                                           \\
                    \cmidrule{2-4}
                                                                    & Full Dataset                                    & 50 Shapes                  & 100 Shapes                 \\
                    \cmidrule[\heavyrulewidth]{1-4}
                    PointNet++ \cite{qi2017pointnet}                & 74.4\%                                          & --                         & --                         \\
                    PCNN \cite{atzmon2018pcnn}                      & 78.0\%                                          & --                         & --                         \\
                    SPHNet \cite{poulenard2019effective}            & 80.1\%                                          & --                         & --                         \\
                    SplineCNN \cite{fey2018splinecnn}               & 53.6\%                                          & --                         & --                         \\
                    SurfaceNetworks \cite{kostrikov2018surface}     & 88.5\%                                          & --                         & --                         \\

                    \cmidrule[\heavyrulewidth]{1-4}
                    DiffusionNet - XYZ \cite{sharp2020diffusionnet} & 90.5 $\pm$ 0.6\%                                & 82.7 $\pm$ 0.63\%          & 83.4 $\pm$ 0.67\%          \\
                    DiffusionNet - HKS                              & 90.6 $\pm$ 0.15\%                               & 82.7 $\pm$ 0.16\%          & 84.5 $\pm$ 0.09\%          \\
                    DiffusionNet - WKS                              & 88.7 $\pm$ 0.26\%                               & 77.6 $\pm$ 0.16\%          & 81.2 $\pm$ 0.20\%          \\
                    DiffusionNet - SHOT                             & 92.1 $\pm$ 0.08\%                               & 81.6 $\pm$ 0.31\%          & 85.7 $\pm$ 0.11\%          \\
                    DiffusionNet - PCH                              & 90.3 $\pm$ 0.1\%                                & 79.9 $\pm$ 0.59\%          & 83.6 $\pm$ 0.08\%          \\
                    DiffusionNet - PCN                              & 90.1 $\pm$ 0.09\%                               & 80.1 $\pm$ 0.29\%          & 83.4 $\pm$ 0.28\%          \\
                    DiffusionNet - \OurMethodName{} (ours)          & \textbf{92.6 $\pm$ 0.02\%}                      & \textbf{83.2 $\pm$ 0.20\%} & \textbf{86.8 $\pm$ 0.09\%} \\ % ~\accuchange{+3.4}   ~\accuchange{+2.7}         \\
                    % DiffusionNet - \OurMethodName{} (ours)          & \textbf{93.1 $\pm$ 0.04\%}                      & \textbf{86.1 $\pm$ 0.13\%} & \textbf{88.4 $\pm$ 0.05\%} \\ % ~\accuchange{+3.4}   ~\accuchange{+2.7}
                    \cmidrule[\heavyrulewidth]{1-4}
               \end{tabular}
          }
          \caption{Accuracy of various mesh and point cloud based methods for RNA segmentation. The reported numbers are mean accuracy over 5 runs randomly initialized. $\pm$ denotes standard deviation.}%\maks{shall we remove the 100 shapes setting?}
          \label{tab:rna_seg}
     \end{center}
    \vspace{\figmargin}
\end{table}
%\setlength{\tabcolsep}{1.4pt}


Next, we conduct experiments in the molecular surface segmentation task, which aims to segment RNA molecules into functional components.
We use the dataset introduced in \cite{poulenard2019effective}, consisting of 640 RNA triangle meshes, where
each vertex is labeled into one of 259 atomic categories.
The dataset has an 80/20\% split for training and test sets.
We feed the baseline features and our \OurMethodName{} respectively as input to DiffusionNet and train it to predict a label at each vertex as output.
% Each experiment is run five times with different random initialization.
% Mean accuracy and its standard deviation are reported.

As shown in \cref{tab:rna_seg}, our approach achieves state-of-the-art segmentation performance when used with the full training set, outperforming both hand-crafted and pre-trained PointContrast features as well as several recent shape segmentation networks, such as \cite{kostrikov2018surface}.
We also perform experiments in more challenging settings, where only a fraction of the training set, with respectively 50 and 100 shapes (corresponding to 9\% and 18\% of the training set), is used.
We observe in \cref{tab:rna_seg} that our method consistently outperforms the competitors by a significant margin when given limited training data.
The results highlight that our pre-training and receptive field size optimization strategies bring significant improvement to downstream organic shape analysis tasks.

\subsection{Partial Animal Matching}
\label{subsec:animal_matching}
We also evaluate how well different geometric features perform on deformable shapes in the presence of significant partiality.
For this, we test on the challenging SHREC16' Cuts dataset \cite{cosmo2016shrec}, where the animal classes (cat, centaur, dog, horse, and wolf) are used for partial shape matching.
We follow the setup of DiffusionNet described in \cref{subsec:human_matching} for correspondence prediction. 

We compare our approach to XYZ and SHOT as they are widely used in partial matching pipelines, in addition to full-fledged methods PFM \cite{Rodol2016} and FSP \cite{Litany2017}, specifically tailored toward partial shape matching. 
The results are summarized in \cref{tab:sup_animal}. Qualitative results are visualized in \cref{fig:qual_sup_animal}. Observe that our approach outperforms the competitors in this setting by a significant margin, including specially-tailored partial matching methods PFM and FSP.

\begin{table}[!t]
    \begin{center}
    \ra{1.0}
        \resizebox{0.9\columnwidth}{!}{%
        \begin{tabular}{@{} lr @{}}
            \toprule
            \textbf{Method / Dataset}              & SHREC'16 CUTS Animals \\
            \midrule
            PFM \cite{Rodol2016}                   & 8.8                   \\
            FSP \cite{Litany2017}                  & 12.2                  \\
            DiffusionNet - XYZ                     & 4.9                   \\
            DiffusionNet - SHOT                    & 4.6                   \\ %19.3
            DiffusionNet - \OurMethodName{} (ours) & \textbf{3.7}          \\
            \bottomrule
        \end{tabular}
        }
        \caption{Performance (mean geodesic error $\times 100$) of various features on the SHREC'16 CUTS Animals benchmark.\vspace{-2mm}}
        \label{tab:sup_animal}
    \end{center}
    \vspace{\figmargin}
\end{table}


\begin{figure}[!t]
    \centering
    \includegraphics[width=\columnwidth]{figures/supervised_animals.pdf}
    \caption{Qualitative comparisons of partial animal matching by texture transfer on the cat class of the SHREC'16 CUTS Animals benchmark.}
    \label{fig:qual_sup_animal}
    \vspace{\figmargin}
\end{figure}


\subsection{Shape Classification}
\label{subsec:shape_classification}

% Finally, to show the utility of our pre-trained feature extractor on man-made objects, we adopt the ShapeNet \cite{chang2015shapenet} classification setup from PointContrast \cite{xie2020pointcontrast}, where pre-trained weights are used as initialization for fine-tuning a classification network.
% For comparison with PointContrast, we conduct experiments on the ShapeNetCore v2 dataset with the same train/test split, and our feature extractor pre-trained with the PointInfoNCE loss is used.
% In addition to fine-tuning on the full training set, a limited training data setup (i.e., 1\% or 10\%) is also considered.

% \cref{table-shapenet_classification_accuracy_retrain} shows the classification accuracy comparison.
% We observe that feature pre-training generally improves the performance across different training setups, compared to training from scratch for the classification networks.
% In particular, our network has better classification accuracy than PointContrast when using 1\%, 10\%, or 100\% training data.

We use the ShapeNet dataset \cite{chang2015shapenet} to demonstrate the effectiveness of our pre-trained feature extractor on man-made objects. We follow the classification setup from PointContrast \cite{xie2020pointcontrast}, using pre-trained weights as initialization for fine-tuning a classification network. Here our feature extractor is pre-trained with the PointInfoNCE loss. We conduct experiments on the ShapeNetCore v2 dataset with the same train/test split as PointContrast. We also consider a limited fine-tuning data setup, using a fraction of the fine-tuning data (1\% or 10\%). The classification accuracy comparison is summarized in \cref{table-shapenet_classification_accuracy_retrain}. It can be seen that feature pre-training improves performance compared to training from scratch. Also, our network achieves higher classification accuracy than PointContrast in all training setups.

% \begin{table}[!t]
%     \begin{center}
%         \resizebox{0.8\columnwidth}{!}{%
%         \ra{1.0}
%         \begin{tabular}{@{} lrrr @{}}
%             \toprule
%                                  & 1\% data      & 10\% data     & 100\% data    \\
%             \midrule
%             \multicolumn{4}{c}{\em{PointContrast}}                               \\
%             % \midrule
%             \addlinespace
%             From scratch         & 53.2          & 74.4          & 76.9          \\
%            %Hardest-Contrastive  & 61.3          & 75.3          & 76.8          \\
%             PointInfoNCE         & 60.7          & 73.7          & 77.2          \\
%             \midrule
%             \multicolumn{4}{c}{\em{\OurMethodName{} (ours)}}                                \\
%             % \midrule
%             \addlinespace
%             From scratch         & 59.5          & 72.2          & 79.0          \\
%             PointInfoNCE         & \textbf{66.5} & \textbf{77.2} & \textbf{81.2} \\
%             \bottomrule
%         \end{tabular}
%         }
%         \caption{ShapeNet classification accuracy with limited labeled training data for fine-tuning.\vspace{-1mm}}
%         \label{table-shapenet_classification_accuracy_retrain}
%     \end{center}
%     \vspace{-3mm}
% \end{table}



\begin{table}[!t]
    \begin{center}
        \resizebox{0.99\columnwidth}{!}{%
        \ra{1.0}
        \begin{tabular}{@{} lrrrr @{}}
            \toprule
                \% train data    &  \multicolumn{2}{c}{\em{PointContrast}} & \multicolumn{2}{c}{\em{\OurMethodName{} (ours)}}   \\
                & From scratch  & PointInfoNCE & From scratch  & PointInfoNCE \\
                \midrule
                1\% data & 53.2 & 60.7 & 59.5 & \textbf{66.5}\\
                10\% data & 74.4  & 73.7 & 72.2 & \textbf{77.2}\\
                100\% data & 76.9 & 77.2 & 79.0 & \textbf{81.2}\\
            
            \bottomrule
        \end{tabular}
        }
        \caption{ShapeNet classification accuracy with limited labeled training data for fine-tuning.\vspace{-1mm}}
        \label{table-shapenet_classification_accuracy_retrain}
    \end{center}
    \vspace{-3mm}
\end{table}





\subsection{Ablation Study}
\label{subsec:ablation_study}

\begin{table}[t]
    \begin{center}
    \ra{1.0}
        \resizebox{0.45\columnwidth}{!}{%
        \begin{tabular}{@{} lrr @{}}
            \toprule
            \textbf{Dataset}                       & \textbf{FR-SR}        & \textbf{SR-FR}     \\
            \midrule
            DFAUST                                 & 27.7                  & 4.4                \\
            3DMatch                                & \textbf{4.1}          & \textbf{3.8}       \\
            \bottomrule
        \end{tabular}
        }
        \caption{Results of using features pre-trained on different datasets in the downstream task of unsupervised non-rigid shape matching. 
        %X-Y means training on X and testing on Y. 
        Values are mean geodesic error $\times 100$ on unit-area shapes.}
        \label{tab:dataset_ablation}
    \end{center}
    \vspace{-4.5mm}
\end{table}


We also evaluate the role of the pretraining dataset for local feature learning. 
For this, we compare 3DMatch used in our experiments to DFAUST \cite{dfaust:CVPR:2017}, a large-scale dataset of human subjects in motion.
We use the unsupervised shape matching task and the evaluation protocol introduced in \cref{subsec:human_matching} to test the generalization of a pre-trained feature extractor to the FR and SR datasets.



% \begin{wrapfigure}[8]{r}{0.4\columnwidth}
%     \centering
%     \includegraphics[width=0.4\columnwidth]{figures/pca_projection.png}
%     \vspace{-10pt}
% \end{wrapfigure}

The comparisons in \cref{tab:dataset_ablation} show that pre-training on 3DMatch leads to more generalizable features and consistent matching performance, even though DFAUST has greater similarity to the downstream human shape datasets FR and SR.
We attribute this to the fact that \textit{local geometries }in 3DMatch, which consists of real-world scans, are richer and more complex than those in template-fitted DFAUST, leading to a more universally useful pre-training signal.
% To validate this, we perform PCA \cite{pca01} on the local patches in 
% % here inset
% each dataset and visualize the projections to principal 
% components in the inset figure, showing that the local patches in DFAUST (red dots) are included in 3DMatch (blue dots).
% Please see the supplementary for more analysis and experiments.

To validate this, we perform PCA analysis \cite{pca01} on the local patches of 3DMatch and DFAUST.
For each dataset, we first randomly extract 200K local patches. We then encode each patch as a high dimensional vector by first orienting it using a local reference frame and then voxelizing it to a small 3D grid of resolution = $16^3$ using the method of \cite{gojcic2019perfect}. The resulting vectors are 4096-dimensional and are fed as input to PCA. In \cref{fig:ds_abla} (a), we report the unexplained variance as a function of the number of principal components. It can be seen that 3DMatch is significantly more diverse than DFAUST since more principal components are needed to explain its full variance. In \cref{fig:ds_abla} (b), we visualize the projection of patches in the first two principal components and observe that local patches in DFAUST (red dots) are included in 3DMatch (blue dots), demonstrating the diversity and richness of 3DMatch once more.

\begin{figure}[!t]
    \centering
    \includegraphics[width=\columnwidth]{figures/pretrain_ds_ablation.pdf}
    \caption{Comparing richness of local geometries in 3DMatch and DFAUST via PCA. (a) We perform PCA on sampled 3D local patches and plot the unexplained variance \wrt the number of principal components. (b) We project the local geometries onto the first two principal components.}
    \label{fig:ds_abla}
    \vspace{\figmargin}
\end{figure}






%%%%%%%%%%%%%%%%%%%%%%%%%%%%%%%
%%%%%%%%%%%%%%%%%%%%%%%%%%%%%%
%%%%%%%%%%%%%%%%%%%%%%%%%%%%%%
%%%%%%%%%%%%%%%%%%%%%%%%%%%%%
% We performed PCA on the local patches of 3DMatch and DFAUST.
% For each dataset, we first randomly extracted 200K local patches. 
% We then encode each patch as a high dimensional vector by first orienting it using a local reference frame and then voxelizing it to a small 3D grid of resolution = $16^3$ using the method of \cite{gojcic2019perfect}. 
% The resulting vectors are 4096-dimensional and fed as input to PCA to analyze the internal richness and complexity of each dataset.
% In \cref{fig:pca_unexplained_supp}, we report the unexplained variance as a function of the number of principal components. It further confirms that 3DMatch is significantly more diverse than DFAUST, since more principal components are needed to explain its full variance. 


% \section{Ablation study}
% \label{sec:ablation}
% \begin{table}[!htb]
\setlength\tabcolsep{6pt}
\centering\footnotesize
\caption{Ablation study on the loss functions used during training. Results are shown for models trained and tested on UBFC-rPPG.}
\begin{tabular}{lrrr}
\toprule
{\bf Loss} & {\bf MAE (bpm)} & {\bf RMSE (bpm)} & {\bf $r$\phantom{xxxx}}
\\
\midrule
    $L_b$   &  3.08 $\pm$ 1.69 &  8.08 $\pm$ 3.61 &  0.87 $\pm 0.08$ \\
    $L_s$   & 45.50 $\pm$ 1.22 & 50.04 $\pm$ 0.94 & -0.04 $\pm 0.08$ \\
    $L_v$   & 22.89 $\pm$ 2.83 & 31.51 $\pm$ 2.36 &  0.22 $\pm 0.09$ \\
    $L_s+L_v$   & 51.24 $\pm$ 5.36 & 57.80 $\pm$ 7.39 & -0.04 $\pm 0.09$ \\
    $L_b+L_s$   &  9.99 $\pm$ 2.55 & 17.14 $\pm$ 2.36 &  0.51 $\pm 0.14$ \\
    $L_b+L_v$   &  4.18 $\pm$ 2.88 &  8.90 $\pm$ 5.24 &  0.82 $\pm 0.14$ \\
    \midrule
    \textbf{$L_b+L_s+L_v$}  & \bf 0.59 $\pm$ 0.00 & \bf 1.83 $\pm$ 0.04 & \bf 0.99 $\pm$ 0.00 \\
% \midrule
% \multirow{3}{*}{\underline{PURE}}
% & b   &  9.08 $\pm$ 2.16 & 18.56 $\pm$  2.34 & 0.66 $\pm$ 0.08 \\
% & s   & 22.10 $\pm$ 0.25 & 30.45 $\pm$  0.36 & 0.04 $\pm$ 0.02 \\
% & v   & 34.37 $\pm$ 3.61 & 44.71 $\pm$  3.62 & 0.14 $\pm$ 0.04 \\
% & sv  & 33.34 $\pm$ 7.79 & 48.78 $\pm$ 12.82 & 0.00 $\pm$ 0.01 \\
% & bs  & 13.44 $\pm$ 1.65 & 21.97 $\pm$  1.28 & 0.51 $\pm$ 0.07 \\
% & bv  &  9.99 $\pm$ 3.16 & 19.43 $\pm$  2.96 & 0.63 $\pm$ 0.10 \\
% & bsv &  4.02 $\pm$ 0.06 & 12.03 $\pm$  0.16 & 0.86 $\pm$ 0.00 \\
% \midrule
% \multirow{3}{*}{\underline{DDPM}}
% & b   & 22.25 $\pm$ 0.97 & 32.56 $\pm$ 0.83 &  0.26 $\pm$ 0.04 \\
% & s   & 47.38 $\pm$ 0.47 & 56.04 $\pm$ 0.41 & -0.01 $\pm$ 0.01 \\
% & v   & 34.04 $\pm$ 0.31 & 42.21 $\pm$ 0.50 &  0.02 $\pm$ 0.02 \\
% & sv  & 50.40 $\pm$ 2.48 & 59.79 $\pm$ 3.06 &  0.01 $\pm$ 0.01 \\
% & bs  & 23.76 $\pm$ 1.00 & 34.03 $\pm$ 0.71 &  0.21 $\pm$ 0.03 \\
% & bv  & 21.95 $\pm$ 1.57 & 32.12 $\pm$ 0.98 &  0.27 $\pm$ 0.06 \\
% & bsv & 18.53 $\pm$ 0.36 & 30.43 $\pm$ 0.41 &  0.38 $\pm$ 0.01 \\
\bottomrule
\end{tabular}
\label{tab:loss_ablation}
\end{table}

\noindent \textbf{Loss ablation.} In order to demonstrate the effectiveness of each component and design of our system, we mainly test three aspects of our proposed method: 

\noindent (i) a video copying task is first undertaken, it consists in transferring a NeRF rendered reference video (20 keyframes) which the trajectory is known, under the same and different scenes, with same/close and different camera initialization positions. The objective is to demonstrate the robustness and characteristics of our two complementary cinematic losses across different scenes and influenced by different level of perturbations (\ie. initial position).

\noindent (ii) we carry out an experiment for retrieving correct timing and focal length information from images rendered with known parameters respectively. This experiment is to prove the ability of recovering these parameters by inverted optimization method on dynamic NeRF networks.

\noindent (iii) we investigate the influence of the guidance map by collecting the motion performance and memory usage of different sampling number (\ie number of pixels with gradient) to demonstrate that the guidance can help the convergence and mitigate the memory usage simultaneously.

Tab.~\ref{tab:loss_ablations} reports the ablation of losses for different setups: i) same/different target scenes to the reference to emphasize the cross-domain transfer ability; ii) same/different initial positions to highlight the robustness. Three metrics are computed: Absolute Trajectory Error (RMSE-ATE) reflects the quality of retrieved motion on $SE(3)$ similar to many tracking works~\cite{iMap,zhu2022nice}. To depict the on-screen composition similarity, we measure Pixel Error (PE) (for the same scene) and average Joint Error (JE) in pixel computed by Litepose (with a more accurate model).

We study several losses combinations: iNeRF~\cite{yen2021inerf}, \ie pixel loss with keypoint-driven sampling (w/o guidance map); pixel loss~\footnote{The difference to iNeRF is that the guidance map keeps gradient for selected pixels but the loss is computed on the entire image with \textit{all} pixels, whereas the iNeRF only computes loss on selected ones}, pose and flow loss followed by combination of flow and pose losses (\textit{flow+pose} in Tab.~\ref{tab:loss_ablations}).

According to Tab.~\ref{tab:loss_ablations}, we can observe:
i) iNeRF and pixel losses behave similarly: they both show good performance and robustness against perturbation for the same scene on motion (ATE) and composition qualities (PE, JE). However pixel-based methods fail frequently (\ie camera moving to non-defined area of the NeRF and yielding numerical error) under different scenes. For the barely succeeded experiments, the performances are low due to the misled pixel information by mismatched appearance from different scenes. ii) Comparing to pixel-based methods, the others (\textit{pose}, \textit{flow}, \textit{flow+pose}) show invariance against appearance changing. Nevertheless, they all act differently: (q) flow loss tends to drift heavily if the initialization position is far to the correct one (see JE and PE). The phenomenon is because of the fact that the flow focus on the inter-frame information and extracts no hint on the compositing; (b) in contrast, pose loss completely ignores the inter-frame motion and causes lower performance on ATE yet relatively better results in PE and JE on the same scene, suggesting possible ambiguities on similar human pose and different camera parameters. Heatmap pose feature also shows robustness against initial perturbation, with a small difference of ATE between the close and the different initializations; (c) by combining the two complementary losses: pose and flow, we achieve overall better performances than pose and flow separately, especially under different scene condition. Yet under the same scene, the performance is lower than pixel level tracking (iNeRF and pixel), reflecting on PE and ATE. This is due to less sharpen "convergence cone" (see Fig.~\ref{fig:loss distribution}), limited resolution on extracted heatmap from image, and possible heatmap ambiguities.

% ##################################################################################

\begin{figure}
  \includegraphics[width=0.47\textwidth]{fig/f-t-ablation_v2.png}
  \caption{Visualization of (i) temporal ablation, and (ii) focal length ablation. We show the initial and final frames of the six-frames clips used for ablation. The right column shows the error distribution on the ablated parameter (time and focal length) using our method (red) and a constant parameter setup (green).}
  \label{fig:f-t-ablation}
\end{figure}

\noindent \textbf{Parameter ablation.} We also undertake ablation for parameters: time $m$ and focal length $\phi$. The experiment is done by applying our method on 6-frames clips, where only time or focal length varies. Each experiment is run by 16 times with a random initial value for the studied parameters ($m$ or $\phi$). For each studied parameter, we show the distribution of the error between the prediction and the ground-truth (\textit{red}). We compare our results with the controlled case where the studied parameter remains constant, \ie equals to the random initial value, all along the 6-frames clips (\textit{blue}).

\noindent Fig~\ref{fig:f-t-ablation} (i) shows the result for the time ablation, where only the arms of the mutant waving and Fig~\ref{fig:f-t-ablation} (ii) shows the result for the focal length ablation, camera pose is fixed but the focal length varies. For both examples, our prediction error (\textit{red}) is smaller than the controlled one (\textit{blue}). Some large error values for our method are due to extreme random initial points, making the convergence harder, and also because we are optimizing \textit{all} cinematic parameters, some effects may have ambiguities by other parameters than the objective one: \eg zooming in and moving forward share similar yet different visual effects. 

% ##################################################################################

\setlength{\tabcolsep}{2pt}
    \begin{table}[]
    \resizebox{1.0\columnwidth}{!}{%
    \begin{tabular}{ll|cccccccccc}
    \toprule
     & No. Guidance      & 500  & 1000 & 2000 & 4000 & 6000 & 8000 & 10000 & 32000 & 66752 &  \\ \cline{2-12} 
     & Avg APE (cm)      & 35.0 & 27.6 & 25.3 & 21.0 & 23.4 & 29.8 & 32.6  & 23.0  & 14.3  &  \\
     & Std APE (cm)      & 0.70 & 0.88 & 0.80 & 0.77 & 0.42 & 1.61 & 1.99  & 0.55  & 0.60  &  \\
     & Mem usage (MB) & 7273 & 7439 & 7457 & 7849 & 8339 & 8871 & 9121  & 14071 & 21151 & \\
    \bottomrule
    \end{tabular}
    }
    \caption{We show the influence of guidance map sampling number to the performance and memory usage.}
    \vspace{-0.2cm}
    \label{tab:guidance_ablations}
\end{table}
\noindent \textbf{Guidance ablation.} Tab.~\ref{tab:guidance_ablations} shows the influence of the guidance map on the performance and memory usage. The ablation is done in similar methodology to the Tab.~\ref{tab:loss_ablations}, ATE are computed under the same scene with different initial position (see more in Suppl. Material). Low performances are reported when the sampling number is insufficient. An experimental optimal number is around 4000 which we used for our method. Higher std and memory usage can be noticed when the sampling number keeps increasing. The performance re-improves when including the gradient of \textit{all} pixel in the image (66752). The low-yield behavior of higher sampling numbers could be due to the confused the gradient direction by non-informative pixels, \eg false detection on the heatmap.

\section{Discussion}
\label{sec:discussion}
\section{Discussion}

\subsection{Improvements over Supervised Learning}
It is initially surprising that unsupervised training leads to similar or improved rPPG estimation models compared to those trained in a supervised manner. However, there are several potential benefits to unsupervised training. From a hardware perspective, one of the difficulties in supervised training is aligning the contact pulse waveform with the video frames~\cite{Zhan2020}. The pulse sensor and camera may have a time lag, effectively giving the model an out-of-phase target at training time. Unsupervised training gives the model freedom to learn the phase directly from the video. The contact-PPG signal is also sensitive to motion and may be noisy. Since motion may co-occur at the face and fingertip, the contact signal may misguide the model towards visual features for which they should be invariant.

From a physiological perspective, the pulse observed optically at the fingertip with a contact sensor has a different phase than that of the face, since blood propagates along a different path before reaching the peripheral microvasculature, making alignment nearly impossible without shifting the targets to rPPG estimates from existing methods~\cite{Speth_CVIU_2021}. Additionally, the morphological shape of the contact-PPG waveform depends on numerous factors such as the wavelength of light (and corresponding tissue penetration depth), external pressure from the oximeter clip, and vasodilation at the measurement site~\cite{Moco2018,Abraham2013}. This indicates that the morphology and phase of the target PPG waveform is likely different from the observed rPPG waveform.

\subsection{Why Does It Work?}
The success of the proposed non-contrastive approach depends on specific properties of the data, model, and how the two interact.
Limited model capacity is actually a strength, since it forces discovering features to generalize across inputs.
An infinite capacity network could discover spurious signals in the training data and fail to generalize.
By constraining the model's predictions to have specific periodic properties the limited-capacity model must find a general set of features to produce a signal that exists in all of the training samples, which happens to be the blood volume pulse in our datasets.

As a beneficial side-effect, the model intrinsically learns to ignore common noise factors such as illumination, rigid motion, non-rigid motion (\eg talking, smiling, etc.), and sensor noise, since they may preside outside the predefined bandlimits or with uniform power spectra.
Even if noise exhibits periodic tendencies within the bandlimits for some samples, those features would produce poor signals on other samples.
Therefore, end-to-end unsupervised approaches are particularly well-suited for periodic problems.



%%%%%%%%% REFERENCES
{\small
\bibliographystyle{unsrt}
\bibliography{egbib,main}
}

% Below we first briefly describe the selected models and then their implementation details during pre-training.

% Traditional convolutional action recognition networks before 2017 are mostly built to process single frame or multiple consecutive frames; however, such simple structures overlook the importance of long-range temporal context in action recognition, which somehow underestimates the intrinsic temporal information within videos. 
Temporal segment networks (TSN) proposes segment-based sampling to learn temporal information across frames. 
Specifically, in TSN, a video is evenly divided into several temporal segments, which one random frame is sampled from. 
Then the output from each segment will be aggregated via pooling to obtain the final prediction. 
Temporal Shift Module (TSM) shifts feature channels along the temporal axis, which facilitates information exchanged among neighboring frames. 
It can be plug-and-played in 2D networks to enable stronger temporal modeling at zero computation and zero parameters.
Thus, TSM can achieve the performance of heavy 3D CNNs while maintaining the efficiency of 2D CNNs.
% TSM introduces stronger temporal learning capacity to 2D networks while maintaining light-weight. 

Inflated 3D ConvNet (I3D) is designed to bootstrap from the corresponding 2D network since (1) the architecture of 2D network is well designed and (2) the  weights of 2D network is well pre-trained, e.g., Inception~\cite{inception} $\rightarrow$ Inception-I3D~\cite{carreira2017quo}. 
% utilize pre-trained weights from the corresponding 2D network since these 2D weights have been well-designed and trained to perceive visual concepts.
I3D initializes its 3D kernels by duplicating the 2D ones along the temporal dimension, which helps the convergence of 3D CNNs. 
Inspired by~\cite{vaswani2017attention}, non-local networks (NL) adapts the non-local operation (i.e., self-attention~\cite{vaswani2017attention}) in its building block to model long-range dependency.
For video action recognition, its goal is to relate the same object, or person-object interaction within a distant time interval in videos.
Similar to TSM, non-local block is compatible to most convolutional networks.


TimeSformer is a pure transformer-based model, which is an extension of ViT~\cite{dosovitskiy2020image} to the spatiotemporal space. 
Given the quadratic complexity of self-attention, TimeSformer compares several attention strategies when considering temporal dimention in videos.
Finally, TimeSformer introduces the divided space-time attention to greatly reduce the computation burden but achieves promising results.
% on most video action recognition datasets. 
% This structure shows both effectiveness and efficiency in their reported results. 
Continuing this modeling shift from CNNs to Transformers, VideoSwin extends Swin Transformer~\cite{liu2021swin} by adding the inductive bias of locality in video transformers. 
Simply speaking, it adapts the idea of 2D shifted window self-attention to 3D space, which results in better speed-accuracy trade-off compared to previous approaches~\cite{bertasius2021space,arnab2021vivit}.
% Similarly, VideoSwin is an extension of Swin Transformer~\cite{liu2021swin}, by adapting the 2D shifted window self-attention to 3D.
% And shifted window ensure the connection across distant regions in the spatiotemporal tensors.


\begin{figure}[t]
\centering
    \includegraphics[width=8cm]{figures/radar_new.pdf}
    \caption{The rank of the averaged performance within different data domains for the 6 models in different settings. The most outside in these radar images means the highest performance. For each domain, we average the top-1 accuracy as the scores in finetuning and average the top-1 accuracy of 16-shot results in few-shot learning. Complete results are shown in Table~\ref{tab:finetune} and Figure~\ref{fewshot}.}
    \label{radar}
\end{figure}

\end{document}
