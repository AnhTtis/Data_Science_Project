
\appendix
\section{Fitting to Period Doubled Decays, Variable Interaction Time}

\begin{figure*}
  \centering
    \includegraphics[width=0.44\textwidth]{appendix_images/t15_t30_fit.pdf}\qquad
      \includegraphics[width=0.44\textwidth]{appendix_images/timescale_dependence_fit.pdf}
  \caption{Left: Exemplary fitting results for variable interaction time DTC experiments. For each fit, the non-dimensional decay timescale $\tau_F$ (units of Floquet periods) is given along with the exponential stretching parameter $b$. (See SM for extended data). Right: Interpolation of the signal lifetimes extracted from left, as a function of the Floquet period $T$, $\tau(T) = aT - b$. The fitted parameters are $a = 37.11$ (non-dimensional) and $b=453.39\mu$s.}
  \label{fig.verbose_fit_internal}
\end{figure*}

Here, we detail the fitting procedure to the period doubled response under time crystalline driving, where the interaction period is variable. The model of interest is a four-parameter model,
\begin{equation}
        S(t) = a\exp(- (t/\tau)^b) + c,
\end{equation}
developed in equation \ref{eq.non_int_decay}. The fitting is performed in Python using SciPy's \verb+curve_fit+ function \cite{2020SciPy-NMeth}. We consider $T\in \lbrace 15, 30, 45, 60, 90, 120 \rbrace \mu$s, which corresponds to dimensionless interaction magnitudes $u\in\lbrace 0.125, 0.25, 0.375, 0.5, 0.75, 1.00\rbrace$. The results are shown in figure \ref{fig.verbose_fit_internal}.
\begin{figure}
  \centering
    \includegraphics[width=0.44\textwidth]{appendix_images/twot_15_30.pdf}
  \caption{Two-timescale fitting for the $T=15\mu$s DTC experiment  and the $T=30\mu$s DTC experiment  in non-dimensional units of Floquet periods.}
  \label{fig.verbose_fit_followup}
\end{figure}

In figure \ref{fig.verbose_fit_internal}, there are two points of note. Firstly, the fitting for $T=15\mu$s shows systematic error -- the model is a bad fit for the given data. In this case, the two-timescale fitting procedure is a better choice and matches the notion that high driving frequency leads to prethermal time crystallinity. As $T$ increases, we similarly argue that the two-timescale fitting procedure is a bad choice. In figure \ref{fig.verbose_fit_followup}, we show the results of the two-timescale fitting procedure for $T=15,30\mu$s. The $T=15$ fitting demonstrates a strongly resolved transition between timescales, whereas the $T=30$ plot indicates that there might be a second timescale, but the transition between prethermalizing and prethermal timescales is weak. Notably, the choice of the cutting point between timescales does not significantly impact the fitting results. Hence, the argument in favor of prethermalization is quite weak, and so we take the single stretched exponential fit to be sufficiently explanatory. Secondly, there is a clear and apparent increase in the signal lifetime as the kicking period $T$ increases. In real-time units for the signal lifetime, this trend is linear, and is shown in figure \ref{fig.verbose_fit_internal} with a linear fitting.

Indeed, we see that the two-timescale fit well describes the $T=15\mu$s interaction time data, hence providing evidence for prethermalization, hence genuine time crystallinity. Further, the linear trend of the physical lifetimes, $\tau(T) = 37.11 T - 453.39$ ($\mu$s), implies that the non-dimensional Floquet period lifetime is  bounded. Indeed, $\tau_F(T) = 37.11 - 453.39/T$ (periods) is bounded above by 37.11 Floquet periods.

\section{Effects of Disordered Fields}

Using Wei16, we can vary the effective disorder strength over a single Floquet Period. Concretely, we have two tunable parameters, $u,v$, such that our Floquet Hamiltonian is 
\begin{equation}
    \hat{H}_{F} = \frac{v}{3} \sum_i r_i \hat{\sigma}_z^{(i)} +  \frac{u}{2} \sum_{i<j} J_0 \bigg(\hat{S}_z^{(i)}\hat{S}_z^{(j)} - \frac{1}{2}\big(\hat{S}_x^{(i)}\hat{S}_x^{(j)}+ \hat{S}_y^{(i)}\hat{S}_y^{(j)}\big)\bigg).
\end{equation}
Above, $r_i$ is the locally disordered field induced by the unpolarized phosphorus spins, and $0 \leq v \leq 0.4$ is the allowed parameter range under Wei16. Then, for selected kicking angles $\theta$, and interaction strengths $u$, we can investigate the effect of disorder on the stability of our time crystal. The results of this investigation are given in figure \ref{fig.disorder_investigation}. In summation, we see no evidence that increasing disorder leads to a longer-lived signal, hence MBL is likely not the stabilization mechanism for this time crystal.

\begin{figure*}
  \centering
  \includegraphics[width=0.44 \textwidth]{appendix_images/dis_trace_175.pdf}\qquad
  \includegraphics[width=0.44\textwidth]{appendix_images/dis_diff_175.pdf}
   \includegraphics[width=0.44\textwidth]{appendix_images/dis_trace_170u10.pdf}\qquad
    \includegraphics[width=0.44\textwidth]{appendix_images/dis_diff_170u10.pdf}
    \caption{Effects of increasing disorder on time crystallinity for $\theta=175^{\circ}$, $u=.05$, and $\theta=170^{\circ}$, $u=0.08,0.10$. On the left, we show the measured magnetization signal for each disorder instance. On the right, we take the difference between experiments with disorder to the experiment with no disorder. Notice that the disordered field does not stabilize the signal; increasing disorder simply leads to increased variability in the measured signal.}
  \label{fig.disorder_investigation}
\end{figure*}

\section{Local $z$ Magnetization State Manipulations}
\begin{figure*}
  \centering
    \includegraphics[width=0.4\textwidth]{appendix_images/nonint_loc.pdf}\qquad
    \includegraphics[width=0.4\textwidth]{appendix_images/nonint_loc_fit.pdf}
  \caption{Results of local $z$ state experiments under periodic driving with interactions turned off via Peng24. On the left, signals are shown for kicking angles $\theta=0^{\circ},170^{\circ}, 180^{\circ}$. We notice that the decay envelope is similar for all three cases, as in the global $z$ case, albeit decaying more rapidly. On the right, we show the fitting of the $\theta=0^{\circ}$ signal to the four-parameter model given in equation \ref{eq.non_int_decay}.}
  \label{fig.nonint_local_comparison}
\end{figure*}

In order to further verify our ability to produce a robust local state, we consider the case where interactions are turned off entirely. Indeed, we want to ensure that the rapid decay observed in figure \ref{fig.local_v_global} is due to time crystal inducing interactions, and not due to poor state preparation. In figure \ref{fig.nonint_local_comparison}, we demonstrate that the local magnetization signal persists for long times when interactions are turned off with Peng24. In fact, the observed decay envelope fits well to the four-parameter model of \ref{eq.non_int_decay}, with a sub-linear exponential argument similar to the non-interacting global magnetization response. The time scale of the decay is much shorter for the local magnetization than global magnetization (1.5 ms vs 6.8 ms), but still quite long compared to $T_2 \approx 30\mu$s. In summation, the results of figure \ref{fig.nonint_local_comparison} demonstrate our state preparation procedure and control scheme are not the causes of the rapid decay seen in figure \ref{fig.local_v_global}.



\section{Two-timescale Fitting Results, Variable Interaction Strength}

In figure \ref{fig.two_timescale_verbose}, we show the two-timescale fitting results for stroboscopic $z$-field strengths of $hT = \pi/2, \pi$. This fitting procedure is performed for all $hT\in\lbrace \pi/4, \pi/2, 3\pi/4, \pi\rbrace$ to populate table \ref{table.timescales}. The fitting for $hT=0$ is shown in figure \ref{fig.two_timescale}. Concretely, $\tau_1$ gives the decay timescale of the initial state to the prethermal time crystalline state, and $\tau_2$ gives the lifetime of the prethermal time crystalline state. 

For each value of $hT$, an optimal $u$ is selected to maximize the two-timescale contrast. Generally, for larger values of $hT$, smaller values of $u$ are used. Given the theoretical expectation that $\tau_2$ depends exponentially on $u$, the quality of the fit is increasingly sensitive to large changes in $u$ with increasing $hT$. The fitting in figure \ref{fig.two_timescale} shows that the prethermal timescale is robust to changes in $u$, which can be attributed the fact that the transition to the time crystalline phase occurs near $u=0.07$. Thus, $\Delta u = 0.02$ is a small fraction of the interaction magnitude at the transition point. Even so, the contrast between timescales is most pronounced for $u=0.07$. However, for $hT=\pi$, the transition point to the time crystalline phase occurs near $u=0.02 (\gamma = \pi/40)$. Given our resolution in $u$ is limited to $0.01$, we are limited to (proportionally) large changes in the interaction magnitude relative to the transition point. If our timing control allowed for a higher resolution in $u$, then we could reconstruct a phase diagram for each $hT$ similar to figure \ref{fig.phase_diagram}. Qualitatively, we expect that the transition boundary be pushed down with increasing $hT$, thus increasing the phase space area occupied by the time crystalline state.

\begin{figure*}
  \centering
  \includegraphics[width=0.4\textwidth]{appendix_images/fit_ht90.pdf}\qquad
  \includegraphics[width=0.4\textwidth]{appendix_images/fit_ht180.pdf}
\caption{Two-timescale fitting results for various applied stroboscopic $z$ fields, in non-dimensional units of Floquet periods. The data (see SM) generated by this procedure is used to populate the entries of table \ref{table.timescales}.} 
\label{fig.two_timescale_verbose}
\end{figure*}
\newpage