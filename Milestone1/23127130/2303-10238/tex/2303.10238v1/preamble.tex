\usepackage{xcolor}
\usepackage{graphicx}
\usepackage[left=13mm,right=13mm,top=35mm,columnsep=15pt]{geometry} 
%\usepackage{caption, subcaption}
\usepackage{subfigure}
\usepackage{float}
\usepackage{adjustbox}
\usepackage{placeins}
\usepackage[T1]{fontenc}
\usepackage{csquotes}
\usepackage[vcentermath]{youngtab}


\usepackage[nointegrals]{wasysym}
\usepackage{amsmath}
\usepackage{amssymb}
\usepackage{amsthm}
\usepackage{braket}
\usepackage{bm}
\usepackage{mathtools}
\usepackage{tensor}
\usepackage{mathrsfs}
%\usepackage{subcaption}

\def\openone{\leavevmode\hbox{\small$1$\normalsize\kern-.33em$1$}}

\renewcommand{\labelenumi}{(\alph{enumi})} 
\let\vaccent=\v 
\renewcommand{\v}[1]{\ensuremath{\mathbf{#1}}} 
\newcommand{\gv}[1]{\ensuremath{\mbox{\boldmath$ #1 $}}} 
\newcommand{\uv}[1]{\ensuremath{\mathbf{\hat{#1}}}}
\newcommand{\inprod}[2]{\ensuremath{\langle #1, #2 \rangle}}
\newcommand{\iprod}[2]{\ensuremath{\langle #1 | #2 \rangle}}
\newcommand{\norm}[1]{\left\lVert#1\right\rVert}
\newcommand{\abs}[1]{\left| #1 \right|} 
\newcommand{\avg}[1]{\left< #1 \right>} 
\let\underdot=\d 
\renewcommand{\d}[2]{\frac{d #1}{d #2}}
\newcommand{\bs}{\boldsymbol}
\newcommand{\imag}{{\rm i}}
\newcommand{\adj}{{\rm ad}}
\newcommand{\pd}[2]{\frac{\partial #1}{\partial #2}} 
\newcommand{\pds}[2]{\frac{\partial^2 #1}{\partial #2^2}}
\newcommand{\pdd}[3]{\frac{\partial^2 #1}{\partial #2 \partial #3}} 
\newcommand{\pdc}[3]{\left( \frac{\partial #1}{\partial #2}
 \right)_{#3}}
\newcommand{\matrixel}[3]{\left< #1 \vphantom{#2#3} \right|
 #2 \left| #3 \vphantom{#1#2} \right>} 
\newcommand{\grad}[1]{\gv{\nabla} #1} % for gradient
\let\divsymb=\div % rename builtin command \div to \divsymb
\renewcommand{\div}[1]{\gv{\nabla} \cdot #1} 
\newcommand{\curl}[1]{\gv{\nabla} \times #1} % for curl
\let\baraccent=\= % rename builtin command \= to \baraccent
\renewcommand{\=}[1]{\stackrel{#1}{=}}
\DeclareMathOperator{\Tr}{\text{tr}}
\newcommand{\ketbra}[2]{\left| #1 \vphantom{#2}\right>\!\!\left< #2\vphantom{#1}\right|}
\newcommand{\be}{\begin{equation}}
\newcommand{\bel}[1]{\begin{equation}\label{#1}}
\newcommand{\ee}{\end{equation}}
\renewcommand\qedsymbol{$\blacksquare$}
\renewcommand{\labelitemi}{\textendash}
\newcommand{\gsize}[1]{\abs{#1}}
\DeclareMathOperator{\gprod}{\circ} % do not use \newcommand
\newcommand{\cclass}[1]{\text{Conj}(#1)}
\newcommand{\ssubg}[0]{<}
\newcommand{\subg}[0]{\leq}
\newcommand{\nsubg}[0]{\mathrel{\unlhd}}
\newcommand{\ii}{\mathrm{i}}
\newcommand{\ggen}[1]{\left<#1\right>}
\newcommand{\dd}{\mathrm{d}}
\newcommand{\C}{\mathbb{C}}
\newcommand{\R}{\mathbb{R}}
\newcommand{\Z}{\mathbb{Z}}
\renewcommand{\H}{\hat{\mathcal{H}}}
\newcommand{\twovec}[2]{\begin{bmatrix} #1\\ #2\end{bmatrix}}
\renewcommand{\Tr}{\operatorname{tr}}

\DeclarePairedDelimiter\ceil{\lceil}{\rceil}
\DeclarePairedDelimiter\floor{\lfloor}{\rfloor}

\graphicspath{ {./images/} }
\graphicspath{ {./appendix_images/} }


\makeatletter

\renewenvironment{widetext@grid}{%
  \par\ignorespaces
  \setbox\widetext@top\vbox{%
   \vskip15\p@
   \hb@xt@\hsize{%
    \leaders\hrule\hfil
    \vrule\@height6\p@
   }%
   \vskip6\p@
  }%
  \setbox\widetext@bot\hb@xt@\hsize{%
    \vrule\@depth6\p@
    \leaders\hrule\hfil
  }%
  \onecolumngrid
%  \dimen@\ht\widetext@top\advance\dimen@\dp\widetext@top
%  \cleaders\box\widetext@top\vskip\dimen@
  \let\set@footnotewidth\set@footnotewidth@ii
}{%
  \par
%  \setbox\widetext@bot\vbox{%
%   \hb@xt@\hsize{\hfil\box\widetext@bot}%
%   \vskip14\p@
%  }%
%  \dimen@\ht\widetext@bot\advance\dimen@\dp\widetext@bot
%  \cleaders\box\widetext@bot\vskip\dimen@
  \twocolumngrid\global\@ignoretrue
  \@endpetrue
}%

\makeatother