%File: anonymous-submission-latex-2023.tex
\documentclass[letterpaper]{article} % DO NOT CHANGE THIS
\usepackage[]{aaai23}  % DO NOT CHANGE THIS
\usepackage{times}  % DO NOT CHANGE THIS
\usepackage{helvet}  % DO NOT CHANGE THIS
\usepackage{courier}  % DO NOT CHANGE THIS
\usepackage[hyphens]{url}  % DO NOT CHANGE THIS
\usepackage{graphicx} % DO NOT CHANGE THIS
\urlstyle{rm} % DO NOT CHANGE THIS
\def\UrlFont{\rm}  % DO NOT CHANGE THIS
\usepackage{natbib}  % DO NOT CHANGE THIS AND DO NOT ADD ANY OPTIONS TO IT
\usepackage{caption} % DO NOT CHANGE THIS AND DO NOT ADD ANY OPTIONS TO IT
\frenchspacing  % DO NOT CHANGE THIS
\setlength{\pdfpagewidth}{8.5in} % DO NOT CHANGE THIS
\setlength{\pdfpageheight}{11in} % DO NOT CHANGE THIS
%
% These are recommended to typeset algorithms but not required. See the subsubsection on algorithms. Remove them if you don't have algorithms in your paper.
\usepackage{algorithm}
\usepackage[noend]{algpseudocode} % algorithmicx
\let\oldReturn\Return
\renewcommand{\Return}{\State\oldReturn}

%
% These are are recommended to typeset listings but not required. See the subsubsection on listing. Remove this block if you don't have listings in your paper.
\usepackage{newfloat}
\usepackage{listings}
\DeclareCaptionStyle{ruled}{labelfont=normalfont,labelsep=colon,strut=off} % DO NOT CHANGE THIS
\lstset{%
	basicstyle={\footnotesize\ttfamily},% footnotesize acceptable for monospace
	numbers=left,numberstyle=\footnotesize,xleftmargin=2em,% show line numbers, remove this entire line if you don't want the numbers.
	aboveskip=0pt,belowskip=0pt,%
	showstringspaces=false,tabsize=2,breaklines=true}
\floatstyle{ruled}
\newfloat{listing}{tb}{lst}{}
\floatname{listing}{Listing}
%
% Keep the \pdfinfo as shown here. There's no need
% for you to add the /Title and /Author tags.
\pdfinfo{
/TemplateVersion (2023.1)
}

% DISALLOWED PACKAGES
% \usepackage{authblk} -- This package is specifically forbidden
% \usepackage{balance} -- This package is specifically forbidden
% \usepackage{color (if used in text)
% \usepackage{CJK} -- This package is specifically forbidden
% \usepackage{float} -- This package is specifically forbidden
% \usepackage{flushend} -- This package is specifically forbidden
% \usepackage{fontenc} -- This package is specifically forbidden
% \usepackage{fullpage} -- This package is specifically forbidden
% \usepackage{geometry} -- This package is specifically forbidden
% \usepackage{grffile} -- This package is specifically forbidden
% \usepackage{hyperref} -- This package is specifically forbidden
% \usepackage{navigator} -- This package is specifically forbidden
% (or any other package that embeds links such as navigator or hyperref)
% \indentfirst} -- This package is specifically forbidden
% \layout} -- This package is specifically forbidden
% \multicol} -- This package is specifically forbidden
% \nameref} -- This package is specifically forbidden
% \usepackage{savetrees} -- This package is specifically forbidden
% \usepackage{setspace} -- This package is specifically forbidden
% \usepackage{stfloats} -- This package is specifically forbidden
% \usepackage{tabu} -- This package is specifically forbidden
% \usepackage{titlesec} -- This package is specifically forbidden
% \usepackage{tocbibind} -- This package is specifically forbidden
% \usepackage{ulem} -- This package is specifically forbidden
% \usepackage{wrapfig} -- This package is specifically forbidden
% DISALLOWED COMMANDS
% \nocopyright -- Your paper will not be published if you use this command
% \addtolength -- This command may not be used
% \balance -- This command may not be used
% \baselinestretch -- Your paper will not be published if you use this command
% \clearpage -- No page breaks of any kind may be used for the final version of your paper
% \columnsep -- This command may not be used
% \newpage -- No page breaks of any kind may be used for the final version of your paper
% \pagebreak -- No page breaks of any kind may be used for the final version of your paperr
% \pagestyle -- This command may not be used
% \tiny -- This is not an acceptable font size.
% \vspace{- -- No negative value may be used in proximity of a caption, figure, table, section, subsection, subsubsection, or reference
% \vskip{- -- No negative value may be used to alter spacing above or below a caption, figure, table, section, subsection, subsubsection, or reference

\setcounter{secnumdepth}{0} %May be changed to 1 or 2 if section numbers are desired.

% The file aaai23.sty is the style file for AAAI Press
% proceedings, working notes, and technical reports.
%

\let\proof\relax
\let\endproof\relax
\usepackage[cmex10]{amsmath}
\usepackage{amsfonts}
\usepackage{amsthm}

\usepackage{multicol}

\usepackage{xspace}

\newcommand\moduleName[1]{\textsf{#1}\xspace}
\newcommand{\MAMO}{\moduleName{MAMO}}
\newcommand{\MAPF}{\moduleName{MAPF}}
% \newcommand\algName[1]{\textsf{#1}\xspace}
\newcommand{\MfM}{\moduleName{\fontfamily{pbk}\selectfont M{\footnotesize 4}M}}
\newcommand{\EMfM}{\moduleName{\fontfamily{pcr}\selectfont E-\fontfamily{pbk}\selectfont M{\footnotesize 4}M}}
\newcommand{\MAPFalg}{\textsc{MAPF}\xspace}
\newcommand{\CBS}{\textsc{CBS}\xspace}
\newcommand{\WHCA}{\textsc{WHCA*}\xspace}
\newcommand{\KPIECE}{\textsc{KPIECE}\xspace}
\newcommand{\RRT}{\textsc{RRT}\xspace}
\newcommand{\PrPl}{\textsc{PP}\xspace}
\newcommand{\UI}{\textsc{UI}\xspace}
\newcommand{\SelSim}{\textsc{SelSim}\xspace}
\newcommand{\Dogar}{\textsc{Dogar}\xspace}

\usepackage[table,xcdraw,dvipsnames,x11names]{xcolor}
\usepackage{color}
\def\D#1{\textcolor{Plum}{#1}}
\def\C#1{\textcolor{BrickRed}{#1}}

\colorlet{documentLinkColor}{red}
\colorlet{documentUrlColor}{blue}
\colorlet{documentCitationColor}{ForestGreen}
% \usepackage[bookmarks=true]{hyperref}
% \hypersetup{
%      colorlinks = true,
%      citecolor = documentCitationColor,
%      linkcolor = documentLinkColor,
%      urlcolor = documentUrlColor,
% }

\usepackage{graphicx}

\newcommand{\PP}{\mathcal{P}}
\newcommand{\X}{\mathcal{X}}
\newcommand{\A}{\mathcal{A}}
\newcommand{\T}{\mathcal{T}}
\newcommand{\K}{\mathcal{K}}
\newcommand{\R}{\mathcal{R}}
\newcommand{\Obs}{\mathcal{O}}
\newcommand{\OI}{\mathcal{I}}
\newcommand{\OM}{\mathcal{M}}

\DeclareMathOperator*{\argmin}{\arg\!\min}
\DeclareMathOperator*{\argmax}{\arg\!\max}

\usepackage{textcomp,gensymb}

\theoremstyle{proposition}
\newtheorem{proposition}{Proposition}

\usepackage{titlesec}
% \titleformat*{\subsubsection}{\bfseries}

\usepackage[binary-units]{siunitx}
\usepackage{booktabs}
\usepackage{multirow}
\usepackage{cite}

\usepackage{mathtools}


\newcommand\blfootnote[1]{%
  \begingroup
  \renewcommand\thefootnote{}\footnote{#1}%
  \addtocounter{footnote}{-1}%
  \endgroup
}

% Title


% Your title must be in mixed case, not sentence case.
% That means all verbs (including short verbs like be, is, using,and go),
% nouns, adverbs, adjectives should be capitalized, including both words in hyphenated terms, while
% articles, conjunctions, and prepositions are lower case unless they
% directly follow a colon or long dash
% \title{Enhanced-\textsf{\fontfamily{pbk}\selectfont M{\Large 4}M}: A Graph Search Formulation for Manipulation Among Movable Objects Guided by Multi-Agent Pathfinding}
\title{Planning for Manipulation among Movable Objects: Deciding Which Objects Go Where, in What Order, and How\footnote{This work was in part supported by ARL grant W911NF-18-2-0218 and ONR grant N00014-18-1-2775.}}

% 1. Enhanced-\textsf{\fontfamily{pbk}\selectfont M{\Large 4}M}: Towards a more complete ...
% 2. Planning for Manipulation Among Movable Objects - deciding who goes where and in what order

\author{
    %Authors
    % All authors must be in the same font size and format.
    Dhruv Saxena,
    Maxim Likhachev
}
\affiliations{
    %Afiliations
    Robotics Institute, Carnegie Mellon University\\
    % 5000 Forbes Ave\\
    % Pittsburgh, PA 15213, USA.\\
    % email address must be in roman text type, not monospace or sans serif
    \{dsaxena, mlikhach\}@andrew.cmu.edu
%
% See more examples next
}
\usepackage{bibentry}

\begin{document}

\maketitle

\begin{abstract}
% The Manipulation Among Movable Objects \raisebox{0.5pt}{(}\MAMO{\raisebox{0.5pt}{)}} domain addresses the challenge of solving manipulation problems in workspaces where no collision-free solution exists.
% For pick-and-place tasks in heavy clutter the robot must rearrange some `movable' objects in the scene in order to retrieve a target object.
% A recently proposed algorithm for this problem, \MfM, utilises guidance from a solution to a Multi-Agent Pathfinding \raisebox{0.5pt}{(}\MAPF{\raisebox{0.5pt}{)}} abstraction of the \MAMO problem.
% The \MAPF solution `provide's information about \emph{which} objects need to be moved \emph{where}.
% \MfM relies on nonprehensile push actions to compute \emph{how} the robot might realise these rearrangements, greedily committing to the first valid push found.
% % However, in computing \emph{how} the robot might realise these rearrangements, \MfM greedily commits to valid nonprehensile rearrangement actions prescribed by the initial \MAPF solution and declares failure when none exist.
% In this paper, we extend \MfM and present Enhanced-\MfM (\EMfM) -- a systematic graph search-based solver for \MAMO problems that searches over different possible rearrangements of the scene, orderings of pushes for movable objects that need to be rearranged, and instantiations of all push actions to rearrange a movable object.
% The search-based nature of \EMfM leads to stronger theoretical properties vis-\`a-vis solvable \MAMO problems.
% We test the performance of \EMfM in simulated and real-world experiments with the PR2 robot and show that it solves many more \MAMO problems than \MfM and other \MAMO baselines with improved planning times despite searching a much larger space of solutions.

We are interested in pick-and-place style robot manipulation tasks in cluttered and confined 3D workspaces among movable objects that may be rearranged by the robot and may slide, tilt, lean or topple. % as the robot interacts with them.
A recently proposed algorithm, \MfM, determines \emph{which} objects need to be moved and \emph{where} by solving a Multi-Agent Pathfinding  \raisebox{1pt}{(}\MAPF{\raisebox{1pt}{)}} abstraction of this problem.
It then utilises a nonprehensile push planner to compute actions for \emph{how} the robot might realise these rearrangements and a rigid body physics simulator to check whether the actions satisfy physics constraints encoded in the problem.
However, \MfM greedily commits to valid pushes found during planning, and does not reason about orderings over pushes if multiple objects need to be rearranged.
Furthermore, \MfM does not reason about other possible \MAPF solutions that lead to different rearrangements and pushes.
In this paper, we extend \MfM and present Enhanced-\MfM (\EMfM) -- a systematic graph search-based solver that searches over orderings of pushes for movable objects that need to be rearranged and different possible rearrangements of the scene.
% To address the higher computational complexity associated with searching for a solution in this much larger space, \EMfM stores information about all successful and unsuccessful pushes found during planning.
% The former are used to avoid simulations of similar pushes seen previously, and the latter help us feedback information to the \MAPF solver to efficiently search the space of rearrangements of the scene.
We introduce several algorithmic optimisations to circumvent the increased computational complexity, discuss the space of problems solvable by \EMfM and show that experimentally, both on the real robot and in simulation, it significantly outperforms the original \MfM algorithm, as well as other state-of-the-art alternatives when dealing with complex scenes.

\end{abstract}


%%%%%%%%%%%%%%%%%%%%%%%%%%%%%%%%%%%%%%%%%%%%%%%%%%%%%%%%%%%%%%%%%%%%%%%%%%%%%%%%
%%%%%%%%%%%%%%%%%%%%%%%%%%%%%%%%%%%%%%%%%%%%%%%%%%%%%%%%%%%%%%%%%%%%%%%%%%%%%%%%

\section{Introduction}
Simple pick-and-place robot manipulation tasks can be difficult to solve for motion planning algorithms that do not reason about how `movable' objects in the confined workspace might need to be rearranged in order to find a feasible solution path.
Such situations are commonly encountered when robot arms have to grasp and extract desired objects from cluttered shelves or pack several objects in a box.
Solving these ``Manipulation Among Movable Objects'' \raisebox{0.5pt}{(}\MAMO{\raisebox{0.5pt}{)}} problems~\cite{alami,StilmanMAMO} requires a planning algorithm to decide \emph{which} objects should be moved~\cite{Hauser14}, \emph{where} to move them, and \emph{how} they may be moved.
For the scene shown in Figure~\ref{fig:intro_fridge} (a), the tomato soup can is the ``object-of-interest'' (OoI) to be retrieved.
In order to do so, the PR2 robot must first move the coffee can and potted meat can out of the way so that the grasp pose for the OoI becomes reachable.

\begin{figure}[t]
    \centering
    \includegraphics[width=\columnwidth]{figures/intro.pdf}
    \caption{(a)~The tomato soup can (yellow outline) is the object-of-interest (OoI) to be retrieved. The potted meat can and coffee can in front of it must be rearranged out of the way in order to retrieve the OoI and solve the \MAMO problem. (b)~Trying to retrieve the beer can (OoI, yellow outline) leads to a complex interaction with the movable potted meat can being tilted by the robot arm.}
    \label{fig:intro_fridge}
\end{figure}

Existing state-of-the-art approaches in literature commonly assume prehensile (pick-and-place) rearrangement actions, e.g.~\citet{StilmanMAMO,wang2022lazy} and/or planar robot-object and object-object interactions, e.g.~\citet{BergSKLM08,homology_nonprehensile}.
Prehensile rearrangements not only preclude manipulation of big, bulky and otherwise ungraspable objects, but also assume access to known grasp poses for all movable objects and availability of stable placement locations for them in a cluttered and confined workspace.
The planar world assumption does not account for realistic physics interactions between objects in a real-world scene.
In contrast to these, our emphasis is on solving \MAMO problems (i) in a 3D workspace where robot actions can lead to complex multi-body interactions where objects tilt, lean on each other, slide, and topple (Figure~\ref{fig:intro_fridge} (b)); and (ii) with nonprehensile push actions for rearranging the clutter in the scene.
With this we allow for more seamless and natural manipulation that rearranges objects aside without picking them up while considering complicated toppling, sliding, and leaning effects.

% To solve the \MAMO problems of interest to us, we must search the composite configuration space that includes the configurations of a $q$ degrees-of-freedom robot in $\mathbb{R}^q$ and all objects in $SE(3)$ for a solution.
The \MAMO problem definition includes information about which objects are \emph{movable} and which are static or \emph{immovable} obstacles.
All objects have a set of \emph{interaction constraints} associated with them that define valid robot-object and object-object interactions in the workspace.
% Since our emphasis is on solving these manipulation problems in the 3D workspace, we must account for complex multi-body interactions where objects tilt, lean on each other, slide, and topple.
% Therefore we keep track of object configurations in the full $SE(3)$ space.
Interaction constraints encode that neither the robot nor any other object can make contact with immovable obstacles (an object that cannot be interacted with, such as a wall), and movable objects cannot fall off the shelf, tilt too far (beyond $\SI{25}{\degree}$), or move with a high instantaneous velocity (above $\SI{1}{\meter\per\second}$).
These constraints help model realistic and desirable robot-object interactions since we want to prevent robots from carelessly hitting, pushing or throwing objects around.
In order to forward simulate the effect of nonprehensile robot pushes on the objects, we use a rigid body physics simulator to evaluate the interaction constraints and determine the resultant state of the workspace.

In recent work~\cite{Saxena23}, we proposed the \MfM algorithm for \MAMO to answer the questions of \emph{which} objects to move \emph{where}, and \emph{how}.
\MfM (``Multi-Agent Pathfinding for Manipulation Among Movable Objects'') relies on an \MAPF abstraction of \MAMO problems where the movable objects are artificially actuated agents with the goal of avoiding collisions with (i) the robot arm as it retrieves the OoI, (ii) each other, and (iii) immovable obstacles.
A solution to this \MAPF abstraction %answers two of the three questions fundamental to \MAMO and
informs \MfM of \emph{which} objects must be moved and \emph{where} so that the OoI can be retrieved to solve the \MAMO problem.
\MfM then samples nonprehensile pushes to try and realise the rearrangements suggested by the \MAPF solution in the real-world, thereby addressing the third question of \emph{how} objects may be moved.

However, \MfM is greedy and can fail to find solutions in many cases where one may exist.
It is greedy in three different ways.
First, \MfM does not search over all possible orderings of object rearrangements if multiple movable objects need to be moved.
It greedily commits to the first valid push it finds and continues searching for a solution from the resultant state of that push.
Second, \MfM never reconsiders solving the \MAPF problem again for a different solution that might require objects to be rearranged differently.
As such, it does not search over all possible rearrangements of the scene.
This is important because in cases where an object cannot be rearranged successfuly as per the \MAPF solution (perhaps due to robot kinematic limits, the presence of immovable obstacles, interaction constraint violations etc.), we must replan the \MAPF solution and consider a different way to rearrange the scene that may indeed be feasible.
Finally, even in cases where we successfully rearrange an object to a particular location, \MfM never reconsiders moving that object differently, which may be required if no solution can be found from the resultant state of the valid push.

This paper extends \MfM and presents Enhanced-\MfM (\EMfM), an algorithm that addresses all three shortcomings of \MfM and does so by searching a much larger space for solutions.
It considers different orderings for rearranging objects, replans \MAPF solutions as and when required, and considers different ways to rearrange any particular object.
Although the search space for \EMfM grows tremendously as a result, \EMfM exploits the information gained during its execution to reduce redundant exploration of the solution space.
There is redundancy in considering the same or similar pushing actions for an object in different nodes of \EMfM search tree such that if one action succeeds or fails (i.e. its validity is determined by forward simulating it for interaction constraint verification), it is likely that the other actions will succeed or fail as well.
We exploit this by introducing caching of positive and negative simulation results and learning a probabilistic estimate of solving a particular subtree of the search, and use these within \EMfM to bias its exploration.
% Like \MfM, in this paper we solve \MAMO problems with nonprehensile rearrangements while reasoning about complex and realistic multi-body interactions where objects lean, tilt and topple. \D{Include figure showing interesting interactions?}
We make the following contributions as part of our \EMfM algorithm:
% To address the higher computational complexity associated with searching for a solution in this much larger space, \EMfM stores information about all successful and unsuccessful pushes found during planning.
% The former are used to avoid simulations of similar pushes seen previously, and the latter help us feedback information to the \MAPF solver to efficiently search the space of rearrangements of the scene.
\begin{enumerate}
    \item A best-first graph search for \MAMO problems that searches over orderings of object rearrangements, different rearrangements of the scene, and different ways to rearrange each object.
    \item Caching results of successful (valid) pushes to avoid simulations of similar pushes repeatedly.
    \item Caching results of unsuccessful (invalid) pushes to feedback information to the \MAPF solver to efficiently search the space of rearrangements of the scene.
    \item A learned probabilistic model for solving a particular subtree to bias exploration of the best-first search.
    \item Significant quantitative improvements over \MfM and several other state-of-the-art \MAMO baselines.
\end{enumerate}

% Traditional planning algorithms~\cite{CohenCL10,RRT,ZuckerRDPKDBS13}, developed with the purpose of avoiding collisions altogether, will inefficiently reason about the necessary reconfiguration of the environment in the problems of interest to us in this paper.
% An exemplar of such problems is shown in Figure~\ref{fig:intro} where the robot, tasked with retrieving \D{OBJECT 1}, necessarily needs to move \D{OBJECT 2} and \D{OBJECT 3} aside to do so.
% This is because solutions to these ``Manipulation Among Movable Objects'' \raisebox{0.5pt}{(}\MAMO{\raisebox{0.5pt}{)}} problems~\cite{alami,StilmanMAMO} lie in a composite space that includes the configurations of the robot and all `movable' objects in the environment.
% Given that this space grows exponentially with the number of movable objects, conventional approaches for robot motion planning will suffer from the curse of dimensionality in their search for a solution~\cite{HauserL10}.
% Since the goal for \MAMO problems is defined only with respect to the target object (hereafter referred to as the object-of-interest, OoI), planning algorithms must reason about \emph{which} objects should be moved~\cite{Hauser14}, \emph{where} to move them, and \emph{how} they can be moved.
% % The answers to the these questions often depend on the capabilities of the particular robot and the ways in which it can interact with the environment.


%%%%%%%%%%%%%%%%%%%%%%%%%%%%%%%%%%%%%%%%%%%%%%%%%%%%%%%%%%%%%%%%%%%%%%%%%%%%%%%%
%%%%%%%%%%%%%%%%%%%%%%%%%%%%%%%%%%%%%%%%%%%%%%%%%%%%%%%%%%%%%%%%%%%%%%%%%%%%%%%%

\section{Related Work}

% \citet{Wilfong91} showed that \MAMO problems which leave goal poses for obstructing movable objects unspecified are NP-hard to solve, while the related problem of rearrangement planning~\cite{Ben-ShaharR98,Ota09} which specifies goal poses for all movable objects is PSPACE-hard.
% \citet{LatombePlanning} provides an excellent review of early work in these domains which were limited to planar environments and geometric solutions.

In recent years, \MAMO planning algorithms have continued to rely on at least one of two simplifying assumptions.
The first limits the action space of the robot to prehensile or pick-and-place rearrangements of movable objects~\cite{StilmanMAMO,KrontirisSDKB14,KrontirisB15,LeeCNPK19,NamLCCK20,ShomeB21,wang2022lazy}.
This simplifies the planning problem as grasped objects behave as rigid bodies attached to the robot end effector, and rearrangement paths can be computed by avoiding collisions with other objects in the scene.
It is important to note that these paths can only be found if we assume (i) all objects that may need to be rearranged are graspable by the robot, (ii) we have known grasp poses for all graspable objects, (iii) the existence of stable placement locations for these objects in the cluttered workspace, and (iv) a relatively large volume of object-free space so that collision- and contact-free rearrangement paths with a grasped object exist.
In our work, we make none of these assumptions, instead relying on nonprehensile pushing actions to rearrange the scene.
This allows us to manipulate a much larger set of objects, and also lets us rearrange multiple objects simultaneously.
However, using these actions within a planning algorithm necessitates the ability to accurately predict their effect on the configurations of movable objects in order to compute the resultant state of the world after the push.
% In case the assumptions stated earlier are indeed true for the problems being solved, the \EMfM algorithm can be easily modified to include pick-and-place rearrangement actions as part of the action space of the robot within its graph search, but we leave this extension as an open direction for future work.

The second simplification assumes planar robot-object and object-object interactions, while allowing nonprehensile push actions for rearrangement.
% ~\cite{BergSKLM08,DogarS12,King-2016-104000,HuangHYB22,gp_nonprehensile,homology_nonprehensile}
This planar assumption halves the size of the configuration space of movable objects from $SE(3)$ to $SE(2)$.
To predict the effect of push actions on the scene, some existing algorithms make use of simple analytical or learned physics models~\cite{BergSKLM08,DogarS12,HuangHYB22}, while others use computationally cheap 2D physics simulators~\cite{King-2016-104000,Huang}.
Assuming planar interactions does not capture the complex multi-body physics of the 3D real-world where objects may tilt, lean on each other, topple etc., something we account for in the \EMfM algorithm.
As part of our experiments, we compare against an implementation of the algorithm from~\cite{DogarS12} that uses the same nonprehensile push planner as \EMfM in conjunction with a 3D rigid body physics simulator.
The original algorithm is not viable for our \MAMO problems as it uses a 2D analytical model to predict the result of planar robot-object interactions, and only allows the robot to rearrange one object at a time.

Existing work which uses a full 3D rigid body physics simulator to forward simulate the effect of robot actions on the scene does not account for the difficult interaction constraints we include in our \MAMO problems which makes it harder to find a feasible solution.
Instead, they either only deal with simple constraints~\cite{SelSim,SPAMP} where objects are not allowed to fall off the workspace shelf, or do not include any such constraints~\cite{gp_nonprehensile,homology_nonprehensile} and ignore cases where objects topple.
We include a comparison against~\cite{SelSim} in our experiments, albeit with the full interaction constraint set that \EMfM considers.

We also include comparisons against two general purpose sampling-based planning algorithms, \KPIECE~\cite{KPIECE} and RRT~\cite{RRT}, and our own recent \MfM algorithm developed for this \MAMO domain.
\KPIECE is a randomised algorithm developed for planning problems where it is expensive to determine the resultant state of an action (like querying a physics-based simulator for the effect of robot pushes).
\MfM, as discussed earlier, decouples the search for a solution to the \MAMO problem between solving an abstract \MAPF problem that reasons about the configuration of movable objects but does not require forward simulating a simulation-based model, and solving a simulation-based arm motion planning problem that does not need to search over the possible configurations of movable objects.

%%%%%%%%%%%%%%%%%%%%%%%%%%%%%%%%%%%%%%%%%%%%%%%%%%%%%%%%%%%%%%%%%%%%%%%%%%%%%%%%
%%%%%%%%%%%%%%%%%%%%%%%%%%%%%%%%%%%%%%%%%%%%%%%%%%%%%%%%%%%%%%%%%%%%%%%%%%%%%%%%

\section{Problem Formulation}

We are interested in solving \MAMO problems with a $q$ degrees-of-freedom robot manipulator $\R$ whose configuration space $\X_\R \subset \mathbb{R}^q$.
The workspace is populated with objects $\Obs = \{O_1, \ldots, O_n\}$ whose configuration spaces $\X_{O_k} \equiv SE(3)$.
We assume we know which objects $\OM \subset \Obs$ are \emph{movable} and which objects $\OI \subset \Obs$ are \emph{immovable}.
Each object $O_k$ is associated with \emph{interaction constraints} described earlier that help determine whether any state $x$ in the search space $\X \coloneqq \X_\R \times \X_{O_1} \times \cdots \times \X_{O_n}$ is valid or not.
% These constraints encode properties of objects such as neither the robot $\R$ nor any movable object $O_m \in \OM$ is allowed to make contact with an immovable obstacle $O_i \in \OI$, and how far movable objects $O_m \in \OM$ are allowed to tilt or how fast they may be moved by $\R$.
The planning algorithm is provided the initial configurations of all movable objects (denoted as $\OM^{\text{init}}$) and immovable obstacles ($\OI$), information about which object is the ``object-of-interest'' (OoI), desired grasp pose for the OoI, and a ``home'' configuration outside the workspace shelf where the OoI must be moved.
Our goal is to find a path of valid states in $\X$ that successfully retrieves the OoI from the cluttered and confined workspace shelf.

% \begin{figure}[t]
%     \centering
%     \includegraphics[width=\columnwidth]{figures/intro_soln.eps}
%     \caption{\textbf{(a)}~The robot must retrieve the OoI (yellow) in the presence of movable (blue) and immovable (red) clutter. \textbf{(b)}~It does so by pushing the movable objects aside before \textbf{(c)}~grasping and extracting the OoI.}
%     \label{fig:eg_solution}
% \end{figure}

% Conceptually this path rearranges some movable objects if necessary and retrieves the OoI while ensuring (i) neither the robot nor any movable object makes contact with immovable `obstacles', and (ii) movable objects are rearranged satisfactorily (without making them topple, fall off the workspace shelf, move very fast etc.).
% Figure~\ref{fig:eg_solution} shows an example of such a path in simulation.
We make no assumptions about the \MAMO problem being \emph{monotone} where each movable object may only be moved once, and we allow the robot to rearrange several movable objects at the same time.
We do assume access to a rigid body physics simulator to evaluate the effect of robot actions on the states of the objects in the workspace.
% While our algorithm only uses nonprehensile push actions for movable object rearrangement, as we discuss later, this can be modified to include prehensile pick-and-place rearrangement actions if we had access to known grasp poses and stable placement poses for the movable objects.

%%%%%%%%%%%%%%%%%%%%%%%%%%%%%%%%%%%%%%%%%%%%%%%%%%%%%%%%%%%%%%%%%%%%%%%%%%%%%%%%
%%%%%%%%%%%%%%%%%%%%%%%%%%%%%%%%%%%%%%%%%%%%%%%%%%%%%%%%%%%%%%%%%%%%%%%%%%%%%%%%

\section{\EMfM}

This paper presents the \EMfM algorithm, an enhanced version of our previous \MfM algorithm.
In this section we provide details about \EMfM, the \MAPF abstraction and nonprehensile push planner used within it, and discuss when and why \EMfM will solve a \MAMO problem (or not).

In order to solve the \MAMO problems of interest to us, \EMfM must answer questions about \emph{which} objects should be moved, \emph{where} they may be moved, and \emph{how} the robot can move them.
Like \MfM, it relies on two modules to answer these questions -- an \MAPF solver~\cite{SharonSFS15} is used to answer the first two questions, while our nonprehensile push planner uses the \MAPF solution to try and answer the third.
Unlike \MfM however, \EMfM runs a best-first search over a graph $G = (V, E)$ with the help of these two modules.
% The push planner also updates the feasible solution space for the \MAPF solver if it determines that certain rearrangements are infeasible due to interaction constraint violations.
The vertices $v \in V$ represent a set of configurations (alternatively a \emph{rearrangement}) of the movable objects $\OM$ in the scene.
% Thus $v = \{ q_{O_k} : O_k \in \OM \}$ with $q_{O_k} \in \X_{O_k}$.
% Since the configurations of immovable obstacles are known prior to planning and cannot be changed by virtue of the definition of interaction constraints, we do not explicitly store them in each vertex $v$.
Edges $e = (u, v) \in E$ represent a successful rearrangement action changing the configuration of \emph{at least} one $O_m \in \OM$ between $u$ and $v$.
% If a rearrangement action from vertex $v$ is unsuccessful due to any interaction constraint violation or reachability constraint, \EMfM will use this information to generate a different rearrangement action from $v$ at a later point in the search.
% Thus given enough time, for any vertex $v \in V$, \EMfM will evaluate all possible rearrangement actions for all objects $O_m \in \OM$.
The overall \EMfM search expands vertices in an order dictated by some priority function $f : V \rightarrow \mathbb{R}_{\geq 0}$.
In contrast \MfM (i) greedily commits to the first valid push found (it does not search over orderings of rearrangements of multiple movable objects like \EMfM), (ii) only obtains a single \MAPF solution for each rearrangement it sees (it never replans the \MAPF solution based on the result of push actions like \EMfM), and consequently (iii) only tries to rearrange a movable object along a single \MAPF solution path for each rearrangement (it does not consider alternate ways to push an object for the same rearrangement like \EMfM).

%%%%%%%%%%%%%%%%%%%%%%%%%%%%%%%%%%%%%%%%%%%%%%%%%%%%%%%%%%%%%%%%%%%%%%%%%%%%%%%%

\subsection{Main Algorithm}

Algorithm~\ref{alg:emfm} contains the pseudocode for \EMfM.
Initially, \EMfM computes a trajectory $\gamma_\text{OoI} \subset \X_\R$ for the robot to grasp and extract the OoI while pretending no movable objects $\OM$ exist in the scene (Line~\ref{line:first_traj}).
The argument for the $\textsc{PlanRetrieval}$ function is the set of objects to be considered as obstacles during planning.
The volume occupied by the robot arm and OoI during execution of $\gamma_\text{OoI}$, written as $\mathcal{V}(\gamma_\text{OoI})$, creates a \emph{negative goal region} (NGR)~\cite{DogarS12}.
We define an NGR parameterised with a robot trajectory as some volume in the workspace that, if free of all objects, will lead to successful retrieval of the OoI upon execution of that robot trajectory.
Note that if the trajectory $\gamma_\text{OoI}$ cannot be found, the overall \MAMO problem as specified is unsolvable.
It may be solvable given a different grasp pose for the OoI, however grasp planning is beyond the scope of this work.
Once the initial NGR $\mathcal{V}(\gamma_\text{OoI})$ has been computed (Line~\ref{line:ngr}), \EMfM executes a best-first search using a priority queue ordered by $f$.
% The queue stores vertices $v \in V$ that represent rearrangements of the workspace.
% Each vertex/rearrangement $v$ is also associated with a set of constraints $\kappa$ that include information about known infeasible successor rearrangements from previous expanions of $v$.
% For example, Fig~\ref{fig:em4m_constraints}

\begin{algorithm}[!t]
\begin{small}
\caption{\EMfM{}}\label{alg:emfm}
% \textbf{Input: } initial movable object configurations $\OM^{\text{init}}$, immovable obstacle configurations $\OI$, object-of-interest OoI, ``home'' configuration, priority function $f$ \\
% \textbf{Output: } Sequence of arm trajectories to rearrange movable objects and retrieve OoI
\begin{algorithmic}[1]
\Procedure{CreateVertex}{$\OM, v, \gamma$}
    \State $v^\prime.\OM \gets \OM, v^\prime.parent \gets v, v^\prime.\gamma \gets \gamma$
    \Return $v^\prime$
\EndProcedure

\Procedure{Done}{$v$}
    % \If{$v.\{\pi_m\}_{m \in \OM}$ exists}
    %     \If{$v.\pi_m = \emptyset \,\forall\,m \in \OM$}
    %         \Return true \label{line:mapf_done}
    %     \EndIf
    % \EndIf
    \If{$v.\OM \cap \mathcal{V}(\gamma_\text{OoI}) = \emptyset$}
        \Return true \label{line:mapf_done}
    \EndIf
    \State $\hat{\gamma}_\text{OoI} \gets \textsc{PlanRetrieval}(v.\OM \cup \OI)$
    \If{$\hat{\gamma}_\text{OoI}$ exists}
        \State $\gamma_\text{OoI} \gets \hat{\gamma}_\text{OoI}$
        \Return true \label{line:new_retrieval}
    \EndIf
    \Return false
\EndProcedure

% \Procedure{ExtractRearrangements}{$v$}
%     \State $\Gamma \gets \{\gamma_\text{OoI}\}$
%     \While{$v.parent \neq \emptyset$}
%         \State $\textsc{PushFront}(\Gamma, v.\gamma)$
%         \State $v \gets v.parent$
%     \EndWhile
%     \Return $\Gamma$
% \EndProcedure

\Procedure{ExpandState}{$v$}
    \State $\kappa \gets \textsc{InvalidGoals}(v)$ \label{line:invalid_goals}
    \State $v.\{\pi_m\}_{m \in \OM} \gets \textsc{Run}\MAPF(v, \kappa, \mathcal{V}(\gamma_\text{OoI}))$ \label{line:call_mapf}
    \If{\MAPF failed}
        \State Remove $v$ from $OPEN$ \label{line:close_vertex}
        \Return
    \EndIf
    \For{$m \in v.\OM$} \label{line:get_succs_begin}
        \If{$v.\pi_m \neq \emptyset$}
            \State $\gamma_m \gets \textsc{PlanPush}(v.\pi_m, v.\OM)$ \label{line:plan_push}
            \State $\OM^\prime, \texttt{valid} \gets \textsc{IsValid}(\gamma_m)$ \label{line:validate_push}
            \If{\texttt{valid}}
                \State $v^\prime \gets \textsc{CreateVertex}(\OM^\prime, v, \gamma_m)$ \label{line:create_succ}
                \State Insert $v^\prime$ into $OPEN$ with priority $f(v^\prime)$ \label{line:insert_succ}
            \Else
                \State Add final state in $v.\pi_m$ to $\textsc{InvalidGoals}(v)$ \label{line:add_invalid_goal}
            \EndIf
        \EndIf
    \EndFor
\EndProcedure

\Procedure{Main}{$\OM^{\text{init}}, \OI$}
    \State $\gamma_\text{OoI} \gets \textsc{PlanRetrieval}(\OI)$ \label{line:first_traj}
    \State Compute $\mathcal{V}(\gamma_\text{OoI})$ \label{line:ngr}
    \State $OPEN \gets \emptyset$, $v_\text{start} \gets \textsc{CreateVertex}(\OM^{\text{init}}, \emptyset, \emptyset)$
    \State Insert $v_\text{start}$ into $OPEN$ with priority $f(v_\text{start})$
    \While{$OPEN$ is not empty \textbf{and} time remains}
        \State $v \gets OPEN.\textsc{top}()$
        \If{\textsc{Done}$(v)$}
            \Return $\textsc{ExtractRearrangements}(v)$
        \EndIf
        \State $\textsc{ExpandState}(v)$ \label{line:expand_state}
    \EndWhile
    \Return $\emptyset$
\EndProcedure

\end{algorithmic}
\end{small}
\end{algorithm}

Every time a vertex $v$ is expanded from this queue during the search (Line~\ref{line:expand_state}), \EMfM calls an \MAPF solver (Line~\ref{line:call_mapf}). % whose solution informs the search about which objects need to move and where.
% The \MAPF solver pretends that all movable objects are artificially actuated robots tasked with the goal of exiting the NGR $\mathcal{V}(\gamma_\text{OoI})$ while avoiding collisions with each other and all immovable obstacles.
The set of solution paths returned by the \MAPF solver is then used by our nonprehensile push planner to generate and evaluate successor rearrangement states (the loop from Line~\ref{line:get_succs_begin}).
For each object $O_m$ that ``moves'' in the \MAPF solution, our push planner tries to compute a trajectory $\gamma_m \subset \X_\R$ to push $O_m$ along its \MAPF solution path $\pi_m$ (Line~\ref{line:plan_push}).
If $\gamma_m$ is found, it is forward simulated in a rigid body physics simulator for interaction constraint verification (Line~\ref{line:validate_push}).
If $\gamma_m$ successfully rearranges the scene, i.e. at least one object is moved and no constraints are violated, \EMfM generates a successor state $v^\prime$ with the resultant rearrangement and adds it to the queue (Lines~\ref{line:create_succ} and~\ref{line:insert_succ}).
% In case $\gamma_m$ cannot be found or violates constraints in simulation, \EMfM includes the last state in $\pi_m$ as an invalid goal location for $O_m$ when the \MAPF solver is called again during a re-expansion of vertex $v$ (\D{LINE ?}).
% In case no invalid pushes were found while expanding $v$, upon re-expansion of $v$ at a later time during the search, \EMfM will ``hallucinate'' the last state of a path $\pi_m$ for $O_m$ that lead to a valid push $\gamma_m$ as an invalid goal location for $O_m$ when the \MAPF solver is called.
% Thus vertices may be re-expanded by \EMfM until all possible rearrangement actions for all objects $O_m \in \OM$ have been considered.

A vertex $v$ is \emph{closed} in Line~\ref{line:close_vertex} and never re-expanded again if the \MAPF solver fails to return a solution in Line~\ref{line:call_mapf}.
Otherwise when a vertex $v$ is re-expanded, we ensure that the \MAPF solver returns a different solution than one obtained during any previous expansion of $v$ by including a set $\kappa$ of invalid goals (Line~\ref{line:invalid_goals}).
$\kappa$ contains configurations for each movable object $O_m \in v.\OM$ that \emph{cannot} be the final state in the path $\pi_m$ found by the \MAPF solver.
This helps \EMfM search over different \MAPF solutions for the same rearrangement $v.\OM$, thereby helping it search over different ways to rearrange $v.\OM$.
% Invalid goals are populated in $\kappa$ in two ways -- all push actions $\gamma_m$ found to be invalid lead to the final state of the corresponding $\pi_m$ being included as an invalid goal for $O_m$ (Line~\ref{line:add_invalid_goal}); additionally if no such invalid pushes remain to be added to $\kappa$, the final states of valid pushes found previously from $v$ are ``hallucinated'' as invalid goals.

\EMfM terminates with $v$ as the goal state when the OoI can be successfully retrieved given the rearrangement $v.\OM$.
This may be achieved in one of two ways.
If the movable objects $\OM$ in $v$ have been successfully rearranged to be outside the initial NGR $\mathcal{V}(\gamma_\text{OoI})$, we know the robot can execute $\gamma_\text{OoI}$ to retrieve the OoI (Line~\ref{line:mapf_done}).
Alternatively, if a different trajectory $\hat{\gamma}_\text{OoI}$ (and therefore its NGR $\mathcal{V}(\hat{\gamma}_\text{OoI})$) can be found in the presence of all objects (movable and immovable) as obstacles, the robot can execute $\hat{\gamma}_\text{OoI}$ to retrieve the OoI without making contact with any other object (Line~\ref{line:new_retrieval}).
% Thus \EMfM implicitly makes the assumption that the OoI retrieval trajectory will only make contact with the OoI.

%%%%%%%%%%%%%%%%%%%%%%%%%%%%%%%%%%%%%%%%%%%%%%%%%%%%%%%%%%%%%%%%%%%%%%%%%%%%%%%%

\subsection{\MAPF Abstraction}

We make the observation that by solving a carefully constructed \MAPF problem with movable objects as agents, we can obtain information about which objects our \MAMO planner should consider rearranging and where.
We use Conflict-Based Search (CBS)~\cite{SharonSFS15} as the solver for our abstract \MAPF problem. % since it is complete with respect to the graphs it searches for single-agent paths.
For a vertex $v$ in \EMfM, we include all movable objects as agents in CBS starting at their current poses in $v.\OM$.
Each agent is assigned a goal of being outside the initial NGR $\mathcal{V}(\gamma_\text{OoI})$, while avoiding collisions with each other and all immovable obstacles $\OI$.
Although agent states in CBS specify their configuration in $\X_{O_m} \equiv SE(3)$, agents use a 2D action space on a four-connected grid in the \MAPF abstraction that only changes the $x-$ or $y-$coordinates of their state.
% Since these agents are artificially actuated for the sake of the \MAPF abstraction, we do not assume nor do we care about any dynamics model for them.
% This allows us to use simple gridworld \MAPF solvers like CBS, instead of more complex multi-robot motion planning algorithms~\cite{shome2020drrt}.
The CBS solution for this \MAPF problem is a set of paths $\{\pi_m\}_{m \in \OM}$ such that $O_m$ ends up outside $\mathcal{V}(\gamma_\text{OoI})$ after following $\pi_m$.
% Our nonprehensile push planner then considers each $\pi_m$ in turn to try and rearrange the movable objects and generate successor rearrangement states in \EMfM.
Figure~\ref{fig:mapf_solution} shows a simulated \MAMO problem, the initial NGR $\mathcal{V}(\gamma_\text{OoI})$ for the scene, and a 2D projection of the scene which shows the \MAPF solution found.

\begin{figure}[t]
    \centering
    \includegraphics[width=\columnwidth]{figures/100011-ngr.pdf}
    \caption{(a)~A \MAMO problem with five movable objects and four immovable obstacles, (b)~the initial NGR $\mathcal{V}(\gamma_\text{OoI})$ found for this scene (in gray), and (c)~a 2D projection of the scene with the \MAPF solution paths in pink. This \MAPF solution suggests that objects $A$ and $B$ should be rearranged as per the pink paths to be outside $\mathcal{V}(\gamma_\text{OoI})$.}
    \label{fig:mapf_solution}
\end{figure}

In order to search over all possible ways to rearrange $v.\OM$, \EMfM includes a set of invalid goals $\kappa$ when it calls CBS in Line~\ref{line:call_mapf}.
For each vertex $v$, $\kappa$ includes information about where each object $O_m \in v.\OM$ \emph{cannot} end up in any future CBS solution.
During a previous expansion of $v$, if path $\pi_m$ led to an invalid push $\gamma_m$, the final state of $\pi_m$ is added as an invalid goal for $O_m \in v.\OM$ since we failed to rearrange $O_m$ along $\pi_m$ (Line~\ref{line:add_invalid_goal}).
As a result, the next solution from CBS would return a new path $\pi^\prime_m \neq \pi_m$ which in turn would cause our push planner to consider a different rearrangement action.
When all invalid pushes from $v$ have been included in the previous call to CBS, we also include the final states of valid pushes found previously as invalid goals in $\kappa$ so as to ensure we consider all possible ways to rearrange $v.\OM$.
If CBS fails to find a solution, or if no new invalid goals can be added to $\kappa$ from the last call to CBS, we \emph{close} vertex $v$ and stop it from being re-expanded (Line~\ref{line:close_vertex}) since no new way to rearrange $v.\OM$ can be found.
Figure~\ref{fig:mapf_constraints} shows the effect of invalid goals $\kappa$ on the \MAPF solution -- when both pushes $\gamma_A$ and $\gamma_B$ computed as per the \MAPF solution in (a) failed, adding the final states of $\pi_A$ and $\pi_B$ to $\kappa$ results in a different \MAPF solution in (b).

\begin{figure}[t]
    \centering
    \includegraphics[width=0.8\columnwidth]{figures/mapf_constraints.pdf}
    \caption{(a)~First \MAPF solution that led to invalid pushes $\gamma_A$ and $\gamma_B$, (b)~Adding the final states of $\pi_A$ and $\pi_B$ (colour coded stars) to $\kappa$ leads to a new \MAPF solution.}
    \label{fig:mapf_constraints}
\end{figure}

%%%%%%%%%%%%%%%%%%%%%%%%%%%%%%%%%%%%%%%%%%%%%%%%%%%%%%%%%%%%%%%%%%%%%%%%%%%%%%%%

\subsection{Nonprehensile Push Planner}

The goal for our push planner is to find robot trajectories $\gamma_m \subset \X_\R$ that rearrange a movable object along the path $\pi_m$ returned by our \MAPF solver.
If we are able to precisely rearrange each $O_m$ to the final state of $\pi_m$, we know the robot can execute $\gamma_\text{OoI}$ to solve the \MAMO problem.
In order to do so, we assume the push planner is provided a shortcut path $\pi_m$ (accounting for immovable obstacles $\OI$) in Line~\ref{line:plan_push}.

\begin{figure}[t]
    \centering
    \includegraphics[width=0.8\columnwidth]{figures/push_planner.pdf}
    \caption{2D illustration of our push planner. Movable object $O_m$ is blue, and an immovable obstacle is drawn in red. The green path is obtained after shortcutting the pink \MAPF solution path $\pi_m$. $x^{aabb}$ is the point-of-intersection between the first segment $(x^1, x^2)$ and the axis-aligned bounding box for $O_m$. The cyan segments depict the path along which inverse kinematics is used to obtain $\gamma_m^i$.}
    \label{fig:push_planner}
\end{figure}

Each call to $\textsc{PlanPush}$ stochastically generates a robot trajectory $\gamma_m$, as shown in Figure~\ref{fig:push_planner}.
The starting location of the push, $x^0_{\text{push}}$, is sampled around the point of intersection $x^{aabb}$ between the first segment of $\pi_m$ and the axis-aligned bounding box of $O_m$.
If the push planner finds an approach path $\gamma_m^0 \subset \X_\R$ to $x^0_{\text{push}}$ in the presence of all objects (movable and immovable) as obstacles, it samples waypoints $x^i_{\text{push}}$ around the corresponding states $x^i$ in $\pi_m$.
It uses inverse kinematics to find segments of the push $\gamma_m^i$ between the waypoints $x^{i-1}_{\text{push}}$ and $x^i_{\text{push}}$.
If all $\gamma_m^i$ are found, the planner returns the final push trajectory as a concatenation of the individual pieces.
In this case, the push $\gamma_m$ is forward simulated in a rigid body physics simulator to detect if any interaction constraints are violated during its execution and get the resultant rearrangement of the scene.

% By ensuring that we retract the robot arm to the final configuration in $\gamma_m^0$ after the push, we can guarantee that the robot can move from one push to the next entirely in free space without making contact with any object.

% %%%%%%%%%%%%%%%%%%%%%%%%%%%%%%%%%%%%%%%%%%%%%%%%%%%%%%%%%%%%%%%%%%%%%%%%%%%%%%%%

% \subsection{Relation to \MfM}

%%%%%%%%%%%%%%%%%%%%%%%%%%%%%%%%%%%%%%%%%%%%%%%%%%%%%%%%%%%%%%%%%%%%%%%%%%%%%%%%

\subsection{What \EMfM Can and Cannot Solve}

Solutions to \MAMO problems lie in a space $\X = \X_\R \times \X_{O_1} \times \cdots \times \X_{O_n}$ that grows exponentially with the number of movable objects.
There are several subtle reasons due to which \EMfM might fail to find solutions to complicated \MAMO problems in this space.
% A rigorous theoretical analysis of \EMfM requires careful consideration of three different graphs that it searches -- the high-level graph $G$ with vertices that denote rearrangements of movable objects and edges that encode rearrangement actions from our nonprehensile push planner, the graphs with vertices in $SE(3)$ with edges corresponding to a 2D action space on a four-connected grid used by the single-agent searches within our \MAPF solver CBS, and finally a graph with vertices in $\X_\R$ and edges representing changes in one of the $q$ degrees-of-freedom that is used when computing any robot trajectory (within the $\textsc{PlanRetrieval}$ and $\textsc{PlanPush}$ functions in Algorithm~\ref{alg:emfm}).
When trying to rearrange an object to a specific location, \EMfM only considers moving it along the particular path $\pi_m$ found by CBS and not along all such paths.
% \EMfM does not compute all ways to push an object to a particular location, i.e. it does not compute all paths (and by extension pushes) that end in the same final state, instead only computing a push $\gamma_m$ for the particular $\pi_m$ found by CBS.
Its reliance on CBS and our push planner also means that \EMfM might fail to find interesting non-monotone solutions where it must rearrange an object $O_a$ partially along its solution path $\pi_a$ before moving $O_b$ along $\pi_b$ and finally going back to move $O_a$ the remainder of the way along $\pi_a$.
The \MAPF abstraction itself uses a simple 2D action space which fails to capture all possible rearrangements of the scene in $\X_{O_1} \times \cdots \times \X_{O_n}$.
Finally, \EMfM does not actively search over all OoI retrieval trajectories $\gamma_\text{OoI} \subset \X_\R$, and consequently the same NGR is used to specify goals for movable objects in all \MAPF calls.

Despite these limitations, by relying on our \MAPF abstraction and nonprehensile push planner, \EMfM does search over `allowed' (i) orderings of movable object rearrangements, (ii) potential rearrangements for the set of movable objects, and (iii) ways to rearrange the same movable object.
In doing so it makes progress towards a complete \MAMO planning algorithm that uses nonprehensile actions for rearrangement in a 3D workspace with complex multi-body interactions where movable objects may tilt, lean, topple etc.
Our quantitative analysis shows that \EMfM performs better than many state-of-the-art planning algorithms for these \MAMO problems.
% Figure~\ref{fig:100011_em4m_soln} shows the solution found by \EMfM for the \MAMO problem from Figure~\ref{fig:mapf_solution} -- the robot pushes two objects before retrieving the OoI.

% \begin{figure}[t]
%     \centering
%     \includegraphics[width=\columnwidth]{figures/100011_em4m_soln.eps}
%     \caption{A sequence of images showing the solution found by \EMfM to a \MAMO problem with five movable objects and four immovable obstacles.}
%     \label{fig:100011_em4m_soln}
% \end{figure}


%%%%%%%%%%%%%%%%%%%%%%%%%%%%%%%%%%%%%%%%%%%%%%%%%%%%%%%%%%%%%%%%%%%%%%%%%%%%%%%%
%%%%%%%%%%%%%%%%%%%%%%%%%%%%%%%%%%%%%%%%%%%%%%%%%%%%%%%%%%%%%%%%%%%%%%%%%%%%%%%%

\section{Speeding up the Algorithm}

This section discusses three algorithmic optimisations we propose as part of \EMfM that significantly improve its quantitative performace.
% We discuss caching information from successful and unsuccesful rearrangement actions to speed up the \EMfM search, and a data-driven priority function that learns a naive probabilistic estimate of the subtree at a vertex $v$ in the \EMfM graph being solvable based on features of the rearrangement $v.\OM$.

%%%%%%%%%%%%%%%%%%%%%%%%%%%%%%%%%%%%%%%%%%%%%%%%%%%%%%%%%%%%%%%%%%%%%%%%%%%%%%%%

\subsection{Caching Unsuccessful Push Actions}

The goal set for movable objects in the \MAPF abstraction includes any configuration outside the initial NGR $\mathcal{V}(\gamma_\text{OoI})$ that is free of collision from all other objects.
Given vertex $v$, consider an object $O_m$ in $v.\OM$ and its corresponding path $\pi_m$ which achieves this goal.
If the resulting push $\gamma_m$ is invalid, the next call to CBS from $v$ will lead to a path $\pi_m^\prime \neq \pi_m$.
However, naively including the last state in $\pi_m$ as an invalid goal state for CBS will likely lead to the new path $\pi_m^\prime$ ending in a neighbouring state of the invalid goal (since CBS is an optimal \MAPF solver).
This in turn will lead to a push $\gamma_m^\prime \approx \gamma_m$ that is also likely to be invalid.

To mitigate this, for every CBS call from a vertex $v$, for each object $O_m$, \EMfM caches the goals that were previously determined to be invalid in a nearest neighbour data structure.
% It goes on to modify the single-agent search in CBS to one towards a `pseudogoal' connected by `pseudoedges' to all states in the abstract \MAPF goal set.
% The cost of these pseudoedges is determined by a lookup into the nearest neighbour data structure that stores invalid goals for objects $O_m$.
We use this cached information to bias the solutions produced by CBS to avoid moving objects to states close to known invalid goals for the respective objects.
This biasing is done by assigning penalties to each potential goal location for each object $O_m$, and then during each low-level search within CBS finding the solution that minimises the summation of getting to a goal plus the penalty associated with the goal.
This can be done by introducing one single `pseudogoal' that the search searches towards and connecting all the potential goal locations to this `pseudogoal' with edges whose cost is proportional to the respective penalty.
This helps penalise paths to states close to known invalid goals, and lets \EMfM search the space of allowed rearrangements of $v.\OM$ more efficiently.
Figure~\ref{fig:neg_db_effect} (a) shows the new \MAPF solution when we include invalid goals naively -- the new paths $\pi_A^\prime$ and $\pi_B^\prime$ are very similar to $\pi_A$ and $\pi_B$ (from Figure~\ref{fig:mapf_constraints} (a)) and end in final states very close to the invalidated goals.
However, using the above explained approach that modifies the single-agent search to use our nearest neighbour data structure for invalid goals, we get a very different \MAPF solution in Figure~\ref{fig:neg_db_effect} (b).
% This allows us to search the space of possible rearrangements much more efficiently.

\begin{figure}[t]
    \centering
    \includegraphics[width=0.8\columnwidth]{figures/neg_db_effect.pdf}
    \caption{(a)~Only adding the final states of paths that led to invalid pushes as invalid goals results in a new \MAPF solution very similar to the previous one (from Figure~\ref{fig:mapf_constraints} (a)), (b)~Using our proposed approach that penalises goal states near known invalid goals, \EMfM is able to find diverse \MAPF solutions quicker.}
    \label{fig:neg_db_effect}
\end{figure}

%%%%%%%%%%%%%%%%%%%%%%%%%%%%%%%%%%%%%%%%%%%%%%%%%%%%%%%%%%%%%%%%%%%%%%%%%%%%%%%%

\subsection{Caching Successful Push Actions}

During the search, CBS solutions in different vertices of the graph \EMfM may contain the same path $\pi_m$ for object $O_m$.
% This can occur if the best rearrangement for $O_m$ is unaffected by the different configurations of other objects in these vertices.
In such cases, if we have computed the push $\gamma_m$ in one of these vertices and found it to be valid in simulation while generating the successor rearrangement state, we would like to reuse this result whenever possible since simulating a push action is computationally expensive (around $2-6\SI{}{\second}$ per action simulation).
This reuse of successful push actions is enabled by storing the successful pushes in a database.

If a push $\gamma_m$ from path $\pi_m$ for object $O_m$ rearranged the scene from $v.\OM$ to $v^\prime.\OM$, we index into this database with the key $(O_m, \pi_m)$.
While any push is simulated, we keep track of objects that are $relevant$ for that push -- these are all objects whose configurations are changed between $v.\OM$ and $v^\prime.\OM$.
For each $(O_m, \pi_m)$ tuple, the database stores the value $(v.\OM, v^\prime.\OM, \gamma_m, relevant \text{ objects})$.
During the expansion of some other vertex $u$, if CBS returns the same path $\pi_m$ for $O_m$, we try to reuse the result of the stored push $\gamma_m$ to generate the successor state $u^\prime$ corresponding to rearranging $O_m$.
However, this reuse is only possible if all $relevant$ objects are in the same configurations in $v.\OM$ and $u.\OM$, and all other `irrelevant' objects in $u.\OM$ are in configurations that will not be affected by $\gamma_m$.
If both these conditions are true, we can simply reuse the result of $\gamma_m$ stored in the database to say that the $relevant$ objects in $u.\OM$ will be rearranged to their repective configurations $v^\prime.\OM$, and the `irrelevant' objects will remain unaffected.


%%%%%%%%%%%%%%%%%%%%%%%%%%%%%%%%%%%%%%%%%%%%%%%%%%%%%%%%%%%%%%%%%%%%%%%%%%%%%%%%

\subsection{Learned Priority Function}

\EMfM searches an extremely large space for \MAMO solutions since it searches over orderings of object rearrangements, different rearrangements of the scene, and different ways to rearrange each object.
In an attempt to focus its search effort on more promising vertices of the search tree, we learn to predict the probability of a particular vertex leading to a solution for the \MAMO problem.
The learned function is used as the priority function $f$ in the best-first \EMfM search.
We predict the probability of a vertex $v$ leading to a solution based on features of the rearrangement $v.\OM$ -- the percentage volume of the initial NGR $\mathcal{V}(\gamma_\text{OoI})$ occupied by movable objects ($\phi_1$), the number of movable objects inside the NGR ($\phi_2$), and for each such object the product of its mass, coefficient of friction and percentage volume inside the NGR ($\phi_3$).
These features indicate how difficult it is to clear the NGR and therefore solve the \MAMO problem.
Our predictive model $f$ makes the Naive Bayes assumption~\cite{JohnL95} that these features are conditionally independent of the others.
Thus if $E$ is the event that vertex $v$ leads to a \MAMO solution,
\begin{align*}
    f(v) = P(E) \times P(\phi_1 \,\lvert\, E) &\times P(\phi_2 \,\lvert\, E) \\
    &\times \Pi_{O_m \in \mathcal{V}(\gamma_\text{OoI})} P(\phi_3 \,\lvert\, E)
\end{align*}

We model $P(\phi_1 \,\lvert\, E)$ as a beta distribution, and $P(\phi_2 \,\lvert\, E)$ and $P(\phi_3 \,\lvert\, E)$ are both exponential distributions.
Their parameters are estimated via maximum likelihood estimation from a dataset of self-supervised \MAMO problems.
We generate this set of problems by running \EMfM breadth-first on 20 \MAMO scenes with a $\SI{10}{\minute}$ planning timeout.
We store each vertex generated during these \EMfM runs as a separate \MAMO problem, and then run \EMfM depth-first with a $\SI{60}{\second}$ timeout on these problems to get our training dataset of $2400$ datapoints.

%%%%%%%%%%%%%%%%%%%%%%%%%%%%%%%%%%%%%%%%%%%%%%%%%%%%%%%%%%%%%%%%%%%%%%%%%%%%%%%%
%%%%%%%%%%%%%%%%%%%%%%%%%%%%%%%%%%%%%%%%%%%%%%%%%%%%%%%%%%%%%%%%%%%%%%%%%%%%%%%%

\section{Experimental Analysis}

This section compares the performance of \EMfM against several \MAMO baselines, studies the effect of the algorithmic improvements we propose in an ablation study, and provides results from real-world experiments on a PR2 robot.


%%%%%%%%%%%%%%%%%%%%%%%%%%%%%%%%%%%%%%%%%%%%%%%%%%%%%%%%%%%%%%%%%%%%%%%%%%%%%%%%

\subsection{Simulation Experiments Against \MAMO Baselines}

Our simulation experiments randomly generate \MAMO workspaces with one OoI, four immovable obstacles, and five, ten or fifteen movable objects.
All object properties (shapes, sizes, mass, coefficient of frictions, initial poses) are randomised and known to each planner prior to planning.
We categorise these workspaces into three difficulty levels based on the number of movable objects inside the initial NGR $\mathcal{V}(\gamma_\text{OoI})$.
Problems are \texttt{Easy}, \texttt{Medium}, or \texttt{Hard} depending on whether there are one, two, or more than two movable objects overlapping with the initial NGR.
Figure~\ref{fig:problem_difficulty} shows sample \MAMO workspaces of each difficulty level.
We test the performance of all algorithms on 98 \texttt{Easy}, 63 \texttt{Medium}, and 39 \texttt{Hard} problems with a $\SI{5}{\minute}$ planning timeout.

\begin{figure}[t]
    \centering
    \includegraphics[width=\columnwidth]{figures/problem_difficulty.pdf}
    \caption{Example \texttt{Easy}, \texttt{Medium}, and \texttt{Hard} scenes.}
    \label{fig:problem_difficulty}
\end{figure}

In addition to (1)~\MfM, we compare the performace of \EMfM against four other baselines.
(2)~We reimplement \textsc{Dogar}~\cite{DogarS12} to use our push planner with a physics-based simulator.
It recursively searches backwards in time for objects that need to be rearranged outside the most recent NGR.
If it rearranges an object successfully, the volume spanned by the rearrangement trajectory is added to the previous NGR and the recursion continues.
However, \textsc{Dogar} only has information about which objects need to be rearranged but not where they should be moved.
Our implementation randomly samples points outside the latest NGR as goal locations for our push planner.
(3)~\textsc{SelSim}~\cite{SelSim} interleaves planning a trajectory while simulating interactions with `relevant' objects with tracking the found trajectory in the presence of all objects.
If tracking violates interaction constraints, the `culprit' object is identified and added to the set of relevant objects for the next iteration.
\textsc{SelSim} uses simple motion primitives that change only one of the $q$ degrees-of-freedom of the robot, which does not lead to meaningful robot-object interactions in this domain.
Finally, we compare against the standard OMPL~\cite{OMPL} implementations of general-purpose sampling-based planning algorithms (4)~\textsc{RRT}~\cite{RRT} and (5)~\textsc{KPIECE}~\cite{KPIECE}.

\begin{table}[]
\centering
\begingroup
\setlength{\tabcolsep}{2pt}
\footnotesize
\begin{tabular}{@{}ccccccc@{}}
\toprule
\multirow{2}{*}{\textbf{Difficulty}}& \multicolumn{6}{c}{\textbf{Planning Algorithm}} \\
\cmidrule{2-7}
& \EMfM & \MfM & \textsc{Dogar} & \textsc{SelSim} & \textsc{RRT} & \textsc{KPIECE}\\ \midrule
\texttt{Easy} (98) & 97 & 78 & 7 & 16 & 33 & 16 \\
\texttt{Medium} (63) & 45 & 25 & 0 & 8 & 7 & 1 \\
\texttt{Hard} (39) & 15 & 7 & 0 & 1 & 1 & 0 \\ \bottomrule
\end{tabular}
\endgroup
\caption{Number of problems solved by various \MAMO planning algorithms in simulation experiments}
\label{tab:sim_exps}
\end{table}

Table~\ref{tab:sim_exps} contains the number of problems solved by each planning algorithm for the different difficulty levels.
It is apparent that \EMfM far outperforms all other algorithms, and that all baselines struggle to solve \texttt{Medium} and \texttt{Hard} problems.
The quantitative performance of all algorithms in terms of total planning time and time spent simulating robot actions is shown in Figure~\ref{fig:baselines_results}.
\EMfM is able to achieve a good balance of time spent computing robot actions (with the \MAPF solver and push planner) and forward simulating them for interaction constraint verification.
Since \MfM never replans the \MAPF solution, it spends most of its time trying to sample push actions to be simulated.
\textsc{Dogar} repeatedly fails to find solutions, even for simple problems, because it (i) has no information about where objects should move, choosing to randomly sample uninformed pushes instead, (ii) only considers pushes to be successful if they rearrange an object to be completely outside the NGR, and (iii) never considers rearranging an object more than once.
The performance of \textsc{SelSim} is particularly interesting.
It is only able to solve problems where the first planned path succeeds when tracked without any interaction constraints being violated, resulting in negligible planning times for its successes (and no time spent in simulation).
Otherwise, owing to its primitive action space, it spends most of its planning budget in simulation trying to rearrange the scene with small robot actions that are incapable of significantly changing object configurations.
\textsc{RRT} and \textsc{KPIECE} perform well as they are able to sample long robot motions that can rearrange objects with favourable physical properties (low masses and coefficients of friction) with high likelihood.
% The poor performance of \textsc{KPIECE} in comparison to \textsc{RRT} can be explained by the 3D projective space being inadequate for capturing information about progress to the goal (of retrieving the OoI) since it contains no notion of which objects still remain to be rearranged, and how difficult that might be.


\begin{figure}[t]
    \centering
    \includegraphics[width=0.88\columnwidth]{figures/baselines_results_2.pdf}
    \caption{(a)~Total planning time and (b)~time spent querying a physics-based simulator for \MAMO planning algorithms across planning problems with varying difficulty levels.}
    \label{fig:baselines_results}
\end{figure}


%%%%%%%%%%%%%%%%%%%%%%%%%%%%%%%%%%%%%%%%%%%%%%%%%%%%%%%%%%%%%%%%%%%%%%%%%%%%%%%%

\subsection{\EMfM Ablation Study}

% To study the effect of the algorithmic improvements we propose as part of \EMfM, we present results from an ablation study where we compare different versions of \EMfM.
To study the effect of the algorithmic improvements we propose as part of \EMfM, we compare four different versions of \EMfM.
% In addition to the proposed algorithm, we include four versions of \EMfM in this study.
\textsc{Neg-DB} only caches information from unsuccessful push actions; % to modify the low-level CBS single-agent search;
\textsc{Pos-DB} only caches information from successful push actions; % and tries to reuse their results whenever possible;
\textsc{No-DB} does not cache any information from push actions;
and \textsc{Tiebreak} assigns priorities by lexicographically tiebreaking \EMfM vertex feature vectors $(\phi_1, \phi_2, \sum_{O_m \in \mathcal{V}(\gamma_\text{OoI})} \phi_3)$.
\textsc{Neg-DB}, \textsc{Pos-DB}, and \textsc{No-DB} all use the learned priority function like \EMfM, while \textsc{Tiebreak} caches information from unsuccessful and successful push actions like \EMfM.

\begin{table}[]
\centering
\begingroup
\setlength{\tabcolsep}{2pt}
\footnotesize
\begin{tabular}{@{}cccccc@{}}
\toprule
\multirow{2}{*}{\textbf{Difficulty}}& \multicolumn{5}{c}{\textbf{Ablation}} \\
\cmidrule{2-6}
& \EMfM & \textsc{Neg-DB} & \textsc{Pos-DB} & \textsc{No-DB} & \textsc{Tiebreak}\\ \midrule
\texttt{Easy} (98) & 97 & 87 & 86 & 82 & 85 \\
\texttt{Medium} (63) & 45 & 24 & 29 & 25 & 36 \\
\texttt{Hard} (39) & 15 & 7 & 8 & 7 & 13 \\ \bottomrule
\end{tabular}
\endgroup
\caption{Number of problems solved \EMfM ablations}
\label{tab:ablation2}
\end{table}

Table~\ref{tab:sim_exps} shows that each of these \EMfM ablations solve fewer \MAMO problems in comparison to \EMfM which combines all of them.
Quantitatively, their performance can be compared from the plots in Figure~\ref{fig:ablation_study}.
While there is no significant difference between the different ablations for \texttt{Easy} problems, for \texttt{Medium} and \texttt{Hard} problems we can see that performance degrades as we remove cached information.
\textsc{Pos-DB} performs worse than \textsc{Neg-DB} since it is not as likely for \EMfM to find the same push multiple times during a search as it is for it to require several different \MAPF solutions.
Finally, even with a naive tiebreaking based priority function, \textsc{Tiebreak} performs only slightly worse than \EMfM for \texttt{Hard} problems.
This suggests that the learned priority function (using the Naive Bayes assumption) is not as useful for these problems, perhaps due to the $\SI{60}{\second}$ timeout imposed during data collection being insufficient to result in a rich set of datapoints for \texttt{Hard} problems.


\begin{figure}[t]
    \centering
    \includegraphics[width=0.9\columnwidth]{figures/ablation_study_2.png}
    \caption{Median time spent calling the \MAPF solver, push planner, and simulator for different \EMfM ablations.}
    \label{fig:ablation_study}
\end{figure}


%%%%%%%%%%%%%%%%%%%%%%%%%%%%%%%%%%%%%%%%%%%%%%%%%%%%%%%%%%%%%%%%%%%%%%%%%%%%%%%%

\subsection{Real-World Experiments}

We ran experiments with the PR2 robot with a refrigerator compartment as our \MAMO workspace (Figure~\ref{fig:intro_fridge}).
Problems were initialised with five objects from the YCB Object Dataset~\cite{YCB}.
The tomato soup can was always our OoI, while all other objects were initialised as movable.
% Their initial poses were localised with a search-based algorithm~\cite{Agarwal2020PERCH2} run on a NVidia Titan X GPU.

We ran \EMfM on 20 perturbations of the scenes in Figure~\ref{fig:intro_fridge} with a $\SI{5}{\minute}$ planning timeout.
15 runs resulted in successful OoI retrieval, with the others failing due to unforeseen discrepancies between simulated and real-world robot-object interactions.
Four failures were due to inaccurate computation of coefficients of friction of movable objects.
One failure was the result of an object getting stuck in ridges in the real-world refrigerator shelf that were not modeled in simulation.
These discrepancies highlight the sim-to-real gap that \EMfM can suffer from, since it blindly relies on the result of the physics based simulator used in the algorithm.
On average, for the 15 successful retrievals, \EMfM took a total time of $39.3 \pm 28.2 \,\SI{}{\second}$ of which $0.9 \pm 1.2 \,\SI{}{\second}$ was spent calling the \MAPF solver, $33.5 \pm 25.8 \,\SI{}{\second}$ was spent planning pushes, and $7.1 \pm 5.3 \,\SI{}{\second}$ was spent simulating them.


%%%%%%%%%%%%%%%%%%%%%%%%%%%%%%%%%%%%%%%%%%%%%%%%%%%%%%%%%%%%%%%%%%%%%%%%%%%%%%%%
%%%%%%%%%%%%%%%%%%%%%%%%%%%%%%%%%%%%%%%%%%%%%%%%%%%%%%%%%%%%%%%%%%%%%%%%%%%%%%%%

\section{Conclusion and Future Work}

The Enhanced-\MfM algorithm presented in this paper builds upon our prior work on Multi-Agent Pathfinding for Manipulation Among Movable Objects~\cite{Saxena23}.
\EMfM utilises an \MAPF abstraction of \MAMO, a nonprehensile push planner, and a rigid body physics simulator within a best-first graph search for solving \MAMO problems that require determing \emph{which} movable objects should be moved, \emph{where} to move them, and \emph{how} they can be moved.
\EMfM searches over different orderings of object rearrangements, different rearrangements of the workspace, and different ways to rearrange the same object.

Currently, the \MAPF solver does not take into account any information about robot kinematics, movable object properties, or immovable obstacle poses (other than for collision checking against agents) when computing solution paths.
Since the \EMfM algorithm uses these paths downstream for nonprehensile push planning, in future work we wish to explore an experience-based learning formulation to take these factors into account as part of the \MAPF cost function to find paths more likely to result in valid pushes.

%%%%%%%%%%%%%%%%%%%%%%%%%%%%%%%%%%%%%%%%%%%%%%%%%%%%%%%%%%%%%%%%%%%%%%%%%%%%%%%%
%%%%%%%%%%%%%%%%%%%%%%%%%%%%%%%%%%%%%%%%%%%%%%%%%%%%%%%%%%%%%%%%%%%%%%%%%%%%%%%%
% \clearpage
% \section{Preparing an Anonymous Submission}

% This document details the formatting requirements for anonymous submissions. The requirements are the same as for camera ready papers but with a few notable differences:

% \begin{itemize}
%     \item Anonymous submissions must not include the author names and affiliations. Write ``Anonymous Submission'' as the ``sole author'' and leave the affiliations empty.
%     \item The PDF document's metadata should be cleared with a metadata-cleaning tool before submitting it. This is to prevent leaked information from revealing your identity.
%     \item References must be anonymized whenever the reader can infer that they are to the authors' previous work.
%     \item AAAI's copyright notice should not be included as a footer in the first page.
%     \item Only the PDF version is required at this stage. No source versions will be requested, nor any copyright transfer form.
% \end{itemize}

% You can achieve all of the above by enabling the \texttt{submission} option when loading the \texttt{aaai23} package:

% \begin{quote}\begin{scriptsize}\begin{verbatim}
% \documentclass[letterpaper]{article}
% \usepackage[submission]{aaai23}
% \end{verbatim}\end{scriptsize}\end{quote}

% The remainder of this document are the original camera-
% ready instructions. Any contradiction of the above points
% ought to be ignored while preparing anonymous submis-
% sions.

% \section{Camera-Ready Guidelines}

% Congratulations on having a paper selected for inclusion in an AAAI Press proceedings or technical report! This document details the requirements necessary to get your accepted paper published using PDF\LaTeX{}. If you are using Microsoft Word, instructions are provided in a different document. AAAI Press does not support any other formatting software.

% The instructions herein are provided as a general guide for experienced \LaTeX{} users. If you do not know how to use \LaTeX{}, please obtain assistance locally. AAAI cannot provide you with support and the accompanying style files are \textbf{not} guaranteed to work. If the results you obtain are not in accordance with the specifications you received, you must correct your source file to achieve the correct result.

% These instructions are generic. Consequently, they do not include specific dates, page charges, and so forth. Please consult your specific written conference instructions for details regarding your submission. Please review the entire document for specific instructions that might apply to your particular situation. All authors must comply with the following:

% \begin{itemize}
% \item You must use the 2023 AAAI Press \LaTeX{} style file and the aaai23.bst bibliography style files, which are located in the 2023 AAAI Author Kit (aaai23.sty, aaai23.bst).
% \item You must complete, sign, and return by the deadline the AAAI copyright form (unless directed by AAAI Press to use the AAAI Distribution License instead).
% \item You must read and format your paper source and PDF according to the formatting instructions for authors.
% \item You must submit your electronic files and abstract using our electronic submission form \textbf{on time.}
% \item You must pay any required page or formatting charges to AAAI Press so that they are received by the deadline.
% \item You must check your paper before submitting it, ensuring that it compiles without error, and complies with the guidelines found in the AAAI Author Kit.
% \end{itemize}

% \section{Copyright}
% All papers submitted for publication by AAAI Press must be accompanied by a valid signed copyright form. They must also contain the AAAI copyright notice at the bottom of the first page of the paper. There are no exceptions to these requirements. If you fail to provide us with a signed copyright form or disable the copyright notice, we will be unable to publish your paper. There are \textbf{no exceptions} to this policy. You will find a PDF version of the AAAI copyright form in the AAAI AuthorKit. Please see the specific instructions for your conference for submission details.

% \section{Formatting Requirements in Brief}
% We need source and PDF files that can be used in a variety of ways and can be output on a variety of devices. The design and appearance of the paper is strictly governed by the aaai style file (aaai23.sty).
% \textbf{You must not make any changes to the aaai style file, nor use any commands, packages, style files, or macros within your own paper that alter that design, including, but not limited to spacing, floats, margins, fonts, font size, and appearance.} AAAI imposes requirements on your source and PDF files that must be followed. Most of these requirements are based on our efforts to standardize conference manuscript properties and layout. All papers submitted to AAAI for publication will be recompiled for standardization purposes. Consequently, every paper submission must comply with the following requirements:

% \begin{itemize}
% \item Your .tex file must compile in PDF\LaTeX{} --- (you may not include .ps or .eps figure files.)
% \item All fonts must be embedded in the PDF file --- including your figures.
% \item Modifications to the style file, whether directly or via commands in your document may not ever be made, most especially when made in an effort to avoid extra page charges or make your paper fit in a specific number of pages.
% \item No type 3 fonts may be used (even in illustrations).
% \item You may not alter the spacing above and below captions, figures, headings, and subheadings.
% \item You may not alter the font sizes of text elements, footnotes, heading elements, captions, or title information (for references and mathematics, please see the limited exceptions provided herein).
% \item You may not alter the line spacing of text.
% \item Your title must follow Title Case capitalization rules (not sentence case).
% \item \LaTeX{} documents must use the Times or Nimbus font package (you may not use Computer Modern for the text of your paper).
% \item No \LaTeX{} 209 documents may be used or submitted.
% \item Your source must not require use of fonts for non-Roman alphabets within the text itself. If your paper includes symbols in other languages (such as, but not limited to, Arabic, Chinese, Hebrew, Japanese, Thai, Russian and other Cyrillic languages), you must restrict their use to bit-mapped figures. Fonts that require non-English language support (CID and Identity-H) must be converted to outlines or 300 dpi bitmap or removed from the document (even if they are in a graphics file embedded in the document).
% \item Two-column format in AAAI style is required for all papers.
% \item The paper size for final submission must be US letter without exception.
% \item The source file must exactly match the PDF.
% \item The document margins may not be exceeded (no overfull boxes).
% \item The number of pages and the file size must be as specified for your event.
% \item No document may be password protected.
% \item Neither the PDFs nor the source may contain any embedded links or bookmarks (no hyperref or navigator packages).
% \item Your source and PDF must not have any page numbers, footers, or headers (no pagestyle commands).
% \item Your PDF must be compatible with Acrobat 5 or higher.
% \item Your \LaTeX{} source file (excluding references) must consist of a \textbf{single} file (use of the ``input" command is not allowed.
% \item Your graphics must be sized appropriately outside of \LaTeX{} (do not use the ``clip" or ``trim'' command) .
% \end{itemize}

% If you do not follow these requirements, your paper will be returned to you to correct the deficiencies.

% \section{What Files to Submit}
% You must submit the following items to ensure that your paper is published:
% \begin{itemize}
% \item A fully-compliant PDF file.
% \item Your \LaTeX{} source file submitted as a \textbf{single} .tex file (do not use the ``input" command to include sections of your paper --- every section must be in the single source file). (The only allowable exception is .bib file, which should be included separately).
% \item The bibliography (.bib) file(s).
% \item Your source must compile on our system, which includes only standard \LaTeX{} 2020 TeXLive support files.
% \item Only the graphics files used in compiling paper.
% \item The \LaTeX{}-generated files (e.g. .aux,  .bbl file, PDF, etc.).
% \end{itemize}

% Your \LaTeX{} source will be reviewed and recompiled on our system (if it does not compile, your paper will be returned to you. \textbf{Do not submit your source in multiple text files.} Your single \LaTeX{} source file must include all your text, your bibliography (formatted using aaai23.bst), and any custom macros.

% Your files should work without any supporting files (other than the program itself) on any computer with a standard \LaTeX{} distribution.

% \textbf{Do not send files that are not actually used in the paper.} We don't want you to send us any files not needed for compiling your paper, including, for example, this instructions file, unused graphics files, style files, additional material sent for the purpose of the paper review, and so forth.

% \textbf{Do not send supporting files that are not actually used in the paper.} We don't want you to send us any files not needed for compiling your paper, including, for example, this instructions file, unused graphics files, style files, additional material sent for the purpose of the paper review, and so forth.

% \textbf{Obsolete style files.} The commands for some common packages (such as some used for algorithms), may have changed. Please be certain that you are not compiling your paper using old or obsolete style files.

% \textbf{Final Archive.} Place your PDF and source files in a single archive which should be compressed using .zip. The final file size may not exceed 10 MB.
% Name your source file with the last (family) name of the first author, even if that is not you.


% \section{Using \LaTeX{} to Format Your Paper}

% The latest version of the AAAI style file is available on AAAI's website. Download this file and place it in the \TeX\ search path. Placing it in the same directory as the paper should also work. You must download the latest version of the complete AAAI Author Kit so that you will have the latest instruction set and style file.

% \subsection{Document Preamble}

% In the \LaTeX{} source for your paper, you \textbf{must} place the following lines as shown in the example in this subsection. This command set-up is for three authors. Add or subtract author and address lines as necessary, and uncomment the portions that apply to you. In most instances, this is all you need to do to format your paper in the Times font. The helvet package will cause Helvetica to be used for sans serif. These files are part of the PSNFSS2e package, which is freely available from many Internet sites (and is often part of a standard installation).

% Leave the setcounter for section number depth commented out and set at 0 unless you want to add section numbers to your paper. If you do add section numbers, you must uncomment this line and change the number to 1 (for section numbers), or 2 (for section and subsection numbers). The style file will not work properly with numbering of subsubsections, so do not use a number higher than 2.

% \subsubsection{The Following Must Appear in Your Preamble}
% \begin{quote}
% \begin{scriptsize}\begin{verbatim}
% \documentclass[letterpaper]{article}
% % DO NOT CHANGE THIS
% \usepackage[submission]{aaai23} % DO NOT CHANGE THIS
% \usepackage{times} % DO NOT CHANGE THIS
% \usepackage{helvet} % DO NOT CHANGE THIS
% \usepackage{courier} % DO NOT CHANGE THIS
% \usepackage[hyphens]{url} % DO NOT CHANGE THIS
% \usepackage{graphicx} % DO NOT CHANGE THIS
% \urlstyle{rm} % DO NOT CHANGE THIS
% \def\UrlFont{\rm} % DO NOT CHANGE THIS
% \usepackage{graphicx}  % DO NOT CHANGE THIS
% \usepackage{natbib}  % DO NOT CHANGE THIS
% \usepackage{caption}  % DO NOT CHANGE THIS
% \frenchspacing % DO NOT CHANGE THIS
% \setlength{\pdfpagewidth}{8.5in} % DO NOT CHANGE THIS
% \setlength{\pdfpageheight}{11in} % DO NOT CHANGE THIS
% %
% % Keep the \pdfinfo as shown here. There's no need
% % for you to add the /Title and /Author tags.
% \pdfinfo{
% /TemplateVersion (2023.1)
% }
% \end{verbatim}\end{scriptsize}
% \end{quote}

% \subsection{Preparing Your Paper}

% After the preamble above, you should prepare your paper as follows:
% \begin{quote}
% \begin{scriptsize}\begin{verbatim}
% \begin{document}
% \maketitle
% \begin{abstract}
% %...
% \end{abstract}\end{verbatim}\end{scriptsize}
% \end{quote}

% \noindent You should then continue with the body of your paper. Your paper must conclude with the references, which should be inserted as follows:
% \begin{quote}
% \begin{scriptsize}\begin{verbatim}
% % References and End of Paper
% % These lines must be placed at the end of your paper
% \bibliography{Bibliography-File}
% \end{document}
% \end{verbatim}\end{scriptsize}
% \end{quote}

% \subsection{Inserting Document Metadata with \LaTeX{}}
% PDF files contain document summary information that enables us to create an Acrobat index (pdx) file, and also allows search engines to locate and present your paper more accurately. \textit{Document metadata for author and title are REQUIRED.} You may not apply any script or macro to the implementation of the title, author, and metadata information in your paper.

% \textit{Important:} Do not include \textit{any} \LaTeX{} code or nonascii characters (including accented characters) in the metadata. The data in the metadata must be completely plain ascii. It may not include slashes, accents, linebreaks, unicode, or any \LaTeX{} commands. Type the title exactly as it appears on the paper (minus all formatting). Input the author names in the order in which they appear on the paper (minus all accents), separating each author by a comma. You may also include keywords in the optional Keywords field.

% \begin{quote}
% \begin{scriptsize}\begin{verbatim}
% \begin{document}\\
% \maketitle\\
% ...\\
% \bibliography{Bibliography-File}\\
% \end{document}\\
% \end{verbatim}\end{scriptsize}
% \end{quote}

% \subsection{Commands and Packages That May Not Be Used}
% \begin{table*}[t]
% \centering

% \begin{tabular}{l|l|l|l}
% \textbackslash abovecaption &
% \textbackslash abovedisplay &
% \textbackslash addevensidemargin &
% \textbackslash addsidemargin \\
% \textbackslash addtolength &
% \textbackslash baselinestretch &
% \textbackslash belowcaption &
% \textbackslash belowdisplay \\
% \textbackslash break &
% \textbackslash clearpage &
% \textbackslash clip &
% \textbackslash columnsep \\
% \textbackslash float &
% \textbackslash input &
% \textbackslash input &
% \textbackslash linespread \\
% \textbackslash newpage &
% \textbackslash pagebreak &
% \textbackslash renewcommand &
% \textbackslash setlength \\
% \textbackslash text height &
% \textbackslash tiny &
% \textbackslash top margin &
% \textbackslash trim \\
% \textbackslash vskip\{- &
% \textbackslash vspace\{- \\
% \end{tabular}
% %}
% \caption{Commands that must not be used}
% \label{table1}
% \end{table*}

% \begin{table}[t]
% \centering
% %\resizebox{.95\columnwidth}{!}{
% \begin{tabular}{l|l|l|l}
%     authblk & babel & cjk & dvips \\
%     epsf & epsfig & euler & float \\
%     fullpage & geometry & graphics & hyperref \\
%     layout & linespread & lmodern & maltepaper \\
%     navigator & pdfcomment & pgfplots & psfig \\
%     pstricks & t1enc & titlesec & tocbind \\
%     ulem
% \end{tabular}
% \caption{LaTeX style packages that must not be used.}
% \label{table2}
% \end{table}

% There are a number of packages, commands, scripts, and macros that are incompatable with aaai23.sty. The common ones are listed in tables \ref{table1} and \ref{table2}. Generally, if a command, package, script, or macro alters floats, margins, fonts, sizing, linespacing, or the presentation of the references and citations, it is unacceptable. Note that negative vskip and vspace may not be used except in certain rare occurances, and may never be used around tables, figures, captions, sections, subsections, subsubsections, or references.


% \subsection{Page Breaks}
% For your final camera ready copy, you must not use any page break commands. References must flow directly after the text without breaks. Note that some conferences require references to be on a separate page during the review process. AAAI Press, however, does not require this condition for the final paper.


% \subsection{Paper Size, Margins, and Column Width}
% Papers must be formatted to print in two-column format on 8.5 x 11 inch US letter-sized paper. The margins must be exactly as follows:
% \begin{itemize}
% \item Top margin: .75 inches
% \item Left margin: .75 inches
% \item Right margin: .75 inches
% \item Bottom margin: 1.25 inches
% \end{itemize}


% The default paper size in most installations of \LaTeX{} is A4. However, because we require that your electronic paper be formatted in US letter size, the preamble we have provided includes commands that alter the default to US letter size. Please note that using any other package to alter page size (such as, but not limited to the Geometry package) will result in your final paper being returned to you for correction.


% \subsubsection{Column Width and Margins.}
% To ensure maximum readability, your paper must include two columns. Each column should be 3.3 inches wide (slightly more than 3.25 inches), with a .375 inch (.952 cm) gutter of white space between the two columns. The aaai23.sty file will automatically create these columns for you.

% \subsection{Overlength Papers}
% If your paper is too long and you resort to formatting tricks to make it fit, it is quite likely that it will be returned to you. The best way to retain readability if the paper is overlength is to cut text, figures, or tables. There are a few acceptable ways to reduce paper size that don't affect readability. First, turn on \textbackslash frenchspacing, which will reduce the space after periods. Next, move all your figures and tables to the top of the page. Consider removing less important portions of a figure. If you use \textbackslash centering instead of \textbackslash begin\{center\} in your figure environment, you can also buy some space. For mathematical environments, you may reduce fontsize {\bf but not below 6.5 point}.


% Commands that alter page layout are forbidden. These include \textbackslash columnsep,  \textbackslash float, \textbackslash topmargin, \textbackslash topskip, \textbackslash textheight, \textbackslash textwidth, \textbackslash oddsidemargin, and \textbackslash evensizemargin (this list is not exhaustive). If you alter page layout, you will be required to pay the page fee. Other commands that are questionable and may cause your paper to be rejected include \textbackslash parindent, and \textbackslash parskip. Commands that alter the space between sections are forbidden. The title sec package is not allowed. Regardless of the above, if your paper is obviously ``squeezed" it is not going to to be accepted. Options for reducing the length of a paper include reducing the size of your graphics, cutting text, or paying the extra page charge (if it is offered).


% \subsection{Type Font and Size}
% Your paper must be formatted in Times Roman or Nimbus. We will not accept papers formatted using Computer Modern or Palatino or some other font as the text or heading typeface. Sans serif, when used, should be Courier. Use Symbol or Lucida or Computer Modern for \textit{mathematics only. }

% Do not use type 3 fonts for any portion of your paper, including graphics. Type 3 bitmapped fonts are designed for fixed resolution printers. Most print at 300 dpi even if the printer resolution is 1200 dpi or higher. They also often cause high resolution imagesetter devices to crash. Consequently, AAAI will not accept electronic files containing obsolete type 3 fonts. Files containing those fonts (even in graphics) will be rejected. (Authors using blackboard symbols must be avoid those packages that use type 3 fonts.)

% Fortunately, there are effective workarounds that will prevent your file from embedding type 3 bitmapped fonts. The easiest workaround is to use the required times, helvet, and courier packages with \LaTeX{}2e. (Note that papers formatted in this way will still use Computer Modern for the mathematics. To make the math look good, you'll either have to use Symbol or Lucida, or you will need to install type 1 Computer Modern fonts --- for more on these fonts, see the section ``Obtaining Type 1 Computer Modern.")

% If you are unsure if your paper contains type 3 fonts, view the PDF in Acrobat Reader. The Properties/Fonts window will display the font name, font type, and encoding properties of all the fonts in the document. If you are unsure if your graphics contain type 3 fonts (and they are PostScript or encapsulated PostScript documents), create PDF versions of them, and consult the properties window in Acrobat Reader.

% The default size for your type must be ten-point with twelve-point leading (line spacing). Start all pages (except the first) directly under the top margin. (See the next section for instructions on formatting the title page.) Indent ten points when beginning a new paragraph, unless the paragraph begins directly below a heading or subheading.


% \subsubsection{Obtaining Type 1 Computer Modern for \LaTeX{}.}

% If you use Computer Modern for the mathematics in your paper (you cannot use it for the text) you may need to download type 1 Computer fonts. They are available without charge from the American Mathematical Society:
% http://www.ams.org/tex/type1-fonts.html.

% \subsubsection{Nonroman Fonts.}
% If your paper includes symbols in other languages (such as, but not limited to, Arabic, Chinese, Hebrew, Japanese, Thai, Russian and other Cyrillic languages), you must restrict their use to bit-mapped figures.

% \subsection{Title and Authors}
% Your title must appear in mixed case (nouns, pronouns, and verbs are capitalized) near the top of the first page, centered over both columns in sixteen-point bold type (twenty-four point leading). This style is called ``mixed case" (or ``title case"), which means that all verbs (including short verbs like be, is, using, and go), nouns, adverbs, adjectives, and pronouns should be capitalized, (including both words in hyphenated terms), while articles, conjunctions, and prepositions are lower case unless they directly follow a colon or long dash.

% Author's names should appear below the title of the paper, centered in twelve-point type (with fifteen point leading), along with affiliation(s) and complete address(es) (including electronic mail address if available) in nine-point roman type (the twelve point leading). You should begin the two-column format when you come to the abstract.

% \subsubsection{Formatting Author Information.}
% Author information has to be set according the following specification depending if you have one or more than one affiliation.  You may not use a table nor may you employ the \textbackslash authorblk.sty package. For one or several authors from the same institution, please just separate with commas and write the affiliation directly below using the macros \textbackslash author and \textbackslash affiliations:

% \begin{quote}\begin{scriptsize}\begin{verbatim}
% \author{
%     Author 1, ..., Author n\\
% }
% \affiliations {
%     Address line\\
%     ... \\
%     Address line\\
% }
% \end{verbatim}\end{scriptsize}\end{quote}


% \noindent For authors from different institutions, use \textbackslash textsuperscript \{\textbackslash rm x \} to match authors and affiliations. Notice that there should not be any spaces between the author name (or comma following it) and the superscript.

% \begin{quote}\begin{scriptsize}\begin{verbatim}
% \author{
%     AuthorOne,\equalcontrib\textsuperscript{\rm 1}
%     AuthorTwo,\equalcontrib\textsuperscript{\rm 2}
%     AuthorThree,\textsuperscript{\rm 3}\\
%     AuthorFour,\textsuperscript{\rm 4}
%     AuthorFive \textsuperscript{\rm 5}}
% }
% \affiliations {
%     \textsuperscript{\rm 1}AffiliationOne,\\
%     \textsuperscript{\rm 2}AffiliationTwo,\\
%     \textsuperscript{\rm 3}AffiliationThree,\\
%     \textsuperscript{\rm 4}AffiliationFour,\\
%     \textsuperscript{\rm 5}AffiliationFive\\
%     \{email, email\}@affiliation.com,
%     email@affiliation.com,
%     email@affiliation.com,
%     email@affiliation.com
% }
% \end{verbatim}\end{scriptsize}\end{quote}

% You can indicate that some authors contributed equally using the \textbackslash equalcontrib command. This will add a marker after the author names and a footnote on the first page.

% Note that you may want to  break the author list for better visualization. You can achieve this using a simple line break (\textbackslash  \textbackslash).

% \subsection{\LaTeX{} Copyright Notice}
% The copyright notice automatically appears if you use aaai23.sty. It has been hardcoded and may not be disabled.

% \subsection{Credits}
% Any credits to a sponsoring agency should appear in the acknowledgments section, unless the agency requires different placement. If it is necessary to include this information on the front page, use
% \textbackslash thanks in either the \textbackslash author or \textbackslash title commands.
% For example:
% \begin{quote}
% \begin{small}
% \textbackslash title\{Very Important Results in AI\textbackslash thanks\{This work is
%  supported by everybody.\}\}
% \end{small}
% \end{quote}
% Multiple \textbackslash thanks commands can be given. Each will result in a separate footnote indication in the author or title with the corresponding text at the botton of the first column of the document. Note that the \textbackslash thanks command is fragile. You will need to use \textbackslash protect.

% Please do not include \textbackslash pubnote commands in your document.

% \subsection{Abstract}
% Follow the example commands in this document for creation of your abstract. The command \textbackslash begin\{abstract\} will automatically indent the text block. Please do not indent it further. {Do not include references in your abstract!}

% \subsection{Page Numbers}

% Do not \textbf{ever} print any page numbers on your paper. The use of \textbackslash pagestyle is forbidden.

% \subsection{Text }
% The main body of the paper must be formatted in black, ten-point Times Roman with twelve-point leading (line spacing). You may not reduce font size or the linespacing. Commands that alter font size or line spacing (including, but not limited to baselinestretch, baselineshift, linespread, and others) are expressly forbidden. In addition, you may not use color in the text.

% \subsection{Citations}
% Citations within the text should include the author's last name and year, for example (Newell 1980). Append lower-case letters to the year in cases of ambiguity. Multiple authors should be treated as follows: (Feigenbaum and Engelmore 1988) or (Ford, Hayes, and Glymour 1992). In the case of four or more authors, list only the first author, followed by et al. (Ford et al. 1997).

% \subsection{Extracts}
% Long quotations and extracts should be indented ten points from the left and right margins.

% \begin{quote}
% This is an example of an extract or quotation. Note the indent on both sides. Quotation marks are not necessary if you offset the text in a block like this, and properly identify and cite the quotation in the text.

% \end{quote}

% \subsection{Footnotes}
% Avoid footnotes as much as possible; they interrupt the reading of the text. When essential, they should be consecutively numbered throughout with superscript Arabic numbers. Footnotes should appear at the bottom of the page, separated from the text by a blank line space and a thin, half-point rule.

% \subsection{Headings and Sections}
% When necessary, headings should be used to separate major sections of your paper. Remember, you are writing a short paper, not a lengthy book! An overabundance of headings will tend to make your paper look more like an outline than a paper. The aaai23.sty package will create headings for you. Do not alter their size nor their spacing above or below.

% \subsubsection{Section Numbers.}
% The use of section numbers in AAAI Press papers is optional. To use section numbers in \LaTeX{}, uncomment the setcounter line in your document preamble and change the 0 to a 1. Section numbers should not be used in short poster papers and/or extended abstracts.

% \subsubsection{Section Headings.}
% Sections should be arranged and headed as follows:
% \begin{enumerate}
% \item Main content sections
% \item Appendices (optional)
% \item Ethical statement (optional, unnumbered)
% \item Acknowledgements (optional, unnumbered)
% \item References (unnumbered)
% \end{enumerate}

% \subsubsection{Appendices.}
% Any appendices must appear after the main content. If your main sections are numbered, appendix sections must use letters instead of arabic numerals. In \LaTeX{} you can use the \texttt{\textbackslash appendix} command to achieve this effect and then use \texttt{\textbackslash section\{Heading\}} normally for your appendix sections.

% \subsubsection{Ethical statement.}
% You can write a statement about the potential ethical impact of your work, including its broad societal implications, both positive and negative. If included, such statement must be written in an unnumbered section titled \emph{Ethical statement}.

% \subsubsection{Acknowledgments.}
% The acknowledgments section, if included, appears right before the references and is headed ``Acknowledgments". It must not be numbered even if other sections are (use \texttt{\textbackslash section*\{Acknowledgements\}} in \LaTeX{}). This section includes acknowledgments of help from associates and colleagues, credits to sponsoring agencies, financial support, and permission to publish. Please acknowledge other contributors, grant support, and so forth, in this section. Do not put acknowledgments in a footnote on the first page. If your grant agency requires acknowledgment of the grant on page 1, limit the footnote to the required statement, and put the remaining acknowledgments at the back. Please try to limit acknowledgments to no more than three sentences.

% \subsubsection{References.}
% The references section should be labeled ``References" and must appear at the very end of the paper (don't end the paper with references, and then put a figure by itself on the last page). A sample list of references is given later on in these instructions. Please use a consistent format for references. Poorly prepared or sloppy references reflect badly on the quality of your paper and your research. Please prepare complete and accurate citations.

% \subsection{Illustrations and  Figures}

% \begin{figure}[t]
% \centering
% \includegraphics[width=0.9\columnwidth]{figure1} % Reduce the figure size so that it is slightly narrower than the column. Don't use precise values for figure width.This setup will avoid overfull boxes.
% \caption{Using the trim and clip commands produces fragile layers that can result in disasters (like this one from an actual paper) when the color space is corrected or the PDF combined with others for the final proceedings. Crop your figures properly in a graphics program -- not in LaTeX}.
% \label{fig1}
% \end{figure}

% \begin{figure*}[t]
% \centering
% \includegraphics[width=0.8\textwidth]{figure2} % Reduce the figure size so that it is slightly narrower than the column.
% \caption{Adjusting the bounding box instead of actually removing the unwanted data resulted multiple layers in this paper. It also needlessly increased the PDF size. In this case, the size of the unwanted layer doubled the paper's size, and produced the following surprising results in final production. Crop your figures properly in a graphics program. Don't just alter the bounding box.}
% \label{fig2}
% \end{figure*}

% % Using the \centering command instead of \begin{center} ... \end{center} will save space
% % Positioning your figure at the top of the page will save space and make the paper more readable
% % Using 0.95\columnwidth in conjunction with the


% Your paper must compile in PDF\LaTeX{}. Consequently, all your figures must be .jpg, .png, or .pdf. You may not use the .gif (the resolution is too low), .ps, or .eps file format for your figures.

% Figures, drawings, tables, and photographs should be placed throughout the paper on the page (or the subsequent page) where they are first discussed. Do not group them together at the end of the paper. If placed at the top of the paper, illustrations may run across both columns. Figures must not invade the top, bottom, or side margin areas. Figures must be inserted using the \textbackslash usepackage\{graphicx\}. Number figures sequentially, for example, figure 1, and so on. Do not use minipage to group figures.

% If you normally create your figures using pgfplots, please create the figures first, and then import them as pdfs with proper bounding boxes, as the bounding and trim boxes created by pfgplots are fragile and not valid.

% When you include your figures, you must crop them \textbf{outside} of \LaTeX{}. The command \textbackslash includegraphics*[clip=true, viewport 0 0 10 10]{...} might result in a PDF that looks great, but the image is \textbf{not really cropped.} The full image can reappear (and obscure whatever it is overlapping) when page numbers are applied or color space is standardized. Figures \ref{fig1}, and \ref{fig2} display some unwanted results that often occur.

% If your paper includes illustrations that are not compatible with PDF\TeX{} (such as .eps or .ps documents), you will need to convert them. The epstopdf package will usually work for eps files. You will need to convert your ps files to PDF in either case.

% \subsubsection {Figure Captions.}The illustration number and caption must appear \textit{under} the illustration. Labels and other text with the actual illustration must be at least nine-point type. However, the font and size of figure captions must be 10 point roman. Do not make them smaller, bold, or italic. (Individual words may be italicized if the context requires differentiation.)

% \subsection{Tables}

% Tables should be presented in 10 point roman type. If necessary, they may be altered to 9 point type. You may not use any commands that further reduce point size below nine points. Tables that do not fit in a single column must be placed across double columns. If your table won't fit within the margins even when spanning both columns, you must split it. Do not use minipage to group tables.

% \subsubsection {Table Captions.} The number and caption for your table must appear \textit{under} (not above) the table.  Additionally, the font and size of table captions must be 10 point roman and must be placed beneath the figure. Do not make them smaller, bold, or italic. (Individual words may be italicized if the context requires differentiation.)



% \subsubsection{Low-Resolution Bitmaps.}
% You may not use low-resolution (such as 72 dpi) screen-dumps and GIF files---these files contain so few pixels that they are always blurry, and illegible when printed. If they are color, they will become an indecipherable mess when converted to black and white. This is always the case with gif files, which should never be used. The resolution of screen dumps can be increased by reducing the print size of the original file while retaining the same number of pixels. You can also enlarge files by manipulating them in software such as PhotoShop. Your figures should be 300 dpi when incorporated into your document.

% \subsubsection{\LaTeX{} Overflow.}
% \LaTeX{} users please beware: \LaTeX{} will sometimes put portions of the figure or table or an equation in the margin. If this happens, you need to make the figure or table span both columns. If absolutely necessary, you may reduce the figure, or reformat the equation, or reconfigure the table.{ \bf Check your log file!} You must fix any overflow into the margin (that means no overfull boxes in \LaTeX{}). \textbf{Nothing is permitted to intrude into the margin or gutter.}


% \subsubsection{Using Color.}
% Use of color is restricted to figures only. It must be WACG 2.0 compliant. (That is, the contrast ratio must be greater than 4.5:1 no matter the font size.) It must be CMYK, NOT RGB. It may never be used for any portion of the text of your paper. The archival version of your paper will be printed in black and white and grayscale. The web version must be readable by persons with disabilities. Consequently, because conversion to grayscale can cause undesirable effects (red changes to black, yellow can disappear, and so forth), we strongly suggest you avoid placing color figures in your document. If you do include color figures, you must (1) use the CMYK (not RGB) colorspace and (2) be mindful of readers who may happen to have trouble distinguishing colors. Your paper must be decipherable without using color for distinction.

% \subsubsection{Drawings.}
% We suggest you use computer drawing software (such as Adobe Illustrator or, (if unavoidable), the drawing tools in Microsoft Word) to create your illustrations. Do not use Microsoft Publisher. These illustrations will look best if all line widths are uniform (half- to two-point in size), and you do not create labels over shaded areas. Shading should be 133 lines per inch if possible. Use Times Roman or Helvetica for all figure call-outs. \textbf{Do not use hairline width lines} --- be sure that the stroke width of all lines is at least .5 pt. Zero point lines will print on a laser printer, but will completely disappear on the high-resolution devices used by our printers.

% \subsubsection{Photographs and Images.}
% Photographs and other images should be in grayscale (color photographs will not reproduce well; for example, red tones will reproduce as black, yellow may turn to white, and so forth) and set to a minimum of 300 dpi. Do not prescreen images.

% \subsubsection{Resizing Graphics.}
% Resize your graphics \textbf{before} you include them with LaTeX. You may \textbf{not} use trim or clip options as part of your \textbackslash includegraphics command. Resize the media box of your PDF using a graphics program instead.

% \subsubsection{Fonts in Your Illustrations.}
% You must embed all fonts in your graphics before including them in your LaTeX document.

% \subsubsection{Algorithms.}
% Algorithms and/or programs are a special kind of figures. Like all illustrations, they should appear floated to the top (preferably) or bottom of the page. However, their caption should appear in the header, left-justified and enclosed between horizontal lines, as shown in Algorithm~\ref{alg:algorithm}. The algorithm body should be terminated with another horizontal line. It is up to the authors to decide whether to show line numbers or not, how to format comments, etc.

% In \LaTeX{} algorithms may be typeset using the {\tt algorithm} and {\tt algorithmic} packages, but you can also use one of the many other packages for the task.

% \begin{algorithm}[tb]
% \caption{Example algorithm}
% \label{alg:algorithm}
% \textbf{Input}: Your algorithm's input\\
% \textbf{Parameter}: Optional list of parameters\\
% \textbf{Output}: Your algorithm's output
% \begin{algorithmic}[1] %[1] enables line numbers
% \State Let $t=0$.
% \While{condition}
% \State Do some action.
% \If {conditional}
% \State Perform task A.
% \Else
% \State Perform task B.
% \EndIf
% \EndWhile
% \State \textbf{return} solution
% \end{algorithmic}
% \end{algorithm}

% \subsubsection{Listings.}
% Listings are much like algorithms and programs. They should also appear floated to the top (preferably) or bottom of the page. Listing captions should appear in the header, left-justified and enclosed between horizontal lines as shown in Listing~\ref{lst:listing}. Terminate the body with another horizontal line and avoid any background color. Line numbers, if included, must appear within the text column.

% \begin{listing}[tb]%
% \caption{Example listing {\tt quicksort.hs}}%
% \label{lst:listing}%
% \begin{lstlisting}[language=Haskell]
% quicksort :: Ord a => [a] -> [a]
% quicksort []     = []
% quicksort (p:xs) = (quicksort lesser) ++ [p] ++ (quicksort greater)
% 	where
% 		lesser  = filter (< p) xs
% 		greater = filter (>= p) xs
% \end{lstlisting}
% \end{listing}

% \subsection{References}
% The AAAI style includes a set of definitions for use in formatting references with BibTeX. These definitions make the bibliography style fairly close to the ones  specified in the Reference Examples appendix below. To use these definitions, you also need the BibTeX style file ``aaai23.bst," available in the AAAI Author Kit on the AAAI web site. Then, at the end of your paper but before \textbackslash end{document}, you need to put the following lines:

% \begin{quote}
% \begin{small}
% \textbackslash bibliography\{bibfile1,bibfile2,...\}
% \end{small}
% \end{quote}

% Please note that the aaai23.sty class already sets the bibliographystyle for you, so you do not have to place any \textbackslash bibliographystyle command in the document yourselves. The aaai23.sty file is incompatible with the hyperref and navigator packages. If you use either, your references will be garbled and your paper will be returned to you.

% References may be the same size as surrounding text. However, in this section (only), you may reduce the size to \textbackslash small if your paper exceeds the allowable number of pages. Making it any smaller than 9 point with 10 point linespacing, however, is not allowed. A more precise and exact method of reducing the size of your references minimally is by means of the following command: \begin{quote}
% \textbackslash fontsize\{9.8pt\}\{10.8pt\}
% \textbackslash selectfont\end{quote}

% \noindent You must reduce the size equally for both font size and line spacing, and may not reduce the size beyond \{9.0pt\}\{10.0pt\}.

% The list of files in the \textbackslash bibliography command should be the names of your BibTeX source files (that is, the .bib files referenced in your paper).

% The following commands are available for your use in citing references:
% \begin{quote}
% {\em \textbackslash cite:} Cites the given reference(s) with a full citation. This appears as ``(Author Year)'' for one reference, or ``(Author Year; Author Year)'' for multiple references.\smallskip\\
% {\em \textbackslash shortcite:} Cites the given reference(s) with just the year. This appears as ``(Year)'' for one reference, or ``(Year; Year)'' for multiple references.\smallskip\\
% {\em \textbackslash citeauthor:} Cites the given reference(s) with just the author name(s) and no parentheses.\smallskip\\
% {\em \textbackslash citeyear:} Cites the given reference(s) with just the date(s) and no parentheses.
% \end{quote}
% You may also use any of the \emph{natbib} citation commands.


% \section{Proofreading Your PDF}
% Please check all the pages of your PDF file. The most commonly forgotten element is the acknowledgements --- especially the correct grant number. Authors also commonly forget to add the metadata to the source, use the wrong reference style file, or don't follow the capitalization rules or comma placement for their author-title information properly. A final common problem is text (expecially equations) that runs into the margin. You will need to fix these common errors before submitting your file.

% \section{Improperly Formatted Files }
% In the past, AAAI has corrected improperly formatted files submitted by the authors. Unfortunately, this has become an increasingly burdensome expense that we can no longer absorb). Consequently, if your file is improperly formatted, it will be returned to you for correction.

% \subsection{\LaTeX{} 209 Warning}
% If you use \LaTeX{} 209 your paper will be returned to you unpublished. Convert your paper to \LaTeX{}2e.

% \section{Naming Your Electronic File}
% We require that you name your \LaTeX{} source file with the last name (family name) of the first author so that it can easily be differentiated from other submissions. Complete file-naming instructions will be provided to you in the submission instructions.

% \section{Submitting Your Electronic Files to AAAI}
% Instructions on paper submittal will be provided to you in your acceptance letter.

% \section{Inquiries}
% If you have any questions about the preparation or submission of your paper as instructed in this document, please contact AAAI Press at the address given below. If you have technical questions about implementation of the aaai style file, please contact an expert at your site. We do not provide technical support for \LaTeX{} or any other software package. To avoid problems, please keep your paper simple, and do not incorporate complicated macros and style files.

% \begin{quote}
% \noindent AAAI Press\\
% 1900 Embarcadero Road, Suite 101\\
% Palo Alto, California 94303-3310 USA\\
% \textit{Telephone:} (650) 328-3123\\
% \textit{E-mail:} See the submission instructions for your particular conference or event.
% \end{quote}

% \section{Additional Resources}
% \LaTeX{} is a difficult program to master. If you've used that software, and this document didn't help or some items were not explained clearly, we recommend you read Michael Shell's excellent document (testflow doc.txt V1.0a 2002/08/13) about obtaining correct PS/PDF output on \LaTeX{} systems. (It was written for another purpose, but it has general application as well). It is available at www.ctan.org in the tex-archive.

% \appendix
% \section{Reference Examples}
% \label{sec:reference_examples}

% \nobibliography*
% Formatted bibliographies should look like the following examples. You should use BibTeX to generate the references. Missing fields are unacceptable when compiling references, and usually indicate that you are using the wrong type of entry (BibTeX class).

% \paragraph{Book with multiple authors~\nocite{em:86}} Use the \texttt{@book} class.\\[.2em]
% \bibentry{em:86}.

% \paragraph{Journal and magazine articles~\nocite{r:80, hcr:83}} Use the \texttt{@article} class.\\[.2em]
% \bibentry{r:80}.\\[.2em]
% \bibentry{hcr:83}.

% \paragraph{Proceedings paper published by a society, press or publisher~\nocite{c:83, c:84}} Use the \texttt{@inproceedings} class. You may abbreviate the \emph{booktitle} field, but make sure that the conference edition is clear.\\[.2em]
% \bibentry{c:84}.\\[.2em]
% \bibentry{c:83}.

% \paragraph{University technical report~\nocite{r:86}} Use the \texttt{@techreport} class.\\[.2em]
% \bibentry{r:86}.

% \paragraph{Dissertation or thesis~\nocite{c:79}} Use the \texttt{@phdthesis} class.\\[.2em]
% \bibentry{c:79}.

% \paragraph{Forthcoming publication~\nocite{c:21}} Use the \texttt{@misc} class with a \texttt{note="Forthcoming"} annotation.
% \begin{quote}
% \begin{footnotesize}
% \begin{verbatim}
% @misc(key,
%   [...]
%   note="Forthcoming",
% )
% \end{verbatim}
% \end{footnotesize}
% \end{quote}
% \bibentry{c:21}.

% \paragraph{ArXiv paper~\nocite{c:22}} Fetch the BibTeX entry from the "Export Bibtex Citation" link in the arXiv website. Notice it uses the \texttt{@misc} class instead of the \texttt{@article} one, and that it includes the \texttt{eprint} and \texttt{archivePrefix} keys.
% \begin{quote}
% \begin{footnotesize}
% \begin{verbatim}
% @misc(key,
%   [...]
%   eprint="xxxx.yyyy",
%   archivePrefix="arXiv",
% )
% \end{verbatim}
% \end{footnotesize}
% \end{quote}
% \bibentry{c:22}.

% \paragraph{Website or online resource~\nocite{c:23}} Use the \texttt{@misc} class. Add the url in the \texttt{howpublished} field and the date of access in the \texttt{note} field:
% \begin{quote}
% \begin{footnotesize}
% \begin{verbatim}
% @misc(key,
%   [...]
%   howpublished="\url{http://...}",
%   note="Accessed: YYYY-mm-dd",
% )
% \end{verbatim}
% \end{footnotesize}
% \end{quote}
% \bibentry{c:23}.

% \vspace{.2em}
% For the most up to date version of the AAAI reference style, please consult the \textit{AI Magazine} Author Guidelines at \url{https://aaai.org/ojs/index.php/aimagazine/about/submissions#authorGuidelines}

% Use \bibliography{yourbibfile} instead or the References section will not appear in your paper
\bibliography{references}
% \nobibliography{aaai23}

% \section{Acknowledgments}
% AAAI is especially grateful to Peter Patel Schneider for his work in implementing the original aaai.sty file, liberally using the ideas of other style hackers, including Barbara Beeton. We also acknowledge with thanks the work of George Ferguson for his guide to using the style and BibTeX files --- which has been incorporated into this document --- and Hans Guesgen, who provided several timely modifications, as well as the many others who have, from time to time, sent in suggestions on improvements to the AAAI style. We are especially grateful to Francisco Cruz, Marc Pujol-Gonzalez, and Mico Loretan for the improvements to the Bib\TeX{} and \LaTeX{} files made in 2020.

% The preparation of the \LaTeX{} and Bib\TeX{} files that implement these instructions was supported by Schlumberger Palo Alto Research, AT\&T Bell Laboratories, Morgan Kaufmann Publishers, The Live Oak Press, LLC, and AAAI Press. Bibliography style changes were added by Sunil Issar. \verb+\+pubnote was added by J. Scott Penberthy. George Ferguson added support for printing the AAAI copyright slug. Additional changes to aaai23.sty and aaai23.bst have been made by Francisco Cruz, Marc Pujol-Gonzalez, and Mico Loretan.

% \bigskip
% \noindent Thank you for reading these instructions carefully. We look forward to receiving your electronic files!

\end{document}
