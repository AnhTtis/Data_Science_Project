\section{Method}\label{Method}

\subsection{Overview}

We use version \texttt{2022.03.30.1} of the lens modeling software {\tt PyAutoLens}\github{https://github.com/Jammy2211/PyAutoLens} \citep{Nightingale2021}. {\tt PyAutoLens} fits the lens galaxy's light and mass and the source galaxy simultaneously. The method assumes a model for the lens's foreground light (e.g. one or more Sersic profiles), which is convolved with the instrumental PSF and subtracted from the observed image. A mass model (e.g. an isothermal mass distribution) ray-traces image-pixels from the image-plane to the source-plane and a pixelized source reconstruction, using an adaptive Voronoi mesh, is performed. \cref{figure:ModelFit} provides an overview of a {\tt PyAutoLens} lens model, where models of Abell 1201 for the image-plane lens galaxy emission and lensed source are shown alongside the source-plane source reconstruction.

By fitting the source's extended surface brightness distribution, {\tt PyAutoLens} considers light rays emanating from different parts of the source, therefore constraining different regions of the lens galaxy's potential. If a small-mass clump is near the lensed source's emission, it may cause observable distortions to one or more of its multiple images. This technique has provided detections of three non-luminous dark matter substructures \citep{Vegetti2010, Vegetti2012, Hezaveh2016, Nightingale2022} in strong lenses, where their presence is inferred by how they perturb the appearance of the lensed images. %lensed source's appearance. 

\begin{figure}
\centering
\includegraphics[width=0.235\textwidth]{smbh_demo/smbh_demo_no_smbh.pdf}
\includegraphics[width=0.235\textwidth]{smbh_demo/smbh_demo_smbh.pdf}
\includegraphics[width=0.235\textwidth]{smbh_demo/smbh_demo_no_smbh_zoom.pdf}
\includegraphics[width=0.235\textwidth]{smbh_demo/smbh_demo_smbh_zoom.pdf}
\caption{
Illustration of how Abell 1201's lens configuration is sensitive to the lens galaxy's SMBH. Both images on the top row are simulated using the same lens mass model (a power-law with external shear) and source galaxy light model (an elliptical Sersic). In the left panel a SMBH is not included, whereas in the right panel a $M_{\rm BH} = 10^{10}$\,M$_{\rm \odot}$ SMBH is included at $(0.0, 0.0)$, which is marked with a black cross. The bottom row shows a zoom in on the counter image. The SMBH changes the location, appearance and brightness of the counter image but does not lead to visible changes in the giant arc. The tangential critical curve is shown by a black line and radial critical curve a white line. The latter does not form for sufficiently steep mass profiles \citep{Kochanek2004a}, including the model with a SMBH shown here. 
} 
\label{figure:SMBHDemo}
\end{figure}

\begin{table}
\tiny
\resizebox{\linewidth}{!}{
\begin{tabular}{ l | l | l } 
\multicolumn{1}{p{1.3cm}|}{\centering Number of \\ Sersics} 
& \multicolumn{1}{p{1.8cm}|}{Aligned Elliptical Components} 
& \multicolumn{1}{p{1.8cm}|}{Aligned Centres} 
\\ \hline
 1 & N/A & N/A \\[2pt]
2 & \checkmark & \checkmark   \\[2pt]
2 & \ding{55} & \checkmark \\[2pt]
2 & \ding{55} & \ding{55}  \\[2pt]
3 & \ding{55} & \ding{55} \\[2pt]
\end{tabular}
}
\caption{The five models for the lens's light that are fitted and compared in the Light pipeline. The lens light model assumes either one, two or three Sersic profiles and makes different assumptions as to whether their $(x,y)$ centre and elliptical components ($\epsilon_{\rm 1}$, $\epsilon_{\rm 2}$) are aligned. A tick mark indicates that this assumption is used in the model, for example the second row is a model where both the elliptical components and centres are aligned.}
\label{table:LightModels}
\end{table}

\begin{table*}
\tiny
\resizebox{\linewidth}{!}{
\begin{tabular}{ l | l | l | l l l l l l} 
\multicolumn{1}{p{1.8cm}|}{\centering \textbf{Model}} 
& \multicolumn{1}{p{1.35cm}|}{\centering \textbf{Component}} 
& \multicolumn{1}{p{1.1cm}|}{\centering \textbf{Represents}} 
& \multicolumn{1}{p{1.5cm}}{\textbf{Parameters}} 
& \multicolumn{1}{p{1.5cm}}{} 
\\ \hline
& & & & & & \\[-4pt]
\textbf{Point Mass} & Mass & Black Hole & $\theta^{\rm smbh}_{\rm E}$: Einstein Radius $(\arcsec)$ \\[2pt]
\hline
& & & & & & \\[-4pt]
\textbf{Sersic} & Light + & Stellar Matter & ($x$,$y$): centre $(\arcsec)$ & ($\epsilon_{\rm 1}$, $\epsilon_{\rm 2}$): elliptical components  \\[2pt]
               &  Mass     & (Bulge, Disk, & $I$: Intensity & $R$: Effective Radius $(\arcsec)$\\[2pt]
              & & Envelope) & $n$: Sersic index &  \\[2pt]
               & & & $\Psi$: Mass-to-Light Ratio (e$^{\rm -}$\,s$^{\rm -1}$) & $\Gamma$: Radial Gradient & \\[2pt]
               \hline
& & & & & & \\[-4pt]
\textbf{Elliptical NFW} & Mass & Dark Matter & ($x^{\rm dark}$,$y^{\rm dark}$): centre $(\arcsec)$ & ($\epsilon_{\rm 1}^{\rm dark}$, $\epsilon_{\rm 2}^{\rm dark}$): elliptical components  & \\[2pt]
            &   &   & $M_{\rm 200}^{\rm dark}$: Mass at 200 ($M_{\rm \odot}$) \\[2pt]
\hline
& & & & & & \\[-4pt]
\textbf{Shear} & Mass & Line-of-sight & ($\epsilon_{\rm 1}^{\rm ext}$, $\epsilon_{\rm 2}^{\rm ext}$): elliptical components  \\[2pt] 
\hline
& & & & & & \\[-4pt]
\textbf{Elliptical} & Mass & Total (Stellar  & ($x^{\rm mass}$,$y^{\rm mass}$): centre $(\arcsec)$ & ($\epsilon^{\rm mass}_{\rm 1}$, $\epsilon^{\rm mass}_{\rm 2}$): elliptical components  \\[2pt]
\textbf{Power-Law (PL)} &      & + Dark Matter) & $\theta^{\rm mass}_{\rm  E}$: Einstein Radius $(\arcsec)$ & $\gamma^{\rm mass}$: density slope \\[2pt]
\hline
& & & & & & \\[-4pt]
\textbf{Broken} & Mass & Total (Stellar  & ($x^{\rm mass}$,$y^{\rm mass}$): centre $(\arcsec)$ & ($\epsilon^{\rm mass}_{\rm 1}$, $\epsilon^{\rm mass}_{\rm 2}$): elliptical components  \\[2pt]
 \textbf{Power-Law (BPL)} &      & + Dark Matter) & $\theta^{\rm mass}_{\rm E}$: Einstein Radius $(\arcsec)$ & $t^{\rm mass}_{\rm 1}$: inner density slope \\[2pt]
         &      &                & $r^{\rm mass}_{\rm B}$: Break Radius $(\arcsec)$ & $t^{\rm mass}_{\rm 2}$: outer density slope \\[2pt]
\end{tabular}
}
\caption{The light and mass profiles used in this work. Column 1 gives the model name. Column 2 whether it models the lens's light, mass or both. Column 3 states what component of mass it represents. Column 4 gives its associated parameters and units.}
\label{table:Models}
\end{table*}

This work uses the same technique, albeit we are in this case investigating whether the perturbing %lensing 
effects of the %lens galaxy's 
central SMBH 
are detected in the lensed source emission. This is why the proximity of Abell 1201's counter-image to the lens galaxy's centre, and therefore SMBH, is so important. A high mass SMBH will induce a local perturbation to the counter image's appearance that does not produce a significant change in the appearance of the other multiple images of source in the giant arc. This is shown in \cref{figure:SMBHDemo}, where two simulated lenses based on our models of Abell 1201 are shown. In the right panel, a $M_{\rm BH} = 10^{10}$\,M$_{\rm \odot}$ SMBH is added to the lens model, which changes the location, appearance and brightness of the counter image without producing a visible change in the giant arc. Our results are therefore not based on whether the source forms a central image \citep{Winn2003, Rusin2005} \footnote{By central image, we are referring to the hypothetical third or fifth image that would form directly over the centre of the lens galaxy, provided its mass distribution were sufficiently cored. We therefore do not consider the counter image located ~0.3 arcsec to the southwest of the lens galaxy a central image, and will always refer to it as the counter-image.}.

% We provide a brief description of the {\tt PyAutoLens} likelihood function below. However, we have received feedback from readers of other papers using {\tt PyAutoLens} (who are less familiar with strong lens modeling) that they were unclear on the procedure that translates a strong lens model to a likelihood value. We therefore provide Jupyter notebooks providing a visual step-by-step guide of the {\tt PyAutoLens} likelihood function used in this work, including URL links to previous literature and explanations of technical aspects of the linear algebra and Bayesian inference. The notebooks can be found at the following link \url{https://github.com/Jammy2211/autolens_likelihood_function}.


%RJS revision: 
At the heart of the {\tt PyAutoLens} model fitting process is the computation of the likelihood function.  We provide a brief description of this calculation in \cref{Coordinates} - \cref{Source} below. Furthermore, to assist readers less familiar with strong lens modelling, we provide Jupyter notebooks providing a visual step-by-step guide, including URL links to previous literature and explanations of technical aspects of the linear algebra and Bayesian inference. The notebooks can be found at the following link: \url{https://github.com/Jammy2211/autolens_likelihood_function}.

Recent works using {\tt PyAutoLens} include modeling strong lenses simulated using stellar dynamics models \citep{Cao2021} and via a cosmological simulation \cite{He2023}, an automated analysis of $59$ lenses \citep{Etherington2022, Etherington2022a} and studies of dark matter substructure \citep{He2022a, He2022b, Amorisco2022}. 

\subsection{Coordinate System}\label{Coordinates}

Light and mass profile quantities are computed using elliptical coordinates $\xi = \sqrt{{x}^2 + y^2/q^2}$, with minor to major axis-ratio $q$ and position angle $\phi$ defined counter clockwise from the positive x-axis. For model-fitting, these are parameterized as two components of ellipticity
\begin{equation}
\epsilon_{1} =\frac{1-q}{1+q} \sin 2\phi, \,\,
\epsilon_{2} =\frac{1-q}{1+q} \cos 2\phi.    
\label{eq: ellip}
\end{equation}
To convert parameters from arc-second units to stellar masses we require the critical surface mass density
\begin{equation}
    \Sigma_\mathrm{crit}=\frac{{\rm c}^2}{4{\rm \pi} {\rm G}}\frac{D_{\rm s}}{D_{\rm l} D_{\rm ls}},
    \label{eq: sigma crit}
\end{equation}
where $D_{\rm l}$, $D_{\rm s}$, and $D_{\rm ls}$ are respectively the angular diameter distances to the lens, to the source, and from the lens to the source, and ${\rm c}$ is the speed of light.

% The more clumpy structure seen in the F390W's lensed source in \cref{figure:Data} also provides different constraints on the lens model than the smoother source observed in the F814W band. Whilst simultaneous lens modeling of both wavebands is possible \citep{Dye2014}, we fit each waveband independently due to computational limitations that come from analysing such high resolution HST imaging.

\subsection{Lens Light Models}\label{LensLightModel}

The lens light profile intensities $I$ are computed using one or more elliptical Sersic profiles \citep{Sersic1968}
\begin{equation}
\label{eqn:Sersic}
I_{\rm  Ser} (\xi_{\rm l}) = I \exp \bigg\{ -k \bigg[ \bigg( \frac{\xi}{R} \bigg)^{\frac{1}{n}} - 1 \bigg] \bigg\} ,
\end{equation}
which has seven free parameters: $(x,\,y)$, the light centre, $(\epsilon_{\rm 1},\,\epsilon_{\rm 2})$ the elliptical components, $I$, the intensity at the effective radius $R$ and $n$, the Sersic index. $k$ is a function of $n$ \citep{Trujillo2004}. These parameters are given superscripts depending on which component of the lens galaxy they are modeling, for example the sersic index of the bulge component is $n^{\rm bulge}$. Models with multiple light profiles are evaluated by summing each individual component’s intensities. Up to three light profiles are fitted to the lens galaxy representing a bulge, bulge + disk or bulge + disk + envelope, with their superscript matching these terms. 
The Sersic profile 
intensities are computed using an adaptive oversampling routine which computes all values to a fractional accuracy of $99.99\%$.

Bayesian model comparison is used to determine the light model complexity, from the five models listed in \cref{table:LightModels}. These models assume one, two or three Sersic profiles and make different assumptions for how their centres and elliptical components are aligned. Model comparison is performed separately for the F390W and F814W images. \cref{LensModels} provides the priors of every Sersic profile parameter assumed in this study.

\subsection{Lens Mass Models}

This work fits a variety of lens galaxy mass models, which are summarised in \cref{table:Models}. We fit \textit{decomposed} mass models, where the light profile(s) that represent the lens's light are translated to stellar density profiles (via a mass-to-light profile) to perform ray-tracing \citep{Nightingale2019}.

The lens's light and stellar mass are modeled as a sum of Sersic profiles, where the Sersic profile given by \cref{eqn:Sersic} is used to give the light matter surface density profile
\begin{equation}
\label{eqn:Sersickap}
\kappa_{Ser} (\xi) = \Psi \bigg[\frac{q  \xi}{R}\bigg]^{\Gamma} I_{Ser} (\xi_l) \, \, ,
\end{equation}
where $\Psi$ gives the mass-to-light ratio in electrons per seconds (the units of the HST imaging) and $\Gamma$ folds a radial dependence into the conversion of mass to light. A constant mass-to-light ratio is given for $\Gamma = 0$. If there are multiple light profile components (e.g. a bulge and disk) they assume independent values of $\Psi$ and $\Gamma$. Deflection angles for this profile are computed via an adapted implementation of the method of \citet{Oguri2021}, which decomposes the convergence profile into multiple cored steep elliptical profiles and efficiently computes the deflection angles from each.

Observationally, early-type galaxies are observed to exhibit steep internal gradients in some spectral features associated with dwarf stars. If these features are truly driven by variations in the initial mass function, as advocated by \citet{VanDokkum2017, LaBarbera2019}, then substantial stellar mass-to-light ratio gradients are expected \citep{Ferreras2019}. Some evidence for such trends have indeed been reported by previous lensing studies using decomposed mass models (e.g. \citealt{Oldham2018}). We therefore fit a stellar mass model which allows for different mass-to-light ratios and radial gradients in each stellar component (bulge, disk and envelope). This ensures that we do not incorrectly favour the inclusion of a SMBH, as could otherwise occur if there is no other way for the lens model to increase the amount of mass centrally.

The dark matter component is given by an elliptical Navarro-Frenk-White (NFW) profile. Parameters associated with the lens's dark matter have superscript `dark'. The NFW represents the universal density profile predicted for dark matter halos by cosmological N-body simulations \citep{Zhao1996, Navarro1997} and with a volume mass density given by
\begin{equation}
    \rho = \frac{\rho_{\rm s}^{\rm dark}}{(r/r_{\rm s}^{\rm dark}) (1 + r/r_{\rm s}^{\rm dark})^2}.
\end{equation}
The halo normalization is given by $\rho_{\rm s}^{\rm dark}$ and the scale radius by $r_{\rm s}^{\rm dark}$. The dark matter normalization is parameterized using the mass at 200 times the critical density of the Universe, $M_{\rm 200}^{\rm dark}$, as a free parameter. The scale radius is set via $M_{\rm 200}^{\rm dark}$ using the mean of the mass-concentration relation of \citet{Ludlow2016}, which uses the lens and source redshifts to convert this to units of solar masses. 

The dark matter model has five free parameters: ($x^{\rm dark}$,\,$y^{\rm dark}$), the centre, ($\epsilon_{\rm 1}^{\rm dark}$,\,$\epsilon_{\rm 2}^{\rm dark}$), the elliptical components and; the mass, $M_{\rm 200}^{\rm dark}$. In \cref{ResultSIE} we fit an elliptical NFW using a parameterization which also varies the concentration of the NFW, to test models which can increase the dark matter central density. The deflection angles of the elliptical NFW are computed via the same method used for the Sersic profile \citep{Oguri2021}.

An external shear field is included and parameterized as two elliptical components $(\epsilon_{\rm 1}^{\rm ext},\,\epsilon_{\rm 2}^{\rm ext})$, where parameters associated with the lens's external shear have superscript `ext'. The shear magnitude $\gamma^{\rm  ext}$ and the orientation of the semi-major axis $\theta^{\rm  ext}$, measured counter-clockwise from north, are given by
\begin{equation}
    \label{eq:shear}
    \gamma^{\rm ext} = \sqrt{\epsilon_{\rm 1}^{\rm ext^{2}}+\epsilon_{\rm 2}^{\rm ext^{2}}}, \, \,
    \tan{2\phi^{\rm ext}} = \frac{\epsilon_{\rm 2}^{\rm ext}}{\epsilon_{\rm 1}^{\rm ext}}.
\end{equation}
Deflection angles are computed analytically.

To test for the presence of a SMBH via Bayesian model comparison, every model is fitted with and without a point-mass, whose parameters have superscript `smbh'. This model includes a single free parameter, the Einstein radius $\theta^{\rm smbh}_{\rm Ein}$, which is related to mass as 
\begin{equation}
\label{eqn:PointMass}
M_{\rm BH} = \Sigma_\mathrm{crit} \, \pi \, (\theta^{\rm smbh}_{\rm Ein})^2 \, .
\end{equation}
$\theta^{\rm smbh}_{\rm Ein}$ is in units of arc-seconds and $M_{\rm BH}$ in stellar masses. Point mass deflection angles are computed analytically. The SMBH $(x^{\rm smbh},\,y^{\rm smbh})$ centre is aligned with the highest Sersic index light profile (e.g.\ the bulge) for decomposed mass models.

In \cref{ResultSIE}, we fit \textit{total} mass models that represent all the mass (e.g. stellar plus dark) in a single profile, either the elliptical power-law (PL) \citep{Tessore2016} or the elliptical broken power-law (BPL) introduced by \citet{Oriordan2019, Oriordan2020, Oriordan2021}. Parameters associated with the total mass model have superscript `mass'. For these models the SMBH $(x^{\rm smbh},\,y^{\rm smbh})$ centre is aligned with the centre of the PL or BPL mass profile. The results of fitting this model are summarized in the main paper.
 
We fit a number of additional lens mass models which make different assumptions, in order to verify that none change any of this paper's main results. An additional galaxy is present towards the right of the giant arc, as shown in the first panel of \cref{figure:Data}. In \cref{MassClump}, we include this galaxy in the lens mass model. In \cref{MassCentreFree}, we fit models where the SMBH position is free to vary. In \cref{Radial}, we fit mass models with a shallower inner density profile, which form a large radial critical curve.

\subsection{Source Reconstruction}\label{Source}

After subtracting the foreground lens emission and ray-tracing the coordinates to the source-plane via the mass model, the source is reconstructed in the source-plane using an adaptive Voronoi mesh which accounts for irregular or asymmetric source morphologies (see \cref{figure:ModelFit}). Our results use the \texttt{PyAutoLens} pixelization \texttt{VoronoiBrightnessImage}, which adapts the centres of the Voronoi pixels to the reconstructed source morphology, such that more resolution is dedicated to its brighter central regions (see \citealt{Nightingale2018}).

The reconstruction computes the linear superposition of PSF-smeared source pixel images that best fits the observed image. This uses the matrix $f_{\rm  ij}$, which maps the $j$th pixel of each lensed image to each source pixel $i$. When constructing $f_{\rm  ij}$ we apply image-plane subgridding of degree $4 \times 4$, meaning that $16 \times j$ sub-pixels are fractionally mapped to source pixels with a weighting of $\frac{1}{16}$, removing aliasing effects \citep{Nightingale2015}.

Following the formalism of \citep[][WD03 hereafter]{Warren2003}, we define the data vector $\vec{D}_{i} = \sum_{\rm  j=1}^{J}f_{ij}(d_{j} - b_{j})/\sigma_{j}^2$ and curvature matrix $\tens{F}_{ik} = \sum_{\rm  j=1}^{J}f_{ij}f_{kj}/\sigma_{j}^2$, where $d_{j}$ are the observed image flux values with statistical uncertainties $\sigma{\rm _j}$, and $ b_{j}$ are the model lens light values. The source pixel surface brightnesses values are given by $s = \tens{F}^{-1} \vec{D}$, which are solved via a linear inversion that minimizes
\begin{equation}
\label{eqn:ChiSquared}
\chi^2 = \sum_{\rm  j=1}^{J} \bigg[ \frac{(\sum_{\rm  i=1}^{I} s_{i} f_{ij}) + b_{j} - d_{j}}{\sigma_{j}} \bigg] \, .
\end{equation}
The term $\sum_{\rm  i=1}^{I} s_{i} f_{ij}$ maps the reconstructed source back to the image-plane for comparison with the observed data. 

This matrix inversion is ill-posed, therefore to avoid over-fitting noise the solution is regularized using a linear regularization matrix $\tens{H}$ (see WD03). The matrix $\tens{H}$ applies a prior on the source reconstruction, penalizing solutions where the difference in reconstructed flux of neighboring Voronoi source pixels is large. We use the \texttt{PyAutoLens} regularization scheme \texttt{AdaptiveBrightness}, which adapts the degree of smoothing to the reconstructed source's luminous emission (see \citealt{Nightingale2018}). The degree of smoothing is chosen objectively using the Bayesian formalism introduced by \citet{Suyu2006}. The likelihood function used in this work is taken from \citep{Dye2008a} and is given by
\begin{eqnarray}
-2 \,{  \mathrm{ln}} \, \mathcal{L} &=& \chi^2 + s^{T}\tens{H}s
+{ \mathrm{ln}} \, \left[ { \mathrm{det}} (\tens{F}+\tens{H})\right]
-{ \mathrm{ln}} \, \left[ { \mathrm{det}} (\tens{H})\right]
\nonumber \\
& &
+ \sum_{\rm  j=1}^{J}
{ \mathrm{ln}} \left[2\pi (\sigma{_j})^2 \right]  \, .
\label{eqn:evidence2}
\end{eqnarray}

The step-by-step Jupyter notebooks linked to above describes how the different terms in this likelihood function compare and ranks different source reconstructions, allowing one to objectively determine the lens model that provides the best fit to the data in a Bayesian context.

\subsection{Data Preparation}

In both the F390W and F814W wavebands there is emission from nearby interloper galaxies towards the right of the giant arc, which can be most clearly seen in the upper left panel of \cref{figure:Data}. Including this emission would negatively impact our analysis, therefore we remove it beforehand. Our lens analysis assumes a circular mask of radius 3.7\,arcsec, whereby all image-pixels outside this circular region are not included in the fitting procedure. The central panels of \cref{figure:Data} show that this mask removes the majority of foreground emission, however a small fraction is still within this circle. We therefore subtract it using a graphical user interface, replacing it with background noise in the image and increasing the RMS noise-map values of these pixels to ensure they do not contribute to the likelihood function. We also consider lens models which include this galaxy in the ray-tracing (see \cref{MassClump}).

\subsection{Light Model Waveband}\label{F390WModel}

The wavelength at which the lens galaxy's emission is observed is important for tracing its stellar mass distribution. The F390W image of Abell 1201 observes the lens galaxy at rest-frame ultra-violet wavelengths, possibly probing younger stellar populations with lower mass-to-light ratios. The F814W image observes rest-frame near infrared (NIR) emission and probes more aged and reddened stellar populations which make up a greater fraction of the stellar mass. This can be seen in \cref{figure:Data}, where only the central regions of the lens are visible in the F390W image compared to the F814W image. The F390W image is therefore less appropriate for constraining the stellar mass component of the lens model. 

Therefore, to fit the decomposed mass model to the F390W image we use the maximum likelihood Sersic light model parameters of the F814W fits that are chosen after the lens light Bayesian model comparison (see \cref{LensLightModel}). The mass-to-light ratio and gradient parameters of each Sersic remain free to vary, ensuring high flexibility in the model's stellar mass distribution. Fits are performed using a lens light subtracted image for the F390W image which is output midway through the analysis. To ease the comparison between fits to the F390W and F814W images, we follow the same approach with the F814W image, using the same fixed maximum likelihood Sersic parameters and fitting a lens subtracted image output midway through the analysis.

\begin{figure*}
\centering
\includegraphics[width=0.24\textwidth]{light_sub/f390w/x2_data.pdf}
\includegraphics[width=0.24\textwidth]{light_sub/f390w/x2_image.pdf}
\includegraphics[width=0.24\textwidth]{light_sub/f390w/x2_norm.pdf}
\includegraphics[width=0.24\textwidth]{light_sub/f390w/x2_norm_zoomed.pdf}
\includegraphics[width=0.24\textwidth]{light_sub/f814w/x3_data.pdf}
\includegraphics[width=0.24\textwidth]{light_sub/f814w/x3_image.pdf}
\includegraphics[width=0.24\textwidth]{light_sub/f814w/x3_norm.pdf}
\includegraphics[width=0.24\textwidth]{light_sub/f814w/x3_norm_zoomed.pdf}
\caption{Lens light subtractions of HST imaging of Abell 1201. The observed images (left column), image-plane model images of the lens and source galaxies (left-centre column), normalized residuals (right-centre column) and a zoom-in of these residuals near the counter image (right column) are shown. The top row shows the F390W and bottom row the F814W wavebands respectively. The lens model is the maximum likelihood model inferred at the end of the first SLaM pipeline run, which produces a lens subtracted image. For the F390W data, a double Sersic lens light model with offset centres and elliptical components is shown, whereas for the F814W data a triple Sersic model is used. The magenta circle indicates residuals that are due to the lens light subtraction. The counter image is fitted poorly in the F390W image, because the mass model (which is an isothermal mass model with shear) does not enable an accurate reconstruction of the lensed source's structure.} 
\label{figure:LightFit2}
\end{figure*}

\subsection{SLaM Pipelines}\label{SLAM}

The models of lens mass, lens light and source light are complex and their parameter spaces highly dimensional. Without human intervention or careful set up, a model-fitting algorithm (e.g. a Markov chain Monte Carlo search) may converge very slowly to the global maximum likelihood solution. 
\texttt{PyAutoLens} therefore applies `non-linear search chaining' to break the search into a sequence of tractable operations.
Using the probabilistic programming language {\tt PyAutoFit} \github{https://github.com/rhayes777/PyAutoFit}, we fit a series of parametric lens models that approximate the form of the desired model, with growing complexity. A fit to the simplest model provides information to initialise a fit to the next model. The final search is started around the global maximum likelihood and with priors reflecting the likelihood surface. Each fit in this chain uses the nested sampler \citep[{\tt dynesty}][]{Speagle2020} \github{https://github.com/joshspeagle/dynesty}.
The models used to perform this analysis extend the Source, Light and Mass (SLaM) pipelines described by \citet[][hereafter E22]{Etherington2022}, \citet{Cao2021} and \citet{He2023}. They are available at \url{https://github.com/Jammy2211/autolens_workspace}.

The first pipeline, called the Source pipeline, initializes the pixelized source model by inferring a robust lens light subtraction (using a double Sersic model) and total mass model (using a PL with $\gamma=2$ plus shear). The highest likelihood lens model and source reconstruction at this stage of the pipeline are shown in \cref{figure:ModelFit}. They give an accurate foreground lens subtraction and reconstruction of the lensed source's light. 

The Light pipeline follows, which uses fixed values of the mass and source parameters corresponding to the maximum likelihood model of the Source pipeline. The lens's mass is therefore again fitted using a total mass model such that the lens light model does not yet contribute to the ray-tracing. The only free parameters in this pipeline are those of the lens light and all five of the models listed in \cref{table:LightModels} are fitted independently, enabling Bayesian model comparison. The results of the Light pipeline, including the models chosen for all subsequent model fits, are presented in \cref{LightModels}.

The final pipeline is the Mass pipeline, which in \citetalias{Etherington2022} directly follows the Light Pipeline, fitting PL mass profiles representing the total mass distribution. In this work, we do not immediately start the Mass pipeline after the Light pipeline, due to the complications of fitting the stellar component of the decomposed models to the F390W imaging data discussed previously (see \cref{F390WModel}). Instead, the lens light models favoured by model comparison are used to output lens-subtracted F390W and F814W images. An analysis of these images is then performed from scratch, starting a new SLaM pipeline fit that uses a scaled down Source pipeline, which removes models that fit the lens light, and which omits the Light pipeline completely (see \citetalias{Etherington2022}). 

When this analysis reaches the Mass pipeline, it fits the decomposed models (models assuming two or three Sersic profiles for the lens light and stellar mass) and the total mass models (the PL and BPL), whose results are described in \cref{ResultSIE}. Every mass model is fitted twice, with and without a point mass representing a SMBH. The Bayesian model comparison of these mass models is the main component of this work's results.

As described in \citetalias{Etherington2022}, the SLaM pipelines use prior passing to initialize the regions of parameter space that \texttt{dynesty} will search in later pipelines, based on the results of earlier pipelines. \cref{LensModels} gives a description of the priors used in this work. We also use the likelihood cap analysis described in \citetalias{Etherington2022} to estimate errors on lens model parameters.

\subsection{Bayesian Evidences}\label{BayesEvi}

The Bayesian evidence, $\mathcal{Z}$, of every lens model we fit is estimated by \texttt{dynesty} and is given by equation 2 of \citet{Speagle2020}. The Bayesian evidence is the integral over all parameters in the model and therefore naturally includes a penalty term for including too much complexity in a model – if a model has more free parameters it is penalized for this complexity. The evidence is computed via sampling of Eq. (9). Our analysis therefore incorporates the principle of Occam's razor, whereby more complex models are only favoured if they improve the fit enough to justify their additional complexity compared a simpler model. To compare models, we use the difference in log evidence, $\Delta \ln \mathcal{Z}$. An increase of $\Delta \ln \mathcal{Z} = 4.5$ for one model over another corresponds to odds of 90:1 in favour of that model. For comparisons of lens models with and without a SMBH this corresponds to a $3\sigma$ detection of the SMBH. An increase of $\Delta \ln \mathcal{Z} = 11$ corresponds to a $5\sigma$ detection. 

However, there are sources of uncertainty in the evidence estimate that means taking these numbers at face value is problematic. For example, there is an error on the evidence estimated by \texttt{dynesty}, with identical runs of a lens model showing variations of $\ln \mathcal{Z} \sim 5$ (due to stochasticity in the \texttt{dynesty} sampling process). Adjusting the priors on the lens model parameters or reparameterizing the model also change its value, with tests showing variations up to $\ln \mathcal{Z} \sim 5$. Accordingly, we consider values of $\Delta \ln \mathcal{Z} > 10 $ sufficient to favour more complex models over simpler ones, including the detection of a SMBH.