\section{Results}\label{Results}

\begin{figure*}
\centering
\includegraphics[width=0.195\textwidth]{light_dark_fit_x3/f390w/grad_data.pdf}
\includegraphics[width=0.195\textwidth]{light_dark_fit_x3/f390w/grad_image.pdf}
\includegraphics[width=0.195\textwidth]{light_dark_fit_x3/f390w/grad_norm.pdf}
\includegraphics[width=0.195\textwidth]{light_dark_fit_x3/f390w/grad_smbh_image.pdf}
\includegraphics[width=0.195\textwidth]{light_dark_fit_x3/f390w/grad_smbh_norm.pdf}
\includegraphics[width=0.195\textwidth]{light_dark_fit_x3/f814w/grad_data.pdf}
\includegraphics[width=0.195\textwidth]{light_dark_fit_x3/f814w/grad_image.pdf}
\includegraphics[width=0.195\textwidth]{light_dark_fit_x3/f814w/grad_norm.pdf}
\includegraphics[width=0.195\textwidth]{light_dark_fit_x3/f814w/grad_smbh_image.pdf}
\includegraphics[width=0.195\textwidth]{light_dark_fit_x3/f814w/grad_smbh_norm.pdf}
\caption{The observed image (left panel), model lensed source and normalized residuals for decomposed model-fits without a SMBH (left-centre and centre panels) and with a SMBH (right-centre and right panel). The top row shows fits to the F390W image and bottom row the F814W image. Minimal residuals are seen in the central regions, indicating the lens light model and subtraction are accurate (the lens light models are visualized in \cref{figure:Light2D}). The magenta circle marks regions of the data where the brightest regions of the lens light were observed and subtracted. The tangential caustic is shown by a black line and radial critical curve and caustic a white line; the latter does not form for models including a SMBH. Source reconstructions with and without a SMBH successfully reproduce the giant arc and counter image, although residuals are present in both indicating their detailed structure is not fitted accurately. Figures \ref{figure:LightDarkF390Wx3} and \ref{figure:LightDarkF814Wx3} show zoom-ins of the counter image to better illustrate how these different models reconstruct the data.} 
\label{figure:LightDarkF390Wx3Global}
\end{figure*}

\setlength{\tabcolsep}{1.0pt}

\begin{tabular}{ l c c | l c c c}

\toprule
Backbone & Params & GFLOPs & Tracker & MOTA$\uparrow$ & IDF1$\uparrow$ & IDs$\downarrow$ \\
\midrule
YOLOX-M & 25.3 M & 118.7 & DeepSORT & 74.5 & 76.2 & 197\\
YOLOX-M & 25.3 M & 118.7 & BYTE & 75.3 & 77.5 & 200\\
YOLOX-S & 8.9 M & 43.0 & DeepSORT & 69.6 & 71.5 & 205\\
YOLOX-S & 8.9 M & 43.0 & BYTE & 71.1 & 73.6 & 224\\
YOLOX-Tiny & 5.0 M & 24.5 & DeepSORT & 68.6 & 72.0 & 224\\
YOLOX-Tiny & 5.0 M & 24.5 & BYTE & 70.5 & 72.1 & 222\\
YOLOX-Nano & 0.9 M & 4.0 & DeepSORT & 61.4 & 66.8 & 212\\
YOLOX-Nano & 0.9 M & 4.0 & BYTE & 64.4 & 68.4 & 161\\
\bottomrule
\end{tabular}

We now present the results of lens modeling of Abell 1201. We first examine the preferred choice of lens light models, inferred using an isothermal mass model that omits a SMBH. Then we present results using the more complex stellar plus dark matter decomposed mass models, which may also include a SMBH. We discuss additional mass models which assume a total mass profile. In each case we examine the reconstruction of the near-centre counter image that is highly sensitive to the central mass distribution and SMBH, as well as the quantitative Bayesian evidence, $\mathcal{Z}$.

\subsection{Lens Light Model}

The choice of lens light model via Bayesian model comparison is described in \cref{LightModels} and summarized as follows:

\begin{itemize}
 \item All light models with two or three Sersic profiles are favoured over models with one Sersic, producing Bayesian evidence increases of $\Delta \ln \mathcal{Z} > 300$.
 \item The two Sersic models whose centres, position angles, and axis ratios are unaligned produce values of $\Delta \ln \mathcal{Z} > 100$ compared to two Sersic models which assume alignment.
 \item For the F814W image, the three-Sersic model marginally gives the highest evidence overall, where $\Delta \ln \mathcal{Z} = 12$ compared with the double Sersic with unaligned geometric parameters. We use this image to create the lens light subtracted image that mass models are fitted to.
 \item The triple Sersic could not be constrained in the F390W band, owing to its observed lower rest-frame wavelength. We therefore use the double Sersic with aligned parameters to create the F390W lens light subtracted image.
\end{itemize}

\begin{figure*}
\centering
\includegraphics[width=0.241\textwidth]{light_dark_fit_x3/f390w/grad_data_zoomed.pdf}
\includegraphics[width=0.241\textwidth]{light_dark_fit_x3/f390w/grad_image_zoomed.pdf}
\includegraphics[width=0.241\textwidth]{light_dark_fit_x3/f390w/grad_norm_zoomed.pdf}
\includegraphics[width=0.241\textwidth]{light_dark_fit_x3/f390w/grad_source_recon.pdf}
\includegraphics[width=0.241\textwidth]{light_dark_fit_x3/f390w/grad_smbh_data_zoomed.pdf}
\includegraphics[width=0.241\textwidth]{light_dark_fit_x3/f390w/grad_smbh_image_zoomed.pdf}
\includegraphics[width=0.241\textwidth]{light_dark_fit_x3/f390w/grad_smbh_norm_zoomed.pdf}
\includegraphics[width=0.241\textwidth]{light_dark_fit_x3/f390w/grad_smbh_source_recon.pdf}
\caption{
Zoom-ins of the observed counter image in the F390W data (left panel), the model lensed source (left-centre panel), the normalized residuals (right-centre panel) and the source reconstruction (right panel). The top and bottom rows show triple Sersic plus NFW decomposed model-fits without and with a SMBH respectively. All models include an external shear. The magenta circle marks regions of the data where the brightest regions of the lens light were observed and subtracted. Models which omit a SMBH form extraneous light in the reconstructed counter image (which is seen just inside the magenta circle), which is not present in the data. The tangential caustic is shown by a black line and the radial critical curve and caustic are shown with a white line; the latter does not form for models including a SMBH.
} 
\label{figure:LightDarkF390Wx3}
\end{figure*}

\cref{figure:LightFit2} shows the highest evidence lens light model fits to the F390W and F814W images. A good fit to the lens galaxy's emission and a clean subtraction is seen. \cref{table:ModelsLight} gives a subset of inferred parameters for fits to the F814W data and the full results of lens light model comparison are presented in \cref{LightModels}.

A small magenta circle is plotted on this figure and subsequent figures to indicate where the centre of the lens galaxy is. Within this magenta circle faint correlated residuals due to a slightly imperfect lens light subtraction can be seen. These are more visible in the F814W image, which is expected given the lens stellar emission is much brighter. The residuals appear as a dipole-like feature, which is commonly seen for lens light subtractions of HST imaging of strong lenses (e.g. \citealt{Etherington2022}). 

We considered whether these residuals might be a central image of the lensed source galaxy, but in this case the feature would be much brighter in the F390W image. Dust absorption could lower the F390W emission, however HST F606W observations of Abell 1201 also show no central emission \citet{Smith2017a}, making dust absorption unlikely. In \cref{Radial} we fit mass models with priors manually tuned to include a centrally cored mass profile, which for the F814W (or the F390W) data do not reconstruct this central emission. Lens modeling therefore confirms it is not a central image.

\subsection{Decomposed Mass Models}\label{Decomp}

\begin{table}
\resizebox{\linewidth}{!}{
\begin{tabular}{ l | l | l | l } 
\multicolumn{1}{p{1.1cm}|}{Filter} 
& \multicolumn{1}{p{1.3cm}|}{Number of Sersics} 
& \multicolumn{1}{p{1.3cm}|}{Includes SMBH?} 
& \multicolumn{1}{p{1.5cm}|}{$\ln \mathcal{Z}$}  
\\ \hline
F390W & 2 & \ding{55} & 125637.18  \\[1pt]
F390W & 2 & \checkmark & 125669.13  \\[0pt]
\hline
F390W & 3 & \ding{55} & 125598.48  \\[1pt]
F390W & 3 & \checkmark & \textbf{125699.06} \\[0pt]
\hline
F814W & 2 & \ding{55} & 78330.51  \\[1pt]
F814W & 2 & \checkmark & 78328.12  \\[0pt]
\hline
F814W & 3 & \ding{55} & 78329.19   \\[1pt]
F814W & 3 & \checkmark & 78332.19 \\[0pt]
\end{tabular}
}
\caption{The Bayesian Evidence, $\ln \mathcal{Z}$, of each model-fit performed by the Mass pipelines using decomposed mass models that assume two and three Sersic profiles, an elliptical NFW and external shear. Fits to both the F390W and F814W images are shown, where the F390W fits assume the Sersic parameters of the F814W image for the stellar mass. The favoured model given our criteria of $\Delta \ln \mathcal{Z} > 10$ is shown in bold. For the F390W image, all models with a SMBH produce $\Delta \ln \mathcal{Z}$ values of at least $30$ above models without a SMBH.}
\label{table:SMBHMCDecomp}
\end{table}

\begin{figure*}
\centering
\includegraphics[width=0.241\textwidth]{light_dark_fit_x3/f814w/grad_data_zoomed.pdf}
\includegraphics[width=0.241\textwidth]{light_dark_fit_x3/f814w/grad_image_zoomed.pdf}
\includegraphics[width=0.241\textwidth]{light_dark_fit_x3/f814w/grad_norm_zoomed.pdf}
\includegraphics[width=0.241\textwidth]{light_dark_fit_x3/f814w/grad_source_recon.pdf}
\includegraphics[width=0.241\textwidth]{light_dark_fit_x3/f814w/grad_smbh_data_zoomed.pdf}
\includegraphics[width=0.241\textwidth]{light_dark_fit_x3/f814w/grad_smbh_image_zoomed.pdf}
\includegraphics[width=0.241\textwidth]{light_dark_fit_x3/f814w/grad_smbh_norm_zoomed.pdf}
\includegraphics[width=0.241\textwidth]{light_dark_fit_x3/f814w/grad_smbh_source_recon.pdf}
\caption{
The same as \cref{figure:LightDarkF390Wx3} but for the F814W data.
} 
\label{figure:LightDarkF814Wx3}
\end{figure*}



\begin{table*}
\tiny
\resizebox{\linewidth}{!}{
\begin{tabular}{ l l l l l l l l l l} 
\multicolumn{1}{p{1.8cm}|}{\centering \textbf{Model}} 
& \multicolumn{1}{p{1.0cm}}{$\Psi^{\rm{bulge}}$ (e$^{\rm -}$\,s$^{\rm -1}$)} 
& \multicolumn{1}{p{1.0cm}}{$\Psi^{\rm{disk}}$ (e$^{\rm -}$\,s$^{\rm -1}$)} 
& \multicolumn{1}{p{1.0cm}}{$\Psi^{\rm{envelope}}$ (e$^{\rm -}$\,s$^{\rm -1}$)} 
& \multicolumn{1}{p{1.0cm}}{$\Gamma^{\rm{bulge}}$} 
& \multicolumn{1}{p{1.0cm}}{$\Gamma^{\rm{disk}}$} 
& \multicolumn{1}{p{1.0cm}}{$\Gamma^{\rm{envelope}}$} 
& \multicolumn{1}{p{1.0cm}}{$\epsilon_{\rm 1}^{\rm{ext}}$} 
& \multicolumn{1}{p{1.0cm}}{$\epsilon_{\rm 2}^{\rm{ext}}$} 
& \multicolumn{1}{p{1.0cm}}{$\theta_{\rm Ein}^{\rm{smbh}}$(\arcsec)} 
\\ \hline
& & & & & & \\[-4pt]

x3 Sersic & 
$1.47^{+0.69}_{-0.55}$ & 
$1.14^{+0.44}_{-0.28}$ & 
$2.32^{+1.05}_{-0.95}$ & 
$0.57^{+0.21}_{-0.32}$ & 
$0.34^{+0.19}_{-0.18}$ & 
$0.12^{+0.08}_{-0.12}$ & 
$-0.097^{+0.042}_{-0.044}$ & 
$0.19^{+0.05}_{-0.07}$  \\[-6pt]

\\ \hline
& & & & & & \\[-2pt]

x2 Sersic & 
$1.65^{+0.69}_{-0.61}$ & 
$0.95^{+0.74}_{-0.30}$ & 
& 
$0.46^{+0.35}_{-0.34}$ & 
$0.29^{+0.16}_{-0.14}$ & 
&
$-0.084^{+0.061}_{-0.057}$ & 
$0.13^{+0.06}_{-0.06}$ \\[-6pt]

\\ \hline
& & & & & & \\[-2pt]

x3 Sersic + SMBH & 
$0.54^{+0.26}_{-0.38}$ & 
$1.45^{+0.55}_{-0.33}$ & 
$1.69^{+0.63}_{-1.05}$ & 
$0.50^{+0.37}_{-0.25}$ & 
$0.41^{+0.14}_{-0.11}$ & 
$0.33^{+0.20}_{-0.12}$ & 
$-0.14^{+0.02}_{-0.02}$ & 
$0.22^{+0.02}_{-0.04}$ & 
$0.48^{+0.07}_{-0.11}$ \\[-6pt]

\\ \hline
& & & & & & \\[-2pt]
x2 Sersic + SMBH & 
$0.87^{+0.61}_{-0.80}$ & 
$1.25^{+0.57}_{-0.46}$ & 
& 
$0.15^{+0.65}_{-1.12}$ & 
$0.48^{+0.12}_{-0.18}$ & 
&
$-0.14^{+0.04}_{-0.03}$ & 
$0.20^{+0.03}_{-0.04}$ & 
$0.42^{+0.09}_{-0.17}$ \\[-2pt]
\end{tabular}
}
\caption{
The inferred stellar mass, shear and SMBH model parameters of the decomposed mass models with two and three Sersic components fitted to the F390W image in the Mass pipeline. Errors are given at 3$\sigma$ confidence intervals.
}
\label{table:ModelsDecompStellar}
\end{table*}






We now present results using decomposed mass models that separately model Abell 1201's stellar and dark matter. Based on the lens light model comparison, we fit models assuming both two and three Sersic profiles (where parameters for the F390W fits use those inferred from fits to the F814W, see \cref{F390WModel}). We fit both models independently to both the F390W and F814W images. Visualization in this section shows the triple Sersic fits, \cref{MassFits} shows figures for the double Sersic fits. 

\cref{table:SMBHMCDecomp} shows the values of $\ln \mathcal{Z}$ for decomposed models with and without a SMBH. Values of $\Delta \ln \mathcal{Z} > 30$ are seen for all model fits to the F390W image with a SMBH compared to those without. The highest overall value of $\ln \mathcal{Z}$ is the triple Sersic decomposed mass model with a SMBH, which is a $\ln \mathcal{Z}$ value more than 60 greater than that for any decomposed model without a SMBH. For the F814W images all models produce nearly consistent values of $\ln \mathcal{Z}$ with or without a SMBH, indicating that the higher S/N of the F390W data or the source's different structure is enabling the SMBH detection. 

\cref{figure:LightDarkF390Wx3Global} shows the reconstructed lensed sources and normalized residuals for fits to the F390W and F814W images with and without a SMBH. All models reproduce the giant arc and counter image. Residuals are seen around the giant arc in the F390W image indicating missing complexity in the mass model. These residuals are seen across all mass models compared in this work (including fits which include the mass of the line-of-sight galaxy to the right of the giant arc, see \cref{MassClump}). We therefore do not anticipate they impact our inference on the SMBH. The reconstructed counter images for the models with and without a SMBH are visibly distinct and they produce different residuals, albeit this is difficult to discern from \cref{figure:LightDarkF390Wx3Global} due to the large arc-second scales over which the image is plotted.

\begin{table*}
\tiny
\resizebox{\linewidth}{!}{
\begin{tabular}{ l l l l l l} 
\multicolumn{1}{p{1.8cm}|}{\centering \textbf{Model}} 
& \multicolumn{1}{p{1.2cm}}{$x^{\rm{dark}}$(\arcsec)} 
& \multicolumn{1}{p{1.2cm}}{$y^{\rm{dark}}$(\arcsec)} 
& \multicolumn{1}{p{1.2cm}}{$\epsilon_{\rm 1}^{\rm{dark}}$} 
& \multicolumn{1}{p{1.2cm}}{$\epsilon_{\rm 2}^{\rm{dark}}$} 
& \multicolumn{1}{p{2.5cm}}{$M_{\rm 200}^{\rm{dark}}$ ($M_{\rm \odot} \times 10^{14}$)} 
\\ \hline
& & & & & \\[-7pt]

x3 Sersic & 
$0.048^{+0.108}_{-0.126}$ & 
$0.050^{+0.100}_{-0.121}$ & 
$0.022^{+0.110}_{-0.090}$ & 
$0.13^{+0.12}_{-0.15}$ & 
$3.43^{+1.24}_{-7.76}$  \\[-7pt]

\\ \hline
& & & & & \\[-7pt]

x2 Sersic & 
$0.031^{+0.059}_{-0.058}$ & 
$0.11^{+0.17}_{-0.12}$ & 
$0.032^{+0.084}_{-0.106}$ & 
$ 0.017^{+0.119}_{-0.097}$ &  
$5.55^{+3.29}_{-2.19}$ \\[-7pt]

\\ \hline
& & & & & \\[-7pt]

x3 Sersic + SMBH & 
$0.13^{+0.08}_{-0.17}$ & 
$0.17^{+0.03}_{-0.10}$ & 
$-0.036^{+0.066}_{-0.115}$ & 
$ 0.19^{+0.12}_{-0.16}$ & 
$1.43^{+6.48}_{-6.28}$ \\[-7pt]

\\ \hline
& & & & & \\[-7pt]
x2 Sersic + SMBH & 
$-0.22^{+0.12}_{-0.12}$ & 
$0.008^{+0.084}_{-0.088}$ & 
$-0.015^{+0.086}_{-0.159}$ & 
$ 0.095^{+0.094}_{-0.067}$ 
& $1.69^{+8.31}_{-4.37}$  \\[-2pt]
\end{tabular}
}
\caption{
The inferred dark matter model parameters of the decomposed mass models with two and three Sersic components fitted to the F390W image in the Mass pipeline. Errors are given at 3$\sigma$ confidence intervals.
}
\label{table:ModelsDecompDark}
\end{table*}

\cref{figure:LightDarkF390Wx3} therefore shows zoom-ins around the counter image for the F390W image, where models without and with a SMBH are shown on the top and bottom rows respectively. The model without a SMBH places extraneous flux in the reconstructed counter image, which is not present when the SMBH is included (this flux can be seen within the radial critical curve shown by a white line and the magenta circle, and is not related to the lens light residuals). \cref{figure:LightDarkF814Wx3} shows zoom-ins for the F814W image where the same extraneous flux is seen for the model not including a SMBH. The same behaviour is seen in \cref{MassFits} for the decomposed models which fits two Sersics instead of three. The inclusion of the SMBH therefore allows the counter image to be reconstructed more accurately in both wavebands, removing central luminous emission that is not observed in the data. Removing this extraneous flux increases $\ln \mathcal{Z}$ for the F390W image data only, implying that the F814W data are too low S/N for changes in the counter image reconstruction to improve the fit in a Bayesian sense.

\begin{figure}
\centering
\includegraphics[width=0.47\textwidth]{light_dark_1d/f390w_light_dark_1d_x3.pdf}
\caption{
The convergence as a function of radius inferred using the F390W image for the decomposed mass models which assume three Sersic profiles, without a SMBH (black) and with a SMBH (red). All models include an external shear. Each line is computed using coordinates that extend radially outwards from the centre of the mass profile and are aligned with its major axis. Shaded regions for each mass model's convergence are shown, corresponding to the inferred $3\sigma$ confidence intervals. The 1D convergence of the SMBH is not included, to make comparison of each mass model's convergence straightforward. 
} 
\label{figure:LightDark1Dx3}
\end{figure}

The inferred 1D convergence profiles for the decomposed models with three Sersic profiles are shown in \cref{figure:LightDark1Dx3}. When a SMBH is included the inferred mass model convergence is shallower. Increasing the central density of the lens galaxy's mass model therefore produces a similar lensing effect to including a SMBH and is an alternative way to improve the counter image fit. We will expand on this further when we discuss the total mass model fits.

\begin{figure*}
\centering
\includegraphics[width=0.98\textwidth]{pdf/grad_smbh_image_base.png}
\caption{
The 2D probability density functions (PDF) of fits to the F390W image of Abell 1201 using the triple Sersic decomposed mass model. Marginalized 2D contours are shown for every lens mass model parameter paired with the SMBH normalization $\theta^{\rm smbh}_{\rm Ein}$ which is related to $M_{\rm BH}$, see \cref{eqn:PointMass}. The inner and outer contours cover the 1 and 2$\sigma$ confidence intervals respectively.
}
\label{figure:LightDarkPDFx3}
\end{figure*}

\cref{figure:LightDarkPDFx3} shows the 2D probability density functions (PDF) of the mass model parameters and the SMBH normalization $\theta^{\rm smbh}_{\rm Ein}$ for fits to the F390W image using the triple Sersic decomposed models. $\theta^{\rm smbh}_{\rm Ein}$ depends on the parameters controlling the mass distribution (e.g., $\Gamma^{\rm bulge}$, $\Gamma^{\rm disk}$, $\Gamma^{\rm env}$). 

\cref{table:ModelsDecompStellar} and \cref{table:ModelsDecompDark} give the inferred parameter estimates of the decomposed models. We can compare our inferred dark matter halo mass to the virial mass estimate of \citet{Rines2013} (from the infall caustic method), which for an NFW dark matter halo gives $M_{\rm 200} = 3.9 \pm 0.1 \times 10^{14} M_{\rm \odot}$. \cref{table:ModelsDecompDark} shows our estimates of $M_{\rm 200}$ range between $M_{\rm 200} = 0 - 5 \times 10^{14} M_{\rm \odot}$. Both models with a SMBH are consistent with \citet{Rines2013}. Our lens model is therefore inferring a realistic dark matter host halo.  

\subsection{Total Mass Models}

The results of fitting total mass models which represent the stars and dark matter with a single projected mass distribution are given in \cref{ResultSIE}. For the power-law (PL) mass model, which has reduced flexibility in adjusting its central density, the inferred $\ln \mathcal{Z}$ values without a SMBH are over 100 below models including a SMBH (PL or decomposed). When the PL includes a SMBH, $\ln \mathcal{Z}$ increases to within $\sim 10$ of the decomposed models with a SMBH. The PL fits therefore strongly favour a SMBH. The counter image reconstructions also reflect those seen above, whereby PL models without a SMBH show extraneous flux which is removed when a SMBH is included. 

For the broken power-law (BPL), which has more flexibility in adjusting its central density, the model without a SMBH infers $\ln \mathcal{Z} = 125699.90$ for the F390W data. This is within $10$ of the highest evidence decomposed and PL models with a SMBH. This model also reconstructs the counter image without extraneous flux. In a Bayesian sense, the BPL model without a SMBH is therefore as likely as any model fitted in this work with a SMBH, calling into question whether a SMBH is necessary in the lens model.  

The inferred BPL model increases its central density above any decomposed model inferred in \cref{Decomp}. In \cref{ResultSIE}, we therefore verify that the decomposed model parameterization can attain the same central density as the BPL. We show that it does when the bulge's radial gradient parameter is increased to $\Gamma^{\rm bulge} = 0.9$. The reason we do not infer this model is because this model is lower likelihood than models inferred above, where $\Gamma^{\rm bulge} = \sim 0.5$, indicating that increasing the stellar mass density produces a different lensing effect to including a SMBH. In \cref{ResultSIE} we also fit models where the dark matter concentration is free to vary, such that it can reach the same central density as the BPL. These models again do not produce solutions with as high an evidence as those presented above.

For the high density BPL model to fit the data as well as the mass models with a SMBH, its ($x^{\rm mass}$,\,$y^{\rm mass}$) centre assumes values that are $\geq 100$\,pc offset from the centre of the bulge's luminous emission. In \cref{ResultSIE}, we show that if the BPL model centre is aligned with the luminous bulge it produces a much lower $\ln \mathcal{Z}$. The BPL model is built-on the assumption that it can simultaneously represent both the stellar and dark matter mass distributions \citep{Oriordan2019}. Therefore, on the grounds that a $100$\,pc offset between the light and total mass distribution is non-physical and breaks the underlying assumption on which the BPL is built, we favour models including a SMBH which do not require this offset. 

\subsection{Alternative Models}

We verify that the inclusion of a SMBH is still favoured for a number of alternative lens galaxy mass models. In \cref{MassClump}, we include the ray-tracing effects of the line-of-sight galaxy to the north-east of the giant arc, by modelling it as a singular isothermal sphere. In \cref{MassCentreFree}, we fit lens models which allow the centre of the SMBH to vary as a free parameter. In \cref{Radial}, we explore a family of solutions where the lens mass model has a shallow (or cored) inner density, therefore forming a larger radial critical curve than those inferred in the main paper. For all alternative models, a SMBH is favoured with the same or greater significance than shown for the models above.

\subsection{SMBH Mass}

\begin{figure}
\centering
\includegraphics[width=0.48\textwidth]{pdf/smbh_1d_mass_pdf.png}
\caption{
The 1D probability density functions (PDF) of the SMBH mass $M_{\rm BH}$ for fits to the F390W image of Abell 1201. Inferred values of $M_{\rm BH}$ are shown for the decomposed mass model with three Sersics, two Sersics and the power-law total mass model. The broken power-law fitted in \cref{ResultSIE} and discarded due to its nonphysical 100pc offset.
}
\label{figure:SMBHPDF}
\end{figure}

The 1D PDFs for $M_{\rm BH}$ for the decomposed two and three Sersic models and PL total mass model are shown in \cref{figure:SMBHPDF}. At $3\sigma$ confidence, the SMBH mass inferred for fits to the F390W image for different mass models (excluding the BPL due to its nonphysical offset centre) are: 

\begin{itemize}
    \item $M_{\rm BH} = 2.22^{+1.41}_{-1.06}\times10^{10}$\,M$_{\rm \odot}$ for the triple Sersic decomposed model.
    \item $M_{\rm BH} = 2.91^{+1.17}_{-0.91}\times10^{10}$\,M$_{\rm \odot}$ for double Sersic fits.
    \item $M_{\rm BH} = 3.95^{+1.44}_{-1.47}\times10^{10}$\,M$_{\rm \odot}$ for the PL model.
\end{itemize}

To estimate a final value of $M_{\rm BH}$ we simply estimate the value which spans the full range of measurements, producing $M_{\rm BH} = 3.27 \pm 2.12\times10^{10}$\,M$_{\rm \odot}$ at $3\sigma$ confidence.

\subsection{Upper Limit Analysis}

Although we have discarded the BPL model on the grounds of physical plausibility, it can still be used to place an upper limit on $M_{\rm BH}$, even with the offset centre. Once a SMBH of sufficiently high mass is included in the mass model, it deforms the counter image reconstruction in a way which cannot be compensated for by reducing the inner density of the mass model. To demonstrate this, \cref{figure:WithSMBH} shows the reconstructed counter images of a BPL model fit without a SMBH and with a SMBH whose mass is fixed to $M_{\rm BH} = 10^{11}$\,M$_{\rm \odot}$. The SMBH displaces the counter image, producing a reconstruction that is not consistent with the observed data.

The value $M_{\rm BH} = 10^{11}$\,M$_{\rm \odot}$ was chosen to visually emphasise how a high mass SMBH disfigures the counter image. We can fit a grid of BPL plus SMBH models where $M_{\rm BH}$ is fixed to incrementally higher values between $1 - 10 \times 10^{10}$\,M$_{\rm \odot}$ to the F390W data. \cref{table:SMBHUpperLimit} shows the $\ln \mathcal{Z}$ values for each fit, where a drop of $\ln \mathcal{Z} = 20$ is seen above masses of $M_{\rm BH} = 5.3 \times 10^{10}$\,M$_{\rm \odot}$. The BPL model with a nonphysical offset centre therefore still places an upper limit of $M_{\rm BH} \leq 5.3 \times 10^{10}$\,M$_{\rm \odot}$.

Whilst in this study Abell 1201's counter image contains sufficient information to provide a measurement of $M_{\rm BH}$, in less fortuitous circumstances upper limits on $M_{\rm BH}$ will still be possible in many strong lenses.

\begin{table}
\tiny
\resizebox{\linewidth}{!}{
\begin{tabular}{ l | l | l} 
\multicolumn{1}{p{1.2cm}}{$\theta_{\rm Ein}^{\rm{smbh}}$ (\arcsec)} 
& \multicolumn{1}{p{2.0cm}}{$M_{\rm BH}$ ($M_{\rm \odot} \times 10^{10}$)} 
& \multicolumn{1}{p{1.0cm}|}{$\ln \mathcal{Z}$}  
\\ \hline
 None  & 0.0   & 125699.90 \\[2pt]
 0.2   & 0.513 & 125706.18 \\[2pt]
 0.3   & 1.145 & 125693.19 \\[2pt]
 0.4   & 2.030 & 125686.63 \\[2pt]
 0.5   & 3.168 & 125657.25 \\[2pt]
 0.6   & 4.557 & 125676.59  \\[2pt]
 0.625 & 4.945 & 125699.68  \\[2pt]
 0.65  & 5.349 & \textbf{125655.86}  \\[2pt]
 0.675 & 5.765 & 125676.04 \\[2pt]
 0.7   & 6.202 & 125636.15 \\[2pt]
 0.725 & 6.651 & 125624.91 \\[2pt]
 0.75  & 7.118 & 125648.55 \\[2pt]
 0.775 & 7.580 & 125656.45 \\[2pt]
 0.8   & 8.099 & 125617.61 \\[2pt] 
 0.9   & 10.248 & 125464.21 \\[2pt]  
\end{tabular}
}
\caption{
The Bayesian evidences, $\ln \mathcal{Z}$, of BPL mass model fits which include a SMBH with a fixed mass. A 1D grid of fits are shown, which iteratively increase the SMBH mass $M_{\rm BH}$. For SMBHs above masses of $M_{\rm BH} = 5.349 \times 10^{10}$\,M$_{\rm \odot}$ all $\ln \mathcal{Z}$ values are at least $20$ below the BPL model without a SMBH where $\ln \mathcal{Z} = 125699.90$. Therefore SMBHs above this mass are ruled out by the data, because they deform the reconstruction of the counter image (see \cref{figure:WithSMBH}).
}
\label{table:SMBHUpperLimit}
\end{table}

\begin{figure}
\centering
\includegraphics[width=0.235\textwidth]{total_fit/f390w/bpl_image_zoomed.pdf}
\includegraphics[width=0.235\textwidth]{total_with_smbh/f390w/0.9_image_zoomed.pdf}
\caption{
The reconstructed counter image for a BPL models without a SMBH (left panel) and including a $M_{\rm BH} = 1.0 \times 10^{11}$\,M$_{\rm \odot}$ SMBH for fits to the F390W image. The fit including a $M_{\rm BH} = 10^{11}$\,M$_{\rm \odot}$ SMBH displaces the reconstructed counter image such that it does not fit the data accurately.
}
\label{figure:WithSMBH}
\end{figure}