\section{Summary}\label{Summary}

We present an analysis of the galaxy-scale strong gravitational lens Abell 1201 using multi-waveband Hubble Space Telescope imaging. Abell 1201 is a particularly unique system for two reasons: (i) its counter image is just $1$\,kpc away from the lens galaxy centre; (ii) it is a cD galaxy located within a galaxy cluster. After extensive strong lens modeling, we show that reconstructions of Abell 1201's counter image provide constraints for mass models that include a central super massive black hole (SMBH). After performing Bayesian model comparison, we find that all but one lens model of Abell 1201 prefer the inclusion of a SMBH. By averaging over these models, we infer a value of $M_{\rm BH} = 3.27 \pm 2.12  \times 10^{10}$\,M$_{\rm \odot}$, in agreement with previous lensing and stellar dynamics models of the system \cite{Smith2017a, Smith2017}. This makes it one of the largest black hole masses measured to date and qualifies it as an ultramassive black hole. Its mass is comparable to other high velocity dispersion $\sigma_{\rm e}$ systems on the $M_{\rm BH}$--$\sigma_{\rm e}$ relation whose masses were measured via stellar orbit analysis.

There is one mass model without a SMBH which, in a Bayesian sense, is as plausible as any model including a SMBH. This model has a lot of flexibility in adjusting its central density and mimics the lensing effect of the SMBH by increasing its density to be extremely peaked; far more so than any other mass model. However, the model simultaneously requires that its mass centre is offset from the luminous centre of the bulge by $\geq 100$\,pc. This offset is not necessary when a SMBH is included in the lens model and we therefore rule-out this model as being nonphysical. Even allowing for this nonphysical offset centre, the model still provides an upper limit of  $M_{\rm BH} \leq 5.3 \times 10^{10}$\,M$_{\rm \odot}$, as including a SMBH above this mass completely deforms the counter image reconstruction. Therefore, even strong lens systems which are not as fortuitous as Abell 1201 in their configuration could provide meaningful constraints on SMBHs as upper limits.

Abell 1201 is the second strong lens to provide constraints on on its central SMBH mass, following the upper limit of $M_{\rm BH} \leq 2 \times 10^8 $\,M$_{\rm \odot}$ placed by \citet{Winn2003} in a strong lens whose central image was observed. Our work is therefore the first to not only place an upper limit but measure $M_{\rm BH}$ and it does so without the rare observation of a central image. This offers hope that many more strong lens systems can potentially constrain the mass of their central SMBH, although the unique properties of Abell 1201 may mean this remains a somewhat rare occurrence. Further investigation is necessary to draw firm conclusions, but with over one hundred thousand strong lens systems set to be discovered in the next decade there is hope that strong lensing can one day constrain the redshift evolution of the $M_{\rm BH}$--$\sigma_{\rm e}$ relation.