\section{Discussion}\label{Discussion}

\subsection{Super Massive Black Holes}

\subsubsection{$M_{\rm BH}$--$\sigma_{\rm e}$ relation}

\begin{figure}
\centering
\includegraphics[width=0.47\textwidth]{bhmasses_jwn_v2.pdf}
\caption{
This work's measurements of Abell 1201's SMBH's mass in comparison to the black-hole mass versus velocity dispersion relation, from the compilation of \citet{Bosch2016}. Abell 1201's $\sigma_{\rm e}$ value is taken from \citet{Smith2017}. This work's measurement of $M_{\rm BH} = 3.27 \pm 2.12\times10^{10}$\,M$_{\rm \odot}$ is shown in black, which comes from averaging over all mass models. The upper limit of $M_{\rm BH} \leq 5.3 \times 10^{10}$\,M$_{\rm \odot}$ inferred for the broken power law mass model (without a SMBH) is shown for completeness, although we have argued this model is less trustworthy due to being nonphysical (see \cref{ResultSIE}). This figure is adapted from \citet{Smith2017a} and shows their inferred SMBH masses in grey, which come from independent analyses using either point-source based strong lens modeling \citep{Smith2017a} or stellar kinematics \citep{Smith2017}. Both works report that a SMBH with $M_{\rm BH} \geq 10^{10}$\,M$_{\rm \odot}$ fits the data, but neither work could break a degeneracy with models that assumed a radial gradient in the conversion of mass to light. Our inferred value of $M_{\rm BH}$ in Abell 1201 makes it one of the highest mass SMBH's measured. The grey dashed and dotted diagonal lines show 1$\sigma$ and 2$\sigma$ scatter of the mean $M_{\rm BH}$--$\sigma_{\rm e}$ relation, with Abell 1201's SMBH approximately a 2$\sigma$ positive outlier.
} 
\label{figure:SMBHRelation}
\end{figure}

\cref{figure:SMBHRelation} shows the inferred value of $M_{\rm BH} = 3.27 \pm 2.12  \times 10^{10}$\,M$_{\rm \odot}$ on the black-hole mass versus velocity dispersion relation.
This figure shows that Abell 1201 has one of the largest reported black hole masses measured so far, making it an ultramassive black hole \citep{Hlavacek-Larrondo2012}. Its mass is comparable to the SMBH of the brightest cluster galaxies NGC 3842 and NGC 4889 \citep{Mcconnell2013} and the field elliptical NGC 1600 \citep{Thomas2016}, all of which are measured via stellar orbit analysis. All three objects have similar values of $\sigma_{\rm e}$ to Abell 1201. 

The SMBH of Abell 1201 is a $\sim$ $2\sigma$ outlier above the scatter of the $M_{\rm BH}$--$\sigma_{\rm e}$ relation. Two other objects with similar $\sigma_{\rm e}$ values to Abell 1201, NGC3842 and NGC1600, are $\sim 1.5$-$2\sigma$ outliers above the mean relation. There are no corresponding outliers at $\sim 1.5\sigma$ below the mean relation, indicating that for $\sigma_{\rm e} > 250$km\,s$^{\rm -1}$ SMBH masses tend to be above the mean $M_{\rm BH}$--$\sigma_{\rm e}$ relation. Although there are too few objects to draw definitive conclusions,  such an upwards kink at high $\sigma_{\rm e}$ is a prediction of different physical processes. For example, binary SMBH scouring, which saturates $\sigma_{\rm e}$ whilst increasing $M_{\rm BH}$ \citep{Kormendy2013a, Thomas2014}, as well as AGN feedback processes \citep{Hlavacek-Larrondo2012}.  

\subsubsection{Stellar Core}

Massive ellipticals are often observed with a stellar core, quantified via the Nuker or cored Sersic models \citep{Hernquist1990, Trujillo2004, Dullo2013, Dullo2014}. BCGs like Abell 1201 may have extremely large and flat cores \citep{Postman2012}. It is posited that these cores form via SMBH scouring, whereby the dissipationless merging of two SMBHs in the centre of a galaxy preferentially ejects high mass stars via three-body interactions \citep{Faber1997, Merritt2006, Kormendy2009,Kormendy2013a, Thomas2014}. We fitted the core-Sersic model to Abell 1201's lens galaxy light during our initial analysis, however the model did not produce an improved fit to the data. Typical core sizes are $0.02$--$0.5$\,kpc \citep{Dullo2019}, therefore if Abell 1201 has a stellar core it may be we simply cannot resolve it, due to the data's resolution of $\sim 120$\,pc\,pixel$^{\rm -1}$. 

Aspects of the lens models which include a SMBH point towards a cored (or at least shallow) inner density. For example, the power-law mass model with a SMBH infers a slope $\gamma^{\rm mass} = 1.65^{+0.12}_{-0.12}$, which is much shallower than many massive elliptical strong lenses with near isothermal slopes of $\gamma^{\rm mass} \sim 2$ \citep{Koopmans2009}. Decomposed models including a SMBH give comparable inner densities. When fitting the core-Sersic model we only included it in the model for the lens galaxy's light. We did not fit it as part of a decomposed mass model and therefore did not try to constrain the stellar core via the ray-tracing and lensing analysis. Future studies hunting for SMBHs in strong lenses may benefit from doing this, because an improved model of the lens's central mass density could help break the degeneracy seen in this work with $M_{\rm BH}$.

\subsubsection{Outlook for Strong Lensing}

Abell 1201 is the second strong lens in which the central SMBH mass has been constrained. It is the first to do so without a central image, as well as the first to provide a measurement of $M_{\rm BH}$ as opposed to an upper limit. This raises a number of questions: what is so special about Abell 1201 that makes it sensitive to its SMBH? Can $M_{\rm BH}$ be measured in other known strong lenses? How common an occurrence will this be amongst the incoming samples of 100,000 strong lenses?

Abell 1201 is a unique strong lens in that its counter image is close to the lens centre and it is a cD galaxy in a galaxy cluster. The cluster potential exerts a large external shear (which is seen in our lens models) that brings the counter-image even closer to the lens centre \citep{Smith2017a}, an effect that is not present in most known galaxy-scale strong lenses, which are typically field galaxies. Thus, a very specific set of circumstances may make Abell 1201 sensitive to its SMBH, and a strategy to finding more systems is to target cD / BCG galaxies with instruments like Multi Unit Spectroscopic Explorer (MUSE). 

On the other hand, some known strong lenses in surveys like the Sloan Lens ACS Survey \citep{Bolton2008a} and Strong Lensing in the Legacy Survey \citep{Sonnenfeld2013b} may be sensitive to their central SMBH and appropriate lens modeling has simply not been performed. Certainly, every strong lens will provide an upper limit on $M_{\rm BH}$, the question is whether any are low enough to be informative for models of galaxy evolution. Whilst the multiple images of strong lenses are predominantly observed at radii well beyond Abell 1201's $1\,$kpc counter image, there are examples of strong lenses where the extended emission of the lensed source goes this close. For example, SLACS1250+0523, which was modeled by \citet{Nightingale2019}. In many surveys, for a candidate strong lens to be worthy of following up with higher resolution imaging, a visible counter image clearly distinct from the lens's emission is typically required. Systems like Abell 1201 may therefore be common in nature but rarely selected for followup. We leave it to future work to investigate what constraints known strong lenses can place on $M_{\rm BH}$. 

% Recent work by \citet{Shajib2022} shows the analysis of a strongly lensed quasar which, when fitted with a decomposed mass model, appears to show qualitatively similar behaviour to Abell 1201. The model appears to have missing mass in its centre which either a radial gradient in the 

It has long been expected that strong lensing can constrain SMBH masses when a central third or fifth image is observed \citep{Rusin2000, Mao2000, Keeton2003, Hezaveh2015}. Such a system was presented by \citet{Winn2003}, who placed an upper limit of $M_{\rm BH} < 2 \times 10^8$\,M$_{\rm \odot}$. These systems require the inner density profile of the lens galaxy to be sufficiently cored that the central image is not demagnifed below the observing instrument's detection limit. Given that no other such observation has been made despite numerous attempts \citep{Jackson2015, Wong2017} this appears to be a rare occurrence. Lower limits on $M_{\rm BH}$ have been placed in systems where a central image is not detected \citep{Quinn2016}.

Abell 1201 demonstrates that a SMBH mass measurement is possible even when the lens's inner density is not cored. This offers hope that large samples of strong lenses can one day constrain the $M_{\rm BH}$--$\sigma_{\rm e}$ relation. This would enable the masses of non-active black holes to be measured at high redshifts, and would provide measurements on the high $\sigma_{\rm e}$ end of the relation where few ETGs are observed in the local Universe. With over 100000 strong lenses set to be observed in the next decade \citep{Collett2015} it is inevitable that more SMBH measurements via strong lensing will be made, however more work is necessary to determine how common an occurrence this will be, and in what types of strong lenses and at how high of a redshift such constraints are feasible. If the detectability of a strong lens's SMBH depends on a specific set of circumstances like Abell 1201, there will also be unavoidable selection effects that must be accounted for.