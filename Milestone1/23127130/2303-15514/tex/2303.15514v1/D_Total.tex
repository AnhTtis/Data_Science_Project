\section{Total Mass Models}\label{ResultSIE}

\begin{table}
\resizebox{\linewidth}{!}{
\begin{tabular}{ l | l | l | l } 
\multicolumn{1}{p{1.5cm}|}{Filter} 
& \multicolumn{1}{p{1.3cm}|}{Model} 
& \multicolumn{1}{p{1.3cm}|}{Includes SMBH?} 
& \multicolumn{1}{p{1.5cm}|}{$\ln \mathcal{Z}$}  
\\ \hline
F390W & PL & \ding{55} & 125562.45 \\[1pt]
F390W & PL & \checkmark & \textbf{125707.20} \\[0pt]
\hline
F390W & BPL & \ding{55} & 125699.90 \\[1pt]
F390W & BPL & \checkmark & 125693.78 \\[0pt]
\hline
F814W & PL &  \ding{55}& 78301.58 \\[1pt]
F814W & PL & \checkmark & \textbf{78330.39} \\[0pt]
\hline
F814W & BPL & \ding{55} & 78331.17 \\[1pt]
F814W & BPL & \checkmark & 78329.28 \\[0pt]
\end{tabular}
}
\caption{
The Bayesian Evidence, $\ln \mathcal{Z}$, of each model-fit using total mass models that collectively represent the lens's stellar and dark matter, where all models also include an external shear. $\ln \mathcal{Z}$ values for both the F390W and F814W images are shown. The favoured models given our criteria of $\Delta \ln \mathcal{Z} > 10$ is shown in bold. The PL mass model without a SMBH produces lower values of $\ln \mathcal{Z}$ than the PL model with a SMBH and both BPL models. The BPL model without a SMBH produces a $\ln \mathcal{Z}$ comparable to all models including a SMBH. The PL models favoured by model comparison are shown in bold; no bold model is shown for the BPL because models with and without a SMBH are both within the threshold of $\Delta \ln \mathcal{Z} > 10$ of one another.
}
\label{table:SMBHMC}
\end{table}

\begin{figure*}
\centering
\includegraphics[width=0.241\textwidth]{total_fit/f390w/pl_data_zoomed.pdf}
\includegraphics[width=0.241\textwidth]{total_fit/f390w/pl_image_zoomed.pdf}
\includegraphics[width=0.241\textwidth]{total_fit/f390w/pl_norm_zoomed.pdf}
\includegraphics[width=0.241\textwidth]{total_fit/f390w/pl_source_recon.pdf}
\includegraphics[width=0.241\textwidth]{total_fit/f390w/pl_smbh_data_zoomed.pdf}
\includegraphics[width=0.241\textwidth]{total_fit/f390w/pl_smbh_image_zoomed.pdf}
\includegraphics[width=0.241\textwidth]{total_fit/f390w/pl_smbh_norm_zoomed.pdf}
\includegraphics[width=0.241\textwidth]{total_fit/f390w/pl_smbh_source_recon.pdf}

\caption{
Zoom-ins of the observed counter image in the F390W data (left panel), the model lensed source (left-centre panel), the normalized residuals (right-centre panel) and the source reconstruction (right panel). The top and bottom rows show the power-law mass model without and with a SMBH respectively. All models include an external shear. Models which omit a SMBH form an additional clump of light in the counter image, which is not present in the data. The tangential caustic is shown by a black line and radial critical curve and caustic a white line; the latter does not form for models including a SMBH.
}
\label{figure:ModelsPLF390W}
\end{figure*}

\begin{figure*}
\centering
\includegraphics[width=0.241\textwidth]{total_fit/f814w/pl_data_zoomed.pdf}
\includegraphics[width=0.241\textwidth]{total_fit/f814w/pl_image_zoomed.pdf}
\includegraphics[width=0.241\textwidth]{total_fit/f814w/pl_norm_zoomed.pdf}
\includegraphics[width=0.241\textwidth]{total_fit/f814w/pl_source_recon.pdf}
\includegraphics[width=0.241\textwidth]{total_fit/f814w/pl_smbh_data_zoomed.pdf}
\includegraphics[width=0.241\textwidth]{total_fit/f814w/pl_smbh_image_zoomed.pdf}
\includegraphics[width=0.241\textwidth]{total_fit/f814w/pl_smbh_norm_zoomed.pdf}
\includegraphics[width=0.241\textwidth]{total_fit/f814w/pl_smbh_source_recon.pdf}

\caption{
The same as \cref{figure:ModelsPLF390W} but for the F814W data.
}
\label{figure:ModelsPLF814W}
\end{figure*}

\begin{figure*}
\centering
\includegraphics[width=0.241\textwidth]{total_fit/f390w/bpl_data_zoomed.pdf}
\includegraphics[width=0.241\textwidth]{total_fit/f390w/bpl_image_zoomed.pdf}
\includegraphics[width=0.241\textwidth]{total_fit/f390w/bpl_norm_zoomed.pdf}
\includegraphics[width=0.241\textwidth]{total_fit/f390w/bpl_source_recon.pdf}
\includegraphics[width=0.241\textwidth]{total_fit/f390w/bpl_smbh_data_zoomed.pdf}
\includegraphics[width=0.241\textwidth]{total_fit/f390w/bpl_smbh_image_zoomed.pdf}
\includegraphics[width=0.241\textwidth]{total_fit/f390w/bpl_smbh_norm_zoomed.pdf}
\includegraphics[width=0.241\textwidth]{total_fit/f390w/bpl_smbh_source_recon.pdf}

\caption{
The same as \cref{figure:ModelsPLF390W} but for the broken power-law (BPL) model and F390W data.
}
\label{figure:ModelsBPLF390W}
\end{figure*}

\begin{figure*}
\centering
\includegraphics[width=0.241\textwidth]{total_fit/f814w/bpl_data_zoomed.pdf}
\includegraphics[width=0.241\textwidth]{total_fit/f814w/bpl_image_zoomed.pdf}
\includegraphics[width=0.241\textwidth]{total_fit/f814w/bpl_norm_zoomed.pdf}
\includegraphics[width=0.241\textwidth]{total_fit/f814w/bpl_source_recon.pdf}
\includegraphics[width=0.241\textwidth]{total_fit/f814w/bpl_smbh_data_zoomed.pdf}
\includegraphics[width=0.241\textwidth]{total_fit/f814w/bpl_smbh_image_zoomed.pdf}
\includegraphics[width=0.241\textwidth]{total_fit/f814w/bpl_smbh_norm_zoomed.pdf}
\includegraphics[width=0.241\textwidth]{total_fit/f814w/bpl_smbh_source_recon.pdf}

\caption{
The same as \cref{figure:ModelsPLF390W} but for the broken power-law (BPL) model and F814W data.
}
\label{figure:ModelsBPLF814W}
\end{figure*}

\begin{figure}
\centering
\includegraphics[width=0.47\textwidth]{total_1d/f390w_total_mass_1d.pdf}
\caption{The convergence as a function of radius inferred using the F390W image for the total mass models: (i) the power-law (black); (ii) the broken power-law (red); (iii) the power law and SMBH (blue) and; (iv) the broken power-law and SMBH (green), where all models include an external shear. Each line is computed using coordinates that extend radially outwards from the centre of the mass profile and are aligned with its major axis. Shaded regions for each mass model's convergence are shown, corresponding to the inferred $3\sigma$ confidence intervals. The BPL model places more mass centrally than all other models, consistent with its ability to reconstruct the counter image accurately.}
\label{figure:Total1D}
\end{figure}

This appendix shows the results of fitting two total mass-models: the power-law (PL) \citep{Tessore2015} and broken power-law (BPL) \citep{Oriordan2019, Oriordan2020, Oriordan2021}. Like in the main paper, we compare fits with and without a point-mass representing a SMBH. We focus on the Bayesian evidence, $\ln \mathcal{Z}$, and the reconstruction of the counter image. We investigate whether the extraneous flux removed by the SMBH for the decomposed models can be removed by either of these profiles without a SMBH. The inferred model parameters for the PL and BPL models are given in \cref{table:ModelsTotal1} and \cref{table:ModelsTotal2}.

\subsection{Profile Equations}

A softened power-law ellipsoid density profile of form
\begin{equation}
\label{eqn:SPLEkap}
\kappa^{\rm mass} (\xi) = \frac{3 - \gamma^{\rm mass}}{1 + q^{\rm mass}} \bigg( \frac{\theta^{\rm mass}_{\rm E}}{\xi} \bigg)^{\gamma^{\rm mass} - 1} ,
\end{equation}
is assumed, where $\theta^{\rm mass}_{\rm E}$ is the model Einstein radius in arcseconds. The power-law density slope is $\gamma^{\rm mass}$, and setting $\gamma^{\rm mass} = 2$ gives the singular isothermal ellipsoid (SIE) model. Deflection angles for the power-law are computed via an implemention of the method of \citep{Tessore2015} in {\tt PyAutoLens}.

We also use the elliptical broken power law (BPL) profile \citep{Oriordan2019, Oriordan2020, Oriordan2021} with convergence 
\begin{equation}\label{equ:power_law}
    \kappa^{\rm mass}{\left(r\right)}=\left\{
    \begin{array}{ll}
    \theta^{\rm mass}_{\rm E}\left(r^{\rm mass}_{\rm b} / r \right)^{t^{\rm mass}_1}, & r \leq r^{\rm mass}_{\rm b} \\
    \theta^{\rm mass}_{\rm E}\left(r^{\rm mass}_{\rm b} / r \right)^{t^{\rm mass}_2}, & r > r^{\rm mass}_{\rm b}
    \end{array}
    \right.,
\end{equation}
where $r^{\rm mass}_{\rm b}$ is the break radius, $\theta^{\rm mass}_{\rm E}$ is the convergence at the break radius, $t^{\rm mass}_1$ is the inner slope and $t^{\rm mass}_2$ is the outer slope. The isothermal case is given by $t^{\rm mass}_1 = t^{\rm mass}_2 = 1.0$.
%When $r^{\rm mass}_{\rm b}=0$, the BPL reduces to the standard PL profile above.

\subsection{Power-Law Models}

We first investigate fits using the simpler PL mass model. The PL parameterization has less flexibility in adjusting its central density compared to the BPL. The top two rows of table \ref{table:SMBHMC} show the $\ln \mathcal{Z}$ values inferred for PL model-fits with and without a SMBH. Models including a SMBH are strongly favoured, giving $\Delta \ln \mathcal{Z} = 145$ for the F390W data and $\Delta \ln \mathcal{Z} = 29$ for the F814W. 

\cref{figure:ModelsPLF390W} shows zoom-ins of the PL model's reconstruction of the counter image. The figure shows the same behaviour seen for the decomposed model in the main paper, whereby the PL model without a SMBH produces central extraneous flux, which the inclusion of the SMBH removes. \cref{figure:ModelsPLF814W} shows this also occurs in the F814W image. The residuals of this extraneous flux are more significant than seen for the decomposed model fitted in the main paper, because of the PL model's reduced flexibility in adjusting its central density.    

When the PL mass model includes a SMBH a value of $M_{\rm BH} = 3.83^{+1.56}_{-1.72} \times 10^{10}$\,M$_{\rm \odot}$ is inferred, which is consistent with the $M_{\rm BH}$ values inferred for the decomposed models. The SMBH changes the ray-tracing such that the lens model can now reproduce the counter image's structure accurately. The PL also infers a shallower slope of $\gamma^{\rm mass} = 1.65^{+0.12}_{-0.12}$, compared to the value $\gamma^{\rm mass} = 1.82^{+0.05}_{-0.05}$ inferred without a SMBH. The model without a SMBH therefore tries (and fails) to better fit the counter image by placing more mass centrally.

Fits using the PL therefore support the inclusion of a SMBH is the lens model, and their reconstruction of the counter image produces the same behaviour seen for the decomposed model in the main paper.

\subsection{Broken Power-Law Models}

\begin{table*}
\tiny
\resizebox{\linewidth}{!}{
\begin{tabular}{ l l l l l l l} 
\multicolumn{1}{p{1.8cm}|}{\centering \textbf{Model}} 
& \multicolumn{1}{p{1.5cm}}{$x^{\rm{mass}}$ (\arcsec)} 
& \multicolumn{1}{p{1.5cm}}{$y^{\rm{mass}}$ (\arcsec)} 
& \multicolumn{1}{p{1.5cm}}{$\epsilon_{\rm 1}^{\rm{mass}}$} 
& \multicolumn{1}{p{1.5cm}}{$\epsilon_{\rm 2}^{\rm{mass}}$} 
& \multicolumn{1}{p{1.5cm}}{$\epsilon_{\rm 1}^{\rm{ext}}$} 
& \multicolumn{1}{p{1.5cm}}{$\epsilon_{\rm 2}^{\rm{ext}}$} 
\\ \hline
& & & & & & \\[-4pt]

PL & 
$0.014^{+0.035}_{-0.041}$ & 
$-0.054^{+0.053}_{-0.053}$ & 
$0.107^{+0.030}_{-0.028}$ & 
$-0.088^{+0.024}_{-0.003}$ & 
$-0.113^{+0.019}_{-0.018}$ & 
$0.152^{+0.018}_{-0.025}$ \\[-7pt]

\\ \hline
& & & & & & \\[-5pt]

BPL & 
$0.045^{+0.031}_{-0.033}$ & 
$0.063^{+0.065}_{-0.061}$ & 
$-0.105^{+0.026}_{-0.029}$ & 
$0.145^{+0.027}_{-0.032}$ & 
$-0.105^{+0.039}_{-0.024}$ & 
$0.145^{+0.025}_{-0.038}$ \\[-7pt]

\\ \hline
& & & & & & \\[-5pt]

PL + SMBH & 
$0.050^{+0.035}_{-0.040}$ & 
$0.063^{+0.045}_{-0.071}$ & 
$0.104^{+0.028}_{-0.026}$ & 
$-0.089^{+0.026}_{-0.037}$ & 
$-0.112^{+0.026}_{-0.021}$ & 
$0.147^{+0.019}_{-0.029}$ \\[-7pt]

\\ \hline
& & & & & & \\[-5pt]
BPL + SMBH & 
$0.052^{+0.001}_{-0.001}$ & 
$0.079^{+0.004}_{-0.009}$ & 
$0.101^{+0.002}_{-0.001}$ & 
$-0.083^{+0.003}_{-0.005}$ & 
$-0.117^{+0.002}_{-0.001}$ & 
$0.146^{+0.022}_{-0.041}$  \\[-2pt]
\end{tabular}
}
\caption{
The inferred geometric model parameters of the power-law (PL) and broken power-law (BPL) total mass models fitted to the F390W image in the Mass pipeline. Errors are given at 3$\sigma$ confidence intervals.
}
\label{table:ModelsTotal1}
\end{table*}

\begin{table*}
\tiny
\resizebox{\linewidth}{!}{
\begin{tabular}{ l l l l l l l} 
\multicolumn{1}{p{1.8cm}|}{\centering \textbf{Model}} 
& \multicolumn{1}{p{1.5cm}}{$\theta_{\rm Ein}^{\rm{mass}}$ (\arcsec)} 
& \multicolumn{1}{p{1.5cm}}{$\gamma^{\rm{mass}}$} 
& \multicolumn{1}{p{1.5cm}}{$t_{\rm 1}^{\rm{mass}}$} 
& \multicolumn{1}{p{1.5cm}}{$t_{\rm 2}^{\rm{mass}}$} 
& \multicolumn{1}{p{1.5cm}}{$\theta_{\rm B}^{\rm{mass}}$ (\arcsec)} 
& \multicolumn{1}{p{1.5cm}}{$\theta_{\rm Ein}^{\rm{smbh}}$ (\arcsec)} 
\\ \hline
& & & & & & \\[-4pt]

PL & 
$1.925^{+0.046}_{-0.040}$ & 
$1.818^{+0.042}_{-0.077}$ & 
& \\[-6pt]

\\ \hline
& & & & & & \\[-2pt]

BPL & 
$1.869^{+0.039}_{-0.041}$ & 
& 
$1.13^{+0.34}_{-0.20}$ & 
$0.64^{+0.09}_{-0.15}$ & 
$0.45^{+0.09}_{-0.22}$ & 
 \\[-6pt]

\\ \hline
& & & & & & \\[-2pt]

PL + SMBH & 
$1.56^{+0.18}_{-0.19}$ & 
$1.66^{+0.08}_{-0.10}$ & 
& 
&
&
$0.53^{+0.12}_{-0.14}$ \\[-6pt]

\\ \hline
& & & & & & \\[-2pt]
BPL + SMBH & 
$1.6165^{+0.0017}_{-0.0092}$ & 
&
$0.6920^{+0.0061}_{-0.0419}$ & 
$0.6637^{+0.0009}_{-0.0036}$ &
$0.2096^{+0.0320}_{-0.0030}$ & 
$0.5544^{+0.0107}_{-0.0013}$ \\[-2pt]
\end{tabular}
}
\caption{
The inferred model parameters of the power-law (PL) and broken power-law (BPL) total mass models fitted to the F390W image in the Mass pipeline. Errors are given at 3$\sigma$ confidence intervals.
}
\label{table:ModelsTotal2}
\end{table*}

We now inspect fits using the BPL, which has much greater flexibility than the PL in controlling its inner density. The bottom two rows of table \ref{table:SMBHMC} show the $\ln \mathcal{Z}$ values inferred for BPL model-fits with and without a SMBH. For the F390W image the $\ln \mathcal{Z}$ value for the BPL model without a SMBH is 125699.90; this is $6.22$ above the BPL model with a SMBH. This value is also within $\Delta \ln \mathcal{Z} \sim 1$ of the decomposed models including a SMBH fitted in the main paper (see table \ref{table:SMBHMCDecomp}). 

\cref{figure:ModelsBPLF390W} shows zoom-ins of the BPL model's reconstruction of the counter image. Irrespective of whether a SMBH is included in the model, the extraneous flux in the reconstructed counter image seen for decomposed models and the PL model without a SMBH is not produced. \cref{figure:ModelsBPLF814W} shows this is also true for fits to the F814W image. 

\cref{figure:Total1D} shows the 1D convergence profiles for the BPL mass models with and without a SMBH. Shaded regions shows $3\sigma$ confidence intervals for each profile. The inner density (e.g. within $0.3$") of the BPL without a SMBH is steeper than the decomposed models fitted in the main paper (and also the PL models). The BPL is therefore able to remove extraneous flux from the the reconstructed counter image because it places more mass centrally than any other mass model. The BPL model including a SMBH infers a shallower density profile, because the SMBH performs the ray-tracing which fits the counter image. 

Fits using the BPL model therefore raise the possibility that a SMBH is not required in the lens mass model.

\subsection{Decomposed Model Validation}

The BPL fits show that if the mass model has a sufficiently high inner density then it can reconstruct the counter image accurately. We therefore check whether the decomposed models fitted in the main paper can place as much mass centrally as the BPL without requiring a SMBH. The blue dashed line in \cref{figure:Total1D} shows that if the bulge of the triple Sersic model assumes a radial gradient parameter with the value $\Gamma^{\rm bulge} = 0.9$, its central density matches that of the BPL. The decomposed model parameterization therefore includes models with inner densities comparable to the BPL. We did not infer them because they correspond to lower likelihood solutions (our inferred value is $\Gamma^{\rm bulge} = 0.52^{+0.21}_{-0.32}$ at 3$\sigma$ confidence). We verify this by fitting decomposed models where a uniform prior on $\Gamma^{\rm bulge}$ for the bulge component is placed between 0.85 and 0.95. The $\ln \mathcal{Z}$ values of this model with three and two Sersics are $125560.70$ and $125630.27$ respectively, well below the value of $125699.06$ found for the triple Sersic decomposed model including a SMBH. 

We also investigate models which make the central dark matter density comparable to that of the BPL. The green dashed line in \cref{figure:Total1D} shows that an NFW profile with a concentration that is a $3.5\sigma$ positive outlier on the mass-concentration relation \citep{Ludlow2016} has a central density close to the BPL. We therefore fit a triple Sersic decomposed models which includes the scatter from the mass-concentration relation $\sigma^{\rm dark}$ as a free parameter with a uniform prior between 2.5 and 4.0. We infer $\ln \mathcal{Z} = 125329.01$, significantly below nearly all model fits, with or without a SMBH. 

We therefore conclude that decomposed models that place as much mass centrally as the BPL model cannot attain a comparable $\ln \mathcal{Z}$ without a SMBH for fits to the F390W data. They are also unable to prevent extraneous flux appearing in the counter image. 

\subsection{Mass Model Centering}\label{Projections}

% \begin{figure}
% \centering
% \includegraphics[width=0.235\textwidth]{total_fit_fixed_centre/bpl_norm.pdf}
% \includegraphics[width=0.235\textwidth]{total_fit_fixed_centre/bpl_image_zoomed.pdf}
% \caption{
% The normalized residuals (left) and model source's counter image (right) for the BPL model (without a SMBH) with a centre fixed to the observed centre of the bulge component of the lens galaxy. The model-fit to the F390W image of Abell 1201 is shown and the model also includes an external shear.
% } 
% \label{figure:ModelsF390WFixedCentre}
% \end{figure}

The centre of the stellar mass component of the decomposed model is tied to that of the lens light, whereas the BPL has full freedom in choosing its center. We now inspect the centering of the decomposed and BPL models in more detail, to see if any model appears more or less realistic or physically plausible. This will allow us to argue in favour or against the need for a SMBH.

Upon inspection of the different mass model parameters, fits using the BPL model (with or without a SMBH) infer mass model centres in the range $0.03 < x^{\rm mass} < 0.06$ and $0.04 < y^{\rm mass} < 0.09$ for the F390W image and $0.0 < x^{\rm mass} < 0.03$ and $0.02 < y^{\rm mass} < 0.07$ for the F814W image. Inspecting the lens light model-fits, the inferred centre of the bulge at $3\sigma$ confidence is $x^{\rm bulge} = -0.008^{+0.003}_{-0.003}$ and $y^{\rm bulge} = 0.003^{+0.003}_{-0.003}$ for the F814W image and $x^{\rm bulge} = -0.013^{+0.007}_{-0.006}$ and $y^{\rm bulge} = 0.007^{+0.003}_{-0.003}$ for the F390W image. The BPL model is therefore shifting its centre $\geq 0.04$\," (a full pixel) away from the bulge centre, a shift which corresponds to $\geq 120$\,pc.

We now fit a BPL model without a SMBH where the centre is fixed to that of the bulge ($x^{\rm mass} = -0.008$ and $y^{\rm mass} = 0.003$). 
% The reconstructed counter image and residuals are shown in \cref{figure:ModelsF390WFixedCentre}. The most notable features are not in the counter image, but the giant arc, where significant residuals are seen. 
This model's fit to the F390W image infers $\ln \mathcal{Z} = 125317.26$, well below the value of $\ln \mathcal{Z} = 125699.90$ inferred for the BPL model with a free centre. When the BPL's centre is consistent with the luminous emission it therefore cannot reconstruct Abell 1201's source accurately. 

We can now explain why decomposed models without a SMBH but with a bulge radial gradient around $\Gamma^{\rm bulge} = 0.9$ or a very concentrated dark matter halo did not give as high $\ln \mathcal{Z}$ values or remove extraneous flux from the reconstructed counter image. Even though their central density is as steep as the BPL model, steepening the mass profile only improves the overall fit when its centre is offset from the bulge by $\geq 120$pc in the positive x and y directions. Thus, not only does the BPL show a nonphysical offset from the bulge, but its ability to reconstruct the counter image accurately is dependent on the existence of this offset. 

We therefore view the decomposed models with a SMBH fitted in the main paper as more reliable than the BPL model without a SMBH and discard the BPL model as nonphysical. 

%The BPL plus SMBH model infers a SMBH mass of $M_{\rm BH} = 3.57^{+1.83}_{-1.62}\times10^{10}$\,M$_{\rm \odot}$ at $3\sigma$ confidence, consistent with all previous $M_{\rm BH}$ values inferred.



