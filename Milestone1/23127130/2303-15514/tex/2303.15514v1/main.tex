\documentclass[useAMS,usenatbib]{mn2e}
\usepackage{amssymb}
\usepackage[english]{babel}
\usepackage{color}
\usepackage{dblfloatfix}
\usepackage{fontawesome}
\usepackage{graphics}
\usepackage{graphicx}
\usepackage{natbib}
\usepackage{hyperref}
\usepackage[capitalise]{cleveref}
\usepackage{pifont}
\usepackage{subfiles}
\usepackage{orcidlink}
\usepackage{wasysym}

\defcitealias{Etherington2022}{E22}

\def\simlt{\lower.5ex\hbox{$\; \buildrel < \over \sim \;$}}
\def\simgt{\lower.5ex\hbox{$\; \buildrel > \over \sim \;$}}

% APC macros
\newcommand{\kms}{\,\mathrm{km \, s^{-1}}}
\newcommand{\msol}{\,\mathrm{M_{\sun}}}
\newcommand{\red}[1]{{\color{red}{{#1}}}}
\newcommand*\tens[1]{\ensuremath{\mathsf{#1}}}

% For RJS comments
\definecolor{purple}{rgb}{0.525,0,0.525}
\newcommand{\rjs}[1]{{\color{purple}{#1}}}
\newcommand{\rjm}[1]{{\color{blue}{{#1}}}}

%%%%%%%%%%%%%%%%%%%%%%%%%%%%%%%%%%%%%%%%%%%%%%%%

\title[Detection of an Ultramassive Black Hole]
{Abell 1201: Detection of an Ultramassive Black Hole in a Strong Gravitational Lens}
\author[Nightingale et al.]
{\parbox{\textwidth}{James.\ W.\ Nightingale$^{1}$\orcidlink{0000-0002-8987-7401}\thanks{e-mail: james.w.nightingale@durham.ac.uk},
Russell J.\ Smith$^{1}$\orcidlink{0000-0001-5998-2297},
Qiuhan He$^{1}$\orcidlink{0000-0003-3672-9365},
Conor M.\ O'Riordan$^{2}$\orcidlink{0000-0003-2227-1998},
Jacob A. Kegerreis$^{3}$\orcidlink{0000-0001-5383-236X},
Aristeidis Amvrosiadis$^{1}$\orcidlink{0000-0002-4465-1564},
Alastair C.\ Edge$^{1}$\orcidlink{0000-0002-3398-6916},
Amy Etherington$^{1}$,
Richard G.\ Hayes$^{1}$,
Ash Kelly$^{1}$\orcidlink{0000-0003-3850-4469},
John R.\ Lucey$^{1}$\orcidlink{0000-0002-9748-961X},
Richard J.\ Massey$^{1}$\orcidlink{0000-0002-6085-3780}\\
}\\
$^{1}$Centre for Extragalactic Astronomy, Department of Physics, Durham University, South Road, Durham, DH1 3LE, UK\\
$^{2}$Max Planck Institute for Astrophysics, Karl-Schwarzschild-Strasse 1, 85748 Garching bei München, Germany\\
$^{3}$NASA Ames Research Center, Moffett Field, CA 94035, USA\\
}

\makeatletter
\newcommand{\github}[1]{%
   \href{#1}{\faGithubSquare}%
}
\makeatother

\begin{document}

\bibliographystyle{mn2e}
\bibpunct{(}{)}{;}{a}{}{;}
\date{\today}
\pagerange{\pageref{firstpage}--\pageref{lastpage}} 
\pubyear{2018}
\maketitle
\label{firstpage}

\begin{abstract}

Supermassive black holes (SMBHs) are a key catalyst of galaxy formation and evolution, leading to an observed correlation between SMBH mass $M_{\rm BH}$ and host galaxy velocity dispersion $\sigma_{\rm e}$. Outside the local Universe, measurements of $M_{\rm BH}$ are usually only possible for SMBHs in an active state: limiting sample size and introducing selection biases. Gravitational lensing makes it possible to measure the mass of non-active SMBHs. We present models of the $z=0.169$ galaxy-scale strong lens Abell~1201. 
A cD galaxy in a galaxy cluster, it has sufficient `external shear' that a magnified image of a $z = 0.451$ background galaxy is projected just $\sim 1$ kpc from the galaxy centre. Using multi-band Hubble Space Telescope imaging and the lens modeling software \texttt{PyAutoLens} we reconstruct the distribution of mass along this line of sight. Bayesian model comparison favours a point mass with $M_{\rm BH} = 3.27 \pm 2.12\times10^{10}$\,M$_{\rm \odot}$ (3$\sigma$ confidence limit); an ultramassive black hole. One model gives a comparable Bayesian evidence without a SMBH, however we argue this model is nonphysical given its base assumptions. This model still provides an upper limit of $M_{\rm BH} \leq 5.3 \times 10^{10}$\,M$_{\rm \odot}$, because a SMBH above this mass deforms the lensed image $\sim 1$ kpc from Abell 1201's centre. This builds on previous work using central images to place upper limits on $M_{\rm BH}$, but is the first to also place a lower limit and without a central image being observed.
The success of this method suggests that surveys during the next decade could measure thousands more SMBH masses, and any redshift evolution of the $M_{\rm BH}$--$\sigma_{\rm e}$ relation. Results are available at \url{https://github.com/Jammy2211/autolens_abell_1201}.


\end{abstract}

\begin{keywords}
quasars: supermassive black holes -- gravitational lensing: strong -- galaxies: evolution -- galaxies: formation
\end{keywords}

\subfile{1_Introduction.tex}
\subfile{2_Data.tex}        
\subfile{3_Method.tex}   
\subfile{4_Results.tex}   
\subfile{6_Discussion.tex}
\subfile{7_Summary.tex}

\subfile{software.tex}

\section*{Data Availability}

Text files and images of every model-fit performed in this work are available at \url{https://github.com/Jammy2211/autolens_abell_1201}. Full \texttt{dynesty} chains of every fit are available at \url{https://zenodo.org/record/7695438}.

\section*{Acknowledgements}

JWN is supported by the UK Space Agency, through grant ST/N001494/1, and a Royal Society Short Industry Fellowship.
RJM is supported by a Royal Society University Research Fellowship and by the STFC via grant ST/T002565/1, and the UK Space Agency via grant ST/W002612/1.
JAK acknowledges support from a NASA Postdoctoral Program Fellowship.
AE is supported by STFC via grants ST/R504725/1 and ST/T506047/1.  
AA and QH acknowledge support from the European Research Council (ERC) through Advanced Investigator grant DMIDAS (GA 786910). 
This work used both the Cambridge Service for Data Driven Discovery (CSD3) and the DiRAC Data-Centric system, project code dp195, which are operated by the University of Cambridge and Durham University on behalf of the STFC DiRAC HPC Facility (www.dirac.ac.uk). These were funded by BIS capital grant ST/K00042X/1, STFC capital grants ST/P002307/1, ST/R002452/1, ST/H008519/1, ST/K00087X/1, STFC Operations grants ST/K003267/1, ST/K003267/1, and Durham University. DiRAC is part of the UK National E-Infrastructure.

% RJS: references should precede appendix
\bibliography{library, citations, manual}            % bibtex file

\appendix
\subfile{A_Lens_Profiles.tex}
\subfile{B_Light_Models.tex}
\subfile{C_Decomposed_Models.tex}
\subfile{D_Total.tex}
\subfile{E_Clump.tex}
\subfile{F_Centre.tex}
\subfile{G_Radial.tex}

%% \bibliography{library, citations}            % bibtex file

\label{lastpage}

\end{document}
