\section{Double Sersic Models}\label{MassFits}

The results of fitting the decomposed model with two Sersic profiles are shown in \cref{figure:LightDarkF390W} and \cref{figure:LightDarkF814W}. These figures follow the same layout as \cref{figure:LightDarkF390Wx3} and \cref{figure:LightDarkF814Wx3} in the main paper. Results show the same behaviour as the triple Sersic fitted in the main paper, including extraneous flux in the counter image reconstruction when the model omits a SMBH. The double Sersic fit with a SMBH has a $\ln \mathcal{Z}$ value $29.93$ below the triple Sersic with a SMBH. This suggests that the lensing effects of the faint inner structure the third Sersic represents plays a role in reconstructing the counter image.

\begin{figure*}
\centering
\includegraphics[width=0.241\textwidth]{light_dark_fit_x2/f390w/grad_data_zoomed.pdf}
\includegraphics[width=0.241\textwidth]{light_dark_fit_x2/f390w/grad_image_zoomed.pdf}
\includegraphics[width=0.241\textwidth]{light_dark_fit_x2/f390w/grad_norm_zoomed.pdf}
\includegraphics[width=0.241\textwidth]{light_dark_fit_x2/f390w/grad_source_recon.pdf}
\includegraphics[width=0.241\textwidth]{light_dark_fit_x2/f390w/grad_smbh_data_zoomed.pdf}
\includegraphics[width=0.241\textwidth]{light_dark_fit_x2/f390w/grad_smbh_image_zoomed.pdf}
\includegraphics[width=0.241\textwidth]{light_dark_fit_x2/f390w/grad_smbh_norm_zoomed.pdf}
\includegraphics[width=0.241\textwidth]{light_dark_fit_x2/f390w/grad_smbh_source_recon.pdf}
\caption{
Zoom-ins of the observed counter image in the F390W data (left panel), the model lensed source (left-centre panel), the normalized residuals (right-centre panel) and the source reconstruction (right panel). The top and bottom rows shows double Sersic plus NFW decomposed model-fits without and with a SMBH respectively. All models include an external shear. Models which omit a SMBH form an additional clump of light in the counter image, which is not present in the data. The tangential caustic is shown by a black line and radial critical curve and caustic a white line; the latter does not form for models including a SMBH.
} 
\label{figure:LightDarkF390W}
\end{figure*}

\begin{figure*}
\centering
\includegraphics[width=0.241\textwidth]{light_dark_fit_x2/f814w/grad_data_zoomed.pdf}
\includegraphics[width=0.241\textwidth]{light_dark_fit_x2/f814w/grad_image_zoomed.pdf}
\includegraphics[width=0.241\textwidth]{light_dark_fit_x2/f814w/grad_norm_zoomed.pdf}
\includegraphics[width=0.241\textwidth]{light_dark_fit_x2/f814w/grad_source_recon.pdf}
\includegraphics[width=0.241\textwidth]{light_dark_fit_x2/f814w/grad_smbh_data_zoomed.pdf}
\includegraphics[width=0.241\textwidth]{light_dark_fit_x2/f814w/grad_smbh_image_zoomed.pdf}
\includegraphics[width=0.241\textwidth]{light_dark_fit_x2/f814w/grad_smbh_norm_zoomed.pdf}
\includegraphics[width=0.241\textwidth]{light_dark_fit_x2/f814w/grad_smbh_source_recon.pdf}
\caption{
The same as \cref{figure:LightDarkF390W} but for the F814W data.
} 
\label{figure:LightDarkF814W}
\end{figure*}
