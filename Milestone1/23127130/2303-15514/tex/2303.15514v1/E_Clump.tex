\section{Line of sight Galaxy}\label{MassClump}

\begin{table}
\resizebox{\linewidth}{!}{
\begin{tabular}{ l | l | l | l } 
\multicolumn{1}{p{1.1cm}|}{Filter} 
& \multicolumn{1}{p{1.3cm}|}{Number of Sersics} 
& \multicolumn{1}{p{1.3cm}|}{Includes SMBH?} 
& \multicolumn{1}{p{1.5cm}|}{$\ln \mathcal{Z}$}  
\\ \hline
F390W & 2 & \ding{55} & 125559.11  \\[1pt]
F390W & 3 & \checkmark & 125608.05 \\[0pt]
\hline
F390W & 2 & \ding{55} &  125588.03 \\[1pt]
F390W & 3 &  \checkmark & 125596.66 \\[0pt]
\hline
F814W & 2 & \ding{55} & 78327.28   \\[1pt]
F814W & 3 & \checkmark & 78322.25  \\[0pt]
\hline
F814W & 2 & \ding{55} & 78318.61    \\[1pt]
F814W & 3 & \checkmark & 78316.00  \\[0pt]
\end{tabular}
}
\caption{
The same as \cref{table:SMBHMC} but with the $z = 0.273$ galaxy around $(4.0\,", \,1.0\,")$ included in the lens galaxy mass model. This table shows fits assuming a decomposed mass model with two and three Sersic profiles.
}
\label{table:SMBHMCDecompClump}
\end{table}

\begin{table}
\resizebox{\linewidth}{!}{
\begin{tabular}{ l | l | l | l } 
\multicolumn{1}{p{1.1cm}|}{Filter} 
& \multicolumn{1}{p{1.3cm}|}{Model} 
& \multicolumn{1}{p{1.3cm}|}{Includes SMBH?} 
& \multicolumn{1}{p{1.5cm}|}{$\ln \mathcal{Z}$}  
\\ \hline
F390W & PL & \ding{55} & 125434.63 \\[1pt]
F390W & PL &  \checkmark & 125589.40 \\[0pt]
\hline
F390W & BPL & \ding{55} & 125239.03 \\[1pt]
F390W & BPL & \checkmark & 125586.22 \\[0pt]
\hline
F814W & PL & \ding{55} & 78264.73 \\[1pt]
F814W & PL & \checkmark & 78323.10 \\[0pt]
\hline
F814W & BPL & \ding{55} & 78314.53 \\[1pt]
F814W & BPL & \checkmark & 78311.49 \\[0pt]
\end{tabular}
}
\caption{
The same as \cref{table:SMBHMC} but with the $z = 0.273$ galaxy around $(4.0\,", \,1.0\,")$ included in the lens galaxy mass model. This table shows fits assuming a PL and BPL lens model.
}
\label{table:SMBHMClump}
\end{table}

\cref{figure:Data} shows line-of-sight emission towards the right of the giant arc, around $(4.0\,", \,1.0\,")$. \citet{Smith2017a} show that this is a $z = 0.273$ galaxy, which is therefore located between the lens and source galaxies. The emission seen in the HST imaging appears as two (or more) distinct blobs. The [OIII] emission shows similar structure indicating this is likely a single galaxy. We fit additional lens models to Abell 1201 which include this galaxy in the lens model as a spherical isothermal mass profile (see \cref{eqn:SPLEkap}) where $\gamma^{\rm mass} = 2$), accounting for multi-plane ray-tracing effects \citep{Schneider2014c}. The centre of this model is fixed to $(3.6\,",\,0.95\,")$ in the image-plane, which is updated when performing multi-plane ray tracing. The $\ln \mathcal{Z}$ of these model fits are given in \cref{table:SMBHMCDecompClump} and \cref{table:SMBHMClump}. All models produce lower $\ln \mathcal{Z}$ values than those inferred in the main paper, indicating that including the galaxy does improve the lens model.