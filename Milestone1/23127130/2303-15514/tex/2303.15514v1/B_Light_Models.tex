\section{Light Models}\label{LightModels}

\begin{figure*}
\centering
\includegraphics[width=0.32\textwidth]{light_2d/light_2d_sersic_x1_1.pdf}
\includegraphics[width=0.32\textwidth]{light_2d/light_2d_no_align_x2_2.pdf}
\includegraphics[width=0.32\textwidth]{light_2d/light_2d_no_align_x2_1.pdf}
\includegraphics[width=0.32\textwidth]{light_2d/light_2d_no_align_x3_1.pdf}
\includegraphics[width=0.32\textwidth]{light_2d/light_2d_no_align_x3_2.pdf}
\includegraphics[width=0.32\textwidth]{light_2d/light_2d_no_align_x3_3.pdf}
\caption{Two dimensional projections of the individual light profiles for the following three lens galaxy light models: (i) a single Sersic profile (top left panel); (ii) a double Sersic profile where the centres and elliptical components are not aligned, representing a central bulge (top middle panel) and extended component (top right panel); (iii) a triple Sersic model where no geometric components are aligned, representing a bulge, an extended component and a third inner component (bottom row). All models are fitted with a fixed isothermal mass profile with external shear and a pixelized source reconstruction which changes for every light profile fitted. Each intensity plot corresponds to the maximum likelihood light model of a model-fit using the F814W image (the F390W image's blue wavelength makes it is less suited to tracing the lens galaxy's stellar mass).} 
\label{figure:Light2D}
\end{figure*}

\begin{figure*}
\centering
\includegraphics[width=0.19\textwidth]{light_fit/f814w/light_fit_sersic_x1_norm.pdf}
\includegraphics[width=0.19\textwidth]{light_fit/f814w/light_fit_align_all_x2_norm.pdf}
\includegraphics[width=0.19\textwidth]{light_fit/f814w/light_fit_align_centre_x2_norm.pdf}
\includegraphics[width=0.19\textwidth]{light_fit/f814w/light_fit_no_align_x2_norm.pdf}
\includegraphics[width=0.19\textwidth]{light_fit/f814w/light_fit_no_align_x3_norm.pdf}
\includegraphics[width=0.19\textwidth]{light_fit/f390w/light_fit_sersic_x1_norm.pdf}
\includegraphics[width=0.19\textwidth]{light_fit/f390w/light_fit_align_all_x2_norm.pdf}
\includegraphics[width=0.19\textwidth]{light_fit/f390w/light_fit_align_centre_x2_norm.pdf}
\includegraphics[width=0.19\textwidth]{light_fit/f390w/light_fit_no_align_x2_norm.pdf}
\includegraphics[width=0.19\textwidth]{light_fit/f390w/light_fit_no_align_x2_norm.pdf}
\includegraphics[width=0.19\textwidth]{light_1d/light_1d_sersic_x1.pdf}
\includegraphics[width=0.19\textwidth]{light_1d/light_1d_align_all_x2.pdf}
\includegraphics[width=0.19\textwidth]{light_1d/light_1d_align_centre_x2.pdf}
\includegraphics[width=0.19\textwidth]{light_1d/light_1d_no_align_x2.pdf}
\includegraphics[width=0.195\textwidth]{light_1d/light_1d_no_align_x3.pdf}
\caption{The normalized residuals of fits to the F814W image (top row), F390W image (middle row) and 1D decomposed intensity profiles (bottom row) of the following five lens galaxy light models model-fits (from left to right): (i) a single Sersic profile; (ii) a double Sersic profile where the centres and elliptical components are aligned; (iii) where their centres are aligns but elliptical components are not; (iv) where neither components are aligned and; (v) a triple Sersic model where no geometric components are aligned. All models are fitted with a fixed isothermal mass profile with external shear and a pixelized source reconstruction which changes for every light profile fitted. 1D profiles are computed using coordinates that extend radially outwards from the centre of the light profile and are aligned with its major axis. Each plot corresponds to the maximum likelihood light model of a model-fit. The shaded regions show estimates of each light profile within $3\sigma$ confidence intervals. The black line shows the tangential critical curve of the mass model, the white line the radial critical curve, and the black cross(es) towards the centre of each figure the centre(s) of each light profile component.} 
\label{figure:LightFit}
\end{figure*}

This section presents the results of fitting the F390W and F814W images of Abell 1201 with different light models, which is performed in the Light pipeline. \cref{figure:Light2D} shows projected two-dimensional images of each model light profile, for the single Sersic model (top left panel), the double Sersic model assuming no geometric alignments (top-centre and top-right panels) and the triple Sersic model (bottom panels). The single Sersic model infers a compact central bulge with Sersic index $n^{\rm bulge} \simeq 4$, consistent with a massive elliptical galaxy. The double Sersic model decomposes the lens galaxy's light into two distinct components, consisting of a compact bulge similar to the single Sersic fit but with a much lower value of $n^{\rm bulge} \simeq 1.25$, surrounded by a more extended and elliptical component where $n^{\rm disk} \simeq 1.3$. The half-light radius of this extended component is $R_{\rm eff}^{\rm disk} \simeq 5.0"$, well beyond the strong lensing features. The triple Sersic model infers these two components, but includes a fainter additional inner structure.

\begin{table}
\resizebox{\linewidth}{!}{
\begin{tabular}{ l | l | l | l | l | l | l | l } 
\multicolumn{1}{p{1.1cm}|}{Filter} 
& \multicolumn{1}{p{1.3cm}|}{\centering Number of \\ Sersics} 
& \multicolumn{1}{p{1.5cm}|}{Aligned Elliptical Components} 
& \multicolumn{1}{p{1.6cm}|}{Aligned Centres} 
& \multicolumn{1}{p{1.5cm}|}{Evidence}  
\\ \hline
& & & & \\[-4pt]
F814W &  1 &       N/A &        N/A & 76664.15 \\[2pt]
F814W & 2 & \checkmark & \checkmark & 77616.09  \\[2pt]
F814W & 2 & \ding{55} & \checkmark & 78049.53 \\[2pt]
F814W & 2 & \ding{55} & \ding{55}  & 78181.48 \\[2pt]
F814W & 3 & \ding{55} & \ding{55}  & \textbf{78193.40} \\[2pt]
\hline
F390W & 1 &       N/A  &        N/A & 123604.22 \\[2pt]
F390W & 2 & \checkmark & \checkmark & 123962.79 \\[2pt] 
F390W & 2 & \ding{55}  & \checkmark & 124275.33 \\[2pt]
F390W & 2 & \ding{55}  & \ding{55}  & \textbf{124663.78} \\[2pt]
F390W & 3 & \ding{55}  & \ding{55}  & N/A \\[2pt]
\end{tabular}
}
\caption{The Bayesian Evidence, $\ln \mathcal{Z}$, of each model-fit performed by the Light pipeline, which compares models with one, two or three Sersic profiles. Fits to both the F814W and F390W images are shown. Models which make different assumptions for the alignment of the $(x,y)$ centre and $(\epsilon_{\rm 1}, \epsilon_{\rm 2}$) elliptical components of the bulge, disk and envelope are shown. A tick mark indicates that this assumption is used in the model, for example the second row is a model where both the elliptical components and centres are aligned. The triple Sersic model for the F390W is omitted because it went to unphysical solutions where one Sersic component was used to fit structure in the lensed source.}
\label{table:LightMC}
\end{table}

The Bayesian evidence values, $\ln \mathcal{Z}$, of the light models informs us which provides the best fit to the data. These are given for both F390W and F814W images in \cref{table:LightMC}. Models assuming a single Sersic profile give significantly worse fits than those using multiple profiles, indicating it does not capture the extended component. Three models assuming two Sersic profiles are compared, where: (i) their centre and elliptical components are aligned; (ii) their centres are aligned but elliptical components are not and; (iii) their centres are also free to vary. For both images model (iii) is preferred, with a value of $\Delta \ln \mathcal{Z} > 100$ the other models for the F814W data. For the F814W image a triple Sersic (with all geometric parameters free to vary) gives a value $\ln \mathcal{Z} = 11.55$ above that of the two Sersic model, indicating that it is the marginally favoured model. 

\cref{figure:LightFit} shows the normalized residuals of these fits. For the single Sersic model and models with geometric alignments residuals are evident around the lens galaxy's centre in both the F814W and F390W bands, consistent with the Bayesian evidences. In the F814W image the double Sersic model with free centres and the triple Sersic model gave a significant increase in $\ln \mathcal{Z}$. However, the improvements are not visible in the residuals, indicating they improve the light model fractionally over many pixels. 

The lower panels of this figure show 1D plots of the intensity as a function of radius for each component. The inner structure contributes to most of the stellar light within $\sim 1.0"$ where the counter image is observed, whereas at the location of the giant arc the extended component makes up over $95\%$ of the total emission. They also show that the outer component makes up the majority of the lens galaxy's total luminous emission, albeit most is beyond the $3.0"$ radius where the lensed source is constrained. 