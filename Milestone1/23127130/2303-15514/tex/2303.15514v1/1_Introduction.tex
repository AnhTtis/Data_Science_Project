\section{Introduction}\label{Intro}

\begin{figure*}
\centering
\includegraphics[width=0.32\textwidth]{data/data_f390w.pdf}
\includegraphics[width=0.32\textwidth]{data/x2_data_sub.pdf}
\includegraphics[width=0.32\textwidth]{data/x2_data_sub_zoom.pdf}
\includegraphics[width=0.32\textwidth]{data/data_f814w.pdf}
\includegraphics[width=0.32\textwidth]{data/x3_data_sub.pdf}
\includegraphics[width=0.32\textwidth]{data/x3_data_sub_zoom.pdf}
\caption{The observed images (left column), masked and lens subtracted images (middle column) and images zoomed in on the central regions containing the counter image (right column) of Abell 1201. The top row show the HST optical image taken using the F390W filter, the bottom row shows an image taken at near infrared wavelengths using the F814W filter, which are both in units of electrons per second. The lens subtractions are performed using the highest likelihood model found for each image, however their visual appearance does not change significantly for other high likelihood models. The counter image of the giant arc can be clearly seen at both wavelengths but has much higher contrast and more clumpy structure in the bluer F390W nwaveband. In the F814W image, residuals from the lens light subtraction around the coordinates (0, 0)\,arcsec are seen; these are not a central image of the source galaxy, which would be brighter in F390W. The black star marks a line-of-sight galaxy at $z = 0.273$ which is included in certain lens models.}
\label{figure:Data}
\end{figure*}

Supermassive black holes (SMBHs) have emerged as an integral part of models of galaxy formation and evolution, owing to the tight correlation observed between SMBH mass, $M_{\rm BH}$, and host galaxy bulge velocity dispersion, bulge mass and other galaxy properties \citep{Kormendy2013, Graham2012, Bosch2016}. It is posited that a SMBH resides at the centre of every galaxy and that galaxies and SMBHs coevolve with one another from their initial formation in the early Universe \citep{Heckman2014, Smith2019}. The mass of an individual SMBH, $M_{\rm BH}$, can be measured via spatially resolved dynamics of nearby tracers such as stars and gas \citep{Davis2017a, Thater2019}. This technique has provided over $100$ measurements of $M_{\rm BH}$ which show tight correlations with other galaxy properties such as bulge luminosity or velocity dispersion \citep{Kormendy1995, Ferrarese2000, Gebhardt2000, Graham2001a}. The need for spectroscopy at high spatial resolution that resolves the SMBH's sphere of influence restricts this approach to nearby galaxies, preventing the study of how these relations evolve with redshift. Spectral fitting of active galactic nuclei \citep{Peterson2004, McLure2004, Shen2013} and reverberation mapping techniques can provide measurements of $M_{\rm BH}$ in higher redshift galaxy populations which therefore enable evolutionary studies. However, these observations necessitate that the galaxy's SMBH is actively accreting, bringing in potential selection effects. A method that can measure $M_{\rm BH}$ for non-active galaxies outside the local Universe would be highly complementary to these existing approaches. Analysing the strong gravitational lensing of background sources, acting in some specific (and perhaps rare) circumstances and configurations, might provide such a technique.

In this paper, we present a re-examination of the strong-lensing brightest cluster galaxy (BCG) in Abell 1201. A tangential gravitational arc was first identified in shallow {\it Hubble Space Telescope} (HST) WFPC2 images of this system, by \cite{Edge2003}. Compared to most cluster lenses, the arc is unusual in being formed at small projected radius from the BCG ($\sim2$\,arcsec; $\sim$5\,kpc). \cite{Edge2003} found that a high ellipticity and/or strong external shear was necessary to match the arc shape. Integral-field spectroscopic data later revealed a faint counter-image to the main arc, projected even closer to the lens centre ($\sim$0.3\,arcsec; $\sim$1\,kpc) \citep{Smith2017a}. Using a simplified position-based model of the lensing configuration, \citep{Smith2017a} argued that an additional mass of $\sim$10$^{10}$\,M$_\odot$ at small radius was necessary to reproduce the counter-image as observed. The spatially-resolved stellar kinematics support this conclusion \citep{Smith2017}. The authors concluded that the necessary central mass could be a SMBH, but with the limited imaging data available, and the rudimentary lensing analysis employed, a degeneracy with the inner stellar mass distribution of the lens could not be excluded. 

\begin{figure*}
\centering
\includegraphics[width=0.24\textwidth]{model_fit/f390w/image_2d.pdf}
\includegraphics[width=0.24\textwidth]{model_fit/f390w/model_image.pdf}
\includegraphics[width=0.24\textwidth]{model_fit/f390w/reconstructed_image.pdf}
\includegraphics[width=0.24\textwidth]{model_fit/f390w/reconstruction.pdf}
\includegraphics[width=0.24\textwidth]{model_fit/f814w/image_2d.pdf}
\includegraphics[width=0.24\textwidth]{model_fit/f814w/model_image.pdf}
\includegraphics[width=0.24\textwidth]{model_fit/f814w/reconstructed_image.pdf}
\includegraphics[width=0.24\textwidth]{model_fit/f814w/reconstruction.pdf}
\caption{Fits to HST imaging of Abell 1201 via {\tt PyAutoLens}. The observed data (left column), the image-plane model images of the lens and source galaxies (left-centre column), the lensed source only (right-centre column) and source-plane pixelized source reconstruction (right column) are shown. The top row shows fits to the F390W and bottom row the F814W wavebands respectively. All images are in units of electrons per second. The lens model is the maximum likelihood model inferred at the end of the first SLaM (see \cref{SLAM}) pipeline run, which produces a lens subtracted image. The black lines show the mass model's tangential critical curve for all panels in the image plane (central columns) and the tangential caustic for panels in the source plane (right hand column).}
\label{figure:ModelFit}
\end{figure*}

%% Would it be possible to plot the source-plane U and I images with exactly the same axis limits for better comparability?

Here, we analyse new HST WFC3/UVIS imaging of higher spatial resolution and greater signal-to-noise ratio, using advanced lens modelling techniques, to reassess the evidence for a lensing-detected SMBH in Abell 1201. We show that the detailed structure observed in the 
counter image constrains the inner mass distribution of the lens, and allows us to place constraints on the central SMBH. We perform a Bayesian model comparison of a variety of lens models that include and omit a point-mass representing a SMBH. The majority of models favour the inclusion of a SMBH and produce consistent estimates of $M_{\rm BH}$, with some dependence on the form and flexibility of the assumed lens galaxy mass model. This work marks the second observation of a strong lens that provides constraints on the SMBH at the centre of its lens galaxy, following the work of \citet{Winn2003} who detected the ``central'' image of a lensed source via radio observations. Our study is the first where a measurement of $M_{\rm BH}$ is inferred via strong lensing (as opposed to an upper limit) and does so without the rare observation of a central image. 

Over the next decade, of order one-hundred thousand strong lenses will be discovered by cosmological surveys such as Euclid, LSST and SKA \citep{Collett2015}, a three orders of magnitude increase over the hundreds of systems that are currently known \citep{Bolton2008a, Shu2016, Sonnenfeld2013b, Bolton2012, Shu2016}. This will naturally lead to the discovery of more exotic and peculiar strong lens systems \citep{OrbandeXivry2008}, whose rare lensing configurations may provide constraints on $M_{\rm BH}$. We conclude with a discussion of whether galaxy-scale strong lensing can become a viable technique to measure large samples of SMBH masses in the future\footnote{For other lensing-related techniques, see also \citet{EventHor} for a measurement of the SMBH at the centre of M87 by mapping the lensed shadow of surrounding gas, \citet{Banik2019} for a discussion of using strong lensing to detect intermediate mass black holes, \citet{Chen2018} and \citet{Mahler2022} for discussions of searching for wandering SMBHs in strong lensing galaxy clusters and \citet{Hezaveh2015, Tamura2015, Wong2017, Quinn2016} for studies based around strong lens central images.}

%% [RJS: \S -> Section per MNRAS style.]
This paper is structured as follows.
In Section~\ref{Data}, we describe the Hubble Space Telescope imaging of Abell 1201.
In Section~\ref{Method}, we describe the {\tt PyAutoLens} method and model fits performed in this work.
In Section~\ref{Results}, we present the results of model fits using a variety of lens models.
In Section~\ref{Discussion}, we discuss the implications of our measurements, and we give a summary in \S\ref{Summary}.
{We assume a Planck 2015 cosmology throughout \citep{PlanckCollaboration2015a}}.
Text files and images of every model-fit performed in this work are available at \url{https://github.com/Jammy2211/autolens_abell_1201}. Full \texttt{dynesty} chains of every fit are available at \url{https://doi.org/10.5281/zenodo.7695438}.
