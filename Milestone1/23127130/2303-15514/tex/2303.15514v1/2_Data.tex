\section{Data}\label{Data}

We acquired the new {\it HST} imaging of Abell 1201 in Programme 14886 using the Ultraviolet and VISible channel on the Wide Field Camera 3 (WFC3/UVIS). A total of five exposures with a total integration time of 7150\,s were taken in the F390W bandpass, tracing the clumpy rest-frame ultraviolet emission from star-forming regions in the source galaxy. This filter probes wavelengths shorter than the 4000-\AA\ break at the redshift of Abell 1201; hence the foreground light of the lens is suppressed and the contrast of the source enhanced. Additionally, we acquired three exposures in F814W, totalling 1009\,s, to trace the distribution of stellar mass in the BCG. The observatory-provided reduced single-exposure images were registered and combined using {\sc astrodrizzle}, projecting onto an output pixel scale of 0.04\,arcsec. An accurate estimate of the point-spread function (PSF) is required for the lens modelling. To this end, we employed  the empirical PSF provided by STScI \footnote{\url{
https://www.stsci.edu/hst/instrumentation/wfc3/data-analysis/psf}}, as appropriate to the position of the target in each individual exposure, and propagated the PSF images through the same stacking process as for the real observation. The final combined images of Abell 1201 in the two bandpasses are shown in \cref{figure:Data}.

\cref{figure:Data} shows the %{\it Hubble Space Telescope (HST)} 
F390W and F814W imaging, 
%of Abell 1201, 
alongside lens-subtracted versions which highlight the lensed source galaxy. There is a giant arc 2.0--3.0\,arcsec away from the lens galaxy on one side of the lens with a counter image just $\sim$0.3\,arcsec ($\sim 0.9\,$kpc) from the lens galaxy centre. The lens itself is a cD galaxy residing in the central regions of a galaxy cluster, in contrast to most galaxy-scale (e.g. Einstein radius $< 5.0$\,arcsec) strong lens systems which are massive elliptical field galaxies and not in a cluster environment. 

The cluster Abell 1201 has also been investigated. X-ray analysis reveals an offset gas core 500 kpc northwest of the lens \citep{Ma2012}, which is interpreted as a tail of gas stripped from the offset core. The gas has different a different density, entropy and temperature than gas in the surrounding area, providing evidence indicative of a minor merger at second core passage. Alignment between the mass distribution of Abell 1201's BCG mass distribution (inferred via lens modeling performed by \citealt{Edge2003}) and the offset core is also noted, which could be the result of a sloshing mechanism.