\section{Models With Shallow Inner Density}\label{Radial}

\begin{figure}
\centering
\includegraphics[width=0.235\textwidth]{radial/demo/grad_data.pdf}
\includegraphics[width=0.235\textwidth]{radial/demo/reconstruction.pdf}
\caption{
An example model where the lens mass model has a shallow inner density, which forms a larger radial critical curve than solutions presented in the main paper. The left panel shows the observed data, with radial and tangential critical curves (white and black respectively) overlaid. The right panel shows the corresponding source plane and source reconstruction, with the radial and tangential caustics (white and black respectively) overlaid.
} 
\label{figure:RadialDemo}
\end{figure}

We encountered an alternative family of solutions which are characterized by: (i) a shallow inner density profile that forms a larger radial critical curve than the solutions presented in the main paper, which cuts through the inner regions of the counter image and; (ii) the counter image reconstruction producing a pair of merging images (the models in the main paper reconstruct a single counter image). An example of such a model is shown in Fig.~\ref{figure:RadialDemo}.

For decomposed models, these solutions are found when the radial gradient parameters (e.g. $\Gamma^{\rm bulge}$) are below zero and there is less mass relative to light. The low Sersic indices of the lens galaxy’s light profiles ($n^{\rm bulge}$ = ~1.28 and $n^{\rm disk}$ = ~1.16) also help to produce a shallow inner density. For the BPL model, these correspond to solutions where the inner slope $t^{\rm mass}_{1} \sim 0.0$, the outer slope $t^{\rm mass}_{2} \sim 0.7$, and the break radius is $r^{\rm mass}_{\rm B} \sim 0.25"$. We verify that this family of models without a SMBH do not fit the data as well as models with a SMBH by performing \texttt{dynesty} fits, where the priors on certain mass-model parameters are constrained to uniform priors that restrict the analysis to these solutions. The priors can be found at \url{https://zenodo.org/record/7695438}.

\begin{figure*}
\centering
\includegraphics[width=0.241\textwidth]{radial/fit/f390w/grad_data_zoomed.pdf}
\includegraphics[width=0.241\textwidth]{radial/fit/f390w/grad_image_zoomed.pdf}
\includegraphics[width=0.241\textwidth]{radial/fit/f390w/grad_norm_zoomed.pdf}
\includegraphics[width=0.241\textwidth]{radial/fit/f390w/grad_source_recon_zoomed.pdf}
\caption{
Zoom-ins of the observed counter image in the F390W data (left panel), the model lensed source (left-centre panel), the normalized residuals (right-centre panel) and the source reconstruction (right panel). These results are for the double Sersic plus NFW decomposed model-fits without a SMBH, where the parameter priors allow for solutions with a shallow inner density and large radial critical curve. The tangential caustic is shown by a black line and the radial critical curve and caustic are shown with a white line.
} 
\label{figure:RadialFit}
\end{figure*}

\begin{table}
\resizebox{\linewidth}{!}{
\begin{tabular}{ l | l | l | l} 
\multicolumn{1}{p{1.1cm}|}{Filter} 
& \multicolumn{1}{p{1.3cm}|}{Model} 
& \multicolumn{1}{p{1.3cm}|}{Shallow Density}  
& \multicolumn{1}{p{1.5cm}|}{SMBH}  
\\ \hline
F390W & Decomposed & 125649.72 & 125699.06  \\[1pt]
F390W & BPL & 125548.22 & 125693.78 \\[0pt]
\hline
F814W & Decomposed & 78289.00  & 78332.19  \\[1pt]
F814W & BPL & 78238.79 & 78329.28  \\[0pt]
\end{tabular}
}
\caption{
The Bayesian evidence, $\ln \mathcal{Z}$, of each model-fit performed by the Mass pipelines using: (i) a decomposed mass model assuming two Sersic profiles, an elliptical NFW and external shear or; (ii) a BPL mass model with external shear. Both models have the priors on various parameters adjusted such that they have a shallower inner density and can form a large radial critical curve. Log evidences are compared to the values found in the main paper, for models including a SMBH. Fits to both the F390W and F814W images are shown, where the F390W fits assume the Sersic parameters of the F814W image for the stellar mass. The favoured model is always that with a SMBH, because models with a shallow inner density fail to reconstruct the counter image's structure (see \cref{figure:RadialFit}).
}
\label{table:SMBHMCRadial}
\end{table}

The maximum likelihood solution for the double Sersic decomposed mass model are shown in \cref{figure:RadialFit}. The reconstructed counter image is split in two, and fails to capture the appearance of the counter image in the data. For this model, the log Bayesian evidence value is $\mathcal{Z} = \sim\,125649$, which is significantly below models with a SMBH which have a log evidence of $\mathcal{Z} =\sim\,125699$. \cref{table:SMBHMCRadial} compares the log Bayesian evidence values for the BPL model fits with a shallower inner density and also includes the values for the F814W. For both the F390W and F814W images, these solutions provide significantly worse fits to the data than models including a SMBH, confirming that they are ruled out by the data.

These fits also confirm that the central emission seen in the F814W data (\cref{figure:LightFit2}; within magenta circle) is not a central image. Lens model fits using cored mass profiles would reconstruct the counter image, if it were the physically correct solution. The fact these solutions are not inferred confirms it is not a central image.