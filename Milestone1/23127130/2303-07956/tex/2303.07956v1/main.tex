\documentclass{article}
\usepackage[utf8]{inputenc}

\usepackage{amssymb,amsmath,amsthm,color}
\usepackage{faktor}
\usepackage{hyperref}
\usepackage{cleveref}
\usepackage{csquotes}
\usepackage{graphicx}
\usepackage{geometry}

\usepackage{thmtools}
\usepackage{thm-restate}
\usepackage{bbm}
\usepackage{dsfont}

\newcommand{\R}{\mathbb{R}}
\newcommand{\Z}{\mathbb{Z}}
\newcommand{\PP}{\mathbb{P}}

\newcommand\restr[2]{\ensuremath{\left.#1\right|_{#2}}}

\newtheorem{theorem}{Theorem}
\newtheorem*{theorem*}{Theorem}
\newtheorem{lemma}[theorem]{Lemma}
\newtheorem{observation}[theorem]{Observation}
\newtheorem{remark}[theorem]{Remark}
\newtheorem{example}[theorem]{Example}
\newtheorem{claim}[theorem]{Claim}
\newtheorem{question}[theorem]{Question}
\newtheorem{conjecture}[theorem]{Conjecture}
\newtheorem{proposition}[theorem]{Proposition}
\newtheorem{fact}[theorem]{Fact}
\newtheorem{definition}[theorem]{Definition}
\newtheorem{corollary}[theorem]{Corollary}

\newcommand{\comm}[1]{\textcolor{red}{[COMMENT: #1]}}


\title{The number of tiles of $\Z^d$}
\date{February 2023}

\author{Itai Benjamini, Gady Kozma, Elad Tzalik }
\begin{document}

\maketitle

\begin{abstract}
    \noindent It is proved that the number of subsets of $[n]^d$ that tile $\Z^d$ is $\left(3^{\frac{1}{3}}\right)^{n^d \pm o(n^d)}$.
\end{abstract}

\section{Introduction}


A set $S \subseteq \Z^d$ tiles $\Z^d$ by translations if one has $\Z^d = \dot\bigcup S+z_i$ for some $z_i \in \Z^d$. Tiling $\Z^d$
with translations is a natural problem that has been widely studied in many works \cite{COVEN,Wang,Newman,LabaLondner,LabaLondner2}.
The main aim of this note is to count how many subsets of $[n]^d$ can tile $\Z^d$. We remark that this differs from the recent work of Stern \cite{Stern}, which studied the problem of counting for a given $S \subseteq \Z^d$, how many different ways there are to tile $S$ ($S$ may not tile $\Z^d$).

Let $t_{n,d}$ the number of subsets of $[n]^d$ that tile $\Z^d$. We use the notation $v_n:=n^d$ the size of $[n]^d$. We prove:


\begin{theorem}\label{thm : number of 1 tiles}
 $(3^{\frac{1}{3}})^{v_n-o(v_n)} \leq t_{n,d} \leq (3^{\frac{1}{3}})^{v_n+o(v_n)}$
\end{theorem}


\section{Proof of \Cref{thm : number of 1 tiles}}

We start with proving the upper-bound. Let $S$ be a subset of $[n]^d$. Since for $\alpha \leq \frac{1}{2}$ one has $\sum_{i=0}^{\alpha v_n}{v_n \choose i} \leq 2^{H(\alpha)v_n}$, where $H$ denotes the binary entropy function, it's enough to count tiles $S$ with $\alpha v_n \leq |S| \leq (1-\alpha)v_n$ for small fixed $\alpha$ ($\alpha=0.1$ is enough) as the number of all other sets is much smaller then the upper-bound.

Let $k$ be a positive integer, and consider a configuration of translations that tile the cube $[kn]^d$, i.e. let $z_1,\ldots z_{\ell} \in \Z^d$ be translations such that $[kn]^d \subseteq \dot{\cup} S+z_i $. Notice that by the assumption on the size of $S$ it is enough to consider sequences with $\ell = O(k^d)$.
We fix $z_1,\ldots,z_{\ell}$ and count the number of tiles $S$ that a set $W$ that contains $[kn]^d$ with this translations. Clearly we may also assume that $z_1=0$ meaning we use $S$ as a tile. Define for $j \in \Z^d$: 

\[A_j = \left\{ j-z_i \mid i \in [\ell] \right\} \cap [n]^d .\] 

\;

\noindent The set of all possible elements of $[n]^d$ that can cover $j$ by the fixed translations. Since $S$ tiles $[kn]^d$ the set $F_j = S\cap A_j$ satisfies $|F_j|=1$ for $j \in [kn]^d$.

Let $\mathcal{S}$ be a uniform tile (conditioned of being of volume at least $\alpha v_n$) that tiles a set containing $[kn]^d$ with the fixed translations $z_1,\ldots,z_\ell$. We bound the entropy of $\mathcal{S}$ and show $H(\mathcal{S}) \leq \log_2 \left( 3^{\frac{1}{3}}+ o_{k}(1) \right) \cdot v_n $ which implies that for every fixed $z_1,\ldots,z_\ell$ we have at most $\left(3^{\frac{1}{3}}+o_{k}(1)\right)^{v_n}$ tiles. Since the volume of $S$ is at least $\alpha v_n$ there is at most $\left(k^d v_n \right)^{O(k^d)}$ such sequences and union bounding over all of them will conclude the proof.

 Write $\mathcal{S}=(X_{i})_{i \in [n]^d}$ where each $X_i$ is the indicator of $i\in \mathcal{S}$ for $i \in [n]^d$. Notice that for each $i \in [n]^d$, the set $\left\{j \mid i \in A_j \right\}$ is exactly $\{ i+z_1,\ldots i+z_\ell\} \cap [kn]^d$. Also, note that for each $z_j$ such that $i+z_j \not \in [kn]^d$, and a tile $S$ the set $S+z_j$ is strictly contained in the $n$-neighbourhood (all vertices of distance $n$) of the boundary of the cube $[kn]^d$. Let $B$ be the size of this $n$-neighbourhood and consider $J= \mid\{ j \mid i+z_j \not \in [kn]^d \} \mid$ then by volume considerations, since all $S+z_j$ fit disjointly into $B$:

\begin{align}\label{eq - vol packing}
    \alpha v_n J \leq |B| \leq  (k+2)^d v_n- (k-2)^d v_n
\end{align}

\;

\noindent and therefore $J \leq \frac{5dk^{d-1}}{\alpha}$ for large enough $n$ and $k=\omega(1)$.

To bound on $H(\mathcal{S})$, we will use Shearer's inequality(\cite{CHUNG198623}) which allows one to upper bound the entropy of $\mathcal{S}$ by entorpy of projections of $\mathcal{S}$. Concretely,  since $(A_j)_{j \in \Z^d}$ cover each $i \in [n]^d$ at least $\ell-J$ times then:

\[ H(\mathcal{S})\leq \frac{1}{\ell-J} \sum_{j} H(\restr{X}{A_j}) .\]


\noindent Where $\restr{X}{A_j}$ is the random variable of the $A_j$ coordinates of $(X_{i})_{i \in [n]^d}$. Notice that $\restr{X}{A_j}$ is exactly the random variable $F_j$, and observe that $F_j$ is \emph{always a set of size exactly one}, since $\mathcal{S}$ is supported on tiles. Thus $H(F_j) \leq \log_2(|A_j|)$, and therefore:

\begin{align}\label{eq : entropy A set bound}
     H(\mathcal{S}) \leq \frac{1}{\ell-J}\sum_{j}\log_2(|A_j|) \leq \frac{1}{\ell-J}\log_2 \left( \Pi_{j} |A_j| \right).
\end{align}


\noindent Notice that $\sum |A_j| \leq n^d \ell$ (as each element of $[n]^d$ participates in $\ell$ shifts). We will need the following observation (which is a discrete form of Jensen's inequality):

\begin{observation}\label{obs : discrete jensen}
    Let $x_1,\ldots ,x_n$ be non-negative integers with $\sum x_i = t$, with $t=an+b$ for $b<n$. Then $\Pi x_i \leq (a+1)^b \cdot a^{n-b}$.
\end{observation}

\begin{proof}
    Since for $k\geq 0$ and $s>1$ we have $k(k+s)<(k+\lfloor \frac{s}{2} \rfloor) (k+\lceil \frac{s}{2} \rceil)$ any maximum contains elements $x_i$ that differ by at most $1$ from one another and the only one such choice.
\end{proof}



\noindent Set $\ell n^d = \lfloor \frac{\ell}{k^d} \rfloor \cdot (kn)^d + \left\{ \frac{\ell}{k^d} \right\} \cdot (kn)^d$ where $\{\}$ denotes the fractional part of a real number. Applying \Cref{obs : discrete jensen}, to the RHS of \Cref{eq : entropy A set bound} we get:

\begin{align}
    H(\mathcal{S}) &\leq \frac{1}{\ell-J} log_2 \left( \left(\lfloor \frac{\ell}{k^d} \rfloor+1 \right)^{\left\{ \frac{\ell}{k^d} \right\} \cdot k^d v_n} \lfloor \frac{\ell}{k^d} \rfloor^{(1-\left\{ \frac{\ell}{k^d} \right\})k^d v_n} \right) \\
    &= \frac{1}{\ell-J} log_2 \left( \left(\lfloor \frac{\ell}{k^d} \rfloor+1 \right)^{\left\{ \frac{\ell}{k^d} \right\} \cdot k^d} \lfloor \frac{\ell}{k^d} \rfloor^{(1-\left\{ \frac{\ell}{k^d} \right\})k^d} \right)\cdot v_n \\
    &= \frac{\ell}{\ell-J} log_2 \left( \left(\lfloor \frac{\ell}{k^d} \rfloor+1 \right)^{\left\{ \frac{\ell}{k^d} \right\} \cdot \frac{k^d}{\ell}} \lfloor \frac{\ell}{k^d} \rfloor^{(1-\left\{ \frac{\ell}{k^d} \right\}) \cdot \frac{k^d}{\ell}} \right)\cdot v_n \label{eq : fractional quantity}
\end{align}

To bound the term inside the $\log$ by $3^{\frac{1}{3}}$, we note the following lemma which follows from simple differentiation:

\begin{lemma}
    Let $N$ be a positive integer integer and $r \in [0,1]$. Then the function:
    \[ f(r) = (N+1)^{r\frac{1}{N+r}} \cdot N^{(1-r)\frac{1}{N+r}}  \]

    \noindent is maximized at $r \in \{ 0,1\}$
\end{lemma}

    By the lemma we get that the term inside the $\log$ in \Cref{eq : fractional quantity} is bounded by $N^{\frac{1}{N}}$ for an integer $N$. Notice that since $x^{\frac{1}{x}}$ increases at $[1,e)$ and decreases at $(e,\infty)$, together with the fact that $2^{\frac{1}{2}}< 3^{\frac{1}{3}}$ we know that:
    
    \[max_{N \in \mathbb{N}} N^{\frac{1}{N}} = 3^{\frac{1}{3}}\]

    \;
    
    \noindent therefore $H(\mathcal{S}) \leq \frac{\ell}{\ell-J} \log_2(3^{\frac{1}{3}}) \cdot v_n$, hence the number of tilings with shifts $z_1,\ldots, z_{\ell}$ and $|S| \geq \alpha v_n$ is at most:
    
    \[\left(3^{\frac{1}{3}} \right)^{\frac{\ell}{\ell-5dk^{d-1}/\alpha}v_n}\]

\;

\noindent In conclusion, setting $k=n^{\frac{1}{2d}}$ concludes the proof of the upper bound as:

\[ t_{n,d} \leq  { v_n \choose \alpha v_n } + (k^d v_n)^{O(k^d)}\cdot \left( 3^{\frac{1}{3}} \right)^{\frac{k^d}{k^d-5d k^{d-1} /\alpha} \cdot v_n} \leq (3^{\frac{1}{3}})^{v_n + o(v_n)}.\]

\begin{remark}
    It also follows from the proof that all translation sequences for which $\frac{\ell}{k^d} \notin (3-\varepsilon, 3+\varepsilon)$ yield a stronger bound on the entropy. This implies that as $n$ goes to $\infty$ the size of a random tile divided by $n^d$ converges to $\frac{1}{3}$ a.s. \end{remark}
\;


For the lower-bound, notice that from each element $(a_1,\ldots a_t) \in \{0,1,2\}^{t}$ one can construct the set $S = \{ a_i t +i\}_{i=1}^t$ contained in $[3t]$. Since $S$ contains one element from each $mod \text{ } t$ representitive it follows that $\Z = \dot\bigcup_{i\in \Z} S+t\cdot i$. Thus there are at least $3^t$ tiles in $[3t]$.

For larger $d$, fix $t$ and consider an element of $(c,u)=\{0,1,2\}^{t \cdot (3t)^{d-1}} \times [3t]^{d-1}$ and take the subset $S_{c,u}=\{v \mid c(v-\lfloor \frac{v_1}{t} \rfloor u) = \lfloor \frac{v_1}{t} \rfloor \}$. Visually, $c$ is a coloring of a face of the cube $[3t]^{d}$ tiled along the direction $u$. Each such $S$ tiles $\Z^d$ by the translations of the lattice generated by the vectors $(t,u) \cup \{3t e_i\}_{i=2}^d$ where $e_i$ is the standard basis vectors. Therefore the number of tiles in $[n]^d$ can be $n^{d-1}(3^{\frac{1}{3}})^{v_n}$


\section{Concluding remarks}

We remark that the proof of the \emph{upper-bound} of \Cref{thm : number of 1 tiles} is robust and can be applied in other situations. E.g. consider the same arguments gives an upper-bound on the number of tiles contained in an $n$-ball around the origin, since the volume growth of balls in $\Z^d$ is polynomial and one has a similar inequality as the one in \Cref{eq - vol packing}. In contrast, we do not know how to obtain a tight lower-bound in such case (we believe the upper-bound is \emph{not tight} in this case). 
In case a space has exponential growth it is not hard to find examples where the number of tiles is larger then $3^{\frac{1}{3}}$ to the power of the volume (consider the $d$-ary tree, where tiles are contained in an $n$-ball around the root and can be shifted down the tree).

We conclude with several conjectures and questions:

\begin{conjecture}
    Fix an integer $k$, and pick $x \in [n]^d$ uniformly at random. Let $\mathcal{S}$ be a random tile contained in $[n]$ and let $B_k$ be the $k$-ball around $x$. Then the local limit of $\restr{\mathcal{S}}{B_k}$ (see \cite{BS} for the definition) as $n$ tends to infinity is a product measure of Bernoulli $\frac{1}{3}$ R.Vs.
\end{conjecture}

\begin{conjecture}
    For each $d$ there is a constant $C(d)$ that depends only on $d$ such that: $t_{n,d} = O(n^{C(d)} \cdot (3^{\frac{1}{3}})^{v_n})$. Can one take $C(d)=d-1$?
\end{conjecture}

\begin{question}
    Let $t_{n,d}^{o}$ denote the number of tiles contained in the $n$-ball (euclidean) around the origin of $\Z^d$. Does $t_{n,d}^{o}$ grows exponentially in the volume of the ball with an exponent strictly smaller then $3^{\frac{1}{3}}$?
\end{question}


% \begin{question}
%     We think it is interesting to understand subsets that do not tile $\Z^d$, but only leave a small fraction of uncovered lattice points. 
%     For $0\leq \tau<1$, and a set $S$, we say $S$ $\tau$-tiles $\Z^d$ if there is $V \subseteq \Z^d$ such that $S$ tiles $V$, and 
%     \[\liminf_{n\rightarrow \infty}{ \frac{|V \cap B_n|}{|B_n|}} \geq 1-\tau\]

%     How many subsets of $[n]^d$ can $\tau$-tile $\Z^d$ as $n \rightarrow \infty$?  
% \end{question}


% we solved it
% \begin{question}
%     Does a set $S \subseteq \Z$ of density $0$ that tiles $\Z$, must be finite?
% \end{question}

\begin{question}
    What is the number of tilings of $\Z^d$ with $k$ tiles, each contained in $[n]^d$? Is it strictly larger then $(3^{\frac{1}{3}})^{k n^{d}}$?
\end{question}

\begin{question}
    Let $S$ be a random tile contained in $[n]^d$. What is the size of the largest connected component in the subgraph induced by $S$? The tiles that achieve the lower-bound correspond to site percolation on $[n]^{d-1}\times [\frac{n}{3}]$ with $p=\frac{1}{3}$. In particular,  site percolation on $\Z^d$ with $p=\frac{1}{3}$ is supercritical for $d\geq 3$ and subcritical for $d \leq 2$ \footnote{See \url{https://en.wikipedia.org/wiki/Percolation_threshold} } which may suggest a phase transition for having a linear sized connected component at $d = 3$.
\end{question}


\begin{question}
    In a recent breakthrough, Greenfeld and Tao \cite{GT} disproved the periodic tiling conjecture and constructed a tile in $\Z^3$ which doesn't admit a periodic tiling.
    
    Is a random tile contained in $[n]^d$ has only periodic tilings w.h.p.? If yes, is this tiling unique?
\end{question}

\begin{question}
    Take a random tile contained in $[n]^d$ and a random lattice $L$ for which it tiles. Does $\frac{1}{n}L$ converges to a limiting distribution on lattices? 
\end{question}

\begin{question}
    Let $M\subseteq \Z$ be any subset of size $n$. Are there at most $\left( 3^{1/3} + o(1) \right)^n$ tiles inside $M$? Is the number of tiles inside a set $M$ is maximized at sets consisting of $n$ consecutive integers? 
\end{question}

\begin{question}
    How many tiles are contained in $1,2,4,\ldots ,2^n$? Are the only tiles of size $1,2$? \footnote{Notice that if one considers $0,1,2,4,\ldots ,2^n$ then one can construct many tiles by choosing a prime of size $\approx \frac{n}{3}$ for which $2$ generates $\mathbb{F}_p^{\times}$ and picking one from each mod $p$ class to be in the tile.}
\end{question}

% \comm{The following two problems are not well defined yet...}
% \begin{question}
%     Add einstein type problem - radius of a tile implies it tiles + compactness and how many sets tile a ball around the origin. In particular - if it tiles the cube does it tile everything? (can we relate this to decidability?)
% \end{question}

% \begin{question}
%     Let $U$ be an open set and consider the best lattice inside it. what do we get?
% \end{question}

\paragraph{Acknoledgment.} I.B and G.K thank the Israel Science Foundation for their support.

\bibliographystyle{plainurl}
\bibliography{ref}

\end{document}

