\documentclass{amsart}
\usepackage{amssymb,latexsym, amsmath, amscd}
\usepackage{enumerate}
\usepackage{multirow}
\usepackage{fullpage, booktabs}
\usepackage{graphicx}
\usepackage[all]{xy}
\makeatletter
\@namedef{subjclassname@2010}{%
\textup{2010} Mathematics Subject Classification}
\makeatother


\newtheorem{thm}{Theorem}
\newtheorem{cor}[thm]{Corollary}
\newtheorem{lem}[thm]{Lemma}
\newtheorem{prop}[thm]{Proposition}

\theoremstyle{definition}
\newtheorem{defin}[thm]{Definition}
\newtheorem{rem}[thm]{Remark}
\newtheorem{exa}[thm]{Example}

\newtheorem*{xrem}{Remark}
\newtheorem*{thmm}{Theorem}

\numberwithin{equation}{section}


\newcommand\leftnarrow{%
        \mathrel{\vcenter{\mathsurround0pt
                \ialign{##\crcr
                        \noalign{\nointerlineskip}$\leftarrow$\crcr
                        \noalign{\nointerlineskip}$\dots$\crcr
                        \noalign{\nointerlineskip}$\leftarrow$\crcr
                }%
        }}%
}

\newcommand\F[1]{F_{#1,3}(\mathbb{R})}

\def\Z{\mathbb Z}
\def\M{\mathcal M}
\def\Y{\Upsilon}
\def\X{\Pi}
\def\R{\mathbb R}
\def\C{\mathbb C}
\def\I{\mathcal I}
\def\P{\mathbb{P}}
\def\Q{{Q}}
\def\algQ{\mathcal{Q}}
\def\PB{\overline{{B}}}
\def\PP{\overline{{P}}}
\def\PBgr{\overline{\mathcal{B}}}
\def\PBgrness{\overline{\mathcal{N}}}
\newcommand{\PPch}{\mathcal{A}}
\def\gr{\mathrm{gr}_*}
\def\h{\mathfrak{h}}
%%%%%%%%%%%%%

\begin{document}


\title{Round twin groups on few strands}

\author[J. Mostovoy]{Jacob Mostovoy}


%\address{Departamento de Matem\'aticas, CINVESTAV-IPN\\ Col. San Pedro Zacatenco, M\'exico, D.F., C.P.\ 07360\\ Mexico}
%\email{jacob@math.cinvestav.mx}


\begin{abstract} We study the fundamental group of the configuration space of $n$ ordered points on the circle no three of which are equal. We compute it for $n<6$ and describe its homology for $n=6$.  We also show how, for arbitrary $n$, this group can be assembled from planar braid groups, relate it to the pure cactus group and construct a cubical complex homotopy equivalent to its classifying space.
\end{abstract}

%\subjclass[2010]{Primary }

%\keywords{}

\maketitle

\section{Introduction}



The pure braid group on $n$ strands, defined as the fundamental group of the space of configurations of $n$ distinct points in $\C$, has many generalizations. The most straightforward way to produce other braid groups is to replace the complex plane with another surface (a manifold of real dimension 2). Configuration spaces on higher-dimensional simply-connected manifolds are simply-connected; however, they also produce braid-like groups if, instead of the fundamental groups one considers other homotopy functors as in \cite{CG}. Configuration spaces in 1-dimensional manifolds seem to be less appealing: indeed, the configuration space of $n$ distinct ordered points in $\R$ consists of $n!$ contractible pieces. Nevertheless, this space has several naturally defined (partial) compactifications whose fundamental groups turn out to be 
of interest.

Two of these ``real versions of the braid groups'' have been studied in some detail. The planar braid group, also known as the twin group, is the fundamental group of the space of $n$-tuples of  particles in $\R$ no three of which are allowed to coincide. Its elements can be represented by $n$-tuples of descending strands in a horizontal strip without triple intersections. Planar braid groups appeared in various contexts; in particular, in the theory of so-called doodles \cite{Kh2}, or in physics in the study of three-body interactions \cite{Threebody}.

Another closely related group is the cactus group \cite{HK}. Its ``pure'' version is the fundamental group of the moduli space of stable real rational curves with $n$ marked points. Although elements of the cactus group are not usually though of as braids, such a representation exists and may be useful in some situations, see \cite{MCact, MRP}.


In this note we observe that there is yet another group that deserves to be considered as the ``real version'' of the pure braid group and fits between pure planar braids and pure cactus groups. It is the fundamental group of the configuration space of $n$ ordered points on the circle $S^1=\R \cup \{\infty\}$ no three of which are allowed to coincide, and such that the $n$th point lies at infinity. We call it the \emph{round twin group on $n$ strands}. Here, we will identify the round twin groups on up to 5 strands and describe the round twin group on 6 strands as the fundamental group of a  link complement in a certain 3-manifold. We also show how round twin groups are assembled from pure planar braid groups and exhibit a long exact sequence involving their cohomology. Finally, we construct cubical complexes homotopy equivalent to the classifying spaces of the round twin groups and show how they can be simplified in the case of 5 and 6 strands so as to give an alternative description of the corresponding round twin groups.


\subsection{Planar braids on a line}
The  \emph{twin group} $\PB_n$, or the group of \emph{planar braids on $n$ strands}, has a presentation with the generators by $\sigma_1, \ldots, \sigma_{n-1}$ and the relations
\[
\begin{array}{rcll}
\sigma_i^2&=& 1&\quad \text{for all\ } 1\leq i < n;\\
\sigma_i\sigma_j&=& \sigma_j\sigma_i &\quad \text{for all\ } 1\leq i, j <n \text{\ with\  } |i-j|>1.
\end{array}
\]
There is a homomorphism of $\PB_n$ onto the symmetric group $S_n$ which sends the generator $\sigma_i$ to the transposition $ (i\  i+1)$. The kernel of this homomorphism is called the \emph{pure twin group} or the \emph{planar pure braid group} $\PP_n$.  
It is the fundamental group of the 
configuration space $M_n=M_{n}(\R)$ of $n$ ordered particles in $\R$ no three of which are allowed to coincide.

The pure twin groups on 3, 4 or 5 strands are free on 1, 7 and 31 generators respectively, while  the pure twin group on 6 strands is a free product of 71 copies of the infinite cyclic group and 20 copies of the free abelian group on 2 generators, see \cite{MRo}. In general, the group $\PP_n$ has a minimal presentation whose relations are commutators \cite{MPres}. The cohomology ring of $\PP_n$  is known (see \cite{B,DT}).


\subsection{On the terminology} The groups of planar braids were discovered independently several times and were given various names: Grothendieck cartographical groups \cite{VV}, twin groups \cite{Kh1,Kh2}, groups of flat braids \cite{Merk2}, traid groups \cite{Threebody}. The most descriptive of those, 
namely, ``flat braids'', has become inoperative after being used for a different object in the theory of virtual knots. The term ``planar braids'' used in \cite{MRo, MPres} is an attempt to produce the closest replacement to ``flat braids''. Unfortunately, it does not generalize too well to the situation considered in the present note, namely, that of planar braids drawn on the vertical wall of a cylinder. One might want to call them ``annular braids'';  as observed in \cite{Fa}, they form annular diagram groups in the terminology of \cite{GS}. However, the term ``annular braids'' has already been used for something entirely different, see \cite{KP}. For this reason, we will mostly use Khovanov's terminology of ``twin groups'' and refer to the elements of the corresponding groups as ``twins''.

\subsection{Twins on a circle} One may consider configuration spaces of points on a circle rather than a line. These lead to  \emph{annular} twin groups. 

The annular pure twin group $\PP_n(S^1)$ is the fundamental group of the configuration space $M_{n}{(S^1)}$ 
of $n$ ordered particles in $S^1$ no three of which may coincide. The space  $M_{n}(S^1)$ 
is an open subset in the $n$-dimensional torus $(S^1)^n$; its complement consists of the points $(z_1,\ldots, z_n)$ which satisfy $$z_i=z_j=z_k$$ for some triple of distinct indices $i,j,k$.

The \emph{full} annular twin group $\PB_n(S^1)$ has a presentation with the generators by $\alpha_1, \ldots, \alpha_{n}$ and $\eta$ and the relations  

\[
\begin{array}{rcll}
\alpha_i^2&=& 1&\quad \text{for all\ } 1\leq i \leq n;\\
\alpha_i\alpha_j&=& \alpha_j\alpha_i &\quad \text{for all\ } 1\leq i, j \leq n \text{\ with\  } i\not\equiv j\pm 1\mod{n};\\
\alpha_i\eta& = &\eta\alpha_{j};&\quad \text{where\ } j=i+1\mod{n}.
\end{array}
\]

Annular twins can be drawn on a vertical cylinder (which, topologically, is an annulus) in the same way as the planar braids are drawn in a plane, namely, as collections of descending strands, see Figure~\ref{first}. It is immediately clear that, with the help of the last relation, one can write a presentation for $\PB_n(S^1)$ which only has two generators. 

\begin{figure}[ht]
\includegraphics[width=3in]{first1.png}
\caption{Generators of $\PB_4(S^1)$: $\alpha_2$ on the left and $\eta$ on the right.}\label{first}
\end{figure}

There is a homomorphism $\PB_n(S^1)\to S_n$ sending $\alpha_i$ to $(i\  i+1)$ for $i<n$ and $\alpha_n$ to $(1\ n)$,
and $\eta$ to $(1\   2\   3\   \ldots\   n)$. The kernel of this homomorphism is precisely the  annular pure twin group $\PP_n(S^1)$. Note that while the symmetric group $S_n$ acts on  $M_{n}(S^1)$ permuting the labels of the particles, the full annular twin group $\PB_n(S^1)$ is \emph{not} the fundamental group of $M_{n}{(S^1)}/S_n$, since this action is not free. (In fact, $M_{n}{(S^1)}/S_n$ is easily seen to be simply connected).


Instead of the annular twin groups, it may be convenient to consider their subgroups that consist only of those twins whose $n$th strand is vertical. Then, if we think of $S^1$ as $\R\cup\{\infty\}$, this strand can be placed at the infinity and the twin can be drawn in a plane as in Figure~\ref{second}. 
\begin{figure}[h]
\includegraphics[width=2in]{second1.png}
\caption{The generator $\zeta$ of $\Y_5\subset \PB_6(S^1)$.}\label{second}
\end{figure}
We call such annular twins \emph{round twins}. The subgroups of all round twins in $\PB_n(S^1)$ and $\PP_n(S^1)$ will be denoted by $\Y_{n-1}$ and $\X_n$ respectively. (The mismatch in the indices has its origin in the standard notation for cactus groups, see Section~\ref{sectcact}). 

A presentation for $\Y_{n}$ is easy to obtain from the presentation for $\PB_{n+1}(S^1)$. For, instance, $\Y_{n}$ can be given by the generators $\sigma_1, \ldots, \sigma_{n-1}$ and $\zeta$ subject to the relations
\begin{equation}\label{relroundtwin}
\begin{array}{rcll}
\sigma_i^2&=& 1&\quad \text{for all\ } 1\leq i < n;\\
\sigma_i\sigma_j&=& \sigma_j\sigma_i &\quad \text{for all\ } 1\leq i, j < n \text{\ with\  } |i- j| <1;\\
\sigma_i\zeta& = &\zeta\sigma_{i+1};&\quad \text{for all\ } 1\leq i < n-1.
\end{array}
\end{equation}
In terms of the generators of $\PB_{n+1}(S^1)$, we have $\sigma_i=\alpha_i$ for $1\leq i <n$ and $\zeta = \alpha_n\eta$.
As in the case of $\PB_{n+1}(S^1)$, one can also write a presentation with two generators only. A presentation for the pure round twin group can be easily constructed by means of the Reidemeister-Schreier method as in \cite[Section~2.4]{MPres}; such a presentation, however, will be far from minimal.

As mentioned in the introduction, it may be tempting to think of twins and annular twins as  ``real versions'' of usual braids, which are paths of configurations of points in a complex plane. From this point of view, the group $\Y_n$ of round twins may be a good analogue of the braid group $B_{n}$. Indeed, we can think of braids on $n$ strands in $\C$ as braids on $n+1$ strands in $\C\cup\{\infty\}$, whose $n+1$st strand is vertical at infinity.

The relationship between the pure annular twin  groups and pure round twin groups is straightforward.
For each $n$, the configuration space $M_{n}{(S^1)}$ splits as a Cartesian product $S^1\times \Q_{n}$, where $\Q_{n}\subset M_{n}{(S^1)}$ is the subspace consisting of the points with $z_{n}=\infty$. As a consequence, we have
$$\PP_n(S^1)=  \Z\times \X_n$$
for all $n$ since $\X_n=\pi_1 \Q_n$. 

For low values of $n$ the pure round twin groups 
can be described explicitly as follows:
\begin{thm}\label{fewstrands}
The group  $\X_1$ is trivial. We have
\begin{align*}
\X_2&=\Z,\\ 
\X_3&=F_2,\\ 
\X_4&=F_4,\\
\X_5&=\pi_1 X_4,\\ 
\end{align*} 
where $X_4$ is the Riemann surface of genus 4 and $F_k$ stands for the free group on $k$ generators.
\end{thm}

When $n\leq 4$ this statement is easy to verify directly from the definition.  For $n=5$, it can be deduced from the fact that $\Q_5$ is a disk bundle over the orienting double cover of  the moduli space $\overline{\mathcal{M}}_{0,5}(\R)$ of stable real rational curves with 5 marked points; $\overline{\mathcal{M}}_{0,5}(\R)$ is a connected sum of 5 real projective planes. 

The connection with the moduli spaces comes from the fact that the space $\Q_n$ is homotopy equivalent to the quotient
$M_n(S^1)/SL(2,\R)$, with the natural action of $SL(2,\R)$ on $S^1=\R P^1$. This connection also gives us the following result:
\begin{thm}\label{sixstrands}
The group $\X_6$ is the fundamental group of the complement of a 10-component link in the 3-manifold which is the boundary of a 4-disk with five 1-handles. Its abelianization $H_1(\X_6)$ is a free abelian group of rank 15, generated by the classes of 1-handles and the meridians of the link components.
\end{thm}

This fact is a direct consequence of the known description of $\overline{\mathcal{M}}_{0,6}(\R)$ as a blowup of a certain configuration of 5 points and 10 lines in $\R\mathbb{P}^3$.
We will also explicitly describe $\Q_6$ as a cubical complex of dimension 2 with 30 vertices, 120 edges and 90 faces. 


\medskip

In general, we will show how to assemble $\X_n$ from the planar pure braid groups $\PP_{k}$ with the tools of Bass-Serre theory. 
 
\begin{thm}
$\X_n$ is the fundamental group of a graph of groups with $n$ vertices: one vertex labelled with $\PP_{n-1}$ and  $n-1$ vertices labelled with $\PP_{n-2}$. 
\end{thm}


It follows from this statement that $\Q_n$ is an Eilenberg-MacLane space and, therefore, its cohomology coincides with that of $\X_n$. (This fact can also be deduced from the existence of a CAT(0) universal cover for $\Q_n$ which we construct explicitly). The description of $\Q_n$ in terms of the groups $\PP_k$, in principle, gives a way to compute the cohomology of $\Q_n$, since the cohomology of $\PP_k$ is known completely. This, however, is not a straightforward task. 

Another possible approach to the cohomology of $\Q_n$ is via explicit cell decompositions. We will construct a cubical complex homotopy equivalent to $\Q_n$. However, computing its homology is a non-trivial combinatorial problem already for $n=6$. 


\medskip

The note has the following structure. In the next section we prove Theorems~\ref{fewstrands} and \ref{sixstrands}. We will not give any introduction to the moduli spaces of stable rational curves referring the reader instead to \cite{Devadoss, HK, Kapranov}. In Section~\ref{sectcact} we compare the round twin groups to the cactus groups, that is, the fundamental groups of the moduli spaces of curves. Namely, we show that the round twin group is a subgroup of the corresponding full cactus group.
In Section~\ref{graphsofgroups} we show that $\Pi_n$ is the fundamental group of a certain graph of planar braid groups and produce an exact sequence which relates its cohomology to that of the planar braid groups.
Finally, in Section~\ref{cubical} we show how to obtain a cubical decomposition for the space $\Q_n$.

\section{Proof of Theorem~\ref{fewstrands} and of Theorem~\ref{sixstrands}}

\subsection{$\X_1$, $\X_2$ and $\X_3$} The group  $\X_1$ is trivial and $\X_2=\Z$ since $\Q_1$ is a point and $\Q_2=S^1$. 
The space $\Q_3$ is the punctured torus $(z_1,z_2)\neq (\infty, \infty)$ and, therefore, $\X_3=F_2.$ 

\subsection{The group $\X_4$} As for $\Q_4$, it is the complement in the torus $(S^1)^3$ to the union of the sets $(\infty, \infty, t)$, $(\infty, t,\infty)$, $(t,\infty,\infty)$ and $(t,t,t)$ with $t\in S^1$. It is shown in Figure~\ref{qfour} as a fundamental region in its universal cover; one has to identify the opposite faces of the cube and remove the black lines. 
\begin{figure}[ht]
\includegraphics[width=1.8in]{oct.png}
\caption{The space $\Q_4$.}\label{qfour}
\end{figure}
$\Q_4$ can be retracted onto the 2-dimensional subcomplex of  $(S^1)^3$ 
which 
is an octahedron with two opposite faces removed and opposite vertices identified;  its fundamental group is $F_4$.

\subsection{$\X_5$ and the moduli space of real stable rational curves with 5 marked points} The group $PSL(2,\R)$ acts on $S^1$ and, hence, on the space $M_{n}{(S^1)}$, by real M\"obius transformations. The quotient space $M_{n}{(S^1)}/PSL(2,\R)$ is, in fact, a subspace of the moduli space $\overline{\mathcal{M}}_{0,n}(\R)$ of stable real rational curves with $n$ marked points. This moduli space is very well-studied; we refer to \cite{Devadoss, Kapranov} for a detailed description of its geometry and combinatorics. A point in $\overline{\mathcal{M}}_{0,n}(\R)$ is a tree of projective lines with $n$ marked points, whose components have no automorphisms, considered up to a M\"obius transformation on each component. A point in this moduli space lies in $M_{n}{(S^1)}/PSL(2,\R)$ if it corresponds to a curve whose graph of components is a star, that is, has all but one vertices univalent, and whose every component represented by a univalent vertex has exactly 2 marked points on it, see Figure~\ref{cact}.
\medskip

\begin{figure}[ht]
\includegraphics[width=1.8in]{cact.png}
\caption{A point in $\overline{\mathcal{M}}_{0,10}(\R)$ coming from $\Q_{10}$.}\label{cact}
\end{figure}


In general, $M_{n}{(S^1)}/PSL(2,\R)$ is the complement to a codimension 1 subset of $\overline{\mathcal{M}}_{0,n}(\R)$. However, when $n=5$, each point of $\overline{\mathcal{M}}_{0,5}(\R)$ comes from $M_{5}{(S^1)}$.
In fact,  the map
$$M_{5}{(S^1)}\to\overline{\mathcal{M}}_{0,5}(\R)$$ factors as
$$M_{5}{(S^1)}\to\M_{5}{(S^1)}/SL(2,\R)\to M_{5}{(S^1)}/PSL(2,\R)=\overline{\mathcal{M}}_{0,5}(\R),$$
where the first map, up to homotopy, is a trivial circle bundle and the second map is the orienting double cover. Since $\overline{\mathcal{M}}_{0,5}(\R)$ is a connected sum of five real projective planes, it follows that $M_{5}{(S^1)}/SL(2,\R)$ is a Riemann surface of genus 4. On the other hand, $M_{5}{(S^1)}/SL(2,\R)$ is homeomorphic to $\Q_5$.

\begin{rem}
The fact that $\X_5=\pi_1 X_4$ can also be established either by means of Corollary~\ref{graph} in Section~\ref{graphsofgroups} or by using an explicit cell decomposition of $\Q_5$.
\end{rem}

\subsection{$\X_6$ and $\overline{\mathcal{M}}_{0,6}(\R)$}
The complement to $M_{6}{(S^1)}/PSL(2,\R)$ in $\overline{\mathcal{M}}_{0,6}(\R)$ is the closure of the subset of all curves with two components, with three marked points on each component. The combinatorial types of all the curves in this closure are shown in Figure~\ref{badsix}.
\begin{figure}[ht]
\includegraphics[width=4in]{badsix.png}
\caption{Points in $\overline{\mathcal{M}}_{0,6}(\R)$ not coming from $M_{6}{(S^1)}$.}\label{badsix}
\end{figure}
In fact, $\overline{\mathcal{M}}_{0,6}(\R)$ is the blowup of $\R P^3$ along the configuration of points and lines shown in
Figure~\ref{blowupsix}; first, one blows up $\R P^3$ at the 5 points and then at the 10 lines connecting them. 
The complement of $M_{6}{(S^1)}/PSL(2,\R)$ is then precisely the exceptional divisor of the blowup along the 10 lines, since this exceptional divisor consists of the curves on Figure~\ref{badsix}; see \cite{Devadoss}. 

This means that $M_{6}{(S^1)}/PSL(2,\R)$ is a complement to a 10-component link in the blowup of $\R P^3$ at 5 points.
The orienting double cover of  the blowup of $\R P^3$ at 5 points consists of two 3-spheres with 6 punctures connected by six 1-tubes; this is readily seen to be the boundary of a 4-disk with five 1-handles. The 10 components of the exceptional divisor lift to 10 circles in this manifold. The space $M_{6}{(S^1)}/PSL(2,\R)$ can be identified with the complement to these circles and   $\Q_6$ is homeomorphic to it. This establishes the first part of Theorem~\ref{sixstrands}. 

In order to compute $H_1(\Q_6)$, we cut $\Q_6$ into 2 parts across the six 1-tubes; note that each of the 10 components of the link is divided into 6 parts by the cut: we have $$\Q_6 = A\cup B,$$ where $A$ and $B$ are two copies of a 3-sphere with 6 balls and 30 intervals connecting them removed. Namely, each of $A$, $B$ can be identified with the complement in $\R^3$ of the union of lines and balls shown in Figure~\ref{blowupsix}. It is straightfoward to see that $A$ and $B$ have the homotopy type of a bouquet (one-point union) of 25 circles, while $A\cap B$ is a disjoint union of 5 bouquets of 7 circles and a bouquet of 19 circles. The Mayer-Vietoris sequence then shows that $H_1(\Q_6)$ is spanned by the meridians and the 5 classes of 1-handles coming from joining the two 3-spheres. (It also follows that $H_2(\Q_6)$ is free abelian of rank 14).

\begin{figure}[h]
\includegraphics[width=2in]{blowupsix.png}
\caption{The configurations of points and lines in $\R P^3$ whose blowup is $\overline{\mathcal{M}}_{0,6}(\R)$.}\label{blowupsix}
\end{figure}

\section{Relationship with the cactus groups}\label{sectcact}

The group $$\Gamma_n=\pi_1 \overline{\mathcal{M}}_{0,n}(\R)$$ is known as the $n$th \emph{pure cactus group}.
The map $$\Q_n\simeq M_n(S^1)/SL(2,\R)\to\overline{\mathcal{M}}_{0,n}(\R)$$ gives rise to a homomorphism
$$\Pi_n\to \Gamma_n$$
for each $n$. While we do not have neat presentations for either of the two series of groups, this homomorphism can be described very explicitly in terms of the corresponding full groups. 

The \emph{full} cactus group $J_n$ (see \cite{HK}) has a presentation with the generators $s_{p,q}$, where $1\leq p< q\leq n$, and the following relations:
\begin{equation}\label{eqcac}
\begin{array}{rcll}
s_{p,q}^2&=& 1,&\\
s_{p,q}s_{m,r}&=& s_{m,r}s_{p,q} &\quad \text{if\ } [p,q]\cap [m,r] =\emptyset,\\
s_{p,q}s_{m,r}&=& s_{p+q-r, p+q-m}s_{p,q} &\quad \text{if\ }  [m,r] \subset [p,q].
\end{array}
\end{equation}
There is a homomorphism $J_n\to S_n$ to the symmetric group: it sends $s_{p,q}$ into the permutation $\tau_{p,q}$ of $\{1, \ldots, n\}$ which reverses the order of $p, p+1,\ldots, q$ and leaves the rest of the elements unchanged. The {pure cactus group} $\Gamma_{n+1}$ is the kernel of this homomorphism.

Consider the homomorphism
$$\kappa: \Y_n\to J_n$$
defined by
$$\begin{array}{lcl}
\kappa(\sigma_i) &= &s_{i,i+1},\\
\kappa(\zeta) &= &s_{1,n} s_{2,n}.
\end{array}
$$
with $\sigma_i$ and $\zeta$ as in (\ref{relroundtwin}).
This homomorphism is clearly well-defined and sends $\Pi_{n+1}\subset  \Y_n$ to $\Gamma_{n+1}$. We will denote by the same letter the map sending words in the generators of  $\Y_n$ to words in the generators of $J_n$.

In order to see that the restriction of $\kappa$ to  $\Pi_n$ is actually induced by the map $$\X_n\to \overline{\mathcal{M}}_{0,n}(\R),$$ one has to recall the geometric meaning of the generators of $J_n$:
the generator $s_{p,q}$ corresponds to a path in  $\overline{\mathcal{M}}_{0,n+1}(\R)$ in which the marked points $p,\ldots, q$ collide and bubble off onto a new component
and then return to the original component in the reversed order (see \cite{HK}). This shows that $\sigma_{i,i+1}$ map into
$s_{i,i+1}$ and $\zeta$ must go to $s_{1,n} s_{2,n}$.

\medskip

It is not hard to see that $\kappa$ is not always injective; for instance, $\Pi_4=F_4$ while $\Gamma_4$ is infinite cyclic.
\begin{prop}
The homomorphism $\kappa:\Y_n\to J_n$ is injective for $n\geq 4$.
\end{prop}

\begin{proof}
If $w$ is a word in the generators $s_{p,q}$ that defines the trivial element of $J_n$, there exists a sequence $w_1,\ldots, w_m$ such that $w_1=w$, $w_n$ is trivial, $w_{i+1}$ is obtained from $w_i$ by applying one of the relations (\ref{eqcac})
once and the length of $w_{i+1}$ is not greater than the length of $w_i$.

This follows from the proof of Proposition~2 in \cite{MCact} where it is shown that two words in the $s_{p,q}$ which represent the same element of $J_n$ and are \emph{locally reduced} (their lengths cannot be decreased by applying the relations (\ref{eqcac})) must have the same length. Indeed, if $w$ can only be taken into the trivial word by a sequence of moves that increases the length at some point, there exists a locally reduced word which represents the trivial element in $J_n$, which is impossible.

The image of a word in the generators $\sigma_i$ and $\zeta$ under $\kappa$ is a word $w$ in the $s_{i,i+1}$, $s_{1,n}$ and $s_{2,n}$.  If it represents the trivial element of $J_n$, it can be transformed into the trivial word by means of the relations that involve only the generators $s_{i,i+1}$, $s_{1,n}$, $s_{2,n}$ and $s_{1,n-1}$ only, since the appearance of any other generator of $J_n$ in the sequence of the words connecting $w$ and $1$ would imply that at some point the length of the word increases. 

Assume that $n\geq 4$; under this condition neither of $s_{1,n}$, $s_{1,n-1}$ or $s_{2,n}$ coincides with any of $s_{i,i+1}$. Let $z$ be the word $s_{1,n} s_{2,n}$. For any word in $u$ in the generators $s_{i,i+1}$, $s_{1,n}$, $s_{2,n}$ and $s_{1,n-1}$, define the word $\mu(u)$ in the $s_{1,2}$,  $\ldots,$ $s_{n-1,n}$, $z$ and $s_{1,n}$ inductively as follows. 

For a word $w$ in the generators $s_{p,q}$, et $\overline{w}$ be the word in which each $s_{p,q}$ is replaced by $s_{n-q+1,n-p+1}$. Now: 
\begin{itemize}
\item if $u=1$ we set $\mu(u)=1$;
\item if $u=s_{i,i+1} v$, we set $\mu(u)=s_{i,i+1} \mu(v)$;
\item if $u=s_{1,n} v$, we set $\mu(u)=\overline{\mu(v)} s_{1,n}$;
\item if $u=s_{2,n} v$, we set $\mu(u)=z^{-1} \overline{\mu(v)} s_{1,n}$;
\item if $u=s_{1,n-1} v$, we set $\mu(u)=z \overline{\mu(v)} s_{1,n}$. 
\end{itemize}

For any word $u$ in  $s_{i,i+1}$, $s_{1,n}$, $s_{2,n}$ and $s_{1,n-1}$, the word $\mu(u)$ is of the form $\mu'(u)  s_{1,n}^k$, where $\mu(u)$ is a word in $s_{i,i+1}$, and $z$ only.


Assume $v$ is a word in the generators  $\sigma_i$ and $\zeta$ such that $\kappa(v)$ defines a trivial element of $J_n$.
Take the sequence of words  $w_1,\ldots, w_n$ in the $s_{p,q}$, such that $w_1=\kappa(v)$, $w_m$ is trivial, $w_{i+1}$ is obtained from $w_i$ by applying one of the relations (\ref{eqcac}) once and the length of $w_{i+1}$ is not greater than the length of $w_i$. Then, the sequence of words  $\mu'(w_1),\ldots, \mu'(w_n)$ in $s_{1,2}$,  $\ldots,$ $s_{n-1,n}$ and $z$ transforms $\kappa(v)$ into the trivial word by means of the relations (\ref{eqcac}). Replacing each $s_{i,i+1}$ with $\sigma_{i}$ and $z$ with $\zeta$, we obtain a sequence of words that transforms $v$ into the trivial word by means of the relations (\ref{relroundtwin}) in $\Y_n$. In particular, this means that $v=1$ in $\Y_n$. 
\end{proof}

\begin{rem}
The fact that the twin groups inject into the cactus groups has been observed in \cite{BCL}.
\end{rem}

\section{Round twins via graphs of groups}\label{graphsofgroups}

Let $n>2$ and assume that the points of the configurations in  $M_{n-1}$ are labelled by the natural numbers from 1 to $n-1$.  
Denote by $M_{n-2,j}$ a copy of $M_{n-2}$ whose configurations are  labelled by natural numbers from 1 to $n-1$ with the label $j$ omitted. 

For a configuration $x\in M_ {n-2,j}$ define 
$\rho_j(x)\in  M_ {n-1}$ 
by adding a point $x_j$ with the label $j$ 
to the right of all the points of $x$, Similarly, 
$\lambda_j(x)$ is defined by adding $x_j$ to the left of $x$. 
These concatenation operations can be considered as maps
$$M_ {n-2,j}\to M_{n-1}.$$
Indeed, in both cases one can assume that all the points of each configuration in $M_ {n-2,j}$ lie in some fixed open interval and choose $x_j$ to be a fixed point outside of this interval. Note that, in general, these maps do not preserve the basepoints. 

In the union $$M_{n-1}\sqcup \bigsqcup_{1\leq j< n} M_{n-2,j} \times [-1,1],$$
identify, for each $x\in M_{n-2,j}$, the point $(x,-1)$ with 
$\lambda_j(x)\in M_{n-1}$ and the point $(x,1)$ with $\rho_j(x)\in M_{n-1}$. Denote the resulting space by $\Q_{n}'$.
\begin{thm}\label{space} The space $\Q_{n}'$ is homotopy equivalent to $\Q_{n}$.
\end{thm}

This result, which will be proved towards the end of this subsection, allows us to express $\X_n$ as the fundamental group of a graph of groups involving $\PP_{n-1}$ and $\PP_{n-2}$.

For each $j$ between 1 and $n-1$, choose a braid $g_{j+}\in\PB_{n-1}$ whose
permutation sends $12\ldots (n-1)$ to $12\ldots\hat{j}\ldots (n-1)j$, and a braid $g_{j-}$ which sends sends $12\ldots (n-1)$ to $j12\ldots\hat{j}\ldots (n-1)$. Let $i:\PP_{n-2}\to \PP_{n-1}$ be the inclusion map which adds one disjoint strand on the right. 
Define
$$(\rho_{j})_*: \PP_{n-2}\to \PP_{n-1}$$
as 
$$x\mapsto g_{j+} x g_{j+}^{-1} $$
and, similarly, let  
$$(\lambda_{j})_*: \PP_{n-2}\to \PP_{n-1}$$
be the map  
$$x\mapsto g_{j-} x g_{j-}^{-1}.$$
Now, define the graph of groups $\Phi_n$ in the following manner. 
The underlying directed graph of $\Phi_n$ has $n$ vertices: one ``central'' vertex and $n-1$ ``peripheral'' vertices, with two edges from each of the peripheral vertices to the central vertex. The central vertex is labelled by $\PP_{n-1}$ and the peripheral vertices  by  $\PP_{n-2}$. Enumerate the copies of $\PP_{n-2}$ from 1 to $n-1$; then, the edges emanating from the $j$th copy of $\PP_{n-2}$ are labelled by $(\rho_{j})_*$ and $(\lambda_{j})_*$. 
\begin{figure}[ht]
\includegraphics[width=2.5in]{graphofgroups.png}
\caption{The graph of groups $\Phi_n$ for $n=6$.}\label{phisix}
\end{figure}
\begin{cor}\label{graph} 
The classifying space $B\Phi_n$ for the graph of groups $\Phi_n$ is homotopy equivalent to $\Q_{n}$.
\end{cor}
According to Theorem~1B.11 of \cite{Hatcher}, this implies that the spaces $\Q_n$ have the homotopy type of the Eilenberg-Maclane spaces $K(\X_n,1)$.

\begin{proof}[Proof of Theorem~\ref{space}]
Let $\Q_{n,I}\subset \Q_n$ be the subspace consisting of the configurations which have either one or two points outside of a finite open interval $I\subset \R\subset S^1$; that is, zero or one points in $\R\backslash I$. We claim that the inclusion of $\Q_{n,I}$ into $\Q_n$ is a homotopy equivalence. Indeed, for any pair of finite open intervals $I\subseteq I'$, the inclusion  $\Q_{n,I}\to \Q_{n,I'}$ is a homotopy equivalence. Any compact subspace of $\Q_n$ lies in $\Q_{n,I}$ for some finite interval $I$, and, therefore, the inclusion 
$\Q_{n,I}\to\Q_n$ induces an isomorphism of homotopy groups. Since both $\Q_{n,I}$ and $\Q_n$ have homotopy type of cell complexes, they are homotopy equivalent.

The space $\Q_{n,I}$ is covered by $n$ open subspaces: $U_j$ with $1\leq j < n$ and $V$. The subspace $U_j$  consists of the configurations whose intersection with $S^1\backslash I$ consists of the point $z_j$ and $z_n=\infty$; the subspace $V$ consists of those configurations whose only point at $\infty$ is $z_n$.


$V$ is homotopy equivalent to $M_{n-1}$ while $U_j$ is homeomorphic to $M_{n-2}\times [-1,1]$. The sets $U_j$  are disjoint, while 
$U_j\cap V$ can be identified with two copies of $M_{n-2}$. The inclusion map
$$U_j\cap V \to U_j$$
is equivalent to the inclusion 
$$M_{n-2}\times \{-1\}\sqcup M_{n-2}\times \{1\} \to M_{n-2}\times [-1,1]$$
of the bases into the cylinder. The map
$$U_j\cap V \to V$$
is equivalent to the concatenation
$$x\mapsto \rho_j(x)$$
on one copy of $U_j$ and to
$$x\mapsto \lambda_j(x)$$
on the other copy. This establishes the Theorem.
\end{proof}

\begin{proof}[Proof of Corollary~\ref{graph}] The construction of the space $\Q_n'$ is very similar to that of the graph of groups $\Phi_n$. Indeed, $M_k$ has the homotopy type of $K(\PP_k)$. The mapping cylinders corresponding to the pair of edges emanating from each peripheral vertex are glued together into a cylinder of the form $M_{n-2}\times [-1,1]$.  

The braids $g_{j+}$ and $g_{j-}$ can be seen as paths  from the basepoint of $V$ to the basepoints of the connected components of $U_j\cap V$; denote these path by $\gamma_{j+}$ or $\gamma_{j-}$ respectively. Then, the maps $(\rho_j(x))_*$ and $(\lambda_j(x))_*$ are of the same form: they send the homotopy class of a path $\alpha$ to that of $\gamma_{j\pm}^{-1}\alpha\gamma_{j\pm}$. This shows that the fundamental group of $\Q_n'$ is precisely the fundamental group of $B\Phi_n$.
\end{proof}

\subsection{The long exact sequence for the cohomology of $\Q_n$}

The inclusion map $M_{n-1} \to \Q_n$ can be deformed so as to obtain a cofibration
$$M_{n-1} \to \Q_n\to \Sigma \left(* \sqcup \bigsqcup_{1\leq j< n} M_{n-2,j}\right),$$
where * is a one-point space and $\Sigma$ denotes the suspension.
This cofibration gives rise to a long exact sequence in cohomology
{\small
$${{\ldots \leftarrow  \widetilde{H}^{k+1}\left(\Sigma \left(* \sqcup \bigsqcup_{1\leq j< n} M_{n-2,j}\right),\Z\right)\leftarrow H^k(M_{n-1},\Z) \leftarrow H^k(\Q_n,\Z)\leftarrow   \widetilde{H}^k\left(\Sigma \left(* \sqcup \bigsqcup_{1\leq j< n} M_{n-2,j}\right),\Z\right)\leftarrow \ldots,}}$$}
which, in view of the suspension isomorphism, translates into
$$ \ldots\leftarrow  \bigoplus_{1\leq j< n} {H}^k(M_{n-2,j},\Z)\xleftarrow{d} H^k(M_{n-1},\Z) \leftarrow H^k(\Q_n,\Z_2)\leftarrow  \bigoplus_{1\leq j< n} {H}^{k-1}(M_{n-2,j},\Z)\leftarrow \ldots$$
In principle, this exact sequence could be used to compute the cohomology of $\Q_n$, since the cohomology of the spaces $M_{k}$ (or, which is the same, the cohomology of the groups $\PP_k$) is known \cite{B, DT}. However, this involves non-trivial combinatorics and is not straightforward even for $n=6$.

\section{The structure of a cubical complex on $\Q_n$}\label{cubical}\label{cubical}

\subsection{The  decomposition of $\Q_n$ into open cells and the dual cubical complex}
Call two configurations in $\Q_n$ equivalent is they can be taken into each other by an orientation-preserving homeomorphism $\R\to\R$ (that is, by an orientation-preserving homeomorphism $S^1\to S^1$ that fixes $\infty\in S^1$). Each equivalence class of configurations can be identified with an open simplex of dimension is $n-1-k$, where $k$ is the number of pairs of coinciding points. The decomposition of  $\Q_n$ into such open simplices fails to be a CW-complex. However, the dual cell decomposition, which has one $k$-cell for each $n-1-k$-simplex, is a CW-complex of the same homotopy type as $\Q_n$. 

A $k$-cell $u$ corresponds to a equivalence class of configurations with $k$ pairs of coinciding points; its boundary consists of all the $k-1$ cells $v$ such that $u$ can be obtained from $v$ by joining together a pair of adjacent points of multiplicity one (one of which can be $z_n=\infty$). It follows that we get, in fact, a cubical complex. The link of each vertex in this complex is a flag and, therefore, its universal cover is a cubical CAT(0) complex. This gives another proof of the fact that $\Q_n$ is an Eilenberg-Maclane space. 

\subsection{Examples: $n\leq 5$}
For low values of $n$ the computations in these cubical complexes are easy. Let us denote a cell by listing the points of a configuration in their natural order, with $z_n$ being the last point. The pairs of coinciding points will be indicated by parentheses and listed in the order of increasing indices. For instance,
$$2(13)5(46)$$
is the notation for the 2-cell dual to the simplex
$$z_2< z_1=z_3<z_5<z_4=z_6=\infty$$
in $\Q_6$.

\medskip

For $n=2$, the cubical complex for $Q_n$ has one vertex $12$ and one edge $(12)$. We see that $Q_2$ is a circle.

\medskip

For $n=3$, there are two vertices $123$ and $213$ and 3 edges: 
$$(12)3, 1(23)\ \text{and}\ 2(13).$$ This is the $\Theta$-graph, which is homotopy equivalent to a bouquet of two circles.

\medskip

When $n=4$, there are 6 vertices:
$$1234, 2314, 3124, 2134, 1324, 3214,$$ 
12 edges
$$(12)34, 1(23)4,12(34), (23)14, 23(14), 3(12)4, 31(24), 2(13)4, 21(34), (13)24, 13(24), 32(14),$$ 
and 3 faces
$$(12)(34), (23)(14), (13)(24).$$ 
At every vertex, precisely two 2-faces meet. Each of the edges lies in the boundary of precisely one face, so that each face can be collapsed onto a ``cross''
without changing the homotopy type of the complex:
\smallskip

$$\includegraphics[width=200pt]{sqsq.png}$$

\smallskip

\noindent After erasing the 6 bivalent vertices, we obtain a connected graph with 3 vertices and 6 edges; it is homotopy equivalent to a bouquet of 4 circles.


\medskip

When $n=5$, there are 24 vertices, 60 edges and 30 faces. The cubical complex is, in fact, a surface: two faces meet at every edge  and at  every vertex 5 faces meet in cyclic order. As we know, this surface is of genus 4; it is the double cover of  $\overline{\mathcal{M}}_{0,5}(\R)$ for which a cubical decomposition is known from \cite{DJS}.

\medskip

For $n=6$, the cubical complex has $120$ vertices, $360$ edges, 
$270$ 2-faces and $30$ 3-faces. 
\begin{itemize}
\item
Every vertex lies in the boundary of precisely two 3-faces. For instance, $123456$ lies in the boundary of $(12)(34)(56)$ and of $(23)(45)(16)$. 
\item
Each edge lies in the boundary of exactly three 2-faces. For example, $(12)3456$ lies in the boundary of  $(12)(34)56$, $(12)3(45)6$ and $(12)34(56)$.  
\item
Out of the 270 2-faces, 180 lie in the boundary of one 3-face (the face $(12)(34)56$ lies in the boundary of $(12)(34)(56)$) and 90 2-faces (such as  $(12)3(45)6$, for example)  in the boundary of no 3-face. 
\end{itemize}
 Therefore, without changing the homotopy type of the complex, each 3-face can be collapsed on its subspace homeomorphic to the cone on the 1-skeleton of a cube: 
\smallskip

$$\includegraphics[width=250pt]{sqsqsq1.png}$$

\smallskip

The resulting 2-dimensional complex has 30 vertices, 120 edges and 90 square 2-faces. Although the boundary of every given face is easy to describe, for the homology computation it is easier to use the connection with the moduli spaces as in the proof of Theorem~\ref{sixstrands}.







\begin{thebibliography}{999}


\bibitem{B}  Y.~Baryshnikov. On the cohomology ring of no k-equal manifolds. Preprint, 1997.\\  Available at {{http://publish.illinois.edu/ymb/files/2018/05/no3.pdf}}.


\bibitem{BCL} P.~Bellingeri, H.~Chemin, V.~Lebed, Cactus groups, twin groups, and right-angled Artin groups, arXiv:2209.08813 [math.CO].


\bibitem{BW} A.~Bj\"orner, V.~Welker, The homology of ``$k$-equal" manifolds and related partition lattices. Advances in Mathematics {\bf 110} (1995) No.2, 277--313. 


\bibitem{CG} F.R.~Cohen, S.~Gitler, Loop spaces of configuration spaces, braid-like groups, and knots, in {\em Cohomological Methods
in Homotopy Theory} in Progress in Mathematics, Vol. 196, Birkh\"auser Verlag, Basel,  2001.



\bibitem{DJS} M.~Davis, T.~Januszkiewicz and R.~Scott,
Nonpositive curvature of blow-ups,
Selecta Mathematica, New Series {\bf 4} (1998) 491--547.

\bibitem{Devadoss} S.~Devadoss, Tessellations of Moduli Spaces and the Mosaic Operad, Contemporary Mathematics, {\bf 239} (1999), 91--114.

\bibitem{DT} N.~Dobrinskaya and V.~Turchin, Homology of non-k-overlapping discs, Homology, Homotopy, and Applications {\bf 17} (2015), no. 2,  261-290.


\bibitem{Fa} D.~Farley,
The planar pure braid group is a diagram group,
	arXiv:2109.02815 [math.GR].


\bibitem{GS}
V.S.~Guba, M.V.~Sapir, Diagram Groups, Memoirs of the American Mathematical Society {\bf130(620)}, 1997. 


\bibitem{Hatcher} A.~Hatcher, Algebraic topology, Cambridge University Press, Cambridge, 2002.


\bibitem{Threebody} N.L.~Harshman, A.C.~Knapp, Anyons from three-body hard-core interactions
in one dimension, Annals of Physics {\bf 412} (2020) 168003.


\bibitem{HK} A.~Henriques, J.~Kamnitzer, Crystals and coboundary categories, 
Duke Mathematical Journal {\bf 132}  (2006) 191--216.


\bibitem{Kapranov} M.~Kapranov, The permutoassociahedron, Mac Lane’s coherence theorem and asymptotic zones for the KZ equation,   Journal of Pure and Applied Algebra, {\bf 85}
(1993) 119--142.

\bibitem{KP} R.~Kent, D.~Peifer, A geometric and algebraic description of annular braid groups, International Journal of Algebra and Computation {\bf 12} (2002) 85--97.

\bibitem{Kh1} 
M.~Khovanov,
Real $K(\pi,1)$ arrangements from finite root systems. 
Math. Res. Lett. 3, No.2, 261-274 (1996). 

\bibitem{Kh2} 
M.~Khovanov, Doodle groups. Transactions of the American Mathematical Society {\bf 349} (1997) No.6, 2297--2315 . 

\bibitem{Merk2}
A.B.~Merkov, Vassiliev invariants classify flat braids. Tabachnikov, S. (ed.), Differential and symplectic topology of knots and curves. Providence, RI: American Mathematical Society. Transl., Ser. 2, Am. Math. Soc. 190(42), 83--102 (1999). 


\bibitem{MCact} J.~Mostovoy, The pure cactus group is residually nilpotent, 
Archiv der Mathematik {\bf 113} (2019) 229--235.

\bibitem{MRo} J.~Mostovoy and C.~Roque-M\'arquez, Planar pure braids on six strands,
 Journal of Knot Theory and Its Ramifications, {\bf 29} (2020) no.\ 01, 1950097. 

\bibitem{MPres} J.~Mostovoy, A presentation for the planar pure braid group, arXiv:2006.08007 [math.GR].

\bibitem{MRP} J.~Mostovoy, A.~Rincon-Prat, Cactus doodles,  	arXiv:2203.08742 [math.GT].


\bibitem{VV}
V.A.~Voevodsky, Flags and Grothendieck Cartographical Group in Higher Dimensions, CSTARCI Math. Preprints, 1990,
{\tt http://www.math.ias.edu/vladimir/files/Voevodsky\_Flags\_and\_cartographic\_group.pdf}

\end{thebibliography}

\end{document}