\documentclass[peerreview]{IEEEtran}

\pdfminorversion=7

%
\usepackage[square,numbers,sort&compress]{natbib}
\bibliographystyle{IEEEtran}
%


\usepackage{amsthm}
\usepackage{amsmath}
\usepackage{mathrsfs}
\usepackage{amssymb} 
\usepackage{bbm,dsfont}   
\usepackage{enumerate} 
%
\usepackage[table]{xcolor}
\usepackage{tikz}
\usetikzlibrary{calc}

\usepackage{physics}

\usepackage{xcolor}

%
\usepackage{hyperref} %
\usepackage%
{hyperref} %
\hypersetup{
    colorlinks,
    linkcolor={blue!80!black},%
    citecolor={green!30!black},
    urlcolor={blue!80!black}
}


\DeclareMathOperator*{\veccat}{%
    \mathchoice%
        {\Bigg\Vert}%
        {\Big\Vert}%
        {\Vert}%
        {\Vert}%
}%






\title{The Multiple-Access Channel with Entangled Transmitters}



\author{
		%\vspace{0.1cm}
    \IEEEauthorblockN{Uzi Pereg\IEEEauthorrefmark{1}\IEEEauthorrefmark{2}, Christian Deppe\IEEEauthorrefmark{3}, and Holger Boche\IEEEauthorrefmark{4}\IEEEauthorrefmark{5}\IEEEauthorrefmark{7}} \\
		\vspace{0.25cm}
    \IEEEauthorblockA{\normalsize
		\IEEEauthorrefmark{1}Faculty of Electrical and Computer Engineering, Technion \\
		\IEEEauthorrefmark{2}Helen Diller Quantum Center, Technion \\
		\IEEEauthorrefmark{3}Institute of Communication Engineering, Technical University of Munich \\
		\IEEEauthorrefmark{4}Theoretical Information Technology, Technical University of Munich\\
		\IEEEauthorrefmark{5}Munich Center for Quantum Science and Technology (MCQST)\\
		\IEEEauthorrefmark{7}Cyber Security in the Age of Large-Scale Adversaries Exzellenzcluster (CASA)\\
    Email: {\tt uzipereg@technion.ac.il, $\{$christian.deppe,boche$\}$@tum.de}}
}




\begin{document}
\maketitle

{}


\begin{abstract}
Communication over a classical multiple-access channel (MAC) with  entanglement resources is considered,
%
whereby two transmitters share entanglement resources a priori before communication begins.
Leditzki et al. (2020)  presented an example of a classical MAC, defined in terms of a pseudo telepathy game, such that the sum rate with entangled transmitters is strictly higher than the best achievable sum rate without such resources. 
%
Here, we derive a full characterization of the capacity region for the \emph{general} MAC with entangled transmitters, and show that the previous result can be obtained as a special case. A single-letter formula is established  involving auxiliary variables and ancillas of finite dimensions.
This, in turn, leads to a sufficient entanglement rate to achieve the rate region.
\end{abstract}

\begin{IEEEkeywords}
Quantum communication, multiple-access channel, entanglement resources.
\end{IEEEkeywords}

%
%


\maketitle



\section{Introduction}
% The multiple-access channel (MAC) is a fundamental model for network communication. %and well-understood models in network communication and information theory \cite{Ahlswede:74p,LiXu:20p}.  
% %
% %
% The MAC is also referred to as the uplink channel \cite{VaeziDingPoor:19b}, since it is interpreted 
% in cellular communication as the link from the mobiles to the base station \cite{ShirvanimoghaddamDohlerJohnson:17m}, and
% in the satellite-based Internet of Things (IoT),  from  ground devices to a satellite in space \cite{GHGCA:18p}.
%
 % Furthermore,  in a wireless local area network (WLAN), the MAC represents  the channel from the terminals to the access point \cite[Section 3.2]{TerplanMorreale:18b}.
%
%

Cooperation in quantum communication networks has  gained considerable attention recently driven by both experimental advancements and theoretical discoveries
%
%Cooperation in quantum communication networks has been extensively studied in recent years, following both experimental progress and theoretical discoveries 
\cite{vanLoockAltBecherBensonBocheDeppe:20p,Pereg:22p,BFSDBFJ:20b}. 
The multiple-access channel (MAC) is a fundamental model for network communication.
Quantum resources in communication over the MAC are considered in the literature in various settings. %
Winter \cite{Winter:01p}  derived a regularized characterization for the classical capacity region of the quantum MAC (see also \cite{Savov:12z}). Furthermore, 
the authors of the present paper \cite{PeregDeppeBoche:22p} considered the quantum MAC with cribbing encoders, whereby Transmitter 2 has access to (part of) the environment of Transmitter 1.
 Boche and N\"otzel \cite{BocheNoetzel:14p} studied the cooperation setting of a classical-quantum %
MAC with conferencing encoders, where the encoders exchange messages between them in a constant rate (see also \cite{DiadamoBoche:19a}).
 Hsieh \etal \cite{HsiehDevetakWinter:08p} and Shi \etal \cite{ShiHsiehGuhaZhangZhuang:21p} addressed the model where each transmitter shares entanglement resources with the receiver independently.
 


Leditzky \etal \cite{LeditzkyAlhejjiLevinSmith:20p} presented an example of a classical MAC
such that  sharing entanglement between transmitters strictly increases the achievable sum rate.
Recently, the classical upper bound has been improved showing that the sum rate increases from at most
$3.02$ to $3.17$ bits per transmission \cite{SeshadriLeditzkySiddhuSmith:22c}. 
The channel construction in \cite{LeditzkyAlhejjiLevinSmith:20p} is based on  a pseudo-telepathy game \cite{BrassardBroadbentTapp:05p} where quantum strategies guarantee a certain win and outperform classical strategies.
Additional observations are developed in 
\cite{DoolittleChitambarLeditzky:22c} as well.
%
 Fawzi and Ferm\'e \cite{FawziFerme:22c} have also  established  separation in a more fundamental example, the binary adder channel.
By analyzing the zero-error capacity region, it was shown that the sum rate increases from $1.5$ without entanglement resources, to $1.5425$ bits per transmission in this case \cite{FawziFerme:22c}. 
We   have shown that the dual property does not hold for the broadcast channel, \ie entanglement between receivers does not increase achievable rates \cite{PeregDeppeBoche:21p2}. %
%Related settings of the quantum MAC involve transmission of quantum information \cite{Yard:05z,YardHaydenDevetak:08p}, error exponents \cite{HayashiCai:17a}, non-additivity effects  \cite{CzekajHorodecki:08a}, security \cite{AghaeeAkhbari:19c,BochJanssenSaeedianaeeni:20p,DasHorodeckiPisarczyk:21a,ChakrabortyNemaSen:21a},  and computation codes \cite{HayashiVazquezCastro:21a}.
 
The potential benefits of the sixth generation of cellular networks (6G) are significant, with anticipated improvements in latency, resilience, computation power, and trustworthiness for future communication systems, such as the tactile internet
\cite{FettwisBoche:21m},
which involves not only data transfer but also the control of physical and virtual objects. Quantum resources are expected to play a crucial role in achieving these gains, as highlighted by \cite{DangAminShihadaAlouini:20p} and \cite{FettwisBoche:21m}. By enabling cooperation between trusted hardware and software components, future communication systems could isolate untrusted components and substantially reduce the attack surface of the communication system  \cite{FettwisBoche:21m,FettwisBoche:22p}. Quantum resources and cooperation also offer additional advantages, such as improved performance gains for communication tasks and the reduction of the attack surface, making them highly promising for 6G networks, as discussed in \cite{TKWIBD:20p} and \cite{FitzekBoche:21p2}. Investigating communication with cooperation in the form of entanglement resources could lead to more efficient protocols for future applications, making this an interesting avenue for further research.



% The sixth generation of cellular networks (6G) is expected to  achieve gains in terms of latency, resilience, computation power, and trustworthiness in future communication systems, such as the tactile internet \cite{FettwisBoche:21m}, which not only transfer data but also control physical and virtual objects, by using quantum resources \cite{DangAminShihadaAlouini:20p}.
% %
% %Cooperation between trusted hardware and software components in future communication systems has the potential to isolate untrusted components such that the attack surface of the communication system is substantially reduced \cite{FettwisBoche:21m,FettwisBoche:22p}. 
% Quantum resources and cooperation
% offer additional advantages, in terms of performance gains for communication tasks and reduction of attack surface, and  
% %
% %
% are of great potential for 6G networks  \cite{TKWIBD:20p,FitzekBoche:21p2}.
% %
% It is therefore interesting to investigate communication over classical channels with cooperation in the form of entanglement resources in order to realize more efficient protocols for future applications. 


%
\begin{figure*}[tb]
\includegraphics[scale=0.75,trim={-2cm 9cm 2cm 6.5cm},clip]{Entangled_MAC_Diag.pdf} %[trim={left bottom right top},clip]

\caption{The classical multiple-access channel $P_{Y|X_1,X_2}$ with pre-shared entanglement resources between the transmitters. The entanglement resources (quantum systems) of Transmitter 1 and Transmitter 2 are marked in red and blue, respectively.
%
}
\label{fig:MentangledTx}
\end{figure*}

Here, %In this paper, 
we consider
%
communication over a two-user classical MAC with  entanglement resources shared
%
between the transmitters a priori before communication begins.
We derive a full characterization of the capacity region for the \emph{general} MAC with entangled transmitters (see Figure~\ref{fig:MentangledTx}), and show that the  result by Leditzky \etal \cite{LeditzkyAlhejjiLevinSmith:20p} can be obtained as a special case. A single-letter formula is established  involving auxiliary variables and ancillas of finite dimensions.
This, in turn, leads to a sufficient entanglement rate to achieve the capacity region.


\section{Preliminaries%Definitions and Related Work
}
\label{sec:def}
%
\subsection{Basic Definitions%Notation, States, and Information Measures
}
\label{subsec:notation}


 We use the following notation conventions. %
%\subsubsection{Discrete Random Variables and Typical Sets}
Script letters $\Xset,\Yset,...$ are used for finite sets.
%
Lowercase letters $x,y,\ldots$  represent constants and values of classical random variables, and uppercase letters $X,Y,\ldots$ represent classical random variables.  
 The distribution of a  random variable $X$ is specified by a probability mass function (pmf) 
	$p_X(x)$ over a finite set $\Xset$.
	The respective product distribution is denoted by $p_X^n$.	%
%
 We use $x^r=\left(x[i] \right)_{i=1}^r$ to denote  a sequence of letters from $\Xset$, where $r$ is a positive integer. %
 A random sequence $X^n$ and its distribution $p_{X^n}(x^n)$ are defined accordingly. 
%
The %weak 
typical set $\tset(p_X)$ is the set of sequences $x^n\in\Xset^n$ such that $\left| p_X(a) -\frac{1}{n}N(a|x^n)%\log p_X^n(x^n)-
%H(X) 
\right|\leq \delta \cdot p_X(a)$, 
for every $a\in\Xset$, where $N(a|x^n)$ is the number of instances of the letter $a\in\Xset$ in the sequence $x^n$.
The definition is extended to a joint distribution $p_{X,Y}$ in a natural manner.
%where $H(X)$ is the entropy of $X\sim p_X$. Furthermore, for a joint distribution, we define $\tset(p_{X,Y})$ as
%  the set of pairs $(x^n,y^n)\in \tset(p_X)\times \tset(p_Y) $ such that 
% $\left| -p_{X,Y}^n(x^n,y^n)\log p_{X,Y}^n(x^n,y^n)-H(X,Y) \right|\leq \delta$.


% \subsubsection{Discrete-Variable Quantum States}
The state of a quantum system $A$ is a density operator $\rho$ on the Hilbert space $\Hset_A$.
A density operator is an Hermitian, positive semidefinite operator, with unit trace, \ie 
 $\rho^\dagger=\rho$, $\rho\succeq 0$, and $\trace(\rho)=1$. The set of all density operators acting on $\Hset_A$ is denoted by $\mathscr{D}(\Hset_A)$. The state is said to be pure if $\rho=\kb{\psi}$, for some vector $|\psi\rangle\in\Hset_A$.
%
A measurement of a quantum system is specified a positive operator-valued measure (POVM), \ie a set of positive semi-denfinite operators  $\{ K_x \}_{x\in\Xset}$
such that $\sum_x K_x=\identity$.
%
According to the Born rule, if the system is in state $\rho$, then the probability to measure $x$ is  $p_X(x)=\trace(K_x \rho)$.
%
%
% Define the quantum entropy of the density operator $\rho$ as
% $%
% H(\rho) \triangleq -\trace[ \rho\log(\rho) ]
% $. 
% %
% Consider the state of a pair of systems $A$ and $B$ on the product space
% $\Hset_A\otimes \Hset_B$.
% Given a bipartite state $\sigma_{AB}$, %
% define the quantum mutual information as
% \begin{align}
% I(A;B)_\sigma=H(\sigma_A)+H(\sigma_B)-H(\sigma_{AB}) \,. %
% \end{align} 
% Furthermore, conditional quantum entropy and mutual information are defined by
% $H(A|B)_{\sigma}=H(\sigma_{AB})-H(\sigma_B)$ and
% $I(A;B|C)_{\sigma}=H(A|C)_\sigma+H(B|C)_\sigma-H(AB|C)_\sigma$, respectively.
A quantum channel $\Pset_{A\to B}$ is a completely-positive trace-preserving (cptp) linear map from $\mathscr{D}(\Hset_A)$ to $\mathscr{D}(\Hset_B)$.


% % \subsubsection{Continuous Random Variables and Typical Sets}
% % The distribution of a real random variable $Z\in\mathbb{R}$ is represented by a cumulative distribution function (cdf) 
% % $F_Z(z)=\prob{Z\leq z}$ over the real line, or alternatively, the probability density function (pdf) $f_Z(z)$,  when it exists.
% %
% %In the continuous case, we use the cumulative distribution function (cdf)
% 	%$F_Z(z)=\prob{Z\leq z}$ for $z\in\mathbb{R}$, or alternatively, the probability density function (pdf) $f_Z(z)$,  when it exists. 
% %	
%  %We use $x^j=(x_1,x_{2},\ldots,x_j)$ to denote  a sequence of numbers. %, with $j\geq 1$. 
%  %A random sequence $X^n$ and its distribution $F_{X^n}(x^n)=\prob{X_1\leq x_1,\ldots,X_n\leq x_n}$ are defined accordingly. 
% %For a pair of integers $i$ and $j$, $1\leq i\leq j$, we define the discrete interval $[i:j]=\{i,i+1,\ldots,j \}$. 
% The notation $\zvec=(z_1,z_{2},\ldots,z_n)$ is used %instead of $x^n$  
% when it is understood from the context that the length of the sequence is $n$, and  the $\ell^2$-norm of $\zvec$ is denoted  by $\norm{\zvec}$. In a similar manner as before, we define $\tset(f_X)$ as the set of $\zvec\in \mathbb{R}^n$ such that 
% $\left| -f_Z^n(\zvec)\log f_Z^n(\zvec)-h(Z) \right|\leq \delta$, where $h(Z)$ is the differential entropy of $Z\sim f_Z$. Furthermore, $\tset(f_{Z_1,Z_2})$ is
%  the set of pairs $(\zvec_1,\zvec_2)\in \tset(f_{Z_1})\times \tset(f_{Z_2}) $ such that 
% $\left| -f_{Z_1,Z_2}^n(\zvec_1,\zvec_2)\log f_{Z_1,Z_2}^n(\zvec_1,\zvec_2)-H(Z_1,Z_2) \right|\leq \delta$.

% \subsubsection{Continouts-Variable Quantum States}


% \subsection{Classical Multiple-Access Channel}
% \label{subsec:MAC}
A discrete memoryless multiple-access channel (MAC) $(\Xset_1,\Xset_2,P_{Y|X_1,X_2},\mathcal{Y})$ consists of finite input alphabets $\Xset_1$, $\Xset_2$, a finite output alphabet $\mathcal{Y}$, and a collection of conditional probability mass functions 
$P_{Y|X_1,X_2}(\cdot|x_1,x_2)$ on $\Yset$, for every $(x_1,x_2)\in\Xset_1 \times \Xset_2$.
We assume that the channel is memoryless. That is, if the inputs are $x_1^n=(x_{1}[i])_{i=1}^n$ and $x_2^n=(x_{2}[i])_{i=1}^n$ are sent through $n$ channel uses, then the output is distributed according to 
$P_{Y|X_1,X_2}^n(y^n|x_1^n,x_2^n)=\prod_{i=1}^n P_{Y|X_1,X_2}\big(y[i] \,\big| x_1[i],x_2[i] \big)$.
%
%
 The transmitters and the receiver are often called Alice 1, Alice 2, and Bob. 




\subsection{Coding with Entangled Transmitters}
\label{sec:Coding}
We consider coding for the classical MAC $P_{Y|X_1,X_2}$ with entanglement resources between the transmitters. 
%
%
%
We denote the entanglement resources of Transmitter 1 and Transmitter 2 by $E_1$ and $E_2$, respectively. Let 
$\Hset_{E_1 E_2}\equiv \Hset_{E_1}\otimes \Hset_{E_2}$ denote the corresponding Hilbert space.

\begin{definition} %
\label{def:ClcapacityE}
A $(2^{nR_1},2^{nR_2},%
n)$   code for the classical MAC $P_{Y|X_1,X_2}$ with
entangled transmitters  consists of the following: 
\begin{itemize}
\item
an entangled state $\Psi_{E_1 E_2}\in\mathscr{D}(\Hset_{E_1 E_2})$ that is shared between the transmitters.
\item 
two message sets  $[1:2^{nR_1}]$ and $ [1:2^{nR_2}]$, assuming $2^{nR_k}$ is an integer;
\item
a pair of encoding POVMs $\Fset_1^{(m_1)}=\{ F^{(m_1)}_{x_1^n} \}_{x_1^n\in\mathcal{X}_1^n}$ and 
$\Fset_2^{(m_2)}=\{ F^{(m_2)}_{x_2^n} \}_{x_2^n\in\Xset_2^n}$ on $E_1$ and $E_2$, respectively; and
\item
a decoding function   $g:\Yset^n\to [1:2^{nR_1}]\times [1:2^{nR_2}] $.
	%
%
 %
\end{itemize}
We denote the code by $\mathscr{C}=(\Psi,\Fset_1,\Fset_2,g)$.
\end{definition}


The communication scheme is depicted in Figure~\ref{fig:MentangledTx}.  
The senders share the entangled pair  $E_1 E_2$. 
Alice $k$ chooses a message $m_k$ according to a uniform distribution over $[1:2^{nR_k}]$, for $k=1,2$.
 To send the message $m_1\in [1:2^{nR_1}]$,
Alice 1 applies the encoding measurement $\Fset_1^{(m_1)}$ to her share of the entanglement resource, $E_1$, and obtains a measurement outcome $x_1^n\in\mathcal{X}_1^n$.
She sends $x_1^n$ through $n$ uses of the classical MAC $P_{Y|X_1,X_2}$. %
Similarly, Alice 2 measures the system $E_2$ using the measurement $\Fset_2^{(m_2)}$, and transmits the outcome $x_2^n$. 
%
%
%
%
%
The joint input distribution is thus
\begin{align}%
f(x_1^n,x_2^n|m_1,m_2)
= \trace \left[  \left(F^{(m_1)}_{x_1^n}\otimes F^{(m_2)}_{x_2^n} \right) \Psi_{E_1 E_2} \right] \,.
\end{align}%



Bob receives the channel output $y^n$, and estimates the message pair as $(\htm_1,\htm_2)=g(y^n)$.
%
%
%
The conditional probability of error of the code $\mathscr{C}=(\Psi,\Fset_1,\Fset_2,g)$ is
\begin{align}
&P_{e}^{(n)}(\mathscr{C}|m_1,m_2)= %
\sum_{y^n: g(y^n)\neq (m_1,m_2)} \left[  \sum_{(x_1^n,x_2^n)\in \Xset_1^n\times \Xset_2^n} f(x_1^n,x_2^n|m_1,m_2) P_{Y|X_1,X_2}^n(y^n|x_1^n,x_2^n) \right] \,.
\end{align}
Hence, the average probability of error is 
\begin{align}
&P_{e}^{(n)}(\mathscr{C})= %
\frac{1}{2^{n(R_1+R_2)}}\sum_{m_1=1}^{2^{nR_1}}\sum_{m_2=1}^{2^{nR_2}}    P_{e}^{(n)}(\mathscr{C}|m_1,m_2) \,.
\end{align}

A $(2^{nR_1},2^{nR_2},n,\eps)$  code satisfies 
$%
P_{e}^{(n)}(\mathscr{C})\leq\eps $. %
%
%
%
%
A rate pair $(R_1,R_2)$ is called achievable with  entangled transmitters   if for every $\eps>0$ and sufficiently large $n$, there exists a 
$(2^{nR_1},2^{nR_2},n,\eps)$ code. 
%
%
 The capacity region $\opC_{\text{ET}}(P_{Y|X_1,X_2})$ of the classical MAC with entangled transmitters
is defined as the set of achievable pairs $(R_1,R_2)$, where the subscript `ET' indicates entanglement resources between the transmitters. %




\section{Main Results}
\label{sec:main}

\subsection{Communication With Unlimited Entanglement Resources}
\label{Subsection:Main_Result}
We state our results on the classical MAC $P_{Y|X_1,X_2}$ with entanglement resources between the transmitters.
%
 Define the rate region $\mathcal{R}_{\text{ET}}(P_{Y|X_1,X_2})$ as follows,
\begin{align}%
\mathcal{R}_{\text{ET}}(P_{Y|X_1,X_2})=
\bigcup_{ p_{V_0} p_{V_1|V_0} p_{V_2|V_0} \,,\; \varphi_{A_1 A_2} \,,\; \Lset_1 \otimes \Lset_2 }
\left\{ \begin{array}{rl}
  (R_1,R_2) \,:\;
	R_1 &\leq I(V_1;Y|V_0 V_2)  \\
  R_2 &\leq I(V_2;Y|V_0 V_1) \\
	R_1+R_2 &\leq I(V_1 V_2;Y|V_0)
	\end{array}
\right\} \,.
\label{eq:inRetx}
\end{align}%
The union on the right-hand side of (\ref{eq:inRetx}) is over the set of all
 entangled states $\varphi_{A_1 A_2}
 \in
\mathscr{D}( \Hset_{A_1}\otimes \Hset_{ A_2})$, classical auxiliary variables $(V_0,V_1,V_2)\sim p_{V_0} p_{V_1|V_0} p_{V_2|V_0}$, and collection of POVMs $\Lset_k(v_0,v_k)=\{L_k(x_k|v_0,v_k)\}_{x_k\in\Xset_k}$, for $v_0\in\Vset_0$, $v_k\in\Vset_k$, $k=1,2$. Given such a state, variables, and POVMs, the joint distribution of $(V_0,V_1,V_2,X_1,X_2,Y)$ is 
\begin{multline}
p_{V_0,V_1,V_2,X_1, X_2, Y}(v_0,v_1,v_2,x_1,x_2,y)=%
 p_{V_0}(v_0) p_{V_1|V_0}(v_1|v_0)p_{V_2|V_0}(v_2|v_0)\\ \cdot\trace\left[ \left( L_1(x_1|v_0,v_1)\otimes L_2(x_2|v_0,v_2) \right)\varphi_{A_1 A_2}  \right] \cdot P_{Y|X_1,X_2}(y|x_1,x_2) \,.
\label{eq:inRsc_Distribution}
\end{multline}
%
%
Before we state the capacity theorem, we give the following lemma. In principle, one may use
 the property stated below in Lemma~\ref{lemm:CardCl} in order to
 compute the region $\mathcal{R}_{\text{ET}}(P_{Y|X_1,X_2})$  %
%in (\ref{eq:inRetx}),
for a given channel.
\begin{lemma}
\label{lemm:CardCl}
The union in (\ref{eq:inRetx}) is exhausted by %
auxiliary variables $V_0$, $V_1$, $V_2$ 
with $|\Vset_0|\leq 3$, $|\Vset_k|\leq 3(|\Xset_1||\Xset_2|+2)$, $k=1,2$, and 
%$|\Xset_1|\leq (|\Hset_{A_1}|^2+2)(|\Hset_{B}|^4+2)$, and   $|\Xset_2|\leq (|\Hset_{A_2}|^2+1)(|\Hset_{B}|^2+2)$.
%
pure states $\varphi_{A_1 A_2}\equiv \ketbra{\phi_{A_1 A_2}}$ of dimension 
\begin{align}
\mathrm{dim}(\Hset_{A_1})=\mathrm{dim}(\Hset_{A_2})\leq
27(|\Xset_1||\Xset_2|+2)^2\min\{ |\Xset_1|,|\Xset_2| \} \,.
\end{align}
\end{lemma}
The proof of  is based on the Fenchel-Eggleston-Carath\'eodory lemma \cite{Eggleston:66p}, using similar arguments as 
in \cite{YardHaydenDevetak:08p} \cite[App. B]{Pereg:22p}.
The details are given in Section%Appendix
~\ref{app:CardCl}. Next, we state our capacity theorem.

\begin{theorem}
\label{theo:etMAC}
The capacity region of a classical MAC $P_{Y|X_1,X_2}$ with entangled transmitters is given by
\begin{align}
\mathcal{C}_{\text{ET}}(P_{Y|X_1,X_2})= \mathcal{R}_{\text{ET}}(P_{Y|X_1,X_2}) \,.
\end{align}
\end{theorem}
The proof of Theorem~\ref{theo:etMAC} is given in Section%Appendix
~\ref{app:etMAC}.

\subsection{Rate-Limited Entanglement Resources}
Now, we consider a setting where the entanglement resources are limited.
A code with rate-limited entanglement resources
is defined such that 
 the encoders have access to Hilbert spaces which grow as a function of the channel uses in a rate limited fashion.
%the encoders can use up to $n\theta_E$ maximally entangled qubit pairs.
The precise definition is given below.
\begin{definition} %
\label{def:ClcapacityE_Limited}
A $(2^{nR_1},2^{nR_2},%
n)$   code for the classical MAC $P_{Y|X_1,X_2}$ with an entanglement rate $\theta_E$ between the
 transmitters  consists of the following: 
\begin{itemize}
\item
an entangled state $\Psi_{E_1 E_2}$ that is shared between the transmitters, with $\mathrm{dim}(\mathcal{H}_k)\leq 2^{n\theta_E}$ for $k=1,2$.
\item 
two message sets  $[1:2^{nR_1}]$ and $ [1:2^{nR_2}]$, assuming $2^{nR_k}$ is an integer;
\item
a pair of encoding POVMs $\Fset_1^{(m_1)}=\{ F^{(m_1)}_{x_1^n} \}_{x_1^n\in\mathcal{X}_1^n}$ and 
$\Fset_2^{(m_2)}=\{ F^{(m_2)}_{x_2^n} \}_{x_2^n\in\Xset_2^n}$ on $E_1$ and $E_2$, respectively; and
\item
a decoding function   $g:\Yset^n\to [1:2^{nR_1}]\times [1:2^{nR_2}] $.
	%
%
 %
\end{itemize}
%We denote the code by $\mathscr{C}=(\Psi,\Fset_1,\Fset_2,g)$.
\end{definition}
The communication scheme is as in
Subsection~\ref{sec:Coding}
(see
%depicted in 
Figure~\ref{fig:MentangledTx}).  
% The senders share the entangled pair  $E_1 E_2$. 
% Alice $k$ chooses a message $m_k$ according to a uniform distribution over $[1:2^{nR_k}]$, for $k=1,2$.
%  To send the message $m_1\in [1:2^{nR_1}]$,
% Alice 1 applies the encoding measurement $\Fset_1^{(m_1)}$ to her share of the entanglement resource, $E_1$, and obtains a measurement outcome $x_1^n\in\mathcal{X}_1^n$.
% She sends $x_1^n$ through $n$ uses of the classical MAC $P_{Y|X_1,X_2}$. %
% Similarly, Alice 2 measures the system $E_2$ using the measurement $\Fset_2^{(m_2)}$, and transmits the outcome $x_2^n$. 
%
%
%
%
%
% The joint input distribution is thus
% \begin{align}%
% f(x_1^n,x_2^n|m_1,m_2)
% = \trace \left[  \left(F^{(m_1)}_{x_1^n}\otimes F^{(m_2)}_{x_2^n} \right) \Psi_{E_1 E_2} \right] \,.
% \end{align}%
%
% Bob receives the channel output $y^n$, and estimates the message pair as $(\htm_1,\htm_2)=g(y^n)$.
% %
% %
% %
% The conditional probability of error of the code $\mathscr{C}=(\Psi,\Fset_1,\Fset_2,g)$ is
% \begin{align}
% &P_{e}^{(n)}(\mathscr{C}|m_1,m_2)= %
% \sum_{y^n: g(y^n)\neq (m_1,m_2)} \left[  \sum_{(x_1^n,x_2^n)\in \Xset_1^n\times \Xset_2^n} f(x_1^n,x_2^n|m_1,m_2) P_{Y|X_1,X_2}^n(y^n|x_1^n,x_2^n) \right] \,.
% \end{align}
% Hence, the average probability of error is 
% \begin{align}
% &P_{e}^{(n)}(\mathscr{C})= %
% \frac{1}{2^{n(R_1+R_2)}}\sum_{m_1=1}^{2^{nR_1}}\sum_{m_2=1}^{2^{nR_2}}    P_{e}^{(n)}(\mathscr{C}|m_1,m_2) \,.
% \end{align}
As before,
a $(2^{nR_1},2^{nR_2},n,\eps)$  code satisfies that the average probability of error is bounded by $\eps$. 
% $%
% P_{e}^{(n)}(\mathscr{C})\leq\eps $. %
%
%
%
%
A rate pair $(R_1,R_2)$ is said to be achievable with  an entanglement rate $\theta_E$    if for every $\eps>0$ and sufficiently large $n$, there exists a 
$(2^{nR_1},2^{nR_2},n,\eps)$ code such that the transmitters share entanglement resources at a rate $\theta_E$. 
%
%
 The  capacity region $\opC_{\text{ET}}(P_{Y|X_1,X_2},\theta_E)$ with rate-limited  entanglement between the transmitters
is defined as the set of achievable pairs $(R_1,R_2)$, with an  entanglement rate $\theta_E$ between the transmitters. %

We  note that by definition, the capacity region with unlimited entanglement between the transmitters is $\mathcal{C}_{\text{ET}}(\cdot)\equiv \mathcal{C}_{\text{ET}}(\cdot,+\infty)$.
Therefore, for every entanglement rate
$\theta_E>0$, we have
$\mathcal{C}_{\text{ET}}(\cdot,\theta_E)\subseteq
\mathcal{C}_{\text{ET}}(\cdot)$.

Based on the achievability proof in Section%Appendix
~\ref{app:etMAC} and Lemma~\ref{lemm:CardCl}, we obtain the following consequence.
\begin{corollary}
\label{coro:etMAC_Limited}
Let $\theta_E>0$ be a given entanglement rate.
A rate pair $(R_1,R_2)$ is achievable with entanglement at rate $\theta_E$ between the transmitters if 
\begin{align}%
%\mathcal{R}_{\text{ET}}(P_{Y|X_1,X_2})=
%\bigcup_{ p_{V_0} p_{V_1|V_0} p_{V_2|V_0} \,,\; \varphi_{A_1 A_2} \,,\; \Lset_1 \otimes \Lset_2 }
%\left\{ 
\begin{array}{rl}
%  (R_1,R_2) \,:\;
	R_1 &\leq I(V_1;Y|V_0 V_2)  \\
  R_2 &\leq I(V_2;Y|V_0 V_1) \\
	R_1+R_2 &\leq I(V_1 V_2;Y|V_0)
	\end{array}
%\right\}
\label{eq:inRetx_Limited}
\end{align}%
for a pure state $\ket{\phi_{A_1 A_2}}$
with entropy
\begin{align}
H(A_1)_\phi=H(A_2)_\phi\leq \theta_E \,,
\end{align}
some classical variables $(V_0,V_1,V_2)\sim p_{V_0} p_{V_1|V_0} p_{V_2|V_0}$, 
  and conditional measurement $\Lset_k(v_0,v_k)=\{L_k(x_k|v_0,v_k)\}_{x_k\in\Xset_k}$, for $v_0\in\Vset_0$, $v_k\in\Vset_k$, $k=1,2$. 
%
% The capacity region of a classical MAC $P_{Y|X_1,X_2}$ with entangled transmitters is given by
% \begin{align}
% \mathcal{C}_{\text{ET}}(P_{Y|X_1,X_2})= \mathcal{R}_{\text{ET}}(P_{Y|X_1,X_2}) \,.
% \end{align}
 \end{corollary}
% The proof of Theorem~\ref{theo:etMAC} is given in Section%Appendix
% ~\ref{app:etMAC}.
We observe that Lemma~\ref{lemm:CardCl} further provides a sufficient entanglement rate to achieve every rate pair in $\mathcal{R}_{\text{ET}}(P_{Y|X_1,X_2})$ (see (\ref{eq:inRetx})).
The rate region $\mathcal{R}_{\text{ET}}(P_{Y|X_1,X_2})$ is also an outer bound on the capacity region with rate-limited entanglement since
$
\mathcal{C}_{\text{ET}}(\cdot,\theta_E) \subseteq
\mathcal{C}_{\text{ET}}(\cdot,+\infty)\equiv \mathcal{C}_{\text{ET}}(\cdot)=
\mathcal{R}_{\text{ET}}(\cdot)
$, where the last equality holds by 
Theorem~\ref{theo:etMAC}.
This yields the following result.
\begin{corollary}
\label{coro:Sufficient_Entanglement}
% Consider a given pure state  $\ket{\phi_{A_1 A_2}}$, a triplet
%  $(V_0,V_1,V_2)\sim p_{V_0} p_{V_1|V_0} p_{V_2|V_0}$, 
%   and a conditional measurement $\Lset_k(v_0,v_k)=\{L_k(x_k|v_0,v_k)\}_{x_k\in\Xset_k}$.
If the entanglement rate satisfies
\begin{align}
\theta_E\geq \log\left[ 27(|\Xset_1||\Xset_2|+2)^2\min\{ |\Xset_1|,|\Xset_2| \}  \right] \,,
\end{align}
% then the rate region $\mathcal{R}_{\text{ET}}(P_{Y|X_1,X_2})$ is achievable  with rate-limited entanglement between the transmitters. 
% \begin{align}%
% %\mathcal{R}_{\text{ET}}(P_{Y|X_1,X_2})=
% %\bigcup_{ p_{V_0} p_{V_1|V_0} p_{V_2|V_0} \,,\; \varphi_{A_1 A_2} \,,\; \Lset_1 \otimes \Lset_2 }
% \left\{ 
% \begin{array}{rl}
% %  (R_1,R_2) \,:\;
% 	R_1 &\leq I(V_1;Y|V_0 V_2)  \\
%   R_2 &\leq I(V_2;Y|V_0 V_1) \\
% 	R_1+R_2 &\leq I(V_1 V_2;Y|V_0)
% 	\end{array}
% \right\}
% \label{eq:inRetx}
% \end{align}%
% for a pure state $\ket{\phi_{A_1 A_2}}$
% with entropy
% \begin{align}
% H(A_1)_\phi=H(A_2)_\phi\leq R_E \,,
% \end{align} 
%
then, the capacity region %of a classical MAC $P_{Y|X_1,X_2}$ 
with rate-limited entanglement between the transmitters is given by
\begin{align}
\mathcal{C}_{\text{ET}}(P_{Y|X_1,X_2},\theta_E)= \mathcal{R}_{\text{ET}}(P_{Y|X_1,X_2}) %\,.
\end{align}
where
$\mathcal{R}_{\text{ET}}(P_{Y|X_1,X_2}) $
is defined in (\ref{eq:inRetx}).
 \end{corollary}

\begin{discussion*}
For the applications in future communication systems mentioned in the introduction, it is important that the capacity region can also be used numerically as a basis for resource allocation. This requirement can be fulfilled by Corollary~\ref{coro:Sufficient_Entanglement}. Another practical requirement is the robust dependence of the communication system performance on the channel parameters. Corollary~\ref{coro:Sufficient_Entanglement} implies that the capacity region of the  MAC with pre-shared entanglement between the transmitters is continuously dependent on the channel matrix for finite alphabets. This result can be shown immediately with the representation of the capacity region in Corollary~\ref{coro:Sufficient_Entanglement} by adapting the proof technique from \cite{BocheSchaeferPoor:20p2}.
\end{discussion*}

\begin{table}
\renewcommand{\arraystretch}{1.3}
\caption{The magic square game: Deterministic strategies}
\label{Table:Magic_Classical}
\centering
\begin{tabular}{| c| c| c| }
\hline
 0 & 0 & 0 \\ 
\hline
 0 & 1 & 1 \\  
\hline
 1 & 0 & ?    \\
\hline
\end{tabular}
\end{table}

\begin{table}
\renewcommand{\arraystretch}{1.3}
\caption{The magic square game: Quantum strategies}
\label{Table:Magic_Quantum}
\centering
\begin{tabular}{| c| c| c| }
\hline
 $\mathsf{X}\otimes\identity$ & $\mathsf{X}\otimes \mathsf{X}$ & $\identity\otimes \mathsf{X}$ \\ 
\hline
 $-\mathsf{X}\otimes \mathsf{Z}$ & $\mathsf{Y}\otimes \mathsf{Y}$ & $-\mathsf{Z}\otimes \mathsf{X}$ \\  
\hline
 $\identity\otimes \mathsf{Z}$ & $\mathsf{Z}\otimes \mathsf{Z}$ & $\mathsf{Z}\otimes \identity$    \\
\hline
\end{tabular}
\end{table}


\subsection{Pseudo-Telepathy Example}
As an example, we consider the channel that was introduced by Leditzki et al. \cite{LeditzkyAlhejjiLevinSmith:20p}.
The channel is defined in terms of a pseudo-telepathy game, i.e., a non-local game such that quantum strategies outperform classical strategies and guarantee winning with certainty. 
In the magic square game,  a referee selects one out of nine cells uniformly at random.
Suppose that the referee selected $(r,c)$.
Then, the referee informs Player 1 of the row index $r$, and Player 2 of the column index $c$.
Each player fills three bits in the respective row $r$ and column $c$. 
% 
In order to win the game, they need to satisfy three requirements: they agree on the bit value in $(r,c)$, the row $r$ has even parity, and the column $c$ has odd parity. 
Tables \ref{Table:Magic_Classical}
and \ref{Table:Magic_Quantum}
demonstrate 
 a classical strategy and a quantum strategy, respectively.
If the game is limited to a classical deterministic strategy, then winning is impossible. An attempt towards winning the game is shown in Table~\ref{Table:Magic_Classical}. 
Furthermore, using randomized strategies, the probability of winning is at most $\frac{8}{9}$.
On the other hand, it can be shown that the following quantum strategy wins the game with probability 1:
\begin{itemize}
\item
Before the game begins,
prepare the state
\begin{align}
\ket{\phi_{A_1' A_1'' A_2' A_2''}}=
\frac{1}{2}\left(
\ket{00} \ket{11}+\ket{11} \ket{00}- \ket{01} \ket{10} - \ket{10} \ket{01} \right) \,.
\label{eq:Magic_A1A2}
\end{align}
Send $A_1',A_1''$ to Player 1, and $A_2',A_2''$ to Player 2.  

\item
Play the game using the following strategy.
Having received the row and column indices, each player measures the observables in Table~\ref{Table:Magic_Quantum} simultaneously, and inserts the measurement outcomes into the corresponding cells, where
$\mathsf{X}$, $\mathsf{Y}$, and $\mathsf{Z}$ are the Pauli operators.
 The observables can be measured simultaneously because the three operators in each row and each column commute.
For example, if the referee selected $r=1$ and $c=2$, then Player 1 measures $\mathsf{X}\otimes\identity$, $\mathsf{X}\otimes \mathsf{X}$, and $\identity\otimes \mathsf{X}$, whereas Player 2 measures $\mathsf{X}\otimes \mathsf{X}$, $\mathsf{Y}\otimes \mathsf{Y}$, and $\mathsf{Z}\otimes \mathsf{Z}$.

\end{itemize}
Leditzki et al. \cite{LeditzkyAlhejjiLevinSmith:20p} defined a MAC such that the channel is ideal for input strategies that win the game, and pure-noise otherwise. 
The precise definition is given below.


In the general description of the non-local game, a referee selects two questions $q_1\in\mathcal{Q}_1$ and $q_2\in\mathcal{Q}_2$ uniformly at random, and
sends the respective question to each player.
The players choose their respective answers $a_1\in\mathcal{A}_1$ and $a_2\in\mathcal{A}_2$ using either deterministic or random strategies $f_k:\mathcal{Q}_k\to \mathcal{A}_k$, for $k=1,2$.
The game is won if $(q_1,q_2,a_1,a_2)\in\mathscr{G}$, where $\mathscr{G}$ is the winning set.
The CHSH game and magic square game are special cases of this description.
%
Leditzki et al. \cite{LeditzkyAlhejjiLevinSmith:20p} defined a classical MAC $P_{Y|X_1,X_2}$, with
\begin{align}
\Xset_k &=\mathcal{Q}_k\times \mathcal{A}_k \,,\; k=1,2\\
\Yset&= \mathcal{Q}_1\times \mathcal{Q}_2
\intertext{such that}
P_{Y|X_1,X_2}\big( (\hat{q}_1,\hat{q}_2) \,\big|  q_1,a_1,q_2,a_2  \big)&=
\begin{cases}
\delta_{q_1, \hat{q}_1} \delta_{q_2, \hat{q}_2} &\text{ if } (q_1,q_2,a_1,a_2)\in\mathscr{G}\,,\\
\frac{1}{|\mathcal{Q}_1| |\mathcal{Q}_2|} &\text{otherwise.}
\end{cases}
\end{align}
In words, if the inputs $X_1=(q_1,a_1)$ and $X_2=(q_2,a_2)$ win the game, then the decoder receives the question pair precisely, i.e.,
$Y=(q_1,q_2)$ with probability $1$. Otherwise, if the game is lost, then the output $Y$ is uniformly distributed over the question set.

Formally, the magic square game is specified by questions $(q_1,q_2)$ from $\{1,2,3\}\times \{1,2,3\}$,
answers $a_1,a_2$ from $\Aset_1=\Aset_2=\{ 0,1 \}^3$, and the winning set $\mathscr{G}=\big\{$ $\big( q_1,q_2, 
(a_1[j],a_2[j])_{j=1,2,3} \big)
\in\Qset_1\times\Qset_2\times\Aset_1\times\Aset_2:$ $a_1[q_1]=a_2[q_2]$,
 $ a_1[1]+ a_1[2] + a_1[3] \mod 2=0 $, and
 $a_2[1]+ a_2[2] + a_2[3] \mod 2=1$
$\big\}$. Without entanglement resources, the sum-rate  is bounded by \cite{SeshadriLeditzkySiddhuSmith:22c}
\begin{align}
R_1+R_2\leq 3.02 \,.
\end{align}
Based on \cite{LeditzkyAlhejjiLevinSmith:20p}, the sum rate $R_1+R_2=2\log(3)\approx 3.17$ is achievable with entangled transmitters.
This can also be obtained as a consequence of our result.
By Theorem~\ref{theo:etMAC}, the capacity region with entangled transmitters is given by 
\begin{align}
\mathcal{C}_{\text{ET}}(P_{Y|X_1,X_2})=
\left\{ \begin{array}{rl}
  (R_1,R_2) \,:\;
	&R_1 \leq \log(3)  \\
  &R_2 \leq \log(3) 
	\end{array}
\right\} \,.
\label{eq:Cet_Magic}
\end{align}
To see why, consider the region in (\ref{eq:inRetx}). The converse part is immediate since 
the set on the right hand side of (\ref{eq:Cet_Magic}) is the capacity region of the noiseless MAC
$\widetilde{P}_{Y|X_1,X_2}\big( (\hat{q}_1,\hat{q}_2) \,\big|  q_1,a_1,q_2,a_2  \big)=\delta_{q_1, \hat{q}_1} \delta_{q_2, \hat{q}_2}$ (with or without entanglement resources).
%the classical conditional mutual information is bounded by $I(X;Y|Z)\leq \log|\Xset|$.
%
As for the direct part, we choose an entangled state, classical variables, and POVMs as follows. Let the entangled state $\ket{\phi_{A_1 A_2}}$ be as in (\ref{eq:Magic_A1A2}), where
$A_k\equiv A_k' A_k''$ for $k=1,2$.
Furthremore, we let $\Vset_0=\emptyset$ and set the joint distribution $p_{V_k}$ to be uniform over $\Vset_k=\{1,2,3\}$, for $k=1,2$.
Thus, the random pair $(V_1,V_2)$ is distributed as the referee's questions.
Given $V_k$, Alice $k$ performs a measurement $\Lset_k(V_k)$ as follows. Alice 1 measures the observables in row number $r=V_1$ in Table~\ref{Table:Magic_Quantum},  obtains a random triplet
$W_1\equiv (a_1[j])_{j=1,2,3}$, and transmits $X_1=(V_1,W_1)$.
Similarly, Alice 2 measures the observables in column number $c=V_2$,  obtains 
$W_2\equiv (a_2[j])_{j=1,2,3}$, and transmits $X_2=(V_2,W_2)$. Since $(V_1,V_2,W_1,W_2)$ win the game, we have $Y=(V_1,V_2)$  with probability 1, hence
$I(V_1 V_2;Y)=H(V_1 V_2)=2\log(3)$, $I(V_1;Y|V_2)=H(V_1)=\log(3)$, and $I(V_2;Y|V_1)=H(V_2)=\log(3)$.
This requires an entanglement rate of 
$\theta_E=2$ qubit pairs per transmission.




% \section*{Acknowledgment}
% U. Pereg was supported by the Israel VATAT Junior Faculty Program for Quantum Science and Technology through Grant 86636903, and the Chaya Career Advancement Chair, Grant 8776026.
%  C. Deppe and H. Boche were supported by
% %
% the German Federal
% Ministry of Education and Research (BMBF) through Grants
% 16KISQ028 (Deppe) and 16KISQ020 %
% (Boche). 
% This work of H. Boche was supported in part by the BMBF within the national initiative for
% ``Post Shannon Communication (NewCom)" under Grant 16KIS1003K, and in
% part by the DFG within the Gottfried Wilhelm
% Leibniz Prize under Grant BO 1734/20-1 and within Germany's Excellence
% Strategy EXC-2092 – 390781972 and EXC-2111 – 390814868. 
% U. Pereg was also supported by the Helen Diller Quantum Center at the Technion.



%\begin{appendices}




\section{Proof of Lemma~\ref{lemm:CardCl} %(Purification and Dimension Bounds)
}
\label{app:CardCl}
In this section, we prove Lemma~\ref{lemm:CardCl} and show that the region $\mathcal{R}_{\text{ET}}(P_{Y|X_1,X_2})$ can be exhausted with pure states and bounded dimensions.

\subsection{Purification}
First, we consider a given Hilbert space
$\Hset_{A_1}\otimes \Hset_{A_2}$, and
show that the union over entangled states $\varphi_{A_1 A_2}$ is exhausted by pure states.
Fix a Hilbert space $\Hset_{A_1}\otimes \Hset_{A_2}$.
Consider a given distribution $\{ p_{V_1|V_0}(\cdot|v_0)p_{V_2|V_0}(\cdot|v_0)\}$, an entangled state 
$\varphi_{A_1 A_2}$, and measurements $\Lset_1(v_0,v_1)$ and $\Lset_2(v_0,v_2)$ on $A_1$ and $A_2$, respectively.
Let $\mathfrak{R}(\varphi,\Lset_1,\Lset_2)$ denote the associated rate region,
\begin{align}%
\mathfrak{R}(\varphi,\Lset_1,\Lset_2)=
%\bigcup_{ p_{V_0} p_{V_1|V_0} p_{V_2|V_0} \,,\; \varphi_{A_1 A_2} \,,\; \Lset_1 \otimes \Lset_2 }
\left\{ \begin{array}{rl}
  (R_1,R_2) \,:\;
	R_1 &\leq I(V_1;Y|V_0 V_2)  \\
  R_2 &\leq I(V_2;Y|V_0 V_1) \\
	R_1+R_2 &\leq I(V_1 V_2;Y|V_0)
	\end{array}
\right\} \,.
\label{eq:inRphi}
\end{align}%
 %Let $(R_1,R_2)$ be a rate pair such that 
%\begin{align}
%R_1&\leq I(V_1;Y|V_0 V_1) \,,\\
%R_2&\leq I(V_2;Y|V_0 V_2) \,,\\
%R_1+R_2&\leq I(V_1 V_2 ;Y|V_0) \,.
%\end{align}
%
Given a joint state $\varphi_{A_1 A_2}$,
consider a spectral decomposition,
\begin{align}
\varphi_{A_1 A_2}=\sum_{z\in\Zset} p_Z(z) \ketbra{ \phi_z }_{A_1 A_2} \,,
\label{eq:Cardinality_Spectral}
\end{align}
where $p_Z$ is a probability distribution, such that the pure states, $\ket{\phi_z}_{A_1 A_2}$, $z\in\Zset$, form an orthonormal basis for 
$\Hset_{A_1} \otimes \Hset_{A_2}$.
Then, $\varphi_{A_1 A_2}$ has the following purification,
\begin{align}
\ket{\psi_{A_1 A_2 E_1 E_2}}= \sum_{z\in\Zset} \sqrt{p_Z(z)} \ket{ \phi_z }_{A_1 A_2}\otimes \ket{z}_{E_1} \otimes \ket{z}_{E_2}
\,.
\end{align}
Now, we consider rate region $\mathfrak{R}(\psi,\Lset_1',\Lset_2')$ that is associated with the following choice of state, distribution, and measurements.
Set the distribution $\{ p_{V_1|V_0}(\cdot|v_0)p_{V_2|V_0}(\cdot|v_0)\}$ as before. Let
Alice 1 and Alice 2 share the state $\ket{\psi_{A_1 E_1 A_2  E_2}}$, and 
suppose that Alice 1 performs a measurement on $(A_1,E_1)$, and Alice 2 on $(A_2,E_2)$, with the following POVMs,
\begin{align}
L_1'(x_1,z|v_0,v_1)&= L(x_1|v_0,v_1)\otimes \ketbra{z}_{E_1} \,,\\
L_2'(x_2,z|v_0,v_2)&= L(x_2|v_0,v_2)\otimes \ketbra{z}_{E_2} \,,
\end{align}
for $(v_0,v_1,v_2,x_1,x_2,z)\in \Vset_0\times\Vset_1\times\Vset_2\times\Xset_1\times\Xset_2\times \Zset$.
The corresponding input distribution $p'_{X_1,X_2|V_0,V_1,V_2}$ is given by
\begin{align}
p'_{X_1,X_2|V_0,V_1,V_2}(x_1,x_2|v_0,v_1,v_2)&=
\sum_{z\in\Zset} p_Z(z) \trace\left[ \left( L_1(x_1|v_0,v_1)\otimes L_2(x_2|v_0,v_2) \right) \ketbra{ \phi_z }_{A_1 A_2}  \right]
\\
&=
 \trace\left[ \left( L_1(x_1|v_0,v_1)\otimes L_2(x_2|v_0,v_2) \right)\left(\sum_{z\in\Zset}  p_Z(z)\ketbra{ \phi_z }_{A_1 A_2} \right) \right]
\\
&=
 \trace\left[ \left( L_1(x_1|v_0,v_1)\otimes L_2(x_2|v_0,v_2) \right)\varphi_{A_1 A_2} \right]
\\
&=
p_{X_1,X_2|V_0,V_1,V_2}(x_1,x_2|v_0,v_1,v_2)
\end{align}
where the third equality follows from (\ref{eq:Cardinality_Spectral}), and the last from (\ref{eq:inRsc_Distribution}).
Therefore, $\mathfrak{R}(\psi,\Lset_1',\Lset_2')=\mathfrak{R}(\varphi,\Lset_1,\Lset_2)$.
We deduce that the entire region $\mathcal{R}_{\text{ET}}(P_{Y|X_1,X_2})$, as in (\ref{eq:inRetx}), can be obtained from pure states.

 
\subsection{Dimension Bounds}
Next, we bound the cardinality of the alphabet $\Vset_0$. Consider a given distribution $\{ p_{V_1|V_0}(\cdot|v_0)p_{V_2|V_0}(\cdot|v_0)\}$, a joint state 
$\varphi_{A_1 A_2}$, and POVM collections, $\Lset_1(v_0,v_1)$ and $\Lset_2(v_0,v_2)$.
%Every density matrix $\rho_A$ has a unique parametric representation $\mathscr{V}(\rho_A)$ of dimension %
%%
 %$|\Hset_A|^2-1$. 
%Then, 
Define a map $\tau_0:\Vset_0\rightarrow \mathbb{R}^{3}$ by
\begin{align}%
\tau_{0}(v_0)= \Big(  %\mathscr{V}(\omega_B^u) \,,\; 
   I(V_1;Y|V_2, V_0=v_0)   \,,\; I(V_2;Y|V_1, V_0=v_0)  \,,\; I(V_1, V_2;Y|V_0=v_0)  \Big) \,.
\end{align}%
%where $\omega_{B}^{u}=\sum_{x_1,x_2}  p_{X_1|U}(x_1|u)p_{X_2|U}(x_2|u) $ $\channel\left( \trace_E\left(\Lset(\theta_{A_1}^{x_1})\right)\otimes \zeta_{A_2}^{x_2}\right) $. 
The map $\tau_0$ can be extended to a map that  acts on probability distributions as follows,
\begin{align}%
T_{0} \,:\; p_{V_0}(\cdot)  \mapsto
\sum_{v_0\in\Vset_0} p_{V_0}(v_0) f_{0}(v_0)= \Big(  I(V_1;Y|V_0 V_2) \,,\; I(V_2;Y|V_0 V_1)   \,,\; I(V_1 V_2;Y|V_0 )   \Big)  \,.
\end{align}%
According to the Fenchel-Eggleston-Carath\'eodory lemma \cite{Eggleston:66p}, any point in the convex closure of a connected compact set within $\mathbb{R}^d$ belongs to the convex hull of $d$ points in the set. 
Since the map $T_{0}$ is linear, it maps  the set of distributions on $\Vset_0$ to a connected compact set in $\mathbb{R}^{3}$. %
Thus, for every  $p_{V_0}$, 
there exists a probability distribution $p_{\bar{V}_0}$ on a subset $\bar{\Vset}_0\subseteq \Vset_0$ of size $%
3$, such that 
$%
T_{0}(p_{{\bar{V}_0}})=T_{0}(p_{V_0}) %
$. %
We deduce that alphabet size can be restricted to $|\Vset_0|\leq 3$, while preserving $I(V_1;Y|V_0 V_2)$, $ I(V_2;Y|V_0 V_1)$, and $I(V_1 V_2;Y|V_0 )$. 

Next, we bound the alphabet size for the auxiliary variables $V_1$ and $V_2$.
Every probability distribution $p_X$ can be represented by %
%
 $|\Xset|-1$ parameters. 
%Every density matrix $\rho_A$ has a unique parametric representation $\mathscr{W}(\rho_A)$ of dimension %
%%
 %$|\Hset_A|^2-1$. 
Then, for every $v_0\in\Vset_0$, define a map $\tau_{12|v_0}:\Vset_1\times \Vset_2\rightarrow \mathbb{R}^{|\Xset_1| |\Xset_2|+2}$ by
\begin{multline}%
\tau_{12|v_0}(v_1,v_2)= \Big(  (p_{X_1,X_2|V_0,V_1,V_2}(\cdot|v_0,v_1,v_2)) \,,\; H(Y|V_0=v_0,V_1=v_1)  \,,\; \\
H(Y|V_0=v_0,V_2=v_2)  \,,\; H(Y|V_0=v_0,V_1=v_1,V_2=v_2)      \Big) \,,
\end{multline}%
where $p_{X_1,X_2|V_0,V_1,V_2}(x_1,x_2|v_0,v_1,v_2)=\trace\left[ (L_1(x_1|v_0,v_1)\otimes L_2(x_2|v_0,v_2)) \varphi_{A_1 A_2} \right]$.
Then, the map $\tau_{12|v_0}$ is extended to %
\begin{align}%
T_{12|v_0} \,:\; p_{V_1,V_2|V_0}(\cdot|v_0)  \mapsto &
\sum_{(v_1,v_2)\in\Vset_1\times \Vset_2} p_{V_1,V_2|V_0}(v_1,v_2|v_0) f_{12|v_0}(v_1,v_2)= \nonumber\\& \Big(  
(p_{X_1,X_2|V_0}(\cdot|v_0)) \,,\;  H(Y|V_0=v_0,V_1) \,,\;  H(Y|V_0=v_0,V_2)  \,,\; H(Y|V_0=v_0,V_1,V_2)   \Big) %
\end{align}%
with
\begin{align}
p_{X_1,X_2|V_0}(x_1,x_2|v_0)\equiv \sum_{v_1,v_2}  p_{V_1|V_0}(v_1|v_0) p_{V_2|V_0}(v_2|v_0)\trace\left[ (L_1(x_1|v_0,v_1)\otimes L_2(x_2|v_0,v_2)) \varphi_{A_1 A_2} \right] \,.
\label{eq:thetaU}
\end{align} 
Thus, by the Fenchel-Eggleston-Carath\'eodory lemma \cite{Eggleston:66p}, for every  $p_{V_1,V_2|V_0}(\cdot|v_0)$, 
there exists a probability distribution $p_{\bar{V}_1,\bar{V}_2|V_0}(\cdot|v_0)$ on a subset $\bar{\Vset}_1\otimes \bar{\Vset}_2\subseteq \Vset_1\otimes \Vset_2$ of size $%
|\Xset_1| |\Xset_2|+2$, such that 
$%
F_{12|v_0}(p_{{\bar{V}_1,\bar{V}_2|V_0}}(\cdot|v_0))=F_{12|v_0}(p_{V_1,V_2|V_0}(\cdot|v_0))) %
$. %
We deduce that alphabet size can be restricted to $|\Vset_k|\leq 3(|\Xset_1| |\Xset_2|+2)$, while preserving  $H(Y|V_0 V_1)$, $H(Y|V_0 V_2)$, 
$H(Y|V_0 V_1 V_2)$,  and  $p_{X_1,X_2|V_0}$.
%\begin{align}
%&p_{X_1,X_2,Y|V_0}(x_1,x_2,y|v_0)=p_{X_1,X_2|V_0}(x_1,x_2|v_0) P_{Y|X_1,X_2}(y|x_1,x_2) \,. 
%\end{align}
 This implies that the distribution $p_{X_1,X_2,Y|V_0}(x_1,x_2,y|v_0)=p_{X_1,X_2|V_0}(x_1,x_2|v_0) P_{Y|X_1,X_2}(y|x_1,x_2)$ remains the same, and so do $H(Y|V_0)$, $I(V_1;Y|V_0 V_2)=
H(Y|V_0 V_2)-H(Y|V_0 V_1 V_2)$, and $I(V_2;Y|V_0 V_1)=
H(Y|V_0 V_1)-H(Y|V_0 V_1 V_2)$, as well as
 $H(Y|V_0)$, $I(V_1 V_2;Y|V_0)=H(Y|V_0)-H(Y|V_0 V_1 V_2)$.
%
%
 This completes the proof for the cardinality bounds.

As for the dimension,
assume without loss of generality that $\Xset_k=\{1,\ldots,|\Xset_k|\}$, for $k=1,2$.
In general, every distribution $p(i,j)$ that is obtained from local measurements of a state $\varphi_{A_1 A_2}$ can also be obtained by 
projective measurements on  systems $A_1' A_2'$ in the state
$\ket{\psi_{A_1' A_2'}}=
\sum_{i=1}^{|\Xset_1|} \sum_{j=1}^{|\Xset_2|} \sqrt{p(i,j)} \ket{i}\otimes \ket{j}$, where $\{\ket{i}\}$ and $\{\ket{j}\}$ are standard bases.
Thus, the union can be restricted to pure states of dimension $\mathrm{dim}(\Hset_{A_k})\leq
|\Vset_0||\Vset_1||\Vset_2||\Xset_k|$ for $k=1,2$.
Since every pure state has a Schmidt decomposition with a rank $r$ that is bounded by $\min\{\mathrm{dim}(\Hset_{A_1}),\mathrm{dim}(\Hset_{A_2})\}$, it suffices to consider
$\mathrm{dim}(\Hset_{A_1})=\mathrm{dim}(\Hset_{A_2})\leq
|\Vset_0||\Vset_1||\Vset_2|\min\{|\Xset_1|,|\Xset_2| \}$.
%
%
\qed










\section{Proof of Theorem~\ref{theo:etMAC}}
\label{app:etMAC}
Consider the classical MAC $P_{Y|X_1,X_2}$ with entanglement resources between the transmitters.

\subsection*{Achievability Proof}
We show that for every $\delta_1,\delta_2,\eps_0>0$, there exists a $(2^{n(R_1-\delta_1)},2^{n(R_2-\delta_2)},n,\eps_0)$ code for $P_{Y|X_1,X_2}$ with entangled transmitters, provided that $(R_1,R_2)\in \mathcal{R}_{\text{ET}}(P_{Y|X_1,X_2})$. 
To prove achievability, we use coded time sharing.





%
Fix a joint distribution $p_{V_0} p_{V_1|V_0} p_{V_2|V_0}$, an entangled state $\varphi_{A_1 A_2}$, and collection of POVMs $\Lset_k(v_0,v_k)=\{ L_k(x_k|v_0,v_k) \}$ for $k=1,2$. 
Suppose that Alice 1 and Alice 2 share an $n$ copies of the  entangled state, 
\begin{align}
\varphi_{A_1^n A_2^n}\equiv \varphi_{A_1 A_2}^{\otimes n} \,.
\end{align}


\vspace{0.2cm}
%
%
The code construction, encoding with shared entanglement, and decoding procedures are described below.

\subsubsection{Code Construction}
Select a random time-sharing sequence $v_0^n$, according to an i.i.d. distribution, $\prod_{i=1}^n p_{V_0}(v_{0,i})$. Furthermore, select
$2^{nR_1}$ conditionally independent sequences, $v_1^n(m_1)$, $m_1\in [1:2^{nR_1}]$,  each distributed as 
$\prod_{i=1}^n p_{V_1|V_0}\big(v_{1}[i] \,\big| v_{0}[i] \big)$.
In a similar manner,
select
$2^{nR_2}$ sequences,  $v_2^n(m_2)$, %$m_2\in [1:2^{nR_2}]$,    
according to  $\prod_{i=1}^n p_{V_2|V_0}\big(v_{2}[i] \,\big| v_{0}[i] \big)$.

%
%
The auxiliary codebooks above are revealed to Alice 1, Alice 2, and Bob.

\vspace{0.1cm}
\subsubsection{Encoder k}
%
%
Given the message $m_k\in [1:2^{nR_k}]$ and the codebooks above,  perform the measurement $\bigotimes_{i=1}^n\left(
\Lset_k\big(v_{0}[i],v_{k}[i] \big) \right)$ on the entangled system $A_k^n$, and transmit the measurement outcome $x_k^n$ through the channel, for $k=1,2$. 

This yields the following input distribution,
\begin{align}
f(x_1^n,x_2^n|m_1,m_2)&=  \trace\left[ \left( L_1^n\big(x_1^n \,\big| v_0^n,v_1^n(m_1) \big) \otimes 
L_2^n \big( x_2^n \,\big| v_0^n,v_2^n(m_2) \big)   \right)  
\varphi_{A_1^n A_2^n}
  \right]
\nonumber\\
&= \prod_{i=1}^n  \trace\left[ \left( L_1\big( x_{1}[i] \,\big| v_{0}[i],v_{1}(m_1)[i] \big) \otimes 
L_2\big( x_{2}[i] \,\big| v_{0}[i],v_{2}(m_2)[i] \big)   \right)  
\varphi_{A_1 A_2}
  \right]	
	\,,
%
\end{align}
where we use the short notations  $L_k^n\big( x_k^n \,\big| v_0^n,v_k^n \big)\equiv 
\bigotimes_{i=1}^n L_k \big( x_{k}[i] \,\big| v_{0}[i],v_{k}[i] \big)  $, 
for $k=1,2$. %
%

\subsubsection{Decoder}
%
%
Let $\delta>0$.
Find a unique pair $(\hm_1,\hm_2)$ such that $(v_0^n,v_1^n(\hm_1),v_2^n(\hm_2),y^n)\in\Aset_\delta^{(n)}(p_{V_0,V_1, V_2, Y})$, where the marginal distribution $p_{V_0,V_1, V_2, Y}$ 
 is induced by the following joint distribution,
\begin{multline}
p_{V_0,V_1,V_2,X_1,X_2,Y}(v_0,v_1,v_2,x_1,x_2,y)= p_{V_0}(v_0) p_{V_1|V_0}(v_1|v_0) p_{V_2|V_0}(v_2|v_0)\\ \cdot \trace\left[ \left( L_1(x_{1}|v_0,v_{1}) \otimes L_2(x_{2}|v_0,v_{2})   \right) \varphi_{A_1 A_2}
  \right]
		\cdot P_{Y|X_1,X_2}(y|x_1,x_2) \,.
	\label{eq:DirectJointDistribution}
\end{multline}

%
%


\subsubsection{Analysis of Probability of Error}
 We use the notation $\eps_i(\delta)$, $i=1,2,\ldots$,
for terms that tend to zero as $\delta\rightarrow 0$.
By symmetry, we may assume without loss of generality that the transmitters send the messages $M_1=M_2=1$.
Consider the following error events,
\begin{align}
%
\mathscr{E}_0=& \{  (V_0^n,V_1^n(1),V_2^n(1),Y^n)\notin \Aset_{\delta_1}^{(n)}(p_{V_0,V_1,V_2,Y}) \} \\
\mathscr{E}_1=& \{  (V_0^n,V_1^n(m_1),V_2^n(1),Y^n)\in \Aset_{\delta}^{(n)}(p_{V_0,V_1,V_2,Y})\text{, for some $m_1\neq 1$}  \}\\
\mathscr{E}_2=& \{  (V_0^n,V_1^n(1),V_2^n(m_2),Y^n)\in \Aset_{\delta}^{(n)}(p_{V_0,V_1,V_2,Y})\text{, for some $m_2\neq 1$}  \}\\
\mathscr{E}_3=& \{  (V_0^n,V_1^n(m_1),V_2^n(m_2),Y^n)\in \Aset_{\delta}^{(n)}(p_{V_0,V_1,V_2,Y})\text{, for some $m_1\neq 1$ and $m_2\neq 1$}  \}
\end{align}
 with $\delta_1\equiv \delta/(2 |\Vset_0| |\Vset_1| |\Vset_2|)$.
By the union of events bound, the probability of error is bounded by
\begin{align}
P_{e}^{(n)}(\mathscr{C}|1,1) %
&\leq %
 \prob{ \mathscr{E}_0 }%
+ \cprob{ \mathscr{E}_1 }{ \mathscr{E}_0^c }
+ \cprob{ \mathscr{E}_2 }{ \mathscr{E}_0^c }
+ \cprob{ \mathscr{E}_3 }{ \mathscr{E}_0^c } 
\label{eq:PeBsc}
\end{align}
where the conditioning on $M_1=M_2=1$ is omitted for convenience of notation.
Observe that the (classical) codewords $V_0^n$, $V_1^n(1)$, $V_2^n(1)$,  channel inputs $X_1^n$, $X_2^n$, and channel output $Y^n$, are jointly i.i.d. according to $p_{V_0,V_1,V_2,X_1,X_2,Y}$, as in (\ref{eq:DirectJointDistribution}).
Hence,
%
by the weak law of large numbers, the first probability term, $\prob{ \mathscr{E}_0 }$, tends to zero as $n\rightarrow\infty$
\cite{CsiszarKorner:82b} \cite[Th. 1.1]{Kramer:08n}.
%

As for the second error term, we have by the union bound:
\begin{align}
\cprob{ \mathscr{E}_1 }{ \mathscr{E}_0^c }\leq 
\sum_{m_1\neq 1} \cprob{(V_0^n,V_1^n(m_1),V_2^n(1),Y^n)\in \Aset_{\delta}^{(n)}(p_{V_1,V_2,Y})}{ \mathscr{E}_0^c } \,.
\label{eq:E1bound1}
\end{align}
%Consider the probability term in the sum above.
 Given $\mathscr{E}_0^c$, it follows that $(V_0^n,V_2^n(1))\in \Aset^{\delta}(p_{V_2}) $. %
Thus, each summand %probability term 
is bounded by
\begin{align}
%
\sum_{(v_0^n,v_2^n)\in \Aset_{\delta}^{(n)}(p_{V_0,V_2})} p_{V_0,V_2}^n(v_0^n,v_2^n) \left[
\sum_{v_1^n,y^n \,:\; (v_0^n,v_1^n,v_2^n,y^n)\in \Aset_{\delta}^{(n)}(p_{V_0,V_1,V_2,Y})}
p_{V_1|V_0}^n(v_1^n|v_0^n)\cdot p_{Y^n|V_0^n,V_2^n}(y^n|v_0^n,v_2^n) \right]
\end{align}
since for every $m_1\neq 1$, the codeword $V_1^n(m_1)$ is conditionally independent of the sequence pair $(V_2^n(1),Y^n)$, given $V_0^n=v_0^n$. 
Then, by standard method-of-types arguments,
the sum within the square brackets is bounded by $2^{-n(I(V_1;Y|V_0 V_2)-\eps_1(\delta))}$ 
\cite{CsiszarKorner:82b} \cite[Th. 1.3]{Kramer:08n}. %(see Lemmas 2.5 and 2.13 in \cite{CsiszarKorner:82b}). 
Hence, by (\ref{eq:E1bound1}),
%
\begin{align}
\cprob{ \mathscr{E}_1 }{ \mathscr{E}_0^c }\leq 
2^{-n[I(V_1;Y|V_0 V_2)-R_1-\eps_1(\delta)]} \,.
\label{eq:E1bound2}
\end{align}
Thereby, the error probability $\cprob{ \mathscr{E}_1 }{ \mathscr{E}_0^c }$  tends to zero as $n\to\infty$, provided that 
\begin{align}
R_1<I(V_1;Y|V_0 V_2)-\eps_1(\delta) \,.
\end{align}
Following similar arguments, the probability terms $ \cprob{ \mathscr{E}_2 }{ \mathscr{E}_0^c }$ and
$ \cprob{ \mathscr{E}_3 }{ \mathscr{E}_0^c } $ also tend to zero, provided that
\begin{align}
R_2<I(V_2;Y|V_0 V_1)-\eps_2(\delta)
\intertext{and}
R_1+R_2<I(V_1 V_2;Y|V_0)-\eps_3(\delta)
 \,.
\end{align}
We conclude that the probability of error, averaged over the class of the codebooks above, tends to zero as $n\to\infty$.
Therefore, there must exist a $(2^{nR_1},2^{nR_2},n,\varepsilon)$ code, for a sufficiently large $n$. 
This completes the achievability proof.

\subsection*{Converse Proof}
Consider the classical MAC with entanglement resources between the transmitters. 
%
We now show the converse part. 
%
 Suppose that Alice 1 and Alice 2 share an entangled state $\Psi_{E_1 E_2}$. Then, Alice $k$ chooses a message $m_k$, uniformly at random from $[1:2^{nR_k}]$, for $k=1,2$. 
%
%
She encodes her message by performing a measurement $\Fset_k^{(m_k)}=\{ F_{x_k^n}^{(m_k)} \}$ on her share of the entangled resources, $E_k$, and sends the measurement outcome $X_k^n$ over the channel. 
%
%
 Bob receives the output $Y^n$ and finds an estimate $(\hm_1,\hm_2)=g(Y^n)$ of the message pair.

Consider a sequence of codes $(\Psi_n,\Fset_{1n},\Fset_{2n},g_n)$ such that the average probability of error tends to zero, hence
the error probabilities $\prob{ \hm_1\neq m_1 |m_2}$, $\prob{ \hm_2\neq m_2 |m_1}$, and $\prob{ (\hm_1,\hm_2)\neq (m_1,m_2)}$,  are bounded by some
$\alpha_n$ which tends to zero as $n\rightarrow \infty$.
%
%
By Fano's inequality \cite{CoverThomas:06b}, it follows that%
\begin{align}
H(m_1|\hm_1,m_2) \leq n\eps_{1n}\\
H(m_2|\hm_2,m_1) \leq n\eps_{2n}\\
H(m_1,m_2|\hm_1,\hm_2) \leq n\eps_{3n}\\
\label{eq:AFWC2c}
\end{align}
where $\eps_{k\,n}$ tend to zero as $n\rightarrow\infty$.
Hence, 
\begin{align}
%
nR_1&= H(m_1|m_2)=I(m_1;\hm_1|m_2)+H(m_1|\hm_1 m_2) 
\nonumber\\
&\leq I(m_1;\hm_1|m_2)+n\eps_{1n} \nonumber\\
&\leq I(m_1;Y^n |m_2)+n\eps_{1n}
\label{eq:ConvIneq1SC}
\end{align}
where the last inequality follows from the data processing inequality. Applying the chain rule, we can rewrite this %inequality 
as
\begin{align}
%
n(R_1-\eps_{1n})
&\leq \sum_{i=1}^n I\left(m_1;Y[i] \,\big|\; Y^{i-1}, m_2 \right) \,. %\nonumber\\
%&\leq \sum_{i=1}^n I(m_1 Y^{i-1};Y_i ) \nonumber\\
%&\leq \sum_{i=1}^n I(m_1 X_1^{i-1} X_2^{i-1};Y_i ) 
\label{eq:ConvIneq1}
\end{align}
where $Y^{i-1}\equiv Y[1],\ldots,Y[i-1]$,
for $i=2,\ldots,n$, and $Y^0\equiv \emptyset$.
Define $V_{0}[i]\equiv Y^{i-1}$, $V_{1}[i]=(m_1,Y^{i-1})$, and $V_{2}[i]=(m_2,Y^{i-1})$, for $i=1,\ldots,n$.
Notice that $V_{1}[i]$ and $V_{2}[i]$ are statistically conditionally independent given $V_{0}[i]$. 
Furthermore, since Alice 1 performs a measurement that depends on her message $m_1$ alone,
the channel input $X_{1}[i]$ is the outcome of a measurement  
$L_1(x_{1}|i,V_{0}[i],V_{1}[i])\equiv F_{x_{1}}^{(m_1|i)}$,
where $F_{a}^{(m_1|i)}=\sum\limits_{x_1^n\in\Xset_1^n\,:\; x_1[i]=a} F_{x_1^n}^{(m_1)}$ for 
$a\in\Xset_1$, $i\in [1:n]$, and $m_1\in [1:2^{nR_1}]$.
Similarly, the  input $X_{2}[i]$ is obtained from a measurement  
$L_2(x_{2}|i,V_{0}[i],V_{2}[i])\equiv F_{x_{2}}^{(m_2|i)}$.

 Then,
\begin{align}
%
R_1-\eps_{1n}
&\leq \frac{1}{n}\sum_{i=1}^n I\left(V_{1}[i];Y[i] \,\big|\; V_{0}[i], V_{2}[i] \right) \nonumber\\
&=  I\left(V_{1}[J];Y[J] \,\big|\; V_{0}[J], V_{2}[J], J \right) \,.
\label{eq:ConvIneq1a}
\end{align}
where the index $J$ is drawn  uniformly at random from $\{1,\ldots,n\}$, and it is uncorrelated with the previous systems.
%
Following the same considerations,
\begin{align}
%
R_2-\eps_{2n}
&\leq  I\left(V_{2}[J];Y[J] \,\big|\; V_{0}[J], V_{1} [J], J \right) 
\label{eq:ConvIneq2}
\intertext{and}
R_1+R_2-\eps_{3n}
&\leq  I\left(V_{1}[J] V_{2}[J];Y[J] |V_{0}[J], J \right) \,. 
\label{eq:ConvIneq3}
\end{align}
The proof follows by defining $V_0\equiv (J,V_{0}[J])$, $V_k\equiv V_{k}[J] $, for $k=1,2$, then
$X_k\equiv X_{k}[J]$, and $Y\equiv Y[J]$. 
\qed











%\end{appendices}


\section*{Acknowledgment}
U. Pereg was supported by the Israel VATAT Junior Faculty Program for Quantum Science and Technology through Grant 86636903,  the Chaya Career Advancement Chair, Grant 8776026, , and the German-Israeli Project Cooperation (DIP), Grant 2032991.
 C. Deppe and H. Boche were supported by
%
the German Federal
Ministry of Education and Research (BMBF) through Grants
16KISQ028 (Deppe) and 16KISQ020 %
(Boche). 
This work of H. Boche was supported in part by the BMBF within the national initiative for
``Post Shannon Communication (NewCom)" under Grant 16KIS1003K, and in
part by the DFG within the Gottfried Wilhelm
Leibniz Prize under Grant BO 1734/20-1 and within Germany's Excellence
Strategy EXC-2092 – 390781972 and EXC-2111 – 390814868. 
U. Pereg was also supported by the Helen Diller Quantum Center at the Technion.


\bibliography{References}{}

%
%


\end{document}
