%File: anonymous-submission-latex-2024.tex
\documentclass[letterpaper]{article} % DO NOT CHANGE THIS
\usepackage[submission]{aaai24}  % DO NOT CHANGE THIS
\usepackage{times}  % DO NOT CHANGE THIS
\usepackage{helvet}  % DO NOT CHANGE THIS
\usepackage{courier}  % DO NOT CHANGE THIS
\usepackage[hyphens]{url}  % DO NOT CHANGE THIS
\usepackage{graphicx} % DO NOT CHANGE THIS
\urlstyle{rm} % DO NOT CHANGE THIS
\def\UrlFont{\rm}  % DO NOT CHANGE THIS
\usepackage{natbib}  % DO NOT CHANGE THIS AND DO NOT ADD ANY OPTIONS TO IT
\usepackage{caption} % DO NOT CHANGE THIS AND DO NOT ADD ANY OPTIONS TO IT
\frenchspacing  % DO NOT CHANGE THIS
\setlength{\pdfpagewidth}{8.5in} % DO NOT CHANGE THIS
\setlength{\pdfpageheight}{11in} % DO NOT CHANGE THIS
%
% These are recommended to typeset algorithms but not required. See the subsubsection on algorithms. Remove them if you don't have algorithms in your paper.
\usepackage{algorithm}
\usepackage{algorithmic}

% Include other packages here, before hyperref.
\usepackage{algorithm}
\usepackage{algorithmic}
\usepackage{adjustbox} %调整表格大小
\usepackage{booktabs}
\usepackage{diagbox}
\usepackage{multirow}
\usepackage{amsmath}
\usepackage{amsfonts}
\usepackage{amssymb}
\usepackage{makecell}
\usepackage{subfigure}
\usepackage{colortbl}
\usepackage{xcolor}
\usepackage[switch]{lineno}


%
% These are are recommended to typeset listings but not required. See the subsubsection on listing. Remove this block if you don't have listings in your paper.
\usepackage{newfloat}
\usepackage{listings}

\DeclareCaptionStyle{ruled}{labelfont=normalfont,labelsep=colon,strut=off} % DO NOT CHANGE THIS
\lstset{%
	basicstyle={\footnotesize\ttfamily},% footnotesize acceptable for monospace
	numbers=left,numberstyle=\footnotesize,xleftmargin=2em,% show line numbers, remove this entire line if you don't want the numbers.
	aboveskip=0pt,belowskip=0pt,%
	showstringspaces=false,tabsize=2,breaklines=true}
\floatstyle{ruled}
\newfloat{listing}{tb}{lst}{}
\floatname{listing}{Listing}
%
% Keep the \pdfinfo as shown here. There's no need
% for you to add the /Title and /Author tags.
\pdfinfo{
/TemplateVersion (2024.1)
}


\setcounter{secnumdepth}{0} %May be changed to 1 or 2 if section numbers are desired.


\title{Exploring Sparse Visual Prompt for Domain Adaptive Dense Prediction}

\author{Senqiao Yang\textsuperscript{\rm 1,2*}, 
Jiarui Wu\textsuperscript{\rm 1,3*},
Jiaming Liu \textsuperscript{\rm 1}\thanks{Equal contribution: liujiaming@bupt.cn}, 
 Xiaoqi Li\textsuperscript{\rm 1},
Qizhe Zhang\textsuperscript{\rm 1},\\
Mingjie Pan\textsuperscript{\rm 1},
Yulu Gan\textsuperscript{\rm 1},
Zehui Chen \textsuperscript{\rm 4},
Shanghang Zhang\textsuperscript{\rm 1}\thanks{Corresponding author: shzhang.pku@gmail.com}\\
\textsuperscript{\rm 1}Peking University, \textsuperscript{\rm 2}Harbin Institute of Technology, Shenzhen,\\ \textsuperscript{\rm 3} Beihang University, \textsuperscript{\rm 4} University of Science and Technology of China
}

% REMOVE THIS: bibentry
% This is only needed to show inline citations in the guidelines document. You should not need it and can safely delete it.
\usepackage{bibentry}
% END REMOVE bibentry
\usepackage{hyperref}
\begin{document}

\maketitle

\begin{abstract}

The visual prompts have provided an efficient manner in addressing visual cross-domain problems. In previous works, \cite{gan2022decorate} first introduces domain prompts to tackle the classification Test-Time Adaptation (TTA) problem by warping image-level prompts on the input and fine-tuning prompts for each target domain. However, since the image-level prompts mask out continuous spatial details in the prompt-allocated region, it will suffer from inaccurate contextual information and limited domain knowledge extraction, particularly when dealing with dense prediction TTA problems. To overcome these challenges, we propose a novel Sparse Visual Domain Prompts (SVDP) approach, which holds minimal trainable parameters (e.g., 0.1\%) in the image-level prompt and reserves more spatial information of the input. To better apply SVDP in extracting domain-specific knowledge, we introduce the Domain Prompt Placement (DPP) method to adaptively allocates trainable parameters of SVDP on the pixels with large distribution shifts. Furthermore, recognizing that each target domain sample exhibits a unique domain shift, we design Domain Prompt Updating (DPU) strategy to optimize prompt parameters differently for each sample, facilitating efficient adaptation to the target domain. Extensive experiments were conducted on widely-used TTA and continual TTA benchmarks, and our proposed method achieves state-of-the-art performance in both semantic segmentation and depth estimation tasks. \href{https://github.com/ICCV2595/SVDP}{Code link.}
\end{abstract}

\section{Introduction}

\begin{figure}[t]
\includegraphics[width=0.48\textwidth]{./images/Intro_V11.pdf}
\vspace{-0.42cm}
\centering
\caption{
\textbf{The motivation and main idea of our method.} 
\textcolor{red}{(a)} Previous dense visual domain prompts (VDP) mask out consecutive spatial details in the warped regions as shown in red circles. In dense prediction DA problems, applying dense VDP will lead to inaccurate context information extraction and severe performance degradation. \textcolor{red}{(b)} We introduce Sparse Visual Domain Prompts (SVDP), which are tailored for addressing the occlusion problem of pixel-wise information and can better extract local domain knowledge for cross-domain learning. Though the parameters of SVDP are less than VDP, SVDP achieves better semantic segmentation performance in the Test Time Adaptation.}
\label{fig:intro}
\vspace{-0.7cm}
\end{figure}



Deep neural networks can achieve promising performance if test data is of the same distribution as the training data. However, it is not the common case in real-world scenarios \cite{Radosavovic2022}, which contain diverse and disparate domains. When applying a pre-trained model in real-world tasks, the domain gap commonly exists \cite{sakaridis2021acdc}, leading to significant performance degradation on target data. Though we can manually collect labeled data for each real-world target domain, it is laborious and time-consuming~\cite{chen2022multi}. To this end, the domain adaptation (DA) methods are introduced and have drawn growing attention in the community.



While DA extensively investigates to address distribution shifts, its typical assumption involves access to raw source data. However, in real-world scenarios, raw data often cannot be publicly accessible due to data protection regulations. Meanwhile, traditional DA methods present resource-intensive backward computation, leading to high training costs \cite{ganin2015unsupervised}. To address this, Test-time adaptation (TTA) \cite{liang2023comprehensive} is gained significant attention, which tackles distribution shifts at test time with only unlabeled test data streams. Prior TTA studies \cite{wang2020tent,Wangetal2022, chen2022contrastive, goyal2022test} predominantly focus on model-based adaptation, utilizing model parameters to fit target domain knowledge.

To better solve the TTA problem, motivated by the recent advances of prompting in NLP \cite{li2021prefix, liu2023pre}, VDP \cite{gan2022decorate} first introduces a prompt-based method to tackle the classification TTA problem. It employs image-level prompts to enhance domain transfer efficiency and effectiveness. Specifically, it randomly warps the dense prompt on the input image and fine-tunes them to extract target domain knowledge. However, this prompt-based technique encounters limitations when applied to dense prediction tasks such as semantic segmentation and depth estimation TTA. Specifically, the dense prompts obscure continuous spatial information in the allocated regions, as illustrated in Figure 1 (a). This occlusion introduced by prompts leads to incomplete semantic knowledge representation, thereby negatively impacting the quality of segmentation maps. Simultaneously, the occluded details within corresponding features impede the extraction of adequate domain knowledge during cross-domain learning.

To this end, as shown in Fig.1 (b), we propose a novel Sparse Visual Domain Prompts (SVDP) approach for effectively extracting target domain knowledge, specially designed to combat domain shifts in dense prediction tasks. By introducing sparse prompts, which entail minimal trainable parameters (e.g., 0.1\%) within the image-level prompt, more spatial information from the input is preserved. Furthermore, the semantic information can be extracted sufficiently (shown in line 2), leading to noticeable improvements in segmentation outcomes (shown in line 3). 
In order to better apply SVDP in the pixel-wise TTA task, we propose the Domain Prompt Placement (DPP) to adaptively allocates trainable parameters of SVDP on the pixel with large distribution shifts. In this way, SVDP excels at extracting local domain knowledge, facilitating the transfer of pixel-wise data distribution from the source to the target domain. Furthermore, recognizing that each target domain sample exhibits a unique domain shift, we design a Domain Prompt Updating (DPU) method to optimize prompt parameters efficiently during the TTA process. Specifically, based on the extent of the domain gap observed in target domain samples, we employ varying weights to update the visual prompts.
It's worth noting that we are the pioneers in designing specific strategies for pixel-level placement and image-level optimization in vision prompt learning, which work in synergy to address domain shifts in dense prediction TTA tasks. 



Since data privacy and transmission limit access to source data in the real world, we evaluate our method on semantic segmentation and depth estimation source-free adaptation settings, including online TTA~\cite{Liangetal2020} and Continual Test-Time Adaptation~\cite{Wangetal2022} (CTTA). Our proposed approach demonstrates superior performance compared to most state-of-the-art (SOTA) methods across three benchmarks, covering Cityscapes to ACDC\cite{sakaridis2021acdc} and KITTI \cite{geiger2012we} to Drivingstereo \cite{yang2019drivingstereo}. 
The main contributions are shown as follows:

1) We are the first for introducing the visual prompt approach to the dense prediction TTA problem. We propose a novel Sparse Visual Domain Prompts (SVDP) approach to better extract local domain knowledge and transfer pixel-wise data distribution from the source to the target domain.

2) In order to efficiently apply SVDP in pixel-wise TTA tasks, we propose Domain Prompt Placement~(DPP) method to adaptively allocates trainable parameters in SVDP based on the degree of distribution shift at the pixel level. And Domain Prompt Updating~(DPU) is designed to optimize prompt parameters differently for each sample, facilitating efficient adaptation on target domains.

3) We conduct extensive experiments to evaluate the effectiveness of our method, which outperforms most SOTA methods on four TTA and two CTTA benchmarks, covering semantic segmentation and depth estimation tasks.


 
\section{Related Work}
\subsection{Test-time adaptation}        
\textbf{Test-time adaptation (TTA)}, also known as source-free domain adaptation~\cite{Boudiafetal2023, Kunduetal2022, Liangetal2020, ShiqiYangetal2021}, aims to adapt a source model to an unknown target domain distribution without using any source domain data. 
% In many real-world scenarios, data privacy and transmission cost limit access to source domain data, resulting in many traditional domain adaptation (DA) algorithms being inapplicable. 
Recent research has focused on using self-training or entropy regularization to fine-tune the source model. Specifically, 
SHOT~\cite{Liangetal2020} optimizes only the feature extractor using information maximization and pseudo labeling. AdaContrast~\cite{Chenetal2022} also uses pseudo labeling for TTA, but introduces self-supervised contrastive learning to enhance performance. In addition to model-level adaptation, ~\cite{Boudiafetal2022} adjusts the output distribution to address this problem. Tent~\cite{DequanWangetal2021} updates the training parameters in the batch normalization layers by entropy minimization. Recent works \cite{niu2023towards,
yuan2023robust} follow Tent to continually explore the robustness of normalization layers in the TTA process.
While the aforementioned works primarily focus on classification tasks, there has been a recent surge of interest in performing TTA on dense prediction tasks~\cite{Shinetal022, Songetal2022, Zhangetal2021}. 
\textbf{Continual Test-Time Adaptation (CTTA)} is a scenario in which the target domain is not static, increasing challenges for traditional TTA methods ~\cite{Wangetal2022}. ~\cite{Wangetal2022} serves as the first approach to tackle this task, using a combination of bi-average pseudo labels and stochastic weight reset. While ~\cite{Wangetal2022,song2023ecotta} addresses the continual shifts at the model level, ~\cite{gan2022decorate} leverages visual domain prompts to address the problem in the classification task at the input level for the first time. In this paper, we evaluate our approach on both TTA and CTTA with a specific focus on the dense prediction task.



\subsection{Prompt learning}
\textbf{Visual prompts} are inspired by their counterparts~\cite{PengfeiLiu2021PretrainPA} which are used in natural language processing (NLP). Language prompts are presented as text instructions to improve the pre-trained language model's understanding of downstream tasks~\cite{Brownetal2020}.
% Besides, prompt has also been widely applied to vision-language models~\cite{Juetal2021, Radfordetal2021, Yaoetal2021, zhouetal2022cocoop, zhouetal2022coop}. 
Recently, researchers have attempted to discard text encoders and use prompts directly for visual tasks. ~\cite{Bahngetal2022} employs visual prompts to pad input images, enabling pre-trained models to adapt to new tasks. Rather than fine-tuning the entire model, VPT~\cite{Conderetal2022, MenglinJia2022VisualPT, Sandleretal2022, ZifengWangetal2021} inserts prompts into image or feature-level patches to adapt Transformer-based models. While these approaches all utilize opaque-block prompts, such prompts can cause performance degradation in dense prediction tasks. 
% Therefore, we propose the use of sparse prompts for the first time to address semantic segmentation.
\textbf{Domain prompts} are first introduced in DAPL~\cite{Geetal2022}, which proposes a novel prompt learning paradigm for unsupervised domain adaptation (UDA). Embedding domain information using prompts can minimize the cost of fine-tuning and enable efficient domain adaptation. Recognizing the potential of prompt learning for UDA, MPA~\cite{chen2022multi} proposes multi-prompt alignment for multi-source UDA. DePT~\cite{Gaoetal2022} combines domain prompts with a hierarchical self-supervised regularization for TTA, which aims to solve the error accumulation problem in self-training. ~\cite{gan2022decorate} further divides domain prompts into domain-specific ones and domain-agnostic ones to address the more challenging CTTA task. However, these studies mainly focus on simple classification DA tasks. Our method, for the first time, applies sparse domain prompts to dense prediction DA tasks. Besides, we are the first to design specific placement and updating strategies for the domain prompt method, which help to jointly ease the domain shift.


\begin{figure*}[ht]
\centering
\includegraphics[width=\linewidth]{images/Framework_V25.pdf}
\vspace{-0.45cm}
\caption{\textbf{The overall framework.} \textbf{Left:} We warp the SVDP into the image and place prompt parameters on the selected pixel by Domain Prompt Placement (DPP) method. The reformulated image serves as the input of the teacher and student model.  
We obtain the uncertainty map as described in Eq. \ref{eq:mc} through the teacher model. The uncertainty map is used to evaluate the degree of pixel-level distribution shift. 
SVDP adopts consistency loss (Eq. \ref{eq:loss}) and exponential moving average (EMA) as the optimization strategies.
\textbf{Right:} Domain Prompt Updating (DPU). Based on the image-level uncertainty value, we adopt different EMA weights to realize stable updating of SVDP parameters, facilitating efficient adaptation to the target domain.}
\label{fig:framework} 
\vspace{-0.3cm}
\end{figure*}






\section{Method}

\subsection{Preliminaries}

\textbf{Test Time Adaptation (TTA)} ~\cite{liang2023comprehensive} aims at adapting a pre-trained model with parameters trained on the source data $(\mathcal{X}_S$, $\mathcal{Y}_S)$ to multiple unlabeled target data distribution $\mathcal{X}_{T_1},\mathcal{X}_{T_2}, \dots, \mathcal{X}_{T_n}$ at inference time. The entire process can not access any source domain data and can only access target domain data once. $\mathcal{X}_{T_i} = \{x_i^T\}_{i=1}^{N_t}$, where $N_t$ denotes the scale of the target domain. The upcoming target domain can be a single one (TTA) or multiple continually changing distributions (CTTA), the latter of which is a more realistic setting that requires the model to achieve stability while preserving plasticity. 

\textbf{Domain Prompt.} Inspired by language prompt in NLP, ~\cite{gan2022decorate} first introduces visual domain prompt (VDP) 
serving as a reminder to continually adapt to the target domain for the classification task, which aims to extract target domain-specific knowledge. Specifically, VDP ($\textbf{p}$) are dense learnable parameters added on the input image. 
\begin{equation}
\widetilde {\textbf{x}} = \textbf{x} + \textbf{p}
\label{eq:dense}
\end{equation}
where x represent the original input image. The reformulated image ${\widetilde {\textbf{x}}}$ will serve as the input for our model instead. 


\subsection{Motivation}
\label{sec:3.2}

\textbf{Sparse Visual Domain Prompt.} Traditional visual prompts \cite{jia2022visual} are deployed on the image or feature level to realize fine-tuning by updating a small number of prompt parameters. Recent works \cite{gan2022decorate, gao2022visual} explore visual prompts in classification DA problems, which extract domain knowledge for the target domain and transfer data distribution from the source to the target domain. 
However, DePT \cite{gao2022visual} concatenate the domain prompts with class token and image tokens to the input of transformer layers, which neglect the local domain knowledge extraction.
Meanwhile, VDP \cite{gan2022decorate} randomly set the locations of dense prompts on the input image, masking out continuous spatial details in prompt allocated regions. 
Different from classification cross-domain learning, dense prediction DA not only requires global domain knowledge but also relies on extracting intact local domain knowledge.  
As shown in Fig.\ref{fig:intro}(a), partial spatial information deficiency caused by dense prompts will lead to inaccurate contextual information and negative effects on target domain knowledge extraction. 
This observation motivates us to propose a novel Sparse Visual Domain Prompts (SVDP), which is tailored for pixel-wise prediction DA tasks. It holds minimal trainable parameters (e.g. 0.1\%) in the image-level prompt and reserves more spatial information of the input.

\textbf{Domain Prompt Placement.} Previous work~\cite{gan2022decorate,gao2022visual} randomly put the prompts on the target domain image to extract global domain knowledge. Specifically, it may set prompts on regions with trivial domain shifts, hindering the extraction of local domain knowledge. Especially in the source-free TTA setting, we can only access target domain data once, which makes the efficiency of transfer learning crucial. Therefore, we propose Domain Prompt Placement (DPP) which efficiently extracts more domain-specific knowledge and addresses local domain shift. Specifically, we measure the degree of domain gap by general uncertainty scheme~\cite{gal2016dropout, guan2021uncertainty, roy2022uncertainty,gan2022cloud} and tactfully place trainable parameters of SVDP on the pixel with large distribution shifts.


\textbf{Domain Prompt Updating.}
The amount of prompt parameters is minimal which brings the challenge of fully learning target domain knowledge during TTA process. Meanwhile, the degree of domain shift is not only various on pixels within the image but also on each target domain test sample. It thus motivates us to update prompt parameters differently for each target sample. Therefore, we design a Domain Prompt Updating (DPU) which efficiently optimizes prompt parameters to fit in target domain distribution. Specifically, we adopt the same uncertainty scheme to measure the degree of domain shift for each target sample.
According to the degree, we update prompt parameters for the individual sample with different updating weights.



\subsection{Sparse visual domain prompt} 
\label{sec:3.3}
SVDP maintains the same resolution as the input image ($\textbf{p} \in \mathbb{R}^{H \times W \times 3}$), it only masks out original information by minimal discrete trainable parameters (e.g. $0.1\%$) on the pixels with large domain shifts. Compared with the previous dense visual prompt, SVDP preserves more contextual information and possesses the capacity to capture local domain knowledge through pixel-wise prompt parameters. The overall framework of our method is shown in Fig .\ref{fig:framework}, and the specially designed prompt Placement and Updating methods are introduced in the following.




\subsection{Domain prompt placing}
\label{sec:3.4}

We propose the Domain Prompt Placement (DPP) strategy of SVDP to efficiently extract local domain knowledge in pixel-wise. We intend to place trainable parameters of SVDP on the
pixel with large distribution shifts and adapt pixel-wise data distribution from source to target domain. 
Though the confidence score is a straightforward measurement to reflect the reliability of prediction, it is trusting less and fluctuates irregularly in pixel-wise cross-domain scenarios. 
As shown in the top of Fig .\ref{fig:DPP}, we thus adopt Dropout method \cite{gal2016dropout} to realize $m$ times forward propagation and obtain $m$ group probabilities for each pixel.
Inspired by \cite{roy2022uncertainty,gan2022cloud}, we calculate the uncertainty value (Eq.\ref{eq:mc}) of the input and figure out the pixel-wise degree of domain shift.
% .Based on the variance of $m$ category of probabilities $p_i(y|x)$ of the model by MC dropout. 
\begin{equation}
\mathcal{U} (\widetilde{x}_j) =  \left( \frac{1}{m} \sum_{i=1}^m \|p_i(\widetilde {y_j}|\widetilde {x_j}) - \mu \|^2 \right) ^{\frac{1}{2}}
\label{eq:mc}
\end{equation}
, where $p_i(\widetilde{y_j}|\widetilde{x_j})$ is the predicted probability of input pixel $\widetilde{x_j}$ in the $i^{th}$ forward propagation, and $\mu$ is the mean prediction ($m$ rounds) of $\widetilde{x_j}$. $\mathcal{U} (\widetilde{x_j})$ thus represents the uncertainty of the source model for pixel-wise target input $\widetilde{x_j}$. As shown in the bottom of Fig .\ref{fig:DPP}, we sort all pixels based on their pixel-wise uncertainty value and place prompt parameters on the pixels with high uncertainty score. 




\begin{figure}[t]
\includegraphics[width=0.45\textwidth]{./images/DPP_V7.pdf}
\centering
\vspace{-0.18cm}
\caption{The detailed process of Domain Prompt Placing. The uncertainty map is estimated by MC Dropout \cite{gal2016dropout}. The SVDP parameters are placed on the pixels with high uncertainty, then warp into the raw image.}
\label{fig:DPP}
\vspace{-0.35cm}
\end{figure}







\subsection{Domain prompt updating}
\label{sec:3.5}
Motivated by the fact that the mean teacher predictions have a higher quality than the standard model \cite{tarvainen2017mean}, we utilize a mean-teacher model to provide more accurate predictions in the TTA process. To be specific, we adopt the widely-used exponential moving average (EMA) to achieve the model and prompt updating. Same as previous works\cite{Wangetal2022}, the teacher model ($\mathcal{T}_{mean}$) is updated by EMA from the student model ($\mathcal{S}_{target}$), shown in Eq. \ref{eq:ema}:
\begin{equation}
 \mathcal{T}_{mean}^{t} = \alpha \mathcal{T}_{mean}^{t-1} + (1-\alpha) \mathcal{S}_{target} ^{t}
\label{eq:ema}
\end{equation}
When $t = 0$ ($t$ is the time step), we utilize the source domain pre-trained model to initialize the weight of the teacher and student model. And we set $\alpha = 0.999$ \cite{AnttiTarvainenetal2017}, which is the updating weight of EMA. 

Different from traditional model updating, we design a special Domain Prompt Updating (DPU) strategy for SVDP to stably fit in target domain distribution. As shown in Fig .\ref{fig:DPU}, we adopt image-level uncertainty value to reflect the degree of domain shift for each target domain sample. 
We calculate the image-level uncertainty value $\mathcal{U}(x)$ by average the pixel-wise uncertainty, shown in Eq. \ref{eq:unc_for_promptregion}:
\begin{equation}
\mathcal{U}(x) = \frac{1}{H \times W}
\sum_{j }^{H \times W} \mathcal{U} (\widetilde{x}_j)
\label{eq:unc_for_promptregion}
\vspace{-0.15cm}
\end{equation}
Based on the image-level uncertainty score, we update prompt parameters for each sample with different weight. 

\begin{equation}
\label{unc_ema_prompt}
\begin{aligned}
p_{t} = \beta p_{t-1} + (1-\beta) p_{t},
\end{aligned}
\end{equation}
Note that, $p_{t}$ represents the parameters of the SVDP that need to be updated. In DPU, we set the prompt EMA updating rate $\beta = 1-(\mathcal{U}(x) \times \theta)$. $\theta$ is intended to bring the value of uncertainty up to the same order of magnitude as the value of the common EMA update parameter~\cite{AnttiTarvainenetal2017}.
As shown in the top of Fig .\ref{fig:DPU}, the prompt EMA weight is set to a large value when the input is of high uncertainty score since the large weight can efficiently adapt to the sample with the large data distribution shift.


\begin{figure}[t]
\includegraphics[width=0.45\textwidth]{./images/DPU_V6.pdf}
\centering
\vspace{-0.25cm}
\caption{The process of Domain Prompt Updating. We adaptively adjust the prompt EMA updating rate for each target domain sample based on image-level uncertainty value.}
\label{fig:DPU}
\vspace{-0.3cm}
\end{figure}


\subsection{Loss function}
We utilize teacher model to generate the pseudo labels ($\widetilde{y}_t$), which is refined by test-time augmentation and confidence filter ~\cite{Wangetal2022}.
Then, we adopt consistency loss ($L_{con}$) as the optimization objective for segmentation task, which is a pixel-wise cross-entropy loss \cite{xie2021segformer}.
\begin{equation}
 \mathcal{L}_{con}(\widetilde {x}) = -
 \frac{1}{H \times W} \sum_{w,h}^{W,H}
 \sum_c^C \widetilde{y}_t(w,h,c) \log \hat{y}_t(w,h,c)
\label{eq:loss}
\end{equation}
Where $\hat{y}_t$ is the output of our student model, $C$ means the amount of categories. The loss function of depth estimation is shown in the supplement.








%%%%%%%%%%%%%%%%%%%%%%%%%%%%%%%
\section{Experiments}

\section{Experimental Results}
\label{sec:experiments}
\subsection{Training Details}
\cite{Kalantari2017DeepHD} provides the first dataset specifically designed for multi-exposure HDR fusion under large motion. It consists of 74 training sets, which we use to supervise the training of our model. We crop the input images to patches of size \(256 \times 256\) at a step size of 64. This totally generates 20128 training samples. To augment training samples, we randomly rotate and flip the training images. The training adopts Adam optimizer. The learning rate is initialized to \(10^{-4}\) and is reduced to \(10^{-5}\) after 20 epochs. It is observed that 40 epochs are sufficient for the training to converge.    

\subsection{Numerical Evaluation}
We numerically measure the performance of our method on the 15 test sets of \cite{Kalantari2017DeepHD}, by Peak Signal-to-Noise Ratio (PSNR) and Structure Similarity, computed in both tonemapping domain (-\(\mu\)) and HDR linear domain (-L). Visual difference metric HDR-VDP-2 is also adopted, where the parameters are set as same as in previous works \cite{wu2018end} and \cite{niu2021hdrgan}. 

Table \ref{table_metrics} compares our model with state-of-the-art models. For \cite{yan2020nonlocal} and \cite{xiong2021hierarchical}, we use the results reported in the publications. Note that \cite{sen2012robust} and \cite{hu2013hdr} are not machine learning based methods. Moreover,  \cite{Kalantari2017DeepHD} and \cite{wu2018end} apply optical flow and homography transformation to preprocess the input images respectively, and hence entail extra computation. 

Table \ref{table_metrics} shows that our method outperforms competing method in terms of PSNR-L, SSIM-$\mu$, SSIM-L and HDR-VDP-2. It ranks the second best in PSNR-$\mu$, being slightly (0.1dB) inferior to \cite{xiong2021hierarchical}. Note that \cite{xiong2021hierarchical} utilizes a pretrained model to detect ghosting regions for training, whereas our method does not require any pretrained model. The high PSNR and SSIM scores varify that our model has strong HDR reconstruction ability and can accurately restore the radiance and structure of the scene in both tonemapping domain and HDR linear domain. Furthermore, its high performance in term of HDR-VDP-2\cite{mantiuk2011hdr} performance indicates that our method can generate HDR image visually close to the target image.

\begin{table*}[ht]
\centering
\begin{tabular}{l|c|c|c|c|c}
\hline
& PSNR-$\mu$ & PSNR-L & SSIM-$\mu$ & SSIM-L & HDR-VDP-2 \\
\hline
\bfseries Sen & 40.97 & 38.36 & 0.9830 & 0.9746 & 60.60\\
\hline
\bfseries Hu  & 35.65 & 30.80 & 0.9725 & 0.9491 & 58.34\\
\hline
\bfseries Kalantari & 42.69 & 41.22 & 0.9888 & 0.9845 & 65.05\\
\hline
\bfseries DeepHDR& 41.99 & 41.22 & 0.9878 & 0.9859 & \underline{65.91}\\
\hline
\bfseries AHDR & 43.62 & 41.03 & 0.9900  &\underline{0.9883} & 63.85 \\
\hline 
\bfseries NHDRRNet& 42.414 & - & 0.9887 & - & 61.21 \\
\hline 
\bfseries HDR-GAN &43.92 & \underline{41.57} &\underline{0.9905} &0.9865 & 65.45\\
\hline 
\bfseries HFNet & \textbf{44.28} & 41.47 & - & - & - \\
\hline 
\bfseries Ours & \underline{44.18} & \textbf{42.19}&\textbf{0.9912} & \textbf{0.9883}& \textbf{67.07} \\
\hline
\end{tabular}
\caption{Numerical performance of the proposed model, evaluated on the dataset by Kalantari-Ramamoorthi. The best and second best results for each metric are marked in \textbf{bold} and \underline{underlined}, respectively}
\label{table_metrics}
\end{table*}

\subsection{Visual Performance Evaluation}

\begin{figure*}[!htb]
\centering
\includegraphics[width=\textwidth]{experiments/kalantari_test.png}
\caption{Visual comparison on the test set of Kalantari-Ramamoorthi dataset. Zoom-in views of reconstruction by each method are presented on the saturated regions that contain moving objects. Our network built with gated Swin Transformer yields noticeably better visual results than other methods.}
\label{fig_kalantari_test}
\end{figure*}
Fig. \ref{fig_kalantari_test} present the visual performance of our method and comparable methods on two examples from \cite{Kalantari2017DeepHD}. We present the zoom-in views of two challenging cases, where large saturated regions contain substantial non-rigid motion in the reference image. The two patch-based methods do not reconstruct the missing details in the saturated regions, as they heavily rely on the details provided by the reference image for registration. Image reconstructed by the optical flow based method \cite{Kalantari2017DeepHD} suffers motion blur artifacts. This is because the convolutions of DeepHDR and HDR-GAN have limited receptive fields, and hence are hampered to repair missing content in misaligned regions by aligned regions. The gating mechanism of AHDR is only applied to low-level features, so the high-level outliers may deteriorate the HDR fusion. In contrast to comparable methods, our model remarkably overcomes the ghosting artifacts.

\begin{figure}[ht]
\centering
\includegraphics[width=\columnwidth]{experiments/sen_test.pdf}
\caption{Visual performance comparison on example images from the dataset by Sen et al. Zoom in views on challenging areas are presented. Although the ground truth is unavailable, it can be clearly observed that our method visually performs better than comparable methods.}
\label{sen_test}
\end{figure}

\begin{figure}[ht]
\centering
\includegraphics[width=\columnwidth]{experiments/tursun_test.pdf}
\caption{Visual performance comparison on example images from the dataset by Tursun et al. Compared to state of the art methods, our method suffers less ghosting artifact.}
\label{tursun_test}
\end{figure}

Fig.\ref{sen_test} and Fig.\ref{tursun_test} present visual performance of our method on two examples from benchmark datasets \cite{sen2012robust} and \cite{tursun2016objective}. As these test datasets   do not provide ground truth image. we mark the visual difference on the results generated by different methods. It can be seen that our method suffers less artifacts than other methods in various scenes with various motion patterns, achieving better visual results. Our method creates high-quality HDR more robustly and generalizes well. 

\subsection{Ablation Study}

\begin{table}[h]
\centering
\resizebox{\columnwidth}{!}{
\begin{tabular}{l|c|c|c|c|c}
\hline
                         & PSNR-$\mu$ & PSNR-l & SSIM-$\mu$ & SSIM-l & HDR-VDP-2 \\ \hline
restormer(w/o ssim loss) & 44.00  & 41.5   & 0.9906 & 0.9873 & 64.72  \\ \hline
Ours(w/o ssim loss)      & 44.07  & 41.83  & 0.9909 & 0.9879 &  64.78  \\ \hline
Ours                     & 44.18  & 42.19  & 0.9912 & 0.9883 & 67.07      \\ \hline
\end{tabular}
}
\caption{Experimental results of ablation study. We compare using Gated Swin Transformer v.s. Gated Transformer, and the combined loss function v.s. the traditional $l_{1}$ norm loss function.}
\label{table_ablation_block_loss}
\end{table}

We verify various components of our method, including Swin Transformer, loss function, and gating mechanism by ablation study.

\subsubsection{Ablation Study on Block Design}
Our model has similar architecture to Restormer, which uses modified Transformer, whereas we use modified Swin Transformer as the building unit. For comparison, we replace the residual modules in each block in our model with multiple transformer layers as in Restormer, with same number of transformer layers. Table \ref{table_ablation_block_loss} presents the results, which show that using Swin Transformer achieves superior performance in all measures. The reason is that the attention module of Restormer is computed channel-wise, but forgoes the cross-exposure spatial dependency to repair the non-aligned area. 

\subsubsection{Ablation Study on Loss Function}
We trained our model under different loss function configurations, as shown in \ref{table_ablation_block_loss}. The results validate that the SSIM loss benefits detail reconstruction.

\subsubsection{Ablation Study on Gating Mechanism}
\begin{table}[h]
\resizebox{\columnwidth}{!}{
\begin{tabular}{l|c|c|c|c|c}
\hline
           & PSNR-$\mu$ & PSNR-l & SSIM-$\mu$ & SSIM-l & HDR-VDP-2 \\ \hline
w/o gating & 43.14  & 41.03  & 0.9904 & 0.9868 &     64.88      \\ \hline
one gating & 43.44  & 41.42  & 0.9909 & 0.9882 &    67.13   \\ \hline
Ours       & 43.61  & 41.74  & 0.9909 & 0.9881 & 66.96     \\ \hline
\end{tabular}
}
\caption{Ablation experimental results to verify the effectiveness of the gating mechanism}
\label{table_ablation_gating}
\end{table}

The gating mechanism is an important component in our model. Ablation study is conducted in the gating mechanism as follows.

\textbf{w/o gating}: The gating mechanism is not used in the feed forward network of all transformer layers in the model, that it, our GST unit degenerate to the vanilla Swin Transformer.

\textbf{one gating}: The gating mechanism is only used in the first Swin Transformer layers subsequent to the embedding layer, but not used for other layers. 

 Table \ref{table_ablation_gating} shows the results of the ablation experiments, where the model is trained for 20 epochs. By removing the gating mechanism, the network relies on self-attention for image alignment, resulting in the lowest performance. On top of it, adding gates to low level layers notably improves the HDR reconstruction. Furthermore, by integrating the gating mechanism with all Swin Transformer layers, the model effectively inpaints information in non-aligned regions and obtains the highest HDR reconstruction results, thus validates the effectiveness of the gating mechanism in our model.

% \clearpage









\bibliography{aaai24}

\clearpage
\appendix

%%%%%%%%% ABSTRACT
\begin{figure*}[ht!]
\centering
\includegraphics[width=\linewidth]{./images/visual_V2.pdf}
\vspace{-0.39cm}
\caption{Qualitative comparison of SVDP with previous SOTA method: CoTTA~\cite{Wangetal2022}, VDP~\cite{gan2022decorate} on ACDC Fog, Night, Rain, and Snow four scenarios. SVDP could better segment different pixel-wise classes such as shown in the white box.}
\label{fig:qualitative} 
\vspace{-0.3cm}
\end{figure*}
\section*{Appendix Overview}
\label{sec: ap}

The following items are included in this supplementary material.

\begin{itemize}
    \item Additional Ablation Studies of
    \begin{itemize}
        \item Depth Estimation on TTA scenario
        \item Domain Prompt Updating
    \end{itemize}
    \item Qualitative Analysis of 
    \begin{itemize}
        \item Semantic Segmentation
        \item Depth Estimation
    \end{itemize}
    \item Additional Quantitative Results
    \item Additional Related Work
\end{itemize}

\section{Additional Ablation Studies}
\subsection{Depth Estimation on TTA scenario}
\label{Sec:ablation ctta}
The proposed method comprises a Sparse Visual Prompt (SVDP), a Domain Prompt Placement (DPP) strategy, and a Domain Prompt Updating (DPU) strategy to mitigate domain shifts in semantic segmentation tasks. In this study, we conduct ablation experiments on the TTA scenario (KITTI-to-DrivingStereo Foggy) to evaluate the effectiveness of each component.

To compare the performance of our method with and without using the teacher-student (TS) structure, a common technique in TTA used to generate pseudo labels in the target domain, we present the results in Tab.~\ref{aptab:ablation} $Ex_{2}$. The results show that without our method, TS only has 0.069 Abs Rel reduces, indicating that our method's improvement does not come from the usage of this prevalent scheme.
In $Ex_{3}$, we introduce SVDP to extract local target domain knowledge without damaging the original spatial information. The results demonstrate that SVDP achieves a 21.0\% $\delta>1.25$ improvement and 0.073 Abs Rel reduction, effectively addressing the domain shift problem. In $Ex_{4}$, DPP achieves a further 3.3\% $\delta>1.25$ improvement and 0.014 Abs Rel reduces by serving as a specially designed prompt placement strategy to assist SVDP in extracting more target domain-specific knowledge.
We evaluate the effectiveness of the DPU in $Ex_{5}$, which adaptively optimizes for different samples during testing. Compared with $Ex_{3}$, DPU reduces the Abs Rel 0.006 and improves 2.2\% $\delta>1.25$ respectively. Finally, in $Ex_{6}$, we show the complete combination of all components, which achieves a total of 69.4\% $\delta>1.25$ improvement and 0.161 Abs Rel reduction. These results demonstrate that all components of our method effectively address the depth estimation domain shift and compensate for each other to achieve superior performance.


\begin{table}[!tb]
\caption{\textbf{Ablation: Contribution of each component on KITTI-to-DrivingStereo Foggy. 
}}
\centering
\setlength\tabcolsep{5pt}%调列距
\renewcommand\arraystretch{1}%调行距
\begin{tabular}{c|cccc|cc}
\toprule
 & \makecell*[c]{TS} & \makecell*[c]{SVDP} & \makecell*[c]{DPP} & \makecell*[c]{DPU}  & Abs Rel$\downarrow$ & $\delta>1.25 \uparrow$\\\midrule
$Ex_{1} $ &  & & & &0.313& 0.040 \\ 
$Ex_{2} $& \checkmark &  & &  & 0.244& 0.482\\
$Ex_{3}$ &\checkmark  & \checkmark &  & & 0.171& 0.692\\
$Ex_{4}$  & \checkmark & \checkmark &\checkmark  &  &0.157& 0.725\\
$Ex_{5}$  & \checkmark & \checkmark & &\checkmark & 0.165& 0.714\\
$Ex_{6} $ & \checkmark &  \checkmark &\checkmark  &\checkmark & 0.152& 0.734\\
\bottomrule
\end{tabular}
\vspace{-0.2cm}
\label{aptab:ablation}
\end{table}

\begin{figure}[!h]
\includegraphics[width=0.43\textwidth]{./images/dpu_senstivity_V9.pdf}
\centering
\vspace{-0.05cm}
\caption{Sensitivity Analysis: The effect of prompt EMA's parameter $\beta$ on depth estimation performance in the CTTA scenario.}
\label{fig:senstivity}
\vspace{-0.25cm}
\end{figure}
\subsection{Domain Prompt Updating}
\label{Sec: sentivity}
For SVDP, we utilize Eq.5 to update the prompt parameters and conduct an analysis of the sensitivity of the parameter $\beta$ in the depth estimation CTTA scenario. As depicted in Fig.~\ref{fig:senstivity}, we investigate the impact of $\delta>1.25$ values on the performance. Specifically, we gradually increase the value of $\beta$ and record the corresponding $\delta>1.25$ values. We observe that the $\delta>1.25$ improves with increasing $\beta$; however, it starts to decrease once $\beta$ exceeds 0.999. Compare with the best fixed $\beta$ value, our proposed DPU (\textcolor{red}{red line}) strategy can further achieve 1.5\% $\delta>1.25$ improvement. Therefore, due to the different degrees of domain shift, we need to update prompt parameters for the each sample with different EMA weights.





\section{Qualitative analysis}
\subsection{Semantic Segmentation}
\label{Sec:qualitative}
To further demonstrate the effectiveness of our proposed method, SVDP, we conduct a qualitative comparison with two current leading methods, CoTTA \cite{Wangetal2022} and VDP \cite{gan2022decorate}, on the CTTA scenario (Cityscapes-to-ACDC). 


The results of the comparison are presented in Fig .\ref{fig:qualitative}. 
In the foggy target-domain, we highlight a white box that contains a tall truck object. This object is difficult to segment as it shares characteristics with the \emph{sign} class. Thanks to the contribution of the Domain Prompt Placement (DPP), our proposed method, SVDP, has a significant advantage in dealing with such confusing semantic segmentation categories with high uncertainty. Our method also outperforms CoTTA and VDP in the remaining three domains. In these domains, our proposed method correctly distinguishes the sidewalk from the road, avoiding misclassification.
Overall, our method can achieve better local segmentation results and neglect the influence of local domain shift.
And our method produces finer results than the previous state-of-the-art methods, with clear visual improvement.

\subsection{Depth estimation}
To demonstrate the generalizability of our proposed method SVDP, we compare it with the current SOTA method on the Depth Estimation task, and the results of the qualitative analysis are shown in Fig.~\ref{fig:qualitative_depth}

Compared with the VDP and CoTTA, We notice that our SVDP can obviously improves the semantic representation of the model, which concentrates more on the foreground object. 
The results verify that our methods can extract the target domain knowledge during the test time adaptation. 
Furthermore, our SVDP enhances the model's depth estimation ability in the regions of edges, sharpness, and long distances.

\begin{figure*}[ht!]
\centering
\includegraphics[width=\linewidth]{./images/visual_depth_v2.pdf}
\vspace{-0.39cm}
\caption{Qualitative comparison of SVDP with previous SOTA method: CoTTA\cite{Wangetal2022}, VDP\cite{gan2022decorate} on Driving Stero  Cloudy, Foggy, Rainy, and Sunny four scenarios. SVDP could make the depth estimation better such as shown in the white circle.}
\label{fig:qualitative_depth} 
\vspace{-0.3cm}
\end{figure*}
\begin{table*}
\caption{Performance Comparison for \textbf{Cityscapes-to-ACDC Fog domain in TTA scenario}. The IoU score of each class and the mIoU score are reported. The best results are highlighted in \textbf{bold}.
  } \label{table:fog acdc}
  \centering
  \resizebox{0.99\textwidth}{!}{
    \def\arraystretch{1.1}
    \begin{tabular}{ l | c c c c c c c c c c c c c c c c c c c | c }
        \Xhline{1.2pt}
        Method & \rotatebox{0}{road} & \rotatebox{0}{side.} & \rotatebox{0}{buil.} & \rotatebox{0}{wall} & \rotatebox{0}{fence} & \rotatebox{0}{pole} & \rotatebox{0}{light} & \rotatebox{0}{sign} & \rotatebox{0}{veg.} & \rotatebox{0}{terr.} & \rotatebox{0}{sky} & \rotatebox{0}{pers.} & \rotatebox{0}{rider} & \rotatebox{0}{car}& \rotatebox{0}{truck} & \rotatebox{0}{bus} & \rotatebox{0}{train} & \rotatebox{0}{mbike} & \rotatebox{0}{bike} & mIoU \\
        \hline
        \hline
        Source  & 94.0 & 63.9 & 79.8 & 55.7 & 24.9 & 45.0 & 41.5 & 69.8 & 86.6 & 71.0 & 97.6 & 64.1 & 66.2 & 87.4 & 73.0 & 92.6 & 87.7 & 50.2 & 61.7 & 69.1 \\
        TENT & 94.0 & 64.0 & 79.7 & 55.4 & 24.6 & 44.6 & 41.4 & 69.9 & 86.7 & 71.1 & 97.6 & 64.0 & 65.9 & 87.4 & 73.0 & 92.6 & 88.0 & 50.2 & 61.9 & 69.0 \\
        CoTTA & 93.9 & 63.6 & 80.0 & 55.5 & 25.1 & 49.0 & 43.4 & 73.0 & 87.0 & 70.7 & 97.8 & 68.6 & 71.3 & 87.1 & 74.8 & 93.6 & 89.1& 58.0 & 66.7 & 70.9 \\
        DePT & 94.0 & 64.0 & 79.9 & 56.1 & 25.3 & 48.8 & 43.5 & 73.0 & 87.1 & 70.6 & 97.5 & 67.9 & \textbf{71.5} & 87.3 & 75.1 & 93.5 & 89.1& 57.4 & 66.5 & 71.0 \\
        VDP & 93.9 & 63.6& 80.0 & 55.6 & 25.1& 49.0& 43.4 & 73.0& 86.9 & 70.7 & 97.7 & 68.5 & 71.1 & 87.2 & 74.7 & 93.5 & 89.2 & 57.9 & 66.6 & 70.9 \\
        \cellcolor{lightgray}\textbf{SVDP} &\cellcolor{lightgray}\textbf{94.4} &\cellcolor{lightgray}\textbf{65.9}&\cellcolor{lightgray}\textbf{80.5}&\cellcolor{lightgray}\textbf{57.8}&\cellcolor{lightgray}\textbf{26.5}&\cellcolor{lightgray}\textbf{50.5}&\cellcolor{lightgray}\textbf{43.9}&\cellcolor{lightgray}\textbf{73.6}&\cellcolor{lightgray}\textbf{87.6}&\cellcolor{lightgray}\textbf{72.0}&\cellcolor{lightgray}\textbf{98.0}&\cellcolor{lightgray}\textbf{68.8}&\cellcolor{lightgray}71.3 
 &\cellcolor{lightgray}\textbf{87.7}&\cellcolor{lightgray}\textbf{77.6}&\cellcolor{lightgray}\textbf{94.4}&\cellcolor{lightgray}\textbf{92.6}&\cellcolor{lightgray}\textbf{60.0}      &\cellcolor{lightgray}\textbf{67.4} 
 &\cellcolor{lightgray}\textbf{72.1}\\\bottomrule


       
        
        \Xhline{1.2pt}
    \end{tabular}
  }

\end{table*}

\begin{table*}
\caption{Performance Comparison for \textbf{Cityscapes-to-ACDC Night domain in TTA scenario}. The IoU score of each class and the mIoU score are reported. The best results are highlighted in \textbf{bold}.
  } 
  \centering
  \resizebox{0.99\textwidth}{!}{
    \def\arraystretch{1.1}
    \begin{tabular}{ l | c c c c c c c c c c c c c c c c c c c | c }
        \Xhline{1.2pt}
        Method & \rotatebox{0}{road} & \rotatebox{0}{side.} & \rotatebox{0}{buil.} & \rotatebox{0}{wall} & \rotatebox{0}{fence} & \rotatebox{0}{pole} & \rotatebox{0}{light} & \rotatebox{0}{sign} & \rotatebox{0}{veg.} & \rotatebox{0}{terr.} & \rotatebox{0}{sky} & \rotatebox{0}{pers.} & \rotatebox{0}{rider} & \rotatebox{0}{car}& \rotatebox{0}{truck} & \rotatebox{0}{bus} & \rotatebox{0}{train} & \rotatebox{0}{mbike} & \rotatebox{0}{bike} & mIoU \\
        \hline
        \hline
        Source  & 87.6 & 46.3 & 61.8 & 27.0 & 25.3 & 40.8 & 38.7 & 39.4 & 47.7& 26.8 & \textbf{11.4} & 48.6 & 39.9 & 76.1 & 15.9 & 24.2 & 52.0 & 26.5 & 29.6 & 40.3 \\
        TENT & \textbf{87.7} & 46.4 & 61.9 & 27.1 & 25.2 & 40.8 & 38.8 & 39.3  & 47.0 & 26.8 & 9.6 & 48.7 & 40.0 & 76.2 & 16.1 & 24.3 & 51.9 & 26.6& 29.7 & 40.2 \\
        CoTTA & 87.6 & 46.7 & \textbf{62.3}& 27.2 & 25.0 & 44.0 & 42.9 & 40.8 & 47.2 & 26.7 & 8.8 & 51.8 & 41.9 & 76.6 & 18.8 & 22.4 & 51.7& 27.8 & 32.1 & 41.2 \\
        DePT & 87.3 & 46.5 & 62.0 & 27.0 & 25.3 & 43.5 & 40.9 & 41.0 & 47.2 & 26.6 & 8.8 & 51.0 & 42.5 & 77.1 & 17.5 & 23.0 & 51.5 & 26.4 & 31.7 & 40.9 \\
        VDP & 87.6 & 46.8 & 62.2 & 27.1 & 25.0 & 44.0 & 42.9 & 41.0 & 47.3 & 26.6 & 9.0 & 51.7 & 41.9 & 76.6& 18.7 & 23.2 & 51.9 & 27.8& 32.0 & 41.2 \\
        \cellcolor{lightgray}\textbf{SVDP} &\cellcolor{lightgray}87.5 &\cellcolor{lightgray}\textbf{46.6}&\cellcolor{lightgray}52.7&\cellcolor{lightgray}\textbf{28.3}&\cellcolor{lightgray}\textbf{23.1}&\cellcolor{lightgray}\textbf{46.0}&\cellcolor{lightgray}\textbf{44.7}&\cellcolor{lightgray}\textbf{41.2}&\cellcolor{lightgray}\textbf{56.1}&\cellcolor{lightgray}\textbf{23.9}&\cellcolor{lightgray}10.5&\cellcolor{lightgray}\textbf{53.4}&\cellcolor{lightgray}\textbf{43.2} 
 &\cellcolor{lightgray}\textbf{78.0}&\cellcolor{lightgray}\textbf{25.7}&\cellcolor{lightgray}\textbf{26.0}&\cellcolor{lightgray}\textbf{46.9}&\cellcolor{lightgray}\textbf{29.7}     &\cellcolor{lightgray}\textbf{34.1} 
 &\cellcolor{lightgray}\textbf{42.0}\\\bottomrule
       

        
        \Xhline{1.2pt}
    \end{tabular}
  }
\label{table:night acdc}
\end{table*}

\begin{table*}[!htb]
\caption{Performance Comparison for \textbf{Cityscapes-to-ACDC Rain domain in TTA scenario}. The IoU score of each class and the mIoU score are reported. The best results are highlighted in \textbf{bold}.
  } 
  \centering
  \resizebox{0.99\textwidth}{!}{
    \def\arraystretch{1.1}
    \begin{tabular}{ l | c c c c c c c c c c c c c c c c c c c | c }
        \Xhline{1.2pt}
        Method & \rotatebox{0}{road} & \rotatebox{0}{side.} & \rotatebox{0}{buil.} & \rotatebox{0}{wall} & \rotatebox{0}{fence} & \rotatebox{0}{pole} & \rotatebox{0}{light} & \rotatebox{0}{sign} & \rotatebox{0}{veg.} & \rotatebox{0}{terr.} & \rotatebox{0}{sky} & \rotatebox{0}{pers.} & \rotatebox{0}{rider} & \rotatebox{0}{car}& \rotatebox{0}{truck} & \rotatebox{0}{bus} & \rotatebox{0}{train} & \rotatebox{0}{mbike} & \rotatebox{0}{bike} & mIoU \\
        \hline
        \hline
        Source  & 82.3 & 47.1 & 89.5 & 36.8 & 26.6 & 51.0 & 64.8 & 62.9 & 89.5 & 60.3 & 97.8 & 46.0 & 53.0 & 81.1 & 25.3 & 65.4 & 56.7 & 47.6 & 51.2 & 59.7 \\
        TENT & 82.4 & 46.7 & 89.6 & 37.3 & 27.0 & 50.6 & 64.6 & 62.9 & 89.5 & 60.4 & 97.7 & 46.7 &54.5 & 81.2 & 25.4 & 65.2 & 56.6 & 47.3 & 51.8 & 59.9 \\
        CoTTA & 83.0 & 48.4& 90.2 & 38.3 & 28.0 & 55.5 & 68.1 & 67.5& 90.3 &61.2 & 98.0 & 54.1 & 60.1 &82.0 & 27.4 & 67.0 & 59.1 & 50.9 & 55.4 & 62.6 \\
        DePT & 82.0 & 47.3 & 89.8& 37.5 & 26.9 & 53.0 & 66.2 & 65.8 & 89.6 & 60.8 & 97.7 & 52.8 & 59.5 & 81.6 & 26.5 & 66.5 & 57.9 & 49.5 & 53.2 & 61.3 \\
        VDP  & 83.0 & 48.3 & 90.2 & 38.2 & 28.0 & 55.5 & 68.2 &67.5 & 90.2 & 61.2 & 97.9 & 54.1 & 59.9 & 82.0 & 27.5 & 67.0 & 59.1 & 50.9 & 55.3 & 62.3 \\
        \cellcolor{lightgray}\textbf{SVDP} &\cellcolor{lightgray}\textbf{85.2} &\cellcolor{lightgray}\textbf{54.4}&\cellcolor{lightgray}\textbf{91.1}&\cellcolor{lightgray}\textbf{43.1}&\cellcolor{lightgray}\textbf{31.8}&\cellcolor{lightgray}\textbf{57.7}&\cellcolor{lightgray}\textbf{69.2}&\cellcolor{lightgray}\textbf{69.9}&\cellcolor{lightgray}\textbf{90.8}&\cellcolor{lightgray}\textbf{62.1}&\cellcolor{lightgray}\textbf{98.2}&\cellcolor{lightgray}\textbf{56.5}&\cellcolor{lightgray}\textbf{60.7} 
 &\cellcolor{lightgray}\textbf{83.0}&\cellcolor{lightgray}\textbf{28.7}&\cellcolor{lightgray}\textbf{67.6}&\cellcolor{lightgray}\textbf{63.6}&\cellcolor{lightgray}\textbf{55.0}      &\cellcolor{lightgray}\textbf{55.7} 
 &\cellcolor{lightgray}\textbf{64.4}\\\bottomrule
       
        
        \Xhline{1.2pt}
    \end{tabular}
  }
\label{table:rain acdc}
\end{table*}



\begin{table*}[!htb]
\caption{Performance Comparison for \textbf{Cityscapes-to-ACDC Snow domain in TTA scenario}. The IoU score of each class and the mIoU score are reported. The best results are highlighted in \textbf{bold}.
  } 
  \centering
  \resizebox{0.99\textwidth}{!}{
    \def\arraystretch{1.1}
    \begin{tabular}{ l | c c c c c c c c c c c c c c c c c c c | c }
        \Xhline{1.2pt}
        Method & \rotatebox{0}{road} & \rotatebox{0}{side.} & \rotatebox{0}{buil.} & \rotatebox{0}{wall} & \rotatebox{0}{fence} & \rotatebox{0}{pole} & \rotatebox{0}{light} & \rotatebox{0}{sign} & \rotatebox{0}{veg.} & \rotatebox{0}{terr.} & \rotatebox{0}{sky} & \rotatebox{0}{pers.} & \rotatebox{0}{rider} & \rotatebox{0}{car}& \rotatebox{0}{truck} & \rotatebox{0}{bus} & \rotatebox{0}{train} & \rotatebox{0}{mbike} & \rotatebox{0}{bike} & mIoU \\
        \hline
        \hline
        Source  & 79.8 & 40.8 & 86.9 & 43.6 & 46.5 & 56.4 & 72.3 & 65.5 & 82.9 &  \textbf{5.7} & 97.1 & 62.8 & 40.4 & 85.4 & 54.7 & 44.5 & 73.1 & 22.7 & 36.0 & 57.8 \\
        TENT & 79.6 & 40.0 & 86.8 & 43.4 & 46.5 & 56.1 & 72.2 & 65.6 & 82.9 &  \textbf{5.7} & 97.1 & 63.0 & 40.9 & 85.5 & 54.7 & 43.1 & 72.7& 23.0 & 36.5& 57.7\\
        CoTTA & 80.1 & 40.7 & 87.5 & 43.9 & 47.7 & 59.9 & 75.3 & 69.2 & 84.0 & 5.1 & 97.2 & 67.3 & 46.9 & 86.2 & 56.1& 43.4 & 74.1 &  25.7& 43.3 & 59.8 \\
        DePT & 79.1 & 40.6 & 86.8 & 43.4 & 47.5 & 59.8 & 75.1 & 69.4 & 83.5 & 5.2 & 97.1 & 67.2 & 46.5 & 86.3 & 56.0 & 44.0 & 73.9 & 25.6 & 43.1 & 59.5 \\
        VDP & 80.1 & 40.8 & 87.5 & 43.9 & 47.8 & 59.9 & 75.1 & 69.4 & 83.9 & 5.1 & 97.2 & 67.2 & 46.7 & 86.2 & 56.2 & 43.9 & 74.0 & 25.7 & 42.9 & 59.7 \\
        \cellcolor{lightgray}\textbf{SVDP} &\cellcolor{lightgray}\textbf{86.6} &\cellcolor{lightgray}\textbf{57.7}&\cellcolor{lightgray}\textbf{88.2}&\cellcolor{lightgray}\textbf{47.0}&\cellcolor{lightgray}\textbf{45.5}&\cellcolor{lightgray}\textbf{62.4}&\cellcolor{lightgray}\textbf{75.9}&\cellcolor{lightgray}\textbf{71.5}&\cellcolor{lightgray}\textbf{84.9}&\cellcolor{lightgray}3.9&\cellcolor{lightgray}\textbf{97.4}&\cellcolor{lightgray}\textbf{68.0}&\cellcolor{lightgray}\textbf{49.3} 
 &\cellcolor{lightgray}\textbf{87.1}&\cellcolor{lightgray}\textbf{57.1}&\cellcolor{lightgray}\textbf{49.2}&\cellcolor{lightgray}\textbf{77.5}&\cellcolor{lightgray}\textbf{26.8 }     &\cellcolor{lightgray}\textbf{45.4} 
 &\cellcolor{lightgray}\textbf{62.2}\\\bottomrule
        \Xhline{1.2pt}
    \end{tabular}
  }
\label{table:snow acdc}
\end{table*}






\section{Additional Quantitative Results}
\label{Sec: quantitative}
We present a comprehensive presentation of experimental results on the Test-time adaptation task for Cityscapes-to-ACDC, as shown in Tab .~\ref{table:fog acdc} - Tab .~\ref{table:snow acdc}. Our findings suggest that our proposed approach can better address the domain shift problem and achieve better IoU value in most categories.




\section{Additional related work}
\label{Sec:related}

\textbf{Semantic segmentation} is a crucial task in many computer vision applications aimed at assigning a categorical label to every pixel in an image. Several representative works in this field include DeepLab \cite{chen2017deeplab}, PSPNet \cite{zhao2017pyramid}, RefineNet \cite{lin2017refinenet}, and Segformer \cite{xie2021segformer}. Despite their high performance, these methods usually require extensive amounts of pixel-level annotated data, which can be laborious and time-consuming to collect. Additionally, they may suffer from poor generalization when applied to new domains.
Recent research has focused on addressing these challenges through domain adaptation strategies. For instance, \cite{yang2020fda} proposes a method that swaps the low-frequency spectrum to align the source and target domains. \cite{tranheden2021dacs} mixes the images from both domains, along with their corresponding labels and pseudo-labels. In contrast, \cite{wu2021dannet} uses adversarial learning to train a domain adaptation network for nighttime semantic segmentation. \cite{hoyer2022daformer} develops a novel model and training strategies to enhance training stability and avoid overfitting to the source domain.
However, these methods often require retraining the model on the source domain, which is inconvenient. Furthermore, they need to be retrained when adapting to a new target domain, incurring additional time and resource costs. Therefore, we propose SVDP to efficiently address the domain shift problem, which leverages a pre-trained model on the source domain and adds only a few parameters to achieve strong generalization capabilities on the target domain.

\textbf{Depth estimation} 
Depth estimation is a crucial task of machine scene comprehension. The ascendancy of deep learning has established it as the dominant approach for supervised depth estimation across both outdoor~\cite{eigen2014depth, geiger2012we, yang2019drivingstereo} and indoor settings~\cite{Silberman:ECCV12, scharstein2014high}. Typical methodologies involve the integration of a universal encoder, responsible for assimilating global context, alongside a decoder designed to retrieve depth details~\cite{Xu_2018_CVPR, Ramamonjisoa_2020_CVPR, lee2019monocular, Ramamonjisoa_2019_ICCV, fu2018deep}. In the context of cross-domain depth estimation, the emphasis is on aligning source and target domains either at the input or feature level~\cite{kundu2018adadepth,zheng2018t2net,zhao2019geometry}. As a case in point, the work presented in~\cite{zhao2019geometry} introduced a geometry-centric symmetric adaptation framework, conceived to optimize both the translation and the depth estimation processes simultaneously.~\cite{li2023test} combines the self-supervised model and the supervised model to tackle this problem. However, these methods always lead to catastrophic forgetting, and thus we introduce the Sparse Visual Domain Prompt to tackle this problem.
% \bibliography{supbib}

\end{document}