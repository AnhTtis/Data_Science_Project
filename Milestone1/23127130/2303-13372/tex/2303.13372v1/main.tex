%%%%%%%% ICML 2023 EXAMPLE LATEX SUBMISSION FILE %%%%%%%%%%%%%%%%%

\documentclass{article}

% Recommended, but optional, packages for figures and better typesetting:
\usepackage{microtype}
\usepackage{graphicx}
%\usepackage{subfigure}
\usepackage{booktabs} % for professional tables
\usepackage{enumitem}
%\usepackage{algorithm}
%\usepackage{algpseudocode}
%\usepackage[algcompatible]{algpseudocode}
\usepackage{amsmath}
\usepackage{array}
\usepackage{multirow}
\usepackage{amsfonts}
\usepackage{caption}
\usepackage{subfig}
\usepackage{array}


% hyperref makes hyperlinks in the resulting PDF.
% If your build breaks (sometimes temporarily if a hyperlink spans a page)
% please comment out the following usepackage line and replace
% \usepackage{icml2023} with \usepackage[nohyperref]{icml2023} above.
\usepackage{hyperref}


% Attempt to make hyperref and algorithmic work together better:
\newcommand{\theHalgorithm}{\arabic{algorithm}}

% Use the following line for the initial blind version submitted for review:
%\usepackage{icml2023}

% If accepted, instead use the following line for the camera-ready submission:
\usepackage[accepted]{icml2023}

% For theorems and such
\usepackage{amsmath}
\usepackage{amssymb}
\usepackage{mathtools}
\usepackage{amsthm}

% if you use cleveref..
\usepackage[capitalize,noabbrev]{cleveref}

%%%%%%%%%%%%%%%%%%%%%%%%%%%%%%%%
% THEOREMS
%%%%%%%%%%%%%%%%%%%%%%%%%%%%%%%%
\theoremstyle{plain}
\newtheorem{theorem}{Theorem}[section]
\newtheorem{proposition}[theorem]{Proposition}
\newtheorem{lemma}[theorem]{Lemma}
\newtheorem{corollary}[theorem]{Corollary}
\theoremstyle{definition}
\newtheorem{definition}[theorem]{Definition}
\newtheorem{assumption}[theorem]{Assumption}
\theoremstyle{remark}
\newtheorem{remark}[theorem]{Remark}

% Todonotes is useful during development; simply uncomment the next line
%    and comment out the line below the next line to turn off comments
%\usepackage[disable,textsize=tiny]{todonotes}
\usepackage[textsize=tiny]{todonotes}

\newcommand{\wwx}[1]{\textcolor{blue}{(Wenxiao: #1)}}
\newcommand{\smk}[1]{\textcolor{red}{(Shoumik: #1)}}
\newcommand{\yck}[1]{\textcolor{purple}{(Yigitcan: #1)}}
\newcommand{\SF}[1]{\textcolor{red}{(SF: #1)}}

% The \icmltitle you define below is probably too long as a header.
% Therefore, a short form for the running title is supplied here:
\icmltitlerunning{Adversarial Robustness of Learning-based Static Malware Classifiers}

\begin{document}

\twocolumn[
\icmltitle{Adversarial Robustness of Learning-based Static Malware Classifiers}



% It is OKAY to include author information, even for blind
% submissions: the style file will automatically remove it for you
% unless you've provided the [accepted] option to the icml2023
% package.

% List of affiliations: The first argument should be a (short)
% identifier you will use later to specify author affiliations
% Academic affiliations should list Department, University, City, Region, Country
% Industry affiliations should list Company, City, Region, Country

% You can specify symbols, otherwise they are numbered in order.
% Ideally, you should not use this facility. Affiliations will be numbered
% in order of appearance and this is the preferred way.
\icmlsetsymbol{equal}{*}

\begin{icmlauthorlist}
\icmlauthor{Shoumik Saha}{yyy}
\icmlauthor{Wenxiao Wang}{yyy}
\icmlauthor{Yigitcan Kaya}{yyy}
\icmlauthor{Soheil Feizi}{yyy}
%\icmlauthor{Firstname5 Lastname5}{yyy}
%\icmlauthor{Firstname6 Lastname6}{sch,yyy,comp}
%\icmlauthor{Firstname7 Lastname7}{comp}
%\icmlauthor{}{sch}
%\icmlauthor{Firstname8 Lastname8}{sch}
%\icmlauthor{Firstname8 Lastname8}{yyy,comp}
%\icmlauthor{}{sch}
%\icmlauthor{}{sch}
\end{icmlauthorlist}

\icmlaffiliation{yyy}{Department of Computer Science, University of Maryland, College Park, United States}
%\icmlaffiliation{comp}{Company Name, Location, Country}
%\icmlaffiliation{sch}{School of ZZZ, Institute of WWW, Location, Country}

\icmlcorrespondingauthor{Shoumik Saha}{smksaha@umd.edu}
%\icmlcorrespondingauthor{Firstname2 Lastname2}{first2.last2@www.uk}

% You may provide any keywords that you
% find helpful for describing your paper; these are used to populate
% the "keywords" metadata in the PDF but will not be shown in the document
\icmlkeywords{Machine Learning, Malware, Security}

\vskip 0.3in
]

% this must go after the closing bracket ] following \twocolumn[ ...

% This command actually creates the footnote in the first column
% listing the affiliations and the copyright notice.
% The command takes one argument, which is text to display at the start of the footnote.
% The \icmlEqualContribution command is standard text for equal contribution.
% Remove it (just {}) if you do not need this facility.


\printAffiliationsAndNotice{}  % leave blank if no need to mention equal contribution
%\printAffiliationsAndNotice{\icmlEqualContribution} % otherwise use the standard text.

\begin{abstract}


Over the past few years, there has been a significant amount of research focused on studying the ReLU activation function, with the aim of achieving neural network convergence through over-parametrization. However, recent developments in the field of Large Language Models (LLMs) have sparked interest in the use of exponential activation functions, specifically in the attention mechanism.

Mathematically, we define the neural function $F: \R^{d \times m} \times  \mathbb{R}^d \rightarrow \mathbb{R}$ using an exponential activation function. Given a set of data points with labels $\{(x_1, y_1), (x_2, y_2), \dots, (x_n, y_n)\} \subset \mathbb{R}^d \times \mathbb{R}$ where $n$ denotes the number of the data. Here $F(W(t),x)$ can be expressed as $F(W(t),x) := \sum_{r=1}^m a_r \exp(\langle w_r, x \rangle)$, where $m$ represents the number of neurons, and $w_r(t)$ are weights at time $t$. It's standard in literature that $a_r$ are the fixed weights and it's never changed during the training. We initialize the weights $W(0) \in \mathbb{R}^{d \times m}$ with random Gaussian distributions, such that $w_r(0) \sim \mathcal{N}(0, I_d)$ and initialize $a_r$ from random sign distribution for each $r \in [m]$.

Using the gradient descent algorithm, we can find a weight $W(T)$ such that $\| F(W(T), X) - y \|_2 \leq \epsilon$ holds with probability $1-\delta$, where $\epsilon \in (0,0.1)$ and $m = \Omega(n^{2+o(1)}\log(n/\delta))$. To optimize the over-parametrization bound $m$, we employ several tight analysis techniques from previous studies [Song and Yang arXiv 2019, Munteanu, Omlor, Song and Woodruff ICML 2022]. 

 

\end{abstract}

\section{Introduction}
\section{Introduction}

The increasing complexity of source code poses a key challenge to the reliability of large-scale software systems. Software bugs in these systems can lead to safety issues~\cite{bug_safety} for users around the world as well as cause non-negligible financial losses~\cite{bug_loss}. As such, developers have to spend a large amount of time and effort on bug fixing. Consequently, \aprfull (\apr), designed to automatically generate patches to fix software bugs, has attracted wide attention from both academia and industry~\cite{long2016prophet, legoues2012genprog, long2015spr, lou2020can, tufano2018empstudy}. 


To achieve \apr, one popular approach is known as Generate-and-Validate (G\&V)~\cite{qi2015gv, ghanbari2019prapr, lou2020can, le2016hdrepair, legoues2012genprog, wen2018capgen, hua2018sketchfix, martinez2016astor, koyuncu2020fixminder, liu2019tbar, liu2019avatar}, which is typically based on the following pipeline: First, fault localization techniques~\cite{wong2016fl, abreu2007ochiai, zhang2013injecting, papadakis2015metallaxis, li2019deepfl, li2017transforming} are applied to determine the suspicious locations in programs where bugs are likely to exist. Then, the buggy locations are used by the \apr tools to generate a list of patches that replace buggy lines with correct lines. Afterward, each patch is validated against the original test suite to identify any \emph{plausible patches} (i.e., passing all tests in the test suite). Finally, to determine the \emph{correct patches}, developers examine the list of plausible patches to see if any of them can correctly fix the bug. 

Traditional \apr tools can mainly be categorized into heuristic-based~\cite{legoues2012genprog, le2016hdrepair, wen2018capgen}, constraint-based~\cite{mechtaev2016angelix, le2017s3, demacro2014nopol, long2015spr} and \template~\cite{ghanbari2019prapr, hua2018sketchfix, martinez2016astor, liu2019tbar, liu2019avatar}. Among these traditional tools, \template \apr tools~\cite{ghanbari2019prapr, liu2019tbar, benton2020effectiveness} have been able to achieve state-of-the-art results. \Template \apr tools typically leverage pre-defined templates (e.g., adding a nullness check) for bug fixing. However, since these fix templates are typically handcrafted, the number and types of bugs they are able to fix can be limited. 



To address the limitations of traditional \apr, researchers have proposed various \learning \apr tools~\cite{li2020dlfix, chen2018sequencer, jiang2021cure, lutellier2020coconut, zhu2021recoder, ye2022rewardrepair} based on the \nmtfull (\nmt) architecture~\cite{sutskever2014mt} where the input is the buggy code snippets and the goal is to translate the buggy code snippets into a fixed version. To accomplish this, \learning \apr tools require supervised training datasets with pairs of both buggy and fixed code snippets in order to learn how to perform this translation step. These training data are usually obtained by mining historical bug fixes using heuristics/keywords~\cite{dallmeier2007benchmark}, which can be imprecise for identifying bug-fixing commits; even the actual bug-fixing commits can include irrelevant code changes, leading to further pollution in the dataset~\cite{xia2022alpharepair}.
% 
Moreover, it can be hard for such \apr tools to generalize and fix bug types unseen during training. 



To better leverage recent advances in \plmfull{s} (\plm{s}), researchers~\cite{xia2022alpharepair, xia2023repairstudy, kolak2022patch, prenner2021codexws} have directly applied \plm{s} to generate patches without bug-fixing datasets. These \llm-based \apr tools work by either directly generating a complete code function~\cite{prenner2021codexws, xia2023repairstudy} or predict/infill the correct code snippet given its surrounding context~\cite{xia2022alpharepair, xia2023repairstudy}. By directly using \llm{s} that are pre-trained on billions of open-source code snippets, \llm-based \apr tools can achieve state-of-the-art performance on many repair datasets~\cite{xia2022alpharepair}. 


% 
%
%

Traditional \apr tools have long used the insight of the \emph{plastic surgery hypothesis}~\cite{barr2014plastic} where it states that the code ingredients to fix a bug already exist within the same project. Traditional \apr tools have manually designed pattern-~\cite{ghanbari2019prapr, saha2017elixir} or heuristic-based~\cite{jiang2018simfix, legoues2012genprog} approaches to finding and using such relevant code ingredients to generate fixes for bugs. However, the plastic surgery hypothesis has been largely ignored in \llm-based \apr. In fact, \llm provides a unique opportunity to fully automate the plastic surgery hypothesis idea via fine-tuning (learning project-specific information via model updates from the buggy project) and prompting (directly providing relevant code ingredients to the model), and make it directly applicable to different languages (since the \llm{s} are typically multi-lingual).%
Moreover, despite the intensive manual efforts involved, traditional \apr tools still cannot fully leverage project-specific information due to large search space for leveraging/composing existing code ingredients. In contrast, the project-specific information can effectively leveraged by \llm{s} due to their power in code understanding/vectorization, e.g., even partial/imprecise information may still guide \llm{s} in correct patch generation!
 To this end, we ask the question: \emph{How useful is the plastic surgery hypothesis in the era of \plm{s}}?








\mypara{Our Work.} To answer the question, we present \ourtech{\xspace} -- a \llm-based approach that automatically utilizes the plastic surgery hypothesis by systematically combining multiple fine-tuning and prompting strategies for \apr. \ourtech fine-tunes \plm{s} using two novel domain-specific training strategies: \textbf{\epfinetune} -- we fine-tune using the original buggy project by aggressively masking out a high percentage of tokens, which allows \plm to learn project-specific code tokens and programming styles; and \textbf{\rofinetune} -- which only masks out a single continuous code sequence per training sample, allowing the model to get used to the final \csapr task of predicting a single continuous code sequence. Furthermore, we directly leverage the ability for \plm{s} to understand natural language instructions and introduce a novel prompting strategy, \textbf{\idprompting}, which uses information retrieval and static analysis to obtain a list of relevant identifiers for the buggy lines. While such relevant identifiers are critical for fixing some difficult bugs, they may not be seen by the \llm during inference due to limited context window size. Through the use of prompting, we directly tell the model to use these extracted identifiers (relevant code ingredients) to generate the correct code. Finally, to perform repair, we combine all four model variants (including the base model, both fine-tuned models and the base model with prompting) for the final repair.





While our insight of leveraging the plastic surgery hypothesis for \llm-based \apr is generalizable across different types of \plm{s}, to implement \ourtech, we choose a recent \plm{\xspace}, \ctfive~\cite{wang2021codet5}, which is pre-trained on millions of open-source code snippets. \ctfive is an encoder-decoder model trained using \mspfull (\msp) objective where a percentage of tokens are masked out and each continuous masked token sequence is referred to as a masked span. Also, although we only extract relevant identifiers from the current buggy project (since this paper focuses on the plastic surgery hypothesis), our work can be easily extended to obtain other code information (such as relevant statements or functions) from other sources, such as  the massive pre-training corpora~\cite{husain2020codesearchnet} or historical bug-fixing datasets~\cite{jiang2019infer}, which can provide more coding knowledge for \llm{s}. Besides, although we mainly focus on using traditional string comparison algorithms for information retrieval in this paper, these techniques can be easily replaced by other frequency-based retrieval~\cite{robertson2009probabilistic} and neural search (or embedding-based search)~\cite{reimers2019sentence}.
  In summary, this paper makes the following contributions:


%


\begin{itemize}[noitemsep, leftmargin=*, topsep=0pt]
    \item \textbf{Dimension.} This paper is the first to revisit the important plastic surgery hypothesis in the era of \llm{s}. It opens up a new dimension for \llm-based \apr to incorporate previously neglected information from the buggy project itself to boost \apr performance. Furthermore, it demonstrates the promising future of retrieval-based prompting for modern \llm-based \apr.
    \item \textbf{Implementation.} We implement \ourtech based on the recent \ctfive model. We augment the model using two novel fine-tuning strategies: \epfinetune and \rofinetune, along with a novel prompting strategy based on information retrieval and static analysis: \idprompting. We combine the patches generated by all four models together and perform patch ranking to speed up \apr.% 
    \item \textbf{Evaluation Study.} We conduct an extensive evaluation against state-of-the-art \apr tools. On the widely studied \dfj 1.2 and 2.0 datasets~\cite{just2014dfj}, \ourtech is able to achieve the new state-of-the-art results of 89 and 44 correct bug fixes (15 and 8 more than best baseline) respectively.  Furthermore, we perform a broad ablation study to justify our design. \ourtech demonstrates for the first time that the plastic surgery hypothesis can substantially boost \llm-based \apr and advance state-of-the-art \apr, while being fully automated and general. Moreover, even partial/imprecise code ingredients may still effectively guide \llm{s} for \apr!
\end{itemize}



\section{Related Work}
\section{Related work}
% There is extensive recent work on speeding up analytical queries due to the need for consistent execution times in the face of the explosive growth in the volume of available data.
% In this section, we divide existing work into two categories: maintaining data freshness in MVs (\Cref{sec:server_side}) and optimizations for minimizing ad-hoc query latency (\Cref{sec:client_side}).

% \subsection{Maintaining Data Freshness in MVs}
% \label{sec:server_side}
% There exists a variety of data warehousing applications aimed at supporting low-latency analytical queries on fresh data.
% In particular, these applications require efficiency in the propagation of newly ingested data into downstream MVs.
 
\mypara{Efficient MV Refresh}
Incremental view maintenance (IVM) aims to update MVs to reflect newly ingested data, taking advantage of already computed results to perform the update in a manner more efficient than computing from scratch (full refresh)
~\cite{ahmad2012dbtoaster,mcsherry2013differential,armbrust2013generalized,zeng2016iolap, palpanas2002incremental, griffin1995incremental, agiwal2021napa, braun2015analytics}. 
There is an abundance of work in IVM, including incremental updates on duplicate values~\cite{griffin1995incremental}, non-distributive aggregate functions~\cite{palpanas2002incremental}, higher-order views~\cite{ahmad2012dbtoaster}, and sliding windows~\cite{braun2015analytics}. 
More recent works also investigate the scalability aspect of IVM, proposing scale-independent updates~\cite{armbrust2013generalized} and sampled views~\cite{zeng2016iolap}. Since \system is applicable to arbitrary SQL statements, \system is orthogonal to and is fully compatible with existing IVM techniques.

\mypara{MV Refresh Scheduling}
There exist works on scheduling the refresh of a MV set focusing on resolving cyclic dependencies~\cite{folkert2005optimizing}, minimizing weighted average staleness~\cite{golab2009scheduling}, and prioritizing MVs with the highest speedups on predicted future queries~\cite{ahmed2020automated}.
\system's scheduling to speed up the end-to-end refresh of the MV set is not addressed in existing works.

\mypara{DAG Workflow Scheduling}
The execution of workloads consisting of individual jobs with acyclic dependencies is a well-studied topic~\cite{apacheoozie,sparkdag,marchal2018parallel,bathie2020revisiting,baruah2022ilp}; many of these techniques can be applied to MV refresh runs studied in this paper.
Existing workflow scheduling systems such as Apache Oozie~\cite{apacheoozie}, Apache Airflow~\cite{airflow}, and Spark DAG scheduler~\cite{sparkdag} automate the execution of user-defined workflows following a topological order.
There are recent works aimed at finding more optimal execution orders in terms of peak memory usage~\cite{marchal2018parallel, bathie2020revisiting} and execution time on parallel platforms~\cite{baruah2022ilp}.
While \system is designed for use with MV refresh runs/workloads, our technique on joint scheduling and optimization can be reasonably applied to general workloads as a possible future direction.

% \paragraph{Incremental MV indexing}
% Update-optimized indices such as the log-structured merge-trees (LSM)~\cite{o1996log} are used for indexing MVs due to frequent updates induced by data ingestion~\cite{gupta2016mesa,agiwal2021napa}.
% \system is orthogonal to indexing: \system is capable of efficiently performing MV refresh runs regardless of whether the individual MVs are indexed or not.

% \subsection{Ad-hoc Query Latency Reduction}
% \label{sec:client_side}

% The minimization of ad-hoc analytical query response times is a well-studied topic due to latency being negatively correlated with the productivity of a data analyst during a data analysis session~\cite{liu2014effects}.
% Sessions are commonly conducted within visualization systems that contain a variety of optimization techniques to ensure that query response times fall within a certain latency tolerance.

% \mypara{Data prefetching}
% Data is often loaded into memory on a by-need basis in visualization systems to minimize interference with user-issued query computations~\cite{mani2017effective,xin2021enhancing,galakatos2017revisiting, yan2020auto, battle2016dynamic, crotty2016case, jalaparti2018netco}. 
% Query-time data retrieval can be significantly expedited by anticipating the data usage of the user in future queries and pre-loading the data into memory during the downtime between user queries (`think time'). SMART~\cite{mani2017effective} prefetches data for modified versions of current user-issued queries with different filters and dimensions. A-WARE~\cite{crotty2016case} maintains a sample store constantly refined through ingesting data based on speculations of future plots.
% ForeCache~\cite{battle2016dynamic} uses an SVM to predict the user's current analysis phase and accordingly prefetches data tiles partitioned based on different numerical values. NetCo predicts future queries via log analysis, and solves an ILP formulation to prefetch data to maximize the number of SLO-meeting queries~\cite{jalaparti2018netco}.
% In the case of MV refresh workloads, `think time' is nonexistent as individual MVs are refreshed back-to-back, rendering data prefetching techniques non-applicable.

\mypara{Intermediate Data Caching}
Some existing data visualization systems cache user-defined variables to support the typical incremental construction of data visualizations~\cite{zgraggen2016progressive, eichmann2020idebench} during data analysis sessions~\cite{jupyter, rstudio, colab}. 
Recent work proposes a management scheme for these cached variables under a memory constraint that greedily keeps variables with the highest estimated time savings based on predicted future user behavior via neural networks~\cite{xin2021enhancing}.
While useful for data visualization, a greedy approach to memory management fails to achieve satisfactory results compared to \system.

\mypara{Intermediate Result Reuse}

There exist works on storing intermediate results from computations to speedup future computations~\cite{yang2018intermediate, dursun2017revisiting, nagel2013recycling, michiardi2019memory, galakatos2017revisiting}.
Studied topics include the identification of reuse opportunities by finding overlaps in computation graphs of successive jobs~\cite{yang2018intermediate, michiardi2019memory},
selective storage under a space constraint with heuristics such as reuse probability~\cite{dursun2017revisiting}, expected savings~\cite{yang2018intermediate}, and recompute-storage cost difference~\cite{nagel2013recycling},
and rewriting incoming jobs to take advantage of stored intermediates~\cite{galakatos2017revisiting}.
These works share similarity with \system in their selection of items to store under a memory constraint, however, \system's problem setting requires it to uniquely consider the joint (re)ordering of job executions along with the selection of items.

% work that considers both job execution (re)order as well as intermediate result caching with a bounded amount of memory. but notably lack the joint aspect of \system and cannot be used to achieve immediate speedup on an incoming MV refresh run if no intermediates are stored beforehand. 

\mypara{Incremental Query Processing} Incremental processing (IQP) is useful for cases where not all data required for a query is immediately available. Similar to online aggregation~\cite{hellerstein1997online}, initial results of a query are computed on a subset of required data and progressively refined as the rest of the required data arrives in a predictable pattern~\cite{tang2019intermittent,wangtempura}. Tang et al. propose a dynamic programming formulation to pick intermediate states to store in memory given a limited memory budget~\cite{tang2019intermittent}. Tempura rewrites the query plan for more efficient execution based on predicted data arrival patterns~\cite{wangtempura}. While similarities exist between the problem setting of IQP and \system, such as management of bounded memory, \system notably includes additional joint optimization for the order of MV updates.

% \paragraph{Sampling}
% Sampling has seen wide use in visualization systems for reducing the computation time of ad-hoc queries by computing an approximate result over a subset of data as exact results are not always required by the user~\cite{crotty2016case, mani2017effective, zgraggen2014panoramicdata, kraska2021northstar, galakatos2017revisiting, kandula2016quickr}. 
% Commonly studied topics in sampling for ad-hoc queries include complex query sampling~\cite{kandula2016quickr}, rare event aggregation~\cite{kraska2021northstar, galakatos2017revisiting}, and maintaining consistency between related sampled visualizations~\cite{zgraggen2014panoramicdata}.
% Sampling server-side at the MV level compromises the assumptions of downstream applications and is thus not considered in \system.

% \paragraph{Progressive visualization}
% The latency tolerance for time-consuming queries can be circumvented by presenting a partially-computed visualization to the user within the tolerance, which is then incrementally refined until it is fully accurate~\cite{rahman2017ve, zgraggen2016progressive, crotty2015vizdom, kraska2021northstar, kamat2017infiniviz}.
% Example plots which benefit from progressive visualization include bar charts~\cite{kamat2017infiniviz} and heatmaps~\cite{rahman2017ve}.
% Similar to sampling, study on this topic is orthogonal to \system as pushing out partially-updated MVs compromises downstream assumptions.

\section{Background and Notation}
\section{Background on Network Calculus}
\label{sec: background}


\begin{figure*}[tbh]
\centering
\begin{subfigure}[b]{0.3\textwidth}
    \centering
    \includegraphics[width=\linewidth]{images/in-out.png}
    \caption{Arrival and departure data and their relation with delay $d(t)$ and backlog $b(t)$. For a FIFO system, the delay is the horizontal distance between $R(t)$ and $R^*(t)$ but some other multiplexing techniques may shift the data to a later priority, causing a longer delay.}
    \label{fig: data in-out}
\end{subfigure}
\hfill
\begin{subfigure}[b]{0.35\textwidth}
    \centering
    \includegraphics[width=\linewidth]{images/arrival-service.png}
    \caption{Characteristics of an arrival curve and a service curve. From any point of observation, the arriving data never exceeds its arrival curve; the departure data is also never less than the service curve with respect to the data arrival.}
    \label{fig: arrival-service curves}
\end{subfigure}
\hfill
\begin{subfigure}[b]{0.33\textwidth}
    \centering
    \includegraphics[width=\linewidth]{images/bound.png}
    \caption{Delay and backlog bounds of a system. Backlog is the maximum vertical distance between $\alpha(t)$ and $\beta(t)$; FIFO delay is their maximum horizontal distance; but for arbitrary multiplexing, the delay guarantee is when the system clears its buffer, thus it's the intersection of $\alpha(t)$ and $\beta(t)$.}
    \label{fig: system bounds}
\end{subfigure}
\caption{Network calculus framework. We let $R(t)$ and $R^*(t)$ be the arrival and departure data flow of a system; $\alpha(t)$ be the piecewise linear concave arrival curve and $\beta(t)$ be the piecewise linear convex service curve of a system.}
% \hossein{Better to show piece-wise linear concave arrival curve and piece-wise linear convex service curve instead of token-bucket and rate-latency.}}
\end{figure*}

We recall some of the network calculus essentials for a better understanding of the framework used in Saihu. In the following context, we use the following notation: $\mbb{R}^+$ is the set of non-negative real numbers; $[x]_+$ denotes $\max(0, x)$

The data flow is by convention modeled as a left-continuous wide-sense increasing function $R(t): \mbb{R}^+ \mapsto \mbb{R}^+$ with respect to time $t$~\cite{ncbook2001leboudec}. 

A system $\mcal{S}$ receives arrival data described as a cumulative function $R(t)$ and delivers departure data as another cumulative function $R^*(t)$. Figure~\ref{fig: data in-out} illustrates such a system $\mcal{S}$. The benefit of representing a system like this is that we can observe system backlog and delay with such a model. 

\begin{definition}[Backlog and Delay~\cite{ncbook2001leboudec}]
    The backlog of a system at time~$t$ is
    \begin{equation}
        b(t) = R(t) - R^*(t)
    \end{equation}
    
    The virtual delay of a FIFO system at time $t$ is
    \begin{equation}
        d_{FIFO}(t) = \inf \lbp \tau \geq 0 : R(t) \leq R^*(t+\tau) \rbp
    \end{equation}
\end{definition}



The backlog of a system can be viewed as the vertical distance between $R$ and $R^*$. The FIFO (\textit{First-in First-out}) delay is the horizontal distance between $R$ and $R^*$. One may obtain other delay values if the multiplexing technique is not FIFO.

% \begin{figure}
%     \centering
%     \includegraphics[width=0.9\linewidth]{images/in-out.png}
%     \caption{In/out data flow; delay and backlog}
%     \label{fig: data in-out}
% \end{figure}

Since we are interested in the system guarantee instead of a single instance of data flow, we would like to have general bounds to the arrival and departure data flows. Therefore, we define \textit{arrival curve} and \textit{service curve} as the bounds of arrival and departure data flows.

\begin{definition}[Arrival Curve~\cite{ncbook2001leboudec}]
    Given a wide-sense increasing function $\alpha: \mbb{R}^+ \mapsto \mbb{R}^+$, we say that a flow $R(t)$ is $\alpha$-constrained if and only if for all $s \leq t$:
    \begin{equation}
        R(t) - R(s) \leq \alpha(t-s)
    \end{equation}
    We say $R(t)$ has $\alpha$ as an arrival curve.
\end{definition}

\begin{definition}[Service Curve~\cite{ncbook2001leboudec}]
    Given a wide-sense increasing function $\beta: \mbb{R}^+ \mapsto \mbb{R}^+$ and $\beta(0) = 0$. A system $\mcal{S}$ having $R(t)$ and $R^*(t)$ as its arrival and departure flows. We say $\mcal{S}$ offers a service curve $\beta$ if and only if
    \begin{equation}
        R^*(t) \geq (R \otimes \beta)(t) =: \inf_{s \leq t} \lbp R(s) + \beta(t-s) \rbp
    \end{equation}
    where $\otimes$ denotes the min-plus convolution
\end{definition}

Figure~\ref{fig: arrival-service curves} illustrates the arrival and service curves. Any segment of arrival flow $R(t)$ is constrained by arrival curve $\alpha$ and the output curve $R^*(t)$ is always no less than the curve $R\otimes\beta$. As a result, an arrival curve upper bounds the incoming traffic, and a service curve lower bounds the outgoing traffic.

% \begin{figure}
%     \centering
%     \includegraphics[width=\linewidth]{images/arrival-service.png}
%     \caption{Arrival/Service curve}
%     \label{fig: arrival-service curves}
% \end{figure}

We consider 2 special types of curves throughout this paper, \textit{token-bucket} (or sometimes called \textit{leaky-bucket}) curve and \textit{rate-Latency} curve.

\begin{definition}[Token-bucket and Rate-latency~\cite{ncbook2001leboudec}]
    A token-bucket curve $\gamma_{r,b}$ with arrival rate $r$ and burst $b$ is defined as
    \begin{equation}
        \gamma_{r,b}(t) = b + rt
    \end{equation}

    A rate-latency curve $\beta_{R,T}$ with service rate $R$ and latency $T$ is defined as
    \begin{equation}
        \beta_{R,T}(t) = R \lb t - T \rb_+
    \end{equation}
\end{definition}

A token-bucket curve is determined by a burst $b$ and an arrival rate~$r$. Burst represents the maximum possible data volume that can arrive simultaneously, and arrival rate represents the maximum long-term data rate~\cite{bouillard2022tradeoff}.
A rate-latency curve is determined by a latency~$T$ and a service rate~$R$. Latency represents the time a server needs before starting to process the incoming data, and service rate represents the minimum rate to process data after the initial latency.

With the help of arrival and service curves, we can derive delay and backlog bounds for a system $\mcal{S}$ illustrated in Figure~\ref{fig: system bounds}. Suppose a system $\mcal{S}$ has arrival curve $\alpha$ and service curve~$\beta$, its worst-case backlog $b^*$ is the maximum vertical distance between~$\alpha$ and~$\beta$. Similarly, depending on the multiplexing technique applied to the system, its worst-case delay bound $d^*$ is the maximum horizontal distance between $\alpha$ and $\beta$ if $\mcal{S}$ is a FIFO system. If we don't have any information about its multiplexing technique, referred to as arbitrary multiplexing, the best we can say is that when $\alpha$ and $\beta$ intersect each other, where all data has been delivered out of the system. Consequently, the worst-case delay bound for arbitrary multiplexing is the time required for $\mcal{S}$ to clear its buffer.

% \begin{figure}
%     \centering
%     \includegraphics[width=\linewidth]{images/bound.png}
%     \caption{System delay/backlog bounds}
%     \label{fig: system bounds}
% \end{figure}

While a service curve captures the slowest possible output speed of a system, a link's transmission capacity limits the speed as well. Hence, we model this phenomenon using a \textit{greedy shaper} with a sub-additive function $\sigma: \mbb{R}^+ \mapsto \mbb{R}^+$ concatenated with a server. We consider a concatenation as shown in Figure \ref{fig: system}. By convention we assume $\sigma(0) = 0$ and $\beta(t) \leq \sigma(t), \forall t \in \mbb{R}^+$, meaning that the buffer is cleared at the beginning and the service never exceed its physical limitation. With the above definition, such greedy shaper conserves the service provided by the system due to theorem \ref{thm: shaping}.

\begin{figure}[thb]
    \centering
    \includegraphics[width=0.7\linewidth]{images/system.png}
    \caption{Shaping of departure data. A flow that has an arrival curve $\alpha$ feeds into a server with an arrival data flow $R(t)$. The server having service curve $\beta$ takes $R(t)$ and gives a departure data flow $R^*(t)$ to a shaper with shaping function $\sigma$. The shaper takes $R^*(t)$ and shape the data flow as another departure $D(t)$.}
    \label{fig: system}
\end{figure}


\begin{theorem}[Shaping conserves service \cite{ncbook2001leboudec}]
\label{thm: shaping}
Following the system shown in Figure \ref{fig: system}, we have
\begin{equation}
     D = R^* \otimes \sigma \geq \lp R \otimes \beta \rp \otimes \sigma = R \otimes \lp \beta \otimes \sigma \rp = R \otimes \beta
\end{equation}
\end{theorem}

In the following context, we model the shaping function $\sigma$ as a token-bucket curve $\gamma_{C,L}$ with transmission capacity $C$ and the packet size $L$ to capture the link capacity and packetization~\cite{bouillard2022tradeoff}.


\section{Vulnerability of MalConv to Patch Appending Attacks}
\label{sec:attacks}
\subsection{Threat Model}
%%%% The section of threat model 
\section{Security of White-Box DNN Watermark} 
% With increasingly more white-box watermarking techniques exploring the usage of the strong correlation to the significant parameters or selected stable activation maps to claim even stronger resilience than previous approaches\cite{ wang2021riga,liu2021greedyresiduals, chen2021lottery,fan2021deepip,lim2022ipcaption,zhang2020passportaware,ong2021iprgan}, recent research at the attack side starts to question for the robustness of existing watermarking schemes. However, to the best of our knowledge, \cite{sokwatermark} is the only existing systematic evaluation on the robustness of black-box DNN watermarks. In contrast, the robustness of white-box watermarks is mainly evaluated by the very scheme designers. 
\subsection{Security Settings}
% 段落整合,攻击分类攻击目标,ambiguity vs remove
%%% 介绍现有攻击的几种目标(ambiguity and removal (maybe others))
\noindent\textbf{Attack Taxonomy.} According to the adversarial goal, we first categorize existing attacks on white-box model watermarking from previous works into \textit{ambiguity attacks} and \textit{removal attacks}. In the former attacks, the adversary aims at constructing a counterfeit watermark, when given the watermarked DNN \cite{zhang2020passportaware}, to pass the verification process. Instead, the removal attacks have a more straightforward goal: invalidating the verification process by removing the secret identity message from the protected model. Considering its severe influence on establishing the model ownership, our work concentrates on devising novel removal attacks to crack the state-of-the-art white-box model watermarks. Below, we formally describe the attack scenario. 
% leaving the secret watermark embedded intact
%  serves as a solid counter-evidence against the lawsuit filed by the owner.

\noindent\textbf{Attack Scenario.} In our threat model, the adversary has obtained an illegal copy of a watermarked model which allows full access to its model parameters. Such model piracy can be accomplished via either algorithmic attacks \cite{tramer2016StealViaApi, yu2020cloudleak} or system attacks exploiting software/hardware vulnerabilities \cite{jeong2021meltdown,yan2020cache}. To conceal the traces of model infringement, the attacker attempts to invalidate the model ownership verification by removing the existing watermarks. 

\noindent\textbf{Attack Budget.} As listed in Table \ref{tab:intro_table_comparison}, the attack budget of a removal attack is mainly measured in the following dimensions: \textit{utility loss}, \textit{training cost}, \textit{dataset access}, \textit{watermark knowledge} (similar to the ones on black-box watermark in \cite{sokwatermark}).

%%% 需要讨论
\begin{itemize}
\item \textbf{Utility Loss:} When removing the watermark, obfuscation on the parameters or the structure of the DNN model seems inevitable. In this case, the obfuscation should not incur a large decrease in the normal model utility, which is otherwise unacceptable because the attacker still requires the normal utility of the pirated model for profits.
\item \textbf{Training Cost:} For watermark removal, the adversary is usually unwilling to cost a similar scale of computing resources as retraining the DNN model from scratch. Typically, the adversary would avoid the expensive model training process, which usually involves the usage of high-end graphical cards for training industry-level models, but prefer to learning-free attacks.  

\item \textbf{Dataset Access:} As the training is usually a private asset of the model owner, the access to the original training data or even a public domain dataset brings an additional attack budget. The adversary would prefer to involve no real data inputs for conducting the attack.  

\item \textbf{Watermark Knowledge:} The adversary should have no knowledge about the adopted watermark embedding and extraction algorithms, which are usually exclusively known to the owner until the ownership verification is launched. 
% To obtain such knowledge, the attacker needs to invoke some precondition attacks to determine the specific location (e.g., the parameters in which layer) of the embedded watermark, which incurs additional attack budgets.   

\end{itemize}



%%%%%%%%%% TODO 
\subsection{Limitation of Existing Removal Attacks}
\label{sec:limitations}
Previous removal attacks
% based on parameter and structure obfuscation 
are all limited in one or more of the above dimensions for fully removing white-box watermarks from a protected model. 

\begin{itemize}

\item \textbf{Pruning} sets a proportion of redundant parameters in DNN to zero, under which previous white-box watermark is highly resistant. To fully remove the watermark, pruning has to remove a substantial amount of weights, which causes an unacceptable utility loss \cite{uchida2017embedding, darvish2019deepsigns}.

\item \textbf{Finetuning} continues the training operation for a few epochs without the watermark-related loss. This removal attack additionally requires a certain amount of domain data and computational resources, otherwise the model utility would degrade \cite{chen2021refit, guo2021ftnotenough}.

\item \textbf{Overwriting} is first proposed in \cite{wang2019overwrite} to show the vulnerability of  \cite{uchida2017embedding}. Considering the adversary has full knowledge about the watermarking process, he/she may confuse the verification by embedding his own identification information. However, the details of watermark schemes are always not available in real-world settings. Meanwhile, overwriting attacks usually fail to encode a new message into the target model following more advanced schemes \cite{fan2021deepip, ong2021iprgan, lim2022ipcaption, zhang2020passportaware}.

\item \textbf{Extraction} refers to an attack class which utilizes knowledge distillation techniques \cite{hinton2015distilling} on the pirated model to obtain an obfuscated model which usually has a different architecture. The extraction attack inevitably involves a substantial amount of training costs to distill a well-trained obfuscated model. Although some very recent works in knowledge distillation eliminate the assumption on dataset access \cite{Yin2020DreamingTD}, most mainstream extraction attacks still use the conventional knowledge distillation approaches and require the access to a domain dataset to reduce the utility loss.     
\end{itemize} 


% We do not consider the model extraction attacks (e.g., \cite{Jagielski2020HighAA}) in our work, as this branch of works mainly aim at stealing models from prediction APIs and also require a substantial amount of training cost to distill a well-trained surrogate models from the victim model. In summary, none of existing removal attacks could meet all the requirements in Section \ref{sec:threat_model} simultaneously. Meanwhile, these attacks are proven to be more effective on black-box model watermarking instead, probably because the black-box model watermarking only rely on the prediction results on a specific set of inputs and hence the behavior can be altered without concerns on the otherwise constraints on modifying some internal parameters which may be related with the white-box model watermarking \cite{ shafieinejad2021robustofbackdoorbased, chen2021refit, aiken2021laundering, guo2021ftnotenough, wang2019neuralcleanse}. With more novel white-box watermarking techniques proposed recently claiming even stronger robustness under limited evaluation, white-box model watermarking attracts increasing attention from the academy and industry as more robust and desirable for practical usages.

%  \item \textbf{No Access to Original Training Data:} As the training is usually a private asset of the model owner, the access to the original training data or even data from a similar domain, is not always practical in reality. Therefore, our work restricts the attacker from obtaining any knowledge of confidential distribution of the training data, since he/she otherwise may legally train his own model from scratch, which makes the white-box watermark removal attack unnecessary.

% \begin{tcolorbox}%[top=3pt]
% \textbf{Summary}: Existing attacks are all limited in one or more aspects for removing the state-of-the-art white-box watermarks from the protected model.   
% \end{tcolorbox}

\subsection{Input-specific Patch Attack} \label{subsec:FGM attack}

There are multiple ways to perform append attack on malware files. The first append attack on MalConv model was presented by \cite{8553214}. In their work, they demonstrated random append and gradient append, and showed gradient append performs significantly better than the former one. However, in gradient append strategy the convergence time grows linearly with the number of appended bytes. So, we used the FGSM (Fast Gradient Sign Method) that was first proposed by \cite{goodfellow2014explaining}. Such FGSM attack has been adopted for malware perturbation by \cite{kreuk2018deceiving}. However, we follow an advanced approach that was proposed by \cite{suciu2019exploring} demonstrated in Algorithm \ref{alg:FGMattack}.

\begin{algorithm}[h!]
%\caption{EmbeddingMapping ($e_x$)}
\begin{algorithmic}

\FUNCTION {$\text{EmbeddingMapping}$ ($e_x$)}
\STATE $e \leftarrow \text{ARRAY}(256)$
\FOR{$byte$ in $0...255$} 
    \STATE {$e[byte] \leftarrow \text{GetEmbeddings}[byte]$} 
\ENDFOR
\FOR{$i$ in $0...|e_x|$} 
    \STATE $x^*[i] \leftarrow argmin_{b \in {0...255}(||e_x[i]-e[b]||_2)}$
\ENDFOR \\
\textbf{return} $x^*$
%\RETURN $x^*$
\ENDFUNCTION
\end{algorithmic}
\end{algorithm}

\begin{algorithm}[h!]
\caption{The FGSM Append attack}\label{alg:FGMattack}
\begin{algorithmic}
\FUNCTION {$\text{FGMAPPEND} (x_0,numBytes,\epsilon)$}
\STATE $x_0\leftarrow\text{PadRandom}(x_0,numBytes)$
\STATE $e\leftarrow\text{GetEmbeddings}(x_0)$
\STATE $e_p\leftarrow\text{GradientAttack}(e,\epsilon)$
\FOR{$i$ in $|x_0| ...|x_0|+numBytes-1$} 
    \STATE {$e[i] \leftarrow e_p[i]$} 
\ENDFOR
\STATE $x^* \leftarrow \text{EmbeddingMapping}(e)$ \\
    \textbf{return} $x^*$
\ENDFUNCTION

\FUNCTION {$\text{GradientAttack}$ ($e,\epsilon$)}
\STATE $e_u \leftarrow e - \epsilon*sign(\nabla_l(e))$ \\
    \textbf{return} $e_u$
\ENDFUNCTION

\end{algorithmic}
\end{algorithm}




In this attack, we start by appending random bytes of $numBytes$ size at the end of original malware sample $x_0$. Since back-propagation of gradient is not possible on embedding layer, we calculate the embedding representation $e$ for $x_0$ and run the FGSM on $e$. This method updates the each value of $e$ by a pre-specified amount $\epsilon$ depending on the sign of the gradient $\nabla_l$. In this paper, we used $L_\infty$ as our distance metric. For the non-differentiability of embedding layer, we mapped the updated embedding value to the byte value using $L_2$ distance metric. We experimented with nearest neighbor and inverse multiplication of embedding matrix, and found that different approach does not affect the evasion rate significantly. We have discussed the performance in \ref{result_adv_patch}.



\subsection{Universal (Input-agnostic) Patch Attack} \label{subsec:uap}

\begin{algorithm}[h!]
\caption{The Universal FGSM Append attack}\label{alg:FGMattack_uni}
\begin{algorithmic}

\FUNCTION{$\text{UniversalPatch} (X, numBytes, \epsilon)$}
\STATE $e_{init} \leftarrow \text{Random[1, numBytes]}$
\FOR{$i$ in $0 ... |X|$}
    \STATE  $ e \leftarrow \text{PERTURB EMBEDDING}(x_i, e_{init}, \epsilon) $
    \STATE $e_{list}.append(e)$
\ENDFOR
\STATE $e_{universal} \leftarrow avg(e_{list})$
\STATE $x_{universal} \leftarrow \text{EmbeddingMapping}(e_{universal})$ \\
\textbf{return} $x_{universal}$
\ENDFUNCTION

\FUNCTION {$\text{PerturbEmbedding} (x_0, e_0,\epsilon)$}
\STATE $e_x\leftarrow\text{GetEmbeddings}(x_0)$
\STATE $e\leftarrow\text{Append}(e_x,e_0)$
%\State $e\leftarrow\text{GetEmbeddings}(x_0)$
\STATE $e_p\leftarrow\text{GradientAttack}(e,\epsilon)$ \\
\textbf{return} $e_p[|x_0|:|x_0|+|e_0|]$
\ENDFUNCTION

\FUNCTION {$\text{GradientAttack} (e,\epsilon)$}
\STATE $e_u \leftarrow e - \epsilon*sign(\nabla_l(e))$ \\
    \textbf{return} $e_u$
\ENDFUNCTION

\end{algorithmic}
\end{algorithm}

Besides input-specific patch attack, we implemented a universal attack strategy that is more-realistic. For this attack, our goal was to generate a universal patch $x_{universal}$ of size $numBytes$ that can be appended at the end of any malware sample $x$, generating $x^{'}$ that results into $F_{\theta}(x^{'}) < 0.5$. The detailed algorithm is shown in \ref{alg:FGMattack_uni}.

For this attack, we started with generating random patch-embedding $e_{init}$ and appended them at the end of the embedding of multiple malware binaries from $X$ set. Then the perturbation for the embedding is generated just like \ref{subsec:FGM attack}. Instead of directly mapping the bytes from the embedding, we collect the perturbed embedding for all malware files and take the average which can be denoted as the embedding representation of the universal patch. Then the mapping of this embedding representation to byte level gives the universal patch that can be used later for evasion. 
We tested our generated universal patch on new diverse unseen malware and the performance details can be found in \ref{result_adv_patch_uni}.



 %2 subsections inside this tex file

\section{Enabling De-randomized Smoothing for Byte Inputs}
\label{sec:defense}
\subsection{Stealthiness of Injected Dummy Neurons}
\label{sec:eval:stealthiness_dn}
Finally, we provide a preliminary study on potential adaptive approaches to detect and eliminate dummy neurons. Specifically, we consider two types of defenders, (a) a \textit{partially knowledgeable defender}, i.e., who knows the existence of dummy neurons but has rare knowledge about the detailed algorithm for generating the dummy neurons, (b) a \textit{skilled defender}, who has a perfect knowledge about our attack framework but does not have access to the original model (a common setting in existing watermarking protocols), and (c) a \textit{fully knowledgeable defender}, who also has the original model for reference. 

\noindent\textbf{(a) A Partially Knowledgeable Defender.} If knowing the existence of dummy neurons in the suspect model, the defender is likely to apply anomaly detection algorithms to detect and eliminate the suspicious neurons from the target layer. Specifically, by considering the incoming and outgoing weights of each neuron as the feature vector, we implement two representative anomaly detection algorithms, i.e., \textit{cluster-based} \cite{chen2018activation_clustering} and \textit{SVD-based} \cite{tran2018spectral}, to evaluate the stealthiness. 

\noindent$\bullet$\textbf{ Experimental Settings.}
We first inject the dummy neurons generated by NeuronZero, NeuronClique, and NeuronSplit into the watermarked models, respectively. Then, we concatenate the flattened incoming and outgoing weights of each neuron as its feature vector. The cluster-based detection leverages K-Means to separate the neurons from the same layer and assigns the abnormal cluster as dummy neurons \cite{chen2018activation_clustering}, while the SVD-based detection utilizes the covariance matrix of the neurons' feature representation to filter outliers \cite{tran2018spectral}.

\noindent$\bullet$\textbf{ Results \& Analysis.} 
As is shown in Fig.\ref{fig:detection}, the dummy neurons with vanishing values generated by NeuronZero are more likely to be recognized as abnormal neurons under both detection approaches, while the dummy neurons produced by NeuronSplit from the original neurons show stronger stealthiness compared to both NeuronZero and NeuronClique, as their weights have the same distribution to the normal ones.
%%%%%%%%%%%%% BEGIN of Detection Rate
\begin{figure}[t]
\begin{center}
\includegraphics[width=0.5\textwidth]{img/detection_rate.pdf}
\caption{Detection rate of anomaly detection algorithms on different types of dummy neurons.}
\label{fig:detection}
\end{center}
\end{figure}
%%%%%%%%%%%%%% END of Detection Rate


%%%%%%%%% BEGIN OF LINEAR ELIMINATION ALGORITHM 
 \begin{algorithm}[h]
 \caption{A possible dummy neuron elimination algorithm.}
 \label{alg:dn_elim}
  {\fontsize{10}{10}\selectfont
 \begin{algorithmic}[1]
 \renewcommand{\algorithmicrequire}{\textbf{Input:}}
 \REQUIRE $W$ (the parameters of the suspect model), $H$ (the number of layers in the suspect model).
 \renewcommand{\algorithmicensure}{\textbf{Output:}}
 \ENSURE $W$, the parameters of the suspect model after dummy neuron elimination.
%  \ENSURE $P = P_w \cup P_b \cup P_s$ (the decoded ParamPool).
% \STATE{$P_w\gets list(),P_b\gets list(), P_s\gets list()$}
 \STATE {$T_{hash} \gets \{\} $}\\
 {\small{/* Find the neurons with proportional incoming weights.*/}}
 \FOR {$l = 1,2,...,H-1$}
  \STATE {$W^{(l)}, W^{(l+1)} \gets W[l],W[l+1]$} {\small{\hskip3em // Obtain the incoming and outgoing weights of the neurons in the $l^\text{th}$ layer}} 
  \STATE {$Ind, \tilde{W}^{(l)}, \tilde{W}^{(l+1)} \gets 0, zeros\_{like}(W^{(l)}), zeros\_{like}(W^{(l+1)})$}

  \FOR {each input weight $W^{(l)}_{\cdot i}$ of the $i^{th}$ neuron in the $l^{th}$ layer}
   \STATE{$w \gets \text{Flatten}(W^{(l)}_{\cdot i}) $}
   \STATE {$w_{norm} \gets \text{Normalize}(W^{(l)}_{\cdot i})$}
   \STATE {$w \gets \frac{W^{(l)}_{\cdot i}}{w_{norm}}$}
   \IF {$w$ not in $T_{hash}$.keys()}
    \STATE {$T_{hash}[w] \gets Ind$}
    \STATE {$Ind \gets Ind + 1$}
    \ELSE
     \STATE{\small{/* Merge the neurons in the same hash bucket.*/}} 
     \STATE {$i' \gets T_{hash}[w]$}
     \STATE {$\tilde{W}^{(l)}_{\cdot i'} \gets w$}
     \STATE {$\tilde{W}^{(l+1)}_{i' \cdot } \gets \tilde{W}^{(l+1)}_{i' \cdot } + w_{norm} \cdot W^{(l+1)}_{i \cdot}$}
    \ENDIF
   \ENDFOR
  \STATE {$W^{(l)}, W^{(l+1)} \gets \tilde{W}^{(l)}, \tilde{W}^{(l+1)}$}
\STATE {Remove the neurons with zero incoming or outgoing weights in $W^{(l)}$, $W^{(l+1)}$.}
  \ENDFOR
 \RETURN  $W$
 \end{algorithmic}}
 \end{algorithm}
%%%%%%%%% END OF LINEAR ELIMINATION ALGORITHM


\noindent\textbf{(b) A Skilled Defender.} Besides the above defense, we consider a more adaptive defender who has perfect knowledge about our proposed attack. From our construction in Section \ref{sec:dn_generation} (combined with the defense results above), the only exploitable information for dummy neuron elimination is in the first layer where the dummy neurons are injected. According to Eq.\ref{eq:NC_incoming}\&\ref{eq:Ns_incoming}, if there are no dummy neurons in the previous layer, then the dummy neurons belonging to the same group generated by \textit{NeuronClique} or \textit{NeuronSplit} would have their incoming weights, if viewed as vectors, proportional to each other. Based on this characteristic, the defender may implement the following procedures to detect and eliminate dummy neurons from each layer:
\begin{itemize}
    \item \textit{Step 1.} Normalize the incoming weights of each neuron in the current layer and move the neurons with the same normalized weight into the same hash bucket.
    \item \textit{Step 2.} Merge the neurons of the same hash bucket into one neuron: Its incoming weights take the normalized weights of either one of the neurons and its outgoing weights take the sum of these neurons.  
    \item \textit{Step 3.} After the merging, check the flattened outgoing weights of each neuron: If the weights are a zero vector, then remove the neuron and its associated weights. 
\end{itemize}

% %%%%%%%% BEGIN OF RESULTS SUMMARY 
% % Table generated by Excel2LaTeX from sheet 'Sheet1'
\begin{table}[htbp]
  \centering
  \caption{Summary of the stealthiness of dummy neuron generation algorithms under elimination algorithms when the defender has a different level of knowledge on the watermarked model.}
  \scalebox{0.55}{
    \begin{tabular}{lcccc}
    \toprule
          & \multicolumn{2}{c}{Preliminary DN} & \multicolumn{2}{c}{Stealthier DN} \\
\cmidrule{2-5}          & \multicolumn{1}{c}{DN Stealthiness} & \multicolumn{1}{c}{Watermark Recovery} & \multicolumn{1}{c}{DN Stealthiness} & \multicolumn{1}{c}{Watermark Recovery} \\
    \midrule
    w/o. Original Model &   \ding{51}  & \ding{55} &  \ding{55}  & \ding{55} \\
    w/. Original Model &   \ding{51}  & \ding{51}  &  \ding{55} & \ding{55}   \\
    \bottomrule
    \end{tabular}}%
  \label{tab:summary_stealthiness}%
\end{table}%

% %%%%%%%% END OF RESULTS SUMMARY 


  By iterating the above procedure from the first hidden layer to the last one, the algorithm is expected to detect the dummy neurons and restore the original neural architecture from the obfuscated model. The detailed algorithm is shown in Algorithm \ref{alg:dn_elim}. Our experiments find, even though there is no dummy neurons after the elimination, the BER of the recognized watermark in the restored model remain over $50\%$ in Table \ref{tab:elim_table}, yielding no evidence for the claimed ownership. This is because, after the merging, the original scale of the parameter could still not be recovered, because the defender does not know how the attacker has rescaled the parameters of the dummy neurons during NeuronClique and NeuronSplit. Therefore, when the defender has no access to the original model, they could not adjust the parameter scale to cancel out the obfuscation effect. Hence, the original watermark is not recovered.
 
% To further improve the stealthiness of our injected neurons, we propose to remove the exploitable trace, i.e., the proportionality between the incoming weights of the dummy neurons, in an \textit{adaptive attack}. Specifically, we first generate their input weight randomly, then choose the bias of each injected neurons to keep them being activated under any valid inputs. Finally, we construct their outgoing weights by solving an under-determined linear equation system for preserving the original model behavior via vanishing the total contribution of the newly inserted dummy neurons. Appendix \ref{sec:app:stealthier_dn} provides the omitted technical details of the adaptive dummy neurons. In this design, the unparalleled incoming weights invalidate the elimination procedure above, which becomes ineffective in merging these stealthier DNs and thus leaves the size of relative parameters/activation maps unmatched for watermark extraction. Even after conducting the error-handling mechanisms in Section \ref{sec:eval}, Table \ref{tab:elim_table} shows the stealthier DN injection inhibits the verification procedures of most white-box watermarking schemes from neither successful detection of dummy neurons nor watermark recovery.

%%%%%%% BEGIN TABLE 
% % Table generated by Excel2LaTeX from sheet 'Sheet1'
% \begin{table}[htbp]
%   \centering
%   \caption{The scaled BER for each white-box watermarking scheme under different attack/defense modes.}
%   \scalebox{0.7}{
%     \begin{tabular}{ccccc}
%     \toprule
%          \multicolumn{1}{l}{DEF/ATK Mode} & \multicolumn{1}{l}{Uchida et al.} & \multicolumn{1}{l}{RIGA} & \multicolumn{1}{l}{IPR-GAN} & \multicolumn{1}{l}{Greedy} \\
%     \midrule
%     \multicolumn{1}{l}{\textbf{Skilled/Original}} & \textbf{52.99\%} & \textbf{54.83\%} & \textbf{62.37\%} & \textbf{51.79\%} \\
%     \multicolumn{1}{l}{\textbf{Skilled/Adaptive}} & -/\textbf{57.00\%}   & -/27.05\%   & -/\textbf{56.45\%}   & \textbf{53.48\%} \\
%     \multicolumn{1}{l}{\textbf{Fully/Adaptive}} & -/\textbf{57.00\%}   & -/27.05\%   & -/\textbf{56.45\%}   & \textbf{53.48\%} \\
%     \midrule
%     \multicolumn{1}{c}{Lottery} & \multicolumn{1}{c}{DeepSigns} & \multicolumn{1}{c}{IPR-IC} & \multicolumn{1}{c}{DeepIPR} & \multicolumn{1}{c}{Passport-Aware} \\
%     \midrule
%     \textbf{54.45\%} & \textbf{52.74\%} & \textbf{53.76\%} & \textbf{57.42\%} & \textbf{54.59\%} \\
%     -/\textbf{56.34\%}   & -/\textbf{52.56\%}   & -/\textbf{54.23\%}   & -/\textbf{55.10\%}   & -/\textbf{52.89\%} \\
%     -/\textbf{56.34\%}   & -/\textbf{52.56\%}   & -/\textbf{54.23\%}   & -/\textbf{55.10\%}   & -/\textbf{52.89\%} \\
%     \bottomrule
%     \end{tabular}}%
%   \label{tab:elim_table}%
% \end{table}%




% Table generated by Excel2LaTeX from sheet 'Sheet1'
\begin{table}[t]
  \centering
  \caption{The scaled BER for each white-box watermarking scheme under dummy neuron elimination.}
  \scalebox{0.85}{
    \begin{tabular}{ccccc}
    \toprule
         \multicolumn{1}{c}{\textbf{Schemes}} & \multicolumn{1}{c}{Uchida et al.} & \multicolumn{1}{c}{RIGA} & \multicolumn{1}{c}{IPR-GAN} & \multicolumn{1}{c}{Greedy} \\
    \midrule
    \multicolumn{1}{l}{\textbf{BER}} & \textbf{52.99\%} & \textbf{54.83\%} & \textbf{62.37\%} & \textbf{51.79\%} \\
    \midrule
    \multicolumn{1}{c}{Lottery} & \multicolumn{1}{c}{DeepSigns} & \multicolumn{1}{c}{IPR-IC} & \multicolumn{1}{c}{DeepIPR} & \multicolumn{1}{c}{Passport-Aware} \\
    \midrule
    \textbf{54.45\%} & \textbf{52.74\%} & \textbf{53.76\%} & \textbf{57.42\%} & \textbf{54.59\%} \\
    \bottomrule
    \end{tabular}}%
  \label{tab:elim_table}%
\end{table}%

%%%%%%% END TABLE 
\noindent\textbf{(c) A Fully Knowledgeable Defender.} Finally, we discuss the case when the defender also refers to the original watermarked model for watermark recovery. Then they can further compare the parameters of the original model with the obfuscated model in order to recover the order and the scale of the parameters. For the dummy neurons constructed in Section \ref{sec:dn_generation}, they can eliminate the dummy neurons and recover the watermark accuracy at the cost of some additional computing power. As the security research on white-box model watermark is an evolving game between the attacks and the defenses, we leave the study on more effective de-obfuscation approaches to future work. 

% In Appendix \ref{sec:app:stealthier_dn}, we develop a more stealthy dummy neuron construction which can adaptively confuse the defender when they try to determine which one of the neurons is from the original model. 
% In our preliminary results reported in Table \ref{tab:full_elim_table}, if the defender just merges the neurons with the same incoming weights as in Algorithm \ref{alg:dn_elim}, the watermark message is not recovered (i.e., BER > 50\%). 


% However, when using the more stealthy dummy neuron construction in the previous part, we find the attacker has the flexibility to set the incoming weights of the dummy neurons to be the same incoming weights of the original neurons. This then confuse the defender when he/she is trying to determine which one of the neurons is the original one. If the defender just merges the neurons with the same incoming weights, the watermark message could hardly be recovered. Our experimental results in the third row of Table \ref{tab:elim_table} validate this point. It is mainly because the outgoing weights of the injected dummy neurons in the new design has not to cancel each other out as for the original design.

% Nevertheless, we would also like to emphasize that referring to the original watermarked model is inconsistent with the basic settings of white-box watermarking. In fact, due to the concerns on opening forgery and ambiguity attack surfaces when an ownership claimer could submit a model for verification, the target model is usually not allowed to be involved in the watermark verification process. Moreover, we admit that the security research on white-box model watermark is again an evolving game between the attacks and the defenses. We leave the study on more effective de-obfuscation approaches as a potential future direction. 


% We present a case-by-case analysis on the stealthiness of preliminary and generalized DN generation algorithms when the attacker has the original watermarked model or not. Table \ref{tab:summary_stealthiness} summarizes our main findings. 


% \noindent$\bullet$\textbf{ Case A. Stealthiness of Preliminary DN.} We first apply the shuffling and scaling invariant transforms for the dummy neuron generation in Section 6.2 to construct Preliminary DN. As the DN elimination procedure shows above, to detect and eliminate the dummy neurons in all the neural layers incurs $O(nd)$ computational complexity. When the defender invests such computational costs to run the elimination algorithm, most of the dummy neurons injected by our approach in the original manuscript would be removed, which results in the following cases depending on whether the defender has access to the original watermarked model. Table \ref{tab:CaseA-12} summarizes the results of out evaluation.



% \noindent\textbf{Case A-1} \textit{(w/o. the original model).}
% Even if most of the dummy neurons are removed, the permutation and the scaling transformations on the weights are hardly canceled out when the defender does not have access to the original version of the watermarked model, which is a common setting in previous works on model watermarking. In the case, the identity information can hardly be recovered, resulting a failed watermark verification. The results in Table \ref{tab:CaseA-12} validate the above point. As is shown, all the nine white-box watermarking schemes is unable to extract the expected signature from the protected model as the scaled BER is higher than 50\% according to \cite{sokwatermark}.

% \noindent\textbf{Case A-2} \textit{(w/. the original model).}
% The situation becomes different when we further assume the defender has the original watermarked model at hand. In this case, after dummy neuron elimination, he/she can further compare the parameters of the original model with the obfuscated model in order to recover the order and the scale of the parameters. From our perspective, for the $i$-th layer, this can done via running the Hungarian algorithm based on the pairwise cosine distance of neurons between the two models. Table \ref{tab:CaseA-12} reports the BER of the obfuscated model after the additional defense steps above. According to the results, we admit that when a skilled defender also has access to the original model, he/she can almost fully eliminate the preliminary DN and recover the watermark accuracy at the cost of some additional computing power.

% However, we would also like to emphasize that referring to the original watermarked model is inconsistent with the basic settings of white-box watermarking. In fact, due to the concerns on opening forgery and ambiguity attack surfaces when an ownership claimer could submit a model for verification, the target model is usually not allowed to be involved in the watermark verification process. Nevertheless, in the following, we present a novel dummy neuron generation algorithm to 

% \noindent$\bullet$\textbf{ Case B. Stealthiness of Generalized DN.}

% However, the above defense algorithm has the major limitation as follows: 
% (i) The detection complexity is under the control of the adversary. In the worst case, the above procedure incurs $O(\sum_{l=1}^{H}(N_l+d_l)^2)$ floating-point operations for calculating the pairwise distance of neurons in all the hidden layers, where $d_l$ is the number of dummy neurons injected by the adversary. In other words, the adversary can inhibit the success of the fully knowledgeable defender by maliciously injecting a large number of dummy neurons. Therefore, the detection algorithm would suffer from an intractable computational cost for eliminating the dummy neurons. (ii) Moreover, we experimentally find 
% even after eliminating the dummy neurons generated by NeuronClique and merges the ones generated by NeuronSplit, the defender can hardly restore the original weights of the merged neurons due to the shuffling and scaling invariance imposed in NeuronSplit. Therefore, the BER of the restored DNN remains over $50\%$, yielding no evidence for existing model piracy.  
 

%%%%%%%%%%%%%%%% $O(\sum_{i=1}^{H}(N_{i=1}^{H})d_i)$ FLOPS



\section{Evaluation}
\label{sec:evaluation}
\section{Results}
\label{sec:results}

\begin{figure}[t]
    \centering
    \includegraphics[width=\textwidth]{figs/Nexf-3D-compare.png}
    \caption{Comparison of 3D oral reconstruction by different methods from PX imaging. The reconstruction results are shown by maximum projection to compare density details. We could easily find that our method show the best performance with clear density density distributions and teeth boundaries.
    }
    \label{fig:3d_compare}
\end{figure}

\subsection{Comparison of 3D reconstruction with other models}
We compare Oral-NeXF with existing deep-learning-based tomography models, and present the results in Figures~\ref{fig:3d_compare} and \ref{tab:compare}, where we observe that Oral-NeXF achieves the best performance. Oral-3D \cite{oral_3d}, ResCNN \cite{x_to_3d}, and GAN \cite{gan} are trained using paired images generated from the reserved 60 cases. Specifically, GAN is trained using the same encoding-decoding network as ResEncoder and the same discriminator as Oral-3D but without any curve information. Moreover, the NAF \cite{naf} model is trained similarly to our work, but utilizes a trainable hash embedding for position encoding and a 3D attenuation coefficient predictor as the neural field function. As shown in the figures, Oral-NeXF achieves remarkable performance with clear details, without requiring prior expert knowledge or additional patient data.

\subsection{Experiment analysis}
Combining the results presented in Figure~\ref{fig:3d_compare} and Table~\ref{tab:compare}, we observe that Oral-NeXF achieves state-of-the-art performance. In contrast, ResEncoder and GAN can only restore the curved shape by learning from numerous paired images. Oral-3D achieves better performance in shape restoration and detail reconstruction, mainly due to prior knowledge of the dental arch shape information that enables the generator to focus on learning inverse projection. On the other hand, NAF fails to generate a detailed structure and contains much noise in the surroundings. As mentioned earlier, a general neural field function with 3D coordinate input and single-head prediction cannot fit PX imaging. This is also demonstrated in our ablation study.


\subsection{Ablation Study}
\label{sec:ablation}
We conduct an ablation study to evaluate the contribution of each component in Oral-NeXF. We use the letters M, D, and S to denote the experiments: 1) replacing the multi-head field function with a single-head predictor and taking in 3D coordinates as input for the positional encoder; 2) using a fixed sampling rate of $N_s=1$ to generate sample points on projection rays; 3) changing the formula in Equation (\ref{eq:render_discrete}) to a weighted sum function that strictly follows the Beer–Lambert law by taking the voxel intensity as Hounsfield units. As shown in Table~\ref{tab:ablation}, the proposed dynamic sampling method plays the most crucial role in 3D reconstruction. This finding is consistent with the experiment in NeRF, where the model uses a coarse network to predict the particle density distribution for high-resolution generation. The drop in Dice and SSIM also highlights the importance of multi-head prediction and soft rendering in Oral-NeXF.

\begin{table}[tp]
    \centering
    \caption{Evaluation of 3D oral reconstruction by PSNR, SSIM, and Dice.}
    \label{tab:compare}
    \setlength\tabcolsep{2pt}
    \begin{tabular}{p{1.8cm}<{\centering}p{1.8cm}<{\centering}p{2.0cm}<{\centering}p{1.8cm}<{\centering}p{1.8cm}<{\centering}p{1.8cm}<{\centering}}
    \hline
    Method&Oral-3D&ResEncoder
    &GAN&NAF&\textbf{Ours}\cr
    \hline
    PSNR&18.59$\pm$0.70 &18.26$\pm$0.62&16.71$\pm$0.89&18.35$\pm$0.86&18.26$\pm$0.50\cr
    SSIM(\%)&76.88$\pm$1.26 &72.67$\pm$1.56&75.10$\pm$1.46&60.69$\pm$2.69&76.67$\pm$1.72\cr
    Dice(\%)&65.94$\pm$4.24&62.52$\pm$5.56&63.96$\pm$7.03&57.20$\pm$3.94&72.09$\pm$3.63\cr
    \hline
    Overall&78.60&75.49&76.93&65.93&\textbf{80.02}\cr
    \hline
    \end{tabular}
\end{table}

\begin{table}[tp]
    \centering
    \caption{Ablation study by removing each component in Oral-NeXF. M: Multi-head Prediction, D: Dynamic Sampling, S: Soft Rendering}
    \label{tab:ablation}
    \setlength\tabcolsep{2pt}
    \begin{tabular}
     {p{0.8cm}<{\centering}p{0.8cm}<{\centering}p{0.8cm}<{\centering}p{2.2cm}<{\centering}p{2.2cm}<{\centering}p{2.2cm}<{\centering}p{2.2cm}<{\centering}}
    \hline
    M&D&S
    &PSNR&SSIM(\%)&Dice(\%)&Overall\cr
    \hline
    \xmark&\cmark&\cmark&17.12$\pm$0.86&71.28$\pm$3.38&61.03$\pm$6.07&72.64(-7.38)\cr
    \cmark&\xmark&\cmark&13.02$\pm$0.52&50.83$\pm$0.65&31.18$\pm$4.70&49.03(-30.99)\cr
    \cmark&\cmark&\xmark&15.80$\pm$0.38&58.72$\pm$0.90&53.01$\pm$3.88&63.79(-16.43)\cr
    \hline
    \end{tabular}
\end{table}



\subsection{Data Visualization}
Several data visualizations exist for examining at a deeper level how machine learning classifiers behave. Taking the MalConv model that was trained, we created two data visualizations using t-Distributed Stochastic Neighbor Embedding (t-SNE) for better insight.

%\textbf{PCA.} PCA is a method of reducing the dimensions of a dataset to the principal components. The last layer of the convolution network was used, reducing 128 dimensions to 3 principal components, then graphing them on a three dimensional plane. It should be noted that PCA is a linear, deterministic visualization, and as a result may not model complex data behavior. 


%Groupings of classifications can be seen in figure \ref{fig:PCA}, dark blue representing benign files, red representing classic malware, and cyan representing adversarial malware.

\begin{figure}[h!]
\centering

\subfloat[\label{fig:tsne_a}Benign vs. Malware vs. Adversarial Malware]{\includegraphics[width = 0.4\textwidth]{Images/tsne_benign_malware.png}} \\

\subfloat[\label{fig:tsne_b}Original Malware vs. Adversarial Malware (UAP) vs. Adversarial Malware(Input-specific)]{\includegraphics[width = 0.4\textwidth]{Images/tsne_i-spec_uap_mw.png}}

\caption{t-SNE plotting on outputs from penultimate layer}
\label{fig:tsne}
\end{figure}

\textbf{t-SNE.} t-SNE is a method of reducing the dimensions of a dataset, but it is a non-linear transformation, and is non deterministic. It can however, help model more complex interactions in the data. 

\iffalse
\begin{figure}[h!]
        \centering
        \includegraphics[width=0.45\textwidth]{Images/tsne_benignvsmalware.png}
        \caption{t-SNE Representation for Benign vs. Malware vs. Adversarial Malware}
        \label{fig:tsne1}
\end{figure}
\fi

At first, we randomly selected some benign, original malware and their adversarial version generated using input-specific patch. We fed these samples into our `original MalConv' model and extracted the output from the penultimate layer. Then these outputs were reduced to two-dimensional space using t-SNE. Clustering groups appear in the plotting of Figure \ref{fig:tsne_a}. Interestingly, the adversarial malware falls into two distinct groups whereas we expected to have one. This implies that -- the  generation of adversarial malware by FGSM append attack can go into two directions. This might help us in future to design more robust model that can identify such groups that are different from original malware and benign. Moreover, some benign files (green color in Figure \ref{fig:tsne_a}) can be noticed in the cluster of original malware which indicates the false-positive cases of the MalConv model.
%the classifier is successfully classifying adversarial malware, it is classifying it into two separate groupings. The reason for this is not immediately clear, and warrants further investigation. 

\iffalse
\begin{figure}[h!]
        \centering
        \includegraphics[width=0.45\textwidth]{Images/tsne_malware_vs_adv_malware.png}
        \caption{t-SNE Representation for Original Malware vs. Adversarial Malware (UAP) vs. Adversarial Malware(Input-specific)}
        \label{fig:tsne2}
\end{figure}
\fi




We plotted another graph for original malware, adversarial malware generated by UAP and input-specific AP following the same setup (see Figure \ref{fig:tsne_b}). We expected to have 3 clusters in this case. Surprisingly, we discovered another cluster consisted of both type of adversarial malware (the leftmost cluster in Figure \ref{fig:tsne_b}). This indicates that some of the input-specific patch and universal patch generated malware has nearly similar representation, which might be the result of using FGSM for both of the attacks.   

\section{Conclusion}
\section{Conclusion}\label{sec:conclusion}
In this work, we focus on addressing the fundamental challenge of OOD detection tasks, which is how to fully understand the semantic discrepancy between the ID/OOD samples. We reveal that the key to success in the realistic SCOOD task is to allocate as many ID samples in the unlabeled set correctly as possible. To this end, we propose a novel uncertainty-aware optimal transport scheme that introduces class-specific energy scores as guidance for effective label assignment. Experimental results show that our method achieves better performance than previous state-of-the-art methods on SCOOD benchmarks.

\textbf{Limitations.} In addition to temperature scaling, other techniques such as feature clipping applied in ReAct~\cite{sun2021react} also enhance the performance of energy score, so how to obtain an OOD score that best fits the SCOOD task can be further explored. Moreover, a setting highly related to SCOOD has been proposed in \cite{katz2022training} and formulated as a constrained optimization problem. We will also theoretically analyze these practical OOD settings in our feature work.

% \section*{Acknowledgments}
\textbf{Acknowledgments.} 
This work is supported by National Key R\&D Program of China under Grant 2020AAA0105701, National Natural Science Foundation of China (NSFC) under Grants 61872327, Major Special Science and Technology Project of Anhui, National Natural Science Foundation of China (62033012) and Ant Group through Ant Research Intern Program.


% Note use of \abovespace and \belowspace to get reasonable spacing
% above and below tabular lines.



% In the unusual situation where you want a paper to appear in the
% references without citing it in the main text, use \nocite
\nocite{langley00}

\bibliography{main}
\bibliographystyle{icml2023}


%%%%%%%%%%%%%%%%%%%%%%%%%%%%%%%%%%%%%%%%%%%%%%%%%%%%%%%%%%%%%%%%%%%%%%%%%%%%%%%
%%%%%%%%%%%%%%%%%%%%%%%%%%%%%%%%%%%%%%%%%%%%%%%%%%%%%%%%%%%%%%%%%%%%%%%%%%%%%%%
% APPENDIX
%%%%%%%%%%%%%%%%%%%%%%%%%%%%%%%%%%%%%%%%%%%%%%%%%%%%%%%%%%%%%%%%%%%%%%%%%%%%%%%
%%%%%%%%%%%%%%%%%%%%%%%%%%%%%%%%%%%%%%%%%%%%%%%%%%%%%%%%%%%%%%%%%%%%%%%%%%%%%%%


\newpage
\newpage
\appendix
\onecolumn
\section{Experimental Setup}
\subsection{MalConv Model} 
\label{app_malconv}
We followed the same implementation as original paper \cite{raff2018malware} where the vocab size in embedding layer is 256, the window size and number of filters in convolution layers are 500 and 128, respectively.
We changed the input length of the model to 250K from 200K so that it can cover more files.
We used the ‘Adam’ optimizer and ‘binary-cross entropy’ loss function for our model implementation, and trained it for 50 epochs with 512 batch size. We saved the best model with respect to the validation accuracy.

%\subsection{Patch Append Attacks} 
%\label{app_attack}

\subsection{`smoothed-MalConv' Model} 
\label{app_smooth_malconv}
We used the `original MalConv' as the base classifier, and changed its input dimension from 250K to $w$, i.e., the ablation size. For example, when $w=50K$, there are 5 ablated sequences in the ablation set $S(x)$ and we trained 5 MalConv models of input dimension $w$ for each ablated sequence. The labels for the ablated sequences are the same as the original label of the file. 

We did not change the architecture and any parameters, e.g., vocab size, window size, number of filters, etc., of the `original MalConv' model. While training, we used Adam optimizer and `binary-cross entropy (without reduction)' as the loss function. We trained `smoothed-MalConv' for four different ablation sizes. All the models were trained for 10 epochs with a batch size of 512.

\section{De-randomized Smoothing Defense}

\label{app_defense}


\iffalse
\begin{figure}[h]
        %\centering
        \includegraphics[width=0.5\textwidth]{Images/chart_vary_UAP_2.png}
        \caption{Standard Accuracy vs UAP size}
        \label{fig:chart_vary_uap_std}
\end{figure}

\begin{figure}[h]
        %\centering
        \includegraphics[width=0.5\textwidth]{Images/chart_vary_UAP_cert.png}
        \caption{Certified Accuracy vs UAP size}
        \label{fig:chart_vary_uap_cert}
\end{figure}

\begin{figure}[h]
        %\centering
        \includegraphics[width=0.5\textwidth]{Images/chart_std_acc_vs_w.png}
        \caption{Standard Accuracy vs Window size}
        \label{fig:chart_vary_std_w}
\end{figure}

\begin{figure}[h]
        %\centering
        \includegraphics[width=0.5\textwidth]{Images/chart_cert_acc_vs_w.png}
        \caption{Certified Accuracy vs Window size}
        \label{fig:chart_vary_cert_w}
\end{figure}
\fi


\begin{figure}[h!]
\centering

\subfloat[\label{fig:chart_vary_uap_std}Standard Accuracy vs UAP size]{\includegraphics[width = 0.45\textwidth]{Images/chart_vary_UAP_2.png}} 
\hfill
\subfloat[\label{fig:chart_vary_uap_cert}Certified Accuracy vs UAP size]{\includegraphics[width = 0.45\textwidth]{Images/chart_vary_UAP_cert.png}}
\\

\subfloat[\label{fig:chart_vary_std_w}Standard Accuracy vs Window size]{\includegraphics[width = 0.45\textwidth]{Images/chart_std_acc_vs_w.png}}
\hfill
\subfloat[\label{fig:chart_vary_cert_w}Certified Accuracy vs Window size]{\includegraphics[width = 0.45\textwidth]{Images/chart_cert_acc_vs_w.png}}



\caption{Line Charts for Standard and Certified Accuracy of `smoothed-MalConv' model}
\label{fig:charts}
\end{figure}

%You can have as much text here as you want. The main body must be at most $8$ pages long.
%For the final version, one more page can be added.
%If you want, you can use an appendix like this one, even using the one-column format.
%%%%%%%%%%%%%%%%%%%%%%%%%%%%%%%%%%%%%%%%%%%%%%%%%%%%%%%%%%%%%%%%%%%%%%%%%%%%%%%
%%%%%%%%%%%%%%%%%%%%%%%%%%%%%%%%%%%%%%%%%%%%%%%%%%%%%%%%%%%%%%%%%%%%%%%%%%%%%%%


\end{document}


% This document was modified from the file originally made available by
% Pat Langley and Andrea Danyluk for ICML-2K. This version was created
% by Iain Murray in 2018, and modified by Alexandre Bouchard in
% 2019 and 2021 and by Csaba Szepesvari, Gang Niu and Sivan Sabato in 2022.
% Modified again in 2023 by Sivan Sabato and Jonathan Scarlett.
% Previous contributors include Dan Roy, Lise Getoor and Tobias
% Scheffer, which was slightly modified from the 2010 version by
% Thorsten Joachims & Johannes Fuernkranz, slightly modified from the
% 2009 version by Kiri Wagstaff and Sam Roweis's 2008 version, which is
% slightly modified from Prasad Tadepalli's 2007 version which is a
% lightly changed version of the previous year's version by Andrew
% Moore, which was in turn edited from those of Kristian Kersting and
% Codrina Lauth. Alex Smola contributed to the algorithmic style files.
