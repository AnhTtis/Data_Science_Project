In this paper, we explore the effectiveness of modern machine learning-based static malware classifiers against patch based attacks.
%
Building on previous research, we apply both input-specific patch attack and universal patch attack against the MalConv malware classifier. 
%
We demonstrate that MalConv has had 97.9\% success rate in detecting non-adversarial malware.
%
Applying input-specific and a universal patch based attacks, has allowed 94.3\% and 83.74\% of adversarial malware to evade detection, respectively.
%
To the best of our knowledge, this is the first time universal patch attacks have been implemented for malware evasion on byte level, as most prior research focuses on hand-crafted features in malware detection.
%
The novel application of de-randomized smoothing to create a new robust model, named `smoothed-MalConv' has allowed us to detect $72\% \sim 93\%$ of adversarial malware, a significant improvement an un-defended model.
%
This strategy is also effective against input-specific patch attacks, as well.

%had to come up with `window-ablation' technique to transform the smoothing strategy from vision to byte files.

We would like to conclude by highlighting several areas and future directions our work identifies:

\begin{itemize}[noitemsep,topsep=0pt, leftmargin=*]
\itemsep0em 
    \item Towards a more realistic threat model, patch attacks need to be transferable to detectors unknown to the adversary;
    \item Incorporating recent defenses from computer vision other than de-randomized smoothing into MalConv models;
    \item Additional detection and reasoning for the different adversarial malware clusters first seen in our data visualizations.
\end{itemize}

Malware detection is inherently is an arms-race and we believe our work will spark further research by developing an effective and practical attack; and by adapting a powerful certified defense to this domain, which lacks an established defense to prevent patch attacks.


%Our work introduces new directions to gain a deeper understanding of malware evasion and classification. 

%Our work demonstrates novel work on both the attack and and defence of static malware classifiers. 

