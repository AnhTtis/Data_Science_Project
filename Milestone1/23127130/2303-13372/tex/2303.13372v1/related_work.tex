

%We repeat the experimental results of the MalConv classifier first outlined in Raff\cite{raff2018malware}, upgrading it to work with the current versions of machine learning libraries. We also apply the strategies outlined in the smoothed vision transform paper\cite{salman2022certified} successfully to our malware classifier. As far as we are aware, we are the first that applies this strategy to the raw byte sequence of malware classification instead of using a feature driven approach.

\textbf{Malware classification.} There have been several studies of how malware executables can be classified using machine learning. As early as 2001 \cite{924286} proposed a data mining technique for malware detection using three different types of static features.
\cite{MalwareImages} demonstrated malware family classification technique using the conversion of binary file to an image. Subsequently, CNN-based techniques for malware detection became popular among security researchers \cite{kalash2018malware} \cite{yan2018detecting}. Eventually, \cite{raff2018malware} proposed a static classifier, named MalConv, that takes raw byte sequence and detects malware using a convolution neural network. This model is considered as one of the state-of-the-art model and hence we used this as our base model.

\textbf{Evasion Techniques.}
Other studies seek to showcase evasion techniques present in the MalConv classifier, such as \cite{8553214} into the utilization of a gradient descent attack on the MalConv model. The paper seeks to add small noise at the end of malware binaries to change the model prediction. Though Kolosnjaji reported success in evasion with modifying as little as 1\% of the original input bytes, the strategy was computationally heavy. It was improved by \cite{kreuk2018deceiving} where they implemented the fast gradient sign method (FGSM), taken from \cite{goodfellow2014explaining}. This work was more improved by taking the vulnerabilities of the model by \cite{suciu2019exploring}. 

Though input-specific patch attack for malware has been heavily explored, universal patch attack is still unexplored. Recently, \cite{labaca2021realizable} proposed a universal adversarial patch (UAP) attack on malware but their work was on feature-space. When they applied tranformation of UAP from feature-space to problem space, they found it less effective for windows malware binaries. In our paper, we address these limitations of existing attack by applying UAP attack on byte-level for windows malware.

\textbf{De-randomized Smoothing.} The computer vision world has had their own problems and solutions to adversarial classification evasion of image based classifiers. Certified robustness against adversarial attacks for image classification was introduced by  \cite{levine2020randomized} \cite{levine2020robustness}. 
Recently, \cite{dosovitskiy2020image} improved this defense strategy by using a vision transformer.
%Recently, \cite{salman2022certified} improved this work by using a vision transformer for the backbone of the derandomized smoothing defense from \cite{levine2020randomized}. 
Though these defense approaches are being heavily used in image classification based models, it is still untouched by the security community. We leveraged the concept of derandomized smoothing by band ablation and adapted this to our MalConv model. To our knowledge, we are the first to use such defense strategy in malware classification.

%Malware classifiers using vision transformers, i.e. converting the executable to an image before classification have been demonstrated in Nataraj\cite{MalwareImages}. These systems are still vulnerable to the patch based evasion techniques, but the application of certain computer vision strategies become applicable.