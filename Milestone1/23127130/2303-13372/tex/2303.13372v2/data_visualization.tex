


\subsection{Data Visualization}
Several data visualizations exist for examining at a deeper level how machine learning classifiers behave. Taking the MalConv model that was trained, we created two data visualizations using t-Distributed Stochastic Neighbor Embedding (t-SNE) for better insight.

%\textbf{PCA.} PCA is a method of reducing the dimensions of a dataset to the principal components. The last layer of the convolution network was used, reducing 128 dimensions to 3 principal components, then graphing them on a three dimensional plane. It should be noted that PCA is a linear, deterministic visualization, and as a result may not model complex data behavior. 


%Groupings of classifications can be seen in figure \ref{fig:PCA}, dark blue representing benign files, red representing classic malware, and cyan representing adversarial malware.

\begin{figure}[h!]
\centering

\subfloat[\label{fig:tsne_a}Benign vs. Malware vs. Adversarial Malware]{\includegraphics[width = 0.4\textwidth]{Images/tsne_benign_malware.png}} \\

\subfloat[\label{fig:tsne_b}Original Malware vs. Adversarial Malware (UAP) vs. Adversarial Malware(Input-specific)]{\includegraphics[width = 0.4\textwidth]{Images/tsne_i-spec_uap_mw.png}}

\caption{t-SNE plotting on outputs from penultimate layer}
\label{fig:tsne}
\end{figure}

\textbf{t-SNE.} t-SNE is a method of reducing the dimensions of a dataset, but it is a non-linear transformation, and is non deterministic. It can however, help model more complex interactions in the data. 

\iffalse
\begin{figure}[h!]
        \centering
        \includegraphics[width=0.45\textwidth]{Images/tsne_benignvsmalware.png}
        \caption{t-SNE Representation for Benign vs. Malware vs. Adversarial Malware}
        \label{fig:tsne1}
\end{figure}
\fi

At first, we randomly selected some benign, original malware and their adversarial version generated using input-specific patch. We fed these samples into our `original MalConv' model and extracted the output from the penultimate layer. Then these outputs were reduced to two-dimensional space using t-SNE. Clustering groups appear in the plotting of Figure \ref{fig:tsne_a}. Interestingly, the adversarial malware falls into two distinct groups whereas we expected to have one. This implies that -- the  generation of adversarial malware by FGSM append attack can go into two directions. This might help us in future to design more robust model that can identify such groups that are different from original malware and benign. Moreover, some benign files (green color in Figure \ref{fig:tsne_a}) can be noticed in the cluster of original malware which indicates the false-positive cases of the MalConv model.
%the classifier is successfully classifying adversarial malware, it is classifying it into two separate groupings. The reason for this is not immediately clear, and warrants further investigation. 

\iffalse
\begin{figure}[h!]
        \centering
        \includegraphics[width=0.45\textwidth]{Images/tsne_malware_vs_adv_malware.png}
        \caption{t-SNE Representation for Original Malware vs. Adversarial Malware (UAP) vs. Adversarial Malware(Input-specific)}
        \label{fig:tsne2}
\end{figure}
\fi




We plotted another graph for original malware, adversarial malware generated by UAP and input-specific AP following the same setup (see Figure \ref{fig:tsne_b}). We expected to have 3 clusters in this case. Surprisingly, we discovered another cluster consisted of both type of adversarial malware (the leftmost cluster in Figure \ref{fig:tsne_b}). This indicates that some of the input-specific patch and universal patch generated malware has nearly similar representation, which might be the result of using FGSM for both of the attacks.   