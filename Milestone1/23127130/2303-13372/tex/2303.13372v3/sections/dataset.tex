\label{sec:dataset}

%\citep{cao2020benign} -- This paper shows that 29.5\% of popular features are generated by benign samples. Thus, it is easier for a machine learning model to learn these benign features and make a decision based on the existence of such features. So, a diverse set of benign samples should cover more features and can make a model more robust which led us to create a diverse dataset to train and evaluate our model.

%\citep{anderson2018ember} -- EMEBR paper discusses why it is tough to collect benign samples due to the copyright law and legal restrictions. And they did not publish the raw executables of the benign files.

%\citep{downing2021deepreflect} -- This paper discusses the importance of diverse benign samples, and they crawled .CNET and MSI to collect benign executables whereas we went further by doing that on other sources (sourceforge, NET, softonic, etc.) including .CNET.

%\citep{yang2021cade, jordaney2017transcend, barbero2022transcending} -- papers on concept drift in malware detection.

Like other domains, malware detection suffers from concept drift too. Previously, \citet{yang2021cade, jordaney2017transcend, barbero2022transcending} demonstrated how concept drift can have a disastrous impact on ML-based malware detection. Therefore, we used $3$ datasets from different times in this work (Table \ref{table:dataset}). However, in the malware domain, having a large dataset to train a machine learning (ML) model may not be enough as maintaining diversity and recency is also crucial \citep{cao2020benign, downing2021deepreflect}.
We found that models trained without diverse benign samples can have a very high false positive rate (see details in Appendix \ref{app:pace_degrade}).
%Otherwise, ML model can get inclined to learn spurious features related to specific signatures in benign binaries (see Appendix \ref{app:pace_degrade}). 
%For example, \citet{cao2020benign, downing2021deepreflect} emphasized the importance of diversity in benign dataset. 

%\subsection{PACE Dataset}
Despite the importance of diverse benign samples, unfortunately, most prior works (\citet{anderson2018ember, downing2021deepreflect}) could not publish raw executables of benign files due to copyright and legal restrictions.
For this work, we crawled popular free websites, e.g., SourceForge, CNET, Net, Softonic, etc., to collect a diverse benign dataset of size 15.5K (Table \ref{table:benign_dataset}), naming \textbf{PACE} (Publicly Accessible Collection(s) of Executables).
We collected the malware from \href{https://virusshare.com/}{VirusShare} at the same time (August 2022) as benign files.
Following the common practice and guidelines, we are publishing the URLs along with the MD5 hash for each raw benign file in our dataset (see Appendix \ref{app:dataset} for more details). We hope this will help researchers to recreate the dataset easily and experiment with a better representative of real-world settings in the future.\footnote{PACE malware samples will also be provided upon request.}

%Besides diversity, we also consider the temporality of datasets in this work. Prior works \citep{yang2021cade, jordaney2017transcend, barbero2022transcending} demonstrated how concept drift can have disastrous impact in malware detection. Therefore, we used $3$ datasets from different times in this work (Table \ref{table:dataset}) for training and made sure that the test set comes from the latest one during evaluation. We collected the malware files that was added to VirusShare \footnote{\url{https://virusshare.com/}} at the same time when we collected the benign files (August, 2022).  

\begin{table}[!htb]
    
    \begin{minipage}{.6\linewidth}
      \caption{Datasets used in this work with collection time, size, and public availability of raw executables}
      \centering
        \resizebox{\linewidth}{!}{
        \begin{tabular}{ llcccc }
        \hline
          Dataset & Collection  & \multicolumn{3}{c}{Number of Binaries} & Public  \\\cline{3-5}
          Name & Time & Malware & Benign & Total & Availability \\
        \hline 
        Ember  & 2017 & 400K & 400K & 800K & \XSolidBrush \\
        VTFeed & 2020 & 139K & 139K & 278K & \XSolidBrush\\
        PACE (Our) & 2022 & 15.5K & 15.5K & 31K & \CheckmarkBold\\
         \hline 
        Total & & 554.5K & 554.5K & 1.1M\\
          \hline
        \end{tabular}
        }
    
    \label{table:dataset}
    \end{minipage}%
    \begin{minipage}{.4\linewidth}
      \centering
        \caption{PACE (Benign) Dataset}
        \resizebox{\linewidth}{!}{
        \begin{tabular}{ lc }
        \hline
          Source  & Number of Binaries \\
         \hline
        SourceForge  & 7,865 \\
        CNET  & 3,661 \\
        Net  & 2,534 \\
        Softonic  & 1,152 \\
        DikeDataset  & 1,082 \\
        Netwindows  & 185 \\
        Manually Obtained   & \multirow{2}{*}{89} \\
        from Windows OS & \\
         \hline
        Total & 15,568\\
          \hline
        \end{tabular}
        }
    
    \label{table:benign_dataset}
    \end{minipage} 
    %\caption{Global caption}
\end{table}


We used a MalConv model pre-trained on Ember \citep{anderson2018ember} dataset provided by the \href{https://en.wikipedia.org/wiki/Endgame,_Inc.}{Endgame Inc.} 
Then we used this model to re-train the MalConv, MalConv (NonNeg), and our DRSM models on both VTFeed  and PACE (our) dataset.\footnote{The authors of \citep{lucas2021malware} assisted in training models on VTFeed, which we could not have done by ourselves since VTFeed is not publicly accessible} We split our dataset into 70:15:15 ratios for train, validation, and test sets, respectively. During evaluation, we made sure that test samples came from the latest dataset (PACE) only.
%We trained and evaluated our models using multiple NVIDIA RTX A4000 and 2 RTX A5000 gpus. 
For more details about model implementation, see Appendix \ref{app:model_impl}.

\iffalse

\begin{table}[h]
\centering
\begin{tabular}{ llcccc }
    \hline
      Dataset & Collection  & \multicolumn{3}{c}{Number of Binaries} & Public  \\\cline{3-5}
      Name & Time & Malware & Benign & Total & Availability \\
    \hline 
    Ember  & 2017 & 400K & 400K & 800K & \XSolidBrush \\
    VTFeed & 2020 & 139K & 139K & 278K & \XSolidBrush\\
    Our Dataset & 2022 & 15.5K & 15.5K & 31K & \CheckmarkBold\\
     \hline 
    Total & & 554.5K & 554.5K & 1.1M\\
      \hline
\end{tabular}
\caption{Datasets used for our work}
\label{table:dataset}
\end{table}

\begin{table}[h]
  %\centering
\begin{center}
\begin{tabular}{ ll }
    \hline
      Source  & Number of Binaries \\
     \hline
    SourceForge  & 7,865 \\
    CNET  & 3,661 \\
    Net  & 2,534 \\
    Softonic  & 1,152 \\
    DikeDataset  & 1,082 \\
    Netwindows  & 185 \\
    Manually Obtained from Windows OS  & 89 \\
     \hline
    Total & 15,568\\
      \hline
\end{tabular}
\end{center}
\caption{Our Published Benign Dataset}
\label{table:benign_dataset}
\end{table}

\fi