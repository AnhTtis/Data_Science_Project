%Machine Learning (ML) models have been utilized for malware detection for over two decades. Consequently, this gave rise to an ongoing arms race between malware authors and antivirus systems, and researchers have proposed mechanisms for defending malware-detection models against evasion attacks. However, the existing defenses against evasion attacks suffer from performance degradation and/or can defend against only specific attacks, which make them less practical in real-life settings. We study this trade-off under multiple adversarial settings and present DRSM (De-Randomized Smoothed MalConv) as a certified defense, which combines our adapted *de-randomized smoothing* technique with state-of-the-art CNN-based malware classifiers. Benefit from the *window ablation smoothing* scheme we propose, the impact of adversarial perturbation is mitigated while the malicious features of a malware file can still be learned. We analyze how different ablation parameters in DRSM can impact standard and certified accuracy, and compare it to the base classifier. To our knowledge, we are the first to introduce certified robustness in the realm of malware detection. Beyond our theoretical formulations, we extensively evaluate our DRSM model against $9$ different attacks in both white-box and black-box settings. Notably, we deliberately include some practical attacks that fall outside of our threat model to showcase the applicability of our proposed method. We then discuss some insights gained from our results. Moreover, due to the scarcity of publicly available benign datasets in the malware domain, we collected $15.5K$ very recent benign raw executables from different sources throughout this work. We are going to publish the dataset following the common standard and guidelines, to allow future research to curate a diverse set of benign programs for experiments that are more representative of the real-world. To facilitate the reproducibility of our findings, we build our framework on top of 'secml-malware' Python library, and open-source the implementation.