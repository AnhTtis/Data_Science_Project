\label{app:dataset}

\subsubsection{Dataset Details}

Our diverse benign dataset contains benign raw executables from $7$ different sources. Figure \ref{fig:cdf_benign_dataset} shows the cumulative distribution function (CDF) of the file sizes of our benign files. 

\begin{figure}[h]
        \centering
        \includegraphics[width=0.60\linewidth]{images/benign_cdf_plot.png}
        \caption{CDF plot of file sizes for our published benign dataset}
        \label{fig:cdf_benign_dataset}
\end{figure}

\textbf{Data Format. } For each benign raw executables, we are going to publish the URL link to download it from with its MD5 hash for the response (See \texttt{`dataset'} folder in our supplementary material). For example, one line from our csv file is -- 

\begin{table}[h]
  %\centering
\begin{center}
\begin{tabular}{ c|c }
    \hline
    URL & MD5 hash  \\
    \hline 
    \href{https://sourceforge.net/projects/pdfcreator/files/PDFCreator/PDFCreator%200.7/PDFCreator-Setup-0_7_0.exe/download}{\shortstack[l]{https://sourceforge.net/projects/pdfcreator/ \\ files/PDFCreator/PDFCreator\%200.7/ \\ PDFCreator-Setup-0\_7\_0.exe/download}}    & {\texttt{afaf0caffeff781f6070f2a9aeb54bdf}} \\
     \hline 
\end{tabular}
\end{center}
%\caption{Other Benign Datasets}
\label{table:other_dataset}
\end{table}




\subsubsection{Other Datasets}
\begin{table}[h]
  %\centering
  \caption{Other Benign Datasets and their public availability}
\begin{center}
\begin{tabular}{ lc }
    \hline
    Benign Datasets & Public Availability \\
    \hline 
    PACE (Our) & Raw Binary Executable \\
    Ember & only Feature Vector  \\
    VTFeed & \XSolidBrush \\
    DeepReflect & only Feature Vector \\
    BODMAS & only Feature Vector \\
     \hline 
    
      \hline
\end{tabular}
\end{center}

\label{table:other_dataset}
\end{table}


\subsubsection{Performance Degradation on Our PACE (Benign) Dataset}
\label{app:pace_degrade}
While there have been works about concept drift on malware \citep{yang2021cade, jordaney2017transcend, barbero2022transcending} and they demonstrated how malwares evolved over the time, there have been very less work on concept drift of benign files. The probable reason can be the common belief that -- benign files do not evolve or change, i.e., the distribution remains same for them. However, we evaluated a version of MalConv model on our PACE (benign) dataset that was trained on (Ember + VTFeed) dataset. 
%First type -- MalConv just trained on Ember dataset which misclassified $1.52\%$ benign files from our dataset. Second type -- 
Surprisingly, this MalConv version was misclassifying $12.22\%$ benign files from PACE dataset while it was still having $98.91\%$ test accuracy on VTFeed dataset. Recall that -- our PACE dataset is the most recent one among these (2022). It is obvious that -- these benign datasets have different distributions due to the variation in collection time. It might be the case that -- with time, different companies release (or update) their softwares for newer version of Windows, and as a result, it causes a shift in benign file distribution too. So, we would suggest researchers to report their model performance on recent datasets in future, especially when it is about security-critical domain like malware detection.