
\section{Galaxy population and evolution}
\label{sec:galaxy}

Future wide-field surveys 
such as LSST \cite{LSST09}, %\footnote{https://www.lsst.org}, %\cite{} 
Euclid \cite{Euclid13}, %\footnote{https://www.euclid-ec.org}.
and the Nancy Grace Roman Space Telescope \cite{Spergel15} %\footnote{https://roman.gsfc.nasa.gov}, 
will detect billions of galaxies, both
near and far, and their statistical properties as galaxy {\it populations}
will provide rich information on galaxy formation and evolution
over cosmic time.
Also the summary statistics of the large-scale galaxy distributions 
are sensitive probes of cosmology \cite{Mo10}.
Reproducing the observed statistical properties of galaxies
is an important goal of theoretical study of galaxy formation.
Here, we introduce ML applications that are aimed at either 
extracting information from observed galaxy populations
or 
at modeling galaxy formation and evolution.

\subsection{Information extraction from observed data}

Fast and automated measurement of galaxy properties 
is urgently needed to analyze the extremely large data from the future wide-field surveys. 
Automated morphological classification such as distinguishing elliptical/spiral galaxies is one of the most important tasks, and
ML-based classification methods have been developed successfully
thanks to existing large "pre-labeled" datasets such as Galaxy Zoo \cite{Lintott11,Willett13}. 
Photometric redshift estimation is also a crucial task for galaxy surveys, and a wide range of ML applications have been explored
(see Refs.~\cite{Brescia21,Newman22} for
recent development).
ML can also be used to identify specific galaxy images such as strong gravitational lensing effect \cite{Hezaveh17} 
and galaxy merger remnants \cite{Bottrell22},
to detect anomalous objects \cite{Lochner22},
to deblend multiple objects \cite{Boucaud20,Arcelin21},
and to deconvolve point spread functions \cite{Wang22}.
Various ML techniques have been applied for these purposes, and extensive comparisons of the methods have been done \cite{Cavuoti12,Tanaka18,Sevilla-Noarbe18,Rafieferantsoa18,Metcalf19,Henghes21}.
In this section, we will not describe the full details of individual methods, 
but rather focus on the recent progress and general aspects of these ML applications.

A crucial advantage of ML methods is
that one can easily treat or incorporate imaging data in addition to photometric data
and spectral energy distribution (SED).
CNNs are known to be promising ML methods to analyze galaxy images.
Well-trained, sophisticated CNNs are able to classify photometric images as accurately as human experts (professional astronomers) and perform much faster \cite{Huertas-Company15,Cheng22}. 
In strong lens searches, 
many promising lens candidates have been newly discovered with the help of CNNs
from KiDS \cite{Petrillo17,Petrillo19a,He20_KiDS_SL,Li21_KiDS_SL},
the Pan-STARRS 3$\pi$ survey \cite{Canameras21},
and the DESI Legacy Imaging Surveys \cite{Huang20,Huang21,Storfer22}. 
Morphological information is also used in photometric redshift estimation. Refs.~\cite{Pasquet19_photoz,Henghes21_CNN} show that CNN-based frameworks estimate photometric redshifts of SDSS galaxies with higher precision than traditional models that use integrated photometric information alone.
Recently, an innovative image analysis model called Vision Transformer was proposed, which works as efficient and accurate as CNNs in estimating galaxy properties, especially when a large training dataset is available \cite{Lin21,Thuruthipilly22}.
Vision Transformer is considered to be more suited for capturing correlations between distant pixels in an image than CNN, and thus further development for applications in astronomy is to be explored.

In order to train supervised ML models that perform classification and regression tasks, one needs to prepare a large number of labeled training data, either observational 
or computer generated. 
While a supervised machine often outperforms conventional methods when sufficient training data is available, 
it may not be able to maximize its full potential otherwise \cite{Cavuoti12}.
An important technique to compensate for insufficient training data is data augmentation. One can use either analytical methods \cite{Hoyle15} or ML generative models \cite{Ravanbakhsh16} to increase effectively the number of training data.
Another solution is to use unsupervised learning models,
which can work even when little or no labeled data are available.
An illustrative example is morphological classification 
of galaxy images without labels such as elliptical and spiral. 
A machine first divides the images into several groups by a clustering algorithm, and then
the members of each group are visually classified to label the group \cite{Martin20}.
Similar unsupervised models can be applied to regression problems \cite{Geach12}. 
Unsupervised ML techniques are also used for anomaly detection (Sec. \ref{sec:SN-anomaly} in this review). Artefacts such as objects with spurious photometry can be removed in performing statistical analysis, and interesting rare objects can be detected with clustering methods \cite{Solarz17} or generative models \cite{Storey-Fisher21}.

Once a machine is trained and optimized, it can quickly classify or analyze a large set of observed galaxy images and generate a labeled catalogue.
Recently, Ref. \cite{Cheng22} provided one of the largest morphological classification catalogs of more than 20 million galaxies from the DES Year 3 data, with an estimated accuracy of over 99 percent for bright galaxies,
by training a machine with a few thousand galaxies from DES Year 1 data.
Photometric redshift catalogs of 39 million KiDS galaxies \cite{deJong17,Bilicki18},
34 million HSC galaxies \cite{Schuldt21},
a billion DESI Legacy Imaging Survey galaxies \cite{Duncan22},
and 3 billion Pan-STARRS1 galaxies \cite{Beck21} 
have also been recently generated with ML analysis.
The ML models can be applied to the future large observations, which will provide catalogs of billions of galaxies,
and will allow us to study statistical properties and to properly extract cosmological information from the data.
In preparation for this, detailed comparisons between various ML models and other conventional methods are being made for upcoming surveys 
such as LSST \cite{Schmidt20} and Euclid \cite{Euclid20,Euclid22}.

\subsection{Modeling galaxy formation and evolution}
\label{sec:galaxy_model}
A number of numerical simulations have been performed to follow the
formation of galaxies starting from realistic cosmological initial conditions.
Modern simulations follow hydrodynamics as well as gravitational interaction of baryons and DM with incorporating sub-grid models
of star formation and stellar feedback effects.
We refer the readers to recent comprehensive review articles on the computer models of galaxy formation and evolution \cite{Vogelsberger20}.
Such sophisticated computer models provide quantitative predictions for statistics of galaxy populations and their spatial distributions, 
and can also be used to estimate various systematic and statistical errors in 
analyses of data from wide-field surveys.


Properly calibrated numerical models with respect to direct observations are indispensable,
but direct simulations are computationally expensive, and
thus are not well suited to generate a large number of realizations with a sufficiently large volume and high resolution.
There are alternative methods such as semi-analytical models (SAMs) that combine physical models of galaxy formation either with analytic prescription of dark halo formation or with outputs of cosmological $N$-body simulations \cite{Somerville15,Behroozi19}. 
Commonly used DM-only simulations require significantly less computational cost 
than hydrodynamics simulations, allowing one to explore
a large number of physical models that connect the properties of DM halos to those of galaxies.
The essential goal is to populate galaxies on top of DM halos 
or on the DM density field, using multiplex relations derived from  direct observations or from numerical models that are calibrated by observations.

ML can be used to infer nonlinear relations between galaxy properties
such as stellar mass and star formation rate to the physical properties of the host dark halos and the environment such
as halo mass, density profile, and the local density (Figure \ref{fig:dm-galaxy}).
Various ML models have been proposed to solve this regression problem. 
Earliest studies built regression machines based on well-known ML classifiers such as SVMs and k-nearest neighbor algorithms (kNNs) to predict DM-galaxy connections \cite{Xu13,Ntampaka15}.
Later, advanced techniques such as decision trees, neural networks, and sparse regression models
have been tested and found to perform better \cite{Kamdar16,Agarwal18}.
Models based on decision trees and sparse regression can also quantify the relative importance of halo parameters in determining galaxy properties from training data with diverse quantities \cite{Kamdar16,Morice-Atkinson18,Moster21,Icaza-Lizaola21,deSanti22}.
Several hybrid approaches combining ML and SAM methods are also proposed \cite{Hearin20,Moews21}.
Recent studies extend the ML methods by training a machine with providing more information. For instance, temporal evolution can be implemented by
either feeding information of halo merger trees \cite{Jo19,Jespersen22} 
or using RNNs to analyze sequential snapshot data \cite{Chittenden22}.
Interestingly, models trained with hydrodynamics simulations can perform better than popular SAMs \cite{Jo19}.
An even faster approach is proposed by Refs. \cite{Zhang19,Kasmanoff20},
which can work with a linear DM density distribution as a basis instead of a DM halo catalog. 
The authors use three-dimensional CNNs to generate galaxy distributions 
from the underlying DM density distribution.
    Such models indeed capture the features of the spatial distribution of galaxies, 
    but also reproduce low-order statistics such as the density power spectra with high accuracy.

\begin{figure}
    \centering
    \includegraphics[width=10cm]{Figures/dm-galaxy.png}
    \caption{Machine learning can be used to generate a mock galaxy catalog (right) from a DM-only simulation (left). One can use regression models such as decision trees and NNs to infer nonlinear relations between galaxies and DM halo properties.}
    \label{fig:dm-galaxy}
\end{figure}

Obviously the accuracy of a ML method critically relies on the accuracy of the training numerical/theoretical models. 
Unfortunately there are more than a handful of features of real galaxies that are {\it not} reproduced 
by popular numerical models.
Thus it is important to reduce the generalization error of ML owing to 
limited or "biased" training data,
and to understand which physical models are more accurate.
Recently, CAMELs project \cite{Paco21,CMD22} provided a large set of numerical simulation data
that are mostly publicly available outputs of cosmological simulations of galaxy formation.
The available simulations differ in physics modelling, assumed cosmology, and numerical methods. Fortunately, the different sets of data can serve as training data 
with model "variations". The rich dataset can be used to robustly predict the connection between DM haloes and galaxies 
\cite{Villanueva-Domingo22,Shao22}
and to constrain cosmological parameters from observations \cite{Villaescusa-Navarro21b} without being biased by a specific model.
The training data covering a sufficiently broad range of physical models also enable us to explore the most appropriate parameters of the subgrid models in the simulation by making direct comparison of simulated galaxies with the observed ones \cite{Maccio22,Thiele22}.

Combined with low-cost DM-only simulations, the above ML techniques allow us to generate a number of mock galaxy catalogs encompassing a large cosmological volume, 
which are difficult to produce with current hydrodynamics simulations.
The predicted clustering features of galaxies are of great importance in 
the analyses of data from future galaxy surveys. 

