\begin{center}
    \caption{Continues from Table~\ref{tab:files_cutouts_1}, and focuses on the origin and assembly of stellar particles (PartType4). Here the dimension of all fields is equal to N = number of stellar particles in each cutout (we hence omit the column for brevity).} 
    \label{tab:files_cutouts_2}
    \begin{tabular}{p{3.7cm} c c p{9.8cm} }
        \hline
        Field & Units & PartType(s) & Description \\
        \hline
        SubfindIDAtFormation &  - & PartType4 & \textsc{Subfind}ID of the subhalo i.e. galaxy in which the stellar particle first appeared; $-1$ if it was formed outside of any subhalo (int16). \\
        SnapNumAtFormation &  - & PartType4 & Snapshot number in which the stellar particle first appeared (int16). \\
        InSitu &  - & PartType4 & $1$ if the stellar particle was formed in situ, $0$ if it was formed ex situ, and $-1$ if it does not currently belong to any subhalo. A stellar particle is considered to have been formed in situ if the subhalo in which it was formed lies along the ``main branch'' of the subhalo in which the stellar particle is currently found. This is decided using the SubLink ``baryonic'' merger trees (int8). Based on \citealt{Rodriguez-Gomez.2016}. \\
        AfterInfall &  - & PartType4 & 1 if the subhalo in which the stellar particle first appeared had already ``infalled'' into the halo \textsc{FoF group} where it is currently found; 0 otherwise; $-1$ if not applicable i.e., if the particle was formed in situ or if it was formed outside of any subhalo (int16). Based on \citealt{Rodriguez-Gomez.2016}.\\
        AccretionOrigin &  - & PartType4 &  This dataset can take the following integer values: 0, 1, and 2 for ex-situ stellar particles that were accreted from completed mergers (i.e., when the subhalo in which the stellar particle formed has already merged with the current subhalo), ongoing mergers (i.e., when the subhalo in which the stellar particle formed has not yet merged with the current subhalo, but will do so at a later snapshot in the simulation), and flybys (i.e., when the subhalo in which the stellar particle formed has not merged with the current subhalo, and will not do so at any future snapshot in the simulation), respectively; and $-1$ if not applicable i.e., if the particle was formed in situ or if it was formed outside of any subhalo (int8). Based on \citealt{Rodriguez-Gomez.2016}.\\
        MergerMassRatio & - & PartType4 & The stellar mass ratio of the merger in which a given ex-situ stellar particle was accreted (if applicable). The mass ratio is measured at the time when the secondary progenitor reaches its maximum stellar mass. NOTE: this quantity was calculated also in the case of flybys, without a merger actually happening. Based on \citealt{Rodriguez-Gomez.2016}.\\
        DistanceAtFormation &  - & PartType4 & The galactocentric distance when the stellar particle was formed, given in units of the stellar half-mass radius of the parent galaxy at the formation time.\\
        SnapNumAtStripping &  - & PartType4 & The snapshot in which the stellar particle last switched galaxies, and has hence remained in its present galaxy -- i.e. time of last stripping.  This field is $-1$ for all in-situ stellar particles and for particles that aren't part of any subhalo (int16).\\
        ProgGalaxyMassAtStripping &  $10^{10}\MSUN$/$h$ & PartType4 & Total stellar mass (as identified by \textsc{Subfind}) of the secondary progenitor galaxy of this stellar particle, from snapshot 'SnapNumAtStripping - 1'.\\
        ProgGalaxyMassInRadAtStripping &  $10^{10}\MSUN$/$h$ & PartType4 & As ProgGalaxyMassAtStripping but for the stellar mass within twice the stellar half mass radius of the secondary progenitor galaxy.\\  
        ProgSubhaloMassAtStripping &  $10^{10}\MSUN$/$h$ & PartType4 & As ProgGalaxyMassAtStripping but for the total subhalo mass of the secondary progenitor galaxy, i.e. the sum of all gravitationally-bound mass. \\ 
        DistanceToHostAtStripping &  ckpc$/h$ & PartType4 & Distance to the centre of the primary galaxy at snapshot SnapNumAtStripping. Here, the centre of the galaxy refers to the position of the most-bound particle of PartType = 4.\\
        %SnapNumAtAccretion &  - & PartType4 & The snapshot in which the stellar particle first appeared in its $z=0$ FoF Halo -- time of accretion.\\
        %ProgM200cAtAccretion &  - & PartType4 & \\
        %ProgM200mAtAccretion &  - & PartType4 & \\
        \hline
    \end{tabular}   
\end{center}
