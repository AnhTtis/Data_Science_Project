\begin{center}
    \caption{Description of the fields in the snapshot cutouts released with this paper (continues in Table~\ref{tab:files_cutouts_2}). There is one hdf5 file for each of the 198 TNG50 MW/M31-like galaxies at each redshift: we provide one cutout per galaxy, each from snapshot 011 to snapshot 099 (i.e. $z=0$) along the main progenitor branch of each galaxy: see text for details. Each cutout contains all particle-level fields present in the original snapshot files across the entire simulated volume (full or mini snapshots depending on the redshift; Section~\ref{sec:fullvolumedata}) and contained within a cube centred on the galaxy position (SubhaloPos) and extending $\pm400$ ckpc along each direction. However, with this release we provide additional quantities for the galaxies under focus that require substantial analysis or post-processing of the whole-simulation data. These additional fields are listed and described below. They are organized by particle type, i.e. PartType = 0, 1, 3, 4, and 5 for gas cells, DM particles, tracers, stars and wind particles, and SMBH particles, respectively. N is the number of each PartType in each file.} 
    \label{tab:files_cutouts_1}
    \begin{tabular}{p{2.5cm} c c c p{9.4cm} }
        \hline
        Field & Dimensions & Units & PartType(s) & Description \\
        \hline
        \textsc{Subfind}ID & N & - & all & The ID of the subhalo that a given particle/cell is a part of at the given redshift, in the space of the {Subfind} i.e. Subhalo catalog of the whole TNG50 simulation. It is set to -1 if the particle/cell does not belong to any subhalo (int32). \\
        MainSnapshotIndex & N & - & all & Index into the main snapshot file(s) of this particular particle/cell. This is useful if one wants to load data from one of the post processing quantities that are available online but are not included in these cutouts (int64). \\
        RotatedCoordinates & N,3 & ckpc$/h$ & all & Spatial position of each particle/cell in the reference system of the main galaxy: a coordinate transformation is first performed to shift to the frame of reference of the (centre of the) main galaxy (SubhaloPos); thus a rotation is performed by diagonalizing the moment of inertia tensor (or mass tensor) such that projecting along the z-axis yields a face-on view of the central galaxy. The mass tensor is computed from stars and star-forming gas within a given aperture: minimum of once (twice) the stellar half mass radius for stars (gas).\\
        RotatedCenterOfMass & N,3 & ckpc$/h$& PartType0 & Spatial position of the center of mass of each gas cell, after having applied the translation+rotation transformation as for RotatedCoordinates. \\
        RotatedVelocities & N,3 & km $\sqrt(a)/s$ & all & Spatial velocity of each particle/cell in the reference system where the main galaxy is at rest: a velocity transformation is first performed with respect to the bulk/peculiar velocity of the main galaxy through the simulated volume (SubhaloVel). Then a rotation is applied as for RotatedCoordinates. \\
        RotatedMagneticField & N,3 & see online
        %$(h/a^2)$(UnitPressure)$^{1/2}$
        & PartType0 & Magnetic field strength of each gas cell after having applied the transformation (rotation only) as for RotatedCoordinates. For the units see description online.\\
        MH & N & $10^{10}\MSUN$/$h$ & PartType0 & Total neutral hydrogen mass of each gas cell. It equals the sum of the mass of atomic HI plus molecular H2: MH = MHI + MH2 \citep{Popping.2020}\\
        MH2* & N & $10^{10}\MSUN$/$h$ & PartType0 & Molecular hydrogen mass of each gas cell, according to three different models and hence three different partitioning schemes (denoted in the names with * = 'GK', 'BR', 'KMT'), based on \citet{Popping.2020}; see also \citet{Diemer.2018}. \\
        Xray\_Emission\_03\_2keV & N & erg cm$^3$/s & PartType0 & Intrinsic and instantaneous X-ray emission of the gas in the soft broad band [0.3-2] keV. This includes contributions from both the continuum and the lines and is given as X-ray cooling rate. The cooling rate is physically similar to the gas field ``GFM\_CoolingRate'' and it is computed assuming an emission model ``APEC'' from the XSPEC package \citep{Smith.2001} using the element abundances traced by the simulation (practically it is done by using the VAPEC model in the XSPEC package).Available only at the $z=0$ snapshot. Based on \textcolor{blue}{Truong et al. 2023 to be submitted}.\\
        Xray\_Emission\_05\_2keV & N & erg cm$^3$/s & PartType0 & Same as above but for the [0.5-2] keV soft X-ray broad band. \\
        Xray\_Emission\_03\_2keV\_C & N & erg cm$^3$/s & PartType0 & Same as Xray\_Emission\_03\_2keV but for continuum emission only. \\
        Xray\_Emission\_05\_2keV\_C & N & erg cm$^3$/s & PartType0 & Same as Xray\_Emission\_05\_2keV but for continuum emission only.\\
        Xray\_Emission\_Line\_X &N & erg cm$^3$/s & PartType0 & Intrinsic and instantaneous X-ray emission of the gas (as Xray\_Emission\_03\_2keV) in narrow bands at nine specific lines, with X = CV (298.97 eV), CVI (367.47 eV), NVI (419.86 eV), NVII (500.36 eV), OVIIf (560.98 eV), OVIIr (573.95 eV), OVIII (653.49 eV), FeXVII (725.05 eV), and NeX (1021.5 eV). Based on \textcolor{blue}{Truong et al. 2023 to be submitted}.\\
        \hline
    \end{tabular}   
\end{center}


