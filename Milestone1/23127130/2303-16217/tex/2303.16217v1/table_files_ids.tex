\begin{center}
    \caption{Description of all fields in the ``TNG50 MW/M31-like catalog'' released with this paper. All fields are float32 unless otherwise specified. The Flag fields are all either 1 or 0. All fields are 198 long: namely, for each field, there is at least one entry per TNG50 MW/M31 analog. All properties refer to $z=0$ and are listed here in conceptual order. Many more properties of the TNG50 MW/M31-like galaxies can be either found in the official \textsc{Subfind} catalogs (by matching the IDs/indexes) or can be computed and measured from the particle-level data also released here (Section~\ref{sec:data_cutouts}).} 
    \label{tab:files_ids}
    \begin{tabular}{p{3cm} p{1cm} p{12.5cm}}
        \hline
        Field & Units & Description \\
        \hline
        \textsc{Subfind}ID &  - & $z=0$ ID of the galaxy, in the space of the {Subfind} i.e. Subhalo catalog of the whole TNG50 simulation. It is 0 indexed (int32). \\
        FlagCentral &  - & 1 if the galaxy is the central i.e. the first/main/most massive Subhalo object of its FoF halo (int8).\\
        FlagDiskyVisual &  - & 1 if the galaxy appears disky and with spiral arms based on the authors' visual inspection of 3-band stellar light images, both face-on and edge-on. See Section~\ref{sec:criterion_diskyness} (int8).\\
        FlagDiskyMinor2Major &  - & 1 if the stellar-mass minor-to-major axis ratio is smaller than a certain value: $c/a \le 0.45$. Based on galaxy shapes from \citet{Pillepich.2019}. See Section~\ref{sec:criterion_diskyness} (int8)\\
        FlagDisky & - & 1 if FlagDiskyVisual=1 OR FlagDiskyMinor2Major=1. Namely 1 if the galaxy has a stellar disky morphology either based on visual inspection or based on the stellar axis ratios. All entries in this catalog should be equal to 1. See Section~\ref{sec:criterion_diskyness} (int8)\\
        FlagIsolated &  - & 1 if there is no galaxy with $M_*(<30{\rm kpc})\ge 10^{10.5}\,\MSUN$ within 500 kpc distance \& if $\MTWOC ({\rm host}) <10^{13}\,\MSUN$. See Section~\ref{sec:criterion_environment} (int8)\\
        FlagMstars &  - & 1 if $M_*(<30{\rm kpc}) = 10^{10.5-11.2}\,\MSUN$. See Section~\ref{sec:criterion_mstars} (int8)\\
        FlagM200c &  - & 1 if $\MTWOC ({\rm host}) = 6\times10^{11}-2\times10^{12}\,\MSUN$. See Section~\ref{sec:selection_comp} (int8)\\
        FlagNoVirgo &  - & 1 if there is no FoF/Group halo with $\MTWOC \gtrsim 10^{14}\,\MSUN$ within 10 Mpc distance. For TNG50, this includes a constraint on the two most massive systems in the box at $z=0$. See Fig.~\ref{fig:environment} and Section~\ref{sec:environment} (int8)\\
        FlagS0 &  - & 1 if the galaxy appears as an S0 per visual inspection, namely it has a disky stellar morphology but with no manifest spiral arms. See Section~\ref{sec:criterion_diskyness} (int8)\\
        FlagBarredZana22 &  - & 1 if the galaxy has a bar according to \citet{Zana.2022}. See Section~\ref{sec:bars} (int8)\\
        FlagBarredRosas22 &  - & 1 if the galaxy has a bar according to \citet{Rosas-Guevara.2020, Rosas-Guevara.2022}. See Section~\ref{sec:bars} (int8)\\
        & & \\
        FlagMWM31 &  - & 1 if FlagDisky=1 \& FlagMstars=1 \& FlagIsolated=1. This is the fiducial selection proposed and discussed in this paper. All entries in this catalog should be equal to 1. See Fig.~\ref{fig:selection}. Other more restrictive selections can be imposed using the provided additional flags (int8)\\
        & & \\
        StellarMass\_30kpc & $\MSUN$& Galaxy stellar mass evaluated by summing up the mass of all stellar particles that are gravitationally bound according to \textsc{Subfind} and within a 3D radius of 30 kpc from the galaxy center. It is the fiducial measure used throughout this paper. \\
        StellarMass\_2r1/2 & $\MSUN$& Galaxy stellar mass evaluated by summing up the mass of all stellar particles that are gravitationally bound according to \textsc{Subfind} and within a 3D radius of twice the stellar half mass radius. This is equivalent to the 5th entry of SubhaloMassInRadType in the official \textsc{Subfind} catalogs \citep{Nelson.2015, Nelson.2019}.\\
        StellarMass\_all& $\MSUN$& Galaxy stellar mass evaluated by summing up the mass of all stellar particles that are gravitationally bound according to \textsc{Subfind}, with no distance limit. This also includes the mass in e.g. the stellar halo.\\
        HaloMass\_M200c & $\MSUN$& Total halo mass in terms of the spherical-overdensity measure $\MTWOC$. For non central galaxies, this is the halo mass of the host. \\
        HaloMass\_Mdyn & $\MSUN$& Total halo mass obtained by summing up the mass of all particles and cells that are gravitationally bound according to \textsc{Subfind}. \\
        HaloVirialRadius\_R200c & kpc & Virial radius of the underlying halo in terms of $\RTWOC$, in analogy with HaloMass\_M200c. \\
        SFR\_inst & $\MSUN$yr$^{-1}$ & Star formation rate of the galaxy based on the instantaneous SFR of the gas, i.e. sum of the individual star formation rates of all gravitationally-bound gas cells in this galaxy. It is equivalent to SubhaloSFR in the official \textsc{Subfind} catalogs \citep{Nelson.2015, Nelson.2019}. \\
        SFR\_50Myr & $\MSUN$yr$^{-1}$ & Time-averaged star formation rate of the galaxy including stars within a 3D aperture of 30 kpc, based on the stellar particles actually produced over the last 50 million years and using their initial mass at birth; as in \citet{Pillepich.2019}.\\
        DiskScaleLength & kpc & Exponential disk length of the galaxy obtained by fitting the stellar mass surface density of stellar particles in disky orbits between 1 and 4 times the stellar half mass radius. See errorbars and fitting procedure in \citet{Sotillo.2022} and \textcolor{blue}{Sotillo et al. submitted}. \\
        DiskScaleHeightThin\_8kpc & pc & Thin stellar disk height of this galaxy obtained by fitting the vertical stellar mass density distribution of disk stars with a double squared hyperbolic secant functional form in an annulus of $7-9$ kpc from the center. See errorbars, fitting procedure, and additional different measurements in \citet{Sotillo.2022} and \textcolor{blue}{Sotillo et al. submitted}. \\
        DiskScaleHeightThick\_8kpc & pc & Same as above, for the thick component. Thin and thick disk heights here are meant as geometrical. \\
        SMBH\_Mass & $\MSUN$& Mass of the central (i.e. most massive) SMBH in this galaxy. The measurement is based on the PartType5/BH\_Mass of the official snapshot data \citep{Nelson.2015, Nelson.2019}. For MW/M31-like galaxies, this mass is to all effects equivalent to SubhaloBHMass in the official \textsc{Subfind} catalog.  \\
        SMBH\_AccretionRate & $\MSUN$yr$^{-1}$& Instantaneous gas mass accretion rate into the SMBH of this galaxy, based on SubhaloBHMdot from the official \textsc{Subfind} catalog \citep{Nelson.2015, Nelson.2019}. \\
        SMBH\_EddingtonRatio & - & Instantaneous Eddington ratio of the SMBH of this galaxy.\\
        SMBH\_FeedbackMode & $\MSUN$& 0 if the SMBH of this galaxy is in thermal feedback mode, 1 if it is exercising kinetic feedback mode. See Sections~\ref{sec:tng50}, \ref{sec:sf}, and \ref{sec:smbhs} and references therein (int8).\\
        SMBH\_CumEnergy\_QM & erg & Cumulative amount injected into the surrounding gas by the central SMBH in the high accretion-state (quasar) mode, total over its entire lifetime. \\
        SMBH\_CumEnergy\_RM & erg & As SMBH\_CumEnergy\_QM but for the energy injected in the low accretion-state (wind or kinetic) mode.\\
        \hline
    \end{tabular}   
\end{center}


