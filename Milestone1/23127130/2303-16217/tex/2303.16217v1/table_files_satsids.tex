
\begin{center}
    \caption{Description of all fields in the ``TNG50 MW/M31-like satellite catalog'' released with this paper. All fields are float32 unless otherwise specified. All fields are 1237 long: namely, for each field, there is one entry per satellite of the TNG50 MW/M31 analogs. All properties refer to $z=0$ and are listed here in conceptual order. The same considerations apply as for the TNG50 MW/M31-like main galaxies.} 
    \label{tab:files_satsids}
    \begin{tabular}{l l p{12cm}}
        \hline
        Field & Units & Description \\
        \hline
        \textsc{Subfind}IDSat &  - & $z=0$ ID of the satellite galaxy, in the space of the {Subfind} i.e. Subhalo catalog of the whole TNG50 simulation. It is 0 indexed (int32). \\
        \textsc{Subfind}IDHost &  - & $z=0$ ID of the host galaxy of this satellite, in the space of the {Subfind} i.e. Subhalo catalog of the whole TNG50 simulation. These can only take the values of the SubhaloIDs of the TNG50 MW/M31 analogs of Table~\ref{tab:files_ids}. The host is the closest main galaxy. It is 0 indexed (int32). \\
        & & \\
        DistanceToHost3D  & kpc & Three-dimensional distance of the satellite galaxy to the closest MW/M31-like hosts.\\ 
        DistanceToHost2D & kpc & Two-dimensional distance of the satellite galaxy  in a random projection to the closest MW/M31-like hosts (along z axis of the simulation box).\\   
        3DVelocityRelToHost & km/s & Three-dimensional relative velocity of the satellite galaxy with respect to the closest MW/M31-like hosts.\\
        LoSVelocityRelToHost & km/s & Line-of-sight relative velocity of the satellite galaxy  with respect to the closest MW/M31-like hosts, in a random projection (along z axis of the simulation box).\\
        StellarMass\_2r1/2 & $\log \MSUN$& Galaxy stellar mass evaluated by summing up the mass of all stellar particles that are gravitationally bound according to \textsc{Subfind} and within a 3D radius of twice the stellar half mass radius. This is equivalent to the 5th entry of SubhaloMassInRadType in the official \textsc{Subfind} catalogs \citep{Nelson.2015, Nelson.2019}.\\
        Magnitude\_VBand & mag & Absolute V-band Buser's magnitude of the satellite galaxy  based on the summed-up luminosities of all the gravitationally-bound stellar particles (Vega magnitudes), from the GFM\_StellarPhotometrics field of the official snapshot data \citep{Nelson.2015, Nelson.2019}.\\
        Magnitude\_rBand & mag & As for Magnitude\_VBand, but in the SDSS r band (AB magnitudes).\\
        Mdyn & $\MSUN$& Total halo mass of the satellite galaxy obtained by summing up the mass of all particles and cells that are gravitationally bound according to \textsc{Subfind}. \\
        GasMass\_2r1/2  & $\log \MSUN$&  Total gas mass of the satellite galaxy evaluated by summing up the mass of all stellar particles that are gravitationally bound according to \textsc{Subfind} and within a 3D radius of twice the stellar half mass radius. This is equivalent to the 1st entry of SubhaloMassInRadType in the official \textsc{Subfind} catalogs \citep{Nelson.2015, Nelson.2019}.\\
        HIGasMass\_2r1/2 & $\log \MSUN$& Atomic hydrogen mass of the satellite galaxy  evaluated by summing up the HI mass of all gas cells that are gravitationally bound according to \textsc{Subfind} and within a 3D radius of twice the stellar half mass radius. This is based on the 'GK' HI+H2 partitioning by \citet{Popping.2020}. \\
        H2GasMass\_2r1/2 & $\log \MSUN$& As HIGasMass\_2r1/2 but for the molecular hydrogen. \\
        HalfLightRadius2D\_VBand  & kpc & Two-dimensional circularized stellar half-light radius of the satellite galaxy  in the V-band, from a face-on projection. Based on \citet{Pillepich.2019}.\\
        StellarVelDisp\_3D & km/s & Three-dimensional standard deviation of the velocities of all stellar particles of the satellite galaxy within twice the stellar half-mass radius weighted by their respective stellar mass. \\
        SurfaceBrightness2D\_rBand & mag arcsec$^{-2}$& Two-dimensional surface brightness of the satellite galaxy in the SDSS r band (see above). \\
        Vmax & km/s & Maximum of the circular velocity profile of the satellite galaxy, accounting for {\it all} matter components. \\
        SFActivity & - & Flag denoting the star-formation activity of the satellite gaalxy: 1, 0, -1 for star-forming, quenched and green-valley galaxies, respectively. This is based on \citet{Pillepich.2019} and on the distance from the ridge of the star-forming main sequence (int8).\\  
        t10 & Gyr ago & Stellar assembly time $\tau_{10}$, i.e. time when the galaxy assembled 10 per cent of its $z=0$ stellar mass. Based on \citet{Joshi.2021}.\\   
        t50 & Gyr ago & As t10 but for the time when the galaxy assembled 50 per cent of its $z=0$ stellar mass. \\   
        t90 & Gyr ago & As t10 but for the time when the galaxy assembled 90 per cent of its $z=0$ stellar mass.\\
        \hline
    \end{tabular}   
\end{center}

