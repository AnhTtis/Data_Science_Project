 \begin{center}
    \footnotesize
    \caption{{\bf Cosmological (magneto-)hydrodynamical simulations of MW/M31-like galaxies or Local Group-like (LG) systems in a $\Lambda$CDM scenario} currently pursued by the community and referenced in this paper. The list deliberately omits idealized and isolated-disk simulations, dark-matter only simulations -- even if they have paved the road for the subsequent hydrodynamical ones, especially in the context of MW-mass haloes --, and cosmological simulations with too poor resolution for galaxies and their stellar disks to be studied. Namely, here we include only cosmological models, i.e. simulations that start from cosmologically-motivated initial conditions on $>>$ 10s Mpc scales, which are run to $z=0$. We hence also list large-volume uniform-resolution projects (specifically those with baryonic mass resolution better than a few $10^7\MSUN$), as they naturally include MW/M31-mass galaxies. As it can be seen from the second column, it is only since about a decade that it has been possible to simulate MW-like galaxies, i.e. with relatively thin 
    stellar disks. The underlying numerical codes are varied, and mostly differ for the adopted  schemes to discretize the collisional component and to numerically solve for the (coupled) equations of (gravity and) hydrodynamics. They span lagrangian and smooth particle hydrodynamics (GASOLINE, GADGET-3, ChaNGa), eulerian and adaptive mesh refinement (RAMSES) and meshless finite-mass (GIZMO) and moving-mesh (AREPO) codes. Also the assumptions on the precise value of the cosmological parameters have evolved through the years. However, what really distinguishes the outcome of these simulations, and hence their galaxies, are the different implementations of the models for star formation, stellar evolution and heavy element production, radiative (metal) cooling, stellar feedback, and SMBH feedback. The details and sophistication of these models differ across simulations, as do their results. Not all simulations include feedback from SMBHs: see 5th column. Only the Auriga, Hestia, and the TNG simulations (TNG100, TNG300, TNG50) include MHD. None include cosmic rays \citep[but see][]{Buck.2020b}. By ``$\#$ MW/M31-like galaxies'', here we mean: for zoom-in simulations, the number of targeted haloes/galaxies (see text for more details on the variously adopted selection criteria); for uniform-resolution box simulations, we quote the number of central galaxies with stellar mass in the $10^{10.5-11.2}\MSUN$, irrespective of morphology.}
\label{tab:sims}

    \begin{tabular}{l c c c c c c c}
        \hline
        Simulation(s) & Year & Code & Technique & SMBH & $\#$  MW/M31-like &$M_{\rm baryon}^{\bigstar}$ & Simulation/Method \\
         & & & & feedback & galaxies & $[\MSUN]$ & Reference(s)\\
        \hline
        
        & & & & & & & \\
        Eris 				&2011 	&GASOLINE 	&zoom 		& no  		& 1 							& $2 \times 10^{4}$ 					&\cite{Guedes.2011} \\
        Agertz's			&2011	&RAMSES	&zoom 		& no 		& 1							& --								&\cite{Agertz.2011}\\
        Few's 				&2012	&RAMSES 	&zoom  		& no  		& 19 							& --								&\cite{Few.2012} \\
        MaGICC 			&2013 	&GASOLINE 	& zoom 		& no  		& 1 							& $2.2 \times 10^{5}$				&\cite{Stinson.2013} \\
        CLUES 			&2014 	&GADGET-3 	&zoom 		& no  		& 1 $^{\vartriangle,\Diamond}$ 		& $5.5 \times 10^{5}$ 				&\cite{Nuza.2014}\\
        Aquarius with Gas	&2014 	&AREPO 		&zoom 		& yes  	& 8  							& $4.1 \times 10^{5}$				&\cite{Marinacci.2014a}\\
        APOSTLE 		&2016  	&GADGET-3 	& zoom 		& yes  	& 12 $^{\vartriangle}$ 			& $1.0 \times10^{4}$ 				&\cite{Sawala.2016, Fattahi.2016}\\
        Latte 			&2016 	&GIZMO 		& zoom 		& no  		& 1 							& $7.1\times10^3$							&\cite{Wetzel.2016} \\
        Auriga L4		&2017 	&AREPO 		&zoom 		& yes  	& 30  						& $5 \times10^{4}$ 					&\cite{Grand.2017} \\        
        Auriga L3		&2017 	&AREPO 		&zoom 		& yes  	& 3  						& $6 \times10^{3}$ 					&\cite{Grand.2017} \\        
        FIRE-2 Suite 		&2017  	&GIZMO 		& zoom 		& no  		& 7 							& $(4.2-7.1)\times10^3$ 					&$\blacklozenge$\\
        ELVIS on FIRE 		&2019  	&GIZMO 		&zoom 		& no  		& 3 $^{\vartriangle}$ 				& $(3.5-4.0)\times10^3$						&\cite{Garrison-Kimmel.2019,Garrison-Kimmel.2019b}\\        
        ARTEMIS 			&2020  	&GADGET-3	&zoom 		& yes  	& 42  						& $3.2 \times 10^{4}$ 				&\cite{Font.2020}\\        
        NIHAO-UHD 		&2020  	&GASOLINE2 	&zoom 		& no  		& 6  							& $(2.0-9.4)\times 10^{4}$ 				&\cite{Buck.2020}\\        
        VINTERGATAN 	&2020 	&RAMSES 	&zoom 		& no  		& 1  							& 7070 								&\cite{Agertz.2021} \\        %$7070$
        %RAMSES-based 	&2020 	&RAMSES 	&zoom 		& no 		& 2  						& $2.9 \times 10^{4}$ 				&\cite{Kretschmer.2020} \\        
        Hestia Int. Res		&2020  	&AREPO 		&zoom 		& yes  	& 13 $^{\vartriangle,\Diamond}$ 	& $1.8 \times 10^{5}$ 				&\cite{Libeskind.2020}\\       
        Hestia High Res	&2020  	&AREPO 		&zoom 		& yes  	& 3 $^{\vartriangle,\Diamond}$ 		& $2.2 \times 10^{4}$ 				&\cite{Libeskind.2020}\\       
        DC Justice League 	&2020  	&ChaNGa 	&zoom 		& yes  		& 2  							& $3.3 \times 10^{3}$				&\cite{Applebaum.2021}\\   
        Auriga L2		&2021 	&AREPO 		&zoom 		& yes  	& 1  						& $800$ 					&\cite{Grand.2021} \\        
        & & & & & & \\
        & & & & & & \\
        Illustris			&2014	&AREPO 		&unif.res. 		& yes  	& $1518$					& $1.3 \times 10^{6}$				&$\spadesuit$\\
        Magneticum 4uhr	&2014	&GADGET-3 	&unif.res. 		& yes  	& 389						& $1.0 \times 10^{7}$				&\cite{Hirschmann.2014}	\\
	Eagle			&2015	&GADGET-3 	&unif.res. 		& yes  	& 812					& $1.8 \times 10^{6}$				&\cite{Schaye.2015, Crain.2015}	\\
	MassiveBlack-II		&2015	&GADGET-3 	&unif.res. 		& yes  	& 1427					& $3.1 \times 10^{6}$				&\cite{Khandai.2015}	\\
	Horizon-AGN		&2015	&RAMSES 	&unif.res. 		& yes   & $\sim{2000}$					& $1.4\times10^6$					&\cite{Dubois.2014}	\\
	Romulus 25		&2016	&ChaNGa 	&unif.res. 		& yes 	& $\sim{20}$ 					& $2.1 \times 10^{5}$				&\cite{Tremmel.2017}	\\
	MUFASA			&2016	&GIZMO 		&unif.res. 		& yes 	& 237						& $1.8 \times 10^{7}$				&\cite{Dave.2016}	\\
	TNG100			&2017	&AREPO 		&unif.res. 		& yes  	& 1606					& $1.4 \times 10^{6}$				&$\clubsuit$	\\
	TNG300			&2017	&AREPO 		&unif.res. 		& yes  	& 23470				& $1.1 \times 10^{7}$				&$\clubsuit$	\\
	FABLE			&2018	&AREPO 		&unif.res. 		& yes  	& 217						& $9.4 \times 10^{6}$				&\cite{Henden.2018}	\\
	Simba 100		&2019	&GIZMO 		&unif.res. 		& yes  	& 2075						& $1.8 \times 10^{7}$				&\cite{Dave.2019}	\\
	\bf{TNG50	}		&\bf{2019}	&\bf{AREPO} 	&\bf{unif.res.} 	& \bf{yes} 	& \bf{198$^{\square}$}  					& \bf{$8.5 \times 10^{4}$}				&\bf{\cite{Pillepich.2019, Nelson.2019b}}\\     
	{\textsc NewHorizon} 		&2020	&RAMSES 	&unif.res. 		& yes 	& $\sim{10}$					& $1.3\times10^4$								&\cite{Dubois.2021}	\\	
	FIREbox 			&2022	&GIZMO 		&unif.res. 		& no		& 53					& $6.3 \times 10^{4}$				&\cite{Feldmann.2022}	\\
	& & & & & & \\       
        \hline
    \end{tabular}   
 \end{center}
$^{\bigstar}$ If constant, the gas particle mass; if variable, the target/average gas cell mass for Lagrangian-type simulations; for grid/Eulerian simulations, this value is not generally comparable. For the latter, when available, we provide the stellar mass resolution or an equivalent estimate.  \\

$^{\vartriangle}$ Local-Group systems, i.e. each with one MW-mass and one M31-mass galaxy \\

$^{\Diamond}$ Constrained initial conditions \\
$\blacklozenge$ \cite{Garrison-Kimmel.2017, Hopkins.2018,Samuel.2020,Garrison-Kimmel.2019}\\

$\spadesuit$ \cite{Vogelsberger.2014a,Vogelsberger.2014b, Genel.2014,Sijacki.2015,Nelson.2015}\\

$\clubsuit$ \cite{Pillepich.2018b, Nelson.2018, Springel.2018, Marinacci.2018,Naiman.2018, Nelson.2019}\\

$^{\square}$ 198 is the number of MW/M31-like galaxies of TNG50 presented throughout this paper and that satisfy the conditions advocated in Section~\ref{sec:selection} and Fig.~\ref{fig:selection}. The total number of galaxies with stellar mass in the $10^{10.5-11.2}\MSUN$ range is 324, including also non-disky and galaxies in group or cluster hosts: of these 325 in TNG50, 221 are centrals.

% In fact, most zoom-in simulations of MW-like galaxies do not include SMBHs and SMBH feedback -- which are instead crucial for the realism of galaxy populations across the mass scales as those produced by large-volume ``uniform-resolution'' runs (lower part of the Table).