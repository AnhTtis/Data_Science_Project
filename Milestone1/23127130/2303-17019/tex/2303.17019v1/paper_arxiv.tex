%
%
\documentclass[preprint,nopreprintline]{elsarticle}


%
\usepackage{etex}               %

%
\usepackage[bookmarks=true,colorlinks=true,linkcolor=blue]{hyperref}

%
\usepackage{xcolor}

%
\usepackage{amsmath}          %
\usepackage{amssymb}          %
\usepackage{mathrsfs}         %
\usepackage{mathtools}        %

%
\usepackage{soul}             %
%
\usepackage[margin=1in]{geometry}

%
\usepackage{caption}          %
\usepackage{makecell,multirow,multicol,booktabs} %
\renewcommand\theadfont{\bfseries}

%
\usepackage[group-separator={,}]{siunitx}

%
\usepackage{tikz}
  \usetikzlibrary{calc,positioning}
\usepackage{pgfplots}
  \pgfplotsset{compat=newest}

%
%
%
%

%
\usepackage{hyperref}

%
\usepackage{amsthm}
\theoremstyle{plain}
\newtheorem{lemma}{Lemma}[section]
\newtheorem{theorem}[lemma]{Theorem}
\newtheorem{corollary}[lemma]{Corollary}
\newtheorem{proposition}[lemma]{Proposition}
\theoremstyle{definition}
\newtheorem{definition}[lemma]{Definition}
\newtheorem{example}[lemma]{Example}
\theoremstyle{remark}
\newtheorem{remark}[lemma]{Remark}

%
\usepackage[section]{algorithm} %
\usepackage{algpseudocode}  %

%
\makeatletter%
\newcounter{algorithmicH} %
\let\oldalgorithmic\algorithmic
\renewcommand{\algorithmic}{%
  \stepcounter{algorithmicH} %
  \oldalgorithmic} %
\renewcommand{\theHALG@line}{ALG@line.\thealgorithmicH.\arabic{ALG@line}}
\makeatother%

%
%
%
%
%
%
%
%
%
%
%
%
%
%
%
%
\newlength{\tfwidth}
\newlength{\tfheight}
\newlength{\tfxa}
\newlength{\tfxb}
\newlength{\tfya}
\newlength{\tfyb}
%
\newcommand{\trimPlotWithBox}[6]{%
\setlength\fboxsep{0pt}%
\setlength\fboxrule{1.0pt}%
\fbox{\includegraphics[width=#2, clip, trim=#3 #4 #5 #6]{#1}}%
}
\newcommand{\trimPlotNoBox}[6]{%
\setlength\fboxsep{1pt}%
\setlength\fboxrule{0.0pt}%
\fbox{\includegraphics[width=#2, clip, trim=#3 #4 #5 #6]{#1}}%
%
}
%
%
\newcommand{\trimPlotb}[6]{%
\setlength{\tfwidth}{19.05cm}%
\setlength{\tfxa}{\tfwidth*\real{#3}}%
\setlength{\tfxb}{\tfwidth*\real{#4}}%
\setlength{\tfya}{\tfwidth*\real{#5}}%
\setlength{\tfyb}{\tfwidth*\real{#6}}%
%
\trimPlotWithBox{#1}{#2}{\tfxa}{\tfya}{\tfxb}{\tfyb}%
%
}
%
%
\newcommand{\trimPlot}[6]{%
\setlength{\tfwidth}{19.05cm}%
\setlength{\tfxa}{\tfwidth*\real{#3}}%
\setlength{\tfxb}{\tfwidth*\real{#4}}%
\setlength{\tfya}{\tfwidth*\real{#5}}%
\setlength{\tfyb}{\tfwidth*\real{#6}}%
%
\trimPlotNoBox{#1}{#2}{\tfxa}{\tfya}{\tfxb}{\tfyb}%
}
%
%
%
%
%
%
%
%
%
%
%
%
%
\newcommand{\trimFigWithBox}[6]{%
\setlength\fboxsep{0pt}%
\setlength\fboxrule{1.0pt}%
\fbox{\includegraphics[width=#2, clip, trim=#3 #4 #5 #6]{#1}}%
}
%
\newcommand{\trimFigNoBox}[6]{%
\setlength\fboxsep{1pt}%
\setlength\fboxrule{0.0pt}%
\fbox{\includegraphics[width=#2, clip, trim=#3 #4 #5 #6]{#1}}%
%
}
%
\newcommand{\trimFigHeightWithBox}[6]{%
\setlength\fboxsep{0pt}%
\setlength\fboxrule{1.0pt}%
\fbox{\includegraphics[height=#2, clip, trim=#3 #4 #5 #6]{#1}}%
}
%
\newcommand{\trimFigHeightNoBox}[6]{%
\setlength\fboxsep{1pt}%
\setlength\fboxrule{0.0pt}%
\fbox{\includegraphics[height=#2, clip, trim=#3 #4 #5 #6]{#1}}%
%
}
%
\newcommand{\trimFigb}[6]{%
\setlength{\tfwidth}{(#2+#2*\real{#3})+#2*\real{#4}}%
\setlength{\tfheight}{(#2+#2*\real{#5})+#2*\real{#6}}%
\setlength{\tfxa}{\tfwidth*\real{#3}}%
\setlength{\tfxb}{\tfwidth*\real{#4}}%
\setlength{\tfya}{\tfheight*\real{#5}}%
\setlength{\tfyb}{\tfheight*\real{#6}}%
%
\trimFigWithBox{#1}{#2}{\tfxa}{\tfya}{\tfxb}{\tfyb}%
}
%
%
\newcommand{\trimFig}[6]{%
\setlength{\tfwidth}{(#2+#2*\real{#3})+#2*\real{#4}}%
\setlength{\tfheight}{(#2+#2*\real{#5})+#2*\real{#6}}%
\setlength{\tfxa}{\tfwidth*\real{#3}}%
\setlength{\tfxb}{\tfwidth*\real{#4}}%
\setlength{\tfya}{\tfheight*\real{#5}}%
\setlength{\tfyb}{\tfheight*\real{#6}}%
%
\trimFigNoBox{#1}{#2}{\tfxa}{\tfya}{\tfxb}{\tfyb}%
}

%
%
%
\newcommand{\trimhb}[6]{%
%
%
%
%
%
%

%
\sbox\figBox{\includegraphics{#1}}
\setlength{\tfwidth}{\the\wd\figBox}
\setlength{\tfheight}{\the\ht\figBox}
\setlength{\tfxa}{\tfwidth*\real{#3}}%
\setlength{\tfxb}{\tfwidth*\real{#4}}%
\setlength{\tfya}{\tfheight*\real{#5}}%
\setlength{\tfyb}{\tfheight*\real{#6}}%
\trimFigHeightWithBox{#1}{#2}{\tfxa}{\tfya}{\tfxb}{\tfyb}%
}
%
%


%
\renewcommand{\num}[2]{#1e#2} %

%
%
%

%
%
%
%
%
%

%
%

%
%

%
%

%
%

%
%

%
%

%
%

%
\bibliographystyle{elsarticle-num}
%

%
% Mathematical symbols and operators
%
% Author: Johann Rudi

% adjoint like
%\newcommand{\trans}{\top}
%\newcommand{\trans}{\intercal}
\newcommand{\trans}{\mathsf{T}}
\newcommand{\transpose}{^{\trans}}
\newcommand{\adjoint}{^{\ast}}
\newcommand{\dual}{^{\prime}}
\newcommand{\perpendicular}{^{\perp}}

% derivatives
\newcommand{\gradient}{\nabla}
\newcommand{\symgradient}{\nabla_{\!\mathrm{s}}}
\newcommand{\divergence}{\nabla\cdot}
\newcommand{\symdivergence}{\nabla_{\!\mathrm{s}}\cdot}
\newcommand{\totalderivative}{D}
\newcommand{\Laplace}{\Delta}

\newcommand{\partd}[2]{\frac{\partial #1}{\partial #2}}
\newcommand{\partdd}[2]{\frac{\partial^2 #1}{\partial {#2}^2}}
\newcommand{\partddd}[2]{\frac{\partial^3 #1}{\partial {#2}^3}}
\newcommand{\totald}[1]{\frac{\totalderivative #1}{\totalderivative t}}
\newcommand{\timed}[1]{\frac{\mathrm{d} #1}{\mathrm{d} t}}

% integration
\newcommand{\intd}{\,\mathrm{d}}
\newcommand{\indicatorfnc}{{\chi\smash[t]{\mathstrut}}}

% norms, floor/ceil
\newcommand{\abs}[1]{\left\lvert#1\right\rvert}
\newcommand{\magn}[1]{\left\lvert#1\right\rvert}
\newcommand{\norm}[1]{\left\lVert#1\right\rVert}
\newcommand{\ceil}[1]{\lceil#1\rceil}
\newcommand{\floor}[1]{\lfloor#1\rfloor}

% linear forms, inner products, duality pairings
\newcommand{\linearform}[2]{#1 \mspace{-2mu} \left(#2\right)}
\newcommand{\bilinearform}[3]{#1 \mspace{-2mu} \left(#2\mathbin{,}#3\right)}
\newcommand{\innerprod}[2]{\left(#1\mathbin{,}#2\right)}
\newcommand{\vecprod}[2]{\left\langle#1\mathbin{,}#2\right\rangle}
\newcommand{\dualprod}[4]{\left\langle#1\mathbin{,}#2\right\rangle_{#3\times#4}}

% function spaces
\newcommand{\Lebesgue}[1]{\ensuremath{ L^{#1} }}
\newcommand{\Lone}{\Lebesgue{1}}
\newcommand{\Ltwo}{\Lebesgue{2}}
\newcommand{\Linf}{\Lebesgue{\infty}}
\newcommand{\LL}{\Ltwo}
\newcommand{\Sobolev}[1]{\ensuremath{ H^{#1} }}

% Landau 'O' and 'o' notation
\newcommand{\LandauO}{\ensuremath{\mathcal{O}}}
\newcommand{\Landauo}{\ensuremath{\mathcal{o}}}

% 'log' type operators
\DeclareMathOperator{\diag}{diag}
\DeclareMathOperator{\diam}{diam}
\DeclareMathOperator{\e}{e}
\DeclareMathOperator{\erf}{erf}
\DeclareMathOperator{\erfc}{erf\mspace{1mu}c}
\DeclareMathOperator{\Id}{Id}
\DeclareMathOperator{\im}{i}
\DeclareMathOperator{\sgn}{sgn}
\DeclareMathOperator{\supp}{supp}
\DeclareMathOperator{\vecspan}{span}
\DeclareMathOperator{\dynamicratio}{DR}

% 'lim' type operators
\DeclareMathOperator*{\argmin}{arg\,min}
\DeclareMathOperator*{\argmax}{arg\,max}
\DeclareMathOperator*{\esssup}{ess\,sup}
\DeclareMathOperator*{\essinf}{ess\,inf}

% range and kernel
\DeclareMathOperator{\range}{Ran}
\DeclareMathOperator{\kernel}{Ker}

% sets
\newcommand{\set}[1]{\ensuremath{\mathbb{#1}}}
\newcommand{\N}{\set{N}}
\newcommand{\I}{\set{I}}
\newcommand{\setP}{\set{P}}
\newcommand{\Q}{\set{Q}}
\newcommand{\R}{\set{R}}
\newcommand{\Z}{\set{Z}}

% bold variables, vectors and matrices
\newcommand{\boldvar}[1]{\ensuremath{\boldsymbol{#1}}}
\newcommand{\boldvec}[1]{\ensuremath{\mathbf{#1}}}
\newcommand{\mat}[1]{\boldvec{#1}}
\ifdefined\vec
  \renewcommand{\vec}[1]{\boldvec{#1}}
\else
  \newcommand{\vec}[1]{\boldvec{#1}}
\fi

\newcommand{\be}{\boldvar{e}}
\newcommand{\bn}{\boldvar{n}}
\newcommand{\bT}{\boldvar{T}}
\newcommand{\bx}{\boldvar{x}}
\newcommand{\bzero}{\boldvar{0}}

\newcommand{\unitvec}{\vec{e}}
\newcommand{\zerovec}{\vec{0}}
\newcommand{\onevec}{\vec{1}}
\newcommand{\IdMat}{\mat{I}}
\newcommand{\ZeroMat}{\mat{0}}
\newcommand{\OneMat}{\mat{1}}

% punctuation at the end of equations
\newcommand{\eqcomma}{,} %{\,,}
\newcommand{\eqperiod}{.} %{\,.}

% about `~`
\newcommand{\about}{\ensuremath{\sim}}

% fluid dynamics
\newcommand{\timesym}{t}
\newcommand{\temp}{T}
\newcommand{\density}{\rho}
\newcommand{\vel}{\boldvar{u}}
\newcommand{\press}{p}

\newcommand{\visc}{\mu}
\newcommand{\bulkvisc}{\lambda}
\newcommand{\Peclet}{P\mspace{-1mu}e}

\newcommand{\strainrate}{\dot{\varepsilon}}
\newcommand{\strainratetensor}{\dot{\boldvar{\varepsilon}}}
\newcommand{\stress}{\sigma}
\newcommand{\stresstensor}{\boldvar{\sigma}}
\newcommand{\viscstress}{\tau}
\newcommand{\viscstresstensor}{\boldvar{\tau}}

\newcommand{\invII}[1]{#1_{\textrm{\textsc{ii}}}} %{#1_{I{\!}I}}
\newcommand{\strainrateinvII}{\invII{\strainrate}}
\newcommand{\viscstressinvII}{\invII{\viscstress}}

\newcommand{\rhsvel}{\boldvar{f}}
\newcommand{\rhspress}{g}

% thermodynamics
\newcommand{\vol}{v}
\newcommand{\heat}{q}

\newcommand{\conduct}{k}

\newcommand{\thermalexpan}{\alpha}
\newcommand{\thermalexpanref}{{\bar\alpha}}
\newcommand{\isothermalcompress}{{\chi\smash[t]{\mathstrut}}_{\temp}}
\newcommand{\isothermalcompressrep}{{\chi\smash[t]{\mathstrut}}_{\temp_{r}}}
\newcommand{\isothermalcompressref}{{{\bar\chi}\smash[t]{\mathstrut}}_{\temp}}

\newcommand{\specificheatconstvol}{c_{\vol}}
\newcommand{\specificheatconstvolrep}{c_{\vol_{r}}}
\newcommand{\specificheatconstvolref}{{\bar c}_{\vol}}
\newcommand{\specificheatconstpress}{c_{\press}}
\newcommand{\specificheatconstpressrep}{c_{\press_{r}}}
\newcommand{\specificheatconstpressref}{{\bar c}_{\press}}

\newcommand{\thermaldiffus}{\kappa}

\newcommand{\intheatproduction}{H}

% mantle flow physics
\newcommand{\BoundarySurf}{\Gamma_{\mathrm{surf}}}
\newcommand{\BoundaryCore}{\Gamma_{\mathrm{core}}}

\newcommand{\plateagemin}{\timesym_{\mathrm{pl,min}}}
\newcommand{\plateageback}{\timesym_{\mathrm{pl,back}}}

\newcommand{\viscmin}{\visc_{\mathrm{min}}}
\newcommand{\viscmax}{\visc_{\mathrm{max}}}
\newcommand{\visclit}{\visc_{\mathrm{lit}}}
\newcommand{\viscast}{\visc_{\mathrm{ast}}}
\newcommand{\viscUM}{\visc_{\textsc{um}}}
\newcommand{\viscLM}{\visc_{\textsc{lm}}}
\newcommand{\visctemp}{a}
\newcommand{\visctempof}[1]{\visctemp\left(#1\right)}
\newcommand{\visctempfactor}{c_{a}}
\newcommand{\visctempfactorLM}{c_{a_{\textsc{lm}}}}
\newcommand{\visctempfactorUM}{c_{a_{\textsc{um}}}}
\newcommand{\viscscalfactorLM}{c_{\textsc{lm}}}
\newcommand{\viscscalfactorUM}{c_{\textsc{um}}}
\newcommand{\activationenergy}{E_{a}}
\newcommand{\activationenergyLM}{E_{a,\textsc{lm}}}
\newcommand{\activationenergyUM}{E_{a,\textsc{um}}}

\newcommand{\weak}{w}
\newcommand{\weakof}[1]{\weak\left(#1\right)}
\newcommand{\weakmin}{w_{\mathrm{min}}}
\newcommand{\weakind}{{\chi\smash[t]{\mathstrut}}}

\newcommand{\stressexp}{n}
\newcommand{\viscstressyield}{\viscstress_{\mathrm{yield}}}
\newcommand{\strainrateshift}{d}

\newcommand{\gravity}{g}
\newcommand{\Prandtl}{P\mspace{-1mu}r}
\newcommand{\Rayleigh}{R\mspace{-1mu}a} % old: \mathrm{Ra}

% infinite-dimensional Stokes system
\newcommand{\FncSpVel}{\Sobolev{1}(\Omega)^{d}}
\newcommand{\FncSpVelBc}{\Sobolev{1}_{0,\bn}(\Omega)^{d}}
\newcommand{\FncSpPress}{\Lebesgue{2}(\Omega)}
\newcommand{\FncSpPressMean}{\Lebesgue{2}(\Omega)/\R}
\newcommand{\FncSpVelGrad}{\Lebesgue{2}(\Omega)^{d \times d}}
\newcommand{\FncSpVelDual}{\Sobolev{-1}(\Omega)^{d}}
\newcommand{\FncSpPressHiReg}{\Sobolev{2}(\Omega)}
\newcommand{\FncSpPressHiRegDual}{\Sobolev{-2}(\Omega)}

\newcommand{\gradvel}{\nabla_{\!\vel}}
\newcommand{\divvel}{\nabla_{\!\vel}\cdot}
\newcommand{\gradpress}{\nabla_{\!\press}}

\newcommand{\ViscStressOp}{A}
\newcommand{\GradOp}{B\adjoint}
\newcommand{\DivOp}{B}
\newcommand{\GradOpExt}{B\adjoint} %{\hat{B}\adjoint}
\newcommand{\DivOpExt}{B} %{\hat{B}}
\newcommand{\SchurOp}{S}
\newcommand{\PressPoissonOp}{K} %{\hat{K}}
\newcommand{\ViscStressApproxOp}{\tilde{\ViscStressOp}}
\newcommand{\SchurApproxOp}{\tilde{\SchurOp}}

\newcommand{\ViscStressApproxOpId}{\tilde{\ViscStressOp}_{1}}
\newcommand{\ViscStressApproxOpVisc}{\tilde{\ViscStressOp}_{1/\visc}}
\newcommand{\ViscStressApproxOpBFBT}{\tilde{\ViscStressOp}_{\mathrm{BFBT}}}
\newcommand{\ViscStressApproxOpWBFBT}{\tilde{\ViscStressOp}_{\mathrm{w\text{-}BFBT}}}
\newcommand{\SchurApproxOpId}{\tilde{\SchurOp}_{1}}
\newcommand{\SchurApproxOpVisc}{\tilde{\SchurOp}_{1/\visc}}
\newcommand{\SchurApproxOpBFBT}{\tilde{\SchurOp}_{\mathrm{BFBT}}}
\newcommand{\SchurApproxOpWBFBT}{\tilde{\SchurOp}_{\mathrm{w\text{-}BFBT}}}

\newcommand{\bfbtweight}{w}
\newcommand{\bfbtweightleft}{\bfbtweight_{l}}
\newcommand{\bfbtweightright}{\bfbtweight_{r}}

% discrete Stokes system
\newcommand{\discrQ}[1]{\ensuremath{ \Q_{#1} }}
\newcommand{\discrP}[1]{\ensuremath{ \setP_{#1} }}
\newcommand{\discrPdisc}[1]{\ensuremath{ \setP_{#1}^{\mathrm{disc}} }}
\newcommand{\discrQdisc}[1]{\ensuremath{ \Q_{#1}^{\mathrm{disc}} }}
\newcommand{\discrPbubble}[1]{\ensuremath{ \setP_{#1}^{\mathrm{bubble}} }}
\newcommand{\discrQPdisc}[2]{\ensuremath{ \discrQ{#1}\times\discrPdisc{#2} }}
\newcommand{\discrQQdisc}[2]{\ensuremath{ \discrQ{#1}\times\discrQdisc{#2} }}
\newcommand{\discrQQ}[2]{\ensuremath{ \discrQ{#1}\times\discrQ{#2} }}
\newcommand{\discrPP}[2]{\ensuremath{ \discrP{#1}\times\discrP{#2} }}
\newcommand{\discrPbubblePdisc}[2]{\ensuremath{ \discrPbubble{#1}\times\discrPdisc{#2} }}

\newcommand{\velvec}{\vec{u}}
\newcommand{\pressvec}{\vec{p}}
\newcommand{\rhsvelvec}{\vec{f}}
\newcommand{\rhspressvec}{\vec{g}}
\newcommand{\rotvec}{\vec{r}}

\newcommand{\ViscStressMat}{\mat{A}}
\newcommand{\GradMat}{\mat{B}\transpose}
\newcommand{\DivMat}{\mat{B}}
\newcommand{\SchurMat}{\mat{S}}
\newcommand{\ViscStressPcMat}{\tilde{\ViscStressMat}}
\newcommand{\SchurPcMat}{\tilde{\SchurMat}}

\newcommand{\MassMat}{\mat{M}}
\newcommand{\MassLumpMat}{\tilde{\mat{M}}}
\newcommand{\VelMassMat}{\MassMat_{\vel}}
\newcommand{\VelMassLumpMat}{\MassLumpMat_{\vel}}
\newcommand{\PressMassMat}{\MassMat_{\press}}
\newcommand{\PressMassLumpMat}{\MassLumpMat_{\press}}

\newcommand{\BfbtScalLeftMat}{\mat{C}}
\newcommand{\BfbtScalRightMat}{\mat{D}}
\newcommand{\WbfbtScalLeftMat}{\BfbtScalLeftMat_{\bfbtweightleft}}
\newcommand{\WbfbtScalRightMat}{\BfbtScalRightMat_{\bfbtweightright}}
\newcommand{\PressPoissonMat}{\mat{K}}
\newcommand{\PressPoissonLeftMat}{\PressPoissonMat_{l}}
\newcommand{\PressPoissonRightMat}{\PressPoissonMat_{r}}
\newcommand{\PressPoissonPcMat}{\tilde{\PressPoissonMat}}
\newcommand{\PressPoissonLeftPcMat}{\PressPoissonPcMat_{l}}
\newcommand{\PressPoissonRightPcMat}{\PressPoissonPcMat_{r}}

% variational formulation of Stokes system
\newcommand{\veltest}{\boldvar{v}}
\newcommand{\presstest}{q}
\newcommand{\velbasis}{\boldvar{\phi}}
\newcommand{\pressbasis}{\psi}

% 1st order variation of Stokes system
\newcommand{\velvar}{\hat{\vel}}
\newcommand{\pressvar}{\hat{\press}}
\newcommand{\residualmom}{\boldvar{r}_{\mathrm{mom}}}
\newcommand{\residualmass}{r_{\mathrm{mass}}}
\newcommand{\residualmomsub}[1]{\boldvar{r}_{\mathrm{mom}#1}}
\newcommand{\residualmasssub}[1]{r_{\mathrm{mass}#1}}

\newcommand{\velvarvec}{\hat{\velvec}}
\newcommand{\pressvarvec}{\hat{\pressvec}}
\newcommand{\residualmomvec}{\vec{r}_{\mathrm{mom}}}
\newcommand{\residualmassvec}{\vec{r}_{\mathrm{mass}}}

% Newton linearization
\newcommand{\srdual}{\boldvar{S}}

%
% Inversion & uncertainty quantification
%

\newcommand{\nparam}{{n_{m}}}
\newcommand{\nobs}{{n_{d}}}
\newcommand{\noise}{\vec{e}_{\mathrm{noise}}}

% parameters
\newcommand{\FncSpParam}{X}
\newcommand{\param}{m}
\newcommand{\incrparam}{\tilde\param}
\newcommand{\varincrparam}{\hat\param}
\newcommand{\paramprior}{\param_{\mathrm{pr}}}
\newcommand{\parammap}{\param_{\mathrm{MAP}}}

\newcommand{\paramvec}{\vec{\param}}
\newcommand{\paramvecmap}{\vec{\param}_{\mathrm{MAP}}}
\newcommand{\paramglovec}{\vec{\param}_{\mathrm{glo}}}
\newcommand{\paramweakvec}{\vec{\param}_{\mathrm{weak}}}
\newcommand{\parampriorvec}{\paramvec_{\mathrm{pr}}}
\newcommand{\parammapvec}{\paramvec_{\mathrm{MAP}}}

% observations and data
\newcommand{\FncSpObs}{Y}
\newcommand{\data}{d}
\newcommand{\datatrue}{\data_{\mathrm{true}}}

\newcommand{\datavec}{\vec{d}}
\newcommand{\datatruevec}{\datavec_{\mathrm{true}}}
%\newcommand{\obs}{\data_{\mathrm{obs}}}
%\newcommand{\obsmap}{\vec{f}}

\newcommand{\ObsOp}{\mathcal{F}}
\newcommand{\ObsOpDeriv}{\boldvar{F}}
\newcommand{\ObsOpJacobian}{\mat{F}}

% probability density functions
\newcommand{\pdf}{\pi}
\newcommand{\pdfcond}[2]{\pdf\left(#1\mid#2\right)}
\newcommand{\pdfprior}{\pdf_{\mathrm{pr}}}
\newcommand{\pdfpost}{\pdf_{\mathrm{post}}}
\newcommand{\pdflike}{\pdf_{\mathrm{like}}}
\newcommand{\pdflikecond}[2]{\pdf_{\mathrm{like}}\left(#1\mid#2\right)}

%
\newcommand{\varisymbol}[1]{\tilde#1}
\newcommand{\variisymbol}[1]{\hat#1}
\newcommand{\VarI}[2]{\delta_{#1}[#2](\varisymbol#1)}
\newcommand{\VarII}[2]{\delta_{#1}[#2](\variisymbol#1)}
\newcommand{\VarVarII}[3]{\delta_{#1}\delta_{#2}[#3]%
                            (\varisymbol#2)(\variisymbol#1)}
\newcommand{\fwd}{u}
\newcommand{\adj}{v}
\newcommand{\incrfwd}{\varisymbol\fwd}
\newcommand{\incradj}{\varisymbol\adj}
\newcommand{\varincrfwd}{\variisymbol\fwd}
\newcommand{\varincradj}{\variisymbol\adj}
\newcommand{\fwdrhs}{f}

\newcommand{\FncSpForward}{V}
\newcommand{\FncSpAdjoint}{V}
\newcommand{\FncSpForwardDual}{\FncSpForward^\prime}
\newcommand{\ForwardOp}{\ensuremath{\mathcal{A}}}
\newcommand{\ObjectiveMAP}{\ensuremath{\mathcal{J}}}

\newcommand{\LagrangianGrad}{\ensuremath{\mathcal{L}_g}}
\newcommand{\LagrangianHess}{\ensuremath{\mathcal{L}_H}}

% Gaussians
\newcommand{\NormalDistrib}{\mathcal{N}}
\newcommand{\CovOp}{\mathscr{C}}
\newcommand{\CovMat}{\mat{\Gamma}}

\newcommand{\CovOpPrior}{\CovOp_\mathrm{pr}}
\newcommand{\CovMatPrior}{\CovMat_\mathrm{pr}}
\newcommand{\CovOpNoise}{\CovOp_\mathrm{noise}}
\newcommand{\CovMatNoise}{\CovMat_\mathrm{noise}}
\newcommand{\CovOpPost}{\CovOp_\mathrm{post}}
\newcommand{\CovMatPost}{\CovMat_\mathrm{post}}

%
\newcommand{\ParamGradientOp}{\ensuremath{\mathcal{G}}}
\newcommand{\ParamHessianOp}{\ensuremath{\mathcal{H}}}
\newcommand{\HessianMisfit}{\mat{H}_\mathrm{misfit}}

%
% Iterative methods
%

\newcommand{\PC}[1]{\tilde{#1}^{-1}}
\newcommand{\IterProc}{\mathcal{I}}
\newcommand{\ErrorOp}{\mathcal{E}}

\newcommand{\MG}{M\mspace{-1mu}G}

%
% Implementation
%

% parallelization
\newcommand{\mpisize}{\ensuremath{n_{\mathrm{mpi}}}}
\newcommand{\ompsize}{\ensuremath{n_{\mathrm{omp}}}}

% MG, general
\newcommand{\MGlevel}{\ensuremath{l}}
\newcommand{\MGnLevels}{\ensuremath{L}}

% GMG
\newcommand{\GMGlevel}{\ensuremath{\MGlevel_{\textsc{gmg}}}}
\newcommand{\GMGnLevels}{\ensuremath{\MGnLevels_{\textsc{gmg}}}}
\newcommand{\GMGnDgNodesPerProc}{\ensuremath{n_{p}}}
\newcommand{\GMGnDgNodesPerProcMin}{\ensuremath{n_{p,\mathrm{min}}}}
\newcommand{\GMGnDofCoarse}{\ensuremath{N_{\MGnLevels-1}}}

% AMG
\newcommand{\AMGlevel}{\ensuremath{\MGlevel_{\textsc{amg}}}}
\newcommand{\AMGnLevels}{\ensuremath{\MGnLevels_{\textsc{amg}}}}
\newcommand{\AMGnDofPerProc}{\ensuremath{m_{p}}}
\newcommand{\AMGnDofCoarse}{\ensuremath{M_{\MGnLevels-1}}}
\newcommand{\AMGnProcMax}{\ensuremath{P_{\textsc{amg}}}}
\newcommand{\AMGmemPerDof}{\ensuremath{c_{\textsc{dof}}}}
\newcommand{\AMGnzzPerLevel}{\ensuremath{D}}
\newcommand{\AMGnzzPerDof}{\ensuremath{d}}
\newcommand{\AMGdofReduction}[2]{\ensuremath{r_{#1 \rightarrow #2}}}
\newcommand{\AMGnzzIncrease}[2]{\ensuremath{f_{#1 \rightarrow #2}}}

% architecture
\newcommand{\archMaxMemPerProc}{\ensuremath{b_{\mathrm{max}}}}


\newcommand{\av}{\boldsymbol{a}}
\newcommand{\bv}{\boldsymbol{b}}
\newcommand{\cv}{\boldsymbol{c}}
\newcommand{\dv}{\boldsymbol{d}}
\newcommand{\ev}{\boldsymbol{e}}
\newcommand{\fv}{\boldsymbol{f}}
\newcommand{\gv}{\boldsymbol{g}}
\newcommand{\hv}{\boldsymbol{h}}
\newcommand{\iv}{\boldsymbol{i}}
\newcommand{\jv}{\boldsymbol{j}}
\newcommand{\kv}{\boldsymbol{k}}
\newcommand{\lv}{\boldsymbol{l}}
\newcommand{\mv}{\boldsymbol{m}}
\newcommand{\nv}{\boldsymbol{n}}
\newcommand{\ov}{\boldsymbol{o}}
\newcommand{\pv}{\boldsymbol{p}}
\newcommand{\qv}{\boldsymbol{q}}
\newcommand{\rv}{\boldsymbol{r}}
\newcommand{\sv}{\boldsymbol{s}}
\newcommand{\tv}{\boldsymbol{t}}
\newcommand{\uv}{\boldsymbol{u}}
\newcommand{\vv}{\boldsymbol{v}}
\newcommand{\wv}{\boldsymbol{w}}
\newcommand{\xv}{\boldsymbol{x}}
\newcommand{\yv}{\boldsymbol{y}}
\newcommand{\zv}{\boldsymbol{z}}

\newcommand{\Av}{\boldsymbol{A}}
\newcommand{\Bv}{\boldsymbol{B}}
\newcommand{\Cv}{\boldsymbol{C}}
\newcommand{\Dv}{\boldsymbol{D}}
\newcommand{\Ev}{\boldsymbol{E}}
\newcommand{\Fv}{\boldsymbol{F}}
\newcommand{\Gv}{\boldsymbol{G}}
\newcommand{\Hv}{\boldsymbol{H}}
\newcommand{\Iv}{\boldsymbol{I}}
\newcommand{\Jv}{\boldsymbol{J}}
\newcommand{\Kv}{\boldsymbol{K}}
\newcommand{\Lv}{\boldsymbol{L}}
\newcommand{\Mv}{\boldsymbol{M}}
\newcommand{\Nv}{\boldsymbol{N}}
\newcommand{\Ov}{\boldsymbol{O}}
\newcommand{\Pv}{\boldsymbol{P}}
\newcommand{\Qv}{\boldsymbol{Q}}
\newcommand{\Rv}{\boldsymbol{R}}
\newcommand{\Sv}{\boldsymbol{S}}
\newcommand{\Tv}{\boldsymbol{T}}
\newcommand{\Uv}{\boldsymbol{U}}
\newcommand{\Vv}{\boldsymbol{V}}
\newcommand{\Wv}{\boldsymbol{W}}
\newcommand{\Xv}{\boldsymbol{X}}
\newcommand{\Yv}{\boldsymbol{Y}}
\newcommand{\Zv}{\boldsymbol{Z}}


%
\newcommand{\indfn}{{\chi\smash[t]{\mathstrut}}}
\newcommand{\indfnAniso}{\boldsymbol{\indfn}}
\newcommand{\indmin}{\indfn_\mathrm{min}}
\newcommand{\indmax}{\indfn_\mathrm{max}}

%
\newcommand{\textcomment}[2]{\emph{\textcolor{#1}{[#2]}}}
\newcommand{\jr}[1]{\textcomment{red}{JR: #1}}
\newcommand{\QT}[1]{{\textcomment{teal}{QT: #1}}}
\newcommand{\EC}[1]{\textcomment{red}{EC: #1}}

%
\newif\ifshowstatus
\showstatustrue %

%
%
%

\begin{document}

\begin{frontmatter}

\title{%
  Scalable Implicit %
  Solvers with Dynamic Mesh Adaptation for a
  Relativistic Drift-Kinetic Fokker--Planck--Boltzmann Model%
  \tnoteref{mytitlenote}}
\tnotetext[mytitlenote]{This work was jointly supported by the
  U.S. Department of Energy through the Fusion Theory Program of the
  Office of Fusion Energy Sciences and the SciDAC partnership on
  Tokamak Disruption Simulation between the Office of Fusion Energy
  Sciences and the Office of Advanced Scientific Computing; and
  through the FASTMath Institute. Los Alamos
  National Laboratory is operated by Triad National Security, LLC, for
  the National Nuclear Security Administration of U.S. Department of
  Energy (Contract No.~89233218CNA000001). Argonne National Laboratory
  is operated under contract DE-AC02-06CH11357.
}

%
\author[anl,vt]{Johann Rudi\corref{cor1}}
\ead{jrudi@vt.edu}
%

\author[anl,vt,bu]{Max Heldman}
\ead{maxh@vt.edu}
%

\author[anl]{Emil M. Constantinescu}
\ead{emconsta@mcs.anl.gov}
%

\author[lanl]{Qi Tang\corref{cor1}}
\ead{qtang@lanl.gov}
%

\author[lanl]{Xian-Zhu Tang}
\ead{xtang@lanl.gov}
%

\address[anl]{Mathematics and Computer Science Division, Argonne National Laboratory, Lemont, IL 60439.}
\address[vt]{Department of Mathematics, Virginia Tech, Blacksburg, VA 24061.}
\address[bu]{Department of Mathematics and Statistics, Boston University, Boston, MA 02215.}
\address[lanl]{Theoretical Division, Los Alamos National Laboratory, Los Alamos, NM 87545.}

\cortext[cor1]{Corresponding author}


\begin{abstract}
In this work we consider a relativistic drift-kinetic model for
runaway electrons along with a Fokker--Planck operator for small-angle
Coulomb collisions, a radiation damping operator, and a secondary
knock-on (Boltzmann) collision source.  We develop a new scalable fully
implicit solver utilizing finite volume and conservative finite
difference schemes and dynamic mesh adaptivity. A new data management
framework in the PETSc library based on the p4est library is
developed to enable simulations with dynamic adaptive mesh
refinement (AMR), parallel computation, and load balancing.  This
framework is tested through the development of the runaway electron
solver that is able to dynamically capture both bulk Maxwellian at the
low-energy region and a runaway tail at the high-energy region.
To effectively capture features via the AMR algorithm, a new AMR
indicator prediction strategy is proposed that is performed alongside the
implicit time evolution of the solution.  This strategy is complemented by the
introduction of computationally cheap feature-based AMR indicators that are
analyzed theoretically.  Numerical results quantify the advantages of the
prediction strategy in better capturing features compared with nonpredictive
strategies; and we demonstrate trade-offs regarding computational costs.  The
full solver is further verified through several benchmark problems including
manufactured solutions and solutions of physics models.  We particularly focus
on demonstrating the advantages of using implicit time stepping and AMR for
runaway electron simulations.
\end{abstract}

\begin{keyword}
Relativistic Fokker--Planck--Boltzmann\sep Adaptive mesh refinement\sep Fully implicit time stepping\sep Runaway electrons
\end{keyword}
\end{frontmatter}

%
%

%
%

%
%
%

\section{Introduction}
\label{sec:introduction}
% \begin{itemize}
%     % Diffusion of FL
%     \item {\st{Diffusion of FL}}
%     % Security threats to FL
%     \item {\st{Security threats to FL with particular focus on model poisoning}}
%     % Limitations of existing countermeasures
%     \item {\st{Current countermeasures (e.g., KRUM) and their limitations}}
%     % Proposed method and its advantages
%     \item {\st{Intuitive description of the proposed method and its difference (i.e., advantages) w.r.t. state of the art}}
%     % Main contributions
%     \item {\st{Summary of the main contributions of this work}}
%     % Paper's structure and organization
%     \item {\st{Paper's structure and organization}}
% \end{itemize}

% Diffusion of FL
Recently, {\em federated learning} (FL) has emerged as the leading paradigm for training distributed, large-scale, and privacy-preserving machine learning (ML) systems~\cite{mcmahan2017googleai,mcmahan2017aistats}. 
The core idea of FL is to allow multiple edge clients to collaboratively train a shared, global model without disclosing their local private training data.
%Specifically, an FL system consists of a central server and many edge clients; 
A typical FL round involves the following steps: {\em(i)} the server randomly picks some clients and sends them the current, global model; {\em(ii)} each selected client locally trains its model with its own private data; then, it sends the resulting local model to the server;\footnote{Whenever we refer to global/local model, we mean global/local model {\em parameters}.} {\em(iii)} the server updates the global model by computing an \emph{aggregation function}, usually the average (FedAvg), on the local models received from clients.
% \begin{enumerate}
%     \item[{\em(i)}] the server sends the current, global model to the clients and appoints some of them for training;
%     \item[{\em(ii)}] each selected client locally trains its copy of the global model with its own private data; then, it sends the resulting local model back to the server;\footnote{Whenever we refer to global/local model, we mean global/local model {\em parameters}.}
%     \item[{\em(iii)}] the server updates the global model by computing an \emph{aggregation function} on the local models received from clients (by default, the average, also referred to as FedAvg~\cite{mcmahan2017aistats}).
% \end{enumerate}
This process goes on until the global model converges. %(e.g., after a certain number of rounds or other similar stopping criteria).
%\\
% The advantages of FL over the traditional, centralized learning paradigm are undoubtedly clear in terms of flexibility/scalability (clients can join/disconnect from the FL network dynamically), network communications (only model weights\footnote{We will use \textit{parameters} and \textit{weights} interchangeably.} are exchanged between clients and server), and privacy (each client's private training data is kept local at the client's end and not uploaded to the server).
\\
% Security threats to FL
%However, the growing adoption of FL also raises security concerns~\cite{costa2022covert}, particularly about its confidentiality, integrity, and availability.
Although its advantages over standard ML, FL also raises security concerns~\cite{costa2022covert}. %, particularly about its confidentiality, integrity, and availability~\cite{costa2022covert}.
% OLD, LONG VERSION
% Indeed, some work deals with privacy leakage that may expose the local data of some clients~\cite{melis2019sp}. 
% A large body of work, instead, investigates attacks that usually aim to detriment the predictive accuracy of the learned global model. For instance, \emph{data poisoning} attacks achieve this goal by letting an adversary pollute the training set of some corrupt FL clients with maliciously crafted examples~\cite{jagielski2018sp}.
% Similarly, in \emph{model poisoning} the attacker attempts to tweak the global model weights~\cite{bhagoji2019pmlr} by directly perturbing the local model's weights of some infected FL clients before these are sent to the central server for aggregation, usually via so-called Byzantine attacks. 
% It turns out that Byzantine model poisoning attacks severely impact standard FedAvg; therefore, more robust aggregation functions must be designed to make FL systems secure.
Here, we focus on \emph{untargeted model poisoning} attacks~\cite{bhagoji2019pmlr}, where an adversary attempts to tweak the global model weights %\footnote{We will use the terms \textit{parameters} and \textit{weights} interchangeably.} 
by directly perturbing the local model's parameters of some infected clients before these are sent to the central server for aggregation.
In doing so, the adversary aims to jeopardize the global model \textit{indiscriminately} at inference time.
Such model poisoning attacks severely impact standard FedAvg; therefore, more robust aggregation functions must be designed to secure FL systems.
\\
% In this paper, we focus on designing a novel robust aggregation scheme at the server's end to contrast the effect of Byzantine model poisoning attacks.
%
% Current countermeasures and their limitations
%Several countermeasures have been proposed in the literature to combat model poisoning attacks on FL systems.
% Some methods use simple statistics more robust than plain average to smooth the impact of malicious updates (e.g., Trimmed Mean and FedMedian~\cite{yin2018icml}). 
% Other defenses implement outlier detection techniques to discard malicious updates from the aggregation performed at the server's end. Those are either based on heuristics (e.g., Krum/Multi-Krum~\cite{blanchard2017nips} and Bulyan~\cite{mhamdi2018pmlr}) or data-driven approaches (e.g., K-means clustering~\cite{shen2016acm} or DnC via spectral analysis~\cite{shejwalkar2021ndss}). 
% Finally, some strategies rely on a centralized ``source of trust'' to spot potential malicious updates (e.g., FLTrust~\cite{cao2020fltrust}).
% Several countermeasures have been proposed in the literature to combat model poisoning attacks on FL systems, i.e., to discard possible malicious local updates from the aggregation performed at the server's end. 
% These techniques range from simple statistics more robust than plain average (e.g., Trimmed Mean and FedMedian~\cite{yin2018icml}) to outlier detection heuristics (e.g., Krum/Multi-Krum~\cite{blanchard2017nips} and Bulyan~\cite{mhamdi2018pmlr}) or data-driven approaches (e.g., spectral analysis via K-means clustering~\cite{shen2016acm} or spectral analysis), or methods based on ``source of trust'' (e.g., FLTrust~\cite{cao2020fltrust}).
% OLD, LONG VERSION
%Several countermeasures have been proposed in the literature to combat Byzantine model poisoning attacks on FL systems.
% Descriptive statistics
% For example, Trimmed Mean and FedMedian aggregate local model updates using more robust statistics than standard average~\cite{yin2018icml}.
%
% % Heuristics for outlier detection
% Many existing Byzantine-resilient strategies implement some outlier detection heuristics to discard the model updates sent by potentially malicious clients from the input of the aggregation function.
% One of the most popular heuristics is Krum~\cite{blanchard2017nips}.
% This strategy tries to mitigate the impact of Byzantine attacks by selecting as a global model the local model with the smallest sum of Euclidean distances to {\em all} the other local models.
% Although powerful, Krum requires the server to know (or, at least, estimate) the number of malicious FL clients upfront, which is generally impossible in a realistic attack scenario. %
% Moreover, Krum may become ineffective for complex, high-dimensional model parameter spaces due to the curse of dimensionality.
% Bulyan~\cite{mhamdi2018pmlr} tries to overcome this issue by combining Krum with a variant of Trimmed Mean.
% % Data-driven outlier detection
% Other strategies use data-driven outlier detection techniques -- e.g., via K-means clustering~\cite{shen2016acm} -- to spot potential malicious local model updates. 
% %For instance, Shen et al. propose to cluster local model updates with K-means and thus identify outliers.
%
% % Other techniques
% As far as the server is concerned, any local model received can be from a potential malicious client. 
% FLTrust~\cite{cao2020fltrust} assumes the server acts as a client, i.e., trains a local model on an additional {\em trustworthy} dataset at the server's end and compares it against all the local models from other clients. 
% This way, the server can rely on some ``source of trust'' when discarding potentially malicious clients.
%\\
% Limitations of existing Byzantine-resilient strategies
Unfortunately, existing defense mechanisms either rely on simple heuristics (e.g., Trimmed Mean and FedMedian by~\cite{yin2018icml}) or need strong and unrealistic assumptions to work effectively (e.g., foreknowledge or estimation of the number of malicious clients in the FL system, as for Krum/Multi-Krum~\cite{blanchard2017nips} and Bulyan~\cite{mhamdi2018pmlr}, which, however, cannot exceed a fixed threshold).
Furthermore, outlier detection methods using K-means clustering~\cite{shen2016acm} or spectral analysis like DnC~\cite{shejwalkar2021ndss} do not directly consider the temporal evolution of local model updates received.
Finally, strategies like FLTrust~\cite{cao2020fltrust} require the server to collect its own dataset and act as a proper client, thereby altering the standard FL protocol.
\\
% OLD, LONG VERSION
% Overall, existing Byzantine-resilient strategies are either simple heuristics (e.g., FedMedian) or, if they are more complex, they rely on strong and unrealistic assumptions to work effectively (e.g., knowing the number of malicious clients in the FL system in advance, as for Krum and alike).
% Furthermore, data-driven outlier detection methods do not consider the temporary evolution of local model updates received (e.g., K-means clustering). 
% Finally, strategies like FLTrust requires the server to collect its own dataset and act as a proper client, thereby altering the standard FL protocol.
%
% Description of the proposed method
This work introduces a novel pre-aggregation \textit{filter} robust to untargeted model poisoning attacks. Notably, this filter $(i)$ operates without requiring prior knowledge or constraints on the number of malicious clients and $(ii)$ inherently integrates temporal dependencies. 
The FL server can employ this filter as a preprocessing step before applying \textit{any} aggregation function, be it standard like FedAvg or robust like Krum or Bulyan.
Specifically, we formulate the problem of identifying corrupted updates as a multidimensional (i.e., matrix-valued) time series anomaly detection task. 
The key idea is that legitimate local updates, resulting from well-calibrated iterative procedures like stochastic gradient descent (SGD) with an appropriate learning rate, show \textit{higher predictability} compared to malicious updates. This hypothesis stems from the fact that the sequence of gradients (thus, model parameters) observed during legitimate training exhibit regular patterns, as validated in Section~\ref{subsec:intuition}. %until convergence. 
%This regularity may be more pronounced for smooth convex loss functions, but it can still be captured within an appropriate time window, even for more complex and convoluted loss surfaces. 
%We provide evidence of this claim in Appendix~B, where we show that the average mutual information (i.e., ``predictability''), calculated over pairs of legitimate model updates sent at different FL rounds, is significantly higher than the corresponding computation for a malicious client.
\\
Inspired by the matrix autoregressive (MAR) framework for multidimensional time series forecasting~\cite{chen2021je}, we propose the FLANDERS ({\em \textbf{F}ederated \textbf{L}earning meets \textbf{AN}omaly \textbf{DE}tection for a \textbf{R}obust and \textbf{S}ecure}) filter.
The main advantages of FLANDERS over existing strategies like FLDetector~\cite{zhao2020multivariate} are its resilience to large-scale attacks, where $50\%$ or more FL participants are hostile, and the capability of working under realistic non-iid scenarios.
We attribute such a capability to two key factors: $(i)$ FLANDERS works without knowing a priori the ratio of corrupted clients, and $(ii)$ it embodies temporal dependencies between intra- and inter-client updates, quickly recognizing local model drifts caused by evil players. Below, we summarize our main contributions:

\begin{itemize}
\item[{\em(i)}]
We provide empirical evidence that the sequence of models sent by legitimate clients is more predictable than those of malicious participants performing untargeted model poisoning attacks.
\\
\item[{\em(ii)}] 
We introduce FLANDERS, the first pre-aggregation filter for FL robust to untargeted model poisoning based on multidimensional time series anomaly detection.
\\
\item[{\em(iii)}] 
We integrate FLANDERS into Flower,\footnote{\scriptsize{\url{https://flower.dev/}}} a popular FL simulation framework for reproducibility.
\\
\item[{\em(iv)}] 
We show that FLANDERS improves the robustness of the existing aggregation methods under multiple settings: different datasets, client's data distribution (non-iid), models, and attack scenarios.
\\
\item[{\em(v)}] 
We publicly release all the implementation code of FLANDERS along with our experiments.\footnote{\scriptsize{\url{https://anonymous.4open.science/r/flanders_exp-7EEB}}}
\end{itemize}

% Paper's structure and organization
The remainder of the paper is structured as follows. %some related work and the current state-of-the-art solutions to security issues that FL entails. 
Section~\ref{sec:background} covers background and preliminaries. 
In Section~\ref{sec:related}, we discuss related work.
Section~\ref{sec:problem} and Section~\ref{sec:method} describe the problem formulation and the method proposed. % to tackle it. 
Section~\ref{sec:experiments} gathers experimental results. %, and Section~\ref{sec:limitations} discusses some limitations of this work.
Finally, we conclude in Section~\ref{sec:conclusion}.
 %discusses the limitations of this work and draws future research directions.
%reports conclusions and draws perspectives for future research directions.

%%%%%%% OLD %%%%%%%
%to overcome the resilience of Byzantine failures in distributed Stochastic Gradient Descent computations. 
% The strength of Krum is its time complexity, which is linear in the gradient dimension. 
% However, the robustness of the approach is guaranteed for gradient-based learning applications only when the majority of the clients are not compromised. 
% Besides, the aggregation mechanism of Krum, as well as that of similar methods, is robust from a coarse-grained perspective and does not provide solutions to errors and perturbations that may occur at inference time.
%A related approach to~\cite{blanchard2017nips} is the work of Su et al.~\cite{su2016dc}. Here, the authors propose an iterated approximate agreement to tackle a multi-layer scenario attacked by Byzantine agents. 
%However, the method works efficiently on the sole discrete context and it is inapplicable to continuous state environments.
%\gabri{Maybe, we should just talk about the main limitations of existing countermeasures without digging into their details (or, we can just mention Krum as this is the most popular one). I will move the description of all these methods to the Related Work section.}

%
%
%

\def\secstatus{finished}

\section{Governing equations}
\label{sec:governing-eqns}

We study the phase-space dynamics of a runaway electron distribution
and consider a simplified version of the relativistic drift-kinetic
Fokker--Planck--Boltzmann equation~\cite{brizard1999nonlinear}. The
governing equations we are considering are effectively described in
the spherical coordinates of $(p, \xi, \theta)$, with $p$ the
normalized momentum magnitude, $\xi = p_\parallel/p$ the pitch,
and $\theta$ the azimuthal angle.  Here $p$ is dimensionless
with normalization by $m_e c$, with $m_e$ the rest mass
of an electron and $c$ the speed of light, and $\xi$ is the cosine
of the polar angle in the standard spherical coordinate.  The parallel
direction,  denoted by subscript $\parallel,$ aligns with the
magnetic field.  We further follow the guiding center model and assume
the solution has azimuthal symmetry, which reduces the momentum space
to two dimensions.  This corresponds to a momentum-space domain of
$[0, +\infty]\times[-1, 1]$.  The normalized relativisitic
Fokker--Planck--Boltzmann (RFP) equation to describe runaway electron
distribution, $f(t, p, \xi),$ in a slab geometry, is given by
\begin{align}
\label{eqn:rfp}
\frac{\partial f}{\partial t} - E \left( \xi \frac{\partial f}{\partial p} + \frac{1-\xi^2}{p} \frac{\partial f}{\partial \xi} \right) = C(f) + \alpha R(f) + S(f)\,,
\end{align}
where $E$ stands for the normalized electric field parallel to the
magnetic field in a tokamak and its normalization scale is
$E_c \coloneqq m_e c/ e \tau_c$ with $e$ the elementary charge
and $\tau_c$ the time scale.  Here the time scale of the
equation, $\tau_c$, is the relativistic electron collision time given
by
\begin{align}
\tau_c \coloneqq \frac{ 4\pi \epsilon_0^2 m_e^2 c^3}{e^4 n_e \ln \Lambda},
\end{align}
where $n_e$ is the (background) thermal electron density, $\epsilon_0$
is the electrical permittivity, and $\ln \Lambda$ is the Coulomb
logarithm.  The so-called Connor--Hastie $E_c$ sets the critical value for
runaway electron generation~\cite{connor1975relativistic}.  In the
current study all those values are given as constant based on
practical devices such as ITER. On the right-hand side are
several secondary effects due to different types of collisions and
synchrotron radiation (defined in detail later), and $\alpha$ is
introduced to describe the intensity of radiation damping. All the physical
quantities appearing in the current work have been summarized
in~\ref{sec:quantities}.

This problem has been well studied by plasma
physicists~\cite{GuoMcDevittTang2017, stahl2017norse,
hesslow2018effect, Guo-etal-PoP-2019, daniel2020fully}, commonly using conservative
finite difference or finite volume schemes on a stretched grid. The
dynamics of the equation is primarily dominated by the advection term
due to the strong electric field. The distribution function has a
sharp boundary layer at $p = p_{\min}$ due to the boundary condition
(defined in detail later); and as time evolves, the distribution
accumulates around $\xi = -1$ if $E>0$ (or $\xi = 1$ if $E<0$) due to
the advection. The distribution is further impacted by the collisions
and the source term in the large $p$ region and eventually forms a
nontrivial tail structure that is of the greatest interest. A majority
of the previous work uses the stretched grid to resolve the localized
distribution and tails. Adaptive mesh refinement has not been
explored on this problem to the best of our knowledge.

We note that in~\cite{daniel2020fully} a different coordinate of
$(p_\parallel, p_\perp)$ is used.
Choosing the coordinate of $(p, \xi)$ for the RFP equation has several advantages. The
solution is highly localized, as found in~\cite{GuoMcDevittTang2017,
stahl2017norse, hesslow2018effect,Guo-etal-PoP-2019, daniel2020fully}. It is more
efficient to resolve the same structure in $(p, \xi)$ while the
coordinate of $(p_\parallel, p_\perp)$ leads to half of the domain
resolving a trivial distribution tail.  Note that the physics of
interest in this study is the small population of runaway electrons.
It is therefore much easier to resolve this small population tail in
$(p, \xi)$ by chopping off the large Maxwellian bulk close to $p = 0$.
%
One disadvantage of using $(p, \xi)$ is the coordinate singularity
when $p_{\min}\rightarrow 0,$ and the fine grid spacing in $\xi$ near
the $\xi=-1$ boundary where runaway electrons of small $p_\perp=\xi p$
are needed to be resolved for large $p.$ Here we set $p_{\min}$ to
correspond to a few times the thermal energy of the background
electrons, and we impose an approximate Dirichlet boundary condition given
by the Maxwellian bulk, which is commonly adopted in the physics
codes~\cite{GuoMcDevittTang2017, stahl2017norse, hesslow2018effect,
Guo-etal-PoP-2019, daniel2020fully} and will be detailed later.

\subsection{Coulomb collision}
In this study the bulk distribution is considered to be close to a
Maxwellian with a thermal speed
much less than the light speed, namely, $v_t/c \ll 1$, where $v_t$
stands for the thermal velocity of the background electrons.  The
(small-angle) Coulomb collision of the runaway electrons, which is of
much lower density, with the
background electrons, $C(f),$ can be approximated by the so-called
test-particle collision operator~\cite{GuoMcDevittTang2017,
stahl2017norse}. We use the test-particle collision operator proposed
in~\cite{papp2011runaway} that is applicable to both thermal
($p\sim v_t/c \ll 1$) and relativistic ($p > 1$) electrons.  This
Coulomb collision operator has the form
\begin{align}
\label{eq:collision}
  C(f) =
    \frac{1}{p^2}\partd{}{p}\left[p^2\left(C_F f + C_A\partd{f}{p}\right)\right] +
    \frac{C_B}{p^2}\partd{}{\xi}\left[\left(1-\xi^2\right)\partd{f}{\xi}\right],
\end{align}
where all the collision coefficients are $p$-dependent and given by
\begin{align*}
  C_F &= \frac{2 c^2}{v_t^2}\Psi(x),\\
  C_A &= \frac{\sqrt{1+p^2}}{p}\Psi(x),\\
  C_B &= \frac{\sqrt{1+p^2}}{2p}
         \left(Z + \erf(x) - \Psi(x) + \frac{v_t^2} {2 c^2} \, \frac{p^2}{1+p^2}\right).
\end{align*}
Here $x$ is $p$-dependent,
\[
  x = \frac{c}{v_t} \, \frac{p}{\sqrt{1+p^2}},
\]
$Z$ stands for a charge number, and $\Psi(x) = [\erf(x) -
x \erf'(x)]/(2x^2)$ is the Chandrasekhar function.  A study of the
collision coefficients in a transition from nonrelativistic to
relativistic regimes is presented in~\cite{GuoMcDevittTang2017}, where the authors show   that the collision coefficients at small $p$ are several
orders of magnitude larger than those at larger $p$ and thus the
diffusion becomes dominant compared with the advection field.
%
Therefore, a fully implicit treatment of the collision operator is
necessary.  When needed, the collision coefficients are also modified
to include the partial screening effect~\cite{hesslow2018effect}, which
addresses the role of bound electrons in collisions between fast
electrons and partially ionized impurities.  Mathematically, it
corresponds to modifying the $C_B$ and the Coulomb logarithm
$\ln \Lambda$, but it does not involve additional numerical
challenges.  For more details, see~\cite{hesslow2017effect,
hesslow2018effect}.

\subsection{Synchrotron radiation}
When there is an acceleration of a charged particle,
there is an energy loss through the emitted
electromagnetic radiation.
In a dominating magnetic field where a charged particle follows cyclotron motion,
relativistic electrons can emit a significant amount of synchrotron radiation.  The
resulting radiation damping operator satisfies
\begin{align}
\label{eq:radiation}
  R(f) = \frac{1}{p^2}\partd{}{p}\left[p^3\gamma(1-\xi^2)f\right] -
               \partd{}{\xi}\left[\frac{\xi(1-\xi^2)}{\gamma}f\right],
\end{align}
where $\gamma=\sqrt{1+p^2}$ is the Lorentz factor.  The dimensionless
parameter for the intensity of damping is formally defined as
$\alpha\coloneqq\tau_c/\tau_s$, a ratio between the collision time
scale and the synchrotron radiation time scale, where
$\tau_s \coloneqq 6\pi \epsilon_0 m_e^3 c^3/e^4 B^2$ and $B$ is the
magnitude of the magnetic field.  Typically it is about 0.1 to 0.3 for
ITER-like tokamaks and 0.001 to 0.05 for a smaller device, which is
the range we use in our simulations.

\subsection{Knock-on collision}
\label{sec:knock_on}
Another important source in runaway electron generations is the
avalanche mechanism due to knock-on collisions between runaway
electrons and the background thermal
electrons~\cite{Rosenbluth_1997,chiu1998fokker}.  The knock-on source
considers the secondary large-angle collisions in addition to the
primary small-angle collisions of the Fokker--Planck
operator~\cite{boozer-pop-2015,breizman2019physics}.  The knock-on
source is
\begin{align}
S(p, \xi) = S_1(p, \xi) + S_2(p,\xi),
\label{eqn:chiu_full}
\end{align}
which consists of the secondary electron source term
\begin{align}
S_1(p, \xi) = \int_{-1}^1 \int_0^\infty \Pi(\gamma', \xi'; \gamma, \xi) f(p', \xi') 2 \pi p'^2 dp' \, d\xi',
\end{align}
%
and the annihilation term
\begin{align}
S_2(p, \xi) = -f(p, \xi)  \int_{-1}^1 \int_0^\infty \Pi(\gamma, \xi; \gamma', \xi') 2 \pi p'^2 dp' \, d\xi',
\end{align}
with $\gamma = \sqrt{1+p^2}$ and $\gamma' = \sqrt{1+p'^2}$.

In this work we use the Chiu model in~\cite{chiu1998fokker} and approximate the scattering rate function in $S_1$ by
\begin{align}
\Pi(\gamma', \xi'; \gamma, \xi) \approx \frac 1 {\ln \Lambda} \frac{p'^2}{2\pi p^2 \, |\xi|} \frac{d \sigma (\gamma', \gamma)}{dp} \, \delta(p' - p^*),
\end{align}
where $\delta(\cdot)$ is a Dirac delta function that indicates only electrons with particular energy at $p^*$ can generate the required secondary electron at a given point $(p, \xi)$
with $p^* = \sqrt{(\gamma^*)^2-1}$ and $\gamma^* = \frac{(\gamma+1)/(\gamma-1) \, \xi^2 + 1}{(\gamma+1)/(\gamma-1) \, \xi^2 - 1}$.
The M{\o}ller cross section for large-angle collisions is used,
\begin{align}
 \frac{d \sigma (\gamma', \gamma) }{dp} = \frac{p}{\gamma}\, \frac{2\pi\gamma'^2 }{(\gamma'-1)^3(\gamma'+1)}\left[
 x^2 -3 x + \left( \frac{\gamma'-1}{\gamma'}\right)^2 (1+x) \right],
\end{align}
where
\begin{align*}
x = \frac 1{\nu(1-\nu)},\quad \text{and} \quad \nu = \frac{\gamma-1}{\gamma'-1}.
\end{align*}
Note that the M{\o}ller cross section is turned on only if $\gamma'\ge 2\gamma-1$;  otherwise $S_1$ is set to  0, which is commonly imposed
(see~\cite{mcdevitt2019avalanche}, for instance).
One can easily show that the source term $S_1$ is  positive only in a thin region of $\xi \in [-\sqrt{\gamma/(\gamma+1)}, -p/(\gamma+1)]$.
An adaptive mesh is beneficial to resolve this thin region.
The annihilation term is integrated in the range of $\gamma' \in [\gamma_0, \frac{\gamma+1}2]$,
where $\gamma_0$ is computed from the computation domain, namely, $\gamma_0 = \sqrt{1+p_{\min}^2}$.
The M{\o}ller cross section is well defined in such a range.
This leads to
\begin{align}
S_2(p,\xi) = -\frac 1{\ln\Lambda} f(p, \xi) \, \sigma(\gamma, \gamma_{0}),
\end{align}
where $\sigma(\gamma, \gamma_{0})$ is the integrated form of the M{\o}ller cross section, given by
\begin{align*}
\sigma(\gamma, \gamma_{0}) = \frac{2\pi}{\gamma^2-1}\left[ \frac{\gamma+1}2 -\gamma_{0} - \gamma^2\left(\frac{1}{\gamma-\gamma_{0}} - \frac 1{\gamma_{0}-1} \right) + \frac{2\gamma-1}{\gamma-1}\ln \frac{\gamma_{0}-1}{\gamma-\gamma_{0}} \right]
\end{align*}
for $\gamma\ge2\gamma_0-1$ or otherwise $\sigma(\gamma, \gamma_0) = 0$.
%


%
%
%
%
%
%
%
%
%
%
%
%
%
%
%
%
%
%
%
%
%
%
%
%
%
%
%
%
%

\subsection{Computational domain and boundary conditions}
Since the focus of this study is on relativistic runaway electrons where
the bulk of the distribution is still assumed to be a Maxwellian, it
is reasonable to use a computational domain: $[p_{\min},
p_{\max}] \times [-1, 1]$, where $p_{\min}$ is chosen as being thermal
(e.g., let $p_{\min} = v_t/c$ or a few times of $v_t/c$).  One can
further assume that the left boundary condition at $p = p_{\min}$ is
Dirichlet and given by the Maxwell--J\"uttner distribution, which is
uniform in $\xi$. At the right boundary of $p = p_{\max}$, a Neumann
boundary condition $\partial f/\partial n = 0$ is prescribed, so one
must check the solution to make sure that $p_{\max}$ is high enough that
$f$ is negligibly small there.  At the top and bottom boundaries, because of the involvement of the coefficient $1-\xi^2$, all the fluxes along
$\xi$ become 0, which is in fact the natural outcome of the coordinate
chosen. Therefore, Neumann boundary conditions are used at the top and
bottom boundaries.


\subsection{Conservation and an alternative RFP form}

The determinant of the Jacobian  corresponding to the transformation
between $(p_\parallel, \vec{p}_\perp)$ to $(p, \xi)$, under azimuthal
symmetry,  is given by $J=p^2$ (see \ref{sec:coordinate}).
Thus, the definition of the divergence of a vector $\vv = (v_1, v_2)$ in $(p,\xi)$ becomes
\[
\nabla \cdot \vv \coloneqq
  \frac 1 {p^2} \Big[ \frac{\partial}{\partial p}(p^2 v_1) +
  \frac{\partial}{\partial\xi} (p^2 v_2) \Big].
\]
One can easily show that the electric field ``advection'' term
in~\eqref{eqn:rfp}, $\big(-E \xi, - E ({1-\xi^2})/{p} \big)$, is
divergence-free. This is in fact more obvious in the coordinate of
$(p_\parallel, \mathbf{p}_\perp)$, as the RFP equation in
$(p_\parallel, \mathbf{p}_\perp)$ becomes
\[
\frac{\partial f}{\partial t} - E \frac{\partial f}{\partial p_\parallel}= C(f) + \alpha R(f) + S(f)
\]
and $E$ is a constant. It is also easy to see that $f p^2$ is a conserved value in the coordinate of $(p, \xi)$, which is consistent with the fact that the distribution function is conservative in $(p_\parallel, \mathbf{p}_\perp)$. As a result, a finite volume method is chosen in this study to satisfy the conservation of the distribution.

In our implementation, we choose to solve $\tilde f \coloneqq fp$. Then we can rewrite the RFP equation~\eqref{eqn:rfp} as
\begin{align}
\label{eqn:conser2}
  \partd{\tilde f}{t} +
  \frac{1}{p}\partd{\Gamma_p(\tilde f)}{p} +
  \partd{\Gamma_\xi(\tilde f)}{\xi}
  =   S(\tilde f),
\end{align}
where the fluxes are
\begin{align}
\label{eqn:conser2-flux-p}
  \Gamma_p(\tilde f) &=
    -p \left[ E\xi +\alpha p \gamma (1-\xi^2) \right] \tilde f -
    (p C_F - C_A) \tilde f - p C_A \partd{\tilde f}{p},\\
\label{eq:conser2-flux-xi}
  \Gamma_\xi(\tilde f) &=
    -(1-\xi^2) \left[ \left(\frac E p - \frac{\alpha  \xi}{\gamma} \right) \tilde f +
                      \frac{C_B}{p^2}\partd{\tilde f }{\xi} \right].
\end{align}
This conservative form is a choice of our implementation. In  previous work~\cite{GuoMcDevittTang2017}, it is shown that solving for $\tilde f$ helps diminish the negativity of the distribution function in the tail region.



%

%
%
%

\def\secstatus{finished}

\section{Adaptive mesh refinement: Indicators and their prediction in time}
\label{sec:amr}

%
%
\editAll{Indicators are flags that correspond to each grid point and determine}
{Indicators are spatial fields that determine at each grid point}
whether to refine or coarsen the mesh.
This section presents the indicators for adaptivity and shows how indicators are
propagated forward in time ahead of the simulation in order to predict the
%
 regions of the mesh that need to be refined and derefined.
We consider an abstract setting; hence the discussion is general and applicable
beyond the RFP equation, which is otherwise the focus of this work.
First, we present how the indicators are computed, and then we propose an approach for
the prediction of indicators.

%
%

%

\subsection{AMR indicators for coarsening and refinement}
\label{sec:amr-indicators}

%
We define an indicator as a functional, $\indfn$, that maps a function, say
$f$, to a single positive scalar value for each mesh cell $\Omega_h$
within the entire domain $\Omega$.  Hence we write
\[
  \indfn(f;\Omega_h)\in\R_+ \quad\text{for each }\Omega_h\subset\Omega.
\]
In the following, the dependence on $\Omega_h$ is assumed implicitly such
that we can use the brief notation $\indfn(f) = \indfn(f;\Omega_h)$.
The result is then used to evaluate the criteria for AMR.  A mesh cell is coarsened if the indicator is below a threshold
\[
  \indfn(f) < \indmin,
\]
and it is refined if the indicator is above a threshold
\[
  \indmax < \indfn(f),
\]
where $0\le\indmin<\indmax$ are (prescribed) constants that stay fixed during
the simulation.  These criteria for coarsening and refinement are carried
out uniformly for every cell of the mesh.  This process yields the requirement
that $\indfn(f)$ has to produce values that are independent of any scale,
because the function $f$ is likely to vary significantly.  Indeed, large
variations of $f$ are the reason to utilize AMR in the first place.

%
To proceed to defining AMR indicators, we introduce the following
preliminary definitions and observations.
Let $f:\Omega\rightarrow\R_+$ be a positive\footnote{%
  To simplify the
  notation, we assume the function $f$ to have positive values. One could, however, remove the assumption on positivity and, in place
  of $f$, use $\abs{f}+\varepsilon$ with a chosen $0 < \varepsilon \ll 1$.
} and continuously differentiable
function, $f\in C^1(\Omega)$, over an open domain $\Omega\in\R^d$.  We define
the gradient scale of a function $f>0$ as the nondimensionalized quantity
$(\gradient f)/f$, and as a result the scale of $f$ is neutralized.
It satisfies the \emph{gradient-scale identity}
\begin{equation}
\label{eq:gs-identity}
  \frac{\gradient f(x)}{f(x)} = \gradient\left(\log f(x)\right),
  \quad\text{for }x\in\Omega.
\end{equation}
%
%
%
%
Since we are concerned with meshes of a finite resolution, we denote
$\Omega_h\subset\Omega$ to be a mesh cell with a \emph{characteristic
size} $h>0$, and we let $\overline{\Omega}_h$ be its closure.  Then we define a
\emph{discrete local gradient magnitude}, which is motivated by finite
difference gradients,
\begin{equation}
\label{eq:discrete-grad}
  G_h[f](x) \coloneqq
    \max_{y\in\overline{\Omega}_h} \frac{\abs{f(x)-f(y)}}{h},
  \quad\text{for }x\in\Omega_h,
\end{equation}
where $\abs{\cdot}$ denotes the absolute value.  Note that for
\eqref{eq:discrete-grad} to be well defined, it is only required
for $f$ to be Lipschitz continuous.

We define two AMR indicators based on the gradient scale (GS) in two different
ways, therefore making use of both sides of the identity \eqref{eq:gs-identity}.

\begin{definition}[Gradient-scale indicator]
\label{def:gs-ind}
Let $f>0$ be Lipschitz continuous, and let $\Omega_h\subset\Omega$ be a
cell of the mesh.
The first version of the GS indicator is
%
\begin{equation}
\label{eq:gs}
  \indfn_{GS}(f) \coloneqq
    h \max_{x\in\overline{\Omega}_h} \left( \frac{G_h[f](x)}{f(x)} \right).
\end{equation}
The second version is
%
\begin{equation}
\label{eq:lgs}
  \indfn_{LGS}(f) \coloneqq
    h \max_{x\in\overline{\Omega}_h} \Bigl( G_h[\log f](x) \Bigr).
\end{equation}
\end{definition}

The indicators in Definition~\ref{def:gs-ind} are derived from a discrete
version of the gradient-scale identity with an additional multiplication by the
characteristic size $h$ of the cell.  The multiplication by $h$ reduces
the value of the indicator as the mesh is refined, and it also neutralizes
division by $h$ appearing in the discrete gradient \eqref{eq:discrete-grad}.
Hence, the indicator becomes nondimensionalized regarding the cell
size.  Overall, because we eliminated the scale of the function $f$ and the
scale of the discrete gradient $h$, the resulting indicators $\indfn_{GS}$ and
$\indfn_{LGS}$ are nondimensional.

An alternative indicator based on the dynamic ratio (i.e., maximum value
divided by minimum value of $f$ in $\overline{\Omega}_h$) is defined next.
Subsequently, a relation between the two definitions of indicators is
established in Proposition~\ref{prop:lgs-vs-ldr}.
%

\begin{definition}[Log-DR indicator]
\label{def:ldr-ind}
We define an AMR indicator based on the logarithm of the dynamic ratio (DR)
within a cell $\Omega_h$
\begin{equation}
\label{eq:ldr}
  \indfn_{LDR}(f) \coloneqq
    \log\left( \frac{\max_{x\in\overline{\Omega}_h} f(x)}
                    {\min_{x\in\overline{\Omega}_h} f(x)} \right).
\end{equation}
\end{definition}

\begin{proposition}[Equivalence between gradient scale and log-DR indicators]
\label{prop:lgs-vs-ldr}
Given a Lipschitz continuous function $f>0$, let $\indfn_{LGS}$ be the gradient-scale AMR indicator defined in \eqref{eq:lgs} and $\indfn_{LDR}$ be the log-DR
indicator defined in \eqref{eq:ldr}.  Then the two indicators satisfy
\begin{equation}
\label{eq:lgs-vs-ldr}
  \indfn_{LGS}(f) = \indfn_{LDR}(f).
\end{equation}
\end{proposition}
\begin{proof}
Starting from the definition \eqref{eq:lgs}, we obtain
\begin{align*}
  \indfn_{LGS}(f)
  &=
    h \max_{x\in\overline{\Omega}_h} \Bigl( G_h[\log f](x) \Bigr)
  \\&=
    h \max_{x,y\in\overline{\Omega}_h}
      \left( \frac{\abs{\log f(x) - \log f(y)}}{h} \right)
  \\&=
    \max_{x,y\in\overline{\Omega}_h} \abs{\log\left(\frac{f(x)}{f(y)}\right)}.
\end{align*}
Utilizing the monotonicity of the logarithm and $f>0$, we continue
\begin{align*}
  \max_{x,y\in\overline{\Omega}_h} \abs{\log\left(\frac{f(x)}{f(y)}\right)}
  &=
    \max\left\{
      \max_{x,y\in\overline{\Omega}_h}  \log \left(\frac{f(x)}{f(y)}\right),
      -\min_{x,y\in\overline{\Omega}_h}  \log \left(\frac{f(x)}{f(y)}\right)
    \right\}
  \\&=
    \max\left\{
       \log \left(\max_{x,y\in\overline{\Omega}_h} \frac{f(x)}{f(y)}\right),
      -\log \left(\min_{x,y\in\overline{\Omega}_h} \frac{f(x)}{f(y)}\right)
    \right\}
  \\&=
    \max\left\{
       \log \left( \frac{\max_{x\in\overline{\Omega}_h} f(x)}
                        {\min_{y\in\overline{\Omega}_h} f(y)} \right),
      -\log \left( \frac{\min_{x\in\overline{\Omega}_h} f(x)}
                        {\max_{y\in\overline{\Omega}_h} f(y)} \right)
    \right\}.
\end{align*}
Additionally, the second argument in the outermost maximum is
\[
  -\log \left( \frac{\min_{x\in\overline{\Omega}_h} f(x)}
                    {\max_{y\in\overline{\Omega}_h} f(y)} \right)
  =
  \log \left( \frac{\min_{x\in\overline{\Omega}_h} f(x)}
                   {\max_{y\in\overline{\Omega}_h} f(y)} \right)^{-1}.
\]
Therefore we can simplify
\[
  \max_{x,y\in\overline{\Omega}_h} \abs{\log\left(\frac{f(x)}{f(y)}\right)}
  =
  \log \left( \frac{\max_{x\in\overline{\Omega}_h} f(x)}
                   {\min_{y\in\overline{\Omega}_h} f(y)} \right),
\]
which shows the claim of the proposition.
\end{proof}

\begin{remark}
The indicator $\indfn_{LGS}$ can be seen in relation to the Lipschitz constant.
Let us assume that $K_h$ is a discrete version of the Lipschitz constant such
that
  $\abs{\log f(x) - \log f(y)} \le K_h \abs{x - y}$
for all $x,y\in\overline{\Omega}_h$ with $h\le\abs{x-y}$.
Then the following estimate holds:
\[
    \max_{\substack{x,y\in\overline{\Omega}_h \\ h\le\abs{x-y}}}
    \frac{\abs{\log f(x) - \log f(y)}}{\abs{x - y}}
  \le
    \max_{x,y\in\overline{\Omega}_h}
    \frac{\abs{\log f(x) - \log f(y)}}{h}
  =
    \frac{\indfn_{LGS}(f)}{h}.
\]
Thus the indicator computes an upper bound on the discrete Lipschitz constant scaled by the characteristic cell size:
  $h K_h \le \indfn_{LGS}(f)$.
\end{remark}

\begin{remark}
In order to compute the indicators \eqref{eq:gs}, \eqref{eq:lgs}, and
\eqref{eq:ldr}, a minimum and/or maximum needs to be computed over a
mesh cell.  In practice this can be done by looping over the nodes or
degrees of freedom of the particular discretization.  Furthermore, it is
sufficient to approximate the discrete local gradient magnitude
\eqref{eq:discrete-grad} by utilizing the routines for gradient computation
that are native to the discretization.

For computing the log-DR indicator in \eqref{eq:ldr} from a discretized $f$
that has been polluted by errors due to finite precision arithmetic or solver
truncations, which are unavoidable in practice, we recommend  computing
$\indfn_{LDR}(f+\epsilon)$ with a small constant $\epsilon>0$.  This ensures
numerically that the argument of $\indfn_{LDR}$ is positive.
\end{remark}

\editTwo{}{
Proposition~\ref{prop:lgs-vs-ldr} shows that the gradient-scale and the log-DR
indicators can be used interchangeably.  Specifically, the simple-to-implement
and computationally cheaper log-DR indicator quantifies features in the solution
equally well as the (slightly) more complex gradient-scale indicator.
To decide which formulation of indicator to use depends on a particular
discretization method and the quantities that are computed in the
implementation.  If, for example, a gradient already needs to be computed, then
the gradient-scale indicator is a cheap byproduct of this computation.  For the
finite difference-based discretization that we consider in this work, the log-DR
indicator presents the computationally cheapest option.  This is why we will
utilize log-DR for our adaptivity criterion in the numerical experiments that
follow in Section~\ref{sec:results}.
}

%

\subsection{Dynamic AMR and indicator prediction in time}
\label{sec:amr-pred}

This section is concerned with adapting a mesh dynamically to a function
$f=f(t)$ that is evolving in time.
Dynamic mesh adaptivity is a requirement that is imposed by the PDEs we aim to
solve (see Section~\ref{sec:governing-eqns}).  These PDEs are
advection-dominated in some parts of the domain and diffusion-dominated in
other parts.

%
The mesh can be adapted in between time steps of the implicit time integration
scheme.  To track the variations of the solution as accurately as possible, one
would ideally adapt the mesh in between every time step.  However, this will
increase the computational cost of a parallel implicit solver, because adapting
the mesh induces additional computational costs, such as
redistributing the mesh cells across compute cores, interpolation
between coarse and fine cells, and reallocating and setting up of
solver components.  If the mesh remains unchanged for too many time steps, on
the other hand, accuracy may suffer, and fine-scale features can become
insufficiently resolved because they will migrate from fine mesh cells to
coarse ones.  Therefore dynamic AMR faces two competing constraints:
(i) how long can the time intervals be between adapting the mesh in order to
adequately track the quantity of interest and
(ii) how large is the additional computational cost that each change of the
mesh induces.

%
The goal in this section is to introduce a new technique to balance these two
constraints.  We want to reduce the frequency of changing the
mesh and at the same time provide sufficient resolution where it is needed
(within a certain time interval).  We also want to avoid overly fine meshes
where resolution is not needed.  It therefore is desired to ``predict''  AMR
indicators in a time evolution simulation.  We will show that such a prediction
is possible in an advection-dominated setting.
%

We denote $\Delta t$ as the length of the time step of the time evolution
scheme.  We define a number $n_\mathrm{adapt}\in\N_+$ (i.e., positive integer)
that determines after how many time steps $\Delta t$ adaptivity is triggered;
therefore the mesh remains fixed during the interval
  $\Delta t_\mathrm{adapt} \coloneqq n_\mathrm{adapt} \Delta t$.
This implies that within the $n_\mathrm{adapt}$ time steps the features of a
function $f(t)$, which we want to resolve,
%
will migrate.  This is what we want to address with AMR prediction.
The superscript notation $f^{(A)}(t)$ indicates that $f(t)$ is discretized (in
space) on a particular mesh $A$.

The diagram in Figure~\ref{fig:diagram-amr} illustrates how dynamic AMR is
carried out between time steps without AMR prediction.
For simplicity of the diagram, we set $n_\mathrm{adapt}=1$, and hence
  $\Delta t_\mathrm{adapt} = \Delta t$.
The diagram shows the sequence of mesh adaptivity, where interpolation from
mesh $A$ to mesh $B$ is performed, alternating with the time evolution scheme,
where $f^{(B)}(t)$ evolves to $f^{(B)}(t + \Delta t)$.

With AMR prediction, we aim to avoid a loss of accuracy when fine-scale
features of $f(t)$ migrate away from fine to coarser regions of the mesh.
We propose to predict the ``path of refinement'' by evolving the AMR indicators
``ahead'' of the quantity of interest, $f(t)$, by intervals of at most
$\Delta t_\mathrm{adapt}$.  In effect, we are leapfrogging the AMR
indicators forward in time relative to the (regular) simulation.
%
Dynamic AMR with prediction is illustrated in the diagram in
Figure~\ref{fig:diagram-amr-prediction}.  For simplicity of the diagram, we set
$n_\mathrm{adapt}=n_\mathrm{pred}=1$, and hence
  $\Delta t_\mathrm{adapt} = \Delta t_\mathrm{pred} = \Delta t$.
First, the indicator $\indfn$ is computed as in the previous case without
prediction; however, $\indfn$ is subsequently evolved forward in time by
$\Delta t_\textrm{pred}$ ahead of the simulation.  To keep the
computational costs low, we propose to use an explicit time integration scheme,
which should be significantly faster to carry out compared with the implicit
scheme used to evolve $f$.  As $\indfn$ evolves, it is simultaneously reduced
into a time-independent indicator $\indfn_\textrm{pred}$.  To adapt
the mesh, we then use
$\indfn_\textrm{pred}$, which predicts the path of adaptivity.

\begin{figure}\centering
  \resizebox{0.98\columnwidth}{!}{
    \tikzset{
  fncSty/.style={font=\footnotesize, align=center},
  meshSty/.style={font=\footnotesize, align=center, draw=darkgray, rounded corners=0.5ex},
  arrowSty/.style={->, >=latex, draw=darkgray, thick},
  arrowLabelSty/.style={font=\scriptsize, align=center},
}
\begin{tikzpicture}
  \def\dh{3.0}
  \def\dv{0.9}

  \node[fncSty]   (fA1) at (-0.5*\dh,-0.0*\dv) {$f^{(A)}(t)$};
  \node[meshSty]  (mA)  at (-0.7*\dh,-1.1*\dv) {mesh $A$};
  \draw[arrowSty] (mA) -- (fA1);

  \node[fncSty]   (iA1) at (0*\dh,-2.6*\dv) {$\indfn^{(A)}(t)$};
  \node[meshSty]  (mB)  at (0.3*\dh,-1.1*\dv) {mesh $B$};
  \draw[arrowSty] (fA1) -- (iA1);
  \draw[arrowSty] (iA1) -- (mB);

  \node[fncSty]   (fB0) at (0.5*\dh,-0*\dv) {$f^{(B)}(t)$};
  \draw[arrowSty] (fA1) -- node[arrowLabelSty,above]{interp.} (fB0);
  \draw[arrowSty] (mB)  -- (fB0);

  \node[fncSty]   (fB1) at (1.5*\dh,-0*\dv) {$f^{(B)}(t + \Delta t)$};
  \draw[arrowSty] (fB0) -- node[arrowLabelSty,above]{evolve} (fB1);

  \node[fncSty]   (iB1) at (2*\dh,-2.6*\dv) {$\indfn^{(B)}(t + \Delta t)$};
  \node[meshSty]  (mC)  at (2.3*\dh,-1.1*\dv) {mesh $C$};
  \draw[arrowSty] (fB1) -- (iB1);
  \draw[arrowSty] (iB1) -- (mC);

  \node[fncSty]   (fC0) at (2.5*\dh,-0*\dv) {$f^{(C)}(t + \Delta t)$};
  \draw[arrowSty] (fB1) -- node[arrowLabelSty,above]{interp.} (fC0);
  \draw[arrowSty] (mC)  -- (fC0);

  \node[fncSty]   (fC1) at (3.5*\dh,-0*\dv) {$f^{(C)}(t + 2\Delta t)$};
  \draw[arrowSty] (fC0) -- node[arrowLabelSty,above]{evolve} (fC1);
\end{tikzpicture}

  }
  \caption{Diagram of \textbf{dynamic AMR without prediction}.
    In the ``interp.'' (interpolation) step, the indicator $\indfn$ is computed
    from $f$, and this indicator is used to adapt the mesh (e.g., from mesh
    $A$ to mesh $B$).
    %
    The ``evolve'' step advances $f$ forward in time by $\Delta
    t_\mathrm{adapt} = \Delta t$.}
  \label{fig:diagram-amr}
  %
  \vskip 8ex
  %
  \resizebox{0.98\columnwidth}{!}{
    \tikzset{
  fncSty/.style={font=\footnotesize, align=center},
  meshSty/.style={font=\footnotesize, align=center, draw=darkgray, rounded corners=0.5ex},
  arrowSty/.style={->, >=latex, draw=darkgray, thick},
  arrowLabelSty/.style={font=\scriptsize, align=center},
}
\begin{tikzpicture}
  \def\dh{3.0}
  \def\dv{0.9}

  \node[fncSty]   (fA1) at (-0.5*\dh,-0.0*\dv) {$f^{(A)}(t)$};
  \node[meshSty]  (mA)  at (-0.7*\dh,-1.1*\dv) {mesh $A$};
  \draw[arrowSty] (mA) -- (fA1);

  \node[fncSty]             (iA1) at (0*\dh,-3*\dv) {$\indfn^{(A)}(t)$};
  \node[fncSty,anchor=west] (iA2) at (1*\dh,-3*\dv) {$\indfn^{(A)}(t + \Delta t)$};
  \draw[arrowSty] (iA1) -- node[arrowLabelSty,below]{evolve (expl.)} (iA2);
  \draw[arrowSty] (fA1) -- (iA1);

  \draw[decoration={brace},decorate] (iA1.north) -- (iA2.north east);
  \node[fncSty]   (pA2) at (0.9*\dh,-2.2*\dv) {$\indfn^{(A)}_\mathrm{pred}(t)$};
  \node[meshSty]  (mB)  at (0.7*\dh,-1.1*\dv) {mesh $B$};
  \draw[arrowSty] (pA2) -- (mB);

  \node[fncSty]   (fB0) at (0.5*\dh,-0*\dv) {$f^{(B)}(t)$};
  \draw[arrowSty] (fA1) -- node[arrowLabelSty,above]{interp.} (fB0);
  \draw[arrowSty] (mB)  -- (fB0);

  \node[fncSty]   (fB1) at (1.5*\dh,-0*\dv) {$f^{(B)}(t + \Delta t)$};
  \draw[arrowSty] (fB0) -- node[arrowLabelSty,above]{evolve}
                           node[arrowLabelSty,below]{(implicit)} (fB1);

  \node[fncSty]             (iB1) at (2*\dh,-3*\dv) {$\indfn^{(B)}(t + \Delta t)$};
  \node[fncSty,anchor=west] (iB2) at (3*\dh,-3*\dv) {$\indfn^{(B)}(t + 2\Delta t)$};
  \draw[arrowSty] (iB1) -- node[arrowLabelSty,below]{evolve (expl.)} (iB2);
  \draw[arrowSty] (fB1) -- (iB1);

  \draw[decoration={brace},decorate] (iB1.north) -- (iB2.north east);
  \node[fncSty]   (pB2) at (2.9*\dh,-2.2*\dv) {$\indfn^{(B)}_\mathrm{pred}(t + \Delta t)$};
  \node[meshSty]  (mC)  at (2.7*\dh,-1.1*\dv) {mesh $C$};
  \draw[arrowSty] (pB2) -- (mC);

  \node[fncSty]   (fC0) at (2.5*\dh,-0*\dv) {$f^{(C)}(t + \Delta t)$};
  \draw[arrowSty] (fB1) -- node[arrowLabelSty,above]{interp.} (fC0);
  \draw[arrowSty] (mC)  -- (fC0);

  \node[fncSty]   (fC1) at (3.5*\dh,-0*\dv) {$f^{(C)}(t + 2\Delta t)$};
  \draw[arrowSty] (fC0) -- node[arrowLabelSty,above]{evolve}
                           node[arrowLabelSty,below]{(implicit)} (fC1);
\end{tikzpicture}

  }
  \caption{Diagram of \textbf{dynamic AMR with prediction}.
    The indicator $\indfn$ is first computed from $f$ and subsequently evolved
    forward in time by $\Delta t_\textrm{pred} = \Delta t$ ahead of the
    simulation, thus predicting the path of AMR.  The evolution of
    $\indfn$ is reduced into one time-independent indicator
    $\indfn_\textrm{pred}$.  In the ``interp.'' (interpolation) step,
    $\indfn_\textrm{pred}$ is used to adapt the mesh (e.g., from mesh $A$ to
    mesh $B$).
    %
    The ``evolve'' step advances $f$ forward in time by $\Delta
    t_\mathrm{adapt} = \Delta t$.}
  \label{fig:diagram-amr-prediction}
\end{figure}

%
%
%
%
%
%
%
%

Algorithm~\ref{alg:amr-prediction} describes in more detail how dynamic AMR
with prediction is carried out during the time evolution of a function $f$.
Specifically, the reduction of the leapfrogged indicator $\indfn$ is performed
by taking the maximum of the evolving indicator across all time steps.

\begin{algorithm}
\caption{AMR prediction by leapfrogging AMR indicator.}
\label{alg:amr-prediction}
\begin{algorithmic}[1]
  \Statex \textbf{Given:} Mesh~$A$ and function $f^{(A)}(t)$ supported on
    mesh~$A$ at time $t$;
    time step length $\Delta t>0$,
    number of steps $n_\mathrm{adapt}\in\N_{+}$ for adaptivity intervals, and
    number of steps $n_\mathrm{pred}\in\N_{+}$ for prediction intervals.
  %
  \While {$t$ has not reached final time}
    \State $\Delta t_\mathrm{pred} = n_\mathrm{pred} \Delta t$
    \State $\Delta t_\mathrm{adapt} = n_\mathrm{adapt} \Delta t$
    %
    \State compute indicator $\indfn^{(A)}(t)$ from function $f^{(A)}(t)$ on mesh $A$
    \State leapfrog $\indfn^{(A)}(t)$ until $\indfn^{(A)}(t + \Delta t_\mathrm{pred})$
      \Comment \emph{(``fast'' scheme)}
    \State reduce $\indfn^{(A)}$ to predictive indicator
      $\indfn^{(A)}_\mathrm{pred}(t) =
        \max_{\tau\in[t, t+\Delta t_\mathrm{pred}]} \indfn^{(A)}(\tau)$
    \State create mesh $B$ by adapting mesh $A$ based on $\indfn^{(A)}_\mathrm{pred}(t)$
      \Comment \emph{(AMR)}
    \State interpolate $f^{(A)}(t)$ to $f^{(B)}(t)$ onto mesh $B$
    \State evolve $f^{(B)}(t)$ until $f^{(B)}(t + \Delta t_\mathrm{adapt})$
      \Comment \emph{(``accurate'' scheme)}
    \State update time step length $\Delta t$
      \Comment \emph{(adaptive time stepping)}
  \EndWhile
\end{algorithmic}

\end{algorithm}

In the general setting, as it has been considered in this section, we have not
made assumptions about the time-stepping methods used to evolve $f(t)$ and to
leap-frog $\indfn$.  The choice of the pairing of these two time-stepping
methods is important when considering the computational complexity of the
overall simulation with dynamic AMR and AMR prediction.
In the context of the governing equations we are aiming to solve
(Section~\ref{sec:governing-eqns}), we are dealing with an advection--diffusion PDE.  The PDE
will require implicit time-stepping schemes, which have a significantly higher
computational complexity when compared with explicit schemes.
The leapfrogging of the indicators $\indfn$, on the other hand, has to be
performed only locally in time: for short time intervals and with a new initial
condition at each invocation of AMR prediction.
Therefore we can relax the requirement for accuracy, and we can approximate
local short-time behavior by eliminating the diffusion term of the PDE.  Hence,
the equation for AMR prediction will be an advection-only PDE, and we will use
the same advection coefficient as for the governing equations of the physical
models (see Section~\ref{sec:governing-eqns})

In
Section~\ref{sec:pred0-vs-pred1} we discuss the increase in accuracy of the numerical solution while keeping the
computational costs low, which is achieved due to AMR prediction.



%
%
%

\def\secstatus{finished}

\section{Relavitistic electron drift-kinetic solver based on PETSc and p4est}
\label{sec:solver}

DMBF (Data Management for Block-structured Forest-of-trees) is a new data
management object in PETSc that we have developed and implemented along with a
new relativistic electron drift-kinetic solver.
We first outline the capabilities of the implementation that are incorporated
into PETSc;  subsequently, we give a high-level description of the new
application code for relativistic electrons.

%---------------------------------------

\subsection{DMBF: New PETSc data management for p4est-based AMR solvers}
\label{sec:dmbf}

Our goal is to discretize the computational domain with locally adaptively
refined quadrilateral meshes in two dimensions.  Extreme local refinement is
critical for resolving the rapidly changing solution, which can vary 15--20
orders of magnitude, while over regions where the solution is relatively flat,
coarser meshes can help reduce the computational cost.  Therefore, a key
requirement of our solver is adaptive mesh refinement.

%%% P4EST
The dynamic mesh adaptivity in parallel is, at its lowest level, enabled by the
p4est library \cite{BursteddeWilcoxGhattas11, IsaacBursteddeWilcoxEtAl15}.
This library implements hierarchically refined quadrilateral and hexahedral meshes
utilizing forest-of-octree algorithms and space-filling curves
\cite{sundar2008bottom}.
Mesh refinement and coarsening are efficient in parallel because p4est performs
these tasks locally on each compute core without communication.  The
representation of the mesh as a quadtree/octree topology enables efficient 2:1
mesh balancing in parallel as well as repartitioning of the mesh cells
across compute cores, for both of which communication is necessary.
Space-filling curves transform a two- or three-dimensional space that is subdivided in cells
into a (one-dimensional) sequence of cells.  This sequence is used for an efficient
partitioning of mesh cells in parallel with the desirable property of
keeping spatially neighboring cells nearby each other in the sequence;
hence exhibiting memory locality.
This property of space-filling curves is responsible for keeping the amount of
communication low.
Overall, the algorithms of p4est have demonstrated scalability up to $O(100,000)$
of processes on distributed-memory CPU-based systems
\cite{BursteddeGhattasGurnisEtAl10, SundarBirosBursteddeEtAl12}; moreover,
scalability to $O(1,000,000)$ of CPU cores have been achived for complex
implicit PDE solvers \cite{rudi2015extreme, RudiStadlerGhattas17,
RudiShihStadler20}.
Generally, tree-based AMR algorithms have shown impressive scalability
results \cite{weinzierl2019peano, FernandoNeilsenLimEtAl19}.

%%% DMBF
DMBF is a new type of data management in PETSc.
%it is thus a particular kind of DM in PETSc.
It provides interfaces for block-structured forest-of-trees meshes.
The block structure is understood in the sense that leaf elements of the trees
can be equipped with blocks of uniformly refined cells.  DMBF implements
PETSc functions that give direct access to the functionality of p4est with the
least amount of overhead compared with existing PETSc data management approaches.
In addition, DMBF provides cellwise data management for storing any kind
of data that is mapped to the cells of the p4est mesh.
Because DMBF associates cellwise data directly with the p4est mesh, the cell
data managed by DMBF is created and destroyed when p4est cells are refined
and coarsened; and DMBF-managed cell data migrates across processes in parallel
as the p4est mesh is repartitioned.  In the context of discretizations of PDEs,
it is essential that data from neighboring cells be shared, which in a
distributed-memory setting is handled via point-to-point communication between
processes in order to communicate data of a ghost layer.  DMBF supports the
communication of DMBF-managed cell data within a ghost layer.

The focus of DMBF in managing general data objects enables scientific
applications that use the DMBF interface to store any kind of cell-dependent
data, for instance, the data required for finite element, finite volume, or
finite difference discretizations.  The shape and size of the cell data are left
entirely up to the application using DMBF and are not imposed by DMBF's
interface.  This approach allows for greater flexibility, because more
applications can utilize DMBF.  At the same time, it requires the
application to implement its own discretization routines (or rely on other
libraries).
We have taken this approach of separating concerns.  We have implemented a new
relativistic electron drift-kinetic solver that is supported by the DMBF
interfaces (see Section~\ref{ref:rfp-solver}).

To perform computations with the DMBF-managed cell data, the DMBF
interface provides iterator functions that call a user-specified function for
each cell.  Additional functions are capable of iterating over the edges/faces
of cells.  The adaptation of a mesh is handled with similar user-provided
functions that act on the data of each cell individually.
%
The DMBF interface is completed by allowing the generation of PETSc vectors and
matrices. Therefore, calculations that utilize DMBF leverage the linear and
nonlinear solvers of PETSc as well as PETSc's time-stepping functionality.

%---------------------------------------

\subsection{Relativistic electron drift-kinetic solver based on DMBF}
\label{ref:rfp-solver}

This section gives a brief overview of the relativistic drift-kinetic Fokker--Planck solver
that is built on the DMBF data structure.

The spatial discretization chooses a finite volume scheme, MUSCL (Monotonic
Upstream-centered Scheme for Conservation Law) \cite{LeVeque02, Toro09},
or a conservative finite difference scheme, QUICK (Quadratic Upstream
Interpolation for Convective Kinematics) \cite{Leonard79}, for the advection
operators.  Both schemes are second-order accurate in space
on a uniform mesh for time-dependent advection problems.
The collision operator is discretized by a central scheme using a three-point stencil that
also gives second-order accuracy. An adaptive quadtree-based
quadrilateral mesh is used. Thus our solver supports the handling of hanging nodes and can be nonsymmetric in a single cell.
More details on the AMR and the implementation can be found
in Section~\ref{sec:dmbf}.

Numerical solutions are always assumed to live on cell centers.  Each cell of
the adapted mesh is further subdiveded into four uniform cells, which
constitute the degrees of freedom (DOFs) of a solution's discretization.
To distinguish between the two notions of cells, we use the term \emph{mesh cells} for the
cells of the adapted p4est-based mesh and the term \emph{FV cells} for the four uniform
cells inside a mesh cell.
%
Additional to the FV cells inside each mesh cell, the numerical schemes need to
access DOFs from neighboring FV cells.  To this end, the DMBF cell data includes
a layer, also referred to as guard layer, of two FV cells around a mesh cell.
This outer layer of DOFs can require point-to-point communication to be filled
with data from other processes.
%
The spatial discretization is coupled with explicit or implicit time stepping.
To ease the mesh adaptivity, we choose single-step integrators.
For the explicit case a SSP Runge--Kutta is chosen, while for the implicit case
a fully implicit DIRK or ESDIRK time integrator is chosen.
Here the explicit integrators are chosen only for testing purposes.
All the production runs use implicit integrators because we are interested in
physics of the collisional time scale.

%- quadrilateral mesh cells\\
%- cell-centered location of degrees of freedom\\
%- faces around a cell-centered node can be non-symmetric, because of mesh adaptivity\\
%- finite volume scheme with piecewise linear reconstruction\\
%- pw lin local reconstruction of q\\
%- total variation diminishing (TVD) [in time] property\\
%- slope limiting with MINMOD\\
%- for advection part of the PDE, this gives up to second-order accuracy\\
%- diffusion flux is handled with a three-point finite difference stencil

%solver
Depending on the choice of the advection scheme, the problem can be nonlinear
(in the case of MUSCL) or linear (in the case of QUICK).  For nonlinear problems, the
JFNK algorithm is the primary solver, and it is further preconditioned with an
approximate Jacobian.  The Jacobian is computed through finite difference
coloring provided by PETSc.  A GMRES iterative solver is used to invert the
linearized system (and also the linear system arising from QUICK).
A hypre \cite{FalgoutYang02} algebraic multigirid (BoomerAMG
\cite{yang2002boomeramg}) solver is used as the preconditioner.  Those choices
of the algorithm are state of the art for large-scale nonlinear simulations.

Two commonly used operators on adaptive meshes are interpolations and
integrations, both of which need to support handling of hanging nodes/edges.  For
interpolation from coarse to fine and reverse, we use tensor product
interpolation matrices because of their lower memory usage per generated Flops.
When needed, a postprocessing step is performed after interpolation to
preserve the positivity and mass conservation of the algorithm.
Computations of fluxes across hanging edges require tailored algorithms for
filling the guard cells; these are derived from the interpolation algorithms
and inherit their properties (tensor product formulation, positivity, and mass
conservation).
%
A line integral along $\xi$ direction is performed to compute runaway
electrons.  A simple quadrature rule is used in each cell, and the positivity
can be enforced at the price of reducing the spatial accuracy to
first-order.

%\noindent
%for interpolation from coarse to fine and reverse, we use:\\
%- tensor product interpolation matrices (low memory, high Flops)\\
%- postprocessing to enforce positivity and mass conservation\\
%-computations for fluxes across haning edges are derived from interpolation
% and inherit their properties (tensor product, positivity, mass conservation)\\



%
%
%

%% Macro setup
\definecolor{purple}{rgb}{1, 0, 1}

\newcommand{\ie}{\emph{i.e.,}\xspace}
\newcommand{\eg}{\emph{e.g.,}\xspace}
\newcommand{\abr}{\emph{abbr.}\xspace}
\newcommand{\ea}{\emph{et al.}\xspace}
\newcommand{\gensync}{\emph{GenSync}\xspace}
\newcommand{\colosseum}{\emph{Colosseum}\xspace}
\newcommand{\srep}{\emph{SREP}\xspace} % Set Reconciliation Enhances
\newcommand{\srepsim}{\emph{SREPSim}\xspace}
% Propagation
\newcommand{\esrep}{\emph{E-SREP}\xspace}
\newcommand{\epsrep}{\emph{EP-SREP}\xspace}
\newcommand{\mesrep}{\emph{ME-SREP}\xspace}
\newcommand{\mempoolsync}{\emph{MempoolSync}}

\newcommand{\fref}[1]{Fig.~\ref{#1}}
\newcommand{\tref}[1]{Table~\ref{#1}}
\newcommand{\aref}[1]{Algorithm~\ref{#1}}
\newcommand{\procref}[1]{Procedure~\ref{#1}}
\newcommand{\sref}[1]{Section~\ref{#1}}
\newcommand{\lineref}[1]{line~\ref{#1}}
\newcommand{\appref}[1]{Appendix~\ref{#1}}

% Change \eqref
\LetLtxMacro{\originaleqref}{\eqref}
\renewcommand{\eqref}{Eq.~\originaleqref}

% Theorems and corollaries
\newcounter{theoremcount}
\setcounter{theoremcount}{0}
\DeclareRobustCommand{\theorem}[1]{%
  \refstepcounter{theoremcount}%
  \noindent\textit{\textbf{Theorem \thetheoremcount\label{theorem:#1}: }}%
}
\DeclareRobustCommand{\theoremref}[1]{Theorem~\ref{theorem:#1}}

\DeclareRobustCommand{\proof}{\emph{Proof:}\xspace}
\DeclareRobustCommand{\qqed}{\hfill$\blacksquare$}

\newcounter{corollcount}
\setcounter{corollcount}{0}
\DeclareRobustCommand{\coroll}[1]{%
  \refstepcounter{corollcount}%
  \noindent\textit{\textbf{Corollary \thecorollcount\label{coroll:#1}: }}%
}
\DeclareRobustCommand{\corollref}[1]{Corollary~\ref{coroll:#1}}

\newcounter{lemmacount}
\setcounter{lemmacount}{0}
\DeclareRobustCommand{\lemma}[1]{%
  \refstepcounter{lemmacount}%
  \noindent\textit{\textbf{Lemma \thelemmacount\label{lemma:#1}: }}%
}
\DeclareRobustCommand{\lemmaref}[1]{Lemma~\ref{lemma:#1}}

\newcounter{definitioncount}
\setcounter{definitioncount}{0}
\DeclareRobustCommand{\definition}[1]{%
  \refstepcounter{definitioncount}%
  \noindent\textit{\textbf{Definition \thedefinitioncount\label{definition:#1}: }}%
}
\DeclareRobustCommand{\defref}[1]{Definition~\ref{definition:#1}}

%notes of different authors
\newif\ifnotes
\notestrue
\notesfalse

\newif\ifdiff
\difftrue
\difffalse

\newcommand{\anote}[1]{\ifnotes $\ll$\textsf{\textcolor{purple}{Ari: {#1}}}$\gg$ \fi}
\newcommand{\nnote}[1]{\ifnotes $\ll$\textsf{\textcolor{orange}{Novak: {#1}}}$\gg$ \fi}
\newcommand{\diff}[1]{\ifdiff\textcolor{orange}{#1}\else#1\fi}

%%% Local Variables:
%%% mode: latex
%%% TeX-master: "main"
%%% End:


%%% Redefining theorem-like environments
\newcounter{environments}

\newcounter{theoremCounter}
\newcounter{lemmaCounter}
\newcounter{definitionCounter}
\newcounter{propositionCounter}
\newcounter{corollaryCounter}
\newcounter{exampleCounter}
\newcounter{remarkCounter}
\newcounter{propertyCounter}
\newcounter{assumptionCounter}
\newcounter{proofCounter}

%\theorempreskip{1pt}
%\theorempostskip{1pt}

\let\proposition\relax
\let\theorem\relax
\let\lemma\relax
\let\definition\relax
\let\corollary\relax
\theoremseparator{.}
\theorembodyfont{\itshape}
\theoremsymbol{$\triangleleft$}
\newtheorem{theorem}[theoremCounter]{Theorem}
\newtheorem{lemma}[lemmaCounter]{Lemma}
\newtheorem{definition}[definitionCounter]{Definition}
\newtheorem{proposition}[propositionCounter]{Proposition}
\newtheorem{corollary}[corollaryCounter]{Corollary}

\let\remark\relax
\let\example\relax
\let\assumption\relax

\theorembodyfont{\normalfont}
\newtheorem{example}[exampleCounter]{Example}
\newtheorem{remark}[remarkCounter]{Remark}

\theoremheaderfont{\itshape}
\theoremsymbol{}
\renewtheorem{property}[remarkCounter]{Property}

\theoremheaderfont{\bfseries}
\theorembodyfont{\itshape}
\newtheorem{assumption}[assumptionCounter]{Assumption}
\theoremheaderfont{\itshape}


\theoremstyle{plain}
\theoremheaderfont{\itshape}
\theorembodyfont{\normalfont}
\let\proof\relax
\theoremseparator{.}
\theoremsymbol{\qedfull}
\newtheorem*{proof}{Proof}
\qedsymbol{\qedfull}

% Reset equation counters for each property!
\makeatletter
\@addtoreset{equation}{property}
\makeatother

% Tikz stuff
\newcommand{\seqarr}
{\begin{tikzpicture}
		\draw[-{Triangle[scale=.7]}] (0,0) --  (.3,0); 
\end{tikzpicture}}

\newcommand{\looparr}
{\begin{tikzpicture}[scale=0.7,baseline=-1.55ex]
		\draw[arrows = {-Stealth[inset=0pt, length=2pt, angle'=60]}] (0,0) arc (102:437:.2cm);
\end{tikzpicture}}

\definecolor{blue-violet}{rgb}{0.54, 0.17, 0.89}
\definecolor{cadmiumorange}{rgb}{0.93, 0.53, 0.18}
\definecolor{yellow-green}{rgb}{0.6, 0.8, 0.2}
\definecolor{green1}{rgb}{0.12, 0.3, 0.17}
\definecolor{byzantium}{rgb}{0.44, 0.16, 0.39}


%
%
%

\section{Results}
\label{results}

\begin{figure*}[ht]
    \centering
    \includegraphics[scale=0.15,trim={0 2.5cm 0 5cm},clip]{images/aoi-single_burst}
    \caption{The time average peak Age of Information with burst and \gls{soa} loss values against the dynamic reliability logic for different network topologies.}
    \label{fig:aoi_burst}\vspace{-0.4cm}
\end{figure*}


This paper focuses on both transport layer and application layer metrics to determine the feasibility of dynamic reliability. For this, we have selected the session packet volume, as transmitted, retransmitted, lost and backlogged packets as \glspl{kpi} for the transport layer; while focusing on the \gls{aoi} for the application layer. The \gls{aoi} was chosen as a crucial indicator for the freshness of packets in real-time applications. More specifically, this work adopts the time average peak \gls{aoi} equation \cite{aoi_equation} depicted in Eq. \ref{aoi}, where $\Delta(r_{i+1})$ is the $i$th update at the time it was received at the server, for a session time period of $\tau$.

\begin{equation}
    \label{aoi}
    \gls{aoi}_\tau = \frac{1}{n-1}\sum_{i=1}^{n-1} \Delta(r_{i+1})
\end{equation}

We include a comparison between the vanilla QUIC implementation which does not enjoy the dynamic reliability extension, with a number of dynamic reliability policies. The tests were run a number of times for statistical significance, with the mean value of vanilla implementation used as a baseline for comparison. The topology utilised both random loss and bursty loss to explore the bounds of dynamic reliability. The \gls{soa} loss in the figures correspond to the loss values presented in Table. \ref{tab:path_char}, for ease of comparison between bursty and random loss scenarios.

\subsection{Transport-Layer KPIs}

To analyse the performance gain at the transport layer due to dynamic reliability, the volume of transmitted and backlogged packets is examined. The figures are in the form of boxplots, which take the vanilla implementation as a benchmark, depicted as the red dashed line.

As seen in Fig. \ref{fig:sent_burst}, the loss plays a crucial role in the performance of the reliability policies. The policies under random loss did incredibly well for the networks with a larger capacity, namely \gls{mmwave} and Sub-6~GHz, whereas for burst loss, the lower network capacities had a larger packet reduction. With the increase in burst loss, the behaviour of the set split reliable policies became unpredictable, if a reliable assignment happened to coincide with a burst loss, the number of transmitted packets increases, and vice versa. On the other hand, in smarter policies, such as Loss-Aware, the performance lightly matched the vanilla baseline, as the reliable assignment dominated the session to compensate for a higher burst loss. Not only that but, the burst loss also impacted the variance of the transmitted packets for the policies.

Unsurprisingly, the unreliable focused policy, 80-20 split, outperformed other policies for all topologies in random and bursty loss scenarios, with an approximate reduction of 80\%. That being said, the majority of the policies reduced the transmitted packets on the link by approximately 70\% for random loss, while the reduction started at $\approx 15\%$ and decreased as the loss increased for the burst loss scenario.

The retransmitted and lost packets, not shown due to space limitations, followed the same trend as the transmitted packets for the random loss scenarios. However, for the burst loss scenarios, the larger capacity networks had a lower reduction in the retransmitted and lost packets. This can be seen as a favorable outcome since the lower capacity networks are scarce on resources. It is important to note that the Loss-Aware policy mimicked the vanilla approach as the burst loss increased, signifying the overwhelming appointment of reliable packets in adapting to the harsh burst loss conditions.
 
Alternatively, Fig. \ref{fig:backlog_burst} clearly shows a stark comparison between the policies and loss scenario in the reduction of the backlogged packets. The Loss-Aware policy for random loss scenario reduced the backlogged packets by up to 50\%, beating all other policies by approximately 30\%. Furthermore, it is clear that the unreliability focused policies resulted in the lowest backlog for the session. In comparison, we notice that the burst loss and the backlogged frequency have a positive correlation, where the maximum reduction of the backlogged packets for the policies is at most 20\%. Much like the transmitted packets, the probability of a burst loss occurrence plays a vital role in the number of retransmissions sent and by extension the number of backlogged packets. Thus, we can conclude that the stress placed on the buffer is a result of the reliable packets which is tightly coupled with the congestion on the session. Whereas, unreliable focused policies did not encounter such a phenomenon regardless if it was experiencing a burst loss.


\subsection{Application-Layer KPIs}

The feasibility of dynamic reliability for real-time applications can be determined by the \gls{aoi}, with comparison across different topologies and policies. If we take a strict approach and consider anything below $10$~ms is real-time \cite{real-time}, then all the reliability policies passed that requirement, which is attractive for real-time applications, as shown in Fig. \ref{fig:aoi_burst}. Utilising the median as an estimate of the runs, the policies in the WLAN and Sub-6~GHz topology with random loss floated around $4-5$~ms with negligible difference, while the \gls{aoi} for \gls{mmwave} was $\approx 2-3$~ms. It is clear that the \gls{aoi} and the network capacity have a negative correlation, as the network capacity decreases, the \gls{aoi} increases. The same correlation is extended to the bursty loss scenarios, where \gls{mmwave} dominated the other topologies. That being said, it is crucial to note that the \gls{aoi} for the reliability policies is often slightly better than or equal to the \gls{aoi} of the vanilla implementation, proving that dynamic reliability reduces the congestion of the session at no cost to the \gls{aoi}.


%
%
%

\def\secstatus{draft}

\section{Conclusion}
\label{sec:conclusion}

In this work we designed and developed adaptive, scalable, fully implicit solvers for the relativistic Fokker--Planck equation in phase space.
One key goal of the work is to develop a scalable and efficient dynamic AMR.
In practice, one needs to determine the requirements on scalability and then
adjust the parameters of the AMR algorithms in terms of refinement frequencies
and the AMR prediction horizon.  The present work proposes a new prediction strategy for refinement indicators,
giving a potential tool to exert control
over accuracy at computational overheads that were demonstrated to be low.
Numerical experiments quantify the predictive approach behaving better than conventional indicators, which lack prediction, in terms of resolving features in the solutions.
When the need for scalability is of higher importance to the application, one can use the
proposed AMR prediction at the expense of further increases of computational
overheads from finer meshes.

%
%
%

We contribute a new scalable implementation of the proposed algorithm using the p4est and PETSc frameworks.
Although the focus of the current work is to develop a solver aiming specifically at runaway electrons during tokamak disruptions,
a large portion of our proposed algorithm is general and thus can be potentially applied to solving many other time-dependent PDEs with dynamic mesh adaptivity.
Several numerical examples are presented in order to verify the solver's accuracy, scalability, and efficiency in adaptivity.
These numerical results confirm the parallel and algorithmic scalabilities and accuracy of the schemes.
A significant improvement of computational cost owing to the AMR algorithm was demonstrated using a practical disruption study.

Future work includes generalizing the fixed external field in the current model to a self-consistent model that evolves the electric field through involving the runaway current.
Another direction to pursue is to extend the computational domain to the low-energy region, which needs a careful treatment of the boundary condition at $p=0$.

%
%
%

\section*{Acknowledgment}
This research used resources provided by the Los Alamos National Laboratory Institutional Computing Program, which is supported by the U.S. Department of Energy National Nuclear Security Administration under Contract No.~89233218CNA000001,
and the National Energy Research Scientific Computing Center
(NERSC), a U.S. Department of Energy Office of Science User Facility located at Lawrence Berkeley National Laboratory, operated under Contract No. DE-AC02-05CH11231 using NERSC award FES-ERCAP0021219.

This research was funded in part and used resources of the Argonne Leadership Computing Facility, which is a DOE Office of Science User Facility supported under Contract No. DE-AC02–06CH11357.

The Texas Advanced Computing Center (TACC) at The University of Texas at Austin
provided HPC resources that have contributed to the research results reported
within this paper.

%
%
%

\clearpage
\appendix

\section{Guiding center coordinate transformation}
\label{sec:coordinate}

The coordinate transformation from
$(\vec{X}, p_\parallel, \vec{p}_\perp)$ to $(\vec{X}, p, \xi)$ is described in this section. The discussion focuses on the phase space.
The differential element for momentum space integration is
\begin{align}
d^3{\vec{p}} = dp_\parallel d^2{\vec{p}_\perp} = dp_\parallel (p_\perp dp_\perp d\theta).
\end{align}
The guiding center model assumes azimuthal symmetry. Thus one can integrate over $\theta\in (0,2\pi)$ (here $p_\perp$ is positive), giving
\begin{align}
\int_\theta d^3\vec{p} = 2\pi p_\perp d p_\parallel d p_\perp.
\end{align}
The transformation from $(p_\parallel, p_\perp)$ to $(p,\xi)$ coordinate is given by
\begin{align}
p & = \sqrt{p_\parallel^2 + p_\perp^2} \\
\xi & = \frac{p_\parallel}{p} = \frac{p_\parallel}{\sqrt{p_\parallel^2+p_\perp^2}}.
\end{align}
The Jacobian of this transformation is
\begin{align}
\frac{\partial (p_\parallel, p_\perp)}{\partial (p,\xi)}
=
\left|\frac{\partial p_\parallel}{\partial p} \frac{\partial p_\perp}{\partial\xi}
- \frac{\partial p_\parallel}{\partial\xi} \frac{\partial p_\perp}{\partial p}\right|
= \left|- \xi \frac{p\xi}{\sqrt{1-\xi^2}} - p\sqrt{1-\xi^2}\right|
= \frac{p}{\sqrt{1-\xi^2}}.
\end{align}
One then has
\begin{align}
\int_\theta d^3\vec{p} = 2\pi p_\perp d p_\parallel d p_\perp
= 2\pi p\sqrt{1-\xi^2} \frac{p}{\sqrt{1-\xi^2}} dp d\xi
= 2\pi p^2 dp d\xi .
\end{align}
Noting that $\xi\in (-1,1),$ the integration over the entire momentum space is simply
$(4\pi/3) p^3,$ as expected.
This also indicates  that the overall Jacobian of $(p,\xi,\theta)$ coordinates is $p^2,$
for gyro-angle independent problems, and the overall Jacobian for the $(p,\xi)$ coordinate
is $2\pi p^2.$ For convenience, we will be using $J=p^2$ for the $(p,\xi)$ coordinates.

%
%
%

\section{Summary of physical quantities in RFP}
\label{sec:quantities}
Table~\ref{table:notation} summarizes some physical quantities and the corresponding notations  in this paper.

\begin{table}
\caption{Physical quantities in the RFP model and the notations used in the current work.\label{table:notation}}
\centering
\vskip 1ex
\footnotesize
\begin{tabular}{lc|lc}
\toprule
  \thead{Description} & \thead{Symbol} &  \thead{Description} & \thead{Symbol} \\
\midrule
  time     & $t$   & electron  charge        & $e$\\
  configuration space    & $\xv$ & electron mass & $m_e$  \\
  phase space & $\pv$  &  speed of light         & $c$ \\
  phase space parallel to $\Bv$      & $p_\parallel$     &  Lorentz factor        & $\gamma$      \\
  phase space perpendicular to $\Bv$ & $p_\bot$      &  electrical permittivity & $\epsilon_0$           \\
  magnitude of the momentum       & $p=\norm{\pv}$       & Coulomb logarithm & $\ln \Lambda$      \\
  pitch angle                     & $\xi=p_\parallel/p$       &   thermal velocity  & $v_t$       \\
%
%
  runaway election density        & $f$        & effective charge number & $Z$               \\
  %
  energy flux                     & $\Gamma_p$         & collision time scale                     & $\tau_c$      \\
  pitch angle flux                & $\Gamma_\xi$     & synchrotron  time scale                 & $\tau_s$        \\
  magnetic field                  & $\Bv$       &      intensity of damping    & $\alpha=\tau_c/\tau_s$             \\
%
%
%
  induced electric field  parallel to $\Bv$        & $E_\parallel$      &     M{\o}ller cross section & ${d\sigma}/{dp}$    \\
\bottomrule
\end{tabular}
\end{table}

%
%
%

%
%
%
%

%
%
%

\clearpage
\bibliography{references}

%
 \begin{center}
	\scriptsize \framebox{\parbox{4in}{Government License (will be removed at publication):
			The submitted manuscript has been created by UChicago Argonne, LLC,
			Operator of Argonne National Laboratory (``Argonne").  Argonne, a
			U.S. Department of Energy Office of Science laboratory, is operated
			under Contract No. DE-AC02-06CH11357.  The U.S. Government retains for
			itself, and others acting on its behalf, a paid-up nonexclusive,
			irrevocable worldwide license in said article to reproduce, prepare
			derivative works, distribute copies to the public, and perform
			publicly and display publicly, by or on behalf of the Government. The Department of Energy will provide public access to these results of federally sponsored research in accordance with the DOE Public Access Plan. http://energy.gov/downloads/doe-public-access-plan.
}}
	\normalsize
\end{center}
%

\end{document}
