\def\secstatus{finished}

\section{Governing equations}
\label{sec:governing-eqns}

We study the phase-space dynamics of a runaway electron distribution
and consider a simplified version of the relativistic drift-kinetic
Fokker--Planck--Boltzmann equation~\cite{brizard1999nonlinear}. The
governing equations we are considering are effectively described in
the spherical coordinates of $(p, \xi, \theta)$, with $p$ the
normalized momentum magnitude, $\xi = p_\parallel/p$ the pitch,
and $\theta$ the azimuthal angle.  Here $p$ is dimensionless
with normalization by $m_e c$, with $m_e$ the rest mass
of an electron and $c$ the speed of light, and $\xi$ is the cosine
of the polar angle in the standard spherical coordinate.  The parallel
direction,  denoted by subscript $\parallel,$ aligns with the
magnetic field.  We further follow the guiding center model and assume
the solution has azimuthal symmetry, which reduces the momentum space
to two dimensions.  This corresponds to a momentum-space domain of
$[0, +\infty]\times[-1, 1]$.  The normalized relativisitic
Fokker--Planck--Boltzmann (RFP) equation to describe runaway electron
distribution, $f(t, p, \xi),$ in a slab geometry, is given by
\begin{align}
\label{eqn:rfp}
\frac{\partial f}{\partial t} - E \left( \xi \frac{\partial f}{\partial p} + \frac{1-\xi^2}{p} \frac{\partial f}{\partial \xi} \right) = C(f) + \alpha R(f) + S(f)\,,
\end{align}
where $E$ stands for the normalized electric field parallel to the
magnetic field in a tokamak and its normalization scale is
$E_c \coloneqq m_e c/ e \tau_c$ with $e$ the elementary charge
and $\tau_c$ the time scale.  Here the time scale of the
equation, $\tau_c$, is the relativistic electron collision time given
by
\begin{align}
\tau_c \coloneqq \frac{ 4\pi \epsilon_0^2 m_e^2 c^3}{e^4 n_e \ln \Lambda},
\end{align}
where $n_e$ is the (background) thermal electron density, $\epsilon_0$
is the electrical permittivity, and $\ln \Lambda$ is the Coulomb
logarithm.  The so-called Connor--Hastie $E_c$ sets the critical value for
runaway electron generation~\cite{connor1975relativistic}.  In the
current study all those values are given as constant based on
practical devices such as ITER. On the right-hand side are
several secondary effects due to different types of collisions and
synchrotron radiation (defined in detail later), and $\alpha$ is
introduced to describe the intensity of radiation damping. All the physical
quantities appearing in the current work have been summarized
in~\ref{sec:quantities}.

This problem has been well studied by plasma
physicists~\cite{GuoMcDevittTang2017, stahl2017norse,
hesslow2018effect, Guo-etal-PoP-2019, daniel2020fully}, commonly using conservative
finite difference or finite volume schemes on a stretched grid. The
dynamics of the equation is primarily dominated by the advection term
due to the strong electric field. The distribution function has a
sharp boundary layer at $p = p_{\min}$ due to the boundary condition
(defined in detail later); and as time evolves, the distribution
accumulates around $\xi = -1$ if $E>0$ (or $\xi = 1$ if $E<0$) due to
the advection. The distribution is further impacted by the collisions
and the source term in the large $p$ region and eventually forms a
nontrivial tail structure that is of the greatest interest. A majority
of the previous work uses the stretched grid to resolve the localized
distribution and tails. Adaptive mesh refinement has not been
explored on this problem to the best of our knowledge.

We note that in~\cite{daniel2020fully} a different coordinate of
$(p_\parallel, p_\perp)$ is used.
Choosing the coordinate of $(p, \xi)$ for the RFP equation has several advantages. The
solution is highly localized, as found in~\cite{GuoMcDevittTang2017,
stahl2017norse, hesslow2018effect,Guo-etal-PoP-2019, daniel2020fully}. It is more
efficient to resolve the same structure in $(p, \xi)$ while the
coordinate of $(p_\parallel, p_\perp)$ leads to half of the domain
resolving a trivial distribution tail.  Note that the physics of
interest in this study is the small population of runaway electrons.
It is therefore much easier to resolve this small population tail in
$(p, \xi)$ by chopping off the large Maxwellian bulk close to $p = 0$.
%
One disadvantage of using $(p, \xi)$ is the coordinate singularity
when $p_{\min}\rightarrow 0,$ and the fine grid spacing in $\xi$ near
the $\xi=-1$ boundary where runaway electrons of small $p_\perp=\xi p$
are needed to be resolved for large $p.$ Here we set $p_{\min}$ to
correspond to a few times the thermal energy of the background
electrons, and we impose an approximate Dirichlet boundary condition given
by the Maxwellian bulk, which is commonly adopted in the physics
codes~\cite{GuoMcDevittTang2017, stahl2017norse, hesslow2018effect,
Guo-etal-PoP-2019, daniel2020fully} and will be detailed later.

\subsection{Coulomb collision}
In this study the bulk distribution is considered to be close to a
Maxwellian with a thermal speed
much less than the light speed, namely, $v_t/c \ll 1$, where $v_t$
stands for the thermal velocity of the background electrons.  The
(small-angle) Coulomb collision of the runaway electrons, which is of
much lower density, with the
background electrons, $C(f),$ can be approximated by the so-called
test-particle collision operator~\cite{GuoMcDevittTang2017,
stahl2017norse}. We use the test-particle collision operator proposed
in~\cite{papp2011runaway} that is applicable to both thermal
($p\sim v_t/c \ll 1$) and relativistic ($p > 1$) electrons.  This
Coulomb collision operator has the form
\begin{align}
\label{eq:collision}
  C(f) =
    \frac{1}{p^2}\partd{}{p}\left[p^2\left(C_F f + C_A\partd{f}{p}\right)\right] +
    \frac{C_B}{p^2}\partd{}{\xi}\left[\left(1-\xi^2\right)\partd{f}{\xi}\right],
\end{align}
where all the collision coefficients are $p$-dependent and given by
\begin{align*}
  C_F &= \frac{2 c^2}{v_t^2}\Psi(x),\\
  C_A &= \frac{\sqrt{1+p^2}}{p}\Psi(x),\\
  C_B &= \frac{\sqrt{1+p^2}}{2p}
         \left(Z + \erf(x) - \Psi(x) + \frac{v_t^2} {2 c^2} \, \frac{p^2}{1+p^2}\right).
\end{align*}
Here $x$ is $p$-dependent,
\[
  x = \frac{c}{v_t} \, \frac{p}{\sqrt{1+p^2}},
\]
$Z$ stands for a charge number, and $\Psi(x) = [\erf(x) -
x \erf'(x)]/(2x^2)$ is the Chandrasekhar function.  A study of the
collision coefficients in a transition from nonrelativistic to
relativistic regimes is presented in~\cite{GuoMcDevittTang2017}, where the authors show   that the collision coefficients at small $p$ are several
orders of magnitude larger than those at larger $p$ and thus the
diffusion becomes dominant compared with the advection field.
%
Therefore, a fully implicit treatment of the collision operator is
necessary.  When needed, the collision coefficients are also modified
to include the partial screening effect~\cite{hesslow2018effect}, which
addresses the role of bound electrons in collisions between fast
electrons and partially ionized impurities.  Mathematically, it
corresponds to modifying the $C_B$ and the Coulomb logarithm
$\ln \Lambda$, but it does not involve additional numerical
challenges.  For more details, see~\cite{hesslow2017effect,
hesslow2018effect}.

\subsection{Synchrotron radiation}
When there is an acceleration of a charged particle,
there is an energy loss through the emitted
electromagnetic radiation.
In a dominating magnetic field where a charged particle follows cyclotron motion,
relativistic electrons can emit a significant amount of synchrotron radiation.  The
resulting radiation damping operator satisfies
\begin{align}
\label{eq:radiation}
  R(f) = \frac{1}{p^2}\partd{}{p}\left[p^3\gamma(1-\xi^2)f\right] -
               \partd{}{\xi}\left[\frac{\xi(1-\xi^2)}{\gamma}f\right],
\end{align}
where $\gamma=\sqrt{1+p^2}$ is the Lorentz factor.  The dimensionless
parameter for the intensity of damping is formally defined as
$\alpha\coloneqq\tau_c/\tau_s$, a ratio between the collision time
scale and the synchrotron radiation time scale, where
$\tau_s \coloneqq 6\pi \epsilon_0 m_e^3 c^3/e^4 B^2$ and $B$ is the
magnitude of the magnetic field.  Typically it is about 0.1 to 0.3 for
ITER-like tokamaks and 0.001 to 0.05 for a smaller device, which is
the range we use in our simulations.

\subsection{Knock-on collision}
\label{sec:knock_on}
Another important source in runaway electron generations is the
avalanche mechanism due to knock-on collisions between runaway
electrons and the background thermal
electrons~\cite{Rosenbluth_1997,chiu1998fokker}.  The knock-on source
considers the secondary large-angle collisions in addition to the
primary small-angle collisions of the Fokker--Planck
operator~\cite{boozer-pop-2015,breizman2019physics}.  The knock-on
source is
\begin{align}
S(p, \xi) = S_1(p, \xi) + S_2(p,\xi),
\label{eqn:chiu_full}
\end{align}
which consists of the secondary electron source term
\begin{align}
S_1(p, \xi) = \int_{-1}^1 \int_0^\infty \Pi(\gamma', \xi'; \gamma, \xi) f(p', \xi') 2 \pi p'^2 dp' \, d\xi',
\end{align}
%
and the annihilation term
\begin{align}
S_2(p, \xi) = -f(p, \xi)  \int_{-1}^1 \int_0^\infty \Pi(\gamma, \xi; \gamma', \xi') 2 \pi p'^2 dp' \, d\xi',
\end{align}
with $\gamma = \sqrt{1+p^2}$ and $\gamma' = \sqrt{1+p'^2}$.

In this work we use the Chiu model in~\cite{chiu1998fokker} and approximate the scattering rate function in $S_1$ by
\begin{align}
\Pi(\gamma', \xi'; \gamma, \xi) \approx \frac 1 {\ln \Lambda} \frac{p'^2}{2\pi p^2 \, |\xi|} \frac{d \sigma (\gamma', \gamma)}{dp} \, \delta(p' - p^*),
\end{align}
where $\delta(\cdot)$ is a Dirac delta function that indicates only electrons with particular energy at $p^*$ can generate the required secondary electron at a given point $(p, \xi)$
with $p^* = \sqrt{(\gamma^*)^2-1}$ and $\gamma^* = \frac{(\gamma+1)/(\gamma-1) \, \xi^2 + 1}{(\gamma+1)/(\gamma-1) \, \xi^2 - 1}$.
The M{\o}ller cross section for large-angle collisions is used,
\begin{align}
 \frac{d \sigma (\gamma', \gamma) }{dp} = \frac{p}{\gamma}\, \frac{2\pi\gamma'^2 }{(\gamma'-1)^3(\gamma'+1)}\left[
 x^2 -3 x + \left( \frac{\gamma'-1}{\gamma'}\right)^2 (1+x) \right],
\end{align}
where
\begin{align*}
x = \frac 1{\nu(1-\nu)},\quad \text{and} \quad \nu = \frac{\gamma-1}{\gamma'-1}.
\end{align*}
Note that the M{\o}ller cross section is turned on only if $\gamma'\ge 2\gamma-1$;  otherwise $S_1$ is set to  0, which is commonly imposed
(see~\cite{mcdevitt2019avalanche}, for instance).
One can easily show that the source term $S_1$ is  positive only in a thin region of $\xi \in [-\sqrt{\gamma/(\gamma+1)}, -p/(\gamma+1)]$.
An adaptive mesh is beneficial to resolve this thin region.
The annihilation term is integrated in the range of $\gamma' \in [\gamma_0, \frac{\gamma+1}2]$,
where $\gamma_0$ is computed from the computation domain, namely, $\gamma_0 = \sqrt{1+p_{\min}^2}$.
The M{\o}ller cross section is well defined in such a range.
This leads to
\begin{align}
S_2(p,\xi) = -\frac 1{\ln\Lambda} f(p, \xi) \, \sigma(\gamma, \gamma_{0}),
\end{align}
where $\sigma(\gamma, \gamma_{0})$ is the integrated form of the M{\o}ller cross section, given by
\begin{align*}
\sigma(\gamma, \gamma_{0}) = \frac{2\pi}{\gamma^2-1}\left[ \frac{\gamma+1}2 -\gamma_{0} - \gamma^2\left(\frac{1}{\gamma-\gamma_{0}} - \frac 1{\gamma_{0}-1} \right) + \frac{2\gamma-1}{\gamma-1}\ln \frac{\gamma_{0}-1}{\gamma-\gamma_{0}} \right]
\end{align*}
for $\gamma\ge2\gamma_0-1$ or otherwise $\sigma(\gamma, \gamma_0) = 0$.
%


%
%
%
%
%
%
%
%
%
%
%
%
%
%
%
%
%
%
%
%
%
%
%
%
%
%
%
%
%

\subsection{Computational domain and boundary conditions}
Since the focus of this study is on relativistic runaway electrons where
the bulk of the distribution is still assumed to be a Maxwellian, it
is reasonable to use a computational domain: $[p_{\min},
p_{\max}] \times [-1, 1]$, where $p_{\min}$ is chosen as being thermal
(e.g., let $p_{\min} = v_t/c$ or a few times of $v_t/c$).  One can
further assume that the left boundary condition at $p = p_{\min}$ is
Dirichlet and given by the Maxwell--J\"uttner distribution, which is
uniform in $\xi$. At the right boundary of $p = p_{\max}$, a Neumann
boundary condition $\partial f/\partial n = 0$ is prescribed, so one
must check the solution to make sure that $p_{\max}$ is high enough that
$f$ is negligibly small there.  At the top and bottom boundaries, because of the involvement of the coefficient $1-\xi^2$, all the fluxes along
$\xi$ become 0, which is in fact the natural outcome of the coordinate
chosen. Therefore, Neumann boundary conditions are used at the top and
bottom boundaries.


\subsection{Conservation and an alternative RFP form}

The determinant of the Jacobian  corresponding to the transformation
between $(p_\parallel, \vec{p}_\perp)$ to $(p, \xi)$, under azimuthal
symmetry,  is given by $J=p^2$ (see \ref{sec:coordinate}).
Thus, the definition of the divergence of a vector $\vv = (v_1, v_2)$ in $(p,\xi)$ becomes
\[
\nabla \cdot \vv \coloneqq
  \frac 1 {p^2} \Big[ \frac{\partial}{\partial p}(p^2 v_1) +
  \frac{\partial}{\partial\xi} (p^2 v_2) \Big].
\]
One can easily show that the electric field ``advection'' term
in~\eqref{eqn:rfp}, $\big(-E \xi, - E ({1-\xi^2})/{p} \big)$, is
divergence-free. This is in fact more obvious in the coordinate of
$(p_\parallel, \mathbf{p}_\perp)$, as the RFP equation in
$(p_\parallel, \mathbf{p}_\perp)$ becomes
\[
\frac{\partial f}{\partial t} - E \frac{\partial f}{\partial p_\parallel}= C(f) + \alpha R(f) + S(f)
\]
and $E$ is a constant. It is also easy to see that $f p^2$ is a conserved value in the coordinate of $(p, \xi)$, which is consistent with the fact that the distribution function is conservative in $(p_\parallel, \mathbf{p}_\perp)$. As a result, a finite volume method is chosen in this study to satisfy the conservation of the distribution.

In our implementation, we choose to solve $\tilde f \coloneqq fp$. Then we can rewrite the RFP equation~\eqref{eqn:rfp} as
\begin{align}
\label{eqn:conser2}
  \partd{\tilde f}{t} +
  \frac{1}{p}\partd{\Gamma_p(\tilde f)}{p} +
  \partd{\Gamma_\xi(\tilde f)}{\xi}
  =   S(\tilde f),
\end{align}
where the fluxes are
\begin{align}
\label{eqn:conser2-flux-p}
  \Gamma_p(\tilde f) &=
    -p \left[ E\xi +\alpha p \gamma (1-\xi^2) \right] \tilde f -
    (p C_F - C_A) \tilde f - p C_A \partd{\tilde f}{p},\\
\label{eq:conser2-flux-xi}
  \Gamma_\xi(\tilde f) &=
    -(1-\xi^2) \left[ \left(\frac E p - \frac{\alpha  \xi}{\gamma} \right) \tilde f +
                      \frac{C_B}{p^2}\partd{\tilde f }{\xi} \right].
\end{align}
This conservative form is a choice of our implementation. In  previous work~\cite{GuoMcDevittTang2017}, it is shown that solving for $\tilde f$ helps diminish the negativity of the distribution function in the tail region.


