% JOURNAL FORMAT STUFF - 
% Are appendices even allowed?
% . stops for abbreivations
% Oxford comma?
% American spellings?
% Equations... should be referenced  from within the main text as Eq. (1), Eq. (2), and so on [or as inequality (1), etc., as appropriate].
% After proofreading the manuscript, compress your .tex manuscript file and all figures (which should be in EPS or PDF format) in a ZIP, TAR or TAR-GZIP package. All files must be referenced at the root level (e.g., file \texttt{figure-1.eps}, not \texttt{/myfigs/figure-1.eps}). 
% The table over runs in Appendix B





% WHAT ALL IS LEFT TO DO SCIENCE-WISE
% Compound adjectives.
% Insert, once ready, the numerical examples.
% Get populating with references.
% Tune up the chiral electric field figure.
% Christmas tree figures
% J=3 analytical eigenstates
% J=2 analytical force coefficients?
% Write a section comparing with the helicity chiral optical force?
% A deflection example?
% In appendix A we spell out individual components of J but in the main text we use eg $\alpha_{ab}$. Which is it?
% Use \Re wherever you can (e.g. coefficients)
% Go through and think very carefull about spaces (\,) and brackets!!
% Make sure you've properly replaced f with F and \mu with \mu_0 as required.
% Have we been consistent throughout with name dropping things?
% Haver we been consistent throughout with commas in equations and equation formatting.
% Have we been consistent throughought with numbers, plus signs and significant figures.
% Have we been consistent throughout with left versus right?
% Perturbation theory?
% Is the result in the abstract correct if we have a B_0?
% Ah, fuck; $E_{0x}$ is simultaneously the $x$ component of $\mathbf{E}_0$ AND the amplitude of the x component of $\tilde{\mathbf{E}}$. Fuck, fuck, fuck.

%%%%%%%%%%%%%%%%%%%%%%%%%%%%%%%%%%%%%%%%%%%%%%%%%%%%%%%%%%%%%%%%%%%%%%%%%%%%
%      
%%%%%%%%%%%%%%%%%%%%%%%%%%%%%%%%%%%%%%%%%%%%%%%%%%%%%%%%%%%%%%%%%%%%%%%%%%%%
%%%%%%%%%%%%%%%%%%%%%%%%%%%%%%%%%%%%%%%%%%%%%%%%%%%%%%%%%%%%%%%%%%%%%%%%%%%%
%       PREAMBLE
%%%%%%%%%%%%%%%%%%%%%%%%%%%%%%%%%%%%%%%%%%%%%%%%%%%%%%%%%%%%%%%%%%%%%%%%%%%%
%%%%%%%%%%%%%%%%%%%%%%%%%%%%%%%%%%%%%%%%%%%%%%%%%%%%%%%%%%%%%%%%%%%%%%%%%%%%
%       
%%%%%%%%%%%%%%%%%%%%%%%%%%%%%%%%%%%%%%%%%%%%%%%%%%%%%%%%%%%%%%%%%%%%%%%%%%%%

\documentclass{optica-article}
\journal{opticajournal} % FOR JOURNALS OR OPTICA OPEN
\articletype{Research Article}
\usepackage{lineno}
% \linenumbers % TURN THIS OFF (LINE NUMBERING) FOR OPTICA OPEN PREPRINT SUBMISSIONS
\begin{document}

%%%%%%%%%%%%%%%%%%%%%%%%%%%%%%%%%%%%%%%%%%%%%%%%%%%%%%%%%%%%%%%%%%%%%%%%%%%%
%      
%%%%%%%%%%%%%%%%%%%%%%%%%%%%%%%%%%%%%%%%%%%%%%%%%%%%%%%%%%%%%%%%%%%%%%%%%%%%
%%%%%%%%%%%%%%%%%%%%%%%%%%%%%%%%%%%%%%%%%%%%%%%%%%%%%%%%%%%%%%%%%%%%%%%%%%%%
%       TITLE, AUTHOR(S) AND ABSTRACT
%%%%%%%%%%%%%%%%%%%%%%%%%%%%%%%%%%%%%%%%%%%%%%%%%%%%%%%%%%%%%%%%%%%%%%%%%%%%
%%%%%%%%%%%%%%%%%%%%%%%%%%%%%%%%%%%%%%%%%%%%%%%%%%%%%%%%%%%%%%%%%%%%%%%%%%%%
%       
%%%%%%%%%%%%%%%%%%%%%%%%%%%%%%%%%%%%%%%%%%%%%%%%%%%%%%%%%%%%%%%%%%%%%%%%%%%%

\title{Strong chiral optical force for small chiral molecules based on electric-dipole interactions}
\author{Robert P. Cameron,\authormark{1,*} Duncan McArthur,\authormark{1} and Alison M. Yao\authormark{2,3}
\address{\authormark{1}SUPA and Department of Physics, University of Strathclyde, Glasgow G4 0NG, UK}

\email{\authormark{*}robert.p.cameron@strath.ac.uk} 
%% email address is required; see note below about the corresponding author designation
% use {asbstract*} to suppress the copyright line. Copyright information will be added in production

\begin{abstract*} 
\noindent We show that a static electric field together with a lin$\bot$lin standing wave can exert a chiral optical force on a small chiral molecule of the form
\begin{align}
\mathbf{F}&\approx kE_{0z}E_{0y}E_{0x}(\mathtt{A}\alpha_{ZY}\mu_{0X}+\mathtt{B}\alpha_{XZ}\mu_{0Y}+\mathtt{C}\alpha_{YX}\mu_{0Z})\cos(2kZ_0)\hat{\mathbf{z}}, \nonumber
\end{align}
which changes sign if either the electric field or the molecule is inverted. Our chiral optical force can be orders of magnitude stronger and scales more favourably with increasing wavelength than other forces proposed to date, as it is based on electric-dipole interactions. Molecular anisotropy is fully accounted for in our theory and indeed plays a vital role. Our chiral optical force applies to most small chiral molecules, including isotopically chiral molecules, and does not involve absorption or require a specific energy-level structure. It could form the basis of robust enantiomer resolution schemes and much more besides. We present quantum-chemical calculations of our chiral optical force for seychellene, using realistic parameters.
\end{abstract*}
% THE ABSTRACT SHOULD BE APPROXIMATELY 100 WORDS, APPARENTLY.
% DO NOT INCLUDE NUMBERS, BULLETS, OR LISTS INSIDE THE ABSTRACT. 

%%%%%%%%%%%%%%%%%%%%%%%%%%%%%%%%%%%%%%%%%%%%%%%%%%%%%%%%%%%%%%%%%%%%%%%%%%%%
%      
%%%%%%%%%%%%%%%%%%%%%%%%%%%%%%%%%%%%%%%%%%%%%%%%%%%%%%%%%%%%%%%%%%%%%%%%%%%%
%%%%%%%%%%%%%%%%%%%%%%%%%%%%%%%%%%%%%%%%%%%%%%%%%%%%%%%%%%%%%%%%%%%%%%%%%%%%
%       INTRODUCTION
%%%%%%%%%%%%%%%%%%%%%%%%%%%%%%%%%%%%%%%%%%%%%%%%%%%%%%%%%%%%%%%%%%%%%%%%%%%%
%%%%%%%%%%%%%%%%%%%%%%%%%%%%%%%%%%%%%%%%%%%%%%%%%%%%%%%%%%%%%%%%%%%%%%%%%%%%
%       
%%%%%%%%%%%%%%%%%%%%%%%%%%%%%%%%%%%%%%%%%%%%%%%%%%%%%%%%%%%%%%%%%%%%%%%%%%%%

\section{Introduction}
\label{Introduction}
% OPENING PARAGRAPH
Interest is growing in the possibility of a chiral optical force that discriminates well between the enantiomers of a small chiral molecule \cite{Marichez19a, Genet22a}. Such a force could be used to resolve enantiomers; a task of vital importance in chemistry and biology \cite{Lough02a, Gardner05a}. The forces proposed to date \cite{Canaguier-Durand13a, Cameron14b}, however, are extremely weak and scale unfavourably with increasing wavelength, as they rely on magnetic-dipole and electric-quadrupole interactions. Furthermore, the effects of molecular anisotropy have been almost entirely ignored. To the best of our knowledge, no experimental observations of chiral optical forces have been reported for small chiral molecules, although chiral optical forces have been demonstrated for large chiral objects, including chiral cantilevers \cite{Zhao17a} and chiral liquid crystal microspheres \cite{Kravets19a}.

% WHAT'S THE BIG PICTURE?
In this paper, we show that a static electric field together with a lin$\bot$lin standing wave can exert a chiral optical force on a small chiral molecule of the form
\begin{align}
\mathbf{F}&\approx kE_{0z}E_{0y}E_{0x}(\mathtt{A}\alpha_{ZY}\mu_{0X}+\mathtt{B}\alpha_{XZ}\mu_{0Y}+\mathtt{C}\alpha_{YX}\mu_{0Z})\cos(2kZ_0)\hat{\mathbf{z}}, \nonumber
\end{align}
which changes sign if either the electric field or the molecule is inverted. Our chiral optical force can be orders of magnitude stronger and scales more favourably with increasing wavelength than other forces proposed to date \cite{Canaguier-Durand13a, Cameron14b}, as it is based on electric-dipole interactions. Molecular anisotropy is fully accounted for in our theory and indeed plays a vital role. Our chiral optical force applies to most small chiral molecules, including isotopically chiral molecules, and does not involve absorption or require a specific energy-level structure. It could form the basis of robust enantiomer resolution schemes and much more besides.

% BREAKDOWN OF SECTIONS
The paper is structured as follows. In Sec. \ref{Electromagnetic field}, we define the electromagnetic field that produces our chiral optical force. In Sec. \ref{Derivation of chiral optical force}, we derive the expression quoted above for our chiral optical force. In Sec. \ref{Numerical results}, we present quantum-chemical calculations of our chiral optical force for seychellene, using realistic parameters. In Sec. \ref{Outlook}, we highlight possible directions for future research.

% CONVENTIONS
We work in vacuum in an inertial frame of reference with time $t$ and position vector $\mathbf{r}=x\hat{\mathbf{x}}+y\hat{\mathbf{y}}+z\hat{\mathbf{z}}$, where $x$, $y$ and $z$ are laboratory-fixed Cartesian coordinates and $\hat{\mathbf{x}}$, $\hat{\mathbf{y}}$ and $\hat{\mathbf{z}}$ are the associated unit vectors. SI units are used, with $c$ being the speed of light and $\hbar$ being the reduced Planck constant. The Einstein summation convention is to be understood with respect to lower-case Roman indices $a,b,\dots\in\{x,y,z\}$ and uppercase Roman indices $A,B,\dots\in\{X,Y,Z\}$, where $X$, $Y$ and $Z$ are molecule-fixed Cartesian coordinates. Complex quantities are decorated with tildes.

%%%%%%%%%%%%%%%%%%%%%%%%%%%%%%%%%%%%%%%%%%%%%%%%%%%%%%%%%%%%%%%%%%%%%%%%%%%%
%      
%%%%%%%%%%%%%%%%%%%%%%%%%%%%%%%%%%%%%%%%%%%%%%%%%%%%%%%%%%%%%%%%%%%%%%%%%%%%
%%%%%%%%%%%%%%%%%%%%%%%%%%%%%%%%%%%%%%%%%%%%%%%%%%%%%%%%%%%%%%%%%%%%%%%%%%%%
%       ELECTROMAGNETIC FIELD
%%%%%%%%%%%%%%%%%%%%%%%%%%%%%%%%%%%%%%%%%%%%%%%%%%%%%%%%%%%%%%%%%%%%%%%%%%%%
%%%%%%%%%%%%%%%%%%%%%%%%%%%%%%%%%%%%%%%%%%%%%%%%%%%%%%%%%%%%%%%%%%%%%%%%%%%%
%       
%%%%%%%%%%%%%%%%%%%%%%%%%%%%%%%%%%%%%%%%%%%%%%%%%%%%%%%%%%%%%%%%%%%%%%%%%%%%

\section{Electromagnetic field}
\label{Electromagnetic field}
% ELECTRIC FIELD FIGURE
\begin{figure}[h!]
\centering
\includegraphics[width=0.75\textwidth]{ElectricField}
\caption{\small Left-handed a) and right-handed b) configurations of the electric field, depicted for $E_{0y}=E_{0x}>0$. The grey lines indicate the polarization state of the standing wave.}
\label{ElectricField}
\end{figure}

% DEFINE THE ELECTROMAGNETIC FIELD
As the origin of our chiral optical force, Curie's principle \cite{Curie94a} combined with geometrical insight from nature\footnote{The asymmetrical hydrozoan \textit{Velella velella} is orientated by gravity (via the surface of the ocean), enabling its enantiomorphous forms to be separated trivially by the wind \cite{Alexander03a, Gardner05a}. For a chiral molecule experiencing our chiral optical force, the static electric field $\mathbf{E}_0$ plays a role analogous to gravity and the standing wave (with electric field $\Re\tilde{\mathbf{E}}$) plays a role somewhat analogous to the wind.} leads us to consider a strong, static, homogeneous electric field $\mathbf{E}_0=\mathbf{E}_0(\mathbf{r},t)$ given by
\begin{align}
\mathbf{E}_0&=E_{0z}\hat{\mathbf{z}} \label{StaticE}
\end{align}
together with an intense, far off-resonance, lin$\perp$lin standing wave with complex electric field $\tilde{\mathbf{E}}=\tilde{\mathbf{E}}(\mathbf{r},t)$ given by
\begin{align}
\tilde{\mathbf{E}}&=(E_{0y}\hat{\mathbf{y}}\mathrm{e}^{\mathrm{i}kz}+\mathrm{i}E_{0x}\hat{\mathbf{x}}\mathrm{e}^{-\mathrm{i}kz})\mathrm{e}^{-\mathrm{i}\omega t}, \label{ComplexE}
\end{align}
where $E_{0z}$ dictates the strength and direction of the static electric field, $E_{0y}$ and $E_{0x}$ dictate the intensity and ellipticity of the standing wave and $\omega$ is the angular frequency of the standing wave, with $k=\omega/c$ being the angular wavenumber. Thus, the electric field $\mathbf{E}=\mathbf{E}(\mathbf{r},t)$ is
\begin{align}
\mathbf{E}&=\mathbf{E}_0+\Re\tilde{\mathbf{E}} \nonumber \\
&=E_{0z}\hat{\mathbf{z}}+E_{0y}\hat{\mathbf{y}}\cos(kz-\omega t)+E_{0x}\hat{\mathbf{x}}\sin(kz+\omega t). \nonumber
\end{align}
To help keep our theory relatively simple, we also consider a strong, static, homogeneous magnetic field $\mathbf{B}_0=\mathbf{B}_0(\mathbf{r},t)$ given by
\begin{align}
\mathbf{B}_0&=B_{0y}\hat{\mathbf{y}}, \label{StaticB}
\end{align}
where $B_{0y}$ dictates the strength and direction of the field. Thus, the magnetic field $\mathbf{B}=\mathbf{B}(\mathbf{r},t)$ is
\begin{align}
\mathbf{B}&=B_{0y}\hat{\mathbf{y}}+\Re(\boldsymbol{\nabla}\times\tilde{\mathbf{E}}/\mathrm{i}\omega)\nonumber \\
&=B_{0y}\hat{\mathbf{y}}-E_{0y}\hat{\mathbf{x}}\cos(kz-\omega t)/c-E_{0x}\hat{\mathbf{y}}\sin(kz+\omega t)/c. \nonumber
\end{align}
In what follows, we take $\mathbf{B}_0$ to decouple the molecule's rotational angular momentum from the molecule's nuclear spins whilst defining a quantisation axis via the rotational Zeeman effect. We work to zeroth order in $\mathbf{B}_0$, which is not trivial as the existence of $\mathbf{B}_0$ still dictates the choice of basis for the molecule's rotational states, according to the rules of degenerate perturbation theory. Let us emphasise here that $\mathbf{B}$ does not contribute \textit{directly} to our chiral optical force, which is based on electric-dipole interactions.

% ELECTRIC FIELD IS SORT OF CHIRAL
The electric field $\mathbf{E}$ has chirality in that the static electric field $E_{0z}\hat{\mathbf{z}}$ and the optical polarization vectors $E_{0y}\hat{\mathbf{y}}$ and $E_{0x}\hat{\mathbf{x}}$ form either a left-handed ($E_{0z}E_{0y}E_{0x}<0$) or a right-handed ($E_{0z}E_{0y}E_{0x}>0$) orthogonal triad, as illustrated in Fig. \ref{ElectricField}. 

% NO RELATION TO LIN PERP LIN
Lin$\perp$lin standing waves are used routinely to laser cool atoms via the Sisyphus effect \cite{Dalibard89a}. There is no obvious conection with our chiral optical force.

% THE GRAVE YARD
% Maybe a comment on the 'optical' frequency.
% Is it false chirality?

%%%%%%%%%%%%%%%%%%%%%%%%%%%%%%%%%%%%%%%%%%%%%%%%%%%%%%%%%%%%%%%%%%%%%%%%%%%%
%      
%%%%%%%%%%%%%%%%%%%%%%%%%%%%%%%%%%%%%%%%%%%%%%%%%%%%%%%%%%%%%%%%%%%%%%%%%%%%
%%%%%%%%%%%%%%%%%%%%%%%%%%%%%%%%%%%%%%%%%%%%%%%%%%%%%%%%%%%%%%%%%%%%%%%%%%%%
%       DERIVATION OF CHIRAL OPTICAL FORCE
%%%%%%%%%%%%%%%%%%%%%%%%%%%%%%%%%%%%%%%%%%%%%%%%%%%%%%%%%%%%%%%%%%%%%%%%%%%%
%%%%%%%%%%%%%%%%%%%%%%%%%%%%%%%%%%%%%%%%%%%%%%%%%%%%%%%%%%%%%%%%%%%%%%%%%%%%
%       
%%%%%%%%%%%%%%%%%%%%%%%%%%%%%%%%%%%%%%%%%%%%%%%%%%%%%%%%%%%%%%%%%%%%%%%%%%%%

\section{Derivation of chiral optical force}
\label{Derivation of chiral optical force}
% DEFINING THE MOLECULE
Consider a small, polar, diamagnetic, chiral molecule in its vibronic ground state with nuclear spins of $0$ or $1/2$, subject to the electromagnetic field defined in section \ref{Electromagnetic field}. Below, we show that the electromagnetic field exerts a chiral optical force on the molecule of the form 
\begin{align}
\mathbf{F}&\approx kE_{0z}E_{0y}E_{0x}(\mathtt{A}\alpha_{ZY}\mu_{0X}+\mathtt{B}\alpha_{XZ}\mu_{0Y}+\mathtt{C}\alpha_{YX}\mu_{0Z})\cos(2kZ_0)\hat{\mathbf{z}}, \nonumber
\end{align}
as claimed in section \ref{Introduction}. 

% VIBRONIC DEGREES OF FREEDOM ASSUMPTIONS
Let us initially freeze the molecule's orientation and centre of mass and focus on the molecule's vibronic degrees of freedom. We take the instantaneous force $\mathbf{f}=\mathbf{f}(t)$ exerted by the electromagnetic field on the molecule to be given by the Lorentz force law
\begin{align}
\mathbf{f}&=\iiint(\rho\mathbf{E}+\mathbf{J}\times\mathbf{B})\,\mathrm{d}^3\mathbf{r}, \label{F1}
\end{align}
where $\rho=\rho(\mathbf{r},t)$ and $\mathbf{J}=\mathbf{J}(\mathbf{r},t)$ are the molecule's electric charge and current densities and the region of integration extends over the entire molecule. As the molecule is electrically neutral, we have
\begin{align}
\rho&=-\boldsymbol{\nabla}\cdot\mathbf{P} \label{rho} \\
\mathbf{J}&=\frac{\partial\mathbf{P}}{\partial t}+\boldsymbol{\nabla}\times\mathbf{M}, \label{J}
\end{align}
where $\mathbf{P}=\mathbf{P}(\mathbf{r},t)$ and $\mathbf{M}=\mathbf{M}(\mathbf{r},t)$ are the molecule's polarization and magnetization fields. Substituting Eqs. (\ref{rho}) and (\ref{J}) into Eq. (\ref{F1}) then making use of the triple product $(\boldsymbol{\nabla}\times\mathbf{M})\times\mathbf{B}=(\mathbf{B}\cdot\boldsymbol{\nabla})\mathbf{M}-M_a\boldsymbol{\nabla}B_a$, integration by parts and the Faraday-Lenz law $\boldsymbol{\nabla}\times\mathbf{E}=-\partial\mathbf{B}/\partial t$, we obtain
\begin{align}
\mathbf{f}&=\iiint(P_a\boldsymbol{\nabla}E_a+M_a\boldsymbol{\nabla}B_a)\,\mathrm{d}^3\mathbf{r}+\frac{\mathrm{d}}{\mathrm{d}t}\iiint\mathbf{P}\times\mathbf{B}\,\mathrm{d}^3\mathbf{r}. \label{F2}
\end{align}
As the molecule is small, we consider only electric-dipole interactions explicitly (working to zeroth order in the static magnetic field $\mathbf{B}_0$). Thus, we take
\begin{align}
\mathbf{P}&\approx\boldsymbol{\mu}\,\delta^3(\mathbf{r}-\mathbf{R}_0) \label{P} \\
\mathbf{M}&\approx 0, \label{M}
\end{align}
where $\boldsymbol{\mu}=\boldsymbol{\mu}(t)$ is the molecule's electric-dipole moment and $\mathbf{R}_0=X_0\hat{\mathbf{x}}+Y_0\hat{\mathbf{y}}+Z_0\hat{\mathbf{z}}$ is the molecule's centre of mass, with $X_0$, $Y_0$ and $Z_0$ being the Cartesian coordinates. Substituting Eqs. (\ref{P}) and (\ref{M}) into Eq. (\ref{F2}), we obtain
\begin{align}
\mathbf{f}&\approx\mu_a\boldsymbol{\nabla}E_a(\mathbf{R}_0)+\frac{\mathrm{d}}{\mathrm{d}t}\left[\boldsymbol{\mu}\times\mathbf{B}(\mathbf{R}_0)\right]. \label{Fbar1}
\end{align}
Again, we consider only electric-dipole interactions explicitly. As the static electric field $\mathbf{E}_0$ is strong and the standing wave (with electric field $\Re\tilde{\mathbf{E}}$) is intense, we seek an expression for our chiral optical force valid to third order in the electric field $\mathbf{E}(=\mathbf{E}_0+\Re\tilde{\mathbf{E}})$. Thus, we take
\begin{align}
\boldsymbol{\mu}&=\boldsymbol{\mu}_0+\Re\tilde{\boldsymbol{\mu}} \label{mu}
\end{align}
with
\begin{align}
\tilde{\mu}_a&\approx\tilde{\alpha}_{ab}(0)E_{0b}+[\tilde{\alpha}_{ab}(\omega)+\tilde{\alpha}_{ab,c}^{(\mu)}(\omega)E_{0c}]\tilde{E}_b \nonumber \\
&+\frac{1}{2}\tilde{\beta}_{abc}(0;0,0)E_{0b}E_{0c}+\frac{1}{2}\tilde{\beta}_{abc}(2\omega;\omega,\omega)\tilde{E}_b\tilde{E}_c+\frac{1}{2}\tilde{\beta}_{abc}(0;\omega,-\omega)\tilde{E}_b\tilde{E}^\ast_c, \label{complexmu}
\end{align}
where $\boldsymbol{\mu}_0$ is the molecule's permanent electric-dipole moment, $\tilde{\boldsymbol{\mu}}$ is the molecule's complex induced electric-dipole moment, $\tilde{\alpha}_{ab}=\tilde{\alpha}_{ab}(\Omega)$ is the molecule's complex vibronic polarisability, $\tilde{\alpha}_{ab,c}^{(\mu)}=\tilde{\alpha}_{ab,c}^{(\mu)}(\Omega)$ is a perturbative correction to $\tilde{\alpha}_{ab}$ due to $\mathbf{E}_0$ and $\tilde{\beta}_{abc}=\tilde{\beta}_{abc}(\Omega+\Omega^\prime;\Omega,\Omega^\prime)$ is the molecule's complex vibronic hyperpolarisability tensor. Substituting Eqs. (\ref{mu}) and (\ref{complexmu}) into Eq. (\ref{Fbar1}) then taking an average over one optical period, we find that the cycle-averaged force $\overline{\mathbf{f}}$ exerted by the electromagnetic field on the molecule is
\begin{align}
\overline{\mathbf{f}}&=\frac{\omega}{2\pi}\int_0^{2\pi/\omega}\mathbf{f}\,\mathrm{d}t \nonumber \\
&\approx\frac{1}{2}\Re\{[\tilde{\alpha}_{ab}(\omega)+\tilde{\alpha}^{(\mu)}_{ab,c}(\omega)E_{0c}]\tilde{E}_b(\mathbf{R}_0)\boldsymbol{\nabla}\tilde{E}^\ast_a(\mathbf{R}_0)\}. \label{Fbar2}
\end{align}
Note that $\tilde{\beta}_{abc}$ contributes nothing to $\overline{\mathbf{f}}$. In general, $\tilde{\alpha}_{ab}$ and $\tilde{\alpha}_{ab,c}^{(\mu)}$ can be partitioned as
\begin{align}
\tilde{\alpha}_{ab}&=\alpha_{ab}(f)+\mathrm{i}\alpha_{ab}(g)-\mathrm{i}[\alpha^\prime_{ab}(f)+\alpha^\prime_{ab}(g)] \label{alpha} \\ 
\tilde{\alpha}^{(\mu)}_{ab,c}&=\alpha^{(\mu)}_{ab,c}(f)+\mathrm{i}\alpha^{(\mu)}_{ab,c}(g)-\mathrm{i}[\alpha^{\prime(\mu)}_{ab,c}(f)+\alpha^{\prime(\mu)}_{ab,c}(g)], \label{alphamu}
\end{align}
where $\alpha_{ab}(f)=\alpha_{ba}(f)$ and $\alpha^{(\mu)}_{ab,c}(f)=\alpha^{(\mu)}_{ba,c}(f)$ are the time-even, dispersive contributions; $\alpha_{ab}(g)=\alpha_{ba}(g)$ and $\alpha^{(\mu)}_{ab,c}(g)=\alpha^{(\mu)}_{ba,c}(g)$ are the time-even, absorptive contributions; $\alpha^\prime_{ab}(f)=-\alpha^\prime_{ba}(f)$ and $\alpha^{\prime(\mu)}_{ab,c}(f)=-\alpha^{\prime(\mu)}_{ba,c}(f)$ are the time-odd, dispersive contributions; $\alpha^\prime_{ab}(g)=-\alpha^\prime_{ba}(g)$ and $\alpha^{\prime(\mu)}_{ab,c}(g)=-\alpha^{\prime(\mu)}_{ba,c}(g)$ are the time-odd, dispersive contributions; $f=f_\Omega$ indicates a dispersive lineshape, and $g=g_\Omega$ indicates an absorptive lineshape \cite{Barron04a}. Substituting Eqs. (\Ref{StaticE}), (\ref{ComplexE}), (\ref{alpha}) and (\ref{alphamu}) into Eq. (\ref{Fbar2}), we obtain
\begin{align}
\overline{\mathbf{f}}&\approx\overline{\mathbf{f}}(f_\omega)+\overline{\mathbf{f}}(g_\omega)+\overline{\mathbf{f}}^\prime(f_\omega)+\overline{\mathbf{f}}^\prime(g_\omega) \label{Fbar3}
\end{align}
with
\begin{align}
\overline{\mathbf{f}}(f_\omega)&\approx k[\alpha_{yx}(f_\omega)+\alpha^{(\mu)}_{yx,z}(f_\omega)E_{0z}]E_{0y}E_{0x}\cos(2kZ_0)\hat{\mathbf{z}}, \label{Ff} \\
\overline{\mathbf{f}}(g_\omega)&\approx\frac{1}{2}k\{[\alpha_{yy}(g_\omega)+\alpha^{(\mu)}_{yy,z}(g_\omega)E_{0z}]E^2_{0y}-[\alpha_{xx}(g_\omega)+\alpha_{xx,z}^{(\mu)}(g_\omega)E_{0z}]E^2_{0x}\}\hat{\mathbf{z}}, \label{Fg} \\
\overline{\mathbf{f}}^\prime(f_\omega)&\approx-k[\alpha^\prime_{yx}(f_\omega)+\alpha_{yx,z}^{\prime(\mu)}(f_\omega)E_{0z}]E_{0y}E_{0x}\sin(2kZ_0)\hat{\mathbf{z}} \label{FPrimef} \\
\overline{\mathbf{f}}^\prime(g_\omega)&\approx0, \label{FPrimeg}
\end{align}
where $\overline{\mathbf{f}}(f_\omega)$ is the time-even, dispersive contribution; $\overline{\mathbf{f}}(g_\omega)$ is the time-even, absorptive contribution; $\overline{\mathbf{f}}^\prime(f_\omega)$ is the time-odd, dispersive contribution, and $\overline{\mathbf{f}}^\prime(g_\omega)$ is the time-odd, absorptive contribution. As the standing wave is far off-resonance, we neglect the absorptive contributions. As the molecule is diamagnetic and thus time-even, we neglect the time-odd contributions. Thus, Eqs. (\ref{Fbar3})-(\ref{FPrimeg}) reduce to
\begin{align}
\overline{\mathbf{f}}&\approx k[\alpha_{yx}(f_\omega)+\alpha^{(\mu)}_{yx,z}(f_\omega)E_{0z}]E_{0y}E_{0x}\cos(2kZ_0)\hat{\mathbf{z}}. \label{Fbar4}
\end{align}
As the rotational motion of a molecule is typically slow relative to vibronic and optical frequencies, we take Eq. (\ref{Fbar4}) to hold even with the molecule's orientation unfrozen.

% ROTATIONAL GENERAL
Let us now focus on the molecule's rotational (and in principle nuclear-spin) degrees of freedom, using a quantum-mechanical description. We take the molecule's rotational state $|\psi\rangle=|\psi(t)\rangle$ to be governed by the Schr\"{o}dinger equation \cite{Gasiorowicz03a}
\begin{align}
\mathrm{i}\hbar\frac{\mathrm{d}|\psi\rangle}{\mathrm{d}t}&=(H^{(0)}+V)|\psi\rangle, \nonumber
\end{align}
where $H^{(0)}$ describes the molecule's rotational degrees of freedom in the absence of the electric field $\mathbf{E}$ and $V$ describes perturbation by $\mathbf{E}$. Again, we are considering only electric-dipole interactions explicitly and we seek an expression for our chiral optical force valid to third order in $\mathbf{E}$. Thus, we take
\begin{align}
V&\approx -\boldsymbol{\mu}_0\cdot\mathbf{E}_0 \label{V}
\end{align}
and consider $V$ to first order. Suppose that the molecule in the absence of $\mathbf{E}$ has orthonormal rotational energy eigenstates $|r\rangle^{(0)}$ and associated energy eigenvalues $E^{(0)}_r$, satisfying
\begin{align}
H^{(0)}|r^{(0)}\rangle&=E^{(0)}_r|r^{(0)}\rangle. \nonumber
\end{align}
According to basic perturbation theory \cite{Gasiorowicz03a}, the molecule in the presence of $\mathbf{E}$ has rotational energy eigenstates $|r\rangle$ and associated energy eigenvalues $E_r$ given by
\begin{align}
|r\rangle&\approx|r\rangle^{(0)}+\sum_{q\ne r}\frac{{}^{(0)}\langle q|V|r\rangle^{(0)}}{E_r^{(0)}-E_q^{(0)}}|q\rangle^{(0)} \label{PerturbedState} \\
E_r&\approx E_r^{(0)}+{}^{(0)}\langle r|V|r\rangle^{(0)}, \label{PerturbedEnergy}
\end{align}
assuming that $V$ is diagonal in degenerate blocks. In general, $|\psi\rangle$ can be expanded in terms of the $|r\rangle$ and the $E_r$ as
\begin{align}
|\psi \rangle&=\sum_r\tilde{a}_r \mathrm{e}^{-\mathrm{i} E_r t/\hbar} |r\rangle, \label{RotationalState}
\end{align}
where the $\tilde{a}_r$ are probability amplitudes. Taking the expectation value of Eq. (\ref{Fbar4}) with respect to Eq. (\ref{RotationalState}) using Eqs. (\ref{V})-(\ref{PerturbedEnergy}), we find that the cycle-averaged, rotationally averaged force $\langle\overline{\mathbf{f}}\rangle=\langle\overline{\mathbf{f}}\rangle_\psi$ exerted by the electromagnetic field on the molecule is
\begin{align}
\langle\overline{\mathbf{f}}\rangle&=\langle \psi|\overline{\mathbf{f}}|\psi\rangle \nonumber \\
&\approx k\langle\alpha_{yx}(f_\omega)+\alpha_{yx,z}^{(\mu)}(f_\omega)E_{0z}\rangle E_{0y}E_{0x}\cos(2kZ_0)\hat{\mathbf{z}} \label{ExpectationForce} 
\end{align}
with
\begin{align}
\langle \alpha_{yx}(f_\omega)+\alpha_{yx,z}^{(\mu)}(f_\omega)E_{0z}\rangle&\approx\sum_r\sum_s\tilde{a}^\ast_r\tilde{a}_s\mathrm{e}^{\mathrm{i}(E_r-E_s)t/\hbar}\Bigg\{{}^{(0)}\langle r|\alpha_{yx}(f_\omega)|s\rangle^{(0)}  \nonumber \\
&-\Bigg[\sum_{q\ne r}\frac{{}^{(0)}\langle r|\mu_{0z}|q\rangle^{(0)}{}^{(0)}\langle q|\alpha_{yx}(f_\omega)|s\rangle^{(0)}}{E_r^{(0)}-E_q^{(0)}} \nonumber \\
&+\sum_{q\ne s}\frac{{}^{(0)}\langle r|\alpha_{yx}(f_\omega)|q\rangle^{(0)}{}^{(0)}\langle q|\mu_{0z}|s\rangle^{(0)}}{E_s^{(0)}-E_q^{(0)}}\Bigg]E_{0z} \nonumber \\
&+{}^{(0)}\langle r|\alpha_{yx,z}^{(\mu)}(f_\omega)|s\rangle^{(0)} E_{0z}\Bigg\}. \label{ExpectationExpression}
\end{align}
To proceed further, we need an explicit model of the molecule's rotational degrees of freedom. 

% ROTATIONAL SPECIFIC
As the molecule's nuclear spins are $0$ or $1/2$, we take the static magnetic field $\mathbf{B}_0$ (considered to zeroth order) to decouple the molecule's rotational angular momentum from the nuclear spins\footnote{A molecule's rotational angular momentum can strongly couple with nuclear spins of $1$ or higher, via electric-quadrupole interactions.} whilst defining a quantisation axis via the rotational Zeeman effect. As the molecule is small, chiral and in its vibronic ground state, we treat it simply as an asymmetric rigid rotor quantised along the $y$ axis (with its nuclear spins neglected), as described in Appx. \ref{Asymmetric rigid rotor}. We take the molecule to occupy a perturbed rotational state $|\psi_{J_\tau,m}\rangle=|\psi_{J_\tau,m}(t)\rangle$ of the form
\begin{align}
|\psi_{J_\tau,m}\rangle&=\mathrm{e}^{-\mathrm{i}E_{J_\tau,m}t/\hbar}|J_{\tau},m\rangle\label{PsiRotor}
\end{align}
and consider
\begin{align}
\mu_{0a}&=\ell_{aA}\mu_{0A}, \label{MuRotation} \\
\alpha_{ab}(f_\omega)&=\ell_{aA}\ell_{bB}\alpha_{AB}(f_\omega) \label{AlphaRotation} \\
\alpha_{ab,c}^{(\mu)}(f_\omega)&=\ell_{aA}\ell_{bB}\ell_{cC}\alpha_{AB,C}^{(\mu)}(f_\omega), \label{AlphaMuRotation}
\end{align}
where the $\ell_{aA}=\ell_{Aa}$ are direction cosines (see Appx. \ref{Asymmetric rigid rotor}).
Substituting Eqs. (\ref{PsiRotor})-(\ref{AlphaMuRotation}) into Eqs. (\ref{ExpectationForce}) and (\ref{ExpectationExpression}) then making use of basic symmetry arguments, we obtain
\begin{align}
\langle\overline{\mathbf{f}}\rangle&\approx kE_{0z}E_{0y}E_{0x}[\mathtt{A}\alpha_{ZY}(f_\omega)\mu_{0X}+\mathtt{B}\alpha_{XZ}(f_\omega)\mu_{0Y}+\mathtt{C}\alpha_{YX}(f_\omega)\mu_{0Z} \nonumber \\
&+\mathtt{a}\alpha_{ZY,X}^{(\mu)}(f_\omega)+\mathtt{b}\alpha_{XZ,Y}^{(\mu)}(f_\omega)+\mathtt{c}\alpha_{YX,Z}^{(\mu)}(f_\omega)]\cos(2kZ_0)\hat{\mathbf{z}}, \label{ChiralOpticalForceFull}
\end{align}
where $\mathtt{A}=\mathtt{A}_{J_\tau,m}$, $\mathtt{B}=\mathtt{B}_{J_\tau,m}$, $\mathtt{C}=\mathtt{C}_{J_\tau,m}$, $\mathtt{a}=\mathtt{a}_{J_\tau,m}$, $\mathtt{b}=\mathtt{b}_{J_\tau,m}$ and $\mathtt{c}=\mathtt{c}_{J_\tau,m}$ are coefficients with values that depend on which perturbed rotational state the molecule occupies, as described in Appx. \ref{Coefficients}. The molecular properties $\alpha_{ZY}(f_\omega)\mu_{0X}$, $\alpha_{XZ}(f_\omega)\mu_{0Y}$, $\alpha_{YX}(f_\omega)\mu_{0Z}$, $\alpha_{ZY,X}^{(\mu)}(f_\omega)$, $\alpha_{XZ,Y}^{(\mu)}(f_\omega)$ and $\alpha_{YX,Z}^{(\mu)}(f_\omega)$ are chirally sensitive, having opposite signs for opposite enantiomers. As the vibronic excitations of a molecule are typically more energetic than the rotational excitations, the $\mathtt{A}$, $\mathtt{B}$ and $\mathtt{C}$ contributions will typically be larger than the $\mathtt{a}$, $\mathtt{b}$ and $\mathtt{c}$ contributions. Neglecting the $\mathtt{a}$, $\mathtt{b}$ and $\mathtt{c}$ contributions in Eq. (\ref{ChiralOpticalForceFull}) then simplifying our notation by writing $\langle\overline{\mathbf{f}}\rangle\rightarrow \mathbf{F}=\mathbf{F}_{J_\tau,m}$ and $\alpha_{AB}(f_\omega)\rightarrow\alpha_{AB}$, we obtain our chiral optical force
\begin{align}
\mathbf{F}&\approx k E_{0z} E_{0y}E_{0x}(\mathtt{A}\alpha_{ZY}\mu_{0X}+\mathtt{B}\alpha_{XZ}\mu_{0Y}+\mathtt{C}\alpha_{YX}\mu_{0Z})\cos(2kZ_0)\hat{\mathbf{z}}, \label{ChiralOpticalForceReduced}
\end{align}
as claimed. As the translational motion of a molecule is typically slow relative to rotational, vibronic and optical frequencies, we take Eqs. (\ref{ChiralOpticalForceFull}) and (\ref{ChiralOpticalForceReduced}) to hold even with the molecule's centre-of-mass unfrozen.

% CHIRALLY SENSITIVE
Our chiral optical force depends on the chirality of the electric field $\mathbf{E}$ through the product $E_{0z}E_{0y}E_{0x}$ and on the chirality of the molecule through the molecular properties $\alpha_{ZY}\mu_{0X}$, $\alpha_{XZ}\mu_{0Y}$ and $\alpha_{YX}\mu_{0Z}$, as illustrated in Fig. \ref{Enantioselectivity}. To distinguish it from other forces proposed to date \cite{Canaguier-Durand13a, Cameron14b, Zhao17a, Kravets19a, Marichez19a, Genet22a}, our chiral optical force might be referred to as a ``chiral optical force from $E_{0z}E_{0y}E_{0x}$'' or ``COFFEEE''.

% ENANTIOSELECTIVITY
\begin{figure}[h!]
\centering
\includegraphics[width=0.75\textwidth]{Enantioselectivity}
\caption{\small For a given configuration of the electric field, our chiral optical force points in opposite directions for opposite enantiomers; compare a) with b) and c) with d). For a given enantiomer, our chiral optical force points in opposite directions depending on whether the electric field is in left-handed or a right-handed configuration; compare a) with c) and b) with d). YES BUT NOTE THAT THIS IS FOR A SINGLE ROTATIONAL STATE!}
\label{Enantioselectivity}
\end{figure}

% WHAT DO WE DO MORE GENERALLY?
Let us emphasise here that the validity of Eqs. (\ref{ChiralOpticalForceFull}) and (\ref{ChiralOpticalForceReduced}) rests on our consideration of the molecule's rotational angular momentum independent of the molecule's (neglected) nuclear spins, the quantisation of this angular momentum along the $y$ axis \footnote{Quantisation along the $x$ axis instead sees our chiral optical force change sign. Quantisation along the $z$ axis sees our chiral optical force vanish.}, and the tacit assumption that it is appropriate to use a perturbative treatment of the molecule's rotational degrees of freedom in which the static electric field $\mathbf{E}_0$ provides the dominant perturbation ($H^{(0)}\gg|-\boldsymbol{\mu}_0\cdot\mathbf{E}_0|\gg|-[\alpha_{ab}(f_\omega)+\alpha^{(\mu)}_{ab,c}(f_\omega)E_{0c}]\tilde{E}_a\tilde{E}_b^\ast/4|$), with no near degeneracies of importance. In a more general and accurate non-perturbative treatment, one should numerically diagonalise the Hamiltonian $H^{(0)}+V$ with the orientational effects of the static magnetic field $\mathbf{B}_0$ and the molecule's nuclear spins included explicitly in $H^{(0)}$ and the orientational effects of not only $\mathbf{E}_0$ but also the standing wave included in $V$.
% THINK VERY, VERY CAREFULLY ABOUT THIS.
% What about electric quadrupole though?!?!

% OTHER ELECTRIC DIPOLE EFFECTS
Our chiral optical force is the latest example of a chiroptical phenomenon that is based, unusually \cite{Barron04a}, on electric-dipole interactions. Others include second-order nonlinear sum- and difference-frequency generation \cite{Giordmaine65a, Rentzepis66a}, photoelectron circular dichroism \cite{Ritchie76a, Bowering01a}, second-harmonic generation circular dichroism from chiral surfaces \cite{Petralli-Mallow93a, Byers94a}, Rayleigh and Raman optical activity from chiral surfaces \cite{Hecht94a}, Coulomb-explosion imaging \cite{Kitamura01a, Pitzer13a, Herwig13a}, chiral microwave three-wave mixing \cite{Hirota12a, Nafie13a, Patterson13a} and photoexcitation circular dichroism and photoexcitation-induced electron circular dichroism \cite{Beaulieu18a}.
% Also perhaps other things cited in the chiral microwave three wave Nature paper, other nonlinear OR etc, the Imperial guy?
% On the hyper-Rayleigh wikipedia page there are lo ads and loads of second-order nonlinear things described.
% Dr Andrés Ordoñez, ICFO Barcelona
% The last decade has brought with it several new effects in the broad field of chiral spectroscopy. A common aspect among these effects, and a fundamental difference with respect to standard chiral effects like circular dichroism, is that they occur within the electric-dipole approximation, and are thus rather strong (zero-th order effects). In this talk, I will discuss two novel electric-dipole chiral effects which rely on recent advances in electron detection capabilities and defy our intuition of chiral light matter interactions. In the first part of the talk, I will show how, when a chiral molecule is ionized by an intense few-cycle linearly polarized IR pulse, its chirality is imprinted on the phase structure of the photoelectron wavepacket, which can be retrieved by measuring the orbital angular momentum of the free electron. In the second part of the talk, I will show that, ionization with bichromatic fields yields an octupolar contribution in the photoelectron angular distribution that changes signs upon reversal of the molecular chirality but does not change sign upon reversal of the instantaneous ellipticity.

% TRANSLATIONAL DEGREES OF FREEDOM
% What does it mean for translation to be slow relative to rotational and optical frequencies, you fool?

% DOES THE LEADING ORDER PIECE OF THE FORCE VANISH ONLY BECAUSE WE'VE CHOSEN QUANTISATION ALONG THE x (OR y OR z) AXIS? Do this require a 'note'.

% It seems we want to choose $\Delta$ throughout this paper such that $-\sin(2kZ_0+\Delta)=\cos(2kZ_0)$. We take $\Delta=-\pi/2$.

% THERE ARE NO ROTATIONAL SUBSCRIPTS HERE.
% Explicit expressions for the polarisabilities?
% There expressions cannot be final. Be careful with approximation signs please.
% ABOVE, WE SAY THE OPTICAL FIELD IS INTENSE BUT HERE ITS STRONG ELECTRIC FIELDS?

% SOMETHING NEEDS TO BE SAID ABOUT HOW ALL OF THIS TREATS ROTATIONAL MOTION ADIABATICALLY AND ALSO HOW ABOUT WE NEGLECT DECAY OF WHATEVER.

% A COMMENT ON SIZE
% MAYBE A COMMENT ON STRENGTH AND WAVELENGTH SCALING RELATIVE TO HELICITY? IS IT REALLY THREE TO FIVE ORDERS OF MAGNITUDE?

%%%%%%%%%%%%%%%%%%%%%%%%%%%%%%%%%%%%%%%%%%%%%%%%%%%%%%%%%%%%%%%%%%%%%%%%%%%%
%      
%%%%%%%%%%%%%%%%%%%%%%%%%%%%%%%%%%%%%%%%%%%%%%%%%%%%%%%%%%%%%%%%%%%%%%%%%%%%
%%%%%%%%%%%%%%%%%%%%%%%%%%%%%%%%%%%%%%%%%%%%%%%%%%%%%%%%%%%%%%%%%%%%%%%%%%%%
%       NUMERICAL RESULTS
%%%%%%%%%%%%%%%%%%%%%%%%%%%%%%%%%%%%%%%%%%%%%%%%%%%%%%%%%%%%%%%%%%%%%%%%%%%%
%%%%%%%%%%%%%%%%%%%%%%%%%%%%%%%%%%%%%%%%%%%%%%%%%%%%%%%%%%%%%%%%%%%%%%%%%%%%
%       
%%%%%%%%%%%%%%%%%%%%%%%%%%%%%%%%%%%%%%%%%%%%%%%%%%%%%%%%%%%%%%%%%%%%%%%%%%%%

\section{Numerical results}
\label{Numerical results}
% CHRISTMAS TREE
\begin{figure}[h!]
\centering
\includegraphics[width=0.75\textwidth]{SeychelleneChristmasTree}
\caption{\small Scatter plot of $E_{0z}E_{0y}E_{0x}(\mathtt{A}\alpha_{ZY}\mu_{0X}+\mathtt{B}\alpha_{XZ}\mu_{0Y}+\mathtt{C}\alpha_{YX}\mu_{0Z})/k_B$ for seychellene. Each circle corresponds to a rotational state with $m=0$ and each cross corresponds to a pair of rotational states with $m=\pm|m|\ne0$. \added[id=dm]{Along the x-axis of this figure we employ a linear scale between the tick marks adjacent to the zero point and a logarithmic scale everywhere else.}}
\label{SeychelleneChristmasTree}
\end{figure}

% THE FIELD
For the static electric field, we take $E_{0z}=1.00\,\mathrm{kV}\,\mathrm{mm}^{-1}$, which is achievable in vacuum. For the standing wave, we take $2\pi/k=1.064\,\mu\mathrm{m}$, which lies in the near infrared and is thus far off resonance for many molecules, and $E_{0x}=E_{0y}=1.00\,\mathrm{kV}\,\mathrm{mm}^{-1}$ corresponding to an optical intensity of $I=\epsilon_0 c E_{0y}E_{0x}$. This is considerably below the multi-photon ionisation threshold. Note that we have fixed the electric field in a right-handed configuration. 
% WHAT IS A REASONABLE VALUE FOR $|B_{0z}|$?

% YOU NEED TO SAY CLEARLY WHAT EQUATION WE'VE USED TO CALCULATE THESE THINGS
CALCULATED USING -> POINT TO THE FORCE!

% THE MOLECULAR PROPERTIES
Let us now consider seychellene, taking $A/2\pi\hbar=0.820\,\mathrm{GHz}$, $B/2\pi\hbar=0.623\,\mathrm{GHz}$, $C/2\pi\hbar=0.596\,\mathrm{GHz}$, $\alpha_{ZY}\mu_{0X}=6.02\times10^{-71}\,\mathrm{C}^3\,\mathrm{m}^3\,\mathrm{J}^{-1}$, $\alpha_{XZ}\mu_{0Y}=2.75\times10^{-71}\,\mathrm{C}^3\,\mathrm{m}^3\,\mathrm{J}^{-1}$ and $\alpha_{YX}\mu_{0Z}=4.15\times 10^{-71}\,\mathrm{C}^3\,\mathrm{m}^3\,\mathrm{J}^{-1}$. For the molecules, we calculate the equilibrium rotational constants $A$, $B$ and $C$ and the combinations $\alpha_{ZY}\mu_{0X}$, $\alpha_{XZ}\mu_{0Y}$ and $\alpha_{YX}\mu_{0Z}$ using Gaussian.
We then calculate the rotational states as described in Appx. \ref{Asymmetric rigid rotor} and the coefficients as described in Appx. \ref{Coefficients}. 

% THE FORCE SCATTER PLOT
Shown in Fig. \ref{SeychelleneChristmasTree} is a scatter plot indicating how the magnitude and sign of our chiral optical force $\mathbf{F}$ depends on the (perturbed) rotational state occupied by the molecule, for the $J\in\{0,\dots,10\}$ manifolds. The largest force is found for the perturbed $J_\tau,m=1_0,0$ state, giving $\Delta U_{1_0,0}/k_B=295\,\mathrm{K}$. For the $J\in\{1,\dots,10\}$ manifolds, there are $777$ states with $\sigma<0$ but $993$ states with $\sigma>0$ (for $J=0$, we take $\sigma=0$), giving an overall preference for $F_z>0$. CONSIDERABLY HIGHER VALUES CAN BE CONSIDERED IF USE OUR NON-PERTURBATIVE TREATMENT.
% For seychellene we probably need to increase $J_\mathrm{max}$.

% A LITTLE COMMENT ON GENERALITY
We have obtained similar results (not reported in this paper) for a handful of other molecules, including isotopically chiral molecules. 

%%%%%%%%%%%%%%%%%%%%%%%%%%%%%%%%%%%%%%%%%%%%%%%%%%%%%%%%%%%%%%%%%%%%%%%%%%%%
%      
%%%%%%%%%%%%%%%%%%%%%%%%%%%%%%%%%%%%%%%%%%%%%%%%%%%%%%%%%%%%%%%%%%%%%%%%%%%%
%%%%%%%%%%%%%%%%%%%%%%%%%%%%%%%%%%%%%%%%%%%%%%%%%%%%%%%%%%%%%%%%%%%%%%%%%%%%
%       OUTLOOK
%%%%%%%%%%%%%%%%%%%%%%%%%%%%%%%%%%%%%%%%%%%%%%%%%%%%%%%%%%%%%%%%%%%%%%%%%%%%
%%%%%%%%%%%%%%%%%%%%%%%%%%%%%%%%%%%%%%%%%%%%%%%%%%%%%%%%%%%%%%%%%%%%%%%%%%%%
%       
%%%%%%%%%%%%%%%%%%%%%%%%%%%%%%%%%%%%%%%%%%%%%%%%%%%%%%%%%%%%%%%%%%%%%%%%%%%%

\section{Outlook}
\label{Outlook}
There is much still to be done:
\begin{itemize}
\item We derived our chiral optical force using a perturbative treatment. It remains for us to develop a more general and accurate non-perturbative treatment. 
\item We have yet to fully investigate how our chiral optical force depends on different aspects of molecular structure, such as conformation and isotopic constitution.
\item There are a wealth of potential applications to be pursued for our chiral optical force, including robust enantiomer resolution schemes.
\item We speculate that other electromagnetic fields can produce chiral optical forces based on electric-dipole interactions. The key idea is the use of three mutually orthogonal electric fields, which can be arranged in either a left-handed or a right-handed configuration. These forces might be referred to as different ``blends'' of COFFEEE, the chiral optical force identified in this paper being the ``static/lin$\bot$lin blend''.
\end{itemize}
We will return to these and related tasks elsewhere.

%%%%%%%%%%%%%%%%%%%%%%%%%%%%%%%%%%%%%%%%%%%%%%%%%%%%%%%%%%%%%%%%%%%%%%%%%%%%
%      
%%%%%%%%%%%%%%%%%%%%%%%%%%%%%%%%%%%%%%%%%%%%%%%%%%%%%%%%%%%%%%%%%%%%%%%%%%%%
%%%%%%%%%%%%%%%%%%%%%%%%%%%%%%%%%%%%%%%%%%%%%%%%%%%%%%%%%%%%%%%%%%%%%%%%%%%%
%       BACKMATTER
%%%%%%%%%%%%%%%%%%%%%%%%%%%%%%%%%%%%%%%%%%%%%%%%%%%%%%%%%%%%%%%%%%%%%%%%%%%%
%%%%%%%%%%%%%%%%%%%%%%%%%%%%%%%%%%%%%%%%%%%%%%%%%%%%%%%%%%%%%%%%%%%%%%%%%%%%
%       
%%%%%%%%%%%%%%%%%%%%%%%%%%%%%%%%%%%%%%%%%%%%%%%%%%%%%%%%%%%%%%%%%%%%%%%%%%%%

\begin{backmatter}
\bmsection{Funding}
The Royal Society (URF$\backslash$R1$\backslash$191243). 

\bmsection{Acknowledgments}
Robert P. Cameron is a Royal Society University Research Fellow and gratefully acknowledges the Royal Society's support.

\bmsection{Disclosures}
The authors declare no conflicts of interest.

\bmsection{Data availability}
No data were generated or analyzed in the presented research.
% DATA AVAILABILITY STATEMENTS ARE NOT REQUIRED FOR PREPRINT SUBMISSIONS 

% \bmsection{Supplemental document}
% WE DON'T HAVE A SUPPLEMENTAL DOCUMENT

\end{backmatter}

%%%%%%%%%%%%%%%%%%%%%%%%%%%%%%%%%%%%%%%%%%%%%%%%%%%%%%%%%%%%%%%%%%%%%%%%%%%%
%      
%%%%%%%%%%%%%%%%%%%%%%%%%%%%%%%%%%%%%%%%%%%%%%%%%%%%%%%%%%%%%%%%%%%%%%%%%%%%
%%%%%%%%%%%%%%%%%%%%%%%%%%%%%%%%%%%%%%%%%%%%%%%%%%%%%%%%%%%%%%%%%%%%%%%%%%%%
%       REFERENCES
%%%%%%%%%%%%%%%%%%%%%%%%%%%%%%%%%%%%%%%%%%%%%%%%%%%%%%%%%%%%%%%%%%%%%%%%%%%%
%%%%%%%%%%%%%%%%%%%%%%%%%%%%%%%%%%%%%%%%%%%%%%%%%%%%%%%%%%%%%%%%%%%%%%%%%%%%
%       
%%%%%%%%%%%%%%%%%%%%%%%%%%%%%%%%%%%%%%%%%%%%%%%%%%%%%%%%%%%%%%%%%%%%%%%%%%%%

\begin{thebibliography}{99}

\bibitem{Marichez19a} V.~Marichez, A.~Tassoni, R.~P.~Cameron, S.~M.~Barnett, R.~Eichhorn, C.~Genet, and T.~M.~Hermans, ``Mechanical chiral resolution,'' {\protect\JournalTitle{Soft Matter}} \textbf{15}, 4593--4608 (2019).

\bibitem{Genet22a} C.~Genet, ``Chiral light-chiral matter interactions: an optical force perspective,'' {\protect\JournalTitle{ACS Photonics}} \textbf{9}, 319--332 (2022).

\bibitem{Lough02a} W.~J.~Lough and I.~W.~Wainer, \textit{Chirality in Natural and Applied Science} (Blackwell Publishing, 2002).

\bibitem{Gardner05a} M.~Gardner, \textit{The New Ambidextrous Universe} (Dover, 2005).

\bibitem{Canaguier-Durand13a} A.~Canaguier-Durand, J.~A.~Hutchison,
C.~Genet, and T.~W.~Ebbesen, ``Mechanical separation of chiral dipoles by chiral light,'' {\protect\JournalTitle{New J. Phys.}} \textbf{15}, 123037 (2014).

\bibitem{Cameron14b} R.~P.~Cameron, S.~M.~Barnett, and A.~M.~Yao, ``Discriminatory optical force for chiral molecules,'' {\protect\JournalTitle{New J. Phys.}} \textbf{16}, 013020 (2014).

\bibitem{Zhao17a} Y.~Zhao, A.~A.~Saleh, M.~A.~van de Haar, B.~Baum, J.~A.~Briggs, A.~Lay, O.~A.~Reyes-Becerra, and J.~A.~Dionne, ``Nanoscopic control and quantification of enantioselective optical forces,'' {\protect\JournalTitle{Nat. Nanotechnol.}}
\textbf{12}, 1055--1059 (2017).

\bibitem{Kravets19a} N.~Kravets, A.~Aleksanyan, and E.~Brasselet, ``Chiral optical Stern-Gerlach Newtonian experiment,'' {\protect\JournalTitle{Phys. Rev. Lett.}} \textbf{122}, 024301 (2019).

\bibitem{Curie94a} P.~Curie, ``Sur la sym\'{e}trie dans les ph\'{e}nom\`{e}nes physiques, sym\'{e}trie d'un champ \'{e}lectrique et d'un champ magn\'{e}tique,'' {\protect\JournalTitle{J. Phys. Theor. Appl.}} \textbf{3}, 393--415 (1894).
% https://jphystap.journaldephysique.org/articles/jphystap/abs/1894/01/jphystap_1894__3__393_0/jphystap_1894__3__393_0.html

\bibitem{Alexander03a} R.~M.~Alexander, \textit{Principles of Animal Locomotion} (Princeton University Press, 2003).

\bibitem{Dalibard89a} J.~Dalibard and C.~Cohen-Tannoudji, ``Laser cooling below the Doppler limit by polarization gradients: simple theoretical models,'' {\protect\JournalTitle{J. Opt. Soc. Am. B}} \textbf{6}, 2023--2045 (1989).

\bibitem{Barron04a} L.~Barron, \textit{Molecular Light Scattering and Optical Activity} (Cambridge, 2004).

\bibitem{Gasiorowicz03a} S.~Gasiorowicz, \textit{Quantum Physics} (Wiley, 2003).

\bibitem{Giordmaine65a} J.~A.~Giordmaine, ``Nonlinear optical properties of liquids,'' {\protect\JournalTitle{Phys. Rev.}} \textbf{138}, A1599--A1606 (1965).

\bibitem{Rentzepis66a} P.~M.~Rentzepis, J.~A.~Giordmaine, and K.~W.~Wecht, ``Coherent optical mixing in optically active liquids,'' {\protect\JournalTitle{Phys. Rev. Lett.}} \textbf{16}, 792--794 (1966).

\bibitem{Ritchie76a} B.~Ritchie, ``Theory of the angular distribution of photoelectrons ejected from optically active molecules and molecular negative ions,'' {\protect\JournalTitle{Phys. Rev. A}} \textbf{13}, 1411--1415 (1976).

\bibitem{Bowering01a} N.~B\"{o}wering, T.~Lischke, B.~Schmidtke, N.~M\"{u}ller, T.~Khalil, and U.~Heinzmann, ``Asymmetry in photoelectron emission from chiral molecules induced by circularly polarized light,'' {\protect\JournalTitle{Phys. Rev. Lett.}} \textbf{86}, 1187--1190 (2001).

\bibitem{Petralli-Mallow93a} T.~Petralli-Mallow, T.~M.~Wong, J.~D.~Byers, H.~I.~Yee, and J.~M.~Hicks, ``Circular dichroism spectroscopy at interfaces: a surface second harmonic generation study,'' {\protect\JournalTitle{J. Phys. Chem.}} \textbf{97}, 1383--1388 (1993).

\bibitem{Byers94a} J.~D.~Byers, H.~I.~Yee, T.~Petralli-Mallow, and J.~M.~Hicks, ``Second-harmonic generation circular-dichroism spectroscopy from chiral monolayers,'' {\protect\JournalTitle{Phys. Rev. B}} \textbf{49}, 14643--14647 (1994).

\bibitem{Hecht94a} L.~Hecht and L.~D.~Barron, ``Rayleigh and Raman optical activity from chiral surfaces,'' {\protect\JournalTitle{Chem. Phys. Lett.}} \textbf{225}, 525--530 (1994).

\bibitem{Kitamura01a} T.~Kitamura, T.~Nishide, H.~Shiromaru, Y.~Achiba, and N.~Kobayashi, ``Direct observation of ``dynamic'' chirality by Coulomb explosion imaging,'' {\protect\JournalTitle{J. Chem. Phys.}} \textbf{115}, 5--6 (2001).

\bibitem{Pitzer13a} M.~Pitzer et al., ``Direct determination of absolute molecular stereochemistry in gas phase by Coulomb explosion imaging,'' {\protect\JournalTitle{Science}} \textbf{341}, 1096--1100 (2013).

\bibitem{Herwig13a} P.~Herwig et al., ``Imaging the absolute configuration of a chiral epoxide in the gas phase,'' {\protect\JournalTitle{Science}} \textbf{342}, 1084--1086 (2013).

\bibitem{Hirota12a} E.~Hirota, ``Triple resonance for a three-level system of a chiral molecule,'' {\protect\JournalTitle{Proc. Jpn. Acad., Ser. B}} \textbf{88}, 120--128 (2012).

\bibitem{Nafie13a} L.~Nafie, ``Handedness detected by microwaves,'' {\protect\JournalTitle{Nature}} \textbf{497}, 446--448 (2013).

\bibitem{Patterson13a} D.~Patterson, M.~Schnell, and J.~M.~Doyle, ``Enantiomer-specific detection of chiral molecules via microwave spectroscopy,'' {\protect\JournalTitle{Nature}} \textbf{497}, 475--477 (2013).

\bibitem{Beaulieu18a} S.~Beaulieu et al., ``Photoexcitation circular dichroism in chiral molecules,'' {\protect\JournalTitle{Nat. Phys.}} \textbf{14}, 484--489 (2018).

% THE REFERENCES BELOW ONLY APPEAR IN THE APPENDICES
\bibitem{Bunker05a} P.~R.~Bunker and P.~Jensen, \textit{Fundamentals of Molecular Symmetry} (Institute of Physics Publishing, 2005).

\bibitem{Wang29a} S.~C.~Wang, ``On the asymmetrical top in quantum mechanics,'' {\protect\JournalTitle{Phys. Rev.}} \textbf{34}, 243--252 (1929).

\bibitem{Bernath05a} P.~F.~Bernath, \textit{Spectra of Atoms and Molecules} (Oxford University Press, 2005).

\bibitem{vanVleck32a} J.~H.~van Vleck, \textit{The Theory of Electric and Magnetic
Susceptibilities} (Oxford University Press, 1932).

\end{thebibliography}

%%%%%%%%%%%%%%%%%%%%%%%%%%%%%%%%%%%%%%%%%%%%%%%%%%%%%%%%%%%%%%%%%%%%%%%%%%%%
%      
%%%%%%%%%%%%%%%%%%%%%%%%%%%%%%%%%%%%%%%%%%%%%%%%%%%%%%%%%%%%%%%%%%%%%%%%%%%%
%%%%%%%%%%%%%%%%%%%%%%%%%%%%%%%%%%%%%%%%%%%%%%%%%%%%%%%%%%%%%%%%%%%%%%%%%%%%
%       APPENDICES
%%%%%%%%%%%%%%%%%%%%%%%%%%%%%%%%%%%%%%%%%%%%%%%%%%%%%%%%%%%%%%%%%%%%%%%%%%%%
%%%%%%%%%%%%%%%%%%%%%%%%%%%%%%%%%%%%%%%%%%%%%%%%%%%%%%%%%%%%%%%%%%%%%%%%%%%%
%       
%%%%%%%%%%%%%%%%%%%%%%%%%%%%%%%%%%%%%%%%%%%%%%%%%%%%%%%%%%%%%%%%%%%%%%%%%%%%

\begin{appendix}

%%%%%%%%%%%%%%%%%%%%%%%%%%%%%%%%%%%%%%%%%%%%%%%%%%%%%%%%%%%%%%%%%%%%%%%%%%%%
%      
%%%%%%%%%%%%%%%%%%%%%%%%%%%%%%%%%%%%%%%%%%%%%%%%%%%%%%%%%%%%%%%%%%%%%%%%%%%%
%%%%%%%%%%%%%%%%%%%%%%%%%%%%%%%%%%%%%%%%%%%%%%%%%%%%%%%%%%%%%%%%%%%%%%%%%%%%
%       ASYMMETRIC RIGID ROTOR
%%%%%%%%%%%%%%%%%%%%%%%%%%%%%%%%%%%%%%%%%%%%%%%%%%%%%%%%%%%%%%%%%%%%%%%%%%%%
%%%%%%%%%%%%%%%%%%%%%%%%%%%%%%%%%%%%%%%%%%%%%%%%%%%%%%%%%%%%%%%%%%%%%%%%%%%%
%       
%%%%%%%%%%%%%%%%%%%%%%%%%%%%%%%%%%%%%%%%%%%%%%%%%%%%%%%%%%%%%%%%%%%%%%%%%%%%

\section{Asymmetric rigid rotor}
\label{Asymmetric rigid rotor}
% OPENING SENTENCE
The quantum-mechanical description of the asymmetric rigid rotor has been covered extensively elsewhere \cite{Wang29a, Bernath05a, Bunker05a}. We use the conventions below, which are slightly unusual as we consider quantisation along the $y$ axis.

% AXES ORIENTATION
We take the molecule-fixed coordinates $X$, $Y$ and $Z$ to be arranged in a $\mathrm{III}^r$ configuration with direction cosines $\ell_{Aa}$ given by
\begin{align}
\ell_{Xx}&=-\cos\theta\cos\phi\cos\chi+\sin\phi\sin\chi, \nonumber \\
\ell_{Xy}&=-\sin\theta\cos\chi, \nonumber \\
\ell_{Xz}&=\cos\theta\sin\phi\cos\chi+\cos\phi\sin\chi, \nonumber \\
\ell_{Yx}&=\cos\theta\cos\phi\sin\chi+\sin\phi\cos\chi, \nonumber \\
\ell_{Yy}&=\sin\theta\sin\chi, \nonumber \\
\ell_{Yz}&=-\cos\theta\sin\phi\sin\chi+\cos\phi\cos\chi, \nonumber \\
\ell_{Zx}&=-\sin\theta\cos\phi, \nonumber \\
\ell_{Zy}&=\cos\theta \nonumber \\
\ell_{Zz}&=\sin\theta\sin\phi, \nonumber
\end{align}
where $0\le\theta\le\pi$, $0\le\phi<2\pi$ and $0\le\chi<2\pi$ are Euler angles relating $X$, $Y$ and $Z$ to the laboratory-fixed coordinates $x$, $y$ and $z$. Thus, the unperturbed Hamiltonian $H^{(0)}$ is
\begin{align}
H^{(0)}&=\frac{1}{\hbar^2}(A J^2_X+B J^2_Y+C J^2_Z) \nonumber 
\end{align}
with
\begin{align}
J_X&=-\mathrm{i}\hbar\left(\sin\chi\frac{\partial}{\partial\theta}-\csc\theta\cos\chi\frac{\partial}{\partial\phi}+\cot\theta\cos\chi\frac{\partial}{\partial\chi}\right), \nonumber \\
J_Y&=-\mathrm{i}\hbar\left(\cos\chi\frac{\partial}{\partial\theta}+\csc\theta\sin\chi\frac{\partial}{\partial\phi}-\cot\theta\sin\chi\frac{\partial}{\partial\chi}\right) \nonumber \\
J_Z&=-\mathrm{i}\hbar\frac{\partial}{\partial\chi}, \nonumber
\end{align}
where $A>B>C$ are the molecule's equilibrium rotational constants and $\mathbf{J}$ is the molecule's rotational angular momentum.

% QUANTISED ALONG $X$, BASICALLY
We take the unperturbed rotational energy eigenstates $|J_{\tau},m\rangle^{(0)}$ and associated energy eigenvalues $E_{J_\tau,m}^{(0)}$ to satisfy
\begin{align}
H^{(0)}|J_\tau,m\rangle^{(0)}&=E_{J_\tau,m}^{(0)}|J_\tau,m\rangle^{(0)}, \nonumber \\
\left(J_X^2+J_Y^2+J_Z^2\right)|J_\tau,m\rangle^{(0)}&=\hbar^2J(J+1)|J_\tau,m\rangle^{(0)} \nonumber \\
J_y|J_\tau,m\rangle^{(0)}&=\hbar m|J_\tau,m\rangle^{(0)} \nonumber
\end{align}
with
\begin{align}
J_x&=-\mathrm{i}\hbar\left(\sin\phi\frac{\partial}{\partial\theta}+\cot\theta\cos\phi\frac{\partial}{\partial\phi}-\csc\theta\cos\phi\frac{\partial}{\partial\chi}\right), \nonumber \\
J_y&=-\mathrm{i}\hbar\frac{\partial}{\partial\phi} \nonumber \\
J_z&=-\mathrm{i}\hbar\left(\cos\phi\frac{\partial}{\partial\theta}-\cot\theta\sin\phi\frac{\partial}{\partial\phi}+\csc\theta\sin\phi\frac{\partial}{\partial\chi}\right), \nonumber
\end{align}
where $J\in\{0,1,\dots\}$ is the rotational angular momentum quantum number, $\tau\in\{-J,\dots,J\}$ is a label that increases with increasing energy and $m\in\{-J,\dots,J\}$ is the laboratory-fixed rotational angular momentum projection quantum number.

% DIAGONALISING THE HAMILTONIAN
To help us identify the unperturbed rotational energy eigenstates $|J_{\tau},m\rangle^{(0)}$ and associated energy eigenvalues $E_{J_\tau,m}^{(0)}$ explicitly, we work in a basis of unperturbed symmetric rigid rotor states $|J,k,m\rangle^{(0)}$ satisfying
\begin{align}
(J_X^2+J_Y^2+J_Z^2)|J,k,m\rangle^{(0)}&=\hbar^2 J(J+1)|J,k,m\rangle^{(0)}, \nonumber \\
J_Z|J,k,m\rangle^{(0)}&=\hbar k|J,k,m\rangle^{(0)} \nonumber \\
J_y|J,k,m\rangle^{(0)}&=\hbar m|J,k,m\rangle^{(0)} \nonumber
\end{align}
with rotational wavefunctions $\langle\theta,\phi,\chi|J,k,m\rangle^{(0)}$ given by
\begin{align}
\langle \theta,\phi,\chi|J,k,m\rangle^{(0)}&=\sqrt{\frac{(J+m)!(J-m)!(J+k)!(J-k)!(2J+1)}{8\pi^2}} \nonumber \\
&\sum_{\sigma=\mathrm{max}(0,k-m)}^{\mathrm{min}(J-m,J+k)}(-1)^\sigma\frac{(\cos\frac{1}{2}\theta)^{2J+k-m-2\sigma}(-\sin\frac{1}{2}\theta)^{m-k+2\sigma}}{\sigma!(J-m-\sigma)!(m-k+\sigma)!(J+k-\sigma)!}\mathrm{e}^{\mathrm{i}m\phi}\mathrm{e}^{\mathrm{i}k\chi}, \nonumber
\end{align}
where $k\in\{-J,\dots,J\}$ is the molecule-fixed rotational angular momentum projection quantum number. The $|J,k,m\rangle^{(0)}$ render the unperturbed rotational Hamiltonian $H^{(0)}$ block diagonal in the quantum number $J$, as
\begin{align}
\langle J^\prime,k^\prime,m^\prime|H^0|J,k,m\rangle&=\delta_{J^\prime J}\delta_{m^\prime m}\Bigg(\frac{1}{2}(A+B)J(J+1)\delta_{k^\prime k}+\left[C-\frac{1}{2}(A+B)\right]k^2\delta_{k^\prime k} \nonumber \\
&+\frac{1}{4}(A-B)\{\sqrt{[J(J+1)-(k-1)(k-2)][J(J+1)-k(k-1)]}\delta_{k^\prime k-2} \nonumber \\
&+\sqrt{[J(J+1)-(k+1)(k+2)][J(J+1)-k(k+1)]}\delta_{k^\prime k+2}\}\Bigg). \nonumber
\end{align}
For $J\in\{0,\dots,5\}$, the diagonalisation of $H^{(0)}$ can completed analytically, giving
\begin{align}
|0_0,0\rangle^{(0)}&=|0,0,0\rangle^{(0)}, \nonumber \\
|1_{-1},m\rangle^{(0)}&=\frac{1}{\sqrt{2}}(|1,1,m\rangle^{(0)}-|1,-1,m\rangle^{(0)}), \nonumber \\
|1_0,m\rangle^{(0)}&=\frac{1}{\sqrt{2}}(|1,1,m\rangle^{(0)}+|1,-1,m\rangle^{(0)}) \nonumber \\
|1_1,m\rangle^{(0)}&=|1,0,m\rangle^{(0)} \nonumber
\end{align}
with
\begin{align}
E_{0_0,m}^{(0)}&=0, \nonumber \\
E_{1_{-1},m}^{(0)}&=B+C, \nonumber \\
E_{1_0,m}^{(0)}&=A+C \nonumber \\
E_{1_1,m}^{(0)}&=A+B, \nonumber 
\end{align}
for example. For $J\in\{6,\dots\}$, the diagonalisation of $H^{(0)}$ must be completed numerically, in general.

% IS THIS ORDERING OF STATES ALWAYS CORRECT WITH REGARDS TO $\tau$? Microwave spectroscopy book gives a recipe for \tau in terms of K_a and K_c.
% Going up to $J=3$ analytically would allow us to quote $\mathtt{A}$, ..., $\mathtt{F}$ analytically for $J=0$, $J=1$ and $J=2$, which is surely good enough for the sake of illustration. 
% NO FUNCTIONAL DEPENDENCES ARE INDICATED IN THIS APPENDIX?

%%%%%%%%%%%%%%%%%%%%%%%%%%%%%%%%%%%%%%%%%%%%%%%%%%%%%%%%%%%%%%%%%%%%%%%%%%%%
%      
%%%%%%%%%%%%%%%%%%%%%%%%%%%%%%%%%%%%%%%%%%%%%%%%%%%%%%%%%%%%%%%%%%%%%%%%%%%%
%%%%%%%%%%%%%%%%%%%%%%%%%%%%%%%%%%%%%%%%%%%%%%%%%%%%%%%%%%%%%%%%%%%%%%%%%%%%
%       COEFFICIENTS
%%%%%%%%%%%%%%%%%%%%%%%%%%%%%%%%%%%%%%%%%%%%%%%%%%%%%%%%%%%%%%%%%%%%%%%%%%%%
%%%%%%%%%%%%%%%%%%%%%%%%%%%%%%%%%%%%%%%%%%%%%%%%%%%%%%%%%%%%%%%%%%%%%%%%%%%%
%       
%%%%%%%%%%%%%%%%%%%%%%%%%%%%%%%%%%%%%%%%%%%%%%%%%%%%%%%%%%%%%%%%%%%%%%%%%%%%

\section{Coefficients}
\label{Coefficients}
% EXPLICIT EXPRESSIONS FOR THE COEFFICIENTS
The coefficients $\mathtt{A}$, $\mathtt{B}$, $\mathtt{C}$, $\mathtt{a}$, $\mathtt{b}$ and $\mathtt{c}$ are given by
\begin{align}
\mathtt{A}&=\sum_{J^\prime=0}^\infty\sum_{\tau^\prime=-J^\prime}^{J^\prime}\sum_{m^\prime=-J^\prime}^{J^\prime}\frac{2}{E_{J^\prime_{\tau^\prime},m^\prime}^{(0)}-E_{J_\tau,m}^{(0)}} \nonumber \\
&\times\Re[{}^{(0)}\langle J_\tau,m|(\ell_{yZ}\ell_{xY}+\ell_{yY}\ell_{xZ})|J^\prime_{\tau^\prime},m^\prime\rangle^{(0)}{}^{(0)}\langle J^\prime_{\tau^\prime},m^\prime|\ell_{zX}|J_\tau,m\rangle^{(0)}], \label{ACoefficient} \\
\mathtt{B}&=\sum_{J^\prime=0}^\infty\sum_{\tau^\prime=-J^\prime}^{J^\prime}\sum_{m^\prime=-J^\prime}^{J^\prime}\frac{2}{E_{J^\prime_{\tau^\prime},m^\prime}^{(0)}-E_{J_\tau,m}^{(0)}} \nonumber \\
&\times\Re[{}^{(0)}\langle J_\tau,m|(\ell_{yX}\ell_{xZ}+\ell_{yZ}\ell_{xX})|J^\prime_{\tau^\prime},m^\prime\rangle^{(0)}{}^{(0)}\langle J^\prime_{\tau^\prime},m^\prime|\ell_{zY}|J_\tau,m\rangle^{(0)}], \label{BCoefficient} \\
\mathtt{C}&=\sum_{J^\prime=0}^\infty\sum_{\tau^\prime=-J^\prime}^{J^\prime}\sum_{m^\prime=-J^\prime}^{J^\prime}\frac{2}{E_{J^\prime_{\tau^\prime},m^\prime}^{(0)}-E_{J_\tau,m}^{(0)}} \nonumber \\
&\times\Re[{}^{(0)}\langle J_\tau,m| 
(\ell_{yY}\ell_{xX}+\ell_{yX}\ell_{xY})|J^\prime_{\tau^\prime},m^\prime\rangle^{(0)}{}^{(0)}\langle J^\prime_{\tau^\prime},m^\prime|\ell_{zZ}|J_\tau,m\rangle^{(0)}], \label{CCoefficient} \\
\mathtt{a}&={}^{(0)}\langle J_\tau,m|\left(\ell_{yZ}\ell_{xY}\ell_{zX}+\ell_{yY}\ell_{xZ}\ell_{zX}\right)|J_\tau,m\rangle^{(0)}, \label{aCoefficient} \\
\mathtt{b}&={}^{(0)}\langle J_\tau,m|\left(\ell_{yX}\ell_{xZ}\ell_{zY}+\ell_{yZ}\ell_{xX}\ell_{zY}\right)|J_\tau,m\rangle^{(0)} \label{bCoefficient} \\
\mathtt{c}&={}^{(0)}\langle J_\tau,m|\left(\ell_{yY}\ell_{xX}\ell_{zZ}+\ell_{yX}\ell_{xY}\ell_{zZ}\right)|J_\tau,m\rangle^{(0)}, \label{cCoefficient}
\end{align}
where terms with $J^\prime_{\tau^\prime},m^\prime =J_\tau,m$ are to be excluded from the summations. For $J\in\{0,1,2,3,4\}$, (\ref{ACoefficient})-(\ref{cCoefficient}) can be evaluated in closed form, as the unperturbed rotational energy eigenstates $|J^\prime_{\tau^\prime},m^\prime\rangle^{(0)}$ and associated energy eigenvalues $E_{J^\prime_{\tau^\prime},m^\prime}^{(0)}$ of importance can be found analytically (see Appx. \ref{Asymmetric rigid rotor}). The results for $J=0$ and $J=1$ are presented in table \ref{LowCoefficients}, where it can be seen that $\mathtt{A}$, $\mathtt{B}$, $\mathtt{C}$, $\mathtt{a}$, $\mathtt{b}$ and $\mathtt{c}$ depend on the magnitude but not the sign of the quantum number $m$ and that they satisfy relationships like
\begin{align}
\sum_{m=-J}^J\mathtt{A}_{J_\tau,m}&=0, \nonumber
\end{align}
in accord with the principle of spectroscopic stability \cite{vanVleck32a}. For $J\in\{5,\dots\}$, (\ref{ACoefficient})-(\ref{cCoefficient}) cannot be evaluated in closed form, as some or all of the $|J^\prime_{\tau^\prime},m^\prime\rangle^{(0)}$ and $E_{J^\prime_{\tau^\prime},m^\prime}^{(0)}$ of importance need to be found numerically (again, see Appx. \ref{Asymmetric rigid rotor}).

% TABLE OF LOW-LYING COEFFICIENTS
\begin{table}[h!]
\begin{tabular}{c|c|c|c|c|c}
& $|\psi_{0_0,0}\rangle$ & $|\psi_{1_{-1}, m}\rangle$ & $|\psi_{1_0, m}\rangle$ & $|\psi_{1_1, m}\rangle$ \\ \hline
$\mathtt{A}$ & $0$ & $\frac{(2-3|m|)(B-C)}{5[3(B-C)^2-8A(B+C)]}$ & $\frac{(3|m|-2)}{20}\left[\frac{3}{(B-C)}+\frac{1}{(B+3C)}\right]$ & $\frac{(2-3|m|)}{20}\left[\frac{3}{(C-B)}+\frac{1}{(C+3B)}\right]$  \\
$\mathtt{B}$ & $0$ & $\frac{(2-3|m|)}{20}\left[\frac{3}{(A-C)}+\frac{1}{(A+3C)}\right]$ & $\frac{(2-3|m|)(C-A)}{5[3(C-A)^2-8B(A+C)]}$ & $\frac{(3|m|-2)}{20}\left[\frac{3}{(C-A)}+\frac{1}{(C+3A)}\right]$ \\
$\mathtt{C}$ & $0$ & $\frac{(3|m|-2)}{20}\left[\frac{3}{(A-B)}+\frac{1}{(A+3B)}\right]$ & $\frac{(2-3|m|)}{20}\left[\frac{3}{(B-A)}+\frac{1}{(B+3A)}\right]$ & $\frac{(2-3|m|)(A-B)}{5[3(A-B)^2-8C(A+B)]}$ \\
$\mathtt{a}$ & $0$ & $0$  & $\frac{(3|m|-2)}{10}$ & $\frac{(2-3|m|)}{10}$ \\
$\mathtt{b}$ & $0$ & $\frac{(2-3|m|)}{10}$ & $0$ & $\frac{(3|m|-2)}{10}$ \\
$\mathtt{c}$ & $0$ & $\frac{(3|m|-2)}{10}$ & $\frac{(2-3|m|)}{10}$ &  $0$ \\
\end{tabular}
\caption{\small The coefficients $\mathtt{A}$, $\mathtt{B}$, $\mathtt{C}$, $\mathtt{a}$, $\mathtt{b}$ and $\mathtt{c}$ for perturbed rotational states in the $J=0$ and $J=1$ manifolds.}
\label{LowCoefficients}
\end{table}
% The results in the table can probably be written in a more systematic way.

\end{appendix}
 
%%%%%%%%%%%%%%%%%%%%%%%%%%%%%%%%%%%%%%%%%%%%%%%%%%%%%%%%%%%%%%%%%%%%%%%%%%%%
%      
%%%%%%%%%%%%%%%%%%%%%%%%%%%%%%%%%%%%%%%%%%%%%%%%%%%%%%%%%%%%%%%%%%%%%%%%%%%%
%%%%%%%%%%%%%%%%%%%%%%%%%%%%%%%%%%%%%%%%%%%%%%%%%%%%%%%%%%%%%%%%%%%%%%%%%%%%
%       END OF THE DOCUMENT
%%%%%%%%%%%%%%%%%%%%%%%%%%%%%%%%%%%%%%%%%%%%%%%%%%%%%%%%%%%%%%%%%%%%%%%%%%%%
%%%%%%%%%%%%%%%%%%%%%%%%%%%%%%%%%%%%%%%%%%%%%%%%%%%%%%%%%%%%%%%%%%%%%%%%%%%%
%       
%%%%%%%%%%%%%%%%%%%%%%%%%%%%%%%%%%%%%%%%%%%%%%%%%%%%%%%%%%%%%%%%%%%%%%%%%%%%

\end{document}

%%%%%%%%%%%%%%%%%%%%%%%%%%%%%%%%%%%%%%%%%%%%%%%%%%%%%%%%%%%%%%%%%%%%%%%%%%%%
%      
%%%%%%%%%%%%%%%%%%%%%%%%%%%%%%%%%%%%%%%%%%%%%%%%%%%%%%%%%%%%%%%%%%%%%%%%%%%%
%%%%%%%%%%%%%%%%%%%%%%%%%%%%%%%%%%%%%%%%%%%%%%%%%%%%%%%%%%%%%%%%%%%%%%%%%%%%
%       THE SCRAP YARD
%%%%%%%%%%%%%%%%%%%%%%%%%%%%%%%%%%%%%%%%%%%%%%%%%%%%%%%%%%%%%%%%%%%%%%%%%%%%
%%%%%%%%%%%%%%%%%%%%%%%%%%%%%%%%%%%%%%%%%%%%%%%%%%%%%%%%%%%%%%%%%%%%%%%%%%%%
%       
%%%%%%%%%%%%%%%%%%%%%%%%%%%%%%%%%%%%%%%%%%%%%%%%%%%%%%%%%%%%%%%%%%%%%%%%%%%%

\begin{align}
\langle \alpha_{xy}(f_\omega)+\alpha_{xy,z}^{(\mu)}(f_\omega)E_{0z}\rangle&\approx\sum_r\sum_s\tilde{c}^\ast_r\tilde{c}_s\mathrm{e}^{\mathrm{i}(E_r-E_s)t/\hbar}\Bigg\{\langle r^0|\ell_{xA}\ell_{yB}|s^0\rangle \alpha_{AB}(f_\omega) \nonumber \\
&+\Bigg[\sum_{q\ne r}\frac{\langle r^0|\ell_{zC}|q^0\rangle\langle q^0|\ell_{xA}\ell_{yB}|s^0\rangle}{E^0_q-E^0_r}+\sum_{q\ne s}\frac{\langle r^0|\ell_{xA}\ell_{yB}|q^0\rangle\langle q^0|\ell_{zC}|s^0\rangle}{E^0_q-E^0_s}\Bigg]\alpha_{AB}(f_\omega)\mu_C E_{0z} \nonumber \\
&+\langle r^0|\ell_{xA}\ell_{yB}\ell_{zC}|s^0\rangle \alpha_{AB,C}^{(\mu)}(f_\omega)E_{0z} \Bigg\}. \nonumber
\end{align}

-----------------------------------------------

% \begin{table}[h!]
%  \begin{center}
%    \caption{Chiral optical force coefficients $\mathtt{A}$, $\mathtt{B}$, $\mathtt{C}$, $\mathtt{E}$, $\mathtt{F}$ and $\mathtt{G}$ for the $J=1$ manifold.}
%    \label{tab:table1}
%    \begin{tabular}{c|c|c|c|c } % <-- Alignments: 1st column left, 2nd middle and 3rd right, with vertical lines in between
%                    & $|1_{-1}, m\rangle$ & $|1_0, m\rangle$ & $|1_1, m\rangle$ \\ \hline
%      $\mathtt{A}$ & $\frac{(3|m|-2)}{60}\left[\frac{(X_--\sqrt{3}Y_-)(3X_-+\sqrt{3}Y_-)}{(E_{2_{-2}}-E_{1_{-1}})}+\frac{(X_+-\sqrt{3}Y_+)(3X_++\sqrt{3}Y_+)}{(E_{2_{2}}-E_{1_{-1}})}\right]$ & $\frac{(2-3|m|)}{20}\left[\frac{3}{(E_{1_1}-E_{1_0})}+\frac{1}{(E_{2_{-1}}-E_{1_0})}\right]$ & $\frac{(3|m|-2)}{20}\left[\frac{3}{(E_{1_0}-E_{1_1})}+\frac{1}{(E_{2_0}-E_{1_1})}\right]$ \\
%      $\mathtt{B}$ & $\frac{(3|m|-2)}{20}\left[\frac{3}{(E_{1_1}-E_{1_{-1}})}+\frac{1}{(E_{2_{-1}}-E_{1_{-1}})}\right]$ & $-\frac{(3|m|-2)}{60}\left[\frac{(3X^2_-+2\sqrt{3}X_-Y_--3Y^2_-)}{(E_{2_{-2}-E_{1_0}})}+\frac{(3X^2_++2\sqrt{3}X_+Y_-+3Y^2_+)}{(E_{2_2-E_{1_0}})}\right]$ & $\frac{(2-3|m|)}{20}\left[\frac{3}{(E_{1_{-1}}-E_{1_1})}+\frac{1}{(E_{2_1}-E_{1_1})}\right]$ \\
%      $\mathtt{C}$ & $\frac{(2-3|m|)}{20}\left[\frac{3}{(E_{1_0}-E_{1_{-1}})}+\frac{1}{(E_{2_0}-E_{1_{-1}})}\right]$ & $\frac{(3|m|-2)}{20}\left[\frac{3}{(E_{1_{-1}}-E_{1_0})}+\frac{1}{(E_{2_1}-E_{1_0})}\right]$ & $\frac{(3|m|-2)}{5\sqrt{3}}\left[\frac{X_-Y_-}{(E_{2_{-2}}-E_{1_1})}+\frac{X_+Y_+}{(E_{2_2}-E_{1_1})}\right]$ \\
%      $\mathtt{D}$ & $0$  & $\frac{(2-3|m|)}{10}$ & $\frac{(3|m|-2)}{10}$ \\
%      $\mathtt{E}$ & $\frac{(3|m|-2)}{10}$ & $0$ & $\frac{(2-3|m|)}{10}$ \\
%      $\mathtt{F}$ & $\frac{(2-3|m|)}{10}$ & $\frac{(3|m|-2)}{10}$ & $0$ \\
%    \end{tabular}
%  \end{center}
% \end{table}

----------------------------------------------------------------

\begin{table}[h!]
\begin{tabular}{c|c|c|c|c}
x & strength & wavelength scaling & dissipative contribution & absolutely discriminatory \\ \hline
Electric chiral optical force & x & $1/\lambda$ & no & yes \\ \hline
Helicity chiral optical force & x & $1/\lambda^2$ & yes & no \\ \hline
\end{tabular}
\caption{\small Blah.}
\label{Table2}
\end{table}

% \begin{table}[h!]
% \begin{tabular}{c|c|c|c|c|c|c}
% & $A/2\pi\hbar f_0$ & $B/2\pi\hbar f_0$ & $C/2\pi\hbar f_0$ & $\alpha_{ZY}\mu_X/\alpha_0\mu_0$ & $\alpha_{XZ}\mu_Y/\alpha_0\mu_0$ & $\alpha_{YX}\mu_Z/\alpha_0\mu_0$ \\ \hline
% ($S$)-norbornenone & $3.65$ & $2.21$ & $2.00$ & $+4.59$ & $+1.52$ & $-0.94$ \\ \hline
% ($R$)-norbornenone & $3.65$ & $2.21$ & $2.00$ & $-4.59$ & $-1.52$ & $+0.94$ \\ \hline
% ($S$,$S$)-Tr\"{o}ger's base & $0.722$ & $0.243$ & $0.237$ & $0$ & $-1.49$ & $0$ \\ \hline
% ($R$,$R$)-Tr\"{o}ger's base & $0.722$ & $0.243$ & $0.237$ & $0$ & $+1.49$ & $0$ 
% \end{tabular}
% \caption{\small Molecular properties calculated using Gaussian, where $f_0=1.00\,\mathrm{GHz}$, $\alpha_0=1.65\times10^{-41}\,\mathrm{C}^2\,\mathrm{m}^2\mathrm{J}^{-1}$ and $\mu_0=3.34\times10^{-30}\,\mathrm{C}\,\mathrm{m}$.}
% \label{MolecularProperties}
% \end{table}
% We believe the electric-dipole moments are given in units of Debye ($3.33564\times 10^{-30}\mathrm{C}\,\mathrm{m}$) and that the polarisability components are given in units of atomic polarisability ($1.648777\dots\times10^{-41}\mathrm{C}^2\,\mathrm{m}^2\mathrm{J}^{-1}$), in which case the chiral quantities above should be multiplied by $5.4997265\times 10^{-71}\mathrm{C}^3\,\mathrm{m}^3\,\mathrm{J}^{-1}.$

% Something to CHECK is whether what Gaussian calls the $X$, $Y$ and $Z$ axes correspond to what we call the $X$, $Y$ and $Z$ axes.

% GRAVEYARD
% Our chiral optical force is conservative and can be expressed in the form
% \begin{align}
% \langle\overline{\mathbf{F}}\rangle&\approx-\frac{\partial U}{\partial Z_0}\hat{\mathbf{z}} \nonumber
% \end{align}
% with
% \begin{align}
% U&\approx-\frac{1}{2}E_{0z}E_{0y}E_{0x}(\mathtt{A}\alpha_{ZY}\mu_X+\mathtt{B}\alpha_{XZ}\mu_Y+\mathtt{C}\alpha_{YX}\mu_Z)\sin(2kZ_0). \label{U} 
% \end{align}
% where $U=U_{J_\tau,m}$ is an effective potential energy. Although we neglected the time-even absorptive contribution $\overline{\mathbf{F}}(g_\omega)$ in deriving (\ref{ChiralOpticalForceReduced}), we note here that its rotationally averaged form $\langle\overline{\mathbf{F}}(g_\omega)\rangle$ vanishes. Our chiral optical force is thus robust against the effects of residual absorption. 

% WE HYPOTHESISE THAT THE SAME RESULTS APPLY FOR QUANTISATION ALONG THE $y$ AXIS BUT WITH FLIPPED SIGNS FOR THE FORCE CONSTANTS AND THAT THE FORCE VANISHES FOR QUANTISATION ALONG THE $z$ AXIS AND FOR QUANTISATION ALONG AN ARBITRARY AXIS WE HAVE A NON-ZERO ACHIRAL CONTRIBUTION TO OUR OPTICAL FORCE.

% For $J\in\{0,\dots,5\}$, the diagonalisation of $H^{(0)}$ can completed analytically, giving
\begin{align}
|0_0,0\rangle^{(0)}&=|0,0,0\rangle^{(0)}, \nonumber \\
|1_{-1},m\rangle^{(0)}&=\frac{1}{\sqrt{2}}(|1,1,m\rangle^{(0)}-|1,-1,m\rangle^{(0)}), \nonumber \\
|1_0,m\rangle^{(0)}&=\frac{1}{\sqrt{2}}(|1,1,m\rangle^{(0)}+|1,-1,m\rangle^{(0)}), \nonumber \\
|1_1,m\rangle^{(0)}&=|1,0,m\rangle^{(0)}, \nonumber \\
|2_{-2},m\rangle^{(0)}&=b_{2_{-2},m}|2,0,m\rangle^{(0)}+c_{2_{-2},m}\frac{1}{\sqrt{2}}(|2,2,m\rangle^{(0)}+|2,-2,m\rangle^{(0)}), \nonumber \\
|2_{-1},m\rangle^{(0)}&=\frac{1}{\sqrt{2}}(|2,2,m\rangle^{(0)}-|2,-2,m\rangle^{(0)}), \nonumber \\
|2_0,m\rangle^{(0)}&=\frac{1}{\sqrt{2}}(|2,1,m\rangle^{(0)}-|2,-1,m\rangle^{(0)}), \nonumber \\
|2_1,m\rangle^{(0)}&=\frac{1}{\sqrt{2}}(|2,1,m\rangle^{(0)}+|2,-1,m\rangle^{(0)}) \nonumber \\
|2_2,m\rangle^{(0)}&=b_{2_2,m}|2,0,m\rangle^{(0)}+c_{2_2,m}\frac{1}{\sqrt{2}}(|2,2,m\rangle^{(0)}+|2,-2,m\rangle^{(0)}) \nonumber \\ \nonumber
\end{align}
with
\begin{align}
E_{0_0,m}^{(0)}&=0, \nonumber \\
E_{1_{-1},m}^{(0)}&=B+C, \nonumber \\
E_{1_0,m}^{(0)}&=A+C, \nonumber \\
E_{1_1,m}^{(0)}&=A+B, \nonumber \\
E_{2_{-2},m}^{(0)}&=2(A+B+C)-2\sqrt{A^2+B^2+C^2-AB-AC-BC}, \nonumber \\
E_{2_{-1},m}^{(0)}&=A+B+4C, \nonumber \\
E_{2_0,m}^{(0)}&=A+4B+C, \nonumber \\
E_{2_1,m}^{(0)}&=4A+B+C \nonumber \\
E_{2_2,m}^{(0)}&=2(A+B+C)+2\sqrt{A^2+B^2+C^2-AB-AC-BC} \nonumber
\end{align}
and
\begin{align}
b_{2_{\pm2},m}&=\sqrt{\frac{3(A-B)^2}{3(A-B)^2+[3(A+B)-E_{2_{\pm2},m}^{(0)}]^2}} \nonumber \\
c_{2_{\pm2},m}&=-b_{2_{\pm2},m}\left[\frac{3(A+B)-E_{2_{\pm2},m}^{(0)}}{\sqrt{3}(A-B)}\right], \nonumber
\end{align}
for example. For $J\in\{6,\dots\}$, the diagonalisation of $H^{(0)}$ must be completed numerically, in general.

\end{document}











\begin{figure}[ht!]
\centering\includegraphics[width=7cm]{opticafig1}
\caption{Sample caption (Fig. 2, \cite{Yelin:03}).}
\end{figure}

\section{Assessing final manuscript length}
The Universal Manuscript Template is based on the Express journal layout and will provide an accurate length estimate for \emph{Optics Express}, \emph{Biomedical Optics Express},  \emph{Optical Materials Express}, and our newest title \emph{Optics Continuum}. \emph{Applied Optics}, JOSAA, JOSAB, \emph{Optics Letters}, \emph{Optica}, and \emph{Photonics Research} publish articles in a two-column layout. To estimate the final page count in a two-column layout, multiply the manuscript page count (in increments of 1/4 page) by 60\%. For example, 11.5 pages in the Universal Manuscript Template are roughly equivalent to 7 composed two-column pages. Note that the estimate is only an approximation, as treatment of figure sizing, equation display, and other aspects can vary greatly across manuscripts. Authors of Letters may use the legacy template for a more accurate length estimate.



Figures and tables should be placed in the body of the manuscript. Standard \LaTeX{} environments should be used to place tables and figures:
\begin{verbatim}
\begin{figure}[htbp]
\centering\includegraphics[width=7cm]{opticafig1}
\caption{Sample caption (Fig. 2, \cite{Yelin:03}).}
\end{figure}
\end{verbatim}





\subsection{Formatting reference items}
Each source must have its own reference number. Footnotes (notes at the bottom of text pages) are not used in our journals. References require all author names, full titles, and inclusive pagination. Examples of common reference types can be found in the  \href{https://opg.optica.org/jot/submit/style/oestyleguide.cfm} {style guide}.

\subsection{Formatting reference citations}
References should be numbered consecutively in the order in which they are referenced in the body of the paper. Set reference callouts with standard \verb+\cite{}+ command or set manually inside square brackets [1].

To reference multiple articles at once, simply use a comma to separate the reference labels, e.g. \verb+\cite{Yelin:03,Masajada:13,Zhang:14}+, produces \cite{Yelin:03,Masajada:13,Zhang:14}.
%Using the \texttt{cite.sty} package will make these citations appear like so: [2--4].

% BIBTEX
Bib\TeX{} may be used to create a file containing the references, whose contents (i.e., contents of \texttt{.bbl} file) can then be pasted into the bibliography section of the \texttt{.tex} file. A Bib\TeX{} style file, \texttt{opticajnl.bst}, is provided.

If your manuscript already contains a manually formatted \verb|\begin{thebibliography}|... \verb|\end{thebibliography}| list, then delete the \texttt{latexmkrc} file (if present) from your submission files. However you should ensure that your manually-formatted reference list adheres to style accurately.

% DATASET
1. M. Partridge, "Spectra evolution during coating," figshare (2014), http://dx.doi.org/10.6084/m9.figshare.1004612.

% CODES
2. C. Rivers, "Epipy: Python tools for epidemiology," figshare (2014) [retrieved 13 May 2015], http://dx.doi.org/10.6084/m9.figshare.1005064.