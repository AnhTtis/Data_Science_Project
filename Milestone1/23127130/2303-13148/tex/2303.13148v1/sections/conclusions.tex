\section{Conclusions}

In this paper we propose a novel approach to OOD detection which uses a generic
pre-trained representation instead of training a discriminative classifier on
the ID classes. In initial experiments, two low-complexity classifiers already
significantly outperform other state-of-the-art methods on most considered OOD
benchmarks, confirming the applicability of the used representation for the OOD
detection.

As a second novelty, we  model the classification scores of the two classifiers
for the ID classes as a multivariate Guassian and show that this permits
addressing OOD detection as a formally defined two-class Neyman-Pearson
task. Compared to traditional logit thresholding, the solution to this task
leads to naturally calibrated OOD detection score connected directly to the
same false negative rate on all ID classes. Moreover, the resulting \grood
method leverages the strengths of both used classifiers leading to
consistent and superior performance over all considered benchmarks.

The proposed \grood method was compared to the state-of-the-art methods on
a very wide range of OOD problems with diverse types and strengths of semantic
and domain shifts. It effectively solves the mixed semantic and distribution
shift benchmarks and achieves the best performance on most of the other
considered problems.

The only observed limitations are related to very small low-contrast images in
SVHN dataset and very fine-grained classification of airplanes, which we
hypothesise requires a more complex use of the ID training data.

The simplicity of the adaptation of the \grood method to a novel problem --
only a multi-class logistic regression, i.e. a linear layer followed by
a softmax, is needed for training -- make the process fast.

We suggest that the proposed method combined with a generic representation is
suitable for most OOD tasks based on natural images and it remains open for
the research community to show any possible failure cases in new benchmarks;
with \grood many of the standard benchmarks are saturated and no longer
stimulate further progress.
