\section{Blockchain-based process orchestration}
\label{sec:procorch}
On most public and permissionless blockchains that allow smart contract deployment, as a rule, the developer cannot update the contract after it is deployed. Accompanied by the risks associated with data integrity errors in a blockchain-based distributed ledger which we wish to treat as a ''single source of truth'', minimizing the probability of smart contract faults is a major concern in the whole industry. Consequently, in addition to domain-specific languages, Model-Driven Engineering (MDE) techniques are steadily gaining ground in smart contract development \cite{model_driven}. In our context, this means that the dominant approach is a model -- usually formulated in BPMN -- to serve as a \textit{specification}, and smart contract logic is generated automatically from the model.

%In our own review of the state of the art, three tools emerged as the most mature and most relevant to our research: Caterpillar, Lorikeet and Chorchain. These provide valuable templates for future research, but in the end, due to the core difference from our own approach, We could not reuse them.

Caterpillar \cite{caterpillar} was the first open-source BPMN-to-Solidity compiler (Solidity is the primary smart contract development language for the Ethereum platform). %Its primary purpose is to execute collaborative business processes between mutually distrusting parties on blockchains. 
Since its initial release, several forks have emerged. Some of these also come with an extended feature set, like Blockchain Studio \cite{Mercenne_Brousmiche_Hamida_2018}, which adds role management, or \cite{Abid_Cheikhrouhou_Jmaiel_2020}, which adds time constraints. 

Lorikeet \cite{lorikeet} is a model-driven engineering approach that integrates assets into business processes. Lorikeet extends the BPMN 2.0 specification with support for asset registries and also transforms models into Solidity smart contracts. The smart contracts handle the orchestration of the process as well as interactions with the tokens.

Chorchain \cite{chorchain} is a tool that takes a BPMN \textit{choreography} and generates an Ethereum smart contract that can be used to execute the model. ChorChain also includes a dedicated modelling tool. The same authors also released two additional but related tools: Multi-Chain \cite{multichain} and FlexChain \cite{flexchain}. Multi-chain is similar to Chorchain, but it is also capable of generating smart contracts for Hyperledger Fabric \cite{hyperledger_fabric}. FlexChain can only produce Solidity smart contracts, but the user can also define a ruleset for each choreography. If a condition in the ruleset is met, then an off-chain processor will perform its underlying action.

Our analysis showed that the process state, as well as the process trace, are easily recoverable from the process manager smart contracts for \textit{all} the tools above.

The \href{https://docs.baseline-protocol.org/}{Baseline protocol}\footnote{\url{https://docs.baseline-protocol.org/}} is a developing open standard that allows enterprises to synchronize complex, multi-party business processes on distributed ledger technologies. Business process workflows in Baseline are formed as state machines. The standard includes some essential and optional privacy-related requirements. The protocol has two reference implementations; however, at the time of this writing, neither of these actually supports privacy/confidentiality measures.

%\todo{Táblázat}


%\begin{longtblr}[
%  caption = {Caption},
%  label = {tab:my_label},
%]{
%  width = \linewidth,
%  colspec = {Q[150]Q[185]Q[300]Q[117]},
%  cell{1}{1} = {c},
%  cell{1}{3} = {c},
%  cell{1}{4} = {c},
%  cell{2}{1} = {c},
%  cell{2}{4} = {c},
%  cell{3}{1} = {c},
%  cell{3}{4} = {c},
%  cell{4}{1} = {c},
%  cell{4}{4} = {c},
%  hlines,
%  vlines,
%}
%Name        & {Modeling\\tool}     & Contribution                                                               & Privacy \\
%Caterpillar & {BPMN\\process}      & first open-source BPMN-to-Solidity compiler                                & traceable    \\
%Lorikeet    & {BPMN\\process}      & {enables fungible/non-fungible asset swap/payment in \\business processes} & traceable    \\
%Chorchain   & {BPMN\\choreography} & generate smart contracts from BPMN choreograpy models                      & traceable    \\
%Multichain  & {BPMN\\choreography} & generate smart contract to multiple DLT systems from models                & traceable    \\
%Flexchain   & {BPMN\\choreography} & used-defined ruleset for each choreography                                 & traceable    
%\end{longtblr}