\section{Introduction}
%\linenumbers
In modern business science, \textit{Business Process Management} (BPM) as a discipline \cite{van2016business} advocates process-focused thinking about internal activities and external collaborations and proved to be a very important tool for controlling and improving key performance indicators. Automating the execution of business processes is a key proposition of BPM and has been supported for a long time by a range of Workflow Engines, and Workflow Enactment Services \cite{POURMIRZA201743}. Today most of these, typically centralized, tools use the leading business process modelling standard, BPMN 2.0 \cite{bpmn}, as a process definition language \cite{CHINOSI2012124}.

Distributed ledger technology, generally implemented on a blockchain basis, is widely recognized as a compelling platform to support the cross-organisational execution of business processes -- even when the organisations can not agree on a trusted (third) party as a middleman \cite{10.1145/3183367}. Importantly, smart contracts can be used to enforce and track the sequences of activities performed by organizations participating in collaborations; store sent and received messages; and either host shared data objects directly or anchor their changes in the blockchain via cryptographic commitments. However, blockchain-assisted BPM is still a relatively new discipline -- among other challenges, the privacy and confidentiality aspects have not yet been sufficiently addressed.

In our paper, we present a novel approach\footnote{This paper is based on the Scientific Student Association report submitted by Balázs Ádám Toldi to the 2022 competition at the Budapest University of Technology and Economics, advised by Imre Kocsis: \url{https://tdk.bme.hu/VIK/sw8/Kollaborativ-munkafolyamatok-titkossagmegorzo}} for orchestrating cross-organizational workflows -- collaborations -- with smart contracts in a confidentiality preserving way. The process logic is defined by BPMN models, and parties not participating in a process instance can not determine the state of the instance, even if they have full access to the transaction sequence of the underlying blockchain and complete knowledge of the process model.

Specifically, we encode the state updates of BPMN process instances as programs of the ZoKrates \cite{zokrates} toolkit, from which zero-knowledge proofs and state commitment update verifier smart contracts are generated. We also define an accompanying state commitment update protocol. We describe our open-source end-to-end framework implementation prototype\footnote{Available at \url{https://github.com/ftsrg/zkWF}} and provide an empirical demonstration of the practical viability of the presented approach.

To the best of our knowledge, our work is the first to offer robust confidentiality protection for blockchain-orchestrated, BPMN-specified collaborations.

%\todo{Todo.}\paragraph{Release?} The source code of our work is available at ..., In the future, We plan to release it under a free license as open-source software.
