\section{Performance}
%In this chapter, We provide quantitative performance figures to demonstrate the feasibility of the zkWF approach.
%\subsection{Hardware used for testing}
In addition to functional testing (compliance with model semantics, proper enforcement of authorization aspects and proper handling of compliant/noncompliant proofs), we used our test suite to evaluate key performance metrics of the approach. Performance tests were performed on a desktop PC (AMD Ryzen 7 2700, 16 GB of DDR4 memory); for gas measurements, we used a private Ethereum test network with geth version {1.10.25}. Note that blockchain-side efficiency measurements are largely irrelevant for Hyperledger Fabric (which has no ''gas'' notion and where the smart contract execution layer is very highly resource-scalable). Table \ref{tab:compare} summarizes the relevant size metrics of our test cases. The size of $P$ is understood in the number of 3-tuples in the array.

%\subsubsection{Software used for testing}
%To run each test scenario, We used our testing framework (section \ref{testing_frame}). To measure the gas required to run on an Ethereum-compatible blockchain, We set up a private test network using geth version 1.10.25.
%\subsubsection{Comparing test cases}
%Table \ref{tab:compare} compares the test cases based on their sizes. These aspects ruffly measure the complexity of each model. In the following sections, We want to find how the complexity of a model is linked to compilation time, setup time(the time it takes to do the trusted setup phase), proof creation time (the amount of time that is needed to produce a proof),  and gas usage on Ethereum.
\begin{table}[H]
    \centering
    \caption{Size characteristics of the test cases and scenario counts}
    \begin{tabular}{|c|c|c|c|c|c|}
        \hline
        \textbf{Case} & \textbf{Vertices} & \textbf{Edges} & \textbf{Executable} & \textbf{Size of $P$} & \textbf{Scenarios}\\
        \hline
        \hline
        Test 1 & 5 & 4 & 3 & 3 & 3\\
        \hline
        Test 2 & 9 & 10 & 5 & 7 & 9\\
        \hline
        Test 3 & 8 & 8 & 4 & 4 & 4\\
        \hline
        Test 4 & 6 & 5 & 2 & 3 & 2\\
        \hline
        Test 5 & 14 & 12 & 10 & 10 & 10\\
         \hline
        Repr. & 68 & 69 & 50 & 54 & 52\\
         \hline
    \end{tabular}
    \label{tab:compare}
\end{table}
%\subsubsection{Results}
%We ran all tests sequentially, and all test cases were successful. 
Table \ref{tab:timing} summarizes the runtimes of the off-chain computations. The compilation and zk-SNARK setup phases were executed once; proving time is the sum of computing the witness and generating the proof, and we give an average over the scenarios. 

In summary, the results indicate that deploying a smart contract and stepping the execution can be done in just a few minutes. We can also see that, although the representative model is 5-6 times larger than the larger ones, proofs only took about $2.3$ times more time to generate. We interpret this as a strong indication that our approach is practically feasible for real-life models.

\begin{table}[H]
    \centering
    \caption{Off-chain computation runtimes}
    \begin{tabular}{|l|c|c|c|c|}
    \hline
    \textbf{Case} &\textbf{Compilation time} & \textbf{Setup time} &  \textbf{Proving time avg.} \\
    \hline
    \hline
        Test 1 & 27.22 s & 129.58 s & 55.0 s \\
    \hline
        Test 2 & 48.32 s & 182.80 s & 88.67 s \\
    \hline
        Test 3 & 28.55 s & 129.69 s & 53.40 s  \\
    \hline
        Test 4 & 27.14 s & 128.82 s & 53.21 s  \\
    \hline
        Test 5 & 30.74 s & 133.44 s & 54.10 s\\
    \hline 
        Repr. & 81.02 s & 187.33 s & 122.47s \\
    \hline
    \end{tabular}
    \label{tab:timing}
\end{table}
%Table \ref{tab:gas_usage} shows how the smart contract performs if deployed on an Ethereum-compatible blockchain. The table demonstrates how much gas is used to deploy the smart contract (one-time fee). It also describes how much gas is required to update the current state on the blockchain.

Table \ref{tab:gas_usage} summarizes smart contract deployment cost to Ethereum and the average gas cost of (updating) smart contract calls in the zkWF protocol. Note that although the representative model is 5-6 times larger than the simple ones, the smart contract call gas cost is only moderately higher. As the hashes, signatures, and proofs have a fixed length, gas usage variability is essentially driven by the size of the encrypted version of the current state.

%It is essential to point out that even though the representative model is 5-6 times larger than the smaller models shown in table \ref{tab:compare}, the gas usage is only $11.90$ per cent higher than the lowest in previous steps.

%This was expected, since the hashes, the signatures, and the proofs have a fixed length. This means the only thing that drives gas usage higher in larger models is the encrypted version of the current state.
\begin{table}[H]
    \centering
    \caption{Gas usage on Ethereum}
    \begin{tabular}{|l|c|c|c|c|}
    \hline
    \textbf{Case} & \textbf{Deployment gas usage} & \textbf{Update gas usage avg.} \\
    \hline
    \hline
        Test 1 &  2,098,786 gas & 490,507 gas \\
    \hline
        Test 2 & 2,098,990 gas & 497,780 gas \\
    \hline
        Test 3 & 2,098,498 gas & 493,705 gas \\
    \hline
        Test 4 & 2,078,071 gas & 503,817 gas \\
    \hline
        Test 5 & 2,161,039 gas & 491,783 gas \\
    \hline 
        Repr. &  2,408,635 gas & $548,898$ gas \\
    \hline
    \end{tabular}
    \label{tab:gas_usage}
\end{table}


%\subsection{Comparison with existing solutions}
Performance-wise, it is hard to compare our approach to others, since, to our knowledge, we are the first to use zero-knowledge proofs to hide the current state of BPMN execution on-chain. Despite some common points, even \cite{toots_msc} remains incommensurable due to differences in goals, basic approach and tools.

That said, the Ethereum smart contracts' gas usage can be certainly compared to existing techniques. The deployment cost is on par with or is less than the existing solutions. In our case, the cost of updating the state is significantly higher compared to previous approaches like Chorchain \cite{chorchain} or Caterpillar \cite{caterpillar}. Chorchain \cite{chorchain} uses about 92,905 gas on average for each message, while Caterpillar \cite{caterpillar} requires similar amounts of gas on average. 

This ''confidentiality premium'', while it may not be currently acceptable on the mainnet (at the time of this writing, the block target size is 15 million gas), can be fully acceptable on permissioned EVM blockchains and sidechains. Additionally, support for the efficient checking of ZKP commitments is being developed very intensively for the Ethereum platform.