\section{Conclusion}
In this paper, we presented a group confidentiality-preserving approach for the smart contract-based orchestration of business collaborations, captured as BPMN 2.0 models. Our protocol is a novel, and to our knowledge, first-of-its-kind solution, which we validated functionally as well as evaluated from the resource usage and gas cost points of view. We also described a full toolchain prototype which we made available as open-source software.

%Currently, the end-to-end toolchain has full integration only for Ethereum; Hyperledger Fabric integration is partial. We will finish the integration with Fabric and also investigate whether we can also support emerging ''blockchain middleware'' technologies, such as Hyperledger Firefly, with the same toolset.

%As we noted, the gas usage of the process manager smart contracts is not satisfactorily low for all deployment options. In addition to potentially utilizing novel developments in Ethereum platform technology, 

%We would also love to reduce the gas usage of this approach on Ethereum to make transactions cheaper. This would be possible by generating proofs for making several steps in a batch.

%Last but not least, there is further work to be done on guaranteeing BPMN operational semantics compliance.