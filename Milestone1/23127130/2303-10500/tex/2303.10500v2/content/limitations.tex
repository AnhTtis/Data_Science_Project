\section{Threats to validity and future work}
\label{sec:futur}
%This section shows the limitations of this approach and the reason behind them.
%Earlier, we addressed our current constraints with respect to BPMN models which are admissible under our current approach; future work will target lifting these limitations. We dealt with the inability of ZoKrates not being able to verify that a given ciphertext is the encrypted form of a given message with a given key by constructing the protocol so that parties submitting a noncompliant ciphertext can always be identified irrepudiably. We accept smart contract gas costs as an outstanding issue; however, it is one which does not truly limit the applicability of our approach on permissioned-consensus platforms and one which we can reasonably expect to become a non-issue for the Ethereum mainnet, too. We will also investigate whether we can sidestep this issue by transforming our approach into a Layer 2 ZKP rollup scheme, where we can, at the very least, amortize gas costs across batches of process manager smart contract updates.

We see compliance with BPMN operational semantics as a non-negligible threat to validity, especially after our planned future extension of the supported BPMN subset. For the approach presented in this paper, we only tested compliant behavior and not formally proved it; this remains for future work. %Here we note that as long as all participants are aware of the way BPMN models are translated to admissible and nonadmissible state changes in zkWF programs, even divergences from the standard-prescribed semantics may be acceptable, but this is clearly not a fully satisfactory answer.

Impractical proof times for much larger BPMN models is also a threat. We plan to introduce the capability to handle \textit{hierarchical} process models. We expect that we can instantiate orchestrator smart contracts for sub-processes in a way that coordinates the commitment-management across the levels, but controls proof obligation complexity by requiring proof generation only for a limited-size model part for each update.

While the ring schedule ''fake updates'' approach is evidently correct for adhering participants (and, we surmise, for mostly adhering participants), side-channel protections is an open line of research. We plan to systematically analyse the ring schedule scheme under various participant failure models and to compare it with delay randomization schemes.

Lastly, we note that there are stronger versions of our group confidentiality model through additional \textit{intra-group} confidentiality constraints; it is an interesting question how our approach can be extended to such settings.

%Formally proving the operational semantics-preserving nature of our transformation logic would be a possible approach, but that would first need migrating the implementation to a transformation model where the translation rules themselves are first-class objects (such as in the model transformation platform VIATRA \cite{10.1007/978-3-319-21155-8_8}), and even then, it would be a highly complicated endeavour. 

%Instead, future work will first investigate proving behavioural equivalence between reference BPMN behaviour and zkWF programs on a \textit{model-by-model basis}, as a prerequisite check integrated into the toolchain. Recent work has formalized BPMN collaboration semantics in a way amenable for state space model checking \cite{CORRADINI2021111007}; this opens up the possibility to perform bisimulation analyses between the state transitions of models under ''authoritative'' semantics and under zkWF program encoded semantics. This approach will have the added benefit that it facilitates the introduction of property checking (soundness, safeness and application-specific requirements) on the BPMN models themselves.