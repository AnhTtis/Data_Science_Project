\section{Introduction}
%\linenumbers
In modern business science, \textit{Business Process Management} (BPM) as a discipline~\cite{van2016business} advocates process-focused thinking about internal activities and external collaborations to improve key performance indicators. Automating the execution of business processes is a key proposition of BPM and has been supported for a long time by various technical solutions~\cite{POURMIRZA201743}. Today, most of these, typically centralized, tools and services use the leading business process modeling standard, Business Process Model and Notation (BPMN) 2.0~\cite{bpmn} as a process definition language~\cite{CHINOSI2012124}.

Distributed ledger technology (DLT), generally implemented on a blockchain basis, is widely recognized as a compelling platform to support the cross-organisational execution of business processes -- even when the organisations cannot agree on a trusted (third) party as a middleman~\cite{10.1145/3183367}. Blockchain-deployed smart contracts can impartially enforce the agreed-on sequences of activities and track sent and received messages. Smart contracts can also host data objects acted on by a process directly or anchor their changes in the blockchain via cryptographic commitments. 

However, blockchain-assisted BPM is still a relatively new discipline -- importantly, known BPMN-based solutions are inadequate from the privacy and confidentiality point of view. This paper presents a novel, collaboration confidentiality-preserving approach and end-to-end prototype tooling for the on-chain process orchestration of cross-organizational, BPMN-based collaborations using zero-knowledge proofs (ZKPs)\footnote{This paper is based on the Scientific Student Association report submitted by Balázs Ádám Toldi to the 2022 competition at the Budapest University of Technology and Economics: \url{https://tdk.bme.hu/VIK/sw8/Kollaborativ-munkafolyamatok-titkossagmegorzo}}. Specifically, for a sufficient subset of BPMN, we present a transformation of the admissible state updates of BPMN process instances to programs of the ZoKrates~\cite{zokrates} toolkit. We assemble state update validity provers from these programs for the participants and proof-verifying orchestrator smart contracts. We define an on-chain process state commitment update protocol, describe our open-source end-to-end implementation prototype\footnote{Available at \url{https://github.com/ftsrg/zkWF}} and evaluate practical viability. %The current implementation supports Ethereum-based blockchains and Hyperledger Fabric and is easy to extend for further platforms.

Our contribution is novel from two aspects. First, to our knowledge, the confidentiality challenges of decentralized BPMN orchestration have not been addressed systematically and constructively yet. Second, we express BPMN execution as an incremental computation in a form amenable to commit-and-prove style zero-knowledge validation in smart contracts. This paves the way for further research on the computational representation of orchestrated BPMN execution against the continuously appearing ZKP advancements.

%\textbf{TBD The paper is structured as follows.}

\section{Motivation and problem statement}
\label{sec:motiv}
BPMN is a standardized approach to visually and precisely express \textit{how} business processes should be performed. BPMN is used in many domains -- including finance, banking, manufacturing, healthcare, logistics and telecommunications -- for capturing processes with well-defined sequences of regularly repeated activities. The BPMN standard defines several model types, \textit{process}, \textit{collaboration} and \textit{choreography} being the most widely used ones. \textit{Process (flow)} models are the simplest: these express the sequence, preconditions and exception handling of a single process performed by a single organization. Collaborations model the individual processes performed by collaborating parties -- usually business entities -- and their messaging-based interactions. Choreography diagrams focus solely on the message exchanges between collaborating entities.

%with message exchange flows and activity sequence coordination.

 %Such tools usually include rich technical support for process automation, e.g., to generate web pages for performing activities or to integrate external activity implementations.

\subsection{Decentralized orchestration}
For over a decade, software tools have been available to assist with \textit{process execution}. The more sophisticated ones track and \textit{orchestrate} activities according to a BPMN model, register activity-related data and perform decision-making on further process evolution. However, centralized orchestration introduces a trusted orchestrator party requirement when we move beyond single-entity processes. With the emergence of blockchain and distributed ledger technology, the potential of decentralizing various aspects of cross-organizational collaboration has been recognized quite early.

%in general, and the orchestration of BPMN-expressed collaborations in particular,

\begin{figure*}
\centering
\includegraphics[width=0.95\textwidth]{img/examplemodel.pdf}
\caption{Car leasing BPMN collaboration example (simplified for presentation). In the depicted example state, green denotes ''completed''; orange is ''active''.}
\label{fig:example}
\end{figure*}

Consider the BPMN car leasing collaboration model in Figure~\ref{fig:example}\footnote{The model was created in the ''Digitisation, artificial intelligence and data age workgroup'' of the ongoing BME-MNB cooperation project. (MNB is the Central Bank of Hungary.). For legibility, the process in Figure~\ref{fig:example} is slightly simplified; the whole model is available in our project repository.}, where the internal processes of a car dealership, a leasing company and a financing bank must be coordinated to accept a leasing application. Individual executions of models are called \textit{instances} and have an \textit{instance state}. The coloring in Figure \ref{fig:example} demonstrates a state: green denotes that the dealership has completed insurance processing and the leasing company and the bank would be able to begin processing. However, for the leasing company to proceed, active downpayment checking (orange) must be finished, then the downpayment filed, and a downpayment notification sent and received.

Using blockchain-deployed smart contracts that track \textit{collaboration state}, the \textit{execution enablement} and \textit{execution obligation} of the activities of the parties can be enforced without a dedicated, trusted party. The transaction journal nature of blockchains can also ensure that the full trace is also stored in an immutable and irrepudiable way. While tracking the internal state of participant-internal processes on-chain is not always desirable, it is a valuable \textit{option}; e.g., when decisions have to be made in a way verifiable by the other collaborating parties.

Orchestrating and journaling messages and collaborative data handling are two further collaboration aspects which can be improved with ''blockchainification''. %These are a) facilitating message passing and securing the messages themselves; and b) reading and writing the data used collaboratively by the processes (expressed via ''data objects'' and ''data stores'' in BPMN). In both cases, mainly for performance and cost reasons, the dominant approach is to use smart contracts to store cryptographic (hash) commitments to externally handled messages and data modifications.
%(expressed via ''data objects'' and ''data stores'' in BPMN). 
%In both cases, to avoid storing sizeable data on-chain, the technical approach is usually to manage cryptographic (hash) commitments to externally handled messages and data modifications in the orchestrator smart contracts.
In both cases, the orchestrator smart contracts usually only manage cryptographic (hash) commitments to externally handled messages and data modifications, to avoid storing sizeable data on-chain.

Tools and approaches exist to create orchestrator smart contracts from BPMN models (see Section~\ref{sec:procorch}). However, no systematic solution exists to protect sensitive collaboration state information in the smart contract state from parties who can read the blockchain but do not participate in the collaboration. In our example, a leasing company may wish that its competitors do not see how many open cases they have, how long it takes to perform key steps in the process, or what lease rates they apply.

Fulfilling such requirements is a confidentiality challenge that contradicts core blockchain design principles. Blockchain nodes must be able to \textit{validate} and \textit{execute} incoming transaction requests to reach consensus on ledger updates, be those changes of the balances of a natively tracked cryptocurrency or state changes of deployed smart contracts. If the transaction details are made ''incomprehensible'' to the nodes, e.g., by off-chain encryption, they can't validate the preconditions for performing the transaction and compute state updates. %In this case, smart contracts can still serve as resilient storage for encrypted state updates or hash commitments but lose the ability to validate the request \textit{semantically}; in our case, whether the right party proposes the right state-advancement of the workflow. 
%For smart contracts, the two dominant, general-purpose cryptographic approaches to the privacy/confidentiality versus transaction validation/executability conundrum are validating transactions with ZKPs and confidentiality-preserving execution using homomorphic encryption, with the prior being significantly better established currently.
For smart contracts, the dominant \textit{cryptographic} answers to this dilemma are validating transactions with ZKPs and confidentiality-preserving execution using homomorphic encryption, with the prior being significantly better established currently.

\subsection{Problem statement: BPMN collaboration confidentiality}
\label{subsec:prob}
We set up our problem statement through a basic system model and the enumeration of required security properties. We target a simple form of collaboration confidentiality (see the properties below) under the assumption that it is not in the interest of any process participant to leak information about process instances; participants neither directly leak information nor help external parties to compromise confidentiality. This is one of the realistic models for our setting, even though the participants do not completely trust the \textit{actions} of each other. We will touch briefly on stronger models in Section~\ref{sec:futur}.

%However, these have not yet been applied to the on-chain orchestration of BPMN-defined collaborations; we present an approach using the non-interactive ZKP route.

\subsubsection{Basic assumptions and terminology} \textit{participants} wish to collaborate in the execution of an \textit{instance} of a previously agreed-on BPMN collaboration definition. All other parties are \textit{process external}. All participants have a cryptographic key pair for signature-based authentication and process activity authorisation. The underlying process model is public knowledge, but the public keys are shared only between the participants. We assume the absence of private key compromises. 

%These key pairs are ''application-level'' and independent of those used for basic transaction signing at the blockchain platform level.

For the underlying blockchain, we assume complete integrity (no successful attack on the consensus) and, for the sake of simplicity, deterministic finality (accepted blocks do not get retracted). Note that even blockchains with probabilistic block finality are usually quasi-deterministically final already at the time scale of a few blocks. On the other hand, process external parties have complete visibility of blockchain transactions. We treat the blockchain as \textit{fair} -- any transaction submitted by a participant is included in a block in a reasonable time, irrespective of concurrent transaction request load. While, in practice, blockchain platforms have strongly varying fault and threat models and sensitivity (see, e.g., \cite{cachin2017blockchain}), these are basic assumptions of normal operational conditions. As a part of platform selection, security and dependability analysis should evaluate the risk of these assumptions not being met.

\subsubsection{System model} the classic Business Process Orchestrator (BPO) middleware pattern \cite{gorton2006essential} facilitates business process execution by providing a message broker and extending it with state management and persistent state storage. The solutions in the state-of-the-art closely match this pattern. (Technically, message passing is only \textit{coordinated} and journaled by the smart contract their core.) The smart contract as a Process Controller \cite{gorton2006essential} also performs authentication and authorization based on the BPMN model to ensure that the stored state sequence never deviates from the model semantics. We also aim to employ a blockchain-deployed smart contract as a BPO.

%through which participants asynchronously track their respective progress in the process, exchange messages and which serves as a trustworthy and resilient log of the execution trace. 

\subsubsection{Security properties} we target a set of integrity, availability and confidentiality guarantees. Integrity and availability properties are already covered by the prior art; our contributions lie in establishing collaboration confidentiality, as defined by properties \textbf{C1} and \textbf{C2}, despite using smart contracts. BPO-SC refers to a per-process instance BPO smart contract.

\smallskip

\noindent \textbf{I1}: The state traces enforced by the BPO-SC always adhere to the operational semantics of the underlying BPMN model.

\smallskip

\noindent \textbf{I2}: Process-external parties cannot influence the BPO-SC state.

\smallskip

\noindent \textbf{I3}: Process state updates can be initiated only by the participants authorized by the model and instance state.

\smallskip

\noindent \textbf{A1}: No external party can influence authorised participants' ability to perform state updates in a bounded time.

\smallskip

\noindent \textbf{A2}: No participant can influence the ability of authorized participants to perform state updates in a bounded time.

\smallskip

\noindent \textbf{A3}: Participants can always learn the trace and current state.

\smallskip

\noindent \textbf{C1}: External parties cannot determine participant identities.

\smallskip

\noindent \textbf{C2}: No external party can learn more about the trajectory, timings and stepwise properties (e.g., process variables and message contents) of the trace during and after execution than the fact that an instance has been started.

%\begin{itemize}
    %\item [I1] For each process instance, the BPO-SC is the authoritative source of the process state, and it stores a full, immutable and irrepudiable log of the instance traces.
%    \item [I1] The state traces enforced by the BPO-SC always adhere to the operational semantics of the underlying BPMN model.
%    \item [I2] Process external parties cannot influence the BPO-SC state.
%    \item [I3] In any given state, process state updates can be initiated only by the participants authorized by the model and instance state.
%    \item [A1] No external party can influence the ability of authorized participants to perform state updates in a bounded time successfully. 
%    \item [A2] No participant can influence the ability of authorized participants to perform state updates in a bounded time successfully.
%    \item [A3] Each participant can always learn the authoritative trace and current state.
%    \item [C1] No external party can determine the identity of the process participants without collusion with one.
%    \item [C2] Without collusion, no external party can learn more about the trajectory, timings and stepwise properties (e.g., process variables and message contents) of the trace during and after execution than the fact that an instance has been started.
%\end{itemize}

%As discussed later, our presented approach fulfils these requirements with only minor caveats. That said, to our knowledge, this work is the first to address these requirements for BPMN-based, decentralized BPOs. %only a relaxed form of A2 is guaranteed (a malicious participant can inhibit further updates, but only in a manner instantly detectable by the other participants) and for C2, there is a tradeoff between confidentiality and not delaying state updates artificially. That said, to our knowledge, this work is the first to address these requirements for BPMN-based, decentralized BPOs.

%-- which, for the sake of simplicity, we approach here briefly as a state machine replication problem. (This is a common shorthand; much of the classic literature on Byzantine consensus protocols formalizes consensus in terms of state machine replication.)


%To the best of our knowledge, our work is the first to offer robust confidentiality protection for blockchain-orchestrated, BPMN-specified collaborations.

%\todo{Todo.}\paragraph{Release?} The source code of our work is available at ..., In the future, We plan to release it under a free license as open-source software.
