\section{Related work}
\label{sec:procorch}
Smart contracts, as a rule, cannot be altered after deployment; thus, to minimize the probability of software faults, domain-specific languages and Model-Driven Engineering (MDE) are steadily gaining ground in smart contract development~\cite{model_driven}. In our context, the established approach is a BPMN model to serve as a \textit{specification}, and orchestrator smart contract logic is generated automatically from the model.

%In our own review of the state of the art, three tools emerged as the most mature and most relevant to our research: Caterpillar, Lorikeet and Chorchain. These provide valuable templates for future research, but in the end, due to the core difference from our own approach, We could not reuse them.
\subsection{Decentralized business process orchestration}
Caterpillar~\cite{caterpillar} was the first open-source BPMN-to-Solidity compiler (Solidity is the primary smart contract development language for the Ethereum platform). %Its primary purpose is to execute collaborative business processes between mutually distrusting parties on blockchains. 
Since its initial release, several forks have emerged. Some of these also come with an extended feature set, like Blockchain Studio~\cite{Mercenne_Brousmiche_Hamida_2018}, which adds role management, or~\cite{Abid_Cheikhrouhou_Jmaiel_2020}, which adds time constraints. Lorikeet~\cite{lorikeet} is a model-driven engineering approach that integrates assets into business processes. Lorikeet extends the BPMN 2.0 specification with support for asset registries and also transforms models into Solidity smart contracts. The smart contracts handle the orchestration of the process as well as interactions with the tokens. Chorchain~\cite{chorchain} takes a BPMN \textit{choreography} and generates an Ethereum smart contract that can be used to execute the model. ChorChain also includes a dedicated modeling tool. The same authors released two further tools: Multi-Chain~\cite{multichain} and FlexChain~\cite{flexchain}. Multi-chain is similar to Chorchain, but it also supports Hyperledger Fabric~\cite{hyperledger_fabric}. FlexChain can only produce Solidity smart contracts, but the user can also define a ruleset for each choreography. If a condition in the ruleset is met, then an off-chain processor will perform its underlying action. 

Our analysis showed that the process state and trace are easily recoverable from the process manager smart contracts for \textit{all} the tools above.

%Beyond BPMN, the \href{https://docs.baseline-protocol.org/}{Baseline protocol}\footnote{\url{https://docs.baseline-protocol.org/}} is a developing open standard that allows enterprises to synchronize complex, multi-party business processes on distributed ledger technologies. Business process workflows in Baseline are formed as state machines. The standard includes some essential and optional privacy-related requirements. The protocol has two reference implementations; however, at the time of this writing, neither of these actually supports privacy/confidentiality measures.

\subsection{Commit-and-prove ZKP with smart contracts}
Zero-Knowledge Proofs (ZKPs) are cryptographic methods to prove the validity of various statements without revealing any additional information~\cite{ZKProofCommunity}. ZKP verification in a smart contract requires a scheme with ''single-shot'' message passing from prover to verifier; in this work, we rely on zk-SNARKs, a family of \textit{noninteractive}, and also \textit{succinct} (small and cheaply verifiable proofs) ZKPs. We use the ZoKrates toolkit as a ZKP front-end with a high-level programming language~\cite{zokrates}. ZoKrates currently supports the Groth16~\cite{Groth_2016}, GM17~\cite{Groth_Maller_2017} and Marlin~\cite{cryptoeprint:2019/1047} proving schemes.

Our contribution implements a commit-and-prove approach. In commit-and-prove schemes, a party first commits to an input and, possibly later, proves some predicate about the input -- without revealing it~\cite{zkws4}. This is a widely used pattern in the smart contract-based application of zero-knowledge proofs. Recent surveys on ZKP schemes, technologies and applications can be found in \cite{9520375} and~\cite{9300214}.

%provides an overview and a formalization framework of commit-and-prove for non-interactive zero-knowledge protocols; however, such mathematical depth is not necessary for our treatment, as for our specific application we utilize a framework that provides a specific zero-knowledge commit-and-prove scheme in the mathematical sense by recording commitments on a blockchain and checking proofs over commitments in smart contracts. This is a widely used pattern in the smart contract-based application of zero-knowledge proofs; \cite{9520375} and \cite{9300214} provide relatively recent surveys on proof schemes, technologies and applications. %In this context, the key contribution of our work is using zero-knowledge commit-and-prove to address the privacy and confidentiality challenges of on-chain BPMN orchestration; during our extensive literature research, we did not find any already-proposed solutions to this specific problem.

%\todo{Táblázat}


%\begin{longtblr}[
%  caption = {Caption},
%  label = {tab:my_label},
%]{
%  width = \linewidth,
%  colspec = {Q[150]Q[185]Q[300]Q[117]},
%  cell{1}{1} = {c},
%  cell{1}{3} = {c},
%  cell{1}{4} = {c},
%  cell{2}{1} = {c},
%  cell{2}{4} = {c},
%  cell{3}{1} = {c},
%  cell{3}{4} = {c},
%  cell{4}{1} = {c},
%  cell{4}{4} = {c},
%  hlines,
%  vlines,
%}
%Name        & {Modeling\\tool}     & Contribution                                                               & Privacy \\
%Caterpillar & {BPMN\\process}      & first open-source BPMN-to-Solidity compiler                                & traceable    \\
%Lorikeet    & {BPMN\\process}      & {enables fungible/non-fungible asset swap/payment in \\business processes} & traceable    \\
%Chorchain   & {BPMN\\choreography} & generate smart contracts from BPMN choreograpy models                      & traceable    \\
%Multichain  & {BPMN\\choreography} & generate smart contract to multiple DLT systems from models                & traceable    \\
%Flexchain   & {BPMN\\choreography} & used-defined ruleset for each choreography                                 & traceable    
%\end{longtblr}