\subsection{Performance evaluation}
%In this chapter, We provide quantitative performance figures to demonstrate the feasibility of the zkWF approach.
%\subsection{Hardware used for testing}
In addition to functional testing (compliance with model semantics, proper enforcement of authorization aspects and proper handling of compliant/noncompliant proofs), we used our test suite to evaluate key performance metrics of the approach. Performance tests were performed on a desktop PC (AMD Ryzen 7 2700, 16 GB of DDR4 memory). 

In Ethereum, smart contract execution steps, measured in ''gas'', incur a cryptocurrency cost, paid by the transaction-requesting user. For measurements of gas used, we used a private, one-node Ethereum test network with version {1.10.25} of geth, the official Go implementation of the Ethereum protocol. Blockchain-side efficiency measurements are largely irrelevant for Hyperledger Fabric, which has no ''gas'' notion and where the smart contract execution layer is highly resource-scalable. Table \ref{tab:compare} summarizes the relevant size metrics of our test cases.

%\subsubsection{Software used for testing}
%To run each test scenario, We used our testing framework (section \ref{testing_frame}). To measure the gas required to run on an Ethereum-compatible blockchain, We set up a private test network using geth version 1.10.25.
%\subsubsection{Comparing test cases}
%Table \ref{tab:compare} compares the test cases based on their sizes. These aspects ruffly measure the complexity of each model. In the following sections, We want to find how the complexity of a model is linked to compilation time, setup time(the time it takes to do the trusted setup phase), proof creation time (the amount of time that is needed to produce a proof),  and gas usage on Ethereum.
\begin{table}[H]
    \centering
    \caption{BPMN model and test case characteristics}
    \begin{tabular}{|c|c|c|c|c|c|}
        \hline
        \textbf{Case} & \textbf{Vertices} & \textbf{Edges} & \textbf{Executable} & \textbf{Size of $P$} & \textbf{Scenarios}\\
        \hline
        \hline
        Test 1 & 5 & 4 & 3 & 3 & 3\\
        \hline
        Test 2 & 9 & 10 & 5 & 7 & 9\\
        \hline
        Test 3 & 8 & 8 & 4 & 4 & 4\\
        \hline
        Test 4 & 6 & 5 & 2 & 3 & 2\\
        \hline
        Test 5 & 14 & 12 & 10 & 10 & 10\\
         \hline
        Repr. & 68 & 69 & 50 & 54 & 52\\
         \hline
    \end{tabular}
    \label{tab:compare}
\end{table}
%\subsubsection{Results}
%We ran all tests sequentially, and all test cases were successful. 
Table \ref{tab:timing} summarizes the runtimes of the off-chain computations. Compilation and zk-SNARK setup were executed once; proving time is the sum of computing the witness and generating the proof, and we give an average over the scenarios. The measurements indicate that our approach is practically feasible for real-life models.

\begin{table}[H]
    \centering
    \caption{Off-chain computation runtimes}
    \begin{tabular}{|l|c|c|c|c|}
    \hline
    \textbf{Case} &\textbf{Compilation time} & \textbf{Setup time} &  \textbf{Proving time avg.} \\
    \hline
    \hline
        Test 1 & 27.22 s & 129.58 s & 55.0 s \\
    \hline
        Test 2 & 48.32 s & 182.80 s & 88.67 s \\
    \hline
        Test 3 & 28.55 s & 129.69 s & 53.40 s  \\
    \hline
        Test 4 & 27.14 s & 128.82 s & 53.21 s  \\
    \hline
        Test 5 & 30.74 s & 133.44 s & 54.10 s\\
    \hline 
        Repr. & 81.02 s & 187.33 s & 122.47s \\
    \hline
    \end{tabular}
    \label{tab:timing}
\end{table}
%Table \ref{tab:gas_usage} shows how the smart contract performs if deployed on an Ethereum-compatible blockchain. The table demonstrates how much gas is used to deploy the smart contract (one-time fee). It also describes how much gas is required to update the current state on the blockchain.

Table \ref{tab:gas_usage} summarizes the gas costs of smart contract deployment and smart contract calls in the zkWF protocol. Note that although the representative model is 5-6 times larger than the simple ones, the smart contract call gas cost is only moderately higher. As the hashes, signatures, and proofs have a fixed length, gas usage variability is driven by the size of the encrypted version of the current state. In the measurements, we use state cleartext instead of ciphertext to eliminate the impact of the not-constrained encryption. %(and possibly prior compression).

%It is essential to point out that even though the representative model is 5-6 times larger than the smaller models shown in table \ref{tab:compare}, the gas usage is only $11.90$ per cent higher than the lowest in previous steps.

%This was expected, since the hashes, the signatures, and the proofs have a fixed length. This means the only thing that drives gas usage higher in larger models is the encrypted version of the current state.
\begin{table}[H]
    \centering
    \caption{Gas usage on Ethereum}
    \begin{tabular}{|l|c|c|c|c|}
    \hline
    \textbf{Case} & \textbf{Deployment gas usage} & \textbf{Update gas usage avg.} \\
    \hline
    \hline
        Test 1 &  2,098,786 gas & 490,507 gas \\
    \hline
        Test 2 & 2,098,990 gas & 497,780 gas \\
    \hline
        Test 3 & 2,098,498 gas & 493,705 gas \\
    \hline
        Test 4 & 2,078,071 gas & 503,817 gas \\
    \hline
        Test 5 & 2,161,039 gas & 491,783 gas \\
    \hline 
        Repr. &  2,408,635 gas & $548,898$ gas \\
    \hline
    \end{tabular}
    \label{tab:gas_usage}
\end{table}


%\subsection{Comparison with existing solutions}
%Performance-wise, it is hard to compare our approach to others, since, to our knowledge, we are the first to use zero-knowledge proofs to hide the current state of BPMN execution on-chain. %Despite some common points, even \cite{toots_msc} remains incommensurable due to differences in goals, basic approach and tools. 
Due to the novelty of our approach, it is comparable with the state of the art only in gas costs. Deployment is on par with, or is better than, the existing solutions. However, the cost of updating the state is significantly higher; ChorChain uses about 92,905 gas on average for each message and Caterpillar is similar to ChorChain. 

This ''confidentiality premium'' is certainly not acceptable on the Ethereum mainnet. Still, it can be argued that the high gas price on the mainnet has ''priced out'' all use cases that were not strictly crypto-financial years ago. On the other hand, at the time of this writing, on multiple alternative EVM-based public blockchains, the gas costs of our operations translate to fractions of 1 USD. Additionally, our approach has evident usage potential on purpose-created, permissioned, cross-organizational blockchains; in this case, the gas cost is a technical consideration and low enough to allow for dozens of transactions per block under customary block gas targets. Lastly, we store encrypted state on-chain ''only'' to fulfil requirement A3 the simplest way; highly available off-chain data storage with blockchain-based integrity assurance is a common technique.
