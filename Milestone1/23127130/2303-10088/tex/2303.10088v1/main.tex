\documentclass[a4paper,twoside,12pt,fleqn]{amsart}
\setlength {\marginparwidth }{2cm}
\usepackage[margin=1.0in]{geometry}


\usepackage{url}
\usepackage{amsmath}
\usepackage{amsthm}
\usepackage{todonotes}
\usepackage{amssymb}
\usepackage{enumitem}
\usepackage{soul}
\usepackage[hidelinks]{hyperref}
\theoremstyle{plain}
\newtheorem{theorem}{Theorem}[section]
\newtheorem{corollary}[theorem]{Corollary}
\newtheorem{proposition}[theorem]{Proposition}
\newtheorem{observation}[theorem]{Observation}
\newtheorem{lemma}[theorem]{Lemma}
\newtheorem{conjecture}[theorem]{Conjecture}
\newtheorem{question}[theorem]{Question}
\newtheorem{problem}[theorem]{Problem}
\newtheorem{fact}[theorem]{Fact}
\theoremstyle{definition}
\newtheorem{definition}[theorem]{Definition}
\newtheorem*{example}{Example}
\newtheorem{remark}[theorem]{Remark}
\newtheorem*{claim}{Claim}
\newcommand{\bb}[1]{\mathbb{#1}}
\newcommand{\cal}[1]{\mathcal{#1}}
\newcommand{\ct}{\mathrm{CT}}
\newcommand{\card}[1]{\left| #1 \right|}    %cardinality
\def\Succ{\mathop{\mathrm{Succ}}\nolimits}
\def\ImmSucc{\mathop{\mathrm{ImmSucc}}\nolimits}
\newcommand{\fr}{Fra\"iss\'e }
\def\cont{^\frown}
\def\L{\mathrm L}
\def\R{\mathrm R}
\def\X{\mathrm X}
\def\str#1{\mathbf {#1}}
\def\lexleq{\leq_{\mathrm{lex}}}
\def\lexlt{<_{\mathrm{lex}}}
\def\eltlt{\vartriangleleft}
\def\eltleq{\trianglelefteq}
\def\Alphabet{\Sigma}
\def\rel#1#2{R_{\mathbf{#1}}^{#2}}
\def\Shape#1#2{\mathrm{Shape}\allowbreak(#1,\allowbreak #2)}
\def\nShape#1#2#3{\mathrm{Shape}_{#1}\allowbreak(#2,\allowbreak #3)}
\newcommand{\AmbStr}[1]{(#1, \allowbreak \lexleq\nobreak,\allowbreak\preceq\nobreak,\allowbreak\eltleq)}

\begin{document}
\title{Characterisation of the big Ramsey degrees of the generic partial order}
\authors{
	\author[M. Balko]{Martin Balko}
	\author[D. Chodounsk\'y]{David Chodounsk\'y}
	\address{Department of Applied Mathematics (KAM), Charles University, Ma\-lo\-stranské~nám\v estí 25, Praha 1, Czech Republic}
	\email{\{balko,chodounsky,hubicka,matej\}@kam.mff.cuni.cz}
	\author[N. Dobrinen]{Natasha Dobrinen}
	\address{Department of Mathematics, University of Notre Dame,
	255 Hurley Bldg
Notre Dame, IN 46556 USA}
	\email{ndobrine@nd.edu}
	\author[J. Hubi\v cka]{Jan Hubi\v cka}
	\author[M. Kone\v cn\'y]{Mat\v ej Kone\v cn\'y}
	\author[L. Vena]{Lluis Vena}
	\address{Universitat Polit\`ecnica de Catalunya, Barcelona, Spain}
	\email{lluis.vena@gmail.com}
	\author[A. Zucker]{Andy Zucker}
	\address{Department of Pure Mathematics, University of Waterloo, 200 University Ave W, Waterloo, ON N2L 3G1, Canada}
	\email{a3zucker@uwaterloo.ca}
}
\maketitle
\pagestyle{plain}

\begin{abstract}
	As a result of 33 intercontinental Zoom calls, we characterise big
	Ramsey degrees of the generic partial order.
	This is an infinitary extension of the well known fact that
	finite partial orders endowed with linear extensions form a Ramsey class (this result
	was announced by Ne\v set\v ril and R\"odl in 1984 with first published proof by Paoli, Trotter and Walker in 1985).
	Towards this, we refine earlier upper bounds obtained by Hubi\v cka based on a new connection of big Ramsey degrees to the Carlson--Simpson theorem
	and we also introduce a new technique of giving lower bounds using an iterated application of the upper-bound theorem. 
\end{abstract}
%
\section{Introduction}
\label{sec:introduction}

Given structures $\str{A}$ and $\str{B}$, we denote by $\binom{\str{B}}{\str{A}}$ the set
of all embeddings from $\str{A}$ to $\str{B}$. We write $\str{C}\longrightarrow (\str{B})^\str{A}_{k,\ell}$ to denote the following statement:
\begin{quote}
	For every colouring $\chi$ of $\binom{\str{C}}{\str{A}}$ with $k$ colours, there exists
	an embedding $f\colon\str{B}\to\str{C}$ such that $\chi$  takes no more than $l$ values on $\binom{f(\str{B})}{\str{A}}$.
\end{quote}
For a countably infinite structure $\str{B}$ and a finite substructure $\str{A}$, the \emph{big Ramsey degree} of $\str{A}$ in $\str{B}$ is
the least number $\ell\in \mathbb N\cup \{\infty\}$ such that $\str{B}\longrightarrow (\str{B})^\str{A}_{k,\ell}$ for every $k\in \mathbb N$.
Similarly, if $\mathcal K$ is a class of finite structures and $\str{A}\in\mathcal K$, the \emph{small Ramsey degree} of $\str{A}$ in~$\mathcal K$ is the least number $\ell\in \mathbb N\cup \{\infty\}$ such that for every $\str{B}\in \mathcal K$ and $k\in \mathbb N$ there exists $\str{C}\in \mathcal K$ such that $\str{C}\longrightarrow (\str{B})^\str{A}_{k,l}$.


A structure is \emph{homogeneous} if every isomorphism between
two of its finite
substructures extends to an automorphism.
It is well known that up to
isomorphism there is a unique homogeneous partial order $\str{P}=(P,\leq_\str{P})$ such that
every countable partial order has an embedding into $\str{P}$. The order $\str{P}$ is called
the \emph{generic partial order} (see e.g.~\cite{Macpherson2011}). We refine the following recent result.

\begin{theorem}[Hubi\v cka~\cite{Hubicka2020CS}]
	The big Ramsey degree of every finite partial order
	in the
	generic partial order $\str{P}$ is finite.
\end{theorem}

We characterise the big Ramsey degrees of (finite substructures of) $\str{P}$ in a similar fashion as the authors did in \cite{Balko2021exact} for binary finitely constrained free amalgamation classes. Similarly to that case, our characterisation is in terms of a tree-like object we call a ``poset diary" where each level has exactly one critical event. It follows that any poset diary which codes the generic poset is a big Ramsey structure for $\str{P}$ (Definition~1.3 of \cite{zucker2017}). 


Relative to the small Ramsey degree case (see e.g.~\cite{NVT14,Bodirsky2015,hubika2020structural}), 
there are fewer classes of structures for which
big Ramsey degrees are fully understood.  The following is the current state of the art:

 \begin{enumerate}
	 \item The Ramsey theorem implies that the big Ramsey degree of every finite linear order in the order of $\omega$ is 1.
	 \item In 1979, Devlin refined upper bounds by Laver and characterised big Ramsey degrees of the order of rationals~\cite{devlin1979,todorcevic2010introduction}.
	 \item In 2006 Laflamme, Sauer and Vuksanovi\'c characterised big Ramsey degrees of the Rado (or random) graph and related random structures in binary languages~\cite{Laflamme2006}. Actual numbers were counted by Larson~\cite{larson2008counting}.
	\item In 
 2008
 Nguyen Van Thé characterised big Ramsey degrees of the ultrametric {U}rysohn spaces~\cite{NVT2008}.
	\item In 2010 Laflamme, Nguyen Van Thé and Sauer~\cite{laflamme2010partition} characterised the big Ramsey degrees of the dense local order, $\mathbf{S}(2)$
 and the $\mathbb{Q}_n$ structures
\item		 
  In 2020
		 Coulson, Dobrinen, and Patel  in ~\cite{coulson2022indivisibility} and ~\cite{coulson2022SDAP} 
   characterised  the big Ramsey degrees of
homogeneous binary relational structures  which satisfy  SDAP$^+$,
recovering work  
in~\cite{Laflamme2006} 
and~\cite{laflamme2010partition} and characterising the big Ramsey degrees of the ordered versions of structures in~\cite{Laflamme2006},
the generic $n$-partite and generic ordered $n$-partite graphs, and the $(\mathbb{Q}_{\mathbb{Q}})_n$ hierarchy.
	\item In 2020 Barbosa characterised big Ramsey degrees of  the directed circular orders $\mathbf{S}(k)$, $k\geq 2$, ~\cite{barbosa2020categorical}.
	\item  In 2020 a characterisation of big Ramsey degrees of the triangle-free Henson graph was obtained by Dobrinen~\cite{dobrinen2020ramsey} and independently by the remaining authors of this article.
	\item Joining efforts,
	the authors were able to fully characterise big Ramsey degrees of free amalgamation classes in finite binary languages described by finitely many forbidden cliques~\cite{Balko2021exact}.
 \end{enumerate}
 
 
If one draws analogy to the small Ramsey results, many of the  aforementioned
characterisations can be seen as infinitary generalisations of special cases of
the Ne\v set\v ril--R\"odl theorem~\cite{Nevsetvril1977}. Small Ramsey degrees (or the \emph{Ramsey expansions} satisfying the \emph{expansion property}, see~\cite{NVT14})
are known for many other classes including the class of partial orders with linear extensions~\cite{Nevsetvril1984,Trotter1985,sokic2012ramsey,masulovic2016pre,solecki2017ramsey,nevsetvril2018ramsey}, metric spaces~\cite{Nevsetvril2007,Dellamonica2012,masulovic2016pre} and other examples derived by rather general structural conditions~\cite{Hubicka2016}.


In this paper, we characterize big Ramsey degrees of the
generic partial order. This class represents an important new example since its finitary
counter-part is not a consequence of the Ne\v set\v ril--R\"odl theorem~\cite{Nevsetvril1977},
and it is  not of the form covered by  \cite{coulson2022SDAP}.
Towards this direction we needed to further refine the new proof technique for upper bounds, based on an application of the Carlson--Simpson
theorem, introduced in \cite{Hubicka2020CS}, and also find a completely new approach to lower bounds. The techniques introduced in this paper
generalize to other classes as we briefly outline in the final section. However, to keep the paper simple, we did not attempt to state the results
in the greatest possible generality.


Our construction makes use of the following partial order introduced in~\cite{Hubicka2020CS} (which is closely tied to
the
order of 1-types within a fixed enumeration of $\str{P}$ discussed in Section~\ref{sec:1types}).
Let $$\Sigma=\{\L,\X,\R\}$$ be an \emph{alphabet} ordered by $\lexlt$ as
$$\L\lexlt \X\lexlt \R.$$  We denote by
$\Sigma^*$ the set of all finite words in the alphabet $\Sigma$, by
$\lexleq$ their lexicographic order, and by $|w|$ the length of the word $w = w_0w_1\cdots w_{|w|-1}$. We denote the empty word by $\emptyset$. Given words $w,w'\in \Sigma^*$, we write $w\sqsubseteq w'$ if $w$ is an initial segment of $w'$.
\begin{definition}[Partial order $(\Sigma^*,\preceq)$]
	For $w,w'\in \Sigma^*$, we set $w\prec w'$ if and only if there exists $i$ such that:
	\begin{enumerate}
		\item  $0\leq i<\min(|w|,|w'|)$,
		\item $(w_i,w'_i)=(\L,\R)$, and
		\item $w_j\leq_{\mathrm{lex}}w'_j$ for every $0\leq j< i$.
	\end{enumerate}
\end{definition}

We call the least $i$ satisfying the condition above the \emph{witness} of $w\prec w'$ and denote it by $i(w,w')$. We say that $w\preceq w'$ if either $w\prec w'$ or $w=w'$.
\begin{proposition}[\cite{Hubicka2020CS}]
	\label{prop:pos}
	The pair $(\Sigma^*,\preceq)$ is a partial order and  $(\Sigma^*,\lexleq)$ is a linear extension of it.
\end{proposition}
This partial order will serve as the main tool for characterising the big Ramsey degrees of $\str{P}$.
The intuition for its definition is described in Section~\ref{sec:1types}.
\begin{proof}[Proof of Proposition~\ref{prop:pos}]
	It is easy to see that $\preceq$ is reflexive and anti-symmetric.  We verify transitivity. Let $w\prec w'\prec w''$
	and put $i=\min(i(w,w'),i(w',w''))$.

	First assume that $i=i(w,w')$.  Then we have $w_i=\L, w'_i=\R$ which implies that $w''_i=\R$.
	For every $0\leq j<i$ it holds that $w_j\leq_{\mathrm{lex}}w'_j\leq_{\mathrm{lex}}w''_j$.
	It follows, by the transitivity of $\leq_{\mathrm{lex}}$, that $w\prec w''$ and $i(w,w'')$ exists with $i(w,w'')\leq i$.

	Now assume that $i=i(w',w'')$. Then we have $w'_i=\L$, $w''_i=\R$, and as $w'_i=\L$ we also have that $w_i=\L$.
	Again, for every $0\leq j<i$ it holds that $w_j\leq_{\mathrm{lex}}w'_j\leq_{\mathrm{lex}}w''_j$.
	Similarly as above, it also follows that $w\preceq w''$ and $i(w,w'')\leq i$.
\end{proof}


Given a word $w$ and an integer $i \geq 0$, we denote by $w|_i$ the \emph{initial segment} of $w$ of length $i$.
For $S\subseteq \Sigma^*$, we let $\overline{S}$ be the set $\{w|_i\colon w\in S, 0\leq i\leq |w|\}$.
Given $\ell\geq 0$, we let $S_\ell=\{w\in S:|w|=\ell\}$ and call it the \emph{level $\ell$ of $S$}. When writing $\overline{S}_\ell$, we always mean level $\ell$ of $\overline{S}$. 
A word $w\in S$ is called a \emph{leaf} of $S$ if there is no word $w'\in S$ extending $w$.
Given a word $w$ and a character $c\in \Sigma$, we denote by $w\cont c$
the word created from $w$ by adding $c$ to the end of $w$. We also set $S\cont c=\{w\cont c:w\in S\}$.


In addition to the partial order $\preceq$ we define the following relation on each $\Sigma^*_\ell$, $\ell\geq 0$,   where $\Sigma^*_\ell$ denotes the set of words of length $\ell$ in the alphabet $\Sigma$.
\begin{definition}[Partial orders $(\Sigma^*_\ell,\eltleq)$]
	Given $\ell>0$ and words $w,w'\in \Sigma^*_\ell$, we write $w\eltleq w'$ if 
	$w_i\lexleq w'_i$ for every $0\leq i<\ell$ (this is the usual element-wise partial order). We write $w\perp w'$ if $w$ and $w'$ are $\eltleq$-incomparable (that is, if neither of $w\eltleq w'$ nor $w'\eltleq w$ holds). We call $w$ and $w'$ \emph{related} if one of expressions $w\preceq w'$, $w'\preceq w$ or $w\perp w'$ holds, otherwise they are \emph{unrelated}.
\end{definition}


Intuitively, $s\eltleq t$ describes those pairs of nodes on the same level where it is possible to extend $s$ and $t$ to nodes with $s'\prec t'$. However, observe that (somewhat counter-intuitively)
it can happen that both $\preceq$ and $\perp$  hold at the same time: For example, we have both that $\L\R\preceq \R\L$ as well as $\L\R\perp \R\L$. Later, we will introduce the notion of compatibility to help us handle these situations.
While it is easy to check that $(\Sigma_\ell^*,\eltleq)$ is a partial order for every $\ell\geq 0$, we do not extend it to all of $\Sigma^*$. 

To characterise big Ramsey degrees of $\str{P}$, we introduce the following  technical definition, our main theorem then characterizes the big Ramsey degree of a given poset in $\str P$ as the number of poset-diaries of a certain kind. This definition has a good intuition behind it which will be explained in Section~\ref{sec:1types}.

\begin{definition}[Poset-diaries]
	\label{def:posetdiary}
	A set $S\subseteq \Sigma^*$ is called a \emph{poset-diary} if no member of $S$ extends any other (i.e.,\ $S$ is an antichain in $(\Sigma^*, \sqsubseteq)$) and precisely one of the following four conditions is satisfied for every level $\ell$ with $0\leq \ell< \sup_{w\in S}|w|$:
	\begin{enumerate}
		\item \textbf{Leaf:}  There is $w\in \overline{S}_\ell$ related to every $u\in \overline{S}_\ell\setminus\{w\}$ and
		      \begin{align*}
			      \qquad \overline{S}_{\ell+1} & =(\overline{S}_\ell\setminus \{w\} )\cont \X.
		      \end{align*}
		\item \textbf{Splitting:}  There is $w\in \overline{S}_\ell$ such that
		      \begin{align*}
			      \begin{split}
				      \qquad \overline{S}_{\ell+1}&=\{z\in \overline{S}_\ell:z\lexlt w\}\cont \X\\
				      &\qquad \cup\{w\cont \X,w\cont\R\}\\
				      &\qquad \cup \{z\in \overline{S}_\ell:w\lexlt z\}\cont \R.
			      \end{split}
		      \end{align*}
		\item \textbf{New $\perp$:} There are unrelated words $v\lexlt w\in \overline{S}_\ell$ such that the following is satisfied.
		      \begin{enumerate}[label=(\Alph*)]
			      \item\label{A2} For every $u\in \overline{S}_\ell$, $v\lexlt u\lexlt w$ implies that at least one of $u\perp v$ or $u\perp w$ holds. 
		      \end{enumerate}
			Moreover:
		      \begin{align*}
			      \begin{split}
				      \qquad \overline{S}_{\ell+1}&=\{z\in \overline{S}_\ell:z\lexlt v\}\cont \X\\
				      &\qquad \cup \{v\cont \R\}\\
				      &\qquad \cup\{z\in \overline{S}_\ell:v\lexlt z\lexlt w\hbox{ and }z\perp v\}\cont \X\\
				      &\qquad \cup\{z\in \overline{S}_\ell:v\lexlt z\lexlt w\hbox{ and } z\not\perp v\}\cont \R\\
				      &\qquad \cup \{{w}\cont \X\}\\
				      &\qquad \cup \{z\in \overline{S}_\ell:w\lexlt z\}\cont \R.
			      \end{split}
		      \end{align*}
		\item \textbf{New $\prec$:}  There are unrelated words $v\lexlt w\in \overline{S}_\ell$ such that the following is satisfied.
		      \begin{enumerate}[label=(B\arabic*)]
			      \item\label{B1} For every $u\in \overline{S}_\ell$ such that $u\lexlt v$, at least one of $u\preceq w$ or $u\perp v$ holds.
			      \item\label{B2} For every $u\in \overline{S}_\ell$ such that $w\lexlt u$, at least one of $v\preceq u$ or $w\perp u$ holds.
		      \end{enumerate}
			Moreover:
		      \begin{align*}
			      \begin{split}
				      \qquad \overline{S}_{\ell+1}&=\{z\in \overline{S}_\ell:z\lexlt v\hbox{ and }z\perp v\}\cont \X \\
				      &\qquad \cup \{z\in \overline{S}_\ell:z\lexlt v\hbox{ and }z\not \perp v\}\cont \L\\
				      &\qquad \cup \{v\cont \L\}\\
				      &\qquad \cup\{z\in \overline{S}_\ell:v\lexlt z\lexlt w\}\cont \X\\
				      &\qquad \cup \{{w}\cont \R\}\\
				      &\qquad \cup \{z\in \overline{S}_\ell:w\lexlt z\hbox{ and }w\perp z\}\cont\X\\
				      &\qquad \cup \{z\in \overline{S}_\ell:w\lexlt z\hbox{ and }w\not\perp z\}\cont\R.
			      \end{split}
		      \end{align*}
	\end{enumerate}
	\begin{figure}
		\centering
		\includegraphics{poset-types-levels.pdf}
		\caption{Possible levels in poset-diaries}
		\label{fig:levels}
	\end{figure}
	See also Figure~\ref{fig:levels}.
\end{definition}

Given a countable partial order $\str{Q}$, we let $T(\str{Q})$ be the set of all poset-diaries $S$ such that $(S,\preceq)$ is isomorphic to $\str{Q}$.
\begin{example}
	Denote by $\str{A}_n$ the anti-chain  with $n$ vertices and by $\str{C}_n$
	the chain with $n$ vertices.
	\begin{align*}
		T(\str{A}_1)=T(\str{C}_1) & =\{\emptyset\},                       \\
		T(\str{A}_2)              & =\{\{\X\R,\R\X\X\},\{\X\R\X,\R\X\}\}, \\
		T(\str{C}_2)              & =\{\{\X\L,\R\R\X\},\{\X\L\X,\R\R\}\}.
	\end{align*}
	\begin{figure}
		\centering
		\includegraphics[page=1]{smalltypes.pdf}
		\caption{Diaries of $\str{A}_2$.}
		\label{fig:types1}
	\end{figure}
	\begin{figure}
		\centering
		\includegraphics[page=2]{smalltypes.pdf}
		\caption{Poset-diaries of $\str{C}_2$.}
		\label{fig:types2}
	\end{figure}
	In all diaries in $T(\str{A}_2)\cup T(\str{C}_2)$, level 0 does splitting, level 1 adds a new $\perp$ or $\prec$, and levels 2 and 3 are leaves.

	Poset-diaries of small partial orders can be determined by an exhaustive search
	tool.\footnote{\url{https://github.com/janhubicka/big-ramsey}}
		 We determined that $|T(\str{C}_3)|=52$, $|T(\str{C}_4)|=11000$ and
	$|T(\str{A}_3)|=84$, $|T(\str{A}_4)|=75642$.  Overall there are:
	\begin{enumerate}
		\item 1 poset-diary of the (unique) partial order of size 1: $S=\{\emptyset\}$,
		\item 4 poset-diaries of partial orders of size 2: $T(\str{A}_2)\cup T(\str{C}_2)$,
		\item 464 poset-diaries of partial orders of size 3,
		\item 1874880 poset-diaries of partial orders of size 4.
	\end{enumerate}
\end{example}
As our main result, we determine the big Ramsey degrees of $\str{P}$ and show that $\str{P}$ admits a big Ramsey structure; while we refer to \cite{zucker2017} for the precise definition, a big Ramsey structure for $\str{P}$ is an expansion $\str{P}^*$ of $\str{P}$ which encodes the exact big Ramsey degrees for all finite substructures simultaneously in a coherent fashion. 

\begin{theorem}
\label{thm:main}
	For every finite partial order $\str{Q}$, the big Ramsey degree of $\str{Q}$ in the generic partial order $\str{P}$ equals $|T(\str{Q})|\cdot |\mathrm{Aut}(\str{Q})|$. Furthermore, any $\str{P}^*\in T(\str{P})$ is a big Ramsey structure for $\str{P}$. Consequently, the topological group $\mathrm{Aut}(\str{P})$ admits a metrizable universal completion flow.
\end{theorem}
Note that the number of poset-diaries is multiplied by the size of the automorphism group since we
define big Ramsey degrees with respect to embeddings (as done, for example, in~\cite{zucker2017}).  
 Big Ramsey degrees are often
defined with respect to substructures (see, for example, \cite{devlin1979,Laflamme2006,larson2008counting,NVT2009,laflamme2010partition}) and in that case the degree would be $|T(\str{Q})|$.
\begin{example}
	\begin{align*}
		|T(\str{A}_1)|\cdot |\mathrm{Aut}(\str{A}_1)|=|T(\str{C}_1)|\cdot |\mathrm{Aut}(\str{C}_1)| & =1,       \\
		|T(\str{A}_2)|\cdot |\mathrm{Aut}(\str{A}_2)|                                               & =4,       \\
		|T(\str{C}_2)|\cdot |\mathrm{Aut}(\str{C}_2)|                                               & =2,       \\
		|T(\str{A}_3)|\cdot |\mathrm{Aut}(\str{A}_3)|                                               & =504,     \\
		|T(\str{C}_3)|\cdot |\mathrm{Aut}(\str{C}_3)|                                               & =52,      \\
		|T(\str{A}_4)|\cdot |\mathrm{Aut}(\str{A}_4)|                                               & =1816128, \\
		|T(\str{C}_4)|\cdot |\mathrm{Aut}(\str{C}_4)|                                               & =11000.   \\
	\end{align*}
\end{example}


\section{Preliminaries}\label{sec:preliminaries}
\subsection{Relational structures}
We use the standard model-theoretic notion of relational structures.
Let $L$ be a language with relational symbols $\rel{}{}\in L$ each equipped with a positive natural number called its {\em arity}.
An \emph{$L$-structure} $\str{A}$ on $A$ is a structure with {\em vertex set} $A$ and  relations $\rel{A}{}\subseteq A^r$ for every symbol $\rel{}{}\in L$ of arity $r$.  If the set $A$ is finite we call $\str A$ a \emph{finite structure}. We consider only structures with finitely many or countably infinitely many vertices.

Since we work with structures in multiple languages, we will list the vertex
set along with the relations of the structure, e.g., $(P,\leq)$ for partial orders.
\subsection{Trees}
For us, a \emph{tree} is a (possibly empty) partially ordered set $(T, <_T)$ such
that, for every $t \in T$, the set $\{s \in T : s <_T t \}$ is finite and linearly ordered by $<_T$.
All nonempty trees we consider are \emph{rooted}, that is, they have a unique minimal element called the \emph{root} of the tree.
An element $t\in T$ of a tree $T$ is called a \emph{node} of $T$ and its \emph{level},
denoted by $\card{t}_T$, is the size of the set $\{s \in T : s <_T t\}$.
Note that the root has level~0.
Given a tree $T$ and nodes $s, t \in T$, we say that $s$ is a \emph{successor} of $t$ in $T$ if $t \leq_T s$.
A \emph{subtree} of a tree $T$ is a subset $T'$ of $T$ viewed as a tree
equipped with the induced partial ordering.

Given words $w,w'\in \Sigma^*$, we write $w\sqsubseteq w'$ if $w$ is an initial segment of $w'$. With this partial order we obtain tree $(\Sigma^*,\sqsubseteq)$
and the notation on words introduced in Section~\ref{sec:introduction} can be viewed as a special case
of the notation introduced here.
\subsection{Parameter words}
To obtain the upper bounds on big Ramsey degrees of $\str{P}$ we apply a Ramsey theorem for parameter words which we briefly review now.

Given a finite alphabet $\Alphabet$ and $k\in \omega\cup \{\omega\}$, a \emph{$k$-parameter word} is a (possibly infinite) string $W$ in the
alphabet $\Alphabet\cup \{\lambda_i\colon 0\leq i<k\}$ containing all symbols ${\lambda_i : 0\leq i < k}$ such that, for every $1\leq j < k$, the first
occurrence of $\lambda_j$ appears after the first occurrence of $\lambda_{j-1}$.
The symbols $\lambda_i$ are called \emph{parameters}.
We will use uppercase characters to denote parameter words and lowercase characters for  words
without parameters.
Let $W$ be an $n$-parameter word and let $U$ be a parameter word of length $k\leq n$, where $k,n\in \omega\cup\{\omega\}$. Then
$W(U)$ is the parameter word created by \emph{substituting} $U$ to $W$. More precisely, $W(U)$ is created from~$W$ by replacing each occurrence of $\lambda_i$, $0\leq i < k$, by $U_i$ and truncating it just
before the first occurrence of $\lambda_k$ in $W$.

We apply the following Ramsey theorem for parameter words, which is  an easy consequence of the Carlson--Simpson theorem\cite{carlson1984}, see also \cite{todorcevic2010introduction,Karagiannis2013}:
\begin{theorem}
	\label{thm:CS}
	Let $\Alphabet$ be a finite alphabet.
	If $\Alphabet^*$ is coloured with finitely many colours, then there exists an infinite-parameter word
	$W$ such that $W[\Alphabet^*]=\{W(s)\colon s\in \Alphabet^*\}$ is monochromatic.
\end{theorem}

\section{Tree of 1-types}\label{sec:1types}

Poset-diaries, which can be compared to \emph{Devlin
	embedding types} (see Chapter 6.3 of \cite{todorcevic2010introduction}) or the \emph{diagonal diaries} introduced in \cite{Balko2021exact}, have an intuitive meaning when
understood in the context of the tree of 1-types of $\str{P}$. We now introduce this tree and its enrichment to an \emph{aged coding tree} and discuss how poset-diaries can be obtained as a suitable abstraction of the aged coding tree. 

An \emph{enumerated structure} is simply a structure $\str{A}$ whose underlying set is $|\str{A}|$. Fix  a countably infinite enumerated structure $\str{A}$.  Given vertices $u,v$ and an integer $n$ satisfying $\min(u,v)\geq n\geq 0$, we write $u\sim^\str{A}_n v$, and say that \emph{$u$ and $v$ are of the same (quantifier-free) type over $0,1,\ldots,n-1$}, if the structure induced by $\str{A}$ on $\{0,1,\ldots, n-1,u\}$ is identical to the structure induced by $\str{A}$ on $\{0,1,\ldots, n-1,v\}$ after renaming vertex $v$ to $u$. We write $[u]^\str{A}_n$ for the $\sim^\str{A}_n$-equivalence class of vertex $u$.
\begin{definition}[Tree of 1-types]
	Let $\str{A}$ be a countably infinite (relational) enumerated structure. Given $n< \omega$, write $\bb{T}_\str{A}(n) = \omega/\!\sim^{\str{A}}_n$. A (quantifier-free) \emph{1-type} is any member of the disjoint union $\bb{T}_\str{A}:=\bigsqcup_{n<\omega} \bb{T}_\str{A}(n)$. We turn $\bb{T}_\str{A}$ into a tree as follows. Given $x\in \bb{T}_\str{A}(m)$ and $y\in \bb{T}_\str{A}(n)$, we declare
	that $x\leq^{\mathbb T}_{\str{A}} y$ if and only if $m\leq n$ and $x\supseteq y$. 
	
	In the case that we have a \fr class $\cal{K}$ in mind (which for us will always be the class of finite partial orders), we can extend the definition to a finite enumerated $\str{A}\in \cal{K}$ as follows. Fix an enumerated \fr limit $\str{K}$ of $\cal{K}$ which has $\str{A}$ as an initial segment. We then set $\bb{T}_\str{A} = \bb{T}_\str{K}({<}|\str{A}|)$. This does not depend on the choice of $\str{K}$.
\end{definition}

In the case that $\str{A}$ is a structure in a finite binary relational language, we can encode $\bb{T}_{\str{A}}$ as a subtree of $k^{<\omega}$ for some $k< \omega$ as follows. Given two enumerated structures $\str{B}$ and $\str{C}$, an \emph{ordered embedding} of $\str{B}$ into $\str{C}$ is any embedding of $\str{B}$ into $\str{C}$ which is an increasing injection of the underlying sets. Write $\mathrm{OEmb}(\str{B}, \str{C})$ for the set of ordered embeddings of $\str{B}$ into $\str{C}$. Fix once and for all an enumeration $\{\str{B}_i: i< k\}$ of the set of enumerated structures of size $2$ which admit an enumerated embedding into $\str{A}$. Given $x\in \bb{T}_\str{A}(m)$, we define $\sigma(x)\in k^m$ where given $j< m$, we set $x(j) = i$ iff for some (equivalently every) $n\in x$ there is $f\in \mathrm{OEmb}(\str{B}_i, \str{A})$ with $\mathrm{Im}(f) = \{m, n\}$. The map $\sigma\colon \bb{T}_\str{A}\to k^{< \omega}$ is then an embedding of trees. We write $\mathrm{CT}^\str{A} = \sigma[\bb{T}_\str{A}]$ and call this the \emph{coding tree} of $\str{A}$. Typically we also endow $\mathrm{CT}^\str{A}$ with \emph{coding nodes}, where for each $n$, the $n^{th}$ coding node of $\mathrm{CT}^\str{A}$ is defined to be $c^\str{A}(n) := \sigma([n]^\str{A}_n)$.

Understanding of the tree of 1-types of a given structure is useful for constructing  unavoidable colorings as well as for showing upper bounds on big Ramsey degrees; see for instance 
\cite{Laflamme2006,
dobrinen2017universal,
dobrinen2019ramsey,
coulson2022indivisibility,coulson2022SDAP,zucker2020,Hubicka2020CS,Hubickabigramsey,Hubicka2020uniform,Balko2021exact}.
We therefore fix an enumerated generic partial order $\str{P}$, and we put $(\mathbb T,\leq_\mathbb T\nobreak )=(\mathbb T_\str{P},\leq^\mathbb T_\str{P})$ and $\mathrm{CT} = \mathrm{CT}^\str{P}$. By identifying the symbols $\{\L, \X, \R\}$ with $\{0, 1, 2\}$, we can identify $\mathrm{CT}$ as a subtree of $(\Sigma^*, \sqsubseteq)$. More concretely, given $x\in \bb{T}(m)$ and $j< m$, we have:
$$\sigma(x)_j=\begin{cases}\L & \emph{if $a<_\str{P}j$ for every (some) $a\in x$,} \\
             \R & \emph{if $j<_\str{P}a$ for every (some) $a\in x$,} \\
             \X & \emph{otherwise.}
	\end{cases}
$$
	\begin{figure}
		\centering
		\includegraphics{types.pdf}
		\caption{Initial part of the tree of types of an enumerated linear order $(\str{Q},\leq_\str{Q})$ (left) and of the enumerated partial order $\str{P}$ (right). The bold node on each level corresponds to the coding node.}
		\label{fig:types}
	\end{figure}

We remark that $\ct$ is a proper subset of $\Sigma^*$, no matter the enumeration we choose.  For example if $\L\R\in \ct$ then $\R\L\notin \ct$ since that would imply an existence of vertices $a,b\in \str{P}$ such that $a<_\str{P} 0<_\str{P}b$ and $b<_\str{P} 1<_\str{P}a$, contradicting the fact that $\str{P}$ is a partial order.

The relations $\prec$, $\lexlt$ and $\perp$, introduced in Section~\ref{sec:introduction}, capture the following properties of types:
\begin{proposition}
	\label{prop:1typerel}
	Let $x\in \bb{T}(m)$ and $y\in \bb{T}(n)$ be 1-types of $\str{P}$.
	\begin{enumerate}[label=(\arabic*)]
		\item\label{prop:1typerel:p0} If there exist $a\in x$ and $b\in y$ satisfying $a<_\str{P} b$, then for every $\ell<\min(m,n)$ it holds that $\sigma(x)_\ell\lexleq \sigma(y)_\ell$.
		\item\label{prop:1typerel:p1} If $\sigma(x)\prec \sigma(y)$, then for every $a\in x$ and $b\in y$ it holds that $a\leq_\str{P} b$.
		\item\label{prop:1typerel:p2} If $\sigma(x)\lexlt \sigma(y)$, then for every $a\in x$ and $b\in y$ it holds that $b\nleq_\str{P} a$.
		\item\label{prop:1typerel:p3} If $m = n$ and $\sigma(x)\perp \sigma(y)$, then for every $a\in x$ and $b\in y$ it holds that $a$ and $b$ are $\leq_\str{P}$ incomparable.
	\end{enumerate}
\end{proposition}
\begin{proof}
	We first verify \ref{prop:1typerel:p0} by contrapositive. Assume there is $\ell<\min(m,n)$ such that $\sigma(y)_\ell \lexlt \sigma(x)_\ell$.
	First consider the case that $(\sigma(x)_\ell,\sigma(y)_\ell)=(\X,\L)$. It follows for any $a\in x$ and $b\in y$ that $a$ is $\leq_\str{P}$-incomparable with $\ell$ and $b<_\str{P} \ell$. It follows that we cannot have $a<_\str{P} b$. The arguments in the cases $(\sigma(x)_\ell,\sigma(y)_\ell)=(\R,\L)$ and $(\sigma(x)_\ell,\sigma(y)_\ell)=(\R,\X)$ are similar.

	To see \ref{prop:1typerel:p1} observe that $\sigma(x)\prec \sigma(y)$ implies the existence of a vertex $\ell\in \str{P}$ satisfying $\ell<\min(m,n)$ and $(\sigma(x)_\ell,\sigma(y)_\ell)=(\L,\R)$. It follows that for any $a\in x$ and $b\in y$ we have $a<_\str{P} \ell<_\str{P} b$.

	To verify \ref{prop:1typerel:p2} observe that there exists $\ell<\min(m, n)$ such that $\sigma(x)_\ell\lexlt \sigma(y)_\ell$. Thus we cannot have $a\in x$, $b\in y$ such that $b<_\str{P} a$, as this would contradict \ref{prop:1typerel:p0}.

	Finally to verify \ref{prop:1typerel:p3} observe that $\sigma(x)\perp \sigma(y)$ implies the existence of vertices $k, \ell\in \str{P}$ satisfying $\max(k, \ell)<\min(m,n)$, $\sigma(x)_k\lexlt \sigma(y)_k$ and $\sigma(y)_\ell\lexlt \sigma(x)_\ell$.
	Hence the existence of $a\in x$ and $b\in y$ with either $a<_\str{P} b$ or $b<_\str{P} a$ contradicts \ref{prop:1typerel:p0}.
\end{proof}

The main difficulty while working with the tree $(\mathbb T,\leq_\mathbb T)$ is the fact that it depends on the choice of an enumeration of $\str{P}$.
For this reason we will focus on the tree $(\Sigma^*,\sqsubseteq)$ which can be seen as an amalgamation of all possible trees $(\mathbb T,\leq_\mathbb T)$ constructed
using all possible enumerations of $\str{P}$.
The next definition captures the main properties of words in $\ct$ which are independent of the choice of enumeration of $\str{P}$.
\begin{definition}[Compatibility]
	\label{defn:comp}
	Words $u\lexleq v\in \Sigma^*$ are \emph{compatible} if the following two conditions are
	satisfied:
	\begin{enumerate}[label=(\arabic*)]
		\item\label{defn:comp:p1} there is no $\ell<\min(|u|,|v|)$ such that $(u_\ell,v_\ell)=(\R,\L)$, and
		\item\label{defn:comp:p2} if there exists $\ell'<\min(|u|,|v|)$ such that $(u_{\ell'},v_{\ell'})=(\L,\R)$, then for every $\ell''<\min(|u|,|v|)$ it holds that $u_{\ell''}\lexleq v_{\ell''}$.
	\end{enumerate}
\end{definition}
\begin{proposition}
	For every $s, t\in \ct$  it holds that $s, t$ are compatible.
\end{proposition}
\begin{proof}
	Suppose $x, y\in \bb{T}$ are such that $\sigma(x) \lexlt \sigma(y)$.
	To see property \ref{defn:comp:p1} of Definition~\ref{defn:comp},
	suppose there were $\ell<\min(i,j)$ such that $(\sigma(x)_\ell,\sigma(y)_\ell)=(\R,\L)$. This implies that $b<_\str{P} \ell \leq_\str{P} a$ for every $a\in x$ and $b\in y$
	which contradicts Proposition~\ref{prop:1typerel}~\ref{prop:1typerel:p2}.

	Property \ref{defn:comp:p2} of Definition~\ref{defn:comp} is a consequence of Proposition~\ref{prop:1typerel}~\ref{prop:1typerel:p0} and the fact that the existence of
	$\ell'$ such that $(\sigma(x)_{\ell'},\sigma(y)_{\ell'})=(\L,\R)$ implies that $a<_\str{P} \ell<_\str{P} b$ for every $a\in x$ and $b\in y$.
\end{proof}
In \cite{zucker2020}, using ideas implicit in the parallel $1$'s of \cite{dobrinen2017universal} and pre-$a$-cliques of \cite{dobrinen2019ramsey}, levels of coding trees are endowed with the structure of \emph{aged sets}. This means that for every $m< \omega$, every set $S\subseteq \ct(m)$ is equipped with a class of finite $|S|$-labeled structures describing exactly which finite structures can be coded by coding nodes above the members of $S$. For the generic partial order, it will be useful to encode this information slightly differently than in \cite{zucker2020,Balko2021exact}, in particular since we want to do this on all of $\Sigma^*$, not just on $\ct$. 
\begin{definition}[Level structure]
	Given $\ell\geq 0$ and $S\subseteq \Sigma^*_\ell$, the \emph{level structure} is the structure $\str{S}=\AmbStr{S}$. 
\end{definition}
\begin{proposition}
	\label{prop:orders}
	For every $\ell>0$ and $S\subseteq \Sigma^*_\ell$, the structure $\str{S}=\AmbStr{S}$ satisfies the following properties:
	\begin{enumerate}[label=(P\arabic*)]
		\item\label{P1} $(S,\preceq)$ is a partial order.
		\item\label{P2} $(S,\eltleq)$ is a partial order.
		\item\label{P3} $(S,\lexleq)$ is a linear order.
		\item\label{P4} For every $u,v\in S$ it holds that $u\preceq v\implies u\lexleq v$ ($\lexleq$ is a linear extension of $\preceq$).
		\item\label{P5} For every $u,v\in S$ it holds that $u\eltleq v\implies u\lexleq v$ ($\lexleq$ is a linear extension of $\eltleq$).
		\item\label{P6} For every $u,v,w\in S$ it holds that $u\preceq v\eltleq w\implies u\preceq w$ and $u\eltleq v\preceq w\implies u\preceq w$.
	\end{enumerate}
	Moreover if all words in $S$ are compatible then
	\begin{enumerate}[label=(P\arabic*),resume]
		\item\label{P7} For every $u,v\in S$ it holds that $u\preceq v\implies u\eltleq v$.
	\end{enumerate}
	\begin{proof}
		Properties \ref{P1} and \ref{P4} are Proposition~\ref{prop:pos}. \ref{P2}, \ref{P3}
		\ref{P5} and \ref{P6} follow directly from the definitions. \ref{P7} is Definition~\ref{defn:comp} \ref{defn:comp:p2}.
	\end{proof}
\end{proposition}
Level structures can be understood as approximations of a given partial order
with a given linear extension from below (using order $\preceq$) and from above
(using $\eltleq$).  This is a natural analog of the age-set structure in $\ct$.

\begin{remark}
	One can, perhaps surprisingly, prove that the class $\mathcal K$ of all finite
	structures satisfying properties \ref{P1}, \ref{P2}, \ldots, \ref{P7} is an
	amalgamation class. As a consequence of the construction from Section~\ref{sec:typeP} one gets
	that for each structure $\str A\in\cal{K}$ there exists $\ell>0$ and $S\subseteq \Sigma^*_\ell$
	such that $\str{S}=\AmbStr{S}$ is isomorphic to $\str{A}$.
\end{remark}

\begin{remark}
	This interesting phenomenon of constructing a class of approximations (or, using the terminology of~\cite{zucker2020,Balko2021exact},  the class of all possible aged sets that can appear on some level of the coding tree) of a
	given amalgamation class exists in other cases. For binary free amalgamation classes, this approximation class corresponds to the union $\bigcup_{\rho} P(\rho)$, where the union is taken over all possible \emph{sorts} $\rho$ (see \cite{Balko2021exact} for the definitions). However, the theory of aged coding trees and the sets $P(\rho)$ can be defined for any strong amalgamation class in a finite binary language.
	Note that while for free amalgamation classes the set $P(\rho)$ is always closed under intersections, this need not be the case in general (indeed, it fails for posets).
	
	
	Another key difference between the free amalgamation case and partial orders is that for free amalgamation classes, we can arrange so that going up and left (that is, by a non-relation) in the coding tree is a safe move, i.e.,\ is an embedding of the level structure from one level to another. Indeed, if this were true for the generic partial order and the coding tree $\ct$ we fixed earlier, one could prove upper bounds for the big Ramsey degrees using forcing arguments much as is done for the free amalgamation case in \cite{zucker2020}.  However, while a weakening of the idea of a ``safe direction" does hold for partial orders (see Proposition~\ref{prop:typeextend}), the proof of Lemma 3.4 from \cite{zucker2020} breaks in the setting of the generic partial order. However, it is possible that the coding tree Milliken theorem still holds. Below, $\mathrm{AEmb}(\ct^\str{A}, \ct)$ refers to the set of \emph{aged embeddings} of the coding tree $\ct^\str{A}$ into $\ct$, the strong similarity maps that respect coding nodes and level structures (see Definition 2.3 of \cite{zucker2020}).
\end{remark}
	
	
	\begin{question}
	    Fix a finite partial order $\str{A}$. Let $r< \omega$ and let $\gamma\colon \mathrm{AEmb}(\ct^\str{A}, \ct)\to r$ be a coloring. Is there $h\in \mathrm{AEmb}(\ct, \ct)$ such that $h\circ \mathrm{AEmb}(\ct^\str{A}, \ct)$ is monochromatic? 
	\end{question}


\section{Poset-diaries and level structures}
Given a poset-diary $S$, one can view $\overline{S}$ as a binary branching tree and each level $\overline{S}_\ell$ as a structure $\str{S}_\ell=\AmbStr{\overline{S}_\ell}$
where the structure $\str{S}_{\ell+1}$ is constructed from the structure $\str{S}_{\ell}$ as described in the following proposition.



\begin{proposition}
\label{prop:levelstr}
	Let $S$ be a poset-diary. Then all words in $\overline{S}$ are mutually compatible, and for each  $\ell\leq \sup_{w\in S}|w|$ the structures $\str{S}_\ell=\AmbStr{\overline{S}_\ell}$ and
	$\str{S}_{\ell+1}=\AmbStr{\overline{S}_\ell}$ are related as follows:
	\begin{enumerate}
		\item If $\overline{S}_\ell$ introduces  a new leaf, then $\str{S}_{\ell+1}$ is isomorphic to $\str{S}_\ell$ with one vertex removed.
		\item If $\overline{S}_\ell$ is splitting, then $\str{S}_{\ell+1}$ 
 is isomorphic to $\str{S}_\ell$ with one vertex $v$ duplicated to $v',v''$ with $v'\lexlt v''$, $v'\not\preceq v''$, $v''\not \preceq v'$,  and $v'\eltleq v''$.
		\item If $\overline{S}_\ell$ has a  new $\perp$, then $\str{S}_{\ell+1}$ is isomorphic to $\str{S}_\ell$ with one pair removed from   relation $\eltlt$ (and thus one pair added to $\perp$).
		\item If $\overline{S}_\ell$ has a new $\preceq$, then $\str{S}_{\ell+1}$ is isomorphic to $\str{S}_\ell$ extended by one pair in relation $\prec$.
	\end{enumerate}
\end{proposition}
To prove Proposition~\ref{prop:levelstr}, we note the following easy observation.
\begin{observation}
	\label{obs:safeext}
	Let $u\lexlt v\in \Sigma^*_i$ for some $i\geq 0$ and $c,c'\in \Sigma$ such that $c\lexleq c'$ and $(c,c')\neq (\L,\R)$. Then
	\begin{enumerate}
		\item $u\preceq v\iff v\cont c\preceq u\cont c'$,
		\item $u\perp v\iff u\cont c\perp v\cont c'$,
		\item if $u$ and $v$ are compatible then $u\cont c$ and $v\cont c'$ are compatible.
	\end{enumerate}
\end{observation}
\begin{proof}[Proof of Proposition~\ref{prop:levelstr}]
	Fix  a poset-diary $S$ and level $\ell<\sup_{w\in S}|w|$ and consider individual cases.
	\begin{enumerate}
		\item Leaf vertex $w$: We have that $|\str{S}_\ell|=|\str{S}_{\ell+1}|+1$ since $w$ is the only vertex of $\str{S}_\ell$ not extended to a vertex in $\str{S}_{\ell+1}$.  The desired isomorphism and mutual compatibility follows by Observation~\ref{obs:safeext}.
		\item Splitting of vertex $w$: Here vertex $w$ is the only vertex with two extensions. The desired isomorphism and mutual compatibility follows again by Observation~\ref{obs:safeext}.
		\item New $v\perp w$: Since $v\lexlt w$ are unrelated and thus $v\eltlt w$, we know that $v\cont \R$ and $w\cont \X$ are compatible and $v\cont \R\perp w\cont \X$ holds. Since we extended by letters $\X$ and $\R$ we know that there are no new pairs in relation $\preceq$.

		      Now assume, for contradiction, that there is $u\in \overline{S}_\ell\setminus\{v,w\}$ unrelated to $v$ but where the extension of $u$ in $\overline{S}_{i+1}$ is related to $v\cont \R\in \overline{S}_{\ell+1}$. Since all words lexicographically before $v$ are extended by $\X$ and all words lexicographically after $w$ by $\R$, by Observation~\ref{obs:safeext}, we conclude that $v\lexlt u\lexlt w$ and $u$ extends by $\X$. Mutual compatibility follows by analogous argument.
		\item New $v\preceq w$: Since $v\lexlt w$ are unrelated, we know that $v\cont \L$ and
		      $w\cont \R$ are compatible and $v\cont \L\preceq w\cont \R$ holds. To see that no
		      additional pair to relation $\perp$ was introduced, observe that for $u,u'\in \overline{S}_i$,
		      $(u,u')\neq (v,w)$ to be extended to
		      $u\cont \L$, ${u'} \cont \R$
			we have, by assumptions \ref{B1} and \ref{B2},  $u\preceq v$ and $w\preceq u'$. Since $v\lexlt w$ is unrelated we also have $v\eltleq v$. By Proposition~\ref{prop:orders}~\ref{P6} $u\preceq v\lexlt w\implies u\preceq v$ and thus also $u\preceq u'$.

		      It remains to consider the possibility that new pairs are added to relation $\perp$. We again consider individual cases.

		      First consider the case that $u$ is unrelated to $v$ but their extensions are newly in $\perp$. Since $v$ extends by $\L$ we know that $u\lexlt v$ and $v$ extends by $\X$. This contradicts construction of $\overline{S}_{i+1}$.

		      The case that $u$ is unrelated to $w$ but their extensions are newly in $\perp$ follows by symmetry.

			It thus remains to consider the case where $u\lexlt u'$, $u,u'\notin\{v,w\}$, are unrelated in $\overline{S}_i$,  however their extensions are related in $\overline{S}_{i+1}$. It is not possible for $u$ to extend by $\R$ and $u'$ by $\L$. So assume that $u$ extends by $\X$ and $v$ extends by $\L$ (the remaining case follows by symmetry). From this we conclude that $u\lexlt u'\lexlt v$, $u\perp v$ and $u'\not\perp v$. Since $u\not\perp u'$ we again obtain a contradiction with Proposition~\ref{prop:orders} \ref{P4} or \ref{P6}.

		      Mutual compatibility follows by analogous argument.

	\end{enumerate}
\end{proof}


\section{A poset-diary coding $\str{P}$}
\label{sec:typeP}
Recall that $\str{P} = (\omega, \leq_{\str{P}})$ denotes a fixed enumerated generic poset. 
We define a function $\varphi\colon \omega\to \Alphabet^*$ by mapping $j< \omega$ to a word $w$ of length $2j+2$ defined by
putting $(w_{2j},w_{2j+1})=(\L,\R)$ and, for every $i<j$, $(w_{2i},w_{2i+1})$ to $(\L,\L)$ if $j\leq_\str{P} i$, $(\R,\R)$ if $i\leq_\str{P} j$ and $(\X,\X)$ otherwise.
We set $T=\overline{\varphi[\omega]}$. The following result is easy to prove by induction.
\begin{proposition}[Proposition 4.7 of \cite{Hubicka2020CS}]
	The function $\varphi$ is an embedding $\str{P}\to (\Sigma^*,\preceq)$.
	More\-over, $\varphi(v)$ is a leaf of $T$ for every $v\in \str{P}$, all words in $T$ are mutually compatible, and if $v,w\in \str{P}$ are incomparable, we have $\varphi(v)\perp\varphi(w)$.
\end{proposition}

We will need the following refinement of this embedding.

\begin{theorem}
	\label{thm:posetemb}
	There exists an embedding $\psi\colon \str{P}\to (\Sigma^*,\preceq)$ such that
	$\psi[\omega]$ is a poset-diary.
\end{theorem}

\begin{proof}
	Fix the embedding $\varphi$ as above and put $T=\overline{\varphi[\omega]}$.
	We proceed by induction on levels of $T$.
	For every $\ell$, we define an integer $N_\ell$ and a function $\psi_\ell\colon T_\ell\to \Sigma^*_{N_\ell}$.
	We will maintain the following invariants:
	\begin{enumerate}
		\item The set $\overline{\psi_\ell[T_\ell]}$ satisfies the conditions of Definition~\ref{def:posetdiary} for all levels with the exception of $N_\ell-1$.
		\item If $\ell>0$, then, for every $u\in T_\ell$, the word $\psi_\ell(u)$ extends $\psi_{\ell-1}(u|_{\ell-1})$.
	\end{enumerate}

	We let $N_0=0$ and put $\psi_0$ to map the empty word to the empty word.
	Now, assume that $N_{\ell-1}$ and $\psi_{\ell-1}$ are already defined.
	We inductively define a sequence of functions $\psi^i_\ell\colon T_\ell\to \Sigma^*_{N_{\ell-1}+i}$. Put $\psi^0_\ell(u)=\psi_{\ell-1}(u|_{\ell-1})$.
	Now, we proceed in steps. At step $j$, apply the first of the following constructions that can be applied and terminate the procedure if none of them applies:
	\begin{enumerate}
		\item\label{posemb1} If $\psi_\ell^{j-1}$ is not injective, let $w\in T_\ell$ be lexicographically least so that $\psi^{j-1}_\ell(w) = \psi^{j-1}_\ell(x)$ for some $x\in T_\ell\setminus\{w\}$. Given $u\in T_\ell$, set $\psi_\ell^j(u) = \psi_\ell^{j-1}(u)^\frown X$ if $u\lexleq w$, and set $\psi_\ell^j(u) = \psi_\ell^{j-1}(u)^\frown R$ if $w\lexlt u$. Then this satisfies the conditions on new splitting at $\psi_{\ell}^{j-1}(w)$ as given in Definition~\ref{def:posetdiary}.
		\item\label{posemb2} If there are words $w$ and $w'$ from $T_\ell$ with $w\lexlt w'$ such that $w\perp w'$ and $\psi^{j-1}_\ell(w)\not \perp\psi^{j-1}_\ell(w')$ and condition~\ref{A2} of Definition~\ref{def:posetdiary} is satisfied for the value range of~$\psi^{j-1}_\ell$, we construct $\psi^j_\ell$ to satisfy the conditions on new $\perp$ for $\psi^{j-1}_\ell(w)$ and $\psi^{j-1}_\ell(w')$ as given by Definition~\ref{def:posetdiary}.
		\item\label{posemb3}  If there are words $w$ and $w'$ from $T_\ell$ with $w\lexlt w'$ such that $w\prec w'$ and $\psi^{j-1}_\ell(w)\not \prec\psi^{j-1}_\ell(w')$ and conditions~\ref{B1} and~\ref{B2} of Definition~\ref{def:posetdiary} are satisfied for the value range of~$\psi^{j-1}_\ell$, we construct $\psi^j_\ell$ to satisfy the conditions on new $\prec$ for $\psi^{j-1}_\ell(w)$ and $\psi^{j-1}_\ell(w')$ as given by Definition~\ref{def:posetdiary}.
	\end{enumerate}


	Let $J$ be the largest index for which for which $\psi^J_\ell$ is defined.
	\begin{claim}
	    $\psi^J_\ell$ is an isomorphism  $\AmbStr{T_\ell}\to
	\AmbStr{\psi^J_\ell[T_\ell]}$.
	\end{claim}
	
	\begin{proof}[Proof of claim]
	    Suppose, to the contrary, that this is not true. If $\psi^J_\ell$ is not a bijection, this means that there are $w,w' \in T_\ell$ such that $\psi^J_\ell(w) = \psi^J_\ell(w')$. But then the conditions in~(\ref{posemb1}) are satisfied, a contradiction with maximality of $J$. So $\psi^J_\ell$ is a bijection. Note that the steps of the construction ensure that $\psi^J_\ell$ respects $\lexlt$. We also have $\psi^J_\ell(w)\perp \psi^J_\ell(w')\implies w\perp w'$ and $\psi^J_\ell(w)\preceq \psi^J_\ell(w')\implies w\preceq w'$ for $w,w'\in T_\ell$.

        If there are $w,w' \in T_\ell$ such that $w\lexlt w'$, $w\perp w'$ and $\psi^J_\ell(w)\not \perp\psi^J_\ell(w')$, pick among all such pairs one minimizing $|\{u\in T_\ell: w\lexlt u\lexleq w'\}|$. Proposition~\ref{prop:orders} implies that the conditions in~(\ref{posemb2}) are satisfied, again a contradiction with maximality of $J$.

        So there are  $w,w' \in T_\ell$ such that $w\lexlt w'$, $w\prec w'$ and $\psi^J_\ell(w)\not \prec\psi^J_\ell(w')$, and we can assume that $w,w'$ maximize $|\{u\in T_\ell: w\lexlt u\lexleq w'\}|$. Proposition~\ref{prop:orders} implies that the conditions in~(\ref{posemb2}) are satisfied, again a contradiction with maximality of $J$. Hence indeed $\psi^J_\ell$ is an isomorphism  $\AmbStr{T_\ell}\to
		\AmbStr{\psi^J_\ell[T_\ell]}$. 
	\end{proof}
	 
  
 
	Finally, we put $N_\ell=|\psi^J_\ell(w)|$ for some $w\in T_\ell$ and $\psi_\ell=\psi^J_\ell$.
	Once all functions $\psi_\ell$ are constructed, we can set $\psi(i) = \psi_{2i+2}(\varphi(i))$. It is easy to verify that this is an embedding $\str{P}\to (\Sigma^*,\preceq)$ such that
	$\psi[\omega]$ is a poset-diary (if it was not it fails at some finite level $\ell$ but the construction ensures that every level adheres to the conditions of Definition~\ref{def:posetdiary}.
\end{proof}


\section{Interesting levels and sub-diaries}
We now aim to prove the upper-bound for big Ramsey degrees of $\str{P}$. Towards this direction we need to define a notion of sub-diary
which corresponds to a subtree of $\Sigma^*$ which preserves all important features of a given subset.  Given $S\subseteq \Sigma^*$ we first
determine which levels contain interesting changes and then define a sub-tree by removing the remaining ``boring" levels from the tree.
This is related to the notion of parameter-space envelopes used in~\cite{Hubicka2020CS}, but sharper, making it possible to get upper bounds tight.

\begin{definition}[Interesting levels]
	Given $S\subseteq \Sigma^*$, we call a level $\overline{S}_i$ \emph{interesting} if
	\begin{enumerate}
		\item the structure $\overline{\str{S}}_i=\AmbStr{\overline{S}_{i}}$ is not isomorphic to  $\overline{\str{S}}_{i+1}=\AmbStr{\overline{S}_{i+1}}$, or 
		\item there exist incompatible $u,v\in \overline{S}_{i+1}$ such that $u|_{i}$ and $v|_{i}$ are compatible, or
        \item 
        there is $u\in S$ with $\lvert u\rvert = i$.
	\end{enumerate}
\end{definition}

\begin{remark}
	Interesting levels are the analog for subsets of $\Sigma^*$ of the notion of \emph{critical level} for a subset of coding nodes in $\ct$; see for instance Definition 5.1 of \cite{zucker2020} or  Definition 5.1.3 of \cite{Balko2021exact} (we note that the two definitions are slightly different).
\end{remark}
Given $S\subseteq \Sigma^*$ and levels $\ell<\ell'$, we call a level $\ell'$ a \emph{duplicate of $\ell$} if $S$ contains no word of length $\ell$ and moreover for every $u\in S$ of length greater than $\ell'$ it holds that $u_\ell=u_{\ell'}$.
By checking definitions of $\lexleq$, $\lexlt$, $\preceq$ and $\eltleq$ one can derive the following simple result.

\begin{observation}
	\label{obs:duplicatelevels}
	For every $S\subseteq \Sigma^*$ and every $\ell<\ell'$ such that the level $\ell'$ is duplicate of $\ell$ it holds that $\ell'$ is not interesting.
\end{observation}

\begin{definition}[Embedding types]
Let $I(S)$ be the set of all interesting levels in $S$. Let $\tau_S\colon S\to \Sigma^*$ be the mapping assigning 
to 
each $w\in S$ 
the word created from $w$ by deleting all characters with indices not in $I(S)$.
Define $\tau(S)=\tau_S[S]$ and call it the \emph{embedding type of $S$}.
\end{definition}
	The following observation is a direct consequence of Definition~\ref{def:posetdiary}.

\begin{observation}\label{obs:types}
	For a poset-diary $S$ and $S'\subseteq S$, $\tau(S')$ is a poset-diary.\qed
\end{observation}

We therefore call $\tau(S')$ the \emph{sub-diary} induced by $S'\subseteq S$.

\begin{definition}
\label{Def:Boring}
Recall that for a set $A=\{u^0\lexlt u^1\lexlt\cdots\lexlt u^{n-1}\}\subseteq\Sigma^*_\ell$ (for some $\ell>0$) and word $e\in\Sigma^*_n$ we put $A\cont e=\{{u^i}\cont e_i:0\leq i<n\}$.
We call word $e\in \Sigma^*_n$ a \emph{boring extension} of $A$ if level $\ell$ of $\AmbStr{S\cont e}$ is not interesting.
We will denote by $\Pi_A$ the set of all boring extensions of $A$ and by $\Pi_A^\star$ the set of all finite words over the alphabet $\Pi_A$. That is, a member of $\Pi_A^\star$ is a sequence $w = (w^0,w^1,\ldots,w^{\lvert w\rvert -1})$ such that for every $i$ we have that $w^i\in \Pi_A$.
\end{definition}

We first prove two properties of boring extensions.

\begin{proposition}
	\label{prop:typeextend}
	For every $\ell\geq 0$, set $S\subseteq \Sigma^*_\ell$ of mutually compatible words, and boring extension $e$ of $S$, there exists a boring extension $e'$ of $\Sigma^*_\ell$ such that $S\cont e\subseteq {\Sigma^*_\ell}\cont e'$.
\end{proposition}
\begin{proof}
	Fix $\ell\geq 0$, $S=\{u^0\lexlt  u^1\lexlt\cdots\lexlt u^{n-1}\}$, $\Sigma^*_\ell=\{v^0\lexlt v^1\lexlt \cdots\lexlt \allowbreak v^{m-1}\}$ and a boring extension $e$ of $S$. For $u\in S$ denote by $i(u)$ the integer $i$ satisfying $u^i=u$.
	For $v\in \Sigma^*_\ell\setminus S$ 
 we say that character $c\in \Sigma$ is \emph{safe for $v$} if
	for every $0\leq j<n$ such that $u^j$ is compatible with $v$, it holds that $\AmbStr{\{u^j,v\}}$ is isomorphic to $\AmbStr{\{{u^j}\cont e_{j},v\cont c\}}$.

	First we check that for every $v\in \Sigma^*\setminus S$ the set of safe characters for $v$ is non-empty.
	To see this, consider vertex $v\in \Sigma^*\setminus S$ such that $\X$ is not safe for $v$. In this case there are two options:


	\begin{enumerate}
		\item There is $u\in S$ compatible with $v$ such that $u \lexlt v$, $u\not\perp v$ and $e_{i(u)}=\R$. In this case we argue that $\R$ is safe for $v$ (in fact, it is the only safe character for $v$). For this we need to check:
		      \begin{enumerate}
			      \item For every $w\in S$ compatible with $v$ such that $v\lexlt w$ and $v\not\perp w$ we have $e_{i(w)}=\R$. This follows from the fact that $v\eltleq w$ and this needs to be preserved by the extension. 
			      \item For every $w\in S$ compatible with $v$ such that $v\lexlt w$  it holds that $e_{i(w)}\neq \L$.
			            From the condition on $S$ we know that $u$ and $w$ are compatible and $u\lexlt v \lexlt w$. But then $u\cont \R$ and $w\cont \R$ are incompatible, which is in contradiction with  the definition of safe extension.
			      \item For every $w\in S$ compatible with $v$ such that $w\lexlt v$ and $v\not\preceq w$, we have that $e_{i(w)}\neq \L$. Suppose for a contradiction that $e_{i(w)} = \L$. As $w$ and $u$ are compatible and $e$ is boring, we know that $w\prec u$. But, by Proposition~\ref{prop:orders}~\ref{P4} and~\ref{P6}, we know that $w\prec u$, $w\not\prec v$, and $u\perp v$ cannot be all satisfied at the same time, a contradiction.
		      \end{enumerate}
		\item There is $u\in S$ compatible with $v$ such that $v \lexlt u$, $v\not\perp u$ and $e_{i(u)}=\L$. In this case we can argue symmetrically to show that $\L$ is the only safe character for $v$.
	\end{enumerate}
	Now we define $e'_j$ to be $e_{i(v^j)}$ whenever $v^j\in S$; otherwise choose the first character from $\X,\L,\R$ that is safe for $v^j$.

	To verify that $e'$ is boring, consider some $0\leq i<j\leq m$. First observe that ${\Sigma^*_\ell}\cont e'$ contains no words of length $\ell$. Also, if $(v^i, v^j)$ are incompatible, then any one-letter extensions will yield an isomorphic level structure. So we may assume that $v^i,v^j\notin S$ and $v^i$, $v^j$ are compatible. We have:
	\begin{enumerate}
		\item $(e'_i,e'_j)\neq (\R,\L)$. Assume the contrary and let $u\in S$ be the vertex that made $\X$ unsafe for $v^i$ and $u'\in S$ be the vertex that made $\X$ unsafe for $v^j$. By the same analysis as above we have $u\lexlt v^i$, $e_{i(u)}=\R$, $u\not\perp v^i$, $v^j\lexlt u'$, $e_{i(u')}=\L$, $v^j\not\perp u'$. From this however we conclude that $u\lexlt u'$, $e_{i(u)}=\R$, $e_{i(u')}=\L$ which contradicts the definition of boring extension and our assumption that $u$ and $u'$ are compatible.
		\item $v^i\not\perp v^j\implies (e'_i, e'_j)\neq (\X,\L)$. Assume the contrary and denote by $u\in S$ the vertex that made $\X$ unsafe for $v^j$. Clearly $v^j\lexlt u$, $v^j\not\perp u$ and $e_{i(u)}=L$.
		      It follows that for every $\ell'<\ell$ we have $v^i_{\ell'}\lexleq v^j_{\ell'}\lexleq u_{\ell'}$. From this we have that $u$ and $v^i$ are compatible and $u\not\perp v^i$ which makes $\X$ unsafe for $v^j$. A contradiction.
		\item $v^i\not\perp v^j\implies (e'_i, e'_j)\neq (\R,\X)$. Assume the contrary and denote by $u\in S$ the vertex that made $\X$ unsafe for $v^i$. Clearly $u\lexlt v^j$, $u\not\perp v^i$ and $e_{i(u)}=\R$.
		      It follows that for every $\ell'<\ell$ we have $u_{\ell'}\lexleq v^i_{\ell'}\lexleq v^j_{\ell'}$. From this we have that $u$ and $v^j$ are compatible and $u\not\perp v^j$ which makes $\R$ unsafe for $v^j$. A contradiction.
		\item $v^i\not\preceq v^j\implies (e'_i,e'_j)\neq (\L,\R)$. Assume the contrary and let $u\in S$ be the vertex that made $\X$ unsafe for $v^i$ and $u'\in S$ be the vertex that made $\X$ unsafe for $v^j$.  We have $v^i\lexlt u\lexlt u'\lexlt v^j$ and $e_{i(u)}=\L$, $e_{i(u')}=\R$. It follows that $v^i\not\perp u$, $u\preceq u'$, $u'\not\perp v^j$. Now for every $\ell'<\ell$ we also have $v^i_{\ell'}\lexlt u_{\ell'}\lexlt u'_{\ell'}\lexlt v^j_{\ell'}$. This yields $v^i\preceq v^j$. A contradiction.

	\end{enumerate}
\end{proof}
\begin{proposition}
	\label{prop:typeextend2}
	Let $0\leq \ell\leq \ell'$ be integers and  $e$ be a  boring extension  of $\Sigma^*_\ell=\{u^0\lexlt u^1\lexlt \cdots \lexlt u^{n-1}\}$. Put $\Sigma^*_{\ell'}=\{v^0\lexlt v^1\lexlt\cdots\lexlt v^{m-1}\}$ and create the word $e'$ of length $m$ by putting, for every $0\leq i<m$, $e'_i=e_j$ where $j$ satisfies $v^i|_{\ell}=u^j$. Then $e'$ is a boring extension of $\Sigma^*_{\ell'}$.
\end{proposition}
\begin{proof}
    The proof is just a straightforward verification of the fact that the relations $\lexleq$, $\preceq$ and $\trianglelefteq$ are determined by first occurrences of certain combinations of letters which this construction does not change.
    
    If $(v^i, v^j)$ are incompatible, then any one-letter extensions will yield an isomorphic level structure. So we may assume that $(v^i, v^j)$ are compatible.

	Let $v^i,v^j$ be compatible words.  We then check that the structure $\AmbStr{\{v^i,v^j\}}$ is isomorphic to $\AmbStr{\{{v^i}\cont e'_i,{v^j}\cont e'_j\}}$ and that ${v^i}\cont e'_i$ and ${v^j}\cont e'_j$ are compatible. If $e'_i=e'_j$ (in particular, this happens whenever $v^i|_\ell=v^j|_\ell$), the result is clear.
 
    So suppose that $e'_i\neq e'_j$; in particular, this implies that $v^i|_\ell\neq v^j|_\ell$. Note that in this case the lexicographic order of $v^i$ and $v^j$ is already determined by their restrictions to level $\ell$ (that is, $v^i \lexlt v^j \iff v^i|_\ell\lexlt v^j|_\ell$), hence $(\{v^i|_\ell,v^j|_\ell\},\lexleq)$, $(\{v^i,v^j\},\lexleq)$, and $(\{{v^i}\cont e'_i,{v^j}\cont e'_j\},\lexleq)$ are isomorphic. Consequently, compatibility of ${v^i}\cont e'_i$ and ${v^j}\cont e'_j$ follows from compatibility of $v^i$ and $v^j$ and compatibility of ${v^i|_\ell}\cont e'_i$ and ${v^j|_\ell}\cont e'_j$.

    Since $\AmbStr{\{v^i|_\ell,v^j|_\ell\}}$ is isomorphic to $\AmbStr{\{{v^i|_\ell}\cont e'_i,{v^j|_\ell}\cont e'_j\}}$ (by the fact that $e$ is a boring extension), we get that either $(e'_i,e'_j)\notin \{(L,R),(R,L)\}$, or $\preceq$ is already defined on $\{v^i|_\ell,v^j|_\ell\}$. In either case, $(\{v^i,v^j\},\preceq)$ and $(\{{v^i}\cont e'_i,{v^j}\cont e'_j\},\preceq)$ are isomorphic. A similar argument can be done for $\trianglelefteq$, and hence indeed $\AmbStr{\{v^i,v^j\}}$ is isomorphic to $\AmbStr{\{{v^i}\cont e'_i,{v^j}\cont e'_j\}}$, that is, $e'$ is a boring extension of $\Sigma^*_{\ell'}$.
 
 
\end{proof}

\section{Upper bounds}
We prove Ramsey-type theorem for the following the following kind of embedings.
\begin{definition}[Shape-preserving functions]
Given $S\subseteq \Sigma^*$ we call function $f\colon S\to \Sigma^*$ \emph{shape-preserving} if  $\tau_S(w)=\tau_{f[S]}(f(w))$
for every $w\in S$. 
\end{definition}

We will generally consider shape-preserving functions only for those sets $S$ such that $S=\tau(S)$ (that is for \emph{embedding types}). However, the next observation follows directly from the definition without this extra assumption:

\begin{observation}\label{obs:shape}
	Let $f\colon S\to \Sigma^*$ be shape-preserving.
	\begin{enumerate}[label=(\roman*)]
		\item\label{item:composition} For every shape-preserving $h\colon f[S]\to \Sigma^*$ it holds that $h\circ f$ is shape-preserving.
		\item\label{item:length} For all $u,v\in S$, $|u|\leq |v|$ implies that $|f(u)|\leq |f(v)|$.
		\item\label{item:succ} For all $u,v\in S$ with $u\sqsubseteq v$, we have $f(u)\sqsubseteq f(v)$.
		\item\label{item:emb} The function $f$ is an embedding $f\colon \AmbStr{S}\to\AmbStr{\Sigma^*}$ and images of pairs of compatible words are also compatible.
            \item\label{item:boring} Let $e$ be a boring extension of $S$. Then $e$ is a boring extension of $f[S]$.
	\end{enumerate}
\end{observation}
Given $S\subseteq \Sigma^*$ and $\ell>0$ we denote by $S_{\leq\ell}=\cup_{i\leq \ell} S_i$ the set of all words in $S$ of length at most $\ell$. We also put $S_{<\ell}=S_{\leq\ell-1}$.
\begin{remark}
	It is also possible to observe that if $S=\Sigma^*_\ell$ for some $\ell>0$, it holds that shape-preserving functions correspond to special strong subtrees as used by the Milliken's tree theorem~\cite{todorcevic2010introduction}. However, the converse is not true:  For example, the function $f\colon \Sigma^*_{<2}\to\Sigma^*$ mapping the empty word to the empty word, $L\mapsto \L\L$, $\X\mapsto \X\R$ and $\R\mapsto \R\R$ is not shape-preserving because all three levels 0, 1 and 2 of $f[\Sigma^*_{<2}]$ are interesting, while it does describe a strong subtree.
\end{remark}
Given $S,S'\subseteq \Sigma^*$ we denote by $\Shape{S}{S'}$ the set of all shape-preserving functions $f$ such that $f[S]\subseteq S'$.
Given integer $n$ we also denote by $\nShape{n}{S}{S'}$ the set of all functions in $\Shape{S}{S'}$ that are  the identity when restricted to $S_{<n}$.


Let $S\subseteq \Sigma^*$.
For  a shape-preserving function $g\colon S\to\Sigma^*$, we denote by $\widetilde{g}$ the function $\{|w|:w\in S\}\to\omega$ defined by $\widetilde{g}(i)=|g(w)|$ for some $w\in S$, $|w|=i$. (Note that by Observation~\ref{obs:shape}~\ref{item:length} this is uniquely defined.) If $S$ is finite, denote by $\max(S)$ the ``last'' level $S_k$ where $k=\max_{w\in S}|w|$.




\begin{observation}
	\label{obs:towords}
	Let $S=\overline S=\tau(S)$ be a finite subset of $\Sigma^*$, $\max(S)=\{u^0\lexlt\allowbreak u^1\lexlt \cdots\allowbreak \lexlt u^n\}$ and $k=\max_{w\in S}{|w|}$.
	There is  a one-to-one correspondence between $\Shape{S}{\Sigma^*}$ and pairs $(g^0,w)$ where $g^0\in\Shape{S_{<k}}{\Sigma^*}$ and $w\in \Pi^*_{\max(S)}$ (recall Definition~\ref{Def:Boring}):
	\begin{enumerate}
		\item
			For every $g\in \Shape{S}{\Sigma^*}$ there exists $w\in \Pi^*_{\max(S)}$ such that for every $0\leq i<n$ it holds that  $g(u^i)=g(u^i|_{k-1})\cont {u^i_{k-1}}\cont {w^0_i}\cont\cdots\cont w^{|w|-1}_i$.
		\item
		      Conversely also for every $g^0\in \Shape{S_{<k}}{\Sigma}$ and every word $w\in \Pi^*_{\max(S)}$ the function $g'\colon S\to \Sigma^*$ defined by $g'(w)=g^0(w)$ for $|w|<k$ and $$g'(u^i)=g^0(u^i|_{k-1})\cont {u^i_{k-1}}\cont {w^0_i}\cont\cdots\cont w^{|w|-1}_i$$ is shape-preserving.
	\end{enumerate}
\end{observation}
\begin{proof}
	To see the first statement assume the contrary and let $g\in \Shape{S}{\Sigma^*}$ be a function for which there is no $w\in \Pi^*_{\max(S)}$ such that for every $0\leq i<n$ it holds that  $$g(u^i)=g(u^i|_{k-1})\cont {u^i_{k-1}}\cont {w^0_i}\cont\cdots\cont w^{|w|-1}_i.$$ Among all such functions $g$ choose one which minimizes $\widetilde{g}(k)$. 

	Because $S=\bar S$ we know that $\tilde{g}(k-1)\in I(g[S])$. Because $g$ is shape-preserving it follows that $g(u^i)_{\tilde{g}(k-1)}=u^i_{k-1}$ and $g(u^i)|_{\tilde{g}(k-1)+1}=g(u^i|_{k-1})\cont {u^i_{k-1}}$.
	It follows that $\widetilde{g}(k)\geq \widetilde{g}(k-1)+2$.

	Notice that $\widetilde{g}(k)=\widetilde{g}(k-1)+2$: As $g$ is shape-preserving, $I(g(S))$ contains no levels between $\widetilde{g}(k-1)$ and $\widetilde{g}(k)$ as otherwise we could remove them, getting a counter example $g'$ with smaller $\widetilde{g'}(k)$. If $\widetilde{g}(k) = \widetilde{g}(k-1)+1$, then we can take $w = \emptyset$, and $g$ would not be a counterexample. But now, observe that level $\widetilde{g}(k)+1$ of $g(S)$ is not interesting and thus corresponds to a boring extension in $\Pi^*_{\max(S)}$, which gives a contradiction. 

	The second statement follows by Proposition~\ref{prop:typeextend2}.
\end{proof}




The following pigeonhole lemma is a consequence of Theorem~\ref{thm:CS}.
\begin{lemma}
	\label{lem:pigeonhole}
	Let $S=\overline{S}=\tau(S)$ be a finite non-empty subset of $\Sigma^*$ of mutually compatible words containing at least one non-empty word. Put $k=\max_{w\in S}|w|$. Let $g^0\in \Shape{S_{<k}}{\Sigma^*}$. Denote by $G$ set of all $g\in \Shape{S}{\Sigma^*}$ extending $g^0$ and put $K=\widetilde{g}^0(k-1)$. Then for every finite coloring $\chi\colon G\to \{0,1,\ldots,r-1\}$ there exists $f\in \nShape{K+1}{\Sigma^*}{\Sigma^*}$ such that
	$\chi$ restricted to $\Shape{S}{f[\Sigma^*]}\cap G$ is constant.
\end{lemma}

\begin{proof}
	By Observation~\ref{obs:towords}, the colouring $\chi\colon G\to \{0,1,\ldots, r-1\}$ gives rise to a colouring $\chi'\colon \Pi^*_{\max(S)}\to\{0,1,\ldots,r-1\}$. Apply Theorem~\ref{thm:CS} to obtain $W$ such that $W[\Pi^*_{\max(S)}]$ is monochromatic with respect to $\chi'$. In order to avoid special cases in the upcoming construction, we will assume that $W$ is indexed from 1 and not from 0.

	For every $u\in \Sigma^*_{\leq K}$ put $f(u)=u$. We will construct the rest of $f$ by induction on levels. Now assume that $f(\Sigma^*_{i-1})$ is already defined for some $i>K$. Put 
	\begin{align*}
	    I = \begin{cases}
	            0 \quad &\text{if } i = K+1,\\
		    \min\{I< \omega: W_I = \lambda_{i-K-2}\} \quad &\text{if } i > K+1.
	        \end{cases}
	\end{align*}
	Let $J$  be the minimal integer such that $W_{J}=\lambda_{i-K-1}$.
	Now define a sequence of functions $f^{i'}\colon \Sigma^*_i\to\Sigma^*_{K+I+i'}$ for every $I\leq i'< J$. Put $f^{I}(u)=f(u|_{i-1})\cont u_{i-1}$  for every $u\in \Sigma^*_i$.
 Now proceed by induction on $i'$.
	Assume that $f^{i'-1}$ is constructed for some $I<i'<J$ and consider two cases:
	\begin{enumerate}
		\item $W_{i'}=\lambda_j$: Put $f^{i'}(u)=f^{i'-1}(u)\cont u_{j+k+1}$ for every $u\in \Sigma^*_i$. 
		\item $W_{i'}=e$ for some $e\in \Pi_S$:
		      Let $e'$ be the extension of $\Sigma^*_{K+1}$ given by Proposition~\ref{prop:typeextend} 
			for boring extension $e$ of $g^1(S_k)$, where $g^1$ is defined by putting $g^1(u)\mapsto g^0(u|_{k-1})\cont u_{k-1}$ (by Observation~\ref{obs:shape}~(\ref{item:boring}), boringness of an extension is preserved by shape-preserving functions).
		      Now let $e''$ be the extension given by Proposition~\ref{prop:typeextend2} for extension $e'$ and level $i$. Enumerate $\Sigma^*_{i}=\{u^0\lexlt u^1\lexlt\cdots\lexlt u^{m-1}\}$ and for $u^j\in S_i$, 
		      put $f^{i'}(u^j)=f^{i'-1}(u)\cont e''_j$ for every $0\leq j<m$.
	\end{enumerate}
	Finally put $f(u)=f^{J-1}(u)$.

	Observe that all levels added by the rules 1 and 2 above are uninteresting since they are either constructed from boring extensions or they are duplicates of levels introduced earlier (in the sense of Observation~\ref{obs:duplicatelevels}).  The last level is interesting because $\tau(\Sigma^*)=\Sigma^*$. From this we get  $f\in \nShape{K+1}{\Sigma^*}{\Sigma^*}$.

 To see that  $\chi$ restricted to $\Shape{S}{f[\Sigma^*]}\cap G$ is constant, pick an arbitrary $g \in \Shape{S}{f[\Sigma^*]}\cap G$. By Observation~\ref{obs:towords} we can decompose $g$ to $g^0$ and a word $w\in \Pi^*_{\max(S)}$ such that $\chi(g)$ is equal to $\chi'(w)$. From our construction it follows that $w\in W(\Sigma^*)$, and so indeed $\chi$ restricted to $\Shape{S}{f[\Sigma^*]}\cap G$ is constant.
\end{proof}


\begin{observation}\label{obs:nobar}
	For every $S=\tau(S)\subseteq \Sigma^*$ and every $f\in \Shape{S}{\Sigma^*}$ there is a unique function $g\in \Shape{\overline{S}}{\Sigma^*}$ extending $f$. It is constructed by putting $g(w|_\ell)=f(w)|_{\widetilde{f}(\ell)}$ for every $w\in S$ and $\ell\leq |w|$. Similarly, for every $g\in \Shape{\overline{S}}{\Sigma^*}$ it holds that $g\restriction S\in \Shape{S}{\Sigma^*}$.
\end{observation}
Notice that $g$ in Observation~\ref{obs:nobar} is well defined by Observation~\ref{obs:shape}~\ref{item:succ}.


\begin{theorem}
	\label{thm:multCS}
	For every finite set $S=\tau(S)\subseteq \Sigma^*$ of mutually compatible words and every finite coloring $\chi\colon \Shape{S}{\Sigma^*}\to\{0,1,\ldots,r-1\}$, there exists
	$f\in \Shape{\Sigma^*}{\Sigma^*}$ such that $\chi$ restricted to $\Shape{S}{f[\Sigma^*]}$ is constant.
\end{theorem}
\begin{proof}
	By Observation~\ref{obs:nobar} we can assume, without loss of generality, that $S=\overline{S}$.
	We will use induction on $k=\max_{w\in S}|w|$.  For $k=0$ we can interpret $\chi$ as coloring of $\Sigma^*$, apply Theorem~\ref{thm:CS} to obtain a monochromatic infinite-parameter word $W$, for every $w\in\Sigma^*$ put $f(w)=W(w)$, and observe that it is shape-preserving.

	Now fix $S$ such that $k=\max_{w\in S}|w|>0$ and a finite colouring $\chi\colon \Shape{S}{\Sigma^*}\to\{0,1,\ldots,r-1\}$.
	We will make use of the following claim:
	\begin{claim}
		There exists $h\in \Shape{\Sigma^*}{\Sigma^*}$ and a colouring $\chi'\colon \Shape{S_{<k}}{\Sigma^*}\to \{0,1,\ldots,\allowbreak r-1\}$  such that for every finite $g\in \Shape{S}{\Sigma^*}$ it holds that $\chi(h\circ g)=\chi'(g \restriction S_{<k})$.
	\end{claim}
	First we show that Theorem~\ref{thm:multCS} follows from the claim. Let $h$ and $\chi'$ be given by the claim. By induction hypothesis there exists $f'\in\Shape{\Sigma^*}{\Sigma^*}$ such that $\chi'$ is constant on $f'[\Sigma^*]$. It is easy to check that $f=h\circ f'$ is shape-preserving and $\chi$ is constant when restricted to $f[\Sigma^*]$.

	\medskip
	It remains to prove the claim. We will obtain $h$ as the limit of the following sequence:
	Pick an enumeration $\Shape{S_{<k}}{\Sigma^*}=\{g^0, g^1,\ldots\}$ such that $0\leq i\leq j$ it holds that $\widetilde{g}^i(k-1)\leq\widetilde{g}^j(k-1)$.
	We construct  a sequence of shape-preserving functions $f^0,f^1,\ldots\in \Shape{\Sigma^*}{\Sigma^*}$  such that for every $i>0$ the following is satisfied:
	\begin{enumerate}
		\item $f^i[\Sigma^*]\subseteq f^{i-1}[\Sigma^*]$ and $f^i(u)=f^{i-1}(u)$ for every $u\in \Sigma^*_{<\widetilde{g}^{i-1}(k-1)+1}$.
		\item There exists $c^{i-1}\in \{0,1,\ldots, r-1\}$ such that $\chi(f^i\circ g)=c^{i-1}$ for every $g\in \Shape{S}{\Sigma^*}$ extending $g^{i-1}$.
	\end{enumerate}

	Put $f^0$ to be the identity $\Sigma^*\to\Sigma^*$. Now assume that $f^{i-1}$ is already constructed.  Consider coloring $\chi^i\colon \Shape{S}{\Sigma^*}\to\{0,1,\ldots,r-1\}$ defined by $\chi^i(g) = \chi(f^{i-1}\circ g)$. Obtain $h^i \in \nShape{\widetilde{g}^{i-1}(k-1)+1}{\Sigma^*}{\Sigma^*}$ by an application of Lemma~\ref{lem:pigeonhole} for coloring $\chi^i$ and function $f^{i-1}\circ g^{i-1}$ (as $g^0$ in the statement) and put $f^i=f^{i-1}\circ h^i$.

	\medskip

	Next we construct the limit shape-preserving $h$.
	For every $i>1$ it holds that $f^i(u)=f^{i-1}(u)$ for all $u\in \Sigma^*$ where $u\leq \widetilde{g}^{i-1}(k-1)$.
	Because $\widetilde{g}^{i-1}(k-1)$ is an increasing function of $i$ and there is no upper bound on the length of words in $\Sigma^*$ it follows that $h(u) = \lim_{i\to \omega} f^i(u)$ is well-defined for every $u\in \Sigma^*$. Moreover, $h$ is shape-preserving, because the failure of shape-preservation is witnessed on a finite set. We also put $\chi'(g^i)=c^i$ for every $i\in \omega$.
\end{proof}

Now we are finally ready to prove a big Ramsey result for $\str P$.
\begin{corollary}
\label{cor:big_ramsey}
	For every finite partial order $\str{Q}$, the big Ramsey degree of $\str{Q}$ in the generic partial order $\str{P}$ is at most $|T(\str{Q})|\cdot |\mathrm{Aut}(\str{Q})|$. 
\end{corollary}
\begin{proof}
	Fix a finite partial order $\str{Q}$ and a coloring $\chi$ of $\str{P}\choose\str{Q}$.
	Choose an arbitrary enumeration $T(\str{Q})=\{S_0,S_1,\ldots,S_{n-1}\}$ and an arbitrary embedding $\eta\colon {(\Sigma^*,\preceq)}\to\str{P}$ (which exists since $\str{P}$ is universal).
	Observe that for every $i\in n$ it holds that $\chi$ and $\eta$ induce a coloring $\chi_i$ of $\Shape{S_i}{\Sigma^*}$ by putting $\chi_i(g)=\chi(\eta\circ g[S_i])$.
	By repeated applications of Theorem~\ref{thm:multCS} we construct a  sequence of functions $\mathrm{Id}=f_0,f_1,\ldots f_{n}\in \Shape{\Sigma^*}{\Sigma^*}$ such that for every $i\in n$ the following is satisfied:
	\begin{enumerate}
		\item $f_{i+1}[\Sigma^*]\subseteq f_{i}[\Sigma^*]$, and
		\item $\chi_i$ restricted to $\Shape{S_i}{f_{i+1}[\Sigma^*]}$ is constant.
	\end{enumerate}
	Let $\psi\colon \str{P}\to{(\Sigma^*,\preceq)}$ be obtained by Theorem~\ref{thm:posetemb}.
	Observe that $f_n\circ \psi$ is the desired embedding $\str{P}\to\str{P}$ where color of every $h\in {\str{P}\choose\str{Q}}$ depends only on $\tau[\eta[h[Q]]]\in T(\str{Q})$.
\end{proof}

\section{The lower bound}\label{sec:lowerBound}

Given a finite partial order $\str{A}$, a \emph{labeled} poset-diary for $\str{A}$ is a pair $(S, f)$, where $S\subseteq \Sigma^*$ is a poset-diary in $T(\str{A})$ and $f\colon \str{A}\to (S, \prec)$ is an isomorphism. Let $T^{lab}(\str{A})$ denote the set of labeled poset-diaries coding $\str{A}$. Note that $|T^{lab}(\str{A})| = |T(\str{A})|\cdot |\mathrm{Aut}(\str{A})|$. Recall the embedding $\psi\colon \str{P}\to (\Sigma^*, \preceq)$ constructed in Theorem~\ref{thm:posetemb}. We define a function (colouring) $\chi_\str{A}\colon\binom{\str{P}}{\str{A}}\to T^{lab}(\str{A})$ by setting $\chi_\str{A}(f)=(\tau(\psi\circ f[A]), \tau_{\psi\circ f[A]}\circ\psi\circ f)$ for every $f\in \binom{\str{P}}{\str{A}}$.
In this section, we show that $\chi_\str{A}$ is a \emph{recurrent coloring} in the following sense: for every $h\in \binom{\str{P}}{\str{P}}$, there is $\phi\in \binom{\str{P}}{\str{P}}$ with $\chi_\str{A}\circ h\circ \phi = \chi_\str{A}$. This allows us to characterize the exact big Ramsey degrees of $\str P$.





\begin{theorem}
	\label{thm:embthm}
	For every finite partial order $\str{A}$ and every $f\in \binom{\str{P}}{\str{P}}$, we have $$\left\{\chi_\str{A}[f\circ g]: g\in \binom{\str{P}}{\str{A}}\right\}=T^{lab}(\str{A}).$$
	Furthermore, for every $f\in \binom{\str{P}}{\str{P}}$, there is $\phi\in \binom{\str{P}}{\str{P}}$ with $\chi_{\str{A}}\circ f\circ \phi = \chi_\str{A}$.
\end{theorem}

\begin{remark}
Zucker~\cite{zucker2017} introduced the notion of \emph{big Ramsey structure}, which is a structure capturing exact big Ramsey degrees for all finite substructures at the same time. Recurrence of the colorings $\chi_\str A$ implies that any poset diary coding $\str{P}$ is a big Ramsey structure for $\str{P}$. In fact, this recurrence property is significantly stronger than asserting that poset diaries coding $\str{P}$ are big Ramsey structures; it tells us that any two poset diaries coding the generic poset are bi-embeddable. This allows us to conclude stronger dynamical properties of the group $G:= \mathrm{Aut}(\str{P})$ than indicated in \cite{zucker2017}, namely that $G$ admits a metrizable universal completion flow which is also a \emph{strong} completion flow (see the discussion preceding Theorem~8.0.6 of~\cite{Balko2021exact} for definitions). We observe that for any finite partial order $\str{B}$, $\chi_\str{B}$ is completely determined by the colorings $\chi_\str{A}$ for finite partial orders $\str{A}$ with $\card{\str{A}}\leq 4$. Thus we obtain a big Ramsey structure for $\str{P}$ in a finite relational language.
\end{remark}

The rest of this section proves Theorem~\ref{thm:embthm}.
First we show, by a repeated application of Lemma~\ref{lem:pigeonhole}, that for every $f\in \binom{{(\Sigma^*,\preceq)}}{{(\Sigma^*,\preceq)}}$ (not necessarily a shape-preserving one) there exists 
$g\in \Shape{\Sigma^*}{\Sigma^*}$ such that $f$ preserves the main features of the tree structure on $g[\Sigma^*]$.


\begin{lemma}
	\label{lem:canonical}
	For every $f\in \binom{{(\Sigma^*,\preceq)}}{{(\Sigma^*,\preceq)}}$ there exists $g\in \Shape{\Sigma^*}{\Sigma^*}$ and a sequence $(N_i)_{i\in\omega}$ satisfying:
	\begin{enumerate}
		\item $N_0=0$,
		\item for every $u\in \Sigma^*$ it holds that $N_{|u|}\leq |f(g(u))|< N_{|u|+1}$, and,
		\item for every $u,v\in \Sigma^*$ and $\ell$ such that $u|_\ell = v|_\ell$, it holds that $f(g(u))|_{N_\ell}=f(g(v))|_{N_\ell}$.
	\end{enumerate}
	See Figure~\ref{fig:canonical}.
\end{lemma}
	\begin{figure}[h]
		\centering
		\includegraphics{canonical.pdf}
		\caption{Function $f\circ g$.}
		\label{fig:canonical}
	\end{figure}
\begin{proof}
Fix an embedding $f\colon{(\Sigma^*,\preceq)}\to{(\Sigma^*,\preceq)}$.
We define a sequence of shape-preserving functions $(g_i)_{i\in\omega}$ and a sequence $(N_i)_{i\in \omega}$ of integers satisfying,
 for every $i>0$, the following three conditions:
\begin{enumerate}[label=(\Alph*)]
	\item\label{g1}$g_i\in \Shape{\Sigma^*}{\Sigma^*}$ and $g_{i-1}\restriction \Sigma^*_{{<}i}=g_i\restriction \Sigma^*_{{<}i}$. 
	\item\label{g2} For every $u\in \Sigma^*_{{<}i}$ it holds that $N_{|u|}\leq f(g_i(u))< N_{|u|+1}$.
	\item\label{g3} For every $u,v\in \Sigma^*$ and $\ell \leq \min(i, |u|,|v|)$ such that $u|_\ell=v|_\ell$ it holds that $f(g_i(u))|_ {N_\ell}=f(g_i(v))|_ {N_\ell}$.
\end{enumerate}

Put $g_0$ to be identity, $N_0=0$ and proceed by induction.
	That $g_{i-1}$ and $N_{i-1}$ are already constructed for some $i > 0$.  Put $$N_i=\max\left\{|f(g_{i-1}(u))|: u\in \Sigma^*_{i-1}\right\}+1.$$

	Enumerate $\Sigma^*_i={w^0,w^1,\ldots,w^{m-1}}$.  By induction we will construct a sequence of functions $g_{i-1}=g^0_i, g^1_i,\ldots, g^m_i\in\Shape{\Sigma^*}{\Sigma^*}$. Assume that $g^j_i$ is constructed.
	Put $S^j_i=\overline{\{w^j\}}$ and define a coloring $\chi^j_i(h)$ of $\nShape{i}{S^j_i}{\Sigma^*}$ by putting $\chi^j_i(h)=g^j_i(h(w^j))|_{N_i}$. Apply Lemma~\ref{lem:pigeonhole}
	on $\chi^j_i$ and obtain $h^j_i$. Put $g^{j+1}_i=h^j_i\circ g^j_i$.
	Finally, put $g_i=g^m_i$.

	To see that $g_i$ satisfies~\ref{g1} note that all $h_i^j$'s are shape-preserving functions and that they are the identity when restricted to $\Sigma^*_{<i}$.
	Property~\ref{g2} follows directly from the choice of $N_i$.
	It remains to verify that~\ref{g3} is satisfied. By~\ref{g1} it is enough to verify this for $\ell = i$. Let $u,v\in \Sigma^*$ be such that $u|_i=v|_i$ and let $c^m_i$ be the constant value of $\chi^m_i$ for $m$ satisfying $w^m=u|_i=v|_i$ on $h_i^m$.
	Notice that $|c^m_i|=N_i$ because there are infinitely many images of words extending $w^m$.
	It follows that $g_i(u)|_{N_i}=g_i(v)|_{N_i}=c^m_i$.

It remains to put $g$ to be the limit of sequence $(g_i)_{i\in \omega}$.
\end{proof}

We denote by $d\colon \Sigma^*\to\Sigma^*$ the function that repeats every letter 3 times.  That is, for every $u\in \Sigma^*$ we put $d(u)=u'$ where $|u'|=3|u|$ and for every $\ell<|u|$ it holds that $u'_{3\ell}=u'_{3\ell+1}=u'_{3\ell+2}=u_\ell$.
Note that $d$ is a shape-preserving function.
\begin{lemma}
	\label{lem:ambpreserving}
	Let $f\in \binom{{(\Sigma^*,\preceq)}}{{(\Sigma^*,\preceq)}}$ be an embedding. Let $g\in \Shape{\Sigma^*}{\Sigma^*}$ and sequence $(N_i)_{i\in \omega}$ be given by Lemma~\ref{lem:canonical}.
	Put $f'=f\circ g\circ d$. 
	Then for every poset-diary $S$ and every $\ell<\sup_{u\in S}|u|$ it holds that $\AmbStr{\overline{S}_\ell}$ is isomorphic to $\AmbStr{\overline{f'[S]}_{N_{3\ell}}}$.
\end{lemma}
\begin{proof}
	We proceed by induction on level $\ell$.  Since $\lvert \overline{S}_0\rvert = 1$ we have: $$\AmbStr{\overline{S}_0}=\AmbStr{\overline{f'[S]}_{N_0}}.$$

	Now assume that $\AmbStr{\overline{S}_\ell}$ is isomorphic to $\AmbStr{\overline{f'[S]}_{N_{3\ell}}}$.  We define function $\mu$ 
	assigning every $u\in S_{\ell+1}$ word $$\mu(u)=f'(u)|_{N_{3\ell+3}}.$$
	We claim that $\mu$ is the isomorphism of $\AmbStr{\overline{S}_{\ell+1}}$ and $\AmbStr{\overline{f'[S]}_{N_{3\ell+3}}}$:

	\begin{enumerate}
		\item $\mu[\overline{S}_{\ell+1}]\subseteq \overline{f'[S]}_{N_{3\ell+3}}$: Recall that $f'=f\circ g\circ d$. For every $u\in \overline{S}_{\ell+1}$ we have $|d(u)|=3\ell+3$. Let $v\in S$ be a sucessor of $u$. Notice that $d(v)$ is a successor of $d(u)$.
			Now because $g$ is given by Lemma~\ref{lem:canonical} we know that both $f'(v)$ and $f'(u)$ are sucessors of $f'(u)|_{N_{3\ell+3}}$.
		\item For every $u\in S$ with $|u|\geq \ell+1$ it holds that $\mu(u|_{\ell+1})\sqsubseteq f'(u)$: This follows directly from the fact that $g$ is constructed using Lemma~\ref{lem:canonical}.
		\item $\mu[\overline{S}_{\ell+1}]\supseteq \overline{f'[S]}_{N_{3\ell+3}}$:  Let $v\in \overline{f'[S]}_{N_{3\ell+3}}$ and 
			choose $u\in S$ such that $v\sqsubseteq f'(u)$.
			It follows from Lemma~\ref{lem:canonical} that $\lvert u\rvert \geq \ell+1$ and $\mu(u|_{\ell+1})=v$.
		\item $\mu$ is injective: 
			Assume that there are $u\lexlt v$ in $\overline{S}_{\ell+1}$ such that $\mu(u)=\mu(v)$.  By the induction hypothesis we then know that level $\ell$ is splitting
			and $u$, $v$ are the splitting words.  It follows that $u=w\cont \X$ and $v=w\cont \R$ for $w=u|_\ell$.
			Put $w'=d(w)\cont \L$ and observe that $d(u)\perp w'$ and $w'\prec d(v)$.  Since $N_{3\ell}\leq |f\circ g(w')|<N_{3\ell+1}$ and $f\circ g$ is an embedding we know that
			there is level $\ell'$ satisfying $N_{3\ell}\leq \ell'<N_{3\ell+1}$ satisfying $f'(u)_{\ell'}=\X$ and $f'(v)_{\ell'}=\R$, which gives a contradiction with $\mu(u)=\mu(v)$, as $\mu(u)\sqsubseteq f'(u)$ and $\mu(v)\sqsubseteq f'(v)$.
		\item $u\lexleq v\implies \mu(u)\lexleq \mu(v)$:
			By the induction hypothesis we need to consider only case where $u|_\ell=v|_\ell$. Therefore $\ell+1$ is a splitting level,  $u=w\cont \X$ and $v=w\cont \R$ for $w=u|_\ell$.
			Let $u'=w\cont \X\L$ and $v'=w\cont \R\R$.  Since $u'\prec v'$ and thus also $f'(u')\prec f'(v')$, and since $\mu(u)\sqsubseteq f(u')$ and $\mu(v)\sqsubseteq f(v')$, we have $\mu(u)\lexleq \mu(v)$.
            \item $\mu(u)\lexleq \mu(v)\implies u\lexleq v$: Follows from the previous point and the fact that $\lexleq$ is a linear order.
		\item $u\prec v\implies \mu(u)\prec \mu(v)$:  Let $\ell'$ be the minimal level such that $u_{\ell'}=\L$ and $v_{\ell'}=\R$.
			Put $w=d(u|_{\ell'})\cont \L\R$. Observe that $d(u)\prec w\prec d(v)$. (That is, $w$ is a \emph{witness} of the fact that $d(u)\prec d(v)$.) Observe also that $f'(u)=f(g(d(u)))\preceq f(g(w))\preceq f(g(d(v)))=f'(v)$. Because $\ell'\leq \ell$ we have $|w|\leq 3\ell+2$ and thus $|f(g(w))|< N_{3\ell+3}$. It follows that $\mu(u)\prec \mu(v)$.
		\item $\mu(u)\prec \mu(v)\implies u\prec v$:  Assume that $\mu(u)\prec \mu(v)$ and $u\not\prec v$. Because $\mu(u)\lexlt \mu(v)$ we also have $u\lexlt v$. Consequently, $u\cont \L\X\not \prec v\cont \X\L$, and so $f'(u\cont \L\X)\not\prec f'(v\cont \X\L)$.  This is a contradiction with $\mu(u)\sqsubseteq f'(u\cont \L\X)$  and $\mu(v)\sqsubseteq f'(v\cont \X\L)$.

		\item $u\eltleq v\implies \mu(u)\eltleq \mu(v)$: Since $u'\prec v'\implies u'\eltlt v'$, we only need to consider the case where $u\eltlt v$ and $u\not \prec v$, and so $f'(u)\not\prec f'(v)$. We have $f'(u\cont \L)\prec f'(v\cont \R)$, and because $\mu(u)\sqsubseteq f'(u)\sqsubseteq f'(u\cont \L)$ 
			and $\mu(v)\sqsubseteq f'(v)\sqsubseteq f'(v\cont \R)$, it follows that $\mu(u)\eltleq \mu(v)$.
		\item $\mu(u)\eltleq \mu(v)\implies u\eltleq v$: If $u\not \eltleq v$ then there exists a level $\ell'<\ell+1$ such that $v_{\ell'}\lexlt u_{\ell'}$. Similarly as in the previous cases, we can produce a witness of this fact and contradict that $\mu(u)\eltleq \mu(v)$.
	\end{enumerate}
\end{proof}
\begin{proof}[Proof of Theorem~\ref{thm:embthm}]
	Fix $f\in \binom{\str{P}}{\str{P}}$. Let $\psi\colon \str{P}\to{(\Sigma^*,\preceq)}$
 be obtained by Theorem~\ref{thm:posetemb}.
	Let $\eta\colon {(\Sigma^*,\preceq)}\to\str{P}$ be an embedding (which exists since $\str{P}$ is universal).
	Now $\psi\circ f\circ\eta$ is an embedding ${(\Sigma^*,\preceq)}\to{(\Sigma^*,\preceq)}$.
	Let $g\colon {(\Sigma^*,\preceq)}\to{(\Sigma^*,\preceq)}$ and $(N_i)_{i\in \omega}$ be obtained by the application of Lemma~\ref{lem:canonical} on $\psi\circ f\circ\eta$.
	Put $f'=\psi\circ f\circ\eta\circ g\circ d$.  We claim that for every poset-diary $S$ it holds that $\tau(f'[S])=S$. 
	By Lemma~\ref{lem:ambpreserving} we know that for every $\ell<\sup_{u\in S}|u|$ it holds that $\AmbStr{\overline{S}_\ell}$ is isomorphic to $\AmbStr{\overline{f'[S]}_{N_{3\ell}}}$.
	By Proposition~\ref{prop:levelstr}, for every $0<\ell<\sup_{u\in S}|u|$ the there is only one difference between  $\AmbStr{\overline{S}_{\ell-1}}$ and  $\AmbStr{\overline{S}_{\ell}}$.
	Consequently there is only one interesting level $\ell'$ of $\overline{f'[S]}$ between $N_{3\ell}$ and $N_{3\ell+3}$ and $\AmbStr{\overline{f'[S]}_{N_{\ell'}}}$ is isomorphic
	to  $\AmbStr{\overline{S}_\ell}$  while  $\AmbStr{\overline{f'[S]}_{N_{\ell'+1}}}$ is isomorphic to $\AmbStr{\overline{S}_{\ell+1}}$. We further note that by the construction of $g$, the map $\tau_{f'[S]}\circ f'$ must be the identity on $S$.

	We can thus put $\phi= \eta\circ g\circ d$.
\end{proof}


\begin{proof}[Proof of Theorem~\ref{thm:main}]
Given a finite poset $\str{A}$, the fact that the big Ramsey degree of $\str{A}$ is exactly $|T^{lab}(\str{A})| = |T(\str{A})|\cdot |\mathrm{Aut}(\str{A})|$ follows from Corollary~\ref{cor:big_ramsey} and Theorem~\ref{thm:embthm}. To conclude that $\str{P}$ admits a big Ramsey structure, we observe that the colorings $\chi_{\str{A}}$ as $\str{A}$ ranges over all finite posets satisfy the hypotheses of Theorem~7.1 from \cite{zucker2017}. Theorem~1.6 from \cite{zucker2017} then shows that $\mathrm{Aut}(\str{P})$ admits a metrizable universal completion flow.
\end{proof}


\section{Concluding remarks}
\subsection{Comparsion to big Ramsey degrees of the order of rationals}
It is natural to ask how the characterisation of big Ramsey degrees of partial orders compares to
that of linear orders.
The big Ramsey degrees of the linear order correspond to \emph{Devlin's diaries} or \emph{types} which, in our setting, can be defined as follows:
\begin{definition}[Devlin diary~\cite{devlin1979}, see also \cite{todorcevic2010introduction}, Definition~6.9]
	A set $S\subseteq \{\L,\R\}^*$ is called a \emph{Devlin embedding type}, if no member of $S$ extends any other and precisely one of the following four conditions is satisfied for every level $\ell$ with $0\leq \ell< \sup_{w\in S}|w|$:
	\begin{enumerate}
		\item \textbf{Leaf:}  There is $w\in \overline{S}_\ell$ and
		      \begin{align*}
			      \qquad \overline{S}_{\ell+1} & =(\overline{S}_\ell\setminus \{w\} )\cont \L.
		      \end{align*}
		\item \textbf{Splitting:}  There is $w\in \overline{S}_\ell$ such that
		      \begin{align*}
			      \begin{split}
				      \qquad \overline{S}_{\ell+1}&=\{z\in \overline{S}_\ell:z\lexlt w\}\cont \L\\
				      &\qquad \cup\{w\cont \L,w\cont\R\}\\
				      &\qquad \cup \{z\in \overline{S}_\ell:w\lexlt z\}\cont \R.
			      \end{split}
		      \end{align*}
	\end{enumerate}
	When $S$ is a Devlin embedding type, we call $\overline{S}$ a \emph{Devlin tree}. Given $n\in \omega$, we let $T'(n)$ be the set of all Devlin embedding types of size $n$.
\end{definition}

\begin{theorem}[Devlin~\cite{devlin1979}, see also \cite{todorcevic2010introduction}]
\label{thm:devlin}
	For every $n\in \omega$, the big Ramsey degree of $(n,\leq)$ in the order of rationals equals $|T'(n)|$.
\end{theorem}
Any infinite Devlin tree whose leaves code the rational order is a big Ramsey structure for the rationals, see also~\cite{zucker2017}.

Comparing Devlin trees to poset-diaries is thus relatively natural.  In a Devlin tree, only two types of events happen: splitting and leaf.
In a poset-diary, the splitting event is different.  If $w$ splits on its level then $w\cont \X$ and $w\cont L$ are incomparable in $\preceq$.  This adds a need
for additional two events: new $\prec$ and new $\perp$, deciding the poset structure between the successors of $w$.
\subsection{The triangle-free graph}
\iffalse
This paper builds on three new proof techniques
\begin{enumerate}
	\item the notion of shape-preserving functions which formalize a form of sub-tree tailored to specific amalgamation class,
	\item corresponding Ramsey-type theorem,
	\item lower bound proof based on iteration of the Ramsey-type theorem.
\end{enumerate}
These techniques can be adapted to other triangle constrained classes in finite
binary languages.  
\fi
We outline how the techniques introduced in this paper can yield a particularly compact characterisation of the big
Ramsey degrees of the generic triangle-free graph. This is a special case of the main result of~\cite{Balko2021exact} (see Example 6.2.10).
However, we now give a more compact description which is similar to Definition~\ref{def:posetdiary}.

We fix an alphabet $\Sigma=\{0,1\}$ and denote by $\Sigma^*$ the set of all
finite words in the alphabet $\Sigma$, by $\lexleq$ their lexicographic order,
and by $|w|$ the length of the word $w$ (whose characters are indexed by
natural numbers starting at $0$).  We will denote the empty word by $\emptyset$.

We consider the following triangle-free graph $\str{G}^\triangle$ introduced
in~\cite{Hubicka2020CS}.

\begin{definition}[Graph $\str{G}^\triangle$]~
	\begin{enumerate}
 		\item Vertex set $G$ of $\str{G}^\triangle$ is $\Sigma^*$.
 		\item Given $u,v\in G$ satisfying $|u|<|v|$ we make $u$ and $v$ adjacent if and only $v_{|u|}=1$ and there is no $i$ satisfying $0\leq i<|u|$ and $u_i=v_i=1$.
		\item There are no other edges in $\str{G}^\triangle$.
	\end{enumerate}
\end{definition}
\begin{definition}[Relation $\perp$]
	Given $u,v\in G$ with $|u|\leq |v|$, we write $u\perp v$ if one of the following is satisfied:
	\begin{enumerate}
		\item There exists $i$ satisfying $0\leq i<|u|$ and $u_i=v_i=1$.
		\item There is no $i$ satisfying $0\leq i<|u|$ and $v_i=1$.
		\item There is no $i$ satisfying $0\leq i<|v|$ and $u_i=1$.
	\end{enumerate}
\end{definition}



\iffalse
Given a word $w$ and an integer $i \geq 0$, we denote by $w|_i$ the \emph{initial segment} of $w$ of length $i$.
For $S\subseteq \Sigma^*$, we let $\overline{S}$ be the set $\{w|_i\colon w\in S, 0\leq i\leq |w|\}$.
The set $\Sigma^*$ can be seen as a rooted ternary tree and sets $S=\overline{S}\subseteq\Sigma^*$ as its subtrees.
Given $i\geq 0$, we let $S_i=\{w\in S:|w|=i\}$ and call it the \emph{level $i$ of $S$}.
A word $w\in S$ is called a \emph{leaf} of $S$ if there is no word $w'\in S$ extending $w$.
Let $L(S)$ be the set of all leafs of $S$. Given a word $w$ and a character $c\in \Sigma$, we denote by $w\cont c$
the word created from $w$ by adding $c$ to the end of $w$. We also set $S\cont c=\{w\cont c:w\in S\}$.
\fi

Let $\str{R}_3$ denote the generic triangle-free graph. To characterise big Ramsey degrees of $\str{R}_3$, we introduce the following  technical definition.

\begin{definition}[Triangle-free diaries]
	\label{def:trianglediary}
	A set $S\subseteq \Sigma^*$ is called a \emph{triangle-free-type} if no member of $S$ extends any other and precisely one of the following four conditions is satisfied for every $i$ with $0\leq i< \sup_{w\in S}|w|$:
	\begin{enumerate}
		\item \textbf{Leaf:}  There is $w\in \overline{S}_i$ with $w\neq 0^i$ such that for every distinct $u,v\in \{z\in \overline{S}_i\setminus\{w\}:z\not\perp w\}$ it holds that $u \perp v$. 
			Moreover:
			\[
			      \overline{S}_{i+1}=\{z\in S_i\setminus\{w\}:z\perp w\}\cont 0 \cup \{z\in S_i\setminus\{w\}:z\not\perp w\}\cont 1. 
			\]
		\item \textbf{Splitting:}  There is $w\in \overline{S}_i$ such that 
			\[
			      \overline{S}_{i+1}=\overline{S}_i\cont 0\cup \{w\}\cont 1.
			\]
		\item \textbf{Non-splitting first neighbour:}  $0^i\in\overline{S}_i$ and 
			\[
			      \overline{S}_{i+1}=(S_i\setminus \{0^i\})\cont 0\cup \{{0^i}\cont 1\}.
			\]
		\item \textbf{New $\perp$:} There are distinct words $v,w\in \overline{S}_i$ with $0^i\not\in \{v, w\}$, $v\not\perp w$ such that
			\[
			      \overline{S}_{i+1}=(\overline{S}_i\setminus \{v,w\})\cont 0\cup \{v,w\}\cont 1.
			\]
	\end{enumerate}
\end{definition}

Given a triangle-free graph $\str{H}$, we let $T^\triangle(\str{H})$ be the set of all triangle-free-types $S$ such that the structure induced by $\str{G}^\triangle$ on $L(S)$ is isomorphic to $\str{H}$. We can now recover the following result of~\cite{Balko2021exact}. 
\iffalse
Notice that the relation $\perp$ describes pairs of vertices which can not be connected by an edge for some reason (i.e.,\ both are adjacent to some common previous vertex) and in the triangle-free types it holds that every non-edge has a reason.
\fi

\begin{theorem}[\cite{Balko2021exact}]\label{thm:trianglefree}
	For every finite triangle-free graph $\str{H}$, the big Ramsey degree of $\str{H}$ in the generic triangle-free graph $\str{R}_3$ equals $|T^\triangle(\str{H})|\cdot |\mathrm{Aut}(\str{H})|$.
\end{theorem}

\section*{Acknowledgement}
J.~H. and M.~K. are supported by the project 21-10775S of  the  Czech  Science Foundation (GA\v CR). This article is part of a project that has received funding from the European Research Council (ERC) under the European Union's Horizon 2020 research and innovation programme (grant agreement No 810115).
M.~B. and J.~H. were supported by the Center for Foundations of Modern Computer Science (Charles University project UNCE/SCI/004). 
N.~D.\ is supported by National Science Foundation grant DMS-1901753. A.~Z.\ was supported by National Science Foundation grant DMS-2054302. L.~V. is supported by Beatriu de Pin\'os BP2018, funded by the AGAUR (Government of Catalonia) and by the Horizon 2020 programme No 801370.

\bibliographystyle{alpha}
\bibliography{ramsey.bib}
\end{document}
