
% IEEEtran V1.7 and later provides for these CLASSINPUT macros to allow the
% user to reprogram some IEEEtran.cls defaults if needed. These settings
% override the internal defaults of IEEEtran.cls regardless of which class
% options are used. Do not use these unless you have good reason to do so as
% they can result in nonIEEE compliant documents. User beware. ;)
%
%\newcommand{\CLASSINPUTbaselinestretch}{1.0} % baselinestretch
%\newcommand{\CLASSINPUTinnersidemargin}{1in} % inner side margin
%\newcommand{\CLASSINPUToutersidemargin}{1in} % outer side margin
%\newcommand{\CLASSINPUTtoptextmargin}{1in}   % top text margin
%\newcommand{\CLASSINPUTbottomtextmargin}{1in}% bottom text margin




%
\documentclass[10pt,journal,compsoc]{IEEEtran}
% If IEEEtran.cls has not been installed into the LaTeX system files,
% manually specify the path to it like:
% \documentclass[10pt,journal,compsoc]{../sty/IEEEtran}


% For Computer Society journals, IEEEtran defaults to the use of 
% Palatino/Palladio as is done in IEEE Computer Society journals.
% To go back to Times Roman, you can use this code:
%\renewcommand{\rmdefault}{ptm}\selectfont





% Some very useful LaTeX packages include:
% (uncomment the ones you want to load)



% *** MISC UTILITY PACKAGES ***
%
%\usepackage{ifpdf}
% Heiko Oberdiek's ifpdf.sty is very useful if you need conditional
% compilation based on whether the output is pdf or dvi.
% usage:
% \ifpdf
%   % pdf code
% \else
%   % dvi code
% \fi
% The latest version of ifpdf.sty can be obtained from:
% http://www.ctan.org/pkg/ifpdf
% Also, note that IEEEtran.cls V1.7 and later provides a builtin
% \ifCLASSINFOpdf conditional that works the same way.
% When switching from latex to pdflatex and vice-versa, the compiler may
% have to be run twice to clear warning/error messages.




% *** CITATION PACKAGES ***
%
\ifCLASSOPTIONcompsoc
  % The IEEE Computer Society needs nocompress option
  % requires cite.sty v4.0 or later (November 2003)
  \usepackage[nocompress]{cite}
\else
  % normal IEEE
  \usepackage{cite}
\fi
% cite.sty was written by Donald Arseneau
% V1.6 and later of IEEEtran pre-defines the format of the cite.sty package
% \cite{} output to follow that of the IEEE. Loading the cite package will
% result in citation numbers being automatically sorted and properly
% "compressed/ranged". e.g., [1], [9], [2], [7], [5], [6] without using
% cite.sty will become [1], [2], [5]--[7], [9] using cite.sty. cite.sty's
% \cite will automatically add leading space, if needed. Use cite.sty's
% noadjust option (cite.sty V3.8 and later) if you want to turn this off
% such as if a citation ever needs to be enclosed in parenthesis.
% cite.sty is already installed on most LaTeX systems. Be sure and use
% version 5.0 (2009-03-20) and later if using hyperref.sty.
% The latest version can be obtained at:
% http://www.ctan.org/pkg/cite
% The documentation is contained in the cite.sty file itself.
%
% Note that some packages require special options to format as the Computer
% Society requires. In particular, Computer Society  papers do not use
% compressed citation ranges as is done in typical IEEE papers
% (e.g., [1]-[4]). Instead, they list every citation separately in order
% (e.g., [1], [2], [3], [4]). To get the latter we need to load the cite
% package with the nocompress option which is supported by cite.sty v4.0
% and later.





% *** GRAPHICS RELATED PACKAGES ***
%
\ifCLASSINFOpdf
  % \usepackage[pdftex]{graphicx}
  % declare the path(s) where your graphic files are
  % \graphicspath{{../pdf/}{../jpeg/}}
  % and their extensions so you won't have to specify these with
  % every instance of \includegraphics
  % \DeclareGraphicsExtensions{.pdf,.jpeg,.png}
\else
  % or other class option (dvipsone, dvipdf, if not using dvips). graphicx
  % will default to the driver specified in the system graphics.cfg if no
  % driver is specified.
  % \usepackage[dvips]{graphicx}
  % declare the path(s) where your graphic files are
  % \graphicspath{{../eps/}}
  % and their extensions so you won't have to specify these with
  % every instance of \includegraphics
  % \DeclareGraphicsExtensions{.eps}
\fi
% graphicx was written by David Carlisle and Sebastian Rahtz. It is
% required if you want graphics, photos, etc. graphicx.sty is already
% installed on most LaTeX systems. The latest version and documentation
% can be obtained at: 
% http://www.ctan.org/pkg/graphicx
% Another good source of documentation is "Using Imported Graphics in
% LaTeX2e" by Keith Reckdahl which can be found at:
% http://www.ctan.org/pkg/epslatex
%
% latex, and pdflatex in dvi mode, support graphics in encapsulated
% postscript (.eps) format. pdflatex in pdf mode supports graphics
% in .pdf, .jpeg, .png and .mps (metapost) formats. Users should ensure
% that all non-photo figures use a vector format (.eps, .pdf, .mps) and
% not a bitmapped formats (.jpeg, .png). The IEEE frowns on bitmapped formats
% which can result in "jaggedy"/blurry rendering of lines and letters as
% well as large increases in file sizes.
%
% You can find documentation about the pdfTeX application at:
% http://www.tug.org/applications/pdftex





% *** MATH PACKAGES ***
%
%\usepackage{amsmath}
% A popular package from the American Mathematical Society that provides
% many useful and powerful commands for dealing with mathematics.
%
% Note that the amsmath package sets \interdisplaylinepenalty to 10000
% thus preventing page breaks from occurring within multiline equations. Use:
%\interdisplaylinepenalty=2500
% after loading amsmath to restore such page breaks as IEEEtran.cls normally
% does. amsmath.sty is already installed on most LaTeX systems. The latest
% version and documentation can be obtained at:
% http://www.ctan.org/pkg/amsmath





% *** SPECIALIZED LIST PACKAGES ***
%\usepackage{acronym}
% acronym.sty was written by Tobias Oetiker. This package provides tools for
% managing documents with large numbers of acronyms. (You don't *have* to
% use this package - unless you have a lot of acronyms, you may feel that
% such package management of them is bit of an overkill.)
% Do note that the acronym environment (which lists acronyms) will have a
% problem when used under IEEEtran.cls because acronym.sty relies on the
% description list environment - which IEEEtran.cls has customized for
% producing IEEE style lists. A workaround is to declared the longest
% label width via the IEEEtran.cls \IEEEiedlistdecl global control:
%
% \renewcommand{\IEEEiedlistdecl}{\IEEEsetlabelwidth{SONET}}
% \begin{acronym}
%
% \end{acronym}
% \renewcommand{\IEEEiedlistdecl}{\relax}% remember to reset \IEEEiedlistdecl
%
% instead of using the acronym environment's optional argument.
% The latest version and documentation can be obtained at:
% http://www.ctan.org/pkg/acronym


%\usepackage{algorithmic}
% algorithmic.sty was written by Peter Williams and Rogerio Brito.
% This package provides an algorithmic environment fo describing algorithms.
% You can use the algorithmic environment in-text or within a figure
% environment to provide for a floating algorithm. Do NOT use the algorithm
% floating environment provided by algorithm.sty (by the same authors) or
% algorithm2e.sty (by Christophe Fiorio) as the IEEE does not use dedicated
% algorithm float types and packages that provide these will not provide
% correct IEEE style captions. The latest version and documentation of
% algorithmic.sty can be obtained at:
% http://www.ctan.org/pkg/algorithms
% Also of interest may be the (relatively newer and more customizable)
% algorithmicx.sty package by Szasz Janos:
% http://www.ctan.org/pkg/algorithmicx




% *** ALIGNMENT PACKAGES ***
%
%\usepackage{array}
% Frank Mittelbach's and David Carlisle's array.sty patches and improves
% the standard LaTeX2e array and tabular environments to provide better
% appearance and additional user controls. As the default LaTeX2e table
% generation code is lacking to the point of almost being broken with
% respect to the quality of the end results, all users are strongly
% advised to use an enhanced (at the very least that provided by array.sty)
% set of table tools. array.sty is already installed on most systems. The
% latest version and documentation can be obtained at:
% http://www.ctan.org/pkg/array


%\usepackage{mdwmath}
%\usepackage{mdwtab}
% Also highly recommended is Mark Wooding's extremely powerful MDW tools,
% especially mdwmath.sty and mdwtab.sty which are used to format equations
% and tables, respectively. The MDWtools set is already installed on most
% LaTeX systems. The lastest version and documentation is available at:
% http://www.ctan.org/pkg/mdwtools


% IEEEtran contains the IEEEeqnarray family of commands that can be used to
% generate multiline equations as well as matrices, tables, etc., of high
% quality.


%\usepackage{eqparbox}
% Also of notable interest is Scott Pakin's eqparbox package for creating
% (automatically sized) equal width boxes - aka "natural width parboxes".
% Available at:
% http://www.ctan.org/pkg/eqparbox




% *** SUBFIGURE PACKAGES ***
%\ifCLASSOPTIONcompsoc
%  \usepackage[caption=false,font=footnotesize,labelfont=sf,textfont=sf]{subfig}
%\else
%  \usepackage[caption=false,font=footnotesize]{subfig}
%\fi
% subfig.sty, written by Steven Douglas Cochran, is the modern replacement
% for subfigure.sty, the latter of which is no longer maintained and is
% incompatible with some LaTeX packages including fixltx2e. However,
% subfig.sty requires and automatically loads Axel Sommerfeldt's caption.sty
% which will override IEEEtran.cls' handling of captions and this will result
% in non-IEEE style figure/table captions. To prevent this problem, be sure
% and invoke subfig.sty's "caption=false" package option (available since
% subfig.sty version 1.3, 2005/06/28) as this is will preserve IEEEtran.cls
% handling of captions.
% Note that the Computer Society format requires a sans serif font rather
% than the serif font used in traditional IEEE formatting and thus the need
% to invoke different subfig.sty package options depending on whether
% compsoc mode has been enabled.
%
% The latest version and documentation of subfig.sty can be obtained at:
% http://www.ctan.org/pkg/subfig




% *** FLOAT PACKAGES ***
%
%\usepackage{fixltx2e}
% fixltx2e, the successor to the earlier fix2col.sty, was written by
% Frank Mittelbach and David Carlisle. This package corrects a few problems
% in the LaTeX2e kernel, the most notable of which is that in current
% LaTeX2e releases, the ordering of single and double column floats is not
% guaranteed to be preserved. Thus, an unpatched LaTeX2e can allow a
% single column figure to be placed prior to an earlier double column
% figure.
% Be aware that LaTeX2e kernels dated 2015 and later have fixltx2e.sty's
% corrections already built into the system in which case a warning will
% be issued if an attempt is made to load fixltx2e.sty as it is no longer
% needed.
% The latest version and documentation can be found at:
% http://www.ctan.org/pkg/fixltx2e


%\usepackage{stfloats}
% stfloats.sty was written by Sigitas Tolusis. This package gives LaTeX2e
% the ability to do double column floats at the bottom of the page as well
% as the top. (e.g., "\begin{figure*}[!b]" is not normally possible in
% LaTeX2e). It also provides a command:
%\fnbelowfloat
% to enable the placement of footnotes below bottom floats (the standard
% LaTeX2e kernel puts them above bottom floats). This is an invasive package
% which rewrites many portions of the LaTeX2e float routines. It may not work
% with other packages that modify the LaTeX2e float routines. The latest
% version and documentation can be obtained at:
% http://www.ctan.org/pkg/stfloats
% Do not use the stfloats baselinefloat ability as the IEEE does not allow
% \baselineskip to stretch. Authors submitting work to the IEEE should note
% that the IEEE rarely uses double column equations and that authors should try
% to avoid such use. Do not be tempted to use the cuted.sty or midfloat.sty
% packages (also by Sigitas Tolusis) as the IEEE does not format its papers in
% such ways.
% Do not attempt to use stfloats with fixltx2e as they are incompatible.
% Instead, use Morten Hogholm'a dblfloatfix which combines the features
% of both fixltx2e and stfloats:
%
% \usepackage{dblfloatfix}
% The latest version can be found at:
% http://www.ctan.org/pkg/dblfloatfix


%\ifCLASSOPTIONcaptionsoff
%  \usepackage[nomarkers]{endfloat}
% \let\MYoriglatexcaption\caption
% \renewcommand{\caption}[2][\relax]{\MYoriglatexcaption[#2]{#2}}
%\fi
% endfloat.sty was written by James Darrell McCauley, Jeff Goldberg and 
% Axel Sommerfeldt. This package may be useful when used in conjunction with 
% IEEEtran.cls'  captionsoff option. Some IEEE journals/societies require that
% submissions have lists of figures/tables at the end of the paper and that
% figures/tables without any captions are placed on a page by themselves at
% the end of the document. If needed, the draftcls IEEEtran class option or
% \CLASSINPUTbaselinestretch interface can be used to increase the line
% spacing as well. Be sure and use the nomarkers option of endfloat to
% prevent endfloat from "marking" where the figures would have been placed
% in the text. The two hack lines of code above are a slight modification of
% that suggested by in the endfloat docs (section 8.4.1) to ensure that
% the full captions always appear in the list of figures/tables - even if
% the user used the short optional argument of \caption[]{}.
% IEEE papers do not typically make use of \caption[]'s optional argument,
% so this should not be an issue. A similar trick can be used to disable
% captions of packages such as subfig.sty that lack options to turn off
% the subcaptions:
% For subfig.sty:
% \let\MYorigsubfloat\subfloat
% \renewcommand{\subfloat}[2][\relax]{\MYorigsubfloat[]{#2}}
% However, the above trick will not work if both optional arguments of
% the \subfloat command are used. Furthermore, there needs to be a
% description of each subfigure *somewhere* and endfloat does not add
% subfigure captions to its list of figures. Thus, the best approach is to
% avoid the use of subfigure captions (many IEEE journals avoid them anyway)
% and instead reference/explain all the subfigures within the main caption.
% The latest version of endfloat.sty and its documentation can obtained at:
% http://www.ctan.org/pkg/endfloat
%
% The IEEEtran \ifCLASSOPTIONcaptionsoff conditional can also be used
% later in the document, say, to conditionally put the References on a 
% page by themselves.





% *** PDF, URL AND HYPERLINK PACKAGES ***
%
%\usepackage{url}
% url.sty was written by Donald Arseneau. It provides better support for
% handling and breaking URLs. url.sty is already installed on most LaTeX
% systems. The latest version and documentation can be obtained at:
% http://www.ctan.org/pkg/url
% Basically, \url{my_url_here}.


% NOTE: PDF thumbnail features are not required in IEEE papers
%       and their use requires extra complexity and work.
%\ifCLASSINFOpdf
%  \usepackage[pdftex]{thumbpdf}
%\else
%  \usepackage[dvips]{thumbpdf}
%\fi
% thumbpdf.sty and its companion Perl utility were written by Heiko Oberdiek.
% It allows the user a way to produce PDF documents that contain fancy
% thumbnail images of each of the pages (which tools like acrobat reader can
% utilize). This is possible even when using dvi->ps->pdf workflow if the
% correct thumbpdf driver options are used. thumbpdf.sty incorporates the
% file containing the PDF thumbnail information (filename.tpm is used with
% dvips, filename.tpt is used with pdftex, where filename is the base name of
% your tex document) into the final ps or pdf output document. An external
% utility, the thumbpdf *Perl script* is needed to make these .tpm or .tpt
% thumbnail files from a .ps or .pdf version of the document (which obviously
% does not yet contain pdf thumbnails). Thus, one does a:
% 
% thumbpdf filename.pdf 
%
% to make a filename.tpt, and:
%
% thumbpdf --mode dvips filename.ps
%
% to make a filename.tpm which will then be loaded into the document by
% thumbpdf.sty the NEXT time the document is compiled (by pdflatex or
% latex->dvips->ps2pdf). Users must be careful to regenerate the .tpt and/or
% .tpm files if the main document changes and then to recompile the
% document to incorporate the revised thumbnails to ensure that thumbnails
% match the actual pages. It is easy to forget to do this!
% 
% Unix systems come with a Perl interpreter. However, MS Windows users
% will usually have to install a Perl interpreter so that the thumbpdf
% script can be run. The Ghostscript PS/PDF interpreter is also required.
% See the thumbpdf docs for details. The latest version and documentation
% can be obtained at.
% http://www.ctan.org/pkg/thumbpdf


% NOTE: PDF hyperlink and bookmark features are not required in IEEE
%       papers and their use requires extra complexity and work.
% *** IF USING HYPERREF BE SURE AND CHANGE THE EXAMPLE PDF ***
% *** TITLE/SUBJECT/AUTHOR/KEYWORDS INFO BELOW!!           ***
\newcommand\MYhyperrefoptions{bookmarks=true,bookmarksnumbered=true,
pdfpagemode={UseOutlines},plainpages=false,pdfpagelabels=true,
colorlinks=true,linkcolor={black},citecolor={black},urlcolor={black},
pdftitle={Bare Demo of IEEEtran.cls for Computer Society Journals},%<!CHANGE!
pdfsubject={Typesetting},%<!CHANGE!
pdfauthor={Michael D. Shell},%<!CHANGE!
pdfkeywords={Computer Society, IEEEtran, journal, LaTeX, paper,
             template}}%<^!CHANGE!
%\ifCLASSINFOpdf
%\usepackage[\MYhyperrefoptions,pdftex]{hyperref}
%\else
%\usepackage[\MYhyperrefoptions,breaklinks=true,dvips]{hyperref}
%\usepackage{breakurl}
%\fi
% One significant drawback of using hyperref under DVI output is that the
% LaTeX compiler cannot break URLs across lines or pages as can be done
% under pdfLaTeX's PDF output via the hyperref pdftex driver. This is
% probably the single most important capability distinction between the
% DVI and PDF output. Perhaps surprisingly, all the other PDF features
% (PDF bookmarks, thumbnails, etc.) can be preserved in
% .tex->.dvi->.ps->.pdf workflow if the respective packages/scripts are
% loaded/invoked with the correct driver options (dvips, etc.). 
% As most IEEE papers use URLs sparingly (mainly in the references), this
% may not be as big an issue as with other publications.
%
% That said, Vilar Camara Neto created his breakurl.sty package which
% permits hyperref to easily break URLs even in dvi mode.
% Note that breakurl, unlike most other packages, must be loaded
% AFTER hyperref. The latest version of breakurl and its documentation can
% be obtained at:
% http://www.ctan.org/pkg/breakurl
% breakurl.sty is not for use under pdflatex pdf mode.
%
% The advanced features offer by hyperref.sty are not required for IEEE
% submission, so users should weigh these features against the added
% complexity of use.
% The package options above demonstrate how to enable PDF bookmarks
% (a type of table of contents viewable in Acrobat Reader) as well as
% PDF document information (title, subject, author and keywords) that is
% viewable in Acrobat reader's Document_Properties menu. PDF document
% information is also used extensively to automate the cataloging of PDF
% documents. The above set of options ensures that hyperlinks will not be
% colored in the text and thus will not be visible in the printed page,
% but will be active on "mouse over". USING COLORS OR OTHER HIGHLIGHTING
% OF HYPERLINKS CAN RESULT IN DOCUMENT REJECTION BY THE IEEE, especially if
% these appear on the "printed" page. IF IN DOUBT, ASK THE RELEVANT
% SUBMISSION EDITOR. You may need to add the option hypertexnames=false if
% you used duplicate equation numbers, etc., but this should not be needed
% in normal IEEE work.
% The latest version of hyperref and its documentation can be obtained at:
% http://www.ctan.org/pkg/hyperref





% *** Do not adjust lengths that control margins, column widths, etc. ***
% *** Do not use packages that alter fonts (such as pslatex).         ***
% There should be no need to do such things with IEEEtran.cls V1.6 and later.
% (Unless specifically asked to do so by the journal or conference you plan
% to submit to, of course. )

\usepackage{amsmath,amsfonts}
\usepackage{algorithmic}
\usepackage{algorithm}
\usepackage{array}
\usepackage{url}
\usepackage{verbatim}
\usepackage{graphicx}
\usepackage{cite}
\usepackage{booktabs}
\usepackage{multirow}
\usepackage{xspace}
\newcommand{\ie}{{\emph{i.e.}},\xspace}
\newcommand{\eg}{{\emph{e.g.}},\xspace}
\newcommand{\etc}{etc.}

\usepackage{subfig}

\usepackage{bbding}

\usepackage{hyperref} 
\hypersetup{
    colorlinks=True,
    linkcolor=magenta,
    filecolor=magenta,      
    urlcolor=magenta,
    citecolor=cyan,
}

\newcommand{\INDSTATE}[1][1]{\STATE\hspace{#1\algorithmicindent}}
\usepackage{color}
\definecolor{GG}{RGB}{48,144,96}

\usepackage{subfig} %subfloat






% correct bad hyphenation here
\hyphenation{op-tical net-works semi-conduc-tor}


\begin{document}
%
% paper title
% Titles are generally capitalized except for words such as a, an, and, as,
% at, but, by, for, in, nor, of, on, or, the, to and up, which are usually
% not capitalized unless they are the first or last word of the title.
% Linebreaks \\ can be used within to get better formatting as desired.
% Do not put math or special symbols in the title.
\title{Adaptive Outlier Optimization for Test-time Out-of-Distribution Detection}
%
%
% author names and IEEE memberships
% note positions of commas and nonbreaking spaces ( ~ ) LaTeX will not break
% a structure at a ~ so this keeps an author's name from being broken across
% two lines.
% use \thanks{} to gain access to the first footnote area
% a separate \thanks must be used for each paragraph as LaTeX2e's \thanks
% was not built to handle multiple paragraphs
%
%
%\IEEEcompsocitemizethanks is a special \thanks that produces the bulleted
% lists the Computer Society journals use for "first footnote" author
% affiliations. Use \IEEEcompsocthanksitem which works much like \item
% for each affiliation group. When not in compsoc mode,
% \IEEEcompsocitemizethanks becomes like \thanks and
% \IEEEcompsocthanksitem becomes a line break with idention. This
% facilitates dual compilation, although admittedly the differences in the
% desired content of \author between the different types of papers makes a
% one-size-fits-all approach a daunting prospect. For instance, compsoc 
% journal papers have the author affiliations above the "Manuscript
% received ..."  text while in non-compsoc journals this is reversed. Sigh.

\author{Puning~Yang, Jian~Liang, Jie~Cao, and~Ran~He\\

%Michael~Shell,~\IEEEmembership{Member,~IEEE,}
%        John~Doe,~\IEEEmembership{Fellow,~OSA,}
%        and~Jane~Doe,~\IEEEmembership{Life~Fellow,~IEEE}% <-this % stops a space
\IEEEcompsocitemizethanks{\IEEEcompsocthanksitem 
Puning Yang, Jian Liang, Jie Cao, Ran He are with the State Key Laboratory of Multimodal Artificial Intelligence Systems, CASIA, Center for Research on Intelligent Perception and Computing, CASIA, Center for Excellence in Brain Science and Intelligence Technology, CAS, and School of Artificial Intelligence, University of Chinese Academy of Sciences, Beijing 100190, China. E-mail: \{puning.yang, jie.cao\}@cripac.ia.ac.cn, liangjian92@gmail.com, rhe@nlpr.ia.ac.cn. (Corresponding author: Ran He.)}% <-this % stops a space
\thanks{Preprint}}

% note the % following the last \IEEEmembership and also \thanks - 
% these prevent an unwanted space from occurring between the last author name
% and the end of the author line. i.e., if you had this:
% 
% \author{....lastname \thanks{...} \thanks{...} }
%                     ^------------^------------^----Do not want these spaces!
%
% a space would be appended to the last name and could cause every name on that
% line to be shifted left slightly. This is one of those "LaTeX things". For
% instance, "\textbf{A} \textbf{B}" will typeset as "A B" not "AB". To get
% "AB" then you have to do: "\textbf{A}\textbf{B}"
% \thanks is no different in this regard, so shield the last } of each \thanks
% that ends a line with a % and do not let a space in before the next \thanks.
% Spaces after \IEEEmembership other than the last one are OK (and needed) as
% you are supposed to have spaces between the names. For what it is worth,
% this is a minor point as most people would not even notice if the said evil
% space somehow managed to creep in.



% The paper headers
\markboth{Journal of \LaTeX\ Class Files,~Vol.~14, No.~8, August~2015}%
{Shell \MakeLowercase{\textit{et al.}}: Bare Advanced Demo of IEEEtran.cls for IEEE Computer Society Journals}
% The only time the second header will appear is for the odd numbered pages
% after the title page when using the twoside option.
% 
% *** Note that you probably will NOT want to include the author's ***
% *** name in the headers of peer review papers.                   ***
% You can use \ifCLASSOPTIONpeerreview for conditional compilation here if
% you desire.



% The publisher's ID mark at the bottom of the page is less important with
% Computer Society journal papers as those publications place the marks
% outside of the main text columns and, therefore, unlike regular IEEE
% journals, the available text space is not reduced by their presence.
% If you want to put a publisher's ID mark on the page you can do it like
% this:
%\IEEEpubid{0000--0000/00\$00.00~\copyright~2015 IEEE}
% or like this to get the Computer Society new two part style.
%\IEEEpubid{\makebox[\columnwidth]{\hfill 0000--0000/00/\$00.00~\copyright~2015 IEEE}%
%\hspace{\columnsep}\makebox[\columnwidth]{Published by the IEEE Computer Society\hfill}}
% Remember, if you use this you must call \IEEEpubidadjcol in the second
% column for its text to clear the IEEEpubid mark (Computer Society journal
% papers don't need this extra clearance.)



% use for special paper notices
%\IEEEspecialpapernotice{(Invited Paper)}



% for Computer Society papers, we must declare the abstract and index terms
% PRIOR to the title within the \IEEEtitleabstractindextext IEEEtran
% command as these need to go into the title area created by \maketitle.
% As a general rule, do not put math, special symbols or citations
% in the abstract or keywords.
\IEEEtitleabstractindextext{%
\begin{abstract}
Out-of-distribution (OOD) detection aims to detect test samples that do not fall into any training in-distribution (ID) classes. Prior efforts focus on regularizing models with ID data only, largely underperforming counterparts that utilize auxiliary outliers. However, data safety and privacy make it infeasible to collect task-specific outliers in advance for different scenarios. Besides, using task-irrelevant outliers leads to inferior OOD detection performance. To address the above issue, we present a new setup called test-time OOD detection, which allows the deployed model to utilize real OOD data from the unlabeled data stream during testing. We propose Adaptive Outlier Optimization (AUTO) which allows for continuous adaptation of the OOD detector. Specifically, AUTO consists of three key components: 1) an in-out-aware filter to selectively annotate test samples with pseudo-ID and pseudo-OOD and ingeniously trigger the updating process while encountering each pseudo-OOD sample; 2) a dynamic-updated memory to overcome the catastrophic forgetting led by frequent parameter updates; 3) a prediction-aligning objective to calibrate the rough OOD objective during testing. Extensive experiments show that AUTO significantly improves OOD detection performance over state-of-the-art methods. Besides, evaluations on complicated scenarios (\eg multi-OOD, time-series OOD) also conduct the superiority of AUTO.

%Out-of-distribution (OOD) detection aims to detect test samples that do not fall into any training in-distribution (ID) classes.
%Prior efforts focus on optimizing models with ID data only and underperform recent works where auxiliary outliers are utilized for model regularization, showcasing the effectiveness of outliers.
%and achieves impressive performance on common OOD benchmarks.
%Compared with previous solutions, AUTO is OOD-free at the training stage and more general via optimizing with real-OOD data.
%However, in practice, real-world applications involve sequentially data stream, unknown and ever-shifting out-of-distribution data.
%However, in practice, real-world applications involve sequentially and ever-shifting OOD data, emerging new challenges for model deployment.
%Numerous prior works confine themselves to an offline manner, where OOD detectors are modified before predicting test samples and remain static and fixed after deployment.
%这里表述了一种现象,fixed的 现象会引起什么样的缺点和问题,然后才是address the problem
%Besides, the test setting in naive OOD detection is simplistic, where all test data arrive at once.
%To address this limitation in a more practical setting, we consider an online framework that allows OOD detector
%However, in practice, except for online data streams, real-world applications also involve unknown and ever-shifting OOD data, emerging new challenges for model deployment.
%In this paper, a more challenging setup is introduced, which emulates continuously evolving online data streams, and we 
\end{abstract}

% Note that keywords are not normally used for peerreview papers.
\begin{IEEEkeywords}
Out-of-distribution Detection, Test-time Optimization.
\end{IEEEkeywords}}


% make the title area
\maketitle


% To allow for easy dual compilation without having to reenter the
% abstract/keywords data, the \IEEEtitleabstractindextext text will
% not be used in maketitle, but will appear (i.e., to be "transported")
% here as \IEEEdisplaynontitleabstractindextext when compsoc mode
% is not selected <OR> if conference mode is selected - because compsoc
% conference papers position the abstract like regular (non-compsoc)
% papers do!
\IEEEdisplaynontitleabstractindextext
% \IEEEdisplaynontitleabstractindextext has no effect when using
% compsoc under a non-conference mode.


% For peer review papers, you can put extra information on the cover
% page as needed:
% \ifCLASSOPTIONpeerreview
% \begin{center} \bfseries EDICS Category: 3-BBND \end{center}
% \fi
%
% For peerreview papers, this IEEEtran command inserts a page break and
% creates the second title. It will be ignored for other modes.
\IEEEpeerreviewmaketitle


\ifCLASSOPTIONcompsoc
\IEEEraisesectionheading{\section{Introduction}\label{sec:introduction}}
\else
\section{Introduction}
\label{sec:introduction}
\fi
% Computer Society journal (but not conference!) papers do something unusual
% with the very first section heading (almost always called "Introduction").
% They place it ABOVE the main text! IEEEtran.cls does not automatically do
% this for you, but you can achieve this effect with the provided
% \IEEEraisesectionheading{} command. Note the need to keep any \label that
% is to refer to the section immediately after \section in the above as
% \IEEEraisesectionheading puts \section within a raised box.




% The very first letter is a 2 line initial drop letter followed
% by the rest of the first word in caps (small caps for compsoc).
% 
% form to use if the first word consists of a single letter:
% \IEEEPARstart{A}{demo} file is ....
% 
% form to use if you need the single drop letter followed by
% normal text (unknown if ever used by the IEEE):
% \IEEEPARstart{A}{}demo file is ....
% 
% Some journals put the first two words in caps:
% \IEEEPARstart{T}{his demo} file is ....
% 
% Here we have the typical use of a "T" for an initial drop letter
% and "HIS" in caps to complete the first word.
\IEEEPARstart{D}{eep} learning models have made tremendous progress in the past few years~\cite{lecun2015deep}.
The primitive to their success is that all classes that appear during inference are present in the training set.
However, such an assumption is criticized as it cannot always be satisfied in real-world applications where all the classes in the test phase would be available in the training phase.
Moreover, It is generally acknowledged that machine learning models often exhibit overconfident predictions on test samples that do not belong to any training classes~\cite{nguyen2015deep,bendale2016towards}.
This triggers a significant matter for robust, trustworthy, and safe AI applications, especially in 
autonomous driving~\cite{filos2020can}, fraud detection~\cite{phua2010comprehensive}, and medical diagnosis~\cite{roy2022does}.
To address the issue, recent efforts focus on out-of-distribution (OOD) detection~\cite{hendrycks2016baseline}.

\begin{table}[ht]
    %\setlength{\belowcaptionskip}{-20pt}
    \centering
    \caption{Comparisons of different detection paradigms.
    $\mathcal{D}^{aux}$ and $\mathcal{D}^{test}$ are datasets sampled from auxiliary data and test data, respectively.
    \textbf{Update}: Does model change at test time?
    \textbf{Online}: Does model continuously update?}
    \begin{tabular}{lccc}
    \toprule
       \multirow{2}{*}{Methods}  & \multicolumn{1}{c}{Training}& \multicolumn{2}{c}{Testing} \\
       &OOD&Update&Online\\
       \midrule
        Score Designing~\cite{hendrycks2016baseline} & -&\XSolidBrush&\XSolidBrush\\
        %ID Regularization~\cite{sehwag2020ssd}&$\mathcal{D}_{in}$&-&\ding{53}\\
        Outlier Exposure~\cite{hendrycks2018deep}&$\mathcal{D}^{aux}$&\XSolidBrush&\XSolidBrush\\
      WOODS~\cite{katzsamuels2022training}&$\mathcal{D}^{test}$&\XSolidBrush&\XSolidBrush\\
        Test Clustering~\cite{zhou2021step}&-&\Checkmark&\XSolidBrush\\
        Test-Time OOD Detection&-&\Checkmark&\Checkmark\\
        \bottomrule
    \end{tabular}
    
    \label{compare_methods}
\end{table}

The goal of OOD detection is to predict whether a test example is from a different distribution from the training data.
Existing solutions involve designing a scoring function that maps the input to the OOD score~\cite{hendrycks2016baseline,liang2018enhancing,lee2018simple,liu2020energy,hendrycks2022scaling} or modifying the training loss to mitigate the overconfident problem~\cite{wei2022mitigating,sehwag2020ssd,ming2023cider}.
Recent findings empirically suggest that exposing models to auxiliary outliers ($\mathcal{D}^{aux}$)~\cite{hendrycks2018deep,liu2020energy,ming2022posterior} significantly outperform the counterparts optimizing with ID data only, highlighting the potential of utilizing extra OOD data.
However, outliers collected offline may not closely match the true distribution of OOD data in deployments, failing to detect test OOD data.
Despite some efforts~\cite{katzsamuels2022training,zhou2021step,fan2022simple} that utilize test data for optimization, there remain some non-negligible limitations.
Firstly, previous test-time explorations require accessibility of test data, which is often impractical due to concerns related to data privacy and security.
Secondly, these explorations require utilizing mounts of test samples at one moment, which is difficult to achieve in online data streams.
Besides, all methods mentioned above are still evaluated in a simplistic offline test setting, where models remain fixed and static at test time (as shown in Table \ref{compare_methods}).
Such a restrictive setting hinders OOD detection in real-world environments, where wild data arrive sequentially and the OOD portion of the data is unknown and even ever-shifting.
For instance, in autonomous driving tasks, the deployed system struggles to perform well in an open environment where wild data arrive sequentially and gradually change with respect to geographic locations, time intervals, and other factors \cite{hoffman2014continuous}.
This motivates us to shift our perspective on OOD detection from the previous stationary to a dynamic setting.

In this paper, we pioneer a more practical setting called \emph{Test-Time OOD Detection (TTOD)}, where OOD detectors dynamically adapt to the current deployment scenario via continuous optimizations on unlabeled test data.
Specifically, the test samples in our setting arrive sequentially from either ID or OOD, inspiring a learner goal that incrementally updates the ID classifier and OOD detector based on predicted results of samples, and minimizes the risk of making incorrect predictions at each timestep.
Besides, the components in TTOD are more complex than those in the naive setting: (1) The ID and OOD mixture ratio are more flexible.
(2) More complex label shifts are considered, which include not only single-OOD scenarios but also multi-OOD and time-series OOD scenarios.
In contrast to the naive OOD detection setting, TTOD brings the benefits of practicality:
(1) Comprehensive: TTOD emulates various realistic scenarios, making it amenable to diverse real-world applications.
(2) Approximate: the online data stream provides a source of true distribution shift information, enabling TTOD to tailor the learning of an OOD detector for the deployed environment.

%In this new setting, we shift our focus entirely to the test phase, which enables OOD detectors to continuously optimize the model using unlabeled test data.
%Notably, test samples will arrive sequentially from either ID or OOD.
%Besides, the OOD detector is assumed static and fixed in the deployed environment.
%This restrictive setting hinders further enhancement of OOD detection performance in the real world.
%Besides, annotating such outliers requires meticulous data scrubbing to prevent any intersection between outliers information and ID data, which is resource-intensive.
%Although these methods have exhibited notable advancements in performance, all of them narrow their focus solely on optimizing models with training data, while disregarding the existence of unlabeled test data after deployment to enhance OOD detection performance.
%Such a restrictive setting hinders OOD detection in real-world environments, where wild data arrive online sequentially and the out-of-distribution portion of the data is unknown and even ever-shifting.
%For instance, in autonomous driving tasks, the deployed system struggles to perform well in an open environment where wild data arrive sequentially and gradually change with respect to geographic locations, time intervals, and other factors \cite{hoffman2014continuous}.
%This motivates us to shift our perspective on OOD detection from the previous stationary to a dynamic setting.

%\begin{figure}
%    \centering
%    \setlength{\belowcaptionskip}{-20pt}
%%    \includegraphics[width=0.9\linewidth]{pic/setting_1.png}
%    \caption{Demonstration of prior OOD detection setting and our test-time setting. Our new setting considers more %realistic scenarios, emulating the real-world environment better.}
%    \label{fig: TTODsetting}
%\end{figure}

%Building on the previous discussion, we pioneer a more practical and challenging setting called \emph{Test-Time OOD Detection (TTOD)}.
%In TTOD, test data will arrive sequentially from either ID or OOD.
%Besides, the OOD component in TTOD is more complex than that in naive OOD detection, where the ID and OOD mixture percentage is more flexible and more complex label shifts are considered, which include not only single-OOD scenarios but also multi-OOD and time-series OOD scenarios.
%A visual description of the new setting is shown in Fig. 1(b).
%Actually, in the context of this practical test stream, we will face the following challenges when performing TTOD.
%1) Learning from unlabeled data. Unlabeled data is unavailable for direct optimization due to the differing objectives between ID and out-of-distribution OOD.
%2) Catastrophic forgetting. In more detail, the model learns well on current OOD detection tasks but sacrifices its performance on source ID tasks.

After formalizing TTOD, we further present a framework called Adaptive oUTlier Optimization (AUTO).
The framework leverages encountered test samples in real-time to perform targeted optimizations on the deployed model, aiming to perform better in making predictions when subsequent test samples arrive.
It comprises three key components: an in-out-aware filter, a dynamic-updated memory bank, and a prediction-aligning objective.
Firstly, to annotate unlabeled test samples as pseudo-ID or pseudo-OOD, we design an in-out-aware filter which is initialized with the statistics of model prediction confidence on ID data.
This filter predicts each test sample and selectively makes annotations, preparing for the optimization process.
Then, based on whether the annotation is pseudo-ID or pseudo-OOD, we directly use the corresponding objective to regularize models.
However, constant iterations lead to significant ID degradation, which is known as catastrophic forgetting.
Thus, we introduce a category-balanced ID memory bank, which contains one sample in each ID class and is updated with pseudo-ID data.
We design an OOD-triggered update strategy, which simultaneously updates model parameters with pseudo-OOD and the memory bank.
Last but not least, we observe that the previous paradigm makes an inappropriate objective on OOD data, which neglects the differences between OOD samples.
%annotation errors at the beginning of the testing process.
Therefore, we design a novel prediction-aligning objective, which slightly adjusts the OOD objective to fit the model’s intuition via aligning the prediction between the initial and current models.

To verify the efficacy of AUTO, we conduct extensive experiments on common CIFAR-10, CIFAR-100, and ImageNet benchmarks.
Except for naive OOD detection settings, we evaluate AUTO on the aforementioned challenging scenarios where test OOD data consists of multiple OOD data or time-series OOD data.
Natural language processing benchmarks are also included to validate the generality of our framework.
The results empirically demonstrate that our framework is capable of capturing the underlying OOD samples and maintaining expertise on ID tasks in learned latent spaces simultaneously for various OOD detection.
In summary, our work has the following contributions:
\begin{itemize}
        \item We firstly formalize the \emph{test-time out-of-distribution detection} setup.
        To the best of our knowledge, our work is the first to explore flexibly utilizing the test data stream for enhancing OOD detection.
        \item We present a new framework, Adaptive oUTlier Optimization, which adaptively mines distinct ID and OOD samples in the test stream while constantly optimizing with them.
        \item We design a novel prediction-aligning criterion, which calibrates the model's predictions for OOD data, resulting in better ID and OOD performance.
        \item More complex scenarios (\eg multi-OOD, time-series OOD) are considered within TTOD, providing a more comprehensive evaluation of OOD detection methods.
        Extensive experiments demonstrate the superiority of AUTO over other methods.
\end{itemize}




%Specifically, we initialize an in-out-aware filter with the statistics of model prediction confidence on ID data.
%The current test sample will be annotated as pseudo-OOD once its prediction confidence is lower than the OOD threshold in the filter.
%Subsequently, we introduce a dynamic-updated memory bank by category-once sampling 



\section{Related Work}

\subsection{Out-of-Distribution Detection}
OOD Detection has been intensively studied in recent years~\cite{hendrycks2016baseline,liang2018enhancing,liu2020energy,huang2021mos,sun2022knn,du2023dream}.
Existing OOD detection methods typically train an offline supervised model on the in-distribution (ID) data, and then derive the OOD detector based on the learned classifier. 
In general, we can roughly categorize them into three categories according to their requirements for training data.

%\noindent{\textbf{Score designing and enhancing.}
The initial category is the score design method, which doesn't necessitate specific training data. Its objective is to construct a scoring function that relates the input to the OOD score, signifying the degree to which the sample is considered out-of-distribution.
The OpenMax score~\cite{bendale2015towards} is the first method to detect OOD samples using the Extreme Value Theory.
Hendrycks \emph{et al.}~\cite{hendrycks2016baseline} present a baseline using the Maximum Softmax Probability (MSP) but may not be suitable for OOD detection \cite{morteza2022provable}.
Various scoring functions have been proposed to seek the properties that better distinguish OOD samples.
These new functions are calculated mainly from the output of the model.
Specifically, some of them are logit-based scores, such as  Energy score \cite{liu2020energy,wang2021canmulti,lin2021mood}, MaxLogit score \cite{hendrycks2022scaling}, DML score \cite{zhang2023decoupling}, and GEN score \cite{liu2023gen}.
Others are feature-based scores, such as Mahalanobis score \cite{lee2018simple}, GradNorm score \cite{huang2021importance}, and FeatureNorm score \cite{yu2023block}.
Except for general score designing literature, some works aim to enhance the aforementioned scores, such as ODIN score~\cite{liang2018enhancing} and React score~\cite{sun2021react}.
%Liang \emph{et al.} propose the ~\cite{liang2018enhancing} that applies temperature scaling to the MSP score, making the trained discriminators more sensitive to OOD samples.
%Furthermore, additional negative adversarial perturbations are utilized to inputs, making ID and OOD samples more distinguishable.
%Sun \emph{et al.} applies feature clipping on the penultimate layer of neural networks, enhancing method is compatible with MSP score~\cite{hendrycks2016baseline} and energy score~\cite{liu2020energy}.

Except for designing scoring functions, the second category is modifying the logit space with novel loss functions, which only requires ID data during training~\cite{sehwag2020ssd,sun2022knn,morteza2022provable,wei2022mitigating,sun2022dice,ming2023cider,zhang2023decoupling,liu2023gen}.
Sun \emph{et al.}~\cite{sun2022knn} replace the classifier with a k-nearest neighbors predictor, thus eliminating the previous assumption about the distribution of feature space.
Ming \emph{et al.}~\cite{ming2023cider} propose CIDER, which jointly optimizes two losses to promote strong ID-OOD separability: a dispersion loss that promotes large angular distances among different class prototypes and a compactness loss that encourages samples to be close to their class prototypes.
Zhang \emph{et al.}~\cite{zhang2023decoupling} utilize the focal loss and the center loss, retraining networks to enhance the DML score.

Last but not least, outlier exposure methods, which require ID and auxiliary OOD data during training, can significantly enhance OOD detection performance.
The pioneering work, Outlier Exposure (OE) \cite{hendrycks2018deep}, optimizes the predictions of auxiliary OOD samples to a uniform distribution, inspiring a new line of work~\cite{liu2020energy,meinke2019towards,mohseni2020self,chen2021robustifying,ming2022posterior}.
Liu \emph{et al.}~\cite{liu2020energy} start from the energy perspective, optimizing OOD data to a higher-energy range.
Ming \emph{et al.}~\cite{ming2022posterior} propose a posterior sampling-based outlier mining framework, learning a more compact decision boundary than that of naive OE.
However, the main challenge of this paradigm lies in obtaining available auxiliary OOD data, which has inspired explorations into data augmentation~\cite{wang2023outofdistribution} and generation~\cite{du2022vos,tao2023nonparametric,du2023dream}.
For instance, Wang \emph{et al.}~\cite{wang2023outofdistribution} generate more outliers with data-augmentation methods, covering wider OOD situations than that of naive OE.
Except for auxiliary OOD data, some works focus on optimizations with test OOD data~\cite{katzsamuels2022training,zhou2021step,fan2022simple}.
For instance, Katzsamel \emph{et al.}~\cite{katzsamuels2022training} leverage the wild test data with an augmented-lagrangian constrained optimization, which makes the pre-trained model perform accurate OOD detection as well as ID classification.
Benefiting from auxiliary OOD data, this direction has demonstrated encouraging OOD detection performance compared to the aforementioned counterparts.

%Wang \emph{et al.}~\cite{wang2023outofdistribution} generate more outliers with data-augmentation methods, covering wider OOD situations than that of naive OE.
%Tao \emph{et al.}~\cite{tao2023nonparametric} propose sampling edge-ID samples in the low-dimensional feature space and directly employ them as latent-space outliers to regularize the model.


%Fan \emph{et al.}
%Another promising direction toward OOD detection involves the auxiliary outliers for model regularization.
% trains models with auxiliary outliers,  such as CCU \cite{meinke2019towards}, SOFL \cite{mohseni2020self}, Energy \cite{liu2020energy}, ATOM \cite{chen2021robustifying}, and POEM \cite{ming2022posterior}.
%Considering the scarcity of training outliers, recent works synthesize pseudo-OOD samples using ID samples (VOS \cite{du2022vos}, NPOS~\cite{tao2023nonparametric}, and DOE~\cite{wang2023outofdistribution}) or sample test data directly \cite{katzsamuels2022training}. 
%It is noteworthy that WOODS \cite{katzsamuels2022training} has tried to leverage test data to improve detection performance.
%However, WOODS requires access to test data at training time, which is not feasible in some security-sensitive scenarios (\eg federated learning).
%In contrast, AUTO directly utilizes data at test time, which is more general and flexible.

\subsection{Test-Time Optimization}
Except for modifying models during training, recent works explore the possibility of generalizing a pre-trained model to the target scenario during the inference phase \cite{liang2023comprehensive}.
Based on the training/test paradigm, existing works generally fall into three categories: source-free domain adaptation \cite{liang2020we}, test-time training \cite{sun2020test}, and test-time adaptation \cite{wang2020tent,iwasawa2021test}.

Source-free domain adaptation~\cite{liang2020we,nayak2021mining,li2021imbalanced,yang2021generalized} intends to transfer the source pre-trained model to the target domain during testing, utilizing all the test data in an offline manner.
Existing solutions leverage different techniques via pseudo-labeling~\cite{liang2018enhancing}, data generation~\cite{nayak2021mining}, memory bank~\cite{yang2021generalized}, clustering~\cite{li2021imbalanced,liu2022source}, and self-supervision~\cite{liang2021source,kundu2022concurrent}.
With sufficient adaptation to the target domain, these approaches always perform impressively.
However, it may be unavailable to access all the test data at once during testing, limiting the compatibility of the source-free setting.
Notably, some OOD detection works \cite{zhou2021step, fan2022simple} have explored similar source-free settings and achieved limited improvements.

Test-time training~\cite{sun2020test,gandelsman2022test,li2021test} introduces self-supervised auxiliary tasks during the training stage and optimizes them at test time to improve the performance of the source model.
To name a few, Sun \emph{et al.}~\cite{sun2020test} propose the pioneering work, which predicts the rotation angle~\cite{gidaris2018unsupervised} as the auxiliary task.
Gandelsman \emph{et al.}~\cite{gandelsman2022test} employ masked autoencoders~\cite{he2022masked} with vision transformer backbones to perform the self-supervised task.
Osowiechi \emph{et al.}~\cite{osowiechi2023tttflow} utilize unsupervised normalizing flows as an alternative auxiliary task.
Taking into account the inclusion of extra auxiliary tasks during training, these methods necessitate retraining the models, which ultimately renders them impractical for deployment in source-restricted scenarios.

Test-time adaptation endeavors to harness online, unlabeled test data streams to dynamically enhance the performance of the pre-trained model in real-time, which has garnered widespread attention in recent years~\cite{wang2020tent,iwasawa2021test,liu2021ttt++,jang2022test,chi2021test,kundu2020universal,royer2015classifier,niu2023towards,niu2022efficient,song2023ecotta}.
The early studies focus on calibrating the statistics of batch normalization layers~\cite{ioffe2015batch,nado2020evaluating}.
Afterwards, the following solutions pay attention to entropy minimization~\cite{wang2020tent}, pseudo-labeling updating~\cite{goyal2022test,jang2022test}, and output alignment~\cite{wang2022continual,dobler2023robust}.
Besides, some researchers attempt to achieve effective test-time adaption via parameter-free \cite{boudiaf2022parameter} or parameter-efficient \cite{gao2022visual} methods.

In short, prior test-time explorations largely focus on OOD generalization and overlook the OOD detection problem.
In contrast, our TTOD setup first fills the gap in OOD detection and test-time optimization.
It takes a more comprehensive consideration of evolving distribution shifts and online data streams and launches AUTO to tackle them by adaptively outlier annotation and optimization.

\subsection{Catastrophic Forgetting}
Forgetting refers to the loss or deterioration of previously acquired information or knowledge, which can be classified as harmful forgetting or beneficial forgetting.
In this paper, we focus on the harmful part, which has been observed not only in continual learning~\cite{parisi2019continual} but also in various other research areas, \eg domain adaptation~\cite{bobu2018adapting}, meta-learning~\cite{gidaris2018dynamic,finn2019online}, federated learning~\cite{karimireddy2020scaffold}, etc.
Considering that our research is related to test-time adaptation and continual learning, we introduce solutions about these areas, which can be divided into:
(1) Freezing the original parameters and introducing new learnable parameters to adapt the model to test-time data, such as VDP~\cite{gan2023decorate} and EcoTTA~\cite{song2023ecotta}. 
(2) Constraining the updates on crucial parameters to avoid introducing new parameters.
For instance, Tent~\cite{wang2020tent} only updates the batch normalization layers.
CoTTA~\cite{wang2022continual} randomly selects layers to update at each iteration.
EATA~\cite{niu2022efficient} updates parameters while calculating their importance to retain parameters crucial for the source domain.
Inspired by the aforementioned explorations, we intuitively overcome the forgetting issue with a memory bank, which is a common practice.
Furthermore, we design a prediction-aligning objective that provides new insights for addressing OOD detection.
This objective aligns the prediction between the initial and current models, improving ID performance effectively.

\subsection{Parameter-Efficient Fine-Tuning}
As the parameter number grows exponentially to billions \cite{brown2020language} or even trillions \cite{fedus2022switch}, it becomes very inefficient to save the fully fine-tuned parameters \cite{he2021towards} for each downstream task. Many recent research works propose a parameter-efficient \cite{houlsby2019parameter,zaken2022bitfit,he2021towards} way to solve this problem by tuning only a small part of the original parameters.
Parameter-efficient fine-tuning (PEFT) is initially proposed in natural language processing tasks~\cite{houlsby2019parameter} and later applied to computer vision tasks, which can be broadly grouped into addition-based approaches~\cite{he2021towards,houlsby2019parameter,jia2022visual,cai2019once,tu2023visual,sung2022lst,wu2021autoformer} and reparameterization approaches~\cite{caelles2017one,guo2021parameter,hu2021lora,lian2022scaling,yosinski2014transferable,zaken2022bitfit,zhao2020masking}.
PEFT is also a common practice in mounts of applications but has not been explored in OOD detection.
In this paper, we design an OOD-triggered strategy, which largely decreases iterations at test time.
Besides, our work is the first to discuss the impact of optimizing various components of the model parameters on the OOD detection performance.


\section{Test-Time OOD Detection}
In this section, we introduce the background of the OOD detection task (Section \ref{preli}) and provide a clear formulation of the test-time OOD detection setup (Section~\ref{setup_ttod}).
%We first introduce preliminaries for OOD detection in section \ref{preli} and then define the test-time OOD detection setup in section~\ref{setup_ttod}.

\subsection{Preliminaries: Naive OOD Detection}
\label{preli}
OOD detection is often formulated as a binary classification problem to distinguish between ID and OOD data.
We start from a multi-class classification task with input space $\mathcal{X} \subseteq \mathbb{R}^{d} $ and ID label space $\mathcal{Y}_{in}=\left \{ 1,...,C \right \}$, where $d$ and $C$ represent input dimensions and number of ID classes, respectively.
The supervised methods aim to learn the joint data distribution $\mathcal{P}_{\mathcal{X}\mathcal{Y}_{in}}$ from the labeled training set $\mathcal{D}_{tr}^{in} = \{x^{tr}_{i},y^{tr}_{i}\}_{i=1}^{N}$.

%\subsubsection{Naive OOD Detection}
In the naive OOD detection setting, we are given a pre-trained model $f_{\theta_{0}}$ trained on $\mathcal{D}_{tr}^{in}$ will then be deployed in open-world scenarios containing OOD samples from unknown classes $y \notin \mathcal{Y}_{in}$.
Let $\mathcal{P}^{in}$ and $\mathcal{P}^{out}_t$ denote the marginal ID and OOD distribution on $\mathcal{X}$, respectively.
The open-world distribution $\mathcal{P}^{open}_t$ can be denoted with the Huber contamination model \cite{huber1992robust}:
\begin{equation}
    \mathcal{P}^{open}_t = \kappa_t\mathcal{P}^{in} + (1-\kappa_t)\mathcal{P}^{out}_t,
    \label{open_equ}
\end{equation}
where $\kappa_{t} \in [0,1]$ is a fixed value that presents the ID and OOD mixture ratio at $t$ time step.
The test set $\mathcal{D}_{te} = \{x^{te}_{i},y^{te}_{i}\}_{i=1}^{N}$ is sampled from $\mathcal{P}^{open}$, and it is noteworthy that all test samples $x^{te}_{i}$ can be accessed at once at any time step during the testing phase.
The goal of OOD detection is to decide if a test sample $x^{te} \in \mathcal{X}$ is from $\mathcal{P}^{in}$ or $\mathcal{P}^{out}$.
The decision process can be described via a thresholding comparison:
\begin{equation}
    D_{\beta}({x^{te}})=\begin{cases}
 {ID} & S({x^{te}}) \ge \beta \\
  {OOD}& S({x^{te}})<\beta
\end{cases},
\end{equation}
where $S(\cdot)$ is a score function and $\beta$ is the threshold.

\begin{figure}[h]
    \centering
    \includegraphics[width=\linewidth]{pic/ttod_men.png}
    \caption{\textbf{Problem Formulation.} In contrast to naive OOD detection, we consider more practical and challenging scenarios, resulting in more comprehensive evaluations.}
    \label{fig:ttod_setting}
\end{figure}

\subsection{New Setup: Test-Time OOD Detection}
\label{setup_ttod}
As mentioned in the automation driving scenario, the naive OOD detection setting remains some impractical details. In the new test-time OOD detection, we make the following changes to the old setup, which emulate the more practical and challenging scenarios (as shown in Figure~\ref{fig:ttod_setting}):

1) \textbf{Accessing test data in an online manner.} Unlike the old offline setup, our new setup requires test data from $\mathcal{P}^{open}_t$ to arrive sequentially from either ID or OOD. Let $x^{te}_0, x^{te}_1, \cdots  ,x^{te}_N$ denote a stream of online unlabeled test samples, where $x^{te}_t$ is the test sample accessed at $t$ time step.

2) \textbf{Continous evolving data distribution.} Unlike the fixed $\kappa_{t}$, our new setting requires diverse (even dynamic) $\kappa_{t}$ to represent non-stationary environments.
Regarding the choice of $\kappa_{t}$, existing literature~\cite{katzsamuels2022training} argues that when the model is deployed in an open environment, the encountered OOD data will be much more than ID data.
Thus, $\kappa_{t}$ should be smaller.
Meanwhile, another~\cite{fan2022simple} believes that in practical applications, the model will likely be deployed in a more familiar environment where ID data will be much more than OOD data.
Thus, $\kappa_{t}$ should be larger.
Based on the above perspectives, we will comprehensively evaluate the performance of AUTO across a broad range of $\kappa_{t}$.

3) \textbf{More intricate OOD components.} 
Except for diverse $\kappa_{t}$, we consider more complex OOD components.
Let $\mathcal{P}^{OOD1}$ and $\mathcal{P}^{OOD1}$ denote different OOD distributions.% (\eg different label spaces). 
Unlike the old single-OOD scenario $\mathcal{P}^{out}_t \subset \mathcal{P}^{OOD1}$, our new setting considers multi-OOD $\mathcal{P}^{outM}_t$ and time-series OOD scenarios $\mathcal{P}^{outT}_t$:
\begin{equation}
    \mathcal{P}^{outM}_t \subset (\mathcal{P}^{OOD1} \cup \mathcal{P}^{OOD2})
\end{equation}
\begin{equation}
   \mathcal{P}^{outT}_t \subset \begin{cases}
   {\mathcal{P}^{OOD1}}& t \in [0,m] \\
 {\mathcal{P}^{OOD2}} & t \in (m,N]
\end{cases},
\end{equation}
where $m$ is a middle-time step during testing.


\begin{figure*}[ht]
    \centering
    \includegraphics[width=0.95\textwidth]{pic/framework.png}
    \caption{\textbf{Illustration of the Adaptive oUTlier Optimization (AUTO) framework.} The key components include an in-out-aware filter, a dynamic ID memory bank, and a prediction-aligning objective.
    Different color means different operations at test time: Each sample is given the MSP score and judged by the filter.
    Then, according to the judgment, the sample will activate different operations.
    For instance, if it is recognized as a pseudo-ID sample, blue lines are activated:
    this sample will be utilized to replace the sample with the same label in the ID memory bank.}
    \label{fig: framework}
\end{figure*}

\section{Method: Adaptive Outlier Optimization}
\label{auto_method}
In this section, we first introduce the proposed Adaptive oUTlier Optimization (AUTO) framework. As illustrated in Figure~\ref{fig: framework}, AUTO comprises three key components: an in-out-aware filter to tackle the selection of training samples (Section \ref{4.1}), a dynamic-updated ID memory bank, and a prediction-aligning objective to tackle the forgetting issue (Section \ref{4.2}), respectively.
Then, we elaborate on the parameter updating strategy for efficient model optimization (Section \ref{4.3}).
Last but not least, the full framework is provided with the above components, which systematically work as a whole and reciprocate each other (Section \ref{4.4}).

\subsection{Adaptive In-Out-Aware Filter}
\label{4.1}

Considering that ID and OOD samples have different optimization objectives, our intuition to utilize online test data is to annotate samples with pseudo-ID and pseudo-OOD.
Extensive prior works \cite{hendrycks2016baseline,liu2020energy,zhou2021step} have indicated that OOD data and ID data exhibit distinct distributions in feature space. 
Therefore, we design an in-out-aware filter, which is initialized with the statistical information of ID data in the softmax space.
%Such a filter keeps track of the overall positioning of ID data in the feature space.
For each incoming test sample, the filter can estimate the distance between this sample and the ID space, enabling rough annotations to be made.
Specifically, 
given ID examples ${x}_{i}^{in} \sim \mathcal{P}^{in}~,i \in [1,N]$, we compute the MSP \cite{hendrycks2016baseline} score $S_0({x}_{i}^{in})$ of each sample and then estimate the mean $\mu^{in}$ and standard deviation $\sigma^{in}$ of the ID data:
\begin{equation}
    \mu^{in}=\frac{\sum_{i=1}^{N}S_0({x}_{i}^{in})}{N}, \sigma^{in}=\sqrt{\frac{\sum_{i=1}^{N}(S_0({x}_{i}^{in})-\mu^{in})^2}{N}}. \\
\end{equation}
Then, the outlier-aware and inner-aware margins are initialized as follows:
\begin{equation}
m^{in}_0=\mu^{in} + k^{in} \times \sigma^{in},~~~m^{out}_0=\mu^{in} - k^{out} \times \sigma^{in},
\end{equation}
where $k^{in}$ and $k^{out}$ are hyper-parameters. 
\textbf{We can regard a sample with a score higher than $m^{in}$ as a pseudo-ID sample $(\hat{x}_t^{in},\hat{y}_t^{in})$, and a sample with a score lower than $m^{out}$ as a pseudo-OOD sample $\hat{x}_t^{out}$.}

During the continuous updating process, the distribution of test data in the feature space undergoes constant changes. A common phenomenon~\cite{hendrycks2016baseline,liu2020energy,hendrycks2022scaling} is that the MSP scores of all samples are decreasing as we update models with outliers (as shown in Figure \ref{fig:filter}). 
Consequently, we have designed targeted update strategies for $m^{in}$ and $m^{out}$.
On the one hand, we keep $m^{in}$ fixed during training, which ensures that the labeling for pseudo-ID samples is correct.
On the other hand, we update $m^{out}$ with a greedy strategy.
We record the mean of historical OOD score values of the pseudo-OOD samples.
Then, we use the mean value to update $m^{out}$ as follows:
Assuming that we have recorded the mean score of $M$ pseudo-OOD samples when the t-th sample inputs:
\begin{equation}
    m^{out}_{t+1} =\begin{cases}
\frac{M \cdot m^{out}_{t} + S_{t}({x}_t)}{M+1}  & \text{ if } \ S_{t}({x}_t)<m^{out}_t ,\\
m^{out}_{t}  & \text{else.}
\end{cases}
    \label{outlier_margin}
\end{equation}


\begin{figure}[h]
    \centering
    \includegraphics[width=\linewidth]{pic/filter.png}
    \caption{The distribution of MSP statistics is changing during testing, thus we update the OOD-aware margin and keep the ID-aware margin fixed.}
    \label{fig:filter}
\end{figure}

With the above annotations, the loss function for each pseudo-OOD sample $\mathcal{L}^{ood}_t$ can be defined through the cross-entropy between the prediction and target uniform vector:
\begin{equation}
 \mathcal{L}^{ood}_t=-\sum_{i=1}^{C} \frac{1}{C}\mathrm {log} \left (   \frac{\mathrm{exp}(f_{\theta_t}^{(i)}(\hat{x}_t^{out}))}{\sum_{j=1}^{C}\mathrm{exp}(f_{\theta_t}^{(j)}(\hat{x}_t^{out}))}\right ) ,
 \label{ood_loss}
\end{equation}
and the loss function for each pseudo-ID sample $\mathcal{L}^{id}_t$ can be defined with the cross-entropy loss:
\begin{equation}
 \mathcal{L}^{id}_t=-\mathrm {log} \left (   \frac{\mathrm{exp}(f_{\theta_t}^{(\hat{y}_t^{in})}(\hat{x}_t^{in}))}{\sum_{i=1}^{C}\mathrm{exp}(f_{\theta_t}^{(i)}(\hat{x}_t^{in}))}\right ).
 \label{id_loss}
\end{equation}
Leveraging these two loss functions defined above, we proceed to discuss how we can optimize the loss in the test data stream.

%Considering %MSP特征空间的变化,我们动态的更新filter.具体的,我们观察到一个common现象~\cite{hendrycks2018deep,liu2020energy} that the MSP scores of all samples are decreasing as we update models with the outliers (as shown in Figure ).
%因此,一方面,我们keep,因为这样能尽量保证annotate为伪ID的{}是ID样本。另一方面,我们基于贪心策略更新{OOD边界}:。。。。。对于{伪ID},the per-sample classification loss {} is . 对于{伪OOD}, 

%On the other hand, 
%Taking this phenomenon into consideration, 
%A sample ${x}^t$ with a score lower than $m_{out}$ is recognized as a pseudo-OOD sample and rewritten as $\hat{{x}}^t_{ood}$.
%Then, we use $\hat{{x}}^t_{ood}$ to optimize the model as follows:
%\begin{equation}
%    \mathcal{L}_{\textrm{ood}}=\ell_{\mathrm{OE}}({h}^{t}(\hat{{x}}^t_{ood})),
%\end{equation}
%where ${h}^{t}$ represents the latest updated model.

%Meanwhile, We record the mean of historical OOD score values of the pseudo-OOD samples.
%Then, we use the mean value to update $m_{out}$ as follows:
%Assuming that we have recorded the mean score of $M$ pseudo-OOD samples when the t-th sample inputs:



\subsection{Anti-Forgetting Components}

\begin{figure}[t]
    %\centering
    \flushleft
    \setlength{\abovecaptionskip}{10pt}
    \setlength{\belowcaptionskip}{-5pt}
    \subfloat[]{
        \label{fig: memory bank}
        \includegraphics[width=0.49\linewidth]{pic/3_2_1.png}}
    \subfloat[]{
        \label{fig: predicting aligning}
        \includegraphics[width=0.49\linewidth]{pic/3_2_2.png}}
    \caption{\textbf{(a):} Models incur catastrophic forgetting due to constant updating, we mitigate the ID degradation with an ID memory.
    \textbf{(b):} We calibrate the objective of model and enhance ID and OOD performance further.}
    \label{fig:anti}
\end{figure}

To optimize the losses, one may intuitively think directly updates the OOD detector once it encounters $\hat{x}_t^{in}$ or $\hat{x}_t^{out}$.
However, as shown in Figure \ref{fig: memory bank}, we notice that such a simplistic optimization strategy significantly underperforms on the ID task.
Considering the constant parameters updating, we realize that the model encounters the catastrophic forgetting issue, a common phenomenon in online learning.
To address this issue, we upgraded the original alternating update strategy to a simultaneous update strategy.
Specifically, we introduce a dynamic memory bank and a prediction-aligning objective to mitigate ID degradation.
\label{4.2}

\textbf{Dynamic ID memory bank.}
We introduce a dynamic memory bank $\mathcal{M}^{id}$ into the ID classification loss formulated in Eq. \ref{id_loss}.
The memory bank stores one sample per category and is initialized with samples randomly selected from training data.
We update the samples in the memory bank with the test-time ID data in the same category.
Concretely, given a test-time sample $\hat{x}^{in}_t$ whose score is higher than the inner-aware margin $m^{in}$ and its pseudo label $\hat{y}^{in}_t$, we utilize it to update the memory bank as follows:
\begin{equation}
    \hat{x}^{in}_t \to {x}_{\mathcal{M}},\quad \text{if } \ \hat{y}^{in}_t= {y}_{\mathcal{M}}.
\end{equation}

Empirically, we notice that the training with only $\mathcal{M}_{id}$ does not help improve OOD detection.
Therefore, we design an OOD-triggered strategy that modifies the model only when encountering a pseudo-OOD sample $\hat{{x}}^{out}_t$, reducing iterations significantly.
We update the model with $\mathcal{L}^{ood}_t$ and $\mathcal{L}^{id}_t$ simultaneously.
As shown in Figure \ref{fig: memory bank}, our new optimization strategy largely mitigates the ID degradation.

Except for considerations focused on ID performance, we also notice the additional computational and time burden brought by the ID memory.
Thus, we introduce a part-activate strategy while ID memory is large.
For instance, we still update an ID memory that contains 1000 samples when evaluating AUTO on the ImageNet-1k dataset, but we only randomly activate 100 of them when we update models.


\begin{figure}[h]
    \centering
    \includegraphics[width=\linewidth]{pic/pa_ood.png}
    \caption{Calibration on the OOD objective. Based on a uniform vector, we propose to consider background information, leading to a more optimal objective that aligns with the model's intuition.}
    \label{fig:pa_ill}
\end{figure}

\textbf{Prediction-Aligning Objective.}
Except for reminding the OOD detector what it has learned before.
We notice that roughly aligning the predictions of all OOD samples with a uniform vector is inappropriate.
Considering the potential similarity between ID images and OOD images (as shown in Figure \ref{fig:pa_ill}), we believe the optimization objective should have slight adjustments based on the model's intuition.
Furthermore, we find that at the beginning of test-time optimization, $m^{out}_0$ may misclassify some ID samples as OOD.
Such misclassifications subsequently confuse the model during optimization.
To address this problem, we propose aligning the predictions of pseudo-OOD samples between the original model and the updated model.
Specifically, at the beginning of the testing stage, we make a duplicate of the model $f_{\theta_0}$ and freeze its parameters.
The prediction of the duplicated model is denoted as ${y}_0$.
Intuitively, if the results predicted by the model remain consistent with ${y}_0$, the performance on the source task will not degrade.
Let $p_t^{{y}}$ denote the softmax probability that the t-th sample belongs to the class ${y}$.
Our objective is:
\begin{equation}
    \mathcal{L}^{PA}_t=\begin{cases}
0,  & \text{ if } \ y_0=y_t \\
p_t^{y_t} - p_t^{y_0} + \phi, & \text{ if } \ y_0 \neq y_t
\end{cases},
\end{equation}
which enforces $p_t^{y_t}$ close to $p_t^{y_0}$ with a margin $\phi$. That means the prediction of $f_{\theta_t}$ is supposed to be higher than that of $f_{\theta_0}$ at least by $\phi$. 
%In this work, we empirically set $\phi$ as a pre-defined value.
%It is worth noting that $\phi$ should not be too large that it may influence the optimization between the pseudo-OOD sample and the uniform distribution.






\subsection{Parameter-Efficient Update}
\label{4.3}
Different from training-time methods, some real-world applications have requirements on the per-sample process time.
Thus, we attempt to accelerate the optimization of each test sample.
Let $\theta$ denote all the parameters of the model, updating $\theta$ is a natural choice, but it is sub-optimal for test-time OOD detection.
%Therefore, we consider optimizing part of the network parameters.
While part-parameter optimization is common in many tasks, there is still a lack of research on identifying which part should be updated to improve OOD detection performance efficiently.
Following the partial updating principle~\cite{wang2020tent}, we explore the influence of optimizing different combinations of parameters, \eg, the last feature block $\theta_{last}$, all batch normalization layers $\theta_{bn}$, and the classifier $\theta_{fc}$.
Table~\ref{tab:opti} displays the results that optimize the above combinations.
We finally optimize $\theta_{last}$ while keeping the remaining parameters fixed during testing.
Besides, OOD-triggered optimization is also an efficient strategy that has been mentioned in Section \ref{4.2}.

 
\subsection{Overall Objective}
\label{4.4}
Finally, the overall objective of AUTO is shown as:
\begin{equation}\mathcal{L}^{total}_t=\mathcal{L}^{id}_t+\lambda^{out}\mathcal{L}^{ood}_t+\lambda^{PA}\mathcal{L}^{PA}_t,
\label{totalloss}
\end{equation}
where $\lambda^{out}$ and $\lambda^{PA}$ are hyper-parameters.
To adequately leverage the information in the outlier and the memory bank,
we repeat the optimization process on each outlier iteratively for a given number of iterations, $T$.
While seemingly separated from each other, the three components of AUTO are working collaboratively.
First, the in-out-aware filter selects high-quality ID and OOD samples from the unlabeled test data, which facilitates the positive update of models.
Second, the anti-forgetting components help the model enlarge the margin between ID and OOD data, which pays back to the filter and helps it select samples more accurately.
The entire training process converges when the three components perform satisfactorily.

\begin{comment}
    
The pseudo-code is illustrated in Alg. \ref{alg1}.
\begin{algorithm}[H]
    \algsetup{linenosize=\small} %\scriptsize
    \caption{Pseudocode of AUTO in a PyTorch-like style.}
    \label{alg1}
    \begin{algorithmic}
    {\small
    \STATE  {\color{GG}\#M\_id: ID memory bank with one sample each class}
    \STATE  {\color{GG}\#m\_i,m\_o: inner-aware margin and outlier-aware margin for selecting test samples.}
    \STATE {\color{GG}\#f\_t: the updated model when the t-th sample inputs}
    \STATE For x in loader: {\color{GG}\# load one sample each time}
    \INDSTATE sm=softmax(f\_t(x)) {\color{GG}\# softmax: 1xC}
    \INDSTATE msp=max(sm) {\color{GG}\# msp score}
    \INDSTATE if msp$>$m\_i: {\color{GG}\# recognized as an ID sample}
    \INDSTATE[2] pred=argmax(sm) {\color{GG}\# prediction of x}
    \INDSTATE[2] {\color{GG}\# search sample with same label in M\_id}
    \INDSTATE[2] for i in range len (M\_id): 
    \INDSTATE[3]    if pred==y\_i: {\color{GG}\# y\_i: labels in M\_id.}
    \INDSTATE[4]        x\_i=x {\color{GG}\# replace sample}
    \INDSTATE if msp$<$m\_o: {\color{GG}\#recognized as an OOD sample.}
    \INDSTATE[2]{\color{GG}\# training T iterations on each sample}
    \INDSTATE[2] ind=0
    \INDSTATE[2] while ind$<$T:
    %\INDSTATE[3]    {\color{GG}\# the overall loss, Eqn. (11)}
    %\INDSTATE[3]    loss=L\_total(x,M\_id)
    \INDSTATE[3]    sm\_t=softmax(f\_t(x))
    \INDSTATE[3]    {\color{GG}\# CELoss: CrossEntropyLoss}
    \INDSTATE[3]    {\color{GG}\# u:uniform distribution}
    \INDSTATE[3]    loss=CELoss(sm\_t, u) 
    \INDSTATE[3]    {\color{GG}\# optimize with ID memory bank}
    \INDSTATE[3]    for i in range len (M\_id):
%    \INDSTATE[4]        {\color{GG}\# CELoss: CrossEntropyLoss}
    \INDSTATE[4]        loss+=CELoss(f\_t(x\_i),y\_i)
    \INDSTATE[3]    {\color{GG}\# semantic consistency}
    \INDSTATE[3]    sm\_0=softmax(f\_0(x))
    \INDSTATE[3]    pred\_t=argmax(sm\_t)
    \INDSTATE[3]    pred\_0=argmax(sm\_0)
    \INDSTATE[3]    if not (pred\_t == pred\_0):
    \INDSTATE[4]        loss+=sm\_t[pred\_t]-sm\_t[pred\_0]+phi
    \INDSTATE[3]    {\color{GG}\#update model}
    \INDSTATE[3]    loss.backward()
    \INDSTATE[3]    update(f\_t.params)
}    
    \end{algorithmic}
\end{algorithm}
\end{comment}








\section{Experiment Result and Discussion}
In this section, we evaluate AUTO for test-time OOD detection on computer vision (CV) and natural language processing (NLP) benchmarks.
We compare AUTO with previous OOD detection methods, both OOD performance and ID performance (Section \ref{exp_res}).
Besides, we present extensive ablation experiments of different components to understand their contribution toward the performance (Section \ref{abla}).

\subsection{Experimental Setup}
\label{setup}

%~\\
%\noindent \emph{\textsf{Datasets}} 
\noindent \textbf{Datasets.}
Following the common CV benchmarks in OOD detection literatures~\cite{liu2020energy,du2022vos}, we evaluate our method on CIFAR-10/100 \cite{krizhevsky2009learning} and ImageNet-1k \cite{deng2009imagenet}.

For CIFAR benchmarks, we consider six common OOD datasets: \textbf{SVHN} \cite{netzer2011reading}, \textbf{Textures} \cite{cimpoi2014describing}, \textbf{LSUN-Crop} \cite{yu2015lsun}, \textbf{LSUN-Resize} \cite{yu2015lsun}, \textbf{iSUN} \cite{xu2015turkergaze}, and \textbf{Places} \cite{zhou2017places}.
We utilize \textbf{ImageNet-1k}~\cite{deng2009imagenet} as the auxiliary outliers.

For the ImageNet benchmark, we use subsets of four datasets from \textbf{SUN} \cite{xiao2010sun}, \textbf{Textures} \cite{cimpoi2014describing}, \textbf{Place} \cite{zhou2017places}, and \textbf{iNaturalist} \cite{van2018inaturalist}.
We utilize \textbf{ImageNet-22k}~\cite{codreanu2017scale} as the auxiliary outliers.
To make test OOD data and auxiliary outliers disjoint, images in ImageNet-1k are removed.

For NLP benchmarks, we evaluate our method on 20 Newsgroups~\cite{pedregosa2011scikit} and TREC~\cite{li2002experimental}. 
For each ID dataset, we consider five common OOD datasets: \textbf{SNLI} \cite{bowman2015large}, \textbf{IMDB} \cite{maas2011learning}, \textbf{Multi30K} \cite{elliott2016multi30k}, \textbf{WMT16}~\cite{bojar2016findings}, and \textbf{Yelp} \cite{asghar2016yelp}.
We utilize \textbf{Gutenburg}~\cite{csaky2021gutenberg}, \textbf{WikiText-2}~\cite{merity2016pointer}, and \textbf{WikiText-103}~\cite{merity2016pointer} as the auxiliary outliers.

Images in CIFAR and ImageNet benchmarks are resized to $32 \times 32$ and $224 \times 224$, respectively.
To simulate the online teat data stream, the batch size is set to 1.
We provide more details in Table \ref{tab:id_data} and \ref{imagenet:ood_data}.

\begin{table}[h]
    \centering
    \caption{Details of ID datasets.}
    \setlength{\belowcaptionskip}{-10pt}
    \begin{tabular}{c|cccc}
    \toprule
         ID Data& CIFAR10/100 &ImageNet-1k & 20-NG &TREC\\
    \midrule
         Training&50,000&1,281,167&11,293&5,452\\
         Testing&10,000&50,000&7,528&500\\
    \bottomrule
    \end{tabular}
    
    \label{tab:id_data}
\end{table}


\begin{table}[h]
    \centering
    \caption{Details of OOD datasets.}
    \resizebox{\linewidth}{!}{
    \begin{tabular}{c|ccccc}
    \toprule
    CIFAR & SVHN & Textures& Places& iSUN& LSUN\\
    OOD &10,000&5640&10,000&10,000&10,000\\
    \midrule
        ImageNet & iNaturalist & Textures& Places50& SUN\\
        OOD &10,000&5640&10,000&10,000\\
    \midrule
    NLP&SNLI&IMDB&Multi30K&WMT16&Yelp\\
    OOD&9,824&25,000&29,000&22,191&50,000\\
    \bottomrule
    \end{tabular}
    }
    
    \label{imagenet:ood_data}
\end{table}

%\noindent \emph{\textsf{Backbones}}
\noindent \textbf{Backbones.}
For CIFAR benchmarks, we train two backbones from scratch: ResNet-34 \cite{he2016identity} and the Wide ResNet \cite{zagoruyko2016wide} architecture with 40 layers and a widen factor of 2.
The models are trained for 100 epochs. 
The start learning rate is to 0.1 which decays by a factor of 10 at epochs 50 and 80.
Batch size is set to 128 for backbones used in CIFAR benchmarks.
For the ImageNet benchmark, we use a pre-trained ResNet-50 model \cite{he2016identity} from the PyTorch \cite{paszke2019pytorch} and a pre-trained Vision Transformer \cite{dosovitskiy2020image} from the Timm library \cite{rw2019timm}.
For NLP benchmarks, we train a 2-layer GRUs \cite{cho2014learning} model for 5 epochs at training time.

%~\\
%\noindent \emph{\textsf{Implementation details}}
\noindent \textbf{Implementation details.}
For OE-based methods, models are fine-tuned for 10 epochs in CV tasks and for 2 epochs in NLP tasks.
For modifying during testing, we use stochastic gradient descent with the learning rate set to that of the last epoch during training, which is 0.001 in all our experiments.
We set weight decay and momentum to zero during test-time OOD detection, inspired by practice in \cite{liu2018rethinking,he2019rethinking}.
Hyper-parameters are adjusted as backbone changes and shown in Table \ref{tab:hyper}.

\begin{table}[h]
    \centering
    \caption{Hyper-parameter setting of different backbones.}
    \begin{tabular}{l|ccccc}
    \toprule
         Model& $\lambda^{out}$ & $\lambda^{PA}$ & $\phi$& $k^{in}$& $k^{out}$\\
    \midrule
         ResNet-34& 0.25&0.2&0.05&0&3\\
         WRN-40-2&0.25&0.1&0.05&0&3\\
         ResNet-50&0.25&0.1&0.005&0&3\\
         ViT-B-16&0.25&0.1&0.005&0&1.5\\
         \bottomrule
    \end{tabular}
    
    \label{tab:hyper}
\end{table}


%These values are selected via a cross-validation strategy which is shown in \textbf{Appendix}.

%~\\
%\noindent \emph{\textsf{Baselines}}
\noindent \textbf{Baselines.}
The compared algorithms include: (1) Methods train model without auxiliary outliers: MSP \cite{hendrycks2016baseline}, ODIN \cite{liang2018enhancing}, Mahalanobis \cite{lee2018simple}, Energy \cite{liu2020energy}, GradNorm \cite{huang2021importance}, MaxLogit~\cite{hendrycks2022scaling}, ViM~\cite{wang2022vim}, LogitNorm \cite{wei2022mitigating}, ReAct \cite{sun2021react}, DICE \cite{sun2022dice}, KNN \cite{sun2022knn}, and DML+~\cite{zhang2023decoupling}.
(2) Methods train model with auxiliary outliers: OE~\cite{hendrycks2018deep}, Energy~\cite{liu2020energy}, VOS \cite{du2022vos}, POEM~\cite{ming2022posterior}, WOODS~\cite{katzsamuels2022training}, NPOS \cite{tao2023nonparametric}, and DOE~\cite{wang2023outofdistribution}.
The reported results in this paper are obtained by reproducing the source code provided by the aforementioned methods.
%~\\
%\noindent \emph{\textsf{Evaluation metrics}}

\noindent \textbf{Evaluation metrics.}
We evaluate our framework and baseline methods using the following metrics: 1) The false positive rate of OOD samples when the true positive rate of in-distribution samples is at 95\% (\textbf{FPR95}); 2) The area under the receiver operating characteristic curve (\textbf{AUROC}); and 3) The ID classification accuracy (\textbf{ID\_Acc}).


\subsection{Performance Analysis}
\label{exp_res}

\setlength{\tabcolsep}{2pt}
\begin{table*}[ht]
    \caption{Comparison with competitive OOD detection methods on CIFAR benchmarks. $\uparrow$ indicates larger values are better and vice versa.
    All values are percentages averaged over six OOD test datasets described in Section \ref{setup}.
    Bold numbers indicate superior results.
    $\mathcal{D}^{aux}$ indicates whether the detector is modified with an outlier dataset during training.}
    \resizebox{\textwidth}{!}{
    \centering
    
    \begin{tabular}{lcrrrrrrrrrrrr}
    \toprule
    \makebox[0.1\textwidth][l]{\multirow{3}{*}{\textbf{Methods}}} & \multirow{3}{*}{$\mathcal{D}^{aux}$}&
    \multicolumn{6}{c}{\textbf{CIFAR-10}} & \multicolumn{6}{c}{\textbf{CIFAR-100}}\\
    %\cmidrule(lr){2-7}\cmidrule(lr){8-13}
    & &\multicolumn{3}{c}{\textbf{ResNet-34}}&\multicolumn{3}{c}{\textbf{WideResNet-40-2}}&\multicolumn{3}{c}{\textbf{ResNet-34}}&\multicolumn{3}{c}{\textbf{WideResNet-40-2}}\\
    \cmidrule(lr){3-5}\cmidrule(lr){6-8}\cmidrule(lr){9-11}\cmidrule(lr){12-14}
    & &\textbf{FPR95}$\downarrow$&\textbf{AUROC}$\uparrow$&\textbf{ID\_Acc}$\uparrow$&\textbf{FPR95}$\downarrow$&\textbf{AUROC}$\uparrow$&\textbf{ID\_Acc}$\uparrow$&\textbf{FPR95}$\downarrow$&\textbf{AUROC}$\uparrow$&\textbf{ID\_Acc}$\uparrow$&\textbf{FPR95}$\downarrow$&\textbf{AUROC}$\uparrow$&\textbf{ID\_Acc}$\uparrow$\\
    \midrule
    MSP \cite{hendrycks2016baseline}&\XSolidBrush&46.49&92.53&94.87 &52.00&90.57&94.53 &83.53&74.34&77.51 &79.15&76.44&75.84\\
    ODIN\cite{liang2018enhancing}   &\XSolidBrush&30.00&93.94&94.87 &34.32&91.38&94.53 &82.76&75.27&77.51 &69.75&81.29&75.84\\
    Mahalanobis \cite{lee2018simple}&\XSolidBrush&44.31&93.31&94.87 &25.61&95.19&94.53 &75.56&80.82&77.51 &71.14&79.71&75.84\\
    Energy \cite{liu2020energy}     &\XSolidBrush&28.77&94.07&94.87 &33.41&91.53&94.53 &82.65&75.33&77.51 &69.65&81.30&75.84\\
    GradNorm \cite{huang2021importance}                &\XSolidBrush&66.20&83.21&94.87 &71.83&61.91&94.53 &76.45&71.46&77.51& 85.28&56.14&75.84\\
    MaxLogit \cite{hendrycks2022scaling}               &\XSolidBrush&28.05&94.02&94.87 &35.01&91.06&94.53 &80.67&77.39&77.51& 73.29&80.25&75.84\\
    ViM \cite{wang2022vim}                             &\XSolidBrush&26.21&94.73&94.87 &23.02&94.98&94.53 &66.22&82.36&77.51& 66.24&82.51&75.84\\
    ReAct \cite{sun2021react}                          &\XSolidBrush&32.57&93.16&94.85 &58.67&82.85&93.41 &74.76&82.01&77.09 &92.01&64.53&64.77\\
    Logit Norm \cite{wei2022mitigating} 
    &\XSolidBrush&18.14&96.61&94.68 &21.03&95.86&94.42 &76.08&76.83&76.40 &54.90&87.60&\textbf{76.02}\\
    KNN \cite{sun2022knn}&\XSolidBrush&36.71&94.15&94.87&36.63&93.31&94.53&71.33&82.44&77.51&59.92&84.36&75.84\\
    DML+\cite{zhang2023decoupling}  &\XSolidBrush&23.66&95.34&94.87 &10.08&98.00   &94.53   &58.02&87.53&77.51&39.20&91.21&75.84\\
    \midrule
    OE~\cite{hendrycks2018deep}&\Checkmark&6.02&98.58&95.08&7.08&98.51&94.44&58.52&87.30&76.84&54.04&85.82&75.59\\
    Energy ~\cite{liu2020energy}&\Checkmark&\textbf{2.93}&98.71&\textbf{95.49}&\textbf{2.91}&\textbf{98.97}&\textbf{94.91}&53.02&90.05&77.19&44.43&90.47&75.75\\
    POEM~\cite{ming2022posterior}&\Checkmark&11.25&97.62&89.57 &7.17&98.37&90.62 &19.78&95.94&69.49&24.30&95.96&69.37\\
    WOODS~\cite{katzsamuels2022training}&\Checkmark&10.10&97.75&94.79&12.14&97.58&94.72&34.90&91.21&77.84&22.65&94.54&75.74\\
    DOE~\cite{wang2023outofdistribution}&\Checkmark&8.93&97.84&94.74&5.00&98.75&94.43&32.43&93.65&76.95&26.09&94.43&74.98\\
    \midrule
    AUTO&\XSolidBrush&6.22&\textbf{98.72}&94.92&9.45&97.94&94.33&\textbf{11.06}&\textbf{97.50}&\textbf{77.91} &\textbf{16.45}&\textbf{95.97}&74.77\\
    \bottomrule
    \end{tabular}
    }
    \setlength{\belowcaptionskip}{-10pt}
    
    \label{tab:cifar}
\end{table*}



Extensive experiments and results are presented here.
Firstly, for fairness, the ratios of in-distribution (ID) to out-of-distribution (OOD) data in Tables \ref{tab:cifar} and \ref{imagenet} follow the same proportions as set in the naive OOD detection, mitigating the adverse impact of complex scenarios in test-time OOD detection on the performance of previous methods.
Then, major solutions are evaluated on new test scenarios (multi-OOD in Table \ref{tab:mix_data} and time-series OOD in Figure \ref{fig:tso}), which further present the superiority of AUTO.
Last but not least, evaluations on NLP benchmarks exhibit the generality and compatibility of AUTO.


 
\textbf{AUTO significantly outperforms counterparts that train models without auxiliary outliers.} 
Compared with methods that optimize models with only ID data, AUTO inherently exhibits superior OOD detection performance, which is attributed to modifications with real-OOD samples at test time.
Meanwhile, AUTO is training-free, maintaining the advantage of not requiring auxiliary outliers during training.
Part of previous explorations require a mount of ID data for retraining to overcome overconfidence.
In contrast, AUTO has no requirements, significantly alleviating the dependence of pretrained models on source data. 
These characteristics make AUTO more suitable for deployments.
%A summary of the CIFAR results is provided in Table \ref{tab:cifar}.
%For the sake of fairness, the ratios of ID to OOD data in Tables \ref{tab:cifar} and \ref{imagenet} follow the same proportions as set in the naive OOD detection.
%A summary of the results is provided in Table 2.  we report the results separately according In general,

\setlength{\tabcolsep}{3.0pt}
\begin{table*}[t]
    \centering
    \caption{Comparison with competitive OOD detection methods on the ImageNet benchmark.}
    \resizebox{\textwidth}{!}{
    \begin{tabular}{lcrrrrrrrrrrr}
    \toprule
    \multicolumn{1}{c}{\multirow{3}{*}{\textbf{Methods}}} & \multirow{3}{*}{$\mathcal{D}^{aux}$}& \multicolumn{8}{c}{\textbf{OOD Datasets}} & \multicolumn{2}{c}{\multirow{2}{*}{\textbf{Average}}}&\multirow{3}{*}{\textbf{ID\_Acc}$\uparrow$}\\ 
    \cmidrule(lr){3-10}
    \multicolumn{2}{c}{ } & \multicolumn{2}{c}{\textbf{SUN}}   & \multicolumn{2}{c}{\textbf{Textures}}  & \multicolumn{2}{c}{\textbf{iNaturalist}} & \multicolumn{2}{c}{\textbf{Places}}  \\ 
    \cmidrule(lr){3-4}\cmidrule(lr){5-6}\cmidrule(lr){7-8}\cmidrule(lr){9-10}\cmidrule(lr){11-12}
    \multicolumn{2}{c}{} &\multicolumn{1}{c}{FPR95$\downarrow$} &AUROC$\uparrow$  & FPR95$\downarrow$ &AUROC$\uparrow$  & FPR95$\downarrow$ &AUROC$\uparrow$  & FPR95$\downarrow$ & \multicolumn{1}{c}{AUROC$\uparrow$}  & \multicolumn{1}{c}{FPR95$\downarrow$ }&AUROC$\uparrow$ \\ 
    \midrule
    \multicolumn{13}{c}{\textbf{Backbone: ResNet-50}}\\
    MSP\cite{hendrycks2016baseline}&\XSolidBrush& 68.53 & 81.75 & 66.15 & 80.46 & 52.69 & 88.42 & 71.59 & 80.63 & 64.74 & 82.82 & \textbf{76.12} \\ 
    ODIN\cite{liang2018enhancing}&\XSolidBrush& 54.04 & 86.89 & 45.50 & 87.57 & 41.50 & 91.38 & 62.12 & 84.45 & 50.79 & 87.57 & \textbf{76.12} \\ 
    G-ODIN\cite{hsu2020generalized}&\XSolidBrush&60.83&85.60&77.85&73.27&61.91&85.40&63.70&83.81&66.07&82.02&\textbf{76.12}\\
    Mahalanobis\cite{lee2018simple}&\XSolidBrush&98.35&42.10&54.78&85.02&96.95&52.60&98.47&42.01&87.14&55.43&76.12\\
    Energy\cite{liu2020energy}&\XSolidBrush& 58.25 & 86.73 & 52.30 & 86.73 & 53.94 & 90.60 & 65.40 & 84.12 & 57.47 & 87.05 & \textbf{76.12} \\
    MaxLogit\cite{hendrycks2018deep}&\XSolidBrush& 60.42&86.44&66.05&84.03&50.82&91.15&54.95&86.39&58.06&87.00&\textbf{76.12}\\
    GradNorm\cite{huang2021importance}&\XSolidBrush&38.53&88.87&46.76&83.66&31.24&91.79&46.29&86.28&40.71&87.65&\textbf{76.12}\\
    ViM\cite{wang2022vim}&\XSolidBrush&91.87&72.65&12.40&97.52&67.95&88.40&91.09&71.47&65.83&82.51&\textbf{76.12}\\
    ReAct\cite{sun2021react}&\XSolidBrush& 23.69 & 94.44 & 46.33 & 90.30 & 19.71 & 96.37 & 33.30 & 91.96 & 30.76 & 93.27 & 74.82 \\ 
    DICE+ReAct\cite{sun2022dice}&\XSolidBrush&  26.49 & 93.83 & 29.36 & 92.65 & 20.07 & 96.11 & 38.35 & 90.61 & 28.57 & 93.30 & 67.01 \\ 
    KNN\cite{sun2022knn}&\XSolidBrush&70.50&80.46&11.26&97.41&60.30&86.09&78.81&74.66&55.22&84.66&\textbf{76.12}\\
    DML+\cite{zhang2023decoupling}&\XSolidBrush&30.73&93.98&36.35&89.02&13.66&97.48&39.82&91.22&30.14&92.93&\textbf{76.12}\\
    %\cmidrule(lr){2-14}
    OE\cite{hendrycks2018deep}&\Checkmark& 80.10&76.55&66.38&82.04&78.31&75.23&70.41&81.78&73.80&78.90&75.51\\
    MixOE\cite{zhang2023mixture}&\Checkmark&74.62&79.81&58.00&85.83&80.51&74.30&84.33&69.20&74.36&77.28&74.62\\
    VOS\cite{du2022vos}&\Checkmark&98.72&38.50&70.20&83.62&94.83&57.69&87.75&65.65&87.87&61.36&74.43\\
    DOE\cite{wang2023outofdistribution}&\Checkmark&80.94&76.26&34.67&88.90&55.87&85.98&67.84&83.05&59.83&83.54&75.50\\
    %\cmidrule(lr){2-14}
    AUTO&\XSolidBrush &\textbf{8.26} & \textbf{97.34} & \textbf{11.21} & \textbf{97.68} & \textbf{2.00} & \textbf{99.39} & \textbf{18.35} & \textbf{94.98} & \textbf{9.96} & \textbf{97.44} & 74.64 \\
    \midrule
    \multicolumn{13}{c}{\textbf{Backbone: ViT-Base-16}}\\
    MSP\cite{hendrycks2016baseline}&\XSolidBrush  & 73.80 & 79.49 & 63.07 & 81.50 & 39.40 & 92.41 & 74.09 & 79.56 & 62.59 & 83.24 & 78.01 \\ 
    ODIN\cite{liang2018enhancing}&\XSolidBrush  & 62.81 & 83.20 & 51.45 & 86.31 & 30.28 & 92.65 & 66.21 & 81.51 & 52.69 & 85.92 & 78.01 \\ 
    Mahalanobis\cite{lee2018simple}&\XSolidBrush&79.88&81.82&72.10&80.33&18.22&95.37&84.05&73.70&63.57&82.81&78.01\\
    Energy\cite{liu2020energy}&\XSolidBrush  & 69.29 & 84.52 & 51.97 & 88.30 & 37.84 & 94.46 & 72.03 & 82.74 & 57.78 & 87.51 & 78.01 \\ 
ReAct\cite{sun2021react}&\XSolidBrush    & 72.19 & 84.12&  53.17 & 88.12& 29.54 & 95.19 & 74.15 & 82.22 & 57.26 & 87.41 & 78.01 \\
    KNN \cite{sun2022knn}&\XSolidBrush& 51.01&89.46&41.12&90.55&7.32&98.50&54.08&88.31&38.38&91.71&78.01\\
    MaxLogit\cite{hendrycks2018deep}&\XSolidBrush&69.99&84.25&54.10&87.75&32.69&94.79& 71.42&82.79&57.05&87.40&78.01\\
    %\cmidrule(lr){2-14}
    VOS\cite{du2022vos}&\Checkmark&43.03&91.92&56.67&87.64&31.65&94.53&41.62&90.23&43.24&90.86&\textbf{79.64}\\
    NPOS\cite{tao2023nonparametric}&\Checkmark&28.96&94.63&57.39&85.91&27.63&94.75&35.45&91.63&37.36&91.73&79.55\\
    %\cmidrule(lr){2-14}
    AUTO&\XSolidBrush &\textbf{9.12} & \textbf{97.22} & \textbf{19.91} & \textbf{95.42} & \textbf{0.77} & \textbf{99.80} & \textbf{19.22} & \textbf{95.19} & \textbf{12.26} & \textbf{96.91} & 79.38\\
    \bottomrule
    \end{tabular}
    }
    \setlength{\belowcaptionskip}{-10pt}
   
    \label{imagenet}
\end{table*}



\textbf{AUTO performs more effectively while encountering larger ID space.}
Compared with methods that optimize models with auxiliary outliers, we notice that AUTO is not the best in evaluations on CIFAR-10 benchmarks.
However, as the complexity of the in-distribution (ID) space increases (from CIFAR-10 to CIFAR-100 and then to ImageNet-1k), the superiority of AUTO gradually becomes evident and establishes itself as the state-of-the-art method. In particular, the lead of AUTO over previous SOTA methods continues to expand (from 1.56\% (CIFAR-100) to 3.90\% (ImageNet-1k)).
The above phenomena suggest that the traditional OE paradigm can form compact decision boundaries in simple ID spaces to handle OOD data.
However, when the ID space becomes complex, previous paradigms struggle to maintain such compact decision boundaries, leading to significant OOD detection degradation.
On the contrary, AUTO adaptively adjusts decision boundaries specifically for the target out-of-distribution (OOD) data in deployment environments, leading to more efficient and better-performing OOD detection.

\textbf{AUTO effectively maintains the ID performance at test time.}
While significantly enhancing OOD detection performance, AUTO also mitigates the ID gradation effectively.
The largest gap in the three benchmarks is 1.48\% on ImageNet-1k with ResNet-50, which is acceptable.
AUTO does not impact the model's handling of the source ID task, which is practical for deploying real-world models.




\begin{figure}
    \centering
    \includegraphics[width=0.95\linewidth]{pic/diff_ratio.png}
   \caption{Effect of mixture ratio $\kappa_{t}$ on ID classification (orange) and OOD detection (blue).}
    \label{fig:dif-rat}
\end{figure}





\textbf{AUTO effectively handle OOD detection under varying ID to OOD data ratios.}
Going beyond the naive OOD detection setup, test-time OOD detection introduces a new setting where the mixed ratio $\kappa_{t}$ of ID to OOD data varies, testing the generality of the solution.
The performance of the AUTO method under different $\kappa_{t}$ conditions is shown in Figure~\ref{fig:dif-rat}. 
As a result, AUTO effectively maintains ID performances on different $\kappa_{t}$.
However, AUTO underperforms when there is very little OOD data.
We record the number of annotations for pseudo-OOD data and find that in scenarios with a higher proportion of OOD data, AUTO relies on continuous accurate annotation to achieve better OOD detection performance.
When there is less OOD data, accurate annotations significantly decrease, affecting the extent to which model performance is improved.



\textbf{AUTO achieves significant enhancement in complex-OOD scenarios.}
Except for various $\kappa_{t}$ scenarios, test-time OOD detection also involves complex OOD components at test time.
We perform experiments on multi-OOD scenarios and time-series scenarios, and the results are presented in Table \ref{tab:mix_data} and Figure \ref{fig:tso}, respectively.
In a word, models' performances in these new scenarios differ from an arithmetic average of performances in single-OOD scenarios.
The intricate composition of data presents challenges for all methods.
Nevertheless, AUTO continues to demonstrate exceptional performance, exhibiting a greater performance advantage over OE and WOODS.
This underscores AUTO's superior capability to handle mixed OOD scenarios.

\begin{table}[ht]
    \centering   
    \caption{Results on the mixed OOD scenarios. Models are ResNet-34 and ResNet-50, respectively.}
    \resizebox{\linewidth}{!}{
    
    \begin{tabular}{clrrr}
    \toprule
        \textbf{Data}& \textbf{Methods}&\textbf{FPR95 $\downarrow$} &\textbf{AUROC $\uparrow$}&\textbf{ID$\_$Acc $\uparrow$}\\
    \midrule
        CIFAR-100&MSP~\cite{hendrycks2016baseline}&82.69&75.07&77.51\\
        + &OE~\cite{hendrycks2018deep}&74.35&80.19&76.37\\
      Places365  &WOODS~\cite{katzsamuels2022training}&73.32&80.35&77.10\\
       SVHN &AUTO&\textbf{36.27}&\textbf{90.11}&\textbf{77.73}\\
    \midrule
    ImageNet&Energy~\cite{liu2020energy}&61.85&85.43&\textbf{76.12}\\
    +&DICE~\cite{sun2022dice}&32.42&92.22&\textbf{76.12}\\
    Places&DOE~\cite{wang2023outofdistribution}&63.34&81.98&75.02\\
    SUN&AUTO&\textbf{13.72}&\textbf{96.62}&73.08\\
    \bottomrule
    \end{tabular}
    }
    \setlength{\belowcaptionskip}{-12pt}
    \label{tab:mix_data}
\end{table}

\begin{figure}[h]
    \centering
    \includegraphics[width=\linewidth]{pic/time_series_ood.png}
    %\setlength{\belowcaptionskip}{-10pt}
    \caption{OOD performance on time-series OOD scenarios, AUROC is reported.
    ResNet-34 is trained on CIFAR-100.}
    \label{fig:tso}
\end{figure}


\setlength{\tabcolsep}{1.0pt}
\begin{table*}[ht]
    \centering
    \caption{Comparison with competitive OOD detection methods in the NLP benchmarks.}
    \resizebox{\textwidth}{!}{

    \begin{tabular}{lcrrrrrrrrrrrrr}
    \toprule
    \multicolumn{1}{l}{\multirow{3}{*}{\textbf{Methods}}} &\multicolumn{1}{c}{\multirow{3}{*}{\textbf{$\mathcal{D}^{aux}$}}} &  \multicolumn{10}{c}{\textbf{OOD Datasets}} & \multicolumn{2}{c}{\multirow{2}{*}{\textbf{Average}}}&\multirow{3}{*}{\textbf{ID\_Acc}$\uparrow$}\\ 
    \cmidrule(lr){3-12}
    \multicolumn{2}{c}{ }   & \multicolumn{2}{c}{\textbf{SNLI}}   & \multicolumn{2}{c}{\textbf{IMDB}}  & \multicolumn{2}{c}{\textbf{Multi30K}} & \multicolumn{2}{c}{\textbf{WMT16}} & \multicolumn{2}{c}{\textbf{Yelp}} \\ 
    \cmidrule(lr){3-4}\cmidrule(lr){5-6}\cmidrule(lr){7-8}\cmidrule(lr){9-10}\cmidrule(lr){11-12}\cmidrule(lr){13-14}
    \multicolumn{2}{c}{}&  \multicolumn{1}{c}{FPR95$\downarrow$} &AUROC$\uparrow$  & FPR95$\downarrow$ &AUROC$\uparrow$  & FPR95$\downarrow$ &AUROC$\uparrow$  & FPR95$\downarrow$ & \multicolumn{1}{c}{AUROC$\uparrow$}  & \multicolumn{1}{c}{FPR95$\downarrow$ }&AUROC$\uparrow$ & \multicolumn{1}{c}{FPR95$\downarrow$ }&AUROC$\uparrow$ \\ 
    \midrule
    \multicolumn{15}{c}{\textbf{ID Data: 20 Newsgroups}}\\
   
    MSP\cite{hendrycks2016baseline}&\XSolidBrush& 37.75&86.32&62.66&80.47&58.79&78.43&45.43&85.40&65.65&78.72&54.06&81.87&73.25\\
    AUTO&\XSolidBrush&36.37&86.61&58.34&81.80&45.93&82.37&33.08&89.64&32.69&89.46&41.28&85.98&\textbf{73.44}\\
    OE (Guten)\cite{hendrycks2018deep}&\Checkmark&4.21&98.22&11.60&96.05&3.92&98.23&2.86&98.79&14.00&94.93&7.32&97.24&72.83\\
    OE (Wiki-103)\cite{hendrycks2018deep}&\Checkmark&2.29&98.85&2.91&98.66&2.90&99.66&0.34&99.73&76.51&75.45&16.59&94.47&72.34\\
    OE (Wiki-2)\cite{hendrycks2018deep}&\Checkmark&3.42&98.57&4.73&98.32&1.39&99.53&0.50&99.76&83.06&74.38&18.62&94.11&72.42\\
    OE+AUTO&\Checkmark&\textbf{2.36}&\textbf{98.89}&\textbf{2.91}&\textbf{98.70}&\textbf{0.86}&\textbf{99.65}&\textbf{0.33}&\textbf{99.76}&\textbf{3.43}&\textbf{98.48}&\textbf{1.98}&\textbf{99.10}&72.21\\
    \midrule
    \multicolumn{15}{c}{\textbf{ID Data: TREC}}\\
    
    MSP\cite{hendrycks2016baseline}&\XSolidBrush& 25.39&93.13&77.15&75.86&66.80&81.75&50.59&84.11&76.95&74.20&59.38&81.81&76.80\\
    AUTO&\XSolidBrush&17.58&94.96&47.85&88.29&2.27&93.01&42.58&87.31&24.02&93.07&30.86&91.33&\textbf{76.97}\\
    OE (Wiki-103)\cite{hendrycks2018deep}&\Checkmark& 19.73&93.20&0.98&99.45&3.32&99.75&10.35&96.79&0.39&99.89&6.95&97.69&63.20\\
    OE (Wiki-2)\cite{hendrycks2018deep}&\Checkmark&8.20&97.21&2.54&99.17&0.39&99.75&0.9 &99.52&\textbf{0.00}&\textbf{100.00}&2.42&99.07&70.60\\
    OE+AUTO& \Checkmark& \textbf{2.58}&\textbf{98.84}&\textbf{0.97}&\textbf{99.46}&\textbf{0.32}&\textbf{99.76} &\textbf{0.87}&\textbf{99.53}&\textbf{0.00}&\textbf{100.00}&\textbf{0.97}&\textbf{99.52}&76.72\\
\bottomrule
    \end{tabular}
    }
    \setlength{\belowcaptionskip}{-10pt}   
    \label{nlp_task}
\end{table*}

\textbf{AUTO possesses outstanding generality and compatibility.}
After assessing AUTO's performance on CV benchmarks, we conducted additional tests on NLP benchmarks, and the results are presented in Table \ref{nlp_task}.
AUTO consistently demonstrates outstanding performance in NLP evaluations, showcasing robust generalization capabilities across diverse modalities.
Furthermore, AUTO improves the performance of models, encompassing not only those trained in an ID manner but also those trained in an OE manner.
Extensive results suggest that AUTO does not conflict with previous out-of-distribution (OOD) detection methods; instead, it serves as a complementary strategy to enhance OOD detection performance when testing models strengthened during training.




\subsection{Component Analysis}
\label{abla}

\textbf{Impact of the core learning objectives.}
We evaluate the impact of different objectives, as presented in Table~\ref{obj_com}.
Our results demonstrate that training the model solely with an ID memory bank leads to similar performance as the method without optimization, indicating that optimizing on ID data alone does not effectively enhance OOD detection. Furthermore, while training on outliers alone improves OOD detection, it results in catastrophic forgetting, as evidenced by the decline in ID classification accuracy.
With the help of the ID memory bank, the model jointly updated by both ID and OOD samples already exhibits progress in both OOD detection and ID classification.
Besides, our prediction-aligning objective enhances both ID and OOD performance further.
\begin{table}[ht]
    \centering
    \caption{Ablation study on different combinations of objectives. Model is trained on CIFAR-100 with ResNet-34.}
    \begin{tabular}{cccrrr}
    \toprule
    $\mathcal{L}^{\textrm{id}}$&$\mathcal{L}^{\textrm{ood}}$&
    \multicolumn{1}{c}{$\mathcal{L}^{\textrm{PA}}$} &\textbf{FPR95 $\downarrow$} &\textbf{AUROC $\uparrow$}&\textbf{ID\_Acc $\uparrow$}  \\
    \midrule
    $\checkmark$&~&~&79.89&76.34&77.50 \\
    &$\checkmark$&&60.26&79.36&65.92\\
    $\checkmark$&$\checkmark$&&12.45&97.37&77.56\\
    $\checkmark$&$\checkmark$&$\checkmark$&\textbf{11.06}&\textbf{97.50}&\textbf{77.91}\\
    \bottomrule
    \end{tabular}    
    \label{obj_com}
\end{table}

\textbf{Impact of different optimization parameters.}
Results presented in Table~\ref{tab:opti} evaluate the efficacy of various optimization parameters.
On the one hand, taking both ID and OOD tasks into account, the performance of optimizing the last parameter block is superior.
On the other hand, models in open-world scenarios, particularly those engaged in online stream applications, need to notice the optimization efficiency.
We note that the inference time per sample is approximately 5ms.
AUTO necessitates only a modest 3.2x increase in processing time, which is tolerable.
Thus, we conclude that the utilization of the last parameter block as the optimization objective is a more efficient strategy.

\begin{table}[ht]
    \centering
    \caption{Ablation study on different modulation parameters. Model is trained on CIFAR-100 with ResNet-34.}
    %\resizebox{\linewidth}{!}{
    \begin{tabular}{lrrrr}
        \toprule
        \multicolumn{1}{c}{\textbf{Modu. Para.}}& \textbf{FPR95} $\downarrow$&\textbf{AUROC} $\uparrow$&\textbf{ID\_Acc} $\uparrow$&\textbf{Time} \\
        \midrule
        %Energy(w. $\matchcal{D}_{aux}$) & 8.3min\\
        No Para.& 83.53&74.34&77.51& 1x \\
        Block 1 & 77.50&78.36&77.64&1.8x\\
        Block 2 & 40.10&88.70&73.32&2.3x\\
        Block 3 & 17.72&95.48&72.40&2.9x\\
        Block 4 & \textbf{11.06}&\textbf{97.50}&\textbf{77.91}&3.2x\\
        BN      & 14.33&96.58&77.16&3.1x\\
        FC      & 77.92&78.76&78.59&1.5x\\
        All Para.& 13.26&97.07&76.95& 11.2x\\
        \bottomrule
    \end{tabular}
    %}
    \setlength{\belowcaptionskip}{-10pt}
    
    \label{tab:opti}
\end{table}

\textbf{Impact of different OOD scoring functions.}
We evaluate AUTO on different scoring functions, and results are shown in Table \ref{tab:ood scores}.
AUTO performs well in the logits space, energy space, and softmax space, with minimal differences in OOD and ID performance across the three spaces. This empirical observation indicates the feasibility and generality of AUTO and the test-time OOD detection paradigm.

\begin{table}[ht]
    \centering
    \caption{Ablation study on different OOD scoring functions. Model is trained on CIFAR-100 with ResNet-34.}
    %\resizebox{\linewidth}{!}{
    \begin{tabular}{lrrr}
    \toprule
    OOD score &\textbf{FPR95 $\downarrow$} &\textbf{AUROC $\uparrow$}&\textbf{ID\_Acc $\uparrow$}  \\
    \midrule
    MSP~\cite{hendrycks2016baseline}&11.40&97.41&77.69 \\
    Energy~\cite{liu2020energy}&11.58&97.49&77.76\\
    MaxLogit~\cite{hendrycks2022scaling}&\textbf{11.06}&\textbf{97.50}&\textbf{77.91}\\
    \bottomrule
    \end{tabular}
    %}
    %
    
    \label{tab:ood scores}
\end{table}

\begin{figure*}[t]
    %\centering
    \flushleft
    \setlength{\abovecaptionskip}{10pt}
    \setlength{\belowcaptionskip}{-5pt}
    \subfloat[FPR95 on different $\lambda^{out}$]{
        \label{fig:subfig:1a}
        \includegraphics[width=0.245\textwidth]{pic/abla_11.png}}
    \subfloat[FPR95 on different $k^{out}$]{
        \label{fig:subfig:1d}
        \includegraphics[width=0.245\textwidth]{pic/abla_41.png}}
    \subfloat[FPR95 on different $\lambda^{PA}$]{
        \label{fig:subfig:1b}
        \includegraphics[width=0.245\textwidth]{pic/abla_21.png}}
    \subfloat[FPR95 on different $\phi$]{
        \label{fig:subfig:1c}
        \includegraphics[width=0.245\textwidth]{pic/abla_31.png}}
    \\
    \subfloat[ID\_Acc on different $\lambda^{out}$]{
        \label{fig:subfig:2a}
        \includegraphics[width=0.245\textwidth]{pic/abla_12.png}}
    \subfloat[ID\_Acc on different $k^{out}$]{
        \label{fig:subfig:2d}
        \includegraphics[width=0.245\textwidth]{pic/abla_42.png}}
    \subfloat[ID\_Acc on different $\lambda^{PA}$]{
        \label{fig:subfig:2b}
        \includegraphics[width=0.245\textwidth]{pic/abla_22.png}}
    \subfloat[ID\_Acc on different $\phi$]{
        \label{fig:subfig:2c}
        \includegraphics[width=0.245\textwidth]{pic/abla_32.png}}   
    \caption{Performance of AUTO with varying $ \{ \lambda^{out}, \lambda^{PA}, \phi, k^{out} \}$ on ResNet. Average FPR95 and ID\_Acc are reported.}
    \label{fig:anti_exp}
\end{figure*}

\textbf{Impact of different memory design details.}
We design a class-wise dynamic ID memory bank in AUTO, which plays an important role in maintaining ID performance.
Results in Table \ref{tab:mem_type} show why we choose class-wise instead of random sampling and why we choose dynamic updating instead of fixed strategy.
(\textbf{Random Sampling}: The initialization of memory is done through random sampling, and with each memory update, the sample with the longest storage time is replaced.)
Considering the randomness of OOD sample occurrence, the number of times each sample is trained in the ID memory may vary. 
Random sampling of the memory can lead to different learning frequencies for samples from different ID categories. 
This, in turn, may cause the model to forget some ID semantics, leading to sub-optimal ID and OOD performances.


\begin{table}[h]
    \centering
    \caption{Ablation study on different ID memory details. Model is trained on CIFAR-100 with ResNet-34.}
    %\resizebox{\linewidth}{!}{
    \begin{tabular}{lrrr}
    \toprule
    Memory &\textbf{FPR95 $\downarrow$} &\textbf{AUROC $\uparrow$}&\textbf{ID\_Acc $\uparrow$}  \\
    \midrule
    Random Sampling&15.24&96.38&77.53 \\
    Class-wise sampling&\textbf{11.06}&\textbf{97.50}&\textbf{77.91}\\
    \midrule
    Fixed &12.44&97.37&77.57\\
    Dynamic &\textbf{11.06}&\textbf{97.50}&\textbf{77.91}\\
    \bottomrule
    \end{tabular}
    %}
    \setlength{\belowcaptionskip}{-15pt}
    
    \label{tab:mem_type}
\end{table}


%\textbf{Impact of the number of used pseudo-OOD samples.}
%As shown in Figure \ref{}, we notice the degradation of OOD performance while the mixture ratio is huge.
%我们记录了不同比例下模型标注的伪OOD样本数量及错误标注量(将real-ID标注为伪OOD)。如图所示,我们可以看到随着错误样本量的减少,OOD检测性能有效提升,然而,同时我们考虑,AUTO提升模型的OOD检测性能是否需要这么多标注的样本。
%因此,我们在naiveOOD detection设定下使用有限量伪OOD样本更新策略,即OOD样本标注量达到一定值后停止更新模型,所得实验结果如图所示。我们conduct that 在连续的正确标注下,AUTO可以促进模型很快的达到OOD检测的最优状态。



\subsection{Hyper-parameter Analysis}
%We present extensive experiments on CIFAR-100 and ImageNet-1k benchmarks to discuss hyper-parameters influence in AUTO.

\textbf{Impact of $\lambda^{out}$.} 
As shown in Figure \ref{fig:subfig:1a} and \ref{fig:subfig:2a}, an appropriate $\lambda^{out}$ is crucial for the proper functioning of AUTO. 
We empirically conduct that OOD regularization is insufficient when $\lambda^{out}$ is too small, and the OOD detection performance cannot be maximally improved.
In contrast, when $\lambda^{out}$ is too large,  the gradient changes significantly during one iteration, leading to rapid forgetting of the original ID knowledge.
Such forgetting leads to a mixture of ID and OOD data in the feature space, causing horrible ID and OOD performances.
In a word, $\lambda^{out}=0.25$ is the best option in our evaluations.

\textbf{Impact of $k^{out}$.}
The boundary for pseudo-OOD annotations is initialized by $k^{out}$, and Figure \ref{fig:subfig:1d} and \ref{fig:subfig:2d} show the impact of $k^{out}$ on OOD detection performance.
We recorded $\mu^{in}$ and $\sigma^{in}$ for ResNet-34, WRN-40-2 on CIFAR and ResNet-50, ViT-B-16 on ImageNet, respectively.
Results are shown in Table~\ref{tab:muandsigma}.
We observe that when $k^{out}$ is small, the pseudo-OOD annotation boundary is initialized to a mixed interval of OOD and ID, leading to sub-optimal performance as the model initially selects ID data as pseudo-OOD samples.
When $k^{out}$ is large, the number of accurately annotated OOD samples decreases significantly, reducing the iteration count, and resulting in sub-optimal performance due to underfitting.

\begin{table}[h]
    \centering
    \caption{ID Statistics for backbones.}
    \begin{tabular}{lcccc}
    \toprule
         Model& ResNet-34 &WRN-40-2  &ResNet-50 &ViT-B-16 \\
         \midrule
         $\mu^{in}$&0.9977&0.9554&0.8460&0.8034\\
         $\sigma^{in}$&0.0111&0.1248&0.2170&0.2435\\
         \bottomrule
    \end{tabular}
    
    \label{tab:muandsigma}
\end{table}


\textbf{Impact of $\lambda^{PA}$ and $\phi$.}
The extent of the prediction-aligning objective is controlled by the parameters $\lambda^{PA}$ and $\phi$, as shown in Figure \ref{fig:subfig:1b}, \ref{fig:subfig:2b} and Figure \ref{fig:subfig:1c}, \ref{fig:subfig:2c}.
$\lambda^{PA}$ is set to constrain the proportion of $\mathcal{L}^{PA}_t$ in the overall objective.
$\phi$ is set to constrain the extent of calibrations in test model predictions.
When $\lambda^{PA}$ or $\phi$ is small, the prediction-aligning regularization is insufficient to correct the model effectively.
In contrast, when $\lambda^{PA}$ is large, the prediction-aligning regularization overly focuses on anti-forgetting issues, weakening the effectiveness of the OOD regularization.
Meanwhile, when $\phi$ is large, the gradients of predictions are significantly changed, noticeably weakening the model's optimization towards a uniform vector in predictions for OOD data.
Considering that a large $\lambda^{PA}$ for $\mathcal{L}^{PA}_t$ can lead to underperformance on both ID and OOD tasks.
To address this issue, we propose a gradually decreasing weighting factor $\beta$, which decreases as the number of iterations increases.
Table \ref{tab:reduce} demonstrates that this factor effectively prevents the degradation of OOD performance, but it also reduces the gain of $\mathcal{L}^{PA}_t$ on ID performance.
Thus, we provide the following recommendations: if you prioritize ID performance, consider using $\mathcal{L}^{PA}_t$ without $\beta$. If you prioritize OOD detection performance, opt for the $\mathcal{L}^{PA}_t$ with $\beta$.
\begin{table}[h]
    \centering
    \caption{A gradually reducing weighting factor for $\mathcal{L}^{PA}_t$ can enhance OOD detection but reduces the gain of $\mathcal{L}^{PA}_t$ on ID performance. Models are trained on CIFAR-100 with ResNet-34 and tested on six OOD datasets.}
    \begin{tabular}{lrrr}
    \toprule
       $\mathcal{L}^{PA}_t$  & \textbf{FPR95} $\downarrow$& \textbf{AUROC} $\uparrow$ &\textbf{ID $\_$Acc} $\uparrow$\\
    \midrule
        with $\beta$ & \textbf{9.92}&\textbf{97.52}&77.84\\
        w/o $\beta$& 10.06&97.50&\textbf{77.91}\\
    \bottomrule
    \end{tabular}
    
    \label{tab:reduce}
\end{table}
%In a word, an appropriate is important for calibrating the prediction of model.


Except for the individual hyper-parameter analyses mentioned above, we also surprisingly observe that the optimal hyper-parameters selected for each model consistently lead to excellent performance across different OOD test environments. 
This suggests that our hyperparameter selection is test-agnostic, meaning that for a given model, a fixed set of hyperparameters can be chosen to handle various deployment scenarios effectively.


\section{Conclusion}
In this paper, we propose the evolving test-time OOD detection problem where OOD detector is modifying online at test time.
Different from previous OOD detection setup, the new paradigm
Test-time OOD detection considers more practical and challenging scenarios.
We further propose a simple yet effective framework, AUTO, which adaptively selects and predicts test samples while updating models with them.
Extensive results demonstrate that our approarch can significantly enhance OOD detection performance while maintaining ID performance at the same time.
In the furture work, we plan to improve the annotation strategy thus enhancing the accuracy of selecting OOD samples in the unlabeled test data stream.
We hope our work could serve as a springboard for future works, provide new insights for revisiting the model development in OOD detection, and draw more attention toward the testing phase.


% if have a single appendix:
%\appendix[Proof of the Zonklar Equations]
% or
%\appendix  % for no appendix heading
% do not use \section anymore after \appendix, only \section*
% is possibly needed

% use appendices with more than one appendix
% then use \section to start each appendix
% you must declare a \section before using any
% \subsection or using \label (\appendices by itself
% starts a section numbered zero.)
%
% use section* for acknowledgment
%\ifCLASSOPTIONcompsoc
  % The Computer Society usually uses the plural form
%  \section*{Acknowledgments}
%\else
  % regular IEEE prefers the singular form
%  \section*{Acknowledgment}
%\fi


%The authors would like to thank...


% Can use something like this to put references on a page
% by themselves when using endfloat and the captionsoff option.
\ifCLASSOPTIONcaptionsoff
  \newpage
\fi



% trigger a \newpage just before the given reference
% number - used to balance the columns on the last page
% adjust value as needed - may need to be readjusted if
% the document is modified later
%\IEEEtriggeratref{8}
% The "triggered" command can be changed if desired:
%\IEEEtriggercmd{\enlargethispage{-5in}}

% references section

% can use a bibliography generated by BibTeX as a .bbl file
% BibTeX documentation can be easily obtained at:
% http://mirror.ctan.org/biblio/bibtex/contrib/doc/
% The IEEEtran BibTeX style support page is at:
% http://www.michaelshell.org/tex/ieeetran/bibtex/
%\bibliographystyle{IEEEtran}
% argument is your BibTeX string definitions and bibliography database(s)
%\bibliography{IEEEabrv,../bib/paper}
%
% <OR> manually copy in the resultant .bbl file
% set second argument of \begin to the number of references
% (used to reserve space for the reference number labels box)
%\newpage
{
\bibliographystyle{IEEEtran}
\bibliography{ref}
}

% biography section
% 
% If you have an EPS/PDF photo (graphicx package needed) extra braces are
% needed around the contents of the optional argument to biography to prevent
% the LaTeX parser from getting confused when it sees the complicated
% \includegraphics command within an optional argument. (You could create
% your own custom macro containing the \includegraphics command to make things
% simpler here.)
%\begin{IEEEbiography}[{\includegraphics[width=1in,height=1.25in,clip,keepaspectratio]{mshell}}]{Michael Shell}
% or if you just want to reserve a space for a photo:

%\begin{IEEEbiography}[{\includegraphics[width=1in,height=1.25in,clip,keepaspectratio]{pic/puning.jpg}}]{Puning Yang}
%received the BE degree in intelligence science and technology from Nankai University, Tianjin, China, in July 2020. He is currently working toward the MS degree in artificial intelligence with the University of Chinese Academy of Sciences, Beijing, China. His research interests focus on open-set recognition, pattern recognition, computer vision, and machine learning.
%\end{IEEEbiography}

% if you will not have a photo at all:
%\begin{IEEEbiography}[{\includegraphics[width=1in,height=1.25in,clip,keepaspectratio]{pic/puning.jpg}}]{Jian Liang}
%received the BE degree in electronic information and technology from Xi’an Jiaotong University, China, and the PhD degree in pattern recognition and intelligent systems from NLPR, CASIA, in July 2013, and January 2019, respectively. He was a research fellow at the National University of Singapore, Singapore from June 2019 to April 2021. Currently, he joins NLPR as an associate professor. His research interests focus on transfer learning, pattern recognition, and computer vision.
%\end{IEEEbiography}

%\begin{IEEEbiography}[{\includegraphics[width=1in,height=1.25in,clip,keepaspectratio]{pic/caojie.jpg}}]{Jie Cao} received the BE degree in automation from North China Electric Power University, Beijing, China, and the PhD degree in pattern recognition and intelligent systems from CASIA, in 2016 and 2021, respectively. Currently, he joins NLPR as an associate professor. His research interests include biometrics, pattern recognition, computer vision, and machine learning.
%\end{IEEEbiography}

%\begin{IEEEbiography}[{\includegraphics[width=1in,height=1.25in,clip,keepaspectratio]{pic/heran.jpg}}]{Ran He}
%received the BE degree in computer science from the Dalian University of Technology, the MS degree in computer science from the Dalian University of Technology, and the PhD degree in pattern recognition and intelligent systems from CASIA, in 2001, 2004, and 2009, respectively. Since September 2010, he has joined NLPR, where he is currently a full professor. His research interests focus on information theoretic learning, pattern recognition, and computer vision. He served as an associate editor of IEEE Transactions on Image Processing and IEEE Transactions on Biometrics, Behavior, and Identity Science and as an area chair of CVPR/ECCV/NeurIPS. He is a fellow of the IAPR.
%\end{IEEEbiography}

% You can push biographies down or up by placing
% a \vfill before or after them. The appropriate
% use of \vfill depends on what kind of text is
% on the last page and whether or not the columns
% are being equalized.

%\vfill

% Can be used to pull up biographies so that the bottom of the last one
% is flush with the other column.
%\enlargethispage{-5in}



% that's all folks
\end{document}


