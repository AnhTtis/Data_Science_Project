\begin{abstract}
This paper examines the art practices, artwork, and motivations of prolific users of the latest generation of text-to-image models. Through interviews, observations, and a user survey, we present a sampling of the artistic styles and describe the developed community of practice around generative AI. We find that: 1) the text prompt \textit{and} the resulting image can be considered collectively as an art piece (\textit{prompts as art}),
and 2) \textit{prompt templates} (prompts with ``slots'' for others to fill in with their own words) are developed to create \textit{generative art styles}. We discover that the value placed by this community on unique outputs leads to artists seeking specialized vocabulary to produce distinctive art pieces (e.g., by reading architectural blogs to find phrases to describe images). We also find that some artists use ``glitches'' in the model that can be turned into artistic styles of their own right. From these findings, we outline specific implications for design regarding future prompting and image editing options.

\end{abstract}