\section{Discussion}
In this research, we have provided a snapshot of the emerging artistic scene enabled by the latest generation of text-to-image models. Our interviews and observations document the types of imagery artists are developing with these new models, as well as an enhanced understanding of the types of results creators seek beyond the images themselves (e.g., prompts and prompt templates as important artifacts in their own right).

In this section, we discuss some implications deriving from artists' goals of seeking novelty, validating originality, and producing reusable prompt templates.  In reviewing design implications for these models, we note that there are countless ways the input and editing interfaces could be improved for interacting with these models (for example, one can find many proposals and working demos online). In this discussion, we restrict ourselves to interfaces that only take text as input, with no post-processing of the image (such techniques that allow in-filling of regions by the model). We limit our discussion to this input modality in part because a number of the artists interviewed were seemingly attracted to the simplicity (and challenge) that this single input modality provides.

\subsection{Aiding Novelty}
In our interviews, the use of thesauri and domain-specific blogs (e.g., architectural blogs) illustrate the desire of artists to identify unique terms that help them produce results that rise above the average. Embedding these search capabilities directly into the tooling could be useful (e.g., quick access to a thesaurus or an embedded search engine). Pushing this idea further, there may also be opportunities to make use of large language models (LLMs) and/or the TTI model's training data to help surface salient terms for a given topic. For example, an LLM may be able to generate terms-of-art for architecture using a prompt like this: \textit{Here are some terms specific to describing the architectural design of a house: 1)}. In our own test with an LLM \footnote{Specific model anonymized for review}, this prompt yielded terms like \textit{entryway}, \textit{foyer}, \textit{entry hall}, and \textit{elevation}. To use the training data itself to identify new spaces, it may be possible to collect terms associated with a particular topic, such as ``house,'' then identify other terms that are more frequently found near that topic in the training data compared to the rest of the training data (e.g., using a method such as term frequency-inverse document frequency (TF-IDF)).

To help users understand how unique their word choices are, one could also visualize the input prompt with respect to how common each individual input token is. For templates, one could also show what the common terms would be for filling in the blanks to help people move beyond those common terms to find more distinct, unique inputs. One possible outcome in helping people find more novel inputs is that their inputs may lie outside of the training distribution, leading to unpredictable results. Providing feedback through mechanisms like visualizations (e.g., showing frequency in the training data) could help users better understand these types of issues should they arise. 

\subsection{Validating Originality}
As mentioned, there is a desire to validate the originality of the image produced by the model. Streamlining the process of doing an image search with an online search engine is one obvious way to address this issue. However, as is the case in aiding novelty (described above), there may be opportunities to take advantage of the training data itself. More specifically, in addition to external search, one could also search for the closest images in the training data to the image produced.

While the above mechanisms would help validate the outputs generated, there may also be opportunities to help validate the originality of the \textit{input}. For example, one might be able to search the training data to identify which parts of a prompt exist within the training data.

\subsection{Materializing Prompt Templates}
Prompt templates provide a way for an artist to derive a new style, then share it with others so they can produce their own unique images. One could imagine embracing this practice and transforming a prompt template into its own first-class interface.

For example, one could imagine allowing users to provide multiple inputs for each slot of a prompt template, then generating the cross-product of all the inputs. One could also make use of the fact that the prompt text for the templates is tokenized into vectors. Specifically, by supplying two different inputs for the same slot, embedding vectors for each input could be obtained and then automatically interpolated between the embeddings to produce a spectrum of outputs. For example, for the shape of a house (in the previous house prompt template), the user could provide two inputs: strawberry and apple. The system could then produce the embedding vectors for those inputs, then interpolate between those embeddings to create a series of images that morph from a strawberry to an apple. However, one thing to keep in mind with these interpolations is they are occurring in the text space rather than image space---the model will be interpolating not between shapes (per se) but between the \textit{linguistic concepts} of strawberries and apples (which may still produce an interesting morphing between these conceptual entities).



\subsection{Prompts and TTI Models as an Art Medium}
One the primary ways the three spotlighted artists (A1, A2, and A3) distinguished themselves from others was their perception of TTI models as an art medium, with a clear focus on exploring the capabilities and limits of medium itself, rather than only on individual outcomes. In this spirit, they embraced 
%The contrast in the perspectives becomes clearer as the artists embrace 
the limitation of not being able to edit the images directly, accepting the text-only input as a defining feature and characteristic of the medium. For example, A2 expressed ``there's beauty in it'' when describing the prompting interaction with the models. In contrast, other participants focused more on the outcome, and expressed desire for additional features, such as direct editing of the generated images. 
%This is similar to how painters explore different paints and brushes and the uniques characteristics. For example, oil paintings or watercolor paintings have their specific styles people can recognize and associate with. 
Embracing the TTI models as-is further reinforces the idea that text prompts are part of the artwork, rather than simply a means to an end. 
%The perspective of prompting interaction as a characteristic of the art medium enforces the idea that text prompts are part of the artwork, and that the exploration of the concept through a collection of prompts and images as the unit of artwork. 


This observation suggests that the design implications for TTI models can be considered from multiple perspectives: prompt-only artists, and creative professionals using the models to achieve specific goals.
%building artist support tools  are different than building tools for creative professionals. 
Creative professionals with a specific design goal may require and request 
%with certain downstream applications will involve 
specific features that offer more fine-grained control, perhaps with tools to help understand model behavior. For example, features related to directly editing the generated images were among the most frequently requested features in 
%are the most asked ones from the creative professionals in 
the survey. 
% \rr{As the communication of design requirements between the ``customer'' and the design professional will become more active and iterative as non-design experts can also learn to use the TTI models to quickly demonstrate and visualize artifacts, we think the traditional pipeline-like design workflow can be transformed to be more synchronous and collaborative. Perhaps as more design tasks involve direct access to shared artifacts by more stakeholders, feedback and collaboration features could become more important}. 
\rr{Given that non-design experts can also learn to use TTI models to quickly demonstrate and visualize artifacts, we hypothesize that this may facilitate more active and iterative communication between design professionals and their clients, with clients more directly engaged in the creative process (as opposed to the more traditional pipeline-like design workflow). If this proves to be true, feedback and collaboration features could become more important}.


\subsection{Limitations}
Our participant pool was drawn among employees of one large US-based corporation, and does not cover other possible ways that culture, community, and collaboration might shape use of TTI models (e.g. on social media). Also, since our analysis was episodic rather than longitudinal, we do not document how artistic prompting strategies may evolve within individuals. For the interactions we \textit{could} observe, observing participants' interactions with the TTI model does not definitively indicate their conceptions of how the model works or how best to prompt it. We also acknowledge that different models behave in different ways due to their structure, training data, and the design of the interface. \rr{In that regard, we also acknowledge some of the findings might be model dependent, and are specific to the models used in the study. Similarities and differences in art forms and art practices might be observable with different models and communities. Another limitation is that while our participants are actively involved in multiple communities, %and we know knowledge sharing across communities are also very active, however, 
we did not ask participants about their experiences with other models or other communities in depth.}
