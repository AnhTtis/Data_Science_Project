\section{Conclusion}
In this paper, we have described a unique moment in time: Recent text-to-image models have given rise to an exceptionally vibrant community of practice, complete with new ideas about what constitute notable outcomes (e.g., new styles, prompts as art, and prompt templates as art). As the larger artistic and creator community adopts these new models and forms of art, there are clear ways in which the tools can improve to better support desired practices including: 1) helping to discover and create novel outputs, 2) providing methods to validate novelty, and 3) elevating the notion of a prompt template into a standalone, first class, interactive object. In considering design implications for these TTI models, our results also suggest the value in distinguishing between prompt artists---those users who embrace the constraint of creating images using only an input prompt---and practitioners, who may desire more fine-grained input and editing controls in comparison.

