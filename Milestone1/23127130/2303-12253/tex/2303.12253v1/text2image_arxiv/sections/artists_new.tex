\section{Survey Results}

We received 161 responses to the survey. Of these responses, 160 answered the question, ``At present, why are you using the [TTI] models?'' Of the responses to this question, 79 (49\%) indicated they use the models to create art, 33 (21\%) reported using it as part of their creative work pipeline, and 126 (79\%) indicated their use was curiosity-driven (not work-related).

In the survey, we also asked participants to estimate the length of time they work with a TTI model when they use one (``When you interact with a model, how much time do you typically spend interacting with it?''). We received 157 responses to this question, with 20\% of respondents indicating that they use a model for one or more hours at a time when they use it (11\% reporting using it for 1-2 hours at a time, 9\% using it for 2 or more hours at a time), 53\% indicating they use a model for 10 minutes to an hour, and 27\% reporting use for less than 10 minutes.

% The remainder of the survey were specific feature requests, which are outside the scope of this study. 

When asked about observed strengths and weaknesses of the various models they've interacted with, survey responses indicated a number of desired capabilities, such as the ability to render text in images, the ability to have more control over spatial arrangements, and the ability for models to handle complex prompts. We also asked participants for desired capabilities when interacting with the model. Seventy six percent of the respondents requested a ``how to build a prompt'' guide, and 75\% desired the ability to fork and remix images, especially for spatial refinement. Seventy six percent of the responses also indicated they would like features like bookmarking, and the ability to directly share outputs to  internal chat groups or social media. As we will see in the artist spotlights below, there is a clear social component to working with these TTI models for our study participants.

Finally, 63\% of the responses desired greater control over the model, such as the ability to assign specific values to each of the prompt words. %, and the ability to turning on and off safety filters.

%Survey responses to the question ``Which image generation models have you used? What strengths and weaknesses have you observed for each? '' discussed how do models handle complex prompts, the ability to render texts in images, the resolution of the images, and the spatial arrangement in the images.

When asked to provide prompting techniques they have learned, %(``Are there any successful/fun tricks or prompts you've used to produce interesting images?''), 
common themes for the strategies included 1) producing specific art styles and eras, such as ``impressionist style'', 2) use of keywords that describe camera lenses and aperture (e.g., ``DSLR photo'',``3D render'',``24 mm, f8, ISO1000''), and 3) domain-specific terms (e.g., ``Line Art'', ``black and white'').


\section{Prompt Artists: Styles, Motivations, Practices}

In this section, we provide a sampling of the vibrant internal TTI artistic community by spotlighting the work of three highly active creators; Shai Noy, Irina Blok, Dan Smith.  \footnote{In the text, we use the terms ``artists,'' ``creators,'' and ``study participants'' interchangeably. We denote the three spotlighted artists as A1, A2, and A3, and other participants by a participant number (e.g., P1, P2, ..., P8).}. For these three artists, we describe and present examples of the styles they have developed and summarize their artistic motivations and goals. We then provide a summary of motivations, styles, and practices observed across the 11 interview participants. In the section that follows, we describe salient high-level, emergent themes arising from the interviews and observations. We want to credit the artists whose artwork are featured, and they had expressed the desire to be associated with their artwork. We use their full name in places where the artwork appears.


\subsection{Shai Noy: The Explorer (A1)}
Shai Noy is a software engineer with no training in the visual arts or design. However, they have produced thousands of images with TTI models. The styles developed by this artist include ``super macro photography'' images (i.e., extremely close-up views of objects, Figure \ref{fig:macro}), and fashion (dresses, suits) made out of unusual materials, such as wood, grass, brick, or ice (Figure \ref{fig:dresses}).

Elements of both discovery and community were emphasized as rewarding for this artist, such as being the first to explore particular concepts and the ability to share discoveries: %``The exploration itself is interesting, a collection of images displays the thought process,'' and 
% ``I first explored these set of concepts'' (A1). Speaking to the community-based aspects of the movement: 
``Everything is more fun when you can share it'' and, ``Art doesn't live in a vacuum, nobody starts from scratch, everything is based on something else. I am proud of being able to recognize the potential'' (A1).

\begin{figure}[t]

    \begin{subfigure}[b]{\textwidth}
        \includegraphics[width=0.24\linewidth]{\imagepath{A1_-_macro_micro_photo_1.png}}
        \hspace{\fill} % note: no blank line here
        \includegraphics[width=0.24\linewidth]{\imagepath{A1_-_macro_micro_photo_4.png}}
        \hspace{\fill} % note: no blank line here
        \includegraphics[width=0.24\linewidth]{\imagepath{A1_-_macro_micro_photo_2.png}}
        \hspace{\fill} % note: no blank line here
        \includegraphics[width=0.24\linewidth]{\imagepath{A1_-_macro_micro_photo_3.png}}
        \caption{Super macro photography} 
        \label{fig:macro}
    \end{subfigure}

    \vspace*{0.2cm} % vertical separation

    \begin{subfigure}[b]{\textwidth}
        \includegraphics[width=0.24\linewidth]{\imagepath{A1_-_A_beautiful_dress_carved_out_of_dead_wood_with_lichen_and_mushrooms,_on_a_mannequin._High_quality,_high_resolution,_studio_lighting.png}}
        \hspace{\fill} % note: no blank line here
        \includegraphics[width=0.24\linewidth]{\imagepath{A1_-_A_beautiful_dress_made_out_of_grass_and_dirt,_on_a_mannequin._High_quality,_High_resolution,_Studio_lighting.png}}
        \hspace{\fill} % note: no blank line here
        \includegraphics[width=0.24\linewidth]{\imagepath{A1_-_A_beautiful_winter_coat_made_out_of_a_brick_fireplace,_on_a_mannequin._High_quality,_high_resolution,_studio_lighting.png}}
        \hspace{\fill} % note: no blank line here
        \includegraphics[width=0.24\linewidth]{\imagepath{A1_-_A_beautiful_winter_jacket_made_out_of_ice,_on_a_mannequin._Studio_lighting,_high_quality,_high_resolution.jpg}}
        \caption{Dresses in various unusual materials}
        \label{fig:dresses}
    \end{subfigure}
    \caption{Selections from Shai Noy (A1 - The Explorer)} 
    \label{fig:artist1}
\end{figure}


\subsection{Irina Blok: The Art Director (A2)}
Irina Blok is a designer who has done visual work for many years using stock images and applications like Photoshop. This artist's styles include origami dancers (Figure \ref{fig:origami}) and reality ``mash-ups,'' such as sliced produce that contains different textures internally (e.g., a sliced head of cabbage that reveals a cross-section of an orange, Figure \ref{fig:mashups}).

Driving these styles is a passion for developing ``impossible objects, something we haven’t seen before, mashups'' (A2). They also seek to create images that differ significantly from the images it was trained on: ``The further away the generated images are from what it was trained on, the more the satisfaction'' and, ``they're [the resulting images] also defying the rules of its training set, like defying [...] gravity. [..] A creative prompt breaks that'' (A2).

Importantly, this artist's output also includes \textit{prompt templates} that describe a particular, parameterized image or visual concept, such as this prompt for generating a house: \textit{Audacious and whimsical fantasy house shaped like <object> with windows and doors, <location>}. These prompt templates are carefully crafted to reliably produce pleasing results for others. We describe this concept more fully in a later section.

% Beyond producing images, A2 also sees text-to-image models as a means to building more effective language learning tools: They find generating images in a reliable and consistent way to be an interesting, engaging method for understanding and learning a language. Specifically, they observed it's a ``tool that allows us to create images in a reliable way, that is also consistent, [...] and used for language learning. Because when we just used text for language learning it's not engaging and it doesn't translate well because the text has to be internalized''. Drawing from previous studies, they added, the models can make ``the visual cues, and they actually contributes to remembering the material and lead to better learning outcomes''.

Notably, when working with the text-to-image models, A2 considers the model to be akin to an artist itself, with themselves the art director. In this relationship, A2 acknowledges a certain lack of control: ``[I] don’t have full control, and there’s beauty in this'' (A2). At the same time, they also consider the model to be a tool: ``[The model] is a brush [...] you just learn to speak its language'' (A2). With these dual views (model as an artist, model as a tool), they note that ``the hardest part is [the] conceptual aspect, being skillful with [the] prompt'' and that ``it’s a thought exercise, it’s not a visual exercise'' (A2). It’s really about how to make people think like an artist''  (A2). In this latter sentiment, this participant was specifically speaking to the need to think like an artist in formulating a prompt, as opposed to formulating a prompt as if one was talking to a machine.

% [TODO: Add: ``There’s a common misconception art is largely about drawing and painting skills. Art is not only about how something looks, it’s about what it says, it tells a story, and has a concept. Art can surprise, provoke, teach, delight and inspire. Art is not just about drawing, art is a way of thinking. [A2]'']

% [TODO: Maybe add: In interacting with the model, Artist 2 considers the model to be akin to an artist itself, with themselves the art director. In this relationship, A2 acknowledges a certain lack of control: ``[I] don’t have full control, and there’s beauty in this'' (A2). At the same time, they also consider the model to be a tool: ``[The model] is a brush [...] you just learn to speak its language'' (A2). With these dual views (model as tool, model as artist), they note that ''the hardest part is [the] conceptual aspect, being skillful with [the] prompt'' and that ``it’s a thought exercise, it’s not a visual exercise. It’s really about how to make people think like an artist'' (A2). [TODO: Verify and clarify these points; there seem to be a lot of interesting nuggets here.] Yet, despite the known challenges of creating a successful prompt, they also suggest that with these models ``anybody can be an artist, and I truly mean it'' (A2).]

\begin{figure}[t]
    \begin{subfigure}[b]{\textwidth}
        \hspace{\fill} % note: no blank line here
        \includegraphics[width=0.24\linewidth]{\imagepath{A2_-_origami_dancer_1.png}}
        \hspace{\fill} % note: no blank line here
        \includegraphics[width=0.24\linewidth]{\imagepath{A2_-_origami_dancer_2.png}}
        \hspace{\fill} % note: no blank line here
        \includegraphics[width=0.24\linewidth]{\imagepath{A2_-_origami_dancer_3.png}}
        \hspace{\fill} % note: no blank line here
        \caption{Origami dancers} 
        \label{fig:origami}
    \end{subfigure}
    
    \vspace*{0.2cm} % vertical separation
    
    \begin{subfigure}[b]{\textwidth}
        \includegraphics[width=0.24\linewidth]{\imagepath{A2_-_Two_objects_1.png}}
        \hspace{\fill} % note: no blank line here
        \includegraphics[width=0.24\linewidth]{\imagepath{A2_-_Two_objects_2.png}}
        \hspace{\fill} % note: no blank line here
        \includegraphics[width=0.24\linewidth]{\imagepath{A2_-_Two_objects_3.png}}
        \hspace{\fill} % note: no blank line here
        \includegraphics[width=0.24\linewidth]{\imagepath{A2_-_Two_objects_4.png}}
        \caption{Reality mash-ups} 
        \label{fig:mashups}
    \end{subfigure}
    \caption{Selections from Irina Blok, (A2 - The Art Director)} 
    \label{fig:artist2}
\end{figure}


\subsection{Dan Smith: The Social Commentator (A3)}

Dan Smith comes from a background in visual media, and is partially driven by the desire to deliver a message about the climate crisis, in order to facilitate change and awareness: ``[I'm] not just having fun ... but [actually] making something that has some power'' (A3). In working towards this goal, they want to make ``something that you look at, and it makes you feel something'' while also making ``something that people would want to look at'' (A3). A3's image styles align with these overall goals: They have a number of pieces that put nature in ``situations where you wouldn't see it'' (Figure \ref{fig:climate-change}) and have created numerous ``hybrid animals'' (Figure \ref{fig:hybrid-animals}). This particular style---hybrid animals---is also inline with a motivation to create images that ``would be hard to [...] visualize or create, if you were [...] a really skilled Photoshop artist'' (see Figure \ref{fig:artist3}).

A3 also heavily considers the quality of the image when assessing its outputs: it must have near expert-level composition, photorealism, and/or artistry. When describing the artwork they created and how they created them, they noted, ``I think the ones that I pick out are [...] standouts for various reasons, and just [...] like what I said, [...] composition photorealism, artistic quality'' (A3). Consistent with their emphasis on overall image quality, they have found ways to address undesirable outputs. For example, in generating images that include animals, they found they needed to adopt a specific strategy to create aesthetically pleasing images: ``I would do `Tall Grass' a lot because early on I discovered that limbs and fingers and paws can get a little wonky'' (A3). The ``tall grass'' addition was their creative strategy to hide feet or paws and suggests an understanding of model limitations, but also a sense of how to cope with these limitations.

\begin{figure}[t]
    \begin{subfigure}[b]{\textwidth}
        \includegraphics[width=0.24\linewidth]{\imagepath{A3_-_Climate_1.png}}
        \hspace{\fill} % note: no blank line here
        \includegraphics[width=0.24\linewidth]{\imagepath{A3_-_Climate_2.png}}
        \hspace{\fill} % note: no blank line here
        \includegraphics[width=0.24\linewidth]{\imagepath{A3_-_Climate_3.png}}
        \hspace{\fill} % note: no blank line here
        \includegraphics[width=0.24\linewidth]{\imagepath{A3_-_Climate_4.png}}
        \caption{Climate change} 
        \label{fig:climate-change}
    \end{subfigure}
    
    \vspace*{0.2cm} % vertical separation
    
    \begin{subfigure}[b]{\textwidth}
        \includegraphics[width=0.24\linewidth]{\imagepath{A3_-_Hybrid_1.png}}
        \hspace{\fill} % note: no blank line here
        \includegraphics[width=0.24\linewidth]{\imagepath{A3_-_Hybrid_2.png}}
        \hspace{\fill} % note: no blank line here
        \includegraphics[width=0.24\linewidth]{\imagepath{A3_-_Hybrid_3.png}}
        \hspace{\fill} % note: no blank line here
        \includegraphics[width=0.24\linewidth]{\imagepath{A3_-_Hybrid_4.png}}
        \caption{Hybrid animals} 
        \label{fig:hybrid-animals}
    \end{subfigure}
    \caption{Selections from Dan Smith (A3 - The Social Commentator)} 
    \label{fig:artist3}
\end{figure}

\subsection{Summarizing Motivations, Styles, and Practices}

One of the primary motivations for interacting with the models was that participants found them fun---the models' output quality enabled people to feel creative, and they were generally interested in interacting with this new class of model. Some people also noted that the model enabled them to engage in their domain interests in a new way. For example, one participant said that the models allowed them to explore their interest in Swiss trains with the model, while another found it compelling to try to create new forms of currency (e.g., new types of coins). One participant used it to create the equivalent of clip art for presentations: ``I use [TTI] model when I need some [...] clipart to use in my presentation, and [TTI] model would be amazing for that...'' (P2).

In comparing the artists spotlighted above to the other participants, one notable difference in practices is that the spotlighted artists placed a particular emphasis on exploring specific themes in depth (e.g., origami figurines). The other participants did not pursue concepts with the same rigor or depth.

