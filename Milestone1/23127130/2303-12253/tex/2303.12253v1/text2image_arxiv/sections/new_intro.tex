\section{Introduction}
Advances in text-to-image(TTI) models have led to significant improvements in the quality of computer-generated, synthetic images \cite{NEURIPS2021_49ad23d1, ho2022cascaded}. A new generation of text-to-image models enable the creation of high-fidelity images via descriptive text prompts by leveraging advances in large language models ~\cite{DALLE279:online,Midjourn12:online,saharia2022Imagen,yu2022Parti,Rombach_2022_CVPR}. 
With broadening access to these models, communities of practice have emerged, enabling people to share designs, prompts, and example images. For instance, there are now tools to help people write prompts~\footnote{https://promptomania.com/prompt-builder/}, and even marketplaces for successful prompts~\footnote{https://promptbase.com/shop}.

% The algorithmic foray into generative art is not a novel idea. 
%The phenomenon of computer-generated art has been extensively studied in the past (e.g., 
Prior work has examined the phenomenon of computer-generated art in a variety of contexts  
% highlighted the tensions of negotiating agency and control when using a machine to create art 
~\cite{10.1145/3326338,aguera2017ArtInAI,mazzone2019ArtCreativityAI}. For example, as an art historian analyzing the AI-assisted art movement, Mazzone et al.~\cite{mazzone2019ArtCreativityAI} describe how the artist's role has adapted to include pre-curation, tweaking, and post-curation. 
% The authors argue that if the AI system can produce novel output, not simply reproductions of examples from its training set, it can produce art. However, they note that subjective perceptions of the output determine whether it is considered art or not~\cite{aican2017Elgammal}. 
More recently,~\citeauthor{hertzmann2020ComputersArt} argues that text-to-image (TTI) models like DALL·E do not themselves create art, but that the artists and technologists who apply them as tools are the ones creating art \cite{hertzmann2020ComputersArt}. With the emergence of this new class of models, which are capable of producing extremely high quality images from textual descriptions (e.g., \cite{PartiPat10:online, ImagenTe79:online, DALLE279:online}), 
we are motivated to understand how this new technology is being adopted and used by creators.

In this research, we provide a snapshot of a vibrant community of \rr{art} practice that has arisen around text-to-image models, \rr{ sharing insights into the ingenuity and creativity of the users of these models.\footnote{We thank our anonymous reviewers for the specific phrase recognizing our contribution.}}. Within a US-based technology company that has produced its own TTI models, we sent a survey to the TTI models users to gain a basic understanding of how and why they are used. We also interviewed and observed 11 prominent users of these models \rr{who are using the models as an art medium}, recruiting from survey respondents and by directly asking prominent users. \rr{They have each generated thousands of images with both the internal and other publicly available models, and are actively sharing their creations in multiple communities.} In studying \rr{the artistically-driven} members of this local community, we sought to understand their practices, their artifacts, and their motivations for engaging extensively with the models. For the purposes of this research, we scope our inquiry to studying interfaces that only accept text as input (recognizing that a wide variety of model capabilities and interfaces are available, including those that enable more fine-grained editing of images). We restrict our scope to text-only interfaces because these were the first interfaces available for models such as DALL-E, and these text-based interfaces have seen considerable use by the public and our internal users. %\rr{Furthermore, while our study focuses on the practices and creations deriving from use of internal models, the study findings highlight themes likely to be found with use of TTI models in general, making the contributions a useful addition to the larger understanding of artistic uses of TTI models.}

% Contributions
Our study reveals that users of these models have developed a range of artistic styles, including origami figurines, fashion (e.g., dresses) made out of materials like bricks, and reality ``mash-ups'' that create hybrids of animals or of fruits and vegetables (see Figures \ref{fig:dresses}, \ref{fig:origami}, \ref{fig:mashups}, \ref{fig:hybrid-animals}). However, we also found that the artistic outputs of this community of users are not limited to the images themselves. For example, the \textit{prompt itself} is an important output, and a piece of the art: a parsimonious, descriptive prompt accompanying the image is seen as a virtuous goal beyond just the image, as it simultaneously acts as a ``title'' for, and description of, the art piece (see Figure \ref{fig:teaser}, A1). Similarly, a \textit{prompt template}---a text prompt with one or more empty ``slots'' for others to fill in---is considered an art piece all on its own (see Figure \ref{fig:teaser}, A2). Among other characteristics, a well-designed prompt template has the property of encapsulating an artistic vision that can nonetheless be customized by future users of that prompt template.

Our results also reveal the lengths some users go to when searching for unique, distinctive outputs. In particular, some creators turn to thesauri or online, domain-specific blogs (e.g., architectural blogs) in search of vocabulary that elevates the model output beyond the ordinary. This focus on vocabulary suggests that capable TTI model artists may also benefit from being highly skilled with natural language. Another creator  explicitly seeks unique model outputs, but through identification of ``glitches'' that can be elevated to styles all on their own. For example, this latter artist found the model did not  render reflections in mirrors perfectly, and explored this concept through a number of pieces (see Figure \ref{fig:teaser}, G2).

Finally, we find that the artists interviewed place a premium on originality, with some turning to image search to validate that their outputs are, in fact, unique.

In sum, this paper presents results from a survey and interview study of heavy users of TTI models, making the following contributions:
\begin{itemize}
    \item \textbf{Usage summary}: From survey data from 161 respondents, we find that when they use a model, 20\% of respondents report using a TTI model for one or more hours at a time, indicating fairly sustained use of these models by a sizable portion of the community surveyed.
    \item \textbf{Sample styles}: We provide a sampling of artistic styles developed by study participants to contextualize the types of outputs being produced by new text-to-image models.
    \item \textbf{Prompt as art}: We find that the prompt itself is often considered a part of the artistic output (in addition to the actual image), with artists pursuing a goal of creating parsimonious, descriptive input prompts.
    \item \textbf{Prompt templates as art}: We discover that artists also produce \textit{prompt templates} to encapsulate a unique visual concept that others can customize.
    \item \textbf{Natural language mastery for visual language artistry}: We describe how TTI artists seek unique natural language in an attempt to elevate their pieces beyond the norm.
    \item \textbf{Glitches as art}: We show how some artists look for ``glitches'' that can be reliably transformed into new styles.
    \item \textbf{Validating originality}: We describe artists' concerns in validating their outputs as original, and how they currently validate through image search.
\end{itemize}

Together, these findings suggest new directions for interactive interfaces and aids for prompt-centric uses of TTI models: 1) Methods and tools to help users locate novel language and capabilities of the model, 2) aiding users in validating the originality of their outputs, and 3) reifying the notion of a prompt template into a standalone computational artifact that supports richer interaction by users of the template. \rr{Importantly, while our results derive from a study of internal TTI models, the implications for design are generally applicable to use of any TTI model (e.g., the notion of a prompt template is useful for any TTI model, as it captures a particular artistic vision in a portable, yet customizable. form).}

In the rest of the paper, we review related work, describe our study method, present results from the survey and interview study, and conclude with a discussion that draws implications for design from the study data.