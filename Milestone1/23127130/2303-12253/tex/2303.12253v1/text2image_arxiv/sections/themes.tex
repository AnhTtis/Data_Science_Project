
\section{Themes: Art Beyond Images, Discovering Unique Points, and Validating Originality}

Across our interviews and observations, a number of themes emerged. First, the notion of what constituted the final artistic output was not always just an image: Some creators consider the \textit{prompt itself} as part of the art. Similarly, a \textit{prompt template} can be considered an artistic output. Second, there was a clear desire for creators to discover new styles possible with the models. This focus on originality extended to one creator going so far as to conduct an image search on successful outputs to validate their originality. We unpack these themes further in this section: 1) Prompt as art, 2) Prompt template as art, 3) Discovering new capabilities of the model, and 4) Validating originality.


\subsection{Prompt as Art}

\rr{For some participants, the prompt itself was part of the overall art, and thus worthy of attention. For these participants, it was important to ``[create] aesthetically pleasing images'' and ``[develop] art concepts'' that \textit{were inherently tied to prompts}.
%, and to use prompts to ``discover new model capabilities.'' 
We detail these motivations and behaviors below.}

While generating aesthetically pleasing, ``glitch-free'' images was a common goal of the creators, other goals were also present in their practices. For some participants, the prompt itself was part of the overall art, and thus worthy of attention 1) on its own, and 2) as it relates to the image: ``It's part of the aesthetic'' (P3), where the prompt is ``like a title of the piece, but you don't get to choose it independently'' (P3). Hinted at in this last comment is the notion that while a prompt could also serve as a title of the art piece, there is a clear dependency on, and perfunctory role for, that prompt as well: The prompt serves as the source material for the model generating the resultant image. Given this dependency, finding a prompt that produces the desired result \textit{and} that can serve as a title for the piece can be challenging, but rewarding when it happens.

\begin{figure}[t]
    \captionsetup[subfigure]{justification=centering}
    \begin{subfigure}[t]{0.4\textwidth}
    \centering
        % \hspace{\fill} % note: no blank line here
        \includegraphics[width=0.6\linewidth]{\imagepath{owl_croissant.png}}
        \subcaption{``a photo of an owl turning into a croissant. f2.2''} 
        \label{fig:owlcroissant}
    \end{subfigure}
        % \hspace{\fill} % note: no blank line here
    \begin{subfigure}[t]{0.4\textwidth}
        \centering
        \includegraphics[width=0.6\linewidth]{\imagepath{bananasaurus.png}}
        \subcaption{``a photo of a bananasaurus rex. a bananasaurus rex has the legs and arms of a banana and the body and head of a tyrannosaurus rex. sigma 85mm f2.2. studio lighting.''} 
        \label{fig:bananasaurus}
        % \hspace{\fill} % note: no blank line here
    \end{subfigure}
    \caption{Artist Ian Fischer (P3) considers (a) as a successful art work given the brevity of the prompt, while (b) a failed attempt because the prompt is too long to get the image they want}
    \label{fig:prompt-as-art}
\end{figure}

This same artist (P3) further described the prompt's role as 
% [TODO: larger context here] 
``communicating the image, and the idea of the image, and how I got it all at the same time'' (P3). % This quote sheds light on how the multimodal nature of TTI models enables artists to think about the way they conceptualize images in a new way. 
This quote sheds light on how art with TTI models can, in some sense, be considered multimodal (text and image) for both the artist and viewer: The prompt and the final output combine into a single, mutually-reinforcing, art piece.
Seen in this light, one can consider the \textit{prompt as art} itself---a well-crafted prompt that creates a compelling image but also accompanies that image, saying something about the image. See Figure \ref{fig:prompt-as-art}a for an example of ``prompt as art,'' as well a prompt that wasn't able to achieve this same level of aesthetic (Figure \ref{fig:prompt-as-art}b).

Diving deeper into this theme, we %were able to learn why the artists regard the prompts itself as an artwork. We 
observed that the artists believe the ``concept'' is a critical component of an artwork. In an internal blog article, A2 describes, ``There’s a common misconception art is largely about drawing and painting skills. Art is not only about how something looks, it’s about what it says, it tells a story, and has a concept. Art can surprise, provoke, teach, delight and inspire. Art is not just about drawing, art is a way of thinking.'' A1 also suggested ``the unit of shareable artwork is not necessarily a specific image but maybe it's the whole exploration of the concept of those images.''

% The exploration of concepts are reflected in the prompts the artists created; others can then view the thought process and the story embedded in the prompts. 

In pursuing ``prompt as art,'' we observed participants impose different constraints when creating their prompts, with some wanting it as descriptive as possible, while others attempting to make it as simple as possible. One participant even discovered delight in their accidental discovery that their ``random string'' produced beautiful output, and named that prompt as their proudest achievement. In this latter case, the pride comes from the joint pairing of the random string and the beautiful output---without the context of the random string, the image has less value, as the random string reveals an unexpected feature of the model.
% ``wpritwuq''

% Regarding the prompt creation process, A2 emphasized ``the hardest part is the concept part, it’s a thought exercise, it’s not a visual exercise, it’s really about how to make people think like an artist.''

% [TODO: This seems to be saying that part of their goal / aesthetic is the prompt. If so, add in here: ``It matches the prompt in a clever or interesting way or the prompt itself is interesting and it sort of says something'' (A3)]


\subsection{Prompt Templates As Art}

\begin{figure}[t]
    % \begin{subfigure}[b]{\textwidth}
        \hspace{\fill} % note: no blank line here
        \includegraphics[width=0.24\linewidth]{\imagepath{strawberry_countryside.png}}
        \hspace{\fill} % note: no blank line here
        \includegraphics[width=0.24\linewidth]{\imagepath{strawberry_paris.png}}
        \hspace{\fill} % note: no blank line here
        \includegraphics[width=0.24\linewidth]{\imagepath{strawberry_tokyo.png}}
        \hspace{\fill} % note: no blank line here
        \caption{Example of the prompt ``Audacious and whimsical fantasy house shaped like <object> with windows and doors, <location>''. All images above use ``strawberry'' as the shape, with the location varied (``countryside'', ``Paris'', ``Tokyo'').} 
        \label{fig:house_template}
    % \end{subfigure}
\end{figure}

In a spirit similar to ``prompt as art,'' five artists sought to produce \textit{prompt templates}. A prompt template is an image description with ``slots'' for someone else to fill in. For example, we previously noted this prompt template created by A2: \textit{Audacious and whimsical fantasy house shaped like <object> with windows and doors, <location>} (see Figure \ref{fig:house_template} for example outputs of this prompt template).

\rr{These prompt templates leverage the capabilities of the models as well as the lightweight, accessible features of prompts. More specifically, when an artist identifies a compelling composition, they can create a text prompt that allows others to create a similar composition, but with their own unique customization to it. %Because of the model's non-deterministic behavior, prompt templates can be thought of as a joint collaboration between the template author, the template user, and the model.
% Given the dynamic, participatory nature of these templates, they can be thought of as shareable interactive art pieces.
The ease with which the templates can be shared also introduces social motivations for producing and distributing templates (e.g., to participate and contribute to a larger community of practice). We elaborate on these points below. }


%\rr{First, we observe that} p
Prompt templates have a number of key features:

\begin{itemize}
    \item They are tightly coupled to and represent a particular artistic concept, vision, or composition (such as a ``whimsical fantasy house''). These features make prompt templates conceptually richer than words or phrases used to specify stylistic characteristics of the image (e.g., ``35mm'' or ``watercolors'', which may be used to produce a particular effect, but don't specify a larger composition).
    \item Others can make use of prompt templates by filling in the blanks. The richness of the models means that the templates guide the overall generated image, but users' unique input can yield a diverse variety of outputs.
    \item There is the intent for the prompt template to provide consistently high quality and delightful output when used by others.
    %, even in the face of unknown inputs by the template user.
    % \item They can be thought of as instructions or source code for generating a particular concept.  
    % \item The prompts are \textit{reliable} and \textit{consistent} in their outputs.
\end{itemize}
% \begin{verbatim}
%     Audacious and whimsical fantasy house shaped like <object> with windows and doors, <location>. [Written by A2]
% \end{verbatim}

Unpacking these concepts in the context of the example prompt above, the skeleton of this prompt embodies a particular (visual) concept: A fantasy home in a given location. Someone making use of this prompt can customize it through two key variables: A shape for the house and a location for the house. While these are seemingly simple variables to customize, the template gives the user great flexibility in terms of the final outputs produced by the model: Any number of shapes can be provided in any number of locations (in fact, the user is free to substitute any text in the slots they wish).

Simultaneously, the prompt cues the model to the \textit{types} of output to produce, as well as details to guide the generation. The phrase ``audacious and whimsical fantasy house'' defines desired attributes of the house, while the specification of ``with windows and doors'' provides additional details that should be included in the generated house. These details in the prompt help increase the reliability of the model output and reduce the likelihood that the downstream user of the prompt template needs to experiment further with the overall prompt structure and content to obtain a good output. %(In a single test of the effectiveness of these additional details, we found the outputs of the model were improved and more consistent when these phrases were present than when they were absent.)

A noteworthy characteristic of these templates and the text-to-image models is the rich interplay that results between the original template and the phrases the user substitutes in the template. For example, a particular choice of location can profoundly influence the resulting house generated by the model---the model does not simply generate the same type of house and situate it in a different location. Instead, the choice of location can directly interact with the other parts of the prompt. For example, creating a strawberry house in the ``countryside'', ``Paris'', or ``Tokyo'' yields qualitatively different outputs for the house, with the house style meshing more naturally in the chosen location (e.g., having styles of windows more typical for a European building for the house in Paris, and doors more typical of Japan for Tokyo---see Figure \ref{fig:house_template}). These interactions between the template and the users' choices enable a diverse variety of outputs that allow a user to explore a wide range of ideas.

To produce prompt templates, one participant described a process whereby they would 1) input a prompt, 2) identify high-quality outcomes, and 3) rewrite the prompt to try to produce that same outcome again. This process could involve several iterations until they get a reliable prompt. Once a prompt is working, participants would sometimes remove content to make it more succinct. For example, they would first emphasize a specific characteristic (like ``high resolution'') by applying many descriptors representing a similar effect (such as ``high res'', ``DSLR'', ``crystal clear'', ``photo realistic''), then start removing the repeated qualifiers in the prompt until the prompt could reliably generate similar outputs to the original, verbose version. % [TODO: Is this true for the prompt templates, or only for prompts themselves?]

As with the notion of a ``prompt as art,'' the creation of these prompt templates can also be considered an artistic outcome in and of itself: The prompt author must first develop a compelling concept, then ensure it has enough capacity to enable people to produce their own unique creations within the frame of the concept. When done well, a prompt template has the quality of attracting users (because of the compelling output produced by the prompt) \textit{and} continually delighting users with the outputs produced using their unique input.

\rr{Elaborating further on this notion of ``prompt template as art,'' a prompt template represents a particular \textit{artistic vision}, with the prompt text capturing the overall design, composition, and aesthetic intentions of the author. Notably, because the artistic vision is captured in natural language text, a prompt template can be used \textit{across} TTI models: Users can customize the template as desired, then tweak the prompt to produce the desired output using whatever model they have at their disposal.}


\subsection{Discovering Unique Points in the Model's Latent Landscape}
A common goal of many artists was to discover new capabilities of the model that others had not yet found. As one participant put it, they tried to ``break'' the model, pushing it to its limits. As with prompt templates, these newly discovered capabilities are often shared so others can apply the concept in their own images. For example, an artist may identify the ability of the model to produce physical sculptures out of unusual materials or to create origami-like figurines (as A2 did). Once a concept and ability have been identified, others can build on the core idea and create their own images or permutations. \rr{In particular, participants felt proud when they ``made the model do X (new concept)'', especially when they discovered a model capability for the first time in the community in which they heavily engage, as the discovery could influence the discourse in the community. %Being able to say they had model respond well to their intent, there's a balance in the human-AI autonomy the participants found to work well. 
We expand on these points below.}

new words to steer the model in new directions. For example, one artist mentioned referencing architectural blogs to learn that domain's vocabulary so it could be applied to their prompts (A2). Another artist mentioned turning to a thesaurus to enrich their vocabulary for the prompt. As one concrete example, A1 (the ``Explorer'') likes to employ ``unusual words'' to steer the model, such as ``intricate'' instead of ``detailed'', explaining: ``If you choose common words, then you get a bit of an uninspiring result quite often. But if you use something a bit more unusual, then you really narrow down [...] the set [of images], and you're going to get into things that use this less common word'' (A1).

What's noteworthy here is that in the pursuit of novel imagery, creators would carefully research and choose \textit{words} to produce specific \textit{visual} outcomes: \textit{What you say and how you say it is critical to producing high-quality imagery with text-to-image models}, requiring TTI users to \textit{enrich their natural language vocabulary} in order to develop and skillfully execute a unique \textit{visual language}.

Driving these practices of discovering new capabilities was a clear desire to push the model away from ``average'' outputs and get it into more unique spaces. In this sense, the artists are navigating and charting the vast latent space of the model and sharing back the most interesting places discovered. When a new, unique output space was discovered, artists expressed a certain satisfaction in having discovered that space: `` A good rule of thumb is to be more descriptive than not [...] if you see that something is lacking, then you try to add more descriptors that will encourage the model, in this direction [...] sometimes you just have to reword a sentence or move something from one sentence to another [...] it's mostly like binary search'' (A1); `` I think your choice of vocabulary is very interesting because you want a large tree. But then, instead of saying a large tree you say mature tree [...] Where do these [...] choices come from? They just come from trial and error'' (A2).

In seeking unique spaces, one participant (P8) expressly hunted for interesting imperfections; instead of \textit{avoiding} glitches, they \textit{sought} glitches. For example, this participant found one of their prompts created imperfections in its output for ``a hybrid of a clock and a snail on an infinite mirror. Steampunk. DSLR photo. astrophotography'' (Figure \ref{fig:breaking_the_model}). Here, A4 explores the imperfections of the infinite mirror: it's ``technically a failure but still amazing'' (P8). P8 also discovered that the model sometimes has trouble generating the backs of things (like cats) and developed that behavior into its own art style (Figure \ref{fig:breaking_the_model}).

\begin{figure}[t]
    % \begin{subfigure}[b]{\textwidth}
        \hspace{\fill} % note: no blank line here
        \includegraphics[width=0.26\linewidth]{\imagepath{P1.png}}
        \hspace{\fill} % note: no blank line here
        \includegraphics[height=0.26\linewidth]{\imagepath{single_cat_back.jpg}}
        \hspace{\fill} % note: no blank line here
        \caption{Artwork by Paul Emmerich(P8). Examples of ``breaking the model'': the artist seeks out and elevates ``glitches'' in the model's output to create new styles. The artist takes advantage of the model's imperfect reflections (on the left) while making creative use of its tendency to transform the backs of cats into cat faces (see the bottom part of the cat's back, which looks like a cat's face).} 
        \label{fig:breaking_the_model}
    % \end{subfigure}
\end{figure}
\subsection{Validating Originality}
% Originality of concept, but can only check originality of artifact. See themselves as explorers. Made the model do X

One question that often arises on the topic of human-in-the-loop, AI-generated art is one of creativity and originality---how much can be attributed to the AI versus the person~\cite{hertzmann2020ComputersArt}. Our participants also struggled with this question, with some seeking to ensure their outputs were novel. \rr{Participants wanted to validate the originality of the artistic concept, but could only check the originality of the artifact.} For example, one participant described a practice of using Google's image search on compelling outputs to ensure originality: After producing a creative output, they use image search to search for that image (or similar images) to ensure it is, in fact, original. This participant mentioned they do this in part because they are sensitized to the fact that machine learning models can sometimes memorize portions of the training data. Given this, they want to ensure their output is \textit{not} in the training data.
