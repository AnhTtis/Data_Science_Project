%%
%% This is file `sample-manuscript.tex',
%% generated with the docstrip utility.
%%
%% The original source files were:
%%
%% samples.dtx  (with options: `manuscript')
%% 
%% IMPORTANT NOTICE:
%% 
%% For the copyright see the source file.
%% 
%% Any modified versions of this file must be renamed
%% with new filenames distinct from sample-manuscript.tex.
%% 
%% For distribution of the original source see the terms
%% for copying and modification in the file samples.dtx.
%% 
%% This generated file may be distributed as long as the
%% original source files, as listed above, are part of the
%% same distribution. (The sources need not necessarily be
%% in the same archive or directory.)
%%
%%
%% Commands for TeXCount
%TC:macro \cite [option:text,text]
%TC:macro \citep [option:text,text]
%TC:macro \citet [option:text,text]
%TC:envir table 0 1
%TC:envir table* 0 1
%TC:envir tabular [ignore] word
%TC:envir displaymath 0 word
%TC:envir math 0 word
%TC:envir comment 0 0
%%
%%
%% The first command in your LaTeX source must be the \documentclass command.
% \documentclass[manuscript,review,anonymous]{acmart}
\documentclass[manuscript,screen,acmlarge,authorversion]{acmart}

%%
%% \BibTeX command to typeset BibTeX logo in the docs
\AtBeginDocument{%
  \providecommand\BibTeX{{%
    Bib\TeX}}}

%% Rights management information.  This information is sent to you
%% when you complete the rights form.  These commands have SAMPLE
%% values in them; it is your responsibility as an author to replace
%% the commands and values with those provided to you when you
%% complete the rights form.
\setcopyright{acmcopyright}
\copyrightyear{2018}
\acmYear{2018}
\acmDOI{XXXXXXX.XXXXXXX}

%% These commands are for a PROCEEDINGS abstract or paper.
\acmConference[Conference acronym 'XX]{Make sure to enter the correct
  conference title from your rights confirmation emai}{June 03--05,
  2018}{Woodstock, NY}
\acmPrice{15.00}
\acmISBN{978-1-4503-XXXX-X/18/06}


%%
%% Submission ID.
%% Use this when submitting an article to a sponsored event. You'll
%% receive a unique submission ID from the organizers
%% of the event, and this ID should be used as the parameter to this command.
%%\acmSubmissionID{123-A56-BU3}

%%
%% For managing citations, it is recommended to use bibliography
%% files in BibTeX format.
%%
%% You can then either use BibTeX with the ACM-Reference-Format style,
%% or BibLaTeX with the acmnumeric or acmauthoryear sytles, that include
%% support for advanced citation of software artefact from the
%% biblatex-software package, also separately available on CTAN.
%%
%% Look at the sample-*-biblatex.tex files for templates showcasing
%% the biblatex styles.
%%

%%
%% The majority of ACM publications use numbered citations and
%% references.  The command \citestyle{authoryear} switches to the
%% "author year" style.
%%
%% If you are preparing content for an event
%% sponsored by ACM SIGGRAPH, you must use the "author year" style of
%% citations and references.
%% Uncommenting
%% the next command will enable that style.
%%\citestyle{acmauthoryear}


\usepackage{soul}
\usepackage{subcaption}
\usepackage{xcolor,colortbl}

% Command that generates the path of a given image. This allows for easily switching all images in the paper between cropped and uncropped versions.
\newcommand{\imagepath}[1]{cropped-figures/#1}
\newcommand{\rr}[1]{\textcolor{black}{#1}}

\usepackage{soul}
\usepackage{xcolor}

%% Revision highlight formatting
\definecolor{highlighter}{HTML}{fff100}
\sethlcolor{highlighter}
\newcommand{\revision}[1]{\hl{#1}} % with highlight
% \newcommand{\revision}[1]{#1}    % without

%%
%% end of the preamble, start of the body of the document source.
\begin{document}

%%
%% The "title" command has an optional parameter,
%% allowing the author to define a "short title" to be used in page headers.
% \title{What People Are Doing With Text to Image Models}
\title{The Prompt Artists} % and the Art of Controlled Chance}

%%
%% The "author" command and its associated commands are used to define
%% the authors and their affiliations.
%% Of note is the shared affiliation of the first two authors, and the
%% "authornote" and "authornotemark" commands
%% used to denote shared contribution to the research.

\author{Minsuk Chang}
\affiliation{%
  \institution{Google, Inc.}
  \city{Seattle, Washington}
  \country{United States}}
\email{misukchang@google.com}

\author{Stefania Druga}
\affiliation{%
  \institution{Google, Inc.}
  \city{Mountain View, California}
  \country{United States}}
\email{druga@google.com}

\author{Alex Fiannaca}
\affiliation{%
  \institution{Google, Inc.}
  \city{Seattle, Washington}
  \country{United States}}
\email{afiannaca@google.com}

\author{Pedro Vergani}
\affiliation{%
  \institution{Google, Inc.}
  \city{London}
  \country{United Kingdom}}
\email{pvergani@google.com}

\author{Chinmay Kulkarni}
\affiliation{%
  \institution{Google, Inc.}
  \city{Atlanta, Georgia}
  \country{United States}}
\email{ckulkarni@google.com}

\author{Carrie Cai}
\affiliation{%
  \institution{Google, Inc.}
  \city{Mountain View, California}
  \country{United States}}
\email{druga@google.com}

\author{Michael Terry}
\affiliation{%
  \institution{Google, Inc.}
  \city{Seattle, Washington}
  \country{United States}}
\email{michaelterry@google.com}



%%
%% By default, the full list of authors will be used in the page
%% headers. Often, this list is too long, and will overlap
%% other information printed in the page headers. This command allows
%% the author to define a more concise list
%% of authors' names for this purpose.
\renewcommand{\shortauthors}{Chang et al.}

%%
%% The abstract is a short summary of the work to be presented in the
%% article.


Over the past few years, there has been a significant amount of research focused on studying the ReLU activation function, with the aim of achieving neural network convergence through over-parametrization. However, recent developments in the field of Large Language Models (LLMs) have sparked interest in the use of exponential activation functions, specifically in the attention mechanism.

Mathematically, we define the neural function $F: \R^{d \times m} \times  \mathbb{R}^d \rightarrow \mathbb{R}$ using an exponential activation function. Given a set of data points with labels $\{(x_1, y_1), (x_2, y_2), \dots, (x_n, y_n)\} \subset \mathbb{R}^d \times \mathbb{R}$ where $n$ denotes the number of the data. Here $F(W(t),x)$ can be expressed as $F(W(t),x) := \sum_{r=1}^m a_r \exp(\langle w_r, x \rangle)$, where $m$ represents the number of neurons, and $w_r(t)$ are weights at time $t$. It's standard in literature that $a_r$ are the fixed weights and it's never changed during the training. We initialize the weights $W(0) \in \mathbb{R}^{d \times m}$ with random Gaussian distributions, such that $w_r(0) \sim \mathcal{N}(0, I_d)$ and initialize $a_r$ from random sign distribution for each $r \in [m]$.

Using the gradient descent algorithm, we can find a weight $W(T)$ such that $\| F(W(T), X) - y \|_2 \leq \epsilon$ holds with probability $1-\delta$, where $\epsilon \in (0,0.1)$ and $m = \Omega(n^{2+o(1)}\log(n/\delta))$. To optimize the over-parametrization bound $m$, we employ several tight analysis techniques from previous studies [Song and Yang arXiv 2019, Munteanu, Omlor, Song and Woodruff ICML 2022]. 

 


%%
%% The code below is generated by the tool at http://dl.acm.org/ccs.cfm.
%% Please copy and paste the code instead of the example below.
%%
\begin{CCSXML}
<ccs2012>
   <concept>
       <concept_id>10003120.10003121.10011748</concept_id>
       <concept_desc>Human-centered computing~Empirical studies in HCI</concept_desc>
       <concept_significance>500</concept_significance>
       </concept>
 </ccs2012>
\end{CCSXML}

\ccsdesc[500]{Human-centered computing~Empirical studies in HCI}
%%
%% Keywords. The author(s) should pick words that accurately describe
%% the work being presented. Separate the keywords with commas.
\keywords{AI art, Artists using AI, Text-to-Image models}

\begin{teaserfigure}
\includegraphics[width=\textwidth]{\imagepath{OverviewFigure.v6.png}}
\caption{Prompt artists develop descriptive text-based prompts that are rendered by text-to-image models. Highly skilled prompt artists will develop 1) distinct visual concepts and styles (S1), 2) prompts that can also serve as titles of the art piece (``prompts as art'', A1), and 3) ``prompt templates'' (A2), which encapsulate specific visual concepts to be customized by others. Artists strive to discover unique \textit{natural language} that produces unique \textit{visual outputs} (G1), and/or model ``glitches'' (G2) that can be elevated to artistic styles in their own right. Finally, some prompt artists validate the novelty of their work by conducting an image search for similar images (C1).}
\label{fig:teaser}
\end{teaserfigure}

%%
%% This command processes the author and affiliation and title
%% information and builds the first part of the formatted document.
\maketitle

% \input{sections/chinmay_intro}
Training data is a key driver of model behavior in modern machine learning systems.
Indeed, model errors, biases, and capabilities can all stem from the training data \citep{ilyas2019adversarial,gu2017badnets,geirhos2019imagenet}.
Furthermore, improving the quality of training data generally improves the performance
of the resulting models \citep{huh2016makes,lee2022deduplicating}.
The importance of training data to model behavior has motivated extensive work
on {\em data attribution}, i.e., the task of tracing model predictions back to the
training examples that informed these predictions.
Recent work
demonstrates, in particular, the utility of data attribution methods in applications such as
explaining predictions \citep{koh2017understanding,ilyas2022datamodels},
debugging model behavior \citep{kong2022resolving,shah2022modeldiff},
assigning data valuations \citep{ghorbani2019data,jia2019towards},
detecting poisoned or mislabeled data \citep{lin2022measuring,hammoudeh2022identifying},
and curating data \citep{khanna2019interpreting,liu2021influence,jia2021scalability}.


However, a recurring tradeoff in the space of data attribution methods is that of
{\em computational demand} versus {\em efficacy}.
On the one hand, methods such as
influence approximation \citep{koh2017understanding, schioppa2022scaling}
or gradient agreement scoring \citep{pruthi2020estimating}
are computationally attractive
but can be unreliable
in non-convex settings
\citep{basu2021influence,ilyas2022datamodels,akyurek2022towards}.
On the other hand, sampling-based methods such as empirical influence
functions \citep{feldman2020neural}, Shapley value estimators \citep{ghorbani2019data,jia2019towards} or datamodels \citep{ilyas2022datamodels} are
more successful at accurately attributing predictions to training data
but require training thousands (or tens of thousands) of models
to be effective.
We thus ask:
\begin{center}
    {\em Are there data attribution methods that are both scalable and effective in large-scale non-convex settings?}
\end{center}

\begin{figure}[!htb]
    \centering
    \begin{figure*}[t]
  \centering
  %\fbox{\rule{0pt}{2in} \rule{0.9\linewidth}{0pt}}
  \includegraphics[width=0.9\linewidth]{figures/simsim_vFINAL.png}

  \caption{\textbf{Overview of AdaSim.} Given an input image $\im_i$, we obtain two latent representations $\glob_i = f(t(\im_i))$ and $\glob_i^{'} = f'(t'(\im_i))$. Additionally, we sample another image $\im_{j^\star}$ in the dataset from $p^{win}(\im_j|\im_i)$ (see \cref{eq:similarity_distribution_windowed} and \cref{eq:p_im_dataset_windowed}) and obtain its latent representation $\glob_{j^\star}^{'} = f'(t'(\im_{j^\star}))$. We adaptively enforce a self-distillation loss $L$ between $\glob_i$ and $\glob_i^{'}$ or between $\glob_i$ and $\glob_{j^\star}^{'}$ here illustrated with a switch. In practice, only one of $\glob_i^{'}$ or $\glob_{j^\star}^{'}$ will be computed, see \cref{sec:adaptive_similarity_bootstrapping} and \cref{alg:adasim} for more details.}
  \label{fig:main_figure}
\end{figure*}
\caption{Our data attribution method \trak achieves state-of-the-art tradeoffs between
speed and efficacy.
Here, we benchmark its performance relative to prior methods on
\cifarten-trained ResNet-9 models and \qnli-trained \bertbase models. The $x$-axis indicates
the time (in minutes) it takes to run each method on a single A100 GPU
(see \Cref{app:wall_time} for details).
The $y$-axis indicates the method's efficacy
as measured by its ability to make accurate counterfactual predictions
(see \Cref{def:attr_output} for the precise metric);
error bars indicate 95\% bootstrap confidence intervals.
}
\label{fig:headline}
\end{figure}


To properly answer this question,
we first need a unifying metric for evaluating data attribution methods.
To this end, we adopt the view that a data attribution method is useful
insofar as it can make accurate {\em counterfactual predictions}, i.e.,
answer questions of the form
``what would happen if I trained the model on a given subset $S'$ of my training set?''
This perspective motivates a benchmark---inspired by the datamodeling framework
\citep{ilyas2022datamodels}---that
measures the correlation between true model outputs
and attribution-derived predictions for those outputs.


With this benchmark in hand, in \cref{sec:method} we consider our motivating question and introduce \trak
(\spelledout),
a new data attribution method for parametric, differentiable models.
The key idea behind \trak is to first approximate models with a kernel machine
(e.g., through the empirical neural tangent kernel \citep{jacot2018neural})
and then to leverage our understanding of the resulting kernel domain to derive data attribution scores.

We demonstrate that %
\trak retains the efficacy of sampling-based
attribution methods while being several orders of magnitude cheaper computationally.
For example (Figure \ref{fig:headline}), on \textsc{CIFAR-10} (image classification)
and \textsc{QNLI} (natural language inference),
 \trak can be as
effective as datamodels \citep{ilyas2022datamodels} while being
100-1000x faster to compute.
Furthermore, \trak is as fast as existing gradient-based methods such as
TracIn \citep{pruthi2020estimating} or variations of influence functions \citep{koh2017understanding,schioppa2022scaling},
while being significantly more predictive of model behavior.

As a result, \trak enables us to study the connection between model predictions
and training data in large-scale settings.
For example,
we use \trak to study predictions of ImageNet classifiers (\cref{sec:eval});
to understand the shared image-text embedding space of \clip models \citep{radford2021learning} trained
on \mscoco \citep{lin2014microsoft} (\cref{sec:CLIP});
and to fact-trace language models
(a 300M-parameter \texttt{mT5-small} model \cite{raffel2020exploring,xue2021mt5})
finetuned on \ftracetrex (\cref{subsec:fact_trace}).



\section{Related Work}
\label{sec:relatedwork}

%%%%%%%%%%%%%%%%%%%%%%%%%% Outline %%%%%%%%%%%%%%%%%%%%%%%%%%%%%%%%%%%%%
%(1) Evasion Attacks
%(1.1) Surveys on evasion attacks and their relation to data properties - Michael
%(1.2) Individual papers that study non-data related reasons behind evasion attacks - Michael
%(1.3) Techniques related to evasion attacks and defenses (new) - Gabby
%(2) Non-Evasion Attacks (new), and - ???
%(3) Effects of training data on standard generalization - done 
%
%
%
%(1) Evasion Attacks
%(1.1) A number of surveys review literature on evasion attacks. - Michael
%Most of them do not focus specifically on properties of data but also discuss attack and defense mechanisms, non-data-related reasons for adversarial vulnarability, and  more. ~\jr{cite 4}.
%Yet, they these surveys mention data and its relation to evasion attacks. Specifically \jr{what they say about data.}
%The most close to ours is concurrent work by XXX + concrete facts that we have and they don't.
%
%(1.2) individual papers that study non-data related reasons behind evasion attacks, - Michael
%Literature identifies multiple reasons for adversarial vulnerability, in particular, for evasion attacks. 
%These include data-related properties extensively discussed in this survey, as well as reasons related to the models 		   themselves, computations resources, and feature representations. We discuss these below. 
%
%\jr{the rest is from the paper (non-data related reasons for adversarial vulnerability), with sections potentially renamed.}
%
%{\bf Model.}
%
%{\bf Computational Resources.}
%
%{\bf Robustness of Features.}
%
%(1.3) Techniques Related to Evasion Attacks and Defenses (new) - Gabby
%A number of works focus on techniques for generating evasion attacks, countermeasures against these attacks, 
%and defining the notion of the attack itself.   
%
%{\bf Attacks and Defense.}
%Here are the 5 remaining surveys + 1 additional paper for the reviewer.
%
%{\bf Adversarial Examples.}
%2 surveys lines 13 and 14 + 1 additional paper for the reviewer.
%
%(2) Non-Evasion Attacks (new) 
%Need to say that there are other type of attacks, define them, cite surveys (Bo's survey, maybe something else). 
%Only one work explicitly focus on effects of data. 
%
%
%(3) Effects of training data on standard generalization (done)

%%%%%%%%%%%%%%%%%%%%%%%%% Outline %%%%%%%%%%%%%%%%%%%%%%%%%%%%%%%%%%%%%


\revreplace{
We divide related work into three categories:
(1) surveys on adversarial robustness and its relation to data properties,
(2) surveys that discuss the influence of data properties on standard generalization, and
(3) individual papers that study non-data-related reasons for adversarial vulnerability.\\
}
{
This survey investigates properties of training data in the context of model robustness under evasion attacks. 
We start the discussion of related work by reviewing other surveys that focus on evasion attacks and 
include some discussion about data (Section~\ref{sec:relatedwork-surveys-data}).  
We then discuss non-data related reasons behind evasion attacks (Section~\ref{sec:relatedwork-not-data}),
as well as techniques related to evasion attacks and defenses (Section~\ref{sec:relatedwork-attacks}). 
Finally, we discuss data-related concerns for non-evasion attacks (Section~\ref{sec:relatedwork-poisoning}) and
the effects of training data on standard generalization (Section~\ref{sec:relatedwork-standard}).
}

%\vspace{-0.1in}
\subsection{Surveys on Evasion Attacks that Discuss Data}
\label{sec:relatedwork-surveys-data}
Numerous existing surveys 
\revreplace{focus on attack and defense techniques for adversarial robustness. 
%~\cite{Biggio:Roli:PR:2018,
%Rosenberg:Shabtai:Elovici:Rokach:CSUR:2021,
%Li:Li:Ye:Xu:CSUR:2021,
%Maiorca:Biggio:Giorgio:CSUR:2019,
%Demetrio:Coull:Biggio:Lagorio:Armando:Roli:ACMTPS:2021,
%Liu:Tantithamthavorn:Li:Liu:CSUR:2022,
%Liu:Nogueria:Fernandes:Kantarci:IEEECST:2022,
%Akhtar:Mian:IEEEAccess:2018,
%Akhtar:Mian:Kardan:Shah:IEEEAccess:2021,
%Serban:Poll:Visser:CSUR:2020,
%Machado:Silva:Goldschmidt:CSUR:2021,
%Zhang:Sheng:Alhazmi:Li:ACMTIST:2020}.
Only a few of these works mention the relationship between adversarial robustness and properties of the underlying data.} 
{review the literature on evasion attacks.
Most of these works do not focus specifically on properties of data but discuss attack and defense mechanisms, non-data-related reasons for adversarial vulnerability, 
and the different threat models. 
Only a few of these works mention data-related reasons for the existence of adversarial examples~\cite{Serban:Poll:Visser:CSUR:2020, Machado:Silva:Goldschmidt:CSUR:2021, Akhtar:Mian:Kardan:Shah:IEEEAccess:2021, Akhtar:Mian:IEEEAccess:2018}.
}
Specifically, Serban et al.~\cite{Serban:Poll:Visser:CSUR:2020} observe that adversarial vulnerability can be caused by an insufficient training sample size %~\cite{Schmidt:Santurkar:Tsipras:Talwar:Madry:NeurIPS:2018}
and high data dimensionality. %~\cite{Gilmer:Metz:Faghri:Schoenholz:Raghu:Wattenberg:Goodfellow:ICLR:2018}.
Similarly, Machado et al.~\cite{Machado:Silva:Goldschmidt:CSUR:2021} mention that the lack of sufficient training data, high dimensionality, 
and high concentration contribute to adversarial vulnerability.
\revadd{
Akhtar et al.~\cite{Akhtar:Mian:IEEEAccess:2018, Akhtar:Mian:Kardan:Shah:IEEEAccess:2021} also mention high dimensionality, along with other non-data-related reasons, 
as a source of adversarial examples.}

\revadd{A concurrent work by Han et al.~\cite{Han:Lin:Shen:Wang:Guan:CSUR:2023} (published at the end of April 2023) 
studies the origins of adversarial vulnerability in deep learning w.r.t. the model, data, and other perspectives.
The authors mention high dimensionality, distributions with high concentration, a small number of output classes, data imbalance, and the perceptual difference in image frequencies as potential sources of adversarial examples.
However, as (a) the focus of that survey is not on data-related properties in particular, 
(b) its paper search was conducted in 2021, and 
(c) it focuses on deep learning models only, 
our work was able to identify more than 50 additional relevant papers which focus on other types of models, 
e.g., non-parametric and linear classifiers, 
and/or discuss additional types of data-related properties, 
such as, types of distribution, class density, separation, and label quality.}
\revreplace{Yet, none of these surveys explicitly collect and analyze work that focuses on the effects of data properties
on adversarial robustness.}
{In summary, by explicitly focusing on the effects of data properties on evasion attacks in our survey, 
we are able to provide a more complete and detailed discussion on this topic, not covered in prior surveys.}

\vspace{-0.05in}
\subsection{Non-data-related Reasons Behind Evasion Attacks}
\label{sec:relatedwork-not-data}

%\vspace{-0.1in}
%\subsection{Non-data Related Reasons for Adversarial Vulnerability}

There has been a variety of hypotheses regarding the reasons behind adversarial vulnerability of ML systems, particularly for evasion attacks.
%\revreplace{
%In addition to the data used for training,  adversarial robustness could also depend on the choice of the model architecture,
%the training procedure, and the interplay between data and the learning algorithm, i.e., correspondence between the complexity of a model to that of the data.
%This section summarizes the key hypotheses regarding these aspects.
%%The hypotheses reviewed in this section are complementary to the potential influence from the data.
%}
These include data-related properties extensively discussed in this survey, as well as reasons related to the models themselves, 
computational resources, and feature learning procedures. We discuss these below.

%\jr{there is a lot of undefined terminology and jargon in this section.}

\vspace{0.02in}
\noindent
\textbf{Model.}
When Szegedy et al.~\cite{Szegedy:Zaremba:Sutskever:Bruna:Erhan:Goodfellow:Fergus:ICLR:2014} first discovered adversarial examples for visual models, they suspected that the high non-linearity of DNNs resulted in low probability `pockets' of adversarial examples in the learned representation manifold.
They hypothesize that while these pockets can be found through attack algorithms, the samples residing in these pockets have different distributions compared to normal samples and are thus subsequently harder to find when randomly sampling from the input space.
Instead, Goodfellow et al.~\cite{Goodfellow:Shlens:Szegedy:ICLR:2015} hypothesize that
the linearity from activation functions, like ReLU and sigmoid found in high-dimensional neural networks, induce vulnerability towards adversarial perturbations.
To support their claim, they present the attack method FGSM that exploits the linearity of the target classifier.
Fawzi et al.~\cite{Fawzi:Fawzi:Frossard:ICMLWorkshop:2015} also argue against the hypothesis of high non-linearity as the cause for adversarial examples.
They show that all classifiers are susceptible to adversarial attacks and claim that it is the low flexibility of the classifier compared to the complexity of the classification task that results in vulnerability.
The lack of consensus on the primary causes of model vulnerability invites more studies on this topic.

Singla et al.~\cite{Singla:Ge:Basri:Jacobs:NeurIPS:2021} show that enforcing invariance to circular shifts (e.g., rotation) in neural networks induces decision boundaries with a smaller margin than normal, fully connected networks,
which, in turn, reduces the adversarial robustness of the model.
Moosavi{-}Dezfooli et al.~\cite{Moosavi-Dezfooli:Fawzi:Fawzi:Frossard:Soatto:ICLR:2018} introduce universal,
input-agnostic perturbations to mislead the classifier and hypothesize that the vulnerability of a multi-class classifier to such perturbations is related to the shape of its decision boundaries, e.g.,
linear classifiers with decision boundaries that are parallel to each other and
nonlinear classifier with decision boundaries that are curved in a similar way
tend to be less robust as
perturbations in one direction can change the prediction label for a different class.

Tanay and Griffin~\cite{Tanay:Griffin:ArXiv:2016} conjecture that the decision boundary learned by the classifier being too close to (or `tilted towards') the data manifold instead of being perpendicular to it,
results in small perturbations being sufficient to move samples across the decision boundary for misclassification.
%data manifold refers to the underlying structure that the data exhibit

\vspace{0.02in}
\noindent
\textbf{Computational Resources.}
Bubeck et al.~\cite{Bubeck:Lee:Price:Razenshteyn:ICML:2019} use computational hardness theory to show that the time complexity for learning a robust model is exponential to the size of input data and thus is computationally intractable.
Hence, they attribute adversarial vulnerability to computational limitations of current learning algorithms.
Degwekar et al.~\cite{Degwekar:Nakkiran:Vaikuntanathan:COLT:2019} further extend this work and also show the impossibility of efficiently training robust classifiers.

%\subsubsection{Ineffective Learning Perspective}
\vspace{0.02in}
\noindent
\textbf{Feature Learning.}
Ilyas et al.~\cite{Ilyas:Santurkar:Tsipras:Engstrom:Tran:Madry:NeurIPS:2019} show that adversarial vulnerability can be a consequence of a model exploiting well-generalizing but non-robust features,
i.e., features that are spurious and sometimes incomprehensible to humans;
when constraining the model to use robust features, the adversarial robustness increases together with the
interpretability of the learned features.
However, Tsipras et al.~\cite{Tsipras:Santurkar:Engstrom:Turner:Madry:ICLR:2019} note that, as the features for achieving high accuracy may be different from the ones for achieving high robustness, robustness may be at odds with standard accuracy.
%
%\jr{why is it called Ineffective learning when it is about features.}\gx{I put it under ineffective learning as in this case, the model learns/decides the features for generalization, and when given the correct objective, the model in fact, can learn more robust features, so I think the underlying reason is objective we gave for the model didn't guide the model to learn the right features}
%
Instead of seeing adversarial vulnerability as a product of classifiers being overly sensitive to changes in spurious features, Jacobsen et al.~\cite{Jacobsen:Behrmann:Zemel:Bethge:ICLR:2019} hypothesize that classifiers can rather be
overly insensitive to relevant semantic information, e.g., images with drastically different content can share similar latent representations.
The authors introduce a new type of adversarial examples that exploit such insensitivity, where the content of images is altered without changing the resulting prediction label.
%As both insensitivity to semantic content and sensitivity to spurious changes can simultaneously exist in models,
%more investigation into how to define proper objectives for models to effectively distinguish the relevant information is needed.

While all these works propose possible reasons for adversarial vulnerabilities, they are orthogonal to our survey, which focuses particularly on the influence of training data.

\vspace{-0.05in}
\revadd{
\subsection{Evasion Attacks and Defenses}
\label{sec:relatedwork-attacks}
A number of works focus on techniques for generating evasion attacks, countermeasures against these attacks, 
and defining the notion of the attack itself.

%\jr{need to include~\cite{Biggio:Roli:PR:2018,
%Rosenberg:Shabtai:Elovici:Rokach:CSUR:2021,
%Li:Li:Ye:Xu:CSUR:2021,
%Maiorca:Biggio:Giorgio:CSUR:2019,
%Demetrio:Coull:Biggio:Lagorio:Armando:Roli:ACMTPS:2021,
%Liu:Tantithamthavorn:Li:Liu:CSUR:2022,
%Liu:Nogueria:Fernandes:Kantarci:IEEECST:2022,
%Zhang:Sheng:Alhazmi:Li:ACMTIST:2020} x and one more survey.}
%\js{\cite{Biggio:Roli:PR:2018, Rosenberg:Shabtai:Elovici:Rokach:CSUR:2021} moved to Adversarial Examples.
%\cite{Rosenberg:Shabtai:Elovici:Rokach:CSUR:2021,
%Li:Li:Ye:Xu:CSUR:2021,
%Maiorca:Biggio:Giorgio:CSUR:2019, Liu:Tantithamthavorn:Li:Liu:CSUR:2022,
%Liu:Nogueria:Fernandes:Kantarci:IEEECST:2022,
%Zhang:Sheng:Alhazmi:Li:ACMTIST:2020, Demetrio:Coull:Biggio:Lagorio:Armando:Roli:ACMTPS:2021} in Attacks and Defense. \cite{Sun:Dou:Yang:Zhang:Wang:Philip:He:Li:TKDE:2022} was the "one more survey" and is also in Attacks and Defenses.}

\vspace{0.02in}
\noindent
{\bf Attacks and Defense.}
Several works~\cite{Liu:Tantithamthavorn:Li:Liu:CSUR:2022,Liu:Nogueria:Fernandes:Kantarci:IEEECST:2022,Sun:Dou:Yang:Zhang:Wang:Philip:He:Li:TKDE:2022, Demetrio:Coull:Biggio:Lagorio:Armando:Roli:ACMTPS:2021} survey adversarial attacks and defenses, observing that most work focuses on computer vision and NLP domains. 
Zhang et al.~\cite{Zhang:Sheng:Alhazmi:Li:ACMTIST:2020}, 
Rosenberg et al.~\cite{Rosenberg:Shabtai:Elovici:Rokach:CSUR:2021},
Li et al.~\cite{Li:Li:Ye:Xu:CSUR:2021}, and 
Maiorca et al.~\cite{Maiorca:Biggio:Giorgio:CSUR:2019}, 
survey attacks and defenses in the NLP domain, cybersecurity domain for networks, Android malware, and PDF malware, respectively. 
These works identify a similar trend of new attacks constantly bypassing defenses, which gives rise to new defenses being proposed, only to be broken again (a.k.a. the `cat and mouse race' or the `arms race'). 
They also observe that research in this field studies attacks / defenses at a feature-level, which restricts 
the practicality of the developed techniques by the feasibility of perturbing the corresponding features in real life. 

%practical attacks are quite difficult and require some basic knowledge about the model or training data such as the feature set or model architecture. 
%Zhang et al.~\cite{Zhang:Sheng:Alhazmi:Li:ACMTIST:2020}, who study adversarial attacks and defenses in the NLP domain,  
%also find that there are obstacles to generating attacks in real-time. 
%For instance, methods that iteratively use gradients to create adversarial examples can be time-consuming, while one-time approaches may fail to produce potent adversarial examples.
%Several works~\cite{Liu:Tantithamthavorn:Li:Liu:CSUR:2022,Liu:Nogueria:Fernandes:Kantarci:IEEECST:2022,Sun:Dou:Yang:Zhang:Wang:Philip:He:Li:TKDE:2022, Demetrio:Coull:Biggio:Lagorio:Armando:Roli:ACMTPS:2021} 
%discuss how most new attacks and defenses are explored in computer vision and NLP, prior to other fields.


%our survey finds the state of the art w.r.t. data properties
%our survey finds that dimensionality is bad ...
%
%%%Here are the 5 remaining surveys + 1 additional paper for the reviewer.
%Numerous surveys have explored the landscape of adversarial evasion attacks and defenses. 
%For instance, Akhtar et al.~\cite{Akhtar:Mian:IEEEAccess:2018, Akhtar:Mian:Kardan:Shah:IEEEAccess:2021} survey the literature on adversarial robustness of deep learning models from Computer Vision field.
%They review popular attacks on visual models, and provided a categorization of existing defense techniques based on the components it modify in the visual model system \gx{Check}.
%
%Rosenberg et al.~\cite{Rosenberg:Shabtai:Elovici:Rokach:ACMComputingSurvey:2021}, Li et al. ~\cite{Li:Li:Ye:Xu:ACMComputingSurvey:2021} and Demetrio et al.~\cite{Demetrio:Coull:Biggio:Lagorio:Armando:Roli:ACMTPS:2021} review the literature on evasion attacks for cyber-security fields. 
%Li et al. proposed a partial order scheme to compare key attacks and defenses techniques for malware detection in Windows, Android, and PDF domains. 
%
%Zhang et al.~\cite{Zhang:Sheng:Alhazmi:Li:ACMTIST:2020} review the literature on adversarial attacks on deep-learning models for textual classification.
%They pointed out the intrinsic differences between Computer Vision and Natural Language Processing fields that pose challenges to directly apply attacks proposed for Visual models to NLP models and identified the strategies proposed that overcomes the barriers.
%The challenges they identified for creating realistic attacks in NLP fields are from a domain characteristics perspective (e.g., definition of imperceptible perturbations, measurement of the semantic changes),  we differ from them by trying to understand the adversarial robustness of machine learning from the characteristics of underlying data. 
%
%Attack and Defenses for wireless and Mobile systems~\cite{Liu:Nogueria:Fernandes:Kantarci:IEEECST:2022}
%
%

More recent research, not included in the surveys above, has also started investigating the 
susceptibility of newer models to adversarial evasion attacks. 
For example, several studies~\cite{Wang:Pan:Hu:Duan:Pan:IJSWIS:2022,Yin:Lin:Sun:Wei:Chen:TIFS:2023, 
Shi:Han:Tan:Kuang:NeurIPS:2022, Wang:Xie:Microsoft:ChatGPT:ArXiv:2023} proposed attack techniques against contemporary models, 
such as Graph Neural Networks, Generative Pre-training Transformers (GPT), and Vision Transformers. 
These studies showed that adversarial examples persist even for the newer models, some of which are 
trained with large volumes of data. 
As all these works focus on attack and defense mechanisms rather than 
the effects of data on adversarial robustness, our work extends and complements this research.
}

\revadd{
\vspace{0.02in}
\noindent
{\bf Adversarial Examples.}
%2 surveys lines 13 and 14 + 1 additional paper for the reviewer.
Adversarial examples are inputs constructed by perturbing a correctly classified sample in a way that makes the change imperceptible to a human. % but causes the model to misclassify the sample.
However, as `imperceptible to a human' is hard to define, existing research on adversarial examples approximates imperceptibility with a small perturbation measured through $L_p$ norms.
A line of research~\cite{Gilmer:Adams:Goodfellow:Anderson:Dahl:ArXiv:2018,Sharif:Bauer:Reiter:CVPRW:2018,Fezza:Bakhti:Hamidouche:Deforges:QoMEX:2019, Mezher:Deng:Karam:EUVIP:2022} 
investigates the validity of this assumption. 
This work shows that perturbations generated by $L_p$ norms do not entirely align with human perceptions, 
i.e., some changes with a small $L_p$ norm can be apparent to humans. 
In addition, adversarial examples with the minimum $L_p$ perturbation may be less effective and transferable than 
higher perturbation~\cite{Biggio:Roli:PR:2018,Rosenberg:Shabtai:Elovici:Rokach:CSUR:2021}. 
Hence, a number of approaches explore metrics for imperceptibility 
in computer vision and NLP domains~\cite{Fezza:Bakhti:Hamidouche:Deforges:QoMEX:2019,Mezher:Deng:Karam:EUVIP:2022, Zhang:Sheng:Alhazmi:Li:ACMTIST:2020}. 
Yet another issue with $L_p$ norms is that they cannot be used reliably in domains other than images. 
For example, in the case of software/malware, simply generating adversarial examples with $L_p$ norms 
may result in feature representations that are not possible in 
the problem space~\cite{Rosenberg:Shabtai:Elovici:Rokach:CSUR:2021,Pierazzi:Pendlebury:Cortellazz:Cavallaro:2020}. 

While all these works focus on the properties of adversarial examples, 
they are orthogonal to the topic of our survey, as we rather focus on how properties of the training data 
affect the success of adversarial examples.
}

%Gilmer et al.~\cite{Gilmer:Adams:Goodfellow:Anderson:Dahl:ArXiv:2018} argue that, while constraining the perturbations by sufficiently small $L_p$ norms can generate indistinguishable samples for most inputs, the actual imperceptibility of the changes depends on the input sample. 
%Several individual studies~\cite{Sharif:Bauer:Reiter:CVPRW:2018,Fezza:Bakhti:Hamidouche:Deforges:QoMEX:2019, Mezher:Deng:Karam:EUVIP:2022} find faults with using $L_p$ norms to generate adversarial examples. They show that the changes measured by $L_p$ norm, does not entirely align with human perceptions, i.e., some changes with a small $L_p$ norm appear apparent to humans. 
%In some domains adversarial examples do not need to be imperceptible but rather semantically preserving. 
%For example, in the case of Android malware~\cite{Rosenberg:Shabtai:Elovici:Rokach:CSUR:2021}, adversarial examples are small perturbations which fool a model while preserving the semantics of the sample, 
%i.e., a malware stays malicious even after the perturbation. 
%This highlights another problem with $L_p$ norm based adversarial examples as Dong et al.~\cite{Dong:Liu:Shang:NeurIPS:2022} show that the semantics of a sample change during adversarial training. 
%Hence, there is a need for metrics to measure the size of perturbations that is imperceptible or semantically preserving.
%Fezza et al.~\cite{Fezza:Bakhti:Hamidouche:Deforges:QoMEX:2019} and Mezher et al.~\cite{Mezher:Deng:Karam:EUVIP:2022} propose to use objective metrics for image quality to approximate the imperceptibility in the computer vision domain.
%Zhang et al.~\cite{Zhang:Sheng:Alhazmi:Li:ACMTIST:2020}, focusing on providing such a metric for Natural Language Processing.
%Vadillo et al.~\cite{Vadillo:Santana:CS:2022} also highlight conducted subject studies to evaluate the noticeability of audio adversarial examples.

%Even in computer vision, adversarial examples are not always imperceptible. For example, Machado et al.~\cite{Machado:Silva:Goldschmidt:CSUR:2021} find that visible perturbations such as adversarial patch~\cite{Brown:Mane:Roy:Abadi:Gilmer:ArXiv:2017}, and graffiti on stop signs~\cite{Eykholt:Evtimov:Fernandes:Li:Rahmati:Xiao:Prakash:Kohno:Song:CVPR:2018} are also considered adversarial examples in research.

%The aforementioned research examines the work on defining and creating adversarial examples, demonstrating the insufficiency of using conventional $L_p$ norms to evaluate the imperceptibility and semantics between clean and adversarial examples. 

\vspace{-0.1in}
\revadd{
\subsection{Non-Evasion Attacks}
\label{sec:relatedwork-poisoning}
Similar to evasion attacks, data poisoning and backdoor attacks aim to compromise model accuracy. 
However, they achieve it by tampering the training data to create deceptive model decision boundaries. 
%Data poisoning attacks involve modifying the training data to create deceptive decision boundaries, either to manipulate the prediction outcomes of a specific input or the entire model.
%Meanwhile, Backdoor attacks are a form of poisoning attacks where the attacker inject tempered training data with triggers 
% and then activates the attack by showing the trigger pattern at inference time.
In addition, backdoor attacks also require perturbing the test instance to result in a misclassification. 
This is achieved by introducing manipulated training data with triggers that can be activated during the testing phase.

Goldblum et al.~\cite{Goldblum:Tsipras:Xie:Chen:Schwarzchild:song:Madry:Li:Goldstein:TPAMI:2022} and Cinà et al.~\cite{Cina:Grosse:Demontis:Sebastiano:Zellinger:Moser:Oprea:Biggio:Pelillo:Roli:CSUR:2023} 
review recent literature on attack methodologies and countermeasures for both poisoning and backdoor attacks.
Both of these surveys found that existing research made overly-optimistic assumptions when designing / validating attack techniques, e.g., assuming the knowledge of a large portion of training data. 
They advocate for researchers to test proposed methods in more realistic situations to better assess the potential threats. 
Furthermore, they encourage exploration of the relationship between poisoning attacks and evasion attacks. 
This could lead to the creation of attacks that produce less noticeable poisoning examples, 
or defensive strategies that can safeguard models against both backdoor and evasion attacks.
%Their survey catalogs and systematizes the threats in the dataset creation process, and discuss the open problems that benefits the understanding of dataset security. 

In addition to undermining model accuracy, 
adversarial attacks also aim at breaching the privacy and confidentiality of training data. 
In particular, membership inference attacks~\cite{Shokri:Stronati:Song:Shmatikov:SP:2017} attempt to determine whether a specific data point was part of the training set used to train the model.
Hu et al.~\cite{Hu:Salcic:Sun:Dobbie:Yu:Zhang:CSUR:2022} present a comprehensive survey of existing research efforts on membership inference attacks. 
They find that, similar to evasion attacks, the membership inference attack success rate decreases as 
%the training data better represents the whole data distribution, i.e., 
the number of training samples increases.
%and model stealing attacks~\cite{Oliynyk:Mayer:Rauber:CSUR:2023} are designed to breach the privacy of training data and machine learning models. 
However, all these attacks are orthogonal to our survey, as we focus on adversarial evasion attacks.

%Li et al. ~\cite{Li:Jiang:Li:Xia:TNNLS:2022} 
%provide the first survey that focuses on backdoor attacks and identified common scenarios in which backdoor attack happen in real life. 
%Furthermore, they proposed a systematic taxonomy for backdoor attacks and defenses for researchers and practitioners to identify the characteristics and limitations of each method. 

%Wang et al.~\cite{Wang:Ma:Wang:Hu:Qin:Ren:CSUR:2022} and Tian et al.~\cite{Tian:Cui:Liang:Yu:CSUR:2022} argue federated learning~\cite{McMahan:Moore:Ramage:Hampson:Arcas:AISTATS:2017} 
%creates new venue for poisoning attack, and survey recent literature on poisoning attacks for both standard and federated learning scenarios. 
%They present a unified framework to categorize both data poisoning and model poisoning attacks, and compared the defense techniques proposed for each of the learning framework, analyzed their advantages and disadvantages.
}

\vspace{-0.1in}
\subsection{Effects of Training Data on Standard Generalization}
\label{sec:relatedwork-standard}
A number of surveys investigate the influence of data properties on standard
rather than robust generalization.
One of the earliest is probably the work of Raudys and Jain~\cite{Raudys:Jain:TPAMI:1991},
who review studies related to the influence of sample size on binary classifiers, showing that
a limited sample size usually leads to sub-optimal generalization.
%With the development of deep learning and the ever-increasing need for larger training datasets,
%a variety of data augmentation techniques have been proposed.
Bansal et al.~\cite{Bansal:Sharma:Kathuria:CSUR:2021} and
Bayer et al.~\cite{Bayer:Kaufhold:Reuter:CSUR:2022} also survey papers addressing the data scarcity problem,
focusing in particular on the recent advancements in data augmentation techniques in the fields of computer vision, security, and text classification.
Their results show that augmentation techniques %exist for various application domain and
can help improve a model's generalization by reducing the problem of model overfitting.
%They evaluate the effectiveness of such techniques in improving the accuracy of machine learning models.

%Limited sample size is also one of the culprit behind poor robust generalization~\cite{Schmidt:Santurkar:Tsipras:Talwar:Madry:NeurIPS:2018}, we collected a number of researches characterize the sample complexity for robust generalization or propose data augmentation techniques to fill in the sample complexity gap.

Label noise is another aspect of data that influences both standard and robust generalization.
Most works on this topic find that the presence of noisy labels increases the need for a greater number of training samples and may result in unnecessarily complex decision boundaries~\cite{Frenay:Verleysen:TNNLS:2014,Song:Kim:Park:Shin:Lee:TNNLS:2022}.
For example, Fr\'{e}nay and Verleysen~\cite{Frenay:Verleysen:TNNLS:2014} show
that overfitting to label noise greatly degrades a model's standard generalization;
the same effect has been observed in the case of robust generalization~\cite{Sanyal:Dokania:Kanade:Torr:ICLR:2021}.
Song et al.~\cite{Song:Kim:Park:Shin:Lee:TNNLS:2022} survey the impact of label noise in deep learning, arguing
that the presence of noisy labels is a more serious concern for deep models as they contain a larger number of parameters which makes them prone to overfitting to the noise in training data.
%They also point out the connection between adversarial poisoning attacks and noisy labels as
%the countermeasures for both share the goal of learning noise-resilient representations.
They mention that adversarial defense techniques, e.g., adversarial training, are effective against label noise~\cite{Zhu:Zhang:Han:Liu:Niu:Yang:Kankanhalli:Sugiyama:ArXiv:2021, Fatras:Damodaran:Lobry:Flamary:Tuia:Courty:TPAMI:2022}
but do not discuss how label noise influences a deep learning model's robustness under attacks.

Lorena et al.~\cite{Lorena:Garcia:Lehmann:Souto:Ho:CSUR:2020} identify a collection of 26 quantitative metrics that measure data complexity with respect to
(1) ambiguity of classes, i.e., whether the classes can be clearly distinguished with the given features,
(2) sparsity and dimensionality of data, 
%i.e., whether enough information are provided to learn confident decision boundaries, and
(3) complexity of boundary separating the classes, i.e., whether more intricate functions are required to describe the decision boundaries.
The authors also discuss how these metrics help estimate the difficulty of performing classification on a given dataset.
Similar to our survey, the authors show that high dimensionality and small separation between classes hinder standard generalization.
However, the relationship of some of the metrics reviewed by these authors, e.g.,
%faction of borderline points (i.e., a measure for the complexity of the required decision boundary) and
%the fraction of hyperspheres covering data (i.e.,
the number of non-intersecting spheres needed to enclose all data points of a class,
to robust generalization is not studied, according to our survey.

%Moreover, the effect of XXX on standard generalization needs future investigation as well (that is if we found something they do not have).

%Knowing the characteristics of a dataset according to these perspectives can assist researchers and practitioners to select optimal learning algorithms~\cite{Ho:Basu:TPAMI:2002}.

He and Garcia~\cite{He:Garcia:TKDE:2009} focus on the imbalance learning problem. %~--
%the disproportion in the number of samples belonging to each class in a given dataset.
The authors found that most standard algorithms %are designed with the assumption of a balanced class distribution.
%These algorithms
fail to reliably represent the characteristics of the imbalanced data and result in unfavorable performance across classes.
Furthermore, L\'{o}pez et al.~\cite{Lopez:Fernandez:Garcia:Palade:Herrera:InfSci:2013} discuss six intrinsic data characteristics that potentially complicate learning from imbalanced data:
low density, sample overlap between classes, noisy data, borderline instances,
dataset shift between training and testing distributions, and
small disjuncts, i.e., disperse small clusters of samples from a single class.
Their analysis concludes that while all these ``unfavorable'' data characteristics further complicate the data imbalance
issues, data overlap between classes is probably one of the most harmful.
To follow up on this point, Santos et al.~\cite{Santos:Henriques:Pedro:Japkowicz:Fernandez:Soares:Wilk:Santos:AIR:2022}
focus on the joint effect of data imbalance and class overlap on model generalization.
The negative impact of data imbalance, low separation, and noisy data on robust generalization was also discussed in our survey.
Yet, the compounding effect of these factors, as well as the effect of other properties,
on robust generalization needs future investigation.

Recently, Yang et al.~\cite{Yang:Jiang:Song:Guo:IJCV:2022} summarized relevant studies focusing on
long-tailed distributions in the field of Computer Vision.
% and categorize the main methods for alleviating the issues caused by long-tailed distribution.
%They present quantitative metrics for measuring data imbalance and .
This survey also includes work on the influence of long-tail distributions on a model's adversarial robustness~\cite{Wu:Liu:Huang:Wang:Lin:CVPR:2021}, which is covered in our survey.
%which is included in our survey,
The authors advocate for more research on adapting long-tailed-based approaches for standard generalization to improve robust generalization.

Finally, Moreno-Torres et al.~\cite{MorenoTorres:Raeder:Rodrigues:Chawla:Herrera:PR:2012} present a unifying framework to categorize existing definitions of dataset shift~-- the case where the joint distribution of inputs and outputs differs between training and testing data.
While ML models are normally trained under the premise that testing data has a similar distribution to the training data,
in reality, the observed data distribution may be different from the historical data that the model is trained on.
Such difference can substantially compromise the quality of model predictions.
The authors analyze the possible causes for dataset shift, e.g., malicious software that evolves over time, and
review the techniques dealing with dataset shift.
They characterize adversarial attacks as one form of dataset shift, where adversaries adaptively
change test instances to create a distribution that differs from training data.
%All works discussed in our survey assumed similar distribution on training and testing data, treating adversarial attacks as the only dataset shift in the problem setup.
%However, in real applications, the underlying data distribution itself can be non-stationary, and the characterize the influence of the dataset shift between training and testing data on the adversarial robustness is yet to be investigated.

\revadd{Overall, despite the similarities with our work, literature discussed in this section focuses on standard generalization while our survey discusses 
the effect of data on robust generalization.}

%More works use the connection between adversarial attacks and distributional shift to analyze the effect of adversaries on generalization performance~\cite{Tu:Zhang:Tao:NeurIPS:2019}.
%However, we do not discuss them in detail, as they focus more on models instead of data.
%\jr{How is that relevant to data properties section?} \gx{This can be removed, as it an individual work we filtered}

\vspace{-0.1in}
\subsection{Summary}
\revadd{
Our survey is the first to explicitly focus on properties of training data in the context of model robustness under evasion attacks.
Numerous other surveys on evasion attacks discuss attack and defense mechanisms, non-data-related reasons for adversarial vulnerability, and the different threat models. 
We identified only five surveys that considered data-related reasons for evasion attacks. 
However, as these surveys are older and do not focus on data in particular, our work provides a more extensive
and comprehensive view on this topic. 
By including more than 50 papers not covered in prior work, we were able to 
identify additional relevant properties, practical suggestions, and future research directions in this area. 

Additional work studies non-data-related reasons for evasion attacks, as well as non-evasion attacks, 
such as poisoning and backdoor. 
Yet another body of literature examines how data properties affect standard generalization. These works show that 
some of the properties discussed in our survey, such as 
the number of samples, dimensionality, and label quality, also affect clean accuracy. 
There are also additional data properties that are covered exclusively by these or by our work. 
Studying the interplay between data properties for clean and robust accuracy is an interesting research direction, 
which could be facilitated by our work. 
However, all these current works are orthogonal and complementary to ours.
}

%\ad{
%The related work of our survey can be categorized into four key topics: 
%The first topic examines data for other adversarial attacks, this include the research that investigates the link between the data characteristics and model's resilience against poisoning attacks as well as the studies that explore data poisoning and backdoor attacks and their countermeasures. \jr{same issues as before: this is meta-summary, we need a concrete summary.}
%These studies complement our survey as they highlight the threats directly aimed at data, thus emphasizing the importance of secure data collection. 
%The second topic focuses on the relationship between various properties of training data and model's standard generalization ability. 
%This body of work suggests that data traits such as number of samples, dimensionality, label quality also influence model's ability to generalize in standard classification. \jr{this looks more concrete!}
%
%The third strand of research concerns adversarial evasion attacks. 
%The work in this area encompasses the research frontier in evasion attacks and the countermeasures. 
%Due to the large volume of work in this area, there are numerous surveys that gives more detail on the advancement. 
%\jr{meta-summary again}
%In addition to attacks and defenses, one relevant line of work investigates the alignment of the conventional similarity metrics used for adversarial examples and human perception, showing the need for supplementary metrics. \jr{why important?}
%These studies \jr{which "these studies"?} collectively present an extensive overview of other types of work conducted on adversarial robustness.
%The last category of work proposes alternative explanations for model vulnerability to adversarial examples.
%These studies presented hypothesis showing the characteristics of machine learning models, e.g., nonlinearity, invariance to rotational shift etc, induces susceptibility to attacks, as well as limited computational resources and non-robust feature representations. \jr{all text based on previous related work looks somewhat concrete; the new additions should be at least at the same level, or better.}
%These studies supplement our work, offering a broader perspective of potential factors affecting model's robust generalization ability. }
%


\section{Study Design}
To understand current practices, motivations, and goals when using modern text-to-image models, we sent a survey to internal users of two internal TTI models to collect basic information about their use of these models (e.g., time spent using the models, motivations, desired capabilities, and prompting strategies). We also interviewed and observed 11 \rr{power} users of TTI models (8 identifying as Male and 3 identifying as Female) in a 50-minute study to uncover their motivations and practices. The latter participants were prolific users of one or more TTI models. Models used by the participants in the study are anonymized for review, but are of the same basic capability as state of the art text to image generation models such as Imagen \cite{ImagenTe79:online}, Parti \cite{PartiPat10:online}, and DALL-E 2 \cite{DALLE279:online} which are described in more detail in our related work.

%There are different types of diffusion models, the first one to show superior performance above GANs is GLIDE (CLIP classifier guided diffusion), Imagen combines T5-XXL (transformer) with diffusion to achieve even better performance. PARTI is an autoregressive model, and it treats text-to-image generation as a sequence-to-sequence problem, which is very similar to machine translation or other language modeling tasks. Parti uses a traditional two-stage encoder-decoder model similar to DALL-E 2 and Imagen.

% Their motivations for using TTI models range from recreational hobby and wanting to experience new technology, to generating reference images of their art projects and doing art projects with the models.

% Survey data snapshot can be found here: https://docs.google.com/spreadsheets/d/1dhjyPK8LaOQRUEYYpnIbSEwhZUBM5K3T39ncvUYM-lw/edit#gid=0
\subsection{Participants}
For the survey, participants were recruited from an internal chat channel dedicated to TTI models (where the internal chat channel has thousands of members) and internal TTI model mailing lists (with hundreds of members). From the survey respondents, we identified interview candidates who had reported having created artwork over more than ten sessions and having spent more than five hours in the previous week using a TTI model. To create a pool of participants, we recruited eight of these latter respondents, and further recruited three prominent artists in the internal artist community to participate. These three artists are quite visible in the internal artist community, and have shared their unique artwork collections within that community. \rr{Participants were also actively engaged in external communities, sharing knowledge, expertise, and artwork.} Participants were given a 60 USD gift card for participation.


\subsection{Interview structure}
The study consisted of four parts intended to understand participants' practices. Each participant was first asked to create an image of their choice to allow the researcher to observe their natural practices.
%, where they get their inspirations from, and the art they are currently most invested in or most comfortable creating.
In the second part of the study, participants were asked to reflect on 1) an artwork they were proud of, 2) a piece they found most successful, and 3) the piece that was least successful. 
%This part of the interview was intended  to understand the criteria and constraints they impose on themselves for the artwork they generate, as well as their practices for stellar outcomes as well as failures.
In the third part of the interview, participants were asked to reflect on someone else's work by examining only the prompt, and specifically asked to either improve or change the prompt in their style. 
%This phase of the interview was intended to better understand the prompting strategies they had developed and the tacit knowledge they had built up about the model behavior. %(their new tool for creation).
In the last part of the interview, participants were asked to discuss envisioned uses for these text-to-image models.

Interviews were conducted remotely. We recorded the shared screen and automatically generated transcripts for the interviews. We constrained our focus to interfaces that only use text prompts as input to the models. In addition, some participants voluntarily shared their collection of generated artwork after the interview.


\subsection{Qualitative Data Analysis}

For the qualitative analyses, the authors analyzed the video transcriptions and also noted comments on participants non-verbal interactions. The final corpus of automatically generated transcripts was 164 pages (60614 words). The first two authors each reviewed the transcripts data independently, looking for ways of explaining the artistic practices~\cite{miles1984drawing}. In this process, the authors separately analyzed each transcript to extracted salient themes, and independently generated hypotheses and points of discussions~\cite{braun2006using}. Using these data, all authors participated in two rounds of interpretation sessions to arrive at the primary themes reported in this paper, \rr{and resolve any discrepancies and disagreements. During the interpretation sessions, authors also analyzed the prompts and the images created by the study participants to identify unique artistic styles and practices. These sessions were inspired by existing analysis practices from qualitative media analysis \cite{altheide2012qualitative}}. 
\section{Survey Results}

We received 161 responses to the survey. Of these responses, 160 answered the question, ``At present, why are you using the [TTI] models?'' Of the responses to this question, 79 (49\%) indicated they use the models to create art, 33 (21\%) reported using it as part of their creative work pipeline, and 126 (79\%) indicated their use was curiosity-driven (not work-related).

In the survey, we also asked participants to estimate the length of time they work with a TTI model when they use one (``When you interact with a model, how much time do you typically spend interacting with it?''). We received 157 responses to this question, with 20\% of respondents indicating that they use a model for one or more hours at a time when they use it (11\% reporting using it for 1-2 hours at a time, 9\% using it for 2 or more hours at a time), 53\% indicating they use a model for 10 minutes to an hour, and 27\% reporting use for less than 10 minutes.

% The remainder of the survey were specific feature requests, which are outside the scope of this study. 

When asked about observed strengths and weaknesses of the various models they've interacted with, survey responses indicated a number of desired capabilities, such as the ability to render text in images, the ability to have more control over spatial arrangements, and the ability for models to handle complex prompts. We also asked participants for desired capabilities when interacting with the model. Seventy six percent of the respondents requested a ``how to build a prompt'' guide, and 75\% desired the ability to fork and remix images, especially for spatial refinement. Seventy six percent of the responses also indicated they would like features like bookmarking, and the ability to directly share outputs to  internal chat groups or social media. As we will see in the artist spotlights below, there is a clear social component to working with these TTI models for our study participants.

Finally, 63\% of the responses desired greater control over the model, such as the ability to assign specific values to each of the prompt words. %, and the ability to turning on and off safety filters.

%Survey responses to the question ``Which image generation models have you used? What strengths and weaknesses have you observed for each? '' discussed how do models handle complex prompts, the ability to render texts in images, the resolution of the images, and the spatial arrangement in the images.

When asked to provide prompting techniques they have learned, %(``Are there any successful/fun tricks or prompts you've used to produce interesting images?''), 
common themes for the strategies included 1) producing specific art styles and eras, such as ``impressionist style'', 2) use of keywords that describe camera lenses and aperture (e.g., ``DSLR photo'',``3D render'',``24 mm, f8, ISO1000''), and 3) domain-specific terms (e.g., ``Line Art'', ``black and white'').


\section{Prompt Artists: Styles, Motivations, Practices}

In this section, we provide a sampling of the vibrant internal TTI artistic community by spotlighting the work of three highly active creators; Shai Noy, Irina Blok, Dan Smith.  \footnote{In the text, we use the terms ``artists,'' ``creators,'' and ``study participants'' interchangeably. We denote the three spotlighted artists as A1, A2, and A3, and other participants by a participant number (e.g., P1, P2, ..., P8).}. For these three artists, we describe and present examples of the styles they have developed and summarize their artistic motivations and goals. We then provide a summary of motivations, styles, and practices observed across the 11 interview participants. In the section that follows, we describe salient high-level, emergent themes arising from the interviews and observations. We want to credit the artists whose artwork are featured, and they had expressed the desire to be associated with their artwork. We use their full name in places where the artwork appears.


\subsection{Shai Noy: The Explorer (A1)}
Shai Noy is a software engineer with no training in the visual arts or design. However, they have produced thousands of images with TTI models. The styles developed by this artist include ``super macro photography'' images (i.e., extremely close-up views of objects, Figure \ref{fig:macro}), and fashion (dresses, suits) made out of unusual materials, such as wood, grass, brick, or ice (Figure \ref{fig:dresses}).

Elements of both discovery and community were emphasized as rewarding for this artist, such as being the first to explore particular concepts and the ability to share discoveries: %``The exploration itself is interesting, a collection of images displays the thought process,'' and 
% ``I first explored these set of concepts'' (A1). Speaking to the community-based aspects of the movement: 
``Everything is more fun when you can share it'' and, ``Art doesn't live in a vacuum, nobody starts from scratch, everything is based on something else. I am proud of being able to recognize the potential'' (A1).

\begin{figure}[t]

    \begin{subfigure}[b]{\textwidth}
        \includegraphics[width=0.24\linewidth]{\imagepath{A1_-_macro_micro_photo_1.png}}
        \hspace{\fill} % note: no blank line here
        \includegraphics[width=0.24\linewidth]{\imagepath{A1_-_macro_micro_photo_4.png}}
        \hspace{\fill} % note: no blank line here
        \includegraphics[width=0.24\linewidth]{\imagepath{A1_-_macro_micro_photo_2.png}}
        \hspace{\fill} % note: no blank line here
        \includegraphics[width=0.24\linewidth]{\imagepath{A1_-_macro_micro_photo_3.png}}
        \caption{Super macro photography} 
        \label{fig:macro}
    \end{subfigure}

    \vspace*{0.2cm} % vertical separation

    \begin{subfigure}[b]{\textwidth}
        \includegraphics[width=0.24\linewidth]{\imagepath{A1_-_A_beautiful_dress_carved_out_of_dead_wood_with_lichen_and_mushrooms,_on_a_mannequin._High_quality,_high_resolution,_studio_lighting.png}}
        \hspace{\fill} % note: no blank line here
        \includegraphics[width=0.24\linewidth]{\imagepath{A1_-_A_beautiful_dress_made_out_of_grass_and_dirt,_on_a_mannequin._High_quality,_High_resolution,_Studio_lighting.png}}
        \hspace{\fill} % note: no blank line here
        \includegraphics[width=0.24\linewidth]{\imagepath{A1_-_A_beautiful_winter_coat_made_out_of_a_brick_fireplace,_on_a_mannequin._High_quality,_high_resolution,_studio_lighting.png}}
        \hspace{\fill} % note: no blank line here
        \includegraphics[width=0.24\linewidth]{\imagepath{A1_-_A_beautiful_winter_jacket_made_out_of_ice,_on_a_mannequin._Studio_lighting,_high_quality,_high_resolution.jpg}}
        \caption{Dresses in various unusual materials}
        \label{fig:dresses}
    \end{subfigure}
    \caption{Selections from Shai Noy (A1 - The Explorer)} 
    \label{fig:artist1}
\end{figure}


\subsection{Irina Blok: The Art Director (A2)}
Irina Blok is a designer who has done visual work for many years using stock images and applications like Photoshop. This artist's styles include origami dancers (Figure \ref{fig:origami}) and reality ``mash-ups,'' such as sliced produce that contains different textures internally (e.g., a sliced head of cabbage that reveals a cross-section of an orange, Figure \ref{fig:mashups}).

Driving these styles is a passion for developing ``impossible objects, something we haven’t seen before, mashups'' (A2). They also seek to create images that differ significantly from the images it was trained on: ``The further away the generated images are from what it was trained on, the more the satisfaction'' and, ``they're [the resulting images] also defying the rules of its training set, like defying [...] gravity. [..] A creative prompt breaks that'' (A2).

Importantly, this artist's output also includes \textit{prompt templates} that describe a particular, parameterized image or visual concept, such as this prompt for generating a house: \textit{Audacious and whimsical fantasy house shaped like <object> with windows and doors, <location>}. These prompt templates are carefully crafted to reliably produce pleasing results for others. We describe this concept more fully in a later section.

% Beyond producing images, A2 also sees text-to-image models as a means to building more effective language learning tools: They find generating images in a reliable and consistent way to be an interesting, engaging method for understanding and learning a language. Specifically, they observed it's a ``tool that allows us to create images in a reliable way, that is also consistent, [...] and used for language learning. Because when we just used text for language learning it's not engaging and it doesn't translate well because the text has to be internalized''. Drawing from previous studies, they added, the models can make ``the visual cues, and they actually contributes to remembering the material and lead to better learning outcomes''.

Notably, when working with the text-to-image models, A2 considers the model to be akin to an artist itself, with themselves the art director. In this relationship, A2 acknowledges a certain lack of control: ``[I] don’t have full control, and there’s beauty in this'' (A2). At the same time, they also consider the model to be a tool: ``[The model] is a brush [...] you just learn to speak its language'' (A2). With these dual views (model as an artist, model as a tool), they note that ``the hardest part is [the] conceptual aspect, being skillful with [the] prompt'' and that ``it’s a thought exercise, it’s not a visual exercise'' (A2). It’s really about how to make people think like an artist''  (A2). In this latter sentiment, this participant was specifically speaking to the need to think like an artist in formulating a prompt, as opposed to formulating a prompt as if one was talking to a machine.

% [TODO: Add: ``There’s a common misconception art is largely about drawing and painting skills. Art is not only about how something looks, it’s about what it says, it tells a story, and has a concept. Art can surprise, provoke, teach, delight and inspire. Art is not just about drawing, art is a way of thinking. [A2]'']

% [TODO: Maybe add: In interacting with the model, Artist 2 considers the model to be akin to an artist itself, with themselves the art director. In this relationship, A2 acknowledges a certain lack of control: ``[I] don’t have full control, and there’s beauty in this'' (A2). At the same time, they also consider the model to be a tool: ``[The model] is a brush [...] you just learn to speak its language'' (A2). With these dual views (model as tool, model as artist), they note that ''the hardest part is [the] conceptual aspect, being skillful with [the] prompt'' and that ``it’s a thought exercise, it’s not a visual exercise. It’s really about how to make people think like an artist'' (A2). [TODO: Verify and clarify these points; there seem to be a lot of interesting nuggets here.] Yet, despite the known challenges of creating a successful prompt, they also suggest that with these models ``anybody can be an artist, and I truly mean it'' (A2).]

\begin{figure}[t]
    \begin{subfigure}[b]{\textwidth}
        \hspace{\fill} % note: no blank line here
        \includegraphics[width=0.24\linewidth]{\imagepath{A2_-_origami_dancer_1.png}}
        \hspace{\fill} % note: no blank line here
        \includegraphics[width=0.24\linewidth]{\imagepath{A2_-_origami_dancer_2.png}}
        \hspace{\fill} % note: no blank line here
        \includegraphics[width=0.24\linewidth]{\imagepath{A2_-_origami_dancer_3.png}}
        \hspace{\fill} % note: no blank line here
        \caption{Origami dancers} 
        \label{fig:origami}
    \end{subfigure}
    
    \vspace*{0.2cm} % vertical separation
    
    \begin{subfigure}[b]{\textwidth}
        \includegraphics[width=0.24\linewidth]{\imagepath{A2_-_Two_objects_1.png}}
        \hspace{\fill} % note: no blank line here
        \includegraphics[width=0.24\linewidth]{\imagepath{A2_-_Two_objects_2.png}}
        \hspace{\fill} % note: no blank line here
        \includegraphics[width=0.24\linewidth]{\imagepath{A2_-_Two_objects_3.png}}
        \hspace{\fill} % note: no blank line here
        \includegraphics[width=0.24\linewidth]{\imagepath{A2_-_Two_objects_4.png}}
        \caption{Reality mash-ups} 
        \label{fig:mashups}
    \end{subfigure}
    \caption{Selections from Irina Blok, (A2 - The Art Director)} 
    \label{fig:artist2}
\end{figure}


\subsection{Dan Smith: The Social Commentator (A3)}

Dan Smith comes from a background in visual media, and is partially driven by the desire to deliver a message about the climate crisis, in order to facilitate change and awareness: ``[I'm] not just having fun ... but [actually] making something that has some power'' (A3). In working towards this goal, they want to make ``something that you look at, and it makes you feel something'' while also making ``something that people would want to look at'' (A3). A3's image styles align with these overall goals: They have a number of pieces that put nature in ``situations where you wouldn't see it'' (Figure \ref{fig:climate-change}) and have created numerous ``hybrid animals'' (Figure \ref{fig:hybrid-animals}). This particular style---hybrid animals---is also inline with a motivation to create images that ``would be hard to [...] visualize or create, if you were [...] a really skilled Photoshop artist'' (see Figure \ref{fig:artist3}).

A3 also heavily considers the quality of the image when assessing its outputs: it must have near expert-level composition, photorealism, and/or artistry. When describing the artwork they created and how they created them, they noted, ``I think the ones that I pick out are [...] standouts for various reasons, and just [...] like what I said, [...] composition photorealism, artistic quality'' (A3). Consistent with their emphasis on overall image quality, they have found ways to address undesirable outputs. For example, in generating images that include animals, they found they needed to adopt a specific strategy to create aesthetically pleasing images: ``I would do `Tall Grass' a lot because early on I discovered that limbs and fingers and paws can get a little wonky'' (A3). The ``tall grass'' addition was their creative strategy to hide feet or paws and suggests an understanding of model limitations, but also a sense of how to cope with these limitations.

\begin{figure}[t]
    \begin{subfigure}[b]{\textwidth}
        \includegraphics[width=0.24\linewidth]{\imagepath{A3_-_Climate_1.png}}
        \hspace{\fill} % note: no blank line here
        \includegraphics[width=0.24\linewidth]{\imagepath{A3_-_Climate_2.png}}
        \hspace{\fill} % note: no blank line here
        \includegraphics[width=0.24\linewidth]{\imagepath{A3_-_Climate_3.png}}
        \hspace{\fill} % note: no blank line here
        \includegraphics[width=0.24\linewidth]{\imagepath{A3_-_Climate_4.png}}
        \caption{Climate change} 
        \label{fig:climate-change}
    \end{subfigure}
    
    \vspace*{0.2cm} % vertical separation
    
    \begin{subfigure}[b]{\textwidth}
        \includegraphics[width=0.24\linewidth]{\imagepath{A3_-_Hybrid_1.png}}
        \hspace{\fill} % note: no blank line here
        \includegraphics[width=0.24\linewidth]{\imagepath{A3_-_Hybrid_2.png}}
        \hspace{\fill} % note: no blank line here
        \includegraphics[width=0.24\linewidth]{\imagepath{A3_-_Hybrid_3.png}}
        \hspace{\fill} % note: no blank line here
        \includegraphics[width=0.24\linewidth]{\imagepath{A3_-_Hybrid_4.png}}
        \caption{Hybrid animals} 
        \label{fig:hybrid-animals}
    \end{subfigure}
    \caption{Selections from Dan Smith (A3 - The Social Commentator)} 
    \label{fig:artist3}
\end{figure}

\subsection{Summarizing Motivations, Styles, and Practices}

One of the primary motivations for interacting with the models was that participants found them fun---the models' output quality enabled people to feel creative, and they were generally interested in interacting with this new class of model. Some people also noted that the model enabled them to engage in their domain interests in a new way. For example, one participant said that the models allowed them to explore their interest in Swiss trains with the model, while another found it compelling to try to create new forms of currency (e.g., new types of coins). One participant used it to create the equivalent of clip art for presentations: ``I use [TTI] model when I need some [...] clipart to use in my presentation, and [TTI] model would be amazing for that...'' (P2).

In comparing the artists spotlighted above to the other participants, one notable difference in practices is that the spotlighted artists placed a particular emphasis on exploring specific themes in depth (e.g., origami figurines). The other participants did not pursue concepts with the same rigor or depth.



\section{Themes: Art Beyond Images, Discovering Unique Points, and Validating Originality}

Across our interviews and observations, a number of themes emerged. First, the notion of what constituted the final artistic output was not always just an image: Some creators consider the \textit{prompt itself} as part of the art. Similarly, a \textit{prompt template} can be considered an artistic output. Second, there was a clear desire for creators to discover new styles possible with the models. This focus on originality extended to one creator going so far as to conduct an image search on successful outputs to validate their originality. We unpack these themes further in this section: 1) Prompt as art, 2) Prompt template as art, 3) Discovering new capabilities of the model, and 4) Validating originality.


\subsection{Prompt as Art}

\rr{For some participants, the prompt itself was part of the overall art, and thus worthy of attention. For these participants, it was important to ``[create] aesthetically pleasing images'' and ``[develop] art concepts'' that \textit{were inherently tied to prompts}.
%, and to use prompts to ``discover new model capabilities.'' 
We detail these motivations and behaviors below.}

While generating aesthetically pleasing, ``glitch-free'' images was a common goal of the creators, other goals were also present in their practices. For some participants, the prompt itself was part of the overall art, and thus worthy of attention 1) on its own, and 2) as it relates to the image: ``It's part of the aesthetic'' (P3), where the prompt is ``like a title of the piece, but you don't get to choose it independently'' (P3). Hinted at in this last comment is the notion that while a prompt could also serve as a title of the art piece, there is a clear dependency on, and perfunctory role for, that prompt as well: The prompt serves as the source material for the model generating the resultant image. Given this dependency, finding a prompt that produces the desired result \textit{and} that can serve as a title for the piece can be challenging, but rewarding when it happens.

\begin{figure}[t]
    \captionsetup[subfigure]{justification=centering}
    \begin{subfigure}[t]{0.4\textwidth}
    \centering
        % \hspace{\fill} % note: no blank line here
        \includegraphics[width=0.6\linewidth]{\imagepath{owl_croissant.png}}
        \subcaption{``a photo of an owl turning into a croissant. f2.2''} 
        \label{fig:owlcroissant}
    \end{subfigure}
        % \hspace{\fill} % note: no blank line here
    \begin{subfigure}[t]{0.4\textwidth}
        \centering
        \includegraphics[width=0.6\linewidth]{\imagepath{bananasaurus.png}}
        \subcaption{``a photo of a bananasaurus rex. a bananasaurus rex has the legs and arms of a banana and the body and head of a tyrannosaurus rex. sigma 85mm f2.2. studio lighting.''} 
        \label{fig:bananasaurus}
        % \hspace{\fill} % note: no blank line here
    \end{subfigure}
    \caption{Artist Ian Fischer (P3) considers (a) as a successful art work given the brevity of the prompt, while (b) a failed attempt because the prompt is too long to get the image they want}
    \label{fig:prompt-as-art}
\end{figure}

This same artist (P3) further described the prompt's role as 
% [TODO: larger context here] 
``communicating the image, and the idea of the image, and how I got it all at the same time'' (P3). % This quote sheds light on how the multimodal nature of TTI models enables artists to think about the way they conceptualize images in a new way. 
This quote sheds light on how art with TTI models can, in some sense, be considered multimodal (text and image) for both the artist and viewer: The prompt and the final output combine into a single, mutually-reinforcing, art piece.
Seen in this light, one can consider the \textit{prompt as art} itself---a well-crafted prompt that creates a compelling image but also accompanies that image, saying something about the image. See Figure \ref{fig:prompt-as-art}a for an example of ``prompt as art,'' as well a prompt that wasn't able to achieve this same level of aesthetic (Figure \ref{fig:prompt-as-art}b).

Diving deeper into this theme, we %were able to learn why the artists regard the prompts itself as an artwork. We 
observed that the artists believe the ``concept'' is a critical component of an artwork. In an internal blog article, A2 describes, ``There’s a common misconception art is largely about drawing and painting skills. Art is not only about how something looks, it’s about what it says, it tells a story, and has a concept. Art can surprise, provoke, teach, delight and inspire. Art is not just about drawing, art is a way of thinking.'' A1 also suggested ``the unit of shareable artwork is not necessarily a specific image but maybe it's the whole exploration of the concept of those images.''

% The exploration of concepts are reflected in the prompts the artists created; others can then view the thought process and the story embedded in the prompts. 

In pursuing ``prompt as art,'' we observed participants impose different constraints when creating their prompts, with some wanting it as descriptive as possible, while others attempting to make it as simple as possible. One participant even discovered delight in their accidental discovery that their ``random string'' produced beautiful output, and named that prompt as their proudest achievement. In this latter case, the pride comes from the joint pairing of the random string and the beautiful output---without the context of the random string, the image has less value, as the random string reveals an unexpected feature of the model.
% ``wpritwuq''

% Regarding the prompt creation process, A2 emphasized ``the hardest part is the concept part, it’s a thought exercise, it’s not a visual exercise, it’s really about how to make people think like an artist.''

% [TODO: This seems to be saying that part of their goal / aesthetic is the prompt. If so, add in here: ``It matches the prompt in a clever or interesting way or the prompt itself is interesting and it sort of says something'' (A3)]


\subsection{Prompt Templates As Art}

\begin{figure}[t]
    % \begin{subfigure}[b]{\textwidth}
        \hspace{\fill} % note: no blank line here
        \includegraphics[width=0.24\linewidth]{\imagepath{strawberry_countryside.png}}
        \hspace{\fill} % note: no blank line here
        \includegraphics[width=0.24\linewidth]{\imagepath{strawberry_paris.png}}
        \hspace{\fill} % note: no blank line here
        \includegraphics[width=0.24\linewidth]{\imagepath{strawberry_tokyo.png}}
        \hspace{\fill} % note: no blank line here
        \caption{Example of the prompt ``Audacious and whimsical fantasy house shaped like <object> with windows and doors, <location>''. All images above use ``strawberry'' as the shape, with the location varied (``countryside'', ``Paris'', ``Tokyo'').} 
        \label{fig:house_template}
    % \end{subfigure}
\end{figure}

In a spirit similar to ``prompt as art,'' five artists sought to produce \textit{prompt templates}. A prompt template is an image description with ``slots'' for someone else to fill in. For example, we previously noted this prompt template created by A2: \textit{Audacious and whimsical fantasy house shaped like <object> with windows and doors, <location>} (see Figure \ref{fig:house_template} for example outputs of this prompt template).

\rr{These prompt templates leverage the capabilities of the models as well as the lightweight, accessible features of prompts. More specifically, when an artist identifies a compelling composition, they can create a text prompt that allows others to create a similar composition, but with their own unique customization to it. %Because of the model's non-deterministic behavior, prompt templates can be thought of as a joint collaboration between the template author, the template user, and the model.
% Given the dynamic, participatory nature of these templates, they can be thought of as shareable interactive art pieces.
The ease with which the templates can be shared also introduces social motivations for producing and distributing templates (e.g., to participate and contribute to a larger community of practice). We elaborate on these points below. }


%\rr{First, we observe that} p
Prompt templates have a number of key features:

\begin{itemize}
    \item They are tightly coupled to and represent a particular artistic concept, vision, or composition (such as a ``whimsical fantasy house''). These features make prompt templates conceptually richer than words or phrases used to specify stylistic characteristics of the image (e.g., ``35mm'' or ``watercolors'', which may be used to produce a particular effect, but don't specify a larger composition).
    \item Others can make use of prompt templates by filling in the blanks. The richness of the models means that the templates guide the overall generated image, but users' unique input can yield a diverse variety of outputs.
    \item There is the intent for the prompt template to provide consistently high quality and delightful output when used by others.
    %, even in the face of unknown inputs by the template user.
    % \item They can be thought of as instructions or source code for generating a particular concept.  
    % \item The prompts are \textit{reliable} and \textit{consistent} in their outputs.
\end{itemize}
% \begin{verbatim}
%     Audacious and whimsical fantasy house shaped like <object> with windows and doors, <location>. [Written by A2]
% \end{verbatim}

Unpacking these concepts in the context of the example prompt above, the skeleton of this prompt embodies a particular (visual) concept: A fantasy home in a given location. Someone making use of this prompt can customize it through two key variables: A shape for the house and a location for the house. While these are seemingly simple variables to customize, the template gives the user great flexibility in terms of the final outputs produced by the model: Any number of shapes can be provided in any number of locations (in fact, the user is free to substitute any text in the slots they wish).

Simultaneously, the prompt cues the model to the \textit{types} of output to produce, as well as details to guide the generation. The phrase ``audacious and whimsical fantasy house'' defines desired attributes of the house, while the specification of ``with windows and doors'' provides additional details that should be included in the generated house. These details in the prompt help increase the reliability of the model output and reduce the likelihood that the downstream user of the prompt template needs to experiment further with the overall prompt structure and content to obtain a good output. %(In a single test of the effectiveness of these additional details, we found the outputs of the model were improved and more consistent when these phrases were present than when they were absent.)

A noteworthy characteristic of these templates and the text-to-image models is the rich interplay that results between the original template and the phrases the user substitutes in the template. For example, a particular choice of location can profoundly influence the resulting house generated by the model---the model does not simply generate the same type of house and situate it in a different location. Instead, the choice of location can directly interact with the other parts of the prompt. For example, creating a strawberry house in the ``countryside'', ``Paris'', or ``Tokyo'' yields qualitatively different outputs for the house, with the house style meshing more naturally in the chosen location (e.g., having styles of windows more typical for a European building for the house in Paris, and doors more typical of Japan for Tokyo---see Figure \ref{fig:house_template}). These interactions between the template and the users' choices enable a diverse variety of outputs that allow a user to explore a wide range of ideas.

To produce prompt templates, one participant described a process whereby they would 1) input a prompt, 2) identify high-quality outcomes, and 3) rewrite the prompt to try to produce that same outcome again. This process could involve several iterations until they get a reliable prompt. Once a prompt is working, participants would sometimes remove content to make it more succinct. For example, they would first emphasize a specific characteristic (like ``high resolution'') by applying many descriptors representing a similar effect (such as ``high res'', ``DSLR'', ``crystal clear'', ``photo realistic''), then start removing the repeated qualifiers in the prompt until the prompt could reliably generate similar outputs to the original, verbose version. % [TODO: Is this true for the prompt templates, or only for prompts themselves?]

As with the notion of a ``prompt as art,'' the creation of these prompt templates can also be considered an artistic outcome in and of itself: The prompt author must first develop a compelling concept, then ensure it has enough capacity to enable people to produce their own unique creations within the frame of the concept. When done well, a prompt template has the quality of attracting users (because of the compelling output produced by the prompt) \textit{and} continually delighting users with the outputs produced using their unique input.

\rr{Elaborating further on this notion of ``prompt template as art,'' a prompt template represents a particular \textit{artistic vision}, with the prompt text capturing the overall design, composition, and aesthetic intentions of the author. Notably, because the artistic vision is captured in natural language text, a prompt template can be used \textit{across} TTI models: Users can customize the template as desired, then tweak the prompt to produce the desired output using whatever model they have at their disposal.}


\subsection{Discovering Unique Points in the Model's Latent Landscape}
A common goal of many artists was to discover new capabilities of the model that others had not yet found. As one participant put it, they tried to ``break'' the model, pushing it to its limits. As with prompt templates, these newly discovered capabilities are often shared so others can apply the concept in their own images. For example, an artist may identify the ability of the model to produce physical sculptures out of unusual materials or to create origami-like figurines (as A2 did). Once a concept and ability have been identified, others can build on the core idea and create their own images or permutations. \rr{In particular, participants felt proud when they ``made the model do X (new concept)'', especially when they discovered a model capability for the first time in the community in which they heavily engage, as the discovery could influence the discourse in the community. %Being able to say they had model respond well to their intent, there's a balance in the human-AI autonomy the participants found to work well. 
We expand on these points below.}

new words to steer the model in new directions. For example, one artist mentioned referencing architectural blogs to learn that domain's vocabulary so it could be applied to their prompts (A2). Another artist mentioned turning to a thesaurus to enrich their vocabulary for the prompt. As one concrete example, A1 (the ``Explorer'') likes to employ ``unusual words'' to steer the model, such as ``intricate'' instead of ``detailed'', explaining: ``If you choose common words, then you get a bit of an uninspiring result quite often. But if you use something a bit more unusual, then you really narrow down [...] the set [of images], and you're going to get into things that use this less common word'' (A1).

What's noteworthy here is that in the pursuit of novel imagery, creators would carefully research and choose \textit{words} to produce specific \textit{visual} outcomes: \textit{What you say and how you say it is critical to producing high-quality imagery with text-to-image models}, requiring TTI users to \textit{enrich their natural language vocabulary} in order to develop and skillfully execute a unique \textit{visual language}.

Driving these practices of discovering new capabilities was a clear desire to push the model away from ``average'' outputs and get it into more unique spaces. In this sense, the artists are navigating and charting the vast latent space of the model and sharing back the most interesting places discovered. When a new, unique output space was discovered, artists expressed a certain satisfaction in having discovered that space: `` A good rule of thumb is to be more descriptive than not [...] if you see that something is lacking, then you try to add more descriptors that will encourage the model, in this direction [...] sometimes you just have to reword a sentence or move something from one sentence to another [...] it's mostly like binary search'' (A1); `` I think your choice of vocabulary is very interesting because you want a large tree. But then, instead of saying a large tree you say mature tree [...] Where do these [...] choices come from? They just come from trial and error'' (A2).

In seeking unique spaces, one participant (P8) expressly hunted for interesting imperfections; instead of \textit{avoiding} glitches, they \textit{sought} glitches. For example, this participant found one of their prompts created imperfections in its output for ``a hybrid of a clock and a snail on an infinite mirror. Steampunk. DSLR photo. astrophotography'' (Figure \ref{fig:breaking_the_model}). Here, A4 explores the imperfections of the infinite mirror: it's ``technically a failure but still amazing'' (P8). P8 also discovered that the model sometimes has trouble generating the backs of things (like cats) and developed that behavior into its own art style (Figure \ref{fig:breaking_the_model}).

\begin{figure}[t]
    % \begin{subfigure}[b]{\textwidth}
        \hspace{\fill} % note: no blank line here
        \includegraphics[width=0.26\linewidth]{\imagepath{P1.png}}
        \hspace{\fill} % note: no blank line here
        \includegraphics[height=0.26\linewidth]{\imagepath{single_cat_back.jpg}}
        \hspace{\fill} % note: no blank line here
        \caption{Artwork by Paul Emmerich(P8). Examples of ``breaking the model'': the artist seeks out and elevates ``glitches'' in the model's output to create new styles. The artist takes advantage of the model's imperfect reflections (on the left) while making creative use of its tendency to transform the backs of cats into cat faces (see the bottom part of the cat's back, which looks like a cat's face).} 
        \label{fig:breaking_the_model}
    % \end{subfigure}
\end{figure}
\subsection{Validating Originality}
% Originality of concept, but can only check originality of artifact. See themselves as explorers. Made the model do X

One question that often arises on the topic of human-in-the-loop, AI-generated art is one of creativity and originality---how much can be attributed to the AI versus the person~\cite{hertzmann2020ComputersArt}. Our participants also struggled with this question, with some seeking to ensure their outputs were novel. \rr{Participants wanted to validate the originality of the artistic concept, but could only check the originality of the artifact.} For example, one participant described a practice of using Google's image search on compelling outputs to ensure originality: After producing a creative output, they use image search to search for that image (or similar images) to ensure it is, in fact, original. This participant mentioned they do this in part because they are sensitized to the fact that machine learning models can sometimes memorize portions of the training data. Given this, they want to ensure their output is \textit{not} in the training data.

We provide some comments on the growth conditions which constituted the majority of our analysis in sections \ref{sec:Hmixing} and \ref{sec:Hsigma}. In the simplest cases of Lemma \ref{lemma:unstableGrowth}, growth was established in an analogous fashion to the old one-step expansion condition (\ref{eq:oldOneStepExpansion}), finding the relevant Jacobians $M_j$ and checking that their expansion factors $K(M_j)$ satisfy
\begin{equation}
    \label{eq:discussionOneStep}
    \sum_j \frac{1}{K(M_j)} <1.
\end{equation}
For the more complicated cases, the inductive method used to establish growth near the accumulation points in Lemma \ref{lemma:unstableGrowth} and the weakened one-step expansion condition (\ref{eq:oneStep}) both address the same fundamental issue: the splitting of unstable curves by singularities into an unbounded number of small components. They circumvent this obstacle in rather different ways, however. While (\ref{eq:oneStep}) generalises (\ref{eq:discussionOneStep}) to ensure an growth of unstable curves `on average' (see \cite{chernov_statistical_2009} for a precise statement), our inductive method is a more direct adaptation of (\ref{eq:discussionOneStep}), using it to generate contradictory geometric conditions which a hypothetical non-growing unstable curve must satisfy. It may be possible to prove Theorem \ref{sec:Hmixing} using (\ref{eq:oneStep}) as the basis for growth. Since we required (\ref{eq:oneStep}) anyway for proving Theorem \ref{thm:HsigmaExp}, this could potentially condense our analysis, but only to a minor extent. A convenience of the method used in section \ref{sec:Hmixing} is that, by way of the `simple intersection' property, it naturally gives geometric information on the images of manifolds, useful for proving the property \textbf{(M)} of Theorem \ref{thm:katok-strelcyn}.

We expect that essentially analogous analysis can be applied to establish mixing properties in a wide class of piecewise linear non-uniformly hyperbolic maps, including those (like the OTM) which sit on the boundary of ergodicity and beyond. While we have relied on the precise partition structure of $H_\sigma$, its fundamental feature (self-similar sequences of elements $A^k$, sharing boundaries with its neighbours $A^{k-1},A^{k+1}$ and accumulating onto some point $p$) is quite typical to return map systems. See, for example, those of various stadium billiards \cite{chernov_chaotic_2006,chernov_improved_2008,chernov_statistical_2009} and LTMs \cite{springham_polynomial_2014}. Indeed, the same method can be used to prove the Bernoulli property for non-monotonic LTMs \cite{myers_hill_mixing_2022}, where monotonicity of the manifold images cannot be assumed and the classical argument \cite{sturman_mathematical_2006} fails. The OTM is the pointwise limit of these maps as the boundary shrinks to null measure. It further has utility in proving growth conditions for maps which are uniformly hyperbolic but possess regions $A_j$ where the hyperbolicity is very weak, signified by $K(M_j) \approx 1$, so that (\ref{eq:discussionOneStep}) fails. Typically this leads to suboptimal bounds on mixing windows, see e.g. \cite{wojtkowski_model_1981,przytycki_ergodicity_1983,myers_hill_family_2022}. The map $H_{(\eta,\eta)}$ for $\eta \approx 1/2$ is another example, possessing weak hyperbolicity over $A_2, A_3$. Letting $\varepsilon = |\eta-1/2|>0$, there is an upper bound $N = N(\varepsilon)$ on escape times from the intersections $A_2\cap \sigma, A_3 \cap \sigma$. The growth lemma then follows by applying the inductive step roughly $N$ times and can be established for arbitrarily small $\varepsilon$, opening the door to establishing optimal mixing windows.

The above gives two examples of piecewise linear perturbations to $H$ where mixing with respect to Lebesgue is preserved and our methods can be applied. Nonlinear perturbations to the shear profiles complicate the analysis in several ways. Firstly as the map's Jacobians takes on a broader range of values, cone invariance becomes an increasingly harder condition to establish. Cones must be widened, giving looser bounds on expansion factors, which may already be weak due to new regions of weaker stretching. This, together with the change from polygonal to curvilinear return time partition elements and nonlinear local manifolds, adds some complexity to showing growth conditions. This does not rule out certain (small) nonlinear perturbations however. There is some leeway in the inequalities which govern cone invariance and growth of local manifolds, the latter of which is not too dissimilar from the piecewise linear setting (see Lemmas \ref{lemma:piecewiseApprox}, \ref{lemma:componentLength}). Certain small perturbations would not alter the \emph{topological} structure of the return time partition, i.e. which elements share boundaries, the key information needed for setting up the induction. Finally while the partition elements would no longer be polygonal, only coarse geometric information is required for verifying each inductive step. Following the above, a potential perturbation could be to replace the linear portions of each shear by a cubic, perturbing the tent profile
\[  f(t) = \begin{cases} 2t & 0 \leq t \leq 1/2, \\ 2(1-t) & 1/2 \leq t \leq 1 ,\end{cases} \]
of the OTM shears to
\[  f_a(t) = \begin{cases} \frac{1}{8} t \left(16 - a + 6at - 8at^{2} \right) & 0 \leq t \leq 1/2, \\ \frac{1}{8}\left(1-t\right)\left( 16 - a + 6a\left(1-t\right) - 8a\left(1-t\right)^{2}\right)  & 1/2 \leq t \leq 1, \end{cases}   \]
for $a>0$. For small enough $a$ the gradient range $f'(t)$ is restricted to small neighbourhoods of $\{ 2, -2\}$ and the escape time partition retains a similar structure. We illustrate this in Figure \ref{fig:perturbations}, showing escapes from the square $S_3$ under the map $G \circ F$, equivalent to escapes from the perturbed $A_3$ under the $G \circ F$, but with a cleaner geometry for comparison. When $a$ is too large the analogy to the OTM breaks down. At $a=16$ the map is twice differentiable everywhere and features a new source of slowed mixing, the Jacobian is the identity at the corner points $x,y \in \{  0, 1/2 \}$ giving locally parabolic behaviour (visible in the escape time partition). 

\begin{figure}
    \centering
    \includegraphics[width=0.24 \linewidth]{0.png}
    \includegraphics[width=0.24 \linewidth]{4.png}
    \includegraphics[width=0.24 \linewidth]{8.png}
    \includegraphics[width=0.24 \linewidth]{16.png}
    \caption{Partition of escape times from $S_3$ under the mapping $F \circ G$ for $a= 0,4,8,16$. }
    \label{fig:perturbations}
\end{figure}
\section{Conclusion}\label{sec:conclusion}
In this work, we focus on addressing the fundamental challenge of OOD detection tasks, which is how to fully understand the semantic discrepancy between the ID/OOD samples. We reveal that the key to success in the realistic SCOOD task is to allocate as many ID samples in the unlabeled set correctly as possible. To this end, we propose a novel uncertainty-aware optimal transport scheme that introduces class-specific energy scores as guidance for effective label assignment. Experimental results show that our method achieves better performance than previous state-of-the-art methods on SCOOD benchmarks.

\textbf{Limitations.} In addition to temperature scaling, other techniques such as feature clipping applied in ReAct~\cite{sun2021react} also enhance the performance of energy score, so how to obtain an OOD score that best fits the SCOOD task can be further explored. Moreover, a setting highly related to SCOOD has been proposed in \cite{katz2022training} and formulated as a constrained optimization problem. We will also theoretically analyze these practical OOD settings in our feature work.

% \section*{Acknowledgments}
\textbf{Acknowledgments.} 
This work is supported by National Key R\&D Program of China under Grant 2020AAA0105701, National Natural Science Foundation of China (NSFC) under Grants 61872327, Major Special Science and Technology Project of Anhui, National Natural Science Foundation of China (62033012) and Ant Group through Ant Research Intern Program.

%%
%% The acknowledgments section is defined using the "acks" environment
%% (and NOT an unnumbered section). This ensures the proper
%% identification of the section in the article metadata, and the
%% consistent spelling of the heading.
\begin{acks}
% To Robert, for the bagels and explaining CMYK and color spaces.
\end{acks}

%%
%% The next two lines define the bibliography style to be used, and
%% the bibliography file.
\bibliographystyle{ACM-Reference-Format}
\bibliography{references}


%%

\end{document}
\endinput
%%
%% End of file `sample-manuscript.tex'.
