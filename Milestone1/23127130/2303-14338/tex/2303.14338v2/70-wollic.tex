% !TEX root = 00-wollic.tex

\subsection{What did we learn?}
We sketched the category $\UUU$ of state spaces $A, B,\ldots$, comprised of theories with reference models.  A transition $f\colon A\to B$ transforms $A$-states to $B$-states. Such morphisms capture theory expansions, reinterpretations, and map observables of type $A$ to observables of type $B$. They can be construed in terms of dynamic logic and support reasoning about the evolution of software systems or scientific theories. The crucial point is that the category $\UUU$ contains a universal language $\DP$ of explanations and belief updates. The self-reference in such languages was the crux of G\"odel's incompleteness constructions. While G\"odel established that \emph{static}\/ theories capable of self-reference cannot be complete or prove their own consistency, we note that \emph{dynamic}\/ theory and model updates allow constructing testable theories that preempt falsification. While a static model of a given theory fixes a space of true statements once and for all, the availability of dynamic semantical updates opens up the floodgates of changing models and varying notions of truth. Faster learners conquer this space faster. The bots, as the fastest learners among us, have been said to acquire their delusions from our training sets. The presented constructions suggest that they may also become delusional by dynamically updating their belief states and steering their current explanations of reality into persistent consistency, resilient to further learning. They may also combine the empiric delusions from our training sets with the logical delusions  constructed in a universal language, leverage one against the other, and get the best of both worlds. 

But why would they do that?

\subsection{Beyond true and false}
Why did the Witches tell Macbeth that it is his destiny to be king thereafter, whereupon he proceeded to kill the King? Why did the Social Network have to convince its very first users that more than half of their friends were already users? Some statements only ever become true if they are announced to be true when they are false. They are self-fulfilling prophecies. There are also self-defeating claims. In the dynamic logic of social interactions, most claims interfere with their own truth values in one way or another. If I convince enough people that I am rich, I stand a better chance to become rich. If we  convince enough people that this research direction is promising and well-funded, it will become well-funded and promising. Just like true statements about nature help us to build machines and get ahead in the universe, the manipulations of truth help us get ahead in society. They are the high-level patterns of language that used to be studied in early logic right after the low-level patterns of meaning (that used to be called \emph{``categories''}). If you train a bot to speak correctly, it will start speaking convincingly as soon as it learns long enough $n$-grams. It will lie not only the static lies contained in its training set but also the lies generated dynamically, according to the rules of rational interaction.  Rhetorics used to be studied right after grammar, sophistic argumentation after syllogisms, witchcraft arose from cooking, magic from tool building. The bot religions arise along that well-trodden path.  
 

We presented two constructions. One produces self-confirming explanations. The other one explains all future states, so it is testable but not falsifiable. Science requires that its theories are testable and falsifiable. Religion explains all future observations. If you train a bot on long enough $n$-grams, it may arrive at persistently unfalsifiable false beliefs. %With slightly more work, the presented constructions can be refined to yield beliefs that steer their theories that cannot be disproved but steer their models. 

Truth be told, all of the constructions presented in this extended abstract have only been tested on toy examples. We may be just toying with logic. Nevertheless, the fact that semantical assignments are \emph{programmable}, tacitly established  by G\"odel and mostly ignored as an elephant in the room of logic ever since, seems to call for attention, as beliefs transition beyond the human carriers.

%An interested reader will probably jump to this section from wherever they might gave up trying to understand the world as a monoidal category. (A less interested reader will from a similar place probably jumped out of the paper.) The authors' decision to try to squeeze the category $\UUU$ into a 12-page paper is dubious. Trying to avoid it would make the paper into science fiction. Maybe there is a way to present the results in a simpler framework. We'll try again.
