% !TEX root = 00-wollic.tex

\subsection{State spaces as objects}\label{Sec:state}
In computation, a state is a family of typed variables with a partial assignment of values. In science, a state is a family of observables, some with expected values. Formally, a state can be viewed as a family of predicates, or a theory in first-order logic,  with a specified model. Both can be presented in the standard Tarskian format, where a theory is a quadruple of sorts, operations, predicates, and axioms, and its  interpretation is an inductively defined model  \cite{Chang-Keisler}. 

\paragraph{Theories as sketches.} In this extended abstract, theories are  presented as categorical \emph{sketches}\/ and their models are specified in extended functorial semantics \cite{AdamekJ:locpac,Ehresmann-Bastiani:sketches,LairC:modelables,LairC:esquissables,MakkaiM:acccfc}. While this may not be the most popular view, it is succinct enough to fit into the available space. The main constructions, presented in Sec.~\ref{Sec:moncom}--\ref{Sec:unfalse}, do not depend on the choice of presentation. The reader could thus skip to Sec.~\ref{Sec:string} and come back as needed. 

\begin{definition}\label{def:state}
A \emph{clone} $\Sigma$ is a cartesian category\footnote{We stick with the traditional terminology where a category is cartesian when it has cartesian products. The cartesian product preserving functors are abbreviated to \emph{cartesian functors}. This clashes with the standard terminology for morphisms between fibrations, but fibrations do not come about in this paper.} freely generated by sorts, operations, and equational axioms of a logical theory. A \emph{theory}\/ is a pair $\Theta=<\Sigma,\Gamma>$, where $\Sigma$ is a clone and $\Gamma$ is a set of cones and cocones in $\Sigma$, capturing the general axioms\footnote{Equational axioms could be subsumed among cones and cocones, and omitted from $\Sigma$, which would boil down to the free category generated by sorts and operations.} of the logical theory. A \emph{model}\/ of $\Theta$ is a cartesian functor $\MMM\colon \Sigma \to \Set$ mapping the $\Gamma$-cones into limit cones and the $\Gamma$-cocones into colimit cocones. A \emph{state of belief} (or \emph{belief state}) is a triple 
\bear
A & = & \left<\Sigma_{A},\Gamma_{A}, \MMM_{A}\right> 
\eear
where $\Theta_{A} = \left<\Sigma_{A},\Gamma_{A}\right>$ is a theory and $\MMM_{A}$ its model in a category $\Set$ of sets and functions. An\/ element of the model $\MMM_{A}$ is called an \emph{observable} of the state $A$.
\end{definition}

\paragraph{States of $A$ as extensions of $\MMM_{A}$.} The reference model $\MMM_A$ determines the notion of truth in the state space $A$. It expresses properties that may not be proved in the theory $\Theta_{A}$ or even effectively specified\footnote{E.g., the set of all true statements of Peano arithmetic is expressed by its standard model, but most of them cannot be described effectively.}. The reference model $\MMM_{A}$ should thus not be thought of as a single object of the category of all models of $\Theta_{A}$ but as the (accessible) subcategory of model extensions of $\MMM_A$. These model extensions are the states of the state space $A$. The structure of a state space can be further refined to capture other features of  theories in science and engineering, including their statistical and complexity-theoretic valuations \cite{RissanenJ:MDL,WallaceCS:MML}. While such refinements have no direct impact on our considerations, they signal that we are in the realm of \emph{inductive}\/ inference, which may feel unusual for the Tarskian framework of static logic, normally concerned with deductive aspects. The fact that the theory $\Theta_{A}$ has a model $\MMM_{A}$ implies that it is logically consistent but it does not imply that it is true within an external frame of reference, a \emph{``reality''}\/ that may drive the state transitions, i.e. the processes of extending and reinterpreting theories. The intuition is that the states in the space $A$ are observables that may never be observed, since $\MMM_{A}$ may be incompatible with the actual observations. The theory $\Theta_{A}$ may be consistent but wrong.

\paragraph{Examples} of state spaces include logical theories with standard models that arise not only in natural sciences but also in social systems, as policy formalizations.  A software specification with a reference implementation can also be viewed as a state space. Updates and evolution of a software system can then be analyzed using a higher-order dynamic logic \cite{PavlovicD:JAMP}. The functorial semantics view was spelled out in \cite{PavlovicD:FOPS}, used in a software synthesis tool \cite{PavlovicD:AMAST08,PavlovicD:ASE01,PavlovicD:SDR}, and applied in algorithm design \cite{PavlovicD:ManaFest,PavlovicD:MPC10}. 

\subsection{Transitions as morphisms}\label{Sec:transition}
Intuitively, a transition $f$ from a state space $A$ to a state space $B$ is a specification that induces a transition from any $A$-state to a $B$-state. We first consider the transitions arising from reinterpreting theories and then expand to modifying the reference models.

\begin{definition}\label{Def:interpretable}
An \emph{interpretation}\/ of state space $A$ in a state space $B$ is a logical interpretation of the theory $\Theta_{B}=<\Sigma_{B},\Gamma_{B}>$ in the theory $\Theta_{A} =<\Sigma_{A},\Gamma_{A}>$ which reduces the reference model $\MMM_{A}$ to $\MMM_{B}$. More precisely, an interpretation  $f\colon A\to B$ is a cartesian functor $f\colon \Sigma_{A}\ot %\ooot{\Gamma_{A}}{\Gamma_{B}} 
\Sigma_{B}$ mapping $\Gamma_{B}$-(co)cones to $\Gamma_{A}$-(co)cones according to a given assignment $f_\Gamma\colon \Gamma_{A}\ot \Gamma_{B}$ and making the following diagram commute
\beq\label{eq:def-intepretable}
\begin{tikzar}{}
\Sigma_{A}\ar{dr}[description]{\MMM_{A}} \&\& \Sigma_{B}\ar{dl}[description]{\MMM_{B}} \ar[bend right]{ll}[swap]{f}\\
\& \Set
\end{tikzar}
\eeq
The models $\MMM_A$ and $\MMM_B$ map the (co)cones from $\Gamma_A$ and $\Gamma_B$ to (co)limits of sets, as required by Def.~\ref{def:state}. \end{definition}

\paragraph{Interpretations as assignments.} The structure of interpretations of software specifications and the method to compose them were spelled out in  \cite{PavlovicD:FOPS,PavlovicD:ASE01}. Since software specifications are finite, an interpretation $f\colon \Sigma_{A}\ot \Sigma_{B}$ boils down to a tuple of assignments 
\[x_{1}:=t_{1}\ ;\  x_{2}:=t_{2}\ ;\ldots;\  x_{n}:=t_{n}\]
of terms $\vec t=<t_{1}, t_{2},\ldots, t_{n}>$ from $\Sigma_{A}$ to variables $\vec x=<x_{1}, x_{2},\ldots, x_{n}>$ from $ \Sigma_{B}$ in such a way that, for each axiom $\gamma\in \Gamma_{B}$, the substitution instance 
\bear
f(\gamma) &= & [\vec{x}:=\vec {t}]\gamma
\eear  
is a theorem derivable from the axioms in $\Gamma_{A}$. In Hoare logic \cite{Hoare-logic}, a state transition $f\colon \Sigma_{A}\ot \Sigma_{B}$ is presented as a triple $\Theta_{A}\uev{\vec{x}:=\vec {t}} \Theta_{B}$.  By definition, this triple is valid if and only if $\Theta_{A}\vdash [\vec{x}:=\vec {t}]\Theta_{B}$, where $[\vec{x}:=\vec {t}]\Theta_{B}$ is the result substituting the $\Theta_{A}$-terms $\vec t$ for $\Theta_{B}$-variables $\vec x$ in all axioms $\gamma\in \Gamma_{B}$. Condition \eqref{eq:def-intepretable} moreover requires that this theory interpretation recovers the model $\MMM_{B}$ from the model $\MMM_{A}$. 

In general, however, it is not always possible to transform all computational states annotated at all relevant program points into one another by mere substitutions. That is why Hoare logic does not boil down to the assignment clause, but specifies the meaning of other program constants in other clauses, which can be viewed as more general state transitions.

%\paragraph{Updates, explanations, predictions.} Updating scientific theories, social policies, or software specifications seldom boils down to variable reassignments. A process that maps observables from one state to another may not even be observable. Computational processes use internal variables, intermediary types, and a variety of program constructs to transform observable inputs of type $A$ to observable outputs of type $B$. An economic process may be claimed to increase employment rates in the general population by cutting taxes on the rich. Such causal links between two types of observables are explained by unobservable mental processes in two social groups and by hidden variables connecting them. A physical theory may transform observable classical measurements as inputs into observable  classical predictions as outputs using an internal model of unobservable quantum interactions.  In general, an explanation is not just a hypothesis that some observables cause some observed effects but it may also include a model of an internal, possibly unobservable causal mechanism that leads from one to the other. The mechanism may be complex and the causal relations may be partial or nondeterministic. But once the observed effects are explained by some observable causes, the observations of the causes can be used to predict the effects. Explanations generally input some observed effects and output some causal mechanisms, and predictions then go the other way around. Explanations are tested by validating the induced predictions. Effectively specified explanations induce computable predictions. Belief transitions formalize both.


\begin{definition}\label{Def:explainable}
A \emph{state transition}\/  $f\colon A\to B$ is a cartesian functor  $f\colon \Theta_{A}\ot %\ooot{\Gamma_{A}}{\Gamma_{B}} 
\Theta_{B}$ mapping $\Gamma_{B}$-(co)cones to $\Gamma_{A}$-(co)cones according to a given assignment $f_\Gamma\colon \Gamma_{A}\ot \Gamma_{B}$ and moreover making the following diagram commute
\beq\label{eq:def-explainable}
\begin{tikzar}{}
\Theta_{A}\ar{dr}[description]{\overline\MMM_{A}} \&\& \Theta_{B}\ar{dl}[description]{\overline\MMM_{B}} \ar[bend right]{ll}[swap]{f}\\
\& \Set
\end{tikzar}
\eeq
where $\overline\MMM_A$ is the extension of $\MMM_A$ along the completion $\Sigma_A\inclusion \Theta_A$ of $\Sigma_A$ under the limits and colimits generated by $\Gamma_A$; ditto for $\overline\MMM_B$.
\end{definition}

\paragraph{General sketches.} In Def.~\ref{Def:interpretable}, theories were presented as pairs $\Theta = <\Sigma,\Gamma>$, where the category $\Sigma$ is comprised of sorts, operations, and equations of the theory, whereas the cones and the cocones in $\Gamma$ specify its predicates and axioms. In Def.~\ref{Def:explainable}, a theory $\Theta$ is presented as the category obtained by completing $\Sigma$ under the limits and the colimits specified by $\Gamma$. This general sketch, with the family of limit cones and colimit cocones from $\Gamma$, is now denoted $\Theta$, by abuse of notation. A detailed construction of this sketch can be found in \cite[\S4.2--3]{MakkaiM:acccfc}. It is a canonical view of the theory derived in the signature $\Sigma$ from the axioms $\Gamma$. Since the category $\Theta$ is the $\Gamma$-completion of $\Sigma$, any functor $\MMM:\Sigma\to\Set$ mapping the $\Gamma$-(co)cones in $\Sigma$ to (co)limit (co)cones in $\Set$ has a unique $\Gamma$-preserving extension $\overline \MMM\colon \Theta\to \Set$. These extensions are displayed in \eqref{eq:def-intepretable}. The upshot of saturating the sketches from Def.~\ref{Def:interpretable} in the form $\Theta=<\Sigma,\Gamma>$ to the general sketches over $\Theta$ in Def.~\ref{Def:explainable} is that the general explainable transitions are now simply the structure-preserving functors displayed in \eqref{eq:def-intepretable}. 

%\paragraph{\em Explanations as programs.} A theory $\Theta$ is derived from the clone (i.e. signature) $\Sigma$ inductively, by imposing the (co)limit (co)cones (i.e. axioms) $\Gamma$. If $\Sigma$ and $\Gamma$ are given effectively, then $\Theta$ can be computed. Since the functor $f\colon \Theta_A\ot \Theta_B$, underlying a belief transition $f\colon A\to B$, maps $\Gamma_{B}$-(co)limits to $\Gamma_{A}$-(co)limits, it follows that it can be recovered from its restriction $\hat f\colon \Theta_{A}\ot \Sigma_{B}$ along $\Sigma_{B}\inclusion \Theta_{B}$. Therefore, whenever $\Theta_A$ and $\Theta_B$ are computable, $f$ is.  However, since there may be \emph{many different ways}\/ to map $\Gamma_{B}$-(co)limits to $\Gamma_{A}$-(co)limits, the functor $f\colon \Theta_A\ot \Theta_B$ is \emph{\textbf{not} uniquely determined}\/ by its restriction $\hat f\colon \Theta_{A}\ot \Sigma_{B}$. The logical intuition is that there may be many different ways to prove the axioms $\Gamma_{B}$ as theorems in $\Theta_{A}$ and many different logical justifications of a belief transition from $A$ to $B$. In summary, an explainable transition $f\colon A\to B$ is effective as long as $A$ and $B$ are effectively given, and the predictions are computable. But computations can be programmed in many different ways. In any case, \emph{the explanations that actually explain an explainable transition can be thought of and formalized as programs that compute its predictions.} If the semantics of a programming language $\EEE$ used for computing the predictions is expressed in terms of a dynamic logic assignment $\TTT^o\times \EEE\times \TTT\to \OOO$, where $\TTT$ is a suitable posetal collapse of the universe of theories, then the sketch morphism or theory mapping $f\colon \Theta_A\ot \Theta_B$ corresponds to a Hoare triple $\Theta_A\uev F\Theta_B$, where $F$ is a program for the computation $f$. 

\subsection{Monoidal category of state spaces and transitions}
Let 
\begin{itemize}
\item $\UUU$ be the category of state spaces from Def.~\ref{def:state} and transitions from Def.~\ref{Def:explainable}, and let
\item $\tot \UUU$ be the category of state spaces from Def.~\ref{def:state} and interpretations from Def.~\ref{Def:interpretable}.
\end{itemize}
In both cases, the monoidal structure is induced by the disjoint unions of theories:
\bea
A\otimes B & = & \Big<\Sigma_{A}+\Sigma_{B}\, ,\,  \Gamma_{A}+\Gamma_{B}\, ,\,  [\MMM_{A}+\MMM_{B}]
\Big>
\eea
where $\MMM_{A\otimes B} = [\MMM_{A}+\MMM_{B}]\colon \Sigma_{A}+\Sigma_{B} \tto{\Gamma_{A\otimes B}}\Set$ maps $\Sigma_{A}$ like $\MMM_{A}$ and $\Sigma_{B}$ like $\MMM_{B}$. The tensor unit is $I = \left<\bot, \bot, \emptyset\right>$, where the truth value $\bot$ denotes the inconsistent theory or sketch, its only axiom, and $\emptyset$ is its empty model. It obviously satisfies $I\otimes A = A = A\otimes I$. The associativity of the tensor $\otimes$ follows from the associativity of the disjoint union $+$. The arrow part of $\otimes$ is induced by the disjoint unions as coproducts. The coproduct structure equips every state space $A$ with a cartesian comonoid structure
\begin{gather}\label{eq:dataserv-text}
 \ \ A\otimes A\  \oot{\ \ \ \ \Delta\ \  \ \ } A \tto{\ \  \ \scun\ \ \ } I\\
\Sigma_{A}+\Sigma_{A}\tto{\ [\id,\id]\ } \Sigma_{A}\oot{\ \ \bot\ \ } \bot\notag
\end{gather}
This provides a categorical mechanism for cloning and erasing states, which makes some observations repeatable and deletable, as required for testing in science and software engineering. However, $\UUU$ is not a cartesian category, and $\otimes$ is not a cartesian product, because some transitions $f:A\to B$ do not in general boil down to functors $\Sigma_{A}\ot \Sigma_{B}$, but only to functors $\Theta_{A}\ot \Sigma_{B}$, where $\Theta_A$ is a completion of $\Sigma_A$ under the $\Gamma_A$-(co)-limits. Intuitively, this means that the axioms of the theory $\Theta_{B}$ may not be interpreted as axioms of $\Theta_{A}$, but may be mapped into theorems, which only arise in the $\Gamma_{A}$-completion. This captures the uncloneable and undeletable states that arise in many sciences, including physics of very small or very large (quantum or cosmological) and economics. The only transitions that preserve the cartesian structure \eqref{eq:dataserv-text} are the interpretations $f:A\to B$, with the underlying functors $\Sigma_{A}\ot \Sigma_{B}$. They form the category $\tot\UUU$, which is the largest cartesian subcategory of $\UUU$. If the states $\alpha \in \UUU(I,A)$ are thought of as observables, the states $a\in \tot \UUU(I,A)$ are the actual observations.

