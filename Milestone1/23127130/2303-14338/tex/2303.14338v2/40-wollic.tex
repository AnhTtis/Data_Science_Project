% !TEX root = 00-wollic.tex

A theory of theories, such as the categorical theory of sketches, is a theory. Category theory is also a theory and functorial semantics provides a categorical theory of reference models. The theory of state spaces from Sec.~\ref{Sec:state} can thus be formalized and presented as a state space in the category $\UUU$. The theory of state spaces from Sec.~\ref{Sec:state} can thus be formalized into a sketch with a reference model and presented as a state space in the category $\UUU$. The theory of state transitions from Sec.~\ref{Sec:transition} is another sketch, and with another reference model it is also a  state space in $\UUU$. Call it $\DP$. The fact that the states in $\DP$ correspond to the transitions in $\UUU$ means that it satisfies a parametrized version of \eqref{eq:univ}. It is a universal language for $\UUU$. Its interpreters follow from its definition, as the models of the theory of transitions. Since there is no room here to spell out the details of a theory of transitions and show that the correspondence of its cartesian models and the transitions in $\UUU$ equips $\DP$ with all interpreters, we  postulate the existence of the interpreters by the following definition. 

\begin{definition}\label{Def:interpreter}
An \emph{universal interpreter}\/ for state spaces $A,B$ is a transition $\{\}\colon \DP\otimes A\to B$ in $\UUU$ which is universal for all parametric families of transitions from $A$ to $B$. This means that for any state space $X$ and any transition $g\in \UUU(X\otimes A, B)$ there is an interpretation $G\in \tot\UUU(X,\DP)$ with
\beq\label{eq:uev}
\begin{split}
\newcommand{\Fee}{\scriptstyle G}
\newcommand{\fee}{g}
\newcommand{\Aee}{\scriptstyle X}
\newcommand{\Bee}{\scriptstyle A}
\newcommand{\Cee}{\scriptstyle B}
\newcommand{\Code}{\scriptstyle \PPp}
\newcommand{\Univ}{\mbox{\large$\{\}$}}
\newcommand{\Dott}{\mbox{\Large$\bullet$}}
\def\JPicScale{.33}
%%Created by jPicEdt 1.4.1_03: mixed JPIC-XML/LaTeX format
%%Thu Mar 16 16:35:34 GMT-10:00 2023
%%Begin JPIC-XML
%<?xml version="1.0" standalone="yes"?>
%<jpic x-min="0" x-max="220" y-min="0" y-max="120" auto-bounding="true">
%<multicurve fill-style= "none"
%	 stroke-width= "0.35"
%	 points= "(80,80);(80,80);(80,40);(80,40)"
%	 />
%<multicurve fill-style= "none"
%	 stroke-width= "0.35"
%	 points= "(80,40);(80,40);(0,40);(0,40)"
%	 />
%<multicurve fill-style= "none"
%	 stroke-width= "0.35"
%	 points= "(0,80);(0,80);(80,80);(80,80)"
%	 />
%<multicurve fill-style= "none"
%	 stroke-width= "0.35"
%	 points= "(60,40);(60,40);(60,0);(60,0)"
%	 />
%<multicurve fill-style= "none"
%	 stroke-width= "0.35"
%	 points= "(40,120);(40,120);(40,80);(40,80)"
%	 />
%<multicurve fill-style= "none"
%	 stroke-width= "0.35"
%	 points= "(0,80);(0,80);(0,40);(0,40)"
%	 />
%<multicurve fill-style= "none"
%	 stroke-width= "0.35"
%	 points= "(20,40);(20,40);(20,0);(20,0)"
%	 />
%<multicurve fill-style= "none"
%	 stroke-width= "0.35"
%	 points= "(220,100);(220,100);(220,60);(220,60)"
%	 />
%<multicurve fill-style= "none"
%	 stroke-width= "0.35"
%	 points= "(180,20);(180,20);(140,20);(140,20)"
%	 />
%<multicurve fill-style= "none"
%	 stroke-width= "0.35"
%	 points= "(140,100);(140,100);(220,100);(220,100)"
%	 />
%<multicurve fill-style= "none"
%	 stroke-width= "0.35"
%	 points= "(200,60);(200,60);(200,0);(200,0)"
%	 />
%<multicurve fill-style= "none"
%	 stroke-width= "0.35"
%	 points= "(180,120);(180,120);(180,100);(180,100)"
%	 />
%<multicurve fill-style= "none"
%	 stroke-width= "0.35"
%	 points= "(140,60);(140,60);(140,20);(140,20)"
%	 />
%<multicurve fill-style= "none"
%	 stroke-width= "0.35"
%	 points= "(160,20);(160,20);(160,0);(160,0)"
%	 />
%<multicurve fill-style= "none"
%	 stroke-width= "0.35"
%	 points= "(220,60);(220,60);(180,60);(180,60)"
%	 />
%<multicurve fill-style= "none"
%	 stroke-width= "0.35"
%	 points= "(140,100);(140,100);(180,60);(180,60)"
%	 />
%<multicurve fill-style= "none"
%	 stroke-width= "0.35"
%	 points= "(140,60);(140,60);(180,20);(180,20)"
%	 />
%<multicurve fill-style= "none"
%	 stroke-width= "0.7"
%	 points= "(160,40);(160,40);(160,80);(160,80)"
%	 />
%<text fill-style= "none"
%	 stroke-width= "0.35"
%	 text-vert-align= "center-v"
%	 anchor-point= "(190,80)"
%	 text-frame= "noframe"
%	 text-hor-align= "center-h"
%	 >
%$\Univ$
%</text>
%<text fill-style= "none"
%	 stroke-width= "0.35"
%	 text-vert-align= "center-v"
%	 anchor-point= "(110,60)"
%	 text-frame= "noframe"
%	 text-hor-align= "center-h"
%	 >
%\EQLS
%</text>
%<text fill-style= "none"
%	 stroke-width= "0.35"
%	 text-vert-align= "center-v"
%	 anchor-point= "(156.25,65)"
%	 text-frame= "noframe"
%	 text-hor-align= "right"
%	 >
%$\Code$
%</text>
%<text fill-style= "none"
%	 stroke-width= "0.35"
%	 text-vert-align= "center-v"
%	 anchor-point= "(185,117.5)"
%	 text-frame= "noframe"
%	 text-hor-align= "left"
%	 >
%$\Cee$
%</text>
%<text fill-style= "none"
%	 stroke-width= "0.35"
%	 text-vert-align= "center-v"
%	 anchor-point= "(45,117.5)"
%	 text-frame= "noframe"
%	 text-hor-align= "left"
%	 >
%$\Cee$
%</text>
%<text fill-style= "none"
%	 stroke-width= "0.35"
%	 text-vert-align= "center-v"
%	 anchor-point= "(15,2.5)"
%	 text-frame= "noframe"
%	 text-hor-align= "right"
%	 >
%$\Aee$
%</text>
%<text fill-style= "none"
%	 stroke-width= "0.35"
%	 text-vert-align= "center-v"
%	 anchor-point= "(55,2.5)"
%	 text-frame= "noframe"
%	 text-hor-align= "right"
%	 >
%$\Bee$
%</text>
%<text fill-style= "none"
%	 stroke-width= "0.35"
%	 text-vert-align= "center-v"
%	 anchor-point= "(40,60)"
%	 text-frame= "noframe"
%	 text-hor-align= "center-h"
%	 >
%$\fee$
%</text>
%<text fill-style= "none"
%	 stroke-width= "0.35"
%	 text-vert-align= "center-v"
%	 anchor-point= "(150,31.25)"
%	 text-frame= "noframe"
%	 text-hor-align= "center-h"
%	 >
%$\Fee$
%</text>
%<text fill-style= "none"
%	 stroke-width= "0.35"
%	 text-vert-align= "center-v"
%	 anchor-point= "(155,2.5)"
%	 text-frame= "noframe"
%	 text-hor-align= "right"
%	 >
%$\Aee$
%</text>
%<text fill-style= "none"
%	 stroke-width= "0.35"
%	 text-vert-align= "center-v"
%	 anchor-point= "(195,2.5)"
%	 text-frame= "noframe"
%	 text-hor-align= "right"
%	 >
%$\Bee$
%</text>
%<text fill-style= "none"
%	 stroke-width= "0.35"
%	 text-vert-align= "center-v"
%	 anchor-point= "(160,40)"
%	 text-frame= "noframe"
%	 text-hor-align= "center-h"
%	 >
%$\Dott$
%</text>
%</jpic>
%%End JPIC-XML
%LaTeX-picture environment using emulated lines and arcs
%You can rescale the whole picture (to 80% for instance) by using the command \def\JPicScale{0.8}
\ifx\JPicScale\undefined\def\JPicScale{1}\fi
\unitlength \JPicScale mm
\begin{picture}(220,120)(0,0)
\linethickness{0.35mm}
\put(80,40){\line(0,1){40}}
\linethickness{0.35mm}
\put(0,40){\line(1,0){80}}
\linethickness{0.35mm}
\put(0,80){\line(1,0){80}}
\linethickness{0.35mm}
\put(60,0){\line(0,1){40}}
\linethickness{0.35mm}
\put(40,80){\line(0,1){40}}
\linethickness{0.35mm}
\put(0,40){\line(0,1){40}}
\linethickness{0.35mm}
\put(20,0){\line(0,1){40}}
\linethickness{0.35mm}
\put(220,60){\line(0,1){40}}
\linethickness{0.35mm}
\put(140,20){\line(1,0){40}}
\linethickness{0.35mm}
\put(140,100){\line(1,0){80}}
\linethickness{0.35mm}
\put(200,0){\line(0,1){60}}
\linethickness{0.35mm}
\put(180,100){\line(0,1){20}}
\linethickness{0.35mm}
\put(140,20){\line(0,1){40}}
\linethickness{0.35mm}
\put(160,0){\line(0,1){20}}
\linethickness{0.35mm}
\put(180,60){\line(1,0){40}}
\linethickness{0.35mm}
\multiput(140,100)(0.12,-0.12){333}{\line(1,0){0.12}}
\linethickness{0.35mm}
\multiput(140,60)(0.12,-0.12){333}{\line(1,0){0.12}}
\linethickness{0.7mm}
\put(160,40){\line(0,1){40}}
\put(190,80){\makebox(0,0)[cc]{$\Univ$}}

\put(110,60){\makebox(0,0)[cc]{\EQLS}}

\put(156.25,65){\makebox(0,0)[cr]{$\Code$}}

\put(185,117.5){\makebox(0,0)[cl]{$\Cee$}}

\put(45,117.5){\makebox(0,0)[cl]{$\Cee$}}

\put(15,2.5){\makebox(0,0)[cr]{$\Aee$}}

\put(55,2.5){\makebox(0,0)[cr]{$\Bee$}}

\put(40,60){\makebox(0,0)[cc]{$\fee$}}

\put(150,31.25){\makebox(0,0)[cc]{$\Fee$}}

\put(155,2.5){\makebox(0,0)[cr]{$\Aee$}}

\put(195,2.5){\makebox(0,0)[cr]{$\Bee$}}

\put(160,40){\makebox(0,0)[cc]{$\Dott$}}

\end{picture}
 
\end{split}
\eeq
\end{definition}

On one hand, a universal interpreter is universal for parametric families. On the other hand, it is a parametric family itself. It is thus capable of interpreting itself.  This capability of self-reflection was crucial for G\"odel's incompleteness construction.  This capability is embodied in the \emph{specializers}, which are derived  directly from Def~\ref{Def:interpreter}. 

\begin{lemma}\label{prop:pev}
For any $X, A, B$ there is an interpretation $\prtial \in \tot\UUU(\DP\times X, \DP)$ which specializes from a given $X\otimes A$-interpreter to an $A$-interpreter, in the sense
\beq\label{eq:pev}
\begin{split}
\newcommand{\Fee}{\prtial}
\newcommand{\fee}{\mbox{\large$\{\}$}}
\newcommand{\Aee}{\scriptstyle X}
\newcommand{\Bee}{\scriptstyle A}
\newcommand{\Cee}{\scriptstyle B}
\newcommand{\Code}{\scriptstyle \DP}
\newcommand{\Univ}{\mbox{\large$\{\}$}}
\newcommand{\Dott}{\mbox{\LARGE$\bullet$}}
\def\JPicScale{.33}
%%Created by jPicEdt 1.4.1_03: mixed JPIC-XML/LaTeX format
%%Fri Mar 17 18:45:57 GMT-10:00 2023
%%Begin JPIC-XML
%<?xml version="1.0" standalone="yes"?>
%<jpic x-min="0" x-max="280" y-min="0" y-max="120" auto-bounding="true">
%<multicurve fill-style= "none"
%	 points= "(120,80);(120,80);(120,40);(120,40)"
%	 />
%<multicurve fill-style= "none"
%	 points= "(120,40);(120,40);(40,40);(40,40)"
%	 />
%<multicurve fill-style= "none"
%	 points= "(0,80);(0,80);(120,80);(120,80)"
%	 />
%<multicurve fill-style= "none"
%	 points= "(100,40);(100,40);(100,0);(100,0)"
%	 />
%<multicurve fill-style= "none"
%	 points= "(80,120);(80,120);(80,80);(80,80)"
%	 />
%<multicurve fill-style= "none"
%	 points= "(0,80);(0,80);(40,40);(40,40)"
%	 />
%<multicurve fill-style= "none"
%	 points= "(60,40);(60,40);(60,0);(60,0)"
%	 />
%<multicurve fill-style= "none"
%	 points= "(280,100);(280,100);(280,60);(280,60)"
%	 />
%<multicurve fill-style= "none"
%	 points= "(240,20);(240,20);(200,20);(200,20)"
%	 />
%<multicurve fill-style= "none"
%	 points= "(200,100);(200,100);(280,100);(280,100)"
%	 />
%<multicurve fill-style= "none"
%	 points= "(260,60);(260,60);(260,0);(260,0)"
%	 />
%<multicurve fill-style= "none"
%	 points= "(240,120);(240,120);(240,100);(240,100)"
%	 />
%<multicurve fill-style= "none"
%	 points= "(160,60);(160,60);(200,20);(200,20)"
%	 />
%<multicurve fill-style= "none"
%	 points= "(220,20);(220,20);(220,0);(220,0)"
%	 />
%<multicurve fill-style= "none"
%	 points= "(280,60);(280,60);(240,60);(240,60)"
%	 />
%<multicurve fill-style= "none"
%	 points= "(200,100);(200,100);(240,60);(240,60)"
%	 />
%<multicurve fill-style= "none"
%	 points= "(200,60);(200,60);(240,20);(240,20)"
%	 />
%<multicurve fill-style= "none"
%	 points= "(220,40);(220,40);(220,80);(220,80)"
%	 stroke-width= "0.7"
%	 />
%<text text-vert-align= "center-v"
%	 fill-style= "none"
%	 anchor-point= "(250,80)"
%	 text-frame= "noframe"
%	 text-hor-align= "center-h"
%	 >
%$\Univ$
%</text>
%<text text-vert-align= "center-v"
%	 fill-style= "none"
%	 anchor-point= "(140,60)"
%	 text-frame= "noframe"
%	 text-hor-align= "center-h"
%	 >
%\EQLS
%</text>
%<text text-vert-align= "center-v"
%	 fill-style= "none"
%	 anchor-point= "(216.25,65)"
%	 text-frame= "noframe"
%	 text-hor-align= "right"
%	 >
%$\Code$
%</text>
%<text text-vert-align= "center-v"
%	 fill-style= "none"
%	 anchor-point= "(245,117.5)"
%	 text-frame= "noframe"
%	 text-hor-align= "left"
%	 >
%$\Cee$
%</text>
%<text text-vert-align= "center-v"
%	 fill-style= "none"
%	 anchor-point= "(85,117.5)"
%	 text-frame= "noframe"
%	 text-hor-align= "left"
%	 >
%$\Cee$
%</text>
%<text text-vert-align= "center-v"
%	 fill-style= "none"
%	 anchor-point= "(55,2.5)"
%	 text-frame= "noframe"
%	 text-hor-align= "right"
%	 >
%$\Aee$
%</text>
%<text text-vert-align= "center-v"
%	 fill-style= "none"
%	 anchor-point= "(95,2.5)"
%	 text-frame= "noframe"
%	 text-hor-align= "right"
%	 >
%$\Bee$
%</text>
%<text text-vert-align= "center-v"
%	 fill-style= "none"
%	 anchor-point= "(80,60)"
%	 text-frame= "noframe"
%	 text-hor-align= "center-h"
%	 >
%$\fee$
%</text>
%<text text-vert-align= "center-v"
%	 fill-style= "none"
%	 anchor-point= "(200,40)"
%	 text-frame= "noframe"
%	 text-hor-align= "center-h"
%	 >
%$\Fee$
%</text>
%<text text-vert-align= "center-v"
%	 fill-style= "none"
%	 anchor-point= "(215,2.5)"
%	 text-frame= "noframe"
%	 text-hor-align= "right"
%	 >
%$\Aee$
%</text>
%<text text-vert-align= "center-v"
%	 fill-style= "none"
%	 anchor-point= "(255,2.5)"
%	 text-frame= "noframe"
%	 text-hor-align= "right"
%	 >
%$\Bee$
%</text>
%<text text-vert-align= "center-v"
%	 fill-style= "none"
%	 anchor-point= "(220,40)"
%	 text-frame= "noframe"
%	 text-hor-align= "center-h"
%	 >
%$\Dott$
%</text>
%<multicurve fill-style= "none"
%	 points= "(20,60);(20,60);(20,0);(20,0)"
%	 />
%<multicurve fill-style= "none"
%	 points= "(200,60);(200,60);(160,60);(160,60)"
%	 />
%<multicurve fill-style= "none"
%	 points= "(180,40);(180,40);(180,0);(180,0)"
%	 />
%<text text-vert-align= "center-v"
%	 fill-style= "none"
%	 anchor-point= "(175,2.5)"
%	 text-frame= "noframe"
%	 text-hor-align= "right"
%	 >
%$\Code$
%</text>
%<text text-vert-align= "center-v"
%	 fill-style= "none"
%	 anchor-point= "(15,2.5)"
%	 text-frame= "noframe"
%	 text-hor-align= "right"
%	 >
%$\Code$
%</text>
%</jpic>
%%End JPIC-XML
%LaTeX-picture environment using emulated lines and arcs
%You can rescale the whole picture (to 80% for instance) by using the command \def\JPicScale{0.8}
\ifx\JPicScale\undefined\def\JPicScale{1}\fi
\unitlength \JPicScale mm
\begin{picture}(280,120)(0,0)
\linethickness{0.3mm}
\put(120,40){\line(0,1){40}}
\linethickness{0.3mm}
\put(40,40){\line(1,0){80}}
\linethickness{0.3mm}
\put(0,80){\line(1,0){120}}
\linethickness{0.3mm}
\put(100,0){\line(0,1){40}}
\linethickness{0.3mm}
\put(80,80){\line(0,1){40}}
\linethickness{0.3mm}
\multiput(0,80)(0.12,-0.12){333}{\line(1,0){0.12}}
\linethickness{0.3mm}
\put(60,0){\line(0,1){40}}
\linethickness{0.3mm}
\put(280,60){\line(0,1){40}}
\linethickness{0.3mm}
\put(200,20){\line(1,0){40}}
\linethickness{0.3mm}
\put(200,100){\line(1,0){80}}
\linethickness{0.3mm}
\put(260,0){\line(0,1){60}}
\linethickness{0.3mm}
\put(240,100){\line(0,1){20}}
\linethickness{0.3mm}
\multiput(160,60)(0.12,-0.12){333}{\line(1,0){0.12}}
\linethickness{0.3mm}
\put(220,0){\line(0,1){20}}
\linethickness{0.3mm}
\put(240,60){\line(1,0){40}}
\linethickness{0.3mm}
\multiput(200,100)(0.12,-0.12){333}{\line(1,0){0.12}}
\linethickness{0.3mm}
\multiput(200,60)(0.12,-0.12){333}{\line(1,0){0.12}}
\linethickness{0.7mm}
\put(220,40){\line(0,1){40}}
\put(250,80){\makebox(0,0)[cc]{$\Univ$}}

\put(140,60){\makebox(0,0)[cc]{\EQLS}}

\put(216.25,65){\makebox(0,0)[cr]{$\Code$}}

\put(245,117.5){\makebox(0,0)[cl]{$\Cee$}}

\put(85,117.5){\makebox(0,0)[cl]{$\Cee$}}

\put(55,2.5){\makebox(0,0)[cr]{$\Aee$}}

\put(95,2.5){\makebox(0,0)[cr]{$\Bee$}}

\put(80,60){\makebox(0,0)[cc]{$\fee$}}

\put(200,40){\makebox(0,0)[cc]{$\Fee$}}

\put(215,2.5){\makebox(0,0)[cr]{$\Aee$}}

\put(255,2.5){\makebox(0,0)[cr]{$\Bee$}}

\put(220,40){\makebox(0,0)[cc]{$\Dott$}}

\linethickness{0.3mm}
\put(20,0){\line(0,1){60}}
\linethickness{0.3mm}
\put(160,60){\line(1,0){40}}
\linethickness{0.3mm}
\put(180,0){\line(0,1){40}}
\put(175,2.5){\makebox(0,0)[cr]{$\Code$}}

\put(15,2.5){\makebox(0,0)[cr]{$\Code$}}

\end{picture}
 
\end{split}
\eeq
\end{lemma}

\paragraph{Hoare logic of interpreters and specializers.} If interpreters are presented as Hoare triples in the form $(X\otimes A)\uev G B$, and if $X\!\pev G$ denotes a specialization of $G$ to $X$ as above, then \eqref{eq:pev} can be written as the invertible Hoare rule
\[\prooftree
(X\otimes A)\uev G B
\Justifies
A\uev{X\!\pev G} B
\endprooftree\]   

\paragraph{Explanations.} Interpretations  (in the sense of Def.~\ref{Def:interpretable}) of arbitrary states from some space $X$ along $G\in \tot\UUU(X,\DP)$ in a universal language $\DP$ can be construed as \emph{explanations}. If $\DP$ is a programming language, they are programs. The idea that explaining a process means programming a computation has been pursued in theory of science from various directions \cite[and references therein]{Osherson:sci-inquiry}. A universal language $\DP$ is thus a universal space of explanations. The idea of programming languages as universal state spaces is pursued in \cite[Ch.~7]{PavlovicD:MonCom}. Just like any universal programming language makes every computation programmable, any universal language from Def.~\ref{Def:interpreter} makes any observable transition explainable. What we cannot explain, we cannot recognize, and therefore we cannot observe. But it gets funny when we take into account how our explanations influence our observations, and how our current explanations can be made to steer future observations. This is sketched in the next two sections. 

