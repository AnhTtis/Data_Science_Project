%%Created by jPicEdt 1.4.1_03: mixed JPIC-XML/LaTeX format
%%Mon Apr 29 06:50:25 GMT+00:00 2019
%%Begin JPIC-XML
%<?xml version="1.0" standalone="yes"?>
%<jpic x-min="0" x-max="190" y-min="0" y-max="110" auto-bounding="true">
%<multicurve fill-style= "none"
%	 stroke-width= "0.75"
%	 points= "(110,70);(110,70);(190,70);(190,70)"
%	 />
%<multicurve fill-style= "none"
%	 stroke-width= "0.75"
%	 points= "(190,40);(190,40);(190,70);(190,70)"
%	 />
%<multicurve fill-style= "none"
%	 stroke-width= "0.75"
%	 points= "(175,95);(175,95);(175,70);(175,70)"
%	 />
%<text text-vert-align= "center-v"
%	 fill-style= "none"
%	 stroke-width= "0.75"
%	 anchor-point= "(165,110)"
%	 text-frame= "noframe"
%	 text-hor-align= "center-h"
%	 >
%$\nameslang$
%</text>
%<text text-vert-align= "center-v"
%	 fill-style= "none"
%	 stroke-width= "0.75"
%	 anchor-point= "(165,0)"
%	 text-frame= "noframe"
%	 text-hor-align= "center-h"
%	 >
%$\inputt$
%</text>
%<multicurve fill-style= "none"
%	 stroke-width= "0.75"
%	 points= "(140,40);(140,40);(190,40);(190,40)"
%	 />
%<multicurve fill-style= "none"
%	 stroke-width= "0.75"
%	 points= "(165,40);(165,40);(165,15);(165,15)"
%	 />
%<text text-vert-align= "center-v"
%	 fill-style= "none"
%	 stroke-width= "0.75"
%	 anchor-point= "(158.12,55)"
%	 text-frame= "noframe"
%	 text-hor-align= "center-h"
%	 >
%\universal
%</text>
%<multicurve fill-style= "none"
%	 stroke-width= "0.75"
%	 points= "(125,55);(125,55);(125,15);(125,15)"
%	 />
%<multicurve fill-style= "none"
%	 stroke-width= "0.75"
%	 points= "(110,70);(110,70);(140,40);(140,40)"
%	 />
%<multicurve fill-style= "none"
%	 stroke-width= "0.75"
%	 points= "(180,40);(180,40);(180,15);(180,15)"
%	 />
%<multicurve fill-style= "none"
%	 stroke-width= "0.75"
%	 points= "(150,40);(150,40);(150,15);(150,15)"
%	 />
%<multicurve fill-style= "none"
%	 stroke-width= "0.75"
%	 points= "(155,95);(155,95);(155,70);(155,70)"
%	 />
%<text text-vert-align= "center-v"
%	 fill-style= "none"
%	 anchor-point= "(125,0)"
%	 text-frame= "noframe"
%	 text-hor-align= "center-h"
%	 >
%$\description$
%</text>
%<text text-vert-align= "center-v"
%	 fill-style= "none"
%	 anchor-point= "(123.75,15)"
%	 text-frame= "noframe"
%	 text-hor-align= "right"
%	 >
%$\dee$
%</text>
%<text text-vert-align= "center-v"
%	 fill-style= "none"
%	 anchor-point= "(147.5,15)"
%	 text-frame= "noframe"
%	 text-hor-align= "right"
%	 >
%$\ceeone$
%</text>
%<text text-vert-align= "center-v"
%	 fill-style= "none"
%	 anchor-point= "(162.5,15)"
%	 text-frame= "noframe"
%	 text-hor-align= "right"
%	 >
%$\ceetwo$
%</text>
%<text text-vert-align= "center-v"
%	 fill-style= "none"
%	 anchor-point= "(177.5,15)"
%	 text-frame= "noframe"
%	 text-hor-align= "right"
%	 >
%$\ceethree$
%</text>
%<text text-vert-align= "center-v"
%	 fill-style= "none"
%	 anchor-point= "(157.5,95)"
%	 text-frame= "noframe"
%	 text-hor-align= "left"
%	 >
%$\eeone$
%</text>
%<text text-vert-align= "center-v"
%	 fill-style= "none"
%	 anchor-point= "(177.5,95)"
%	 text-frame= "noframe"
%	 text-hor-align= "left"
%	 >
%$\eetwo$
%</text>
%<text text-vert-align= "center-v"
%	 fill-style= "none"
%	 anchor-point= "(0,110)"
%	 text-frame= "noframe"
%	 text-hor-align= "center-h"
%	 >
%\nothing
%</text>
%<text text-vert-align= "center-v"
%	 fill-style= "none"
%	 anchor-point= "(0,0)"
%	 text-frame= "noframe"
%	 text-hor-align= "center-h"
%	 >
%\nothing
%</text>
%</jpic>
%%End JPIC-XML
%PSTricks content-type (pstricks.sty package needed)
%Add \usepackage{pstricks} in the preamble of your LaTeX file
%You can rescale the whole picture (to 80% for instance) by using the command \def\JPicScale{0.8}
\ifx\JPicScale\undefined\def\JPicScale{1}\fi
\psset{unit=\JPicScale mm}
\psset{linewidth=0.3,dotsep=1,hatchwidth=0.3,hatchsep=1.5,shadowsize=1,dimen=middle}
\psset{dotsize=0.7 2.5,dotscale=1 1,fillcolor=black}
\psset{arrowsize=1 2,arrowlength=1,arrowinset=0.25,tbarsize=0.7 5,bracketlength=0.15,rbracketlength=0.15}
\begin{pspicture}(0,0)(190,110)
\psline[linewidth=0.75](110,70)(190,70)
\psline[linewidth=0.75](190,40)(190,70)
\psline[linewidth=0.75](175,95)(175,70)
\rput(165,110){$\nameslang$}
\rput(165,0){$\inputt$}
\psline[linewidth=0.75](140,40)(190,40)
\psline[linewidth=0.75](165,40)(165,15)
\rput(158.12,55){\universal}
\psline[linewidth=0.75](125,55)(125,15)
\psline[linewidth=0.75](110,70)(140,40)
\psline[linewidth=0.75](180,40)(180,15)
\psline[linewidth=0.75](150,40)(150,15)
\psline[linewidth=0.75](155,95)(155,70)
\rput(125,0){$\description$}
\rput[r](123.75,15){$\dee$}
\rput[r](147.5,15){$\ceeone$}
\rput[r](162.5,15){$\ceetwo$}
\rput[r](177.5,15){$\ceethree$}
\rput[l](157.5,95){$\eeone$}
\rput[l](177.5,95){$\eetwo$}
\rput(0,110){\nothing}
\rput(0,0){\nothing}
\end{pspicture}
