%%Created by jPicEdt 1.4.1_03: mixed JPIC-XML/LaTeX format
%%Sun Apr 28 21:03:37 GMT+00:00 2019
%%Begin JPIC-XML
%<?xml version="1.0" standalone="yes"?>
%<jpic x-min="0" x-max="190" y-min="0" y-max="110" auto-bounding="true">
%<multicurve fill-style= "none"
%	 stroke-width= "0.75"
%	 points= "(0,70);(0,70);(50,70);(50,70)"
%	 />
%<multicurve fill-style= "none"
%	 stroke-width= "0.75"
%	 points= "(0,70);(0,70);(0,40);(0,40)"
%	 />
%<multicurve fill-style= "none"
%	 stroke-width= "0.75"
%	 points= "(50,40);(50,40);(50,70);(50,70)"
%	 />
%<multicurve fill-style= "none"
%	 stroke-width= "0.75"
%	 points= "(15,95);(15,95);(15,70);(15,70)"
%	 />
%<text text-vert-align= "center-v"
%	 fill-style= "none"
%	 stroke-width= "0.75"
%	 anchor-point= "(25,110)"
%	 text-frame= "noframe"
%	 text-hor-align= "center-h"
%	 >
%$\nameslang$
%</text>
%<text text-vert-align= "center-v"
%	 fill-style= "none"
%	 stroke-width= "0.75"
%	 anchor-point= "(25,0)"
%	 text-frame= "noframe"
%	 text-hor-align= "center-h"
%	 >
%$\inputt$
%</text>
%<multicurve fill-style= "none"
%	 stroke-width= "0.75"
%	 points= "(0,40);(0,40);(50,40);(50,40)"
%	 />
%<multicurve fill-style= "none"
%	 stroke-width= "0.75"
%	 points= "(10,40);(10,40);(10,15);(10,15)"
%	 />
%<text text-vert-align= "center-v"
%	 fill-style= "none"
%	 stroke-width= "0.75"
%	 anchor-point= "(25,55)"
%	 text-frame= "noframe"
%	 text-hor-align= "center-h"
%	 >
%\machine
%</text>
%<multicurve fill-style= "none"
%	 stroke-width= "0.75"
%	 points= "(35,95);(35,95);(35,70);(35,70)"
%	 />
%<multicurve fill-style= "none"
%	 stroke-width= "0.75"
%	 points= "(40,40);(40,40);(40,15);(40,15)"
%	 />
%<multicurve fill-style= "none"
%	 stroke-width= "0.75"
%	 points= "(25,40);(25,40);(25,15);(25,15)"
%	 />
%<text text-vert-align= "center-v"
%	 fill-style= "none"
%	 anchor-point= "(190,110)"
%	 text-frame= "noframe"
%	 text-hor-align= "center-h"
%	 >
%\nothing
%</text>
%<text text-vert-align= "center-v"
%	 fill-style= "none"
%	 anchor-point= "(190,0)"
%	 text-frame= "noframe"
%	 text-hor-align= "center-h"
%	 >
%\nothing
%</text>
%<text text-vert-align= "center-v"
%	 fill-style= "none"
%	 anchor-point= "(7.5,13.75)"
%	 text-frame= "noframe"
%	 text-hor-align= "right"
%	 >
%$\ceeone$
%</text>
%<text text-vert-align= "center-v"
%	 fill-style= "none"
%	 anchor-point= "(22.5,13.75)"
%	 text-frame= "noframe"
%	 text-hor-align= "right"
%	 >
%$\ceetwo$
%</text>
%<text text-vert-align= "center-v"
%	 fill-style= "none"
%	 anchor-point= "(37.5,13.75)"
%	 text-frame= "noframe"
%	 text-hor-align= "right"
%	 >
%$\ceethree$
%</text>
%<text text-vert-align= "center-v"
%	 fill-style= "none"
%	 anchor-point= "(17.5,93.75)"
%	 text-frame= "noframe"
%	 text-hor-align= "left"
%	 >
%$\eeone$
%</text>
%<text text-vert-align= "center-v"
%	 fill-style= "none"
%	 anchor-point= "(37.5,93.75)"
%	 text-frame= "noframe"
%	 text-hor-align= "left"
%	 >
%$\eetwo$
%</text>
%</jpic>
%%End JPIC-XML
%PSTricks content-type (pstricks.sty package needed)
%Add \usepackage{pstricks} in the preamble of your LaTeX file
%You can rescale the whole picture (to 80% for instance) by using the command \def\JPicScale{0.8}
\ifx\JPicScale\undefined\def\JPicScale{1}\fi
\psset{unit=\JPicScale mm}
\psset{linewidth=0.3,dotsep=1,hatchwidth=0.3,hatchsep=1.5,shadowsize=1,dimen=middle}
\psset{dotsize=0.7 2.5,dotscale=1 1,fillcolor=black}
\psset{arrowsize=1 2,arrowlength=1,arrowinset=0.25,tbarsize=0.7 5,bracketlength=0.15,rbracketlength=0.15}
\begin{pspicture}(0,0)(190,110)
\psline[linewidth=0.75](0,70)(50,70)
\psline[linewidth=0.75](0,70)(0,40)
\psline[linewidth=0.75](50,40)(50,70)
\psline[linewidth=0.75](15,95)(15,70)
\rput(25,110){$\nameslang$}
\rput(25,0){$\inputt$}
\psline[linewidth=0.75](0,40)(50,40)
\psline[linewidth=0.75](10,40)(10,15)
\rput(25,55){\machine}
\psline[linewidth=0.75](35,95)(35,70)
\psline[linewidth=0.75](40,40)(40,15)
\psline[linewidth=0.75](25,40)(25,15)
\rput(190,110){\nothing}
\rput(190,0){\nothing}
\rput[r](7.5,13.75){$\ceeone$}
\rput[r](22.5,13.75){$\ceetwo$}
\rput[r](37.5,13.75){$\ceethree$}
\rput[l](17.5,93.75){$\eeone$}
\rput[l](37.5,93.75){$\eetwo$}
\end{pspicture}
