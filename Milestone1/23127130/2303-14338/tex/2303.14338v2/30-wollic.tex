% !TEX root = 00-wollic.tex

Constructions in monoidal categories yield to insightful presentations in terms of string diagrams \cite[Ch.~1]{Joyal-Street:geometry,PavlovicD:MonCom}. We will need them to present the constructions like \eqref{eq:fixpoint} and in particular  \eqref{eq:ana}. While commutative diagrams like \eqref{eq:def-intepretable} display compositions of morphisms and abbreviate their equations, string diagrams display \emph{de}\/compositions of morphisms. Monoidal categories come with two composition operations: the categorical (sequential) morphism composition $\circ$ and the monoidal (parallel) composition $\otimes$. The former is drawn along the vertical axis, the latter along the horizontal axis. The objects are drawn as strings, the morphisms as boxes. A morphism $A\tto f B$ is presented as a box $f$ with a string $A$ hanging from the bottom and a string $B$ sticking out from the top. The identities are presented as invisible boxes: the identity on $A$ is just the string $A$. The unit type $I$ is presented as the invisible string. There are thus boxes with no strings attached. The composite morphism $g\circ f =(A\tto f B\tto g C)$ is drawn bottom-up, by hanging the box $f$ on the string $B$ under the box $g$. The monoidal composition is presented as the horizontal adjacency: the composite $(g\circ f)\otimes (s\circ t)$ is drawn by placing the boxes $g\circ f$ next to the boxes for $s\circ t$:

\beq\label{eq:godement}
\begin{split}
\newcommand{\machine}{$f$}
\newcommand{\gee}{$g$}
\newcommand{\kee}{$s$}
\newcommand{\hee}{t}
\newcommand{\nameslang}{\scriptstyle B}
\newcommand{\seqcompp}{\scriptstyle {\color{red}g\circ f}}
\newcommand{\parcompp}{\scriptstyle {\color{blue}f\otimes t}}
\newcommand{\inputt}{\scriptstyle A} 
\newcommand{\outpt}{$\scriptstyle C$}
\newcommand{\otherinputt}{\scriptstyle U}
\newcommand{\otheroutpt}{\scriptstyle V} 
\newcommand{\outpttt}{$\scriptstyle W$}
\def\JPicScale{.5}
%%Created by jPicEdt 1.4.1_03: mixed JPIC-XML/LaTeX format
%%Fri Mar 17 19:14:21 GMT-10:00 2023
%%Begin JPIC-XML
%<?xml version="1.0" standalone="yes"?>
%<jpic x-min="-5" x-max="115" y-min="-5" y-max="115" auto-bounding="true">
%<multicurve fill-style= "none"
%	 points= "(15,40);(15,40);(45,40);(45,40)"
%	 stroke-width= "1"
%	 />
%<multicurve fill-style= "none"
%	 points= "(15,40);(15,40);(15,20);(15,20)"
%	 stroke-width= "1"
%	 />
%<multicurve fill-style= "none"
%	 points= "(45,20);(45,20);(45,40);(45,40)"
%	 stroke-width= "1"
%	 />
%<multicurve fill-style= "none"
%	 arrow-head-inset-scale= "0"
%	 points= "(30,70);(30,70);(30,40);(30,40)"
%	 arrow-head-width-minimum= "1.5"
%	 arrow-head-length-scale= "1.5"
%	 stroke-width= "1"
%	 left-arrow= "head"
%	 />
%<text text-vert-align= "center-v"
%	 fill-style= "none"
%	 anchor-point= "(27.5,55)"
%	 text-frame= "noframe"
%	 stroke-width= "1"
%	 text-hor-align= "right"
%	 >
%$\nameslang$
%</text>
%<text text-vert-align= "center-v"
%	 fill-style= "none"
%	 anchor-point= "(28.75,-3.75)"
%	 text-frame= "noframe"
%	 stroke-width= "1"
%	 text-hor-align= "right"
%	 >
%$\inputt$
%</text>
%<multicurve fill-style= "none"
%	 points= "(15,20);(15,20);(45,20);(45,20)"
%	 stroke-width= "1"
%	 />
%<multicurve fill-style= "none"
%	 arrow-head-inset-scale= "0"
%	 points= "(30,20);(30,20);(30,-5);(30,-5)"
%	 arrow-head-width-minimum= "1.5"
%	 arrow-head-length-scale= "1.5"
%	 stroke-width= "1"
%	 left-arrow= "head"
%	 />
%<text text-vert-align= "center-v"
%	 fill-style= "none"
%	 anchor-point= "(30,30)"
%	 text-frame= "noframe"
%	 stroke-width= "1"
%	 text-hor-align= "center-h"
%	 >
%\machine
%</text>
%<multicurve fill-style= "none"
%	 points= "(15,90);(15,90);(45,90);(45,90)"
%	 stroke-width= "1"
%	 />
%<multicurve fill-style= "none"
%	 points= "(15,90);(15,90);(15,70);(15,70)"
%	 stroke-width= "1"
%	 />
%<multicurve fill-style= "none"
%	 points= "(45,70);(45,70);(45,90);(45,90)"
%	 stroke-width= "1"
%	 />
%<multicurve fill-style= "none"
%	 arrow-head-inset-scale= "0"
%	 points= "(30,115);(30,115);(30,90);(30,90)"
%	 arrow-head-width-minimum= "1.5"
%	 arrow-head-length-scale= "1.5"
%	 stroke-width= "1"
%	 left-arrow= "head"
%	 />
%<multicurve fill-style= "none"
%	 points= "(15,70);(15,70);(45,70);(45,70)"
%	 stroke-width= "1"
%	 />
%<text text-vert-align= "center-v"
%	 fill-style= "none"
%	 anchor-point= "(30,80)"
%	 text-frame= "noframe"
%	 stroke-width= "1"
%	 text-hor-align= "center-h"
%	 >
%\gee
%</text>
%<text text-vert-align= "center-v"
%	 fill-style= "none"
%	 anchor-point= "(32.5,113.75)"
%	 text-frame= "noframe"
%	 stroke-width= "1"
%	 text-hor-align= "left"
%	 >
%\outpt
%</text>
%<multicurve fill-style= "none"
%	 points= "(65,40);(65,40);(95,40);(95,40)"
%	 stroke-width= "1"
%	 />
%<multicurve fill-style= "none"
%	 points= "(65,40);(65,40);(65,20);(65,20)"
%	 stroke-width= "1"
%	 />
%<multicurve fill-style= "none"
%	 points= "(95,20);(95,20);(95,40);(95,40)"
%	 stroke-width= "1"
%	 />
%<multicurve fill-style= "none"
%	 arrow-head-inset-scale= "0"
%	 points= "(80,70);(80,70);(80,40);(80,40)"
%	 arrow-head-width-minimum= "1.5"
%	 arrow-head-length-scale= "1.5"
%	 stroke-width= "1"
%	 left-arrow= "head"
%	 />
%<multicurve fill-style= "none"
%	 points= "(65,20);(65,20);(95,20);(95,20)"
%	 stroke-width= "1"
%	 />
%<multicurve fill-style= "none"
%	 arrow-head-inset-scale= "0"
%	 points= "(80,20);(80,20);(80,-5);(80,-5)"
%	 arrow-head-width-minimum= "1.5"
%	 arrow-head-length-scale= "1.5"
%	 stroke-width= "1"
%	 left-arrow= "head"
%	 />
%<text text-vert-align= "center-v"
%	 fill-style= "none"
%	 anchor-point= "(77.5,-3.75)"
%	 text-frame= "noframe"
%	 stroke-width= "1"
%	 text-hor-align= "right"
%	 >
%$\otherinputt$
%</text>
%<text text-vert-align= "center-v"
%	 fill-style= "none"
%	 anchor-point= "(77.5,55)"
%	 text-frame= "noframe"
%	 stroke-width= "1"
%	 text-hor-align= "right"
%	 >
%$\otheroutpt$
%</text>
%<text text-vert-align= "center-v"
%	 fill-style= "none"
%	 anchor-point= "(80,30)"
%	 text-frame= "noframe"
%	 stroke-width= "1"
%	 text-hor-align= "center-h"
%	 >
%$\hee$
%</text>
%<parallelogram p3= "(50,5)"
%	 fill-style= "none"
%	 p2= "(50,105)"
%	 p1= "(10,105)"
%	 stroke-color= "#ff0066"
%	 stroke-width= "0.45"
%	 />
%<parallelogram p3= "(115,15)"
%	 fill-style= "none"
%	 p2= "(115,45)"
%	 p1= "(-5,45)"
%	 stroke-color= "#3300ff"
%	 stroke-width= "0.45"
%	 />
%<text text-vert-align= "bottom"
%	 fill-style= "none"
%	 anchor-point= "(113.75,16.25)"
%	 text-frame= "noframe"
%	 stroke-width= "1"
%	 text-hor-align= "right"
%	 >
%$\parcompp$
%</text>
%<text text-vert-align= "top"
%	 fill-style= "none"
%	 anchor-point= "(48.75,103.12)"
%	 text-frame= "noframe"
%	 stroke-width= "1"
%	 text-hor-align= "right"
%	 >
%$\seqcompp$
%</text>
%<multicurve fill-style= "none"
%	 points= "(65,90);(65,90);(65,70);(65,70)"
%	 stroke-width= "1"
%	 />
%<multicurve fill-style= "none"
%	 points= "(95,70);(95,70);(95,90);(95,90)"
%	 stroke-width= "1"
%	 />
%<multicurve fill-style= "none"
%	 arrow-head-inset-scale= "0"
%	 points= "(80,115);(80,115);(80,90);(80,90)"
%	 arrow-head-width-minimum= "1.5"
%	 arrow-head-length-scale= "1.5"
%	 stroke-width= "1"
%	 left-arrow= "head"
%	 />
%<multicurve fill-style= "none"
%	 points= "(65,70);(65,70);(95,70);(95,70)"
%	 stroke-width= "1"
%	 />
%<multicurve fill-style= "none"
%	 points= "(65,90);(65,90);(95,90);(95,90)"
%	 stroke-width= "1"
%	 />
%<text text-vert-align= "center-v"
%	 fill-style= "none"
%	 anchor-point= "(80,80)"
%	 text-frame= "noframe"
%	 stroke-width= "1"
%	 text-hor-align= "center-h"
%	 >
%\kee
%</text>
%<text text-vert-align= "center-v"
%	 fill-style= "none"
%	 anchor-point= "(82.5,113.75)"
%	 text-frame= "noframe"
%	 stroke-width= "1"
%	 text-hor-align= "left"
%	 >
%\outpttt
%</text>
%</jpic>
%%End JPIC-XML
%PSTricks content-type (pstricks.sty package needed)
%Add \usepackage{pstricks} in the preamble of your LaTeX file
%You can rescale the whole picture (to 80% for instance) by using the command \def\JPicScale{0.8}
\ifx\JPicScale\undefined\def\JPicScale{1}\fi
\psset{unit=\JPicScale mm}
\psset{linewidth=0.3,dotsep=1,hatchwidth=0.3,hatchsep=1.5,shadowsize=1,dimen=middle}
\psset{dotsize=0.7 2.5,dotscale=1 1,fillcolor=black}
\psset{arrowsize=1 2,arrowlength=1,arrowinset=0.25,tbarsize=0.7 5,bracketlength=0.15,rbracketlength=0.15}
\begin{pspicture}(0,0)(115,115)
\psline[linewidth=1](15,40)(45,40)
\psline[linewidth=1](15,40)(15,20)
\psline[linewidth=1](45,20)(45,40)
\psline[linewidth=1,arrowsize=1.5 2,arrowlength=1.5,arrowinset=0]{<-}(30,70)(30,40)
\rput[r](27.5,55){$\nameslang$}
\rput[r](28.75,-3.75){$\inputt$}
\psline[linewidth=1](15,20)(45,20)
\psline[linewidth=1,arrowsize=1.5 2,arrowlength=1.5,arrowinset=0]{<-}(30,20)(30,-5)
\rput(30,30){\machine}
\psline[linewidth=1](15,90)(45,90)
\psline[linewidth=1](15,90)(15,70)
\psline[linewidth=1](45,70)(45,90)
\psline[linewidth=1,arrowsize=1.5 2,arrowlength=1.5,arrowinset=0]{<-}(30,115)(30,90)
\psline[linewidth=1](15,70)(45,70)
\rput(30,80){\gee}
\rput[l](32.5,113.75){\outpt}
\psline[linewidth=1](65,40)(95,40)
\psline[linewidth=1](65,40)(65,20)
\psline[linewidth=1](95,20)(95,40)
\psline[linewidth=1,arrowsize=1.5 2,arrowlength=1.5,arrowinset=0]{<-}(80,70)(80,40)
\psline[linewidth=1](65,20)(95,20)
\psline[linewidth=1,arrowsize=1.5 2,arrowlength=1.5,arrowinset=0]{<-}(80,20)(80,-5)
\rput[r](77.5,-3.75){$\otherinputt$}
\rput[r](77.5,55){$\otheroutpt$}
\rput(80,30){$\hee$}
\newrgbcolor{userLineColour}{1 0 0.4}
\pspolygon[linewidth=0.45,linecolor=userLineColour](10,105)(50,105)(50,5)(10,5)
\newrgbcolor{userLineColour}{0.2 0 1}
\pspolygon[linewidth=0.45,linecolor=userLineColour](-5,45)(115,45)(115,15)(-5,15)
\rput[br](113.75,16.25){$\parcompp$}
\rput[tr](48.75,103.12){$\seqcompp$}
\psline[linewidth=1](65,90)(65,70)
\psline[linewidth=1](95,70)(95,90)
\psline[linewidth=1,arrowsize=1.5 2,arrowlength=1.5,arrowinset=0]{<-}(80,115)(80,90)
\psline[linewidth=1](65,70)(95,70)
\psline[linewidth=1](65,90)(95,90)
\rput(80,80){\kee}
\rput[l](82.5,113.75){\outpttt}
\end{pspicture}

\end{split}
\eeq

The middle-two-interchange law $(g\circ f)\otimes(s\circ t) =  (g\otimes s)\circ(f\otimes t)$ corresponds to the two ways of reading the diagram: vertical-first and horizontal-first, marked by the red and the blue rectangle respectively. The string diagrams corresponding to the cartesian comonoids \eqref{eq:dataserv-text} are 
\beq\label{eq:dataserv}
\begin{split}
\newcommand{\AAh}{\scriptstyle A}
\newcommand{\ccopy}{\cmn}
\newcommand{\delete}{\scun}
\def\JPicScale{.4} 
\input{PICS/dataserv-1.tex}
\end{split}
\eeq
The equations that make them into commutative comonoids look like this:
\begin{alignat}{11}
\def\JPicScale{1} %%Created by jPicEdt 1.4.1_03: mixed JPIC-XML/LaTeX format
%%Sat May 15 12:50:08 GMT-10:00 2021
%%Begin JPIC-XML
%<?xml version="1.0" standalone="yes"?>
%<jpic x-min="-0.63" x-max="15" y-min="-10.62" y-max="7.5" auto-bounding="true">
%<multicurve fill-style= "none"
%	 stroke-width= "0.55"
%	 points= "(5.62,-0.62);(5.62,-0.62);(10.62,-5.62);(10.62,-5.62);(10.62,-5.62);(9.38,-5.62);
%	(9.38,-5.62);(9.38,-5.62);(10,-5.62);(10,-5.62)"
%	 />
%<ellipse p3= "(2.81,2.19)"
%	 p2= "(2.81,-0.94)"
%	 fill-style= "solid"
%	 p1= "(5.94,-0.94)"
%	 stroke-width= "0.55"
%	 closure= "open"
%	 angle-end= "0"
%	 angle-start= "0"
%	 />
%<ellipse p3= "(8.75,-3.75)"
%	 p2= "(8.75,-6.88)"
%	 fill-style= "solid"
%	 p1= "(11.88,-6.88)"
%	 stroke-width= "0.55"
%	 closure= "open"
%	 angle-end= "0"
%	 angle-start= "0"
%	 />
%<multicurve fill-style= "none"
%	 stroke-width= "0.55"
%	 points= "(15,7.5);(15,7.5);(15,5);(15,5);(15,5);(15,-0.63);
%	(15,-0.63);(15,-0.63);(11.25,-4.38);(11.25,-4.38)"
%	 />
%<multicurve fill-style= "none"
%	 stroke-width= "0.55"
%	 points= "(9.37,7.5);(9.37,7.5);(9.37,6.87);(9.37,6.87);(9.37,6.87);(9.38,5);
%	(9.38,5);(9.38,5);(5.63,1.25);(5.63,1.25)"
%	 />
%<multicurve fill-style= "none"
%	 stroke-width= "0.55"
%	 points= "(-0.63,7.5);(-0.63,7.5);(-0.63,6.87);(-0.63,6.87);(-0.63,6.87);(-0.62,5);
%	(-0.62,5);(-0.62,5);(3.13,1.25);(3.13,1.25)"
%	 />
%<multicurve fill-style= "none"
%	 stroke-width= "0.55"
%	 points= "(10,-6.88);(10,-6.88);(10,-10);(10,-10);(10,-10);(10,-10.62);
%	(10,-10.62);(10,-10.62);(10,-10);(10,-10)"
%	 />
%</jpic>
%%End JPIC-XML
%PSTricks content-type (pstricks.sty package needed)
%Add \usepackage{pstricks} in the preamble of your LaTeX file
%You can rescale the whole picture (to 80% for instance) by using the command \def\JPicScale{0.8}
\ifx\JPicScale\undefined\def\JPicScale{1}\fi
\psset{unit=\JPicScale mm}
\psset{linewidth=0.3,dotsep=1,hatchwidth=0.3,hatchsep=1.5,shadowsize=1,dimen=middle}
\psset{dotsize=0.7 2.5,dotscale=1 1,fillcolor=black}
\psset{arrowsize=1 2,arrowlength=1,arrowinset=0.25,tbarsize=0.7 5,bracketlength=0.15,rbracketlength=0.15}
\begin{pspicture}(0,0)(15,7.5)
\psline[linewidth=0.55](5.62,-0.62)
(10.62,-5.62)
(9.38,-5.62)(10,-5.62)
\rput{0}(4.38,0.62){\psellipse[linewidth=0.55,fillstyle=solid](0,0)(1.56,-1.57)}
\rput{0}(10.32,-5.31){\psellipse[linewidth=0.55,fillstyle=solid](0,0)(1.56,-1.57)}
\psline[linewidth=0.55](15,7.5)
(15,5)
(15,-0.63)(11.25,-4.38)
\psline[linewidth=0.55](9.37,7.5)
(9.37,6.87)
(9.38,5)(5.63,1.25)
\psline[linewidth=0.55](-0.63,7.5)
(-0.63,6.87)
(-0.62,5)(3.13,1.25)
\psline[linewidth=0.55](10,-6.88)
(10,-10)
(10,-10.62)(10,-10)
\end{pspicture}
\   &\ \ =\ \   \def\JPicScale{1} %%Created by jPicEdt 1.4.1_03: mixed JPIC-XML/LaTeX format
%%Sat May 15 12:54:01 GMT-10:00 2021
%%Begin JPIC-XML
%<?xml version="1.0" standalone="yes"?>
%<jpic x-min="-0.63" x-max="15.63" y-min="-10.62" y-max="8.12" auto-bounding="true">
%<ellipse p3= "(9.06,2.19)"
%	 p2= "(9.06,-0.94)"
%	 fill-style= "solid"
%	 p1= "(12.19,-0.94)"
%	 stroke-width= "0.55"
%	 closure= "open"
%	 angle-end= "0"
%	 angle-start= "0"
%	 />
%<multicurve fill-style= "none"
%	 stroke-width= "0.55"
%	 points= "(15.62,8.12);(15.62,8.12);(15.62,7.5);(15.62,7.5);(15.62,7.5);(15.63,5.62);
%	(15.63,5.62);(15.63,5.62);(11.88,1.87);(11.88,1.87)"
%	 />
%<multicurve fill-style= "none"
%	 stroke-width= "0.55"
%	 points= "(5.62,8.12);(5.62,8.12);(5.62,7.5);(5.62,7.5);(5.62,7.5);(5.63,5.62);
%	(5.63,5.62);(5.63,5.62);(9.38,1.87);(9.38,1.87)"
%	 />
%<ellipse p3= "(2.81,-4.06)"
%	 p2= "(2.81,-7.19)"
%	 fill-style= "solid"
%	 p1= "(5.94,-7.19)"
%	 stroke-width= "0.55"
%	 closure= "open"
%	 angle-end= "0"
%	 angle-start= "0"
%	 />
%<multicurve fill-style= "none"
%	 stroke-width= "0.55"
%	 points= "(-0.63,7.5);(-0.63,7.5);(-0.62,1.25);(-0.62,1.25);(-0.62,1.25);(-0.62,-0.63);
%	(-0.62,-0.63);(-0.62,-0.63);(3.13,-4.38);(3.13,-4.38)"
%	 />
%<multicurve fill-style= "none"
%	 stroke-width= "0.55"
%	 points= "(9.38,-0.62);(9.38,-0.62);(4.38,-5.62);(4.38,-5.62);(4.38,-5.62);(4.38,-5.62);
%	(4.38,-5.62);(4.38,-5.62);(4.38,-5.62);(4.38,-5.62)"
%	 />
%<multicurve fill-style= "none"
%	 stroke-width= "0.55"
%	 points= "(4.38,-6.88);(4.38,-6.88);(4.38,-10);(4.38,-10);(4.38,-10);(4.38,-10.62);
%	(4.38,-10.62);(4.38,-10.62);(4.38,-10);(4.38,-10)"
%	 />
%</jpic>
%%End JPIC-XML
%PSTricks content-type (pstricks.sty package needed)
%Add \usepackage{pstricks} in the preamble of your LaTeX file
%You can rescale the whole picture (to 80% for instance) by using the command \def\JPicScale{0.8}
\ifx\JPicScale\undefined\def\JPicScale{1}\fi
\psset{unit=\JPicScale mm}
\psset{linewidth=0.3,dotsep=1,hatchwidth=0.3,hatchsep=1.5,shadowsize=1,dimen=middle}
\psset{dotsize=0.7 2.5,dotscale=1 1,fillcolor=black}
\psset{arrowsize=1 2,arrowlength=1,arrowinset=0.25,tbarsize=0.7 5,bracketlength=0.15,rbracketlength=0.15}
\begin{pspicture}(0,0)(15.63,8.12)
\rput{0}(10.62,0.62){\psellipse[linewidth=0.55,fillstyle=solid](0,0)(1.57,-1.56)}
\psline[linewidth=0.55](15.62,8.12)
(15.62,7.5)
(15.63,5.62)(11.88,1.87)
\psline[linewidth=0.55](5.62,8.12)
(5.62,7.5)
(5.63,5.62)(9.38,1.87)
\rput{0}(4.38,-5.62){\psellipse[linewidth=0.55,fillstyle=solid](0,0)(1.57,-1.57)}
\psline[linewidth=0.55](-0.63,7.5)
(-0.62,1.25)
(-0.62,-0.63)(3.13,-4.38)
\pscustom[linewidth=0.55]{\psline(9.38,-0.62)(4.38,-5.62)
\psbezier(4.38,-5.62)(4.38,-5.62)(4.38,-5.62)
\psbezier(4.38,-5.62)(4.38,-5.62)(4.38,-5.62)
}
\psline[linewidth=0.55](4.38,-6.88)
(4.38,-10)
(4.38,-10.62)(4.38,-10)
\end{pspicture}
 &&\qquad\qquad&&& 
\def\JPicScale{1} 
\input{PICS/cun-left-1.tex}  & \ = \ &\ \, \def\JPicScale{1} %%Created by jPicEdt 1.4.1_03: mixed JPIC-XML/LaTeX format
%%Sat May 15 12:55:13 GMT-10:00 2021
%%Begin JPIC-XML
%<?xml version="1.0" standalone="yes"?>
%<jpic x-min="0" x-max="0" y-min="-10" y-max="8.74" auto-bounding="true">
%<multicurve fill-style= "none"
%	 stroke-width= "0.55"
%	 points= "(0,8.74);(0,8.74);(0,-9.38);(0,-9.38);(0,-9.38);(0,-10);
%	(0,-10);(0,-10);(0,-9.38);(0,-9.38)"
%	 />
%</jpic>
%%End JPIC-XML
%PSTricks content-type (pstricks.sty package needed)
%Add \usepackage{pstricks} in the preamble of your LaTeX file
%You can rescale the whole picture (to 80% for instance) by using the command \def\JPicScale{0.8}
\ifx\JPicScale\undefined\def\JPicScale{1}\fi
\psset{unit=\JPicScale mm}
\psset{linewidth=0.3,dotsep=1,hatchwidth=0.3,hatchsep=1.5,shadowsize=1,dimen=middle}
\psset{dotsize=0.7 2.5,dotscale=1 1,fillcolor=black}
\psset{arrowsize=1 2,arrowlength=1,arrowinset=0.25,tbarsize=0.7 5,bracketlength=0.15,rbracketlength=0.15}
\begin{pspicture}(0,0)(0,8.74)
\psline[linewidth=0.55](0,8.74)
(0,-9.38)
(0,-10)(0,-9.38)
\end{pspicture}
 &\ \, = \ \ &  \def\JPicScale{1} %%Created by jPicEdt 1.4.1_03: mixed JPIC-XML/LaTeX format
%%Sat May 15 12:56:33 GMT-10:00 2021
%%Begin JPIC-XML
%<?xml version="1.0" standalone="yes"?>
%<jpic x-min="-0.63" x-max="12.19" y-min="-10.62" y-max="8.12" auto-bounding="true">
%<ellipse p3= "(9.06,2.81)"
%	 p2= "(9.06,-0.31)"
%	 fill-style= "solid"
%	 p1= "(12.19,-0.31)"
%	 stroke-width= "0.55"
%	 closure= "open"
%	 angle-end= "0"
%	 angle-start= "0"
%	 />
%<ellipse p3= "(2.81,-3.44)"
%	 p2= "(2.81,-6.56)"
%	 fill-style= "solid"
%	 p1= "(5.94,-6.56)"
%	 stroke-width= "0.55"
%	 closure= "open"
%	 angle-end= "0"
%	 angle-start= "0"
%	 />
%<multicurve fill-style= "none"
%	 stroke-width= "0.55"
%	 points= "(4.38,-6.25);(4.38,-6.25);(4.38,-10);(4.38,-10);(4.38,-10);(4.38,-10.62);
%	(4.38,-10.62);(4.38,-10.62);(4.38,-10);(4.38,-10)"
%	 />
%<multicurve fill-style= "none"
%	 stroke-width= "0.55"
%	 points= "(-0.63,8.12);(-0.63,8.12);(-0.62,1.87);(-0.62,1.87);(-0.62,1.87);(-0.62,0);
%	(-0.62,0);(-0.62,0);(3.13,-3.75);(3.13,-3.75)"
%	 />
%<multicurve fill-style= "none"
%	 stroke-width= "0.55"
%	 points= "(9.38,0);(9.38,0);(4.38,-5);(4.38,-5);(4.38,-5);(4.38,-5);
%	(4.38,-5);(4.38,-5);(4.38,-5);(4.38,-5)"
%	 />
%</jpic>
%%End JPIC-XML
%PSTricks content-type (pstricks.sty package needed)
%Add \usepackage{pstricks} in the preamble of your LaTeX file
%You can rescale the whole picture (to 80% for instance) by using the command \def\JPicScale{0.8}
\ifx\JPicScale\undefined\def\JPicScale{1}\fi
\psset{unit=\JPicScale mm}
\psset{linewidth=0.3,dotsep=1,hatchwidth=0.3,hatchsep=1.5,shadowsize=1,dimen=middle}
\psset{dotsize=0.7 2.5,dotscale=1 1,fillcolor=black}
\psset{arrowsize=1 2,arrowlength=1,arrowinset=0.25,tbarsize=0.7 5,bracketlength=0.15,rbracketlength=0.15}
\begin{pspicture}(0,0)(12.19,8.12)
\rput{0}(10.62,1.25){\psellipse[linewidth=0.55,fillstyle=solid](0,0)(1.56,-1.56)}
\rput{0}(4.38,-5){\psellipse[linewidth=0.55,fillstyle=solid](0,0)(1.57,-1.56)}
\psline[linewidth=0.55](4.38,-6.25)
(4.38,-10)
(4.38,-10.62)(4.38,-10)
\psline[linewidth=0.55](-0.63,8.12)
(-0.62,1.87)
(-0.62,0)(3.13,-3.75)
\pscustom[linewidth=0.55]{\psline(9.38,0)(4.38,-5)
\psbezier(4.38,-5)(4.38,-5)(4.38,-5)
\psbezier(4.38,-5)(4.38,-5)(4.38,-5)
}
\end{pspicture}
&&\qquad\qquad&&&
\def\JPicScale{1} \input{PICS/cmn.tex} &\   = \   \def\JPicScale{1} %%Created by jPicEdt 1.4.1_03: mixed JPIC-XML/LaTeX format
%%Thu Mar 16 14:52:39 GMT-10:00 2023
%%Begin JPIC-XML
%<?xml version="1.0" standalone="yes"?>
%<jpic x-min="0" x-max="5" y-min="-10" y-max="7.5" auto-bounding="true">
%<ellipse fill-style= "solid"
%	 stroke-width= "0.5"
%	 p3= "(0.93,-3.44)"
%	 p2= "(0.93,-6.56)"
%	 p1= "(4.06,-6.56)"
%	 closure= "open"
%	 angle-end= "0"
%	 angle-start= "0"
%	 />
%<multicurve fill-style= "none"
%	 stroke-width= "0.5"
%	 points= "(2.5,-5);(2.5,-5);(2.5,-10);(2.5,-10)"
%	 />
%<multicurve fill-style= "none"
%	 stroke-width= "0.5"
%	 points= "(5,-2.5);(5,-2.5);(2.5,-5);(2.5,-5)"
%	 />
%<multicurve fill-style= "none"
%	 stroke-width= "0.5"
%	 points= "(0,-2.5);(0,-2.5);(2.5,-5);(2.5,-5)"
%	 />
%<multicurve fill-style= "none"
%	 stroke-width= "0.5"
%	 points= "(5,0);(5,0);(5,-2.5);(5,-2.5)"
%	 />
%<multicurve fill-style= "none"
%	 stroke-width= "0.5"
%	 points= "(0,0);(0,0);(0,-2.5);(0,-2.5)"
%	 />
%<multicurve fill-style= "none"
%	 stroke-width= "0.5"
%	 points= "(0,5);(0,5);(5,0);(5,0)"
%	 />
%<multicurve fill-style= "none"
%	 stroke-width= "0.5"
%	 points= "(0,0);(0,0);(5,5);(5,5)"
%	 />
%<multicurve fill-style= "none"
%	 stroke-width= "0.5"
%	 points= "(5,7.5);(5,7.5);(5,5);(5,5)"
%	 />
%<multicurve fill-style= "none"
%	 stroke-width= "0.5"
%	 points= "(0,7.5);(0,7.5);(0,5);(0,5)"
%	 />
%</jpic>
%%End JPIC-XML
%PSTricks content-type (pstricks.sty package needed)
%Add \usepackage{pstricks} in the preamble of your LaTeX file
%You can rescale the whole picture (to 80% for instance) by using the command \def\JPicScale{0.8}
\ifx\JPicScale\undefined\def\JPicScale{1}\fi
\psset{unit=\JPicScale mm}
\psset{linewidth=0.3,dotsep=1,hatchwidth=0.3,hatchsep=1.5,shadowsize=1,dimen=middle}
\psset{dotsize=0.7 2.5,dotscale=1 1,fillcolor=black}
\psset{arrowsize=1 2,arrowlength=1,arrowinset=0.25,tbarsize=0.7 5,bracketlength=0.15,rbracketlength=0.15}
\begin{pspicture}(0,0)(5,7.5)
\rput{0}(2.49,-5){\psellipse[linewidth=0.5,fillstyle=solid](0,0)(1.56,-1.56)}
\psline[linewidth=0.5](2.5,-5)(2.5,-10)
\psline[linewidth=0.5](5,-2.5)(2.5,-5)
\psline[linewidth=0.5](0,-2.5)(2.5,-5)
\psline[linewidth=0.5](5,0)(5,-2.5)
\psline[linewidth=0.5](0,0)(0,-2.5)
\psline[linewidth=0.5](0,5)(5,0)
\psline[linewidth=0.5](0,0)(5,5)
\psline[linewidth=0.5](5,7.5)(5,5)
\psline[linewidth=0.5](0,7.5)(0,5)
\end{pspicture}
 \hspace{2em}
\label{eq:comonoid}%\\%[1ex]
\notag
\end{alignat}
\smallskip

\paragraph{State parametrization and updating.} Products $A\otimes B$ denote a space where $A$ and $B$ but do not interfere. In a diagram, they are just parallel strings. Since the product states from the space $X\otimes A$ do not interfere, a transition $g\colon X\otimes A \to B$ can be viewed as $X$-parametrized family $g_{x}\colon A\to B$, as it was viewed in Sec.~\ref{Sec:Intro}. Since the product states from $X\otimes B$ also remain separate, a transition $q\colon X\otimes A\to X\otimes B$ can be viewed as $X$-updating process, as it was also viewed in Sec.~\ref{Sec:Intro}. The corresponding string diagrams are 
\beq\label{eq:gq}
\begin{split}
\newcommand{\fee}{g}
\newcommand{\qee}{q}
\newcommand{\Aee}{\scriptstyle X}
\newcommand{\Bee}{\scriptstyle A}
\newcommand{\Cee}{\scriptstyle B}
\def\JPicScale{.25}
%%Created by jPicEdt 1.4.1_03: mixed JPIC-XML/LaTeX format
%%Fri Mar 17 19:08:11 GMT-10:00 2023
%%Begin JPIC-XML
%<?xml version="1.0" standalone="yes"?>
%<jpic x-min="0" x-max="300" y-min="0" y-max="120" auto-bounding="true">
%<multicurve fill-style= "none"
%	 points= "(80,80);(80,80);(80,40);(80,40)"
%	 />
%<multicurve fill-style= "none"
%	 points= "(80,40);(80,40);(0,40);(0,40)"
%	 />
%<multicurve fill-style= "none"
%	 points= "(0,80);(0,80);(80,80);(80,80)"
%	 />
%<multicurve fill-style= "none"
%	 points= "(60,40);(60,40);(60,0);(60,0)"
%	 />
%<multicurve fill-style= "none"
%	 points= "(40,120);(40,120);(40,80);(40,80)"
%	 />
%<multicurve fill-style= "none"
%	 points= "(0,80);(0,80);(0,40);(0,40)"
%	 />
%<multicurve fill-style= "none"
%	 points= "(20,40);(20,40);(20,0);(20,0)"
%	 />
%<text text-vert-align= "center-v"
%	 fill-style= "none"
%	 anchor-point= "(45,117.5)"
%	 text-frame= "noframe"
%	 text-hor-align= "left"
%	 >
%$\Cee$
%</text>
%<text text-vert-align= "center-v"
%	 fill-style= "none"
%	 anchor-point= "(15,2.5)"
%	 text-frame= "noframe"
%	 text-hor-align= "right"
%	 >
%$\Aee$
%</text>
%<text text-vert-align= "center-v"
%	 fill-style= "none"
%	 anchor-point= "(55,2.5)"
%	 text-frame= "noframe"
%	 text-hor-align= "right"
%	 >
%$\Bee$
%</text>
%<text text-vert-align= "center-v"
%	 fill-style= "none"
%	 anchor-point= "(40,60)"
%	 text-frame= "noframe"
%	 text-hor-align= "center-h"
%	 >
%$\fee$
%</text>
%<multicurve fill-style= "none"
%	 points= "(300,80);(300,80);(300,40);(300,40)"
%	 />
%<multicurve fill-style= "none"
%	 points= "(300,40);(300,40);(220,40);(220,40)"
%	 />
%<multicurve fill-style= "none"
%	 points= "(220,80);(220,80);(300,80);(300,80)"
%	 />
%<multicurve fill-style= "none"
%	 points= "(280,40);(280,40);(280,0);(280,0)"
%	 />
%<multicurve fill-style= "none"
%	 points= "(280,120);(280,120);(280,80);(280,80)"
%	 />
%<multicurve fill-style= "none"
%	 points= "(220,80);(220,80);(220,40);(220,40)"
%	 />
%<multicurve fill-style= "none"
%	 points= "(240,40);(240,40);(240,0);(240,0)"
%	 />
%<text text-vert-align= "center-v"
%	 fill-style= "none"
%	 anchor-point= "(285,117.5)"
%	 text-frame= "noframe"
%	 text-hor-align= "left"
%	 >
%$\Cee$
%</text>
%<text text-vert-align= "center-v"
%	 fill-style= "none"
%	 anchor-point= "(235,2.5)"
%	 text-frame= "noframe"
%	 text-hor-align= "right"
%	 >
%$\Aee$
%</text>
%<text text-vert-align= "center-v"
%	 fill-style= "none"
%	 anchor-point= "(275,2.5)"
%	 text-frame= "noframe"
%	 text-hor-align= "right"
%	 >
%$\Bee$
%</text>
%<multicurve fill-style= "none"
%	 points= "(240,120);(240,120);(240,80);(240,80)"
%	 />
%<text text-vert-align= "center-v"
%	 fill-style= "none"
%	 anchor-point= "(245,117.5)"
%	 text-frame= "noframe"
%	 text-hor-align= "left"
%	 >
%$\Aee$
%</text>
%<text text-vert-align= "center-v"
%	 fill-style= "none"
%	 anchor-point= "(260,60)"
%	 text-frame= "noframe"
%	 text-hor-align= "center-h"
%	 >
%$\qee$
%</text>
%</jpic>
%%End JPIC-XML
%LaTeX-picture environment using emulated lines and arcs
%You can rescale the whole picture (to 80% for instance) by using the command \def\JPicScale{0.8}
\ifx\JPicScale\undefined\def\JPicScale{1}\fi
\unitlength \JPicScale mm
\begin{picture}(300,120)(0,0)
\linethickness{0.3mm}
\put(80,40){\line(0,1){40}}
\linethickness{0.3mm}
\put(0,40){\line(1,0){80}}
\linethickness{0.3mm}
\put(0,80){\line(1,0){80}}
\linethickness{0.3mm}
\put(60,0){\line(0,1){40}}
\linethickness{0.3mm}
\put(40,80){\line(0,1){40}}
\linethickness{0.3mm}
\put(0,40){\line(0,1){40}}
\linethickness{0.3mm}
\put(20,0){\line(0,1){40}}
\put(45,117.5){\makebox(0,0)[cl]{$\Cee$}}

\put(15,2.5){\makebox(0,0)[cr]{$\Aee$}}

\put(55,2.5){\makebox(0,0)[cr]{$\Bee$}}

\put(40,60){\makebox(0,0)[cc]{$\fee$}}

\linethickness{0.3mm}
\put(300,40){\line(0,1){40}}
\linethickness{0.3mm}
\put(220,40){\line(1,0){80}}
\linethickness{0.3mm}
\put(220,80){\line(1,0){80}}
\linethickness{0.3mm}
\put(280,0){\line(0,1){40}}
\linethickness{0.3mm}
\put(280,80){\line(0,1){40}}
\linethickness{0.3mm}
\put(220,40){\line(0,1){40}}
\linethickness{0.3mm}
\put(240,0){\line(0,1){40}}
\put(285,117.5){\makebox(0,0)[cl]{$\Cee$}}

\put(235,2.5){\makebox(0,0)[cr]{$\Aee$}}

\put(275,2.5){\makebox(0,0)[cr]{$\Bee$}}

\linethickness{0.3mm}
\put(240,80){\line(0,1){40}}
\put(245,117.5){\makebox(0,0)[cl]{$\Aee$}}

\put(260,60){\makebox(0,0)[cc]{$\qee$}}

\end{picture}
 
\end{split}
\eeq

\paragraph{Shape conventions.} While the boxes in \eqref{eq:godement} and \eqref{eq:gq} are rectangular, the cartesian ``boxes'' in \eqref{eq:dataserv} are reduced to black dots. In general, the boxes denoting general transitions can vary in shape, and fixed shapes are used for generic notations. E.g., the interpreters, introduced in \eqref{eq:uev} below, are denoted by trapezoids, and the interpretations, that are fed to them, by triangles. A black dot on a box signals that it is cartesian, i.e. belongs to $\tot\UUU$.

\paragraph{Projections.} Using the cartesian structure from \eqref{eq:dataserv}, a state updating transition $q$ can still be decomposed like before 
\beq
\sta q = \left(X\otimes A\tto q X\times B\tto{\id\otimes \scun} X\right)\quad \out q = \left(X\otimes A\tto q X\otimes B\tto{\scun\otimes \id} B\right)
\eeq
In general, however, although the transitions $u\colon Z\to U$ and $v\colon Z\to V$ can be paired into $<u,v> =(Z\tto\cmn Z\otimes Z \tto{u\otimes v} U\otimes V)$, the pair $<\sta q, \out q>$ may not be equal to $q$ in the universe $\UUU$, unless it happens to be cloneable, in the sense that it commutes with $\cmn$.