\usepackage{etex}

\usepackage{tikz}
\usetikzlibrary{cd,decorations,shapes}
\newenvironment{tikzar}[1][]{{}\kern-4pt\begin{tikzcd}[ampersand replacement=\&,#1]}%
{\end{tikzcd}\kern-4pt{}}



\newcommand{\arrowequal}[1]{\arrow[-,shift left=1pt]{#1}\arrow[-,shift right=1pt]{#1}}
%\newcommand{\arrowpb}{\arrow[phantom]{dr}[near start]{\lrcorner}}

\usepackage{graphicx}
\usepackage{pstricks}

\usepackage{bbold}
\usepackage{bbm}
\usepackage%[only,sslash]
{stmaryrd}


\usepackage[USenglish]{babel}
\usepackage{amssymb}
 \usepackage{amstext}
 \usepackage{amsmath}
\usepackage{amsthm}
%\usepackage{txfonts}  

\usepackage{enumerate} 

%


%%%%%%%%%%%%%%%% \input{prooftree}
%SYNTAX:
%%
%%      \prooftree
%%              hyp1            produces:
%%              hyp2
%%              hyp3            hyp1    hyp2    hyp3
%%      \justifies              -------------------- rulename
%%              concl                   concl
%%      \thickness=0.08em
%%      \shiftright 2em
%%      \using
%%              rulename
%%      \endprooftree
%%
%% where the hypotheses may be similar structures or just formulae.
%%
%% To get a vertical string of dots instead of the proof rule, do
%%
%%      \prooftree                      which produces:
%%              [hyp]
%%      \using                                  [hyp]
%%              name                              .
%%      \proofdotseparation=1.2ex                 .name
%%      \proofdotnumber=4                         .
%%      \leadsto                                  .
%%              concl                           concl
%%      \endprooftree
%%
%% Within a prooftree, \[ and \] may be used instead of \prooftree and
%% \endprooftree; this is not permitted at the outer level because it
%% conflicts with LaTeX. Also,
%%      \Justifies
%% produces a double line. In LaTeX you can use \begin{prooftree} and
%% \end{prootree} at the outer level (however this will not work for the inner
%% levels, but in any case why would you want to be so verbose?).
%%
%% All of of the keywords except \prooftree and \endprooftree are optional
%% and may appear in any order. They may also be combined in \newcommand's
%% eg "\def\Cut{\using\sf cut\thickness.08em\justifies}" with the abbreviation
%% "\prooftree hyp1 hyp2 \Cut \concl \endprooftree". This is recommended and
%% some standard abbreviations will be found at the end of this file.
%%
%% \thickness specifies the breadth of the rule in any units, although
%% font-relative units such as "ex" or "em" are preferable.
%% It may optionally be followed by "=".
%% \proofrulebreadth=.08em or \setlength\proofrulebreadth{.08em} may also be
%% used either in place of \thickness or globally; the default is 0.04em.
%% \proofdotseparation and \proofdotnumber control the size of the
%% string of dots
%%
%% If proof trees and formulae are mixed, some explicit spacing is needed,
%% but don't put anything to the left of the left-most (or the right of
%% the right-most) hypothesis, or put it in braces, because this will cause
%% the indentation to be lost.
%%
%% By default the conclusion is centered wrt the left-most and right-most
%% immediate hypotheses (not their proofs); \shiftright or \shiftleft moves
%% it relative to this position. (Not sure about this specification or how
%% it should affect spreading of proof tree.)
%
% global assignments to dimensions seem to have the effect of stretching
% diagrams horizontally.
%
%%==========================================================================

\def\introrule{{\cal I}}\def\elimrule{{\cal E}}%%
\def\andintro{\using{\land}\introrule\justifies}%%
\def\impelim{\using{\Rightarrow}\elimrule\justifies}%%
\def\allintro{\using{\forall}\introrule\justifies}%%
\def\allelim{\using{\forall}\elimrule\justifies}%%
\def\falseelim{\using{\bot}\elimrule\justifies}%%
\def\existsintro{\using{\exists}\introrule\justifies}%%

%% #1 is meant to be 1 or 2 for the first or second formula
\def\andelim#1{\using{\land}#1\elimrule\justifies}%%
\def\orintro#1{\using{\lor}#1\introrule\justifies}%%

%% #1 is meant to be a label corresponding to the discharged hypothesis/es
\def\impintro#1{\using{\Rightarrow}\introrule_{#1}\justifies}%%
\def\orelim#1{\using{\lor}\elimrule_{#1}\justifies}%%
\def\existselim#1{\using{\exists}\elimrule_{#1}\justifies}

%%==========================================================================

\newdimen\proofrulebreadth \proofrulebreadth=.05em
\newdimen\proofdotseparation \proofdotseparation=1.25ex
\newdimen\proofrulebaseline \proofrulebaseline=2ex
\newcount\proofdotnumber \proofdotnumber=3
\let\then\relax
\def\hfi{\hskip0pt plus.0001fil}
\mathchardef\squigto="3A3B
%
% flag where we are
\newif\ifinsideprooftree\insideprooftreefalse
\newif\ifonleftofproofrule\onleftofproofrulefalse
\newif\ifproofdots\proofdotsfalse
\newif\ifdoubleproof\doubleprooffalse
\let\wereinproofbit\relax
%
% dimensions and boxes of bits
\newdimen\shortenproofleft
\newdimen\shortenproofright
\newdimen\proofbelowshift
\newbox\proofabove
\newbox\proofbelow
\newbox\proofrulename
%
% miscellaneous commands for setting values
\def\shiftproofbelow{\let\next\relax\afterassignment\setshiftproofbelow\dimen0 }
\def\shiftproofbelowneg{\def\next{\multiply\dimen0 by-1 }%
\afterassignment\setshiftproofbelow\dimen0 }
\def\setshiftproofbelow{\next\proofbelowshift=\dimen0 }
\def\setproofrulebreadth{\proofrulebreadth}

%=============================================================================
\def\prooftree{% NESTED ZERO (\ifonleftofproofrule)
%
% first find out whether we're at the left-hand end of a proof rule
\ifnum  \lastpenalty=1
\then   \unpenalty
\else   \onleftofproofrulefalse
\fi
%
% some space on left (except if we're on left, and no infinity for outermost)
\ifonleftofproofrule
\else   \ifinsideprooftree
        \then   \hskip.5em plus1fil
        \fi
\fi
%
% begin our proof tree environment
\bgroup% NESTED ONE (\proofbelow, \proofrulename, \proofabove,
%               \shortenproofleft, \shortenproofright, \proofrulebreadth)
\setbox\proofbelow=\hbox{}\setbox\proofrulename=\hbox{}%
\let\justifies\proofover\let\leadsto\proofoverdots\let\Justifies\proofoverdbl
\let\using\proofusing\let\[\prooftree
\ifinsideprooftree\let\]\endprooftree\fi
\proofdotsfalse\doubleprooffalse
\let\thickness\setproofrulebreadth
\let\shiftright\shiftproofbelow \let\shift\shiftproofbelow
\let\shiftleft\shiftproofbelowneg
\let\ifwasinsideprooftree\ifinsideprooftree
\insideprooftreetrue
%
% now begin to set the top of the rule (definitions local to it)
\setbox\proofabove=\hbox\bgroup$\displaystyle % NESTED TWO
\let\wereinproofbit\prooftree
%
% these local variables will be copied out:
\shortenproofleft=0pt \shortenproofright=0pt \proofbelowshift=0pt
%
% flags to enable inner proof tree to detect if on left:
\onleftofproofruletrue\penalty1
}

%=============================================================================
% end whatever box and copy crucial values out of it
\def\eproofbit{% NESTED TWO
%
% various hacks applicable to hypothesis list 
\ifx    \wereinproofbit\prooftree
\then   \ifcase \lastpenalty
        \then   \shortenproofright=0pt  % 0: some other object, no indentation
        \or     \unpenalty\hfil         % 1: empty hypotheses, just glue
        \or     \unpenalty\unskip       % 2: just had a tree, remove glue
        \else   \shortenproofright=0pt  % eh?
        \fi
\fi
%
% pass out crucial values from scope
\global\dimen0=\shortenproofleft
\global\dimen1=\shortenproofright
\global\dimen2=\proofrulebreadth
\global\dimen3=\proofbelowshift
\global\dimen4=\proofdotseparation
\global\count255=\proofdotnumber
%
% end the box
$\egroup  % NESTED ONE
%
% restore the values
\shortenproofleft=\dimen0
\shortenproofright=\dimen1
\proofrulebreadth=\dimen2
\proofbelowshift=\dimen3
\proofdotseparation=\dimen4
\proofdotnumber=\count255
}

%=============================================================================
\def\proofover{% NESTED TWO
\eproofbit % NESTED ONE
\setbox\proofbelow=\hbox\bgroup % NESTED TWO
\let\wereinproofbit\proofover
$\displaystyle
}%
%
%=============================================================================
\def\proofoverdbl{% NESTED TWO
\eproofbit % NESTED ONE
\doubleprooftrue
\setbox\proofbelow=\hbox\bgroup % NESTED TWO
\let\wereinproofbit\proofoverdbl
$\displaystyle
}%
%
%=============================================================================
\def\proofoverdots{% NESTED TWO
\eproofbit % NESTED ONE
\proofdotstrue
\setbox\proofbelow=\hbox\bgroup % NESTED TWO
\let\wereinproofbit\proofoverdots
$\displaystyle
}%
%
%=============================================================================
\def\proofusing{% NESTED TWO
\eproofbit % NESTED ONE
\setbox\proofrulename=\hbox\bgroup % NESTED TWO
\let\wereinproofbit\proofusing
\kern0.3em$
}

%=============================================================================
\def\endprooftree{% NESTED TWO
\eproofbit % NESTED ONE
% \dimen0 =     length of proof rule
% \dimen1 =     indentation of conclusion wrt rule
% \dimen2 =     new \shortenproofleft, ie indentation of conclusion
% \dimen3 =     new \shortenproofright, ie
%                space on right of conclusion to end of tree
% \dimen4 =     space on right of conclusion below rule
  \dimen5 =0pt% spread of hypotheses
% \dimen6, \dimen7 = height & depth of rule
%
% length of rule needed by proof above
\dimen0=\wd\proofabove \advance\dimen0-\shortenproofleft
\advance\dimen0-\shortenproofright
%
% amount of spare space below
\dimen1=.5\dimen0 \advance\dimen1-.5\wd\proofbelow
\dimen4=\dimen1
\advance\dimen1\proofbelowshift \advance\dimen4-\proofbelowshift
%
% conclusion sticks out to left of immediate hypotheses
\ifdim  \dimen1<0pt
\then   \advance\shortenproofleft\dimen1
        \advance\dimen0-\dimen1
        \dimen1=0pt
%       now it sticks out to left of tree!
        \ifdim  \shortenproofleft<0pt
        \then   \setbox\proofabove=\hbox{%
                        \kern-\shortenproofleft\unhbox\proofabove}%
                \shortenproofleft=0pt
        \fi
\fi
%
% and to the right
\ifdim  \dimen4<0pt
\then   \advance\shortenproofright\dimen4
        \advance\dimen0-\dimen4
        \dimen4=0pt
\fi
%
% make sure enough space for label
\ifdim  \shortenproofright<\wd\proofrulename
\then   \shortenproofright=\wd\proofrulename
\fi
%
% calculate new indentations
\dimen2=\shortenproofleft \advance\dimen2 by\dimen1
\dimen3=\shortenproofright\advance\dimen3 by\dimen4
%
% make the rule or dots, with name attached
\ifproofdots
\then
        \dimen6=\shortenproofleft \advance\dimen6 .5\dimen0
        \setbox1=\vbox to\proofdotseparation{\vss\hbox{$\cdot$}\vss}%
        \setbox0=\hbox{%
                \advance\dimen6-.5\wd1
                \kern\dimen6
                $\vcenter to\proofdotnumber\proofdotseparation
                        {\leaders\box1\vfill}$%
                \unhbox\proofrulename}%
\else   \dimen6=\fontdimen22\the\textfont2 % height of maths axis
        \dimen7=\dimen6
        \advance\dimen6by.5\proofrulebreadth
        \advance\dimen7by-.5\proofrulebreadth
        \setbox0=\hbox{%
                \kern\shortenproofleft
                \ifdoubleproof
                \then   \hbox to\dimen0{%
                        $\mathsurround0pt\mathord=\mkern-6mu%
                        \cleaders\hbox{$\mkern-2mu=\mkern-2mu$}\hfill
                        \mkern-6mu\mathord=$}%
                \else   \vrule height\dimen6 depth-\dimen7 width\dimen0
                \fi
                \unhbox\proofrulename}%
        \ht0=\dimen6 \dp0=-\dimen7
\fi
%
% set up to centre outermost tree only
\let\doll\relax
\ifwasinsideprooftree
\then   \let\VBOX\vbox
\else   \ifmmode\else$\let\doll=$\fi
        \let\VBOX\vcenter
\fi
% this \vbox or \vcenter is the actual output:
\VBOX   {\baselineskip\proofrulebaseline \lineskip.2ex
        \expandafter\lineskiplimit\ifproofdots0ex\else-0.6ex\fi
        \hbox   spread\dimen5   {\hfi\unhbox\proofabove\hfi}%
        \hbox{\box0}%
        \hbox   {\kern\dimen2 \box\proofbelow}}\doll%
%
% pass new indentations out of scope
\global\dimen2=\dimen2
\global\dimen3=\dimen3
\egroup % NESTED ZERO
\ifonleftofproofrule
\then   \shortenproofleft=\dimen2
\fi
\shortenproofright=\dimen3
%
% some space on right and flag we've just made a tree
\onleftofproofrulefalse
\ifinsideprooftree
\then   \hskip.5em plus 1fil \penalty2
\fi
}
%%%%%%%%%%%%%%% END prooftree



    
\renewcommand{\to}{\xrightarrow{}}%{\longrightarrow}
\newcommand{\ot}{\xleftarrow{}}%{\longleftarrow}
\newcommand{\tto}[1]{\xrightarrow{#1}}
\newcommand{\oot}[1]{\xleftarrow{#1}}
\newcommand{\ttto}[2]{\xrightarrow[#2]{#1}}
\newcommand{\ooot}[2]{\xleftarrow[#2]{#1}}
\newcommand{\mono}{\rightarrowtail}
\newcommand{\epi}{\twoheadrightarrow}
\newcommand{\ipe}{\twoheadlefttarrow}
\newcommand{\inclusion}{\hookrightarrow}
\newcommand{\mmono}[1]{\stackrel{#1}\rightarrowtail}
\newcommand{\eepi}[1]{\stackrel{#1}\twoheadrightarrow}
\newcommand{\iipe}[1]{\stackrel{#1}\twoheadleftarrow}
\newcommand{\iinclusion}[1]{\stackrel{#1}\hookrightarrow}
\renewcommand{\mapsfrom}{\mathrel{\reflectbox{\ensuremath{\mapsto}}}}
\newcommand{\pfn}{\rightharpoonup}
\newcommand{\ppfn}[1]{\stackrel{#1}\rightharpoonup}

\newcommand{\xxp}[2]{\left[{#1},{#2}\right]}
\newcommand{\xxpp}[2]{\xxp{{#1}^+}{#2}}
%\newcommand{\xxpp}[2]{\llbracket{#1}^+,{#2}\rrbracket}

\newcommand{\tleft}{\triangleleft}%{-}%
\newcommand{\tright}{\triangleright}%{+}%
\newcommand{\tstall}{{\scriptstyle \Box}}%{{\rm o}}%



\newcommand{\WP}{\mbox{\Large $\wp$}}
\newcommand{\WPf}{\mbox{\Large $\wp$}_{f}}
\newcommand{\Set}{\mathsf{Set}}
\newcommand{\Pfn}{{\sf Pfn}}
\newcommand{\Vect}{{\sf Vec}}
\newcommand{\FSet}{{\sf FSet}}
\newcommand{\FVec}{{\sf FVec}}
\newcommand{\Cat}{{\sf Cat}}
\newcommand{\Pos}{{\sf Pos}}
\newcommand{\CAT}{{\sf CAT}}
\newcommand{\Rel}{{\sf Rel}}
\newcommand{\FRel}{{\sf FRel}}
\newcommand{\FHilb}{{\sf FHilb}}
\newcommand{\Hilb}{{\sf Hilb}}
\newcommand{\im}{{\rm im}}
\newcommand{\id}{{\rm id}}
\newcommand{\blank}{\sqcup}

%\newcommand{\cata}[1]{\llparenthesis {#1} \rrparenthesis}
%\newcommand{\ana}[1]{\left\llbracket {#1} \right\rrbracket}
%\newcommand{\run}[1]{\left\{ {#1} \right\}}
%\newcommand{\Run}[1]{\left\{\!\lvert {#1} \rvert\!\right\}}
%\newcommand{\RRun}{\left\{\!\lvert \, \rvert\!\right\}}

\newcommand{\AAA}{{\cal A}}
\newcommand{\BBB}{{\cal B}}
\newcommand{\CCC}{{\cal C}}
\newcommand{\DDD}{{\cal D}}
\newcommand{\EEE}{{\cal E}}
\newcommand{\FFF}{{\cal F}}
\newcommand{\GGG}{{\cal G}}
\newcommand{\HHH}{{\cal H}}
\newcommand{\III}{{\cal I}}
\newcommand{\JJJ}{{\cal J}}
\newcommand{\KKK}{{\cal K}}
\newcommand{\LLL}{{\cal L}}
\newcommand{\MMM}{{\cal M}}
\newcommand{\NNN}{{\cal N}}
\newcommand{\OOO}{{\cal O}}
\newcommand{\PPP}{{\cal P}}
\newcommand{\QQQ}{{\cal Q}}
\newcommand{\RRR}{{\cal R}}
\newcommand{\SSS}{{\cal S}}
\newcommand{\TTT}{{\cal T}}
\newcommand{\UUU}{{\cal U}}
\newcommand{\VVV}{{\cal V}}
\newcommand{\WWW}{{\cal W}}
\newcommand{\XXX}{{\cal X}}
\newcommand{\YYY}{{\cal Y}}
\newcommand{\ZZZ}{{\cal Z}}
\renewcommand{\Bbb}{\mathbb}
\newcommand{\AAa}{{\Bbb A}}
\newcommand{\BBb}{{\Bbb B}}
\newcommand{\CCc}{{\Bbb C}}
\newcommand{\DDd}{{\Bbb D}}
\newcommand{\EEe}{{\Bbb E}}
\newcommand{\FFf}{{\Bbb F}}
\newcommand{\GGg}{{\Bbb G}}
\newcommand{\HHh}{{\Bbb H}}
\newcommand{\IIi}{{\Bbb I}}
\newcommand{\JJj}{{\Bbb J}}
\newcommand{\KKk}{{\Bbb K}}
\newcommand{\LLl}{{\Bbb L}}
\newcommand{\MMm}{{\Bbb M}}
\newcommand{\NNn}{{\Bbb N}}
\newcommand{\OOo}{{\Bbb O}}
\newcommand{\PPp}{{\Bbb P}}
\newcommand{\QQq}{{\Bbb Q}}
\newcommand{\RRr}{{\Bbb R}}
\newcommand{\SSs}{{\Bbb S}}
\newcommand{\TTt}{{\Bbb T}}
\newcommand{\UUu}{{\Bbb U}}
\newcommand{\VVv}{{\Bbb V}}
\newcommand{\WWw}{{\Bbb W}}
\newcommand{\XXx}{{\Bbb X}}
\newcommand{\YYy}{{\Bbb Y}}
\newcommand{\ZZz}{{\Bbb Z}}

\newcommand{\OOne}{\mathbbm{1}}
\newcommand{\TTwo}{\mathbbm{2}}
\newcommand{\TThree}{\mathbbm{3}}
\newcommand{\FFour}{\mathbbm{4}}
\newcommand{\FFive}{\mathbbm{5}}
\newcommand{\SSix}{\mathbbm{6}}
\newcommand{\SSeven}{\mathbbm{7}}
\newcommand{\EEight}{\mathbbm{8}}
\newcommand{\NNine}{\mathbbm{9}}

\newcommand{\AaA}{{\mathfrak A}}
\newcommand{\BbB}{{\mathfrak B}}
\newcommand{\CcC}{{\mathfrak C}}
\newcommand{\DdD}{{\mathfrak D}}
\newcommand{\EeE}{{\mathfrak E}}
\newcommand{\FfF}{{\mathfrak F}}
\newcommand{\GgG}{{\mathfrak G}}
\newcommand{\HhH}{{\mathfrak H}}
\newcommand{\IiI}{{\mathfrak I}}
\newcommand{\JjJ}{{\mathfrak J}}
\newcommand{\KkK}{{\mathfrak K}}
\newcommand{\LlL}{{\mathfrak L}}
\newcommand{\MmM}{{\mathfrak M}}
\newcommand{\NnN}{{\mathfrak N}}
\newcommand{\OoO}{{\mathfrak O}}
\newcommand{\PpP}{{\mathfrak P}}
\newcommand{\QqQ}{{\mathfrak Q}}
\newcommand{\RrR}{{\mathfrak R}}
\newcommand{\SsS}{{\mathfrak S}}
\newcommand{\TtT}{{\mathfrak T}}
\newcommand{\UuU}{{\mathfrak U}}
\newcommand{\VvV}{{\mathfrak V}}
\newcommand{\WwW}{{\mathfrak W}}
\newcommand{\XxX}{{\mathfrak X}}
\newcommand{\YyY}{{\mathfrak Y}}
\newcommand{\ZzZ}{{\mathfrak Z}}

\newcommand{\mathbold}[1]{\mbox{\boldmath $#1$}}

\newcommand{\aaa}{{\mathbold a}}
\newcommand{\bbb}{{\mathbold b}}
\newcommand{\ccc}{{\mathbold c}}
\newcommand{\ddd}{{\mathbold d}}
\newcommand{\eee}{{\mathbold e}}
\newcommand{\fff}{{\mathbold f}}
\renewcommand{\ggg}{{\mathbold g}}
\newcommand{\hhh}{{\mathbold h}}
\newcommand{\iii}{{\mathbold i}}
\newcommand{\jjj}{{\mathbold j}}
\newcommand{\kkk}{{\mathbold k}}
\newcommand{\elll}{{\mathbold \ell}}
\newcommand{\mmm}{{\mathbold m}}
\newcommand{\nnn}{{\mathbold n}}
\newcommand{\ooo}{{\mathbold o}}
\newcommand{\ppp}{{\mathbold p}}
\newcommand{\qqq}{{\mathbold q}}
\newcommand{\rrr}{{\mathbold r}}
\newcommand{\sss}{{\mathbold s}}
\newcommand{\ttt}{{\mathbold t}}
\newcommand{\uuu}{{\mathbold u}}
\newcommand{\vvv}{{\mathbold v}}
\newcommand{\www}{{\mathbold w}}
\newcommand{\xxx}{{\mathbold x}}
\newcommand{\yyy}{{\mathbold y}}
\newcommand{\zzz}{{\mathbold z}}


\mathcode`\<="4268 %left delimiter
\mathcode`\>="5269 %right delimiter
\mathchardef\gt="313E %relation >
\mathchardef\lt="313C %relation <


%  %===========================================================================
%  %      P. TAYLOR'S "END OF PROOF BOX":
%  %
%  %
%  %  The complexity of the macro necessary to get a little box on the
%  %  right-hand-side at the end of a proof is amazing.  It really does
%  %  have to be this long!  Otherwise you're liable to get it at the
%  %  beginning of the next line, or even on the next page.
%  %
%  \def\pushright#1{{%              set up
%     \parfillskip=0pt            % so \par doesnt push \square to left
%     \widowpenalty=10000         % so we dont break the page before \square
%     \displaywidowpenalty=10000  % ditto
%     \finalhyphendemerits=0      % TeXbook exercise 14.32
%    %
%    %                 horizontal
%     \leavevmode                 % \nobreak means lines not pages
%     \unskip                     % remove previous space or glue
%     \nobreak                    % don't break lines
%     \hfil                       % ragged right if we spill over
%     \penalty50                  % discouragement to do so
%     \hskip.2em                  % ensure some space
%     \null                       % anchor following \hfill
%     \hfill                      % push \square to right
%     {#1}                        % the end-of-proof mark (or whatever)
%    %
%    %                   vertical
%     \par}}                      % build paragraph
% 
%  % prefer proofs with statements, also space after
%  \def\qed{\pushright{$\square$}\penalty-700 \smallskip}

%%=====================================================================
%
%       For laying out a proof:  \begin{prf}{} ... \end{prf}
%       The argument is in case the proof is a continuation as in:
%               \begin{prf}{\ref{four-color} continued} ... \end{prf}
%       if you do not use the argument be careful to use the brackets!
%
%  							due to RAGS
%-----------------------------------------------------------------------

% \newenvironment{prf}[1]{\begin{trivlist} \item[{\bf ~Proof}#1.]}%
% {\qed\end{trivlist}}

\newcommand{\be}[1]{\begin{#1}}
\newcommand{\bbe}[2]{\begin{#1}[#2]}
\newcommand{\ee}[1]{\end{#1}}
\newcommand{\beq}{\begin{equation}}
\newcommand{\eeq}{\end{equation}}
\newcommand{\ba}[1]{\begin{array}{#1}}
\newcommand{\ea}{\end{array}}
\newcommand{\bea}{\begin{eqnarray}}
\newcommand{\eea}{\end{eqnarray}}
\newcommand{\bear}{\begin{eqnarray*}}
\newcommand{\eear}{\end{eqnarray*}}

\theoremstyle{plain}
\newtheorem{theorem}{Theorem}[section]

\newtheorem*{proposition*}{Proposition}

\newtheorem{proposition}[theorem]{Proposition}
\newtheorem{definition}[theorem]{Definition}
\newtheorem{corollary}[theorem]{Corollary}
\newtheorem{lemma}[theorem]{Lemma}
\theoremstyle{remark}
\newtheorem*{remark}{Remark}
\newtheorem*{terminology}{Terminology}
\theoremstyle{remark}
\newtheorem*{notation}{Notation}
\theoremstyle{remark}
\newtheorem*{explanation}{Explanation}
\theoremstyle{remark}
\newtheorem*{exercise}{Exercise}


%\newtheorem{theorem}{Theorem}[section]
%\newtheorem{definition}[theorem]{Definition}
%\newtheorem{proposition}[theorem]{Proposition}
%\newtheorem{lemma}[theorem]{Lemma}
%\newtheorem{corollary}[theorem]{Corollary}
%
%
%
%%\newtheorem{cond}{}[thm]
%%\renewcommand{\thecond}{{(\alph{cond})}}
%%\newenvironment{condition}{\vspace{-.5\baselineskip}\begin{cond}}{\end{cond}}
%%\newtheorem{prenumb}[thm]{\hspace{-1ex}}
%%\newenvironment{numb}{\begin{prenumb}\rm}{\end{prenumb}}
%
%\newcommand{\nthm}[1]{\newtheorem{#1}[theorem]{#1}}
%\newcommand{\nenv}[1]{\newtheorem{pre#1}[theorem]{#1}%
%		      \newenvironment{#1}{\begin{pre#1}\rm}%
%					    {\end{pre#1}}}  
%					    
%\nenv{Remark}
%\nenv{Terminology}
%\nenv{Notation}	
%\nenv{Explanation}				    
%
%\newenvironment{spec}{\begin{verbatim}\tt}{\end{verbatim}}
%
%



%\newcommand{\tot}[1]{{#1}^\bullet}
%\newcommand{\Base}[1]{{#1}^\flat}
%			    
%\newcommand{\comp}{\, ;}
%
%\newcommand{\supp}{{\sf supp}}
%
%\newcommand{\restr}{\!\restriction}
%\newcommand{\halts}{\!\downarrow}
%\newcommand{\grad}[1]{\left\|{#1}\right\|}
%\newcommand{\tran}{\shortrightarrow}
%\newcommand{\opls}{\oplus}
%
%\newcommand{\initstate}{q}
%\newcommand{\initdata}{a}
%\newcommand{\uu}{w}
%\newcommand{\KT}{T}
%\newcommand{\CXX}{c}
%\newcommand{\UX}[2]{u\left({#1},{#2}\right)}
%\newcommand{\TM}{\TTT}
%
%\newcommand{\iif}{\mbox{\it if}}
%\newcommand{\true}{\mathtt{t}}
%\newcommand{\false}{\mathtt{f}}

\newenvironment{prf}[1]{\begin{trivlist} \item[{\bf ~Proof #1.}]}%
{\qed\end{trivlist}}


\newcommand{\bpr}{\begin{prf}{\!\!}}
\newcommand{\epr}{\end{prf}}
\newcommand{\bprf}[1]{\begin{prf}{#1}}
\newcommand{\eprf}{\end{prf}}	

