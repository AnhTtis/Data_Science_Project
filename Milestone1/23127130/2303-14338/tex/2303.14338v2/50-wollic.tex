% !TEX root = 00-wollic.tex

When a state change depends on our explanations, then we can find an explanation consistent with its own impact: the state changes the way the explanation predicts. More precisely, if a family of transitions in the form $t\colon \DP\otimes X\otimes A\to B$, then the predictions $t_{\ell x}$ can be steered by varying the explanations $\ell$ for every $x$ until a family of explanations $\enco t \colon X\to \DP$ is found, which is self-confirming at all states $x$, i.e. it satisfies $t(\enco t_{x}, x, a) = \uev{\enco t_{x}} a$. 

\begin{proposition}
For any belief transition $t \in \UUU(\DP\otimes X\otimes A, B)$ there is an explanation $\enco t\in \tot \UUU(X,\DP)$ such that
\beq\label{eq:fund}
\begin{split}
\newcommand{\DOTT}{\mbox{\Large$\bullet$}}
\newcommand{\Aah}{\scriptscriptstyle A}
\newcommand{\Xah}{\scriptscriptstyle X}
\newcommand{\grr}{t}
\newcommand{\Bah}{\scriptscriptstyle B}
\newcommand{\GRR}{\scriptstyle \enco t}
\newcommand{\UK}{\mbox{\large$\{\}$}}
\def\JPicScale{.55}
%%Created by jPicEdt 1.4.1_03: mixed JPIC-XML/LaTeX format
%%Thu Mar 16 22:45:56 GMT-10:00 2023
%%Begin JPIC-XML
%<?xml version="1.0" standalone="yes"?>
%<jpic x-min="0" x-max="115" y-min="0" y-max="90" auto-bounding="true">
%<multicurve fill-style= "none"
%	 points= "(5,70);(5,70);(55,70);(55,70)"
%	 />
%<multicurve fill-style= "none"
%	 points= "(55,70);(55,70);(55,50);(55,50)"
%	 />
%<multicurve fill-style= "none"
%	 points= "(55,50);(55,50);(5,50);(5,50)"
%	 />
%<multicurve fill-style= "none"
%	 points= "(5,50);(5,50);(5,70);(5,70)"
%	 />
%<multicurve fill-style= "none"
%	 points= "(30,70);(30,70);(30,90);(30,90)"
%	 />
%<multicurve fill-style= "none"
%	 points= "(10,50);(10,50);(10,35);(10,35)"
%	 stroke-width= "0.65"
%	 />
%<multicurve fill-style= "none"
%	 points= "(30,50);(30,50);(30,20);(30,20)"
%	 />
%<multicurve fill-style= "none"
%	 points= "(0,45);(0,45);(20,25);(20,25)"
%	 />
%<multicurve fill-style= "none"
%	 points= "(20,25);(20,25);(0,25);(0,25)"
%	 />
%<multicurve fill-style= "none"
%	 points= "(0,25);(0,25);(0,45);(0,45)"
%	 />
%<multicurve fill-style= "none"
%	 points= "(75,70);(75,70);(115,70);(115,70)"
%	 />
%<multicurve fill-style= "none"
%	 points= "(115,70);(115,70);(115,50);(115,50)"
%	 />
%<multicurve fill-style= "none"
%	 points= "(115,50);(115,50);(95,50);(95,50)"
%	 />
%<multicurve fill-style= "none"
%	 points= "(95,50);(95,50);(75,70);(75,70)"
%	 />
%<multicurve fill-style= "none"
%	 points= "(100,70);(100,70);(100,90);(100,90)"
%	 />
%<multicurve fill-style= "none"
%	 points= "(10,25);(10,25);(10,20);(10,20)"
%	 />
%<multicurve fill-style= "none"
%	 points= "(95,25);(95,25);(75,45);(75,45)"
%	 />
%<multicurve fill-style= "none"
%	 points= "(75,25);(75,25);(95,25);(95,25)"
%	 />
%<multicurve fill-style= "none"
%	 points= "(75,45);(75,45);(75,25);(75,25)"
%	 />
%<multicurve fill-style= "none"
%	 points= "(85,60);(85,60);(85,35);(85,35)"
%	 stroke-width= "0.65"
%	 />
%<text fill-style= "none"
%	 text-vert-align= "center-v"
%	 anchor-point= "(10,35)"
%	 text-frame= "noframe"
%	 text-hor-align= "center-h"
%	 >
%\DOTT
%</text>
%<text fill-style= "none"
%	 text-vert-align= "center-v"
%	 anchor-point= "(85,35)"
%	 text-frame= "noframe"
%	 text-hor-align= "center-h"
%	 >
%\DOTT
%</text>
%<text fill-style= "none"
%	 text-vert-align= "center-v"
%	 anchor-point= "(46.25,0)"
%	 text-frame= "noframe"
%	 text-hor-align= "right"
%	 >
%$\Aah$
%</text>
%<text fill-style= "none"
%	 text-vert-align= "center-v"
%	 anchor-point= "(107.5,0)"
%	 text-frame= "noframe"
%	 text-hor-align= "right"
%	 >
%$\Aah$
%</text>
%<text fill-style= "none"
%	 text-vert-align= "center-v"
%	 anchor-point= "(30,60)"
%	 text-frame= "noframe"
%	 text-hor-align= "center-h"
%	 >
%$\grr$
%</text>
%<text fill-style= "none"
%	 text-vert-align= "center-v"
%	 anchor-point= "(32.5,90)"
%	 text-frame= "noframe"
%	 text-hor-align= "left"
%	 >
%$\Bah$
%</text>
%<text fill-style= "none"
%	 text-vert-align= "center-v"
%	 anchor-point= "(102.5,90)"
%	 text-frame= "noframe"
%	 text-hor-align= "left"
%	 >
%$\Bah$
%</text>
%<text fill-style= "none"
%	 text-vert-align= "center-v"
%	 anchor-point= "(100,60)"
%	 text-frame= "noframe"
%	 text-hor-align= "center-h"
%	 >
%$\UK$
%</text>
%<text fill-style= "none"
%	 text-vert-align= "center-v"
%	 anchor-point= "(65,60)"
%	 text-frame= "noframe"
%	 text-hor-align= "center-h"
%	 >
%\EQLS
%</text>
%<text fill-style= "none"
%	 text-vert-align= "center-v"
%	 anchor-point= "(5,30)"
%	 text-frame= "noframe"
%	 text-hor-align= "center-h"
%	 >
%$\GRR$
%</text>
%<text fill-style= "none"
%	 text-vert-align= "center-v"
%	 anchor-point= "(80,30)"
%	 text-frame= "noframe"
%	 text-hor-align= "center-h"
%	 >
%$\GRR$
%</text>
%<multicurve fill-style= "none"
%	 points= "(50,50);(50,50);(50,0);(50,0)"
%	 />
%<multicurve fill-style= "none"
%	 points= "(110,50);(110,50);(110,0);(110,0)"
%	 />
%<multicurve fill-style= "none"
%	 points= "(10,20);(10,20);(20,10);(20,10)"
%	 />
%<multicurve fill-style= "none"
%	 points= "(30,20);(30,20);(20,10);(20,10)"
%	 />
%<multicurve fill-style= "none"
%	 points= "(20,10);(20,10);(20,0);(20,0)"
%	 />
%<multicurve fill-style= "none"
%	 points= "(85,25);(85,25);(85,0);(85,0)"
%	 />
%<text fill-style= "none"
%	 text-vert-align= "center-v"
%	 anchor-point= "(82.5,0)"
%	 text-frame= "noframe"
%	 text-hor-align= "right"
%	 >
%$\Xah$
%</text>
%<text fill-style= "none"
%	 text-vert-align= "center-v"
%	 anchor-point= "(17.5,0)"
%	 text-frame= "noframe"
%	 text-hor-align= "right"
%	 >
%$\Xah$
%</text>
%<text fill-style= "none"
%	 text-vert-align= "center-v"
%	 anchor-point= "(20,10)"
%	 text-frame= "noframe"
%	 text-hor-align= "center-h"
%	 >
%\DOTT
%</text>
%</jpic>
%%End JPIC-XML
%LaTeX-picture environment using emulated lines and arcs
%You can rescale the whole picture (to 80% for instance) by using the command \def\JPicScale{0.8}
\ifx\JPicScale\undefined\def\JPicScale{1}\fi
\unitlength \JPicScale mm
\begin{picture}(115,90)(0,0)
\linethickness{0.3mm}
\put(5,70){\line(1,0){50}}
\linethickness{0.3mm}
\put(55,50){\line(0,1){20}}
\linethickness{0.3mm}
\put(5,50){\line(1,0){50}}
\linethickness{0.3mm}
\put(5,50){\line(0,1){20}}
\linethickness{0.3mm}
\put(30,70){\line(0,1){20}}
\linethickness{0.65mm}
\put(10,35){\line(0,1){15}}
\linethickness{0.3mm}
\put(30,20){\line(0,1){30}}
\linethickness{0.3mm}
\multiput(0,45)(0.12,-0.12){167}{\line(1,0){0.12}}
\linethickness{0.3mm}
\put(0,25){\line(1,0){20}}
\linethickness{0.3mm}
\put(0,25){\line(0,1){20}}
\linethickness{0.3mm}
\put(75,70){\line(1,0){40}}
\linethickness{0.3mm}
\put(115,50){\line(0,1){20}}
\linethickness{0.3mm}
\put(95,50){\line(1,0){20}}
\linethickness{0.3mm}
\multiput(75,70)(0.12,-0.12){167}{\line(1,0){0.12}}
\linethickness{0.3mm}
\put(100,70){\line(0,1){20}}
\linethickness{0.3mm}
\put(10,20){\line(0,1){5}}
\linethickness{0.3mm}
\multiput(75,45)(0.12,-0.12){167}{\line(1,0){0.12}}
\linethickness{0.3mm}
\put(75,25){\line(1,0){20}}
\linethickness{0.3mm}
\put(75,25){\line(0,1){20}}
\linethickness{0.65mm}
\put(85,35){\line(0,1){25}}
\put(10,35){\makebox(0,0)[cc]{\DOTT}}

\put(85,35){\makebox(0,0)[cc]{\DOTT}}

\put(46.25,0){\makebox(0,0)[cr]{$\Aah$}}

\put(107.5,0){\makebox(0,0)[cr]{$\Aah$}}

\put(30,60){\makebox(0,0)[cc]{$\grr$}}

\put(32.5,90){\makebox(0,0)[cl]{$\Bah$}}

\put(102.5,90){\makebox(0,0)[cl]{$\Bah$}}

\put(100,60){\makebox(0,0)[cc]{$\UK$}}

\put(65,60){\makebox(0,0)[cc]{\EQLS}}

\put(5,30){\makebox(0,0)[cc]{$\GRR$}}

\put(80,30){\makebox(0,0)[cc]{$\GRR$}}

\linethickness{0.3mm}
\put(50,0){\line(0,1){50}}
\linethickness{0.3mm}
\put(110,0){\line(0,1){50}}
\linethickness{0.3mm}
\multiput(10,20)(0.12,-0.12){83}{\line(1,0){0.12}}
\linethickness{0.3mm}
\multiput(20,10)(0.12,0.12){83}{\line(1,0){0.12}}
\linethickness{0.3mm}
\put(20,0){\line(0,1){10}}
\linethickness{0.3mm}
\put(85,0){\line(0,1){25}}
\put(82.5,0){\makebox(0,0)[cr]{$\Xah$}}

\put(17.5,0){\makebox(0,0)[cr]{$\Xah$}}

\put(20,10){\makebox(0,0)[cc]{\DOTT}}

\end{picture}

\end{split}
\eeq 
\end{proposition}

\bpr
Let $T\in \tot\UUU(X,\DP)$ be an explanation of the transition on the left in \eqref{eq:fund}. 
\beq
\begin{split}
\newcommand{\DOTT}{\mbox{\Large$\bullet$}}
\newcommand{\Aah}{\scriptscriptstyle A}
\newcommand{\Xah}{\scriptscriptstyle X}
\newcommand{\grr}{t}
\newcommand{\Bah}{\scriptscriptstyle B}
\newcommand{\Grr}{\scriptstyle T}
\newcommand{\UK}{\universal}
\newcommand{\PK}{\prtial}
\def\JPicScale{.45}
%%Created by jPicEdt 1.4.1_03: mixed JPIC-XML/LaTeX format
%%Thu Mar 16 22:24:27 GMT-10:00 2023
%%Begin JPIC-XML
%<?xml version="1.0" standalone="yes"?>
%<jpic x-min="0" x-max="155" y-min="0" y-max="120" auto-bounding="true">
%<multicurve fill-style= "none"
%	 points= "(25,100);(25,100);(75,100);(75,100)"
%	 />
%<multicurve fill-style= "none"
%	 points= "(75,100);(75,100);(75,80);(75,80)"
%	 />
%<multicurve fill-style= "none"
%	 points= "(75,80);(75,80);(25,80);(25,80)"
%	 />
%<multicurve fill-style= "none"
%	 points= "(25,80);(25,80);(25,100);(25,100)"
%	 />
%<multicurve fill-style= "none"
%	 points= "(50,100);(50,100);(50,120);(50,120)"
%	 />
%<multicurve fill-style= "none"
%	 points= "(30,80);(30,80);(30,60);(30,60)"
%	 stroke-width= "0.65"
%	 />
%<multicurve fill-style= "none"
%	 points= "(70,80);(70,80);(70,0);(70,0)"
%	 />
%<multicurve fill-style= "none"
%	 points= "(20,70);(20,70);(40,50);(40,50)"
%	 />
%<multicurve fill-style= "none"
%	 points= "(40,50);(40,50);(20,50);(20,50)"
%	 />
%<multicurve fill-style= "none"
%	 points= "(20,50);(20,50);(0,70);(0,70)"
%	 />
%<multicurve fill-style= "none"
%	 points= "(0,70);(0,70);(20,70);(20,70)"
%	 />
%<multicurve fill-style= "none"
%	 points= "(30,50);(30,50);(30,30);(30,30)"
%	 stroke-width= "0.65"
%	 />
%<multicurve fill-style= "none"
%	 points= "(100,100);(100,100);(155,100);(155,100)"
%	 />
%<multicurve fill-style= "none"
%	 points= "(155,100);(155,100);(155,80);(155,80)"
%	 />
%<multicurve fill-style= "none"
%	 points= "(155,80);(155,80);(120,80);(120,80)"
%	 />
%<multicurve fill-style= "none"
%	 points= "(120,80);(120,80);(100,100);(100,100)"
%	 />
%<multicurve fill-style= "none"
%	 points= "(140,100);(140,100);(140,120);(140,120)"
%	 />
%<multicurve fill-style= "none"
%	 points= "(130,80);(130,80);(130,0);(130,0)"
%	 stroke-width= "0.65"
%	 />
%<multicurve fill-style= "none"
%	 points= "(150,80);(150,80);(150,0);(150,0)"
%	 />
%<multicurve fill-style= "none"
%	 points= "(120,50);(120,50);(100,70);(100,70)"
%	 />
%<multicurve fill-style= "none"
%	 points= "(100,50);(100,50);(120,50);(120,50)"
%	 />
%<multicurve fill-style= "none"
%	 points= "(100,70);(100,70);(100,50);(100,50)"
%	 />
%<multicurve fill-style= "none"
%	 points= "(110,90);(110,90);(110,60);(110,60)"
%	 stroke-width= "0.65"
%	 />
%<multicurve fill-style= "none"
%	 points= "(10,60);(10,60);(10,50);(10,50)"
%	 stroke-width= "0.65"
%	 />
%<text fill-style= "none"
%	 text-vert-align= "center-v"
%	 anchor-point= "(30,30)"
%	 text-frame= "noframe"
%	 text-hor-align= "center-h"
%	 >
%\DOTT
%</text>
%<text fill-style= "none"
%	 text-vert-align= "center-v"
%	 anchor-point= "(30,60)"
%	 text-frame= "noframe"
%	 text-hor-align= "center-h"
%	 >
%\DOTT
%</text>
%<text fill-style= "none"
%	 text-vert-align= "center-v"
%	 anchor-point= "(110,60)"
%	 text-frame= "noframe"
%	 text-hor-align= "center-h"
%	 >
%\DOTT
%</text>
%<text fill-style= "none"
%	 text-vert-align= "center-v"
%	 anchor-point= "(67.5,0)"
%	 text-frame= "noframe"
%	 text-hor-align= "right"
%	 >
%$\Aah$
%</text>
%<text fill-style= "none"
%	 text-vert-align= "center-v"
%	 anchor-point= "(147.5,0)"
%	 text-frame= "noframe"
%	 text-hor-align= "right"
%	 >
%$\Aah$
%</text>
%<text fill-style= "none"
%	 text-vert-align= "center-v"
%	 anchor-point= "(20,60)"
%	 text-frame= "noframe"
%	 text-hor-align= "center-h"
%	 >
%$\PK$
%</text>
%<text fill-style= "none"
%	 text-vert-align= "center-v"
%	 anchor-point= "(50,90)"
%	 text-frame= "noframe"
%	 text-hor-align= "center-h"
%	 >
%$\grr$
%</text>
%<text fill-style= "none"
%	 text-vert-align= "center-v"
%	 anchor-point= "(52.5,120)"
%	 text-frame= "noframe"
%	 text-hor-align= "left"
%	 >
%$\Bah$
%</text>
%<text fill-style= "none"
%	 text-vert-align= "center-v"
%	 anchor-point= "(142.5,120)"
%	 text-frame= "noframe"
%	 text-hor-align= "left"
%	 >
%$\Bah$
%</text>
%<text fill-style= "none"
%	 text-vert-align= "center-v"
%	 anchor-point= "(140,90)"
%	 text-frame= "noframe"
%	 text-hor-align= "center-h"
%	 >
%$\UK$
%</text>
%<text fill-style= "none"
%	 text-vert-align= "center-v"
%	 anchor-point= "(90,90)"
%	 text-frame= "noframe"
%	 text-hor-align= "center-h"
%	 >
%\EQLS
%</text>
%<text fill-style= "none"
%	 text-vert-align= "center-v"
%	 anchor-point= "(105,55)"
%	 text-frame= "noframe"
%	 text-hor-align= "center-h"
%	 >
%$\Grr$
%</text>
%<multicurve fill-style= "none"
%	 points= "(50,80);(50,80);(50,30);(50,30)"
%	 />
%<multicurve fill-style= "none"
%	 points= "(50,30);(50,30);(30,10);(30,10)"
%	 />
%<multicurve fill-style= "none"
%	 points= "(30,10);(30,10);(30,0);(30,0)"
%	 />
%<multicurve fill-style= "none"
%	 points= "(50,10);(50,10);(50,0);(50,0)"
%	 stroke-width= "0.65"
%	 />
%<multicurve stroke-overstrike= "true"
%	 fill-style= "none"
%	 points= "(10,50);(10,50);(50,10);(50,10)"
%	 stroke-width= "0.65"
%	 stroke-overstrike-width= "0.5"
%	 />
%<multicurve fill-style= "none"
%	 points= "(110,50);(110,50);(110,0);(110,0)"
%	 />
%<text fill-style= "none"
%	 text-vert-align= "center-v"
%	 anchor-point= "(107.5,0)"
%	 text-frame= "noframe"
%	 text-hor-align= "right"
%	 >
%$\Xah$
%</text>
%<text fill-style= "none"
%	 text-vert-align= "center-v"
%	 anchor-point= "(27.5,0)"
%	 text-frame= "noframe"
%	 text-hor-align= "right"
%	 >
%$\Xah$
%</text>
%</jpic>
%%End JPIC-XML
%LaTeX-picture environment using emulated lines and arcs
%You can rescale the whole picture (to 80% for instance) by using the command \def\JPicScale{0.8}
\ifx\JPicScale\undefined\def\JPicScale{1}\fi
\unitlength \JPicScale mm
\begin{picture}(155,120)(0,0)
\linethickness{0.3mm}
\put(25,100){\line(1,0){50}}
\linethickness{0.3mm}
\put(75,80){\line(0,1){20}}
\linethickness{0.3mm}
\put(25,80){\line(1,0){50}}
\linethickness{0.3mm}
\put(25,80){\line(0,1){20}}
\linethickness{0.3mm}
\put(50,100){\line(0,1){20}}
\linethickness{0.65mm}
\put(30,60){\line(0,1){20}}
\linethickness{0.3mm}
\put(70,0){\line(0,1){80}}
\linethickness{0.3mm}
\multiput(20,70)(0.12,-0.12){167}{\line(1,0){0.12}}
\linethickness{0.3mm}
\put(20,50){\line(1,0){20}}
\linethickness{0.3mm}
\multiput(0,70)(0.12,-0.12){167}{\line(1,0){0.12}}
\linethickness{0.3mm}
\put(0,70){\line(1,0){20}}
\linethickness{0.65mm}
\put(30,30){\line(0,1){20}}
\linethickness{0.3mm}
\put(100,100){\line(1,0){55}}
\linethickness{0.3mm}
\put(155,80){\line(0,1){20}}
\linethickness{0.3mm}
\put(120,80){\line(1,0){35}}
\linethickness{0.3mm}
\multiput(100,100)(0.12,-0.12){167}{\line(1,0){0.12}}
\linethickness{0.3mm}
\put(140,100){\line(0,1){20}}
\linethickness{0.65mm}
\put(130,0){\line(0,1){80}}
\linethickness{0.3mm}
\put(150,0){\line(0,1){80}}
\linethickness{0.3mm}
\multiput(100,70)(0.12,-0.12){167}{\line(1,0){0.12}}
\linethickness{0.3mm}
\put(100,50){\line(1,0){20}}
\linethickness{0.3mm}
\put(100,50){\line(0,1){20}}
\linethickness{0.65mm}
\put(110,60){\line(0,1){30}}
\linethickness{0.65mm}
\put(10,50){\line(0,1){10}}
\put(30,30){\makebox(0,0)[cc]{\DOTT}}

\put(30,60){\makebox(0,0)[cc]{\DOTT}}

\put(110,60){\makebox(0,0)[cc]{\DOTT}}

\put(67.5,0){\makebox(0,0)[cr]{$\Aah$}}

\put(147.5,0){\makebox(0,0)[cr]{$\Aah$}}

\put(20,60){\makebox(0,0)[cc]{$\PK$}}

\put(50,90){\makebox(0,0)[cc]{$\grr$}}

\put(52.5,120){\makebox(0,0)[cl]{$\Bah$}}

\put(142.5,120){\makebox(0,0)[cl]{$\Bah$}}

\put(140,90){\makebox(0,0)[cc]{$\UK$}}

\put(90,90){\makebox(0,0)[cc]{\EQLS}}

\put(105,55){\makebox(0,0)[cc]{$\Grr$}}

\linethickness{0.3mm}
\put(50,30){\line(0,1){50}}
\linethickness{0.3mm}
\multiput(30,10)(0.12,0.12){167}{\line(1,0){0.12}}
\linethickness{0.3mm}
\put(30,0){\line(0,1){10}}
\linethickness{0.65mm}
\put(50,0){\line(0,1){10}}
\linethickness{0.65mm}
\multiput(10,50)(0.12,-0.12){333}{\line(1,0){0.12}}
\linethickness{0.3mm}
\put(110,0){\line(0,1){50}}
\put(107.5,0){\makebox(0,0)[cr]{$\Xah$}}

\put(27.5,0){\makebox(0,0)[cr]{$\Xah$}}

\end{picture}

\end{split}
\eeq
$H$ exists by Def.~\ref{Def:interpreter}. Then $\enco t_{x} =\pev{Tx}$ is self-confirming, because
\beq
\begin{split}
\newcommand{\DOTT}{\mbox{\Large$\bullet$}}
\newcommand{\Aah}{\scriptscriptstyle A}
\newcommand{\Xah}{\scriptscriptstyle X}
\newcommand{\grr}{t}
\newcommand{\Bah}{\scriptscriptstyle B}
\newcommand{\Grr}{\scriptstyle T}
\newcommand{\Psee}{\enco t}
\newcommand{\UK}{\universal}
\newcommand{\PK}{\prtial}
\def\JPicScale{.45}
%%Created by jPicEdt 1.4.1_03: mixed JPIC-XML/LaTeX format
%%Thu Mar 16 23:04:27 GMT-10:00 2023
%%Begin JPIC-XML
%<?xml version="1.0" standalone="yes"?>
%<jpic x-min="-0" x-max="320" y-min="0" y-max="140" auto-bounding="true">
%<multicurve fill-style= "none"
%	 points= "(30,120);(30,120);(80,120);(80,120)"
%	 />
%<multicurve fill-style= "none"
%	 points= "(80,120);(80,120);(80,100);(80,100)"
%	 />
%<multicurve fill-style= "none"
%	 points= "(80,100);(80,100);(30,100);(30,100)"
%	 />
%<multicurve fill-style= "none"
%	 points= "(30,100);(30,100);(30,120);(30,120)"
%	 />
%<multicurve fill-style= "none"
%	 points= "(55,120);(55,120);(55,140);(55,140)"
%	 />
%<multicurve fill-style= "none"
%	 points= "(35,100);(35,100);(35,80);(35,80)"
%	 stroke-width= "0.65"
%	 />
%<multicurve fill-style= "none"
%	 points= "(75,100);(75,100);(75,0);(75,0)"
%	 />
%<multicurve fill-style= "none"
%	 points= "(25,90);(25,90);(45,70);(45,70)"
%	 />
%<multicurve fill-style= "none"
%	 points= "(45,70);(45,70);(25,70);(25,70)"
%	 />
%<multicurve fill-style= "none"
%	 points= "(25,70);(25,70);(5,90);(5,90)"
%	 />
%<multicurve fill-style= "none"
%	 points= "(5,90);(5,90);(25,90);(25,90)"
%	 />
%<multicurve fill-style= "none"
%	 points= "(35,70);(35,70);(35,45);(35,45)"
%	 stroke-width= "0.65"
%	 />
%<multicurve fill-style= "none"
%	 points= "(105,120);(105,120);(160,120);(160,120)"
%	 />
%<multicurve fill-style= "none"
%	 points= "(160,120);(160,120);(160,100);(160,100)"
%	 />
%<multicurve fill-style= "none"
%	 points= "(160,100);(160,100);(125,100);(125,100)"
%	 />
%<multicurve fill-style= "none"
%	 points= "(125,100);(125,100);(105,120);(105,120)"
%	 />
%<multicurve fill-style= "none"
%	 points= "(145,120);(145,120);(145,140);(145,140)"
%	 />
%<multicurve fill-style= "none"
%	 points= "(135,100);(135,100);(135,45);(135,45)"
%	 stroke-width= "0.65"
%	 />
%<multicurve fill-style= "none"
%	 points= "(155,100);(155,100);(155,0);(155,0)"
%	 />
%<multicurve fill-style= "none"
%	 points= "(125,70);(125,70);(105,90);(105,90)"
%	 />
%<multicurve fill-style= "none"
%	 points= "(105,70);(105,70);(125,70);(125,70)"
%	 />
%<multicurve fill-style= "none"
%	 points= "(105,90);(105,90);(105,70);(105,70)"
%	 />
%<multicurve fill-style= "none"
%	 points= "(115,110);(115,110);(115,80);(115,80)"
%	 stroke-width= "0.65"
%	 />
%<multicurve fill-style= "none"
%	 points= "(15,80);(15,80);(15,72.5);(15,72.5)"
%	 stroke-width= "0.65"
%	 />
%<text fill-style= "none"
%	 text-vert-align= "center-v"
%	 anchor-point= "(35,52.5)"
%	 text-frame= "noframe"
%	 text-hor-align= "center-h"
%	 >
%\DOTT
%</text>
%<text fill-style= "none"
%	 text-vert-align= "center-v"
%	 anchor-point= "(35,80)"
%	 text-frame= "noframe"
%	 text-hor-align= "center-h"
%	 >
%\DOTT
%</text>
%<text fill-style= "none"
%	 text-vert-align= "center-v"
%	 anchor-point= "(115,80)"
%	 text-frame= "noframe"
%	 text-hor-align= "center-h"
%	 >
%\DOTT
%</text>
%<text fill-style= "none"
%	 text-vert-align= "center-v"
%	 anchor-point= "(72.5,0)"
%	 text-frame= "noframe"
%	 text-hor-align= "right"
%	 >
%$\Aah$
%</text>
%<text fill-style= "none"
%	 text-vert-align= "center-v"
%	 anchor-point= "(152.5,0)"
%	 text-frame= "noframe"
%	 text-hor-align= "right"
%	 >
%$\Aah$
%</text>
%<text fill-style= "none"
%	 text-vert-align= "center-v"
%	 anchor-point= "(25,80)"
%	 text-frame= "noframe"
%	 text-hor-align= "center-h"
%	 >
%$\PK$
%</text>
%<text fill-style= "none"
%	 text-vert-align= "center-v"
%	 anchor-point= "(55,110)"
%	 text-frame= "noframe"
%	 text-hor-align= "center-h"
%	 >
%$\grr$
%</text>
%<text fill-style= "none"
%	 text-vert-align= "center-v"
%	 anchor-point= "(57.5,140)"
%	 text-frame= "noframe"
%	 text-hor-align= "left"
%	 >
%$\Bah$
%</text>
%<text fill-style= "none"
%	 text-vert-align= "center-v"
%	 anchor-point= "(147.5,140)"
%	 text-frame= "noframe"
%	 text-hor-align= "left"
%	 >
%$\Bah$
%</text>
%<text fill-style= "none"
%	 text-vert-align= "center-v"
%	 anchor-point= "(145,110)"
%	 text-frame= "noframe"
%	 text-hor-align= "center-h"
%	 >
%$\UK$
%</text>
%<text fill-style= "none"
%	 text-vert-align= "center-v"
%	 anchor-point= "(95,110)"
%	 text-frame= "noframe"
%	 text-hor-align= "center-h"
%	 >
%\EQLS
%</text>
%<text fill-style= "none"
%	 text-vert-align= "center-v"
%	 anchor-point= "(110,75)"
%	 text-frame= "noframe"
%	 text-hor-align= "center-h"
%	 >
%$\Grr$
%</text>
%<multicurve fill-style= "none"
%	 points= "(55,100);(55,100);(55,30);(55,30)"
%	 />
%<multicurve fill-style= "none"
%	 points= "(35,35);(35,35);(35,0);(35,0)"
%	 />
%<multicurve stroke-overstrike= "true"
%	 fill-style= "none"
%	 points= "(15,72.5);(15,72.5);(35,52.5);(35,52.5)"
%	 stroke-width= "0.65"
%	 stroke-overstrike-width= "0.5"
%	 />
%<multicurve fill-style= "none"
%	 points= "(115,70);(115,70);(115,0);(115,0)"
%	 />
%<text fill-style= "none"
%	 text-vert-align= "center-v"
%	 anchor-point= "(112.5,0)"
%	 text-frame= "noframe"
%	 text-hor-align= "right"
%	 >
%$\Xah$
%</text>
%<text fill-style= "none"
%	 text-vert-align= "center-v"
%	 anchor-point= "(32.5,0)"
%	 text-frame= "noframe"
%	 text-hor-align= "right"
%	 >
%$\Xah$
%</text>
%<multicurve fill-style= "none"
%	 points= "(25,95);(25,95);(-0,95);(-0,95)"
%	 />
%<multicurve fill-style= "none"
%	 points= "(25,95);(25,95);(50,70);(50,70)"
%	 />
%<multicurve fill-style= "none"
%	 points= "(50,70);(50,70);(50,30);(50,30)"
%	 />
%<multicurve fill-style= "none"
%	 points= "(50,30);(50,30);(0,30);(0,30)"
%	 />
%<multicurve fill-style= "none"
%	 points= "(45,35);(45,35);(25,55);(25,55)"
%	 />
%<multicurve fill-style= "none"
%	 points= "(25,35);(25,35);(45,35);(45,35)"
%	 />
%<multicurve fill-style= "none"
%	 points= "(25,55);(25,55);(25,35);(25,35)"
%	 />
%<text fill-style= "none"
%	 text-vert-align= "center-v"
%	 anchor-point= "(35,45)"
%	 text-frame= "noframe"
%	 text-hor-align= "center-h"
%	 >
%\DOTT
%</text>
%<text fill-style= "none"
%	 text-vert-align= "center-v"
%	 anchor-point= "(30,40)"
%	 text-frame= "noframe"
%	 text-hor-align= "center-h"
%	 >
%$\Grr$
%</text>
%<multicurve fill-style= "none"
%	 points= "(55,30);(55,30);(35,10);(35,10)"
%	 />
%<text fill-style= "none"
%	 text-vert-align= "center-v"
%	 anchor-point= "(35,10)"
%	 text-frame= "noframe"
%	 text-hor-align= "center-h"
%	 >
%\DOTT
%</text>
%<multicurve fill-style= "none"
%	 points= "(0,95);(0,95);(0,30);(0,30)"
%	 />
%<multicurve fill-style= "none"
%	 points= "(145,35);(145,35);(125,55);(125,55)"
%	 />
%<multicurve fill-style= "none"
%	 points= "(125,35);(125,35);(145,35);(145,35)"
%	 />
%<multicurve fill-style= "none"
%	 points= "(125,55);(125,55);(125,35);(125,35)"
%	 />
%<text fill-style= "none"
%	 text-vert-align= "center-v"
%	 anchor-point= "(135,45)"
%	 text-frame= "noframe"
%	 text-hor-align= "center-h"
%	 >
%\DOTT
%</text>
%<text fill-style= "none"
%	 text-vert-align= "center-v"
%	 anchor-point= "(130,40)"
%	 text-frame= "noframe"
%	 text-hor-align= "center-h"
%	 >
%$\Grr$
%</text>
%<multicurve fill-style= "none"
%	 points= "(135,30);(135,30);(115,10);(115,10)"
%	 />
%<multicurve fill-style= "none"
%	 points= "(135,35);(135,35);(135,30);(135,30)"
%	 />
%<multicurve fill-style= "none"
%	 points= "(185,120);(185,120);(240,120);(240,120)"
%	 />
%<multicurve fill-style= "none"
%	 points= "(240,120);(240,120);(240,100);(240,100)"
%	 />
%<multicurve fill-style= "none"
%	 points= "(240,100);(240,100);(205,100);(205,100)"
%	 />
%<multicurve fill-style= "none"
%	 points= "(205,100);(205,100);(185,120);(185,120)"
%	 />
%<multicurve fill-style= "none"
%	 points= "(225,120);(225,120);(225,140);(225,140)"
%	 />
%<multicurve fill-style= "none"
%	 points= "(215,100);(215,100);(215,45);(215,45)"
%	 stroke-width= "0.65"
%	 />
%<multicurve fill-style= "none"
%	 points= "(235,100);(235,100);(235,0);(235,0)"
%	 />
%<text fill-style= "none"
%	 text-vert-align= "center-v"
%	 anchor-point= "(232.5,0)"
%	 text-frame= "noframe"
%	 text-hor-align= "right"
%	 >
%$\Aah$
%</text>
%<text fill-style= "none"
%	 text-vert-align= "center-v"
%	 anchor-point= "(227.5,140)"
%	 text-frame= "noframe"
%	 text-hor-align= "left"
%	 >
%$\Bah$
%</text>
%<text fill-style= "none"
%	 text-vert-align= "center-v"
%	 anchor-point= "(225,110)"
%	 text-frame= "noframe"
%	 text-hor-align= "center-h"
%	 >
%$\UK$
%</text>
%<text fill-style= "none"
%	 text-vert-align= "center-v"
%	 anchor-point= "(175,110)"
%	 text-frame= "noframe"
%	 text-hor-align= "center-h"
%	 >
%\EQLS
%</text>
%<text fill-style= "none"
%	 text-vert-align= "center-v"
%	 anchor-point= "(212.5,0)"
%	 text-frame= "noframe"
%	 text-hor-align= "right"
%	 >
%$\Xah$
%</text>
%<multicurve fill-style= "none"
%	 points= "(225,35);(225,35);(205,55);(205,55)"
%	 />
%<multicurve fill-style= "none"
%	 points= "(205,35);(205,35);(225,35);(225,35)"
%	 />
%<multicurve fill-style= "none"
%	 points= "(205,55);(205,55);(205,35);(205,35)"
%	 />
%<text fill-style= "none"
%	 text-vert-align= "center-v"
%	 anchor-point= "(215,45)"
%	 text-frame= "noframe"
%	 text-hor-align= "center-h"
%	 >
%\DOTT
%</text>
%<text fill-style= "none"
%	 text-vert-align= "center-v"
%	 anchor-point= "(210,40)"
%	 text-frame= "noframe"
%	 text-hor-align= "center-h"
%	 >
%$\Grr$
%</text>
%<multicurve fill-style= "none"
%	 points= "(215,35);(215,35);(215,0);(215,0)"
%	 />
%<multicurve fill-style= "none"
%	 points= "(285,120);(285,120);(320,120);(320,120)"
%	 />
%<multicurve fill-style= "none"
%	 points= "(320,120);(320,120);(320,100);(320,100)"
%	 />
%<multicurve fill-style= "none"
%	 points= "(320,100);(320,100);(305,100);(305,100)"
%	 />
%<multicurve fill-style= "none"
%	 points= "(305,100);(305,100);(285,120);(285,120)"
%	 />
%<multicurve fill-style= "none"
%	 points= "(305,120);(305,120);(305,140);(305,140)"
%	 />
%<multicurve fill-style= "none"
%	 points= "(315,100);(315,100);(315,0);(315,0)"
%	 />
%<text fill-style= "none"
%	 text-vert-align= "center-v"
%	 anchor-point= "(312.5,0)"
%	 text-frame= "noframe"
%	 text-hor-align= "right"
%	 >
%$\Aah$
%</text>
%<text fill-style= "none"
%	 text-vert-align= "center-v"
%	 anchor-point= "(307.5,140)"
%	 text-frame= "noframe"
%	 text-hor-align= "left"
%	 >
%$\Bah$
%</text>
%<text fill-style= "none"
%	 text-vert-align= "center-v"
%	 anchor-point= "(310,110)"
%	 text-frame= "noframe"
%	 text-hor-align= "center-h"
%	 >
%$\UK$
%</text>
%<text fill-style= "none"
%	 text-vert-align= "center-v"
%	 anchor-point= "(255,110)"
%	 text-frame= "noframe"
%	 text-hor-align= "center-h"
%	 >
%\EQLS
%</text>
%<multicurve fill-style= "none"
%	 points= "(295,110);(295,110);(295,80);(295,80)"
%	 stroke-width= "0.65"
%	 />
%<multicurve fill-style= "none"
%	 points= "(285,90);(285,90);(305,70);(305,70)"
%	 />
%<multicurve fill-style= "none"
%	 points= "(305,70);(305,70);(285,70);(285,70)"
%	 />
%<multicurve fill-style= "none"
%	 points= "(285,70);(285,70);(265,90);(265,90)"
%	 />
%<multicurve fill-style= "none"
%	 points= "(265,90);(265,90);(285,90);(285,90)"
%	 />
%<multicurve fill-style= "none"
%	 points= "(295,70);(295,70);(295,45);(295,45)"
%	 stroke-width= "0.65"
%	 />
%<multicurve fill-style= "none"
%	 points= "(275,80);(275,80);(275,72.5);(275,72.5)"
%	 stroke-width= "0.65"
%	 />
%<text fill-style= "none"
%	 text-vert-align= "center-v"
%	 anchor-point= "(295,52.5)"
%	 text-frame= "noframe"
%	 text-hor-align= "center-h"
%	 >
%\DOTT
%</text>
%<text fill-style= "none"
%	 text-vert-align= "center-v"
%	 anchor-point= "(295,80)"
%	 text-frame= "noframe"
%	 text-hor-align= "center-h"
%	 >
%\DOTT
%</text>
%<text fill-style= "none"
%	 text-vert-align= "center-v"
%	 anchor-point= "(285,80)"
%	 text-frame= "noframe"
%	 text-hor-align= "center-h"
%	 >
%$\PK$
%</text>
%<multicurve fill-style= "none"
%	 points= "(295,35);(295,35);(295,0);(295,0)"
%	 />
%<multicurve stroke-overstrike= "true"
%	 fill-style= "none"
%	 points= "(275,72.5);(275,72.5);(295,52.5);(295,52.5)"
%	 stroke-width= "0.65"
%	 stroke-overstrike-width= "0.5"
%	 />
%<text fill-style= "none"
%	 text-vert-align= "center-v"
%	 anchor-point= "(292.5,0)"
%	 text-frame= "noframe"
%	 text-hor-align= "right"
%	 >
%$\Xah$
%</text>
%<multicurve fill-style= "none"
%	 points= "(285,95);(285,95);(260,95);(260,95)"
%	 />
%<multicurve fill-style= "none"
%	 points= "(285,95);(285,95);(310,70);(310,70)"
%	 />
%<multicurve fill-style= "none"
%	 points= "(310,70);(310,70);(310,30);(310,30)"
%	 />
%<multicurve fill-style= "none"
%	 points= "(310,30);(310,30);(260,30);(260,30)"
%	 />
%<multicurve fill-style= "none"
%	 points= "(305,35);(305,35);(285,55);(285,55)"
%	 />
%<multicurve fill-style= "none"
%	 points= "(285,35);(285,35);(305,35);(305,35)"
%	 />
%<multicurve fill-style= "none"
%	 points= "(285,55);(285,55);(285,35);(285,35)"
%	 />
%<text fill-style= "none"
%	 text-vert-align= "center-v"
%	 anchor-point= "(295,45)"
%	 text-frame= "noframe"
%	 text-hor-align= "center-h"
%	 >
%\DOTT
%</text>
%<text fill-style= "none"
%	 text-vert-align= "center-v"
%	 anchor-point= "(290,40)"
%	 text-frame= "noframe"
%	 text-hor-align= "center-h"
%	 >
%$\Grr$
%</text>
%<multicurve fill-style= "none"
%	 points= "(260,95);(260,95);(260,30);(260,30)"
%	 />
%<multicurve fill-style= "none"
%	 points= "(195,110);(195,110);(195,72.5);(195,72.5)"
%	 stroke-width= "0.65"
%	 />
%<text fill-style= "none"
%	 text-vert-align= "center-v"
%	 anchor-point= "(215,52.5)"
%	 text-frame= "noframe"
%	 text-hor-align= "center-h"
%	 >
%\DOTT
%</text>
%<multicurve stroke-overstrike= "true"
%	 fill-style= "none"
%	 points= "(195,72.5);(195,72.5);(215,52.5);(215,52.5)"
%	 stroke-width= "0.65"
%	 stroke-overstrike-width= "0.5"
%	 />
%<text fill-style= "none"
%	 text-vert-align= "center-v"
%	 anchor-point= "(10,50)"
%	 text-frame= "noframe"
%	 text-hor-align= "center-h"
%	 >
%$\Psee$
%</text>
%<text fill-style= "none"
%	 text-vert-align= "center-v"
%	 anchor-point= "(270,50)"
%	 text-frame= "noframe"
%	 text-hor-align= "center-h"
%	 >
%$\Psee$
%</text>
%<text fill-style= "none"
%	 text-vert-align= "center-v"
%	 anchor-point= "(115,10)"
%	 text-frame= "noframe"
%	 text-hor-align= "center-h"
%	 >
%\DOTT
%</text>
%</jpic>
%%End JPIC-XML
%LaTeX-picture environment using emulated lines and arcs
%You can rescale the whole picture (to 80% for instance) by using the command \def\JPicScale{0.8}
\ifx\JPicScale\undefined\def\JPicScale{1}\fi
\unitlength \JPicScale mm
\begin{picture}(320,140)(0,0)
\linethickness{0.3mm}
\put(30,120){\line(1,0){50}}
\linethickness{0.3mm}
\put(80,100){\line(0,1){20}}
\linethickness{0.3mm}
\put(30,100){\line(1,0){50}}
\linethickness{0.3mm}
\put(30,100){\line(0,1){20}}
\linethickness{0.3mm}
\put(55,120){\line(0,1){20}}
\linethickness{0.65mm}
\put(35,80){\line(0,1){20}}
\linethickness{0.3mm}
\put(75,0){\line(0,1){100}}
\linethickness{0.3mm}
\multiput(25,90)(0.12,-0.12){167}{\line(1,0){0.12}}
\linethickness{0.3mm}
\put(25,70){\line(1,0){20}}
\linethickness{0.3mm}
\multiput(5,90)(0.12,-0.12){167}{\line(1,0){0.12}}
\linethickness{0.3mm}
\put(5,90){\line(1,0){20}}
\linethickness{0.65mm}
\put(35,45){\line(0,1){25}}
\linethickness{0.3mm}
\put(105,120){\line(1,0){55}}
\linethickness{0.3mm}
\put(160,100){\line(0,1){20}}
\linethickness{0.3mm}
\put(125,100){\line(1,0){35}}
\linethickness{0.3mm}
\multiput(105,120)(0.12,-0.12){167}{\line(1,0){0.12}}
\linethickness{0.3mm}
\put(145,120){\line(0,1){20}}
\linethickness{0.65mm}
\put(135,45){\line(0,1){55}}
\linethickness{0.3mm}
\put(155,0){\line(0,1){100}}
\linethickness{0.3mm}
\multiput(105,90)(0.12,-0.12){167}{\line(1,0){0.12}}
\linethickness{0.3mm}
\put(105,70){\line(1,0){20}}
\linethickness{0.3mm}
\put(105,70){\line(0,1){20}}
\linethickness{0.65mm}
\put(115,80){\line(0,1){30}}
\linethickness{0.65mm}
\put(15,72.5){\line(0,1){7.5}}
\put(35,52.5){\makebox(0,0)[cc]{\DOTT}}

\put(35,80){\makebox(0,0)[cc]{\DOTT}}

\put(115,80){\makebox(0,0)[cc]{\DOTT}}

\put(72.5,0){\makebox(0,0)[cr]{$\Aah$}}

\put(152.5,0){\makebox(0,0)[cr]{$\Aah$}}

\put(25,80){\makebox(0,0)[cc]{$\PK$}}

\put(55,110){\makebox(0,0)[cc]{$\grr$}}

\put(57.5,140){\makebox(0,0)[cl]{$\Bah$}}

\put(147.5,140){\makebox(0,0)[cl]{$\Bah$}}

\put(145,110){\makebox(0,0)[cc]{$\UK$}}

\put(95,110){\makebox(0,0)[cc]{\EQLS}}

\put(110,75){\makebox(0,0)[cc]{$\Grr$}}

\linethickness{0.3mm}
\put(55,30){\line(0,1){70}}
\linethickness{0.3mm}
\put(35,0){\line(0,1){35}}
\linethickness{0.65mm}
\multiput(15,72.5)(0.12,-0.12){167}{\line(1,0){0.12}}
\linethickness{0.3mm}
\put(115,0){\line(0,1){70}}
\put(112.5,0){\makebox(0,0)[cr]{$\Xah$}}

\put(32.5,0){\makebox(0,0)[cr]{$\Xah$}}

\linethickness{0.3mm}
\put(-0,95){\line(1,0){25}}
\linethickness{0.3mm}
\multiput(25,95)(0.12,-0.12){208}{\line(1,0){0.12}}
\linethickness{0.3mm}
\put(50,30){\line(0,1){40}}
\linethickness{0.3mm}
\put(0,30){\line(1,0){50}}
\linethickness{0.3mm}
\multiput(25,55)(0.12,-0.12){167}{\line(1,0){0.12}}
\linethickness{0.3mm}
\put(25,35){\line(1,0){20}}
\linethickness{0.3mm}
\put(25,35){\line(0,1){20}}
\put(35,45){\makebox(0,0)[cc]{\DOTT}}

\put(30,40){\makebox(0,0)[cc]{$\Grr$}}

\linethickness{0.3mm}
\multiput(35,10)(0.12,0.12){167}{\line(1,0){0.12}}
\put(35,10){\makebox(0,0)[cc]{\DOTT}}

\linethickness{0.3mm}
\put(0,30){\line(0,1){65}}
\linethickness{0.3mm}
\multiput(125,55)(0.12,-0.12){167}{\line(1,0){0.12}}
\linethickness{0.3mm}
\put(125,35){\line(1,0){20}}
\linethickness{0.3mm}
\put(125,35){\line(0,1){20}}
\put(135,45){\makebox(0,0)[cc]{\DOTT}}

\put(130,40){\makebox(0,0)[cc]{$\Grr$}}

\linethickness{0.3mm}
\multiput(115,10)(0.12,0.12){167}{\line(1,0){0.12}}
\linethickness{0.3mm}
\put(135,30){\line(0,1){5}}
\linethickness{0.3mm}
\put(185,120){\line(1,0){55}}
\linethickness{0.3mm}
\put(240,100){\line(0,1){20}}
\linethickness{0.3mm}
\put(205,100){\line(1,0){35}}
\linethickness{0.3mm}
\multiput(185,120)(0.12,-0.12){167}{\line(1,0){0.12}}
\linethickness{0.3mm}
\put(225,120){\line(0,1){20}}
\linethickness{0.65mm}
\put(215,45){\line(0,1){55}}
\linethickness{0.3mm}
\put(235,0){\line(0,1){100}}
\put(232.5,0){\makebox(0,0)[cr]{$\Aah$}}

\put(227.5,140){\makebox(0,0)[cl]{$\Bah$}}

\put(225,110){\makebox(0,0)[cc]{$\UK$}}

\put(175,110){\makebox(0,0)[cc]{\EQLS}}

\put(212.5,0){\makebox(0,0)[cr]{$\Xah$}}

\linethickness{0.3mm}
\multiput(205,55)(0.12,-0.12){167}{\line(1,0){0.12}}
\linethickness{0.3mm}
\put(205,35){\line(1,0){20}}
\linethickness{0.3mm}
\put(205,35){\line(0,1){20}}
\put(215,45){\makebox(0,0)[cc]{\DOTT}}

\put(210,40){\makebox(0,0)[cc]{$\Grr$}}

\linethickness{0.3mm}
\put(215,0){\line(0,1){35}}
\linethickness{0.3mm}
\put(285,120){\line(1,0){35}}
\linethickness{0.3mm}
\put(320,100){\line(0,1){20}}
\linethickness{0.3mm}
\put(305,100){\line(1,0){15}}
\linethickness{0.3mm}
\multiput(285,120)(0.12,-0.12){167}{\line(1,0){0.12}}
\linethickness{0.3mm}
\put(305,120){\line(0,1){20}}
\linethickness{0.3mm}
\put(315,0){\line(0,1){100}}
\put(312.5,0){\makebox(0,0)[cr]{$\Aah$}}

\put(307.5,140){\makebox(0,0)[cl]{$\Bah$}}

\put(310,110){\makebox(0,0)[cc]{$\UK$}}

\put(255,110){\makebox(0,0)[cc]{\EQLS}}

\linethickness{0.65mm}
\put(295,80){\line(0,1){30}}
\linethickness{0.3mm}
\multiput(285,90)(0.12,-0.12){167}{\line(1,0){0.12}}
\linethickness{0.3mm}
\put(285,70){\line(1,0){20}}
\linethickness{0.3mm}
\multiput(265,90)(0.12,-0.12){167}{\line(1,0){0.12}}
\linethickness{0.3mm}
\put(265,90){\line(1,0){20}}
\linethickness{0.65mm}
\put(295,45){\line(0,1){25}}
\linethickness{0.65mm}
\put(275,72.5){\line(0,1){7.5}}
\put(295,52.5){\makebox(0,0)[cc]{\DOTT}}

\put(295,80){\makebox(0,0)[cc]{\DOTT}}

\put(285,80){\makebox(0,0)[cc]{$\PK$}}

\linethickness{0.3mm}
\put(295,0){\line(0,1){35}}
\linethickness{0.65mm}
\multiput(275,72.5)(0.12,-0.12){167}{\line(1,0){0.12}}
\put(292.5,0){\makebox(0,0)[cr]{$\Xah$}}

\linethickness{0.3mm}
\put(260,95){\line(1,0){25}}
\linethickness{0.3mm}
\multiput(285,95)(0.12,-0.12){208}{\line(1,0){0.12}}
\linethickness{0.3mm}
\put(310,30){\line(0,1){40}}
\linethickness{0.3mm}
\put(260,30){\line(1,0){50}}
\linethickness{0.3mm}
\multiput(285,55)(0.12,-0.12){167}{\line(1,0){0.12}}
\linethickness{0.3mm}
\put(285,35){\line(1,0){20}}
\linethickness{0.3mm}
\put(285,35){\line(0,1){20}}
\put(295,45){\makebox(0,0)[cc]{\DOTT}}

\put(290,40){\makebox(0,0)[cc]{$\Grr$}}

\linethickness{0.3mm}
\put(260,30){\line(0,1){65}}
\linethickness{0.65mm}
\put(195,72.5){\line(0,1){37.5}}
\put(215,52.5){\makebox(0,0)[cc]{\DOTT}}

\linethickness{0.65mm}
\multiput(195,72.5)(0.12,-0.12){167}{\line(1,0){0.12}}
\put(10,50){\makebox(0,0)[cc]{$\Psee$}}

\put(270,50){\makebox(0,0)[cc]{$\Psee$}}

\put(115,10){\makebox(0,0)[cc]{\DOTT}}

\end{picture}

\end{split}
\eeq
\epr