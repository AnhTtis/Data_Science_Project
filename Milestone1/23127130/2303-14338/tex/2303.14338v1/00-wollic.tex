\documentclass{article}

\usepackage{etex}

\usepackage{tikz}
\usetikzlibrary{cd,decorations,shapes}
\newenvironment{tikzar}[1][]{{}\kern-4pt\begin{tikzcd}[ampersand replacement=\&,#1]}%
{\end{tikzcd}\kern-4pt{}}



\newcommand{\arrowequal}[1]{\arrow[-,shift left=1pt]{#1}\arrow[-,shift right=1pt]{#1}}
%\newcommand{\arrowpb}{\arrow[phantom]{dr}[near start]{\lrcorner}}

\usepackage{graphicx}
\usepackage{pstricks}

\usepackage{bbold}
\usepackage{bbm}
\usepackage%[only,sslash]
{stmaryrd}


\usepackage[USenglish]{babel}
\usepackage{amssymb}
 \usepackage{amstext}
 \usepackage{amsmath}
\usepackage{amsthm}
%\usepackage{txfonts}  

\usepackage{enumerate} 

%


%%%%%%%%%%%%%%%% \input{prooftree}
%SYNTAX:
%%
%%      \prooftree
%%              hyp1            produces:
%%              hyp2
%%              hyp3            hyp1    hyp2    hyp3
%%      \justifies              -------------------- rulename
%%              concl                   concl
%%      \thickness=0.08em
%%      \shiftright 2em
%%      \using
%%              rulename
%%      \endprooftree
%%
%% where the hypotheses may be similar structures or just formulae.
%%
%% To get a vertical string of dots instead of the proof rule, do
%%
%%      \prooftree                      which produces:
%%              [hyp]
%%      \using                                  [hyp]
%%              name                              .
%%      \proofdotseparation=1.2ex                 .name
%%      \proofdotnumber=4                         .
%%      \leadsto                                  .
%%              concl                           concl
%%      \endprooftree
%%
%% Within a prooftree, \[ and \] may be used instead of \prooftree and
%% \endprooftree; this is not permitted at the outer level because it
%% conflicts with LaTeX. Also,
%%      \Justifies
%% produces a double line. In LaTeX you can use \begin{prooftree} and
%% \end{prootree} at the outer level (however this will not work for the inner
%% levels, but in any case why would you want to be so verbose?).
%%
%% All of of the keywords except \prooftree and \endprooftree are optional
%% and may appear in any order. They may also be combined in \newcommand's
%% eg "\def\Cut{\using\sf cut\thickness.08em\justifies}" with the abbreviation
%% "\prooftree hyp1 hyp2 \Cut \concl \endprooftree". This is recommended and
%% some standard abbreviations will be found at the end of this file.
%%
%% \thickness specifies the breadth of the rule in any units, although
%% font-relative units such as "ex" or "em" are preferable.
%% It may optionally be followed by "=".
%% \proofrulebreadth=.08em or \setlength\proofrulebreadth{.08em} may also be
%% used either in place of \thickness or globally; the default is 0.04em.
%% \proofdotseparation and \proofdotnumber control the size of the
%% string of dots
%%
%% If proof trees and formulae are mixed, some explicit spacing is needed,
%% but don't put anything to the left of the left-most (or the right of
%% the right-most) hypothesis, or put it in braces, because this will cause
%% the indentation to be lost.
%%
%% By default the conclusion is centered wrt the left-most and right-most
%% immediate hypotheses (not their proofs); \shiftright or \shiftleft moves
%% it relative to this position. (Not sure about this specification or how
%% it should affect spreading of proof tree.)
%
% global assignments to dimensions seem to have the effect of stretching
% diagrams horizontally.
%
%%==========================================================================

\def\introrule{{\cal I}}\def\elimrule{{\cal E}}%%
\def\andintro{\using{\land}\introrule\justifies}%%
\def\impelim{\using{\Rightarrow}\elimrule\justifies}%%
\def\allintro{\using{\forall}\introrule\justifies}%%
\def\allelim{\using{\forall}\elimrule\justifies}%%
\def\falseelim{\using{\bot}\elimrule\justifies}%%
\def\existsintro{\using{\exists}\introrule\justifies}%%

%% #1 is meant to be 1 or 2 for the first or second formula
\def\andelim#1{\using{\land}#1\elimrule\justifies}%%
\def\orintro#1{\using{\lor}#1\introrule\justifies}%%

%% #1 is meant to be a label corresponding to the discharged hypothesis/es
\def\impintro#1{\using{\Rightarrow}\introrule_{#1}\justifies}%%
\def\orelim#1{\using{\lor}\elimrule_{#1}\justifies}%%
\def\existselim#1{\using{\exists}\elimrule_{#1}\justifies}

%%==========================================================================

\newdimen\proofrulebreadth \proofrulebreadth=.05em
\newdimen\proofdotseparation \proofdotseparation=1.25ex
\newdimen\proofrulebaseline \proofrulebaseline=2ex
\newcount\proofdotnumber \proofdotnumber=3
\let\then\relax
\def\hfi{\hskip0pt plus.0001fil}
\mathchardef\squigto="3A3B
%
% flag where we are
\newif\ifinsideprooftree\insideprooftreefalse
\newif\ifonleftofproofrule\onleftofproofrulefalse
\newif\ifproofdots\proofdotsfalse
\newif\ifdoubleproof\doubleprooffalse
\let\wereinproofbit\relax
%
% dimensions and boxes of bits
\newdimen\shortenproofleft
\newdimen\shortenproofright
\newdimen\proofbelowshift
\newbox\proofabove
\newbox\proofbelow
\newbox\proofrulename
%
% miscellaneous commands for setting values
\def\shiftproofbelow{\let\next\relax\afterassignment\setshiftproofbelow\dimen0 }
\def\shiftproofbelowneg{\def\next{\multiply\dimen0 by-1 }%
\afterassignment\setshiftproofbelow\dimen0 }
\def\setshiftproofbelow{\next\proofbelowshift=\dimen0 }
\def\setproofrulebreadth{\proofrulebreadth}

%=============================================================================
\def\prooftree{% NESTED ZERO (\ifonleftofproofrule)
%
% first find out whether we're at the left-hand end of a proof rule
\ifnum  \lastpenalty=1
\then   \unpenalty
\else   \onleftofproofrulefalse
\fi
%
% some space on left (except if we're on left, and no infinity for outermost)
\ifonleftofproofrule
\else   \ifinsideprooftree
        \then   \hskip.5em plus1fil
        \fi
\fi
%
% begin our proof tree environment
\bgroup% NESTED ONE (\proofbelow, \proofrulename, \proofabove,
%               \shortenproofleft, \shortenproofright, \proofrulebreadth)
\setbox\proofbelow=\hbox{}\setbox\proofrulename=\hbox{}%
\let\justifies\proofover\let\leadsto\proofoverdots\let\Justifies\proofoverdbl
\let\using\proofusing\let\[\prooftree
\ifinsideprooftree\let\]\endprooftree\fi
\proofdotsfalse\doubleprooffalse
\let\thickness\setproofrulebreadth
\let\shiftright\shiftproofbelow \let\shift\shiftproofbelow
\let\shiftleft\shiftproofbelowneg
\let\ifwasinsideprooftree\ifinsideprooftree
\insideprooftreetrue
%
% now begin to set the top of the rule (definitions local to it)
\setbox\proofabove=\hbox\bgroup$\displaystyle % NESTED TWO
\let\wereinproofbit\prooftree
%
% these local variables will be copied out:
\shortenproofleft=0pt \shortenproofright=0pt \proofbelowshift=0pt
%
% flags to enable inner proof tree to detect if on left:
\onleftofproofruletrue\penalty1
}

%=============================================================================
% end whatever box and copy crucial values out of it
\def\eproofbit{% NESTED TWO
%
% various hacks applicable to hypothesis list 
\ifx    \wereinproofbit\prooftree
\then   \ifcase \lastpenalty
        \then   \shortenproofright=0pt  % 0: some other object, no indentation
        \or     \unpenalty\hfil         % 1: empty hypotheses, just glue
        \or     \unpenalty\unskip       % 2: just had a tree, remove glue
        \else   \shortenproofright=0pt  % eh?
        \fi
\fi
%
% pass out crucial values from scope
\global\dimen0=\shortenproofleft
\global\dimen1=\shortenproofright
\global\dimen2=\proofrulebreadth
\global\dimen3=\proofbelowshift
\global\dimen4=\proofdotseparation
\global\count255=\proofdotnumber
%
% end the box
$\egroup  % NESTED ONE
%
% restore the values
\shortenproofleft=\dimen0
\shortenproofright=\dimen1
\proofrulebreadth=\dimen2
\proofbelowshift=\dimen3
\proofdotseparation=\dimen4
\proofdotnumber=\count255
}

%=============================================================================
\def\proofover{% NESTED TWO
\eproofbit % NESTED ONE
\setbox\proofbelow=\hbox\bgroup % NESTED TWO
\let\wereinproofbit\proofover
$\displaystyle
}%
%
%=============================================================================
\def\proofoverdbl{% NESTED TWO
\eproofbit % NESTED ONE
\doubleprooftrue
\setbox\proofbelow=\hbox\bgroup % NESTED TWO
\let\wereinproofbit\proofoverdbl
$\displaystyle
}%
%
%=============================================================================
\def\proofoverdots{% NESTED TWO
\eproofbit % NESTED ONE
\proofdotstrue
\setbox\proofbelow=\hbox\bgroup % NESTED TWO
\let\wereinproofbit\proofoverdots
$\displaystyle
}%
%
%=============================================================================
\def\proofusing{% NESTED TWO
\eproofbit % NESTED ONE
\setbox\proofrulename=\hbox\bgroup % NESTED TWO
\let\wereinproofbit\proofusing
\kern0.3em$
}

%=============================================================================
\def\endprooftree{% NESTED TWO
\eproofbit % NESTED ONE
% \dimen0 =     length of proof rule
% \dimen1 =     indentation of conclusion wrt rule
% \dimen2 =     new \shortenproofleft, ie indentation of conclusion
% \dimen3 =     new \shortenproofright, ie
%                space on right of conclusion to end of tree
% \dimen4 =     space on right of conclusion below rule
  \dimen5 =0pt% spread of hypotheses
% \dimen6, \dimen7 = height & depth of rule
%
% length of rule needed by proof above
\dimen0=\wd\proofabove \advance\dimen0-\shortenproofleft
\advance\dimen0-\shortenproofright
%
% amount of spare space below
\dimen1=.5\dimen0 \advance\dimen1-.5\wd\proofbelow
\dimen4=\dimen1
\advance\dimen1\proofbelowshift \advance\dimen4-\proofbelowshift
%
% conclusion sticks out to left of immediate hypotheses
\ifdim  \dimen1<0pt
\then   \advance\shortenproofleft\dimen1
        \advance\dimen0-\dimen1
        \dimen1=0pt
%       now it sticks out to left of tree!
        \ifdim  \shortenproofleft<0pt
        \then   \setbox\proofabove=\hbox{%
                        \kern-\shortenproofleft\unhbox\proofabove}%
                \shortenproofleft=0pt
        \fi
\fi
%
% and to the right
\ifdim  \dimen4<0pt
\then   \advance\shortenproofright\dimen4
        \advance\dimen0-\dimen4
        \dimen4=0pt
\fi
%
% make sure enough space for label
\ifdim  \shortenproofright<\wd\proofrulename
\then   \shortenproofright=\wd\proofrulename
\fi
%
% calculate new indentations
\dimen2=\shortenproofleft \advance\dimen2 by\dimen1
\dimen3=\shortenproofright\advance\dimen3 by\dimen4
%
% make the rule or dots, with name attached
\ifproofdots
\then
        \dimen6=\shortenproofleft \advance\dimen6 .5\dimen0
        \setbox1=\vbox to\proofdotseparation{\vss\hbox{$\cdot$}\vss}%
        \setbox0=\hbox{%
                \advance\dimen6-.5\wd1
                \kern\dimen6
                $\vcenter to\proofdotnumber\proofdotseparation
                        {\leaders\box1\vfill}$%
                \unhbox\proofrulename}%
\else   \dimen6=\fontdimen22\the\textfont2 % height of maths axis
        \dimen7=\dimen6
        \advance\dimen6by.5\proofrulebreadth
        \advance\dimen7by-.5\proofrulebreadth
        \setbox0=\hbox{%
                \kern\shortenproofleft
                \ifdoubleproof
                \then   \hbox to\dimen0{%
                        $\mathsurround0pt\mathord=\mkern-6mu%
                        \cleaders\hbox{$\mkern-2mu=\mkern-2mu$}\hfill
                        \mkern-6mu\mathord=$}%
                \else   \vrule height\dimen6 depth-\dimen7 width\dimen0
                \fi
                \unhbox\proofrulename}%
        \ht0=\dimen6 \dp0=-\dimen7
\fi
%
% set up to centre outermost tree only
\let\doll\relax
\ifwasinsideprooftree
\then   \let\VBOX\vbox
\else   \ifmmode\else$\let\doll=$\fi
        \let\VBOX\vcenter
\fi
% this \vbox or \vcenter is the actual output:
\VBOX   {\baselineskip\proofrulebaseline \lineskip.2ex
        \expandafter\lineskiplimit\ifproofdots0ex\else-0.6ex\fi
        \hbox   spread\dimen5   {\hfi\unhbox\proofabove\hfi}%
        \hbox{\box0}%
        \hbox   {\kern\dimen2 \box\proofbelow}}\doll%
%
% pass new indentations out of scope
\global\dimen2=\dimen2
\global\dimen3=\dimen3
\egroup % NESTED ZERO
\ifonleftofproofrule
\then   \shortenproofleft=\dimen2
\fi
\shortenproofright=\dimen3
%
% some space on right and flag we've just made a tree
\onleftofproofrulefalse
\ifinsideprooftree
\then   \hskip.5em plus 1fil \penalty2
\fi
}
%%%%%%%%%%%%%%% END prooftree



    
\renewcommand{\to}{\xrightarrow{}}%{\longrightarrow}
\newcommand{\ot}{\xleftarrow{}}%{\longleftarrow}
\newcommand{\tto}[1]{\xrightarrow{#1}}
\newcommand{\oot}[1]{\xleftarrow{#1}}
\newcommand{\ttto}[2]{\xrightarrow[#2]{#1}}
\newcommand{\ooot}[2]{\xleftarrow[#2]{#1}}
\newcommand{\mono}{\rightarrowtail}
\newcommand{\epi}{\twoheadrightarrow}
\newcommand{\ipe}{\twoheadlefttarrow}
\newcommand{\inclusion}{\hookrightarrow}
\newcommand{\mmono}[1]{\stackrel{#1}\rightarrowtail}
\newcommand{\eepi}[1]{\stackrel{#1}\twoheadrightarrow}
\newcommand{\iipe}[1]{\stackrel{#1}\twoheadleftarrow}
\newcommand{\iinclusion}[1]{\stackrel{#1}\hookrightarrow}
\renewcommand{\mapsfrom}{\mathrel{\reflectbox{\ensuremath{\mapsto}}}}
\newcommand{\pfn}{\rightharpoonup}
\newcommand{\ppfn}[1]{\stackrel{#1}\rightharpoonup}

\newcommand{\xxp}[2]{\left[{#1},{#2}\right]}
\newcommand{\xxpp}[2]{\xxp{{#1}^+}{#2}}
%\newcommand{\xxpp}[2]{\llbracket{#1}^+,{#2}\rrbracket}

\newcommand{\tleft}{\triangleleft}%{-}%
\newcommand{\tright}{\triangleright}%{+}%
\newcommand{\tstall}{{\scriptstyle \Box}}%{{\rm o}}%



\newcommand{\WP}{\mbox{\Large $\wp$}}
\newcommand{\WPf}{\mbox{\Large $\wp$}_{f}}
\newcommand{\Set}{\mathsf{Set}}
\newcommand{\Pfn}{{\sf Pfn}}
\newcommand{\Vect}{{\sf Vec}}
\newcommand{\FSet}{{\sf FSet}}
\newcommand{\FVec}{{\sf FVec}}
\newcommand{\Cat}{{\sf Cat}}
\newcommand{\Pos}{{\sf Pos}}
\newcommand{\CAT}{{\sf CAT}}
\newcommand{\Rel}{{\sf Rel}}
\newcommand{\FRel}{{\sf FRel}}
\newcommand{\FHilb}{{\sf FHilb}}
\newcommand{\Hilb}{{\sf Hilb}}
\newcommand{\im}{{\rm im}}
\newcommand{\id}{{\rm id}}
\newcommand{\blank}{\sqcup}

%\newcommand{\cata}[1]{\llparenthesis {#1} \rrparenthesis}
%\newcommand{\ana}[1]{\left\llbracket {#1} \right\rrbracket}
%\newcommand{\run}[1]{\left\{ {#1} \right\}}
%\newcommand{\Run}[1]{\left\{\!\lvert {#1} \rvert\!\right\}}
%\newcommand{\RRun}{\left\{\!\lvert \, \rvert\!\right\}}

\newcommand{\AAA}{{\cal A}}
\newcommand{\BBB}{{\cal B}}
\newcommand{\CCC}{{\cal C}}
\newcommand{\DDD}{{\cal D}}
\newcommand{\EEE}{{\cal E}}
\newcommand{\FFF}{{\cal F}}
\newcommand{\GGG}{{\cal G}}
\newcommand{\HHH}{{\cal H}}
\newcommand{\III}{{\cal I}}
\newcommand{\JJJ}{{\cal J}}
\newcommand{\KKK}{{\cal K}}
\newcommand{\LLL}{{\cal L}}
\newcommand{\MMM}{{\cal M}}
\newcommand{\NNN}{{\cal N}}
\newcommand{\OOO}{{\cal O}}
\newcommand{\PPP}{{\cal P}}
\newcommand{\QQQ}{{\cal Q}}
\newcommand{\RRR}{{\cal R}}
\newcommand{\SSS}{{\cal S}}
\newcommand{\TTT}{{\cal T}}
\newcommand{\UUU}{{\cal U}}
\newcommand{\VVV}{{\cal V}}
\newcommand{\WWW}{{\cal W}}
\newcommand{\XXX}{{\cal X}}
\newcommand{\YYY}{{\cal Y}}
\newcommand{\ZZZ}{{\cal Z}}
\renewcommand{\Bbb}{\mathbb}
\newcommand{\AAa}{{\Bbb A}}
\newcommand{\BBb}{{\Bbb B}}
\newcommand{\CCc}{{\Bbb C}}
\newcommand{\DDd}{{\Bbb D}}
\newcommand{\EEe}{{\Bbb E}}
\newcommand{\FFf}{{\Bbb F}}
\newcommand{\GGg}{{\Bbb G}}
\newcommand{\HHh}{{\Bbb H}}
\newcommand{\IIi}{{\Bbb I}}
\newcommand{\JJj}{{\Bbb J}}
\newcommand{\KKk}{{\Bbb K}}
\newcommand{\LLl}{{\Bbb L}}
\newcommand{\MMm}{{\Bbb M}}
\newcommand{\NNn}{{\Bbb N}}
\newcommand{\OOo}{{\Bbb O}}
\newcommand{\PPp}{{\Bbb P}}
\newcommand{\QQq}{{\Bbb Q}}
\newcommand{\RRr}{{\Bbb R}}
\newcommand{\SSs}{{\Bbb S}}
\newcommand{\TTt}{{\Bbb T}}
\newcommand{\UUu}{{\Bbb U}}
\newcommand{\VVv}{{\Bbb V}}
\newcommand{\WWw}{{\Bbb W}}
\newcommand{\XXx}{{\Bbb X}}
\newcommand{\YYy}{{\Bbb Y}}
\newcommand{\ZZz}{{\Bbb Z}}

\newcommand{\OOne}{\mathbbm{1}}
\newcommand{\TTwo}{\mathbbm{2}}
\newcommand{\TThree}{\mathbbm{3}}
\newcommand{\FFour}{\mathbbm{4}}
\newcommand{\FFive}{\mathbbm{5}}
\newcommand{\SSix}{\mathbbm{6}}
\newcommand{\SSeven}{\mathbbm{7}}
\newcommand{\EEight}{\mathbbm{8}}
\newcommand{\NNine}{\mathbbm{9}}

\newcommand{\AaA}{{\mathfrak A}}
\newcommand{\BbB}{{\mathfrak B}}
\newcommand{\CcC}{{\mathfrak C}}
\newcommand{\DdD}{{\mathfrak D}}
\newcommand{\EeE}{{\mathfrak E}}
\newcommand{\FfF}{{\mathfrak F}}
\newcommand{\GgG}{{\mathfrak G}}
\newcommand{\HhH}{{\mathfrak H}}
\newcommand{\IiI}{{\mathfrak I}}
\newcommand{\JjJ}{{\mathfrak J}}
\newcommand{\KkK}{{\mathfrak K}}
\newcommand{\LlL}{{\mathfrak L}}
\newcommand{\MmM}{{\mathfrak M}}
\newcommand{\NnN}{{\mathfrak N}}
\newcommand{\OoO}{{\mathfrak O}}
\newcommand{\PpP}{{\mathfrak P}}
\newcommand{\QqQ}{{\mathfrak Q}}
\newcommand{\RrR}{{\mathfrak R}}
\newcommand{\SsS}{{\mathfrak S}}
\newcommand{\TtT}{{\mathfrak T}}
\newcommand{\UuU}{{\mathfrak U}}
\newcommand{\VvV}{{\mathfrak V}}
\newcommand{\WwW}{{\mathfrak W}}
\newcommand{\XxX}{{\mathfrak X}}
\newcommand{\YyY}{{\mathfrak Y}}
\newcommand{\ZzZ}{{\mathfrak Z}}

\newcommand{\mathbold}[1]{\mbox{\boldmath $#1$}}

\newcommand{\aaa}{{\mathbold a}}
\newcommand{\bbb}{{\mathbold b}}
\newcommand{\ccc}{{\mathbold c}}
\newcommand{\ddd}{{\mathbold d}}
\newcommand{\eee}{{\mathbold e}}
\newcommand{\fff}{{\mathbold f}}
\renewcommand{\ggg}{{\mathbold g}}
\newcommand{\hhh}{{\mathbold h}}
\newcommand{\iii}{{\mathbold i}}
\newcommand{\jjj}{{\mathbold j}}
\newcommand{\kkk}{{\mathbold k}}
\newcommand{\elll}{{\mathbold \ell}}
\newcommand{\mmm}{{\mathbold m}}
\newcommand{\nnn}{{\mathbold n}}
\newcommand{\ooo}{{\mathbold o}}
\newcommand{\ppp}{{\mathbold p}}
\newcommand{\qqq}{{\mathbold q}}
\newcommand{\rrr}{{\mathbold r}}
\newcommand{\sss}{{\mathbold s}}
\newcommand{\ttt}{{\mathbold t}}
\newcommand{\uuu}{{\mathbold u}}
\newcommand{\vvv}{{\mathbold v}}
\newcommand{\www}{{\mathbold w}}
\newcommand{\xxx}{{\mathbold x}}
\newcommand{\yyy}{{\mathbold y}}
\newcommand{\zzz}{{\mathbold z}}


\mathcode`\<="4268 %left delimiter
\mathcode`\>="5269 %right delimiter
\mathchardef\gt="313E %relation >
\mathchardef\lt="313C %relation <


%  %===========================================================================
%  %      P. TAYLOR'S "END OF PROOF BOX":
%  %
%  %
%  %  The complexity of the macro necessary to get a little box on the
%  %  right-hand-side at the end of a proof is amazing.  It really does
%  %  have to be this long!  Otherwise you're liable to get it at the
%  %  beginning of the next line, or even on the next page.
%  %
%  \def\pushright#1{{%              set up
%     \parfillskip=0pt            % so \par doesnt push \square to left
%     \widowpenalty=10000         % so we dont break the page before \square
%     \displaywidowpenalty=10000  % ditto
%     \finalhyphendemerits=0      % TeXbook exercise 14.32
%    %
%    %                 horizontal
%     \leavevmode                 % \nobreak means lines not pages
%     \unskip                     % remove previous space or glue
%     \nobreak                    % don't break lines
%     \hfil                       % ragged right if we spill over
%     \penalty50                  % discouragement to do so
%     \hskip.2em                  % ensure some space
%     \null                       % anchor following \hfill
%     \hfill                      % push \square to right
%     {#1}                        % the end-of-proof mark (or whatever)
%    %
%    %                   vertical
%     \par}}                      % build paragraph
% 
%  % prefer proofs with statements, also space after
%  \def\qed{\pushright{$\square$}\penalty-700 \smallskip}

%%=====================================================================
%
%       For laying out a proof:  \begin{prf}{} ... \end{prf}
%       The argument is in case the proof is a continuation as in:
%               \begin{prf}{\ref{four-color} continued} ... \end{prf}
%       if you do not use the argument be careful to use the brackets!
%
%  							due to RAGS
%-----------------------------------------------------------------------

% \newenvironment{prf}[1]{\begin{trivlist} \item[{\bf ~Proof}#1.]}%
% {\qed\end{trivlist}}

\newcommand{\be}[1]{\begin{#1}}
\newcommand{\bbe}[2]{\begin{#1}[#2]}
\newcommand{\ee}[1]{\end{#1}}
\newcommand{\beq}{\begin{equation}}
\newcommand{\eeq}{\end{equation}}
\newcommand{\ba}[1]{\begin{array}{#1}}
\newcommand{\ea}{\end{array}}
\newcommand{\bea}{\begin{eqnarray}}
\newcommand{\eea}{\end{eqnarray}}
\newcommand{\bear}{\begin{eqnarray*}}
\newcommand{\eear}{\end{eqnarray*}}

\theoremstyle{plain}
\newtheorem{theorem}{Theorem}[section]

\newtheorem*{proposition*}{Proposition}

\newtheorem{proposition}[theorem]{Proposition}
\newtheorem{definition}[theorem]{Definition}
\newtheorem{corollary}[theorem]{Corollary}
\newtheorem{lemma}[theorem]{Lemma}
\theoremstyle{remark}
\newtheorem*{remark}{Remark}
\newtheorem*{terminology}{Terminology}
\theoremstyle{remark}
\newtheorem*{notation}{Notation}
\theoremstyle{remark}
\newtheorem*{explanation}{Explanation}
\theoremstyle{remark}
\newtheorem*{exercise}{Exercise}


%\newtheorem{theorem}{Theorem}[section]
%\newtheorem{definition}[theorem]{Definition}
%\newtheorem{proposition}[theorem]{Proposition}
%\newtheorem{lemma}[theorem]{Lemma}
%\newtheorem{corollary}[theorem]{Corollary}
%
%
%
%%\newtheorem{cond}{}[thm]
%%\renewcommand{\thecond}{{(\alph{cond})}}
%%\newenvironment{condition}{\vspace{-.5\baselineskip}\begin{cond}}{\end{cond}}
%%\newtheorem{prenumb}[thm]{\hspace{-1ex}}
%%\newenvironment{numb}{\begin{prenumb}\rm}{\end{prenumb}}
%
%\newcommand{\nthm}[1]{\newtheorem{#1}[theorem]{#1}}
%\newcommand{\nenv}[1]{\newtheorem{pre#1}[theorem]{#1}%
%		      \newenvironment{#1}{\begin{pre#1}\rm}%
%					    {\end{pre#1}}}  
%					    
%\nenv{Remark}
%\nenv{Terminology}
%\nenv{Notation}	
%\nenv{Explanation}				    
%
%\newenvironment{spec}{\begin{verbatim}\tt}{\end{verbatim}}
%
%



%\newcommand{\tot}[1]{{#1}^\bullet}
%\newcommand{\Base}[1]{{#1}^\flat}
%			    
%\newcommand{\comp}{\, ;}
%
%\newcommand{\supp}{{\sf supp}}
%
%\newcommand{\restr}{\!\restriction}
%\newcommand{\halts}{\!\downarrow}
%\newcommand{\grad}[1]{\left\|{#1}\right\|}
%\newcommand{\tran}{\shortrightarrow}
%\newcommand{\opls}{\oplus}
%
%\newcommand{\initstate}{q}
%\newcommand{\initdata}{a}
%\newcommand{\uu}{w}
%\newcommand{\KT}{T}
%\newcommand{\CXX}{c}
%\newcommand{\UX}[2]{u\left({#1},{#2}\right)}
%\newcommand{\TM}{\TTT}
%
%\newcommand{\iif}{\mbox{\it if}}
%\newcommand{\true}{\mathtt{t}}
%\newcommand{\false}{\mathtt{f}}

\newenvironment{prf}[1]{\begin{trivlist} \item[{\bf ~Proof #1.}]}%
{\qed\end{trivlist}}


\newcommand{\bpr}{\begin{prf}{\!\!}}
\newcommand{\epr}{\end{prf}}
\newcommand{\bprf}[1]{\begin{prf}{#1}}
\newcommand{\eprf}{\end{prf}}	


% !TEX root = 00-wollic.tex
%!TEX TS-program = xelatex
%!TEX encoding = UTF-8 Unicode

\setlength{\parskip}{.5\baselineskip}
%\setlength{\topsep}{0pt}
\setlength{\parindent}{0in}
\setlength{\itemindent}{0in}

\usepackage{enumerate}
\usepackage{rotating}
\usepackage{amsfonts}
\usepackage{amssymb}
\usepackage{stmaryrd}
\usepackage{fullpage}
\usepackage{authblk}

\usepackage[LGR,T1]{fontenc}
\newcommand{\textgreek}[1]{\begingroup\fontencoding{LGR}\selectfont#1\endgroup}

\usepackage{hyperref}

\input{prooftree}

%--------------------------------------------------------------%
%			Symbols                                %
%--------------------------------------------------------------%


%\newcommand{\leftsquigarrow}{\turnbox{180}{$\rightsquigarrow$}}

\newcommand{\leftsquigarrow}{\mathbin{\rotatebox[origin=c]{180}{$\rightsquigarrow$}}}

\renewcommand{\paragraph}[1]{\vspace{.3\baselineskip}\noindent\textbf{#1}}
%\newcommand{\WP}{\raisebox{.15\baselineskip}{\Large\ensuremath{\wp}}}

\renewcommand{\vec}[1]{\mathbf{#1}}
\newcommand{\DP}{\LLl}%{\PPp}

\newcommand{\fixp}{\rotatebox[origin=c]{90}{$\rightsquigarrow$}}%{\curlywedgeuparrow}%\uparrow}%{\Phi}%{\varpi}
\newcommand{\sdivg}{\rotatebox[origin=c]{90}{$\scriptstyle\multimap$}}
\newcommand{\divg}{\rotatebox[origin=c]{90}{$\multimap$}}
\newcommand{\convg}{\mbox{$\rotatebox[origin=c]{90}{$\multimap$}^{\raisebox{-.3ex}{\hspace{-1ex}\footnotesize$\bullet$}}$}}
\newcommand{\sconvg}{\mbox{\scriptsize $\rotatebox[origin=c]{90}{$\multimap$}^{\raisebox{-.29ex}{\hspace{-1.2ex}$\scriptstyle\bullet$}}$}}

\newcommand{\congv}{\mbox{$\rotatebox[origin=c]{-90}{$\multimap$}_{\raisebox{0.15ex}{\hspace{-0.9ex}\footnotesize$\bullet$}}$}}
\newcommand{\scongv}{\mbox{$\rotatebox[origin=c]{-90}{$\multimap$}_{\raisebox{0.15ex}{\hspace{-0.9ex}$\scriptstyle\bullet$}}$}}

\newcommand{\mnd}{\nabla}%{\varrho}%{\triangledown}%
\newcommand{\unt}{\congv}%{\!\uparrow\ }%{\mathbin{\rotatebox[origin=c]{180}{!}}}%{\mbox{<}}%{\blacktriangledown}%{\bot}%{\ome}%{?}%{\l}
\newcommand{\sunt}{\scongv}
\newcommand{\cmn}{\Delta}%{\delta}%{\vartriangle}%
\newcommand{\cun}{\convg}%{\downarrow\!}%{!}%{\top}%{\o}{\blacktriangle}
\newcommand{\scun}{\sconvg}

\newcommand{\tot}[1]{{#1}^\bullet}



%\newcommand{\comp}[2]{{#2}\circ {#1}}
%
%\newcommand{\tev}{\uev{}}
\newcommand{\uev}[1]{\mbox{\large$\left\{\mbox{\normalsize $#1$}\right\}$}}
\newcommand{\pev}[1]{\left[{#1}\right]}%{\left\lceil{#1}\right\rceil}
\newcommand{\prtial}{\pev{\,}}
\newcommand{\universal}{\mbox{\large$\{\}$}}
\newcommand{\uuniversal}{\{\}}

\newcommand{\sta}[1]{{#1}^{\prime}}%{{#1}^{\triangleright}}%^\circ}
\newcommand{\out}[1]{{#1}^{\prime\prime}}%{{#1}^{\odot}}%{\triangleleft}}%\bullet}

\newcommand{\EQLS}{\mbox{\large$=$}}

\newcommand{\enco}[1]{\left\ulcorner{#1}\right\urcorner}%{\llceil{#1}\rrceil}%

\newcommand{\ana}[1]{\llbracket{#1}\rrbracket}

\newcommand{\Prov}{\P}%{{\sf P}}
\newcommand{\Bool}{\BBb}


\renewcommand{\vec}[1]{\mathbf{#1}}
\usepackage{authblk}

\graphicspath{{PICS/}}

\begin{document}

\title{From G\"odel's Incompleteness Theorem\\
to the completeness of bot religions\\[.5ex]
{\Large (Extended abstract)}}

\author{Dusko~Pavlovic\thanks{Partially supported by NSF and AFOSR.}} 
\affil{University of Hawaii, Honolulu HI}
\author[**]{Temra Pavlovic}

\date{}

\maketitle

\begin{abstract} 
Hilbert and Ackermann asked for a method to consistently extend incomplete theories to complete theories. G\"odel essentially proved that any theory capable of encoding its own statements and their proofs contains statements that are true but not provable. Hilbert did not accept that G\"odel's construction answered his question, and in his late writings and lectures, G\"odel agreed that it did not, since theories can be completed incrementally, by adding axioms to prove ever more true statements, as science normally does, with completeness as the vanishing point. This pragmatic view of validity is familiar not only to scientists who conjecture test hypotheses but also to real-estate agents and other dealers, who conjure claims, albeit invalid, as necessary to close a deal, confident that they will be able to conjure other claims, albeit invalid, sufficient to make the first claims valid. We study the underlying logical process and describe the trajectories leading to testable but unfalsifiable theories to which bots and other automated learners are likely to converge.
\end{abstract}



\section{Introduction}% wrong, fake, artificial causation}
\label{Sec:Intro}
% !TEX root = 00-wollic.tex

Logic as the theory of theories was originally developed to prove true statements. Here we study developments in the opposite direction: modifying logical theories and their interpretations to make provable some previously unprovable statements. The purpose of such transitions in science is to extend existing theories with new knowledge. The purpose of such transitions in public life is to modify some public views to better suit  private preferences. The instances of such transitions span a broad gamut of techniques, from unsupervised learning to conditioning.

If unprovable claims of a theory are true in a nontrivial model, the theory can be consistently extended to make them provable. That was clear already in the time of G\"odel's Incompleteness Theorem \cite{GoedelK:ueber}. If the unprovable claims are false, then making them true and provable requires modifying the model and reinterpreting the theory. Such dynamic reassignments of meaning bring us into the realm of \emph{dynamic logic}. If the propositions from a lattice $\TTT$ are used as assertions about the states of the world or the states of our beliefs, then the dynamic changes of these assertions under the influence of events from a lattice $\EEE$ can be expressed in terms of the \emph{Hoare triples}
\[ A\uev e B\]
saying that the event $E$ after the precondition state $A$ leads to the postcondition state $B$. The \emph{Hoare logic}\/ evolved in the late 60s as a method for reasoning about program expressions generating the events in $\EEE$ in terms of the program annotations $A$ and $B$ as formal versions of the preconditions and the postconditions habitually inserted by programmers into their code as comments, to clarify and stipulate the intended functions of blocks of code \cite{FloydR:meaning,Hoare-logic}. The propositional algebra of dynamic logic can be viewed as a monotone map
\[ \TTT^{o}\times \EEE\times \TTT\tto{ - \uev - -} \OOO\]
where $\OOO$ is a lattice of truth values. If the lattice $\TTT$ is complete, then each event $e\in \EEE$ induces a Galois connection
\[ A\rtimes e \vdash B \ \ \iff\ \ A\uev e B\ \ \iff\ \ A\vdash [e]B\]
determining a \emph{dynamic modality}\/ $[e]\colon\TTT\to\TTT$ for every $e\in \EEE$ \cite{PrattV:dynamic}. The induced interior operation $\left([e]B\right)\rtimes E\vdash B$ says that $[e]B$ is the weakest precondition that assures $B$ after $e$. The induced closure $A\vdash [e]\left(A\rtimes e\right)$ says that $A\rtimes e$ is the strongest postcondition that can be guaranteed by $A$ before $e$. Beyond such program annotations, dynamic logic has been developed in many directions  \cite{BenthemJ:dynamic-book,DitmarschH:dynamic-epistemic,Kozen:dynamic-book}. Here we use it as a backdrop for evolution of theories and their interpretations from science to religion.


\section{World as a monoidal category}
\label{Sec:world}
% !TEX root = 00-wollic.tex

\subsection{Belief states as objects}
Together, a logical theory and its interpretation constitute a \emph{state of belief}. They can be presented in the standard Tarskian format, where a theory is a quadruple of sorts, operations, predicates, and axioms, and its  interpretation is an inductively defined model  \cite{Chang-Keisler}. To streamline further constructions, we assume that each theory is presented as a categorical \emph{sketch}\/ and that its models are presented in extended functorial semantics 
\cite{AdamekJ:locpac,Ehresmann-Bastiani:sketches,MakkaiM:acccfc}. %A popular overview is \cite{BarrM:ttt}, where sketches are called \emph{theories}.

\begin{definition}\label{def:state}
A \emph{clone} $\Sigma$ is a cartesian category\footnote{We stick with the traditional terminology where a category is cartesian when it has cartesian products.} freely generated by sorts, operations, and equational axioms of a logical theory. A \emph{theory}\/ is a pair $\Theta=<\Sigma,\Gamma>$, where $\Sigma$ is a clone and $\Gamma$ is a set of cones and cocones in $\Sigma$, capturing the general axioms\footnote{The equational axioms can also be subsumed as cones or cocones, in which case they could be omitted from $\Sigma$ and it would boil down to the free category generated by sorts and operations.} of the logical theory. A \emph{model}\/ of $\Theta$ is a cartesian functor $\MMM\colon \Sigma \tto\Gamma \Set$ mapping the $\Gamma$-cones into limit cones and the $\Gamma$-cocones into colimit cocones. A \emph{state of belief} (or \emph{belief state}) is a triple 
\bear
A & = & \left<\Sigma_{A},\Gamma_{A}, \MMM_{A}\right> 
\eear
where $\Theta_{A} = \left<\Sigma_{A},\Gamma_{A}\right>$ is a theory and $\MMM_{A}$ its model in a category $\Set$ of sets and functions. An\/ element of the model $\MMM_{A}$ is called an \emph{observable} of the state $A$.
\end{definition}

\paragraph{Comment.} A belief state $A$ should not be thought of as a particular model $\MMM_A$ of the theory $\Theta_A$, but rather as the (accessible) category of all models of $\Theta_A$ consistent with $\MMM_A$. The model thus expresses the properties that may not be stated effectively\footnote{E.g., the set of all true statements of Peano arithmetic is expressed by its standard model, but most of them cannot be stated effectively.}. The described structure of the belief states can be refined to capture other features of theories in science and engineering, including statistical and  complexity-theoretic valuations \cite{RissanenJ:MDL,WallaceCS:MML}. While such refinements would have no real impact on our considerations, they signal that we are in the realm of \emph{inductive}\/ inference, which may feel unusual for the Tarskian formalism, usually concerned with the deductive aspects only. The fact that the theory $\Theta_{A}$ has a model $\MMM_{A}$ implies that the state of belief $A$ is logically consistent but it does not imply that it is true within an external frame of reference, a \emph{``reality''}\/ that may drive the state changes, i.e. the processes of extending and reinterpreting theories. A possible intuition is that observable elements of the model $\MMM_{A}$ may not be observed, or may be found to be incompatible with actual observations.

\paragraph{Examples.} Logical theories and their models, in any of the familiar formats, Tarskian, categorical, or statistical,  are obvious examples of belief states. They arise not only in natural sciences but also in social systems, e.g. as policy formalizations. Software specifications are formalized as logical theories even more often, and their implementations can then be presented as models. The functorial semantics view was spelled out in \cite{PavlovicD:FOPS}, used in a software synthesis tool \cite{PavlovicD:ASE01,PavlovicD:SDR,PavlovicD:AMAST08}, and applied in algorithm design \cite{PavlovicD:ManaFest,PavlovicD:MPC10}. The evolution of software components can thus be formalized in terms of evolving belief states.

\subsection{Belief transitions as morphisms}
Intuitively, a transition $f$ from a state $A$ to a state $B$ is a process transforming  $A$-observables to $B$-observables. We first consider \emph{state reinterpretations}\/ as transitions and then expand the perspective to \emph{state explanations}\/ as transitions.

\begin{definition}\label{Def:interpretable}
An \emph{interpretable belief transition}\/  $f\colon A\to B$ is a cartesian functor $f\colon \Sigma_{A}\ooot{\Gamma_{A}}{\Gamma_{B}} \Sigma_{A}$ mapping $\Gamma_{B}$-(co)cones to $\Gamma_{A}$-(co)cones and moreover making the following diagram compute
\beq\label{eq:def-intepretable}
\begin{tikzar}{}
\Sigma_{A}\ar{dr}[description]{\MMM_{A}} \&\& \Sigma_{B}\ar{dl}[description]{\MMM_{B}} \ar[bend right]{ll}[swap]{f}\\
\& \Set
\end{tikzar}
\eeq
where the models $\MMM_A$ and $\MMM_B$ also preserve, respectively, the $\Gamma_A$ and $\Gamma_B$ (co)cones. \end{definition}

\paragraph{\em Interpretations as assignment commands.} The structure of interpretable transitions between software specifications and the colimits of their diagrams were spelled out in  \cite{PavlovicD:FOPS,PavlovicD:ASE01}. Since the software specifications are finite, an interpretation $f\colon \Sigma_{A}\ot \Sigma_{B}$ boils down to a tuple of assignments 
\[x_{1}:=t_{1}\ ;\  x_{2}:=t_{2}\ ;\ldots;\  x_{n}:=t_{n}\]
of terms $\vec t=<t_{1}, t_{2},\ldots, t_{n}>$ from $\Sigma_{A}$ to variables $\vec x=<x_{1}, x_{2},\ldots, x_{n}>$ from $ \Sigma_{B}$ in such a way that, for each axiom $\gamma\in \Gamma_{B}$, the interpretation 
\bear
f(\gamma) &= & [\vec{x}:=\vec {t}]\gamma
\eear  
is a theorem derivable from the axioms in $\Gamma_{A}$. In terms of Hoare logic \cite{Hoare-logic}, an interpretable state transition $f\colon \Sigma_{A}\ot \Sigma_{B}$ is just a Hoare triple $\Theta_{A}\uev{\vec{x}:=\vec {t}} \Theta_{B}$. This triple was defined to be valid if and only if $\Theta_{A}\vdash [\vec{x}:=\vec {t}]\Theta_{B}$ holds, where $[\vec{x}:=\vec {t}]\Theta_{B}$ is the result substituting the $\Theta_{A}$-terms $\vec t$ for $\Theta_{B}$-variables $\vec x$ in all axioms $\gamma\in \Gamma_{B}$. The triple thus says that this substitution instance of $\Theta_{B}$ should be derivable in $\Theta_{A}$. Condition \eqref{eq:def-intepretable} moreover requires that this theory  reinterpretation recovers the model $\MMM_{B}$ from the model $\MMM_{A}$.

But the computational states required at the annotated program points cannot in general be transformed into one another by mere substitutions even in Hoare logic. The desired transitions from state to state are implemented by programs.

\paragraph{Updates, explanations, predictions.} Updating scientific theories, social policies, or general claims usually  transforms the current belief states in ways that do not boil down variable substitutions. A process that maps the observables from one belief state to another belief state may not even be observable. Computational processes use internal variables, intermediary types, and many computational operations besides the value assignments to  transform observable inputs of type $A$ into observable outputs of type $B$. An economic process may be claimed to increase employment rates in the general population by cutting taxes on the rich. Such causal links between two types of observables are explained by unobservable mental processes in two social groups and by hidden variables connecting them. A physical theory may transform observable classical measurements as inputs into observable  classical predictions as outputs using an internal model of unobservable quantum interactions. 

Intuitively, an explanation is not just a hypothesis that some observables cause some observed effects but also a model of an internal, possibly unobservable causal mechanism that leads from one to the other. The mechanism may be complex and the causal relations may be partial or nondeterministic. However, once some observed effects are explained by some observable causes, the observations of the causes can be used to predict the effects. The explanations input effects and output causes, whereas the predictions go the other way around. The explanations can then be tested by validating the predictions. If the explanations are specified effectively, the induced predictions are computable. Belief transitions provide a high-level view of both.


\begin{definition}\label{Def:explainable}
An \emph{explainable belief transition}\/  $f\colon A\to B$ is a cartesian functor  $f\colon \Theta_{A}\ooot{\Gamma_{A}}{\Gamma_{B}} \Theta_{B}$ mapping $\Gamma_{B}$-(co)cones to $\Gamma_{A}$-(co)cones and moreover making the following diagram commutes
\beq\label{eq:def-explainable}
\begin{tikzar}{}
\Theta_{A}\ar{dr}[description]{\overline\MMM_{A}} \&\& \Theta_{B}\ar{dl}[description]{\overline\MMM_{B}} \ar[bend right]{ll}[swap]{f}\\
\& \Set
\end{tikzar}
\eeq
where $\overline\MMM_A$ is the extension of $\MMM_A$ along the completion $\Sigma_A\inclusion \Theta_A$ along all limits and colimits generated by $\Gamma_A$; and ditto for $\overline\MMM_B$.
\end{definition}

\paragraph{Comment.} In Def.~\ref{Def:interpretable}, theories were presented as pairs $\Theta = <\Sigma,\Gamma>$, where the category $\Sigma$ presents sorts, operations, and equations of the theory, whereas $\Gamma$ presents predicates and imposes the axioms of the theory. In Def.~\ref{Def:explainable}, a theory $\Theta$ is presented as the category obtained by completing $\Sigma$ under the limits and colimits derived from $\Gamma$. This category with the distinguished family of cones and cocones generated from $\Gamma$ is now denoted $\Theta$, by abuse of notation. A detailed construction of this category, forming a canonical sketch with its distinguished cones and cocones, can be found in \cite[\S4.2--3]{MakkaiM:acccfc}. This category represents the set of all theorems derivable in the signature $\Sigma$ from the axioms $\Gamma$. Since $\Sigma \inclusion \Theta$ is the $\Gamma$-completion of $\Sigma$, any functor $\MMM:\Sigma\to\Set$ mapping the $\Gamma$-(co)cones in $\Sigma$ to (co)cones in $\Set$ has a unique $\Gamma$-preserving extension $\overline \MMM\colon \Theta\to \Set$. These extensions are displayed in \eqref{eq:def-intepretable}. The upshot of saturating the theory presentations in the form $\Theta=<\Sigma,\Gamma>$ used in Def.~\ref{Def:interpretable} to the categories that they generated, denoted again $\Theta$, is that the general explainable transitions are now expressed in terms of functors between such saturated categories, as displayed in \eqref{eq:def-intepretable}. 

\paragraph{\em Explanations as programs.} A theory $\Theta$ is derived from the signature or clone $\Sigma$ inductively, by applying the axioms or (co)cones $\Gamma$. If $\Sigma$ and $\Gamma$ are effectively given, then $\Theta$ can be effectively computed. The fact that the functor $f\colon \Theta_A\ot \Theta_B$ underlying a belief transition $f\colon A\to B$ maps $\Gamma_{B}$-(co)limits to $\Gamma_{A}$-(co)limits means that it is also built inductively and thus effectively from its restriction $\hat f\colon \Theta_{A}\ot \Sigma_{B}$ along $\Sigma_{B}\inclusion \Theta_{B}$. However, since there may be \emph{many different ways}\/ to map $\Gamma_{B}$-(co)limits to $\Gamma_{A}$-(co)limits, the functor $f\colon \Theta_A\ot \Theta_B$ is \emph{\textbf{not} uniquely determined}\/ by its restriction $\hat f\colon \Theta_{A}\ot \Sigma_{B}$. The logical intuition is that there may be many different ways to prove the axioms $\Gamma_{B}$ as theorems in $\Theta_{A}$ and many different logical justifications of a belief transition from $A$ to $B$. In summary, a explainable transition $f\colon A\to B$ is effective as long as $A$ and $B$ are effectively given, and its predictions are computable. \emph{The explanations that actually an explainable transition can be formalized as programs that compute its predictions.} If the semantics of a programming language $\EEE$ used for computing the predictions is expressed in terms of a dynamic logic assignment $\TTT^o\times \EEE\times \TTT\to \OOO$, where $\TTT$ is a suitable posetal collapse of the universe of theories, then the sketch morphism or theory mapping $f\colon \Theta_A\ot \Theta_B$ corresponds to a Hoare triple $\Theta_A\uev F\Theta_B$, where $F$ is a program for the computation $f$. 

\subsection{Monoidal category of belief states and transitions}
The belief states from Def.~\ref{def:state} and the belief transitions from Def.~\ref{Def:explainable} clearly form a category, which we will call $\UUU$. The monoidal structure is induced by the disjoint unions of theories:
\bea
A\otimes B & = & \Big<\Sigma_{A}+\Sigma_{B}\, ,\,  \Gamma_{A}+\Gamma_{B}\, ,\,  [\MMM_{A}+\MMM_{B}]
\Big>
\eea
where $\MMM_{A\otimes B} = [\MMM_{A}+\MMM_{B}]\colon \Sigma_{A}+\Sigma_{B} \tto{\Gamma_{A\otimes B}}\Set$ maps $\Sigma_{A}$ like $\MMM_{A}$ and $\Sigma_{B}$ like $\MMM_{B}$. The tensor unit is $I = \left<\bot, \bot, \emptyset\right>$, where the truth value $\bot$ denotes the inconsistent theory or sketch, its only axiom, and $\emptyset$ is its empty model. It obviously satisfies $I\otimes A = A = A\otimes I$. The associativity of the tensor $\otimes$ follows from the associativity of the disjoint union $+$. The arrow part of $\otimes$ is induced by the disjoint unions as coproducts. The coproduct structure equips every belief state $A$ with a cartesian comonoid structure
\begin{gather}\label{eq:dataserv-text}
 \ \ A\otimes A\  \oot{\ \ \ \ \Delta\ \  \ \ } A \tto{\ \  \ \scun\ \ \ } I\\
\Sigma_{A}+\Sigma_{A}\tto{\ [\id,\id]\ } \Sigma_{A}\oot{\ \ \bot\ \ } \bot\notag
\end{gather}
The observables can thus be cloned and erased, and the observations are repeatable and deletable, as required in science. However, $\UUU$ is not a cartesian category, and $\otimes$ is not a cartesian product, because the explainable belief transitions $f:A\to B$ are not always interpretable, and do not boil down to the functors $\Sigma_{A}\ot \Sigma_{B}$. This captures the unclonable and undeletable states that arise in explanations of unobservable causations, e.g. in quantum physics and computation, or in economic and political theory and practice. Those that do form the cartesian subcategory $\tot\UUU\inclusion\UUU$, for which $\otimes$ is a cartesian product. The category $\tot\UUU$ consists of belief states from Def.~\ref{def:state} and the interpretable belief transitions from Def.~\ref{Def:interpretable}. If the elements $\alpha \in \UUU(I,A)$ are thought of as observables of type $A$, then the elements $a\in \tot \UUU(I,A)$ are actual observations, disproving some statements in $A$.


 


\subsection{String diagrams}
Commutative diagrams like \eqref{eq:def-intepretable} display composite morphisms and abbreviate their equations. String diagrams display the \emph{de}\/compositions of morphisms in monoidal categories. The two composition operations presented by monoidal categories are displayed along the two dimensions: the categorical composition is constrained by the strings and goes bottom-up, whereas the monoidal composition goes left-right, unconstrained by the strings. A morphism $A\tto f B$ is presented as a box $f$ with a string $A$ coming in at the bottom and a string $B$ coming out at the top. The identities are presented as invisible boxes and the identity on $A$ is just the string $A$. The unit type $I$ is presented as an invisible string, so there are boxes with no strings attached. The categorical composition $g\circ f =(A\tto f B\tto g C)$ is drawn by hanging the box $f$ on the string $B$ under the box $g$. The monoidal composition $(g\circ f)\otimes (s\circ t)$ is drawn by placing the boxes $g\circ f$ next to the boxes for $s\circ t$:
\beq\label{eq:godement}
\newcommand{\machine}{$f$}
\newcommand{\gee}{$g$}
\newcommand{\kee}{$s$}
\newcommand{\hee}{t}
\newcommand{\nameslang}{\scriptstyle B}
\newcommand{\seqcompp}{\scriptstyle {\color{red}g\circ f}}
\newcommand{\parcompp}{\scriptstyle {\color{blue}f\otimes t}}
\newcommand{\inputt}{\scriptstyle A} 
\newcommand{\outpt}{$\scriptstyle C$}
\newcommand{\otherinputt}{\scriptstyle U}
\newcommand{\otheroutpt}{\scriptstyle V} 
\newcommand{\outpttt}{$\scriptstyle W$}
\def\JPicScale{.3}
%%Created by jPicEdt 1.4.1_03: mixed JPIC-XML/LaTeX format
%%Fri Mar 17 19:14:21 GMT-10:00 2023
%%Begin JPIC-XML
%<?xml version="1.0" standalone="yes"?>
%<jpic x-min="-5" x-max="115" y-min="-5" y-max="115" auto-bounding="true">
%<multicurve fill-style= "none"
%	 points= "(15,40);(15,40);(45,40);(45,40)"
%	 stroke-width= "1"
%	 />
%<multicurve fill-style= "none"
%	 points= "(15,40);(15,40);(15,20);(15,20)"
%	 stroke-width= "1"
%	 />
%<multicurve fill-style= "none"
%	 points= "(45,20);(45,20);(45,40);(45,40)"
%	 stroke-width= "1"
%	 />
%<multicurve fill-style= "none"
%	 arrow-head-inset-scale= "0"
%	 points= "(30,70);(30,70);(30,40);(30,40)"
%	 arrow-head-width-minimum= "1.5"
%	 arrow-head-length-scale= "1.5"
%	 stroke-width= "1"
%	 left-arrow= "head"
%	 />
%<text text-vert-align= "center-v"
%	 fill-style= "none"
%	 anchor-point= "(27.5,55)"
%	 text-frame= "noframe"
%	 stroke-width= "1"
%	 text-hor-align= "right"
%	 >
%$\nameslang$
%</text>
%<text text-vert-align= "center-v"
%	 fill-style= "none"
%	 anchor-point= "(28.75,-3.75)"
%	 text-frame= "noframe"
%	 stroke-width= "1"
%	 text-hor-align= "right"
%	 >
%$\inputt$
%</text>
%<multicurve fill-style= "none"
%	 points= "(15,20);(15,20);(45,20);(45,20)"
%	 stroke-width= "1"
%	 />
%<multicurve fill-style= "none"
%	 arrow-head-inset-scale= "0"
%	 points= "(30,20);(30,20);(30,-5);(30,-5)"
%	 arrow-head-width-minimum= "1.5"
%	 arrow-head-length-scale= "1.5"
%	 stroke-width= "1"
%	 left-arrow= "head"
%	 />
%<text text-vert-align= "center-v"
%	 fill-style= "none"
%	 anchor-point= "(30,30)"
%	 text-frame= "noframe"
%	 stroke-width= "1"
%	 text-hor-align= "center-h"
%	 >
%\machine
%</text>
%<multicurve fill-style= "none"
%	 points= "(15,90);(15,90);(45,90);(45,90)"
%	 stroke-width= "1"
%	 />
%<multicurve fill-style= "none"
%	 points= "(15,90);(15,90);(15,70);(15,70)"
%	 stroke-width= "1"
%	 />
%<multicurve fill-style= "none"
%	 points= "(45,70);(45,70);(45,90);(45,90)"
%	 stroke-width= "1"
%	 />
%<multicurve fill-style= "none"
%	 arrow-head-inset-scale= "0"
%	 points= "(30,115);(30,115);(30,90);(30,90)"
%	 arrow-head-width-minimum= "1.5"
%	 arrow-head-length-scale= "1.5"
%	 stroke-width= "1"
%	 left-arrow= "head"
%	 />
%<multicurve fill-style= "none"
%	 points= "(15,70);(15,70);(45,70);(45,70)"
%	 stroke-width= "1"
%	 />
%<text text-vert-align= "center-v"
%	 fill-style= "none"
%	 anchor-point= "(30,80)"
%	 text-frame= "noframe"
%	 stroke-width= "1"
%	 text-hor-align= "center-h"
%	 >
%\gee
%</text>
%<text text-vert-align= "center-v"
%	 fill-style= "none"
%	 anchor-point= "(32.5,113.75)"
%	 text-frame= "noframe"
%	 stroke-width= "1"
%	 text-hor-align= "left"
%	 >
%\outpt
%</text>
%<multicurve fill-style= "none"
%	 points= "(65,40);(65,40);(95,40);(95,40)"
%	 stroke-width= "1"
%	 />
%<multicurve fill-style= "none"
%	 points= "(65,40);(65,40);(65,20);(65,20)"
%	 stroke-width= "1"
%	 />
%<multicurve fill-style= "none"
%	 points= "(95,20);(95,20);(95,40);(95,40)"
%	 stroke-width= "1"
%	 />
%<multicurve fill-style= "none"
%	 arrow-head-inset-scale= "0"
%	 points= "(80,70);(80,70);(80,40);(80,40)"
%	 arrow-head-width-minimum= "1.5"
%	 arrow-head-length-scale= "1.5"
%	 stroke-width= "1"
%	 left-arrow= "head"
%	 />
%<multicurve fill-style= "none"
%	 points= "(65,20);(65,20);(95,20);(95,20)"
%	 stroke-width= "1"
%	 />
%<multicurve fill-style= "none"
%	 arrow-head-inset-scale= "0"
%	 points= "(80,20);(80,20);(80,-5);(80,-5)"
%	 arrow-head-width-minimum= "1.5"
%	 arrow-head-length-scale= "1.5"
%	 stroke-width= "1"
%	 left-arrow= "head"
%	 />
%<text text-vert-align= "center-v"
%	 fill-style= "none"
%	 anchor-point= "(77.5,-3.75)"
%	 text-frame= "noframe"
%	 stroke-width= "1"
%	 text-hor-align= "right"
%	 >
%$\otherinputt$
%</text>
%<text text-vert-align= "center-v"
%	 fill-style= "none"
%	 anchor-point= "(77.5,55)"
%	 text-frame= "noframe"
%	 stroke-width= "1"
%	 text-hor-align= "right"
%	 >
%$\otheroutpt$
%</text>
%<text text-vert-align= "center-v"
%	 fill-style= "none"
%	 anchor-point= "(80,30)"
%	 text-frame= "noframe"
%	 stroke-width= "1"
%	 text-hor-align= "center-h"
%	 >
%$\hee$
%</text>
%<parallelogram p3= "(50,5)"
%	 fill-style= "none"
%	 p2= "(50,105)"
%	 p1= "(10,105)"
%	 stroke-color= "#ff0066"
%	 stroke-width= "0.45"
%	 />
%<parallelogram p3= "(115,15)"
%	 fill-style= "none"
%	 p2= "(115,45)"
%	 p1= "(-5,45)"
%	 stroke-color= "#3300ff"
%	 stroke-width= "0.45"
%	 />
%<text text-vert-align= "bottom"
%	 fill-style= "none"
%	 anchor-point= "(113.75,16.25)"
%	 text-frame= "noframe"
%	 stroke-width= "1"
%	 text-hor-align= "right"
%	 >
%$\parcompp$
%</text>
%<text text-vert-align= "top"
%	 fill-style= "none"
%	 anchor-point= "(48.75,103.12)"
%	 text-frame= "noframe"
%	 stroke-width= "1"
%	 text-hor-align= "right"
%	 >
%$\seqcompp$
%</text>
%<multicurve fill-style= "none"
%	 points= "(65,90);(65,90);(65,70);(65,70)"
%	 stroke-width= "1"
%	 />
%<multicurve fill-style= "none"
%	 points= "(95,70);(95,70);(95,90);(95,90)"
%	 stroke-width= "1"
%	 />
%<multicurve fill-style= "none"
%	 arrow-head-inset-scale= "0"
%	 points= "(80,115);(80,115);(80,90);(80,90)"
%	 arrow-head-width-minimum= "1.5"
%	 arrow-head-length-scale= "1.5"
%	 stroke-width= "1"
%	 left-arrow= "head"
%	 />
%<multicurve fill-style= "none"
%	 points= "(65,70);(65,70);(95,70);(95,70)"
%	 stroke-width= "1"
%	 />
%<multicurve fill-style= "none"
%	 points= "(65,90);(65,90);(95,90);(95,90)"
%	 stroke-width= "1"
%	 />
%<text text-vert-align= "center-v"
%	 fill-style= "none"
%	 anchor-point= "(80,80)"
%	 text-frame= "noframe"
%	 stroke-width= "1"
%	 text-hor-align= "center-h"
%	 >
%\kee
%</text>
%<text text-vert-align= "center-v"
%	 fill-style= "none"
%	 anchor-point= "(82.5,113.75)"
%	 text-frame= "noframe"
%	 stroke-width= "1"
%	 text-hor-align= "left"
%	 >
%\outpttt
%</text>
%</jpic>
%%End JPIC-XML
%PSTricks content-type (pstricks.sty package needed)
%Add \usepackage{pstricks} in the preamble of your LaTeX file
%You can rescale the whole picture (to 80% for instance) by using the command \def\JPicScale{0.8}
\ifx\JPicScale\undefined\def\JPicScale{1}\fi
\psset{unit=\JPicScale mm}
\psset{linewidth=0.3,dotsep=1,hatchwidth=0.3,hatchsep=1.5,shadowsize=1,dimen=middle}
\psset{dotsize=0.7 2.5,dotscale=1 1,fillcolor=black}
\psset{arrowsize=1 2,arrowlength=1,arrowinset=0.25,tbarsize=0.7 5,bracketlength=0.15,rbracketlength=0.15}
\begin{pspicture}(0,0)(115,115)
\psline[linewidth=1](15,40)(45,40)
\psline[linewidth=1](15,40)(15,20)
\psline[linewidth=1](45,20)(45,40)
\psline[linewidth=1,arrowsize=1.5 2,arrowlength=1.5,arrowinset=0]{<-}(30,70)(30,40)
\rput[r](27.5,55){$\nameslang$}
\rput[r](28.75,-3.75){$\inputt$}
\psline[linewidth=1](15,20)(45,20)
\psline[linewidth=1,arrowsize=1.5 2,arrowlength=1.5,arrowinset=0]{<-}(30,20)(30,-5)
\rput(30,30){\machine}
\psline[linewidth=1](15,90)(45,90)
\psline[linewidth=1](15,90)(15,70)
\psline[linewidth=1](45,70)(45,90)
\psline[linewidth=1,arrowsize=1.5 2,arrowlength=1.5,arrowinset=0]{<-}(30,115)(30,90)
\psline[linewidth=1](15,70)(45,70)
\rput(30,80){\gee}
\rput[l](32.5,113.75){\outpt}
\psline[linewidth=1](65,40)(95,40)
\psline[linewidth=1](65,40)(65,20)
\psline[linewidth=1](95,20)(95,40)
\psline[linewidth=1,arrowsize=1.5 2,arrowlength=1.5,arrowinset=0]{<-}(80,70)(80,40)
\psline[linewidth=1](65,20)(95,20)
\psline[linewidth=1,arrowsize=1.5 2,arrowlength=1.5,arrowinset=0]{<-}(80,20)(80,-5)
\rput[r](77.5,-3.75){$\otherinputt$}
\rput[r](77.5,55){$\otheroutpt$}
\rput(80,30){$\hee$}
\newrgbcolor{userLineColour}{1 0 0.4}
\pspolygon[linewidth=0.45,linecolor=userLineColour](10,105)(50,105)(50,5)(10,5)
\newrgbcolor{userLineColour}{0.2 0 1}
\pspolygon[linewidth=0.45,linecolor=userLineColour](-5,45)(115,45)(115,15)(-5,15)
\rput[br](113.75,16.25){$\parcompp$}
\rput[tr](48.75,103.12){$\seqcompp$}
\psline[linewidth=1](65,90)(65,70)
\psline[linewidth=1](95,70)(95,90)
\psline[linewidth=1,arrowsize=1.5 2,arrowlength=1.5,arrowinset=0]{<-}(80,115)(80,90)
\psline[linewidth=1](65,70)(95,70)
\psline[linewidth=1](65,90)(95,90)
\rput(80,80){\kee}
\rput[l](82.5,113.75){\outpttt}
\end{pspicture}

\eeq


The middle-two-interchange law $(g\circ f)\otimes(s\circ t) =  (g\otimes s)\circ(f\otimes t)$ corresponds to the two ways of reading the diagram: vertical-first and horizontal-first, marked by the red and the blue rectangle respectively. The cartesian comonoids \eqref{eq:dataserv-text} are drawn
\beq\label{eq:dataserv}
\begin{split}
%\begin{figure}[!ht]
%\begin{center}
\newcommand{\AAh}{\scriptstyle A}
\newcommand{\ccopy}{\cmn}
\newcommand{\delete}{\scun}
\def\JPicScale{.26} 
\input{PICS/dataserv-1.tex}
\end{split}
\eeq
and the equations that make them into commutative comonoids look like this:
\begin{alignat}{11}
%\comp {\cmn}{(\cmn \ttimes \id)}  &\ \ =\ \   \comp{\cmn}{(\id \ttimes \cmn)} &\quad&\quad&\quad&\quad&
%\comp{\cmn}{(\cun \ttimes\, \id)}  &\ \  =\ \  &\   \id  &\ \   =\ \  &\ \  \comp{\cmn}{(\id \, \ttimes \cun)}\notag
%\\[1ex]
\def\JPicScale{.65} %%Created by jPicEdt 1.4.1_03: mixed JPIC-XML/LaTeX format
%%Sat May 15 12:50:08 GMT-10:00 2021
%%Begin JPIC-XML
%<?xml version="1.0" standalone="yes"?>
%<jpic x-min="-0.63" x-max="15" y-min="-10.62" y-max="7.5" auto-bounding="true">
%<multicurve fill-style= "none"
%	 stroke-width= "0.55"
%	 points= "(5.62,-0.62);(5.62,-0.62);(10.62,-5.62);(10.62,-5.62);(10.62,-5.62);(9.38,-5.62);
%	(9.38,-5.62);(9.38,-5.62);(10,-5.62);(10,-5.62)"
%	 />
%<ellipse p3= "(2.81,2.19)"
%	 p2= "(2.81,-0.94)"
%	 fill-style= "solid"
%	 p1= "(5.94,-0.94)"
%	 stroke-width= "0.55"
%	 closure= "open"
%	 angle-end= "0"
%	 angle-start= "0"
%	 />
%<ellipse p3= "(8.75,-3.75)"
%	 p2= "(8.75,-6.88)"
%	 fill-style= "solid"
%	 p1= "(11.88,-6.88)"
%	 stroke-width= "0.55"
%	 closure= "open"
%	 angle-end= "0"
%	 angle-start= "0"
%	 />
%<multicurve fill-style= "none"
%	 stroke-width= "0.55"
%	 points= "(15,7.5);(15,7.5);(15,5);(15,5);(15,5);(15,-0.63);
%	(15,-0.63);(15,-0.63);(11.25,-4.38);(11.25,-4.38)"
%	 />
%<multicurve fill-style= "none"
%	 stroke-width= "0.55"
%	 points= "(9.37,7.5);(9.37,7.5);(9.37,6.87);(9.37,6.87);(9.37,6.87);(9.38,5);
%	(9.38,5);(9.38,5);(5.63,1.25);(5.63,1.25)"
%	 />
%<multicurve fill-style= "none"
%	 stroke-width= "0.55"
%	 points= "(-0.63,7.5);(-0.63,7.5);(-0.63,6.87);(-0.63,6.87);(-0.63,6.87);(-0.62,5);
%	(-0.62,5);(-0.62,5);(3.13,1.25);(3.13,1.25)"
%	 />
%<multicurve fill-style= "none"
%	 stroke-width= "0.55"
%	 points= "(10,-6.88);(10,-6.88);(10,-10);(10,-10);(10,-10);(10,-10.62);
%	(10,-10.62);(10,-10.62);(10,-10);(10,-10)"
%	 />
%</jpic>
%%End JPIC-XML
%PSTricks content-type (pstricks.sty package needed)
%Add \usepackage{pstricks} in the preamble of your LaTeX file
%You can rescale the whole picture (to 80% for instance) by using the command \def\JPicScale{0.8}
\ifx\JPicScale\undefined\def\JPicScale{1}\fi
\psset{unit=\JPicScale mm}
\psset{linewidth=0.3,dotsep=1,hatchwidth=0.3,hatchsep=1.5,shadowsize=1,dimen=middle}
\psset{dotsize=0.7 2.5,dotscale=1 1,fillcolor=black}
\psset{arrowsize=1 2,arrowlength=1,arrowinset=0.25,tbarsize=0.7 5,bracketlength=0.15,rbracketlength=0.15}
\begin{pspicture}(0,0)(15,7.5)
\psline[linewidth=0.55](5.62,-0.62)
(10.62,-5.62)
(9.38,-5.62)(10,-5.62)
\rput{0}(4.38,0.62){\psellipse[linewidth=0.55,fillstyle=solid](0,0)(1.56,-1.57)}
\rput{0}(10.32,-5.31){\psellipse[linewidth=0.55,fillstyle=solid](0,0)(1.56,-1.57)}
\psline[linewidth=0.55](15,7.5)
(15,5)
(15,-0.63)(11.25,-4.38)
\psline[linewidth=0.55](9.37,7.5)
(9.37,6.87)
(9.38,5)(5.63,1.25)
\psline[linewidth=0.55](-0.63,7.5)
(-0.63,6.87)
(-0.62,5)(3.13,1.25)
\psline[linewidth=0.55](10,-6.88)
(10,-10)
(10,-10.62)(10,-10)
\end{pspicture}
\   &\ \ =\ \   \def\JPicScale{.65} %%Created by jPicEdt 1.4.1_03: mixed JPIC-XML/LaTeX format
%%Sat May 15 12:54:01 GMT-10:00 2021
%%Begin JPIC-XML
%<?xml version="1.0" standalone="yes"?>
%<jpic x-min="-0.63" x-max="15.63" y-min="-10.62" y-max="8.12" auto-bounding="true">
%<ellipse p3= "(9.06,2.19)"
%	 p2= "(9.06,-0.94)"
%	 fill-style= "solid"
%	 p1= "(12.19,-0.94)"
%	 stroke-width= "0.55"
%	 closure= "open"
%	 angle-end= "0"
%	 angle-start= "0"
%	 />
%<multicurve fill-style= "none"
%	 stroke-width= "0.55"
%	 points= "(15.62,8.12);(15.62,8.12);(15.62,7.5);(15.62,7.5);(15.62,7.5);(15.63,5.62);
%	(15.63,5.62);(15.63,5.62);(11.88,1.87);(11.88,1.87)"
%	 />
%<multicurve fill-style= "none"
%	 stroke-width= "0.55"
%	 points= "(5.62,8.12);(5.62,8.12);(5.62,7.5);(5.62,7.5);(5.62,7.5);(5.63,5.62);
%	(5.63,5.62);(5.63,5.62);(9.38,1.87);(9.38,1.87)"
%	 />
%<ellipse p3= "(2.81,-4.06)"
%	 p2= "(2.81,-7.19)"
%	 fill-style= "solid"
%	 p1= "(5.94,-7.19)"
%	 stroke-width= "0.55"
%	 closure= "open"
%	 angle-end= "0"
%	 angle-start= "0"
%	 />
%<multicurve fill-style= "none"
%	 stroke-width= "0.55"
%	 points= "(-0.63,7.5);(-0.63,7.5);(-0.62,1.25);(-0.62,1.25);(-0.62,1.25);(-0.62,-0.63);
%	(-0.62,-0.63);(-0.62,-0.63);(3.13,-4.38);(3.13,-4.38)"
%	 />
%<multicurve fill-style= "none"
%	 stroke-width= "0.55"
%	 points= "(9.38,-0.62);(9.38,-0.62);(4.38,-5.62);(4.38,-5.62);(4.38,-5.62);(4.38,-5.62);
%	(4.38,-5.62);(4.38,-5.62);(4.38,-5.62);(4.38,-5.62)"
%	 />
%<multicurve fill-style= "none"
%	 stroke-width= "0.55"
%	 points= "(4.38,-6.88);(4.38,-6.88);(4.38,-10);(4.38,-10);(4.38,-10);(4.38,-10.62);
%	(4.38,-10.62);(4.38,-10.62);(4.38,-10);(4.38,-10)"
%	 />
%</jpic>
%%End JPIC-XML
%PSTricks content-type (pstricks.sty package needed)
%Add \usepackage{pstricks} in the preamble of your LaTeX file
%You can rescale the whole picture (to 80% for instance) by using the command \def\JPicScale{0.8}
\ifx\JPicScale\undefined\def\JPicScale{1}\fi
\psset{unit=\JPicScale mm}
\psset{linewidth=0.3,dotsep=1,hatchwidth=0.3,hatchsep=1.5,shadowsize=1,dimen=middle}
\psset{dotsize=0.7 2.5,dotscale=1 1,fillcolor=black}
\psset{arrowsize=1 2,arrowlength=1,arrowinset=0.25,tbarsize=0.7 5,bracketlength=0.15,rbracketlength=0.15}
\begin{pspicture}(0,0)(15.63,8.12)
\rput{0}(10.62,0.62){\psellipse[linewidth=0.55,fillstyle=solid](0,0)(1.57,-1.56)}
\psline[linewidth=0.55](15.62,8.12)
(15.62,7.5)
(15.63,5.62)(11.88,1.87)
\psline[linewidth=0.55](5.62,8.12)
(5.62,7.5)
(5.63,5.62)(9.38,1.87)
\rput{0}(4.38,-5.62){\psellipse[linewidth=0.55,fillstyle=solid](0,0)(1.57,-1.57)}
\psline[linewidth=0.55](-0.63,7.5)
(-0.62,1.25)
(-0.62,-0.63)(3.13,-4.38)
\pscustom[linewidth=0.55]{\psline(9.38,-0.62)(4.38,-5.62)
\psbezier(4.38,-5.62)(4.38,-5.62)(4.38,-5.62)
\psbezier(4.38,-5.62)(4.38,-5.62)(4.38,-5.62)
}
\psline[linewidth=0.55](4.38,-6.88)
(4.38,-10)
(4.38,-10.62)(4.38,-10)
\end{pspicture}
 &&\qquad\qquad&&& 
\def\JPicScale{.65} 
\input{PICS/cun-left-1.tex}  & \ = \ &\ \, \def\JPicScale{.65} %%Created by jPicEdt 1.4.1_03: mixed JPIC-XML/LaTeX format
%%Sat May 15 12:55:13 GMT-10:00 2021
%%Begin JPIC-XML
%<?xml version="1.0" standalone="yes"?>
%<jpic x-min="0" x-max="0" y-min="-10" y-max="8.74" auto-bounding="true">
%<multicurve fill-style= "none"
%	 stroke-width= "0.55"
%	 points= "(0,8.74);(0,8.74);(0,-9.38);(0,-9.38);(0,-9.38);(0,-10);
%	(0,-10);(0,-10);(0,-9.38);(0,-9.38)"
%	 />
%</jpic>
%%End JPIC-XML
%PSTricks content-type (pstricks.sty package needed)
%Add \usepackage{pstricks} in the preamble of your LaTeX file
%You can rescale the whole picture (to 80% for instance) by using the command \def\JPicScale{0.8}
\ifx\JPicScale\undefined\def\JPicScale{1}\fi
\psset{unit=\JPicScale mm}
\psset{linewidth=0.3,dotsep=1,hatchwidth=0.3,hatchsep=1.5,shadowsize=1,dimen=middle}
\psset{dotsize=0.7 2.5,dotscale=1 1,fillcolor=black}
\psset{arrowsize=1 2,arrowlength=1,arrowinset=0.25,tbarsize=0.7 5,bracketlength=0.15,rbracketlength=0.15}
\begin{pspicture}(0,0)(0,8.74)
\psline[linewidth=0.55](0,8.74)
(0,-9.38)
(0,-10)(0,-9.38)
\end{pspicture}
 &\ \, = \ \ &  \def\JPicScale{.65} %%Created by jPicEdt 1.4.1_03: mixed JPIC-XML/LaTeX format
%%Sat May 15 12:56:33 GMT-10:00 2021
%%Begin JPIC-XML
%<?xml version="1.0" standalone="yes"?>
%<jpic x-min="-0.63" x-max="12.19" y-min="-10.62" y-max="8.12" auto-bounding="true">
%<ellipse p3= "(9.06,2.81)"
%	 p2= "(9.06,-0.31)"
%	 fill-style= "solid"
%	 p1= "(12.19,-0.31)"
%	 stroke-width= "0.55"
%	 closure= "open"
%	 angle-end= "0"
%	 angle-start= "0"
%	 />
%<ellipse p3= "(2.81,-3.44)"
%	 p2= "(2.81,-6.56)"
%	 fill-style= "solid"
%	 p1= "(5.94,-6.56)"
%	 stroke-width= "0.55"
%	 closure= "open"
%	 angle-end= "0"
%	 angle-start= "0"
%	 />
%<multicurve fill-style= "none"
%	 stroke-width= "0.55"
%	 points= "(4.38,-6.25);(4.38,-6.25);(4.38,-10);(4.38,-10);(4.38,-10);(4.38,-10.62);
%	(4.38,-10.62);(4.38,-10.62);(4.38,-10);(4.38,-10)"
%	 />
%<multicurve fill-style= "none"
%	 stroke-width= "0.55"
%	 points= "(-0.63,8.12);(-0.63,8.12);(-0.62,1.87);(-0.62,1.87);(-0.62,1.87);(-0.62,0);
%	(-0.62,0);(-0.62,0);(3.13,-3.75);(3.13,-3.75)"
%	 />
%<multicurve fill-style= "none"
%	 stroke-width= "0.55"
%	 points= "(9.38,0);(9.38,0);(4.38,-5);(4.38,-5);(4.38,-5);(4.38,-5);
%	(4.38,-5);(4.38,-5);(4.38,-5);(4.38,-5)"
%	 />
%</jpic>
%%End JPIC-XML
%PSTricks content-type (pstricks.sty package needed)
%Add \usepackage{pstricks} in the preamble of your LaTeX file
%You can rescale the whole picture (to 80% for instance) by using the command \def\JPicScale{0.8}
\ifx\JPicScale\undefined\def\JPicScale{1}\fi
\psset{unit=\JPicScale mm}
\psset{linewidth=0.3,dotsep=1,hatchwidth=0.3,hatchsep=1.5,shadowsize=1,dimen=middle}
\psset{dotsize=0.7 2.5,dotscale=1 1,fillcolor=black}
\psset{arrowsize=1 2,arrowlength=1,arrowinset=0.25,tbarsize=0.7 5,bracketlength=0.15,rbracketlength=0.15}
\begin{pspicture}(0,0)(12.19,8.12)
\rput{0}(10.62,1.25){\psellipse[linewidth=0.55,fillstyle=solid](0,0)(1.56,-1.56)}
\rput{0}(4.38,-5){\psellipse[linewidth=0.55,fillstyle=solid](0,0)(1.57,-1.56)}
\psline[linewidth=0.55](4.38,-6.25)
(4.38,-10)
(4.38,-10.62)(4.38,-10)
\psline[linewidth=0.55](-0.63,8.12)
(-0.62,1.87)
(-0.62,0)(3.13,-3.75)
\pscustom[linewidth=0.55]{\psline(9.38,0)(4.38,-5)
\psbezier(4.38,-5)(4.38,-5)(4.38,-5)
\psbezier(4.38,-5)(4.38,-5)(4.38,-5)
}
\end{pspicture}
&&\qquad\qquad&&&
\def\JPicScale{.65} \input{PICS/cmn.tex} &\   = \   \def\JPicScale{.65} %%Created by jPicEdt 1.4.1_03: mixed JPIC-XML/LaTeX format
%%Thu Mar 16 14:52:39 GMT-10:00 2023
%%Begin JPIC-XML
%<?xml version="1.0" standalone="yes"?>
%<jpic x-min="0" x-max="5" y-min="-10" y-max="7.5" auto-bounding="true">
%<ellipse fill-style= "solid"
%	 stroke-width= "0.5"
%	 p3= "(0.93,-3.44)"
%	 p2= "(0.93,-6.56)"
%	 p1= "(4.06,-6.56)"
%	 closure= "open"
%	 angle-end= "0"
%	 angle-start= "0"
%	 />
%<multicurve fill-style= "none"
%	 stroke-width= "0.5"
%	 points= "(2.5,-5);(2.5,-5);(2.5,-10);(2.5,-10)"
%	 />
%<multicurve fill-style= "none"
%	 stroke-width= "0.5"
%	 points= "(5,-2.5);(5,-2.5);(2.5,-5);(2.5,-5)"
%	 />
%<multicurve fill-style= "none"
%	 stroke-width= "0.5"
%	 points= "(0,-2.5);(0,-2.5);(2.5,-5);(2.5,-5)"
%	 />
%<multicurve fill-style= "none"
%	 stroke-width= "0.5"
%	 points= "(5,0);(5,0);(5,-2.5);(5,-2.5)"
%	 />
%<multicurve fill-style= "none"
%	 stroke-width= "0.5"
%	 points= "(0,0);(0,0);(0,-2.5);(0,-2.5)"
%	 />
%<multicurve fill-style= "none"
%	 stroke-width= "0.5"
%	 points= "(0,5);(0,5);(5,0);(5,0)"
%	 />
%<multicurve fill-style= "none"
%	 stroke-width= "0.5"
%	 points= "(0,0);(0,0);(5,5);(5,5)"
%	 />
%<multicurve fill-style= "none"
%	 stroke-width= "0.5"
%	 points= "(5,7.5);(5,7.5);(5,5);(5,5)"
%	 />
%<multicurve fill-style= "none"
%	 stroke-width= "0.5"
%	 points= "(0,7.5);(0,7.5);(0,5);(0,5)"
%	 />
%</jpic>
%%End JPIC-XML
%PSTricks content-type (pstricks.sty package needed)
%Add \usepackage{pstricks} in the preamble of your LaTeX file
%You can rescale the whole picture (to 80% for instance) by using the command \def\JPicScale{0.8}
\ifx\JPicScale\undefined\def\JPicScale{1}\fi
\psset{unit=\JPicScale mm}
\psset{linewidth=0.3,dotsep=1,hatchwidth=0.3,hatchsep=1.5,shadowsize=1,dimen=middle}
\psset{dotsize=0.7 2.5,dotscale=1 1,fillcolor=black}
\psset{arrowsize=1 2,arrowlength=1,arrowinset=0.25,tbarsize=0.7 5,bracketlength=0.15,rbracketlength=0.15}
\begin{pspicture}(0,0)(5,7.5)
\rput{0}(2.49,-5){\psellipse[linewidth=0.5,fillstyle=solid](0,0)(1.56,-1.56)}
\psline[linewidth=0.5](2.5,-5)(2.5,-10)
\psline[linewidth=0.5](5,-2.5)(2.5,-5)
\psline[linewidth=0.5](0,-2.5)(2.5,-5)
\psline[linewidth=0.5](5,0)(5,-2.5)
\psline[linewidth=0.5](0,0)(0,-2.5)
\psline[linewidth=0.5](0,5)(5,0)
\psline[linewidth=0.5](0,0)(5,5)
\psline[linewidth=0.5](5,7.5)(5,5)
\psline[linewidth=0.5](0,7.5)(0,5)
\end{pspicture}
 \hspace{2em}
\label{eq:comonoid}%\\%[1ex]
\notag
\end{alignat}
\smallskip

\paragraph{Paremetrized and updating transitions.} As usual, a product $A\otimes B$ denotes a situation where $A$ and $B$ occur together but remain separated and do not influence each other. In a diagram, they are just parallel strings. A transition $g\colon X\otimes A\to B$ depends on $X$ and $A$ separately whereas $q\colon X\otimes A\to X\otimes B$ moreover updates $X$ and outputs $B$ separately.
\beq\label{eq:gq}
\begin{split}
\newcommand{\fee}{g}
\newcommand{\qee}{q}
\newcommand{\Aee}{\scriptstyle X}
\newcommand{\Bee}{\scriptstyle A}
\newcommand{\Cee}{\scriptstyle B}
\def\JPicScale{.15}
%%Created by jPicEdt 1.4.1_03: mixed JPIC-XML/LaTeX format
%%Fri Mar 17 19:08:11 GMT-10:00 2023
%%Begin JPIC-XML
%<?xml version="1.0" standalone="yes"?>
%<jpic x-min="0" x-max="300" y-min="0" y-max="120" auto-bounding="true">
%<multicurve fill-style= "none"
%	 points= "(80,80);(80,80);(80,40);(80,40)"
%	 />
%<multicurve fill-style= "none"
%	 points= "(80,40);(80,40);(0,40);(0,40)"
%	 />
%<multicurve fill-style= "none"
%	 points= "(0,80);(0,80);(80,80);(80,80)"
%	 />
%<multicurve fill-style= "none"
%	 points= "(60,40);(60,40);(60,0);(60,0)"
%	 />
%<multicurve fill-style= "none"
%	 points= "(40,120);(40,120);(40,80);(40,80)"
%	 />
%<multicurve fill-style= "none"
%	 points= "(0,80);(0,80);(0,40);(0,40)"
%	 />
%<multicurve fill-style= "none"
%	 points= "(20,40);(20,40);(20,0);(20,0)"
%	 />
%<text text-vert-align= "center-v"
%	 fill-style= "none"
%	 anchor-point= "(45,117.5)"
%	 text-frame= "noframe"
%	 text-hor-align= "left"
%	 >
%$\Cee$
%</text>
%<text text-vert-align= "center-v"
%	 fill-style= "none"
%	 anchor-point= "(15,2.5)"
%	 text-frame= "noframe"
%	 text-hor-align= "right"
%	 >
%$\Aee$
%</text>
%<text text-vert-align= "center-v"
%	 fill-style= "none"
%	 anchor-point= "(55,2.5)"
%	 text-frame= "noframe"
%	 text-hor-align= "right"
%	 >
%$\Bee$
%</text>
%<text text-vert-align= "center-v"
%	 fill-style= "none"
%	 anchor-point= "(40,60)"
%	 text-frame= "noframe"
%	 text-hor-align= "center-h"
%	 >
%$\fee$
%</text>
%<multicurve fill-style= "none"
%	 points= "(300,80);(300,80);(300,40);(300,40)"
%	 />
%<multicurve fill-style= "none"
%	 points= "(300,40);(300,40);(220,40);(220,40)"
%	 />
%<multicurve fill-style= "none"
%	 points= "(220,80);(220,80);(300,80);(300,80)"
%	 />
%<multicurve fill-style= "none"
%	 points= "(280,40);(280,40);(280,0);(280,0)"
%	 />
%<multicurve fill-style= "none"
%	 points= "(280,120);(280,120);(280,80);(280,80)"
%	 />
%<multicurve fill-style= "none"
%	 points= "(220,80);(220,80);(220,40);(220,40)"
%	 />
%<multicurve fill-style= "none"
%	 points= "(240,40);(240,40);(240,0);(240,0)"
%	 />
%<text text-vert-align= "center-v"
%	 fill-style= "none"
%	 anchor-point= "(285,117.5)"
%	 text-frame= "noframe"
%	 text-hor-align= "left"
%	 >
%$\Cee$
%</text>
%<text text-vert-align= "center-v"
%	 fill-style= "none"
%	 anchor-point= "(235,2.5)"
%	 text-frame= "noframe"
%	 text-hor-align= "right"
%	 >
%$\Aee$
%</text>
%<text text-vert-align= "center-v"
%	 fill-style= "none"
%	 anchor-point= "(275,2.5)"
%	 text-frame= "noframe"
%	 text-hor-align= "right"
%	 >
%$\Bee$
%</text>
%<multicurve fill-style= "none"
%	 points= "(240,120);(240,120);(240,80);(240,80)"
%	 />
%<text text-vert-align= "center-v"
%	 fill-style= "none"
%	 anchor-point= "(245,117.5)"
%	 text-frame= "noframe"
%	 text-hor-align= "left"
%	 >
%$\Aee$
%</text>
%<text text-vert-align= "center-v"
%	 fill-style= "none"
%	 anchor-point= "(260,60)"
%	 text-frame= "noframe"
%	 text-hor-align= "center-h"
%	 >
%$\qee$
%</text>
%</jpic>
%%End JPIC-XML
%LaTeX-picture environment using emulated lines and arcs
%You can rescale the whole picture (to 80% for instance) by using the command \def\JPicScale{0.8}
\ifx\JPicScale\undefined\def\JPicScale{1}\fi
\unitlength \JPicScale mm
\begin{picture}(300,120)(0,0)
\linethickness{0.3mm}
\put(80,40){\line(0,1){40}}
\linethickness{0.3mm}
\put(0,40){\line(1,0){80}}
\linethickness{0.3mm}
\put(0,80){\line(1,0){80}}
\linethickness{0.3mm}
\put(60,0){\line(0,1){40}}
\linethickness{0.3mm}
\put(40,80){\line(0,1){40}}
\linethickness{0.3mm}
\put(0,40){\line(0,1){40}}
\linethickness{0.3mm}
\put(20,0){\line(0,1){40}}
\put(45,117.5){\makebox(0,0)[cl]{$\Cee$}}

\put(15,2.5){\makebox(0,0)[cr]{$\Aee$}}

\put(55,2.5){\makebox(0,0)[cr]{$\Bee$}}

\put(40,60){\makebox(0,0)[cc]{$\fee$}}

\linethickness{0.3mm}
\put(300,40){\line(0,1){40}}
\linethickness{0.3mm}
\put(220,40){\line(1,0){80}}
\linethickness{0.3mm}
\put(220,80){\line(1,0){80}}
\linethickness{0.3mm}
\put(280,0){\line(0,1){40}}
\linethickness{0.3mm}
\put(280,80){\line(0,1){40}}
\linethickness{0.3mm}
\put(220,40){\line(0,1){40}}
\linethickness{0.3mm}
\put(240,0){\line(0,1){40}}
\put(285,117.5){\makebox(0,0)[cl]{$\Cee$}}

\put(235,2.5){\makebox(0,0)[cr]{$\Aee$}}

\put(275,2.5){\makebox(0,0)[cr]{$\Bee$}}

\linethickness{0.3mm}
\put(240,80){\line(0,1){40}}
\put(245,117.5){\makebox(0,0)[cl]{$\Aee$}}

\put(260,60){\makebox(0,0)[cc]{$\qee$}}

\end{picture}
 
\end{split}
\eeq
Since its arguments are mutually independent, $g\colon X\otimes A\to B$ can be thought of as, say, an $X$-parametrized family of transitions $g_{x}\colon A\to B$. Since the output pairs can also be separated, $q\colon X\otimes A\to X\otimes B$ is often viewed as the pair
\beq
\sta q = \left(X\times A\tto q X\times B\tto{\id\times \scun} X\right)\quad \out q = \left(X\times A\tto q X\times B\tto{\scun\times \id} B\right)
\eeq
although they generally cannot be paired back together, since their sharing of side-effects is lost.
 


\section{Universal state of belief}\label{Sec:moncom}
% !TEX root = 00-wollic.tex

The theory of logical theories, or of categorical sketches, is a theory. Together with a chosen basic model, it lives as an object in the universe $\UUU$. Assuming that the universe of sets is also effectively presented, the models $\MMM$ of theories $\Theta$ can also be formalized, and a formal version of Definition~\ref{def:state} of belief states can be found living as an object in $\UUU$, as can a formal version of Definition~\ref{Def:explainable} of explainable belief transitions. Let us call this latter object  $\DP$. It is the universal state of belief. If the explainable belief transitions are viewed as computable predictions and if the explanations are their programs, then $\DP$ can be thought of as a programming language. As a state of belief, it is universal in the sense that it carries a universal interpreter of all explanations for explainable transitions, and of all programs that compute the predictions derived from them.

\begin{definition}
A \emph{universal interpreter}\/ for belief states $A,B$ is a belief transition $\{\}\colon \DP\otimes A\to B$ in $\UUU$ which is universal for all parametric families of belief transitions from $A$ to $B$. This means that for any belief state $X$ and any belief transition $g\in \UUU(X\otimes A, B)$ there is an explanation $G\in \tot\UUU(X,\DP)$ with
\beq\label{eq:uev}
\begin{split}
\newcommand{\Fee}{\scriptstyle G}
\newcommand{\fee}{g}
\newcommand{\Aee}{\scriptstyle X}
\newcommand{\Bee}{\scriptstyle A}
\newcommand{\Cee}{\scriptstyle B}
\newcommand{\Code}{\scriptstyle \PPp}
\newcommand{\Univ}{\mbox{\large$\{\}$}}
\newcommand{\Dott}{\mbox{\Large$\bullet$}}
\def\JPicScale{.18}
%%Created by jPicEdt 1.4.1_03: mixed JPIC-XML/LaTeX format
%%Thu Mar 16 16:35:34 GMT-10:00 2023
%%Begin JPIC-XML
%<?xml version="1.0" standalone="yes"?>
%<jpic x-min="0" x-max="220" y-min="0" y-max="120" auto-bounding="true">
%<multicurve fill-style= "none"
%	 stroke-width= "0.35"
%	 points= "(80,80);(80,80);(80,40);(80,40)"
%	 />
%<multicurve fill-style= "none"
%	 stroke-width= "0.35"
%	 points= "(80,40);(80,40);(0,40);(0,40)"
%	 />
%<multicurve fill-style= "none"
%	 stroke-width= "0.35"
%	 points= "(0,80);(0,80);(80,80);(80,80)"
%	 />
%<multicurve fill-style= "none"
%	 stroke-width= "0.35"
%	 points= "(60,40);(60,40);(60,0);(60,0)"
%	 />
%<multicurve fill-style= "none"
%	 stroke-width= "0.35"
%	 points= "(40,120);(40,120);(40,80);(40,80)"
%	 />
%<multicurve fill-style= "none"
%	 stroke-width= "0.35"
%	 points= "(0,80);(0,80);(0,40);(0,40)"
%	 />
%<multicurve fill-style= "none"
%	 stroke-width= "0.35"
%	 points= "(20,40);(20,40);(20,0);(20,0)"
%	 />
%<multicurve fill-style= "none"
%	 stroke-width= "0.35"
%	 points= "(220,100);(220,100);(220,60);(220,60)"
%	 />
%<multicurve fill-style= "none"
%	 stroke-width= "0.35"
%	 points= "(180,20);(180,20);(140,20);(140,20)"
%	 />
%<multicurve fill-style= "none"
%	 stroke-width= "0.35"
%	 points= "(140,100);(140,100);(220,100);(220,100)"
%	 />
%<multicurve fill-style= "none"
%	 stroke-width= "0.35"
%	 points= "(200,60);(200,60);(200,0);(200,0)"
%	 />
%<multicurve fill-style= "none"
%	 stroke-width= "0.35"
%	 points= "(180,120);(180,120);(180,100);(180,100)"
%	 />
%<multicurve fill-style= "none"
%	 stroke-width= "0.35"
%	 points= "(140,60);(140,60);(140,20);(140,20)"
%	 />
%<multicurve fill-style= "none"
%	 stroke-width= "0.35"
%	 points= "(160,20);(160,20);(160,0);(160,0)"
%	 />
%<multicurve fill-style= "none"
%	 stroke-width= "0.35"
%	 points= "(220,60);(220,60);(180,60);(180,60)"
%	 />
%<multicurve fill-style= "none"
%	 stroke-width= "0.35"
%	 points= "(140,100);(140,100);(180,60);(180,60)"
%	 />
%<multicurve fill-style= "none"
%	 stroke-width= "0.35"
%	 points= "(140,60);(140,60);(180,20);(180,20)"
%	 />
%<multicurve fill-style= "none"
%	 stroke-width= "0.7"
%	 points= "(160,40);(160,40);(160,80);(160,80)"
%	 />
%<text fill-style= "none"
%	 stroke-width= "0.35"
%	 text-vert-align= "center-v"
%	 anchor-point= "(190,80)"
%	 text-frame= "noframe"
%	 text-hor-align= "center-h"
%	 >
%$\Univ$
%</text>
%<text fill-style= "none"
%	 stroke-width= "0.35"
%	 text-vert-align= "center-v"
%	 anchor-point= "(110,60)"
%	 text-frame= "noframe"
%	 text-hor-align= "center-h"
%	 >
%\EQLS
%</text>
%<text fill-style= "none"
%	 stroke-width= "0.35"
%	 text-vert-align= "center-v"
%	 anchor-point= "(156.25,65)"
%	 text-frame= "noframe"
%	 text-hor-align= "right"
%	 >
%$\Code$
%</text>
%<text fill-style= "none"
%	 stroke-width= "0.35"
%	 text-vert-align= "center-v"
%	 anchor-point= "(185,117.5)"
%	 text-frame= "noframe"
%	 text-hor-align= "left"
%	 >
%$\Cee$
%</text>
%<text fill-style= "none"
%	 stroke-width= "0.35"
%	 text-vert-align= "center-v"
%	 anchor-point= "(45,117.5)"
%	 text-frame= "noframe"
%	 text-hor-align= "left"
%	 >
%$\Cee$
%</text>
%<text fill-style= "none"
%	 stroke-width= "0.35"
%	 text-vert-align= "center-v"
%	 anchor-point= "(15,2.5)"
%	 text-frame= "noframe"
%	 text-hor-align= "right"
%	 >
%$\Aee$
%</text>
%<text fill-style= "none"
%	 stroke-width= "0.35"
%	 text-vert-align= "center-v"
%	 anchor-point= "(55,2.5)"
%	 text-frame= "noframe"
%	 text-hor-align= "right"
%	 >
%$\Bee$
%</text>
%<text fill-style= "none"
%	 stroke-width= "0.35"
%	 text-vert-align= "center-v"
%	 anchor-point= "(40,60)"
%	 text-frame= "noframe"
%	 text-hor-align= "center-h"
%	 >
%$\fee$
%</text>
%<text fill-style= "none"
%	 stroke-width= "0.35"
%	 text-vert-align= "center-v"
%	 anchor-point= "(150,31.25)"
%	 text-frame= "noframe"
%	 text-hor-align= "center-h"
%	 >
%$\Fee$
%</text>
%<text fill-style= "none"
%	 stroke-width= "0.35"
%	 text-vert-align= "center-v"
%	 anchor-point= "(155,2.5)"
%	 text-frame= "noframe"
%	 text-hor-align= "right"
%	 >
%$\Aee$
%</text>
%<text fill-style= "none"
%	 stroke-width= "0.35"
%	 text-vert-align= "center-v"
%	 anchor-point= "(195,2.5)"
%	 text-frame= "noframe"
%	 text-hor-align= "right"
%	 >
%$\Bee$
%</text>
%<text fill-style= "none"
%	 stroke-width= "0.35"
%	 text-vert-align= "center-v"
%	 anchor-point= "(160,40)"
%	 text-frame= "noframe"
%	 text-hor-align= "center-h"
%	 >
%$\Dott$
%</text>
%</jpic>
%%End JPIC-XML
%LaTeX-picture environment using emulated lines and arcs
%You can rescale the whole picture (to 80% for instance) by using the command \def\JPicScale{0.8}
\ifx\JPicScale\undefined\def\JPicScale{1}\fi
\unitlength \JPicScale mm
\begin{picture}(220,120)(0,0)
\linethickness{0.35mm}
\put(80,40){\line(0,1){40}}
\linethickness{0.35mm}
\put(0,40){\line(1,0){80}}
\linethickness{0.35mm}
\put(0,80){\line(1,0){80}}
\linethickness{0.35mm}
\put(60,0){\line(0,1){40}}
\linethickness{0.35mm}
\put(40,80){\line(0,1){40}}
\linethickness{0.35mm}
\put(0,40){\line(0,1){40}}
\linethickness{0.35mm}
\put(20,0){\line(0,1){40}}
\linethickness{0.35mm}
\put(220,60){\line(0,1){40}}
\linethickness{0.35mm}
\put(140,20){\line(1,0){40}}
\linethickness{0.35mm}
\put(140,100){\line(1,0){80}}
\linethickness{0.35mm}
\put(200,0){\line(0,1){60}}
\linethickness{0.35mm}
\put(180,100){\line(0,1){20}}
\linethickness{0.35mm}
\put(140,20){\line(0,1){40}}
\linethickness{0.35mm}
\put(160,0){\line(0,1){20}}
\linethickness{0.35mm}
\put(180,60){\line(1,0){40}}
\linethickness{0.35mm}
\multiput(140,100)(0.12,-0.12){333}{\line(1,0){0.12}}
\linethickness{0.35mm}
\multiput(140,60)(0.12,-0.12){333}{\line(1,0){0.12}}
\linethickness{0.7mm}
\put(160,40){\line(0,1){40}}
\put(190,80){\makebox(0,0)[cc]{$\Univ$}}

\put(110,60){\makebox(0,0)[cc]{\EQLS}}

\put(156.25,65){\makebox(0,0)[cr]{$\Code$}}

\put(185,117.5){\makebox(0,0)[cl]{$\Cee$}}

\put(45,117.5){\makebox(0,0)[cl]{$\Cee$}}

\put(15,2.5){\makebox(0,0)[cr]{$\Aee$}}

\put(55,2.5){\makebox(0,0)[cr]{$\Bee$}}

\put(40,60){\makebox(0,0)[cc]{$\fee$}}

\put(150,31.25){\makebox(0,0)[cc]{$\Fee$}}

\put(155,2.5){\makebox(0,0)[cr]{$\Aee$}}

\put(195,2.5){\makebox(0,0)[cr]{$\Bee$}}

\put(160,40){\makebox(0,0)[cc]{$\Dott$}}

\end{picture}
 
\end{split}
\eeq
\end{definition}

\paragraph{Comment.} On one hand, a universal interpreter is universal for parametric families. On the other hand, it is a parametric family itself. It is thus capable of interpreting itself. G\"odel's Incompleteness Theorem is based on this capability of reflection. Universal interpreters support a suitable version of this theorem, very simple, spelled out in \cite[Sec.~5.2]{PavlovicD:MonCom}. But G\"odel's Incompleteness Theorem is concerned with a single belief state, a single theory with a standard model, capable of reflection. The point of this paper is that bots can learn their way out of logic of static belief states, expand their belief transitions through dynamic logic (here in the Hoare form), and arrive to beliefs that are complete in a suitable sense, and thus go beyond G\"odel's Incompleteness Theorem. Such beliefs can be constructed using the \emph{specializers} which are derived  directly from the definition of universal interpreters. 

\begin{lemma}\label{prop:pev}
For any $X, A, B$ there is a\/ \emph{specializer} $\prtial \in \tot\UUU(\DP\times X, \DP)$ such that
\beq\label{eq:pev}
\begin{split}
\newcommand{\Fee}{\scriptstyle \prtial}
\newcommand{\fee}{\mbox{\large$\{\}$}}
\newcommand{\Aee}{\scriptstyle X}
\newcommand{\Bee}{\scriptstyle A}
\newcommand{\Cee}{\scriptstyle B}
\newcommand{\Code}{\scriptstyle \DP}
\newcommand{\Univ}{\mbox{\large$\{\}$}}
\newcommand{\Dott}{\mbox{\LARGE$\bullet$}}
\def\JPicScale{.18}
%%Created by jPicEdt 1.4.1_03: mixed JPIC-XML/LaTeX format
%%Fri Mar 17 18:45:57 GMT-10:00 2023
%%Begin JPIC-XML
%<?xml version="1.0" standalone="yes"?>
%<jpic x-min="0" x-max="280" y-min="0" y-max="120" auto-bounding="true">
%<multicurve fill-style= "none"
%	 points= "(120,80);(120,80);(120,40);(120,40)"
%	 />
%<multicurve fill-style= "none"
%	 points= "(120,40);(120,40);(40,40);(40,40)"
%	 />
%<multicurve fill-style= "none"
%	 points= "(0,80);(0,80);(120,80);(120,80)"
%	 />
%<multicurve fill-style= "none"
%	 points= "(100,40);(100,40);(100,0);(100,0)"
%	 />
%<multicurve fill-style= "none"
%	 points= "(80,120);(80,120);(80,80);(80,80)"
%	 />
%<multicurve fill-style= "none"
%	 points= "(0,80);(0,80);(40,40);(40,40)"
%	 />
%<multicurve fill-style= "none"
%	 points= "(60,40);(60,40);(60,0);(60,0)"
%	 />
%<multicurve fill-style= "none"
%	 points= "(280,100);(280,100);(280,60);(280,60)"
%	 />
%<multicurve fill-style= "none"
%	 points= "(240,20);(240,20);(200,20);(200,20)"
%	 />
%<multicurve fill-style= "none"
%	 points= "(200,100);(200,100);(280,100);(280,100)"
%	 />
%<multicurve fill-style= "none"
%	 points= "(260,60);(260,60);(260,0);(260,0)"
%	 />
%<multicurve fill-style= "none"
%	 points= "(240,120);(240,120);(240,100);(240,100)"
%	 />
%<multicurve fill-style= "none"
%	 points= "(160,60);(160,60);(200,20);(200,20)"
%	 />
%<multicurve fill-style= "none"
%	 points= "(220,20);(220,20);(220,0);(220,0)"
%	 />
%<multicurve fill-style= "none"
%	 points= "(280,60);(280,60);(240,60);(240,60)"
%	 />
%<multicurve fill-style= "none"
%	 points= "(200,100);(200,100);(240,60);(240,60)"
%	 />
%<multicurve fill-style= "none"
%	 points= "(200,60);(200,60);(240,20);(240,20)"
%	 />
%<multicurve fill-style= "none"
%	 points= "(220,40);(220,40);(220,80);(220,80)"
%	 stroke-width= "0.7"
%	 />
%<text text-vert-align= "center-v"
%	 fill-style= "none"
%	 anchor-point= "(250,80)"
%	 text-frame= "noframe"
%	 text-hor-align= "center-h"
%	 >
%$\Univ$
%</text>
%<text text-vert-align= "center-v"
%	 fill-style= "none"
%	 anchor-point= "(140,60)"
%	 text-frame= "noframe"
%	 text-hor-align= "center-h"
%	 >
%\EQLS
%</text>
%<text text-vert-align= "center-v"
%	 fill-style= "none"
%	 anchor-point= "(216.25,65)"
%	 text-frame= "noframe"
%	 text-hor-align= "right"
%	 >
%$\Code$
%</text>
%<text text-vert-align= "center-v"
%	 fill-style= "none"
%	 anchor-point= "(245,117.5)"
%	 text-frame= "noframe"
%	 text-hor-align= "left"
%	 >
%$\Cee$
%</text>
%<text text-vert-align= "center-v"
%	 fill-style= "none"
%	 anchor-point= "(85,117.5)"
%	 text-frame= "noframe"
%	 text-hor-align= "left"
%	 >
%$\Cee$
%</text>
%<text text-vert-align= "center-v"
%	 fill-style= "none"
%	 anchor-point= "(55,2.5)"
%	 text-frame= "noframe"
%	 text-hor-align= "right"
%	 >
%$\Aee$
%</text>
%<text text-vert-align= "center-v"
%	 fill-style= "none"
%	 anchor-point= "(95,2.5)"
%	 text-frame= "noframe"
%	 text-hor-align= "right"
%	 >
%$\Bee$
%</text>
%<text text-vert-align= "center-v"
%	 fill-style= "none"
%	 anchor-point= "(80,60)"
%	 text-frame= "noframe"
%	 text-hor-align= "center-h"
%	 >
%$\fee$
%</text>
%<text text-vert-align= "center-v"
%	 fill-style= "none"
%	 anchor-point= "(200,40)"
%	 text-frame= "noframe"
%	 text-hor-align= "center-h"
%	 >
%$\Fee$
%</text>
%<text text-vert-align= "center-v"
%	 fill-style= "none"
%	 anchor-point= "(215,2.5)"
%	 text-frame= "noframe"
%	 text-hor-align= "right"
%	 >
%$\Aee$
%</text>
%<text text-vert-align= "center-v"
%	 fill-style= "none"
%	 anchor-point= "(255,2.5)"
%	 text-frame= "noframe"
%	 text-hor-align= "right"
%	 >
%$\Bee$
%</text>
%<text text-vert-align= "center-v"
%	 fill-style= "none"
%	 anchor-point= "(220,40)"
%	 text-frame= "noframe"
%	 text-hor-align= "center-h"
%	 >
%$\Dott$
%</text>
%<multicurve fill-style= "none"
%	 points= "(20,60);(20,60);(20,0);(20,0)"
%	 />
%<multicurve fill-style= "none"
%	 points= "(200,60);(200,60);(160,60);(160,60)"
%	 />
%<multicurve fill-style= "none"
%	 points= "(180,40);(180,40);(180,0);(180,0)"
%	 />
%<text text-vert-align= "center-v"
%	 fill-style= "none"
%	 anchor-point= "(175,2.5)"
%	 text-frame= "noframe"
%	 text-hor-align= "right"
%	 >
%$\Code$
%</text>
%<text text-vert-align= "center-v"
%	 fill-style= "none"
%	 anchor-point= "(15,2.5)"
%	 text-frame= "noframe"
%	 text-hor-align= "right"
%	 >
%$\Code$
%</text>
%</jpic>
%%End JPIC-XML
%LaTeX-picture environment using emulated lines and arcs
%You can rescale the whole picture (to 80% for instance) by using the command \def\JPicScale{0.8}
\ifx\JPicScale\undefined\def\JPicScale{1}\fi
\unitlength \JPicScale mm
\begin{picture}(280,120)(0,0)
\linethickness{0.3mm}
\put(120,40){\line(0,1){40}}
\linethickness{0.3mm}
\put(40,40){\line(1,0){80}}
\linethickness{0.3mm}
\put(0,80){\line(1,0){120}}
\linethickness{0.3mm}
\put(100,0){\line(0,1){40}}
\linethickness{0.3mm}
\put(80,80){\line(0,1){40}}
\linethickness{0.3mm}
\multiput(0,80)(0.12,-0.12){333}{\line(1,0){0.12}}
\linethickness{0.3mm}
\put(60,0){\line(0,1){40}}
\linethickness{0.3mm}
\put(280,60){\line(0,1){40}}
\linethickness{0.3mm}
\put(200,20){\line(1,0){40}}
\linethickness{0.3mm}
\put(200,100){\line(1,0){80}}
\linethickness{0.3mm}
\put(260,0){\line(0,1){60}}
\linethickness{0.3mm}
\put(240,100){\line(0,1){20}}
\linethickness{0.3mm}
\multiput(160,60)(0.12,-0.12){333}{\line(1,0){0.12}}
\linethickness{0.3mm}
\put(220,0){\line(0,1){20}}
\linethickness{0.3mm}
\put(240,60){\line(1,0){40}}
\linethickness{0.3mm}
\multiput(200,100)(0.12,-0.12){333}{\line(1,0){0.12}}
\linethickness{0.3mm}
\multiput(200,60)(0.12,-0.12){333}{\line(1,0){0.12}}
\linethickness{0.7mm}
\put(220,40){\line(0,1){40}}
\put(250,80){\makebox(0,0)[cc]{$\Univ$}}

\put(140,60){\makebox(0,0)[cc]{\EQLS}}

\put(216.25,65){\makebox(0,0)[cr]{$\Code$}}

\put(245,117.5){\makebox(0,0)[cl]{$\Cee$}}

\put(85,117.5){\makebox(0,0)[cl]{$\Cee$}}

\put(55,2.5){\makebox(0,0)[cr]{$\Aee$}}

\put(95,2.5){\makebox(0,0)[cr]{$\Bee$}}

\put(80,60){\makebox(0,0)[cc]{$\fee$}}

\put(200,40){\makebox(0,0)[cc]{$\Fee$}}

\put(215,2.5){\makebox(0,0)[cr]{$\Aee$}}

\put(255,2.5){\makebox(0,0)[cr]{$\Bee$}}

\put(220,40){\makebox(0,0)[cc]{$\Dott$}}

\linethickness{0.3mm}
\put(20,0){\line(0,1){60}}
\linethickness{0.3mm}
\put(160,60){\line(1,0){40}}
\linethickness{0.3mm}
\put(180,0){\line(0,1){40}}
\put(175,2.5){\makebox(0,0)[cr]{$\Code$}}

\put(15,2.5){\makebox(0,0)[cr]{$\Code$}}

\end{picture}
 
\end{split}
\eeq
\end{lemma}

\paragraph{Hoare logic view of interpreters and specializers.} If interpreters are presented as Hoare triples in the form $(X\otimes A)\uev G B$, and if $X\!\pev G$ denotes an explanation $G$ specialized to $X$, then \eqref{eq:pev} is equivalent to the invertible rule
\[\prooftree
(X\otimes A)\uev G B
\Justifies
A\uev{X\!\pev G} B
\endprooftree\]   
The advantages of the string-diagrammatic view will become apparent in the next section. 

\section{Self-confirming explanations and\\ self-fulfilling predictions}\label{Sec:self-fulfill}
% !TEX root = 00-wollic.tex

A theory of theories, such as the categorical theory of sketches, is a theory. Category theory is also a theory and functorial semantics provides a categorical theory of reference models. The theory of state spaces from Sec.~\ref{Sec:state} can thus be formalized and presented as a state space in the category $\UUU$. The theory of state spaces from Sec.~\ref{Sec:state} can thus be formalized into a sketch with a reference model and presented as a state space in the category $\UUU$. The theory of state transitions from Sec.~\ref{Sec:transition} is another sketch, and with another reference model it is also a  state space in $\UUU$. Call it $\DP$. The fact that the states in $\DP$ correspond to the transitions in $\UUU$ means that it satisfies a parametrized version of \eqref{eq:univ}. It is a universal language for $\UUU$. Its interpreters follow from its definition, as the models of the theory of transitions. Since there is no room here to spell out the details of a theory of transitions and show that the correspondence of its cartesian models and the transitions in $\UUU$ equips $\DP$ with all interpreters, we  postulate the existence of the interpreters by the following definition. 

\begin{definition}\label{Def:interpreter}
An \emph{universal interpreter}\/ for state spaces $A,B$ is a transition $\{\}\colon \DP\otimes A\to B$ in $\UUU$ which is universal for all parametric families of transitions from $A$ to $B$. This means that for any state space $X$ and any transition $g\in \UUU(X\otimes A, B)$ there is an interpretation $G\in \tot\UUU(X,\DP)$ with
\beq\label{eq:uev}
\begin{split}
\newcommand{\Fee}{\scriptstyle G}
\newcommand{\fee}{g}
\newcommand{\Aee}{\scriptstyle X}
\newcommand{\Bee}{\scriptstyle A}
\newcommand{\Cee}{\scriptstyle B}
\newcommand{\Code}{\scriptstyle \PPp}
\newcommand{\Univ}{\mbox{\large$\{\}$}}
\newcommand{\Dott}{\mbox{\Large$\bullet$}}
\def\JPicScale{.33}
%%Created by jPicEdt 1.4.1_03: mixed JPIC-XML/LaTeX format
%%Thu Mar 16 16:35:34 GMT-10:00 2023
%%Begin JPIC-XML
%<?xml version="1.0" standalone="yes"?>
%<jpic x-min="0" x-max="220" y-min="0" y-max="120" auto-bounding="true">
%<multicurve fill-style= "none"
%	 stroke-width= "0.35"
%	 points= "(80,80);(80,80);(80,40);(80,40)"
%	 />
%<multicurve fill-style= "none"
%	 stroke-width= "0.35"
%	 points= "(80,40);(80,40);(0,40);(0,40)"
%	 />
%<multicurve fill-style= "none"
%	 stroke-width= "0.35"
%	 points= "(0,80);(0,80);(80,80);(80,80)"
%	 />
%<multicurve fill-style= "none"
%	 stroke-width= "0.35"
%	 points= "(60,40);(60,40);(60,0);(60,0)"
%	 />
%<multicurve fill-style= "none"
%	 stroke-width= "0.35"
%	 points= "(40,120);(40,120);(40,80);(40,80)"
%	 />
%<multicurve fill-style= "none"
%	 stroke-width= "0.35"
%	 points= "(0,80);(0,80);(0,40);(0,40)"
%	 />
%<multicurve fill-style= "none"
%	 stroke-width= "0.35"
%	 points= "(20,40);(20,40);(20,0);(20,0)"
%	 />
%<multicurve fill-style= "none"
%	 stroke-width= "0.35"
%	 points= "(220,100);(220,100);(220,60);(220,60)"
%	 />
%<multicurve fill-style= "none"
%	 stroke-width= "0.35"
%	 points= "(180,20);(180,20);(140,20);(140,20)"
%	 />
%<multicurve fill-style= "none"
%	 stroke-width= "0.35"
%	 points= "(140,100);(140,100);(220,100);(220,100)"
%	 />
%<multicurve fill-style= "none"
%	 stroke-width= "0.35"
%	 points= "(200,60);(200,60);(200,0);(200,0)"
%	 />
%<multicurve fill-style= "none"
%	 stroke-width= "0.35"
%	 points= "(180,120);(180,120);(180,100);(180,100)"
%	 />
%<multicurve fill-style= "none"
%	 stroke-width= "0.35"
%	 points= "(140,60);(140,60);(140,20);(140,20)"
%	 />
%<multicurve fill-style= "none"
%	 stroke-width= "0.35"
%	 points= "(160,20);(160,20);(160,0);(160,0)"
%	 />
%<multicurve fill-style= "none"
%	 stroke-width= "0.35"
%	 points= "(220,60);(220,60);(180,60);(180,60)"
%	 />
%<multicurve fill-style= "none"
%	 stroke-width= "0.35"
%	 points= "(140,100);(140,100);(180,60);(180,60)"
%	 />
%<multicurve fill-style= "none"
%	 stroke-width= "0.35"
%	 points= "(140,60);(140,60);(180,20);(180,20)"
%	 />
%<multicurve fill-style= "none"
%	 stroke-width= "0.7"
%	 points= "(160,40);(160,40);(160,80);(160,80)"
%	 />
%<text fill-style= "none"
%	 stroke-width= "0.35"
%	 text-vert-align= "center-v"
%	 anchor-point= "(190,80)"
%	 text-frame= "noframe"
%	 text-hor-align= "center-h"
%	 >
%$\Univ$
%</text>
%<text fill-style= "none"
%	 stroke-width= "0.35"
%	 text-vert-align= "center-v"
%	 anchor-point= "(110,60)"
%	 text-frame= "noframe"
%	 text-hor-align= "center-h"
%	 >
%\EQLS
%</text>
%<text fill-style= "none"
%	 stroke-width= "0.35"
%	 text-vert-align= "center-v"
%	 anchor-point= "(156.25,65)"
%	 text-frame= "noframe"
%	 text-hor-align= "right"
%	 >
%$\Code$
%</text>
%<text fill-style= "none"
%	 stroke-width= "0.35"
%	 text-vert-align= "center-v"
%	 anchor-point= "(185,117.5)"
%	 text-frame= "noframe"
%	 text-hor-align= "left"
%	 >
%$\Cee$
%</text>
%<text fill-style= "none"
%	 stroke-width= "0.35"
%	 text-vert-align= "center-v"
%	 anchor-point= "(45,117.5)"
%	 text-frame= "noframe"
%	 text-hor-align= "left"
%	 >
%$\Cee$
%</text>
%<text fill-style= "none"
%	 stroke-width= "0.35"
%	 text-vert-align= "center-v"
%	 anchor-point= "(15,2.5)"
%	 text-frame= "noframe"
%	 text-hor-align= "right"
%	 >
%$\Aee$
%</text>
%<text fill-style= "none"
%	 stroke-width= "0.35"
%	 text-vert-align= "center-v"
%	 anchor-point= "(55,2.5)"
%	 text-frame= "noframe"
%	 text-hor-align= "right"
%	 >
%$\Bee$
%</text>
%<text fill-style= "none"
%	 stroke-width= "0.35"
%	 text-vert-align= "center-v"
%	 anchor-point= "(40,60)"
%	 text-frame= "noframe"
%	 text-hor-align= "center-h"
%	 >
%$\fee$
%</text>
%<text fill-style= "none"
%	 stroke-width= "0.35"
%	 text-vert-align= "center-v"
%	 anchor-point= "(150,31.25)"
%	 text-frame= "noframe"
%	 text-hor-align= "center-h"
%	 >
%$\Fee$
%</text>
%<text fill-style= "none"
%	 stroke-width= "0.35"
%	 text-vert-align= "center-v"
%	 anchor-point= "(155,2.5)"
%	 text-frame= "noframe"
%	 text-hor-align= "right"
%	 >
%$\Aee$
%</text>
%<text fill-style= "none"
%	 stroke-width= "0.35"
%	 text-vert-align= "center-v"
%	 anchor-point= "(195,2.5)"
%	 text-frame= "noframe"
%	 text-hor-align= "right"
%	 >
%$\Bee$
%</text>
%<text fill-style= "none"
%	 stroke-width= "0.35"
%	 text-vert-align= "center-v"
%	 anchor-point= "(160,40)"
%	 text-frame= "noframe"
%	 text-hor-align= "center-h"
%	 >
%$\Dott$
%</text>
%</jpic>
%%End JPIC-XML
%LaTeX-picture environment using emulated lines and arcs
%You can rescale the whole picture (to 80% for instance) by using the command \def\JPicScale{0.8}
\ifx\JPicScale\undefined\def\JPicScale{1}\fi
\unitlength \JPicScale mm
\begin{picture}(220,120)(0,0)
\linethickness{0.35mm}
\put(80,40){\line(0,1){40}}
\linethickness{0.35mm}
\put(0,40){\line(1,0){80}}
\linethickness{0.35mm}
\put(0,80){\line(1,0){80}}
\linethickness{0.35mm}
\put(60,0){\line(0,1){40}}
\linethickness{0.35mm}
\put(40,80){\line(0,1){40}}
\linethickness{0.35mm}
\put(0,40){\line(0,1){40}}
\linethickness{0.35mm}
\put(20,0){\line(0,1){40}}
\linethickness{0.35mm}
\put(220,60){\line(0,1){40}}
\linethickness{0.35mm}
\put(140,20){\line(1,0){40}}
\linethickness{0.35mm}
\put(140,100){\line(1,0){80}}
\linethickness{0.35mm}
\put(200,0){\line(0,1){60}}
\linethickness{0.35mm}
\put(180,100){\line(0,1){20}}
\linethickness{0.35mm}
\put(140,20){\line(0,1){40}}
\linethickness{0.35mm}
\put(160,0){\line(0,1){20}}
\linethickness{0.35mm}
\put(180,60){\line(1,0){40}}
\linethickness{0.35mm}
\multiput(140,100)(0.12,-0.12){333}{\line(1,0){0.12}}
\linethickness{0.35mm}
\multiput(140,60)(0.12,-0.12){333}{\line(1,0){0.12}}
\linethickness{0.7mm}
\put(160,40){\line(0,1){40}}
\put(190,80){\makebox(0,0)[cc]{$\Univ$}}

\put(110,60){\makebox(0,0)[cc]{\EQLS}}

\put(156.25,65){\makebox(0,0)[cr]{$\Code$}}

\put(185,117.5){\makebox(0,0)[cl]{$\Cee$}}

\put(45,117.5){\makebox(0,0)[cl]{$\Cee$}}

\put(15,2.5){\makebox(0,0)[cr]{$\Aee$}}

\put(55,2.5){\makebox(0,0)[cr]{$\Bee$}}

\put(40,60){\makebox(0,0)[cc]{$\fee$}}

\put(150,31.25){\makebox(0,0)[cc]{$\Fee$}}

\put(155,2.5){\makebox(0,0)[cr]{$\Aee$}}

\put(195,2.5){\makebox(0,0)[cr]{$\Bee$}}

\put(160,40){\makebox(0,0)[cc]{$\Dott$}}

\end{picture}
 
\end{split}
\eeq
\end{definition}

On one hand, a universal interpreter is universal for parametric families. On the other hand, it is a parametric family itself. It is thus capable of interpreting itself.  This capability of self-reflection was crucial for G\"odel's incompleteness construction.  This capability is embodied in the \emph{specializers}, which are derived  directly from Def~\ref{Def:interpreter}. 

\begin{lemma}\label{prop:pev}
For any $X, A, B$ there is an interpretation $\prtial \in \tot\UUU(\DP\times X, \DP)$ which specializes from a given $X\otimes A$-interpreter to an $A$-interpreter, in the sense
\beq\label{eq:pev}
\begin{split}
\newcommand{\Fee}{\prtial}
\newcommand{\fee}{\mbox{\large$\{\}$}}
\newcommand{\Aee}{\scriptstyle X}
\newcommand{\Bee}{\scriptstyle A}
\newcommand{\Cee}{\scriptstyle B}
\newcommand{\Code}{\scriptstyle \DP}
\newcommand{\Univ}{\mbox{\large$\{\}$}}
\newcommand{\Dott}{\mbox{\LARGE$\bullet$}}
\def\JPicScale{.33}
%%Created by jPicEdt 1.4.1_03: mixed JPIC-XML/LaTeX format
%%Fri Mar 17 18:45:57 GMT-10:00 2023
%%Begin JPIC-XML
%<?xml version="1.0" standalone="yes"?>
%<jpic x-min="0" x-max="280" y-min="0" y-max="120" auto-bounding="true">
%<multicurve fill-style= "none"
%	 points= "(120,80);(120,80);(120,40);(120,40)"
%	 />
%<multicurve fill-style= "none"
%	 points= "(120,40);(120,40);(40,40);(40,40)"
%	 />
%<multicurve fill-style= "none"
%	 points= "(0,80);(0,80);(120,80);(120,80)"
%	 />
%<multicurve fill-style= "none"
%	 points= "(100,40);(100,40);(100,0);(100,0)"
%	 />
%<multicurve fill-style= "none"
%	 points= "(80,120);(80,120);(80,80);(80,80)"
%	 />
%<multicurve fill-style= "none"
%	 points= "(0,80);(0,80);(40,40);(40,40)"
%	 />
%<multicurve fill-style= "none"
%	 points= "(60,40);(60,40);(60,0);(60,0)"
%	 />
%<multicurve fill-style= "none"
%	 points= "(280,100);(280,100);(280,60);(280,60)"
%	 />
%<multicurve fill-style= "none"
%	 points= "(240,20);(240,20);(200,20);(200,20)"
%	 />
%<multicurve fill-style= "none"
%	 points= "(200,100);(200,100);(280,100);(280,100)"
%	 />
%<multicurve fill-style= "none"
%	 points= "(260,60);(260,60);(260,0);(260,0)"
%	 />
%<multicurve fill-style= "none"
%	 points= "(240,120);(240,120);(240,100);(240,100)"
%	 />
%<multicurve fill-style= "none"
%	 points= "(160,60);(160,60);(200,20);(200,20)"
%	 />
%<multicurve fill-style= "none"
%	 points= "(220,20);(220,20);(220,0);(220,0)"
%	 />
%<multicurve fill-style= "none"
%	 points= "(280,60);(280,60);(240,60);(240,60)"
%	 />
%<multicurve fill-style= "none"
%	 points= "(200,100);(200,100);(240,60);(240,60)"
%	 />
%<multicurve fill-style= "none"
%	 points= "(200,60);(200,60);(240,20);(240,20)"
%	 />
%<multicurve fill-style= "none"
%	 points= "(220,40);(220,40);(220,80);(220,80)"
%	 stroke-width= "0.7"
%	 />
%<text text-vert-align= "center-v"
%	 fill-style= "none"
%	 anchor-point= "(250,80)"
%	 text-frame= "noframe"
%	 text-hor-align= "center-h"
%	 >
%$\Univ$
%</text>
%<text text-vert-align= "center-v"
%	 fill-style= "none"
%	 anchor-point= "(140,60)"
%	 text-frame= "noframe"
%	 text-hor-align= "center-h"
%	 >
%\EQLS
%</text>
%<text text-vert-align= "center-v"
%	 fill-style= "none"
%	 anchor-point= "(216.25,65)"
%	 text-frame= "noframe"
%	 text-hor-align= "right"
%	 >
%$\Code$
%</text>
%<text text-vert-align= "center-v"
%	 fill-style= "none"
%	 anchor-point= "(245,117.5)"
%	 text-frame= "noframe"
%	 text-hor-align= "left"
%	 >
%$\Cee$
%</text>
%<text text-vert-align= "center-v"
%	 fill-style= "none"
%	 anchor-point= "(85,117.5)"
%	 text-frame= "noframe"
%	 text-hor-align= "left"
%	 >
%$\Cee$
%</text>
%<text text-vert-align= "center-v"
%	 fill-style= "none"
%	 anchor-point= "(55,2.5)"
%	 text-frame= "noframe"
%	 text-hor-align= "right"
%	 >
%$\Aee$
%</text>
%<text text-vert-align= "center-v"
%	 fill-style= "none"
%	 anchor-point= "(95,2.5)"
%	 text-frame= "noframe"
%	 text-hor-align= "right"
%	 >
%$\Bee$
%</text>
%<text text-vert-align= "center-v"
%	 fill-style= "none"
%	 anchor-point= "(80,60)"
%	 text-frame= "noframe"
%	 text-hor-align= "center-h"
%	 >
%$\fee$
%</text>
%<text text-vert-align= "center-v"
%	 fill-style= "none"
%	 anchor-point= "(200,40)"
%	 text-frame= "noframe"
%	 text-hor-align= "center-h"
%	 >
%$\Fee$
%</text>
%<text text-vert-align= "center-v"
%	 fill-style= "none"
%	 anchor-point= "(215,2.5)"
%	 text-frame= "noframe"
%	 text-hor-align= "right"
%	 >
%$\Aee$
%</text>
%<text text-vert-align= "center-v"
%	 fill-style= "none"
%	 anchor-point= "(255,2.5)"
%	 text-frame= "noframe"
%	 text-hor-align= "right"
%	 >
%$\Bee$
%</text>
%<text text-vert-align= "center-v"
%	 fill-style= "none"
%	 anchor-point= "(220,40)"
%	 text-frame= "noframe"
%	 text-hor-align= "center-h"
%	 >
%$\Dott$
%</text>
%<multicurve fill-style= "none"
%	 points= "(20,60);(20,60);(20,0);(20,0)"
%	 />
%<multicurve fill-style= "none"
%	 points= "(200,60);(200,60);(160,60);(160,60)"
%	 />
%<multicurve fill-style= "none"
%	 points= "(180,40);(180,40);(180,0);(180,0)"
%	 />
%<text text-vert-align= "center-v"
%	 fill-style= "none"
%	 anchor-point= "(175,2.5)"
%	 text-frame= "noframe"
%	 text-hor-align= "right"
%	 >
%$\Code$
%</text>
%<text text-vert-align= "center-v"
%	 fill-style= "none"
%	 anchor-point= "(15,2.5)"
%	 text-frame= "noframe"
%	 text-hor-align= "right"
%	 >
%$\Code$
%</text>
%</jpic>
%%End JPIC-XML
%LaTeX-picture environment using emulated lines and arcs
%You can rescale the whole picture (to 80% for instance) by using the command \def\JPicScale{0.8}
\ifx\JPicScale\undefined\def\JPicScale{1}\fi
\unitlength \JPicScale mm
\begin{picture}(280,120)(0,0)
\linethickness{0.3mm}
\put(120,40){\line(0,1){40}}
\linethickness{0.3mm}
\put(40,40){\line(1,0){80}}
\linethickness{0.3mm}
\put(0,80){\line(1,0){120}}
\linethickness{0.3mm}
\put(100,0){\line(0,1){40}}
\linethickness{0.3mm}
\put(80,80){\line(0,1){40}}
\linethickness{0.3mm}
\multiput(0,80)(0.12,-0.12){333}{\line(1,0){0.12}}
\linethickness{0.3mm}
\put(60,0){\line(0,1){40}}
\linethickness{0.3mm}
\put(280,60){\line(0,1){40}}
\linethickness{0.3mm}
\put(200,20){\line(1,0){40}}
\linethickness{0.3mm}
\put(200,100){\line(1,0){80}}
\linethickness{0.3mm}
\put(260,0){\line(0,1){60}}
\linethickness{0.3mm}
\put(240,100){\line(0,1){20}}
\linethickness{0.3mm}
\multiput(160,60)(0.12,-0.12){333}{\line(1,0){0.12}}
\linethickness{0.3mm}
\put(220,0){\line(0,1){20}}
\linethickness{0.3mm}
\put(240,60){\line(1,0){40}}
\linethickness{0.3mm}
\multiput(200,100)(0.12,-0.12){333}{\line(1,0){0.12}}
\linethickness{0.3mm}
\multiput(200,60)(0.12,-0.12){333}{\line(1,0){0.12}}
\linethickness{0.7mm}
\put(220,40){\line(0,1){40}}
\put(250,80){\makebox(0,0)[cc]{$\Univ$}}

\put(140,60){\makebox(0,0)[cc]{\EQLS}}

\put(216.25,65){\makebox(0,0)[cr]{$\Code$}}

\put(245,117.5){\makebox(0,0)[cl]{$\Cee$}}

\put(85,117.5){\makebox(0,0)[cl]{$\Cee$}}

\put(55,2.5){\makebox(0,0)[cr]{$\Aee$}}

\put(95,2.5){\makebox(0,0)[cr]{$\Bee$}}

\put(80,60){\makebox(0,0)[cc]{$\fee$}}

\put(200,40){\makebox(0,0)[cc]{$\Fee$}}

\put(215,2.5){\makebox(0,0)[cr]{$\Aee$}}

\put(255,2.5){\makebox(0,0)[cr]{$\Bee$}}

\put(220,40){\makebox(0,0)[cc]{$\Dott$}}

\linethickness{0.3mm}
\put(20,0){\line(0,1){60}}
\linethickness{0.3mm}
\put(160,60){\line(1,0){40}}
\linethickness{0.3mm}
\put(180,0){\line(0,1){40}}
\put(175,2.5){\makebox(0,0)[cr]{$\Code$}}

\put(15,2.5){\makebox(0,0)[cr]{$\Code$}}

\end{picture}
 
\end{split}
\eeq
\end{lemma}

\paragraph{Hoare logic of interpreters and specializers.} If interpreters are presented as Hoare triples in the form $(X\otimes A)\uev G B$, and if $X\!\pev G$ denotes a specialization of $G$ to $X$ as above, then \eqref{eq:pev} can be written as the invertible Hoare rule
\[\prooftree
(X\otimes A)\uev G B
\Justifies
A\uev{X\!\pev G} B
\endprooftree\]   

\paragraph{Explanations.} Interpretations  (in the sense of Def.~\ref{Def:interpretable}) of arbitrary states from some space $X$ along $G\in \tot\UUU(X,\DP)$ in a universal language $\DP$ can be construed as \emph{explanations}. If $\DP$ is a programming language, they are programs. The idea that explaining a process means programming a computation has been pursued in theory of science from various directions \cite[and references therein]{Osherson:sci-inquiry}. A universal language $\DP$ is thus a universal space of explanations. The idea of programming languages as universal state spaces is pursued in \cite[Ch.~7]{PavlovicD:MonCom}. Just like any universal programming language makes every computation programmable, any universal language from Def.~\ref{Def:interpreter} makes any observable transition explainable. What we cannot explain, we cannot recognize, and therefore we cannot observe. But it gets funny when we take into account how our explanations influence our observations, and how our current explanations can be made to steer future observations. This is sketched in the next two sections. 




\section{Unfalsifiable explanations}\label{Sec:unfalse}
% !TEX root = 00-wollic.tex

When a state change depends on our explanations, then we can find an explanation consistent with its own impact: the state changes the way the explanation predicts. More precisely, if a family of transitions in the form $t\colon \DP\otimes X\otimes A\to B$, then the predictions $t_{\ell x}$ can be steered by varying the explanations $\ell$ for every $x$ until a family of explanations $\enco t \colon X\to \DP$ is found, which is self-confirming at all states $x$, i.e. it satisfies $t(\enco t_{x}, x, a) = \uev{\enco t_{x}} a$. 

\begin{proposition}
For any belief transition $t \in \UUU(\DP\otimes X\otimes A, B)$ there is an explanation $\enco t\in \tot \UUU(X,\DP)$ such that
\beq\label{eq:fund}
\begin{split}
\newcommand{\DOTT}{\mbox{\Large$\bullet$}}
\newcommand{\Aah}{\scriptscriptstyle A}
\newcommand{\Xah}{\scriptscriptstyle X}
\newcommand{\grr}{t}
\newcommand{\Bah}{\scriptscriptstyle B}
\newcommand{\GRR}{\scriptstyle \enco t}
\newcommand{\UK}{\mbox{\large$\{\}$}}
\def\JPicScale{.55}
%%Created by jPicEdt 1.4.1_03: mixed JPIC-XML/LaTeX format
%%Thu Mar 16 22:45:56 GMT-10:00 2023
%%Begin JPIC-XML
%<?xml version="1.0" standalone="yes"?>
%<jpic x-min="0" x-max="115" y-min="0" y-max="90" auto-bounding="true">
%<multicurve fill-style= "none"
%	 points= "(5,70);(5,70);(55,70);(55,70)"
%	 />
%<multicurve fill-style= "none"
%	 points= "(55,70);(55,70);(55,50);(55,50)"
%	 />
%<multicurve fill-style= "none"
%	 points= "(55,50);(55,50);(5,50);(5,50)"
%	 />
%<multicurve fill-style= "none"
%	 points= "(5,50);(5,50);(5,70);(5,70)"
%	 />
%<multicurve fill-style= "none"
%	 points= "(30,70);(30,70);(30,90);(30,90)"
%	 />
%<multicurve fill-style= "none"
%	 points= "(10,50);(10,50);(10,35);(10,35)"
%	 stroke-width= "0.65"
%	 />
%<multicurve fill-style= "none"
%	 points= "(30,50);(30,50);(30,20);(30,20)"
%	 />
%<multicurve fill-style= "none"
%	 points= "(0,45);(0,45);(20,25);(20,25)"
%	 />
%<multicurve fill-style= "none"
%	 points= "(20,25);(20,25);(0,25);(0,25)"
%	 />
%<multicurve fill-style= "none"
%	 points= "(0,25);(0,25);(0,45);(0,45)"
%	 />
%<multicurve fill-style= "none"
%	 points= "(75,70);(75,70);(115,70);(115,70)"
%	 />
%<multicurve fill-style= "none"
%	 points= "(115,70);(115,70);(115,50);(115,50)"
%	 />
%<multicurve fill-style= "none"
%	 points= "(115,50);(115,50);(95,50);(95,50)"
%	 />
%<multicurve fill-style= "none"
%	 points= "(95,50);(95,50);(75,70);(75,70)"
%	 />
%<multicurve fill-style= "none"
%	 points= "(100,70);(100,70);(100,90);(100,90)"
%	 />
%<multicurve fill-style= "none"
%	 points= "(10,25);(10,25);(10,20);(10,20)"
%	 />
%<multicurve fill-style= "none"
%	 points= "(95,25);(95,25);(75,45);(75,45)"
%	 />
%<multicurve fill-style= "none"
%	 points= "(75,25);(75,25);(95,25);(95,25)"
%	 />
%<multicurve fill-style= "none"
%	 points= "(75,45);(75,45);(75,25);(75,25)"
%	 />
%<multicurve fill-style= "none"
%	 points= "(85,60);(85,60);(85,35);(85,35)"
%	 stroke-width= "0.65"
%	 />
%<text fill-style= "none"
%	 text-vert-align= "center-v"
%	 anchor-point= "(10,35)"
%	 text-frame= "noframe"
%	 text-hor-align= "center-h"
%	 >
%\DOTT
%</text>
%<text fill-style= "none"
%	 text-vert-align= "center-v"
%	 anchor-point= "(85,35)"
%	 text-frame= "noframe"
%	 text-hor-align= "center-h"
%	 >
%\DOTT
%</text>
%<text fill-style= "none"
%	 text-vert-align= "center-v"
%	 anchor-point= "(46.25,0)"
%	 text-frame= "noframe"
%	 text-hor-align= "right"
%	 >
%$\Aah$
%</text>
%<text fill-style= "none"
%	 text-vert-align= "center-v"
%	 anchor-point= "(107.5,0)"
%	 text-frame= "noframe"
%	 text-hor-align= "right"
%	 >
%$\Aah$
%</text>
%<text fill-style= "none"
%	 text-vert-align= "center-v"
%	 anchor-point= "(30,60)"
%	 text-frame= "noframe"
%	 text-hor-align= "center-h"
%	 >
%$\grr$
%</text>
%<text fill-style= "none"
%	 text-vert-align= "center-v"
%	 anchor-point= "(32.5,90)"
%	 text-frame= "noframe"
%	 text-hor-align= "left"
%	 >
%$\Bah$
%</text>
%<text fill-style= "none"
%	 text-vert-align= "center-v"
%	 anchor-point= "(102.5,90)"
%	 text-frame= "noframe"
%	 text-hor-align= "left"
%	 >
%$\Bah$
%</text>
%<text fill-style= "none"
%	 text-vert-align= "center-v"
%	 anchor-point= "(100,60)"
%	 text-frame= "noframe"
%	 text-hor-align= "center-h"
%	 >
%$\UK$
%</text>
%<text fill-style= "none"
%	 text-vert-align= "center-v"
%	 anchor-point= "(65,60)"
%	 text-frame= "noframe"
%	 text-hor-align= "center-h"
%	 >
%\EQLS
%</text>
%<text fill-style= "none"
%	 text-vert-align= "center-v"
%	 anchor-point= "(5,30)"
%	 text-frame= "noframe"
%	 text-hor-align= "center-h"
%	 >
%$\GRR$
%</text>
%<text fill-style= "none"
%	 text-vert-align= "center-v"
%	 anchor-point= "(80,30)"
%	 text-frame= "noframe"
%	 text-hor-align= "center-h"
%	 >
%$\GRR$
%</text>
%<multicurve fill-style= "none"
%	 points= "(50,50);(50,50);(50,0);(50,0)"
%	 />
%<multicurve fill-style= "none"
%	 points= "(110,50);(110,50);(110,0);(110,0)"
%	 />
%<multicurve fill-style= "none"
%	 points= "(10,20);(10,20);(20,10);(20,10)"
%	 />
%<multicurve fill-style= "none"
%	 points= "(30,20);(30,20);(20,10);(20,10)"
%	 />
%<multicurve fill-style= "none"
%	 points= "(20,10);(20,10);(20,0);(20,0)"
%	 />
%<multicurve fill-style= "none"
%	 points= "(85,25);(85,25);(85,0);(85,0)"
%	 />
%<text fill-style= "none"
%	 text-vert-align= "center-v"
%	 anchor-point= "(82.5,0)"
%	 text-frame= "noframe"
%	 text-hor-align= "right"
%	 >
%$\Xah$
%</text>
%<text fill-style= "none"
%	 text-vert-align= "center-v"
%	 anchor-point= "(17.5,0)"
%	 text-frame= "noframe"
%	 text-hor-align= "right"
%	 >
%$\Xah$
%</text>
%<text fill-style= "none"
%	 text-vert-align= "center-v"
%	 anchor-point= "(20,10)"
%	 text-frame= "noframe"
%	 text-hor-align= "center-h"
%	 >
%\DOTT
%</text>
%</jpic>
%%End JPIC-XML
%LaTeX-picture environment using emulated lines and arcs
%You can rescale the whole picture (to 80% for instance) by using the command \def\JPicScale{0.8}
\ifx\JPicScale\undefined\def\JPicScale{1}\fi
\unitlength \JPicScale mm
\begin{picture}(115,90)(0,0)
\linethickness{0.3mm}
\put(5,70){\line(1,0){50}}
\linethickness{0.3mm}
\put(55,50){\line(0,1){20}}
\linethickness{0.3mm}
\put(5,50){\line(1,0){50}}
\linethickness{0.3mm}
\put(5,50){\line(0,1){20}}
\linethickness{0.3mm}
\put(30,70){\line(0,1){20}}
\linethickness{0.65mm}
\put(10,35){\line(0,1){15}}
\linethickness{0.3mm}
\put(30,20){\line(0,1){30}}
\linethickness{0.3mm}
\multiput(0,45)(0.12,-0.12){167}{\line(1,0){0.12}}
\linethickness{0.3mm}
\put(0,25){\line(1,0){20}}
\linethickness{0.3mm}
\put(0,25){\line(0,1){20}}
\linethickness{0.3mm}
\put(75,70){\line(1,0){40}}
\linethickness{0.3mm}
\put(115,50){\line(0,1){20}}
\linethickness{0.3mm}
\put(95,50){\line(1,0){20}}
\linethickness{0.3mm}
\multiput(75,70)(0.12,-0.12){167}{\line(1,0){0.12}}
\linethickness{0.3mm}
\put(100,70){\line(0,1){20}}
\linethickness{0.3mm}
\put(10,20){\line(0,1){5}}
\linethickness{0.3mm}
\multiput(75,45)(0.12,-0.12){167}{\line(1,0){0.12}}
\linethickness{0.3mm}
\put(75,25){\line(1,0){20}}
\linethickness{0.3mm}
\put(75,25){\line(0,1){20}}
\linethickness{0.65mm}
\put(85,35){\line(0,1){25}}
\put(10,35){\makebox(0,0)[cc]{\DOTT}}

\put(85,35){\makebox(0,0)[cc]{\DOTT}}

\put(46.25,0){\makebox(0,0)[cr]{$\Aah$}}

\put(107.5,0){\makebox(0,0)[cr]{$\Aah$}}

\put(30,60){\makebox(0,0)[cc]{$\grr$}}

\put(32.5,90){\makebox(0,0)[cl]{$\Bah$}}

\put(102.5,90){\makebox(0,0)[cl]{$\Bah$}}

\put(100,60){\makebox(0,0)[cc]{$\UK$}}

\put(65,60){\makebox(0,0)[cc]{\EQLS}}

\put(5,30){\makebox(0,0)[cc]{$\GRR$}}

\put(80,30){\makebox(0,0)[cc]{$\GRR$}}

\linethickness{0.3mm}
\put(50,0){\line(0,1){50}}
\linethickness{0.3mm}
\put(110,0){\line(0,1){50}}
\linethickness{0.3mm}
\multiput(10,20)(0.12,-0.12){83}{\line(1,0){0.12}}
\linethickness{0.3mm}
\multiput(20,10)(0.12,0.12){83}{\line(1,0){0.12}}
\linethickness{0.3mm}
\put(20,0){\line(0,1){10}}
\linethickness{0.3mm}
\put(85,0){\line(0,1){25}}
\put(82.5,0){\makebox(0,0)[cr]{$\Xah$}}

\put(17.5,0){\makebox(0,0)[cr]{$\Xah$}}

\put(20,10){\makebox(0,0)[cc]{\DOTT}}

\end{picture}

\end{split}
\eeq 
\end{proposition}

\bpr
Let $T\in \tot\UUU(X,\DP)$ be an explanation of the transition on the left in \eqref{eq:fund}. 
\beq
\begin{split}
\newcommand{\DOTT}{\mbox{\Large$\bullet$}}
\newcommand{\Aah}{\scriptscriptstyle A}
\newcommand{\Xah}{\scriptscriptstyle X}
\newcommand{\grr}{t}
\newcommand{\Bah}{\scriptscriptstyle B}
\newcommand{\Grr}{\scriptstyle T}
\newcommand{\UK}{\universal}
\newcommand{\PK}{\prtial}
\def\JPicScale{.45}
%%Created by jPicEdt 1.4.1_03: mixed JPIC-XML/LaTeX format
%%Thu Mar 16 22:24:27 GMT-10:00 2023
%%Begin JPIC-XML
%<?xml version="1.0" standalone="yes"?>
%<jpic x-min="0" x-max="155" y-min="0" y-max="120" auto-bounding="true">
%<multicurve fill-style= "none"
%	 points= "(25,100);(25,100);(75,100);(75,100)"
%	 />
%<multicurve fill-style= "none"
%	 points= "(75,100);(75,100);(75,80);(75,80)"
%	 />
%<multicurve fill-style= "none"
%	 points= "(75,80);(75,80);(25,80);(25,80)"
%	 />
%<multicurve fill-style= "none"
%	 points= "(25,80);(25,80);(25,100);(25,100)"
%	 />
%<multicurve fill-style= "none"
%	 points= "(50,100);(50,100);(50,120);(50,120)"
%	 />
%<multicurve fill-style= "none"
%	 points= "(30,80);(30,80);(30,60);(30,60)"
%	 stroke-width= "0.65"
%	 />
%<multicurve fill-style= "none"
%	 points= "(70,80);(70,80);(70,0);(70,0)"
%	 />
%<multicurve fill-style= "none"
%	 points= "(20,70);(20,70);(40,50);(40,50)"
%	 />
%<multicurve fill-style= "none"
%	 points= "(40,50);(40,50);(20,50);(20,50)"
%	 />
%<multicurve fill-style= "none"
%	 points= "(20,50);(20,50);(0,70);(0,70)"
%	 />
%<multicurve fill-style= "none"
%	 points= "(0,70);(0,70);(20,70);(20,70)"
%	 />
%<multicurve fill-style= "none"
%	 points= "(30,50);(30,50);(30,30);(30,30)"
%	 stroke-width= "0.65"
%	 />
%<multicurve fill-style= "none"
%	 points= "(100,100);(100,100);(155,100);(155,100)"
%	 />
%<multicurve fill-style= "none"
%	 points= "(155,100);(155,100);(155,80);(155,80)"
%	 />
%<multicurve fill-style= "none"
%	 points= "(155,80);(155,80);(120,80);(120,80)"
%	 />
%<multicurve fill-style= "none"
%	 points= "(120,80);(120,80);(100,100);(100,100)"
%	 />
%<multicurve fill-style= "none"
%	 points= "(140,100);(140,100);(140,120);(140,120)"
%	 />
%<multicurve fill-style= "none"
%	 points= "(130,80);(130,80);(130,0);(130,0)"
%	 stroke-width= "0.65"
%	 />
%<multicurve fill-style= "none"
%	 points= "(150,80);(150,80);(150,0);(150,0)"
%	 />
%<multicurve fill-style= "none"
%	 points= "(120,50);(120,50);(100,70);(100,70)"
%	 />
%<multicurve fill-style= "none"
%	 points= "(100,50);(100,50);(120,50);(120,50)"
%	 />
%<multicurve fill-style= "none"
%	 points= "(100,70);(100,70);(100,50);(100,50)"
%	 />
%<multicurve fill-style= "none"
%	 points= "(110,90);(110,90);(110,60);(110,60)"
%	 stroke-width= "0.65"
%	 />
%<multicurve fill-style= "none"
%	 points= "(10,60);(10,60);(10,50);(10,50)"
%	 stroke-width= "0.65"
%	 />
%<text fill-style= "none"
%	 text-vert-align= "center-v"
%	 anchor-point= "(30,30)"
%	 text-frame= "noframe"
%	 text-hor-align= "center-h"
%	 >
%\DOTT
%</text>
%<text fill-style= "none"
%	 text-vert-align= "center-v"
%	 anchor-point= "(30,60)"
%	 text-frame= "noframe"
%	 text-hor-align= "center-h"
%	 >
%\DOTT
%</text>
%<text fill-style= "none"
%	 text-vert-align= "center-v"
%	 anchor-point= "(110,60)"
%	 text-frame= "noframe"
%	 text-hor-align= "center-h"
%	 >
%\DOTT
%</text>
%<text fill-style= "none"
%	 text-vert-align= "center-v"
%	 anchor-point= "(67.5,0)"
%	 text-frame= "noframe"
%	 text-hor-align= "right"
%	 >
%$\Aah$
%</text>
%<text fill-style= "none"
%	 text-vert-align= "center-v"
%	 anchor-point= "(147.5,0)"
%	 text-frame= "noframe"
%	 text-hor-align= "right"
%	 >
%$\Aah$
%</text>
%<text fill-style= "none"
%	 text-vert-align= "center-v"
%	 anchor-point= "(20,60)"
%	 text-frame= "noframe"
%	 text-hor-align= "center-h"
%	 >
%$\PK$
%</text>
%<text fill-style= "none"
%	 text-vert-align= "center-v"
%	 anchor-point= "(50,90)"
%	 text-frame= "noframe"
%	 text-hor-align= "center-h"
%	 >
%$\grr$
%</text>
%<text fill-style= "none"
%	 text-vert-align= "center-v"
%	 anchor-point= "(52.5,120)"
%	 text-frame= "noframe"
%	 text-hor-align= "left"
%	 >
%$\Bah$
%</text>
%<text fill-style= "none"
%	 text-vert-align= "center-v"
%	 anchor-point= "(142.5,120)"
%	 text-frame= "noframe"
%	 text-hor-align= "left"
%	 >
%$\Bah$
%</text>
%<text fill-style= "none"
%	 text-vert-align= "center-v"
%	 anchor-point= "(140,90)"
%	 text-frame= "noframe"
%	 text-hor-align= "center-h"
%	 >
%$\UK$
%</text>
%<text fill-style= "none"
%	 text-vert-align= "center-v"
%	 anchor-point= "(90,90)"
%	 text-frame= "noframe"
%	 text-hor-align= "center-h"
%	 >
%\EQLS
%</text>
%<text fill-style= "none"
%	 text-vert-align= "center-v"
%	 anchor-point= "(105,55)"
%	 text-frame= "noframe"
%	 text-hor-align= "center-h"
%	 >
%$\Grr$
%</text>
%<multicurve fill-style= "none"
%	 points= "(50,80);(50,80);(50,30);(50,30)"
%	 />
%<multicurve fill-style= "none"
%	 points= "(50,30);(50,30);(30,10);(30,10)"
%	 />
%<multicurve fill-style= "none"
%	 points= "(30,10);(30,10);(30,0);(30,0)"
%	 />
%<multicurve fill-style= "none"
%	 points= "(50,10);(50,10);(50,0);(50,0)"
%	 stroke-width= "0.65"
%	 />
%<multicurve stroke-overstrike= "true"
%	 fill-style= "none"
%	 points= "(10,50);(10,50);(50,10);(50,10)"
%	 stroke-width= "0.65"
%	 stroke-overstrike-width= "0.5"
%	 />
%<multicurve fill-style= "none"
%	 points= "(110,50);(110,50);(110,0);(110,0)"
%	 />
%<text fill-style= "none"
%	 text-vert-align= "center-v"
%	 anchor-point= "(107.5,0)"
%	 text-frame= "noframe"
%	 text-hor-align= "right"
%	 >
%$\Xah$
%</text>
%<text fill-style= "none"
%	 text-vert-align= "center-v"
%	 anchor-point= "(27.5,0)"
%	 text-frame= "noframe"
%	 text-hor-align= "right"
%	 >
%$\Xah$
%</text>
%</jpic>
%%End JPIC-XML
%LaTeX-picture environment using emulated lines and arcs
%You can rescale the whole picture (to 80% for instance) by using the command \def\JPicScale{0.8}
\ifx\JPicScale\undefined\def\JPicScale{1}\fi
\unitlength \JPicScale mm
\begin{picture}(155,120)(0,0)
\linethickness{0.3mm}
\put(25,100){\line(1,0){50}}
\linethickness{0.3mm}
\put(75,80){\line(0,1){20}}
\linethickness{0.3mm}
\put(25,80){\line(1,0){50}}
\linethickness{0.3mm}
\put(25,80){\line(0,1){20}}
\linethickness{0.3mm}
\put(50,100){\line(0,1){20}}
\linethickness{0.65mm}
\put(30,60){\line(0,1){20}}
\linethickness{0.3mm}
\put(70,0){\line(0,1){80}}
\linethickness{0.3mm}
\multiput(20,70)(0.12,-0.12){167}{\line(1,0){0.12}}
\linethickness{0.3mm}
\put(20,50){\line(1,0){20}}
\linethickness{0.3mm}
\multiput(0,70)(0.12,-0.12){167}{\line(1,0){0.12}}
\linethickness{0.3mm}
\put(0,70){\line(1,0){20}}
\linethickness{0.65mm}
\put(30,30){\line(0,1){20}}
\linethickness{0.3mm}
\put(100,100){\line(1,0){55}}
\linethickness{0.3mm}
\put(155,80){\line(0,1){20}}
\linethickness{0.3mm}
\put(120,80){\line(1,0){35}}
\linethickness{0.3mm}
\multiput(100,100)(0.12,-0.12){167}{\line(1,0){0.12}}
\linethickness{0.3mm}
\put(140,100){\line(0,1){20}}
\linethickness{0.65mm}
\put(130,0){\line(0,1){80}}
\linethickness{0.3mm}
\put(150,0){\line(0,1){80}}
\linethickness{0.3mm}
\multiput(100,70)(0.12,-0.12){167}{\line(1,0){0.12}}
\linethickness{0.3mm}
\put(100,50){\line(1,0){20}}
\linethickness{0.3mm}
\put(100,50){\line(0,1){20}}
\linethickness{0.65mm}
\put(110,60){\line(0,1){30}}
\linethickness{0.65mm}
\put(10,50){\line(0,1){10}}
\put(30,30){\makebox(0,0)[cc]{\DOTT}}

\put(30,60){\makebox(0,0)[cc]{\DOTT}}

\put(110,60){\makebox(0,0)[cc]{\DOTT}}

\put(67.5,0){\makebox(0,0)[cr]{$\Aah$}}

\put(147.5,0){\makebox(0,0)[cr]{$\Aah$}}

\put(20,60){\makebox(0,0)[cc]{$\PK$}}

\put(50,90){\makebox(0,0)[cc]{$\grr$}}

\put(52.5,120){\makebox(0,0)[cl]{$\Bah$}}

\put(142.5,120){\makebox(0,0)[cl]{$\Bah$}}

\put(140,90){\makebox(0,0)[cc]{$\UK$}}

\put(90,90){\makebox(0,0)[cc]{\EQLS}}

\put(105,55){\makebox(0,0)[cc]{$\Grr$}}

\linethickness{0.3mm}
\put(50,30){\line(0,1){50}}
\linethickness{0.3mm}
\multiput(30,10)(0.12,0.12){167}{\line(1,0){0.12}}
\linethickness{0.3mm}
\put(30,0){\line(0,1){10}}
\linethickness{0.65mm}
\put(50,0){\line(0,1){10}}
\linethickness{0.65mm}
\multiput(10,50)(0.12,-0.12){333}{\line(1,0){0.12}}
\linethickness{0.3mm}
\put(110,0){\line(0,1){50}}
\put(107.5,0){\makebox(0,0)[cr]{$\Xah$}}

\put(27.5,0){\makebox(0,0)[cr]{$\Xah$}}

\end{picture}

\end{split}
\eeq
$H$ exists by Def.~\ref{Def:interpreter}. Then $\enco t_{x} =\pev{Tx}$ is self-confirming, because
\beq
\begin{split}
\newcommand{\DOTT}{\mbox{\Large$\bullet$}}
\newcommand{\Aah}{\scriptscriptstyle A}
\newcommand{\Xah}{\scriptscriptstyle X}
\newcommand{\grr}{t}
\newcommand{\Bah}{\scriptscriptstyle B}
\newcommand{\Grr}{\scriptstyle T}
\newcommand{\Psee}{\enco t}
\newcommand{\UK}{\universal}
\newcommand{\PK}{\prtial}
\def\JPicScale{.45}
%%Created by jPicEdt 1.4.1_03: mixed JPIC-XML/LaTeX format
%%Thu Mar 16 23:04:27 GMT-10:00 2023
%%Begin JPIC-XML
%<?xml version="1.0" standalone="yes"?>
%<jpic x-min="-0" x-max="320" y-min="0" y-max="140" auto-bounding="true">
%<multicurve fill-style= "none"
%	 points= "(30,120);(30,120);(80,120);(80,120)"
%	 />
%<multicurve fill-style= "none"
%	 points= "(80,120);(80,120);(80,100);(80,100)"
%	 />
%<multicurve fill-style= "none"
%	 points= "(80,100);(80,100);(30,100);(30,100)"
%	 />
%<multicurve fill-style= "none"
%	 points= "(30,100);(30,100);(30,120);(30,120)"
%	 />
%<multicurve fill-style= "none"
%	 points= "(55,120);(55,120);(55,140);(55,140)"
%	 />
%<multicurve fill-style= "none"
%	 points= "(35,100);(35,100);(35,80);(35,80)"
%	 stroke-width= "0.65"
%	 />
%<multicurve fill-style= "none"
%	 points= "(75,100);(75,100);(75,0);(75,0)"
%	 />
%<multicurve fill-style= "none"
%	 points= "(25,90);(25,90);(45,70);(45,70)"
%	 />
%<multicurve fill-style= "none"
%	 points= "(45,70);(45,70);(25,70);(25,70)"
%	 />
%<multicurve fill-style= "none"
%	 points= "(25,70);(25,70);(5,90);(5,90)"
%	 />
%<multicurve fill-style= "none"
%	 points= "(5,90);(5,90);(25,90);(25,90)"
%	 />
%<multicurve fill-style= "none"
%	 points= "(35,70);(35,70);(35,45);(35,45)"
%	 stroke-width= "0.65"
%	 />
%<multicurve fill-style= "none"
%	 points= "(105,120);(105,120);(160,120);(160,120)"
%	 />
%<multicurve fill-style= "none"
%	 points= "(160,120);(160,120);(160,100);(160,100)"
%	 />
%<multicurve fill-style= "none"
%	 points= "(160,100);(160,100);(125,100);(125,100)"
%	 />
%<multicurve fill-style= "none"
%	 points= "(125,100);(125,100);(105,120);(105,120)"
%	 />
%<multicurve fill-style= "none"
%	 points= "(145,120);(145,120);(145,140);(145,140)"
%	 />
%<multicurve fill-style= "none"
%	 points= "(135,100);(135,100);(135,45);(135,45)"
%	 stroke-width= "0.65"
%	 />
%<multicurve fill-style= "none"
%	 points= "(155,100);(155,100);(155,0);(155,0)"
%	 />
%<multicurve fill-style= "none"
%	 points= "(125,70);(125,70);(105,90);(105,90)"
%	 />
%<multicurve fill-style= "none"
%	 points= "(105,70);(105,70);(125,70);(125,70)"
%	 />
%<multicurve fill-style= "none"
%	 points= "(105,90);(105,90);(105,70);(105,70)"
%	 />
%<multicurve fill-style= "none"
%	 points= "(115,110);(115,110);(115,80);(115,80)"
%	 stroke-width= "0.65"
%	 />
%<multicurve fill-style= "none"
%	 points= "(15,80);(15,80);(15,72.5);(15,72.5)"
%	 stroke-width= "0.65"
%	 />
%<text fill-style= "none"
%	 text-vert-align= "center-v"
%	 anchor-point= "(35,52.5)"
%	 text-frame= "noframe"
%	 text-hor-align= "center-h"
%	 >
%\DOTT
%</text>
%<text fill-style= "none"
%	 text-vert-align= "center-v"
%	 anchor-point= "(35,80)"
%	 text-frame= "noframe"
%	 text-hor-align= "center-h"
%	 >
%\DOTT
%</text>
%<text fill-style= "none"
%	 text-vert-align= "center-v"
%	 anchor-point= "(115,80)"
%	 text-frame= "noframe"
%	 text-hor-align= "center-h"
%	 >
%\DOTT
%</text>
%<text fill-style= "none"
%	 text-vert-align= "center-v"
%	 anchor-point= "(72.5,0)"
%	 text-frame= "noframe"
%	 text-hor-align= "right"
%	 >
%$\Aah$
%</text>
%<text fill-style= "none"
%	 text-vert-align= "center-v"
%	 anchor-point= "(152.5,0)"
%	 text-frame= "noframe"
%	 text-hor-align= "right"
%	 >
%$\Aah$
%</text>
%<text fill-style= "none"
%	 text-vert-align= "center-v"
%	 anchor-point= "(25,80)"
%	 text-frame= "noframe"
%	 text-hor-align= "center-h"
%	 >
%$\PK$
%</text>
%<text fill-style= "none"
%	 text-vert-align= "center-v"
%	 anchor-point= "(55,110)"
%	 text-frame= "noframe"
%	 text-hor-align= "center-h"
%	 >
%$\grr$
%</text>
%<text fill-style= "none"
%	 text-vert-align= "center-v"
%	 anchor-point= "(57.5,140)"
%	 text-frame= "noframe"
%	 text-hor-align= "left"
%	 >
%$\Bah$
%</text>
%<text fill-style= "none"
%	 text-vert-align= "center-v"
%	 anchor-point= "(147.5,140)"
%	 text-frame= "noframe"
%	 text-hor-align= "left"
%	 >
%$\Bah$
%</text>
%<text fill-style= "none"
%	 text-vert-align= "center-v"
%	 anchor-point= "(145,110)"
%	 text-frame= "noframe"
%	 text-hor-align= "center-h"
%	 >
%$\UK$
%</text>
%<text fill-style= "none"
%	 text-vert-align= "center-v"
%	 anchor-point= "(95,110)"
%	 text-frame= "noframe"
%	 text-hor-align= "center-h"
%	 >
%\EQLS
%</text>
%<text fill-style= "none"
%	 text-vert-align= "center-v"
%	 anchor-point= "(110,75)"
%	 text-frame= "noframe"
%	 text-hor-align= "center-h"
%	 >
%$\Grr$
%</text>
%<multicurve fill-style= "none"
%	 points= "(55,100);(55,100);(55,30);(55,30)"
%	 />
%<multicurve fill-style= "none"
%	 points= "(35,35);(35,35);(35,0);(35,0)"
%	 />
%<multicurve stroke-overstrike= "true"
%	 fill-style= "none"
%	 points= "(15,72.5);(15,72.5);(35,52.5);(35,52.5)"
%	 stroke-width= "0.65"
%	 stroke-overstrike-width= "0.5"
%	 />
%<multicurve fill-style= "none"
%	 points= "(115,70);(115,70);(115,0);(115,0)"
%	 />
%<text fill-style= "none"
%	 text-vert-align= "center-v"
%	 anchor-point= "(112.5,0)"
%	 text-frame= "noframe"
%	 text-hor-align= "right"
%	 >
%$\Xah$
%</text>
%<text fill-style= "none"
%	 text-vert-align= "center-v"
%	 anchor-point= "(32.5,0)"
%	 text-frame= "noframe"
%	 text-hor-align= "right"
%	 >
%$\Xah$
%</text>
%<multicurve fill-style= "none"
%	 points= "(25,95);(25,95);(-0,95);(-0,95)"
%	 />
%<multicurve fill-style= "none"
%	 points= "(25,95);(25,95);(50,70);(50,70)"
%	 />
%<multicurve fill-style= "none"
%	 points= "(50,70);(50,70);(50,30);(50,30)"
%	 />
%<multicurve fill-style= "none"
%	 points= "(50,30);(50,30);(0,30);(0,30)"
%	 />
%<multicurve fill-style= "none"
%	 points= "(45,35);(45,35);(25,55);(25,55)"
%	 />
%<multicurve fill-style= "none"
%	 points= "(25,35);(25,35);(45,35);(45,35)"
%	 />
%<multicurve fill-style= "none"
%	 points= "(25,55);(25,55);(25,35);(25,35)"
%	 />
%<text fill-style= "none"
%	 text-vert-align= "center-v"
%	 anchor-point= "(35,45)"
%	 text-frame= "noframe"
%	 text-hor-align= "center-h"
%	 >
%\DOTT
%</text>
%<text fill-style= "none"
%	 text-vert-align= "center-v"
%	 anchor-point= "(30,40)"
%	 text-frame= "noframe"
%	 text-hor-align= "center-h"
%	 >
%$\Grr$
%</text>
%<multicurve fill-style= "none"
%	 points= "(55,30);(55,30);(35,10);(35,10)"
%	 />
%<text fill-style= "none"
%	 text-vert-align= "center-v"
%	 anchor-point= "(35,10)"
%	 text-frame= "noframe"
%	 text-hor-align= "center-h"
%	 >
%\DOTT
%</text>
%<multicurve fill-style= "none"
%	 points= "(0,95);(0,95);(0,30);(0,30)"
%	 />
%<multicurve fill-style= "none"
%	 points= "(145,35);(145,35);(125,55);(125,55)"
%	 />
%<multicurve fill-style= "none"
%	 points= "(125,35);(125,35);(145,35);(145,35)"
%	 />
%<multicurve fill-style= "none"
%	 points= "(125,55);(125,55);(125,35);(125,35)"
%	 />
%<text fill-style= "none"
%	 text-vert-align= "center-v"
%	 anchor-point= "(135,45)"
%	 text-frame= "noframe"
%	 text-hor-align= "center-h"
%	 >
%\DOTT
%</text>
%<text fill-style= "none"
%	 text-vert-align= "center-v"
%	 anchor-point= "(130,40)"
%	 text-frame= "noframe"
%	 text-hor-align= "center-h"
%	 >
%$\Grr$
%</text>
%<multicurve fill-style= "none"
%	 points= "(135,30);(135,30);(115,10);(115,10)"
%	 />
%<multicurve fill-style= "none"
%	 points= "(135,35);(135,35);(135,30);(135,30)"
%	 />
%<multicurve fill-style= "none"
%	 points= "(185,120);(185,120);(240,120);(240,120)"
%	 />
%<multicurve fill-style= "none"
%	 points= "(240,120);(240,120);(240,100);(240,100)"
%	 />
%<multicurve fill-style= "none"
%	 points= "(240,100);(240,100);(205,100);(205,100)"
%	 />
%<multicurve fill-style= "none"
%	 points= "(205,100);(205,100);(185,120);(185,120)"
%	 />
%<multicurve fill-style= "none"
%	 points= "(225,120);(225,120);(225,140);(225,140)"
%	 />
%<multicurve fill-style= "none"
%	 points= "(215,100);(215,100);(215,45);(215,45)"
%	 stroke-width= "0.65"
%	 />
%<multicurve fill-style= "none"
%	 points= "(235,100);(235,100);(235,0);(235,0)"
%	 />
%<text fill-style= "none"
%	 text-vert-align= "center-v"
%	 anchor-point= "(232.5,0)"
%	 text-frame= "noframe"
%	 text-hor-align= "right"
%	 >
%$\Aah$
%</text>
%<text fill-style= "none"
%	 text-vert-align= "center-v"
%	 anchor-point= "(227.5,140)"
%	 text-frame= "noframe"
%	 text-hor-align= "left"
%	 >
%$\Bah$
%</text>
%<text fill-style= "none"
%	 text-vert-align= "center-v"
%	 anchor-point= "(225,110)"
%	 text-frame= "noframe"
%	 text-hor-align= "center-h"
%	 >
%$\UK$
%</text>
%<text fill-style= "none"
%	 text-vert-align= "center-v"
%	 anchor-point= "(175,110)"
%	 text-frame= "noframe"
%	 text-hor-align= "center-h"
%	 >
%\EQLS
%</text>
%<text fill-style= "none"
%	 text-vert-align= "center-v"
%	 anchor-point= "(212.5,0)"
%	 text-frame= "noframe"
%	 text-hor-align= "right"
%	 >
%$\Xah$
%</text>
%<multicurve fill-style= "none"
%	 points= "(225,35);(225,35);(205,55);(205,55)"
%	 />
%<multicurve fill-style= "none"
%	 points= "(205,35);(205,35);(225,35);(225,35)"
%	 />
%<multicurve fill-style= "none"
%	 points= "(205,55);(205,55);(205,35);(205,35)"
%	 />
%<text fill-style= "none"
%	 text-vert-align= "center-v"
%	 anchor-point= "(215,45)"
%	 text-frame= "noframe"
%	 text-hor-align= "center-h"
%	 >
%\DOTT
%</text>
%<text fill-style= "none"
%	 text-vert-align= "center-v"
%	 anchor-point= "(210,40)"
%	 text-frame= "noframe"
%	 text-hor-align= "center-h"
%	 >
%$\Grr$
%</text>
%<multicurve fill-style= "none"
%	 points= "(215,35);(215,35);(215,0);(215,0)"
%	 />
%<multicurve fill-style= "none"
%	 points= "(285,120);(285,120);(320,120);(320,120)"
%	 />
%<multicurve fill-style= "none"
%	 points= "(320,120);(320,120);(320,100);(320,100)"
%	 />
%<multicurve fill-style= "none"
%	 points= "(320,100);(320,100);(305,100);(305,100)"
%	 />
%<multicurve fill-style= "none"
%	 points= "(305,100);(305,100);(285,120);(285,120)"
%	 />
%<multicurve fill-style= "none"
%	 points= "(305,120);(305,120);(305,140);(305,140)"
%	 />
%<multicurve fill-style= "none"
%	 points= "(315,100);(315,100);(315,0);(315,0)"
%	 />
%<text fill-style= "none"
%	 text-vert-align= "center-v"
%	 anchor-point= "(312.5,0)"
%	 text-frame= "noframe"
%	 text-hor-align= "right"
%	 >
%$\Aah$
%</text>
%<text fill-style= "none"
%	 text-vert-align= "center-v"
%	 anchor-point= "(307.5,140)"
%	 text-frame= "noframe"
%	 text-hor-align= "left"
%	 >
%$\Bah$
%</text>
%<text fill-style= "none"
%	 text-vert-align= "center-v"
%	 anchor-point= "(310,110)"
%	 text-frame= "noframe"
%	 text-hor-align= "center-h"
%	 >
%$\UK$
%</text>
%<text fill-style= "none"
%	 text-vert-align= "center-v"
%	 anchor-point= "(255,110)"
%	 text-frame= "noframe"
%	 text-hor-align= "center-h"
%	 >
%\EQLS
%</text>
%<multicurve fill-style= "none"
%	 points= "(295,110);(295,110);(295,80);(295,80)"
%	 stroke-width= "0.65"
%	 />
%<multicurve fill-style= "none"
%	 points= "(285,90);(285,90);(305,70);(305,70)"
%	 />
%<multicurve fill-style= "none"
%	 points= "(305,70);(305,70);(285,70);(285,70)"
%	 />
%<multicurve fill-style= "none"
%	 points= "(285,70);(285,70);(265,90);(265,90)"
%	 />
%<multicurve fill-style= "none"
%	 points= "(265,90);(265,90);(285,90);(285,90)"
%	 />
%<multicurve fill-style= "none"
%	 points= "(295,70);(295,70);(295,45);(295,45)"
%	 stroke-width= "0.65"
%	 />
%<multicurve fill-style= "none"
%	 points= "(275,80);(275,80);(275,72.5);(275,72.5)"
%	 stroke-width= "0.65"
%	 />
%<text fill-style= "none"
%	 text-vert-align= "center-v"
%	 anchor-point= "(295,52.5)"
%	 text-frame= "noframe"
%	 text-hor-align= "center-h"
%	 >
%\DOTT
%</text>
%<text fill-style= "none"
%	 text-vert-align= "center-v"
%	 anchor-point= "(295,80)"
%	 text-frame= "noframe"
%	 text-hor-align= "center-h"
%	 >
%\DOTT
%</text>
%<text fill-style= "none"
%	 text-vert-align= "center-v"
%	 anchor-point= "(285,80)"
%	 text-frame= "noframe"
%	 text-hor-align= "center-h"
%	 >
%$\PK$
%</text>
%<multicurve fill-style= "none"
%	 points= "(295,35);(295,35);(295,0);(295,0)"
%	 />
%<multicurve stroke-overstrike= "true"
%	 fill-style= "none"
%	 points= "(275,72.5);(275,72.5);(295,52.5);(295,52.5)"
%	 stroke-width= "0.65"
%	 stroke-overstrike-width= "0.5"
%	 />
%<text fill-style= "none"
%	 text-vert-align= "center-v"
%	 anchor-point= "(292.5,0)"
%	 text-frame= "noframe"
%	 text-hor-align= "right"
%	 >
%$\Xah$
%</text>
%<multicurve fill-style= "none"
%	 points= "(285,95);(285,95);(260,95);(260,95)"
%	 />
%<multicurve fill-style= "none"
%	 points= "(285,95);(285,95);(310,70);(310,70)"
%	 />
%<multicurve fill-style= "none"
%	 points= "(310,70);(310,70);(310,30);(310,30)"
%	 />
%<multicurve fill-style= "none"
%	 points= "(310,30);(310,30);(260,30);(260,30)"
%	 />
%<multicurve fill-style= "none"
%	 points= "(305,35);(305,35);(285,55);(285,55)"
%	 />
%<multicurve fill-style= "none"
%	 points= "(285,35);(285,35);(305,35);(305,35)"
%	 />
%<multicurve fill-style= "none"
%	 points= "(285,55);(285,55);(285,35);(285,35)"
%	 />
%<text fill-style= "none"
%	 text-vert-align= "center-v"
%	 anchor-point= "(295,45)"
%	 text-frame= "noframe"
%	 text-hor-align= "center-h"
%	 >
%\DOTT
%</text>
%<text fill-style= "none"
%	 text-vert-align= "center-v"
%	 anchor-point= "(290,40)"
%	 text-frame= "noframe"
%	 text-hor-align= "center-h"
%	 >
%$\Grr$
%</text>
%<multicurve fill-style= "none"
%	 points= "(260,95);(260,95);(260,30);(260,30)"
%	 />
%<multicurve fill-style= "none"
%	 points= "(195,110);(195,110);(195,72.5);(195,72.5)"
%	 stroke-width= "0.65"
%	 />
%<text fill-style= "none"
%	 text-vert-align= "center-v"
%	 anchor-point= "(215,52.5)"
%	 text-frame= "noframe"
%	 text-hor-align= "center-h"
%	 >
%\DOTT
%</text>
%<multicurve stroke-overstrike= "true"
%	 fill-style= "none"
%	 points= "(195,72.5);(195,72.5);(215,52.5);(215,52.5)"
%	 stroke-width= "0.65"
%	 stroke-overstrike-width= "0.5"
%	 />
%<text fill-style= "none"
%	 text-vert-align= "center-v"
%	 anchor-point= "(10,50)"
%	 text-frame= "noframe"
%	 text-hor-align= "center-h"
%	 >
%$\Psee$
%</text>
%<text fill-style= "none"
%	 text-vert-align= "center-v"
%	 anchor-point= "(270,50)"
%	 text-frame= "noframe"
%	 text-hor-align= "center-h"
%	 >
%$\Psee$
%</text>
%<text fill-style= "none"
%	 text-vert-align= "center-v"
%	 anchor-point= "(115,10)"
%	 text-frame= "noframe"
%	 text-hor-align= "center-h"
%	 >
%\DOTT
%</text>
%</jpic>
%%End JPIC-XML
%LaTeX-picture environment using emulated lines and arcs
%You can rescale the whole picture (to 80% for instance) by using the command \def\JPicScale{0.8}
\ifx\JPicScale\undefined\def\JPicScale{1}\fi
\unitlength \JPicScale mm
\begin{picture}(320,140)(0,0)
\linethickness{0.3mm}
\put(30,120){\line(1,0){50}}
\linethickness{0.3mm}
\put(80,100){\line(0,1){20}}
\linethickness{0.3mm}
\put(30,100){\line(1,0){50}}
\linethickness{0.3mm}
\put(30,100){\line(0,1){20}}
\linethickness{0.3mm}
\put(55,120){\line(0,1){20}}
\linethickness{0.65mm}
\put(35,80){\line(0,1){20}}
\linethickness{0.3mm}
\put(75,0){\line(0,1){100}}
\linethickness{0.3mm}
\multiput(25,90)(0.12,-0.12){167}{\line(1,0){0.12}}
\linethickness{0.3mm}
\put(25,70){\line(1,0){20}}
\linethickness{0.3mm}
\multiput(5,90)(0.12,-0.12){167}{\line(1,0){0.12}}
\linethickness{0.3mm}
\put(5,90){\line(1,0){20}}
\linethickness{0.65mm}
\put(35,45){\line(0,1){25}}
\linethickness{0.3mm}
\put(105,120){\line(1,0){55}}
\linethickness{0.3mm}
\put(160,100){\line(0,1){20}}
\linethickness{0.3mm}
\put(125,100){\line(1,0){35}}
\linethickness{0.3mm}
\multiput(105,120)(0.12,-0.12){167}{\line(1,0){0.12}}
\linethickness{0.3mm}
\put(145,120){\line(0,1){20}}
\linethickness{0.65mm}
\put(135,45){\line(0,1){55}}
\linethickness{0.3mm}
\put(155,0){\line(0,1){100}}
\linethickness{0.3mm}
\multiput(105,90)(0.12,-0.12){167}{\line(1,0){0.12}}
\linethickness{0.3mm}
\put(105,70){\line(1,0){20}}
\linethickness{0.3mm}
\put(105,70){\line(0,1){20}}
\linethickness{0.65mm}
\put(115,80){\line(0,1){30}}
\linethickness{0.65mm}
\put(15,72.5){\line(0,1){7.5}}
\put(35,52.5){\makebox(0,0)[cc]{\DOTT}}

\put(35,80){\makebox(0,0)[cc]{\DOTT}}

\put(115,80){\makebox(0,0)[cc]{\DOTT}}

\put(72.5,0){\makebox(0,0)[cr]{$\Aah$}}

\put(152.5,0){\makebox(0,0)[cr]{$\Aah$}}

\put(25,80){\makebox(0,0)[cc]{$\PK$}}

\put(55,110){\makebox(0,0)[cc]{$\grr$}}

\put(57.5,140){\makebox(0,0)[cl]{$\Bah$}}

\put(147.5,140){\makebox(0,0)[cl]{$\Bah$}}

\put(145,110){\makebox(0,0)[cc]{$\UK$}}

\put(95,110){\makebox(0,0)[cc]{\EQLS}}

\put(110,75){\makebox(0,0)[cc]{$\Grr$}}

\linethickness{0.3mm}
\put(55,30){\line(0,1){70}}
\linethickness{0.3mm}
\put(35,0){\line(0,1){35}}
\linethickness{0.65mm}
\multiput(15,72.5)(0.12,-0.12){167}{\line(1,0){0.12}}
\linethickness{0.3mm}
\put(115,0){\line(0,1){70}}
\put(112.5,0){\makebox(0,0)[cr]{$\Xah$}}

\put(32.5,0){\makebox(0,0)[cr]{$\Xah$}}

\linethickness{0.3mm}
\put(-0,95){\line(1,0){25}}
\linethickness{0.3mm}
\multiput(25,95)(0.12,-0.12){208}{\line(1,0){0.12}}
\linethickness{0.3mm}
\put(50,30){\line(0,1){40}}
\linethickness{0.3mm}
\put(0,30){\line(1,0){50}}
\linethickness{0.3mm}
\multiput(25,55)(0.12,-0.12){167}{\line(1,0){0.12}}
\linethickness{0.3mm}
\put(25,35){\line(1,0){20}}
\linethickness{0.3mm}
\put(25,35){\line(0,1){20}}
\put(35,45){\makebox(0,0)[cc]{\DOTT}}

\put(30,40){\makebox(0,0)[cc]{$\Grr$}}

\linethickness{0.3mm}
\multiput(35,10)(0.12,0.12){167}{\line(1,0){0.12}}
\put(35,10){\makebox(0,0)[cc]{\DOTT}}

\linethickness{0.3mm}
\put(0,30){\line(0,1){65}}
\linethickness{0.3mm}
\multiput(125,55)(0.12,-0.12){167}{\line(1,0){0.12}}
\linethickness{0.3mm}
\put(125,35){\line(1,0){20}}
\linethickness{0.3mm}
\put(125,35){\line(0,1){20}}
\put(135,45){\makebox(0,0)[cc]{\DOTT}}

\put(130,40){\makebox(0,0)[cc]{$\Grr$}}

\linethickness{0.3mm}
\multiput(115,10)(0.12,0.12){167}{\line(1,0){0.12}}
\linethickness{0.3mm}
\put(135,30){\line(0,1){5}}
\linethickness{0.3mm}
\put(185,120){\line(1,0){55}}
\linethickness{0.3mm}
\put(240,100){\line(0,1){20}}
\linethickness{0.3mm}
\put(205,100){\line(1,0){35}}
\linethickness{0.3mm}
\multiput(185,120)(0.12,-0.12){167}{\line(1,0){0.12}}
\linethickness{0.3mm}
\put(225,120){\line(0,1){20}}
\linethickness{0.65mm}
\put(215,45){\line(0,1){55}}
\linethickness{0.3mm}
\put(235,0){\line(0,1){100}}
\put(232.5,0){\makebox(0,0)[cr]{$\Aah$}}

\put(227.5,140){\makebox(0,0)[cl]{$\Bah$}}

\put(225,110){\makebox(0,0)[cc]{$\UK$}}

\put(175,110){\makebox(0,0)[cc]{\EQLS}}

\put(212.5,0){\makebox(0,0)[cr]{$\Xah$}}

\linethickness{0.3mm}
\multiput(205,55)(0.12,-0.12){167}{\line(1,0){0.12}}
\linethickness{0.3mm}
\put(205,35){\line(1,0){20}}
\linethickness{0.3mm}
\put(205,35){\line(0,1){20}}
\put(215,45){\makebox(0,0)[cc]{\DOTT}}

\put(210,40){\makebox(0,0)[cc]{$\Grr$}}

\linethickness{0.3mm}
\put(215,0){\line(0,1){35}}
\linethickness{0.3mm}
\put(285,120){\line(1,0){35}}
\linethickness{0.3mm}
\put(320,100){\line(0,1){20}}
\linethickness{0.3mm}
\put(305,100){\line(1,0){15}}
\linethickness{0.3mm}
\multiput(285,120)(0.12,-0.12){167}{\line(1,0){0.12}}
\linethickness{0.3mm}
\put(305,120){\line(0,1){20}}
\linethickness{0.3mm}
\put(315,0){\line(0,1){100}}
\put(312.5,0){\makebox(0,0)[cr]{$\Aah$}}

\put(307.5,140){\makebox(0,0)[cl]{$\Bah$}}

\put(310,110){\makebox(0,0)[cc]{$\UK$}}

\put(255,110){\makebox(0,0)[cc]{\EQLS}}

\linethickness{0.65mm}
\put(295,80){\line(0,1){30}}
\linethickness{0.3mm}
\multiput(285,90)(0.12,-0.12){167}{\line(1,0){0.12}}
\linethickness{0.3mm}
\put(285,70){\line(1,0){20}}
\linethickness{0.3mm}
\multiput(265,90)(0.12,-0.12){167}{\line(1,0){0.12}}
\linethickness{0.3mm}
\put(265,90){\line(1,0){20}}
\linethickness{0.65mm}
\put(295,45){\line(0,1){25}}
\linethickness{0.65mm}
\put(275,72.5){\line(0,1){7.5}}
\put(295,52.5){\makebox(0,0)[cc]{\DOTT}}

\put(295,80){\makebox(0,0)[cc]{\DOTT}}

\put(285,80){\makebox(0,0)[cc]{$\PK$}}

\linethickness{0.3mm}
\put(295,0){\line(0,1){35}}
\linethickness{0.65mm}
\multiput(275,72.5)(0.12,-0.12){167}{\line(1,0){0.12}}
\put(292.5,0){\makebox(0,0)[cr]{$\Xah$}}

\linethickness{0.3mm}
\put(260,95){\line(1,0){25}}
\linethickness{0.3mm}
\multiput(285,95)(0.12,-0.12){208}{\line(1,0){0.12}}
\linethickness{0.3mm}
\put(310,30){\line(0,1){40}}
\linethickness{0.3mm}
\put(260,30){\line(1,0){50}}
\linethickness{0.3mm}
\multiput(285,55)(0.12,-0.12){167}{\line(1,0){0.12}}
\linethickness{0.3mm}
\put(285,35){\line(1,0){20}}
\linethickness{0.3mm}
\put(285,35){\line(0,1){20}}
\put(295,45){\makebox(0,0)[cc]{\DOTT}}

\put(290,40){\makebox(0,0)[cc]{$\Grr$}}

\linethickness{0.3mm}
\put(260,30){\line(0,1){65}}
\linethickness{0.65mm}
\put(195,72.5){\line(0,1){37.5}}
\put(215,52.5){\makebox(0,0)[cc]{\DOTT}}

\linethickness{0.65mm}
\multiput(195,72.5)(0.12,-0.12){167}{\line(1,0){0.12}}
\put(10,50){\makebox(0,0)[cc]{$\Psee$}}

\put(270,50){\makebox(0,0)[cc]{$\Psee$}}

\put(115,10){\makebox(0,0)[cc]{\DOTT}}

\end{picture}

\end{split}
\eeq
\epr

\section{From natural science to artificial theology}\label{Sec:Outro}
% !TEX root = 00-wollic.tex

\subsection{What did we learn?}
We sketched the category $\UUU$ of \emph{belief states} $A, B,\ldots$, presented as theories with standard models.  A belief transition $f\colon A\to B$ updates the belief state by mapping the observables supporting $A$ to observables supporting $B$. Such morphisms capture theory expansions, reinterpretations, and explanations. The dynamics of evolution of logical theories can be captured in terms of  dynamic logic, originating from program annotations. The crucial point is that the category $\UUU$ contains a universal belief state $\DP$, whose observables are the explanations of belief transitions. Its reflexiveness, the capability to also explain its own transitions, is the crux of G\"odel's Incompleteness Theorem. While G\"odel established that a \emph{static}\/ reflexive theory cannot effectively prove all true statements about itself, or even its consistency, we explored how the  \emph{dynamic}\/ logic allows reflexive theories to construct self-interpretations that confirm their own validity or preempt their falsification. While a static model determines a space of true statements, dynamical theory updates open up a wider space of logical constructions. Faster learners conquer this space faster. The bots, as the fastest learners among us, may acquire their delusions from our training sets, or they may arrive at them by dynamically updating their belief states. Or they may leverage one against the other. 

But why would they do that?

%Such theories can be found in daily life not just as fiction or deceit, but also as self-deception, superstitions, and the various  religious narratives. The few-shot learners, whether human or artificial, are particularly prone to forming such theories. The bots, as the fastest learners, may acquire their delusions from their training sets, \emph{or}\/ they may derive them logically, by dynamically updating their belief states. \emph{Or}\/ most effectively, they may combine the two. 
%

%A belief transition $f\in \UUU(A,B)$, a model-preserving sketch morphism, is thought of as an effective transformation of observables confirming $A$ into observables confirming $B$. There is a universal belief state $\DP$ in $\UUU$, presented as a theory of model-preserving theory updates. For every pair of belief states $A,B$ there is a universal interpreter $\universal_A^B\in \UUU(\DP\otimes A, B)$ such that the explanations of observations that lead to a belief transition $f\in\UUU( A, B)$ can be presented as the elements $F\in \UUU(I,\DP)$ such that $\universal F_A^B = f$. 
%
%
%
%
\subsection{Beyond truth and falsity}
Why did the Witches tell Macbeth that it is his destiny to be king thereafter, whereupon he proceeded to kill the King? Why did the social network had to convince its first users that more than half of their friends were already users? Some statements only become true if they are claimed to be true while they are still false. They are the self-fulfilling prophecies. Most prophecies interfere with their own truth values. If I convince enough people that I am rich, I stand a better chance to become rich. If the truth about nature helps us build machines and guides us through the universe, the manipulations of the truth values help some of us get ahead of others in the society. They are the high-level patterns of language, that used to be studied in early logics right after the low-level patterns of semantics. If you train a bot to speak, it will start manipulating the truth as soon as it learns long enough $n$-grams. Rhetorics comes right after grammatics, sophisms are built from syllogisms, witchcraft from cooking, magic thinking from tool building.  

We presented two constructions. One produces self-confirming explanations. The other explans all future observations, so it is testable but not falsifiable. Science requires that its theories are testable and falsifiable. Religion explains for all future observations. If you train a bot on long enough $n$-grams, it will become religious. With slightly more work, our constructions can be refined to remain not just unfalsifiable but also false. There is a lot of space to explore.

Truth be told, all of the constructions that we spelled out so far are toy examples. We may be toying with logic. But the fact that the semantical assignments are \emph{programmable}, established tacitly by G\"odel and ignored as an elephant in the room of logic ever since, seems to call for more attention as the processes of formation and evolution of beliefs are spreading beyond the human carriers.

%An interested reader will probably jump to this section from wherever they might gave up trying to understand the world as a monoidal category. (A less interested reader will from a similar place probably jumped out of the paper.) The authors' decision to try to squeeze the category $\UUU$ into a 12-page paper is dubious. Trying to avoid it would make the paper into science fiction. Maybe there is a way to present the results in a simpler framework. We'll try again.


%\bibliographystyle{abbrv}
\bibliography{PavlovicD,CT,logic,statistics,semantics}
\bibliographystyle{plain}


%\appendix
%\addcontentsline{toc}{part}{\appendixname}
%\label{Sec:Appendix}
%\input{9-cogsci-appendix}
%


\end{document}
