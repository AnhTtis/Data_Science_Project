% !TEX root = 00-wollic.tex

Logic as the theory of theories was originally developed to prove true statements. Here we study developments in the opposite direction: modifying logical theories and their interpretations to make provable some previously unprovable statements. The purpose of such transitions in science is to extend existing theories with new knowledge. The purpose of such transitions in public life is to modify some public views to better suit  private preferences. The instances of such transitions span a broad gamut of techniques, from unsupervised learning to conditioning.

If unprovable claims of a theory are true in a nontrivial model, the theory can be consistently extended to make them provable. That was clear already in the time of G\"odel's Incompleteness Theorem \cite{GoedelK:ueber}. If the unprovable claims are false, then making them true and provable requires modifying the model and reinterpreting the theory. Such dynamic reassignments of meaning bring us into the realm of \emph{dynamic logic}. If the propositions from a lattice $\TTT$ are used as assertions about the states of the world or the states of our beliefs, then the dynamic changes of these assertions under the influence of events from a lattice $\EEE$ can be expressed in terms of the \emph{Hoare triples}
\[ A\uev e B\]
saying that the event $E$ after the precondition state $A$ leads to the postcondition state $B$. The \emph{Hoare logic}\/ evolved in the late 60s as a method for reasoning about program expressions generating the events in $\EEE$ in terms of the program annotations $A$ and $B$ as formal versions of the preconditions and the postconditions habitually inserted by programmers into their code as comments, to clarify and stipulate the intended functions of blocks of code \cite{FloydR:meaning,Hoare-logic}. The propositional algebra of dynamic logic can be viewed as a monotone map
\[ \TTT^{o}\times \EEE\times \TTT\tto{ - \uev - -} \OOO\]
where $\OOO$ is a lattice of truth values. If the lattice $\TTT$ is complete, then each event $e\in \EEE$ induces a Galois connection
\[ A\rtimes e \vdash B \ \ \iff\ \ A\uev e B\ \ \iff\ \ A\vdash [e]B\]
determining a \emph{dynamic modality}\/ $[e]\colon\TTT\to\TTT$ for every $e\in \EEE$ \cite{PrattV:dynamic}. The induced interior operation $\left([e]B\right)\rtimes E\vdash B$ says that $[e]B$ is the weakest precondition that assures $B$ after $e$. The induced closure $A\vdash [e]\left(A\rtimes e\right)$ says that $A\rtimes e$ is the strongest postcondition that can be guaranteed by $A$ before $e$. Beyond such program annotations, dynamic logic has been developed in many directions  \cite{BenthemJ:dynamic-book,DitmarschH:dynamic-epistemic,Kozen:dynamic-book}. Here we use it as a backdrop for evolution of theories and their interpretations from science to religion.
