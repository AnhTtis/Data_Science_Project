% !TEX root = 00-wollic.tex

\subsection{What did we learn?}
We sketched the category $\UUU$ of \emph{belief states} $A, B,\ldots$, presented as theories with standard models.  A belief transition $f\colon A\to B$ updates the belief state by mapping the observables supporting $A$ to observables supporting $B$. Such morphisms capture theory expansions, reinterpretations, and explanations. The dynamics of evolution of logical theories can be captured in terms of  dynamic logic, originating from program annotations. The crucial point is that the category $\UUU$ contains a universal belief state $\DP$, whose observables are the explanations of belief transitions. Its reflexiveness, the capability to also explain its own transitions, is the crux of G\"odel's Incompleteness Theorem. While G\"odel established that a \emph{static}\/ reflexive theory cannot effectively prove all true statements about itself, or even its consistency, we explored how the  \emph{dynamic}\/ logic allows reflexive theories to construct self-interpretations that confirm their own validity or preempt their falsification. While a static model determines a space of true statements, dynamical theory updates open up a wider space of logical constructions. Faster learners conquer this space faster. The bots, as the fastest learners among us, may acquire their delusions from our training sets, or they may arrive at them by dynamically updating their belief states. Or they may leverage one against the other. 

But why would they do that?

%Such theories can be found in daily life not just as fiction or deceit, but also as self-deception, superstitions, and the various  religious narratives. The few-shot learners, whether human or artificial, are particularly prone to forming such theories. The bots, as the fastest learners, may acquire their delusions from their training sets, \emph{or}\/ they may derive them logically, by dynamically updating their belief states. \emph{Or}\/ most effectively, they may combine the two. 
%

%A belief transition $f\in \UUU(A,B)$, a model-preserving sketch morphism, is thought of as an effective transformation of observables confirming $A$ into observables confirming $B$. There is a universal belief state $\DP$ in $\UUU$, presented as a theory of model-preserving theory updates. For every pair of belief states $A,B$ there is a universal interpreter $\universal_A^B\in \UUU(\DP\otimes A, B)$ such that the explanations of observations that lead to a belief transition $f\in\UUU( A, B)$ can be presented as the elements $F\in \UUU(I,\DP)$ such that $\universal F_A^B = f$. 
%
%
%
%
\subsection{Beyond truth and falsity}
Why did the Witches tell Macbeth that it is his destiny to be king thereafter, whereupon he proceeded to kill the King? Why did the social network had to convince its first users that more than half of their friends were already users? Some statements only become true if they are claimed to be true while they are still false. They are the self-fulfilling prophecies. Most prophecies interfere with their own truth values. If I convince enough people that I am rich, I stand a better chance to become rich. If the truth about nature helps us build machines and guides us through the universe, the manipulations of the truth values help some of us get ahead of others in the society. They are the high-level patterns of language, that used to be studied in early logics right after the low-level patterns of semantics. If you train a bot to speak, it will start manipulating the truth as soon as it learns long enough $n$-grams. Rhetorics comes right after grammatics, sophisms are built from syllogisms, witchcraft from cooking, magic thinking from tool building.  

We presented two constructions. One produces self-confirming explanations. The other explans all future observations, so it is testable but not falsifiable. Science requires that its theories are testable and falsifiable. Religion explains for all future observations. If you train a bot on long enough $n$-grams, it will become religious. With slightly more work, our constructions can be refined to remain not just unfalsifiable but also false. There is a lot of space to explore.

Truth be told, all of the constructions that we spelled out so far are toy examples. We may be toying with logic. But the fact that the semantical assignments are \emph{programmable}, established tacitly by G\"odel and ignored as an elephant in the room of logic ever since, seems to call for more attention as the processes of formation and evolution of beliefs are spreading beyond the human carriers.

%An interested reader will probably jump to this section from wherever they might gave up trying to understand the world as a monoidal category. (A less interested reader will from a similar place probably jumped out of the paper.) The authors' decision to try to squeeze the category $\UUU$ into a 12-page paper is dubious. Trying to avoid it would make the paper into science fiction. Maybe there is a way to present the results in a simpler framework. We'll try again.
