%%Created by jPicEdt 1.4.1_03: mixed JPIC-XML/LaTeX format
%%Sat Sep 02 17:30:15 HST 2017
%%Begin JPIC-XML
%<?xml version="1.0" standalone="yes"?>
%<jpic x-min="-0.63" x-max="15" y-min="-10.62" y-max="7.5" auto-bounding="true">
%<multicurve points= "(5.62,-0.62);(5.62,-0.62);(10.62,-5.62);(10.62,-5.62);(10.62,-5.62);(9.38,-5.62);
%	(9.38,-5.62);(9.38,-5.62);(10,-5.62);(10,-5.62)"
%	 fill-style= "none"
%	 />
%<ellipse p3= "(2.81,2.19)"
%	 p2= "(2.81,-0.94)"
%	 p1= "(5.94,-0.94)"
%	 closure= "open"
%	 angle-end= "0"
%	 fill-style= "solid"
%	 angle-start= "0"
%	 />
%<ellipse p3= "(8.75,-3.75)"
%	 p2= "(8.75,-6.88)"
%	 p1= "(11.88,-6.88)"
%	 closure= "open"
%	 angle-end= "0"
%	 fill-style= "solid"
%	 angle-start= "0"
%	 />
%<multicurve points= "(15,7.5);(15,7.5);(15,5);(15,5);(15,5);(15,-0.63);
%	(15,-0.63);(15,-0.63);(11.25,-4.38);(11.25,-4.38)"
%	 fill-style= "none"
%	 />
%<multicurve points= "(9.37,7.5);(9.37,7.5);(9.37,6.87);(9.37,6.87);(9.37,6.87);(9.38,5);
%	(9.38,5);(9.38,5);(5.63,1.25);(5.63,1.25)"
%	 fill-style= "none"
%	 />
%<multicurve points= "(-0.63,7.5);(-0.63,7.5);(-0.63,6.87);(-0.63,6.87);(-0.63,6.87);(-0.62,5);
%	(-0.62,5);(-0.62,5);(3.13,1.25);(3.13,1.25)"
%	 fill-style= "none"
%	 />
%<multicurve points= "(10,-6.88);(10,-6.88);(10,-10);(10,-10);(10,-10);(10,-10.62);
%	(10,-10.62);(10,-10.62);(10,-10);(10,-10)"
%	 fill-style= "none"
%	 />
%</jpic>
%%End JPIC-XML
%PSTricks content-type (pstricks.sty package needed)
%Add \usepackage{pstricks} in the preamble of your LaTeX file
%You can rescale the whole picture (to 80% for instance) by using the command \def\JPicScale{0.8}
\ifx\JPicScale\undefined\def\JPicScale{1}\fi
\psset{unit=\JPicScale mm}
\psset{linewidth=0.3,dotsep=1,hatchwidth=0.3,hatchsep=1.5,shadowsize=1,dimen=middle}
\psset{dotsize=0.7 2.5,dotscale=1 1,fillcolor=black}
\psset{arrowsize=1 2,arrowlength=1,arrowinset=0.25,tbarsize=0.7 5,bracketlength=0.15,rbracketlength=0.15}
\begin{pspicture}(0,0)(15,7.5)
\psline(5.62,-0.62)
(10.62,-5.62)
(9.38,-5.62)(10,-5.62)
\rput{0}(4.38,0.62){\psellipse[fillstyle=solid](0,0)(1.56,-1.57)}
\rput{0}(10.32,-5.31){\psellipse[fillstyle=solid](0,0)(1.56,-1.57)}
\psline(15,7.5)
(15,5)
(15,-0.63)(11.25,-4.38)
\psline(9.37,7.5)
(9.37,6.87)
(9.38,5)(5.63,1.25)
\psline(-0.63,7.5)
(-0.63,6.87)
(-0.62,5)(3.13,1.25)
\psline(10,-6.88)
(10,-10)
(10,-10.62)(10,-10)
\end{pspicture}
