%%Created by jPicEdt 1.4.1_03: mixed JPIC-XML/LaTeX format
%%Tue Aug 13 16:23:02 GMT-10:00 2019
%%Begin JPIC-XML
%<?xml version="1.0" standalone="yes"?>
%<jpic x-min="0" x-max="130" y-min="1.25" y-max="71.25" auto-bounding="true">
%<multicurve fill-style= "none"
%	 points= "(0,47.5);(0,47.5);(40,47.5);(40,47.5)"
%	 stroke-width= "0.75"
%	 />
%<multicurve fill-style= "none"
%	 points= "(0,47.5);(0,47.5);(0,22.5);(0,22.5)"
%	 stroke-width= "0.75"
%	 />
%<multicurve fill-style= "none"
%	 points= "(40,22.5);(40,22.5);(40,47.5);(40,47.5)"
%	 stroke-width= "0.75"
%	 />
%<multicurve fill-style= "none"
%	 points= "(20,65);(20,65);(20,47.5);(20,47.5)"
%	 stroke-width= "0.75"
%	 />
%<text text-vert-align= "center-v"
%	 fill-style= "none"
%	 anchor-point= "(20,71.25)"
%	 text-frame= "noframe"
%	 stroke-width= "0.75"
%	 text-hor-align= "center-h"
%	 >
%$\nameslang$
%</text>
%<text text-vert-align= "center-v"
%	 fill-style= "none"
%	 anchor-point= "(30,1.25)"
%	 text-frame= "noframe"
%	 stroke-width= "0.75"
%	 text-hor-align= "center-h"
%	 >
%$\inputt$
%</text>
%<multicurve fill-style= "none"
%	 points= "(0,22.5);(0,22.5);(40,22.5);(40,22.5)"
%	 stroke-width= "0.75"
%	 />
%<multicurve fill-style= "none"
%	 points= "(30,22.5);(30,22.5);(30,7.5);(30,7.5)"
%	 stroke-width= "0.75"
%	 />
%<text text-vert-align= "center-v"
%	 fill-style= "none"
%	 anchor-point= "(20,35)"
%	 text-frame= "noframe"
%	 stroke-width= "0.75"
%	 text-hor-align= "center-h"
%	 >
%\machine
%</text>
%<multicurve fill-style= "none"
%	 points= "(70,53.12);(70,53.12);(130,53.12);(130,53.12)"
%	 stroke-width= "0.75"
%	 />
%<multicurve fill-style= "none"
%	 points= "(70,41.25);(70,41.25);(70,16.25);(70,16.25)"
%	 stroke-width= "0.75"
%	 />
%<multicurve fill-style= "none"
%	 points= "(130,28.12);(130,28.12);(130,53.12);(130,53.12)"
%	 stroke-width= "0.75"
%	 />
%<multicurve fill-style= "none"
%	 points= "(110,65);(110,65);(110,53.12);(110,53.12)"
%	 stroke-width= "0.75"
%	 />
%<text text-vert-align= "center-v"
%	 fill-style= "none"
%	 anchor-point= "(110,71.25)"
%	 text-frame= "noframe"
%	 stroke-width= "0.75"
%	 text-hor-align= "center-h"
%	 >
%$\nameslang$
%</text>
%<text text-vert-align= "center-v"
%	 fill-style= "none"
%	 anchor-point= "(110,1.25)"
%	 text-frame= "noframe"
%	 stroke-width= "0.75"
%	 text-hor-align= "center-h"
%	 >
%$\inputt$
%</text>
%<multicurve fill-style= "none"
%	 points= "(95,28.12);(95,28.12);(130,28.12);(130,28.12)"
%	 stroke-width= "0.75"
%	 />
%<multicurve fill-style= "none"
%	 points= "(110,28.12);(110,28.12);(110,7.5);(110,7.5)"
%	 stroke-width= "0.75"
%	 />
%<multicurve fill-style= "none"
%	 points= "(70,16.25);(70,16.25);(95,16.25);(95,16.25)"
%	 stroke-width= "0.75"
%	 />
%<text text-vert-align= "center-v"
%	 fill-style= "none"
%	 anchor-point= "(108.12,40)"
%	 text-frame= "noframe"
%	 stroke-width= "0.75"
%	 text-hor-align= "center-h"
%	 >
%\universal
%</text>
%<text text-vert-align= "center-v"
%	 fill-style= "none"
%	 anchor-point= "(75,23.75)"
%	 text-frame= "noframe"
%	 stroke-width= "0.75"
%	 text-hor-align= "center-h"
%	 >
%\program
%</text>
%<text text-vert-align= "center-v"
%	 fill-style= "none"
%	 anchor-point= "(55,35)"
%	 text-frame= "noframe"
%	 stroke-width= "0.75"
%	 text-hor-align= "center-h"
%	 >
%\EQLS
%</text>
%<multicurve fill-style= "none"
%	 points= "(70,41.25);(70,41.25);(95,16.25);(95,16.25)"
%	 stroke-width= "0.75"
%	 />
%<multicurve fill-style= "none"
%	 points= "(83.12,40);(83.12,40);(83.12,28.12);(83.12,28.12)"
%	 stroke-width= "0.75"
%	 />
%<multicurve fill-style= "none"
%	 points= "(70,53.12);(70,53.12);(95,28.12);(95,28.12)"
%	 stroke-width= "0.75"
%	 />
%<multicurve fill-style= "none"
%	 points= "(10,22.5);(10,22.5);(10,7.5);(10,7.5)"
%	 stroke-width= "0.75"
%	 />
%<multicurve fill-style= "none"
%	 points= "(80,15);(80,15);(80,7.5);(80,7.5)"
%	 stroke-width= "0.75"
%	 />
%</jpic>
%%End JPIC-XML
%PSTricks content-type (pstricks.sty package needed)
%Add \usepackage{pstricks} in the preamble of your LaTeX file
%You can rescale the whole picture (to 80% for instance) by using the command \def\JPicScale{0.8}
\ifx\JPicScale\undefined\def\JPicScale{1}\fi
\psset{unit=\JPicScale mm}
\psset{linewidth=0.3,dotsep=1,hatchwidth=0.3,hatchsep=1.5,shadowsize=1,dimen=middle}
\psset{dotsize=0.7 2.5,dotscale=1 1,fillcolor=black}
\psset{arrowsize=1 2,arrowlength=1,arrowinset=0.25,tbarsize=0.7 5,bracketlength=0.15,rbracketlength=0.15}
\begin{pspicture}(0,0)(130,71.25)
\psline[linewidth=0.75](0,47.5)(40,47.5)
\psline[linewidth=0.75](0,47.5)(0,22.5)
\psline[linewidth=0.75](40,22.5)(40,47.5)
\psline[linewidth=0.75](20,65)(20,47.5)
\rput(20,71.25){$\nameslang$}
\rput(30,1.25){$\inputt$}
\psline[linewidth=0.75](0,22.5)(40,22.5)
\psline[linewidth=0.75](30,22.5)(30,7.5)
\rput(20,35){\machine}
\psline[linewidth=0.75](70,53.12)(130,53.12)
\psline[linewidth=0.75](70,41.25)(70,16.25)
\psline[linewidth=0.75](130,28.12)(130,53.12)
\psline[linewidth=0.75](110,65)(110,53.12)
\rput(110,71.25){$\nameslang$}
\rput(110,1.25){$\inputt$}
\psline[linewidth=0.75](95,28.12)(130,28.12)
\psline[linewidth=0.75](110,28.12)(110,7.5)
\psline[linewidth=0.75](70,16.25)(95,16.25)
\rput(108.12,40){\universal}
\rput(75,23.75){\program}
\rput(55,35){\EQLS}
\psline[linewidth=0.75](70,41.25)(95,16.25)
\psline[linewidth=0.75](83.12,40)(83.12,28.12)
\psline[linewidth=0.75](70,53.12)(95,28.12)
\psline[linewidth=0.75](10,22.5)(10,7.5)
\psline[linewidth=0.75](80,15)(80,7.5)
\end{pspicture}
