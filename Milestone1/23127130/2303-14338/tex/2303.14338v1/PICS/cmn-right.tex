%%Created by jPicEdt 1.4.1_03: mixed JPIC-XML/LaTeX format
%%Sat Sep 02 17:32:58 HST 2017
%%Begin JPIC-XML
%<?xml version="1.0" standalone="yes"?>
%<jpic x-min="-0.63" x-max="15.63" y-min="-10.62" y-max="8.12" auto-bounding="true">
%<ellipse p3= "(9.06,2.19)"
%	 p2= "(9.06,-0.94)"
%	 p1= "(12.19,-0.94)"
%	 closure= "open"
%	 angle-end= "0"
%	 fill-style= "solid"
%	 angle-start= "0"
%	 />
%<multicurve points= "(15.62,8.12);(15.62,8.12);(15.62,7.5);(15.62,7.5);(15.62,7.5);(15.63,5.62);
%	(15.63,5.62);(15.63,5.62);(11.88,1.87);(11.88,1.87)"
%	 fill-style= "none"
%	 />
%<multicurve points= "(5.62,8.12);(5.62,8.12);(5.62,7.5);(5.62,7.5);(5.62,7.5);(5.63,5.62);
%	(5.63,5.62);(5.63,5.62);(9.38,1.87);(9.38,1.87)"
%	 fill-style= "none"
%	 />
%<ellipse p3= "(2.81,-4.06)"
%	 p2= "(2.81,-7.19)"
%	 p1= "(5.94,-7.19)"
%	 closure= "open"
%	 angle-end= "0"
%	 fill-style= "solid"
%	 angle-start= "0"
%	 />
%<multicurve points= "(-0.63,7.5);(-0.63,7.5);(-0.62,1.25);(-0.62,1.25);(-0.62,1.25);(-0.62,-0.63);
%	(-0.62,-0.63);(-0.62,-0.63);(3.13,-4.38);(3.13,-4.38)"
%	 fill-style= "none"
%	 />
%<multicurve points= "(9.38,-0.62);(9.38,-0.62);(4.38,-5.62);(4.38,-5.62);(4.38,-5.62);(4.38,-5.62);
%	(4.38,-5.62);(4.38,-5.62);(4.38,-5.62);(4.38,-5.62)"
%	 fill-style= "none"
%	 />
%<multicurve points= "(4.38,-6.88);(4.38,-6.88);(4.38,-10);(4.38,-10);(4.38,-10);(4.38,-10.62);
%	(4.38,-10.62);(4.38,-10.62);(4.38,-10);(4.38,-10)"
%	 fill-style= "none"
%	 />
%</jpic>
%%End JPIC-XML
%PSTricks content-type (pstricks.sty package needed)
%Add \usepackage{pstricks} in the preamble of your LaTeX file
%You can rescale the whole picture (to 80% for instance) by using the command \def\JPicScale{0.8}
\ifx\JPicScale\undefined\def\JPicScale{1}\fi
\psset{unit=\JPicScale mm}
\psset{linewidth=0.3,dotsep=1,hatchwidth=0.3,hatchsep=1.5,shadowsize=1,dimen=middle}
\psset{dotsize=0.7 2.5,dotscale=1 1,fillcolor=black}
\psset{arrowsize=1 2,arrowlength=1,arrowinset=0.25,tbarsize=0.7 5,bracketlength=0.15,rbracketlength=0.15}
\begin{pspicture}(0,0)(15.63,8.12)
\rput{0}(10.62,0.62){\psellipse[fillstyle=solid](0,0)(1.56,-1.56)}
\psline(15.62,8.12)
(15.62,7.5)
(15.63,5.62)(11.88,1.87)
\psline(5.62,8.12)
(5.62,7.5)
(5.63,5.62)(9.38,1.87)
\rput{0}(4.38,-5.62){\psellipse[fillstyle=solid](0,0)(1.57,-1.57)}
\psline(-0.63,7.5)
(-0.62,1.25)
(-0.62,-0.63)(3.13,-4.38)
\pscustom[]{\psline(9.38,-0.62)(4.38,-5.62)
\psbezier(4.38,-5.62)(4.38,-5.62)(4.38,-5.62)
\psbezier(4.38,-5.62)(4.38,-5.62)(4.38,-5.62)
}
\psline(4.38,-6.88)
(4.38,-10)
(4.38,-10.62)(4.38,-10)
\end{pspicture}
