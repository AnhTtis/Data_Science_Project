%%Created by jPicEdt 1.4.1_03: mixed JPIC-XML/LaTeX format
%%Mon May 17 23:15:37 GMT-10:00 2021
%%Begin JPIC-XML
%<?xml version="1.0" standalone="yes"?>
%<jpic x-min="0" x-max="200" y-min="0" y-max="150" auto-bounding="true">
%<multicurve fill-style= "none"
%	 points= "(80,80);(80,80);(80,40);(80,40)"
%	 stroke-width= "0.75"
%	 />
%<multicurve fill-style= "none"
%	 points= "(80,40);(80,40);(0,40);(0,40)"
%	 stroke-width= "0.75"
%	 />
%<multicurve fill-style= "none"
%	 points= "(0,80);(0,80);(80,80);(80,80)"
%	 stroke-width= "0.75"
%	 />
%<multicurve fill-style= "none"
%	 points= "(60,40);(60,40);(60,0);(60,0)"
%	 stroke-width= "0.75"
%	 />
%<multicurve fill-style= "none"
%	 points= "(40,120);(40,120);(40,80);(40,80)"
%	 stroke-width= "0.75"
%	 />
%<multicurve fill-style= "none"
%	 points= "(0,80);(0,80);(0,40);(0,40)"
%	 stroke-width= "0.75"
%	 />
%<multicurve fill-style= "none"
%	 points= "(20,40);(20,40);(20,0);(20,0)"
%	 stroke-width= "0.75"
%	 />
%<multicurve fill-style= "none"
%	 points= "(200,100);(200,100);(200,60);(200,60)"
%	 stroke-width= "0.75"
%	 />
%<multicurve fill-style= "none"
%	 points= "(160,20);(160,20);(120,20);(120,20)"
%	 stroke-width= "0.75"
%	 />
%<multicurve fill-style= "none"
%	 points= "(120,100);(120,100);(200,100);(200,100)"
%	 stroke-width= "0.75"
%	 />
%<multicurve fill-style= "none"
%	 points= "(180,60);(180,60);(180,0);(180,0)"
%	 stroke-width= "0.75"
%	 />
%<multicurve fill-style= "none"
%	 points= "(160,120);(160,120);(160,100);(160,100)"
%	 stroke-width= "0.75"
%	 />
%<multicurve fill-style= "none"
%	 points= "(120,60);(120,60);(120,20);(120,20)"
%	 stroke-width= "0.75"
%	 />
%<multicurve fill-style= "none"
%	 points= "(140,20);(140,20);(140,0);(140,0)"
%	 stroke-width= "0.75"
%	 />
%<multicurve fill-style= "none"
%	 points= "(200,60);(200,60);(160,60);(160,60)"
%	 stroke-width= "0.75"
%	 />
%<multicurve fill-style= "none"
%	 points= "(120,100);(120,100);(160,60);(160,60)"
%	 stroke-width= "0.75"
%	 />
%<multicurve fill-style= "none"
%	 points= "(120,60);(120,60);(160,20);(160,20)"
%	 stroke-width= "0.75"
%	 />
%<multicurve fill-style= "none"
%	 points= "(140,40);(140,40);(140,80);(140,80)"
%	 stroke-width= "1.4"
%	 />
%<text text-vert-align= "center-v"
%	 fill-style= "none"
%	 anchor-point= "(170,80)"
%	 text-frame= "noframe"
%	 stroke-width= "0.75"
%	 text-hor-align= "center-h"
%	 >
%$\Univ$
%</text>
%<text text-vert-align= "center-v"
%	 fill-style= "none"
%	 anchor-point= "(100,60)"
%	 text-frame= "noframe"
%	 stroke-width= "0.75"
%	 text-hor-align= "center-h"
%	 >
%\EQLS
%</text>
%<text text-vert-align= "center-v"
%	 fill-style= "none"
%	 anchor-point= "(136.25,65)"
%	 text-frame= "noframe"
%	 stroke-width= "0.75"
%	 text-hor-align= "right"
%	 >
%$\Code$
%</text>
%<text text-vert-align= "center-v"
%	 fill-style= "none"
%	 anchor-point= "(165,117.5)"
%	 text-frame= "noframe"
%	 stroke-width= "0.75"
%	 text-hor-align= "left"
%	 >
%$\Cee$
%</text>
%<text text-vert-align= "center-v"
%	 fill-style= "none"
%	 anchor-point= "(45,117.5)"
%	 text-frame= "noframe"
%	 stroke-width= "0.75"
%	 text-hor-align= "left"
%	 >
%$\Cee$
%</text>
%<text text-vert-align= "center-v"
%	 fill-style= "none"
%	 anchor-point= "(15,2.5)"
%	 text-frame= "noframe"
%	 stroke-width= "0.75"
%	 text-hor-align= "right"
%	 >
%$\Aee$
%</text>
%<text text-vert-align= "center-v"
%	 fill-style= "none"
%	 anchor-point= "(55,2.5)"
%	 text-frame= "noframe"
%	 stroke-width= "0.75"
%	 text-hor-align= "right"
%	 >
%$\Bee$
%</text>
%<text text-vert-align= "center-v"
%	 fill-style= "none"
%	 anchor-point= "(40,60)"
%	 text-frame= "noframe"
%	 stroke-width= "0.75"
%	 text-hor-align= "center-h"
%	 >
%$\fee$
%</text>
%<text text-vert-align= "center-v"
%	 fill-style= "none"
%	 anchor-point= "(130,31.25)"
%	 text-frame= "noframe"
%	 stroke-width= "0.75"
%	 text-hor-align= "center-h"
%	 >
%$\Fee$
%</text>
%<text text-vert-align= "center-v"
%	 fill-style= "none"
%	 anchor-point= "(135,2.5)"
%	 text-frame= "noframe"
%	 stroke-width= "0.75"
%	 text-hor-align= "right"
%	 >
%$\Aee$
%</text>
%<text text-vert-align= "center-v"
%	 fill-style= "none"
%	 anchor-point= "(175,2.5)"
%	 text-frame= "noframe"
%	 stroke-width= "0.75"
%	 text-hor-align= "right"
%	 >
%$\Bee$
%</text>
%<text text-vert-align= "center-v"
%	 fill-style= "none"
%	 anchor-point= "(100,150)"
%	 text-frame= "noframe"
%	 stroke-width= "0.75"
%	 text-hor-align= "center-h"
%	 >
%\EQLS
%</text>
%<text text-vert-align= "center-v"
%	 fill-style= "none"
%	 anchor-point= "(40,150)"
%	 text-frame= "noframe"
%	 text-hor-align= "center-h"
%	 >
%$\LHS$
%</text>
%<text text-vert-align= "center-v"
%	 fill-style= "none"
%	 anchor-point= "(160,150)"
%	 text-frame= "noframe"
%	 text-hor-align= "center-h"
%	 >
%$\RHS$
%</text>
%<text text-vert-align= "center-v"
%	 fill-style= "none"
%	 anchor-point= "(140,40)"
%	 text-frame= "noframe"
%	 text-hor-align= "center-h"
%	 >
%$\Dott$
%</text>
%</jpic>
%%End JPIC-XML
%PSTricks content-type (pstricks.sty package needed)
%Add \usepackage{pstricks} in the preamble of your LaTeX file
%You can rescale the whole picture (to 80% for instance) by using the command \def\JPicScale{0.8}
\ifx\JPicScale\undefined\def\JPicScale{1}\fi
\psset{unit=\JPicScale mm}
\psset{linewidth=0.3,dotsep=1,hatchwidth=0.3,hatchsep=1.5,shadowsize=1,dimen=middle}
\psset{dotsize=0.7 2.5,dotscale=1 1,fillcolor=black}
\psset{arrowsize=1 2,arrowlength=1,arrowinset=0.25,tbarsize=0.7 5,bracketlength=0.15,rbracketlength=0.15}
\begin{pspicture}(0,0)(200,150)
\psline[linewidth=0.75](80,80)(80,40)
\psline[linewidth=0.75](80,40)(0,40)
\psline[linewidth=0.75](0,80)(80,80)
\psline[linewidth=0.75](60,40)(60,0)
\psline[linewidth=0.75](40,120)(40,80)
\psline[linewidth=0.75](0,80)(0,40)
\psline[linewidth=0.75](20,40)(20,0)
\psline[linewidth=0.75](200,100)(200,60)
\psline[linewidth=0.75](160,20)(120,20)
\psline[linewidth=0.75](120,100)(200,100)
\psline[linewidth=0.75](180,60)(180,0)
\psline[linewidth=0.75](160,120)(160,100)
\psline[linewidth=0.75](120,60)(120,20)
\psline[linewidth=0.75](140,20)(140,0)
\psline[linewidth=0.75](200,60)(160,60)
\psline[linewidth=0.75](120,100)(160,60)
\psline[linewidth=0.75](120,60)(160,20)
\psline[linewidth=1.4](140,40)(140,80)
\rput(170,80){$\Univ$}
\rput(100,60){\EQLS}
\rput[r](136.25,65){$\Code$}
\rput[l](165,117.5){$\Cee$}
\rput[l](45,117.5){$\Cee$}
\rput[r](15,2.5){$\Aee$}
\rput[r](55,2.5){$\Bee$}
\rput(40,60){$\fee$}
\rput(130,31.25){$\Fee$}
\rput[r](135,2.5){$\Aee$}
\rput[r](175,2.5){$\Bee$}
\rput(100,150){\EQLS}
\rput(40,150){$\LHS$}
\rput(160,150){$\RHS$}
\rput(140,40){$\Dott$}
\end{pspicture}
