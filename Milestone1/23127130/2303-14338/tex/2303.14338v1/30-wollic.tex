% !TEX root = 00-wollic.tex

The theory of logical theories, or of categorical sketches, is a theory. Together with a chosen basic model, it lives as an object in the universe $\UUU$. Assuming that the universe of sets is also effectively presented, the models $\MMM$ of theories $\Theta$ can also be formalized, and a formal version of Definition~\ref{def:state} of belief states can be found living as an object in $\UUU$, as can a formal version of Definition~\ref{Def:explainable} of explainable belief transitions. Let us call this latter object  $\DP$. It is the universal state of belief. If the explainable belief transitions are viewed as computable predictions and if the explanations are their programs, then $\DP$ can be thought of as a programming language. As a state of belief, it is universal in the sense that it carries a universal interpreter of all explanations for explainable transitions, and of all programs that compute the predictions derived from them.

\begin{definition}
A \emph{universal interpreter}\/ for belief states $A,B$ is a belief transition $\{\}\colon \DP\otimes A\to B$ in $\UUU$ which is universal for all parametric families of belief transitions from $A$ to $B$. This means that for any belief state $X$ and any belief transition $g\in \UUU(X\otimes A, B)$ there is an explanation $G\in \tot\UUU(X,\DP)$ with
\beq\label{eq:uev}
\begin{split}
\newcommand{\Fee}{\scriptstyle G}
\newcommand{\fee}{g}
\newcommand{\Aee}{\scriptstyle X}
\newcommand{\Bee}{\scriptstyle A}
\newcommand{\Cee}{\scriptstyle B}
\newcommand{\Code}{\scriptstyle \PPp}
\newcommand{\Univ}{\mbox{\large$\{\}$}}
\newcommand{\Dott}{\mbox{\Large$\bullet$}}
\def\JPicScale{.18}
%%Created by jPicEdt 1.4.1_03: mixed JPIC-XML/LaTeX format
%%Thu Mar 16 16:35:34 GMT-10:00 2023
%%Begin JPIC-XML
%<?xml version="1.0" standalone="yes"?>
%<jpic x-min="0" x-max="220" y-min="0" y-max="120" auto-bounding="true">
%<multicurve fill-style= "none"
%	 stroke-width= "0.35"
%	 points= "(80,80);(80,80);(80,40);(80,40)"
%	 />
%<multicurve fill-style= "none"
%	 stroke-width= "0.35"
%	 points= "(80,40);(80,40);(0,40);(0,40)"
%	 />
%<multicurve fill-style= "none"
%	 stroke-width= "0.35"
%	 points= "(0,80);(0,80);(80,80);(80,80)"
%	 />
%<multicurve fill-style= "none"
%	 stroke-width= "0.35"
%	 points= "(60,40);(60,40);(60,0);(60,0)"
%	 />
%<multicurve fill-style= "none"
%	 stroke-width= "0.35"
%	 points= "(40,120);(40,120);(40,80);(40,80)"
%	 />
%<multicurve fill-style= "none"
%	 stroke-width= "0.35"
%	 points= "(0,80);(0,80);(0,40);(0,40)"
%	 />
%<multicurve fill-style= "none"
%	 stroke-width= "0.35"
%	 points= "(20,40);(20,40);(20,0);(20,0)"
%	 />
%<multicurve fill-style= "none"
%	 stroke-width= "0.35"
%	 points= "(220,100);(220,100);(220,60);(220,60)"
%	 />
%<multicurve fill-style= "none"
%	 stroke-width= "0.35"
%	 points= "(180,20);(180,20);(140,20);(140,20)"
%	 />
%<multicurve fill-style= "none"
%	 stroke-width= "0.35"
%	 points= "(140,100);(140,100);(220,100);(220,100)"
%	 />
%<multicurve fill-style= "none"
%	 stroke-width= "0.35"
%	 points= "(200,60);(200,60);(200,0);(200,0)"
%	 />
%<multicurve fill-style= "none"
%	 stroke-width= "0.35"
%	 points= "(180,120);(180,120);(180,100);(180,100)"
%	 />
%<multicurve fill-style= "none"
%	 stroke-width= "0.35"
%	 points= "(140,60);(140,60);(140,20);(140,20)"
%	 />
%<multicurve fill-style= "none"
%	 stroke-width= "0.35"
%	 points= "(160,20);(160,20);(160,0);(160,0)"
%	 />
%<multicurve fill-style= "none"
%	 stroke-width= "0.35"
%	 points= "(220,60);(220,60);(180,60);(180,60)"
%	 />
%<multicurve fill-style= "none"
%	 stroke-width= "0.35"
%	 points= "(140,100);(140,100);(180,60);(180,60)"
%	 />
%<multicurve fill-style= "none"
%	 stroke-width= "0.35"
%	 points= "(140,60);(140,60);(180,20);(180,20)"
%	 />
%<multicurve fill-style= "none"
%	 stroke-width= "0.7"
%	 points= "(160,40);(160,40);(160,80);(160,80)"
%	 />
%<text fill-style= "none"
%	 stroke-width= "0.35"
%	 text-vert-align= "center-v"
%	 anchor-point= "(190,80)"
%	 text-frame= "noframe"
%	 text-hor-align= "center-h"
%	 >
%$\Univ$
%</text>
%<text fill-style= "none"
%	 stroke-width= "0.35"
%	 text-vert-align= "center-v"
%	 anchor-point= "(110,60)"
%	 text-frame= "noframe"
%	 text-hor-align= "center-h"
%	 >
%\EQLS
%</text>
%<text fill-style= "none"
%	 stroke-width= "0.35"
%	 text-vert-align= "center-v"
%	 anchor-point= "(156.25,65)"
%	 text-frame= "noframe"
%	 text-hor-align= "right"
%	 >
%$\Code$
%</text>
%<text fill-style= "none"
%	 stroke-width= "0.35"
%	 text-vert-align= "center-v"
%	 anchor-point= "(185,117.5)"
%	 text-frame= "noframe"
%	 text-hor-align= "left"
%	 >
%$\Cee$
%</text>
%<text fill-style= "none"
%	 stroke-width= "0.35"
%	 text-vert-align= "center-v"
%	 anchor-point= "(45,117.5)"
%	 text-frame= "noframe"
%	 text-hor-align= "left"
%	 >
%$\Cee$
%</text>
%<text fill-style= "none"
%	 stroke-width= "0.35"
%	 text-vert-align= "center-v"
%	 anchor-point= "(15,2.5)"
%	 text-frame= "noframe"
%	 text-hor-align= "right"
%	 >
%$\Aee$
%</text>
%<text fill-style= "none"
%	 stroke-width= "0.35"
%	 text-vert-align= "center-v"
%	 anchor-point= "(55,2.5)"
%	 text-frame= "noframe"
%	 text-hor-align= "right"
%	 >
%$\Bee$
%</text>
%<text fill-style= "none"
%	 stroke-width= "0.35"
%	 text-vert-align= "center-v"
%	 anchor-point= "(40,60)"
%	 text-frame= "noframe"
%	 text-hor-align= "center-h"
%	 >
%$\fee$
%</text>
%<text fill-style= "none"
%	 stroke-width= "0.35"
%	 text-vert-align= "center-v"
%	 anchor-point= "(150,31.25)"
%	 text-frame= "noframe"
%	 text-hor-align= "center-h"
%	 >
%$\Fee$
%</text>
%<text fill-style= "none"
%	 stroke-width= "0.35"
%	 text-vert-align= "center-v"
%	 anchor-point= "(155,2.5)"
%	 text-frame= "noframe"
%	 text-hor-align= "right"
%	 >
%$\Aee$
%</text>
%<text fill-style= "none"
%	 stroke-width= "0.35"
%	 text-vert-align= "center-v"
%	 anchor-point= "(195,2.5)"
%	 text-frame= "noframe"
%	 text-hor-align= "right"
%	 >
%$\Bee$
%</text>
%<text fill-style= "none"
%	 stroke-width= "0.35"
%	 text-vert-align= "center-v"
%	 anchor-point= "(160,40)"
%	 text-frame= "noframe"
%	 text-hor-align= "center-h"
%	 >
%$\Dott$
%</text>
%</jpic>
%%End JPIC-XML
%LaTeX-picture environment using emulated lines and arcs
%You can rescale the whole picture (to 80% for instance) by using the command \def\JPicScale{0.8}
\ifx\JPicScale\undefined\def\JPicScale{1}\fi
\unitlength \JPicScale mm
\begin{picture}(220,120)(0,0)
\linethickness{0.35mm}
\put(80,40){\line(0,1){40}}
\linethickness{0.35mm}
\put(0,40){\line(1,0){80}}
\linethickness{0.35mm}
\put(0,80){\line(1,0){80}}
\linethickness{0.35mm}
\put(60,0){\line(0,1){40}}
\linethickness{0.35mm}
\put(40,80){\line(0,1){40}}
\linethickness{0.35mm}
\put(0,40){\line(0,1){40}}
\linethickness{0.35mm}
\put(20,0){\line(0,1){40}}
\linethickness{0.35mm}
\put(220,60){\line(0,1){40}}
\linethickness{0.35mm}
\put(140,20){\line(1,0){40}}
\linethickness{0.35mm}
\put(140,100){\line(1,0){80}}
\linethickness{0.35mm}
\put(200,0){\line(0,1){60}}
\linethickness{0.35mm}
\put(180,100){\line(0,1){20}}
\linethickness{0.35mm}
\put(140,20){\line(0,1){40}}
\linethickness{0.35mm}
\put(160,0){\line(0,1){20}}
\linethickness{0.35mm}
\put(180,60){\line(1,0){40}}
\linethickness{0.35mm}
\multiput(140,100)(0.12,-0.12){333}{\line(1,0){0.12}}
\linethickness{0.35mm}
\multiput(140,60)(0.12,-0.12){333}{\line(1,0){0.12}}
\linethickness{0.7mm}
\put(160,40){\line(0,1){40}}
\put(190,80){\makebox(0,0)[cc]{$\Univ$}}

\put(110,60){\makebox(0,0)[cc]{\EQLS}}

\put(156.25,65){\makebox(0,0)[cr]{$\Code$}}

\put(185,117.5){\makebox(0,0)[cl]{$\Cee$}}

\put(45,117.5){\makebox(0,0)[cl]{$\Cee$}}

\put(15,2.5){\makebox(0,0)[cr]{$\Aee$}}

\put(55,2.5){\makebox(0,0)[cr]{$\Bee$}}

\put(40,60){\makebox(0,0)[cc]{$\fee$}}

\put(150,31.25){\makebox(0,0)[cc]{$\Fee$}}

\put(155,2.5){\makebox(0,0)[cr]{$\Aee$}}

\put(195,2.5){\makebox(0,0)[cr]{$\Bee$}}

\put(160,40){\makebox(0,0)[cc]{$\Dott$}}

\end{picture}
 
\end{split}
\eeq
\end{definition}

\paragraph{Comment.} On one hand, a universal interpreter is universal for parametric families. On the other hand, it is a parametric family itself. It is thus capable of interpreting itself. G\"odel's Incompleteness Theorem is based on this capability of reflection. Universal interpreters support a suitable version of this theorem, very simple, spelled out in \cite[Sec.~5.2]{PavlovicD:MonCom}. But G\"odel's Incompleteness Theorem is concerned with a single belief state, a single theory with a standard model, capable of reflection. The point of this paper is that bots can learn their way out of logic of static belief states, expand their belief transitions through dynamic logic (here in the Hoare form), and arrive to beliefs that are complete in a suitable sense, and thus go beyond G\"odel's Incompleteness Theorem. Such beliefs can be constructed using the \emph{specializers} which are derived  directly from the definition of universal interpreters. 

\begin{lemma}\label{prop:pev}
For any $X, A, B$ there is a\/ \emph{specializer} $\prtial \in \tot\UUU(\DP\times X, \DP)$ such that
\beq\label{eq:pev}
\begin{split}
\newcommand{\Fee}{\scriptstyle \prtial}
\newcommand{\fee}{\mbox{\large$\{\}$}}
\newcommand{\Aee}{\scriptstyle X}
\newcommand{\Bee}{\scriptstyle A}
\newcommand{\Cee}{\scriptstyle B}
\newcommand{\Code}{\scriptstyle \DP}
\newcommand{\Univ}{\mbox{\large$\{\}$}}
\newcommand{\Dott}{\mbox{\LARGE$\bullet$}}
\def\JPicScale{.18}
%%Created by jPicEdt 1.4.1_03: mixed JPIC-XML/LaTeX format
%%Fri Mar 17 18:45:57 GMT-10:00 2023
%%Begin JPIC-XML
%<?xml version="1.0" standalone="yes"?>
%<jpic x-min="0" x-max="280" y-min="0" y-max="120" auto-bounding="true">
%<multicurve fill-style= "none"
%	 points= "(120,80);(120,80);(120,40);(120,40)"
%	 />
%<multicurve fill-style= "none"
%	 points= "(120,40);(120,40);(40,40);(40,40)"
%	 />
%<multicurve fill-style= "none"
%	 points= "(0,80);(0,80);(120,80);(120,80)"
%	 />
%<multicurve fill-style= "none"
%	 points= "(100,40);(100,40);(100,0);(100,0)"
%	 />
%<multicurve fill-style= "none"
%	 points= "(80,120);(80,120);(80,80);(80,80)"
%	 />
%<multicurve fill-style= "none"
%	 points= "(0,80);(0,80);(40,40);(40,40)"
%	 />
%<multicurve fill-style= "none"
%	 points= "(60,40);(60,40);(60,0);(60,0)"
%	 />
%<multicurve fill-style= "none"
%	 points= "(280,100);(280,100);(280,60);(280,60)"
%	 />
%<multicurve fill-style= "none"
%	 points= "(240,20);(240,20);(200,20);(200,20)"
%	 />
%<multicurve fill-style= "none"
%	 points= "(200,100);(200,100);(280,100);(280,100)"
%	 />
%<multicurve fill-style= "none"
%	 points= "(260,60);(260,60);(260,0);(260,0)"
%	 />
%<multicurve fill-style= "none"
%	 points= "(240,120);(240,120);(240,100);(240,100)"
%	 />
%<multicurve fill-style= "none"
%	 points= "(160,60);(160,60);(200,20);(200,20)"
%	 />
%<multicurve fill-style= "none"
%	 points= "(220,20);(220,20);(220,0);(220,0)"
%	 />
%<multicurve fill-style= "none"
%	 points= "(280,60);(280,60);(240,60);(240,60)"
%	 />
%<multicurve fill-style= "none"
%	 points= "(200,100);(200,100);(240,60);(240,60)"
%	 />
%<multicurve fill-style= "none"
%	 points= "(200,60);(200,60);(240,20);(240,20)"
%	 />
%<multicurve fill-style= "none"
%	 points= "(220,40);(220,40);(220,80);(220,80)"
%	 stroke-width= "0.7"
%	 />
%<text text-vert-align= "center-v"
%	 fill-style= "none"
%	 anchor-point= "(250,80)"
%	 text-frame= "noframe"
%	 text-hor-align= "center-h"
%	 >
%$\Univ$
%</text>
%<text text-vert-align= "center-v"
%	 fill-style= "none"
%	 anchor-point= "(140,60)"
%	 text-frame= "noframe"
%	 text-hor-align= "center-h"
%	 >
%\EQLS
%</text>
%<text text-vert-align= "center-v"
%	 fill-style= "none"
%	 anchor-point= "(216.25,65)"
%	 text-frame= "noframe"
%	 text-hor-align= "right"
%	 >
%$\Code$
%</text>
%<text text-vert-align= "center-v"
%	 fill-style= "none"
%	 anchor-point= "(245,117.5)"
%	 text-frame= "noframe"
%	 text-hor-align= "left"
%	 >
%$\Cee$
%</text>
%<text text-vert-align= "center-v"
%	 fill-style= "none"
%	 anchor-point= "(85,117.5)"
%	 text-frame= "noframe"
%	 text-hor-align= "left"
%	 >
%$\Cee$
%</text>
%<text text-vert-align= "center-v"
%	 fill-style= "none"
%	 anchor-point= "(55,2.5)"
%	 text-frame= "noframe"
%	 text-hor-align= "right"
%	 >
%$\Aee$
%</text>
%<text text-vert-align= "center-v"
%	 fill-style= "none"
%	 anchor-point= "(95,2.5)"
%	 text-frame= "noframe"
%	 text-hor-align= "right"
%	 >
%$\Bee$
%</text>
%<text text-vert-align= "center-v"
%	 fill-style= "none"
%	 anchor-point= "(80,60)"
%	 text-frame= "noframe"
%	 text-hor-align= "center-h"
%	 >
%$\fee$
%</text>
%<text text-vert-align= "center-v"
%	 fill-style= "none"
%	 anchor-point= "(200,40)"
%	 text-frame= "noframe"
%	 text-hor-align= "center-h"
%	 >
%$\Fee$
%</text>
%<text text-vert-align= "center-v"
%	 fill-style= "none"
%	 anchor-point= "(215,2.5)"
%	 text-frame= "noframe"
%	 text-hor-align= "right"
%	 >
%$\Aee$
%</text>
%<text text-vert-align= "center-v"
%	 fill-style= "none"
%	 anchor-point= "(255,2.5)"
%	 text-frame= "noframe"
%	 text-hor-align= "right"
%	 >
%$\Bee$
%</text>
%<text text-vert-align= "center-v"
%	 fill-style= "none"
%	 anchor-point= "(220,40)"
%	 text-frame= "noframe"
%	 text-hor-align= "center-h"
%	 >
%$\Dott$
%</text>
%<multicurve fill-style= "none"
%	 points= "(20,60);(20,60);(20,0);(20,0)"
%	 />
%<multicurve fill-style= "none"
%	 points= "(200,60);(200,60);(160,60);(160,60)"
%	 />
%<multicurve fill-style= "none"
%	 points= "(180,40);(180,40);(180,0);(180,0)"
%	 />
%<text text-vert-align= "center-v"
%	 fill-style= "none"
%	 anchor-point= "(175,2.5)"
%	 text-frame= "noframe"
%	 text-hor-align= "right"
%	 >
%$\Code$
%</text>
%<text text-vert-align= "center-v"
%	 fill-style= "none"
%	 anchor-point= "(15,2.5)"
%	 text-frame= "noframe"
%	 text-hor-align= "right"
%	 >
%$\Code$
%</text>
%</jpic>
%%End JPIC-XML
%LaTeX-picture environment using emulated lines and arcs
%You can rescale the whole picture (to 80% for instance) by using the command \def\JPicScale{0.8}
\ifx\JPicScale\undefined\def\JPicScale{1}\fi
\unitlength \JPicScale mm
\begin{picture}(280,120)(0,0)
\linethickness{0.3mm}
\put(120,40){\line(0,1){40}}
\linethickness{0.3mm}
\put(40,40){\line(1,0){80}}
\linethickness{0.3mm}
\put(0,80){\line(1,0){120}}
\linethickness{0.3mm}
\put(100,0){\line(0,1){40}}
\linethickness{0.3mm}
\put(80,80){\line(0,1){40}}
\linethickness{0.3mm}
\multiput(0,80)(0.12,-0.12){333}{\line(1,0){0.12}}
\linethickness{0.3mm}
\put(60,0){\line(0,1){40}}
\linethickness{0.3mm}
\put(280,60){\line(0,1){40}}
\linethickness{0.3mm}
\put(200,20){\line(1,0){40}}
\linethickness{0.3mm}
\put(200,100){\line(1,0){80}}
\linethickness{0.3mm}
\put(260,0){\line(0,1){60}}
\linethickness{0.3mm}
\put(240,100){\line(0,1){20}}
\linethickness{0.3mm}
\multiput(160,60)(0.12,-0.12){333}{\line(1,0){0.12}}
\linethickness{0.3mm}
\put(220,0){\line(0,1){20}}
\linethickness{0.3mm}
\put(240,60){\line(1,0){40}}
\linethickness{0.3mm}
\multiput(200,100)(0.12,-0.12){333}{\line(1,0){0.12}}
\linethickness{0.3mm}
\multiput(200,60)(0.12,-0.12){333}{\line(1,0){0.12}}
\linethickness{0.7mm}
\put(220,40){\line(0,1){40}}
\put(250,80){\makebox(0,0)[cc]{$\Univ$}}

\put(140,60){\makebox(0,0)[cc]{\EQLS}}

\put(216.25,65){\makebox(0,0)[cr]{$\Code$}}

\put(245,117.5){\makebox(0,0)[cl]{$\Cee$}}

\put(85,117.5){\makebox(0,0)[cl]{$\Cee$}}

\put(55,2.5){\makebox(0,0)[cr]{$\Aee$}}

\put(95,2.5){\makebox(0,0)[cr]{$\Bee$}}

\put(80,60){\makebox(0,0)[cc]{$\fee$}}

\put(200,40){\makebox(0,0)[cc]{$\Fee$}}

\put(215,2.5){\makebox(0,0)[cr]{$\Aee$}}

\put(255,2.5){\makebox(0,0)[cr]{$\Bee$}}

\put(220,40){\makebox(0,0)[cc]{$\Dott$}}

\linethickness{0.3mm}
\put(20,0){\line(0,1){60}}
\linethickness{0.3mm}
\put(160,60){\line(1,0){40}}
\linethickness{0.3mm}
\put(180,0){\line(0,1){40}}
\put(175,2.5){\makebox(0,0)[cr]{$\Code$}}

\put(15,2.5){\makebox(0,0)[cr]{$\Code$}}

\end{picture}
 
\end{split}
\eeq
\end{lemma}

\paragraph{Hoare logic view of interpreters and specializers.} If interpreters are presented as Hoare triples in the form $(X\otimes A)\uev G B$, and if $X\!\pev G$ denotes an explanation $G$ specialized to $X$, then \eqref{eq:pev} is equivalent to the invertible rule
\[\prooftree
(X\otimes A)\uev G B
\Justifies
A\uev{X\!\pev G} B
\endprooftree\]   
The advantages of the string-diagrammatic view will become apparent in the next section. 