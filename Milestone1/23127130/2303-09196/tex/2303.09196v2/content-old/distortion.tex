\section{Distortion Representation} \label{sec:distortion-representation}
The notion of law-invariant, coherent risk measures has been studied in depth before 
(see \cite[\S6.3.5]{Shapiro2021}, \cite{Shapiro2013}, \cite{Bertsimas2009b} and references within). 
They are usually motivated through an axiomatic framework. Our interest originates from 
the fact that they are a class of risk measures that satisfy a representation like the supremum in \cref{eq:distortion-bound}.

In our setting risk measures are defined by considering 
a finite sample space $\Omega^n = \{\omega_1, \dots, \omega_n\}$ with distribution $\one_n/n = (1/n, \dots, 1/n)$ and $\sigma$-algebra $\F$. 
Then any random variable $X \colon \Omega^n \to \Re$ is uniquely characterized by some vector in $x = (X(\omega_i))_{i=1}^{n} \in \Re^n$. 
Its components will later encode the samples, similarly to $L(\theta)$ in \eqref{eq:robustified-erm}.
The underlying distribution encodes that the samples are exchangeable and has 
the consequence that identically distributed random variables are characterized as permutations (cf. \ilarxiv{\cref{app:distortion-representation}}\ilpub{\cite[App.~C]{TR}})
\begin{lemma} \label{lem:identically-distributed}
    Consider two random variables $X, Y \colon \Omega^n \to \Re$ with $x, y \in \Re^n$ their vector characterization. Then 
    \begin{equation*}
        X \deq Y \quad \Leftrightarrow \quad \exists \pi \in \Pi^n \colon x = \pi(y),
    \end{equation*}
\end{lemma}

We consider coherent risks, which can be represented as support functions \cite[\S6.3]{Shapiro2021}%\cite{Pichler2021}
\begin{proposition} \label{def:ambiguity}
    For a $\rho \colon \Re^n \to \Re$, let $\amb \dfn \dom\rho^*$ be the \emph{ambiguity set}. 
    Then $\rho$ is coherent iff
    \begin{equation} \label{eq:ambiguity-representation}
        \rho(X) = \sup_{\mu \in \amb} \, \<\mu, X\>
    \end{equation}
    for some non-empty, closed and convex $\amb \subseteq \Delta^n$.
    Moreover $\rho$ is also law-invariant (i.e. $\rho(X) = \rho(Y)$ if $X \deq Y$) 
    iff $\pi \amb = \amb$ for all $\pi \in \Pi^n$.
\end{proposition}

We provide a motivation for \cref{def:ambiguity} in terms of the usual 
definition of a coherent risk measure in \ilarxiv{\cref{app:distortion-representation}}\ilpub{\cite[App.~C]{TR}}.

We refer to \cref{eq:ambiguity-representation} as the \emph{ambiguity representation} and define an analogous 
\emph{distortion representation} using the domain of the \emph{ordered convex conjugate}:
\begin{equation} \label{eq:ordered-conjugate}
    \rho^\diamond(\mu) \dfn \sup_{X \in \Re^n_{\uparrow}} \left\{ \<\mu, X\> - \rho(X) \right\},
\end{equation}
which differs from $\rho^*$ as the supremum is taken over $\Re^n_{\uparrow}$. 

\begin{theorem} \label{thm:distortion-representation}
    For a law-invariant, coherent risk $\rho \colon \Re^n \to \Re$, let $\distort \dfn \dom\rho^\diamond$
    denote the \emph{distortion set}. Then
    \begin{equation} \label{eq:distortion-representation}
        \rho(X) = \sup_{\mu \in \distort} \, \<\mu, X_{\uparrow}\>
    \end{equation}
    for $\distort$ closed, convex,
    \begin{enumerate*}[(i)]
        \item $\one_n/n \in \distort$,
        \item $\distort + (\M^n)^\circ = \distort$,
        \item $\sum_{i=1}^{n} \mu_i = 1$ for all $\mu \in \distort$ and
        \item any $\mu$ s.t. $\pi(\mu) \in \distort$ for all $\pi \in \Pi_n$ is nonnegative.
    \end{enumerate*}

    Moreover, letting $\amb$ be as in \cref{def:ambiguity},
    \begin{align} \label{eq:distort-ambiguity-relation}
        \distort &= \amb + (\M^n)^\circ& \text{and}& &\amb &= \bigcap_{\pi \in \Pi^n} \pi\left( \distort \right).
    \end{align}
\end{theorem}
\begin{proof}
    We only derive a valid $\distort$ such that \cref{eq:distortion-representation} holds, as this is the main property 
    we require further on. From \cref{eq:ambiguity-representation}:
    \begin{equation*}
        \rho(X) = \sup_{\mu \in \amb} \<\mu, X\> = \sup_{\mu \in \amb} \<\mu, X_{\uparrow}\> = \sup_{\mu \in \amb + (\M^n)^\circ } \<\mu, X_{\uparrow}\>.
    \end{equation*}
    The second and third equality follow by permutation invariance and the definition of the polar cone respectively. 
    So $\amb + (\M^n)^\circ $ is valid. The rest is shown in \ilarxiv{\cref{app:distortion-representation}}\ilpub{\cite[App.~C]{TR}}.
\end{proof}

The full derivation of \cref{eq:distort-ambiguity-relation} in \ilarxiv{\cref{app:distortion-representation}}\ilpub{\cite[App.~C]{TR}} makes extensive use of 
the link between $\Pi^n$, $\Re_{\uparrow}^n$ and vector majorization \cite{Marshall2011}.
% stated in terms of a group-induced cone ordering \cite{Steerneman1990}. 


\begin{figure}
    \vspace{0.7em}
    \includegraphics{assets/transform-converted-to.pdf}
    \vspace{0.5em}
    \centering
    \caption{Graphical depiction of \cref{thm:distortion-representation} for $n = 3$.
    The intersection with the simplex $\Delta^n$ is shown.} \label{fig:graphical-equivalence} \vspace{-1.5em}
\end{figure}


A visualization of \cref{eq:distort-ambiguity-relation} is provided in \cref{fig:graphical-equivalence}.
We highlight the link with the supremum in \cref{eq:distortion-bound}. In fact, when $\distort = \dom\rho^\diamond$ 
it cannot be further increased in size. Any set $\distort'$ satisfying \cref{eq:distortion-representation}
would have to be a subset of $\distort$. So $\prob[W \in \distort]$ cannot be increased further for the given risk measure. 
Moreover $\distort$ is invariant under addition with $(\M^n)^\circ$ as required by \cref{eq:distortion-bound}.

We conclude this section with:
\begin{theorem} \label{thm:main-theorem}
    A law-invariant, coherent risk $\rho \colon \Re^n \to \Re$ is $\varepsilon$-calibrated as in \cref{eq:well-callibrated} with $\varepsilon = \prob[W \notin \distort  = \amb + (\M^n)^\circ]$
    for $W$ uniformly distributed over $\Delta^n$.
\end{theorem}
\begin{proof}
    This follows by combining \cref{eq:distortion-representation} with \cref{thm:distortion-bound}. 
\end{proof}

\cref{thm:main-theorem} is considered the main contribution of this work. 
In the next section we show how it can be used to calibrate several existing risk measures at a specified level $\varepsilon$. 

\begin{remark} \label{rem:robust-sample}
    In \cref{thm:main-theorem} it is assumed that $L(\theta)$ includes some $\bar{\ell}(\theta) \geq \esssup_{\xi \in \Xi} \ell(\xi, \theta)$. 
    In practice, this value can be difficult to compute. So in practice we often replace it with an additional sample. 
    % This heuristic is similar 
    % to the one used in the scenario approach \cite{Campi2018}. Its properties and associated guarantees will be studied in future work. 
\end{remark}



% \section{Distortion Representation} \label{sec:distortion-representation}
% We derive the distortion representation of law-invariant, coherent risk measures. It is analogous to the ambiguity
% set representation, but only exists under law-invariance. The distortion representation will cleanly interface with the 
% statistical framework from the next section. 

% In our setting risk measures are defined by considering 
% a finite sample space $\Omega^n = \{\omega_1, \dots, \omega_n\}$ with distribution $\one_n/n = (1/n, \dots, 1/n)$. 
% Then any random vector $X \colon \Omega^n \to \Re$ is uniquely characterized by some vector in $\Re^n$. 
% The support will later encode the samples. The underlying distribution encodes the 
% fact that the samples are exchangeable and has 
% the consequence that identically distributed random variables are characterized in terms of permutations.
% \begin{lemma} \label{lem:identically-distributed}
%     Consider two random variables $X, Y \in \Re^n$ with underlying measure $\one_n/n = (1/n, \dots, 1/n)$. Then 
%     \begin{equation*}
%         X \deq Y \quad \Leftrightarrow \quad \exists \pi \in \Pi^n \colon X = \pi(Y). 
%     \end{equation*}
% \end{lemma}
% \begin{proof}
%     Proof deferred to \cref{app:distortion-representation}.
% \end{proof}

% Throughout the rest of this section we view \cref{lem:identically-distributed} as a definition of $X \deq Y$. 
% We consider risk measures as $\rho \colon \Re^n \to \eRe$. 


% \newcommand{\cohasm}[1]{\hyperref[def:coh:#1]{\sc a\oldstylenums{#1}}}
% \begin{definition}[coherent risk] \label{def:coh}
%     Consider a real valued function $\rho \colon \Re^n \to \eRe$
%     that is proper \footnote{
%         That is $\rho(X) > -\infty$ for all $X \in \Re^n$ and the domain $\mathrm{dom}(\rho) \dfn \{X \in \Re^n \colon \rho(X) < +\infty\}$ is nonempty.
%     } and lsc.. Assume (for $X, Y \in \Re^n$):
%     \begin{enumeratass}
%         \item \emph{convex}: $\rho(\alpha X + (1-\alpha) Y) \leq \alpha \rho(X) + (1-\alpha) \rho(Y)$, $\forall \alpha \in [0, 1]$;\label{def:coh:1}
%         \item \emph{monotone}: if $Y \geq X$, then $\rho(Y) \geq \rho(X)$;\label{def:coh:2}
%         \item \emph{translation equivariance}: $\rho(X+a) = \rho(X) + a$, $\forall a \in \Re$;\label{def:coh:3}
%         \item \emph{pos. homogeneity}: if $t > 0$, then $\rho(tX) = t\rho(X)$;\label{def:coh:4}
%     \end{enumeratass}
%     then $\rho$ is a \emph{coherent risk measure}. Consider also
%     \begin{enumeratass}[resume]
%         \item \emph{law invariance}: if $X \deq Y$, then $\rho(Y) = \rho(X)$. \label{def:coh:5}
%     \end{enumeratass}
% \end{definition}
% The first four assumptions were considered in \cite[\S1]{Ruszczynski2006}, while \cohasm{5} is considered in 
% \cite[Def.~4.4]{Bertsimas2009b}. By \cref{lem:identically-distributed} it implies that $\rho(X) = \rho(\pi X)$ 
% for all $\pi \in \Pi^n$. 

% The convex conjugate $\rho^* \colon \Re^n \to \eRe$ of a risk measure is given as 
% \begin{equation} \label{eq:convex-conjugate}
%     \rho^*(\mu) = \sup_{X \in \Re^n} \left\{ \<\mu, X\> - \rho(X) \right\}.
% \end{equation}

% Its domain is affected by the assumptions in \cref{def:coh}. 
% \begin{proposition} \label{prop:conjugate-duality}
%     Suppose that $\rho \colon \Re^n \to \eRe$ is proper, lsc., and convex. Then, $\rho = \rho^{**}$
%     with $\amb \dfn \mathrm{dom}(\rho^*)$ a non-empty and convex set. Moreover
%     \begin{enumerate}[itemsep=0pt]
%         \item \cohasm{2} holds iff $\mu \geq 0$ for any $\mu \in \amb$; \label{prop:conjugate-duality:2}
%         \item \cohasm{3} holds iff $\sum_{i=1}^{n} \mu_i = 1$ for any $\mu \in \amb$; \label{prop:conjugate-duality:3}
%         \item \cohasm{4} holds iff $\amb$ is closed and \label{prop:conjugate-duality:4}
%         \begin{equation*}
%             \rho(X) = \sup_{\mu \in \amb} \, \<\mu, X\>, \quad \forall X \in \Re^n
%         \end{equation*}
%         \item \cohasm{5} holds iff $\rho^*(\pi \mu) = \rho^*(\mu)$ for all $\pi \in \Pi^n, \mu \in \Re^n$. \label{prop:conjugate-duality:5}
%     \end{enumerate}
%     We assumed the underlying measure of the random variables is $\one_n/n = (1/n, \dots, 1/n)$ for (iv). 
% \end{proposition}
% \begin{proof}
%     By \cite[Prop.~2.112]{Bonnans2000} $\rho^*$ is proper and convex. 
%     Hence its domain must be non-empty and convex. 
%     Now \emph{(i)}--\emph{(iii)} follows from \cite[Thm.~2.2]{Ruszczynski2006}. 

%     By \cref{lem:identically-distributed}, \cohasm{5} holds iff $\rho(\pi X) = X$ for all $\pi \in \Pi^n$. 
%     We first show that \cohasm{5} implies $\rho^*(\pi \mu) = \rho^*(\mu)$ for all $\pi \in \Pi^n$.
%     First note that 
%     \begin{align*}
%         &\sup_{X} \left\{ \<\mu, X\> - \rho(X) \right\} = \sup_{X} \left\{ \<\mu, \pi(X)\> - \rho(\pi(X)) \right\} \\
%         &\qquad = \sup_{X} \left\{ \<\pi^{-1}(\mu), X\> - \rho(X) \right\}
%     \end{align*}
%     Since $\{\pi^{-1} \colon \pi \in \Pi^n\} = \Pi^n$ we have shown $\rho^*(\pi \mu) = \rho^*(\mu)$ for all $\mu \in \Re^n$
%     and all $\pi \in \Pi^n$. For the reverse implication we can apply the same reasoning and using that $\rho = \rho^{**}$. 
%     Note that this result specializes \cite[Prop.~2]{Shapiro2013}.
% \end{proof}

% We have the following corollary.
% \begin{corollary} \label{cor:ambiguity}
%     Some risk $\rho \colon \Re^n \to \eRe$ is coherent iff 
%     \begin{equation} \label{eq:ambiguity-representation}
%         \rho(X) = \sup_{\mu \in \amb} \, \<\mu, X\>
%     \end{equation}
%     for some non-empty, closed and convex $\amb \subseteq \Delta^n$.
%     Moreover $\rho$ is also law-invariant iff $\pi \amb = \amb$ for all $\pi \in \Pi^n$. 
%     The set $\amb$ is referred to as the \emph{ambiguity set}. 
% \end{corollary}

% Everything above is a summary of existing characterizations of risk measures.
% These were developed based on the convex conjugate. We now re-derive similar properties based on a different 
% notion of a conjugate. Define the \emph{ordered conjugate} of $\rho \colon \Re^n \to \eRe$ as 
% \begin{equation} \label{eq:ordered-conjugate}
%     \rho^{\diamond}(\mu) \dfn \sup_{X \in \M^n} \left\{ \<\mu, X\> - \rho(X) \right\}.
% \end{equation}
% For this new conjugate we have an alternative to \cref{prop:conjugate-duality} that has law-invariance embedded into it.

% We first consider some properties of \cref{eq:ordered-conjugate} by relating it to an infimal convolution.
% \begin{lemma} \label{lem:infimal-convolution}
%     Given some proper, convex, lsc. and law-invariant risk measure $\rho \colon \Re^n \to \eRe$ and $(\M^n)^\circ \dfn (\M^n)^\circ$
%     the polar of the monotone cone. Then 
%     \begin{equation*}
%         \rho^{\diamond} = (\rho + \indi_{\M^n})^* = \cl(\rho^* \episum \indi_{(\M^n)^\circ}).
%     \end{equation*}
%     Moreover if $\dom(\rho) \cap \intr(\M^n) = \emptyset$ then the $\cl$ is spurious. 
%     We say $\rho$ is \emph{montone-proper} when this holds. 
% \end{lemma}
% \begin{proof}
%     The first equality holds by definition of the conjugate and the indicator. 
%     Suppose $\dom(\rho) \cap \M^n = \emptyset$. By law invariance and \cref{lem:identically-distributed} 
%     this implies $\rho(\pi(X)) = +\infty$ for all $X \in \M^n$ and $\pi \in \Pi^n$. So $\dom \rho = \emptyset$, contradicting 
%     properness of $\rho$. Hence $\dom(\rho) \cap \M^n \neq \emptyset$. The rest of the result then 
%     follows from \cite[thm.~11.23]{Rockafellar1998}, where we used $\indi_{\M^n}^* = \indi_{(\M^n)^\circ}$ (cf. \cite[Ex.~11.4]{Rockafellar1998}). 
% \end{proof}

% We relate the domain of $\rho^{\diamond}$ to that of $\rho^*$. 
% \begin{proposition} \label{prop:fundamental-equivalence-lemma}
%     Let $\rho \colon \Re^n \to \eRe$ be a monotone-proper, lsc., convex and law-invariant risk measure. 
%     Then
%     \begin{equation*}
%         \mathrm{dom}\left( \rho^{\diamond} \right) = \mathrm{dom}\left( \rho^* \right) + (\M^n)^\circ.
%     \end{equation*}
%     Moreover
%     \begin{equation*}
%         \mathrm{dom}\left( \rho^* \right) = \bigcap_{\pi \in \Pi^n} \pi\left( \mathrm{dom}\left( \rho^{\diamond} \right) \right).
%     \end{equation*}
% \end{proposition}
% \begin{proof}
%     The first result follows from \cref{lem:infimal-convolution} and \cite[Ex.~1.28]{Rockafellar1998}. %
%     % For the remainder let $\distort \dfn \mathrm{dom}\left( \rho^{\diamond} \right)$ and $\amb \dfn \mathrm{dom}\left( \rho^* \right)$.
%     % We first prove an auxiliary result:
%     % \begin{equation} \label{eq:cone-invariance}
%     %     \text{If } \mu \in \distort \text{ then } \mu - (\M^n)^\circ \subseteq \distort. 
%     % \end{equation}
%     % Note that, for $s \in (\M^n)^\circ$ implies 
%     % \begin{equation*}
%     %     \begin{alignedat}{1}
%     %     &\<s, X\> \geq 0, \, \forall X \in \M^n \,\Leftrightarrow \, \<\mu - s, X\> \leq \<\mu, X\>, \, \forall X \in \M^n\\
%     %     &\qquad                                   \Leftrightarrow \, \<\mu - s\> - \rho(X) \leq \<\mu, X\> - \rho(X), \, \forall X \in \M^n.
%     %     \end{alignedat}
%     % \end{equation*}
%     % Thus if $\mu \in \distort$ -- i.e. the supremum of the final expression is finite -- then $\mu - s \in \distort$. We have therefore shown \cref{eq:cone-invariance}. 
%     %
%     % Next we show $\distort \subseteq \amb - (\M^n)^\circ$. For any element $\nu \in \amb - (\M^n)^\circ$ we can pick some $\mu \in \amb$ 
%     % and some $s \in (\M^n)^\circ$ such that $\nu = \mu - s$ by construction. By permutation invariance of $\set{A}$ (cf. \cref{prop:conjugate-duality:5})
%     % we have $\mu_{\uparrow} \in \set{A}$. Hence 
%     % \begin{align*}
%     %     &\sup_{X \in \Re^n} \left\{ \<\mu_{\uparrow}, X\> - \rho(X) \right\} = \sup_{X \in \Re^n} \left\{ \<\mu_{\uparrow}, X_{\uparrow} \> - \rho(X) \right\} \\
%     %     &\qquad = \sup_{X \in \M^n} \left\{ \<\mu_{\uparrow}, X \> - \rho(X) \right\} = \rho^{\diamond}(\mu_{\uparrow})
%     % \end{align*}
%     % are all finite. For the first equality we used \cite[Prop.~6.A.3]{Marshall2011} which states $\<\mu_{\uparrow}, X\> \leq \<\mu_{\uparrow}, X_{\uparrow}\>$.
%     % For the second we used permutation invariance of $\rho$, which holds by assumption and by \cref{lem:identically-distributed}. 
%     % We have thus shown that $\mu_{\uparrow} \in \distort$. Hence, by \cref{eq:cone-invariance}, we have $\mu_{\uparrow} - (\M^n)^\circ \subseteq \distort$.
%     % Then, by \cref{lem:permutation-hull-contained}, $\hull(\{\pi(\mu) \colon \pi \in \Pi^n\}) - (\M^n)^\circ \subseteq \distort$.
%     % Since $s \in (\M^n)^\circ$ and $\mu \in \hull(\{\pi(\mu) \colon \pi \in \Pi^n\})$ we have shown $\nu = \mu - s \in \distort$. 
%     %
%     % To show $\amb - (\M^n)^\circ \subseteq \distort$ it is sufficient to prove $\amb \subseteq \distort$ by using \cref{eq:cone-invariance}. 
%     % This holds trivially since 
%     % \begin{equation*}
%     %     \sup_{X\in\Re^n} \left\{ \<\mu, X\> - \rho(X) \right\} \geq \sup_{X \in \M^n} \left\{ \<\mu, X\> - \rho(X) \right\}.
%     % \end{equation*}
%     % So whenever $\mu \in \amb$, which implies the left-hand side is finite, then the right-hand side is finite too. Thus $\mu \in \distort$. 
%     The second result follows from \cref{lem:inverse-cone-shift}. 
% \end{proof}


% We can use this lemma to generalize \cref{prop:conjugate-duality}. 
% \begin{proposition} \label{prop:ordered-conjugate-duality}
%     Suppose that $\rho \colon \Re^n \to \eRe$ is monotone-proper, lsc., law-invariant and convex. Then
%     $\distort \dfn \mathrm{dom}(\rho^\diamond)$ is a convex set such that $\distort \cap \M^n \neq \emptyset$. 
%     Moreover
%     \begin{enumerate}[itemsep=0pt]
%         \item If $\mu \in \distort$ then $\{\mu\} + (\M^n)^\circ \subseteq \distort$;
%         \item \cohasm{2} holds iff any $\mu$ s.t. $\pi(\mu) \in \distort$ for all permutations $\pi \in \Pi^n$
%         is nonnegative. 
%         \item \cohasm{3} holds iff $\sum_{i=1}^{n} \mu_i = 1$ for any $\mu \in \distort$; 
%         \item \cohasm{4} holds iff $\distort$ is closed and
%         \begin{equation*} 
%             \rho(X) = \sup_{\mu \in \distort} \, \<\mu, X_{\uparrow}\>, \quad \forall X \in \Re^n
%         \end{equation*}
%     \end{enumerate}
% \end{proposition}
% \begin{proof}
%     The fact that $\distort$ is convex and that \emph{(ii)} holds follow
%     directly from \cref{prop:fundamental-equivalence-lemma} and \cref{prop:conjugate-duality}. 
%     Similarly $\distort$ should contain all permutations of at least one vector $\mu$, since otherwise $\dom\rho^*$ is 
%     empty by \cref{prop:fundamental-equivalence-lemma}. This can only be the case when $\mu_{\uparrow} \in \distort$ by \cref{lem:opposite-hull-contained}. 
%     So $\distort \cap \M^n \neq \emptyset$. 

%     Meanwhile \emph{(i)} follows from \cref{prop:fundamental-equivalence-lemma}
%     and $(\dom \rho^* + (\M^n)^\circ) + (\M^n)^\circ = \dom \rho^* + (\M^n)^\circ$ since $(\M^n)^\circ$ is a convex cone.

%     Let $\bar{\rho} = \rho + \iota_{\Re^n}$ as in \cref{lem:infimal-convolution}.
%     Then $\distort = \mathrm{dom}(\bar{\rho}^*)$. 
%     By permutation invariance $\bar{\rho}(X_{\uparrow}) = \rho(X_{\uparrow}) = \rho(X)$ for all $X \in \Re^n$. 
%     We can then prove \emph{(iii)} by noting that $(X + a)_{\uparrow} = X_{\uparrow} + a$. 
%     Hence \cohasm{3} holds for $\bar{\rho}$
%     iff it holds for $\rho(X) = \bar{\rho}(X_{\uparrow})$. Thus we can apply \cref{prop:conjugate-duality:3} 
%     to $\bar{\rho}$ in order to link \cohasm{3} with $\mathrm{dom}(\rho^{\diamond}) = \mathrm{dom}(\bar{\rho}^{*})$. 
%     For \emph{(iv)} we use a similar argument noting that $(\alpha X)_{\uparrow} = \alpha X_{\uparrow}$ for all $X \in \Re^n$
%     and all $\alpha \geq 0$ and apply \cref{prop:conjugate-duality:4}. The only property of $\rho$ 
%     that does not transfer to $\bar{\rho}$ is monotonicity. So \emph{(ii)} is more complex. 
% \end{proof}

% We conclude with an anolog to \cref{cor:ambiguity}, defining the \emph{distortion representation} of a law-invariant, coherent risk measure.
% The relation with the ambiguity representation from \cref{prop:fundamental-equivalence-lemma} is presented graphically in \cref{fig:graphical-equivalence}.
% \begin{corollary}
%     Some risk $\rho \colon \Re^n \to \eRe$ is law-invariant and coherent iff 
%     \begin{equation} \label{eq:distortion-representation}
%         \rho(X) = \sup_{\mu \in \distort} \, \<\mu, X_{\uparrow}\>
%     \end{equation}
%     for a closed and convex \emph{distortion set} $\distort$ with
%     \begin{enumerate*}[(i)]
%         \item $\one_n/n \in \distort$;
%         \item $\mu + (\M^n)^\circ \subseteq \distort$ for all $\mu \in \distort$;
%         \item $\sum_{i=1}^{n} \mu_i = 1$;
%         \item any $\mu$ s.t. $\pi(\mu) \in \distort$ for all $\pi \in \Pi_n$ is nonnegative.
%     \end{enumerate*}
% \end{corollary}
% \begin{proof}
%     Any coherent risk measure has $\dom(\rho) = \Re^n$.
%     So $\rho$ is automatically monotone-proper. The other facts then follow from \cref{prop:ordered-conjugate-duality},
%     with the exception of \emph{(i)}.  
%     To prove this final fact note that for any $\mu \in \dom\rho^*$ we have $\hull(\{\pi(\mu) \colon \pi \in \Pi_n\}) \subseteq \dom \rho^*$,
%     by law-invariance and convexity. Let $E \in \Re^{n \times n}$ denote the matrix of all ones. 
%     Then $E\mu/n \in \hull(\{\pi(\mu) \colon \pi \in \Pi_n\})$. Since $\sum_{i=1}^{n} \mu_i = 1$ we have 
%     $E\mu/n = \one_n /n$. So if $\one_n /n \notin \dom \rho^*$ then $\dom \rho^*$ 
%     has to be empty by contradiction. Thus $\distort$ should contain $\one_n/n$ since otherwise $\dom\rho^* = \emptyset$ 
%     by \cref{prop:fundamental-equivalence-lemma}. 
% \end{proof}

% \begin{figure}
%     \tikzset{
    add hatch/.style 2 args={
        pattern=hatch, 
        pattern color=#1, 
        hatch size=5pt, 
        hatch angle=#2,
        hatch line width=0.1pt
    }
}

\begin{tikzpicture}
    \definecolor{crimson}{RGB}{214,39,40}
    \definecolor{steelblue}{RGB}{31,119,180}

    \begin{scope}[scale=3.2]
        % \path [semithick, draw=steelblue, add hatch={steelblue}{0}]
        % (0.3334,0.0)
        % --(-0.5,0.0)
        % --(-0.25,0.433)
        % --(0.0833,0.433)
        % --cycle;

        % \path [draw=gray, add hatch={gray}{-60}]
        % (0.0,0.0)
        % --(-0.5,0.0)
        % --(-0.0833,0.722)
        % --(0.1667,0.2887)
        % --cycle;

        % \path [draw=black, add hatch={black}{60}]
        % (-0.3334,0.0)
        % --(0.5,0.0)
        % --(0.25,0.433)
        % --(-0.08333,0.433)
        % --cycle;

        \path [draw=gray, add hatch={gray!50!white}{-120}]
        (-0.41667,0.1443)
        --(0,0.866)
        --(0.25,0.433)
        --(0.08334,0.1443)
        --cycle;

        % \path [draw=gray, add hatch={gray}{120}]
        % (0.0,0.0)
        % --(0.5,0.0)
        % --(0.08334,0.722)
        % --(-0.1667,0.2887)
        % --cycle;

        % \path [draw=gray, add hatch={gray}{180}]
        % (-0.25,0.433)
        % --(0,0.866)
        % --(0.41667,0.1443)
        % --(-0.08333,0.14434)
        % --cycle;

        \path [semithick, draw=steelblue, fill=steelblue!30!white, fill opacity=0.6]
        (0.3334,0.0)
        --(-0.5,0.0)
        --(-0.25,0.433)
        --(0.0833,0.433)
        --cycle;

        \path [thick, draw=crimson, fill=crimson!30!white]
        (0.1667,0.2887)
        --(0.0833,0.1443)
        --(-0.08333,0.14434)
        --(-0.1667,0.2887)
        --(-0.08333,0.433)
        --(0.0833,0.433)
        --cycle;

        \node [align=center, text=crimson] (amb) at (0, 0.2887) {$\amb$};
        \node [above right, xshift=5pt, align=center, text=steelblue] (dist) at (-0.5, 0.0) {$\distort$};
        \node [below, yshift=-15pt, align=center, text=gray] (permute) at (0.0, 0.866) {$\pi(\distort)$};

        \path [draw=black, fill=none] (-0.5, 0.0) -- (0.5, 0.0) -- ( 0.0, 0.866) -- cycle;
    \end{scope}

    \coordinate (box-east) at (current bounding box.east);
    \coordinate (box-north east) at (current bounding box.north east);
    \coordinate (box-south east) at (current bounding box.south east);
    
    % \node[anchor=north west, xshift=1cm, text=steelblue] at (box-north east) {$\distort$};
    % \node[anchor=south west, xshift=1cm, text=crimson] at (box-south east) {$\amb$};
    \node[anchor=west, xshift=0.2cm, yshift=0.2cm] (a) at (box-east) {${\color{crimson}\amb} = \bigcap_{\pi \in \Pi^n} \pi({\color{steelblue}\distort})$};
    \node[anchor=north west, yshift=-0.1cm, align=left] (b) at (a.south west) {${\color{steelblue}\distort} = {\color{crimson}\amb} + (\M^n)^\circ$};

    \draw [crimson, very thin, -{Latex[length=2mm]}] (amb.north) |- (a.west);
    \draw [steelblue, very thin, -{Latex[length=2mm]}] (dist.east) -| ([xshift=-0.7cm]b.west) -- (b.west);
\end{tikzpicture}
%     \vspace{1em}
%     \centering
%     \caption{Graphical depiction of \cref{prop:fundamental-equivalence-lemma} for $n = 3$.
%     The intersection with the simplex $\Delta^n$ is shown. Let $\amb \dfn \mathrm{dom}(\rho^*)$, 
%     $\distort \dfn \mathrm{dom}(\rho^{\diamond})$.} \label{fig:graphical-equivalence}
% \end{figure}

