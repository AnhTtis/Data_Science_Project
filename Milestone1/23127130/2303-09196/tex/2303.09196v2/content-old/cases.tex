\section{Case Studies} \label{sec:case-studies} 

To illustrate the validity and potential of our method we provide several simple case studies. These 
are convex for maximum interpretability, as in the non-convex case a worse generalization performance might be caused by local optima. 
Nonetheless our method is also applicable in non-convex settings, where stochastic gradient
descent methods can be used (cf. \cite{Mehta2022} for simple distortions and \cite{Chouzenoux2019} for divergences).
In the convex case we use duality to reformulate the proxy cost in \cref{eq:ordered-risk}. See \cite{Schuurmans2023,Bertsimas2009b} for details.

We present problems of the form \cref{eq:erm} and employ ordered risk minimization \cref{eq:ordered-risk} 
or using \cref{eq:biased-cvar}, which we refer to as $\bar{\CVAR}$. For divergences we use either the \emph{total variation (TV)} or
\emph{Kullback-Leibler (KL)} and the radius is calibrated using \cref{thm:divergence-bound} (with $\beta = 0.005$ and $m=10\,000$).
The value of $\gamma$ in \cref{eq:biased-cvar} is calibrated using \cref{eq:calibration-cvar}.

% Throughout this section we will consider the following risk measures: \emph{(SAA)} takes $\rho(X) = \sum_{i=1}^{n} X_i/n$; 
% \emph{($\bar{\CVAR}$)} takes $\rho = \bar{\CVAR}^n_{\gamma}$ with $\gamma$ determined such that $\rho$ is $\varepsilon$-calibrated
% using \cref{rem:order-statistics-bound} and \cite{Moscovich2020} or using \cref{eq:asymptotic} when $n \geq 100$; \emph{(TV)} and \emph{(KL)} take $\rho$ 
% as a \emph{total-variation} and a \emph{Kullback-Leibler} $\phi$-divergence risk respectively and both are $\varepsilon$-calibrated as in \cref{thm:divergence-bound} with $\beta = 0.005$ and $m = 10\,000$. 

\subsection{Newsvendor}
We begin with a toy problem, illustrating the behavior of our method in low-sample settings. 
Let $\xi \colon \Omega \to \Re$ be Beta distributed with $\alpha = 0.1$, $\beta = 0.2$, scaled by a factor $\bar{D} \dfn 100$. 
Consider a newsvendor problem \cite[\S1.2.1]{Shapiro2021}:
\begin{equation*} \label{eq:newsvendor}
    \begin{alignedat}{2}
        &\minimize_{\theta \in \Re} & \quad & \E\left[\smashunderbracket{c\theta + b[\xi - \theta]_+ +  h[\theta - \xi]_+}{\ell(\theta, \xi)}\right],
    \end{alignedat}
\end{equation*}
with $b = 14$, $h = 2$ amd $c = 1$. For samples $\{\xi_i\}_{i=1}^{n-1}$ with $n=20$ let $\ell_i(\theta) = \ell(\theta, \xi_i)$ 
for $i \in [n-1]$, $\ell_n(\theta) = \max\{(c - b)\theta + b \bar{D}, (c + h)\theta\}$ a robust upper bound. 
We replace the expectation by a data-driven proxy as described at the start of the section.
For the \emph{sample average approach (SAA)} we take $\sum_{i=1}^{n-1} \ell_i(\theta)/(n-1)$.

\begin{figure}
    \centering
    \vspace{0.2em}
    \begin{figure}[t]
    \centering
    %\includegraphics[width=\textwidth]{figures/pipeline.png}
    \includegraphics[width=.44\textwidth]{figures/boxes2.png}
    \vspace{-10pt}
    \caption{\textbf{Visualization of the point annotation and automatic bounding box generation from points.} The red point represents the mean of the five annotation points. The points annotation captures diverse patterns in various action types.}%\bc{This is directly modified by MCN, will modify the figure soon}}
    % \label{fig:pipeline}
   \label{fig:boxes}
   \vspace{-10pt}
    
\end{figure}\vspace{-0.5em}
    \caption{Box plots showing newsvendor expected cost (left); and difference between the predicted cost and expected cost (right). The colored area is the \emph{inter-quartile range (IQR)}, while the whiskers show the 
    range of samples truncated to $1.5$ times the IQR. Outliers outside of this range are depicted as diamonds. The red dashed lines depict the robust performance. The blue dashed line is the optimal cost.} \label{fig:newsvendor-example}%
    \vspace{-1em}
\end{figure}


The calibration problems are solved for $\delta = 0.2$. 
Their performance is compared over $200$ sampled data sets in \cref{fig:newsvendor-example}.
The left plot shows the actual expected cost for the minimizers.
The blue dashed line is the true optimum of \cref{eq:newsvendor}. See \cite[\S1.2.1]{Shapiro2021} for details on how to compute these values. 
Note how the SAA performs decently in the median, but has significantly more variance. The outliers above $240$ were omitted, the largest of which was $428.2$. 
Moreover, the right plot depicts the difference between the optimum value of the proxy cost, and the true cost. 
The SAA often underestimates its true cost, while our methods overestimate it. The dashed red line depicts the behavior when taking $\amb = \Delta^n$
in \cref{eq:ordered-risk} (cf. \cite[Eq.~1.9]{Shapiro2021}). As we almost never perform worse than this robust method, 
this shows that our methods learn from data without over-fitting on the sample. 

In large sample cases, we can use the largest sample as an approximation of $\ell_n(\theta)$. 
This heuristic is similar to the one used in the scenario approach, the consequences of which have been studied in detail (cf.\ \cite{Ramponi2018}).
In combination with some regularization, this significantly boosts 
the performance of our method, as shown in the next \ilarxiv{examples}\ilpub{example}.


\begin{arxiv}
\subsection{Regression}
Let $T_k \colon \Re \to \Re$ denote the Chebychev polynomials of the first kind for $k \geq 0$ and $f_d(x) = (T_k(x))_{k=0}^{d} \in \Re^{d+1}$ 
a feature vector. Consider a lasso regression problem:
\begin{equation} \label{eq:regression}
    \begin{alignedat}{2}
        &\minimize_{\theta \in \Re^{d+1}} &\qquad & \E\left[(\<f_d(X), \theta\> - Y)^2\right] + \lambda \nrm{\theta}_1.
    \end{alignedat}    
\end{equation}
Assuming access to samples $\{(X_i, Y_i)\}_{i=1}^n$, we replace the expectation with the proxy costs described above,
where $\ell_i(\theta) = (\<f_d(X_i), \theta\> - Y_i)^2$ for $i \in [n]$. 
So we approximate the robust term with the largest sample.

For the parameters $\theta_\star = (0, 0, 0.2, 0.5, 1.0)$ the data is generated as $Y_i = \<f_4(X_i), \theta_\star\> + E_i$ with $X_i \deq \mathcal{U}(-1, 1)$ 
and $E_i \deq \mathcal{U}(-0.2, 0.2)$ for $i \in [n]$. We over-parametrize the problem, taking $d = 20$, to illustrate the regularizing effect of our method.
A fit is plotted for $\lambda = 0.2$ and $n = 50$ in \cref{fig:regression-example}. Note how the risk measures all perform similarly, 
while SAA has a worse fit.

\begin{figure}
    \centering
    \begin{table}[t!]
\setlength{\tabcolsep}{2.5pt}
\caption{Detailed results over downstream regression tasks. The best results among fair representation learning baselines are highlighted in bold.
}
\centering
\label{table:regressionresults}
\resizebox{1\columnwidth}{!}{
\begin{tabular}{@{}lccc|ccc@{}}
\toprule
\multicolumn{1}{l}{\multirow{2}{*}{Method}} & \multicolumn{3}{c|}{Students} & \multicolumn{3}{c}{Communities} \\ \cmidrule(l){2-7} 
\multicolumn{1}{c}{} &  \multicolumn{1}{c}{RMSE ($\downarrow$)} & \multicolumn{1}{c}{$\Delta DP$ ($\downarrow$)} & \multicolumn{1}{c|}{$\Delta CP$ ($\downarrow$)} &  \multicolumn{1}{c}{RMSE} & \multicolumn{1}{c}{$\Delta DP$} & \multicolumn{1}{c}{$\Delta CP$} \\ \midrule
LR & 0.0392 & 0.0177 & 0.0794 & 0.0145 & 0.2146 & 0.1559 \\
C-LR & 0.0399 & 0.0081 & 0.0777 & 0.0141 & 0.1643 & 0.0854 \\
SCARF & 0.0746 & 0.0196 & 0.0412 & 0.0821 & 0.1631 & 0.0881 \\ \midrule
VFAE & 0.0425 & 0.0084 & \textbf{0.0519} & \textbf{0.0191} & 0.1714 & 0.0844 \\
LAFTR & 0.0371 & 0.0223 & 0.0758 & 0.0194 & 0.1725 & 0.1311 \\
MIFR & 0.0475 & 0.0056 & 0.0785 & 0.0261 & 0.1139 & 0.0916 \\
L-MIFR & 0.0485 & 0.0063 & 0.0797 & 0.0258 & 0.1161 & 0.0902 \\
C-InfoNCE & 0.0415 & \textbf{0.0049} & 0.0626 & 0.0317 & 0.0841 & 0.0884 \\
WeaC-InfoNCE & 0.0407 & 0.0054 & 0.0676 & 0.0319 & \textbf{0.0813} & 0.0891 \\ \midrule
\textbf{DualFair} & \textbf{0.0382} & 0.0054 & 0.0735 & 0.0247 & 0.1130 & \textbf{0.0816} \\ \bottomrule
\end{tabular}
}
\end{table}
    \caption{Regression using $n = 50$ samples with $d=20$ and $\lambda = 0.2$ for different risk measures.} \label{fig:regression-example}
    \vspace{-1em}
\end{figure}

The methods are evaluated quantitatively by sampling an additional $100\,000$ data points and computing a sample approximation of the cost of \cref{eq:regression}.
The resulting performance is compared for several tunings in \cref{tab:regression-quantitative}, where any parameters not mentioned are kept as specified above. 
It is of note that our methods are significantly less sensitive to tuning parameters compared to the SAA. In fact, our methods outperform SAA for all tunings investigated. 


{\setlength{\tabcolsep}{4pt}\tiny
\begin{table}
    \vspace{1.2em}
   \begin{tabular}{lllllll}
    \toprule
    &  &  & SAA & TV & $\bar{\CVAR}$ & KL \\
    $\varepsilon$ & $d$ & $\lambda$ &  &  &  &  \\
    \midrule
    \multirow[t]{10}{*}{$0.05$} & \multirow[t]{5}{*}{$10$} & $0.001$ & $0.019\,(03)$ & $0.018\,(02)$ & $0.018\,(02)$ & $0.018\,(02)$ \\
    &  & $0.01$ & $0.018\,(05)$ & $0.017\,(04)$ & $0.017\,(04)$ & $0.017\,(04)$ \\
    &  & $0.05$ & $0.023\,(05)$ & $0.017\,(05)$ & $0.017\,(05)$ & $0.018\,(05)$ \\
    &  & $0.2$ & $0.073\,(13)$ & $0.034\,(07)$ & $0.035\,(07)$ & $0.042\,(04)$ \\
    % &  & $0.5$ & $0.278\,(37)$ & $0.108\,(13)$ & $0.112\,(13)$ & $0.137\,(19)$ \\
    & \multirow[t]{5}{*}{$20$} & $0.001$ & $0.023\,(03)$ & $0.023\,(04)$ & $0.023\,(04)$ & $0.023\,(03)$ \\
    &  & $0.01$ & $0.019\,(05)$ & $0.019\,(04)$ & $0.019\,(04)$ & $0.019\,(04)$ \\
    &  & $0.05$ & $0.024\,(05)$ & $0.018\,(05)$ & $0.018\,(05)$ & $0.018\,(05)$ \\
    &  & $0.2$ & $0.073\,(13)$ & $0.034\,(07)$ & $0.035\,(07)$ & $0.042\,(04)$ \\
    % &  & $0.5$ & $0.278\,(37)$ & $0.108\,(13)$ & $0.112\,(13)$ & $0.137\,(19)$ \\
    \midrule
    % \multirow[t]{10}{*}{$0.1$} & \multirow[t]{5}{*}{$10$} & $0.001$ & $0.019\,(03)$ & $0.018\,(02)$ & $0.018\,(02)$ & $0.018\,(02)$ \\
    % &  & $0.01$ & $0.018\,(05)$ & $0.017\,(04)$ & $0.017\,(04)$ & $0.017\,(04)$ \\
    % &  & $0.05$ & $0.023\,(05)$ & $0.017\,(05)$ & $0.018\,(05)$ & $0.018\,(05)$ \\
    % &  & $0.2$ & $0.073\,(13)$ & $0.037\,(07)$ & $0.038\,(06)$ & $0.044\,(05)$ \\
    % &  & $0.5$ & $0.278\,(37)$ & $0.120\,(14)$ & $0.124\,(14)$ & $0.147\,(21)$ \\
    % & \multirow[t]{5}{*}{$20$} & $0.001$ & $0.023\,(03)$ & $0.023\,(04)$ & $0.023\,(04)$ & $0.023\,(03)$ \\
    % &  & $0.01$ & $0.019\,(05)$ & $0.019\,(04)$ & $0.019\,(04)$ & $0.019\,(04)$ \\
    % &  & $0.05$ & $0.024\,(05)$ & $0.018\,(05)$ & $0.018\,(05)$ & $0.018\,(04)$ \\
    % &  & $0.2$ & $0.073\,(13)$ & $0.037\,(07)$ & $0.038\,(06)$ & $0.044\,(05)$ \\
    % &  & $0.5$ & $0.278\,(37)$ & $0.120\,(14)$ & $0.124\,(14)$ & $0.147\,(21)$ \\
    % \midrule
    \multirow[t]{10}{*}{$0.2$} & \multirow[t]{5}{*}{$10$} & $0.001$ & $0.019\,(03)$ & $0.019\,(02)$ & $0.019\,(02)$ & $0.018\,(02)$ \\
    &  & $0.01$ & $0.018\,(05)$ & $0.017\,(04)$ & $0.017\,(04)$ & $0.017\,(04)$ \\
    &  & $0.05$ & $0.023\,(05)$ & $0.018\,(05)$ & $0.018\,(05)$ & $0.018\,(05)$ \\
    &  & $0.2$ & $0.073\,(13)$ & $0.040\,(06)$ & $0.041\,(06)$ & $0.047\,(06)$ \\
    % &  & $0.5$ & $0.278\,(37)$ & $0.137\,(16)$ & $0.142\,(17)$ & $0.162\,(24)$ \\
    & \multirow[t]{5}{*}{$20$} & $0.001$ & $0.023\,(03)$ & $0.023\,(03)$ & $0.023\,(03)$ & $0.023\,(03)$ \\
    &  & $0.01$ & $0.019\,(05)$ & $0.019\,(04)$ & $0.019\,(04)$ & $0.019\,(04)$ \\
    &  & $0.05$ & $0.024\,(05)$ & $0.018\,(04)$ & $0.018\,(04)$ & $0.018\,(04)$ \\
    &  & $0.2$ & $0.073\,(13)$ & $0.040\,(06)$ & $0.041\,(06)$ & $0.047\,(06)$ \\
    % &  & $0.5$ & $0.278\,(37)$ & $0.137\,(16)$ & $0.142\,(17)$ & $0.162\,(24)$ \\
    \bottomrule
   \end{tabular}
   \centering
   \caption{Regression generalization performance for various tuning parameters. Values are reported as \emph{mean (standard deviation $\cdot\, 10^{3}$)}
   computed over $10$ training sets. The same $10$ sets were used for every selection of parameters and method. Note that $\varepsilon$ 
   does not affect SAA.} \label{tab:regression-quantitative} \vspace{-1em}
\end{table}
}
\end{arxiv}

\subsection{Support Vector Machines}
Consider a classification problem with $X \deq \mathcal{N}(0, I_2)$ normally distributed and $Y = 1$ if $X_{1} X_{2} \geq 0$ 
and $Y = -1$ otherwise%\footnote{Inspired by \url{https://scikit-learn.org/stable/auto_examples/svm/plot_svm_nonlinear.html}}
. A \emph{Support Vector Machine (SVM)}
solves:
\begin{equation*}
    \begin{alignedat}{2}
        &\minimize_{(f, b) \in \set{H} \times \Re} &\qquad & \frac{1}{2} \nrm{f}_{\set{H}}^2 + \lambda\E\left[ 1 - Y (f(X) - b) \right]_+
    \end{alignedat}
\end{equation*}
with $\lambda > 0$ and $\set{H}$ some \emph{reproducing kernel Hilbert Space (RKHS)} \cite[Def.~2.9]{Scholkopf2002}. The 
resulting classifier is then given by $\mathrm{sign}(f(X) - b)$. Henceforth $\set{H}$ is the RKHS associated with the \emph{radial basis function} kernel 
\cite[\S2.3]{Scholkopf2002} with some standard deviation $\sigma$.
Solving the primal problem is difficult for two reasons: \emph{(i)} 
the true expectation is often unknown; \emph{(ii)} optimizing over the infinite dimensional $\set{H}$ is intractable in general. 
We resolve \emph{(i)} by replacing the expectation with a proxy-cost as described above and \emph{(ii)} through the usual duality trick \cite[\S7.4]{Scholkopf2002}.
Details are deferred to \ilarxiv{\cref{app:svm}}\ilpub{\cite[App.~B]{TR}}. 

The proxy cost of three of the risks above -- SAA, TV and $\bar{\CVAR}$ -- is a maximum of linear functions and the dual problem is a QP. 
The sample average -- C-SVC in \cite[\S7.5]{Scholkopf2002} -- is the usual choice. We illustrate the superior performance our calibrated risks. 

\begin{figure}
    \vspace{1em}
    \centering
    \def\svgwidth{0.85\columnwidth}
    %% Creator: Inkscape inkscape 0.92.3, www.inkscape.org
%% PDF/EPS/PS + LaTeX output extension by Johan Engelen, 2010
%% Accompanies image file 'assets/_svm.pdf' (pdf, eps, ps)
%%
%% To include the image in your LaTeX document, write
%%   \input{<filename>.pdf_tex}
%%  instead of
%%   \includegraphics{<filename>.pdf}
%% To scale the image, write
%%   \def\svgwidth{<desired width>}
%%   \input{<filename>.pdf_tex}
%%  instead of
%%   \includegraphics[width=<desired width>]{<filename>.pdf}
%%
%% Images with a different path to the parent latex file can
%% be accessed with the `import' package (which may need to be
%% installed) using
%%   \usepackage{import}
%% in the preamble, and then including the image with
%%   \import{<path to file>}{<filename>.pdf_tex}
%% Alternatively, one can specify
%%   \graphicspath{{<path to file>/}}
%% 
%% For more information, please see info/svg-inkscape on CTAN:
%%   http://tug.ctan.org/tex-archive/info/svg-inkscape
%%
\begingroup%
  \makeatletter%
  \providecommand\color[2][]{%
    \errmessage{(Inkscape) Color is used for the text in Inkscape, but the package 'color.sty' is not loaded}%
    \renewcommand\color[2][]{}%
  }%
  \providecommand\transparent[1]{%
    \errmessage{(Inkscape) Transparency is used (non-zero) for the text in Inkscape, but the package 'transparent.sty' is not loaded}%
    \renewcommand\transparent[1]{}%
  }%
  \providecommand\rotatebox[2]{#2}%
  \newcommand*\fsize{\dimexpr\f@size pt\relax}%
  \newcommand*\lineheight[1]{\fontsize{\fsize}{#1\fsize}\selectfont}%
  \ifx\svgwidth\undefined%
    \setlength{\unitlength}{576bp}%
    \ifx\svgscale\undefined%
      \relax%
    \else%
      \setlength{\unitlength}{\unitlength * \real{\svgscale}}%
    \fi%
  \else%
    \setlength{\unitlength}{\svgwidth}%
  \fi%
  \global\let\svgwidth\undefined%
  \global\let\svgscale\undefined%
  \ifx\svgfont\undefined%
  \global\let\svgfont\footnotesize
  \fi%    
  \makeatother%
  \begin{picture}(1,0.3125)%
    \lineheight{1}%
    \setlength\tabcolsep{0pt}%
    \put(0,0){\includegraphics[width=\unitlength,page=1]{assets/_svm.pdf}}%
    \put(0.08892572,0.015){\makebox(0,0)[t]{\lineheight{1.25}\smash{\begin{tabular}[t]{c}\svgfont{}-2\end{tabular}}}}%
    \put(0,0){\includegraphics[width=\unitlength,page=2]{assets/_svm.pdf}}%
    \put(0.15403411,0.015){\makebox(0,0)[t]{\lineheight{1.25}\smash{\begin{tabular}[t]{c}\svgfont{}0\end{tabular}}}}%
    \put(0,0){\includegraphics[width=\unitlength,page=3]{assets/_svm.pdf}}%
    \put(0.21914249,0.015){\makebox(0,0)[t]{\lineheight{1.25}\smash{\begin{tabular}[t]{c}\svgfont{}2\end{tabular}}}}%
    \put(0,0){\includegraphics[width=\unitlength,page=4]{assets/_svm.pdf}}%
    \put(0.04421875,0.07804534){\makebox(0,0)[rt]{\lineheight{1.25}\smash{\begin{tabular}[t]{r}\svgfont{}-2\end{tabular}}}}%
    \put(0,0){\includegraphics[width=\unitlength,page=5]{assets/_svm.pdf}}%
    \put(0.04421875,0.15052219){\makebox(0,0)[rt]{\lineheight{1.25}\smash{\begin{tabular}[t]{r}\svgfont{}0\end{tabular}}}}%
    \put(0,0){\includegraphics[width=\unitlength,page=6]{assets/_svm.pdf}}%
    \put(0.04421875,0.22299904){\makebox(0,0)[rt]{\lineheight{1.25}\smash{\begin{tabular}[t]{r}\svgfont{}2\end{tabular}}}}%
    \put(0,0){\includegraphics[width=\unitlength,page=7]{assets/_svm.pdf}}%
    \put(0.15403411,0.27625){\makebox(0,0)[t]{\lineheight{1.25}\smash{\begin{tabular}[t]{c}{}SAA\end{tabular}}}}%
    \put(0,0){\includegraphics[width=\unitlength,page=8]{assets/_svm.pdf}}%
    \put(0.41600906,0.015){\makebox(0,0)[t]{\lineheight{1.25}\smash{\begin{tabular}[t]{c}\svgfont{}-2\end{tabular}}}}%
    \put(0,0){\includegraphics[width=\unitlength,page=9]{assets/_svm.pdf}}%
    \put(0.48111746,0.015){\makebox(0,0)[t]{\lineheight{1.25}\smash{\begin{tabular}[t]{c}\svgfont{}0\end{tabular}}}}%
    \put(0,0){\includegraphics[width=\unitlength,page=10]{assets/_svm.pdf}}%
    \put(0.54622581,0.015){\makebox(0,0)[t]{\lineheight{1.25}\smash{\begin{tabular}[t]{c}\svgfont{}2\end{tabular}}}}%
    \put(0,0){\includegraphics[width=\unitlength,page=11]{assets/_svm.pdf}}%
    \put(0.37130207,0.07804534){\makebox(0,0)[rt]{\lineheight{1.25}\smash{\begin{tabular}[t]{r}\svgfont{}-2\end{tabular}}}}%
    \put(0,0){\includegraphics[width=\unitlength,page=12]{assets/_svm.pdf}}%
    \put(0.37130207,0.15052219){\makebox(0,0)[rt]{\lineheight{1.25}\smash{\begin{tabular}[t]{r}\svgfont{}0\end{tabular}}}}%
    \put(0,0){\includegraphics[width=\unitlength,page=13]{assets/_svm.pdf}}%
    \put(0.37130207,0.22299904){\makebox(0,0)[rt]{\lineheight{1.25}\smash{\begin{tabular}[t]{r}\svgfont{}2\end{tabular}}}}%
    \put(0,0){\includegraphics[width=\unitlength,page=14]{assets/_svm.pdf}}%
    \put(0.48111746,0.27625){\makebox(0,0)[t]{\lineheight{1.25}\smash{\begin{tabular}[t]{c}{}TV\end{tabular}}}}%
    \put(0,0){\includegraphics[width=\unitlength,page=15]{assets/_svm.pdf}}%
    \put(0.74309238,0.015){\makebox(0,0)[t]{\lineheight{1.25}\smash{\begin{tabular}[t]{c}\svgfont{}-2\end{tabular}}}}%
    \put(0,0){\includegraphics[width=\unitlength,page=16]{assets/_svm.pdf}}%
    \put(0.80820078,0.015){\makebox(0,0)[t]{\lineheight{1.25}\smash{\begin{tabular}[t]{c}\svgfont{}0\end{tabular}}}}%
    \put(0,0){\includegraphics[width=\unitlength,page=17]{assets/_svm.pdf}}%
    \put(0.87330919,0.015){\makebox(0,0)[t]{\lineheight{1.25}\smash{\begin{tabular}[t]{c}\svgfont{}2\end{tabular}}}}%
    \put(0,0){\includegraphics[width=\unitlength,page=18]{assets/_svm.pdf}}%
    \put(0.6983854,0.07804534){\makebox(0,0)[rt]{\lineheight{1.25}\smash{\begin{tabular}[t]{r}\svgfont{}-2\end{tabular}}}}%
    \put(0,0){\includegraphics[width=\unitlength,page=19]{assets/_svm.pdf}}%
    \put(0.6983854,0.15052219){\makebox(0,0)[rt]{\lineheight{1.25}\smash{\begin{tabular}[t]{r}\svgfont{}0\end{tabular}}}}%
    \put(0,0){\includegraphics[width=\unitlength,page=20]{assets/_svm.pdf}}%
    \put(0.6983854,0.22299904){\makebox(0,0)[rt]{\lineheight{1.25}\smash{\begin{tabular}[t]{r}\svgfont{}2\end{tabular}}}}%
    \put(0,0){\includegraphics[width=\unitlength,page=21]{assets/_svm.pdf}}%
    \put(0.80820078,0.27625){\makebox(0,0)[t]{\lineheight{1.25}\smash{\begin{tabular}[t]{c}{}$\bar{\CVAR}$\end{tabular}}}}%
    \put(0,0){\includegraphics[width=\unitlength,page=22]{assets/_svm.pdf}}%
    \put(0.28692881,0.07081707){\makebox(0,0)[lt]{\lineheight{1.25}\smash{\begin{tabular}[t]{l}\svgfont{}-10\end{tabular}}}}%
    \put(0,0){\includegraphics[width=\unitlength,page=23]{assets/_svm.pdf}}%
    \put(0.28692881,0.13394156){\makebox(0,0)[lt]{\lineheight{1.25}\smash{\begin{tabular}[t]{l}\svgfont{}0\end{tabular}}}}%
    \put(0,0){\includegraphics[width=\unitlength,page=24]{assets/_svm.pdf}}%
    \put(0.28692881,0.19706604){\makebox(0,0)[lt]{\lineheight{1.25}\smash{\begin{tabular}[t]{l}\svgfont{}10\end{tabular}}}}%
    \put(0,0){\includegraphics[width=\unitlength,page=25]{assets/_svm.pdf}}%
    \put(0.61401214,0.07997168){\makebox(0,0)[lt]{\lineheight{1.25}\smash{\begin{tabular}[t]{l}\svgfont{}-10\end{tabular}}}}%
    \put(0,0){\includegraphics[width=\unitlength,page=26]{assets/_svm.pdf}}%
    \put(0.61401214,0.1417839){\makebox(0,0)[lt]{\lineheight{1.25}\smash{\begin{tabular}[t]{l}\svgfont{}0\end{tabular}}}}%
    \put(0,0){\includegraphics[width=\unitlength,page=27]{assets/_svm.pdf}}%
    \put(0.61401214,0.20359611){\makebox(0,0)[lt]{\lineheight{1.25}\smash{\begin{tabular}[t]{l}\svgfont{}10\end{tabular}}}}%
    \put(0,0){\includegraphics[width=\unitlength,page=28]{assets/_svm.pdf}}%
    \put(0.94109546,0.08010896){\makebox(0,0)[lt]{\lineheight{1.25}\smash{\begin{tabular}[t]{l}\svgfont{}-10\end{tabular}}}}%
    \put(0,0){\includegraphics[width=\unitlength,page=29]{assets/_svm.pdf}}%
    \put(0.94109546,0.14185908){\makebox(0,0)[lt]{\lineheight{1.25}\smash{\begin{tabular}[t]{l}\svgfont{}0\end{tabular}}}}%
    \put(0,0){\includegraphics[width=\unitlength,page=30]{assets/_svm.pdf}}%
    \put(0.94109546,0.2036092){\makebox(0,0)[lt]{\lineheight{1.25}\smash{\begin{tabular}[t]{l}\svgfont{}10\end{tabular}}}}%
    \put(0,0){\includegraphics[width=\unitlength,page=31]{assets/_svm.pdf}}%
  \end{picture}%
\endgroup%

    \caption{SVM classifiers trained using $n = 250$ samples with $\sigma=0.25$ and $\lambda = 10^4$ for different risks. The red and blue markers are samples for $Y=1$ and $-1$ respectively. 
    The line is the decision boundary and the color axis depicts $f(X) - b$.} \label{fig:svm-example}
    \vspace{-0.5em}
\end{figure}

In \cref{fig:svm-example}, the three classifiers produced by the three proxy costs above are depicted. 
Note how both TV and $\bar{\CVAR}$ perform similarly and both visibly better than the usual SAA. 
% \ilpub{In further experiments in the technical report \cite{TR}, the performance is compared for several tunings. It is of note that our methods are
% significantly less sensitive to tuning parameters compared to the SAA and outperform it for most parameter values.}%
% \begin{arxiv}%
Quantitative performance is compared through the fraction of 
incorrectly labeled samples in a test set of $10^5$ samples, which 
we refer to as the misclassification rate. The performance is compared for several tunings in \cref{tab:svm-quantitative}, 
where any parameters not mentioned are kept as specified above. It is of note that our methods are significantly less 
sensitive to tuning parameters compared to the SAA. In fact, even for the tunings where SAA performs best, our methods 
perform better for the same tuning, for reasonable choices of $\delta$. 

{\setlength{\tabcolsep}{4pt}
\begin{table}
    \vspace{1.2em}
    \centering
    % \begin{tabular}{llllll}
%     \toprule
%     &  &  & SAA & TV & $\bar{\CVAR}$ \\
%    $\delta$ & $\sigma$ & $\lambda$ &  &  &  \\
%    \midrule
%    \multirow[t]{9}{*}{$0.05$} & \multirow[t]{3}{*}{$0.05$} & $10^4$ & $0.438\,(0.048)$ & $0.140\,(0.137)$ & $0.135\,(0.082)$ \\
%    &  & $10^6$ & $0.168\,(0.056)$ & $0.025\,(0.011)$ & $0.027\,(0.013)$ \\
%    &  & $10^8$ & $0.044\,(0.012)$ & $0.015\,(0.005)$ & $0.018\,(0.005)$ \\
%    & \multirow[t]{3}{*}{$0.25$} & $10^4$ & $0.085\,(0.033)$ & $0.023\,(0.008)$ & $0.020\,(0.010)$ \\
%    &  & $10^6$ & $0.033\,(0.011)$ & $0.016\,(0.006)$ & $0.015\,(0.006)$ \\
%    &  & $10^8$ & $0.020\,(0.010)$ & $0.020\,(0.009)$ & $0.021\,(0.009)$ \\
%    & \multirow[t]{3}{*}{$0.5$} & $10^4$ & $0.047\,(0.018)$ & $0.020\,(0.010)$ & $0.020\,(0.008)$ \\
%    &  & $10^6$ & $0.025\,(0.009)$ & $0.022\,(0.011)$ & $0.022\,(0.011)$ \\
%    &  & $10^8$ & $0.028\,(0.013)$ & $0.029\,(0.013)$ & $0.027\,(0.013)$ \\
%    \midrule
%   \multirow[t]{9}{*}{$0.2$} & \multirow[t]{3}{*}{$0.05$} & $10^4$ & $0.438\,(0.048)$ & $0.208\,(0.113)$ & $0.207\,(0.118)$ \\
%    &  & $10^6$ & $0.168\,(0.056)$ & $0.036\,(0.019)$ & $0.043\,(0.023)$ \\
%    &  & $10^8$ & $0.044\,(0.012)$ & $0.016\,(0.007)$ & $0.018\,(0.007)$ \\
%    & \multirow[t]{3}{*}{$0.25$} & $10^4$ & $0.085\,(0.033)$ & $0.032\,(0.017)$ & $0.024\,(0.014)$ \\
%    &  & $10^6$ & $0.033\,(0.011)$ & $0.015\,(0.006)$ & $0.015\,(0.006)$ \\
%    &  & $10^8$ & $0.020\,(0.010)$ & $0.022\,(0.010)$ & $0.020\,(0.009)$ \\
%    & \multirow[t]{3}{*}{$0.5$} & $10^4$ & $0.047\,(0.018)$ & $0.024\,(0.008)$ & $0.021\,(0.009)$ \\
%    &  & $10^6$ & $0.025\,(0.009)$ & $0.021\,(0.011)$ & $0.020\,(0.010)$ \\
%    &  & $10^8$ & $0.028\,(0.013)$ & $0.028\,(0.013)$ & $0.028\,(0.013)$ \\
%    \bottomrule
%    \end{tabular}


   \begin{tabular}{llllll}
    \toprule
    &  &  & SAA & TV & $\bar{\CVAR}$ \\
   $\delta$ & $\sigma$ & $\lambda$ &  &  &  \\
   \midrule
   \multirow[t]{20}{*}{$0.01$} & \multirow[t]{4}{*}{$0.01$} & $10^4$ & $0.500\,(0.002)$ & $0.212\,(0.055)$ & $0.108\,(0.043)$ \\
    &  & $10^6$ & $0.471\,(0.050)$ & $0.152\,(0.184)$ & $0.066\,(0.023)$ \\
    &  & $10^8$ & $0.362\,(0.038)$ & $0.423\,(0.158)$ & $0.277\,(0.195)$ \\
    &  & $10^{10}$ & $0.081\,(0.033)$ & $0.035\,(0.019)$ & $0.072\,(0.040)$ \\
    & \multirow[t]{4}{*}{$0.05$} & $10^4$ & $0.474\,(0.052)$ & $0.109\,(0.135)$ & $0.080\,(0.037)$ \\
    &  & $10^6$ & $0.170\,(0.057)$ & $0.038\,(0.021)$ & $0.046\,(0.026)$ \\
    &  & $10^8$ & $0.053\,(0.020)$ & $0.018\,(0.008)$ & $0.018\,(0.008)$ \\
    &  & $10^{10}$ & $0.039\,(0.029)$ & $0.043\,(0.020)$ & $0.027\,(0.015)$ \\
    & \multirow[t]{4}{*}{$0.25$} & $10^4$ & $0.097\,(0.047)$ & $0.038\,(0.030)$ & $0.033\,(0.027)$ \\
    &  & $10^6$ & $0.042\,(0.012)$ & $0.016\,(0.003)$ & $0.016\,(0.003)$ \\
    &  & $10^8$ & $0.022\,(0.010)$ & $0.019\,(0.005)$ & $0.017\,(0.004)$ \\
    &  & $10^{10}$ & $0.067\,(0.082)$ & $0.084\,(0.128)$ & $0.033\,(0.021)$ \\
    & \multirow[t]{4}{*}{$0.5$} & $10^4$ & $0.055\,(0.021)$ & $0.020\,(0.005)$ & $0.020\,(0.005)$ \\
    &  & $10^6$ & $0.030\,(0.008)$ & $0.020\,(0.004)$ & $0.020\,(0.004)$ \\
    &  & $10^8$ & $0.023\,(0.008)$ & $0.024\,(0.007)$ & $0.023\,(0.007)$ \\
    &  & $10^{10}$ & $0.202\,(0.140)$ & $0.128\,(0.128)$ & $0.163\,(0.136)$ \\
    & \multirow[t]{4}{*}{$0.75$} & $10^4$ & $0.047\,(0.014)$ & $0.022\,(0.006)$ & $0.022\,(0.006)$ \\
    &  & $10^6$ & $0.031\,(0.011)$ & $0.026\,(0.008)$ & $0.026\,(0.008)$ \\
    &  & $10^8$ & $0.036\,(0.017)$ & $0.035\,(0.014)$ & $0.031\,(0.010)$ \\
    &  & $10^{10}$ & $0.302\,(0.149)$ & $0.273\,(0.134)$ & $0.249\,(0.134)$ \\
    \midrule
   \multirow[t]{20}{*}{$0.05$} & \multirow[t]{4}{*}{$0.01$} & $10^4$ & $0.500\,(0.002)$ & $0.256\,(0.103)$ & $0.146\,(0.054)$ \\
    &  & $10^6$ & $0.471\,(0.050)$ & $0.208\,(0.184)$ & $0.103\,(0.055)$ \\
    &  & $10^8$ & $0.362\,(0.038)$ & $0.418\,(0.176)$ & $0.192\,(0.164)$ \\
    &  & $10^{10}$ & $0.081\,(0.033)$ & $0.041\,(0.024)$ & $0.064\,(0.040)$ \\
    & \multirow[t]{4}{*}{$0.05$} & $10^4$ & $0.474\,(0.052)$ & $0.126\,(0.100)$ & $0.111\,(0.054)$ \\
    &  & $10^6$ & $0.170\,(0.057)$ & $0.052\,(0.035)$ & $0.052\,(0.041)$ \\
    &  & $10^8$ & $0.053\,(0.020)$ & $0.020\,(0.008)$ & $0.022\,(0.011)$ \\
    &  & $10^{10}$ & $0.039\,(0.029)$ & $0.027\,(0.013)$ & $0.022\,(0.017)$ \\
    & \multirow[t]{4}{*}{$0.25$} & $10^4$ & $0.097\,(0.047)$ & $0.045\,(0.038)$ & $0.038\,(0.035)$ \\
    &  & $10^6$ & $0.042\,(0.012)$ & $0.019\,(0.004)$ & $0.016\,(0.003)$ \\
    &  & $10^8$ & $0.022\,(0.010)$ & $0.019\,(0.004)$ & $0.018\,(0.003)$ \\
    &  & $10^{10}$ & $0.067\,(0.082)$ & $0.061\,(0.114)$ & $0.043\,(0.045)$ \\
    & \multirow[t]{4}{*}{$0.5$} & $10^4$ & $0.055\,(0.021)$ & $0.023\,(0.007)$ & $0.023\,(0.006)$ \\
    &  & $10^6$ & $0.030\,(0.008)$ & $0.020\,(0.003)$ & $0.020\,(0.004)$ \\
    &  & $10^8$ & $0.023\,(0.008)$ & $0.024\,(0.007)$ & $0.024\,(0.007)$ \\
    &  & $10^{10}$ & $0.202\,(0.140)$ & $0.145\,(0.110)$ & $0.114\,(0.110)$ \\
    & \multirow[t]{4}{*}{$0.75$} & $10^4$ & $0.047\,(0.014)$ & $0.022\,(0.005)$ & $0.022\,(0.006)$ \\
    &  & $10^6$ & $0.031\,(0.011)$ & $0.026\,(0.007)$ & $0.026\,(0.007)$ \\
    &  & $10^8$ & $0.036\,(0.017)$ & $0.035\,(0.016)$ & $0.031\,(0.010)$ \\
    &  & $10^{10}$ & $0.302\,(0.149)$ & $0.290\,(0.112)$ & $0.252\,(0.139)$ \\
    \midrule
   \multirow[t]{20}{*}{$0.1$} & \multirow[t]{4}{*}{$0.01$} & $10^4$ & $0.500\,(0.002)$ & $0.329\,(0.097)$ & $0.168\,(0.078)$ \\
    &  & $10^6$ & $0.471\,(0.050)$ & $0.301\,(0.215)$ & $0.142\,(0.077)$ \\
    &  & $10^8$ & $0.362\,(0.038)$ & $0.248\,(0.178)$ & $0.184\,(0.127)$ \\
    &  & $10^{10}$ & $0.081\,(0.033)$ & $0.042\,(0.026)$ & $0.057\,(0.033)$ \\
    & \multirow[t]{4}{*}{$0.05$} & $10^4$ & $0.474\,(0.052)$ & $0.154\,(0.093)$ & $0.145\,(0.075)$ \\
    &  & $10^6$ & $0.170\,(0.057)$ & $0.059\,(0.048)$ & $0.060\,(0.046)$ \\
    &  & $10^8$ & $0.053\,(0.020)$ & $0.022\,(0.011)$ & $0.021\,(0.012)$ \\
    &  & $10^{10}$ & $0.039\,(0.029)$ & $0.030\,(0.021)$ & $0.030\,(0.020)$ \\
    & \multirow[t]{4}{*}{$0.25$} & $10^4$ & $0.097\,(0.047)$ & $0.046\,(0.042)$ & $0.041\,(0.036)$ \\
    &  & $10^6$ & $0.042\,(0.012)$ & \ccell[gray]{0.9}{$0.015\,(0.003)$} & \ccell[gray]{0.9}{$0.015\,(0.003)$} \\
    &  & $10^8$ & $0.022\,(0.010)$ & $0.020\,(0.005)$ & $0.018\,(0.004)$ \\
    &  & $10^{10}$ & $0.067\,(0.082)$ & $0.076\,(0.119)$ & $0.037\,(0.039)$ \\
    & \multirow[t]{4}{*}{$0.5$} & $10^4$ & $0.055\,(0.021)$ & $0.025\,(0.009)$ & $0.024\,(0.009)$ \\
    &  & $10^6$ & $0.030\,(0.008)$ & $0.021\,(0.004)$ & $0.021\,(0.005)$ \\
    &  & $10^8$ & $0.023\,(0.008)$ & $0.024\,(0.007)$ & $0.024\,(0.007)$ \\
    &  & $10^{10}$ & $0.202\,(0.140)$ & $0.145\,(0.097)$ & $0.112\,(0.110)$ \\
    & \multirow[t]{4}{*}{$0.75$} & $10^4$ & $0.047\,(0.014)$ & $0.022\,(0.006)$ & $0.022\,(0.006)$ \\
    &  & $10^6$ & $0.031\,(0.011)$ & $0.026\,(0.007)$ & $0.026\,(0.007)$ \\
    &  & $10^8$ & $0.036\,(0.017)$ & $0.034\,(0.013)$ & $0.031\,(0.010)$ \\
    &  & $10^{10}$ & $0.302\,(0.149)$ & $0.286\,(0.127)$ & $0.257\,(0.144)$ \\
    \midrule
   \multirow[t]{20}{*}{$0.2$} & \multirow[t]{4}{*}{$0.01$} & $10^4$ & $0.500\,(0.002)$ & $0.361\,(0.079)$ & $0.235\,(0.094)$ \\
    &  & $10^6$ & $0.471\,(0.050)$ & $0.328\,(0.180)$ & $0.198\,(0.106)$ \\
    &  & $10^8$ & $0.362\,(0.038)$ & $0.217\,(0.131)$ & $0.184\,(0.083)$ \\
    &  & $10^{10}$ & $0.081\,(0.033)$ & $0.046\,(0.023)$ & $0.061\,(0.035)$ \\
    & \multirow[t]{4}{*}{$0.05$} & $10^4$ & $0.474\,(0.052)$ & $0.225\,(0.142)$ & $0.197\,(0.100)$ \\
    &  & $10^6$ & $0.170\,(0.057)$ & $0.074\,(0.050)$ & $0.075\,(0.051)$ \\
    &  & $10^8$ & $0.053\,(0.020)$ & $0.025\,(0.015)$ & $0.026\,(0.015)$ \\
    &  & $10^{10}$ & $0.039\,(0.029)$ & $0.033\,(0.023)$ & $0.036\,(0.026)$ \\
    & \multirow[t]{4}{*}{$0.25$} & $10^4$ & $0.097\,(0.047)$ & $0.049\,(0.044)$ & $0.045\,(0.040)$ \\
    &  & $10^6$ & $0.042\,(0.012)$ & $0.019\,(0.005)$ & $0.016\,(0.003)$ \\
    &  & $10^8$ & $0.022\,(0.010)$ & $0.017\,(0.004)$ & $0.017\,(0.004)$ \\
    &  & $10^{10}$ & $0.067\,(0.082)$ & $0.070\,(0.113)$ & $0.039\,(0.033)$ \\
    & \multirow[t]{4}{*}{$0.5$} & $10^4$ & $0.055\,(0.021)$ & $0.027\,(0.014)$ & $0.027\,(0.015)$ \\
    &  & $10^6$ & $0.030\,(0.008)$ & $0.021\,(0.005)$ & $0.020\,(0.005)$ \\
    &  & $10^8$ & $0.023\,(0.008)$ & $0.024\,(0.007)$ & $0.024\,(0.007)$ \\
    &  & $10^{10}$ & $0.202\,(0.140)$ & $0.186\,(0.144)$ & $0.112\,(0.101)$ \\
    & \multirow[t]{4}{*}{$0.75$} & $10^4$ & $0.047\,(0.014)$ & $0.022\,(0.005)$ & $0.022\,(0.006)$ \\
    &  & $10^6$ & $0.031\,(0.011)$ & $0.026\,(0.007)$ & $0.026\,(0.007)$ \\
    &  & $10^8$ & $0.036\,(0.017)$ & $0.033\,(0.012)$ & $0.031\,(0.010)$ \\
    &  & $10^{10}$ & $0.302\,(0.149)$ & $0.289\,(0.129)$ & $0.250\,(0.140)$ \\
    \midrule
   \multirow[t]{20}{*}{$0.25$} & \multirow[t]{4}{*}{$0.01$} & $10^4$ & $0.500\,(0.002)$ & $0.365\,(0.113)$ & $0.257\,(0.124)$ \\
    &  & $10^6$ & $0.471\,(0.050)$ & $0.359\,(0.170)$ & $0.228\,(0.133)$ \\
    &  & $10^8$ & $0.362\,(0.038)$ & $0.221\,(0.100)$ & $0.199\,(0.099)$ \\
    &  & $10^{10}$ & $0.081\,(0.033)$ & $0.053\,(0.030)$ & $0.070\,(0.044)$ \\
    & \multirow[t]{4}{*}{$0.05$} & $10^4$ & $0.474\,(0.052)$ & $0.243\,(0.141)$ & $0.224\,(0.133)$ \\
    &  & $10^6$ & $0.170\,(0.057)$ & $0.077\,(0.051)$ & $0.078\,(0.054)$ \\
    &  & $10^8$ & $0.053\,(0.020)$ & $0.026\,(0.017)$ & $0.026\,(0.016)$ \\
    &  & $10^{10}$ & $0.039\,(0.029)$ & $0.038\,(0.019)$ & $0.030\,(0.025)$ \\
    & \multirow[t]{4}{*}{$0.25$} & $10^4$ & $0.097\,(0.047)$ & $0.050\,(0.044)$ & $0.048\,(0.043)$ \\
    &  & $10^6$ & $0.042\,(0.012)$ & $0.019\,(0.004)$ & $0.016\,(0.003)$ \\
    &  & $10^8$ & $0.022\,(0.010)$ & $0.018\,(0.005)$ & $0.017\,(0.004)$ \\
    &  & $10^{10}$ & $0.067\,(0.082)$ & $0.054\,(0.067)$ & $0.038\,(0.024)$ \\
    & \multirow[t]{4}{*}{$0.5$} & $10^4$ & $0.055\,(0.021)$ & $0.029\,(0.015)$ & $0.028\,(0.016)$ \\
    &  & $10^6$ & $0.030\,(0.008)$ & $0.021\,(0.004)$ & $0.020\,(0.005)$ \\
    &  & $10^8$ & $0.023\,(0.008)$ & $0.024\,(0.007)$ & $0.023\,(0.007)$ \\
    &  & $10^{10}$ & $0.202\,(0.140)$ & $0.149\,(0.116)$ & $0.111\,(0.106)$ \\
    & \multirow[t]{4}{*}{$0.75$} & $10^4$ & $0.047\,(0.014)$ & $0.023\,(0.007)$ & $0.022\,(0.005)$ \\
    &  & $10^6$ & $0.031\,(0.011)$ & $0.026\,(0.007)$ & $0.026\,(0.007)$ \\
    &  & $10^8$ & $0.036\,(0.017)$ & $0.034\,(0.012)$ & $0.031\,(0.010)$ \\
    &  & $10^{10}$ & $0.302\,(0.149)$ & $0.267\,(0.124)$ & $0.236\,(0.137)$ \\
   \end{tabular}
   
    \caption{%
%    \ilarxiv{%
    SVM misclassification rates for various tuning parameters. Reported values are the \emph{mean (standard deviation)}
    over $10$ training sets. The same $10$ sets were used for every parameter selection and method. Note that $\delta$ 
    does not affect SAA. The lowest values in a column are bold. Observe that in the rows where SAA achieves its best performance, our methods still perform better.%
%    }%
    } \label{tab:svm-quantitative} \vspace{-1.5em}
\end{table}
}
% \end{arxiv}

\begin{figure}
    \centering
    \begin{tikzpicture}
    \definecolor{darkorange_}{RGB}{255,127,14}
    \definecolor{forestgreen_}{RGB}{44,160,44}
    \definecolor{steelblue_}{RGB}{31,119,180}
    
    \begin{axis}[
        numeric axis,
        xmode=log,
        xmin=10, xmax=1000,
        ymin=0, ymax=0.5,
        legend style={at={(axis cs: 1000, 0.6)}, anchor=north east, draw=none, yshift=-6pt, xshift=-6pt},
        legend cell align={left},
        xlabel={$n$},
        ylabel={misclassification rate},
        width=\columnwidth,
        height=0.4\columnwidth,
        darkorange/.style={darkorange_, dash dot},
        forestgreen/.style={forestgreen_, dashed},
        steelblue/.style={steelblue_},
    ]
    \addlegendimage{steelblue}
    \addlegendentry{\footnotesize{}SAA};
    \addlegendimage{darkorange}
    \addlegendentry{\footnotesize{}TV};
    \addlegendimage{forestgreen}
    \addlegendentry{\footnotesize{}$\bar{\CVAR}$};

    \addplot [draw=steelblue_, fill=steelblue_, mark=-, only marks]
    table{%
    x  y
    10 0.18885
    10 0.39347
    };
    \addplot [draw=steelblue_, fill=steelblue_, mark=-, only marks]
    table{%
    x  y
    15 0.19874
    15 0.31073
    };
    \addplot [draw=steelblue_, fill=steelblue_, mark=-, only marks]
    table{%
    x  y
    23 0.16067
    23 0.24473
    };
    \addplot [draw=steelblue_, fill=steelblue_, mark=-, only marks]
    table{%
    x  y
    35 0.12376
    35 0.19851
    };
    \addplot [draw=steelblue_, fill=steelblue_, mark=-, only marks]
    table{%
    x  y
    53 0.11064
    53 0.18576
    };
    \addplot [draw=steelblue_, fill=steelblue_, mark=-, only marks]
    table{%
    x  y
    81 0.09463
    81 0.17309
    };
    \addplot [draw=steelblue_, fill=steelblue_, mark=-, only marks]
    table{%
    x  y
    123 0.07884
    123 0.15004
    };
    \addplot [draw=steelblue_, fill=steelblue_, mark=-, only marks]
    table{%
    x  y
    187 0.07128
    187 0.13723
    };
    \addplot [draw=steelblue_, fill=steelblue_, mark=-, only marks]
    table{%
    x  y
    284 0.05175
    284 0.11733
    };
    \addplot [draw=steelblue_, fill=steelblue_, mark=-, only marks]
    table{%
    x  y
    432 0.05331
    432 0.10888
    };
    \addplot [draw=steelblue_, fill=steelblue_, mark=-, only marks]
    table{%
    x  y
    657 0.04509
    657 0.09446
    };
    \addplot [draw=steelblue_, fill=steelblue_, mark=-, only marks]
    table{%
    x  y
    1000 0.04217
    1000 0.0867
    };
    \addplot [draw=darkorange_, fill=darkorange_, mark=-, only marks]
    table{%
    x  y
    10 0.16592
    10 0.39681
    };
    \addplot [draw=darkorange_, fill=darkorange_, mark=-, only marks]
    table{%
    x  y
    15 0.13697
    15 0.28749
    };
    \addplot [draw=darkorange_, fill=darkorange_, mark=-, only marks]
    table{%
    x  y
    23 0.10255
    23 0.17783
    };
    \addplot [draw=darkorange_, fill=darkorange_, mark=-, only marks]
    table{%
    x  y
    35 0.07374
    35 0.13799
    };
    \addplot [draw=darkorange_, fill=darkorange_, mark=-, only marks]
    table{%
    x  y
    53 0.04261
    53 0.10511
    };
    \addplot [draw=darkorange_, fill=darkorange_, mark=-, only marks]
    table{%
    x  y
    81 0.03782
    81 0.09301
    };
    \addplot [draw=darkorange_, fill=darkorange_, mark=-, only marks]
    table{%
    x  y
    123 0.02718
    123 0.06968
    };
    \addplot [draw=darkorange_, fill=darkorange_, mark=-, only marks]
    table{%
    x  y
    187 0.02711
    187 0.0468
    };
    \addplot [draw=darkorange_, fill=darkorange_, mark=-, only marks]
    table{%
    x  y
    284 0.0188
    284 0.04456
    };
    \addplot [draw=darkorange_, fill=darkorange_, mark=-, only marks]
    table{%
    x  y
    432 0.01779
    432 0.04326
    };
    \addplot [draw=darkorange_, fill=darkorange_, mark=-, only marks]
    table{%
    x  y
    657 0.00834000000000001
    657 0.02489
    };
    \addplot [draw=darkorange_, fill=darkorange_, mark=-, only marks]
    table{%
    x  y
    1000 0.00788999999999995
    1000 0.02727
    };
    \addplot [draw=forestgreen_, fill=forestgreen_, mark=-, only marks]
    table{%
    x  y
    10 0.16397
    10 0.39681
    };
    \addplot [draw=forestgreen_, fill=forestgreen_, mark=-, only marks]
    table{%
    x  y
    15 0.13622
    15 0.28749
    };
    \addplot [draw=forestgreen_, fill=forestgreen_, mark=-, only marks]
    table{%
    x  y
    23 0.11745
    23 0.18719
    };
    \addplot [draw=forestgreen_, fill=forestgreen_, mark=-, only marks]
    table{%
    x  y
    35 0.07229
    35 0.1427
    };
    \addplot [draw=forestgreen_, fill=forestgreen_, mark=-, only marks]
    table{%
    x  y
    53 0.04505
    53 0.10902
    };
    \addplot [draw=forestgreen_, fill=forestgreen_, mark=-, only marks]
    table{%
    x  y
    81 0.03645
    81 0.09358
    };
    \addplot [draw=forestgreen_, fill=forestgreen_, mark=-, only marks]
    table{%
    x  y
    123 0.02934
    123 0.07226
    };
    \addplot [draw=forestgreen_, fill=forestgreen_, mark=-, only marks]
    table{%
    x  y
    187 0.02444
    187 0.05278
    };
    \addplot [draw=forestgreen_, fill=forestgreen_, mark=-, only marks]
    table{%
    x  y
    284 0.01676
    284 0.0448
    };
    \addplot [draw=forestgreen_, fill=forestgreen_, mark=-, only marks]
    table{%
    x  y
    432 0.01941
    432 0.04194
    };
    \addplot [draw=forestgreen_, fill=forestgreen_, mark=-, only marks]
    table{%
    x  y
    657 0.00814000000000004
    657 0.02967
    };
    \addplot [draw=forestgreen_, fill=forestgreen_, mark=-, only marks]
    table{%
    x  y
    1000 0.00602999999999998
    1000 0.03102
    };
    \addplot [steelblue]
    table {%
    10 0.18885
    10 0.39347
    };
    \addplot [steelblue]
    table {%
    15 0.19874
    15 0.31073
    };
    \addplot [steelblue]
    table {%
    23 0.16067
    23 0.24473
    };
    \addplot [steelblue]
    table {%
    35 0.12376
    35 0.19851
    };
    \addplot [steelblue]
    table {%
    53 0.11064
    53 0.18576
    };
    \addplot [steelblue]
    table {%
    81 0.09463
    81 0.17309
    };
    \addplot [steelblue]
    table {%
    123 0.07884
    123 0.15004
    };
    \addplot [steelblue]
    table {%
    187 0.07128
    187 0.13723
    };
    \addplot [steelblue]
    table {%
    284 0.05175
    284 0.11733
    };
    \addplot [steelblue]
    table {%
    432 0.05331
    432 0.10888
    };
    \addplot [steelblue]
    table {%
    657 0.04509
    657 0.09446
    };
    \addplot [steelblue]
    table {%
    1000 0.04217
    1000 0.0867
    };
    \addplot [steelblue]
    table {%
    10 0.289249666666667
    15 0.245928666666667
    23 0.193936333333333
    35 0.166404
    53 0.143886
    81 0.131428333333333
    123 0.113370666666667
    187 0.0979253333333333
    284 0.081105
    432 0.077855
    657 0.0665683333333333
    1000 0.0620413333333333
    };
    \addplot [darkorange]
    table {%
    10 0.16592
    10 0.39681
    };
    \addplot [darkorange]
    table {%
    15 0.13697
    15 0.28749
    };
    \addplot [darkorange]
    table {%
    23 0.10255
    23 0.17783
    };
    \addplot [darkorange]
    table {%
    35 0.07374
    35 0.13799
    };
    \addplot [darkorange]
    table {%
    53 0.04261
    53 0.10511
    };
    \addplot [darkorange]
    table {%
    81 0.03782
    81 0.09301
    };
    \addplot [darkorange]
    table {%
    123 0.02718
    123 0.06968
    };
    \addplot [darkorange]
    table {%
    187 0.02711
    187 0.0468
    };
    \addplot [darkorange]
    table {%
    284 0.0188
    284 0.04456
    };
    \addplot [darkorange]
    table {%
    432 0.01779
    432 0.04326
    };
    \addplot [darkorange]
    table {%
    657 0.00834000000000001
    657 0.02489
    };
    \addplot [darkorange]
    table {%
    1000 0.00788999999999995
    1000 0.02727
    };
    \addplot [darkorange]
    table {%
    10 0.273421
    15 0.202971666666667
    23 0.145454333333333
    35 0.10497
    53 0.0682196666666667
    81 0.0654166666666667
    123 0.049121
    187 0.0404373333333333
    284 0.0333993333333333
    432 0.0311653333333333
    657 0.0191353333333333
    1000 0.0194006666666667
    };
    \addplot [forestgreen]
    table {%
    10 0.16397
    10 0.39681
    };
    \addplot [forestgreen]
    table {%
    15 0.13622
    15 0.28749
    };
    \addplot [forestgreen]
    table {%
    23 0.11745
    23 0.18719
    };
    \addplot [forestgreen]
    table {%
    35 0.07229
    35 0.1427
    };
    \addplot [forestgreen]
    table {%
    53 0.04505
    53 0.10902
    };
    \addplot [forestgreen]
    table {%
    81 0.03645
    81 0.09358
    };
    \addplot [forestgreen]
    table {%
    123 0.02934
    123 0.07226
    };
    \addplot [forestgreen]
    table {%
    187 0.02444
    187 0.05278
    };
    \addplot [forestgreen]
    table {%
    284 0.01676
    284 0.0448
    };
    \addplot [forestgreen]
    table {%
    432 0.01941
    432 0.04194
    };
    \addplot [forestgreen]
    table {%
    657 0.00814000000000004
    657 0.02967
    };
    \addplot [forestgreen]
    table {%
    1000 0.00602999999999998
    1000 0.03102
    };
    \addplot [forestgreen]
    table {%
    10 0.274501666666667
    15 0.202802333333333
    23 0.149837666666667
    35 0.105479333333333
    53 0.0759403333333333
    81 0.0632976666666667
    123 0.0510246666666667
    187 0.0385383333333333
    284 0.0336413333333333
    432 0.030228
    657 0.0202433333333333
    1000 0.0183866666666667
    };
    \end{axis}
    
    \end{tikzpicture}
    \vspace{-0.5em}
    \caption{SVM misclassification rates for varying sample counts $n$. 
    The center line depicts the mean, while the intervals depicts the empirical $0.2$-confidence interval.} \label{fig:svm-complexity} \vspace{-1.5em}
\end{figure}

We can also examine the effect of varying the sample count $n$. For each such value we train the classifiers, again using the parameters used to produce \cref{fig:svm-example}, 
for $30$ training sets. The resulting misclassification rates are depicted in \cref{fig:svm-complexity}. Again note that $\bar{\CVAR}$ and TV both outperform SAA. 

