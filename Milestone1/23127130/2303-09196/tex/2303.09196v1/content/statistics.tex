\section{Statistical Framework} \label{sec:statistical-framework}
In this section we find upper bounds for $\E[\ell(\theta, \xi)]$ that hold point-wise over $\theta$. These are related to classical results in order statistics. 
The bounds are presented to highlight the connection with the \emph{distortion representation} of law-invariant, coherent risk measures described in the next section. 
To simplify the notation, we focus on scalar random variables $Z \colon \Omega \to \Re$ defined on some abstract probability space $(\Omega, \F, \prob)$. 
Specifically, we fix the value of $\theta$ and take $Z = \ell(\theta, \xi)$. 
Then consider an \iid{} sample $(Z_i)_{i=1}^{n-1}  = (\ell(\theta, \xi_i))_{i=1}^{n-1}$ and some upper bound $\esssup[Z] \leq Z_{n} < +\infty$ -- one could take $Z_n = \sup_{\xi \in \Xi} \ell(\theta, \xi)$ --
with order statistic $Z_{(1)} \leq \dots \leq Z_{(n-1)} \leq \esssup[Z] \leq Z_{(n)}$.
We assume throughout that the expectation $\E[Z]$ is well defined. 

Throughout the section we assume that $Z$ is a continuous random variable for clarity of the exposition. The final theorem however holds for any random variable. 
For continuous $Z$ we have the following \cite[Thm.~8.7.4]{Wilks1964}.
\begin{proposition} \label{thm:elementary-coverage}
    Let $Z_{(1)} \leq \dots \leq Z_{(n-1)}$ be the order statistics of an \iid{} sample from a continuous random variable $Z$. 
    Moreover let $Z_{(0)} = -\infty$ and $Z_{(n)} \geq \esssup[Z]$. Then the random vector of so-called \emph{coverages}:
    \begin{equation*}
        W \dfn \left(F(Z_{(i)}) - F(Z_{(i-1)})\right)_{i=1}^{n}
    \end{equation*}
    is uniformly distributed over $\Delta^{n}$ with $F(z) = \prob[Z \leq z]$. 
\end{proposition}

We exploit this property by writing the expectation of $Z$ in terms of these 
coverages using the law of total expectation
\begin{align*}
    \E[Z] = \int_{-\infty}^{Z_{(n)}} Z \di F(Z) = \sum_{i=1}^{n} \int_{Z_{(i-1)}}^{Z_{(i)}} Z \di F(Z).
\end{align*}
The equality holds for any realization of the data. For each term
$\int_{Z_{(i-1)}}^{Z_(i)} Z \di F(Z) \leq Z_{(i)} \int_{Z_{(i-1)}}^{Z_(i)} \di F(Z) = Z_{(i)} W_i$. Hence
\begin{equation} \label{eq:known-distortion-bound}
    \E[Z] \leq \sum_{i=1}^{n} Z_{(i)} W_i = \<W, \hat{Z}_{\uparrow}\>,
\end{equation}
for any realization of the data. Computing the value of $W$ when only $Z_1, \dots, Z_n$ is known
is not possible in practice. Instead consider some set $\distort$ 
such that 
\begin{equation} \label{eq:confidence-distortion}
    \prob[W \in \distort] = 1 - \varepsilon   
\end{equation}
and take the supremum over its elements in \cref{eq:known-distortion-bound}. 

Such a confidence level is similar to that of ambiguity sets in classical data-driven DRO. There
$\prob[\mu_{\star} \in \amb] \geq 1 - \varepsilon$, which places randomness inside the data-driven $\amb$, 
while the true distribution $\mu_{\star}$ of $\xi$ in \cref{eq:expected-risk-minimization} is fixed and unknown. Instead in \cref{eq:confidence-distortion} $\distort$ is fixed and $W$ acts as the 
random variable, with known distribution by \cref{thm:elementary-coverage}. Thus calibrating $\distort$ is easier compared to calibrating $\set{A}$, which
requires conservative concentration inequalities. 

We summarize this discussion in a theorem.
\begin{theorem} \label{thm:distortion-bound}
    Given an \iid{} sample $\{Z_i\}_{i=1}^{n-1}$ from some random variable $Z$ and some $Z_{n} \geq \esssup[Z] < +\infty$,
    let $\hat{Z} = (Z_1, \dots, Z_n) \in \Re^{n}$. Consider some non-empty $\distort \subseteq \Re^n$ such that $\distort + (\M^n)^\circ = \distort$.
    Then
    \begin{equation} \label{eq:distortion-bound}
        \prob\left[ \sup_{\mu \in \distort} \<\mu, \hat{Z}_{\uparrow}\> \geq \E[Z] \right] \geq \prob[W \in \distort],
    \end{equation}
    for $W$ uniformly distributed over the simplex $\Delta^{n}$.
\end{theorem}
\begin{proof}
    Proof deferred to \cref{app:statistical-framework}.
\end{proof}

Using the notation from the introduction, take $\hat{Z} = L(\theta)$ and reflect on the similarity of the $\sup$ to \cref{eq:robustified-erm}. 
The difference is that $L(\theta)$ is ordered, which leads to the \emph{distortion representation} of risk measures studied in the next section. 
Invariance $\distort + (\M^n)^\circ = \distort$ enables non-continuous random variables $Z$. 
Interestingly the distortion representation derived below automatically satisfies this property. 
