%%%%%%%%%%%%%%%%%%%%%%%%%%%%%%%%%%%%%%%%%%%%%%%%%%%%%%%%%%%%%%%%%%%%%%%%%%%%%%%%%%%%%%%%%%%%%%%%%%%%%%%%%%%%%%%%%%%%%%%%
%23456789012345678901234567890123456789012345678901234567890123456789012345678901234567890123456789012345678901234567890
%        1         2         3         4         5         6         7         8         9         10        11       12

\newcommand{\thetitle}{Robustified Empirical Risk Minimization with Law-Invariant, Coherent Risk Measures}

\documentclass[todo=false, usebiblatex=false, pub=true, color=true]{kulconf}                       % Comment this line out if you need a4paper
\configuretemplate{P. Coppens}{\thetitle}

\usepackage[arxiv=true]{arxivix}
\usepackage{tikzutils}
\usepackage{booktabs, multirow, tabularx}
\usepackage{array}
\usepackage[ruled, linesnumbered]{algorithm2e}

% bibliography
% \addbibresource{IEEEabrv.bib}
\addbibresource{bibliography.bib}

%In case you encounter the following error:
%Error 1010 The PDF file may be corrupt (unable to open PDF file) OR
%Error 1000 An error occurred while parsing a contents stream. Unable to analyze the PDF file.
%This is a known problem with pdfLaTeX conversion filter. The file cannot be opened with acrobat reader
%Please use one of the alternatives below to circumvent this error by uncommenting one or the other
%\pdfobjcompresslevel=0
% \pdfminorversion=4

% See the \addtolength command later in the file to balance the column lengths
% on the last page of the document


\title{\thetitle}
\author{Peter Coppens and Panagiotis Patrinos$^\dagger$
\thanks{$^\dagger$P. Coppens and P. Patrinos are with the Department of Electrical
Engineering (ESAT-STADIUS), KU Leuven, Kasteelpark Arenberg
10, 3001 Leuven, Belgium.
        {Email: \tt\footnotesize peter.coppens@kuleuven.be, panos.patrinos@kuleuven.be}}%
\thanks{This work was supported by: the Research Foundation
Flanders (FWO) PhD grant 11E5520N and research projects G0A0920N, G086518N and G086318N;
Research Council KU Leuven C1 project No. C14/18/068; Fonds de la Recherche 
Scientifique - FNRS and the FWO - Vlaanderen under 
EOS project no 30468160 (SeLMA); EU's Horizon 2020 research and innovation programme: Marie Skłodowska-Curie grant No. 953348.}%
}%


%%%%%%%%%%%%%%%%%%%%%%%%%%%%%%%%%%%%%%%%%%%%%%%%%%%%%%%%%%%%%%%%%%%%%%%%%%%%%%%%%%%%%%%%%%%%%%%%%%%%%%%%%%%%%%%%%%%%%%%%
% notation

%%%%%%%%%%%%%%%%%%%%%%%%%%%%%%%%%%%%%%%%%%%%%%%%%%%%%%%%%%%%%%%%%%%%%%%%%%%%%%%%%%%%%%%%%%%%%%%%%%%%%%%%%%%%%%%%%%%%%%%%


\begin{document}

\maketitle
\thispagestyle{empty}
\pagestyle{empty}


%%%%%%%%%%%%%%%%%%%%%%%%%%%%%%%%%%%%%%%%%%%%%%%%%%%%%%%%%%%%%%%%%%%%%%%%%%%%%%%%%%%%%%%%%%%%%%%%%%%%%%%%%%%%%%%%%%%%%%%%
\begin{abstract}
    In this work we consider law-invariant, coherent risk measures as a proxy-cost for data-driven expected risk 
    minimization. We show how such risks serve as point-wise high-confidence upper bounds of the expected risk. 
    The confidence level can be determined tightly for any number of samples. Conversely we also illustrate how to 
    calibrate risk measures to act as a high-confidence upper bound with some user specified confidence. Numerical 
    experiments then illustrate how the resulting proxy-cost both generalizes better and is less sensitive to tuning 
    parameters compared to the usual empirical risk minimization approach. 
\end{abstract}


\begin{pub}
\begin{IEEEkeywords}
\todo*{Placeholder keywords.}
\end{IEEEkeywords}
\end{pub}

%%%%%%%%%%%%%%%%%%%%%%%%%%%%%%%%%%%%%%%%%%%%%%%%%%%%%%%%%%%%%%%%%%%%%%%%%%%%%%%%%%%%%%%%%%%%%%%%%%%%%%%%%%%%%%%%%%%%%%%%

\section{Introduction}

The increasing complexity of source code poses a key challenge to the reliability of large-scale software systems. Software bugs in these systems can lead to safety issues~\cite{bug_safety} for users around the world as well as cause non-negligible financial losses~\cite{bug_loss}. As such, developers have to spend a large amount of time and effort on bug fixing. Consequently, \aprfull (\apr), designed to automatically generate patches to fix software bugs, has attracted wide attention from both academia and industry~\cite{long2016prophet, legoues2012genprog, long2015spr, lou2020can, tufano2018empstudy}. 


To achieve \apr, one popular approach is known as Generate-and-Validate (G\&V)~\cite{qi2015gv, ghanbari2019prapr, lou2020can, le2016hdrepair, legoues2012genprog, wen2018capgen, hua2018sketchfix, martinez2016astor, koyuncu2020fixminder, liu2019tbar, liu2019avatar}, which is typically based on the following pipeline: First, fault localization techniques~\cite{wong2016fl, abreu2007ochiai, zhang2013injecting, papadakis2015metallaxis, li2019deepfl, li2017transforming} are applied to determine the suspicious locations in programs where bugs are likely to exist. Then, the buggy locations are used by the \apr tools to generate a list of patches that replace buggy lines with correct lines. Afterward, each patch is validated against the original test suite to identify any \emph{plausible patches} (i.e., passing all tests in the test suite). Finally, to determine the \emph{correct patches}, developers examine the list of plausible patches to see if any of them can correctly fix the bug. 

Traditional \apr tools can mainly be categorized into heuristic-based~\cite{legoues2012genprog, le2016hdrepair, wen2018capgen}, constraint-based~\cite{mechtaev2016angelix, le2017s3, demacro2014nopol, long2015spr} and \template~\cite{ghanbari2019prapr, hua2018sketchfix, martinez2016astor, liu2019tbar, liu2019avatar}. Among these traditional tools, \template \apr tools~\cite{ghanbari2019prapr, liu2019tbar, benton2020effectiveness} have been able to achieve state-of-the-art results. \Template \apr tools typically leverage pre-defined templates (e.g., adding a nullness check) for bug fixing. However, since these fix templates are typically handcrafted, the number and types of bugs they are able to fix can be limited. 



To address the limitations of traditional \apr, researchers have proposed various \learning \apr tools~\cite{li2020dlfix, chen2018sequencer, jiang2021cure, lutellier2020coconut, zhu2021recoder, ye2022rewardrepair} based on the \nmtfull (\nmt) architecture~\cite{sutskever2014mt} where the input is the buggy code snippets and the goal is to translate the buggy code snippets into a fixed version. To accomplish this, \learning \apr tools require supervised training datasets with pairs of both buggy and fixed code snippets in order to learn how to perform this translation step. These training data are usually obtained by mining historical bug fixes using heuristics/keywords~\cite{dallmeier2007benchmark}, which can be imprecise for identifying bug-fixing commits; even the actual bug-fixing commits can include irrelevant code changes, leading to further pollution in the dataset~\cite{xia2022alpharepair}.
% 
Moreover, it can be hard for such \apr tools to generalize and fix bug types unseen during training. 



To better leverage recent advances in \plmfull{s} (\plm{s}), researchers~\cite{xia2022alpharepair, xia2023repairstudy, kolak2022patch, prenner2021codexws} have directly applied \plm{s} to generate patches without bug-fixing datasets. These \llm-based \apr tools work by either directly generating a complete code function~\cite{prenner2021codexws, xia2023repairstudy} or predict/infill the correct code snippet given its surrounding context~\cite{xia2022alpharepair, xia2023repairstudy}. By directly using \llm{s} that are pre-trained on billions of open-source code snippets, \llm-based \apr tools can achieve state-of-the-art performance on many repair datasets~\cite{xia2022alpharepair}. 


% 
%
%

Traditional \apr tools have long used the insight of the \emph{plastic surgery hypothesis}~\cite{barr2014plastic} where it states that the code ingredients to fix a bug already exist within the same project. Traditional \apr tools have manually designed pattern-~\cite{ghanbari2019prapr, saha2017elixir} or heuristic-based~\cite{jiang2018simfix, legoues2012genprog} approaches to finding and using such relevant code ingredients to generate fixes for bugs. However, the plastic surgery hypothesis has been largely ignored in \llm-based \apr. In fact, \llm provides a unique opportunity to fully automate the plastic surgery hypothesis idea via fine-tuning (learning project-specific information via model updates from the buggy project) and prompting (directly providing relevant code ingredients to the model), and make it directly applicable to different languages (since the \llm{s} are typically multi-lingual).%
Moreover, despite the intensive manual efforts involved, traditional \apr tools still cannot fully leverage project-specific information due to large search space for leveraging/composing existing code ingredients. In contrast, the project-specific information can effectively leveraged by \llm{s} due to their power in code understanding/vectorization, e.g., even partial/imprecise information may still guide \llm{s} in correct patch generation!
 To this end, we ask the question: \emph{How useful is the plastic surgery hypothesis in the era of \plm{s}}?








\mypara{Our Work.} To answer the question, we present \ourtech{\xspace} -- a \llm-based approach that automatically utilizes the plastic surgery hypothesis by systematically combining multiple fine-tuning and prompting strategies for \apr. \ourtech fine-tunes \plm{s} using two novel domain-specific training strategies: \textbf{\epfinetune} -- we fine-tune using the original buggy project by aggressively masking out a high percentage of tokens, which allows \plm to learn project-specific code tokens and programming styles; and \textbf{\rofinetune} -- which only masks out a single continuous code sequence per training sample, allowing the model to get used to the final \csapr task of predicting a single continuous code sequence. Furthermore, we directly leverage the ability for \plm{s} to understand natural language instructions and introduce a novel prompting strategy, \textbf{\idprompting}, which uses information retrieval and static analysis to obtain a list of relevant identifiers for the buggy lines. While such relevant identifiers are critical for fixing some difficult bugs, they may not be seen by the \llm during inference due to limited context window size. Through the use of prompting, we directly tell the model to use these extracted identifiers (relevant code ingredients) to generate the correct code. Finally, to perform repair, we combine all four model variants (including the base model, both fine-tuned models and the base model with prompting) for the final repair.





While our insight of leveraging the plastic surgery hypothesis for \llm-based \apr is generalizable across different types of \plm{s}, to implement \ourtech, we choose a recent \plm{\xspace}, \ctfive~\cite{wang2021codet5}, which is pre-trained on millions of open-source code snippets. \ctfive is an encoder-decoder model trained using \mspfull (\msp) objective where a percentage of tokens are masked out and each continuous masked token sequence is referred to as a masked span. Also, although we only extract relevant identifiers from the current buggy project (since this paper focuses on the plastic surgery hypothesis), our work can be easily extended to obtain other code information (such as relevant statements or functions) from other sources, such as  the massive pre-training corpora~\cite{husain2020codesearchnet} or historical bug-fixing datasets~\cite{jiang2019infer}, which can provide more coding knowledge for \llm{s}. Besides, although we mainly focus on using traditional string comparison algorithms for information retrieval in this paper, these techniques can be easily replaced by other frequency-based retrieval~\cite{robertson2009probabilistic} and neural search (or embedding-based search)~\cite{reimers2019sentence}.
  In summary, this paper makes the following contributions:


%


\begin{itemize}[noitemsep, leftmargin=*, topsep=0pt]
    \item \textbf{Dimension.} This paper is the first to revisit the important plastic surgery hypothesis in the era of \llm{s}. It opens up a new dimension for \llm-based \apr to incorporate previously neglected information from the buggy project itself to boost \apr performance. Furthermore, it demonstrates the promising future of retrieval-based prompting for modern \llm-based \apr.
    \item \textbf{Implementation.} We implement \ourtech based on the recent \ctfive model. We augment the model using two novel fine-tuning strategies: \epfinetune and \rofinetune, along with a novel prompting strategy based on information retrieval and static analysis: \idprompting. We combine the patches generated by all four models together and perform patch ranking to speed up \apr.% 
    \item \textbf{Evaluation Study.} We conduct an extensive evaluation against state-of-the-art \apr tools. On the widely studied \dfj 1.2 and 2.0 datasets~\cite{just2014dfj}, \ourtech is able to achieve the new state-of-the-art results of 89 and 44 correct bug fixes (15 and 8 more than best baseline) respectively.  Furthermore, we perform a broad ablation study to justify our design. \ourtech demonstrates for the first time that the plastic surgery hypothesis can substantially boost \llm-based \apr and advance state-of-the-art \apr, while being fully automated and general. Moreover, even partial/imprecise code ingredients may still effectively guide \llm{s} for \apr!
\end{itemize}


\begin{table}[h]
\small
\centering
\caption{
Statistics of \NAME v1.0.
}
% \begin{tabular}{P{1.7cm}P{1.2cm}P{1.2cm}P{1.0cm}P{1.2cm}P{1.2cm}P{1.2cm}P{1.1cm}P{1.1cm}P{1.1cm}}
\begin{tabular}{lr}
\toprule[1.2pt]
Statistics \\
\midrule
\# of prompts & 4,081 \\
$\ $ - \# of COCO captions & 2,000 \\
$\ $ - \# of DrawBench, PartiPrompt, PaintSkill prompts & 2,081 \\
\midrule
\# of questions & 25,829 \\
$\ $ - \# of binary questions & 17,226 \\
$\ $ - \# of multiple-choice questions & 8,603 \\
\midrule
avg. \# of questions per prompt & 6.3 \\
avg. \# of words per prompt & 10.5 \\
avg. \# of elements per prompt & 4.3 \\
\bottomrule[1.2pt]
\end{tabular}
\label{tab:statistics}
% \bottomrule
\vspace{-3mm}
\end{table}
\section{Distortion Representation} \label{app:distortion-representation}
This section contains the proofs associated with \cref{sec:distortion-representation}.

\paragraph*{Proof of \cref{lem:identically-distributed}}  
We will need to use the vector and function characterization of random variables simultaneously throughout this proof. 
So to avoid the abuse of notation used throughout the rest 
of the paper we use capital letters when $X \colon \Omega^n \to \Re$ is 
implied and $x \in \Re^n$ for the vector representation with
\begin{equation*}
    x_i = X(\omega_i), \quad \forall i \in [n].
\end{equation*}
The random variables $X, Y \colon \Omega^n \to \Re$ are identically distributed when 
$\prob[X \in \set{X}] = \prob[Y \in \set{X}]$
for every measurable set $\set{X} \in \B$, with $\B$ the Borel sigma algebra over the reals. Using the vector representation
$\prob[X \in \set{X}] = \frac{1}{n} \sum_{i=1}^{n} \bm{1}_{\set{X}}(x_i)$.
Thus $X \deq Y$ holds iff  
\begin{equation} \label{eq:equal-distribution}
    \sum_{i=1}^{n} \bm{1}_{\set{X}}(x_i) = \sum_{i=1}^{n} \bm{1}_{\set{X}}(y_i), \quad \forall \set{X} \in \B.
\end{equation}

To show $\Leftarrow$ observe that \eqref{eq:equal-distribution} is permutation invariant. So 
$\sum_{i=1}^{n} \bm{1}_{\set{X}}(y_{\pi(i)}) = \sum_{i=1}^{n} \bm{1}_{\set{X}}(y_i)$ for all $\set{X} \in \F$. 

To show $\Rightarrow$ let $m$ denote the number of distinct elements of $x$, the values of which we define as $\{z_j\}_{j=1}^{m}$.
Then let $\set{C}_j = \{i \colon x_i = z_j\}$ for $j \in [m]$. We have 
\begin{equation*}
    \sum_{i=1}^{n} \bm{1}_{\set{X}}(x_i) = \sum_{j=1}^{m} |\set{C}_j| \bm{1}_{\set{X}}(z_j) = \sum_{i=1}^{n} \bm{1}_{\set{X}}(y_i), \quad \forall \set{X} \in \B
\end{equation*}
since $X \deq Y$ by assumption. For any $j \in [m]$ let $\set{X} = \{z_j\}$ and $\set{D}_j = \{i \colon y_i = z_j\}$. Then the above implies that $|\set{D}_j| = |\set{C}_j|$ 
for all $j \in [m]$. Take $\pi \colon [n] \to [n]$ any of the bijections such that $\pi(\set{D}_j) = \set{C}_j$. Then $y_{\pi(i)} = z_{j} = x_i$ for all $i \in \set{D}_j$ and all $j \in [m]$. 
Thus $x = \pi y$, proving the claimed result. \qed\\


% Consider $\bar{\pi}(i) = \left\{ j \colon y_j \in \{z_i\} \right\}$ and let $|\bar{\pi}(i)| = |\set{C}_i|$ for each $i \in [m]$. 
% Also, by construction $\bar{\pi}(i) \cap \bar{\pi}(j) = \emptyset$
% for each pair $i \neq j$. Hence we can define a permutation (i.e. a bijection) $\pi \colon [n] \to [n]$ by arbitrarily pairing up elements 
% between $\set{C}_i$ and $\bar{\pi}(i)$ such that $\pi^{-1}(i) \in \bar{\pi}^{-1}(i) = \set{C}_i$ for all $i \in [m]$. 
% Then taking $y = \pi x$ implies $y_{\pi^{-1}(j)} = z_i$ for $j \in \set{C}_i$ and all $i \in [m]$. Therefore 
% \begin{equation*}
%     \sum_{j=1}^{n} \bm{1}_{\set{X}}(y_j) = \sum_{j=1}^{n} \bm{1}_{\set{X}}(y_{\pi^{-1}(j)}) = \sum_{i=1}^{m} |\set{C}_i| \bm{1}_{\set{X}}(z_i).
% \end{equation*} 
% We have thus shown that \cref{eq:equal-distribution} holds. \qed

We next state the usual definition of law-invariant, coherent risk measures:
\newcommand{\cohasm}[1]{\hyperref[def:coh:#1]{\sc a\oldstylenums{#1}}}
\begin{definition}[coherent risk] \label{def:coh}
    Consider a $\rho \colon \Re^n \to \eRe$
    that is proper \footnote{
        That is $\rho(X) > -\infty$ for all $X \in \Re^n$ and the domain $\mathrm{dom}(\rho) \dfn \{X \in \Re^n \colon \rho(X) < +\infty\}$ is nonempty.
    } and lsc. Assume (for $X, Y \in \Re^n$):
    \begin{enumeratass}[leftmargin=2em]
        \item \emph{convex}: $\rho(\alpha X + (1-\alpha) Y) \leq \alpha \rho(X) + (1-\alpha) \rho(Y)$, $\forall \alpha \in [0, 1]$;\label{def:coh:1}
        \item \emph{monotone}: if $Y \geq X$, then $\rho(Y) \geq \rho(X)$;\label{def:coh:2}
        \item \emph{translation equivariance}: $\rho(X+a) = \rho(X) + a$, $\forall a \in \Re$;\label{def:coh:3}
        \item \emph{pos. homogeneity}: if $t > 0$, then $\rho(tX) = t\rho(X)$.\label{def:coh:4}
    \end{enumeratass}
    Then $\rho$ is a \emph{coherent risk measure}. Consider also
    \begin{enumeratass}[leftmargin=2em]
        \item[\textsc{A\oldstylenums{5}}.] \emph{law invariance}: if $X \deq Y$, then $\rho(Y) = \rho(X)$. \label{def:coh:5}
    \end{enumeratass}
\end{definition}
The first four assumptions were considered in \cite[\S1]{Ruszczynski2006}, while \cohasm{5} is considered in 
\cite[Def.~4.4]{Bertsimas2009b}. By \cref{lem:identically-distributed} it implies that $\rho(X) = \rho(\pi X)$ 
for all $\pi \in \Pi^n$. 

The convex conjugate $\rho^* \colon \Re^n \to \eRe$ of a risk measure is given as 
\begin{equation} \label{eq:convex-conjugate}
    \rho^*(\mu) = \sup_{X \in \Re^n} \left\{ \<\mu, X\> - \rho(X) \right\}.
\end{equation}

Its domain is affected by the assumptions in \cref{def:coh}. 
\begin{proposition} \label{prop:conjugate-duality}
    Suppose that $\rho \colon \Re^n \to \eRe$ is proper, lsc., and convex. Then, $\rho = \rho^{**}$
    with $\amb \dfn \mathrm{dom}(\rho^*)$ a non-empty and convex set. Moreover
    \begin{enumerate}[itemsep=0pt]
        \item \cohasm{2} holds iff $\mu \geq 0$ for any $\mu \in \amb$; \label{prop:conjugate-duality:2}
        \item \cohasm{3} holds iff $\sum_{i=1}^{n} \mu_i = 1$ for any $\mu \in \amb$; \label{prop:conjugate-duality:3}
        \item \cohasm{4} holds iff $\amb$ is closed and \label{prop:conjugate-duality:4}
        \begin{equation*}
            \rho(X) = \sup_{\mu \in \amb} \, \<\mu, X\>, \quad \forall X \in \Re^n
        \end{equation*}
        \item \cohasm{5} holds iff $\rho^*(\pi \mu) = \rho^*(\mu)$ for all $\pi \in \Pi^n, \mu \in \Re^n$. \label{prop:conjugate-duality:5}
    \end{enumerate}
    We assumed the underlying measure of the random variables is $\one_n/n = (1/n, \dots, 1/n)$ for (iv). 
\end{proposition}
\begin{proof}
    By \cite[Prop.~2.112]{Bonnans2000} $\rho^*$ is proper and convex. 
    Hence its domain must be non-empty and convex. 
    Now \emph{(i)}--\emph{(iii)} follows from \cite[Thm.~2.2]{Ruszczynski2006}. 

    By \cref{lem:identically-distributed}, \cohasm{5} holds iff $\rho(\pi X) = X$ for all $\pi \in \Pi^n$. 
    We first show that \cohasm{5} implies $\rho^*(\pi \mu) = \rho^*(\mu)$ for all $\pi \in \Pi^n$.
    First note that 
    \begin{align*}
        &\sup_{X} \left\{ \<\mu, X\> - \rho(X) \right\} = \sup_{X} \left\{ \<\mu, \pi(X)\> - \rho(\pi(X)) \right\} \\
        &\qquad = \sup_{X} \left\{ \<\pi^{-1}(\mu), X\> - \rho(X) \right\}
    \end{align*}
    Since $\{\pi^{-1} \colon \pi \in \Pi^n\} = \Pi^n$ we have shown $\rho^*(\pi \mu) = \rho^*(\mu)$ for all $\mu \in \Re^n$
    and all $\pi \in \Pi^n$. For the reverse implication we can apply the same reasoning and using that $\rho = \rho^{**}$. 
    Note that this result specializes \cite[Prop.~2]{Shapiro2013}.
\end{proof}

% Note that \cref{def:ambiguity} is a direct consequence of \cref{prop:conjugate-duality}.
% Hence the definition is consistent with the usual definition of law-invariant coherent risk measures in \cref{def:coh}. 

% As discussed in \cref{sec:distortion-representation} we will derive an analogous representation to 
% the ambiguity representation in \cref{def:ambiguity} using the ordered convex conjugate:
% \begin{equation} 
%     \rho^{\diamond}(\mu) \dfn \sup_{X \in \M^n} \left\{ \<\mu, X\> - \rho(X) \right\}. \tag{\ref{eq:ordered-conjugate}}
% \end{equation}
To find the equivalent of \cref{prop:conjugate-duality} for the ordered conjugate $\rho^\diamond$, we
begin by relating $\rho^\diamond$ to $\rho^*$ in terms of an infimal convolution.
Then we relate their domains.
\begin{lemma} \label{lem:infimal-convolution}
    Given some proper, convex, lsc. and law-invariant risk measure $\rho \colon \Re^n \to \eRe$. Then 
    \begin{equation*}
        \rho^{\diamond} = (\rho + \indi_{\M^n})^* = \rho^* \episum \indi_{(\M^n)^\circ}.
    \end{equation*}
\end{lemma}
\begin{proof}
    The first equality holds by definition of the conjugate and the indicator. 
    Plugging in the associated definitions and changing signs makes the second equality equivalent to
    \begin{align*}
        \inf_{x} \, \rho(x) + (\indi_{\M^n}(x) - \<x, y\>) = - \inf_u \rho^*(u) + \indi^*_{\Re^n_{\uparrow}}(y-u).
    \end{align*}
    Letting $f(x) = \rho(x)$ and $g(x) = \indi_{\M^n}(x) - \<x, y\>$. Note that $f^{*} = \rho^*$ and $g^*(u) = \indi^*_{\M^n}(y+u)$. 
    Thus we require
    \begin{align*}
        \inf_x \, f(x) + g(x) = - \inf_u f^{*}(u) + g^{*}(-u),
    \end{align*}
    which by \cite[Fact.~15.25]{Bauschke2011}, since $g$ is polyhedral, requires $\dom g \cap \ri \dom f \neq \emptyset$. 
    
    By properness, $\dom \rho \neq \emptyset$, which by convexity of $\rho$ and \cite[Fact.~6.14(i)]{Bauschke2011} implies $\ri \dom \rho \neq \emptyset$.
    Moreover, by permutation invariance $x \in \dom \rho$ implies $\pi x \in \dom\rho$ for all $\pi \in \Pi^n$ since $\rho(\pi(x)) = \rho(x) < \infty$. 
    Thus $\pi \dom \rho = \dom \rho$. Moreover, by \cite[Prop.~2.44]{Rockafellar1998} and linearity of permutations we have $\ri (\dom \rho) = \ri (\pi \dom \rho) = \pi \ri (\dom \rho)$. 
    Thus $\dom g \cap \ri \dom f = \Re^n_{\uparrow} \cap \ri (\dom\rho) \neq \emptyset$. 

    Finally note that $\indi_{\M^n}^* = \indi_{(\M^n)^\circ}$ by \cite[Ex.~11.4]{Rockafellar1998}. 
\end{proof}

\begin{proposition} \label{prop:fundamental-equivalence-lemma}
    Let $\rho \colon \Re^n \to \eRe$ be lsc., convex and law-invariant risk measure. 
    Then
    \begin{align*}
        \mathrm{dom}\left( \rho^{\diamond} \right) &= \mathrm{dom}\left( \rho^* \right) + (\M^n)^\circ \\
        \mathrm{dom}\left( \rho^* \right) &= \bigcap_{\pi \in \Pi^n} \pi\left( \mathrm{dom}\left( \rho^{\diamond} \right) \right).
    \end{align*}
\end{proposition}
\begin{proof}
    The first result follows from \cref{lem:infimal-convolution} and \cite[Ex.~1.28]{Rockafellar1998}. %
    The second result follows from \cref{cor:cone-shift-invariance}. 
\end{proof}

The equivalence derived through \cref{prop:fundamental-equivalence-lemma} enables us to 
re-derive many of the statements in \cref{prop:conjugate-duality} for $\rho^\diamond$. 

\begin{proposition} \label{prop:ordered-conjugate-duality}
    Suppose that $\rho \colon \Re^n \to \eRe$ is lsc., law-invariant and convex. Then
    $\distort \dfn \mathrm{dom}(\rho^\diamond)$ is a convex set such that $\distort \cap \M^n \neq \emptyset$. 
    Moreover
    \begin{enumerate}[itemsep=0pt]
        \item If $\mu \in \distort$ then $\mu + (\M^n)^\circ \subseteq \distort$;
        \item \cohasm{2} holds iff any $\mu$ s.t. $\pi(\mu) \in \distort$ for all permutations $\pi \in \Pi^n$
        is nonnegative. 
        \item \cohasm{3} holds iff $\sum_{i=1}^{n} \mu_i = 1$ for any $\mu \in \distort$; 
        \item \cohasm{4} holds iff $\distort$ is closed and
        \begin{equation*} 
            \rho(X) = \sup_{\mu \in \distort} \, \<\mu, X_{\uparrow}\>, \quad \forall X \in \Re^n
        \end{equation*}
    \end{enumerate}
\end{proposition}
\begin{proof}
    By \cref{prop:conjugate-duality} we have convexity of $\dom\rho^*$. This implies convexity of $\rho^\diamond$ 
    (Minkowski sum of convex sets). Next assume $x \in \dom \rho^*$ then, by \cref{prop:fundamental-equivalence-lemma}, 
    $\pi(x) \in \distort$ for all $\pi \in \Pi^n$. Thus $x_{\uparrow} \in \distort$. 
    So $\dom \rho^* \neq \emptyset$, which holds by \cref{prop:conjugate-duality}, implies $\distort \cap \M^n \neq \emptyset$.
    % Similarly, assume there is some $\mu_{\uparrow} \in \distort \cap \M^n \neq \emptyset$. 
    % By \cref{prop:permuto-hull} we have $\hull (\Pi^n \mu) \in \mu_{\uparrow} + (\M^n)^\circ$.
    % Thus $\pi(\hull (\Pi^n \mu)) \in \distort$ for all $\pi$ or equivalently $\hull (\Pi^n \mu) \subseteq \cap_{\pi} \pi(\distort)$. 
    % Note that $\dom \rho^* = \cap_{\pi} \pi(\distort)$ by \cref{prop:permuto-hull}. So $\dom\rho^* \neq \emptyset$.
    
    \emph{(i)} holds by \cref{prop:fundamental-equivalence-lemma}. To show \emph{(ii)} we use \cref{prop:fundamental-equivalence-lemma}
    and specifically $\dom \rho^* = \cap_{\pi} \pi(\distort)$. So $x \in \dom \rho^*$ iff $\pi(x) \in \distort$ for all $\pi \in \Pi^n$.
    Note that \cohasm{2} holds iff $x \geq 0$ for all $x \in \dom \rho^*$. So \cohasm{2} holds iff every $x$ such that $\pi(x) \in \distort$ for all $\pi \in \Pi^n$
    is nonnegative, proving \emph{(ii)}.

    Let $\bar{\rho} = \rho + \iota_{\Re^n}$ as in \cref{lem:infimal-convolution}.
    Then $\distort = \mathrm{dom}(\bar{\rho}^*)$. 
    By permutation invariance $\bar{\rho}(X_{\uparrow}) = \rho(X_{\uparrow}) = \rho(X)$ for all $X \in \Re^n$. 
    We can then prove \emph{(iii)} by noting that $(X + a)_{\uparrow} = X_{\uparrow} + a$. 
    Hence \cohasm{3} holds for $\bar{\rho}$
    iff it holds for $\rho(X) = \bar{\rho}(X_{\uparrow})$. Thus we can apply \cref{prop:conjugate-duality:3} 
    to $\bar{\rho}$ in order to link \cohasm{3} with $\mathrm{dom}(\rho^{\diamond}) = \mathrm{dom}(\bar{\rho}^{*})$. 
    For \emph{(iv)} we use a similar argument noting that $(\alpha X)_{\uparrow} = \alpha X_{\uparrow}$ for all $X \in \Re^n$
    and all $\alpha \geq 0$ and apply \cref{prop:conjugate-duality:4}. The only property of $\rho$ 
    that does not transfer to $\bar{\rho}$ is monotonicity. So \emph{(ii)} is more complex. 
\end{proof}

We can now prove the main theorem of \cref{sec:distortion-representation}.
\paragraph*{Proof of \cref{thm:distortion-representation}} 
We need to show that \cref{prop:ordered-conjugate-duality} implies the properties of $\distort$
as stated in the theorem. \cref{prop:ordered-conjugate-duality} directly implies the following properties:
convexity, closedness, \emph{(ii)}, \emph{(iii)} and \emph{(iv)} as well as \cref{eq:distortion-representation}. 
The one remaining property is \emph{(i)}. Note that, for any $\mu \in \dom\rho^*$ we have 
$\hull(\Pi^n \mu) \subseteq \dom \rho^*$ by permutation invariance (so $\mu_{\uparrow} \in \dom\rho^*$).
Then \cref{prop:fundamental-equivalence-lemma} and \cref{prop:permuto-hull} imply $\hull(\Pi^n \mu) \subseteq \dom \rho^\diamond$. 
Take $E \in \Re^{n \times n}$ a matrix of all ones. 
Then $E\mu/n \in \hull(\Pi^n \mu) $. Since $\sum_{i=1}^{n} \mu_i = 1$ by \emph{(iii)} we have 
$E\mu/n = \one_n /n$. So if $\one_n /n \notin \dom \rho^*$ then $\dom \rho^*$ 
is empty by contradiction. Thus $\distort$ contains $\one_n/n$ since otherwise $\dom\rho^* = \emptyset$.
Finally, \cref{eq:distort-ambiguity-relation} rephrases \cref{prop:fundamental-equivalence-lemma}. \qed{}
Given measurements $\{\hat \Omega_i\}$ with uncertainties $\sigma^2_i$, as shown in Sec.\ref{sec:pe} the following likelihood function can be used to perform parameter estimation on the \gls{gwb}:
	\begin{equation}
    \label{eq:likelihood-again}
    p(\{\hat \Omega_f\} | {\bm \Theta})
    	= \mathcal{N} \exp\left[
        	-\frac{1}{2}\sum_f\frac{\left(\hat \Omega_f -  \Omega_{\rm M}(f|{\bm \Theta})\right)^2}{\sigma_f^2}\right].
    \end{equation}
Here, the $\{\hat \Omega_f\}$ are a set of estimators for the \gls{gw} energy density at discrete frequencies $f$, $\Omega_{\rm M}(f|{\bm \Theta})$ is a model for the energy density with parameters ${\bm \Theta}$, and $\mathcal{N}$ is a normalization constant.
We will consider only a single baseline and neglect the sum over detector pairs $IJ$ appearing in Eq.~\eqref{eq:likelihood}; if multiple detector pairs exist, the derivation below can be replicated for each pair.

Eq.~\eqref{eq:likelihood-again} assumes that our estimators $\{\hat \Omega_f\}$ are direct, unbiased measurements of the underlying energy-density spectrum.
In general, however, the imperfect amplitude and phase calibration of \gls{gw} detectors will break this assumption.
We can account for calibration uncertainty by amending our likelihood to introduce a new parameter $\lambda$:
	\begin{equation}
    \label{eq:likelihood-calibration-uncertainty}
    p(\{\hat \Omega_f\} | {\bm \Theta},\lambda)
    	= \mathcal{N} \exp\left[
        	-\frac{1}{2}\sum_f\frac{\left(\hat \Omega_f -  \lambda\Omega_{\rm M}(f|{\bm \Theta})\right)^2}{\sigma_f^2}\right].
    \end{equation}
The parameter $\lambda$ is an unknown multiplicative factor that encapsulates potential calibration inaccuracy.
In the case of perfect amplitude calibration ($\lambda=1$), then $\{\hat \Omega_f\}$ are direct measurements of the underlying (unknown) energy spectrum.
But if our calibration is imperfect ($\lambda\ne1$), then $\{\hat \Omega_f\}$ are instead measurements of some multiple $\lambda \Omega(f)$ of the \gls{gwb} spectrum.
Although we do not know $\lambda$, it is possible to estimate the \textit{uncertainty} on instrumental calibration.
We will therefore model $\lambda$ itself as an unknown variable drawn from a normal distribution centered at 1 (corresponding to perfect calibration) but with a variance $\epsilon^2$:
	\begin{equation}
    p(\lambda) \propto 
    	\exp\left[-\frac{1}{2\epsilon^2}\left(\lambda-1\right)^2\right],
    \end{equation}
where $\epsilon$ is a known amplitude calibration uncertainty. 
Additionally, we impose the constraint that $\lambda$ be positive: we expect errors in the amplitude of strain measurements but not their \textit{sign}.
In this case, the probability distribution for $\lambda$ becomes
	\begin{equation}
    \label{eq:plambda}
    p(\lambda) = \sqrt{\frac{2}{\pi}}
  		\frac{1}{\epsilon\left[1
        	+\mathrm{Erf}(\frac{1}{\sqrt{2\epsilon^2}})\right]}
        \exp\left[-\frac{1}{2\epsilon^2}\left(\lambda-1\right)^2\right],
    \end{equation}
normalized to unity on the interval $\lambda\in(0,\infty)$.
Eq. \eqref{eq:plambda} is our prior on $\lambda$.

We can now use Eq. \eqref{eq:plambda} to marginalize our likelihood (Eq.~\eqref{eq:likelihood-calibration-uncertainty}) over the unknown calibration factor $\lambda$.
The marginalized likelihood is given by
	\begin{equation}
    \begin{aligned}
    p(\{\hat \Omega_f \} | {\bm \Theta} )
        &= \int p(\{\hat \Omega_f \} | {\bm\Theta},\lambda) \,p(\lambda) d\lambda \\
        &= \mathcal{N} \sqrt{\frac{2}{\pi}}
  			\frac{1}{\epsilon\left[1
        		+\mathrm{Erf}(\frac{1}{\sqrt{2\epsilon^2}})\right]}
            \int_0^\infty \exp\left[
            	-\frac{1}{2}\sum_f \frac{\left(\hat \Omega_f - \lambda\Omega_{\rm M}(f|{\bm \Theta})\right)^2}{\sigma^2_f}
                -\frac{1}{2}\frac{\left(\lambda-1\right)^2}{\epsilon^2}
                \right] d\lambda.
    \end{aligned}
    \end{equation}
If we define 
	\begin{equation}
    A({\bm \Theta}) = \frac{1}{\epsilon^2}+\sum_f\frac{\Omega_{\rm M}(f|{\bm \Theta})^2}{\sigma^2_f},
    \end{equation}
    \begin{equation}
    B({\bm \Theta}) = \frac{1}{\epsilon^2}+\sum_f\frac{\hat \Omega_f \Omega_{\rm M}(f|{\bm\Theta})}{\sigma^2_f},
    \end{equation}
and
	\begin{equation}
    C({\bm \Theta}) = \frac{1}{\epsilon^2}+\sum_f\frac{\hat \Omega^2_f}{\sigma^2_f},
    \end{equation}
the marginal likelihood can be more concisely expressed as
	\begin{equation}
    \label{eq:likelihood-calib-2}
    p(\{\hat \Omega_f \} | {\bm \Theta} )
    	= \mathcal{N} \sqrt{\frac{2}{\pi}}
  			\frac{1}{\epsilon\left[1 +\mathrm{Erf}(\frac{1}{\sqrt{2\epsilon^2}})\right]}
            \int_0^\infty \exp\left[-\frac{1}{2}\left(
            	A({\bm \Theta})\lambda^2 - 2B({\bm \Theta})\lambda + C({\bm \Theta})
                \right)\right] d\lambda;
    \end{equation}
this expression can be analytically integrated to obtain
	\begin{equation}
    p(\{\hat \Omega_f \} | {\bm \Theta} ) =
    	\mathcal{N} \frac{1}{\epsilon\sqrt{A({\bm \Theta})}}
        \left[\frac{1+\mathrm{Erf}(\frac{B({\bm \Theta})}{\sqrt{2A({\bm \Theta})}})}
        	{1+\mathrm{Erf}(\frac{1}{\sqrt{2\epsilon^2}})}\right]
        \exp\left[-\frac{1}{2}\left(C({\bm \Theta})-\frac{B({\bm \Theta})^2}{A({\bm \Theta})}\right)\right].
    \end{equation}

Marginalization of calibration uncertainty is built into the {\tt pygwb\_pe} module, and this calculation is automatically triggered when passing a calibration error $\epsilon\neq 0$. Additional information on the treatment of calibration uncertainties can be found in \cite{Whelan:2012ur}.
%% The Case Studies
\subsection{Attacking Weight-based Watermarks}
\label{sec:eval_weight}
% \noindent\textit{7.2.1$\quad$Uchida et al.}\\
\noindent\textbf{(1) Uchida et al.}
% This method utilizes a watermark-related regularization term to embed owner-specific information into the certain weights of a layer of the target model, 
Uchida et al.\cite{uchida2017embedding} introduce one of the earliest white-box schemes which embed the model watermark into the convolutional weights of the target model. 
% As we mentioned in Section \ref{sec: white-box wm}
To extract the model watermark, the scheme first averages the convolutional weights $w \in \mathbb{R}^{N_{l-1}\times N_l \times w \times h}$ of the watermark-related layer through first dimension and flattens the weight to $\hat w \in \mathbb{R}^{(N_l \cdot w \cdot h)}$. Then, a binary string $s'$ is obtained from $\hat w$ through a pre-defined linear matrix $A$ and a threshold function $T_h$ at 0, i.e.,
%%%%%%%%%%%%%%%%%%%%% begin eq uchida %%%%%%%%%%%%%%%%%%%%%
% \begin{equation}\label{eq:uchida}
$
s' = T_h(X \cdot \hat w),
$
% \end{equation}
%%%%%%%%%%%%%%%%%%%%% end eq uchida %%%%%%%%%%%%%%%%%%%%%
which is matched with the owner-specific message $s$ in terms of BER for validation.
% where $T_h$ is a hard threshold function which outputs $1$ when the input is positive and $0$ otherwise. Finally, the extracted binary string $s'$ is matched with the owner-specific message $s$ in terms of BER to assert the model ownership.

\noindent$\bullet$\textbf{ Discussion.}
% As the linear transformation matrix in \ref{eq:uchida} is predefined before the watermark embedding procedure,  As the extraction process of Uchida et al. only consists of a mask matrix and a linear transformation matrix, \mytodo{the element-level representations of watermark-related parameters play a crucial role to the final extracted binary string. (please be more pointed at its own weakness)}
Although this approach is previously shown to be vulnerable to overwriting, known attacks however require the specific prior knowledge of the watermark algorithm as well as a domain dataset, both of which are usually impractical. Our attack reveals the insecurity of this algorithm by directly modifying the structure of the target model while leaving the utility intact. Specifically, the dimension of $\hat w$ is increased once the adversary injects dummy neurons into watermark-related layers. As a result, during the extraction procedure of $s_i'$ from the victim model, the second dimension of pre-defined linear transformation matrix $X$ is unmatched to the dimension of expanded $\hat w$ any longer.

%%%%%%%%%%%%% BEGIN OF ls fig
%\begin{figure}[t]
%\begin{center}
%\includegraphics[width=0.45\textwidth]{img/ber_p_uchida_dn.pdf}
%\caption{BER of WRN watermarked by \cite{uchida2017embedding} after an $\alpha$ ratio of dummy neurons are injected by our attacks. The dashed horizontal lines reports the BER of an irrelevant model.}
%\label{fig:ber_p_uchida}
%\end{center}
%\vspace{-0.25in}
%\end{figure}
%%%%%%%%%%%%%% END OF ls fig

\noindent$\bullet$\textbf{ Evaluation Results.}
We run the code of \cite{code-uchida} they publicly released to reproduce a watermarked model of Wide Residual Network (WRN) trained on CIFAR10 dataset. We perform our removal attack to inject dummy neurons generated via different methods into each layer, which presents the same original utility with classification accuracy of $91.55\%$ while the verification procedure is inhibited if with no error handling mechanism. As Fig.\ref{fig:scaled_ber} shows that applying Max-First cannot restore the embedded watermark from structural obfuscated model by \textit{NeuronClique} and \textit{NeuronSplit}, as the BER is increased over $50\%$ once we add less than $5\%$ dummy neurons, indicating the innate vulnerability of this scheme.


% ########################################################################
\noindent\textbf{(2) RIGA.} Wang et al. \cite{wang2021riga} replace the linear transformation matrix in Uchida et al. with a learnable fully-connected neural network (FCN) to boost the encoding capacity of watermarking messages. That is, the intuition behind this watermark scheme is closely identical to Uchida et al. We present the full details in Appendix \ref{sec:app:eval} and report the results in Fig.\ref{fig:scaled_ber}.
% \noindent\textbf{Protection Mechanism.}
% Wang et al. \cite{wang2021riga} enhance the covertness and robustness of prior white-box watermarking methods against watermark detection and removal attacks based on adversarial training and more sophisticated transformation function. They train a watermark detector to serve as a discriminator to encourage the distribution of watermark-related weights to be similar to that of unwatermarked models. Meanwhile, they replace the watermark extractor, which is previously implemented with a predefined linear transformation \cite{uchida2017embedding}, with a learnable fully-connected neural network (FCN), for boosting the encoding capacity of watermarking messages. Similar to Uchida et al.\cite{uchida2017embedding}, the watermark-related weights are first selected from the target model and then projected to a binary string $s'$ via the FCN-based extractor during the ownership verification procedure.



% \noindent\textbf{Discussion.}
% Simply replacing the linear transformation matrix in Uchida et al. \cite{uchida2017embedding} to a learnable extractor can not completely eliminate the removal threats from our attack based on model structural obfuscation. As a result, RIGA has the similar vulnerability of \cite{uchida2017embedding} as their watermark extraction procedures only differ into the type of extractor, which is also inexecutable due to the incompatible input dimension of the trained extractor for RIGA.


% \noindent\textbf{Evaluation Results.}
% We follow their evaluation settings to watermark Inception-V3 trained on CelebA, which achieves $95.90\%$ accuracy and $0\%$ BER \cite{code-riga}. We employ the default setups that the watermark is embedded into the third convolutional layer of the target model and the extractor is a multiple layer perceptron with one hidden layer. With our attack framework, we successfully inhibit the ownership verification of RIGA without any loss to the utility of victim model. Even applying the error-handling mechanisms, the BER of extracted message is increased to an unacceptable level. For example, when we utilize Max-First error-handling to obtain the embedded watermark, the BER is increased to $23.75\%$ when we inject the dummy neurons generated via \textit{NeuronSplit}.

% ########################################################################
\noindent\textbf{(3) IPR-GAN.}
To extend the watermarking primitive to generative adversarial networks (GANs) \cite{goodfellow2014gan}, Ong et al. present the first model watermark framework which first invokes black-box verification to collect some evidence from the suspect model via remote queries, and then utilizes the white-box verification for further extracting the watermark from the specific weights of suspicious model through the law enforcement. Different from \cite{uchida2017embedding}, Ong et al. embed the identification information into the scale parameters $\gamma$ of the normalization layers, rather than the convolutional weights. Correspondingly, the transformation function used in watermark verification stage consists of only a hard threshold function $T_h$, which actually extracts the signs of $\gamma$ in selected normalization layers as a binary string, i.e., 
%%%%%%%%%%%%%%%%%%%%% begin eq iprgan %%%%%%%%%%%%%%%%%%%%%
% \begin{equation}\label{eq:iprgan}
$s' = T_h(\gamma).$
% \end{equation}
%%%%%%%%%%%%%%%%%%%%% end eq iprgan %%%%%%%%%%%%%%%%%%%%%

\noindent$\bullet$\textbf{ Discussion.}
We focus on the white-box part of the watermark method. Previous works have shown that the scale parameters $\gamma$ of normalization layers are more stable than the convolution weights against existing removal attacks and ambiguity attacks, as small perturbation to these watermark-related parameters would cause significant drops to the original model utility. However, the number of scale weights in the watermark-related layer can be increased once we inject a group of dummy neurons. As a result, the length of binary string $s'$ extracted by the transformation function $T_h$ in this watermarking scheme is incompatible with the length of the target watermark any longer.
%As a result, the extracted binary string in Equation \ref{eq:iprgan} can be arbitrarily modified. 
% The capacity of embedded information is constrained by the total number of channels in normalization layers.

\noindent$\bullet$\textbf{ Evaluation Results.}
We follow their evaluation setups to watermark DCGAN trained on the CUB200 dataset, which achieves $54.33$ in terms of FID and has $0\%$ BER \cite{code-iprgan}.
As the watermark verification procedure is blocked with our proposed removal attacks, we leverage the error-handling methods on the scale weights to measure the BER of the extracted signature, which only has at most $55.86\%$ matched bits to the pre-defined binary signature while the FID of the generator is perfectly preserved as $54.33$, as  Fig.\ref{fig:scaled_ber} shows.

% ########################################################################
\noindent\textbf{(4) Greedy.}
Liu et al. \cite{liu2021greedyresiduals} propose to greedily select fewer yet more important model weights called the \textit{greedy residuals}, which is more important to the normal model utility and hence improves the watermark robustness against previous attacks. Specifically, the method extracts the identity information by first applying the transformation function $A$ on the one-dimensional average pooling over the flattened parameters $\hat w$ in the chosen convolutional layers, and then greedily takes the largest absolute values in each row by a ratio of $\eta$ to build the residuals. Finally, the secret binary string can be extracted from the signs of residuals by hard threshold function $T_h$ after being averaged to a real-valued vector.
Formally, the extraction procedure can be written as
%%%%%%%%%%%%%%%%%%%%% begin eq greedy %%%%%%%%%%%%%%%%%%%%%
% \begin{equation}\label{eq:greedy}
$
s' = T_h(Avg(Greedy(Avg\_pool\_1D(\hat w)))).
$
% \end{equation}
%%%%%%%%%%%%%%%%%%%%% end eq greedy %%%%%%%%%%%%%%%%%%%%%

\noindent$\bullet$\textbf{ Discussion.}
Greedy Residuals utilize some important parameters for embedding, which are more stable than choosing all the convolution weights in the specific layer proved in their ablation evaluations. Moreover, this watermark scheme greedily select the larger absolute values in each row from the intermediate feature matrix to build the residuals with fixed number of values, the verification procedure is not inhibited with the injection of dummy neurons. However, as the average pooled feature matrix before residual construction is impacted by some arbitrary values introduced by the injected dummy neurons, the extracted watermark after our attack would be perturbed unexpectedly. 
% As Greedy Residuals is still executable on the obfuscated model, we do not report the results of error-handling the watermark extraction algorithm in Table \ref{tab:eval}.

\noindent$\bullet$\textbf{ Evaluation Results.}
We run the source codes of Greedy Residuals publicly released by the authors \cite{code-greedy} to reproduce a watermarked ResNet18 training on Caltech256 dataset with $55.05\%$ accuracy and $0\%$ BER. We embed the secret watermark on the parameters of the first convolution layer with $\eta = 0.5$. We prove that our removal attack can utterly destroy the model watermark embedded into the residual of fewer parameters, for example, leading to an increase in the BER to $57.57\%$ after injecting dummy neurons from \textit{NeuronClique} whereas the model utility remains unchanged.

% ########################################################################
% sparsity pattern 
\noindent\textbf{(5) Lottery.}
The Lottery Ticket Hypothesis (LTH) explores a new scheme for compressing the full model to reduce the training and inference costs. As the topological information of a found sparse sub-network (i.e., the winning ticket) is a valuable asset to the owners, Chen et al. propose a watermark framework to protect the IP of these sub-networks \cite{chen2021lottery}. Specifically, they take the structural property of the original model into account for ownership verification via embedding the watermark into the weight mask in several layers with highest similarity to enforce the sparsity masks to follow certain 0-1 pattern. The proposed lottery verification uses the QR code to increase the capacity of the watermark method. For watermark verification, this algorithm first selects a set of watermark-related weight masks $m$ and then averages the chosen masks to a 2-dimensional matrix, which is further transformed to a QR code via $Sign$ function, i.e., $s' = Sign(Avg(m))$ and can be validated via a QR scanner. 
%%%%%%%%%%%%%%%%%%%%% begin eq lottery %%%%%%%%%%%%%%%%%%%%%
% \begin{equation}\label{eq:lottery}
% s' = T_h(Avg(m))
% \end{equation}
%%%%%%%%%%%%%%%%%%%%% end eq lottery %%%%%%%%%%%%%%%%%%%%%

\noindent$\bullet$\textbf{ Discussion.}
While most existing watermark techniques explore the specific model weights or prediction to embed the secret watermark, the lottery verification leverages the sparse topological information (i.e., the weight masks) to protect the winning ticket by embedding a QR code which can be further decoded into the secret watermark. However, our attack directly obfuscates the topological information of the target model by injecting groups of dummy neurons with the corresponding weight masks, which enlarges the shape of extracted QR code from the weight mask of trained winning ticket unexpectedly. As this QR code without valid shape is not decodable, we leverage the error-handling mechanism to extract the embedded watermark for ownership verification, where remains a large distortion.

%%%%%%%%%%%%% BEGIN OF ls fig
\begin{figure}[t]
\begin{center}
\includegraphics[width=0.4\textwidth]{img/qrcode_dn.png}
\vspace{-0.2in}
\caption{The QR codes extracted from ResNet-20 watermarked by\cite{chen2021lottery} after an $\alpha$ ratio of dummy neurons are added.}
\label{fig:qrcode}
\end{center}
\vspace{-0.3in}
\end{figure}
%%%%%%%%%%%%%% END OF ls fig

\noindent$\bullet$\textbf{ Evaluation Results.}
We follow the evaluation settings in the original paper to watermark a ResNet20 training on CIFAR-100 dataset, which achieves $66.41\%$ accuracy and $0\%$ BER \cite{code-lottery}. Although the QR code has the ability to correct some errors which improves the robustness against existing attacks, the identity information decoding procedure from the extracted QR code is invalidated by adding only a few (e.g., $1\%$)  dummy neurons in the victim model as Fig.\ref{fig:qrcode} shows. Moreover, the original design of Lottery Ticket is inhibited (due to the unmatched size) when attackers insert the dummy neurons into the protected model, while it retains robust against structure obfuscation with our proposed error-handling mechanisms. In other words, Lottery Ticket would have better robustness against neural structural obfuscation than other schemes if the unmatched size of neural layers are properly addressed in its design. 

% ########################################################################
\subsection{Attacking Activation-based Watermarks}
\noindent\textbf{(1) DeepSigns.}
As a representative of activation-based white-box watermarking schemes, DeepSigns proposes to embed the model watermark into the probability density function (PDF) of the intermediate activation maps obtained in different layers on the white-box scenario. Specifically, DeepSigns adopts a Gaussian Mixture Model (GMM) as the prior probability to characterize the hidden representations, and considers the mean values of the PDF at specific layers to share the same role as the watermark-related weights in Uchida et al. \cite{uchida2017embedding}. Similar to the verification procedure of \cite{uchida2017embedding}, a transformation matrix $A$, randomly sampled in embedding procedure, projects the mean values of chosen intermediate features to a real-valued vector. With the final hard threshold function, the resulted binary string $s'$ is matched to the owner-specific watermark for claiming the model ownership. 

%%%%%%%%%%%%%%%%%%%%% begin eq greedy %%%%%%%%%%%%%%%%%%%%%
% \begin{equation}\label{eq:deepsign}
% s' = T_h(A \cdot \frac{1}{|D_T|}h^l(x_t; W^l_{in}))
% \end{equation}
%%%%%%%%%%%%%%%%%%%%% end eq greedy %%%%%%%%%%%%%%%%%%%%%

\noindent$\bullet$\textbf{ Discussion.}
% Most known attacks focus on carefully fine-tuning the victim model or utilizing some transformations to the trigger data to remove the inner watermarks embedded by DeepSigns \cite{chen2021refit}. As the element-level representations of the weights at specific layer (or channel) are closely related to the output feature maps, DeepSigns is almost as vulnerable as Uchida et al's under these attacks. However, these known attacks again involve impractical assumptions on the attacker's knowledge and sacrifice much of the normal model utility for fully removing the watermark. As a comparison, our attack utilize the local features of 
The notable difference between DeepSigns and \cite{uchida2017embedding} is where to embed the model watermark. 
However, as the hidden features utilized by DeepSigns are generated by the weights in the preceding layer, e.g., $a_i = W_i\cdot x+b_i$, the structural information of target model is closely related to the shape of output feature maps. 
For example, the shape of the watermark-related layer's output is expanded after injecting dummy neurons, which inhibits the ownership verification due to the incompatible dimension of the output activation maps and the pre-defined transform matrix.
As a result, DeepSigns is almost as vulnerable as \cite{uchida2017embedding} under our attack.


\noindent$\bullet$\textbf{ Evaluation Results.}
We run the source code of DeepSigns from \cite{code-deepsign} to watermark a wide residual network trained on CIFAR10. This watermarked WRN achieves $89.94\%$ accuracy and $0\%$ BER. With the error-handling method, it is shown that the ownership verification of the target model is completely confused by our removal attacks. For example, the BER is increased to $47.59\%$ with dummy neurons from \textit{NeuronSplit} while the original model functionality is intact.

\noindent\textbf{(2) IPR-IC.}
As previous watermarking schemes are mainly designed for image classification models, they are insufficient to IP protection for image captioning models and cause inevitable degradation to the image captioning performance. Therefore, Lim et al. \cite{lim2022ipcaption} embed a unique signature into Recurrent Neural Network (RNN) through hidden features. At the ownership verification stage, the mask matrix $M$ first selects the hidden memory state $h$ of given watermarking image in protected RNN model. Then, the hard threshold function transforms the chosen $h$ to a binary string $s'$, which can be formally written as
%%%%%%%%%%%%%%%%%%%%% begin eq lstm %%%%%%%%%%%%%%%%%%%%%
$
s' = T_h(h).$
%%%%%%%%%%%%%%%%%%%%% end eq lstm %%%%%%%%%%%%%%%%%%%%%

\noindent$\bullet$\textbf{ Discussion.}
Similar to DeepSigns \cite{darvish2019deepsigns} and IPR protection on GANs \cite{ong2021iprgan}, the topological information is closely related to the shape of hidden memory state. Although the protected image captioning model contains an RNN architecture, we can adopt our structural obfuscation method to expand the size of hidden state, e.g., with vanishing weights, to produce an equivalence of the original model with the same output.

\noindent$\bullet$\textbf{ Evaluation Results.}
We run the official implementation \cite{code-captioning} to reproduce a watermarked Resnet50-LSTM trained on COCO, which achieves $72.06$ BLEU-1 and has $0\%$ BER. With our proposed removal attacks, the signature extracted from the hidden memory state $h$ is not compatible to the size of the owner-specific binary message. We leverage error-handling mechanisms, e.g., Max-First to extract the identity message, which is perturbed with $56.95\%$ and $52.60\%$ BER for \textit{NeuronClique} and \textit{NeuronSplit}, while no loss is brought to the image captioning performance.

% \subsection {Passport-based}
% ########################################################################
\subsection{Attacking Passport-based Watermarks} %\label{section:DeepIPR}
\noindent\textbf{(1) DeepIPR.} 
DeepIPR is one of the earliest passport-based DNN ownership verification schemes \cite{fan2021deepip}. By inserting owner-specific passport layers during the watermark embedding procedure, DeepIPR is designed to claim the ownership not only based on the extracted signature from the specific model parameters but also on the model performance with the private passport layer. Consequently, this scheme shows high robustness to previous removal attacks and especially to the ambiguity attacks, which mainly forge counterfeit watermarks to cast doubts on the ownership verification.

In our evaluation, we focus on the following passport verification scheme in \cite{fan2021deepip}. This scheme generates two types of passport layers simultaneously by performing a multi-task learning, i.e., public passports for distribution and private passports for verification, both of which are actually based on normalization layers. Generally, DeepIPR leverages pre-defined digital passports $P=\{P_{\gamma}, P_{\beta}\}$ to obtain the scale and the bias parameters of the private passport, which are written as:
$
\gamma = Avg(W_c \odot P_{\gamma}), \beta =  Avg(W_c \odot P_{\beta}),
$
where $W_c$ is the filters of the precedent convolution layer, and $\odot$ denotes the convolution operation. DeepIPR adopts a similar watermark extraction process as \cite{uchida2017embedding}, where the transformation function $A$ converts the signs of the private $\gamma$ into a binary string to match the target signature.

\noindent$\bullet$\textbf{ Discussion.}
As the private $\gamma$ and $\beta$ are obtained from the preceding convolution weights, DeepIPR actually embeds the secret signature in the hidden output of the convolutional layer with the weights $W_c$ given the input $P_{\gamma}$ or $P_{\beta}$. Similar to the activation-based watermarking scheme, the unmatched extracted watermark can not be used to ownership verification due to the expanded shape of $W_c$.
% Specifically, the signs of private $\gamma$ can be modified directly by changing the signs of related convolution weights, \mytodo{which causes a significant damage to the watermark detection procedure and model performance with private layers (this sentence is ambiguous. Needs discussion).}
% However, because the joint training for different passport layers leads to different distribution of the feature maps before normalization, DeepIPR calculates statistic values of normalization layers on-the-fly by replacing the batch normalization to other normalization, which causes noticeable drops to the original performance



% %%%%% BEGIN sig TABLE
% % For accuracy, x/y stands for the model performance with public passport and private passport respectively.


% Table generated by Excel2LaTeX from sheet 'Sheet1'
\begin{table}[htbp]
  \centering
   \caption{Extracted signatures and the model accuracy when an $\alpha$ ratio of neurons are modified by our attack.}
    \begin{tabular}{clc}
    \toprule
    \textbf{$\alpha$} & \textbf{Signature s} & \textbf{BER (\%)} \\
    \midrule
    0     &$this\ is\ my\ signature$ &  0  \\
    % 1/16  & $ôhis\backslash xa0ió\ íq\backslash xa0óùgîaôõðe$ & 8.04  & 74.68 (1.00) \\
    % 1/8   & $ÔHéS\backslash xa0Éñ@\backslash xadY\backslash xa0ÓëN¦AôEòE$ & 16.46  & 74.68 (1.00) \\
    % 1/4   &$XîEw.oN\&!\backslash x7f\$õuaF÷zã~a$ &  30.28  & 74.68 (1.00) \\
    % 1/2   & $±*:ã\backslash x12â\%\backslash x9e W3Ú,\backslash x87d\backslash x11E\backslash x91$ & 45.83  & 74.68 (1.00) \\
    % 1     & $@zs\backslash x9d~ó6\backslash x03Ë6á\backslash x95Ó\backslash x00\backslash x1a=\backslash x8dñUÇ$ & 50.96  & 74.68 (1.00) 
    0.25  & $s(gÂÊ'ëË=b\}2Ûe¡öDüûj$ & 43.75  \\
    0.5   & $ÎÍ¿±C¾Ýz!²/ü½¤L°Ã9Ûå$ & 51.25   \\
    0.75   & $p£äêßGÊqXBu¨oÓ\{G 襦$ &  47.50  \\
    1   & $h!*ƧxkiC_"h!*ƧxkiC$ & 39.38  \\
    \bottomrule
    \end{tabular}%
  \label{tab:sig1}%
\end{table}%

% %%%%% END sig TABLE



\noindent$\bullet$\textbf{ Evaluation Results.}
We evaluate our attack on the watermarked ResNet18 trained on the CIFAR-100 dataset with DeepIPR \cite{code-deepipr} which achieves $67.94\%$ accuracy. When we inject an $\alpha$ proportion of neurons with our attack, the signature extracted from the victim model becomes totally unreadable, from \textit{``this is my signature''} ($\alpha =0$, BER$=0\%$) to \textit{``ÎÍ¿±C¾Ýzü½¤L°!²/Ã9Ûå''} ($\alpha=0.5$, BER$=51.25\%$). By injecting $50\%$ of dummy neurons, our attack
successfully increases the BER to almost random, while causing no change in the accuracy of the model with the public passports. 

% ########################################################################
\noindent\textbf{(2) Passport-aware Normalization.} Zhang et al. \cite{zhang2020passportaware} propose another passport-based watermark method without modifying the target network structure by maintaining the statistic values independently for passport layer. As the watermark embedding and extracting procedures are nearly identical to DeepIPR, we report the results in  Fig.\ref{fig:scaled_ber} and provide the details of Passport-aware Normalization in Appendix \ref{sec:app:eval}.
% \noindent\textbf{Protection Mechanism.} 
% Zhang et al. \cite{zhang2020passportaware} propose another passport-based watermark method without modifying the target network structure, which would otherwise incur notable performance drops. They adopt a simple but effective strategy by training the passport-free and passport-aware branches in an alternating order and maintaining the statistic values independently for the passport-aware branch at the inference stage. Similar to DeepIPR, the authors design the learnable $\gamma, \beta$ to be relevant to the original model for stronger ownership claim. During the extraction of model watermarks, the transformation function $A$ first projects the $\gamma$ by an additional FCN model to an equal-length vector and then utilizes the signs of the vector to match the target signature.

% \noindent\textbf{Discussion.}
% While this method improves DeepIPR in terms of model performance by preserving the network structure and improving transformation function $A$ with linear transformation and sign function, we discover it is still inexecutable because of the incompatible dimensions between the extracted watermark and target one.


% \noindent\textbf{Evaluation Results.}
% When we embed the model watermark into a ResNet18 trained on the CIFAR-100 via passport-aware normalization \cite{code-aware}, we are able to achieve $0\%$ BER, while preventing the original model utility from unacceptable drops. Our proposed structural obfuscation attacks demonstrate sufficient effectiveness to remove this white-box watermark and invalidate the passport-aware branch independently as Table \ref{tab:eval} shows. For example, with the error-handling of Max-First, the injection of dummy neurons generated by \textit{NeuronSplit} can boost the BER to $49.61\%$.


\printbibliography{}
\appendices
\section{Monotone Cone}
Let $\Re^n_{\uparrow} \dfn \{x \in \Re^n \colon x_1 \leq x_2 \leq \dots \leq x_n\}$ 
denote the monotone cone. This cone and its dual have a history in isotonic regression \cite{Barlow1972}
and majorization \cite{Steerneman1990}. 

% We begin by deriving the polar of $\Re^n_{\uparrow}$. The following property will be useful for this.
% \begin{lemma} \label{lem:inner-product}
%     Let $x, y \in \Re^n$ and $x_{0} = 0$. Then 
%     \begin{equation*}
%         \ssum_{i=1}^{n} x_i y_i = \ssum_{i=1}^{n} (x_{i} - x_{i-1}) \left( \ssum_{j=i}^{n} y_j \right).
%     \end{equation*}
% \end{lemma}
% \begin{proof}
%     For some $k \in [n]$, $x_k$ enters in the sum on the right-hand side for $i = k$ and $i=k+1$.
%     The full term is thus
%     \begin{align*}
%         x_k \left(\ssum_{j=k}^{n} y_j\right) - x_k \left(\ssum_{j=k+1}^{n} y_j\right) = x_k y_k.
%     \end{align*}
%     Using this argument for each $k$ completes the proof. 
% \end{proof}

We first derive the polar of $\M^n$.
\begin{lemma} \label{prop:dual-monotone-cone}
    Let $\M^n$ be the monotone cone.
    Then 
    \begin{align*}
        (\M^n)^\circ = \left\{ x \in \Re^n \colon \sum_{i=1}^k x_i \geq 0, \forall k \in [n-1], \sum_{i=1}^n x_i = 0 \right\}. 
    \end{align*}
\end{lemma}
\begin{proof}
    The monotone cone is polyhedral with $\M^n = \{x \colon Mx \leq 0\}$ for 
    $M \in \Re^{n-1 \times n}$ with $Mx = (x_1 - x_2, x_2 - x_3, \dots, x_{n-1} - x_n)$. 
    The definition of the polar cone is thus 
    \begin{align*}
        (\M^n)^\circ &= \left\{ y \in \Re^n \colon \<x, y\> \leq 0, \forall x \text{ s.t. } M x \leq 0 \right\}.
    \end{align*}
    By Farkas' lemma \cite[p.~263]{Boyd2004} we have either $Mx \leq 0$ and $\<x, y\> > 0$ or $\trans{M} \lambda = y$ and $\lambda \geq 0$. 
    So
    \begin{align*}
        (\M^n)^\circ = \left\{ y \in \Re^n \colon y = \trans{M} \lambda, \lambda \geq 0 \right\}.
    \end{align*}
    Note that $\<\lambda, Mx\> = \sum_{i=1}^{n-1} \lambda_i (x_i - x_{i+1}) = \sum_{j=1}^{n} x_j (\lambda_{j} - \lambda_{j-1}) = \<\trans{M} \lambda, x\>$,
    where $\lambda_0 = \lambda_{n} = 0$. Thus $y \in (\M^n)^\circ$ iff $y = \trans{M} \lambda$ for $\lambda \geq 0$. Here $y = \trans{M} \lambda$ holds iff 
    \begin{align*}
        y_j &= \lambda_j - \lambda_{j-1}, &&\forall j \in [n].  \\
        \Leftrightarrow \quad \ssum_{j=1}^k y_j &= \ssum_{j=1}^k \lambda_j - \lambda_{j-1} = \lambda_k, &&\forall k \in [n]. 
    \end{align*}
    Since $\lambda \geq 0$ we have $\sum_{j=1}^k y_j = \lambda_k \geq 0$ for $k \in [n-1]$ and $\sum_{j=1}^{n} y_j = \lambda_n = 0$. 
    % We can rewrite the inner product using \cref{lem:inner-product} as
    % \begin{equation*}
    %     \ssum_{i=1}^{n} (y_{(i)} - y_{(i-1)}) ( \ssum_{j=i}^{n} x_j ) \leq 0, \quad \forall y \in \Re^n.
    % \end{equation*}
    % First note that $y_{(1)} - y_{(0)} = y_{(1)}$ has arbitrary sign.
    % Since $y_{(1)} - y_{(0)} = y_{(1)}$ has arbitrary sign, we have $\sum_{i=1}^{n} x_i = 0$.  
    % Note that $y_{(i)} \geq y_{(i-1)}$ for all $i \in [2, n]$. Therefore, $\forall k \in [n-1]$,
    % \begin{align*}
    %     \ssum_{i=k+1}^n x_i \leq 0 \, &\Leftrightarrow \, \ssum_{i=1}^{n} x_i - \ssum_{i=k+1}^n x_i \geq 0 \\
    %     &\Leftrightarrow \, \ssum_{i=1}^{k} x_i \geq 0,
    % \end{align*}
    % thereby proving the required result.
\end{proof}

% The cone is deeply linked to the permutation group $\Pi^n$ of bijections $\pi \colon [n] \to [n]$. 
% Consider \cite[Eq.~2.2]{Steerneman1990}
% \newcommand{\match}{\mmathcal{m}}
% \begin{definition}
%     Consider the \emph{matching function}
%     \begin{equation*}
%         \match(u, x) = \sup_{\pi \in \Pi^n} \, \<u, \pi x\>, \quad \forall x,u \in \Re^n.
%     \end{equation*}
% \end{definition}

% The matching function was introduced to study group induced cone orderings (e.g. majorization) \cite{Steerneman1990}.
% The ordering induced by the permutation group is majorization \cite{Marshall2011}:
% \begin{definition}
%     For $x, y \in \Re^n$, then $x \slt y$ iff $x \in \hull (\Pi^n y)$. 
% \end{definition}

% We first link matching with majorization \cite[\S2]{Steerneman1990}
% \begin{proposition}
%     For $x, y \in \Re^n$, then $x \slt y$ iff
%     \begin{equation*}
%         \match(u, x) \leq \match(u, y) \quad \text{for all } u \in \Re^n.
%     \end{equation*} 
%     % which holds iff $x_{\uparrow} - y_{\uparrow} \in (\M^n)^\circ$.
% \end{proposition}

% Meanwhile matching relates to $\M^n$ as follows 
% \begin{proposition} \label{prop:fundamental-region}
%     For each $x, y \in \M^n$: 
%     \begin{enumerate}[(i)]
%         \item $\Pi^n x \cap \M^n \neq \emptyset$; \, (ii) $\match(x, y) = \<x, y\>$
%     \end{enumerate}
% \end{proposition}
% \begin{proof}
%     \emph{(ii)} follows by \cite[Prop.~6.A.3]{Marshall2011} and \emph{(i)} follows by noting 
%     that $\exists \pi \in \Pi^n$ s.t. $\pi(x) = x_{\uparrow}$.
% \end{proof}
% The above facts identify $\M^n$ as the closure of a \emph{fundamental region} \cite[Thm.~3.1]{Steerneman1990}
% and $x \slt y$ as a \emph{group-induced cone ordering} \cite[Def.~2.1]{Steerneman1990}.
% This has several consequences. Most importantly it implies the following \cite[\S2]{Steerneman1990}:
% \begin{proposition} \label{prop:group-induced-cone-order}
%     Consider $x, y \in \Re^n$. Then 
%     \begin{equation*}
%         x \slt y \quad \text{iff} \quad x_{\uparrow} - y_{\uparrow} \in (\M^n)^\circ.
%     \end{equation*}
% \end{proposition}

We relate $(\M^n)^\circ$ to majorization. 
\begin{lemma} \label{prop:group-induced-cone-order}
    Consider $x, y \in \Re^n$. Then
    \begin{equation*}
        x_{\uparrow} \in y_{\uparrow} + (\M^n)^\circ \quad \text{iff} \quad x \slt y,
    \end{equation*}
    where $x \slt y$ denotes that $x$ is majorized by $y$. 
\end{lemma}
\begin{proof}
    The definition of majorization \cite[Def.~1.A.1]{Marshall2011} is:
    \begin{align*}
        x \slt y \, \Leftrightarrow \, \sum_{i=1}^{n} x_{(i)} \geq \sum_{i=1}^{n} y_{(i)} \text{ and } \sum_{i=1}^{n} x_i = \sum_{i=1}^{n} y_i.
    \end{align*}
    This is clearly equivalent to \cref{prop:dual-monotone-cone} applied to $x_{\uparrow} - y_{\uparrow}$. 
\end{proof}

% The characterization in terms of a group-induced cone ordering also implies the following 
% characterization of the convex hulls of the orbit under $\Pi^n$:
Majorization is related to the convex hull of the orbit under $\Pi^n$. Specifically \cite[Cor.~2.B.3]{Marshall2011}:
\begin{equation} \label{eq:permuto-hull-eq}
    x \slt y \quad \Leftrightarrow \quad x \in \hull(\Pi^n y). 
\end{equation}

This has the following useful consequence.
\begin{lemma} \label{prop:permuto-hull}
    Consider $y \in \Re^n$. Then 
    \begin{equation*}
        \hull (\Pi^n y) = \bigcap_{\pi \in \Pi^n} \pi(y_{\uparrow} + (\M^n)^\circ).
    \end{equation*}
    % \begin{enumerate}[(i)]
    %     \item \label{prop:permuto-hull:a} $\hull (\Pi^n y) \subseteq y_{\uparrow} + (\M^n)^\circ$; 
    %     \item \label{prop:permuto-hull:b} 
    %     % \item $\hull (\Pi^n y) = \cup_{\pi \in \Pi^n} \pi(\M^n \cap (y + (\M^n)^\circ))$.
    % \end{enumerate}
\end{lemma}
\begin{proof}
    The result is a specialization of \cite[Lem.~4.2]{Steerneman1990}. Take $x \in \hull (\Pi^n y)$.
    Then $x \slt y$ by \cref{eq:permuto-hull-eq} and $x_{\uparrow} - y_{\uparrow} \in (\M^n)^\circ$ by \cref{prop:group-induced-cone-order}.
    Using the definition of the polar cone this gives:
    \begin{align*}
        &\<x_{\uparrow} - y_{\uparrow}, u_{\uparrow}\> \leq 0, &&\forall u_{\uparrow} \in \M^n \\
        \Leftrightarrow \quad &\<x_{\uparrow}, u_{\uparrow}\> \leq \<y_{\uparrow}, u_{\uparrow}\>, &&\forall u_{\uparrow} \in \M^n \\
        \Rightarrow \quad &\<x, u_{\uparrow}\> \leq \<y_{\uparrow}, u_{\uparrow}\>, &&\forall u_{\uparrow} \in \M^n, \\
        \Leftrightarrow \quad &x - y_{\uparrow} \in (\M^n)^\circ,
    \end{align*}
    where we used $\<x, u_{\uparrow}\> \leq \<x_{\uparrow}, u_{\uparrow}\>$ \cite[Prop.~6.A.3]{Marshall2011} for the third implication
    and the definition of the polar cone again for the fourth. Thus we have shown that $x \in y_{\uparrow} + (\M^n)^\circ$.
    As we could have taken any $x \in \hull(\Pi^n y)$ we thus showed $\hull (\Pi^n y) \subseteq y_{\uparrow} + (\M^n)^\circ$.
    We can take permutations of both sides without affecting the result (since permutations are invertible). Moreover $\pi \hull(\Pi^n y) = \hull(\Pi^n y)$.
    Thus $\hull (\Pi^n y) \subseteq \pi(y_{\uparrow} + (\M^n)^\circ)$ for all $\pi \in \Pi^n$ and $\hull (\Pi^n y) \subseteq \cap_{\pi} \pi(y_{\uparrow} + (\M^n)^\circ)$. 

    For the converse let $x \in \cap_{\pi \in \Pi^n} \pi(y_{\uparrow} + (\M^n)^\circ)$. So $x_{\uparrow} \in y_{\uparrow} + (\M^n)^\circ$. 
    By \cref{prop:group-induced-cone-order} we have $x \slt y$. Applying \cref{eq:permuto-hull-eq} then implies $x \in \hull(\Pi^n y)$.
    Therefore $\cap_{\pi \in \Pi^n} \pi(y_{\uparrow} + (\M^n)^\circ) \subseteq \hull(\Pi^n y) $, completing the proof.
\end{proof}

We can generalize the above to apply to convex, permutation invariant sets. 
\begin{proposition} \label{cor:cone-shift-invariance}
    Consider some convex permutation invariant $\set{A} \subseteq \Re^n$ (i.e. $x \in \set{A}$ implies $\pi x \in \set{A}$ for all $\pi \in \Pi^n$).
    Then 
    \begin{equation*}
        \set{A} = \bigcap_{\pi \in \Pi^n} \pi(\set{A} + (\M^n)^{\circ}).
    \end{equation*}
\end{proposition}
\begin{proof}
    Let $x \in \set{A}$. By \cref{prop:permuto-hull} we have $\pi(x) \in x_{\uparrow} + \M^n$ for all $\pi \in \Pi^n$. 
    Noting that $x_{\uparrow} \in \set{A}$ by permutation invariance gives $\pi(x) \in \set{A} + \M^n$ for all $\pi \in \Pi^n$. 
    Thus $x \in \cap_{\pi} \pi(\set{A} + (\M^n)^\circ)$ and $\set{A} \subseteq \cap_{\pi} \pi(\set{A} + (\M^n)^\circ)$.

    For the converse let $x \in \cap_{\pi} \pi(\set{A} + (\M^n)^\circ)$. Then $\pi(x) \in \set{A} + (\M^n)^\circ$ for all $\pi \in \Pi^n$.
    Specifically $x_{\uparrow} \in \set{A} + (\M^n)^\circ$, which implies $\exists y \in \set{A}$ such that $x_{\uparrow} \in y + (\M^n)^\circ$. 
    Note that $y \in y_{\uparrow} + (\M^n)^\circ$ by \cref{prop:permuto-hull}. That is $\exists s \in (\M^n)^\circ$ such that $y = y_{\uparrow} + s$
    and $x_{\uparrow} \in y + s + (\M^n)^\circ$. Since $(\M^n)^\circ$ is a convex cone, $s + s' \in (\M^n)^\circ$ for $s, s' \in (\M^n)^\circ$.
    Thus $x_{\uparrow} \in y_{\uparrow} + (\M^n)^\circ$, which by \cref{prop:group-induced-cone-order} implies $x \slt y$ or $x \in \hull(\Pi^n y)$ \cref{eq:permuto-hull-eq}. 
    From $y \in \set{A}$ and permutation invariance $\Pi^n y \subseteq \set{A}$. Moreover, by convexity, $\hull(\Pi^n y) \subseteq \set{A}$. Thus $x \in \set{A}$ 
    and $\cap_{\pi} \pi(\set{A} + (\M^n)^\circ) \subseteq \set{A}$. 
\end{proof}

We finish with a characterization of the convex hull of the permutations of some vector as a polyhedral set.
\begin{proposition} \label{lem:simple-distortion-characterization}
    Consider a vector $x \in \Re^n$ with $d$ distinct elements, which we store in $\bar{x} \in \Re^d$. Let $c \in \N^d$ denote 
    the number of copies of each component of $\bar{x}$ in $x$. Then 
    \begin{align*}
        \hull\left( \Pi^n x \right) = \{ S\bar{x} \colon S\one_d = \one_n, \trans{\one}_n S = \trans{c}, S \geq 0\}.
    \end{align*}
\end{proposition}
\begin{proof}
    Since the doubly stochastic matrices are the convex hull of the permutation matrices \cite[2.A.2]{Marshall2011}, the convex hull of all permutations of $x$ is
    \begin{equation*}
        \hull\left( \Pi^n x \right) = \left\{ H x \colon H \one_n = \one_n, \trans{\one}_n H = \one_n, H \geq 0 \right\}.
    \end{equation*}
    Consider
    \begin{equation*}
        \trans{C} \dfn \smallmat{
            1 & \dots & 1 \\ &&& \ddots \\ &&&& 1 & \dots & 1
         } \in \Re^{d \times n},
    \end{equation*}
    where row $i$ contains $c_i$ repetitions of $1$ for each $i \in [d]$ such that $x = C \bar{x}$. The pseudo-inverse $C^{\dagger}$ is given as 
    \begin{equation*}
        C^{\dagger} \dfn \smallmat{
            1/c_1 & \dots & 1/c_1 \\ &&& \ddots \\ &&&& 1/c_d & \dots & 1/c_d
        } \in \Re^{d \times n}.
    \end{equation*}
    So $\bar{x} = C^{\dagger} x$ and $C^{\dagger} C = I_d$. 

    Take $H$ doubly stochastic. Then $v = H C C^{\dagger} x = HC \bar{x} \in \hull\left( \Pi^n x \right)$ and $S = HC$ satisfies the conditions above. 
    Specifically  $H \trans{C} \one_d = H \one_n = \one_n$ and $\trans{\one_n} H \trans{C} = \trans{\one_n} \trans{C} = \trans{\one_d}$ and $H \trans{C} \geq 0$.
    Let $\set{R}$ denote the right-most set in the proposition. We have shown $v \in \set{R}$ or $\hull\left( \Pi^n x \right) \subseteq \set{R}$. 
    For the reverse implication take some $v \in \set{R}$. So $v = S y = S C^{\dagger} C y = S C^{\dagger} x$. It is easy to verify that $S C^{\dagger}$ is doubly stochastic.
    Thus $v \in \hull\left( \Pi^n x \right)$. 
\end{proof}


% \section{Monotone Cone}

% Consider the monotone cone. We first derive its dual. 
% \begin{lemma} \label{lem:dual-monotone-cone}
%     Let $\M^n$ be the monotone cone.
%     Then 
%     \begin{align*}
%         (\M^n)^\circ &\dfn (\M^n)^\circ \\
%         &= \left\{ x \in \Re^n \colon \sum_{i=k}^n x_i \leq 0, \forall k \in [2, n], \sum_{i=1}^n x_i = 0 \right\}. 
%     \end{align*}
% \end{lemma}
% \begin{proof}
%     Plugging in the definition of the dual cone gives 
%     \begin{align*}
%         (\M^n)^\circ &= \left\{ y \in \Re^n \colon \<x, y\> \leq 0, \forall x \in \M^n \right\}.
%     \end{align*}
%     We can replace the inner product using \cref{cor:inner-product} with 
%     \begin{equation*}
%         \ssum_{i=1}^{n} (x_{(i)} - x_{(i-1)}) ( \ssum_{j=i}^{n} y_j ) \leq 0, \quad \forall x \in \Re^n
%     \end{equation*}
%     Note that $x_{(i)} \geq x_{(i-1)}$ for all $i \in [2, n]$, which gives the first set of constraints. 
%     The final constraint follows by noting that $x_{(1)} - x_{(0)} = x_{(1)}$ has arbitrary sign.  
% \end{proof}

% We do the same for the nonnegative monotone cone.
% \begin{lemma} \label{lem:dual-nn-monotone-cone}
%     Let $\pM^n$ be the nonnegative monotone cone.
%     Then 
%     \begin{align*}
%         (\M^n)^\circ_+ &\dfn (\pM^n)^* = \left\{ x \in \Re^n \colon \sum_{i=k}^n x_i \geq 0, \forall k \in [n] \right\}. 
%     \end{align*}
% \end{lemma}
% \begin{proof}
%     The proof is analogous to that of \cref{lem:dual-nn-monotone-cone}. Except for the final step where we known $x_{(1)} \geq 0$.
% \end{proof}

% We consider properties of the \emph{cumulative cone}. We first derive a generator characterization.
% \begin{lemma} \label{lem:cumulative-cone}
%     Let $(\M^n)^\circ$ be as given in \cref{lem:dual-monotone-cone}. Then
%     \[(\M^n)^\circ = \{(\tau_2-\tau_1, \tau_3 - \tau_2, \dots, -\tau_{n}) \colon \tau \in \Re^{n}_+, \tau_1 = 0\}.\]
% \end{lemma}
% \begin{proof}
%     Take some $x \in (\M^n)^\circ$ and let $\tau_{k} = -\sum_{i=k}^{n} x_i$ for $k \in [n]$.
%     Since $x \in (\M^n)^\circ$ we have $\tau \in \Re^{n}_+$ and $\tau_1 = 0$. Moreover
%     \begin{align*}
%         &(\tau_2-\tau_1, \dots, -\tau_{n}) \\
%         &= (\ssum_{i=1}^{n}x_i - \ssum_{i=2}^{n}x_i, \dots, \ssum_{i=n}^{n}x_i) \\
%         &= (x_1, \dots, x_n).
%     \end{align*}
%     To show the other inclusion let $\tau \in \Re^{n}_+$ and $\tau_1 = 0$ and $(x_1, \dots, x_n) = (\tau_2-\tau_1, \dots, -\tau_{n})$. 
%     Then $\sum_{i=k}^{n} x_i = -\tau_k \leq 0$ for $k \in [2, n]$ and $\sum_{i=1}^n x_i = \tau_1 = 0$. 
% \end{proof}

% We prove some properties of $(\M^n)^\circ$ related to permutations.
% \begin{lemma} \label{lem:permutation-hull-contained}
%     Let $(\M^n)^\circ$ be as in \cref{lem:dual-monotone-cone}. Then, $\forall x \in \Re^n$,
%     \begin{equation*}
%         \hull\left( \{\pi(x) \colon \pi \in \Pi^n\} \right) \subset x_{\uparrow} + (\M^n)^\circ. 
%     \end{equation*}
% \end{lemma}
% \begin{proof}
%     We begin by showing that for every $x$ we have $x \in x_{\uparrow} + (\M^n)^\circ$, which holds iff 
%         $x - x_{\uparrow} \in (\M^n)^\circ$
%     Let $g^{(k)} = (0, \dots, 0, 1, \dots, 1) \in \Re^n$ the $k$'th row of $G$ as in \cref{lem:upper-triangular}. 
%     Since clearly $\sum_{i=1}^{n} x_{i} - x_{(i)} = 0$ we should only show, for all $k \in [2, n]$, 
%     \begin{align*}
%         &\langle g^{(k)}, x - x_{\uparrow} \rangle = \sum_{i=1}^{n} g^{(k)}_i (x_{i} - x_{(i)}) \leq 0,
%     \end{align*}
%     which by construction implies $x - x_{\uparrow} \in (\M^n)^\circ$.
%     The above inequality holds iff 
%     \begin{align*}
%         &\Leftrightarrow \quad \sum_{i=1}^{n} g^{(k)}_{(i)} x_{i} \leq \sum_{i=1}^{n} g^{(k)}_{(i)} x_{(i)} \\
%         &\Leftrightarrow \quad \sum_{i=1}^{n} g^{(k)}_{[i]} x_{n-i+1} \leq \sum_{i=1}^{n} g^{(k)}_{[i]} x_{[i]}
%     \end{align*}
%     where we used the fact that $g^{(k)}_{\uparrow} = g^{(k)}$
%     and $x_{[i]} = x_{(n - i + 1)}$. The final inequality follows from \cite[Prop.~6.A.3]{Marshall2011}.

%     Thus we have shown that $\pi(x_{\uparrow}) \in x_{\uparrow} + (\M^n)^\circ$ for all $\pi \in \Pi^n$. 
%     Adding $x_{\uparrow}$ does not affect convexity of $(\M^n)^\circ$.
%     So since $\pi(x) \in x_{\uparrow} + (\M^n)^\circ$ for all $\pi \in \Pi^n$ with a convex right-hand side
%     we conclude $\hull\left( \{\pi(x) \colon \pi \in \Pi^n\} \right) \subset x_{\uparrow} + (\M^n)^\circ$.
% \end{proof}

% \begin{lemma} \label{lem:opposite-hull-contained}
%     For any $x \in \Re^n$ such that $x \neq x_{\uparrow}$
%     \begin{equation*}
%         \exists \pi \in \Pi^n\colon \pi(x) \notin x + (\M^n)^\circ.
%     \end{equation*}
% \end{lemma}
% \begin{proof}
%     The proof is similar to that of \cite[Thm.~368]{Hardy1952}. We have $\pi(x) \notin x + (\M^n)^\circ$ iff 
%     \begin{equation*}
%         \exists k \in [2, n]\colon \<g^{(k)}, \pi(x)\> > \<g^{(k)}, x\>,
%     \end{equation*}
%     with $g^{(k)} = (0, \dots, 0, 1, \dots, 1) \in \Re^n$ the $k$'th row of $G$ as in \cref{lem:upper-triangular}.
%     Take $i$ such that $x_{i} > x_{i+1}$, which exist by $x \neq x_{\uparrow}$. Take $k$ such that $g^{(k)}_{i} = 0$
%     and $g^{(k)}_{i+1} = 1$ (i.e. $k = i + 1$). Let $\pi(i) = i+1$, $\pi(i+1) = i$ and $\pi(j) = j$ for all $j \in [n] \setminus \{i, i+1\}$. 
%     Then
%     \begin{align*}
%         \<g^{(k)}, \pi(x)\> &= \ssum_{j=i+2}^n x_j + x_{i} \\ &> \ssum_{j=i+2}^n x_j + x_{i+1} = \<g^{(k)}, x\>,
%     \end{align*}
%     which completes the proof.
% \end{proof}

% We relate $(\M^n)^\circ$ to majorization. 
% \begin{lemma} \label{lem:majorization}
%     Consider $x, y \in \Re^n$ and $(\M^n)^\circ \dfn (\M^n)^\circ$. Then
%     \begin{equation*}
%         x_{\uparrow} \in y_{\uparrow} + (\M^n)^\circ \quad \text{iff} \quad x \slt y,
%     \end{equation*}
%     where $x \slt y$ denotes that $x$ is majorized by $y$. 
% \end{lemma}
% \begin{proof}
%     Note that $x_{\uparrow} - y_{\uparrow} \in (\M^n)^\circ$ holds iff $\sum_{i=1}^{n} x_i - y_i = 0$ and $\forall k \in [2, n]$,
%     \begin{align*}
%         &&\ssum_{i=k}^{n} x_{(i)} - y_{(i)} &\leq 0 \\
%         &\Leftrightarrow&-\ssum_{i=k}^{n} x_{(i)} &\geq -\ssum_{i=k}^{n} y_{(i)} \\
%         &\Leftrightarrow&\ssum_{i=1}^{n} x_{i} - \ssum_{i=k}^{n} x_{(i)} &\geq \ssum_{i=1}^{n} y_{i} - \ssum_{i=k}^{n} y_{(i)} \\
%         &\Leftrightarrow&\ssum_{i=1}^{k-1} x_{(i)} &\geq \ssum_{i=1}^{k-1} y_{(i)}. 
%     \end{align*}
%     The last inequality and $\sum_{i=1}^{n} x_i - y_i = 0$ defines $x \slt y$ \cite[Def.~1.A.1]{Marshall2011}. 
%     Since all steps held with iff we have shown the required result.
% \end{proof}

% \begin{proposition} \label{lem:simple-distortion-characterization}
%     Consider a vector $x \in \Re^n$ with $d$ distinct elements, which we store in $y \in \Re^d$. Let $c \in \N^d$ denote 
%     the number of copies of each component of $y$ in $x$. Then 
%     \begin{align*}
%         \bigcap_{\pi \in \Pi^n} \pi(x_{\uparrow} + \set{G}_n) &= \hull\left( \left\{\pi(x) \colon \pi \in \Pi^n\right\} \right) \\[-1em]   
%         &= \{ Sy \colon S\one_d = \one_n, \trans{\one}_n S = \trans{c}, S \geq 0\}.
%     \end{align*}
% \end{proposition}
% \begin{proof}
%     Let $\set{L}$ and $\set{M}$ denote the left and middle set respectively. We first show $\set{M} \subseteq \set{L}$. 
%     This follows from \cref{lem:permutation-hull-contained}, since $\set{M} \subset x_{\uparrow} + \set{G}_n$ and $\pi(\set{M}) = \set{M}$
%     for any $\pi \in \Pi^n$. For the reverse implication $\set{L} \subseteq \set{M}$ take some $v \in \set{L}$. 
%     Then by \cite[Cor.~2.B.3]{Marshall2011} we have $v \slt x$, which by \cref{lem:majorization} implies $v_{\uparrow} \in x_{\uparrow} + (\M^n)^\circ$. 
%     So by \cref{lem:permutation-hull-contained} we have $\pi(v) \in x_{\uparrow} + (\M^n)^\circ$ for all $\pi \in \Pi^n$. Applying $\pi^{-1} \in \Pi^n$ to both sides 
%     then implies $v \in \set{L}$. 
    
%     Since the doubly stochastic matrices are the convex hull of the permutation matrices \cite[2.A.2]{Marshall2011}, the convex hull of all permutations of $x$ iss
%     \begin{equation*}
%         \set{S} \dfn \left\{ H x \colon H \one_n = \one_n, \trans{\one}_n H = \one_n, H \geq 0 \right\}.
%     \end{equation*}
%     Consider
%     \begin{equation*}
%         \trans{C} \dfn \begin{bmatrix}
%             1 & \dots & 1 \\ &&& \ddots \\ &&&& 1 & \dots & 1
%         \end{bmatrix} \in \Re^{d \times n},
%     \end{equation*}
%     where row $i$ contains $c_i$ repetitions of $1$ for each $i \in [d]$ such that $x = C y$. The pseudo-inverse $C^{\dagger}$ is given as 
%     \begin{equation*}
%         C^{\dagger} \dfn \begin{bmatrix}
%             1/c_1 & \dots & 1/c_1 \\ &&& \ddots \\ &&&& 1/c_d & \dots & 1/c_d
%         \end{bmatrix} \in \Re^{d \times n}.
%     \end{equation*}
%     So $y = C^{\dagger} x$ and $C^{\dagger} C = I_d$. 

%     Take $H$ doubly stochastic. Then $v = H C C^{\dagger} x = HC y \in \set{S}$ and $S = HC$ satisfies the conditions above. 
%     Specifically  $H \trans{C} \one_d = H \one_n = \one_n$ and $\trans{\one_n} H \trans{C} = \trans{\one_n} \trans{C} = \trans{\one_d}$ and $H \trans{C} \geq 0$.
%     Let $\set{R}$ denote the right-most set in the statement of the lemma. We have shown $v \in \set{R}$ or $\set{R} \subseteq \set{S}$. 
%     For the reverse implication take some $v \in \set{R}$. So $v = S y = S C^{\dagger} C y = S C^{\dagger} x$. It is easy to verify that $S C^{\dagger}$ is doubly stochastic.
%     Thus $v \in \set{S}$. 
% \end{proof}

% \begin{proposition} \label{lem:inverse-cone-shift}
%     Consider some permutation invariant set $\set{A} \subseteq \Re^n$. Then 
%     \begin{equation*}
%         \set{A} = \bigcap_{\pi \in \Pi^n} \pi \left( \set{A} + (\M^n)^\circ \right). 
%     \end{equation*}
% \end{proposition}
% \begin{proof}
%     We first show $\cap_{\pi} \pi \left( \set{A} + (\M^n)^\circ \right) \subseteq \set{A}$. Assume $x \in \set{A}$. We have 
%     $\set{A} + (\M^n)^\circ = \{y + s \colon y \in \set{A}, s \in (\M^n)^\circ\} \supset \{y \colon y \in \set{A}\} = \set{A}$,
%     where the $\supset$ follows from $0 \in (\M^n)^\circ$. Hence $x \in \set{A} +(\M^n)^\circ$. We can make the same argument 
%     for any permutation $\pi(\set{A} + (\M^n)^\circ) \supset \pi(\set{A})$, since $\pi(\set{A}) = \set{A}$. Hence $x \in \cap_{\pi} \pi \left( \set{A} + (\M^n)^\circ \right)$. 

%     To show $\set{A} \subseteq \cap_{\pi} \pi \left( \set{A} + (\M^n)^\circ \right)$ we proceed by contradiction. Assume 
%     that there is some $x \in \set{A}$ such that $x \notin \cap_{\pi} \pi \left( \set{A} + (\M^n)^\circ \right)$. 
%     That is, there exists some permutation $\pi$ such that $x \notin \pi \left( \set{A} + (\M^n)^\circ \right)$. 
%     This holds iff $\pi^{-1}(x) \notin \set{A} + (\M^n)^\circ$. 

%     Since $\pi^{-1}$ is also a permutation there is thus some $\pi \in \Pi^n$ such that $\pi(x) \notin \set{A} + (\M^n)^\circ$. 
%     Note however that $x_{\uparrow}$ is also in $\set{A}$ by permutation invariance. Moreover by \cref{lem:permutation-hull-contained} $s = x - x_{\uparrow} \in (\M^n)^\circ$
%     for all $x \in \Re^n$. Hence $s = \pi(x) - x_{\uparrow} \in (\M^n)^\circ$ for all $\pi \in \Pi^n$. Thus there is some $s \in (\M^n)^\circ$ and some 
%     $y = x_{\uparrow} \in \set{A}$ such that $\pi(x) = x_{\uparrow} + s$. Therefore $\pi(x) \in \set{A} + (\M^n)^\circ$, which contradicts 
%     our initial assumption.
% \end{proof}

% \todo*{Can we show both previous statements simultaneously?}
% \begin{theorem} %\label{lem:inverse-cone-shift}
%     Consider some convex $\set{A} \subseteq \Re^n$. Then 
%     \begin{equation*}
%         \hull\left( \bigcup_{\pi \in \Pi^n} \pi(\set{A} \cap \M^n) \right) = \bigcap_{\pi \in \Pi^n} \pi \left( \set{A} + (\M^n)^\circ \right). 
%     \end{equation*}
% \end{theorem}
% \begin{proof}
%     Let $\set{L}$ and $\set{R}$ denote the left- and right-hand side set respectively. We first show $\set{L} \subseteq \set{R}$. 
%     Assume $x \in \set{L}$. That is, there are some $v^{(i)}_{\uparrow} \in \set{A}$ and $\pi^{(i)} \in \Pi^n$ for $i \in [m]$ such that 
%     $x \in \hull(\{\pi^{(i)}(v^{(i)}_{\uparrow})\}_{i=1}^m)$. By \cref{lem:permutation-hull-contained} we also have $\pi^{(i)}(v^{(i)}_{\uparrow}) \in \set{R}$
%     for all $i \in [m]$. Also note that, by convexity of $\set{A}$, $\set{R}$ is also convex. Hence any convex combination of $\{\pi^{(i)}(v^{(i)}_{\uparrow})\}_{i=1}^m$
%     should be in $\set{R}$, implying that $x$ is. 

%     Next we show that $\set{R} \subseteq \set{L}$. We proceed by contradiction, assuming $x \in \set{R}$ and $x \notin \set{L}$. 
%     Consider the set of vectors such that $x \in \hull(\{\pi(v) \colon \pi \in \Pi^n\})$,
%     which corresponds with $\{v \colon x \slt v\}$ by \cite[Cor.~2.B.3]{Marshall2011}. Take some $v$ in this set. 
%     Assume $v_{\uparrow} \in \set{A}$ then clearly $\hull(\{\pi(v) \colon \pi \in \Pi^n\}) \subseteq \set{L}$
%     so $x \in \set{L}$, which contradicts our assumption. So we have shown that for any $v \sgt x$ we have
%     $v_{\uparrow} \notin \set{A}$. \todo*{Stuck \dots}


% \end{proof}

% For the next part we will consider minimization of convex separable functions over $(\M^n)^\circ$. Specifically
% \begin{equation}
%     \min_{s \in (\M^n)^\circ} \quad \varphi(x + s) \dfn \sum_{i=1}^{n} \phi_i(x_i + s_i).
% \end{equation}
% We begin by formulating the problem in terms of the convex conjugate and the ordered conjugate from \cref{eq:ordered-conjugate}.

% \begin{lemma} \label{lem:infimum-dual-monotone}
%     For any convex function $\varphi \colon \Re^n \to \eRe$
%     \begin{equation} \label{eq:minimization-at-hand}
%         \inf_{s \in (\M^n)^\circ} \, \varphi(x + s) = (\varphi^*)^\diamond(x).
%     \end{equation}
% \end{lemma}
% \begin{proof}
%     Note that $s \in (\M^n)^\circ$ holds iff $\<\lambda, s\> \geq 0$ for all $\lambda \in \Re^n_{\uparrow}$ by definition of the 
%     dual cone. Alternatively $\sup_{\lambda \in \Re^n_{\uparrow}} -\<\lambda, s\>$ should be finite. This allows us to 
%     write a saddle problem associated with \cref{eq:minimization-at-hand}
%     \begin{equation*}
%         \inf_{s \in \Re^n} \, \sup_{\lambda \in \Re^n_{\uparrow}} \, \varphi(x+s) - \<s, \lambda\>. 
%     \end{equation*}
%     This, using strong duality\todo{cite!} and a change of variables with $y = x+s$, allows us to write the dual of \cref{eq:minimization-at-hand} 
%     \begin{align*}
%         &\sup_{\lambda \in \Re^n_{\uparrow}} \, \left( \inf_{s \in \ne_nRe^n} \, \varphi(y) - \<y, \lambda\> \right) + \<y, \lambda\>\\
%         &\sup_{\lambda \in \Re^n_{\uparrow}} \, - \left( \sup_{s \in \Re^n} \, \<y, \lambda\> - \varphi(y) \right) + \<x, \lambda\>.
%     \end{align*}
%     Plugging in the definition of the convex conjugate for the inner suprema and the ordered conjugate \cref{eq:ordered-conjugate} gives 
%     $(\varphi^*)^\diamond(x)$, completing the proof.
% \end{proof}
\begin{table}[h]
\small
\centering
\caption{
Statistics of \NAME v1.0.
}
% \begin{tabular}{P{1.7cm}P{1.2cm}P{1.2cm}P{1.0cm}P{1.2cm}P{1.2cm}P{1.2cm}P{1.1cm}P{1.1cm}P{1.1cm}}
\begin{tabular}{lr}
\toprule[1.2pt]
Statistics \\
\midrule
\# of prompts & 4,081 \\
$\ $ - \# of COCO captions & 2,000 \\
$\ $ - \# of DrawBench, PartiPrompt, PaintSkill prompts & 2,081 \\
\midrule
\# of questions & 25,829 \\
$\ $ - \# of binary questions & 17,226 \\
$\ $ - \# of multiple-choice questions & 8,603 \\
\midrule
avg. \# of questions per prompt & 6.3 \\
avg. \# of words per prompt & 10.5 \\
avg. \# of elements per prompt & 4.3 \\
\bottomrule[1.2pt]
\end{tabular}
\label{tab:statistics}
% \bottomrule
\vspace{-3mm}
\end{table}
\section{Distortion Representation} \label{app:distortion-representation}
This section contains the proofs associated with \cref{sec:distortion-representation}.

\paragraph*{Proof of \cref{lem:identically-distributed}}  
We will need to use the vector and function characterization of random variables simultaneously throughout this proof. 
So to avoid the abuse of notation used throughout the rest 
of the paper we use capital letters when $X \colon \Omega^n \to \Re$ is 
implied and $x \in \Re^n$ for the vector representation with
\begin{equation*}
    x_i = X(\omega_i), \quad \forall i \in [n].
\end{equation*}
The random variables $X, Y \colon \Omega^n \to \Re$ are identically distributed when 
$\prob[X \in \set{X}] = \prob[Y \in \set{X}]$
for every measurable set $\set{X} \in \B$, with $\B$ the Borel sigma algebra over the reals. Using the vector representation
$\prob[X \in \set{X}] = \frac{1}{n} \sum_{i=1}^{n} \bm{1}_{\set{X}}(x_i)$.
Thus $X \deq Y$ holds iff  
\begin{equation} \label{eq:equal-distribution}
    \sum_{i=1}^{n} \bm{1}_{\set{X}}(x_i) = \sum_{i=1}^{n} \bm{1}_{\set{X}}(y_i), \quad \forall \set{X} \in \B.
\end{equation}

To show $\Leftarrow$ observe that \eqref{eq:equal-distribution} is permutation invariant. So 
$\sum_{i=1}^{n} \bm{1}_{\set{X}}(y_{\pi(i)}) = \sum_{i=1}^{n} \bm{1}_{\set{X}}(y_i)$ for all $\set{X} \in \F$. 

To show $\Rightarrow$ let $m$ denote the number of distinct elements of $x$, the values of which we define as $\{z_j\}_{j=1}^{m}$.
Then let $\set{C}_j = \{i \colon x_i = z_j\}$ for $j \in [m]$. We have 
\begin{equation*}
    \sum_{i=1}^{n} \bm{1}_{\set{X}}(x_i) = \sum_{j=1}^{m} |\set{C}_j| \bm{1}_{\set{X}}(z_j) = \sum_{i=1}^{n} \bm{1}_{\set{X}}(y_i), \quad \forall \set{X} \in \B
\end{equation*}
since $X \deq Y$ by assumption. For any $j \in [m]$ let $\set{X} = \{z_j\}$ and $\set{D}_j = \{i \colon y_i = z_j\}$. Then the above implies that $|\set{D}_j| = |\set{C}_j|$ 
for all $j \in [m]$. Take $\pi \colon [n] \to [n]$ any of the bijections such that $\pi(\set{D}_j) = \set{C}_j$. Then $y_{\pi(i)} = z_{j} = x_i$ for all $i \in \set{D}_j$ and all $j \in [m]$. 
Thus $x = \pi y$, proving the claimed result. \qed\\


% Consider $\bar{\pi}(i) = \left\{ j \colon y_j \in \{z_i\} \right\}$ and let $|\bar{\pi}(i)| = |\set{C}_i|$ for each $i \in [m]$. 
% Also, by construction $\bar{\pi}(i) \cap \bar{\pi}(j) = \emptyset$
% for each pair $i \neq j$. Hence we can define a permutation (i.e. a bijection) $\pi \colon [n] \to [n]$ by arbitrarily pairing up elements 
% between $\set{C}_i$ and $\bar{\pi}(i)$ such that $\pi^{-1}(i) \in \bar{\pi}^{-1}(i) = \set{C}_i$ for all $i \in [m]$. 
% Then taking $y = \pi x$ implies $y_{\pi^{-1}(j)} = z_i$ for $j \in \set{C}_i$ and all $i \in [m]$. Therefore 
% \begin{equation*}
%     \sum_{j=1}^{n} \bm{1}_{\set{X}}(y_j) = \sum_{j=1}^{n} \bm{1}_{\set{X}}(y_{\pi^{-1}(j)}) = \sum_{i=1}^{m} |\set{C}_i| \bm{1}_{\set{X}}(z_i).
% \end{equation*} 
% We have thus shown that \cref{eq:equal-distribution} holds. \qed

We next state the usual definition of law-invariant, coherent risk measures:
\newcommand{\cohasm}[1]{\hyperref[def:coh:#1]{\sc a\oldstylenums{#1}}}
\begin{definition}[coherent risk] \label{def:coh}
    Consider a $\rho \colon \Re^n \to \eRe$
    that is proper \footnote{
        That is $\rho(X) > -\infty$ for all $X \in \Re^n$ and the domain $\mathrm{dom}(\rho) \dfn \{X \in \Re^n \colon \rho(X) < +\infty\}$ is nonempty.
    } and lsc. Assume (for $X, Y \in \Re^n$):
    \begin{enumeratass}[leftmargin=2em]
        \item \emph{convex}: $\rho(\alpha X + (1-\alpha) Y) \leq \alpha \rho(X) + (1-\alpha) \rho(Y)$, $\forall \alpha \in [0, 1]$;\label{def:coh:1}
        \item \emph{monotone}: if $Y \geq X$, then $\rho(Y) \geq \rho(X)$;\label{def:coh:2}
        \item \emph{translation equivariance}: $\rho(X+a) = \rho(X) + a$, $\forall a \in \Re$;\label{def:coh:3}
        \item \emph{pos. homogeneity}: if $t > 0$, then $\rho(tX) = t\rho(X)$.\label{def:coh:4}
    \end{enumeratass}
    Then $\rho$ is a \emph{coherent risk measure}. Consider also
    \begin{enumeratass}[leftmargin=2em]
        \item[\textsc{A\oldstylenums{5}}.] \emph{law invariance}: if $X \deq Y$, then $\rho(Y) = \rho(X)$. \label{def:coh:5}
    \end{enumeratass}
\end{definition}
The first four assumptions were considered in \cite[\S1]{Ruszczynski2006}, while \cohasm{5} is considered in 
\cite[Def.~4.4]{Bertsimas2009b}. By \cref{lem:identically-distributed} it implies that $\rho(X) = \rho(\pi X)$ 
for all $\pi \in \Pi^n$. 

The convex conjugate $\rho^* \colon \Re^n \to \eRe$ of a risk measure is given as 
\begin{equation} \label{eq:convex-conjugate}
    \rho^*(\mu) = \sup_{X \in \Re^n} \left\{ \<\mu, X\> - \rho(X) \right\}.
\end{equation}

Its domain is affected by the assumptions in \cref{def:coh}. 
\begin{proposition} \label{prop:conjugate-duality}
    Suppose that $\rho \colon \Re^n \to \eRe$ is proper, lsc., and convex. Then, $\rho = \rho^{**}$
    with $\amb \dfn \mathrm{dom}(\rho^*)$ a non-empty and convex set. Moreover
    \begin{enumerate}[itemsep=0pt]
        \item \cohasm{2} holds iff $\mu \geq 0$ for any $\mu \in \amb$; \label{prop:conjugate-duality:2}
        \item \cohasm{3} holds iff $\sum_{i=1}^{n} \mu_i = 1$ for any $\mu \in \amb$; \label{prop:conjugate-duality:3}
        \item \cohasm{4} holds iff $\amb$ is closed and \label{prop:conjugate-duality:4}
        \begin{equation*}
            \rho(X) = \sup_{\mu \in \amb} \, \<\mu, X\>, \quad \forall X \in \Re^n
        \end{equation*}
        \item \cohasm{5} holds iff $\rho^*(\pi \mu) = \rho^*(\mu)$ for all $\pi \in \Pi^n, \mu \in \Re^n$. \label{prop:conjugate-duality:5}
    \end{enumerate}
    We assumed the underlying measure of the random variables is $\one_n/n = (1/n, \dots, 1/n)$ for (iv). 
\end{proposition}
\begin{proof}
    By \cite[Prop.~2.112]{Bonnans2000} $\rho^*$ is proper and convex. 
    Hence its domain must be non-empty and convex. 
    Now \emph{(i)}--\emph{(iii)} follows from \cite[Thm.~2.2]{Ruszczynski2006}. 

    By \cref{lem:identically-distributed}, \cohasm{5} holds iff $\rho(\pi X) = X$ for all $\pi \in \Pi^n$. 
    We first show that \cohasm{5} implies $\rho^*(\pi \mu) = \rho^*(\mu)$ for all $\pi \in \Pi^n$.
    First note that 
    \begin{align*}
        &\sup_{X} \left\{ \<\mu, X\> - \rho(X) \right\} = \sup_{X} \left\{ \<\mu, \pi(X)\> - \rho(\pi(X)) \right\} \\
        &\qquad = \sup_{X} \left\{ \<\pi^{-1}(\mu), X\> - \rho(X) \right\}
    \end{align*}
    Since $\{\pi^{-1} \colon \pi \in \Pi^n\} = \Pi^n$ we have shown $\rho^*(\pi \mu) = \rho^*(\mu)$ for all $\mu \in \Re^n$
    and all $\pi \in \Pi^n$. For the reverse implication we can apply the same reasoning and using that $\rho = \rho^{**}$. 
    Note that this result specializes \cite[Prop.~2]{Shapiro2013}.
\end{proof}

% Note that \cref{def:ambiguity} is a direct consequence of \cref{prop:conjugate-duality}.
% Hence the definition is consistent with the usual definition of law-invariant coherent risk measures in \cref{def:coh}. 

% As discussed in \cref{sec:distortion-representation} we will derive an analogous representation to 
% the ambiguity representation in \cref{def:ambiguity} using the ordered convex conjugate:
% \begin{equation} 
%     \rho^{\diamond}(\mu) \dfn \sup_{X \in \M^n} \left\{ \<\mu, X\> - \rho(X) \right\}. \tag{\ref{eq:ordered-conjugate}}
% \end{equation}
To find the equivalent of \cref{prop:conjugate-duality} for the ordered conjugate $\rho^\diamond$, we
begin by relating $\rho^\diamond$ to $\rho^*$ in terms of an infimal convolution.
Then we relate their domains.
\begin{lemma} \label{lem:infimal-convolution}
    Given some proper, convex, lsc. and law-invariant risk measure $\rho \colon \Re^n \to \eRe$. Then 
    \begin{equation*}
        \rho^{\diamond} = (\rho + \indi_{\M^n})^* = \rho^* \episum \indi_{(\M^n)^\circ}.
    \end{equation*}
\end{lemma}
\begin{proof}
    The first equality holds by definition of the conjugate and the indicator. 
    Plugging in the associated definitions and changing signs makes the second equality equivalent to
    \begin{align*}
        \inf_{x} \, \rho(x) + (\indi_{\M^n}(x) - \<x, y\>) = - \inf_u \rho^*(u) + \indi^*_{\Re^n_{\uparrow}}(y-u).
    \end{align*}
    Letting $f(x) = \rho(x)$ and $g(x) = \indi_{\M^n}(x) - \<x, y\>$. Note that $f^{*} = \rho^*$ and $g^*(u) = \indi^*_{\M^n}(y+u)$. 
    Thus we require
    \begin{align*}
        \inf_x \, f(x) + g(x) = - \inf_u f^{*}(u) + g^{*}(-u),
    \end{align*}
    which by \cite[Fact.~15.25]{Bauschke2011}, since $g$ is polyhedral, requires $\dom g \cap \ri \dom f \neq \emptyset$. 
    
    By properness, $\dom \rho \neq \emptyset$, which by convexity of $\rho$ and \cite[Fact.~6.14(i)]{Bauschke2011} implies $\ri \dom \rho \neq \emptyset$.
    Moreover, by permutation invariance $x \in \dom \rho$ implies $\pi x \in \dom\rho$ for all $\pi \in \Pi^n$ since $\rho(\pi(x)) = \rho(x) < \infty$. 
    Thus $\pi \dom \rho = \dom \rho$. Moreover, by \cite[Prop.~2.44]{Rockafellar1998} and linearity of permutations we have $\ri (\dom \rho) = \ri (\pi \dom \rho) = \pi \ri (\dom \rho)$. 
    Thus $\dom g \cap \ri \dom f = \Re^n_{\uparrow} \cap \ri (\dom\rho) \neq \emptyset$. 

    Finally note that $\indi_{\M^n}^* = \indi_{(\M^n)^\circ}$ by \cite[Ex.~11.4]{Rockafellar1998}. 
\end{proof}

\begin{proposition} \label{prop:fundamental-equivalence-lemma}
    Let $\rho \colon \Re^n \to \eRe$ be lsc., convex and law-invariant risk measure. 
    Then
    \begin{align*}
        \mathrm{dom}\left( \rho^{\diamond} \right) &= \mathrm{dom}\left( \rho^* \right) + (\M^n)^\circ \\
        \mathrm{dom}\left( \rho^* \right) &= \bigcap_{\pi \in \Pi^n} \pi\left( \mathrm{dom}\left( \rho^{\diamond} \right) \right).
    \end{align*}
\end{proposition}
\begin{proof}
    The first result follows from \cref{lem:infimal-convolution} and \cite[Ex.~1.28]{Rockafellar1998}. %
    The second result follows from \cref{cor:cone-shift-invariance}. 
\end{proof}

The equivalence derived through \cref{prop:fundamental-equivalence-lemma} enables us to 
re-derive many of the statements in \cref{prop:conjugate-duality} for $\rho^\diamond$. 

\begin{proposition} \label{prop:ordered-conjugate-duality}
    Suppose that $\rho \colon \Re^n \to \eRe$ is lsc., law-invariant and convex. Then
    $\distort \dfn \mathrm{dom}(\rho^\diamond)$ is a convex set such that $\distort \cap \M^n \neq \emptyset$. 
    Moreover
    \begin{enumerate}[itemsep=0pt]
        \item If $\mu \in \distort$ then $\mu + (\M^n)^\circ \subseteq \distort$;
        \item \cohasm{2} holds iff any $\mu$ s.t. $\pi(\mu) \in \distort$ for all permutations $\pi \in \Pi^n$
        is nonnegative. 
        \item \cohasm{3} holds iff $\sum_{i=1}^{n} \mu_i = 1$ for any $\mu \in \distort$; 
        \item \cohasm{4} holds iff $\distort$ is closed and
        \begin{equation*} 
            \rho(X) = \sup_{\mu \in \distort} \, \<\mu, X_{\uparrow}\>, \quad \forall X \in \Re^n
        \end{equation*}
    \end{enumerate}
\end{proposition}
\begin{proof}
    By \cref{prop:conjugate-duality} we have convexity of $\dom\rho^*$. This implies convexity of $\rho^\diamond$ 
    (Minkowski sum of convex sets). Next assume $x \in \dom \rho^*$ then, by \cref{prop:fundamental-equivalence-lemma}, 
    $\pi(x) \in \distort$ for all $\pi \in \Pi^n$. Thus $x_{\uparrow} \in \distort$. 
    So $\dom \rho^* \neq \emptyset$, which holds by \cref{prop:conjugate-duality}, implies $\distort \cap \M^n \neq \emptyset$.
    % Similarly, assume there is some $\mu_{\uparrow} \in \distort \cap \M^n \neq \emptyset$. 
    % By \cref{prop:permuto-hull} we have $\hull (\Pi^n \mu) \in \mu_{\uparrow} + (\M^n)^\circ$.
    % Thus $\pi(\hull (\Pi^n \mu)) \in \distort$ for all $\pi$ or equivalently $\hull (\Pi^n \mu) \subseteq \cap_{\pi} \pi(\distort)$. 
    % Note that $\dom \rho^* = \cap_{\pi} \pi(\distort)$ by \cref{prop:permuto-hull}. So $\dom\rho^* \neq \emptyset$.
    
    \emph{(i)} holds by \cref{prop:fundamental-equivalence-lemma}. To show \emph{(ii)} we use \cref{prop:fundamental-equivalence-lemma}
    and specifically $\dom \rho^* = \cap_{\pi} \pi(\distort)$. So $x \in \dom \rho^*$ iff $\pi(x) \in \distort$ for all $\pi \in \Pi^n$.
    Note that \cohasm{2} holds iff $x \geq 0$ for all $x \in \dom \rho^*$. So \cohasm{2} holds iff every $x$ such that $\pi(x) \in \distort$ for all $\pi \in \Pi^n$
    is nonnegative, proving \emph{(ii)}.

    Let $\bar{\rho} = \rho + \iota_{\Re^n}$ as in \cref{lem:infimal-convolution}.
    Then $\distort = \mathrm{dom}(\bar{\rho}^*)$. 
    By permutation invariance $\bar{\rho}(X_{\uparrow}) = \rho(X_{\uparrow}) = \rho(X)$ for all $X \in \Re^n$. 
    We can then prove \emph{(iii)} by noting that $(X + a)_{\uparrow} = X_{\uparrow} + a$. 
    Hence \cohasm{3} holds for $\bar{\rho}$
    iff it holds for $\rho(X) = \bar{\rho}(X_{\uparrow})$. Thus we can apply \cref{prop:conjugate-duality:3} 
    to $\bar{\rho}$ in order to link \cohasm{3} with $\mathrm{dom}(\rho^{\diamond}) = \mathrm{dom}(\bar{\rho}^{*})$. 
    For \emph{(iv)} we use a similar argument noting that $(\alpha X)_{\uparrow} = \alpha X_{\uparrow}$ for all $X \in \Re^n$
    and all $\alpha \geq 0$ and apply \cref{prop:conjugate-duality:4}. The only property of $\rho$ 
    that does not transfer to $\bar{\rho}$ is monotonicity. So \emph{(ii)} is more complex. 
\end{proof}

We can now prove the main theorem of \cref{sec:distortion-representation}.
\paragraph*{Proof of \cref{thm:distortion-representation}} 
We need to show that \cref{prop:ordered-conjugate-duality} implies the properties of $\distort$
as stated in the theorem. \cref{prop:ordered-conjugate-duality} directly implies the following properties:
convexity, closedness, \emph{(ii)}, \emph{(iii)} and \emph{(iv)} as well as \cref{eq:distortion-representation}. 
The one remaining property is \emph{(i)}. Note that, for any $\mu \in \dom\rho^*$ we have 
$\hull(\Pi^n \mu) \subseteq \dom \rho^*$ by permutation invariance (so $\mu_{\uparrow} \in \dom\rho^*$).
Then \cref{prop:fundamental-equivalence-lemma} and \cref{prop:permuto-hull} imply $\hull(\Pi^n \mu) \subseteq \dom \rho^\diamond$. 
Take $E \in \Re^{n \times n}$ a matrix of all ones. 
Then $E\mu/n \in \hull(\Pi^n \mu) $. Since $\sum_{i=1}^{n} \mu_i = 1$ by \emph{(iii)} we have 
$E\mu/n = \one_n /n$. So if $\one_n /n \notin \dom \rho^*$ then $\dom \rho^*$ 
is empty by contradiction. Thus $\distort$ contains $\one_n/n$ since otherwise $\dom\rho^* = \emptyset$.
Finally, \cref{eq:distort-ambiguity-relation} rephrases \cref{prop:fundamental-equivalence-lemma}. \qed{}
Given measurements $\{\hat \Omega_i\}$ with uncertainties $\sigma^2_i$, as shown in Sec.\ref{sec:pe} the following likelihood function can be used to perform parameter estimation on the \gls{gwb}:
	\begin{equation}
    \label{eq:likelihood-again}
    p(\{\hat \Omega_f\} | {\bm \Theta})
    	= \mathcal{N} \exp\left[
        	-\frac{1}{2}\sum_f\frac{\left(\hat \Omega_f -  \Omega_{\rm M}(f|{\bm \Theta})\right)^2}{\sigma_f^2}\right].
    \end{equation}
Here, the $\{\hat \Omega_f\}$ are a set of estimators for the \gls{gw} energy density at discrete frequencies $f$, $\Omega_{\rm M}(f|{\bm \Theta})$ is a model for the energy density with parameters ${\bm \Theta}$, and $\mathcal{N}$ is a normalization constant.
We will consider only a single baseline and neglect the sum over detector pairs $IJ$ appearing in Eq.~\eqref{eq:likelihood}; if multiple detector pairs exist, the derivation below can be replicated for each pair.

Eq.~\eqref{eq:likelihood-again} assumes that our estimators $\{\hat \Omega_f\}$ are direct, unbiased measurements of the underlying energy-density spectrum.
In general, however, the imperfect amplitude and phase calibration of \gls{gw} detectors will break this assumption.
We can account for calibration uncertainty by amending our likelihood to introduce a new parameter $\lambda$:
	\begin{equation}
    \label{eq:likelihood-calibration-uncertainty}
    p(\{\hat \Omega_f\} | {\bm \Theta},\lambda)
    	= \mathcal{N} \exp\left[
        	-\frac{1}{2}\sum_f\frac{\left(\hat \Omega_f -  \lambda\Omega_{\rm M}(f|{\bm \Theta})\right)^2}{\sigma_f^2}\right].
    \end{equation}
The parameter $\lambda$ is an unknown multiplicative factor that encapsulates potential calibration inaccuracy.
In the case of perfect amplitude calibration ($\lambda=1$), then $\{\hat \Omega_f\}$ are direct measurements of the underlying (unknown) energy spectrum.
But if our calibration is imperfect ($\lambda\ne1$), then $\{\hat \Omega_f\}$ are instead measurements of some multiple $\lambda \Omega(f)$ of the \gls{gwb} spectrum.
Although we do not know $\lambda$, it is possible to estimate the \textit{uncertainty} on instrumental calibration.
We will therefore model $\lambda$ itself as an unknown variable drawn from a normal distribution centered at 1 (corresponding to perfect calibration) but with a variance $\epsilon^2$:
	\begin{equation}
    p(\lambda) \propto 
    	\exp\left[-\frac{1}{2\epsilon^2}\left(\lambda-1\right)^2\right],
    \end{equation}
where $\epsilon$ is a known amplitude calibration uncertainty. 
Additionally, we impose the constraint that $\lambda$ be positive: we expect errors in the amplitude of strain measurements but not their \textit{sign}.
In this case, the probability distribution for $\lambda$ becomes
	\begin{equation}
    \label{eq:plambda}
    p(\lambda) = \sqrt{\frac{2}{\pi}}
  		\frac{1}{\epsilon\left[1
        	+\mathrm{Erf}(\frac{1}{\sqrt{2\epsilon^2}})\right]}
        \exp\left[-\frac{1}{2\epsilon^2}\left(\lambda-1\right)^2\right],
    \end{equation}
normalized to unity on the interval $\lambda\in(0,\infty)$.
Eq. \eqref{eq:plambda} is our prior on $\lambda$.

We can now use Eq. \eqref{eq:plambda} to marginalize our likelihood (Eq.~\eqref{eq:likelihood-calibration-uncertainty}) over the unknown calibration factor $\lambda$.
The marginalized likelihood is given by
	\begin{equation}
    \begin{aligned}
    p(\{\hat \Omega_f \} | {\bm \Theta} )
        &= \int p(\{\hat \Omega_f \} | {\bm\Theta},\lambda) \,p(\lambda) d\lambda \\
        &= \mathcal{N} \sqrt{\frac{2}{\pi}}
  			\frac{1}{\epsilon\left[1
        		+\mathrm{Erf}(\frac{1}{\sqrt{2\epsilon^2}})\right]}
            \int_0^\infty \exp\left[
            	-\frac{1}{2}\sum_f \frac{\left(\hat \Omega_f - \lambda\Omega_{\rm M}(f|{\bm \Theta})\right)^2}{\sigma^2_f}
                -\frac{1}{2}\frac{\left(\lambda-1\right)^2}{\epsilon^2}
                \right] d\lambda.
    \end{aligned}
    \end{equation}
If we define 
	\begin{equation}
    A({\bm \Theta}) = \frac{1}{\epsilon^2}+\sum_f\frac{\Omega_{\rm M}(f|{\bm \Theta})^2}{\sigma^2_f},
    \end{equation}
    \begin{equation}
    B({\bm \Theta}) = \frac{1}{\epsilon^2}+\sum_f\frac{\hat \Omega_f \Omega_{\rm M}(f|{\bm\Theta})}{\sigma^2_f},
    \end{equation}
and
	\begin{equation}
    C({\bm \Theta}) = \frac{1}{\epsilon^2}+\sum_f\frac{\hat \Omega^2_f}{\sigma^2_f},
    \end{equation}
the marginal likelihood can be more concisely expressed as
	\begin{equation}
    \label{eq:likelihood-calib-2}
    p(\{\hat \Omega_f \} | {\bm \Theta} )
    	= \mathcal{N} \sqrt{\frac{2}{\pi}}
  			\frac{1}{\epsilon\left[1 +\mathrm{Erf}(\frac{1}{\sqrt{2\epsilon^2}})\right]}
            \int_0^\infty \exp\left[-\frac{1}{2}\left(
            	A({\bm \Theta})\lambda^2 - 2B({\bm \Theta})\lambda + C({\bm \Theta})
                \right)\right] d\lambda;
    \end{equation}
this expression can be analytically integrated to obtain
	\begin{equation}
    p(\{\hat \Omega_f \} | {\bm \Theta} ) =
    	\mathcal{N} \frac{1}{\epsilon\sqrt{A({\bm \Theta})}}
        \left[\frac{1+\mathrm{Erf}(\frac{B({\bm \Theta})}{\sqrt{2A({\bm \Theta})}})}
        	{1+\mathrm{Erf}(\frac{1}{\sqrt{2\epsilon^2}})}\right]
        \exp\left[-\frac{1}{2}\left(C({\bm \Theta})-\frac{B({\bm \Theta})^2}{A({\bm \Theta})}\right)\right].
    \end{equation}

Marginalization of calibration uncertainty is built into the {\tt pygwb\_pe} module, and this calculation is automatically triggered when passing a calibration error $\epsilon\neq 0$. Additional information on the treatment of calibration uncertainties can be found in \cite{Whelan:2012ur}.
\section{Tractable Reformulations} \label{app:conic-risk}
We review tractable reformulations of the proxy costs presented in this work. All of these are often referred to as \emph{risk measures} in literature \cite{Schuurmans2023}.
Specifically the ones we consider are \emph{conic-representable}, the consequences of which are discussed in this section.
We also highlight dedicated algorithms from literature, whenever they exist.
\subsection{Conic representable ambiguity}
We begin by considering the general class of conic representable risk measures \cite{Schuurmans2023}.
\begin{theorem} \label{thm:conic-ambiguity}
    For matrices $E \in \Re^{d \times n}$ and $F \in \Re^{d \times r}$ and some cone $\set{K}$. Let
    \begin{equation} \label{eq:conic-ambiguity-set}
        \amb = \left\{ \mu \in \Delta^n \colon \exists \nu \in \Re^r \colon E \mu + F \nu \leqc{\set{K}} b \right\},
    \end{equation}
    denote the ambiguity set with $\rho(X) = \sup_{\mu \in \amb} \<\mu, X\>$.
    If $\set{K}$ is polyhedral (i.e. it is represented by an intersection of halfplanes) or $\ri \amb \neq \emptyset$, then
    \begin{equation} \label{eq:dual-risk-representation}
        \begin{alignedat}{2}
            \rho(X) =& \min_{(\lambda, \tau)  \in \set{K}^* \times \Re} &\quad & \tau + \<b, \lambda\>\\
            &\sttshort&& \trans{E} \lambda + \tau \geq X, \, \trans{F} \lambda \geq 0.
        \end{alignedat}
    \end{equation}
\end{theorem}
\begin{proof}
    First note that \cref{eq:dual-risk-representation} is the Lagrangian dual associated with $\rho(X) = \sup_{\mu \in \amb} \<\mu, X\>$.
    Specifically, we can rewrite the supremum in standard dual form as
    \begin{equation*}
        \rho(X) = \left\{ \max_{y} \<(X, 0), y\> \colon  \begin{bmatrix} E & F \\ -\trans{\one_n} & 0 \\ \trans{\one_n} & 0 \end{bmatrix} y \leqc{\set{G}^*} \begin{bmatrix} b \\ 0 \\ 1 \end{bmatrix} \right\},
    \end{equation*}
    with $\set{G}^* = \set{K} \times \Re^n_+ \times \{0\}$ and $y = (\mu, \nu)$. Its dual is given as $\set{G} = \set{K}^* \times \Re^n_+ \times \Re$. 
    The primal is \cite[Ex.~5.12]{Boyd2004}
    \begin{equation}
        \begin{alignedat}{2}
            \rho(X) =& \min_{x \in \set{G}} &\quad & \<(b, 0, 1), x\>\\
            &\sttshort&& \begin{bmatrix} \trans{E} & - \one_n & \one_n \\ \trans{F} & 0 & 0 \end{bmatrix} = \begin{bmatrix} b \\ 0 \end{bmatrix}.
        \end{alignedat}
    \end{equation}
    If we partition $x = (\lambda, \beta, \tau)$ and eliminate $\beta$ from the above problem then \cref{eq:dual-risk-representation} is recovered. 
    The conditions on $\set{K}$ and $\ri \amb$ follow from the discussion at the start of \cite[\S{}5.2.3]{Boyd2004}.
\end{proof}

\begin{remark}
    Consider \cref{eq:robustified-erm} with $L(\theta) \colon \Re^{n_\theta} \to \Re^n$ convex.
    Then the dual problem for $\rho[L(\theta)]$ in \cref{eq:dual-risk-representation} has the constraint 
    \begin{equation*}
        \trans{E} \lambda + \tau \geq L(\theta),
    \end{equation*}
    which is clearly convex. The constraint can be made conic through application of an epigraph trick on $L(\theta)$. 
\end{remark}

Besides the classical positive orthant $\Re^n_+$, which is self dual, we will need two more cones below.
The exponential cone in $\Re^3$ and its dual are:
\begin{align*}
    \set{E}^3 &\dfn \{x \colon x_1 \leq x_2 \log(\tfrac{x_3}{x_2}), x_2 > 0\}, \, \\
    (\set{E}^3)^* &= \{y \colon y_1 < 0, y_3 > 0, -y_1 \log(\tfrac{-y_1}{y_3}) + y_1 - y_2 \leq 0\}.
\end{align*}
The second-order cone is self dual and is given as:
\begin{equation*}
    \set{Q}^n \dfn \{x = (t, z) \in \Re^{n+1} \colon  t \geq \nrm{y}_2\}.
\end{equation*}

\subsection{Distortion risk}
Optimizing distortion risks has been considered in literature before and dedicated algorithms were developed \cite{Mehta2022}. 
For purpose of generality we provide a reformulation in terms of \cref{thm:conic-ambiguity}.
\begin{proposition} \label{prop:distortion-ambiguity}
    Let $\rho$ be a distortion risk with $\distort \dfn \dom \rho^{\diamond} = \mu_{\uparrow} + (\M^n)^\circ$ 
    for some $\mu \in \Delta^n$. 
    Then the ambiguity set $\amb \dfn \dom \rho^*$ is given as:
    \begin{align*}
        \amb = \left\{ S \bar{\mu} \colon S \one_d = \one_n, \trans{\one}_n S = c, S \geq 0 \right\},
    \end{align*}
    with $\bar{\mu} \in \Re^d$ containing the $d \leq n$ unique elements of $\mu$ and $c \in \Re^d$ the number of copies of each element of $\bar{\mu}$ in $\mu$. 
    Moreover    
    \begin{equation} \label{eq:dual-distortion-representation}
        \begin{alignedat}{2}
            \rho(X) =& \min_{y, w} &\quad & \trans{\one_n}y + \trans{c} w \\
            &\sttshort&& y_i + w_j \geq X_i \bar{\mu}_j, \, (i, j) \in [n] \times [d].
        \end{alignedat}
    \end{equation}
\end{proposition}
\begin{proof}
    This follows from \cref{prop:fundamental-equivalence-lemma} and \cref{lem:simple-distortion-characterization}.
    For the final result, first note that $\amb$ can be written as in \cref{thm:conic-ambiguity} with constraints $I_n \mu - (\trans{\bar{\mu}} \kron I_n) \nu = 0$,
    $(\trans{\one_d} \kron I_n) \nu = \one_n$, $(I_d \kron \trans{\one_n})  = \one_d$ and $\nu = \vec(S) \geq 0$, with $\kron$ the Kronecker product and $\vec$ the vectorization operation. 
    Dualizing and simplifying gives the required result. Note that the associated cones are all polyhedral. Hence \cref{thm:conic-ambiguity} is applicable. 
\end{proof}

A similar characterization was provided in \cite[Thm.~4.3]{Bertsimas2009b}, which did not exploit sparsity of $\mu$ as we do. 
Note how for $\bar{\CVAR}$ we have $d = 4$ for any value of $n$. This fixed sparsity means that the number of constraints and variables in \cref{eq:dual-distortion-representation}
grow linearly with $n$. Without exploiting sparsity, the growth would be quadratic.

\subsection{$\phi$-divergences}
Optimization of $\phi$-divergence based risk measures has been considered in \cite{Ben-Tal2013} and \cite{Chouzenoux2019}.
Specifically \cite[Prop.~9]{Chouzenoux2019}:
\begin{equation*}
    \rho(X) = \min_{(\lambda, \mu) \in \Re_+ \times \Re} \, \lambda \alpha + \mu + \frac{1}{n} \sum_{i=1}^{n} \lambda \phi^* \left( \frac{X_i - \mu}{\lambda} \right).
\end{equation*}
This formulation is especially useful as it supports stochastic gradient descent algorithms \cite{Chouzenoux2019}. 

To support reformulating $\rho(X)$ as a conic program in the convex case, 
we also characterize the ambiguity sets associated with the $\phi$-divergences listed in \cref{tab:phi-table} in terms of conic constraints.
The associated risk measure can be minimized through application of \cref{thm:conic-ambiguity}. For all, except the Burg Entropy and $\chi^2$-distance 
we follow \cite[App.~A]{Schuurmans2023}.

\paragraph*{KL-divergence}
\begin{align*}
    &I_{\phi}(\mu, \one_n/n) \leq \alpha \, \Leftrightarrow \, \tfrac{1}{n} \ssum_{i=1}^{n} \log(n \mu_i) \leq \alpha \\
    &\Leftrightarrow \quad  \exists \nu \in \Re^n \colon \begin{cases}
        \log(\mu_i n) / n \leq \nu_i, \quad \forall i \in [n] \\
        \sum_{i=1}^{n} \nu_i \leq \alpha,
    \end{cases} \\
    &\Leftrightarrow \quad  \exists \nu \in \Re^n \colon \begin{cases}
        (-\nu_i, 1/n, \mu_i) \in \set{E}^3, \quad \forall i \in [n] \\
        \sum_{i=1}^{n} \nu_i \leq \alpha.
    \end{cases}
\end{align*}

\paragraph*{Burg Entropy}
\begin{align*}
    &I_{\phi}(\mu, \one_n/n) \leq \alpha \, \Leftrightarrow \, \tfrac{1}{n} \ssum_{i=1}^{n} (n \mu_i - \log(n \mu_i) - 1) \leq \alpha \\
    &\Leftrightarrow \quad - \tfrac{1}{n} \log(n \mu_i) \leq \alpha \\
    &\Leftrightarrow \quad  \exists \nu \in \Re^n \colon \begin{cases}
        - \log(\mu_i n) / n \leq \nu_i, \quad \forall i \in [n] \\
        \sum_{i=1}^{n} \nu_i \leq \alpha,
    \end{cases} \\
    &\Leftrightarrow \quad  \exists \nu \in \Re^n \colon \begin{cases}
        (-\nu_i, 1/n, \mu_i) \in \set{E}^3, \quad \forall i \in [n] \\
        \sum_{i=1}^{n} \nu_i \leq \alpha.
    \end{cases}
\end{align*}

\paragraph*{Hellinger Distance}
Using the fact that 
\begin{equation*}
    I_{\phi}(\mu, \one_n/n) = \ssum_{i=1}^{n} \left(\sqrt{\mu_i} - \sqrt{\tfrac{1}{n}}\right)^2 = 2 \left(1 - \ssum_{i=1}^{n} \sqrt{\tfrac{\mu_i}{n}}\right),
\end{equation*}
we have
\begin{align*}
    &I_{\phi}(\mu, \one_n/n) \leq \alpha \, \Leftrightarrow \, \ssum_{i=1}^{n} \sqrt{\mu_i/n} \geq 1 - \alpha/2 \\
    &\Leftrightarrow \quad  \exists \nu \in \Re^n \colon \begin{cases}
        1 - \alpha/2 \leq \ssum_{i=1}^{n} \sqrt{1/n} \nu_i, \quad \forall i \in [n] \\
        \nu_i^2 \leq \mu_i.
    \end{cases}
\end{align*}
The second constraint can be reformulated as 
\begin{align*}
    \nu_i^2 \leq \mu_i &\Leftrightarrow 4\nu_i^2 \leq (\mu_i + 1 + \mu_i - 1) (\mu_i + 1 - (\mu_i - 1)) \\
        &\Leftrightarrow 4\nu_i^2 \leq  (\mu_i + 1)^2 - (\mu_i - 1)^2 \\
        &\Leftrightarrow \nrm{(2 \nu_i, \mu_i - 1)}_2 \leq \mu_i + 1 \\
        &\Leftrightarrow (\mu_i + 1, 2\nu_i, \mu_i - 1) \in \set{Q}^2.
\end{align*}

\paragraph*{$\chi^2$ Distance}
Using the fact that 
\begin{equation*}
    I_{\phi}(\mu, \one_n/n) = \tfrac{1}{n} \ssum_{i=1}^{n} \tfrac{(n\mu_i - 1)^2}{n \mu_i} = \sum_{i=1}^{n} (\mu_i - \tfrac{1}{n})^2/\mu_i,
\end{equation*}
we can rewrite 
\begin{align*}
    &I_{\phi}(\mu, \one_n/n) \leq \alpha \,\Leftrightarrow \, \ssum_{i=1}^{n} (\mu_i^2 - 2 \tfrac{\mu_i}{n} + \tfrac{1}{n^2})/\mu_i \leq \alpha \\
    &\Leftrightarrow \quad \ssum_{i=1}^{n} \mu_i - \tfrac{2}{n} + \tfrac{1}{\mu_i n^2} \leq \alpha \\
    &\Leftrightarrow \quad \ssum_{i=1}^{n} \tfrac{1}{\mu_i n^2} \leq \alpha + 1 \\
    &\Leftrightarrow \quad  \exists \nu \in \Re^n \colon \begin{cases}
        \ssum_{i=1}^{n} \nu_i/n^2 \leq \alpha+1, \quad \forall i \in [n] \\
        1/\mu_i \leq \nu_i.
    \end{cases}
\end{align*}
The second constraint can be reformulated as 
\begin{align*}
    1/\mu_i \leq \nu_i &\Leftrightarrow 4 \leq (\mu_i + \nu_i + \mu_i - \nu_i) (\mu_i + \nu_i - (\mu_i - \nu_i)) \\
    &\Leftrightarrow 4 \leq (\mu_i + \nu_i)^2 - (\mu_i - \nu_i)^2 \\
    &\Leftrightarrow \nrm{(2, \mu_i - \nu_i)} \leq \mu_i + \nu_i \\
    &\Leftrightarrow (\mu_i + \nu_i, 2, \mu_i - \nu_i) \in \set{Q}^2.
\end{align*}

\paragraph*{Total variation}
\begin{align*}
    &I_{\phi}(\mu, \one_n/n) \leq \alpha \, \Leftrightarrow \, \nrm{\mu - \one_n/n}_1 \leq \alpha \\ 
    & \quad \Leftrightarrow \, \exists \nu \in \Re^n \colon \begin{cases}
        - \nu_i \leq \mu_i - 1/n \leq \nu_i, \, \forall i \in [n] \\
        \ssum_{i=1}^{n} \nu_i \leq \alpha.
    \end{cases}
\end{align*}
%% Creator: Inkscape inkscape 0.92.3, www.inkscape.org
%% PDF/EPS/PS + LaTeX output extension by Johan Engelen, 2010
%% Accompanies image file 'assets/_svm.pdf' (pdf, eps, ps)
%%
%% To include the image in your LaTeX document, write
%%   \input{<filename>.pdf_tex}
%%  instead of
%%   \includegraphics{<filename>.pdf}
%% To scale the image, write
%%   \def\svgwidth{<desired width>}
%%   \input{<filename>.pdf_tex}
%%  instead of
%%   \includegraphics[width=<desired width>]{<filename>.pdf}
%%
%% Images with a different path to the parent latex file can
%% be accessed with the `import' package (which may need to be
%% installed) using
%%   \usepackage{import}
%% in the preamble, and then including the image with
%%   \import{<path to file>}{<filename>.pdf_tex}
%% Alternatively, one can specify
%%   \graphicspath{{<path to file>/}}
%% 
%% For more information, please see info/svg-inkscape on CTAN:
%%   http://tug.ctan.org/tex-archive/info/svg-inkscape
%%
\begingroup%
  \makeatletter%
  \providecommand\color[2][]{%
    \errmessage{(Inkscape) Color is used for the text in Inkscape, but the package 'color.sty' is not loaded}%
    \renewcommand\color[2][]{}%
  }%
  \providecommand\transparent[1]{%
    \errmessage{(Inkscape) Transparency is used (non-zero) for the text in Inkscape, but the package 'transparent.sty' is not loaded}%
    \renewcommand\transparent[1]{}%
  }%
  \providecommand\rotatebox[2]{#2}%
  \newcommand*\fsize{\dimexpr\f@size pt\relax}%
  \newcommand*\lineheight[1]{\fontsize{\fsize}{#1\fsize}\selectfont}%
  \ifx\svgwidth\undefined%
    \setlength{\unitlength}{576bp}%
    \ifx\svgscale\undefined%
      \relax%
    \else%
      \setlength{\unitlength}{\unitlength * \real{\svgscale}}%
    \fi%
  \else%
    \setlength{\unitlength}{\svgwidth}%
  \fi%
  \global\let\svgwidth\undefined%
  \global\let\svgscale\undefined%
  \ifx\svgfont\undefined%
  \global\let\svgfont\footnotesize
  \fi%    
  \makeatother%
  \begin{picture}(1,0.3125)%
    \lineheight{1}%
    \setlength\tabcolsep{0pt}%
    \put(0,0){\includegraphics[width=\unitlength,page=1]{assets/_svm.pdf}}%
    \put(0.08892572,0.015){\makebox(0,0)[t]{\lineheight{1.25}\smash{\begin{tabular}[t]{c}\svgfont{}-2\end{tabular}}}}%
    \put(0,0){\includegraphics[width=\unitlength,page=2]{assets/_svm.pdf}}%
    \put(0.15403411,0.015){\makebox(0,0)[t]{\lineheight{1.25}\smash{\begin{tabular}[t]{c}\svgfont{}0\end{tabular}}}}%
    \put(0,0){\includegraphics[width=\unitlength,page=3]{assets/_svm.pdf}}%
    \put(0.21914249,0.015){\makebox(0,0)[t]{\lineheight{1.25}\smash{\begin{tabular}[t]{c}\svgfont{}2\end{tabular}}}}%
    \put(0,0){\includegraphics[width=\unitlength,page=4]{assets/_svm.pdf}}%
    \put(0.04421875,0.07804534){\makebox(0,0)[rt]{\lineheight{1.25}\smash{\begin{tabular}[t]{r}\svgfont{}-2\end{tabular}}}}%
    \put(0,0){\includegraphics[width=\unitlength,page=5]{assets/_svm.pdf}}%
    \put(0.04421875,0.15052219){\makebox(0,0)[rt]{\lineheight{1.25}\smash{\begin{tabular}[t]{r}\svgfont{}0\end{tabular}}}}%
    \put(0,0){\includegraphics[width=\unitlength,page=6]{assets/_svm.pdf}}%
    \put(0.04421875,0.22299904){\makebox(0,0)[rt]{\lineheight{1.25}\smash{\begin{tabular}[t]{r}\svgfont{}2\end{tabular}}}}%
    \put(0,0){\includegraphics[width=\unitlength,page=7]{assets/_svm.pdf}}%
    \put(0.15403411,0.27625){\makebox(0,0)[t]{\lineheight{1.25}\smash{\begin{tabular}[t]{c}{}SAA\end{tabular}}}}%
    \put(0,0){\includegraphics[width=\unitlength,page=8]{assets/_svm.pdf}}%
    \put(0.41600906,0.015){\makebox(0,0)[t]{\lineheight{1.25}\smash{\begin{tabular}[t]{c}\svgfont{}-2\end{tabular}}}}%
    \put(0,0){\includegraphics[width=\unitlength,page=9]{assets/_svm.pdf}}%
    \put(0.48111746,0.015){\makebox(0,0)[t]{\lineheight{1.25}\smash{\begin{tabular}[t]{c}\svgfont{}0\end{tabular}}}}%
    \put(0,0){\includegraphics[width=\unitlength,page=10]{assets/_svm.pdf}}%
    \put(0.54622581,0.015){\makebox(0,0)[t]{\lineheight{1.25}\smash{\begin{tabular}[t]{c}\svgfont{}2\end{tabular}}}}%
    \put(0,0){\includegraphics[width=\unitlength,page=11]{assets/_svm.pdf}}%
    \put(0.37130207,0.07804534){\makebox(0,0)[rt]{\lineheight{1.25}\smash{\begin{tabular}[t]{r}\svgfont{}-2\end{tabular}}}}%
    \put(0,0){\includegraphics[width=\unitlength,page=12]{assets/_svm.pdf}}%
    \put(0.37130207,0.15052219){\makebox(0,0)[rt]{\lineheight{1.25}\smash{\begin{tabular}[t]{r}\svgfont{}0\end{tabular}}}}%
    \put(0,0){\includegraphics[width=\unitlength,page=13]{assets/_svm.pdf}}%
    \put(0.37130207,0.22299904){\makebox(0,0)[rt]{\lineheight{1.25}\smash{\begin{tabular}[t]{r}\svgfont{}2\end{tabular}}}}%
    \put(0,0){\includegraphics[width=\unitlength,page=14]{assets/_svm.pdf}}%
    \put(0.48111746,0.27625){\makebox(0,0)[t]{\lineheight{1.25}\smash{\begin{tabular}[t]{c}{}TV\end{tabular}}}}%
    \put(0,0){\includegraphics[width=\unitlength,page=15]{assets/_svm.pdf}}%
    \put(0.74309238,0.015){\makebox(0,0)[t]{\lineheight{1.25}\smash{\begin{tabular}[t]{c}\svgfont{}-2\end{tabular}}}}%
    \put(0,0){\includegraphics[width=\unitlength,page=16]{assets/_svm.pdf}}%
    \put(0.80820078,0.015){\makebox(0,0)[t]{\lineheight{1.25}\smash{\begin{tabular}[t]{c}\svgfont{}0\end{tabular}}}}%
    \put(0,0){\includegraphics[width=\unitlength,page=17]{assets/_svm.pdf}}%
    \put(0.87330919,0.015){\makebox(0,0)[t]{\lineheight{1.25}\smash{\begin{tabular}[t]{c}\svgfont{}2\end{tabular}}}}%
    \put(0,0){\includegraphics[width=\unitlength,page=18]{assets/_svm.pdf}}%
    \put(0.6983854,0.07804534){\makebox(0,0)[rt]{\lineheight{1.25}\smash{\begin{tabular}[t]{r}\svgfont{}-2\end{tabular}}}}%
    \put(0,0){\includegraphics[width=\unitlength,page=19]{assets/_svm.pdf}}%
    \put(0.6983854,0.15052219){\makebox(0,0)[rt]{\lineheight{1.25}\smash{\begin{tabular}[t]{r}\svgfont{}0\end{tabular}}}}%
    \put(0,0){\includegraphics[width=\unitlength,page=20]{assets/_svm.pdf}}%
    \put(0.6983854,0.22299904){\makebox(0,0)[rt]{\lineheight{1.25}\smash{\begin{tabular}[t]{r}\svgfont{}2\end{tabular}}}}%
    \put(0,0){\includegraphics[width=\unitlength,page=21]{assets/_svm.pdf}}%
    \put(0.80820078,0.27625){\makebox(0,0)[t]{\lineheight{1.25}\smash{\begin{tabular}[t]{c}{}$\bar{\CVAR}$\end{tabular}}}}%
    \put(0,0){\includegraphics[width=\unitlength,page=22]{assets/_svm.pdf}}%
    \put(0.28692881,0.07081707){\makebox(0,0)[lt]{\lineheight{1.25}\smash{\begin{tabular}[t]{l}\svgfont{}-10\end{tabular}}}}%
    \put(0,0){\includegraphics[width=\unitlength,page=23]{assets/_svm.pdf}}%
    \put(0.28692881,0.13394156){\makebox(0,0)[lt]{\lineheight{1.25}\smash{\begin{tabular}[t]{l}\svgfont{}0\end{tabular}}}}%
    \put(0,0){\includegraphics[width=\unitlength,page=24]{assets/_svm.pdf}}%
    \put(0.28692881,0.19706604){\makebox(0,0)[lt]{\lineheight{1.25}\smash{\begin{tabular}[t]{l}\svgfont{}10\end{tabular}}}}%
    \put(0,0){\includegraphics[width=\unitlength,page=25]{assets/_svm.pdf}}%
    \put(0.61401214,0.07997168){\makebox(0,0)[lt]{\lineheight{1.25}\smash{\begin{tabular}[t]{l}\svgfont{}-10\end{tabular}}}}%
    \put(0,0){\includegraphics[width=\unitlength,page=26]{assets/_svm.pdf}}%
    \put(0.61401214,0.1417839){\makebox(0,0)[lt]{\lineheight{1.25}\smash{\begin{tabular}[t]{l}\svgfont{}0\end{tabular}}}}%
    \put(0,0){\includegraphics[width=\unitlength,page=27]{assets/_svm.pdf}}%
    \put(0.61401214,0.20359611){\makebox(0,0)[lt]{\lineheight{1.25}\smash{\begin{tabular}[t]{l}\svgfont{}10\end{tabular}}}}%
    \put(0,0){\includegraphics[width=\unitlength,page=28]{assets/_svm.pdf}}%
    \put(0.94109546,0.08010896){\makebox(0,0)[lt]{\lineheight{1.25}\smash{\begin{tabular}[t]{l}\svgfont{}-10\end{tabular}}}}%
    \put(0,0){\includegraphics[width=\unitlength,page=29]{assets/_svm.pdf}}%
    \put(0.94109546,0.14185908){\makebox(0,0)[lt]{\lineheight{1.25}\smash{\begin{tabular}[t]{l}\svgfont{}0\end{tabular}}}}%
    \put(0,0){\includegraphics[width=\unitlength,page=30]{assets/_svm.pdf}}%
    \put(0.94109546,0.2036092){\makebox(0,0)[lt]{\lineheight{1.25}\smash{\begin{tabular}[t]{l}\svgfont{}10\end{tabular}}}}%
    \put(0,0){\includegraphics[width=\unitlength,page=31]{assets/_svm.pdf}}%
  \end{picture}%
\endgroup%


\end{document}

