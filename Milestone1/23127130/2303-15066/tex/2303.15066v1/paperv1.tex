%\documentclass[useAMS,usenatbib]{mn2e}
\documentclass[a4paper,fleqn,usenatbib]{mnras}
\usepackage{graphicx}
\usepackage{amssymb}
\usepackage{amsmath}
\usepackage{times} 
\usepackage{multirow}
\usepackage{url}
\hypersetup{draft} %****** remove
\usepackage{cuted}
\usepackage[english]{babel}
\usepackage{enumerate}
\usepackage[normalem]{ulem}
\usepackage{enumitem}
\usepackage{multirow}
\usepackage{float}
\usepackage{booktabs}
%\usepackage{bm}
\usepackage{makecell}
\usepackage{placeins}
%\usepackage{ae,aecompl}
%\documentclass[useAMS,usenatbib,twocolumn,preprintnumbers,nofootinbib]{revtex4}
%\usepackage{amsmath}
%\usepackage{amsfonts}
%\usepackage{amssymb}
%preprint,
%\usepackage{graphicx}% Include figure files
%\usepackage{dcolumn}% Align table columns on decimal point
%\usepackage{bm}% bold math



\title[]{Low-redshift estimates of the absolute scale of baryon acoustic oscillations}



\author[Lemos et al.]{Tha\'{\i}s Lemos,$ ^{1} $\thanks{e-mail: thaislemos@on.br}
Ruchika,$^{2}$\thanks{e-mail: ruchika@ctp-jamia.res.in}
Jailson Alcaniz,$^{1}$ \thanks{e-mail: alcaniz@on.br}\\
$^{1}$Departamento de Astronomia, Observat\'orio Nacional, 20921-400 Rio de Janeiro, RJ, Brasil \\
$^{2}$Centre for Theoretical Physics, Jamia Millia Islamia, New Delhi-110025, India
}

% These dates will be filled out by the publisher
\date{Accepted XXX. Received YYY; in original form ZZZ}

% Enter the current year, for the copyright statements etc.
\pubyear{20}

\begin{document}


\maketitle


\begin{abstract}
We investigate... 
\end{abstract}

\begin{keywords}
write here the keywords
\end{keywords}

%%----------------------------------------------------------------
\section{Introduction}\label{sec1}

%The standard $\Lambda$-Cold Dark Matter ($\Lambda$CDM) model successfully describes the current cosmological observations. %\footnote{For a recent review on possible tensions involving estimates of the $\Lambda$CDM model parameters, see \cite{DiValentino:2021izs}.}. 
%Among these observations, measurements of the Cosmic Microwave Background (CMB) and Type Ia Supernovae (SNe Ia) have been used to infer the angular diameter distance $d_{A}(z)$ out to $z \approx 1100$ and luminosity distances $d_{L}(z)$ out to $z \approx 2$, respectively, being also highly complementary tools for measuring the cosmic expansion history and constraining cosmological parameters such as the matter density parameter, $\Omega_m$, and the local expansion rate, $H_{0}$ (see e.g., \cite{Weinberg:2013agg} and references therein). 

The Baryon Acoustic Oscillations (BAO) arise due to the competing effects of radiation pressure and gravity in the early Universe. When photons
and baryons decoupled, the sound waves freeze out, leaving a fundamental scale in large-scale structure in the Universe (Peebles \&
Yu 1970; Sunyaev \& Zel’dovich 1970), which  %. Since such scale remains imprinted in the galaxy distribution, the BAO length scale  
can be used as a robust cosmic ruler at intermediate redshifts \citep{Weinberg:2013agg}. When combined with Cosmic Microwave Background (CMB)
and type Ia Supernova (SNe) measurements, which provide the angular diameter distance $d_A(z)$ out to $z \approx 1100$ and luminosity distances $d_L(z)$ out
to $z \approx 2$, respectively, BAO measurements place the best constraints on cosmological
parameters today, including the matter density parameter, $\Omega_m$, and the
spatial curvature parameter, $\Omega_k$ (see, e.g., di Valentino et al. 2020). %\cite{Aubourg:2014yra} and references therein)

The position of the BAO feature observed in the large-scale distribution of galaxies is determined by the comoving sound horizon size at the drag epoch,
\begin{equation}
r_{s} (z_{\rm{drag}})=r_{d} = \int_{z_{\rm{drag}}}^{\infty}{\frac{c_s(z)}{H(z)}}dz\;,
\end{equation}
where $c_s$ is the sound speed of photon-baryon fluid and $z_{\rm{drag}} \approx 1100$ is the redshift at which baryons were released from photons. Such a  characteristic scale %provides a standard ruler that 
can be measured in the CMB anisotropy spectrum and in the distribution of large-scale structure at lower $z$. However, as well known, the comoving length $r_d$ is calibrated at $z \gtrsim 1000$ using a combination of observations and theory, which makes its estimates vulnerable to systematic errors from possible unknown physics in the early universe \citep{Sutherland}.

The BAO feature can also be separated into transversal and radial modes,
providing independent estimates of angular diameter distances and
the expansion rate $H(z)$ (Seo \& Eisenstein 2003). Measurements of the BAO scale are usually obtained from the application of the spatial 2-point correlation function to a large distribution of galaxies, where the BAO signature appears as a bump at the corresponding
scale. This type of measurements requires a fiducial cosmology  to transform the measured angular positions and redshifts into comoving distances and constrains the quantity $r_s/D_V$, where  
\begin{equation}
D_V(z) =  \left[(1+z)^2 {d}^2_{A}(z) \frac{cz}{H(z)}\right]^{1/3}  
\end{equation}
%Such mesurements are obtained by calculating the spatial 2-point correlation function to a largedistribution of galaxies, which 
is  the dilation scale (Eisenstein et al. 2005).% \citep{Aubourg:2014yra}. 

Another possibility is to apply the angular 2-point correlation function,
which involves only the angular separation $\theta$ between pairs, yielding information of $d_A(z)$ without introducing a fiducial cosmology, provided that the comoving sound horizon $r_s$ is known (Carneiro et al. 2009). These measurements use thin-enough redshift bins to obtain the angular BAO scale given by \citep{Carvalho2016}
\begin{equation}\label{eq2}
    \theta_{\text{BAO}}(z)=\frac{{r}_{s}}{(1+z)\text{d}_{A}(z)}\;.
\end{equation}

%However, it is possible to obtain BAO measurements without introducing a fiducial cosmology from the angular 2-point correlation, which involves only the angular separation $\theta$ between pairs of galaxies (Carneiro et al. 2009). Using thin-enough redshift bins, one measures the angular BAO scale given by \citep{Carvalho2016}
%\begin{equation}\label{eq2}
%    \theta_{\text{BAO}}(z)=\frac{{r}_{s}}{(1+z)\text{d}_{A}(z)}\;.
%\end{equation}
%where $\theta_{\textrm{BAO}}$ is the angular BAO scale. 


There is a slight difference between the comoving sound horizon at the drag epoch $r_d$ and the comoving length scale of the BAO feature in a galaxy survey $r_s$, which results from the non-linear growth of structure and evolution of perturbations (Eisenstein et al. 2007; Padmanabhan et al. 2012). Currently, the best-inferred value of sound horizon $r_d = 147.21 \pm 0.23$ Mpc (1$\sigma$) was determined from the CMB power spectrum by the Planck mission (Aghanim et al., 2018). Such an estimate is obtained in the context of the $\Lambda$CDM model and does not consider possible new or unknown physics at earlier times\footnote{The above estimate of $r_d$ is obtained assuming the standard recombination history with the effective number of neutrino species $N_{eff} = 3.046$ and the usual evolution of matter and radiation energy densities.}. Therefore, measuring $r_s$ at low-$z$ and comparing it with the sound horizon estimates from CMB $r_d$ constitutes an important consistency test of the $\Lambda$CDM model and its assumptions\footnote{The BAO length scale also plays a role in the discussions of the $H_0$-tension problem, as an increase in the pre-recombination expansion rate implies a reduction of the sound horizon at recombination and an increase in $H_0$.}.% (see, e.g. [REF]).}. 

%%------------------------------------------------------  TABLE  ------------------------------------------------------

\begin{table*}
\centering
\caption{BAO measurements from angular separation of pairs of galaxies.}
\begin{tabular}{| c | c | c || c | c | c |}
\hline
$\Bar{z}$ & $\theta_{BAO}(z)[^{\circ}]$ & Reference & $\Bar{z}$ & $\theta_{BAO}(z)[^{\circ}]$ & Reference\\
\hline
0.45 & 4.77 $\pm$ 0.17 & \cite{Carvalho2016} & 0.57 & 4.59 $\pm$ 0.36 & \cite{Carvalho2017} \\
0.47 & 5.02 $\pm$ 0.25 & \cite{Carvalho2016} & 0.59 & 4.39 $\pm$ 0.33 & \cite{Carvalho2017} \\
0.49 & 4.99 $\pm$ 0.21 & \cite{Carvalho2016} & 0.61 & 3.85 $\pm$ 0.31 & \cite{Carvalho2017} \\
0.51 & 4.81 $\pm$ 0.17 & \cite{Carvalho2016} & 0.63 & 3.90 $\pm$ 0.43 & \cite{Carvalho2017} \\
0.53 & 4.29 $\pm$ 0.30 & \cite{Carvalho2016} & 0.65 & 3.55 $\pm$ 0.16 & \cite{Carvalho2017} \\
0.55 & 4.25 $\pm$ 0.25 & \cite{Carvalho2016} & 2.225 & 1.77 $\pm$ 0.31 & \cite{Carvalho2018} \\
\hline
%\multicolumn{6}{p{2.5cm}}{\,} 
\end{tabular}
\label{tab:BAO_data}
\end{table*}


This idea was first discussed by \cite{Sutherland}, who derived an elegant approximation for $D_V$ in terms of $d_L$, thereby providing a way to measure the absolute BAO length using only low-$z$ ($z \lesssim 0.3$) observations. In this paper, we follow a different route and derive estimates of $r_s$ combining SNe measurements from the Pantheon sample \citep{Scolnic} with eleven $\theta_{\text{BAO}}$ measurements lying in the redshift interval $0.45 \leq z \leq 0.65$ \citep{Carvalho2016,Carvalho2017} together with a high-$z$ angular BAO measurement at $z = 2.225$ \citep{Carvalho2018}. Considering different measurements and estimates of the Hubble constant $H_0$, we compare our estimates of $r_{s}$ with the current estimates of the sound horizon at drag epoch $r_{d}$ and discuss potential mismatches between them.

We organize this paper as follows. In Section 2, we briefly review the physics of the BAOs. We describe the data used in our analysis and the methodology in Sections 3 and 4, respectively. Our results are discussed in Section 4. Section 5 presents our main conclusions.



%%---------------------------------------------------------------------------------------------------------------------------
\section{Data sets}\label{sec3} 
\indent

In order to perform our analysis, we 
%In what follows, we present the data sets used in our analysis.
%\subsection{Angular BAO}
use a set of 12 $\theta_{\textrm{BAO}}(z)$ measurements obtained from public data of the Sloan Digital Sky Survey (SDSS), namely DR10, DR11, and DR12Q (quasars) \citep{Carvalho2016,Carvalho2017,Carvalho2018}. As mentioned earlier, these measurements are derived by calculating the 2PACF between pairs of objects  and considering thin redshift slices with a fair number of cosmic tracers \citep{sanchez}. 
 The compiled BAO dataset is shown in Table \ref{tab:BAO_data}.


%BAO angular measurements can put constraints on the absolute scale $r_s$, once we know the angular diameter distance to that redshift (see Eq.  \ref{eq2}).  While using this dataset, one must remember that since it adopts a model-independent methodology, its error are larger than those that adopt some fiducial cosmology. The current errors in the former approach reach up to 10$\%$ whereas, in the latter, within a few percent 


%\subsection{Type Ia Supernovae}

We also use the Pantheon Sample, which comprises 1048 SNe data points ranging in the redshift interval $0.01 \leq  z \leq  2.3$. It includes 279 SNe ($0.03 \leq z \leq  0.68$) discovered by Pan-STARSS1 (PS1) Medium Deep Survey with distance estimates from SDSS, SNLS, and various low-$z$ SNe along with HST samples. These data have been corrected for bias corrections in the light curve fit parameters using the BEAMS with Bias Corrections (BBC) method. Therefore, the systematic uncertainty related to the photometric calibration has been substantially reduced. Corrected magnitudes of the 1048 SNe, along with their redshift, can be found in \citep{Scolnic}.

\section{Methodology}

To estimate the absolute BAO scale $r_s$ from Eq. (\ref{eq2}) can be estimated by taking mean value of $r_s(z)$ from just low redshift observations. To estimate  it ($r_s(z) = \theta_{\textrm{BAO}} (1+z) D_A(z)$) just from low redshift observations in a completely model independent way, we need to i) know transversal BAO observational values ($\theta_{\textrm{BAO}}(z)$) and ii) calculate angular diameter distance in a completely model independent way. We met first requirement by directly using $\theta_{\textrm{BAO}}(z)$ values from Table \ref{tab:BAO_data}. For completing second requirement, we utilised SNe IA datasets given in Section \ref{sec3}. SNe Ia data comprises of apparent magnitudes values $m_B$ which can be converted to distance modulus values $\mu$ by assuming some values of absolute magnitude values of SNe Ia $M_B$ (as $\mu = m_B - M_B$). The Equation \ref{dl} tells us how luminosity distance to a Type Ia Supernovae can be calculated once apparent magnitude $m_b$ and Absolute Magnitude $M_B$ of Type Ia Supernovae are provided. Using the same, one can easily calculate luminosity distance and hence angular diameter distance corresponding to each redshift by utilising cosmic distance duality relation (CDDR) $ D_{L} = (1+z)^2 D_{A}$.

\begin{equation}\label{dl}
    \text{D}_{L}=10^{(\mu - 25)/5},
\end{equation}
where $\mu = m_{b} - M_{B}$ is the distance modulus, $m_b$  and $M_B$ are the apparent and absolute magnitude of SNI-a respectively.


We have only 12 known $\theta_{\textrm{BAO}}(z)$ data sets but 1048 Sne Ia datasets. To know $r_s(z)$ values, we need 12 SNe Ia angular diameter distance values approximately as same redshift values as that of $\theta_{\textrm{BAO}}(z)$. For that, we adopted two ways. First one, we name as Binning Method and second one we call as Gaussian Process. In First method, we bin the 1048 SNe Ia samples into 12 bins (corresponding to 12 redshifts as of $\theta_{\textrm{BAO}}(z)$ values) . In second way, we use Gaussian Process which helps us to find angular diameter distance to SNe Ia exactly at same redshift as that of $\theta_{\textrm{BAO}}(z)$.





%Both utilises cosmic distance duality relation (CDDR) $ D_{L} = (1+z)^2 D_{A}$ to compare transversal BAO data and Pantheon data at a particular or effective redshift and then finally infer $r_{s}$. CDDR relates luminosity distance with angular diameter distance and it is only valid if photon number remains conserved and gravity is described by metric theory. We expect to obtain luminosity distance from SNe-Ia sample and then calculate angular diameter distance. After that we simply take transversal BAO data $\theta_{\textrm{BAO}}$ and substitute that in Equation 2 along with $D_{A}$ from Pantheon sample to get constrains on $r_{s}$.\\

%SNe data-sets do not directly measure luminosity distance. The measurements are taken of flux or apparent magnitude $m_B$ and then using standard Absolute Magnitude values of all SNe-Ia, we calculate distance modulus to the galaxies which inturn provides the Luminosity Distance [\ref{dl}]. 





%$M_{B}$ = -19.23 $\pm$ 0.01 which we calculated from the latest, most precise model independent measurement of $H_{0}$ (SH0ES Measurement) $H_{0}^{SH0ES}$ = 74.03 $\pm $ 1.42  km s$^{-1}$Mpc$^{-1}$ \citep{Riess2019}.\\~\\
Both the above-mentioned methods to calculate angular diameter distance $D_A$ corresponding to each redshift as that of $\theta_{\textrm{BAO}}(z)$ are given in detail here:\\~\\
\textbf{Binning Supernovae Sample:} We have considered the supernovae between the redshift interval 0.44 to 0.66. And divided the data in 12 redshift bins equal to Transversal BAO datasets. However, for $\Bar{z} = 0.57$ and $\Bar{z} = 2.225$, there are no SNe in the catalog. That's why we don't have binning results in Table \ref{tab:BAO_bingp} and Figure \ref{fig:res2021}.  

It should be noted that observational data from BAO $\theta_{\textrm{BAO}}$, may not be at exact redshift as of the calculated $D_{L}$ from Supernovae sample. Some studies tried to remove this error by exploiting the fact that distance modulus is piece wise linear function of log(z) \citep{Valerio2019} [Equation 20]. However, in Table \ref{tab:BAO_bingp}, we can see that the percentage error that comes while taking $z_{\alpha}$ for calculations and not $z_{\textrm{BAO}}$ is just 0.6$\%$. While binning the Supernovae data sample, we give more weights to the redshifts $z_{\alpha}$. So, it gives us the freedom to choose $z_{\textrm{BAO}}$ nearly equal to the $z_{\alpha}$ for each SN bin. And using this approach, we extremely decrease the possibility of allowed error because of $z_{\textrm{BAO}}$ $\neq$ $z_{\alpha}$ of Supernovae sample.\\
\textbf{Gaussian Process Method:} There is also one another technique where we can redo the analysis and confirm the results. This method may be not as effective as the Binning Method because here we can not assign weights to the $z_{\alpha}$ but surely it can confirm our results. Here we reconstruct apparent magnitude function along with redshift from the Supernovae Observational Sample (Figure \ref{gp}). And we calculate $D_{L}$ at the exact same redshift as we know from $z_{\textrm{BAO}}$ sample. Here we also we tried making bins with the weighted average and we find that the results remains the same. G.P. takes care that the error discussed in \citep{Valerio2019} [Equation 20] doesn't even arrive. We discuss the results and comparison with the Binning Method in Table \ref{tab:BAO_bingp}.  


\begin{figure}
\includegraphics[width=0.4\textwidth]{Figures/m_b.png}
%\vspace{-0.5cm}
\caption{This plot shows Gaussian Process reconstruction of apparent magnitude observations with redshift from Pantheon Sample.}
\label{gp}
%colour_magnitude_diagram
\end{figure}

%As is known, Supernovae Ia are standardized candles,  being their absolute magnitude $M_{B}$ almost the same after the standardization \citep{Valerio2019,Scolnic}. Considering it, we calculate $M_{B}$ using many %observational results of Hubble Constant.  We show the results for both the Binning method and G.P. in Table 2 taking same value of absolute magnitude to keep coherency and so that it can be compared. And that latest %and most accurate model independent value \citep{Efstathiou2021} of the absolute magnitude that we have taken in Table 2 is 
%\begin{equation}
%    M_{B}=-19.214 \pm 0.037
%\end{equation}
Both the above-mentioned methods calculate angular diameter distances $D_A$ corresponding to same redshifts as that of Transveral BAO datasets ($\theta_{\textrm{BAO}}$) without using any cosmological or cosmographic model but exploiting the cosmic-distance duality relation.


%Both the above-mentioned methods calibrate supernovae with BAO measurements without using any cosmological or cosmographic model but exploiting the cosmic-distance duality relation. We use the expression Equation \ref{dl} to calculate Luminosity Distances from the apparent magnitude measurements. Finally, along with the observations of $\theta_{\textrm{BAO}}$ from Table 1, we use Expression \ref{eq2} to get constraints on cosmological parameter $r_{s}$ in a model independent way.


%To estimate the uncertainty of our value inferred to sound horizon in case of Binning, we propagated the errors considering two sources of %this. First we have the error associated the binning of the data that is standard deviation ($\delta$). The other source is related to the %fact that $\theta_{\textrm{BAO}}$ and $\mu$ have errors associated and will bias the sound horizon's error. In the end the total uncertainty of sound %horizon will be: 
%\begin{equation}
%    \Sigma^{2}=\sigma^{2}+\delta^{2},
%\end{equation}
%where $\sigma$ is the uncertainty due to apparent magnitude and angular BAO scale. 

%While Gaussian Processes reconstruction, error bars are given in the output file itself. We have quotted the results along with error bars %in Table \ref{tab:BAO_bin}.



To estimate the uncertainty in our calculated value of absolute scale of BAO in case of Binning, we need to propagate the errors from luminosity distance considering two sources of this. First source, we have the error associated with the binning of the data that is standard deviation ($\delta$).

\begin{equation}
    \delta = \sqrt{\left(\frac{1}{N} \Sigma_{i=1}^{N}(x_i- \mu)^2 \right)}, 
\end{equation}
where N is the total no. of elements in one bin. $x_{i}$ being the value of $i^{th}$ observable in that bin and $\mu$ being the mean of all observables in that bin.
The other source is due to the error in apparent magnitude values. We denote it by $\sigma$. In the end, the total uncertainty in luminosity distance will be:
\begin{equation}
    \Sigma^{2}_{D_L}=\sigma^{2}+\delta^{2}.
\end{equation}

Then the uncertainty of the sound horizon will be a contribution of luminosity distance and $\theta_{\textrm{BAO}}$ that can be written as:
\begin{equation}
    \sigma_{r_{s}}=\sqrt{\left( \frac{\partial r_{s}}{\partial D_{L}}\right)^{2} \Sigma^{2}_{D_L}+\left( \frac{\partial r_{s}}{\partial \theta_{BAO}}\right)^{2} \sigma_{\theta_{BAO}}^{2}}.
\end{equation}

While Gaussian Processes reconstruction, $D_{L}$ error bars are only associated with the apparent magnitude. We have quotted the results along with error bars in Table \ref{tab:BAO_bingp}. 

In the next section, we present and discuss our results about $r_{s}$ i.e absolute scale of BAO compared directly to the high redshift measurement values of $r_{d}$. 

%----------------------------------------------------------------------------------------------------------------------

\section{Model Independent Bounds on $r_{s}$ from different measurements}\label{sec4}
\indent 

In Table \ref{tab:BAO_bingp}, we have quoted our results where we calculated $D_{L}$ by binning supernoave measurements \citep{Scolnic} assuming the latest precise value of absolute magnitude $M_{B} = - 19.214 \pm 0.037$ mag of SH0ES 2021 as mentioned in \citep{Efstathiou2021}. It is calculated by combining geometrical distance estimate from Detached Eclipsing Binaries in LMC \citep{piet}, MASER NGC4258 \citep{reid}, and recent parallex measurements of 75 Milky Way Cepheids with HST photometry \citep{riess21} GAIA Early Data Release 3 (EDR3, \citep{lind20a}) and is also the most precise and latest model independent measure of Absolute Magnitude so far. Then in the same table, we have mentioned results from Gaussian Process for easy comparisons. In Binning as we can see, number of observable data points in each bin varies. Also the error bar of obtained acoustic BAO scale $r_s$ is quite larger. One attempt to make our analysis bias free of unequal no. of observable in each bin and to decrease the error in a bias independent way (which may come when we choose some cosmological model)  is to implement Gaussian Processes (here after G.P.). Here G.P. fulfills that requirement completely. As now, we can directly calculate $D_{L}$ or $r_{s}$ at exact redshift as $z_{\textrm{BAO}}$. Also $r_{s}$ error bar has decreased a lot while using Gaussian Process method. Important point to note here is that the results via both the methods agree within one sigma (see Figure \ref{fig:res2021}) irrespective of the method used (Binning or G.P.).

%%%%%%%%%%%%%%%%%%%%%%%%%%%%%%%%%%%%%%%%%%%%%%%%%%%%%
\begin{table*}
\begin{center}
\scalebox{0.7}{
\renewcommand{\arraystretch}{1.3}
\begin{tabular}{|c|c|c|c|c|c|c|c|c|c|c|c|c|}
\hline
\hline
BAO &  & & SNe Type-A &Binning  &  &  & & &GP & & & \\ 
\hline
\hline
$a$-th bin &  $z_{\rm{bao}}$ & $\theta_{\textrm{BAO}}$& $[z^{\rm a}_l,z^{\rm a}_r]$ & ${z}_a$ & $n_a$ & $D_{L}$ & $D_{A}$ & $r_{s}$(Mpc) & $z_{GP} = z_{\rm{bao}}$ & $D_{L}$ & $D_{A}$ & $r_{s}$(Mpc) \\ \hline
$1$ &  $0.45$ & 4.77 $\pm$ 0.17 & [0.44007, 0.45173]  & 0.44730 & 10 & 2304.34 $\pm$ 226.62 & 1096.00 $\pm$ 107.79 & 132.39 $\pm$ 13.85 &0.45 & 2361.35 $\pm$ 45.00 & 1123.06 $\pm$ 21.40 & 135.57 $\pm$ 5.48\\ \hline
$2$ & $0.47$ & 5.02 $\pm$ 0.25 & $[0.4664, 0.47175]$  & 0.46925 & 7 & 2606.23 $\pm$ 300.43 & 1206.08 $\pm$ 139.03 & 155.20 $\pm$ 19.49 & 0.47& 2486.40 $\pm$ 47.71 & 1150.42 $\pm$ 22.07  & 148.18 $\pm$ 7.91\\\hline
$3$ & $0.49$ & 4.99 $\pm$ 0.21 & [0.4804, 0.49737]   & 0.48635 & 7 & 2581.83 $\pm$ 206.25 & 1162.93 $\pm$ 92.90 & 150.69 $\pm$ 13.61  & 0.49 & 2609.44 $\pm$ 50.48 & 1175.31 $\pm$ 22.74 & 152.52 $\pm$ 7.06 \\ \hline
$4$ & $0.51$ & 4.81 $\pm$ 0.17 & [0.50718, 0.51476]  & 0.51092 & 9 & 2753.04 $\pm$ 246.42 & 1207.42 $\pm$ 108.07 & 153.13 $\pm$ 14.74  & 0.51 & 2734.46 $\pm$ 53.36 & 1199.05 $\pm$ 23.40 & 152.01 $\pm$ 6.14\\ \hline
$5$ & $0.53$ &  4.29 $\pm$ 0.30 & [0.52851, 0.53433]  & 0.53235 & 4 & 2697.60 $\pm$ 203.15 & 1152.38 $\pm$ 86.78 & 132.04 $\pm$ 13.57   & 0.53 & 2861.86 $\pm$ 56.19 & 1222.48 $\pm$ 24.00  & 140.05 $\pm$ 10.17 \\ \hline
$6$ & $0.55$ & 4.25 $\pm$ 0.25 & [0.54539, 0.55381]  & 0.55009 & 6 & 2958.21$\pm$ 286.74 & 1231.30 $\pm$ 119.35 & 141.56 $\pm$ 16.05   & 0.55 & 2996.40 $\pm$ 58.93 & 1246.98 $\pm$ 24.52  & 143.38 $\pm$ 8.89 \\ \hline
$7$ & $0.57$ & 4.59 $\pm$ 0.36 & [0.565, 0.575]  & - & 0 &  - & - & -  & 0.57 & 3134.79 $\pm$ 61.61 & 1271.70 $\pm$ 24.99 & 159.95 $\pm$ 12.93\\ \hline
$8$ & $0.59$ & 4.39 $\pm$ 0.33 & [0.58575, 0.59185]  & 0.58878 & 5  & 3304.57 $\pm$ 286.21 & 1307.13 $\pm$ 113.21 & 159.31 $\pm$ 18.27  & 0.59 & 33275.58 $\pm$  64.78 & 1295.44 $\pm$ 25.62 & 157.83 $\pm$ 12.27\\ \hline
$9$ & $0.61$ & 3.85 $\pm$ 0.31 & [0.61124, 0.60825]  & 0.61 & 4 & 3551.85 $\pm$ 328.27 & 1370.26 $\pm$ 126.64 & 148.43 $\pm$ 18.05  & 0.61 & 3409.94 $\pm$  68.98 & 1315.44 $\pm$ 26.61 & 142.31 $\pm$ 11.81 \\ \hline
$10$ & $0.63$ & 3.90 $\pm$ 0.43 & [0.62522, 0.63222]  & 0.62964 & 6 & 3346.38 $\pm$ 340.44 & 1259.50 $\pm$ 128.13 & 140.10 $\pm$ 20.99  & 0.63 & 3536.01 $\pm$ 74.38 & 1330.64 $\pm$ 27.99 & 147.65 $\pm$ 16.57\\ \hline
$11$ & $0.65$ & 3.55 $\pm$ 0.16  & [0.64191, 0.64864]  & 0.64509 & 5 & 3654.40 $\pm$ 307.19 & 1342.29 $\pm$ 112.83 & 137.20 $\pm$ 13.09   & 0.65 & 3651.93 $\pm$ 79.92 &1341.31 $\pm$ 29.35 & 137.13 $\pm$ 6.87 \\ \hline
$12$ & $2.225$ & 1.77 $\pm$ 0.31 & -  & - & 0 & - & - & - & - & - & - & - \\ \hline
\hline
\hline
\end{tabular}}
\caption{This table encloses the values of Sound Horizon $r_{s}$ using CDDR by both methods Binning and G.P. when applied to Supernovae datasets along with the angular BAO $\theta$ observations. Here we assume the value of absolute magnitude as $M_{B}$=-19.214 $\pm$ 0.037 (SH0ES 2021).}
\label{tab:BAO_bingp}
\end{center}
\end{table*}

\begin{figure*}
\includegraphics[width=0.4\textwidth]{Figures/Efstathiou.png}
%\vspace{-0.5cm}
\caption{This plot shows the acoustic scale of bao found in this work using transversal bao data along with pantheon 2021 datasets. The red points are results from Gaussian Processes and blue points are obtained via binning the Pantheon Dataset for SH0ES 2021 analysis. The black horizontal line in all figure denotes CMB $r_{d}$.}
\label{fig:res2021}
%colour_magnitude_diagram
\end{figure*}

Using one particular value of absolute magnitude $M_{B}$ makes our results dependent on the choice of $M_{B}$. 
\begin{equation}\label{mbh0}
    \text{log $H_{0}$} = \frac{M_{B} +5 \alpha_B + 25}{5}.
\end{equation}
In our study, we have assumed various values of Hubble Constant from seven measurements which include Planck, WMAP, BBN, SHOES 2020, Masers, SHOES 2019, and TFR. Then we use equation \ref{mbh0} to calculate corresponding $M_B$ which we further use to analyse the SNe Ia data set to calculate distance modulus $\mu$ and hence luminosity distance (Equation \ref{dl}). $\alpha_B = 0.71273 \pm 0.00176$ in  equation \ref{mbh0} is the intercept of SNIa magnitude-redshift relation and is independent of CMB or BAO data \citep{Reiss2016} and therefore can be used in performing model independent analysis.   \\~\\


We extended the same analysis as done in table \ref{tab:BAO_bingp} and Figure \ref{fig:res2021} for each seven measurements. Corresponding figures to each measurement are given in Appendix. We quote only final values in Table \ref{tab:all_results}. From there, we can easily conclude that low redshift measurements which allows only higher value of $H_{0}$ comparatively supports smaller value of $r_{s}$ as compare to PLANCK, WMAP and BBN (high redshift measurements).

%%%%%%%%%%%%%%%%%%%%%%%%%%%%%%%%%%%%%%%%%%%%%%%%%%%%%
\begin{table*}
\scalebox{0.9}{
\renewcommand{\arraystretch}{1.3}
\begin{tabular}{|c|c|c|c|c|c|c|c|c}
\hline
\multicolumn{2}{|c|}{ } &  \multicolumn{3}{|c|}{ Binning } &  \multicolumn{3}{|c|}{ G.P.}\\
\hline
\hline
Measurement & $H_{0}$(Km/s/Mpc) & $r_{s}$(Mpc)   & $\eta = \frac{r_s}{r_d}$ & $\sigma$ & $r_{s}$(Mpc) & $\eta = \frac{r_s}{r_d}$ & $\sigma$\\ 
\hline
\hline
Planck & 67.36 $\pm$ 0.54   & 159.44 $\pm$ 17.88    & 1.084 & 0.691 & 161.59 $\pm$ 10.96 & 1.099 & 1.323\\ \hline
WMAP &  67.6 $\pm$ 1.1 & 158.70 $\pm$ 17.93  & 1.079 & 0.647  & 160.84 $\pm$ 11.13 & 1.093 & 1.235\\ \hline
BBN & 68.5 $\pm$ 2.2  & 156.53 $\pm$ 18.21   & 1.064 &  0.518 & 158.64 $\pm$ 11.83 & 1.078 & 0.976\\ \hline
SH0ES 2021 &  74.1 $\pm$ 1.3  & 145.01 $\pm$ 16.39    & 0.986 & 0.127 & 146.96 $\pm$ 10.19 & 0.999 & 0.0127\\ \hline
SH0ES 2020 & 73.2 $\pm$ 1.3  & 146.82 $\pm$ 16.76    & 0.998 & 0.016 & 148.80 $\pm$ 10.35 & 1.011 & 0.1652\\ \hline
Masers &  73.9 $\pm$ 3.0 & 145.41 $\pm$ 17.29   & 0.989 & 0.097 & 147.37 $\pm$ 11.57 & 1.002 & 0.0242\\ \hline
SH0ES 2019 &  74.0 $\pm$ 1.4 & 145.14 $\pm$ 16.44   & 0.987 & 0.118 & 147.10 $\pm$ 10.25 & 1.000 & 0.000975\\ \hline
TFR & 76.0 $\pm$ 2.6  & 141.45 $\pm$ 16.45   & 0.962 & 0.343  &  143.35 $\pm$ 10.80 & 0.974 & 0.346\\ \hline
\hline
\end{tabular}}
\caption{This Table encloses the mean values of Sound Horizon when different values of $H_{0}$ (hence absolute magnitude) from various measurements are assumed for finding $D_{L}$ and hence $D_{A}$ from Supernovae sample. The parameter $\eta$ is defined as $r_{s}/r_{d}$. And the parameter $\sigma$ tells about the extent of tension in acoustic scale of BAO $r_{s}$ and CMB inferred parameter sound horizon at drag epoch $r_{d}= 147.09 \pm 0.26$ Mpc.}
\label{tab:all_results}
\end{table*}

\begin{figure}
\includegraphics[width=0.4\textwidth]{Figures/newcomapareall.png}
%\vspace{-0.5cm}
\caption{This plot shows different $H_{0}$ values obtained from various measurements -both low and high redshift on X axis when transversal bao data is used along with Pantheon using Binning method. On Y axis are the constraints on the $r_{s}$ (absolute scale of Baryon Acoustic Oscillations), we obtained in our analysis.}
\label{fig:rs_H0}
%colour_magnitude_diagram
\end{figure}

We inferred that the mean value of acoustic scale of BAO, $r_{s}$ for very low redshift measurements is compatible with the CMB $r_{d}$ values within one sigma (Figure \ref{fig:rs_H0}).  And not just that, the mean values obtained from model independent method (both binning and G.P.) for low redshift measurements coincide with Planck $r_{d}$ value. But when we consider the high redshift measurements mean values of $r_{s}$ are far from mean value of Planck $r_{d}$ though data points are having less than one sigma tension with the Planck $r_{d}$ value. 

Then we go on to say that $r_{d}$ and $r_{s}$ are related by
\begin{equation}
    r_{s} = \eta r_{d},
\end{equation}
where $\eta$ = 1 tells that the high redshift CMB inferred sound horizon at drag epoch is same as the absolute scale of Baryon Acoustic Oscillations. In Table \ref{tab:all_results}, we can see fluctuations in value of parameter $\eta$ around 1. Also in the same table, we mentioned the tension denoted by $\sigma$ in $r_{d}$ and $r_{s}$ measurements. The tension is less than one in all of the cases because of large error bar in $r_{s}$ measurements (binning results). Though it is visible that $r_{s}$ value for high redshift measurements is always more than or equal to 0.5 $\sigma$ away from Planck $r_{d}$ value which for low redshift measurements is always less than 0.35 (more closer to Planck best-fit). 

In future, with more model independent probes or large no. of datasets, the error bar on $r_s$ will decrease but mean values will more or less remain same. So in this work we put a lot of emphasis on mean value obtained from model independent analysis rather than saying our results of $r_s$ are within one sigma of Planck $r_{d}$. In next section, we try to see how much standard $\Lambda$CDM is consistent with the obtained model independent behavior.  



\subsection{$\Lambda$CDM bounds}\label{sec4}
In the last section, we found out the model independent constraints on $r_{s}$, absolute scale of BAO from both high and low redshift measurements. After finding that our next aim is to check if $\Lambda$CDM behaviour is consistent with the values obtained in a model independent way. So, we assumed $\Lambda$CDM cosmology in a  spatially flat FLRW universe along with normalising present scale factor $a_{0}$ to 1. 0 subscript denotes the present day values.

\begin{equation}
    \frac{H^{2}(z)}{H_{0}^{2}} = \Omega_{m0}(1+z)^{3} + (1 - \Omega_{m0})
\end{equation}
where $\Omega_{m0}$ is the present day matter density. We use the same dataset for BAO (transversal) and Pantheon as mentioned in Section \ref{sec3}. Along with the dataset, we used uniform prior on $\Omega_{m0}$ , $h$ and $r_{d}$ as mentioned in Table \ref{Table:prior}. And the same Gaussian priors on $M_B$ as used in the analysis above for each measurement (also mentioned in same table). We used publicly available code \textit{emcee} \citep{ForemanMackey:2012ig} for generating MCMC chains for the above mentioned dataset. And the bounds that we obtained are given in Table \ref{Table:lcdmcompare}. In Figure \ref{fig:compare_rs}, we can see that the mean values of both the analysis (model independent and $\Lambda$CDM using transversal BAO data) almost overlap over each other in all the measurements. So, we can explicitly say from here that $\Lambda$CDM behavior is well allowed in the range that we obtained from the model independent way. 

\begin{table}
\centering
%\resizebox{1\textwidth}{!}{\begin{minipage}{\textwidth}
\begin{tabular}{ccc}
\Xhline{\arrayrulewidth}
\textbf{Parameter}  & \textbf{Prior (uniform)}\\ 
\Xhline{\arrayrulewidth}
$\Omega_{m0}$ &  [0.1, 0.9]\\
$h$ &  [0.50, 0.90] \\
$r_{d}$ &  [130, 170]\\
\Xhline{\arrayrulewidth}
\textbf{Measurements of $M_B$}  & \textbf{Prior (Gaussian($\mathcal{N}(\mu,\,\sigma^{2})$))}\\ 
\Xhline{\arrayrulewidth}
%#PLANCK, WMAP, BBN, SH0ES 2020, Masers, SH0ES 2019,TFR
% -19.421639607542033, -19.41391652029182, -19.385197142537873, -19.24109459470804, -19.22042780802587, -19.216611253088992, -19.159582038596042, -19.21455896010336
% 0.01950575410988461, 0.03641393523057987, 0.07029372651452852, 0.03955569494449577, 0.08858994274222648, 0.041997568662917165, 0.07480661845941766, 0.03909917855148818
Planck & $\mathcal{N}(-19.422,0.019^{2} )$ \\
WMAP & $\mathcal{N}(-19.414,0.036^{2} )$ \\
BBN &  $\mathcal{N}(-19.385,0.070^{2} )$\\
SH0ES 2021 & $\mathcal{N}(-19.214,0.039^{2} )$  \\
SH0ES 2020 & $\mathcal{N}(-19.241,0.040^{2} )$ \\
Masers &  $\mathcal{N}(-19.220,0.089^{2} )$\\
SH0ES 2019 & $\mathcal{N}(-19.217,0.042^{2} )$ \\
TFR & $\mathcal{N}(-19.159,0.075^{2} )$ \\
\Xhline{\arrayrulewidth} 
\end{tabular} 
\caption{Parameters used and their prior while doing $\Lambda CDM$ analysis. }
\label{Table:prior}
 %\end{minipage}}
\end{table}

\begin{figure}
\includegraphics[width=0.4\textwidth]{Figures/newcomparelcdm.png}
%\vspace{-0.5cm}
\caption{This plot shows different $H_{0}$ values obtained from various measurements (low and redshift) on X axis. On Y axis are the constraints on the $r_{s}$ (absolute scale of Baryon Acoustic Oscillations). On this $H_{0} - r_{s}$ plane, the coloured error bars are for Model Independent analysis and black ones are for $\Lambda$CDM (both for transversal BAO + Pantheon dataset). }
\label{fig:compare_rs}
%colour_magnitude_diagram
\end{figure}


\begin{table*}
\centering
\scalebox{0.9}{\renewcommand{\arraystretch}{1.5}
{\begin{tabular}{|c|c|c|c|c|c|c|}
 \hline
 \hline
 \multicolumn{1}{|c|}{ } &  \multicolumn{3}{|c|}{ Transversal BAO + Pantheon} &  \multicolumn{3}{|c|}{ Volumetric BAO + Pantheon}\\
%  & Transversal BAO& & & Volumetric BAO& & \\
 \hline
 Measurements & $r_s$ (Mpc) & h & $M_B$  & $r_s$ (Mpc)& h & $M_B$  \\
 \hline
  \hline
PLANCK  & $158.27 \pm 2.99$ & $0.68 \pm 0.01$ & $-19.42 \pm 0.02$ & $149.46 \pm 1.91$ & $0.68 \pm 0.01$ & $-19.42 \pm 0.02$\\ \hline
WMAP & $157.70 \pm 3.71$& $0.68 \pm 0.01$ & $-19.41 \pm 0.04$ & $148.89 \pm 2.88$ & $0.68 \pm 0.01$ & $-19.41 \pm 0.04$\\ \hline
BBN & $155.65 \pm 5.67$ & $0.69 \pm 0.02$& $-19.38 \pm 0.07$ & $147.03 \pm 4.92$& $0.69 \pm 0.02$ & $-19.38 \pm 0.07$\\ \hline
 SH0ES 2021  & $143.87 \pm 3.37$ & $0.75 \pm 0.013$ & $-19.21 \pm 0.04$&$136.01 \pm 2.58$ & $0.75 \pm 0.013$ & $-19.22 \pm 0.04$\\ \hline
 SH0ES 2020 & $145.58 \pm 3.57$& $0.74 \pm 0.01$ & $-19.24 \pm 0.04$&$137.63 \pm 2.74$& $0.74 \pm 0.01$ & $-19.24 \pm 0.04$\\ \hline
 Masers & $144.57 \pm 6.26$ & $0.74 \pm 0.03$& $-19.22 \pm 0.09$ &$137.544 \pm 4.87$ & $0.74 \pm 0.03$& $-19.24 \pm 0.07$ \\  \hline
 SH0ES 2019& $144.05 \pm 3.68$& $0.74 \pm 0.01$& $-19.22 \pm 0.04$& $136.07 \pm 2.76$& $0.74 \pm 0.01$& $-19.22 \pm 0.04$\\ \hline
 TFR &$140.57 \pm 5.01$ & $0.76 \pm 0.03$ & $-19.16 \pm 0.07$  &$134.92 \pm 3.47 $ & $0.75 \pm 0.02$ & $-19.20 \pm 0.05$ \\
 \hline
\end{tabular}}}
\caption{Constraints on parameters for $\Lambda$CDM for Transversal and Volumetric BAO data. The error bars quoted are at $1 \sigma$ confidence interval. We got $O_{m0} = 0.29 \pm 0.02$ bounds for all Transversal BAO and $O_{m0} = 0.29 \pm 0.01$ for all Volumetric BAO.}
\label{Table:lcdmcompare}
\end{table*}


\subsection{Tension between Transversal BAO mesurements and Volumetric Data}\label{sec4}
We did the same \textit{emcee} analysis \citep{ForemanMackey:2012ig} as done in the last section with the same Pantheon dataset and Absolute Magnitude Prior but different BAO dataset. The BAO dataset we used is as follows:
\\
\textbf{BAO dataset used:}
We use isotropic BAO measurements by 6dF survey ($z=0.0106$) \citep{Beutler:2011hx}, SDSS-MGS survey ($z=0.15$) \citep{Ross:2014qpa} as well as by eBOSS quasar clustering ($z=1.52$) \citep{Ata:2017dya} and anisotropic BAO measurements by BOSS. Then we also consider BAO measurement by BOSS-DR12 using Lyman-$\alpha$ samples at $z=2.4$ \citep{Bourboux:2017cbm}. Hereafter, we will call all these data together as " volumetric BAO" data.
    
In Figure \ref{fig:compare_bao}, we can explicitly see that the constraints on $r_{s}$ while using transversal BAO data are more than one sigma away from planck $r_d$ for high redshift measurements. On the other hand while using volumetric BAO data, $r_{s}$ and $r_{d}$ have less than one sigma tension for high redshift measurements.  For low redshift measurements, we get constraints on $r_{s}$ within one sigma as obtained by Planck (147.09 $\pm $ 0.26 Mpc) for transversal BAO and the opposite for volumetric BAO, which is we get lower values of $r_{s}$ which are at more than one sigma tension with Planck $r_{d}$. 

This can also be seen in Figure \ref{fig:final_plot}. We take model $\Lambda$CDM for the contours and the error bar are for the model independent analysis (again using transversal BAO data). We have taken Pantheon datasets and $M_{B}$ prior for all the contours except we have taken different datasets of BAO for comparison. In this figure, we can see that model independent analysis behavior that we obtained is completely in alignment with the $\Lambda$CDM analysis. And if in future we get better constraints by model independent analysis, they will agree to volumetric BAO for high redshift measurements and transversal BAO for low redshift measurements. We have shown contours only for two measurements SH0ES 2019 and WMAP. But the behaviour for all low-redshift measurements (SH0ES 2020, SH0ES 2021, TFR, and Masers) will be approximately same as SH0ES 2019. And all high-redshift measurements(Planck and BBN) will have approximately same contours as WMAP. 

\begin{figure}
\includegraphics[width=0.4\textwidth]{Figures/newcomparebao.png}
\vspace{-0.5cm}
\caption{This plot shows different $H_{0}$ values obtained from various measurements (low and high redshift) on X axis. On Y axis are the constraints on the $r_{s}$ (absolute scale of Baryon Acoustic Oscillations). On this $H_0 - r_s$ plane, the teal coloured error bars are when we use volumetric BAO + Pantheon data  and black ones are for transversal BAO data along with Pantheon data (both $\Lambda$CDM).}
\label{fig:compare_bao}
%colour_magnitude_diagram
\end{figure}

\begin{figure}
\includegraphics[width=0.4\textwidth]{Figures/final_plot.pdf}
%\vspace{-0.5cm}
\caption{This figure shows contour plots obtained from two measurements (SH0ES 2019 and WMAP) on $r_{s} H_{0}$ plane. Each analysis done with taking two different datasets: Transversal BAO + Pantheon and Volumetric BAO + Pantheon while assuming $\Lambda CDM$ model. The magenta error bar represents error bar on $r_{s}$ from model independent analysis using binning.}
\label{fig:final_plot}
%colour_magnitude_diagram
\end{figure}

    

%%----------------------------------------------------------------------------------------------------------------------------
\section{Check for Model Independent analysis}

This result is completely in alignment with the fact that the product of Hubble Constant and BAO acoustic scale ($H_{0} \times r_{s}$) is constant as seen in Figure \ref{fig:H0rs} . We have shown both binning and G.P. results in Figure \ref{fig:H0rs} for model independent case and also for model $\Lambda CDM$ using both Pantheon+Transversal BAO and Pantheon+Volumetric BAO. We can see in lower panel of Figure \ref{fig:H0rs} that the binning and G.P. constraints on product of $H_0r_s$ coincide within one sigma. We can say that it is the artifact of the fact that BAO measurements put bounds on $H_0r_{s}$ rather than individually putting constraints on $r_{s}$ or $H_{0}$ \citep{bernal}. Also in CMB measurements, we measure the angular acoustic scale $r_{d}/ D_{A}(z_{d})$  where $z_{d}$ is the drag epoch redshift. This claim is supported by another work that price of shift in value of $H_{0}$ have to be paid by the shift of $r_{d}$ i.e. sound horizon at drag epoch \citep{evslin} so that the product $H_{0} r_{d}$ remains constant. In same figure, top panel we see that due to assuming a model $\Lambda CDM$, the error bar has decreased a lot. But for Planck and WMAP, product of $H_0r_s$ doesnot coincide for Pantheon+Transversal BAO and Pantheon+Volumetric BAO. For all other measurements the constraints on $H_0r_s$ still coincide. 

\begin{figure}
\includegraphics[width=0.48\textwidth]{Figures/H0rs_all.png}
%\vspace{-0.5cm}
\caption{This plot explicitly shows that we get constant product of $H_{0} \times r_{s}$ for all low and high redshift analysis while using model independent analysis (Binning and G.P.) and also using model $\Lambda CDM$ (for both Pantheon+Transveral/Volumetric Bao datasets). }
\label{fig:H0rs}
%colour_magnitude_diagram
\end{figure}

%-----------------------------------------------
\section{Conclusions}\label{sec6}
We found that the absolute scale of Baryon Acoustic Oscillations can be found out just by low redshift measurements. Values of this absolute scale ranges from 141.45 Mpc to 159.44 Mpc when taking $H_{0}$ values from different measurements when we did binning for Supernovae data. On the other hand when we used Gaussian Process Reconstruction, the values of $r_{s}$ varies from 143.35 Mpc to 161.59 Mpc. In our study, both the methods agree with each other within one sigma and also with CMB $r_{d}$ i.e. sound horizon at drag epoch values. We defined a parameter $\eta$ which tells the deviation of $r_{s}$ from $r_{d}$ and we found out that it's values goes from 0.962 to 1.084 for binning results and from 0.974 to 1.099 for Gaussian process reconstruction.

Our results completely verify the results of \citep{Nunes} that transversal BAO provides higher values of $r_{s}$ than volumetric BAO datasets.

We find that the error bars in the estimates of absolute scale of BAO i.e. $r_{s}$ are really high. The constraints may get better as new data in future comes up. Our results supports the claim that the product of $H_{0}r_{s}$ is constant. 

In our work, we also see that in $\Lambda$CDM while taking transversal BAO data, high redshift measurements e.g PLANCK, WMAP and BBN have more deviation with Planck $r_{d}$ then when we consider low redshift measurements e.g. SH0ES 2019, SH0ES 2020, MASERS, TFR (their mean values nearly coincide). On the other hand, while using Volumetric BAO data, high redshift measurements values of $r_{s}$ are coinciding with PLANCK $r_{d}$ and low redshift measurements are having more than one sigma deviation with Planck $r_{d}$.

\textcolor{red}{For $\Lambda$CDM, if $r_{s}$ has to be in alignment with Planck $r_{d}$, then we need to use Transversal datasets of BAO with low redshift measurements and Volumetric datasets of BAO for high redshift measurements (Figure 5).}


%%----------------------------------------------------------------------------------------------------------------------------
\section*{Acknowledgements}







%%**************************************************************************************************
\begin{thebibliography}{99}
%%\bibitem[\protect\citeauthoryear{}{}]{}
%%%%%%%%%%%%%%%%%%%%%%%%%%%%%%%%%

\bibitem[\protect\citeauthoryear{Weinberg et al.}{2013}]{Weinberg:2013agg}
D.~H.~Weinberg, M.~J.~Mortonson, D.~J.~Eisenstein, C.~Hirata, A.~G.~Riess and E.~Rozo,
%``Observational Probes of Cosmic Acceleration,''
Phys. Rept. \textbf{530} (2013), 87-255
%doi:10.1016/j.physrep.2013.05.001
[arXiv:1201.2434 [astro-ph.CO]].


\bibitem[\protect\citeauthoryear{Aubourg et al.}{2015}]{Aubourg:2014yra}
\'E.~Aubourg, S.~Bailey, J.~E.~Bautista, F.~Beutler, V.~Bhardwaj, D.~Bizyaev, M.~Blanton, M.~Blomqvist, A.~S.~Bolton and J.~Bovy, \textit{et al.}
%``Cosmological implications of baryon acoustic oscillation measurements,''
Phys. Rev. D \textbf{92}, no.12, 123516 (2015)
%doi:10.1103/PhysRevD.92.123516
[arXiv:1411.1074 [astro-ph.CO]].

\bibitem[\protect\citeauthoryear{Padmanabhan et al.}{2012}]{Padmanabhan:2012hf}
N.~Padmanabhan, X.~Xu, D.~J.~Eisenstein, R.~Scalzo, A.~J.~Cuesta, K.~T.~Mehta and E.~Kazin,
%``A 2 per cent distance to $z$=0.35 by reconstructing baryon acoustic oscillations - I. Methods and application to the Sloan Digital Sky Survey,''
Mon. Not. Roy. Astron. Soc. \textbf{427} (2012) no.3, 2132-2145
%doi:10.1111/j.1365-2966.2012.21888.x
[arXiv:1202.0090 [astro-ph.CO]].

\bibitem[\protect\citeauthoryear{Aghanim et al.}{2018}]{Planck2018}
Aghanim N., et al., 2018; 1807.06209. 

\bibitem[\protect\citeauthoryear{Aghanim et al.}{2020}]{Planck2020}
Aghanim, N.; Akrami, Y.; Ashdown, M.; Aumont, J.; Baccigalupi, C.; Ballardini, M.; Banday, A.; Barreiro, R.; Bartolo, N.; Basak, S.
Planck 2018 results-VI. Cosmological parameters. Astron. Astrophys. 2020, 641, A6.

\bibitem[\protect\citeauthoryear{Camarena \& Marra}{2019}]{Valerio2019}
Camarena D., Marra V., 2019; arXiv:1910.14125.

\bibitem[\protect\citeauthoryear{Etherington}{1933}]{etherington}
I. M. H. Etherington, Phil. Mag., 15, 761 (1933); reprinted in GRG, 39, 1055 (2007).

\bibitem[\protect\citeauthoryear{Carvalho et al.}{2016}]{Carvalho2016}
Carvalho G. C., Bernui A., Benetti M., Carvalho J. C., Alcaniz J. S., 2016, Phys. Rev., D93,
023530; arXiv:1507.08972 [astro-ph.CO].

\bibitem[\protect\citeauthoryear{Carvalho et al.}{2017}]{Carvalho2017}
Carvalho G. C., Bernui A., Benetti M., Carvalho J. C., Alcaniz J. S., 2017;
arXiv:1709.00271 [astro-ph.CO].

\bibitem[\protect\citeauthoryear{de Carvalho et al.}{2018}]{Carvalho2018}
de Carvalho E., Bernui A., Carvalho G.C., Novaes C.P., Xavier H.S., 2018, JCAP, 1804, 064; arXiv:1709.00113 [astro-ph.CO].

\bibitem[\protect\citeauthoryear{Di Valentino et al.}{2021}]{DiValentino:2021izs}
~Di Valentino, E.,~Mena, O., Pan, S., Visinelli, L., Yang, W., Melchiorri, A., Mota, D. F., Riess, A. G. and Silk, J., 2021, 
%``In the realm of the Hubble tension\textemdash{}a review of solutions,''
Class. Quant. Grav., 18  no.15, 153001. 
[arXiv:2103.01183 [astro-ph.CO]].

\bibitem[\protect\citeauthoryear{Eisenstein \& Hu}{1998}]{Eisenstein1998}
Eisenstein D. J., Hu W., 1998, Apj, 496, 605; arXiv:astro-ph/9709112.

\bibitem[\protect\citeauthoryear{Beutler et al.}{2011}]{Beutler:2011hx}
F. Beutler, C. Blake, M. Colless, D. H. Jones, L. Staveley-
Smith, L. Campbell, Q. Parker, W. Saunders, and
F. Watson, Mon. Not. Roy. Astron. Soc. 416, 3017
(2011), arXiv:1106.3366 [astro-ph.CO]


\bibitem[\protect\citeauthoryear{Ross et al.}{2011}]{Ross:2014qpa}
A. J. Ross, L. Samushia, C. Howlett, W. J. Percival,
A. Burden, and M. Manera, Mon. Not. Roy. Astron.
Soc. 449, 835 (2015), arXiv:1409.3242 [astro-ph.CO] 

\bibitem[\protect\citeauthoryear{Ata et al.}{2011}]{Ata:2017dya}
M. Ata et al., Mon. Not. Roy. Astron. Soc. 473, 4773
(2017), arXiv:1705.06373 [astro-ph.CO] 

\bibitem[\protect\citeauthoryear{Alam et al.}{2011}]{Alam:2016hwk}
S. Alam et al. (BOSS), Mon. Not. Roy. Astron. Soc. 470,
2617 (2011), arXiv:1607.03155 [astro-ph.CO] 


\bibitem[\protect\citeauthoryear{Bourboux et al.}{2017}]{Bourboux:2017cbm}
H. du Mas des Bourboux et al., Astron. Astrophys. 608,
A130 (2017), arXiv:1708.02225 [astro-ph.CO]

\bibitem[\protect\citeauthoryear{Reiss et al.}{2016}]{Reiss2016}
https://doi.org/10.48550/arXiv.1604.01424



\bibitem[\protect\citeauthoryear{ForemanMackey et al.}{2013}]{ForemanMackey:2012ig}
D. Foreman-Mackey, D. W. Hogg, D. Lang,
and
J. Goodman, Publ. Astron. Soc. Pac. 125, 306 (2013),
arXiv:1202.3665 [astro-ph.IM]


\bibitem[\protect\citeauthoryear{Yang et al.}{2019}]{yang}
DOI:10.1016/j.astropartphys.2019.01.005


\bibitem[\protect\citeauthoryear{Evslin et al.}{2018}]{evslin}
Phys. Rev. D 97, 103511 

\bibitem[\protect\citeauthoryear{Liao et al}{2019}]{liao}
DOI:10.3847/1538-4357/ab4819

\bibitem[\protect\citeauthoryear{Holanda et al}{2012}]{holanda1}
R.F.L. Holanda et al JCAP06(2012)022

\bibitem[\protect\citeauthoryear{Holanda et al}{2016}]{holanda2}
 R. F. L. Holanda, V. C. Busti F. S. Lima and J. S. Alcaniz, arXiv:1611.09426 (2016).
 
 \bibitem[\protect\citeauthoryear{Fu et al}{2017}]{fu}
 DOI:10.1142/S0218271817500973

\bibitem[\protect\citeauthoryear{Lv et al}{2016}]{lvetal}
https://doi.org/10.1016/j.dark.2016.06.003

\bibitem[\protect\citeauthoryear{Eisenstein et al.}{2007}]{Eisenstein2007}
Eisenstein D. J., Seo H.J., Sirko E., Spergel D.N., 2007, ApJ, 664, 675.

\bibitem[\protect\citeauthoryear{Eisenstein et al.}{2005}]{Eisenstein2005}
Eisenstein D. J., Zehavi I., Hogg D. et al, 2005, ApJ, 633, 560; arXiv:astro-ph/0501171. 

\bibitem[\protect\citeauthoryear{Efstathiou}{2021}]{Efstathiou2021}
Efstathiou G., 2021, MNRAS, 505, 3866; arXiv:2103.08723

\bibitem[\protect\citeauthoryear{Peebles \& Yu}{1970}]{Peebles}
Peebles P. J. E., Yu J. T., 1970, ApJ, 162, 815. 

\bibitem[\protect\citeauthoryear{Riess et al.}{2019}]{Riess2019}
Riess A. G., Casertano S., Yuan W., Macri L. M., Scolnic D., 2019, ApJ, 876, 85; 1903.07603.

\bibitem[\protect\citeauthoryear{Sánchez et al.}{2011}]{sanchez}
Sánchez E. et al., 2011, MNRAS, 411, 277.

\bibitem[\protect\citeauthoryear{Scolnic et al.}{2018}]{Scolnic}
Scolnic D. M., et al., 2018, ApJ, 859, 101; arXiv:1710.00845. 

\bibitem[\protect\citeauthoryear{Table for SNe-Ia dataset}{2018}]{sntable}
http://dx.DOI.org/10.17909/T95Q4X.

\bibitem[\protect\citeauthoryear{Seo \& Eisenstein}{2003}]{Seo2003}
 Seo H. J., Eisenstein D.J, 2003, ApJ, 598, 720.
 
\bibitem[\protect\citeauthoryear{Sunyaev \& Zel'dovich}{1970}]{Sunyaev}
Sunyaev R. A., Zel'dovich Ya.B., 1970, Ap\&SS, 7, 3. 

\bibitem[\protect\citeauthoryear{Sutherland}{2012}]{Sutherland}
Sutherland W., 2012, MNRAS, 426, 1280; arXiv:1205.0715.

\bibitem[\protect\citeauthoryear{Bernel et al.}{2016}]{bernal}
 José Luis Bernal et al JCAP10(2016)019.

\bibitem[\protect\citeauthoryear{Weinberg et al.}{2012}]{Weinberg}
Weinberg D.H., Mortonson M.J., Eisenstein D.J., Hirata C., Reiss A.G., Rozo E., 2012, Phys. Reports; arXiv:1201.2434.

\bibitem[\protect\citeauthoryear{Alam et al.}{2017}]{alam}
S. Alam et al. (BOSS), Mon. Not. Roy. Astron. Soc.
470, 2617 (2017), arXiv:1607.03155 [astro-ph.CO].

\bibitem[\protect\citeauthoryear{Reid et al.}{2019}]{reid}
Reid M. J., Pesce D. W., Riess A. G., 2019, arXiv e-prints, p.
arXiv:1908.05625

\bibitem[\protect\citeauthoryear{Pietrzy´nski et al.}{2019}]{piet}
Pietrzy´nski G., et al., 2019, Nature, 567, 200

\bibitem[\protect\citeauthoryear{Jarah et al.}{2017}]{jarah}
DOI:10.1103/PhysRevD.97.103511

\bibitem[\protect\citeauthoryear{Riess et al.}{2021}]{riess21}
Riess A. G., Casertano S., Yuan W., Bowers J. B., Macri L., Zinn
J. C., Scolnic D., 2021, ApJ, 908, L6

\bibitem[\protect\citeauthoryear{Lindegren et al.}{2020a,b}]{lind20a}
Lindegren L., et al., 2020a, arXiv e-prints, p. arXiv:2012.01742\\
Lindegren L., et al., 2020b, arXiv e-prints, p. arXiv:2012.03380

\bibitem[\protect\citeauthoryear{Beutler et al.}{2011}]{beutler}
F. Beutler, C. Blake, M. Colless, D. H. Jones,
L. Staveley-Smith, L. Campbell, Q. Parker, W. Saun-
ders, and F. Watson, Mon. Not. Roy. Astron. Soc. 416,
3017 (2011), arXiv:1106.3366 [astro-ph.CO].

\bibitem[\protect\citeauthoryear{Ross et al.}{2015}]{ross}
A. J. Ross, L. Samushia, C. Howlett, W. J. Percival,
A. Burden, and M. Manera, Mon. Not. Roy. Astron.
Soc. 449, 835 (2015), arXiv:1409.3242 [astro-ph.CO].

\bibitem[\protect\citeauthoryear{Aiola et al.}{2020}]{WMAP}
S. Aiola  et al. (ACT), 2020, JCAP, 12, 047, arXiv:2007.07288.

\bibitem[\protect\citeauthoryear{D'Amico et al.}{2020}]{BOSS}
G. D’Amico, J. Gleyzes, N. Kokron, K. Markovic, L. Senatore, P. Zhang, F. Beutlar and Gil-Mar in ,H 2020, JCAP, 05, 005, arXiv:1909.05271

%G. D’Amico, J. Gleyzes, N. Kokron, K. Markovic, L. Senatore, P. Zhang, F. Beutler and Gil-Mar ́ın ,H
2020, JCAP, 05, 005, arXiv:1909.05271.

\bibitem[\protect\citeauthoryear{Riess et al.}{2020}]{Riess2020}
Riess A. G.,  Casertano S., Yuan W., Bowers J. B., Macri L., Zinn J. C., Scolnic D., 2020, ApJ, 908, 1,	arXiv:2012.08534.

\bibitem[\protect\citeauthoryear{Pesce et al.}{2020}]{Masers}
D. W. Pesce et al., 2020, ApJL, 891, L1.

\bibitem[\protect\citeauthoryear{Kourkchi et al.}{2020}]{TFR}
E. Kourkchi, R. B. Tully, G. S. Anand, H. M. Courtois, A. Dupuy, J. D. Neill, L. Rizzi, Mark. Seibert, 2020, ApJ, 896, 3, arXiv:2004.14499.


\bibitem[\protect\citeauthoryear{Nunes et al.}{2020}]{Nunes}
Rafael C. Nunes, Armando Bernui, 2020,
https://doi.org/10.48550/arXiv.2008.03259

\end{thebibliography}



%%%%%%%%%%%%%%%%%%%%%%%%%%%%%%%%%%%%%%%%%%%%%%%%%%%%%%%%%%%%%%%%%%%%%%%%%%%%%%%%%%%%%%%%%%%%%%%%%%%%%%%%%%%55

%\begin{figure*}
%\includegraphics[width=0.8\textwidth]{Figures/Graph2.png}
%\vspace{-0.5cm}
%\caption{The acoustic scale found in this work compared with CMB $r_{s}$: the blue circles are our result and the red line is CMB $r_{s}$ ($r_{s}(z_{d})=147.09 \pm 0.5$ Mpc)}
%\label{fig3}
%colour_magnitude_diagram
%\end{figure*}



%%-----------------------------------------------------------------------------
%\section{Completeness of the Results}\label{sec5}
%Along with using the CDDR with $\theta_{\textrm{BAO}}$ for Supernaovae sample to calculate $D_{A}$ and then constrain $r_{s}$ i.e. absolute scale of BAO, we also used  BAO $D_{V}/r_{s}$ observational data sample observations with a new approximation which connects $D_V$ to the
% luminosity distance $D_{L}$ at a relatively higher redshift and then again constrain $r_{s}$. For converting $D_{V}(z)$ to $D_{L}(\frac{4}{3}z)$, we use:

%\begin{equation}
 %   D_{V}(z) \simeq \frac{3}{4} D_{L} (\frac{4}{3}z )(1+\frac{4}{3}z )^{-1}(1-0.0245z^{3}+0.0105z^{4}).
%\end{equation}


%As concluded in \citep{Sutherland}, this approximation   doesn't explicitly depend on value of $H_{0}$ so, can be used for any analysis where we want to bypass the distance ladder. Also, it is quite accurate in the WMAP allowed neighbourhood of concordance $\Lambda$CDM model or time varying Dark Energy Fields so apt for our analysis. There are only two data points in the required redshift range mentioned in the table \ref{tablesuth}. The results of the following analysis for these two points is given in table \ref{tabledv} and Figure \ref{suth} which agrees with the analysis done above for all other redshifts and to each other (G.P. and Binning).

%\begin{table}\label{dv1}
%\caption{BAO $D_{V}$ measurements using $r_{s}$ fiducial as 148.69 Mpc where dz is the dimensionless ratio $D_{V}(z)/r_{s}$}
%\begin{tabular}{| c | c | c |}
%\hline
%$\Bar{z}$ & Reported Parameter & Reference \\
%\hline
%0.106 & $dz^{-1} = 0.3360 \pm 0.0150$ & \citep{beutler} \\
%0.150 & $dz = (664 \pm 25)/rs_{fid}$ & \citep{ross}  \\
%\hline
%\multicolumn{6}{p{2.5cm}}{\,} 
%\end{tabular}
%\label{tablesuth}
%\end{table}

%\begin{table}\label{dv2}
%\caption{$r_{s}$ measurements from BAO $D_{V}$ Observations }
%\scalebox{0.9}{
%\begin{tabular}{| c | c | c |c |}
%\hline
% & Binning & &   \\
%\hline
%$\Bar{z}$ & $D_{L}(\Bar{4/3z})$ & $D_{V}(\Bar{z}) & %r_{s}(\Bar{z})(Mpc)\\
%\hline
%0.106 & 652.42 $\pm$ 36.37 &  428.67 $\pm$ 23.89 & 144.03 $\pm$ 10.29 \\
%0.150 & 935.72 $\pm$ 59.54 & 584.66 $\pm$ 37.21 & 130.84 $\pm$ 35.33 \\
%\hline
%\hline
% &  G.P. & & \\
% \hline
% $\Bar{z}$ & $D_{L}(\Bar{4/3z})$ & $D_{V}(\Bar{z}) & r_{s}(\Bar{z})(Mpc) \\
% \hline
% 0.106 & 652.29 $\pm$ 39.73  &  428.60 $\pm$ 26.10 &  144.01 $\pm$ 10.88\\
%0.150 &  979.42 $\pm$ 72.48 & 611.74 $\pm$ 45.27  & 136.90 $\pm$ 37.78\\
%\hline
%\hline
%\multicolumn{6}{p{2.5cm}}{\,} 
%\end{tabular}}
%\label{tabledv}
%\end{table}
%\begin{figure}
%\includegraphics[width=0.4\textwidth]{Figures/Sutherland_equation.png}
%\vspace{-0.5cm}
%\caption{This plot shows Binning and Gaussian Process reconstruction results when using $D_{V}$ observational datasets.}
%\label{suth}
%colour_magnitude_diagram
%\end{figure}
%\section{Discussion}\label{sec5}


%1) It can be seen tension in $H_0$ travels to tension in Absolute Magnitude of Supernoave and then $r_s$. (Figure \ref{fig1} and \ref{mbh0})

%2) We measure rs from low redshift measurement and it is coming near planck $r_{s}$. Now this is something important. All the previous study suggest Planck(H0 = 67 and rs= 147.5 ) and if we take local H0 value (H0 reiss = around 73. and rs comes around 135 and to increase that studies suggest early dark energy but we are already getting that)\\
%Adding onto this, can we say that we can not use high redshift H0 values to find absolute magnitude of Supernovae because supernoave are low redshift observations and it would be better to find their absolute magnitude by just low redshift measurements.

%3) Figure 3 paper 1709.00271.pdf


%\begin{figure*}
%\includegraphics[width=0.4\textwidth]{Figures/MB_H0.png}
%\vspace{-0.5cm}
%\caption{This plots shows the values of Absolute Magnitude of Supernoaves that we have considered for respective $H_0$ values.}
%\label{mbh0}
%colour_magnitude_diagram
%\end{figure*}




%%%%%%%%%%%%%%%%%%%%%%%%%%%%%%%%%%%%%%%%%%%%%%%%%%%%%%%%%%%%%%%%%%%%%%%%%%%%%%%%%%%%%%%%%%%%%%%%%%%%%%%%%%%%%%%%%%%%%%%%%%%%%%5


%-----------------------APPENDIX_B------------------------------------------------
\appendix{Appendix}
\section{Supplementary Figures}

\begin{figure*}
\begin{center}
\resizebox{210pt}{160pt}{\includegraphics{Figures/SH0ES_2020.png}}
\hspace{1mm}\resizebox{220pt}{160pt}{\includegraphics{Figures/SH0ES_2019.png}}\\
\hspace{1mm}\resizebox{220pt}{160pt}{\includegraphics{Figures/Masers.png}}
\resizebox{220pt}{160pt}{\includegraphics{Figures/TFR.png}} 
\resizebox{210pt}{160pt}{\includegraphics{Figures/BBN.png}}
\hspace{1mm}\resizebox{220pt}{160pt}{\includegraphics{Figures/WMAP.png}}
\resizebox{210pt}{160pt}{\includegraphics{Figures/Planck.png}}\\
\end{center}
\caption{This plot shows direct comparison between estimation of $r_{s}$ from Gaussian Process reconstruction or via binning method from Pantheon Sample when we assume $H_{0}$ from very low redshift measurements (say SHOES 2020, SHOES 2019, MASERS, and TFR) and high redshift measurements (like BBN, WMAP, and Planck).}
\label{lowexp}
%colour_magnitude_diagram
\end{figure*}
% %\appendix{The BAO bump shift}
% \section{The BAO bump, $\theta_{FIT}$, shift}
% 
% Here we describe how to quantify the BAO bump shift due to the fact that we are using a non-null bin 
% redshift of data, $\delta z = 0.05$, when calculating the $\theta_{\textrm{BAO}}$ at $z=2.225$. 
% %angular diameter distance 
% 
% \noindent
% (i)  $\delta z = 0.0 ,  \theta = 1.4967$ \\
% (ii) $\delta z = 0.05, \theta = 1.4937$ \\
% 
% In other words, the difference between the BAO bumps is $\Delta \theta = 0.002^{\circ}$, that corresponds 
% to a shift of 0.2\%, in fact quite negligible. 
% 
% %%----------------------------------------  fig ?  ----------------------------
% \begin{figure}[h]
% \mbox{\hspace{-0.4cm}
% %\includegraphics[scale=0.47]{figure/angular-correlation-z2p25-dif-3-2.pdf}
% \includegraphics[width=9.5cm, height=6.5cm]{figure/angular-correlation-z2p25-dif-3-2.pdf}
% }
% \caption{Analyses of the shift in the BAO bump due to a non-null bin redshift, i.e., $\delta z \ne 0$, 
% that is, because data in analysis belong to the redshift shell $z \in [2.20, 2.25]$. 
% %using the DR12Q data. 
% The grey line is the expected for the case $\delta z = 0$, instead the red line consider data in a bin 
% of width $\delta z = 0.05$.}
% \label{BAOshift}
% \end{figure}
% %%----------------------------------------  fig ?  ----------------------------


\end{document}
% ****** End of file apssamp.tex ******
%On the other hand, the position of the baryon acoustic oscillation (BAO) feature observed in the large-scale distribution of galaxies is determined by the comoving sound horizon size at the drag epoch. Such a  characteristic scale provides a fundamental standard ruler that can be measured in the CMB anisotropy spectrum and in distribution of large-scale structure at low-$z$, and used to estimate cosmological parameters [REF]. As is well known, the comoving length $r_s(z_{\rm{drag}})$ is calibrated at $z > 1000$ using a combination of CMB observations and theory. 
%%%%%%%%%%%%%%
% the text below was taken from Sutherland
%this leaves us vulnerable to systematic errors from possible unknown new physics at early times (see Section for discussion). Low-redshift measurements of the BAO scale essentially measure a ratio of rs relative to some distance which is itself dependent on cosmological parameters H0, Ωm, w etc. Therefore, in a joint fit to CMB+BAO data, a wrong assumption in the CMB measurement of rs may be masked by biased values of cosmological parameters, i.e. a ‘precisely wrong’ outcome (see Section for an example).

%From last few decades, observational cosmology has helped a lot in completing and confirming our picture of known Universe today. Observational signatures of Cosmic Microwave Background (CMB), Baryon Acoustic Oscillations (BAO), measurements of  Type Ia Supernovae (SNe Ia) and  clustering of galaxies at various epochs have put very strong constraints on the cosmological parameters assuming $\Lambda$CDM Cosmology. But there has been many evidences which shows the hints of deviation from the assumed $\Lambda$CDM Cosmology. Like SH0ES Team \citep{Riess2019} which uses Cepheids as calibrators to study SN Type-I measures $H_{0}$ = 74.03 $\pm$ 1.42 km/s/Mpc without assuming any cosmology. Planck on the other hand finds $H_{0}$ measurement as 67.27 $\pm$ 0.60 km/s/Mpc \citep{Planck2020} using CMB Temperature, Polarisation, Lensing and 6 parameter concordance $\Lambda$CDM cosmology. And there is 4.4$\sigma$ discrepancy between high redshift Planck measurement and Cosmology independent low redshift SH0ES measurement. This mismatch between the the cosmological parameter value $H_0$ explaining the expansion of universe obtained from low and high redshift measurements is called Hubble Tension. It is always encouraging to look for different techniques through which we can find the evolution of Universe and confirm our results in a model independent way. There are claims showing Hubble Tension directly related to tension in parameter $r_{d}$ sound horizon at drag epoch \cite{jarah} because BAO measures combination of $r_{d} H_{0}$ rather than $H_{0}$ and $r_{d}$ individually. We have local independent measurements of Hubble Constant. Also it is necessary to find the value of BAO absolute scale $r_{s}$ at low redshifts in a model independent way. In all the measurements, the two important basic observable that goes into the picture are Luminosity Distance and Angular Diameter Distance. Using these observables only, we get constraints on various cosmological parameters and that adds to complete understanding of evolution of our  Universe. Given any redshift, the Luminosity Distance and Angular Distance combines together to form this relation popularly known as the Cosmological Distance Duality Relation (Hereafter CDDR) which we will be using in our study. The standard Ruler i.e. sound horizon at the drag epoch $r_{d}$ most of the times is inferred from the CMB observations and theory. But studies points out the inferred length of this comoving ruler may be biased by many non standard physics phenomenon of early universe. And this can lead to bias in inference of all cosmological parameters. Therefore, it becomes extremely important to calculate the length of absolute scale in Baryon Acoustic Oscillations denoted as $r_{s}$ just via low redshift measurements. In this work, we in completely model independent way aim to calculate $r_{s}$ explicitly by low redshift measurements i.e. from Supernovae Pantheon Sample and Transversal BAO Measurements. Then we try to compare this with the standard ruler measurements, the sound horizon at drag epoch i.e. $r_{d}$ and see if both comes out to be the same or they have some tension within themselves.


%stated as:
%\begin{equation}\label{cddr}
%    \eta(z) = \frac{(1+z)^2 D_A(z)}{D_L(z)} = 1.
%\end{equation}
%This identity was first proved by Etherington in 1933 \citep{etherington}. Also known as Astronomical version of "Etherington's Reciprocity Theorem", CDDR makes use of the fact that geometric properties remains invariant when the roles of source and observer are transposed in astronomical observations. Etherington's Reciprocity theorem holds true when below mentioned conditions are well satisfied:
%\begin{itemize}
%    \item The space-time is described by metric theory of gravity.
%    \item The photons follows null geodesic.
%    \item Number of Photons is conserved.
%\end{itemize}
%The first two conditions here are more fundamental and related to nature of space-time but the condition 3 on the other hand, usually corresponds to astroparticle mechanism of Particle Physics. Though there are some possibilities of violation of CDDR because of  possibility of new dramatic physics, non-standard mechanism like photon-axion conversion, unrecognized systematic uncertainties, violation of photon number conservation like presence of some opaque source in between (more related to astro particle physics) or some modified gravity theory in which one finds different Luminosity Distance from GW's than inferred from EM signals. But there has not been any confirmations of CDDR violations. If there happens to be the case of cosmic opacity then the third conditions break down because cosmic opacity makes observed flux change by factor of $e^{-\tau(z)}$ where $\tau(z)$ is the optical depth. If $\tau(z) > 0 $ then SN(z) observed $D_L$ would be affected but better understanding of Standard Siren can help us obtain opacity free measurements of $D_{L}$ in near future. Also, worth noting in strong lensing what matters is the measurement angle of the source positions and Intensity measure only contribute to Signal-to-Noise Ratio (SNR). Cosmic Opacity could change the absolute Intensity but not the relative Intensity. So, distance measurements from Strong Lensing are also opacity free \citep{liao}.

%In recent work, \citep{yang} denies the possibility of violation of CDDR by using data from SN-Ia, GRB, Angular Diameter Distance from Strong Lensing Observations(SGL). Also they performed the analysis by simulating 600, 900, 1200  Gravitational Waves data as it is also insensitive to the number of photons or any change in Astroparticle Physics Mechanism. This result is also supported by another work in same direction \citep{holanda1} which makes use of measurements of X-ray surface brightness observations and the gas mass fraction of galaxy clusters from Sunyaev-Zeldovich observations. Another work which supports the assumption of zero opacity \citep{lvetal} and \citep{holanda2}  where they confirmed the validity of CDDR with the latest GRBs data and the SGL along with Union2.1 within 1.5$\sigma$ CL when a power law model(Plaw) is used to study the mass distribution in lensing systems. \citep{fu} also agrees well with the validity of CDDR.
 
%Assuming no modification in Space-time or General Relativity and considering opacity term zero which takes care of the conservation of photons, we use CDDR in our work to determine the length of the sound horizon at low redshift without assuming any preferred cosmology. We use the Luminosity Distance measurements from Supernovae to calculate Angular Diameter Distance. Around the same redshift, we use of Observation of Angular BAO Scale $\theta_{\textrm{BAO}}$ and that together puts constraint on sound horizon $r_{s}$.

%Although the BAO feature evolves by a small amount during the cosmic evolution \citep{Padmanabhan:2012hf}, it is certainly the most robust cosmic ruler at intermediate redshifts currently available. Moreover, its length scale also plays a role in the discussions about the current tensions in the standard cosmology, as some possible solutions for the mismatch between local measurements of $H_0$ and the value inferred from CMB observations assuming the $\Lambda$CDM model, known as the Hubble tension, suggest an increase of the pre-recombination expansion rate, which implies a reduction of the sound horizon at recombination with an increase in $H_0$ (see, e.g. [REF]). More importantly, independent estimates of $r_s$ can be used as a probe of the standard assumptions of the early universe cosmology.


%\section{BAO FEATURES}\label{sec2} 



%If any of the assumptions above are wrong then this can influence the value of CMB $r_{d}$ and will affect the results of others cosmological parameters. 

%BAO scale appears as a bump in the correlation function of CMB \citep{Eisenstein2005} and is given by expression: 
%\begin{equation}\label{rd}
%    r_{s}(z_{d})=r_{d}=\int_{z_{d}}^{\infty} \frac{c_{s}}{H(z)}dz,
%\end{equation}
%being $c_{s}$ the sound speed in photon-baryon fluid and $z_{d}$ was fitted by \cite{Eisenstein1998}. 
%As is well known, there is a difference between $r_{s}$ and $r_{d}$ due to the fact that $r_{s}$ is the comoving length scale of BAO in a galaxy and $r_{d}$ is the comoving horizon at the drag epoch, when photons and baryons decoupled. The difference between these terms is associated with the non-linear growth of structure and evolution of perturbations. The shifts was predicted to be lower than 0.6 percent \citep{Eisenstein2007,Sutherland}. 


%In CMB era, $r_{d}$ calculations were done without assuming flatness of the universe or any details about dark energy.Only assumptions that are taken for inferring $r_d$ are very basic but weakly tested. The list of assumptions \citep{Sutherland} is as follows:
%\begin{itemize}
%     \item General Relativity stands correct,
%     \item Radiation content is taken the standard amount,
%     \item Early dark energy contribution is taken negligible,
%     \item Recombination history was kept standard allowing negligible variations in fundamental constants,
%     \item Contribution of isocurvature fluctuations is also taken negligible amount,
%     \item Smooth and almost a power-law primordial power spectrum is assumed,
%     \item Densities of matter and radiation are assumed to scale as the standard powers of scale factor; fraction of late-decaying %particles at $z \lesssim 10^{6}$ is taken negligible etc. 
%\end{itemize}

% and then we can compare with the sound horizon in the CMB era and test the assumptions about $z > 100$ universe, including \citep{Sutherland}:





%\cite{Sutherland} in his work presented the idea how to determine the length of the horizon scale at low redshift independent of cosmological model, combining measurements of BAO acoustic scale ($\theta_{\textrm{BAO}}$) at some effective redshift with luminosity distance $D_{L}(z)$ from supernovae observations. The motivation to estimate $r_{s}$ at low redshift in model-independent way is to compare its the value with the one inferred by Planck mission at CMB era. Relating these values is a powerful test of standard cosmology at $z > 1000$.

%\sout{In this work, we present a inference of $r_{s}$ at low-redshift independent of any model, calibrating supernovae and BAO data using the method that combine $D_{L}(z)$ and $D_{A}(z)$, also known as \textit{distance-duality relation} and then gives an estimate of model-independent BAO length.  Also, we used the absolute magnitude $M_{B}$ from various model independent measurements of Hubble constant $H_{0}$ to make our analysis independent of any bias which may come from assuming any cosmological model or any one local measurement of Hubble Constant.  }\textcolor{red}{Better suited in introduction}


%~\citep{} 
%\cite{} 

%%-------------------------------------------------------------------------------------------------------------------------




%{In next section, we present the method used to link supernovae with BAO measurements and the data that we used to obtain the absolute BAO scale.} 
   

