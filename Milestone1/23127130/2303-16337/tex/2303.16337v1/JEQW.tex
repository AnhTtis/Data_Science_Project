\documentclass[aps,pre,amsmath,lengthcheck,superscriptaddress]{revtex4-2}

\bibliographystyle{apsrev4-2}
\usepackage{graphicx}\graphicspath{ {figures/} }
\usepackage{hyperref}
\hypersetup{colorlinks,allcolors=blue,breaklinks}

%new commands

\newcommand{\be}{\begin{equation}}
\newcommand{\ee}{\end{equation}}
\newcommand{\ba}{\begin{align}}
\newcommand{\ea}{\end{align}}
\newcommand{\bi}{\begin{itemize}}
\newcommand{\ei}{\end{itemize}}
\newcommand{\R}{\mathbb{R}}
\newcommand{\bra}[1]{\langle #1|}
\newcommand{\ket}[1]{|#1\rangle}
\newcommand{\braket}[2]{\left\langle #1|#2\right\rangle}
\newcommand{\ketbrad}[1]{|#1\rangle\!\langle #1|}
\newcommand{\tr}[1]{\mathrm{tr}\left\{#1\right\}}
\newcommand{\ptr}[2]{\mathrm{tr_{#1}}\left\{#2\right\}}
\newcommand{\di}[1]{\mathrm{div}\left\{#1\right\}}
\newcommand{\la}{\left\langle}
\newcommand{\ra}{\right\rangle}
\newcommand{\pd}{\partial}
\newcommand{\de}[1]{\delta\left(#1\right)}
\newcommand{\td}{\mathrm{d}}
\newcommand{\ma}[1]{\max{\left\{#1\right\}}}
\newcommand{\mi}[1]{\min{\left\{#1\right\}}}
\newcommand{\etal}{\textit{et al.}}
\newcommand{\e}[1]{\exp{\left(#1\right)}}
\newcommand{\lo}[1]{\ln{\left(#1\right)}}
\newcommand{\id}{\mathbb{I}}
\newcommand{\com}[2]{\left[#1,\,#2\right]}
\newcommand{\acom}[2]{\left\{#1,\,#2\right\}}
\newcommand{\co}[1]{\cos{\left(#1\right)}}
\newcommand{\si}[1]{\sin{\left(#1\right)}}
\newcommand{\sh}[1]{\sinh{\left(#1\right)}}
\newcommand{\ch}[1]{\cosh{\left(#1\right)}}
\newcommand{\shi}[1]{\mathrm{shi}{\left(#1\right)}}
\newcommand{\cohi}[1]{\mathrm{chi}{\left(#1\right)}}
\newcommand{\ct}[1]{\coth{\left(#1\right)}}
\newcommand{\bla}{bla\\bla\\bla\\bla\\bla}

\newcommand{\mb}[1]{\mbox{\boldmath$#1$}}
\newcommand{\mc}[1]{\mathcal{#1}}
\newcommand{\mbb}[1]{\mathbb{#1}}
\newcommand{\mf}[1]{\mathfrak{#1}}
\newcommand{\mrm}[1]{\mathrm{#1}}

\begin{document}

\title{Jarzynski equality for quasistatic work}

\author{Pierre Naz\'e}
\email{pierre.naze@unesp.br}

\affiliation{\it Departamento de F\'isica, Instituto de Geoci\^encias e Ci\^encias Exatas, Universidade Estadual Paulista ``J\'ulio de Mesquita Filho'', 13506-900, Rio Claro, SP, Brazil}

\date{\today}

\begin{abstract}

To prove a fluctuation theorem that connects the fluctuations of a work with the average quasistatic work, I show that a thermally isolated system performing an adiabatic driven process has associated another system performing an isothermal driven one. This isothermal process has a temperature equal to that one of the initial equilibrium state of the thermally isolated system, and its difference of Helmholtz free energy between the final and initial equilibrium states is equal to the average quasistatic work of the original process. The application of the usual Jarzynski equality for this isothermal process is therefore straightforward. However, for the complexity of the new system, check proofs corroborating this Jarzynski equality are lacking.

\end{abstract}

\maketitle

{\it Introduction.} Jarzynski equalities (or fluctuations theorems) of all types have been proved over the last decades \cite{sevick2008,esposito2009,seifert2012}. However, few attempts have been made to prove it in the context of showing a connection between the fluctuations of a work performed and the average quasistatic work \cite{jarzynski2020}. In this work, for thermally isolated systems, I prove
\be
\langle e^{-\beta_0 W_{\rm ad}} \rangle = e^{-\beta_0 W_{\rm qs}},
\label{eq:jarzynskiequality}
\ee
where $W_{\rm ad}$ is a work performed, $W_{\rm qs}$ is the average quasistatic work, and $\beta_0:=(k_B T_0)^{-1}$, with $k_B$ being Boltzmann constant, and $T_0$ the temperature of the initial thermal equilibrium of the thermally isolated system. The operation $\langle\cdot\rangle$ is the non-equilibrium average of the associated system. 

{\it Associated isothermal.} Consider a system with Hamiltonian $\mathcal{H}$ dependent on some external parameter $\lambda(t)=\lambda_0+g(t)\delta\lambda$, with $g(0)=0$ and $g(\tau)=1$, where $\tau$ is the switching time of the process. Initially, the system is in thermal equilibrium at a temperature $T_0$. When the process starts, the system is decoupled from the heat bath and adiabatically evolves in time, that is, without any source of heat. The average work performed along the process is
\be
\langle W \rangle = \int_0^\tau \langle \partial_\lambda \mathcal{H} \rangle(t)\dot{\lambda}(t)dt.
\ee
In particular, for quasistatic processes, we called it average quasistatic work
\be
W_{\rm qs} = \lim_{\tau\rightarrow\infty} \langle W \rangle(\tau). 
\ee
The idea to prove equality~\eqref{eq:jarzynskiequality} is to find a different system, performing an isothermal process in a temperature $T_0$, whose difference of Helmholtz free energy between the final and initial state is equal to the average quasistatic work of the original system. In this manner, the application of Jarzynski equality is straightforward.

We start defining a system of interest, with a not-known yet Hamiltonian $\mathcal{H}_{\rm ad}({\bf z}({\bf z_0},t),\lambda(t))$, and a heat bath of temperature $T_0$, with Hamiltonian $\mathcal{H}_{\rm bath}({\bf z'}({\bf z'_0},t))$. Here, ${\bf z}$ and ${\bf z'}$ are points in the phase space evolved accordingly to its respective Hamiltonians from its initial points ${\bf z_0}$ and ${\bf z'_0}$ until the instant of time $t$. Both systems are weakly coupled, such that the total system is given by the Hamiltonian $\mathcal{H}_{\rm total}$. It is expected that the system of interest at the end of our argument crucially depends on the adiabatic process, which would justify its index. 

The average work performed by the system of interest when the parameter is changed from $\lambda_0$ to $\lambda_0+\delta\lambda$ in a process of switching time $\tau$ is
\be
\langle W_{\rm ad}\rangle(\tau) = \int_0^\tau \langle\partial_\lambda \mathcal{H}_{\rm ad}\rangle(t)\dot{\lambda}(t)dt.
\ee
It is not hard to see that it holds the equality of the First Law of Thermodynamics
\be
\langle U_{\rm ad} \rangle = \langle W_{\rm ad}\rangle +\langle Q_{\rm ad} \rangle,
\label{eq:1stlaw}
\ee
where the average energy of the system of interest and the average heat of the process are
\be
\langle U_{\rm ad}\rangle = \langle \mathcal{H}_{\rm ad} \rangle|_0^\tau,
\ee
\be
\langle Q_{\rm ad}\rangle = \langle \mathcal{H}_{\rm bath} \rangle|_0^\tau.
\ee 
Remark that the non-equilibrium averages were taken considering as the non-equilibrium probability distribution the solution of Liouville equation
\be
\frac{\partial\rho_{\rm ad}}{\partial t} = \mathcal{L}_{\rm ad}\rho_{\rm ad},
\ee
where $\mathcal{L}_{\rm ad}:=-\{\cdot,\mathcal{H}_{\rm total}\}$ is the Liouville operator. Here, $\{\cdot,\cdot\}$ is the Poisson bracket. The initial probability distribution is the canonical one
\be
\rho_{\rm ad}(0)=\frac{e^{-\beta_0 \mathcal{H}_{\rm total}}}{\mathcal{Z}_{\rm total}},
\ee
with $\mathcal{Z}_{\rm total}$ being the partition function of the total system. We respectively define now the average internal and external entropies of the system of interest
\be
\langle S_{\rm ad}^i \rangle = \frac{1}{T_0}(\langle W_{\rm ad} \rangle-W_{\rm qs}), 
\label{eq:internalentropy}
\ee
\be
\langle S_{\rm ad}^e \rangle = \frac{\langle Q_{\rm ad} \rangle}{T_0}.
\label{eq:externalentropy}
\ee
The average entropy of the system of interest is 
\be
\langle S_{\rm ad} \rangle = \langle S^i_{\rm ad} \rangle+\langle S^e_{\rm ad} \rangle.
\label{eq:entropy}
\ee
Applying Eq.~\eqref{eq:1stlaw} in Eq.~\eqref{eq:entropy}, we obtain
\be
W_{\rm qs} = \langle U_{\rm ad} \rangle-T_0\langle S_{\rm ad} \rangle,
\label{eq:key}
\ee
which holds for any switching time, in particular for quasistatic processes. Equation~\eqref{eq:key} express that the system of interest is performing a process in contact with a heat bath of temperature $T_0$, such that its difference of Helmholtz free energy is equal to the average quasistatic work of the adiabatic process.

{\it Jarzynski equality.} It remains to find the Hamiltonians $\mathcal{H}_{\rm ad}$ and $\mathcal{H}_{\rm bath}$ of the system of interest and heat bath. In order to determine them, I consider the regime of quasistatic processes. First, I express
\be
W_{\rm qs} = \langle E_{\rm ad} \rangle_0|_{\lambda_0}^{\lambda_0+\delta\lambda},
\ee
where $\langle\cdot\rangle_0$ is the canonical ensemble of the Hamiltonian $\mathcal{H}(\lambda_0)$ calculated at $T_0$. The partition function of the system of interest will be
\be
\mathcal{Z}_{\rm ad} = e^{-\beta_0 \langle E_{\rm ad} \rangle_0}.
\label{eq:partitionfunction}
\ee
To find $\mathcal{H}_{\rm ad}$, I use the expression for the average energy in thermal equilibrium states
\be
\int \mathcal{H}_{\rm ad} e^{-\beta_0 \mathcal{H}_{\rm ad}} d{\bf z_0} =-\partial_{\beta_0} \mathcal{Z}_{\rm ad}.
\ee
Using Eq.~\eqref{eq:partitionfunction}, we have
\be
\int \mathcal{H}_{\rm ad} e^{-\beta_0 \mathcal{H}_{\rm ad}} d{\bf z_0} = \int \mathcal{J} e^{-\beta_0 \mathcal{H}} d{\bf z_0},
\label{eq:key2}
\ee
where
\be
\mathcal{J} = [(1+\beta_0 \langle\mathcal{H}\rangle_0)E_{\rm ad}-\beta_0 E_{\rm ad}\mathcal{H}]e^{-\beta_0 \langle E_{\rm ad}\rangle_0}.
\ee
Assuming that the coordinates of the phase spaces of the system of interest and of the thermally isolated system are the same, we equal the integrands of Eq.~\eqref{eq:key2} and solve the equation for $\mathcal{H}_{\rm ad}$. We have
\be
\mathcal{H}_{\rm ad}(\lambda) = -\frac{1}{\beta_0} \mathcal{W}_k(-\beta_0 \mathcal{J}(\lambda)e^{-\beta_0 \mathcal{H}(\lambda)}),
\label{eq:hamiltoniang}
\ee
where $\mathcal{W}_k$ is Lambert's $\mathcal{W}$ function of branch $k$. Observe that $\mathcal{H}_{\rm ad}$ crucially depends on $\mathcal{H}$. Also, in principle, for each branch $k$ we have a new system of interest. 

Finally, about the heat bath, we can assume that it is the same as the one of the preparation of the original system. This at least solves the problem that both baths must have the same coordinates in their phase spaces. 

Since now every term of the isothermal process performed is of our knowledge, from Ref.~\cite{jarzynski1997} it holds the equality
\be
\langle e^{-\beta_0 W_{\rm ad}}\rangle = e^{-\beta_0 W_{\rm qs}}.
\ee
Also, using Jensen's inequality, we have
\be
\langle W_{\rm ad}\rangle \ge W_{\rm qs},
\ee
showing that $\langle S_{\rm ad}^i\rangle\ge 0$, as it was expected for an average internal entropy. Also, given the complexity of the new Hamiltonian~\eqref{eq:hamiltoniang}, it seems hard to find a relation between $\langle W_{\rm ad}\rangle$ and $\langle W \rangle$. Therefore, the expected result $\langle W\rangle\ge W_{\rm qs}$ is not so obvious to obtain.

{\it Attempt of testing.} In order to corroborate Jarzynski equality~\eqref{eq:jarzynskiequality}, we made an attempt using the classical harmonic oscillator with a time-dependent linear stiffness parameter
\be
\mathcal{H} = \frac{p^2}{2}+\left(\lambda_0+\frac{t}{\tau}\delta\lambda\right)\frac{q^2}{2}.
\ee
The proceeding of finding analytically $\mathcal{H}_{\rm ad}$ is not so hard. However, sampling the initial conditions from the canonical ensemble of the system of interest is an impossible task at this moment. Indeed, this probability distribution is very involved, and numerical methods furnishing this sampling have not been implemented by the available mathematical software of the community ({\it e.g.} {\sc Mathematica}).

{\it Conclusion.} I proved a Jarzynski equality relating the fluctuations of a work with the average quasistatic work. Because of the complexity of the new system of interest, the expected result $\langle W\rangle\ge W_{\rm qs}$ is not so obvious to obtain. Also, for the same reason, mathematical software can not sample numerical initial conditions to be used in the evaluation of the averages for the initial canonical ensemble. Therefore, check proofs corroborating this Jarzynski equality are lacking.

\begin{acknowledgments}
I thank Marcus V. S. Bonan\c{c}a, Artur Soriani, and Sebastian Deffner for reading the manuscript.
\end{acknowledgments}

\bibliography{bibliography.bib}
\bibliographystyle{apsrev4-2}

\end{document}

