\section{Experimental Results}
\label{sec:expers}
In this section, we experimentally evaluate our implementation of \Vp.
In this implementation, we used Minisat~\cite{minisat} as an internal
SAT-solver. The source of \Vp and some examples can be downloaded
from~\cite{ver_pqe}. We conducted three experiments in which we
verified solutions obtained by the PQE algorithm called
\egp~\cite{eg_pqe_tech}. In the experiments we solved the PQE problem
of taking a clause $C$ out of formula \prob{X}{F(X,Y)} i.e.  finding a
formula $H(Y)$ such that $\prob{X}{F} \equiv H \wedge \prob{X}{F
  \setminus \s{C}}$. In Subsections~\ref{ssec:rmv_bps}
and~\ref{ssec:unrem_bps} we consider large formulas appearing in the
process of ``property generation''.  Namely, these formulas were
constructed when generating properties of circuits from the HWMCC-13
set as described in~\cite{eg_pqe_tech}. In
Subsection~\ref{ssec:rand_form}, we consider small random formulas. In
the experiments we used a computer with
Intel\textsuperscript{\textregistered} Core\textsuperscript{TM}
i5-10500 CPU @ 3.10GHz.



\subsection{Formulas where all boundary points are removable}
\label{ssec:rmv_bps}



In this subsection, we consider the PQE problems of taking $C$ out of
\prob{X}{F(X,Y)} where all $C$-boundary points of $F$ are
$Y$-removable.  In~\cite{eg_pqe_tech}, we generated 3,736 of such
formulas. In Table~\ref{tbl:rmv_bps}, we give a sample of 7 formulas.
The first column of the table gives the name of the circuit of the
HWMCC-13 set used to generate the PQE problem. (The real names of
circuits \ti{exmp1}, \ti{exmp2} and \ti{examp3} in the HWMCC-13 set
are \ti{mentorbm1}, \ti{bob12m08m}, and \ti{bob12m03m} respectively.)

%
% Verification on formulas where all bound. pnts
% are removable
%
%\vspace{5pt}
\begin{wraptable}{l}{2.8in}
%\begin{table}
\centering
%\small
  % \vspace{15pt}
\scriptsize
\captionsetup{justification=centering}
\caption{\small{\Vp on formulas where all boundary points are removable}}
%\vspace{-10pt}
%\begin{center}
%\renewcommand{\arraystretch}{1.2} % Default value: 1
%\begni{tabular}{|p{22pt}|p{36pt}|c|c|c|c|} \hline
  \begin{tabular}{|p{25pt}|p{28pt}|p{25pt}|p{21pt}|p{17pt}|p{30pt}|p{29pt}|} \hline
 name & cla- & vari- &size & size &\tiny\ti{EG-PQE}$^+$& \tiny\ti{VerPQE}\\
 of  & uses  & ables & of set & of $H$ & run & run\\ 
 circ. & of $F$  & of $F$       &~~$Y$   &    &time\,(s)  & time\,(s)  \\ \hline
exmp1     & 64,365  &  26,998  & 4,376 &~1 &~~ 0.2  &~~ 0.03 \\ \hline
6s207   & 73,457  &  30,540  & 3,012 &~6  & ~~ 0.2  &~~ 0.04     \\ \hline
exmp2     & 84,009  &  32,147  & 1,994 &~1 & ~~ 0.1  &~~ 0.1  \\ \hline
exmp3     & 94,523  &  41,354  & 5,174 &~825 & ~~ 11   &~~ 0.4  \\ \hline
6s249   & 226,666 &  78,289  & 1,111 &~1 & ~~ 0.4  &~~ 0.1  \\ \hline
6s428   & 231,506 &  92,274  & 3,790 &~118 &~~  2.8  &~~ 0.2  \\ \hline
6s311   & 259,086 &  87,974  & 519   &~80 & ~~ 2.0  &~~ 0.1  \\ \hline
%           &        &       &     &       &      \\ \hline
% exmp1 - mentorbm1,  exmp2 - bob12m08m, exmp3 - bob12m03m
\end{tabular}                
%\end{center}
%\vspace{-10pt}
\label{tbl:rmv_bps}
%\end{table}
\end{wraptable}







The second and third co-lumns give the number of clauses and variables
of formula $F$. The fourth column shows the size of the set $Y$ i.e.
the number of unquantified variables in \prob{X}{F(X,Y)}. The next
column gives the number of clauses in the solution $H$ found by \egp.
The last two columns show the time taken by \egp and \Vp (in seconds)
to finish the PQE problem and verify the solution. As we mentioned in
Subsection~\ref{ssec:perf}, if all $C$-boundary points of $F$ are
$Y$-removable the same applies to formula $F \wedge H$. So, \Vp should
be very efficient even for large formulas.  Table~\ref{tbl:rmv_bps}
substantiates this intuition.


\subsection{Formulas with unremovable boundary points}
%
% Verification on formulas with unremovable
% boundary points
%
\vspace{5pt}
\begin{wraptable}{l}{2.8in}
%\begin{table}
\centering
%\small
  % \vspace{15pt}
\scriptsize
\captionsetup{justification=centering}
\caption{\small{\Vp on formulas with unremovable boundary points. The
time limit is 600 sec.}}
%\begin{center}
%\renewcommand{\arraystretch}{1.2} % Default value: 1
%\begni{tabular}{|p{22pt}|p{36pt}|c|c|c|c|} \hline
  \begin{tabular}{|p{25pt}|p{28pt}|p{25pt}|p{20pt}|p{17pt}|p{30pt}|p{29pt}|} \hline
 name    & cla-     & vari-  &size     & size &\tiny\ti{EG-PQE}$^+$ & \tiny\ti{VerPQE}\\
 of    & uses     & ables  & of set  & of $H$ & run  & run\\ 
 circ.   &  of $F$    & of $F$       &~$Y$    & & time\,(s) & time\,(s)  \\ \hline
 6s209   &  25,086  &  14,868  & 5,759 &~5 &~~ 0.1  & $>$600 \\ \hline 
 6s413   &  29,321  & 14,063   & 4,343 &~18 &~~ 0.2  & $>$600 \\ \hline
 6s276   &  35,810  & 17,631   & 3,201 &~11 &~~ 0.1  & $>$600 \\ \hline
 6s176   &  39,704  & 15,754   & 1,566 &~0 &~~ 0.9  & $>$600 \\ \hline
 6s207   &  73,457  & 30,540   & 3,012 &~20 &~~ 0.5  & $>$600 \\ \hline
 6s110   &  83,396  & 34,165   & 807   &~6 &~~ 0.2  & ~~0.1    \\ \hline
 6s275   &  109,328 & 49,130   & 3,196 &~2 &~~ 0.1  & $>$600 \\ \hline

\end{tabular}                
%\end{center}
%\vspace{-20pt}
\label{tbl:unrem_bps}
%\end{table}
\end{wraptable}






%
Here we consider the same PQE problems as in the previous
subsection. The only difference is that the formula $F$ contains
$C$-boundary points that are $Y$\ti{-unremovable}.
In~\cite{eg_pqe_tech}, we generated 3,094 of such formulas. In
Table~\ref{tbl:unrem_bps}, we give a sample of 7 formulas.  The name
and meaning of each column is the same as in Table~\ref{tbl:rmv_bps}.

\label{ssec:unrem_bps}

Table~\ref{tbl:unrem_bps} shows that \Vp failed to verify 6 out of 7
solutions in the time limit of 600 sec. whereas the corresponding
problems were easily solved by \egp. (The reason is that \egp uses a
more powerful technique of proving redundancy of $C$ than plugging
unremovable boundary points as \Vp does.) So, solutions $H$ obtained
for large formulas \prob{X}{F} where $F$ has a lot of unremovable
boundary points cannot be efficiently verified by \Vp.


\subsection{Random formulas}
\label{ssec:rand_form}
 In this subsection, we continue consider formulas with
 $Y$-unremovable $C$-boundary points. Only, in contrast to the
 previous subsection, here we consider small random formulas. In this
 experiment we verified solutions obtained for formulas whose number
 of variables ranged from 70 to 85. To get more reliable data, for
 each size we generated 100 random PQE problems and computed the
 average result. For each example, the formula $F$ had 20\% of
 two-literal and 80\% of three-literal clauses.

%
% Verification on random formulas
%
%\vspace{-5pt}
\begin{wraptable}{l}{2.5in}
%\begin{table}
\centering
%\small
  % \vspace{15pt}
\scriptsize
\captionsetup{justification=centering}
\caption{\small{\Vp on random formulas}}
\vspace{-5pt}
%\begin{center}
%\renewcommand{\arraystretch}{1.2} % Default value: 1
%\begni{tabular}{|p{22pt}|p{36pt}|c|c|c|c|} \hline
  \begin{tabular}{|p{25pt}|p{15pt}|p{17pt}|p{20pt}|p{18pt}|p{30pt}|p{29pt}|} \hline
 num- & cla- & vari- &size & size&\tiny\ti{EG-PQE}$^+$& \tiny\ti{VerPQE}\\
 ber of  &uses   & ables  & of set & of $H$  & run & run\\ 
 prob. & of $F$ &  of $F$      & ~$Y$  &     &time\,(s)  & time\,(s)  \\ \hline
 ~100 & 140     &~70  &~35 &~28 &~~ 0.01  &~~ 1.0  \\ \hline
  ~100 & 150     &~75   &~37 &~41 &~~ 0.01 &~~ 4.7  \\ \hline
   ~100 & 160     &~80  &~40  &~69 &~~ 0.03 &~~ 11.5  \\ \hline
    ~100 & 170     &~85   &~42 &~63 &~~ 0.03  &~~ 98.3   \\ \hline

\end{tabular}                
%\end{center}
\vspace{5pt}
\label{tbl:rf}
%\end{table}
\end{wraptable}







The results of this experiment are shown in Table~\ref{tbl:rf}. Let us
explain the meaning of each column of this table using its first line.
The first column indicates that we generated 100 PQE problems of the
same size shown in the next three columns. That is for all 100
problems corresponding to the first line of Table~\ref{tbl:rf} the
number of clauses, variables and the size of the set $Y$ was 140, 70
and 35 respectively. The last three columns of the first line show the
\ti{average} results over 100 examples. For instance, the first column
of the three says that the average size of the solution $H$ found by
\egp was 28 clauses. Table~\ref{tbl:rf} shows that the performance of
\Vp drastically drops as the number of variables grows due to the
exponential blow-up of the set of $Y$-unremovable $C$-boundary points.
