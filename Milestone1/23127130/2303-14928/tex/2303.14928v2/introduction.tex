\section{Introduction}
Earlier, we introduced a generalization of Quantifier Elimination (QE)
called \ti{partial} QE (or PQE for short)~\cite{hvc-14}.  PQE allows
to unquantify a \ti{part} of the formula. So, QE is just a special
case of PQE where the entire formula gets unquantified.  The appeal of
PQE is twofold. First, it can be much more efficient than QE if only a
small part of the formula gets unquantified. Second, many known
verification problems like SAT, equivalence checking, model checking
and new problems like property generation can be solved in terms of
PQE~\cite{hvc-14,south_korea,fmcad16,mc_no_inv2,eg_pqe_tech}. So, PQE
can be used to design new efficient algorithms. To make PQE practical,
one needs to verify the correctness of the solution provided by a PQE
solver. Such verification is the focus of this paper.



We consider PQE on propositional formulas in conjunctive normal form
(CNF)\footnote{Every formula is a propositional CNF formula unless otherwise
stated. Given a CNF formula $F$ represented as the conjunction of
clauses $C_1 \wedge \dots \wedge C_k$, we will also consider $F$ as
the \ti{set} of clauses \s{C_1,\dots,C_k}.
} with existential quantifiers. PQE is
defined as follows. Let $F(X,Y)$ be a propositional CNF formula where
$X,Y$ are sets of variables. Let $G$ be a subset of clauses of $F$.
Given a formula \prob{X}{F}, find a quantifier-free formula $H(Y)$
such that $\prob{X}{F}\equiv H\wedge\prob{X}{F \setminus G}$.  In
contrast to QE, only the clauses of $G$ are taken out of the scope of
quantifiers here (hence the name partial QE).  We will refer to $H$ as
a \tb{solution} to PQE. As we mentioned above, PQE \ti{generalizes}
QE. The latter is just a special case of PQE where $G = F$ and the
entire formula is unquantified.
%


To verify the solution $H$ above one needs to check if
$\prob{X}{F}\equiv H\wedge\prob{X}{F \setminus G}$ indeed holds. If
derivation of $H$ is done in some proof system, one can check the
correctness of $H$ by verifying the proof (like it is done for
SAT-solvers). Since, PQE is currently in its infancy and no well
established proof system exists we use a more straightforward
approach.  Namely, we present a very simple SAT-based verification
algorithm called \Vp that does not require any knowledge of how the
solution $H$ is produced. A flaw of \Vp is that, in general, it does
not scale well. Nevertheless, \Vp can be quite useful in two
scenarios. First, \Vp is efficient enough to handle PQE problems formed
from random formulas of up to 70-80 variables. Such examples can can
be employed when debugging a PQE solver. Second, \Vp can efficiently
verify even large PQE problems for a particular class of formulas
described in Subsection~\ref{ssec:perf}.

The paper is structured as follows. Basic definitions are given in
Section~\ref{sec:basic}. Section~\ref{sec:ver_pqe} formally describes
how a solution to PQE can be verified. The verification algorithm
called \Vp is presented in
Section~\ref{sec:ver_alg}. Section~\ref{sec:expers} gives experimental
results. Some background is provided in Section~\ref{sec:bkgr} and
conclusions are made in Section~\ref{sec:concl}.

