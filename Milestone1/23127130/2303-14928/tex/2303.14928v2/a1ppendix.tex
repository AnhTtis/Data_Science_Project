%\clearpage
\vspace{20pt}
\appendix
\noindent{\large \tb{Appendix}}
\section{Proofs Of Propositions}
\label{app:proofs}
\setcounter{proposition}{0}
%
%  proposition about F -> H
%
\begin{proposition}
%\label{prop:sol_impl}
Let $H$ be a solution to the PQE problem of
Definition~\ref{def:pqe_prob}.  That is $\prob{X}{F}\equiv
H\wedge\prob{X}{F \setminus G}$. Then $F \imp H$ (i.e. $F$ implies
$H$).
\end{proposition}
%
% proof
%
\begin{proof}
By conjoining both sides of the equality with $H$ one concludes
that\linebreak $H \wedge \prob{X}{F} \equiv
H\wedge\prob{X}{F \setminus G}$, which entails
$H \wedge \prob{X}{F} \equiv \prob{X}{F}$. Then\linebreak $\prob{X}{F} \imp H$
and thus $F \imp H$.
\end{proof}
%
%  main proposition
%
\begin{proposition}
%\label{prop:main}
Formula $H(Y)$ is a solution to the PQE problem of taking $G$ out
of the scope of quantifiers in \prob{X}{F(X,Y)} if and only if
  \begin{enumerate}
  \item[a)]  $H$ is implied by $F$;
  \item[b)]  $G$ is redundant in $H \wedge \prob{X}{F}$ i.e.
    $H \wedge \prob{X}{F} \equiv H \wedge \prob{X}{F \setminus G}$
 \end{enumerate}
\end{proposition}
%
% proof
%
\begin{proof}\noindent\tb{The if part.} Given the two conditions above,
one needs to prove that $\prob{X}{F} \equiv
H \wedge \prob{X}{F \setminus G}$.  Assume the contrary i.e.
$\prob{X}{F} \not\equiv H \wedge \prob{X}{F \setminus G}$.  Consider
the two possible cases. The first case is that there exists a full
assignment \pnt{y} to $Y$ such that $F$ is satisfiable in
subspace \pnt{y} whereas $H \wedge (F \setminus G)$ is unsatisfiable
in this subspace. Since $F \setminus G$ is satisfiable in the
subspace \pnt{y}, $H$ is unsatisfiable in this subspace.  So, $F$ does
not imply $H$ and we have a contradiction.

The second case is that $F$ is unsatisfiable in the subspace \pnt{y}
whereas\linebreak $H \wedge (F \setminus G)$ is satisfiable
there. Then $H \wedge F$ is unsatisfiable in subspace \pnt{y} too.
So, $H \wedge \prob{X}{F} \neq H \wedge \prob{X}{F \setminus G}$ in
subspace \pnt{y} and hence $G$ is not redundant in
$H \wedge \prob{X}{F}$. So, we have a contradiction again.
%

\vspace{5pt}
\noindent\tb{The only if part}. Given
$\prob{X}{F} \equiv H \wedge \prob{X}{F \setminus G}$, one needs to
prove the two conditions above. The first condition (that $H$ is
implied by $F$) follows from Proposition~\ref{prop:sol_impl}.  Now
assume that the second condition (that $G$ is redundant in
$H \wedge \prob{X}{F}$) does not hold. That is
$H \wedge \prob{X}{F} \not\equiv H \wedge \prob{X}{F \setminus G}$.
Note that if $H \wedge F$ is satisfiable in a subspace \pnt{y}, then
$H \wedge (F \setminus G)$ is satisfiable too. So, the only case to
consider here is that $H \wedge F$ is unsatisfiable in a
subspace \pnt{y} whereas $H \wedge (F \setminus G)$ is satisfiable
there.  This means that $F$ is unsatisfiable in the
subspace \pnt{y}. Then $\prob{X}{F} \neq H \wedge \prob{X}{F \setminus
G}$ in this subspace and we have a contradiction.
\end{proof}
%
% Proposition about redudnancy
%
\begin{proposition}
% \label{prop:form_red}
  Let $F(X,Y)$ be a formula. Let $G$ be a non-empty subset of clauses of $G$.
  The formula $G$ is redundant in \prob{X}{F} if and only if every
  $G$-boundary point of $F$ (if any) is $Y$-unremovable.
\end{proposition}
%
% Proof
%
\begin{proof}
\noindent\tb{The if part.} Given that every $G$-boundary point of $F$ 
is $Y$-unremovable, one needs to show that $G$ is redundant
in \prob{X}{F} i.e.  $\prob{X}{F} \equiv \prob{X}{F \setminus
G}$. Assume that this is not true.  Then there is a full
assignment \pnt{y} to $Y$ such that $F$ is unsatisfiable in
subspace \pnt{y} whereas $F \setminus G$ is satisfiable there. This
means that there is an assignment (\pnt{x},\pnt{y}) falsifying $F$ and
satisfying $F \setminus G$. Since this assignment falsifies $G$, it is
a $G$-boundary point. This boundary point is $Y$-\ti{removable},
because $F$ is unsatisfiable in subspace \pnt{y}. So, we have a
contradiction.

\vspace{5pt}
\noindent\tb{The only if part}. Given that $G$ is redundant
in \prob{X}{F}, one needs to show that every $G$-boundary point of $F$
is $Y$-unremovable. Assume the contrary i.e. there is a $Y$-removable
$G$-boundary point of $F$. This means that there is an assignment
(\pnt{x},\pnt{y}) falsifying $G$ and satisfying $F \setminus G$ such
that $F$ is unsatisfiable in subspace \pnt{y}.  Then
$\prob{X}{F} \neq \prob{X}{F \setminus G}$ in subspace \pnt{y} and so,
we have a contradiction.
\end{proof}
