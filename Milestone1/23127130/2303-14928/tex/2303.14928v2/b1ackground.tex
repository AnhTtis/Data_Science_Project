\section{Some Background}
\label{sec:bkgr}
In this section, we give some background on boundary points.  The
notion of a boundary point with respect to a variable was introduced
in~\cite{esspnts}. (At the time it was called an \ti{essential}
point).  Given a formula $F(X)$, a boundary point with respect to a
variable $x \in X$ is a full assignment \pnt{p} to $X$ such that each
clause falsified by \pnt{p} contains $x$. Later we showed a relation
between a resolution proof and boundary points~\cite{sat09}. Namely,
it was shown that if $F$ is unsatisfiable and contains a boundary
point with respect to a variable $x$, any resolution proof that $F$ is
unsatisfiable has to contain a resolution on $x$. In~\cite{hvc-10}, we
presented an algorithm that performs SAT-solving via boundary point
elimination.

In~\cite{tech_rep_edpll,fmsd14}, we introduced the notion of a
boundary point with respect to a subset of variables rather than a
single variable. Using this notion we formulated a QE algorithm that
builds a solution by eliminating removable boundary
points. In~\cite{eg_pqe_tech}, we formulated two PQE algorithms called
\Eg and \egp.  The algorithm \Eg is quite similar to \Vp and
implicitly employs the notion of a boundary point we introduced here
i.e. the notion formulated with respect to a subset of \ti{clauses}
rather than variables. In this report, when describing \Vp we use this
notion of a boundary point \ti{explicitly}.

