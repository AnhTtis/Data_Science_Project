\documentclass[a4paper,11pt]{article}
\usepackage[utf8]{inputenc}
\usepackage[T1]{fontenc}
\usepackage[english]{babel}
\usepackage{amssymb}
\usepackage[dvipsnames]{xcolor}
\usepackage{amsmath}
\usepackage{amsfonts}
\usepackage{comment} %pour cacher des choses dans le latex \begin{comment}
\usepackage{bbm}
\usepackage{color}
\usepackage{amsthm}
\usepackage{geometry}
\geometry{left=3cm,right=3cm,top=2cm,bottom=2cm} %calibre la taille de la page 
\usepackage{fancyhdr}
\usepackage{esint}
\usepackage{authblk}
\usepackage{multicol}
%\usepackage[cm]{fullpage} % permets de mettre en pleine page 
\usepackage{parskip}
\usepackage{url}




\usepackage[mathscr]{euscript}   % euscript package
\usepackage[color,all]{xy}      % Xy-Pic for diagrams
\usepackage{graphicx}
\usepackage[normalem]{ulem}	% Strikethrough
\usepackage{tikz}

%\setlength{\parindent}{0pt}
%\usepackage{hyperref}
%\usepackage{showkeys}
\usepackage{enumerate}
%\usepackage{enumitem}
\newcommand{\HRule}{\rule{\linewidth}{0.5mm}}

\renewcommand{\thefootnote}{\alph{footnote}}


\pagestyle{fancy}
\fancyhead{}
\fancyfoot[C]{ A groupoid approach to the Wodzicki residue}
\fancyfoot[LE,RO]{\thepage}
\renewcommand{\headrulewidth}{0pt}


\newtheorem{thm}{Theorem}[section]
\newtheorem{coro}{Corollary}[section]
\newtheorem{ppt}{Proposition}[section]
\newtheorem{defi}{Definition}[section]
\newtheorem{lem}{Lemma}[section]
\newtheorem{remark}{Remark}
\newtheorem{exemple}{Example}
\numberwithin{equation}{section}

\author[1]{Nathan Couchet}
\affil[1]{Université Clermont Auvergne, CNRS, LMBP, F-63000 Clermont-Ferrand, France\\ \texttt{nathan.couchet@uca.fr}}



\author[2]{Robert Yuncken \thanks{This article/publication is based upon work from COST Action CaLISTA CA21109 supported by COST (European Cooperation in Science and Technology). www.cost.eu.} }
\affil[2]{Institut Élie Cartan de Lorraine, Université de Lorraine, CNRS, IECL, F-57000 Metz, France \\ \texttt{robert.yuncken@univ-lorraine.fr}}



\title{ A groupoid approach to the Wodzicki residue}
\date{}

\usepackage{xcolor}
\usepackage[colorlinks=true,linkcolor=blue,urlcolor=blue,citecolor=blue]{hyperref}



\newcommand{\Mm}{\mathcal{M}_{m}(\mathbb{C})}
\newcommand{\Mn}{\mathcal{M}_{n}(\mathbb{C})}
\newcommand{\J}{\mathcal{J}}
\newcommand{\hahb}{\mathbb{C}^{n} \otimes \mathbb{C}^{m}}
\newcommand{\A}{\mathcal{A}} %on fait une lettre ronde avec la commande mathcal{}
\newcommand{\B}{\mathcal{B}}
\newcommand{\C}{\mathbb{C}} %on fait le C complexe double barre
\newcommand{\R}{\mathbb{R}}
\newcommand{\Sc}{\mathcal{S}}
\newcommand{\V}{\mathcal{V}}
\newcommand{\ii}{\imath}
\newcommand{\m}{\medbreak} %saut d'une ligne
\newcommand{\bg}{\bigbreak}%saut de plusieurs lignes
\newcommand{\N}{\mathbb{N}}
\newcommand{\x}{\xi}
\newcommand{\Scn}{\mathcal{S}({\R}^n)}
\newcommand{\DT}{\mathcal{S} '({\R}^n)}
% n'accepte pas les chiffres apres newcommand{pas de chiffre}
\newcommand{\Scdn}{\mathcal{S}({\R}^{2n})} % fonction Schwartz Deux n variables Scdn
\newcommand{\Lcn}{L^2({\R}^n)} %fonction L^2 Carré intégrale n variables  Lcn
\newcommand{\Lcdn}{L^2({\R}^{2n})} %fonction L^2 Carré intégrables n variables

\newcommand{\HW}{\mathbb{H}_{1}}
\newcommand{\Lr}{L^1({\R}^{2})}
\newcommand{\supp}{\mathrm{supp}}

\newcommand{\Bor}{\B_{\infty}(\Omega)} %fonction bornée sur Oméga
\newcommand{\Lh}{\mathcal{L}_{c}(H)} %application linéaire continue sur H
\newcommand{\Co}{\mathcal{C}^{0}(\Omega)} %fonction continue sur Oméga
\newcommand{\bigslant}[2]{{\raisebox{.2em}{$#1$}\left/\raisebox{-.2em}{$#2$}\right.}}


%\newcommand{\sphm}{\mathcal{S}_{phg}^m(\R^n)} %symbole polyhomogène d'ordre m sur R^n
\newcommand{\sphmu}{\mathcal{S}_{phg,G}^m(U \times \R^d)}

%\newcommand{\hsmn}{\mathcal{H}\mathcal{S}^m(\R^n)} %fonction homogène modulo Schwartz
\newcommand{\hsmdu}{\mathcal{H}\mathcal{S}_G^m(U \times \R^d)}


%\newcommand{\hsmdu}{\mathcal{H}\mathcal{S}_G^m(U \times \R^d \times \R)}

\newcommand{\sphmdu}{\mathcal{S}_{phg,G}^m(U \times \R^{d+1})}

\newcommand{\sphhmdu}{\mathcal{S}_{phg,G}^m(U \times \R^{d+1})}

\newcommand{\hshmdu}{\mathcal{H}\mathcal{S}_G^m(U \times \R^{d+1} \times \R)}

\newcommand{\Hn}{\mathbb{H}_{n}}
\newcommand{\hn}{\mathfrak{h}_{n}}
\newcommand{\g}{\mathfrak{g}}
\newcommand{\Hom}{\mathrm{Hom}}

\newcommand{\Exp}{\mathbb{E}\mathrm{xp}^{\overline{X}} }


\newcommand{\TM}{\mathbb{T} M}
\newcommand{\THU}{\mathbb{T}_HU}
\newcommand{\THn}{\mathbb{T}_H \mathbb{H}_n}
\newcommand{\THM}{\mathbb{T}_HM}
\newcommand{\kgt}{\tilde{\mathbbm{k}}}
\newcommand{\kg}{\mathbbm{k}}
\newcommand{\Gu}{G^{(0)}}

%%indique qu'on numérote les objets selons les sections
\newtheorem{theoreme}{Theorem}[section]
\newtheorem{proposition}[theoreme]{Proposition}
\newtheorem{lemma}[theoreme]{Lemma}
\newtheorem{corollary}[theoreme]{Corollary}


\theoremstyle{definition}

\newtheorem{definition}[theoreme]{Definition} %%on va définir un commande qui va numéroter les théorèmes définitions lemmes par section
%%  {Définition} sera afficher {definition} sert au début de chaque définition lorsqu'on met \begin{definition} [section]



%%pour ajouter la numérotation des \subsubsection
\setcounter{tocdepth}{3}
\setcounter{secnumdepth}{3}



\begin{document}




\maketitle




%\textbf{\textsc{Mathematical Subject Classification} (2020)}: Here you set the MSC classes. You'll find it on https://zbmath.org/classification/ \\
 
\textbf{\textsc{Keywords}}: Wodzicki residue, pseudodifferential calculus, classical symbol, tangent groupoid, filtered manifold, non-commutative geometry. 

\textbf{\textsc{MSC:}}  Primary: 47G30; secondary: 22A22, 35S05, 58H05, 58J40. 
 
\begin{abstract}
Originally, the noncommutative residue was studied in  the 80's by Wodzicki in his thesis \cite{wodzicki1984LocalInvariantSpectralSym} and Guillemin \cite{guillemin1985newProofsWeylsformula}. In this article we give a definition of the Wodzicki residue, using the langage of $r$-fibered distributions from \cite{lescure2017convolution}, \cite{Yuncken2019groupoidapproach}, in the context of filtered manifolds. We show that this groupoidal residue behaves like a trace on the algebra of pseudodifferential operators on filtered manifolds and coincides with the usual residue Wodzicki in the case where the manifold is trivially filtered. Moreover, in the context of Heisenberg calculus, we show that the groupoidal residue coincides with Ponge's definition \cite{ponge2007residueHeisenberg} for contact and codimension 1 foliation Heisenberg manifolds.
\end{abstract}

\section{Introduction.}

One of the remarkable features of the theory of pseudodifferential operators is the noncommutative residue of Wodzicki, which he defined in 1984 in his thesis \cite{wodzicki1984spectralAssymetryNoncommRes}. The noncommutative residue was also defined in 1985 in a article of Guillemin, \cite[Definition 6 p 151]{guillemin1985newProofsWeylsformula}, in which he proposes to associate to an operator $P$ a zeta function:

\begin{equation}
\zeta(P,s) =\sum_{k} \lambda_k^s,
\end{equation}
where $\lambda_k$ are the eigenvalues of $P$, and then shows that it admits a meromorphic continuation. The residues of this zeta function are linked to the number of eigenvalues of $P$ denoted by $N(\lambda)=card \{ \lambda_k, ~ \lambda_k \leq \lambda \}$, leading Guillemin to a Weyl-type formula, as in \cite{weyl1911asymptotische}. 
 
Let $P$ be a classical pseudodifferential operator of ordrer $m \in \mathbb{Z}$ on a manifold $M$ of dimension $d$. This means that in any chart the symbol of $P$ admits an asymptotic expansion: 

\begin{equation}
a(x,\xi) \sim  \sum_k a_{m-k}(x,\xi).
\end{equation}

In the following discussion, we fix a chart $(U,\phi)$ and we identify $U$ with its image in $\R^d$.
%We start by defining the Wodzicki residue on a compact manifold $M$ on which we fix a smooth measure $dx$. 


\begin{definition}{ \cite[p 58]{vassout2001feuilletages} , \cite[1.8 Formule locale]{kassel1989residu}, \cite[§ 7]{wodzicki1984LocalInvariantSpectralSym}}  \label{751} \m
We define the residue at $x$ of a classical pseudodifferential operator $P$ on $M$ of order $m$ to be:
\begin{equation} \label{766}
Res_x^W(P)=\frac{1}{(2 \pi)^d} \Big( \int_{\mathbb{S}^{d-1}} a_{-d}(x,\xi) d \sigma(\xi) \Big) dx,
\end{equation}
where $a_{-d}(x,\xi)$ is the homogeneous part of order $-d$ in the variable $\xi$ coming from the asymptotic expansion of $P$ in any chart.
Moreover, if $M$ is a compact riemannian manifold, we define also the (global) residue of $P$:
\begin{equation} \label{790}
Res^W(P)=\int_M Res_x^W(P) dx,
\end{equation}
where $dx$ is the smooth measure coming from the riemannian structure.
\end{definition}

It is a profound theorem that these quantitites are independent of the choice of chart. In this paper, we propose to give another definition of this residue for operators of order $-dim(M)$, which extends naturally to filtered manifolds, using the groupoidal calculus from \cite{Yuncken2019groupoidapproach} --- that is using tangent groupoids $\TM, \THM$. We therefore manage to circumvent the chart machinery. Moreover our definition extends essentially without change to define non commutative residue on pseudodifferential operator on any filtered manifold, including that of Ponge on a Heisenberg manifold \cite{ponge2007residueHeisenberg}, using the filtered tangent groupoid of \cite{choi2019tangent},\cite{VanErp2017groupoid}. Note that the compact hypothesis is to make \eqref{790} valid, we can define $Res_x^W$ for $M$ non-compact.
\bg
The noncommutative residue has proved to be connected with other geometrical objets. For instance :

\begin{enumerate}
\item There is link between this residue for a differential operator on a compact manifold and the asymptotic expansion of the trace of the heat operator $e^{-tP}$, see \cite{ackermann1996noteWodzickiRes}.
\item Connes showed in 1988 - see \cite[Proposition 5 p313]{connes1994noncommutative}, \cite[section 2.6 p17]{cardona2020dixmiertracesdiscretepsido}, \cite[section 7.6]{graciabondia2001Elementsofnoncom}, \cite[Proposition 4.11 p 16]{ponge2021connesweyllaws} - that when $P$ is a pseudodifferential operator of order $-dim(M)$ and $M$ is compact, then the Dixmier trace of $P$ coincides to this residue up to a constant. 
\item Ponge showed in \cite[Proposition 6.3 p 454]{ponge2007residueHeisenberg} the link between the residue of the Kohn-Laplacian $\square_b$ on a $CR$ compact manifold, see \cite[Equation (6.9)]{ponge2007residueHeisenberg},  and the volume of $M$, defined at \cite[Equation (6.4)]{ponge2007residueHeisenberg}.
\end{enumerate}






  

Let us very briefly recall the groupoid approach to pseudodifferential operators, first observed by Debord-Skandalis \cite{debord2014adiabatic} in 2014 and developped by van Erp and the second author some years later \cite{Yuncken2019groupoidapproach}. The tangent groupoid of Connes is:
\begin{equation}
\TM=M  \times M \times \R^* \bigcup TM \times \{ 0 \},
\end{equation}
which is seen as a smooth glueing of the tangent bundle $TM$ with a family of pair groupoids $M \times M$ over $\R^*$. In the filtered case the appropriate substitute of the tangent bundle $TM$ is a bundle of nilpotent osculating groups $\mathcal{T}_HM$ whose fibers are denoted by $\mathcal{T}_HM_x$ and we define the filtered tangent groupoid by:
\begin{equation}
\THM=M  \times M \times \R^* \bigcup \mathcal{T}_HM \times \{ 0 \}.
\end{equation}

\bg
The bundle of osculating groups admits a family of automorphism $(\delta_s)_{s >0}$ called dilations which generalises the homotheties on $\TM$ in the trivially-filtered case. 

Using these dilations we obtain a smooth $\R_+^*$-action on $\THM$, namely the Debord-Skandalis action \cite{debord2014adiabatic}, see also \cite{Yuncken2019groupoidapproach} where it is called the \textit{``zoom action''} and \cite{couchet2022polyhomo}:
% \bg
\begin{definition} \label{799}
Let $M$ be a filtered manifold.
We define the Debord-Skandalis action of $\R_+^*$ on $\THM$, $s \in \R_+^* \mapsto \alpha_s \in Aut(\THM)$ by:
\begin{equation}
\left\{
    \begin{array}{ll}
     \alpha_s(y,x,t)=(y,x,s^{-1}t) ~~ (x,y) \in M,  \\
    \alpha_s(x,\xi,0)=(x,\delta_s(\xi),0) ~~ x \in M, ~ \xi \in\mathcal{T}_HM_x.
    \end{array}
\right.
\end{equation}
\end{definition} 


The key point in the groupoidal approach to pseudodifferential operators \cite{Yuncken2019groupoidapproach} is to consider 
the set of distributions on the tangent groupoid $\THM$ which are essentially homogeneous for the Debord-Skandalis action in the following sense. A $r$-fibered distribution on $\THM$ is a continuous $C^\infty(M \times \R)$-linear map  $\kg : C^\infty(\THM) \rightarrow C^\infty(M \times \R)$. This definition implies that the support of such a distribution is $r$-proper and we denote them by $\mathcal{E}_r'(\THM)$. They were first studied by Androulidakis-Skandalis \cite{androulidakis2010pseudodifferential} and Lescure-Manchon-Vassout \cite{lescure2017convolution}  in the general case of submersions. Thanks to a result from \cite{lescure2017convolution}, $r$-fibered distributions can be seen as smooth maps $\kg$ from $M \times \R$ to  
compactly supported distributions in the $r$-fibers, 
% where for all $(x,t) \in M \times \R$,
$\kg(x,t) \in \mathcal{E}'(r^{-1}(x,t))$, and whose support is $r$-proper. We say that a $r$-fibered distribution is properly supported if its support is also $s$-proper. 

Recall that $C_p^\infty(\THM,\Omega_r)$ denotes the proper smooth sections of the 1-density bundle tangent to the $r$-fibers, see \cite[Definition 5.9]{couchet2022polyhomo}.
We say that a properly supported $r$-fibered distribution $\kg$ is essentially homogeneous of order $m$ for the Debord-Skandalis if:
\begin{equation} \label{793}
s \in \R_+^* \mapsto s^{-m} \alpha_{s*} \kg - \kg \in  C_p^\infty(\THM,\Omega_r).
\end{equation}
We refer to the function appearing in \eqref{793} as the co-cycles of $\kg$. The set of essentially homogeneous distributions of this kind is denoted 
$\boldsymbol{\Psi}_{\text{vEY}}^m(\THM)$. We will refer to notations and concepts from \cite{Yuncken2019groupoidapproach} and \cite[section 5.3]{couchet2022polyhomo}. Thanks to this, van Erp and the second author define a pseudodifferential operator as the restriction at $t=1$ of an element $\kg \in \boldsymbol{\Psi}_{\text{vEY}}^m(\THM)$,  see \cite[Theorem 2 p3]{Yuncken2019groupoidapproach}. These elements are the $H$-pseudodifferential operators on the filtered manifold $M$ and are denoted by $\boldsymbol{\Psi}_{H}^m(M)$.
\bg
%We denote by $\boldsymbol{\Psi}_{\text{vEY}}^m(\THM)|_{t=1}$  (resp. $\boldsymbol{\Psi}_{\text{vEY}}^m(\TM)|_{t=1}$ ) the groupoidal calculus of vEY in the case of filtered manifolds $M$ (resp. a trivially filtered manifolds). The essentially homogeneous $r$-fibered distributions of order $m$ on $\THM$ (resp. $\TM$) with respect to the Debord-Skandalis action, that are elements $\kg \in \mathcal{E}_r'(\THM)$ (resp.  $\kg \in \mathcal{E}_r'(\TM)$) such that their co-cycles satify:
%are denoted by $\boldsymbol{\Psi}_{\text{vEY}}^m(\THM)$  (resp. $\boldsymbol{\Psi}_{\text{vEY}}^m(\TM)$).
One of the main theorems of the article \cite{Yuncken2019groupoidapproach} is that the groupoidal calculus coincides, in the case of a trivially filtered manifold, with the classical calculus of Kohn-Nirenberg and Hörmander, namely:
\begin{equation} \label{780}
\boldsymbol{\Psi}_{\text{Hör}}^m(M)=\boldsymbol{\Psi}_{H}^m(M).
\end{equation}



A simple but important first observation of the present paper is the following. We will consider $m \in \mathbb{Z}$ in  the whole paper. If $M$ is a filtered manifold, we denote $d_H$ the homogeneous dimension of $M$. 
 Note that, if we fix a smooth measure $dx$ on $M$, we obtain a canonical smooth family of 1-densities $d \lambda_x$ on the tangent fibers $T_xM$ and hence on the osculating groups $\mathcal{T}_HM_x$. 
 %\color{red} \textbf{Bob : Donc ça veut dire que dans tout le papier $M$ est une variété riemannienne compacte ??} \color{black}
%Note that the density $d \lambda_x$ appearing in \eqref{805} can be identified canonically with 

%we fix a smooth family of Haar measures on the fibers of the osculating groupoid $\mathcal{T}_HM$, which allow us to identify smooth functions with smooth densities. 

\begin{lemma} \label{759}
Let $M$ be a filtered manifold and $\kg \in \boldsymbol{\Psi}_{\text{vEY}}^{- d_H}(\THM)$. 
For every $x \in M$, the function defined by:

\begin{equation} \label{805}
s \mapsto \Big( s^{d_H} \alpha_{s*} \kg - \kg \Big)|_{(x,0,0)},
\end{equation}
is a group homomorphism from $(\R_+^*,\times)$ to  $(\C,+)$. More precisely,  there exists a constant $r_x \in \C$ such that for all $s>0$:
\begin{equation}
\Big( s^{d_H} \alpha_{s*} \kg - \kg \Big)|_{(x,0,0)}=r_x \log(s)d \lambda_x.
\end{equation}
\end{lemma}

We shall give the proof in the next section.
\bg
We may define the quantity $r_x dx$ to be the groupoidal residue of the pseudodifferential operator with kernel $\kg|_{t=1}$, which is indeed a polyhomogeneous pseudodifferential operator by \eqref{780}. As we shall see in Lemma \ref{797}, $r_x dx$ does not depend on the $r$-fibered distribution $\kg$ representing the pseudo-differential operator $P$ at $t=1$. Therefore we can define:  
% reformulate this as follows:

%Moreover we shall see that the constant $r_x$ depends only on the value of $\kg|_{t=1}$. 



\begin{definition} \label{726}
Let $M$ be a filtered manifold of homogeneous dimension $d_H$ and $P \in \bold{\Psi}_{H}^{m}(M)$ with $m \leq -d_H$. Let $\kg \in \bold{\Psi}_{vEY}^{-d_H}(\THM)$ be an element in vEY groupoidal calculus such that $\kg|_{t=1}$ is the Schwartz kernel of $P$. We define the groupoidal residue of $P$ at $x \in M$, denoted $Res_x(P)$,   for any $s \in \R_+^* \setminus \{ 1 \}$ to be:
\begin{equation} \label{727}
Res_x(P):= \frac{1 }{\log(s)} \Big( s^{d_H} \alpha_{s*} \kg - \kg \Big)|_{(x,0,0)}.
\end{equation}
\end{definition}

The above definition is for scalar-valued operators. For operators between vector bundles, we should take, for any $s \in \R_+^* \setminus \{ 1 \}$:

\begin{equation} \label{798}
Res_x(P):= \frac{1 }{\log(s)} \mathrm{Tr} \Big( s^{d_H} \alpha_{s*} \kg - \kg \Big)|_{(x,0,0)}.
\end{equation}
For details on the groupoidal calculus with vector-bundle coefficients, see \cite{dave2022gradedHypoellipticity}. We will restrict our attention to scalar-valued operators for simplicity.
\bg
Now come the main results of this paper:
\begin{enumerate}
\item We shall show, see Theorem \ref{746}, that similarly to the Wodzicki residue $Res_x^W$, the groupoidal residue $Res_x$ from Definition \ref{726} defines a trace on operators of appropriate order.
More precisely if $P \in \bold{\Psi}_{H}^{m}(M)$ where $M$ is a filtered manifold and $m \leq -d_H$, $Q \in \bold{\Psi}_{H}^{0}(M)$ then :
\begin{equation} 
Res_x([P,Q]) =0,
\end{equation}
where $[~,~]$ denotes the commutator of operators.
\item  We shall show, see Theorem \ref{758}, that the groupoidal residue from Definition \ref{726} coincides with the Wodzicki residue in the case of a trivially filtered manifold.
\item  We shall show, see Theorem \ref{774} and Corollary \ref{779}, that the definition of the noncommutative residue made by Raphaël Ponge \cite{ponge2007residueHeisenberg} in 2007 coincides with groupoidal residue from Definition \ref{726} for $\V$-pseudo-differential operators in the calculus of BG, \cite[section § 10]{Beals2016Heisenbergcalculus} where $M$ is a contact manifold or a foliation of codimension 1.
\end{enumerate}

Definition \ref{726} only applies to pseudo-differential operators of order $\leq -d_H$. It should be possible to extend this definition to operators of arbitrary order. In the case of the classical unfiltered calculus, this will be treated in a forthcoming article by Higson, Sukochev and Zanin \cite{higson2023NoncommutativeResidue}. We learnt of their work while writing this article. We refer to that article for details. 

%In an article in preparation  they will propose a definition for the noncommutative residue of a classical pseudodifferential operator of arbitrary order using an elaboration of the groupoidal method described here.




\section{Basic properties of the Wodzicki residue on a filtered manifold.}

Maintaining the notation of the introduction, let us start by proving Lemmas \ref{759} and \ref{797}, which show the well-definedness of our residue. Our conventions for the filtered tangent groupoid are such that the range and source maps are given by :

\begin{equation}
\left\{ \begin{array}{ll}
r(y,x,t)&=y, ~ s(y,x,t)=x, ~ t \neq 0 \\
r(x,v,0)&=x, ~ s(x,v,0)=x.
\end{array}
\right.
\end{equation}


\begin{proof}[Proof of Lemma \eqref{759}]
Set:
\begin{equation} \label{806}
F : s \in \R_+^* \mapsto F_s:=\Big( s^{d_H} \alpha_{s*} \kg - \kg \Big) \in C_p^\infty(\THM,\Omega_r).
\end{equation}
We will show that it is a group morphism when restricted at $(x,0,0)$. First, the reader can easily check that:
\begin{equation} \label{761}
F_{st}=s^{d_H} \alpha_{s*} F_t +F_s, ~ s,t >0.
\end{equation}

%Then $F$ is a group morphism if:
%\begin{align}
%F_{st}|_{(x,0,0)} & \underbrace{=}_{\eqref{761}} s^{d_H} \alpha_{s*} F_t|_{(x,0,0)} +F_s|_{(x,0,0)} \\
% & = F_s|_{(x,0,0)}+ F_t|_{(x,0,0)}, \label{787}
%\end{align}
%is true. So the only way to have \eqref{787} is to show:
Thus we need to show:
\begin{equation} \label{760}
 s^{d_H} \alpha_{s*} F_t|_{(x,0,0)}=F_t|_{(x,0,0)}. 
\end{equation}
We are therefore going to prove \eqref{760}.
In the fiber $(x,.,0)$, we can write:
\begin{equation} \label{804}
 F_s |_{(x,.,0)}=f_x d \lambda_x,
\end{equation}
for some $f_x \in C_c^\infty(\mathcal{T}_HM_x)$ and where $d\lambda_x$ is a Haar measure on the  nilpotent (graded) osculating group $\mathcal{T}_HM_x$ such that $\delta_{s*}(d \lambda_x)=s^{-d_H} d\lambda_x$.  The Debord-Skandalis action at $t=0$ acts by the dilations $\delta_s$.
So for the fiber $(x,.,0)$ we have:
\begin{align}
\alpha_{s*}( F_s )|_{(x,.,0)} &=\delta_{s*}(F_s|_{(x,.,0)}) \\
 & \underbrace{=}_{\eqref{804}} \Big((\delta_s^{-1})^*f_x \Big)  \delta_{s*} (d\lambda_x) \\
& =s^{-d_H} \Big((\delta_s^{-1})^*f_x \Big)  d\lambda_x. \label{794}
\end{align}

%=s^{-d_H} d\lambda_x
Since the Debord-Skandalis action $\alpha_s$ fixes the points $(x,0,0)$, this gives 
$s^{d_H} \alpha_{s*} F_t|_{(x,0,0)}=F_t|_{(x,0,0)}$ as desired. Therefore, $s \mapsto F_s|_{(x,0,0)}$ is a morphism from $(\R_+^*,\times)$ to $(\C,+)$. Moreover, note that the map from \eqref{806} is smooth, see \cite[Lemma 21]{Yuncken2019groupoidapproach}.
%So there exists $r_x \in \C$ such that:
%\begin{equation}
%F_s|_{(x,0,0)}=r_x \log(s), \forall ~ s >0.
%\end{equation}
\end{proof}


%The following lemma ensures the well-definedness of the groupoidal residue.

\begin{lemma} \label{797} 
Let $M$ be a filtered manifold and $P \in \boldsymbol{\Psi}_{H}^{m}(M)$, with $m \leq -d_H$.
The Definition \ref{726} of the groupoidal residue does not depend on the $r$-fibered distribution $\kg$ that represents $P$.
\end{lemma}

\begin{proof}
Let $\kg,\kg' \in \bold{\Psi}_{vEY}^{-d_H}(\THM)$ such that:
\begin{equation}
\kg|_{t=1}=P, ~ \kg'|_{t=1}=P.
\end{equation} 
Let $x \in M$ be fixed.
We need to show that the two co-cycles \eqref{727} related to these $r$-fibered distributions give the same groupoidal residue at $x$. Thanks to \cite[Corollary 33]{Yuncken2019groupoidapproach}, we have:
\begin{equation} \label{796}
\kg|_{t=0}-\kg'|_{t=0} \in C_p^\infty(\mathcal{T}_HM,\Omega_r).
\end{equation}
%Using this, one shows that the co-cycle \eqref{727} at $(x,0,0)$ for $\kg - \kg'$ 
%is zero.  
Using \eqref{796}, there exists $f_x \in C_c^\infty(\mathcal{T}_HM_x)$ such that:
\begin{equation}
(\kg-\kg')|_{(x,.,0)}=f_x d\lambda_x,
\end{equation}

where $d\lambda_x$ is a left Haar measure on the nilpotent (graded) Lie group $\mathcal{T}_HM_x$. To conclude, remember that the Debord-Skandalis action fixes $(x,0,0)$. Indeed, we compute:
\begin{align}
\Big( s^{d_H} \alpha_{s*}( \kg-\kg') -( \kg-\kg')  \Big)|_{(x,0,0)} &=  s^{d_H} \alpha_{s*}( \kg-\kg')|_{(x,0,0)} - ( \kg-\kg')|_{(x,0,0)}  \\
& = s^{d_H} \delta_{s*}( f_x d\lambda_x) - f_x d\lambda_x \\ 
& = 0,
\end{align}
as in the previous proof.
\end{proof}

Now  we can conclude that the groupoidal residue is well-defined.
\bg
Let us now recall a basic fact about convolution Lie groups. In the following Lemma the commutators are in the sense of convolution product. Also recall that every connected nilpotent Lie group is unidomular, see \cite[Proposition 2.30]{folland2016harmonicanalysis} or \cite[Proposition 5.5.4 et Corollary 5.5.5]{faraut2008analysis}.
%The reader shall recall the definition of an algebroid, see \cite{moerdijk2003introduction}, \cite{mackenzie2005generalTheoryGroupoidsAlgebroids}.
%Also recall that every connected nilpotent Lie group is unidomular, see \cite[proposition 2.30]{folland2016harmonicanalysis} or \cite[proposition 5.5.4 et corollaire 5.5.5]{faraut2008analysis}.

\begin{lemma} \label{757}
Let $G$ be a connected nilpotent Lie group $G$. Given  $f \in C_c^\infty(G)$ and $g \in \mathcal{E}'(G)$ then $[f,g](e)=0$.
\end{lemma}

\begin{proof}
Recall that convolution on Lie group is defined by:
\begin{equation}
\underbrace{u \star v}_{\in ~ \mathcal{E}'(G)}=\underbrace{(u \otimes v)}_{\in ~ \mathcal{E}'(G \times G)} \circ ~ m^*,
\end{equation}
where  $u \in \mathcal{E}'(G), ~ v  \in \mathcal{E}'(G), ~  (m^*f)(x,y)=f(xy).$ We may also take $v \in \mathcal{D}'(G)$ and in this case $u \star v \in \mathcal{D}'(G)$, see  \cite[Theorem 5.1.1 p51 on $\R^d$]{friedlander1998introduction}.
The commutator is well-defined because $C_c^\infty(G)$ is a two-sided ideal in $\mathcal{E}'(G)$. 
%(voir \cite[thm 5.2.1 p53 sur $\R^d$]{friedlander1998introduction})
Still inspired by the \cite[Theorem 5.2.1 p53 on $\R^d$]{friedlander1998introduction}, one can prove that for $f \in C_c^\infty(G)$ and $g \in \mathcal{E}'(G)$:
\begin{equation}
f \star g(x)= \langle g(y),f(xy^{-1}) \rangle ,
\end{equation}

%\begin{center}
%\color{red} \textbf{Bob : Je pense que le résultat est faux si le groupe n'est pas unimodulaire, il doit y avoir une fonction modulaire qui apparaît dans l'équation précédente (j'ai fait l'exemple avec des fonctions, voir le livre de Folland A course in Abstract Harmonic Analysis p56)} \color{black}
%\end{center}
and:
\begin{equation}
g \star f(x)= \langle g(y),f(y^{-1}x) \rangle.
\end{equation}
See that it is in the two previous equations that we use the unimodularness of $G$.
Then we have proved $f \star g(e)= g \star f(e)$,
that is $[f,g](e)=0$.
% First recall definition of convolution of $r$-fibered distributions, see \cite[section 2.2]{Yuncken2019groupoidapproach}. In our case, $E$ is seen as a groupoid, where we have set the range and source maps to be $r,s=\pi$. Let $\phi \in C^\infty(E)$ and $x \in M$. We compute:
%\begin{equation}
%\langle f \star g, \phi \rangle(x)=\langle f^x(\gamma),\langle g^{s(\gamma)}(\beta),\phi(\gamma \beta) \rangle \rangle, ~ \gamma \in \pi^{-1}(x), ~ \beta \in \pi^{-1}(\pi(\gamma)).
%\end{equation}

%The previous formula expresses the fact that the convolution is processed fiber by fiber on the nilpotent (a fortiori unimodular) Lie group $\pi^{-1}(x)$. With what we have just proved in point one:

%\begin{equation}
%\langle f \star g, \phi \rangle(x)=\langle g \star f, \phi \rangle(x),
%\end{equation}

%that is $[f,k](x)=0$.
\end{proof}


\begin{theoreme} \label{746} \m
Let $M$ be a filtered manifold of homogeneous dimension $d_H$. Let $P \in \Psi_{H}^{m}(M)$ be a pseudodifferential operator $M$ with $m \leq -d_H$ and $Q \in \Psi_{H}^{0}(M)$. 
Then the groupoidal residue of Definition \ref{726} satisfies the trace property, that is for all $x \in M$ :
\begin{equation} \label{747}
Res_x([P,Q]) =0.
\end{equation}

\end{theoreme}


\begin{proof} 
We denote by $\kg_P$ and $\kg_Q$ 
the two associated essentially homogeneous  $r$-fibered distributions of order $-d_H$ and  $0$ respectively. 
%on the nose outside $[-1,1]$ such that their restrictions at $t=1$ are respectively equal to  $P$ and $Q$, again this is due to \cite[Proposition 42]{Yuncken2019groupoidapproach}. 
By definition, there exists $f,g \in C_p^\infty(\THM,\Omega_r)$ such that:
\begin{align}
Res_x([P,Q]) &= Res_x(PQ)-Res_x(QP) \\
             &= \frac{1}{log(s)} \Big( (s^{d_H}  \alpha_{s*} (\kg_P \star \kg_Q) - \kg_P \star \kg_Q  )|_{(x,0,0)} \nonumber \\  & ~~~ - (s^{d_H} \alpha_{s*} ( \kg_Q \star \kg_P) -  \kg_Q \star \kg_P  )|_{(x,0,0)}  \Big) \label{792} \\
             & = \frac{1}{log(s)} \Big(s^{d_H} (( s^{-d_H} \kg_P + f) \star (\kg_Q +g) - \kg_P \star \kg_Q  )|_{(x,0,0)} \nonumber \\
              & ~~~ -  s^{d_H} ((  \kg_Q + g) \star (s^{-d_H} \kg_P +f) + \kg_Q \star \kg_P )|_{(x,0,0)} \Big) \label{789} \\
           & = \frac{1}{log(s)}  \Big( \kg_P \star g +f \star \kg_Q - \kg_Q \star f - g \star \kg_P \Big)|_{(x,0,0)} \\
          &= \frac{1}{log(s)}  \Big( [\kg_P,g]+[\kg_Q,f] \Big)|_{(x,0,0)} \\
          & = 0. 
\end{align}

The equality in equation \eqref{789} is true since $\kg_P$ and $\kg_Q$  are essentially homogeneous $r$-fibered distributions respectively of order $-d_H$ and $0$ respectively and $\alpha_s$ is a groupoid automorphism for all $s>0$.
The last equality is true by virtue of Lemma  \ref{757} applied fiberwise to the osculating groups which are by definition connected (graded) nilpotent Lie groups. Indeed the convolution in the brackets $[\kg_P,g]$ and $[\kg_Q,f]$ are done fiberwise. In the fiber $(x,0,0)$ the convolution is done between a distribution with compact support and a function on $\mathcal{T}_HM_x$ with compact support. 
\end{proof}


\begin{remark}
Theorem \ref{746} is in particular satisfied for $Q \in \Psi_{H}^{m'}(M)$ with $m' \leq -1$. In this case, we would have $PQ$ and $QP$ in the space $\Psi_{H}^{-d_H-1}(M)$. Therefore, the two $r$-fibered distributions $\kg_P \star \kg_Q$, $\kg_Q \star \kg_P$ would be two continuous functions, thanks to \cite[Theorem 52]{Yuncken2019groupoidapproach}. Therefore the two co-cycles in \eqref{792} would be $0$, as again the Debord-Skandalis action fixes $(x,0,0)$. So it is only the critical case $m'=0$ which remained to be proven.
\end{remark}

%le commutateur $[P,Q] \in \Psi_H^{m-m'}(\THM)|_{t=1}$ et donc



\section{Pseudo-homogeneous functions and kernels}

It is well-known, see for instance \cite{hsiao2008boundaryintegraleq},\cite[p4]{Beals2016Heisenbergcalculus}, that classical pseudodifferential operators admit also a kernel expansion in terms of (pseudo)-homogeneous functions which is precisely linked to the asymptotic expansion of the symbol. Let us recall the details.

 
Consider $\R^d$ equipped with a one-parameter family of dilations $(\delta_s)_{s>0}$. Typically, these will come from a grading on $\R^d$ where $\delta_s$ acts on the subspace of graded degree $k$ by multiplication by $s^k$. We will be particularly interested in the trivial dilation structure:
 \begin{equation} \label{778}
\delta_s(\xi_1,...,\xi_d)=(s\xi_1,...,s\xi_d),
\end{equation}
and shall also encounter the Heisenberg dilation structure:
\begin{equation} \label{784}
\delta_s(\xi_0,\xi_1,...,\xi_d)=(s^2\xi_0,s\xi_1,...,s\xi_d).
\end{equation}

%As in \cite[section 1.1]{couchet2022polyhomo}, we equip $\R^d$ with dilations $(\delta_s)$. In the case where the weights in the dilatations are all one, we say that the dilations are trivial.
%We denote by $\mathcal{H}^m(U \times \R^d)$ the space of trivial homogeneous functions on $U \times \R^d \setminus \{ 0 \}$ of order $m$ on the nose in the second variable. In the case where the dilatations are non trivial, we denote the space of  homogeneous functions on $U \times \R^d \setminus \{ 0 \}$ of order $m$ on the nose in the second variable by $\mathcal{H}_G^m(U \times \R^d)$, like we did in \cite{couchet2022polyhomo}. Note that in this article, we 

%and the trivial ones:
Now we can define:
\begin{definition}{ \cite[Definition 7.1.1 p353 ]{hsiao2008boundaryintegraleq},\cite[Definition (15.19)]{Beals2016Heisenbergcalculus} } \m
Let $U \subset \R^d$ be an open subset. We denote by $\mathcal{H}_G^m(U \times \R^d)$ the space of functions $f \in C^\infty(U \times \R^d \setminus \{ 0 \})$ homogeneous of order $m$ satisfying:
\begin{equation}
f(x,\delta_s(\xi))=s^m f(x,\delta_s(\xi)), ~ s>0,
\end{equation} 
where $\delta_s$ is as in \eqref{778} or \eqref{784}.
We define the set $\Psi \mathrm{hf}_G^m(U \times \R^d)$ of smooth pseudo-homogeneous functions of degree $m$ in the second variable as follows. If $m \notin \N$ then :
\begin{equation}
\Psi \mathrm{hf}_G^m(U \times \R^d)=\mathcal{H}_G^m(U \times \R^d).
\end{equation}
If $m \in \N$ then $\Psi \mathrm{hf}_G^m(U \times \R^d)$ is the set of $k \in C^\infty(U \times \R^d \setminus \{ 0 \})$ of the form:
\begin{equation}
k(x,\xi) = f(x,\xi)+\log(|\xi|)p(x,\xi),
\end{equation}
where $p$ is a homogeneous polynomial in $\xi$ of degree $m$ having $C^\infty$-coefficients in $x$, where the function $f \in \mathcal{H}_G^m(U \times \R^d)$ and $|~|$ is a homogeneous quasi-norm on $\R^d$, in the sense of dilations \eqref{778} or \eqref{784}.
\end{definition}

%\begin{exemple}
%It is clear that for all $m \in \R$, $\mathcal{H}_G^m(U \times \R^d) \subset \Psi \mathrm{hf}_G^m(U \times \R^d)$.
%\end{exemple}
We also recall, see \cite{couchet2022polyhomo}:

\begin{definition} 
Let $U \subset \R^d$ be an open subset. We denote by $\mathcal{H} \mathcal{S}_G^m(U \times \R^d)$ the space of functions $f \in C^\infty(U \times \R^d)$ which are homogeneous of order $m$ modulo Schwartz-class meaning:
\begin{equation}
f(x,\delta_s(\xi)) - s^mf(x,\xi),
\end{equation}
is a Schwartz-class function of $\xi$, with smooth dependance on $x$.
\end{definition}


\begin{definition}{ \cite[Equation (7.1.2) p354 ]{hsiao2008boundaryintegraleq} \cite[Definition 3.5 p414]{ponge2007residueHeisenberg}, \cite[Equations (15.40)-(15.41)]{Beals2016Heisenbergcalculus} } \label{765} \m
A distribution kernel $k \in \mathcal{D}'(U \times U)$ is said to have a pseudo-homogeneous expansion of degree $m \in \R$ if:
\begin{equation} \label{773}
k \sim \sum_{j} k_{m+j},
\end{equation}
where $k_{m+j} \in \Psi \mathrm{hf}_G^{m+j}(U \times \R^d)$ and where the symbol $\sim$ means that for all $N \in \N$, there exists $J_N \in \N$:
\begin{equation}
k - \sum_{j \in J_N} k_{m+j} \in C^N(U \times U),
\end{equation}
where $C^N(U \times U)$ denotes the space of class $C^N$ functions. The space of kernels having a pseudo-homogeneous expansion of degree $m$ is denoted  $\Psi \mathrm{hk}_G^m(U)$.
\end{definition}

\begin{remark}
Our use of $\mathcal{D}'(U \times U)$ in the above definition follows the standard practice of most works on pseudodifferential operators. Stricly speaking, when we realize these as kernels in the groupoid calculus, we will need to replace these by $r$-fibred distributions :
\begin{equation}
k(x,y)dy \in \mathcal{D}_r'(U \times U),
\end{equation}
see \cite{lescure2017convolution}, \cite{Yuncken2019groupoidapproach}. The Lebesgue measure $dy$ is homogeneous of degree $-d$ with respect to the trivial dilation $\delta_s$ so this introduces a degree-shift in orders of kernels, see the next theorem. Thus, the kernel of a pseudodifferential operator on $U$ of order $m<0$ will be given by an element of $\Psi hk_G^{-m-d}(U)$. This is a well-known technical detail, and we will not remark on it further.
\end{remark}

We now state an important theorem, also see \cite[thm 7.1.1, 7.1.6, 7.1.7, 7.1.8]{hsiao2008boundaryintegraleq}.

\begin{theoreme}{Seeley 1969 \cite[theorem 1 p 209]{seeley2011topics}} \m \label{764}
Let $U \subset \R^d$ be an open subset and $m<0$. Then $P \in \boldsymbol{\Psi}_{\text{Hör}}^m(U)$ if and only if:
\begin{equation}
Pu(x)=\int_U k(x,x-y)u(y)dy, ~ u \in C_c^\infty(U),
\end{equation}
with Schwartz kernel satisfying $k \in \Psi \mathrm{hk}_G^{-m-d}(U)$. 
\end{theoreme}

Moreover, in the case of the trivial dilations structure, the asymptotic expansion of the symbol:

\begin{equation}
a \sim \sum_{j=0}^\infty a_{m-j}, ~ a_{m-j} \in \mathcal{H}_G^{m-j}(U \times \R^d),
\end{equation}

and the kernel:

\begin{equation}
k \sim \sum_{j=0}^\infty k_{m+j}, ~ k_{m+j} \in \Psi hf_G^{m+j}(U \times \R^d),
\end{equation}

are related by an adapted Fourier transform as follows see \cite[Equation (7.1.81) p393]{hsiao2008boundaryintegraleq}. Take  $\psi \in C_c^\infty(\R^d)$ is any cut-off function satisfying:
\begin{equation}
\psi(z):=
\left\{ \begin{array}{lr}
1 & \mbox{if} ~ |z| \leq \frac{1}{2}, \\
0 & \mbox{if} ~ |z| >1.
\end{array} 
\right. 
\end{equation}
Set $\kappa=-m-d$. Then for $m-j <0$:
\begin{equation} \label{776}
a_{m-j}(x,\xi)=\lim\limits_{t \to + \infty} \int_{\R^d} k_{\kappa+j}(x,z) \psi(\frac{z}{t})e^{-i \xi.z} dz, ~ x \in U.
\end{equation}

%\subsection{Outline of the strategy}

%\color{red} \textbf{A supprimer car sera mise dans l'introduction} \color{black}



%Now, inspired by \eqref{762}, we shall define, in the case of filtered manifold of homogeneous dimension $d_H$, the Wodzicki residue for operators $P \in \bold{\Psi}_{vEY}^m(M)$, $m \leq -d_H$ by:

%\begin{equation}  \label{775}
%Res_x(P) = \frac{1 }{\log(s)} \Big( s^{d_H} \alpha_{s*} \kg - \kg \Big)|_{(x,0,0)}.
%\end{equation}
%Note that this is licit to define this formula without worried about which $s$ to choose, thanks to the Lemma \ref{759}. In Theorem \ref{746}, we shall show that our Wodzicki residue is still a trace. 





\section{The Wodziciki residue coincides with the groupoidal residue}

We shall prove in this section that the groupoidal residue $Res_x(P)$ and the Wodzicki residue $Res_x^W(P)$ coincide when $P$ is a classical pseudodifferential operator of order $\leq -d$ on a trivially filtered manifold, see Theorem \ref{758}. We begin by recalling exponential coordinates $\Exp$ on $\TM$, see also \cite{Yuncken2019groupoidapproach}. 
\bg
Given a vector field $X$ on $M$ and a point $x \in M$, we write $\exp(X).x$ for the time one flow of $x$ along $X$ if defined.
If $\overline{X}=(X_1,...,X_n)$ is a local frame of vector fields and  $v \in \R^n$ then we set $v.\overline{X}=\sum_{k=1}^n v_kX_k$. Also, note that the dilations $\delta_s$ on $\R^d$ in this case are given by $\delta_s(v)=sv$.  The following Lemma lists the properties of the exponential charts of $\THM$ which we will need in the sequel.
\bg

%The reader shall now recall the \cite[Definition 5.3]{couchet2022polyhomo} of exponential charts $\Exp$.

%\begin{center}
%\color{red} \textbf{Bob, peux tu relire super soigneusement le lemme suivant, je l'ai déjà relu bcp de fois et corrigé aussi mais je veux être sûr qu'il n'y ait pas d'erreur dedans} \color{black}
%\end{center}

\begin{lemma}{\cite[ Lemma 27 p 14]{Yuncken2019groupoidapproach}, \cite[Proposition 5.13]{couchet2022polyhomo}, \cite{couchet2023thesis} } \label{750} \m
Let $M$ be a smooth manifold of dimension $d$ and $\kg \in \boldsymbol{\Psi}_{vEY}^m(\TM)$. Given $x_0 \in M$, $(U_0,\phi)$ a chart on $x_0$ and $\overline{X}=(X_1,...,X_d)$ a local frame on $x_0$, we have: 
\begin{enumerate}
\item There exists an open neighbourhood $U$ of $U_0 \times \{ 0 \}$ with $U \subset U_0 \times \R^d$ such that:
\begin{equation}
Exp^{\overline{X}} : U \rightarrow M \times M, (x,v) \mapsto (x,\exp(v.\overline{X}).x),
\end{equation}
is a diffeomorphism onto its image.
\item The derivative of $Exp^{\overline{X}}$ at $(x_0,0)$ is  : 
\begin{equation}
d_{(x,0)} Exp^{\overline{X}} : (w,v) \in T_xU_0 \times \R^d \mapsto (w, v.\overline{X}|_x).
\end{equation}
\item Put $\tilde{\mathbb{U}}:=\{(x,v,t) \in U_0 \times \R^d \times \R, ~ (x,\delta_t(v)) \in U \}$. Then the map:
\begin{equation}
\Exp : \tilde{\mathbb{U}} \rightarrow \TM, (x,v,t) \mapsto \begin{cases} \Exp(x,v,t)=(Exp^{\overline{X}}(x,\delta_t(-v)),t), &  t \neq 0 \\
\Exp(x,v,0)=(x,v.\overline{X}|_x,0) & t=0,
\end{cases}
\end{equation}
defines  the inverse of a smooth chart for the tangent groupoid $\TM$. 
\item Let  $\mathbb{U}=\Exp(\tilde{\mathbb{U}}) \in \TM$ be the domain of this chart. Then:
\begin{equation}
\mathbb{U}=\Big(TM \times \{ 0 \} \Big) \bigcup \Big(Exp^{\overline{X}}(U) \times \R^* \Big),
\end{equation} 
is an open neighbourhood  of $\Big(TM \times \{ 0 \} \Big) \bigcup \Big(\mathrm{diag}(U_0)\times \R^* \Big)$ in $\TM$, where $\mathrm{diag}(U_0)=\{ (x,x), ~ x \in U_0 \}$. Moreover $\mathbb{U}$ is invariant for the Debord-Skandalis action \ref{799}, and the pullback of this action under $\Exp$:
\begin{equation} \label{772}
\widetilde{\alpha}_{s}:=(\Exp)^{-1} \circ \alpha_s \circ \Exp : \tilde{\mathbb{U}} \rightarrow  \tilde{\mathbb{U}},
\end{equation} 
is given by:
\begin{equation} \label{763}
\widetilde{\alpha}_{s}(x,v,t)=(x,\delta_s(v),s^{-1}t).
\end{equation}

\item There exists $\chi_{\mathbb{U}} \in C_c^\infty(\TM)$ invariant under the Debord-Skandalis action $\alpha_s$ such that:
 \begin{equation} 
\chi_{\mathbb{U}} = \left\{
    \begin{array}{ll}
        1 & \mbox{in a neighbourhood of } \{(x_0,x_0)\} \times \R,  \\
        0 & \mbox{outside } \mathbb{U}.
    \end{array}
\right. 
\end{equation}
%Plus précisément $\mathbb{U}=\Big(TM \times \{ 0 \} \Big) \cup Exp^{\overline{X}}(U) \times \R^*$ est un ouvert de $M \times \R \cup \TM|_{t=0}$.
\item $\kg \chi_{\mathbb{U}} \in \boldsymbol{\Psi}_{vEY}^m(\TM)$ has support in $\mathbb{U}$. Moreover $\kgt=(\Exp)_*^{-1}(\kg \chi_{\mathbb{U}})$ has support in $\tilde{\mathbb{U}}$ and is essentially homogeneous for the action $\tilde{\alpha}_s$.
\end{enumerate} 
\end{lemma}



\begin{lemma} \label{754}
Let $U \subset \R^d$ be an open subset. If $k_0 \in \Psi \mathrm{h f}_G^0(U \times \R^d)$ then $Res_x(P)=Res_x^W(P)$ is verified for the operator $P \in \bold{\Psi}_{\text{Hör}}^{-d}(U)$ with Schwartz kernel $\chi(x-y) k_0(x,x-y)$ where $\chi \in C_c^\infty(\R^d)$ is $1$ in a neighbourhood of 0 and 0 at infinity.
\end{lemma}



\begin{proof}
By definition of  $\Psi \mathrm{h f}_G^0(U \times \R^d)$, we can write $k_0(x,z)=f_0(x,z)+ \log(|z|)p_0(x)$, where $p_0$ is a smooth function on $U$ and $f_0$ is a smooth function homogeneous of order $0$ with respect to $z$. Let $\chi$ be as in the statement. Then we set:
\begin{enumerate}
\item $l(x,y)=\chi(x-y) \ln(|x-y|)p_0(x)dy$ to be the kernel of the operator $P=Op(l)$ whose kernel's asymptotic expansion is given by $l_0(x,z)=\ln(|z|) p_0(x)dz$ and $l_j(x,z)=0$ if $j>0$.
\item $r(x,y)=\chi(x-y) f_0(x,x-y)dy$ to be the kernel of the operator $P=Op(r)$ whose kernel's asymptotic expansion is given by $r_0(x,z)= f_0(x,z)dz$ and $r_j(x,z)=0$ if $j>0$.
\end{enumerate}

We compute the Wodzicki residue at $x \in U$ respectively for each of these operators, using Definition \ref{766}.
In both cases, recall that equation \eqref{776} gives us the link between the asymptotic symbol expansion and the asymptotic kernel expansion.

\begin{enumerate}
\item We get, by denoting $\mathcal{F}_2$ the Fourier transform with respect to the second variable:
\begin{align} 
a_{-d}(x,\xi) & \underbrace{=}_{\eqref{776}} \mathcal{F}_2 \Big(p_0(x) \log(|.|) \Big)(\xi) \\
&= p_0(x) \mathcal{F}_2( \log(|.|))(\xi), \label{777}
\end{align}
where $\mathcal{F}_2( \log(|.|))$ is interpreted as the Fourier transform of the tempered distribution $z \mapsto \log(|z|)$.
%and it is a result from \cite{hsiao2008boundaryintegraleq} that it coincides on $\xi \neq 0$ with the smooth function $a_{-d}(x,\xi)$.
Now, the Fourier transform of the logarithm in $\R^d$ is well known and given, for $\xi \neq 0$ by:
%\begin{equation} \label{756}
%\mathcal{F}_2( \log(|z|))(\xi)=- \frac{\pi^{\frac{d+1}{2}}}{|\xi|^d} \frac{\Gamma(d)}{\Gamma(\frac{d+1}{2})}.
%\end{equation}
%Thanks to the properties of Gamma function it come :
\begin{equation} \label{767}
\mathcal{F}_2( \log(|.|))(\xi)=- \frac{1}{|\xi|^d} \frac{(2 \pi)^d}{\omega_d},
\end{equation}
where $\omega_d=\frac{(2 \pi)^d}{\sqrt{\pi}^d \Gamma(\frac{d}{2}) 2^{d-1}}$ denotes the surface area of the unit $(d-1)$-sphere $\mathbb{S}^{d-1}:=\{\xi \in \R^d, ~ |\xi|=1 \} \subset \R^d$.
Then in \eqref{777} we may now write:
\begin{align}
a_{-d}(x,\xi) & \underbrace{=}_{\eqref{767}} - \frac{p_0(x)}{|\xi|^d} \frac{(2 \pi)^d}{\omega_d}.
\end{align}


Therefore from \eqref{766} we get:
\begin{align}
Res_x^W(Op(l)) & = \frac{1}{(2 \pi)^d} \int_{\mathbb{S}^{d-1}} - \frac{p_0(x)}{|\xi|^d} \frac{(2 \pi)^d}{\omega_d} d \sigma(\xi) dx  = -p_0(x) dx,  \label{786}
\end{align}
where $d \sigma$ denotes the usual surface measure on $\mathbb{S}^{d-1}$.

\item As $f_0 \in \mathcal{H}^0(U \times \R^d)$, we may extend it to $f_0 \in C^\infty(U) \otimes L^\infty(\R^d)$ by attributing any value at $\xi=0$ for all $x \in U$.  The result is a tempered distribution (generalized function) which is homogeneous of degree 0.
%Moreover this extension, still denoted by $f_0$, is a temperate distribution on the $\xi$-fibers, namely $f_0 \in S'(U \times \R^d)=C^\infty(U) \otimes S'(\R^d)$ and it is a homogeneous distribution of degree $0$. Indeed for $x \in U$ fixed and for $\phi \in S(\R^d)$ we have:
%\begin{align}
%\langle \delta_{s*} f_0(x,.),\phi \rangle & = \int_{\R^d} f_0(x,s^{-1} \xi) \phi(\xi) d\xi \\
%& = <f_0(x,.),\phi>.
%\end{align} 

We now proceed with $x \in U$ fixed. 
 Thanks to \cite[Proposition 2.4.7 p 140]{grafakos2008classical}, or \cite[p86]{coifman1978deladespsidos},
there exists  $b_x \in \C$ and $\Omega_x$ a smooth function on the sphere $\mathbb{S}^{d-1}$ with integral 0 on $\mathbb{S}^{d-1}$ such that:
\begin{equation} \label{803}
\mathcal{F}_2(f_0(x,.))(\xi)=b_x \delta_0 + W_{\Omega_x}(\xi),
\end{equation}
where $W_{\Omega_x}$ is the principal value distribution whose restriction to $\R^d \setminus \{ 0 \}$ is:
\begin{equation} \label{782}
\Omega_x \Big(\frac{\xi}{|\xi|} \Big) \frac{1}{|\xi|^d},
\end{equation}

see \cite[Equation (2.4.12) ]{grafakos2008classical}.
%Moreover, thanks to \cite[Proposition 2.4.8]{grafakos2008classical}, we have that 
When $\xi \neq 0$ we have that:
\begin{align}
a_{-d}(x,\xi) & \underbrace{=}_{\eqref{776}}  \mathcal{F}_2(f_0(x,.))(\xi)  \\
& \underbrace{=}_{\eqref{803}} W_{\Omega_x}(\xi), \label{783}
\end{align}
is smooth in $\R^d \setminus \{0 \}$. 
%In reality this is the case also because by definition, $\mathcal{F}_2(f_0(x,.))(\xi)=a_{-d}(x,\xi)$, meaning that \cite[Proposition 2.4.8]{grafakos2008classical} does not need to be appealed to. 
It follows that:
\begin{align}
\int_{\mathbb{S}^{d-1}} a_{-d}(x,\xi) d \sigma(\xi) &  \underbrace{=}_{\eqref{783}} \int_{\mathbb{S}^{d-1}} W_{\Omega_x}(\xi) d \sigma(\xi) \\
& \underbrace{=}_{\eqref{782}} \int_{\mathbb{S}^{d-1}} \Omega_x(\xi) d \sigma(\xi) \\
& = 0,
\end{align}
 %& = \int_{\mathbb{S}^{d-1}} \Omega_x \Big(\frac{\xi}{|\xi|} \Big) \frac{1}{|\xi|^d} d \sigma(\xi) \\
where the last equality is true by the assumption on $\Omega_x$. We therefore have:
  
\begin{equation} \label{785}
Res_x^W(Op(r))=0.
\end{equation}  
  
\end{enumerate} 



Now we look at the co-cycles at $x$ of the $r$-fibred distributions essentially homogeneous associated to the operators $Op(l),Op(r)$ and prove that at $(x,0,0)$ we recover the residue values \eqref{786} and \eqref{785}. First, we can respectively define two elements in $\boldsymbol{\Psi}_{\text{vEY}}^m(\TM)$ such that their restrictions in $t=1$ give the kernels $l$ and $r$.

\begin{enumerate}
\item Set: 
\begin{equation}
 \left\{
    \begin{array}{ll}
        \mathbbm{l}(x,y,t)= \frac{1}{t^d} \chi(\frac{x-y}{t}) \log(\frac{|x-y|}{t}) p_0(x)dy & \mbox{if } t \neq 0 \\
       \mathbbm{l}(x,v,0)=\chi(v)\log(|v|) p_0(x)d \lambda_x(v) & \mbox{if} ~ t=0,
    \end{array}
\right.
\end{equation}
where $d \lambda_x$ denotes the Haar measure on the tangent space $T_xM$ at $x$.
Writing this in exponential coordinates according to Lemma \ref{750} with respect to the standard coordinate frame $\overline{X}$ for $\R^d$, we get:
\begin{equation} 
\tilde{\mathbbm{l}}(x,v,t)=\Big( \mathbb{E} xp^{\overline{X}}\Big)_*^{-1} \mathbbm{l}(x,v,t)=\chi(v)\log(|v|) p_0(x)d \lambda_x(v).
\end{equation}
Recalling $\tilde{\alpha}_s$ from Lemma \ref{750} and using the fact that $\delta_{s*}(d \lambda_x)=s^{-d}d \lambda_x$, we get:
\begin{align}
s^d \tilde{\alpha}_{s*} \tilde{\mathbbm{l}}(x,v,t) -\tilde{\mathbbm{l}}(x,v,t) &=  s^d \tilde{\mathbbm{l}} \Big(x,\delta_{s^{-1}}(v),st \Big) \delta_{s*} (d \lambda_x(v)) - \tilde{\mathbbm{l}}(x,v,t)  \\
& = - \log(s) \chi(s^{-1}v)p_0(x)d \lambda_x(v) \nonumber  \\
&  ~ ~ ~+ \Big( \chi(s^{-1}v) - \chi(v) \Big)\log(|v|) p_0(x) d \lambda_x(v) \label{800}.
\end{align}
We deduce that :
\begin{equation}
s^d \tilde{\alpha}_{s*} \tilde{\mathbbm{l}} - \tilde{\mathbbm{l}} \in C_p^\infty(\tilde{\mathbb{U}},\Omega_r).
\end{equation}
%for all $t \in \R$ and for the exponential chart $\Exp$ in a local frame $\overline{X}$ of $U_0$, where $U_0$ is defined in Lemma \ref{750} \color{red} \textbf{En a-t-on vraiment besoin de ce $U_0$?} \color{black}
\item
Set:
\begin{equation} \left\{
    \begin{array}{ll}
        \mathbbm{r}(x,y,t)=  \chi(\frac{x-y}{t}) f_0(x,\frac{x-y}{t})dy & \mbox{if } t \neq 0 \\
       \mathbbm{r}(x,v,0)=\chi(v)f_0(x,v)d \lambda_x(v) & \mbox{if} ~ t=0,
    \end{array}
\right.
\end{equation}
In the same exponential coordinates we get:
\begin{equation} \label{801}
\tilde{\mathbbm{r}}(x,v,t)=\Big( \mathbb{E} xp^{\overline{X}}\Big)_*^{-1}   \mathbbm{r}(x,v,t)=\chi(v) f_0(x,v)d \lambda_x(v),
\end{equation}
and the homogeneity of $f_0$ gives:
\begin{equation}
s^{d} \tilde{\alpha}_{s*}  \tilde{\mathbbm{r}} -\tilde{\mathbbm{r}} \in C_p^\infty(\tilde{\mathbb{U}},\Omega_r).
\end{equation}
%for all $t \in \R$ and for the exponential chart $\Exp$ in a local frame $\overline{X}$ of $U_0$, where $U_0$ is defined in .
\end{enumerate}
We move now to the computations of the co-cycles restricted in $(x,0,0)$. 

\begin{enumerate}
\item Using \eqref{800} we get, for all $s \in \R_+^* \setminus \{ 1\}$:
\begin{align}
\Big( \tilde{\alpha}_{s*} \tilde{\mathbbm{l}} -s^{-d} \tilde{\mathbbm{l}} \Big)|_{(x,0,0)} & = - \frac{1}{s^d} \log(s) p_0(x) d \lambda_x \\
&=  \frac{1}{s^d} log(s)  Res_x^W(Op(l)),
\end{align}
where we use the canonical identification of the smooth family of 1-densities $d \lambda_x$ with the smooth measure $dx$ on $M$.
Then:
\begin{equation}
 \frac{1}{\log(s)} \Big(s^d \tilde{\alpha}_{s*}\tilde{\mathbbm{l}} -  \tilde{\mathbbm{l}}  \Big)|_{(x,0,0)} = Res_x^W(Op(l)).
\end{equation}
\item Using \eqref{801} we get, for all $s \in \R_+^* \setminus \{ 1\}$:
\begin{align}
\Big( \tilde{\alpha}_{s*} \tilde{\mathbbm{r}} -s^{-d} \tilde{\mathbbm{r}} \Big)|_{(x,0,0)} &=\Big( f_0(x,v) [ \chi(\frac{v}{s}) - \chi(v) ] d \lambda_x(v) \Big)|_{(x,0,0)} \\
&= 0 \\
& =  \frac{1}{s^d} \log(s)  Res_x^W(Op(r)).
\end{align}

%\color{red} \textbf{Oui $f_0(x,0)$ n'est pas défini mais ce n'est pas grave avec les cut-offs $\chi(\frac{v}{s}) - \chi(v)$  non ? } \color{black}
Then: 


\begin{equation}
 \frac{1}{\log(s)} \Big( s^d \tilde{\alpha}_{s*}\tilde{\mathbbm{r}} - \tilde{\mathbbm{r}} \Big)|_{(x,0,0)} = Res_x^W(Op(r)).
\end{equation}

\end{enumerate}
This completes the proof.
\end{proof}

%In the following we denote by $TM$ the tangent bundle of a smooth manifold $M$.

\begin{theoreme} \label{758} \m
Let $M$ be a (trivially) filtered manifold of dimension $d$ and $P \in \boldsymbol{\Psi}_{H}^m(M)$  a classical pseudodifferential operator of order $m$ on $M$, with $m \leq -d$, $m \in \mathbb{Z}$. Let $\kg$ be any essentially homogeneous $r$-fibered distribution of order $-d$ that extends $P$ at $t=1$. Then:
\begin{equation} \label{753} 
 Res_x^W(P) =\frac{1}{\log(s)} \Big( s^d \alpha_{s*} \kg - \kg \Big)|_{(x,0,0)} , ~ \forall ~ s \in \R_+^* \setminus \{ 1\}, ~ \forall ~ x \in M.
\end{equation}
\end{theoreme}


\begin{proof}
%\begin{center}
%\color{red} \textbf{Rédaction Nathan à la main :} \color{black}
%\end{center}

%\color{red} \textbf{cette rédaction est basée sur le le thm 54 et corollaire 55 de l'article préliminaire de vEY} \color{black}

%Si $m \leq -1-d$, alors  $a_{-d}(x,\xi)=0$ pour tout $x \in M$.  Donc, avec la définition \ref{751}, le résidu d'un tel opérateur est donc nul.
%On sait d'après \cite{Yuncken2019groupoidapproach} proposition 42, que l'on peut supposer $\kg$ homogène pile d'ordre $m$ en dehors de $[-1,1]$. On fixe un $x \in M$. Alors avec le lemme \ref{750} pour lequel on gardera les notations, $\kgt=(\Exp)_*^{-1}(\kg \chi_{\mathbb{U}})$ est à support dans $\tilde{\mathbb{U}}$, prolongée par 0 ailleurs et est essentiellement homogène pour l'action $\tilde{\alpha}_s$. Reprenant la preuve du  \cite[théorème 5.17]{couchet2022polyhomo} il vient que $u=\mathcal{F}_2(\kgt) \in \mathcal{H} \mathcal{S}^m(U_0 \times \R^d \times \R)$.  Alors avec le  \cite[théorème 2.1]{couchet2022polyhomo}, il existe $u' \in \mathcal{H}^m(U_0 \times \R^d \times \R)$ et $u'' \in S_G(U_0 \times \R^d \times \R)$ tel qu'en dehors d'un compact $K \subset \R^d$ contenant $0$ on ait :

%\begin{equation}
%u=\chi_K u'+u'',
%\end{equation}
%où $\chi_K$ est une fonction cut-off qui vaut $0$ sur $K$ et 1 en l'infini. Alors il vient :
%\begin{equation} \label{752}
%\kgt=\mathcal{F}_2^{-1}(u)=\mathcal{F}_2^{-1}(\chi_K u')+ \underbrace{\mathcal{F}_2^{-1}(u'')}_{ \in ~ S_G(U_0 \times \R \times \R) ~ \subset ~ C^0(U_0 \times \R \times \R)}.
%\end{equation}
%Il est maintenant facile de voir que $v \mapsto \chi_K(v) u'(.,v,.)$ est intégrable. En effet, avec la preuve du lemme 3.1 p6 de \cite{couchet2022polyhomo}, on peut dominer la précédente fonction par $\xi \mapsto (1+|\xi|)^m$, laquelle, de façon générale, est intégrable sur $\R^d$ si et seulement si $m < - d$. Ceci étant le cas ici puisqu'on a supposé $m \leq d-1$, donc $\chi_{V} u'(x,.,t) \in L^1(\R)$ pour tous $(x,t) \in \R \times \R$. 
%D'une part, la transformée de Fourier $\mathcal{F}_2 : L^1(U \times \R^d) \rightarrow C_0^0(U \times \R^d)$ et d'autre part $\kgt$ est une application $r$-fibrée (car $\kg$ l'est par hypothèse) donc, de l'équation \eqref{752} on peut affirmer en fait que :
%\begin{equation} 
%\kgt \in C^\infty(U_0 \times \R,C_c^0(\R^d,|\Omega^1|)).
%\end{equation}
Consider first the case  $m \leq -d-1$.
Thanks to  \cite[Theorem 52]{Yuncken2019groupoidapproach} we already know that:
\begin{equation} \label{795}
\kg \in C^0(\TM,\Omega_r),
\end{equation}
as we have supposed here $m \leq -d-1$. That means:
\begin{equation}
\kg(x,v,0)=\mathbbm{l}_0(x,v) d \lambda_x,
\end{equation}
for some $\mathbbm{l}_0 \in C^0(TM)$ and where $d \lambda_x$ is the Haar measure on $T_xM$.

%\begin{center}
%\color{red} \textbf{Rédaction Bob :} \color{black}
%\end{center}
% Then $\kg_0:=\kg|_{t=0} \in \mathcal{E}_r'(TM)$ satisfies the following condition :

%\begin{equation}
%s^{-d} \delta_{s*} \kg_0 - \kg_0 \in C_p^\infty(TM),
%\end{equation}
%where $(\delta_s)$ is acting in the second variable in $(x,\xi,0) \in TM$. Thus its fiberwise Fourier transform on the second variable $\mathcal{F}_2(\kg_0)$ belongs to  $\mathcal{H} \mathcal{S}_G^m(TM)$. Since $m< -d$, it follows \color{red} \textbf{Oui mais c'est bien à cause du lemme de Taylor sur les fibrés, voir lemme 52 article préliminaire Bob } \color{black}  that $\mathcal{F}_2(\kg_0)$ is fiberwise $L^1$, so $\kg_0$ is continuous. 

%By Taylor's lemma on bundle, $\mathcal{F}(kg_0)$ is fiberwise $L^1$ so $\kg_0$ is continuous.



%We set $\kg_0:=\kg|_{t=0} \in \mathcal{E}_r'(TM)$ \color{red} \textbf{Bob : mais avec \eqref{795} on sait qqch de plus fort non ? on sait que $\kg_0=f d\lambda$, où $f \in C^0(TM)$, $d \lambda$ est un mesure (mais sur quel espace ? c'est un système de Haar sur le groupoïde $TM$ ?) } \color{black}.
%Ainsi $\kgt_0= \tilde{\mathbbm{l}}_0 dv$, où  $ \tilde{\mathbbm{l}}_0 \in  C^0(U_0 \times \R^d)$ et $dv$ est la mesure de Lebesgue sur $\R^d$, où l'indice $0$ désigne la restriction en $t=0$.

We can now evaluate the co-cycle term to term. Moreover the points $(x,0,0)$ are fixed by the Debord-Skandalis action. 
%On peut donc évaluer le co-cycle terme à terme car $\kgt$ est maintenant fonction et les points $(x,0,0)$ sont fixés par l'action de Debord-Skandalis et de même pour l'action $\tilde{\alpha_{s}}$. 
Thus, using the facts that $\delta_{s*} d \lambda_x=s^{-d} d \lambda_x$,
% and $\tilde{\alpha}_{s}^{-1*} (\tilde{\mathbbm{l}})|_{(x,0,0)} =(\delta_s^{-1})^*\tilde{\mathbbm{l}}_0=\tilde{\mathbbm{l}}_0$, 
we get:
\begin{align}
\frac{1}{\log(s)} \Big(s^d \tilde{\alpha}_{s*} \kg - \kg \Big)|_{(x,0,0)} & = \frac{1}{\log(s)} \Big(  \mathbbm{l}_0(x,0) d \lambda_x - \mathbbm{l}_0(x,0) d \lambda_x \Big) = 0.
\end{align}
This agrees with the Wodzicki residue in this case. Indeed, for an operator of this order the term $a_{-d}$ appearing in the asymptotic expansion of its symbol is always zero.
\bg
If $m=-d$ and $(U_0,\phi)$ is a chart on $x \in M$,
then the kernel admits an asymptotic expansion  $k \sim \sum_{j=0}^{+ \infty} k_{j}$ in $U_0$, thanks to Seeley's Theorem \ref{764}. Thus we may write:
\begin{equation} \label{802}
k(x,x-y) - \chi(x-y) k_0(x,x-y) \sim \sum_{j \geq 1} k_j(x,x-y),
\end{equation}
with $\chi \in C_c^\infty(\R^d)$ is such that $\chi$ is equal to 1 in a neighbourhood of $0$ and $0$ at infinity. Let us denote the left hand side of \eqref{802} by $\overline{k}(x,x-y)$. Then $\overline{P}=Op(\overline{k}) \in \bold{\Psi}_{\text{Hör}}^{m-1}(M)$ and using what we did just before, $Res_x^W(\overline{P})=Res_x(\overline{P})$ for all $x \in M$.  It suffices to apply Lemma \ref{754} to the function  $k_0 \in \Psi \mathrm{hf}_G^0$ to conclude, as we have $k(x,z)=\overline{k}(x,z)+\chi(z)k_0(x,z).$
\end{proof}



\section{Ponge's non commutative residue for an Heisenberg manifold.}

In his article \cite{ponge2007residueHeisenberg}, Ponge defined a noncommutative residue that fits the context of a Heisenberg manifold.
In this section, we will show that Ponge's residue coincides with the groupoidal residue of Definition \ref{726} for pseudodifferential operators of order $\leq -d_H$ on a contact manifold or a codimensional one foliation. Again, we will restrict our attention to scalar-valued operators to simplify notation, although one can easily generalise to vector bundles using \eqref{798}.
\bg
Let $M$ be a Heisenberg manifold of dimension $d+1$ with hyperplane bundle $\V \leq TM$. The algebra of Heisenberg pseudodifferential operators of BG \cite{Beals2016Heisenbergcalculus} is denoted $\bold{\Psi}_\V^{\bullet}(M)$. 
It is shown in \cite{couchet2022polyhomo} that this coincides with the groupoidal calculus when $M=\Hn \times \R^m$, $\Hn$ being the $2n+1$ dimensional Heisenberg group, or $M$ is a contact manifold or a codimensional one foliation. That is:
\begin{equation}
\bold{\Psi}_\V^{m}(M)=\bold{\Psi}_{H}^{m}(M),
\end{equation}
for $M$ in these cases, though we expect it to be true also for a general Heisenberg manifold.
\bg 
Ponge's noncommutative residue is defined as follows.
Let $\Big( X_j \Big)_{j \in \{ 0,...,d \}}$ be a local $H$-frame of vector fields on an open subset $U \subset M$ and $\Psi_x : U \rightarrow \R^{d+1}$ be a privileged change of coordinates centered at $x$, see \cite[p 415 and Definitions 2.3 ,2.4]{ponge2007residueHeisenberg}. The latter assertion means that if $\Big( X_j \Big)_{j \in \{0,...,d \}}$ is a local $H$-frame of vector fields, then we have $\Psi_x(x)=0$ and $(\Psi_x)_*X_j(x)=\partial_j|_x$. In this context, the noncommutative residue of Ponge of a Heisenberg pseudodifferential operator $P \in \bold{\Psi}_{\V}^{-d_H}(U)$ of degree $-d_H$ on $U$, is defined as follows.
\bg
Let $p_{-d-2}$ be the term of degree $-d_H=-(d+2)$ from the asymptotic expansion of the symbol of $P \in \bold{\Psi}_{\V}^{-d_H}(U)$, see \cite[2.7 p409]{ponge2007residueHeisenberg}. Then set the noncommutative residue of $P$ at $x$: 
\begin{equation}
c_P(x)=\frac{|d \Psi_x|}{(2 \pi)^{d+1}} \int_{\mathbb{S}^{d}} p_{-(d+2)}(x,\xi) d\xi,
\end{equation}
where $|d \Psi_x|$ is the jacobian of $\Psi_x$, see \cite[Lemma 3.9]{ponge2007residueHeisenberg}.



The reader can compare this definition in contrast with the "non-graded" non commutative residue \eqref{766}.
% defined by Kassel \cite{kassel1989residu} or \cite{vassout2001feuilletages}.
We begin with the case of the model groups $M=\Hn \times \R^m=\R^{d+1}$, where $d=2n+m$. If $n=0$ then $\mathbb{H}_0=\R$ by convention. We equip $M$ with the model vector fields $\overline{X}=(X_0,X_1,...,X_d)$ of \cite[chapter 1 p12-13]{Beals2016Heisenbergcalculus}, also see \cite[section 5.1]{couchet2022polyhomo}, so that $(X_0,...,X_{2n})$ generate $\Hn$, $X_0$ is central and $(X_{2n+1},...,X_d)$ are the usual vectors fields on $\R^{d+1}$. Recall that $\mathcal{E}_r'(M \times M)$ denotes the set of $r$-fibered distributions on the pair groupoid $M \times M$.

\begin{theoreme} \label{774}
Given a model manifold $M=\Hn \times \R^m$ of homogeneous dimension $d_H=d+2$, $2n+m=d$, with the standard model structure as in \cite{couchet2022polyhomo} and $P \in \bold{\Psi}_{\V}^{m}(M)$, with $m \leq -d_H$, then the residue of Ponge at $x$ and our residue from Definition \ref{726} coincide:
\begin{equation} \label{781}
Res_x(P)=c_P(x).
\end{equation}
\end{theoreme}

\begin{proof}
We will rely heavily on \cite{Beals2016Heisenbergcalculus} and \cite{couchet2022polyhomo}. We denote $k \in \mathcal{E}_r'(M \times M)$ the kernel of $P$. Since the Heisenberg and groupoidal calculi coincide, see \cite[Theorem 5.16]{couchet2022polyhomo}, there exists $\kg \in \bold{\Psi}_{vEY}^{-d_H}(\THM)$ such that $\kg|_{t=1}=k$. Moreover we may suppose, see \cite[proposition 42]{Yuncken2019groupoidapproach} that $\kg$ is homogeneous on the nose of order $-d_H$ outside $t \in [-1,1]$. 
We pull this back via exponential coordinates, again as in \cite[section 5]{couchet2022polyhomo}, yielding $\kgt \in \mathcal{E}_r'(M \times \R^{d+1} \times \R)$ with the equality $\kg=\Exp_* \Big( \kgt \Big)$. Note that in the case of model manifold, the exponential coordinate chart is globally define.
\bg
Next we must consider the symbol of $P$. By the definition of the Heisenberg calculus, \cite[Chapter 3, § 10]{Beals2016Heisenbergcalculus}, $P$ is defined starting from a graded-polyhomogeneous function $f \in S_{phg,G}^{-d_H}(M \times \R^{d+1})$. Letting $\overline{\sigma}(x,\xi)=(x,\sigma_0(x,\xi),...,\sigma_d(x,\xi))$ be the coordinate transform \cite[Equations (10.14),(10.15)]{Beals2016Heisenbergcalculus} obtained from the symbols of the model vector fields $(X_0,...,X_d)$, Beals and Greiner define the $\V$-symbol associated to $f$ by:
\begin{equation}
q(x,\xi)=\overline{\sigma}^*f(x,\xi).
\end{equation}
See also \cite[Definition 5.5]{couchet2022polyhomo}.
The symbol and kernel are related by fiberwise Fourier transform, after the abovementioned coordinate changes. Explicitly, we have:
\begin{equation}
f=\mathcal{F}_2(\kgt)|_{t=1},
\end{equation}
where $\mathcal{F}_2$ is the fiberwise Fourier transform in the second variable. Extending this fiberwise Fourier transform to all $t \in \R$, let us put:
\begin{equation}
u=\mathcal{F}_2(\kgt) \in C^\infty(M \times \R^{d+1} \times \R).
\end{equation}
Note that the Debord-Skandalis action transforms under the Fourier transform as:
\begin{equation}
\mathcal{F}_2 \circ \tilde{\alpha}_{s*} \circ \mathcal{F}_2^{-1}=\beta_s^*,
\end{equation}
where $\beta_s : M \times \R^{d+1} \times \R \rightarrow M \times \R^{d+1} \times \R$ are the dilations:
\begin{equation}
\beta_s(x,v,t)=(x,\delta_s(v),st),
\end{equation}
see \cite[Proposition .13]{couchet2022polyhomo}. The essential homogeneity of $\kg$ and consequently of $\kgt=(\Exp_*)^{-1}(\kg)$, therefore implies that $u \in \mathcal{H}\mathcal{S}_G^m(M \times \R^{d+1} \times \R)$ where the homogeneity modulo Schwartz is with respect to the dilations $\beta_s$ (thanks to the hypothesis on the homogeneity of $\kg$ outside $t \in [-1,1]$), see the proof of \cite[Theorem 5.16]{couchet2022polyhomo}. 
\bg
Set $u_0=u|_{t=0}$. By \cite[Proposition 3.2]{couchet2022polyhomo} we have $u_0 \in \mathcal{H} \mathcal{S}_G^{-d_H}(M \times \R^{d+1})$. Therefore by a well-known Lemma, eg \cite[Theorem 2.1]{couchet2022polyhomo}, outside a compact neighbourhood containing $0$ of $\xi$ we may write:
\begin{equation}
u_0=u_0'+u_0'',
\end{equation}
where $u_0' \in \mathcal{H}_G^{-d_H}(M \times \R^{d+1})$ and $u_0'' \in S_G(M \times \R^{d+1})$. 
\bg
We may extend the homogeneous function $u_0'$ as a tempered distribution (not necessarily homogeneous), still denoted $u_0'$. Indeed, we may first extend $u_0'$ to a distribution such as in \cite[section § 15]{Beals2016Heisenbergcalculus}, \cite[Lemma 3.1]{ponge2008heisenberg}, or in \cite[Theorem 3.2.4]{hormander1983analysis},
% Moreover, this extension, still denoted $u_0'$ is a tempered distribution (not compulsory homogeneous),
and it is tempered because it has polynomial growth at infinity. 
% \cite[§ section 32 and theorem p 154]{donoghue1969distributions}. 
Therefore we can find $u_0''' \in C^\infty(M,\mathcal{E}'(\R^{d+1}))=C^\infty(M) \otimes \mathcal{E}'(\R^{d+1})$ such that the following holds everywhere:
\begin{equation} \label{769}
u_0=u_0'+u_0''+u_0'''.
\end{equation}
We can compute for all $s \in \R_+^* \setminus \{ 1 \}$:
\begin{align}
Res_x(P)&= \frac{1 }{\log(s)} \Big( s^{d_H} \alpha_{s*} \kg - \kg \Big)|_{(x,0,0)} \\
& =\frac{1 }{\log(s)} \Big( s^{d_H} \alpha_{s*} \Exp_* \Big( \kgt \Big) - \Exp_* \Big( \kgt \Big) \Big)|_{(x,0,0)} \label{771} \\ 
&=\frac{1 }{\log(s)} \Big( s^{d_H} \tilde{\alpha}_{s*} \kgt - \kgt \Big) \circ \Big( \Exp \Big)^{-1} |_{(x,0,0)}, \label{768}
\end{align}
where we used in \eqref{771} the equality $\tilde{\alpha}_s=\Big( \Exp \Big)^{-1} \circ \alpha_s \circ \Exp$, 
see \eqref{772}.
Also recall that $\Exp|_{M \times \{ 0 \} \times \{ 0 \}}=id_{M \times \{ 0 \} \times \{ 0 \}}$.
We next continue to compute in \eqref{768}:
\begin{align}
Res_x(P)&=\frac{1 }{\log(s)} \Big( s^{d_H} \tilde{\alpha}_{s*} \mathcal{F}_2^{-1}(u) - \mathcal{F}_2^{-1}(u) \Big)|_{(x,0,0)} \\
&= \frac{1 }{\log(s)} \mathcal{F}_2^{-1} \Big(s^{d_H} \beta_s^*u -u \Big)|_{(x,0,0)},
\end{align}
where we use the equality $\mathcal{F}_2 \circ \tilde{\alpha}_{s*} \mathcal{F}_2^{-1}=\beta_s^*$ recalled earlier. Now we use \eqref{769}:
\begin{align}
Res_x(P)
&= \frac{1 }{\log(s)} \mathcal{F}_2^{-1} \Big(s^{d_H} \beta_s^*(u_0'+u_0''+u_0''') -(u_0'+u_0''+u_0''') \Big)|_{(x,0,0)} \\
&= \frac{1 }{\log(s)} \mathcal{F}_2^{-1} \Big(s^{d_H} \beta_s^*u_0' -u_0'+ s^{d_H} \beta_s^*(u_0''+u_0''') -(u_0''+u_0''')   \Big)|_{(x,0,0)}.
\end{align}
%The $\beta_s^*$ applies to functions so here it should be interpreted as $(\beta_{s^{-1}})_*$ \color{red} \textbf{Bob are you Ok ?} \color{black}

Since $u_0''$ is Schwartz class in $\xi$ and $u_0'''$ is compactly supported in $\xi$, their fiberwise Fourier transforms are smooth. 
%If $u \in S(\R^{d+1}) \hookrightarrow S'(\R^{d+1})$ or $u \in \mathcal{E}'(\R^{d+1})$ then using the definition of the Fourier transform of a distribution and \cite[Theorem 8.4.1]{friedlander1998introduction} it comes when evaluating in $\xi=0$:
%\begin{equation}
%\langle s^{d_H} \beta_s^* u,1 \lambda \rangle= \langle u,1 \lambda \rangle,
%\end{equation}
%where $\lambda$ is the Lebesgue measure.


Therefore $\mathcal{F}_2^{-1}(u_0'')$ and $\mathcal{F}_2^{-1}(u_0''')$, %understood as the fiberwise Fourier transform of Schwartz and compactly distributions,  
can be evaluated at $(x,0,0)$ and as in the proof of Theorem \ref{758}, for all $s \in \R_+^* \setminus \{ 1 \}$, we get:
\begin{equation}
 \frac{1 }{\log(s)} \mathcal{F}_2^{-1} \Big(s^{d_H} \beta_s^*(u_0''+u_0''') -(u_0''+u_0''')   \Big)|_{(x,0,0)}=0.
\end{equation}
Hence:
\begin{equation} \label{770}
Res_x(P) = \frac{1 }{\log(s)} \mathcal{F}_2^{-1} \Big(s^{d_H} \beta_s^*u_0' -u_0' \Big)|_{(x,0,0)}. 
\end{equation}

Finally, we use \cite[Lemma 3.1 p 414 and Equation (3.2)]{ponge2007residueHeisenberg} which assert that we have for all  $s \in \R_+^* \setminus \{ 1 \}$:
\begin{equation} \label{807}
\beta_s^* u_0'=s^{-d_H}u'_0+s^{-d_H} \log(s) c_0(u_0') \delta_0,
\end{equation}
where:
\begin{equation}
c_0(u_0')=\int_{|\xi|=1} u_0'(x,\xi) d \sigma(\xi).
\end{equation}

Then, using \eqref{807}, Equation \eqref{770} becomes:
\begin{align}
Res_x(P)&= \frac{1 }{\log(s)} \mathcal{F}_2^{-1} \Big(\log(s) c_0(u_0') \delta_0 \Big)|_{(x,0,0)} \\
& =\frac{1}{(2 \pi)^{d+1}}  \int_{|\xi|=1} u_0'(x,\xi) d \sigma(\xi),
\end{align}
where the constant $\frac{1}{(2 \pi)^{d+1}}$ appears in the inverse Fourier transform formula. 
Now, thanks to \cite[Equation (3.25) p 27]{Beals2016Heisenbergcalculus} we see that:
\begin{equation}
^t \Big( d \Psi_x  \Big) \xi=\sigma(x,\xi),
\end{equation}
where $\sigma$ is defined in \cite[Equation (5.12)]{couchet2022polyhomo}. The reader can compute the Jacobian of $\Psi_x$ and see that it is triangular and unipotent and so has determinant equal to one, see also \cite{couchet2023thesis}.
 
%Now it remains to show that :
According to \cite[the proof p8 of Theorem 1.12]{couchet2022polyhomo}, the purely homogeneous component $u_0'$ of $u_0$ is the first term in the asymptotic expansion of the polyhomogeneous function $f=u|_{t=1}$, see \cite[Equation (3.16)]{couchet2022polyhomo} and the remarks which follow. Therefore:
\begin{equation}
u_0'(x,\xi)=p_{-(d+2)}(x,\xi),
\end{equation}
where $p_{-(d+2)}$ is the term of degree $-(d+2)$ in the asymptotic expansion of $f=u|_{t=1}$. 
%One can see from the proof of \cite[Theorem 1.12 (see the proof p8)]{couchet2022polyhomo} that this is the case. 
We have shown that :
\begin{equation}
Res_x(P)=\frac{|d \Psi_x|}{(2 \pi)^{d+1}}  \int_{|\xi|=1} p_{-(d+2)}(x,\xi) d \sigma(\xi).
\end{equation}
\end{proof}


\begin{corollary} \label{779}
If $M$ is a contact manifold or a codimensional one foliation, then the groupoidal residue of Definition \ref{726} agree with's Ponge noncommutative residue for operators $P \in \bold{\Psi}_\V^{-d_H}(M)$, meaning that \eqref{781} still holds.
\end{corollary}

\begin{proof}
Darboux' Theorem for a contact manifold or Frobenius' Theorem for a codimensional one foliation, implies that around any $x \in M$, there is a local coordinate system which identifies $M$ with the model 
space $\Hn$, $d=2n$, or $\mathbb{H}_0 \times \R^d$. Since both Ponge's residue and the groupoidal residue are independent of (priveleged) coordinates, the result follows.
\end{proof}




\bibliographystyle{plain}

%rafraîchir la bibliothèque Editer -> rafraîchir la bibliothèque
%compiler une fois en Pdf Latex
%Compiler en bibtex
%recompiler en Pdf Latex
%compiler en compilation rapide

\bibliography{bibarticle}





\end{document}
%%% Local Variables:
%%% mode: latex
%%% TeX-master: t
%%% End:
