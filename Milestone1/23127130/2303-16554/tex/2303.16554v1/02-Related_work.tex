\section{Related work} \label{sec:related_work}

\subsection{Unconvetional Visual Communication}

%% VLC
Visible light communication (VLC) is the transmission of digital data using photons in the visible light spectrum as medium (\SIrange{380}{700}{\nano\meter} wavelength). 
Compared to the radio-frequency spectrum, the light spectrum is less congested and harder to eavesdrop from a distance, especially when used in enclosed environments.
Conversely, long-range links are difficult to implement, and require line-of-sight; moreover outdoor applications must account for other illuminants, such as sunlight.
Implementations that use dedicated hardware typically rely on LEDs for the transmitter, and photo-diodes for the receiver; these implementations modulate light at very high frequencies, can achieve high data rates~\cite{4758667, 5200518, 5875646}, and can be also used for indoor localization~\cite{4463523}.  

%% In robotics
Various applications in robotics take advantage of VLC, e.g. for disaster site monitoring~\cite{9700576,9780193}, agriculture~\cite{soil} and covert communications~\cite{9199639}.  Applications to existing hardware often repurpose off-the-shelf cameras on the receiving end; using a camera allows a receiver to deal with multiple transmitters simultaneously, by identifying each source of light in the image space~\cite{9700576,9780193,trajectories}. By using high-speed cameras and optics with very narrow field of view, long-range line-of-sight links have been demonstrated~\cite{9700576}.

%% Rolling shutter
%While limited camera frame rates severely constrain data throughput, a recent line of research~\cite{6477759} mitigates this problem by exploiting the rolling-shutter method implemented by the majority of inexpensive off-the-shelf CMOS cameras; in these cameras, each line of the sensor is exposed at a slightly different time period; then, an illuminant that affects most of the image and is modulated faster than the camera frame rate, will yield detectable horizontal bands that can be detected and decoded~\cite{Hsu:20}.  However, this approach is not suitable for applications such as ours, in which the light source only affects a few pixels of the image.

%% ML applications
In the context of VLC, machine learning techniques are used on the receiving end to implement signal demodulation~\cite{9352471}, to recover the modulated signal from rolling-shutter images~\cite{6477759,Hsu:20}, or to find locations of transmitters in images, before analyzing those regions of interest separately~\cite{9700576,9780193,trajectories}.  In contrast, our approach feeds raw images directly to a CNN in order to directly extract binary signal information (LEDs on or off), independently of the relative location of the drone that is transmitting the signal.


\subsection{MCUs with limited Security}

Although onboard $\mu$CUs are sufficient for flight controller only functionality, 
for advanced autonomous flight capabilities and security, industry tends to prefer using more high-end platforms 
like NVIDIA Jetson TX2, Intel Aero, ODROID XU4, etc.~\cite{Iqbal2021ccwc}. One such example of a drone is 
Skydio 2~\cite{Skydio2}, which has a size of 223×273 mm with a built-in Jetson TX2 module.
Similarly, Qualcomm recently released Qualcomm Flight RB5 for high-end AI and security capabilities~\cite{QualcommRB5}.

