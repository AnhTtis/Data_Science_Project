\section{Security Use Case} \label{sec:sec_usecase}

Our use case envisions multiple nano-drones cooperatively operating and exchanging periodic data (e.g., mission commands, etc.) via radio (e.g., WiFi, BTLE, etc.).
In this scenario, we address the following two \textit{threat models}.

\begin{figure}[t]
  \includegraphics[width=\columnwidth]{images/roc-curve.png}
  \caption{Performance on the testing set. \textbf{A)} ROC Curves for full-precision and quantized models. \textbf{B)} Zoom-in of A. \textbf{C)} Confusion matrix.}
  \label{fig:roc-curve}
\end{figure}

\textbf{Man-In-The-Middle.}
A man-in-the-middle attack enables the attacker to intercept the communication and exchange malicious data with the drones.
Following the zero trust policy~\cite{rose2020zero}, where we always authenticate and never trust, the drones periodically check that the received data are original and transmitted by an authenticated fleet member.
If the authenticity cannot be verified, OpenTitan assumes that both the communication radio channel and the rest of the SoC are potentially compromised.
Therefore, to notify the rest of the fleet about this situation, it enables the UVC blinking procedure by driving its secure GPIOs (exclusively connected to the secure subsystem) to transmit an informative SOS message.
Other fleet drones -- in line-of-sight with the transmitter one -- can simultaneously or alternatively monitor peers' activity, distributing and time-interleaving the computational overhead for the message decoding.

\textbf{Anomaly/Intrusion Detection.}
For this use case, we assume that there is a minimal anomaly/intrusion detection mechanism~\cite{Galvan2021anomaly,lunardi2022arcade} running on the Ibex core of OpenTitan, which is a secure region by construction.
Since OpenTitan is the master of the TLUL-to-AXI interface on the host AXI-4 crossbar, it can monitor the activity of sensors (e.g., accelerometer, cameras, etc.).
Then, if OpenTitan detects an anomaly, it can assume that the host domain, including the communication links, is compromised and triggers the UVC procedure.
The specific implementation of a detection mechanism is out of the scope of this paper.
Instead, we focus our work on the OpenTitan integration/isolation from the rest of the SoC and the implementation of the attack response method, i.e., the UVC exchanging SOS messages by demonstrating it with two Crazyflie nano-drones equipped with the AI-deck.