\section{System Architecture} \label{sec:system}

This section presents the SoC architecture in Figure~\ref{fig:soc}.
It consists of a heterogeneous system architecture composed of a 64-bit application processor implementing flight control functions as well as auxiliary functions such as network stack, a parallel programmable accelerator for mission control functions, and a secure enclave based on OpenTitan IPs.

The SoC is built around the CVA6 core, a 6-stages, single-issue, in-order, 64-bit RISC-V core supporting the RV64GC ISA variant, virtual memory, three execution privilege levels, physical memory protection (PMP), and is capable of booting the Linux OS.
CVA6 has \SI{16}{\kilo\byte} of L1 I-cache and \SI{32}{\kilo\byte} of write-through L1 D-cache, which enable simple coherency with other masters to the crossbar interconnect, which implements high-bandwidth, low-latency 64-bit AXI4 protocol.
The \textit{host domain} contains a scratchpad memory (L2SPM) and a complete set of peripherals such as I2C, (Q)SPI, CPI, SDIO, UART, CAN, PWM, I2S.
Moreover, the host embeds also a standard Platform Level Interrupt Controller (PLIC), a Core Local Interrupt (CLINT), a controller for Cypress Semiconductor's external HyperRAM memories, and a Last Level Cache (LLC) to filter accesses to the external HyperRAM memory improving system performance.
Peripheral data is transferred from/to the scratchpad memory through a dedicated DMA, called $\mu$DMA.

The Programmable Multi-Core Accelerator (PMCA) of the system is built around 8 CV32E-based processors which share 16×\SI{8}{\kilo\byte} SRAM banks (\SI{128}{\kilo\byte} L1SPM).
The cores implement RV32 extension with many ML and DSP features such as hardware loops, MAC\&Load operation, SIMD operations, and post-increment LD/ST.
With SIMD, the operands’ width can be reduced to double or quadruple the number of operations per cycle.
The cluster also implements FPUs supporting FP32 and FP16 with SIMD support and features a two-level I-cache (\SI{512}{\byte} for each core and \SI{4}{\kilo\byte} shared) to speed-up execution of data-parallel tasks typical of drone mission control functions and deep neural networks for objects and pattern recognition.
The architecture of the cluster is optimized for ML algorithms in embedded applications: it exploits scratchpad memories with DMA access, double buffering and custom ISA extensions to optimize memory utilization and computation.

The third key component of the drone navigation SoC is the secure enclave based on the OpenTitan architecture, acting as an on-chip Root of Trust (RoT), providing security services.
The Ibex core is the main processor and is in charge to orchestrate the secure boot and all the RoT functionalities.
Ibex controls four main components.
The AON domain includes power, reset, clock management and a timer.
The secure memories module includes One Time Programmable (OTP) memories which store security keys and seeds.
The crypto module includes specialized accelerators such as Advanced Encryption Standard (AES), Hashing (HMAC and KMAC) and Big Number Accelerator (OTBN) for Rivest–Shamir–Adleman (RSA) and Elliptic Curve Cryptography (ECC).
The security module includes key manager, life cycle controller and alert handler.

The host domain requires to exploit the hardware cryptography accelerators of the secure subsystem but it must not access directly its internal memory map for security reasons.
Instead, it can only encode commands writing into a dedicated mailbox compliant with ARM System Control and Management Interface (SCMI).
The host domain populates the shared memory of the mailbox and then it raises an interrupt to the secure subsystem's core through a dedicated memory mapped register.
The Ibex core reads the content of the mailbox and executes the command encoded in it.
At the end of the execution, the Ibex core raises back another interrupt to the host domain's core.
Moreover, OpenTitan has its own internal timer that can trigger periodic interrupts.
In this way the Ibex core can be periodically waken up in order to perform anomaly detection checks by analyzing the content of CVA6 and cluster's memories as well as external peripherals.