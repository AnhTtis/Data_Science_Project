\section{Introduction} \label{sec:intro}

Autonomous Micro Aerial Vehicles (MAVs) are progressively gaining importance thanks to their ubiquitous sensing capabilities.
In the Internet of Things (IoT) ecosystem, nano-drones, i.e., palm-sized MAVs, can acquire and process information from different locations by flying where their presence is more important~\cite{lakshman2021integration}.
Therefore, they can exchange crucial data with fixed infrastructure or other drones, i.e., swarm operations.
Their miniaturized form factor enables a wide range of applicability, for example, in narrow spaces~\cite{quadrotor_nanodrone} and human surroundings~\cite{blimp1_nanodrone}, but it limits the class of processors they can host aboard.
This \textit{i}) lower-bounds the computational/memory complexity of the algorithms that can run aboard and \textit{ii}) forces the main drone's mission computer to simple microcontroller units (MCUs) that lack advanced cyber-security features.

In this emerging new era of connected and collaborating IoT devices/nano-drones, reliable security and privacy mechanisms are needed to protect assets and data collected or generated~\cite{hwang2015iot}.
The security cornerstone of IoT devices is the Root of Trust (RoT), where critical assets are kept isolated and protected, the code executed is authenticated, and its integrity is verified~\cite{RoT}.
Most modern IoT devices rely on hardware to ensure their RoT and therefore build the whole security stack on top of it, following the \textit{chain of trust} principle~\cite{RoT}.
Despite RoTs provide a solid hardware/software security foundation, there are several types of attacks potentially compromising the drones' operations, such as man-in-the-middle, denial of service, spoofing, jamming, rogue data injection, routing attack, etc.~\cite{Yahuza2021iod}.

Current Commercial Off-The-Shelf (COTS) nano-drones platforms, such as the Bitcraze Crazyflie typically host low-power 32-bit MCUs such as the STM32F4 as main mission computer~\cite{quadrotor_nanodrone}.
This class of MCUs provides sufficient computing power to guarantee basic functionalities such as low-level control loops, state estimation, and cryptographic encoding.
Although they lack both a security enclave and RoT; therefore, they can not guarantee hardware isolation of code execution or support full-fledged operating systems capable of software isolation of different parts of the applications running on them.
Similarly, more computationally-capable SoCs for nano-drones, such as the GWT GAP8 processors~\cite{GAP8} available as a companion board for the Crazyflie nano-drone, i.e., AI-deck, still lacks RoT and security enclave able to take control of the whole system in case of attacks.
In this work, we present an open-source SoC design of a mission computer for autonomous nano-drones, which includes silicon secure enclave and RoT by integrating the OpenTitan reference design~\footnote{\url{https://opentitan.org/}}. 

\begin{figure*}[t!]
\centerline{\includegraphics[width=\textwidth]{images/he-soc-drone-horizontal.png}}
\caption{A) The Bitcraze Crazyflie equipped with the GWT GAP8 SoC. B) The proposed SoC architecture envisioned as alternative MCU aboard the nano-drone.}
\label{fig:soc}
\end{figure*}

The SoC is built around a 64-bit RISC-V CVA6 core featuring full Linux support and an 8-core cluster of 32-bit RISC-V cores acting as a software-programmable accelerator enabling vision-based tasks.
Our work uses and enhances OpenTitan, the first collaborative open-source RISC-V-based silicon RoT, to support service request handling through an System Control and Management Interface (SCMI)~\footnote{\url{https://developer.arm.com/documentation/den0056/d/?lang=en}} mailbox, master on the host domain, and secure GPIO handling connected to LEDs.

We showcase our system design with a novel and field-proven use case of \textit{Unconventional Visual Communication} (UVC) between nano-drones exchanging messages by LED blinking.
In the context of visible light communication, machine learning techniques are used on the receiving end to implement signal demodulation~\cite{9352471}, to recover the modulated signal from rolling-shutter images~\cite{6477759,Hsu:20}, or to find locations of transmitters in images, before analyzing those regions of interest separately~\cite{9700576,9780193,trajectories}. 
In contrast, our approach feeds raw images directly to a CNN to directly extract binary signal information (LEDs on or off), independently of the relative location of the drone transmitting the signal.

Once a cyber-attack compromises a nano-drone in the fleet, the radio channel cannot be trusted and the UVC is triggered, which depends exclusively on the secure OpenTitan sub-module and results in an SOS message emitted by the blinking LEDs.
We show how other nano-drones equipped with the same SoC can reconstruct the SOS message by analyzing a video stream.
By running a convolutional neural network (CNN), we assess the LED state of each input image.
Then, a simple state machine continuously analyzes the series produced by the CNN and retrieve any custom message, such as the SOS one. 

Our main contribution is the development of a novel SoC for drones' autonomous navigation providing cyber-security features which are keys in the proposed UVC-based use case.
In detail: \textit{i}) we integrate the OpenTitan secure subsystem into the navigation controller SoC; \textit{ii}) we develop and field-test a simple light-based communication between multiple nano-drones in the swarm.
With a power envelope of \SI{250}{\milli\watt} and a silicon footprint of \SI{9}{\square\milli\meter}, the proposed SoC can recognize an SOS message in \SI{1.3}{\second} performing 2.3$\times$ faster than a Crazyflie nano-drone equipped with an AI-deck, while offering support for a security enclave and full-fledged operating system.