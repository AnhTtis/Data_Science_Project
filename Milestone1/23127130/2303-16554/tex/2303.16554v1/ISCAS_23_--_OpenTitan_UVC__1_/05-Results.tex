\section{Experimental results} \label{sec:results}

\subsection{Deployment on Crazyflie with AI-deck}

Figure~\ref{fig:roc-curve} reports the CNN performance on the testing set; the full-precision model achieves an Area Under the ROC curve score of 98.88\%, with negligible performance loss ($-$0.09\%) after 8-bit integer quantization.
After binarizing outputs at a threshold of 0.5, the model achieves an accuracy of 95.1\%.

\begin{table}[t]
    \centering
    \caption{Power consumption and Area occupation}
    \begin{tabular}{|c|c|c|c|c|c|} \hline
                           & Area      & Leakage & Dynamic            & Max Freq & Max Power \\ 
                          & ($mm^2$)   & ($mW$)  & ($\frac{uW}{MHz}$) & ($MHz$)     &  ($mW$) \\ \hline
         Top              & 7.28       & 4.23    & 214.7              & 450       & 100.53 \\\hline
         CVA6             & 0.49       & 4.79    & 47.5               & 900       & 47.54  \\ \hline
         PMCA             & 1.56       & 5.78    & 206                & 400       & 88.18  \\ \hline
         Mem Ctrl.        & 0.27       & 0.14    & 2.3                & 450       & 1.16   \\ \hline
         Opentitan        & 0.86       & 4.53    & 16                & 350       & 10.13   \\ \hline
         Total            & 7.28       & 19.47   & 486.5             & -         & 247.54  \\ \hline
    \end{tabular}
    \label{tab:power_table}
\end{table}

The end-to-end message transmission is assessed with an experiment in which the transmitter drone sends a sequence of 256 messages with a payload values from 0x00 to 0xFF. 
The observer drone, placed at a fixed distance of \SI{30}{\centi\meter}, is always in line-of-sight with the transmitter one and decodes the received messages, at 30 FPS, with the quantized CNN.
All 256 messages are decoded correctly. 
Figure~\ref{fig:incremental-samples} reports a subsequence of the received messages and a supplementary video demonstration is provided at \url{https://youtu.be/TClcuUWJe0U}.


\subsection{Physical Implementation \& Performance Evaluation}

The proposed SoC has been implemented in the Global Foundries \SI{22}{\nano\meter} FDX technology, employing the Synopsys Design Compiler for the logical synthesis and the place and route with Cadence Innovus. 
For the SoC's signoff we used the Synopsys PrimeTime, considering the worst case operating corner for a nominal supply voltage of \SI{0.8}{\volt} (SS, \SI{0.72}{\volt}, 125\textdegree C/-40\textdegree C), while power analysis was performed in typical operating conditions (TT, \SI{0.8}{\volt}, 25\textdegree C)
The layouts of the SoC and the main subsystems composing it are shown in Figure~\ref{fig:layout}, while Table~\ref{tab:power_table} summarizes the physical implementation.

\begin{figure}[t!]
  \centering\includegraphics[width=\columnwidth]{images/orizontal_layout.png}
  \caption{In the middle, the layout of the SoC. Around it, the layouts of the PMCA, secure subsystem, HyperBus and CVA6.}
  \label{fig:layout}
\end{figure}

To evaluate the performance of the proposed SoC on the UVC use-case, we first deploy our CNN on the PMCA of GAP8 SoC hosted on the COTS Carzyflie nano-drone.
The full inference of the DNN on the proposed SoC takes \SI{3.7}{\mega cycles}, meaning that each payload's bit can be recognized by the receiver drone in \SI{126}{\milli\second} and that a full message described in Figure~\ref{fig:incremental-samples} can be recognized in \SI{1.3}{\second} by our proposed SoC.
This is 2.3$\times$ faster than the same application running on the GAP8 SoC (@\SI{175}{\mega\hertz}).
By coupling a linux-capable application core with a parallel programmable accelerator and a secure enclave within the power budget of \SI{250}{\milli\watt} and a footprint of \SI{9}{\square\milli\meter}, our SoC represents an appealing solution for secure and high-performance mission computers for nano-drones, paving the way for a wide range of new secure applications.