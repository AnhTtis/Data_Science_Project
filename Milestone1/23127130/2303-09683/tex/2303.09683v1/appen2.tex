
\section*{Algorithm to generate simulated sequence by predicting the order of nucleotides in a sequence using the probability of dinucleotide frequency - Dutta et.al in 1992  \cite{Dutta1992}}
\textit{Input}: DNA Sequence\\
\textit{Output}: CGR of simulated sequence.\\ \\
\indent  Step 1. Determine the frequency of occurrences of dinucleotides from a given
 sequence.\\
\\ \indent  Step 2. Set the length L of the hypothetical sequence to be
generated.\\
\\ \indent  Step 3. Randomly choose the first base N$\left( 1 \right)$ of the
sequence.\\
\\ \indent  Step 4. Set $i$ = N$\left( 1 \right)$, Set j = 2;\\
\\ \indent  Step 5. Repeat 6  and 7 while $j \le L$\\
\\ \indent  Step 6. Generate a number $n$ between 0 and 1.
\\ \indent \hspace{1.0cm} if 0 $<n < P_{iA}$ then N(j) = A
\\ \indent \hspace{1.0cm} if $P_{iA} < n < P_{iT} $ then N(j) = T
\\ \indent \hspace{1.0cm} if $P_{iT} <  n < P_{iG} $ then N(j) = G
\\ \indent \hspace{1.0cm} if $P_{iG} <  n < P_{iC} $ then N(j) = C
\\ \indent where  $P_{iA}$, $P_{iT}$,  $P_{iG}$ and  $P_{iC}$ represent the probability of A,T,G,C follow the $ith$ nucleotide.\\
\\ \indent  Step 7. Set i = N(j) and j = j +1.\\
\\ \indent Step 8. Generate CGR for the hypothetical sequence and compare with the CGR of the original
sequence.\\
\\ \indent  Step 9. If the CGRs do not match then reset the values of the
probabilities.\\
\\ Guidelines for setting the probabilities were  also provided based on trial and error method, also said,
the algorithm could be modified to use the probability occurrences of tri-, tetra- or $k$-nucleotides.
Therefore, \cite{Dutta1992} concluded, the sparse region in the CGRs of vertebrate gene (Fig 4.2) are due to
the rare occurrences of the dinucleotide and not due to any non-random occurrence of single nucleotide.
