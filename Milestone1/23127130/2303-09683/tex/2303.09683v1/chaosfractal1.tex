\section{Fractal}
\paragraph
\indent A \textit{fractal} is a geometric figure that does not
become less complex  when you break it down into smaller and smaller
parts. This implies, a fractal is scale invariant. The word fractal
was coined by Mandelbrot from the Latin word \textit{fractus}
meaning broken or uneven, to describe objects that are too
irregular to fit into traditional geometry \cite{Falconer1990}.
For example, if we take a straight line and remove the middle
third from it, we obtain two small straight line segments and if
we continue this process repeatedly for smaller segments, in the
limit we obtain a fractal called the Cantor set (fig 3.1). Similar
examples are the Koch curve and the Sierpinski triangle (fig 3.2,
fig 3.3). In all of the above examples, the same structure has been
repeated at all scales. Therefore, these fractals are known as
\textit{self-similar} fractals. It is not necessary for fractals
to be self-similar. For example a coastline, the human body, or the sky on a
partly cloudy day are fractals without being self-similar\cite{Baranger}.
\begin{figure}
\begin{center}
\includegraphics{cantor.eps}
\caption{The middle third Cantor set generated by repeated removal
of middle third of interval} \label{fig:3.1}
\end{center}
\end{figure}
\begin{figure}
\begin{center}
\includegraphics[scale=0.60]{koch.eps}
\caption{The Koch curve generated by replacing the middle third of
each interval by the other two sides of an equilateral triangle
\cite{Fractal}:(Used)} \label{fig:3.2}
\end{center}
\end{figure}
\begin{figure}
\begin{center}
\includegraphics{sierp_det.eps}
\caption{The Sierpinski triangle generated by repeatedly removing
the inverted equilateral triangle from the center of the initial
equilateral triangle.} \label{fig:3.3}
\end{center}
\end{figure}
\subsection{Properties of fractals}\label{subsec:fractdim}
\paragraph
\indent Based on the examples above, we have the following
properties of fractals \cite{Falconer1990}. A fractal set F
\begin{enumerate}
    \item has a fine structure i.e detail on arbitrarily small
    scale;
    \item is too irregular to be described by traditional
    geometry;
    \item often has some form of self-similarity, perhaps
    approximate or statistical. For example, any part of the Cantor set F  in the interval
    $[0,\frac{1}{3}]$ and the interval $[\frac{2}{3},1]$ are geometrically similar to
    F. Figure 3.1, 3.2 and 3.3 contain copies of itself at
    different scales;
    \item is in most cases defined recursively. For example, the
    Cantor set is generated by repeatedly removing the middle
    third of intervals and the Sierpinski triangle is obtained by
    repeatedly removing the inverted triangle.
    \item usually has the fractal dimension (see \ref{subsec:fractdim}) greater than the
    topological dimension (see below) or Covering Dimension.
\end{enumerate}
 The topological dimension is defined as
    \\ \\
\indent  A covering of a subset $\mathcal{S}$ of a topological
space $\mathcal{X}$ is a collection $\mathcal{C}$ of open subsets
in $\mathcal{X}$ whose union contains all of $\mathcal{S}$\\ \\
\indent  A refinement of a covering $\mathcal{C}$ of S is another
$\mathcal{C}^\prime$ of $\mathcal{S}$ such that each set
$\mathcal{B}$ in $\mathcal{C}^\prime$ is contained in some set A
in $\mathcal{C}$. The idea is that the sets in
$\mathcal{C}^\prime$ are in some sense ``smaller'' than those in
$\mathcal{C}$ and provide a more finely detailed coverage of
$\mathcal{S}$.\\ \\\indent A topological space $\mathcal{X}$ has
topological dimension $m$ if every covering $\mathcal{C}$ of
$\mathcal{X}$ has a refinement $\mathcal{C}^\prime$ in which every
point of $\mathcal{X}$ occurs in at most $m+1$ sets in $\mathcal{C}^\prime$,
and $m$ is the smallest such integer.

\subsection{Mathematical fractals}
\paragraph
\indent Mathematically, a great variety of fractals could be
generated by iterating a collection of transformations, forming
what is known as an Iterated Function System (IFS). If all the
transformations in an IFS are \textit{contractive mappings} then
iterating these transformations would definitely converge to a
unique shape. A contractive mapping is a transformation $f$ that
reduces the distance between every pair of points. That is, there
is a number $s$ between 0 and 1 and
\\
$$dist(f(x,y),f(x',y')) \leq s * dist((x,y),(x',y'))$$
\\Formally, a contractive mapping, an IFS and an affine transformation are defined as
\paragraph
\indent \textbf{Definition 1:} A transformation $f$: $\mathcal{X}
\rightarrow \mathcal{X}$ on a metric space ($\mathcal{X},d$) is
called a contractive mapping if there is a constant 0 $\leq s \leq
1$ such that $d(f(x),f(y)) \leq s * d(x,y)$  $\forall  x,y
\varepsilon \mathcal{X}$, where $d$ is the Euclidean distance and
any $s$ the contraction factor of $f$.\\ \\
\indent \textbf{Definition 2:} Let $\mathcal{T}_1$,
$\mathcal{T}_2$, ..,$\mathcal{T}_N$ be a family of contractions on
$\Re^k$ and $\mathcal{S}$ be a closed bounded subset of $\Re^k$.
Then the system { $\mathcal{S}$: $\mathcal{T}$ = $U_{i=1}^n$
$\mathcal{T}_i$} is called an \textit{iterated function system}.\\
\\ \indent \textbf{Definition 3:} An \textit{affine transformation}
of $\Re^n$ is achieved by applying a linear transformation
followed by a translation. An affine transformation $\mathcal{T}$
of $\Re^n$ is represented in matrix-vector form as
\begin{center} $\mathcal{T}(x) = \mathcal{A}x +b,  x \in \Re^{n}$
and $\mathcal{A}$ is a transformation matrix \end{center}
\subsection{Example}
\paragraph
\indent The following example explains the mathematically
generated self-similar fractal called the Sierpinski triangle (fig
3.3). Consider $\mathcal{E}_{0}$ to be a unit triangle. The
contractive mapping for producing the
Sierpinski Triangle is given by three affine transformations. The three affine transformations for the Sierpinski triangle are\\
$$\mathcal{T}_{1}\left(\biggl[\begin{array}{c} x_{1} \\ x_{2} \end{array}\biggr]\right) = 1/2\biggl[\begin{array}{cc} 1 & 0 \\ 0 & 1 \end{array}\biggr] \hspace{0.4cm} \biggl[\begin {array}{c}x_{1}\\ x_{2}\end{array}\biggr] + \biggl[\begin{array}{c}0\\ 0\end{array}\biggr]$$
$$\mathcal{T}_{2}\left(\biggl[\begin{array}{c} x_{1} \\ x_{2} \end{array}\biggr]\right) = 1/2\biggl[\begin{array}{cc} 1 & 0 \\ 0 & 1 \end{array}\biggr] \hspace{0.2cm} \biggl[\begin {array}{c}x_{1}\\ x_{2}\end{array}\biggr] + \biggl[\begin{array}{c}1/2\\ 0\end{array}\biggr]$$
$$\mathcal{T}_{3}\left(\biggl[\begin{array}{c} x_{1} \\ x_{2} \end{array}\biggr]\right) = 1/2\biggl[\begin{array}{cc} 1 & 0 \\ 0 & 1 \end{array}\biggr] \hspace{0.2cm} \biggl[\begin {array}{c}x_{1}\\ x_{2}\end{array}\biggr] + \biggl[\begin{array}{c}1/4\\ \sqrt(3)/4\end{array}\biggr]$$
\\with a contractive factor $\frac{1}{2}$.
Applying $\mathcal{T}_{1}\left(\mathcal{E}_{0}\right)$,
$\mathcal{T}_{2}\left(\mathcal{E}_{0}\right)$ and
$\mathcal{T}_{3}\left(\mathcal{E}_{0}\right)$ to the triangle
$\mathcal{E}_{0}$ produces three smaller equilateral triangles
$\mathcal{E}_1$. Similarly, applying all the three transformations
to all the three vertices of each of the smaller triangles
$\mathcal{E}_1$ produces nine smaller triangles $\mathcal{E}_2$.
The iterative application of the three affine transformations
produces smaller and smaller triangles resulting in the Sierpinski
triangle (fig 3.3).\\  The iterative scheme is
\begin{center}$\mathcal{E}_0$ = a compact set\\
$\mathcal{E}_1 = \mathcal{T}(\mathcal{E}_0) =
\mathcal{T}_1(\mathcal{E}_0) \bigcup \mathcal{T}_2(\mathcal{E}_0)
\bigcup \mathcal{T}_3(\mathcal{E}_0)$ \\
$\mathcal{E}_2 = \mathcal{T}(\mathcal{E}_1) =
\mathcal{T}_1(\mathcal{E}_1) \bigcup \mathcal{T}_2(\mathcal{E}_1)
\bigcup
\mathcal{T}_3(\mathcal{E}_1)$\\
$\vdots$ \\
$\mathcal{E}_n = \mathcal{T}(\mathcal{E}_n-1) =
\mathcal{T}_1(\mathcal{E}_n-1) \bigcup
\mathcal{T}_2(\mathcal{E}_n-1) \bigcup
\mathcal{T}_3(\mathcal{E}_n-1)$\end{center} This sequence would
converge to a unique shape (the Sierpinski triangle) called an
\textit{attractor}. Since all the
transformations are applied in each step, this approach is a \textit{deterministic approach}. \\
\subsection{Fractal dimension}
\paragraph
\indent
 The \emph{dimension} is a topological measure of spacial extent. For example, a point has a dimension 0, a line has
 a dimension 1, a square has a dimension 2 and a cube has a dimension 3. However, topological dimension cannot
 be used to measure fractals, because, for example, when trying to measure the length of the Koch curve using line
 segments, as the number of line segments needed to measure the length increases, the length of the Koch curve
 increases, leading to infinity (fig 3.4). Table 3.1 lists the increasing length of the Koch curve as the
 number of line segments needed to measure the Koch curve increases (The initial line is of length 1). \\
\begin{table}
\begin{center}
 \begin{tabular}{|l|l|l|l|}
\hline n & length of the segment & number of segments &
Length of the Koch curve $\mathcal{L}_n$\\
\hline 0 & 1 & 1 & $\mathcal{L}_n = 1$\\
\hline 1 & $1/3$ & 4 & $\mathcal{L}_n = 4/3$\\
\hline 2 & $1/9 = 1/3^2$ & $16 = 4^2$ & $\mathcal{L}_n = 16/9 =
(4/3)^2$\\
\hline 3 & $1/27 = 1/3^3$ & $64 = 4^3$ & $\mathcal{L}_n
= 64/37 = (4/3)^3$\\ \hline \ldots & \ldots & \ldots & \ldots \\
\hline n & $1/3^n$ & $4^n$ & $\mathcal{L}_n = (4/3)^n$\\
\hline
\end{tabular}
\end{center}
\caption{Dimension of the Koch curve using length of line
segments} \label{Table: 3.1}
\end{table}
\\Similarly, when trying to compute the area of the Koch curve by covering it with triangles, as the
number of triangles needed to cover the Koch curve increases, the
area of the Koch curve decreases leading to zero (fig 3.4). The
initial triangle is an isosceles triangle with base 1 and height
 $\sqrt(3)/6$. In the next stage the Koch curve is covered with three smaller triangles whose
 base and height are reduced by $1/3$ compared to the initial triangle. As the process continues, at every
 stage the area of the triangles are reduced leading to zero. In the above metioned examples the dimension of the Koch curve leads to either infinity or zero producing no limiting value. Measuring an object in an dimension lower than the object produces infinity and higher than the object produced zero. This implies the dimension of the
 Koch curve is $>$ 1 but $<$ 2 a fractional value. Table 3.2 lists the area of the Koch curve for various
 sizes of the triangle needed to cover the Koch curve.
 \begin{table}
 \begin{center}
 \begin{tabular}{|l|l|l|l|}
\hline n & Area of the triangle & number of triangles &
Area of the Koch curve $\mathcal{A}_n$\\
\hline 0 & $\sqrt(3)/12$ & 1 & $\mathcal{A}_n = \sqrt(3)/12$\\
\hline 1 & $(\sqrt(3)/12) * (1/9)$ & 4 & $\mathcal{A}_n = (\sqrt(3)/12)*(4/9)$\\
\hline 2 & $(\sqrt(3)/12) * (1/81)$ & $16 = 4^2$ & $\mathcal{A}_n = (\sqrt(3)/12)*(16/81)$\\
\hline \ldots & \ldots & \ldots & \ldots \\
\hline n & $(\sqrt(3)/12) *(1/9)^n$ & $4^n$ & $\mathcal{A}_n = (\sqrt(3)/12)*(4/9)^n$\\
 \hline
\end{tabular}
\end{center}
\caption{Dimension of the Koch curve using area of triangles}
\label{Table: 3.2}
\end{table}
 \begin{figure}
\begin{center}
\includegraphics[scale=0.50]{koch_topdim.eps}
\caption{Topological dimension of the Koch curve
\cite{Fractal}:(Adapted)} \label{fig:3.4}
\end{center}
\end{figure}
\paragraph
\indent
 Therefore, we use a better dimension called \textit{Box counting
 dimension} \cite{Fractal} to calculate the dimension of fractals. The box counting dimension of a fractal is
 calculated by covering the fractal with boxes and
 calculating the number of boxes $\mathcal{N}_{r}$ of size $r$ needed to cover the
 fractal (fig 3.5). The size of a fractal set is measured by its dimension($d_f$) given as:
\begin{center} $ d_f = \lim_{r \rightarrow 0}\frac{\log \mathcal{N}(r)}{\log r}$ \end{center}
For better approximation, the number of boxes needed to cover the fractal for various box sizes $r$ is calculated and the fractal dimension is the slope of log-log
 plot of size of the boxes against the number of such boxes needed to cover
 the fractal.
\begin{figure}
\begin{center}
\includegraphics[scale=0.55]{KochBoxDim.eps}
\caption{Box counting dimension of the Koch curve
\cite{Fractal}:(Adapted), r - length of the sides and N(r) - no. of boxes needed to cover the fractal} \label{fig:3.5}
\end{center}
\end{figure}
\paragraph
\indent \textit{Example:} Figure 3.5 depicts the number of boxes
needed to cover the Koch curve for varying box sizes. The slope of
a log-log plot of size of the boxes against the number of boxes
needed to cover the Koch curve gives the fractal dimension of the
curve.
\section{Chaos game}
\paragraph
\indent Another approach to generate fractals is the random
approach of \textit{Chaos Game}. Consider a triangle and the three
transformations defined in section 3.1.3. Let the initial set be a
single point. Assume that, at each step, one of the three
transformations is randomly chosen and applied. Therefore, the
output at each stage is a single point. After some transient
behavior the points generated form a fractal - Sierpinski triangle. The iteration scheme is
\begin{center}$y_{0}$ = start point\end{center}
\begin{center}$y_{1} = \mathcal{T}_{1}(y_0)$ or $\mathcal{T}_{2}(y_0$) or $\mathcal{T}_{3}(y_0)$ \end{center}
\begin{center} $\vdots$ \end{center}
\begin{center}$y_{n} = \mathcal{T}_{1}(y_{n-1})$ or $\mathcal{T}_{2}(y_{n-1})$ or $\mathcal{T}_{3}(y_{n-1})$ \end{center}
\paragraph \indent For example, consider a triangle with vertices (0,0),(1,0) and
($1/2,\sqrt(3)/2$)(fig 3.6). The center of the triangle is chosen
as the starting point. Choose randomly one of the transformation,
say $\mathcal{T}_2$, and apply it to the center point: This would
produce a point (say $p$), which is the midpoint of the center and
the vertex (1,0). Again, randomly choose another transformation,
say $\mathcal{T}_3$, and apply it to the previously produced point
($p$): The new point produced is the midpoint of the point $p$ and
the vertex ($1/2,\sqrt(3)/2$). As this process is continued for
large number of times,  the image produced looks like a Sierpinski
triangle. The chaos game can be played with any number of vertices, four, five, six and so on. If the vertices are not selected uniformly at random in chaos game then various patterns are produced. This reveals some kind of order in the sequence.
\begin{figure}
\begin{center}
\includegraphics[scale=0.65]{sierp_chaos.eps}
\caption{Sierpinski triangle using the chaos Game} \label{fig:3.6}
\end{center}
\end{figure}
\paragraph
\indent Intuitively, order (non-randomness) means the sequence has a
structure. Therefore, the chaos game can be a used as a tool to study the non-randomness of any sequence visually.
If the chaos game can be extended to play on DNA or protein sequences then various patterns/structure in them could be revealed.
