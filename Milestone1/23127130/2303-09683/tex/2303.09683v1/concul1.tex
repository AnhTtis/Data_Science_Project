\paragraph
\indent  This thesis presented the fundamentals of DNA and protein
sequences, mathematical background of chaos and fractals, two
dimensional chaos game representation while also exploring a new
approach for CGR to study sequence similarity. Chapter 3 and 4
presented the biological and mathematical background for the
thesis. Chapter 4 presented the two dimensional CGR of DNA and
protein sequences studied in the past. Chapter 5, presented the
new three dimensional approach to CGR for protein sequences that
provided a holistic approach to sequence analysis
\paragraph
\indent First, in Chapter 2, the biological introduction to the
thesis was presented. The meaning, structure and the functionality
of DNA and protein sequences and the synthesis of protein sequence
from DNA were explained. The representation of sequences/species
relatedness was explained through phylogenetic tree and the
analysis of sequences/species relatedness was explained through
the bioinformatic technique, sequence alignment. Also, the process
of multiple sequence alignment was briefly explained through a
widely used program CLUSTALW.
\paragraph
\indent Next, in Chapter 3, the mathematics background to understand the chaos game for
biological sequences was presented. The mathematical background of
chaos game for generating fractals  and its ability to reveal the
structure present in the non-random sequences were explained.
Also, the concepts of fractal and the need for fractal dimension
were explained with various examples.
\paragraph
\indent Then, in Chapter 4, the literature on chaos game
representation of DNA and protein sequences in two dimension was
presented.  A detailed description of the methods, novel advances
and limitations of chaos game on protein sequence in two dimension
was emphasized.
\paragraph
\indent Finally, in Chapter 5, a new approach and results of chaos
game representation of protein sequence in three dimension was
given. The selection of the geometric solid icosahedron to represent
twenty amino acids and its structure were explained. The new three
dimensional approach was taken to present a chaos game model by
mapping amino acids on to the icosahedron faces based on the dinucleotide
relatedness of amino acid from codons. The new approach (3D-CGR)
was used to study sequence relatedness, the effect of dinucleotide
biases at the amino acid level on the 3D-CGR deduced protein homology, and shuffled motif detection.
\paragraph
\indent The 3D-CGR was evaluated using phylogenetic trees. Trees
generated using 3D-CGR were able to distinguish protein families
and species relatedness within the families of sequences.  Also,
the effect of varying the mapping of 20 amino acids on the faces
of the icosahedron analysed using phylogenetic tree showed very
small difference in the branching of protein families and branch
lengths of closely related sequences between the phylogenetic
trees generated by random mappings and mapping based on the
dinucleotide relatedness of codon. The comparison of phylogenetic
trees of 3D-CGR and CLUSTALW revealed the significant difference
in branch length between closely related sequences indicated
3D-CGR could be used for measuring the amount of divergence
between sequences within a family. Also, 3DCGR can detect sequence
relatedness by detecting multiple motifs present in sequences
irrespective of the order of the motifs therefore, it could be a
useful tool in studying protein evolution due to exon shuffling.
\paragraph
\indent The Image distance measure used for generating
phylogenetic trees was compared with Euclid and Pearson distance.
All the three distance measures were able to distinguish protein
families and relatedness of species within the families. Finally,
sequence relatedness explored using fractal curves did not provide
much information except the fractal curves generated were similar
for protein families.
\paragraph
\indent The patterns produced by 3D-CGR were not visible to the
naked eye. Therefore, it is hoped the future research in 3D-CGR on
protein sequences could represent the 'cloud of points' in space
as a structure for visual comparison. Also, the research could be
extended to detect the position of shuffled motifs in 3D-CGR in
order to study the protein evolution due to exon shuffling.
