
\section*{Algorithm to determine the subsequence represented by any point in a CGR - Dutta et.al \cite{Dutta1992}}
\textit{Input}:  A point (X,Y) on the CGR.\\
\textit{Output}: Sequence that generated the point (X,Y)\\ \\
\indent Step 1. Let (0,0), (2,0),(2,2) and (0,2) be the vertices
of the square and (CX,CY) represent the center of the square. CX =
1 and CY =1\\
\\ \indent Step 2. Let (X,Y) be  the coordinates of the point whose sequence we want to determine (within the resolution limit of the monitor $\pm \delta $ of the
monitor).\\
\\  \indent  Step 3.  L is the length of the subsequence to be
generated.\\
\\  \indent Let PX and PY be two variables that hold the coordinates of the points as they are generated. Based on the center coordinate and the resolution limit of the monitor, determine in which quadrant the point belongs and change the center accordingly. The process is repeated until the X and Y values are equal to the center +/- the resolution limit of the monitor. These are given by the following steps:\\
\\  \indent  Step 4.  Repeat step 4 L times
\\  \indent \hspace{0.5cm} if  X $>$ CX and Y $> $CY then
\\ \indent \hspace{1.0cm} if X $>$ CX + $\delta$ then PX = 2
\\ \indent \hspace{1.0cm} if X $<$ CX + $\delta$ then PX = 1
\\ \indent \hspace{1.0cm} if Y $>$ CY + $\delta$ then PY = 2
\\ \indent \hspace{1.0cm} if Y $<$ CY + $\delta$ then PY = 1
\\ \indent \hspace{1.0cm} if X = CX  $\pm \delta$  or Y = CY $\pm \delta $ then goto step 6
\\ \indent \hspace{0.5cm} set CX = CX + $\left(-1\right)^{PX} $ $\left(1/2\right)$
\\ \indent \hspace{0.5cm} set CY = CY + $\left(-1\right)^{PY} $ $\left(1/2\right)$
\\ \indent \hspace{0.5cm} if PX =2 and PY = 2 then  N = G
\\ \indent \hspace{0.5cm} if PX =2 and PY = 1 then  N = T
\\ \indent \hspace{0.5cm} if PX =1 and PY = 2 then  N = C
\\ \indent \hspace{0.5cm} if PX =1 and PY = 1 then  N = A
\\ \indent Step 5. At each step set the value of CX and CY as follows, For the \textit{ith} step:
\\ \indent \hspace{0.5cm} CX = CX + $\left(-1\right)^{PX} $ $\left(1/2\right)^i$
\\ \indent \hspace{0.5cm} CY = CY + $\left(-1\right)^{PY} $ $\left(1/2\right)^i$
\\ \indent Step 6. When CX  - $\delta < X < CX + \delta$ and CY - $\delta < Y < CY + \delta$ then the length of the sequence is reached or the limit of the resolution of the CGR is
reached.\\
\\
Both CX and CY should reach the resolution limit simultaneously.
\\ The above algorithm can also be used to determine missing sequences as well.

