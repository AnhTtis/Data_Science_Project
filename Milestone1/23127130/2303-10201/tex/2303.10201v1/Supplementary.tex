% ****** Start of file apssamp.tex ******
%
%   This file is part of the APS files in the REVTeX 4.1 distribution.
%   Version 4.1r of REVTeX, August 2010
%
%   Copyright (c) 2009, 2010 The American Physical Society.
%
%   See the REVTeX 4 README file for restrictions and more information.
%
% TeX'ing this file requires that you have AMS-LaTeX 2.0 installed
% as well as the rest of the prerequisites for REVTeX 4.1
%
% See the REVTeX 4 README file
% It also requires running BibTeX. The commands are as follows:
%
%  1)  latex apssamp.tex
%  2)  bibtex apssamp
%  3)  latex apssamp.tex
%  4)  latex apssamp.tex
%
\documentclass[%
preprint,
superscriptaddress,
%groupedaddress,
%unsortedaddress,
%runinaddress,
%frontmatterverbose, 
%preprint,
%showpacs,preprintnumbers,
%nofootinbib,
%nobibnotes,
%bibnotes,
 amsmath,amssymb,
 aps,
%pra,
%prb,
%rmp,
%prstab,
%prstper,
%floatfix,
]{revtex4-1}

\usepackage{graphicx}% Include figure files
\usepackage{dcolumn}% Align table columns on decimal point
\usepackage{bm}% bold math
%\usepackage{hyperref}% add hypertext capabilities
%\usepackage[mathlines]{lineno}% Enable numbering of text and display math
%\linenumbers\relax % Commence numbering lines

\usepackage{todonotes}
\usepackage{xcolor}
\usepackage[normalem]{ulem}
\usepackage{braket}


\usepackage[pagebackref=false]{hyperref}
\renewcommand{\thefigure}{S\arabic{figure}}

\begin{document}
\author{L.~Banszerus}
\thanks{These two authors contributed equally.}
\author{S.~M\"oller}
\thanks{These two authors contributed equally.}
\affiliation{JARA-FIT and 2nd Institute of Physics, RWTH Aachen University, 52074 Aachen, Germany,~EU}%
\affiliation{Peter Gr\"unberg Institute  (PGI-9), Forschungszentrum J\"ulich, 52425 J\"ulich,~Germany,~EU}
\author{K.~Hecker}
\author{E.~Icking}
\affiliation{JARA-FIT and 2nd Institute of Physics, RWTH Aachen University, 52074 Aachen, Germany,~EU}%
\affiliation{Peter Gr\"unberg Institute  (PGI-9), Forschungszentrum J\"ulich, 52425 J\"ulich,~Germany,~EU}
\author{K.~Watanabe}
\affiliation{Research Center for Functional Materials, 
National Institute for Materials Science, 1-1 Namiki, Tsukuba 305-0044, Japan}
\author{T.~Taniguchi}
\affiliation{International Center for Materials Nanoarchitectonics, 
National Institute for Materials Science,  1-1 Namiki, Tsukuba 305-0044, Japan}%
\author{F.~Hassler}
\affiliation{JARA-Institute for Quantum Information, RWTH Aachen University, 52056 Aachen, Germany, EU}
\author{C.~Volk}
\author{C.~Stampfer}
\email{stampfer@physik.rwth-aachen.de}
\affiliation{JARA-FIT and 2nd Institute of Physics, RWTH Aachen University, 52074 Aachen, Germany,~EU}%
\affiliation{Peter Gr\"unberg Institute  (PGI-9), Forschungszentrum J\"ulich, 52425 J\"ulich,~Germany,~EU}%


\title{Supporting Information:\\ Particle-hole symmetry protects spin-valley blockade in graphene quantum dots}


\date{\today}

\maketitle
\tableofcontents

\clearpage
\subsection{Charge stability diagrams for opposite bias voltages in DQD \#1}

\begin{figure}[!thb]
\centering
\includegraphics[draft=false,keepaspectratio=true,clip,width=\linewidth]{FigS1A.pdf}
\caption[Fig01]{Charge stability diagrams of DQD \#1 (as in Fig.~1d of the main text) measured at a bias voltage of \textbf{a} $V_\mathrm{SD} = 1$~mV and \textbf{b} $V_\mathrm{SD} = -1$mV (T=10mK).
The dashed circles mark the formation of single electron -- single hole DQDs using the hole QD and an electron QD to the left (red) or right (black) of the hole QD. \textbf{c-d} Schematics of the valence and conduction band edge profiles along the p-type
channel. An electron-hole double quantum dot is formed using the hole QD and the electron QD underneath the left (right) FG (see red (black) circles in Fig.~S1a,b).
 }
\label{S1}
\end{figure}

Fig.~\ref{S1} compares charge stability diagrams measured at positive and negative bias voltage in DQD \#1 (c.f. Figs. 1, 2 and 3 in the main text). 
The dashed lines indicate the charge transitions of the electron (black) and hole (red) QDs. 
Electron-hole (e-h) DQDs are formed at the intersections of these charging lines. 
For the left electron-hole DQD ($(0h,0e)\leftrightarrow(1h,1e)$ transition, see red circle), transport is blocked at positive bias, while for the right electron-hole DQD ($(0h,0e)\leftrightarrow(1h,1e)$ transition, see black circle), transport is blocked at negative bias. The data in the main text has been obtained in the latter regime.

\clearpage
\subsection{Extracting $\Delta_\text{SO}$ from measurements on a single-electron DQD in the same device}
To compare the measured value for $\Delta_\mathrm{SO}$ in the electron-hole DQD and to demonstrate that the magnitude of the SO gap is symmetric for electrons and holes, we present measurements of  $\Delta_\mathrm{SO}$ in an electron-electron DQD. Fig.~\ref{S5} shows a close-up of the first triple point of an electron-electron DQD formed in the same device (c.f. Fig.~\ref{S1}). Transport via a ground state and an excited state can be observed. 
We extract their energy splitting by fitting two Lorentzian peaks to a linecut through the triple point (see Fig.~\ref{S5}b). The determined value of $\Delta_\mathrm{SO} = 68 \pm 7~\mu$eV is in good agreement with the ones observed in the electron-hole DQD regime. A detailed discussion of $\Delta_\mathrm{SO}$ and the single particle spectrum in the electron DQD in this device is given in Ref.~\cite{Banszerus2021Sep}.


\begin{figure}[!thb]
\centering
\includegraphics[draft=false,keepaspectratio=true,clip,width=0.9\linewidth]{FigS2A.pdf}
\caption[Fig01]{
\textbf{a} Charge stability diagrams of the $(1e,0e)\leftrightarrow(0e,1e)$ transition of an electron-electron DQD measured at $V_\mathrm{SD} = 1~$mV and $B_\perp = 0~$T (T=10mK).
A ground state and an excited state transition are visible (see black arrows). 
\textbf{b} Cut along the yellow dashed line in a. Two Lorentzian peaks (dashed lines) are fitted to the data.
Inset: Schematic energy diagrams of an electron-electron DQD in the finite bias regime for different interdot detuning
energies $\varepsilon$, illustrating resonant transport from the left (L) to the right (R) QD through the ground state of each QD (transition (i)) and resonant transport at $\varepsilon = \Delta_\mathrm{SO}$ (transition (ii)).
}
\label{S5}
\end{figure}

\clearpage
\subsection{Additional data set for another e-h double quantum dot (DQD \#2) in the same device }

\begin{figure}[!thb]
\centering
\includegraphics[draft=false,keepaspectratio=true,clip,width=0.85\linewidth]{FigS3A.pdf}
\caption[Fig03]{ \textbf{a} and \textbf{b} Gate configurations used to form DQD \#1 and DQD \#2 in the device, respectively. \textbf{c}
 Charge stability diagram of an e-h DQD formed with the second set of gate fingers (DQD \#2, see panel b). The dashed circle marks the $(0h,0e)\rightarrow(1h,1e)$ transition. $V_\mathrm{SD} = 1~$mV (T=10mK).
}
\label{S3}
\end{figure}

\begin{figure}[!thb]
\centering
\includegraphics[draft=false,keepaspectratio=true,clip,width=1\linewidth]{FigS4A.pdf}
\caption[Fig03]{  
\textbf{a, b} Close-ups of the $(0h,0e)\rightarrow(1h,1e)$ triple point at $V_\mathrm{SD} = 0.5~$mV and $V_\mathrm{SD} = 1.5~$mV, respectively. Transport only occurs via the $\alpha$ and $\beta$ transition.
\textbf{c} Charge stability diagram as in c measured at $B_\perp = 0.6$~T. 
\textbf{d} Charge stability diagram as in b at $B_\parallel = 0.7$~T.
\textbf{e, f} Charge stability diagrams as in b and c at $V_\mathrm{SD} = -0.5$~mV and $V_\mathrm{SD} = -1.5$~mV. Transport is strongly suppressed, only co-tunneling can be observed. 
\textbf{g, h}  Charge stability diagrams as in g measured at $B_\perp = 0.6$~T and $B_\parallel = 0.7$~T.
}
\label{S32}
\end{figure}

A second e-h DQD has been studied, formed with a different set of gate fingers on the same gated bilayer graphene device as presented in the main text (DQD \#2 depicted in Fig.~S2b). 
The single electron -- single hole transition, $(0h,0e)\rightarrow(1h,1e)$, is highlighted by the dashed circle in the charge stability diagram (see Fig.~\ref{S3}c). 

Measurements of that bias triangle are shown in Figs.~\ref{S32} for different $V_\mathrm{SD}$ and magnetic fields, showing good agreement with the data presented for DQD \#1 in Fig.~2.
In contrast to the data presented in the main manuscript, co-tunneling is more pronounced due to a strong coupling of the hole QD to the reservoir. 


\begin{figure}[!thb]
\centering
\includegraphics[draft=false,keepaspectratio=true,clip,width=\linewidth]{FigS5A.pdf}
\caption[Fig01]{
\textbf{a} Energy dispersion of single-particle states in the first orbital for electrons and holes as a function of in-plane ($B_\parallel$, left) and out-of-plane ($B_\perp$, right) magnetic fields. States and transitions are labelled as in Fig.~3a of the main text.
\textbf{b} Current through DQD \# 2 as a function of the detuning energy $\widetilde \varepsilon$ (see yellow dashed line in Fig.~\ref{S32}b) and $B_\perp$ at $V_\mathrm{SD} = 1.5~$mV. The white dashed line marks the onset of the bias transport window.
\textbf{c} Current through the device as a function of $\widetilde \varepsilon$ and $B_\parallel$ at $V_\mathrm{SD} = 1.5~$mV. \textbf{d, e} Data acquired in the blockade regime ($V_\mathrm{SD} = -1.5~$mV). The current has been measured as a function of $B_\perp$ and $B_\parallel$, respectively. Data has been symmetrized around $B = 0$.
}
\label{S4}
\end{figure}




The magnetic field dependent spectrum of the first electron and the first hole states is depicted in Fig.~\ref{S4}a (c.f. Fig.~3a of the main text).
Figs.~\ref{S4}b and c show measurements complementary to the one presented in Fig.~3 of the main text, recorded for DQD \#2 shown in Fig.~\ref{S3}b. 
The measurements show that the difference in detuning energy between  $\alpha$ and $\beta$ is independent of $B_\perp$, the energy splitting measures $\Delta \varepsilon = 150 \pm 10 \,\mu$eV, which corresponds to $2\Delta_\mathrm{SO}$.
The background current originates from co-tunneling in the bias transport window (its onset is highlighted by the white dashed line), which shifts in energy with increasing $|B_\perp|$. This is due to the fact that the bias window is defined by the (forbidden) ground state transition $\ket{K'\uparrow}_e \leftrightarrow  \ket{K'\uparrow}_h$, which requires less detuning for increasing $|B_\perp|$.
The same measurement for parallel magnetic fields shows the effect of the spins being continuously canted into the BLG plane. 
The difference in detuning of the transitions $\alpha$ and $\beta$ increases while a third resonance, $\gamma$, emerges.
The data is in good qualitative and quantitative agreement with the data presented in Fig.~3 of the main text. 
Figs.~\ref{S4}d and \ref{S4}e show magneto-transport data in the single-particle blockade regime. The spin-valley blockade is not lifted under the influence of both in-plane and out-of-plane magnetic fields. 
Transport via co-tunneling is suppressed at increasing $B_\perp$ as (also in this case) the tunneling barriers turn more opaque due to magnetic confinement.

\clearpage
\subsection{Simulation of magnetotransport through an e-h DQD}
We simulate transport within the DQD bias triangles and along the detuning cuts by solving the rate equations for the electron and hole QD states presented in Fig.~3a following the approach used in Ref.~\cite{Knothe2022Apr}. The energy of the respective electron and hole states is given by 


\begin{align}
    \mathrm{H_{e}} &= \frac{1}{2} \Delta_\text{SO} \tau_z s_z + \frac{1}{2} g_\text{s} \mu_\text{B} \mathbf{B} \cdot \mathbf{s} + \frac{1}{2} g_\text{v} \mu_\text{B} \mathrm{B}_z \tau_z  \\
    \mathrm{H_h} &=  - \mathrm{H_{e}},
\end{align}
with the spin and valley g-factors $g_\text{s} = 2 $ and $g_\text{v} = 15$,
the Bohr magneton $\mu_\text{B}$, the proximity enhanced (intrinsic) Kane-Mele spin-orbit coupling $\Delta_\text{SO} = 70\, \mu$eV and the Pauli matrices $s_i$ and $\tau_i$ which act on spin and valley, respectively. We approximate the effect of the right (R) and left (L) finger gate on the charging energy of the system by 
\begin{equation}
    E_\text{c} (N_\text{R}, N_\text{L}) = e N_\text{R}  V_\text{R} + e N_\text{L}  V_\text{L} \,,
\end{equation}
with the absolute value of the elementary charge, $e$, the QD occupation number $N_\text{L} = -1$ (1h),  $N_\text{R} = 1$ (1e) and the gate voltages $V_\text{R}$ and $V_\text{L}$. For describing transport through the e-h DQD, we focus on the $(0,0) \rightarrow (-1,1) \rightarrow (-1, 0) \rightarrow (0,0)$ charge cycle and only consider sequential tunneling. There are in total 25 possible states of the system $\chi$ = (hole QD state, electron QD state) with
\begin{align}
    \chi &= (\overline{\phi_\text{h}}\, , \overline{\psi_\text{e}}) \\
    \overline{\phi_\text{h}}\, , \overline{\psi_\text{e}}  &\in \{0,K{\uparrow},K{\downarrow},K'{\uparrow},K'{\downarrow} \}.  
\end{align}
%
Here, $ \overline{\phi_\text{h}}\, , \overline{\psi_\text{e}}$ describe the state of the left and right QD, which includes the four single particle states, as well as the QD being empty.


We assume no mixing between lead and QD states and equal tunnel probabilities to and from the leads for all states, $\gamma^\text{L,R} = 1.7$ GHz. Thus, we obtain the transition rates between QD states involving tunneling processes from the leads (L,R) by computing
\begin{align}
W^\text{L,R}_{\chi \leftarrow \chi^{\prime}} &= \gamma^\text{L,R} \,  f(E_{\chi} - E_{\chi^{\prime}} - \mu^\text{L,R}),
\end{align}
with the Fermi-function, $f$, at $T = 0.1\,$K, and the electron and hole QD states $\phi_\text{h}, \psi_\text{e}$. Note that hole states only tunnel to the left lead and electron states only tunnel to the right lead. 

For interdot transitions, we assume no mixing of electron and hole states due to the small interdot tunnel coupling. For simplicity, relaxation is neglected. We obtain the rates of the interdot transition by computing 
\begin{align}
W^\text{inter}_{(0,0) \leftarrow (\phi_\text{h}, \psi_\text{e})} &=  W^\text{inter}_{(\phi_\text{h}, \psi_\text{e}) \leftarrow (0,0) }= G^\text{inter} \, \braket{\phi_\text{h} | \psi_\text{e}}  \frac{1}{\sqrt{2 \pi \sigma}} \mathrm{exp}\left(  {-\frac{(E_{(0,0)} - E_{(\phi_\text{h}, \psi_\text{e})})^2}{4 \sigma^2}} \right),
\label{align:interdot}
\end{align} 
 with the interdot tunnel rate $\gamma^\text{inter} = 6$ kHz and $\phi_\text{h}, \psi_\text{e} \in \{ K{\uparrow},K{\downarrow},K'{\uparrow},K'{\downarrow} \}$. The Gaussian energy smearing models the experimentally observed peaks with an estimated width of the resonances $\Gamma = 40 \,\mu$eV. We expect that this smearing originates from voltage fluctuations of the finger gates. The overlap between electron and hole states is given by $\braket{\phi_\text{h} | \psi_\text{e}} = (\sigma_y \tau_x s_y)_{\phi_\text{h}, \psi_\text{e}}$ in order to assure that only electrons and holes with opposite quantum numbers are created (or annihilated). With equation (\ref{align:interdot}) we implicitly assume that the states in the left and right QD have no coherent phase relation. 

We solve the master equation of the probabilities, $\mathrm{P}_{\chi}$, for the system to be in state $\chi$,
\begin{equation}
\dot{\mathrm{P}}_{\chi} =\sum_{\chi'} (W_{ \chi \leftarrow \chi'} \, \dot{\mathrm{P}}_{\chi'}  - W_{ \chi' \leftarrow \chi} \, \dot{\mathrm{P}}_{\chi} ) ,
\label{eqn:rateeqn}
\end{equation}

in the stationary limit, $\dot{\mathrm{P}}_{\chi} = 0$, normalizing the probabilities to $\sum_{\chi} \mathrm{P}_{\chi} = 1$.
In the stationary limit, we can compute the current through the double QD by computing the current flow from the right QD to lead R:
\begin{equation} 
 I^\text{R}= e \sum_{\phi_\text{h}, \psi_\text{e} } \left( W^\text{R}_{(\phi_\text{h}, 0) \leftarrow (\phi_\text{h}, \psi_\text{e})} \dot{\mathrm{P}}_{(\phi_\text{h}, \psi_\text{e})}
-  W^\text{R}_{(\phi_\text{h}, \psi_\text{e}) \leftarrow (\phi_\text{h}, 0) } \dot{\mathrm{P}}_{(\phi_\text{h}, 0)} \right).
 \label{eqn:Iseq}
\end{equation}

We follow this procedure for different magnetic fields and different gate voltage combinations $V_\text{L}$, $V_\text{R}$. The result is shown in Fig.~\ref{S6}, where we are able to reproduce the experimental data of Figs.~2b-d and Figs.~2f-h. Additionally, we simulate the current along the detuning axis of the $(0h,0e)\rightarrow(1h,1e)$ triple point as a function of parallel magnetic field, which is presented in Fig.~3e of the main manuscript.



\begin{figure}[!thb]
\centering
\includegraphics[draft=false,keepaspectratio=true,clip,width=\linewidth]{FigS6A.pdf}
\caption[S6]{  
Charge stability diagrams of the first triple point simulated by solving the rate equation. 
\textbf{a - c} depict the forward bias direction ($V_\mathrm{SD}=1$~mV) for different magnetic fields, showing the same features as the experimental data presented in Fig.~2. 
\textbf{d - f} show the blocked bias direction ($V_\mathrm{SD}=-1$~mV) for the same magnetic fields. For zero magnetic field, the blockade is lifted at the corners of the bias triangle, where back and forth tunneling to source (or drain) allows lifting the blockade. The effect is even larger at finite parallel magnetic fields, where the spins are tilted into the plane of the BLG.}
\label{S6}
\end{figure}



\clearpage
\subsection{Electron-hole symmetry breaking due to Rashba spin-orbit coupling}
Since we are explicitly breaking the inversion symmetry of BLG with a perpendicular electric field, extrinsic (Rashba) spin-orbit coupling poses an additional mechanism to break the electron-hole symmetry in our DQD system. The corresponding full spin-orbit Hamiltonian acting on the low energy bands is then given by [6]
\begin{equation*}
\begin{split}
H_\mathrm{SO} =  \Psi^\dag \Big( & \frac{1}{4} [(\Delta^\text{t}_\mathrm{SO} + \Delta^\text{b}_\mathrm{SO}) \sigma_z - (\Delta^\text{t}_\mathrm{SO} -  \Delta^\text{b}_\mathrm{SO}) \sigma_0 ] \tau_z s_z  
 + \frac{1}{2} \lambda_\text{ex} (\sigma_y s_x + i \tau_z \sigma_x s_y ) \Big)  \Psi ,   
\end{split}
\end{equation*}
with the Pauli matrices $\tau, \sigma, s$ as defined in the main text, the extrinsic (Rashba) SO coupling  $\lambda_\text{ex}$, which scales linearly with the applied electric displacement field, and the proximity enhanced intrinsic (Kane-Mele) spin-orbit coupling energies $\Delta^\text{t}_\mathrm{SO}$ and $\Delta^\text{b}_\mathrm{SO}$ for the top and bottom layer of the BLG \footnote{$\Delta^\text{t}_\mathrm{SO}$, $\Delta^b_\mathrm{SO}$ and $\lambda_\text{ex}$ correspond to $\lambda_{I1}$ and $\lambda'_{I1}$ and $\lambda_3$ in Ref. \cite{Konschuh2012Mar}} \cite{Konschuh2012Mar}.
%
The influence of the proximity enhanced Kane-Mele spin-orbit coupling on electron-hole symmetry is discussed in the main text.

For understanding the influence of the extrinsic (Rashba) term, we note that for Fermi energies close to the band edge, the sublattice space is equivalent to the layer space and therefore to conduction and valence band. 
%
This is caused by the fact that excess charge is strongly layer polarized, only leading to a small admixture of the sublattices~\cite{McCann2013Apr, Banszerus2020May}.
%
The extrinsic SO term couples the two sublattices via $\sigma_{x,y}$ and therefore to the two layers, which experience a potential difference due to the electric displacement field.
%
As a consequence, the extrinsic spin-orbit term is suppressed to first order by~$\lambda_\mathrm{ex}^2/E_\mathrm{g}^2$.
%
Theoretical predictions of $\lambda_\text{ex}$ are at least three orders of magnitude smaller than the band gap ($E_\mathrm{g}$), rendering extrinsic spin-orbit coupling irrelevant for our system ~\cite{Konschuh2012Mar, Banszerus2021Sep}.


\clearpage
\subsection{Electron-hole symmetry breaking due to different valley g-factors in the electron and hole QDs}

We investigate how asymmetric valley g-factors would affect the transition spectrum of the e-h DQD. In Fig.~\ref{S7}a-d we simulate the current through the device as a function of the detuning energy $\widetilde \varepsilon$ and perpendicular magnetic field, $B_\perp$, for different combinations of valley g-factors in the hole and electron QD, respectively. 
As clearly visible in Figs.~S7a,b, both the $\alpha$ and $\beta$ transition split due to the difference in valley g-factors (see colored lines in Fig.~\ref{S7}a) by $\Delta E = \frac{1}{2} \mu_\text{B} |g^\text{e}_\text{v}-g^\text{h}_\text{v}| B_\perp$. For equal valley g-factors, $\alpha$ and $\beta$ do not show any $B_\perp$-dependence, as shown in Fig.~S7c. A tiny asymmetry in valley g-factors is allowed without significantly changing the observed features for magnetic fields below 1T, as shown in Fig.~S7d, where a g-factor asymmetry of 0.1 is assumed.


\begin{figure}[!thb]
\centering
\includegraphics[draft=false,keepaspectratio=true,clip,width=0.8\linewidth]{FigS7A.pdf}
\caption[Fig07]{Calculation of the current through the device as a function of the detuning energy $\widetilde \varepsilon$ (see arrow in Fig.~2c of the main text) and perpendicular magnetic field at a finite bias of $V_\mathrm{SD} = 1~$mV.
In \textbf{a}, the valley g-factors of the two QDs are chosen asymmetrically ($g^\text{e}_\text{v}=15$ for the electron QD and $g^\text{h}_\text{v}=20$ for the hole QD), resulting in a splitting of both, the $\alpha$ and $\beta$ transition, which scales with the difference in the valley g-factors. In \textbf{b}, the valley g-factors of the two QDs are chosen less asymmetrically ($g^\text{e}_\text{v}=15$ for the electron QD and $g^\text{h}_\text{v}=17$ for the hole QD), resulting in a smaller splitting of both, the $\alpha$ and $\beta$ transition, which scales with the difference in the valley g-factors. In \textbf{c} the valley g-factors are chosen symmetrically ($g_\text{v}=15$), and no dependence on $B_\perp$ is observed. In \textbf{d}, the experimentally observed g-factor difference of $g^\text{e}_\text{v}=15$ and $g^\text{h}_\text{v}=15.1$ is used for the simulation.
}
\label{S7}
\end{figure}


To quantitatively estimate the valley g-factor asymmetry, we fit Gaussian peaks with width, $\Gamma$, to the detuning cuts presented in Fig.~3b in the main manuscript, allowing for a constant background and assuming equal width for both peaks, i.e. the $\alpha$ and $\beta$ peak. Such a fit is exemplarily shown in Fig.~S8a. The fitted width of the two peaks increases slightly for increasing $B_\perp$, as shown in Fig~\ref{S8}b. Attributing this effect entirely to a difference of the electron and hole g-factors, we obtain a maximum g-factor difference of $g_\text{v} \approx 0.1$ (c.f. with Fig.~S7d). 


\begin{figure}[!thb]
\centering
\includegraphics[draft=false,keepaspectratio=true,clip,width=0.85\linewidth]{FigS8A.pdf}
\caption[Fig08]{\textbf{a} Exemplary line trace of the tunneling current as a function of the detuning. The sum of two Gauss curves with width $\Gamma$ is fitted to the data (see dashed line). \textbf{b}  $\Gamma$ extracted from the line fits as shown in a, as a function of $B_\perp$. Attributing the linear broadening of $\alpha$ and $\beta$ to an asymmetry of valley g-factors between electron and hole QD yields $\Delta g \approx 0.11$ .
}
\label{S8}
\end{figure}

% \bibliography{Literature}

%merlin.mbs apsrev4-1.bst 2010-07-25 4.21a (PWD, AO, DPC) hacked
%Control: key (0)
%Control: author (8) initials jnrlst
%Control: editor formatted (1) identically to author
%Control: production of article title (-1) disabled
%Control: page (0) single
%Control: year (1) truncated
%Control: production of eprint (0) enabled
\begin{thebibliography}{6}%
\makeatletter
\providecommand \@ifxundefined [1]{%
 \@ifx{#1\undefined}
}%
\providecommand \@ifnum [1]{%
 \ifnum #1\expandafter \@firstoftwo
 \else \expandafter \@secondoftwo
 \fi
}%
\providecommand \@ifx [1]{%
 \ifx #1\expandafter \@firstoftwo
 \else \expandafter \@secondoftwo
 \fi
}%
\providecommand \natexlab [1]{#1}%
\providecommand \enquote  [1]{``#1''}%
\providecommand \bibnamefont  [1]{#1}%
\providecommand \bibfnamefont [1]{#1}%
\providecommand \citenamefont [1]{#1}%
\providecommand \href@noop [0]{\@secondoftwo}%
\providecommand \href [0]{\begingroup \@sanitize@url \@href}%
\providecommand \@href[1]{\@@startlink{#1}\@@href}%
\providecommand \@@href[1]{\endgroup#1\@@endlink}%
\providecommand \@sanitize@url [0]{\catcode `\\12\catcode `\$12\catcode
  `\&12\catcode `\#12\catcode `\^12\catcode `\_12\catcode `\%12\relax}%
\providecommand \@@startlink[1]{}%
\providecommand \@@endlink[0]{}%
\providecommand \url  [0]{\begingroup\@sanitize@url \@url }%
\providecommand \@url [1]{\endgroup\@href {#1}{\urlprefix }}%
\providecommand \urlprefix  [0]{URL }%
\providecommand \Eprint [0]{\href }%
\providecommand \doibase [0]{http://dx.doi.org/}%
\providecommand \selectlanguage [0]{\@gobble}%
\providecommand \bibinfo  [0]{\@secondoftwo}%
\providecommand \bibfield  [0]{\@secondoftwo}%
\providecommand \translation [1]{[#1]}%
\providecommand \BibitemOpen [0]{}%
\providecommand \bibitemStop [0]{}%
\providecommand \bibitemNoStop [0]{.\EOS\space}%
\providecommand \EOS [0]{\spacefactor3000\relax}%
\providecommand \BibitemShut  [1]{\csname bibitem#1\endcsname}%
\let\auto@bib@innerbib\@empty
%</preamble>
\bibitem [{\citenamefont {Banszerus}\ \emph {et~al.}(2021)\citenamefont
  {Banszerus}, \citenamefont {M{\ifmmode\ddot{o}\else\"{o}\fi}ller},
  \citenamefont {Steiner}, \citenamefont {Icking}, \citenamefont {Trellenkamp},
  \citenamefont {Lentz}, \citenamefont {Watanabe}, \citenamefont {Taniguchi},
  \citenamefont {Volk},\ and\ \citenamefont {Stampfer}}]{Banszerus2021Sep}%
  \BibitemOpen
  \bibfield  {author} {\bibinfo {author} {\bibfnamefont {L.}~\bibnamefont
  {Banszerus}}, \bibinfo {author} {\bibfnamefont {S.}~\bibnamefont
  {M{\ifmmode\ddot{o}\else\"{o}\fi}ller}}, \bibinfo {author} {\bibfnamefont
  {C.}~\bibnamefont {Steiner}}, \bibinfo {author} {\bibfnamefont
  {E.}~\bibnamefont {Icking}}, \bibinfo {author} {\bibfnamefont
  {S.}~\bibnamefont {Trellenkamp}}, \bibinfo {author} {\bibfnamefont
  {F.}~\bibnamefont {Lentz}}, \bibinfo {author} {\bibfnamefont
  {K.}~\bibnamefont {Watanabe}}, \bibinfo {author} {\bibfnamefont
  {T.}~\bibnamefont {Taniguchi}}, \bibinfo {author} {\bibfnamefont
  {C.}~\bibnamefont {Volk}}, \ and\ \bibinfo {author} {\bibfnamefont
  {C.}~\bibnamefont {Stampfer}},\ }\href {\doibase 10.1038/s41467-021-25498-3}
  {\bibfield  {journal} {\bibinfo  {journal} {Nat. Commun.}\ }\textbf {\bibinfo
  {volume} {12}},\ \bibinfo {pages} {5250} (\bibinfo {year}
  {2021})}\BibitemShut {NoStop}%
\bibitem [{\citenamefont {Knothe}\ \emph {et~al.}(2022)\citenamefont {Knothe},
  \citenamefont {Glazman},\ and\ \citenamefont {Fal{'}ko}}]{Knothe2022Apr}%
  \BibitemOpen
  \bibfield  {author} {\bibinfo {author} {\bibfnamefont {A.}~\bibnamefont
  {Knothe}}, \bibinfo {author} {\bibfnamefont {L.~I.}\ \bibnamefont {Glazman}},
  \ and\ \bibinfo {author} {\bibfnamefont {V.~I.}\ \bibnamefont {Fal{'}ko}},\
  }\href {\doibase 10.1088/1367-2630/ac5d00} {\bibfield  {journal} {\bibinfo
  {journal} {New J. Phys.}\ }\textbf {\bibinfo {volume} {24}},\ \bibinfo
  {pages} {043003} (\bibinfo {year} {2022})}\BibitemShut {NoStop}%
\bibitem [{Note1()}]{Note1}%
  \BibitemOpen
  \bibinfo {note} {$\Delta ^\protect \text {t}_\protect \mathrm {SO}$, $\Delta
  ^b_\protect \mathrm {SO}$ and $\lambda _\protect \text {ex}$ correspond to
  $\lambda _{I1}$ and $\lambda '_{I1}$ and $\lambda _3$ in Ref. \cite
  {Konschuh2012Mar}}\BibitemShut {NoStop}%
\bibitem [{\citenamefont {Konschuh}\ \emph {et~al.}(2012)\citenamefont
  {Konschuh}, \citenamefont {Gmitra}, \citenamefont {Kochan},\ and\
  \citenamefont {Fabian}}]{Konschuh2012Mar}%
  \BibitemOpen
  \bibfield  {author} {\bibinfo {author} {\bibfnamefont {S.}~\bibnamefont
  {Konschuh}}, \bibinfo {author} {\bibfnamefont {M.}~\bibnamefont {Gmitra}},
  \bibinfo {author} {\bibfnamefont {D.}~\bibnamefont {Kochan}}, \ and\ \bibinfo
  {author} {\bibfnamefont {J.}~\bibnamefont {Fabian}},\ }\href {\doibase
  10.1103/PhysRevB.85.115423} {\bibfield  {journal} {\bibinfo  {journal} {Phys.
  Rev. B}\ }\textbf {\bibinfo {volume} {85}},\ \bibinfo {pages} {115423}
  (\bibinfo {year} {2012})}\BibitemShut {NoStop}%
\bibitem [{\citenamefont {McCann}\ and\ \citenamefont
  {Koshino}(2013)}]{McCann2013Apr}%
  \BibitemOpen
  \bibfield  {author} {\bibinfo {author} {\bibfnamefont {E.}~\bibnamefont
  {McCann}}\ and\ \bibinfo {author} {\bibfnamefont {M.}~\bibnamefont
  {Koshino}},\ }\href {\doibase 10.1088/0034-4885/76/5/056503} {\bibfield
  {journal} {\bibinfo  {journal} {Rep. Prog. Phys.}\ }\textbf {\bibinfo
  {volume} {76}},\ \bibinfo {pages} {056503} (\bibinfo {year}
  {2013})}\BibitemShut {NoStop}%
\bibitem [{\citenamefont {Banszerus}\ \emph {et~al.}(2020)\citenamefont
  {Banszerus}, \citenamefont {Frohn}, \citenamefont {Fabian}, \citenamefont
  {Somanchi}, \citenamefont {Epping}, \citenamefont
  {M{\ifmmode\ddot{u}\else\"{u}\fi}ller}, \citenamefont {Neumaier},
  \citenamefont {Watanabe}, \citenamefont {Taniguchi}, \citenamefont {Libisch},
  \citenamefont {Beschoten}, \citenamefont {Hassler},\ and\ \citenamefont
  {Stampfer}}]{Banszerus2020May}%
  \BibitemOpen
  \bibfield  {author} {\bibinfo {author} {\bibfnamefont {L.}~\bibnamefont
  {Banszerus}}, \bibinfo {author} {\bibfnamefont {B.}~\bibnamefont {Frohn}},
  \bibinfo {author} {\bibfnamefont {T.}~\bibnamefont {Fabian}}, \bibinfo
  {author} {\bibfnamefont {S.}~\bibnamefont {Somanchi}}, \bibinfo {author}
  {\bibfnamefont {A.}~\bibnamefont {Epping}}, \bibinfo {author} {\bibfnamefont
  {M.}~\bibnamefont {M{\ifmmode\ddot{u}\else\"{u}\fi}ller}}, \bibinfo {author}
  {\bibfnamefont {D.}~\bibnamefont {Neumaier}}, \bibinfo {author}
  {\bibfnamefont {K.}~\bibnamefont {Watanabe}}, \bibinfo {author}
  {\bibfnamefont {T.}~\bibnamefont {Taniguchi}}, \bibinfo {author}
  {\bibfnamefont {F.}~\bibnamefont {Libisch}}, \bibinfo {author} {\bibfnamefont
  {B.}~\bibnamefont {Beschoten}}, \bibinfo {author} {\bibfnamefont
  {F.}~\bibnamefont {Hassler}}, \ and\ \bibinfo {author} {\bibfnamefont
  {C.}~\bibnamefont {Stampfer}},\ }\href {\doibase
  10.1103/PhysRevLett.124.177701} {\bibfield  {journal} {\bibinfo  {journal}
  {Phys. Rev. Lett.}\ }\textbf {\bibinfo {volume} {124}},\ \bibinfo {pages}
  {177701} (\bibinfo {year} {2020})}\BibitemShut {NoStop}%
\end{thebibliography}%


\end{document}