\begin{figure}
    \centering
    \setlength{\tabcolsep}{0pt}
    \renewcommand{\arraystretch}{1}
    {\small
    \begin{tabular}{c c c}
        \includegraphics[width=0.33\linewidth]{figures/ablation/dining_proxy.jpg} & 
        \includegraphics[width=0.33\linewidth]{figures/ablation/dining_no_scene.jpg} & 
        \includegraphics[width=0.33\linewidth]{figures/ablation/dining_with_scene.jpg} \\ 
        \includegraphics[width=0.33\linewidth]{figures/ablation/garage_proxy.jpg} & 
        \includegraphics[width=0.33\linewidth]{figures/ablation/garage_no_scene.jpg} & 
        \includegraphics[width=0.33\linewidth]{figures/ablation/garage_with_scene.jpg}\\
        \includegraphics[width=0.33\linewidth]{figures/ablation/living_room_proxy.jpg} & 
        \includegraphics[width=0.33\linewidth]{figures/ablation/living_no_scene.jpg} & 
        \includegraphics[width=0.33\linewidth]{figures/ablation/living_with_scene.jpg}  \\
        & Local & Global-Local \\
        & Training & Training

    \end{tabular}}
    \caption{{\bf Impact of Global-Local training.}
    Here we show the importance of global training for three different scenes. Observe how our Global-Local training policy results in more coherent scenes, compared to scenes where each object is optimized independently that are inconsistent.} 
    \label{fig:ablation}
\end{figure} 



