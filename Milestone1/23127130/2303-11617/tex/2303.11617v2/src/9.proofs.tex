\subsection{Proof of Lemma \ref{lem:bound}}

\begin{proof}
    Let $f \in \mathcal{A}$ and $I \in P(f)$ be a compatible interval for the function $f$. If $f$ is linear in $I$, then for all $x < y \in I$, $T[f, x]$ coincides with $f$ on $\cc{x}{y}$, and so do $T[f, y]$ and $T[f, x, y]$. Besides $f'' = 0$ so inequalities \eq{bound_T} and \eq{bound_DT} hold trivially. We thus suppose that $f$ is either strictly convex or strictly concave in $I$, which means that $f''$ is zero at the two ends of $I$ and nowhere else in $I$.

    For $\delta \geq 0$, we define the set $E_\delta = \{(x, y) \in I^2, y - x \geq \delta\}$. We also write $E = E_0$. Let now $\delta > 0$ be fixed. For all $(x, y) \in E_\delta$, the quantity $\aabs{f''}_{L^{p/(2p+1)}(\oo{x}{y})}$ is not zero since $y > x$ and the integrand only vanishes on a set of zero measure. This enables us to consider the function
    \[r: E_\delta \ni (x, y) \mapsto \frac{\aabs{f - T[f, x, y]}_{L^p(\oo{x}{y})}}{\aabs{f''}_{L^{p/(2p+1)}(\oo{x}{y})}}.\]
    We want to show that $r$ can be defined on $E$ as a whole, and that $r$ is bounded on $E$.

        {\itshape (i) Continuity and differentiability of $c$ on $E_\delta$.}
    By definition of $I$, for all $x < y \in I$, the tangents to $f$ at $x$ and $y$ intersect at an abscissa that we write $c(x, y)$, with $x < c(x, y) < y$. More precisely, for finite $x < y \in I$, the function $c$ has the following expression
    \[c(x, y) = -\frac{[f(y) - y f'(y)] - [f(x) - x f'(x)]}{f'(y) - f'(x)},\]
    in which the denominator is not zero because $f'$ is injective in $I$. We extend $c$ to possible infinite values of $x$ or $y$ by setting $c(x, +\infty) = \lim_{y \to +\infty} c(x, y)$ and $c(-\infty, y) = \lim_{x \to -\infty} c(x, y)$. Since $f$ and $f'$ are continuous on $I$, and since $f'$ is injective on $I$, we conclude that $c$ is continuous on $E_\delta$. By using the differentiation rule for quotients, similar arguments show that $c$ is differentiable on $E_\delta$.

        {\itshape (ii) Continuity of $r$ on $E_\delta$.}
    Since $f$ and $f'$ are continuous on $\RR$, the map $x \mapsto T[f, x]$ is continuous. Besides, $c$ is continuous so the map $(x, y) \to T[f, x, y]$ is also continuous. Finally, any map $(x, y) \to \int_x^y \phi(t, x, y) \D t$ is continuous whenever $\phi$ is continuous. Here, $(x, y) \mapsto f - T[f, x, y]$ and $f''$ are continuous, so we conclude that the numerator and the denominator of $r$ are continuous. Moreover, since $y - x \geq \delta > 0$, the denominator of $r$ is not zero, which makes $r$ continuous on $E_\delta$.

    The fact that $f$ belongs to $\mathcal{A}$ ensures that the numerator of $r$ is bounded. Besides, the denominator is a strictly increasing function of $y - x$, which is bounded from below by $\delta \aabs{f''}_{L^\infty(\oo{x}{y})}$, and from above because $f''$ is in $L^{p/(2p+1)}(\RR)$. This proves that $r(x, y)$ is bounded on $E_\delta$. Thus the only case where $r$ could be unbounded is when $x$ and $y$ get arbitrarily close ($\delta$ goes to zero).

        {\itshape (iii) Continuity and differentiability of $c$ on $E$.}
    We now show that $c$ can be continuously extended to $E$ and that this extension is differentiable on $E$. To do so, we prove that for all element $x \in I$, $\lim_{y \to x} c(x, y) = x$. One can prove that for all $y \in I$, $\lim_{x \to y} c(x, y) = y$ using a similar argument. Let $\epsilon \leq y - x$, so that $x + \epsilon \in I$. We start by rearranging the terms in the expression of $c(x, x + \epsilon)$ and obtain the following

    \[c(x, x + \epsilon) = x + \frac{f'(x + \epsilon) - \epsilon^{-1} (f(x + \epsilon) - f(x))}{f'(x + \epsilon) - f'(x)} \epsilon.\]
    Now, let $k \geq 2$ be the smallest integer such that the $k$-th derivative of $f$ at $x$ is not zero. By definition of a compatible interval, $k$ will be equal to $2$ for all interior points of $I$. Since $f$ is analytical in the neighbourhood of the zeros of $f''$, there also exists such a $k > 2$ for the ends of $I$. We can always take $\epsilon$ as small as needed so that $f$ is analytical around $x$ in a neighbourhood of radius $\epsilon$. We obtain a Taylor expansion of the fraction above by expanding its numerator and denominator to order $k-1$ around $x$, expressed at $x + \epsilon$, and arrive at
    \[c(x, x + \epsilon) = x + \frac{k - 1}{k} \epsilon + O(\epsilon^2).\]
    This shows that the function $\hat{c}$ defined by $\hat{c}(x, y) = c(x, y)$ if $x < y$ and $\hat{c}(x, x) = x$ is continuous and differentiable on $E$.

        {\itshape (iv) Continuity of $r$ on $E$.}
    Let $x \in I$. We show that $\epsilon \mapsto r(x, x + \epsilon)$ has a finite limit when $\epsilon$ goes to zero by finding an equivalent of the numerator and the denominator of $r$. We write the numerator as $(I_1(\epsilon)^p + I_2(\epsilon)^p)^{1/p}$, where $I_1$ and $I_2$ are the integrals on $\cc{x}{c(x, x+\epsilon)}$ and $\cc{c(x, x+\epsilon)}{x + \epsilon}$ respectively. The denominator of $r$ is written $I_3(\epsilon)$. Here again, let $k \geq 2$ be the smallest integer such that the $k$-th derivative of $f$ at $x$ is not zero.

    We use the mean-value form of the remainder in the $k$-th-order Taylor expansion of $f$ to obtain that for all $t \in \oo{x}{c(x, x + \epsilon)}$, there exists $\eta_t \in \cc{x}{t}$ such that $f(t) - T[f, x](t) = \frac{1}{k!} f^{(k)}(\eta_t) (t - x)^k$. Then, owing to the mean-value theorem for integrals since $f^{(k)}$ is continuous, there exists $\eta \in \oo{x}{c(x, x + \epsilon)}$ such that
    \[I_1(\epsilon)^p = \int_{x}^{c(x, x + \epsilon)} \frac{1}{k!} |f^{(k)}(\eta_t) (t - x)^k|^p \D t = \frac{1}{k!^p (kp+1)} |f^{(k)}(\eta)|^p (c(x, x + \epsilon) - x)^{kp+1}.\]
    We can now use the previous Taylor expansion of $c$ around $(x, x)$ and also expand $f^{(k)}(\eta)$ around $x$ since $\eta$ goes to $x$ as $\epsilon$ approaches zero. We finally obtain the following expansion for $I_1(\epsilon)^p$
    \[I_1(\epsilon)^p = \frac{1}{k!^p (kp+1)} \left(\frac{k-1}{k}\right)^{kp+1} |f^{(k)}(x)|^p \epsilon^{kp+1} + o(\epsilon^{kp+1}).\]

    We obtain a $k$-th order Taylor expansion of $f - T[f, x + \epsilon]$ as follows: there exist $\eta_1, \eta_2 \in \oo{x}{x + \epsilon}$, such that for all $t \in \oo{c(x, x + \epsilon)}{x + \epsilon}$, there exists $\eta_t$ between $x$ and $t$ such that
    \[f(t) - T[f, x + \epsilon](t) = \frac{1}{k!} \left[f^{(k)}(\eta_t) (t - x)^k - k f^{(k)}(\eta_2) \epsilon^{k-1} (t - x) + k f^{(k)}(\eta_2) \epsilon^k - f^{(k)}(\eta_1) \epsilon^k\right] + o(\epsilon^k).\]
    We observe that $\eta_1$, $\eta_2$ and $\eta_t$ go to $x$ as $\epsilon$ approaches zero. This means that the Taylor expansion of $f^{(k)}(\eta_t)$ is $f^{(k)}(x) + O(\epsilon)$, and similarly at $\eta_1$ and $\eta_2$. Besides, using the expansion of $c(x, x+ \epsilon)$, we show that $x + \frac{k-1}{k} \epsilon + o(\epsilon) \leq t \leq x + \epsilon$, which means that $t - x$ is of the order of $\epsilon$. After keeping the terms of order $\epsilon^k$, factoring by $f^{(k)}(x) \epsilon^k$ and introducing the change of variable $u = \frac{t - x}{\epsilon}$, we obtain
    \[d_2(t) = \frac{f^{(k)}(x)}{k!} \left[u^k - k u + k - 1\right] \epsilon^k + o(\epsilon^k).\]
    The bounds of the integral become $\frac{k-1}{k} + o(1)$ and $1$, and we decompose the integration domain into $\cc{\frac{k-1}{k}}{1}$ and $\cc{\frac{k-1}{k} + o(1)}{\frac{k-1}{k}}$. The second integral is going to be negligible against the first because its integrand is bounded and its integration domain has a length of $o(1)$. Putting everything together, we obtain the following expansion for $I_2(\epsilon)^p$
    \[I_2(\epsilon)^p = \frac{1}{k!^p} \left(\int_{\frac{k-1}{k}}^{1} \abs{u^k - 1 - k (u - 1)}^p \D u\right) |f^{(k)}(x)|^p \epsilon^{kp+1} + o(\epsilon^{kp+1}).\]

    Using a similar argument, the Taylor expansion of the integrand of the denominator of $r$ is $\frac{1}{(k-2)!} f^{(k)}(x)(t - x)^{k-2} + o((t - x)^{k-2})$ and taking the $L^{p/(2p+1)}$ norm of this relationship, we find that the expansion of $I_3(\epsilon)^p$ is
    \[I_3(\epsilon)^p = \frac{1}{(k-2)!^p} \left(\frac{2p+1}{kp+1}\right)^{2p+1} |f^{(k)}(x)|^p \epsilon^{kp+1} + o(\epsilon^{kp+1}).\]
    Altogether, in the expression of $r(x, x + \epsilon)^p$, we can factor the numerator and the denominator by $|f^{(k)}(x)|^p \epsilon^{kp+1}$ and taking the limit as $\epsilon$ goes to zero, we obtain
    \[\lim_{\epsilon \to 0} r(x, x + \epsilon)^p = \frac{1}{(k(k-1))^p} \left(\frac{kp+1}{2p+1}\right)^{2p+1} \left(\frac{1}{kp+1} \left(\frac{k-1}{k}\right)^{kp+1} + \int_{\frac{k-1}{k}}^{1} \abs{u^k - 1 - k (u - 1)}^p \D u\right).\]
    This shows that the function $\hat{r}$ defined by $\hat{r}(x, y) = r(x, y)$ if $x < y$ and $\hat{r}(x, x)$ as the quantity above is continuous on $E$ as a whole. Besides $\hat{r}$ has finite limits on $\partial E$, so we conclude that it is bounded in $E$.

        {\itshape (v) Proof of the other inequalities.}
    The same proof can be adapted for the bound on the derivatives, and the bound in $L^\infty$ norms. The arguments are the same everywhere, except for the boundedness around the points of the type $(x, x)$, where other equivalents are found. It is still possible to define the corresponding function $r$ on $E$ as a whole and to show that it is bounded in $E$.
\end{proof}

\subsection{Proof of Proposition \ref{prop:cpwl}}

\begin{proof}
    This proof is inspired by the heuristics given in \cite{berjon2015}. We keep the notations of the definition in \subsect{approximation_space}. Let $I$ be a compatible interval of $f$. If $f$ is linear on $I$, then $f$ coincides with its tangent at either end of $I$ so it is enough to define the restriction of $\pi_{n, p}[f]$ to $I$ as this tangent. We thus decompose the total squared $L^p$ norm on the $\kappa(f)$ compatible intervals of $f$ where $f$ is not linear. We decide to place $n' = \floor{n/\kappa(f)}$ points in each interval.

    Let $I = \cc{a}{b}$ be a compatible interval for $f$ where $f$ is not linear. We write $a < \xi_1 < \ldots < \xi_{n'} < b$ the free points in $I$. For convenience we introduce $\xi_0 = a$ and $\xi_{n' + 1} = b$. We split the integral on $I$ on the intervals $\cc{\xi_i}{\xi_{i+1}}$ and using \eq{bound_T} from \lem{bound}, we readily have
    \[\aabs{f - \pi_{n, p}[f]}_{L^p(I)}^p = \sum_{i = 0}^{n'} \aabs{f - T[f, \xi_i, \xi_{i+1}]}_{L^p(\cc{\xi_i}{\xi_{i+1}})}^p \leq A_p(f)^p \sum_{i = 0}^{n'} \aabs{f''}_{L^{p/(2p+1)}(\cc{\xi_i}{\xi_{i+1}})}^{2p+1}.\]
    We now define the $(\xi_i)_{1 \leq i \leq n'}$ such that $\aabs{f''}_{L^{p/(2p+1)}(\cc{\xi_i}{\xi_{i+1}})}^{p/(2p+1)}$ are all equal to the same constant that we write $C_I$. By the linearity of the integral, we find that $(n' + 1) C_I = \aabs{f''}_{L^{p/(2p+1)}(I)}^{p/(2p+1)}$ and $\xi_i$ is such that
    \[\int_{\xi_0}^{\xi_i} \abs{f''(t)}^{p/(2p+1)} \D t = i C_I = \frac{i }{n' + 1} \int_I \abs{f''(t)}^{p/(2p+1)} \D t.\]
    The sum above evaluates to $(n' + 1) C_I^{2p+1} = (n' + 1)^{-2p} \aabs{f''}_{L^{p/(2p+1)}(I)}^{p}$. We apply the same method in all the compatible intervals for $f$ that we write $I_k$ for $1 \leq k \leq \kappa(f)$. Since $2/5 \leq p/(2p+1) \leq 1/2$ and $f'' \in L^q$ for all $q \geq 2/5$, we bound the $L^{p/(2p+1)}$ norm of $f''$ on each $I_k$ by that on $\RR$ as a whole. We finally obtain
    \begin{align*}
        \aabs{f - \pi_{n, p}[f]}_{L^p(\RR)}^p & = \sum_{k = 1}^{\kappa(f)} \aabs{f - \pi_{n, p}[f]}_{L^p(I_k)}^p \leq \sum_{k = 1}^{\kappa(f)} A_p(f)^p (n'+1) C_{I_k}^{2p+1}                                \\
                                              & \leq \kappa(f) \frac{A_p(f)^p \aabs{f''}_{L^{p/(2p+1)}(\RR)}^p}{(n'+1)^{2p}} \leq \frac{A_p(f)^p \kappa(f)^{2p+1} \aabs{f''}_{L^{p/(2p+1)}(\RR)}^p}{n^{2p}},
    \end{align*}
    where we obtained the last inequality by multiplying the numerator and denominator by $\kappa(f)^{2p}$ and used the fact that for all $m, n > 0$, $m \floor{n/m} \geq n - m$. We apply the same method to $f'$ using \eq{bound_T} from \lem{bound}. In each $I_k$, we sample the $(\xi_i)_{1 \leq i \leq n'}$ such that $\int_{\xi_0}^{\xi_i} \abs{f''(t)}^{p/(p+1)} \D t = i D_{I_k}$ where $(n' + 1) D_{I_k} = \aabs{f''}_{L^{p/(p+1)}(I_k)}^{p/(p+1)}$. Then it holds
    \[\aabs{f - \pi_{n, p}[f]}_{L^p(\RR)}^p \leq \sum_{k=1}^{\kappa(f)} B_p(f)^p (n'+1) D_{I_k}^{p+1} \leq \frac{B_p(f)^p \kappa(f)^{p+1} \aabs{f''}_{L^{p/(p+1)}(\RR)}^p}{n^{p}}.\]
    The same bounds are obtained for the $L^\infty$ norms, by considering the maximum $L^\infty$ norm over the $n'+1$ subsegments and sampling the $(\xi_i)$ according to $\int_{\xi_0}^{\xi_i} \abs{f''}^{1/2}$ for the first bound and $\int_{\xi_0}^{\xi_i} \abs{f''}$ for the second bound.
\end{proof}

\subsection{Proof of Lemma \ref{lem:convexity}}

\begin{proof}
    We know that the cells in $\tau_{\RR^d}(\pmb{\Theta})$ are all convex since they are defined as the interior of intersections of half-planes. The final mesh $\tau_\Omega(\pmb{\Theta})$ is the intersection of $\tau_{\RR^d}(\pmb{\Theta})$ against $\Omega$. Let $K \in \tau_\Omega(\pmb{\Theta})$. There exists $K' \in \tau_{\RR^d}(\pmb{\Theta})$ such that $K = K' \cap \Omega$. If $K' \subseteq \Omega$, then $K = K'$ is convex. By the contraposition principle, if $K$ is not convex, then $K' \nsubseteq  \Omega$. This shows that if $K$ is a non-convex cell in $\tau_\Omega(\pmb{\Theta})$, then $K$ intersects with both $\Omega$ and $\RR^d \backslash \Omega$, which means that $K$ intersects with the boundary of $\Omega$. In the case where $\Omega$ is convex, the intersection of two convex sets being convex implies that $K = K' \cap \Omega$ is always convex.
\end{proof}

\subsection{Proof of Proposition \ref{prop:nns}}

\begin{proof}
    We show the two bounds \eq{nns_1} and \eq{nns_2} on a single-layer network, and extend them to arbitrary \acp{nn} by induction. We also prove that $\aabs{\nabla u}_{L^\infty(\Omega)} \leq C_u \aabs{f'}_{L^\infty(\RR)}^\alpha$ and $\aabs{\nabla v}_{L^\infty(\Omega)} \leq C_v \aabs{g'}_{L^\infty(\RR)}^\alpha$ for some constants $C_u$ and $C_v$ that only depend on $\pmb{\Theta}$, and some integer $\alpha$.

    To simplify notations, we write $\aabs{\cdot}_\Omega$ instead of $\aabs{\cdot}_{L^\infty(\Omega)}$ and similarly $\aabs{\cdot}_\RR$ instead of $\aabs{\cdot}_{L^\infty(\RR)}$. For an affine map $\pmb{\Theta}$ associated with a matrix $\pmb{W}$, we write $\aabs{\pmb{\Theta}}_\infty = \max_{i, j} \abs{\pmb{W}_{i, j}}$. In this proof, we write $C_1$, $C_2$, $C_3$, $C_u$, $C_v$ and $\alpha$ the constants in the hypothesis of the induction and write the new constants after the induction with a prime symbol.

        {\itshape (i) Initialisation.} We take an affine map $\pmb{\Theta}$ from $\RR^d$ to $\RR^m$ associated with a weight matrix $\pmb{W} \in \RR^{m \times d}$ and we consider the functions $u = f \circ \pmb{\Theta}$ and $v = g \circ \pmb{\Theta}$. The partial derivatives of $u$ are given by $\partial_j u_i = \pmb{W}_{i, j} f' \circ \pmb{\Theta}_i$. We readily obtain that $\aabs{u_i - v_i}_\Omega \leq \aabs{f - g}_\RR$ and $\aabs{\partial_j u_i - \partial_j v_i}_\Omega \leq \abs{\pmb{W}_{i, j}} \aabs{f' - g'}_\RR$. We also notice that $\partial_j u_i$ is bounded by $\abs{\pmb{W}_{i, j}} \aabs{f'}_\RR$. Similarly, $\partial_j v_i$ is bounded by an analogous constant. We conclude by taking the maximum on $1 \leq i \leq m$ and $1 \leq j \leq d$. We can take $C_1 = 1$, $C_2 \geq 0$, $C_3 = C_u = C_v = \aabs{\pmb{\Theta}}_\infty$ and $\alpha = 1$.

        {\itshape (ii) Induction with affine map.} Let $u$ and $v$ be two functions from $\RR^d$ to $\RR^n$ such that the two bounds hold and $\nabla u$ and $\nabla v$ are bounded in $\Omega$. Let $\pmb{\Theta}$ be an affine map from $\RR^n$ to $\RR^m$. We want to show that the two bounds still hold for $\pmb{\Theta} \circ u - \pmb{\Theta} \circ v$. We use the triangle inequality and the induction hypothesis to obtain that
    \[\aabs{(\pmb{\Theta} \circ u)_i - (\pmb{\Theta} \circ v)_i}_\Omega \leq \sum_{k=1}^{n} \abs{\pmb{W}_{i, k}} \aabs{u_k - v_k}_\Omega \leq n C_1 \aabs{\pmb{\Theta}}_\infty \aabs{f - g}_\RR.\]
    The partial derivatives of $\pmb{\Theta} \circ u$ are given by $\partial_j (\pmb{\Theta} \circ u)_i = \sum_{k=1}^n \pmb{W}_{i, k} \partial_j u_k$. Using the triangle inequality and the induction hypothesis, we arrive at
    \[\aabs{\partial_j (\pmb{\Theta} \circ u)_i - \partial_j (\pmb{\Theta} \circ v)_i}_\Omega \leq \sum_{k=1}^n \abs{\pmb{W}_{i, k}} \aabs{\partial_j u_k - \partial_j v_k}_\Omega \leq n \aabs{\pmb{\Theta}}_\infty (C_1 \aabs{f - g}_\RR + C_2 \aabs{f' - g'}_\RR).\]
    We verify that $\aabs{\partial_j (\pmb{\Theta} \circ u)_i}_\Omega \leq C_u \sum_{k = 1}^{n} \abs{\pmb{W}_{i, k}} \aabs{f'}_\RR^\alpha$ and the same argument shows that $\partial_j v_i$ is bounded by a similar constant. We conclude by taking the maximum on $1 \leq i \leq m$ and $1 \leq j \leq d$. We can take $C_1' = C_2' = n C_1 \aabs{\pmb{\Theta}}_\infty$, $C_3' = n C_2 \aabs{\pmb{\Theta}}_\infty$, $C_u' = n C_u \aabs{\pmb{\Theta}}_\infty$, $C_v' = n C_v \aabs{\pmb{\Theta}}_\infty$ and $\alpha' = \alpha$.

        {\itshape (iii) Induction with activation.} Let $u$ and $v$ be two functions from $\RR^d$ to $\RR^n$ such that the two bounds hold and $\nabla u$ and $\nabla v$ are bounded in $\Omega$. We want to show that the two bounds still hold for $f \circ u - g \circ v$. Using the triangle inequality, the fact that $f$ is Lipschitz and the induction hypothesis, we obtain
    \begin{align*}
        \aabs{(f \circ u)_i - (g \circ v)_i}_\Omega & \leq \aabs{(f \circ u)_i - (f \circ v)_i}_\Omega + \aabs{(f \circ v)_i - (g \circ v)_i}_\Omega \\
                                                    & \leq (C_1 \lip(f) + 1) \aabs{f - g}_\RR.
    \end{align*}
    The partial derivatives of $f \circ u$ are given by $\partial_j (f \circ u)_i = (f' \circ u_i) \partial_j u_i$. The triangle inequality leads to
    \begin{align*}
        \aabs{\partial_j (f \circ u)_i - \partial_j (g \circ v)_i}_\Omega & \leq \aabs{(f' \circ u_i) \partial_j u_i - (f' \circ u_i) \partial_j v_i}_\Omega + \aabs{(f' \circ u_i) \partial_j v_i - (f' \circ v_i) \partial_j v_i}_\Omega \\
                                                                          & + \aabs{(f' \circ v_i) \partial_j v_i - (g' \circ v_i) \partial_j v_i}_\Omega.
    \end{align*}
    We bound the first term by factorising by $f' \circ u_i$ and the induction hypothesis. We use the fact that $f'$ is Lipschitz and the induction hypothesis to bound the second term. The third term is readily bounded by $\aabs{\nabla v}_\Omega \aabs{f' - g'}_\RR$. Altogether, we obtain the following bound
    \begin{align*}
        \aabs{\partial_j (f \circ u)_i - \partial_j (g \circ v)_i}_\Omega & \leq (C_2 \aabs{f'}_\RR + C_1 C_v \lip(f') \aabs{g'}_\RR^\alpha) \aabs{f - g}_\RR \\
                                                                          & + (C_3 \aabs{f'}_\RR + C_v \aabs{g'}_\RR^\alpha) \aabs{f' - g'}_\RR.
    \end{align*}
    We observe that $\aabs{\partial_j u_i}_\Omega \leq C_u \aabs{f'}_\RR^{\alpha + 1}$ and $\partial_j v_i$ is bounded by a similar constant. We conclude by taking the maximum on $1 \leq i \leq m$ and $1 \leq j \leq d$. We can take $C_1' = C_1 \lip(f) + 1$, $C_2' = C_2 \aabs{f'}_\RR + C_1 C_v \lip(f') \aabs{g'}_\RR^\alpha$, $C_3' = C_3 \aabs{f'}_\RR + C_v \aabs{g'}_\RR^\alpha$, $C_u' = C_u$, $C_v' = C_v$ and $\alpha' = \alpha + 1$.
\end{proof}