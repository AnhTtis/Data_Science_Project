In this section, we suppose that we are given a smooth activation $\rho$ and we consider the \ac{nn} $u = \mathcal{N}(\pmb{\Theta}, \rho)$. We draw inspiration from the \ac{fem} and we would like to construct a mesh of the domain such that the behaviour of the \ac{nn} is close to linear on each cell of the mesh. We design a method that approximates the smooth activation function $\rho$ by a \ac{cpwl} function $\pi[\rho]$ and bound the distance between the two functions.

\subsection{Assumptions on the activation function}

Since we have no prior bound on the parameters of the network, we need the approximation of $\rho$ by $\pi[\rho]$ to hold on $\RR$ as a whole. We impose that $\pi[\rho] - \rho$ is square integrable on $\RR$, which implies that $\rho$ has a linear behaviour at infinity. If $\rho$ is twice differentiable on $\RR$, we know that $\pi[\rho] - \rho$ behaves as $x^2 f''(x)$ at infinity. As a consequence, one way to enforce the integrability condition is to make sure that $x^2 f''(x)$ is of the order of $|x|^{-\alpha}$ for a given $\alpha > 1/2$, which is equivalent to $|x|^\beta f''(x)$ being bounded on $\RR$ for a given $\beta > 5/2$.

As will be made clear in the proof of \lem{bound}, we also need $f$ to be analytical in the neighbourhood of the zeros of $f''$. Altogether, we restrict our approach to functions that belong to the functional space $\mathcal{A}$ defined as
\[\mathcal{A} = \left\{\begin{array}{l}
        f \in \mathcal{C}^2(\RR), \abs{x}^\alpha f''(x) \text{ is bounded for a given } \alpha > 5/2, \\
        f''(x) = 0 \implies f \text{ is analytical in a neighbourhood of } x
    \end{array}\right\}.\]
We point out that for all $f \in \mathcal{A}$, $\pi[f] - f \in L^p(\RR)$ for all $p \geq 2$ and $f'' \in L^p(\RR)$ for all $p \geq 2/5$.

\subsection{Approximation space}
\label{subsect:approximation_space}

In our experiments, we observed that the way we approximate the activation function by a \ac{cpwl} counterpart plays an important role in the performance of our adaptive quadrature. There are existing algorithms and heuristics to build $\pi[\rho]$ \cite{berjon2015} but we found that $\pi[\rho]$ needs to satisfy extra properties for our method to be most efficient in practice. In particular, we noticed that it is preferable that the \ac{cpwl} approximation of $\rho$ lie below $\rho$ in the regions where $\rho$ is convex, and above $\rho$ where $\rho$ is concave. To that aim, we wish to decompose $\RR$ into intervals where $\rho$ is either strictly convex, linear or strictly concave. We can then approximate $\rho$ in each interval by connecting the tangents to $\rho$ at free points. More formally, we define the partition induced by $f$ as follows.

\begin{definition}[Partition induced by a function]
    Let $f \in \mathcal{A}$. We define the set $Z_0(f) = \{I \subset \RR, f''_{|I} = 0\}$ that contains the intervals (possibly isolated points) where $f''$ is zero and $Z(f) = \partial Z_0(f) \cup \{-\infty, +\infty\}$ the set of points where $f''$ is zero, excluding intervals but including $\pm \infty$. We write $z(f) = \card{Z(f)}$ and $\xi_1 < \ldots < \xi_{z(f)}$ the ordered elements of $Z(f)$. The \emph{partition induced by $f$} is defined as $P(f) = \{[\xi_i, \xi_{i+1}], 1 \leq i \leq z(f) - 1\}$, that is the collection of intervals whose ends are two consecutive elements of $Z(f)$. The elements of $P(f)$ are called \emph{compatible intervals for $f$}. We finally introduce $1 \leq \kappa(f) \leq z(f) - 1$ the number of segments in $P(f)$ where $f$ is not linear.
\end{definition}

By way of example, the second derivative of $\abse$ is always positive so $Z(\abse) = \varnothing$ and $P(\abse) = \{\RR\}$. However, $Z(\tanh) = \{0\}$ and $P(\tanh) = \{\RR_-, \RR_+\}$.

For $x \in \RR$, we write $T[f, x]$ the tangent to $f$ at $x$ and we extend this notation to $x = \pm \infty$ by considering the asymptotic tangents to $f$. From the construction of $P(f)$, it is easy to see that for all $I \in P(f)$ and $x < y \in I$, the two tangents $T[f, x]$ and $T[f, y]$ intersect at a unique $z \in \oo{x}{y}$. We write $T[f, x, y]$ the \ac{cpwl} function defined on $\cc{x}{y}$ as $T[f, x]$ on $\cc{x}{z}$ and $T[f, y]$ on $\cc{z}{y}$.

We are now ready to define our approximation space. For $n \geq z(f)$, we define the space $\mathcal{A}_n(f)$ of \ac{cpwl} functions that connect the tangents to $f$ at $Z(f)$ as well as at $n - z(f)$ other free points that do not belong to $Z(f)$. As an example, the function plotted in \fig{cpwl_tanh} belongs to $\mathcal{A}_7(\tanh)$. There are four free points because $z(\tanh) = 3$. The square points are fixed and they are located at $\pm \infty$ and zero.

\subsection{Convergence rate}

For $f \in \mathcal{A}$, $n \geq z(f)$ and $p \geq 2$, we write $\pi_{n, p}[f]$ the best approximation of $f$ in $\mathcal{A}_n(f)$ in the $L^p$ norm. In this paragraph, we bound the distance between $f$ and $\pi_{n, p}[f]$, and determine the convergence rate of this approximation. We start by finding an upper bound for the distance between $f$ and $T[f, x, y]$ for $x < y \in I$, where $I$ is a compatible interval for $f$.

\begin{lemma}
    \label{lem:bound}
    Let $f \in \mathcal{A}$ and $p \in \co{2}{\infty}$. There exist constants $A_p(f)$, $A_\infty(f)$, $B_p(f)$ and $B_\infty(f)$ such that for all interval $I$ compatible for $f$ and $J = \cc{x}{y} \subset I$, it holds
    \begin{align}
        \aabs{f - T[f, x, y]}_{L^p(J)} \leq A_p(f) \aabs{f''}_{L^{p/(2p+1)}(J)},  &  & \aabs{f - T[f, x, y]}_{L^\infty(J)} \leq A_\infty(f) \aabs{f''}_{L^{1/2}(J)}, \label{eq:bound_T} \\
        \aabs{f' - T[f, x, y]'}_{L^p(J)} \leq B_p(f) \aabs{f''}_{L^{p/(p+1)}(J)}, &  & \aabs{f' - T[f, x, y]'}_{L^\infty(J)} \leq B_\infty(f) \aabs{f''}_{L^1(J)}. \label{eq:bound_DT}
    \end{align}
\end{lemma}

Thanks to this lemma, we can bound the distance between $f$ and $\pi_{n, p}[f]$ and prove a quadratic convergence for all $p$.

\begin{proposition}[Best approximation on $\mathcal{A}_n$]
    \label{prop:cpwl}
    Let $f \in \mathcal{A}$, $n \geq z(f)$ and $p \in \co{2}{\infty}$. Using the constants from \lem{bound}, it holds
        {\small
            \begin{align*}
                \aabs{f - \pi_{n, p}[f]}_{L^p(\RR)} \leq \frac{A_p(f) \kappa(f)^{(2p + 1)/p} \aabs{f''}_{L^{p/(2p+1)}(\RR)}}{n^2}, &  & \aabs{f - \pi_{n, p}[f]}_{L^\infty(\RR)} \leq \frac{A_\infty(f) \kappa(f)^2 \aabs{f''}_{L^{1/2}(\RR)}}{n^2}, \\
                \aabs{f' - \pi_{n, p}[f]'}_{L^p(\RR)} \leq \frac{B_p(f) \kappa(f)^{(p + 1)/p} \aabs{f''}_{L^{p/(p+1)}(\RR)}}{n},   &  & \aabs{f' - \pi_{n, p}[f]'}_{L^\infty(\RR)} \leq \frac{B_\infty(f) \kappa(f) \aabs{f''}_{L^{1}(\RR)}}{n}.
            \end{align*}
        }
\end{proposition}

\subsection{Numerical implementation}

In practice, we choose $p = 2$ and implement the least-square approximation of $\rho$ on $\mathcal{A}_n(\rho)$. We write $\pi_n$ instead of $\pi_{n, 2}$. Since we are searching for the best approximant within $\mathcal{A}_n[\rho]$, the unknowns to be determined are the locations $(\xi_i)_{1 \leq i \leq n}$ of the tangents. We write the total $L^2$ norm between $\rho$ and $\rho_n$ in terms of $\xi_i$ and seek to minimise this quantity via a gradient descent algorithm.

We are interested in two activation functions: $\abse$ and $\tanh$. Both functions are symmetric around the origin, which allows us to restrict the optimisation domain. We also verify that both functions belong to $\mathcal{A}$.
\begin{itemize}
    \item The relationship $\abse(x) - \abse(-x) = x$ holds for all $x \in \RR$. We impose the same condition on $\pi_n[\abse]$ and verify that $\abse - \pi_n[\abse]$ is even. Since $\abse$ takes bounded values on $\RR_-$, we approximate $\abse$ by $\pi_n[\abse]$ on $\RR_-$ only and recover $\pi_n[\abse]$ as a whole by symmetry. The second derivative of $\abse$ is $x \mapsto \frac{1}{2 \gamma} (1 + \overline{x}^2)^{-3/2}$. It follows a polynomial decal rate with $\alpha = 3 > 5/2$. We verify that the second derivative of $\abse$ is never zero. The asymptotic tangents to $\abse$ intersect, so we can approximate $\abse$ with $2$ pieces and this approximation is $\ReLU$.
    \item The $\tanh$ function is odd, so imposing that $\pi_n[\tanh]$ is odd enforces that $\tanh - \pi_n[\tanh]$ is even. We decide to perform the minimisation on $\RR_+$ and obtain an approximation of $\tanh$ on $\RR$ as a whole by symmetry. The second derivative of $\tanh$ is $x \mapsto -2 \tanh(x) (1 - \tanh(x)^2)$. This is asymptotically equivalent to $-8 \exp(-2x)$. In particular $x \mapsto x^3 \tanh''(x)$ is bounded on  $\RR$. Moreover, $\tanh''(x) = 0$ if and only if $x = 0$, and $\tanh$ is analytic in a neighbourhood of zero (its Taylor series converges within a radius of $\pi / 2$). Since the asymptotic tangents to $\tanh$ do not intersect, the minimum number of pieces to approximate $\tanh$ on $\RR$ is three and the corresponding \ac{cpwl} function is known as the hard $\tanh$ activation.
\end{itemize}

As an example, we plot $\pi_7[\tanh]$ in \fig{cpwl_tanh}. The convergence plot in $L^2$ norm of our method applied to both functions is shown in \fig{convergence_rate}, together with the theoretical convergence rate of $-2$. We verify that the convergence is asymptotically quadratic. We notice that the approximation of $\abse$ shows a preasymptotic behaviour in which the effective convergence rate is closer to $1.5$.

\begin{figure}
    \centering
    \subfloat[Visualisation of $\pi_7(\tanh)$.\label{fig:cpwl_tanh}]{
        \includegraphics[width=0.45\linewidth]{cpwl.pdf}
    }
    \hfill
    \subfloat[Convergence plot in $L^2$ norm.\label{fig:convergence_rate}]{
        \includegraphics[width=0.45\linewidth]{convergence.pdf}
    }
    \caption{Visualisation of $\pi_7(\tanh)$ and convergence plot of our proposed method in $L^2$ norm for $\abse$ and $\tanh$. On \protect\subref{fig:mesh_1}, the square pink points are fixed, while the circle green points are chosen so as to minimise the $L^2$ norm of $\tanh - \pi_7[\tanh]$.}
    \label{fig:cpwl_convergence}
\end{figure}
