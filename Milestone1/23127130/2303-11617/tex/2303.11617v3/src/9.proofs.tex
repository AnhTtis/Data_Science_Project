\subsection{Proof of Lemma \ref{lem:integrability}}
\label{proof:lem:integrability}

\begin{proof}
    Let $\alpha > 0$ and $\rho \in \mathcal{A}^\alpha$. We only show the lemma at $+\infty$, as it is obtained by symmetry at $-\infty$ by considering $\tilde{\rho}: x \mapsto \rho(-x)$. Let $M > 0$ and define $I = \oo{M}{+\infty}$. Since $x \mapsto \abs{x}^{2 + \alpha} \rho''(x)$ is bounded on $\RR$, the $p$-test for improper integrals implies that $\rho''$ is integrable on $I$. Besides, for all $x \geq M$, one can write
    $$\rho'(x) = \rho'(x) - \rho'(M) + \rho'(M) = \int_M^x \rho''(t) \D{t} + \rho'(M).$$
    This shows that $\rho'$ has a finite limit at $+\infty$, that we write $A$. Furthermore, it holds
    $$\abs{\rho'(x) - A} = \abs{\int_x^\infty \rho''(t) \D{t}} \leq \int_x^\infty \abs{\rho''(t)} \D{t} \lesssim \int_x^\infty \frac{1}{t^{2 + \alpha}} \D{t} = \frac{1}{(1 + \alpha) x^{1 + \alpha}}.$$
    This proves that $x \mapsto \abs{x}^{1 + \alpha} \abs{\rho'(x) - A}$ is bounded on $I$. We introduce the function $\eta: x \mapsto \rho(x) - A x$, which is such that $\eta'(x) = \rho'(x) - A$ so $x \mapsto \abs{x}^{1 + \alpha} \abs{\eta'(x)}$ is bounded. Applying the same arguments as above to $\eta$ shows that it has a finite limit $B$ at $+\infty$ and that $x \mapsto \abs{x}^\alpha \abs{\eta(x) - B} = \abs{x}^\alpha \abs{\rho(x) - A x - B}$ is bounded. We conclude that $\rho$ behaves linearly at $+\infty$, as it has a slant asymptote $\rho_{+\infty}: x \mapsto A x + B$ at $+\infty$.

    Using the $p$-test for improper integrals again, $x \mapsto \abs{x}^\alpha \abs{\rho(x) - \rho_{+\infty}(x)}$ being bounded on $I$ implies that $\rho - \rho_{+\infty}$ belongs to $L^p(I)$ for all $p > 1/\alpha$. Since $\rho$ and $\rho_{+\infty}$ are continuous on $\cc{0}{M}$, $\rho - \rho_{+\infty}$ is bounded on $\cc{0}{M}$ and the $L^p$ integrability of $\rho - \rho_{+\infty}$ extends to $\RR_+$ as a whole. Similar arguments show that $\rho' - \rho_{+\infty}' \in L^q(\RR_+)$ for all $q > 1/(1 + \alpha)$. Finally, $\rho''$ is bounded on $\RR$ (by definition of $\mathcal{A}^\alpha$) so in particular it is bounded on $\cc{0}{M}$. Together with the boundedness of $x \mapsto \abs{x}^{2 + \alpha} \rho''(x)$, we conclude that $\rho'' \in L^r(\RR_+)$ for all $r > 1/(2 + \alpha)$.
\end{proof}

\subsection{Proof of Lemma \ref{lem:bound}}
\label{proof:lem:bound}

\begin{proof}
    Let $\alpha > 0$, $\rho \in \mathcal{A}^\alpha$, $1/\alpha < p < \infty$ and $I \in P(\rho)$ be a compatible interval for the function $\rho$. If $\rho$ is linear on $I$, then for all $x < y \in I$, $T[\rho, x]$ coincides with $\rho$ on $\oo{x}{y}$, and so do $T[\rho, y]$ and $T[\rho, x, y]$. Besides, $\rho'' = 0$ so the proposition holds trivially. We thus suppose that $\rho$ is either strictly convex or strictly concave on $I$, which means that $\rho''$ is zero at the two ends of $I$ and nowhere else in $I$.

    For $\delta \geq 0$, we define the set $E_\delta = \{(x, y) \in I^2, y - x \geq \delta\}$. We also write $E = E_0$. Let now $\delta > 0$ be fixed. For all $(x, y) \in E_\delta$, the quantity $\aabs{\rho''}_{L^{p/(2p+1)}(\oo{x}{y})}$ is not zero since $y > x$ and the integrand only vanishes on a set of zero measure. Besides, if $\oo{x}{y}$ is infinite, $T[\rho, x, y]$ coincides with $\rho_{\pm \infty}$ and since $p > 1/\alpha$, \lem{integrability} ensures that $\rho - T[\rho, x, y]$ belongs to $L^p(\oo{x}{y})$. This enables us to consider the function
    \[r: E_\delta \ni (x, y) \mapsto \frac{\aabs{\rho - T[\rho, x, y]}_{L^p(\oo{x}{y})}}{\aabs{\rho''}_{L^{p/(2p+1)}(\oo{x}{y})}}.\]
    We want to show that $r$ can be defined on $E$ as a whole, and that $r$ is bounded on $E$.

        {\itshape (i) Continuity and differentiability of $c$ on $E_\delta$.}
    By definition of $I$, for all $x < y \in I$, the tangents to $\rho$ at $x$ and $y$ intersect at an abscissa that we write $c(x, y)$, with $x < c(x, y) < y$. More precisely, for finite $x < y \in I$, the function $c$ has the following expression
    \[c(x, y) = -\frac{[\rho(y) - y \rho'(y)] - [\rho(x) - x \rho'(x)]}{\rho'(y) - \rho'(x)},\]
    in which the denominator is not zero because $\rho'$ is injective on $I$. We extend $c$ to possible infinite values of $x$ or $y$ by setting $c(x, +\infty) = \lim_{y \to +\infty} c(x, y)$ and $c(-\infty, y) = \lim_{x \to -\infty} c(x, y)$. Since $\rho$ and $\rho'$ are continuous on $I$, and since $\rho'$ is injective on $I$, we conclude that $c$ is continuous on $E_\delta$. By using the differentiation rule for quotients, similar arguments show that $c$ is differentiable on $E_\delta$.

        {\itshape (ii) Continuity of $r$ on $E_\delta$.}
    Since $\rho$ and $\rho'$ are continuous on $\RR$, the map $x \mapsto T[\rho, x]$ is continuous. Besides, $c$ is continuous so the map $(x, y) \to T[\rho, x, y]$ is also continuous. Finally, any map $(x, y) \to \int_x^y \phi(t, x, y) \D t$ is continuous whenever $\phi$ is continuous. Here, $(x, y) \mapsto \rho - T[\rho, x, y]$ and $\rho''$ are continuous, so we conclude that the numerator and the denominator of $r$ are continuous. Moreover, since $y - x \geq \delta > 0$, the denominator of $r$ is not zero, which makes $r$ continuous on $E_\delta$.

    \lem{integrability} ensures that the numerator of $r$ is bounded. The denominator is bounded from below because $y - x > \delta$ and $\rho''$ only vanishes on a set of zero measure. It is bounded from above because when $p > 1/\alpha$, we verify that $p/(2p+1) > 1/(\alpha+2)$ so \lem{integrability} ensures that $\rho''$ belongs to $L^{p/(2p+1)}(\RR)$. This proves that $r(x, y)$ is bounded on $E_\delta$. Thus the only case where $r$ could be unbounded is when $x$ and $y$ get arbitrarily close ($\delta$ goes to zero).

        {\itshape (iii) Continuity and differentiability of $c$ on $E$.}
    We now show that $c$ can be continuously extended to $E$ and that this extension is differentiable on $E$. To do so, we prove that for all element $x \in I$, $\lim_{y \to x} c(x, y) = x$. One can prove that for all $y \in I$, $\lim_{x \to y} c(x, y) = y$ using a similar argument. Let $\epsilon \leq y - x$, so that $x + \epsilon \in I$. We start by rearranging the terms in the expression of $c(x, x + \epsilon)$ and obtain the following
    \[c(x, x + \epsilon) = x + \frac{\rho'(x + \epsilon) - \epsilon^{-1} (\rho(x + \epsilon) - \rho(x))}{\rho'(x + \epsilon) - \rho'(x)} \epsilon.\]
    Now, let $k \geq 2$ be the smallest integer such that the $k$-th derivative of $\rho$ at $x$ is not zero. By definition of a compatible interval, $k$ will be equal to $2$ for all interior points of $I$. Since $\rho$ is analytic in the neighbourhood of the zeros of $\rho''$, there also exists such a $k > 2$ for the ends of $I$. We can always take $\epsilon$ as small as needed so that $\rho$ is analytic around $x$ in a neighbourhood of radius $\epsilon$. We obtain a Taylor expansion of the fraction above by expanding its numerator and denominator to order $k-1$ around $x$, expressed at $x + \epsilon$, and arrive at
    \[c(x, x + \epsilon) = x + \frac{k - 1}{k} \epsilon + O(\epsilon^2).\]
    This shows that the function $\hat{c}$ defined by $\hat{c}(x, y) = c(x, y)$ if $x < y$ and $\hat{c}(x, x) = x$ is continuous and differentiable on $E$.

        {\itshape (iv) Continuity of $r$ on $E$.}
    Let $x \in I$. We show that $\epsilon \mapsto r(x, x + \epsilon)$ has a finite limit when $\epsilon$ goes to zero by finding an equivalent of the numerator and the denominator of $r$. We write the numerator as $(I_1(\epsilon)^p + I_2(\epsilon)^p)^{1/p}$, where $I_1$ and $I_2$ are the integrals on $\oo{x}{c(x, x+\epsilon)}$ and $\oo{c(x, x+\epsilon)}{x + \epsilon}$ respectively. The denominator of $r$ is written $I_3(\epsilon)$. Here again, let $k \geq 2$ be the smallest integer such that the $k$-th derivative of $\rho$ at $x$ is not zero.

    We use the mean-value form of the remainder in the $k$-th-order Taylor expansion of $\rho$ to obtain that for all $t \in \oo{x}{c(x, x + \epsilon)}$, there exists $\eta_t \in \oo{x}{t}$ such that $\rho(t) - T[\rho, x](t) = \frac{1}{k!} \rho^{(k)}(\eta_t) (t - x)^k$. Then, owing to the mean-value theorem for integrals since $\rho^{(k)}$ is continuous, there exists $\eta \in \oo{x}{c(x, x + \epsilon)}$ such that
    \[I_1(\epsilon)^p = \int_{x}^{c(x, x + \epsilon)} \abs{\frac{1}{k!} \rho^{(k)}(\eta_t) (t - x)^k}^p \D t = \frac{1}{k!^p (kp+1)} |\rho^{(k)}(\eta)|^p (c(x, x + \epsilon) - x)^{kp+1}.\]
    We can now use the previous Taylor expansion of $c$ around $(x, x)$ and also expand $\rho^{(k)}(\eta)$ around $x$ since $\eta$ goes to $x$ as $\epsilon$ approaches zero. We finally obtain the following expansion for $I_1(\epsilon)^p$
    \[I_1(\epsilon)^p = \frac{1}{k!^p (kp+1)} \left(\frac{k-1}{k}\right)^{kp+1} |\rho^{(k)}(x)|^p \epsilon^{kp+1} + o(\epsilon^{kp+1}).\]
    We obtain a $k$-th order Taylor expansion of $\rho - T[\rho, x + \epsilon]$ as follows: there exist $\eta_1, \eta_2 \in \oo{x}{x + \epsilon}$, such that for all $t \in \oo{c(x, x + \epsilon)}{x + \epsilon}$, there exists $\eta_t$ between $x$ and $t$ such that
    \[\rho(t) - T[\rho, x + \epsilon](t) = \frac{1}{k!} \left[\rho^{(k)}(\eta_t) (t - x)^k - k \rho^{(k)}(\eta_2) \epsilon^{k-1} (t - x) + k \rho^{(k)}(\eta_2) \epsilon^k - \rho^{(k)}(\eta_1) \epsilon^k\right] + o(\epsilon^k).\]
    We observe that $\eta_1$, $\eta_2$ and $\eta_t$ go to $x$ as $\epsilon$ approaches zero. This means that the Taylor expansion of $\rho^{(k)}(\eta_t)$ is $\rho^{(k)}(x) + O(\epsilon)$, and similarly at $\eta_1$ and $\eta_2$. Besides, using the expansion of $c(x, x+ \epsilon)$, we show that $x + \frac{k-1}{k} \epsilon + o(\epsilon) \leq t \leq x + \epsilon$, which means that $t - x$ is of the order of $\epsilon$. After keeping the terms of order $\epsilon^k$, factoring by $\rho^{(k)}(x) \epsilon^k$ and introducing the change of variable $u = \frac{t - x}{\epsilon}$, we obtain
    \[\rho(t) - T[\rho, x + \epsilon](t) = \frac{\rho^{(k)}(x)}{k!} \left[u^k - k u + k - 1\right] \epsilon^k + o(\epsilon^k).\]
    The bounds of the integral become $\frac{k-1}{k} + o(1)$ and $1$, and we decompose the integration domain into $\oo{\frac{k-1}{k}}{1}$ and $\oo{\frac{k-1}{k} + o(1)}{\frac{k-1}{k}}$. The second integral is going to be negligible against the first because its integrand is bounded and its integration domain has a length of $o(1)$. Putting everything together, we obtain the following expansion for $I_2(\epsilon)^p$
    \[I_2(\epsilon)^p = \frac{1}{k!^p} \left(\int_{\frac{k-1}{k}}^{1} \abs{u^k - 1 - k (u - 1)}^p \D u\right) |\rho^{(k)}(x)|^p \epsilon^{kp+1} + o(\epsilon^{kp+1}).\]
    Using a similar argument, the Taylor expansion of the integrand of the denominator of $r$ is $\frac{1}{(k-2)!} \rho^{(k)}(x)(t - x)^{k-2} + o((t - x)^{k-2})$ and taking the $L^{p/(2p+1)}$ norm of this relationship, we find that the expansion of $I_3(\epsilon)^p$ is
    \[I_3(\epsilon)^p = \frac{1}{(k-2)!^p} \left(\frac{2p+1}{kp+1}\right)^{2p+1} |\rho^{(k)}(x)|^p \epsilon^{kp+1} + o(\epsilon^{kp+1}).\]
    Altogether, in the expression of $r(x, x + \epsilon)^p$, we can factor the numerator and the denominator by $|\rho^{(k)}(x)|^p \epsilon^{kp+1}$ and taking the limit as $\epsilon$ goes to zero, we obtain
    \[\lim_{\epsilon \to 0} r(x, x + \epsilon)^p = \frac{1}{(k(k-1))^p} \left(\frac{kp+1}{2p+1}\right)^{2p+1} \left(\frac{1}{kp+1} \left(\frac{k-1}{k}\right)^{kp+1} + \int_{\frac{k-1}{k}}^{1} \abs{u^k - 1 - k (u - 1)}^p \D u\right).\]
    This shows that the function $\hat{r}$ defined by $\hat{r}(x, y) = r(x, y)$ if $x < y$ and $\hat{r}(x, x)$ as the quantity above is continuous on $E$ as a whole. Besides, $\hat{r}$ has finite limits on $\partial E$, so we conclude that it is bounded on $E$.

        {\itshape (v) Proof of the other inequalities.}
    The same proof can be adapted for the bound on the derivatives, and the bounds in $L^\infty$ norm (\lem{integrability} guarantees that $\rho'' \in L^{1/2}(\RR)$). There is one important change when bounding the derivative: all instances of $k$ must be replaced by $k-1$ (since $\rho^{(k)} = (\rho')^{(k-1)}$), so the appropriate norm of $\rho''$ needed to cancel $\epsilon^{(k-1)p + 1}$ is $p/(p+1)$. The other arguments are the same everywhere, up to different constants in the equivalent of $r$ at points of the type $(x, x$). It is still possible to define the corresponding function $r$ on $E$ as a whole and to show that it is bounded on $E$.
\end{proof}

\subsection{Proof of Proposition \ref{prop:cpwl}}
\label{proof:prop:cpwl}

\begin{proof}
    This proof is inspired by the heuristics given in \cite{berjon2015}. We keep the notations of the definition in \subsect{approximation_space}. Let $\alpha > 0$, $\rho \in \mathcal{A}^\alpha$, $1/\alpha < p \leq \infty$, $q \geq p$ and $I$ be a compatible interval for $\rho$. If $\rho$ is linear on $I$, then $\rho$ coincides with its tangent at either end of $I$ so it is enough to define the restriction of $\pi_{n, p}[\rho]$ to $I$ as this tangent. We thus decompose the integral norms of $\rho - \pi_{n, p}[\rho]$ and $\rho' - \pi_{n, p}[\rho]'$ on the $\kappa(\rho)$ compatible intervals for $\rho$ where $\rho$ is not linear. We decide to place $n' = \floor{n/\kappa(\rho)}$ points in each interval.

        {\itshape (i) Bound in one compatible interval for $q < \infty$.}
    Since $\pi_{n, p}[\rho]$ coincides with $\rho_{\pm \infty}$ on the first and last compatible intervals for $\rho$, \lem{integrability} ensures that when $q \geq p > 1/\alpha$, the functions $\rho - \pi_{n, p}[\rho]$ and $\rho' - \pi_{n, p}[\rho]'$ belong to $L^q(\RR_{\pm \infty})$.

    Let $I = \oo{a}{b}$ be a compatible interval for $\rho$ such that $\rho$ is not linear on $I$. Let $a < \xi_1 < \ldots < \xi_{n'} < b$ the free points in $I$. For convenience we introduce $\xi_0 = a$ and $\xi_{n' + 1} = b$. For $0 \leq k \leq n'$, we write $I_k = \oo{\xi_k}{\xi_{k+1}}$. By construction of $\pi_{n, p}[\rho]$, the restriction of $\pi_{n, p}[\rho]$ to $I_k$ coincides with $T[\rho, \xi_k, \xi_{k+1}]$. We apply \lem{bound} to obtain the upper bounds
    \begin{align}
        \aabs{\rho - \pi_{n, p}[\rho]}_{L^q(I)}^q   & = \sum_{0 \leq k \leq n'} \aabs{\rho - T[\rho, \xi_k, \xi_{k+1}]}_{L^q(I_k)}^q \leq A_q(\rho)^q \sum_{0 \leq k \leq n'} \aabs{\rho''}_{L^{q/(2q+1)}(I_k)}^q, \label{eq:bound_P}   \\
        \aabs{\rho' - \pi_{n, p}[\rho]'}_{L^q(I)}^q & = \sum_{0 \leq k \leq n'} \aabs{\rho' - T[\rho, \xi_k, \xi_{k+1}]'}_{L^q(I_k)}^q \leq B_q(\rho)^q \sum_{0 \leq k \leq n'} \aabs{\rho''}_{L^{q/(q+1)}(I_k)}^q. \label{eq:bound_DP}
    \end{align}
    We introduce the notations $q_1 = q/(q+1)$ and $q_2 = q/(2q+1)$ to make the rest of the proof more readable. We are now looking for a choice of free points that provides a satisfying upper bound for both inequalities. We do so by equidistributing the $L^{q_2}$ norm of $\rho''$ on the subsegments $I_k$. Let $\eta$ be the function defined on $I$ by
    \[\eta(\xi) = \int_{\xi_0}^{\xi} \abs{\rho''(t)}^{q_2} \D{t}.\]
    Since $\rho''$ may only vanish on a set of zero measure, it is clear that $\eta$ is strictly increasing. Besides, we check that $\eta(\xi_0) = 0$ and $\eta(\xi_{n'+1}) = \aabs{\rho''}_{L^{q_2}(I)}^{q_2}$. Therefore, by the intermediate value theorem, we can select $\xi_k$ such that $\eta(\xi_k) = k (n'+1)^{-1} \aabs{\rho''}_{L^{q_2}(I)}^{q_2}$ for all $1 \leq k \leq n'$.

    Now, the terms in the sum of \eq{bound_P} are equal to $\left(\aabs{\rho''}_{L^{q_2}(I_k)}^{q_2}\right)^{q/q_2} = (\eta(\xi_{k+1}) - \eta(\xi_k))^{2q+1}$, and with this choice of $(\xi_k)$ these terms are all constant to $(n'+1)^{-(2q+1)} \aabs{\rho''}_{L^{q_2}(I)}^{q}$. Therefore, \eq{bound_P} becomes
    \begin{align}
        \aabs{\rho - \pi_{n, p}[\rho]}_{L^q(I)}^q \leq A_q(\rho)^q (n'+1)^{-2q} \aabs{\rho''}_{L^{q_2}(\RR)}^{q}, \label{eq:bound_P2}
    \end{align}
    where we used the fact that $\rho''$ is $L^{q_2}$ integrable on $\RR$ as a whole to bound the $L^{q_2}$ norm on $I$.

    The partition of $I$ above does not equidistribute $\aabs{\rho''}_{L^{q_1}(I_k)}^{q_1}$ so we cannot use the same method to bound \eq{bound_DP}. Instead, we break down $\abs{\rho''}^{q_1}$ into $\abs{\rho''}^{q_1-q_2} \abs{\rho''}^{q_2}$ and use the fact that $\rho''$ is bounded to move the first factor out of the integral:
    \[
        \aabs{\rho''}_{L^{q_1}(I_k)}^{q_1} \leq \aabs{\rho''}_{L^\infty(I_k)}^{q_1 - q_2} \aabs{\rho''}_{L^{q_2}(I_k)}^{q_2} \leq (n' + 1)^{-1} \aabs{\rho''}_{L^\infty(\RR)}^{q_1 - q_2} \aabs{\rho''}_{L^{q_2}(\RR)}^{q_2}.
    \]
    We make use of this result to bound \eq{bound_DP}. After simplification, we have
    \begin{align}
        \aabs{\rho' - \pi_{n, p}[\rho]'}_{L^q(I)}^q \leq B_q(\rho)^q (n' + 1)^{-q} \aabs{\rho''}_{L^\infty(\RR)}^{q^2/(2q+1)} \aabs{\rho''}_{L^{q_2}(\RR)}^{q(q+1)/(2q+1)}. \label{eq:bound_DP2}
    \end{align}

    {\itshape (ii) Bound on $\RR$ as a whole for $q < \infty$.}
    Let $\mathcal{I}(\rho)$ denote the set of compatible intervals for $\rho$. We apply the strategy presented in {\itshape (i)} to all $I \in \mathcal{I}(\rho)$ and find an upper bound for the $L^q$ norm on $\RR$ as a whole. Applying \eq{bound_P2}, we finally reach
    \begin{align*}
        \aabs{\rho - \pi_{n, p}[\rho]}_{L^q(\RR)}^q & = \sum_{I \in \mathcal{I}(\rho)} \aabs{\rho - \pi_{n, p}[\rho]}_{L^q(I)}^q \leq \sum_{I \in \mathcal{I}(\rho)} A_q(\rho)^q (n'+1)^{-2q} \aabs{\rho''}_{L^{q_2}(\RR)}^{q} \\
                                                    & \leq \kappa(\rho) A_q(\rho)^q (n'+1)^{-2q} \aabs{\rho''}_{L^{q_2}(\RR)}^{q}                                                                                              \\
                                                    & \leq A_q(\rho)^q \kappa(\rho)^{2q+1} \aabs{\rho''}_{L^{q_2}(\RR)}^{q} n^{-2q}.
    \end{align*}
    The last inequality comes from the fact that for all $m, n > 0$, $m \floor{n/m} \geq n - m$. We conclude by raising both sides to the power $1/q$: the $L^q$ norm of $\rho - \pi_{n, p}[\rho]$ decays as $n^{-2}$. We apply the same method to $\rho' - \pi_{n, p}[\rho]$: using \eq{bound_DP2} yields
    \begin{align*}
        \aabs{\rho' - \pi_{n, p}[\rho]'}_{L^q(\RR)}^q & \leq B_q(\rho)^q \kappa(\rho)^{q+1} \aabs{\rho''}_{L^\infty(\RR)}^{q^2/(2q+1)} \aabs{\rho''}_{L^{q_2}(\RR)}^{q(q+1)/(2q+1)} n^{-q}.
    \end{align*}
    Here again, we conclude by raising both sides to the power $1/q$: the $L^q$ norm of $\rho' - \pi_{n, p}[\rho]'$ decays as $n^{-1}$.

        {\itshape (iii) Extension to $p = \infty$.}
    The same bounds are obtained for the $L^\infty$ norm, by considering the maximum $L^\infty$ norm over the $n'+1$ subsegments and sampling the free points so as to equidistribute $\abs{\rho''}^{1/2}$.
\end{proof}

\subsection{Proof of Lemma \ref{lem:convexity}}
\label{proof:lem:convexity}

\begin{proof}
    We know that the cells in $\tau_{\RR^d}(\bm{\vartheta})$ are all convex since they are defined as the interior of intersections of half-planes. The final mesh $\tau_\Omega(\bm{\vartheta})$ is the intersection of $\tau_{\RR^d}(\bm{\vartheta})$ against $\Omega$. Let $K \in \tau_\Omega(\bm{\vartheta})$. There exists $K' \in \tau_{\RR^d}(\bm{\vartheta})$ such that $K = K' \cap \Omega$. If $K' \subseteq \Omega$, then $K = K'$ is convex. By the contraposition principle, if $K$ is not convex, then $K' \nsubseteq  \Omega$. This shows that if $K$ is a non-convex cell in $\tau_\Omega(\bm{\vartheta})$, then $K$ intersects with both $\Omega$ and $\RR^d \backslash \Omega$, which means that $K$ intersects with the boundary of $\Omega$. In the case where $\Omega$ is convex, the intersection of two convex sets being convex implies that $K = K' \cap \Omega$ is always convex.
\end{proof}

\subsection{Proof of Proposition \ref{prop:nns}}
\label{proof:prop:nns}

\begin{proof}
    Let $\Omega \subset \RR^d$ be a bounded domain. Let $\rho$ and $\sigma$ be two continuous functions almost everywhere differentiable such that $\rho - \sigma$, $\rho'$ and $\sigma'$ are bounded on $\RR$, and $\rho$ and $\rho'$ are Lipschitz continuous. We recall that we define the $L^\infty$ norm of a vector- or matrix-valued function as the maximum $L^\infty$ norm of its coordinates.

    We prove the proposition by induction, starting from the identity and showing that the inequalities still hold after composition by a linear map or an activation function. As will be made clear in {\itshape (iii)}, we also need to show that $\nabla u_\sigma$ is bounded in $\Omega$. In this proof, the new constants after the induction are denoted with a prime symbol.

        {\itshape (i) Induction hypotheses.}
    Suppose that $u_\rho$ and $u_\sigma$ take values in $\RR^n$ and that the following inequalities hold for some constants $C_1$, $C_2$, $C_3$, $C_{\bm{\vartheta}}$ and integer $\alpha$
    \begin{align*}
        \aabs{u_\rho - u_\sigma}_{L^\infty(\Omega, \RR^n)}                          & \leq C_1 \aabs{\rho - \sigma}_{L^\infty(\RR)},                                              \\
        \aabs{\nabla u_\rho - \nabla u_\sigma}_{L^\infty(\Omega, \RR^{d \times n})} & \leq C_2 \aabs{\rho - \sigma}_{L^\infty(\RR)} + C_3 \aabs{\rho' - \sigma'}_{L^\infty(\RR)}, \\
        \aabs{\nabla u_\sigma}_{L^\infty(\Omega, \RR^{d \times n})}                 & \leq C_{\bm{\vartheta}} \aabs{\sigma'}_{L^\infty(\RR)}^\alpha.
    \end{align*}

    {\itshape (ii) Composition with a linear map.}
    Let $\bm{\Theta}: \bm{x} \to \bm{W} \bm{x} + \bm{b}$, where $\bm{W} \in \RR^{m \times n}$ and $\bm{b} \in \RR^{m}$. Let $1 \leq i \leq m$. The triangle inequality yields
    \begin{align*}
        \aabs{(\bm{\Theta} \circ u_\rho)_i - (\bm{\Theta} \circ u_\sigma)_i}_{L^\infty(\Omega)} & \leq \sum_{1 \leq j \leq n} \abs{\bm{W}_{ij}} \aabs{(u_\rho - u_\sigma)_j}_{L^\infty(\Omega)} \leq \aabs{\pmb{\Theta}}_{\infty, 1} \aabs{u_\rho - u_\sigma}_{L^\infty(\Omega, \RR^n)},
    \end{align*}
    where we have introduced the norm $\aabs{\bm{\Theta}}_{\infty, 1} = \max_{1 \leq i \leq m} \sum_{1 \leq j \leq n} \abs{\bm{W}_{ij}}$. Let $1 \leq k \leq d$. We follow the same method for the norm of $\partial_k (\bm{\Theta} \circ u_\rho) - \partial_k (\bm{\Theta} \circ u_\sigma)$: it holds
    \begin{align*}
        \aabs{\partial_k (\bm{\Theta} \circ u_\rho)_i - \partial_k (\bm{\Theta} \circ u_\sigma)_i}_{L^\infty(\Omega)} & \leq \aabs{\bm{\Theta}}_{\infty, 1} \aabs{\nabla u_\rho - \nabla u_\sigma}_{L^\infty(\Omega, \RR^{d \times n})}.
    \end{align*}
    The same argument shows that $\aabs{\partial_k (\bm{\Theta} \circ u_\sigma)_i}_{L^\infty(\Omega)} \leq \aabs{\bm{\Theta}}_{\infty, 1} \aabs{\nabla u_\sigma}_{L^\infty(\Omega, \RR^{d \times n})}$. We conclude by taking the maximum on $1 \leq i \leq m$ and $1 \leq k \leq d$: the three bounds still hold for constants $C_1'$, $C_2'$, $C_3'$ and $C_{\bm{\vartheta}}'$ obtained by multiplying the original constants $C_1$, $C_2$, $C_3$ and $C_{\bm{\vartheta}}$ by $\aabs{\bm{\Theta}}_{\infty, 1}$, and taking $\alpha' = \alpha$.

        {\itshape (iii) Composition with an activation function.}
    We now compose $u_\rho$ by $\rho$ and $u_\sigma$ by $\sigma$. Let $1 \leq i \leq n$. We use the triangle inequality again and obtain
    \begin{align*}
        \aabs{(\rho \circ u_\rho)_i - (\sigma \circ u_\sigma)_i}_{L^\infty(\Omega)} & \leq \aabs{\rho \circ (u_\rho)_i - \rho \circ (u_\sigma)_i}_{L^\infty(\Omega)} + \aabs{\rho \circ (u_\sigma)_i - \sigma \circ (u_\sigma)_i}_{L^\infty(\Omega)}.
    \end{align*}
    We bound the first term using the Lipschitz continuity of $\rho$. The second term is bounded by the $L^\infty$ norm of $\rho - \sigma$. Altogether, applying the induction hypothesis, we reach
    \begin{align*}
        \aabs{(\rho \circ u_\rho)_i - (\sigma \circ u_\sigma)_i}_{L^p(\Omega)} & \leq \underbrace{\left[C_1 \lip{\rho} + 1\right]}_{= C_1'} \aabs{\rho - \sigma}_{L^\infty(\RR)}.
    \end{align*}
    We now turn to the bound for the gradient. By the chain rule, $\partial_k (\rho \circ (u_\rho)_i) = \rho' \circ (u_\rho)_i \partial_k (u_\rho)_i$. We use the triangle inequality to decompose the distance between $\partial_k (\rho \circ u_\rho)_i$ and $\partial_k (\rho \circ u_\rho)_i$ into three terms:
    \begin{align*}
        \aabs{\partial_k (\rho \circ u_\rho)_i - \partial_k (\sigma \circ u_\sigma)_i}_{L^\infty(\Omega)} & \leq \aabs{\rho' \circ (u_\rho)_i \partial_k (u_\rho)_i - \rho' \circ (u_\rho)_i \partial_k (u_\sigma)_i}_{L^\infty(\Omega)}       \\
                                                                                                          & + \aabs{\rho' \circ (u_\rho)_i \partial_k (u_\sigma)_i - \rho' \circ (u_\sigma)_i \partial_k (u_\sigma)_i}_{L^\infty(\Omega)}      \\
                                                                                                          & + \aabs{\rho' \circ (u_\sigma)_i \partial_k (u_\sigma)_i - \sigma' \circ (u_\sigma)_i \partial_k (u_\sigma)_i}_{L^\infty(\Omega)}.
    \end{align*}
    The first term is easily bounded using the fact that $\rho'$ is bounded on $\RR$. The second term is bounded owing to $\rho'$ being Lipschitz continuous and $\nabla u_\sigma$ being bounded. We bound the last term by using the $L^\infty$ norm of $\rho' - \sigma'$ and the fact that $\nabla u_\sigma$ is bounded. We finally obtain the following upper bound
    \begin{align*}
        \aabs{\partial_k (\rho \circ u_\rho)_i - \partial_k (\sigma \circ u_\sigma)_i}_{L^p(\Omega)} & \leq  \underbrace{\left[C_2 \aabs{\rho'}_{L^\infty(\RR)} + C_1 C_{\bm{\vartheta}} \lip{\rho'} \aabs{\sigma'}_{L^\infty(\RR)}^\alpha\right]}_{= C_2'} \aabs{\rho - \sigma}_{L^p(\RR)} \\
                                                                                                     & + \underbrace{\left[C_3 \aabs{\rho'}_{L^\infty(\RR)} + C_{\bm{\vartheta}} \aabs{\sigma'}_{L^\infty(\RR)}^\alpha\right]}_{= C_3'} \aabs{\rho' - \sigma'}_{L^p(\RR)}.
    \end{align*}
    Since $\sigma'$ is bounded on $\RR$ it holds $\aabs{\partial_k (\sigma \circ u_\sigma)_i}_{L^\infty(\Omega)} \leq \aabs{\sigma'}_{L^\infty(\RR)} \aabs{\nabla u_\sigma}_{L^\infty(\Omega, \RR^{d \times n})}$. We conclude by taking the maximum on $1 \leq i \leq m$ and $1 \leq k \leq d$: the three bounds still hold for constants $C_1'$, $C_2'$ and $C_3'$ indicated above, $C_{\bm{\vartheta}}' = C_{\bm{\vartheta}}$ and $\alpha' = \alpha + 1$.
\end{proof}