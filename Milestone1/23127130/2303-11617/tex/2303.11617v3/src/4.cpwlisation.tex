In this section, we suppose that we are given a smooth activation $\rho$ and we consider the \ac{nn} $u = \mathcal{N}(\bm{\vartheta}, \rho)$. We draw inspiration from the \ac{fem} and we would like to construct a mesh of the domain such that the behaviour of the \ac{nn} is close to linear in each cell of the mesh. We design a method that approximates the smooth activation function $\rho$ by a \ac{cpwl} function $\pi[\rho]$ and bound the distance between the two functions.

\subsection{Assumptions on the activation function}

Since we have no prior bound on the parameters of the network, we need the approximation of $\rho$ by $\pi[\rho]$ to hold on $\RR$ as a whole. Since $\pi[\rho]$ is going to be piecewise linear, we need $\rho$ to behave linearly at infinity in order for $\rho - \pi[\rho]$ to be integrable on $\RR$ as a whole. Therefore, we require that $\rho$ be twice differentiable and that $\rho''$ vanish fast enough at infinity. As will be made clear in the proof of \lem{bound}, we also need $\rho$ to be analytic in the neighbourhood of the zeros of $\rho''$. Altogether, we restrict our approach to functions that belong to the functional spaces $\mathcal{A}^\alpha$ for $\alpha > 0$, defined as
\[
    \mathcal{A}^\alpha = \left\{\begin{array}{l}
        \rho: \RR \to \RR \text{ twice differentiable on } \RR \text{ such that}                          \\
        \rho'' \text{ and } x \mapsto \abs{x}^{2 + \alpha} \rho''(x) \text{ are bounded on } \RR \text{,} \\
        \rho \text{ is analytic in a neighbourhood of the zeros of } \rho''
    \end{array}\right\}.
\]
For convenience, we also introduce the union space $\mathcal{A} = \bigcup_{\alpha > 0} \mathcal{A}^\alpha$. The following lemma shows important properties of functions in $\mathcal{A}^\alpha$, which we use in the remainder of this paper.

\begin{lemma}[Some properties of functions in $\mathcal{A}^\alpha$]
    \label{lem:integrability}
    For all $\alpha > 0$ and $\rho \in \mathcal{A}^\alpha$,
    \begin{itemize}
        \item $\rho$ behaves linearly at infinity, i.e. there exist linear functions (slant asymptotes) $\rho_{+ \infty}$ and $\rho_{- \infty}$ such that $\rho - \rho_{\pm \infty}$ vanishes at $\pm \infty$.
        \item For all $p > 1/\alpha$, $\rho - \rho_{\pm \infty} \in L^p(\RR_\pm)$.
        \item For all $q > 1/(1 + \alpha)$, $\rho' - \rho_{\pm \infty}' \in L^q(\RR_\pm)$.
        \item For all $r > 1/(2 + \alpha)$, $\rho'' \in L^r(\RR)$.
    \end{itemize}
\end{lemma}
\begin{proof}
    See \ref{proof:lem:integrability}.
\end{proof}

\subsection{Approximation space}
\label{subsect:approximation_space}

In our experiments, we observed that the way we approximate the activation function by a \ac{cpwl} counterpart plays an important role in the performance of our adaptive quadrature. There are existing algorithms and heuristics to build $\pi[\rho]$ \cite{berjon2015} but we found that $\pi[\rho]$ needs to satisfy extra properties for our method to be most efficient in practice. In particular, we noticed that it is preferable that the \ac{cpwl} approximation of $\rho$ lie below $\rho$ on the regions where $\rho$ is convex, and above $\rho$ where $\rho$ is concave. To that aim, we wish to decompose $\RR$ into intervals where $\rho$ is either strictly convex, linear, or strictly concave. We can then approximate $\rho$ on each interval by connecting the tangents to $\rho$ at free points. More formally, we define the partition induced by $\rho$ as follows.

\begin{definition}[Partition induced by a function]
    Let $\rho \in \mathcal{A}$. We define the set $Z_0(\rho) = \{I \subset \RR, \rho''_{|I} = 0\}$ that contains the intervals (possibly isolated points) where $\rho''$ is zero and $Z(\rho) = \partial Z_0(\rho) \cup \{-\infty, +\infty\}$ the set of points where $\rho''$ is zero, excluding intervals but including $\pm \infty$. We write $z(\rho) = \card{Z(\rho)}$ and $\xi_1 < \ldots < \xi_{z(\rho)}$ the ordered elements of $Z(\rho)$. The \emph{partition induced by $\rho$} is defined as $P(\rho) = \{\oo{\xi_i}{\xi_{i+1}}, 1 \leq i \leq z(\rho) - 1\}$, that is the collection of intervals whose ends are two consecutive elements of $Z(\rho)$. The elements of $P(\rho)$ are called \emph{compatible intervals for $\rho$}. We finally introduce $1 \leq \kappa(\rho) \leq z(\rho) - 1$ the number of segments in $P(\rho)$ where $\rho$ is not linear.
\end{definition}

By way of example, the second derivative of $\abse$ is always positive so $Z_0(\abse) = \varnothing$, and we infer that $Z(\abse) = \{-\infty, +\infty\}$, $P(\abse) = \{\RR\}$, $z(\abse) = 2$ and $\kappa(\abse) = 1$. However, the second derivative of $\tanh$ vanishes at zero so $Z_0(\tanh) = \{0\}$, and it follows that $Z(\tanh) = \{-\infty, 0, +\infty\}$, $P(\tanh) = \{\oo{-\infty}{0}, \oo{0}{+\infty}\}$, $z(\tanh) = 3$ and $\kappa(\tanh) = 2$.

For $\rho \in \mathcal{A}$ and $x \in \RR$, we write $T[\rho, x]$ the tangent to $\rho$ at $x$ and we extend this notation to $x = \pm \infty$ by considering the slant asymptotes to $\rho$. From the construction of $P(\rho)$, it is easy to see that for any interval $I \in P(\rho)$ such that $\rho$ is not linear on $I$ and for all $x < y \in I$, the two tangents $T[\rho, x]$ and $T[\rho, y]$ intersect at a unique point in $\oo{x}{y}$ that we write $c(x, y)$. Let $T[\rho, x, y]$ denote the \ac{cpwl} function defined on $\oo{x}{y}$ as $T[\rho, x]$ on $\oc{x}{c(x,y)}$ and $T[\rho, y]$ on $\co{c(x,y)}{y}$.

We are now ready to define our approximation space. For $n \geq z(\rho)$, we define the space $\mathcal{A}_n(\rho)$ of \ac{cpwl} functions that connect tangents to $\rho$ at $Z(\rho)$ as well as $n - z(\rho)$ other free points that do not belong to $Z(\rho)$. More formally, let $\mathcal{S}_n$ denote the set of increasing sequences of length $n$ in $\RR \backslash Z(\rho)$. Next, let $\mathcal{S}_n(\rho)$ be the set of increasing sequences of length $n$ in $\RR$ obtained by concatenating the elements of $\mathcal{S}_{n - z(\rho)}$ with $Z(\rho)$. For $S \in \mathcal{S}_n(\rho)$ and $1 \leq k \leq n$, let $S_k$ be the $k$-th element of $S$. For convenience, we set the convention $S_0 = -\infty$ and $S_{n+1} = +\infty$. Using the extensions of $T$ and $c$ to $\pm \infty$, $\mathcal{A}_n(\rho)$ can be defined as
\[
    \mathcal{A}_n(\rho) = \left\{\begin{array}{l}
        \rho_n: \RR \to \RR \text{ such that there exists } S \in \mathcal{S}_n(\rho) \text{ and for all } 1 \leq k \leq n \text{,}  \\
        \text{the restriction of } \rho_n \text{ to } \oo{c(S_{k-1}, S_k)}{c(S_k, S_{k+1})} \text{ is } T[\rho, S_k]
    \end{array}\right\}.
\]
As an example, the function plotted in \fig{cpwl_tanh} belongs to $\mathcal{A}_7(\tanh)$. There are four free points because $z(\tanh) = 3$. The square points are fixed and they are located at $\pm \infty$ and zero.

\subsection{Convergence rate}

For $\rho \in \mathcal{A}$, $n \geq z(\rho)$ and $p \geq 2$, we write $\pi_{n, p}[\rho]$ the best approximation of $\rho$ in $\mathcal{A}_n(\rho)$ in the $L^p$ norm. In this paragraph, we bound the distance between $\rho$ and $\pi_{n, p}[\rho]$, and determine the convergence rate of this approximation. We start by finding an upper bound for the distance between $\rho$ and $T[\rho, x, y]$ for $x < y \in I$, where $I$ is a compatible interval for $\rho$.

\begin{lemma}[Distance between a function and its tangents]
    \label{lem:bound}
    Let $\alpha > 0$ and $\rho \in \mathcal{A}^\alpha$. Let $p \in \oo{1/\alpha}{\infty}$. There exist constants $A_p(\rho)$ and $B_p(\rho)$ such that for all interval $I$ compatible for $\rho$ and for all $J = \oo{x}{y} \subset I$, it holds
    \begin{align*}
        \aabs{\rho - T[\rho, x, y]}_{L^p(J)} \leq A_p(\rho) \aabs{\rho''}_{L^{p/(2p+1)}(J)}, \\
        \aabs{\rho' - T[\rho, x, y]'}_{L^p(J)} \leq B_q(\rho) \aabs{\rho''}_{L^{p/(p+1)}(J)}.
    \end{align*}
    These bounds extend to the case $p = \infty$: there exist constants $A_\infty(\rho)$ and $B_\infty(\rho)$ such that for all interval $I$ compatible for $\rho$ and for all $J = \oo{x}{y} \subset I$, it holds
    \begin{align*}
        \aabs{\rho - T[\rho, x, y]}_{L^\infty(J)} \leq A_\infty(\rho) \aabs{\rho''}_{L^{1/2}(J)}, \\
        \aabs{\rho' - T[\rho, x, y]'}_{L^\infty(J)} \leq B_\infty(\rho) \aabs{\rho''}_{L^1(J)}.
    \end{align*}
\end{lemma}
\begin{proof}
    See \ref{proof:lem:bound}.
\end{proof}

Thanks to this lemma, we can bound the distance between $\rho$ and $\pi_{n, p}[\rho]$ and prove a quadratic convergence for all $p$. We also prove a linear convergence rate for the distance between $\rho'$ and $\pi_{n, p}[\rho]'$.

\begin{proposition}[Best approximation in $\mathcal{A}_n$]
    \label{prop:cpwl}
    Let $\alpha > 0$ and $\rho \in \mathcal{A}^\alpha$. Let $n \geq z(\rho)$, $p \in \oo{1/\alpha}{\infty}$ and $q \geq p$. Using the constants from \lem{bound}, it holds
    \begin{align*}
        \aabs{\rho - \pi_{n, p}[\rho]}_{L^q(\RR)}   & \leq \frac{A_q(\rho) \kappa(\rho)^{(2q + 1)/q} \aabs{\rho''}_{L^{q/(2q+1)}(\RR)}}{n^2},                                                    \\
        \aabs{\rho' - \pi_{n, p}[\rho]'}_{L^q(\RR)} & \leq \frac{B_q(\rho) \kappa(\rho)^{(q+1)/q} \aabs{\rho''}_{L^{q/(2q+1)}(\RR)}^{(q+1)/(2q+1)} \aabs{\rho''}_{L^\infty(\RR)}^{q/(2q+1)}}{n}.
    \end{align*}
    These bounds extend to the cases $p < \infty$, $q = \infty$ and $p = q = \infty$: using the constants from \lem{bound}, it holds
    \begin{align*}
        \aabs{\rho - \pi_{n, p}[\rho]}_{L^\infty(\RR)}   & \leq \frac{A_\infty(\rho) \kappa(\rho)^2 \aabs{\rho''}_{L^{1/2}(\RR)}}{n^2},                                       \\
        \aabs{\rho' - \pi_{n, p}[\rho]'}_{L^\infty(\RR)} & \leq \frac{B_\infty(\rho) \kappa(\rho) \aabs{\rho''}_{L^{1/2}(\RR)}^{1/2} \aabs{\rho''}_{L^\infty(\RR)}^{1/2}}{n}.
    \end{align*}
\end{proposition}
\begin{proof}
    See \ref{proof:prop:cpwl}.
\end{proof}

\subsection{Numerical implementation}

In practice, we choose $p = 2$ and implement the least-square approximation of $\rho$ on $\mathcal{A}_n(\rho)$. We write $\pi_n$ instead of $\pi_{n, 2}$. Since we are searching for the best approximant within $\mathcal{A}_n[\rho]$, the unknowns to be determined are the locations of the tangents $(\xi_k)_{1 \leq k \leq n-z(\rho)}$. We write the total $L^2$ norm between $\rho$ and $\pi_n[\rho]$ in terms of $(\xi_k)$ and seek to minimise this quantity via a gradient descent algorithm.

We are interested in two activation functions: $\abse$ and $\tanh$. Both functions are symmetric around the origin, which allows us to restrict the optimisation domain. We also verify that both functions belong to $\mathcal{A}$.
\begin{itemize}
    \item The relationship $\abse(x) - \abse(-x) = x$ holds for all $x \in \RR$. We impose the same condition on $\pi_n[\abse]$ and verify that $\abse - \pi_n[\abse]$ is even. Since $\abse$ takes bounded values on $\RR_-$, we approximate $\abse$ by $\pi_n[\abse]$ on $\RR_-$ only and recover $\pi_n[\abse]$ as a whole by symmetry. The second derivative of $\abse$ is $x \mapsto \frac{1}{2 \gamma} (1 + \overline{x}^2)^{-3/2}$, which decays as $x^{-3}$. We verify that the second derivative of $\abse$ is never zero. This shows that $\abse \in \mathcal{A}^\alpha$ with $\alpha = 1$. The slant asymptotes of $\abse$ intersect, so we can approximate $\abse$ with $2$ pieces and this approximation is $\ReLU$.
    \item The $\tanh$ function is odd, so imposing that $\pi_n[\tanh]$ is odd enforces that $\tanh - \pi_n[\tanh]$ is even. We decide to perform the minimisation on $\RR_+$ and obtain an approximation of $\tanh$ on $\RR$ as a whole by symmetry. The second derivative of $\tanh$ is $x \mapsto -2 \tanh(x) (1 - \tanh(x)^2)$. This is asymptotically equivalent to $-8 \exp(-2x)$. In particular, for all $\alpha > 0$, $x \mapsto x^{2 + \alpha} \tanh''(x)$ is bounded on $\RR$. Moreover, $\tanh''(x) = 0$ if and only if $x = 0$, and $\tanh$ is analytic in a neighbourhood of zero (its Taylor series converges within a radius of $\pi / 2$). This shows that $\tanh \in \mathcal{A}^\alpha$ for all $\alpha > 0$. Since the slant asymptotes of $\tanh$ do not intersect, the minimum number of pieces to approximate $\tanh$ on $\RR$ is three and the corresponding \ac{cpwl} function is known as the hard $\tanh$ activation.
\end{itemize}

As an example, we plot $\pi_7[\tanh]$ in \fig{cpwl_tanh}. The convergence plot in $L^2$ norm of our method applied to both functions is shown in \fig{convergence_rate}, together with the theoretical convergence rate of $-2$. We verify that the convergence is asymptotically quadratic. We notice that the approximation of $\abse$ shows a preasymptotic behaviour in which the effective convergence rate is closer to $1.5$.

\begin{figure}
    \centering
    \subfloat[Visualisation of $\pi_7(\tanh)$.\label{fig:cpwl_tanh}]{
        \includegraphics[width=0.45\linewidth]{cpwl.pdf}
    }
    \hfill
    \subfloat[Convergence plot in $L^2$ norm.\label{fig:convergence_rate}]{
        \includegraphics[width=0.45\linewidth]{convergence.pdf}
    }
    \caption{Visualisation of $\pi_7(\tanh)$ and convergence plot of our proposed method in $L^2$ norm for $\abse$ and $\tanh$. On \protect\subref{fig:cpwl_tanh}, the square pink points are fixed, while the circle green points are chosen to minimise the $L^2$ norm of $\tanh - \pi_7[\tanh]$.}
    \label{fig:cpwl_convergence}
\end{figure}
