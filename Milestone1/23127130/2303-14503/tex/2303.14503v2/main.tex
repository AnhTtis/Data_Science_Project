\documentclass[12pt]{article}
\textwidth=17.5cm\textheight=25.5cm\hoffset=-2cm\voffset=-3.5cm
%\usepackage[cp1251]{inputenc}
%\usepackage[russian]{babel}
\usepackage[english]{babel}
\usepackage{amsfonts,amsmath,epsf,amssymb,amsthm,graphicx,afterpage}%,url}
\usepackage[
    draft = false,
    unicode = true,
    colorlinks = true,
    allcolors = blue,
    hyperfootnotes = true,
    citecolor = red
]{hyperref}
\usepackage{epsfig}
\graphicspath{{morefig/}{figures/}{figures00/}{figures02/}{figures05/}{figlms/}}
\newcommand{\?}{\nobreak\hskip.167em\nobreak\hskip0pt}
\DeclareGraphicsExtensions{.pdf,.png,.jpg}
\def\R{\mathbb R}
\def\K{K_5- (4, 5)}
\def\Cycle{\Sigma^1}
\def\ZerSphere{\Sigma^0}
\newcommand{\lk}{\operatorname{lk}}
\def\fSigmC{\displaystyle f|_{\Cycle}}
\def\fSigmP{\gamma^0}

\def\gSigmC{\displaystyle g|_{\Cycle}}
\def\gSigmP{\gamma^0}

\let\fiverm\tiny
\let\Bbb=\mathbb
\long\def\comment#1\endcomment{}

\newtheoremstyle{mydefinition}% name
{3pt}%      Space above
{3pt}%      Space below
{\normalfont}%         Body font
{\parindent}%         Indent amount (empty = no indent, \parindent = para indent)
{\bfseries}% Thm head font
{.}%        Punctuation after thm head
{ }%     Space after thm head: " " = normal interword space;
%       \newline = linebreak
{}%         Thm head spec (can be left empty, meaning `normal')

%\theoremstyle{mydefinition}
\theoremstyle{plain}
\newtheorem{theorem}{Theorem}
\newtheorem{pr}[theorem]{Problem}
\newtheorem{example}[theorem]{Example}
\newtheorem{lemma}[theorem]{Lemma}
\newtheorem{???sentence}[theorem]{Предложение}
\newtheorem{Assertion}[theorem]{Assertion}
\newtheorem{approvalo}[theorem]{Утверждение}

\theoremstyle{definition}
\newtheorem{remark}[theorem]{Remark}

\title{On drawing $K_5$ minus an edge in the plane}
\author{Garaev T. R. }

%\date{August 2021}

\begin{document}

\maketitle

\section{Introduction and main result}

A map $g:K\rightarrow \mathbb{R}^2$ (continuous, or PL, see Remark \ref{r:mot}.a) is called an \textbf{almost embedding}  if $g(\alpha)\cap g(\beta) = \varnothing$ for any two disjoint edges $\alpha, \beta \subset K$.
In this text we give a restriction on certain winding numbers for an almost embedding of graph $K_5$ minus an edge in the plane. 

The main result is Theorem \ref{p:Theor}; an elementary reformulation is given in Remark \ref{r:simp}.a.
Some motivations are given in Remarks \ref{r:mot} and \ref{r:wu}.

Almost embeddings naturally appear in studies of embeddings, in combinatorial geometry and topological combinatorics.
%For example, proofs of non-embeddability of graph in $\R^2$ show that these graph are not almost embeddable to $\R^2$, see \cite[Theorem 1.1 ]{Sch11}.
%Proofs of non-existence of almost embedding of some complexes in $\mathbb{R}^d$ usually show that these complexes are not embeddable to $\mathbb{R}^d$.
%For example, if ???$d \leq \frac{3(k+1)}{2}$ and a $k$-complex almost embeds in $\mathbb{R}^d$, then the complex PL embeds in $\mathbb{R}^d$.
See more motivations in \cite[\S 1, ‘Motivation and background’]{SkTa19} and in \cite[Remark 6.7.5]{Sk}. %\cite{MiPeSh07}.???? %конкретные мотивировки
%[Theorem 1.2]

Denote by

$\bullet$ $[n]$ the set $\{1, 2, \ldots, n\}$;

$\bullet$ $K_5$ the complete graph with the vertex set $[5]$; 

$\bullet$ $K_3$ the complete graph  with the vertex set $[3]$;  

$\bullet$ $(l, k)$ the edge between vertices $l$ and $k$ in a graph;  

$\bullet$ $\K$ the graph obtained from $K_5$ by deleting the edge $(4, 5)$;

$\bullet$ $w(\gamma, A)$ the \emph{winding number}\footnote{See definition in \url{https://en.wikipedia.org/wiki/Winding_number}.} of a closed oriented curve $\gamma :S^1\to \R^2$ around a point $A\in\mathbb{R}^2 \backslash \gamma(S^1)$.

We assume that $K_3$ has the orientation inherited from cyclic permutation $(1\ 2\ 3).$

\begin{theorem}\label{p:Theor}
    For any continuous almost embedding $g:\K \rightarrow \mathbb{R}^2$ we have
    $$l(g):=w(g|_{K_3}, g(4))-w(g|_{K_3}, g(5))=\pm 1.$$
\end{theorem}

A map $g:K\to \R^2$ of graph $K$ is a \emph{piecewise linear} if there is a  subdivision $K'$ of $K$ such that the corresponding map $g':K'\to \R^2$ is linear. 
We write 'PL' instead of 'piecewise linear'.

We write 'embedding' and 'almost embedding' instead of 'PL or continuous embedding' and 'PL or continuous almost embedding' respectively, see Remark \ref{r:mot}.a. 


\begin{remark}[motivation]\label{r:mot}
(a) Theorem \ref{p:Theor} is equivalent to the same statement for PL map, but even for PL map Theorem \ref{p:Theor} is not obvious.

(b) It is obvious that $l(g)=\pm 1$ for any embedding $g:\K \rightarrow \mathbb{R}^2$. 
%the integer $l(g)$ equals $\pm 1$.
It is well-known that for any  almost embedding $g:\K \rightarrow \mathbb{R}^2$ the integer $l(g)$ is odd, see e.g. \cite[Lemma 1.3 for $k=1,\ l=0$]{SkKa23}. 
In the second arXiv version of \cite{SkKa23} Conjecture 1.6(a) states that for any integer $k$ there is an almost embedding $g:\K \to \R^2$ such that $l(g)=2k+1$. %Сказать, что многомерная аналог гипотезы верен.
So Theorem \ref{p:Theor} disproves this conjecture.
For different results in similar situations see (f) and Remark \ref{r:simp}.a'. 

(c) For any integer $k$ there is an almost embedding $g:\K \to \R^2$ such that $w(g|_{K_3}, g(4))=k$.
E.g. Figure % все figer на Figer
\ref{ris:winding} presents an almost embedding $g:\K \to \R^2$ with $w(g|_{K_3}, g(4))=2$.
%This shows that Theorem \ref{p:Theor} is not obvious. 

\begin{figure}[h]
    \centering
\includegraphics[scale=0.8]{winding.eps}
	\caption{An almost embedding $g:\K \to \R^2$ such that $w(g|_{K_3}, g(4))=2$.}
    \label{ris:winding}
\end{figure}

(d) For an edge $ab$ in graph $K_{3, 3}$ denote by $K_{2, 2}$ the graph $K_{3, 3}-a-b$. 
We conjecture that for any almost embedding $g: K_{3, 3}- ab\to \mathbb{R}^2$ we have $w(g|_{K_{2, 2}}, g(a))-w(g|_{K_{2, 2}}, g(b))=\pm 1$.

(e) %There are similar to 
%Theorem \ref{p:Theor} states that $l(g)=\pm 1$ for any embedding $g:\K \rightarrow \mathbb{R}^2$. 
There are similar to the second sentence of (b) results for linking numbers  for embeddings of graphs into $\mathbb R^3$ (see survey \cite{Sk24}), and also in higher dimensions (see the introduction to \cite{SkKa23}). 
As opposed to Theorem \ref{p:Theor}, in those results the linking number can assume any odd value, see (f). 
%\cite[Theorem 1.4]{SkKa23}. 
A relation between $l(g)$ and linking number is described below. 

Let us define the \textit{linking number} $\lk{(\gamma^1, \gamma^0)}\in \mathbb Z$  of disjoint oriented closed polygonal line 
$\gamma^1:S^1\to \mathbb{R}^2$ and ordered pair  $\gamma^0$ of points in $\mathbb{R}^2 \backslash \gamma^1(S^1)$.     
Take an oriented polygonal line $M$ such that no two vertices from $M$ and vertex from $\gamma^1$ lie in a line, and such that $\partial M$ with order inherited from $M$ coincides with $\gamma^0$. 
See definition of general position in $\R^2$.
By $\lk{(\gamma^1, \gamma^0)}$ denote the sum of the signs of the intersection points of $M$ and $\gamma^1(S^1)$. 
It is well known that

$\bullet$ the linking number is well defined;

$\bullet$ for $\gamma^0=(P, Q)$ we have 
$\lk{(\gamma^1, \gamma^0)}=w(\gamma^1, Q)-w(\gamma^1, P).$

Hence $l(g)=\lk(g|_{K_3}, (g(4), g(5)))$.
Thus Theorem \ref{p:Theor} gives a constraint on the value of the linking number.

(f) Theorem 1.4 in \cite{SkKa23} shows that the higher dimensional analog of Theorem \ref{p:Theor} is false.

\end{remark}

\begin{remark}[more general context: embeddings, almost embeddings and Wu invariant]\label{r:wu}

(a) Take a cycle $\Gamma$ in a graph $K$, and is a point $pt$ in $K\backslash \Gamma$.   For an (almost) embedding $g:K\to \R^2$ the integer $w(g|_{\Gamma}, g(pt))$ is an (almost) isotopy invariant of $g$. 
(This is a part of Haefliger-Wu invariant of $g$, see (d).) 

For some $K$ there is an integer which is the value of this invariant 
%$w(g(\gamma), g(pt))$ 
for some almost embedding, but not for any embedding. % заменить гамма на Гамма
For example, for graph $K_3\sqcup \{4\}$ the value of $w(g|_{K_3}, g(4))$ for an almost embedding $g:K_3\sqcup \{4\}\to \R^2$ can be any integer, but for any embedding $g:K_3\sqcup \{4\}\to \R^2$ we have $w(g|_{K_3}, g(4))\in \{-1, 0, 1\}$. 
Analogous statement holds if we replace the graph $K_3\sqcup \{4\}$ by the graph $K_4$. 
See also (c).   

(b) The following integer is an almost isotopy invariant of almost embedding $g:K_{3,1}\to \mathbb{R}^2$. 
Denote by $K_{3, 1}$ the graph with vertices $\{O, A_1, A_2, A_3\}$, where $\deg O=3$ and $\deg A_m=1$ for $m\in [3]$.
Denote by $l_m$ the edge $(O, A_m)$ for $m\in [3]$. 
%In the plane let $l_1,l_2,l_3$ be polygonal lines in general, joining a point $O$ to points $A_1,A_2,A_3$, respectively and such that $A_i\not\in l_j$ for $i\neq j$.
The invariant is defined to be the number of turns during the following rotation of vector.

From $\overrightarrow{g(A_1)g(A_2)}$ to $\overrightarrow{g(A_1)g(A_3)}$ (as the second point of the vector moves) along $g(l_2)\cup g(l_3)$, then

from $\overrightarrow{g(A_1)g(A_3)}$ to $\overrightarrow{g(A_2)g(A_3)}$ (as the first point of the vector moves) along $g(l_1)\cup g(l_2)$, then

from $\overrightarrow{g(A_2)g(A_3)}$ to $\overrightarrow{g(A_2)g(A_1)}$ (as the second point of the vector moves) along $g(l_3)\cup g(l_1)$, then

from $\overrightarrow{g(A_2)g(A_1)}$ to $\overrightarrow{g(A_3)g(A_1)}$ along $g(l_2)\cup g(l_3)$, then

from $\overrightarrow{g(A_3)g(A_1)}$ to $\overrightarrow{g(A_3)g(A_2)}$ along $g(l_1)\cup g(l_2)$, then

from $\overrightarrow{g(A_3)g(A_2)}$ to $\overrightarrow{g(A_1)g(A_2)}$ along $g(l_3)\cup g(l_1)$.
%так определяется число...

For any integer $k$ there is an almost embedding $g:K_{3, 1}\to \R^2$ such that the invariant equals  $2k+1$.
%E.g.
For example, in Figure \ref{ris:K_3,1} we present an almost embedding $g:K_{3, 1}\to \R^2$ whose invariant equals $3$. 

\begin{figure}[h]
    \centering
\includegraphics[scale=0.8]{K_3,1.eps}
	\caption{example of almost embedding whose invariant equals 3.}
 \label{ris:K_3,1}
\end{figure}  

If PL map $g:K_{3, 1}\to \R^2$ has no self-intersections, then the invariant equals $1$. 
(This is proved by induction on the number of segments in $g(l_i)$.) 

(c) For any graph $K$ denote 
$$\widetilde{K}:=\cup \{\tau \times \sigma \subset K\times K: \tau, \sigma  \text{ are non-adjacent edges of } K\}.$$ 
Then the map $\widetilde g:\widetilde{K} \to S^1$ is
well-defined by the Gauss formula $\widetilde g(x, y):=\dfrac{g(x)-g(y)}{|g(x)-g(y)|}$. 
The equivariant homotopy class $\alpha (g)$ of the map $\widetilde g$ is called \textit{the Haefliger-Wu invariant}. 
The integers in (a), (b) are `parts' of  $\alpha(g)$.
Also integer $l(g)$ is `part` of $\alpha(g)$ for an almost embedding $g: \K\to \R^2$.
For `cohomological' definitions of the Haefliger-Wu invariant see \cite[\S 1.6]{Sk}, \cite[\S 1.6]{Sk23}, \cite[Theorem 4.6, \S5]{Sk06}.  

%Let $a$ be a vertex of $K$, and $\gamma$ a simple cycle in $K-a$. 
%The number $w(g(\gamma), g(a))=\mbox{deg} \frac{g(x)-g(a)}{|g(x)-g(a)|}|_{\gamma}$, and hence is a part of the Wu invariant. 
%, namely, the `value of $U(g)$ at $a\times\gamma$'.

%The Wu invariant assumes values in a  certain group (\emph{symmetric cohomology group} $H^1_s(\widetilde{K}, \mathbb{Z})$ of the configuration space $\widetilde{K}$ of ordered pairs of distinct points in $K$). 
%all elements (of the group in which the Wu invariant assumes its values)
%Not all values!!! of $U(g)$ at $a\times\gamma$ are possible. 

%!!!E.g. for any embedding $g$, vertex $a$ of $K$, and simple cycle $\gamma$ in $K-a$ the number $w(g(\gamma), g(a))$ is in $\{+1,0,-1\}$, but not any integer. 
%Analogous statement for almost embeddings is incorrect: for graph $K_3\bigcup \{4\}$ the value at $4\times K_3$ of Wu invariant for an almost embedding can be any integer.

Theorem \ref{p:Theor} gives a restriction on values of Haefliger-Wu invariant for $\K$. %(see (b)).
So Theorem \ref{p:Theor} and the analogous conjecture for $K_{3,3}$ without edge (see Remark \ref{r:mot}.d) are first (and presumably important) steps towards the interesting problem of description the values of $\alpha (g)$ for almost embeddings $g$.

%(g) The statement in (g') means that the set of all values of Haefliger-Wu invariant realizable by embeddings of $K_{3,1}\to\R^2$ does not coincides with the set of all values of Haefliger-Wu invariant realizable by almost embeddings of $K_{3,1}\to\R^2$.
\end{remark}
 
\begin{remark}[idea of proof and simple reformulation]\label{r:simp}
(a) We illustrate our main idea by proving Theorem \ref{p:Theor} assuming Lemma \ref{p:GreatLemma2}.
In that proof we will reduce Theorem \ref{p:Theor} to the following statement.  

In the plane let $A_1, A_2, A_3$ be vertices of a regular triangle, and $O$ its center.
For $m\in [3]$ %аналогично
let $L_m$ be a polygonal line  joining two of the vertices distinct from $A_m$, and disjoint with the ray $OA_m$.
Then $w(L_1\cup L_2\cup L_3, O) = \pm 1$.

This statement is equivalent to the following exercise.

Let $a_1, a_2, a_3$ be pairwise distinct points on $S^1$.  
Let $g : S^1\to S^1$ be a continuous map such that $ga_m = a_m \not\in gL_m$ for each $m\in [3]$. 
Then $\deg g = 1$.

(a') Theorem \ref{p:Theor} is equivalent to the following generalization of (a).  

In the plane let $A_1, A_2, A_3$ be vertices of a regular triangle, $O$ its center and $O'$ is some point in $\R^2\backslash \{A_1, A_2, A_3, O\}$.
For $m\in [3]$ let 

$\bullet$ $l_i^+$ be a polygonal line  joining $A_i$ and $O$; 

$\bullet$ $l_i^-$ be a polygonal line  joining $A_i$ and $O'$;

$\bullet$$L_m$ be a polygonal line  joining two of the vertices distinct from $A_m$.

Assume that $l_i^+\cap l_j^-=L_i\cap l_j^-=L_i\cap l_j^+= \emptyset$ for $i\neq j$.
Then $w(L_1\cup L_2\cup L_3, O) = \pm 1$.

(b) Some theorems in topology of the plane have technical proofs, while attempts for simpler proofs led to mistakes. 
Examples are the Jordan curve theorem, and completeness of the van Kampen planarity  obstruction. 
For a proof of the latter see \cite{MiPeSh07}. 
This justifies the need of a careful proof of Theorem \ref{p:Theor} (or Lemma \ref{p:GreatLemma2}), and explains why that proof is technical.  
\end{remark}


%\section{Proof of theorem \ref{p:Theor} using lemma \ref{p:GreatLemma2}}

%Denote $S^2:=\{(x,y,z) : x^2+y^2+z^2=1\}.$

%Similarly to the text before theorem \ref{p:Theor} define linking number for disjoint closed oriented polygonal line $\gamma^1$ on $S^2$ and ordered pair $\gamma^0$ of points in $S^2\backslash \gamma^1$.  

Denote by $K_{2,3}$ complete bipartite graph with parts $[3]$ and $\{4, 5\}.$ 
We write $gA$ instead of $g(A)$. 

\begin{lemma}\label{p:GreatLemma2}
    For  any % перенести PL во второй параграф
    almost  embedding $g:\K \rightarrow \R^2$ there is a PL almost embedding $f:\K \rightarrow \R^2$ such that $f|_{K_{2,3}}$ is an embedding and $l(f) = l(g)$.
\end{lemma}


\textit{Proof of Theorem \ref{p:Theor} assuming Lemma \ref{p:GreatLemma2}.}
%We may assume that $g$ is a PL almost embedding. 
By Lemma \ref{p:GreatLemma2} we may assume that $g|_{K_{2,3}}$ is a PL embedding. 
Consider $\R^2$ as a subset of $S^2_{PL}:=\{(x, y, z) \in \R^3\  :\ \max \{|x|, |y|, |z|\}=1 \}$. 
A known result states that for any PL embeddings $g_1, g_2: K_{2, 3}\to S^2_{PL}$ there is a PL homeomorphism $\psi:S^2_{PL}\to S^2_{PL}$ such that $\psi\circ g_1 = g_2$, see e.g. \cite[Theorem 1.6.1]{Sk23}. %посмотреть в Sk23 
Hence it suffices to prove the analog of Theorem \ref{p:Theor} for $\R^2$ replaced by $S^2_{PL}$, and under the additional assumption that $g4=(0, 0, 1)$, $g5=(0, 0, -1)$ and $g([4, j]\cup [j, 5])$ is a meridian (i.e. is polygonal line joining $(0, 0, 1)$ to $(0, 0, -1)$, and lying in intersection of $S^2_{PL}$ and some plane passing through the $z$-axis) for any $j\in [3]$. 
%$L=\{(x, y, z)\in \R^3\ :\ ax+by=0\}$) for any $j\in [3]$. 
This analog is equivalent to Remark \ref{r:simp}.a'. $\Box$

\section{Proof of Lemma \ref{p:GreatLemma2}}

For $g:X\to Y$ and for $A\subset Y$ denote $g^{-1}A:=\{x\in X\ :\ g(x)\in A\}$.

For any points $p, q$ in the edge $(i, j)$ of the graph $\K$ denote by $[p, q]\subset (i, j)$ the part of the edge $(i, j)$ between $p$ and $q$.   

Some points in the plane are in \textit{general position}, if no three of them lie in a line and
no three segments joining them have a common interior point.

A linear map $g:K\to \R^2$ of a graph is in general position, if the image of vertices in $K$ in general position.

A PL map $g:\K \rightarrow \R^2$ is said to be \emph{in general position} if there is a  subdivision $K'$ of $K$ such that the corresponding map $g':K'\to \R^2$ is linear and in general position.

\begin{lemma}\label{p:LessLemma3}
    Suppose that $g:\K \rightarrow \R^2$ is a PL almost  embedding in general position and points $p_1, q_1, p_2, q_2\in \K$ are such that
        
        (0) $g|_{[p_1, q_1]}$ and $g|_{[p_2, q_2]}$ have no self-intersections; 
        
        (1) $g[p_1, q_1]\cap g[p_2, q_2]=\{gp_1=gp_2, gq_1=gq_2\}$;
        
        (2) either $[p_1, q_1] \subset(i, 4)$ and $[p_2, q_2] \subset(i, 5)$ for some $i\in [3]$, or $[p_1, q_1] \subset(k, l)$ and $[p_2, q_2] \subset(k, m)$ for some $k\in\{4, 5\}$, $l\neq m\in [3]$;
        
        (3) $g[5]$ is contained in the closure of some connected component of $\R^2\backslash (g[p_1, q_1]\cup g[p_2, q_2]).$ 
        
    Then there is a PL almost embedding $f:\K \rightarrow \R^2$ such that $l(f) = l(g)$ and the number of the self-intersection points of $f$ is less than the number of the self-intersection points of $g$.
 
\end{lemma}


\smallskip
\textit{Proof of Lemma \ref{p:LessLemma3}.}    Denote $\gamma :=[p_1, q_1]\cup [p_2, q_2]$. Denote by $U$ the complement to the closure of the connected component from condition $(3)$. 

A interval $J \subset g^{-1}(\mbox{Cl}U)$ 
%K_5-(4,5)-\gamma$ 
is called {\it $U$-interval} if $gJ$ joins either two points  in $g[p_1, q_1]$ or two points in $g[p_2, q_2]$. 
%and $gJ\subset \mbox{Cl}\  U$.
    
    
    \begin{figure}[h]
    \centering
\includegraphics[scale=0.8]{lemma3.1.eps}
       \label{ris:Lemm3.1}
	 \caption{Upper Figure is used if $p_1, q_1 \in [5]$ or $p_2, q_2 \in [5]$, and lower is used otherwise}
    
\end{figure}
    
    For any $p, q \in \gamma$ denote by $[p, q]_{\varepsilon}$ intersection of some small neighborhood of $[p, q]$ and the edge containing $[p, q]$.
    
    {\it Suppose that there are no $U$-intervals.} Change $g$ on the ${[p_1, q_1]_{\varepsilon}\cup [p_2, q_2]_{\varepsilon}}$ as in Figure %\ref{ris:Lemm3.1}
    1. Denote the resulting map by $f$.  
    For any interval $J$ denote by $\alpha_J$ the edge of $\K$ containing $J$. 
    
    Consider any edge $\beta$ of $\K$ 
    % that without loss of generate 
    non-adjacent to  $\alpha_{[p_1, q_1]}$. 
    Since $g$ is an almost embedding, $g\beta \cap g\alpha_{[p_1, q_1]}=\emptyset$. 
    Hence the intersection $g\beta\cap \mbox{Cl}U$ consists of images of $U$-intervals. 
    Since there are no $U$-intervals, it follows that $g\beta \cap g\gamma=\emptyset$.
    Analogously $g\beta \cap g\gamma=\emptyset$   for any edge $\beta$ non-adjacent to $\alpha_{[p_2, q_2]}$. 
    Hence $f([p_1, q_1]_{\varepsilon}\cup [p_2, q_2]_{\varepsilon})$ does not intersect any edge non-adjacent to one of the edges $\alpha_{[p_1, q_1]}$ and $\alpha_{[p_2, q_2]}$.
    
    The restrictions of $f$ and $g$ to the complement of $[p_1, q_1]_{\varepsilon}\cup[p_2, q_2]_{\varepsilon}$ in $\K$ coincide. 
    Hence for any non-adjacent edges $\beta', \beta$ distinct from $\alpha_{[p_1, q_1]}$ and $\alpha_{[p_2, q_2]}$ we have $$f\beta \cap f\beta' = g\beta \cap g\beta' = \emptyset.$$ Hence $f$ is an almost embedding. 
    
    Since the restrictions of $f$ and $g$ to $K_3 \cup \{4, 5\}$ coincide, it follows that $l(f)=l(g)$. Then $f$  satisfies the conclusion of Lemma \ref{p:LessLemma3}.
    
    \begin{figure}[h]
    \centering
        \includegraphics[scale=0.6]{lemma3.2.eps}
         \caption{}
         \label{ris:Lemm3-2}
\end{figure}

    
    {\it Suppose that there is a  $U$-interval.}
    For any $U$-interval $[p, q]$ denote by $\bar p, \bar q\in \gamma$ the points such that $gp=g\bar p$, $gq=g\bar q$ and $\{p, q\}\neq \{\bar p, \bar q\}$. 
    Take a  $U$-interval $[p, q]$ such that for any other  $U$-interval $[p', q']$ 
    the interval $[\bar p, \bar q]$ does not contain $[\bar{p'}, \bar{q'}]$. 
    
    Denote by $\Gamma:=[p, q] \cup [\bar p, \bar q]$. 
    Denote by $V$ the connected component of $\R^2\backslash g\Gamma$ such that $V\subset U$. 
    Similarly define {\it $V$-interval}. 
    For every $V$-interval $[p', q']$ we have $[\bar p', \bar q']\subset [p, q]$. 
    By the choice of $[p, q]$ there are no $V$-intervals. 
    Change $g$ on the ${[p, q]_{\varepsilon}\cup [\bar p, \bar q]_{\varepsilon}}$ as in Figure \ref{ris:Lemm3-2}. 
    Proof that $f$ is an almost embedding is obtained from such a proof for the case where there are no $U$-intervals by  replacing $U$ by $V$, $[p_1, q_1]$ by $[p, q]$, $[p_2, q_2]$ by $[\bar p, \bar q]$, and $\gamma$ by $\Gamma$.

%Let us define the \textbf{linking number} $\lk{(\gamma^1, \gamma^0)}\in \mathbb Z$  of disjoint oriented closed polygonal line $\gamma^1$ in $\mathbb{R}^2$ and ordered pair  $\gamma^0$ of points in $\mathbb{R}^2 \backslash \gamma^1$. Take an oriented polygonal line $M$  in general position to $\gamma^1$ such that $\partial M$ with order inherited from $M$ coincides with $\gamma^0$. By $\lk{(\gamma^1, \gamma^0)}$ denote the sum of the signs of the intersection points of $M$ and $\gamma^1$. It is well known that

%$\bullet$ the linking number is well defined,

%$\bullet$ for $\gamma^0=(P, Q)$ we have $$\lk{(\gamma^1, \gamma^0)}=w(\gamma^1, Q)-w(\gamma^1, P).$$
    
If $p, q \notin K_3$, then the restrictions of $f$ and $g$ to $K_3$ and $\{4, 5\}$ coincide. 
    So $l(f) = l(g)$. 
    
Assume that $p, q \in K_3$.
Denote by $\Tilde{\Gamma}:S^1\to \R^2$ the composition of some homeomorphism $h:S^1\to g(\Gamma)$ and inclusion $i:g(\Gamma)\to \R^2$.
Then $fK_3=gK_3+g\Gamma$. 
So we have
$$l(f) = w(f|_{K_3}, f4)- w(f|_{K_3}, f5)= (w(g|_{K_3}, f4) + w(\Tilde{\Gamma}, f4)) - (w(g|_{K_3}, f5)+w(\Tilde{\Gamma}, f5))=$$
$$(w(g|_{K_3}, f4) -w(g|_{K_3}, f5)) + (w(\Tilde{\Gamma}, f4) -w(\Tilde{\Gamma}, f5))=l(g)+w(\Tilde{\Gamma}, f4) -w(\Tilde{\Gamma}, f5)=l(g),$$
where last equality holds because $g(\Gamma) \subset \mbox{Cl}\ U$ and $g\{4, 5\} \not \subset U$.  
    
    It follows that $f$  satisfies the conclusions  of Lemma \ref{p:LessLemma3}.
    $\Box$
    
\medskip    
\textit{Proof of Lemma \ref{p:GreatLemma2}.} 
It is sufficient prove Lemma \ref{p:GreatLemma2} under additional assumption that $g$ is a PL map in general position.

Figure \ref{ris:Reid} shows that for any PL almost embedding $g:\K \rightarrow S^2_{PL}$ there is a PL almost embedding $f:\K \rightarrow S^2_{PL}$ such that for any edge $\alpha$ of $K_{2,3}$ the restriction of $f$ to $\alpha$ is embedding and  $l(f) = l(g)$. 

\begin{figure}[h]
    \centering
        \includegraphics[scale=1.3]{R_0.eps}
        \caption{}
        \label{ris:Reid}
\end{figure}


Then it is sufficient prove Lemma \ref{p:GreatLemma2} under additional assumption that restriction of $g$ to any edge of $K_{2,3}$ is embedding.


If the restriction of $f$ to $K_{2,3}$ is a PL embedding, then take $f=g$. 
In the  opposite case it is sufficient to show that there is map $f$ with fewer number of self-intersection points and $l(f) = l(g)$. 

In the following text we consider the case when $g(i, j)\cap g(i, k)\neq\emptyset$ 
for some $i \in \{4, 5\}$ and $j\neq k\in [3]$. 
The case when $g(i, 4)\cap g(i, 5)\neq \emptyset$ for some $i \in [3]$, is considered similarly, 
replacing $(1, 2, 3, 4, 5)$ by $(4, 5, 3, 2, 1)$.

Without loss of generality assume $g(5, 1)\cap g(5,2)\neq \{g5\}$. Denote $p_1=p_2=5$. Denote by $q_1\in (5, 1)$ the first intersection point with $(5, 2)$ and denote by $q_2\in (5, 2)$ the point such that points $gq_1=gq_2$. Since $q_1$ is the first intersection point with $(5, 2)$, we have $g[5, q_1]\cap g[5, q_2]=\{g5, gq_1\}$. Hence points $p_1, q_1, p_2, q_2$ satisfy the property $(1)$ of Lemma \ref{p:LessLemma3} (fig. \ref{ris:LastConst}). Points $p_1, q_1, p_2, q_2$ satisfy the property $(2)$ of Lemma \ref{p:LessLemma3} for $k=5, l=1, m=2$.   

\begin{figure}[h]
    \centering
        \includegraphics[scale=0.6]{g5-g1,g2.eps}
	 \caption{}
  \label{ris:LastConst}
\end{figure}

Then $g[5, q_1]\cup g[5, q_2]$ divide $\R^2\backslash g[5, q_1]\cup g(5, q_2)$ into two connected components. 

If  $g[4]$ is contained in one of the connected components, then points $p_1=5, q_1$ and $p_2=5, q_2$ satisfy the property $(3)$ of Lemma \ref{p:LessLemma3}. So by Lemma \ref{p:LessLemma3} there is map $f$ with fewer number of self-intersection points and satisfying property $l(f) = l(g)$. 

Now suppose that $g[4]$ is not contained in the one connected component. Since $g$ is PL almost embedding, we have $g(3, 4)\cap (g[5,q_1]\cup g[5, q_2])\subset g(3, 4)\cap (g[5, 1]\cup g[5, 2])=\emptyset$. It follows that there is a connected component $U$ which contains $g3$ and $g4$. Since $g[4]$ is not contained in $U$, without loss of generality assume that $g2 \notin U.$

Since $g(4, 2)\cap g(5, 1)=\emptyset$, $g4 \in U$ and $g2 \notin U$ it follows that $g(4, 2)\cap g(5, 2)\neq \{g2\}$. Denote by $p_1' \in g(4, 2)$ the first intersection point between $(4, 2)$ and $(5, 2)$. Denote $p_2':=(g^{-1}gp_1')\cap (5, 2)$. Since $\{g2\}\subset g[p_1', 2]\cap g[5, 2]\neq \emptyset$ it follows that there is the first intersection point $q_1'\in [p_1, 2]$ between $[p_1', 2]$ and $(5, 2)$. Denote $q_2':=(g^{-1}gq_1')\cap (5, 2)$. Since $q_1'$ is the first intersection point between $[p_1', 2]$ and $(5, 2)$, we have $g[p_1', q_1']\cap g[p_2', q_2']=\{gp_1, gq_1\}$. Hence points $p_1', p_2', q_1', q_2'$ satisfy the property $(1)$. Points $p_1', p_2', q_1', q_2'$ satisfy the property $(2)$ of Lemma \ref{p:LessLemma3} for $i=2$ (fig. \ref{ris:LastConst}).


\begin{figure}[h]
    \begin{center}
        
        \includegraphics[width=100mm]{g4-g2.eps}
        
        \label{ris:MainCont}
	 \caption{}
    \end{center}
\end{figure}

Then $g[p_1', q_1']\cup g[p_2', q_2']$ divides $\R^2\backslash g[p_1', q_1']\cup g[p_2', q_2']$ into two connected component. From Jordan curve theorem it follows that:
    if  $U,V\subset S^2_{PL}$ are open polygons and $U\cap \partial  V=\emptyset$, then either $U\cap V = \emptyset$ or $U\cap Int(\R^2\backslash V) = \emptyset$.

%\textit{Proof of lemma \ref{p:SmallLemma2}.} For any points $p, q\in U$ take a curve $\gamma_{p, q}: [0, 1] \rightarrow U$ joining $p$ and $q$. Since $U\cap \partial  V=\emptyset$ it follows that either $\gamma_{p, q}^{-1}(V\cap \gamma_{p, q}[0, 1])=\emptyset$ or $\gamma_{p, q}^{-1}(Int(S^2_{PL}\backslash V)\cap \gamma_{p, q}[0, 1])=\emptyset$. Since $U$ is connected it follows the statement of the lemma.  $\Box$

It follows that $g3, g4, g5\notin V$. Since $g$ is PL almost embedding, we have $g(1, 3)\cap (g[p_1', q_1']\cup g[p_2', q_2'])=\emptyset$. It follows that $g1\notin V$. 

If  $g2\notin V$, then points $p_1', p_2', q_1', q_2'$ satisfy the property $(3)$ of Lemma \ref{p:LessLemma3}. So by Lemma \ref{p:LessLemma3} there is map $f$ with fewer number of self-intersection points and  $l(f)=l(g)$.  

\begin{figure}[h]
    \centering
        
        \includegraphics[scale=0.6]{gq.eps}
        
	 \caption{}
  \label{ris:gq}
\end{figure}

Now assume that $g2 \in V$. Since $g$ in general position it follows that there is point $q\in [q_2', 2]$ such that $gq\in (\R^2\backslash V)\cap g[q_2', 2]$ and $q\neq q_2'$ (fig. \ref{ris:gq}). Since $g2 \in V$ and $gq \notin V$ it follows that  $g[q, 2]\cap (g[p_1', q_1']\cup g[p_2', q_2'])\neq \emptyset$.
Then $g(5, 2)$ has self-intersection point or $g(5, 2)\cap g[p_2', q_2'] \neq \{gp_2', gq_2'\}$. This contradiction proves the Lemma \ref{p:GreatLemma2}.

\begin{thebibliography}{3}

        \bibitem[SkTa19]{SkTa19}
        {\it A. Skopenkov, M.Tancer.} Hardness of almost embedding simplicial complexes in $\R^d$, Discr. and Comp. Geom.61:2 (2019), 452-463, https://arxiv.org/abs/1703.06305
    
        \bibitem[Sk06]{Sk06}
        {\it  Skopenkov A.} Embedding and knotting of manifolds in Euclidean spaces. arXiv:math/0604045.
        
        \bibitem[SkKa23]{SkKa23}
        {\it Karasev R., Skopenkov A.} Some `converses' to intrinsic linking theorems. Discrete Comput. Geom. 70(2023), no.3, 921--930. arXiv:2008.02523.

        
        \bibitem[MiPeSh07]{MiPeSh07}
        {\it Michael J. Pelsmajer, Marcus Schaefer, and Daniel \v Stefankovi\v c}. Removing even crossings. J. Combin. Theory Ser. B, 97(4):489–500, 2007.

        %\bibitem[Sch11]{Sch11}
        %{\it M. Schaefer.} Hanani-Tutte and related results. In Geometry -- intuitive, discrete,
        %and convex, Bolyai Soc. Math. Stud., 24 (2013), 259-299. https://ovid.cs.depaul.edu/documents/htsurvey.pdf.
        
        \bibitem[Sk23]{Sk23} \emph{A. Skopenkov.} Invariants of graph drawings in the plane (in Russian). Mat. Prosveschenie 31 (2023), 74-127.

        \bibitem[Sk24]{Sk24} \emph{A. Skopenkov.} arXiv:1402.0658v6 [math.MG] 22 Oct 2023
Realizability of hypergraphs and intrinsic link theory, Mat. Prosveschenie, 32. (2024), 125–159 https://arxiv.org/pdf/1402.0658.pdf

        {\bibitem[Sk]{Sk} \emph{A. Skopenkov.} Algebraic Topology From Algorithmic Standpoint, draft of a book, mostly in Russian,
        %\linebreak
        \url{http://www.mccme.ru/circles/oim/algor.pdf}.}

    
            
\end{thebibliography}

\end{document} 




%It's not hard to see that after transformation as in the picture \ref{ris:4i i5 1} and \ref{ris:4i i5 2} winding number of $\gSigmC$ around $g(4)$ and $g(5)$ don't change.



%And $\lk(f(B_{j})\cup L_{j})\in \{-1, 0, 1\}$, where the orientation of the polygonal line $f(B_{j})\cup L_{j}$ is inherited from the polygonal line $f(S^1)$.

%We have that $[3]$ contained either into one arc $B_i$ or into two arcs $B_i$ and $B_j$ or into three arcs $B_i, B_j$ and $B_k$.


%Suppose that $[3]$ contained into one arc $B_i$. Since $$\lk(\fSigmC, \fSigmP)= \lk(\gamma_1^1, \fSigmP)+ \ldots + \lk(\gamma_n^1, \fSigmP)$$ and $\lk(\gamma_i^1, \fSigmP)\in \{-1, 0, 1\}$, we see that exist such $k$ that $B_k$ is subset of some edge and  $$\lk(f(B_{k})\cup L_k, f(\ZerSphere))\neq 0.$$ Without loss of generality, we can assume that $B_{k}\subset (1,2)$. Then $$(f(B_{k})\cup L_k)\cap f((4, 3)\cup (3, 5))\neq \emptyset.$$ Since $f((4, 3)\cup (3, 5)) \cap L =\emptyset$, we have $f((1, 2)) \cap f((4, 3)\cup (3, 5)) \neq \emptyset$,  that contradicts the condition of almost embedding. 

%Suppose that $[3]$ contained into two arcs $B_i$ and $B_j$. %Without loss of generality, we can assume that $[2]\subset B_i$, $f^{-1}(A_i)\in (3, 1)$, $f^{-1}(A_{i+1})\in (2, 3)$,  $f^{-1}(A_j)\in (3, 1)$, $f^{-1}(A_{j+1})\in (2, 3)$. 
%Take an open linearly connected neighborhood $V\subset S^2_{PL}$ such that $|\partial  V \cap f(B_i)|=|\partial  V \cap f(B_j)|=2$. Take a point $P\in V$. Let connect  $P$ with $\partial  V \cap f(B_i)$ and $\partial  V \cap f(B_j)$. Let consider $f(1), f(2), f(3), f(4), f(5), P$ and polygonal lines, that connect them. This is how almost embedding of $K_{3, 3}$ is constructed. This is a contradiction.

%\begin{center}
        
%        \includegraphics[scale=0.8]{K_33.eps}
	 
% \end{center}

%Suppose that $[3]$ contained into three arcs $B_i, B_j$ and $B_k$. %Without loss of generality, we can assume that $[2]\subset B_i$, $f^{-1}(A_i)\in (3, 1)$, $f^{-1}(A_{i+1})\in (2, 3)$,  $f^{-1}(A_j)\in (3, 1)$, $f^{-1}(A_{j+1})\in (2, 3)$. 
%Take an open linearly connected neighborhood $V\subset S^2_{PL}$ such that $|\partial  V \cap f(B_i)|=|\partial  V \cap f(B_j)|=|\partial  V \cap f(B_k)|=2$. Take a point $P\in V$. Let connect  $P$ with $\partial  V \cap f(B_i),$  $\partial  V \cap f(B_j)$ and $\partial  V \cap f(B_k)$. Let consider $(1), f(2), f(3), f(4), f(5), P$ and polygonal lines, that connect them. This is how almost embedding of $K_{3, 3}$ is constructed. This is a contradiction. $\Box$