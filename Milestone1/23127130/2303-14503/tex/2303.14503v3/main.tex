	\documentclass[12pt]{article}
\textwidth=17.5cm\textheight=25.5cm\hoffset=-2cm\voffset=-3.5cm
%\usepackage[cp1251]{inputenc}
\usepackage[english]{babel}
\usepackage[shortlabels]{enumitem}
\usepackage{tikz}
\usepackage{amsfonts,amsmath,epsf,amssymb,amsthm,graphicx,afterpage}%,url}
\usepackage[
    draft = false,
    unicode = true,
    colorlinks = true,
    allcolors = blue,
    hyperfootnotes = true,
    citecolor = red
]{hyperref}
\usepackage{epsfig}
\graphicspath{{morefig/}{figures/}{figures00/}{figures02/}{figures05/}{figlms/}}
\newcommand{\?}{\nobreak\hskip.167em\nobreak\hskip0pt}
\DeclareGraphicsExtensions{.pdf,.png,.jpg}
\def\R{\mathbb R}
\def\K{K_5-45}
\def\Cycle{\Sigma^1}
\def\ZerSphere{\Sigma^0}
\newcommand{\lk}{\operatorname{lk}}
\def\fSigmC{\displaystyle f|_{\Cycle}}
\def\fSigmP{\gamma^0}
\DeclareMathOperator{\Cl}{Cl}
\DeclareMathOperator{\Int}{Int}
\def\gSigmC{\displaystyle g|_{\Cycle}}
\def\gSigmP{\gamma^0}

\let\fiverm\tiny
\let\Bbb=\mathbb
\long\def\comment#1\endcomment{}

\newtheoremstyle{mydefinition}% name
{3pt}%      Space above
{3pt}%      Space below
{\normalfont}%         Body font
{\parindent}%         Indent amount (empty = no indent, \parindent = para indent)
{\bfseries}% Thm head font
{.}%        Punctuation after thm head
{ }%     Space after thm head: " " = normal interword space;
%       \newline = linebreak
{}%         Thm head spec (can be left empty, meaning `normal')

%\theoremstyle{mydefinition}
\theoremstyle{plain}
\newtheorem{theorem}{Theorem}
\newtheorem{pr}[theorem]{Problem}
\newtheorem{example}[theorem]{Example}
\newtheorem{lemma}[theorem]{Lemma}
\newtheorem{sentence}[theorem]{Предложение}
\newtheorem{Assertion}[theorem]{Assertion}
\newtheorem{approvalo}[theorem]{Утверждение}

\theoremstyle{definition}
\newtheorem{remark}[theorem]{Remark}
\newtheorem{construction}[theorem]{Construction}

\title{On winding numbers of $K_5$ minus an edge in the plane}
\author{Garaev T. R. }

%\date{August 2021}

\begin{document}

\maketitle

\begin{abstract}
	Let $K$ be the graph on vertices $\{1, 2, 3, 4, 5\}$, and having all edges except $(4, 5)$. 
	A continuous map $f:K\to \R^2$ is called an \emph{almost embedding} if $f$-images of non-adjacent edges are disjoint. 
	Take the winding numbers of the $f$-image of the oriented cycle $(1, 2, 3)$ around $f(4)$ and around $f(5)$. 
	We prove that the difference of these numbers equals $\pm 1$.
	This is surprising, because in other similar situations analogous statement is wrong. 
	%Ссылками разобраться??? 
\end{abstract}
\tableofcontents

\section{Main result}\label{s:mainres}

Denote by $[n]$ the set $\{1, \ldots, n\}$.
Denote by $K_n$ the complete graph with the vertex set $[n]$, and by $\K$ the graph obtained from $K_5$ by deleting the edge $45$. 
We identify a graph and the body of the graph.
A map $f:\K\to\R^2$ (continuous, or PL)  is called an {\it almost embedding} if $f$-images of non-adjacent edges are disjoint 
(see definition of PL map at the beginning of \S\ref{s:discuss}). 
%See a comment Remark \ref{r:almost_emb}.\ref{r,e:almost_emb-PL-Con}


We give a restriction on certain winding numbers for almost embeddings of graph $K_5$ minus an edge in the plane, see rigorous formulation in Theorem \ref{p:Theor}.
Motivation, discussion, and proof of Theorem \ref{p:Theor} assuming Lemma \ref{p:GreatLemma2} are given in \S \ref{s:discuss}.
Lemma \ref{p:GreatLemma2} is proved in \S \ref{s:proof}. 

For a graph $K$ denote by

$\bullet$ $ij$ the edge joining vertices $i$ and $j$ in $K$;

$\bullet$ $j_1\ldots j_n$ the simple oriented cycle $(j_1, \ldots, j_n)$ in $K$.

Denote by $w(\gamma, A)$ the \emph{winding number}\footnote{See definition for $\gamma$ a closed polygonal line in \cite[\S 2.3]{Sk18}, and for $\gamma$ a closed curve in 
\linebreak	
	\url{https://en.wikipedia.org/wiki/Winding_number}.}
 of a closed curve $\gamma :S^1\to \R^2$ around a point $A\in\mathbb{R}^2 \backslash \gamma(S^1)$.

Let $K$ be a graph, and $f:K\to\R^2$ a continuous map.  
Let $C$ be an oriented cycle in $K$, and $v$ a vertex of $K$ such that $f(v)\notin f(C)$. 
Set 
$$w_f(C, v):=w(f|_C, f(v)).$$ 

\begin{theorem}\label{p:Theor}
	If Lemmas \ref{l:improv_exist} and \ref{l:improv_exist_last_case} are correct, then for any continuous almost embedding $g:\K \rightarrow \mathbb{R}^2$ we have
    $$l(g):=w_g(123, 4)-w_g(123, 5)=\pm 1.$$
\end{theorem}

Lemmas \ref{l:improv_exist} and \ref{l:improv_exist_last_case} correct a mistake in arXiv version 2 of this paper. 
I found and corrected the mistake during my work on critical remarks by E. Alkin. 
My work on his new critical remarks (sent about May, 18, 2025) is not yet completed. 
So the arguments for Lemmas \ref{l:improv_exist} and \ref{l:improv_exist_last_case} is named `sketch of a proof'.
%The author did not find any mistakes in the proofs of Lemmas \ref{l:improv_exist} and \ref{l:improv_exist_last_case}, but they have not been verified by a referee.

Theorem \ref{p:Theor} is interesting because:

$\bullet$ a similar invariant for $K_4$ assumes any odd value, see definition in Remark  \ref{r:all_odd}.\ref{r,e:all_odd-cyc_K4}; 

$\bullet$ a similar invariant for $K_{3,3}$ conjecturally assumes any odd value, see definition in Remark \ref{r:all_odd}.\ref{r,e:all_odd-K33};

$\bullet$ the analogous statements in higher dimensional Euclidean spaces are wrong, see \cite[Theorem 1.3]{Ni22} and Remark \ref{r:mot}, so Theorem \ref{p:Theor} disproves Conjecture 1.6.a from \cite{KS20};

$\bullet$ $l(g)$ being odd is a special case of the famous van Kampen theorem, which has no integer version as explained in Remarks \ref{r:vaKa}.\ref{r,e:vaKa-def}, \ref{r,e:vaKa-int_ana_ref}. 


\section{Discussion of Theorem \ref{p:Theor}}  \label{s:discuss}

Remarks \ref{r:almost_emb}, \ref{r:mot} and \ref{r:all_odd}-\ref{r:vaKa}, \ref{r:just} are formally not used in the proof of Theorem \ref{p:Theor}.

A map $g:K \rightarrow \R^2$ of graph $K$ is said to be \textbf{piecewise linear}, if there is a  subdivision $K'$ of $K$ such that the corresponding map $g':K'\to \R^2$ is linear on any edge of $K'$.
We write 'PL' instead of 'piecewise linear'.

We define an almost embedding for any graph.
For a graph $\K$ this definition is different from but equivalent to the one given in \S 1.
A map $f:K\to \R^2$ of a graph $K$ is called an \textbf{almost embedding} if $f(\alpha)\cap f(\beta)=\varnothing$ for any two non-adjacent simplices (i.e. vertices or edges) $\alpha, \beta \subset K$.
In other words, if

$\bullet$ the images of non-adjacent edges are disjoint,

$\bullet$ the image of a vertex is not contained in the image of any edge non-adjacent to this vertex, 

$\bullet$ the images of distinct vertices are distinct.

We write 'embedding' and 'almost embedding' instead of 'PL or continuous embedding' and 'PL or continuous almost embedding' respectively, because of Remark \ref{r:almost_emb}.\ref{r,e:almost_emb-PL-Con}.

\begin{remark}[almost embeddings]\label{r:almost_emb}	
	\begin{enumerate}[(a)]
		\item\label{r,e:almost_emb-other_branches} Almost embeddings naturally appear in topological graph theory, in combinatorial geometry, in topological combinatorics, and in studies of embeddings (of graphs in surfaces, and of hypergraphs in higher-dimensional Euclidean space).
		See more motivations in \cite[\S 1, ‘Motivation and background’]{ST17}, \cite[\S6.10, ‘Almost embeddings, $\mathbb{Z}_2$- and $\mathbb{Z}$-embeddings’]{Sk}.
		
		\item\label{r,e:almost_emb-PL-Con}
		The property of being an almost embedding is stable, i.e., is preserved under small enough perturbation of a map (as opposed to the property of being an embedding). 
		Thus any continuous almost embedding can be approximated by a PL almost embedding. 
		For this reason Theorem \ref{p:Theor} is equivalent to the same statement for PL almost embeddings. 
		However even for this case Theorem \ref{p:Theor} is not obvious. 
		
		\item\label{r,e:almost_emb-his_alg} 
		An algebraic version of almost embeddings ($\mathbb{Z}_2$-embeddings) appeared in 1930s and is actively studied in graph theory since 2000s. 
		See e.g. surveys \cite{SS13}, \cite[\S6.10 `Almost embeddings, $\mathbb{Z}_2$- and $\mathbb{Z}$-embeddings']{Sk}, and the papers \cite{FK19}, \cite{Ky16}.
		The invariant $l(g)$, and the similar invariants for graphs $K_4$ and $K_{3,3}$ defined in Remarks \ref{r:all_odd}.\ref{r,e:all_odd-cyc_K4},\ref{r,e:all_odd-K33} assume only odd values for $\mathbb{Z}_2$-embeddings $g$. 
		These are proved similarly to the analogues of these results for almost embeddings, proved in Remarks \ref{r:vaKa}.\ref{r,e:vaKa-def},\ref{r,e:vaKa-kam_k33}, \ref{r:all_odd}.\ref{r,e:all_odd-cyc_K4}.
		Presumably the analog of Theorem~\ref{p:Theor}  for $\mathbb{Z}_2$-embeddings is incorrect.
		We conjecture that the analog of Theorem~\ref{p:Theor}  for $\mathbb{Z}$-embeddings (defined in \cite[\S 1.1]{Sk21}) is correct.
		
		\item\label{r,e:almost_emb-emb} The analog of Theorem~\ref{p:Theor} for embeddings instead of almost embeddings is much simpler (and is close to Jordan Curve Theorem). 
	\end{enumerate}
\end{remark}

\begin{remark}[motivation]\label{r:mot}
A \emph{hypergraph} is a higher-dimensional analog of graph: together with edges joining pairs of points one considers triangles spanned by triples of points, etc., see definiton e.g. in \cite[\S 6.3]{Sk}.   
A classical problem in topology, combinatorics and computer science is to find criteria for realizability (and algorithms recognizing realizability) of hypergraphs in Euclidean space of given dimension $d$. 

Such a criterion was obtained in 1930s-1960s by classical figures in topology, see survey \cite[\S4, \S5]{Sk06}.
The criterion is stated in terms of certain configuration space, yielded many specific corollaries, and works for $2d\ge3k+3$, where $k$ is the dimension of the hypergraph, see survey \cite[\S5]{Sk06}. 
A polynomial algorithm based on this criterion was obtained in 2013 \cite{CKV}. 
The non-existence of a polynomial algorithm for $2d<3k+2$ was announced in 2019 by 
Marek Filakovsk\'y, Ulrich Wagner and Stephan Zhechev \cite{FWZ}. 
A mistake was found in 2020 by Arkadiy Skopenkov (and recognized by the authors). 
The mistake was that in a higher-dimensional analog of Theorem \ref{p:Theor} for embeddings certain linking number can assume value distinct from $\pm1$. 
In 2020 Roman Karasev and Arkadiy Skopenkov showed that this linking number for almost embeddings assume any odd value \cite[Theorem 1.5]{KS20}.

The analogous (to the linking number) invariants for graphs in the plane are $l(g)$ (by Remark~\ref{r:link}), and the similar invariants for the graphs $K_4$ and $K_{3,3}$ defined in Remarks \ref{r:all_odd}.\ref{r,e:all_odd-cyc_K4},\ref{r,e:all_odd-K33}.  
%We conjecture that these invariants assume any odd values, see Remarks \ref{r:all_odd}.\ref{r,e:all_odd-cyc_K4}, \ref{r,e:all_odd-K33}.
These analogues and other similar invariants assume \emph{any} odd values, see the first two bullets after Theorem~\ref{p:Theor}, and 
Remarks \ref{r:oth_inv}.\ref{r,e:oth_inv-cyc_wu},\ref{r,e:oth_inv-tri_wu}. %(or assume any integer value in Remark \ref{r:all_odd}.\ref{r,e:all_odd-diff} or conjecturally any even value in Remark \ref{r:oth_inv}.\ref{r,e:oth_inv-wu_diff}).
The invariant $l(g)$ and its analogues assume {\it only} odd values, see proofs in Remarks \ref{r:all_odd}.\ref{r,e:all_odd-cyc_K4},  \ref{r:vaKa}.\ref{r,e:vaKa-def},\ref{r,e:vaKa-kam_k33}.  



%Theorem \ref{p:Theor} shows that the analogs of \cite[Theorem 1.5]{KS20} and conjectures from Remarks \ref{r:all_odd}.\ref{r,e:all_odd-cyc_K4},\ref{r,e:all_odd-K33} for $l(g)$ are false. 

\end{remark}

\begin{remark}[linking number]\label{r:link} 
	\begin{enumerate}[(a)]
		\item \label{r,e:link-def_plain}
		Let $A,B,C,D$ be points in the plane, of which no three belong to a line. 
		Define the sign of intersection point of oriented segments $\overrightarrow{AB}$ and	$\overrightarrow{CD}$ as the number $+1$ if $ABC$ is oriented clockwise and the number $-1$ otherwise.
		
	Let us define the \textit{linking number} $\lk{(\gamma, (P, Q))}\in \mathbb Z$  of a disjoint oriented closed polygonal line $\gamma=A_1\ldots A_n$ and an ordered pair  $(P, Q)$ of points from the complement of $\gamma$.     
	Take an oriented polygonal line $M$ joining $P$ to $Q$ such that
	for any segment $AB$ of $M$ and segment $CD$ of $\gamma$ we have $\{A, B\}\cap CD=\{C, D\}\cap AB=\varnothing$.
	By $\lk{(\gamma, (P, Q))}$ denote the sum of the signs of the intersection points of such pairs of segments.  
	%$M$ and $\gamma$. 
	It is well known that
	
	$\bullet$ the linking number is well defined;
	
	$\bullet$ for PL map $g:\K\to \R^2$ in general position we have 
	$\lk{(\gamma, (Q, P))}=w(\gamma, Q)-w(\gamma, P).$
	
	Hence $l(g)=\lk(g|_{123}, (g4, g5))$.
	
	\item \label{r,e:link-def_sphere} 
	Denote $S^2_{PL}:=\{(x, y, z) \in \R^3\  :\ \max \{|x|, |y|, |z|\}=1 \}$.
	Similarly we define the \textit{linking number} on the sphere, replacing $\R^2$ by $S^2_{PL}$.
	Identify $S^2_{PL}\backslash (0, 0, 1)$ with $\R^2$ by some homeomorphism, then the restriction of linking number on $S^2_{PL}$ to $\R^2$ coincides with the linking number on $\R^2$.   
\end{enumerate}
\end{remark}

A homotopy $F:K\times [0, 1]\to\R^2$ is called an {\it (almost) isotopy}, if for any $t\in [0, 1]$ the map $F|_{K\times t}$ is an (almost) embedding.

\begin{remark}[some invariants of (almost) embeddings]\label{r:all_odd}
In this remark we introduce some invariants of an (almost) embedding up to (almost) isotopy.

	\begin{enumerate}[(a)]	
	\item\label{r,e:all_odd-diff}  
	Take some oriented cycle $C$ in a graph $K$, and some vertex $v$ in $K\backslash C$.
	For an (almost) embedding $g:K\to \R^2$ the integer $w_g(C, v)$ is an (almost) isotopy invariant of $g$. 
	
	For some $K$ there is an integer that is the value of this invariant for some almost embedding, but not for any embedding.
	E.g., for graph $123\sqcup \{4\}$ the number $w_g(123, 4)$ for an almost embedding $g:123\sqcup \{4\}\to \R^2$ can be any integer, but for any embedding $g:123\sqcup \{4\}\to \R^2$ we have $w_g(123, 4)\in \{-1, 0, 1\}$ (the latter is close to Jordan Curve Theorem). 
	Analogous statement holds if we replace the graph $123\sqcup \{4\}$ by the graph $K_4$. 
	
	For any integer $k$ there is an almost embedding $g:\K \to \R^2$ such that $w_g(123, 4)=k$, see Figure \ref{ris:winding}. 
	However for any embedding $g:\K \to \R^2$ we have $w_g(123, 4)\in \{-1, 0, 1\}$.
	
	\begin{figure}[h]
		\centering
		\includegraphics[scale=0.8]{almembk5n3.eps}
		\caption{An almost embedding $g:\K \to \R^2$ such that $w_g(123, 4)=3$.}
		\label{ris:winding}
	\end{figure}
	
	\item\label{r,e:all_odd-cyc_K4}
	For $j\in [4]$ denote by $C_j$ the oriented cycle in $K_4$ obtained by deleting $j$ from $1234$.  
	Radon theorem for the plane \cite[Lemma 2.2.3]{Sk18} implies that for any almost embedding $g:K_4\to \R^2$ the integer $$l(g):=w_g(C_4, 4)+w_g(C_3, 3)+w_g(C_2, 2)+w_g(C_1, 1)$$ is odd, see survey \cite[Theorem 5.2]{ABM+}. 
	For any integers $n_1,n_2,n_3,n_4$ whose sum is odd there is an almost embedding $f:K_4\to\R^2$ such that $w_f(C_j, j)=n_j$ for every $j=1,2,3,4$ \cite[Theorem 2]{AM25}.
		
	\item\label{r,e:all_odd-K33} Take an edge $ab$ of $K_{3, 3}$. %turgor 
	Denote by $C$ somehow oriented cycle $K_{3,3}-a-b$ of length $4$. 
	It is well-known that for any almost embedding $g:K_{3,3}-ab \rightarrow \mathbb{R}^2$ the integer $l(g):=w_g(C, a)-w_g(C, b)$ is odd, see Remark \ref{r:vaKa}.\ref{r,e:vaKa-kam_k33}.
	We conjecture that for any integer $k$ there is an almost embedding $g: K_{3, 3}- ab\to \mathbb{R}^2$ such that $l(g)=2k+1$.
	The idea was proposed by A. Lazarev, but the proof has not been published at the time of writing. 
	\end{enumerate}
\end{remark}

Denote by $K_{2,3}$ the complete bipartite graph with parts $[3]$ and $\{4, 5\}.$
For $j\in \{4, 5\}$ denote by $[3]*j$ the complete bipartite graph with parts $[3]$ and $\{j\}$.

\begin{remark}[other invariants of (almost) embeddings]\label{r:oth_inv}
	\begin{enumerate}[(a)]
	\item\label{r,e:oth_inv-cyc_wu}
	%Denote by $l_m$ for $m\in [3]$ the edge joining two of the vertices distinct from $m$.
	The  {\it cyclic Wu number} of a map $g:K_3\to \mathbb{R}^2$ is defined to be the number of revolutions in the following rotation of vector:  

	from $\overrightarrow{g(1)g(2)}$ to $\overrightarrow{g(1)g(3)}$, as the second point of the vector moves along $g|_{23}$, then

	from $\overrightarrow{g(1)g(3)}$ to $\overrightarrow{g(2)g(3)}$, as the first point of the vector moves along $g|_{12}$, then

	from $\overrightarrow{g(2)g(3)}$ to $\overrightarrow{g(2)g(1)}$, as the second point of the vector moves along $g|_{31}$, then

	from $\overrightarrow{g(2)g(1)}$ to $\overrightarrow{g(3)g(1)}$, as the first point of the vector moves along $g|_{23}$, then

	from $\overrightarrow{g(3)g(1)}$ to $\overrightarrow{g(3)g(2)}$, as the second point of the vector moves along $g|_{12}$, then

	from $\overrightarrow{g(3)g(2)}$ to $\overrightarrow{g(1)g(2)}$, as the first point of the vector moves along $g|_{31}$. 

	This equals twice the (non-integer) number of revolutions in the first three rotations above.
	The cyclic Wu number is an (almost) isotopy invariant of an (almost) embedding $g:K_{3}\to \mathbb{R}^2$. 

	\begin{figure}[h]
	\centering
	\includegraphics[scale=0.1]{off5.eps}
	\caption{An almost embedding $g:K_{3}\to \mathbb{R}^2$ whose cyclic Wu number 5}
	\label{ris:K_3}
	\end{figure}

	The cyclic Wu number is odd for any almost embedding. 
	For any integer $k$ there is an almost embedding $g:K_{3}\to \R^2$ whose cyclic Wu number equals  $2k+1$, see Figure \ref{ris:K_3}.

	For any embedding $g:K_{3}\to \R^2$ the cyclic Wu number equals $\pm 1$ (this is close to Jordan Curve Theorem).

	%The cyclic Wu number is similar to, but distinct from the \emph{degree} of a closed curve. 

	\item\label{r,e:oth_inv-tri_wu}
	The {\it triodic  Wu number} of a map $g:[3]*j\to \mathbb{R}^2$ is defined to be twice the number of revolutions in the following rotation of vector: 

	from $\overrightarrow{g(1)g(2)}$ to $\overrightarrow{g(1)g(3)}$, as the second point of the vector moves along $g|_{2j\cup j3}$, then

	from $\overrightarrow{g(1)g(3)}$ to $\overrightarrow{g(2)g(3)}$, as the first point of the vector moves along $g|_{1j\cup j2}$, then

	from $\overrightarrow{g(2)g(3)}$ to $\overrightarrow{g(2)g(1)}$, as the second point of the vector moves along $g|_{3j\cup j1}$.

	The triodic Wu number is an (almost) isotopy invariant of an (almost) embedding $g:[3]*j\to \mathbb{R}^2$. 

	\begin{figure}[h]
		\centering
		\includegraphics[scale=0.8]{triodic_gauss.eps}
		\caption{An almost embedding $g:[3]*j\to \mathbb{R}^2$ whose triodic Wu number 3}
		\label{ris:K_3,1}
	\end{figure}  

	The triodic Wu number is odd for any almost embedding. 
	For any integer $k$ there is an almost embedding $g:[3]*j\to \R^2$ whose triodic Wu number equals $2k+1$, see Figure \ref{ris:K_3,1}. 

	For any PL and apparently continuous embedding $g:[3]*j\to \R^2$ the triodic Wu number equals $\pm 1$ (for PL embedding this is proved by induction on the number of segments in $g(4i)$ for $i\in [3]$). 

	\begin{figure}[h]
	\centering
	\includegraphics[scale=1]{k23-conj.eps}
	\caption{An almost embedding $g:K_{2, 3}\to \R^2$ such that $d(g)=0$}
	\label{ris:K_2,3}
\end{figure}
		
	\item\label{r,e:oth_inv-wu_diff}  
	For an almost embedding $g:K_{2,3}\to \R^2$ denote by $d(g)$ the difference between triodic Wu number of $g|_{[3]*4}$ and the triodic Wu number of $g|_{[3]*5}$. 
	By \ref{r,e:oth_inv-tri_wu} the values of triodic Wu number are odd. 
 	So $d(g)$ is even.
	We conjecture that for any integer $k$ there is an almost embedding $g:K_{2,3}\to \R^2$ such that $d(g)=2k$, cf. Figure \ref{ris:K_2,3}.
		
	Since for any almost embedding $g:\K\to \R^2$ we have $2 l(g)=d(g|_{K_{2, 3}})$, see \cite[Statement 7.5]{ABM+r}, one might attempt to prove Theorem \ref{p:Theor} by considering only the restriction of an almost embedding $g:\K\to \R$ to the subgraph $K_{2,3}$. 
	Figure \ref{ris:K_2,3} shows that this approach does not work.
	\end{enumerate}
\end{remark}

\begin{remark}[more general context: Wu invariant of (almost) embeddings]\label{r:wu}
For any graph $K$ denote 
$$\widetilde{K}:=\cup \{\sigma\times \tau  \subset K\times K: \sigma, \tau  \text{ are non-adjacent edges of } K\}.$$ 
Then the map $\widetilde g:\widetilde{K} \to S^1$ is
well-defined by the Gauss formula $\widetilde g(x, y):=\dfrac{g(x)-g(y)}{|g(x)-g(y)|}$. 
Define the involution $t:\widetilde{K_5}\to \widetilde{K_5}$ by $t(x, y) = (y, x)$. 
The map $\widetilde g:\widetilde{K} \to S^1$ is equivariant with respect to $t$ and antipodal map of $S^1$.
The equivariant homotopy class $\alpha (g)$ of the map $\widetilde g$ is called \textit{the Haefliger-Wu invariant} of $g$ \cite[\S5]{Sk06}. 
The integers from Remark \ref{r:all_odd} are `parts' of  $\alpha(g)$. 
For definitions of `cohomological' version of this invariant, called Wu invariant, see \cite[\S 1.6]{Sk}, %\cite[\S 1.6]{Sk24},
 \cite[Theorem 4.4]{Sk06}.  

Theorem \ref{p:Theor} gives a restriction on values of the Haefliger-Wu invariant for almost embeddings $\K\to \R^2$. 
So Theorem \ref{p:Theor} is a (presumably important) step towards the interesting problem of describing the values of $\alpha (g)$ for almost embeddings $g$ of an arbitrary graph \cite[Open Problem 7]{AM25}.
\end{remark}

Some points in the plane are in \textit{general position}, if no three of them lie in a line and no three segments joining them have a common interior point.

A PL map $g:K \rightarrow \R^2$ is in \emph{general position}, if there is a  subdivision $K'$ of $K$ such that the corresponding map $g':K'\to \R^2$ is linear on any edge and the images of vertices in $K'$ in general position.

\begin{remark}[van Kampen Theorem and its integer version]\label{r:vaKa}
\begin{enumerate}[(a)]
\item\label{r,e:vaKa-def} For a general position PL map $g:K_5\to \R^2$ let the \emph{van Kampen number} $v(g)\in \mathbb{Z}_2$ be the sum mod $2$ of the numbers $|g\sigma \cap g\tau|$ over all non-ordered pairs $\{\sigma, \tau\}$ of non-adjacent edges of $K_5$.
The van Kampen Theorem for the plane states that this number is odd, see survey \cite[Lemma 1.4.3]{Sk18}.
This is equivalent to $l(g)$ being odd, see proof in \ref{r,r:vaKa-eq}.

Theorem \ref{p:Theor} is an integer analog of $l(g)$ being odd. 
However, it is known that there are no integer analogs of the van Kampen Theorem, see a rigorous formulation in \ref{r,e:vaKa-int_ana_ref}. 


\item\label{r,r:vaKa-eq} 
%The equivalence holds because 
For any PL general position almost embedding $g: \K\to \R^2$ and some general position extension $\bar{g}:K_5\to \R^2$ of $g$ we have $v(\bar{g})\underset2\equiv l(g)$ because  
$$v(\bar{g})\underset2\equiv \sum_{\{\sigma, \tau\}} |\bar{g}\sigma\cap \bar{g}\tau| \underset2\equiv|\bar{g}(45)\cap \bar{g}(12)|+|\bar{g}(45)\cap \bar{g}(23)|+|\bar{g}(45)\cap \bar{g}(31)| \stackrel{(1)}{\underset2\equiv} l(g),$$ where congruence $(1)$ follows from the equation in the second bullet of Remark \ref{r:link}.\ref{r,e:link-def_plain}. 

\item\label{r,e:vaKa-int_ana_ref} Proposition. {\it Take some orientations on edges in $K_5$.
For any cell subcomplex $C$ of $\widetilde{K_5}$ and PL general position map $g:K_5\to \R^2$ denote  $V_C(g):=\sum_{\sigma\times \tau\subset C} g(\sigma) \cdot g(\tau)$, where $\sigma, \tau$ are edges in $K_5$, and $g(\sigma) \cdot g(\tau)$ is the sum of signs of intersection points of $g(\sigma)$ and $g(\tau)$ (for definition of $\widetilde{K}$ see Remark~\ref{r:wu}).
If $V_C$ does not depend on such $g$, then $V_C(g)=0$.  }

\emph{Proof.} 
Define the involution $t:\widetilde{K_5}\to \widetilde{K_5}$ by $t(x, y) = (y, x)$. 
Since $g(\sigma) \cdot g(\tau)=-g(\tau) \cdot g(\sigma)$, we may assume that $C$ and $t(C)$ have no common $2$-cells. 
Since $V_C$ does not depend on $g$, by \cite[Lemma 3.3]{Sh57} the projection of $C$ to $\widetilde{K_5}/t$ is an integer $2$-cycle for some orientation on 2-cells of $\widetilde{K_5}/t$.
So $C=\varnothing$ by known fact \ref{r,e:vaKa-pro_surf}.  

\item\label{r,e:vaKa-pro_surf} Proposition. \emph{The 2-complex $\widetilde{K_5}/t$ has only empty integer $2$-cycle (for definition of $\widetilde{K}$ and $t$ see Remark~\ref{r:wu}).}

\emph{Proof.}
It suffices to show that $\widetilde{K_5}/t$ is a compact connected non-orientable $2$-manifold, because then $\widetilde{K_5}/t$ has only empty integer $2$-cycle, see e.g. \cite[6.2, second bullet]{Mu}, cf. \cite[3.4]{Sa91}.

Since $\widetilde{K_5}/t$ is a finite cell complex, $\widetilde{K_5}/t$ is a compact complex.

Since the link of any vertex of $\widetilde{K_5}/t$ is circular, $\widetilde{K_5}/t$ is a 2-manifold.

The connectivity of $\widetilde{K_5}/t$ is proved by checking that for each two vertices of $\widetilde{K_5}/t$ there is a cell, 
containing these vertices.

By counting cells we see that $\widetilde{K_5}$ has Euler characteristic $-10$. 
For any cell $\sigma \subset \widetilde{K_5}$ the preimage $t^{-1}(\sigma)$ consists of two cells of the same dimension.
Hence $\widetilde{K_5}/t$ has Euler characteristic $\frac{-10}{2}=-5$.
Since the Euler characteristic of a closed orientable $2$-manifold is even, see e.g. \cite[6.1, twelfth bullet]{Mu}, the 2-manifold $\widetilde{K_5}/t$ is non-orientable. 

\item\label{r,e:vaKa-kam_k33} For a general position PL map $g:K_{3,3}\to \R^2$ let the van Kampen number $v(g)\in \mathbb{Z}_2$ be the sum mod $2$ of the numbers $|g(\sigma) \cap g(\tau)|$ over all non-ordered pairs $\{\sigma, \tau\}$ of non-adjacent edges of $K_{3,3}$.
It is known that this number is odd, see survey \cite[Remark 1.4.4.a]{Sk18}.
Analogously to \ref{r,e:vaKa-def}, \ref{r,r:vaKa-eq} the number $l(g)$ and $v(g)$ are congruent modulo $2$, and there are no integer analogs of the van Kampen number.

\end{enumerate}
\end{remark}

\begin{lemma}\label{p:GreatLemma2}
	If Lemmas \ref{l:improv_exist} and \ref{l:improv_exist_last_case} are correct, then for  any % перенести PL во второй параграф
	almost  embedding $g:\K \rightarrow \R^2$ there is a PL almost embedding $f:\K \rightarrow \R^2$ such that $f|_{K_{2,3}}$ is a PL embedding and $l(f) = l(g)$.
\end{lemma}

\begin{remark}[idea of proof and a simple reformulation]\label{r:simp}
	\begin{enumerate}[(a)]
		\item\label{r,e:simp-com} We illustrate our main idea by deducing Theorem \ref{p:Theor} from Lemma \ref{p:GreatLemma2}.
		In that proof we reduce Theorem \ref{p:Theor} to the statement in \ref{r,e:simp-el}.  
		Theorem \ref{p:Theor} is equivalent to the statement \ref{r,e:simp-el'} generalizing \ref{r,e:simp-el}.
		
		\item\label{r,e:simp-el} Proposition. 
		\emph{Let $A_1, A_2, A_3$ be the vertices of a regular triangle in the plane, and $O$ its center.
			For $m\in [3]$ %аналогично
			let $l_m$ be a polygonal line  joining two of the vertices distinct from $A_m$, and disjoint with the ray $OA_m$.
			Then $w(l_1l_2l_3, O) = \pm 1$.}
		
		This proposition is equivalent to the following exercise.
		
		Let $a_1, a_2, a_3$ be pairwise distinct points on $S^1$.  
		Let $g : S^1\to S^1$ be a continuous map such that $ga_m = a_m \notin gl_m$ for $m\in [3]$. 
		Then $\deg g = 1$.
		
		\item\label{r,e:simp-el'} Proposition. 
		{\it Let $A_1, A_2, A_3$ be the vertices of a regular triangle in the plane, $O$ its center, and $O'$ some point in $\R^2\backslash \{A_1, A_2, A_3, O\}$.
		For $m\in [3]$ let 
		
		$\bullet$ $l_m$ be a polygonal line  joining two of the vertices distinct from $A_m$; 
		
		$\bullet$ $k_i$ and $k_i'$ be polygonal lines  joining $A_i$ to $O$, and $A_i$ to $O'$.
		%TG
		
		Assume that $k_i\cap k_j' = l_i\cap k_j' = l_i\cap k_j= \varnothing$ for $i\neq j$.
		Then $w(l_1l_2l_3, O) - w(l_1l_2l_3, O') = \pm 1$.
	    }
		
	\end{enumerate}
\end{remark} 

\begin{proof}[Proof of Theorem \ref{p:Theor} assuming Lemma \ref{p:GreatLemma2}]
	%We may assume that $g$ is a PL almost embedding. 
	By Lemma \ref{p:GreatLemma2} it suffices to prove Theorem \ref{p:Theor} under additional assumption that $g|_{K_{2,3}}$ is a PL embedding. 
	Recall the following known result\footnote{This result can first be proved for trees, and then for the general case can be reduced to the case of trees by taking a maximal tree.
	The details are technical (as it often happens   in plane topology). 
	This result is stated in 
		%the book 
	\cite{Wu65} with a reference to a paper by McLane-Adkisson, which could not be found in the cited collection of papers.
	Experts in topological graph theory confirm that this theorem is well known (and valid).}. 
	
	\emph{Theorem}.
	Two embeddings of a connected graph in the plane are isotopic if and only if their restrictions to any triod and to any simple cycle are isotopic. 
	
	As in Remark \ref{r:link}.\ref{r,e:link-def_sphere} identify $\R^2$ with a subset of $S^2_{PL}$. 
	From the theorem we obtain that for any PL embeddings $g_1, g_2: K_{2, 3}\to S^2_{PL}$ there is a PL homeomorphism $\psi:S^2_{PL}\to S^2_{PL}$ such that $\psi\circ g_1 = g_2$.
	Hence by Remark \ref{r:link}.\ref{r,e:link-def_sphere} it suffices to prove the analog of Theorem \ref{p:Theor} for $\R^2$ replaced by $S^2_{PL}$, for $l(g)$ replaced by $\lk(g|_{K_3}, (g(4), g(5)))$, and under the additional assumption that $g|_{K_{2, 3}}$ is a standard embedding (i.e. $g(4)=(0, 0, 1)$, $g(5)=(0, 0, -1)$ and $g(4j\cup j5)$ is a polygonal line which is the intersection of $S_{PL}^2$ and some half-plane bounded by the $z$-axis, for any $j\in [3]$). 
	%$L=\{(x, y, z)\in \R^3\ :\ ax+by=0\}$) for any $j\in [3]$. 
	This analog is equivalent to the proposition of Remark \ref{r:simp}.\ref{r,e:simp-el}. 
\end{proof}

\begin{remark}\label{r:just}
	Some theorems in PL topology of the plane have technical proofs, while attempts for simpler proofs led to mistakes. 
	An example is the completeness of the van Kampen planarity  obstruction.
	For a mistake in the proof of \cite[(2.2)]{Sa91} see \cite[footnote 10]{Sc13}; for a correct proof see e.g. \cite[proof of Theorem 3.1]{MPS}.
	This justifies the need of a careful proof of Theorem \ref{p:Theor} (or of Lemma \ref{p:GreatLemma2}), and explains why that proof is technical. 
	
\end{remark}

\section{Proof of Lemma \ref{p:GreatLemma2}}\label{s:proof}

We write $gA$ instead of $g(A)$. 

For $g:X\to Y$ and for $A\subset Y$ denote $g^{-1}A:=\{x\in X\ :\ g(x)\in A\}$.

For any points $p, q$ in edge $ij$ of the graph $\K$ denote by $[p, q]\subset ij$ the part of the edge $ij$ between $p$ and $q$.   

A PL almost  embedding $f:\K \rightarrow \R^2$ in general position is called an {\bf improvement} of PL almost  embedding $g:\K \rightarrow \R^2$, if $l(f) = l(g)$ and the number of the self-intersection points of $f$ is fewer than the number of the self-intersection points of $g$.
%A PL almost  embedding $g:\K \rightarrow \R^2$ is {\bf interesting} if there is an improvement of $g$.

In this section, except in the proof of Lemma~\ref{p:GreatLemma2}, we assume that $g:\K\to\R^2$ is a PL almost embedding in general position.
In this section all arc is a part of edge of $\K$.

The following Lemma~\ref{l:LessLemma3} is the main tool in the proof of Lemma~\ref{p:GreatLemma2}, enabling the elimination of self‑intersections.

\begin{lemma}[van Kampen trick]\label{l:LessLemma3}
    Assume that for arcs $I, \bar{I}\subset \K$ we have
        
        (0) $g|_{I}$ and $g|_{\bar{I}}$ are embeddings; 
        
        (1) $gI\cap g\bar{I}=g\partial I=g\partial \bar{I}$;
        
        (2) $I, \bar{I}\subset K_{2, 3}$;
        
        (3) $g[5]$ is contained in the closure of some connected component of $\R^2\backslash (gI\cup g\bar{I}).$ 
        
    Then there is an improvement of $g$.
 
\end{lemma}

\begin{proof}
    Denote by $U$ the complement to the closure of the connected component from condition $(3)$.
    For any arc $J$ denote by $\alpha_J$ the edge of $\K$ containing $J$.
    For any arc $J \subset g^{-1}(\Cl U)$ denote by $J^{\varepsilon}$ the intersection of some small open neighborhood of $J$ and $\alpha_J$.
    
    {\it Case 1: there is an arc $J$ in $g^{-1}(\Cl U)$ such that $g|_J$ is not embedding.}  
    Modify  $g$ on $J$ to match $f$ as shown in Figure~\ref{ris:Reid} such that $fJ \subset \Cl U$, and such that for any edge $\beta$ distinct from $\alpha_J$ we have $g\beta\cap g\alpha_J=f\beta\cap f\alpha_J$.
    Hence $f$ is almost embedding.
    The number of self-intersection points of $f$ is fewer than the number of self-intersection points of $g$.
    In the next paragraph we show that $l(f) = l(g)$. 
    
    \begin{figure}[h]
    \centering
        \includegraphics[scale=0.6]{R_0.eps}
         \caption{Transformation of a map}
         \label{ris:Reid}
\end{figure}
    
    If $J \not\subset 123$, then the restrictions of $f$ and $g$ to $123$ and $\{4, 5\}$ coincide. 
    So $l(f) = l(g)$. 
    Assume that $J \subset 123$.
    Let $\Gamma : S^1 \to \mathbb{R}^2$ be a map such that 
    
    $\bullet$ $\Gamma$ is a map onto $gJ\cup fJ$;
    
    $\bullet$ $\Gamma$-image of one semicircle is $gJ$ and the image of the other semicircle is $fJ$;
    
    $\bullet$ $\Gamma \{-1, 1\}=g\{\partial J\}$;
    
    $\bullet$ for any $s\in \{4, 5\}$ we have $w_f(123, s)=w_g(123, s)+ w(\Gamma, gs)$. 
    
    Since $\Gamma S^1\subset \Cl U$, and $g4, g5 \notin U$, and there is a curve $p: [0, 1] \to \R^2 \setminus U$ joining points $g4$ and $g5$ such that $p(\Int [0, 1] ) \cap \Cl U = \varnothing$, we have $w(\Gamma, g4)=w(\Gamma, g5)$. 
    Hence
    $$l(f) = w_f(123, 4)- w_f(123, 5)=  (w_g(123, 4)+ w(\Gamma, g4))- (w_g(123, 5)+ w(\Gamma, g5))=$$
    $$ = w_g(123, 4)- w_g(123, 5) + (w(\Gamma, g4)-w(\Gamma, g5))=w_g(123, 4)- w_g(123, 5)=l(g).$$
    
    Then $f$ is improvement of $g$.
    
    {\it Case 2: for any arc $J$ in $g^{-1}(\Cl U)$ the restriction $g|_J$ is embedding.}
    An arc $J\subset g^{-1}(\Cl U)$ is called \textbf{extendable} if there is an arc $J'\subset g^{-1}(\Cl U)$ such that $J\neq J'$ and $g(\partial J)=g(\partial J')$.
    The arc $\bar J\subset g^{-1}(\Cl U)$ denotes some of this counterpart of $J$ satisfying these conditions.
    Since $g$ is in general position and $g$ restricted to any arc in $ g^{-1}(\Cl U)$ is embedding, we have that for any extendable arc $J$ there is a unique $\bar{J}$.
    For an extendable arc $J$ denote by $U_J$ the open component in $\R^2\backslash g(J\cup \bar{J})$ such that $U_J\subset U$.
    An extendable arc $J$ is called \textbf{minimal} if for every extendable arc $J'\notin \{J, \bar J\}$ we have $U_{J'}\not\subset U_J$.
    
    Let $J$ and $J'\notin \{J, \bar{J}\}$ be extendable arcs. 
    Since $\partial U_J=g(J\cup \bar{J})\neq g(J'\cup \bar{J'})=\partial U_{J'}$, we have $U_J\neq U_{J'}$. 
    Hence if $J$ and $J'\notin \{J, \bar{J}\}$ are extendable arcs such that $U_{J'} \subset U_J$, then $U_{J'} \varsubsetneq U_J$.
    
    
    Let us show that there is a minimal arc $J$.
    Denote $J_0:=I$.
    If $J_0$ is minimal, then we are done.
    Otherwise there is an extendable arc $J_1\notin \{J_0, \bar{J_0}\}$ such that $U_{J_1}\varsubsetneq U_{J_0}$.
    If $J_1$ is minimal, then we are done.
    Otherwise there is an extendable arc $J_2\notin \{J_1, \bar{J_1}\}$ such that $U_{J_2}\varsubsetneq U_{J_1}$.
    Proceeding in this way, we produce a sequence $$U_{J_0}\varsupsetneq U_{J_1}\varsupsetneq U_{J_2}\varsupsetneq \ldots$$ 
    Since the set of extendable arcs is finite, we have that the sequence must terminate.
    Hence there is a minimal arc $J$. 
     
    %Denote $\gamma_J:=J\cup \bar{J}$.
    Take  $J^{\varepsilon}, \bar{J}^{\varepsilon}$ sufficiently small such that for every edge $\beta$ distinct from $\alpha_{J^{\varepsilon}}$ and $\alpha_{\bar{J}^{\varepsilon}}$ we have 
    \begin{equation}
    g\beta\cap g(J^{\varepsilon}\cup \bar{J}^{\varepsilon})=g\beta\cap g(J\cup \bar{J}).
    \tag{*}
    \end{equation}
    Take sufficiently small open neighbourhood $U_J^{\varepsilon}$ of $U_J$ such that $J^{\varepsilon}, \bar{J}^{\varepsilon}\subset g^{-1}\Cl U_J^{\varepsilon}$, and $g4, g5\notin U_J^{\varepsilon}$, and there is a curve $p: [0, 1] \to \R^2 \setminus U_J^{\varepsilon}$ joining points $g4$ and $g5$ such that $p(\Int [0, 1] ) \cap \Cl U_J^{\varepsilon} = \varnothing$.
    Modify  $g$ on ${J^{\varepsilon}\cup \bar{J}^{\varepsilon}}$ to match $f$ as shown in Figure~\ref{ris:Lemm3.1} such that $f(J^{\varepsilon}\cup \bar{J}^{\varepsilon})\subset \Cl U_J^{\varepsilon}$, and such that for any edge $\beta$ distinct from $\alpha_J$ and $\alpha_{\bar{J}}$ we have 
    \begin{equation}
    g\beta\cap f(J^{\varepsilon}\cup \bar{J}^{\varepsilon})=g\beta\cap g(J^{\varepsilon}\cup \bar{J}^{\varepsilon}).
    \tag{**}
    \end{equation}
    \begin{figure}[h]
    \centering
\includegraphics[scale=0.8]{change_map.eps}
	 \caption{The upper figure is used if $\partial J \cap [5]\neq\varnothing$ or $\partial \bar{J} \cap [5]\neq\varnothing$, and the lower figure is used otherwise}
    \label{ris:Lemm3.1}
\end{figure}

    \textit{Proof that $l(f) = l(g)$.}
    If $J$ and $\bar{J}$ are not in $123$, then the restrictions of $f$ and $g$ coincide on $123$ and $\{4, 5\}$.
    So $l(f) = l(g)$. 
    
    The proof that $l(f) = l(g)$ under the additional assumption that $J$ is subset of $123$ is obtained from the proof of $l(f) = l(g)$ under the additional assumption that $J\subset 123$ in the case 1 by replacing $J$ with $\Cl J^{\varepsilon}$ and $U$ with $U_J^{\varepsilon}$.
    
    Analogously $l(f) = l(g)$ under the additional assumption that $\bar{J}$ is subset of $123$.
    
    Assume that $J, \bar{J}\subset 123$ (cf. the penultimate paragraph in case 1).
    Let $\Gamma, \bar{\Gamma} : S^1 \to \mathbb{R}^2$ be maps such that 
    
    $\bullet$ $\Gamma$ is a map onto $g\Cl J^{\varepsilon}\cup f\Cl J^{\varepsilon}$ and $\bar{\Gamma}$ is a map onto $g\Cl \bar{J}^{\varepsilon}\cup f\Cl \bar{J}^{\varepsilon}$;
    
    $\bullet$ $\Gamma$-image of one semicircle is $g\Cl J^{\varepsilon}$ and the $\Gamma$-image of the other semicircle is $f\Cl J^{\varepsilon}$, and $\bar{\Gamma}$-image of one semicircle is $g\Cl \bar{J}^{\varepsilon}$ and the $\bar{\Gamma}$-image of the other semicircle is $f\Cl \bar{J}^{\varepsilon}$;
    
    $\bullet$ $\Gamma \{-1, 1\}=g\{\partial J^{\varepsilon}\}$ and $\bar{\Gamma} \{-1, 1\}=g\{\partial \bar{J}^{\varepsilon}\}$;
    
    $\bullet$ for any $s\in \{4, 5\}$ we have $w_f(123, s)=w_g(123, s)+ w(\Gamma, gs)+ w(\bar{\Gamma}, gs)$.
    
    From the choose of $U_J^{\varepsilon}$ and from $\Gamma S^1, \bar{\Gamma}S^1\subset \Cl U_J^{\varepsilon}$, we have $w(\Gamma, g4)=w(\Gamma, g5)$ and $w(\bar{\Gamma}, g4)=w(\bar{\Gamma}, g5)$. 
    Hence
    $$l(f) = w_f(123, 4)- w_f(123, 5)=  (w_g(123, 4)+ w(\Gamma, g4)+w(\bar{\Gamma}, g4))- (w_g(123, 5)+ w(\Gamma, g5)+w(\bar{\Gamma}, g5))=$$
    $$ = w_g(123, 4)- w_g(123, 5) + (w(\Gamma, g4)-w(\Gamma, g5) + w(\bar{\Gamma}, g4) - w(\bar{\Gamma}, g5))=w_g(123, 4)- w_g(123, 5)=l(g).$$
    
    \textit{Proof that $f$ is almost embedding.}
    The restrictions of $f$ and $g$ to the complement of $J^{\varepsilon}\cup \bar{J}^{\varepsilon}$ in $\K$ coincide. 
    Hence for any non-adjacent edges $\beta, \beta'$ distinct from $\alpha_{J}$ and $\alpha_{\bar{J}}$ we have $$f\beta \cap f\beta' = g\beta \cap g\beta' = \varnothing.$$ 
    
    Consider any edge $\beta$ of $\K$ non-adjacent to  $\alpha_{J}$. 
    Since $g$ is an almost embedding, we have $g\beta \cap g\alpha_{J}=\varnothing$. 
    Since $g$ restricted to any arc in $g^{-1}(\Cl U)$ is embedding and $g\beta \cap g\alpha_{J}=\varnothing$, we have that $g\beta\cap \Cl U_J$ consists of images of extendable arcs $J'$ such that $g(\partial J')\subset g\bar{J}$. 
    Hence $U_{J'}\subset U_{\bar{J}}=U_J$.
    Since $J$ is a minimal arc, we have $g\beta \cap g(J\cup \bar{J})\subset g\beta\cap\Cl U_J=\varnothing$.
    We have $$f\beta\cap f\alpha_{J}=(f\beta\cap f(\alpha_{J}\backslash J^{\varepsilon}))\cup(f\beta\cap fJ^{\varepsilon})\stackrel{(1)}{\subset} (g\beta\cap g(\alpha_{J}\backslash J^{\varepsilon}))\cup(g\beta\cap g(J\cup \bar{J}))\stackrel{(2)}{=}\varnothing,\quad \text{where}$$
    
    $\bullet$ inclusion $(1)$ holds because the restrictions of $f$ and $g$ to the complement of $J^{\varepsilon}\cup \bar{J}^{\varepsilon}$ in $\K$ coincide and because $f\beta\cap fJ^{\varepsilon}=g\beta\cap fJ^{\varepsilon}\subset g\beta\cap f(J^{\varepsilon}\cup \bar{J}^{\varepsilon})\overset{(^{**})}{=}g\beta\cap g(J^{\varepsilon}\cup \bar{J}^{\varepsilon})\overset{(^*)}{=}g\beta \cap g(J\cup \bar{J})$, and
    
    $\bullet$ equation $(2)$ holds because $g$ is an almost embedding and because $g\beta\cap g(J\cup \bar{J}) =\varnothing$.
    
    Analogously $f\beta\cap f\alpha_{\bar{J}}=\varnothing$   for any edge $\beta$ non-adjacent to $\alpha_{\bar{J}}$. 
    
    Then $f$ is an almost embedding. 
    
    Then $f$ is improvement of $g$.
\end{proof}

\begin{figure}[h]
	\centering
	\includegraphics[scale=0.7]{U_exist.eps}
	\caption{}
	\label{ris:LastConst}
\end{figure}
    
    We say that triple $(i, j, k)$ of points from $[5]$ is \textbf{interesting} if 
    
    $\bullet$ $ij, ik$ are edges of $K_{2, 3}$,
    
    $\bullet$ $g|_{ij}$ and $g|_{ik}$ are embeddings, and
    
    $\bullet$ $g(ij)\cap g(ik)\varsupsetneq\{gi\}$.
    
    For an interesting triple $(i, j, k)$ denote by $p=p_{i, j, k, g}\in ij$ the first point after $i$ of the path $g|_{ij}$ (starting from $i$) contained in $g(ik)$. 
    Denote by $q=q_{i, j, k, g}\in ik$ the point such that $gp=gq$.
    Points $p, q$ are called \textbf{improving for the triple} $(i, j, k)$.
    Open component $U$ in $\R^2\backslash (g[i,p]\cup g[i, q])$ is called \textbf{improving for the triple} $(i, j, k)$ if  $g([5]\backslash \{i, j, k\})\subset~U$. 
	
    Triple $(i, j, k)$ is interesting iff $(i, k, j)$ is interesting. 
    The orders of points $j$ and $k$ is important for the definition of improving points and improving component.
	
	\begin{lemma}[fig. \ref{ris:LastConst}]\label{l:improv}
		For any interesting triple $(i, j, k)$ there is a unique open component $U$ improving for the triple $(i, j, k)$, and $gi\in\partial U$.
	\end{lemma}

	\begin{proof}	
	Denote by $s$ and $t$ points in $[5]\backslash \{i, j, k\}$.
	We have $$g(st)\cap (g[i,p]\cup g[i, q])\subset g(st)\cap (g(ij)\cup g(ik))=(g(st)\cap g(ij))\cup(g(st)\cap g(ik)).$$ 
	Since $g$ is PL almost embedding, we have $(g(st)\cap g(ij))\cup(g(st)\cap g(ik))=\varnothing$. 
	It follows that there is a unique component $U$ that contains $gs$ and $gt$. 	
	
	Since $g|_{ij}$ and $g|_{ik}$ are embeddings, and $g[i,p]\cap g[i, q]=\{gi, gp\}$, we have $\R^2\backslash (g[i, p]\cup g[i, q])$ has two connected components.
	Hence $g[i, p]\cup~g[i, q]=\partial U$.
	Then $gi\in g[i, p]\cup~g[i, q]=\partial U$. 
\end{proof}   

\begin{lemma}\label{l:improv_exist}
	Assume that restriction of $g$ to any edge of $K_{2,3}$ is an embedding.
	Assume that open component $U$ is improving for the triple $(i, j, k)$.
	If either $gj, gk\in U$, or $gk\notin U$, then there is an improvement of $g$.
\end{lemma}    

\begin{proof}[Sketch of proof.]
	Denote by $p\in ij$ and $q\in ik$ the improving points for the triple $(i, j, k)$.
	
	{\it Suppose that $gj, gk \in U$.}
	In this paragraph we show that arcs $I=[i, p]$ and $\bar{I}=[i, q]$ satisfy the properties from Lemma~\ref{l:LessLemma3}.
	Since the restriction of $g$ to any edge of $K_{2,3}$ is an embedding, we have that $g|_{I}$ and $g|_{\bar{I}}$ are embeddings.
	Hence arcs $I$ and $\bar{I}$ satisfy the property $(0)$ of Lemma \ref{l:LessLemma3}. 
	Since $p$ is the first point after $i$ of the path $g|_{ij}$ (starting from $i$) contained in $g(ik)$, we have $g[i, p]\cap g[i, q]=\{gi, gp\}$. 
	Hence arcs $[i, p]$ and $[i, q]$ satisfy the property $(1)$ of Lemma \ref{l:LessLemma3}. 
	Since $ij, ik\subset K_{2,3}$, we have that $[i, p]$ and $[i, q]$ satisfy the property $(2)$ of Lemma \ref{l:LessLemma3}.
	Since $gj, gk \in U$, $gi\in \partial U$ and $U$ is improving for the triple $(i, j, k)$, we have $g[5] \subset \Cl U$. 
	Then arcs $I:=[i, p], \bar{I}:=[i, q]$ satisfy the property $(3)$ of Lemma \ref{l:LessLemma3}. 
	
	So we are done by Lemma \ref{l:LessLemma3}.
	
	{\it Suppose that $gk\notin U$.}
	Denote by $s$ and $t$ points from $[5]\backslash\{i, j, k\}$ such that edge $ks$ is in $K_{2,3}$.
	Since $gs\in U$ and $gk\notin U$, we have $\varnothing \neq g(ks)\cap\partial U= g(ks) \cap (g[i, p] \cup g[i, q]) = g(ks)\cap g[i, q]\not\ni gk$, because $k\notin [i, q]$. 
	Hence $g(ks)\cap g(ik)\varsupsetneq \{gk\}$.
	Since $ks, ki$  are edges of $K_{2, 3}$, $g|_{ks}$ and $g|_{ki}$ are embeddings, and $g(ks)\cap g(ik)\varsupsetneq \{gk\}$, we have $(k, s, i)$ is interesting triple.
	Denote by $p'\in ks$ and $q'\in ki$ the improving points for the triple $(k, s, i)$.
	From Lemma~\ref{l:improv} for the triple $(k, s, i)$, we have that there is a unique improving open component $U'$ for the triple $(k, s, i)$.
	In the next two paragraphs we show that $U\cap \partial U'=\varnothing$.
	
	Since  $U\cap \partial U'=U\cap (g[k, p']\cup g[k, q'])$, it suffices to show that $U\cap g[k, p']=\varnothing$ and $U\cap g[k, q']=\varnothing$.
	Let us show that $U\cap g[k, p']=\varnothing$. 
	Since $g|_{k s}$ has no self-intersection and $gk\notin U$, it suffices to show that $g[k, p']\cap \partial U\subset \{gp'\}$.
	We have $$g[k, p']\cap \partial U=g[k, p']\cap (g[i, p]\cup g[i, q])\stackrel{(1)}{=}$$$$\stackrel{(1)}{=}g[k, p']\cap g[i, q]\stackrel{(2)}{\subset} \{gp'\},\quad\text{where,} $$
	
	$\bullet$ equation (1) holds from the definition of an almost embedding, and
	
	$\bullet$ inclusion (2) holds because $g[k, p']\cap g(ki)=\{gk, gp'\}$ and because from the definition of $q$ we have $k\notin [i, q]$.
	
	Let us show that $U\cap g[k, q']=\varnothing$. 
	From the definition of $q$ we have $g(ki)\cap \partial U=g(ki)\cap (g[i, p]\cup g[i, q])=g[i, q]$.
	Since $g|_{ki}$ has no self-intersection, $gk\notin U$ and $g(ki)\cap \partial U = g[i, q]$, we have $g(ki)\cap U=\varnothing$.
	Hence $g[k, q']\cap U\subset g(ki)\cap U=\varnothing$.
	
	Since $U\cap \partial U'=\varnothing$, we have either $U\subset U'$, or $U \cap U' = \varnothing$.
	From the definition of $U$ and $U'$, we have $gt\in U$ and $gt\in U'$.
	Hence $U\subset U'$.
	From Lemma~\ref{l:improv} for the triples $(i, j, k)$ and $(k, s, i)$, we have $gi\in\Cl U$ and $gk\in\Cl U'$.
	Hence $g\{i, s, t\}\subset\Cl U$ and $g\{k, j, t\}\subset\Cl U'$.
	Since $g\{i, s, t\}\subset\Cl U$ and $g\{k, j, t\}\subset\Cl U'$, and $U\subset U'$, we have $g[5]\subset\Cl U\cup\Cl U'=\Cl U'$. 
	Then arcs $I:=[k, p']$ and $\bar{I}:=[k, q']$ satisfy the property (3) of Lemma \ref{l:LessLemma3}.
	
	In this paragraph we show that arcs $I:=[k, p']$ and $\bar{I}:=[k, q']$ satisfy the properties $(0)-(2)$ from Lemma~\ref{l:LessLemma3}.
	Since the restriction of $g$ to any edge of $K_{2,3}$ is an embedding, we have that $g|_{I}$ and $g|_{\bar{I}}$ are embeddings.
	Hence arcs $I$ and $\bar{I}$ satisfy the property $(0)$ of Lemma \ref{l:LessLemma3}. 
	Since $p'$ is the first point after $i$ of the path $g|_{ks}$ (starting from $k$) contained in $g(ki)$, we have $g[k, p']\cap g[k, q']=\{gk, gp'\}$. 
	Hence arcs $[k, p']$ and $[k, q']$ satisfy the property $(1)$ of Lemma \ref{l:LessLemma3}. 
	Since $ks, ki\subset K_{2,3}$, we have that $[k, p']$ and $[k, q']$ satisfy the property $(2)$ of Lemma \ref{l:LessLemma3}.
	
	So we are done by Lemma \ref{l:LessLemma3}.
\end{proof}

\begin{lemma}\label{l:improv_exist_last_case}
	Assume that $(i, j, k)$ and $(k, i, l)$ are interesting triples and $j\neq l$.
	Assume that $U, U'$ are the open components improving for the triples $(i, j, k)$ and $(k, i, l)$ respectively.
	If $gj\notin U$, $gk\in U$, $gi\notin U'$, and $gl\in U'$, then there is an improvment of $g$.
\end{lemma}   

\begin{proof}[Sketch of proof.]
	Since $gj\notin U$ and $gi\notin U'$, we have $gj\notin \Cl U$ and $gi\notin \Cl U'$.
	Since $gj\notin \Cl U$ and $gj\in \Cl U'$, we have $\Cl U'\not\subset \Cl U$.
	Since $gi\in \partial U$ and $gi\in \R^2\backslash\Cl U'$, we have $\R^2\backslash\Cl U'\not\subset \Cl U$.
	Hence we have $\partial U'\not\subset \Cl U$.
	
	Denote by $p\in ij$ and $q\in ik$ the improving points for the triple $(i, j, k)$.
	Denote by $p'\in ki$ and $q'\in kl$ the improving points for the triple $(k, i, l)$.
	In this paragraph we show that there is a point $a\in [k, q']$ such that $ga\notin \Cl U$.
	We have
	$$g(ik)\cap \Cl U=(g[i, q]\cup g[q, k])\cap \Cl U=(g[i, q]\cap\Cl U)\cup (g[q, k]\cap\Cl U)=g(ik),$$
	where the last equation holds because $g[i, q]\subset \partial U$, and because $gk\in U$ and $g[q, k]\cap \partial U=gq$.
	Hence $g[k, p']\cap \Cl U=g[k, p']$.
	Since $\partial U'\not\subset \Cl U$ and since $g[k, p']\cap \Cl U=g[k, p']$, we have that there is a point $a\in [k, q']$ such that $ga\notin \Cl U$.
	
	Since $gk, gl\in \Cl U$, $ga\notin \Cl U$ and $\{k, l, a\}\subset kl$, then there are the first points $a_1\in[a, k]$ and $a_2\in [a, l]$ of the passes $g|_{[a, k]}$ and $g|_{[a, l]}$ respectively contained in $\partial U$.
	Since $g(kl)\cap g(ij)=\varnothing$, we have $ga_1, ga_2\in g(ik)$.
	Denote by $b_1, b_2\in ik$ points such that $ga_1=gb_1$ and $ga_2=gb_2$.
	
	In this paragraph we show that arcs $I=[a_1, a_2]$, $\bar{I}=[b_1, b_2]$ satisfy the properties of Lemma \ref{l:LessLemma3}.
	Since $g|_{ki}$ and $g|_{kl}$ are embeddings, we have that arcs $I$, $\bar{I}$ satisfy the property $(0)$ of Lemma \ref{l:LessLemma3}. 
	From the definition of $a_1, a_2, b_1, b_2$, we have that arcs $I$, $\bar{I}$ satisfy the property $(1)$ of Lemma \ref{l:LessLemma3}.
	Since $I\subset kl$, $\bar{I}\subset ki$, we have that $I, \bar{I}$ satisfy the property $(2)$ of Lemma \ref{l:LessLemma3}.
	Let us show that arcs $I$, $\bar{I}$ satisfy the property $(3)$ of Lemma \ref{l:LessLemma3}.
	Since $gI\cap U=\varnothing$ and $g\bar{I}\subset \partial U$, we have $(gI\cup g\bar{I})\cap U=\varnothing$.
	Then there is an open component $W$ in $\R^2\backslash (gI\cup g\bar{I})$, such that $\Cl W\cap \Cl U = \Cl U$.
	Then $gk, gb, gl, gi \in \Cl W$, where $b$ is a point from $[5]\backslash\{i, j, k, l\}$. 
	Since $g(jb)\cap \partial W\subset g(jb)\cap (g(ki)\cup g(kl))=\varnothing$ and $gb\in \Cl W$, we have $gj\in \Cl W$.
	Then arcs $I$, $\bar{I}$ satisfy the property $(3)$ of Lemma \ref{l:LessLemma3}.
	
	Since arc $I, \bar{I}$ satisfy the properties of Lemma \ref{l:LessLemma3}, we have that there is an improvement of $g$.
\end{proof} 
 
\textbf{Proof of Lemma \ref{p:GreatLemma2} using  Lemmas \ref{l:improv_exist}, \ref{l:improv_exist_last_case}.} 

It suffices to prove Lemma \ref{p:GreatLemma2} under the additional assumptions that 

$\bullet$ $g$ is a PL map in general position, and

%Figure \ref{ris:Reid} shows that for any PL almost embedding $g:\K \rightarrow \R^2$ there is an PL almost embedding $f:\K \rightarrow \R^2$ such that the restriction of $f$ to any edge of $K_{2,3}$ is an embedding and  $l(f) = l(g)$. 
%Then it suffices to prove Lemma \ref{p:GreatLemma2} under further additional assumption that restriction of $g$ to any edge of $K_{2,3}$ is embedding.

$\bullet$  $g|_{K_{2,3}}$ is not a PL embedding.

It suffices to prove that if $g|_{K_{2,3}}$ is not a PL embedding, then there is an improvement of $g$.  

\textit{Case 0:} {\it there is an edge $\alpha$ in $K_{2, 3}$ such that $g|_{\alpha}$ is not an embedding.}
Figure \ref{ris:Reid} shows how to improve $g$.
Since $\alpha$ is not in $123$, we have $l(f)=l(g)$. 

\textit{Case 1:} {\it the restriction of $g$ to any edge of $K_{2,3}$ is an embedding and there are no self-intersections of $g|_{ij\cup ik}$ for any $i \in \{4, 5\}$, $j\neq k\in [3]$.}
Since $g|_{K_{2, 3}}$ is not a PL embedding and restriction of $g$ to any edge of $K_{2,3}$ is embedding, there is a self-intersection of $g|_{i4\cup i5}$ for some $i \in [3]$.
Without loss of generality, assume that $g(14)\cap g(15)\varsupsetneq \{g1\}$.
Then  $(1, 5, 4)$ is interesting triple.
Denote by $p\in 15$ and $q\in 14$ the interesting points for the triple $(1, 5, 4)$.
By Lemma~\ref{l:improv} for the triple $(1, 5, 4)$ there is a unique open component $U$ improving for the triple $(1, 5, 4)$.
If $\{g4, g5\}\not\subset U$, then for some $i\in\{4, 5\}$ we have $g(i2)\cap \partial U \subset g(i2)\cap g(i1)\not\ni gi$.
This contradicts the assumption that there are no self-intersections of $g|_{ij\cup ik}$ for any $i \in \{4, 5\}$, and $j\neq k\in [3]$.
Hence $g[5]\subset \Cl U$.
It follows that intervals $I:=[1, p], \bar{I}:=[1, q]$ satisfy the property $(3)$ of Lemma \ref{l:LessLemma3}.
Since $g|_{K_{2, 3}}$ is embedding we have that arcs $I$, $\bar{I}$ satisfy the property $(0)$ of Lemma \ref{l:LessLemma3}.
From the definition of $p$ and $q$, we have that arcs $I$, $\bar{I}$ satisfy the property $(1)$ of Lemma \ref{l:LessLemma3}.
Since $I\subset 15$, $\bar{I}\subset 14$, we have that arcs $I, \bar{I}$ satisfy the property $(2)$ of Lemma \ref{l:LessLemma3}.
So we are done by Lemma \ref{l:LessLemma3}.

\textit{Case 2:} {\it the restriction of $g$ to any edge of $K_{2,3}$ is an embedding and there is a self-intersection of $g|_{ij\cup ik}$ for some $i \in \{4, 5\}$, $j\neq k\in [3]$.}
Without loss of generality, assume that $g(51) \cap g(52)\varsupsetneq \{g5\}$.
Then  $(5, 1, 2)$ and $(5, 2, 1)$ are interesting triples.
By Lemma~\ref{l:improv} for the triple $(5, 1, 2)$ there is a unique open component $U$ improving for the triple $(5, 1, 2)$.
By Lemma~\ref{l:improv} for the triple $(5, 2, 1)$ there is a unique open component $U'$ improving for the triple $(5, 2, 1)$.
If $g1, g2\in U$, or $g2\notin U$, then the existence of an improvement of $g$ follows from Lemma~\ref{l:improv_exist} for the triple $(5, 1, 2)$.
Hence it suffices to prove Lemma~\ref{p:GreatLemma2} under the additional assumption that $g2\in U$ and $g1\notin U$.
If $g2, g1\in U'$, or $g1\notin U'$, then the existence of an improvement of $g$ follows from Lemma~\ref{l:improv_exist} for the triple $(5, 2, 1)$.
Hence it suffices to prove Lemma~\ref{p:GreatLemma2} under the additional assumption that $g1\in U'$ and $g2\notin U'$.

Since $g2\notin U'$ and $g4\in U'$, we have $\varnothing\neq g(24)\cap \partial U'= g(24)\cap g[5, q]$.
Then $g(24)\cap g(25)\varsupsetneq \{g2\}$.
Hence $(2, 5, 4)$ is interesting triple. 
By Lemma~\ref{l:improv} for the triple $(2, 5, 4)$ there is a unique open component $W$ improving for the triple $(2, 5, 4)$.
If $g5, g4\in W$, or $g4\notin W$, then the existence of an improvement of $g$ follows from Lemma~\ref{l:improv_exist} for the triple $(2, 5, 4)$.
Then it suffices to prove Lemma~\ref{p:GreatLemma2} under the additional assumption that $g4\in W$ and $g5\notin W$. 

Since $g2\in U$, $g1\notin U$, $g4\in W$ and $g5\notin W$, we have that Lemma~\ref{p:GreatLemma2} follows from Lemma~\ref{l:improv_exist_last_case} for the triples $(5, 1, 2)$ and $(2, 5, 4)$. $\Box$

\begin{thebibliography}{3}
	
		\bibitem[ABM+]{ABM+}
		{\it Alkin E., Bordacheva E., Miroshnikov A., Nikitenko O., Skopenkov A.}Invariants of almost embeddings of graphs in the plane: results and problems, \href{https://arxiv.org/abs/2408.06392}{arXiv:2408.06392v1}.
	
		\bibitem[ABM+r]{ABM+r}
		{\it Alkin E., Bordacheva E., Miroshnikov A., Nikitenko O., Skopenkov A.}Invariants of almost embeddings of graphs in the plane,  \href{https://arxiv.org/abs/2410.09860}{arXiv:2410.09860v1}.
		
		\bibitem[AM25]{AM25} 
		\emph{E. Alkin, A. Miroshnikov,} On winding numbers of almost embeddings of $K_4$ in the plane, \href{https://arxiv.org/abs/2501.15642}{arXiv:2501.15642v1}.
	
		\bibitem[CKV]{CKV} \emph{M. Čadek, M. Krčál and L. Vokřínek.} Algorithmic solvability of the lifting-extension problem, Discr. Comp. Geom. 57 (2017), 915–965. \href{https://arxiv.org/abs/1307.6444}{arXiv:1307.6444v4}.
		
		\bibitem[FK19]{FK19} \emph{R. Fulek, J. Kyn{\v{c}}l.}
		$\mathbb{Z}_2$-genus of graphs and minimum rank of partial symmetric matrices,
		35th Intern. Symp. on Comp. Geom. (SoCG 2019), Article No. 39; pp. 39:1--39:16, \linebreak
		\href{https://drops.dagstuhl.de/opus/volltexte/2019/10443/pdf/LIPIcs-SoCG-2019-39.pdf}{https://drops.dagstuhl.de/opus/volltexte/2019/10443/pdf/LIPIcs-SoCG-2019-39.pdf}.
		We refer to numbering in arXiv version: \href{https://arxiv.org/abs/1903.08637}{arXiv:1903.08637v1}.
		
		\bibitem[FKT]{FKT} \emph{M. H. Freedman, V. S. Krushkal and P. Teihner.} Van Kampen's embedding
		obstruction is in complete for $2$-complexes in $\R^4$, Math. Res. Letters. 1994. 1. P. 167-176, \href{https://people.mpim-bonn.mpg.de/teichner/Math/ewExternalFiles/VanKampen-Journal.pdf}{https://people.mpim-bonn.mpg.de/teichner/Math/ewExternalFiles/VanKampen-Journal.pdf}
		
		\bibitem[FWZ]{FWZ} \emph{M. Filakovsk´y, U. Wagner, S. Zhechev.} Embeddability of simplicial complexes is
		undecidable. Proceedings of the 2020 ACM-SIAM Symposium on Discrete Algorithms, \href{https://epubs.siam.org/doi/epdf/10.1137/1.9781611975994.47}{https://epubs.siam.org/doi/epdf/10.1137/1.9781611975994.47}    
		
		%\bibitem[Ma97]{Ma97} \emph{Yu. Makarychev.} A short pro of of Kuratowski's graph planarity criterion, J. of Graph Theory, 25 (1997), 129-131.
		
		\bibitem[KS20]{KS20}
		{\it R.Karasev, A. Skopenkov.} Some `converses' to intrinsic linking theorems. Discrete Comput. Geom. 70(2023), no.3, 921--930. \href{https://arxiv.org/abs/2008.02523}{arXiv:2008.02523v2}.
		  
		\bibitem[Ky16]{Ky16} \emph{J. Kyn{\v{c}}l.} Simple realizability of complete abstract topological graphs simplified, Discrete Comput. Geom. 64 (2020) 1--27.\href{https://arxiv.org/abs/1608.05867}{arXiv:1608.05867v2}.
		
		
		%https://doi.org/10.1007/s00454-020-00204-0
		
		\bibitem[MPS]{MPS}
		{\it Michael J. Pelsmajer, Marcus Schaefer, and Daniel \v Stefankovi\v c}. Removing even crossings. J. Combin. Theory Ser. B, 97(4):489–500, 2007, \href{https://www.sciencedirect.com/science/article/pii/S0095895606001018}{https://www.sciencedirect.com/science/article/pii/S0095895606001018}      
		
		\bibitem[Mu]{Mu} \emph{Daniel Müllner.} 2-manifolds, \href{http://www.map.mpim-bonn.mpg.de/2-manifolds}{http://www.map.mpim-bonn.mpg.de/2-manifolds}.
		
		\bibitem[Ni22]{Ni22}
		{\it Ryo Nikkuni.} Converses to generalized Conway-Gordon type congruences, Tokyo J. Math. 47(2): 353-364 (December 2024), \href{https://arxiv.org/abs/2211.00408}{arXiv:2211.00408v3}.

		\bibitem[RS72]{RS72} \emph{C. P. Rourke and B. J. Sanderson.} Introduction to Piecewise-Linear Topology, Springer, Ergebnisse der Mathematik und ihrer Grenzgebiete, Vol. 69, 1972.

		\bibitem[Sa91]{Sa91} \emph{K. S. Sarkaria.} A one-dimensional Whitney trik and Kuratowski's graph planarity criterion, Israel J. Math. 73 (1991), 79- 89. 
		 \href{http://kssarkaria.org/docs/One-dimensional.pdf}{http://kssarkaria.org/docs/One-dimensional.pdf}.
		
		\bibitem[Sc13]{Sc13} \emph{M. Schaefer.} Hanani-Tutte and related results. In Geometry-intuitive, discrete, and convex, Bolyai Soc. Math. Stud., 24 (2013), 259-299. \href{http://ovid.cs.depaul.edu/documents/htsurvey.pdf}{http://ovid.cs.depaul.edu/documents/htsurvey.pdf}.

		\bibitem[Sh57]{Sh57} \emph{A. Shapiro}, Obstructions to the embedding of a complex in a Euclidean space, I, The first obstruction, Ann. of Math. (2) 66 (1957), 256–269.
		
		{\bibitem[Sk]{Sk} \emph{A. Skopenkov.} Algebraic Topology From Algorithmic Standpoint, draft of a book, mostly in Russian,
			%\linebreak
			\href{http://www.mccme.ru/circles/oim/algor.pdf}{http://www.mccme.ru/circles/oim/algor.pdf}.}

		\bibitem[Sk06]{Sk06}
		{\it A. Skopenkov.} Embedding and knotting of manifolds in Euclidean spaces. \href{https://arxiv.org/abs/math/0604045}{arXiv:0604045v1} 
		
		\bibitem[Sk16]{Sk16} \emph{A. Skopenkov.} A user's guide to the topological Tverberg Conjecture,
		\href{https://arxiv.org/abs/math/1605.05141}{arXiv:1605.05141v5}. Abridged earlier published version: Russian Math. Surveys, 73:2 (2018), 323-353.
		
		\bibitem[Sk18]{Sk18} \emph{A. Skopenkov.} Invariants of graph drawings in the plane. Arnold Math. J., 6 (2020)
		21-55; full version: \href{https://arxiv.org/abs/1805.10237}{arXiv:1805.10237v3}.
		
		\bibitem[Sk18r]{Sk18r}\emph{A. Skopenkov.} Invariants of graph drawings in the plane, 
		Math. education, 31 (2023), 74-127. 
		
		\bibitem[Sk21]{Sk21}\emph{A. Skopenkov.} Embeddings of $k$-complexes in $2k$-manifolds and minimum rank of partial symmetric matrices. \href{https://arxiv.org/abs/2112.06636}{arXiv:2112.06636v4}. 
		%\bibitem[Sk24]{Sk24} \emph{A. Skopenkov.} Realizability of hypergraphs and intrinsic link theory, Mat. Prosveschenie, 32. (2024), 125–159. arXiv:1402.0658.pdf.

		\bibitem[SS13]{SS13} \emph{M. Schaefer and D. \v Stefankovi\v c.} Block additivity of $\mathbb{Z}_2$-embeddings. In Graph drawing, volume 8242 of Lecture Notes in Comput. Sci., 185--195.
		Springer, Cham, 2013. \href{http://ovid.cs.depaul.edu/documents/genus.pdf}{http://ovid.cs.depaul.edu/documents/genus.pdf}.

		\bibitem[ST17]{ST17}
		{\it A. Skopenkov, M.Tancer.} Hardness of almost embedding simplicial complexes in $\R^d$, Discr. and Comp. Geom.61:2 (2019), 452-463, \href{https://arxiv.org/abs/1703.06305}{arXiv:1703.06305v2}.

		\bibitem[Wu65]{Wu65} \emph{W.~T.~Wu}. A Theory of Embedding, Immersion and Isotopy of Polytopes in an Euclidean Space, Science Press, Peking, 1965

        %\bibitem[Sch11]{Sch11}
        %{\it M. Schaefer.} Hanani-Tutte and related results. In Geometry -- intuitive, discrete,
        %and convex, Bolyai Soc. Math. Stud., 24 (2013), 259-299. https://ovid.cs.depaul.edu/documents/htsurvey.pdf.
    	
\end{thebibliography}

\end{document} 

