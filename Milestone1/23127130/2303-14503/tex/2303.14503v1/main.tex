\documentclass[12pt]{article}
\textwidth=17.5cm\textheight=25.5cm\hoffset=-2cm\voffset=-3.5cm
%\usepackage[cp1251]{inputenc}
%\usepackage[russian]{babel}
\usepackage[english]{babel}
\usepackage{amsfonts,amsmath,epsf,amssymb,amsthm,graphicx,afterpage}%,url}
\usepackage{epsfig}
\graphicspath{{morefig/}{figures/}{figures00/}{figures02/}{figures05/}{figlms/}}
\newcommand{\?}{\nobreak\hskip.167em\nobreak\hskip0pt}
\DeclareGraphicsExtensions{.pdf,.png,.jpg}
\def\K{K_5\backslash (4, 5)}
\def\Cycle{\Sigma^1}
\def\ZerSphere{\Sigma^0}
\newcommand{\lk}{\operatorname{lk}}
\def\fSigmC{\displaystyle f|_{\Cycle}}
\def\fSigmP{\gamma^0}

\def\gSigmC{\displaystyle g|_{\Cycle}}
\def\gSigmP{\gamma^0}

\let\fiverm\tiny
\let\Bbb=\mathbb
\long\def\comment#1\endcomment{}

\newtheoremstyle{mydefinition}% name
{3pt}%      Space above
{3pt}%      Space below
{\normalfont}%         Body font
{\parindent}%         Indent amount (empty = no indent, \parindent = para indent)
{\bfseries}% Thm head font
{.}%        Punctuation after thm head
{ }%     Space after thm head: " " = normal interword space;
%       \newline = linebreak
{}%         Thm head spec (can be left empty, meaning `normal')

%\theoremstyle{mydefinition}
%\theoremstyle{definition}
\theoremstyle{plain}
\newtheorem{theorem}{Theorem}
\newtheorem{pr}[theorem]{Problem}
\newtheorem{example}[theorem]{Example}
\newtheorem{remark}[theorem]{Remark}
\newtheorem{lemma}[theorem]{Lemma}
\newtheorem{???sentence}[theorem]{Предложение}
\newtheorem{Assertion}[theorem]{Assertion}
\newtheorem{approvalo}[theorem]{Утверждение}


\title{On drawing $K_5$ minus an edge in the plane}
\author{Garaev T. R. }

%\date{August 2021}

\begin{document}

\maketitle

\section{Introduction and main result}

%Text dedicated to almost embedding of graph $K_5$ minus an edge in the plain. The main result is formulated in the theorem \ref{p:Theor}.

A continuous map $g:K\rightarrow \mathbb{R}^2$ of a graph $K$  is called an \textbf{almost embedding}  if $g(\alpha)\cap g(\beta) = \varnothing$ for any two disjoint edges $\alpha, \beta \subseteq K$.

Denote $[n]:=\{1, 2, \ldots, n\}$.

Denote by $K_5$ the complete graph with the vertices set $[5]$. Denote by $(l, k)$ the edge between vertices $l$ and $k$ in a graph. Denote by $\K$ the graph obtained from the graph $K_5$ by deleting the edge $(4, 5)$.


Denote by $\Cycle$ the complete graph $K_3$ with the vertices set $[3]$ ordered by cyclic permutation $(1\ 2\ 3)$. 

Let us define the \textbf{linking number} $\lk{(\gamma^1, \gamma^0)}\in \mathbb Z$  of disjoint oriented closed polygonal line $\gamma^1$ in $\mathbb{R}^2$ and ordered pair  $\gamma^0$ of points in $\mathbb{R}^2 \backslash \gamma^1$. Take an oriented polygonal line $M$  in general position to!!!with $\gamma^1$ such that $\partial M$ with order inherited from $L$ coincides with $\gamma^0$. By $\lk{(\gamma^1, \gamma^0)}$ denote the sum of the signs of the intersection points of $M$ and $\gamma^1$. It is well known that the linking number is well defined.


\begin{theorem}\label{p:Theor}
    For any almost embedding $g:\K \rightarrow \mathbb{R}^2$ the absolute value of $\lk(g|_{\Cycle}, (g(4), g(5)))$ does not exceed $3$.
\end{theorem}


It is well-known that for any  almost embedding $g:\K \rightarrow \mathbb{R}^2$ the absolute value of linking coefficient $\lk(g|_{\Cycle}, (g(4), g(5)))$ is odd. 
There are similar results for embeddings of graphs into $\mathbb R^3$, see survey \cite{Sk}.
Theorem \ref{p:Theor} gives a constraint on the absolute value of the linking coefficient.  Theorem \ref{p:Theor} disproves conjecture $1.6 (a)$ in the first arXiv version of \cite{SkKa}. We conjecture that for any almost embedding $g:\K \rightarrow \mathbb{R}^2$ the absolute value does not exceed $1$.

Some theorems in topology of the plane have technical proofs, e.g. Jordan curve theorem and completeness of the van Kampen planarity  obstruction in \cite{Sh}. Theorem \ref{p:Theor} is no exception. 

\section{Proof of the theorem \ref{p:Theor} using lemma \ref{p:GreatLemma2}}

Denote $S^2:=\{(x,y,z) : \max \{|x|, |y|, |z|\}=1\}.$

Similarly to the text before theorem \ref{p:Theor} define linking number for disjoint oriented polygonal line $\gamma^1$ on $S^2$ and ordered pair $\gamma^0$ of points in $S^2\backslash \gamma^1$.  

A polygonal line $L$ is called \textbf{simple} if $L$ has no self-intersections. 

Denote by $K_{2,3}$ complete bipartite graph with parts $[3]$ and $\{4, 5\}.$ 

In the following text we sometimes  write $fA$ instead of $f(A).$


\begin{lemma}\label{p:GreatLemma2}
    For  any PL almost  embedding $g:\K \rightarrow S^2$ there is a PL almost embedding $f:\K \rightarrow S^2$ such that
    
    $(1)$ the restriction of $f$ to $K_{2,3}$ is a PL embedding;
    
    $(2)$ $\lk(\displaystyle f|_{\Cycle}, (f4, f5)) = \lk(\displaystyle g|_{\Cycle}, (g4, g5))$.
    %\end{enumerate}
    
\end{lemma}


\textit{Proof of theorem \ref{p:Theor} using lemma \ref{p:GreatLemma2}.} %By lemma \ref{p:GreatLemma2} there is an almost embedding $f: \K \to \mathbb R^2$ such  that properties (1) and (3) from lemma \ref{p:GreatLemma2} hold and%  \begin{enumerate}
       %(2') $f((4, i)\cup (i, 5))\cap f((4, j)\cup (j, 5))=\{f4, f5\}$ for any $1\leq i\neq j\leq 3.$
    %\end{enumerate}
It is sufficient prove the analog of theorem \ref{p:Theor} with $\mathbb R^2$ replace by $S^2$.
It is sufficient prove theorem \ref{p:Theor} under additional assumption that $g$ is in general position PL map. Denote by $f$ the map from lemma \ref{p:GreatLemma2}. There is a simple polygonal line $L\subset \mathbb{R}^2$ joining $f4$ and $f5$, in general position with $\displaystyle f|_{\Cycle}$, and such that $L\cap f((4,j)\cup(j,5))=\{f4, f5\}$ for any $1\leq j \leq 3$. 

We may assume that points of $\fSigmC^{-1}L$ split $\Cycle$ into  consecutive arcs $B_1, \ldots, B_n$. Denote by $L_i$  the polygonal line in $L$ joining the ends of $f(B_i)$. Denote by $\gamma_i^1$ the closed polygonal line $f(B_{i})\cup L_{i}$ with the orientation inherited from $\Sigma^1$ (fig. 1).

\begin{figure}[h]
\begin{center}
         %\includegraphics[scale=0.9]{SplitGamma.png}
         \includegraphics[width=110mm]{SplitGamma.png}
         \label{ris:Gamma_i}
	 \caption{}
 \end{center}
 \end{figure}

Denote $\gamma^0:=(f4, f5).$

Below we prove that 

$$(a)\qquad\lk(\fSigmC, \gamma^0)= \lk(\gamma_1^1, \gamma^0)+ \ldots + \lk(\gamma_n^1, \gamma^0);$$

$$(b)\qquad\lk(\gamma_i^1, \fSigmP)\in \{-1, 0, 1\};$$

$$(c)\qquad\text{if}\quad f[3]\cap \gamma_i^1 = \emptyset,\quad\text{then}\quad\lk(\gamma_i^1, \fSigmP)= 0.$$



Since $\gamma_i^1\cap f[3]$ are pairwise disjoint for different $i$, and $|f[3]|=3$, by (b) and (c) there are at most 3 subscripts $i$ such that $\lk(\gamma_i^1, \gamma^0)\neq 0$. Then from (a) it follows that $|\lk(\fSigmC, \gamma^0)|\leq 3$.


\smallskip
\textit{Proof of (a).}
Take a polygonal line $L'$ joining $f4$ and $f5$ such  that 
$L'\cap L=\{f4, f5\}$ and $L'$ is in general position with $\displaystyle f|_{\Cycle}$. 
We have $$f\Sigma^1\cap L'=(fB_1\cup \ldots \cup fB_n)\cap L'=((fB_1\cup L_1)\cup \ldots \cup (fB_n\cup L_n))\cap L'=(\gamma_1^1\cup \ldots \cup \gamma_n^1)\cap L'.$$ The third equation follows from $L'\cap L_i=\emptyset.$
Since the orientation of $\gamma_i^1$ is inherited from $\Sigma^1$, the signs of the same intersection points of $f\Sigma^1\cap L'$ and $\gamma_i^1\cap L'$ coincide. Then equation (a) follows from the definition of linking number.


\smallskip
\textit{Proof of (b).}
Take a polygonal line $L'$ joining $f4$ and $f5$ such  that: 

$\bullet$ $L'\cap L=\{f4, f5\}$; 

$\bullet$ $L'$ is in general position with $\displaystyle f|_{\Cycle}$; 

$\bullet$ $i$-th vertex of $L'$ are close to $i$-th vertex of $L$, and $L'$ is on "one side" of $L$ (fig. 2). 

\begin{figure}[h]
\begin{center}
         \includegraphics[scale=0.8]{ExampleL_.png}
	 \label{ris:LChange}
	 \caption{}
 \end{center}
 \end{figure}

 Since $L\cap fB_i=f\partial B_i$, we have $|L'\cap fB_i|\leq 2$. If $|L'\cap fB_i|=2,$ then $\lk(fB_{i}\cup L_{i}, \gamma^0)=0.$ If $|L'\cap fB_i|\leq 1,$ then $|\lk(fB_{i}\cup L_{i}, \gamma^0)|\leq 1.$
 
 \smallskip


\textit{Proof of (c).}
%Let us prove  in the next paragraph that if 
Since $f[3]\cap \gamma_i^1 = \emptyset$, without loss of generality $B_i\subset (1,2)$. In the definition of $\lk (\gamma_i^1, \gamma^0)$ take $M=f((4, 3)\cup (3, 5))$. Since $L\cap M =\emptyset$ and $f$ is almost embedding, we have $f(1, 2) \cap M =\emptyset$ and $\gamma_i^1\cap M$. It follows that $\lk(\gamma_i^1, \fSigmP)= 0$
 $\Box$

\section{Proof of lemma \ref{p:GreatLemma2}}



%The PL almost embeddings $g, f: \K \rightarrow S^2$  are called \textit{similar} if images of any edge from $\K$ after applying $g$ and $f$ have not self-intersection and . 

For any points $p, q$ in the edge $(i, j)$ of the graph $\K$ denote by $[p, q]\subset (i, j)$ part of the edge $(i, j)$ between $p$ and $q$.   

\begin{lemma}\label{p:LessLemma3}
    Suppose that $g:\K \rightarrow S^2$ is a PL almost  embedding in general position and points $p_1, q_1, p_2, q_2\in \K$ are such that
        
        (0) $g|_{[p_1, q_1]}$ and $g|_{[p_2, q_2]}$ have no self-intersections; 
        
        (1) $g[p_1, q_1]\cap g[p_2, q_2]=\{gp_1=gp_2, gq_1=gq_2\}$;
        
        (2) either $[p_1, q_1] \subset(i, 4)$ and $[p_2, q_2] \subset(i, 5)$ for some $i\in [3]$, or $[p_1, q_1] \subset(k, l)$ and $[p_2, q_2] \subset(k, m)$ for some $k\in\{4, 5\}$, $l\neq m\in [3]$;
        
        (3) $g[5]$ is contained in the closure of some connected component of $S^2\backslash (g[p_1, q_1]\cup g[p_2, q_2]).$ 
        
    Then there is a PL almost embedding $f:\K \rightarrow S^2$ such that $$\lk(\displaystyle f|_{\Cycle}, (f4, f5)) = \lk(\displaystyle g|_{\Cycle}, (g4, g5))$$ 
    and the number of the self-intersection points of $f$ is less than the number of the self-intersection points of $g$.
    
\end{lemma}

%\begin{figure}[h]
%    \begin{center}
        
%        \includegraphics[scale=0.8]{DeletOfSelfInt.png}
        
%        \label{ris:SelfInt}
%	 \caption{}
%    \end{center}
%\end{figure}
\smallskip
\textit{Proof.}    Denote $\gamma :=[p_1, q_1]\cup [p_2, q_2]$. Denote by $U$ the complement to the closure of the connected component from condition $(3)$. 

A interval $J \subset g^{-1}(\mbox{Cl}U)$ 
%K_5-(4,5)-\gamma$ 
is called {\it $U$-interval} if $gJ$ joins either two points  in $g[p_1, q_1]$ or two points in $g[p_2, q_2]$. 
%and $gJ\subset \mbox{Cl}\  U$.
    
    
    \begin{figure}[h]
    \begin{center}
        
        \includegraphics[scale=0.8]{lemma3.1.png}
        
        \label{ris:Lemm3.1}
	 \caption{Upper figure is used if $p_1, q_1 \in [5]$ or $p_2, q_2 \in [5]$, and lower is used otherwise}
    \end{center}
\end{figure}
    
    For any $p, q \in \gamma$ denote by $[p, q]_{\varepsilon}$ intersection of some small neighborhood of $[p, q]$ and the edge containing $[p, q]$.
    
    {\it Suppose that there are no $U$-intervals.} Change $g$ on the ${[p_1, q_1]_{\varepsilon}\cup [p_2, q_2]_{\varepsilon}}$ as in figure \ref{ris:Lemm3.1}. Denote the resulting map by $f$.  
    %Since there are no significant pairs, it follows that $g\Cycle$ does not intersect edge that is nonadjacent to both edges either $(i, 4)$ and $(i, 5)$, or $(k, l)$ and $(k, m)$ from condition $(2)$. 
    For any interval $J$ in some edge $\alpha$ of $\K$ denote $\alpha_{J}:=\alpha$. 
    
    Consider any edge $\beta$ of $\K$ 
    % that without loss of generate 
    nonadjacent to  $\alpha_{[p_1, q_1]}$. Since $g$ is almost embedding, $g\beta \cap g\alpha_{[p_1, q_1]}=\emptyset$. 
    Hence the intersection $\beta\cap g^{-1}(\mbox{Cl}U)$ consists of $U$-intervals. 
    Since there are no $U$-intervals, it follows that $g\beta \cap g\gamma=\emptyset$.
    Analogously $g\beta \cap g\gamma=\emptyset$   for any edge $\beta$ nonadjacent to $\alpha_{[p_2, q_2]}$. 
%    It follows that $g\gamma$ do not intersect images of any edge nonadjacent to one of the edges $\alpha_{[p_1, q_1]}$ and $\alpha_{[p_2, q_2]}$. 
    Hence $f|_{[p_1, q_1]_{\varepsilon}\cup [p_2, q_2]_{\varepsilon}}$ does not intersect any edge nonadjacent to one of the edges $\alpha_{[p_1, q_1]}$ and $\alpha_{[p_2, q_2]}$.
    
    The restrictions of $f$ and $g$ to the complement to $[p_1, q_1]_{\varepsilon}\cup[p_2, q_2]_{\varepsilon}$ coincide. 
    Hence for any nonadjacent edges $\beta', \beta$ distinct from $\alpha_{[p_1, q_1]}$ and $\alpha_{[p_2, q_2]}$ we have $$f\alpha \cap f\beta = g\alpha \cap g\beta = \emptyset.$$ Hence $f$ is an almost embedding. 
    
    Since the restrictions of $f$ and $g$ to $\Cycle \cup \{4, 5\}$ coincide, it follows that $$\lk(\displaystyle f|_{\Cycle}, (f4, f5)) = \lk(\displaystyle g|_{\Cycle}, (g4, g5)).$$ Then $f$  satisfies the conclusion of lemma \ref{p:LessLemma3}.
    
    \begin{figure}[h]
    \begin{center}
        
        \includegraphics[scale=0.6]{lemma3.2.png}
        
        \label{ris:Lemm3.2}
	 \caption{}
    \end{center}
\end{figure}

    
    {\it Suppose that there is a  $U$-interval.}
    For any $U$-interval $[p, q]$ denote by $\bar p, \bar q\in \gamma$ the points such that $gp=g\bar p$ and $gq=g\bar q$. %and Define the partial order on the set of significant pairs such that $\{p, q\}$ less than $\{p', q'\}$ if ???$$[\gamma \cap g^{-1}gp, \gamma \cap g^{-1}gq] \subset [\gamma \cap g^{-1}gp', \gamma \cap g^{-1}gq'],$$ where $[\gamma \cap g^{-1}gp, \gamma \cap g^{-1}gq]$ is the segment in $\gamma$ connecting $\{gp, gq\}$
    Take a  $U$-interval $[p, q]$ such that for any other  $U$-interval $[p', q']$ 
    the interval $[\bar p, \bar q]$ does not contain $[\bar{p'}, \bar{q'}]_{\varepsilon}$. 
    
    Denote by $\Gamma:=[p, q] \cup [\bar p, \bar q]$. Denote by $V$ the connected component of $S^2\backslash g\Gamma$ such that $V\subset U$. Similarly define {\it $V$-interval}. 
    For every $V$-interval $[p', q']$ we have $[\bar p', \bar q']\subset [p, q]$. 
    By the choice of $[p, q]$ there are no $V$-intervals. 
    Change $g$ on the ${[p, q]_{\varepsilon}\cup [\bar p, \bar q]_{\varepsilon}}$ as in figure 4. 
    Proof that $f$ is an almost embedding is obtained from such a proof for the case where there are no $U$-intervals by  replacing $U$ by $V$, $[p_1, q_1]$ by $[p, q]$, $[p_2, q_2]$ by $[\bar p, \bar q]$, and $\gamma$ by $\Gamma$.
    
    %Denote the resulting map by $f$.  Since $[\bar p, \bar q]$ does not contain $[\bar p', \bar q']$ for any other  $U$-interval, it follows that $f|_{[p, q]_{\varepsilon}}$ intersects only with edge, that intersect $f|_{[\bar p, \bar q]_{\varepsilon}}$. Hence $f[p, q]_{\varepsilon}\cup f[\bar p, \bar q]_{\varepsilon}$ does not intersect any edges that is nonadjacent to both edges $\alpha_{[p, q]}$ and $\alpha_{[\bar p, \bar q]}$. Since restrictions of $f$ and $g$ to the complement to $[p, q]_{\varepsilon}\cup[\bar p, \bar q]_{\varepsilon}$ coincide, it follows that for any nonadjacent edges $\alpha, \beta$ in the complement to $[p, q]_{\varepsilon}\cup[\bar p, \bar q]_{\varepsilon}$ we have $$f\alpha \cap f\beta = g\alpha \cap g\beta = \emptyset.$$ It follows that $f$ is an almost embedding.
    
    If $p, q \notin \Cycle$, then the restrictions of $f$ and $g$ to $\Cycle$ and $\{4, 5\}$ coincide. 
    Otherwise $$\lk(\displaystyle f|_{\Cycle}, (f4, f5)) = \lk(\displaystyle g|_{\Cycle}, (g4, g5)) + \lk(\Gamma, (g4, g5)).$$ 
    Since $\Gamma \subset \mbox{Cl}\ U$ and $g\{4, 5\} \not \subset U$ it follows that $\lk(\Gamma, (g4, g5))=0$. 
    Hence in both cases 
    $$\lk(\displaystyle f|_{\Cycle}, (f4, f5)) = \lk(\displaystyle g|_{\Cycle}, (g4, g5)).$$   
    
    It follows that $f$  satisfies the conclusions  of lemma \ref{p:LessLemma3}.
    $\Box$
    
\medskip    
\textit{Proof of lemma \ref{p:GreatLemma2}.} 
It is sufficient prove lemma \ref{p:GreatLemma2} under additional assumption that $g$ is in general position.

Figure 5 shows that for any PL almost embedding $g':\K \rightarrow S^2$ there is a PL almost embedding $f':\K \rightarrow S^2$ such that for any edge $e$ of $K_{2,3}$ the restriction of $f'$ to $e$ is embedding and  $$\lk(\displaystyle f'|_{\Cycle}, (f'4, f'5)) = \lk(\displaystyle g'|_{\Cycle}, (g'4, g'5)).$$ 

\begin{figure}[h]
    \begin{center}
        
        \includegraphics[scale=1.3]{R_0.png}
        
        \label{ris:Reid}
	 \caption{}
    \end{center}
\end{figure}

Then it is sufficient prove lemma \ref{p:GreatLemma2} under additional assumption that restriction of $g$ to any edge of $K_{2,3}$ is embedding.


If the restriction of $f$ to $K_{2,3}$ is a PL embedding, then take $f=g$. 
In the  opposite case it is sufficient to show that there is map $f$ with fewer number of self-intersection points and satisfying property $(2)$ of lemma \ref{p:GreatLemma2}. 

In the following text we consider the case when $g(i, j)\cap g(i, k)\neq\emptyset$ 
for some $i \in \{4, 5\}$ and $j\neq k\in [3]$. 
The case when $g(i, 4)\cap g(i, 5)\neq \emptyset$ for some $i \in [3]$, is considered similarly, 
replacing $(1, 2, 3, 4, 5)$ by $(4, 5, 3, 2, 1)$.

Without loss of generality assume $g(5, 1)\cap g(5,2)\neq \{g5\}$. Denote by $q_1\in (5, 1)$ and by $q_2\in (5, 2)$ the points such that points $p_1, q_1$ and $p_2, q_2$ where $p_1=p_2=5$, satisfy the property $(1)$ of lemma \ref{p:LessLemma3} (fig. 6). Points $p_1, q_1$ and $p_2, q_2$ satisfy the property $(2)$ of lemma \ref{p:LessLemma3} for $k=5, l=1, m=2$.   

\begin{figure}[h]
    \begin{center}
        
        \includegraphics[scale=0.6]{g5-g1,g2.png}
        
        \label{ris:LastConst}
	 \caption{}
    \end{center}
\end{figure}

Then $g[5, q_1]\cup g[5, q_2]$ divide $S^2\backslash g[5, q_1]\cup g(5, q_2)$ into two connected components. 

If  $g[4]$ is contained in one of the connected components, then points $p_1=5, q_1$ and $p_2=5, q_2$ satisfy the property $(3)$ of lemma \ref{p:LessLemma3}. So by lemma \ref{p:LessLemma3} there is map $f$ with fewer number of self-intersection points and satisfying property $(2)$ of lemma \ref{p:GreatLemma2}. 

Now suppose that $g[4]$ is not contained in the one connected component. Since $g$ is PL almost embedding, we have $g(3, 4)\cap (g[5,q_1]\cup g[5, q_2])=\emptyset$. It follows that there is a connected component $U$ which contains $g3$ and $g4$. Since $g[4]$ is not contained in $U$, without loss of generality assume that $g2 \notin U.$

Since $g(4, 2)\cap g(5, 1)=\emptyset$, $g4 \in U$ and $g2 \notin U$ it follows that $g(4, 2)\cap g(5, 2)\neq \{g2\}$. Denote by $p_1' \in g(4, 2)$ the first intersection point between $(4, 2)$ and $(5, 2)$. Denote $p_2':=(g^{-1}gp_1')\cap (5, 2)$. Since $\{g2\}\subset g[p_1', 2]\cap g[5, 2]\neq \emptyset$ it follows that there is the first intersection point $q_1'\in [p_1, 2]$ between $[p_1', 2]$ and $(5, 2)$. Denote $q_2':=(g^{-1}gq_1')\cap (5, 2)$. Points $p_1', p_2', q_1', q_2'$ satisfy the property $(1)$ from the construction and property $(2)$ of lemma \ref{p:LessLemma3} for $i=2$ (fig. 7).


\begin{figure}[h]
    \begin{center}
        
        \includegraphics[width=100mm]{g4-g2.png}
        
        \label{ris:MainCont}
	 \caption{}
    \end{center}
\end{figure}

Then $g[p_1', q_1']\cup g[p_2', q_2']$ divides $S^2\backslash g[p_1', q_1']\cup g[p_2', q_2']$ into two connected component. From Jordan curve theorem it follows that:
    if  $U,V\subset S^2$ are open polygons and $U\cap \partial  V=\emptyset$, then either $U\cap V = \emptyset$ or $U\cap Int(S^2\backslash V) = \emptyset$.

%\textit{Proof of lemma \ref{p:SmallLemma2}.} For any points $p, q\in U$ take a curve $\gamma_{p, q}: [0, 1] \rightarrow U$ joining $p$ and $q$. Since $U\cap \partial  V=\emptyset$ it follows that either $\gamma_{p, q}^{-1}(V\cap \gamma_{p, q}[0, 1])=\emptyset$ or $\gamma_{p, q}^{-1}(Int(S^2\backslash V)\cap \gamma_{p, q}[0, 1])=\emptyset$. Since $U$ is connected it follows the statement of the lemma.  $\Box$

It follows that $g3, g4, g5\notin V$. Since $g$ is PL almost embedding, we have $g(1, 3)\cap (g[p_1', q_1']\cup g[p_2', q_2'])=\emptyset$. It follows that $g1\notin V$. 

If  $g2\notin V$, then points $p_1', p_2', q_1', q_2'$ satisfy the property $(3)$ of lemma \ref{p:LessLemma3}. So by lemma \ref{p:LessLemma3} there is map $f$ with fewer number of self-intersection points and satisfying property $(2)$ of lemma \ref{p:GreatLemma2}.  

\begin{figure}[h]
    \begin{center}
        
        \includegraphics[scale=0.6]{gq.png}
        
        \label{ris:gq}
	 \caption{}
    \end{center}
\end{figure}

Now assume that $g2 \in V$. Since $g$ in general position it follows that there is point $q\in [q_2', 2]$ such that $gq\in (S^2\backslash V)\cap g[q_2', 2]$ and $q\neq q_2'$ (fig. 8). Since $g2 \in V$ and $gq \notin V$ it follows that  $g[q, 2]\cap (g[p_1', q_1']\cup g[p_2', q_2'])\neq \emptyset$.
Then $g(5, 2)$ has self-intersection point or $g(5, 2)\cap g[p_2', q_2'] \neq \{gp_2', gq_2'\}$. This contradiction proves the lemma \ref{p:GreatLemma2}.

 \begin{thebibliography}{3}
        
         \bibitem{Sk}
        {\it  Skopenkov A.} Realizability of hypergraphs and Ramsey link theory // arXiv:1402.0658.
        
        \bibitem{SkKa}
        {\it Karasev R., Skopenkov A.} Some ‘converses’ to intrinsic linking theorems // arXiv:2008.02523.
        
        \bibitem{Sh}
        {\it Michael J. Pelsmajer, Marcus Schaefer, and Daniel ˇStefankoviˇc}. Removing even crossings. J. Combin. Theory Ser. B, 97(4):489–500, 2007.
            
 \end{thebibliography}

\end{document} 




%It's not hard to see that after transformation as in the picture \ref{ris:4i i5 1} and \ref{ris:4i i5 2} winding number of $\gSigmC$ around $g(4)$ and $g(5)$ don't change.



%And $\lk(f(B_{j})\cup L_{j})\in \{-1, 0, 1\}$, where the orientation of the polygonal line $f(B_{j})\cup L_{j}$ is inherited from the polygonal line $f(S^1)$.

%We have that $[3]$ contained either into one arc $B_i$ or into two arcs $B_i$ and $B_j$ or into three arcs $B_i, B_j$ and $B_k$.


%Suppose that $[3]$ contained into one arc $B_i$. Since $$\lk(\fSigmC, \fSigmP)= \lk(\gamma_1^1, \fSigmP)+ \ldots + \lk(\gamma_n^1, \fSigmP)$$ and $\lk(\gamma_i^1, \fSigmP)\in \{-1, 0, 1\}$, we see that exist such $k$ that $B_k$ is subset of some edge and  $$\lk(f(B_{k})\cup L_k, f(\ZerSphere))\neq 0.$$ Without loss of generality, we can assume that $B_{k}\subset (1,2)$. Then $$(f(B_{k})\cup L_k)\cap f((4, 3)\cup (3, 5))\neq \emptyset.$$ Since $f((4, 3)\cup (3, 5)) \cap L =\emptyset$, we have $f((1, 2)) \cap f((4, 3)\cup (3, 5)) \neq \emptyset$,  that contradicts the condition of almost embedding. 

%Suppose that $[3]$ contained into two arcs $B_i$ and $B_j$. %Without loss of generality, we can assume that $[2]\subset B_i$, $f^{-1}(A_i)\in (3, 1)$, $f^{-1}(A_{i+1})\in (2, 3)$,  $f^{-1}(A_j)\in (3, 1)$, $f^{-1}(A_{j+1})\in (2, 3)$. 
%Take an open linearly connected neighborhood $V\subset S^2$ such that $|\partial  V \cap f(B_i)|=|\partial  V \cap f(B_j)|=2$. Take a point $P\in V$. Let connect  $P$ with $\partial  V \cap f(B_i)$ and $\partial  V \cap f(B_j)$. Let consider $f(1), f(2), f(3), f(4), f(5), P$ and polygonal lines, that connect them. This is how almost embedding of $K_{3, 3}$ is constructed. This is a contradiction.

%\begin{center}
        
%        \includegraphics[scale=0.8]{K_33.png}
	 
% \end{center}

%Suppose that $[3]$ contained into three arcs $B_i, B_j$ and $B_k$. %Without loss of generality, we can assume that $[2]\subset B_i$, $f^{-1}(A_i)\in (3, 1)$, $f^{-1}(A_{i+1})\in (2, 3)$,  $f^{-1}(A_j)\in (3, 1)$, $f^{-1}(A_{j+1})\in (2, 3)$. 
%Take an open linearly connected neighborhood $V\subset S^2$ such that $|\partial  V \cap f(B_i)|=|\partial  V \cap f(B_j)|=|\partial  V \cap f(B_k)|=2$. Take a point $P\in V$. Let connect  $P$ with $\partial  V \cap f(B_i),$  $\partial  V \cap f(B_j)$ and $\partial  V \cap f(B_k)$. Let consider $(1), f(2), f(3), f(4), f(5), P$ and polygonal lines, that connect them. This is how almost embedding of $K_{3, 3}$ is constructed. This is a contradiction. $\Box$

%Since $|lk(f(\Cycle), f(\ZerSphere))|>3$, then there is arc $B_{j, j+1}$, such that $lk(f(B_{j,j+1})\cup L_{A_{j+1},A_j}) \neq 0$ and $B_{j, j+1}\cap \{1,2,3\} = \emptyset$. 
%Without loss of generality, we can assume that $B_{j, j+1}\subset (1,2)$. Then $$(f(B_{j, j+1})\cup L_{A_{j+1},A_j})\cap f((4,3)\cup (3,5))\neq \emptyset.$$ Since $f((4,3)\cup (3,5)) \cap L =\emptyset$, therefore $f(\sigma_{1,2}) \cap L \neq \emptyset$,  that contradicts the condition of almost embedding. 

 \textit{Achievement of property $(2b)$ using lemma \ref{p:LessLemma3} and keeping property 1 and property $(2a)$.}

Suppose that property $(2b)$ is not fulfilled. It is sufficient to show that there is map $f$ with few the number of self-intersection points and with keeping properties $(1)$, $(2a)$ and $(3)$ of lemma \ref{p:GreatLemma2}. Without loss of generality assume $g(1, 4)\cap g(1, 5)\neq \emptyset$. Denote by $q_1\in (1, 4)$ and by $q_2\in (1, 5)$ the points such that points $p_1=1, q_1$ and $p_2=1, q_2$ satisfy the properties $(1)$ and $(2)$ of lemma \ref{p:LessLemma3}.

Then $g(1, q_1)\cup g(1, q_2)$ divide $S^2\backslash g(1, q_1)\cup g(1, q_2)$ into two connected components. 

Suppose that $g([5]\backslash 1)$ is contained in the same connected component. Then points $p_1=1, q_1$ and $p_2=1, q_2$ satisfy the property $(3)$ of lemma \ref{p:LessLemma3}. It follows that we can reduce the number of self-intersection points.