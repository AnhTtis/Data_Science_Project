%%%use the first documentclass for prl-like version, second one for drafts
%\documentclass[twocolumn,showpacs,floatfix,aps]{revtex4}
%%\documentclass[preprint,showpacs,preprintnumbers]{revtex4}
%\usepackage{amsmath}
%\usepackage{amsfonts}
%\usepackage{amssymb}
%\usepackage{graphicx}
%\usepackage[usenames]{color}
%\usepackage{footnote}
%\newcommand{\half}{\tfrac{1}{2}}
%\usepackage{bbm}

\documentclass[onecolumn,floatfix,prd,aps,12pt]{revtex4-2}
%\documentclass[rmp,aps,preprint,showpacs,floatfix]{revtex4}
\usepackage[applemac]{inputenc}
\usepackage[T1]{fontenc}
\pdfoutput=1
\usepackage{enumerate}
\usepackage{braket}
\usepackage{amsmath}
%\usepackage[english,frenchb]{babel}   % To remove for English text only
\usepackage{subfigure}
\usepackage{lipsum}
%\usepackage{widetext}
\usepackage{amsfonts}
\usepackage{amssymb, mathrsfs}
\usepackage{bm}
\usepackage[pdftex]{graphicx}
\usepackage{footnote}
\usepackage{hyperref}  % Ajoute les hyperlien
\hypersetup{colorlinks=true,allcolors=blue}
\usepackage{hypcap}   % Corrige la position du lien pour les images
\usepackage{bookmark}
%\usepackage{pdfpages}  -- Package not compatible with revTex 4-2
%\usepackage{caption}
\usepackage{dsfont,mathrsfs,color,url,verbatim,booktabs}

\def\beq{\begin{equation}}
\def\eeq{\end{equation}}
\def\bsp{\begin{split}}
\def\esp{\end{split}}
\def\bea{\begin{eqnarray}}
\def\eea{\end{eqnarray}}
\def\ba{\begin{array}}
\def\ea{\end{array}}
\def\nn{\nonumber \\}
\def\vsp#1{\vspace{#1}}
\def\ha{\hat  h (\hat  X, \hat  P)}
\def\ev{\exp {-i{\ha\over\hbar}}}
\def\undertext#1{$\underline{\hbox{#1}}$}
\def\haf#1{{{#1}\over 2}}
\def\eps{\epsilon}
\def\om{\omega}
\def\rw{\rightarrow}
\def\Rw{\Rightarrow}
\def\lb{\left(}
\def\rb{\right)}
\def\lr{\left|}
\def\rr{\right|}
\def\l.{\left.}
\def\r.{\right.}
\def\iy{\infty}
\def\ra{\rangle}
\def\la{\langle}
\def\pr{\prime}
\def\pa{\partial}
\def\inty{\int_{-\infty}^\infty}


\def\hsp#1{\hspace{#1}}
%math symbols
\def\part{\partial}
\def\tfrac#1#2{{\textstyle{#1\over #2}}}
\def\half{\tfrac{1}{2}}
\def\x{\times}
\def\ox{\otimes}
\def\dag{\dagger}
\def\dagg#1{#1^{\dag}}
\def\bra#1{{\cal{h}} #1 \mid}
\def\ket#1{\mid #1 {\cal{i}}}
\def\ketbra#1#2{\ket{#1} \bra{#2}}
\def\abs#1{\mid #1 \mid}
\def\norm#1{\| #1 \|}
\def\Id{\ensuremath{\mathbbm{1}}}
\def\Re{{\mathfrak{Re}} }
\def\Im{{\mathfrak{Im}} }
\def\vect{\overrightarrow}
\def\eq{\Leftrightarrow}
\def\implique{\Rightarrow}
\def\si{\Leftarrow}
\def\grad{\vect{\nabla}}
\def\laplacien{\nabla^2}
\def\div{\nabla {\bf{\cdot}}}
\def\rot{\nabla \times}
\def\aprime{{\alpha^\prime}}
\def\Tr{\mbox{Tr}}
\def\Str{\mbox{Str}}
\def\qed{\ensuremath{\Box}}
\def\N{\ensuremath{\mathbb{N}}}
\def\Z{\ensuremath{\mathbb{Z}}}
\def\Q{\ensuremath{\mathbb{Q}}}
\def\R{\ensuremath{\mathbb{R}}}
\def\C{\ensuremath{\mathbb{C}}}

\newcommand\omegaLC{{\overset{\circ}{\omega}}{}}


%%Environement Maths%%%%%%%%%%%%%%%%%
\newtheorem{theorem}{Theorem}[section]
\newtheorem{lemma}[theorem]{Lemma}
\newtheorem{proposition}[theorem]{Proposition}
\newtheorem{corollary}[theorem]{Corollary}
\newenvironment{proof}[1][Proof]{\begin{trivlist}
\item[\hskip \labelsep {\bfseries #1}]}{\end{trivlist}}
\newenvironment{definition}[1][Definition]{\begin{trivlist}
\item[\hskip \labelsep {\bfseries #1}]}{\end{trivlist}}
\newenvironment{example}[1][Example]{\begin{trivlist}
\item[\hskip \labelsep {\bfseries #1}]}{\end{trivlist}}
\newenvironment{remark}[1][Remark]{\begin{trivlist}
\item[\hskip \labelsep {\bfseries #1}]}{\end{trivlist}}

\newcommand\GLC{{\overset{\ \circ}{G}}{}}



\begin{document}

\preprint{arXiv:23xx.xxxxx}
\title{Teleparallel Minkowski Spacetime with Perturbative Approach for Teleparallel Gravity on Proper Frame}

\author{A. Landry}
\email{A.Landry@dal.ca}
\affiliation{Department of Mathematics and Statistics, Dalhousie University, P.O. Box 15 000, Halifax, Nova Scotia, Canada, B3H 4R2}

\author{R. J. {van den Hoogen}}
\email{rvandenh@stfx.ca}
\affiliation{Department of Mathematics and Statistics, St. Francis Xavier University, Antigonish, Nova Scotia, Canada, B2G 2W5}


\date{\today}

\begin{abstract}


%\section*{Abstract}

In this paper, we first develop a complete perturbation theory requiring only the perturbation of the fundamental quantities describing Teleparallel Gravity. We first obtain the physical quantities by perturbing the coframes taking into account the gauge metric and spin-connection conditions. We obtain the perturbed field equations involving these perturbed quantities. We will study some specific cases of perturbations of coframes and finally discuss the stability of the Minkowski background. Our perturbation framework is based on using a proper orthonormal frame throughout, which is possible since we remain with a theory of Teleparallel Gravity.

\end{abstract}


%\pacs{}
%% Put Correct Pacs Numbers

\maketitle

\tableofcontents



%%-------------------------------------------------------------------------
%%----------------------   SECTION       ----------------------------------
%%-------------------------------------------------------------------------
\section{Introduction}\label{sect1}


There are two major classes of theories for physical phenomena: gravitational theories and quantized theories \cite{peskin,sred,schiff,griffiths}. The first class of theories are used to explain phenomena at the astrophysical scale, for example General Relativity (GR) has been very successful in explaining astrophysical phenomena \cite{wein,misner,podolsky,will2018}. However, the second class of theories concerns phenomena occurring at the microscopic scale involving fundamental quantum particles. Attempts have been made to reconcile the two classes of theories in order to have a general all encompassing theory. A theory capable of dealing with very low amplitude physical and geometrical quantities, as is the case for theories based on quantization, is desirable.

Indeed, Quantum Mechanics (QM) as well as Quantum Field Theory (QFT) have well-established perturbative theories: a potential is perturbed generating a correction of the eigenvalues of the energies as well as corrections to the wave functions \cite{peskin,sred,schiff,griffiths}. QM and QFT are well established and have been used to describe gravitational corrections of curved spacetimes of physical phenomena that can occur at the microscopic scale \cite{landryhammad1,landryhammad2,landryhammad3,landryhammad4}. Unfortunately, this perturbative approach to GR is problematic primarily because one requires an identifiable background on which to do the perturbations \cite{cosmomodels}. One can of course use gauge invariant variables to address this challenge.

Recently, there has been a growing interest in the development of Teleparallel Gravity as an alternative theory to GR \cite{andrade1,booktegr1,bibletegr,coley2021a,coley2020,coley2019,coley2022a,tdsprep}. Teleparallel Gravity needs to be better understood and developed in order to address foundational, physical and geometrical problems. Here we will illuminate some of the challenges and nuances present within perturbative approaches to Teleparallel Gravity.

Golovnev and Guzman \cite{minkowbackpert1} studied some very specific perturbations in a Minkowski background. They assumed a very specific form for the co-frame and the boosted coframe. Although, in the Minkowski background, one is still left with specific cases of perturbations and a particular case of coframe to be perturbed. In another paper, Golovnev and Guzman \cite{perturbpert2} tried to describe Minkowski spacetime itself via a specific perturbation of the trivial tetrad. Again, they used a specific case of perturbations on the co-frame. These approaches do not really advance towards a general theory perturbations within teleparallel gravity.

Recently, within a cosmological setting, Bahamonde et al. \cite{bahamondepert} investigated perturbations occurring on a FLRW-type background. They defined a very specific form of perturbation compatible with this background. They then obtain the perturbed field equations.  In addition, they investigated the consequent effects of perturbations on the torsion and on different physical quantities. Most of the types of perturbations studied lead to the flat FLRW background case under some precise limits. On the other hand, some perturbation modes do not propagate, which maintains the strong coupling. This is the case of the scalar and pseudo-scalar parts of perturbations. Here we still have work with a limited scope, hence the need for a more general theory of perturbation in teleparallel gravity.


Bamba and Cai's papers focus on Gravitational Waves (GW) in Teleparallel Gravity \cite{gwperturb1,gwperturb2}. GWs are a class of wave-like perturbations of Minkowski spacetime. They are still dealing here with a specific case of perturbation. In Bamba \cite{gwperturb1}, they place themselves in the Minkowski Background to process the GWs in Teleparallel Gravity. In Cai \cite{gwperturb2}, they place ourselves in the FLRW background. They therefore have a generalization of Bamba's work for GWs compatible with the cosmological models. In addition, in \cite{gwperturb2}, they add the effects of scalar fields in their perturbations. Not only are they still dealing with specific cases of perturbations, but they are moving from the Minkowski background to the FLRW background. However, they still don't have a general theory for the Minkowski background. Therefore, a more general and fundamental theory applicable for any perturbation and any co-frame in Minkowski spacetime in Teleparallel Gravity is needed.


We begin this paper with a definition of Minkowski geometry and Minkowski spacetime within Teleparallel Gravity. Then, we will investigate the effect of perturbations in Teleparallel Gravity. After, we will study the stability of Minkowski spacetime by using the perturbed quantities and field equations.


In Teleparallel Gravity, co-frames encode both the gravitational and inertial effects. Our goal is to explore the perturbations of gravity and therefore we shall carefully construct a perturbative theory that achieves this goal. If we transform initially to ``proper'' frames which encode only the gravitational effects and then perform perturbations on all physical quantities, and consequently ensuring that the resulting perturbed theory is still within the class of Teleparallel theories of gravity will yield the general allowable form for perturbations within Teleparallel Gravity. We will perturb physical quantities which maintain the ``proper frames'' thus avoiding the challenge of interpreting the spurious inertial effects that may appear in ``non-proper frames'' \cite{tegrspinconnect1,andrade1,hohmann2022,bibletegr,booktegr1}.


We want to highlight the effects of perturbations in Teleparallel Gravity. For example, in absolute vacuum, one can highlight the effects of perturbations modifying this same vacuum. For example, we will determine the gravitational Energy-Momentum associated with a perturbation. We will apply this theory of perturbations in Teleparallel Gravity to some examples and problems of Physics \cite{teleparabackground,cosmocons2,bibletegr}. Particularly, we will study by these coframe perturbations the stability of the Minkowski background and determine the required symmetries conditions to satisfy.


This paper is divided as follows. In section \ref{sect2}, we present a summary of Teleparallel Gravity and propose a definition of Minkowski geometry within Teleparallel Gravity. In section \ref{sect3}, we will define the perturbations maintaining the ``proper frames'', the orthonormal framework and we will also provide the perturbed Field Equations (FE). In section \ref{sect6}, we will explore some coframe perturbations to determine the stability criterions for Minkowski spacetime. We can also generalize these criterions to null and constant torsion spacetimes.


%%-------------------------------------------------------------------------
%%----------------------   SECTION       ----------------------------------
%%-------------------------------------------------------------------------

\section{Teleparallel Theories of Gravity}\label{sect2}




\subsection{Notation}\label{sect210}

Greek indices $(\mu,\nu,\dots)$ are employed to represent spacetime coordinate indices, while Latin indices $(a,b,\dots)$, are employed to represent frame or tangent-space indices. As is standard notation, round parentheses surrounding indices represents symmetrization, while square brackets represent anti-symmetrization.  Any quantity that is computed using a Levi-Civita connection $\omegaLC^a_{\phantom{a}b\mu}$, will have a circle above symbol.  A comma will denote a partial derivative. The metric signature is assumed to be $(-,+,+,+)$.



\subsection{Torsion-based Theories}\label{sect211}


Torsion-based theories of gravity are a subclass of Einstein-Cartan theories \cite{einsteincartan,booktegr1,bibletegr}. This superclass of theories contains theories based solely on the curvature, for example, General Relativity, or $f\left(R\right)$ theories where $R$ is Ricci curvature scalar.  Einstein-Cartan theories of gravity also contain theories of gravity based solely on the torsion, for example, Teleparallel theories of gravity including, New General Relativity \cite{Hayashi:1979qx} and $f\left(T\right)$ theories where $T$ is the torsion scalar.  In addition, theories of gravity based on both the curvature and torsion scalars  ($f\left(R,T\right)$-type) are also subclasses of Einstein-Cartan theories of gravity.  Recently, there has been an emergence of theories based on the non-metricity ($f\left(Q\right)$-type), although they are less known \cite{mondth1,nonmetricity1,bibletegr}. In this paper we are interested in Teleparallel gravity, and in particular $f(T)$ teleparallel gravity \cite{andrade1,booktegr1,coley2021a,coley2020,coley2019,coley2022a,bibletegr,hohmann2022}.



\subsection{Geometrical Framework for Teleparallel Gravity}\label{sect212}

Let $M$ be a $4$-dimensional differentiable manifold with coordinates $x^\mu$.  Then the geometry of the manifold is characterized by the three geometrical objects
\begin{itemize}
\item{\bf The Co-frame:} $h^a=h^a_{\;\;\mu}dx^\mu$. This quantity generally encodes both the gravitational and inertial effects in a gravitational system. The dual of the co-frame is defined as the vector field $h_a=h_a^{~\mu}\frac{\partial}{\partial x^\mu}$ such that $h^a_{~\mu}h_b^{~\mu}=\delta^a_b$.
\item{\bf The Gauge Metric:} $g_{ab}$. This object expresses the ``metric'' of the tangent space such that $g_{ab}=g(h_a,h_b)$. Having a metric allows one to define lengths and angles.
\item {\bf The Spin-connection:} $\omega^a_{\;\;b}=\omega^a_{\;\;b\mu}dx^\mu$. Having a connection, allows one to ``parallel transport'' or equivalently allows one to define a covariant differentiation.
\end{itemize}

In Teleparallel Gravity, the co-frame, gauge metric and spin connection are restricted and interdependent, characterized by the following two postulates  \cite{andrade1,booktegr1,bibletegr}:
\begin{itemize}
\item{\bf Null Curvature:}
    \begin{equation}\label{101}
        R^a_{\;\;b\nu\mu}\equiv\omega^a_{~b\mu,\nu}-\omega^a_{~b\nu,\mu}
                         +\omega^a_{~c\nu}\omega^c_{~b\mu}
                         -\omega^a_{~c\mu}\omega^c_{~b\nu}=0
    \end{equation}
\item{\bf Null Non-Metricity:}
    \begin{equation}\label{102}
        Q_{ab\mu}\equiv-g_{ab,\mu}+\omega^c_{~a\mu}g_{cb}+\omega^c_{~b\mu}g_{ac}=0
    \end{equation}
\end{itemize}

In Teleparallel Gravity, the only remaining non-null field strength is the torsion defined as
\begin{equation}\label{103}
T^a_{\phantom{a}\mu\nu}= h^a_{\phantom{a}\nu,\mu}-  h^a_{\phantom{a}\mu,\nu}+\omega^a_{\phantom{a}b\mu}h^b_{\phantom{a}\nu}-\omega^a_{\phantom{a}b\nu}h^b_{\phantom{a}\mu}
\end{equation}
It is now possible to construct a gravitational theory that depends only on the torsion. However before proceeding, we illustrate the effects of gauge transformations on the geometry and how we can judiciously choose a gauge to simplify our computations.


\subsection{Linear Transformations and Gauge Choices}

From the Principle of Relativity we impose the requirement that the physical gravitational system  under consideration be invariant under $GL(4,\mathbb{R})$ local linear transformations of the frame.  These types of transformations allow one to pass from one frame of reference to another frame of reference.  For the fundamental geometrical quantities $\{h^a, g_{ab},\omega^a_{~bc}\}$, we have the following transformation rules under a general linear transformation $M^a_{~b} \in GL(4,\mathbb{R})$:
\begin{eqnarray}
h'^a_{~\mu}&=&M^a_{~b}\,h^b_{~\mu}, \label{217}
\\
g'_{ab}&=&M_a^{~e}\,M_b^{~f}\,g_{ef},\label{217b}
\\
\omega'^a_{\,~b\mu} &=& M^a_{~e}\,\omega^e_{~f\mu} \,M_b^{~f}+M^a_{~e}\,\partial_\mu\,M_b^{~e}.\label{218}
\end{eqnarray}
where $M_b^{~a}=(M^{-1})^a_{~b}$ represents the inverse matrix. Equation \eqref{218} shows that the Spin-connection transforms non-homogeneously under a general linear transformation.




\subsubsection{Gauge Choices and Teleparallel Gravity}

Physical phenomena must respect the principle of Gauge Invariance. The physical phenomenon must be explainable and valid, regardless of the gauge and its possible transformations. If this general principle is important for quantized theories, then this same principle is also important for Teleparallel Gravity. Generally, we have tremendous choice of gauge depending on the assumed symmetries of the physical system.   However, once we have made a gauge choice, the consequent field equations describing the theory must transform covariantly (i.e., they are invariant) under any remaining gauge freedom.

\paragraph{Proper Orthonormal Frame}

The Null Curvature postulate guarantees that there exists an element $M^a_{~b} \in GL(4,\mathbb{R})$ such that
\begin{equation}
\omega^a_{~b\mu} \equiv (M^{-1})^{a}_{~b}\partial_\mu(M^b_{~c})
\end{equation}
Since the connection transforms non-homogeneously under local linear transformations, we can always apply the linear transformation $M^a_{~b}$ to transform to a proper frame in which $ \omega^a_{~b\mu} = 0$.  Further, within this proper frame, given the Null Non-Metricity postulate it is then possible to apply a second constant linear transformation to bring the gauge metric to some desired form. For example, we can transform to a gauge in which the spin connection is null and the gauge metric is $g_{ab}=\mathrm{Diag}[-1,1,1,1]$ which we will call a ``proper orthonormal frame''. The only remaining gauge freedom in this case are global (constant) Lorentz transformations.

\paragraph{Orthonormal Frame}
If one prefers not to be restricted to a proper frame, then there is more flexibility.  Since the gauge metric is symmetric, we can still always choose an ``orthonormal frame'' in which the gauge metric becomes $g_{ab}=\mathrm{Diag}[-1,1,1,1]$ but the spin connection may be non-trivial. Assuming an orthonormal frame, the remaining gauge freedom is represented by proper orthochronous Lorentz transformations $SO^+(1,3)$ subgroup of $GL(4,\mathbb{R})$. Other gauge choices might include Complex-Null, Half-Null, Angular-Null and others \cite{coley2021a,coley2020,coley2019}.  In the orthonormal frame, given the Null Curvature postulate, there exists a $\Lambda^a_{~b} \in SO^+(1,3)$ such that the spin connection is
\begin{equation}
\omega^a_{~b\mu} \equiv (\Lambda^{-1})^{a}_{~b}\partial_\mu(\Lambda^b_{~c})
\end{equation}
and given the Null Non-Metricity postulate we have the restriction $\omega_{(ab)\mu}=0$.



However, in either choice of gauge, we note that the spin connection, $\omega^a_{~b\mu}$, is not a true dynamical variable and only encodes inertial effects present in the choice of frame \cite{andrade1,booktegr1,coley2021a,coley2020,coley2019,coley2022a,bibletegr,hohmann2022,tegrspinconnect1}.




\subsection{Action for $f(T)$ Teleparallel Gravity}

In principle one can construct a Lagrangian density from any scalars built from the torsion tensor.  One such scalar is \cite{andrade1,booktegr1,coley2021a,coley2020,coley2019,coley2022a,bibletegr,hohmann2022}:
\begin{eqnarray}\label{212}
T=\frac{1}{4}T^a_{~bc} T_a^{~bc}+\frac{1}{2}T^a_{~bc} T^{cb}_{~~a}-T^a_{~ca} T^{bc}_{~~b}.
\end{eqnarray}
which we will call ``the'' torsion scalar $T$ .  Another related scalar, for example used in New General Relativity \cite{Hayashi:1979qx}, is
\begin{eqnarray}\label{212b}
\widetilde{T}=c_1T^a_{~bc} T_a^{~bc}+c_2T^a_{~bc} T^{cb}_{~~a}+c_3T^a_{~ca} T^{bc}_{~~b}
\end{eqnarray}
Other torsion scalars could be included, but these scalars are not invariant under $SO^+(1,3)$ and include parity violating terms \cite{Hayashi:1979qx}.

Here we are interested in a particular class of Teleparallel Gravity theories, $f(T)$ teleparallel gravity.  The action describing the $f(T)$ teleparallel theory of gravity containing matter is \cite{andrade1,booktegr1,coley2021a,coley2020,coley2019,coley2022a,bibletegr,hohmann2022}:
\begin{equation}\label{201}
S_{f\left(T\right)}=
%\int \,d^4\,x\,\left[\mathcal{L}_{f\left(T\right)}+\mathcal{L}_{Matter}\right]=
\int \,d^4\,x\,\left[\frac{h}{2\,\kappa}\,f\left(T\right)+\mathcal{L}_{Matter}\right].
\end{equation}
where $h=\mbox{Det}\left(h^a_{~\mu}\right)$ is the determinant of the veilbein, the parameter $\kappa$ is the gravitational coupling constant which contains the physical constants and $f\left(T\right)$ is an arbitrary function of the torsion Scalar, $T$, given by equation \eqref{212}.



\subsection{Field Equations for $f(T)$ Teleparallel Gravity }\label{sect22}


From the Lagrangian defined by equation (\ref{201}) and by Least-Action principle, the Symmetric and Antisymmetric FEs are respectively \cite{coley2021a,coley2020,coley2019}:
\begin{eqnarray}\label{221}
\kappa \Theta_{\left(ab\right)}\,&=&\,f_{TT}\left(T\right)\,S_{\left(ab\right)}^{~~~\mu}\,\partial_{\mu} T+f_{T}\left(T\right)\,\overset{\ \circ}{G}_{ab}+\frac{g_{ab}}{2}\,\left[f\left(T\right)-T\,f_{T}\left(T\right)\right],
\nonumber\\
0 \,&=&\,f_{TT}\left(T\right)\,S_{\left[ab\right]}^{~~~\mu}\,\partial_{\mu} T,
\end{eqnarray}
where $\overset{\ \circ}{G}_{ab}$ is the Einstein tensor computed from the Levi-Civita connection of the metric.  We define the super-potential \cite{andrade1,booktegr1,coley2020,coley2021a}:
\begin{equation}\label{213}
S_a^{~\mu\nu}=\frac{1}{2}\left(T_a^{~\mu\nu}+T_{~~a}^{\nu\mu}-T_{~~a}^{\mu\nu} \right) -h_a^{~\nu}\,T^{\rho\mu}_{~~\rho}+h_a^{~\mu}\,T^{\rho\nu}_{~~\rho}.
\end{equation}
%where $h\equiv Det\left(h_a^{\;\;\mu}\right)$ is the determinant of the coframe, $T^b_{\;\;a\mu}$ is the torsion Tensor, $\kappa $ is the gravitational coupling constant and $S_a^{\;\;\mu\nu}$ is defined from eq. (\ref{213}).
The canonical Energy-Momentum is defined as \cite{coley2020}:
\begin{equation}\label{202}
h\,\Theta_a^{~\mu} \equiv \frac{\delta \mathcal{L}_{Matter}}{\delta h^a_{~\mu}}.
%\quad\text{and} \quad -\frac{1}{2} \frac{\delta \mathcal{L}_{Matter}}{\delta g_{ab}} \equiv T_{\mu\nu}\quad\quad \text{in GR}.
\end{equation}
With an orthonormal gauge choice, and consequent invariance under $SO^+(1,3)$ transformations, it can be shown that
\begin{equation}
\Theta_{[ab]}=0,
\end{equation}
and the metrical energy momentum $T_{ab}$ and the symmetric part of canonical energy momentum satisfy
\begin{equation}
\Theta_{(ab)}=T_{ab}\equiv\frac{1}{2}\frac{\delta L_{Matt}}{\delta g_{ab}}.
\end{equation}



\section{Constant Torsion Spacetimes} \label{sect222}


A class of interesting spacetimes are those leading to a constant Scalar torsion, i.e. $T=T_0=\text{Const}$. This class of spacetimes includes the Minkowski spacetime, amongst others.  In this case, the equations (\ref{221}) will simplify with $\partial_{\mu} T=0$ as follows, leaving only the symmetric part of the field equations:
\begin{eqnarray}\label{231}
\kappa \Theta_{\left(ab\right)}\,&=& f_{T}\left(T_0\right)\,\overset{\ \circ}{G}_{ab}+\frac{g_{ab}}{2}\,\left[f\left(T_0\right)-T_0\,f_{T}\left(T_0\right)\right].
\end{eqnarray}
The antisymmetric part of the field equations becomes identically satisfied.
% and is respected by default.
%Moreover, we notice that even the symmetric part was greatly simplified, because we no longer need the Superpotential $S_{ab}^{\;\;\;\mu}$ term.
We can now divide equation (\ref{231}) by $f_{T}\left(T_0\right)$ to obtain:
\begin{eqnarray}\label{231a}
\kappa_{eff} \Theta_{\left(ab\right)}\,&=& \overset{\ \circ}{G}_{ab}+g_{ab}\,\left[\frac{f\left(T_0\right)}{2\,f_{T}\left(T_0\right)}-\frac{T_0}{2}\right]
\nonumber\\
&=& \overset{\ \circ}{G}_{ab}+g_{ab}\,\Lambda\left(T_0\right).
\end{eqnarray}
where we define the re-scaled gravitational coupling constant $\kappa_{eff}=\frac{\kappa}{f_{T}\left(T_0\right)}$ and an effective cosmological constant $\Lambda\left(T_0\right)$ both dependent on the value of $T=T_0$.  We observe that if $T=T_0=\text{Const}$ then the $f(T)$ teleparallel field equations reduce to those of GR having a re-scaled gravitational coupling and a cosmological constant.

Due to its importance in characterizing the Minkowski geometry, we carefully consider the case $T_0=0$ for further consideration.
%In general, for a non-vacuum spacetimes, $T_0\neq 0$. However, for the case where $T_0=0$, there are two types of situations: the pure Minkowski spacetime representing the perfect vacuum and the case where the Minkowski spacetime cannot be defined. In the last case, we will also see that the perfect vacuum is not possible.


\subsection{Null Torsion Scalar spacetimes}\label{sect224}

When $T_0=0$, the field equations reduce to:
\begin{eqnarray}\label{232}
\kappa_{eff} \Theta_{\left(ab\right)}\,&=& \overset{\ \circ}{G}_{ab}+g_{ab}\,\left[\frac{f\left(0\right)}{2\,f_{T}\left(0\right)}\right],
\nonumber\\
&=& \overset{\ \circ}{G}_{ab}+g_{ab}\,\Lambda\left(0\right).
\end{eqnarray}
where $\kappa_{eff}=\frac{\kappa}{f_{T}\left(0\right)}$ and $\Lambda\left(0\right)=\frac{f(0)}{2\,f_{T}\left(0\right)}$.  If $f(0)\neq 0$, then the Cosmological Constant $\Lambda(0)\neq 0$.
%However, with this non-zero cosmological constant by default, we cannot declare that this spacetime is Minkowski, because this $\Lambda(0)$ will give an Energy-Momentum and/or a non-zero Einstein Tensor by default.


\subsubsection{Definition: Minkowski Geometry and Minkowski spacetime}\label{sect223}

Before obtaining the field equations and introducing the perturbations on such, one must clearly define the true nature of the Minkowski spacetime in Teleparallel Gravity. This will make it possible to better understand the nature and origin of the equations involving the dominant quantities with respect to the perturbed quantities. The Minkowski geometry is characterized as follows:
\begin{itemize}
\item{Maximally symmetric:}  The Minkowski geometry is invariant under a $G_{10}$ group of transformations \cite{coley2020}.
\item{Null Curvature:} $R_{~b\mu\nu}^{a}=0$
\item{Null Torsion:} $T^{a}_{~\mu\nu}=0$
\item{Null Non-Metricity:} $Q_{ab\mu}=0$
\end{itemize}
One of the consequences is that Minkowski geometry is a smooth geometry everywhere without condition and singularity.

We distinguish between Minkowski geometry and Minkowski spacetime in teleparallel gravity as follows.  Minkowski geometry is defined independently of any field equations, while Minkowski spacetime is a Minkowski geometry that is a solution to the teleparallel gravity field equations where the matter source is vacuum, $\Theta_{ab}=0$.

If the geometry is Minkowski, then the torsion scalar is identically zero. Note that the converse is not necessarily true. The Einstein tensor $\overset{\ \circ}{G}_{ab}=0$, and since the matter source is vacuum, $\Theta_{ab}=0$, the field equations \eqref{232} reduce to
\begin{equation}\label{242}
0 = \frac{f\left(0\right)}{2}\,g_{ab}.
\end{equation}
From the field equations \eqref{242}, if the geometry is Minkowski and $\Theta_{ab}=0$ then $f(0) = 0$.  In this case the solution is a Minkowski spacetime, a Minkowski geometry that satisfies the field equations in vacuum.  Alternatively, if $f(0)\not=0$ then a solution to the field equations \eqref{242}  necessarily requires a non-null $\Theta_{ab}$, and consequently this spacetime is not a Minkowski spacetime even though the geometry is Minkowski.  Of course, the non-trivial $\Theta_{ab}$ can be interpreted as the energy density of the vacuum.  Expressing the statement clearly, Minkowski geometry is a solution to the vacuum  $f(T)$ teleparallel gravity field equations  only if $f(0)=0$.


\section{Perturbations in Teleparallel Geometries}\label{sect3}

\subsection{Proper Orthonormal Perturbation of the Co-frame}\label{sect_perturbed_frames}

As described earlier, a teleparallel geometry is characterized in general via the triplet of quantities, the co-frame one form, $h^a$, the spin connection one-form, $\omega^a_{~b}$, and the metric tensor field $g_{ab}$ with two constraints Null Curvature and Null Non-metricity.  As argued earlier, assuming that the physical system is invariant under $GL(4,\mathbb{R})$ linear transformations, means that even before constructing a perturbative theory, one can always choose to begin in a ``proper orthonormal frame'' as our background without loss of generality:
\begin{equation}
{h}^a={h}^a_{~\mu}dx^\mu, \qquad {\omega}^a_{~b}=0,\qquad {g}_{ab}=\eta_{ab}=\mathrm{Diag}[-1,1,1,1].
\end{equation}

Now we apply a perturbation to all three quantities as follows
\begin{equation}
h'^a={h}^a+\delta h^a, \qquad \omega'^a_{~b}=\delta\omega^a_{~b},\qquad g'_{ab}=\eta_{ab}+\delta g_{ab}
\end{equation}
The perturbed geometry is no longer expressed in a proper orthonormal frame.  The perturbed system is only proper if $\delta \omega^a_{~b}=0$ and orthonormal if $\delta g_{ab}=0$. However, we shall show that we can always transform to a proper orthonormal perturbation scheme.

We note that the perturbed geometry given by the triplet $\{h'^a,\omega'^a_{~b},g'_{ab}\}$ must still satisfy the Null Curvature and Null Non-Metricity constraints, else one is moving outside of the theory of teleparallel gravity. In general the perturbations $\delta h^a$, $\delta\omega^a_{~b}$ and $\delta g_{ab}$ are not all independent.  The Null Curvature constraint for the perturbed connection $\omega'^a_{~b}$ implies that there exists some local linear transformation $L^a_{~b} \in GL(4,\mathbb{R})$ such that
\begin{equation}
\delta\omega^a_{~b}=(L^{-1})^a_{~c}dL^c_{~b}
\end{equation}
where $d$ indicates the exterior derivative.  This means we can apply this general linear transformation to the perturbed system to express it in a perturbed proper frame
\begin{equation} \label{properperturbdeframe}
\bar{h}'^a=L^a_{~b}({h}^b+\delta h^b), \qquad \bar{\omega}'^a_{~b}=0,\qquad \bar{g}'_{ab}=(L^{-1})^c_{~a}(L^{-1})^d_{~b}(\eta_{cd}+\delta g_{cd})
\end{equation}
where we have used a bar to indicate that we are now in a proper frame.

The Null Non-Metricity condition applied to this ``perturbed proper frame'' \eqref{properperturbdeframe} means that  $\bar{g}'_{ab}$ is a symmetric matrix of constants which can diagonalized.  That is, there exists a matrix $P^a_{~b}\in GL(4,\mathbb{R})$ of constants such that $\bar{g}'_{ab}=(P^{-1})^c_{~a}(P^{-1})^d_{~b}\eta_{cd}$.  So we can apply this constant transformation $P^a_{~b}$ to the ``perturbed proper frame'' \eqref{properperturbdeframe} to obtain a ``perturbed proper orthonormal frame'' without loss of generality.
\begin{subequations}
\begin{eqnarray}
\hat{h}'^a&=&P^a_{~b}\bar{h}'^b = P^a_{~b}L^b_{~c}({h}^c+\delta h^c),\\
\hat{\omega}'^a_{~b}&=& 0,\\
\hat{g}_{ab}' &=& \eta_{ab}.
\end{eqnarray}
\end{subequations}

We observe that we can investigate perturbations in teleparallel geometries by simply looking at the perturbations in co-frame using proper orthonormal frames.  Doing so ensures that Null Curvature and Null Non-Metricity constraints are respected.  If we define the composition of the two linear transformation as matrix $M^a_{~b}=P^a_{~c}L^c_{~b}\in GL(4,\mathbb{R})$, then the ``perturbed proper orthonormal frame'' becomes
\begin{equation}\label{300}
\hat{h}'^a=M^a_{~b}\left({h}^b + \delta h^b\right).
\end{equation}
which encodes all possible perturbations within a proper orthonormal framework. If $M^a_{~b}=\delta^a_b$ then the only perturbations are perturbations in the original proper orthonormal frame.  The matrix $M^a_{~b}$ encodes the perturbations that took place originally in the spin connection and metric, but ensures that the resulting perturbed system is teleparallel in nature. For completeness the original perturbations can be expressed in terms of $M^a_{~b}$ as
\begin{equation}\label{300a}
\delta \omega^a_{~b}=(M^{-1})^a_{~c}dM^c_{~b},\qquad \delta g_{ab}=(M^{-1})^c_{~a}(M^{-1})^d_{~b}\eta_{cd}-\eta_{ab}
\end{equation}

Now, in a perturbative approach, to first order we have that
\begin{eqnarray}
M^a_{~b}&\approx& \delta^a_b+\mu^a_{~b} \\
\delta h^a &\approx & \nu^a_{~b} h^b
\end{eqnarray}
for some $\mu^a_{~b}$ and $\nu^a_{~b} \in \mathfrak{gl}(4,\mathbb{R})$. Therefore, putting it all together, we have to first order
\begin{subequations}\label{proper_perturb}
\begin{eqnarray}
\hat{h}'^a&=& h^a+(\mu^a_{~b}+\nu^a_{~b})h^b=h^a+\lambda^a_{~b}h^b,\\
\hat{\omega}'^a_{~b}&=& 0,\\
\hat{g}_{ab}' &=& \eta_{ab},
\end{eqnarray}
\end{subequations}
where $\lambda^a_{~b} \in M(4,\mathbb{R})$, the set of $4\times 4$ real valued matrices. Perturbations of the independent quantities in teleparallel geometry can always be transformed to the form \eqref{proper_perturb}. The matrix $\lambda$ can be invariantly decomposed into a trace, symmetric trace-free, and anti-symmetric parts.


For the next section and in the appendix, we will apply the perturbations 
\begin{equation}
\delta h^a=\lambda^a_{~b}h^b, \qquad \delta \omega^a_{~b}=0, \qquad \delta g_{ab}=0,
\label{proper_perturbations}
\end{equation}
to the $f(T)$ teleparallel field equations in a proper orthonormal frame. In particular, we will look at perturbations of constant scalar torsion spacetimes.
 

\subsection{Perturbed $f(T)$ Teleparallel Field Equations: General}\label{sect33}


Considering perturbations of the field equations \eqref{221} we obtain
\begin{subequations}\label{301}
\begin{eqnarray}
\kappa \left[\Theta_{\left(ab\right)} +\delta \Theta_{(ab)}\right]&=&
 f_{TT} \left(T+\delta T\right)\,\left[S_{(ab)}^{~~~\mu}+\delta S_{(ab)}^{~~~\mu}\right]\left[\partial_{\mu}T + \partial_{\mu}\left(\delta T\right)\right]
 \nonumber\\
 &&\quad 
 +f_T\left(T+\delta T\right)\,\left[\overset{\ \circ}{G}_{ab}+\delta \overset{\ \circ}{G}_{ab}\right]
\nonumber\\
&&\quad
+\frac{g_{ab}}{2}\left[f\left(T+\delta T\right)-\left(T+\delta T\right)\,f_T \left(T+\delta T\right)\right],
\\
0 &=& f_{TT}\left(T+\delta T\right)\,\left[S_{[ab]}^{~~~\mu}+\delta S_{[ab]}^{~~~\mu}\right]\partial_{\mu}\left(T+\delta T\right),
\end{eqnarray}
\end{subequations}
which to first order in perturbations yields
\begin{subequations}\label{302}
\begin{eqnarray}
\kappa\,\delta \Theta_{\left(ab\right)} &\approx& 
 \left[f_{TTT}\,S_{(ab)}^{~~~\mu}\partial_{\mu}T+f_{TT}\,\left(\overset{\ \circ}{G}_{ab}-\frac{T}{2}\,g_{ab}\right)\right]\,\delta T +f_{T}\,\delta \overset{\ \circ}{G}_{ab}
\nonumber\\
&&\quad + f_{TT}\left[\delta S_{(ab)}^{~~~\mu}\,\partial_{\mu}T+S_{(ab)}^{~~~\mu}\,\partial_{\mu}\left(\delta T\right)\right]  +O\left(|\delta h|^2\right),\\
%%
0 & \approx & 
 f_{TTT}\,\left[S_{[ab]}^{~~~\mu}\partial_{\mu} T\right]\,\delta T
+f_{TT}\,\left[S_{[ab]}^{~~~\mu}\partial_{\mu}\left(\delta T\right)+\delta S_{[ab]}^{~~~\mu}\partial_{\mu} T\right]+O\left(|\delta h|^2\right),
\end{eqnarray}
\end{subequations}
where we no longer explicitly show the functional dependence in $F$.

In Appendix \ref{appena} perturbations of different dependent quantities are explicitly computed in terms of the perturbations \eqref{proper_perturbations}, for example $\delta T, \delta S_{[ab]}^{~~~\mu}$ etc. Here the $\delta T$ is given by equation \eqref{a004a} and the $\delta S_{ab}^{~~~\mu}$ is given by equation \eqref{a014a}. Equations \eqref{302} give us expressions for the perturbations to the matter resulting from perturbations in the co-frame, and constraints on perturbations to the antisymmetric part of the super-potential.

%{\color{red} Should we absolutely put $\delta T$, $\delta S_{[ab]}^{~~~\mu}$, $\delta S_{(ab)}^{~~~\mu}$ and $\delta \overset{\ \circ}{G}_{ab}$ in terms of $\delta h^a_{~\mu}$ inside the equations above? }

 


\subsection{Perturbed $f(T)$ Teleparallel Field Equations: Constant Torsion Scalar}\label{sect34}

To study the effects of perturbations of the co-frame in constant Scalar torsion spacetimes, one substitutes $T=T_0=\text{Const}$ into equations \eqref{302}. This means $\partial_{\nu}T=0$. If we divide by $f_{T}\left(T_0\right)$, the equations (\eqref{302}) become:
\begin{subequations}
\begin{eqnarray}
\kappa_{eff}\,\delta \Theta_{(ab)} 
&\approx & \delta \overset{\ \circ}{G}_{ab}+\frac{f_{TT}\left(T_0\right)}{f_{T}\left(T_0\right)} \left[S_{(ab)}^{~~~\mu}\,\partial_{\mu}\left(\delta T\right)+\delta T\,\left(\overset{\ \circ}{G}_{ab}-\frac{T_0}{2}\,g_{ab}\right)\right]+O\left(|\delta h|^2\right),\label{306}\\
%%
0 & \approx & \left(\frac{f_{TT}\left(T_0\right)}{f_{T}\left(T_0\right)}\right)S_{[ab]}^{~~~\mu}\partial_{\mu}(\delta T)+O\left(|\delta h|^2\right),\label{307}
\end{eqnarray}
\end{subequations}
where $\kappa_{eff}=\frac{\kappa}{f_{T}\left(T_0\right)}$.


\subsection{Perturbed $f(T)$ Teleparallel Field Equations: Zero Torsion Scalar}\label{sect36}

For spacetimes that have a zero Scalar torsion, $T=0$, equations \eqref{306} and \eqref{307} become:
\small
\begin{subequations}
\begin{eqnarray}
\kappa_{eff}\,\delta \Theta_{\left(ab\right)} &\approx & \delta \overset{\ \circ}{G}_{ab}+\frac{f_{TT}\left(0\right)}{f_{T}\left(0\right)} \left[S_{(ab)}^{~~~\mu}\,\partial_{\mu}\left(\delta T\right)+\delta T\,\overset{\ \circ}{G}_{ab}\right]+O\left(|\delta h|^2\right),\label{306a}\\
%%
0 & \approx & \left(\frac{f_{TT}\left(0\right)}{f_{T}\left(0\right)}\right)S_{[ab]}^{~~~\mu}\partial_{\mu}(\delta T)+O\left(|\delta h|^2\right),\label{307a}
\end{eqnarray}
\end{subequations}
\normalsize
where $\kappa_{eff}=\frac{\kappa}{f_{T}\left(0\right)}$. In general,  $S_{ab}^{~~\mu}\neq 0$, $\overset{\ \circ}{G}_{ab}\neq 0$ and $f(0) \neq 0$ and therefore these equations represent perturbation in non-Minkowski but zero scalar torsion spacetimes. With the condition $f(0) \neq 0$, we may have a non-zero energy-momentum here. This situation is not compatible with a  Minkowski spacetime as defined in Section \ref{sect223}.



\subsection{Perturbed $f(T)$ Teleparallel Field Equations: Minkowski}\label{sect35}

For Minkowski spacetimes as defined in Section \ref{sect223}, since the torsion tensor is zero by definition, the Superpotential terms $S_{(ab) }^{~~~\mu}=S_{[ab]}^{~~~\mu}=0$.  Further,  the Einstein Tensor $\overset{\ \circ}{G}_{ab} =0$ and as argued before $f(0)=0$ so the equations \eqref{306a} and \eqref{307a} reduce as follows:
\begin{subequations}
\begin{eqnarray}
\kappa_{eff}\,\delta \Theta_{(ab)} &\approx & \delta \overset{\ \circ}{G}_{ab}+O\left(|\delta h|^2\right), \label{303a}
\\
0 & \approx & O\left(|\delta h|^2\right). \label{303b}
\end{eqnarray}
\end{subequations}
Equation \eqref{303b} for antisymmetric part of the field equations is identically satisfied while equation \eqref{303a} shows that a variation $\delta \overset{\ \circ}{G}_{ab}$ associated with a perturbation is directly related to a variation of the Energy-Momentum Tensor $\delta \Theta_{\left (ab\right)}$. This shows that the perturbations of Minkowski spacetime as defined in Section \ref{sect223} for $f(T)$ teleparallel gravity follow the perturbative treatments of Minkowski spacetime in GR.

 

\section{Effects of perturbations and the Minkowski spacetime symmetries conditions for stability}\label{sect6}



\subsection{Rotation/boost perturbation in Minkowski Background}\label{sect61}


We would like to know if orthonormal coframe perturbations as expressed by equation \eqref{proper_perturbations} leads to the stability of a pure Minkowski spacetime background. To achieve this goal, we will first test the stability for the rotation/boost perturbations as described in equation \eqref{proper_perturbations}. Secondly, we will also test the stability and its impact for a translated form of this equation \eqref{proper_perturbations}. We will finish by studying effects of the trace, symmetric and antisymmetric parts of perturbation and their respective impacts on torsion and superpotential perturbations.


In fact, the equation \eqref{proper_perturbations} for orthonormal gauge is exactly the rotation/boost perturbation in Minkowski spacetime. The perturbation is described as follows:
\begin{equation}\label{431}
\delta h^a_{\;\;\mu} = \lambda^a_{\;\;b}\,h^b_{\;\;\mu} .
\end{equation}
By substituting equation \eqref{a015b} inside eqs \eqref{303a}, the field equation with the equation \eqref{431} perturbation inside is exactly:
\begin{eqnarray}\label{433}
\kappa_{eff}\,\delta \Theta_{(ab)} &\approx & \left(h_a^{\;\;\mu} h_b^{\;\;\nu}\right)\Bigg[h_k^{~\alpha}\,h^m_{~\mu}\,\delta {\overset{\ \circ}{R}}_{~m\alpha\nu}^{k}-\frac{\eta^{cd}\,\eta_{ef}}{2}\,\left[h_c^{\;\;\sigma}\,h_d^{\;\;\rho}\,h^e_{\;\;\mu}\,h^f_{\;\;\nu}\right]\,h_k^{~\alpha}\,h^m_{~\sigma}\,\delta \overset{\ \circ}{R}_{~m\alpha\rho}^{k}\Bigg]
\nonumber\\
&&\quad+O\left(|\delta h|^2\right),  
\nonumber\\
0 & \approx & O\left(|\delta h|^2\right). 
\end{eqnarray}
Here we obtain perturbed FEs in terms of $\delta \overset{\ \circ}{R}_{~m\alpha\rho}^{k}$ and $h^a_{\;\;\mu}$. If we have that $\delta \overset{\ \circ}{R}_{~m\alpha\nu}^{k} \rightarrow 0$, then we get that $\delta \Theta_{(ab)}\rightarrow 0 $ for equation \eqref{431} as also wanted by GR and TEGR. We might also express equations \eqref{433} in terms of $\lambda^a_{\;\;b}$ and we have shown that pure Minkowski spacetime is stable from the zero curvature criteria as wanted by teleparallel postulates.

From equations \eqref{a004a} and by substituting the equation \eqref{431}, the torsion scalar perturbation $\delta T$ is expressed as:
\small
\begin{align}\label{434}
\delta T =& \frac{1}{4}\Bigg[\left(\partial_{\mu}\left(\lambda^a_{\;\;b}\,h^b_{\;\;\nu}\right)-\partial_{\nu}\left(\lambda^a_{\;\;b}\,h^b_{\;\;\mu}\right)\right) \left(\partial^{\mu}\,h_a^{\;\;\nu}-\partial^{\nu}\,h_a^{\;\;\mu}\right)+\left(\partial_{\mu}\,h^a_{\;\;\nu}-\partial_{\nu}\,h^a_{\;\;\mu}\right)
\nonumber\\
&\times\left(\partial^{\mu}\,\left(\lambda_a^{\;\;b}\,h_b^{\;\;\nu}\right)-\partial^{\nu}\,\left(\lambda_a^{\;\;b}\,h_b^{\;\;\mu}\right)\right)\Bigg]+\frac{1}{2}\Bigg[\left(\partial_{\mu}\left(\lambda^a_{~b}\,h^b_{~\nu} \right)-\partial_{\nu}\left(\lambda^a_{~b}\,h^b_{~\mu} \right)\right)\left(\partial^{\nu}\,h^{\mu}_{~a}-\partial^{\mu}\,h^{\nu}_{~a}\right)
\nonumber\\
&+\left(\partial_{\mu}\,h^a_{\;\;\nu}-\partial_{\nu}\,h^a_{\;\;\mu}\right)\left(\partial^{\nu}\,\left(\lambda^b_{~a}\,h^{\mu}_{~b}\right)-\partial^{\mu}\,\left(\lambda^b_{~a}\,h^{\nu}_{~b}\right)\right)\Bigg]
\nonumber\\
& -\Bigg[\left(\lambda_a^{\;\;b}\,h^{~\nu}_{b}\,\left[\partial_{\mu}\,h^a_{\;\;\nu}-\partial_{\nu}\,h^a_{\;\;\mu}\right]+h^{~\nu}_{a}\,\left[\partial_{\mu}\left(\lambda^a_{~b}\,h^b_{\;\;\nu}\right)-\partial_{\nu}\left(\lambda^a_{~b}\,h^b_{\;\;\mu}\right)\right]\right)\left(h^a_{~\rho}\left(\partial^{\rho}\,h^{\mu}_{~a}-\partial^{\mu}\,h^{\rho}_{~a}\right)\right)
\nonumber\\
&+\left(h^{~\nu}_{a}\,\left[\partial_{\mu}\,h^a_{\;\;\nu}-\partial_{\nu}\,h^a_{\;\;\mu}\right]\right)\left(\lambda^a_{~b} h^b_{~\rho}\left(\partial^{\rho}\,h^{\mu}_{~a}-\partial^{\mu}\,h^{\rho}_{~a}\right)+h^a_{~\rho}\left(\partial^{\rho}\,\left(\lambda^b_{~a}\,h^{\mu}_{~b}\right)-\partial^{\mu}\,\left(\lambda^b_{~a}\,h^{\rho}_{~b}\right)\right)\right)\Bigg]
\nonumber\\
&+O\left(|\delta h|^2\right)
\nonumber\\
&\rightarrow 0 \quad\quad\quad\quad\quad\quad\quad\, \text{for}\; T^a_{~\mu\nu}=\partial_{\mu}\,h^a_{\;\;\nu}-\partial_{\nu}\,h^a_{\;\;\mu }\rightarrow 0.
\end{align}
\normalsize
From the last line of the equation \eqref{434}, we obtain that the condition for $\delta T \rightarrow 0$ is described by the zero torsion tensor criteria $T^a_{~~\mu\nu}=0$ relation as:
\begin{eqnarray}\label{436a}
\partial_{\mu}\left(h^a_{\;\;\nu}\right)\approx  \partial_{\nu}\left(h^a_{\;\;\mu}\right)
\end{eqnarray}

From equation \eqref{a014a} and by substituting the equation \eqref{431}, the superpotential perturbation $\delta S_{ab}^{~~~\mu}$ is expressed as:
\small
\begin{align}\label{435}
\delta S_{ab}^{~~~\mu} =& \Bigg[\frac{1}{2}\left(\partial_a\left(\lambda_b^{~c}\, h_c^{~\mu}\right)- \partial_b\left(\lambda_a^{~c}\,  h_c^{~\mu}\right)\right)+\left( \partial_b\left(\lambda_{~a}^c\,h_{~c}^{\mu}\right)-\partial_a\left(\lambda_{~b}^c\, h_{~c}^{\mu}\right)\right)-\lambda_{~b}^e\,h_{~e}^{\mu} \left(h_{~\rho}^{c} \left[\partial_c\,h_a^{~\rho}-\partial_a\,h_c^{~\rho}\right]\right)
\nonumber\\
&-h_{~b}^{\mu}\left(\lambda^c_{~e}\,h_{~\rho}^{e}\left[\partial_c\,h_a^{~\rho}-\partial_a\,h_c^{~\rho}\right]+h_{~\rho}^{c} \left[\partial_c\,\left(\lambda_a^{~f} h_f^{~\rho}\right)-\partial_a\,\left(\lambda_c^{~f} h_f^{~\rho}\right)\right]\right)
\nonumber\\
&+\lambda_{~a}^e h_{~e}^{\mu}\left(h_{~\rho}^{c} \left[\partial_c\,h_b^{~\rho}-\partial_b\,h_c^{~\rho}\right]\right)+h_{~a}^{\mu}\lambda^c_{~e} h_{~\rho}^{e} \left[\partial_c\,h_b^{~\rho}-\partial_b\,h_c^{~\rho}\right]
\nonumber\\
&+ h_{~a}^{\mu}\,h_{~\rho}^{c} \left[\partial_c\,\left(\lambda_b^{~f} h_f^{~\rho}\right)-\partial_b\,\left(\lambda_c^{~f} h_f^{~\rho}\right)\right]\Bigg]+O\left(|\delta h|^2\right)
\nonumber\\
& \rightarrow \Bigg[\frac{1}{2}\left(\partial_a\left(\lambda_b^{~c}\, h_c^{~\mu}\right)- \partial_b\left(\lambda_a^{~c}\,  h_c^{~\mu}\right)\right)+\left( \partial_b\left(\lambda_{~a}^c\,h_{~c}^{\mu}\right)-\partial_a\left(\lambda_{~b}^c\, h_{~c}^{\mu}\right)\right)
\nonumber\\
&\quad\quad-h_{~b}^{\mu}\,h_{~\rho}^{c} \left[\partial_c\,\left(\lambda_a^{~f} h_f^{~\rho}\right)-\partial_a\,\left(\lambda_c^{~f} h_f^{~\rho}\right)\right]+h_{~a}^{\mu}\,h_{~\rho}^{c} \left[\partial_c\,\left(\lambda_b^{~f} h_f^{~\rho}\right)-\partial_b\,\left(\lambda_c^{~f} h_f^{~\rho}\right)\right]\Bigg]
\nonumber\\
&\quad\quad+O\left(|\delta h|^2\right) \quad\quad \text{by applying equation \eqref{436a} (the zero torsion criteria)}     
\nonumber\\
& \rightarrow  0  \quad\quad\quad\quad\quad\quad\quad\,\, \text{for}\; \delta T^a_{~\mu\nu}=\partial_{\mu}\,\left(\lambda^a_{~c}\, h^c_{\;\;\nu}\right)-\partial_{\nu}\,\left(\lambda^a_{~c}\,h^c_{\;\;\mu }\right)\rightarrow 0.
\end{align}
\normalsize
From the last line of the equation \eqref{435}, we obtain that the condition for $\delta S_{ab}^{~~~\mu}\rightarrow 0$ is also described by the zero perturbed torsion tensor criteria $\delta T^a_{~~\mu\nu}=0$ relation as:
\begin{eqnarray}\label{436}
\partial_{\mu}\left(\lambda^a_{\;\;b}\,h^b_{\;\;\nu}\right) \approx  \partial_{\nu}\left(\lambda^a_{\;\;b}\,h^b_{\;\;\mu}\right).
\end{eqnarray}
The equation \eqref{436} (zero perturbed torsion criteria) is complementary to the equation \eqref{436a} (zero torsion criteria) for obtaining the limit $\delta S_{ab}^{~~~\mu} \rightarrow 0$. We apply the equation \eqref{436a} before applying the equation \eqref{436}. From here, the equations \eqref{436a} and \eqref{436} are the \textbf{two fundamental symmetries conditions for Minkowski spacetime stability}. 


If we set $\delta T \rightarrow 0$ and $\delta S_{ab}^{~~~\mu}\rightarrow 0$ for equations \eqref{306a} and \eqref{307a} for all zero torsion spacetimes, we still respect the equations \eqref{436a} and \eqref{436} as for pure Minkowski spacetimes. Hence the zero torsion tensor and zero perturbed torsion tensor criterions are still valid for all zero torsion spacetimes, Minkowski or not.


Even for constant torsion spacetimes, by always setting $\delta T \rightarrow 0$ and $\delta S_{ab}^{~~~\mu}\rightarrow 0$ inside equations \eqref{306} and \eqref{307}, we respect again equations \eqref{436a} and \eqref{436} as for zero torsion scalar spacetimes. This is another generalization of the Minkowski spacetime result to a most general class of spacetimes as the constant torsion ones.


There are some other consequencies for Minkowski spacetime on proper frame. By applying the null covariant derivative criteria to equation \eqref{431}, we also get as relation:
\begin{eqnarray}\label{432}
0 = \nabla_{\mu}\,\delta h^a_{\;\;\nu} &=& \nabla_{\mu}\,\left(\lambda^a_{\;\;b}\,h^b_{\;\;\nu}\right) = \partial_{\mu}\,\delta h^a_{\;\;\nu}-\Gamma^{\rho}_{\;\;\nu\mu}\,\delta h^a_{\;\;\rho}-\delta \Gamma^{\rho}_{\;\;\nu\mu} h^a_{\;\;\rho}
\nonumber\\
&=&\partial_{\mu}\,\left(\lambda^a_{\;\;b}\,h^b_{\;\;\nu}\right)-\left(h_c^{\;\;\rho}\,\partial_{\mu}\,h^c_{\;\;\nu}\right)\,\left(\lambda^a_{\;\;b}\,h^b_{\;\;\rho}\right)-\delta \Gamma^{\rho}_{\;\;\nu\mu} h^a_{\;\;\rho}
\nonumber\\
& &\Rightarrow \delta \Gamma^{\rho}_{\;\;\nu\mu} = h_a^{\;\;\rho}\left[\partial_{\mu}\left(\lambda^a_{\;\;b}\,h^b_{\;\;\nu}\right)-\left(h_c^{\;\;\sigma}\,\partial_{\mu}\,h^c_{\;\;\nu}\right) \left(\lambda^a_{\;\;b}\,h^b_{\;\;\sigma}\right)\right].
\end{eqnarray}
where $\Gamma^{\rho}_{\;\;\nu\mu}=h_c^{\;\;\rho}\,\partial_{\mu}\,h^c_{\;\;\nu}$ is the Weitzenbock connection for a proper frame. For trivial coframes as $h^a_{\;\;\mu}=\delta^a_{\;\;\mu}=Diag[1,1,1,1]$, the equation \eqref{432} becomes:
\begin{eqnarray}\label{432a}
\delta \Gamma^{\rho}_{\;\;\nu\mu} = h_a^{\;\;\rho}\left[\partial_{\mu}\left(\lambda^a_{\;\;b}\,h^b_{\;\;\mu}\right)\right]=\delta_a^{\;\;\rho}\,\partial_{\mu}\left(\lambda^a_{\;\;b}\right)\,\delta^b_{\;\;\mu}.
\end{eqnarray}
In the next subsection, we will study the effect of a translation applied to the perturbation described by equation \eqref{431} on equations \eqref{432} and \eqref{432a}. The goal is to know the effects of perturbations on the Weitzenbock connection and its perturbation.

We can now see by the equations (\ref{433}) to (\ref{432a}) the effect of the perturbation described by equation (\ref{431}) maintaining the proper frame and respecting the $GL(4,\mathbb{R})$ invariance transformation. In addition, the equations (\ref{436a}) and (\ref{436}) gives the Minkowski spacetime stability conditions on proper frames for the perturbation described by equation (\ref{431}) \cite{stabilitychristo,stability2,stability3,stability4}. 


\subsection{General Linear perturbation in Minkowski Background}\label{sect62}


A more general perturbation scheme requires one to deal with the following general linear perturbation:
\begin{equation}\label{441}
\delta h^a_{\;\;\mu} = \lambda^a_{\;\;b}\,h^b_{\;\;\mu}+\epsilon^a_{\;\;\mu} ,
\end{equation}
where $|\lambda^a_{\;\;b}|, |\epsilon^a_{\;\;\mu}|\ll 1$. We have here the transformation described by equation \eqref{431} superposed with a translation in Minkowski tangent space. 


For equation \eqref{441} perturbation, the equations \eqref{433} becomes as:
\begin{eqnarray}\label{443}
\kappa_{eff}\,\delta \Theta_{(ab)} &\approx & \left(h_a^{\;\;\mu} h_b^{\;\;\nu}\right)\Bigg[h_k^{~\alpha}\,h^m_{~\mu}\,\delta {\overset{\ \circ}{R}}^k_{~m\alpha\nu}-\frac{\eta^{cd}\,\eta_{ef}}{2}\,\left[h_c^{\;\;\sigma}\,h_d^{\;\;\rho}\,h^e_{\;\;\mu}\,h^f_{\;\;\nu}\right]\,h_k^{~\alpha}\,h^m_{~\sigma}\,\delta {\overset{\ \circ}{R}}^k_{~m\alpha\rho}\Bigg]
\nonumber\\
&&\quad+O\left(|\delta h|^2\right),  
\nonumber\\
0 & \approx & O\left(|\delta h|^2\right). 
\end{eqnarray}
Here again we obtain perturbed FEs in terms of $\delta {\overset{\ \circ}{R}}^k_{~m\alpha\rho}$ and $h^a_{\;\;\mu}$. As for equation \eqref{433}, $\delta  {\overset{\ \circ}{R}}^k_{~m\alpha\nu} \rightarrow 0$, then we still get that $\delta \Theta_{(ab)}\rightarrow 0 $ for equation \eqref{441} as also wanted by GR and TEGR  \cite{stabilitychristo,stability2,stability3,stability4}. We might express equations \eqref{443} in terms of $\lambda^a_{\;\;b}$ and $\epsilon^a_{\;\;\mu}$. Here again, we have shown that pure Minkowski spacetime is still stable from the zero curvature criteria as wanted by teleparallel postulates.


From equations \eqref{a004a} and by substituting the equation \eqref{441}, the torsion scalar perturbation $\delta T$ is expressed as:
\small
\begin{align}\label{444}
\delta T =& \frac{1}{4}\Bigg[\left(\partial_{\mu}\left(\lambda^a_{\;\;b}\,h^b_{\;\;\nu}+\epsilon^a_{\;\;\nu}\right)-\partial_{\nu}\left(\lambda^a_{\;\;b}\,h^b_{\;\;\mu}+\epsilon^a_{\;\;\mu}\right)\right) \left(\partial^{\mu}\,h_a^{\;\;\nu}-\partial^{\nu}\,h_a^{\;\;\mu}\right)+\left(\partial_{\mu}\,h^a_{\;\;\nu}-\partial_{\nu}\,h^a_{\;\;\mu}\right)
\nonumber\\
&\times\left(\partial^{\mu}\,\left(\lambda_a^{\;\;b}\,h_b^{\;\;\nu}+\epsilon_a^{\;\;\nu}\right)-\partial^{\nu}\,\left(\lambda_a^{\;\;b}\,h_b^{\;\;\mu}+\epsilon_a^{\;\;\mu}\right)\right)\Bigg]
\nonumber\\
&+\frac{1}{2}\Bigg[\left(\partial_{\mu}\left(\lambda^a_{~b}\,h^b_{~\nu} +\epsilon^a_{\;\;\nu}\right)-\partial_{\nu}\left(\lambda^a_{~b}\,h^b_{~\mu}+\epsilon^a_{\;\;\mu} \right)\right)\left(\partial^{\nu}\,h^{\mu}_{~a}-\partial^{\mu}\,h^{\nu}_{~a}\right)
\nonumber\\
&+\left(\partial_{\mu}\,h^a_{\;\;\nu}-\partial_{\nu}\,h^a_{\;\;\mu}\right)\left(\partial^{\nu}\,\left(\lambda^b_{~a}\,h^{\mu}_{~b}+\epsilon^{\mu}_{~a}\right)-\partial^{\mu}\,\left(\lambda^b_{~a}\,h^{\nu}_{~b}+\epsilon^{\nu}_{~a}\right)\right)\Bigg]
\nonumber\\
& -\Bigg[\left(\left(\lambda_a^{\;\;b}\,h^{~\nu}_{b}+\epsilon^{~\nu}_{a}\right)\,\left[\partial_{\mu}\,h^a_{\;\;\nu}-\partial_{\nu}\,h^a_{\;\;\mu}\right]+h^{~\nu}_{a}\,\left[\partial_{\mu}\left(\lambda^a_{~b}\,h^b_{\;\;\nu}+\epsilon^a_{\;\;\nu}\right)-\partial_{\nu}\left(\lambda^a_{~b}\,h^b_{\;\;\mu}+\epsilon^a_{\;\;\mu}\right)\right]\right)
\nonumber\\
&\times\left(h^a_{~\rho}\left(\partial^{\rho}\,h^{\mu}_{~a}-\partial^{\mu}\,h^{\rho}_{~a}\right)\right)+\left(h^{~\nu}_{a}\,\left[\partial_{\mu}\,h^a_{\;\;\nu}-\partial_{\nu}\,h^a_{\;\;\mu}\right]\right)
\nonumber\\
&\times\left(\left(\lambda^a_{~b} h^b_{~\rho}+\epsilon^a_{~\rho}\right)\left(\partial^{\rho}\,h^{\mu}_{~a}-\partial^{\mu}\,h^{\rho}_{~a}\right)+h^a_{~\rho}\left(\partial^{\rho}\,\left(\lambda^b_{~a}\,h^{\mu}_{~b}+\epsilon^{\mu}_{~a}\right)-\partial^{\mu}\,\left(\lambda^b_{~a}\,h^{\rho}_{~b}+\epsilon^{\rho}_{~a}\right)\right)\right)\Bigg]
\nonumber\\
&+O\left(|\delta h|^2\right)
\nonumber\\
&\rightarrow 0 \quad\quad\quad\quad\quad\quad\quad\, \text{for}\; T^a_{~\mu\nu}=\partial_{\mu}\,h^a_{\;\;\nu}-\partial_{\nu}\,h^a_{\;\;\mu }\rightarrow 0.
\end{align}
\normalsize
The condition for $\delta T \rightarrow 0$ is still described by the equation \eqref{436a} for the zero torsion tensor criteria $T^a_{~~\mu\nu}=0$.


From equation \eqref{a014a} and by substituting the equation \eqref{441}, the superpotential perturbation $\delta S_{ab}^{~~~\mu}$ is expressed as:
\small
\begin{align}\label{445}
\delta S_{ab}^{~~~\mu} =& \Bigg[\frac{1}{2}\left(\partial_a\left(\lambda_b^{~c}\, h_c^{~\mu}+ \epsilon_b^{~\mu}\right)- \partial_b\left(\lambda_a^{~c}\,  h_c^{~\mu}+\epsilon_a^{~\mu}\right)\right)+\left( \partial_b\left(\lambda_{~a}^c\,h_{~c}^{\mu}+\epsilon_{~a}^{\mu}\right)-\partial_a\left(\lambda_{~b}^c\, h_{~c}^{\mu}+\epsilon_{~b}^{\mu}\right)\right)
\nonumber\\
&-\left(\lambda_{~b}^e\,h_{~e}^{\mu}+\epsilon_{~b}^{\mu}\right) \left(h_{~\rho}^{c} \left[\partial_c\,h_a^{~\rho}-\partial_a\,h_c^{~\rho}\right]\right)
\nonumber\\
&-h_{~b}^{\mu}\left(\left(\lambda^c_{~e}\,h_{~\rho}^{e}+\epsilon_{~\rho}^{c}\right)\left[\partial_c\,h_a^{~\rho}-\partial_a\,h_c^{~\rho}\right]+h_{~\rho}^{c} \left[\partial_c\,\left(\lambda_a^{~f} h_f^{~\rho}+\epsilon_a^{~\rho}\right)-\partial_a\,\left(\lambda_c^{~f} h_f^{~\rho}+\epsilon_c^{~\rho}\right)\right]\right)
\nonumber\\
&+\left(\lambda_{~a}^e h_{~e}^{\mu}+\epsilon_{~a}^{\mu}\right)\left(h_{~\rho}^{c} \left[\partial_c\,h_b^{~\rho}-\partial_b\,h_c^{~\rho}\right]\right)+h_{~a}^{\mu}\left(\lambda^c_{~e} h_{~\rho}^{e}+\epsilon_{~\rho}^{c}\right) \left[\partial_c\,h_b^{~\rho}-\partial_b\,h_c^{~\rho}\right]
\nonumber\\
&+ h_{~a}^{\mu}\,h_{~\rho}^{c} \left[\partial_c\,\left(\lambda_b^{~f} h_f^{~\rho}+\epsilon_b^{~\rho}\right)-\partial_b\,\left(\lambda_c^{~f} h_f^{~\rho}+\epsilon_c^{~\rho}\right)\right]\Bigg]+O\left(|\delta h|^2\right)
\nonumber\\
& \rightarrow \Bigg[\frac{1}{2}\left(\partial_a\left(\lambda_b^{~c}\, h_c^{~\mu}+ \epsilon_b^{~\mu}\right)- \partial_b\left(\lambda_a^{~c}\,  h_c^{~\mu}+\epsilon_a^{~\mu}\right)\right)+\left( \partial_b\left(\lambda_{~a}^c\,h_{~c}^{\mu}+\epsilon_{~a}^{\mu}\right)-\partial_a\left(\lambda_{~b}^c\, h_{~c}^{\mu}+\epsilon_{~b}^{\mu}\right)\right)
\nonumber\\
&\quad\quad-h_{~b}^{\mu}\,h_{~\rho}^{c} \left[\partial_c\,\left(\lambda_a^{~f} h_f^{~\rho}+\epsilon_a^{~\rho}\right)-\partial_a\,\left(\lambda_c^{~f} h_f^{~\rho}+\epsilon_c^{~\rho}\right)\right]
\nonumber\\
&\quad\quad+h_{~a}^{\mu}\,h_{~\rho}^{c} \left[\partial_c\,\left(\lambda_b^{~f} h_f^{~\rho}+\epsilon_b^{~\rho}\right)-\partial_b\,\left(\lambda_c^{~f} h_f^{~\rho}+\epsilon_c^{~\rho}\right)\right]\Bigg]+O\left(|\delta h|^2\right)
\nonumber\\
& \quad\quad\quad\quad\quad\, \text{by applying equation \eqref{436a} (i.e. the zero torsion criteria)}    , 
\nonumber\\
& \rightarrow  0  \quad\quad\quad\, \text{if we respect}\; \partial_{a}\epsilon_b^{\;\;\mu}=\partial_{b}\epsilon_a^{\;\;\mu}=0\; 
\nonumber\\
&\quad\quad\quad\quad\quad\,\;\text{(constant translation condition for equation \eqref{441})} 
\nonumber\\
&\quad\quad\quad\quad\quad\,\;\text{and after we apply the equation \eqref{436} criteria.}
\end{align}
\normalsize
The condition for $\delta S_{ab}^{~~~\mu}\rightarrow 0$ is still described by the equation \eqref{436} for zero perturbed torsion tensor criteria $\delta T^a_{~~\mu\nu}=0$ only if the constant translation criteria is respected as:
\begin{eqnarray}\label{446}
\partial_{\mu}\epsilon^a_{\;\;\nu}=\partial_{\nu}\epsilon^a_{\;\;\nu}=0
\end{eqnarray}
Hence for equation \eqref{441} perturbation, we still respect the equations \eqref{436a} and \eqref{436} as the two first symmetries conditions for Minkowski spacetime stability, but we must also respect the equation \eqref{446} before equation \eqref{436}. A simple translation does not affect these equations \eqref{436a} and \eqref{436} only if we respect equation \eqref{446}, the translation term $\epsilon^a_{\;\;\nu}$ must be constant inside equation \eqref{441}. This constant translation criteria as expressed by equation \eqref{446} is a \textbf{third symmetry condition for Minkowski spacetime stability}. 


As for equations \eqref{432} and \eqref{432a}, we apply the null covariant derivative criteria to equation \eqref{441} and we get as relation:
\begin{eqnarray}\label{442}
0 &=&\partial_{\mu}\,\left(\lambda^a_{\;\;b}\,h^b_{\;\;\nu}+\epsilon^a_{\;\;\nu}\right)-\left(h_c^{\;\;\rho}\,\partial_{\mu}\,h^c_{\;\;\nu}\right)\,\left(\lambda^a_{\;\;b}\,h^b_{\;\;\rho}+\epsilon^a_{\;\;\rho}\right)-\delta \Gamma^{\rho}_{\;\;\nu\mu} h^a_{\;\;\rho}
\nonumber\\
& &\Rightarrow \delta \Gamma^{\rho}_{\;\;\nu\mu} = h_a^{\;\;\rho}\left[\partial_{\mu}\left(\lambda^a_{\;\;b}\,h^b_{\;\;\nu}\right)-\left(h_c^{\;\;\sigma}\,\partial_{\mu}\,h^c_{\;\;\nu}\right) \left(\lambda^a_{\;\;b}\,h^b_{\;\;\sigma}+\epsilon^a_{\;\;\sigma}\right)\right].
\end{eqnarray}
where $\Gamma^{\sigma}_{\;\;\nu\mu}=h_c^{\;\;\sigma}\,\partial_{\mu}\,h^c_{\;\;\nu}$ is the Weitzenbock connection for a proper frame and $\partial_{\mu}\,\epsilon^a_{\;\;\nu}=0$ because constant translation. The equation \eqref{442} is slightly different from equation \eqref{432} by the term $-\left(h_c^{\;\;\sigma}\,\partial_{\mu}\,h^c_{\;\;\nu}\right)\,\epsilon^a_{\;\;\sigma}$. For non-trivial coframes (i.e. $\partial_{\mu}\,h^c_{\;\;\nu}\neq 0$), the equation \eqref{442} is not invariant under the equation \eqref{446}. For trivial coframes (i.e. $\partial_{\mu}\,h^c_{\;\;\nu}= 0$), the equation \eqref{442} becomes exactly the equation \eqref{432a} as for the perturbation described by equation \eqref{431}. From this result, we now respect the constant coframe criteria as (or null Weitzenbock connection $\Gamma^{\rho}_{\;\;\nu\mu}=0$ criteria):
\begin{eqnarray}\label{447}
\partial_{\mu}\,h^c_{\;\;\nu}=0.
\end{eqnarray}
With the equation \eqref{447}, we also satisfies the invariance under the equation \eqref{446}, the constant translation criteria for the Weitzenbock connection perturbation. Hence, the equation \eqref{432} and \eqref{442} show that the Weitzenbock connection perturbation $\delta \Gamma^{\rho}_{\;\;\nu\mu}$ is invariant only if we respect the equation \eqref{447}, the constant coframe criteria. This criteria as expressed by equation \eqref{447} is a \textbf{fourth symmetry condition for Minkowski spacetime stability}.


Now, the equations (\ref{443}) to (\ref{447}) generalize the equations  (\ref{433}) to (\ref{432a}) by applying a constant translation $\epsilon^a_{\;\;\nu}$ to the linear transformation described by equation \eqref{431} which maintains the proper frame and the invariance under $GL(4,\mathbb{R})$ transformation. By respecting equation \eqref{446}, the constant translation criteria, we still respect equations \eqref{436a} and \eqref{436} for the equation \eqref{441} and this generalization shows that Minkowski spacetime and all zero torsion spacetimes are stable everytimes \cite{stabilitychristo,stability2,stability3,stability4}. However, the equations \eqref{432} and \eqref{442} both giving the equation \eqref{432a} show that the Weitzenbock connection perturbation $\delta \Gamma^{\rho}_{\;\;\nu\mu}$ is invariant only if we work with constant or trivial coframes respecting the equation \eqref{447}.



\subsection{Perturbations on Trivial Coframes by each part of perturbation}\label{sect63}


Before properly dealing with more complex cases of coframes, it is imperative to deal with perturbations on the trivial coframe. This coframe is defined as follows:
\begin{equation}\label{667}
h^a_{\;\;\mu}=\delta^a_{\;\;\mu}=Diag \left[1,\,1,\,1,\,1\right] .
\end{equation}
The coframe described by equation \eqref{667} is defined in the orthonormal gauge. This equation \eqref{667} respects the equations \eqref{447}, the 4th symmetry condition for Minkowski spacetime stability. From there, we will study the following general perturbations which will be applied to equation (\ref{667}) in terms of $\lambda^a_{~b}$ and respecting equations \eqref{436a}, \eqref{436} and if necessary equation \eqref{446} as:

\begin{enumerate}
	\item Trace:
	\begin{equation}\label{697}
	\left(\lambda^a_{~b}\right)_{Trace}=\lambda = Trace \left[Diag \left[a_{00},\,a_{11},\,a_{22},\,a_{33}\right]\right]=a_{00}+a_{11}+a_{22}+a_{33}.
	\end{equation}
   The equation \eqref{431} will be exactly $\left(\delta h^a_{\;\;\mu}\right)_{Trace} = \frac{\lambda}{4}\delta^a_{\;\;\mu}$ and by setting $h^a_{\;\;\mu}=\delta^a_{\;\;\mu}$, the equations \eqref{434} and \eqref{435} are:
	\begin{align}\label{634}
    \delta T \approx O\left(|\delta h|^2\right)\rightarrow 0, 
    \end{align}
    which respects equation \eqref{436a} and
    \small
    \begin{align}\label{635}
    \delta S_{ab}^{~~~\mu} & \rightarrow \Bigg[\frac{1}{8}\left(\partial_a\left(\lambda\, \delta_b^{~\mu}\right)- \partial_b\left(\lambda\,\delta_a^{~\mu}\right)\right)+\frac{1}{4}\left( \partial_b\left(\lambda\,\delta_{~a}^{\mu}\right)-\partial_a\left(\lambda\,\delta_{~b}^c\, h_{~c}^{\mu}\right)\right)
\nonumber\\
&\quad\quad-\frac{1}{4}\delta_{~b}^{\mu}\,\delta_{~\rho}^{c} \left[\partial_c\,\left(\lambda\,\delta_a^{~\rho}\right)-\partial_a\,\left(\lambda\,\delta_c^{~\rho}\right)\right]+\frac{1}{4}\,\delta_{~a}^{\mu}\,\delta_{~\rho}^{c} \left[\partial_c\,\left(\lambda\,\delta_b^{~\rho}\right)-\partial_b\,\left(\lambda\, \delta_c^{~\rho}\right)\right]\Bigg]
\nonumber\\
&\quad\quad+O\left(|\delta h|^2\right) \quad\quad \text{by applying equation \eqref{436a} (the zero torsion criteria)}     
\nonumber\\
& \rightarrow  0  \quad\quad\quad\quad\quad\quad\quad\, \text{for}\; \delta T^a_{~\mu\nu}=\partial_{\mu}\,\left(\lambda\right)\,\delta^a_{\;\;\nu}-\partial_{\nu}\,\left(\lambda\right)\,\delta^a_{\;\;\mu }\rightarrow 0.
    \end{align}
    \normalsize
    The equation \eqref{436} will be expressed as:
    \begin{eqnarray}\label{636}
    \partial_{\mu}\,\left(\lambda\right)\,\delta^a_{\;\;\nu}\approx \partial_{\nu}\,\left(\lambda\right)\,\delta^a_{\;\;\mu }.
    \end{eqnarray}
    
   

	\item Symmetric (null diagonal components):
	\small
	\begin{equation}\label{697a}
	\left(\lambda^a_{~b}\right)_{Sym}=\tilde{\lambda}^a_{~b} = \left[\begin{array}{cccc}
	0 & b_{10} & b_{20} & b_{30} \\
	b_{10} & 0 & b_{12} & b_{13} \\
	b_{20} &  b_{12} & 0 & b_{23} \\
	b_{30} & b_{13} & b_{23} & 0
	\end{array}
	\right].
	\end{equation}
	\normalsize
	The equation \eqref{431} will be exactly $\left(\delta h^a_{\;\;\mu}\right)_{Sym} = \tilde{\lambda}^a_{\;\;b}\,\delta^b_{\;\;\mu}$ and by setting $h^a_{\;\;\mu}=\delta^a_{\;\;\mu}$, the equation \eqref{434} is still expressed by the equation \eqref{634} respecting the equation \eqref{436a} and equation \eqref{435} is:
    \small    
    \begin{align}\label{635a}
    \delta S_{ab}^{~~~\mu} =& \Bigg[\frac{1}{2}\left(\partial_a\left(\tilde{\lambda}_b^{~c}\, \delta_c^{~\mu}\right)- \partial_b\left(\tilde{\lambda}_a^{~c}\,  \delta_c^{~\mu}\right)\right)+\left( \partial_b\left(\tilde{\lambda}_{~a}^c\,\delta_{~c}^{\mu}\right)-\partial_a\left(\tilde{\lambda}_{~b}^c\, \delta_{~c}^{\mu}\right)\right)
\nonumber\\
&\quad\quad-\delta_{~b}^{\mu}\,\delta_{~\rho}^{c} \left[\partial_c\,\left(\tilde{\lambda}_a^{~f} \delta_f^{~\rho}\right)-\partial_a\,\left(\tilde{\lambda}_c^{~f} \delta_f^{~\rho}\right)\right]+\delta_{~a}^{\mu}\,\delta_{~\rho}^{c} \left[\partial_c\,\left(\tilde{\lambda}_b^{~f} \delta_f^{~\rho}\right)-\partial_b\,\left(\tilde{\lambda}_c^{~f} \delta_f^{~\rho}\right)\right]\Bigg]
\nonumber\\
&\quad\quad+O\left(|\delta h|^2\right) \quad\quad \text{by applying equation \eqref{436a} (the zero torsion criteria)}     
\nonumber\\
& \rightarrow  0  \quad\quad\quad\quad\quad\quad\quad\, \text{for}\; \delta T^a_{~\mu\nu}=\partial_{\mu}\,\left(\lambda^a_{~c}\right)\delta^c_{\;\;\nu}-\partial_{\nu}\,\left(\lambda^a_{~c}\right),\delta^c_{\;\;\mu }\rightarrow 0.
    \end{align}
    \normalsize
The equation \eqref{436} will be expressed as:
    \begin{eqnarray}\label{636a}
    \partial_{\mu}\,\left(\lambda^a_{~c}\right)\delta^c_{\;\;\nu}\approx \partial_{\nu}\,\left(\lambda^a_{~c}\right),\delta^c_{\;\;\mu }.
    \end{eqnarray}		
	
	\item Antisymmetric (null diagonal components):
	\small
	\begin{equation}\label{697b}
	\left(\lambda^a_{~b}\right)_{AntiSym}=\bar{\lambda}^a_{~b}  = \left[\begin{array}{cccc}
	0 & b_{10} & b_{20} & b_{30} \\
	-b_{10} & 0 & b_{12} & b_{13} \\
	-b_{20} & -b_{12} & 0 & b_{23} \\
	-b_{30} & -b_{13} & -b_{23} & 0
	\end{array}
	\right].
	\end{equation}
	\normalsize
	The equation \eqref{431} will be exactly $\left(\delta h^a_{\;\;\mu}\right)_{AntiSym} = \bar{\lambda}^a_{\;\;b}\,\delta^b_{\;\;\mu}$ and by setting $h^a_{\;\;\mu}=\delta^a_{\;\;\mu}$, the equation \eqref{434} is still expressed by the equation \eqref{634} respecting the equation \eqref{436a} and equation \eqref{435} is:
    \small    
    \begin{align}\label{635b}
    \delta S_{ab}^{~~~\mu} =& \Bigg[\frac{1}{2}\left(\partial_a\left(\bar{\lambda}_b^{~c}\, \delta_c^{~\mu}\right)- \partial_b\left(\bar{\lambda}_a^{~c}\,  \delta_c^{~\mu}\right)\right)+\left( \partial_b\left(\bar{\lambda}_{~a}^c\,\delta_{~c}^{\mu}\right)-\partial_a\left(\bar{\lambda}_{~b}^c\, \delta_{~c}^{\mu}\right)\right)
\nonumber\\
&\quad\quad-\delta_{~b}^{\mu}\,\delta_{~\rho}^{c} \left[\partial_c\,\left(\bar{\lambda}_a^{~f} \delta_f^{~\rho}\right)-\partial_a\,\left(\bar{\lambda}_c^{~f} \delta_f^{~\rho}\right)\right]+\delta_{~a}^{\mu}\,\delta_{~\rho}^{c} \left[\partial_c\,\left(\bar{\lambda}_b^{~f} \delta_f^{~\rho}\right)-\partial_b\,\left(\bar{\lambda}_c^{~f} \delta_f^{~\rho}\right)\right]\Bigg]
\nonumber\\
&\quad\quad+O\left(|\delta h|^2\right) \quad\quad \text{by applying equation \eqref{436a} (the zero torsion criteria)}     
\nonumber\\
& \rightarrow  0  \quad\quad\quad\quad\quad\quad\quad\, \text{for}\; \delta T^a_{~\mu\nu}=\partial_{\mu}\,\left(\bar{\lambda}^a_{~c}\right)\, \delta^c_{\;\;\nu}-\partial_{\nu}\,\left(\bar{\lambda}^a_{~c}\right)\,\delta^c_{\;\;\mu } \rightarrow 0.
    \end{align}
    \normalsize
 The equation \eqref{436} will be expressed as:
    \begin{eqnarray}\label{636b}
    \partial_{\mu}\,\left(\bar{\lambda}^a_{~c}\right)\, \delta^c_{\;\;\nu}\approx \partial_{\nu}\,\left(\bar{\lambda}^a_{~c}\right)\,\delta^c_{\;\;\mu }.
    \end{eqnarray}	   
	
	\item Mixed (general case: a combination of the three previous sorts):
	\begin{eqnarray}\label{697c}
	\left(\lambda^a_{~b}\right)_{Mixed}=\lambda^a_{~b} &=&\frac{\delta^a_{~b}}{4}\,\left(\lambda^a_{~b}\right)_{Trace}+\left(\lambda^a_{~b}\right)_{Sym}+\left( \lambda^a_{~b}\right)_{AntiSym} ,
	\nonumber\\
	&=& \frac{\lambda}{4}\delta^a_{~b} + \tilde{\lambda}^a_{~b} + \bar{\lambda}^a_{~b}.
	\end{eqnarray}
The equation \eqref{431} will be exactly $\left(\delta h^a_{\;\;\mu}\right)_{Mixed} = \lambda^a_{\;\;b}\,\delta^b_{\;\;\mu}$ and we obtain exactly the equations \eqref{434} and \eqref{435} by respecting equations \eqref{436a} and \eqref{436} by superposition. In equation \eqref{697c}, the first two terms (Trace and Symmetric terms) represent the symmetric part of $\left(\lambda^a_{~b}\right)_ {Mixed}$ and the last term (Antisymmetric term) represents the Antisymmetric part of $\left(\lambda^a_{~b}\right)_{Mixed}$. For every cases, we satisfy the equations \eqref{436a}, \eqref{436}, \eqref{446} and \eqref{447} in supplement of the energy-momentum stability from $\delta \overset{\ \circ}{R}^k_{~m\alpha\nu} \rightarrow 0$ leading to $\delta \Theta_{(ab)}\rightarrow 0 $ \cite{stabilitychristo,stability2,stability3,stability4}.
\end{enumerate}	



For the trivial coframe cases expressed by equation (\ref{667}), we verify the energy-momentum stability by equations \eqref{433} and the four other symmetries condition stated by equations \eqref{436a}, \eqref{436}, \eqref{446} and \eqref{447} are all satisfied. The Minkowski spacetime is stable with these four symmetries conditions. From these considerations for the pure Minkowski spacetime, we have showed that $\delta \Theta_{(ab)} \rightarrow 0$ by equations \eqref{433} and \eqref{443} when all perturbed quantities goes to zero. From this, we must absolutely have $\Theta_{(ab)}=0$ when we are in the pure vacuum: the full absence of a gravitational source.



%{\color{blue}
%These results for the Energy-Momentum Tensors also confirm that we could have the non-zero Torsion scalars for each perturbation too. In order to simplify the calculations, we have imposed the null Superpotential condition $S_a^{\;\;\mu\nu}=0$ in these MAPLE codes. This addition makes it possible to shorten the calculation times and to have simpler results. On the other hand, as we can see in eq. (\ref{213}), the null Superpotential condition imposes additional conditions on the Torsion Tensor and therefore on the perturbations themselves.

%}


\section{Discussion and Conclusion}\label{sect8}



We obtained in section \ref{sect3} the perturbed field equations (perturbed FEs) in terms of the perturbed torsion $\delta T$ and perturbed superpotential $\delta S_{ab}^{~~~\mu}$. These two quantities are themselves dependent on the coframe perturbation $\delta h^a_{\;\;\mu}$. These perturbed field equations make it possible to relate these perturbed quantities to the perturbation of the Energy-Momentum $\delta \Theta_{\left(ab\right)}$. This is analogous to the field equations for the non-perturbed quantities and how they relate to the physical quantities in the Energy-Momentum $\Theta_{\left(ab\right)}$. We have obtained the general forms of the perturbed field equations for any $f\left(T\right)$ theory with a particular emphasis on constant torsion scalar, null torsion scalar and pure Minkowski spacetimes. We also obtained the orthonormal coframe perturbation $\delta h^a_{~\mu}$ respecting the invariance under $GL(4,\mathbb{R})$ transformation group: it is a rotation/boost perturbation. This orthonormal coframe perturbation as defined in equation \eqref{proper_perturbations} respects the teleparallel postulates. The perturbed field equations will still remain valid for the coframe perturbation in every gauge, not only orthonormal.


In section \ref{sect6}, we look at field equations \eqref{433} and \eqref{443} when the curvature perturbation criteria $\delta {\overset{\ \circ}{R}}^k_{~m\alpha\nu}$ goes to zero, We observe that the energy-momentum perturbation $\delta \Theta_{\left(ab\right)}$ also goes to zero as in General Relativity. In General Relativity, it is known that a curvature perturbation leads to a energy-momentum perturbation. We prove that the same thing occurs for the Teleparallel Minkowski spacetime with equations \eqref{433} and \eqref{443}. 


Then, we got by the null torsion tensor and the null perturbed torsion tensor criteria as defined by equations \eqref{436a} and \eqref{436} that the torsion scalar perturbation $\delta T$ and superpotential perturbation $\delta S_{ab}^{~~~\mu}$ go to zero for pure Minkowski spacetime when we use the equation \eqref{431} perturbation (boost/rotation perturbation). These equations \eqref{436a} and \eqref{436} are the two first fundamental Minkowski spacetime stability conditions on proper frames. However, if we use the more general linear perturbation as defined by equation \eqref{441}, we need to respect the constant translation criteria as defined by equation \eqref{446} in order for the superpotential perturbation $\delta S_{ab}^{~~~\mu}$ to go to zero. This is a third Minkowski spacetime stability condition for proper frames to respect for equation \eqref{441} perturbation. In this way, by respecting equation \eqref{446} we then respect the equations \eqref{436a} and \eqref{436} as for equation \eqref{431} perturbation.


Another consequence from equation \eqref{441} perturbation is about the Weitzenbock connection perturbation $\delta \Gamma^{\rho}_{\;\;\nu\mu}$. The equations \eqref{432} and \eqref{442} have shown that we need to respect the constant coframe criteria as defined by equation \eqref{447}. This equation \eqref{447} is a fourth Minkowski spacetime stability conditions for proper frames to respect for equation \eqref{441} perturbation allowing the invariance for Weitzenbock connection perturbation.


To generalize, these steps applied for the Minkowski spacetime giving these stability criteria can also be applied for null torsion scalar spacetimes as well as constant torsion scalar spacetimes. Indeed, with the analysis made in sections \ref{sect61} and \ref{sect62} and the stability criteria obtained for the Minkowski spacetime, the equations \eqref{306a} and \eqref{307a} for the null torsion scalar spacetimes make it possible to generalize these treatments and to obtain in the end the same stability criteria which are the equations \eqref{436a}, \eqref{436} and if necessary the equations \eqref{446} and \eqref{447}. This is also the case for the constant torsion scalar spacetimes described by the equations \eqref{306} and \eqref{307} if we still take the limits where $\delta T \rightarrow 0$ and $\delta S_{ab}^{~~~\mu} \rightarrow 0$ as for Minkowski and null torsion scalar spacetimes.


Afterward, we could also apply this approach to perturbations in Teleparallel Gravity for spacetimes other than constant torsion scalar. We would like to know if the null torsion tensor and null perturbed torsion tensor criteria and if applicable the constant translation and the constant coframe criteria are also existing for other spacetimes as stability conditions. We are interested to test these stability conditions for spacetimes with some spherical symmetries \cite{sspaper,tdsprep}. 


From there, we can use in Teleparallel Gravity and the results from the perturbative approach to deal with more concrete problems of the astrophysical and/or cosmological type. We may also test the stability of the cosmological models by using the perturbation analysis tools developed in this paper. We know much more about the stability conditions for Minkowski and related spacetimes to better study these applications in these areas of subjects. There are some works which are coming much later about these subjects. We would then unequivocally show the full power of Teleparallel Gravity as well as theories based on spacetime torsion in general.


%%-------------------------------------------------------------------------
%%----------------------   ACKNOWLEDGEMENTS    ----------------------------
%%-------------------------------------------------------------------------

\section*{Acknowledgments}


RvdH is supported by the Natural Sciences and Engineering Research Council of Canada and the Dr. W.F. James Chair of Studies in the Pure and Applied Sciences at St.F.X. AL is supported by an AARMS fellowship. 




%%-------------------------------------------------------------------------
%%----------------------   BIBLIO        ----------------------------------
%%-------------------------------------------------------------------------


\bibliographystyle{unsrt}
\bibliography{referencesminkow}

%%-------------------------------------------------------------------------
%%----------------------   APPENDICES    ----------------------------------
%%-------------------------------------------------------------------------

\appendix

\section{Perturbed Physical Quantities in Teleparallel Theories}\label{appena}




To complete the analysis of Teleparallel theories and geometries, we want to perturb various physical quantities that may be involved. As explained in section \ref{sect_perturbed_frames} we are able to always consider perturbations of the co-frame only within a proper orthonormal gauge. 
\begin{subequations}
\begin{eqnarray}
\hat{h}'^a_{~\mu}&=& h^a_{~\mu}+\delta h^a_{~\mu},\\
\hat{\omega}'^a_{~\,b\mu}&=& 0,\\
\hat{g}'_{ab} &=& \eta_{ab},
\end{eqnarray}
\end{subequations}
where $\delta h^a=\delta h^a_{~\mu}\,dx^{\mu}=\lambda^a_{~b}h^b$.  Here we apply the coframe perturbations to the main physical and geometrical quantities involved in the Teleparallel Gravity.

\begin{enumerate}

\item The Inverse coframe perturbation $\delta h_a^{~\mu}$: 
\begin{eqnarray}\label{a000}
h_a^{~\mu}+\delta h_a^{~\mu} &=& h_a^{~\mu} + \left[\lambda_{~a}^{b}\right]^{-1}\,h_a^{~\mu} ,
\nonumber\\
&= & h_a^{~\mu} + \lambda_a^{~b}\,h_a^{~\mu},
\nonumber\\
\Rightarrow \delta h_a^{~\mu} &=& \lambda_a^{~b}\,h_a^{~\mu}
\end{eqnarray}

\item Determinant of the co-frame $h=\text{Det}(h^a_{~\mu})$:
\begin{eqnarray}\label{a001}
h+\delta h &=& \text{Det}(h^a_{~\mu}+\delta h^a_{~\mu}) \nonumber\\
&\approx & h + \text{Det}(\lambda^a_{~b}\, h^b_{~\mu}) = h + \lambda\,h
\nonumber\\
\Rightarrow \delta h &\approx& \lambda\,h
\end{eqnarray}
where $\lambda = \text{Det}(\lambda^a_{~b})\ll 1$ and $\text{Det}(\delta h^a_{~\mu})=\text{Det}(\lambda^a_{~b}\,h^b_{~\mu})=\lambda\,h$.

\item Metric Tensor $g_{\mu\nu}$:
\begin{eqnarray}\label{a003}
g_{\mu\nu}+\delta g_{\mu\nu} &=& \eta_{ab} \left[h^a_{\;\;\mu}+\delta h^a_{\;\;\mu}\right]\left[h^b_{\;\;\nu}+\delta h^b_{\;\;\nu}\right],
\nonumber\\
&\approx & g_{\mu\nu} + \eta_{ab} \left[\delta h^a_{\;\;\mu} h^b_{\;\;\nu}+h^a_{\;\;\mu}\delta h^b_{\;\;\nu}\right]+O\left(|\delta h|^2\right),
\nonumber\\
\Rightarrow \delta g_{\mu\nu} &\approx & \eta_{ab} \left[\delta h^a_{\;\;\mu} h^b_{\;\;\nu}+h^a_{\;\;\mu}\delta h^b_{\;\;\nu}\right]+O\left(|\delta h|^2\right).
\end{eqnarray}

\item Torsion Tensor $T^a_{\;\;\mu\nu}$ and $T^{\rho}_{\;\;\mu\nu}$:
\begin{eqnarray}\label{a005}
T^a_{\;\;\mu\nu}+\delta T^a_{\;\;\mu\nu} &=& \partial_{\mu} h^a_{\;\;\nu}+\partial_{\mu}\left(\delta h^a_{\;\;\nu}\right)-\partial_{\nu} h^a_{\;\;\mu}+\partial_{\nu}\left(\delta h^a_{\;\;\mu}\right)
\nonumber\\
&\approx & T^a_{\;\;\mu\nu}+\left[\partial_{\mu}\left(\delta h^a_{\;\;\nu}\right)-\partial_{\nu}\left(\delta h^a_{\;\;\mu}\right)\right]+O\left(|\delta h|^2\right)
\nonumber\\
\Rightarrow \delta T^a_{\;\;\mu\nu} &\approx& \left[\partial_{\mu}\left(\delta h^a_{\;\;\nu}\right)-\partial_{\nu}\left(\delta h^a_{\;\;\mu}\right)\right]+O\left(|\delta h|^2\right)
\end{eqnarray}
If we also have that $T^{\rho}_{\;\;\mu\nu}=h^{~\rho}_{a}\,T^a_{\;\;\mu\nu}$, then:
\begin{eqnarray}\label{a005a}
T^{\rho}_{\;\;\mu\nu} + \delta T^{\rho}_{\;\;\mu\nu} &=& \left(h^{~\rho}_{a}+\delta h^{~\rho}_{a}\right)\left(T^a_{\;\;\mu\nu}+\delta T^a_{\;\;\mu\nu}\right)
\nonumber\\
&\approx & T^{\rho}_{\;\;\mu\nu}+\delta h^{~\rho}_{a}\,T^a_{\;\;\mu\nu}+h^{~\rho}_{a}\,\delta T^a_{\;\;\mu\nu}+O\left(|\delta h|^2\right)
\nonumber\\
\Rightarrow \delta T^{\rho}_{\;\;\mu\nu} &\approx& \delta h^{~\rho}_{a}\,\left[\partial_{\mu}\,h^a_{\;\;\nu}-\partial_{\nu}\,h^a_{\;\;\mu}\right]+h^{~\rho}_{a}\,\left[\partial_{\mu}\left(\delta h^a_{\;\;\nu}\right)-\partial_{\nu}\left(\delta h^a_{\;\;\mu}\right)\right]+O\left(|\delta h|^2\right)
\nonumber\\
\end{eqnarray}



\item Torsion scalar $T$:
\begin{eqnarray}\label{a004}
T+\delta T &=& \frac{1}{4}\left(T^a_{\;\;\mu\nu}+\delta T^a_{\;\;\mu\nu}\right)\left(T_a^{\;\;\mu\nu}+\delta T_a^{\;\;\mu\nu}\right)+\frac{1}{2}\left(T^a_{\;\;\mu\nu}+\delta T^a_{\;\;\mu\nu}\right)\left(T^{\nu\mu}_{\;\;a}+\delta T^{\nu\mu}_{\;\;a}\right)
\nonumber\\
&&\qquad -\left(T^{\nu}_{\;\;\mu\nu}+\delta T^{\nu}_{\;\;\mu\nu}+\right)\left(T^{\rho\mu}_{\;\;\rho}+\delta T^{\rho\mu}_{\;\;\rho}\right)
\nonumber\\
&=& T+\frac{1}{4}\left(\delta T^a_{\;\;\mu\nu}T_a^{\;\;\mu\nu}+T^a_{\;\;\mu\nu}\delta T_a^{\;\;\mu\nu}\right)
+\frac{1}{2}\left(\delta T^a_{\;\;\mu\nu}T^{\nu\mu}_{\;\;a}+T^a_{\;\;\mu\nu}\delta T^{\nu\mu}_{\;\;a}\right)
\nonumber\\
&& \qquad-\left(\delta T^{\nu}_{\;\;\mu\nu}T^{\rho\mu}_{\;\;\rho}+T^{\nu}_{\;\;\mu\nu}\delta T^{\rho\mu}_{\;\;\rho}\right) +O\left(|\delta h|^2\right)
\nonumber\\
\Rightarrow  \delta T &=& \frac{1}{4}\left(\delta T^a_{\;\;\mu\nu}T_a^{\;\;\mu\nu}+T^a_{\;\;\mu\nu}\delta T_a^{\;\;\mu\nu}\right)
+\frac{1}{2}\left(\delta T^a_{\;\;\mu\nu}T^{\nu\mu}_{\;\;a}+T^a_{\;\;\mu\nu}\delta T^{\nu\mu}_{\;\;a}\right)
\nonumber\\
&&\qquad -\left(\delta T^{\nu}_{\;\;\mu\nu}T^{\rho\mu}_{\;\;\rho}+T^{\nu}_{\;\;\mu\nu}\delta T^{\rho\mu}_{\;\;\rho}\right)+O\left(|\delta h|^2\right)
\end{eqnarray}
In terms of equations \eqref{a005} and \eqref{a005a}, the equation \eqref{a004} becomes as:
\small
\begin{align}\label{a004a}
\delta T =& \frac{1}{4}\Bigg[\left(\partial_{\mu}\left(\delta h^a_{\;\;\nu}\right)-\partial_{\nu}\left(\delta h^a_{\;\;\mu}\right)\right) \left(\partial^{\mu}\,h_a^{\;\;\nu}-\partial^{\nu}\,h_a^{\;\;\mu}\right)+\left(\partial_{\mu}\,h^a_{\;\;\nu}-\partial_{\nu}\,h^a_{\;\;\mu}\right)
\nonumber\\
&\times\,\left(\partial^{\mu}\,\left(\delta h_a^{\;\;\nu}\right)-\partial^{\nu}\,\left(\delta h_a^{\;\;\mu}\right)\right)\Bigg]+\frac{1}{2}\Bigg[\left(\partial_{\mu}\left(\delta h^a_{\;\;\nu}\right)-\partial_{\nu}\left(\delta h^a_{\;\;\mu}\right)\right)\left(\partial^{\nu}\,h^{\mu}_{~a}-\partial^{\mu}\,h^{\nu}_{~a}\right)
\nonumber\\
& +\left(\partial_{\mu}\,h^a_{\;\;\nu}-\partial_{\nu}\,h^a_{\;\;\mu}\right)\left(\partial^{\nu}\,\left(\delta h^{\mu}_{~a}\right)-\partial^{\mu}\,\left(\delta h^{\nu}_{~a}\right)\right)\Bigg]
\nonumber\\
& -\Bigg[\left(\delta h^{~\nu}_{a}\,\left[\partial_{\mu}\,h^a_{\;\;\nu}-\partial_{\nu}\,h^a_{\;\;\mu}\right]+h^{~\nu}_{a}\,\left[\partial_{\mu}\left(\delta h^a_{\;\;\nu}\right)-\partial_{\nu}\left(\delta h^a_{\;\;\mu}\right)\right]\right)\left(h^a_{~\rho}\left(\partial^{\rho}\,h^{\mu}_{~a}-\partial^{\mu}\,h^{\rho}_{~a}\right)\right)
\nonumber\\
&+\left(h^{~\nu}_{a}\,\left[\partial_{\mu}\,h^a_{\;\;\nu}-\partial_{\nu}\,h^a_{\;\;\mu}\right]\right)\left(\delta h^a_{~\rho}\left(\partial^{\rho}\,h^{\mu}_{~a}-\partial^{\mu}\,h^{\rho}_{~a}\right)+h^a_{~\rho}\left(\partial^{\rho}\,\left(\delta h^{\mu}_{~a}\right)-\partial^{\mu}\,\left(\delta h^{\rho}_{~a}\right)\right)\right)\Bigg]
\nonumber\\
&+O\left(|\delta h|^2\right).
\end{align}
\normalsize



\item Lagrangian density $\mathcal{L}_{Grav}$:
\begin{eqnarray}\label{a009}
\mathcal{L}_{Grav}+\delta \mathcal{L}_{Grav} &=& \frac{1}{2\kappa}\left(h+\delta h\right)\,f\left(T+\delta T\right), \nonumber\\
&\approx& \mathcal{L}_{Grav}+\frac{1}{2\kappa}\left[\delta h\,f\left(T\right)+h\,f_T\left(T\right)\,\delta T\right]+O\left(|\delta h|^2\right),
\nonumber\\
\Rightarrow \delta \mathcal{L}_{Grav} &\approx & \frac{1}{2\kappa}\left[\delta h\,f\left(T\right)+h\,f_T\left(T\right)\,\delta T\right]+O\left(|\delta h|^2\right).
\end{eqnarray}


\item Sum of Torsion and Ricci Curvature scalar $\overset{\ \circ}{R}+T$:
Here $\overset{\ \circ}{R}$ is the Ricci scalar computed from the Levi-Civita connection.
\begin{eqnarray}\label{a013}
\delta (\overset{\ \circ}{R}+T) &=&\, \delta \left[\frac{2}{h}\,\delta_{\mu}\left(h\,T^{\nu\mu}_{\;\;\nu}\right)\right] = 2 \left[\delta \left(\frac{1}{h}\right)\,\delta_{\mu}\left(h\,T^{\nu\mu}_{\;\;\nu}\right)+\frac{1}{h}\delta_{\mu}\left[\delta\left(h\,T^{\nu\mu}_{\;\;\nu}\right)\right]\right]
\nonumber\\
& \approx & \frac{2}{h}\,\left[-\frac{\delta h}{h}(\delta_{\mu}h)\,T^{\nu\mu}_{\;\;\nu}+\left(\delta_{\mu}(\delta h)\right)\,T^{\nu\mu}_{\;\;\nu}+ \left(\delta_{\mu} h\right)\,\delta T^{\nu\mu}_{\;\;\nu}+h \,\delta_{\mu} \left(\delta T^{\nu\mu}_{\;\;\nu}\right)\right]
\nonumber\\
& &+O\left(|\delta h|^2\right)
\end{eqnarray}
By using equation \eqref{a005a}, the equation \eqref{a013} becomes as:
\begin{align}\label{a013a}
\delta (\overset{\ \circ}{R}+T) \approx & \frac{2}{h}\,\Bigg[-\frac{\delta h}{h}(\delta_{\mu}h)\,\left(h^a_{~\nu}\left[\partial^{\nu}\,h^{\mu}_{~a}-\partial^{\mu}\,h^{\nu}_{~a}\right]\right)+\left(\delta_{\mu}(\delta h)\right)\,\left(h^a_{~\nu}\left[\partial^{\nu}\,h^{\mu}_{~a}-\partial^{\mu}\,h^{\nu}_{~a}\right]\right)
\nonumber\\
&+ \left(\delta_{\mu} h\right)\,\left(\delta h^a_{~\nu}\left[\partial^{\nu}\,h^{\mu}_{~a}-\partial^{\mu}\,h^{\nu}_{~a}\right]+h^a_{~\nu}\left[\partial^{\nu}\,\left(\delta h^{\mu}_{~a}\right)-\partial^{\mu}\,\left(\delta h^{\nu}_{~a}\right)\right]\right)
\nonumber\\
&+h \,\delta_{\mu} \left(\delta h^a_{~\nu}\left[\partial^{\nu}\,h^{\mu}_{~a}-\partial^{\mu}\,h^{\nu}_{~a}\right]+h^a_{~\nu}\left[\partial^{\nu}\,\left(\delta h^{\mu}_{~a}\right)-\partial^{\mu}\,\left(\delta h^{\nu}_{~a}\right)\right]\right)\Bigg]+O\left(|\delta h|^2\right)
\nonumber\\
\end{align}




\item Superpotential $S_a^{\;\;\mu\nu}$:
\small
\begin{align}\label{a014}
S_{ab}^{\;\;\;\mu} +\delta S_{ab}^{\;\;\;\mu} =&\, \frac{1}{2}\left(T_{ab}^{\;\;\mu}+\delta T_{ab}^{\;\;\mu}+T^{\mu}_{\;\;ba}+\delta T^{\mu}_{\;\;ba}-T^{\mu}_{\;\;ab}-\delta T^{\mu}_{\;\;ab}\right)
\nonumber\\
&\qquad -\left(h_{~b}^{\mu}+\delta h_{~b}^{\mu}\right)\left(T_{\rho a}^{\;\;\rho}+\delta T_{\rho a}^{\;\;\rho}\right)
+\left(h_{~a}^{\mu}+\delta h_{~a}^{\mu}\right)\left(T_{\rho b}^{\;\;\rho}+\delta T_{\rho b}^{\;\;\rho}\right)
\nonumber\\
\approx& \, S_a^{\;\;\mu\nu}+\Bigg[\frac{1}{2}\left(\delta T_{ab}^{\;\;\mu}+\delta T^{\mu}_{\;\;ba}-\delta T^{\mu}_{\;\;ab}\right)-\delta h_{~b}^{\mu} T_{\rho a}^{\;\;\rho}-h_{~b}^{\mu}\delta T_{\rho a}^{\;\;\rho}+\delta h_{~a}^{\mu}T_{\rho b}^{\;\;\rho}
\nonumber\\
&\quad +h_{~a}^{\mu}\delta T_{\rho b}^{\;\;\rho}\Bigg]+O\left(|\delta h|^2\right)
\nonumber\\
\Rightarrow \delta S_{ab}^{\;\;\;\mu}  \approx&  \left[\frac{1}{2}\left(\delta T_{ab}^{\;\;\mu}+2\,\delta T^{\mu}_{\;\;ba}\right)-\delta h_{~b}^{\mu} T_{\rho a}^{\;\;\rho}-h_{~b}^{\mu}\delta T_{\rho a}^{\;\;\rho}+\delta h_{~a}^{\mu}T_{\rho b}^{\;\;\rho}+h_{~a}^{\mu}\delta T_{\rho b}^{\;\;\rho}\right]+O\left(|\delta h|^2\right)
\nonumber\\
\end{align}
\normalsize


In terms of $\delta h^a_{~\mu}$, the equation \eqref{a014} becomes:
\small
\begin{align}\label{a014a}
& \delta S_{ab}^{\;\;\;\mu}  \approx 
\nonumber\\
& \Bigg[\frac{1}{2}\left(\partial_a\left(\delta h_b^{~\mu}\right)- \partial_b\left(\delta h_a^{~\mu}\right)\right)+\left( \partial_b\left(\delta h_{~a}^{\mu}\right)-\partial_a\left(\delta h_{~b}^{\mu}\right)\right)-\delta h_{~b}^{\mu} \left(h_{~\rho}^{c} \left[\partial_c\,h_a^{~\rho}-\partial_a\,h_c^{~\rho}\right]\right)
\nonumber\\
&-h_{~b}^{\mu}\left(\delta h_{~\rho}^{c}\left[\partial_c\,h_a^{~\rho}-\partial_a\,h_c^{~\rho}\right]+h_{~\rho}^{c} \left[\partial_c\,\left(\delta h_a^{~\rho}\right)-\partial_a\,\left(\delta h_c^{~\rho}\right)\right]\right)+\delta h_{~a}^{\mu}\left(h_{~\rho}^{c} \left[\partial_c\,h_b^{~\rho}-\partial_b\,h_c^{~\rho}\right]\right)
\nonumber\\
&+h_{~a}^{\mu}\left(\delta h_{~\rho}^{c} \left[\partial_c\,h_b^{~\rho}-\partial_b\,h_c^{~\rho}\right]+ h_{~\rho}^{c} \left[\partial_c\,\left(\delta h_b^{~\rho}\right)-\partial_b\,\left(\delta h_c^{~\rho}\right)\right]\right)\Bigg]+O\left(|\delta h|^2\right)
\end{align}
\normalsize



\item Einstein Tensor $\overset{\ \circ}{G}_{\mu\nu}$:
\begin{eqnarray}\label{a015}
\overset{\ \circ}{G}_{ab}+\delta \overset{\ \circ}{G}_{ab} &=& \left(\overset{\ \circ}{G}_{\mu\nu}+\delta \overset{\ \circ}{G}_{\mu\nu}\right)\left(h_a^{\;\;\mu}+\delta h_a^{\;\;\mu}\right)\left(h_b^{\;\;\nu}+\delta h_b^{\;\;\nu}\right)
\nonumber\\
&\approx & \overset{\ \circ}{G}_{ab} +\left[\overset{\ \circ}{G}_{\mu\nu}\left(\delta h_a^{\;\;\mu} h_b^{\;\;\nu}+h_a^{\;\;\mu}\delta h_b^{\;\;\nu}\right)+\delta \overset{\ \circ}{G}_{\mu\nu}\left(h_a^{\;\;\mu} h_b^{\;\;\nu}\right) \right]+O\left(|\delta h|^2\right)
\nonumber\\
\Rightarrow \delta \overset{\ \circ}{G}_{ab} &\approx & \left[\overset{\ \circ}{G}_{\mu\nu}\left(\delta h_a^{\;\;\mu} h_b^{\;\;\nu}+h_a^{\;\;\mu}\delta h_b^{\;\;\nu}\right)+\delta \overset{\ \circ}{G}_{\mu\nu}\left(h_a^{\;\;\mu} h_b^{\;\;\nu}\right) \right]+O\left(|\delta h|^2\right).
\end{eqnarray}
If $\overset{\ \circ}{G}_{\mu\nu}=\overset{\ \circ}{R}_{\mu\nu}-\frac{1}{2}\,g^{\sigma\rho}\,g_{\mu\nu}\,\overset{\ \circ}{R}_{\sigma\rho}=\overset{\ \circ}{R}_{\mu\nu}-\frac{\eta^{cd}\,\eta_{ab}}{2}\left[h_c^{\;\;\sigma}\,h_d^{\;\;\rho}\,h^a_{\;\;\mu}\,h^b_{\;\;\nu}\right]\,\overset{\ \circ}{R}_{\sigma\rho} $, then we get from equation \eqref{a003} that:
\begin{align}\label{a015a}
\delta \overset{\ \circ}{G}_{\mu\nu} \approx &\, \delta \overset{\ \circ}{R}_{\mu\nu}-\frac{\eta^{cd}\,\eta_{ab}}{2}\,\Bigg[\left[h_c^{\;\;\sigma}\,h_d^{\;\;\rho}\,h^a_{\;\;\mu}\,h^b_{\;\;\nu}\right]\,\delta \overset{\ \circ}{R}_{\sigma\rho}+\Bigg[\delta h_c^{\;\;\sigma}\,h_d^{\;\;\rho}\,h^a_{\;\;\mu}\,h^b_{\;\;\nu}+h_c^{\;\;\sigma}\,\delta h_d^{\;\;\rho}\,h^a_{\;\;\mu}\,h^b_{\;\;\nu}
\nonumber\\ 
&+h_c^{\;\;\sigma}\,h_d^{\;\;\rho}\,\delta h^a_{\;\;\mu}\,h^b_{\;\;\nu}+h_c^{\;\;\sigma}\,h_d^{\;\;\rho}\,h^a_{\;\;\mu}\,\delta h^b_{\;\;\nu}\Bigg]\,\overset{\ \circ}{R}_{\sigma\rho}\Bigg]+O\left(|\delta h|^2\right)
\end{align}
By substituting equation \eqref{a015a} into equation \eqref{a015}, we obtain that:
\small
\begin{align}\label{a015c}
\delta \overset{\ \circ}{G}_{ab} \approx & \Bigg[\overset{\ \circ}{R}_{\mu\nu}-\frac{\eta^{cd}\,\eta_{ef}}{2}\left[h_c^{\;\;\sigma}\,h_d^{\;\;\rho}\,h^e_{\;\;\mu}\,h^f_{\;\;\nu}\right]\,\overset{\ \circ}{R}_{\sigma\rho}\Bigg]\left(\delta h_a^{\;\;\mu} h_b^{\;\;\nu}+h_a^{\;\;\mu}\delta h_b^{\;\;\nu}\right)
\nonumber\\
&+\left(h_a^{\;\;\mu} h_b^{\;\;\nu}\right)\Bigg[\delta \overset{\ \circ}{R}_{\mu\nu}-\frac{\eta^{cd}\,\eta_{ef}}{2}\,\Bigg[\left[h_c^{\;\;\sigma}\,h_d^{\;\;\rho}\,h^e_{\;\;\mu}\,h^f_{\;\;\nu}\right]\,\delta \overset{\ \circ}{R}_{\sigma\rho}
\nonumber\\
&+\left[\delta h_c^{\;\;\sigma}\,h_d^{\;\;\rho}\,h^e_{\;\;\mu}\,h^f_{\;\;\nu}+h_c^{\;\;\sigma}\,\delta h_d^{\;\;\rho}\,h^e_{\;\;\mu}\,h^f_{\;\;\nu}+h_c^{\;\;\sigma}\,h_d^{\;\;\rho}\,\delta h^e_{\;\;\mu}\,h^f_{\;\;\nu}+h_c^{\;\;\sigma}\,h_d^{\;\;\rho}\,h^e_{\;\;\mu}\,\delta h^f_{\;\;\nu}\right]\,\overset{\ \circ}{R}_{\sigma\rho}\Bigg]\Bigg]
\nonumber\\ 
 &+O\left(|\delta h|^2\right)
\end{align}
\normalsize
Now if we have that $\overset{\ \circ}{R}_{\mu\nu}=h_k^{~\alpha}\,h^m_{~\mu}\,\overset{\ \circ}{R}^k_{~m\alpha\nu}$, then equation \eqref{a015c} becomes
\small
\begin{align}\label{a015d}
\delta \overset{\ \circ}{G}_{ab} \approx & \Bigg[h_k^{~\alpha}\,h^m_{~\mu}\,\overset{\ \circ}{R}^k_{~m\alpha\nu}-\frac{\eta^{cd}\,\eta_{ef}}{2}\left[h_c^{\;\;\sigma}\,h_d^{\;\;\rho}\,h^e_{\;\;\mu}\,h^f_{\;\;\nu}\right]\,h_k^{~\alpha}\,h^m_{~\sigma}\,\overset{\ \circ}{R}^k_{~m\alpha\rho}\Bigg]\left(\delta h_a^{\;\;\mu} h_b^{\;\;\nu}+h_a^{\;\;\mu}\delta h_b^{\;\;\nu}\right)
\nonumber\\
&+\left(h_a^{\;\;\mu} h_b^{\;\;\nu}\right)\Bigg[\left[\left(\delta h_k^{~\alpha}\,h^m_{~\mu}+h_k^{~\alpha}\,\delta h^m_{~\mu}\right)\,\overset{\ \circ}{R}^k_{~m\alpha\nu}+ h_k^{~\alpha}\,h^m_{~\mu}\,\delta \overset{\ \circ}{R}^k_{~m\alpha\nu}\right]
\nonumber\\
&-\frac{\eta^{cd}\,\eta_{ef}}{2}\,\Bigg[\left[h_c^{\;\;\sigma}\,h_d^{\;\;\rho}\,h^e_{\;\;\mu}\,h^f_{\;\;\nu}\right]\,\left[\left(\delta h_k^{~\alpha}\,h^m_{~\sigma}+h_k^{~\alpha}\,\delta h^m_{~\mu}\right)\,\overset{\ \circ}{R}^k_{~m\alpha\rho}+ h_k^{~\alpha}\,h^m_{~\sigma}\,\delta \overset{\ \circ}{R}^k_{~m\alpha\rho}\right]
\nonumber\\
&+\left[\delta h_c^{\;\;\sigma}\,h_d^{\;\;\rho}\,h^e_{\;\;\mu}\,h^f_{\;\;\nu}+h_c^{\;\;\sigma}\,\delta h_d^{\;\;\rho}\,h^e_{\;\;\mu}\,h^f_{\;\;\nu}+h_c^{\;\;\sigma}\,h_d^{\;\;\rho}\,\delta h^e_{\;\;\mu}\,h^f_{\;\;\nu}+h_c^{\;\;\sigma}\,h_d^{\;\;\rho}\,h^e_{\;\;\mu}\,\delta h^f_{\;\;\nu}\right]
\nonumber\\
&\quad\times\,h_k^{~\alpha}\,h^m_{~\sigma}\,\overset{\ \circ}{R}^k_{~m\alpha\rho}\Bigg]\Bigg]+O\left(|\delta h|^2\right)
\end{align}
\normalsize

For pure Minkowski spacetime, we have that $\overset{\ \circ}{R}^k_{~m\alpha\rho}=0$ by default and equation \eqref{a015d} reduces as:
\small
\begin{align}\label{a015b}
\delta \overset{\ \circ}{G}_{ab} \approx & \left(h_a^{\;\;\mu} h_b^{\;\;\nu}\right)\Bigg[h_k^{~\alpha}\,h^m_{~\mu}\,\delta \overset{\ \circ}{R}^k_{~m\alpha\nu}-\frac{\eta^{cd}\,\eta_{ef}}{2}\,\left[h_c^{\;\;\sigma}\,h_d^{\;\;\rho}\,h^e_{\;\;\mu}\,h^f_{\;\;\nu}\right]\,h_k^{~\alpha}\,h^m_{~\sigma}\,\delta \overset{\ \circ}{R}^k_{~m\alpha\rho}\Bigg]+O\left(|\delta h|^2\right).
\nonumber\\
\end{align}
\normalsize
This equation \eqref{a015b} is useful for equations \eqref{433} and \eqref{443} and the energy-momentum stability test.


\end{enumerate}




\section{General Perturbed Torsion-based Field Equation by linearization}\label{appenc}

Here we can also obtain the perturbed field equation (equations (\ref{301}) and (\ref{302})) using equation (\ref{a009}) with matter contribution as follows:
\begin{eqnarray}\label{320}
\delta \mathcal{L} &\approx & \frac{1}{2\kappa}\left[\delta h\,f\left(T\right)+h\,f_T\left(T\right)\,\delta T\right]+\delta \mathcal{L}_{Matter} +O\left(|\delta h|^2\right)
\end{eqnarray}
As for non-perturbed FEs, we have here that $\delta \Theta_{(ab)}=\delta T_{ab}\equiv \frac{1}{2}\frac{\delta \left(\delta L_{Matt}\right)}{\delta g_{ab}}$.


\noindent For the term $\frac{1}{2\kappa}\,\delta h\,f\left(T\right)$, we obtain by analogy with equation (\ref{221}) the following part (here $\delta g_{ab}=0$ for orthonormal framework):
\begin{eqnarray}\label{321}
\frac{\delta h\,f\left(T\right)}{2\kappa}\, & \rightarrow & f_{TT}\left[\delta S_{\left(ab\right)}^{~~~\mu}\,\partial_{\mu}T+S_{\left(ab\right)}^{~~~\mu}\,\partial_{\mu}\left(\delta T\right)\right]+f_{T}\,\delta \overset{\ \circ}{G}_{ab}-\frac{g_{ab}}{2}\,f_{T}\,\delta T \quad\quad\text{Symmetric}
\nonumber\\
&\rightarrow & f_{TT}\,\left[S_{\left[ab\right]}^{~~~\mu}\partial_{\mu}\left(\delta T\right)+\delta S_{\left[ab\right]}^{~~~\mu}\partial_{\mu} T\right] \quad\quad \quad\quad \quad\quad\quad\quad\quad\quad\quad\quad \text{Antisymmetric}
\nonumber\\
\end{eqnarray}
At equation (\ref{321}), we only perturb the physical quantities linked by $\delta h$ giving $\delta T$, $\delta \overset{\ \circ}{G}_{ab}$ and $\delta S_{ab}^{\;\;\mu}$. We do not perturb $f(T)$ and its derivatives.

\noindent For the term $\frac{1}{2\kappa}\,h\,f_T\left(T\right)\,\delta T$, we still obtain by analogy with equation (\ref{221}) the part (here again $\delta g_{ab}=0$):
\begin{eqnarray}\label{322}
\frac{h\,f_T\left(T\right)\,\delta T}{2\kappa}\, & \rightarrow & \left[f_{TTT}\,S_{\left(ab\right)}^{~~~\mu}\partial_{\mu}T+f_{TT}\,\overset{\ \circ}{G}_{ab}+ \frac{g_{ab}}{2}\left(f_T-T\,f_{TT}\right)\right]\,\delta T \quad\quad\text{Symmetric}
\nonumber\\
&\rightarrow & f_{TTT}\,\left[S_{\left[ab\right]}^{~~~\mu}\partial_{\mu} T\right]\,\delta T \quad\quad\quad\quad\quad\quad\quad\quad\quad\quad\quad\quad\quad\quad\quad\text{Antisymmetric}
\nonumber\\
\end{eqnarray}
At equation (\ref{322}), we only change $f(T) \rightarrow f_T(T)\,\delta T$, $f_T (T) \rightarrow f_{TT}(T)\,\delta T$ and $f_{TT}(T) \rightarrow f_{TTT}(T)\,\delta T$. We does not perturb the physical quantities themselves.


By adding the equations (\ref{321}) and (\ref{322}), we obtain exactly at 1st order the equations (\ref{301}) and (\ref{302}). This is the sign that the linearization of gravity and the direct perturbation of the field equation described by the equations (\ref{221}) are both equivalent. By these two methods, we obtain the field equation described by equations (\ref{301}) and (\ref{302}), which is in the order of things.



\end{document} 