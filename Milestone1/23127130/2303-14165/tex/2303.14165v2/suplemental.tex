\UseRawInputEncoding 
\documentclass[aps,prb,citenumautoscript,reprint,
 amsmath,amssymb,nofootinbib,superscriptaddress,longbibliography
]{revtex4-1}
 
 % Some packages I think we need
\usepackage{graphicx}% Include figure files
\usepackage{bm}% bold math
\usepackage{float}
\usepackage{siunitx}
\usepackage{color}


% make footnotes use star (we'll use this to make the email reference since revtex defaults to letters)
\usepackage[symbol]{footmisc}
\renewcommand{\thefootnote}{\fnsymbol{footnote}}

%number pages with S-#
\renewcommand{\thepage}{S-\arabic{page}}  

% this puts the page numbers at the bottom right
\usepackage{fancyhdr}
\fancyhf{}
\renewcommand{\headrulewidth}{0pt}
\cfoot{\thepage}
\pagestyle{fancy}

\newcommand{\red}[1]{\textcolor{red}{#1}}
\newcommand{\blue}[1]{\textcolor{blue}{#1}}

\begin{document}

\title{Supporting Information\\
\small The Effect of Dust and Hotspots on the Thermal Stability of Laser Sails}

\author{Gabriel R. Jaffe$^{\dagger}$}
%\thanks{The authors contribute equally to this paper.}
\affiliation{Department of Physics, University of Wisconsin-Madison, Madison WI 53706 USA}

\author{Gregory R. Holdman$^{\dagger}$}
%\thanks{The authors contribute equally to this paper.}
\affiliation{Department of Physics, University of Wisconsin-Madison, Madison WI 53706 USA}


\author{Min Seok Jang}
\affiliation{
School of Electrical Engineering, Korea Advanced Institute of Science and Technology, Daejeon 34141, Korea
}%

\author{Demeng Feng}
\affiliation{
Department of Electrical and Computer Engineering, University of Wisconsin-Madison, Madison WI 53706 USA
}%

\author{Mikhail A. Kats}
\affiliation{
Department of Electrical and Computer Engineering, University of Wisconsin-Madison, Madison WI 53706 USA
}%


\author{Victor Watson Brar*$^,$}
%\thanks{vbrar@wisc.edu}%
%\email[]{}
\affiliation{%
Department of Physics, University of Wisconsin-Madison, Madison WI 53706 USA
}%

\date{\today}


\maketitle

% this adds page number to title page
\thispagestyle{fancy}

% centers and displays the email line under the affiliations
\onecolumngrid
\begin{center}
*Email: vbrar@wisc.edu\newline$^{\dagger}$The authors contribute equally to this paper\clearpage
\end{center}

%=====================================
\section{The effect of dust particle size}

The size and areal density of dust particles is an important factor in determining whether thermal runaway occurs.  Following a scaling argument described below, we estimate that isolated dust particles with radii $<$350\,nm will not initiate thermal runaway.  This limit is supported by simulations using different dust sizes, plots of which can be seen below.  Furthermore, we calculate analytically that the total areal coverage of dust of a 10\,m$^2$ sail must stay below 1\,mm$^2$ to prevent thermal runaway.

In the first simulation shown below, the dust particle radius is 200\,nm and no thermal runaway is observed.  In the second simulation, the dust particle has a radius of 400\,nm and thermal runaway destroys the metasurface.  

% ---------------- Figure S1 ----------------
\begin{figure}[h]
	\centering
	\let\nobreakspace\relax
	\renewcommand{\figurename}{Figure S}
\includegraphics[width = 1\linewidth]{./200nmDustSim.png}
\caption{Simulation of the same metasurface depicted in Fig. 3a of the main text but with a dust particle with a radius of 200nm.  The particle is small enough that thermal runaway does not spread.  The left plot shows a 3D temperature map of the metasurface after 1ms.  The right plot shows the total heat in the simulated volume as a function of time which appears to have reached a steady state.
}
\label{fig:S1}
\end{figure}
% -------------------------------------------

% ---------------- Figure S2 ----------------
\begin{figure}[h]
	\centering
	\let\nobreakspace\relax
	\renewcommand{\figurename}{Figure S}
\includegraphics[width = 1\linewidth]{./400nmDustSim.png}
\caption{Simulation of the same metasurface depicted in Fig. 3a of the main text but with a dust particle with a radius of 400nm.  The particle is large enough that thermal runaway destroys the metasurface, leaving predominantly a mesh of molten SiO2.  The left plot shows a 3D temperature map of the metasurface after 1ms.  The right plot shows the total heat in the simulated volume as a function of time.  The onset of thermal runaway can be seen around 200�s, where the rate of heat absorption dramatically increases.  The total heat in then begins to decrease as metasurface material evaporates.
}
\label{fig:S2}
\end{figure}
% -------------------------------------------

% ---------------- Figure S3 ----------------
\begin{figure}[h]
	\centering
	\let\nobreakspace\relax
	\renewcommand{\figurename}{Figure S}
\includegraphics[width = .50\linewidth]{./NetHeatAddedToMetasurface.png}
\caption{The total absorbed heat minus the total emitted heat as a function of temperature of the metasurface calculated from Fig. 2c of the manuscript.  Positive fluxes represent net absorption and negative values represent net emission of the metasurface.
}
\label{fig:S3}
\end{figure}
% -------------------------------------------

Under steady state conditions, the temperature rise of a Si pillar with a fixed heat input from a dust particle can be characterized as $T_{pillar}= R \cdot P$, where $P = \pi a^2 \cdot 1\,GW m^{-2}$   is the heat absorbed by the dust particle of radius $a$, and $R$ is thermal resistance, units of K W$^{-1}$, between a single Si pillar and the surrounding metasurface. In Fig.\,3b of the main text we see that increasing the in-plane thermal conductivity of the substrate by a factor of 8, and as a result, decreasing $R$ by the same factor, is sufficient to keep $T_{pillar}$ small enough to prevent thermal runaway from spreading.  Reducing the particle radius by a factor of $\sqrt{8}$ should have the same effect of as decreasing $R$ by a factor of 8.  Using this approach, we estimate that isolated particles with radii $<$\,350\,nm will not cause thermal runaway for the case of a 200\,nm thick SiO$_2$ substrate.

If we assume a homogenous areal distribution of dust across the entire sail, we can analytically calculate the minimum areal coverage of dust that would cause thermal runaway. Plotted below, we show the heat flux absorbed minus the heat emitted for the metasurface calculated from the curves shown in Fig. 2c of the manuscript.  We see that the maximum net radiated heat flux of the metasurface is $\sim$300\,W\,m$^{-2}$ which occurs at $\sim$\,450\,K.  If the absorbed power flux of the dust particles on the metasurface is greater than this value, the sail will be unable to sufficiently cool radiatively and will always enter thermal runaway.  Dust areal coverage of 1\,mm$^{2}$ on a 10\,m$^2$ sail would absorb the necessary 3000\,W to push the sail into thermal runaway.  Alternatively, this corresponds to a 7.5\,nm particle on one out of every 1600 Si pillars.  



\end{document}
