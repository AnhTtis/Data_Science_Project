\documentclass[10pt,twocolumn,letterpaper]{article}

\usepackage{iccv}
\usepackage{times}
\usepackage{epsfig}
\usepackage{graphicx}
\usepackage{amsmath}
\usepackage{amssymb}
\usepackage{soul}

\usepackage[table]{xcolor}
\definecolor{mydarkblue}{rgb}{0,0.08,0.45}

%\usepackage[%pagebackref,
%            breaklinks,
%            colorlinks,
%            linkcolor=mydarkblue,
%            citecolor=mydarkblue,
%            filecolor=mydarkblue,
%            urlcolor=mydarkblue,]{hyperref}

% Include other packages here, before hyperref.
%\newcommand{\etal}{\textit{et al}. }
%\newcommand{\ie}{\textit{i}.\textit{e}. }
%\newcommand{\eg}{\textit{e}.\textit{g}. }
%\newcommand{\etc}{\textit{etc}}

% packages from  FAME paper
\usepackage{algorithm}
\usepackage{algorithmic}
\usepackage{bbding}
\usepackage{booktabs}
\newcommand\norm[1]{\left\lVert#1\right\rVert}
\newcommand\round[1]{\left[#1\right]}
\usepackage{multirow}
\usepackage{arydshln}
\usepackage[accsupp]{axessibility}
\usepackage{xcolor}
\usepackage{amsfonts}

% packages from severe paper
%\usepackage[table,usenames,dvipsnames]{xcolor}
%\usepackage{tikz}
\usepackage{comment}
\usepackage{color}
\usepackage{tabularx,ragged2e}
% \usepackage{subfig} % this has issues when used with subcaption
\usepackage{csquotes}
\usepackage{arydshln}
\usepackage{cite}
\usepackage{subfiles}
\usepackage{fontawesome}
% \usepackage[table,RGB]{xcolor}
\usepackage{inconsolata}
\usepackage{epsfig}
\usepackage{amsmath}
\usepackage{amssymb}
\usepackage{multirow}
\usepackage{booktabs}
\usepackage{cellspace}
\usepackage{subcaption}
\usepackage[hang,flushmargin]{footmisc}
%
\usepackage{booktabs,tabularx, colortbl} 
\usepackage{collcell}
\usepackage{pgfplots}
\pgfplotsset{compat=newest}
\usepgfplotslibrary{colormaps}
\usepackage{caption}
\usepackage{array}
\usepackage{enumitem}
\usepackage{adjustbox}


\newcommand{\note}[1]{\textcolor{blue}{#1}}
\newcommand{\notered}[1]{\textcolor{red}{#1}}
\newcommand{\hd}[1]{\textcolor{orange}{#1}}
\newcommand{\cs}[1]{\textcolor{magenta}{#1}}


\definecolor{heatmap-yellow}{rgb}{1.0, 1.0, 0.8509803921568627}
\definecolor{heatmap-blue}{rgb}{0.03137254901960784, 0.11372549019607843, 0.34509803921568627}
\newcommand*\colourcheck[1]{%
  \expandafter\newcommand\csname #1check\endcsname{\textcolor{#1}{\ding{51}}}%
}
\newcommand*\colourcross[1]{%
  \expandafter\newcommand\csname #1cross\endcsname{\textcolor{#1}{\ding{55}}}%
}
\colourcheck{blue}
\colourcheck{green}
% \colourcheck{red}
\colourcross{red}

\newcommand{\pink}[1]{{\color[HTML]{FFCCC9} #1}}
\newcommand{\prettygreen}[1]{{\color[HTML]{32CB00} #1}}
\definecolor{audiogreen}{RGB}{234,241,230}
\definecolor{visualblue}{RGB}{231,240,244}

\newcommand{\red}[1]{\color{red}{#1}}
\definecolor{darkgreen}{rgb}{0.0, 0.75, 0.0}
\newcommand{\green}[1]{\textcolor{darkgreen}{#1}}
\definecolor{bananayellow}{rgb}{1.0, 0.88, 0.21}
\newcommand{\yellow}[1]{\textcolor{bananayellow}{#1}}

% \usepackage{subcaption}
% \usepackage{subcaption}
% \usepackage{subfig}
\usepackage{multirow}


% matplotlib PiYG
\definecolor{lowcolor}{HTML}{ef8a62}
\definecolor{centralcolor}{HTML}{f7f7f7}
\definecolor{highcolor}{HTML}{67a9cf}
\definecolor{same_attr}{HTML}{009900}
\definecolor{diff_attr}{HTML}{FE0000}
\definecolor{Gray}{gray}{0.9}
\usepackage{color, colortbl}

%\definecolor{lowcolor}{rgb}{0.9673202614379085, 0.968473663975394, 0.9656286043829296}
%\definecolor{centralcolor}{rgb}{0.57185, 0.83323, 0.9450}
%\definecolor{highcolor}{rgb}{0.1764, 0.698, 0.9216}
\pgfplotsset{%
  colormap={PiYG}{%
    rgb=(0.93725490196, 0.54117647058, 0.38431372549)%
    rgb=(0.96862745098, 0.96862745098, 0.96862745098)%
    rgb=(0.40392156862, 0.66274509803, 0.81176470588)%
  }%
}

\def\pgfplotsshowcolormap#1{%
    \pgfplotscolormapifdefined{#1}{\relax}{%
        \pgfplotsset{colormap/#1}%
    }%
    \pgfplotscolormaptoshadingspec{#1}{1cm}\result
    \def\tempb{\pgfdeclarehorizontalshading{tempshading}{0.2cm}}%
    \expandafter\tempb\expandafter{\result}%
    \pgfuseshading{tempshading}%
}

\newlength{\savedtabcolsep}
\setlength{\savedtabcolsep}{\tabcolsep}
\setlength{\tabcolsep}{1ex}

\newcommand{\midsepremove}{\aboverulesep=0mm \belowrulesep=0mm}%
\usepackage{listofitems}
\setsepchar{ }

%\newcommand{\ApplyGradient}[1]{%
%  \pgfmathsetmacro{\Val}{#1}%
%  \pgfmathsetmacro{\MiddleVal}{\MinVal + (\MaxVal - \MinVal) / 2}%
%  \ifdim #1 pt > \MiddleVal pt%
%      \pgfmathsetmacro{\PercentColor}{max(min(100.0*(#1 - \MiddleVal)/(\MaxVal-\MiddleVal),100.0),0.00)}%
%      \edef\HeatCell{\noexpand\cellcolor{highcolor!\PercentColor!centralcolor}}%
%      \HeatCell$#1$%
%  \else
%      \pgfmathsetmacro{\PercentColor}{max(min(100.0*(\MiddleVal - #1)/(\MiddleVal-\MinVal),100.0),0.00)}%
%      \edef\HeatCell{\noexpand\cellcolor{lowcolor!\PercentColor!centralcolor}}%
%      \HeatCell$#1$%
%  \fi%
%}
%
%\newcommand{\ApplyGradientReverse}[1]{%
%  \pgfmathsetmacro{\Val}{#1}%
%  \pgfmathsetmacro{\MiddleVal}{\MinVal + (\MaxVal - \MinVal) / 2}%
%  \ifdim #1 pt > \MiddleVal pt%
%      \pgfmathsetmacro{\PercentColor}{max(min(100.0*(#1 - \MiddleVal)/(\MaxVal-\MiddleVal),100.0),0.00)}%
%      \edef\HeatCell{\noexpand\cellcolor{lowcolor!\PercentColor!centralcolor}}%
%      \HeatCell$#1$%
%  \else
%      \pgfmathsetmacro{\PercentColor}{max(min(100.0*(\MiddleVal - \Val)/(\MiddleVal-\MinVal),100.0),0.00)} %
%      \edef\HeatCell{\noexpand\cellcolor{highcolor!\PercentColor!centralcolor}}%
%      \HeatCell$#1$%
%  \fi%
%}
\newcommand{\ApplyGradient}[1]{%
    \readlist*\mylist{#1}%
  \pgfmathsetmacro{\Val}{\mylist[1]}
  \pgfmathsetmacro{\MiddleVal}{\MinVal + (\MaxVal - \MinVal) / 2}%
  \ifdim \Val pt > \MiddleVal pt%
      \pgfmathsetmacro{\PercentColor}{max(min(100.0*(\Val - \MiddleVal)/(\MaxVal-\MiddleVal),100.0),0.00)}%
      \edef\HeatCell{\noexpand\cellcolor{highcolor!\PercentColor!centralcolor}}%
      \HeatCell$#1$%
  \else
      \pgfmathsetmacro{\PercentColor}{max(min(100.0*(\MiddleVal - \Val)/(\MiddleVal-\MinVal),100.0),0.00)}%
      \edef\HeatCell{\noexpand\cellcolor{lowcolor!\PercentColor!centralcolor}}%
      \HeatCell$#1$%
  \fi%
}

\newcommand{\ApplyGradientReverse}[1]{%
    \readlist*\mylist{#1}%
  \pgfmathsetmacro{\Val}{\mylist[1]}%
  \pgfmathsetmacro{\MiddleVal}{\MinVal + (\MaxVal - \MinVal) / 2}%
  \ifdim \Val pt > \MiddleVal pt%
      \pgfmathsetmacro{\PercentColor}{max(min(100.0*(\Val - \MiddleVal)/(\MaxVal-\MiddleVal),100.0),0.00)}%
      \edef\HeatCell{\noexpand\cellcolor{lowcolor!\PercentColor!centralcolor}}%
      \HeatCell$#1$%
  \else
      \pgfmathsetmacro{\PercentColor}{max(min(100.0*(\MiddleVal - \Val)/(\MiddleVal-\MinVal),100.0),0.00)} %
      \edef\HeatCell{\noexpand\cellcolor{highcolor!\PercentColor!centralcolor}}%
      \HeatCell$#1$%
  \fi%
}

\newcolumntype{\C}[2]{>{\def\MinVal{#1}\def\MaxVal{#2}\collectcell\ApplyGradient}c<{\endcollectcell}}

\newcolumntype{\CR}[2]{>{\def\MinVal{#1}\def\MaxVal{#2}\collectcell\ApplyGradientReverse}c<{\endcollectcell}}


% Code to generate part-box
% Reference: https://tex.stackexchange.com/questions/66820/how-to-create-highlight-boxes-in-latex
\usepackage[many]{tcolorbox}
\newtcolorbox{myboxi}[2][]{
  breakable,
  title=#1,
  colback=white,
  colbacktitle=white,
  coltitle=black,
  fonttitle=\itshape,
  bottomrule=-0.1pt,
  toprule=-0.2pt,
  leftrule=5pt,
  rightrule=0pt,
  titlerule=0pt,
  arc=0pt,
  boxsep=1mm,
  outer arc=0pt,
  colframe=#2,
}

% end of packages from severe
% If you comment hyperref and then uncomment it, you should delete
% egpaper.aux before re-running latex.  (Or just hit 'q' on the first latex
% run, let it finish, and you should be clear).
\usepackage[pagebackref=true,breaklinks=true,letterpaper=true,colorlinks,bookmarks=false]{hyperref}

\iccvfinalcopy % *** Uncomment this line for the final submission

%\def\iccvPaperID{1703} % *** Enter the ICCV Paper ID here
\def\httilde{\mbox{\tt\raisebox{-.5ex}{\symbol{126}}}}

% Pages are numbered in submission mode, and unnumbered in camera-ready
\ificcvfinal\pagestyle{empty}\fi

\begin{document}

%%%%%%%%% TITLE
%\title{Tubelet contrastive video representation learning enables data-efficiency and better generalization}
\title{Tubelet-Contrastive Self-Supervision for Video-Efficient Generalization}
%\title{Better generalization, data efficiency  with Tubelet contrastive video representation learning }


\author{First Author\\
Institution1\\
Institution1 address\\
{\tt\small firstauthor@i1.org}
% For a paper whose authors are all at the same institution,
% omit the following lines up until the closing ``}''.
% Additional authors and addresses can be added with ``\and'',
% just like the second author.
% To save space, use either the email address or home page, not both
\and
Second Author\\
Institution2\\
First line of institution2 address\\
{\tt\small secondauthor@i2.org}
}

\author{Fida Mohammad Thoker, Hazel Doughty, Cees  G. M.  Snoek\\
University of Amsterdam}


\maketitle
% Remove page # from the first page of camera-ready.
%\ificcvfinal\thispagestyle{empty}\fi

% Commented our for Appendix submission
\begin{abstract}
The current study investigated possible human-robot kinaesthetic interaction using a variational recurrent neural network model, called PV-RNN, which is based on the free energy principle.
Our prior robotic studies using PV-RNN showed that the nature of interactions between top-down expectation and bottom-up inference is strongly affected by a parameter, called the meta-prior, which regulates the complexity term in free energy.
% The current study examines how the behaviours of robots alter by changing the meta-prior $w$ in human-robot kinaesthetic interaction.
The current study examines how changing the meta-prior $w$ in the interaction phase affects the counter force generated when an experimenter attempts to induce movement pattern transitions familiar to the robot through its prior training.
The study also compares the counter force generated when trained transitions are induced by a human experimenter and when untrained transitions are induced.
Our experimental results indicated that (1) the human experimenter needs more/less force to induce trained transitions when $w$ is set with larger/smaller values, (2) the human experimenter needs more force to act on the robot when he attempts to induce untrained as opposed to trained movement pattern transitions.
Our analysis of time development of essential variables and values in PV-RNN during bodily interaction clarified the mechanism by which gaps in actional intentions between the human experimenter and the robot can be manifested as reaction forces between them.


%% Hiroki writing 2022-11-4
%Current study investigates the dynamics of the latent states during human-robot kinaesthetic interaction using PV-RNN.
%We have achieved to observe and analyse the internal state of an RNN model based on the free energy principle, during real-time human-robot interaction.
%Essential characteristics observed in the previous study of this variational recurrent neural network model, PV-RNN, is that by changing a meta prior $w$, the balance between the top-down intention and the bottom-up perceptual reality changes.
%In the current study, we examined how changing the weighting parameter $w$ between accuracy and complexity in free energy principle affects the humanoid robot's behaviour through human-robot interaction. We have conducted some human-robot kinaesthetic interaction experiments with various $w$ and quantitatively analysed the latent variable and the force applied to the humanoid robot. We have observed that the force required to change the robot's intention has increased, both when the top-down intention was strengthened by changing the $w$ and when corresponding switch of its primitive was against the experience of the RNN during its training. The study confirms through quantitative analysis that by increasing or decreasing the $w$ in PV-RNN, humanoid robot leads or follows the human counterpart during the human-robot kinaesthetic interaction.

\begin{comment}
Comment from Jun #2
・最後にQualitativeな結果(インパクト)が欲しい
・Current study investigates the problem on~と書き出すのが一般的
・最初の一文と最後の一文を対応させる
・最後の一文はもう少しAbstractかつ包括的に
\end{comment}

\begin{comment}
Comment from Jun #1
We investigated how the kinaesthetic human-robot interaction can affect the internal state of a model based on the free energy principle. 
=> how the internal state is affected is not the most important point in this study. This part should be rewritten.

The key function of this variational recurrent neural network model, PV-RNN, is that by changing a meta prior $w$, it takes a balance between the "complexity” term and the ”accuracy” term which corresponds to a top-down intention and a bottom-up perceptual reality in the free energy principle, respectively. 
=> This is not key function of PV-RNN. It is an essential characteristics observed in the previous study. The grammar after $w$ is something strange. Rewrite these.

This research has conducted a human-robot interaction experiment with a robotic agent in a kinaesthetic sense.
=> The sentence is not good. "in a kinaesthetic sense" is grammatically wrong.
MODIFIED => "In the current study human-robot interaction experiments using the kinaesthetic sense were conducted."

We investigated that when human forces the agent to switch primitives from one to another, larger force was required both when the human intention is conflictive against the top-down the intention of the agent and when the agent has a stronger top-down intention by modifying the $w$.
=> You should write the essential results of the experiments rather than what we investigated and also how these results could contribute to the studies on human-robot interaction.
\end{comment}

\end{abstract}
% \begin{figure}[t]
%     % \begin{subfigure}{1\linewidth}
%     %   \centering
%     % %   \includegraphics[width=1\linewidth]{figs/fig_1_moti_textattn.pdf}  
%     % %   \includegraphics[width=1\linewidth]{figs/fig_1_moti_textattn_v2.pdf}  
%     %   \includegraphics[width=1\linewidth]{figs/fig_1_moti_textattn_v5.pdf}  
%     %   \vspace{-0.5cm}
%     %     \caption{Amount of attention added to each video clip from the source video and query text in the self-attention layers of Moment-DETR encoder.}
%     %     % \caption{Distribution of attention for source and query in Moment-DETR encoder}
%     %     % Visualization of video clip's self-attention score in Moment-DETR encoder.
%     %   \label{fig:fig1_text_attn_ex}
%     % \end{subfigure}%\hfill% or  or \hspace{0.3\textwidth}
%     \vspace{0.2cm}
%     % \begin{subfigure}{1\linewidth}
%       \centering
%     %   \includegraphics[width=1\linewidth]{figs/fig1_moti_negattn.pdf}  
%       \includegraphics[width=1\linewidth]{figs/fig1_moti_negattn_v3.pdf}  
%       \vspace{-0.4cm}
%     %   \caption{Correspondence of saliency scores on the relevance between video clips and the text query.}
%     % \caption{Predicted saliency scores against the video relevant positive query and video irrelevant negative query}
%       \label{fig:fig1_neg_attn_ex}
%     % \end{subfigure}%\hfill% or  or \hspace{0.3\textwidth}
%     \caption{
%     % 원준 원본
%     % (a) Comparison between attention scores of source and query for each video clip~(We sum the attention scores from video and text). 
%     % We observe that the attention scores are dominated by other clips in the source video. 
%     % Text queries do not account for much attention regardless of the relevance to the video clips.
%     % \textbf{(a)} Inspection of the query dependency in Moment-DETR encoder.
%     % % We visualize the attention score of video tokens in the transformer encoder and observe that text query accounts for only a low portion of attention.
%     % % This tendency occurs regardless of the relevance between the text query and video clips. 
%     % We visualize the attention score of video tokens in the transformer encoder and observe 1) text query only accounts for a low portion of attention, and 2) relevance between video-query pair does not affect the attention scores ratio of text.
%     \textbf{(b)} Comparison of highlight-ness when relevant and non-relevant queries are input.
%     As observed in , existing work only uses queries to play an insignificant role, thereby may not be capable of detecting false queries and considering the video-query relevance even when the problem in (a) is resolved. 
%     % \SE{} % 이 부분이 "not capable of" 란 용어가 세다는 피드백이 있는 듯 합니다. 이러한 능력이 없다는 것은 굉장히 강한 어조인거 같기는 하고, 이러한 경우들이 종종 있다거나 좀 약화시킬 필요가 있어보이긴 하네요.
%     On the other hand, our QD-DETR yields a query-dependent representation that the relevance between the source video and query text is updated in the saliency scores.
%     There is a large gap between positive and negative saliency scores, and scores are consistent since the clips are all highly correlated to others.
%     }
%     \label{fig:motivation_ex}
%     % \captionsetup{belowskip=13pt}
%     % \setlength{\belowcaptionskip}{-10pt}
% \end{figure}
\begin{figure}
    \centering
    \includegraphics[width=1\linewidth]{figs/fig1_moti_negattn_1111.pdf}
    % \includegraphics[width=1\linewidth]{figs/fig1_moti_negattn_1109.pdf}
    % \includegraphics[width=1\linewidth]{figs/fig1_moti_negattn_stat.pdf}
    \vspace{-0.6cm}
    \caption{
        % \SE{} % 수정 필요
        Comparison of highlight-ness~(saliency score) when relevant and non-relevant queries are given.
        We found that the existing work only uses queries to play an insignificant role, thereby may not be capable of detecting negative queries and video-query relevance; saliency scores for clips in ground-truth~(GT) moments are low and equivalent for positive and negative queries.
        % This also results in mispredicted moments when ground-truth~(GT) moment is dominated by clips unrelated to GT since their prediction is highly focused on the video.
        % \SE{} % 여기 한번 더 보면 좋을 듯 합니다. GT moment에 unrelated한 clip이 많으면? label이 틀렷을 경우를 말씀하시는건지?
        % As observed in saliency graph, existing work only uses queries to play an insignificant role, thereby may not be capable of detecting false queries and considering the video-query relevance.
        On the other hand, query-dependent representations of QD-DETR result in corresponding saliency scores to the video-query relevance and precisely localized moments.
        % On the other hand, our QD-DETR yields a query-dependent representation that the
        % saliency scores are in accordance with the relevance between the video and query.
        % text is in accordance with the saliency scores.
        % There is a large gap between positive and negative saliency scores, and scores are consistent since the clips are all highly correlated to others.
}
    \label{fig:motivation_ex}
\end{figure}


\section{Introduction}
% 원준 원본
% Along with the advance of digital devices and platforms, video is now one of the most desired data type for consumers. However, although the large information capacity of videos may be beneficial in many aspects, e.g., informative and entertaining, on the contrary perspective, videos are time-consuming, and hard to search for desirable moments. 
% This has led many creators to use extra manpower to crop and edit the video to generate highlight clips to gain the consumer’s attention.
Along with the advance of digital devices and platforms, video is now one of the most desired data types for consumers~\cite{apostolidis2021video,wu2017deep}.
% SE: Video aware deep learning application & survey papers?
Although the large information capacity of videos might be beneficial in many aspects, e.g., informative and entertaining, inspecting the videos is time-consuming, so that it is hard to capture the desired moments~\cite{anne2017localizing,apostolidis2021video}. 
% This has led many creators to use extra manpower to crop and edit the video to generate highlight clips to gain the consumer’s attention.


% On the other side, 
Indeed, the need to retrieve user-requested or highlight moments within videos is greatly raised.
Numerous research efforts were put into the search for the requested moments in the video~\cite{anne2017localizing, gao2017tall, liu2015multi, escorcia2019temporal} and summarizing the video highlights~\cite{zhang2016video, mahasseni2017unsupervised, badamdorj2022contrastive, wei2022learning}.
% Numerous research efforts were put into the search for the requested moments in the video~\cite{anne2017localizing, gao2017tall, liu2015multi, escorcia2019temporal}, summarizing the video to generate highlights was another popular topic~\cite{zhang2016video, mahasseni2017unsupervised, badamdorj2022contrastive, wei2022learning}.
Recently, Moment-DETR~\cite{momentdetr} further spotlighted the topic by proposing a QVHighlights dataset that enables the model to perform both tasks, retrieving the moments with their highlight-ness, simultaneously.

% 원준 원본
% To detect the desired moments, previous works employed transformer encoder-decoder architectural designs to fuse the text query into the video representations. Moment-DETR~\cite{mDETR} modified detection transformer to process capture the moment as a set, and UMT~\cite{umt} implemented transformer decoder as to output clip-wise saliency. 
% Yet to their outstanding breakthroughs in the literature of moment retrieval with the seminal architectures, their limitation is that the role of the given text query is insignificant in representing the query-conditioned video representation; the attention mechanism of moment DETR is not explicitly conditioned on the text query, and the text query is conditioned on multi-modal clips where the differences between the clips are smoothed after encoding process in UMT.



% \begin{figure}[t]
% \centering
%     \begin{subfigure}[l]{0.37\linewidth}
%       \centering
%       \vspace{0.20cm}
%     %   \includegraphics[width=1\linewidth]{figs/fig_1_moti_textattn.pdf}  
%     %   \includegraphics[width=1\linewidth]{figs/fig_1_moti_textattn_v2.pdf}  
%       \includegraphics[width=1\linewidth]{figs/fig1_moti_violin_a.pdf}  
%       \vspace{-0.60cm}
%     %   \caption{text attention}
%         \caption{Importance of queries in video representation}
%       \label{fig:fig1_text_attn}
%     \end{subfigure}%\hfill% or  or \hspace{0.3\textwidth}
%     \vspace{0.2cm}
%     \begin{subfigure}[r]{0.61\linewidth}
%       \centering
%     %   \includegraphics[width=1\linewidth]{figs/fig1_moti_negattn.pdf}  
%       \includegraphics[width=1\linewidth]{figs/fig1_moti_violin_b.pdf}  
%     %   \caption{neg attention}
%         % \caption{Relation between the highlight-ness and the relevance between videos and query texts.}
%         \caption{Highlight-ness~(saliency) histogram of positive and negative video-query pairs\SE{}}
%       \label{fig:fig1_neg_attn}
%     \end{subfigure}%\hfill% or  or \hspace{0.3\textwidth}
%     % \vspace{-0.2cm}
%     \caption{Overall statistics for attention scores in Fig.~\ref{fig:motivation_ex} in QVHighlights dataset. 
%     (a) For the attention scores that measure how much the text query is generally involved in video representation, we use violin plots to show the probability density. We plot the score for each layer in the encoder.
%     % (b) Using the histogram, we compare how the baseline and QD-DETR yield different salient scores given the positive and negative video-text pairs.
%     (b) Saliency histogram shows the distributional gap between positive and negative video-text query pairs of baseline~(Moment-DETR) and proposed QD-DETR.\SE{}
%     }
%     \label{fig:motivation}
%     % \captionsetup{belowskip=13pt}
%     % \setlength{\belowcaptionskip}{-10pt}
% \end{figure}

% \begin{figure}[t]
% \centering

%     \begin{subfigure}[r]{1\linewidth}
%       \centering
%       \hspace{-0.2cm}
%     %   \includegraphics[width=1\linewidth]{figs/fig1_moti_negattn.pdf}  
%       \includegraphics[width=1.1\linewidth]{figs/fig1_moti_violin_a_v2.pdf}  
%     %   \caption{neg attention}
%         % \caption{Relation between the highlight-ness and the relevance between videos and query texts.}
%         \vspace{-0.5cm}
%         % \caption{Saliency histogram of positive and negative video-query pairs}
%         \caption{We plot the histograms and its average value~(dotted line) to compare saliency scores when true and false text queries are given for each method. (left) Since the video representations do not include much textual information, both the true and false queries yield similar saliency scores. (Middle) Even when the video representation is enforced to be updated with the textual information, the issue is not much resolved. (Right) By extracting discriminative features in the text query, distributions are differentiated.
%         % \SE{} % R1@0.5 설명
%         Also, R1@0.5 indicates evaluation metric, Recall at 1 with IoU 0.5 threshold on QVhighlight \textit{val} set.
%         }
%       \label{fig:fig1_neg_attn}
%     \end{subfigure}%\hfill% or  or \hspace{0.3\textwidth}
%     \\
%     \begin{tabular}{cc}
%     \hspace{-0.2cm}
%         \begin{minipage}{.4\linewidth}
%             \begin{subfigure}[l]{1\linewidth}
%               \centering
%             %   \vspace{0.20cm}
%             %   \includegraphics[width=1\linewidth]{figs/fig_1_moti_textattn.pdf}  
%             %   \includegraphics[width=1\linewidth]{figs/fig_1_moti_textattn_v2.pdf}  
%               \includegraphics[width=1\linewidth]{figs/fig1_moti_violin_a.pdf}  
%               \vspace{-0.60cm}
%             %   \caption{text attention}
%                 \caption{Importance of queries in video representation}
%               \label{fig:fig1_text_attn}
%             \end{subfigure}%\hfill% or  or \hspace{0.3\textwidth}
%         \end{minipage}
        
%         \begin{minipage}{.6\linewidth}
%             \vspace{-0.2cm}
%             \caption{Overall statistics of Fig.~\ref{fig:motivation_ex} in QVHighlights dataset. 
%             (a) Saliency histogram shows the distributional gap between positive and negative video-text query pairs.
%             % (a) For the attention scores that measure how much the text query is generally involved in video representation, we use violin plots to show the probability density. We plot the score for each layer in the encoder.
%             % (b) Using the histogram, we compare how the baseline and QD-DETR yield different salient scores given the positive and negative video-text pairs.
%             % (b) Text ratio in self-attention layer to  of Moment-DETR
%             % (b) Ratio of text when representing video tokens in self-attention of Moment-DETR.
%             % (b) Magnitude of attention text query involved.
%             % (b) Attention score of video tokens
%             % (b) Magnitude of text query to refine the video tokens in self-attention layer of Moment-DETR.
%             (b) Probability density depicting the weight of the text query in attention score for video clips. Scores are from the self-attention layers in Moment-DETR encoder.
%             % (b) The text query ratio in attention score of video clips (Self-attention layer in Moment-DETR encoder). We use violin plots to show probability density.
%             % 텍스트 쿼리가, 비디오 피쳐에 얼만큼 attend 하는지
%             }
%         \end{minipage}
    
%     \end{tabular}
%     \vspace{-0.5cm}
%     \label{fig:moti}
%     % \captionsetup{belowskip=13pt}
%     % \setlength{\belowcaptionskip}{-10pt}
% \end{figure}


% \begin{figure}
%     \centering
%     % \includegraphics[width=1\linewidth]{figs/fig1_moti_negattn_1109.pdf}
%     \includegraphics[width=1\linewidth]{figs/fig1_moti_negattn_stat_v2.pdf}
%     \vspace{-0.8cm}
%     \caption{
%         Histogram of saliency when the positive and negative queries are given. We plot the histograms and its average value~(dotted line) to compare saliency scores when relevant~(positive) and irrelevant~(negative) text queries are given for each method. (Left) Since the video representations do not properly reflect textual information, both the positive and negative queries yield similar saliency scores. 
%         % (Middle) Even when the video representation is enforced to be updated with the textual information, the issue is not much resolved. 
%         (Right) By representing video clips in query-dependent manner, distributions are differentiated.
%     }
%     \vspace{-0.6cm}
%     \label{fig:motivation}
% \end{figure}


% One of the demanding task is moment retrieval task, which is detecting the desired moments from the given query, typically the text query.
When describing the moment, one of the most favored types of query is the natural language sentence~(text)\cite{anne2017localizing}. 
While early methods utilized convolution networks~\cite{zhang2020learning, gao2021fast, wang2020temporally}, recent approaches have shown that deploying the attention mechanism of transformer architecture is more effective to fuse the text query into the video representation.
% To handle these modalities, previous works simply employed the attention mechanism of transformer architecture to fuse the text query into the video representation.
For example, Moment-DETR~\cite{momentdetr} introduced the transformer architecture which processes both text and video tokens as input by modifying the detection transformer~(DETR), and UMT~\cite{umt} proposed transformer architectures to take multi-modal sources, e.g., video and audio. 
Also, they utilized the text queries in the transformer decoder.
Although they brought breakthroughs in the field of MR/HD with seminal architectures, they overlooked the role of the text query.
To validate our claim, we investigate the Moment-DETR~\cite{momentdetr} in terms of the impact of text query in MR/HD~(Fig.\ref{fig:motivation_ex}).
Given the video clips with a relevant positive query and an irrelevant negative query, we observe that the baseline often neglects the given text query when estimating the query-relevance scores, i.e., saliency scores, for each video clip.
% the output saliency score, i.e. query-relevance scores.
% Based on the observation, we traced the actual saliency prediction of the model against both the video-relevant query and the irrelevant dummy one where we find that the baseline often neglects the given text query when estimating the query-relevance scores of video clips.
% For example, in Fig.~\ref{fig:motivation_ex}, saliency scores are not affected even when the query is substituted with the dummy.
% % General statistics for Fig.~\ref{fig:motivation_ex} is shown in Fig.~\ref{fig:motivation}. 
% General statistics corresponding to Fig.~\ref{fig:motivation_ex} are also shown in Fig.~\ref{fig:motivation}.



% The limitation of the concrete baseline~\cite{momentdetr} is inspected in two different aspects; 1) Utilization of text-query in the encoding process and 2) the output saliency score, i.e. query-relevance scores.
% Firstly, we visualize the attention score when video clips are given as a query in self-attention. 
% We observe that the text queries have relatively small impacts compared to other video features, as shown in Fig.~\ref{fig:fig1_text_attn_ex}.
% That is, the text does not account for much in representing every video clip, although the goal of MR/HD is to detect query-relevant moments.
% Based on the observation, we traced the actual saliency prediction of the model against both the video-relevant query and the irrelevant dummy one where we find that the baseline often neglects the given text query when estimating the query-relevance scores of video clips.
% For example, in Fig.~\ref{fig:motivation_ex}, saliency scores are not affected even when the query is substituted with the dummy.
% % General statistics for Fig.~\ref{fig:motivation_ex} is shown in Fig.~\ref{fig:motivation}. 
% General statistics are also shown in Fig.~\ref{fig:motivation}.

% Consequently, in Fig.~\ref{fig:fig1_neg_attn_ex}~(b), we found that the baseline often neglects the given text query when estimating the query-relevance scores of video clips; 
% For example, 


% We validate the previous work sometimes neglects the given query when estimating the saliency of video clips.
% For example, there is an example that the saliency scores from positive and negative queries cannot be distinguishable, as shown in Fig.~\ref{fig:fig1_neg_attn_ex}.
% % 우리는 추가로 text attention을 추가도 해봤지만, 효과가 있긴 했으나, still 이슈가 있는 것을 확인하였다?
% % Still, we observe that assuring the high attendance of text queries does not resolve the overlap which motivates us to question the quality of the naive use of task-agnostic text representation~\cite{momentdetr, umt}.
% We found that introducing the text-attention for ensuring the high attendance of text queries relieve the overlap, but there still be a severe overlap.


% To validate their limitations, we inspect the impacts of text queries in the concrete baseline~\cite{momentdetr} with the two different aspects, 1) tendency of attention in self-attention layer and 2) saliency score, i.e. query-relevance scores. \SE{} % attention 이 갑자기 등장하는가?
% Firstly, we visualize the attention score when video clips are given as a query in self-attention. We observe the text queries have relatively low attention scores compared to the video features, as shown in Fig.~\ref{fig:fig1_text_attn_ex}.
% That is, the text does not account for much in representing every video clip, although the goal of MR/HD is to detect query-relevant moments.
% Based on this observation, we trace the actual saliency prediction of the model against both positive and negative text queries.
% We validate the previous work sometimes neglects the given query when estimating the saliency of video clips.
% For example, there is an example that the saliency scores from positive and negative queries cannot be distinguishable, as shown in Fig.~\ref{fig:fig1_neg_attn_ex}.
% % 우리는 추가로 text attention을 추가도 해봤지만, 효과가 있긴 했으나, still 이슈가 있는 것을 확인하였다?
% % Still, we observe that assuring the high attendance of text queries does not resolve the overlap which motivates us to question the quality of the naive use of task-agnostic text representation~\cite{momentdetr, umt}.
% We found that introducing the text-attention for ensuring the high attendance of text queries relieve the overlap, but there still be a severe overlap.



% Thus, we 
% query dependency를 높이기 위해 
% Cross-attention? text-attention? detailed explanation on text-attention should be needed?
% By handling these two issues, we find that more precise retrieval can be achieved.
% 
% 
%
% By projecting video-discriminative text features with high text attendance to source video, we f 
% We also find the need to improve the quality of query features since assuring high text attendance also results in...
% pairs are not finetuned to be discriminative that even the similarity within the pairs does not reflect the relevance between the query and the video clips.
% General statistics for Fig.~\ref{fig:motivation_ex} is shown in Fig.~\ref{fig:motivation}. 
% \SE{} % 이거 ??로 뜨는데, 위처럼 figure 그리면 label이 안되는걸까요
% \SE{}
% 형님 아래 사항 생각 좀 해보는게 좋을 거 같아요.
% fig 1. (a) 그림만 봤을 때 모든 clip에 대해 text attention이 일정이상 존재하긴 하니까, 뭔가 not assured to be conditioned가 와닿지 않는거 같아요.
% + 왜 text가 항상 attend 해야하나?
% not assured to be conditioned --> text shows relatively low affects compared to video 같이 실제 나타난 현상까지 같이 적으면 어떨까 싶어요.
% fig 1. (b) 덜 반영한다?

% \SU{}
% 일단 text가 attend 잘 되어야 한다는 것에 좀 궁금점이 생깁니다. 결국에는 text와 관련있는 frame들을 attend해서 higlight를 찾아야 하는게 아닐까요? 그리고, 현제 저희의 모델 구조상 text query가 Key와 Value로 거의 활용되고 있는데 그렇다면 결국에는 해당 모델은 text에 대한 attention이 전혀 없다고 봐도 무방하지 않을까요? 그런 면에서 text attention을 강조하는게 좀 걸리긴 합니다.

% Specifically, the text query is not assured to be explicitly conditioned on every clip of the video, and as the query texts are evenly treated, discriminative keywords may not be spotlighted.
% attention mechanism of Moment-DETR is not explicitly conditioned on the text query as shown in Fig~\ref{}(d), and in UMT, the text are only used for conditioning the queries while the video representation are refined itself by self-attention.

% \begin{figure}[t]
%     \begin{subfigure}{1\linewidth}
%       \centering
%     %   \includegraphics[width=1\linewidth]{figs/fig_1_moti_textattn.pdf}  
%     %   \includegraphics[width=1\linewidth]{figs/fig_1_moti_textattn_v2.pdf}  
%       \includegraphics[width=1\linewidth]{figs/fig_1_moti_textattn_v4.pdf}  
%       \vspace{-0.5cm}
%     %   \caption{text attention}
%         \caption{Distribution of attention scores in Moment-DETR encoder}
%       \label{fig:fig1_text_attn}
%     \end{subfigure}%\hfill% or  or \hspace{0.3\textwidth}
%     \vspace{0.2cm}
%     \begin{subfigure}{1\linewidth}
%       \centering
%     %   \includegraphics[width=1\linewidth]{figs/fig1_moti_negattn.pdf}  
%       \includegraphics[width=1\linewidth]{figs/fig1_moti_negattn_v2.pdf}  
%       \vspace{-0.5cm}
%     %   \caption{neg attention}
%         \caption{Saliency score against positive and negative text queries}
%       \label{fig:fig1_neg_attn}
%     \end{subfigure}%\hfill% or  or \hspace{0.3\textwidth}
%     \vspace{0.2cm}
%     \begin{subfigure}{1\linewidth}
%       \centering
%     %   \includegraphics[width=1\linewidth]{figs/fig1_moti_violin.pdf}  
%       \includegraphics[width=1\linewidth]{figs/fig1_moti_violin_v2.pdf}  
%       \vspace{-0.5cm}
%       \caption{violin}
%       \label{fig:fig1_violin}
%     \end{subfigure}%\hfill% or  or \hspace{0.3\textwidth}
%     \vspace{-0.2cm}
%     \caption{(a) 1. portion of text attention vs. video attention 2. relation with text query and content (e.g. fg, bg) of clip seems not to affect the attention score
%     (b) 1. high variability even though entire clips are highly correlated with the given text query 2. positive and negative query makes overlaps on saliency score distribution
%     (3) actual distribution on validation dataset.}
%     \label{fig:motivation}
%     % \captionsetup{belowskip=13pt}
%     % \setlength{\belowcaptionskip}{-10pt}
% \end{figure}

To this end, we propose Query-Dependent DETR~(QD-DETR) that produces query-dependent video representation.
% Our key focus is to ensure each clip in predicted moments is explicitly conditioned by the query, particularly on the video-descriptive portion of the text query.
% Our key focus is to ensure that query-relevant clips are predicted by enforcing each clip to be explicitly conditioned by the query.
%Our key focus is to ensure that the model prediction for each clip is highly relevant to the query.
Our key focus is to ensure that the model's prediction for each clip is highly dependent on the query.
% by enforcing each clip to be explicitly conditioned by the query. :)
% hmm...
% \SE {} % "query-relevant clips are predicted" 이 문장이 좀 애매한거 같습니다. relevant 클립을 놓지지 않고 찾는 것을 보장한다? 이런 느낌인지 아니면 높은 saliency 를 주는게 목적이다? model prediction이 query-relevance를 반영하는 것을 보장한다?
% Our key focus is to ensure that the model prediction reflects query-relevance of clips by enforcing each clip to be explicitly conditioned by the query.
First, to fully utilize the contextual information in the query, we revise the transformer encoder to be equipped with cross-attention layers at the very first layers.
% 상익's thought :  single video - query간의 관계만 고려 - 같은 word가 더 많이 쓰이는 것을 보고 
% 교수님's thought : neg pair 를 쓰면 쿼리를 보지 않고서는 video clip간만 고려하는 것이 사라짐. 왜냐면 0으로 내보내야 하기 때문. --> SE: relative difference 만 고려하다가, 
By inserting a video as the query and a text as the key and value of the cross-attention layers, our encoder enforces the engagement of the text query in extracting video representation.
% 원준 교수님 코멘트 반영해서 다시
Then, in order to not only inject a lot of textual information into the video feature but also make it fully exploited, we leverage the negative video-query pairs generated by mixing the original pairs.
Specifically, the model is learned to suppress the saliency scores of such  negative~(irrelevant) pairs.
Our expectation is the increased contribution of the text query in prediction since the videos will be sometimes required to yield high saliency scores and sometimes low ones depending on whether the text query is relevant or not.
% \SE{}
% learns to?
% By suppressing the saliency scores of the irrelevant video-query pairs, the model learns to spotlight only the video-specific discriminative words in the query.
% % \SE{} % ====================== 상익 수정 ========================
% However, this architectural design still lacks the capability of identifying the video-descriptive keywords in the query.
% % However, this architectural design still lacks in identifying proper query relevance.
% This is because the current training scheme only focuses on the interactions of video and clips within a single video while neglecting information shared throughout the entire video.
% % We argue the problem of the current training scheme that only focuses on distinguishing the clips in a single video while neglecting information shared throughout the entire video.
% Therefore, we leverage the negative video-query relationships to enhance the capability of identifying the contextual similarity of query and video clips.
% 
% 원준 원본 
% However, this architectural design heavily relies on the quality of the text query.
% Therefore, we leverage the negative video-query relationships to enable the model to emphasize key corresponding query features.
% By suppressing the saliency scores of the irrelevant video-query pairs, the model learns to spotlight only the video-specific discriminative words in the query.
% =========================================================
Lastly, to apply the dynamic criterion to mark highlights for each instance, we deploy a saliency token to represent the entire video and utilize it as an input-adaptive saliency criterion. 
With all components combined, our QD-DETR produces query-dependent video representation by integrating source and query modalities.
This further allows the use of positional queries~\cite{dabdetr} in the transformer decoder.
% Furthermore, we can exploit the advanced DETR decoder architectures using the positional information, e.g., DAB-DETR, since our encoded tokens consist of identical position representations from a single modality.
% \SE{} % ====================== 상익 수정 ========================
% Furthermore, we can exploit the advanced DETR decoder architectures using the positional information, e.g., DAB-DETR, since our video clip tokens consist of identical position representations from a single modality.
% 원준 원본
% It also enables the use of advanced DETR decoder architectures, e.g., DAB-DETR, for the first time, as these works exploit the position information within a single modality.
% =========================================================
Overall, our superior performances over the existing approaches validate the significance of the role of text query for MR/HD.
% Our extensive experiments on QVHighlights, TVSum, and Charades-STA datasets validate the significance of considering the role and the quality of text query.

% All components combined with dynamic anchor moments for the query of decoder, our FOQUE fosters the query-dependent video representation, thereby making the 
% All components combined, our modified transformer encoding process fosters the query-dependent video representation thereby achieving the state-of-the-art results on various benchmarks of moment-retrieval and highlight detection.
	
% -	Video Platform & Streamer & Consumer의 증가. 
% Video는 다른 데이터 타입보다 정보가 많아 유용하지만, 이는 다른 말로 해석하면 video를 보는 것은 time-consuming 하고, 원하는 것을 찾아보기에는 힘들 수 있음.
% 따라서, 많은 매체에서는 사람들의 더 많은 이목을 끌기 위해 highlight 비디오라는 것을 편집하여 공유도 함.
% 하지만, highlight video를 만들기 위해 사람의 노력이 필요한 현 시점에서, This spotlights the need to retrieve the user-requested / Highlight moments in the video.

% -	이전에도 이러한 문제를 해결하기 위해 (asdfasdf) for moment retrieval, (asdfasdf) for highlight detection 등이 제안 되었지만, 이들은 비디오의 특정 영역을 찾는다는 공통된 목적을 가지고 있으면서도, 데이터 셋의 한계로 인해 따로 연구되었음. 이를 문제 삼으며, 최근에는 두 task를 동시에 학습할 수 있는 dataset이 소개 되었는데, 컴퓨터비전에서 최근 각광을 받고 있는 Transformer 모델 도입과 함께 큰 발전을 거듭하고 있음.

% -	구체적으로, 이 두가지 task를 수행하기 위해서는 transformer를 두가지 방법으로 이용할 수 있는데, moment-DETR 처럼 moment 를 clip의 set 단위로 예측할 수 있고, UMT 처럼 clip-wise prediction을 할 수 있음. 하지만, 이들은 query를 condition이 아닌 video와 동등한 레벨로 취급하거나 [mDETR], 매 클립이 self-attention으로 mixing 된 후에 condition을 걸어주어 clip간의 차이를 확실하지 이용하지 못하였고, 또한, 확실하게 condition으로 주지 못하였고, video와 query 사이의 관계를 한정적으로만 이용하였다.

% -	따라서, we explore three different ways to fully exploit query information. First, we design one-way cross-attention layer to condition every clip with the query features. Then, we utilized the negative video-text pairs to better model the relationships between the video and the text embeddings. Lastly, we define the saliency token to be the video-query dependent saliency estimator.


















% ===================== neg pair 부분 ===========================
% Nevertheless, the current training scheme, only considering the given video-query pair, still disturbs the model from identifying proper query-relevance prediction.
% In detail, the model focus on learning the fine-grained discrepancy between video clips, while neglecting the information they share, which contains significant clues to understand the context of video.
% Therefore, we leverage the negative video-query relationships to enhance the capability of identifying the contextual similarity of query and video clips.
% Therefore, we leverage the negative video-query relationships by suppressing those pairs, so that enhance the capability of identifying the contextual similarity of query and video clips.
% We hypothsize the diversity in query-video pairs are insufficient to learn the general relationship between text query and video.
% Therefore, we leverage the negative video-query relationships by suppressing the saliency scores of the irrelevant video-query pairs.
% However, this architectural design still lacks in identifying proper query relevance.
% We argue that the current training scheme only focuses on learning the fine-grained discrepancy between clips in a single video, while neglecting the information they share, which contains significant clues to understand the context of the video.
% Therefore, we leverage the negative video-query relationships to enhance the capability of identifying the contextual similarity of query and video clips.
% However, this architectural design still lacks in identifying proper query relevance.
% We argue the problem of the current training scheme that only focuses on learning the fine-grained discrepancy between clips in a single video.
% That is, the current design neglects the information shared throughout the video, although it contains significant clues to understand the context of the video.
\section{Related Work}
\label{sec:related_work}
\subsection{Co-Speech Gesture Synthesis}
The early approaches for generating co-speech gestures often involve creating linguistic rules to translate speech input into a sequence of pre-collected gesture segments, which are typically referred to as rule-based methods \cite{cassell1994rulefullbody,cassell2001beat,kipp2004gesture,kopp2006bml}. \citet{wagner2014rulereview} provide a comprehensive review of these methods. Rule-based methods produce interpretable and controllable results, but creating gesture datasets and rules requires significant effort. To alleviate the manual effort of designing rules in rule-based methods, data-driven approaches have gradually become predominant in this field. \citet{nyatsanga2023data_driven_gesture_survey} offer a thorough survey of these methods. Early data-driven approaches aim to directly learn mapping rules from data through statistical models \cite{neff2008videogesture,levine2009prosodygesture,levine2010gesturecontroller} and combine them with predefined gesture units for gesture generation. Later, the powerful modeling capability of deep neural networks makes it possible to train complex end-to-end models using raw speech-gesture data directly. One option is deterministic models, such as MLP \cite{kucherenko2020gesticulator}, CNN \cite{habibie2021videogesture}, RNN \cite{yoon2019robot,yoon2020trimodalgesture,bhattacharya2021affectivegesture,liu2022hierarchicalgesture}, and Transformer \cite{bhattacharya2021text2gestures}. Another choice is generative models, including flow-based models \cite{alexanderson2020stylegesture,ye2022styleflowgesture}, VAEs \cite{li2021audio2gesture,ghorbani2022zeroeggs}, and VQ-VAE \cite{yi2022talkshow,yazdian2022gesture2vec,liu2022vqgesturevideo}. Due to the inherent many-to-many relationship between speech and gesture, end-to-end models can generate natural-looking gestures but face challenges in ensuring content matching between speech and generated gestures \cite{yoon2022genea}. To address this issue, some neural systems aim to explicitly model both rhythm and semantics from the perspective of model structure \cite{kucherenko2021speech2properties2gestures,ao2022rhythmicgesticulator,liu2022disco} or training supervision strategy \cite{liang2022seeg}. Furthermore, hybrid systems, such as the combination of deep features and motion graphs \cite{zhou2022gesturemaster}, have been proposed to harness the advantages of different approaches. Recently, diffusion models \cite{sohldickstein2015diffusion,song2020improvedscore,ho2020ddpm} have demonstrated impressive results in image synthesis \cite{ramesh2022dalle2} and human motion generation \cite{tevet2022humanmotiondiffusion, zhang2022motiondiffuse}. Inspired by these works, our system adapts the latent diffusion model \cite{rombach2022latentdiffusion} for the co-speech gesture generation task and achieves appealing results.

\subsection{Style Control for Human Motion}
A typical approach to style control for human motion involves specifying a motion clip as a reference and transferring the reference clip's style to the source motion. This task is also known as \emph{style transfer}. Early works in motion style transfer integrate traditional machine learning techniques with manually defined features to infer motion styles \cite{hsu2005motion_style_translation,ma2010motion_style_transfer,xia2015realtime_motion_style_transfer,yumer2016spectral_motion_style_transfer}. Recently, deep learning-based methods have significantly enhanced motion quality. \citet{holden2016deepmotion} first propose a learning framework enabling motion style control through optimization in the motion manifold space. \citet{du2019stylemotioncvae} improve transfer efficiency by training a conditional VAE. \citet{mason2018few-shot_motion_style_transfer} use few-shot learning to generate stylized locomotion. \citet{aberman2020adain} employ a temporally invariant adaptive instance normalization (AdaIN) layer for target style injection, eliminating the need for paired data during training. \citet{wen2021stylemotionflow} achieve unsupervised style transfer using a flow model. \citet{jang2022motionpuzzle} introduce a method capable of controlling styles for individual body parts.

Previous co-speech gesture synthesis systems with style control can be categorized based on whether or not they require style labels. For methods needing labeled data, early works can only learn an individual style for one generator \cite{levine2010gesturecontroller,neff2008videogesture,ginosar2019stylegesture}. \citet{ahuja2022lowresource} propose a strategy that efficiently adapts the source generator to another speaker style using low-resource data. Some works learn a speaker style embedding space with labeled speaker-motion data, enabling gesture style control by sampling from this space \cite{ahuja2020stylegesture,yoon2020trimodalgesture,bhattacharya2021affectivegesture}. \citet{alexanderson2020stylegesture} aimat controlling fine-grained styles, such as gesturing speed and spatial scope, using preprocessed control signal-motion data. Their later work \cite{alexanderson2022diffusiongesture} utilizes a diffusion model for audio-driven motion synthesis, achieving label-based style control by training the model on labeled data. For methods not requiring style labels, \citet{habibie2022motionmatching} propose a motion matching framework to achieve flexible style control. Other studies achieve arbitrary style control by imitating an example given as a video \cite{liu2022hierarchicalgesture} or a motion clip \cite{ghorbani2022zeroeggs,ye2022styleflowgesture,kuriyama2022tokenizedgestures}.  In this work, we utilize a CLIP-based encoder to extract a style embedding from an arbitrary text prompt and incorporate it into the generator via an AdaIN layer, guiding the synthesis of stylized gestures. Our system supports fine-grained multimodal style prompts as opposed to label-based style control. It employs a self-supervised learning scheme and eliminates the need for labeled data. Additionally, we use an autoregressive model rather than a parallel model, making it potentially suitable for real-time applications.
\section{Method}
\label{sec: method}
% This section introduces the rendering pipeline of our proposed hierarchical compositional scene. 
% our pipeline consists of three processes, including decomposing the text into editable 3D layout, rendering the compositional views with local (object) NeRFs and global (scene) NeRF and the joint optimization on these hierarchical 3D representations.

% Note that the transformation between the object and the scene frame is defined by ${p}_o$ and ${D}_o$. 
%
% Next, we build a residual connection to add ${\sigma}_o$ and the referenced global color, and the rendering result will be used to calculate the SDS loss based on the global text.  
% Fig.~\ref{fig:framework} illustrates our pipeline, which consists of three main components, including the editable 3D scene layout based on multi-object text (Sec.~\ref{ssec:layout}), the scene rendering pipeline that composites the predictions from all local NeRFs (Sec.~\ref{ssec:render}), and the joint optimization on both local and global representation models (Sec.~\ref{sec:optimization}).
% To elaborate, our editable 3D scene layout represents a global frame of the scene by decomposing it into a set of local frames, where each is parameterized by a local NeRF, a 3D bounding box, and a corresponding local text prompt.
% For instance, the text prompt `A teddy bear and a stuffed monkey sit side by side' is interpreted as a 3D scene layout, as shown in Fig.~\ref{fig:framework}.  
% The whole 3D layout, \ie, scene frame, consists of two 3D bounding boxes, \ie local frames \#1 and \#2, with specific local text prompts, \ie, `a teddy bear' and `a stuffed monkey'. 
% %
% To render the scene view, we first calculate the ray-box intersections between the boxes and rays $({\boldsymbol{r}}_o, \boldsymbol{\phi}_d, {\boldsymbol{\theta}}_d)$, where the ${\boldsymbol{r}}_o$ is the ray origin and the $({\boldsymbol{r}}_o, \boldsymbol{\phi}_d)$ is its direction.
% Then, to infer each object's properties in local NeRFs, we sample the global points $({\boldsymbol{x}}_g, {\boldsymbol{y}}_g, {\boldsymbol{z}}_g)$ in the global frame within the ray-box intersection intervals and project them into the normalized local location $({\boldsymbol{x}}_l, {\boldsymbol{y}}_l, {\boldsymbol{z}}_l)$ in the local frame.
% %
% Given the local sampling points $({\boldsymbol{x}}_l, {\boldsymbol{y}}_l, {\boldsymbol{z}}_l)$, the implicit local NeRF ${\boldsymbol{\theta}}_l$ outputs four pseudo-color channels ${\boldsymbol{C}}_l$ and density $\boldsymbol{\sigma}$, which can be used to render a local view of the local frame to match its local text prompt.
% %
% We further calibrate the predicted pseudo-color $\boldsymbol{C}_l$ from local frames by adding the global embeddings ${\boldsymbol{emb}}_g$ to improve the global view consistency.
% Then, the calibrated predictions after composition are used to reconstruct the scene view by volumetric rendering along the rays.
% %
% Lastly, the rendered views based on local and global frames are guided by score distillation sampling loss $\nabla \mathcal{L}_{\text{SDS}}$~\cite{poole2022dreamfusion} to optimize all the learnable parameters. 
To resolve the issue of guidance collapse, our principal strategy is to \textit{decompose the scene into reusable components and compose/recompose them into a unified and consistent one}.
This enables flexible control over the generated content with direct use of prompts and box layouts, as illustrated in \cref{fig:teaser}.
%
Our proposed CompoNeRF confers several key benefits:
1) \textbf{Semantic Coherence}: It reliably creates 3D objects with detailed textures and global consistency, exemplified by authentic light interactions, such as reflections on the bed surface.
2) \textbf{Modularity and Reusability}: CompoNeRF functions as an ensemble of independently trained NeRF models. These can be efficiently stored and later retrieved from a cached dataset, enabling their reuse in various cases.
3) \textbf{Editability}: Our approach allows for flexible scene modification, such as interchanging the lamp for a vase filled with sunflowers or altering its scale, by simply adjusting the box dimensions for later finetuning. This feature enhances flexibility and creative possibilities. 


% Furthermore, the usage of layout boxes enables more flexible control over the generated content compared with the intricate sketch shape in Latent-NeRF\cite{metzer2022latent}. 
\begin{figure*}[t]
    \centering
    \includegraphics[width=0.9\linewidth]{figures/method.pdf}
    % \vspace{-12pt}
    \caption{\textbf{Framework Overview}.
The CompoNeRF model unfolds in three stages: 1) Editing 3D scene, which initiates the process by structuring the scene with 3D boxes and textual prompts; 2) Scene rendering, which encapsulates the composition/recomposition process, facilitating the transformation of NeRFs to a global frame, ensuring cohesive scene construction. Here, we specify design choices between density-based or color-based(without refining density) composition; 3) Joint Optimization, which leverages textual directives to amplify the rendering quality of both global and local views, while also integrating revised text prompts and NeRFs for refined scene depiction.
  % The model is structured into three components: Composition, Decomposition, and Recomposition. Composition deals with the foundational setup, detailed with choices for density-based and color-based composition. Decomposition utilizes the modularity of the CompoNeRF feature, caching each NeRF module offline for efficient recalibration. Recomposition reuses these cached NeRFs and adjusts the semantic context, providing a revised output with the inclusion of the offline NeRF enhancements.
    % Our model consists of two branches where the upper part is individual NeRFs, and the lower part denotes global calibration with our tailored composition model. The specific designs for density-based and color-based composition modules are highlighted. 
    % CompoNeRF consists of three parts: 1). The editable 3D scene layout configures the scene representations with 3D boxes and text prompts; 2).  The scene rendering includes the global calibration and the compositional process; 3). The joint optimization applies global and local text guidance on global and local render views.
    % The global frame (scene space) contains a set of local frames. Each is  represented by a local NeRF associated with a 3D box and text prompt defined by the editable 3D layout.
    % The scene view is volumetric rendered by sampling the points $({\boldsymbol{x}}_g, \boldsymbol{y}_g, \boldsymbol{z}_g)$ intersected with any local frame along the ray $(\boldsymbol{r}_o, {\boldsymbol{\phi}}_d, \boldsymbol{\theta}_d)$.
    % The sampling points are first inferred through the local NeRF with the local frame locations $({\boldsymbol{x}}_l, \boldsymbol{y}_l, \boldsymbol{z}_l)$ projected from the global location $({\boldsymbol{x}}_g, \boldsymbol{y}_g, \boldsymbol{z}_g)$.
    % And then, all the local predictions are calibrated by a global MLP with conditional input to render the scene view.
    % During the optimization, the text guidance is applied to both local views predicted by local frames only and global views predicted by the composition of all local frame predictions.
    }
    \label{fig:framework}
    % \vspace{-8pt}
\end{figure*}

\subsection{Preliminaries}
Defining individual object bounding boxes as \textit{local frames} and the overall scene coordinate system as the \textit{global frame}, we build the foundation of NeRF and diffusion processes.

\label{sec:background}
\noindent \textbf{3D Representation in Latent Space.}
Our methodology capitalizes on the state-of-the-art text-to-image generative model—Stable Diffusion as described by Rombach et al\cite{rombach2022high}.
We build upon the Latent-NeRF framework~\cite{metzer2022latent}, which computes latent colors for individual objects by considering their sample positions within a localized frame. Specifically, it maps a three-dimensional point in local coordinates \(\boldsymbol{x}_l = (x_l, y_l, z_l)\) to a volumetric density \(\boldsymbol{\sigma}_l\) and an associated color \(\boldsymbol{C}_l\), expressed as \((\boldsymbol{C}_l, \boldsymbol{\sigma}_l) = f_{\boldsymbol{\theta}_l}(x_l, y_l, z_l)\). Here, \(f\) represents a Multi-Layer Perceptron (MLP) characterized by parameters \(\boldsymbol{\theta}_l\).
 This NeRF-generated color is then assessed in the context of the Stable Diffusion model, using text prompts to guide NeRF toward spatially coherent inference with intricate context.
% to infer pseudo-color for each object using local NeRF.
% Specifically, the representation maps a point $\boldsymbol{x}_l = \left({x}_l, {y}_l, {z}_l\right)\in [-1, 1]$ in the local frame to its corresponding volumetric density $\boldsymbol{\sigma}_l$ and emitted color $\boldsymbol{C}_l$, \ie,  $\left(\boldsymbol{C}_l, {\boldsymbol{\sigma}_l}\right)=\boldsymbol{\theta}_{_l}\left({x_l}, {y}_l, {z}_l\right)$.
% The predicted pseudo-color is fed forward into the decoder of the Stable Diffusion model to obtain the final rendering result.

\noindent \textbf{Volume Rendering with Multiple Objects.}
% For each local frame $j$ with NeRF parameterized as $\theta_j$, we follow original NeRF design\cite{nerf} to integrate $(\boldsymbol{C}_l, \boldsymbol{\sigma}_l)$ of   sampled points from any hit ray $r_l=(\boldsymbol{o}_l, \boldsymbol{d}_l)$ by,
% For consistent scene rendering, object transmittance $T_k$ must be recalculated in the global frame based on independent properties inferred from local NeRFs. Hence, we sort predictions according to their distance to $\boldsymbol{o}_g$. 
% Similar to \cref{eq:volrend}, global color $\hat{\boldsymbol{C}}_g$ of ray $\boldsymbol{r}_g=(\boldsymbol{o}_g, \boldsymbol{d}_g)$ is predicted by the volumetric rendering integrating over $m$ objects,
We extend the volume rendering process to accommodate multiple objects by assigning each a local frame, denoted as $j$, with NeRF parameters $\boldsymbol{\theta}_{l, j}$. Drawing from the foundational NeRF approach \cite{nerf}, in each local frame, we integrate the color $\boldsymbol{C}_l$ and density $\boldsymbol{\sigma}_l$ for points $\boldsymbol{x}_l$ sampled along a ray $\boldsymbol{r}_l$, emanates from the camera origin $\boldsymbol{o}_l$ in direction $\boldsymbol{d}_l$. This is formalized in the predicted color integration for $\hat{\boldsymbol{C}}_l$ as:
{\setlength\abovedisplayskip{2pt}
\setlength\belowdisplayskip{2pt}
\begin{equation}
\label{eq:volrend}
{\hat{\boldsymbol{C}}_l}({\boldsymbol{r}_l})=\sum_{k=1}^{N} T_{l, k} \left(1-\exp \left(-\sigma_{l, k} \delta_k\right) \right) {\boldsymbol{C}}_{l,k},
\end{equation}}where $T_{l, k}=\exp \left(-\sum_{j=1}^{k-1} \sigma_{l,j} \delta_j\right)$ represents the transmittance to the $k$-th of total $N$ sample, calculated exponentially over the cumulative density along $\boldsymbol{r}_l$, and $\delta_k$ is the interval between adjacent samples.
%
To synthesize a coherent scene, we transition from processing individual local frames to a collective global frame. Within this global context, we reconcile object attributes inferred from their individual local NeRFs for refined $\boldsymbol{\sigma}_g, \boldsymbol{C}_g$ along with $T_{g, k}$. The samples $\boldsymbol{x}_g$ are ordered based on their spatial distances from the origin $\boldsymbol{o}_g$ following the coordinate transformation. We then express the volumetric rendering of a ray $\boldsymbol{r}_g$ integrating $m$ objects within the global frame as follows:
{
\setlength\abovedisplayskip{2pt}
\setlength\belowdisplayskip{2pt}
\begin{equation}
\label{eq:multi_volrend}
{\hat{\boldsymbol{C}}_g}({\boldsymbol{r}_g})=\sum_{k=1}^{m*N} T_{g, k} \left(1-\exp \left(-\sigma_{g, k} \delta_k\right) \right) {\boldsymbol{C}}_{g,k}. 
\end{equation}}

\noindent \textbf{Score Distillation Sampling.}
% During the SDS process, a noise image $\boldsymbol{X}_t$ is first generated by adding a sampled noise $\epsilon \sim \mathcal{N}(0, I)$ in noise level $t$ into a rendered view $\boldsymbol{X}$ from a NeRF.
To facilitate the conversion from text descriptions to 3D models, DreamFusion~\cite{poole2022dreamfusion} utilizes Score Distillation Sampling (SDS), leveraging the generative capabilities of a diffusion model, denoted as $\phi$, to guide the optimization of NeRF parameters, symbolized as $\boldsymbol{\theta}$.
%
Initially, SDS creates a noisy image $\boldsymbol{X}_t$ by infusing a randomly sampled noise $\epsilon$, which follows a normal distribution $\mathcal{N}(0, I)$, into a NeRF-rendered image $\boldsymbol{X}$ at a given noise level $t$.
The diffusion model $\phi$ then estimates the noise $\epsilon_\phi\left(\boldsymbol{X}_t, t, T\right)$ from this noisy image, conditioned by the noise level $t$ and an optional text prompt $T$. 
The key step in SDS involves calculating the gradient of the loss function, which measures the discrepancy between the estimated noise and the originally added noise:
{\setlength\abovedisplayskip{2pt}
\setlength\belowdisplayskip{2pt}
\begin{equation}
\label{eq:sds_loss}
\nabla_\theta \mathcal{L}_{\text{SDS}}(\boldsymbol{X}_t, T)=  w(t)\left(\epsilon_\phi\left(\boldsymbol{X}_t, t, T\right)-\epsilon\right),
\end{equation}}where $w(t)$ is a weighting function that adjusts the influence of the gradient based on the noise level. 
The gradients across all rendered views direct the update of $\boldsymbol{\theta}$, ensuring that the NeRF-generated images align with the text descriptions. Additionally, we incorporate the 'perturb and average' technique from SJC for more robust $\mathcal{L}_{\text{SDS}}$. For a comprehensive understanding of these methods, the reader is directed to the detailed explanations provided in \cite{poole2022dreamfusion,wang2022score}.

%
%
% \subsection{Editable 3D Scene Layout}
% \label{ssec:layout}
% The 3D scene layout explicitly combines language structures with 3D layouts in an editable way.
% Given the input text prompt $T$, the attribute-object pairs can be easily obtained based on user control.
% Note that the text prompt indicates the multi-object text prompt by default.
% % available for free in many structured representations, such as the constituency tree.
% As shown in Fig.~\ref{fig:framework}, we can extract multiple noun phrases with their binding attributes and map these local text prompts into corresponding regions.
% Specifically, we define the scene structure with $m$ local frames, each employs a local NeRF $\boldsymbol{\theta}_l$ as representation, the local text prompt $T_{l} \subseteq{T}$ and its spatial layout with 3D boxes $\mathbf{b} = \{\mathbf{p}, \mathbf{s}\} \in  \mathbb{R}^6$ of each object entity, where $\mathbf{p}=\{p_x, p_y, p_z\}$ refers to the center point and $\mathbf{s}=\{s_x, s_y, s_z\}$ denotes the box scale. 
% \textit{Our editable 3D layout is easy to be collected and edited with its simplicity, allowing for versatile and interactive user control by modifying the box's or text's properties to define a new scene}.
% Moreover, as depicted in Fig.~\ref{fig:teaser}, each component in a 3D scene layout can be replaced or re-composited with other trained local NeRFs, which is more friendly for flexible user editions compared with using only text prompts.
% We fine-tuned the new layout by global rendering, which enables scalable re-editing.
% Each relationship $r_k \in R$ is a triplet in a <subject-predictive object> format, where a subject node is. After we generate the scene graph from the complex prompts, we can sample the closest relationship with the 2d spatial layout as the initial 3D position. fine-tuned the new layout by global rendering, which enables scalable re-editing
%
% \subsection{Scene Rendering Pipeline}
% \label{ssec:render}
% In CompoNeRF, the scene images are rendered by a ray-casting approach following the design of NeRF.
% % Each ray to be cast is generated based on the camera pose, intrinsic, and transformation.
% The camera is defined by a pinhole camera model, casting a set of rays $(\boldsymbol{r}_o, \boldsymbol{\phi}_d, {\boldsymbol{\theta}}_d)=\boldsymbol{o}+t\boldsymbol{d}$ through each pixel on the frame of size $H \times W$, where the $\boldsymbol{r}_o \in  \mathbb{R}^3$ is the origin and the $(\boldsymbol{\phi}_d, \boldsymbol{\theta}_d)$ is the viewing direction.
% Along this ray, we sample all the points intersected with any layout box of local frames.
% For each hit sampled point, the color and volumetric density are computed through the local NeRF of the hit local frame.
% The ray color perdition is calculated by the differentiable integration applied on all the point-predicted colors and volumetric density along the ray.
%
% \noindent \textbf{Ray-box Intersection with Local Frames.}
% Given a ray $\boldsymbol{r}_i$, each box $\boldsymbol{b}_j$ of the local frame is applied with the AABB ray intersection test algorithm to check the intersections.
% When the ray $r_i$ is hit with a box $\boldsymbol{b}_j$ of the local frame, we use the entrance and exit points as near $\boldsymbol{t}_{in}$ and far $\boldsymbol{t}_{out}$ bounds to sample $N$ equidistant quadrature points, $
% \boldsymbol{t}_{i,j,n}=\frac{n-1}{N-1}\left(\boldsymbol{t}_{out}-\boldsymbol{t}_{in}\right)+\boldsymbol{t}_{in} , n \in \left[1, N\right]$
% % Despite each local frame only having a small number of hit rays compared to the scene, we observe that it is enough to represent each object accurately while maintaining short rendering times.
% Note that the coordinates of sampled points are first projected into normalized coordinates using the box scale of local frames to enable each local NeRF to learn the scale-independent representation.
% The bounding box $\mathbf{b}$ of the local frame in global coordinate can be transformed into a canonical bounding box by ${(\mathbf{b}} - \boldsymbol{p}) / \mathbf{s}$.
% Considering the rendering efficiency, we only calculate the valid points, interacted with the boxes, and set all the empty points with a constant background color.
%
% The appearance of a set object representations depends on its interaction with the scene and illumination which should be decided by the local frame location.
% To ensure the volumetric consistency, we only calibrate the emitted color with scene location, while the gradient still can be propagated.
% Since the overall color depends on both the global  positions $({x}_w, {y}_w, {z}_w)$ and ray directions $({\phi}_d, {\theta}_d)$, the global color embedding is learned based on both the positions and ray directions.
% Since the overall color depends on both the global  positions $({x}_w, {y}_w, {z}_w)$ and ray directions $({\phi}_d, {\theta}_d)$, the global color embedding is learned based on both the positions and ray directions.
% \subsection{The Proposed CompoNeRF}
% \subsubsection{Composition Module}
% CompoNeRF aims to composite multiple NeRFs to reconstruct multi-object scenes with both box and prompt guidance.
% %
% Our framework, as shown in \cref{fig:framework}, applies the AABB ray intersection test algorithm to check for intersections on each box in the global frame. We then samples $\boldsymbol{x}_g$ within the ray box intervals, and project them to $\boldsymbol{x}_l$ to infer  $\left(\boldsymbol{C}_l, {\boldsymbol{\sigma}_l}\right)$ in separate NeRF models. 
% %
% We then utilize volume rendering to obtain rendered views for each local frame respectively. 
% %
% After that, they would be passed on to our tailored composition Module to infer 
% $\left(\boldsymbol{C}_g, {\boldsymbol{\sigma}_g}\right)$
% for global rendering. 
% Next, we match local and global texts with their corresponding image outputs by SDS losses. 
% We also support recomposition by passing samples from cached models into $\boldsymbol{x}_l$ to continue the above process.
\begin{figure}[t!]
    \centering
    \includegraphics[width=\linewidth]{figures/abls.pdf}
    % \vspace{-22pt}
    % \caption{Ablation study on text guidance. (a) without local SDS losses. (b) without global SDS losses. (c) vanilla SDS losses without perturb and average scoring~\cite{wang2022score}. (d) full model.}
    \caption{\textbf{Design Impact Comparison: Density vs. Color-based Methods.} The top row illustrates the density-based approach's detailed rendering and quick convergence in the 'table wine' scene. The bottom row highlights the color-based method's enhancements and its drawbacks, such as geometric and shadow inaccuracies, particularly in close-up views and slow convergence.
    % \textbf{(a)} global text guidance(integrating local frames by \cref{eq:multi_volrend}) and global calibration(integrating local frames, then aligning the rendering result directly with the full text). 
    }
    \label{fig:abls}
    % \vspace{-20pt}
\end{figure}
\subsection{The Proposed CompoNeRF}
\subsubsection{Composition Module}
CompoNeRF is designed to composite multiple NeRFs to reconstruct scenes featuring multiple objects, utilizing guidance from both bounding boxes and textual prompts. Within our framework, depicted in \cref{fig:framework}, the Axis-Aligned Bounding Box (AABB) ray intersection test algorithm is applied to ascertain intersections across each box in the global frame. Subsequently, we sample points \(\boldsymbol{x}_g\) within the intervals of the ray-box and project them to \(\boldsymbol{x}_l\) to deduce the corresponding color \(\boldsymbol{C}_l\) and density \(\boldsymbol{\sigma}_l\) within individual NeRF models.
%
These properties are processed through our composition module to infer the global color \(\boldsymbol{C}_g\) and density \(\boldsymbol{\sigma}_g\), crucial for the global rendering.
%
Volume rendering techniques~\cite{kajiya1984ray} are then employed to procure the rendered views for both local and global frames. We propose dual SDS losses to ensure coherence between the image outputs and their corresponding textual descriptions. Additionally, our approach facilitates recomposition by channeling samples from cached models back into local frames along with the text revision, thereby streamlining the integration.

% As shown in \cref{fig:abls}(a), we verify its necessity by dropping $\nabla \mathcal{L}_{\text{SDS}_g}$. 
% %
% Compared with our full model, its layout does not fit our shared sense of a room, \ie, \emph{nightstand} is usually lower than \emph{bed}; \emph{lamp} needs a base to support it. Additionally,  it lacks global consistency, such as light reflection, to make it more realistic. 
% %
% Therefore, we leverage the full text semantics to ensure consistent global rendering across local frames. 
% %
% Instead of conditioning the global rendering view with the full prompt directly, we note that global calibration is necessary for geometry and color to be learned sufficiently.
% For example, we observe that geometric completeness and texture of \emph{nightstand} are not ideal. Although reflection appears around \emph{nightstand}, \emph{bed} is stripped of the light. 
% %
% Therefore, we opt to leverage the correlation between the rendering output of the combined NeRFs and the overall semantics to perform multi-object scene reconstruction.  
%

\noindent\textbf{Global Composition.}
The independent optimization of each local frame may inadvertently result in a lack of global coherence within the scene. To address this, our scene composition process is designed to integrate these frames, thereby achieving a more consistent result.
%
Before exploring the specifics of the module, it is imperative to discuss two critical design decisions within the composition module, as depicted in \cref{fig:framework}.
%
Upon integrating the properties inferred from \(\boldsymbol{x}_g\) into the composition module, they are fine-tuned through gradients derived from the global SDS loss.  This process leads to a critical consideration: the necessity and implications of refining the global density \(\boldsymbol{\sigma}_g\). This can be divided into two approaches: \textbf{1) Density-based:} The advantage of adjusting \(\boldsymbol{\sigma}_g\) is that it can adjust geometry, thus yielding a scene more congruent with the global text prompt. 
However, this comes at the cost of potentially compromising the optimal color \(\boldsymbol{C}_g\), as calibrating \(\boldsymbol{\sigma}_g\) introduces more uncertainty for subsequent color refinement as it requires prior density features $\boldsymbol{h}$ as shown at \cref{fig:compo}. 
\textbf{2) Color-based:} Conversely, directly employing \(\boldsymbol{\sigma}_l\) mitigates this uncertainty but at the expense of reduced geometric control, presenting a challenging balance to strike in the pursuit of precise scene composition.
% , which may lead to suboptimal outcomes.
%
After thorough experiments, exemplified in \cref{fig:abls}, we have opted for the density-based approach to refine \(\boldsymbol{\sigma}_g\)  prioritizing both \textbf{accuracy and efficiency}. The test revealed that it excels in rendering intricate details, such as enhanced wood grain textures and more naturally contoured 'salad', as accentuated by boxes. This method also demonstrated a swifter convergence rate. Conversely, while the color-based improved reflections and reduced flickering on the 'wine cup', it was plagued by issues such as sparse density, which adversely brings holes at the base of the 'cup' and the corner of the 'table'.
Furthermore, upon close examination, it becomes evident that shadow artifacts of 'wine' on the 'table' are pronounced, suggesting that its disadvantages outweigh its advantages.
%  in this context
% \textbf{Global Composition.}
% Each local frame is optimized independently, causing a lack of global connections for scene composition.
% Before delving into module details, there are two choices (see \cref{fig:framework}) on the composition module design we need to elaborate on first. 
% %
% In \cref{fig:framework}, by taking $\boldsymbol{x}_g$ into the composition module, their inferred properties are calibrated with gradients propagated from the global SDS loss. 
% However, it remains unclear whether $\boldsymbol{\sigma}_g$ should be refined or not. 
% %
% The trade-off on its usage is the density adjustment bringing a more reasonable layout and more geometric details that fit the global text prompt. While its potential downside is that $\boldsymbol{C}_g$ may not be optimal as $\boldsymbol{\sigma}_g$ has more uncertainty compared to $\boldsymbol{\sigma}_l$, bringing sub-optimal rendering results. 

% We choose the density-based method after comparing them with the experiment shown in \cref{fig:abls}. 
% %
% Specifically, we test both designs on the scene \emph{table wine} and discover that the density-based design provides more intrinsic details(as indicated by green boxes), \eg, enriched wood grains, and a more natural shape for \emph{salad} and has much faster convergence speed. In contrast, the color-based method enhances the reflection and smooths flickering on \emph{wine cup}, (as indicated by red boxes), but it suffers from 1) sparse density, resulting in poorly generated geometry at the base of  \emph{cup} and the wood \emph{table} corner. Additionally, shadow artifacts appeared on \emph{table} when viewed up close, outweighing benefits of the color-based method.

\begin{figure}[t!]
    \centering
    \includegraphics[width=\linewidth]{figures/compo_module.pdf}
    % \vspace{-24pt}
    % \caption{Ablation study on text guidance. (a) without local SDS losses. (b) without global SDS losses. (c) vanilla SDS losses without perturb and average scoring~\cite{wang2022score}. (d) full model.}
    \caption{\textbf{Detail of Composition module}: density-based design. 
    }
    \label{fig:compo}
    % \vspace{-18pt}
\end{figure}
\noindent\textbf{Network Design.}
The compositional framework of our network, as delineated in \cref{fig:compo}, is predicated on an architecture that employs a suite of MLPs, represented as \(\{\boldsymbol{\theta}_l\}_{l=1}^{m}\),  each dedicated to a distinct local frame. To harmonize \(\boldsymbol{\sigma}_l\) and \(\boldsymbol{C}_l\), we incorporate global MLPs, including density calibrator $f_{\boldsymbol{\theta}_{g_d}}$ and color calibrator $f_{\boldsymbol{\theta}_{g_c}}$.
%
A transformation module complements this system, tasked with maintaining the spatial coherence between the global and local frames. It governs the transformation of sampling points $\boldsymbol{x}$, ray directions $\boldsymbol{d}$, and adjacent sampling distances $\delta$. This module also orders the points $\{\boldsymbol{x}_{g,j}\}_j$ by their distance to the global camera origin $\boldsymbol{o}_g$, ensuring that each local point $\boldsymbol{x}_l$ is accurately matched with its corresponding global point $\boldsymbol{x}_g$ for subsequent volume rendering. 
%
The network design is:
{
\setlength\abovedisplayskip{4.5pt}
\setlength\belowdisplayskip{4.5pt}
\begin{align}
\label{eq:g_c_d}
{\boldsymbol{\sigma}_g}  &= \alpha_d f_{\boldsymbol{\theta}_{g_d}}({\boldsymbol{x}_g}) + \boldsymbol{\sigma}_l, \\
{\boldsymbol{C}_g}  &= \alpha_c f_{\boldsymbol{\theta}_{g_c}}(\boldsymbol{h}, {\boldsymbol{d}_g}) + \boldsymbol{C}_l. 
\end{align}}In contrast to the local frames, the global frame's color output $\boldsymbol{C}_g$ is inferred based on $\boldsymbol{h}$ and conditional on $\boldsymbol{d}_g$ to enable a view-dependent lighting effect.
% Denote the density features as $\boldsymbol{h}$. 
%
%
Residual learning is leveraged here, where \(\boldsymbol{\sigma}_l, \boldsymbol{C}_l\) serve as foundational elements that support the learning of global density \(\boldsymbol{\sigma}_g\) and color \(\boldsymbol{C}_g\). The parameters \(\alpha_d, \alpha_c\) are adjustable, allowing fine-tuning of the influence that local components exert on the global outputs.
%
It is imperative to acknowledge that in our color-based method, density calibration is intentionally excluded to concentrate solely on the refinement of color dynamics as shown at \cref{fig:framework}. This is achieved by conditioning the process on both spatial and directional global inputs \((\boldsymbol{x}_g, \boldsymbol{d}_g)\), as demonstrated in the following equations:
\begin{align}
\setlength\abovedisplayskip{4.5pt}
\setlength\belowdisplayskip{4.5pt}
\label{eq:g_c_c}
\boldsymbol{\sigma}_g = \boldsymbol{\sigma}_l, \quad
{\boldsymbol{C}_g} = \alpha_c f_{\boldsymbol{\theta}_{g_c}}({\boldsymbol{x}_g}, {\boldsymbol{d}_g}) + \boldsymbol{C}_l.
\end{align}
The integration of extra $\boldsymbol{x}_g$ aims to facilitate a fair comparison under same inputs with the density-based. It enhances the visual appeal of effects like the wine cup's reflection, as demonstrated in \cref{fig:abls}. However, this method is not without its compromises. It tends to produce artifacts and is characterized by a slower convergence rate. Additionally, this approach limits the ability to precisely control density, subsequently impacting the intricate geometric details.


\begin{figure*}[t!]
    \centering
    \includegraphics[width=\linewidth]{figures/sota.pdf}
    % \vspace{-24pt}
    \caption{\textbf{Qualitative comparison with other text-to-3D methods using multi-object text prompts}. Cases 1-3 demonstrate simpler settings characterized by compositions involving two objects. In contrast, Cases 4-8 delve into more intricate scenarios featuring compositions with more than two objects. Smaller images are presented to illustrate the generated local NeRFs(partially shown in Cases 4-8).}
    \label{fig:sota}
    % \vspace{-5pt}
\end{figure*}
%
% \begin{table*}[t!]
% \centering
% \resizebox{\textwidth}{!}
% {
% \begin{tabular}{cccccccc}
% \toprule
% Method            & \rotatebox{60}{table wine}  & \rotatebox{60}{teddy monkey} & \rotatebox{60}{computer mouse} & \rotatebox{60}{bed room}  & \rotatebox{60}{chess} & \rotatebox{60}{pisa tower} & \rotatebox{60}{astronaut} & \rotatebox{60}{tesla}  \\ \midrule
% LatentNeRF  & 21.55 & 27.38 & 17.13 & 21.86 & 31.19 & 24.31 & 27.07 & 25.16 \\
% SJC & 23.33 & 27.37 & 18.00 & 22.54 & 30.53 & \textbf{26.18 }& 27.84 & 23.55 \\
% CompoNeRF & \textbf{32.68} & \textbf{28.57}	 &\textbf{ 22.34} &\textbf{ 28.65} & \textbf{31.45} & \textbf{28.96} & 25.82 & 25.95 & 24.42 & \textbf{32.71} & \textbf{26.13 }& \textbf{26.38} & \textbf{30.98} & \textbf{33.37} \\
% \bottomrule
% \end{tabular}
% }
% \vspace{-10pt}
% \caption{Performance of our CompoNeRF in different 3D scenes. We use CLIP score \cite{parmar2023zero,zhang2023sine,wang2023imagen} as our evaluation metric, which is a common evaluation metric in text-to-image generation tasks to evaluate the similarity of the generated image to the text prompt. }
% \label{perclass}
% \end{table*}
%
\begin{table*}[t!]
% \scalebox{0.8}
\renewcommand{\arraystretch}{1.2}
\fontsize{4pt}{4pt}
\selectfont 
\centering
% \vspace{-8pt}
\resizebox{\textwidth}{!}
{
% \begin{tabular}{lcccccccc}
% \hline
% Method     & table\_wine    & tesla          & pyramid        & chess          & apple and banana      & astronaut      & glass\_balls   & Eiffel\_tower    \\ \hline
% LatentNeRF & 21.55          & 25.16          & 27.43          & 31.19          & 27.69          & 27.07          & 29.51          & 26.32          \\
% SJC        & 23.33          & 23.55          & 25.62          & 30.53          & 28.21          & 27.84          & 28.76          &27.41 \\
% \textbf{CompoNeRF(Ours)}     & \textbf{32.68} & \textbf{26.13} & \textbf{28.96} & \textbf{31.45} & \textbf{33.37} & \textbf{32.71} & \textbf{30.98} & \textbf{28.44}          \\ \hline
% \end{tabular}
\begin{tabular}{lcccccccc}
\hline
Method                   & Case 1         & Case 2         & Case 3         & Case 4         & Case 5         & Case 6         & Case 7         & Case 8         \\ 
\hlineB{1.1}
LatentNeRF               & 25.16          & 27.07          & 27.69          & 31.19          & 21.55          & 26.32          & 27.43          & 29.51          \\
SJC                      & 23.55          & 27.84          & 28.21          & 30.53          & 23.33          & 27.41          & 25.62          & 28.76          \\
\textbf{CompoNeRF (Ours)} & \textbf{26.13} & \textbf{32.71} & \textbf{33.37} & \textbf{31.45} & \textbf{36.06} & \textbf{28.44} & \textbf{28.96} & \textbf{30.98} \\ \hlineB{1.1}
\end{tabular}
}

% \vspace{-6pt}
\caption{\textbf{Performance comparison of our CompoNeRF in different 3D scenes}. For our evaluation metric, we utilize the average of CLIP scores~\cite{parmar2023zero,zhang2023sine,wang2023imagen} across different views, which serve to assess the similarity between the generated images and the global text prompt. }
\label{tb:perclass}
\end{table*}
% \cref{fig:framework} depicts the network architecture of the composition module. Denote $m$ as local MLP $\{\boldsymbol{\theta}_l\}_{l=1}^{m}$ for each local frame. Then, we introduce the global MLPs including density $\boldsymbol{\theta}_{g_d}$ and $\boldsymbol{\theta}_{g_c}$ calibrators to refine $\boldsymbol{\sigma}_l$ and $\boldsymbol{C}_l$. 
% %
% In detail, the network design is, 
% {
% % \setlength\abovedisplayskip{4.5pt}
% % \setlength\belowdisplayskip{4.5pt}
% \begin{align}
% \label{eq:g_c_d}
% {\boldsymbol{\sigma}_g}  &= \alpha_d \boldsymbol{\theta}_{g_d}({\boldsymbol{\sigma}_l}) + \boldsymbol{\sigma}_l, \\  
% {\boldsymbol{C}_g}  &= \alpha_c \boldsymbol{\theta}_{g_c}({\boldsymbol{C}_l},  {\boldsymbol{d}_g}) + \boldsymbol{C}_l, 
% \end{align}}
% %
% where residual $\boldsymbol{\sigma}_l, \boldsymbol{C}_l$ assist in learning $\boldsymbol{\sigma}_g$ and $\boldsymbol{C}_g$, while $\alpha_d, \alpha_c$ balance their contribution as learnable parameters.
% %
% Note that the color-based omits density calibration, and simply uses the shared color refinement.



% The 3D boxes are only used for the spatial configuration of local NeRFs, while the implicit representation of local NeRFs is inferred by the canonical samples inside the local frame without considering the global relationship across different objects.
% To relieve such location-dependent effects, we further calibrate the output color and density from the local NeRF with global coordinates $({\boldsymbol{x}}_g, {\boldsymbol{y}}_g, {\boldsymbol{z}}_g)$ and ray directions $\left({\boldsymbol{\phi}}_{d}, {\boldsymbol{\theta}}_{d}\right)$ as the conditional input.
% % to inject the global visual clues.
% %
% %
% Specifically, we adopt a shared MLP $\boldsymbol{\theta}_{g}$ to calibrate all the predicted object colors, that is,
% {\setlength\abovedisplayskip{4.5pt}
% \setlength\belowdisplayskip{4.5pt}
% \begin{align}
% \label{eq:MLP_dyn_2}
% {\boldsymbol{C}_g} = {\boldsymbol{C}_l} + \boldsymbol{emb}_{g} &= {\boldsymbol{C}_l} + \boldsymbol{\theta}_{g}({\boldsymbol{x}}_g, {\boldsymbol{y}}_g, {\boldsymbol{z}}_g, {\boldsymbol{\phi}}_{d}, {\boldsymbol{\theta}}_{d}),
% \end{align}}
% where ${\boldsymbol{C}_l}$ is the color predicted by the local NeRF.
% Therefore, the scene color can preserve the view-consistent behavior from the original architecture and add consistency across poses for the volumetric density.
% Since the color and density values share the same latent expression in $({\boldsymbol{x}}_l, {\boldsymbol{y}}_l, {\boldsymbol{z}}_l)$, we only calibrate the emitted scene color explicitly with the scene location, as the densities of local NeRFs also are implicitly adjusted during optimization.

% \noindent \textbf{Global and Local Volumetric Rendering.}
% After compositing all the interacted points, each ray $\boldsymbol{r}_i$ collects a set sampling points by $\{\boldsymbol{t}_{i,j,n} \}_{j=1, n=1}^{m_j, N}$, where $m_j$ is the number of the hit object.
% For each sampling point, the inference results with the respective 3D representations are the local color $\boldsymbol{c}_{l}$, global color $\boldsymbol{c}_{g}$, and density $\sigma$.

% In fact, the local view $\hat{C}_{l,j}$ of single object $j$ also can be rendered by the sampled points  belongs to the same local frames as shown at Fig.~\ref{fig:framework}.

\subsubsection{Recomposition}
Our architecture advances scene reconstruction by providing an intuitive interface for layout manipulation.  This capability is crucial for the reconfiguration of scene elements into novel scenes, as depicted in \cref{fig:framework}. Here, the input panel allows for adjustments in the attributes of bounding boxes, such as modifying the position and scale of the 'apple' bounding box prior to composition. The refinement process further involves sampling ray-box intervals from the global frame, leading to transformed coordinates with the corresponding ray samples that are then incorporated into the pipeline, as demonstrated in \cref{fig:compo}.
%
Each bounding box represents an individual NeRF, providing the flexibility to move, scale, or remove elements as needed. CompoNeRF's capabilities also extend to textual edits, exemplified by the transformation of 'wine' into 'juice'.
%
Since NeRFs have been well trained, we only finetune \(\theta_g, \theta_l\) to align text prompts to promote consistency of both local and global views.
%
Moreover, the NeRFs once retrained within the edited scene, are also structured to be decomposable and cacheable in future scene compositions.
% Our CompoNeRF architecture facilitates the seamless reconstruction of scenes leveraging existing models. It enables precise editing of bounding boxes parameterized by \(\{\boldsymbol{\theta}_l\}_{l=1}^{m}\), allowing for their reconfiguration into new layouts. Refer to \cref{fig:framework}, the input panel permits the modification of attributes such as the position and scale of the 'apple' node's bounding box prior to composition. The process is further refined by sampling from the updated ray-box intervals within the global frame, which are then projected onto \(\boldsymbol{x}_l\), ensuring a streamlined reconstruction that integrates the 'apple' effectively. This addition is executed with careful attention to color consistency, positioning the 'apple' adjacent to the 'French bread' to complement the scene's overall palette. Each bounding box represents an individual NeRF, which means they can be manipulated through moving, scaling, and removal operations. CompoNeRF also extends its editing prowess to textual modifications, as evidenced by the 'wine cup' now appearing filled with juice—a change propagated through both subtexts and the global test. 
% %
% Since NeRFs have been well trained, we only finetune $\theta_g, \theta_l$ to align text prompts to promote consistency of both local and global views . 
% %
% Moreover, the NeRFs, once retrained within the reimagined scene, are also structured to be decomposable and cacheable for subsequent scene compositions.

% , as shown in Fig.~\ref{fig:framework}.
% For each scene described by the multi-object text prompt $T$, we
% To enhance the guidance of local representations, we use the local text prompt $T_l \subseteq T$ of a single object to optimize the local NeRFs in local views.
% The scene views $\hat{\boldsymbol{X}}_g=\{\hat{\boldsymbol{C}}_{g,i}\}_{i=1}^{H\times W}$ is obtained from the predicted pixel values of $H \times W$ rays by compositing all the ray-box interaction values.
% Similarly, the rendered view $\hat{\boldsymbol{X}}_{l,j}$ of the local frame $\boldsymbol{\theta}_j$ without compositing other objects can be calculated by $\hat{\boldsymbol{C}}_{l,j}$, as depicted in Sec.~\ref{ssec:render}.
% We use the local color instead of the globally calibrated color to obtain a local view because the local NeRF should learn the object identity unrelated to its placed position, as the position can be different during user edition.
% % Compared to cropping the local region from a global view for training, separate rendering can avoid the undesired information from other objects brought by the occlusion and resolution adjustments.
% Formally, we employ the following loss as the learning objective,
\begin{figure*}[t!]
    \centering
    \includegraphics[width=\linewidth]{figures/editing.pdf}
    % \vspace{-23pt}
    \caption{\textbf{Scene Editing Outcome:} Demonstrated here are the stages of our recomposition, utilizing cached source scenes. Each NeRF is individually identified by colorful labels. These decomposed nodes are then positioned in the initial layout and subsequently calibrated to form the final composition. The detailed description of the ambient environment is underscored, enhancing the scene's realism.}
    \label{fig:app}
    % \vspace{-12pt}
\end{figure*} 
\subsubsection{Optimization}
\label{sec:optimization}
During optimization, our method employs dual text guidance to align rendering results with both global and local textual descriptions. The optimization objective is:
{
\small
\setlength\abovedisplayskip{2pt}
\setlength\belowdisplayskip{2pt}
\begin{equation}
\label{eqn:loss_f}
\mathcal{L}= {\alpha_g}\nabla\mathcal{L}_{\text{SDS}}(\hat{\boldsymbol{X}}_{g}, T) + {\alpha_l}\sum_{j=1}^{m} \nabla\mathcal{L}_{\text{SDS}}(\hat{\boldsymbol{X}}_{l,j}, T_{l,j}) + \beta\mathcal{L}_{\text{sparse}},\nonumber
\end{equation}
}where $T$ signifies the global text prompt, while $T_{l}$ pertains to a specific object within the global context. The hyperparameters $\alpha_{g}, \alpha_{l}$, and $\beta$ modulate the respective loss weights. 
% $\nabla \mathcal{L}_{\text{SDS}}$ is the score distillation sampling loss, as described in Sec.~\ref{sec:background}.
As suggested in~\cite{metzer2022latent}, we use $L_{\text{sparse}}$ included to penalize the binary entropy of local NeRFs' densities, thereby mitigating the issue of extraneous floating radiance.
Additionally, incorporating directional cues such as "front view" or "side view" into the input text, as suggested by \cite{poole2022dreamfusion,metzer2022latent} proves beneficial in specifying camera poses during the training phase, further enhancing the alignment of our generated scenes with the intended perspectives.
% Note that the global calibration in the scene frame can adaptively revise both $({C}_l, {\sigma})$ in local NeRF with $\nabla \mathcal{L}_{SDS}$ along with the back-propagating gradient.

\vspace{-0.3em}
\section{Experiments}
\vspace{-0.3em}
\subsection{Datasets, Evaluation \& Implementation}
\vspace{-0.3em}

\noindent\textbf{Pretraining Datasets. } 
Following prior work \cite{videomoco-pan2021videomoco,gdt-patrick2020multimodal,dave2022tclr,ctp-wang2021unsupervised,rspnet-chen2020RSPNet,huang2021ascnet}  we use \textbf{Kinetics-400}~\cite{Kinetics-400-arxiv} for self-supervised pretraining. Kinetics-400 is a large-scale action recognition dataset containing 250K videos of 400 action classes.
To show data efficiency, we also pretrain with \textbf{Mini-Kinetics}~\cite{mini_kinetics}, a subset containing 85K videos of 200 action classes. %For both cases, we use the training set without action labels for self-supervised pretraining. 

\noindent \textbf{Downstream Evaluation. } 
To evaluate the video representations learned by our tubelet contrast, we finetune and evaluate our model on various downstream datasets summarized in Table~\ref{tab:severe_desc}.
Following previous self-supervised work, we evaluate on standard benchmarks: \textbf{UCF101}~\cite{UCF-101-arxiv} and \textbf{HMDB51}~\cite{HMDB-51-ICCV}. These action recognition datasets contain coarse-grained actions with domains similar to Kinetics-400. 
%UCF101 contains 13,320 videos from 101 action classes while HMDB51 contains 6,890 videos from 51 action classes. 
For both, we report top-1 accuracy on split 1 from the original papers.
%
We examine the generalizability of our model with the \textbf{SEVERE} benchmark proposed in our previous work~\cite{thoker2022severe}. This consists of eight experiments over four downstream generalization factors: \textit{domain shift}, \textit{sample efficiency}, \textit{action granularity}, and \textit{task shift}. %SEVERE contains two experimental setups for each factor.
\textit{Domain shift} is evaluated on Something-Something v2~\cite{SS-v2-arxiv} (SSv2) and FineGym~\cite{Gym-99-arxiv} (Gym99)  
which vary in domain relative to Kinetics-400. 
\textit{Sample efficiency} evaluates low-shot action recognition on UCF101~\cite{UCF-101-arxiv} and FineGym~\cite{Gym-99-arxiv} with 1,000 training samples, referred to as UCF ($10^{3}$) and Gym ($10^{3}$).
\textit{Action granularity} evaluates semantically similar actions using FX-S1 and UB-S1 subsets from  FineGym~\cite{Gym-99-arxiv}. In both subsets, action classes belong to the same element of a gymnastic routine, \eg, FX-S1 is types of jump.  
\textit{Task shift} evaluates tasks beyond single-label action recognition. Specifically, it uses temporal repetition counting on UCFRep~\cite{ucfrep-zhang2020context}, a subset of UCF-101~\cite{ucfrep-zhang2020context}, and multi-label action recognition on Charades~\cite{charades-sigurdsson:hal-01418216}. %where videos are an average of 30 seconds long. 
The experimental setups are detailed in Table~\ref{tab:severe_desc} and all follow SEVERE~\cite{thoker2022severe}. 

\noindent\textbf{Tubelet Generation and Transformation.} Our clips are 16 $112 {\times} 112$ frames with standard spatial augmentations: random crops, horizontal flip, and color jitter. %Accordingly, we set $T{=}16$, $W{=}112$ and $H{=}112$ for tubelet generation.  
We randomly crop 2 patches to generate $M{=}2$ tubelets (Eq.~\ref{eq:overlay_M_tubelets}). The patch size $H^{'}{\times} W^{'}$ is uniformly sampled from $[16{\times}16, 64{\times64}]$. We also randomly sample a patch shape from a set of predefined shapes. For linear motions, we use $\Delta{=}[40{-}80]$ displacement difference. For non-linear motion, we use $N{=}48$ and a smoothing factor of $\sigma{=}8$ (Eq.~\ref{eq:nonlinear_motion}). For linear motion and all tubelet transformations, we use $K{=}3$ keyframes.

\noindent\textbf{Networks, Pretraining and Finetuning.}
We use R(2+1)D-18~\cite{tran2018closer} as the video encoder, following previous self-supervision works~\cite{pace,gdt-patrick2020multimodal,rspnet-chen2020RSPNet,fame,dave2022tclr,videomoco-pan2021videomoco}. The projection head is a 2-layer MLP with 128D output. We use momentum contrast~\cite{moco} to increase the number of negatives $|\mathcal{N}|$ (Eq.~\ref{eq:tcl}) to 16,384 for Mini-Kinetics and 65,536 for Kinetics. 
We use temperature $\tau{=}0.2$ (Eq.~\ref{eq:tcl}). The model is optimized using SGD with momentum 0.9,  learning rate 0.01, and weight decay 0.0001. We use a batch size of 32 for Mini-Kinetics and 128 for Kinetics, a cosine scheduler~\cite{cosine_sch}, and pretrain for 100 epochs. After pretraining, we replace the projection head with a task-dependent head as in SEVERE~\cite{thoker2022severe} and finetune the whole network with labels for the downstream task.  We provide finetuning details in the supplementary. %
\begin{table}
    \centering
    \setlength{\tabcolsep}{2pt}
    \resizebox{\linewidth}{!}{
    \begin{tabular}{lcccc} \toprule  
    %Pre-training& UCF ($10^{3}$) & Gym ($10^{3}$) &UB-S1 & SSv2-Sub\\
    & UCF ($10^{3}$) & Gym ($10^{3}$) & SSv2-Sub & UB-S1 \\
         \midrule
         \rowcolor{Gray}
         \textbf{Temporal Contrast} & & & &\\
         Baseline    & 57.5 & 29.5        & 44.2  & 84.8      \\
         \rowcolor{Gray}
         \textbf{Tubelet Contrast} & & & &\\
         Tubelet Generation & 48.2  &28.2    & 40.1  & 84.1      \\
         Tubelet Motion
          & 63.0 & 45.6          & 47.5  & 90.3      \\
         Tubelet Transformation  & 65.5 & 48.0 & 47.9  & 90.9      \\
        \bottomrule
    \end{tabular}}
    \vspace{-0.8em}
    \caption{\textbf{Tubelet-Contrastive Learning} considerably outperforms temporal contrast on multiple downstream settings. Tubelet motion and transformations are key. % 
    }
    \label{tab:ablation_main}
    %\vspace{-0.5em}
\end{table}

\vspace{-0.3em}
\subsection{Ablation Studies \& Analysis}
\vspace{-0.3em}
To ablate the effectiveness of individual components we rely on Mini-Kinetics for self-supervised pretraining. For downstream evaluation, we report on UCF ($10^{3}$), Gym ($10^{3}$), Something-Something v2 and UB-S1. To decrease the finetuning time we use a subset of Something Something (SSv2-Sub) with 25\% of the training data. Unless specified otherwise, we use non-linear motion and the rotation transformation to simulate the tubelets.

\noindent\textbf{Tubelet-Contrastive Learning.}
Table~\ref{tab:ablation_main} demonstrates the benefits brought by our tubelet-contrastive learning. We first observe that our full tubelet-contrastive model improves considerably over the temporal contrastive baseline, which uses MoCo~\cite{moco} with a temporal crop augmentation. This improvement is independent of the downstream dataset used but is especially observable with Gym ($10^{3}$) (+18.5\%) and UB-S1 (+6.1\%) where temporal cues are more crucial than the spatial cues to recognize actions. Our model is also effective on UCF ($10^{3}$) (+8.0\%) where spatial cues are often as important as temporal ones. These results demonstrate that learning similarities between synthetic tubelets produces generalizable, but motion-focused, video representations required for finer temporal understanding.

It is clear that the motion within tubelets is critical to our model's success as contrasting static tubelets obtained from our tubelet generation (Section~\ref{section:tbcl}) actually  decreases the performance from the temporal contrast baseline. 
%This demonstrates that tubelet motion is key to learning effective video representations. 
When tubelet motion is added (Section~\ref{section:motions}), performance improves considerably, \eg Gym ($10^{3}$) +17.4\% and SSv2-Sub +7.4\%. 
Finally, adding more motion types via tubelet transformations (Section~\ref{section:transformations}) further improves the video representation quality, \eg UCF ($10^{3}$) +2.5\% and  Gym ($10^{3}$) +2.4\%. This highlights the importance of including a variety of motions beyond what is present in the pretraining data to learn generalizable video representations. %

\begin{table}
    \centering
    \resizebox{\linewidth}{!}{
    \begin{tabular}{lcccc} \toprule Tubelet Motion & UCF ($10^{3}$) & Gym ($10^{3}$) &  SSv2-Sub & UB-S1  \\
         \midrule
         No motion & 48.2  &28.2      & 40.1 & 84.1 \\
         Linear   & 55.5 & 34.6       & 45.3 & 88.5 \\
         Non-Linear   & 63.0 & 45.6   & 47.5 & 90.3 \\
        \bottomrule
    \end{tabular}}
    \vspace{-0.8em}
    \caption{\textbf{Tubelet Motions.} Learning from tubelets with non-linear motion benefits multiple downstream settings.}
    \label{tab:ablation_motions}
    %\vspace{-0.8em}
\end{table}

\begin{figure*}[t!]
\centering
\includegraphics[width=0.98\linewidth]{figures/data-eff.png}
\vspace{-1em}
\caption{\textbf{Video-Data Efficiency of Tubelet-Contrastive Learning.} Our approach maintains performance when using only 25\% of the pretraining data. When using 5\% of the pretraining data, our approach is still more effective than using 100\% with the baseline for Gym($10^3$), UB-S1, and HMDB51. Results are averaged over three pretraining runs with different seeds.%
\vspace{-1em}
}
\label{fig:data_efficiency}
\end{figure*}

 \begin{table}
    \centering
    \resizebox{0.95\linewidth}{!}{
    \begin{tabular}{lcccc} \toprule Transformation & UCF ($10^{3}$) & Gym ($10^{3}$)  & SSv2-Sub &UB-S1\\
         \midrule
         None   & 63.0 & 45.6 & 47.5 & 90.5    \\
         Scale  & 65.1 & 46.5 & 47.0 & 90.5   \\
         Shear  & 65.2 & 47.5 & 47.3 & 90.9   \\
         Rotation & 65.5 & 48.0 & 47.9 & 90.9   \\
        \bottomrule
    \end{tabular}}
    \vspace{-0.5em}
    \caption{\textbf{Tubelet Transformation.} Adding motion patterns to tubelet-contrastive learning through transformations improves downstream performance. Best results for rotation.}
    \label{tab:ablation_transformations}
    %\vspace{-0.5em}
\end{table}


\begin{table}
    \centering
    \resizebox{0.88\linewidth}{!}{
    \begin{tabular}{lcccc}
         \toprule \#Tubelets & UCF ($10^{3}$) & Gym ($10^{3}$)  & SSv2-Sub & UB-S1\\
         \midrule
         1 & 62.0 & 39.5 &  47.1 & 89.5 \\
         2 & 65.5 & 48.0 &  47.9 & 90.9 \\
         3 & 66.5 & 46.0 &  47.5 & 90.9 \\
        \bottomrule
    \end{tabular}}
    \vspace{-0.5em}
    \caption{\textbf{Number of Tubelets.} Overlaying two tubelets in positive pairs improves downstream performance. %
    }
    \label{tab:ablation_num_tubelets}
    \vspace{-0.8em}
\end{table}

\begin{figure}
\centering
\includegraphics[width=\linewidth]{figures/grad_cam.pdf}
\vspace{-2em}
\caption{\textbf{Class-Agnostic Activation Maps without Finetuning} for the temporal contrastive baseline and our tubelet-contrast.  Our model better attends to regions with motion. }
\vspace{-1em}
\label{fig:caam}
\end{figure}

\noindent\textbf{Tubelet Motions.} Next, we ablate the impact of the  tubelet motion type (Section~\ref{section:motions}) without transformations. We compare the performance of static tubelets with no motion, linear motion, and non-linear motion in Table~\ref{tab:ablation_motions}. Tubelets with simple linear motion already improve performance for all four datasets, \eg +6.4\% on Gym ($10^3$). %\cs{highlight one result?}. 
Using non-linear motion further improves results, for instance with an additional +11.0\% improvement on Gym ($10^3$). %\cs{highlight another result?}. 
We conclude that learning from non-linear motions provides more generalizable video representations. 
%


\noindent\textbf{Tubelet Transformation.}
Table~\ref{tab:ablation_transformations} compares the three proposed tubelet transformations (Section~\ref{section:transformations}). % 
All four datasets benefit from transformations. We find that scaling improves results on UCF ($10^3$) (+2.1\%) and Gym ($10^3$) (+0.9\%) while maintaining performance on UB-S1 and SSv2-Sub. Shearing gives better performance over scaling on all datasets, \eg, Gym ($10^3$) (+1.0\%) however, performance for SSv2-Sub is still similar to using no transformation. Rotation is the most effective transformation, increasing performance on all four datasets.
These differences are likely due to the types of motion present in the downstream datasets. For instance, Gym ($10^3$) and UB-S1 contain gymnastic videos where actors are often spinning and turning but do not change in scale due to the fixed camera, therefore rotation is more helpful than scaling. %
We also experiment with combinations of transformations in supplementary and observe that it does not give further improvement. %Thus,  we choose rotation as the tubelet transformation for our final model.   





\noindent\textbf{Number of Tubelets.} %
We investigate the number of tubelets overlaid on each video in Table~\ref{tab:ablation_num_tubelets}. One tubelet is already more effective than temporal contrastive learning, \eg 29.5\% vs. 39.5\% for Gym ($10^3$). Adding two tubelets improves accuracy on all datasets, \eg +8.5\% for Gym ($10^3$). %

\noindent\textbf{Qualitative Analysis. } %\cs{This takes too much space, can easily be shortened.}
To understand what our model learns Figure~\ref{fig:caam} visualizes class agnostic activation maps~\cite{CAAM} without finetuning for the temporal contrastive baseline and our approach. % 
Without previously seeing any FineGym data, our approach attends to motions. The temporal contrastive baseline instead attends to the background. % 

\vspace{-0.3em}
\subsection{Video-Data Efficiency} 
\vspace{-0.3em}
To demonstrate our method's data efficiency, we pretrain using subsets of the Kinetics-400 training data. In particular, we sample $5\%,10\%,25\%,33\% \text{ and } 50\%$ of the Kinetics-400 training set with three different seeds and use this to pretrain our model and the temporal contrastive baseline. %We repeat the experiment three times to sample different subsets of data with different seeds. 
We compare the effectiveness of these representations after finetuning on UCF ($10^{3}$), Gym($10^{3}$), SSv2-Sub, UB-S1, and HMDB51 in Figure~\ref{fig:data_efficiency}.
%
On all downstream setups, our method maintains similar performance when reducing the pretraining data to just 25\%, while the temporal contrastive baseline performance decreases significantly. Our method is less effective when using only 5\% or 10\% of the data. Remarkably, with this little data, our model still outperforms the baseline with 100\% for Gym ($10^3$), UB-S1, and HMDB. We attribute our model's data efficiency to the tubelets we add to the pretraining data. In particular, our non-linear motion and transformations generate a variety of synthetic tubelets that 
simulate a greater variety of fine-grained motions than are present in the original data. 
%when overlaid onto different backgrounds steers the model to focus on learning fine-grained motion patterns while ignoring shortcuts apparent in the background.








\begin{table}[t!]
\centering
    \setlength{\tabcolsep}{6pt}
\resizebox{0.92\linewidth}{!}{
    \begin{tabular}{llcc}
        \toprule
        \textbf{Method} & \textbf{Modality} & UCF101 & HMDB51\\
        
        \midrule
        Pace Prediction~\cite{pace} & RGB &   77.1 & 36.6 \\
        VideoMoCo~\cite{videomoco-pan2021videomoco} & RGB &   78.7 & 49.2 \\
        RSPNet~\cite{rspnet-chen2020RSPNet} & RGB  &  81.1 &44.6 \\
        SRTC~\cite{srtc} & RGB &    82.0 & 51.2 \\
        FAME~\cite{fame}  & RGB &    {84.8}	& {53.5} \\
        MCN~\cite{mcn} & RGB     & 84.8 & 54.5 \\
        \color{gray}
        AVID-CMA~\cite{avid-cma-morgado2021audio} & \color{gray} RGB+Audio & \color{gray} 87.5 & \color{gray} 60.8 \\
        TCLR~\cite{dave2022tclr} & RGB &   88.2 &60.0 \\
        TE~\cite{jenni2021time_eqv} & RGB & 88.2 &	62.2\\	
        CtP~\cite{ctp-wang2021unsupervised} & RGB & {88.4}	& {61.7} \\
        \color{gray}
        MotionFit~\cite{motion_fit} & \color{gray} RGB+Flow &   \color{gray} 88.9 & \color{gray} 61.4 \\ 
        \color{gray} GDT~\cite{gdt-patrick2020multimodal} & \color{gray} RGB+Audio &  \color{gray} 89.3 & \color{gray} 60.0 \\
        
        \midrule
        %\rowcolor{visualblue}
        \textbf{\textit{This paper $^\dagger$}} & RGB &  \textbf{90.7}	& \textbf{65.0} \\
        %\rowcolor{visualblue}
        \textbf{\textit{This paper}} & RGB & \textbf{91.0}	& \textbf{64.1} \\
        \bottomrule
    \end{tabular}
    }
    \vspace{-0.8em}
    \caption{\textbf{Standard Evaluation: UCF101 and HMDB51} using R(2+1)D. 
    %All models utilize Kinetics-400 for self-supervised pretraining. 
    Gray lines indicate use of additional modalities during self-supervised pretraining. Note that our method pretrained on Mini-Kinetics ($\dagger$) outperforms all methods which pretrain on the 3${\times}$ larger Kinetics-400.
    }
    \vspace{-0.8em}
    \label{tab:ucf_hmdb_sota}
\end{table}

\begin{table*}[t!]
    \centering
    \midsepremove
    \setlength{\tabcolsep}{5pt}
    \resizebox{\linewidth}{!}{\begin{tabular}
     {l  ccc cc ccccc}
    \toprule
    
      \addlinespace[0.1cm]
        & \multicolumn{2}{Sc}{\textbf{Domains}}   & \multicolumn{2}{Sc}{\textbf{Samples}}  & \multicolumn{2}{Sc}{\textbf{Actions}} & \multicolumn{2}{Sc}{\textbf{Tasks}}\\
      \cmidrule(lr){2-3} \cmidrule(lr){4-5} \cmidrule(lr){6-7} \cmidrule(lr){8-9}
         %\multicolumn{1}{l}{\textbf{Methods}}  \\ 
      \addlinespace[0.1cm]
         &   \multicolumn{1}{c}{SS-v2} & \multicolumn{1}{c}{Gym-99}  & \multicolumn{1}{c}{UCF ($10^{3}$)} & \multicolumn{1}{c}{Gym ($10^{3}$)}  & \multicolumn{1}{c}{FX-S1 } & \multicolumn{1}{c}{UB-S1}&  \multicolumn{1}{c}{UCF-RC$\downarrow$} &  \multicolumn{1}{c}{Charades-MLC} & \textbf{Mean } & \textbf{Avg Rank$\downarrow$}\\
         \midrule
           \addlinespace[0.01cm]
         Supervised~\cite{tran2018closer}               & 60.8   & 92.1   & 86.6   & 51.3   & 79.0   & 87.1 & 0.132 & 23.5   &  18.0                     & -            \\
         \midrule
         None                     & 57.1   & 89.8   & 38.3   & 22.7   & 46.6   & 82.3 & 0.217 & ~7.9    &  -                         &11.1            \\
         SeLaVi~\cite{selavi}                  & 56.2   & 88.9   & 69.0   & 30.2   & 51.3   & 80.9 & 0.162 & ~8.4    & ~5.7                       & ~9.3            \\
         MoCo~\cite{moco}                     & 57.1   & 90.7   & 60.4   & 30.9   & 65.0   & 84.5 & 0.208 & ~8.3    & ~6.6                       & ~8.5            \\
         VideoMoCo~\cite{videomoco-pan2021videomoco}                & 59.0   & 90.3   & 65.4   & 20.6   & 57.3   & 83.9 & 0.185 & 10.5   & ~5.6                       & ~8.1            \\
         Pretext-Contrast~\cite{pretext-contrast}         & 56.9   & 90.5   & 64.6   & 27.5   & 66.1   & 86.1 & 0.164 & ~8.9    & ~7.6                       & ~7.5            \\
         AVID-CMA~\cite{avid-cma-morgado2021audio}                 & 52.0   & 90.4   & 68.2   & 33.4   & 68.0   & 87.3 & 0.148 & ~8.2    & ~8.7                       & ~6.6            \\
         GDT~\cite{gdt-patrick2020multimodal}                      & 58.0   & 90.5   & \textbf{78.4}   & 45.6   & 66.0   & 83.4 & \textbf{0.123} & ~8.5    & 11.8    & ~5.4            \\
         RSPNet~\cite{rspnet-chen2020RSPNet}                   & 59.0   & 91.1   & 74.7   & 32.2   & 65.4   & 83.6 & 0.145 & ~9.0    & ~9.6                       & ~5.3            \\
         CtP~\cite{ctp-wang2021unsupervised}                      & 59.6   & 92.0   & 61.0   & 32.9   & 79.1   & 88.8 & 0.178 & ~9.6    & 10.2                      & ~4.9            \\
         TCLR~\cite{dave2022tclr}                     & 59.8   & 91.6   & 72.6   & 26.3   & 60.7   & 84.7 & 0.142 & \textbf{12.2}   & ~8.8              & ~4.6            \\
         \midrule                                                                  \addlinespace[0.03cm]                          %                  \rowcolor{visualblue}       
         %\rowcolor{audiogreen}
         \textbf{\textit{This paper $^\dagger$}}                      & {59.4}   & {92.2}   & {65.5}   & \textbf{48.0}   & {78.3}   & 90.9 & {0.150}  & {9.0}   & 13.1  &~3.6    \\
                  \textbf{\textit{This paper}}                      & \textbf{60.2}   & \textbf{92.8}   & {65.7}   & 47.0   & \textbf{80.1}   & \textbf{91.0} & {0.150}  & {10.3}   & \textbf{13.6}  & ~\textbf{2.5}    \\
        \addlinespace[0.01cm]
         \bottomrule
    \end{tabular}
    }
    \vspace{-0.8em}
         \caption{\textbf{SEVERE Generalization Benchmark.}
         Comparison with prior self-supervised methods for generalization to downstream domains, fewer samples, action granularity, and tasks. $\downarrow$ indicates lower is better. Results for baselines are taken from SEVERE \cite{thoker2022severe}.
         Our method generalizes best, even when using the 3x smaller Mini-Kinetics dataset ($\dagger$) for pretraining.}
    \label{tab:severe-performance}
    \vspace{-0.5em}
\end{table*}

\vspace{-0.3em}
\subsection{Standard Evaluation: UCF101 and HMDB51}
\vspace{-0.3em}
We first show the effectiveness of our proposed method on standard coarse-grained action recognition benchmarks UCF101 and HMBD51, where we compare with prior video self-supervised works. For a fair comparison, we only report methods in Table~\ref{tab:ucf_hmdb_sota} that use the R(2+1)D-18 backbone and Kinetics-400 as the pretraining dataset. 

%\noindent\textbf{RGB-only Comparison.} 
First, we observe that %among the self-supervised methods that only use the RGB modality for pretraining, 
our method obtains the best results for UCF101 and HMDB51. % with R(2+1)D-18 backbone. 
 The supplementary material shows we also achieve similar improvement with the R3D backbone. In particular, with R(2+1)D our method beats CtP~\cite{ctp-wang2021unsupervised} by 2.6\% and 2.4\%, TCLR~\cite{dave2022tclr} by 2.8\% and 4.1\%, and TE~\cite{jenni2021time_eqv} by 2.8\% and 1.9\% all of which aim to learn finer temporal representations.   This confirms that explicitly contrasting tubelet-based motion patterns   
 results in a better video representation than learning temporal distinctiveness or prediction. %explicitly contrasting motion patterns in the form of tubelets  results in learning much better temporal representations than learning to track patches (CtP) or temporal distinctiveness (TCLR) or temporal equivariance (TE).
%Notably, FAME~\cite{fame} shares the same motivation as ours, be it they learn motion-aware representations 
We also outperform FAME~\cite{fame} by 6.2\% and 9.6\% on UCF101 and HMDB51. FAME, like our method, aims to learn a motion-focus representation.
It does this by pasting the foreground region of one video onto the background of another to construct positive pairs for contrastive learning. We however are not limited by the motions present in the set of pretraining videos as we simulate new motion patterns for learning. We also outperform prior multi-modal works which incorporate audio or explicitly learn motion from optical flow. %
Since our model is data-efficient, we can pretrain on Mini-Kinetics and still outperform all baselines which are trained on the 3x larger Kinetics-400. 


\vspace{-0.3em}
\subsection{SEVERE Generalization Benchmark} 
\vspace{-0.3em}
Next, we compare to prior works on the challenging SEVERE benchmark \cite{thoker2022severe}, which evaluates self-supervised video representations for generalizability along \textit{domain shift}, \textit{sample efficiency}, \textit{action granularity}, and \textit{task shift}. We strictly follow the same setup as in the original SEVERE benchmark using the author-provided codebase. We use an R(2+1)D-18 backbone pretrained on Kinetics-400 with our tubelet-contrastive learning before finetuning on the different downstream settings.
%
The results are shown in Table~\ref{tab:severe-performance}.




\noindent \textbf{Domain Shift.} Among the evaluated methods our proposal achieves the best performance on SS-v2 and Gym-99. These datasets differ considerably from Kinetics-400, particularly in regard to the actions they contain, the environment and the viewpoint. Our improvement demonstrates that the representation from our tubelet-contrastive learning is robust to a variety of domain shifts. %
\noindent \textbf{Sample Efficiency.}
For the sample efficiency, we achieve a considerable improvement over all prior works on Gym ($10^{3}$), \eg +20.7\%  over TCLR, +14.8\%  over RSPNet and  +14.1\%  over CtP~\cite{ctp-wang2021unsupervised}. Notably, the gap between the second best method GDT~\cite{gdt-patrick2020multimodal} and all other works is large, demonstrating the challenge. For UCF ($10^{3}$), our method is on par with VideoMoCo\cite{videomoco-pan2021videomoco} and CtP~\cite{ctp-wang2021unsupervised} but is outperformed by GDT and RSPNet. %\etc. 
This is likely because most actions in UCF101 require more spatial than temporal understanding, thus it benefits from the additional augmentations used by GDT and RSPNet. Our motion-focused representation  requires more examples for finetuning on such datasets. 

\noindent \textbf{Action Granularity.} For fine-grained actions in FX-S1 and UB-S1, our method achieves the best performance, even outperforming supervised Kinetics-400 pre-training (top row).
 We achieve a considerable improvement over other RGB-only models, \eg, +19.6\% and +6.3\% over TCLR, as well as audio-visual models, \eg, +14.1\% and +7.6\% over GDT. 
These results demonstrate that the video representation learned by our method are better suited  to fine-grained actions than existing self-supervised methods. We additionally report results on  Diving48~\cite{diving48} in the supplementary. 


\noindent \textbf{Task Shift.} 
For the task shift to repetition counting, our method is on par with AVID-CMA and RSPNet, but worse than GDT. For multi-label action recognition on Charades, our approach is 3rd, comparable to VideoMoCo but worse than TCLR.
 %
 %
 This suggests the video representations learned by our method are somewhat transferable to tasks beyond single-label action recognition. However, the remaining gap between supervised and self-supervised highlights the need for future work to explore task generalizability further.

\noindent \textbf{Overall SEVERE Performance.} 
Finally, we compare the average rank and average improvement over no pretraining for all methods across all generalizability factors. Our method achieves the  best average rank (2.5) and has the best mean improvement (13.6). When pretraining with the 3x smaller Mini-Kinetics our method still achieves impressive results. We conclude our method improves the generalizability of video self-supervised representations across these four downstream factors while being data-efficient.


\section{Conclusion}
Throughout the paper, we first analyze the current evaluation methods for diffusion-based adversarial purification and then propose a recommendation for the reliable evaluation of the robustness of adversarial purification. We further investigate the influence of hyperparameters of the diffusion model on the robustness of the purification. Based on our analysis, we propose a new strategy to maximize the benefit of the purification methods.

%CS: for the project and final version
\vspace{1em}
\noindent\textbf{Acknowledgements.}
This work is part of the research programme Perspectief EDL with project number P16-25 project 3, which is financed by the Dutch Research Council (NWO) domain Applied and Engineering/ Sciences (TTW).

{\small
\bibliographystyle{ieee_fullname}
\bibliography{egbib}
}
\appendix
\section{Appendix for Proofs}

\paragraph{Proof of Theorem \ref{thm:main}.}

\begin{proof}
\label{proof:main}
Our proof has two steps. In Step 1, we will show that SimCLR is equivalent to minimizing the cross entropy loss defined in Eqn.~(\ref{eqn:cross-entropy}). 
In Step 2, we will show  that minimizing the cross-entropy loss 
is equivalent to spectral clustering on $\bfpi$. 
Combining the two steps together, we have proved our theorem. 

\textbf{Step 1: } SimCLR is equivalent to minimizing the cross entropy loss.

The cross-entropy loss takes expectation over 
$\bfW_\bfX\sim \mathbb{P}(\cdot ; \bfpi)$, 
which means $\bfW_\bfX$ has exactly one non-zero entry in each row $i$. By Lemma~\ref{lem:multinomial}, we know every row $i$ of $\bfW_\bfX$ is independent of other rows. Moreover, 
$\bfW_{\bfX,i}\sim \mathcal{M}(1, \bfpi_i/\sum_j \bfpi_{i,j})=\mathcal{M}(1, \bfpi_i)$, because $\bfpi_i$ itself is a probability distribution.
Similarly, we know $\bfW_\bfZ$ also has the row-independent property by sampling over $\mathbb{P}(\cdot;\bfK_\bfZ)$.
Therefore, by Lemma~\ref{lem:cross_split}, we know Eqn.~(\ref{eqn:cross-entropy}) is equivalent to:
\[
 -\sum_{i=1}^n \mathbb{E}_{\bfW_{\bfX,i}}[\log \mathbb{P}(\bfW_{\bfZ,i}=\bfW_{\bfX,i};\bfK_\bfZ)],
\]

This expression takes expectation over $\bfW_{\bfX,i}$ for the given row $i$. Notice that 
$\bfW_{\bfX,i}$ has exactly one non-zero entry, which equals $1$ (same for $\bfW_{\bfZ,i}$). 
As a result
we expand the above expression to be:
\begin{equation}
 -\sum_{i=1}^n \sum_{j\neq i} \Pr(\bfW_{\bfX,i,j}=1)\log \Pr(\bfW_{\bfZ,i,j}=1).
\label{eqn:detailed-expansion}    
\end{equation}


By Lemma~\ref{lem:multinomial}, $\Pr(\bfW_{\bfZ,i,j}=1)=\bfK_{\bfZ,i,j}/\|\bfK_{\bfZ,i}\|_1$ for $j\neq i$. Recall that $\bfK_\bfZ=(k(\bfZ_i-\bfZ_j))_{(i,j)\in[n]^2}$, which means 
$\bfK_{\bfZ,i,j}/\|\bfK_{\bfZ,i}\|_1=\frac{\exp(-\|\bfZ_i-\bfZ_j\|^2/{2\tau})}{\sum_{k\neq i}
\exp(-\|\bfZ_i-\bfZ_k\|^2/{2\tau})
}$ for $j\neq i$, when $k$ is the Gaussian kernel with variance $\tau$. 

Notice that $\bfZ_i=f(\bfX_i)$, so we know
\begin{equation}
-\log \Pr(\bfW_{\bfZ,i,j}=1)=
-\log \frac{\exp(-\|f(\bfX_i)-f(\bfX_j)\|^2/{2\tau})}{\sum_{k\neq i}
\exp(-\|f(\bfX_i)-f(\bfX_k)\|^2/{2\tau}),
}
\label{eqn:infonce-equivalence}    
\end{equation}


The right hand side is exactly the InfoNCE loss defined in Eqn.~(\ref{eqn:infonce}).
Inserting Eqn.~(\ref{eqn:infonce-equivalence}) into Eqn.~(\ref{eqn:detailed-expansion}), we get the SimCLR algorithm, which first samples augmentation pairs $(i,j)$ with $\Pr(\bfW_{\bfX,i,j}=1)$ for each row $i$, and then optimize the InfoNCE loss. 

\textbf{Step 2: } minimizing the cross entropy loss 
is equivalent to spectral clustering on $\bfpi$.


By Lemma~\ref{lem:convert_to_spectral}, we may further convert the loss to 
\begin{equation}
\label{eqn:main-theorem-repul-attr}
\min_{\bfZ}
-\sum_{(i,j)\in [n]^2} \mathbf{P}_{i,j}
\log k (\bfZ_i-\bfZ_j)+\log \mathbf{R}(\bfZ).
\end{equation}
Since $k$ is the Gaussian kernel, this reduces to \[
\min_\bfZ \mathrm{tr}(\bfZ^\top \mathbf{L}(\bfpi) \bfZ)
+\log \mathbf{R}(\bfZ),
\]

where we use the fact that $\mathbb{E}_{\bfW_\bfX\sim \mathbb{P}(\cdot; \bfpi)}[\mathbf{L}(\bfW_\bfX)]
=\mathbf{L}(\bfpi)
$, because the Laplacian operator is linear and $
\mathbb{E}_{\bfW_\bfX\sim \mathbb{P}(\cdot; \bfpi)}(\bfW_\bfX)=\bfpi
$.
\end{proof}

\paragraph{Proof of Theorem \ref{thm:clip}.}
\begin{proof}
Since $\bfW_\bfX\sim \mathbb{P}(\cdot;\bfpi_{\mathbf{A}, \mathbf{B}})$, we know 
$\bfW_\bfX$ has exactly one non-zero entry in each row, denoting the pair that got sampled. 
A notable difference compared to the previous proof is we now have $n_\mathcal{A}+n_\mathcal{B}$ objects in our graph. CLIP deals with this by taking a mini-batch of size $2N$, 
such that $n_\mathcal{A}=n_\mathcal{B}=N$, and adding the $2N$ InfoNCE losses together. We label the objects in $\mathcal{A}$ as $[n_\mathcal{A}]$, and the objects in $\mathcal{B}$ as $\{n_\mathcal{A}+1, \cdots, n_\mathcal{A}+n_\mathcal{B}\}$. 

Notice that $\bfpi_{\mathbf{A}, \mathbf{B}}$ is a bipartite graph, so the edges of objects in $\mathcal{A}$ will only connect to object in $\mathcal{B}$ and vice versa. We can define the similarity matrix in $\cZ$ as $\bfK_\bfZ$, 
where $\bfK_\bfZ(i, j+n_\mathcal{A})=\bfK_\bfZ(j+n_\mathcal{A},i)= k(\bfZ_i-\bfZ_j)$ for $i\in [n_\mathcal{A}], j\in [n_\mathcal{B}]$, and otherwise we set $\bfK_\bfZ(i,j)=0$. 
The rest is same as the previous proof. 
\end{proof}

\paragraph{Proof of Theorem \ref{thm:exponential}.}

\begin{proof}
\label{proof:exponential}
Since the objective function consists of a linear term combined with an entropy regularization, which is a strongly concave function, the maximization problem is a convex optimization problem. Owing to the implicit constraints provided by the entropy function, the problem is equivalent to having only the equality constraint. We then introduce the Lagrangian multiplier $\lambda$ and obtain the following relaxed problem:

$$
\widetilde{E}(\boldsymbol{\alpha})=\psi_{1}-\sum_{i=1}^n \alpha_{i} \psi_{i}+\tau \sum_{i=1}^n \alpha_{i}\log \alpha_{i}+\lambda\left(\boldsymbol{\alpha}^{\top} \mathbf{1}_n-1\right).
$$

As the relaxed problem is unconstrained, taking the derivative with respect to $\alpha_{i}$ yields

$$
\frac{\partial \widetilde{E}(\boldsymbol{\alpha})}{\partial \alpha_{i}}=-\psi_{i}+\tau\left(\log \alpha_{i}+\alpha_{i} \frac{1}{\alpha_{i}}\right)+\lambda=0.
$$

Solving the above equation implies that $\alpha_{i}$ takes the form
$
\alpha_{i}=\exp \left(\frac{1}{\tau} \psi_{i}\right) \exp \left(\frac{-\lambda}{\tau}-1\right).
$ Since $\alpha_{i}$ lies on the probability simplex, the optimal $\alpha_{i}$ is explicitly given by
$
\alpha^{*}_{i}=\frac{\exp \left(\frac{1}{\tau} \psi_{i}\right)}{\sum_{i^{\prime}=1}^n \exp \left(\frac{1}{\tau} \psi_{i^{\prime}}\right)} .
$ Substituting the optimal point into the objective function, we obtain
$$
\begin{aligned}
E\left(\boldsymbol{\alpha}^*\right)  &=\psi_1-\sum_{i=1}^n \frac{\exp \left(\frac{1}{\tau} \psi_{i}\right)}{\sum_{i^{\prime}=1}^n \exp \left(\frac{1}{\tau} \psi_{i^{\prime}}\right)} \psi_{i}+\tau \sum_{i=1}^n \frac{\exp \left(\frac{1}{\tau} \psi_{i}\right)}{\sum_{i^{\prime}=1}^n \exp \left(\frac{1}{\tau} \psi_{i^{\prime}}\right)}\log \frac{\exp \left(\frac{1}{\tau} \psi_{i}\right)}{\sum_{i^{\prime}=1}^n \exp \left(\frac{1}{\tau} \psi_{i^{\prime}}\right)} \\
& =\psi_1 - \tau \log \left(\sum_{i=1}^n \exp \left(\frac{1}{\tau} \psi_{i}\right)\right).
\end{aligned}
$$
Thus, the Lagrangian dual function is given by
\begin{equation*}
-E\left(\boldsymbol{\alpha}^*\right)= -\tau \log \frac{\exp \left(\frac{1}{\tau} \psi_{1}\right)}{\sum_{i=1}^n \exp \left(\frac{1}{\tau} \psi_{i}\right)}.\qedhere
\end{equation*}
\end{proof}



\section{More on Experiments} \label{section: experiment_details}

\paragraph{CIFAR-10 and CIFAR-100} CIFAR-10 ~\citep{krizhevsky2009learning} and CIFAR-100 ~\citep{krizhevsky2009learning} are well-known classic image classification datasets. Both CIFAR-10 and CIFAR-100 contain a total of 60k $32 \times 32$ labeled images of different classes, with 50k for training and 10k for testing. CIFAR-10 is similar to CIFAR-100, except there are 10 different classes in CIFAR-10 and 100 classes in CIFAR-100.

\paragraph{TinyImageNet} TinyImageNet ~\citep{le2015tiny} is a subset of ImageNet ~\citep{deng2009imagenet}. There are 200 different object classes in TinyImageNet, with 500 training images, 50 validation images, and 50 test images for each class. All the images in TinyImageNet are colored and labeled with a size of $64 \times 64$.

\textbf{Pseudo-code.} Algorithm \ref{alg:Training Procedure} presents the pseudo-code for our empirical training procedure.

\begin{algorithm}[!htbp]
\caption{Training Procedure}
\label{alg:Training Procedure}
\begin{algorithmic}[1]
\REQUIRE trainable encoder network $f$, batch size $N$, augmentation strategy \textit{aug}, loss function $L$ with hyperparameters \textit{args}
\FOR {sampled minibatch ${x_i}_{i=1}^N$}
\FORALL{$i \in { 1, ..., N }$}
\STATE draw two augmentations $t_i = \textit{aug}\left(x_i\right) $, $t_i' = \textit{aug}\left(x_i\right) $
\STATE $z_i = f\left(t_i\right)$, $z_i' = f\left(t_i'\right)$
\ENDFOR
\STATE compute loss $\mathcal{L} = L(N, z, z', \textit{args})$
\STATE update encoder network $f$ to minimize $\mathcal{L}$
\ENDFOR
\STATE \textbf{Return} encoder network $f$
\end{algorithmic}
\end{algorithm}

We also provide the pseudo-code for our core loss function used in the training procedure in Algorithm \ref{alg:Core loss}. The pseudo-code is almost identical to SimCLR's loss function, with the exception of an extra parameter $\gamma$.

\begin{algorithm}[!htbp]
\caption{Core loss function $\mathcal{C}$}
\label{alg:Core loss}
\begin{algorithmic}[1]
\REQUIRE batch size $N$, two encoded minibatches $z_1, z_2$, $\gamma$, temperature $\tau$
\STATE $z = \textit{concat}\left(z_1, z_2\right)$
\FOR {$i \in {1, ..., 2N }, j \in {1, ..., 2N}$ }
\STATE $s_{i,j} = \Vert z_i - z_j \Vert_2^{\gamma}$
\ENDFOR
\STATE \textbf{define} $l(i, j)$ \textbf{as} $l(i, j) = - \log \frac{exp\left(s_{i,j}/\tau \right)}{\sum_{k=1}^{2N} \mathbf{1}{[k \ne i]} exp\left(s{i, j} / \tau \right)} $
\STATE \textbf{Return} $\frac{1}{2N} \sum_{k=1}^N\left[l(i, i+N) + l(i+N, i)\right]$
\end{algorithmic}
\end{algorithm}

Utilizing the core loss function $\mathcal{C}$, we can define all kernel loss functions used in our experiments in Table \ref{table: loss definition}. For all $z_i \in z$ with even dimensions $n$, we define $z_{L_i} = z_i\left[0:n/2\right]$ and $z_{R_i} = z_i\left[n/2:n\right]$.

\begin{table}[ht]
\centering
\begin{tabular}{{@{}l|l@{}}}
Kernel  &  Loss function \\ \midrule
Laplacian & $\mathcal{C}\left(N, z, z', \gamma=1, \tau\right)$\\ \midrule
Sum       & $\lambda * \mathcal{C}\left(N, z, z', \gamma=1, \tau_1\right) + (1-\lambda) * \mathcal{C}\left(N, z, z', \gamma=2, \tau_2\right)$  \\ \midrule
Concatenation Sum&$\lambda * \mathcal{C}\left(N, z_L, z'_L, \gamma=1, \tau_1\right) + (1-\lambda) * \mathcal{C}\left(N, z_R, z'_R, \gamma=2, \tau_2\right)$\\ \midrule
$\gamma = 0.5$ & $\mathcal{C}\left(N, z, z', \gamma=0.5, \tau\right)$          \\ 

\end{tabular}

\caption{Definition of kernel loss functions in our experiments}
\label {table: loss definition}
\end{table}

\textbf{Baselines.} We reproduce the SimCLR algorithm using PyTorch Lightning~\citep{PytorchLightning}.

\textbf{Encoder details.}
The encoder $f$ consists of a backbone network and a projection network. We employ ResNet50~\citep{ResNet} as the backbone and a 2-layer MLP (connected by a batch normalization~\citep{ioffe2015batch} layer and a ReLU \cite{nair2010rectified} layer) with hidden dimensions 2048 and output dimensions 128 (or 256 in the concatenation kernel case).

\textbf{Encoder hyperparameter tuning.}
For each encoder training case, we randomly sample 500 hyperparameter groups (sample details are shown in Table \ref{table: Hyperparameter sample}) and train these samples simultaneously using Ray Tune ~\citep{RayTune}, with the ASHA scheduler~\citep{li2018massively}. Ultimately, the hyperparameter group that maximizes the online validation accuracy (integrated in PyTorch Lightning) within 5000 validation steps is chosen for the given encoder training case.

\begin{table}[ht]
\centering

\begin{tabular}{@{}l|l|l@{}}
\midrule
Hyperparameter  & Sample Range & Sample Strategy \\ \midrule
start learning rate & $\left[10^{-2}, 10\right]$ & log uniform \\ \midrule
$\lambda$       & $\left[0, 1\right]$ & uniform \\ \midrule
$\tau$, $\tau_1$, $\tau_2$ & $\left[0, 1\right]$ & log uniform \\ \midrule
\end{tabular}

\caption{Hyperparameters sample strategy}
\label {table: Hyperparameter sample}
\end{table}

\textbf{Encoder training.} 
We train each encoder using the LARS optimizer~\citep{LARSOptimizer}, LambdaLR Scheduler in PyTorch, momentum 0.9, weight decay $10^{-6}$, batch size 256, and the aforementioned hyperparameters for 400 epochs on a single A-100 GPU.

\textbf{Image transformation.} The image transformation strategy, including augmentation, is identical to the default transformation strategy provided by PyTorch Lightning.

\textbf{Linear evaluation.}
The linear head is trained using the SGD optimizer with a cosine learning rate scheduler, batch size 64, and weight decay $10^{-6}$ for 100 epochs. The learning rate starts at $0.3$ and ends at $0$.

\textbf{Moco Experiments.} We also tested our method based on MoCo~\citep{he2019moco}. The results are summarized in Table \ref{tab:results-moco}. Here we choose ResNet18~\citep{ResNet} as the backbone and set a temperature of $0.1$ as default. For our simple sum kernel, we set $\lambda=0.8$. The results show that our method outperforms the original MoCo method.

\begin{table}[thb]
\centering
\caption{MoCo Experiment Results on CIFAR-10 and CIFAR-100.}
\label{tab:results-moco}
\resizebox{\textwidth}{!}{%
\begin{tabular}{@{}c|ccc|ccc@{}}
\toprule
\multirow{3}{*}{Method} & \multicolumn{3}{c|}{CIFAR-10} & \multicolumn{3}{c}{CIFAR-100} \\ \cmidrule(lr){2-4} \cmidrule(lr){5-7} 
                        & 200 epochs & 400 epochs    & 1000 epochs   & 200 epochs & 400 epochs & 1000 epochs         \\ \midrule
MoCo (repro.)         & $76.41 \pm 0.12$    & $80.01 \pm 0.15$          & $84.45 \pm 0.08$    & $\mathbf{47.02 \pm 0.11}$ & $52.50 \pm 0.07$ & $57.62 \pm 0.15$            \\
\midrule
Laplacian Kernel        & ${78.09 \pm 0.10}$    & $\mathbf{83.85 \pm 0.09}$          & $\mathbf{88.34 \pm 0.16}$    & $46.12 \pm 0.22$   & $53.44 \pm 0.17$ & $59.10 \pm 0.14$        \\
Simple Sum Kernel & $\mathbf{78.12 \pm 0.15}$   & $83.23 \pm 0.18$ & $87.50 \pm 0.20$ & $46.65 \pm 0.06$ & $\mathbf{53.62 \pm 0.19}$ & $\mathbf{59.83 \pm 0.12}$\\
\bottomrule
\end{tabular}
}
\end{table}



\section{More Experiments on Synthetic Data}


Consider a scenario with $n$ clusters, each containing $k$ vertices. Let the probability of vertices $u$ and $v$ from the same cluster belonging to $\bfpi$ be $p$. Conversely, for vertices $u$ and $v$ from different clusters, let the probability of belonging to $\pi$ be $q$. We generate the graph $\bfpi$ randomly, based on $p$ and $q$. We experiment with values of $k=100$ and $n=6$ for ease of visualization, embedding all points in a two-dimensional space. Each vertex's initial position originates from a normal distribution. In each iteration, we sample a subgraph of $\bfpi$ uniformly, ensuring each vertex has an out-degree of $1$. We then optimize the corresponding vectors using InfoNCE loss with an SGD optimizer and iterate until convergence. Our experimental setup consists of an SGD learning rate of $1$, an InfoNCE loss temperature of $0.5$, and a batch size of $50$. We evaluate two scenarios with different $p$ and $q$ values: $p=1$, $q=0$, and $p=0.75$, $q=0.2$. The results of these experiments are visualized in Figure \ref{fig:vis-spectral-cluster}. The obtained embeddings exhibit the hallmark pattern of spectral clustering of graph $\bfpi$.

\begin{figure}[!tb]
\centering
\subfigure{
\includegraphics[width=1\textwidth]{Figures/cluster_pi.png}
\label{fig:vis-cluster}
}
\subfigure{
\includegraphics[width=1\textwidth]{Figures/noised_cluster_pi.png}
\label{fig:vis-noised-cluster}
}
\caption{Visualizations of the optimization process using InfoNCE Loss on the vectors corresponding to $\bfpi$. Points of identical color belong to the same cluster within $\bfpi$. To showcase the internal structure of $\bfpi$, we randomly select 10 vertices from each cluster to display the edge distribution of $\bfpi$.}
\label{fig:vis-spectral-cluster}
\end{figure}



\end{document}