\title{
    Translate your gibberish: black-box adversarial attack on machine translation systems
}
\titlerunning{Black-box adversarial attack on machine translation systems}

\author{
%Anonymous authors
    Andrei Chertkov\inst{1, 2}\orcidID{0000-0001-9990-6598}
    \and
    Olga Tsymboi\inst{3, 4}\orcidID{0000-0002-8078-1876}
    \and
    Mikhail Pautov\inst{1}\orcidID{0000-0003-0438-6361}
    \and
    Ivan Oseledets\inst{1,2,5}\orcidID{0000-0003-2071-2163}
}
\authorrunning{A. Chertkov et al.}

\institute{
    Skolkovo Institute of Science and Technology, Moscow, Russia
    \email{\{a.chertkov,mikhail.pautov,i.oseledets\}@skoltech.ru}
    \and
    Institute of Numerical Mathematics, Russian Academy of Sciences
    \and
    Moscow Institute of Physics and Technology, Moscow, Russia
    \email{tsimboy.oa@phystech.edu}
    \and
    Sber AI Lab, Moscow, Russia
    \and
    AIRI, Moscow, Russia
}

\maketitle

\begin{abstract}
    Neural networks are deployed widely in natural language processing tasks on the industrial scale, and perhaps the most often they are used as compounds of automatic machine translation systems. In this work, we present a simple approach to fool state-of-the-art machine translation tools in the task of translation from  Russian to English and vice versa. Using a novel black-box gradient-free tensor-based optimizer, we show that many online translation tools, such as Google, DeepL, and Yandex, may both produce wrong or offensive translations for nonsensical adversarial input queries and refuse to translate seemingly benign input phrases. This vulnerability may interfere with understanding a new language and simply worsen the user's experience while using machine translation systems, and, hence, additional improvements of these tools are required to establish better translation.

\keywords{
    Natural language processing
    \and
    Machine translation
    \and
    Adversarial attack
    \and
    Black-box optimization
}
\end{abstract}