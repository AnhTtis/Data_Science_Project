\documentclass[prd,twocolumn,showpacs,preprintnumbers,amsmath,amssymb,floatfix]{revtex4-1}
%\documentstyle[prb,aps,psfig]{revtex}
%\usepackage{simplewick}% wick contraction
\usepackage{graphicx}% Include figure files
%\useackage{dcolumn}% Align table columns on decimal point
%\usepackage{bm}% bold math
%\usepackage{psfig}
\usepackage{float}
\usepackage{trfsigns} 

\newcommand {\sll}[1]{{\slashed #1}}

%\usepackage{trfsigns}


\usepackage{tensor}
\newcommand{\te}[2]{\tensor{#1}{#2}}

\newcommand {\con}[3]{\contraction[0.5ex]{}{#1}{#2}{#3}\nomathglue{#1#2#3}}
\newcommand \be{\begin{eqnarray}}
\newcommand \ee{\end{eqnarray}}


\newcommand \ba{\begin{align}}
\newcommand \eea{\end{align}}
\newcommand {\ket}[1]{|#1\rangle}
\newcommand {\bra}[1]{\langle #1|}
\newcommand {\f}[1]{(\ref{#1})}
\newcommand {\z}[1]{\cite{#1}}
\newcommand {\p}[1]{\partial_{#1}}
\newcommand \s{{\backslash\hspace{-1ex}i%\O
}}
\newcommand \bd{\mathbf}
\newcommand \V{\vec}
%\newcommand {\V}[1]{{\bf #1}}
\newcommand {\D}[1]{{\dot {\V #1}}}
%\renewcommand{\vec}[1]{\text{\boldmath{$ #1 $}}}
\newcommand {\ov}[1]{\overline{#1}}
\DeclareMathOperator{\Tr}{Tr}

\newcommand{\rpm}{\textcircled{$\scriptstyle\pm$}}
\newcommand{\rrmp}{\textcircled{$\scriptstyle\mp$}}

\newcommand{\ch}[2]{\genfrac{\{}{\}}{0pt}{}{#1}{#2}}


\begin{document}
%\twocolumn[\hsize\textwidth\columnwidth\hsize
           \csname @twocolumnfalse\endcsname
\title{Time behaviour of Hubble parameter by torsion}
\author{Klaus Morawetz$^{1,2}$
%, Vinod Ashokan$^3$, Neil Drummond$^4$, 
%Renu Bala$^5$,
%Kare Narain Pathak$^3$
%, Gianaurelio Cuniberti$^6$
}
\affiliation{$^1$M\"unster University of Applied Sciences,
Stegerwaldstrasse 39, 48565 Steinfurt, Germany}
\affiliation{$^2$International Institute of Physics- UFRN,
Campus Universit\'ario Lagoa nova,
59078-970 Natal, Brazil
}
%\affiliation{$^{3}$ Max-Planck-Institute for the Physics of Complex Systems, 01187 Dresden, Germany}
%\affiliation{$^3$ Centre for Advanced Study in Physics, Panjab University, 160014 Chandigarh, India}
%\affiliation{$^4$Department of Physics, Lancaster University, Lancaster LA1 4YB, United Kingdom}
%\affiliation{$^5$ Department of Physics, MCM DAV College for Women, 160036 Chandigarh, India}
%\affiliation{$^6$ Institute for Materials Science, TU Dresden, 01062 Dresden, Germany}



\begin{abstract}
Consequences of the consistent exact solution of Einstein-Cartan equation on the time dependence of Hubble parameter is discussed. The torsion leads to a space and time dependent expansion parameter which results into nontrivial windows of Hubble parameter between diverging behaviour. Only one window shows a period of decreasing followed by increasing time dependence. Provided a known cosmological constant and the present values of Hubble and deceleration parameter this changing time can be given in the past as well as the ending time of the windows or universe. From the metric with torsion outside matter it is seen that torsion can feign dark matter. 
\end{abstract}
%\pacs{
%03.65.Nk %      Scattering theory
%,21.45.+v %     Few-body systems
%,72.10.Fk %     Scattering by point defects, dislocations, surfaces, and other imperfections (including Kondo effect)
%,03.65.Ge %     Solutions of wave equations: bound states
%,34.80.Pa %     Coherence and correlation in electron scattering
%,34.10.+x %     General theories and models of atomic and molecular collisions and interactions (including statistical theories, transition state, stochastic and trajectory models, etc.)
%,68.65.Hb %Quantum dots
%,73.22.-f %     Electronic structure of nanoscale materials: clusters, nanoparticles, nanotubes, and nanocrystals
%,79.20.Rf %    Atomic, molecular, and ion beam impact and interactions with surfaces
%, 61.14.Dc %    Theories of diffraction and scattering
%,61.46.+w %     Nanoscale materials: clusters, nanoparticles, nanotubes, and nanocrystals
%71.45.Gm, %      Exchange, correlation, dielectric and magnetic response functions, plasmons
%78.20.-e, %      Optical properties of bulk materials and thin films
%78.47.+p, %      Time-resolved optical spectroscopies and other ultrafast optical measurements in condensed matter
%42.65.Re, %     Ultrafast processes; optical pulse generation and pulse compression
%82.53.Mj, %     Femtosecond probing of semiconductor nanostructures
%71.10.+w, %Theories and models of many-electron systems
% 71.70.Ej, %Spin-orbit coupling, Zeeman and Stark splitting, Jahn-Teller effect
%75.76.+j, %Spin transport effects
% 85.75.Ss %Magnetic field sensors using spin polarized transport
% 03.75.-b, %Matter waves
%74.20.-z, %Theories and models of superconducting state
%67.85.-d, %Ultracold gases, trapped gases
%64.70.Nd, %Structural transitions in nanoscale materials
%75.70.Ak, %Magnetic properties of monolayers and thin films
%75.30.Fv, %Spin-density waves
%}
\maketitle
%    \vskip2pc]

\section{Introduction}

There is an ongoing discrepancy of Hubble data from the early universe obtained by background radiation and data from present galaxies \cite{Riess22}:
"We find a 5$\sigma$ difference with the prediction of H0 from Planck cosmic microwave background
observations under $\lambda$CDM, with no indication that the discrepancy arises from measurement uncertainties or
analysis variations considered to date. The source of this now long-standing discrepancy between direct and
cosmological routes to determining H0 remains unknown." This discrepancy is also further supported by quasars at far distance \cite{Li21}.
In figure~\ref{hubble_riess} the data of the earlier paper are presented and obviously the discrepancy has increased now by 5$\sigma$. 
\begin{figure}
\centerline{\includegraphics[width=8.5cm]{hubble_riess1.eps}}
\caption{\label{hubble_riess} The present data of Hubble constant from \cite{Riess19}.}  
\end{figure}

Considering the time dependence of the scale parameter
\be
R(t)=R_0(t_0)[1+H_0(t-t_0)-\frac q 2 H^2(t-t_0)^2+...]
\ee
with the Hubble and deceleration parameters 
\be
H&=&{\dot R \over R}\qquad q=-{\ddot R R\over R^2}
\label{hq}
\ee
we adopt for the moment (\ref{hq}) as a literal definition at presence which would lead to a time change of Hubble parameter
\be
\dot H={\ddot R R-{\dot R}^2\over R^2}.
\ee
Demanding $\dot H >0$ means $\ddot R R>{\dot R}^2$ and with the help of (\ref{hq}) one finds the equivalence
\be
\dot H >0 \,\leftrightarrow \, q<-1.
\ee
Consequently the search is going on for such a negative deceleration parameter. A analysis of the Planck data and SHOES collaboration \cite{CM20} indeed
seems to indicate parameters
\be
H_0&=&75.35\pm 1.68 {\rm km/sMpc}\nonumber\\
q_0&=&-1.08\pm 0.29.
\ee

The theoretical challenge is how the Hubble parameter could increase with time. One possible solution is provided by torsion leading to a metric due to an exact solution \cite{Mo21,*Mo21e} of Einstein-Cartan equations. In the latter the additional gravitational potential due to torsion becomes
\be
U_{ik}={1\over 2} \left [\te{s}{_i^j}s_{jk}+\sigma^2\left (u_iu_k +{g_{ik}\over 2}\right )
\right ]
\label{U1}
\ee 
in the Einstein-Cartan equations
\be
G_{ik}=P_{ik}-\left (\lambda+\frac P 2\right ) g_{ik}=\kappa {T}_{ik}+\kappa \varepsilon Z_{ik}+\kappa^2 U_{ik}.
\label{EC}
\ee
Here we use $P=\te{P}{_i^i}$ and the cosmological constant $\lambda$. An additional part by torsion comes from the Belinfante-Rosenfeld equation \cite{BEL40,Ros40} which relates the dynamical metric ${\cal T}$ and the canonical energy-momentum $T$ tensor by 
\be
{\cal T}_{ik}&=&T_{ik}+\varepsilon Z_{ik}
\nonumber\\
Z_{ik}&=&
-\frac 1 2 \nabla_l (\te{s}{_k^l} u_i+\te{s}{_i^l}u_k+\te{s}{_k_i}u^l).
\label{BR}
\ee
This difference we indicate by a constant $\epsilon=0,1$ dependent whether we use the metric variational principle or the torsion tensor as variation variable. The metric and torsion are dependent on each other even outside matter which consistency results into the antisymmetric spin tensor in terms of the Weyssenhoff spin liquid parameter $2 \sigma^2=\te{s}{^m_l}\te{s}{_m^l}$ with the only nonzero components in spherical coordinates \cite{Mo21}
\be
s_{2,3}(t,r,\theta, \phi)=-s_{3,2}=\sigma r^2 \sin\theta 
%={2\over \kappa} r \sqrt{1-C}  \sin\theta
.
\ee
Besides the Schwarzschild solution with zero torsion it has been found that there exists exactly a second solution of these equations (\ref{EC}) outside matter withe the metric \cite{Mo21}
\ba
ds^2&=
\hat r^2 d\hat t^2\!-\!{3 \over C\!+\! \lambda \hat r^2} d \hat r^2
\!-\!\hat r^2(d\theta^2\!+\!\sin^2\theta d\hat \phi^2)
\label{metricd}
\end{align}
where we abbreviate 
\be
C={3 (1+2 \varepsilon)\over 2(2+3\varepsilon)}
=\left \{\begin{array}{ll}
3/4 & \epsilon=0\cr
{9/10} & \epsilon=1
%\cr
%3/2 &x=-1
\end{array}\right . .
\label{Con}
\ee
A further transformation 
\ba
&\hat r^2=\cos^2 \tilde t \left [c(\tilde r)^2\tan^2\tilde t -1\right ],\,
\coth\hat t=c(\tilde r) \tan \tilde t
\end{align}
with 
\ba
c(\tilde r)={\rm tan} \left (c+ \sqrt{C\over 3} \,{\rm ln}\, \tilde r\right ),
\quad \tilde r=\sqrt{|\lambda|\over 3} r,
\quad \tilde t=\sqrt{|\lambda|\over 3} t
\label{itrafo3}
\end{align}
translates the metric (\ref{metricd}) into a Friedman-Lama\^itre-Robertson-Walker metric 
\be
ds^2=d t^2-a(r, t)\left (d r^2+r^2 d\Omega^2\right ).
\label{ds}
\ee
The expansion or scale parameter becomes now space and time dependent
\be
a(r, \tilde t)=R^2(r,t)={C \sin^2{\sqrt{|\lambda|\over 3}t}\over |\lambda| r^2\cos^2{\left (2 c+ \sqrt{C\over 3} {\rm ln} \,\sqrt{|\lambda|\over 3}r\right )}}
\label{Ryx1}
\ee
with an arbitrary constant $c$. It provides a time-like universe for ${\tilde r}^2<\coth^2 \tilde t-1$ and a space-like otherwise.


\section{Time dependence of Hubble parameter}

\paragraph{Local simplistic picture}
From the seemingly factorization of time and space dependence in the expansion parameter (\ref{Ryx1}) one might be tempted to calculate  locally the Hubble constant $H$ and the delay parameter $q$ directly via (\ref{hq})
as
\be
H(t)&=&{\dot R\over R}=\sqrt{|\Lambda|\over 3}\cot\sqrt{|\Lambda|\over 3} t 
\newline\\
q(t)&=&-{R {\ddot R}\over {\dot R}^2}=\tan^2\sqrt{|\Lambda|\over 3} t ={\lambda\over 3 H^2(t)}
\label{Ht}
\ee
which divides out the spatial dependence seemingly. This result is plotted in figure~\ref{hubble}. We see that we have only a time-decreasing Hubble parameter with $q_0>0$ in contrast to the data above. The reason is that the light coming from the past is running on $r=r_0+c (t-t_0)$ and we cannot just simply divide out the spatial dependence. This we will consider more closely later. 

First let us see to what values the local assumption (\ref{Ht}) will lead which might explain a discrepancy to earlier data. We assume that the starting time $t_0$ is fixed at $H(t_0)=0$. Then the Hubble constant at present time $H_0=H(t_p)$ and $q_0$ is related with the cosmological constant as 
\be
\lambda=3 q_0 H_0^2\sim H_0^2
\ee 
which is in agreement of high-redshift rotation curves and MOND calculations \cite{milgrom17}. Within this simplistic picture we assume the older value $q_0\approx 1/2$ and see from (\ref{Ht})
that the initial time is $t_0=0$ for $H=\infty$, the present and final time where the Hubble constant vanishes read
\be
%t_0&=&{\Pi \over 2 \sqrt{q_0} H_0}
%\nonumber\\
t_p
&=&{1\over \sqrt{q_0}H_0}
(\pi-{\rm arccot}{\sqrt{q_0}})-t_0\approx 11.8\times 10^9 {\rm a}
\nonumber\\
t_\infty&=&{\pi\over \sqrt{q_0}H_0}-t_0\approx 30.2\times 10^9 {\rm a},
\ee
respectively, with $1/H_0\approx 13.6\times 10^9 a$.

\begin{figure}
\includegraphics[width=8cm]{hubble_t.eps}
%\\[1ex]
%\includegraphics[width=4.2cm]{hubble_t_change.eps}
\caption{\label{hubble} Simplistic (locally fixed) time dependence of Hubble constant 
%(left) and its relative time change (right) 
assuming a zero time at vanishing Hubble constant. The present is indicated by dashed lines.}  
\end{figure}

The relative time change of the Hubble constant plotted in figure~\ref{hubble} in this simplistic picture would be at present
%\be
${\dot H/ H}(t_0)\approx -2\times 10^{-10}{\rm a}^{-1}$.
%\ee 
All these times agree astonishingly well with the Big Bang scenario though only locally measured and with the wrong time dependence. We will see that the values changes only slightly if we consider the time the light travels from far distances but will get the right time dependence. The time dependence of the Hubble constant comes here in local approximation only from the structure of the metric and the cosmological constant and is independent of the factor $C$ describing the spatial dependence of the expansion parameter.

\paragraph{Hubble parameter from distant light}

The spatial and time dependence of the expansion parameter (\ref{Ryx1}) allows to consider the time the light is traveling from far distances. In figure~\ref{a_rt} we plot the square of this expansion parameter. Contrary to the simplistic picture before we measure light from distant objects. This means we cannot consider space and time independently as before. Instead we have to consider Hubble parameter at the light path $r=r_0+c (t-t_0)$ indicated by the red line in figure~\ref{a_rt}. One sees the oscillating behaviour with respect to space and time. The spatial variation shows interestingly one additional maxima at large distances before it falls of rapidly. 

\begin{figure}
\centerline{\includegraphics[width=8.5cm]{a_rt_c.eps}
}
\caption{\label{a_rt} The time and space dependence of the expansion parameter (\ref{Ryx1}) with the light path $r=r_0+c (t-t_0)$ (red).}  
\end{figure}

Working in dimensionless values
\be
H=\sqrt{\lambda \over 3} h,\quad \bar r =\sqrt{\lambda \over 3 c^2} r,\quad \tau=\sqrt{\lambda \over 3} t
\ee
we obtain with time derivatives along the light path the Hubble and deceleration parameter
\be
h(\tau)&=&\cot \tau+{\tan{\ln({\bar r_0}+\tau)\over \sqrt{3}}-\sqrt{3}
\over \sqrt{3}({\bar r_0}+\tau)}
\nonumber\\
q(\tau)&=&-1+{1
\over
h ({\bar r_0}+t)}
+{({\bar r_0}+t) \csc^2 t+\cot t
\over
3 h^2 ({\bar r_0}+t)^2}
\nonumber\\
&&
-
{
\sec^2\left(
{\log ({\bar r_0}+t)
\over \sqrt{3}}
\right)
\over
3 h^2 ({\bar r_0}+t)^2}.
\ee
%Since we have $q_0$ and $H_0$ at present time we have two equations for the unknown $r_0, \tau_0, \lambda$. We consider $r_0$ as the unknown parameter of present location of the universe and discuss possible values in the following.

\begin{figure}
\includegraphics[width=8.5cm]{hq_c.eps}\\[1ex]
\includegraphics[width=8.5cm]{hq_c_spec.eps}
\caption{\label{hq_c} The dimensionless Hubble parameter (left) and deceleration parameter (right) as function of dimensionless time assuming a present position of $r_0=0$. Below a zoom of the only possible scenario with $\dot h>0$.}  
\end{figure}

In figure~\ref{hq_c} we see that the periodic oscillation along the light cone reveals only a certain window where the Hubble parameter can increase with time  $\dot H>0$ as observed. Choosing this interval we can determine the initial and final time of this universe window as the present cosmos by $H(t_0)=H(t_f)=\infty$, the present time by $H(t_p)=72$km/sMpc and the time where the Hubble parameter changes from falling to increasing time by $\dot H(t_c)=0$. 
For any parameter $r_0$ we can now determine these times together with the cosmological constant plotted in figure~\ref{age_t}.
  
\begin{figure}
\centerline{\includegraphics[width=8.5cm]{age_t.eps}}
\caption{\label{age_t} The age of present universe together with the age where the Hubble parameter changes from falling into increasing value vs. the dimensionless parameter of present location $r_0$. The data at presence are middle (black) lines and the corresponding cosmological constant are red lines.}  
\end{figure}

We see that the unknown parameter $\bar r_0$ as the starting place in figure~\ref{a_rt} determines all three values of initial time, ending time as well as the cosmological constant due to the known present data. The time in the past is than given where the Hubble constant has changed from decreasing to increasing behaviour. Larger $\bar r_0$ implies larger times accompanied by larger cosmological constant. In turn if we know the cosmological constant by other measurements we know $\bar r_0$ and the times are fixed. Please note that we have set the timescale to initially $t_0=0$ such that only the differences in times matter. As a note, the oscillating behaviour as big bounce instead of big bang has been reported in \cite{POP10,Pop12} due to torsion.

\section{Conclusion}

A time dependence of the Hubble and deceleration parameter is found from the exact solution of Einstein-Cartan equations. The latter provides a spatial and time dependent expansion or scale parameter. It is shown that the evolution of the universe is starting with an decreasing Hubble parameter switching to an increasing one within a certain evolution window among possible cosmoses. The seemingly dependence of the certain times and Hubble behaviour on the position parameter $r_0$ is not indicating a violation of equivalence principle. It appears here as a artificial unknown parameter for large scale structures determined by the cosmological constant. Since the Einstein-Cartan equations complete the equivalence principle \cite{Hey75} or more recently \cite{Pra22} and since we have used an exact solution of the latter we can conclude that locally there exist a transformation to a frame where the gravitational force vanishes though the large scale time and space structure of the expansion parameter looks nonholonomic.

As a second hint we should note that the torsion can mime dark matter. This can be seen as follows. We rewrite (\ref{metricd}) into the Schwarzschild form:
\be
ds^2=\left (1-\frac a r\right ) d t^2-{1\over 1-\frac b r} dr^2-r^2 d\Omega^2
\ee
with
\be
a=r-\frac {|\Lambda|}{ 3} r^3,\qquad b=\left (1-\frac C 3 \right ) r-\frac {|\Lambda|} 3 r^3.
\ee 
If we compare this result of the new metric with
the standard Schwarzschild solution with zero torsion and the extension to include the cosmological constant known as Kottler solution \cite{CXZ87}
\be
a^K=2 M-\frac {|\Lambda|}{ 3} r^3,\qquad b^K=2 M -\frac {|\Lambda|}{ 3} r^3
\ee 
we can conclude that the new metric resulting from torsion induces a mass like term
\be
M^{\rm torr}=\frac 1 2 \left (1-\frac C 3 \right ) r
\ee
which increases with larger distances. This can probably mime an additional gravitational mass \cite{Pop11} modifying the outer rotation of large galaxies \cite{mor21}. Recent investigations for torsion leading to dark energy can be found in \cite{ben22}. 

%\section*{References}

\begin{thebibliography}{17}
\expandafter\ifx\csname natexlab\endcsname\relax\def\natexlab#1{#1}\fi
\expandafter\ifx\csname bibnamefont\endcsname\relax
  \def\bibnamefont#1{#1}\fi
\expandafter\ifx\csname bibfnamefont\endcsname\relax
  \def\bibfnamefont#1{#1}\fi
\expandafter\ifx\csname citenamefont\endcsname\relax
  \def\citenamefont#1{#1}\fi
\expandafter\ifx\csname url\endcsname\relax
  \def\url#1{\texttt{#1}}\fi
\expandafter\ifx\csname urlprefix\endcsname\relax\def\urlprefix{URL }\fi
\providecommand{\bibinfo}[2]{#2}
\providecommand{\eprint}[2][]{\url{#2}}

\bibitem[{\citenamefont{Riess et~al.}(2022)\citenamefont{Riess, Yuan, Macri,
  Scolnic, Brout, Casertano, Jones, Murakami, Anand, Breuval et~al.}}]{Riess22}
\bibinfo{author}{\bibfnamefont{A.~G.} \bibnamefont{Riess}},
  \bibinfo{author}{\bibfnamefont{W.}~\bibnamefont{Yuan}},
  \bibinfo{author}{\bibfnamefont{L.~M.} \bibnamefont{Macri}},
  \bibinfo{author}{\bibfnamefont{D.}~\bibnamefont{Scolnic}},
  \bibinfo{author}{\bibfnamefont{D.}~\bibnamefont{Brout}},
  \bibinfo{author}{\bibfnamefont{S.}~\bibnamefont{Casertano}},
  \bibinfo{author}{\bibfnamefont{D.~O.} \bibnamefont{Jones}},
  \bibinfo{author}{\bibfnamefont{Y.}~\bibnamefont{Murakami}},
  \bibinfo{author}{\bibfnamefont{G.~S.} \bibnamefont{Anand}},
  \bibinfo{author}{\bibfnamefont{L.}~\bibnamefont{Breuval}},
  \bibnamefont{et~al.}, \bibinfo{journal}{The Astrophysical Journal Letters}
  \textbf{\bibinfo{volume}{934}}, \bibinfo{pages}{L7} (\bibinfo{year}{2022}),
  \urlprefix\url{https://doi.org/10.3847/2041-8213/ac5c5b}.

\bibitem[{\citenamefont{Li et~al.}(2021)\citenamefont{Li, Keeley, Shafieloo,
  Zheng, Cao, Biesiada, and Zhu}}]{Li21}
\bibinfo{author}{\bibfnamefont{X.}~\bibnamefont{Li}},
  \bibinfo{author}{\bibfnamefont{R.~E.} \bibnamefont{Keeley}},
  \bibinfo{author}{\bibfnamefont{A.}~\bibnamefont{Shafieloo}},
  \bibinfo{author}{\bibfnamefont{X.}~\bibnamefont{Zheng}},
  \bibinfo{author}{\bibfnamefont{S.}~\bibnamefont{Cao}},
  \bibinfo{author}{\bibfnamefont{M.}~\bibnamefont{Biesiada}}, \bibnamefont{and}
  \bibinfo{author}{\bibfnamefont{Z.-H.} \bibnamefont{Zhu}},
  \bibinfo{journal}{Monthly Notices of the Royal Astronomical Society}
  \textbf{\bibinfo{volume}{507}}, \bibinfo{pages}{919} (\bibinfo{year}{2021}),
  ISSN \bibinfo{issn}{0035-8711},
  \eprint{https://academic.oup.com/mnras/article-pdf/507/1/919/39805710/stab2154.pdf},
  \urlprefix\url{https://doi.org/10.1093/mnras/stab2154}.

\bibitem[{\citenamefont{Riess et~al.}(2019)\citenamefont{Riess, Casertano,
  Yuan, Macri, and Scolnic}}]{Riess19}
\bibinfo{author}{\bibfnamefont{A.~G.} \bibnamefont{Riess}},
  \bibinfo{author}{\bibfnamefont{S.}~\bibnamefont{Casertano}},
  \bibinfo{author}{\bibfnamefont{W.}~\bibnamefont{Yuan}},
  \bibinfo{author}{\bibfnamefont{L.~M.} \bibnamefont{Macri}}, \bibnamefont{and}
  \bibinfo{author}{\bibfnamefont{D.}~\bibnamefont{Scolnic}},
  \bibinfo{journal}{The Astrophysical Journal} \textbf{\bibinfo{volume}{876}},
  \bibinfo{pages}{85} (\bibinfo{year}{2019}),
  \urlprefix\url{https://doi.org/10.3847/1538-4357/ab1422}.

\bibitem[{\citenamefont{Camarena and Marra}(2020)}]{CM20}
\bibinfo{author}{\bibfnamefont{D.}~\bibnamefont{Camarena}} \bibnamefont{and}
  \bibinfo{author}{\bibfnamefont{V.}~\bibnamefont{Marra}},
  \bibinfo{journal}{Phys. Rev. Research} \textbf{\bibinfo{volume}{2}},
  \bibinfo{pages}{013028} (\bibinfo{year}{2020}),
  \urlprefix\url{https://link.aps.org/doi/10.1103/PhysRevResearch.2.013028}.

\bibitem[{\citenamefont{Morawetz}(2021)}]{Mo21}
\bibinfo{author}{\bibfnamefont{K.}~\bibnamefont{Morawetz}},
  \bibinfo{journal}{Classical and Quantum Gravity}
  \textbf{\bibinfo{volume}{38}}, \bibinfo{pages}{205003}
  (\bibinfo{year}{2021}),
  \urlprefix\url{https://doi.org/10.1088/1361-6382/ac2417}.

\bibitem[{\citenamefont{errata}(2022)}]{Mo21e}
\bibinfo{author}{\bibnamefont{errata}}, \bibinfo{journal}{Class. Quantum Grav.}
  \textbf{\bibinfo{volume}{40}}, \bibinfo{pages}{029501}
  (\bibinfo{year}{2022}).

\bibitem[{\citenamefont{Belinfante}(1940)}]{BEL40}
\bibinfo{author}{\bibfnamefont{F.}~\bibnamefont{Belinfante}},
  \bibinfo{journal}{Physica} \textbf{\bibinfo{volume}{7}}, \bibinfo{pages}{449
  } (\bibinfo{year}{1940}), ISSN \bibinfo{issn}{0031-8914},
  \urlprefix\url{http://www.sciencedirect.com/science/article/pii/S003189144090091X}.

\bibitem[{\citenamefont{Rosenfeld}(1940)}]{Ros40}
\bibinfo{author}{\bibfnamefont{L.}~\bibnamefont{Rosenfeld}},
  \emph{\bibinfo{title}{Sur le tenseur d'impulsion-{\'e}nergie}}
  (\bibinfo{publisher}{Palais des Acad{\'e}mies (impr. de G. Thone)},
  \bibinfo{address}{Bruxelles}, \bibinfo{year}{1940}).

\bibitem[{\citenamefont{Milgrom}(2017)}]{milgrom17}
\bibinfo{author}{\bibfnamefont{M.}~\bibnamefont{Milgrom}},
  \emph{\bibinfo{title}{High-redshift rotation curves and mond}}
  (\bibinfo{year}{2017}), \urlprefix\url{https://arxiv.org/abs/1703.06110}.

\bibitem[{\citenamefont{Popławski}(2010)}]{POP10}
\bibinfo{author}{\bibfnamefont{N.~J.} \bibnamefont{Popławski}},
  \bibinfo{journal}{Physics Letters B} \textbf{\bibinfo{volume}{694}},
  \bibinfo{pages}{181 } (\bibinfo{year}{2010}), ISSN \bibinfo{issn}{0370-2693},
  \urlprefix\url{http://www.sciencedirect.com/science/article/pii/S0370269310011561}.

\bibitem[{\citenamefont{Pop\l{}awski}(2012)}]{Pop12}
\bibinfo{author}{\bibfnamefont{N.}~\bibnamefont{Pop\l{}awski}},
  \bibinfo{journal}{Phys. Rev. D} \textbf{\bibinfo{volume}{85}},
  \bibinfo{pages}{107502} (\bibinfo{year}{2012}),
  \urlprefix\url{https://link.aps.org/doi/10.1103/PhysRevD.85.107502}.

\bibitem[{\citenamefont{Von Der~Heyde}(1975)}]{Hey75}
\bibinfo{author}{\bibfnamefont{P.}~\bibnamefont{Von Der~Heyde}},
  \bibinfo{journal}{Lett. Nuovo Cimento} \textbf{\bibinfo{volume}{14}},
  \bibinfo{pages}{250} (\bibinfo{year}{1975}).

\bibitem[{\citenamefont{Pradisi and Salvio}(2022)}]{Pra22}
\bibinfo{author}{\bibfnamefont{G.}~\bibnamefont{Pradisi}} \bibnamefont{and}
  \bibinfo{author}{\bibfnamefont{A.}~\bibnamefont{Salvio}},
  \bibinfo{journal}{Eur. Phys. J. C} \textbf{\bibinfo{volume}{82}},
  \bibinfo{pages}{840} (\bibinfo{year}{2022}).

\bibitem[{\citenamefont{Chongming et~al.}(1987)\citenamefont{Chongming, Xuejun,
  and Zhun}}]{CXZ87}
\bibinfo{author}{\bibfnamefont{X.}~\bibnamefont{Chongming}},
  \bibinfo{author}{\bibfnamefont{W.}~\bibnamefont{Xuejun}}, \bibnamefont{and}
  \bibinfo{author}{\bibfnamefont{H.}~\bibnamefont{Zhun}},
  \bibinfo{journal}{Gen. Relat. Gravit.} \textbf{\bibinfo{volume}{19}},
  \bibinfo{pages}{1203} (\bibinfo{year}{1987}).

\bibitem[{\citenamefont{Pop\l{}awski}(2011)}]{Pop11}
\bibinfo{author}{\bibfnamefont{N.~J.} \bibnamefont{Pop\l{}awski}},
  \bibinfo{journal}{Phys. Rev. D} \textbf{\bibinfo{volume}{83}},
  \bibinfo{pages}{084033} (\bibinfo{year}{2011}),
  \urlprefix\url{https://link.aps.org/doi/10.1103/PhysRevD.83.084033}.

\bibitem[{\citenamefont{González-Morán
  et~al.}(2021)\citenamefont{González-Morán, Chávez, Terlevich, Terlevich,
  Fernández-Arenas, Bresolin, Plionis, Melnick, Basilakos, and
  Telles}}]{mor21}
\bibinfo{author}{\bibfnamefont{A.~L.} \bibnamefont{González-Morán}},
  \bibinfo{author}{\bibfnamefont{R.}~\bibnamefont{Chávez}},
  \bibinfo{author}{\bibfnamefont{E.}~\bibnamefont{Terlevich}},
  \bibinfo{author}{\bibfnamefont{R.}~\bibnamefont{Terlevich}},
  \bibinfo{author}{\bibfnamefont{D.}~\bibnamefont{Fernández-Arenas}},
  \bibinfo{author}{\bibfnamefont{F.}~\bibnamefont{Bresolin}},
  \bibinfo{author}{\bibfnamefont{M.}~\bibnamefont{Plionis}},
  \bibinfo{author}{\bibfnamefont{J.}~\bibnamefont{Melnick}},
  \bibinfo{author}{\bibfnamefont{S.}~\bibnamefont{Basilakos}},
  \bibnamefont{and} \bibinfo{author}{\bibfnamefont{E.}~\bibnamefont{Telles}},
  \bibinfo{journal}{Monthly Notices of the Royal Astronomical Society}
  \textbf{\bibinfo{volume}{505}}, \bibinfo{pages}{1441} (\bibinfo{year}{2021}),
  ISSN \bibinfo{issn}{0035-8711},
  \eprint{https://academic.oup.com/mnras/article-pdf/505/1/1441/38444969/stab1385.pdf},
  \urlprefix\url{https://doi.org/10.1093/mnras/stab1385}.

\bibitem[{\citenamefont{Benisty et~al.}(2022)\citenamefont{Benisty, Guendelman,
  and Stoecker}}]{ben22}
\bibinfo{author}{\bibfnamefont{D.}~\bibnamefont{Benisty}},
  \bibinfo{author}{\bibfnamefont{E.}~\bibnamefont{Guendelman}},
  \bibnamefont{and} \bibinfo{author}{\bibfnamefont{H.}~\bibnamefont{Stoecker}},
  \bibinfo{journal}{The European Physical Journal}
  \textbf{\bibinfo{volume}{82}}, \bibinfo{pages}{264} (\bibinfo{year}{2022}),
  ISSN \bibinfo{issn}{1434-6044}.

\end{thebibliography}

%\bibliography{entropy,bose,kmsr,kmsr1,kmsr2,kmsr3,kmsr4,kmsr5,kmsr6,kmsr7,delay2,spin,spin1,refer,delay3,gdr,chaos,sem3,sem1,sem2,short,cauchy,genn,paradox,deform,shuttling,blase,spinhall,spincurrent,tdgl,pattern,zitter,graphene,quench,msc_nodouble,iso,march,weyl,anomal,darkmatter,rel}

%\bibliography{bose,kmsr,kmsr1,kmsr2,kmsr3,kmsr4,kmsr5,kmsr6,kmsr7,delay2,spin,spin1,refer,delay3,gdr,chaos,sem3,sem1,sem2,short,cauchy,genn,paradox,deform,shuttling,blase,spinhall,spincurrent,tdgl,pattern,zitter,isospin,quench,hubbard,march,polariton,anomal,weyl,entropy}
%\bibliographystyle{prsty}
\bibliographystyle{apsrev}

\end{document}


 
