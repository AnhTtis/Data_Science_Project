\section{Related Work}

\subsection{Intent Classification}
Traditionally, benchmark datasets \cite{price1990evaluation, coucke2018snips, eric2019multiwoz} for intent identification have sufficient labeled datasets for training, and the task has been solved through the classification method. For example, \citet{goo2018slot} classified the intent and slot information using the attention mechanism, and \citet{kim2017onenet} enriched word embeddings by using semantic lexicons and adapted this strategy to the intent classification. In addition, \citet{wang2015mining} grouped the intent of tweets into six categories, used a graph embedding consisting of tweet nodes, and classified their intents. However, labeling the intent for the raw dialogue dataset requires extensive human labor, so building a new labeled intent dataset in the real world is challenging. Therefore, a robust intent induction model that can be applied to a new domain as an unsupervised method is required.


\subsection{Intent induction with unsupervised method}
The representative method of unsupervised intent induction utilizes the clustering method. \citet{liu2021open} is one example of research that enhanced the clustering algorithm. They proposed a balanced score metric to obtain similar-sized clusters in K-means clustering and found proper K-values that were more stable than naive K-means. \citet{chatterjee2020intent} also enhanced the clustering algorithm by utilizing the outlier information of the density-based clustering model, which is called ITER-DBSCAN. Their work shows greater accuracy on imbalanced intent data. 
On the other hand, there has been research that improved the dialogue representations for better clustering results. For example, \citet{perkins2019dialog} iteratively enhanced the dialogue embedding by reflecting the clustering score, and \citet{lin2019deep} proposed a BiLSTM \cite{mesnil2014using} embedding model with margin loss that is effective in detecting unknown intents. However, robust intent induction in diverse domains was not examined in previous research. Therefore, as a strategy for enhancing the embedding dialogue model, we propose the DORIC method, which robustly embeds diverse dialogue domains.