% This must be in the first 5 lines to tell arXiv to use pdfLaTeX, which is strongly recommended.
\pdfoutput=1
% In particular, the hyperref package requires pdfLaTeX in order to break URLs across lines.

\documentclass[11pt]{article}

% Remove the "review" option to generate the final version.
\usepackage[]{ACL2023}
\usepackage{graphicx}

% Standard package includes
\usepackage{times}
\usepackage{anyfontsize}
\usepackage{latexsym}
\usepackage{amsmath}
\usepackage{amssymb}
\usepackage{multirow}
\usepackage{xcolor}
\usepackage{makecell}
\usepackage{latexsym}
\newcommand{\code}[1]{{\texttt{#1}}}
% For proper rendering and hyphenation of words containing Latin characters (including in bib files)
\usepackage[T1]{fontenc}
% For Vietnamese characters
% \usepackage[T5]{fontenc}
% See https://www.latex-project.org/help/documentation/encguide.pdf for other character sets

% This assumes your files are encoded as UTF8
\usepackage[utf8]{inputenc}

% This is not strictly necessary, and may be commented out.
% However, it will improve the layout of the manuscript,
% and will typically save some space.
\usepackage{microtype}

% This is also not strictly necessary, and may be commented out.
% However, it will improve the aesthetics of text in
% the typewriter font.
\usepackage{inconsolata}


% If the title and author information does not fit in the area allocated, uncomment the following
%
%\setlength\titlebox{<dim>}
%
% and set <dim> to something 5cm or larger.

\title{DORIC : Domain Robust Fine-Tuning for Open Intent Clustering through Dependency Parsing}

% Author information can be set in various styles:
% For several authors from the same institution:
% \author{Author 1 \and ... \and Author n \\
%         Address line \\ ... \\ Address line}
% if the names do not fit well on one line use
%         Author 1 \\ {\bf Author 2} \\ ... \\ {\bf Author n} \\
% For authors from different institutions:
% \author{Author 1 \\ Address line \\  ... \\ Address line
%         \And  ... \And
%         Author n \\ Address line \\ ... \\ Address line}
% To start a seperate ``row'' of authors use \AND, as in
% \author{Author 1 \\ Address line \\  ... \\ Address line
%         \AND
%         Author 2 \\ Address line \\ ... \\ Address line \And
%         Author 3 \\ Address line \\ ... \\ Address line}

% \author{First Author \\
%   Affiliation / Address line 1 \\
%   Affiliation / Address line 2 \\
%   Affiliation / Address line 3 \\
%   \texttt{email@domain} \\\And
%   Seungyeon Seo \\
%   Affiliation / Address line 1 \\
%   Affiliation / Address line 2 \\
%   Affiliation / Address line 3 \\
%   \texttt{email@domain} \\\And
%   Seungyeon Seo \\
%   Affiliation / Address line 1 \\
%   Affiliation / Address line 2 \\
%   Affiliation / Address line 3 \\
%   \texttt{email@domain} \\\And
%   Seungyeon Seo \\
%   Affiliation / Address line 1 \\
%   Affiliation / Address line 2 \\
%   Affiliation / Address line 3 \\
%   \texttt{email@domain} \\}


\author{Jihyun Lee$^1$, Seungyeon Seo$^1$, Yunsu Kim$^{1,2}$, Gary Geunbae Lee$^{1,2}$ \\
  $^1$Graduate School of Artificial Intelligence, POSTECH, Republic of Korea\\
  $^2$Department of Computer Science and Engineering, POSTECH, Republic of Korea\\
  \texttt{\{jihyunlee, ssy319, yunsu.kim, gblee\}@postech.ac.kr} \\
}


\begin{document}
\maketitle
\begin{abstract}

We present our work on Track 2 in the Dialog System Technology Challenges 11 (DSTC11). DSTC11-Track2 aims to provide a benchmark for zero-shot, cross-domain, intent-set induction. In the absence of in-domain training dataset, robust utterance representation that can be used across domains is necessary to induce users' intentions. To achieve this, we leveraged a multi-domain dialogue dataset to fine-tune the language model and proposed extracting Verb-Object pairs to remove the artifacts of unnecessary information. Furthermore,  we devised the method that generates each cluster's name for the explainability of clustered results. Our approach achieved 3rd place in the precision score and showed superior accuracy and normalized mutual information (NMI) score than the baseline model on various domain datasets. 
\end{abstract}

\section{Introduction}
Deep learning~\cite{dl} has been highly successful in computer vision~\cite{sg1,od1,app-detection,zhou2024diffdet4sar,li2024predicting,yang2024saratr,LiSARATRX25}, largely due to the availability of large-scale labeled datasets. However, in many practical scenarios, obtaining such large amounts of labeled data is difficult or costly. To address this challenge, Few-shot learning (FSL) aims to enable models to learn new tasks with only a limited number of labeled samples. Consequently, this problem has garnered significant attention in both academia and industry due to its broad real-world applications. While humans can easily distinguish between objects after seeing only a few examples, machines struggle to achieve similar efficiency. In domains such as natural scene images, large datasets are readily available, but FSL is crucial in scenarios where collecting large amounts of data is difficult. Since the problem was first introduced in 2006~\cite{fsl-1}, numerous methods have been proposed to tackle the challenges of FSL~\cite{fslsurvey,fslsurvey22,fslsurvey20,fsl18,fslsurvey1}.

With the development of FSL, challenges such as limited training data, domain variations, and task modifications have led to the emergence of various FSL variants, including semi-supervised FSL~\cite{semifsl}, unsupervised FSL~\cite{ufsl1,ufsl2}, zero-shot learning (ZSL)\cite{zsl1}, and cross-domain FSL (CDFSL)~\cite{feature-wise,bscd-fsl}, among others. These variants represent distinctive cases of FSL in terms of sample availability and domain learning. This paper focuses specifically on CDFSL variants. The traditional FSL problem assumes that both prior knowledge and target tasks come from the same domain, which is often restrictive in real-world applications. CDFSL addresses this issue by overcoming the domain gap between auxiliary data (which provides prior knowledge) and the target data in FSL tasks, as show in Figure~\ref{int}. For instance, in art image recognition tasks involving scribble, cartoon, or sketch images, FSL could theoretically leverage prior knowledge from related domains like cartoons and sketches. However, such data is often scarce due to copyright restrictions and the high cost of collection. As a result, researchers have turned to data-rich domains, such as natural scene images, to address the challenges of few-shot image recognition in the field of art.
However, the significant domain gap between these domains often leads to performance degradation in FSL. CDFSL faces challenges from both transfer learning and FSL, including domain gaps, class shifts, and the scarcity of labeled samples in the target domain, making it a more complex task. Since its formal introduction in 2020~\cite{feature-wise}, CDFSL has garnered widespread attention, with numerous methods published in top venues~\cite{bscd-fsl,st,dynamic,hybrid_1,feature_reweight_6}. Figure~\ref{imaging} presents the milestones of CDFSL technologies from 2019 to the present, showcasing representative CDFSL methods and related benchmarks.
\begin{figure}%[b]
	\centering
  \vspace{-0.3cm}
 	\includegraphics[width=0.9\linewidth]{CDFSLProblem-10.pdf}
  \vspace{-0.3cm}
	\caption{\textcolor{black}{The difference of few-shot learning and cross-domain few-shot learning.}}
 \vspace{-0.4cm}
	\label{int}
\end{figure}


So far, several surveys have provided comprehensive overviews and future directions for FSL~\cite{fsl18,fslsurvey,fslsurvey1,fslsurvey20,fslsurvey22}. For example,\cite{fsl18} categorizes FSL into experiential and conceptual learning, while\cite{fslsurvey} focuses on empirical risk minimization and defines FSL by experience, task, and performance, introducing CDFSL as a branch of FSL. Both~\cite{fslsurvey1} and~\cite{fslsurvey20} highlight CDFSL as a variant of FSL, discussing meta-learning approaches and benchmarks. Lastly,~\cite{fslsurvey22} offers a taxonomy based on prior knowledge and emphasizes that current methods have yet to fully tackle cross-domain challenges. Collectively, these works point to cross-domain learning as a promising area for future FSL research. Currently, there are two elementary surveys on CDFSL~\cite{wang2023survey,deng2023survey}. \cite{wang2023survey} classifies methods into benchmark, single source, and multiple source categories, while~\cite{deng2023survey} categorizes algorithms into data augmentation and feature alignment paradigms. In contrast, to stimulate future research and help newcomers better understand this challenging problem, this paper offers the first classification grounded in theoretical analysis and provides a comprehensive review, offering deeper insights into the core principles of CDFSL. Firstly, this paper compiles and analyzes a broad range of literature on the topic. An analysis of the reference index reveals that even before the formal introduction of CDFSL, some works had already tried to solve cross-domain issues within the FSL framework~\cite{clc, rffl}. Following its formal introduction as a branch of FSL, CDFSL has garnered significant attention. Additionally, we define CDFSL using both machine learning theory~\cite{ml,erm1} and transfer learning principles~\cite{tltheory}. Secondly, our analysis highlights that the unique challenge in CDFSL lies in the unreliable nature of two-stage empirical risk minimization. The details are discussed in Section~\ref{background}. To address these challenges, the paper organizes CDFSL research into four categories: $\mathcal{D}$-Extension, $\mathcal{H}$-Constraint, $\Delta$-Adaptation, and hybrid approaches. We also compile relevant datasets and benchmarks to evaluate the methods, and analyze their performance, as discussed in Sections~\ref{methods} and~\ref{performance}. Finally, we explore future research directions for CDFSL by considering three perspectives, including problem set-ups, applications, and theories, which provide a comprehensive understanding of the field and its potential for future growth. Contributions of this survey can be summarized as follows:

\begin{itemize}
    \item We analyzed existing CDFSL papers and provided a comprehensive survey. We also defined CDFSL formally, connecting it to classic ML~\cite{ml,erm1} and transfer learning theory~\cite{tltheory}. This helps guide future research in the field.
    \item We listed relevant learning problems for CDFSL with examples, clarifying their relation and differences. This helps position CDFSL among various learning problems. We also analyzed unique issues and challenges of CDFSL, helping to explore a scientific taxonomy for CDFSL work.
    \item We conducted an extensive literature review, organizing it into a unified taxonomy based on $\mathcal{D}$-Extension, $\mathcal{H}$-Constraint, $\Delta$-Adaptation, and hybrid approaches. We introduced applicable scenarios for each taxonomy to help discuss its pros and cons. We also presented datasets and benchmarks for CDFSL, summarizing insights from performance results to improve understanding of CDFSL methods.
    \item We proposed promising future directions for CDFSL in problem set-ups, applications, and theories, based on current weaknesses and potential improvements.
\end{itemize}

\begin{figure}
	\centering
  \vspace{-0.3cm}
        \includegraphics[width=\linewidth]{response/crop_fig2.pdf}
 \vspace{-0.5cm}
	\caption{Chronological milestones of CDFSL from 2019 to the present, including representative CDFSL approaches and related benchmarks. Key events include the release of Meta-Dataset~\cite{meta-dataset} and BSCD-FSL~\cite{bscd-fsl} in 2020, the introduction of pioneering works such as~\cite{feature-wise}, and subsequent contributions like~\cite{feature_reweight_1,lscdfsl}. Later works~\cite{st,dynamic,hybrid_1,hybrid_4,hybrid_2} explored new setups, while~\cite{boosting,ata,data_target_1,feature_reweight_5,parameter_weight_2,confess,feature_reweight_9} focused on improving performance. Please see Section~\ref{methods} for details.}
 \vspace{-0.3cm}
	\label{imaging}
\end{figure}

The remainder of this survey is organized as follows: Section \ref{background} provides an overview of CDFSL, including its definition, challenges, and taxonomy. Section \ref{methods} covers approaches to CDFSL in detail, while Section \ref{performance} presents performance results and evaluates methods. Section \ref{future} explores future directions in set-ups, applications, and theories. Finally, Section \ref{conclusion} concludes the survey.
\section{Related Work}

\subsection{Intent Classification}
Traditionally, benchmark datasets \cite{price1990evaluation, coucke2018snips, eric2019multiwoz} for intent identification have sufficient labeled datasets for training, and the task has been solved through the classification method. For example, \citet{goo2018slot} classified the intent and slot information using the attention mechanism, and \citet{kim2017onenet} enriched word embeddings by using semantic lexicons and adapted this strategy to the intent classification. In addition, \citet{wang2015mining} grouped the intent of tweets into six categories, used a graph embedding consisting of tweet nodes, and classified their intents. However, labeling the intent for the raw dialogue dataset requires extensive human labor, so building a new labeled intent dataset in the real world is challenging. Therefore, a robust intent induction model that can be applied to a new domain as an unsupervised method is required.


\subsection{Intent induction with unsupervised method}
The representative method of unsupervised intent induction utilizes the clustering method. \citet{liu2021open} is one example of research that enhanced the clustering algorithm. They proposed a balanced score metric to obtain similar-sized clusters in K-means clustering and found proper K-values that were more stable than naive K-means. \citet{chatterjee2020intent} also enhanced the clustering algorithm by utilizing the outlier information of the density-based clustering model, which is called ITER-DBSCAN. Their work shows greater accuracy on imbalanced intent data. 
On the other hand, there has been research that improved the dialogue representations for better clustering results. For example, \citet{perkins2019dialog} iteratively enhanced the dialogue embedding by reflecting the clustering score, and \citet{lin2019deep} proposed a BiLSTM \cite{mesnil2014using} embedding model with margin loss that is effective in detecting unknown intents. However, robust intent induction in diverse domains was not examined in previous research. Therefore, as a strategy for enhancing the embedding dialogue model, we propose the DORIC method, which robustly embeds diverse dialogue domains.
\section{DSTC11 Intent Clustering Task}
In this task, participants are required to assign an intent label to each dialogue turn. A set of dialogues are provided as input, and each turn is pre-labeled with both its speaker role (i.e., Agent or Customer) and dialogue acts (i.e., InformIntent or not). One development dataset and two test datasets are provided, and each dataset consists of approximately 1K customer support spoken conversations with manual transcriptions and annotations. The development dataset derives from an insurance-related customer support service, and each conversation has an average of 70 turns. In addition, the development dataset contains ground truth intent annotations that allow participants to test and evaluate the model. The number of intent types and the domains of the test dataset are not revealed until the development phase ends. Note that no training dataset is given, as this challenge aims to zero-shot intent induction.
\section{Proposed Modification}
\label{sec:method}

In the previous section, we provided a theoretical analysis of the conditions under which computed ``probe'' functions within the deep functional map pipeline can be used as pointwise descriptors directly, and lead to the same point-to-point maps as computed by the functional maps. In \cref{sec:application}, we provide an extensive evaluation of using learned feature functions for pointwise map computation and thus affirm the validity of \cref{thm:equivalence} in practice.

% \maks{is this paragraph necessary?} 
Our main observation is that the two approaches for point-to-point map computation are indeed often equivalent in practice, especially in ``easy'' cases, where existing state-of-the-art approaches lead to highly accurate maps. In contrast, we found that in more challenging cases, where existing methods fail, the two approaches are not equivalent and can lead to significantly different results. 

% Inspired by our analysis above, we thus propose to use the structural properties suggested in \cref{thm:equivalence} as a way to bridge this gap and, possibly improve the overall accuracy. As we demonstrate in \cref{sec:application}, our proposed modifications while being relatively simple, significantly improve the quality of the computed correspondences, especially in ``difficult'' matching scenarios.

Motivated by our analysis, we propose to use the structural properties suggested in \cref{thm:equivalence} as a way to bridge this gap and improve the overall accuracy. Our proposed modifications are relatively simple, but they significantly improve the quality of computed correspondences, especially in ``difficult'' matching scenarios, as we demonstrate in \cref{sec:application}.

% The two key assumptions in \cref{thm:equivalence} are \textit{basis-aligning} functional maps and \textit{complete} feature extractors. We thus propose to modify the functional map pipeline so that the conditions of the theorem are satisfied. As mentioned above, the basis-aligning property is closely related to \textit{properness} and thus we propose to approach it by imposing that the predicted functional map arises from some pointwise correspondence. For feature completeness, we propose a simple modification of the feature extractor so that it produces \textit{smooth features}. In what follows, we will use the same notation as \cref{sec:notation}.

The two key assumptions in \cref{thm:equivalence} are \textit{basis-aligning} functional maps and \textit{complete} feature extractors. We propose modifying the functional map pipeline to satisfy the conditions of the theorem. Since the basis-aligning property is closely related to \textit{properness}, we propose to impose that the predicted functional map to be proper, \ie arises from some pointwise correspondence. For feature completeness, we suggest modifying the feature extractor to produce \textit{smooth features}. We use the same notation as in \cref{sec:notation}.


\subsection{Enforcing Properness}
\label{sec:proper_fmap}

In this section, we propose two ways to enforce functional map properness and associated losses  for both supervised and unsupervised training.

\paragraph{The adjoint method}
Given feature functions $F_1, F_2$, produced by a feature extractor, we compute the functional map $\C_{12-pred}$ as explained in \cref{sec:background}. To compute a proper functional from it, we first convert $\C_{12-pred}$ into a p2p map $\Pi_{21-pred}$ in a differentiable way and then compute the ``differentiable'' proper functional map $\C_{21-proper} = \Phi_{2}^{\dagger} \Pi_{21-pred} \Phi_{1}$. 

To compute $\Pi_{21-pred}$, denoting $G_1 = \Phi_{1}$ and $G_2 = \Phi_{2}\C_{21-pred}$, we use:
\begin{align}
& \Pi_{21-pred}^{i, j} = \dfrac{\exp\big(\langle G_2^{i}, G_1^{j}\rangle / \tau\big)}{\sum_{k=1}^{n_1}\exp\big(\langle G_2^{i}, G_1^{k} \rangle / \tau\big)}.\label{eq:diff_p2p}%\\
%&s(\mathbf{x}, \mathbf{y}) = \mathbf{x} \cdot \mathbf{y}.\label{eq:FeatureDistanceFunc}
\end{align}
%
Here $\langle \cdot,\cdot \rangle$ is the scalar product measuring the similarity between $G_1$ and $G_2$, and $\tau$ is a temperature hyper-parameter. $\Pi_{21-pred}$ can be seen as a soft point-to-point map, formulated based on the adjoint conversion method described in \cite{Pai_2021_CVPR}, and computed in a differentiable manner, hence it can be used inside a neural network.

\paragraph{The feature-based method}
The feature-based method is similar to the adjoint method in spirit, the only difference being that $\Pi_{21-pred}$ is computed using the predicted features instead of the fmap. For this, we use \cref{eq:diff_p2p}, with $G_1 = F_1$ and $G_2 = F_2$. The modified deep functional map pipeline is illustrated in \cref{fig:fmap-pipeline}.

\begin{figure}
    \centering
    \includegraphics[width=\columnwidth]{figures/fmap_pipeline.pdf}
    \caption{An overview of our revised deep functional map pipeline. The extracted features are used to compute the functional map and the proper functional map, as explained in \cref{sec:proper_fmap}}
    \label{fig:fmap-pipeline}
    \vspace{-1em}
\end{figure}

In addition to $C_{12-pred}$, the two previous methods allow to calculate $C_{21-proper}$. We adapt the functional map losses to take into account this modification.

In the supervised case, we modify the supervised loss (see \cref{eq:sup_loss}) by simply introducing an additional term into the loss: 
\begin{align}
\mathcal{L}_{proper} = \| \C_{12-pred} - \C_{12-proper} \|_F^2 \label{eq:sup_loss_proper}.
\end{align}

The motivation behind this loss is that we want the predicted functional map to be as close as possible to the ground truth and stay within the space of proper functional maps.% a proper one , the gradient is strong enough to force the features to produce a  using \cref{eq:fmap_basic}, that is close to a proper one.

In the unsupervised setting, we simply impose the standard unsupervised losses on the differentiable proper functional map $\C_{12-proper}$ instead of $\C_{12-pred}$. Specifically, in our experiments below, we use the following unsupervised losses: 
%
\begin{align}
\nonumber
\mathcal{L}_{unsup}(\C_{12}, \C_{21}) &= \| \C_{12} \C_{21} - \mathbb{I} \|_F^2 + \| \C_{21} \C_{12} - \mathbb{I} \|_F^2 \\
& + \| \C_{12}^{\top} \C_{12} - \mathbb{I} \|_F^2 + \| \C_{21}^{\top} \C_{21} - \mathbb{I} \|_F^2
\label{eq:unsup_loss_proper}
\end{align}

%\maks{add the losses here.}


% \souhaib{how to justify that the results obtained with NN are better than fmap}












%%%%%%%%%%%%%%%%%%%%%%%%%%%%%%%%%%%%%%%%%%%%%%%%%%%%
%%%%%%%%%%%%%%%%%%%%%%%%%%%%%%%%%%%%%%%%%%%%%%%%%%%%%%%%%%%%%%%%%%%%%%%%%%%%%%%%%%%%%%%%%%%%%%%%%%%%%%%%
%%%%%%%%%%%%%%%%%%%%%%%%%%%%%%%%%%%%%%%%%%%%%%%%%%%%%%%%%%%%%%%%%%%%%%%%%%%%%%%%%%%%%%%%%%%%%%%%%%%%%%%%
%%%%%%%%%%%%%%%%%%%%%%%%%%%%%%%%%%%%%%%%%%%%%%%%%%%%
%%%%%%%%%%%%%%%%%%%%%%%%%%%%%%%%%%%%%%%%%%%%%%%%%%%%

\subsection{As Smooth As Possible Feature Extractor}
Another fundamental assumption of \cref{thm:equivalence} is the completeness of the features produced by a neural network. 

We have experimented with several ways to impose it and have found that it is not easy to satisfy it exactly in general because it would require the network to always produce features in some target  subspace, which is not explicitly specified in advance. Moreover, we have found that explicitly projecting feature functions to a small reduced subspace can also hinder learning. 

To circumvent this, we propose instead to \textit{encourage} this property by promoting the feature extractor to produce smooth features. 

The motivation for this is as follows. If $F_i$ is complete, then there exist coefficients $a_1 ... a_k$ such that $F_i = \sum_{j=1}^k a_j \Phi_i^j$, where $k$ is the size of the functional map used in \cref{eq:fmap_basic}.
However, it's known that Fourier coefficients for smooth functions decay rapidly (faster than any polynomial, if $f$ is of class $C^l$, the coefficients are $o(n^{-l})$), which means that the smoother the function is, the closer it will be to being complete for some index $k$.

Inspired by this, we propose the following simple modification to feature extractors used for deep functional maps. Since feature extractors are made of multiple layers, we propose to project the output of each layer into the Laplacian basis, diffuse it over the surface following \cite{sharp2021diffusion}, and then project it back to the ambient space before feeding it to the next layer, see \cref{fig:feat-extract-modif}. Concretely, for shape $S$, if $f_i$ is the output of layer $i$, we feed to layer $i+1$ the function $f^{'}_i$, such that $f^{'}_i = \Phi_j e^{-t \Delta} \Phi_j^{\dagger} f_i$, where $\Phi_j$ denotes the first $j$ eigenfunctions of the Laplacian, $\Delta$ is a diagonal matrix containing the first j eigenvalues, and $t$ is a learnable parameter. Please note there is no need to do this operation for the final layer, since the features will be projected into the Laplacian basis anyway, for computing the functional map. In practice, we observed that it is beneficial to set $j$ to \textit{be larger} than the size of the functional maps in \cref{eq:fmap_basic}. This allows the network to impose smoothness, while still allowing degrees of freedom to enable optimization.

%Also note, that DiffusionNet \cite{sharp2021diffusion} does this operation by construction for each layer, which can in part explain its success.
%
%\souhaib{what about the receiptive field}


%
%
% - smooth the input before feeding them to the network (doesn't work practically)
% 
% - smooth the features at the end of each layer
% 
% - for better results, increase the receiptive field of the features using diffusion 
%
%
\vspace{-1em}

\paragraph{Implementation details} we provide implementation details, for all our experiments, in the supplementary. Our code and data will be released after publication.

\begin{figure}
    \centering
    \includegraphics[width=\columnwidth]{figures/feature_extractor.pdf}
    \caption{An overview of the feature extractor modification is shown here. The features are made smooth by projecting them into the Laplacian basis at the end of each layer.}
    \label{fig:feat-extract-modif}
    % \vspace{-1.5em}
\end{figure}
\section{Experiment}

\begin{figure*}
\centering
\begin{tabular}{cccc}
\hline
Dev (Insurance) & Test1 (Banking) &  Test2 (Finance) & SGD (Tourism)\\
\hline
\multicolumn{4}{c}{\textit{Baseline}} \\ \hline
\includegraphics[width=0.2\textwidth]{img/baseline/insurance.png}& 
\includegraphics[width=0.2\textwidth]{img/baseline/banking.png}&
\includegraphics[width=0.2\textwidth]{img/baseline/finance.png}&
\includegraphics[width=0.2\textwidth]{img/baseline/SGD2.png}\\
\hline
\multicolumn{4}{c}{\textit{DORIC}} \\ \hline
\includegraphics[width=0.2\textwidth]{img/Ours/insurance.png}&
\includegraphics[width=0.2\textwidth]{img/Ours/banking.png}&
\includegraphics[width=0.2\textwidth]{img/Ours/finance.png}&
\includegraphics[width=0.2\textwidth]{img/Ours/SGD2.png}\\
\hline

\end{tabular}
\caption{Visulization on dev (insurance), test1 (banking), test2 (finance) and SGD (tourism) dataset.}
\label{fig:base_ours}
\end{figure*}


\subsection{Experimental setup}
\textbf{Dataset}
\noindent
To demonstrate the performance of our model, we used the development (dev) (insurance), test1 (banking), and test2 (finance) datasets. These datasets have domains that are different from the fine-tuning dataset (tourism), so we were able to examine our method's effectiveness in diverse domains. Additionally, we used the Schema-Guided Dialogue Dataset (SGD) dataset; we extracted tourism-related domains from the SGD dataset to make the same domain environment with fine-tuning dataset. The number of intents for each dataset is shown in Table~\ref{tab:dataset}.

\begin{table}[h]
\centering
\small
\begin{tabular}{lll}
\hline
\textbf{Dataset} & \textbf{Domain} & \textbf{\# of intents}\\ \hline
Dev  & Insurance & 22\\
Test1 & Banking & 29\\
Test2 & Fianace & 39\\
SGD & Tourism & 6\\

\hline


\end{tabular}
\caption{Domain and number of intents type for each dataset.}
\label{tab:dataset}
\end{table}

\noindent
\textbf{Evaluation} 
NMI and accuracy were the primary metrics used for the evaluation, and to provide additional metrics, precision was also used. The higher NMI value denotes that clustering has reduced more entropy. 1:1 alignments between the induced intents and the gold intents were computed by the Hungarian algorithm \cite{kuhn1955hungarian}.\\

\noindent\textbf{Setup} 
We employ the pre-trained SBERT \cite{reimers2019sentence} for the baseline embedding model. The pre-trained parameters were from the huggingface \cite{wolf2019huggingface} \textit{all-mpnet-base-v2} version. In the SCL function, we set the $\tau$ as 0.07 and trained a maximum of five epochs with early stopping. In the K-means clustering, we set the minimum cluster number as five and the max cluster number as 50 and use silhouette score for comparing the clustered result, which is based on tightness and separation \cite{rousseeuw1987silhouettes}.

\subsection{Intent clustering result}

\begin{table}[h]
\centering
\small

\begin{tabular}{ll|lll}

\hline
\textbf{Model} & \textbf{Data} & \textbf{NMI}& \textbf{ACC} & {\fontsize{8}{7}\textbf{Precision}} \\ \hline

\multicolumn{5}{c}{\textit{Different Domain with Fine-tuning Dataset}} \\ \hline
Baseline & Dev(Insurance)                & 59.31 & 46.14 &65.98 \\
DORIC   & Dev(Insurance)                 & \textbf{65.16} & \textbf{56.68}  &\textbf{67.63}\\
\hline
Baseline & Test1(Banking) & 65.71 & 51.85   &60.68\\
DORIC   & Test1(Banking)   & \textbf{71.02} & \textbf{52.35}  & \textbf{73.92}\\
  \hline
Baseline & Test2(Finance)         & 60.26 & 59.75  &69.25 \\
DORIC   &  Test2(Finance)           & \textbf{69.64} & \textbf{65.14} &\textbf{75.14} \\
 \hline
\multicolumn{5}{c}{\textit{Same Domain with Fine-tuning Dataset}} \\ \hline
Baseline & SGD(Tourism)  & 60.54 & 63.67  &49.90\\
DORIC   & SGD(Tourism) & \textbf{65.32} & \textbf{68.36}  &\textbf{51.00}\\

\hline

\end{tabular}


\caption{Comparison of baseline and DORIC in different dataset. NMI, ACC and Precision are reported.}
\label{tab:main_result}
\end{table}
The results of DORIC after evaluation on the dev (insurance), test1 (finance), test2 (banking), and SGD (tourism) datasets are shown in Table~\ref{tab:main_result}. These results show that our model outperforms the baseline model in terms of the NMI, ACC, and precision on all datasets. Except for the SGD dataset, the dataset’s domains are all different from the fine-tuning dataset MultiWOZ2.2 (tourism), which demonstrates that our intent induction framework is robust to diverse domain datasets. The visualization of experimental results in Figure~\ref{fig:base_ours} also exhibits the aligned result with Table~\ref{tab:main_result}; compared to the baseline, DORIC embeds utterances with the same label at a closer distance.

\subsection{Intent label generation with hypernym}
\begin{table}[]
\centering
\small
{
\begin{tabular}{lll}
\hline

\textbf{Idx} & \textbf{Generated name}& \textbf{Ground-truth}    \\ \hline
0               & create-account  & CreateAccount             \\
1 & cancel-billing & CancelAutomaticBilling    \\
2\dag & \textbf{add-child} & AddDependent \\
3 & report-accident & ReportAutomobileAccident\\
4 & change-address & ChangeAddress\\
5\dag & get-quote & GetQuote\\
6 & change-plan & ChangePlan\\
7 & file-claim & FileClaim \\
8  & pay-bill & PayBill\\
9 & check-balance & CheckAccountBalance\\
10 & change-question & ChangeSecurityQuestion\\
11 & cancel-plan & CancelPlan\\
\hline
\multicolumn{3}{c}{\textit{Without hypernym}} \\ 
\hline
2 & \textbf{add-son} & AddDependent \\
5 & get-quote & GetQuote\\
\hline
\end{tabular}%
}
\caption{Example of generated intent labels and ground truth. Cluster name with \dag means using hypernym.}
\label{tab:label_generation}
\end{table} 

Table~\ref{tab:label_generation} shows examples of the generated intent labels, and cluster with \dag denotes the clusters with the hypernyms following section~\ref{label_generation}. As shown in the table, our proposed method successfully explains the cluster results compared to the ground truth label. Furthermore, using hypernyms enables the grouping of detailed information in the cluster. For instance, Cluster 2 obtains a more comprehensive label, \code{add-child} than \code{add-son} by using the hypernyms. We also present the more detailed results for the dev and test data in Appendix A.1.



\section{Analysis}
\subsection{Verb-Object structure in fine-tuning}
\begin{table}[h!]
\centering
\small
\begin{tabular}{ll|ll}
\hline
\textbf{Method} & \textbf{Dataset (domain)} & \textbf{NMI}& \textbf{ACC} \\ \hline

\multicolumn{4}{c}{\textit{Different Domain with Fine-tuning Dataset}} \\ \hline


Sentence & Dev (Insurance)  & 62.13 & 55.35 \\
Verb-Obj    & Dev (Insurance)          & \textbf{65.16} & \textbf{56.68} \\
\hline
Sentence & Test1 (Banking)     & 68.91 & \textbf{53.22} \\
Verb-Obj     & Test1 (Banking)      & \textbf{71.02} & 52.35 \\
 \hline
 
 
Sentence & Test2 (Finance)              & 64.94 & \textbf{67.07} \\
Verb-Obj     & Test2 (Finance)  & \textbf{69.64} & 65.14 \\
 \hline

\multicolumn{4}{c}{\textit{Same Domain with Fine-tuning Dataset}} \\ \hline
Sentence & SGD (Tourism)  & 65.24 & 68.35 \\
Verb-Obj     & SGD (Tourism) & \textbf{65.32} & \textbf{68.36} \\
\hline
\end{tabular}
\caption{The NMI and accuracy result on DSTC11 development, test, and SGD dataset according to fine-tuning utterance format.}
\label{tab:extract}
\end{table}

To examine the effect of extracting Verb-Object structures from the sentence, we compare our proposed method with methods that use the whole sentence during the fine-tuning stage (Table~\ref{tab:extract}). Using the Verb-Object structure demonstrates superior NMI results in both different-domain and same-domain environments; this result indicates that fine-tuning with Verb-Object information has helped reduce the clustering uncertainty. However, the accuracy doesn't significantly differ between the Verb-Object form and the whole sentence form in the tourism domain, which is identical to the fine-tuning dataset domain.


\subsection{Analysis of loss}

\begin{table}[h]
\centering
\small
\begin{tabular}{ll|ll}
\hline
\textbf{Loss function} & \textbf{Dataset (domain)} & \textbf{NMI}& \textbf{ACC} \\ \hline

\multicolumn{4}{c}{\textit{Different Domain with Fine-tuning Dataset}} \\ \hline
Cross Entropy & Dev (Insurance)  & 61.98 & 53.69 \\
SCL    & Dev (Insurance)          & \textbf{65.16} & \textbf{56.68} \\
\hline
Cross Entropy & Test1 (Banking)              & 67.67 & 52.16 \\
SCL   & Test1 (Banking)    & \textbf{71.02} & \textbf{52.35} \\
\hline
Cross Entropy & Test2 (Finance)     & 64.09 & 63.07 \\
SCL   & Test2 (Finance)    & \textbf{69.64} & \textbf{65.14} \\
\hline
 
 
\multicolumn{4}{c}{\textit{Same Domain with Fine-tuning Dataset}} \\ \hline
Cross Entropy & SGD (Tourism) & 64.11 & \textbf{70.31} \\
SCL   & SGD (Tourism) & \textbf{65.32} & 68.36 \\
\hline



\end{tabular}
\caption{The NMI and accuracy result on DSTC11 development, test, and SGD dataset according to the loss function.}
\label{tab:class_contrast}
\end{table}

To investigate the effect of SCL during fine-tuning, we compare the result with the cross-entropy loss in Table~\ref{tab:class_contrast}. In most cases, the SCL loss demonstrates better results by a large margin; however, on the SGD dataset, the NMI and ACC results were slightly or no different than the cross-entropy loss. Considering that the SGD dataset is the only dataset with the same domain with the fine-tuning dataset (tourism), this result indicates that SCL is more useful when it is used in a domain-across environment.


\section{Conclusion}
In this paper, we describe our solution for the DSTC11 intent induction competition. We leveraged the SBERT model to embed sentences and fine-tuned the model using dependency parsing results. Additionally, we used supervised contrastive loss during fine-tuning to make the model robust in multiple domains. During the analysis, both dependency parsing and SCL helped to make the intent induction model more domain robust. Furthermore, our intent label generation with hypernym methods allows us to explain the clustering results. According to the results, our approach achieved 3rd place in terms of the precision score and demonstrated better NMI and accuracy compared to the baseline model.


\section*{Limitations}
Our contribution has two limitations. First, although DORIC shows superior performance in the domain across the environment, the increase was insignificant in the same domain environment. Second, we thoroughly examine the embedding methods, but we adapt this method only to the K-means clustering. In the future, we plan to devise a progressed clustering method that fits our embedding method.


\section*{Acknowledgements}
% should rewrite this

This work was partly supported by Institute of Information \& communications Technology Planning \& Evaluation (IITP) grant funded by the Korea government(MSIT) (No.2019-0-01906, Artificial Intelligence Graduate School Program(POSTECH)) and  ITRC(Information Technology Research Center) support program(IITP-2023-2020-0-01789).




% Entries for the entire Anthology, followed by custom entries
\bibliography{anthology,custom,jihyun}
\bibliographystyle{acl_natbib}

\appendix
\label{sec:appendix}
\section{Appendix for Proofs}

\paragraph{Proof of Theorem \ref{thm:main}.}

\begin{proof}
\label{proof:main}
Our proof has two steps. In Step 1, we will show that SimCLR is equivalent to minimizing the cross entropy loss defined in Eqn.~(\ref{eqn:cross-entropy}). 
In Step 2, we will show  that minimizing the cross-entropy loss 
is equivalent to spectral clustering on $\bfpi$. 
Combining the two steps together, we have proved our theorem. 

\textbf{Step 1: } SimCLR is equivalent to minimizing the cross entropy loss.

The cross-entropy loss takes expectation over 
$\bfW_\bfX\sim \mathbb{P}(\cdot ; \bfpi)$, 
which means $\bfW_\bfX$ has exactly one non-zero entry in each row $i$. By Lemma~\ref{lem:multinomial}, we know every row $i$ of $\bfW_\bfX$ is independent of other rows. Moreover, 
$\bfW_{\bfX,i}\sim \mathcal{M}(1, \bfpi_i/\sum_j \bfpi_{i,j})=\mathcal{M}(1, \bfpi_i)$, because $\bfpi_i$ itself is a probability distribution.
Similarly, we know $\bfW_\bfZ$ also has the row-independent property by sampling over $\mathbb{P}(\cdot;\bfK_\bfZ)$.
Therefore, by Lemma~\ref{lem:cross_split}, we know Eqn.~(\ref{eqn:cross-entropy}) is equivalent to:
\[
 -\sum_{i=1}^n \mathbb{E}_{\bfW_{\bfX,i}}[\log \mathbb{P}(\bfW_{\bfZ,i}=\bfW_{\bfX,i};\bfK_\bfZ)],
\]

This expression takes expectation over $\bfW_{\bfX,i}$ for the given row $i$. Notice that 
$\bfW_{\bfX,i}$ has exactly one non-zero entry, which equals $1$ (same for $\bfW_{\bfZ,i}$). 
As a result
we expand the above expression to be:
\begin{equation}
 -\sum_{i=1}^n \sum_{j\neq i} \Pr(\bfW_{\bfX,i,j}=1)\log \Pr(\bfW_{\bfZ,i,j}=1).
\label{eqn:detailed-expansion}    
\end{equation}


By Lemma~\ref{lem:multinomial}, $\Pr(\bfW_{\bfZ,i,j}=1)=\bfK_{\bfZ,i,j}/\|\bfK_{\bfZ,i}\|_1$ for $j\neq i$. Recall that $\bfK_\bfZ=(k(\bfZ_i-\bfZ_j))_{(i,j)\in[n]^2}$, which means 
$\bfK_{\bfZ,i,j}/\|\bfK_{\bfZ,i}\|_1=\frac{\exp(-\|\bfZ_i-\bfZ_j\|^2/{2\tau})}{\sum_{k\neq i}
\exp(-\|\bfZ_i-\bfZ_k\|^2/{2\tau})
}$ for $j\neq i$, when $k$ is the Gaussian kernel with variance $\tau$. 

Notice that $\bfZ_i=f(\bfX_i)$, so we know
\begin{equation}
-\log \Pr(\bfW_{\bfZ,i,j}=1)=
-\log \frac{\exp(-\|f(\bfX_i)-f(\bfX_j)\|^2/{2\tau})}{\sum_{k\neq i}
\exp(-\|f(\bfX_i)-f(\bfX_k)\|^2/{2\tau}),
}
\label{eqn:infonce-equivalence}    
\end{equation}


The right hand side is exactly the InfoNCE loss defined in Eqn.~(\ref{eqn:infonce}).
Inserting Eqn.~(\ref{eqn:infonce-equivalence}) into Eqn.~(\ref{eqn:detailed-expansion}), we get the SimCLR algorithm, which first samples augmentation pairs $(i,j)$ with $\Pr(\bfW_{\bfX,i,j}=1)$ for each row $i$, and then optimize the InfoNCE loss. 

\textbf{Step 2: } minimizing the cross entropy loss 
is equivalent to spectral clustering on $\bfpi$.


By Lemma~\ref{lem:convert_to_spectral}, we may further convert the loss to 
\begin{equation}
\label{eqn:main-theorem-repul-attr}
\min_{\bfZ}
-\sum_{(i,j)\in [n]^2} \mathbf{P}_{i,j}
\log k (\bfZ_i-\bfZ_j)+\log \mathbf{R}(\bfZ).
\end{equation}
Since $k$ is the Gaussian kernel, this reduces to \[
\min_\bfZ \mathrm{tr}(\bfZ^\top \mathbf{L}(\bfpi) \bfZ)
+\log \mathbf{R}(\bfZ),
\]

where we use the fact that $\mathbb{E}_{\bfW_\bfX\sim \mathbb{P}(\cdot; \bfpi)}[\mathbf{L}(\bfW_\bfX)]
=\mathbf{L}(\bfpi)
$, because the Laplacian operator is linear and $
\mathbb{E}_{\bfW_\bfX\sim \mathbb{P}(\cdot; \bfpi)}(\bfW_\bfX)=\bfpi
$.
\end{proof}

\paragraph{Proof of Theorem \ref{thm:clip}.}
\begin{proof}
Since $\bfW_\bfX\sim \mathbb{P}(\cdot;\bfpi_{\mathbf{A}, \mathbf{B}})$, we know 
$\bfW_\bfX$ has exactly one non-zero entry in each row, denoting the pair that got sampled. 
A notable difference compared to the previous proof is we now have $n_\mathcal{A}+n_\mathcal{B}$ objects in our graph. CLIP deals with this by taking a mini-batch of size $2N$, 
such that $n_\mathcal{A}=n_\mathcal{B}=N$, and adding the $2N$ InfoNCE losses together. We label the objects in $\mathcal{A}$ as $[n_\mathcal{A}]$, and the objects in $\mathcal{B}$ as $\{n_\mathcal{A}+1, \cdots, n_\mathcal{A}+n_\mathcal{B}\}$. 

Notice that $\bfpi_{\mathbf{A}, \mathbf{B}}$ is a bipartite graph, so the edges of objects in $\mathcal{A}$ will only connect to object in $\mathcal{B}$ and vice versa. We can define the similarity matrix in $\cZ$ as $\bfK_\bfZ$, 
where $\bfK_\bfZ(i, j+n_\mathcal{A})=\bfK_\bfZ(j+n_\mathcal{A},i)= k(\bfZ_i-\bfZ_j)$ for $i\in [n_\mathcal{A}], j\in [n_\mathcal{B}]$, and otherwise we set $\bfK_\bfZ(i,j)=0$. 
The rest is same as the previous proof. 
\end{proof}

\paragraph{Proof of Theorem \ref{thm:exponential}.}

\begin{proof}
\label{proof:exponential}
Since the objective function consists of a linear term combined with an entropy regularization, which is a strongly concave function, the maximization problem is a convex optimization problem. Owing to the implicit constraints provided by the entropy function, the problem is equivalent to having only the equality constraint. We then introduce the Lagrangian multiplier $\lambda$ and obtain the following relaxed problem:

$$
\widetilde{E}(\boldsymbol{\alpha})=\psi_{1}-\sum_{i=1}^n \alpha_{i} \psi_{i}+\tau \sum_{i=1}^n \alpha_{i}\log \alpha_{i}+\lambda\left(\boldsymbol{\alpha}^{\top} \mathbf{1}_n-1\right).
$$

As the relaxed problem is unconstrained, taking the derivative with respect to $\alpha_{i}$ yields

$$
\frac{\partial \widetilde{E}(\boldsymbol{\alpha})}{\partial \alpha_{i}}=-\psi_{i}+\tau\left(\log \alpha_{i}+\alpha_{i} \frac{1}{\alpha_{i}}\right)+\lambda=0.
$$

Solving the above equation implies that $\alpha_{i}$ takes the form
$
\alpha_{i}=\exp \left(\frac{1}{\tau} \psi_{i}\right) \exp \left(\frac{-\lambda}{\tau}-1\right).
$ Since $\alpha_{i}$ lies on the probability simplex, the optimal $\alpha_{i}$ is explicitly given by
$
\alpha^{*}_{i}=\frac{\exp \left(\frac{1}{\tau} \psi_{i}\right)}{\sum_{i^{\prime}=1}^n \exp \left(\frac{1}{\tau} \psi_{i^{\prime}}\right)} .
$ Substituting the optimal point into the objective function, we obtain
$$
\begin{aligned}
E\left(\boldsymbol{\alpha}^*\right)  &=\psi_1-\sum_{i=1}^n \frac{\exp \left(\frac{1}{\tau} \psi_{i}\right)}{\sum_{i^{\prime}=1}^n \exp \left(\frac{1}{\tau} \psi_{i^{\prime}}\right)} \psi_{i}+\tau \sum_{i=1}^n \frac{\exp \left(\frac{1}{\tau} \psi_{i}\right)}{\sum_{i^{\prime}=1}^n \exp \left(\frac{1}{\tau} \psi_{i^{\prime}}\right)}\log \frac{\exp \left(\frac{1}{\tau} \psi_{i}\right)}{\sum_{i^{\prime}=1}^n \exp \left(\frac{1}{\tau} \psi_{i^{\prime}}\right)} \\
& =\psi_1 - \tau \log \left(\sum_{i=1}^n \exp \left(\frac{1}{\tau} \psi_{i}\right)\right).
\end{aligned}
$$
Thus, the Lagrangian dual function is given by
\begin{equation*}
-E\left(\boldsymbol{\alpha}^*\right)= -\tau \log \frac{\exp \left(\frac{1}{\tau} \psi_{1}\right)}{\sum_{i=1}^n \exp \left(\frac{1}{\tau} \psi_{i}\right)}.\qedhere
\end{equation*}
\end{proof}



\section{More on Experiments} \label{section: experiment_details}

\paragraph{CIFAR-10 and CIFAR-100} CIFAR-10 ~\citep{krizhevsky2009learning} and CIFAR-100 ~\citep{krizhevsky2009learning} are well-known classic image classification datasets. Both CIFAR-10 and CIFAR-100 contain a total of 60k $32 \times 32$ labeled images of different classes, with 50k for training and 10k for testing. CIFAR-10 is similar to CIFAR-100, except there are 10 different classes in CIFAR-10 and 100 classes in CIFAR-100.

\paragraph{TinyImageNet} TinyImageNet ~\citep{le2015tiny} is a subset of ImageNet ~\citep{deng2009imagenet}. There are 200 different object classes in TinyImageNet, with 500 training images, 50 validation images, and 50 test images for each class. All the images in TinyImageNet are colored and labeled with a size of $64 \times 64$.

\textbf{Pseudo-code.} Algorithm \ref{alg:Training Procedure} presents the pseudo-code for our empirical training procedure.

\begin{algorithm}[!htbp]
\caption{Training Procedure}
\label{alg:Training Procedure}
\begin{algorithmic}[1]
\REQUIRE trainable encoder network $f$, batch size $N$, augmentation strategy \textit{aug}, loss function $L$ with hyperparameters \textit{args}
\FOR {sampled minibatch ${x_i}_{i=1}^N$}
\FORALL{$i \in { 1, ..., N }$}
\STATE draw two augmentations $t_i = \textit{aug}\left(x_i\right) $, $t_i' = \textit{aug}\left(x_i\right) $
\STATE $z_i = f\left(t_i\right)$, $z_i' = f\left(t_i'\right)$
\ENDFOR
\STATE compute loss $\mathcal{L} = L(N, z, z', \textit{args})$
\STATE update encoder network $f$ to minimize $\mathcal{L}$
\ENDFOR
\STATE \textbf{Return} encoder network $f$
\end{algorithmic}
\end{algorithm}

We also provide the pseudo-code for our core loss function used in the training procedure in Algorithm \ref{alg:Core loss}. The pseudo-code is almost identical to SimCLR's loss function, with the exception of an extra parameter $\gamma$.

\begin{algorithm}[!htbp]
\caption{Core loss function $\mathcal{C}$}
\label{alg:Core loss}
\begin{algorithmic}[1]
\REQUIRE batch size $N$, two encoded minibatches $z_1, z_2$, $\gamma$, temperature $\tau$
\STATE $z = \textit{concat}\left(z_1, z_2\right)$
\FOR {$i \in {1, ..., 2N }, j \in {1, ..., 2N}$ }
\STATE $s_{i,j} = \Vert z_i - z_j \Vert_2^{\gamma}$
\ENDFOR
\STATE \textbf{define} $l(i, j)$ \textbf{as} $l(i, j) = - \log \frac{exp\left(s_{i,j}/\tau \right)}{\sum_{k=1}^{2N} \mathbf{1}{[k \ne i]} exp\left(s{i, j} / \tau \right)} $
\STATE \textbf{Return} $\frac{1}{2N} \sum_{k=1}^N\left[l(i, i+N) + l(i+N, i)\right]$
\end{algorithmic}
\end{algorithm}

Utilizing the core loss function $\mathcal{C}$, we can define all kernel loss functions used in our experiments in Table \ref{table: loss definition}. For all $z_i \in z$ with even dimensions $n$, we define $z_{L_i} = z_i\left[0:n/2\right]$ and $z_{R_i} = z_i\left[n/2:n\right]$.

\begin{table}[ht]
\centering
\begin{tabular}{{@{}l|l@{}}}
Kernel  &  Loss function \\ \midrule
Laplacian & $\mathcal{C}\left(N, z, z', \gamma=1, \tau\right)$\\ \midrule
Sum       & $\lambda * \mathcal{C}\left(N, z, z', \gamma=1, \tau_1\right) + (1-\lambda) * \mathcal{C}\left(N, z, z', \gamma=2, \tau_2\right)$  \\ \midrule
Concatenation Sum&$\lambda * \mathcal{C}\left(N, z_L, z'_L, \gamma=1, \tau_1\right) + (1-\lambda) * \mathcal{C}\left(N, z_R, z'_R, \gamma=2, \tau_2\right)$\\ \midrule
$\gamma = 0.5$ & $\mathcal{C}\left(N, z, z', \gamma=0.5, \tau\right)$          \\ 

\end{tabular}

\caption{Definition of kernel loss functions in our experiments}
\label {table: loss definition}
\end{table}

\textbf{Baselines.} We reproduce the SimCLR algorithm using PyTorch Lightning~\citep{PytorchLightning}.

\textbf{Encoder details.}
The encoder $f$ consists of a backbone network and a projection network. We employ ResNet50~\citep{ResNet} as the backbone and a 2-layer MLP (connected by a batch normalization~\citep{ioffe2015batch} layer and a ReLU \cite{nair2010rectified} layer) with hidden dimensions 2048 and output dimensions 128 (or 256 in the concatenation kernel case).

\textbf{Encoder hyperparameter tuning.}
For each encoder training case, we randomly sample 500 hyperparameter groups (sample details are shown in Table \ref{table: Hyperparameter sample}) and train these samples simultaneously using Ray Tune ~\citep{RayTune}, with the ASHA scheduler~\citep{li2018massively}. Ultimately, the hyperparameter group that maximizes the online validation accuracy (integrated in PyTorch Lightning) within 5000 validation steps is chosen for the given encoder training case.

\begin{table}[ht]
\centering

\begin{tabular}{@{}l|l|l@{}}
\midrule
Hyperparameter  & Sample Range & Sample Strategy \\ \midrule
start learning rate & $\left[10^{-2}, 10\right]$ & log uniform \\ \midrule
$\lambda$       & $\left[0, 1\right]$ & uniform \\ \midrule
$\tau$, $\tau_1$, $\tau_2$ & $\left[0, 1\right]$ & log uniform \\ \midrule
\end{tabular}

\caption{Hyperparameters sample strategy}
\label {table: Hyperparameter sample}
\end{table}

\textbf{Encoder training.} 
We train each encoder using the LARS optimizer~\citep{LARSOptimizer}, LambdaLR Scheduler in PyTorch, momentum 0.9, weight decay $10^{-6}$, batch size 256, and the aforementioned hyperparameters for 400 epochs on a single A-100 GPU.

\textbf{Image transformation.} The image transformation strategy, including augmentation, is identical to the default transformation strategy provided by PyTorch Lightning.

\textbf{Linear evaluation.}
The linear head is trained using the SGD optimizer with a cosine learning rate scheduler, batch size 64, and weight decay $10^{-6}$ for 100 epochs. The learning rate starts at $0.3$ and ends at $0$.

\textbf{Moco Experiments.} We also tested our method based on MoCo~\citep{he2019moco}. The results are summarized in Table \ref{tab:results-moco}. Here we choose ResNet18~\citep{ResNet} as the backbone and set a temperature of $0.1$ as default. For our simple sum kernel, we set $\lambda=0.8$. The results show that our method outperforms the original MoCo method.

\begin{table}[thb]
\centering
\caption{MoCo Experiment Results on CIFAR-10 and CIFAR-100.}
\label{tab:results-moco}
\resizebox{\textwidth}{!}{%
\begin{tabular}{@{}c|ccc|ccc@{}}
\toprule
\multirow{3}{*}{Method} & \multicolumn{3}{c|}{CIFAR-10} & \multicolumn{3}{c}{CIFAR-100} \\ \cmidrule(lr){2-4} \cmidrule(lr){5-7} 
                        & 200 epochs & 400 epochs    & 1000 epochs   & 200 epochs & 400 epochs & 1000 epochs         \\ \midrule
MoCo (repro.)         & $76.41 \pm 0.12$    & $80.01 \pm 0.15$          & $84.45 \pm 0.08$    & $\mathbf{47.02 \pm 0.11}$ & $52.50 \pm 0.07$ & $57.62 \pm 0.15$            \\
\midrule
Laplacian Kernel        & ${78.09 \pm 0.10}$    & $\mathbf{83.85 \pm 0.09}$          & $\mathbf{88.34 \pm 0.16}$    & $46.12 \pm 0.22$   & $53.44 \pm 0.17$ & $59.10 \pm 0.14$        \\
Simple Sum Kernel & $\mathbf{78.12 \pm 0.15}$   & $83.23 \pm 0.18$ & $87.50 \pm 0.20$ & $46.65 \pm 0.06$ & $\mathbf{53.62 \pm 0.19}$ & $\mathbf{59.83 \pm 0.12}$\\
\bottomrule
\end{tabular}
}
\end{table}



\section{More Experiments on Synthetic Data}


Consider a scenario with $n$ clusters, each containing $k$ vertices. Let the probability of vertices $u$ and $v$ from the same cluster belonging to $\bfpi$ be $p$. Conversely, for vertices $u$ and $v$ from different clusters, let the probability of belonging to $\pi$ be $q$. We generate the graph $\bfpi$ randomly, based on $p$ and $q$. We experiment with values of $k=100$ and $n=6$ for ease of visualization, embedding all points in a two-dimensional space. Each vertex's initial position originates from a normal distribution. In each iteration, we sample a subgraph of $\bfpi$ uniformly, ensuring each vertex has an out-degree of $1$. We then optimize the corresponding vectors using InfoNCE loss with an SGD optimizer and iterate until convergence. Our experimental setup consists of an SGD learning rate of $1$, an InfoNCE loss temperature of $0.5$, and a batch size of $50$. We evaluate two scenarios with different $p$ and $q$ values: $p=1$, $q=0$, and $p=0.75$, $q=0.2$. The results of these experiments are visualized in Figure \ref{fig:vis-spectral-cluster}. The obtained embeddings exhibit the hallmark pattern of spectral clustering of graph $\bfpi$.

\begin{figure}[!tb]
\centering
\subfigure{
\includegraphics[width=1\textwidth]{Figures/cluster_pi.png}
\label{fig:vis-cluster}
}
\subfigure{
\includegraphics[width=1\textwidth]{Figures/noised_cluster_pi.png}
\label{fig:vis-noised-cluster}
}
\caption{Visualizations of the optimization process using InfoNCE Loss on the vectors corresponding to $\bfpi$. Points of identical color belong to the same cluster within $\bfpi$. To showcase the internal structure of $\bfpi$, we randomly select 10 vertices from each cluster to display the edge distribution of $\bfpi$.}
\label{fig:vis-spectral-cluster}
\end{figure}


\end{document}




% % Entries for the entire Anthology, followed by custom entries
% \bibliography{anthology,custom}
% \bibliographystyle{acl_natbib}

% \appendix

% \section{Example Appendix}
% \label{sec:appendix}

% This is a section in the appendix.

% \end{document}
