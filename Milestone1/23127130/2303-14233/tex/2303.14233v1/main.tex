% Template for PLoS
% Version 3.5 March 2018
%
% % % % % % % % % % % % % % % % % % % % % %
%
% -- IMPORTANT NOTE
%
% This template contains comments intended 
% to minimize problems and delays during our production 
% process. Please follow the template instructions
% whenever possible.
%
% % % % % % % % % % % % % % % % % % % % % % % 
%
% Once your paper is accepted for publication, 
% PLEASE REMOVE ALL TRACKED CHANGES in this file 
% and leave only the final text of your manuscript. 
% PLOS recommends the use of latexdiff to track changes during review, as this will help to maintain a clean tex file.
% Visit https://www.ctan.org/pkg/latexdiff?lang=en for info or contact us at latex@plos.org.
%
%
% There are no restrictions on package use within the LaTeX files except that 
% no packages listed in the template may be deleted.
%
% Please do not include colors or graphics in the text.
%
% The manuscript LaTeX source should be contained within a single file (do not use \input, \externaldocument, or similar commands).
%
% % % % % % % % % % % % % % % % % % % % % % %
%
% -- FIGURES AND TABLES
%
% Please include tables/figure captions directly after the paragraph where they are first cited in the text.
%
% DO NOT INCLUDE GRAPHICS IN YOUR MANUSCRIPT
% - Figures should be uploaded separately from your manuscript file. 
% - Figures generated using LaTeX should be extracted and removed from the PDF before submission. 
% - Figures containing multiple panels/subfigures must be combined into one image file before submission.
% For figure citations, please use "Fig" instead of "Figure".
% See http://journals.plos.org/plosone/s/figures for PLOS figure guidelines.
%
% Tables should be cell-based and may not contain:
% - spacing/line breaks within cells to alter layout or alignment
% - do not nest tabular environments (no tabular environments within tabular environments)
% - no graphics or colored text (cell background color/shading OK)
% See http://journals.plos.org/plosone/s/tables for table guidelines.
%
% For tables that exceed the width of the text column, use the adjustwidth environment as illustrated in the example table in text below.
%
% % % % % % % % % % % % % % % % % % % % % % % %
%
% -- EQUATIONS, MATH SYMBOLS, SUBSCRIPTS, AND SUPERSCRIPTS
%
% IMPORTANT
% Below are a few tips to help format your equations and other special characters according to our specifications. For more tips to help reduce the possibility of formatting errors during conversion, please see our LaTeX guidelines at http://journals.plos.org/plosone/s/latex
%
% For inline equations, please be sure to include all portions of an equation in the math environment.  For example, x$^2$ is incorrect; this should be formatted as $x^2$ (or $\mathrm{x}^2$ if the romanized font is desired).
%
% Do not include text that is not math in the math environment. For example, CO2 should be written as CO\textsubscript{2} instead of CO$_2$.
%
% Please add line breaks to long display equations when possible in order to fit size of the column. 
%
% For inline equations, please do not include punctuation (commas, etc) within the math environment unless this is part of the equation.
%
% When adding superscript or subscripts outside of brackets/braces, please group using {}.  For example, change "[U(D,E,\gamma)]^2" to "{[U(D,E,\gamma)]}^2". 
%
% Do not use \cal for caligraphic font.  Instead, use \mathcal{}
%
% % % % % % % % % % % % % % % % % % % % % % % % 
%
% Please contact latex@plos.org with any questions.
%
% % % % % % % % % % % % % % % % % % % % % % % %

\documentclass[10pt,letterpaper]{article}
\usepackage[top=0.85in,left=2.75in,footskip=0.75in]{geometry}

\usepackage{float}

% amsmath and amssymb packages, useful for mathematical formulas and symbols
\usepackage{amsmath,amssymb}

% Use adjustwidth environment to exceed column width (see example table in text)
\usepackage{changepage}

% Use Unicode characters when possible
\usepackage[utf8x]{inputenc}

% textcomp package and marvosym package for additional characters
\usepackage{textcomp,marvosym}

% cite package, to clean up citations in the main text. Do not remove.
\usepackage{cite}

% Use nameref to cite supporting information files (see Supporting Information section for more info)
\usepackage{nameref,hyperref}

% line numbers
% \usepackage[right]{lineno}

% ligatures disabled
\usepackage{microtype}
\DisableLigatures[f]{encoding = *, family = * }

% color can be used to apply background shading to table cells only
\usepackage[table]{xcolor}

% array package and thick rules for tables
\usepackage{array}

% create "+" rule type for thick vertical lines
\newcolumntype{+}{!{\vrule width 2pt}}

% create \thickcline for thick horizontal lines of variable length
\newlength\savedwidth
\newcommand\thickcline[1]{%
  \noalign{\global\savedwidth\arrayrulewidth\global\arrayrulewidth 2pt}%
  \cline{#1}%
  \noalign{\vskip\arrayrulewidth}%
  \noalign{\global\arrayrulewidth\savedwidth}%
}

% \thickhline command for thick horizontal lines that span the table
\newcommand\thickhline{\noalign{\global\savedwidth\arrayrulewidth\global\arrayrulewidth 2pt}%
\hline
\noalign{\global\arrayrulewidth\savedwidth}}


% Remove comment for double spacing
%\usepackage{setspace} 
%\doublespacing

% Text layout
\raggedright
\setlength{\parindent}{0.5cm}
\textwidth 5.25in 
\textheight 8.75in

% Bold the 'Figure #' in the caption and separate it from the title/caption with a period
% Captions will be left justified
\usepackage[aboveskip=1pt,labelfont=bf,labelsep=period,justification=raggedright,singlelinecheck=off]{caption}
\renewcommand{\figurename}{Fig}

% Use the PLoS provided BiBTeX style
\bibliographystyle{plos2015}

% Remove brackets from numbering in List of References
\makeatletter
\renewcommand{\@biblabel}[1]{\quad#1.}
\makeatother



% Header and Footer with logo
\usepackage{lastpage,fancyhdr,graphicx}
\usepackage{epstopdf}
%\pagestyle{myheadings}
\pagestyle{fancy}
\fancyhf{}
%\setlength{\headheight}{27.023pt}
%\lhead{\includegraphics[width=2.0in]{PLOS-submission.eps}}
\rfoot{\thepage/\pageref{LastPage}}
\renewcommand{\headrulewidth}{0pt}
\renewcommand{\footrule}{\hrule height 2pt \vspace{2mm}}
\fancyheadoffset[L]{2.25in}
\fancyfootoffset[L]{2.25in}
\lfoot{\today}

%% Include all macros below

\newcommand{\lorem}{{\bf LOREM}}
\newcommand{\ipsum}{{\bf IPSUM}}

%% END MACROS SECTION


\begin{document}
\vspace*{0.2in}

% Title must be 250 characters or less.
\begin{flushleft}
{\Large
\textbf\newline{A computer vision based optical method for measuring fluid level in cell culture plates} % Please use "sentence case" for title and headings (capitalize only the first word in a title (or heading), the first word in a subtitle (or subheading), and any proper nouns).
}
\newline
% Insert author names, affiliations and corresponding author email (do not include titles, positions, or degrees).
\\
Pierre V. Baudin\textsuperscript{1,2,*},
Mircea Teodorescu\textsuperscript{1,2},
\\
\bigskip
\textbf{1} Department of Electrical and Computer Engineering,   University of California Santa Cruz, Santa Cruz, CA 95064, USA
\\
\textbf{2} Genomics Institute, University of California Santa Cruz, Santa Cruz, CA 95064, USA
\\
\bigskip

% Insert additional author notes using the symbols described below. Insert symbol callouts after author names as necessary.
% 
% Remove or comment out the author notes below if they aren't used.
%
% Primary Equal Contribution Note
% \Yinyang These authors contributed equally to this work.

% Additional Equal Contribution Note
% Also use this double-dagger symbol for special authorship notes, such as senior authorship.
% \ddag These authors also contributed equally to this work.

% Current address notes
% \textcurrency Current Address: Dept/Program/Center, Institution Name, City, State, Country % change symbol to "\textcurrency a" if more than one current address note
% \textcurrency b Insert second current address 
% \textcurrency c Insert third current address

% Deceased author note
% \dag Deceased

% Group/Consortium Author Note
% \textpilcrow Membership list can be found in the Acknowledgments section.

% Use the asterisk to denote corresponding authorship and provide email address in note below.
* pvbaudin@ucsc.edu

\end{flushleft}
% Please keep the abstract below 300 words
\section*{Abstract}
For a transparent well with a known volume capacity, changes in fluid level result in predictable changes in magnification of an overhead light source. For a given well size and fluid, the relationship between volume and magnification can be calculated if the fluid’s index of refraction is known or in a naive fashion with a calibration procedure. Light source magnification can be measured through a camera and processed using computer vision contour analysis with OpenCV. This principle was applied in the design of a 3D printable sensing device using a raspberry pi zero and a camera



% \linenumbers

% Use "Eq" instead of "Equation" for equation citations.
\section*{Introduction}\label{sec:intro}

% Motivation:
% Applications that require fluid sensing 
% Common methods of fluid sensing and their drawbacks

The automation of cell culture procedures has the potential to greatly increase the throughput and consistency of cell culture based experiments \cite{doulgkeroglou_automation_2020}. Manipulation of fluids of is a key feature for any automated culture platform. Systems must have the ability to deliver or move precise quantities of fluids. To automate movement of fluids, some systems use a "lab-on-a-chip" approach \cite{figeys_lab---chip_2000}, utilizing microfluidic channels to transport fluids. Other systems use robot manipulators to deliver measured quantities of fluids with micro-pipettes \cite{doulgkeroglou_automation_2020, fleischer_analytical_2018, steffens_versatile_2017}. Regardless of mechanism behind the delivery of fluids, being able to detect the presence and volume of fluids within provides key error detection and correction functionality that open loop methods lack.

Surface level measurement within a container of known volume is a simple way to determine fluiod volume. This can be done with electrically active probes dipped into a fluid being measured \cite{singh_review_2019}. These probes can be effective but direct contact with the fluid can be problematic for cell culture applications wherein any potential contamination vector increases the risk of failure \cite{lincoln_chapter_1998}. Non-contact sensing is preferable for sensitive applications. Optical fluid sensing methods take advantage of how fluids interact with light. Several optical approaches for measuring fluids in culture plates exist. One method uses computer vision with well placed light sources and sensors placed at specific angles to measure ray deflection \cite{jain_total-internal-reflection_2021}. This method uses reflectance of the fluid surface, which makes it usable no matter the solid mass contents of the well. However, the angular requirements of the sensor electronics makes tight packing of independent sensing elements infeasible, making parallel measurements within a plate complex or impossible. Another published method involves imaging the distortions in a printed grid to visualize changes in refraction related to fluid level \cite{litt_visualization_1989}. While this approach can be used in parallel on many wells at once, it is highly affected by occlusion resulting from material in the plate and can be affected by ambient lighting conditions.

% Two other published methods use overhead image measurements to infer fluid level from resulting distortions, one tracking sizes of added droplet sizes in microplates \cite{thurow_fast_2011}, 


%% why are these approaches lacking 

Here we propose a method for non-contact optical detection of fluid level using a CMOS camera sensor and an LED. We detail the principles at work and show a 3D printed device that can be used to perform measurements in 24 well culture plates. Similar to the approach shown in \cite{litt_visualization_1989}, this method relies on the lens-like properties of fluids and the resulting changes of an image viewed through them. Instead of viewing a grid, we detect the apparent size of an overhead light source. Changes in the apparent distance of the light source correlate to changes in fluid level. A bright overhead light source can shine through some samples, making this approach usable in scenarios with solid occlusion that would disrupt the viewing of a grid. Additionally, the electronics required are set up directly inline with the sample wells, allowing electronics to be packed into a grid allowing parallel sensing of many wells at once. 
\section*{Methods}\label{sec:methods}
\subsection*{Operating Principle}
\label{subsec:principle}
\begin{figure}[H]
    \centering
    \includegraphics[width = \columnwidth]{figure-pics/generalized-operating-principle.png}
    \caption{\textbf{Apparent distance of the light source changes based on fluid depth for fluids with index of refraction greater than air} (relationship is inverted if index is less than air). Refraction angles can be determined with Snell's law with known indices of refraction of the 2 media ($n_1 sin(\theta_1) = n_2 sin(\theta_2)$). In the case of second transition back into the original medium, the relationship $\theta_3 = \theta_1$ can be derived, hence the labeling of the third angle as $\theta_1$ in this diagram. $L$ = fluid depth, $I$ = apparant distance of object. $h_0$ = distance between object and top of fluid, $h_1$ = distance between bottom of fluid and measurement plane.}
    \label{fig:principle}
\end{figure}
% h_0 = (h - h_1) - L
% $$
% $h - h_1$ is constant and we will now label it as $C_0$
% $$
% \begin{aligned}
% &x_{0}=(C_0 - L) \tan \theta_{1} \\
% &C_1 = \tan \theta_{1} \\
% &x_{1}= x_0 + L \tan \theta_{2} \\
% &C_2 = \tan \theta_{2} \\
% &x_f = x_1 + h_1 C_1 \\
% &x_f = (C_0 - L)C_1 + L C_2 +h_1 C_1 \\
% &x_f = C_0C_1 - LC_1 + L C_2 +h_1 C_1 \\
% &x_f = L(-C_1 + C_2) + h_1C_1 + C_0C_1 \\
% & \tan(\theta_1) = \frac{x_f}{I} \\

The measurement principle behind this system relies on the lensing effects of fluids with different indices of refraction. In figure \ref{fig:principle} the diagrams represent how the apparent distance of the light source changes based on fluid depth. In the case of a fluid with a index of refraction greater than air (like water), the light source will appear closer as the fluid level increases. \\

To derive a function relating the fluid depth to the apparent distance of the light source, we define the following constants.
$$
\begin{aligned}
& C_0 = h - h_1 \\
& C_1 = \tan \theta_{1} \\
& C_2 = \tan \theta_{2}
\end{aligned}
$$
These can be considered constants because our light source sends rays in all directions below it, so any angle $\theta_1$ can be chosen so long as it results in a ray that intersects the fluid, and from this chosen angle $\theta_2$ can be derived via Snell's Law. In terms of these constants the resulting transfer function is:
$$
I=\frac{L(-C_1 + C_2) + h_1C_1 + C_0C_1}{C_0}
$$
This function relates the apparent distance of the image $I$ to the fluid level $L$ and is a first degree polynomial. The full derivation can be found in section \ref{S1_Appendix} of the supplemental materials. This function is a useful approximation that ignores several factors including refraction caused by the plate material and refraction from the meniscus geometry of the fluid surface, our data show that accurate results can be obtained with this approximation in most scenarios. Meniscus has a substantial effect on measurement in two extremes, an almost empty well, and an almost full well. These meniscus effects are further explored in section \ref{sec:results}

% one edge case where the meniscus has a substantial effect of the measurement is examined in section \ref{sec:min-invert}. Additionally, users should be aware that the effect of meniscus geometry on the flatness of the surface is minimized in the center of the fluid container, making that the optimal position for measurement.

% $$
% \begin{aligned}
% &h = h_0 + L + h_1 \\
% &x_{1}=h_{0} \tan \theta_{1} \\
% &x_{2}=h_{f} \tan \theta_{2} \\
% &x=h_{2} \tan \theta_{1} \\
% &x=\left(h_{0}+h_{2}\right) \tan \theta_{1}+L \tan \theta_{2} \\
% &\theta_{x}=90-\theta_{1} \\
% &h_{I}=\frac{x}{\tan \theta_{x}} \\
% &h_{I}=\frac{\left(h_{0}+h_{1}\right) \tan \theta_{1}+L \tan \theta_{2}}{\tan \left(90-\theta_{1}\right)}
% \end{aligned}
% $$




% \begin{figure}[H]
%     \centering
%     \includegraphics[width = 0.25\columnwidth]{2_Methods/Pictures/grabstract.png}
%     \caption{hardware setup placeholder}
%     \label{fig:setup}
% \end{figure}

The polynomial can be characterized analytically so long as we can determine how the image distance relates to size on the camera sensor. This can be determined by taking a picture of a ruled measurement calibration slide at several known distances. It can also be characterized by a simple calibration step, being that the relationship is linear, a 2 point calibration would theoretically be sufficient, but for greater accuracy a 5 point calibration would be more prudent. Once this polynomial is characterized, image data is sufficient for capturing fluid level

\subsection*{Hardware}
To test this approach, we designed a 3D printable rig that holds a raspberry pi camera and a white LED. Figure \ref{fig:exploded-view} shows 3D renders of the parts comprising the measurement device and figure \ref{fig:device-with-plate} shows the experimental setup.

\begin{figure}[H]
    \centering
    \includegraphics[width=\textwidth]{figure-pics/hardware-render.png}
    \caption{\textbf{3D renders of measurement rig} A: a 3W white LED, B: 3D printed enclosure with a plug to hold the light in place, C: Raspberry Pi Spy Camera, D: Raspberry Pi Zero W, E: rigid plate holder, F: experimental setup. Rendered using Fusion360, real setup shown in figure \ref{fig:device-with-plate} }
    \label{fig:exploded-view}
\end{figure}
\begin{figure}[H]
    \centering
    \includegraphics[width=\textwidth]{figure-pics/physical-setup.png}
    \caption{\textbf{Experimental setup with 24 Well plate} }
    \label{fig:device-with-plate}
\end{figure}


\subsection*{Analysis Pipeline}
Using openCV \cite{opencv_library}, a free and open-source machine vision toolkit, we can capture and measure the size of spot light image. This can be done on static images or on videos including live feeds. We do this by detecting the central contour in the image and fitting an ellipse to it. The result can be calculated fast and is robust to occlusion and noise. In order to do this On the raspberry pi zero we run a video stream using the open-source RPi Cam Web Interface \cite{noauthor_rpi-cam-web-interface_nodate}. The python code for this image analysis is lightweight and can be run internally on the raspberry pi or on an external system. For external system use, the code grabs frames from the MJPEG stream generated by the camera web interface.

% The relevant code can be found on github, link is included in the supplemental materials.




Figure \ref{fig:detection-process} details the process through which a data point is obtained. Once an ellipse is fit to the central contour of the image, we filter the output by accepting the average value of a rolling buffer when the standard deviation of that buffer falls below a threshold value. This is to compensate for the disturbance of the fluid surface when new fluid is added or if the plate is shaken. By only accepting reads when the reading has stabilized, we avoid wildly fluctuating erroneous readings any time the fluid is disturbed.

\begin{figure}[H]
    \centering
    \includegraphics[width=\textwidth]{figure-pics/Detection-Process-Flowchart.png}
    \caption{\textbf{Flowchart representing detection process pipeline}}
    \label{fig:detection-process}
\end{figure}

\section*{Results}\label{sec:results}
%% meniscus formation and miniscus inversion
% show plot of error
% justify using only the linear portion
% show plot of linear portion
% show that error is actually minimized at 2

\begin{figure}[H]
    \centering
    \includegraphics[width=\textwidth]{figure-pics/meniscus-invert-v2.png}
    \caption{\textbf{Contour ellipse perimeter as a function of fluid volume.}
    Region A shows behavior before meniscus has reached stable geometry. Region B has stable meniscus and shows linear response as predicted in section \ref{subsec:principle}, Region C shows when the well begins to get full, the surface tension causes Meniscus Inversion. Region D is where fluid has spilled over the edge of the well, flattening the meniscus.}
    \label{fig:data-with-meniscus}
\end{figure}

The plot shown in figure \ref{fig:data-with-meniscus} shows the output of our sensing system for from the first possible reading to the overflowing of the well. Figure \ref{fig:Dry-Fill} shows why the first possible reading from a previously dry well occurs around 0.4ml. When the fluid level is very low in a well, the geometry of the meniscus cannot fully form. The center of the meniscus will rise more slowly before the full geometry of the meniscus is formed, this is because of the fluid volume held in the meniscus edges. Therefore, before the full formation of the meniscus, the overhead light image viewed from the center of the meniscus grows more slowly up until point A shown in figure \ref{fig:data-with-meniscus}. As the well starts to get full, a new distortion of the meniscus geometry occurs due to the surface tension of the fluid. The image grows substantially faster due to the magnifying properties of the convex meniscus, this phenomenon begins at point B. Finally, when the well overflows, the meniscus flattens resulting in a final level of unchanging magnification beginning at point C.

The region between points A and B are where this sensing process will be most generalizable, as the relationship between the spotlight image contour size and the fluid level is highly linear. 
% \begin{figure}[H]
%     \centering
%     \includegraphics[width=\textwidth]{3_Results/Pictures/Error-comparison.png}
%     \caption{Error comparison, meniscus geometry formation and inversion visible as large increase in error}
%     \label{fig:full-dataset-error-polyorder}
% \end{figure}

\subsection*{Calibration}
Deriving an effective transfer function for this system can be done by fitting a curve to a set of calibration points. Deciding on the best fit curve is a question of trade offs. The phenomenon we are measuring is linear within the range of volumes that have a stable meniscus geometry. A simple 2 point linear fit with points from the linear region, generates outputs with average error below 100$\mu$l within the linear range. Outside this range, the estimation error increases substantially. 

In order to represent the entire curve, we need a high order polynomial fit. Using a least squares approximation method, we can fit polynomials to any subset of points we choose. Considering the phenomena shown in figure \ref{fig:data-with-meniscus}, the curve we are attempting to characterize has 3 distinct regions. Calibration points should be selected to include points from each of these regions. Precision measurements of well volume inside the meniscus inversion region is unlikely to be very accurate, as disruptions in surface tension can cause overflow to occur at different points. Therefore for calibration it should be sufficient to model the linear region occurring at the start of meniscus inversion. This is adequate to detect that the fluid level is greater than the capacity of the well. Since the other 2 regions are linear, we can adequately fit a curve with 2 data points selected from each region. Comparisons of different order curve fits can be seen in figure \ref{fig:polynomial-curve-compare-by-order}



% Figure \ref{fig:total_error_points} shows the total error for each curve when applied to a test set of data from measurements withheld from the curve fitting operation. Each set of calibration points includes 2 points close to each inflection as well as a random selection from the points between those. For example, the 10 point calibration set consists of 2 points near the beginning and end, and a random selection of 6 other points from the middle region. Curves are compared to the baseline 2 point linear fit shown as a blue dotted line. The data show that around 10 points, the higher order fits start to outperform the linear fit. However, past a the 4th order fit, gains from increasing the polynomial order diminish greatly, and beyond 5th order going higher frequently diminishes performance. 

% \begin{figure}[H]
%     \centering
%     \includegraphics[width=\textwidth]{figure-pics/total_error_comparison_from_points2.png}
%     \caption{Comparing magnitude of total error from least squares polynomial fits of different order}
%     \label{fig:total_error_points}
% \end{figure}



\begin{figure}[H]
    \centering
    \includegraphics[width=\textwidth]{figure-pics/polynomial-fit-comparison-A6-100-6point.png}
    \caption{\textbf{Curve comparisons} Comparing 2 point linear fit with 6 point least square polynomial fits of different orders}
    \label{fig:polynomial-curve-compare-by-order}
\end{figure}


% \begin{figure}[H]
%     \centering
%     \includegraphics[width=\textwidth]{figure-pics/Error-comparison_10_point.png}
%     \caption{Comparing errors from polynomial fits}
%     \label{fig:polynomial-error-compare-by-order}
% \end{figure}

% Figures \ref{fig:polynomial-curve-compare-by-order} and \ref{fig:polynomial-error-compare-by-order} show the curves generated from these calibration points and the error per point. Here we can see that while the 4th and 5th order fits have the same overall error magnitude, the 5th order curve generates considerably less error in the regions near the inflection points. For this reason we can say the a 10 point calibration with a 5th order polynomial fit would be a solid choice for this system's transfer function. However if the users needs are constrained to the linear region, then the inaccuracies of a simple 2 point calibration may be a worthwhile trade off for its simplicity and speed of setup.

\subsubsection*{Generalizability of transfer function}
To evaluate the consistency of this measurement principle, data was captured in 2 different wells during 2 different runs. The resulting error magnitudes are shown in figure \ref{fig:polynomial-error-compare-by-order-and-well}

\begin{figure}[H]
    \centering
    \includegraphics[width=\textwidth]{figure-pics/error-compare-wells.png}
    \caption{\textbf{Error Comparison} Comparing errors across 4 runs for various polynomial fits}
    \label{fig:polynomial-error-compare-by-order-and-well}
\end{figure}

As expected, the error from the 2 point linear fit is substantial outside the central linear range, but within this range, the measurement error can be reasonable depending on the precision needs of the user. Having the option of a simple 2 point calibration for certain uses is an advantage. The other polynomial fits use the 6 points shown in the curve in figure \ref{fig:polynomial-curve-compare-by-order}. It is interesting to note that the 4th order curve fit performs better than the 5th order fit. Most datapoints shown fall within 100$\mu l$ error with occasional outlier measurements with a maximum error of 190$\mu l$. Knowing this measurement precision means that volumetric fluid dosing measured with this particular implementation of this principle should be tolerant to errors up to 190$\mu l$.

% \begin{itemize}
%     \item surface geometry is not flat, therefore measurements must be taken at the same position for any given well
%     \item curves will differ based on well geometry, material, and fluid properties
%     \item captured data shows that 
% \end{itemize}







% A better approach might be to take what we know about the physical principles at play here to create a piecewise transfer function for the three distinct regions of interest. If we separate meniscus formation, the linear region, and meniscus inversion, we can...

\subsection*{Dry Well}
When a well is completely dry, droplets hold their shape rather than forming a layer along the bottom. Once $~$0.5ml have been placed in the dry well, a layer adheres to the bottom and the measurement starts to work. This is shown in figure \ref{fig:Dry-Fill} The well can also be moistened beforehand with a small amount of water, doing so disrupts the formation of droplets and allows the measurements to work at smaller fluid volumes.

\begin{figure}[H]
    \centering
    \includegraphics[width=0.75\textwidth]{figure-pics/Dry-Fill.png}
    \caption{\textbf{Limitations with dry wells} before a fluid film has formed across the dry bottom of the well, measurements have no significance}
    \label{fig:Dry-Fill}
\end{figure}

\section*{Discussion}\label{sec:discussion}
By leveraging the availability of low cost camera hardware and open source machine vision tools, devices that use computer vision for sensing tasks are likely to become more ubiquitous. The proliferation of cheap mass-produced camera sensors is an opportunity for designers to consider many new methods of measurement.

The device shown here serves as a simple proof of concept for the use of this fluid level measurement principle. For practical purposes, this should be deployed with many cameras in parallel, each monitoring a single well. Previous work has been published laying out the design and use of a 24 well parallel microscope system using similar camera hardware and LEDs, this system is called the "Picroscope" \cite{ly_picroscope_2021, baudin_low_2021}. The  picroscope is also built to be compatible with a microfluidic cell culture feeding platform \cite{seiler_modular_2022}. Applying this principle to this system can enable feedback control for fluid contents of the wells in this and other microfluidic "lab-on-a-chip" type systems.

It would also be possible to use this principle without the overhead LED being permanently affixed above the culture plate. In applications using pipette robots, LEDs could be attached to the end effector, if the arm positions the LED at a known position, spot measurements can be taken and used to detect potential dosing errors or losses due to evaporation.

The principle can also be applied to much larger containers where fluid level measurements are important. Fluid storage containers exist for many applications and while approaches for measuring level in large containers exist, few of those applications are applicable to small volume containers like culture plates. The approach presented here may be broadly applicable to many fields. Furthermore, with different light frequency selections, this principle could be applied to fluids that are opaque in the visible light spectrum but transparent to infrared.

This proposed fluid level sensing method is a simple, reliable, and accurate approach that has been validated for cell culture plates and has potential for uses in other applications. 

\section*{Supporting information}

% Include only the SI item label in the paragraph heading. Use the \nameref{label} command to cite SI items in the text.
% \paragraph*{S1 Fig.}
% \label{S1_Fig}
% {\bf Bold the title sentence.} Add descriptive text after the title of the item (optional).

% \paragraph*{S2 Fig.}
% \label{S2_Fig}
% {\bf Lorem ipsum.} Maecenas convallis mauris sit amet sem ultrices gravida. Etiam eget sapien nibh. Sed ac ipsum eget enim egestas ullamcorper nec euismod ligula. Curabitur fringilla pulvinar lectus consectetur pellentesque.

% \paragraph*{S1 File.}
% \label{S1_File}
% {\bf Lorem ipsum.}  Maecenas convallis mauris sit amet sem ultrices gravida. Etiam eget sapien nibh. Sed ac ipsum eget enim egestas ullamcorper nec euismod ligula. Curabitur fringilla pulvinar lectus consectetur pellentesque.

% \paragraph*{S1 Video.}
% \label{S1_Video}
% {\bf Lorem ipsum.}  Maecenas convallis mauris sit amet sem ultrices gravida. Etiam eget sapien nibh. Sed ac ipsum eget enim egestas ullamcorper nec euismod ligula. Curabitur fringilla pulvinar lectus consectetur pellentesque.

\paragraph*{S1 Appendix.}
\label{S1_Appendix}
{\bf Distance function derivation.} detailed derivation of image distance transfer function

% \paragraph*{S1 Table.}
% \label{S1_Table}
% {\bf Lorem ipsum.} Maecenas convallis mauris sit amet sem ultrices gravida. Etiam eget sapien nibh. Sed ac ipsum eget enim egestas ullamcorper nec euismod ligula. Curabitur fringilla pulvinar lectus consectetur pellentesque.

\section*{Acknowledgments}
P.V. Baudin thanks David Haussler for the support and guidance he provides to the many projects in our research group.

% \nolinenumbers

% Either type in your references using
% \begin{thebibliography}{}
% \bibitem{}
% Text
% \end{thebibliography}
%
% or
%
% Compile your BiBTeX database using our plos2015.bst
% style file and paste the contents of your .bbl file
% here. See http://journals.plos.org/plosone/s/latex for 
% step-by-step instructions.
% 
\begin{thebibliography}{10}

    \bibitem{doulgkeroglou_automation_2020}
    Doulgkeroglou MN, Di~Nubila A, Niessing B, König N, Schmitt RH, Damen J,
      et~al.
    \newblock Automation, {Monitoring}, and {Standardization} of {Cell} {Product}
      {Manufacturing}.
    \newblock Frontiers in Bioengineering and Biotechnology. 2020;8.
    
    \bibitem{figeys_lab---chip_2000}
    Figeys D, Pinto D.
    \newblock Lab-on-a-{Chip}: {A} {Revolution} in {Biological} and {Medical}
      {Sciences}.
    \newblock Analytical Chemistry. 2000;72(9):330 A--335 A.
    \newblock doi:{10.1021/ac002800y}.
    
    \bibitem{fleischer_analytical_2018}
    Fleischer H, Baumann D, Joshi S, Chu X, Roddelkopf T, Klos M, et~al.
    \newblock Analytical {Measurements} and {Efficient} {Process} {Generation}
      {Using} a {Dual}–{Arm} {Robot} {Equipped} with {Electronic} {Pipettes}.
    \newblock Energies. 2018;11(10):2567.
    \newblock doi:{10.3390/en11102567}.
    
    \bibitem{steffens_versatile_2017}
    Steffens S, Nüßer L, Seiler TB, Ruchter N, Schumann M, Döring R, et~al.
    \newblock A versatile and low-cost open source pipetting robot for automation
      of toxicological and ecotoxicological bioassays.
    \newblock PLOS ONE. 2017;12(6):e0179636.
    \newblock doi:{10.1371/journal.pone.0179636}.
    
    \bibitem{singh_review_2019}
    Singh Y, Raghuwanshi SK, Kumar S.
    \newblock Review on {Liquid}-level {Measurement} and {Level} {Transmitter}
      {Using} {Conventional} and {Optical} {Techniques}.
    \newblock IETE Technical Review. 2019;36(4):329--340.
    \newblock doi:{10.1080/02564602.2018.1471364}.
    
    \bibitem{lincoln_chapter_1998}
    Lincoln CK, Gabridge MG.
    \newblock Chapter 4 {Cell} {Culture} {Contamination}: {Sources},
      {Consequences}, {Prevention}, and {Elimination}.
    \newblock In: Mather JP, Barnes D, editors. Methods in {Cell} {Biology}.
      vol.~57 of Animal {Cell} {Culture} {Methods}. Academic Press; 1998. p.
      49--65.
    \newblock Available from:
      \url{https://www.sciencedirect.com/science/article/pii/S0091679X0861571X}.
    
    \bibitem{jain_total-internal-reflection_2021}
    Jain U, Gauthier A, van~der Meer D.
    \newblock Total-internal-reflection deflectometry for measuring small
      deflections of a fluid surface.
    \newblock Experiments in Fluids. 2021;62(11):235.
    \newblock doi:{10.1007/s00348-021-03328-y}.
    
    \bibitem{litt_visualization_1989}
    Litt GJ. Visualization device; 1989.
    \newblock Available from:
      \url{https://patents.google.com/patent/US4824230A/en}.
    
    \bibitem{opencv_library}
    Bradski G.
    \newblock {The OpenCV Library}.
    \newblock Dr Dobb's Journal of Software Tools. 2000;.
    
    \bibitem{noauthor_rpi-cam-web-interface_nodate}
    {RPi}-{Cam}-{Web}-{Interface} - {eLinux}.org;.
    \newblock Available from: \url{https://elinux.org/RPi-Cam-Web-Interface}.
    
    \bibitem{ly_picroscope_2021}
    Ly VT, Baudin PV, Pansodtee P, Jung EA, Voitiuk K, Rosen YM, et~al.
    \newblock Picroscope: low-cost system for simultaneous longitudinal biological
      imaging.
    \newblock Communications Biology. 2021;4(1):1--11.
    \newblock doi:{10.1038/s42003-021-02779-7}.
    
    \bibitem{baudin_low_2021}
    Baudin PV, Ly VT, Pansodtee P, Jung EA, Currie R, Hoffman R, et~al.
    \newblock Low cost cloud based remote microscopy for biological sciences.
    \newblock Internet of Things. 2021; p. 100454.
    \newblock doi:{10.1016/j.iot.2021.100454}.
    
    \bibitem{seiler_modular_2022}
    Seiler ST, Mantalas GL, Selberg J, Cordero S, Torres-Montoya S, Baudin PV,
      et~al.. Modular automated microfluidic cell culture platform reduces
      glycolytic stress in cerebral cortex organoids; 2022.
    \newblock Available from:
      \url{https://www.biorxiv.org/content/10.1101/2022.07.13.499938v1}.
    
    \end{thebibliography}



\end{document}

