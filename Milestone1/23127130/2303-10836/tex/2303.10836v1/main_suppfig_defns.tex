\def\supfigcell{
    %\fig-supFig1
    \begin{figure*}[h]
    	\includegraphics[width=3 in]{Supp1.png}%
    	\caption{\textbf{Uniaxial stress cell}
    	Uniaxial stress cell for ARPES.
    	\label{fig:supFigCell}}
    \end{figure*}
}



\def\supFigStrain{
    \begin{figure*}[h]
    	\includegraphics[width=6 in]{Supp2.pdf}%
    	\caption{\textbf{Strain measurements via strain gauge}
    	(a) A picture of the strain gauge mounted on fixtures. (b) Strain responses as a function of applied voltage on piezo actuators.
    	\label{fig:supFigStrain}}
    \end{figure*}
}

\def\supFigResistivity{
    \begin{figure*}[h]
        \includegraphics[width=0.8\linewidth]{Supp3.pdf}%
    	\caption{\textbf{Uniaxial stress cell}
    	Uniaxial stress cell for ARPES.
    	\label{fig:supFigResistivity}}
    \end{figure*}
    
}

\def\supFigStrainBS{
    \begin{figure*}[h]
    	\includegraphics[width=6.3 in]{Supp4.pdf}%
    	\caption{\textbf{Strain dependent band structure}
    	(a) Estimated strain based on strain gauge measurement. Band structure cut along the $\Gamma$-$Z$(crystallographic $c$) direction of \hf\ with the applied voltage: (b) inner piezo actuator of 0 V and outer piezo actuators of 0 V, (c) inner piezo actuator of 120 V and outer piezo actuators of -50 V, (d) inner piezo actuator of 0 V and outer piezo actuators of 0 V, (e) inner piezo actuator of 0 V and outer piezo actuators of 80 V, (f) inner piezo actuator of -50 V and outer piezo actuators of 120 V, (g) inner piezo actuator of 0 V and outer piezo actuators of 0 V, and (h) inner piezo actuator of 120 V and outer piezo actuators of 0 V.
    	\label{fig:supFigStrainBS}}
    \end{figure*}
}

\def\supfigisoenergy{
    \begin{figure*} [h]
    	\includegraphics[width=6.3 in]{Supp5.pdf}
    	\caption{\textbf{Constant energy contour plots}
    	(a)-(c) Constant energy contour plots of compressed sample 2 at $E_{F}$, $E_{B}\,=\,250$\,meV, $E_{B}\,=\,500$\,meV, respectively. (d)-(f) Constant energy contour plots from IGF calculations at $E_{F}$, $E_{B}\,=\,100$\,meV, $E_{B}\,=\,300$\,meV, respectively. (g)-(i) Constant energy contour plots of compressed sample 2 at $E_{B}\,=\,900$, $E_{B}\,=\,1$\,eV, $E_{B}\,=\,1.2$\,eV, respectively. (j)-(l) Constant energy contour plots from IGF calculations at $E_{B}\,=\,800$\,meV, $E_{B}\,=\,950$\,meV, $E_{B}\,=\,1.05$\,eV, respectively. (m) Band structure of compressed sample 2 along the $k_y$. The dashed line marks the energy from which the constant energy contour plots are taken. (n) Band structure from IGF calculations along the $k_y$. The dashed line marks the energy from which the constant energy contour plots are taken. Red dashed lines indicate another linear band crossing point.  
    	\label{fig:isoenergy}}
    \end{figure*}
}

\def\supfiglinear{
    \begin{figure*}[h]
    	\includegraphics[width=4 in]{Supp6.pdf}
    	\caption{\textbf{Electronic band dispersion in compressed \hf\ at the crossing points}
    	 (a)-(d) Electronic band dispersion at $k_x$ = -0.16\,\AA$^{-1}$, $k_x$ = -0.28\,\AA$^{-1}$, $k_x$ = -0.29\,\AA$^{-1}$, and $k_x$ = -0.32 \AA$^{-1}$ that are marked in Fig.\ \ref{fig:isoenergy} (c)-(f), respectively. Red dashed lines indicate the Dirac point at \eb\ = 500 meV, 900 meV, 1 eV, and 1.2 eV.
    	\label{fig:linear}}
    \end{figure*}
}

\def\supFIGSTRcal{
    \begin{figure*}[!p]
    	\includegraphics[width=0.85\linewidth]{Supp7.pdf}
    	\caption{\textbf{Structural changes with Te vacancies}
    	Calculated structural changes for HfTe$_{5}$ with the inclusion of Te vacancies on the three unique Te sites in the $Cmcm$ structure.
    	\label{fig:supFIGSTRcal}}
    \end{figure*}
}


\def\supFigArpesFit{
    \begin{figure*} [h]
    	\includegraphics[width=5.5 in]{Supp8.pdf}
    	\caption{\textbf{Detailed analysis of ARPES data}
    	(a) Band structure of unstrained sample 2 along the $\Gamma$ - Y direction. Black dashed lines mark the binning range for the momentum distribution curve (MDC) fitting. Red dashed lines indicate the binning range for the energy distribution curve (EDC) fitting. The magenta rectangular box shows the area where the averaged background is obtained. 
    	(b) Average amplitudes of bands based on the MDC and EDC fitting in the marked range shown in (a). 
    	(c) Normalized background signal over amplitudes of bands. The amplitudes are based on (b). \label{fig:supFigArpesFit}}
    \end{figure*}
}

