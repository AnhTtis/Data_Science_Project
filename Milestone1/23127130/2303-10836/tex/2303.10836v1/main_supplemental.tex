\clearpage\pagenumbering{arabic}
\section{Supplementary Information}

\renewcommand{\thepage}{S\arabic{page}}  
\renewcommand{\thesection}{S\arabic{section}}   
\renewcommand{\thetable}{S\arabic{table}}   
\renewcommand{\thefigure}{S\arabic{figure}}
%\renewcommand{\figurename}{F}
\setcounter{figure}{0}
\setcounter{table}{0}

%%%%%%%%%%%%%%%%%%%%%%%%%%%%%%%%%%%%%%%%%%%%%%%%% STRAIN CELL
\subsection{Uniaxial stress cell}

The uniaxial stress cell we used for this experiment is based on three piezoelectric stacks, similar to Ref. \cite{hicks2014}. As we have eight available electrical channels on our system, we used four to control the piezoelectric stacks and the rest of the four channels for strain gauge measurements. We attached a Cu thermal braiding near the sample in order to ground the sample and achieve better temperature control. In addition, we put a Cu shielding on top of the cell to shield the field from the piezoelectric stacks, as shown in Fig.\,\ref{fig:supFigCell} on the right side.
\supfigcell

%%%%%%%%%%%%%%%%%%%%%%%%%%%%%%%%%%%%%%%%%%%%%%%%% STRAIN MEASUREMENTS
\newpage\subsection{Strain measurements}
\supFigStrain
The strain was measured using a strain gauge (C5K-XX-S5198-350/39F, Micro-Measurements.) mounted with the Stycast epoxy (2850FT) underneath. We first drive the inner piezo actuator from 0 V to 120 V, followed by -50 V, back to 120 V, and ended up at 0 V. While applying voltage on the inner piezo actuator; the outer piezo actuators were connected in a short circuit. At the same time, we measured changes in the resistance of the strain gauge. With the given gauge factor of 1.84 from the manufacturer, the strain was calculated. 
The results are shown in Fig.\ \ref{fig:supFigStrain} (b) in the black line. A clear hysteresis was detected with the maximum compressive strain of $\sim\,-0.42\,\%$ at 120 V and maximum tensile strain of $\sim\,0.27\,\%$ at -50 V. We then put the inner piezo actuator in a short circuit and derive the outer piezo actuators with the same applied voltage sequence. The red line in Fig.\ \ref{fig:supFigStrain} (b) shows a similar hysteresis loop but the opposite sign compared to the hysteresis loop from the inner piezo actuator. The maximum compressive strain of $\sim\,-0.3\,\%$ at -50 V and the maximum tensile strain of $\sim\,-0.39\,\%$ at 120 V were observed.   

%%%%%%%%%%%%%%%%%%%%%%%%%%%%%%%%%%%%%%%%%%%%%%%%% DEFECT DENSITY
\newpage\subsection{Estimate of Defect Density}
For the Te-poor compound \hfd, Lv et al. \cite{Lv2018Tu} have reported a linear relationship between the Te deficit $\delta$ with the temperature of the maximum longitudinal resistivity $\rho_{xx}$, which is plotted in Fig.\ \ref{fig:supFigResistivity} (red dots).  Since this temperature is very similar to $T_{max}$ as defined in our main text, we have used their transport measurements to derive estimates of $\delta$ for our samples (1,2), which are $\delta$ = (.066, .022), respectively.
\supFigResistivity

%%%%%%%%%%%%%%%%%%%%%%%%%%%%%%%%%%%%%%%%%%%%%%%%% STRAINED BS
\newpage\subsection{Strain dependent band structure}
Figure \ref{fig:supFigStrainBS} presents band structure changes with full-cycle piezo actuator movements. (0\,V - compress - tensile - compress) It demonstrates not only apparent changes in band structure with the applied uniaxial stress but also reproducible results.  
\supFigStrainBS


%%%%%%%%%%%%%%%%%%%%%%%%%%%%%%%%%%%%%%%%%%%%%%%%% ISO ENERGY PLOTS
\newpage\subsection{Comparison between ARPES and IGF calculations}
Figures \ref{fig:isoenergy} (a)-(c) and (g)-(i) show constant energy contour plots of \hf\ under compressive strain (sample 2). Corresponding IGF calculations are also plotted in Figs.\ \ref{fig:isoenergy} (d)-(f) and (j)-(l). The binding energies of each plot, Figs.\ \ref{fig:isoenergy} (a)-(l), are marked in Figs.\ \ref{fig:isoenergy} (m) and (n). Comparing ARPES and IGF calculations, we find that sample 2 with compressive strain (or unstrained sample 1) is very close to but not exactly the same as the equilibrium IGF calculation results. First of all, the chemical potential is shifted about 150\,meV downward in ARPES data compared to the IGF calculations. More specifically, the Fermi surface from ARPES measurement is composed of dots which are the residual intensities from the top of the hole bands below $E_{F}$ (Fig.\ \ref{fig:isoenergy} (a)). On the other hand, $E_{F}$ crosses the hole bands in the IGF calculation. As a result, the astroid shape of the Fermi surface is observed in the IGF calculations (Fig.\ \ref{fig:isoenergy} (d)). The astroids become more clear at higher binding energies in both ARPES and IGF calculation results (Fig.\ \ref{fig:isoenergy} (b), (c), and (d)-(f)). ARPES results at the binding energies of 250\,meV and 500\,meV are very similar to the IGF calculation results at 100\,meV and 300\,meV, respectively. This means 150\,meV - 200\,meV band shifts in ARPES data. Secondly, the spacing between two surface bands, $\alpha$, and $\beta$, gets larger with tensile strain in the IGF calculations. (See Fig.\,\ref{fig:band} (h) and (i)) In fact, the spacing between the surface bands in compressed sample 2 and unstrained sample 1 is slightly larger than that of the IGF calculation at equilibrium, which means the samples at those conditions are slightly stretched. 
\supfigisoenergy

%%%%%%%%%%%%%%%%%%%%%%%%%%%%%%%%%%%%%%%%%%%%%%%%% LINEAR BAND CROSSINGS OFF-AXIS
\newpage\subsection{Linear band crossings away from high symmetry points}
The astroids shown in Fig.\ \ref{fig:isoenergy} start to cross at \eb\,=\,500\,meV. Crossing points are marked with red dashed lines in Figs.\ \ref{fig:isoenergy}(c), and (g)-(i). With larger binding energy, the points move outwards in the $k_{x}$ direction. Interestingly, the crossing points of the steroids marked in Fig.\ \ref{fig:isoenergy} with red dashed lines correspond to linear crossing in the band structure as shown in Fig.\ \ref{fig:linear}.  
\supfiglinear


%%%%%%%%%%%%%%%%%%%%%%%%%%%%%%%%%%%%%%%%%%%%%%%%% BOND LENGTHS
\newpage\subsection{Effect of vacancies and strain on bond lengths}



The structural changes are quantified by comparing the Hf-Hf nearest-neighbor distances in the A, B, and C directions as indicated on the crystal structure in Fig. \ref{fig:theory-defects}(a) with the values for the stoichiometric case. For each calculation, one Te vacancy was included in a $2\times 2\times 2$ super cell comprising 16 formula units of \hf\, resulting in a 1.25\% defect density.

\begin{table}[h!]
%\captionsetup{justification=centering}
\caption{\label{tab:bondlengths} Bond lengths in \AA\, as shown in Fig. \ref{fig:theory-defects}(a), for structures with different Te vacancies and strain.} 
\begin{tabular}{l*{3}{>{\centering\arraybackslash}p{4em}}}
\toprule
Structure       & A    & B    & C    \\
\midrule
2\% Compressive & 3.876 & 7.547 & 7.337 \\
\myrowcolour
Equilibrium     & 3.953 & 7.545 & 7.327 \\
2\% Tensile     & 4.034 & 7.554 & 7.311 \\
\myrowcolour
Te-1 Vacancy    & 3.927 & 7.555 & 7.299 \\
Te-2 Vacancy    & 3.956 & 7.568 & 7.322 \\
\myrowcolour
Te-3 Vacancy    & 3.958 & 7.561 & 7.302 \\
\bottomrule
\end{tabular}
\end{table}

\supFIGSTRcal

\newpage

\subsection{Detailed analysis of ARPES data at different strains}

Fig.\ \ref{fig:supFigArpesFit} (b) and (c) indicate a decrease in the amplitude of the bands and an increase of background noise with tensile strain in HfTe$_{5}$.
\supFigArpesFit

\newpage 