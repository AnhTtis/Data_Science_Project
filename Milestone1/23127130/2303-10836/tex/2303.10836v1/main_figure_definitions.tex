\def\figone{
    \begin{figure} [h]
    	\includegraphics[width=3 in]{Figure1.pdf}%
    	\caption{\textbf{Electrical transport.} 
    	(a) Temperature-dependent resistance ratio (left and bottom axis) and MR (right and top axis) of \hf\ sample 1 that is post-annealed at 380~$^{o}$C. 
        (b) Temperature-dependent resistance ratio (left and bottom axis) and MR (right and top axis) of \hf\ sample 2 that is post-annealed at 250~$^{o}$C.
    	\label{fig:Fig1_1}}
    \end{figure}
}

\def\figtwo-theory-defects{
    \begin{figure*} [h]
    \includegraphics[width=6in]{Figure2.pdf}
        \caption{\textbf{Effect of Te vacancies and strain on the crystal structure and electronic structure of \hf} 
    	(a) The conventional cell of \hf, depicting the three unique Te sites and Hf-Hf bonds. (b) The primitive cell of \hf. The primitive cell has half the volume and number of atoms of the conventional cell, and was used for all bulk DFT calculations. 
    	(c) The Brillouin zone (BZ) of the primitive cell, along with the BZ of the $(1\bar{1}0)$ surface. $b_i$ are the reciprocal lattice basis vectors of the bulk BZ, corresponding to $a_i$ in (b).
    	(d)-(f) Calculated electronic band structures of \hf\ under uniaxial compressive strain, at equilibrium, and under uniaxial tensile strain, as defined in the main text. The two topological gaps of interest are shaded and labeled as (I) and (II). The Fermi level in each is set to 0 eV and marked by a dashed line. 
        \label{fig:theory-defects}}
    \end{figure*}
}

\def\figthree-band{
    \begin{figure*} [h]
    	\includegraphics[width=6.5 in]{Figure3.pdf} 
    	\caption{\textbf{Non-trivial topological states at $\bar{\Gamma}$}
    	(a) Electronic band structure along the $k_{x}$ of sample 1 at ambient condition. 
    	(b-c) Slab band structures along the $\bar{\Gamma} - \bar{X}$ direction for equilibrium, and tensile strain, respectively.
    	(d)-(f) Calculated surface band structure along the $\bar{\Gamma} - \bar{Y}$ direction of pristine \hf\ with compressive strain, at equilibrium, and with tensile strain, respectively. The gray shading represents bulk bands projected onto the $(1\bar{1}0)$ surface, while the colored lines represent the slab bands. The color and line width of the slab bands are given by Eq. \eqref{eq:weighted_projections} in the supplementary information, with thin black lines representing bulk character and thick red lines representing surface character. 
    	\label{fig:strain-phase}}
\end{figure*}
}


\def\figfour-strain-phase{
    \begin{figure*} [h]
    	\includegraphics[width=6 in]{Figure4.pdf}%
    	\caption{\textbf{Topological phase diagram and electronic band structure along $k_{y}$ with different Te vacancies and strain}
            (a) Schematic phase diagram, indicating the location of ARPES band structure measurements in the space of defect density \textit{vs.} strain. Blue markers correspond to Sample 1 and (b,c); Red markers correspond to Sample 2 and (d-f). %\omarcomment{I think we should remove or at least center the background in (a).}
    	(b-f) ARPES band structures acquired for positions indicated in (a).
    	\label{fig:band}}
    \end{figure*}
}

\def\bandref #1{Fig.\ \ref{fig:band}(#1)}
\def\bandrefs #1{Figs.\ \ref{fig:band}(#1)}

