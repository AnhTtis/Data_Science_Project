\documentclass[pra,twocolumn,amsmath,amssymb,superscriptaddress]{revtex4-1}
\usepackage{epsfig,amsmath}
\usepackage{subfigure}
\usepackage{graphicx}% Include figure files
\usepackage{dcolumn}% Align table columns on decimal point
\usepackage{stmaryrd}
\usepackage{mathrsfs}
\usepackage{pifont}
\usepackage{amsthm}
\usepackage{amssymb}
\usepackage{bm}
\usepackage{latexsym}
\usepackage[colorlinks=true,linkcolor=blue,citecolor=blue]{hyperref}
\usepackage{color}
\usepackage{epstopdf}
\usepackage{booktabs}

\newcommand{\red}[1]{\textcolor{red}{#1}}
\newcommand{\non}{\nonumber}
\newcommand{\blue}[1]{\textcolor{blue}{#1}}

\begin{document}

\title{Magnon blockade in magnon-qubit systems}

\author{Zhu-yao Jin}
\affiliation{School of Physics, Zhejiang University, Hangzhou 310027, Zhejiang, China}

\author{Jun Jing}
\email{Email address: jingjun@zju.edu.cn}
\affiliation{School of Physics, Zhejiang University, Hangzhou 310027, Zhejiang, China}

\date{\today}

\begin{abstract}
We consider a hybrid system that is established by the direct interaction between a magnon and a superconducting transmon qubit. Through weakly driving the magnon and probing the qubit, a high-degree magnon blockade can be realized, which paves a revenue toward quantum manipulation at the level of a single magnon. Our magnon-blockade proposal is optimized when the magnon-qubit transversal coupling strength is equivalent to the detuning of the qubit and the probing field or that of the magnon and the driving field. Under this condition, the analytical expression of the equal-time second-order correlation function $g^{(2)}(0)$ can be minimized when the probing intensity is about three times of the driving intensity. Moreover, the degree of the magnon blockade could be further enhanced by choosing proper driving intensity and system decay rate. With experimental-relevant parameters, the correlation function attains $g^{(2)}(0)\sim10^{-7}$, about two orders lower than that for the photon blockade in cavity optomechanical systems. Also we discuss the effects on $g^{(2)}(0)$ from the thermal noise on magnon and qubit and the extra longitudinal interaction between the two components. Our optimized conditions for blockade still hold in both nonideal situations.
\end{abstract}

\maketitle

\section{Introduction}

Strong coupling among distinct quantum systems allows for efficient quantum state engineering, which plays a crucial role in quantum information processing~\cite{Wallquist2009Hybrid,Kimble2008internet} and quantum network~\cite{Kimble2008internet}. Beyond the conventional paradigm, e.g., the strong coupling to the optical modes by large electric dipole moments of the system, it was found that the strong light-matter interaction can also be established through magnetic dipole moments~\cite{Imamo2009Cavity}. Extensive attentions arise in the last decade to multiple systems of the ferromagnetic materials with high spin densities, including ultracold atomic clouds~\cite{Verd2009Strong}, molecule nanomagnets in cavity~\cite{Eddins2014collective}, nitrogen vacancy centers in diamond~\cite{Kubo2010Strong,Ams2011Cavity,Marcos2010coupling,Ranjan2013probing}, and ion doped crystals~\cite{Schuster2010high,Probst2013Anisotropic,Tkal2014strong}. Hybrid magnon systems of yttrium iron garnet (YIG) spheres~\cite{Tabuchi2014Hybridizing,Huebl2013High,Zhang2014Strongly,Goryachev2014High,
Bai2015Spin,Soykal2010Strong,Bourhill2016Ultrahigh,Zare2015Magnetic,Viola2016Coupled,
Zhang2016Opto,Osada2016Cavity,Haigh2016Triple} demonstrate distinguishable advantages by virtue of a high order of magnitude of spin density ($\rho_s=4.22\times10^{27}$ m$^{-3}$), an extremely low damping rate of magnons (the quanta of collective spin excitations), and the nonlinear amplification and control of magnons via the interaction between spin excitations. Recent experiments~\cite{Yutaka2015Coherent,Dany2017Resolving,Wolski2020Dissipation,Dany2020Entanglement,
Kounalakis2022Analog} suggest that the magnon-qubit system could be a promising information carrier since the indirect and direct interactions between a YIG sphere and a superconducting qubit have been built in a microwave cavity.

Many outstanding quantum properties of hybrid systems of magnons, such as polariton bistability~\cite{wang2018bistability}, distant spin currents~\cite{Bai2017Cavity}, and long-range magnon-magnon entanglement~\cite{Ren2022Long}, have been observed in a mesoscopic scale. Owing to these recent advances, there is growing interest in understanding the properties in more microscopic scales, including cooling magnons by a monochromatic laser source~\cite{Sharma2018Optical}, storing information in the magnon dark modes~\cite{Zhang2015magnon}, entangling magnon-magnon system in a cavity~\cite{Yuan2020enhancement,Azimi2021Magnon}, and generating the magnon-photon-phonon entanglement and magnon-photon Bell state~\cite{Li2018Magnon,Yuan2020Steady}. Not only do they relate to the foundations of quantum theory, but they are also the doorway to the quantum realm of magnon systems. In pursuit of the quantum manipulation at the level of a single magnon, a superposed state of a single magnon and vacuum has been created via the Autler-Townes effect~\cite{Xu2022Quantum}. While in general, a certain blockade is required to suppress the population on quantum states with more than one quantum. A group of particularly important examples are those in cavity optomechanics~\cite{Zheng2011Cavity,Huang2013photon,Liao2010correlated,
Ghosh2019Dynamical,Lu2015Squeezed,Peyronel2012quantum,Reinhard2012strongly,
Reinhard2012strongly,Liew2010Single,Bamba2011Origin,Snijders2018Observation,
Vaneph2018Observation,Hartmann2007strong,Chang2014Quantum}. Photon blockade prevents a second photon entering the cavity mode~\cite{Imamo1997strongly,Hartmann2007strong,Chang2014Quantum}, leading to antibunching in photon correlation measurements. Similarly, the manipulation over a single magnon level could be supported by the magnon blockade that occurs under the anharmonicity in the energy spectrum.

In this work, we propose to create and enhance the magnon blockade in a hybrid system, where a YIG sphere directly couples to a superconducting transmon qubit through the magnetic stray field~\cite{Kounalakis2022Analog}. This system features the transversal (exchange) and longitudinal interactions between magnon and qubit, which are tunable by the externally flux $\Phi_b$ on qubit and the magnon-qubit distance. A general goal in quantum blockade is to minimize the expected value of the equal-time second-order correlation function $g^{(2)}(0)$, that is commonly used to characterize and distinguish the bunching and antibunching of the interested quanta. To enhance the blockade degree of the magnon mode, a probing field and a driving field are applied to qubit and magnon, respectively. $g^{(2)}(0)$ is obtained by solving the master equation for the composite system under decoherence~\cite{Carmichael1999statistical}. In the steady-state regime, the nonclassical antibunching effect can be observed when $g^{(2)}(0)<1$. This is a pure quantum scenario where the magnons have the tendency to be detected further away in space (time) than those of a classical state. Our magnon-blockade proposal is verified to be optimized under the particular conditions that the magnon-qubit transversal coupling strength and the detuning of the qubit (magnon) and the probing (driving) field are the same in magnitude and the intensity of the probing field is three times that of the driving field. And the degree of magnon blockade is further enhanced to $g^{(2)}(0)\sim10^{-7}$ with proper driving intensity and system decay rate, which is about two orders lower than the minimum value of the $g^{(2)}$ function for the photon blockade to our best knowledge~\cite{Wang2020Photon}. We also find that the optimal conditions for the high-degree magnon blockade are retained under the influences of the thermal noise and the extra longitudinal interaction between magnon and qubit.

The rest of this work is structured as follows. In Sec.~\ref{modelandHam}, we introduce the hybrid magnon-qubit system under external driving and probing after a brief review of the physics underlying quantum blockade. In Sec.~\ref{OptimalDrivingStrength}, we investigate the optimal conditions for magnon blockade with respect to both dynamics and steady states under the experimental-relevant regimes of parameters about Rabi frequencies of driving and probing fields. The numerical results are confirmed by an analytical model in Appendix~\ref{Analytic model}. An optimized driving intensity to minimize $g^{(2)}(0)$ of the steady state can be further located by choosing proper transversal coupling strength and the system decay rate. In Sec.~\ref{discussion}, we discuss the reliability of the optimal conditions under two nonideal situations. We conclude the whole work in Sec.~\ref{Conclusion}.

\section{model and Hamiltonian}\label{modelandHam}

A typical photon-blockade effect in cavity quantum electrodynamics (QED) relies on the strong energy-spectrum anharmonicity. Taking the classic Jaynes-Cummings (JC) model~\cite{Yutaka2015Coherent,Jaynes1963comparison} as an example, the strong coupling between the cavity and the two-level system induces a large vacuum Rabi splitting~\cite{Imamo1997strongly,Hartmann2007strong,Chang2014Quantum}, which gives rise to anharmonic ladders of dressed states described by $|n,\pm\rangle$ with eigenvalues $E_{n,\pm}=n\omega\pm\sqrt{n}J$. Here $\omega$ and $J$ are respectively the bare frequencies of photon and qubit and their coupling strength, $n$ denotes the excitation number, and $+$ ($-$) represents the higher (lower) branch. In this nonlinear energy-level spectrum, the transitions between $|0,g\rangle$ and $|1,\pm\rangle$ are allowed by the resonant driving while the transitions $|1,\pm\rangle\rightarrow|2,\pm\rangle$ are suppressed due to the energy mismatch, i.e., $2(\omega\pm J)\neq2\omega\pm\sqrt{2}J$.

Given the similarity in the anharmonic spectrum, our study is conducted on a hybrid system, in which a YIG particle is directly coupled to a superconducting transmon qubit~\cite{Kounalakis2022Analog}. The original Hamiltonian reads ($\hbar\equiv1$)
\begin{equation}\label{Ham_initial}
H=\omega_q\sigma_+\sigma_-+\omega_mm^\dagger m+J(m\sigma_++m^\dagger\sigma_-),
\end{equation}
where $\sigma_+$ ($\sigma_-$) and $m^\dagger$ ($m$) are the creation (annihilation) operators of the qubit and magnon mode with frequencies $\omega_q$ and $\omega_m$, respectively, and $J$ is the transversal (exchange) coupling strength between them. The qubit in our system is formed by a superconducting quantum interference device (SQUID) loop and a capacitor. The magnetic moments of the YIG sphere emit a stray field that induces a flux through the SQUID loop and then establish direct magnon-qubit interaction, including the transversal and longitudinal parts. The transversal interaction, akin to the JC model~\cite{Yutaka2015Coherent,Jaynes1963comparison}, is the key element to realize the magnon blockade. The longitudinal interaction, which is analogous to the radiation-pressure in optomechanics~\cite{Shevchuk2017Strong,Rodrigues2019coupling,Kounalakis2020Flux}, can be omitted when fixing the externally applied flux on qubit as $\Phi_b/\Phi_0\approx0.5$ ($\Phi_0$ is the magnetic flux quantum). The effect of the longitudinal interaction on magnon blockade will be discussed in Sec.~\ref{secLongitudinal}. Under the flux on qubit $\Phi_b/\Phi_0\approx0.5$, the coupling strength $J/2\pi$ can be tuned as strong as about $40$ MHz~\cite{Kounalakis2022Analog} by modulating the magnon position. The magnon frequency $\omega_m/2\pi$ is about $1\sim10$ GHz, and its decay rate $\kappa_m=\omega_m\alpha_G$ is about $0.1\sim1$ MHz in terms of the Gilbert damping constant $\alpha_G$. The qubit frequency $\omega_q/2\pi$ is tunable from $1$ to $10$ GHz by modulating the externally flux and the SQUID asymmetry $a$~\cite{Kounalakis2022Analog}, which is about $1.5$ GHz under $\Phi_b/\Phi_0\approx0.5$ and $a=0.1$.

Aiming to manipulate and enhance the degree of magnon blockade, we apply a probing field and a driving field on qubit and magnon, respectively. The two fields are assumed to have the same frequency $\omega$, which is in the order of GHz~\cite{wang2018bistability}. Then the total Hamiltonian reads,
\begin{equation}\label{Ham_tot}
\begin{aligned}
H_{\rm tot}&=\omega_q\sigma_+\sigma_-+\omega_mm^\dagger m+J(m\sigma_++m^\dagger\sigma_-)\\
&+\Omega_m(m^\dagger e^{-i\omega t}+m e^{i\omega t})+ \Omega_q(\sigma_+e^{-i\omega t}+\sigma_-e^{i\omega t}),
\end{aligned}
\end{equation}
where $\Omega_q=k\sqrt{P_d}$ with $k=103$ MHz/mW$^{1/2}$~\cite{wang2019simulation} and the field power $P_d$ about $350$ mW~\cite{wang2018bistability} and $\Omega_m=\sqrt{2S}\Omega_s$ with $S$ the total spin number of the macrospin and $\Omega_s$ the coupling strength of the drive field with the macrospin~\cite{wang2016magnon}. The time dependence of the Hamiltonian can be removed in the rotating frame with respect to $H'=\omega(m^\dagger m+\sigma_+\sigma_-)$. Then the effective Hamiltonian of this hybrid system becomes
\begin{equation}\label{Ham_eff}
\begin{aligned}
H_{\rm eff}&=(\Delta_+-\Delta_-)\sigma_+\sigma_-+(\Delta_++\Delta_-)m^\dagger m\\
&+J(m\sigma_++m^\dagger\sigma_-)+\Omega_m(m^\dagger+m)+\Omega_q(\sigma_++\sigma_-),
\end{aligned}
\end{equation}
where $\Delta_{\pm}\equiv(\Delta_m\pm\Delta_q)/2$ and $\Delta_q\equiv\omega_q-\omega$ ($\Delta_m\equiv\omega_m-\omega$) is the detuning of the qubit (magnon) and the probing (driving) field.

\section{Magnon blockade and optimal conditions}\label{OptimalDrivingStrength}

The density matrix $\rho_m$ of the magnon mode is obtained by partial tracing the full density matrix $\rho$ over the degrees of freedom of qubit, $\rho_m={\rm Tr}_q[\rho]$. Using a master-equation approach to take decoherence into account~\cite{Carmichael1999statistical}, we are able to study the dynamics as well as the steady state of the full density matrix. The master equation reads,
\begin{equation}\label{masterequation}
\frac{\partial}{\partial t}\rho=-i[H_{\rm eff},\rho]+\frac{\kappa_m}{2}\mathcal{L}_m[\rho]+\frac{\kappa_q}{2}\mathcal{L}_{\sigma_-}[\rho],
\end{equation}
where the dissipator is defined as the Lindblad superoperators $\mathcal{L}_o[\rho]=2o\rho o^\dagger-o^\dagger o\rho-\rho o^\dagger o$~\cite{Scully1997quantum} with $o=m, \sigma_-$ indicating respectively the decay channel for the magnon mode and the qubit. The decay rates of magnon and qubit are $\kappa_m$ and $\kappa_q$, respectively, which are set as  $\kappa_m=\kappa_q=\kappa$ for simplicity. The degree of the magnon blockade is characterized by the equal-time second-order correlation function~\cite{Carmichael1999statistical,Eleuch2008Photon}
\begin{equation}\label{g2define}
g^{(2)}(0)\equiv g^{(2)}(t,t)=\frac{\langle m^\dagger m^\dagger mm\rangle}{\langle m^\dagger m\rangle^2},
\end{equation}
where the expectation values are calculated by $\rho_m$ after solving Eq.~(\ref{masterequation}) with the effective Hamiltonian in Eq.~(\ref{Ham_eff}). The correlation function $g^{(2)}(0)=1$ indicates Poissonian distribution of the magnons, $g^{(2)}(0)>1$ indicates super-Poissonian corresponding to a magnon bunching effect, and $0<g^{(2)}(0)<1$ indicates sub-Poissonian corresponding to a magnon antibunching effect. The asymptotic limit $g^{(2)}(0)\rightarrow0$ describes the magnon blockade.

\begin{figure}[htbp]
\centering
\includegraphics[width=0.95\linewidth]{dynamical.eps}
\includegraphics[width=0.95\linewidth]{Delta.eps}
\caption{(a) Dynamics of the logarithmic function of the equal-time second-order correlation function $\log_{10}g^{(2)}(t,t)$ under $\Delta_-=0$ with various $\Delta_+$. (b) Steady-state $\log_{10}g^{(2)}(0)$ as a function of the ratio $\Delta_-/\Delta_+$. $J/2\pi=20$ MHz, $\Omega_m/2\pi=0.1$ MHz ($0.037$ $\mu$W), $\Omega_q=\Omega_m$ and $\kappa/2\pi=1$ MHz. }\label{dynamical}
\end{figure}

Due to Eq.~(\ref{Ham_eff}), the logarithmic function of the equal-time second-order correlation function $\log_{10}g^{(2)}(0)$ relies on the average $\Delta_+$ of the two detunings $\Delta_m$ and $\Delta_q$ and their discrepancy $\Delta_-$. To build intuition to locate the optimized parameters for the antibunching effect, we first demonstrate the behaviors of $\log_{10}g^{(2)}(0)$ with respect to the temporal evolution and the steady state in Fig.~\ref{dynamical}(a) and \ref{dynamical}(b), respectively. In Fig.~\ref{dynamical}(a) for $\Delta_m=\Delta_q$, it is shown that all of the $g^{(2)}$ functions approach their asymptotic values when $\kappa t\ge20$. Antibunching effect emerges when $\Delta_+/J$ is close to unit. In particular, when $\Delta_+/J=1$, the $g^{(2)}$ function finds its minimum value as $g^{(2)}(t,t)\rightarrow10^{-2}$. It motivates us to explore the optimal value of the discrepancy $\Delta_-$ under fixed $\Delta_+$. In Fig.~\ref{dynamical}(b) for the steady value, it is found that $g^{(2)}$ functions take the minimum value when $\Delta_-/\Delta_+=0$. A nonzero $\Delta_-$ induces a negative effect on blockade. Then we set $\Delta_-=0$, i.e., $\omega_q=\omega_m$, in the following investigation.

\begin{figure}[htbp]
\centering
\includegraphics[width=0.95\linewidth]{conditionDelta.eps}
\caption{Steady-state $\log_{10}g^{(2)}(0)$ versus the ratio $\Delta_+/J$ of the detuning average to the transversal coupling strength. $J/2\pi=20$ MHz, $\Omega_m/2\pi=0.1$ MHz, and $\kappa/2\pi=1$ MHz.}\label{conditionDelta}
\end{figure}

To further confirm the conditions under which the antibunching effect occurs, we show in Fig.~\ref{conditionDelta} the dependence of $\log_{10}g^{(2)}(0)$ on the ratio of the detuning average and the magnon-qubit transversal coupling strength $\Delta_+/J$ under various ratios of field intensities (Rabi frequencies) $\Omega_q$ and $\Omega_m$. The intensities of both probing and driving fields are experimentally tunable through modulating the field power~\cite{wang2018bistability}. In Fig.~\ref{conditionDelta}, the quantum-classical boundary $g^{(2)}(0)=1$ separates the bunching effect [$g^{(2)}(0)>1$] and the antibunching effect [$g^{(2)}(0)<1$]~\cite{Rabl2011photon}. For all ratios of $\Omega_q/\Omega_m$, strong bunching effect occurs around $\Delta_+/J=\pm0.7$, where $g^{(2)}(0)\sim10^{2}$ and strong antibunching effect occurs around $\Delta_+/J=\pm1$. When $\Delta_+/J=-1$, one can find that the expecting value of the $g^{(2)}$ function is not sensitive to $\Omega_q/\Omega_m$. In contrast, more higher degree of magnon blockade favors the condition $\Delta_+/J=1$ and can be modulated by tuning the ratio of the probing-field and driving-field strength. In particular, we have $g^{(2)}(0)=10^{-1.99}$, $g^{(2)}(0)=10^{-2.58}$, $g^{(2)}(0)=10^{-2.99}$, and $g^{(2)}(0)=10^{-2.21}$ under $\Omega_q/\Omega_m=1$, $\Omega_q/\Omega_m=2$, $\Omega_q/\Omega_m=5$, $\Omega_q/\Omega_m=10$, respectively. We are therefore guided to investigate an optimized condition about $\Omega_q/\Omega_m$ to further enhance the degree of the magnon blockade.

\begin{figure}[htbp]
\centering
\includegraphics[width=0.95\linewidth]{conditionDrive.eps}
\caption{Steady-state $\log_{10}g^{(2)}{(0)}$ versus $\Omega_q/\Omega_m$ under $\Delta_+/J=1$. The other parameters are set as $\Omega_m/2\pi=0.1$ MHz and $\kappa/2\pi=1$ MHz.}\label{conditionDrive}
\end{figure}

In Fig.~\ref{conditionDrive} with various transversal coupling strengths, we show the dependence of $\log_{10}g^{(2)}(0)$ on the ratio of probing intensity and driving intensity under $\Delta_+=J$. Note for all $J$'s, the $g^{(2)}$ function takes local minimum values under the conditions of $\Omega_q/\Omega_m=1$ and $\Omega_q/\Omega_m=3$. One can locates the global minimum values at $\Omega_q/\Omega_m=3$, which monotonically increase with increasing transversal interaction strength. $J$ is upper-bounded in current experiments (about $J/2\pi\sim40$ MHz~\cite{Kounalakis2022Analog}), nevertheless, the degree of magnon blockade is enhanced from $g^{(2)}(0)=10^{-4.44}$ under $J/2\pi=14$ MHz to $g^{(2)}(0)=10^{-6.03}$ under $J/2\pi=35$ MHz.

The ratio of the Rabi frequency of the probing field $\Omega_q$ and that of the driving field $\Omega_m$ in locating the minimum $g^{(2)}$ function could be self-consistently estimated by an analytical model~\cite{Kar2013Single} for an approximated expression of the $g^{(2)}$ function. In the weak-driving limit, the state of the composite system could be truncated to a subspace with a few excitations. It can be approximately written as $|\psi\rangle=C_{g0}|g0\rangle+C_{g1}|g1\rangle+C_{e0}|e0\rangle+C_{e1}|e1\rangle+C_{g2}|g2\rangle$, where $C_{g(e),n}$ are the probability amplitudes for the state $|g(e),n\rangle$ with $g$ ($e$) the ground (excited) state of qubit and $n$ the excitation number of magnon. And the system dynamics is determined by a non-Hermitian Hamiltonian $H_{\rm non}=H_{\rm eff}-i\kappa(\sigma_+\sigma_-+m^\dagger m)/2$, where $H_{\rm eff}$ is the effective Hamiltonian in Eq.~(\ref{Ham_eff}) and $\kappa$ is the system decay rate in Eq.~(\ref{masterequation}). According to the Schr\"odinger equation $i\partial|\psi\rangle/\partial t=H_{\rm non}|\psi\rangle$, we have
\begin{equation}\label{differential}
\begin{aligned}
i\dot{C}_{g0}&=0,\\
i\dot{C}_{e0}&=JC_{g1}+\Omega_qC_{g0}+(\Delta_+-i\kappa/2)C_{e0},\\
i\dot{C}_{g1}&=JC_{e0}+\Omega_mC_{g0}+(\Delta_+-i\kappa/2)C_{g1},\\
i\dot{C}_{e1}&=\sqrt{2}JC_{g2}+\Omega_qC_{g1}+\Omega_mC_{e0}+2(\Delta_+-i\kappa/2)C_{e1},\\
i\dot{C}_{g2}&=\sqrt{2}JC_{e1}+\sqrt{2}\Omega_mC_{g1}+2(\Delta_+-i\kappa/2)C_{g2},
\end{aligned}
\end{equation}
which yield straightforwardly the steady-state amplitudes (see Appendix~\ref{Analytic model} for details) under the condition of $\Delta_+=J$. The $g^{(2)}$ function for the steady state can then be expressed by
\begin{equation}\label{g2analytical}
g^{(2)}(0)=\frac{2|C_{g2}|^2}{\left(|C_{g1}|^2+|C_{e1}|^2+2|C_{g2}|^2\right)^2}\approx\frac{2|C_{g2}|^2}{|C_{g1}|^4},
\end{equation}
where the second equivalence is approximately valid under $|C_{g0}|\gg|C_{g1}|, |C_{e0}|\gg|C_{e1}|, |C_{g2}|$. This result could be verified by the numerical simulation (See Fig.~\ref{comparison} in Appendix~\ref{Analytic model}). Through analyzing the extremum value of Eq.~(\ref{g2analytical}), it is confirmed that the strongest magnon antibunching occurs when $\Omega_q\approx3\Omega_m$, which is stable and independent of dimensional parameters. In addition, it is also qualitatively derived that when $\Omega_q\approx\Omega_m$, the $g^{(2)}$ function might take a local minimum value as implied in Fig.~\ref{conditionDrive}, which is however unstable due to the dependence of driving intensities.

\begin{figure}[htbp]
\centering
\includegraphics[width=0.95\linewidth]{optdecayJ.eps}
\includegraphics[width=0.95\linewidth]{optdecayOm.eps}
\caption{Steady-state $\log_{10}g^{(2)}{(0)}$ versus (a) the transversal coupling strength $J$ or (b) the driving field strength $\Omega_m$ and the decay rate $\kappa$ under $\Delta_+=J$ and $\Omega_q=3\Omega_m$. In (a) $\Omega_m/2\pi=0.1$ MHz and in (b) $J/2\pi=35$ MHz.}\label{optdecayOm}
\end{figure}

It is interesting to justify the introduction about the probing and driving fields into our model. The absence of these two fields is found to yield a vanishing equal-time second-order correlation function. Under the master equation~(\ref{masterequation}), the steady state of the composite system would be the ground state determined by the JC Hamiltonian and the dissipators for both magnon and qubit. Also it is verified that only under either driving field or probing field, the degree of the magnon blockade is much lower than that under both fields~\cite{Li2021strong}.



\begin{figure}[htbp]
\centering
\includegraphics[width=0.95\linewidth]{fixJOm.eps}
\includegraphics[width=0.95\linewidth]{fixdecay.eps}
\caption{Steady-state $\log_{10}g^{(2)}{(0)}$ as a function of (a) the driving field strength $\Omega_m$ and (b) the decay rate $\kappa$. In (a) $\kappa/2\pi=0.5$ MHz and in (b) $J/2\pi=35$ MHz. It is still under the optimized conditions $\Delta_+=J$ and $\Omega_q=3\Omega_m$.}\label{fixdecay}
\end{figure}

Nevertheless it seems important to explicitly show that both probing and driving fields are indispensable to enhance magnon blockade. We are motivated to find proper driving intensity from the aspect of the system decay rate. We explore the connections in Fig.~\ref{optdecayOm} among transversal coupling strength $J$, optimal driving field strength $\Omega_m$, and decay rate $\kappa$. In Fig.~\ref{optdecayOm}(a) with a fixed driving intensity, a region is distinguished by the white dashed lines where the $g^{(2)}$ function exhibits a non-monotonic function of the decay rate and its magnitude is less than $10^{-7}$. We then fix a strong transversal coupling strength $J/2\pi=35$ MHz that is close to the upperbound in current experiments and optimize the $g^{(2)}$ function in the space of $\Omega_m$ and $\kappa$ in Fig.~\ref{optdecayOm}(b). It is interesting to verify that there does exit a lowerbound of the driving intensity for an efficient blockade. The region for $g^{(2)}\leq10^{-7}$ is also located by the white dashed lines, where $\Omega_m$ ranges from $0.01$ MHz to $0.13$ MHz and $\kappa$ ranges from $0.35$ MHz to $0.62$ MHz.

More transparent relations among $\Omega_m$, $J$, and $\kappa$ are demonstrated by Fig.~\ref{fixdecay}. In both Fig.~\ref{fixdecay}(a) under a fixed decay rate and Fig.~\ref{fixdecay}(b) under a fixed transversal coupling strength, the magnon blockade turns out to be inefficient and even disappears when the driving intensity $\Omega_m$ approaches vanishing. It is found that the $g^{(2)}$ function could be locally minimized by $\Omega_m$. In Fig.~\ref{fixdecay}(a) with $\kappa/2\pi=0.5$ MHz, the optimal value for $\Omega_m$ becomes larger under a stronger transversal coupling. In particular, we have $\Omega_m/2\pi=0.021$ MHz and $\Omega_m/2\pi=0.033$ MHz when $J/2\pi=14$ MHz and $J/2\pi=35$ MHz, respectively. In Fig.~\ref{fixdecay}(b) with $J/2\pi=35$ MHz, $\Omega_m$ is optimized with a properly chosen decay rate. When $\kappa/2\pi=0.5$ MHz, a high-degree blockade is achieved with $g^{(2)}(0)=10^{-7.24}$.

In a system consisting of a cavity and a single three-level atom~\cite{Tang2019Strong}, the dressed-state splitting between the higher and lower branches was enhanced by the optical Stark shift, and then the $g^{(2)}$ function can be made three orders of magnitude smaller than that in the JC model [$g^{(2)}(0)\sim10^{-3}$]. In a cavity QED system~\cite{Li2021strong}, both emitter and cavity are driven by a coherent laser field, and the $g^{(2)}$ function can reach as low as $g^{(2)}(0)\sim10^{-4}$. And in a double-cavity system~\cite{Wang2020Photon}, the photon blockade with $g^{(2)}(0)\sim10^{-5}$ can be observed. In comparison to these optomechanical systems, the magnon-qubit system is featured with strong transversal coupling strength and dramatically low damping rate. We can therefore realize a higher degree of blockade [$g^{(2)}(0)\sim10^{-7}$] under the optimized conditions $\Delta_+=J$ and $\Omega_q=3\Omega_m$ with proper driving field strength and system decay rate.

\section{Nonideal situations}\label{discussion}

We have created the magnon blockade in a transversally coupled magnon-qubit system and found its optimal conditions. The preceding evaluation about the $g^{(2)}$ function is performed under a zero-temperature environment. And the external magnetic flux is fixed at a proper value to avoid the longitudinal (radiation-pressure like) interaction between magnon and qubit. In this section, we would consider the nonideal situations.

\subsection{Thermal noise effects on magnon blockade}\label{thermalnoise}

\begin{figure}[htbp]
\centering
\includegraphics[width=0.95\linewidth]{Oqnth.eps}
\caption{(a) Steady-state $\log_{10}g^{(2)}{(0)}$ as a function of the ratio $\Omega_q/\Omega_m$ with various occupation number of thermal field $\bar{n}$. (b) Steady-state $\log_{10}g^{(2)}{(0)}$ as a function of the temperature $T$ under $\Omega_q=3\Omega_m$. $\Delta_+=J$, $J/2\pi=35$ MHz, $\Omega_m/2\pi=0.033$ MHz, and $\kappa/2\pi=0.5$ MHz.}\label{Oqnth}
\end{figure}

Taking account a non-zero temperature bosonic environment into consideration, we can discuss the effect from a thermal noise on the magnon blockade. Under the resonant condition $\Delta_-=0$, the master equation~(\ref{masterequation}) becomes
\begin{equation}\label{masterequation_nth}
\begin{aligned}
\frac{\partial}{\partial t}\rho&=-i[H_{\rm eff},\rho]+\frac{\kappa_m}{2}(\bar{n}+1)\mathcal{L}_m[\rho]+\frac{\kappa_m}{2}\bar{n}\mathcal{L}_{m^\dagger}[\rho]\\
&+\frac{\kappa_q}{2}(\bar{n}+1)\mathcal{L}_{\sigma_-}[\rho]+\frac{\kappa_q}{2}\bar{n}\mathcal{L}_{\sigma_+}[\rho],\\
\end{aligned}
\end{equation}
where $\bar{n}=[\exp(\hbar\omega_m/k_BT)-1]^{-1}$ is the thermal occupation number of the surrounding environmental field. The $g^{(2)}$ function is then numerically solved by Eq.~(\ref{masterequation_nth}) with the effective Hamiltonian in Eq.~(\ref{Ham_eff}). In the hybrid magnon-qubit system~\cite{Kounalakis2022Analog}, the direct interaction between magnon and qubit is established at a cryogenic temperature about $5$ mK~\cite{Kounalakis2022Analog}. With the magnon (qubit) frequency $1.5$ GHz, the thermal occupation number is about $\sim10^{-8}$, $\sim10^{-7}$, $\sim10^{-6}$ when $T=3.9$ mK,  $T=4.5$ mK, and $T=5.3$ mK, respectively. In Fig.~\ref{Oqnth}(a) with $J/2\pi=35$ MHz and $\Omega_m/2\pi=0.033$ MHz, one can find that the optimal condition around $\Omega_q=3\Omega_m$ can be maintained when $\bar{n}<10^{-6}$, although the magnitude of the $g^{(2)}$ function increases with $\bar{n}$. In particular, $g^{(2)}(0)=10^{-5.89}$ when $\bar{n}=10^{-8}$ and $g^{(2)}(0)=10^{-4.91}$ when $\bar{n}=10^{-7}$. In addition, a dramatic bunching phenomenon occurs under $\Omega_q=\Omega_m$ for arbitrary $\bar{n}$.

The experimental feasibility of our proposal for magnon blockade is clearly shown in Fig.~\ref{Oqnth}(b) about the temperature dependence of the $g^{(2)}$ function on the environmental temperature. It is interesting to find that the magnon blockade approaches a saturated lowerbound with $g^{(2)}(0)=10^{-7.24}$ when $T<2.6$ mK. And the blockade completely vanishes with $g^{(2)}(0)\ge1$ when $T>16.8$ mK or $\bar{n}=0.014$. This result quantifies the restriction from the thermal noise on the magnon blockade.

\subsection{Extra longitudinal interaction between magnon and qubit}\label{secLongitudinal}

In this subsection, we discuss the effect from the extra magnon-qubit longitudinal interaction $g_{\rm rp}\sigma_+\sigma_-(m+m^\dagger)$ on the magnon blockade. The coupling strength $g_{\rm rp}$ can be modulated by the externally applied flux $\Phi_b$ on qubit~\cite{Kounalakis2022Analog}. In the rotating frame with respect to $H'=\omega(m^\dagger m+\sigma_+\sigma_-)$, the effective time-dependent Hamiltonian becomes
\begin{equation}\label{Ham_longitudinal}
\begin{aligned}
H_{\rm eff}&=\Delta_+\sigma_+\sigma_-+\Delta_+m^\dagger m+J(m\sigma_++m^\dagger\sigma_-)\\
&+g_{\rm rp}\sigma_+\sigma_-(me^{-i\omega t}+m^\dagger e^{i\omega t})\\
&+\Omega_m(m^\dagger+m)+\Omega_q(\sigma_++\sigma_-).
\end{aligned}
\end{equation}

\begin{figure}[htbp]
\centering
\includegraphics[width=0.95\linewidth]{longitudinal.eps}
\caption{(a) Steady-state $\log_{10}g^{(2)}(0)$ as a function of the ratio $\Omega_q/\Omega_m$ with various ratios $g_{\rm rp}/J$. (b) Steady-state $\log_{10}g^{(2)}(0)$ as a function of the ratio $g_{\rm rp}/J$ under the conditions of $\Omega_q=3\Omega_m$. The other parameters are set as $\Delta_+=J$, $\omega/2\pi=1.5$ GHz and $\kappa/2\pi=0.5$ MHz.}\label{longitudinalfig}
\end{figure}

Using the master equation~(\ref{masterequation}) with the effective Hamiltonian $H_{\rm eff}$ in Eq.~(\ref{Ham_longitudinal}), we plot the steady-state $g^{(2)}$ function in Fig.~\ref{longitudinalfig}(a) versus the ratio $\Omega_q/\Omega_m$ under various ratios $g_{\rm rp}/J$. Note for the two coupling strengths $J$, we use their corresponding optimized driving intensity $\Omega_m$ obtained in Fig.~\ref{fixdecay}(a). It is interesting to found that under a sufficiently large $J$, the optimal ratio $\Omega_q/\Omega_m$ deviates gradually from $\Omega_q/\Omega_m=3$ and the expectation value of the $g^{(2)}$ function becomes even larger by increasing the extra longitudinal interaction. In particularly, when the transversal interaction strength is as weak as $J/2\pi=14$ MHz, the optimal condition $\Omega_q/\Omega_m=3$ remains valid even when the extra longitudinal interaction is as strong as $g_{\rm rp}/J=0.3$. In fact, the presence of a nonvanishing $g_{\rm rp}$ has no significant impact on the $g^{(2)}$ function for arbitrary $\Omega_q/\Omega_m$. While when $J/2\pi=35$ MHz, the optimized ratio of driving and probing fields becomes about $\Omega_q/\Omega_m\approx3.1$ with $g^{(2)}(0)=10^{-6.93}$ under $g_{\rm rp}/J=0.1$, $\Omega_q/\Omega_m\approx3.2$ with $g^{(2)}(0)=10^{-6.65}$ under $g_{\rm rp}/J=0.2$, and $\Omega_q/\Omega_m\approx3.3$ with $g^{(2)}(0)=10^{-6.40}$ under $g_{\rm rp}/J=0.3$.

Moreover, we plot the steady-state $g^{(2)}$ function in Fig.~\ref{longitudinalfig}(b) under the optimized conditions of $\Delta_+=J$ and $\Omega_q=3\Omega_m$ with proper transversal coupling strength and driving intensity that minimize the $g^{(2)}$ function in Fig.~\ref{fixdecay}(a). It is confirmed that the sensitivity of the equal-time correlation function to the ratio $g_{\rm rp}/J$ is enhanced by increasing the transversal coupling strength $J$. In Fig.~\ref{longitudinalfig}(b), for a weak transversal interaction with $J/2\pi=14$ MHz, the $g^{(2)}$ function is almost invariant in the presence of the extra longitudinal interaction. When $g_{\rm rp}/J<0.04$, it is hold that the degree of magnon blockade increases with the transversal coupling strength $J$. However, the $g^{(2)}$ function under $J/2\pi=35$ MHz becomes larger than that under $J/2\pi=28$ MHz, that under $J/2\pi=21$ MHz, and that under $J/2\pi=14$ MHz, when $g_{\rm rp}/J$ approaches $0.04$, $0.08$, and $0.18$, respectively. It is enhanced by nearly three orders in magnitude when $g_{\rm rp}/J$ increases from zero to $0.5$. It indicated that the magnon blockade can no longer be promoted by increasing the transversal interaction in the presence of a significant longitudinal interaction. While with a moderate $J$, e.g., $J/2\pi=21$ MHz, we can achieve $g^{(2)}(0)\sim 10^{-6}$ even when $g_{\rm rp}/J=0.3$. Our magnon-blockade proposal as well as the optimal condition for $\Omega_q/\Omega_m$ is therefore robust in the presence of the extra longitudinal interaction.

\section{Conclusion}\label{Conclusion}

In summary, we propose a protocol to generate and optimize the magnon blockade in a magnon-qubit system established by the direct interaction between a YIG sphere and a superconducting transmon qubit. Assisted by a weak driving field on the magnon, a weak probing field on the qubit, and a strong transversal magnon-qubit interaction, we can achieve a high-degree magnon blockade evaluated by the steady-state equal-time second-order correlation function $g^{(2)}(0)$ through optimizing the relevant system parameters. It is found on both analytical derivation and numerical simulation that $g^{(2)}(0)\sim10^{-7}$ is available with experimental-relevant parameters, when the transversal coupling strength equals the detuning average of magnon and qubit and the driving-field intensity is about three times of the probing-field intensity. It is about two orders in magnitude lower than that for the photon blockade in cavity optomechanical systems. We also discuss the upper-bound of the environmental temperature to restrict the magnon blockade and the influence from the extra longitudinal interaction between magnon and qubit. Our magnon-blockade proposal provides a systematic optimization process for magnon blockade. It paves a way towards quantum manipulation at the level of a single magnon.

\section*{Acknowledgments}

We acknowledge grant support from the National Science Foundation of China (Grant No. 11974311).

\appendix

\section{Analytical model of optimized conditions for blockade}\label{Analytic model}

This appendix contributes to confirming the analytical expression for the steady-state equal-time second-order correlation function in Eq.~(\ref{g2analytical}) and then locating the optimized point for blockade in the ratio of $\Omega_q/\Omega_m$. By virtue of the analytic model~\cite{Kar2013Single}, the steady-state solution for the dynamical equation~(\ref{differential}) can be obtained from
\begin{equation}\label{differentialsteady}
\begin{aligned}
0&=JC_{g1}+\Omega_qC_{g0}+(\Delta_+-i\kappa/2)C_{e0},\\
0&=JC_{e0}+\Omega_mC_{g0}+(\Delta_+-i\kappa/2)C_{g1},\\
0&=\sqrt{2}JC_{g2}+\Omega_qC_{g1}+\Omega_mC_{e0}+2(\Delta_+-i\kappa/2)C_{e1},\\
0&=\sqrt{2}JC_{e1}+\sqrt{2}\Omega_mC_{g1}+2(\Delta_+-i\kappa/2)C_{g2}.
\end{aligned}
\end{equation}
Under the condition of $\Delta_+=J$, we have
\begin{equation}\label{steadysolutions}
\begin{aligned}
C_{g0}&\approx1\\
C_{e0}&=-\frac{4J(\Omega_m-\Omega_q)+2i\Omega_q\kappa}{\kappa^2+4iJ\kappa},\\
C_{g1}&=-\frac{4J(\Omega_q-\Omega_m)+2i\Omega_m\kappa}{\kappa^2+4iJ\kappa},\\
C_{g2}&=\frac{2\sqrt{2}(A+Bi)}{(\kappa^2-2J^2+4iJ\kappa)(\kappa^2+4iJ\kappa)},\\
C_{e1}&=-\frac{4(C+Di)}{(\kappa^2-2J^2+4iJ\kappa)(\kappa^2+4iJ\kappa)},\\
\end{aligned}
\end{equation}
where
\begin{equation}\label{ABCDE}
\begin{aligned}
A&=2J^2(3\Omega_m^2+\Omega_q^2-4\Omega_m\Omega_q)-\Omega_m^2\kappa^2,\\
B&=4J\kappa(\Omega_m\Omega_q-\Omega_m^2),\\
C&=2J^2(2\Omega_m^2+\Omega_q^2-3\Omega_m\Omega_q)+\Omega_m\Omega_q\kappa^2,\\
D&=J\kappa(4\Omega_m\Omega_q-2\Omega_m^2-\Omega_q^2).\\
\end{aligned}
\end{equation}
With the steady amplitudes in Eq.~(\ref{steadysolutions}), the $g^{(2)}$ function defined in Eq.~(\ref{g2define}) can be expressed by
\begin{equation}\label{g2analyticalDetailed}
\begin{aligned}
&g^{(2)}(0)=\frac{2|C_{g2}|^2}{(|C_{g1}|^2+|C_{e1}|^2+2|C_{g2}|^2)^2}\approx\frac{2|C_{g2}|^2}{|C_{g1}|^4}\\
&=\frac{(A^2+B^2)(\kappa^4+16J^2\kappa^2)}{[(2J^2-\kappa^2)^2+16J^2\kappa^2][4J^2(\Omega_q-\Omega_m)^2+\Omega_m^2\kappa^2]^2},
\end{aligned}
\end{equation}
which is valid under $|C_{g0}|\gg|C_{g1}|, |C_{e0}|\gg|C_{e1}|, |C_{g2}|$. Using $\Omega_q=(l+1)\Omega_m$ and $r=\kappa/J$, we can rewrite the $g^{(2)}$ function in Eq.~(\ref{g2analyticalDetailed}) as a function of the dimensionless parameters $l$ and $r$:
\begin{equation}\label{g2analyticalDetailed_n}
g^{(2)}(0)=\frac{4(l-2)^2l^2+4r^2(3l+2)l+r^4}{\left[1+\frac{4(1-r^2)}{r^4+16r^2}\right](4l^2+r^2)^2}.
\end{equation}
The first derivative of $g^{(2)}(0)$ with respect to $l$ is
\begin{equation}\label{g2analyticalderive}
\begin{aligned}
\frac{dg^{(2)}(0)}{dl}&=\frac{l^4+bl^3+cl^2+dl+f}{\left[1+\frac{4(1-r^2)}{r^4+16r^2}\right](l^2+\frac{r^2}{4})^3}
\end{aligned}
\end{equation}
where $b=-5r^2/4-2$, $c=-9r^2/4$, $d=r^4/8+r^2/2$, and $f=r^4/8$. With the \emph{Galois theory}~\cite{Harold1984Galois}, the numerator of Eq.~(\ref{g2analyticalderive}) can be reorganized into
\begin{equation}\label{reducedsteptwo}
\left(l^2+\frac{1}{2}bl+\frac{1}{2}y\right)^2-(l-l')^2,\\
\end{equation}
where
\begin{equation}\label{yl'}
\begin{aligned}
l'&=\frac{2d-by}{b^2-4c+4y},\\
y&=\frac{c}{3}+\frac{-1-\sqrt{3}i}{12}(\sqrt{h_1}+\sqrt{h_1+h_2})^{\frac{1}{3}}\\
&+\frac{-1+\sqrt{3}i}{12}(\sqrt{h_1}-\sqrt{h_1+h_2})^{\frac{1}{3}}
\end{aligned}
\end{equation}
with $h_1=(-36cc_1+8c^3-108d_1)^2$, $h_2=(12c_1-4c^2)^3$, $c_1=-(4f-bd)$, and $d_1=-(b^2-4c)f-d^2$. Then we can find two real roots of $l=\Omega_q/\Omega_m-1$ for $dg^{(2)}(0)/dl=0$:
\begin{equation}\label{roots}
\begin{aligned}
l_1&=\frac{-\frac{1}{2}b+1+\sqrt{\frac{b^2}{4}-b-2y-4l'+1}}{2},\\
l_2&=\frac{-\frac{1}{2}b+1-\sqrt{\frac{b^2}{4}-b-2y-4l'+1}}{2},
\end{aligned}
\end{equation}
that are larger than $-1$ in the positive region of $r$.

\begin{figure}[htbp]
\centering
\includegraphics[width=0.95\linewidth]{t1t2.eps}
\caption{(a) $l_1$ and (b) $l_2$ versus the ratio $r$ of the system decay rate and the transversal coupling strength. (c) and (d): Analytical [Eq.~(\ref{g2analyticalDetailed_n})] and numerical results [Eq.~(\ref{masterequation})] for the steady-state $\log_{10}g^{(2)}{(0)}$ versus the ratio $r$ with the roots $l_1$ and $l_2$, respectively. $\Delta_+=J$, $J/2\pi=35$ MHz, and $\Omega_m/2\pi=0.033$ MHz.}\label{t1t2}
\end{figure}

\begin{figure}[htbp]
\centering
\includegraphics[width=0.95\linewidth]{comparison.eps}
\caption{Steady-state $\log_{10}g^{(2)}{(0)}$ versus the ratio of the probing field strength to the driving field strength $\Omega_q/\Omega_m$. $\Delta_+=J$ and $\Omega_m/2\pi=0.1$ MHz}\label{comparison}
\end{figure}

$l$ is a function solely determined by $r$, i.e., the ratio of the system decay rate and the transversal coupling strength $\kappa/J$. In Figs.~\ref{t1t2}(a) and \ref{t1t2}(b), we can find that the two extremum points $l_1\approx2$ and $l_2\approx0$ for a range of experimental-relevant parameters $r\in(0, 0.25)$. Accordingly, we plot both the analytical and numerical results for the $g^{(2)}$ function in Figs.~\ref{t1t2}(c) and \ref{t1t2}(d). It is found in Fig.~\ref{t1t2}(c) that when $r\ge0.014$, the numerical simulation could be perfectly interpreted by the analytical evaluation around the root $l_1\approx2$. Particularly the turning point of $r\approx0.014$ corresponds to the purple line in Fig.~\ref{fixdecay}(b). The model for any blockade becomes invalid with no dissipation. It is thus reasonable to understand the derivation between the numerical simulation and analytical evaluation when $r$ becomes smaller than a turning point. In contrast, it is found in Fig.~\ref{t1t2}(d) that the analytical model around the root $l_2\approx0$ is unable to interpret the numerical results in quantity. Also the value of the $g^{(2)}$ function around $l_2$ is much higher than that around $l_1$ under the same $r$.

The optimized condition $l_1\approx2$ or $\Omega_q/\Omega_m\approx3$ can be further confirmed by the second derivative of the $g^{(2)}$ function with respect to $l$:
\begin{equation}\label{secondderivative}
\frac{d^2[g^{(2)}(0)]}{dl^2}=\frac{-2l^5+b'l^4+c'l^3+d'l^2+f'l+g'}
{\left[1+\frac{4(1-r^2)}{r^4+16r^2}\right](l^2+\frac{r^2}{4})^4}
\end{equation}
where $b'=36-6b$, $c'=2c-bc+r^2$, $d'=d-6d+9r^2$, $f'=cr^2-6f$, and $g'=dr^2/4$. We have $d^2[g^{(2)}(0)]/dl^2>0$ for $l_1=2$ and $r\in(0, 0.25)$. It is therefore found that the $g^{(2)}$ function takes the minimum value when $\Omega_q\approx3\Omega_m$. In Fig.~\ref{comparison} under various transversal coupling strengths $J$ and system decay rates $\kappa$, it is demonstrated that for a comparatively strong decay $\kappa=1$ MHz, the analytical results by Eq.~(\ref{g2analyticalDetailed}) or Eq.~(\ref{g2analytical}) for the steady-state equal-time second-order correlation function match perfectly with the numerical results by master equation. While for a comparatively weak decay $\kappa=0.5$ MHz, there exists small deviation between them when $\Omega_q/\Omega_m\geq3$.

\bibliographystyle{apsrevlong}
\bibliography{ref}
\end{document}
