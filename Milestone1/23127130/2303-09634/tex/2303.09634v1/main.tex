%%%%%%%% ICML 2021 EXAMPLE LATEX SUBMISSION FILE %%%%%%%%%%%%%%%%%

%\documentclass{article}
\documentclass[manuscript]{acmart}
\pdfoutput=1
\settopmatter{printacmref=false} % Removes citation information below abstract
\renewcommand\footnotetextcopyrightpermission[1]{} % removes footnote with conference information in first column
\pagestyle{plain} % removes running headers

%% \BibTeX command to typeset BibTeX logo in the docs
\AtBeginDocument{%
  \providecommand\BibTeX{{%
    \normalfont B\kern-0.5em{\scshape i\kern-0.25em b}\kern-0.8em\TeX}}}

%% Rights management information.  This information is sent to you
%% when you complete the rights form.  These commands have SAMPLE
%% values in them; it is your responsibility as an author to replace
%% the commands and values with those provided to you when you
%% complete the rights form.
% \setcopyright{acmcopyright}
% \copyrightyear{2018}
% \acmYear{2018}
% \acmDOI{XXXXXXX.XXXXXXX}


%% These commands are for a PROCEEDINGS abstract or paper.
\acmConference[FunCausal '23]{}{May 09--12,
  2023}{Grenoble, France}

%%
%%  Uncomment \acmBooktitle if the title of the proceedings is different
%%  from ``Proceedings of ...''!
%%
\acmBooktitle{Proceedings of Fundamental Challenges in Causality, May 09--12, 2023, Grenoble, France}
% \acmPrice{15.00}
% \acmISBN{978-1-4503-XXXX-X/18/06}


%%
%% Submission ID.
%% Use this when submitting an article to a sponsored event. You'll
%% receive a unique submission ID from the organizers
%% of the event, and this ID should be used as the parameter to this command.
%%\acmSubmissionID{123-A56-BU3}

%%
%% For managing citations, it is recommended to use bibliography
%% files in BibTeX format.
%%
%% You can then either use BibTeX with the ACM-Reference-Format style,
%% or BibLaTeX with the acmnumeric or acmauthoryear sytles, that include
%% support for advanced citation of software artefact from the
%% biblatex-software package, also separately available on CTAN.
%%
%% Look at the sample-*-biblatex.tex files for templates showcasing
%% the biblatex styles.
%%

%%
%% The majority of ACM publications use numbered citations and
%% references.  The command \citestyle{authoryear} switches to the
%% "author year" style.
%%
%% If you are preparing content for an event
%% sponsored by ACM SIGGRAPH, you must use the "author year" style of
%% citations and references.
%% Uncommenting
%% the next command will enable that style.
%%\citestyle{acmauthoryear}

% Recommended, but optional, packages for figures and better typesetting:
\usepackage{microtype}
\usepackage{graphicx}
\usepackage{subfigure}
\usepackage{booktabs} % for professional tables



\usepackage{amsfonts}
\usepackage{amsmath}
\usepackage{xfrac}
\usepackage{enumitem}% http://ctan.org/pkg/enumitem
%\usepackage{flushend}

% Attempt to make hyperref and mic work together better:


\begin{document}

\title{Causal Temporal Graph Convolutional Neural Networks (CTGCN)}
%%
%% The "author" command and its associated commands are used to define
%% the authors and their affiliations.
%% Of note is the shared affiliation of the first two authors, and the
%% "authornote" and "authornotemark" commands
%% used to denote shared contribution to the research.
\author{Abigail Langbridge}
\email{abigail.langbridge18@imperial.ac.uk}
% \orcid{1234-5678-9012}
\affiliation{%
  \institution{Imperial College London}
  \city{London}
  \country{UK}
  \postcode{SW7 2BX}
}
\author{Fearghal O'Donncha}
\affiliation{%
  \institution{IBM Research Europe}
  \city{Dublin}
  \country{Ireland}
}
\author{Amadou Ba}
\affiliation{%
  \institution{IBM Research Europe}
  \city{Dublin}
  \country{Ireland}
}
\author{Fabio Lorenzi}
\affiliation{%
  \institution{IBM Research Europe}
  \city{Dublin}
  \country{Ireland}
}
\author{Christopher Lohse}
\affiliation{%
  \institution{Trinity College Dublin}
  \city{Dublin}
  \country{Ireland}
}
\author{Joern Ploennigs}
\affiliation{%
  \institution{University of Rostock}
  \city{Rostock}
  \country{Germany}
}

%%
%% By default, the full list of authors will be used in the page
%% headers. Often, this list is too long, and will overlap
%% other information printed in the page headers. This command allows
%% the author to define a more concise list
%% of authors' names for this purpose.
\renewcommand{\shortauthors}{Langbridge et al.}


\setlength{\abovedisplayskip}{6pt}
\setlength{\belowdisplayskip}{6pt}

% -------------------------------------------
\begin{abstract}

Many large-scale applications can be elegantly represented using graph structures. Their scalability, however, is often limited by the domain knowledge required to apply them. To address this problem, we propose a novel Causal Temporal Graph Convolutional Neural Network (CTGCN). Our CTGCN architecture is based on a causal discovery mechanism, and is capable of discovering the underlying causal processes. The major advantages of our approach stem from its ability to overcome computational scalability problems with a divide and conquer technique, and from the greater explainability of predictions made using a causal model. We evaluate the scalability of our CTGCN on two datasets to demonstrate that our method is applicable to large scale problems, and show that the integration of causality into the TGCN architecture improves prediction performance up to 40\,\% over typical TGCN approach. Our results are obtained without requiring additional domain knowledge, making our approach adaptable to various domains, specifically when little contextual knowledge is available.
\end{abstract}


%%
%% Keywords. The author(s) should pick words that accurately describe
%% the work being presented. Separate the keywords with commas.
\keywords{Graph Neural Networks, Causal Inference, Time Series, Scaling AI, Spatiotemporal}

\received{10 February 2023}
% \received[revised]{12 March 2009}
% \received[accepted]{5 June 2009}

%%
%% This command processes the author and affiliation and title
%% information and builds the first part of the formatted document.
\maketitle

% -------------------------------------------
\section{Introduction}
Numerous real-world processes contain interconnected dynamics characterised by a complex organizational structure. Examples include environmental systems, energy grids, and epidemic outbreaks. Many classical mathematical frameworks exist to model these dynamics such as Navier-Stokes or convective-diffusion equations. While these explicitly encode known relationships, machine learning provides opportunity to resolve established and latent dynamics.

Graphs have been historically used to model many of the underlying structures and physical behaviours of spatial systems such as road networks \cite{thomson1995graph}, building thermodynamics \cite{Tabunschikov92}, and water and energy grids \cite{mackaness1993use}. This allows practitioners to explicitly encode their domain knowledge. A fundamental assumption in these models is that the underlying structure of these dynamics are well known to be modelled. But, most real-world applications do not admit a-priori knowledge of these structures. Therefore, we aim to create a graph model that learns this structure using causal discovery.

Recent Temporal Graph Convolutional Neural Networks (TGCN) \cite{sanchez2018graph,dai2021graph} utilise this available domain knowledge in graphs by combining learning over temporal and graphical features.
%can overcome this limitation and offer the possibility to directly learn the graph structure and encode the topological structure in the modelling framework. 
These TGCN models also assume that spatial information or similar prior knowledge regarding connections is available \cite{pvnetwork_spatial, Yu2018SpatioTemporalGC}.
However, in practise, many use cases for spatiotemporal modelling do not have well-defined graph structures. In contrast, in most practical applications we see clients collect new timeseries data faster than they are able to specify contextual information which grows the problem.

This lack or uncertaincy of a-priori spatial knowledge is intensified for downstream tasks of the TGCN. First, incorrect graph models will strongly influence the prediction performance of the model limiting optimization and diagnostic tasks. Second, recent works have investigated the efficacy of post-hoc explanations for graph convolutional neural network (GCNN) predictions \cite{GNNExplainer}, with some approaches using causal inference methods \cite{OrphicX}, but these fundamentally rely on the correctness and completeness of the graph over which they are explaining.

Some work has investigated treating the graph as a learnable parameter to optimise for a given downstream task \cite{dgm_learning, latent_graph_learning}. \citet{adaptive_dep_lrn} proposed an efficient method to construct a dynamic dependency graph based on statistical structure learning models. Prominent limitations of these methods is that they are prone to learning spurious correlations and not the underlying causes. Other work looks into identifying the graph from physics equations from scientific papers utilizing NLP approaches \cite{ba2022automated}, but it assumes that systems exhibit static physical behaviour.

Also traditional Bayesian structure learning approaches are often not applicable to high-dimensional, real-world data due to their computational cost and unrealistic assumptions which include stationarity, the absence of latent confounders in data, and that none of the underlying relationships are contemporaneous \cite{bayesian_cost}.

In this work, we present a novel, scalable method for deducing causal relationships in large observational data, which we integrate into a TGCN architecture to overcome the limitation of requiring a-priori domain knowledge. This gives our causal-informed TGCN (CTGCN) visibility of the underlying structural causal model in an automated way and facilitates its self-adaptation and scalability in diverse large-scale applications. Furthermore, we investigate the computational complexity and predictive skill of the causal discovery process in order to improve the performance for large scale, real-world applications and non-IID datasets. We further study the effects of decomposing the causal discovery problem on different downstream predictive task, as we posit that integrating structural causal models into the graph convolution layer of a TGCN model improves model performance, robustness and explainability.

The main contributions of this work are:
\begin{itemize}[noitemsep,topsep=0pt]
    \item We develop a novel causal-informed TGCN discovery method in order to facilitate adoption for large-scale, interconnected, and complex applications.
    \item We extend existing causal discovery methods with spatial and temporal decomposition to improve their scalability for large-scale applications.
    \item We demonstrate that the integration of causal structure information into predictive models through a graph convolution layer improves forecasting performance.
\end{itemize}

% Many real-world processes such as environmental systems, energy grids, and epidemic outbreaks are characterised by distinct spatial and temporal relationships. Many classical mathematical frameworks exist to model these dynamics such as Navier-Stokes or convective-diffusion equations. More recently, machine learning (ML) has achieved success in problems with clear spatial (e.\,g. satellite image processing) or temporal (e.\,g. speech recognition) dependencies. However, applying these ML models in practice often requires domain knowledge in order to fine-tune them to individual systems.

% Graphs have been historically used to model many of the underlying structures and physical behaviours of spatial systems such as road networks \cite{thomson1995graph}, building thermodynamics \cite{Tabunschikov92}, and water and energy grids \cite{mackaness1993use}. This allows practitioners to explicitly encode their domain knowledge. A fundamental assumption in these models is that the underlying structure of these dynamics are known to be encoded in the graph model.

% Recent Temporal Graph Convolutional Neural Networks (TGCN)~\cite{sanchez2018graph,dai2021graph} utilise this available domain knowledge in graphs by combining learning over temporal and graphical features.
% %can overcome this limitation and offer the possibility to directly learn the graph structure and encode the topological structure in the modelling framework. 
% These TGCN models therefore assume that the spatial information, or some prior knowledge regarding connections is available \cite{pvnetwork_spatial, Yu2018SpatioTemporalGC}.
% However, in industry, many practical use cases for spatiotemporal modelling do not have well-defined graph structures. In contrast, in most practical applications we see clients collect new timeseries data faster than they are able to specify contextual information which intensifies the problem.

% Some work has investigated treating the graph as a learnable parameter to optimise for a given downstream task \cite{dgm_learning, latent_graph_learning}. \citet{adaptive_dep_lrn} proposed an efficient method to construct a dynamically changing dependency graph based on statistical structure learning models. Prominent limitations of these graph construction methods is that they are prone to learning spurious correlations and not accurately representing the underlying causes.

% In this work, we also utilise TGCN for spatiotemporal forecasting and capture the relationships between component systems with a graph. For each node, we model the temporal data using a sequence to sequence learner. %This allows us to transform our spatiotemporal problem into a TGCN. However, one of the major limitations of TGCN comes from defining the adjacency matrix used in its realization. 
% However, we propose a novel extension of TGCN, where its adjacency matrix is derived from a causal discovery mechanism to overcome the problem of requiring a-priori domain knowledge. 
% %Practical applications require discovery of a representative adjacency matrix that scales to real-world systems.
% This gives TGCN visibility of the underlying structural causal model in an automated way and facilitates its self-adaptation and scalability in diverse large-scale applications. Furthermore, we investigate the computational complexity and predictive skill of the causal discovery process in order to improve the performance for large scale applications and non-IID datasets. We further study the effects of decomposing the causal discovery problem on different downstream predictive task, as we posit that integrating structural causal models into the graph convolution layer of a TGCN model improves model performance and robustness.

% The main contributions of this work are:
% \begin{itemize}[noitemsep,topsep=0pt]
%     \item We develop a novel causal-informed TGCN discovery method in order to facilitate adoption for large-scale, interconnected, and complex applications.
    
%     \item 
%     % the adaptation 
%     We extend existing causal discovery methods with spatial and temporal decomposition to improve their scalability for large-scale applications.
%     % The development of a TGCN approach that uses a causal discovery method  in order to make it a scalable and semi-automatic tool capable of detecting causal relationships in complex systems, and considering that results in the adjacency matrix of the TGCN.
    
%     \item We demonstrate that the integration of causal structure information into predictive models through a graph convolution layer improves forecasting performance.
%     \item We evaluate the effectiveness of the method on three datasets from different domains.
% \end{itemize}

\section{Related Work}
First, we provide a background on spatiotemporal graph convolutional neural networks (GCNN). We then discuss the utility of existing causal inference methods for large-scale data.


\subsection{Spatiotemporal GCNN}

Our approach relies on GCNN \cite{KipfW17}, initially introduced by \citet{Bruna2014SpectralNA}, and extended by \citet{Duvenaud2015ConvolutionalNO} with fast localised convolutions.
% The effectiveness of GCNN for addressing node classification and link prediction prompted its use for time series data. However, very limited approaches address the use of GCNN with sequence-to-sequence learning, 
Approaches such as TGCN \cite{Yu2018SpatioTemporalGC,li2016gated,8809901} augment this method with sequence-to-sequence learning to encode temporal dynamics.
The penetration of GCNN into sensor-driven applications using time series data led to the extension of GCNN with sequence-to-sequence learning methods such as Gated Recurrent Unit (GRU) or Long Short-Term Memory (LSTM) for forecasting applications \cite{pathak2018forecasting}. The combination of GCNN with sequence-to-sequence learning methods is generally motivated by the need to simultaneously capture spatial and temporal dependencies.
In this case, the GCNN is used to capture spatial dependencies by building a filter that acts on the nodes and their $n$-th order neighbourhood (usually first order). This enables the filter to capture spatial features between the nodes, and further the GCNN can be developed by stacking multiple convolutional layers. The sequence-to-sequence learning method is employed to capture dynamic changes of the systems.

However, modelling complex topological structures with sequence-to-sequence learning usually fails in capturing a system's underlying causal processes and thus restricts the input-output relationships to correlations. Schölkopf highlights the robustness of similar approaches as a key problem, suggesting the integration of causal modelling in ML could overcome this \cite{causal4ml}. Many real-world problems that we are interested in studying violate the IID assumption, which underpins many correlational approaches. 
% Schölkopf argues that theories from causality could help overcome the shortcomings of existing approaches by improving model generalisability to new data and even new problems.
We propose an extension of the TGCN architecture that introduces a scalable causal convolution to capture the characteristics of the underlying system.

 % The promising results of GCNN-based architecture for link prediction and node classification tasks led to its application on spatial-temporal problems (\citet{Yu2018SpatioTemporalGC}, \citet{8809901}). These Approaches try to capture the spatial and temporal dynamics in different parts of the model. While some works capture temporal characteristics by applying a one-dimensional convolution to capture the key characteristics (\citet{Yu2018SpatioTemporalGC}, \citet{liu2022fine}), other methods capture temporal dynamics by sequence-to-sequence models such as Gated Recurrent Unit (GRU) or Long Short-term Memory (LSTM) (\citet{8809901}, \citet{li2016gated}).The spatial characteristics are usually encoded by using multiple GCNN layers (\citet{Yu2018SpatioTemporalGC},{\citet{liu2022fine}); the use of the GCNN is to add as a filter that takes temporal features of adjacent nodes in the $n$-th order neighbourhood (usually a first order is considered) into account for representing the nodes in a latent space.}Simultaneously capturing spatial and temporal characteristics in a single model proved to be effective on various forecasting tasks (\citet{8809901}, \citet{wu2020connecting},\citet{wang2018spatio}).


\subsection{Causal Discovery}

Time-series data has long been a challenging problem in causal discovery: while the canonical order of data facilitates the directing of causal links (thus overcoming Markov equivalence), strong autocorrelation and the presence of unobserved confounders reduces detection power~\cite{PCMCI+}. The scale of most modern data exacerbates these problems, rendering many methods intractable.

% Traditional Bayesian structure learning approaches are often not applicable to high-dimensional, real-world data due to their computational cost and unrealistic assumptions which include stationarity, the absence of latent confounders in data, and that none of the underlying relationships are contemporaneous \cite{bayesian_cost}. 

Conditional independence (CI) methods have shown significant promise in time-series causal discovery due in part to their flexibility to utilise various CI tests. 
% , from simple linear partial correlation to non-linear non-parametric methods including neural networks.
The PC algorithm presented in the seminal work \citet{pc} utilises sparsity to improve the scalability of the CI method. This has led to incremental improvements on the base PC algorithm, firstly in the form of the Fast Causal Inference (FCI) algorithm and its adaptation to time series data \cite{fci,ts_fci}, and further for improving detection power with PCMCI$^+$ \cite{PCMCI+} and latent variable detection with LPCMCI \cite{LPCMCI}.

% \citet{deep_causal_discovery} designed a temporal causal discovery framework consisting of multiple attention-based CNNs that predicted multiple time series using a single series. The authors infer that series are causally related when a series has high attention. While performance on the causal discovery task is good, and some assumptions of other causal models are relaxed, the method does not solve the problem of scalability.

These approaches work well for small scale problems, but face serious combinatorial explosion problems which renders these promising approaches infeasible for large scale applications. 
% This renders these promising approaches infeasible for large scale applications to automate TGCN models in various scenarios. 
In order for our proposed causal TGCN to be tractable for real-world systems, a more scalable approach to causal discovery must be developed. The score-based fast Greedy Equivalence Search (fGES) \cite{FGES} highlights the appetite for scalable causal discovery algorithms, however to the best of the authors' knowledge no equivalent approach to the acceleration of CI-based methods exists, and score-based methods are limited by their inability to overcome Markov equivalence.

 % We can have paragraph here on GNN for smart cities applications: energy, water, transport



% -------------------------------------------
\section{Methodology}

To overcome the challenges in applying TGCN at large scale, we developed a novel causality-based TGCN approach that is capable of automatically identifying the causal relationships of spatiotemporal systems and configuring TGCN models accordingly.

Figure~\ref{fig:block-diagram} shows the workflow of the methodology. We use as input a multivariate timeseries dataset, and first do causal discovery using PCMCI$^+$ \cite{PCMCI+}. To overcome the underlying scalability problems, we decompose the identification problem temporally and spatially and then recombine the results in a matrix aggregation step. Then we transform the discovered causal relationships into the adjacency matrix of a TGCN.
% In the following section we introduce this framework starting with GNNs, to causal inference identification, temporal and spatial decomposition and result transformation for an evaluation task.

\begin{figure*}[t!]
\vskip 0.2in
\begin{center}
\centerline{\includegraphics[width=0.9\textwidth]{block_diagram.png}}
\caption{Block diagram of the proposed causal TGCN architecture.}
\label{fig:block-diagram}
\end{center}
\vskip -0.2in
\end{figure*}


\subsection{Graph Convolutional Neural Networks}\label{sec:gcnn}
%We provide in this section a brief background on Graph Convolutional Neural Networks (GCNN) and the approach to causal discovery in order to build the adjacency matrix.

%\subsubsection{Background on GCNN}

We consider as input a feature matrix $\mathcal{X} \in \mathbb{R}^{N \times P}$ of a multi-variate spatiotemporal system, with the number of features $N$, the length of the time series $P$ and an observation vector $\mathcal{X}_{t} \in \mathbb{R}^{N}$ at time $t$.

Each feature $\mathcal{X}^j \in \mathcal{X}$ can be represented as node in a graph $\mathcal{G}$. As our first modification of a traditional GCNN model, we define a \textit{causal adjacency matrix} $A$ that defines the causal relationship for each directed edge $a_{j,k}$ between feature pairs $\mathcal{X}^j$ and $\mathcal{X}^k$. We distinguish two forms: (i) first a binary causal adjacency matrix where the relationship $a_{j,k} \in \{0,1\}$ defines the binary existence of a causal link. We also consider (ii) a weighted causal adjacency matrix, where $a_{j,k} \in \mathbb{R}^{+}_0$ provides the weight of the causal relationship assuming that any $a_{j,k}=0$ signifies no causal relationship.

To evaluate the performance of our approach, we define the spatiotemporal forecasting problem as learning the mapping function $\operatorname{f}$ using the structural information provided by the adjacency matrix $A$ and the features $\mathcal{X}$.
% As target of our approach we state the problem of spatiotemporal forecasting that consists of learning the mapping function $\operatorname{f}$ using the structural information provided by the adjacency matrix $A$ and the features $\mathcal{X}$.

A GCNN model constructs the mapping function $\operatorname{f}$ as filter in the Fourier domain. The filter then acts on the nodes of the graph and its first order neighbourhood. This allows the topological structure and the spatial features between the nodes to be captured. Subsequently, the GCNN model can be established by stacking multiple convolutional layers. The GCNN is given by
\begin{align}
\operatorname{f}\left(\mathcal{X}, A\right) = \sigma\left(\hat{A}\, \operatorname{ReLU}\!\!\left(\hat{A}\,\mathcal{X} W^{(0)}\right)\,W^{(1)}\right),
\end{align}

where $\mathcal{X}$ is the feature matrix, $A$ represents the adjacency matrix, $\hat{A} = \tilde{D}^{-\frac{1}{2}}\tilde{A} \tilde{D}^{-\frac{1}{2}}$ is a preprocessing step, $\tilde{A} = A+ I_{N}$ is a matrix that considers the features of the nodes for which the learning is conducted, $\tilde{D}$ is a degree matrix, where $\tilde{D} = \sum_{j} \tilde{A}_{ij}$, $W^{(0)}$ and $W^{(1)}$ represent the weight matrix in the first and second neighborhood, and $\sigma$, and $\operatorname{ReLU}$ represent the activation functions.


\subsection{Causal Inference of the Adjacency Matrix}
% \textcolor{blue}{We develop in this section the the approach to inferring the adjacency matrix through a causal discovery mechanism....}

To solve the above GCNN problem, we need to identify the adjacency matrix $A$. We frame this as a causal learning problem on the structure of the underlying causal processes:
\begin{equation}
    \mathcal{X}^j_t = \operatorname{g}_j\left(\mathcal{P}(\mathcal{X}^j_t), \eta^j_t \right),
\label{eq:scm}
\end{equation}
where $\operatorname{g}_j$ are arbitrary measurable functions that depend non-trivially on the causal parents $\mathcal{P}$ of the given node $j$, and $\eta$ is independent noise obscuring the causal processes.

As described in Section 1.3, there are various methods for discovering these causal relationships. In this work, we adapt the constraint-based method PCMCI$^+$ presented by \citet{PCMCI+}. This method increases the detection power over highly autocorrelated data compared to seminal methods \cite{pc}, and its ability to detect contemporaneous links makes it particularly suitable for dealing with noisy real-world spatiotemporal data.

%The PCMCI$^+$ algorithm \cite{PCMCI+} is a constraint-based method for deducing the SCM from observational data. It is an extension of the PC algorithm \cite{pc}, improving detection power through the optimisation of the conditioning sets. PCMCI$^+$ can detect contemporaneous links, is robust to the presence of independent variables, and benefits from strong autocorrelation. Combined with the flexibility to select CI tests appropriate to the data, this makes the algorithm a strong candidate for the development of a more scalable solution.

Causal discovery methods such as PCMCI$^+$ are bound by assumptions that the observations are faithful to and fully representative of the underlying processes, and that these processes are stationary and acyclic. PCMCI$^+$ is based on Runge's definition of a momentary conditional independence (MCI) test \cite{PCMCI} that, for each lagged observation $\mathcal{X}^j_{t-\tau}$, $\mathcal{X}^k_t$ of a feature pair $j,k$, tests for the existence of a causal relationship given a lag $\tau \in (1,\dots,\tau_{max})$. If the $p$-value of the test is above the significance threshold $\alpha$, we consider a binary causal relationship between $\mathcal{X}^j_{t-\tau}$, $\mathcal{X}^k_t$ where $c_{j,k,t,\tau}=\operatorname{L}(p_{j,k,t,\tau}>\alpha)$ with the logical function $L(b)$ which is $1$ if condition $b$ evaluates true and $0$ otherwise.

The biggest limitation of the approach is its scalability. As an MCI test is computed for each lagged combination of features, we observe a worst-case time complexity of $\mathcal{O}(P \cdot ((N \cdot \tau_{max})^2 + e^N))$ for PCMCI$^+$. While this is a significant improvement on the original PC algorithm's worst case $\mathcal{O}(P \cdot e^{N \cdot \tau_{max}})$, it is still intractable for large $N$ and $P$. The choice of CI tests also affects the runtime, with methods more robust to non-linear relationships, and therefore more generalisable, increasing computational cost.



\subsection{Temporal Decomposition}
A key limitation of existing causal discovery methods is the assumption of stationarity, which is unrealistic for data spanning months or years and encompassing varying temporal dynamics. To overcome this, and improve the scalability of our method, we propose splitting the data along the time axis into approximately stationary periods with $P_T$ observations.

These periods are typically derived based on generalised domain expertise or statistical analysis of exogenous features. Human systems such as buildings, energy, transport, or finance often exhibit daily, weekly, or monthly patterns, while natural systems such as oceans, agriculture, and weather often exhibit daily, seasonal, or annual dynamics. 

% And aggregated into prediction periods [how?] -- We don't really need to cover this. Typically the prediction period is a business problem and not defined by the scientist 

\subsection{Spatial Decomposition}
% Use of dynamic time warping to cluster data by similarity of time-series behaviour \& semi-automatic selection of sensible number of clusters through elbow plot

% BUT for systems with multiple features per variable we could produce a 'cluster' for each node which contains the constituent variables [IS THIS VALID? TEST WITH HEATING DATA]

% Causal learning can then be conducted within each cluster and aggregated into an N x N adjacency matrix

% Collapsing on the time axis as above
Modern systems in manufacturing, smart cities, and the automotive industry are highly monitored with thousands of IoT devices. The large spatial dimensionality is computationally challenging for existing CI algorithms. Intelligent spatial decompostion can dramatically reduce computational expense. Many decomposition approaches exist such as statistical, domain-inspired, or rule-based methods. 

Dynamic time warping (DTW) is a pattern-matching approach to the alignment of time-series data first proposed for speech recognition by \citet{DTW}. It has more recently been popularised as a method for the unsupervised classification of large temporal data in applications from finance to astronomy \cite{cluster-review}. The principal advantage of the DTW approach over Euclidean distances lies in the fact that one-to-many relationships can be mapped between candidate timeseries, which facilitates the identification of shared trends even if they occur on different timescales.

By utilising unsupervised DTW clustering on the feature axis, we can decompose the causal discovery problem into several sub-problems which we solve independently. We evaluate PCMCI$^+$ within each cluster and for each temporal period. This divide-and-conquer approach reduces the worst-case complexity of PCMCI$^+$ to $\mathcal{O}(D \cdot P_T \cdot (( N_C \cdot \tau_{max})^2 + e^{N_C}))$ with $P_T$ being the temporal decomposition period, $D$ being the cluster number and $N_C$ being the maximum cluster size.
We posit that by clustering in this way, we minimise the number of cross-cluster relationships in the underlying causal model, and therefore maximise the recall of the decomposed problem.

Further, as DTW clustering is an unsupervised method, this decomposition requires few additional parameters to be run automatically as shown in Table \ref{tab:ts-pcmci}.
% The complete spatial decomposition method is given in Algorithm \ref{alg:spatial-decomp}.

% Introduces more new params:
% \begin{itemize}
%     \item Number of clusters (can be selected automatically)
%     \item Collection method (for PeMSD7 we get good results even with no within-cluster links, but may not be universally appropriate)
% \end{itemize}

% \begin{algorithm}[tb]
% \caption{Spatial decomposition algorithm, where DTW is an adapted k-means clustering algorithm, and PCMCI$^+$ is the causal discovery process.}
% \label{alg:spatial-decomp}
% \begin{algorithmic}
% \STATE {\bfseries Input:} data $\mathcal{X}_\mathcal{T}$ [$P_T \times N$], num clusters $D$, parameters $p$
% \STATE Initialise A_\mathcal{T}[$n \times n$] = 0
% \STATE clusters = DTW($X$, $D$)
% \FOR{$j=0$ {\bfseries to} $D-1$}
%     \STATE S = nodes in $j$th cluster
%     \STATE R = X[:,S]
%     \IF {Length(S) $> 1$}
%         \STATE $C_\mathcal{T,S} = \operatorname{PCMCI}^+(R, p)$ and eq.~(\ref{eq:agg})
%     \ENDIF
% \ENDFOR
% % \STATE A = A + A$^T$
% % \STATE A[A $>$ 1] = 1
% \STATE {\bfseries return} A
% \end{algorithmic}
% \end{algorithm}

\subsection{Adjacency Matrix Construction} \label{sec:adj}

The causal discovery steps outlined above produce results $c_{j,k,t,\tau}=\operatorname{L}(p_{j,k,t,\tau}>\alpha)$ for each $\tau$-lagged timestep $t$ and detect causal relationships which span from contemporaneous ($\tau = 0$) up to a maximum lag ($\tau_{max}$) which exceed some significance threshold $\alpha$. These parameters and their selection are summarised in Table \ref{tab:pcmci-param}.

After the temporal and spatial decomposition we retrieve $c_{j,k,t,\tau}$ for all temporal and spatial clusters and aggregate them in form of our causal adjacency matrix $A$. We first aggregate all test results within each temporal $\mathcal{T}$ and spatial $\mathcal{S}$ sample set and perform a majority vote
\begin{equation}
\hat{c}_{j,k,\mathcal{T},\mathcal{S}} = \operatorname{MV}\!\!\left(\sum_{t=0}^{P_T} \sum_{\tau=0}^{\operatorname{min}(t,\tau_{max})}\!\!\!\!\!\!\!\!\frac{1}{P_T (\tau_{max}-1)} c_{j,k,t,\tau}\!\!\right)\!\!,
\label{eq:agg}
\end{equation}
with the majority voting function $\,\operatorname{MV}(v)$ that is $1$ if $v>0.5$ and $0$ otherwise.
This filters out causal relationships that were only discovered in specific time steps, but, are not common within a temporal and spatial sample set. We then aggregate the votes from all temporal and spatial sample sets in our causal adjacency matrix $A$. We use the following aggregation strategies: 
\begin{equation}
\begin{aligned}
a_{j,k}^\text{ANY;W} = \sum_{\mathcal{S}} \sum_{\mathcal{T}} \hat{c}_{j,k,\mathcal{T,S}},\\
%a_{j,k}^\text{MT;W} = \operatorname{MV}\left(\tfrac{1}{T\cdot C}a_{j,k}^\text{ANY;W}\right)\cdot a_{j,k}^\text{ANY;W},\\
a_{j,k}^\text{MT;W} = \operatorname{MV}(T^{-1} a_{j,k}^\text{ANY;W})\cdot a_{j,k}^\text{ANY;W},\\
a_{j,k}^\text{ANY;UW} = \operatorname{B}(a_{j,k}^\text{ANY;W}),\\
a_{j,k}^\text{MT;UW} = \operatorname{B}(a_{j,k}^\text{MT;W}).
\end{aligned}
\label{eq:agg2}
\end{equation}
The weighted aggregation $a_{j,k}^\text{ANY;W}$ computes the total sum of causality test results. The weighted majority vote $a_{j,k}^\text{MT;W}$ considers the weight of relationships occurring in a majority of sample sets in the number $T$ of temporal sets. The unweighted aggregations $a_{j,k}^\text{ANY;UW}$ and $a_{j,k}^\text{MT;UW}$ reduce the weighted adjacency matrix to a binary one using the binary function $\,\operatorname{B}(v)$ that is $1$ if $v>0$ and $0$ otherwise.
If we do not model the directed behaviour of the system, we can simplify this to an undirected matrix via $a_{j,k}=\operatorname{B}(a_{j,k}+a_{k,j})$.

%The resulting $N \times N \times \tau_{max}$ causal process graph contains each node's detected relationships at each lag. To convert this into a causal adjacency matrix that we can integrate into the TGCN architecture, we collapse this into an $N \times N$ directed adjacency matrix by summing on the lag axis. 

%For problems that have been spatially decomposed, we introduce a matrix aggregation step where the full $N \times N$ adjacency matrix is populated with the results of each subproblem, and this is followed by a temporal aggregation if applicable. 
% This result is then symmetric-normalised to produce a final weighted adjacency matrix.

\subsection{Evaluating Performance}\label{sec:prediction}
Forecasting is a common task in spatiotemporal applications for monitoring, anomaly detection, optimisation, etc. %Consequently, it is a key development area for many scientists and practitioners and a key task on which to evaluate the performance of our method.
We consider the task of windowed forecasting, where a model is trained to infer the subsequent $q$ measurements %$[\mathcal{X}_{t+1},...,\mathcal{X}_{t+q}]$
given a finite history of length $\lambda$.

Despite the growing popularity of sequence-to-sequence learners for forecasting applications, we consider the foundational case of a simple one-dimensional convolution along the time axis to capture the temporal behaviour of the system. This has the dual benefit of reducing training overhead for improved scalability and clearly demonstrating the effectiveness of our method even with a simplistic architecture. The simpler network structure also enables explainability \cite{GNNExplainer} that is important for practical applications.

We design a forecasting architecture which begins with a temporal convolution, followed by the causal graph convolution layer. We posit that this ordering allows the temporal dynamics of the system to be captured at each node, and then these distilled features can be used to inform forecasting at causally-related nodes.

% These temporal features are then convoluted over the causal graph which we calculate using the method in the previous section. We posit that this convolutional layer over the detected causal graph improves forecasting performance by encouraging the model to learn functional relationships which adhere to the true underlying causal relationships between variables.

The TGCN we employ builds on \citet{KipfW17} and is implemented in PyTorch Geometric \cite{pyg}. A final linear layer produces a forecast of length $q$. 
The model is trained in batches using RMSE loss. Hyperparameters are tuned for each dataset using the adaptive grid search method provided by Optuna \cite{optuna}.


% -------------------------------------------
\section{Experiments}
We demonstrate the performance of our method in two contexts, selected to demonstrate the generalisability of the approach.
% to heterogeneous, multi-scale datasets from various disciplines. Appendix Table \ref{tab:params} summarises all parameters used for causal discovery and learning.
Experiments were conducted on an Apple M1 Max 32GB MacBook Pro (2021) running MacOS Monterey 12.6 and Python 3.9.12.

Code to replicate our experiments is provided at \url{https://tinyurl.com/ctgcn-code} and will be open-sourced


\subsection{Building Heating System}
The first dataset is heterogeneous and multivariate, and generated from a simulated heating, ventilation, air conditioning (HVAC) system which controls the temperature of two rooms and an interconnecting corridor \cite{ploennigs2017semantic}. It consists of data from thirty sensors measuring the system variables including the internal status of the system's boiler, chiller and ventilation units, the temperature within each room and externally, and room occupancy. The data is sampled every 10 minutes over a total period of one year.

We included this dataset in our experiments as, due to its simulated nature, we have a ground truth about the causal relationships which we can use to investigate the quality of the causal discovery, as well as to measure the performance of the CTGCN on the downstream forecasting task. The dataset is further strongly heterogenous with datapoints representing sensors with different characteristics and value ranges. It is known that this is a challenging task for GCNNs \cite{zhao2021heterogeneous,ba2022automated}, and may also adversely effect the causal discovery, rendering this a particularly interesting study.

The ground truth adjacency matrix is defined based on the underlying physics of the simulation \cite{wetter2014modelica}, and contains 50 relationships between the 30 sensors.

For this dataset we evaluate performance for both temporal and spatiotemporal decomposition in terms of the accuracy of causal discovery, the performance of forecasting the temperature of one of the rooms, and the runtime.

Our temporal decomposition consists of a daily temporal split with $P_T = 144$ based on auto-correlation analysis, yielding 365 causal sub-problems. We select one hour ($\tau_{max}=6$) as the maximal time lag based on system dynamics. For the spatial decomposition we use 10 clusters after running an elbow test.

\subsection{Highway Traffic Flow}
We further verify our method on homogeneous real-world traffic flow data collected by the California Department of Transport. The data consists of 30-second traffic flow speed measurements collected from 228 locations across the California state highway system over weekdays in May and June 2012. We downsample the data to five-minute intervals to align with work by \citet{Yu2018SpatioTemporalGC} and facilitate comparison. In that work, Yu et al. also proposes a distance-thresholding mechanism to identify the adjacency matrix
\begin{equation}
w_{ij} =
\begin{cases}
    exp\!\!\left(-\frac{d^2_{i,j}}{\sigma^2}\right)\!,&\!\!\text{if } i \neq j \text{ and } exp\!\!\left(-\frac{d^2_{i,j}}{\sigma^2}\right)\!\geq\!\epsilon;\\
    0,              &\!\!\text{otherwise}.
\end{cases}
\label{eq:distance-adj}
\end{equation}
We evaluate the effectiveness of our method against their results using their parameters $\sigma^2 = 10$ and $\epsilon = 0.5$.

To overcome the non-stationarity of the data, we split the data by date, conducting causal discovery for each of the 44 days of data. We also conduct spatial decomposition with 25 clusters estimated using an elbow test. The maximal lag we consider is $\tau_{max} = 9$ to align with \citet{Yu2018SpatioTemporalGC}.

For this dataset, we demonstrate the runtime improvement of spatiotemporal decomposition over temporal, and evaluate the compromise between runtime and prediction accuracy.

% \subsection{Salmon Louse Transmission}
% We explore the scalability of our method on data from 579 Norwegian salmon farms, containing information about the number of salmon lice measured weekly at each location over a period of eleven years from 2012 to 2023, and extracted from the BarentsWatch portal \cite{barentswatch}.

% Parasite infestation such as sea lice are one of the most serious threats to the aquaculture industry \cite{shinn2015economic}. Sea cages with high fish densities offer an ideal habitat for parasites to thrive and reproduce. Sea lice infestations occur rapidly, cannot be predicted reliably, and techniques to ameliorate infestations are limited \cite{costello2009global}.
% Parasite control treatment in salmon farms constitutes 7.5\% of total production costs \cite{odonncha2019precision}.
% Parasite control is made harder when several farms occupy the same region, such as the same fjordic enclosure, particularly if farms from different companies fail to coordinate their anti-parasite treatments \cite{morro2022offshore}. Instead, accurate prediction allows farms to implement avoidance rather than mitigation strategies.

% Similar to the traffic flow dataset, we calculate a weighted adjacency matrix $W$ using Equation \ref{eq:distance-adj} to act as a performance benchmark. Parameters $\sigma^2$ and $\epsilon$ are set to 1 and 0.99 respectively. We compare the RMSE of the benchmark to our causal adjacency matrix when forecasting louse infections over a period of eight weeks.

% We temporally decompose the data into eleven yearly sub-problems, within which the farms are further decomposed into 55 clusters. Due to the larger number of variables and correspondingly high computational cost we do not examine the temporally decomposed case.


%Additional scaling experiments were conducted on a cluster of x86-64 Intel(R) Xeon(R) Platinum 8260 CPU @ 2.40GHz 32GB Virtual Machine running CentOS Linux 7 (Core) and Python 3.9.12.

% -------------------------------------------
\section{Results}
\subsection{Building Heating System}
We first compare the ground-truth with the temporal TGCN and the spatiotemporal (DTW clustered) TGCN approach with different aggregation strategies. Table~\ref{tab:results-building} shows the precision, accuracy and F1 score for the different approaches outlined in Section \ref{sec:adj}.
% The aggregation approach AVG takes the mean accuracy, precision and F1 score considering each day as a sample, the MT aggregation uses the Majority Threshold approach, and ANY considers a causal relationship to exist if discovered on any day.

The accuracy of all approaches is higher than 83\,\% due to a high true negative rate as the association matrix is sparse, hence precision is a more relevant metric. Precision is only 14.2\,\% to 16.6\,\% for the temporal approach, which may be surprising as this approach should be able to discover all potential causal relationships. However, this results in a high false positive rate with about 150 causal relationships discovered. The spatiotemporal approach has a significantly higher precision with 32.8\,\% to 39.2\,\% discovering an average of 25 relationships. Despite this, the method is not able to discover all relationships due to the heterogenous nature of the dataset and our spatial decomposition. We see that data are grouped semantically by the DTW clustering step, e.\,g.\ separate clusters are created for temperature, CO2, occupancy and humidity. As our algorithm does not consider causal relationships to exist between clusters, we miss these relationships (See Appendix~\ref{sec:graph} for details). Nonetheless, the spatiotemporal approach has a higher accuracy, precision and a significantly lower compute time. The temporal approach took 288 hours (12 days) to compute, while the spatiotemporal approach computed in 49 hours (2 days, see Table~\ref{tab:runtimes}).

The matrix aggregation step also improves the result. One performance improvement approach could be to sample individual days from the dataset to estimate the adjacency matrix. This approach (AVG) has a mean precision of 32.8\,\%. But, analysing the full dataset and applying the Majority Threshold (MT) or the ANY aggregation we improve precision to 38.4\,\% and 39.2\,\%, respectively.

\begin{table}[t]
\caption{Results for the building dataset.}
\label{tab:results-building}
\begin{center}
\begin{small}
\begin{sc}
\vskip -0.15in
\begin{tabular}{{p{0.41\columnwidth} p{0.14\columnwidth}  p{0.14\columnwidth} p{0.10\columnwidth}}} 
\toprule
Approach          & Precision & Accuracy & F1\\
\midrule
Temporal AVG   & 14.2\,\%  & 83.1\,\%   & 14.1\,\% \\
Temporal  MT   & 16.6\,\%  & 84.7\,\%   & 15.2\,\% \\
%Temporal 1/3V  & 17.4\,\%  & 84.3\,\%   & 16.6\,\% \\
Temporal ANY   & 15.2\,\%  & 83.9\,\%   & 14.6\,\% \\
\midrule
Spatiotemporal   AVG   & 32.8\,\%  & 88.2\,\%   & 21.9\,\% \\
Spatiotemporal   MT    & 38.4\,\%  & 88.8\,\%   & 26.3\,\% \\
%Spatiotemporal   1/3V  & 35.7\,\%  & 88.4\,\%   & 25.6\,\% \\
Spatiotemporal   ANY   & 39.2\,\%  & 88.8\,\%   & 28.2\,\% \\
\bottomrule
\end{tabular}
\end{sc}
\end{small}
\end{center}
\vskip -0.2in
\end{table}

We further compare the prediction performance of the discovered adjacency matrix to the ground truth when forecasting the room temperature. This has multiple causal relationships and relevant practical applications in energy management. As detailed in Section \ref{sec:prediction}, we utilise a CTGCN that is configured with the different association matrices resulting from the temporal and spatiotemporal clustering and the different aggregation methods for prediction. Table~\ref{tab:pred-results-building} compares the results. We provide also the RMSE for an unconnected and fully connected TGCN for context.

The RMSE for the ground truth prediction is significantly better than the unconnected and fully connected TGCN demonstrating the well-known benefits of using domain knowledge in TGCN configuration.
The unweighted temporal and spatiotemporal CTGCN outperform the unconnected TGCN but not the ground truth. This is expected given that not all causal relationships were discovered. However, the temporal and spatiotemporal CTGCN is able to outperform the ground truth TGCN when weighted by frequency. The frequency clearly encodes important information about the relevance of a causal relationship that is not contained in the binary ground truth, and these weights allow the TGCN to compensate for the missing edges. In result, the spatiotemporal weighted majority threshold had the lowest RMSE, showing that our CTGCN can outperform the ground truth in a prediction problem.

\begin{table}[t]
\caption{Comparison of the prediction accuracy of different aggregation methods for the building dataset.}
\label{tab:pred-results-building}
\begin{center}
\begin{small}
\begin{sc}
\vskip -0.15in
\begin{tabular}{lccc}
\toprule
Approach & RMSE\\
\midrule
Ground Truth TGCN                      & \textbf{0.8297}\\
\midrule
Unconnected TGCN                      & 0.8981\\
Fully connected TGCN                    & 0.8548\\
\midrule
Temporal MT Unweighted CTGCN             & 0.8668\\
Temporal MT Weighted CTGCN             & 0.8469\\
Temporal ANY Unweighted CTGCN            & 0.8735\\
Temporal ANY Weighted CTGCN              & \textbf{0.8250}\\
\midrule
Spatiotemporal MT Unweighted CTGCN  & 0.8668\\
Spatiotemporal MT Weighted CTGCN   & \textbf{0.8209}\\
Spatiotemporal ANY Unweighted CTGCN & 0.8306\\
Spatiotemporal ANY Weighted CTGCN  & 0.8735\\
\bottomrule
\end{tabular}
\end{sc}
\end{small}
\end{center}
\vskip -0.2in
\end{table}



\subsection{Highway Traffic Flow}
%\subsubsection{Temporal Decomposition}
% [INFO / STATS ABOUT THE CAUSAL GRAPH - DENSITY, EDGE COUNT, MAP PLOT?]

% Runtimes therefore limited number of samples. 1-5, 20-24 and 40-44 to sample throughout dataset

The mean runtime of each day of the temporally-decomposed traffic data was 35 hours, corresponding to an estimated total runtime of more than 64 days (Table \ref{tab:runtimes}). As such, causal discovery was not conducted for every temporal split with this method. Instead, we selected the first, last, and central five days of data to get an overview of the entire period and include variations in causal relationships over time. Forecasting performance was measured across the entire dataset to test the representativity of the causal discovery method when forecasting on unseen data.%, and for consistency with the spatial decomposition results.

% \begin{table}[t]
% \caption{Statistics summarising the density and mean degree of the calculated temporal-decomposed causal graphs. [update with new graphs ???]}
% \label{tab:temporal-stat}
% \vskip 0.15in
% \begin{center}
% \begin{small}
% \begin{sc}
% \begin{tabular}{lccc}
% \toprule
% & Density & Mean Degree\\
% \midrule
% Benchmark & 0.03640 & 8.29825\\
% \midrule
% Min  & 0.01120 & 2.55263\\
% Mean & 0.01188 & 2.70760\\
% Max  & 0.01493 & 3.40351\\
% \bottomrule
% \end{tabular}
% \end{sc}
% \end{small}
% \end{center}
% \vskip -0.1in
% \end{table}

% ??? The resulting graphs were sparser and with lower connectivity than the distance benchmark, as shown in Table \ref{tab:temporal-stat}.

As evidenced in Fig.~\ref{fig:decomp-performance} and Appendix Table \ref{tab:pred-results-traffic}, downstream performance is significantly improved by introducing the temporally decomposed causal adjacency matrix into the TGCN architecture. Forecasting RMSE was on average 30.4\% improved over the benchmark, with the best-performing and worst-performing days yielding 33.5\% and 23.6\% improvements respectively.

% \begin{table}[t]
% \caption{Comparison of the downstream accuracy of different aggregation methods.}
% \label{tab:temporal-agg}
% \vskip 0.15in
% \begin{center}
% \begin{small}
% \begin{sc}
% \begin{tabular}{lccc}
% \toprule
% Aggregation & RMSE\\
% \midrule
% Frequency Weighted              & ???\\
% Unweighted                      & ???\\
% Majority Threshold Weighted     & ???\\
% Majority Threshold Unweighted   & ???\\
% \bottomrule
% \end{tabular}
% \end{sc}
% \end{small}
% \end{center}
% \vskip -0.1in
% \end{table}

% In order to combine the causal graphs from individual days and overcome this variation in performance, four aggregation methods were compared: weighting edges by the frequency of their occurrence, including all edges that are present in any causal graph equally weighted, and applying the previous two methods to only edges present in at least half of the tested days. Each method was evaluated on the same forecasting task, and the results are given in Table \ref{tab:temporal-agg}.

To combine the causal graphs calculated from different days of data, the majority threshold aggregation method was used. This aggregation improves the performance of the temporal decomposition, resulting in an RMSE 36.0\,\% lower than the benchmark, and lower than any of the daily graphs. This result is notable as it demonstrates that by combining causal graphs from different sections of the data we can better capture the overall causal relationships.

To compare our results to a state-of-the art approach, we added the STGCN results from \citet{Yu2018SpatioTemporalGC}. Note that the RMSE of the same distance based benchmark is higher than the STGCN due to our simpler sequence-to-sequence learner. But, the improved adjacency matrix of the Temporal MT graph outperforms even the more advanced STGCN.

%\subsubsection{Spatial Decomposition}

\begin{table}[t]
\caption{Comparison of causal discovery runtimes in hours for the datasets.}
\label{tab:runtimes}
\begin{center}
\begin{small}
\begin{sc}
\vskip -0.15in
\begin{tabular}{p{0.2\columnwidth} p{0.2\columnwidth} p{0.23\columnwidth} p{0.15\columnwidth}}
\toprule
Dataset            & Temporal  & Spatiotemporal & Factor\\
\midrule
%Building          & 00:47\,h  & 00:08\,h & 5.8x \\
Building           & 287.7\,h  & 49.8\,h & 5.8\,x \\
%Traffic          & 35.05\,h  & 2.42\,h & 14.5x \\
Traffic       & 1540.0\,h  & 106.2\,h & 14.5\,x \\
%Salmon       & - & 2.42\,h & 14.5x \\
% Salmon$^*$       & -  & 11.9\,h & - \\
\bottomrule
\end{tabular}
\end{sc}
\end{small}
\end{center}
\vskip -0.25in
\end{table}


% \begin{table}[t]
% \caption{Statistics summarising the density and mean degree of the calculated clustered causal graphs.}
% \label{tab:temporal-spatial-stat}
% \vskip 0.15in
% \begin{center}
% \begin{small}
% \begin{sc}
% \begin{tabular}{lccc}
% \toprule
% & Density & Mean Degree\\
% \midrule
% Benchmark & 0.03640 & 8.29825\\
% \midrule
% Min  & 0.00773 & 1.76316\\
% Mean & 0.00858 & 1.95534\\
% Max  & 0.01020 & 2.32456\\
% \bottomrule
% \end{tabular}
% \end{sc}
% \end{small}
% \end{center}
% \vskip -0.1in
% \end{table}

\begin{figure}[t]
% \vskip 0.2in
\begin{center}
\centerline{\includegraphics[width=0.9\columnwidth]{PEMSD7.pdf}}
\vskip -0.15in
\caption{Forecasting RMSE on the traffic dataset compared to STGCN \cite{Yu2018SpatioTemporalGC}}
\label{fig:decomp-performance}
\end{center}
\vskip -0.3in
\end{figure}

As shown in Table \ref{tab:runtimes}, introducing spatial decomposition accelerates causal discovery on this data by more than 14 times. 
%\textcolor{red}{As summarised in Table \ref{tab:temporal-spatial-stat}, the density and mean degree of the resulting graphs are reduced as compared to Table \ref{tab:temporal-stat}, which is explained by the absence of cross-cluster connections in the resulting graphs [---THESE TWO TABLES SHOULD POINT TO TABLE \ref{tab:runtimes}? -- they point to the currently commented out Density \& Mean Degree table - could go in appendix?]}.
Despite this, Fig.~\ref{fig:decomp-performance} demonstrates that predictive performance was not negatively affected, with performance improving in some cases. The minimum, mean and maximum improvement over the benchmark are 26.1\%, 33.5\% and 37.4\% respectively.
Spatiotemporal performance is further improved through MT aggregation, with RMSE 39.1\% lower than the benchmark and 4.9\% lower than Temporal MT.


% \subsection{Salmon Louse Transmission}
% The introduction of spatiotemporal decomposition on the salmon dataset makes the problem tractable. Causal discovery runs within twelve hours, with each of the eleven years taking on average 1 hour.

% The accuracy of the forecast also improves, with the average RMSE of a two-month forecast nearly 2\,\% lower than the best-performing distance-based benchmark graph as evidenced by Fig.~\ref{fig:salmon-performance} and Table \ref{tab:pred-results-salmon}. We validate the trend identified in the traffic data, where the MT aggregation improves performance. In this case, the MT aggregation yields a performance improvement of 3.6\,\% over the distance benchmark and 1.0\,\% over the mean spatiotemporal.

% We suggest that the performance improvement is smaller with this data as compared to previous case studies due to the highly dynamic nature of the system and relatively low temporal resolution which may not capture all of the dynamics. We assume that performance can be improved in future work with a more complex sequence-to-sequence learner that can better capture these dynamics.


% \begin{figure}[t]
% %\vskip 0.2in
% \begin{center}
% \centerline{\includegraphics[width=0.9\columnwidth]{Figures/Sealice.pdf}}
% \caption{Forecasting RMSE on the salmon dataset}
% \label{fig:salmon-performance}
% \end{center}
% \vskip -0.3in
% \end{figure}

% -------------------------------------------
\section{Discussion}

Our experiments have shown that using an automatically discovered causal adjacency matrix with a TGCN architecture can improve the forecasting performance over typical spatial approaches. Further, we demonstrate that through context-aware decomposition of the causal discovery problem we accelerate the discovery process such that previously intractable problems can be computed on a single machine. This is particularly notable for the traffic dataset, which is accelerated 14 times when the problem is both spatially and temporally decomposed over a temporal decomposition.

% Shallower Model
Importantly, our approach reports high predictive skill with a simplistic sequence-to-sequence learner. Due to the lightweight architecture---constructed of a simple 1-D convolution and a graph convolution layer---we posit that the predictions from this model are more interpretable than more complex SOTA architectures \cite{zeng2022transformers}. Simple weight visualisation or gradient based approaches can provide efficient insight into model performance for such 1-layer systems \cite{nguyen2019understanding}.  
A notable advantage of the proposed framework is that a causal model accurately representing underlying system dynamics allows us to retain performance with the lightweight architecture.

% More robust representation of dynamics
Causal discovery can generate a more practical representation of  latent and contemporaneous relationships. The power of GNNs has been demonstrated across many applications in recent years such as estimating travel times in Google Maps, powering content recommendations in Pinterest, and providing product recommendations in Amazon \cite{velivckovic2023everything}. We propose an agnostic and scalable framework for effective graph generation for such applications. 

% Decomposition provides advantages
We also illustrate that the method used to aggregate causal results from sub-problems is important, with the weighted majority threshold method yielding performance improvements over any of its constituent graphs. We posit that this is due to its ability to to capture information about the relevance of most common causal relationships in the data and filters non-stationary relationships.

However, our results are not exhaustive: we suggest that by implementing a more balanced clustering method, we might see improved precision. Also, the scalability remains a problem for very large datasets ($N >> 1000$) with dense causal models where the spatial decomposition capability is limited. There also remains opportunity to expand this work toward the explainability of the CTGCN model.

% There is opportunity to extend this work by integrating a more powerful sequence-to-sequence learner, such as LSTM or GRU, into the CTGCN architecture that can better capture complex temporal dynamics. 


% -------------------------------------------
\section{Conclusion} 
While TGCN demonstrates excellent forecasting skill across a wide variety of applications, the need for manually configured adjacency matrices hinder large-scale deployments. We demonstrate that causal discovery methods generate a more robust graph structure that better capture system dynamics, and demonstrate improved predictive skill on two real-world datasets and in comparison to state of the art.
We address computational scalability and non-stationarity by implementing efficient spatial and temporal decomposition.

We show that CTGCN demonstrates particularly impressive performance when data is decomposed based on stationarity, causal relationships are calculated for each sub-problem, and aggregated to incorporate these temporal dynamics.
% Future work in the space is in the direction of a fully automatic pipeline for causal TGCN by identifying decomposition parameters automatically.
Future work in this space is in the direction of improving the clustering method, automatic parameter identification, and an evaluation of explainability.

% -------------------------------------------
% \begin{figure}[ht]
% \vskip 0.2in
% \begin{center}
% \centerline{\includegraphics[width=\columnwidth]{icml_numpapers}}
% \caption{Historical locations and number of accepted papers for International
% Machine Learning Conferences (ICML 1993 -- ICML 2008) and International
% Workshops on Machine Learning (ML 1988 -- ML 1992). At the time this figure was
% produced, the number of accepted papers for ICML 2008 was unknown and instead
% estimated.}
% \label{icml-historical}
% \end{center}
% \vskip -0.2in
% \end{figure}


% \begin{algorithm}[tb]
%    \caption{Bubble Sort}
%    \label{alg:example}
% \begin{algorithmic}
%    \STATE {\bfseries Input:} data $x_i$, size $m$
%    \REPEAT
%    \STATE Initialize $noChange = true$.
%    \FOR{$i=1$ {\bfseries to} $m-1$}
%    \IF{$x_i > x_{i+1}$}
%    \STATE Swap $x_i$ and $x_{i+1}$
%    \STATE $noChange = false$
%    \ENDIF
%    \ENDFOR
%    \UNTIL{$noChange$ is $true$}
% \end{algorithmic}
% \end{algorithm}


% \begin{table}[t]
% \caption{Classification accuracies for naive Bayes and flexible
% Bayes on various data sets.}
% \label{sample-table}
% \vskip 0.15in
% \begin{center}
% \begin{small}
% \begin{sc}
% \begin{tabular}{lcccr}
% \toprule
% Data set & Naive & Flexible & Better? \\
% \midrule
% Breast    & 95.9$\pm$ 0.2& 96.7$\pm$ 0.2& $\surd$ \\
% Cleveland & 83.3$\pm$ 0.6& 80.0$\pm$ 0.6& $\times$\\
% Glass2    & 61.9$\pm$ 1.4& 83.8$\pm$ 0.7& $\surd$ \\
% Credit    & 74.8$\pm$ 0.5& 78.3$\pm$ 0.6&         \\
% Horse     & 73.3$\pm$ 0.9& 69.7$\pm$ 1.0& $\times$\\
% Meta      & 67.1$\pm$ 0.6& 76.5$\pm$ 0.5& $\surd$ \\
% Pima      & 75.1$\pm$ 0.6& 73.9$\pm$ 0.5&         \\
% Vehicle   & 44.9$\pm$ 0.6& 61.5$\pm$ 0.4& $\surd$ \\
% \bottomrule
% \end{tabular}
% \end{sc}
% \end{small}
% \end{center}
% \vskip -0.1in
% \end{table}


% -------------------------------------------
% This must be in the first 5 lines to tell arXiv to use pdfLaTeX, which is strongly recommended.
\pdfoutput=1
% In particular, the hyperref package requires pdfLaTeX in order to break URLs across lines.

\documentclass[11pt]{article}

% Remove the "review" option to generate the final version.
%\usepackage[review]{ACL2023}
\usepackage{ACL2023}

% Standard package includes
\usepackage{times}
\usepackage{latexsym}

% For proper rendering and hyphenation of words containing Latin characters (including in bib files)
\usepackage[T1]{fontenc}
% For Vietnamese characters
% \usepackage[T5]{fontenc}
% See https://www.latex-project.org/help/documentation/encguide.pdf for other character sets

% This assumes your files are encoded as UTF8
\usepackage[utf8]{inputenc}

% This is not strictly necessary, and may be commented out.
% However, it will improve the layout of the manuscript,
% and will typically save some space.
\usepackage{microtype}

% This is also not strictly necessary, and may be commented out.
% However, it will improve the aesthetics of text in
% the typewriter font.
\usepackage{inconsolata}


% If the title and author information does not fit in the area allocated, uncomment the following
%
%\setlength\titlebox{10cm}
%
% and set <dim> to something 5cm or larger.

%%%%%%%%%%%%%%%%%%%%%%%%%%%%%%%%%%
\usepackage{graphicx}
\usepackage{amsfonts}
\usepackage{amsmath}
\usepackage{bigdelim}
\usepackage{diagbox}
\usepackage{amsthm}
\usepackage{makecell}
\usepackage{mathtools}
\usepackage{booktabs}
\usepackage[shortlabels]{enumitem}
\graphicspath{ {figs/} }

\theoremstyle{remark}
\newtheorem*{question}{Question}

\newcommand{\tk}[1]{\textcolor{blue}{{#1}}}
\newcommand{\sy}[1]{\textcolor{red}{{#1}}}
\newcommand{\mg}[1]{\textcolor{purple}{{#1}}}
\newcommand{\lh}[1]{\textcolor{green}{{#1}}}
\newcommand{\lc}[1]{\textcolor{green}{{#1}}}

% Rounded color box
\definecolor{light_blue}{HTML}{cfdfff}
\usepackage[most]{tcolorbox}
\tcbset{on line, 
        boxsep=1pt, left=0pt,right=0pt,top=0pt,bottom=0pt,
        colframe=white,colback=light_blue,  
        highlight math style={enhanced}
        }

\newcommand{\quash}[1]{}  %Anything in \quash is ignored
\newcommand{\gpt}{\textsc{GPT-2}}
\newcommand{\bert}{\textsc{BERT}}
\newcommand{\bertlarge}{\textsc{BERT-large}}
\newcommand{\mask}{\texttt{[MASK]}}
\newcommand{\cls}{\texttt{[CLS]}}
\newcommand{\sep}{\texttt{[SEP]}}
\newcommand{\mat}{\texttt{mat}}
\newcommand{\id}{\texttt{id}}
\newcommand{\matl}{\texttt{mat}_{\ell \rightarrow \ell'}}
\newcommand{\matattnl}{\texttt{mat\_attn}_{\ell \rightarrow \ell'}}
\newcommand{\matffl}{\texttt{mat\_ffn}_{\ell \rightarrow \ell'}}
\newcommand{\matlnl}{\texttt{mat\_ln1\_ln2}_{\ell \rightarrow \ell'}}
\newcommand{\idl}{\texttt{id}_{\ell \rightarrow \ell'}}
\newcommand{\matlL}{\texttt{mat}_{\ell \rightarrow L}}
\newcommand{\matattnlL}{\texttt{mat\_attn}_{\ell \rightarrow L}}
\newcommand{\matfflL}{\texttt{mat\_ffn}_{\ell \rightarrow L}}
\newcommand{\matlnlL}{\texttt{mat\_ln1\_ln2}_{\ell \rightarrow L}}
\newcommand{\idlL}{\texttt{id}_{\ell \rightarrow L}}

\definecolor{blue(munsell)}{rgb}{0.0, 0.5, 0.69}
%%%%%%%%%%%%%%%%%%%%%%%%%%%%%%%%%%

\title{Jump to Conclusions: Short-Cutting Transformers\\With Linear Transformations}

% Author information can be set in various styles:
% For several authors from the same institution:
% \author{Author 1 \and ... \and Author n \\
%         Address line \\ ... \\ Address line}
% if the names do not fit well on one line use
%         Author 1 \\ {\bf Author 2} \\ ... \\ {\bf Author n} \\
% For authors from different institutions:
% \author{Author 1 \\ Address line \\  ... \\ Address line
%         \And  ... \And
%         Author n \\ Address line \\ ... \\ Address line}
% To start a seperate ``row'' of authors use \AND, as in
% \author{Author 1 \\ Address line \\  ... \\ Address line
%         \AND
%         Author 2 \\ Address line \\ ... \\ Address line \And
%         Author 3 \\ Address line \\ ... \\ Address line}

\author{Alexander Yom Din$^{1}$ ~~~~~ Taelin Karidi$^{1}$ ~~~~~ Leshem Choshen$^{1}$ ~~~~~
Mor Geva$^{2}$ 
\vspace{0.2cm} \\
$^1$Hebrew University of Jerusalem ~~~ $^2$Google Research \\
\small{\texttt{\{alexander.yomdin, taelin.karidi, leshem.choshen\}@mail.huji.ac.il}}, \small{\texttt{pipek@google.com}}}

\quash{
\author{Alexander Yom Din \\
  Hebrew University of Jerusalem \\ \texttt{alexander.yomdin@mail.huji.ac.il} \\\And
  Taelin Karidi \\
  Hebrew University of Jerusalem \\
  \texttt{taelin.karidi@mail.huji.ac.il} \\\And
  Leshem Choshen \\
  Hebrew University of Jerusalem \\ \texttt{leshem.choshen@mail.huji.ac.il} \\\And
  Mor Geva \\
  Google Research \\
  \texttt{pipek@google.com} \\}
}

\begin{document}
\maketitle



\begin{abstract}
% \vspace{-1em}
The diffusion-based generative models have achieved remarkable success in text-based image generation. However, since it contains enormous randomness in generation progress, it is still challenging to apply such models for real-world visual content editing, especially in videos. 
In this paper, we propose \texttt{FateZero}, a zero-shot text-based editing method on real-world videos without per-prompt training or use-specific mask. 
\RM{Specifically, different from a pipeline of two independent inversion and then generation stages, we find the intermediate attention maps during inversions store better structure and motion information. We thus reform them to temporally casual attention and replace them in the generation progress. To further reduce the unnecessary semantic leakage of source video and enhance the editing quality, we then remix the temporally casual attentions via the cross-attention features of the source prompt as the mask.}
To edit videos consistently, we propose several techniques based on the pre-trained models. Firstly, in contrast to the straightforward DDIM inversion technique, our approach captures intermediate attention maps during inversion, which effectively retain both structural and motion information. These maps are directly fused in the editing process rather than generated during denoising. To further minimize semantic leakage of the source video, we then fuse self-attentions with a blending mask obtained by cross-attention features from the source prompt. Furthermore, we have implemented a reform of the self-attention mechanism in denoising UNet by introducing spatial-temporal attention to ensure frame consistency.
Yet succinct, our method is the first one to show the ability of zero-shot text-driven video style and local attribute editing from the trained text-to-image model. We also have a better zero-shot shape-aware editing ability based on the text-to-video model~\cite{tuneavideo}. \RM{Besides video, our unified method also achieves state-of-the-art performance in zero-shot image editing.\chenyang{Need exp or remove the zero-shot image}} Extensive experiments demonstrate our superior temporal consistency and editing capability than previous works.
% The code will be released.
% \chenyang{emphasize: our observation at inversion time} \xiaodong{replacing the bold part to the actual pipeline: \textbf{Specifically, we work on replacing and mixing the attention maps between the inversion and generation since the self-attention map keeps the structure of the original natural image and the cross-attention is semantic-related, after remixing, we replace them in the corresponding generation steps for denoising.}}
% \footnote{Since there is no general video diffusion model is publicly available, we use one-shot video generation method~(Tune-A-Video~\cite{tuneavideo}) as the pretrained video diffusion model for zero-shot video editing\xiaodong{can be removed if we actually zero-shot on video}.}.
\end{abstract}
\section{Introduction}

The ability to reason about plans is critical for performing long-horizon tasks \citep{erol1996hierarchical, sohn2018hierarchical, sharma-etal-2022-skill}, compositional generalization \citep{corona-etal-2021-modular} and generalization to unseen tasks and environments \citep{shridhar2020alfred}.
Consider a simple long-horizon planning scenario where a robot is tasked with preparing a meal and serving it on the table. 
This presents a non-trivial planning problem since the agent needs to understand the sequence of operations required to perform the task and search for the relevant objects in the unfamiliar environment by interacting with various objects. %



Large language models have been recently shown to possess commonsense knowledge about the world such as object affordances and physical dynamics \citep{ouyang2022training,chowdhery2022palm}.
Early approaches considered text based environments and fine-tuned PLMs to predict actions given the history of past observations and actions \citep{jansen-2020-visually,micheli-fleuret-2021-language,yao-etal-2020-keep}.
Recent work has used this ability to reason about plans from text instructions in simulated household environments with simplifying assumptions such as text-only environment observations or feedback \citep{huang2022language,ahn2022can,li2022pre,logeswaran-etal-2022-shot}.


We focus on \emph{visually grounded planning} with PLMs --- the ability to adapt plans based on interaction and visual feedback from the environment.
While PLMs have strong planning commonsense priors, predictions from a PLM may not be directly realizable in the environment since the observation and action spaces are unknown.
This requires \emph{grounding} the PLM in the environment and adapting it to observe visual feedback, which is highly non-trivial.
Some prior works assume the availability of a pre-trained affordance function \citep{ahn2022can} or a success detector \citep{mirchandani2021ella}.
Notably, SayCan \citep{ahn2022can} completely decouples the PLM from observation information by selecting actions that have both high affordability (through a pre-trained affordance model) and high PLM likelihood.
Although this partially addresses the grounding problem, the use of visual feedback for action affordance alone is limited.
Often an agent must choose one of many affordable actions using information from observations.
For example, a driving agent should re-navigate and possibly turn around when encountering a ``road closed'' sign, but both turning around and driving forward are indistinguishable to SayCan because they are both affordable and the PLM is blind to observations.

Another workaround explored in prior work is translating the information in the visual observations to text using a pre-trained captioning system \citep{shridhar2021alfworld,huang2022language}.
However, it can be difficult to faithfully describe an image in words and information is lost in this inherently noisy process, which limits the information available to the planner.



Recent work shows that PLMs can be adapted for various natural language tasks by inserting tunable embeddings or soft prompts at the input of the PLM (also called prompt tuning or prefix tuning)~\citep{li-liang-2021-prefix,lester-etal-2021-power}.
This approach also extends to multi-modal understanding tasks such as image captioning \citep{mokady2021clipcap} and VQA \citep{tsimpoukelli2021multimodal} where images are encoded as soft prompts and finetuned for the target task.
Transformer based architectures have also been successfully applied to offline Reinforcement Learning in recent work \citep{chen2021decision,janner2021offline,li2022pre,reid2022can}.

Taking inspiration from these works, we propose the simple approach of embedding visual observations (`visual prompts') and \textit{directly inserting them as PLM input embeddings}.
The visual encoder and PLM are jointly trained for the target task, an approach we call \textbf{\oursfull}~(\ours).
By teaching the PLM to use observations for planning in an end to end manner, we remove the dependency on external data such as captions and affordability information that was used in prior work.
We show that this simple approach performs better than prior PLM-based planning approaches on two embodied planning benchmarks based on ALFWorld~\citep{shridhar2021alfworld} and Virtualhome~\cite{puig2018virtualhome}.



\section{Related Work}

%Here we summarize prior work on transfer learning and property inference.

%\shortsection{Transfer Learning}
%%Transfer learning reuses features learned by pre-trained models for new tasks, with the pretext that inherent similarities in the generic features will be useful for the downstream tasks and hence reducing their cost of downstream training. Specifically, the downstream model trainer will use a pre-trained upstream model as the starting point for the downstream training, with inclusion of (or replacement with) the task-specific classification layer/module. The downstream model is then trained by either updating all layers of the model (including ones reused from upstream model) or freezing some earlier layers of the reused parts as the ``feature extractor'' and only updating the rest. The latter approach is more popular as the reused feature extractors can already learn useful feature representations and the training cost is also much lower and affordable for individuals with limited computational resources. We study the vulnerability of the latter transfer learning approach in this paper. 


%\shortsection{Transfer Learning} 
Several works have demonstrated risks associated with transfer learning across a variety of attack goals. Wang et al.~\cite{wang2018great} and Yao et al.~\cite{yao2019latent} consider manipulating the upstream model such that the fine-tuned downstream models contain backdoors, misclassifying test inputs that contain predefined backdoor triggers. These transfer manipulations are tailored to their particular attack goals and cannot be applied for the property inference goal considered in this paper. Zou et al.~\cite{zou2020privacy} study the threat of membership inference attacks on transfer learning, but with normally trained upstream models.  
%\dnote{its clear that the goals are different for these attacks, but how similar are the methods?} \ynote{similarity of the methods? more details about the methods? do not know what is expected here}
%In contrast, we investigate the possibility of boosting the effectiveness of property inference by manipulating the upstream model training. % Schuster et al.~\cite{schuster2020humpty} show that the attacker can modify the corpus on which the word embedding is trained such that the downstream NLP models which use that embedding will behave abnormally.

%\shortsection{Property Inference}
The risk of property inference was introduced by Ateniese et al.~\cite{ateniese2015hacking}, % introduces the threat of inferring properties of the training data from pre-trained models, 
and several subsequent works have developed property inference (also known as distribution inference) attacks~\cite{Wang2022GroupPI, suri2022formalizing, Jurez2022BlackBoxAF, Hartmann2022DistributionIR}.
% Ganju et al.~\cite{ganju2018property} and Suri and Evans~\cite{suri2022formalizing} 
These works study property inference against normally trained models, and they launch attacks using a variety of black-box and white-box attacks. All the white-box attacks use meta-classifiers, which take the permutation-invariant representation~\cite{ganju2018property} of the model parameters as the features. We use the state-of-the-art white-box attack~\cite{suri2022formalizing} in our experiments.
%We will use the state-of-the-art white-box method proposed by Ganju et al.~\cite{ganju2018property} and later extended by suri et al.~\cite{suri2022formalizing} in this paper.
%\dnote{do we use these attacks?} 
Melis et al.~\cite{melis2019exploiting} and Zhang et al.~\cite{zhang2021leakage} focus on property inference in distributed training scenarios. In their settings, the attacker is a participant in the global model training and conducts property inference using meta-classifiers that are trained on model outputs or gradients. Similarly, Suri et al.~\cite{suri2022subject} focus on federated learning settings where the attacker is a participant (or the central server) that utilizes black-box attacks for inferring membership of data from particular subjects. %\dnote{if we use black-box attacks, explain which ones, or how ours are related to previous ones} 
For our experiments, We improve the black-box meta-classifier proposed by Zhang et al.~\cite{zhang2021leakage} using the ``query tuning'' technique in Xu et al.~\cite{xu2019detecting}. 

The closest works to ours are Chase et al.~\cite{saeed} and Chaudhari et al.~\cite{Chaudhari2022SNAPEE}, which both consider a scenario where the attacker can manipulate some of the training data of the model to induce a model that significantly increases property inference risk.
% \dnote{it enables precise property inference attacks?}.
These works assume an adversary with the ability to poison the victim's training data, while the adversary in our scenario has no access to the victim's training data, and therefore, their methods are not applicable.
% \dnote{example how different from ours, and why the methods are not applicable}
%Thus, their methods are not applicable to our transfer learning scenario.
%Their methods rely on inducing certain behavior correlated with the properties to be inferred, and thus are not applicable to our transfer learning scenario. \anote{Still a bit unclear why that is the case.}
%
There are also works similar to ours that leverage ``adversarial initializations'' for attack purposes.
% \cite{grosse2019adversarial, boenisch2021curious, wen2022fishing, fowl2021robbing}.
Grosse et al.~\cite{grosse2019adversarial} focus on scenarios where the attacker can control the parameter initialization of a model, and demonstrate that the attacker can use special initializations to damage the performance of the trained model. %This attack is orthogonal to ours.
Other works \cite{boenisch2021curious, wen2022fishing, fowl2021robbing} show that the malicious central server in a federated learning protocol can reconstruct some training samples via falsifying the global model in some training rounds and then analyzing the submitted gradients. These kinds of attacks do not apply to our transfer-learning scenario since the attacker cannot access the downstream gradients, and can only manipulate the upstream training.

\iffalse %%%%%%%%%%%%%%%%%%%%%%%%%%%%%%%%

In this section, we provide the background and also the summary of prior attacks on transfer learning (Section~\ref{sec:transfer_learning}) and property inference (Section~\ref{sec:property_inference}). Then, we introduce the closely related manipulation attacks against machine learning models to boost different privacy risks in Section~\ref{sec:active_inference_attacks}.

%\anote{Do we really need a dedicated section for this? It's barely 2 paragraphs right now.}

%\dnote{the most closely related work to ours are works that attempt to amplify inference attacks by poisoning models, the two most relevant I know of are \url{https://www.computer.org/csdl/proceedings-article/sp/2022/131600b569/1CIO8nmuota} and \url{https://arxiv.org/abs/2204.00032}, but need to look thoroughly for others. We should definitely be describing this and relating it to our work, probably in the introduction. Most of what is here is Background, but should be clear what this section is for (not muddling background and related work)}

\subsection{Transfer Learning} \label{sec:transfer_learning}
Transfer learning reuses features learned by pre-trained models for new tasks, with the pretext that inherent similarities in generic features can be useful for downstream tasks, thus reducing the cost of downstream training. Specifically, the downstream model trainer uses a pre-trained upstream model as the starting point for downstream training, with the inclusion (or replacement) of task-specific classification layers/modules. The downstream model is then trained by either updating all layers of the model (including ones reused from the upstream model) or freezing some earlier layers of the reused parts as the ``feature extractor'' and only updating the rest. The latter approach is more popular as the reused feature extractors can already learn useful feature representations and the training cost is also much lower and affordable for individuals with limited computational resources. We study the vulnerability of the latter transfer learning approach in this paper. 
%mainly in two ways:  1) all the layers (including ones reused from ) and tune the full model; the other one is to freeze some earlier layers of the model as the feature extractor and only tune the rest later layers. The second update strategy could achieve better efficiency since the frozen layers can already produce meaningful feature representations~\cite{wang2018great,yao2019latent}, and we will study the transfer learning using this strategy. 

Recently, various attacks have been proposed for the transfer learning setting, but with different attack goals from ours. Wang et al.~\cite{wang2018great} generate adversarial examples against black-box student models that transfer knowledge from publicly available teacher models without repeated queries. Yao et al.~\cite{yao2019latent} propose to manipulate the upstream model such that the downstream models derived from the upstream model contain backdoors, which would misclassify test inputs that contain some predefined backdoor triggers. Zou et al.~\cite{zou2020privacy} study the threat of membership inference attacks on transfer learning and the upstream models are trained normally. In contrast, we investigate the possibility of boosting the effectiveness of property inference by manipulating the upstream model training. Schuster et al.~\cite{schuster2020humpty} show that the attacker can modify the corpus on which the word embedding is trained such that the downstream NLP models which use that embedding will behave abnormally.

%This additionally allows model trainers to achieve satisfactory performance with limited training samples, leading to reduced computational costs. The most common approach reuses parameters in the earlier layers of the pre-trained model, either by fixing them as the feature extractor or just using them for initialization, to conduct downstream training.

\subsection{Property Inference} \label{sec:property_inference}

\shortsection{Property Inference Attacks} In property inference attacks, the adversary aims to infer some sensitive properties of some data, given a model trained on it. For example, the adversary may be interested in sensitive properties like the presence of people of a specific race in the dataset~\cite{ateniese2015hacking, melis2019exploiting}), or even be curious about the 
the statistics of the training set (e.g, the ratio of people with a specific gender~\cite{saeed, ganju2018property, suri2022formalizing, zhang2021leakage}).


Ateniese et al.~\cite{ateniese2015hacking} were the first to identify the threat of inferring properties of the training data from pre-trained models. Ganju et al.~\cite{ganju2018property} and Suri and Evans~\cite{suri2022formalizing} 
study property inference against normally trained models, and they launch attacks using white-box meta-classifiers, which utilize the permutation-invariance representation~\cite{ganju2018property} of the model parameters, while other works focus on distributed training~\cite{zhang2021leakage} where the attacker is a participant in the global model training and conducts property inference using meta-classifiers trained on model outputs. Similarly, Suri et al.~\cite{suri2022subject} focus on federated learning, where the attacker is a participant (or the central server) that utilizes black-box attacks for inferring membership of data from particular subjects. Chase et al.~\cite{saeed} propose an active property inference attack for data poisoning scenarios, which we will cover and compare to in Section~\ref{sec:active_inference_attacks}.

%The closest work to ours are by Chase et al.~\cite{saeed} and Tramer et al.~\cite{tramer2022truth}. In their work, the attacker can manipulate some of the training data of the model such that a model trained (from scratch) on the poisoned data has an increased inference risk. However, their methods are not applicable to the transfer learning scenario. 
%In this work, we will focus on the property inference in transfer learning scenarios in which the attacker releases the upstream model and infer sensitive properties of the downstream models tuned from that upstream model.
% 

\shortsection{Defenses}
Defending against property inference attacks is an open problem. There are no studies in the current literature on active adversaries, and only a couple on passive ones. Ma et. al.~\cite{ma2021nosnoop} propose a defense against property inference attacks on data batches in the  collaborative learning setting. However, adversaries in the transfer-learning setting do not have access to batch-wise gradients of the downstream trainer. Chen and Ohrimenko~\cite{chen2022protecting} utilize mechanisms that add carefully-crafted noise to features to provide theoretical guarantees against inference adversaries, but focus on query-based access to the underlying dataset, not a machine learning model trained on it. These existing defenses thus do not apply to our threat model.

%propose a framework that reduces property inference to Boolean functions of individual members, posing the ratio of members satisfying the given function in a dataset as the property. These property inference attacks have since then been proposed as distribution inference attacks~\cite{suri2022formalizing}, presenting such attacks as inferring properties of the distributions used to sample datasets, differentiating them from exact inference attacks like dataset inference~\cite{maini2021dataset}. Nearly all property inference attacks use meta-classifiers to perform inference: training models on versions of datasets with and without the target property, followed by training a meta-classifier on top of these classifiers's model representations. These representations can take several forms: using model weights themselves with permutation-invariance~\cite{ganju2018property}, or model activations or logits for a generated set of query points~\cite{xu2019detecting}. However, the capability of such approaches is limited: the most that these attacks have been shown to work is medium-sized convolutional networks on the CelebA dataset~\cite{suri2022formalizing}.


\subsection{Active Privacy Attacks} \label{sec:active_inference_attacks}
% Perhaps the closely related works to ours as ones that proactively enhance the effectiveness of privacy attacks by manipulating the model training process in certain ways~\cite{saeed, melis2019exploiting, nasr2019comprehensive, tramer2022truth}. 
%shown that the adversary can, by using proactive ways, achieve stronger attacks that infer private information from deep learning systems~\cite{nasr2019comprehensive, melis2019exploiting, tramer2022truth, saeed}. In this section, we introduce the ones that are close to ours.

In the decentralized federated learning training, by submitting specially crafted gradients to the central server, malicious agents can increase membership inference risk~\cite{nasr2019comprehensive} and property inference risks~\cite{melis2019exploiting} of other benign agents' training data. However, these attacks do not apply to transfer learning scenario, as the attacker cannot control model gradients of downstream training. In the centralized setting, researchers propose attacks to poison the victim's training data such that the impacts of attribute inference and membership inference~\cite{tramer2022truth} and property inference~\cite{saeed} attacks are amplified on the poisoned model.
The ability to poison the victim's data is a threat model orthogonal to ours, since we have no access to the victim's downstream data. While there is scope to combine such approaches for stronger attacks (albeit with stronger access assumptions), we choose to focus on the scenario with no read/write access to the victim's data.

\fi %%%%%%%%%%%%%%%%%%%%%%%%%%%%%%%%

\section{Linear Shortcut Across Blocks}
\label{sec:layer_jump}

To use a hidden representation from layer $\ell<L$ as a final representation, we propose to cast it using linear regression, while skipping the computation in-between these layers. More generally, this approach can be applied to cast any $\ell$-th hidden representation to any subsequent layer $\ell'>\ell$.


\subsection{Method}
\label{subsec:methodology_linear_shortcut}

Given a source layer $\ell$ and a target layer $\ell'$ such that $0 \leq \ell < \ell' \leq L$, our goal is to learn a mapping
%$A_{\ell', \ell} \in \mathbb{R}^{d_h \times d_h}$
from hidden representations at layer $\ell$ to those at layer $\ell'$. To this end, we first collect a set of corresponding hidden representation pairs $(h^\ell, h^{\ell'})$. Concretely, we run a set $\mathcal{T}$ of input sequences through the model, and for each input $s$, we extract the hidden representations $h_{i_s}^{\ell}, h_{i_s}^{\ell'}$, where $i_s$ is a random position in $s$.
Next, we learn a matrix $A_{\ell', \ell} \in \mathbb{R}^{d_h \times d_h}$ by fitting linear regression over $\mathcal{T}$, i.e., $A_{\ell', \ell}$ is a numerical minimizer for:
$$ A \mapsto \sum_{s \in \mathcal{T}} || A \cdot h_{i_s}^\ell - h_{i_s}^{\ell'} ||^2,$$ 
and define the mapping of a representation $h$ from layer $\ell$ to layer $\ell'$ as:
\begin{equation}
\label{eq:linear_jump}
    \matl{} (h) \coloneqq A_{\ell', \ell} \cdot h.
\end{equation}


\subsection{Baseline}
\label{subsec:baseline}

We evaluate 
% our method against 
the prevalent approach of ``reading'' hidden representations directly, without any transformation. 
Namely, the propagation of a hidden representation from layer $\ell$ to layer $\ell'$ is given by the identity function, dubbed \id{}:

$$ \idl{} (h) \coloneqq h.$$

% Notably, 
This baseline 
assumes that representations at different layers operate in the same linear space.

\subsection{Quality of Fit}
\label{subsec:experiments_r2}

We first evaluate our method by measuring how well the learned linear mappings approximate the representations at the target layer. To this end, we calculate the (coordinate-averaged) $r^2$-score of our mapping's outputs with respect to the representations obtained from a full inference pass, and compare to the same for the \id{} baseline.


\paragraph{Models.}

We use \gpt{} \cite{radford2019language}, a decoder-only auto-regressive LM, with $L = 48$, $d_h = 1600$, and \bert{} \cite{devlin-etal-2019-bert}, an encoder-only model trained with masked language modeling, with $L=24$, $d_h=1024$.
% \footnote{\label{footnote:hf}We use models and data from Huggingface \cite{wolf-etal-2020-transformers,lhoest-etal-2021-datasets}.}
%For masked token prediction, we use a masked LM head pre-trained for our \bert{} model.

% \footnote{Specifically, we use the Huggingface Transformers \cite{wolf-etal-2020-transformers} implementations of all these models.}

%\sy{We use \gpt{} \cite{radford2019language}, a decoder-only auto-regressive LM, coming in four scales; $\texttt{gpt2}$ ($L = 12$, $d_h = 768$), $\texttt{gpt2-medium}$ ($L = 24$, $d_h = 1024$), $\texttt{gpt2-large}$ ($L = 36$, $d_h = 1280$) and $\texttt{gpt2-xl}$ ($L = 48$, $d_h = 1600$). Also, we use \bert{} \cite{devlin-etal-2019-bert}, an encoder-only model trained with masked language modeling, coming in two scales;  \texttt{bert-base-uncased} ($L=12$, $d_h=768$) and \texttt{bert-large-uncased} ($L=24$, $d_h=1024$). For masked token prediction, we use masked LM heads pre-trained for our models. Specifically, we use the Huggingface Transformers \cite{wolf-etal-2020-transformers} implementations of all these models. The plots presented in this section are for $48$-layered \gpt{} and $24$-layered \bert{}.}

%\sy{We use \gpt{} \cite{radford2019language}, a decoder-only auto-regressive LM, in the Huggingface \cite{wolf-etal-2020-transformers} implementation\footnote{\url{https://huggingface.co/gpt2}}, coming in four scales; $\texttt{gpt2}$ ($L = 12$, $d_h = 768$), $\texttt{gpt2-medium}$ ($L = 24$, $d_h = 1024$), $\texttt{gpt2-large}$ ($L = 36$, $d_h = 1280$) and $\texttt{gpt2-xl}$ ($L = 48$, $d_h = 1600$). Also, we use \bert{} \cite{devlin-etal-2019-bert}, an encoder-only model trained with masked language modeling, in the Hugginface implementation, coming in two scales;  \texttt{bert-base-uncased}\footnote{\url{https://huggingface.co/bert-base-uncased}} ($L=12$, $d_h=768$) and \texttt{bert-large-uncased}\footnote{\url{https://huggingface.co/bert-large-uncased}} ($L=24$, $d_h=1024$). For masked token prediction, we use the \texttt{BertForMaskedLM} heads from Huggingface, pretrained for these models. The plots presented in this section are for $48$-layered \gpt{} and $24$-layered \bert{}.}

\paragraph{Data.}
We sample random sentences from Wikipedia,
% \footref{footnote:hf} 
collecting 9,000 (resp. 3,000) sentences for the training set $\mathcal{T}$ (resp. validation set $\mathcal{V}$).\footnote{We use sentences rather than full documents to simplify the analysis.}
%\sy{We use two data sources to evaluate our method. One is Wikiepdia \cite{lhoest-etal-2021-datasets}\footnote{\url{https://huggingface.co/datasets/wikipedia}}; we use \texttt{spaCy}\footnote{\url{https://spacy.io/}} to divide documents into sentences\footnote{We use sentences rather than full documents to simplify the analysis.}\footnote{We pick randomly a Wikipedia document and then pick randomly a sentence ending in a newline character in it. \sy{[maybe this footnote is not needed?]}}, collecting 9,000 (resp. 3,000) random sentences for the training set $\mathcal{T}$ (resp. validation set $\mathcal{V}$). The second is a news article sentences dataset, the 10K English 2020 news sentences corpus
% \footnote{\url{https://downloads.wortschatz-leipzig.de/corpora/eng_news_2020_10K.tar.gz}} from the Leipzig Corpora Collection \cite{goldhahn-etal-2012-building}, which we randomly divide into a training set $\mathcal{T}$ consisting of 9,000 examples and a validation set $\mathcal{V}$ consisting of 1,000 examples.
% We truncate sentences to the maximal token length allowed by the model \mg{do we ever need to truncate? a sentence has about 10 words and the max. input len is thousands} \sy{[I surely did not need to in Leipzig, but discovered (via a transformers runtime warning) that I do need to for some (probably a minority) of the Wikipedia sentences. This probably has to do with that it is not really ``sentences" necessarily, for example, I noticed that it has some listings or something like that (bulleted items)... So some minority might get very long I guess...]}.
For each example $s$, we select a random position $i_s$ and extract the hidden representations $h_{i_s}^{\ell}$ at that position from all the layers.
For \bert{}, we first replace the input token at position $i_s$ with a \mask{} token, as our motivation is interpreting predictions, which are obtained via masked tokens in \bert{} (see \S\ref{subsec:BERT}).
Thus, in this case, the hidden representations we consider
%in the case of \bert{}
are of \mask{} tokens only.
%As we observed highly similar results for the two data sources across all our experiments, throughout the paper we will mainly report results for Wikipedia (except for \S\ref{sec:robustness}, where we cross-validate).


\begin{figure}[t]
\includegraphics[scale=0.2]{figs/r2_scores_48.pdf}
% \includegraphics[width=\columnwidth]{figs/r2_scores_48.pdf}
\caption{The coordinate-averaged $r^2$-score of $\matl{}$ (left) and $\idl{}$ (right) (\gpt{}).}
\label{fig:r2_scores}
\end{figure}


\begin{figure}[t]
\setlength{\belowcaptionskip}{-10pt}
\includegraphics[scale=0.2]{figs/bertmask_r2_scores_24.pdf}
% \includegraphics[width=\columnwidth]{figs/bertmask_r2_scores_24.pdf}
\caption{The coordinate-averaged $r^2$-score of $\matl{}$ (left) and $\idl{}$ (right) (\bert{}).}
\label{fig:bertmask_r2_scores}
\end{figure}



\paragraph{Evaluation.}
For every pair of layers $\ell, \ell'$, such that $0 \leq \ell < \ell' \leq L$, we use the training set $\mathcal{T}$ to fit linear regression as described in \S\ref{subsec:methodology_linear_shortcut}, and obtain a mapping $\matl{}$. 
Next, we evaluate the quality of $\matl{}$ as well as of $\idl{}$ using the $r^2$-coefficient, uniformly averaged over all coordinates. Concretely, we compute the $r^2$-coefficient of each of the predicted representations $\matl{} (h_{i_s}^{\ell})$ and $\idl{} (h_{i_s}^{\ell})$ versus the true representations $h_{i_s}^{\ell'}$
over all $s \in \mathcal{V}$.
%as we vary $s \in \mathcal{V}$.
%for every $s \in \mathcal{V}$.



\paragraph{Results.}
Results for \gpt{} and \bert{} are presented in Figs.~\ref{fig:r2_scores} and~\ref{fig:bertmask_r2_scores}, respectively.
In both models, \mat{} consistently yields better approximations than \id{}, as it obtains higher $r^2$-scores (in blue) across the network. 
This gap between \mat{} and \id{} is especially evident in \bert{}, where \id{} completely fails to map the representations between most layers, suggesting that hidden representations are modified  substantially by every transformer block.
Overall, this highlights the shortcoming of existing practices to inspect representations in the same linear space, and the gains from using our method to approximate future layers.
% in the network.
\section{Linear Shortcut for Language Modeling}
\label{sec:prediction}

We saw that our method approximates future hidden representations substantially better than a naive propagation. 
In this section, we will show that this improvement also translates to better predictive abilities from earlier layers. Specifically, we will use our method to estimate how often intermediate representations encode the final prediction, in the context of two fundamental LM tasks; next token prediction and masked token prediction.

\paragraph{Evaluation Metrics.}
Let $h, h' \in \mathbb{R}^{d_h}$ be a final representation and a substitute final representation obtained by some mapping, and denote by $\delta (h), \delta (h') \in \mathbb{R}^{d_v}$ their corresponding output probability distributions (obtained through projection to the output vocabulary -- see details below). 
We measure the prediction quality of $h'$ with respect to $h$ using two metrics:
\begin{itemize}
[leftmargin=*,topsep=1pt,parsep=1pt]
    \item \textbf{Precision@$k$} ($\uparrow$ is better): This checks whether the token with the highest probability according to $\delta(h')$ appears in the top-$k$ tokens according to $\delta(h)$. Namely, we sort $\delta(h)$ and assign a score of $1$ if $\arg\max(\delta(h'))$ appears in the top-$k$ tokens by $\delta(h)$, and $0$ otherwise.
    
    \item \textbf{Surprisal} ($\downarrow$ is better): We measure the minus log-probability according to $\delta(h)$, of the highest-probability token according to $\delta(h')$. Intuitively, low values mean that the model sees the substitute result as probable and hence not surprising.
\end{itemize}

\noindent We report the average Precision@$k$ and Surprisal over the validation set $\mathcal{V}$.



\subsection{Next Token Prediction}
\label{subsec:next_token_prediction_task}

Auto-regressive LMs output for every position a probability distribution over the vocabulary for the next token. Specifically, the output distribution for every position $i$ is given by $\delta (h_i^L)$, where:
\begin{equation}\label{eq:output_distribution}
    \delta (h) = \texttt{softmax} ( E^\top \cdot h) \in \mathbb{R}^{d_v}
\end{equation}
For some LMs, including \gpt{}, a layer normalization $\texttt{ln\_f}$ is applied to the final layer representation before this conversion (i.e., computing $\delta (\texttt{ln\_f}(h))$ rather than $\delta (h)$).

Recall that our goal is to measure how well this distribution can be estimated from intermediate representations, i.e. estimating $\delta (h_i^L)$ from $\delta (h_i^\ell)$ where $\ell<L$. To this end, we first run examples from the validation set through the model, while extracting for each example $s$ the hidden representation of a random position $i_s$ at every layer. Next, we apply our mappings $\matlL{}$ and the $\idlL{}$ baseline to cast the hidden representations of every layer $\ell$ to final layer substitutes (see \S\ref{sec:layer_jump}). Last, for each layer, we convert its corresponding final-layer substitute to an output distribution (Eq.~\ref{eq:output_distribution}) and compute the average Precision@$k$ (for $k=1,5,10$) and Surprisal scores with respect to the final output distribution, over the validation set.

\paragraph{Results.}
Figs.~\ref{fig:pre} and~\ref{fig:surp} show the average Precision@$k$ and Surprisal scores per layer in $48$-layered \gpt{}, respectively (the plots for the other \gpt{} models are presented in \S\ref{sec:app_scale}). Across all layers, \mat{} outperforms \id{} in terms of both scores, often by a large margin (e.g. till layer $44$ the Precision@$1$ achieved by \mat{} is bigger than that of $\id{}$ by more than $0.2$). 
This shows that linear mappings enable not just better estimation of final layer representations, but also of the predictions they induce. Moreover, the relatively high Precision@$k$ scores of \mat{} in early layers ($0.62$-$0.82$ for $k=10$, $0.52$-$0.74$ for $k=5$, and $0.28$-$0.45$ for $k=1$) suggest that early representations already encode a good estimation of the final prediction. Also, the substantially lower Surprisal scores of \mat{} compared to \id{} imply that our method allows for a more representative reading into the layer-wise prediction-formation of the model than allowed through direct projection to the vocabulary.

\begin{figure}[t]
\centering
\includegraphics[scale=0.4]{figs/pre_48.pdf}
\caption{Precision@$k$ ($k = 1,5, 10$) of $\matlL{}$ and $\idlL{}$ for next token prediction in $48$-layered \gpt{}.}
\label{fig:pre}
\end{figure}

\begin{figure}[t]
\centering
\includegraphics[scale=0.35]{figs/surp_48.pdf}
\caption{Surprisal for $\matlL$ and the baseline $\idlL{}$ ($48$-layered \gpt{} next token prediction task). A 95\% confidence interval surrounds the lines.}
\label{fig:surp}
\end{figure}

\subsection{Masked Token Prediction}
\label{subsec:BERT}

We now conduct the same experiment for the task of masked language modeling, where the model predicts a probability distribution of a masked token in the input rather than the token that follows the input. Unlike next token prediction, where the output distribution is computed from representations of varying input tokens, in masked token prediction the output is always obtained from representations of the same input token (i.e. \texttt{[MASK]}).

For this experiment, we use \bert{}, on top of which we use a pretrained masked language model head $\delta$; given a token sequence $s$, a \mask{} token inside it and its final representation $h$, $\delta (h) \in \mathbb{R}^{d_v}$
 is a probability distribution over tokens giving the model's assessment
 of the likelihood of tokens to be fitting in place of the \mask{} token in $s$.


\begin{figure}[t]
\centering
\includegraphics[scale=0.4]{figs/bertmask_pre_24.pdf}
\caption{Precision@$k$ ($k = 1,5, 10$) for  $\matlL{}$ and the baseline $\idlL{}$ ($24$-layered \bert{} masked token prediction task).}
\label{fig:bertmask_pre}
\end{figure}

\begin{figure}[t]
\centering
\includegraphics[scale=0.35]{figs/bertmask_surp_24.pdf}
\caption{Surprisal for $\matlL{}$ and the baseline $\idlL{}$ ($24$-layered \bert{} masked token prediction task). A 95\% confidence interval surrounds the lines.}
\label{fig:bertmask_surp}
\end{figure}

\paragraph{Results.}
Figs.~\ref{fig:bertmask_pre} and~\ref{fig:bertmask_surp} present the average Precision@$k$ and Surprisal scores per layer in $24$-layered \bert{} (the plots for the $12$-layered \bert{} model are presented in \S\ref{sec:app_scale}), overall showing trends similar to those observed for next token prediction in \gpt{} (\S\ref{subsec:next_token_prediction_task}). This is despite the differences between the two tasks and the considerable architectural differences between \bert{} and \gpt{}.
Notably, the superiority of \mat{} over \id{} in this setting is even more prominent; 
while \mat{}'s precision is between $0.2-0.6$ in the first ten layers (Fig.~\ref{fig:bertmask_pre}), \id{}'s precision for all values of $k$ is close to zero, again strongly indicating that our method allows for better reading into early layer hidden representations. 
More generally, \mat{} improves the Precision@$1$ of \id{} by more than $17\%$ at most layers, and unveils that a substantial amount of predictions ($>25\%$ starting from layer $3$) appear already in the very first layers.
Interestingly, the (rough) divide between the first half of layers and last half of layers for $\id{}$ in Figs.~\ref{fig:bertmask_pre},~\ref{fig:bertmask_surp} seems to align with the two-hump shape of the blue region for $\mat{}$ in Fig.~\ref{fig:bertmask_r2_scores}.

\paragraph{Analysis.}
We manually compare the predictions of our mapping $\matlL{}$ with $\idlL{}$, for a $24$-layered \bert{} model.  Concretely, we select 50 random sentences from the Leipzig dataset. Next, for each layer $\ell$, we manually analyze how many of the top-$5$ tokens according to $\matlL{}$ and $\idlL{}$ fit into context. We consider a token to fit into context if it is grammatically plausible within the sentence (see Tab.~\ref{tab:manual} for concrete examples).
In the resulting $1250$ instances (i.e. $50$ sentences $\times$ $25$ representations), we observe a substantially higher plausibility rate of $85.36\%$ for \mat{} compared to $52.8\%$ for \id{}. In fact, only in less than $4.3\%$ of the instances there are more plausible tokens among the top-$5$ tokens according to \id{} than among the top-$5$ tokens according to \mat{}, further supporting the Surprisal results above.

\begin{table*}
\footnotesize
\setlength{\belowcaptionskip}{-15pt}
\begin{tabular}{p{0.3\linewidth}ccccc}
& $\texttt{id}_{4 \rightarrow 24}$ & $\texttt{mat}_{4 \rightarrow 24}$ & $\texttt{id}_{12 \rightarrow 24}$ & $\texttt{mat}_{12 \rightarrow 24}$ & $\texttt{id}_{24 \rightarrow 24}$ \\ \midrule
\multirow{5}{=}{aldridge had shoulder surgery in \mask{}.} & fellowship & \tcbox{time} & cyclist & \tcbox{2009} & \tcbox{september} \\
& employment & \tcbox{it} & emergencies & \tcbox{2008} & \tcbox{november} \\
& agreement & her & seniors & \tcbox{2010} & \tcbox{december} \\
& \#\#ostal & them & cycling & \tcbox{2006} & \tcbox{august} \\
& \#\#com & work & \tcbox{pennsylvania} & \tcbox{2007} & \tcbox{july} \\ \midrule
\multirow{5}{=}{on your next view you will be asked to \mask{} continue reading.} & \#\#com & be & be & be & \tcbox{please} \\
& accreditation & get & undergo & \tcbox{please} & \tcbox{simply} \\ 
& $	\copyright$ & go & spartans & help & \tcbox{also} \\ 
& fellowship & \tcbox{help} & seniors & \tcbox{simply} & \tcbox{again} \\ 
& summer & have & * & say & \tcbox{immediately} \\ \bottomrule
\end{tabular}
\caption{Examples of top-$5$ predictions at layers $4$, $12$ and $24$, under the mappings $\matlL{}$ and $\idlL{}$, for a $24$-layered \bert{} model. Grammatically plausible predictions (according to a human annotator) are marked in \tcbox{blue}. Note that at layer $24$ the predictions of $\matlL{}$ and $\idlL{}$ are the same (by definition).} 
\label{tab:manual}
\end{table*}

\section{Implication to Early Exiting}
\label{sec:applications}

%The fact that it is often possible to approximate
The possibility of approximating
the final prediction already in the early layers has important implications for efficiency; applying our linear mapping instead of executing transformer blocks of quadratic time complexity, could save a substantial portion of the computation. In this section, we demonstrate this in the context of early exiting.

When 
% performing transformer model inference under 
using an early exit strategy \cite{schwartz-etal-2020-right, xin-etal-2020-deebert, schuster2022confident}, one aims at deciding dynamically at which layer to stop the computation and ``read'' the prediction from the hidden representation of that layer.
More precisely, under a confidence measure paradigm, one decides to stop the computation for a position $i$ at layer $\ell$ based on a confidence criterion, that is derived from casting the hidden representation $h_i^\ell$ as a final-layer representation and converting it to an output probability distribution. Specifically, following \citet{schuster2022confident}, a decision to exit is made if the difference between the highest and the second highest probabilities is bigger than $$ 0.9 \cdot \lambda + 0.1 \cdot {\rm exp} (-4 i / N),$$
where $N$ is the average length of the input until position $i_s$ for $s \in \mathcal{V}$, and $\lambda$ is a hyper-parameter.

\begin{figure}[t]
\setlength{\belowcaptionskip}{-10pt}
\centering
\includegraphics[width=\columnwidth]{figs/ee_gpt2bert.pdf}
\caption{Precision@$1$ with early exit and ``fixed exit'', applied to the $24$-layer \gpt{} for next token prediction (left) and the $24$-layer \bert{} for masked token prediction (right). Varying the confidence parameter $\lambda$, the $x$-coordinate is the average number of layers processed before an early exit decision is reached.}
\label{fig:ee_gpt2bert}
\end{figure}

\quash{
\begin{figure}[t]
\setlength{\belowcaptionskip}{-10pt}
\centering
\includegraphics[scale=0.35]{figs/ee_pre1_24.pdf}
\caption{Precision@$1$ for the various early exit methods, and previous ``fixed exit'' methods for comparison ($24$-layer \gpt{} next token prediction task). Varying the confidence parameter $\lambda$, the $x$-coordinate is the average number of layers processed before an early exit decision is reached.}
\label{fig:ee_pre1}
\end{figure}
}

\paragraph{Experiment.}
We assess the utility of our mapping $\matlL{}$ for early exit as a plug-and-play replacement for $\idlL{}$, through which intermediate representations are cast into final-layer representations.
We use \gpt{} for the next token prediction and \bert{} for masked token prediction (both with 24 layers).
We run each of the models over the validation set examples, while varying the confidence parameter $\lambda$ and using either $\idlL{}$ or $\matlL{}$ for casting intermediate representations.
Furthermore, we compare these early exit variants to the ``fixed exit'' strategy from \S\ref{sec:prediction}, where the computation is stopped after a pre-defined number of layers rather than relying on a dynamic decision.
We evaluate each variant in terms of both prediction's accuracy, using the Precision@$1$ metric (see \S\ref{sec:prediction}), and efficiency, measured as the average number of transformer layers processed during inference.


\paragraph{Results.}
%Figs.~\ref{fig:ee_pre1} and~\ref{fig:bertmask_ee_pre1}
Fig.~\ref{fig:ee_gpt2bert}
plots the average Precision@$1$ score against the average number of layers processed, for $24$-layer \gpt{} and $24$-layer \bert{}. For both models, under an early exit strategy our mapping \mat{} again provides a substantial improvement over \id{}.
For example, aiming at $95\%$ average precision, \mat{} saves $\sim3.3$ ($13.8$\%) layers in \gpt{} compared to only $\sim1.4$ ($5.9$\%) layers by \id{}, and $\sim4.8$ ($20$\%) layers in \bert{} versus $\sim3.5$ ($14.6$\%) layers by \id{}.
These results highlight the potential gains prominent early exit methods can obtain by using our method.
Notably, in both models and for each of the mapping methods, early exit obtains better results than fixed layer exit, as expected. 

\quash{
\begin{figure}[t]
\setlength{\belowcaptionskip}{-10pt}
\centering
\includegraphics[scale=0.35]{figs/bertmask_ee_pre1_24.pdf}
\caption{Precision@$1$ for the various early exit methods, and previous ``fixed exit'' methods for comparison ($24$-layer \bert{} masked token prediction task). Varying the confidence parameter $\lambda$, the $x$-coordinate is the average number of layers processed before an early exit decision is reached.}
\label{fig:bertmask_ee_pre1}
\end{figure}
}
\section{Linear Shortcut Across Sub-Modules}
\label{sec:submodules}

% Our experiments show that
% , despite the commonly-applied simplification by interpretability works, transformer layers do not operate in the same linear space and 
% there is a major gap in approximating future representations using an identity mapping (\S\ref{sec:layer_jump}, \S\ref{sec:prediction}).
% Here, 
In this section, we investigate whether discrepancies across layers result from specific sub-modules or are a general behaviour of all sub-modules in the network.  
This is done by extending our approach to test how well particular components in transformer blocks can be linearly approximated. 


\paragraph{Method.}

Consider \gpt{} for definiteness, then:
% we have 
$$ \texttt{b}_{\ell} = \texttt{b}_{\ell}^{\texttt{ffn}} \circ \texttt{b}_{\ell}^{\texttt{attn}}$$ 
% with
\begin{equation}\label{eq:attn} \texttt{b}^{\texttt{attn}}_{\ell} (H) = \texttt{attn}_{\ell} (\texttt{ln1}_{\ell} (H)) + H,\end{equation} 
where $\texttt{attn}_{\ell}$ is
%a multi-head self-attention
a MHSA
layer and \texttt{ln1} is a layer normalization (LN), and 
$$ \texttt{b}^{\texttt{ffn}}_{\ell} (H) = \texttt{ffn}_{\ell} (\texttt{ln2}_{\ell} (H)) + H,$$  
where $\texttt{ffn}_{\ell}$ is
%a feed-forward network
an FFN
layer and $\texttt{ln2}$ is a LN.
\quash{
Given a block $\texttt{b}_\ell$ and one of its sub-modules $\texttt{ln1}_\ell, \ \texttt{attn}_\ell, \ \texttt{ln2}_\ell$, or $\texttt{ffn}_\ell$, we fit linear regression approximating the output of the sub-module given its input and then use it in order to define mappings, as we now describe.
}
Given a block $\texttt{b}_\ell$ and one of its sub-modules $\texttt{ln1}_\ell, \ \texttt{attn}_\ell, \ \texttt{ln2}_\ell$, or $\texttt{ffn}_\ell$, we fit linear regression approximating the output of the sub-module given its input, and then use it to define mappings $\matattnl{}$, $\matlnl{}$ and $\matffl{}$.
%We provide the definition of $\matattnl{}$ below, and that of the other two in App. \ref{sec:app_submodule_skip_description}.
We provide the formal definitions of these mappings in App. \ref{sec:app_submodule_skip_description}.
\iffalse
\paragraph{$\matattnl{}$.}
%Illustrating this on $\texttt{attn}_\ell$ for definiteness,
For an input $s$, let $v^\ell_{i_s}$ be the vector at position $i_s$ in the output of $\texttt{attn}_\ell (\texttt{ln1}_\ell (H^{\ell - 1}))$. We denote by $A_\ell^{\texttt{attn}} \in \mathbb{R}^{d_h \times d_h}$ the matrix numerically minimizing 
$$ A \mapsto \sum_{s \in \mathcal{T}} || A \cdot \texttt{ln1}_\ell (h^{\ell-1}_{i_s}) - v^\ell_{i_s}||^2,$$
and define an attention sub-module replacement (Eq.~\ref{eq:attn}) by $$
\texttt{b}^{\overline{\texttt{attn}}}_\ell (h) \coloneqq A_{\ell}^{\texttt{attn}} \cdot \texttt{ln1}_\ell (h) + h. $$
We then define a mapping between two layers ${\ell \rightarrow \ell'}$ by:
$$ \matattnl{} (h) \coloneqq $$
$$ \texttt{b}^{\texttt{ffn}}_{\ell'} ( \texttt{b}^{\overline{\texttt{attn}}}_{\ell'} ( \ldots (\texttt{b}^{\texttt{ffn}}_{\ell+1} ( \texttt{b}^{\overline{\texttt{attn}}}_{\ell+1} (h)))\ldots)).$$ 
Namely, when applying each $\ell''$-th block, $\ell < \ell'' \leq \ell'$, we replace its attention sub-module $\texttt{attn}_{\ell''}$ by its linear approximation.
%In an analogous way, we consider the mappings $\matffl{}$ and $\matlnl{}$, where in the latter we perform the linear shortcut both for \texttt{ln1} and for \texttt{ln2} (see~\S\ref{sec:app_submodule_skip_description} for precise descriptions).
Importantly, unlike the original attention module, the approximation $\texttt{b}^{\overline{\texttt{attn}}}_\ell$ operates on each position independently, and therefore applying $\matattnl{}$ disables any contextualization between the layers $\ell$ and $\ell'$. Note that this is not the case for $\matffl{}$ and $\matlnl{}$, which retain the self-attention sub-modules and operate contextually.
\fi

\paragraph{Evaluation.}


We analyze the $24$-layered \gpt{}, and proceed completely analogously to \S\ref{subsec:next_token_prediction_task}, evaluating the Precision@$1$ and Surprisal metrics for the mappings $\matattnlL{}$, $\matfflL{}$ and $\matlnlL{}$.

\begin{figure}[t]
\setlength{\belowcaptionskip}{-0pt}
\centering
%\includegraphics[scale=0.2]
\includegraphics[width=\columnwidth]{figs/parts_presurp_24.pdf}
\caption{Precision@$1$ and Surprisal for the various sub-module linear mappings, and $\matlL{}$ for comparison ($24$-layer \gpt{} next token prediction task). A 95\% confidence interval surrounds the Surprisal lines.}
\label{fig:parts_presurp}
\end{figure}

\quash{
\begin{figure}[t]
\centering
\includegraphics[scale=0.4]{figs/parts_pre1_24.pdf}
\caption{Precision@$1$ for the various sub-module linear shortcut mappings, and the mapping $\matlL{}$ for comparison (\gpt{} next token prediction task).}
\label{fig:parts_pre1}
\end{figure}

\begin{figure}[t]
\centering
\includegraphics[scale=0.35]{figs/parts_surp_24.pdf}
\caption{Surprisal for the various sub-module linear shortcut mappings, and the mapping $\matlL{}$ for comparison (\gpt{} next token prediction task). A 95\% confidence interval surrounds the lines.}
\label{fig:parts_surp}
\end{figure}
}

\paragraph{Results.}
Fig.~\ref{fig:parts_presurp} shows the average Precision@$1$ and Surprisal scores per layer.
From a certain layer (\textasciitilde$7$), all sub-module mappings achieve better results than the full-block mapping $\matlL{}$. Thus, it is not just the cumulative effect of all the sub-modules in the transformer block that is amenable to linear approximation, but also individual sub-modules can be linearly approximated. 
Furthermore, the linear approximation of attention sub-modules is less harmful than that of the FFN or LN sub-modules. 
% Hypothetically, 
A possible reason is that the linear replacement of FFN or LN ``erodes'' the self-attention computation after a few layers. 
Moreover, the good performance of $\matattnlL{}$ suggests that contextualization often exhausts itself in early layers; speculatively, it is only in more delicate cases that the self-attention of late layers adds important information. Last, remark the sharp ascent of the scores for layer normalization in layers $5$-$8$, for which we do not currently see a particular reason. To conclude, we see that the possibility of linear approximation permeates
%the various
transformer components.


\section{Related Work}

Recently, there was a lot of interest in utilizing intermediate representations in transformer-based LMs, both for interpretability and for efficiency.

In the direction of interpretability, one seeks to understand the prediction construction process of the model \cite{tenney-etal-2019-bert, voita-etal-2019-bottom}.

More recent works use mechanistic interpretability and view the inference pass as a residual stream of information \cite{dar2022analyzing,geva-etal-2022-transformer}. Additionally, there are works on probing, attempting to understand what features are stored in the hidden representations \cite{adi2017finegrained, conneau-etal-2018-cram,liu-etal-2019-linguistic}. Our work is different in that it attempts to convert intermediate representations into a final-layer form, which is interpretable by design.

In the direction of efficiency, there is the thread of work on early exit, where computation is cut at a dynamically-decided earlier stage \cite{schwartz-etal-2020-right,xin-etal-2020-deebert,schuster2022confident}. Other works utilize a fixed early stage network to parallelize inference \citep{leviathan2022fast, chen2023accelerating}. However, intermediate representations are directly propagated in these works, which we show is substantially worse than our approach. Moreover, our method requires training considerably less parameters than methods such as \citet{schuster-etal-2021-consistent}, that learn a different output softmax for each intermediate layer.  

More broadly, skipping transformer layers and analyzing the linearity properties of transformer components have been discussed in prior works \cite{Zhao2021of,mickus-etal-2022-dissect,wang-etal-2022-skipbert,lamparth2023analyzing}.


\section{Conclusion and Future Work}

We present a simple and effective method for enhancing utilization of hidden representations in transformer-based LMs, that uses 
pre-fitted context-free and token-uniform linear mappings.
Through a series of experiments on different data sources, model architectures and scales, we show that our method consistently outperforms the prevalent practice of interpreting representations in the final-layer space of the model, yielding better approximations of succeeding representations and the predictions they induce, thus allowing a more faithful interpretation of the model's prediction-formation.
We demonstrate the practicality of our method for improving computation efficiency, saving a substantial amount of compute on top of prominent early exiting approaches. 
Also, by extending our method to sub-modules, 
% more specifically the attention sub-modules, 
we observe that replacing a part of the transformer inference by a non-contextual linear computation often results in a small deterioration of the prediction.
This opens new research directions for improving model efficiency,
% and parallelizability.
% including breaking the computation into several parallelizable tasks.
including breaking the computation into parallel tasks.

\section*{Limitations}

Although we see in this work that there is more linear structure to transformer inference than could be explained solely by the residual connection, we do not elucidate a reason for that. We also do not try to formulate formal criteria according to which to judge, in principle, the quality of ways of short-cutting transformer inference in-between layers. In addition, our experiments cover only English data.


%\section*{Ethics Statement}
%Scientific work published at ACL 2023 must comply with the ACL Ethics Policy.\footnote{\url{https://www.aclweb.org/portal/content/acl-code-ethics}} We encourage all authors to include an explicit ethics statement on the broader impact of the work, or other ethical considerations after the conclusion but before the references. The ethics statement will not count toward the page limit (8 pages for long, 4 pages for short papers).

\section*{Acknowledgements}

We thank Tal Schuster for constructive comments.

% Entries for the entire Anthology, followed by custom entries
\bibliography{anthology,custom}
\bibliographystyle{acl_natbib}

\appendix

\section{Descriptions of $\matattn{}$, $\matff{}$ and $\matln{}$}
\label{sec:app_submodule_skip_description}

Here we detail the definitions of the mappings $\matattnl{}$, $\matffl{}$ and $\matlnl{}$ utilized in \S\ref{sec:submodules}.

\paragraph{Description of $\matattnl{}$.}
%Illustrating this on $\texttt{attn}_\ell$ for definiteness,
For an input $s$, let $v^\ell_{i_s}$ be the vector at position $i_s$ in the output of $\texttt{attn}_\ell (\texttt{ln1}_\ell (H^{\ell - 1}))$. We denote by $A_\ell^{\texttt{attn}} \in \mathbb{R}^{d_h \times d_h}$ the matrix numerically minimizing 
$$ A \mapsto \sum_{s \in \mathcal{T}} || A \cdot \texttt{ln1}_\ell (h^{\ell-1}_{i_s}) - v^\ell_{i_s}||^2,$$
and define an attention sub-module replacement (Eq.~\ref{eq:attn}) by $$
\texttt{b}^{\overline{\texttt{attn}}}_\ell (h) \coloneqq A_{\ell}^{\texttt{attn}} \cdot \texttt{ln1}_\ell (h) + h. $$
We then define a mapping between two layers ${\ell \rightarrow \ell'}$ by:
$$ \matattnl{} (h) \coloneqq $$
$$ \texttt{b}^{\texttt{ffn}}_{\ell'} ( \texttt{b}^{\overline{\texttt{attn}}}_{\ell'} ( \ldots (\texttt{b}^{\texttt{ffn}}_{\ell+1} ( \texttt{b}^{\overline{\texttt{attn}}}_{\ell+1} (h)))\ldots)).$$ 
Namely, when applying each $\ell''$-th block, $\ell < \ell'' \leq \ell'$, we replace its attention sub-module $\texttt{attn}_{\ell''}$ by its linear approximation.
%In an analogous way, we consider the mappings $\matffl{}$ and $\matlnl{}$, where in the latter we perform the linear shortcut both for \texttt{ln1} and for \texttt{ln2} (see~\S\ref{sec:app_submodule_skip_description} for precise descriptions).
Importantly, unlike the original attention module, the approximation $\texttt{b}^{\overline{\texttt{attn}}}_\ell$ operates on each position independently, and therefore applying $\matattnl{}$ disables any contextualization between the layers $\ell$ and $\ell'$. Note that this is not the case for $\matffl{}$ and $\matlnl{}$, which retain the self-attention sub-modules and operate contextually.

\paragraph{Description of $\matffl{}$.}
Let $v^\ell_{i_s}$ be the vector at position $i_s$ in the output of $\texttt{ln2}_{\ell} (\texttt{b}_\ell^{\texttt{attn}} (H^{\ell - 1}))$, for a given input $s$. We denote by $A_\ell^{\texttt{ffn}} \in \mathbb{R}^{d_h \times d_h}$ the matrix numerically minimizing 
$$ A \mapsto \sum_{s \in \mathcal{T}} || A \cdot v^{\ell}_{i_s} - \texttt{ffn}_{\ell} (v^\ell_{i_s})||^2,$$
and define a replacement of the feed-forward sub-module $\texttt{b}_{\ell}^{\texttt{ffn}}$ by $$ \texttt{b}^{\overline{\texttt{ffn}}}_\ell (H) \coloneqq A_{\ell}^{\texttt{ffn}} \cdot \texttt{ln2}_\ell (H) + H.$$
We then define a mapping between two layers ${\ell \rightarrow \ell'}$ by:
$$ \matffl{} (H) \coloneqq $$
$$ \texttt{b}^{\overline{\texttt{ffn}}}_{\ell'} ( \texttt{b}^{\texttt{attn}}_{\ell'} ( \ldots (\texttt{b}^{\overline{\texttt{ffn}}}_{\ell+1} ( \texttt{b}^{\texttt{attn}}_{\ell+1} (H))\ldots)).$$

\paragraph{Description of $\matlnl{}$.}
Let $v^\ell_{i_s}$ be the vector at position $i_s$ in the output of $\texttt{b}^{\texttt{attn}}_{\ell} (H^{\ell - 1})$, for a given input $s$. We denote by $A_\ell^{\texttt{ln1}} \in \mathbb{R}^{d_h \times d_h}$ the matrix numerically minimizing 
$$ A \mapsto \sum_{s \in \mathcal{T}} || A \cdot h^{\ell}_{i_s} - \texttt{ln1}_{\ell} (h^\ell_{i_s})||^2$$ and we denote by $A_\ell^{\texttt{ln2}} \in \mathbb{R}^{d_h \times d_h}$ the matrix numerically minimizing $$ A \mapsto \sum_{s \in \mathcal{T}} || A \cdot v^{\ell}_{i_s} - \texttt{ln2}_{\ell} (v^\ell_{i_s})||^2.$$ We define a replacement of the block $\texttt{b}^{\texttt{attn}}_{\ell}$ by \begin{equation} \texttt{b}^{\overline{\texttt{ln1}}}_\ell (H) \coloneqq \texttt{attn}_{\ell} (A_{\ell}^{\texttt{ln1}} \cdot H) + H\end{equation} and we define a replacement of the block $\texttt{b}^{\texttt{ffn}}_{\ell}$ by \begin{equation} \texttt{b}^{\overline{\texttt{ln2}}}_\ell (H) \coloneqq \texttt{ffn}_{\ell} (A_{\ell}^{\texttt{ln2}} \cdot H) + H.\end{equation}
We then define a mapping between two layers ${\ell \rightarrow \ell'}$ by:
$$ \matlnl{} (H) \coloneqq $$
$$ \texttt{b}^{\overline{\texttt{ln2}}}_{\ell'} ( \texttt{b}^{\overline{\texttt{ln1}}}_{\ell'} ( \ldots (\texttt{b}^{\overline{\texttt{ln2}}}_{\ell+1} ( \texttt{b}^{\overline{\texttt{ln1}}}_{\ell+1} (H))\ldots)).$$


\end{document}


\newpage
\pagenumbering{roman}
\renewcommand\thefigure{\thesection\arabic{figure}} 
\renewcommand\thetable{\thesection\arabic{table}} 
\setcounter{figure}{0}
\setcounter{table}{0} 

\appendix

\section{Causal TGCN Parameters}

%\subsection{Parameters for Causal Decomposition}
Tables \ref{tab:pcmci-param} to \ref{tab:ts-pcmci} detail the parameters required to decompose the causal discovery problem and give a brief summary of the selection method for each parameter.
\begin{table}[h!]
\caption{Parameters necessary for the PCMCI$^+$ algorithm.}
\label{tab:pcmci-param}
%\vskip 0.15in
\begin{center}
\begin{small}
\begin{sc}
\begin{tabular}{{p{0.2\columnwidth} p{0.65\columnwidth}}}
\toprule
Parameter & Selection Method\\
\midrule
$\tau_{max}$ & Domain knowledge. Method is robust to overestimates at cost of long runtimes.\\
\midrule
$\alpha$ & Significance threshold for causal link detection.\\
\midrule
CI test & Contextual knowledge of the likely nature of causal relationships.\\
\bottomrule
\end{tabular}
\end{sc}
\end{small}
\end{center}
\vskip -0.2in
\end{table}

\begin{table}[h!]
\caption{Additional Parameters necessary for the time-decomposed modification of PCMCI$^+$.}
\label{tab:t-pcmci}
%\vskip 0.15in
\begin{center}
\begin{small}
\begin{sc}
\begin{tabular}{{p{0.2\columnwidth} p{0.65\columnwidth}}}
\toprule
Parameter & Selection Method\\
\midrule
Period & Data analysis \& context.\\
\midrule
Aggregation method & Downstream performance.\\
\bottomrule
\end{tabular}
\end{sc}
\end{small}
\end{center}
\vskip -0.2in
\end{table}

\begin{table}[h!]
\caption{Additional parameters necessary for the space- decomposed modification of PCMCI$^+$.}
\label{tab:ts-pcmci}
%\vskip 0.15in
\begin{center}
\begin{small}
\begin{sc}
\begin{tabular}{{p{0.2\columnwidth} p{0.65\columnwidth}}}
\toprule
Parameter & Selection Method\\
\midrule
Number of clusters & Elbow plot.\\
\bottomrule
\end{tabular}
\end{sc}
\end{small}
\end{center}
\vskip -0.2in
\end{table}

\section{Traffic Flow Results}
Numeric results for the traffic flow case study are given in Table \ref{tab:pred-results-traffic}.

\begin{table}[ht]
\caption{Comparison of the prediction accuracy of different aggregation methods for the traffic flow dataset.}
\label{tab:pred-results-traffic}
%\vskip -0.15in
\begin{center}
\begin{small}
\begin{sc}
\begin{tabular}{lccc}
\toprule
Approach & RMSE\\
\midrule
Distance Benchmark   TGCN               & 9.64\\
\midrule
STGCN (1st order; \citet{Yu2018SpatioTemporalGC}) & 7.03\\
STGCN (Cheb; \citet{Yu2018SpatioTemporalGC})      & 6.77\\
\midrule
Temporal CTGCN & 6.73 \\
Spatiotemporal CTGCN & 6.41 \\
\midrule
Temporal MT CTGCN & 6.17 \\
Spatiotemporal MT CTGCN& \textbf{5.88} \\
\bottomrule
\end{tabular}
\end{sc}
\end{small}
\end{center}
% \vskip -0.3in
\end{table}


% \begin{table}[ht]
% \caption{Comparison of the prediction accuracy of different aggregation methods for the salmon louse dataset.}
% \label{tab:pred-results-salmon}
% %\vskip -0.15in
% \begin{center}
% \begin{small}
% \begin{sc}
% \begin{tabular}{lccc}
% \toprule
% Approach & RMSE\\
% \midrule
% Distance Benchmark TGCN                 & 1.17\\
% \midrule
% Spatiotemporal CTGCN & 1.15 \\
% \midrule
% Spatiotemporal MT CTGCN & \textbf{1.13} \\
% \bottomrule
% \end{tabular}
% \end{sc}
% \end{small}
% \end{center}
% \vskip -0.3in
% \end{table}

% \begin{table*}[ht]
% \caption{Hyperparameters for each dataset, selected according to the methods detailed above and in the main text.}
% \label{tab:params}
% %\vskip -0.15in
% \begin{center}
% \begin{small}
% \begin{sc}
% \begin{tabular}{p{0.5\columnwidth} p{0.35\columnwidth} p{0.35\columnwidth}  p{0.35\columnwidth}}
% \toprule
% Parameter & Building Heating & Traffic Flow & Salmon Louse\\
% \midrule
% $\alpha$            & 0.1   & 0.1   & 0.1\\
% $\tau_{max}$        & 6 (1h)& 9 (45min)& 6 (6wk)\\
% CI test             & GPDC  & GPDC  & GPDC\\
% \midrule
% Period              & 144 (1D)& 288 (1D)& -\\
% Clusters            & 10    & 25    & 55\\
% \midrule
% Window length       & 12 (2h)  & 9 (45min)& 12 (12wk)\\
% Forecast length     & 12 (2h)  & 12 (1h)& 8 (8wk)\\
% Temporal Kernel Size& 3     & 3     & 2\\
% GCN Latent Features &32, 32 &32, 32 & 64, 32\\
% \midrule
% Epochs              & 50    & 50    & 25\\
% Batch Size          & 128   & 128   & 32\\
% \bottomrule
% \end{tabular}
% \end{sc}
% \end{small}
% \end{center}
% \vskip -0.3in
% \end{table*}

\section{Causal Graph for the Building Dataset}
\label{sec:graph}

\begin{figure*}[hb]
\begin{center}
\centerline{\includegraphics[width=\textwidth]{GroundTruth.png}}
\vskip -0.15in
\caption{Directed graph representation of the ground truth adjacency matrix based on the physics of the simulation \cite{ploennigs2017semantic}}
\label{fig:graph-ground-truth}
\end{center}
\vskip -0.2in
\end{figure*}

\begin{figure*}[hb]
\begin{center}
\centerline{\includegraphics[width=\textwidth]{SpatioTempD1.png}}
\vskip -0.2in
\caption{Directed graph representation of the adjacency matrix for one day using the spatial-temporal approach showing correctly identified causal relationships (TP), missing (FN), or incorrectly identified ones (FP)}
\label{fig:graph-spationtempD1}
\end{center}
\vskip -0.3in
\end{figure*}

Figures~\ref{fig:graph-ground-truth} and \ref{fig:graph-spationtempD1} show a graph representation of the ground truth adjacency matrix and the adjacency matrix discovered for a single day with the spatial-temporal approach for the building dataset.

The ground truth Figures~\ref{fig:graph-ground-truth} shows the time series of the dataset as nodes and the correct causal relationships as directed edges. Its has on the left the boiler system that is feeding a HVAC in the center which is heating/cooling the rooms on the right with occupants. 

The goal of the causal discovery is to identify this graph from the time series data. We split all time series into days and compute for each day an adjacency matrix by clustering the timeseries with DTW in smaller clusters, within which we compute the causal relationships that form the adjacency matrix.

Figure~\ref{fig:graph-spationtempD1} shows the adjacency matrix resulting from the first day as example. The edges here color-code if the edge is also in the ground truth graph (TP), not in it (FP), or missing (FN). The node color shows the clusters the time series belongs to. We can identify clusters around similar time series semantics and characteristics like CO2 cluster, a humidity cluster, or a temperature cluster. Most of the missing edges (FN) are between clusters as we ignore them for performance reasons. The additionally detected ones (FP) are within a cluster and occur in some cases where there is actually a multi-hop relationship passing trough another cluster. Good example here are the incorrect relationships around the AHU\_Supply\_Air\_Temperature. It is isolated in its own cluster without causal relationships to other nodes. To cope with this, the causal discovery identifies additional relationships that bypass this isolated node. These bypass relationships are in this case not necessarily wrong and thus also do not have negative effects on the prediction performance of the TGCN.


\end{document}


% This document was modified from the file originally made available by
% Pat Langley and Andrea Danyluk for ICML-2K. This version was created
% by Iain Murray in 2018, and modified by Alexandre Bouchard in
% 2019 and 2021. Previous contributors include Dan Roy, Lise Getoor and Tobias
% Scheffer, which was slightly modified from the 2010 version by
% Thorsten Joachims & Johannes Fuernkranz, slightly modified from the
% 2009 version by Kiri Wagstaff and Sam Roweis's 2008 version, which is
% slightly modified from Prasad Tadepalli's 2007 version which is a
% lightly changed version of the previous year's version by Andrew
% Moore, which was in turn edited from those of Kristian Kersting and
% Codrina Lauth. Alex Smola contributed to the algorithmic style files.