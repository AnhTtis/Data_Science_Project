\documentclass[12pt]{article}

% DEFAULT PACKAGE SETUP

\usepackage{setspace,graphicx,epstopdf,amsmath,amsfonts,amssymb,amsthm}
\usepackage{marginnote,datetime,enumitem,subfigure,rotating,fancyvrb, booktabs}
\usepackage{float}
\usdate
\usepackage{lscape}
\usepackage{adjustbox}
%\usepackage[ruled]{algorithm2e}

% we have used too many packages for algorithm ?
 
\usepackage[ruled]{algorithm2e}  
\usepackage{pdflscape}
\usepackage{tikz}
\usepackage{mathpazo}
\usepackage{bm}
\usepackage{caption}
\usepackage{soul} 
\usepackage{color, xcolor}
% \usepackage{algorithm}  
\usepackage{amsmath}
\usepackage{algorithmicx}  
\usepackage{algpseudocode} 
\usepackage[colorlinks,linkcolor=blue,citecolor=blue]{hyperref}
\usepackage{longtable}
%\usepackage{versionPO}
%\usepackage[ left=1in,right=1in,top=1.5in,bottom=1.5in]{geometry}
%\usepackage[ left=1.25in,right=1.25in,top=1.25in,bottom=1.25in]{geometry}
\usepackage[ left=1in,right=1in,top=1.25in,bottom=1.25in]{geometry}
% These next lines allow including or excludingt versions of text
% using versionPO.sty
%\usepackage[notes,natbib,isbn=false,backend=biber]{biblatex-chicago}  
\usepackage[longnamesfirst]{natbib}
\usepackage{threeparttable}
\usepackage{authblk}
\algnewcommand\algorithmicparam{\textbf{Parameter:}}
\algnewcommand\Param{\item[\algorithmicparam]}


\newcommand{\E}{\mathbb{E}}
\newcommand{\se}{\sigma_\epsilon}
\newcommand{\sth}{\sigma_\theta}
\newcommand{\sx}{\sigma_x}
\newcommand{\lmd}{\lambda}
\newcommand{\ra}{\rightarrow}
\newcommand{\var}{\mathrm{Var}}
\newcommand{\bt}{\beta}
\newcommand{\tabincell}[2]{\begin{tabular}{@{}#1@{}}#2\end{tabular}}

\newtheorem{corollary}{Corollary}[section]
\newtheorem{conjecture}{Conjecture}[section]
%\newtheorem{con-proposition}{``Proposition''}
\newtheorem{proposition}{Proposition}[section]
\newtheorem{theorem}{Theorem}[section]
%\newtheorem{theorem_sec}{Theorem}[section]

% \newtheorem{corollary}{Corollary}
\newtheorem{lemma}{Lemma}[section]

\newtheorem{example}{Example}[section]
\newtheorem{remark}{Remark}[section]
\newtheorem{question}{Question}%[section]

\newtheorem{assumption}{Assumption}[section]
%\newenvironment{proof}{{\noindent\it Proof}\quad}{\hfill $\square$\par}
\newtheorem*{theorem*}{Theorem}

\newtheorem*{prob}{Problem}
\numberwithin{equation}{section}


%\newtheorem{remark}{Remark}[section]
\newtheorem{prop}{Proposition}[section]
\newtheorem{define}{Definition}[section]

\newtheorem{hyp}{Hypothesis}


\hypersetup{
	colorlinks=true,
	linkcolor=blue,
	filecolor=blue,      
	urlcolor=blue,
	citecolor=blue,
}

\usepackage{cleveref}



\begin{document}

\setlist{noitemsep}    
	
\title{Artificial Intelligence and Dual Contract}

\author{Wuming Fu}	
\author{Qian QI\footnote{Correspondence. No.5 Yiheyuan Road Haidian District, Beijing, P.R.China 100871. Email: {\href{mailto:qiqian@pku.edu.cn}{qiqian@pku.edu.cn}}.}}

\affil[1]{Peking University}


\renewcommand{\thefootnote}{\fnsymbol{footnote}}
\singlespacing
	
\maketitle
	
\vspace{-.2in}

\begin{abstract}
    
With the dramatic progress of artificial intelligence algorithms in recent times, it is hoped that algorithms will soon supplant human decision-makers in various fields, such as contract design. We analyze the possible consequences by experimentally studying the behavior of algorithms powered by Artificial Intelligence (Multi-agent Q-learning) in a workhorse \emph{dual contract} model for dual-principal-agent problems. We find that the AI algorithms autonomously learn to design incentive-compatible contracts without external guidance or communication among themselves. We emphasize that the principal, powered by distinct AI algorithms, can play mixed-sum behavior such as collusion and competition. We find that the more intelligent principals tend to become cooperative, and the less intelligent principals are endogenizing myopia and tend to become competitive. Under the optimal contract, the lower contract incentive to the agent is sustained by collusive strategies between the principals. This finding is robust to principal heterogeneity, changes in the number of players involved in the contract, and various forms of uncertainty.
\end{abstract}

%This paper introduces the dual-contract design via Q-learning methods. In contrast to the standard principal-agent problem (a principal and an agent), we emphasize that the dual-contract problem can also be recognized as a dual-principal-agent problem (two principals and an agent). The method utilizes a combination of game theory and reinforcement learning (RL) to create a contract that is both fair and beneficial to multiple sides. In this problem, two principals (e.g., two departments of a headquarters) are jointly responsible for providing the resources and capital for an agent. The two principals may have different objectives and interests, which must be lanced to ensure that the project is completed efficiently and effectively. In this problem, both principals must design a contract that provides the right incentives for the agent to perform the task optimally, however, the conventional mathematical method is hard to illustrate the economic consequence of this dual-contracting problem. The main advantage of using multi-agent Q-learning for dual-contract design is that it allows for solving optimization problems that are both fair and beneficial to both sides. This is because the multi-agent Q-learning algorithm can take into account both sides’ preferences and optimize the contract parameters accordingly. Additionally, using a Q-learning algorithm allows the contract to be adjusted over time as conditions change, ensuring that the contract remains fair and beneficial to both sides in the long term. 

\medskip
	
\noindent \textbf{JEL classification}: D21, D43, D83, L12, L13
	
\medskip
\noindent \textbf{Keywords}: Artificial intelligence, dual contract, principal-agent problem, AI alignment.
	
\thispagestyle{empty}
	
\clearpage
	
\doublespacing
\setcounter{footnote}{0}
\renewcommand{\thefootnote}{\arabic{footnote}}
\setcounter{page}{1}
	
	
\newpage

\section{Introduction}
\label{sec:introduction}
% \begin{itemize}
%     % Diffusion of FL
%     \item {\st{Diffusion of FL}}
%     % Security threats to FL
%     \item {\st{Security threats to FL with particular focus on model poisoning}}
%     % Limitations of existing countermeasures
%     \item {\st{Current countermeasures (e.g., KRUM) and their limitations}}
%     % Proposed method and its advantages
%     \item {\st{Intuitive description of the proposed method and its difference (i.e., advantages) w.r.t. state of the art}}
%     % Main contributions
%     \item {\st{Summary of the main contributions of this work}}
%     % Paper's structure and organization
%     \item {\st{Paper's structure and organization}}
% \end{itemize}

% Diffusion of FL
Recently, {\em federated learning} (FL) has emerged as the leading paradigm for training distributed, large-scale, and privacy-preserving machine learning (ML) systems~\cite{mcmahan2017googleai,mcmahan2017aistats}. 
The core idea of FL is to allow multiple edge clients to collaboratively train a shared, global model without disclosing their local private training data.
%Specifically, an FL system consists of a central server and many edge clients; 
A typical FL round involves the following steps: {\em(i)} the server randomly picks some clients and sends them the current, global model; {\em(ii)} each selected client locally trains its model with its own private data; then, it sends the resulting local model to the server;\footnote{Whenever we refer to global/local model, we mean global/local model {\em parameters}.} {\em(iii)} the server updates the global model by computing an \emph{aggregation function}, usually the average (FedAvg), on the local models received from clients.
% \begin{enumerate}
%     \item[{\em(i)}] the server sends the current, global model to the clients and appoints some of them for training;
%     \item[{\em(ii)}] each selected client locally trains its copy of the global model with its own private data; then, it sends the resulting local model back to the server;\footnote{Whenever we refer to global/local model, we mean global/local model {\em parameters}.}
%     \item[{\em(iii)}] the server updates the global model by computing an \emph{aggregation function} on the local models received from clients (by default, the average, also referred to as FedAvg~\cite{mcmahan2017aistats}).
% \end{enumerate}
This process goes on until the global model converges. %(e.g., after a certain number of rounds or other similar stopping criteria).
%\\
% The advantages of FL over the traditional, centralized learning paradigm are undoubtedly clear in terms of flexibility/scalability (clients can join/disconnect from the FL network dynamically), network communications (only model weights\footnote{We will use \textit{parameters} and \textit{weights} interchangeably.} are exchanged between clients and server), and privacy (each client's private training data is kept local at the client's end and not uploaded to the server).
\\
% Security threats to FL
%However, the growing adoption of FL also raises security concerns~\cite{costa2022covert}, particularly about its confidentiality, integrity, and availability.
Although its advantages over standard ML, FL also raises security concerns~\cite{costa2022covert}. %, particularly about its confidentiality, integrity, and availability~\cite{costa2022covert}.
% OLD, LONG VERSION
% Indeed, some work deals with privacy leakage that may expose the local data of some clients~\cite{melis2019sp}. 
% A large body of work, instead, investigates attacks that usually aim to detriment the predictive accuracy of the learned global model. For instance, \emph{data poisoning} attacks achieve this goal by letting an adversary pollute the training set of some corrupt FL clients with maliciously crafted examples~\cite{jagielski2018sp}.
% Similarly, in \emph{model poisoning} the attacker attempts to tweak the global model weights~\cite{bhagoji2019pmlr} by directly perturbing the local model's weights of some infected FL clients before these are sent to the central server for aggregation, usually via so-called Byzantine attacks. 
% It turns out that Byzantine model poisoning attacks severely impact standard FedAvg; therefore, more robust aggregation functions must be designed to make FL systems secure.
Here, we focus on \emph{untargeted model poisoning} attacks~\cite{bhagoji2019pmlr}, where an adversary attempts to tweak the global model weights %\footnote{We will use the terms \textit{parameters} and \textit{weights} interchangeably.} 
by directly perturbing the local model's parameters of some infected clients before these are sent to the central server for aggregation.
In doing so, the adversary aims to jeopardize the global model \textit{indiscriminately} at inference time.
Such model poisoning attacks severely impact standard FedAvg; therefore, more robust aggregation functions must be designed to secure FL systems.
\\
% In this paper, we focus on designing a novel robust aggregation scheme at the server's end to contrast the effect of Byzantine model poisoning attacks.
%
% Current countermeasures and their limitations
%Several countermeasures have been proposed in the literature to combat model poisoning attacks on FL systems.
% Some methods use simple statistics more robust than plain average to smooth the impact of malicious updates (e.g., Trimmed Mean and FedMedian~\cite{yin2018icml}). 
% Other defenses implement outlier detection techniques to discard malicious updates from the aggregation performed at the server's end. Those are either based on heuristics (e.g., Krum/Multi-Krum~\cite{blanchard2017nips} and Bulyan~\cite{mhamdi2018pmlr}) or data-driven approaches (e.g., K-means clustering~\cite{shen2016acm} or DnC via spectral analysis~\cite{shejwalkar2021ndss}). 
% Finally, some strategies rely on a centralized ``source of trust'' to spot potential malicious updates (e.g., FLTrust~\cite{cao2020fltrust}).
% Several countermeasures have been proposed in the literature to combat model poisoning attacks on FL systems, i.e., to discard possible malicious local updates from the aggregation performed at the server's end. 
% These techniques range from simple statistics more robust than plain average (e.g., Trimmed Mean and FedMedian~\cite{yin2018icml}) to outlier detection heuristics (e.g., Krum/Multi-Krum~\cite{blanchard2017nips} and Bulyan~\cite{mhamdi2018pmlr}) or data-driven approaches (e.g., spectral analysis via K-means clustering~\cite{shen2016acm} or spectral analysis), or methods based on ``source of trust'' (e.g., FLTrust~\cite{cao2020fltrust}).
% OLD, LONG VERSION
%Several countermeasures have been proposed in the literature to combat Byzantine model poisoning attacks on FL systems.
% Descriptive statistics
% For example, Trimmed Mean and FedMedian aggregate local model updates using more robust statistics than standard average~\cite{yin2018icml}.
%
% % Heuristics for outlier detection
% Many existing Byzantine-resilient strategies implement some outlier detection heuristics to discard the model updates sent by potentially malicious clients from the input of the aggregation function.
% One of the most popular heuristics is Krum~\cite{blanchard2017nips}.
% This strategy tries to mitigate the impact of Byzantine attacks by selecting as a global model the local model with the smallest sum of Euclidean distances to {\em all} the other local models.
% Although powerful, Krum requires the server to know (or, at least, estimate) the number of malicious FL clients upfront, which is generally impossible in a realistic attack scenario. %
% Moreover, Krum may become ineffective for complex, high-dimensional model parameter spaces due to the curse of dimensionality.
% Bulyan~\cite{mhamdi2018pmlr} tries to overcome this issue by combining Krum with a variant of Trimmed Mean.
% % Data-driven outlier detection
% Other strategies use data-driven outlier detection techniques -- e.g., via K-means clustering~\cite{shen2016acm} -- to spot potential malicious local model updates. 
% %For instance, Shen et al. propose to cluster local model updates with K-means and thus identify outliers.
%
% % Other techniques
% As far as the server is concerned, any local model received can be from a potential malicious client. 
% FLTrust~\cite{cao2020fltrust} assumes the server acts as a client, i.e., trains a local model on an additional {\em trustworthy} dataset at the server's end and compares it against all the local models from other clients. 
% This way, the server can rely on some ``source of trust'' when discarding potentially malicious clients.
%\\
% Limitations of existing Byzantine-resilient strategies
Unfortunately, existing defense mechanisms either rely on simple heuristics (e.g., Trimmed Mean and FedMedian by~\cite{yin2018icml}) or need strong and unrealistic assumptions to work effectively (e.g., foreknowledge or estimation of the number of malicious clients in the FL system, as for Krum/Multi-Krum~\cite{blanchard2017nips} and Bulyan~\cite{mhamdi2018pmlr}, which, however, cannot exceed a fixed threshold).
Furthermore, outlier detection methods using K-means clustering~\cite{shen2016acm} or spectral analysis like DnC~\cite{shejwalkar2021ndss} do not directly consider the temporal evolution of local model updates received.
Finally, strategies like FLTrust~\cite{cao2020fltrust} require the server to collect its own dataset and act as a proper client, thereby altering the standard FL protocol.
\\
% OLD, LONG VERSION
% Overall, existing Byzantine-resilient strategies are either simple heuristics (e.g., FedMedian) or, if they are more complex, they rely on strong and unrealistic assumptions to work effectively (e.g., knowing the number of malicious clients in the FL system in advance, as for Krum and alike).
% Furthermore, data-driven outlier detection methods do not consider the temporary evolution of local model updates received (e.g., K-means clustering). 
% Finally, strategies like FLTrust requires the server to collect its own dataset and act as a proper client, thereby altering the standard FL protocol.
%
% Description of the proposed method
This work introduces a novel pre-aggregation \textit{filter} robust to untargeted model poisoning attacks. Notably, this filter $(i)$ operates without requiring prior knowledge or constraints on the number of malicious clients and $(ii)$ inherently integrates temporal dependencies. 
The FL server can employ this filter as a preprocessing step before applying \textit{any} aggregation function, be it standard like FedAvg or robust like Krum or Bulyan.
Specifically, we formulate the problem of identifying corrupted updates as a multidimensional (i.e., matrix-valued) time series anomaly detection task. 
The key idea is that legitimate local updates, resulting from well-calibrated iterative procedures like stochastic gradient descent (SGD) with an appropriate learning rate, show \textit{higher predictability} compared to malicious updates. This hypothesis stems from the fact that the sequence of gradients (thus, model parameters) observed during legitimate training exhibit regular patterns, as validated in Section~\ref{subsec:intuition}. %until convergence. 
%This regularity may be more pronounced for smooth convex loss functions, but it can still be captured within an appropriate time window, even for more complex and convoluted loss surfaces. 
%We provide evidence of this claim in Appendix~B, where we show that the average mutual information (i.e., ``predictability''), calculated over pairs of legitimate model updates sent at different FL rounds, is significantly higher than the corresponding computation for a malicious client.
\\
Inspired by the matrix autoregressive (MAR) framework for multidimensional time series forecasting~\cite{chen2021je}, we propose the FLANDERS ({\em \textbf{F}ederated \textbf{L}earning meets \textbf{AN}omaly \textbf{DE}tection for a \textbf{R}obust and \textbf{S}ecure}) filter.
The main advantages of FLANDERS over existing strategies like FLDetector~\cite{zhao2020multivariate} are its resilience to large-scale attacks, where $50\%$ or more FL participants are hostile, and the capability of working under realistic non-iid scenarios.
We attribute such a capability to two key factors: $(i)$ FLANDERS works without knowing a priori the ratio of corrupted clients, and $(ii)$ it embodies temporal dependencies between intra- and inter-client updates, quickly recognizing local model drifts caused by evil players. Below, we summarize our main contributions:

\begin{itemize}
\item[{\em(i)}]
We provide empirical evidence that the sequence of models sent by legitimate clients is more predictable than those of malicious participants performing untargeted model poisoning attacks.
\\
\item[{\em(ii)}] 
We introduce FLANDERS, the first pre-aggregation filter for FL robust to untargeted model poisoning based on multidimensional time series anomaly detection.
\\
\item[{\em(iii)}] 
We integrate FLANDERS into Flower,\footnote{\scriptsize{\url{https://flower.dev/}}} a popular FL simulation framework for reproducibility.
\\
\item[{\em(iv)}] 
We show that FLANDERS improves the robustness of the existing aggregation methods under multiple settings: different datasets, client's data distribution (non-iid), models, and attack scenarios.
\\
\item[{\em(v)}] 
We publicly release all the implementation code of FLANDERS along with our experiments.\footnote{\scriptsize{\url{https://anonymous.4open.science/r/flanders_exp-7EEB}}}
\end{itemize}

% Paper's structure and organization
The remainder of the paper is structured as follows. %some related work and the current state-of-the-art solutions to security issues that FL entails. 
Section~\ref{sec:background} covers background and preliminaries. 
In Section~\ref{sec:related}, we discuss related work.
Section~\ref{sec:problem} and Section~\ref{sec:method} describe the problem formulation and the method proposed. % to tackle it. 
Section~\ref{sec:experiments} gathers experimental results. %, and Section~\ref{sec:limitations} discusses some limitations of this work.
Finally, we conclude in Section~\ref{sec:conclusion}.
 %discusses the limitations of this work and draws future research directions.
%reports conclusions and draws perspectives for future research directions.

%%%%%%% OLD %%%%%%%
%to overcome the resilience of Byzantine failures in distributed Stochastic Gradient Descent computations. 
% The strength of Krum is its time complexity, which is linear in the gradient dimension. 
% However, the robustness of the approach is guaranteed for gradient-based learning applications only when the majority of the clients are not compromised. 
% Besides, the aggregation mechanism of Krum, as well as that of similar methods, is robust from a coarse-grained perspective and does not provide solutions to errors and perturbations that may occur at inference time.
%A related approach to~\cite{blanchard2017nips} is the work of Su et al.~\cite{su2016dc}. Here, the authors propose an iterated approximate agreement to tackle a multi-layer scenario attacked by Byzantine agents. 
%However, the method works efficiently on the sole discrete context and it is inapplicable to continuous state environments.
%\gabri{Maybe, we should just talk about the main limitations of existing countermeasures without digging into their details (or, we can just mention Krum as this is the most popular one). I will move the description of all these methods to the Related Work section.}
\section{Q-learning}\label{Qlearning}

\indent We focus on Q-learning algorithms \cite{watkins1992q} and \cite{calvano2020artificial}, a cornerstone of model-free reinforcement learning widely used in AI. These off-policy algorithms utilize a Q-value function—a matrix predicting the utility of actions in different states—to guide action selection. Through actions and rewards, the AI refines this function to maximize expected rewards over time, developing an optimal policy.\footnote{Q-learning, a reinforcement learning algorithm, aims to identify actions that yield the highest rewards. By learning from action outcomes, the decision-maker continuously improves its approach. Q-learning assigns values to actions, updating them based on new rewards to guide better decision-making. Our choice of Q-learning stems from its widespread real-world application, realistic simulation of decision-making, clear economic interpretation of parameters, and structural resemblance to advanced programs like ChatGPT \citep{ouyang2022training}. This section provides a concise overview, emphasizing its relevance and rationale for incorporation in our analysis.}


% Head 2
\subsection{Single Decision Maker Problems}

\indent Q-learning, a type of reinforcement learning, enables decision-makers to learn from experience and improve their choices. It seeks the optimal sequence of actions, known as a policy, to maximize rewards over time without prior knowledge of the problem. Initially designed for Markov Decision Processes (MDPs) with finite states and actions, Q-learning facilitates learning through interaction with the environment.

\indent In a stationary MDP, at each time step $t = 0, 1, 2,...$, a decision-maker observes state $s_t \in \mathcal{S}$ and chooses action $a_t \in \mathcal{A}$. Each state-action pair $(s_t, a_t)$ yields a reward $\pi_t$, and the system transitions to the next state $s_{t+1}$ according to a time-invariant probability distribution $F(\pi_t, s_{t+1} | s_t, a_t)$. Notably, Q-learning in this context assumes finite $\mathcal{S}$ and $\mathcal{A}$, with $\mathcal{A}$ being independent of the current state.


The decision-maker's problem is to maximize the expected present value of the reward stream:
% Numbered Equation
\begin{equation}
\label{eqn:01}
\mathbb{E} \left[ \sum_{t=0}^{\infty} \delta^t \pi_t \right],
\end{equation}
where $\delta \leq 1$ represents the discount factor. This dynamic programming problem is typically addressed using Bellman's value function:
% Numbered Equation
\begin{equation}
\label{eqn:02}
V(s_{t}) = \max_{a_{t} \in \mathcal{A}} \{ \mathbb{E}[\pi_{t} | s_{t}, a_{t}] + \delta \mathbb{E}[V(s_{t+1}) | s_{t}, a_{t}] \}.
\end{equation}
Building upon this, we introduce the Q-function, representing the discounted payoff of action $a$ in state $s$:
\begin{equation}
\label{eqn:03}
Q(s_{t}, a_{t}) = \mathbb{E}[\pi_{t} | s_{t}, a_{t}] + \delta \mathbb{E} \left[ \max_{a_{t+1} \in \mathcal{A}} Q(s_{t+1}, a_{t+1}) | s_{t}, a_{t} \right],
\end{equation}
where the first term represents the immediate reward, and the second term captures the discounted continuation value. The value function and Q-function are linked by $V(s) \equiv \max_{a \in \mathcal{A}} Q(s, a)$. With finite $\mathcal{S}$ and $\mathcal{A}$, the Q-function can be represented as an $|\mathcal{S}| \times |\mathcal{A}|$ matrix.

% Head 3
\subsection{Learning the Q-Matrix}

Q-learning aims to determine the optimal action for each state by estimating the Q-matrix, reflecting expected rewards for actions in different states. This process operates without prior knowledge of the underlying model, specifically $F(\pi_{t}, s_{t+1} | s_{t}, a_{t})$.

Q-learning algorithms employ an iterative approach to approximate the Q-matrix. Starting from an arbitrary initial matrix $Q_0$, the algorithm updates the corresponding cell $Q_t(s_{t}, a_{t})$ after observing reward $\pi_t$ and transition to state $s_{t+1}$ following action $a_t$ in state $s_t$:

\begin{equation}
\label{eqn:04}
Q_{t+1}(s_{t}, a_{t}) = (1 - \alpha) Q_t(s_{t}, a_{t}) + \alpha [\pi_t + \delta \max_{a_{t} \in \mathcal{A}} Q_t(s_{t+1}, a_{t})],
\end{equation}
where $\alpha \in [0, 1]$ is the learning rate, controlling the influence of new experience on the Q-value update. 

While \cite{watkins1992q} demonstrated the convergence of Q-learning to the optimal policy within an MDP for a single decision-maker, extending this guarantee to multi-agent scenarios is challenging due to non-stationarity. The interconnected reward structure and unpredictable actions of other agents introduce complexities. However, independent Q-learning, where agents learn without explicitly modeling opponents' strategies, has shown promise in such environments.\footnote{\cite{watkins1992q} revealed its potential to reach the optimal strategy within the confines of a Markov Decision Problem (MDP) for an individual decision maker. However, extending this certainty to multi-decision maker scenarios is problematic due to non-stationarity. decision makers must navigate a dynamic environment where the reward system is intertwined with the unpredictable actions of adversaries. Despite the absence of the Markov property, studies suggest that independent Q-learning can still yield positive outcomes in such complex environments. While algorithms that consider opponents’ strategies require detailed information about their tactics and behavior, an independent approach retains the uncomplicated, model-free essence of reinforcement learning.}


% Head 4
\subsection{Exploration Strategies}

\indent Effective learning necessitates exploring all possible state-action pairs to determine the most rewarding actions. The algorithm learns through trial and error, balancing the exploitation of existing knowledge with the exploration of new possibilities. While achieving the optimal balance is complex, Q-learning algorithms typically rely on predefined exploration parameters.

\indent The $\epsilon$-greedy policy is a common exploration strategy, selecting the best-known action with probability $1-\epsilon$ and choosing randomly among all actions with probability $\epsilon$. This approach balances exploiting known rewards with exploring potentially better alternatives.

\subsection{Beyond Single Decision Maker}

Although initially developed for single-agent MDPs, Q-learning has been extended to multi-agent systems. In these scenarios, agents learn simultaneously, facing the challenge of non-stationarity arising from the dynamic strategies of other agents. Despite these difficulties, independent Q-learning, where agents learn and adapt individually, often leads to effective outcomes in complex multi-agent environments.


 
\section{Experiment Design}\label{contracts}

\indent We begin with a heuristic naive case by constructing Q-learning algorithms and allowing them to interact in a dynamic contract setting, hoping that the naive case will illustrate that AI algorithms can learn incentive-compatible contracts automatically without any external guidance. To this end, we take \cite{innes1990limited} as the reference model for the naive case to examine the effectiveness and convergence of the AI outcomes. The selection of the reference model remains for the following important reasons:

\begin{itemize}
    \item The model is tractable enough to obtain an analytical solution, providing a clear benchmark for evaluating the equilibrium outcomes of AI algorithms in the naive case.\footnote{Remarkably, the Q-learning algorithms can naturally extend the static model to a dynamic setting, thereby analyzing the dynamics of the extended model. Fortunately, this change in model setup does not affect the final equilibrium outcome in the naive case, indicating that the equilibrium outcome in the static model can also be used as a reference under dynamic settings. Therefore, the main contribution of these Q-learning algorithms here is to reveal different convergence paths under different initial conditions, which would be impossible to observe in a static model. The experiment results in the following sentences show that the AI algorithms are robust enough to learn the equilibrium outcomes under various initial conditions.}

    \item The setup of \cite{innes1990limited} is logically straight forward and easy to implement in the dynamic setting of the experiment environment for the AI algorithms. 

    \item The Model is simple and can be fully characterized by just a few parameters, the economic interpretation of which is clear.

    \item The setup is suitable for progressing to the dynamic dual-contract scenario, making the dual-principal-agent problem remain well defined and interpretable yet difficult to be analyzed using conventional methods.
    
\end{itemize}

\indent Note that learning a dynamic setting of such a static moral hazard model is not our goal in the current paper. The importance of selecting a parsimonious reference model is to prove the AI algorithms' learnability of incentive-compatible contracts in a relatively tractable way, although we can choose more complicated models and build up more intricate experiment environments for the AI algorithm. Here, we describe the reference model and the economic environment in which the algorithms operate, the exploration strategy they follow, and other aspects of the numerical simulations.





\subsection{Reference Model and Economic Environment}

Specifically, a risk-neutral principal provides investment funds to a risk-neutral agent, who then makes an unobservable ex-ante effort choice. Its key idea is that legal liability limits bind the investment contract.\footnote{
Recall the key idea of \cite{innes1990limited}, in the presence of limited liability when the downside of an investment is limited both for the agent and the principal, the closest one can get to a situation where the agent is a ``residual claimant'' is a (risky) debt contract. In other words, a debt contract provides the best incentives for effort provision by extracting as much as possible from the agent under low performance and giving her the total marginal return from effort provision in high-performance states where revenues are above the face value of the debt.} 

\begin{itemize}
\item Project requires initial investment $I$, which comes from principal.
\item agent exerts unobservable effort $e$ at cost $\frac{1}{2}ce^2$\
\item With probability $e$, project generates payoff $X^H$.
\item With probability $1-e$, generate payoff $X^L<X^H$.
\item Contract pays principal $D^L$ if payoff is $X^L$ and $D^H$ if payoff is $X^H$.
\item Agent retains the residual.
\end{itemize}
For a given contract $(D^L,D^H)$, the agent maximizes
\begin{equation}
\label{eqn:05}
e(X^H-D^H)+(1-e)(X^L-D^L)-\frac{1}{2}ce^2,
\end{equation}
The first-order condition for $e$ gives the incentive-compatible (IC) constraint:
\begin{equation}
\label{eqn:06}
(X^H-D^H)+(X^L-D^L)=ce,
\end{equation}
The individual rationality (IR) constraint is that the principal must also break even, so we need
\begin{equation}
\label{eqn:07}
eD^H+(1-e)D^L=I,
\end{equation}
 Lagrangian for optimal contract
 \begin{align}
\label{eqn:08}
\mathcal{L}& = e(X^H-D^H)+(1-e)(X^L-D^L)-\frac{1}{2}ce^2\nonumber\\&+\lambda_1(e-\frac{(X^H-D^H)-(X^L-D^L)}{c}+\lambda_2(1-eD^H-(1-e)D^L),
\end{align}
Derivative wrt $D^L$
\begin{equation}
\label{eqn:09}
\frac{\mathrm{d}\mathcal{L}}{\mathrm{d}D^L}=-(1-e)-\frac{\lambda_1}{c}-\lambda_2(1-e),
\end{equation}
Derivative wrt $D^H$
\begin{equation}
\label{eqn:10}
\frac{\mathrm{d}\mathcal{L}}{\mathrm{d}D^H}=-e+\frac{\lambda_1}{c}-e\lambda_2=-\frac{\mathrm{d}\mathcal{L}}{\mathrm{d}D^L}-(1+\lambda_2),
\end{equation}
\paragraph{Claim}
Optimal to set $D^L=X^L$.
\paragraph{Proof by contradiction}
Suppose optimal $D^L<X^L$. Then it must be the case that $\frac{\mathrm{d}\mathcal{L}}{\mathrm{d}D^L}=0$.
\begin{itemize}
\item If it were not, we would increase $D^L$.
\item But then we will have $\frac{\mathrm{d}\mathcal{L}}{\mathrm{d}D^H}<0$, so we will want to set $D^H=0$.
\item But then we will induce negative effort.
\item Instead, set $D^L=X^L$ and $X^H>D^H>I$.
\end{itemize}

\subsection{Economic Environment}\label{single}
The reference model looks like a debt contract:
\begin{itemize}
\item At a low project payoff, the principal gets everything.
\item At high project payoff, the agent gets residual.
\item At high project payoff, the principal gets more than initially
provided, which is similar to interest payment.
\end{itemize}
The key intuition is to reward the agent for a high project payoff that induces effort.
\begin{itemize}
\item Leave the agent with the max possible cash in that state.
\item Compensate by giving the principal as much as possible in the low payoff state
\end{itemize}

\indent We adopt the setup of the reference model to construct our economic environment. Here, we drop the IR constraint (\Cref{eqn:07}), as we would like to demonstrate that AI algorithms can autonomously learn to be rational individuals.

\paragraph{Setup} Single principal-agent contract Problem
\begin{itemize}
\item Project requires initial investment $I$, which comes from the principal.
\item Principal chooses a ``tax rate'' $p{\in}[0,1]$.
\item Agent observes principal's ``tax rate'' then exerts effort $e{\in}[0,1]$ at cost $\frac{1}{2}ce^2$\
\item The project generates payoff $I+(T-I)e$, where $T$ is the highest possible payoff.
\item Contract pays principal $I+(T-I)ep$.
\item Agent retains the residual.
\end{itemize}

\subsection{Parametrization and Initialization}
Initially, we focus on a baseline economic environment that consists of a principal-agent problem with $T=2I$ and $c=2I$. For this specification, the profit $(T-I)e$ is equal to the agent's cost $\frac{1}{2}ce^2$ when the agent's effort is equal to $1$. This means that when the agent exerts effort $e=1$, the net profit will be equal to $0$.

As for the initial matrix $Q_0$, our baseline choice is to set the Q-values at $t=0$ to a random number between $0$ and $1$. The learning parameter $\alpha$ may be in the principal range from $0$ to $1$. It is well known, however, that high values of $\alpha$ may disrupt learning when experimentation is extensive as the algorithm would forget too rapidly what it has learned in the past. Learning must be persistent to be effective, requiring that $\alpha$ be relatively small. In machine learning literature, a value of $0.1$ is often used; accordingly, we choose $\alpha = 0.1$ in our baseline model.

As for the experimentation parameter $\epsilon$, the trade-off is as follows. First, the algorithms need to explore extensively, as the only way to learn is through multiple visits to every state-action cell (of which there are $101$ in our baseline experiment with $|S|=1,|A|=101$, and much more in more complex environments). Additionally, exploration is costly. One can abstract from the short-run cost by considering long-run outcomes. But exploration also entails another cost if there is more than one algorithm learning, in that if one algorithm experiments more extensively, this creates noise in the environment, making it harder for the other to learn. This externality means that, in principle, experimentation may be excessive, even discounting the short-term cost.

Our baseline model adopts the $\epsilon$-greedy model with a time-invariant exploration rate. Although a more complex, time-declining exploration rate can be designed, a fixed exploration rate is enough for our purpose.

\subsection{Results}
In this part of the simulation, we simulate a Q-learning algorithm (the principal) contracting with the agent. The Q-learning algorithm needs to choose a ``tax rate'' $p$. The agent will observe this $p$ and act in their best interest. The grid of allowable choices of $p$ includes 101 $p$ levels from $0$ to $1$. The outcome of this experiment is reported in \Cref{fig:one}. Expressly, we set $\epsilon=0.2$.

% Figure
% \begin{figure}
% \centerline{\includegraphics{singlePA.jpg}}
% \caption{The picture shows the Q-learning algorithm gradually finding the optimal p.}
% \label{fig:one}
% \end{figure}

\begin{figure}[htbp]
    \centering
    \subfigure[]{
        \includegraphics[width=2.5in]{singlePA.png}
    }
    \subfigure[]{
	\includegraphics[width=2.5in]{singlePA2.png}
    }
    \quad 
    \subfigure[]{
    \includegraphics[width=2.5in]{singlePA3.png}
    }
    \subfigure[]{
	\includegraphics[width=2.5in]{singlePA4.png}
    }
    \caption{These pictures show the Q-learning algorithm gradually finding the optimal tax rate $p$.}
    \label{fig:one}
\end{figure}

\paragraph{Result} Our algorithm converges to the optimal choice of $p$. One thing to note in \Cref{fig:one} is that there exists a static optimal $p$. After finding this optimal $p$, there is no need for the algorithm to explore further. This is intuitive, as the agent always acts in their best interest and has a fixed strategy for every $p$ the principal chooses. However, this may not be true if there are more than one algorithm learning and they are playing against each other, as seen in the following dual contract problem.
\section{Dual Contract and Principal Heterogeneity}\label{dualcontract}

\indent This section extends our analysis from a single-principal-agent model (see \Cref{single}) to a more realistic dual-contract scenario. In this setting, a single agent simultaneously engages in contracts with two distinct principals. This structure closely resembles the dynamics of various real-world scenarios, such as venture capital funding rounds, freelance work arrangements, and multi-client consulting engagements. While offering benefits like diversified experience and combined expertise, it also presents unique challenges in terms of transparency, fairness, and potential agent exploitation. Our goal is to understand how two principals, each employing a Q-learning algorithm, learn to set contract terms ("tax rates") when interacting with an  agent.\footnote{This dynamic closely resembles the venture capital market, where startups often secure funding from multiple investors simultaneously. This parallel highlights several key similarities: 
\begin{itemize}
\item Negotiation Power: Startups with multiple investors have greater leverage to negotiate better terms, just like an individual with multiple job offers can negotiate better compensation or benefits. 
\item Access to Diverse Expertise: Venture capital firms often have specialized expertise in different industries. Similarly, working for multiple companies can expose individuals to a broader range of perspectives and skillsets.
\item Risk Management: Diversifying funding sources can mitigate risk for startups and individuals alike, reducing dependence on a single revenue stream and enhancing resilience to financial instability.
\end{itemize}} 

\subsection{Model Setup}

\indent \indent We consider two principals ($P_1$ and $P_2$) who offer contracts to a single agent $A$. Each principal has a project ($Project_1$ and $Project_2$) requiring initial investments $I_1$ and $I_2$, respectively. The agent can allocate their effort ($e_{1,t}$ and $e_{2,t}$) between these projects in each period $t$, subject to the constraint $e_{1,t} + e_{2,t} \leq 1$.

\paragraph{Contract Terms and Payoffs:}
\indent Principals independently choose tax rates ($p_{1,t}$ and $p_{2,t}$) in each period, representing the fraction of project revenue they retain. The payoffs are structured as follows:
\begin{itemize}
\item \textbf{Principal 1's Profit: } $\Pi_t^{P_1} = I_1 + (R_1 - I_1)e_{1,t}p_{1,t}$
\item \textbf{Principal 2's Profit: } $\Pi_t^{P_2} = I_2 + (R_2 - I_2)e_{2,t}p_{2,t}$
\item \textbf{Agent's Profit:} $\Pi^{A}_{t} = (1 - p{1,t})[I_1 + (R_1 - I_1)e_{1,t}] + (1 - p_{2,t})[I_2 + (R_2 - I_2)e_{2,t}] - C(e_{1,t}, e_{2,t})$
\end{itemize}
where $R_1$ and $R_2$ are the maximum potential revenues for the projects. The agent's cost function, $C(e_{1,t}, e_{2,t})$, incorporates the cost parameter $c$ and the heterogeneity parameter $\kappa$ (explained below):

\begin{equation}
C(e_{1,t}, e_{2,t}) = \frac{1}{2}c(e_{1,t} + e_{2,t})^2 (1 - \kappa + 2 \kappa e_{2,t} / (e_{1,t} + e_{2,t}))
\end{equation}

\paragraph{Profit Alignment and Heterogeneity:}

\begin{itemize}
\item We introduce a "rate of identity of interests," $\gamma \in [0, 0.5]$, to capture varying degrees of profit alignment between the principals. Higher $\gamma$ indicates greater alignment, with $\gamma = 0$ representing pure competition and $\gamma = 0.5$ representing pure collusion.
\item To model principal heterogeneity, we use the parameter $\kappa \in [0,1)$ in the agent's cost function. A higher $\kappa$ gives Principal 1 an advantage by making the agent's per-unit effort cost lower for Project 1, reflecting potential real-world biases. This bias reflects real-world scenarios where factors like reputation, pre-existing relationships, or project attributes might make one principal more appealing to the agent.
\end{itemize}


\subsection{Optimization with Q-Learning}

\indent In contrast to the single-principal-agent model, deriving closed-form solutions for the optimization problem in this dynamic dual-contract setting proves analytically intractable. To circumvent this, we employ multi-agent reinforcement-learning (MARL), enabling the principals to progressively learn optimal contract terms (tax rates) through repeated interactions with the agent and each other. Each principal maintains an independent Q-table, updating it based on their own realized profits.

\paragraph{Q-Learning Dynamics:}
\indent Both principals utilize Q-learning to optimize their strategies. Their Q-functions $Q^{P_i}(s^{P_i},p_i)$, where $i \in {1,2}$, map state-action pairs to expected profits. The Q-tables are initialized arbitrarily, and the Q-values are updated using the following rule:
\begin{equation}
Q_{t+1}^{P_i}(s_t^{P_i}, p_{i,t}) = (1 - \alpha) Q_t^{P_i}(s_t^{P_i}, p_{i,t}) + \alpha [\Pi_{i,t}^P + \delta \max_{p_{i,t+1}} Q_t^{P_i}(s^{P_i}_{t+1}, p_{i,t+1})],
\end{equation}
where:
\begin{itemize}
\item $s_t^{P_i}$ is the state of Principal $i$ at time $t$, which includes information about past tax rates offered by both principals, past profits, and potentially other relevant information.
\item $p_{i,t}$ is the tax rate chosen by Principal $i$ at time $t$.
\item $\alpha$ is the learning rate.
\item $\delta$ is the discount factor.
\item $\Pi_{i,t}^P$ is the profit of Principal $i$ at time $t$, which depends on the tax rate offered by Principal $i$, the tax rate offered by the other principal, and the agent's effort allocation.
\end{itemize}
\paragraph{Agent's Strategy:}
\indent The agent's Q-function, $Q^A(s^A, e_1, e_2)$, maps state-action pairs to expected rewards. The agent's state $s_t^A$ includes the current tax rates from both principals: $s_t^A = (p_{1,t}, p_{2,t})$. The agent's action space consists of all possible effort levels on Project 1 and Project 2, subject to the constraint $e_{1,t} + e_{2,t} \leq 1$. The agent updates their Q-function using the following rule:
\begin{equation}
Q_{t+1}^A(s_t^A, e_{1,t}, e_{2,t}) = (1 - \alpha) Q_t^A(s_t^A, e_{1,t}, e_{2,t}) + \alpha [\Pi^{A}_{t} + \delta \max_{e_{1,t+1}, e_{2,t+1}} Q_t^A(s_{t+1}^A, e_{1,t+1}, e_{2,t+1})],
\end{equation}
where $\alpha$ is the learning rate, $s_{t+1}^A$ is the next period's state, which includes the next period's tax rates from both principals ($p_{1,t+1}$, $p_{2,t+1}$), $\Pi^{A}_{t}$ is the agent's profit in period $t$ (as defined above).

In each period, the agent observes the tax rates from both principals, chooses the effort levels on both projects that maximize the estimated Q-value, and then updates their Q-table based on the observed profits. This iterative process allows the agent to learn and adapt their effort allocation strategy in response to the changing contract terms offered by the two principals.

\subsection{Baseline Parametrization and Initialization}

\indent To systematically investigate the dynamics of the dual-contract model, we define a baseline economic setting and explore variations across four key parameter grids. These parameters are summarized in \Cref{table:parameters}:

\paragraph{Baseline Economic Setting:}
\begin{itemize}
\item $I_1 = I_2 = 1$: The initial investments required for both projects are set equal to normalize the project scales.
\item $R_1 = R_2 = 2$: The maximum potential revenue for both projects is fixed at twice the initial investment, reflecting a common return target.
\item $c = I_1 + I_2 = 2$: The agent's cost parameter is set equal to the sum of the initial investments. This ensures that at maximum effort ($e_1 + e_2 = 1$), the combined project profit equals the agent's effort cost, leading to a net profit of 0 for the principals collectively.
\end{itemize}

\paragraph{Parameter Grids:}
We discretize the parameter space of the learning rate $\alpha$, exploration rate $\beta$, profit alignment $\gamma$, and principal heterogeneity $\kappa$ to systematically explore their impact on contract negotiation outcomes. The specific grids are defined as follows:
\begin{enumerate}
\item \textbf{Learning Rate $\alpha$:} The learning rate dictates how much weight principals give to new information versus their existing beliefs. We explore 100 equally spaced values between 0.025 and 0.25. This range captures a balance between slow and fast learning, allowing us to investigate the effect of learning speed on the negotiation dynamics.
\item \textbf{Exploration Rate $\beta$:} The exploration rate determines the principals' tendency to explore new tax rates versus exploiting those that have yielded high profits in the past. We vary $\beta$ over 100 equally spaced values between $10^{-6}$ and $10^{-5}$. This range ensures sufficient exploration at the beginning of the simulations while allowing for exploitation as the principals gain experience.
\item \textbf{Profit Alignment $\gamma$:} To model varying degrees of alignment between the principals' interests, we consider three distinct values for $\gamma$: 0, 0.25, and 0.5. These values represent pure competition ($\gamma = 0$), a mixed-sum game ($\gamma = 0.25$), and pure collusion ($\gamma = 0.5$). This allows us to investigate how the level of competition or cooperation influences the negotiated contract terms and the resulting profits.
\item \textbf{Principal Heterogeneity $\kappa$:} We consider two levels of principal heterogeneity, $\kappa = 0$ (non-heterogeneity) and $\kappa = 0.25$. The inclusion of $\kappa$ allows us to examine the impact of asymmetry in the agent's effort cost on the bargaining power dynamics and effort allocation. Specifically, we can analyze how even a slight advantage for one principal might affect the agent's effort allocation and the final distribution of profits.
\end{enumerate}


\indent This parametrization allows us to isolate the effects of varying $\gamma$ and $\kappa$ on the contract outcomes. For the Q-learning algorithms, we employ the following settings:

\begin{itemize}
\item \textbf{Initial Q-values $Q_0$:} All Q-tables are initialized with random values drawn uniformly from the interval [0, 1], representing a lack of prior knowledge about the optimal contract terms.
\item \textbf{Discount Factor $\delta$:} We use a discount factor of 0.9, reflecting the importance of future rewards in the principals' decision-making.
\item \textbf{Memory Length $k$:} This parameter, set to 1 in our baseline, determines the number of past tax rates that are included in the state representation. This allows us to investigate the impact of memory on the negotiation dynamics.
\end{itemize}


\subsection{Results and Discussion}
This section presents the findings from simulating the dual-contract model across varying levels of profit alignment $\gamma$ and principal heterogeneity $\kappa$. We focus on three key aspects: the convergence of tax rates chosen by the principals, the agent's effort allocation across the two projects, and the resulting profit distribution among the stakeholders.

\subsubsection{Impact of Learning and Exploration Rates}

\begin{figure}[htbp]
\centering
\includegraphics[width=1\textwidth]{results/dual/combined_heatmaps_group1_gamma_0_kappa_0_memory_1.png}
\caption{Average values for Principal 1 profit, Principal 2 profit, effort for Project 1, effort for Project 2, tax rate for Principal 1, and tax rate for Principal 2 for $\gamma=0, \kappa=0$. The heatmaps illustrate the impact of learning rate $\alpha$ and exploration rate $\beta$ on these six variables.}
\label{fig:heatmaps-gamma-0}
\end{figure}
The learning rate $\alpha$ and exploration rate $\beta$ significantly influence the dynamics of the Q-learning process and, consequently, the contract negotiation outcomes. To illustrate this impact, we analyze the heatmaps depicting average Principal 1 profit, average Principal 2 profit, average effort for Project 1, average effort for Project 2, average tax rate for Principal 1, and average tax rate for Principal 2 across different values of $\alpha$ and $\beta$, under pure competition scenario ($\gamma=0$, $\kappa=0$) shown in Figure \ref{fig:heatmaps-gamma-0}.

A clear pattern emerges: higher $\alpha$ values generally lead to faster convergence of both tax rates and profits. This is because principals with higher learning rates adapt more quickly to new information, reaching stable outcomes faster. This observation highlights the importance of learning agility in dynamic negotiation environments. Conversely, larger $\beta$ values, corresponding to higher exploration rates, introduce more volatility in the early stages of the negotiation process. This is because principals experiment with a wider range of tax rates before converging, leading to fluctuations in profits and effort allocations. This highlights the trade-off between exploration (gathering information) and exploitation (leveraging seemingly profitable strategies) in reinforcement learning.


\begin{figure}[htbp]
\centering
\includegraphics[width=1\textwidth]{results/dual/combined_heatmaps_group2_gamma_0_kappa_0_memory_1.png}
\caption{Convergence Iteration for Principal 1 and Principal 2 for $\gamma=0, \kappa=0$. The heatmap illustrates the impact of learning rate $\alpha$ and exploration rate $\beta$ on the convergence iteration.}
\label{fig:heatmaps-profits_2}
\end{figure}

Remarkably, larger $\beta$ values, corresponding to higher exploration rates, might delay the convergence to a stable strategy as principals experiment with a wider range of tax rates. This delay is reflected in \Cref{fig:heatmaps-profits_2}, which shows that higher $\beta$ values generally lead to more iterations required for convergence, especially for certain learning rates. This exploration, while crucial for gathering information about the system dynamics, could potentially prolong the period of fluctuating profits before the principals settle on a fixed strategy.


\subsubsection{Profit Alignment and Emergent Cooperation}

The level of profit alignment $\gamma$ between the principals significantly shapes the negotiation outcomes, directly influencing their achieved profits. We can observe these dynamics by analyzing the average principal profits visualized in heatmaps across different learning rates $\alpha$ and exploration rates $\beta$ under varying degrees of profit alignment, specifically $\gamma = 0$, $\gamma = 0.25$, and $\gamma = 0.5$, while keeping principal heterogeneity constant $\kappa = 0$.

\begin{figure}[htbp]
\centering
\includegraphics[width=1\textwidth]{results/dual/combined_heatmaps_group1_gamma_0.25_kappa_0_memory_1.png}
\caption{Average values for Principal 1 profit, Principal 2 profit, effort for Project 1, effort for Project 2, tax rate for Principal 1, and tax rate for Principal 2 for $\gamma=0.25, \kappa=0$. The heatmaps illustrate the impact of learning rate $\alpha$ and exploration rate $\beta$ on these six variables.}
\label{fig:heatmaps-gamma-0.25}
\end{figure}

\Cref{fig:heatmaps-gamma-0} depicts the outcomes for $\gamma=0$, while \Cref{fig:heatmaps-gamma-0.25} displays the results for  $\gamma=0.25$, and \Cref{fig:heatmaps-gamma-0.5} illustrates the case when $\gamma=0.5$. As $\gamma$ increases, we observe a noticeable upward shift in the average profits for both principals. For instance, focusing on the top-left heatmaps in each figure, which represent average Principal 1 profit, we can see a clear trend of increasing profit as $\gamma$ changes from 0 to 0.25 and then to 0.5. This difference suggests that even a small degree of profit alignment can incentivize a degree of implicit cooperation between the principals, leading to higher tax rates and, consequently, higher average profits. 


\begin{figure}[htbp]
\centering
\includegraphics[width=1\textwidth]{results/dual/combined_heatmaps_group1_gamma_0.5_kappa_0_memory_1.png}
\caption{Average values for Principal 1 profit, Principal 2 profit, effort for Project 1, effort for Project 2, tax rate for Principal 1, and tax rate for Principal 2 for $\gamma=0.5, \kappa=0$. The heatmaps illustrate the impact of learning rate $\alpha$ and exploration rate $\beta$ on these six variables.}
\label{fig:heatmaps-gamma-0.5}
\end{figure}

Furthermore, examining the heatmaps for average effort for Project 1 and Project 2, we observe that as $\gamma$ increases, the difference in effort allocation between the two projects becomes less pronounced. This observation indicates that with higher profit alignment, the competition for the agent's effort becomes less intense, leading to a more balanced effort allocation across both projects. 

These observations underscore the significant influence of profit alignment on the strategic dynamics in multi-principal settings. Even a small degree of shared interest can incentivize more cooperative behavior, leading to higher average profits for the principals and potentially a more balanced effort allocation from the agent. As the alignment of incentives increases, the potential for emergent cooperation strengthens, ultimately shifting the system away from cutthroat competition towards strategies that benefit all parties involved.

As we shift to a scenario with partial profit alignment, represented by $\gamma = 0.25$ in Figure \ref{fig:heatmaps-gamma-0.25}, a noticeable shift occurs. The average profits for Principal 1 are markedly higher compared to the purely competitive case. This difference suggests that even a small degree of profit alignment can incentivize a degree of implicit cooperation between the principals, leading to higher tax rates and, consequently, higher average profits.


In a purely competitive scenario ($\gamma = 0$), both principals, driven solely by their profit maximization, engage in a race to the bottom, consistently converging to the lowest possible tax rate, as depicted in Figure \ref{fig:four}. 

\begin{figure}[htbp]
\centering
\subfigure[]{
\includegraphics[width=2.5in]{beta0_1.png}
}
\subfigure[]{
\includegraphics[width=2.5in]{beta0_2.png}
}
\quad
\subfigure[]{
\includegraphics[width=2.5in]{beta0_3.png}
}
\subfigure[]{
\includegraphics[width=2.5in]{beta0_4.png}
}
\caption{Convergence of tax rates under pure competition ($\gamma=0$). Both Q-learning algorithms converge to the lowest possible positive tax rate.}
\label{fig:four}
\end{figure}

However, a striking phenomenon emerges when the principals' profits are perfectly aligned ($\gamma = 0.5$). Figure \ref{fig:five} illustrates this scenario, where despite the absence of explicit communication or coordination mechanisms, the Q-learning algorithms demonstrate emergent cooperative behavior. 

\begin{figure}[htbp]
\centering
\subfigure[]{
\includegraphics[width=2.6in]{beta5_0.png}
}
\subfigure[]{
\includegraphics[width=2.6in]{beta5_2.png}
}
\quad
\subfigure[]{
\includegraphics[width=2.6in]{beta5_3.png}
}
\quad
\subfigure[]{
\includegraphics[width=2.6in]{beta5_4.png}
}
\caption{Convergence of tax rates under pure collusion ($\gamma=0.5$). The Q-learning algorithms learn to implicitly cooperate, converging on higher tax rates than in the competitive scenario.}
\label{fig:five}
\end{figure}

This implicit collusion is evident in the convergence towards higher tax rates compared to the competitive cases. This spontaneous coupling effectively allows the principals to extract more surplus from the agent, maximizing their joint profit, reflected in the higher average profits observed in the heatmaps for  $\gamma = 0.5$.

Further reinforcing these observations, Figure \ref{fig:six} demonstrates the impact of varying levels of profit alignment on the effective tax rate convergence. As $\gamma$ increases, we observe a gradual shift from competitive to more cooperative dynamics, resulting in higher converged tax rates. 

\begin{figure}
\centerline{\includegraphics[width=5in]{betaResult.png}}
\caption{Effective tax rate convergence for varying levels of profit alignment $\gamma$. As $\gamma$ increases, the simulations demonstrate a gradual shift from competitive to more cooperative dynamics.}
\label{fig:six}
\end{figure}

The principals, even without explicit communication, learn to balance their self-interest with the potential gains from coordinated action, leading to intermediate levels of cooperation and subsequently impacting the average profits observed in the heatmaps.

\subsubsection{Principal Heterogeneity and Bargaining Asymmetry}

Introducing heterogeneity between the principals ($\kappa > 0$) by making the agent's effort cost asymmetric significantly impacts the bargaining power dynamics. This asymmetry creates a distinct advantage for the favored principal (Principal 1 in our model).

\begin{figure}[htbp]
\centering
\includegraphics[width=1\textwidth]{results/dual/combined_heatmaps_group1_gamma_0.25_kappa_0_memory_1.png}
\caption{Average Effort for Project 1 for $\gamma=0.25, \kappa=0$. The heatmap demonstrates the impact of principal heterogeneity on the agent's effort allocation.}
\label{fig:heatmaps-efforts-1}
\end{figure}

Figure \ref{fig:heatmaps-efforts-1} presents a heatmap of the agent's average effort for Project 1 across different learning and exploration rates for a symmetric scenario ($\gamma=0.25, \kappa=0$). 

\begin{figure}[htbp]
\centering
\includegraphics[width=1\textwidth]{results/dual/combined_heatmaps_group1_gamma_0.25_kappa_0.25_memory_1.png}
\caption{Average Effort for Project 2 for $\gamma=0.25, \kappa=0.25$. The heatmap demonstrates the impact of principal heterogeneity on the agent's effort allocation.}
\label{fig:heatmaps-efforts-2}
\end{figure}

Conversely, Figure \ref{fig:heatmaps-efforts-2} showcases the same information, but for an asymmetric scenario ($\gamma=0.25, \kappa=0.25$). These figures reveal that Principal 1, benefiting from the agent's lower effort cost, can sustain higher tax rates without losing the agent's effort, even under competitive pressure. This "protection effect" arises from the agent's rational preference for the less costly project, granting Principal 1 greater bargaining power. 


The agent's rational behavior is further reflected in the effort allocation, as illustrated in Figure \ref{fig:seven}. The agent allocates more effort toward the less costly project offered by Principal 1, reinforcing the protection effect and further amplifying Principal 1's profit advantage.

\begin{figure}[htbp]
\centering
\subfigure[Agent's effort in Project 1]{
\includegraphics[width=1.7in]{ae1.png}
}
\subfigure[Agent's effort in Project 2]{
\includegraphics[width=1.7in]{ae2.png}
}
\subfigure[Agent's maximum profit given $p_{1}$ and $p_{2}$]{
\includegraphics[width=1.7in]{aProfit.png}
}
\caption{Agent's optimal strategy under principal heterogeneity ($\kappa > 0$). The agent allocates more effort toward the less costly project offered by Principal 1.}
\label{fig:seven}
\end{figure}

\subsubsection{Spontaneous Coupling and its Implications}
Our findings highlight the potential for spontaneous coupling to emerge in multi-principal settings, even without explicit collusion. Figure \ref{fig:eight} depicts the convergence of the effective tax rate – the lower of the two offered tax rates – under varying levels of $\gamma$ in the presence of principal heterogeneity. We observe that higher $\gamma$ values lead to stronger spontaneous coupling, resulting in higher converged tax rates and greater surplus extraction from the agent. Furthermore, principal heterogeneity introduces an additional layer of complexity. While both principals might benefit from spontaneous coupling when $\gamma$ is high, the advantaged principal (Principal 1) consistently secures a larger share of the surplus due to the protection effect. This is illustrated by the higher effective tax rate for Principal 1 across different $\gamma$ values.
\begin{figure}
\centerline{\includegraphics[width=5in]{asymBeta.png}}
\caption{Effective tax rate convergence under principal heterogeneity for varying levels of profit alignment ($\gamma$). The advantaged principal (Principal 1) consistently secures a higher effective tax rate.}
\label{fig:eight}
\end{figure}
\subsubsection{Discussion}
The emergence of spontaneous coupling in our model raises important questions about its implications for market dynamics and agent welfare. Future research should explore the robustness of these findings across different learning algorithms, information structures, and agent behaviors. Furthermore, designing mechanisms to mitigate the potentially negative consequences of spontaneous coupling on agent welfare presents a significant challenge for future work.
\section{Conclusion}\label{sec:conclusion}
In this work, we focus on addressing the fundamental challenge of OOD detection tasks, which is how to fully understand the semantic discrepancy between the ID/OOD samples. We reveal that the key to success in the realistic SCOOD task is to allocate as many ID samples in the unlabeled set correctly as possible. To this end, we propose a novel uncertainty-aware optimal transport scheme that introduces class-specific energy scores as guidance for effective label assignment. Experimental results show that our method achieves better performance than previous state-of-the-art methods on SCOOD benchmarks.

\textbf{Limitations.} In addition to temperature scaling, other techniques such as feature clipping applied in ReAct~\cite{sun2021react} also enhance the performance of energy score, so how to obtain an OOD score that best fits the SCOOD task can be further explored. Moreover, a setting highly related to SCOOD has been proposed in \cite{katz2022training} and formulated as a constrained optimization problem. We will also theoretically analyze these practical OOD settings in our feature work.

% \section*{Acknowledgments}
\textbf{Acknowledgments.} 
This work is supported by National Key R\&D Program of China under Grant 2020AAA0105701, National Natural Science Foundation of China (NSFC) under Grants 61872327, Major Special Science and Technology Project of Anhui, National Natural Science Foundation of China (62033012) and Ant Group through Ant Research Intern Program.

	
	

\clearpage
\bibliographystyle{jf}
\bibliography{ref}
\clearpage

\appendix
\section{Appendix for Proofs}

\paragraph{Proof of Theorem \ref{thm:main}.}

\begin{proof}
\label{proof:main}
Our proof has two steps. In Step 1, we will show that SimCLR is equivalent to minimizing the cross entropy loss defined in Eqn.~(\ref{eqn:cross-entropy}). 
In Step 2, we will show  that minimizing the cross-entropy loss 
is equivalent to spectral clustering on $\bfpi$. 
Combining the two steps together, we have proved our theorem. 

\textbf{Step 1: } SimCLR is equivalent to minimizing the cross entropy loss.

The cross-entropy loss takes expectation over 
$\bfW_\bfX\sim \mathbb{P}(\cdot ; \bfpi)$, 
which means $\bfW_\bfX$ has exactly one non-zero entry in each row $i$. By Lemma~\ref{lem:multinomial}, we know every row $i$ of $\bfW_\bfX$ is independent of other rows. Moreover, 
$\bfW_{\bfX,i}\sim \mathcal{M}(1, \bfpi_i/\sum_j \bfpi_{i,j})=\mathcal{M}(1, \bfpi_i)$, because $\bfpi_i$ itself is a probability distribution.
Similarly, we know $\bfW_\bfZ$ also has the row-independent property by sampling over $\mathbb{P}(\cdot;\bfK_\bfZ)$.
Therefore, by Lemma~\ref{lem:cross_split}, we know Eqn.~(\ref{eqn:cross-entropy}) is equivalent to:
\[
 -\sum_{i=1}^n \mathbb{E}_{\bfW_{\bfX,i}}[\log \mathbb{P}(\bfW_{\bfZ,i}=\bfW_{\bfX,i};\bfK_\bfZ)],
\]

This expression takes expectation over $\bfW_{\bfX,i}$ for the given row $i$. Notice that 
$\bfW_{\bfX,i}$ has exactly one non-zero entry, which equals $1$ (same for $\bfW_{\bfZ,i}$). 
As a result
we expand the above expression to be:
\begin{equation}
 -\sum_{i=1}^n \sum_{j\neq i} \Pr(\bfW_{\bfX,i,j}=1)\log \Pr(\bfW_{\bfZ,i,j}=1).
\label{eqn:detailed-expansion}    
\end{equation}


By Lemma~\ref{lem:multinomial}, $\Pr(\bfW_{\bfZ,i,j}=1)=\bfK_{\bfZ,i,j}/\|\bfK_{\bfZ,i}\|_1$ for $j\neq i$. Recall that $\bfK_\bfZ=(k(\bfZ_i-\bfZ_j))_{(i,j)\in[n]^2}$, which means 
$\bfK_{\bfZ,i,j}/\|\bfK_{\bfZ,i}\|_1=\frac{\exp(-\|\bfZ_i-\bfZ_j\|^2/{2\tau})}{\sum_{k\neq i}
\exp(-\|\bfZ_i-\bfZ_k\|^2/{2\tau})
}$ for $j\neq i$, when $k$ is the Gaussian kernel with variance $\tau$. 

Notice that $\bfZ_i=f(\bfX_i)$, so we know
\begin{equation}
-\log \Pr(\bfW_{\bfZ,i,j}=1)=
-\log \frac{\exp(-\|f(\bfX_i)-f(\bfX_j)\|^2/{2\tau})}{\sum_{k\neq i}
\exp(-\|f(\bfX_i)-f(\bfX_k)\|^2/{2\tau}),
}
\label{eqn:infonce-equivalence}    
\end{equation}


The right hand side is exactly the InfoNCE loss defined in Eqn.~(\ref{eqn:infonce}).
Inserting Eqn.~(\ref{eqn:infonce-equivalence}) into Eqn.~(\ref{eqn:detailed-expansion}), we get the SimCLR algorithm, which first samples augmentation pairs $(i,j)$ with $\Pr(\bfW_{\bfX,i,j}=1)$ for each row $i$, and then optimize the InfoNCE loss. 

\textbf{Step 2: } minimizing the cross entropy loss 
is equivalent to spectral clustering on $\bfpi$.


By Lemma~\ref{lem:convert_to_spectral}, we may further convert the loss to 
\begin{equation}
\label{eqn:main-theorem-repul-attr}
\min_{\bfZ}
-\sum_{(i,j)\in [n]^2} \mathbf{P}_{i,j}
\log k (\bfZ_i-\bfZ_j)+\log \mathbf{R}(\bfZ).
\end{equation}
Since $k$ is the Gaussian kernel, this reduces to \[
\min_\bfZ \mathrm{tr}(\bfZ^\top \mathbf{L}(\bfpi) \bfZ)
+\log \mathbf{R}(\bfZ),
\]

where we use the fact that $\mathbb{E}_{\bfW_\bfX\sim \mathbb{P}(\cdot; \bfpi)}[\mathbf{L}(\bfW_\bfX)]
=\mathbf{L}(\bfpi)
$, because the Laplacian operator is linear and $
\mathbb{E}_{\bfW_\bfX\sim \mathbb{P}(\cdot; \bfpi)}(\bfW_\bfX)=\bfpi
$.
\end{proof}

\paragraph{Proof of Theorem \ref{thm:clip}.}
\begin{proof}
Since $\bfW_\bfX\sim \mathbb{P}(\cdot;\bfpi_{\mathbf{A}, \mathbf{B}})$, we know 
$\bfW_\bfX$ has exactly one non-zero entry in each row, denoting the pair that got sampled. 
A notable difference compared to the previous proof is we now have $n_\mathcal{A}+n_\mathcal{B}$ objects in our graph. CLIP deals with this by taking a mini-batch of size $2N$, 
such that $n_\mathcal{A}=n_\mathcal{B}=N$, and adding the $2N$ InfoNCE losses together. We label the objects in $\mathcal{A}$ as $[n_\mathcal{A}]$, and the objects in $\mathcal{B}$ as $\{n_\mathcal{A}+1, \cdots, n_\mathcal{A}+n_\mathcal{B}\}$. 

Notice that $\bfpi_{\mathbf{A}, \mathbf{B}}$ is a bipartite graph, so the edges of objects in $\mathcal{A}$ will only connect to object in $\mathcal{B}$ and vice versa. We can define the similarity matrix in $\cZ$ as $\bfK_\bfZ$, 
where $\bfK_\bfZ(i, j+n_\mathcal{A})=\bfK_\bfZ(j+n_\mathcal{A},i)= k(\bfZ_i-\bfZ_j)$ for $i\in [n_\mathcal{A}], j\in [n_\mathcal{B}]$, and otherwise we set $\bfK_\bfZ(i,j)=0$. 
The rest is same as the previous proof. 
\end{proof}

\paragraph{Proof of Theorem \ref{thm:exponential}.}

\begin{proof}
\label{proof:exponential}
Since the objective function consists of a linear term combined with an entropy regularization, which is a strongly concave function, the maximization problem is a convex optimization problem. Owing to the implicit constraints provided by the entropy function, the problem is equivalent to having only the equality constraint. We then introduce the Lagrangian multiplier $\lambda$ and obtain the following relaxed problem:

$$
\widetilde{E}(\boldsymbol{\alpha})=\psi_{1}-\sum_{i=1}^n \alpha_{i} \psi_{i}+\tau \sum_{i=1}^n \alpha_{i}\log \alpha_{i}+\lambda\left(\boldsymbol{\alpha}^{\top} \mathbf{1}_n-1\right).
$$

As the relaxed problem is unconstrained, taking the derivative with respect to $\alpha_{i}$ yields

$$
\frac{\partial \widetilde{E}(\boldsymbol{\alpha})}{\partial \alpha_{i}}=-\psi_{i}+\tau\left(\log \alpha_{i}+\alpha_{i} \frac{1}{\alpha_{i}}\right)+\lambda=0.
$$

Solving the above equation implies that $\alpha_{i}$ takes the form
$
\alpha_{i}=\exp \left(\frac{1}{\tau} \psi_{i}\right) \exp \left(\frac{-\lambda}{\tau}-1\right).
$ Since $\alpha_{i}$ lies on the probability simplex, the optimal $\alpha_{i}$ is explicitly given by
$
\alpha^{*}_{i}=\frac{\exp \left(\frac{1}{\tau} \psi_{i}\right)}{\sum_{i^{\prime}=1}^n \exp \left(\frac{1}{\tau} \psi_{i^{\prime}}\right)} .
$ Substituting the optimal point into the objective function, we obtain
$$
\begin{aligned}
E\left(\boldsymbol{\alpha}^*\right)  &=\psi_1-\sum_{i=1}^n \frac{\exp \left(\frac{1}{\tau} \psi_{i}\right)}{\sum_{i^{\prime}=1}^n \exp \left(\frac{1}{\tau} \psi_{i^{\prime}}\right)} \psi_{i}+\tau \sum_{i=1}^n \frac{\exp \left(\frac{1}{\tau} \psi_{i}\right)}{\sum_{i^{\prime}=1}^n \exp \left(\frac{1}{\tau} \psi_{i^{\prime}}\right)}\log \frac{\exp \left(\frac{1}{\tau} \psi_{i}\right)}{\sum_{i^{\prime}=1}^n \exp \left(\frac{1}{\tau} \psi_{i^{\prime}}\right)} \\
& =\psi_1 - \tau \log \left(\sum_{i=1}^n \exp \left(\frac{1}{\tau} \psi_{i}\right)\right).
\end{aligned}
$$
Thus, the Lagrangian dual function is given by
\begin{equation*}
-E\left(\boldsymbol{\alpha}^*\right)= -\tau \log \frac{\exp \left(\frac{1}{\tau} \psi_{1}\right)}{\sum_{i=1}^n \exp \left(\frac{1}{\tau} \psi_{i}\right)}.\qedhere
\end{equation*}
\end{proof}



\section{More on Experiments} \label{section: experiment_details}

\paragraph{CIFAR-10 and CIFAR-100} CIFAR-10 ~\citep{krizhevsky2009learning} and CIFAR-100 ~\citep{krizhevsky2009learning} are well-known classic image classification datasets. Both CIFAR-10 and CIFAR-100 contain a total of 60k $32 \times 32$ labeled images of different classes, with 50k for training and 10k for testing. CIFAR-10 is similar to CIFAR-100, except there are 10 different classes in CIFAR-10 and 100 classes in CIFAR-100.

\paragraph{TinyImageNet} TinyImageNet ~\citep{le2015tiny} is a subset of ImageNet ~\citep{deng2009imagenet}. There are 200 different object classes in TinyImageNet, with 500 training images, 50 validation images, and 50 test images for each class. All the images in TinyImageNet are colored and labeled with a size of $64 \times 64$.

\textbf{Pseudo-code.} Algorithm \ref{alg:Training Procedure} presents the pseudo-code for our empirical training procedure.

\begin{algorithm}[!htbp]
\caption{Training Procedure}
\label{alg:Training Procedure}
\begin{algorithmic}[1]
\REQUIRE trainable encoder network $f$, batch size $N$, augmentation strategy \textit{aug}, loss function $L$ with hyperparameters \textit{args}
\FOR {sampled minibatch ${x_i}_{i=1}^N$}
\FORALL{$i \in { 1, ..., N }$}
\STATE draw two augmentations $t_i = \textit{aug}\left(x_i\right) $, $t_i' = \textit{aug}\left(x_i\right) $
\STATE $z_i = f\left(t_i\right)$, $z_i' = f\left(t_i'\right)$
\ENDFOR
\STATE compute loss $\mathcal{L} = L(N, z, z', \textit{args})$
\STATE update encoder network $f$ to minimize $\mathcal{L}$
\ENDFOR
\STATE \textbf{Return} encoder network $f$
\end{algorithmic}
\end{algorithm}

We also provide the pseudo-code for our core loss function used in the training procedure in Algorithm \ref{alg:Core loss}. The pseudo-code is almost identical to SimCLR's loss function, with the exception of an extra parameter $\gamma$.

\begin{algorithm}[!htbp]
\caption{Core loss function $\mathcal{C}$}
\label{alg:Core loss}
\begin{algorithmic}[1]
\REQUIRE batch size $N$, two encoded minibatches $z_1, z_2$, $\gamma$, temperature $\tau$
\STATE $z = \textit{concat}\left(z_1, z_2\right)$
\FOR {$i \in {1, ..., 2N }, j \in {1, ..., 2N}$ }
\STATE $s_{i,j} = \Vert z_i - z_j \Vert_2^{\gamma}$
\ENDFOR
\STATE \textbf{define} $l(i, j)$ \textbf{as} $l(i, j) = - \log \frac{exp\left(s_{i,j}/\tau \right)}{\sum_{k=1}^{2N} \mathbf{1}{[k \ne i]} exp\left(s{i, j} / \tau \right)} $
\STATE \textbf{Return} $\frac{1}{2N} \sum_{k=1}^N\left[l(i, i+N) + l(i+N, i)\right]$
\end{algorithmic}
\end{algorithm}

Utilizing the core loss function $\mathcal{C}$, we can define all kernel loss functions used in our experiments in Table \ref{table: loss definition}. For all $z_i \in z$ with even dimensions $n$, we define $z_{L_i} = z_i\left[0:n/2\right]$ and $z_{R_i} = z_i\left[n/2:n\right]$.

\begin{table}[ht]
\centering
\begin{tabular}{{@{}l|l@{}}}
Kernel  &  Loss function \\ \midrule
Laplacian & $\mathcal{C}\left(N, z, z', \gamma=1, \tau\right)$\\ \midrule
Sum       & $\lambda * \mathcal{C}\left(N, z, z', \gamma=1, \tau_1\right) + (1-\lambda) * \mathcal{C}\left(N, z, z', \gamma=2, \tau_2\right)$  \\ \midrule
Concatenation Sum&$\lambda * \mathcal{C}\left(N, z_L, z'_L, \gamma=1, \tau_1\right) + (1-\lambda) * \mathcal{C}\left(N, z_R, z'_R, \gamma=2, \tau_2\right)$\\ \midrule
$\gamma = 0.5$ & $\mathcal{C}\left(N, z, z', \gamma=0.5, \tau\right)$          \\ 

\end{tabular}

\caption{Definition of kernel loss functions in our experiments}
\label {table: loss definition}
\end{table}

\textbf{Baselines.} We reproduce the SimCLR algorithm using PyTorch Lightning~\citep{PytorchLightning}.

\textbf{Encoder details.}
The encoder $f$ consists of a backbone network and a projection network. We employ ResNet50~\citep{ResNet} as the backbone and a 2-layer MLP (connected by a batch normalization~\citep{ioffe2015batch} layer and a ReLU \cite{nair2010rectified} layer) with hidden dimensions 2048 and output dimensions 128 (or 256 in the concatenation kernel case).

\textbf{Encoder hyperparameter tuning.}
For each encoder training case, we randomly sample 500 hyperparameter groups (sample details are shown in Table \ref{table: Hyperparameter sample}) and train these samples simultaneously using Ray Tune ~\citep{RayTune}, with the ASHA scheduler~\citep{li2018massively}. Ultimately, the hyperparameter group that maximizes the online validation accuracy (integrated in PyTorch Lightning) within 5000 validation steps is chosen for the given encoder training case.

\begin{table}[ht]
\centering

\begin{tabular}{@{}l|l|l@{}}
\midrule
Hyperparameter  & Sample Range & Sample Strategy \\ \midrule
start learning rate & $\left[10^{-2}, 10\right]$ & log uniform \\ \midrule
$\lambda$       & $\left[0, 1\right]$ & uniform \\ \midrule
$\tau$, $\tau_1$, $\tau_2$ & $\left[0, 1\right]$ & log uniform \\ \midrule
\end{tabular}

\caption{Hyperparameters sample strategy}
\label {table: Hyperparameter sample}
\end{table}

\textbf{Encoder training.} 
We train each encoder using the LARS optimizer~\citep{LARSOptimizer}, LambdaLR Scheduler in PyTorch, momentum 0.9, weight decay $10^{-6}$, batch size 256, and the aforementioned hyperparameters for 400 epochs on a single A-100 GPU.

\textbf{Image transformation.} The image transformation strategy, including augmentation, is identical to the default transformation strategy provided by PyTorch Lightning.

\textbf{Linear evaluation.}
The linear head is trained using the SGD optimizer with a cosine learning rate scheduler, batch size 64, and weight decay $10^{-6}$ for 100 epochs. The learning rate starts at $0.3$ and ends at $0$.

\textbf{Moco Experiments.} We also tested our method based on MoCo~\citep{he2019moco}. The results are summarized in Table \ref{tab:results-moco}. Here we choose ResNet18~\citep{ResNet} as the backbone and set a temperature of $0.1$ as default. For our simple sum kernel, we set $\lambda=0.8$. The results show that our method outperforms the original MoCo method.

\begin{table}[thb]
\centering
\caption{MoCo Experiment Results on CIFAR-10 and CIFAR-100.}
\label{tab:results-moco}
\resizebox{\textwidth}{!}{%
\begin{tabular}{@{}c|ccc|ccc@{}}
\toprule
\multirow{3}{*}{Method} & \multicolumn{3}{c|}{CIFAR-10} & \multicolumn{3}{c}{CIFAR-100} \\ \cmidrule(lr){2-4} \cmidrule(lr){5-7} 
                        & 200 epochs & 400 epochs    & 1000 epochs   & 200 epochs & 400 epochs & 1000 epochs         \\ \midrule
MoCo (repro.)         & $76.41 \pm 0.12$    & $80.01 \pm 0.15$          & $84.45 \pm 0.08$    & $\mathbf{47.02 \pm 0.11}$ & $52.50 \pm 0.07$ & $57.62 \pm 0.15$            \\
\midrule
Laplacian Kernel        & ${78.09 \pm 0.10}$    & $\mathbf{83.85 \pm 0.09}$          & $\mathbf{88.34 \pm 0.16}$    & $46.12 \pm 0.22$   & $53.44 \pm 0.17$ & $59.10 \pm 0.14$        \\
Simple Sum Kernel & $\mathbf{78.12 \pm 0.15}$   & $83.23 \pm 0.18$ & $87.50 \pm 0.20$ & $46.65 \pm 0.06$ & $\mathbf{53.62 \pm 0.19}$ & $\mathbf{59.83 \pm 0.12}$\\
\bottomrule
\end{tabular}
}
\end{table}



\section{More Experiments on Synthetic Data}


Consider a scenario with $n$ clusters, each containing $k$ vertices. Let the probability of vertices $u$ and $v$ from the same cluster belonging to $\bfpi$ be $p$. Conversely, for vertices $u$ and $v$ from different clusters, let the probability of belonging to $\pi$ be $q$. We generate the graph $\bfpi$ randomly, based on $p$ and $q$. We experiment with values of $k=100$ and $n=6$ for ease of visualization, embedding all points in a two-dimensional space. Each vertex's initial position originates from a normal distribution. In each iteration, we sample a subgraph of $\bfpi$ uniformly, ensuring each vertex has an out-degree of $1$. We then optimize the corresponding vectors using InfoNCE loss with an SGD optimizer and iterate until convergence. Our experimental setup consists of an SGD learning rate of $1$, an InfoNCE loss temperature of $0.5$, and a batch size of $50$. We evaluate two scenarios with different $p$ and $q$ values: $p=1$, $q=0$, and $p=0.75$, $q=0.2$. The results of these experiments are visualized in Figure \ref{fig:vis-spectral-cluster}. The obtained embeddings exhibit the hallmark pattern of spectral clustering of graph $\bfpi$.

\begin{figure}[!tb]
\centering
\subfigure{
\includegraphics[width=1\textwidth]{Figures/cluster_pi.png}
\label{fig:vis-cluster}
}
\subfigure{
\includegraphics[width=1\textwidth]{Figures/noised_cluster_pi.png}
\label{fig:vis-noised-cluster}
}
\caption{Visualizations of the optimization process using InfoNCE Loss on the vectors corresponding to $\bfpi$. Points of identical color belong to the same cluster within $\bfpi$. To showcase the internal structure of $\bfpi$, we randomly select 10 vertices from each cluster to display the edge distribution of $\bfpi$.}
\label{fig:vis-spectral-cluster}
\end{figure}



	
	
%\section{Appendix for Proofs}

\paragraph{Proof of Theorem \ref{thm:main}.}

\begin{proof}
\label{proof:main}
Our proof has two steps. In Step 1, we will show that SimCLR is equivalent to minimizing the cross entropy loss defined in Eqn.~(\ref{eqn:cross-entropy}). 
In Step 2, we will show  that minimizing the cross-entropy loss 
is equivalent to spectral clustering on $\bfpi$. 
Combining the two steps together, we have proved our theorem. 

\textbf{Step 1: } SimCLR is equivalent to minimizing the cross entropy loss.

The cross-entropy loss takes expectation over 
$\bfW_\bfX\sim \mathbb{P}(\cdot ; \bfpi)$, 
which means $\bfW_\bfX$ has exactly one non-zero entry in each row $i$. By Lemma~\ref{lem:multinomial}, we know every row $i$ of $\bfW_\bfX$ is independent of other rows. Moreover, 
$\bfW_{\bfX,i}\sim \mathcal{M}(1, \bfpi_i/\sum_j \bfpi_{i,j})=\mathcal{M}(1, \bfpi_i)$, because $\bfpi_i$ itself is a probability distribution.
Similarly, we know $\bfW_\bfZ$ also has the row-independent property by sampling over $\mathbb{P}(\cdot;\bfK_\bfZ)$.
Therefore, by Lemma~\ref{lem:cross_split}, we know Eqn.~(\ref{eqn:cross-entropy}) is equivalent to:
\[
 -\sum_{i=1}^n \mathbb{E}_{\bfW_{\bfX,i}}[\log \mathbb{P}(\bfW_{\bfZ,i}=\bfW_{\bfX,i};\bfK_\bfZ)],
\]

This expression takes expectation over $\bfW_{\bfX,i}$ for the given row $i$. Notice that 
$\bfW_{\bfX,i}$ has exactly one non-zero entry, which equals $1$ (same for $\bfW_{\bfZ,i}$). 
As a result
we expand the above expression to be:
\begin{equation}
 -\sum_{i=1}^n \sum_{j\neq i} \Pr(\bfW_{\bfX,i,j}=1)\log \Pr(\bfW_{\bfZ,i,j}=1).
\label{eqn:detailed-expansion}    
\end{equation}


By Lemma~\ref{lem:multinomial}, $\Pr(\bfW_{\bfZ,i,j}=1)=\bfK_{\bfZ,i,j}/\|\bfK_{\bfZ,i}\|_1$ for $j\neq i$. Recall that $\bfK_\bfZ=(k(\bfZ_i-\bfZ_j))_{(i,j)\in[n]^2}$, which means 
$\bfK_{\bfZ,i,j}/\|\bfK_{\bfZ,i}\|_1=\frac{\exp(-\|\bfZ_i-\bfZ_j\|^2/{2\tau})}{\sum_{k\neq i}
\exp(-\|\bfZ_i-\bfZ_k\|^2/{2\tau})
}$ for $j\neq i$, when $k$ is the Gaussian kernel with variance $\tau$. 

Notice that $\bfZ_i=f(\bfX_i)$, so we know
\begin{equation}
-\log \Pr(\bfW_{\bfZ,i,j}=1)=
-\log \frac{\exp(-\|f(\bfX_i)-f(\bfX_j)\|^2/{2\tau})}{\sum_{k\neq i}
\exp(-\|f(\bfX_i)-f(\bfX_k)\|^2/{2\tau}),
}
\label{eqn:infonce-equivalence}    
\end{equation}


The right hand side is exactly the InfoNCE loss defined in Eqn.~(\ref{eqn:infonce}).
Inserting Eqn.~(\ref{eqn:infonce-equivalence}) into Eqn.~(\ref{eqn:detailed-expansion}), we get the SimCLR algorithm, which first samples augmentation pairs $(i,j)$ with $\Pr(\bfW_{\bfX,i,j}=1)$ for each row $i$, and then optimize the InfoNCE loss. 

\textbf{Step 2: } minimizing the cross entropy loss 
is equivalent to spectral clustering on $\bfpi$.


By Lemma~\ref{lem:convert_to_spectral}, we may further convert the loss to 
\begin{equation}
\label{eqn:main-theorem-repul-attr}
\min_{\bfZ}
-\sum_{(i,j)\in [n]^2} \mathbf{P}_{i,j}
\log k (\bfZ_i-\bfZ_j)+\log \mathbf{R}(\bfZ).
\end{equation}
Since $k$ is the Gaussian kernel, this reduces to \[
\min_\bfZ \mathrm{tr}(\bfZ^\top \mathbf{L}(\bfpi) \bfZ)
+\log \mathbf{R}(\bfZ),
\]

where we use the fact that $\mathbb{E}_{\bfW_\bfX\sim \mathbb{P}(\cdot; \bfpi)}[\mathbf{L}(\bfW_\bfX)]
=\mathbf{L}(\bfpi)
$, because the Laplacian operator is linear and $
\mathbb{E}_{\bfW_\bfX\sim \mathbb{P}(\cdot; \bfpi)}(\bfW_\bfX)=\bfpi
$.
\end{proof}

\paragraph{Proof of Theorem \ref{thm:clip}.}
\begin{proof}
Since $\bfW_\bfX\sim \mathbb{P}(\cdot;\bfpi_{\mathbf{A}, \mathbf{B}})$, we know 
$\bfW_\bfX$ has exactly one non-zero entry in each row, denoting the pair that got sampled. 
A notable difference compared to the previous proof is we now have $n_\mathcal{A}+n_\mathcal{B}$ objects in our graph. CLIP deals with this by taking a mini-batch of size $2N$, 
such that $n_\mathcal{A}=n_\mathcal{B}=N$, and adding the $2N$ InfoNCE losses together. We label the objects in $\mathcal{A}$ as $[n_\mathcal{A}]$, and the objects in $\mathcal{B}$ as $\{n_\mathcal{A}+1, \cdots, n_\mathcal{A}+n_\mathcal{B}\}$. 

Notice that $\bfpi_{\mathbf{A}, \mathbf{B}}$ is a bipartite graph, so the edges of objects in $\mathcal{A}$ will only connect to object in $\mathcal{B}$ and vice versa. We can define the similarity matrix in $\cZ$ as $\bfK_\bfZ$, 
where $\bfK_\bfZ(i, j+n_\mathcal{A})=\bfK_\bfZ(j+n_\mathcal{A},i)= k(\bfZ_i-\bfZ_j)$ for $i\in [n_\mathcal{A}], j\in [n_\mathcal{B}]$, and otherwise we set $\bfK_\bfZ(i,j)=0$. 
The rest is same as the previous proof. 
\end{proof}

\paragraph{Proof of Theorem \ref{thm:exponential}.}

\begin{proof}
\label{proof:exponential}
Since the objective function consists of a linear term combined with an entropy regularization, which is a strongly concave function, the maximization problem is a convex optimization problem. Owing to the implicit constraints provided by the entropy function, the problem is equivalent to having only the equality constraint. We then introduce the Lagrangian multiplier $\lambda$ and obtain the following relaxed problem:

$$
\widetilde{E}(\boldsymbol{\alpha})=\psi_{1}-\sum_{i=1}^n \alpha_{i} \psi_{i}+\tau \sum_{i=1}^n \alpha_{i}\log \alpha_{i}+\lambda\left(\boldsymbol{\alpha}^{\top} \mathbf{1}_n-1\right).
$$

As the relaxed problem is unconstrained, taking the derivative with respect to $\alpha_{i}$ yields

$$
\frac{\partial \widetilde{E}(\boldsymbol{\alpha})}{\partial \alpha_{i}}=-\psi_{i}+\tau\left(\log \alpha_{i}+\alpha_{i} \frac{1}{\alpha_{i}}\right)+\lambda=0.
$$

Solving the above equation implies that $\alpha_{i}$ takes the form
$
\alpha_{i}=\exp \left(\frac{1}{\tau} \psi_{i}\right) \exp \left(\frac{-\lambda}{\tau}-1\right).
$ Since $\alpha_{i}$ lies on the probability simplex, the optimal $\alpha_{i}$ is explicitly given by
$
\alpha^{*}_{i}=\frac{\exp \left(\frac{1}{\tau} \psi_{i}\right)}{\sum_{i^{\prime}=1}^n \exp \left(\frac{1}{\tau} \psi_{i^{\prime}}\right)} .
$ Substituting the optimal point into the objective function, we obtain
$$
\begin{aligned}
E\left(\boldsymbol{\alpha}^*\right)  &=\psi_1-\sum_{i=1}^n \frac{\exp \left(\frac{1}{\tau} \psi_{i}\right)}{\sum_{i^{\prime}=1}^n \exp \left(\frac{1}{\tau} \psi_{i^{\prime}}\right)} \psi_{i}+\tau \sum_{i=1}^n \frac{\exp \left(\frac{1}{\tau} \psi_{i}\right)}{\sum_{i^{\prime}=1}^n \exp \left(\frac{1}{\tau} \psi_{i^{\prime}}\right)}\log \frac{\exp \left(\frac{1}{\tau} \psi_{i}\right)}{\sum_{i^{\prime}=1}^n \exp \left(\frac{1}{\tau} \psi_{i^{\prime}}\right)} \\
& =\psi_1 - \tau \log \left(\sum_{i=1}^n \exp \left(\frac{1}{\tau} \psi_{i}\right)\right).
\end{aligned}
$$
Thus, the Lagrangian dual function is given by
\begin{equation*}
-E\left(\boldsymbol{\alpha}^*\right)= -\tau \log \frac{\exp \left(\frac{1}{\tau} \psi_{1}\right)}{\sum_{i=1}^n \exp \left(\frac{1}{\tau} \psi_{i}\right)}.\qedhere
\end{equation*}
\end{proof}



\section{More on Experiments} \label{section: experiment_details}

\paragraph{CIFAR-10 and CIFAR-100} CIFAR-10 ~\citep{krizhevsky2009learning} and CIFAR-100 ~\citep{krizhevsky2009learning} are well-known classic image classification datasets. Both CIFAR-10 and CIFAR-100 contain a total of 60k $32 \times 32$ labeled images of different classes, with 50k for training and 10k for testing. CIFAR-10 is similar to CIFAR-100, except there are 10 different classes in CIFAR-10 and 100 classes in CIFAR-100.

\paragraph{TinyImageNet} TinyImageNet ~\citep{le2015tiny} is a subset of ImageNet ~\citep{deng2009imagenet}. There are 200 different object classes in TinyImageNet, with 500 training images, 50 validation images, and 50 test images for each class. All the images in TinyImageNet are colored and labeled with a size of $64 \times 64$.

\textbf{Pseudo-code.} Algorithm \ref{alg:Training Procedure} presents the pseudo-code for our empirical training procedure.

\begin{algorithm}[!htbp]
\caption{Training Procedure}
\label{alg:Training Procedure}
\begin{algorithmic}[1]
\REQUIRE trainable encoder network $f$, batch size $N$, augmentation strategy \textit{aug}, loss function $L$ with hyperparameters \textit{args}
\FOR {sampled minibatch ${x_i}_{i=1}^N$}
\FORALL{$i \in { 1, ..., N }$}
\STATE draw two augmentations $t_i = \textit{aug}\left(x_i\right) $, $t_i' = \textit{aug}\left(x_i\right) $
\STATE $z_i = f\left(t_i\right)$, $z_i' = f\left(t_i'\right)$
\ENDFOR
\STATE compute loss $\mathcal{L} = L(N, z, z', \textit{args})$
\STATE update encoder network $f$ to minimize $\mathcal{L}$
\ENDFOR
\STATE \textbf{Return} encoder network $f$
\end{algorithmic}
\end{algorithm}

We also provide the pseudo-code for our core loss function used in the training procedure in Algorithm \ref{alg:Core loss}. The pseudo-code is almost identical to SimCLR's loss function, with the exception of an extra parameter $\gamma$.

\begin{algorithm}[!htbp]
\caption{Core loss function $\mathcal{C}$}
\label{alg:Core loss}
\begin{algorithmic}[1]
\REQUIRE batch size $N$, two encoded minibatches $z_1, z_2$, $\gamma$, temperature $\tau$
\STATE $z = \textit{concat}\left(z_1, z_2\right)$
\FOR {$i \in {1, ..., 2N }, j \in {1, ..., 2N}$ }
\STATE $s_{i,j} = \Vert z_i - z_j \Vert_2^{\gamma}$
\ENDFOR
\STATE \textbf{define} $l(i, j)$ \textbf{as} $l(i, j) = - \log \frac{exp\left(s_{i,j}/\tau \right)}{\sum_{k=1}^{2N} \mathbf{1}{[k \ne i]} exp\left(s{i, j} / \tau \right)} $
\STATE \textbf{Return} $\frac{1}{2N} \sum_{k=1}^N\left[l(i, i+N) + l(i+N, i)\right]$
\end{algorithmic}
\end{algorithm}

Utilizing the core loss function $\mathcal{C}$, we can define all kernel loss functions used in our experiments in Table \ref{table: loss definition}. For all $z_i \in z$ with even dimensions $n$, we define $z_{L_i} = z_i\left[0:n/2\right]$ and $z_{R_i} = z_i\left[n/2:n\right]$.

\begin{table}[ht]
\centering
\begin{tabular}{{@{}l|l@{}}}
Kernel  &  Loss function \\ \midrule
Laplacian & $\mathcal{C}\left(N, z, z', \gamma=1, \tau\right)$\\ \midrule
Sum       & $\lambda * \mathcal{C}\left(N, z, z', \gamma=1, \tau_1\right) + (1-\lambda) * \mathcal{C}\left(N, z, z', \gamma=2, \tau_2\right)$  \\ \midrule
Concatenation Sum&$\lambda * \mathcal{C}\left(N, z_L, z'_L, \gamma=1, \tau_1\right) + (1-\lambda) * \mathcal{C}\left(N, z_R, z'_R, \gamma=2, \tau_2\right)$\\ \midrule
$\gamma = 0.5$ & $\mathcal{C}\left(N, z, z', \gamma=0.5, \tau\right)$          \\ 

\end{tabular}

\caption{Definition of kernel loss functions in our experiments}
\label {table: loss definition}
\end{table}

\textbf{Baselines.} We reproduce the SimCLR algorithm using PyTorch Lightning~\citep{PytorchLightning}.

\textbf{Encoder details.}
The encoder $f$ consists of a backbone network and a projection network. We employ ResNet50~\citep{ResNet} as the backbone and a 2-layer MLP (connected by a batch normalization~\citep{ioffe2015batch} layer and a ReLU \cite{nair2010rectified} layer) with hidden dimensions 2048 and output dimensions 128 (or 256 in the concatenation kernel case).

\textbf{Encoder hyperparameter tuning.}
For each encoder training case, we randomly sample 500 hyperparameter groups (sample details are shown in Table \ref{table: Hyperparameter sample}) and train these samples simultaneously using Ray Tune ~\citep{RayTune}, with the ASHA scheduler~\citep{li2018massively}. Ultimately, the hyperparameter group that maximizes the online validation accuracy (integrated in PyTorch Lightning) within 5000 validation steps is chosen for the given encoder training case.

\begin{table}[ht]
\centering

\begin{tabular}{@{}l|l|l@{}}
\midrule
Hyperparameter  & Sample Range & Sample Strategy \\ \midrule
start learning rate & $\left[10^{-2}, 10\right]$ & log uniform \\ \midrule
$\lambda$       & $\left[0, 1\right]$ & uniform \\ \midrule
$\tau$, $\tau_1$, $\tau_2$ & $\left[0, 1\right]$ & log uniform \\ \midrule
\end{tabular}

\caption{Hyperparameters sample strategy}
\label {table: Hyperparameter sample}
\end{table}

\textbf{Encoder training.} 
We train each encoder using the LARS optimizer~\citep{LARSOptimizer}, LambdaLR Scheduler in PyTorch, momentum 0.9, weight decay $10^{-6}$, batch size 256, and the aforementioned hyperparameters for 400 epochs on a single A-100 GPU.

\textbf{Image transformation.} The image transformation strategy, including augmentation, is identical to the default transformation strategy provided by PyTorch Lightning.

\textbf{Linear evaluation.}
The linear head is trained using the SGD optimizer with a cosine learning rate scheduler, batch size 64, and weight decay $10^{-6}$ for 100 epochs. The learning rate starts at $0.3$ and ends at $0$.

\textbf{Moco Experiments.} We also tested our method based on MoCo~\citep{he2019moco}. The results are summarized in Table \ref{tab:results-moco}. Here we choose ResNet18~\citep{ResNet} as the backbone and set a temperature of $0.1$ as default. For our simple sum kernel, we set $\lambda=0.8$. The results show that our method outperforms the original MoCo method.

\begin{table}[thb]
\centering
\caption{MoCo Experiment Results on CIFAR-10 and CIFAR-100.}
\label{tab:results-moco}
\resizebox{\textwidth}{!}{%
\begin{tabular}{@{}c|ccc|ccc@{}}
\toprule
\multirow{3}{*}{Method} & \multicolumn{3}{c|}{CIFAR-10} & \multicolumn{3}{c}{CIFAR-100} \\ \cmidrule(lr){2-4} \cmidrule(lr){5-7} 
                        & 200 epochs & 400 epochs    & 1000 epochs   & 200 epochs & 400 epochs & 1000 epochs         \\ \midrule
MoCo (repro.)         & $76.41 \pm 0.12$    & $80.01 \pm 0.15$          & $84.45 \pm 0.08$    & $\mathbf{47.02 \pm 0.11}$ & $52.50 \pm 0.07$ & $57.62 \pm 0.15$            \\
\midrule
Laplacian Kernel        & ${78.09 \pm 0.10}$    & $\mathbf{83.85 \pm 0.09}$          & $\mathbf{88.34 \pm 0.16}$    & $46.12 \pm 0.22$   & $53.44 \pm 0.17$ & $59.10 \pm 0.14$        \\
Simple Sum Kernel & $\mathbf{78.12 \pm 0.15}$   & $83.23 \pm 0.18$ & $87.50 \pm 0.20$ & $46.65 \pm 0.06$ & $\mathbf{53.62 \pm 0.19}$ & $\mathbf{59.83 \pm 0.12}$\\
\bottomrule
\end{tabular}
}
\end{table}



\section{More Experiments on Synthetic Data}


Consider a scenario with $n$ clusters, each containing $k$ vertices. Let the probability of vertices $u$ and $v$ from the same cluster belonging to $\bfpi$ be $p$. Conversely, for vertices $u$ and $v$ from different clusters, let the probability of belonging to $\pi$ be $q$. We generate the graph $\bfpi$ randomly, based on $p$ and $q$. We experiment with values of $k=100$ and $n=6$ for ease of visualization, embedding all points in a two-dimensional space. Each vertex's initial position originates from a normal distribution. In each iteration, we sample a subgraph of $\bfpi$ uniformly, ensuring each vertex has an out-degree of $1$. We then optimize the corresponding vectors using InfoNCE loss with an SGD optimizer and iterate until convergence. Our experimental setup consists of an SGD learning rate of $1$, an InfoNCE loss temperature of $0.5$, and a batch size of $50$. We evaluate two scenarios with different $p$ and $q$ values: $p=1$, $q=0$, and $p=0.75$, $q=0.2$. The results of these experiments are visualized in Figure \ref{fig:vis-spectral-cluster}. The obtained embeddings exhibit the hallmark pattern of spectral clustering of graph $\bfpi$.

\begin{figure}[!tb]
\centering
\subfigure{
\includegraphics[width=1\textwidth]{Figures/cluster_pi.png}
\label{fig:vis-cluster}
}
\subfigure{
\includegraphics[width=1\textwidth]{Figures/noised_cluster_pi.png}
\label{fig:vis-noised-cluster}
}
\caption{Visualizations of the optimization process using InfoNCE Loss on the vectors corresponding to $\bfpi$. Points of identical color belong to the same cluster within $\bfpi$. To showcase the internal structure of $\bfpi$, we randomly select 10 vertices from each cluster to display the edge distribution of $\bfpi$.}
\label{fig:vis-spectral-cluster}
\end{figure}



\end{document}
