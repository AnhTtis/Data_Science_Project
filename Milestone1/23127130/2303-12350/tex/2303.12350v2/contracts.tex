\section{Experiment Design}\label{contracts}

\indent Increasingly, algorithms are replacing human decision-makers, even in complex settings involving contracts and incentives. This raises a fundamental question: can algorithms autonomously learn to design and navigate contracts that incentivize desired behavior? To best understand how Q-learning algorithms works in a dynamic contract setting, we first explores this questions through the lens of Q-learning algorithms in a single-principal-agent problem, thereby extending the problem to dual-contract.


\subsection{Q-Learning in Repeated Games}

While initially developed for stationary Markov decision processes, Q-learning can be applied to repeated games like contractual settings \citep{calvano2020artificial}. However, standard Q-learning faces challenges in such environments:
\begin{itemize}
\item \textbf{Non-Stationarity:} Unlike stationary settings, players' strategies in repeated games evolve, making the environment non-stationary from any single player's perspective.
\item \textbf{Expanding State Space:} The history of actions, which forms the state space, grows with each iteration, posing computational challenges.
\end{itemize}

\subsubsection{Addressing the Challenges: Bounded Memory}

\indent To ensure tractability and potential convergence, we consider a naive case with \textbf{bounded memory}. This means each agent's decision depends only on the past $k$ interactions, limiting the state space's growth.\footnote{Bounding memory, while simplifying the problem, does not guarantee convergence in multi-agent Q-learning. The inherent non-stationarity from interacting adaptive agents persists. We investigate this issue in our experiments.}


\subsection{Dynamic Agency and Economic Environment}\label{single}

To facilitate a seamless transition to the dual-principal-agent scenario, we initiate our analysis within the context of a canonical dynamic single-principal-agent model. This approach builds upon the seminal work of \cite{innes1990limited} in the static context, which we adapt to the dynamic setting due to its inherent advantages:


\begin{itemize}
\item \textbf{Analytical Tractability:} The reference model admits a closed-form solution, providing a clear benchmark for evaluating the algorithm's performance in dynamic environments.\footnote{See \Cref{appendix} for details on the reference model.}

\item \textbf{Simplicity and Interpretability:} The model, built on intuitive economic parameters, aids in understanding the algorithm's learning process.

\item \textbf{Extensibility:} The framework naturally extends to a dynamic dual-contract paradigm, preserving interpretability while introducing analytical intractability.
\end{itemize}

\indent Building upon the reference model, We outline the dynamic model, economic environment, exploration strategy, and experimental design below.

\subsubsection{Model Setup}
The dynamic model involves a risk-neutral principal (investor) who offers a contract to a risk-neutral agent (entrepreneur). The agent's hidden effort level, which impacts project outcomes, is not directly observable by the principal, leading to the classic moral hazard problem. The principal's objective is to learn the optimal contract that maximizes their payoff, while simultaneously incentivizing the agent to exert effort. The model incorporates the following key features:
\paragraph{Key Features:}
\begin{itemize}
\item \textbf{Dynamic Setting:} Interactions occur over discrete time periods, allowing for learning and adaptation.
\item \textbf{Hidden Action (Moral Hazard):} The principal cannot directly observe the agent's effort, creating a challenge for incentive alignment.
\item \textbf{Limited Liability:} Similar to a debt contract, the agent's payoff is bounded below, influencing strategic interactions.
\item \textbf{Relaxed IR Constraint:} We relax the individual rationality constraint to focus on the algorithm's ability to learn incentive-compatible contracts without this assumption.\footnote{Namely, we remove the individual rationality (IR) constraint (see \Cref{eqn:07} in the Appendix) to allow AI algorithms to learn rational behavior autonomously.} 
\end{itemize}


\paragraph{Formal Structure:}
\begin{itemize}
\item \textbf{Time:} Discrete periods, $t = 1, 2, ..., T$.
\item \textbf{Project:} Requires initial investment $I$ from the principal.
\item \textbf{Outcomes:}
\begin{itemize}
\item $Revenue_t = I + (R-I)e_t$: Total revenue generated in period $t$, where $R > I$ is the exogenous maximum revenue, $I$ is the initial investment, and $e_t \in [0, 1]$ is the agent's effort.
\end{itemize}
\item \textbf{Contract Payments:}
\begin{itemize}
\item $\Pi_t^{P} = I + (R-I)e_{t}p_t$: Principal's profit in period $t$, which the principal aims to maximize by strategically setting the tax rate $p_t$ while anticipating the agent's effort response.
\item $\Pi_t^{A} = (1 - p_t)(R - I)e_t - \frac{1}{2}ce_t^2$: Agent's profit in period $t$, where $c$ is a cost parameter.
\end{itemize}
\item \textbf{Actions:}
\begin{itemize}
\item $p_t \in [0, 1]$: Principal's tax rate in period $t$, representing the share of the project's revenue the principal receives.
\item $e_t \in [0, 1]$: Agent's hidden effort level in period $t$.
\end{itemize}
\item \textbf{State Variables:}
\begin{itemize}
\item $s_t^{P} = (p_{t-1}, p_{t-2}, ..., p_{t-k}, \Pi^{P}_{t-1}, \Pi^{P}_{t-2}, ..., \Pi^{P}_{t-k})$: Principal's state, representing the past $k$ tax rates offered and the past $k$ profits.
\item $s_t^{A} = p_t$: Agent's state (observing only the current tax rate).
\end{itemize}
\end{itemize}

\paragraph{Key Points:}
\begin{itemize}
\item No IR constraint to showcase autonomous learning of rational behavior.
\item Dynamic learning with agents updating Q-functions based on observed outcomes.
\item Debt contract analogy with the model structure.
\end{itemize}

\paragraph{Q-Learning Optimization:}
Both the principal and the agent utilize Q-learning to optimize their strategies:
\begin{itemize}
\item \textbf{Agent:} $Q^{A}(s_t^{A}, e_t)$ estimates the expected discounted future profit:
\begin{equation}
Q^{A}(p_t, e_t) = \Pi_t^{A} + \delta \max_{e_{t+1}} Q^{A}(p_{t+1}, e_{t+1}),
\end{equation}
where $\delta$ is the discount factor.
\item \textbf{Principal:} $Q^{P}(s_t^{P}, p_t)$ estimates the expected discounted future revenue:
\begin{equation}
Q^{P}(s_t^P,p_t) = \Pi^{P}_t + \delta \max_{p_{t+1}} Q^{P}(s_{t+1}^P, p_{t+1}).
\end{equation}
\end{itemize}
\paragraph{Action Space:}
The principal's action space $\mathcal{A}$ consists of 101 possible tax rates, evenly spaced between 0\% and 100\% ($p \in {0, 0.01, ..., 0.99, 1}$). 

\paragraph{Q-Learning Dynamics:}
The principal's Q-function, $Q^P(s^{P},p)$, maps state-action pairs to expected rewards. The Q-table is initialized arbitrarily, and the Q-values are updated using the following rule:
\begin{equation}
Q_{t+1}^P(s_t^{P}, p_t) = (1 - \alpha) Q_t^P(s_t^{P}, p_t) + \alpha [\Pi_t^P + \delta \max_{p_{t+1} } Q_t^P(s^{P}_{t+1}, p_{t+1})],
\end{equation}
where $\alpha$ is the learning rate. This update rule allows the algorithm to gradually learn from experience and refine its estimates of the expected rewards for each state-action pair.

The agent's Q-function, $Q^A(s^A,e)$, also maps state-action pairs to expected rewards. The agent's state $s_t^A$ is simply the current tax rate $p_t$. The agent's action space is the set of possible effort levels. The agent updates their Q-function using a similar rule:
\begin{equation}
Q_{t+1}^A(s_t^A, e_t) = (1 - \alpha) Q_t^A(s_t^A, e_t) + \alpha [\Pi_t^A + \delta \max_{e_{t+1}} Q_t^A(s_{t+1}^A, e_{t+1})],
\end{equation}
where $\alpha$ is the learning rate for the agent (which could be different from the principal's learning rate), $s_{t+1}^A$ is the next period's tax rate.
In each period, both the principal and the agent observe the outcome (revenue) and update their Q-tables accordingly. This iterative process allows both players to learn the optimal strategies for maximizing their payoffs in this dynamic contract setting.
\paragraph{Memory:}
In our implementation, the Q-table serves as the principal's memory. It stores the current estimates of expected rewards for each state-action pair, denoted as $Q^P(s^P,p)$, where:
\begin{itemize}
\item $s \in \mathcal{S}$: Represents the state, which in this case is derived from the history of the past $k$ tax rates and profit as defined above.
\item $a \in \mathcal{A}$: Represents the action, which is the chosen tax rate $p_t$.
\end{itemize}
The parameter $k$ controls the extent of the principal's memory. The state space $\mathcal{S}$ consists of all possible combinations of the past $k$ tax rates and profit.\footnote{Note that the Q-table does not retain the complete history of interactions beyond what is encapsulated in the current state $s$. Formally, let $H_t = (p_0, \Pi^{P}_0, p_1, \Pi^{P}_1, ..., p_{t-1}, \Pi^{P}_{t-1}, p_t, \Pi^{P}_t)$ denote the complete history of actions and profit up to time $t$.} In our bounded memory approach, the principal's decision at time $t$ depends only on the current state $s_t$, which summarizes the past $k$ interactions, and the Q-table:
\begin{equation}
p_t = f(s_t^P, Q^P),
\end{equation}
where $f$ is the decision rule of the Q-learning algorithm, which, in this case, is the $\epsilon$-greedy strategy. This simplification, while making the problem computationally tractable, might limit the algorithm's ability to leverage the full information contained in the complete history. The influence of the memory length $k$ on the learning process and the algorithm's performance is a key aspect of our investigation.


\paragraph{Exploration:}

The principal employs an $\epsilon$-greedy exploration strategy, characterized by a time-decaying exploration rate $\epsilon_t$. In each iteration, the principal chooses the action with the highest estimated Q-value (exploitation) with probability $1-\epsilon_t$ and selects a random action (exploration) with probability $\epsilon_t$. The decaying exploration rate allows the algorithm to initially explore the action space extensively and gradually shift towards exploiting the learned knowledge as its confidence in the estimated Q-values increases.
We parameterize the exploration rate using:
\begin{equation}
\epsilon_t = e^{-\beta t},
\end{equation}
where $\beta$ controls the rate of decay. Higher values of $\beta$ lead to faster decay, resulting in quicker transitions from exploration to exploitation.


\subsection{Baseline Parametrization and Initialization}

To create a realistic contract learning scenario, we establish a specific set of parameters for our simulations. These parameters are summarized in \Cref{table:parameters}.
\begin{table}[ht!]
\centering
\caption{Parameter Values}
\label{table:parameters}
\begin{tabular}{lcc}
\toprule
Parameter & Single-Principal-Agent Model & Dual-Principal-Agent Model \\
\midrule
Maximum Revenue & $R = 2I$ & $R_1 = R_2 = 2$ \\
Initial Investment & $I = 1$ & $I_1 = I_2 = 1$ \\
Agent's Cost Parameter & $c = 2I$ & $c = I_1 + I_2=2$ \\
Discount Factor & $\delta = 0.9$ & $\delta = 0.9$ \\
Memory Length & $k = 5$ & $k = 1$ \\
Learning Rate & $\alpha \in [0.025, 0.25]$ & $\alpha \in [0.025, 0.25]$ \\
Exploration Rate Decay & $\beta \in [10^{-6} , 10^{-5}]$ & $\beta \in [10^{-6} , 10^{-5}]$ \\
Profit Alignment & Not applicable & $\gamma = 0, 0.25, 0.5$ \\
Principal Heterogeneity & Not applicable & $\kappa = 0, 0.25$ \\
\bottomrule
\end{tabular}
\begin{tablenotes}
\footnotesize
\item \textit{Note:} Values for $R$, $I$, $c$, and $\delta$ are kept consistent between the two models for comparability. The dual-principal-agent model introduces two additional parameters: $\gamma$ (profit alignment) and $\kappa$ (principal heterogeneity).
\end{tablenotes}
\end{table}

We fix the maximum revenue $R$ at twice the initial investment $I$, meaning $R = 2I$, and set the agent's cost parameter $c$ equal to $2I$, so $c=2I$. This setup ensures that incentivizing the agent's effort is essential for maximizing profit, as simply offering a high revenue share wouldn't guarantee high effort. Looking ahead to future profits, we use a discount factor $\delta = 0.9$, indicating that both the agent and principal value future gains but don't disregard immediate rewards. To ensure unbiased learning, we initialize the Q-table with random values, signifying no pre-existing knowledge of the optimal contract. Lastly, to manage computational complexity, we limit the principal's memory $k$ to 5 periods, meaning only the past 5 tax rates and profits influence its decisions.

\paragraph{Alpha-Beta Grids:}
Understanding the interplay between learning rate $\alpha$ and exploration decay $\beta$ is crucial for effectively applying Q-learning algorithms to our problems. To systematically explore this interplay, we employ a grid search approach across a range of $\alpha$ and $\beta$ values.
\begin{itemize}
\item $\alpha$: Represents the learning rate, which determines the weight assigned to new information during Q-value updates. \footnote{The learning parameter $\alpha$ may be in the principal range from $0$ to $1$. It is well known, however, that high values of $\alpha$ may disrupt learning when experimentation is extensive as the algorithm would forget too rapidly what it has learned in the past. Learning must be persistent to be effective, requiring that $\alpha$ be relatively small. In machine learning literature, a value of 0.1 is often used. We set the $\alpha \in [0.025, 0.25]$ in the parameter grids by following \cite{calvano2020artificial}.}
\item $\beta$: Governs the decay rate of the exploration parameter $\epsilon$ over time, influencing the balance between exploration (trying new actions) and exploitation (choosing actions with the highest known Q-values).
\end{itemize}
We discretize the parameter space by constructing uniform grids for both $\alpha$ and $\beta$. Specifically, $\alpha$ is drawn from 100 equally spaced values within the interval [0.025, 0.25], while $\beta$ ranges across 100 equally spaced values from $10^{-6}$ to $10^{-5}$. This procedure yields 10,000 unique $(\alpha, \beta)$ pairs. For each pair, we execute the Q-learning algorithm and evaluate its performance based on four key metrics:
\begin{itemize}
\item \textbf{Convergence Speed:} Measured as the number of iterations required for the algorithm to reach a stable tax rate policy.
\item \textbf{Profitability:} Calculated as the average profit accrued by the principal upon convergence of the algorithm.
\item \textbf{Stability:} Quantified by the magnitude of fluctuations in the chosen tax rate post-convergence. Lower fluctuations indicate higher stability.
\item \textbf{Optimality:} Assessed by the proximity of the learned tax rate to the theoretically optimal tax rate.
\end{itemize}


\subsection{Results}

\indent To ensure the robustness of our findings and account for the stochastic nature of the Q-learning process, we conduct \textbf{1000 independent sessions} of the simulation for each combination of learning rate $\alpha$ and exploration rate $\beta$ on the grid. In each session, the principal's Q-table is initialized randomly, and the algorithm interacts with the agent for a predetermined number of iterations. During each session, we record the principal's profit, agent's profit, tax rate chosen by the principal, and effort exerted by the agent in each iteration. Additionally, we track whether the algorithm converges to a stable tax rate, recording the converged tax rate and the number of iterations required for convergence. We then calculate the average of each metric over all iterations within a simulation. Finally, we average each metric across all 1000 simulations for a given ($\alpha$, $\beta$) pair to produce the results presented in \Cref{fig:subplots_combined}.

\begin{figure}[h!]
\centering
\includegraphics[width=1\textwidth]{results/single/subplots_combined.png}
\caption{Impact of Learning Rate $\alpha$ and Exploration Rate $\beta$ on Q-learning Dynamics in a Dynamic Contract Setting.
The heatmaps depict the average values of six key metrics over 1000 simulation sessions for each combination of $\alpha$ and $\beta$.
\textbf{Panel A} illustrates the average profit accrued by the principal.
\textbf{Panel B} shows the average profit gained by the agent.
\textbf{Panel C} presents the average tax rate chosen by the principal.
\textbf{Panel D} depicts the average effort exerted by the agent.
\textbf{Panel E} highlights the converged tax rate, if achieved.
\textbf{Panel F} displays the number of iterations required for convergence.}
\label{fig:subplots_combined}
\end{figure}

\indent \Cref{fig:subplots_combined} visualizes the impact of learning rate $\alpha$ and exploration rate $\beta$ on six key aspects of the Q-learning dynamics: average principal profit (Panel A), average agent profit (Panel B), average tax rate (Panel C), average agent effort (Panel D), converged tax rate (Panel E), and convergence iteration (Panel F).

\indent \textbf{Panel A (Average Principal Profit):} Higher learning rates consistently correspond to higher average principal profits for a given exploration rate. This suggests that a principal who can quickly integrate new information achieves superior performance. However, the magnitude of this effect diminishes as the exploration rate rises, indicating that excessive exploration can limit the benefits of a high learning rate.

\indent \textbf{Panel B (Average Agent Profit):} The pattern observed in Panel B reveals an inverse relationship between average agent profit and learning rate, particularly at lower exploration rates. This implies that the principal's enhanced ability to learn and optimize their strategy might come at the expense of the agent's payoff.

\indent \textbf{Panel C (Average Tax Rate):} A clear negative correlation exists between learning rate and the average tax rate employed by the principal. As the learning rate increases, the principal appears to converge towards lower tax rates, potentially indicating a shift towards less extractive and more collaborative contracts that encourage higher agent effort in the long run.

\indent \textbf{Panel D (Average Agent Effort):} Mirroring the trend in average agent profit (Panel B), agent effort generally declines as the learning rate rises. This reinforces the notion of a potential trade-off where the principal's increased learning efficiency might lead to lower agent incentives and effort.

\indent \textbf{Panel E (Converged Tax Rate):} This panel reveals intriguing dynamics in the convergence behavior. For lower exploration rates, the algorithm consistently converges to a stable tax rate, with higher learning rates generally leading to lower converged rates. However, as the exploration rate increases, the region of convergence shrinks, and at very high exploration rates, the algorithm fails to converge to a stable tax rate. This highlights the potential for instability and difficulty in reaching a clear optimal strategy when exploration is excessive.\footnote{In the context of Q-learning for contract design, the converged tax rate represents the final, stable tax rate the principal settles on after the algorithm has learned the optimal contract. This rate emerges from the algorithm's iterative process of experimenting with different tax rates, ultimately identifying the most effective balance between incentivizing the agent's effort and maximizing the principal's profit.
The converged tax rate offers crucial insights:
\begin{itemize}
\item Long-Term Contract Structure: It provides a glimpse into the enduring nature of the optimal contract that emerges from the learning process.
\item Efficiency: A lower converged tax rate, while maintaining high agent effort, typically suggests a more efficient contract design, highlighting the algorithm's ability to achieve optimal outcomes.
\end{itemize}
}

\indent \textbf{Panel F (Convergence Iteration):} The heatmap for convergence iteration illustrates the complex interplay of learning and exploration rates in determining how quickly the algorithm settles on a stable strategy. While higher learning rates generally accelerate convergence, particularly at lower exploration rates, there are regions where higher exploration leads to faster convergence, suggesting that a degree of exploration can be beneficial. However, very high exploration rates consistently hinder convergence regardless of the learning rate, emphasizing the importance of balancing exploration and exploitation for efficient learning.



\subsection{Statistical Analysis: Testing for Significance}

To rigorously assess the relationship between the learning rate $\alpha$ and the algorithm's performance, we conducted a series of statistical tests. We employed a two-way ANOVA (Analysis of Variance) with $\alpha$ and $\beta$ as independent variables and average profitability and convergence speed as dependent variables. The ANOVA model allowed us to test the null hypothesis of no significant difference in the dependent variables across different levels of $\alpha$ and $\beta$.


\begin{table}[ht!]
\centering
\caption{Impact of Learning Rate on Q-Learning Dynamics}
\label{table:alpha_impact}
\begin{threeparttable}
\begin{tabular}{p{3cm}p{3cm}p{3cm}p{4cm}}  \toprule
\textbf{Observation} & \textbf{Learning Rate $\alpha$} & \textbf{Exploration Rate $\beta$} & \textbf{Explanation} \\ \midrule
Higher $\alpha$ leads to higher principal profit. & Positive Correlation & Weakens with increasing $\beta$ &  At $\beta = 10^{-6}$, increasing $\alpha$ from 0.025 to 0.25 leads to a  principal profit increase (Panel A, Figure \ref{fig:subplots_combined}).  \\  \midrule
Higher $\alpha$ is associated with lower agent profit. & Negative Correlation & Stronger at lower $\beta$  & At $\beta = 10^{-6}$, increasing $\alpha$ from 0.025 to 0.25 decreases average agent profit (Panel B, Figure \ref{fig:subplots_combined}). \\ \midrule
Higher $\alpha$ results in lower average tax rates. & Negative Correlation & Consistent  &  Higher $\alpha$ generally corresponds to lower average tax rates, especially at lower exploration rates (Panel C, Figure \ref{fig:subplots_combined}).\\ \midrule
Higher $\alpha$ can be linked to lower average agent effort. & Negative Correlation &  Mirrors agent profit trend  &  This is likely due to lower tax rates associated with higher $\alpha$, leading to reduced immediate incentives for the agent (Panel D, Figure \ref{fig:subplots_combined}). \\ 
\bottomrule
\end{tabular}
\begin{tablenotes}
\footnotesize
\item \textit{Notes:}  ANOVA and t-tests reveal a statistically significant effect of $\alpha$ on profitability and convergence speed across various $\beta$ values. All observations are based on 1000 independent simulation runs for each parameter combination.
\end{tablenotes}
\end{threeparttable}
\end{table}

The results of the ANOVA analysis revealed statistically significant main effects of the learning rate $\alpha$ on both average profitability and convergence speed (p-value $<$ 0.05). This finding indicates that the choice of learning rate has a statistically significant impact on the algorithm's performance in this dynamic contract setting, independent of the exploration rate.

\begin{table}[h!]
\centering
\caption{Impact of Exploration Rate on Convergence Dynamics}
\label{table:beta_impact}
\begin{threeparttable}
\begin{tabular}{p{3cm}p{3cm}p{3cm}p{4cm}} \toprule
\textbf{Observation} & \textbf{Learning Rate $\alpha$} & \textbf{Exploration Rate $\beta$} & \textbf{Explanation} \\ \midrule
Convergence to a stable tax rate (Converged Tax Rate) exhibits complex dynamics. & Varies &  Region of convergence shrinks with increasing $\beta$ &  At low $\beta$ (around $10^{-6}$), convergence is consistent, with higher $\alpha$ leading to lower converged tax rates (around 0.04 for $\alpha$ = 0.25).  As $\beta$ increases, convergence becomes less frequent (Panel E, Figure \ref{fig:subplots_combined}). \\
 \midrule
Convergence speed, measured by the number of iterations (Convergence Iteration), is influenced by both $\alpha$ and $\beta$. &  Higher $\alpha$ typically accelerates convergence & High $\beta$ hinders convergence &  Higher $\alpha$ speeds up convergence, especially at lower $\beta$. For example, at $\beta = 10^{-6}$, increasing $\alpha$ from 0.025 to 0.25 reduces convergence iterations from 300 to 240.  However, high $\beta$ slows down convergence (Panel F, Figure \ref{fig:subplots_combined}).  \\
\bottomrule
\end{tabular}
\begin{tablenotes}
\footnotesize
\item \textit{Notes:} This table focuses on the impact of exploration rate on the convergence dynamics of the Q-learning algorithm. Key takeaway: Understanding the interplay of $\alpha$ and $\beta$ is crucial for optimizing algorithm performance. All observations are based on 1000 independent simulation runs for each parameter combination.
\end{tablenotes}
\end{threeparttable}
\end{table}

To further explore the specific relationships between pairs of learning rates, we conducted pairwise t-tests. These tests consistently confirmed the significant differences observed in the ANOVA analysis, reinforcing the conclusion that the learning rate plays a critical role in shaping the algorithm's behavior and outcomes.

The results summarized in \Cref{table:alpha_impact} and \Cref{table:beta_impact}, combined with the statistical analysis, provide a clear understanding of the interplay between learning rate $\alpha$ and exploration rate $\beta$ in the context of Q-learning for dynamic contract design.

\subsection{Discussion of Results: Implications}

\paragraph{Learning Rate Dominance:} Our findings demonstrate that the learning rate $\alpha$ significantly influences algorithm performance, leading to higher principal profits and lower agent profits, while generally resulting in lower average tax rates and agent effort.
\paragraph{Exploration's Complex Role:} The exploration rate $\beta$ exhibits a complex impact: while moderate exploration can be beneficial, high levels hinder convergence and slow down learning.
\paragraph{Balancing is Key:} Optimizing algorithm performance requires balancing exploration and exploitation. Future research should investigate this interplay, along with the effects of memory length and contract complexity.

\paragraph{Real-World Relevance:} These insights are crucial for developing and implementing Q-learning algorithms in dynamic contractual settings. By understanding the sensitivity to key parameters like $\alpha$ and $\beta$, we can design more efficient and effective algorithms.
