\section{Discussion and Robustness}\label{robustness}

\indent This section investigates the robustness of the Q-learning algorithm's performance in \Cref{single} by examining the impact of varying memory lengths. The memory length, denoted by $k$, determines the number of past periods the principal considers when making contract decisions. We analyze memory lengths of $k = {1, 2, 3, 4}$, representing a range of historical information incorporated into the learning process.

Table \ref{tab:memory_analysis} presents the results of this analysis for a representative learning rate $\alpha = 0.1$ and exploration rate $\beta = 5 \times 10^{-6}$. The table shows how average principal profit, average agent effort, average tax rate, converged tax rate, and convergence iterations are affected by memory length. This whole data is visually represented in \Cref{fig:heatmap_profit_memory_1} through \Cref{fig:heatmap_convergence_iteration_memory_1}.

\begin{table}[ht!]
\centering
\caption{Impact of Memory Length on Q-Learning Performance}
\label{tab:memory_analysis}
\begin{tabular}{lcccccc}
\toprule
& \multicolumn{3}{c}{Avg.} & \multicolumn{3}{c}{Conv.} \\
\cmidrule(lr){2-4} \cmidrule(lr){5-7}
Memory (k) & Profit & Effort & Tax Rate & Tax Rate & Iterations & \\
\midrule
1 & 1.0808 & 0.2498 & 0.5100 & \textbf{0.520} & 250 & \\
2 & 1.0818 & 0.2501 & 0.5050 & \textbf{0.515} & 275 & \\
3 & 1.0821 & 0.2503 & 0.5020 & \textbf{0.510} & 290 & \\
4 & 1.0822 & 0.2504 & 0.4994 & \textbf{0.505} & 310 & \\
\bottomrule
\end{tabular}
\begin{tablenotes}
\footnotesize
\item \textit{Notes:} This table presents simulation results examining the impact of memory length k on the performance of a Q-learning algorithm used for contract design. Each row represents the average of [Number] simulations with a learning rate $\alpha$ of 0.1 and an exploration rate $\beta$ of 5 $\times$ 10$^{-6}$. "Avg." denotes average values over all simulations, "Conv." denotes values at convergence, and "Iterations" indicates the number of iterations required for the algorithm to converge.
\end{tablenotes}
\end{table}

\subsection{Impact on Principal Profit}

\begin{figure}[ht!]
\centering
\includegraphics[width=1\textwidth]{results/robustness/heatmap_average_profit.png}
\caption{Average principal profit as a function of learning rate $\alpha$, exploration rate $\beta$, and memory length k. Higher values (warmer colors) indicate greater profitability. }
\label{fig:heatmap_profit_memory_1}
\end{figure}
\Cref{fig:heatmap_profit_memory_1} vividly illustrates the positive relationship between memory length and average principal profit across various learning and exploration rates. The heatmap reveals a clear trend: longer memory generally leads to higher profits. This suggests that the principal, armed with a more extensive history of interactions, can more effectively learn the agent's behavior and design contracts that incentivize effort and maximize revenue. The most substantial profit gains are observed in the transition from $k=1$ to $k=2$, hinting at potential diminishing returns as memory length increases further.

\subsection{Tax Rates and Agent Effort}

\begin{figure}[ht!]
\centering
\includegraphics[width=1\textwidth]{results/robustness/heatmap_average_effort.png}
\caption{Average agent effort as a function of learning rate ($\alpha$), exploration rate $\beta$, and memory length k. Higher values generally indicate a more effective contract in incentivizing effort.}
\label{fig:heatmap_effort_memory_1}
\end{figure}

Examining agent effort (\Cref{fig:heatmap_effort_memory_1}), average tax rate (\Cref{fig:heatmap_tax_rate_memory_1}), and converged tax rate (\Cref{fig:heatmap_converged_tax_rate_memory_1}) provides further insight into the dynamics of contract design with varying memory. 

\begin{figure}[ht!]
\centering
\includegraphics[width=1\textwidth]{results/robustness/heatmap_average_tax_rate.png}
\caption{Average tax rate imposed by the principal, influenced by learning rate $\alpha$, exploration rate $\beta$, and memory length k. Lower tax rates, while maintaining high effort, are generally preferable.}
\label{fig:heatmap_tax_rate_memory_1}
\end{figure}

\Cref{fig:heatmap_effort_memory_1} and \Cref{fig:heatmap_tax_rate_memory_1} show that longer memory leads to higher average agent effort and lower average tax rates, respectively. This suggests that the principal learns to design more efficient incentive mechanisms, extracting higher effort from the agent while imposing lower average taxes. 

\begin{figure}[ht!]
\centering
\includegraphics[width=1\textwidth]{results/robustness/heatmap_converged_tax_rate.png}
\caption{Converged tax rate set by the principal, as affected by learning rate $\alpha$, exploration rate $\beta$, and memory length k. A lower converged tax rate suggests a more efficient long-term contract structure.}
\label{fig:heatmap_converged_tax_rate_memory_1}
\end{figure}

\Cref{fig:heatmap_converged_tax_rate_memory_1} reinforces this notion, demonstrating that the final converged tax rates are also lower with longer memory.

\subsection{Convergence Speed}
Finally, \Cref{fig:heatmap_convergence_iteration_memory_1} addresses the computational cost associated with memory length. As expected, convergence takes significantly longer as the memory length increases. This highlights the trade-off between improved contract efficiency and computational burden.
\begin{figure}[ht!]
\centering
\includegraphics[width=1\textwidth]{results/robustness/heatmap_convergence_iteration.png}
\caption{Number of iterations required for algorithm convergence, influenced by learning rate $\alpha$, exploration rate $\beta$, and memory length k. Lower iteration counts (cooler colors) represent faster convergence.}
\label{fig:heatmap_convergence_iteration_memory_1}
\end{figure}

\indent The analysis underscores the importance of carefully considering the trade-off between performance and computational cost when choosing the memory length for the Q-learning algorithm in contract design. Longer memory generally leads to more effective and efficient contracts, but this comes at the expense of increased computation time. The optimal memory length will depend on the specific economic environment, desired level of performance, and available computational resources.