\documentclass[conference]{IEEEtran}
\IEEEoverridecommandlockouts
%
\usepackage{cite}
\usepackage{amsmath,amssymb,amsfonts}
\usepackage{algorithmic}
\usepackage{graphicx}
\usepackage{textcomp}
\usepackage{xcolor}
\usepackage[a4paper, total={184mm,239mm}]{geometry}

\usepackage{balance}
\usepackage{MnSymbol}
\usepackage{algorithmic}
\usepackage[linesnumbered,lined,commentsnumbered]{algorithm2e}%
\bibliographystyle{IEEEtran}    

\usepackage{makecell}

%
%

\def\BibTeX{{\rm B\kern-.05em{\sc i\kern-.025em b}\kern-.08em
    T\kern-.1667em\lower.7ex\hbox{E}\kern-.125emX}}
\begin{document}
%
\thispagestyle{plain}
\pagestyle{plain}

%
%
%
%
%

\title{Efficient deadlock avoidance in \\2D mesh NoCs that use OQ or VOQ routers}


\author{
\IEEEauthorblockN{Philippos Papaphilippou}
\IEEEauthorblockA{\textit{Department of Computing, %
%
Imperial College London, UK\,$^{*}$ \IEEEcompsocitemizethanks{$^*$The author is now with Huawei Technologies R\&D (UK) Limited.}}\\
%
p.papaphilippou17@alumni.imperial.ac.uk}
%
%
%
%
%
%
%
%
%
%
%
%
%
%
%
%
%
%
%
%
%
%
%
%
%
%
%
%
%
%
}

\maketitle

\begin{abstract}
%
%
%


Network-on-chips (NoCs) are currently a widely used approach for achieving scalability of multi-cores to many-cores, as well as for interconnecting other vital system-on-chip (SoC) components. Each entity in 2D mesh-based NoCs has a router responsible for forwarding packets between the dimensions as well as the entity itself, and it is essentially a 5-port switch. With respect to the routing algorithm, there are important trade-offs between routing performance and the efficiency of overcoming potential deadlocks. Common deadlock avoidance techniques including the turn model usually involve restrictions of certain paths a packet can take at the cost of a higher probability for network congestion. In contrast, deadlock resolution techniques, as well as some avoidance schemes, provide more path flexibility at the expense of hardware complexity, such as by incorporating (or assuming) dedicated buffers. 

This paper provides a deadlock avoidance algorithm for NoC routers based on output-queues (OQs) or virtual-output queues (VOQs). The proposed approach features fewer path restrictions than common techniques, and can be based on existing routing algorithms as a baseline, deadlock-free or not. This requires no modification to the queueing topology, and the required logic is minimal. %
Our algorithm approaches the performance of fully-adaptive algorithms, while maintaining deadlock freedom. %

\end{abstract}

\begin{IEEEkeywords}
%
NoC, SoC, deadlock avoidance, VOQs, OQs, NoC router
\end{IEEEkeywords}

\section{Introduction}

A network-on-chip (NoC) is an interesting and diverse approach for interconnecting a high number of computing entities. With the increase of the number of entities in today's processors and their heterogeneity, NoCs have an increasing presence in research. This also covers field-programmable gate arrays (FPGA), where NoCs are applied to prototyping, CGRA implementation \cite{9912662} or simply for connecting systems of smaller logic components. One of the fundamental challenges in NoCs are deadlocks, and this is usually solved at the routing level.

%

While the NoC research mostly focused on virtual channels for buffering, in FPGAs and embedded systems, it is also common to use NoC designs with routers based on output-queued (OQ) switches \cite{kapre2006packet, huan2012fpga} or input-queued (IQ) switches with virtual output queues (VOQs) \cite{wang2010performance, ahmed2014graceful}. These have a queue for every input-output combination. As a result, each router in the nodes of such NoCs has a queue per pair combination of the 5 directions (\{E, S, W, N, C\} for east, south, west, north and centre respectively). %

\vspace{0.3em}
\paragraph*{Limitations in state-of-the-art} While there is a plethora of research on deadlock freedom in virtual channels (VCs), NoCs that have a VOQ-like queue organisation still rely on simpler routing algorithms to achieve deadlock avoidance. This is because there is no theoretical background to increase path freedom for approaching full-adaptivity. While it is possible to adapt some methods from VCs, they are costly to apply, such as by introducing numerous additional %
queues for implementing escape channels. This limitation has an impact on routing performance, while the studied queue organisation remains popular among FPGA and embedded applications \cite{kapre2006packet, huan2012fpga, tpds23switches}. 

%
\vspace{0.5em}

\paragraph*{Insights} The main idea of this paper is that in NoC routers with VOQ-like queue organisation (one queue per input-output pair), deadlock avoidance can be achieved with more path flexibility than traditional models by \emph{calculating a worst case occupancy of specific queues}. This is based on the observation that \emph{this queue topology already implies turn information, which can be exploited to relax the turn model} for building more flexible deadlock-free routing algorithms. 

\vspace{0.5em}

\paragraph*{Motivation} In order to demonstrate the importance of path flexibility, figure \ref{fimot} presents simplified results from a \(8\times8\) NoC simulation with output-queues under 3 routing algorithms. The first algorithm is dimension order routing (DOR) which limits 1/2 of the possible turns, the second is "north last" which forbids 1/4 of the available turns. The third one, however, does not have any turn limitation, but has a potential for creating deadlocks. The traffic model for this example was purposely selected to produce lower probability for deadlocks, in order to show the potential impact of path flexibility on performance. 

\begin{figure}[h!]
\centering
\includegraphics[width=0.42\textwidth, trim=0 0 0 0]{motivation.pdf}
\caption{NoC throughput under different amounts of turn restrictions}\label{fimot}
%
\end{figure}

For instance, for a 55\% input rate of bit-reverse traffic \cite{dally2004principles}, DOR yields 0.4 
of throughput (extracted over injected packets), while the heuristic-based ``north-last" and the full-freedom ones yield 0.59 and 0.72 respectively. The goal is to approach the performance of the latter, but with deadlock freedom (these numerical values are case-specific, see section \ref{routper} for a more detailed discussion and simulations).
%



%
\vspace{0.5em}

\paragraph*{Contribution} The proposed solution is a hybrid approach that solves the deadlock problem more efficiently than traditional routing algorithms. It gives the illusion of full freedom to any routing algorithm, deadlock-free or not, while ensuring deadlock freedom. A fallback algorithm such as XY or YX dimension order routing (DOR) is activated only locally and when deemed necessary according to the freedom condition. The freedom condition requires minimal information to decide if a packet can perform an originally restricted-turn. This is achieved without assuming additional queues (such as escape channels \cite{verbeek2011necessary}, interface buffers \cite{farrokhbakht2021pitstop}), deadlock detection \cite{ramrakhyani2018synchronized} and recovery routines \cite {farrokhbakht2021pitstop}, misrouting (non-minimal path), packet reordering and global knowledge that are usually found in the literature on virtual channels. An intentional but indirect outcome of this paper is also the increased routing performance of example hybrid algorithms exploiting the additional flexibility of the formally proven deadlock-avoidance technique.

Following is background information (section \ref{backr}) on the related router architectures and the baseline deadlock avoidance model. Section \ref{sol} introduces the proposed algorithm, while section \ref{proof} is a proof for the correctness of the proposal. The evaluation (section \ref{eval}) includes simulations of example routing algorithms based on the proposed approach, as well as a study on the router resource utilisation. Finally, the paper concludes with related work (section \ref{rlw}), proposed future work (section \ref{ftw}) and a conclusion section (\ref{conc}).

\section{Background}\label{backr}

\subsection{Network-on-chip routers}

NoC routers are essentially switches, forwarding packets from port to port. The most common NoC routers are based on an input-queued switch with virtual channels (VCs). On each port (\{E, S, W, N, C\} for east, south, west, north and centre), the input is connected to a demultiplexer splitting the packets/flits into a fixed number of VCs (buffers). A virtual channel allocation scheme is one of the first decisions a NoC router makes on packet arrival, which can also relate to quality of service (QoS). 

Each group of VCs are then demultiplexed into the assumed crossbar (functionally a superset of permutation networks) connected to the outputs (ports), resulting in a \(5\times5\) switch. The buffering is necessary to facilitate the temporary storage of incoming packets until a passable matching is achieved between the output ports and all virtual channels. A matching is calculated on every fixed period of time and is used by the crossbar. See figure \ref{firouters} (right) for an (equivalent) illustration of a router with 5 VCs per input.

\begin{figure}[h!]
\centering
\includegraphics[width=0.42\textwidth, trim=0 0 0 0]{routersfig.pdf}
\caption{Output and input-queued NoC routers}\label{firouters}
%
\end{figure}

\paragraph{Input-queued routers with VOQs} 

Instead of virtual channels (VC), virtual output queues (VOQs) can be used to implement switches, and are especially common in network switches. There is a queue for every input-output port combination. Thus, there are \(P\) groups of \(P\) VOQs, resulting in \(P^2\) total buffers for a \(P\)-port switch (in NoCs, \(P=5\)).

A disadvantage of virtual output queues over virtual channels in NoCs is that they work as a static virtual allocation scheme in a switch with 4 or 5 virtual channels. This essentially means that the queue utilisation may be less efficient, and the buffer space is generally higher. It is also more challenging to reduce those queues (one way is to restrict routability, see the discussion of section \ref{resul}). Additionally, supporting quality of service can challenge scalability. 

However, as both VCs and VOQs support memory sharing per queue group, their differences can become less significant according to the implementation, also considering that 4 VCs may already be a norm in modern processors \cite{dai2022full, ma2012whole}. %


\paragraph{Output-queued (OQ) routers} 

Another approach to NoC router implementation is output-queues. They eliminate the use of a crossbar in favour of simpler logic. In this case, the queue organisation is the same with virtual output queues, and is illustrated in figure \ref{firouters} left. 

On FPGAs, this is a prominent NoC router architecture \cite{fpt21switch}, as with the split-merge switch \cite{huan2012fpga,kapre2006packet}. The split-merge switch is an output-queued switch meaning that, upon arrival, the incoming packets are immediately split across queues according to the destination port. Those queues are then grouped based on their destination port, and are multiplexed only once. This results in highly pipelinable logic, the split and merge units, equivalent to demultiplexers and multiplexers respectively. 

There is no need for an expensive scheduling algorithm, such as by featuring an iterative approach to perform well. There is only arbitration near the output, for every output port, which can be fulfilled by only using priority encoders. This also results in high scheduling performance, as it naturally provides more connectivity combinations than input queued for the same queue topology. This is because of the absence of additional arbitration steps near the inputs, which would otherwise serialise dequeueing from the output queues coming from the same input port.

A potential disadvantage of output-queued switches as NoCs routers is the limited scope for memory sharing, the associated cost of which also relates to QoS support. Both  memory sharing and QoS are seemingly less popular on FPGAs.

Our proposal provides the theoretical foundations to provide more flexibility in routing for such VOQ-like queue topologies, instead of relying on worse-performing traditional routing.

\subsection{Turn Model} %
\label{sec:turn-model}

%

The turn model can be used to create routing algorithms based on turn restriction to avoid forming any possible cyclic dependency \cite{glass1992turn}. The resulting set of possible algorithms can be summarised by the rules illustrated in figure \ref{fiturn}. There are clockwise and counterclockwise turns for which at least one turn from both must be forbidden. However, for each diagonal direction there must be at least one way for packets to travel, hence the ``\(\geq1\)'' sum rule per column in the figure. This model is a superset of the older ``dimension ordered routing'' (DOR), i.e. XY DOR for first traversing along the x-axis (no %
``\(\rcurvearrownw,\lcurvearrowne,\lcurvearrowsw,\rcurvearrowse\)'' turns) and YX DOR (no %
``\(\lcurvearrownw,\rcurvearrowsw,\rcurvearrowne,\lcurvearrowse\)'' turns).

When reducing the restrictions as much as allowed by this model, there needs to be one clockwise and one counterclockwise turn restriction, out of which they are not on a common path, as explained above. This results in 12 possible combinations for the forbidden turns (permutations of 2 from 4). In other words, the turn model gives 12 turn-restriction-based algorithms. However, if a rotated mesh is considered equivalent, this reduces to only 3, the popular ``west-first'' (no %
``\(\rcurvearrownw,\lcurvearrowsw\)'' turns), ``north-last'' (no %
``\(\rcurvearrownw,\lcurvearrowne\)'' turns) and ``negative-first'' (no  %
``\(\rcurvearrownw,\lcurvearrowse\)'' turns) routing algorithms.

%




\begin{figure}[h!]
\centering
\includegraphics[width=0.39\textwidth, trim=0 0 0 0]{turns2.pdf}
\caption{Turn model for deadlock avoidance}\label{fiturn}
%
\end{figure}

The deadlocks that happen in NoC routers with this queue topology are a bit less trivial than the classical example on a simple \(2\times2\) mesh. In a deadlocked \(2\times2\) mesh with a single queue (of length 1) per link per node, the remaining packets all have 2-hop path, as 1-hop paths would have been consumed directly by the destination, and 3-hop paths would have been U-turns (this paper only studies minimal path routing). In this case, there is a cyclic dependency of the four buffers. In the OQ and VOQ case, however, it is impossible to create a deadlock on a \(2\times2\) mesh, as all different 2-hop paths never pass from the same queue. 

\begin{figure}[h!]
\centering
\includegraphics[width=0.45\textwidth, trim=0 0 0 0]{voq_deadlock.pdf}
\caption{Deadlock state example from a simulation of a \(3\times4\) NoC with OQs }\label{fivd}
%
\end{figure}


Figure \ref{fivd} illustrates a deadlock state %
that is observed in a simulation of \(3\times4\) mesh NoC with OQs of length 2 that uses a routing algorithm without deadlock avoidance. The numbered nodes represent the NoC nodes, while the rectangular nodes represent the filled output queues, encoded as the concatenation of the node number and the turn of the queue (e.g. ``2) SW'' stands for the south-to-west forwarding (northwest direction) queue inside node 2). Note that the depicted arrows do not represent the specific head dependencies, though these are sometimes apparent, such as through the lack of activity in certain links.




\section{Proposed algorithm}\label{sol}

The proposed algorithm is hybrid, and solves deadlocks by selecting between a base algorithm and a fallback algorithm based on the freedom condition \(F\) (see section \ref{frco}) for every individual next hop decision. The fallback routing algorithm in our case shall follow the turn model, and can be for example XY DOR, ``west-first'', etc. The algorithm \ref{alg2} presents the distributed algorithm among the NoC's nodes that relates to the routing decisions.

\begin{algorithm}[h!]
\small
%
\LinesNumbered
\SetSideCommentRight
\SetKwComment{Comment}{$\triangleright$\ }{}
  %
 %
 %
%
int \(inp=from(p)\)\Comment*[r]{the input port receiving \(p\)}
reg \(sel\)\Comment*[r]{the output port selection}
 \While{forever}{
 receive (\(p\));\\
\eIf {F(p)==True}{
        \(sel\) $\leftarrow$ \(base\_algorithm(p)\);\\        
}{
        \(sel\) $\leftarrow$ \(fallback\_algorithm(p)\)\Comment*[r]{Turn model}
}    
OQs[\(inp\)][\(sel\)].\(enqueue(p)\);
 }
 %
\caption{Proposed %
routing algorithm for (V)OQs}\label{alg2}
\end{algorithm}

As the goal of this algorithm is to provide path flexibility, such as for better routing performance, the base algorithm is expected to allow a superset of the turns allowed by the fallback algorithm. The base algorithm can be any arbitrary routing algorithm, deadlock-free or not, which is also very useful for adopting algorithms that would otherwise require a specialised queue organisation for achieving deadlock freedom. Some examples include O1-Turn and LEF \cite{lef} that assert virtual channel allocation requirements. Like with turn model algorithms, the flexibility of the base algorithm is also appropriate for adaptive routing, where %
the routing decisions are based on heuristics such as for estimating network congestion. %

Sections \ref{frco}  and \ref{fradapt} provide a detailed description of the proposed freedom condition. %


\subsection{Assumptions} 

The proposed approach is studied under the following assumptions. %

\begin{enumerate}

\item The NoC is a 2D mesh and only minimal paths are allowed (no misrouting). The path length is always the Manhattan distance (i.e. \(\lvert x_1-x_2 \rvert + \lvert y_2-y_2\rvert\) between nodes with coordinates \((x_1,y_1)\) and \( (x_2,y_2)\)).

\item The target queue organisation is output-queues (OQs) and virtual-output queues on input-queued switches (VOQs). This totals \(5\times5\) distinct queues for any combination of an input and an output port including the core (N,W,S,E and C). This notion is kept for consistency, though the %
assumption of no misrouting results in \(4 \times 5\) queues, as sending to oneself is not supported. %

\item There are no separate interface queues, though the model can be adopted accordingly. %
%
%

\item For multi-flit packets, the head flit contains the number of flits for the whole packet. Otherwise, a maximum packet length can be used, where there is an upper limit. %
The paper focuses mostly on single-flit packets for brevity. %


\item %
Whole packet forwarding (WPF) is applied to enable an efficient use of the queues \cite{ma2012whole}. WPF allows output channels to be reused for more packets even if they are not empty, assuming the tail flit of the last packet is already received. This is especially useful in NoCs working as system interconnects and inside FPGAs, as shorter packets are more frequent \cite{ma2013novel}. 

\item The series of events when a node receives a packet is the following: routing (proposed algorithm), virtual channel allocation (fixed because of the (V)OQ organisation%
), scheduling (only applicable to the crossbar when using VOQs) and finally switch traversal.

\item A 5-bit vector (or equivalent) per NoC link indicates that a downstream router can accept a packet in each one of its %
queues connected to that port. In order for a packet to be considered ready for transmission, all bits for its possible path (up to two for minimal path) shall be high. The routing decisions of an upstream node is assumed not to influence the routing decisions of a downstream node, hence this requirement. %

\item The router's actions are considered to happen in the same cycle, thus the flow control is simplistic. This and the latter assumption can be generalised for pipelined implementations, such as by using a credit system, including for the freedom condition. 

\end{enumerate}
The following conventions are used in the nomenclature.

\begin{itemize}


\item The term ``turn'' is also sometimes used for forwardings from and to a port that is on the opposite side (going straight), as well as to the node's centre (``C''). These are considered legal %
in accordance to
the turn model. %
The centre in this case is a ``sink'', being able to consume packets directly \cite{holsmark2009deadlock, duato1993new}. A node's centre as a consumer cannot contribute to a deadlock, as every inserted packet is eventually consumed in a dedicated (virtual) output queue. A similar argument could be made for straight forwardings, though different models can also break circles while not being limited to turns only.

\item The turn and queue naming follows the directional path with respect to the ports of the same/upstream node, rather than real-life directions. For example, when mentioning a NW turn, it is not the northwest direction (i.e. between north and west), but the progression from the north port to the west port (southwest direction). 





%

%

%

\end{itemize}

%

\subsection{Freedom condition}\label{frco}
The freedom condition (\(F\)) is a sufficient %
condition for the avoidance of deadlocks, and is applied on packet arrival into any non-sink node (sink is the final destination). %

Let \(p\) a packet received from any port \(from(p) \in \{E, S, W, N, C\}\), and \(dest(p) \neq p\), where \(dest(p)\) is the destination node of \(p\).
Let \(sel \in \{E, S, W, N, C\}\) the proposed direction of the baseline routing algorithm that does not necessarily feature deadlock avoidance. 

The freedom condition is only useful whenever the next hop implied in \(sel\) could make \(p\) land in a queue of the next hop that would be a restricted turn based on the turn model. Using minimal path routing, when \(p\) arrives at the next hop, according to its designated final destination it can go straight (``\(\nearrow\)''), to the centre, clockwise (``\(\lcirclearrowright\)'') or counterclockwise (``\(\rcirclearrowleft\)''). Note that the latter two are the only ones that can be forbidden turns, and are mutually exclusive. Then,

\(F (p) \equiv        \)
%
%
%
%
%
%
%
%
%
%
\[ \begin{cases} %
      size(p) +  \sum_{p' \in q'_\lcirclearrowright} size(p') +\sum_{d \in \{C, \nearrow, \rcirclearrowleft\} } &  q'_\lcirclearrowright \in next( p) \land  \\
       \ \ \ \sum_{\{p' | q'_\lcirclearrowright \in next(p') \land p' \in q_d\}}  size(p') \leq cap(q'_\lcirclearrowright)& \ \ \ \neg  turn(q'_\lcirclearrowright) \\
       \\
      size(p) +\sum_{p' \in q'_\rcirclearrowleft} size(p') +\sum_{d \in \{C, \nearrow, \lcirclearrowright\} } &  q'_\rcirclearrowleft \in next(p) \land  \\
        \ \ \ \sum_{\{p' | q'_\rcirclearrowleft \in next(p') \land p' \in q_d\} } size(p') \leq cap(q'_\rcirclearrowleft)& \ \ \ \neg  turn(q'_\rcirclearrowleft) \\
\\
       \top & Otherwise,\\
   \end{cases}
\]
where %
\(cap (x)\) is the %
capacity of queue \(x\), %
\(q_d\) is a queue of the current node (receiving \(p\)) that receives packets %
from direction \(d\), while \(q'_d\) is a queue of the next hop as pointed by \(sel\). %
\(turn(x)\) indicates that the queue \(x\) corresponds to a legal turn based on an algorithm derived by the original turn model. 
%
The \(size(x)\) function indicates the number of flits a packet \(x\) consists of. This is useful in the cases where multi-flit packets are allowed, where this function only applies to the head flit of the packet, and the availability of this information is assumed. For non head-flits of a packet \(x\), \(size(x)=0\). %

In other words, if a packet \(p\) can do a clockwise or a counterclockwise turn on the next hop, and those turns are considered restricted based on the fallback algorithm, we check the worst case occupancy for that queue of the next hop. This is achieved by summing up the contents of all queues that feed into that queue (only the packets that can go into that queue), plus its own contents, and checking whether the addition of the incoming packet would cause an overflow in the worst case. The worst case is for all packets in the sum to end up in the restricted-turn queue and that for any reason it stops being consumed in the meantime. The time aspect for the worst case occupancy is for until packet \(p\) manages to be enqueued into the restricted turn/queue. Whenever \(F(p)=\bot\), \(p\) takes an alternative queue/ output port/ direction, which ensures the \(p\) will not be able to make a turn not following the turn model in the routing step of the next hop.



%

%

\subsection{Freedom condition adaptation}\label{fradapt}

In order to simplify the presentation of the freedom condition, this subsection provides an adaptation example based also on implementation practicality. The section provides a simplified sufficient condition \(F'\) based on system implementation assumptions and a higher degree of approximation for the queue occupancy, but with the same worst case (i.e. \(F'(p)\rightarrow F(p)\)).

%

One modification is to replace the packet counts having a potential queue target inside the next hop with the entire occupancy of each of the corresponding queues (denoted by \(occ(x)\) for each queue \(x\)). Although there can be an overhead in the path flexibility the algorithm provides, it could also be implemented more efficiently as the queue length circuitry is likely to be already existent. %

An adaptation is that the ``north last'' routing algorithm is selected as the fallback condition (or its least subset still following the turn model). By having the same output port (``N'') as the potential direction for the start of both forbidden turns for the next hop (SW, SE, i.e. ``\(\rcurvearrownw,\lcurvearrowne\)''), there needs to be less logic for the queue capacity checks in total. This is because \(q_C, q_\nearrow, q_\lcirclearrowright\) and \(q_\rcirclearrowleft\) represent the same queues for the turns SW and SE, happening at \(q'_{\rcirclearrowleft}\) and \(q'_\lcirclearrowright\)  respectively in the next hop across the N direction. Under this assumption, the counterclockwise, straight and clockwise of the current node in \(F\) become the EN, NN and WN queues (``\(\rcurvearrowne, \uparrow, \lcurvearrownw\)'') and for the next hop the SW, SN, SE (``\(\rcurvearrownw, \uparrow, \lcurvearrowne\)'') respectively.

The link overhead in this case is two southerly wires between each consecutive neighbouring nodes in the \(y\)-axis, of bit widths equal to \(\lceil \log_2(cap(q'_\lcurvearrowne))\rceil\) and \(\lceil \log_2(cap(q'_\rcurvearrownw))\rceil\) correspondingly. This metric makes the simplifying assumption that the occupancy of the next hop can be provided within the same cycle. A practical implementation with a credit system, such as with pipelined router implementations, is also likely to use fewer wires between the nodes.

The size of the packets (\(size(x)\) function) is also replaced by 1, as only single-flits are considered for brevity. %

%
%
%
%
%
%
%
%
%
%
%
%
%
%
%
%
%
%
%
%
%
\(F'(p) \equiv        \)
\[ \begin{cases} 
      1 +  occ(q'_\lcurvearrowne) +\sum\limits_{d \in \{C, \uparrow, \rcurvearrowne\} } occ (q_d)  \leq cap(q'_\lcurvearrowne) &  q'_{\lcurvearrowne} \in next(p) \\
       \\
      1 +  occ(q'_\rcurvearrownw) +\sum\limits_{d \in \{C, \uparrow, \lcurvearrownw\} } occ (q_d)  \leq cap(q'_\rcurvearrownw) &  q'_{\rcurvearrownw} \in next(p) \\
\\
       \top & Otherwise,\\
   \end{cases}
\]

Figure \ref{fiproof} illustrates the potential paths that involve non-allowable turns/queues by ``north-last'' as the fallback routing algorithm that restricts SW and SE turns. When a packet can make such turns in the next hop (associated with the queues \(q'_{\rcurvearrownw}\) and \(q'_{\lcurvearrowne}\)), then exactly one of the first two pieces of the piecewise function \(F'\) is used.

\begin{figure}[h!]
\centering
\includegraphics[width=0.45\textwidth, trim=0 0 0 0]{proof_help_4.pdf}
\caption{Packet paths involving originally-forbidden turns on the next hop}\label{fiproof}
%
\end{figure}

For the sake of notation simplicity both \(F\) and its variation \(F'\) are only valid when applied ``serially'' on the set of incoming packets per cycle (but still being a combinational circuit operating in the same cycle). In practice, their computation can be implemented in parallel for each input port, but %
%
there needs to be a simple arbitration step for synchronisation, such as with a priority encoder among the input ports. This is for whenever two packets can compete for the same potential queue of the next hop (on the current node packets from different ports/directions are still placed in different queues).


Without loss of generality, this version of the freedom condition is used for the remainder of the paper. %

\section{Proof}\label{proof}
%

In order to prove that the algorithm is always deadlock-free, a proof by %
contradiction is provided. %

Let \(c=\{q_0, q_1, ..., q_{n-1}, q_0\}\) a cyclic dependency \cite{duato1995necessary} between \(n\) buffers that has been allowed by the routing algorithm at time \(t\). As this is a deadlock, all queues in the cycle are full, with each head packet only able to be served by queues of the subsequent node (including the subsequent queue in the cycle). That is \(occ(q_i)=cap(q_i)\) \( \forall i \in \{0, 1, ..., n-1\}\), where \(occ (q_i)\) and  \(cap (q_i)\) is the occupancy and capacity of queue \(q_i\) respectively, Additionally, \(q_{j+1} \in next(head(q_j))\) for every \(j\) rotation function of \(i\), where \(next(head(q_j))\) is the set %
containing all possible buffer destinations of the head of queue \(q_j\). Note that as long as a packet is in a queue, the direction for the next hop (also associated with a single node) is already determined, but the queue placement in the receiving node will still be decided upon arrival on the next hop at time \(t+1\). %

For every \(<q_j, q_{j+1}>\) pair, a head packet proceeds from \(q_j\) to \(q_{j+1}\) if and only if the algorithm considers the packet making a legal forwarding to \(node(q_{j+1})\), and the \(node(q_{j+1})\) notifies that it will be able to accommodate it on cycle \(t+1\). %
%

As each queue represents a turn (a permutation of 2-selection from \{E, S, W, N, C\}), based on the turn model, it is impossible for the cyclic dependency to be based only out of turn/queues being inline with %
the turn model. That is \(\neg( \forall i, q_i \in c \land   turn(q_i))\), where \(turn(q_i)\) denotes that \(q_i\) follows the turn model. In other words, there is at least one queue/turn not following the turn model, i.e. \(\exists i, \ q_i \in c \land  \neg turn(q_i)\), for \(c\) to be able to be formed. %

Based on the turn model, as summarised in figure \ref{fiturn}, the forbidden turns alone are not able to form the circle \(c\) either, as they will consist of a strict subset of the 4 turns in the clockwise direction and a strict subset of the 4 turns in the counterclockwise direction. Therefore,  \(\neg(\forall i, q_i \in c \land   \neg turn(q_i)) \leftrightarrow        \exists i, \ q_i \in c \land  turn(q_i)\) as well. 

Thus, there is at least one consecutive queue pair that consist of one following the turn model and one that does not, i.e. \(\exists l=(i+k)\ \mathrm{mod}\ n, k \in \mathbb{Z}, turn(q_l) \land \neg turn(q_{l+1}) \). 

%
As a reminder, when a next hop decision \(sel \in \{E, S, W, N, C\}\) is made by the routing algorithm, %
there are two possible outcomes with respect to complying to the turn model. Either the next hop will only have the option to place the packet in queues that all follow the model, or the next hop will also have the option to make a restricted turn. Therefore, although \(sel\) specifies which virtual output queue the packet is stored in the upstream node as well, whether a queue follows the turn model is only reflected in the set of the potential queues for the next hop. For example, if a packet at (6, 0) has destination (8, 8) and has followed XY DOR routing until now, a \(sel=N\) before the \(x\) coordinate becomes the same with the packet's destination (not permitted in  XY DOR) would still make it stored in a WN queue (for ``\(\rcurvearrowne\)") locally. %
This turn is allowable in XY DOR, but the packet can land on either a NS queue (for ``\(\uparrow\)") or a SE queue (for ``\(\lcurvearrowne\)") on the next hop. The turn represented by the latter queue is originally not allowed by the turn model for XY DOR.

As \(q_{l+1} \in next(head(q_l))\), it means that at the time of insertion of the now \(head(q_l)\) to \(q_l\), the \(F'\) condition has been met. %
Assuming \(head(q_l)\) could do a clockwise rotation (\(q_{l+1} = q'_{\lcurvearrowne} \in next((head(q_l))\)) when it arrived at the node of \(q_{l+1}\), at the time of insertion  (similar for the counterclockwise equivalent):
%
%
\begin{align}
1 +  occ(q'_\lcurvearrowne) +\sum_{d \in \{C, \uparrow, \rcurvearrowne\} } occ (q_d)  &\leq cap(q'_\lcurvearrowne) \\
\rightarrow 1+ occ(q_{l+1}) +\sum_{d \in \{C, l, l'\}}{(occ(q_{d}))}  &\leq cap(q_{l+1}),
\end{align}
where \(q_{l'}\) is the other queue in the \(node(q_l)\) than \(q_l\), either representing the straight or counterclockwise movement from the previous hop of the (now) \(head(q_l)\).  There is also \(q_{C}\) that starts from the centre of \(node(q_l)\)). These queues (\(q_l\), \(q_{l'}\) and \(q_{C}\)) are the only queues that can feed \(q_{l+1}\), since the remaining %
 two are \(q_{Back}\) and \(q_{\lcurvearrownw}\). %
 A packet in \(q_{\lcurvearrownw}\) (also a clockwise turn) cannot arrive to \(q'_{\lcurvearrowne}\) (a clockwise turn), as U-turns are not allowed in minimal path routing. Similarly, \(q_{Back}\), i.e. the queue from port \(sel\) to port \(sel\) is disabled and only kept for notation consistency. %
 

%
%
%
%
%
%
 

This essentially means that at the time \(head(q_l)\) was enqueued into \(q_l\), the sum of all packets that could be destined for \(q_{l+1}\) plus 1 for the now \(head(q_l)\) would have %
been able to fit %
inside \(q_{l+1}\) %

Following the worst case approach,
%
the queue \(q_{l+1}\) stops being consumed in the meantime for any reason, such as congestion. %
Any packet after (the now) \(head(q_l)\) arrived in \(node(q_l)\) would have followed the fallback routing algorithm, and therefore it would either not being allowed to be forwarded to \(node(q_{l+1})\), or it would be stored in different queues of the \(node(q_{l+1})\). 
This is because we have the following cases for any successor packet \(p'\) to \(head(q_l)\) for \(node(q_{l})\) with respect to the north direction: %

\begin{itemize}
\item[--] \(p'\) arrives from the centre, west or south port of \(node(q_i)\) and \(p'\) can land in \(q_{l+1}\). As \(F'(p')=\bot\), \(p'\) will follow the fallback algorithm which will place it in other queues than \(q_C\), \(q_l\) or \(q_{l'}\) among the queues fed by centre, west or south. %
%
As the turn model always provides full-routability without relying on the non-allowable turns, any forthcoming packet that can land in \(q_{l+1}\) %
can simply follow the fallback routing algorithm, which will ensure a next hop without a queue violating the turn model. Such packets would land in other nodes than \(node(q_{l+1})\). %
\item[--] \(p'\) arrives from any port and \(p'\) cannot land in \(q_{l+1}=q'_\lcurvearrowne\) or \(q'_{l+1}=q'_\rcurvearrownw\). %
If the destination of \(p'\) is towards the north, it can still be placed in \(q_C\), \(q_l\) or \(q_{l'}\) according to its input port. The fallback condition will always hold (i.e. \(F'(p')=\top\)). This case is when a packet will always go straight for the remainder of its path, and this happens when the current %
\(y\)-axis coordinate is the same as the one of the packet destination. As the selected turn model only restricts north, \(p'\) could still pass through \(node(q_{l+1})\), but would land in its \(q'_{\uparrow}\) or \(q'_{C}\) queues only.
\item[--] \(p'\) arrives from any port and \(p'\) can land in \(q'_{l+1}=q'_\rcurvearrownw\). When  \(p'\) can land in \(q'_{l+1}=q'_\rcurvearrownw\), it cannot also have \(q_{l+1}\) as a potential destination, as U-turns are not allowed in minimal routing. In such cases, the equivalent observations are made for the values of \(F'(p')\) for the counterclockwise case. \(p'\) can still be placed in \(q_C\), \(q_l\) or \(q_{l'}\) (only \(q_{\uparrow}\) from the latter two), but for clockwise turns such packets can only make \(F'(p')=\bot\) earlier rather than the opposite (from \(\bot\) to \(\top\)). %
\end{itemize}

%
%

%

\begin{figure*}[h!]
\centering
\includegraphics[width=1\textwidth, trim=0 0 0 0]{out_lat_p.pdf}
\caption{Comparison of average packet latency using different routing algorithms under different traffic patterns}\label{figlat}
%
\end{figure*}

In all the aforementioned cases, \(head(q_l)\) would have been the last packet of all queues \(q_l\), \(q_{l'}\) and \(q_{C}\) that can land in \(q_{l+1}\) (for which \(\neg turn(q_{l+1})\)) on the next hop. As \(F'(p')=\top\) for \(p'\) being the \(head(q_l)\) at the time of insertion, it would have been able to be forwarded before \(q_{l+1}\) became full. This packet would have been able to become the tail of \(q_{l+1}\) at least. Thus, when \(q_{l+1}\) is full, the head of \(q_l\) could not have a dependency to \(q_{l+1}\), i.e. \(q_{l+1} \notin next(head(q_l))\). As \(q_{l+1} \in next(head(q_l))\), this is a contradiction.


\section{Evaluation}\label{eval}


The proposed technique is evaluated based on indicative adaptations. First, section \ref{routper} uses high-level simulation to comment on the algorithmic performance. Then, section \ref{resul} provides more details on the resource utilisation based on a NoC implementation in hardware description language (HDL).

\subsection{Routing performance}\label{routper}

The performance of the proposed routing algorithm methodology is %
studied under a variety of synthetic traffic models in simulation. The presented results are for an \(8 \times 8\) 2D mesh NoC, the routers of which use output queues. 

Two example routing algorithms are provided, and are based on the proposed approach. The first one is ``XY/O1-Turn" which uses XY DOR as a fallback algorithm (equivalent to ``north-last" in our arrangement, as \(F'=\top\) for south ports). It is based on O1-Turn \cite{o1turn} which randomly selects XY or YX DOR for the whole path of a packet, and does not provide adaptiveness. The second example is ``XY/Adaptive", which alternates between ``north-last" and full-freedom according to \(F'\) per packet arrival.

As adaptive routing algorithms like ``north-last" introduce a degree of turn freedom, heuristics are used to avoid congestion. For the purposes of this study, the adaptiveness heuristic is consistent across all experiments. This is enabled where applicable, including for the novel ``XY/Adaptive". The higher priority is given to the turn of which the direction maps to the queue with the least occupancy. In this case, the information used by the heuristic is still local to the node, but the queue occupancy is also indirectly associated with the congestion in the corresponding next hop.

The traffic models used in this evaluation are selected with diversity in mind. First, uniform Bernoulli arrivals and uniform bursty traffic are the most common models for interconnection circuits and relate to system interconnect use cases \cite{lef, tpds23switches}. Then, bit complement, bit reverse, bit rotate \cite{dally2004principles} and butterfly produce destination permutations based on the corresponding bit manipulation operations on the destination address. These and the transpose model are based on applications, such as the latter for FFT \cite{dally2004principles}. Finally, hotspot is the Bernoulli model modified for the central node %
to receive requests with four times higher probability than the rest of the nodes, modelling system-on-chip behaviour \cite{dally2004principles,lef}.


\begin{figure*}[h!]
\centering
\includegraphics[width=1\textwidth, trim=0 0 0 0]{out_throughput_p.pdf}
\caption{Comparison of throughput among different routing algorithms under different traffic patterns}\label{figthr}
%
\end{figure*}


Figure \ref{figlat} presents the performance results from this experiment with respect to the average packet latency. Each output queue has a depth of 16 flits-packets. There are only single-flit packets and the forwarding from a node can happen under a cycle, also according to the queue states. %
Each simulation has a region of interest equal to 5,000 cycles after a 500-cycle warmup period. %
When the average latency is predicted to become above 1500 cycles, the simulation stops early and the series stops to save simulation time. Each data point is an average of 5 runs. The observations for the equivalent experiment with virtual output queues are similar but not shown for brevity.

As a general observation the novel ``XY/O1-Turn" and ``XY/Adaptive" are the winners in the majority of the traffic patterns, achieving the lowest average packet latency for almost any presented injection rate. Two noticeable exceptions are the uniform Bernoulli and bit-complement cases. In the first, both fully-adaptive algorithms are marginally worse than with dimension order routing (DOR). In the second case, ``XY/Adaptive" comes third, but it is not a close third, so it could be said that `XY/O1-Turn" is a more balanced solution.




The second-class performance of the full-adaptiveness examples under certain traffic cases is expected, but it is not a limitation of the deadlock avoidance model. There are traffic patterns where additional turn-freedom is not always beneficial, at least when the adaptiveness heuristic has a more-local scope \cite{ebrahimi2012catra}. Hence, instead of a fine-tuned routing algorithm, the main focus is the theoretical model that allows a superset of potential packet paths than is currently feasible. For instance, as XY and YX DOR use a subset of the allowable turns, future adaptiveness heuristics could still use the proposed model while reverting to XY or YX DOR where deemed beneficial, for example.


Another observation is that overall ``XY/O1-Turn" and ``XY/Adaptive" are more well-rounded than the alternative algorithms in this selection, always being among the top performers. For example, odd-even routing \cite{chiu2000odd} sometimes performs the worst, as under Bernoulli traffic and the bit-complement cases, whereas for bit reverse and transpose traffic is the next best alternative to the proposed ones.

Figure \ref{figthr} presents the throughput results from the same experiment for numerical comparison examples. %
The throughput is defined as the average number of extracted packets per node per cycle. On average for all the traffic models, under a 35\% injection rate, ``XY/Adaptive" provides 1.23, 1.22, 1.17, 1.28, 1.19 and 1.27x %
the throughput of XY, YX DOR, west-first, negative-first, north-last and odd-even respectively. At 35\% the corresponding numbers for ``XY/O1-Turn" remain very similar at 1.23, 1.22, 1.17, 1.29,  1.19 and 1.28 times.

%

\subsection{Resource utilisation}\label{resul}

In order to study the hardware utilisation of the proposed deadlock avoidance methodology, an \(8\times8\) NoC is implemented in System Verilog and synthesised using yosys \cite{wolf2016yosys}. 

Each NoC router has a \(5\times5\) output-queued switch, with 25 queues in total. The size of each queue is 8 packets, and each packet is 64-bit wide. The router queues are based on a rather popular formally-verified synchronous queue \cite{bushnyuzi}. As building an optimised and specialised implementation is outside the scope of this paper, the router designs do not feature pipelining and the NoC nodes do not perform a useful task (random packet generation). The results are presented for both a standard cell library (``cmos\_cells.lib'') and 6-input lookup tables that are found in modern FPGAs (6-LUTs).

Table presents the synthesis results for 4 variations of the NoC router based on their routing algorithm. From left to right, each approach is expected to use more resources. North-last uses queue information in its heuristic, while DOR does not. ``Full-freedom" (deadlock-prone) uses a superset of this information, as its heuristic is responsible for the turns from the north as well. This is the equivalent of ``4/4 of turns'' from figure \ref{fimot}. Finally, the proposed example routing algorithm ``XY/Adaptive" includes the signals required for the freedom condition, hence also the two extra wires in the y-axis carrying occupancy information (for queues \(q'_{\rcirclearrowleft}\) and \(q'_\lcirclearrowright\)). This is reflected in the addition of 4 public wires, two as inputs in the north direction and two as outputs of the downstream router for monitoring the two restricted turns. 


\begin{table}[h!] 
  \caption{Routing algorithm impact on router resource utilisation} 
\label{tab1} 
%
\centering
  
 \setlength{\tabcolsep}{5pt}
\begin{tabular} {l|r r r r r }

&XY DOR&North-last&\thead{Full-freedom \\ (deadlock-prone)}&XY/Adaptive \\

\hline
\\
&\multicolumn{4}{c}{\textit{Standard cell library-based}}\\
\\
Wires&10707&10985&11230&11458\\
Public wires&183&183&183&187\\
Cells&5630&5797&5962&6062\\
NAND&3476&3803&3566&3823\\
NOR&1455&1344&1649&1518\\
NOT&659&610&707&681\\
\\
&\multicolumn{4}{c}{\textit{Look-up table-based}}\\
\\
Wires&885&957&977&1002\\
Public wires&182&182&182&186\\
Cells&1132&1204&1224&1245\\
6-LUT&1092&1164&1184&1205\\
\\
&\multicolumn{4}{c}{\textit{Common}}\\
\\
DFF&15&15&15&15\\Sync. FIFOs&25&25&25&25\\

%
%
\end{tabular}
%
\end{table}

%
It is also worth mentioning that for the %
turn-restriction based (XY DOR and north-last) yosys was not able to optimise unused queues away, as 25 queues were used in all cases. As the proposal is based on the observation that (V)OQs imply turn information, we know that the first two cases could result in (4 for XY DOR, and 2 for north-last) fewer queues. Additionally, as synthesis is also based on heuristics, small variations are expected, such as in the distribution of logic cells in the standard library.


As can be observed, ``XY/Adaptive" being a superset of the ``full-freedom" in terms of logic complexity, it has a small but measurable overhead on the wire and cell utilisation. By using the standard cell library, the most noticeable change is in the NAND gate count that increases by 7\%. For the FPGA case, the proposed approach only uses 21 more LUTs %
($<$2\% increase).

%



\section{Related Work}\label{rlw}

Deadlock avoidance, or detection and recovery in NoCs is still an active and thorough research topic. A great majority of the related literature is about NoCs  based on virtual channels (VCs) instead of (virtual) output queues. Thus, our research aims to close this knowledge gap, and can also be considered the (V)OQ-equivalent solution of modern deadlock research such as for improving deadlock avoidance in FPGA-based NoCs.

%

%

%

There are a number of works also focusing on enabling fully-adaptive routing, but with VCs \cite{ma2013novel,puente2001adaptive,verbeek2011necessary}. Initially, this could be achieved with additional virtual channels called ``escape" channels, where certain packets could resort to for avoiding deadlocks. Conditional forwarding \cite{yu2016conditional} is a deadlock avoidance mechanism and has similar aspects to our proposed approach. This is because it also uses a function (``conditional forwarding flow control") to determine if a packet can follow a restricted path. It aims to increase flexibility in VC allocation, though. This is achieved by eliminating the need to have separate escape channels, which is a common aspect with our approach on (V)OQs.

Irrespective of the freedom condition, the proposed algorithm as illustrated in algorithm \ref{alg2} is also inspired by the DyAD routing algorithm \cite{hu2004dyad}. DyAD also selects between two component algorithms, though the algorithm selection in DyAD only relates to routing performance instead of deadlock avoidance. 

\section{Future work}\label{ftw}

Future research could focus on exploiting the additional flexibility provided by the model, in order to build and deploy new routing algorithms in OQ and VOQ-based NoCs. This could also be done in the context of novel uses of FPGAs \cite{arc22fpgaext}. It would also be appropriate to elaborate on a flow control-based use, such as for building optimised and specialised implementations of routers with algorithms relying on the freedom condition.
 
It is worth mentioning that there could be additional and/or different simplifications to \(F'\) (see section \ref{fradapt}) 
based on restrictions or architectural assumptions that still satisfy \(F\). One such example could be a quantisation of the queue length measurements for reducing the related signals. %
Another example is to force the 1st hop of all packets to follow the turn model so that \(q_C\) will not need to be checked alongside the other queues that contribute to the contents of \(q'_{\rcirclearrowleft}\) and \(q'_\lcirclearrowright\). The trade-offs between the decisions in such a design space, especially with respect to circuit complexity and algorithm performance could be interesting %
to explore further.



\section{Conclusions}\label{conc}

This paper relaxes the turn model for building fully-adaptive routing algorithms on NoCs with an output-queued or virtual output-queued router architecture. These two queueing topologies are useful in applications such as in FPGAs, where deadlock avoidance is traditionally achieved using non-fully-adaptive algorithms. The proposed algorithm is a hybrid algorithm, with a fallback component being activated only locally based on the proposed freedom condition. The provided example routing algorithms ``XY/Adaptive" and ``XY/O1-Turn" generally outperform older algorithms significantly in simulation. These presented examples are relatively well-rounded across the traffic model selection, though the proposed approach can be used to build other novel routing algorithms and is easily generalisable. An example implementation of ``XY/Adaptive" is shown to have minimal overhead on resource utilisation over when not featuring deadlock avoidance for full routing freedom.

\section*{Acknowledgment}

Initial discussions with Thiem Van Chu, such as about the related work are greatly appreciated.

%
%
%
%
%
%
%
%
%
%
%
%
%
%
%
%
%
%
%
%
%
%
%
%
%
%
%
%
%
%
%
%
%
%
%
%
%
%
%
%
%
%
%
%
%
%
%
%
%
%
%
%
%
%
%
%
%
%
%
%
%
%
%
%
%
%
%
%
%
%
%
%
%
%
%
%
%
%
%
%
%
%
%
%
%
%
%
%
%
%
%
%
%
%
%
%
%
%
%
%
%
%
%
%
%
%
%
%
%
%
%
%
%
%
%
%
%
%
%
%
%
%
%
%
%
%
%
%
%
%
%
%
%
%
%
%
%
%
%
%
%
%
%
%
%
%
%
%
%
%
%
%
%
%
%
%
%
%
%
%
%
%
%
%
%
%
%
%
%
%
%
%
%
%
%
%
%
%
%
%
%
%
%

%
%
%
%
%
%
%
%
%
%
%
%
%
%
%
%
%
%
%
%

%
%
%
%
%
%
%
%
%
%
%
%
\balance
\bibliography{bibliography}

\end{document}
