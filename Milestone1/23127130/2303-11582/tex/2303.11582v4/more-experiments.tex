\section{Further Experimental Results}

\subsection{Unknown measurement variance}
\label{section:unknown-s2}

We consider the case where the measurement variance is unknown to the experimenter. 
We fix $K = 100$ and $n = 10,000$ in the Gamma-Gumbel experiment. The experimenter believes
the measurement variance for all arms is identically equal to $s^{2} = 1$, but the 
true measurement variance is $s^{2} = Y_{i}$ where $Y_{i} \sim \mathsf{Lognormal}(0, \varsigma^{2})$
for $\varsigma \in \{0.25, 0.5, 1.0\}$. The 50\% confidence intervals for $Y_{i}$ under these distributions are
\begin{align*}
    \varsigma = 0.25 &: 50\% \text{ CI } [0.84, 1.18] \\
    \varsigma = 0.50 &: 50\% \text{ CI }[0.75, 1.40] \\
    \varsigma = 1.00 &: 50\% \text{ CI }[0.50, 1.96]
\end{align*}
Intuitively, when $\varsigma = 1.0$, the actual measurement variance can easily range from 50\% to 196\% of the variance 
assumed by the experimenter. We observe in Figure \ref{fig:var_perturb} that despite large mismatch between
the experimenter's belief of the measurement variance and the true measurement variance, 
$\algo$ is still able to retain much of its performance benefits over other policies. This illustrates that the method
is rather robust to estimation errors of the measurement variance.

\begin{figure}[t]
     \centering
     \subfloat{\includegraphics[height=5.7cm]{./fig/reg_Gumbel_10000_100_1.0_Flat_var.png}} %
     \caption{\label{fig:var_perturb} Comparison of performance across
       variance perturbations.  Relative gains over the uniform allocation as measured
       by the Bayes simple regret for the finite batch problem with $K = 100$
       treatment arms, batch size $n = 10,000$ in the $\textsf{Gamma-Gumbel experiment}$.
       Despite drastic variance mis-measurement ($\varsigma = 1$), $\algo$ maintains performance
       benefits over other Bayesian policies (except Policy Gradient).}    
   \end{figure}