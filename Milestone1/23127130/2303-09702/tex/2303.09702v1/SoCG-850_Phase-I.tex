\section{Phase I: Finding the Farthest Edge Voronoi Diagram Restricted to the Polygon Boundary}
\label{section:Phase-I}

\changed{Based on Assumptions~\ref{assumptions}, the boundary of $P$ consists of chains with a single farthest edge, separated by isolated points (not vertices) that have two farthest edges. 
Our goal is to find these points.}
The first step of the algorithm is to find the farthest edge from each vertex of the polygon in linear time.  To do this, we extend the algorithm of Hershberger and Suri~\cite{hershberger1997matrix} that finds the farthest \emph{vertex} from each vertex.  Details are in
Appendix~\ref{appendix:Hershberger-Suri}.
The next step is to fill in the Voronoi diagram along the polygon edges.
For an edge $ab$ where
vertices $a$ and $b$ have the same farthest edge, i.e., $F(a) = F(b)$,  all points on the edge $ab$ have the same farthest edge, by
the Ordering Property. 
An edge $ab$ with 
$F(a) \ne F(b)$ is a \defn{transition edge}. 
We will find the farthest edge Voronoi diagram on one transition edge in linear time.  To handle \emph{all} the transition edges in linear time, we will show that for each transition edge $ab$ we can restrict our attention to a subpolygon $H(a,b)$ called the \defn{hourglass} of $ab$.
(Hourglasses have been used
previously for shortest paths, Voronoi diagrams, and geodesic centers~\cite{SPT_linear,GUIBAS1989126,chazelle1989visibility,aronov1993farthest,linear_time_geodesic}.)
In Appendix~\ref{appendix:hourglasses} we show that the hourglasses of all transition edges can be found in linear time and that the sum of their sizes is linear. 

\remove{
in Section~\ref{section:functions_to_capture} we show how to find the farthest edge Voronoi diagram on a transition edge inside its hourglass in time linear in the size of the hourglass. 
For the second step, we use a coarse cover (see Section~\ref{section:chord_oracle}) of the transition edge.
}


\remove{
\anna{Keep this here? Or later?}  Later.
As explained in the Introduction, there is a \reviewerchange{linear-time algorithm} to find the farthest \emph{vertex} Voronoi diagram restricted to the boundary of a polygon due to
Oh, Barba and Ahn~\cite[Section 3]{oh2020geodesic}.
Our approach is roughly similar to theirs (with farthest edges in place of farthest vertices), but there are some significant differences.  First of all, we will use 
the Voronoi diagram on the polygon boundary to easily find a coarse cover of the polygon, whereas previous papers did things in reverse---Ahn et al.~had a more complicated way of finding a coarse cover of the polygon
which Oh et al.~then used to find the boundary Voronoi diagram. 
Another difference is that our algorithm to find the Voronoi diagram on one transition edge is considerably simpler---see Section~\ref{section:functions_to_capture} for details.
}



In this section, we show how to construct the 
farthest edge Voronoi diagram along one polygon edge $ab$ 
in time linear in the size of the polygon.
We do not assume that the polygon is an hourglass.
For purposes of description, imagine $ab$ horizontal with $a$ at the left, and the polygon interior above $ab$.
We use the \defn{coarse cover} (Definition~\ref{defn:coarse-cover}) of the edge $ab$,
which can be found in linear time (Lubiw and Naredla~\cite{lubiw_et_al:LIPIcs.ESA.2021.65}). 
Elements of the coarse cover are triples $(I,f,e)$ where $I$ is a subinterval of $ab$ and $f(x) = d(x,e)$ for any $x \in I$. 
By resolving overlaps of coarse cover intervals $I$, we find
the upper envelope of the coarse cover functions $f$, which immediately gives the Voronoi diagram on $ab$. 
This is easy if we %
sort the endpoints of the intervals $I$, but 
we cannot afford to sort. Instead, we will insert the coarse cover elements %
one by one, maintaining a list $M$ of [pairwise internally] disjoint subintervals of $ab$ together with an associated distance function $f_M(x)$. 
\changed{An efficient insertion order depends on the fact that}
\remove{When we consider  $(I,f,e)$ we will insert into $M$ the initial subinterval of $I$ in which $f(x) > f_M(x)$. 
We must be able to quickly find the left endpoint of $I$ in $M$ and to ensure that one subinterval of $I$ is all we need to add. 
This 
requires}
elements of the coarse cover of edge $ab$ are associated with edges of the shortest path trees ${T}_a$ and ${T}_b$ (that consist of the shortest paths from $a$ and $b$, respectively, to all the edges of $P$). 
We will use the ordering of the trees as embedded in the plane.



Oh, Barba, Ahn~\cite{oh2020geodesic} gave a \reviewerchange{linear-time algorithm} to find the farthest \emph{vertex} Voronoi diagram on the boundary of $P$. 
\changed{The approach is similar, but they add coarse cover elements by iterating over the sites (the vertices in their case), which involves a complicated algorithm to sweep back and forth along $M$ maintaining a shortest path to the current vertex, and a tricky amortized analysis (see~\cite[Lemma 7]{oh2020geodesic}).
Our approach is simpler and more general.} 

\remove{
Their algorithm also works by 
adding coarse cover elements to obtain the upper envelope of the distance functions.
Their coarse cover is obtained differently, and 
their algorithm is quite 
complicated---they
iterate over the sites (the vertices, in their case),  adding all coarse cover elements associated with each site.  To find the correct insertion point on $ab$, 
they sweep back and forth
maintaining a shortest path from the current point on $ab$ to the current vertex.  Their argument that updating the path costs amortized linear time depends 
on properties of the coarse cover that are not made explicit (see Lemma 7 in their paper).
Our algorithm is much simpler, both in its description and its analysis, because we
use the relationship between our coarse cover and shortest path trees, which allows us to iterate over edges of the shortest path trees rather than iterating over the sites.
}

\subsection{Farthest Edge Voronoi Diagram on One Edge}
\label{sec:one-edge-Vor}

Lubiw and Naredla~\cite{lubiw2021visibility, lubiw_et_al:LIPIcs.ESA.2021.65} constructed 
a coarse cover
(see Definition~\ref{defn:coarse-cover})
of an edge $ab$ from the shortest path trees $T_a$ and $T_b$.  They first augment $T_a$ and $T_b$ with $0$-length edges so that the paths to every 
polygon edge $e$ end with  
a tree edge perpendicular to $e$. In particular, every polygon edge corresponds to a leaf in each tree.

Direct edges of $T_a$ and $T_b$ away from their roots. 
Each edge $uv$ of $T_a \setminus T_b$ with $u \ne a$
corresponds to an \defn{$a$-side} coarse cover element $(I,f,e)$ where $e$ corresponds to the farthest leaf of $T_a$ descended from $v$.
For example, in Figure~\ref{fig:example_2_e4}, see edge $a_3$ of $T_a$ and interval $I_{a_3}$.
There are symmetrically defined \defn{$b$-side} coarse cover elements.
Each edge $uv$ of $T_a \cap T_b$ with $u$ visible from $ab$ corresponds to a \defn{central triangle} coarse cover element $(I,f,e)$ where $e$ corresponds to the farthest leaf of $T_a$ descended from $v$.
For example, see edge $a_5=b_5$ and interval $I_{a_5}$.
Each polygon edge $e$ that has an interior point visible from $ab$ corresponds to a \defn{central trapezoid} coarse cover element $(I,f,e)$ where $I$ consists of the points on $ab$ whose shortest paths to $e$ arrive perpendicularly.
For example, see edge $e_4$ and interval $I_{e_4}$.


\remove{
\changed{We first describe results of
Lubiw and Naredla~\cite{lubiw2021visibility, lubiw_et_al:LIPIcs.ESA.2021.65} on constructing 
a coarse cover
(see Definition~\ref{defn:coarse-cover})
of an edge $ab$ from the shortest path trees $T_a$ and $T_b$.} 
Direct the edges of $T_a$ and $T_b$ from the roots to the leaves (which correspond to the edges of the polygon).

\anna{Idea: Remove definition of coarse cover.  Just explain enough to get Lemma 3.  Remove figure.  Instead, put Example 2 of the algorithm.}

For any node $v$ in $T_a$ define \defn{$\ell_a(v)$} to
be the maximum length of a directed path in $T_a$ from $v$ to a  leaf node %
representing a terminal point on some 
\changed{polygon edge},
and define 
\defn{$F_a(v)$} to be that \changed{polygon edge}.
Define  \defn{$\ell_b$} and \defn{$F_b$} similarly.
These values can be computed in linear time~\cite{lubiw2021visibility}. 


Define \defn{$p_a(u)$} and \defn{$p_b(u)$} to be the parents of node $u$ in $T_a$ and $T_b$, respectively.
As noted by 
Pollack et al.~\cite{pollack_sharir}, a vertex $u$ is visible from some point on $ab$
if and only if $p_a(u) \ne p_b(u)$. 
Furthermore, note that if $u$ is visible from some point on $ab$, then extending the edge from $u$ through $p_a(u)$ reaches a point
\defn{$x_a(u)$}
on $ab$ from which $u$ is visible.
Similarly, extending the edge from $u$ through $p_b(u)$ reaches a point
\defn{$x_b(u)$}
on $ab$ from which $u$ is visible.

In order to avoid special cases,~\cite{lubiw2021visibility} assumes that  for any edge $e$ the last segment of the path $\pi(a,e)$ is orthogonal to $e$, and they add $0$-length edges to make this true. 
The extension of such a $0$-length edge is orthogonal to $e$. 
Note that, as a special case, there is a $0$-length edge added to $T_a$ at the root $a$ to account for the path (of length 0) from $a$ to the polygon edge incident to $a$.  The $0$-length edges ensure that every polygon edge is associated with a unique leaf of $T_a$.
Note that if $\pi(a,e)$ and $\pi(b,e)$ terminate at the same endpoint of $e$, then the added 0-length segment is common to both trees.

The following defines a coarse cover $\cal T$ that can be found in linear time~\cite{lubiw2021visibility}.
See Figure~\ref{fig:coarse-cover}.

\begin{figure}[htb]
    \centering
    \includegraphics[width=\textwidth]{RS_figures/coarse_cover_elements_edge-cropped.pdf}
    \caption{
    Coarse cover elements: 
    (a) an $a$-side triangle element associated with edge $uv$;
(b) a central triangle element associated with edge $uv$;
(c) 
a central triangle element associated with $0$-length edge $uv$;
(d) a central trapezoid element %
associated with polygon edge $e$.
\anna{Drop $I_2$}
    }
    \label{fig:coarse-cover}
\end{figure}




\begin{enumerate}
\squeezelist

\item ({\bf $a$-side triangle element})
\label{list:coarse_1} 
For each directed edge $(u,v)$ in ${T}_a$ where %
\changed{$u \ne a$}, and $u$ and $v$ are both visible from $ab$, add 
$(I,f,e)$ where $I = [x_a(u), x_a(v)]$, $e = F_a(v)$ and $f(x) = d_2(x,u) + |uv| + \ell_a(v)$. 
\remove{
\item \label{list:coarse_1} For each directed edge $(u,v)$ in ${T}_a$ where \changed{$u \ne a$}, and $u$ and $v$ are both visible from $ab$, there is an associated 
{\bf $a$-side triangle} that has apex $u$ and base $I = [x_a(u), x_a(v)] \subseteq ab$. 
The associated coarse cover element is $(I,f,e)$ where $e = F_a(v)$ and $f(x) = d_2(x,u) + |uv| + \ell_a(v)$.
}
Define {\bf $b$-side triangle elements} 
symmetrically.

\item ({\bf central triangle element})
\label{list:coarse_2}
For each edge $(u,v)$ common to ${T}_a$ and ${T}_b$ with $u$ visible from $ab$, 
add $(I,f,e)$ where $I = [x_a(u), x_b(u)]$, $e = F_a(v)= F_b(v)$ and $f(x) = d_2(x,u) + |uv| + \ell_a(v)$. 

\remove{
\item \label{list:coarse_2} For each edge $(u,v)$ that is common to ${T}_a$ and ${T}_b$ where $u$ is visible from $ab$ and $v$ is not 
(i.e., $u = p_a(v) = p_b(v)$) there is an associated 
{\bf central triangle} that has apex $u$ and 
base $I = [x_a(u), x_b(u)] \subseteq ab$.
The associated coarse cover element is $(I,f,e)$ where $e = F_a(v)= F_b(v)$ and $f(x) = d_2(x,u) + |uv| + \ell_a(v)$.
}

\item ({\bf central trapezoid element})
\label{list:coarse_3}
For each
polygon edge $e$
such that the terminal points \defn{$t(a,e)$} of $\pi(a,e)$ and \defn{$t(b,e)$} of $\pi(b,e)$ are distinct,
add $(I,f,e)$ where 
$I$ is bounded by the 
two lines perpendicular to $e$ emanating from $t(a,e)$ and $t(b,e)$,
$e$ is the polygon edge and $f(x) = d_2(x,{\bar e})$ where $\bar e$ is the line through $e$. 

\remove{
\item \label{list:coarse_3} For each
polygon edge $e$
such that the terminal points \defn{$t(a,e)$} of $\pi(a,e)$ and \defn{$t(b,e)$} of $\pi(b,e)$ are distinct, 
there is a {\bf central trapezoid}
with base $I \subseteq ab$ bounded by the 
two lines perpendicular to $e$ emanating from $t(a,e)$ and $t(b,e)$---these lines are the extensions of the (possibly $0$-length) last edges of the paths.
The associated coarse cover element is $(I,f,e)$ where $e$ is the polygon edge and $f(x) = d_2(x,{\bar e})$ where $\bar e$ is the line through $e$.
}
\end{enumerate}



Note that coarse cover elements of Type~\ref{list:coarse_1} and Type~\ref{list:coarse_2} 
are associated with edges $(u,v)$ of $T_a$ and $T_b$ with at least one node visible from $ab$, and
elements of Type~\ref{list:coarse_3} %
are %
associated with %
polygon edges, 
and thus 
with leaves of $T_a$.

\remove{
\begin{obs}
\label{obs:consecutive-coarse-cover}
If $uv$ and $vw$ are edges of $T_a$ that have associated coarse cover elements $(I_1, f,e)$ and $(I_2,f',e')$, then
the right endpoint of $I_1$ is $x_a(v)$ and the left endpoint of $I_2$ is $x_a(v)$, i.e., $I$ and $I'$ appear in that order along $ab$ and intersect in a single point.  
A similar property holds for $T_b$.
\end{obs}

\begin{lem} 
\label{lem:coarse-cover-path}
Suppose edge $uv$ of $T_a$ has an associated coarse cover element 
$(I,f,F_a(v))$.
Then:

\begin{enumerate}
\item 
On the path $\pi(a,u)$ all edges except the first one have associated coarse cover elements.

\item On the path $\pi(v,F_a(v))$ let $x$ be the last vertex visible from $ab$.  All edges on $\pi(v,x)$ have associated coarse cover elements for the polygon edge $F_a(v)$.  Furthermore, if $x$ is a leaf then there is a central trapezoid associated with $F_a(v)$, and otherwise there is an edge $xy$ in $\pi(v,F_a(v))$ and it is associated with a central triangle for $F_a(v)$.
\end{enumerate}
A similar property holds for $T_b$.
\end{lem}
}

} %


\begin{lem} 
(proved in Appendix~\ref{appendix:one-edge-Vor})
\label{lem:consecutive-coarse-cover}
\label{cor:consecutive-coarse-cover}
For any edge $e$ of $P$, let $C(e)$ be the set of coarse cover elements $(I,f,e)$ for $e$.
If $C(e)$ is nonempty, then its elements correspond to a (possibly empty) path in $T_a$ directed towards a leaf, followed by a central triangle or trapezoid, followed by a (possibly empty) path in $T_b$ directed towards the root.  Furthermore, the corresponding intervals on $ab$ appear in order, are [internally] disjoint, and their union is an interval.
\end{lem}





\changed{We next construct a single tree $T$ whose edges correspond to coarse cover elements of $ab$.
Then we incrementally construct the farthest edge Voronoi diagram on $ab$ by adding coarse cover elements in a depth first search (DFS) order of $T$.
}

\remove{
\changed{This section contains the algorithm to find the farthest edge Voronoi diagram on polygon edge $ab$.  We first construct a tree $T$ whose edges correspond to coarse cover elements of $ab$.  Then we incrementally construct the farthest edge Voronoi diagram on $ab$ by adding coarse cover elements in a depth first search (DFS) order of $T$.}
}

\subparagraph*{Constructing 
tree $T$.}
Starting with $T_a$,
attach an edge for each central trapezoid element to the associated leaf vertex of $T_a$; add the path of $b$-side triangle elements for each polygon edge $e$ after the central triangle or trapezoid for $e$; and contract original edges of $T_a$ that are not associated with coarse cover elements.
See Figure~\ref{fig:example_2_e4}(right).
We give more detail of these steps in Appendix~\ref{appendix:one-edge-Vor}.
\remove{
\begin{enumerate}
\squeezelist
\item For each central trapezoid, say associated with polygon edge $e$, there is a leaf $l$
of $T$ corresponding to $e$. Attach a new edge in $T$ descending from $l$ and associate the central trapezoid with it.


\item 
For each polygon edge $e$ \changed{that has $b$-side triangles associated with it, those triangles} 
correspond to a  path $\pi$ in $T_b$, that is directed in leaf-to-root order, see Corollary~\ref{cor:consecutive-coarse-cover}.  The tree $T$ currently has an edge, say $g$, associated with the central triangle/trapezoid for $e$.  Attach the path $\pi$ at end of the edge $g$.

\item 
Finally, we delete and contract some of the original edges of $T_a$.
Delete from $T_a$ any 
original directed edge $uv$ such that $u$ is not visible from $ab$ (these edges do not have associated coarse cover elements).  By Lemma~\ref{lem:coarse-cover-path}, the result is still a tree.  Next, contract \changed{the original} edges of $T_a$ incident to $a$ (these edges also do not have associated coarse cover elements).  \changed{The edges of $T_a$ that remain in $T$ are the ones that 
have associated coarse cover elements}, i.e., all $a$-side  and central triangles.


\end{enumerate}
}
The resulting tree $T$ can be constructed in linear time and its edges are in one-to-one correspondence with the coarse cover elements. 

\begin{obs}
\label{obs:consecutive-coarse-cover-T}
If $uv$ and $vw$ are edges of $T$, then the
corresponding coarse cover intervals
$I_1$ and $I_2$
appear in that order along $ab$ and intersect in a single point. 
\end{obs}

\remove{
\begin{figure}
\centering
\includegraphics[width=0.4\textwidth]{RS_figures/example_2_for_DFS_algorithm_3_original-cropped.pdf}
\caption{
The shortest path trees rooted at $a$ and $b$ are shown in red and blue respectively.
We have constructed the diagram so that the farthest edge from points on $ab$ is either $e_4$ or $e_7$.
Tree edges not on the paths to these two edges will be ignored.
The crossover point $t$ on $ab$ is at equal geodesic distance from $e_4$ and $e_7$. Our Voronoi diagram algorithm will assign the interval $at$ to $e_4$ and $tb$ to $e_7$.
}
\label{fig:example_2}
\end{figure}
}

\begin{figure}
    \centering
    \begin{subfigure}[t]{0.43\textwidth}
        \centering
        \includegraphics[width=\textwidth]{RS_figures/example_2_for_DFS_algorithm_2_coarse_cover_for_e4_1-cropped.pdf}
    \end{subfigure}%
    \begin{subfigure}[t]{0.43\textwidth}
        \centering
        \includegraphics[width=\textwidth]{RS_figures/example_2_for_DFS_algorithm_2_coarse_cover_for_e7_3-cropped.pdf}
    \end{subfigure}
    \begin{subfigure}[t]{0.13\textwidth}
        \centering
        \includegraphics[width=\textwidth]{RS_figures/Ta_and_T_example_2_1_JUST_T_rotated_1-cropped.pdf}
    \end{subfigure}
    \caption{Coarse cover elements corresponding to some (not all) edges of $T_a$ (red) and $T_b$ (blue):  (left) coarse cover elements for $e_4$; 
(middle) coarse cover elements for $e_7$;
(right) the corresponding part of tree $T$.
\changed{When {\tt Insert} compares $I_{a_4}$ to the new interval $I_{a_6}$ it finds cross-over point $t$.}
}
\label{fig:example_2_e4}
\end{figure}



\subparagraph*{DFS Algorithm for the Voronoi Diagram.}
We add the coarse cover elements following a DFS of $T$ with children of a node in clockwise order. 
We maintain a list $M$ of interior disjoint subintervals of $ab$ whose union  is an interval starting at $a$. Each subinterval in $M$ records the coarse cover element it came from. 
Define $f_M$ to be the distance function determined by the intervals of $M$.
\changed{Initially, $M$ is the single point $a$, and $f_M$ is $- \infty$.}
At the end $M$ will be the upper envelope of the coarse cover functions (though this property is not guaranteed throughout). To handle edge $uv$ of $T$ with associated coarse cover element $(I,f,e)$, we compare $f$ to $f_M$ beginning at the left endpoint of $I$. We maintain a pointer $p_u$ that gives an interval of $M$ containing this endpoint. 
The recursive routine 
{\tt Insert}$(u,p_u)$ inserts into $M$ the portions of coarse cover elements that are associated with $u$'s subtree and that define the upper envelope.  
At the top level, we call {\tt Insert}$(a,p_a)$, where $p_a$ points to $a$.


\remove{
We add the coarse cover elements one by one, ordered  according to
a depth first search of the tree $T$ in which we explore children of a vertex in clockwise order (i.e., with the most counterclockwise child first). 
We maintain a list $M$ of interior disjoint subintervals of $ab$ whose union  is an interval starting at $a$. With each subinterval in $M$ we record the coarse cover element it came from.  We define $f_M$ to be the distance function determined by the intervals of $M$.  
In the end $M$ will be the farthest edge Voronoi diagram on $ab$. 




During the depth first search of $T$, when we traverse an edge $uv$ with an associated coarse cover element $(I,f,e)$, we ensure that any part of this element that defines the final upper envelope is inserted into $M$.
(Note the subtlety that 
$M$ need not be the upper envelope of the coarse cover elements examined so far.)

In fact, the algorithm will insert only one subinterval of $I$ beginning at its left endpoint.
If all of $I$ is inserted, then the algorithm recursively continues the depth first search on vertex $v$, and otherwise the whole subtree rooted at $v$ is abandoned. 
There are two issues: correctness, which is addressed later;
and efficiency, which we discuss 
here.
In interval $I =[l,r]$ 
we must compare the distance function $f$ to the distance function $f_M$ determined by the intervals of $M$. We begin at $l$, the left endpoint of $I$.
To compare $f(l)$ with $f_M(l)$, we must locate $l$ in $M$ at unit cost. 
If $u$ is the root of $T$, then $l=a$, and we begin at the start of $M$.  Otherwise, the edge of $T$ from the parent of $u$ to $u$ has an associated coarse cover element with an interval $I'$.  
By Observation~\ref{obs:consecutive-coarse-cover-T}, the right endpoint of $I'$ is $l$.
Thus, we will store with vertex $u$ a pointer $p_u$ to the interval of $M$ that contains the right endpoint of $I'$.
This allows us to access $f_M(l)$ at unit cost.



We define a recursive routine, {\tt Insert}$(u,p_u)$ where $u$ is a vertex of $T$ and $p_u$ is a pointer into the list $M$ as described above. 
The goal is to insert into $M$ the portions of coarse cover elements that are associated with the subtree of $T$ rooted at $u$ and that determine the final upper envelope.  
Initially $M$ has a single endpoint $a$ and pointer $p_a$ points to it. At the top level we call {\tt Insert}$(a,p_a)$.
}

\remove{
\noindent
\ind{1}{\tt Insert}$(u, p_u)$
\ \ \ $u$ is a node of $T$ and $p_u$ is a pointer to an interval of $M$\\
\ind{1}1: \ind{1}{\bf for} each child $v$ of $u$ in clockwise order {\bf do}\\
\ind{1}2:\ind{2}$(I,f,e) :=$
the coarse cover element associated with the edge $uv$ of $T$.\\
\ind{1}3:\ind{2} $l :=$ left endpoint of $I$; $r :=$ right endpoint of $I$\\
\ind{4} {\bf Invariant:} $p_u$ points to an interval of $M$ that contains $l$.\\
\ind{1}4:\ind{2} {\bf if} $f(l^+) > f_M(l^+)$ where $l^+$ is just to the right of $l$  {\bf then}\\
\ind{1}5:\ind{3} replace intervals of $M$ starting at $p_u$ with a subinterval of $I$ ending at\\
\ind{6}the ``cross-over'' point $t < r$ where $f_M$ starts to dominate $f$, or at $r$\\
\ind{1}6:\ind{3} {\bf if} $f$ dominates until $r$ and $v$ is not a leaf of $T$ {\bf then}\\ \ind{1}6:\ind{4} {\bf call} {\tt Insert}$(v,p_v)$, where $p_v$ is a pointer to interval $I$ in $M$
}

\newchanged{
\medskip
\noindent
\ind{1} {\tt Insert}$(u, p_u)$
\ \ \ \# $u$ is a node of $T$ and $p_u$ is a pointer to an interval of $M$

\smallskip
\noindent
\ind{2} {\bf for} each child $v$ of $u$ in clockwise order {\bf do}\\
\ind{3} $(I,f,e) :=$
the coarse cover element associated with the edge $uv$ of $T$\\
\ind{3} $l :=$ left endpoint of $I$; \ $r :=$ right endpoint of $I$\\
\ind{3} {\bf Invariant:} $p_u$ points to an interval of $M$ that contains $l$\\
\ind{3} {\bf if} $f(l^+) > f_M(l^+)$ where $l^+$ is just to the right of $l$  {\bf then}\\
\ind{4} replace intervals of $M$ starting at $p_u$ with a subinterval of $I$ ending at\\
\ind{5}the ``cross-over'' point $t < r$ where $f_M$ starts to dominate $f$, or at $r$\\
\ind{4} {\bf if} $f$ dominates until $r$ and $v$ is not a leaf of $T$ {\bf then}\\ \ind{5} {\bf call} {\tt Insert}$(v,p_v)$, where $p_v$ is a pointer to interval $I$ in $M$
}

\remove{
\smallskip
\noindent
{\tt Insert}$(u, p_u)$
\ \ \ $u$ is a node of $T$ and $p_u$ is a pointer to an interval of $M$ \attention{ Formatting.}

\smallskip

For each child $v$ of $u$ in clockwise order: 
\begin{enumerate}
\squeezelist    

\item 
$(I,f,e) :=$
the coarse cover element associated with the edge $uv$ of $T$.  Suppose $I=[l,r]$. Our invariant is that $p_u$ points to an interval of $M$ that contains $l$.

\item if $f(l^+) > f_M(l^+)$ where $l^+$ is just to the right of $l$  then 
\begin{itemize}
\item \label{algorithm_step:premature_replace} replace the intervals of $M$ starting at $p_u$ with a subinterval of $I$ that ends at the ``cross-over'' point $t$ where $f_M$ starts to dominate $f$, or at $r$, the right endpoint of $I$, whichever comes first

\item if $f$ dominates until $r$ and $v$ is not a leaf of $T$, then call {\tt Insert}$(v,p_v)$, where $p_v$ is a pointer to interval $I$ in $M$

\end{itemize}

\end{enumerate}
}




\medskip
\noindent
{\bf Runtime:}
Each edge of ${T}$ is handled once, and causes at most one new interval to be inserted into $M$, so the total number of endpoints inserted into $M$ is $O(n)$.
We can access $f_M(l^+)$ in constant time using the pointer $p_u$. 
Then 
the endpoints of intervals of $M$ that we traverse as we do the insertion vanish from $M$. 
Thus the runtime is $O(n)$.



\smallskip
\noindent
{\bf Correctness:}
\changed{The following lemma implies that the final 
$M$ is the upper envelope of the coarse cover functions.}
\remove{
By definition, the upper envelope of the distance functions of the coarse cover elements of $ab$
is the geodesic edge radius. 
We will prove that our algorithm only discards cover elements (partially or fully) in intervals where their function value is exceeded by some other element.
The discarded parts are thus not relevant for finding the upper envelope.
At any point during the execution of the algorithm, $M$ is a list of coarse cover elements with disjoint
intervals  on $ab$ whose union is an interval starting at $a$.
All the coarse cover elements that remain in the end are part of $M$ and hence $M$ must be the upper envelope.
The following lemma completes the proof of correctness:
} %
\begin{lem}
\label{lem:DFS-correct}
The algorithm only discards pieces of coarse cover elements that do not form part of the final upper envelope.
\end{lem}



\begin{proof}
We examine the behaviour of the algorithm for edge $uv$ of $T$ with associated coarse cover element $(I,f,e)$, where $I=[l,r]$.   
We insert the subinterval $[l,t]$ into $M$ (or no subinterval). 
Because  $f(x) \ge f_M(x)$ for $x \in [l,t]$, any subintervals of $M$ that are removed due to the insertion do not determine the upper envelope, so their removal is correct. 

If we insert all of interval $I$ into $M$ and recursively call Insert$(v,p_v)$, then this is correct by induction.  
So suppose we insert a proper subinterval of $I$ or none of $I$.  
We must prove that no later part of $I$, and no element of the coarse cover associated with edges of the subtree rooted at $v$ determines the upper envelope. 
Let $t^+$ be a point just to the right of $t$ (or just to the right of $l$ if we insert no part of $I$).  Then $f_M(t^+) > f(t^+)$.
Number the polygon edges $e_1, e_2, \ldots, e_m$ clockwise from $a$ to $b$.  
Suppose that $e = e_i$, so $f(x) = d(x,e_i)$ for $x \in I$, in particular, $f(t^+) = d(t^+,e_i)$.
Suppose that 
$f_M(t^+) = d(t^+,e_k)$. 
Then $d(t^+,e_k) > d(t^+, e_i)$.  

Now consider the edges of $T$ descended from $v$ plus the edge $uv$.  Consider the corresponding coarse cover elements, $C_v$, and let $e_j$ be 
any polygon edge associated with any element in $C_v$. 
Note that the intervals on $ab$ associated with coarse cover elements of $C_v$ lie to the right of $r$, except for $I$ associated with $uv$.
We will prove that for any point $x \in ab$ to the right of $t^+$, $d(x,e_k) > d(x,e_j)$, which implies that none of the coarse cover elements in $C_v$ determines the upper envelope, nor does any part of $I$ to the right of $t$. Thus the algorithm is correct to discard them.

\changed{We first prove the result for $x = t^+$.}
If $uv$ corresponds to a central triangle/trapezoid for $e_i$
or a $b$-side triangle, then $T$ has a single path descending from $v$, all of whose edges are associated with $e_i$, i.e., $j=i$.  Otherwise, 
\changed{by the definition of
an $a$-side coarse cover element, $e_i$ corresponds to the farthest leaf of $T_a$ descended from $v$, which implies that  
}
$d(x,e_i) \ge d(x,e_j)$ for all $x \in I$, and in particular for $x=t^+$. 
Thus, in either case we have 
$d(t^+,e_k) > d(t^+, e_i) \ge d(t^+,e_j)$.

We next claim that $k < j$.  
The current $f_M$ values arise from tree edges already processed. 
These consist of: (1) edges on the path from $a$ to $u$; and (2)  edges of $T$ counterclockwise from this path.  
Edges on the path from $a$ to $u$ have coarse cover intervals on $ab$ to the left of $l$, by Observation~\ref{obs:consecutive-coarse-cover-T}.
Thus type (1) edges do not determine $f_M(t^+)$. 
By the depth-first-search order, type (2) edges have coarse cover elements corresponding to polygon edges counterclockwise from $e_j$.  Thus $k<j$.


To complete the proof of Lemma~\ref{lem:DFS-correct},
consider any point $x \in ab$ to the right of $t^+$. The clockwise ordering around the polygon boundary is $x,t^+, e_k, e_j$, so by
Lemma~\ref{lem:new_order}
and the fact that  $d(t^+,e_k) > d(t^+,e_j)$, we get $d(x,e_k) >  d(x,e_j)$, as required.  
\end{proof}

