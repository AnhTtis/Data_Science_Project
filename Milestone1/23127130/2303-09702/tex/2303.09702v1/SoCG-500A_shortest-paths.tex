\subsection{Details on Shortest Paths To/From Edges}
\label{appendix:shortest-paths}

In this section we give \reviewerchange{linear-time algorithm}s to find  shortest paths from a  given  point to all edges of the polygon, and to find shortest paths from a  given  edge to all  vertices of the polygon. In fact, in both cases, we will augment to a \defn{shortest path map} that divides the polygon into regions (triangles and trapezoids) in which the shortest paths are combinatorially the same.


\medskip
\noindent{\bf Shortest Paths from a Point to all Edges and Vertices.}
For a point $p$ in polygon $P$, define \defn{$T_p$} be the \defn{shortest path tree} that consists of shortest paths from $p$ to all the edges and vertices of the polygon.  
In some situations we 
will only care about the shortest paths to edges, but we will still use the notation $T_p$ and just clarify what we mean.

\begin{lem}\label{SPT_lemma}
There is a \reviewerchange{linear-time algorithm} to find, given a point $p$ in a polygon, the shortest path tree $T_p$
and its augmentation to a shortest path map.
\end{lem}

\begin{proof}
The idea is simple.  Construct the shortest path tree from $p$ to all vertices and augment to the shortest path map using the algorithm by Guibas et al.~\cite{SPT_linear}. 
Regions of the shortest path map are triangles. 
Check each triangle in $O(1)$ time to see if it contains the last segment of a shortest path from $p$ to an edge. 
For further details see~\cite[Section 4.1.2]{lubiw_et_al:LIPIcs.ESA.2021.65, lubiw2021visibility} , which solves the more general case of shortest paths to a set of chords in a polygon when no two chords nest.
Note that these algorithms assume $p$ is on the boundary of the polygon, but we can handle an interior point $p$ by first cutting the polygon at a chord through $p$ in linear time and then finding shortest paths on each side of the chord.
\end{proof}




\medskip
\noindent{\bf Shortest Paths from an Edge to all Vertices.}
For an edge $e= ab$ of polygon $P$, define \defn{$T(e)$} to be the forest of shortest paths from $e$ to all vertices of the polygon.  %
We will use the shortest path trees $T_a$ and $T_b$ at the endpoints of $e$ to construct $T(e)$.

A vertex $v$ is \defn{visible} from $e$ if there is a line segment $xv$ inside $P$ for some point $x \in e$, and $v$ is \defn{orthogonally visible} from $e$ if $xv$ can be orthogonal to $e$.
Pollack et al.~\cite{pollack_sharir} note that vertex
$v$ is visible from $e$ iff $v$ has different parents $p_a(v)$ and $p_b(v)$ in $T_a$ and $T_b$.
Furthermore, if $p_a(v) \ne p_b(v)$ then $v$ is visible from the interval $[x_a(v),x_b(v)]$ in $e$ where $x_a(v)$ and $x_b(v)$ are the intersections of $e$ with the lines from $v$ to  $p_a(v)$ and $p_b(v)$ respectively. 






\begin{figure}
\begin{subfigure}{\textwidth}
  \centering
  \includegraphics[width=\textwidth]{RS_figures/separate_shortest_path_trees_2-cropped.pdf}
%  \caption{Shortest Path Trees from the endpoints of edge $e=ab$}
  \label{fig:sfig1}
\end{subfigure}%
\newline
\begin{subfigure}{\textwidth}
  \centering
  \includegraphics[width=.8\textwidth]{RS_figures/edge_spm_1-cropped.pdf}
%  \caption{The resulting shortest path map for the edge $e$}
\label{fig:sfig2}
\end{subfigure}
\caption{The shortest path trees from the endpoints of an edge can be used to construct its shortest path forest and map.
The color of 
an edge
indicates the shortest path tree it originates from. 
Green edges 
indicate orthogonal visibility from $e$.}
\label{fig:setvtree}
\end{figure}

\begin{lem}\label{ShortestPathForest_lemma}
There is a \reviewerchange{linear-time algorithm} to find, given an edge $e$ of a polygon, the forest $T(e)$ that consists of shortest paths from $e$ to all the vertices of the polygon, 
and to augment this to the shortest path map.
\end{lem}
\begin{proof}
See Figure~\ref{fig:setvtree}.
To construct $T(e)$,  we define the parent of each vertex of $P$. If $p_a(v) = p_b(v)$ then $v$ has the same parent in $T(e)$.  Otherwise, $v$ is visible from $e$.  If the angles $\angle v x_a(v) b$ and $\angle v x_b(v) a$ are both $\le \pi /2$ then $v$ is orthogonally visible from $e$, and we define the parent of $v$ to be the foot of the perpendicular from $v$ to $e$.   
And if one of the angles is obtuse, then the parent of $v$ in $T(e)$ is whichever of $p_a(v)$ or $p_b(v)$ that leads to the obtuse angle. 


We now have the shortest path forest $T(e)$.
To augment to the shortest path map, we first 
construct the vertex shortest path maps for subtrees rooted at $a$, $b$, and all the orthogonally visible vertices.
This %
takes linear time.
Finally, we can extend the perpendiculars
from orthogonally visible vertices (plus the endpoints of $e$) until they intersect $\partial P$.
\changed{This splits
the polygon into} trapezoids and triangles giving the required shortest path map.
\changed{The runtime is linear.}
\end{proof}









Note that we can easily modify the algorithm in Lemma~\ref{ShortestPathForest_lemma} to construct the shortest path forest from any %
chord of a given simple polygon in linear time.
