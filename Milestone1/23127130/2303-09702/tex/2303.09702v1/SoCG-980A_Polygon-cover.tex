\section{Extra Material for Section~\ref{section:PhaseII}, Phase II
} 


\subsection{Finding a Coarse Cover of the Polygon by Triangles} \label{appendix:coarse-cover}


This section contains the
first step of Phase II, which is 
to construct in linear time a \defn{coarse cover} 
of the polygon
as specified in Definition~\ref{defn:coarse-cover}.

Let $X$ be the set of vertices of the farthest edge Voronoi diagram on the boundary of $P$.  These are points on $\partial P$ that each have two farthest edges. The points of $X$ partition $\partial P$ into \defn{chains $C(e)$}, where $C(e)$ consists of the points on $\partial P$ whose farthest edge is $e$. We begin by expanding each chain $C(e)$  to a subpolygon
$Y(e)$
that contains the Voronoi region of the edge $e$.
After that, we will partition each polygon $Y(e)$ into triangles to obtain the coarse cover.


Suppose the chain $C(e)$ goes clockwise from $p(e) \in X$ to $q(e) \in X$.  The \defn{edge funnel} \defn{$Y(e)$} is the polygon  bounded by the chain $C(e)$, the \defn{walls} $\pi(p(e),e)$ and  $\pi(q(e),e)$, and the \defn{base} $b(e)$, which is the part
of $e$ between the terminal points $t(p(e),e)$ and $t(q(e),e)$.
See Figure~\ref{fig:flower}.
The \defn{size} of the edge funnel $Y(e)$ is its number of  vertices.
By Property~\ref{prop:path-to-same-edge}, the walls of an edge funnel may merge but never cross, so each edge funnel is a weakly simple polygon.

Edge funnels are an extension of the well-studied vertex funnels that are used for computing shortest paths (see~\cite{lee1984euclidean,SPT_linear}), and that were used by Ahn et al.~\cite{linear_time_geodesic} to compute the geodesic vertex center.
To build their coarse cover of the polygon, Ahn et al.~needed hourglasses as well as [vertex] funnels, so their method was more complicated. 
By contrast, our coarse cover is constructed from edge funnels alone because we did extra work ahead of time using hourglasses to compute the farthest edge Voronoi diagram restricted to the polygon boundary. 




\begin{lem}
\label{lem:voronoi_containment}
For any point $p \in P$, if $e$ is a farthest edge from $p$, then $p \in Y(e)$.
\end{lem}
\begin{proof}
Consider the shortest path $\pi(p,e)$ from the 
point $p$ to the edge $e$
and extend the first segment of the path backwards 
until it intersects the boundary $\partial P$ at point $p'$.
Since the result is a locally shortest path, it must be the shortest path from $p'$ to $e$.
Thus the distance from $p'$ to $e$ is $|p'p| + d(p,e)$.
We now show that $e$ is a farthest edge from $p'$.
Consider any other edge $e'$. We have  $d(p,e') \leq d(p,e)$.
Then
$d(p',e') \leq |p'p|  + d(p,e') \leq |p'p| + d(p,e)  = d(p',e)$.
Thus $e$ is a farthest edge from $p'$ so  $p'$ 
lies on the chain $C(e)$.
By Property~\ref{prop:path-to-same-edge} of Lemma~\ref{lem:ordering-properties},
$\pi(p',e)$ 
does not cross the walls of the edge funnel $Y(e)$, so it lies inside $Y(e)$.
Therefore, the point $p$ lies in $Y(e)$ since $p$ is a point on $\pi(p',e)$.
\end{proof}


\begin{lem}
\label{lemma:flowers_total_linear}
The set of  edge funnels $Y(e)$ corresponding to all the edges $e$ of the polygon can be constructed in $O(n)$ time.  The sum of all their sizes is $O(n)$.
\end{lem}
\begin{proof}
The farthest edge Voronoi diagram on $\partial P$ gives us the chains $C(e)$, so we only need to find the walls of the edge funnels, which are the shortest paths from the endpoints of $C(e)$ to $e$.
Equivalently, we must find,
for each Voronoi vertex $p \in X$, the shortest paths to $p$'s farthest edges. Note that there are $O(n)$ points in $X$, and 
by Lemma~\ref{lem:1D-Voronoi-edges}, 
each $p \in X$ has two farthest edges.

Recall that by Lemma~\ref{lem:constant_splits} there is a \reviewerchange{linear-time algorithm} to find a set of five farthest edge separators such that for every point $p \in \partial P$, one of the separators has $p$ to its 
right  
and, by definition of a farthest edge separator, has all farthest edges of $p$ to the 
left. . 
It therefore suffices to focus on one of these farthest
edge separators $\gamma = \pi(a,b)$, 
and give a \reviewerchange{linear-time algorithm} to find the shortest path from each point $p \in X$ that lies to the right of $\gamma$ to each of $p$'s farthest edges. 
By Lemma~\ref{lem:separator-paths}, each such path $\pi(p,F(p))$ is contained (except for one edge) in the shortest path trees $T(a)$ and $T(b)$. By Lemma~\ref{lemma:walls_can_be_constructed_fast},  
after a linear time preprocessing of the trees $T(a)$ and $T(b)$, each path $\pi(p,F(p))$ can be found in time proportional to its
number of edges, which we denote by $| \pi(p,F(p)) |$. 
Finally, we note that 
the two walls of one edge funnel may share edges, but we claim that walls of different edge funnels do not share edges if they originate from points in $X$ to the right of $\gamma$. 
\newchanged{Consider $p, q \in X$ and the paths $\pi(p,F(p))$, $\pi(q,F(q))$ with $F(p) \ne F(q)$. By Claim~\ref{claim:across-separator}, the paths cross, and then Property~\ref{prop:no-shared-chord} implies that the paths do not share any edges (chords).
}


We therefore have 
$\sum | \pi(p,F(p)) | \in O(|T(a)| + |T(b)| + n)$, 
where the last term accounts for the one edge of each path that is not in the trees. Thus the total run time to find all the shortest paths is $O(n)$.
\end{proof}



\subparagraph*{Defining the coarse cover $\cal T$ of polygon $P$.}
The idea is to partition each funnel $Y(e)$ into triangles in time linear in $O(|Y(e)|)$, and then take the union over all funnels.
We first use Lemma~\ref{ShortestPathForest_lemma} to   partition  $Y(e)$ in linear time into its \emph{shortest path map} from its base edge $b(e)$.  Recall that
the shortest path map 
partitions $Y(e)$ into regions such that shortest paths to $e$  from points in the same region are combinatorially the same. 
In addition,
if a region of the shortest path map contains any vertex $v$ whose shortest path to $e$ splits the region, then we subdivide the region at the path. All these subdivisions can be found in linear time, and the resulting subdivided regions $T$  
are either triangles or trapezoids; see Figure~\ref{fig:coarse-cover}.

Next we define distance functions on the triangles and trapezoids.
If $T$ is a triangle, then the shortest path to $e$ from any point $p \in T$ goes through an apex $v$ of the triangle, and the distance from $p$ to $e$ is $f_T(p) := d_2(p,v) + \kappa$ where $d_2$ is Euclidean distance (ignoring $P$) and  $\kappa$ is $d(v,e)$ which is independent of $p$.
If $T$ is a trapezoid, then the shortest path to $e$ from any point $p$ of $T$ is a straight line segment meeting $e$ at right angles, and the distance from  $p $ to $e$ is $f_T(p) := d_2(p,{\bar e})$, where ${\bar e}$ is the line through $e$ and $d_2$ is Euclidean distance (ignoring $P$). 
For the convenience of not having to say ``triangles and trapezoids,'' we will further partition each trapezoid into two triangles, each inheriting a distance function of the form $d_2(p,{\bar e})$.

 
Define \defn{${\cal T}(e)$} to contain
the triple $(T, f_T, e)$ for each triangle $T$ in the partition of $Y(e)$.
Along with the triples we store the following:


\begin{enumerate}
\squeezelist
\item Each triangle $T$ 
is given by its three sides: one side is a subsegment of an edge and the other two are chords (recall that a chord may include, or be, an edge of $P$).  A chord is given by its two endpoints and the vertex/edge containing each endpoint.  
\item Furthermore, we store a list of chords used as  sides of triangles of ${\cal T}(e)$, and for each chord, list the one or two triangles it is a side of.
Each chord is given by its two endpoints on $\partial P$.
\end{enumerate}



\begin{claim}
\label{claim:cover-one-funnel}
${\cal T}(e)$ can be computed in time $O(|Y(e)|)$, and has size $O(|Y(e)|)$.
\end{claim}

 

Define 
\defn{${\cal T}$} 
to be $ \bigcup_e {\cal T}(e)$. 
We prove that $\cal T$ is a coarse cover of $P$ according to Definition~\ref{defn:coarse-cover}.
(Note that a chord may appear as a side of triangles in more than one ${\cal T}(e)$.  We could, in fact, identify these, but instead we simply allow duplicates in the list of chords.)




\begin{lem} 
\label{lem:coarse-cover}
The set $\cal T$ 
is a coarse cover of $P$.
Furthermore, $\cal T$
has size $O(n)$ and  can be computed in time $O(n)$.  
\end{lem}
\begin{proof}
From Lemma~\ref{lemma:flowers_total_linear}, we can construct all the edge funnels $Y(e)$ in time $\sum_e |Y(e)| \in O(n)$, and from Claim~\ref{claim:cover-one-funnel}, we
can compute ${\cal T}(e)$ in time $O|Y(e)|)$.  Thus we can compute $\cal T$ in time $O(n)$.


To prove that $\cal T$ is a coarse cover, first observe that the functions $f_T$ have the correct forms. 
By Lemma~\ref{lem:voronoi_containment}, for any point $p \in P$, if $e$ is a farthest edge from $p$, then $p$ is in the edge funnel $Y(e)$ so $p$ is contained in some triangle $T$ in the partition of $Y(e)$, and is therefore contained in a triple $(T,f_T,e)$ of ${\cal T}(e)$.
\end{proof}
