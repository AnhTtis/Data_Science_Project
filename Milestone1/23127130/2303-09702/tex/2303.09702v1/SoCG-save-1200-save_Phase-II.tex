\section{Phase II: Finding the Geodesic Edge Center}
\label{section:final_edge_center}
\label{section:PhaseII}
\label{section:Phase-II}


The first step of Phase II is 
to construct a 
\defn{coarse cover} (Definition~\ref{defn:coarse-cover}) 
of the polygon
in linear time.
\changed{As shown in Figure~\ref{fig:coarse-cover} the ``funnel'' $Y(e)$ consisting of shortest paths between a chain on $\partial P$ with farthest edge $e$ and $e$ itself is partitioned into its shortest path map.  If the result includes trapezoids, we partition each one into two triangles\footnote{Thus our triangles are not necessarily ``apexed'' triangles as in~\cite{linear_time_geodesic}.}. 
Each triangle is bounded by two polygon chords and a segment of a polygon edge, and the distance to $e$ has the form required by Definition~\ref{defn:coarse-cover}.}
See Appendix~\ref{appendix:coarse-cover} for details.
We seek the point inside $P$ that minimizes the upper envelope of the functions of the coarse cover. 
\changed{Note that Phase I can detect if the edge center lies \emph{on} $\partial P$.}


\begin{figure}
\centering
\includegraphics[trim={0 0cm 0 0}, width=.8\textwidth]{RS_figures/edge_funnel_2-cropped.pdf}
\caption{
\changed{Constructing the coarse cover by partitioning the funnel 
$Y(e)$ (shaded) into a shortest path map from $e$ to its chain on $\partial P$ (in red).} 
}
\label{fig:coarse-cover}
\label{fig:flower}
\end{figure}


Our final divide-and-conquer algorithm follows the vertex center algorithm of Ahn et al.~\cite{linear_time_geodesic}, generalized to farthest edges, and repairing flaws in their approach.
At each step of the algorithm we have a subpolygon $Q$ whose interior contains the center together with the coarse cover elements whose triangles intersect the interior of $Q$, and we shrink the subpolygon and eliminate a constant fraction of the 
coarse cover.
Each recursive step takes time linear 
in the size of the subproblem (the size of $Q$ plus the size of its coarse cover).
The subpolygons we work with are  \defn{$3$-geodesic cells} defined as follows.
A geodesic path $\gamma$ joining two points on $\partial P$ divides $P$ into two closed weakly simple polygons that we will refer to as \defn{geodesic half-polygons}.
A \defn{$3$-geodesic cell}
of $P$ is the intersection of at most three geodesic half-polygons.
\newchanged{We only recurse on \defn{simple} $3$-geodesic cells.} 
See Figure~\ref{fig:shatter}.

The algorithm has two stages.
In Stage 1
no triangle of the coarse cover contains $Q$ (this is true initially when $Q=P$), so every triangle
has a chord crossing $Q$ and we
use $\epsilon$-net techniques on the set of such chords to reduce to a smaller cell $Q'$ that is crossed by a fraction of the chords, and hence by a %
fraction of the coarse cover triangles.
Once $Q$ is contained in a triangle of the coarse cover
we show (see Lemma~\ref{lem:floating_cells_constant_size} in Appendix~\ref{appendix:Q-bounds}) that it must have constant size.
In fact, we will exit Stage 1 as soon as $Q$ has constant size.
It is then easy to reduce $Q$ to a triangle. 
After that, we switch to Stage 2, where the convexity of $Q$ allows us to use 
a Megiddo-style prune-and-search technique to recursively reduce the size of the subproblem. 
Stage 2 is deferred to Appendix~\ref{section:constantQ}.  




\subsection{Stage 1: Algorithm for Large $Q$}
\label{section:largeQ}

Consider a subproblem corresponding to a 
\newchanged{simple}
$3$-geodesic cell $Q$ with $|Q| >6$.
We give an algorithm that either finds the edge center or reduces to a subproblem with $|Q| \le 6$, which is handled by Stage 2. 
In Stage 1, 
 no triangle of the coarse cover
contains $Q$ (Lemma~\ref{lem:floating_cells_constant_size} in Appendix~\ref{appendix:Q-bounds}), so each one has a chord crossing $Q$---we denote this set of 
chords by ${\cal K}(Q)$.

To apply $\epsilon$-net techniques
\newchanged{we first define a range space.}
Consider a $3$-geodesic cell $G$ 
defined by geodesic paths $\gamma_1, \gamma_2, \gamma_3$.
The boundary of $G$ has a subpath of each $\gamma_i$ called a \defn{geodesic chain}.
Around $\partial G$, two consecutive geodesic chains may be joined by a subpath of $\partial P$ called a 
\defn{polygon chain} or may share a common endpoint in the interior of $P$ which we call an \defn{interior vertex} of $Q$.   
See Figure~\ref{fig:shatter}.
Observe that 
$G$ is the geodesic convex hull of its (at most three)
polygon chains and interior vertices.
\newchanged{Define a \defn{$3$-hull} to be the geodesic convex hull of at most three polygon chains and interior vertices. (These are more general since a line segment inside $P$ is a $3$-hull but not a $3$-geodesic cell.)
Define the \defn{$3$-hull range space} as follows. 
The ground set is 
a set $\cal K$ of chords of $P$,
and for each $3$-hull $H$ of $P$ 
there is a range ${\cal K}(H)$ 
consisting of all chords of 
$\cal K$ that \defn{cross} $H$.
Here a chord \defn{crosses} a set  if both its open half-polygons contain points of the set.
}



The algorithm finds an $\epsilon$-net of the \newchanged{$3$-hull range space} on ${\cal K}(Q)$, 
which is 
a constant-sized set $N \subseteq {\cal K}(Q)$ such that any \newchanged{$3$-hull (hence, any $3$-geodesic cell)} not intersected by a chord of $N$ is intersected by only a constant fraction of the chords of ${\cal K}(Q)$---this is the important property that allows us to discard a fraction of the chords.
The set of chords $N$ forms an arrangement that partitions $Q$ into cells.
We use the chord oracle to determine which cell contains the center. We then 
add geodesic paths to 
subdivide this cell into 
a constant number of $3$-geodesic cells and use a \defn{geodesic oracle}
(see Appendix~\ref{appendix:geodesic-oracle})
to 
\newchanged{find a simple}
$3$-geodesic cell, say $Q'$, that contains the center point. The algorithm recurses on $Q'$, whose coarse cover is a fraction of the size.


More details of the algorithm can be found in Appendix~\ref{appendix:Algorithm-Stage1}.  
For now, we expand on the aspects of the algorithm that differ from the approach of Ahn et al.~\cite{linear_time_geodesic}.
Instead of $3$-geodesic cells, their algorithm works with \emph{$4$-cells}---they use chords rather than geodesic paths to bound their cells, and they use four of them rather than three.  
The number (three versus four) is not significant, but we 
\newchanged{use geodesic cells and hulls}
to obtain the following two results. 

\subparagraph{1.} The
\newchanged{$3$-hull}
range space has finite VC-dimension.  This implies that constant-sized $\epsilon$-nets exist. Furthermore, there is a ``subspace oracle'' 
that allows us to find an $\epsilon$-net $N$ in deterministic linear time~\cite[Chapter 47, Theorem 47.4.3]{toth2017handbook}. For 
further 
background see Appendix~\ref{appendix:epsilon-net-overview}. 

Ahn et al.~claim that their range space 
\newchanged{(of chords crossing $4$-cells)}
has finite VC-dimension but their proof is flawed.
Our proof shows that their range space does in fact have finite VC-dimension.
\newchanged{They do not mention subspace oracles, without which their algorithm runs in expected linear time rather than deterministic linear time as claimed.}
We expand on these aspects in Section~\ref{section:epsilon-net-results} below.

\subparagraph{2.} A cell of the arrangement of $N$ can be partitioned into 
$O(1)$ $3$-geodesic cells.

The method used by Ahn et al.~to 
subdivide a cell of $N$ into $4$-cells by adding a constant number of chords is
incomplete.
They subdivide using vertical chords---this is why they prefer $4$-cells over $3$-cells---but their algorithm does not add enough vertical chords. 
For a counter-example, \newchanged{see %
Appendix~\ref{section:counterexample}}.
We see how to repair their partition step %
but we find 
$3$-geodesic cells more natural.

\remove{
\begin{figure}
  \centering
  \includegraphics[width=.15\textwidth]{RS_figures/decomposition_counterexample-cropped.pdf}
  \caption{
  \changed{(a) Ahn et al.~partition a cell of the arrangement of $N$ by adding vertical chords from the endpoints and intersection points of chords $N$.  (b) A counterexample to their claim that this produces 4-cells.} 
  \old{
  Schematic outlining the problem with Ahn et al.'s proposed vertical decomposition of the polygon into 4-cells.
  The blue chords represent an $\epsilon$-net of their range space.
  Using their decomposition scheme, the polygon is not partitioned into 4-cells, contrary to their claim.}
  \anna{It would be nice to include a version of their partition figure  beside this (we should redraw to avoid copyright).}
  }
\label{fig:decomposition_counterexample}
\end{figure}
}


\subsection{$\epsilon$-Net Results for Stage 1}
\label{section:epsilon-net-results}

In this section we expand on the $\epsilon$-net results that are needed for Stage 1 of the algorithm as described above.  We also give details of the flaws in the approach of Ahn et al.~\cite{linear_time_geodesic}.
For an overview of $\epsilon$-nets as used for geometric divide-and-conquer, see Appendix~\ref{appendix:epsilon-net-overview}.
To show that $\epsilon$-nets exist we need the following result.

\begin{lem}
\label{lem:constant_shattering_dimension}
The \newchanged{$3$-hull}
range space has 
VC-dimension less than 259.
\end{lem}



Our proof of Lemma~\ref{lem:constant_shattering_dimension} works equally well for
\newchanged{$4$-hulls}---the
bound becomes %
$372$.
Since $4$-cells are a subset of $4$-hulls, our proof implies finite VC-dimension ($\le 372$) for the $4$-cell range space which repairs the claim by Ahn et al.\footnote{In response to our enquiries, Eunjin Oh independently suggested a similar remedy.}.
We explain the flaw in %
their proof.
Let us refer to the set of chords intersecting a $4$-cell as a ``$4$-cell range''. 
Ahn et al.~prove that the $1$-cell range space has 
VC-dimension at most 65,535. 
They note that a $4$-cell is the intersection of four $1$-cells, and then claim in their Lemma 9.1 that
this implies finite VC-dimension for the $4$-cell range space.  As justification, they refer to 
Proposition 10.3.3 of Matousek's text~\cite{matousek2002}, which 
states 
that the VC-dimension is bounded for any family whose sets can be defined by a formula of Boolean connectives (union, intersection, set difference).
However, Matousek's proposition cannot be applied in this situation because, although a $4$-cell is the intersection of four $1$-cells,  
it is not true that 
a $4$-cell \emph{range} is the intersection of four $1$-cell \emph{ranges}.
In particular, a chord can intersect two $1$-cells, but not intersect the intersection of the two $1$-cells.  For example, a line of slope $-1$ can intersect the $+x$ half-plane and the $+y$ half-plane without intersecting the $+x,+y$ quadrant.






\begin{proof}[Proof of Lemma~\ref{lem:constant_shattering_dimension}]
We will prove that the shattering dimension is 6 and then apply the result that a range 
space with shattering dimension $d$ has VC-dimension bounded by $12d \ln{(6d)}$ (Lemma 5.14 from Har-Peled~\cite{har2011geometric}). 
For $d=6$ this is $< 259$.

For an explanation of shattering dimensions see Appendix~\ref{appendix:epsilon-net-overview}.
We must show that for
\changed{a set $\cal K$ of chords with 
$| {\cal K}| = m$, the number of distinct ranges is $O(m^6)$.}
We prove that 
the range space for ${\cal K}$ is the same 
if we replace 
\newchanged{$3$-hulls}
by   ``$3$-anchor hulls'' that are defined in terms of $\cal K$, more precisely, 
in terms of the arrangement $A({\cal K})$
of the chords ${\cal K}$ plus the edges of $P$.
Define an \defn{anchor} to be an interior face, edge, or vertex of 
$A({\cal K)}$, or a 
\changed{polygon chain}
with endpoints in 
$V({\cal K})$, the set of endpoints of chords $\cal K$.
A \defn{$3$-anchor hull} is the geodesic convex hull of at most three anchors.




\begin{figure}
  \centering
  \includegraphics[width=0.7\textwidth]{RS_figures/threeGthreeAcells_3-cropped.pdf}
  \caption{
  A simple 3-geodesic cell $Q$ defined by the red geodesics.
  \newchanged{As a $3$-hull, $Q$ is the geodesic convex hull of}
polygon chain $c_3$ (bold black) and %
points $v_1$ and $v_2$.
  Solid blue chords cross $Q$, while dashed blue chords do not.
  The $3$-anchor hull
  $\psi(Q)$ (lightly shaded) is the geodesic
  convex hull of:
  anchor $a_1$, the face in the chord arrangement containing $v_1$;
  anchor $a_2$, the edge with $v_2$ in its interior; and
  anchor $a_3$, the polygon chain extending $c_3$ to chord endpoints. 
  The same %
  chords cross $Q$ and $\psi(Q)$.
}
\label{fig:shatter}
\end{figure}


\begin{lem}
\label{lem:anchor-ranges}
The set of ranges ${\cal R} = \{ {\cal K}(Q) \mid Q \text{ is 
a $3$-hull
} \}$ is the same as 
the set of ranges
$\overline{\cal R} = \{ {\cal K}({Q}) \mid {Q} \text{ is 
a $3$-anchor hull } \}$.
\end{lem}

\begin{proof}
To prove ${\cal R} \subseteq \overline{\cal R}$, consider a 
\newchanged{$3$-hull $Q$, the geodesic convex hull of at most three points and polygon chains.}
\newchanged{Replace any point $p$}
by the 
smallest (by containment) vertex, edge, or face of $A({\cal K)}$ that contains $p$.
See Figure~\ref{fig:shatter}.
Replace any 
polygon chain $C$ %
by the smallest chain of 
$\partial P$ 
containing $C$ and with endpoints in $V({\cal K})$.
Let 
$\psi(Q)$ be the geodesic convex hull of these anchors.  
Then $\psi(Q)$ is a $3$-anchor hull that contains $Q$, and it is straight-forward to prove that 
${\cal K}(Q) = {\cal K}(\psi(Q))$
(see Claim~\ref{claim:same-crossing-chords} in Appendix~\ref{appendix:epsilon-nets}).

\remove{\begin{claim}
\label{claim:same-crossing-chords}
A chord of $\cal{K}'$ crosses $Q$ if and only if it crosses $a(Q)$,
i.e., ${\cal K}'(Q) = {\cal K}'({a(Q)})$.
\label{claim:expansions}
\end{claim}

\begin{proof} 
One direction of the proof is simple: If a chord crosses $Q$, it must cross $a(Q)$ since $Q \subseteq a(Q)$.

For the other direction we prove
that if a chord $k \in {\cal K}'$ does not cross the 
$3$-geodesic cell $Q$, 
then it does not cross the 
$3$-anchor cell $a(Q)$.
Suppose that a chord $k \in \mathcal{K}'$ does not cross $Q$. 
Then $Q$ is contained in one of the closed half-polygons, say $H$, defined by $k$.  This implies that the polygon chains and interior vertices of $Q$ are contained in $H$.  Since the corresponding anchors were defined to not cross chords of $\cal K'$, they are contained in $H$. Thus $a(Q)$, being the geodesic convex hull of sets in $H$, is also in $H$.  So $k$ does not cross $a(Q)$. 
\end{proof}
}

For the other direction, let $Q$ be a $3$-anchor hull.
Replace an anchor that is a face, edge, or vertex of $A({\cal K})$ by a point in the interior of that face, edge, or vertex. Leave the polygon chain anchors as is.
Let $\gamma(Q)$ be the
geodesic convex hull of the resulting points and chains.
Observe that
\newchanged{$\gamma(Q)$ is a $3$-hull and}
$\psi(\gamma(Q)) = Q$.
Thus by Claim~\ref{claim:same-crossing-chords}, ${\cal K}(Q) = {\cal K}(\gamma(Q))$.
\end{proof}

\changed{
To complete the proof of Lemma~\ref{lem:constant_shattering_dimension} we claim that the number of $3$-anchor hulls of ${\cal K}$ is $O(m^6)$.
An anchor may be a interior vertex, edge, or face of $A(\cal K)$, of which there are $O(m^2)$ possibilities.
Otherwise, an anchor is 
a chain of $\partial P$ between vertices of $V({\cal K})$, 
also with
$O(m^2)$ possibilities.
Thus the number of $3$-anchor hulls 
is $O((m^2)^3) = O(m^6)$.
}
\end{proof}

\paragraph*{Subspace Oracle}

\changed{In Appendix~\ref{appendix:epsilon-nets} we prove that the $3$-hull range space has a subspace oracle, i.e., that there is a deterministic algorithm that, given a subset ${\cal K}' \subseteq {\cal K}$ with $|{\cal K}'| = m$, computes the set
of ranges ${\cal R} = \{ {\cal K}'(Q) \mid Q \text{ is 
a $3$-hull}\}$
in time 
$O(m^7)$.
The idea is 
\newchanged{to use Lemma~\ref{lem:anchor-ranges} and}
to construct $A({\cal K}')$ minus the edges of $P$, 
and find, for each 
chord $K \in {\cal K}'$, which of the $O(m^2)$ anchors in $A({\cal K}')$ intersect each side of $K$, and then, for each of the $O(m^6)$ 3-anchor hulls, eliminate the chords that have all three anchors to one side, leaving the chords that cross the hull. 
}

\remove{
\begin{lem}
\label{lem:subspace-oracle}
The $3$-geodesic cell range space has a subspace oracle. 
\end{lem}

\begin{proof} We must provide a deterministic algorithm that, given a subset ${\cal K}' \subseteq {\cal K}$ with $|{\cal K}'| = m$, computes the set
$\mathcal{R}_{|{\cal K}'}$
in time $O(m^{d+1})$, where $d=6$ is the shattering dimension of the $3$-geodesic cell range space.  


In Lemma~\ref{lem:constant_shattering_dimension} we proved that 
the number of ranges in 
$\mathcal{R}_{|{\cal K}'}$ is $O(m^6)$.  Each range has at most $m$ chords in it, so the desired output, $\mathcal{R}_{|{\cal K}'}$, has size $O(m^7)$, and our goal it to compute that output in time proportional to its size.
To do this, we will basically continue the proof of Lemma~\ref{lem:constant_shattering_dimension}.
Refer back to that proof 
to see that what we
must do is find the $O(m^6)$ $3$-anchor cells $Q$, and compute for each $Q$, the set of chords of $\cal K'$ that cross it. 

To avoid dependence on $n$, we will work with the graph $G(A({\cal K'}))$
and represent each $3$-anchor cell combinatorially via its anchors. 

\anna{Here (or above?) do we need to define dummy edges?}

\old{
For a subset $\mathcal{K}' \subseteq {\cal K}$, 
define \defn{$A(\cal{K'})$} to be
the subdivision of the plane formed by the set of chords $\mathcal{K'}$ and the edges of $P$.
This is an arrangement of line segments, with the special property that all segment endpoints are on the outer face. 
Let \defn{$V({\cal K'})$} denote the endpoints of the chords in $\cal K'$.
If $\cal{K'}$ has size $m$, then
$A(\cal{K'})$ has $O(m^2)$ faces, $O(m^2)$ internal vertices and edges, and $n+2m$ external vertices and edges on the boundary of $P$. 
In particular, the external vertices are the vertices of $P$ union $V({\cal K'})$.
For our algorithms we will avoid the dependence on $n$ by working with a combinatorial version of $A(\cal{K'})$ in which each minimal chain \anurag{should this be maximal chain?}
\anna{I think minimal is correct}
along $\partial P$ with endpoints in $V({\cal K'})$
is represented by a single ``dummy edge''. 
Note that the number of dummy edges is at most $2m$.  
Let \defn{$G(A({\cal K'}))$}
denote this planar graph, which has $O(m^2)$ vertices, edges, and faces.  
}

We first compute $G(A({\cal K'}))$ as follows. 
Compute 
the arrangement of the $m$ line segments $\cal K'$ in time $O(m^2)$ \anna{cite an algorithm (Chazelle and Edelsbrunner? -- or is there a simpler more basic one?)}.  
Then, traverse the outer face of the arrangement, adding dummy edges corresponding to subchains of $\partial P$ between vertices of  $V({\cal K}')$. 
We thus compute $G(A({\cal K'}))$ in time $O(m^2)$.

Next, we enumerate all of the possible anchors: the $O(m^2)$ internal vertices, edges, and faces of $G(A({\cal K'}))$, and the $O(m^2)$ polygon chains, each represented by two endpoints in $V({\cal K'})$. 

For each of the $m$ chords $k$ of $\cal K'$ we enumerate the
anchors that lie in each of the two closed half-polygons $H$ defined by $k$.
In particular, we can traverse $G(A({\cal K'}))$ in time $O(m^2)$ to find the 
vertices, edges, and faces that lie in $H$.  We can also decide which of the $O(m^2)$  polygon chains lie entirely in $H$, based on where the endpoints lie.  This takes time $O(m^2)$ per chord, for a total of $O(m^3)$.


Finally, we can enumerate all the $O(m^6)$ choices of at most three anchors that determine a $3$-anchor cell $Q$.  For each choice we spend $O(m)$ time to find the set of chords crossing $Q$---begin with all of $\cal K'$ and eliminate chords that have all three anchors on the same side, since these are precisely the chords do not cross $Q$.
This gives us the set of chords crossing $Q$.
\end{proof}
}










