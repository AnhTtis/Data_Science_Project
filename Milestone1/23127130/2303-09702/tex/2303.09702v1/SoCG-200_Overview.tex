\section{Overview of the %
Algorithm}
\label{sec:overview}

\remove{
\anna{Could skip:} The algorithm to find the geodesic edge center has two phases.  In Phase I we find the \reviewerchange{geodesic farthest-edge} Voronoi diagram restricted to the polygon boundary.  In Phase II we reduce the problem to finding a point in the polygon that minimizes the upper envelope of a linear number of easy-to-compute
convex functions each defined on a triangle inside the polygon. 
That problem is then solved using a two-stage divide-and-conquer algorithm based first on $\epsilon$-nets and then on 
prune-and-search techniques like those used in Megiddo's linear programming algorithm. 
A more detailed overview is below, but we first outline the previous work that our algorithm builds upon, and explain what is novel about our contributions.
}

\changed{Before giving the overview of our algorithm, we outline the previous work that our algorithm builds upon, and explain what is novel about our contributions.}

Pollack et al.~\cite{pollack_sharir} gave an $O(n \log n)$ time algorithm to find the geodesic vertex center of a simple polygon. 
A main ingredient is to solve the problem one dimension down.
In particular, they develop an $O(n)$ time 
\emph{chord oracle}
that, given
a chord of the polygon, finds the \emph{relative center} restricted to the chord
and from that, determines whether the 
center of the polygon lies to left or right of the chord.  
By applying the chord oracle $O(\log n)$ times, they limit the 
search to a convex subpolygon where Euclidean distances can be used.
This reduces the problem to finding 
a minimum disc that encloses some disks, which Megiddo~\cite{megiddo_spanned_ball} solved in linear time using the same approach as  for his linear programming algorithm.
The chord oracle was extended to handle farthest \emph{edges} instead of vertices by
Lubiw and Naredla~\cite{lubiw2021visibility}. 

The idea used in the chord oracle algorithm is central to further developments. Expressed in general terms, 
the goal is to find a point in a domain (either a chord or the whole polygon) that minimizes the maximum distance to a %
site (a vertex or edge of the polygon). The idea is to first find what we will call a \defn{coarse cover} of the domain by a linear number of elementary regions $R$ (intervals or triangles), 
\changed{each with an associated easy-to-compute convex function $f_R$ that captures the geodesic distance to a potential farthest edge, and with the property that
the upper envelope of the functions is 
the geodesic radius function.} %
Thus, the goal is to find the point $x$ that minimizes the upper envelope of the  functions $f_R$. 
When the domain is a chord,
the chord oracle solves this in linear time.

\changed{When the domain is the whole polygon, and the sites are vertices, 
Ahn et al.~\cite{linear_time_geodesic} gave a \reviewerchange{linear-time algorithm}.
They find a coarse cover of the whole polygon starting from 
Hershberger and Suri's algorithm~\cite{hershberger1997matrix} to find the farthest vertex from each vertex.  
They then use divide-and-conquer based on $\epsilon$-nets---their big innovation---to reduce the domain to a triangle. After that, the vertex center is found using  
Megiddo-style prune-and-search techniques 
like those used by 
Pollack et al.}

Our algorithm uses a similar approach, modified to deal with farthest \emph{edges} rather than vertices. 
Another difference is that 
we give a simpler method of finding a coarse cover of the polygon by first finding 
the \reviewerchange{geodesic farthest-edge} Voronoi diagram on the polygon boundary.
\changed{There is a \reviewerchange{linear-time algorithm} to find the geodesic farthest \emph{vertex} Voronoi diagram on the polygon boundary by Oh, Barba, and Ahn~\cite{oh2020voronoi}.
Our algorithm is considerably simpler, and the idea of using the boundary Voronoi diagram to find the center is novel.}

Other differences between our approach and that of Ahn et al. are introduced in order to 
repair 
some flaws in 
their paper.
\changed{
They use $\epsilon$-net techniques, but their range space does not have the necessary properties for finding $\epsilon$-nets in deterministic linear time. We remedy this by using a different range space, thereby repairing and generalizing their result.}


\paragraph*{Algorithm Overview}

\paragraph*{Phase I: Finding the Farthest Edge Voronoi Diagram Restricted to
the Polygon Boundary (Section~\ref{section:Phase-I})}
We first show that the \reviewerchange{linear-time algorithm} of Hershberger and Suri~\cite{hershberger1997matrix} that finds the farthest \emph{vertex} from each vertex can be modified to find the farthest \emph{edge} from each vertex.
A polygon edge $e$ whose endpoints have the same farthest edge $g$ is then part of the farthest Voronoi region of $g$.
To find the Voronoi diagram on a 
\emph{transition edge}
$e$ 
that has different farthest edges at its endpoints, we must find
the upper envelope of the coarse cover of $e$.   
\changed{We use the fact that the coarse cover of $e$ is 
constructed from two shortest path trees inside a smaller subpolygon called the \emph{hourglass} of $e$. The hourglasses of all transition edges can be found in linear time. 
In each hourglass, 
the shortest path trees allows us to construct the upper envelope incrementally in linear time---this is a main new aspect of our work.}




\paragraph*{Phase II: Finding the Geodesic Edge Center (Section~\ref{section:PhaseII})}
We first find a coarse cover of the polygon by triangles, 
each bounded by two polygon chords plus a segment of an edge, and 
each with an associated convex function that captures the %
geodesic distance to a potential farthest edge---the potential farthest edges are 
those that have 
non-empty Voronoi regions
on the boundary of $P$.
\remove{We do this by expanding the Voronoi regions on the boundary of $P$ into the interior of $P$.}
The problem of finding the edge center is then reduced to the problem of finding
the point that minimizes the upper envelope of the coarse cover functions.


To find this point we use divide-and-conquer, reducing in each step to a smaller subpolygon with a constant fraction of the coarse cover elements. 
There are two stages.
In Stage 2, once the subpolygon is a triangle, the prune-and-search approach of Megiddo's can be applied.  
In Stage 1
every coarse cover triangle that intersects the subpolygon has a boundary chord crossing the subpolygon, and $\epsilon$-net techniques are used to reduce the number of such chords, and hence the number of coarse cover elements.
Our approach follows that of Ahn et al.~\cite{linear_time_geodesic} but 
we repair some flaws---another main new aspect of our work.
Ahn et al.~recurse on subpolygons called ``4-cells'' that are the intersection of four half-polygons (a \defn{half-polygon} is the subpolygon to one side of a chord).
\newchanged{
We instead recurse on ``$3$-anchor hulls'' that are the geodesic convex hulls of at most three points or subchains of the polygon boundary.} 
These define a 
range space 
whose ground set is 
a set of chords and whose ranges are subsets of chords that cross a
$3$-anchor hull. 
We prove that our range space has finite VC-dimension, which repairs the faulty proof in Ahn et al.~for 4-cells.
Even more crucially, we give a ``subspace oracle'' that permits an $\epsilon$-net to be found in \emph{deterministic} linear time, 
\newchanged{something missing from their approach.}


