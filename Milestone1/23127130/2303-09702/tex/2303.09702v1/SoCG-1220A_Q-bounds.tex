\subsection{
Details for Section~\ref{section:final_edge_center}, Stage 1 Subproblems
}
\label{appendix:Q-bounds}
\changed{Stage 1 of the 
algorithm to find the edge center recurses on subproblems, each consisting of 
a subpolygon $Q$ that is a \newchanged{simple
$3$-anchor hull,} %
together with the coarse cover elements that intersect the interior of $Q$. 
We give some properties of these.} 


\begin{obs}
\label{obs:3-geo-cell}
Let $Q$ be a 
\newchanged{$3$-anchor hull.}

\begin{enumerate}
\squeezelist

\item $Q$ is a closed connected weakly-simple polygon, and is \defn{geodesically convex in $P$}, meaning that for any two points $a$ and $b$ in $Q$, the geodesic path from $a$ to $b$ in $P$ is contained in $Q$.  This implies that the intersection of $Q$ with a chord [geodesic] of $P$ is a chord [geodesic] of $Q$.


\item The boundary of $Q$ consists of: 
\newchanged{the at most three anchors that are subchains of $\partial P$;
and at most three geodesic paths between pairs of anchors.
}


\item 
\newchanged{If $v$ is a vertex of $Q$ but not a vertex of $P$, then $v$ is a point anchor or the endpoint of an anchor chain that is interior to an edge of $P$.  In either case, $v$ is a convex vertex of $Q$.}

\end{enumerate}
\end{obs}


Each subproblem consists of the following.
\begin{enumerate}
\squeezelist
\item $Q$, a 
\newchanged{simple $3$-anchor hull} %
of $P$
that contains the geodesic edge center in its interior.
\item the set \defn{${\cal T}(Q)$}
of all coarse cover elements whose triangles intersect the interior of $Q$. 
Each triangle of the coarse cover is specified by its two defining chords of $P$ and the subsegment of an edge of $P$ that forms its third side.
Note: In this section we will
refer to triangles of the coarse cover  elements of ${\cal T}(Q)$ as
``triangles of ${\cal T}(Q)$.'' 

\item 
the set \newchanged{${\cal K}_T(Q)$ of 
defining chords of triangles of ${\cal T}(Q)$},
each given by its endpoints on the boundary of $P$, and each recording the one or two triangles of ${\cal T}(Q)$ that it is a side of.  
We also maintain the subset \newchanged{${\cal K}(Q)$ of chords that cross $Q$}, each given by its endpoints on the boundary of $Q$ (as well as its endpoints on the boundary of $P$).
\end{enumerate}

To solve a subproblem in Stage 1 means finding a point in $Q$ that minimizes the upper envelope of the functions of the coarse cover  $\mathcal{T}(Q)$, or reducing to $|Q| < 6$.


Define \defn{$t(Q)$} $:= |\mathcal{T}(Q)|$.
The \defn{size of a subproblem} is $|Q| + t(Q)$, where $|Q|$ is the number of vertices of $Q$ (as a polygon).
Initially, $Q$ is $P$,
$\mathcal{T}(Q)$ is all of $\cal T$, and ${\cal K}_T(Q)$ and ${\cal K}(Q) $ are all of $\mathcal{K}$.  
The size of the initial problem is $O(n)$, because $\cal T$ has linear size by Lemma~\ref{lem:coarse-cover}. 
Our goal is to  spend linear time in the size of a subproblem to reduce the size by a constant fraction.




We need some results about the size of $Q$.

\begin{lem}
\label{lem:floating_cells_constant_size}
If \newchanged{a simple $3$-anchor hull}
$Q$ is contained in a 
triangle of the coarse cover then $|Q| \le 6$.
\end{lem}
\begin{proof}
Let $T$ be a triangle of ${\cal T}(Q)$
that contains $Q$.
We claim that the boundary of $Q$ has at most three edges that are subsegments of edges of $P$.  Any such segment must lie on the boundary of $T$, and each of the three sides of $T$ can contain at most one such segment by our assumption that no three vertices of $P$ are collinear
(Assumption~\ref{assumption:no_three_points_collinear}).

We next claim that each of the at most three
geodesic chains on the boundary of $Q$ consists of a single segment.  This is because an internal vertex $v$ of a geodesic chain is a vertex of $P$, which must then be on the boundary of $T$ (since no point on the boundary  of $P$ lies in the interior of $T$). 
But then the internal angle of $Q$ at $v$ is $\le \pi$, so $v$ is not an internal vertex of a geodesic path.

Thus $Q$ has at most six edges.
\end{proof}




\begin{claim}
\label{observation:cell_triangles}
\newchanged{If $Q$ is  a simple $3$-anchor hull, then}
$|Q| \le 3 t(Q) + 6 \le 9 t(Q)$.
\end{claim}
\begin{proof}
\newchanged{Because $Q$ is simple, every vertex $v$ of $Q$ has interior points of $Q$ in its neighbourhood, so $v$}
must be contained in some triangle of ${\cal T}(Q)$ since ${\cal T}(Q)$ is 
a coarse cover of $Q$. 
By Observation~\ref{obs:3-geo-cell}, all but 6 of the vertices of $Q$ are vertices of $P$.


To complete the proof we show that each triangle $T$ of the coarse cover contains at most three vertices of $P$.
No vertex of $P$ is internal to $T$. Since $P$ does not have 3 collinear vertices (by Assumption~\ref{assumption:no_three_points_collinear}), each side of $T$ contains at most 2 vertices of $P$.  Furthermore, one side of $T$---call it $s_1$---is a subsegment of an edge of $P$, so it cannot contain vertices in its interior. 
Triangles of the coarse cover either have a vertex of $P$ as an apex opposite $s_1$, or arise from subdividing a trapezoid (see Appendix~\ref{appendix:coarse-cover}).  In the first case, $T$ has at most one more vertex on each side incident to the apex for a total of at most three vertices of $P$.  In the second case, $T$ has a side that is a diagonal of a trapezoid and contains no vertices in its interior, though it may have a vertex of $P$ at its intersection with $s_1$, and 
the third side of $T$ has at most two vertices of $P$, for a total of at most three vertices of $P$.
Thus $T$ contains at most three vertices of $P$.


This shows that $|Q| \le 3t(Q) + 6$.  For the second part of the inequality, note that $t(Q) \ge 1$.
\end{proof}

We note that the above Claim depends on the assumption that $Q$ is a $3$-anchor hull of $P$.
If we constructed $3$-anchor hulls of $3$-anchor hulls, then
the number of vertices that are not vertices of $P$ would grow. 


We also need the following relationships between the number of  chords  and the 
number of coarse cover triangles.

\begin{claim}
\label{observation:chords_triangles}
$|\mathcal{K}(Q)| \leq |\mathcal{K}_T(Q)| \leq 2 t(Q)$.  If $Q$ is not contained in a triangle of ${\cal T}(Q)$, then $t(Q) \le 2|{\cal K}(Q)|$.
\end{claim}
\begin{proof}
For the first inequality, ${\cal K}(Q) \subseteq {\cal K}_T(Q)$ and every triangle of the coarse cover has two chords (the third side is a piece of a polygon edge).  For the second inequality, since no triangle of ${\cal T}(Q)$ contains $Q$, each one has at least one chord that crosses $Q$, and each chord comes from the coarse cover ${\cal T}(e)$ of a  
funnel $Y(e)$ and is a side of one or two coarse cover triangles in ${\cal T}(e)$.  (If a chord arises from more more than one $Y(e)$, we duplicate it in $\cal K$, see the definition of ${\cal T}(e)$ in
Appendix~\ref{appendix:coarse-cover}.)
\end{proof}
