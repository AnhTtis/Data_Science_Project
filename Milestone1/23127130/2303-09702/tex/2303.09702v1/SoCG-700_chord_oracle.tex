\subsection{Chord Oracles and Coarse Covers}\label{section:chord_oracle}


In this section we 
describe the chord oracle results that we need from previous work, 
and we give a unified explanation of those algorithms and our current algorithm in terms of coarse covers.
The basic function of a chord  oracle is to decide, given a chord $K$, whether the center lies to the left or right (or on) the chord.
Pollack et al.~\cite{pollack_sharir} gave a linear time chord oracle for the geodesic vertex center, 
which is at the heart of all further geodesic center algorithms.
Lubiw and Naredla~\cite{lubiw2021visibility} extended the chord oracle to the case of the geodesic \emph{edge} center. %

In both cases, 
a main 
step is 
the ``one-dimension down''
problem of 
finding 
the \defn{relative center}, which is a point $c_K$ on $K$ that minimizes the 
geodesic radius function $r(x)$.
\changed{The directions of the first segments of the paths from $c_K$ to its farthest sites determine
whether the center of $P$ lies 
left/right/on $K$
(see Appendix~\ref{appendix:oracles}).}



\remove{
\changed{Algorithms to find the [relative] center 
depend on the following convexity property of the geodesic radius function $r(x)$.}
\remove{
The algorithms to find the relative center on a chord and to find the center of a polygon depend on a crucial convexity property.
Define the \defn{geodesic radius function}, $r(x)$, for $x \in P$
to be the maximum geodesic distance from $x$ to a site (a vertex or edge). 
Thus the center is the point $x$ that minimizes $r(x)$.}
A function is \defn{geodesically convex} on $P$ if the function is convex on every geodesic path in $P$.
The following result was proved for vertex sites by Pollack et al.~\cite{pollack_sharir} and for edge sites by Lubiw and Naredla~\cite{lubiw2021visibility}.
\anna{This could go in the appendix, since we don't appeal to it from the main text.}

\begin{lem}
\label{lem:geodesically-convex}
The geodesic radius function $r(x)$ is geodesically convex.
\end{lem}

}


Algorithms to find the relative 
center of a chord 
or the 
center of a polygon
rely on a basic convexity property of the geodesic radius function (see Lemma~\ref{lem:geodesically-convex} in Appendix~\ref{appendix:oracles}), and all
follow the same pattern, which can be 
formalized via the concept of a \emph{coarse cover} of the chord/polygon. 
The idea is that 
a \emph{coarse cover} for a domain (a chord/polygon)  
is a set of elementary regions $R$ (intervals/triangles) covering the domain,
where each region $R$ has 
an associated 
easy-to-compute convex function $f_R$, 
such that the upper envelope of the $f_R$'s is the geodesic radius function.
We give a precise definition 
for the case of farthest edges
(following~\cite{lubiw2021visibility} and specialized for our Assumptions~\ref{assumptions}).

\begin{definition}
\label{defn:coarse-cover}
A \defn{coarse cover} of chord $K$ [or polygon $P$] is  
a set of 
triples $(R,f,e)$
where 
\begin{enumerate}
\squeezelist
\item 
$R$ is a subinterval of $K$ [or a triangle of $P$], $f$ is a function defined on domain $R$, and $e$ is an edge of $P$.

\item For all $x \in R$, $f(x) = d(x,e)$ and
either: $f(x) = d_2(x,v) + \kappa$ where $d_2$ is Euclidean distance, $\kappa$ is a constant
and
$v$ is 
a vertex of $P$; or
$f(x) = d_2(x, {\bar e})$, where $d_2$ is Euclidean distance and $\bar e$ is the line through $e$.

\remove{
has one of the following forms:
\begin{itemize}
\squeezelist
\item $f(x) = d_2(x,v) + \kappa$ where $d_2$ is Euclidean distance, $\kappa$ is a constant
and
$v$ is 
a vertex of $P$.

\item $f(x) = d_2(x, {\bar e})$, where $d_2$ is Euclidean distance and $\bar e$ is the line through $e$.
\end{itemize}
}


\item 
For any point $x \in K$ [or $P$], and any edge $e$ that is farthest from $x$, there is a triple $(R,f,e)$ in the coarse cover with $x \in R$. 


\end{enumerate}
\end{definition}

Condition (3) 
implies that the upper envelope of the functions of the coarse cover is the geodesic radius function. Thus the
[relative] center problem breaks into two subproblems:
(1) find a coarse cover; and (2)  find the point $x$ that minimizes the upper envelope of the coarse cover functions.
The high-level idea for solving step (2) in linear time  (for a chord or polygon domain) is to 
recursively reduce the domain (the search space) to a subinterval or subpolygon while eliminating elements 
of the coarse cover whose functions are strictly dominated by others.
\changed{As for step (1)---constructing a coarse cover---see Section~\ref{section:Phase-I} for a chord and Section~\ref{section:Phase-II} for a polygon.

We call the chord oracle in Phase II when we use divide-and-conquer to search for the center in successively smaller subpolygons.  We actually need two variations of the basic chord oracle.  
First, we need a \emph{geodesic oracle} that tests which side of a geodesic contains the center.  Secondly, we 
do not construct a coarse cover of a chord/geodesic from scratch; rather, we intersect the triangles of the coarse cover of the subpolygon with the chord/geodesic, 
thus avoiding runtime dependence on $n$.
These variations are described in Appendix~\ref{appendix:oracles}.
}

\remove{
\changed{More details on the chord oracle. 
After finding the relative center $c_K$, the last step of the chord oracle involves testing the maximum values of the coarse cover functions at $c_K$ to determine whether the center of $P$ lies left/right/on $K$.   
}

\remove{
We note that the definition in~\cite{lubiw2021visibility} allows an exception to condition (3): in case 
two triples have identical $R$ and $f$
(which implies that for all $x \in R$ the paths to two half-polygons have the same length and the the same first segment $xv$),
then we may eliminate one of them.
We do not need that exception here because we assumed that no vertex $v$ is equidistant from two or more edges (Assumption~\ref{assumption:unique_farthest_neighbor}).
}

We use a coarse cover of a chord in Phase I of our algorithm.  The chord oracle is used in Phase II. In that situation, we have a coarse cover of the polygon . . . 
where
we have a coarse cover of the polygon and are searching for a point in the polygon that minimizes the upper envelope of the coarse cover functions.  We then make calls to the chord oracle as we use
divide-and-conquer to restrict the search space to a subpolygon and
eliminate elements of the coarse cover.
As noted by Ahn et al., 
a coarse cover $\cal T$ of the polygon provides a coarse cover ${\cal T}_K$ of any chord $K$, simply by intersecting the triangles of $\cal T$ with $K$.
Substituting this for step (1) of the chord oracle 
eliminates the dependence on $n$ when the chord oracle is called during the divide-and-conquer step.
This idea is fleshed out in
\anna{Fix ref.} Section~\ref{sec:geodesic-oracle}, where we describe the type of subpolygons we recurse on, and what it means to have a coarse cover of such a subpolygon. We also generalize the chord oracle to a \emph{geodesic oracle} that tests which side of a geodesic path contains the center of the polygon.
}


