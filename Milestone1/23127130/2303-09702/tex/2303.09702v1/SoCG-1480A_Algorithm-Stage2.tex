\subsection{Stage 2: Algorithm for $Q$ a Triangle}
\label{section:constantQ}
\label{section:constantQ_Q}


In this section we 
\newchanged{outline the} algorithm to solve a subproblem for a
subpolygon $Q$ with $|Q| \le 6$ 
and its associated sets ${\cal T}(Q)$ and ${\cal K}_T(Q)$. 
Some of the triangles of  ${\cal T}(Q)$ may contain $Q$. 
We can triangulate $Q$ in constant time and apply the chord oracle to determine which triangle contains the center. Thus we will assume that $Q$ is a triangle.  

We must find the point that minimizes the upper envelope of the functions of the coarse cover ${\cal T}(Q)$.  
We crucially use the properties that the upper envelope is a geodesically convex function (Lemma~\ref{lem:geodesically-convex}) and that $Q$ is convex---together these imply that the upper envelope is a convex function.
We use a Megiddo-style prune-and-search technique, following the same approach as Ahn et al.~\cite[Section7]{linear_time_geodesic} but 
modified to deal with the edge center rather than the vertex center.



Each triangle $T$ of the coarse cover is the domain of a distance function to some edge $e$ of $P$.
Definition~\ref{defn:coarse-cover} tells us that functions associated with coarse cover elements have two different forms. Accordingly, we partition ${\cal T} (Q)$ into:
\begin{enumerate}
    \squeezelist
    \item \textbf{${\cal T}_1$}: Coarse cover elements whose associated functions have the form $d_2(x,v)+ \kappa$, where $v$ is a polygon vertex and $\kappa$ is a constant.
    \item \textbf{${\cal T}_2$}: Coarse cover elements whose associated functions have the form $d_2(x,{\bar e})$, where $\bar e$ is the line through polygon edge $e$.
\end{enumerate}



To determine the edge center, we must locate a point $x = (x_1, x_2)$ and a value $\rho$ to solve the following


\begin{equation}\label{program:optimization_floating}
\begin{array}{ll@{}ll}
\text{minimize}  & \rho &\\
\text{subject to}&
x \in Q\\
&d_2(x,v) + \kappa \le \rho \ \ \ &
\text{$x \in T \cap Q$; $v$, $\kappa$, and $T$
from 
an element of ${\cal T}_1$}\\
&d_2(x,{\bar e}) \le \rho & 
\text{$x \in T \cap Q$; $\bar e$ and $T$
from 
an element of  ${\cal T}_2$}\\
\end{array}
\end{equation}


We show how to solve 
Problem (\ref{program:optimization_floating}) 
\newchanged{in linear time} when the upper envelope of the coarse cover functions is convex.
(Without this condition 
the problem becomes hard since we then have unrelated convex constraints defined on different subdomains $T$.)



The constraints corresponding to ${\cal T}_1$ will be referred to as \textit{disk constraints}.
The constraints corresponding to ${\cal T}_2$ will be referred to as \textit{half-plane constraints}.
\newchanged{Ahn et al.~\cite[Section 7]{linear_time_geodesic} solve Problem (\ref{program:optimization_floating}) when there are no half-plane constraints.
Following their approach, we first describe previous work that handles the case when all triangles of the coarse cover contain $Q$.}





\subparagraph*{Special Case: All Triangles Contain $Q$.}
\newchanged{Note that in this case there is no need to assume that the upper envelope of the coarse cover functions is convex, since this}
follows immediately from the fact that each constraint
is convex on $Q$.
\begin{enumerate}
\squeezelist

\item \label{halfplane_constraints} 
Suppose all the constraints are half-plane constraints.
In this case, the problem 
is simply
linear programming in fixed
dimension 
which was solved in linear time by Megiddo~\cite{megiddo_linear} and Dyer~\cite{dyer1984linear}.
The idea 
is to pair up the lines that define the half-planes, and compute the angle bisector of each pair.
Knowing which side of the bisector contains the optimum point allows us to restrict the domain and discard one of the two constraints.
Find an appropriately-sized cutting of the 
bisectors.
\nnewchanged{
If we find which simplex of the cutting contains the optimum point, we can discard a constant fraction of the constraints.
The simplex can be found using an ``oracle'' that finds the optimum restricted to a line, i.e., in one lower dimension, 
and then testing whether this solution is the  global optimum, and if not, finding which side of the line contains the optimum.} 
The ``oracle'' on a line uses the prune-and-search technique applied repeatedly to the median point.

    \item \label{disk_constraints} 
    Suppose all the constraints are disk constraints.
This special case was also solved by Megiddo~\cite{megiddo_spanned_ball} and the solution was used in the geodesic center algorithm of Pollack et al.~\cite{pollack_sharir}.
The idea is again to pair up the constraints.
Although the constraints are non-linear, Megiddo showed that in the three-dimensional space of $x_1,x_2,\rho$, the locus of points where two constraints are equally tight is a plane that acts as the bisector between the two constraints.
\nnewchanged{The methods used to solve linear programming in three dimensions can then be applied to solve the problem in linear time.}

\item Finally, suppose there are both half-plane and disk constraints.
A linear-time algorithm for this case is given by Lubiw and Naredla~\cite{lubiw2021visibility} in their solution of the visibility center problem.
\newchanged{The idea is to pair up the half-plane constraints and separately pair up the disk constraints. After computing the bisector of each pair, the}
prune-and-search approach described above will %
prune away a constant fraction of 
both 
types of constraints in linear time.
\end{enumerate}

\subparagraph*{General Case.}
\newchanged{The new complication is that}
each constraint applies only in a triangular subdomain.
The idea for the solution one dimension down 
(with interval subdomains on a line) comes from the linear-time chord oracle of Pollack et al.~\cite{pollack_sharir}.  This was extended by Ahn et al.~\cite{linear_time_geodesic} to two dimensions. 
\newchanged{They dealt only with disk constraints, but we can extend the approach to handle both disk constraints and half-plane constraints, by pairing each constraint with another of the same type.

We outline the approach of Ahn et al.~\cite[Section 7.1]{linear_time_geodesic}.}
The basic idea is to add the subdomain boundary lines to the set of bisectors.
\newchanged{Each triangle of the coarse cover is bounded by two chords of $P$.
A pair of constraints (of the same type) then involves five linear constraints (a ``plane-set''): two for each triangular subdomain plus one bisecting plane.
Using cuttings and a ``side-decision'' algorithm
\nnewchanged{(which Megiddo called an ``oracle'')} 
we can in linear time restrict our search to a constant sized convex region $Q'\subseteq Q$ 
such that some constant fraction of the pairs of constraints have the property that no member of their plane-set
intersects $Q'$.
The claim is that at least one of each such pair can be eliminated. 
If $Q'$ is outside either of the two triangular domains, then the corresponding constraint is irrelevant
Otherwise, $Q'$ is inside both the domains.
In this case, we use the fact that it lies on one side of the bisector plane.
One constraint dominates over the other on this side of the bisector plane, and the other one may be ignored.
The last remaining ingredient is   
the ``side-decision'' algorithm which involves solving Problem~(\ref{program:optimization_floating}) 
restricted to a plane---this is the same problem down a dimension---and then testing
whether this solution is a local (hence global) solution and if not, finding which side of the plane contains the optimum.  
}




This completes the outline for solving Problem~\ref{program:optimization_floating} in linear-time.


