\section{Introduction}\label{section:introduction}

The most basic ``center'' problem is Sylvester's problem: given $n$ points in the plane, find the smallest disc that encloses the points. The center of this disc is a point that minimizes the maximum distance to any of the given points.
We consider a center problem that differs in two ways from Sylvester's problem. First, the domain is a simple polygon and the distance measure is not Euclidean distance, but rather the shortest path, or ``geodesic'' distance inside the polygon.
Second, the sites are not points but rather the edges of the polygon.  More precisely, the problem is to find, given a simple polygon in the plane, the \defn{geodesic edge center}, which is a point in the polygon that minimizes the maximum geodesic distance to a polygon edge. 
See Figure~\ref{fig:center-example}.
More formally, let $E$ be the set of edges of the polygon $P$, and for point $p \in P$ and edge $e \in E$, define \defn{$d(p,e)$} to be the geodesic distance from $p$ to $e$.  
\changed{Define the \defn{geodesic radius} of a point $p \in P$ to be $r(p) := \max \{d(p,e): e \in E\}$.
Then the \defn{geodesic edge center} is a point $p \in P$ that minimizes
$r(p)$.}

\begin{figure}[htb]
  \centering
  \includegraphics[width=.65\textwidth]{RS_figures/EdgeCenter-cropped.pdf}
  \caption{Point $c$ is the edge center of this polygon.
  Edges $e, e', e''$ (in blue)
  are geodesically farthest
  from $c$---the geodesic paths (in red) from $c$ to these edges all
have the same length.}
\label{fig:good_four_cell}
\label{fig:center-example}
\end{figure}

Our main result is a linear-time algorithm to find the geodesic edge center of a simple $n$-vertex polygon.  This improves the previous $O(n \log n)$ time algorithm by Lubiw and Naredla~\cite{lubiw2021visibility}.
The algorithm follows the strategies used to find the geodesic \emph{vertex} center, which is a point in the polygon that minimizes the maximum geodesic distance to a polygon vertex.  
In 1989, Pollack, Sharir and Rote~\cite{pollack_sharir} gave an $O(n \log n)$ time algorithm for the geodesic vertex center problem.  A main tool---which is used in all subsequent algorithms---is
a linear time \emph{chord oracle} that finds, given a chord, which side of the chord contains the center. 
In 2016, Ahn, Barba, Bose, De Carufel, Korman, and Oh~\cite{linear_time_geodesic} improved the runtime for the geodesic vertex center to $O(n)$.  
Their most important new contribution is the use of $\epsilon$-nets to perform a divide-and-conquer search. 
Our algorithm follows the approach of Ahn et al., modified to deal with farthest edges rather than farthest vertices.  We simplify some aspects and we repair some %
errors in their approach.  
\newchanged{The edge center problem is more general than the vertex center problem via the reduction of splitting each vertex into  two vertices joined by a very short edge.}
\remove{Our result is more general, since farthest vertices can be converted to farthest edges by 
adding a small edge at each vertex.}


In general, the center of a set of sites can be determined from the farthest Voronoi diagram of those sites, but computing the Voronoi diagram can be more costly. 
As the first step of 
our center algorithm we give a \reviewerchange{linear-time} algorithm to compute the \reviewerchange{geodesic farthest-edge} Voronoi diagram restricted to the boundary of the polygon. 
\changed{Computing the whole \reviewerchange{geodesic farthest-edge} Voronoi diagram in linear time is an open problem.}


\remove{
\attention{I think the next paragraph must either be removed or modified! Khramtcova and Papadopoulou don't note anywhere the case for a convex polygon but it works in linear expected time. We can get the expected time part down to output sensitive linear + use aurenhammer et al. to get a deterministic \reviewerchange{linear-time algorithm} when the size of the Voronoi diagram is $O(\frac{n}{\log n})$.}
\anna{OK, we should remove this part.  Also, I don't see `output sensitive linear time' since it takes $O(n)$ to compute the Voronoi diagram on $\partial P$---you can't ignore that runtime.  I think we should just drop the whole thing.}
\anurag{removed.}
}





\subparagraph*{Background on Centers and Farthest Voronoi Diagrams.} 

\changed{
Megiddo~\cite{megiddo_linear} gave a linear time  algorithm to find the center of a set of points in the plane (Sylvester's problem) using his ``prune-and-search'' technique, which is used in the final stages of all geodesic center algorithms.
However, computing the farthest Voronoi diagram of points in the plane takes $\Theta(n \log n)$ time~\cite{shamos1975closest}.
}



\remove{
The center of a set of points in the plane (Sylvester's problem) 
can be determined from the farthest point Voronoi diagram---the center is a vertex of the Voronoi diagram unless the center has only two farthest points, in which case it lies on an edge of the Voronoi diagram.
Shamos and Hoey~\cite{shamos1975closest} gave an $O(n \log n)$ time algorithm to find the farthest point Voronoi diagram and the center. 
There is an $\Omega(n \log n)$ lower bound for finding Voronoi diagrams of points in the plane, but the center problem is strictly easier---Megiddo~\cite{megiddo_linear} used his ``prune-and-search'' technique to give a \reviewerchange{linear-time algorithm} to find the center of a set of points in the plane. 
It turns out that all algorithms for geodesic center problems end up using Megiddo's prune-and-search techniques in the last stages.  
}

Our problem involves distances that are geodesic rather that Euclidean, and sites that are segments (edges) rather than points. \changed{These have been studied separately, although there is almost no work
combining them.}

For Euclidean distances, 
Megiddo's method extends to \reviewerchange{linear-time algorithm}s to find the center of %
line segments or lines in the plane~\cite{bhattacharya1994optimal}.
The farthest Voronoi diagram of segments in the plane was considered by Aurenhammer et al.~\cite{aurenhammer2006farthest},
who called it a
``stepchild in
the vast Voronoi diagram literature''.  
They gave an $O(n \log n)$ time algorithm which was improved to output-sensitive time $O(n \log h)$, where $h$ is the number of faces of the diagram~\cite{papadopoulou2013farthest}.

\remove{
There are algorithms 
to find the Euclidean center of more general sites such as convex polygons~\cite{jadhav1996optimal}, and to find the smallest disc that \emph{contains} a set of discs~\cite{megiddo_spanned_ball} which can be viewed as a center problem for points with additive weights, and is one of the ingredients in the algorithms for finding the vertex center of a polygon.
With regard to Voronoi diagrams, 
there is an algorithm to find 
the farthest Voronoi diagram of convex polygons in the plane~\cite{cheong2011farthest},
and more generalizations can be found in 
the book by Aurenhammer et al.~\cite{aurenhammer2013voronoi}.
}



For geodesic 
distances with point sites Ahn et al. gave 
a \reviewerchange{linear-time algorithm} to find the geodesic center of the vertices of a polygon~\cite{linear_time_geodesic}.
The corresponding farthest Voronoi diagram 
can be found in time $O(n \log \log n)$~\cite{oh2020geodesic}, and in expected linear time~\cite{barba2019optimal}. 
More generally, for $m$ points inside an $n$-vertex polygon, an algorithm to find their farthest Voronoi diagram 
was first given by Aronov et al.~\cite{aronov1993farthest} with run-time $O((n+m)\log (n+m))$, and improved in a sequence of papers~\cite{oh2020geodesic,barba2019optimal,oh2020voronoi}, culminating in an optimal run time of   
$O(n + m \log m)$~\cite{wang2021optimal}. \changed{This is also the best-known bound for finding the center of $m$ points in a simple polygon.}
For sites more general than point sites inside a polygon, the only result we are aware of is %
an $O((n+m) \log (n+m))$ time algorithm to find the geodesic center of $m$ \emph{half-polygons}~\cite{lubiw2021visibility}, 
with edges being a special case. 

Finally, we mention a curious difference between nearest and farthest site Voronoi diagrams of edges
in a polygon.
The nearest Voronoi diagram of the edges of a polygon is the medial axis, 
one of the most famous and useful Voronoi diagrams.  The medial axis can be found in linear time~\cite{chin1999finding}.
By contrast,
the \emph{farthest} Voronoi diagram of edges in a polygon has received virtually no attention,
\changed{except for a convex polygon (which avoids geodesic issues)
where there is an $O(n \log n)$ time algorithm~\cite{drysdale2008nlogn}, and a recent 
expected \reviewerchange{linear-time algorithm}~\cite{khramtcova2014expected}.
}

\remove{
\subparagraph*{Techniques and Our Contributions.}
Our algorithm to find the geodesic edge center of a polygon has two phases.  In the first phase we find the \reviewerchange{geodesic farthest-edge} Voronoi diagram restricted to the polygon boundary.  In the second phase we reduce the problem to finding a point in the polygon that minimizes the upper envelope of a linear number of easy-to-compute
\changed{convex} functions each defined on a triangle inside the polygon. 
That problem is then solved using a divide-and-conquer algorithm based on $\epsilon$-nets
\changed{and Megiddo's prune and search techniques}.  
A more detailed overview of our algorithm can be found in Section~\ref{sec:overview}, but since our algorithm follows and builds upon a number of previous results, we first summarize the previous results
\changed{(in historical order)}
and explain what is novel about our contributions.

\begin{enumerate}

\item \changed{Meggido, 1983~\cite{megiddo_linear}, Dyer, 1984~\cite{dyer1984linear}.  A \reviewerchange{linear-time algorithm} for linear programming in two and three dimensions.
There are two ideas from these famous prune-and-search algorithms that are used in  algorithms for geodesic centers, including ours. The first idea is to
make repeated calls to an ``oracle'' that solves the problem one dimension down.
The second idea is to eliminate constraints by pairing them up and considering the ``bisecting'' plane or line where the two constraints are equal. 
The oracle can be used to determine which 
side of the bisector contains the optimum solution, thus eliminating one 
of the two constraints.
Using techniques 
now known as cuttings or epsilon-nets, a small number of bisectors can be tested to 
eliminate a large number of constraints, resulting in a \reviewerchange{linear-time algorithm}.
}

\item Pollack, Sharir, Rote, 1989~\cite{pollack_sharir}.  An $O(n \log n)$ time algorithm to find the geodesic vertex center of a simple polygon. 
\changed{As for the linear programming algorithms described above, a main ingredient is to solve the problem one dimension down.
In particular, they develop an $O(n)$ time 
``chord oracle''
that, given
a chord of the polygon, finds the \emph{relative center} restricted to the chord
and from that, determines whether the 
center of the polygon lies to left or right of the chord.  
By applying the chord oracle $O(\log n)$ times, they limit the 
search to a subpolygon where Euclidean distances can be used.
This reduces the problem to finding 
a minimum disc that encloses some disks, which Megiddo~\cite{megiddo_spanned_ball} solved in linear time using the same approach as described above for linear programming.
}


Lubiw and Naredla~\cite{lubiw2021visibility} redid the chord oracle to handle farthest \emph{edges} instead of vertices. 

The idea used in the chord oracle algorithm is central to further developments. Expressed in general terms, 
the goal is to find a point in a domain (either a chord or the whole polygon) that minimizes the maximum distance to a %
site (a vertex or edge of the polygon). The idea is to first find what we will call a \emph{coarse cover} of the domain by a linear number of elementary regions $R$ (intervals or triangles) with an easy-to-compute \changed{convex} function $f_R$ defined on each region $R$, such that for any point $x$ in the domain, the maximum distance from $x$ to a site is the maximum of $f_R(x)$ over regions $R$ containing $x$.  Thus, the goal is to find the point $x$ that minimizes the upper envelope of the convex functions.
\changed{When the domain is a chord, the 
chord oracle finds the point $x$ in linear time using Megiddo's technique of pairing the constraints (i.e., the functions) in order to prune away a fraction of them in each round. 
When the domain is the whole polygon, 
the innovation of Ahn et al.~described in (4) below is to use similar ideas, together with $\epsilon$-net techniques, to obtain a \reviewerchange{linear-time algorithm}. 
However, finding an appropriate coarse cover of the polygon 
in linear time depends on another breakthrough, which we discuss next.
}

\item Hershberger, Suri, 1997~\cite{hershberger1997matrix}. A \reviewerchange{linear-time algorithm} to find the farthest vertex from each vertex of a polygon. This important result is 
the cornerstone of \reviewerchange{linear-time algorithm}s
{to find the geodesic diameter and center of a polygon}. 
The problem is transformed to finding row maxima in a matrix of distances between pairs of vertices.
The resulting matrix is ``totally monotone'' so the row maxima can be found with a linear number of matrix accesses using the algorithm of 
Aggarwal et al.~\cite{aggarwal1989linear}.  
Hershberger and Suri show
how to access the required distance matrix entries in amortized constant time each.

We show that Hershberger and Suri's algorithm extends to finding the farthest \emph{edge} from each vertex.

\item Ahn, Barba, Bose, De Carufel, Korman, Oh, 2016~\cite{linear_time_geodesic}.  A \reviewerchange{linear-time algorithm} to find the geodesic vertex center of a simple polygon.  
Using the algorithm of Hershberger and Suri as a starting point, they find a coarse cover of the whole polygon.  
They then use divide-and-conquer based on $\epsilon$-nets to reduce the domain to a triangle. After that, the vertex center is found using  
Megiddo-style prune-and-search techniques as described in (1) above.

Our algorithm uses a similar approach. 
One difference is that 
we give a simpler method of finding a coarse cover of the polygon by first finding 
the \reviewerchange{geodesic farthest-edge} Voronoi diagram on the polygon boundary.

Other differences are introduced in order to 
repair 
some flaws in the algorithm of Ahn et al. 
They 
recurse on subpolygons called ``$4$-cells'' that are the intersection of four half-polygons (a half-polygon is the subpolygon to one side of a chord).
However, 
the resulting range space does not in fact 
not 
have the necessary properties for finding $\epsilon$-nets.  
In addition, their step of partitioning a larger cell into $4$-cells needs repairing.  

We remedy these issues by recursing on
subpolygons that are the intersection of three 
\emph{geodesic} half-polygons, i.e., instead of cutting off the region to one side of a chord, we cut off the region to one side of a geodesic path joining two points on the boundary of the polygon.
We can then prove the requisite $\epsilon$-net properties, and we can partition larger cells into our subpolygons. 
Our approach works both for the geodesic edge center and for the geodesic vertex center, thus repairing the result of Ahn et al.
For more details, see
Section~\ref{sec:phase-II-overview}.

\item Oh, Barba, Ahn, 2020~\cite{oh2020geodesic}. This paper uses the coarse cover of the polygon from (4) above, and finds the farthest vertex Voronoi diagram inside a polygon, although not in linear time. 
Of relevance here is their first step, which is a \reviewerchange{linear-time algorithm} to find the Voronoi diagram restricted to the polygon boundary. 
We reverse the order of dependence of (4) and (5).
First, 
using the Hershberger-Suri result (3) \changed{and the coarse cover of an edge from (6) below}, we  give a considerably simpler \reviewerchange{linear-time algorithm} to find the farthest edge/vertex Voronoi diagram restricted to the polygon boundary---a problem of independent interest.  
After that, we use the boundary Voronoi diagram to produce a coarse cover of the polygon and then proceed to find the geodesic edge center.


\item Naredla and Lubiw~\cite{lubiw2021visibility}. An $O(n \log n)$ time algorithm to find the edge center of a simple polygon, based on the algorithm of Pollack et al.~described in (1) above.  In fact, the algorithm solves the more general problem of finding the center of a set of $O(n)$ half-polygons (edges being the special case where the half-polygons hug the polygon boundary).
One ingredient that we will resuse in the present paper is a linear time chord oracle for the edge center, including finding a coarse cover of a chord.

\end{enumerate}
} %


