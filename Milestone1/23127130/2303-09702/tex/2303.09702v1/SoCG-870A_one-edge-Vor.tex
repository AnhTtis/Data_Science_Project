\subsection{Details for Section~\ref{sec:one-edge-Vor}, Farthest Edge Voronoi Diagram on One Edge}
\label{appendix:ab-course-cover}
\label{appendix:one-edge-Vor}

We need some more results to 
prove Lemma~\ref{lem:consecutive-coarse-cover}. 
First we need more details of the construction of the coarse cover~\cite{lubiw2021visibility}.
Define \defn{$p_a(u)$} and \defn{$p_b(u)$} to be the parents of node $u$ in $T_a$ and $T_b$, respectively.
As noted by 
Pollack et al.~\cite{pollack_sharir}, a vertex $u$ is visible from some point on $ab$
if and only if $p_a(u) \ne p_b(u)$. 
If $u$ is visible from some point on $ab$, then extending the edge from $u$ through $p_a(u)$ reaches a point
\defn{$x_a(u)$}
on $ab$ from which $u$ is visible.
Similarly, extending the edge from $u$ through $p_b(u)$ reaches a point
\defn{$x_b(u)$}
on $ab$ from which $u$ is visible.
According to the definition \reviewerchange{in\cite{lubiw2021visibility}}, if
edge $uv$ of $T_a$
has an associated 
$a$-side coarse cover element $(I,f,e)$, then 
$I=[x_a(u), x_a(v)]$.
Similarly for $b$-side elements.
\reviewerchange{If
edge $uv$ of $T_a \cap T_b$
has an associated 
central triangle coarse cover element $(I,f,e)$, then 
$I=[x_a(u), x_b(u)]$.
And if polygon edge $e$ has an associated central trapezoid coarse cover element $(I,f,e)$, then $I$ consists of the points of $ab$ whose shortest paths to $e$ arrive perpendicularly, and with the added $0$-length edges this is
$I = [x_a(t(a,e)),x_b(t(b,e))]$.
}

\begin{obs}
\label{obs:consecutive-coarse-cover}
If $uv$ and $vw$ are edges of $T_a$ that have associated coarse cover elements $(I_1, f,e)$ and $(I_2,f',e')$, then
the right endpoint of $I_1$ is $x_a(v)$ and the left endpoint of $I_2$ is $x_a(v)$, i.e., 
\reviewerchange{$I_1$ and $I_2$ appear in that order along $ab$ and intersect in a single point.  
This observation is also true for an edge $uv$ and a central trapezoid at $v$ if $v$ happens to be a leaf.}
A similar property holds for $T_b$.
\end{obs}

\begin{lem}
\label{lem:coarse-cover-path}
Suppose edge $uv$ of $T_a$ has an associated coarse cover element
\changed{$(I,f,e)$ for polygon edge $e$.}
Then:

\begin{enumerate}
\item 
On the path $\pi(a,u)$ all edges except the first one have associated coarse cover elements.

\item On the path $\pi(v,e)$ let $x$ be the last vertex visible from $ab$.  All edges on $\pi(v,x)$ have associated coarse cover elements for the polygon edge $e$.  Furthermore, if $x$ is a leaf then there is a central trapezoid associated with $e$, and otherwise there is an edge $xy$ in $\pi(v,e)$ and it is associated with a central triangle for $e$.
\end{enumerate}
A similar property holds for $T_b$.
\end{lem}

\begin{proof}
The first statement just depends on the fact that if $u$ is visible from $ab$ (i.e., has different parents in $T_a$ and $T_b$) then the same is true for every vertex on the path $\pi(a,u)$.

For the second statement, note that $e$ is the farthest edge from $v$ in the subtree of $v$.
Let $w$ be any vertex on $\pi(v,x)$, and $e'$ any polygon edge for which the terminal of the path $\pi(a,e')$ 
lies in the subtree of $w$.
We have $d(v,e') = d(v,w) + d(w,e')$.
Since $d(v,e) > d(v,e')$, the previous equality implies $d(w, e) > d(w,e')$.
So the farthest edge from $w$ is $e$,
and the coarse cover element for the tree edge joining $w$ to its parent $p_a(w)$ is also associated with $e$.


If $x$ is a leaf, the terminal points 
$t(a,e)$ and $t(b,e)$ 
are distinct (due to the introduction of 0-length segments).
From the definition of the coarse cover elements, this means there is a central trapezoid associated with $e$.
If $x$ is not a leaf, let $xy$ be the first edge on the path $\pi(x,e)$ (which is a subpath of $\pi(v,e)$).
The vertex (or terminal point) $y$ is not visible from $ab$ and $xy$ is a tree edge in both $T_a$ and $T_b$.
The coarse cover element for $xy$ is a central triangle associated with $e$.
\end{proof}

Lemma~\ref{lem:consecutive-coarse-cover} follows immediately from Lemma~\ref{lem:coarse-cover-path}.

\subparagraph*{Details on Constructing Tree $T$.}

\begin{enumerate}
\squeezelist
\item For each central trapezoid coarse cover element, say associated with polygon edge $e$, there is a leaf $l$
of $T$ corresponding to $e$. Attach a new edge in $T$ descending from $l$ and associate the central trapezoid element with it.


\item 
For each polygon edge $e$ that has $b$-side triangle elements associated with it, those triangles 
correspond to a  path $\pi$ in $T_b$, that is directed in leaf-to-root order by Lemma~\ref{lem:consecutive-coarse-cover}.  The tree $T$ currently has an edge, say $g$, associated with the central triangle/trapezoid for $e$.  Attach the path $\pi$ at end of the edge $g$.

\item 
Finally, we  contract any original edge of $T_a$ that is not associated with a coarse cover element.
\changed{These are edges
$uv$ such that $u$ is not visible from $ab$ plus the original edges incident to $a$.
}
\end{enumerate}
