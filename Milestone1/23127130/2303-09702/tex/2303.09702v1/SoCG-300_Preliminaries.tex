\section{Preliminaries}
\label{section:prelims}

Although our algorithm follows the pattern of the geodesic vertex center algorithm by Ahn et al.~\cite{linear_time_geodesic}, we must re-do everything from the ground up to deal with farthest edges. 
In this section we summarize the basic results we need, deferring proofs and details to the appendix.

\remove{
\anna{Give a better intro to this section, describing what each subsection is.}

We discussed a few preliminary concepts required to define the edge center in the Introduction (Section~\ref{section:introduction}).
This section discusses a few other concepts that will be  useful in describing and understanding our algorithm.
}

\subparagraph*{Notation and Definitions.}
\label{sec:notation}
A \defn{chord} of a polygon is a straight line segment in the (closed) polygon with both endpoints on the boundary, $\partial P$. 
For a point $p \in P$ and a point or line segment $s$ in $P$, \defn{$\pi(p,s)$} is the unique \defn{shortest} (or \defn{geodesic}) path from $p$ to $s$, and \defn{$t(p,s)$} is its \defn{terminal point}.
The length of $\pi(p,s)$, denoted \defn{$d(p,s)$}, is the \defn{geodesic distance} from $p$ to $s$.
For point $p$ in $P$, a \defn{farthest edge}, \defn{$F(p)$}, is an edge of $P$ that maximizes $d(p,e)$.
The \defn{geodesic radius} of $p$ is \defn{$r(p)$} $:= d(p,F(p))$.
The \defn{geodesic edge center} is a point $p \in P$ that minimizes
$r(p)$.


\remove{
\anna{This definition should take into account that 2D regions can have ties (or else refer to general position assumptions that prevent this).}
The subdivision of the polygon $P$ into regions with the same farthest edge is referred to as the \defn{\reviewerchange{geodesic farthest-edge} Voronoi diagram} and denoted \defn{${\cal V} (P)$}.
The diagram consists of two-
dimensional faces, edges (straight or curved) and vertices.
For brevity, we will sometimes refer to this as simply the `farthest edge Voronoi diagram'.
The restriction of the diagram to the boundary $\partial P$ is denoted ${\cal V} (\partial P)$.
More generally, the restriction of the diagram to a subset $S \subseteq P$, is denoted ${\cal V}(S)$.
For polygon $P$ and edge $e$, define the \defn{farthest Voronoi region of $e$} to be  
$\defn{V(e)} := \{ p \in P : \text{ for any other edge $f$, } d(p,e) \ge d(p,f) \}$.
}


\subparagraph*{General Position Assumptions.}
\label{sec:general-position}
\remove{
\begin{figure}
  \centering
  \includegraphics[width=.8\textwidth]{RS_figures/two_dimensional_bisector-cropped.pdf}
  \caption{
  Vertex $r$ 
  \changed{is equidistant from edges $e$ and $e'$ (colored red), and so is every point in the shaded region.
  The bisector between the two edges is thus two-dimensional.
  Assumption~\ref{assumption:unique_farthest_neighbor} forbids this situation.
  }
\anna{Please change the V between $e$ and $f$ so that $r$ does not lie on the bisector of the tip of the V (that confuses the situation).}
\anurag{changed}
\anna{The previous figure was much more enlightening.  This one is too complicated.  Please go back to the previous figure---just tweak the reflex vertex in the middle.}
  }
\label{fig:twod_voronoi}
\end{figure}
} %
\changed{As is standard for Voronoi diagrams of segments, e.g., see~\cite{aurenhammer2006farthest}, we use the following tie-breaking rule %
to prevent 2-dimensional Voronoi regions with more than one farthest edge.}

\remove{
To simplify the description of our algorithm, we %
\changed{make some} general position assumptions.
\anna{The next sentence is fine for a thesis, but should be skipped for a paper.}
General position assumptions are commonly used in computational geometry to deal with degeneracy in the input.
Previous algorithms on geodesic vertex centers and the corresponding Voronoi diagrams~\cite{linear_time_geodesic,aronov1993farthest,oh2020geodesic,wang2021optimal} 
used assumptions of a similar flavour.
In actual implementations, general position may be achieved by an infinitesimal perturbation of the input %
points~\cite{10.1145/77635.77639}.
Our first assumption is  common and basic.
} %




\smallskip\noindent
{\bf Tie-Breaking Rule.} 
Suppose that $p$ is a point of $P$, $e$ and $f$ are two edges that meet at reflex vertex $u$, and
$\pi(p,e) = \pi(p,f) = \pi(p,u)$. 
\changed{Let line $b$ be the angle bisector of $u$.
For $p$ not on $b$, break the tie $d(p,e)=d(p,f)$
by saying that the distance to the edge on the opposite side of $b$ is greater.}


\medskip
\changed{We make the following general position assumptions, which we claim can be effected by perturbing vertices.}

\changed{
\begin{assumptions}
\label{assumptions}
(1) No three vertices of 
    $P$ are collinear.
(2) After imposing the tie-breaking rule, no vertex is equidistant from two or more edges.
(3) No point on the polygon boundary has more than two farthest edges and no point in the interior of the polygon has more than a constant number of farthest edges.
\end{assumptions}
}
It follows that the set of points with more than one farthest edge 
is 1-dimensional, does not contain any vertex of $P$, and intersects $\partial P$ in isolated points; see Lemma~\ref{lem:1D-Voronoi-edges} in Appendix~\ref{appendix:general-position}.

\remove{
\begin{assumption}\label{assumption:no_three_points_collinear}
    No three vertices of 
    $P$ are collinear.
\end{assumption}



\begin{assumption}\label{assumption:unique_farthest_neighbor}
After imposing the tie-breaking rule, no vertex is equidistant from two or more edges.
\end{assumption}


It follows that the set of points with more than one farthest edge 
is 1-dimensional, does not contain any vertex of $P$, and intersects $\partial P$ in isolated points, see Lemma~\ref{lem:1D-Voronoi-edges} in Appendix~\ref{appendix:general-position}.
We also want Voronoi vertices to be ``nice'':




\begin{assumption}
\label{assumption:Voronoi-vertices}
No point on the polygon boundary has more than two farthest edges. No point in the interior of the polygon has more than a constant number of farthest edges.
\end{assumption}
} %
