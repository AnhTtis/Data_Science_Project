\subsubsection{Extensions of the Chord Oracle}
\label{appendix:geodesic-oracle}


In this section we give two extensions of the chord oracle that we use in 
Phase II.  
The divide and conquer algorithm in Phase II recurses on subpolygons that are \newchanged{simple $3$-anchor hulls.} 
The first extension of the chord oracle is
a version that works on a chord of a subpolygon (a \newchanged{simple $3$-anchor hull}) and uses the coarse cover of the subpolygon to get a coarse cover of the chord.
The runtime will be linear in the size of the subpolygon's coarse cover, which decreases during the divide-and-conquer algorithm.
The second extension is 
a generalization of the chord oracle to a \emph{geodesic oracle}.
\newchanged{A geodesic path divides a polygon into regions, and the geodesic oracle tells us which region contains the center.}
We need this because 
our subpolygons are bounded by geodesics.


Let $Q$ be a 
\newchanged{simple $3$-anchor hull}
in $P$.
By Observation~\ref{obs:3-geo-cell}, $Q$ is geodesically convex in $P$
so the intersection of $Q$ with a chord [geodesic] of $P$ is a chord [geodesic] of $Q$.

We say that a subset ${\cal T}$ of the coarse cover of $P$ is a \defn{coarse cover of $Q$ in $P$} if condition 3 of the definition of a chord cover (Definition~\ref{defn:coarse-cover}) holds for all points in 
the interior of
$Q$, i.e., for any point $x$ in the interior of $Q$
and any edge $e$ of $P$ that is farthest from $x$, there is a triple $(R,f,e)$ in the coarse cover with $x \in R$.
\newchanged{(In particular, we get a coarse cover of $Q$ by  taking all the coarse cover elements whose triangles intersect the interior of $Q$.)}

Recall that each triangle $T$ of the coarse cover of $P$ is bounded by a segment of an edge of $P$ and two chords, and we store the endpoints of the chords on $\partial P$.  
As we recurse on subpolygons $Q$ we will maintain the endpoints of these chords on $\partial Q$.


\begin{lem}
\label{lem:generalized-chord-oracle} 
For the geodesic \reviewerchange{edge center} problem, 
there is an algorithm that
takes as input 
a 
\newchanged{simple $3$-anchor hull}
$Q$ known to contain the center of $P$ in its interior,
a coarse cover $\cal T$ of $Q$ in $P$,
and a chord $K$ of $Q$, 
and decides whether the center lies left/right/on $K$. The runtime is 
$O(|{\cal T}|)$. 
\end{lem}

\begin{proof}
We first construct a coarse cover ${\cal T}_K$ of $K$. 
For each triple $(T,f,e)$
in ${\cal T}$, let $T_K$ be the intersection of triangle $T$ with $K$.
Note that $T_K$ is a subsegment of $K$ and can be found in constant time from the boundary chords of $T$.
Add the triple $(T_K,f,e)$ to ${\cal T}_K$.  The resulting set ${\cal T}_K$ is a coarse cover of $K$ of size at most $O(|{\cal T}|)$.


Next, we follow Steps 2 and 3 of the 
chord oracle algorithm---Step 2 finds the relative center on $K$, and Step 3 decides whether the center lies to the left or right (or on) $K$.
As noted above,
each step takes time $O(|{\cal T}_K|)$.  
\end{proof}

Next we generalize Lemma~\ref{lem:generalized-chord-oracle} to a geodesic path.



\begin{lem}
\label{lem:geodesic-oracle} For the geodesic \reviewerchange{edge center} problem, 
there is an algorithm that
takes as input 
a 
\newchanged{simple $3$-anchor hull} %
$Q$ known to contain the center of $P$
in its interior, a coarse cover $\cal T$ of $Q$ in $P$,
and a geodesic $\gamma = \pi(a,b)$ in $Q$ with $a,b \in \partial Q$, 
\newchanged{and finds which subregion formed by $\gamma$ contains the center, or if the center lies on the geodesic.}
The runtime is 
$O(|Q| + |{\cal T}|)$.
\end{lem}

The idea is similar to that of  Lemma~\ref{lem:generalized-chord-oracle}.  We must first describe how to intersect the triangles of the coarse cover of $Q$ with the geodesic $\gamma$.
Since coarse cover triangles are bounded by chords, we can 
use the following result.

\begin{lem}
\label{lem:intersection-with-geodesic}
There is an algorithm that takes as input a \newchanged{simple} subpolygon $Q$, a geodesic $\gamma = \pi(a,b)$ in $Q$ with $a,b \in \partial Q$, and a set of chords $\cal K$ of $Q$, and finds the intersections of the chords of $\cal K$ with $\gamma$.  Each chord of $\cal K$ is given by its endpoints together with the identity of the edge of $Q$ containing the endpoint.
The runtime is $O(|Q| +  |{\cal K}|)$.
\end{lem}
\begin{proof}
Suppose the 
geodesic $\gamma$ has $g$ segments.  Then it divides the boundary of $Q$ into $g+1$ subchains, and we can traverse $\partial Q$ once to identify, for each edge of $Q$, which subchain contains it.

Direct $\gamma$ from $a$ to $b$.  This also directs the subchains of $Q$.  Identify each segment $s = uv$ of $\gamma$ with the subchain $c_s$ that ends at $v$. 
The subchain $c_s$ is unique except for the last segment incident to 
vertex $b$ where two subchains end---use one of the two subchains and ignore the other one. 

Observe that if chord $K \in {\cal K}$ crosses segment $s$, then $K$ has an endpoint in $c_s$.
Thus 
we can iterate through the chords $K$, finding which subchain contains each endpoint, and testing whether the associated segment of $\gamma$ intersects $K$.

The total time is $O(|Q| + |{\cal K}|)$.
\end{proof}

With Lemma~\ref{lem:intersection-with-geodesic} in hand, we can prove Lemma~\ref{lem:geodesic-oracle}.

\begin{proof}[Proof of Lemma~\ref{lem:geodesic-oracle}]
For each triangle of the coarse cover ${\cal T}$, the endpoints of its defining chords on $\partial Q$ are known.
Denoting the set of these defining chords by ${\cal K}$, we apply Lemma~\ref{lem:intersection-with-geodesic} to determine the intersections of the chords in ${\cal K}$ with $\gamma$. 
This takes $O(|Q| + |{\cal K}|)$ time, or equivalently, $O(|{Q}| + |{\cal T}| )$ time.

From the chord intersections, we can determine the intersections of the triangles of the coarse cover $\cal T$ with the segments of $\gamma$.
Each segment $s$ of $\gamma$ is a chord of $Q$.
Let ${\cal T}_s$ be the coarse cover elements whose triangles intersect the interior of $s$.
Each intersection is an interval of $s$ and the set of these intersections gives a coarse cover of $s$ of size $O(|{\cal T}_s|)$.
The chord oracle of Lemma~\ref{lem:generalized-chord-oracle} then determines whether the edge center lies left/right/on the segment $s$
in time 
$O(|{\cal T}_s|)$. 
Running the algorithm for \textit{all} the segments of $\gamma$ will take $O(|{\cal T}|)$ time in total
because each triangle of $\cal T$ intersects the interior of at most one segment of $\gamma$.
If the center lies on one of the segments, then it lies on $\gamma$.
Otherwise, 
since the segments  partition $Q$ into disjoint regions, knowing which side of each segment  contains the center tells us the region that contains the center.

The algorithm takes $O(|{Q}| + |{\cal T}|)$ time.
\end{proof}


