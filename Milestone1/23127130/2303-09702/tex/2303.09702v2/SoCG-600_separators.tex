\subparagraph*{Separators and Funnels.}
\label{sec:separators}



A geodesic path between two vertices of $P$ separates $\partial P$ into two parts, and when we focus on which vertices/edges are in opposite parts,
we call the geodesic path 
a ``separator''.
Separators, first introduced by Suri~\cite{suri1989}, are a
main tool for finding all farthest vertices in a polygon. 
In Appendix~\ref{appendix:separators} we extend the %
basic properties of separators to the case of farthest \emph{edges} and prove:
(1) 
If vertex $v$ and edge $e$
are separated by a geodesic path $\pi(a,b)$, then the shortest path from $v$ to $e$ is contained, except for one edge, in the shortest path trees of $a$ and $b$;
(2) A constant number of separators suffice to separate every vertex from its farthest edge.
