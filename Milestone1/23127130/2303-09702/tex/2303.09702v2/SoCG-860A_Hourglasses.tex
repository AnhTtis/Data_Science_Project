\subsection{
Hourglasses}
\label{appendix:hourglasses}

\changed{In this section we show that to find the Voronoi diagram on a transition edge $ab$ it suffices to look at the \emph{hourglass} of $ab$, and we show that all the hourglasses can be found in linear time, and the sum of their sizes is linear.}

Hourglasses were first used in algorithms for shortest paths~\cite{SPT_linear,GUIBAS1989126,chazelle1989visibility}, and then used in algorithms to find the farthest vertex geodesic Voronoi diagram (Aronov et al.~\cite{aronov1993farthest}) and in algorithms to find the geodesic [vertex] center (Ahn et al.~\cite{linear_time_geodesic}).


\begin{figure}
\centering
\includegraphics[width=0.6\textwidth]{RS_figures/hourglass_ab-cropped.pdf}
\caption{
The hourglass for a transition edge $ab$.
The walls $\pi(a,F(b))$ and $\pi(b,F(a))$ are represented by dashed polylines.
}
\label{fig:hourglass}
\end{figure}

Let $ab$
be 
a transition edge directed counterclockwise.
Note that
$\pi(a,F(a))$ and $\pi(b,F(b))$ cross each other by
the Ordering Property~\ref{prop:anti-parallel} of Lemma~\ref{lem:ordering-properties}. 
The \defn{hourglass} $H(a,b)=H(e)$ is
the subpolygon of $P$ bounded by $ab$, $\pi(a,F(b))$, $\pi(b,F(a))$ and the clockwise portion of $\partial P$ between the terminals $t(a,F(b))$ and $t(b,F(a))$. See Figure~\ref{fig:hourglass}. 
Recall that: every vertex of $P$ has a unique farthest edge by Assumption~\ref{assumption:unique_farthest_neighbor}; and    $t(a,F(b))$ is the terminal point of the geodesic path from vertex $a$ to the edge $F(b)$.
The geodesics $\pi(a,F(b))$ and $\pi(b,F(a))$  are called the \defn{walls} of the hourglass $H(a,b)$, and the part of $\partial P$ clockwise from $t(b,F(a))$ to $t(a,F(b))$ is called the \defn{chain} of the hourglass.
The \defn{size} of an hourglass is its number of vertices.

The following lemma justifies restricting attention to the hourglass of a transition edge $ab$ in order to find the farthest edge Voronoi diagram restricted to $ab$.
It is a consequence of the Ordering Property from Lemma~\ref{lem:ordering-properties}.

\begin{lem}\label{lemma:hourglasses_have_necessary_information}
Let $p$ be a point on the transition edge $ab$ and  
let $e$ be a farthest edge from $p$ in $P$.  Then $e$ lies in the chain of the hourglass $H(a,b)$.
\end{lem}
\begin{proof}
Suppose $e$ lies in the clockwise chain of $\partial P$ from $F(a)$ to $b$. Then the clockwise ordering around $\partial P$ is $p,a,F(a),e$ in contradiction to the Ordering Property (Property~\ref{prop:anti-parallel} of Lemma~\ref{lem:ordering-properties}).  Similarly, if $e$ lies in the clockwise chain $\partial P$ from $a$ to $F(b)$, the clockwise ordering  $b,p,e, F(b)$ contradicts the Ordering Property.   
\end{proof}



Let $\cal H$ be the set of  hourglasses of all the transition edges of $P$. 
In the remainder of this subsection we show how to find 
$\cal H$ in linear time.

\begin{lem}\label{Total_Size_of_All_Hourglasses_Lemma}
All the hourglasses of $\mathcal{H}$ can be constructed in $O(n)$ time.
In particular, the sum of 
their sizes
is $O(n)$.
\end{lem}

\begin{proof}
Recall that by Lemma~\ref{lem:constant_splits} there are five farthest edge separators such that for every edge $ab$, one of the separators has $a$ and $b$ to its right and $F(a)$ and $F(b)$ to its left.
Let $\gamma = \pi(p,q)$ be a farthest edge separator and let 
${\cal H}_\gamma$ be the set of hourglasses of transition edges that lie to the right of $\gamma$.
It suffices to 
prove the lemma for one set ${\cal H}_\gamma$.
Each hourglass in $\mathcal{H}_{\gamma}$ consists of a transition edge $ab$ to the right of $\gamma$, two walls, and a chain to the left of $\gamma$.
In 
Appendix~\ref{appendix:Hershberger-Suri}
we found the farthest edge from each vertex in linear time, 
so we 
know $F(a)$ and $F(b)$.
Because the hourglass chains are internally disjoint, we just need to show that we can find all the walls of the hourglasses in $\mathcal{H}_{\gamma}$ in linear time.  

Each wall is a shortest path from a vertex to the right of the separator $\gamma=\pi(p,q)$ to an edge to the left of $\gamma$, so by Lemma~\ref{lem:separator-paths} each wall consists of edges of  the shortest path trees $T_p$ and $T_q$, except for at most one edge crossing $\gamma$. 
The set of crossing edges has size $O(n)$ because there are $O(n)$ hourglasses. 
By Lemma~\ref{SPT_lemma} the shortest path trees $T_p$ and $T_q$ can be found in 
time $O(n)$ and have size $O(n)$.
By Lemma~\ref{lemma:walls_can_be_constructed_fast} we can find each wall in time proportional to the size of the wall. 
Thus we can find all the walls in time $O(n)$ so long as we show that each polygon chord is in $O(1)$ walls.
(Note that walls of hourglasses are not paths to farthest edges, so we cannot simply apply Property~\ref{prop:no-shared-chord} that crossing paths to farthest edge do not share directed chords.)


\begin{claim}
Any chord of the polygon is in $O(1)$ walls of hourglasses of $\mathcal{H}_{\gamma}$. 
\label{claim:hourglass-wall}
\end{claim}

\begin{proof}
Let $t_1, \ldots, t_k$ be the transition edges to the right of $\gamma$ in clockwise order. 
If two transition edges are close together in this ordering, then their walls may have common chords, but we will show that if $t_i$ and $t_j$ are separated by at least three transition edges, i.e., 
$j-i \ge 4$, then the walls of the hourglasses $H(t_i)$ and $H(t_j)$ have no common chords.  Note that this proves the Claim.

So, consider $t_i$ and $t_j$ with $j-i \ge 4$, and suppose for a contradiction that a wall of $H(t_i)$ and a wall of $H(t_j)$ share a common chord $f$.  Take the intermediate transition edge $t_k = (u,v)$, where $k = i+2$.  Then the farthest edges of the endpoints of $t_i, t_j$ and $t_k$ are all distinct (this is the reason for choosing $i,j,k$ as we did), and all lie to the right of $\gamma$. 
We will show that the paths $\pi(u,F(u))$ and $\pi(v,F(v))$ also share the chord $f$.  This means that we have crossing paths to distinct farthest edges and the paths share a chord, which contradicts Property~\ref{prop:no-shared-chord} from Lemma~\ref{lem:ordering-properties}.

\begin{figure}[ht]
\centering
\includegraphics[width=.75\textwidth]{RS_figures/sandwiched-cropped.pdf}
\caption{Illustration for Claim~\ref{claim:hourglass-wall}. The walls $\pi(p_1,e_1)$ and $\pi(p_2,e_2)$ share the chord $f = xy$, which forces the path $\pi(u,F(u))$ to also use $f$.
}
\label{fig:shared-walls}
\end{figure}


It remains to show that $\pi(u,F(u))$ uses the chord $f$. (The case of $\pi(v,F(v))$ is exactly the same.)  The idea is that this path is ``squashed'' between the two walls that use $f$.
Suppose that the wall $\pi(p_1, e_1)$ of $H(t_i)$ and the wall $\pi(p_2,e_2)$ of $H(t_j)$ both use chord $f$. 
See Figure~\ref{fig:shared-walls}.
Here $p_1$ and $p_2$ are distinct vertices on the right %
of the separator $\gamma$ and $e_1$ and $e_2$ are distinct edges on the 
left %
of $\gamma$.  Vertex $u$ lies between $p_1$ and $p_2$ in clockwise order, and $F(u)$ lies between $e_1$ and $e_2$ in clockwise order, and all are distinct.  
Let $f = xy$ where $x$ and $y$ are vertices of the polygon.
Because shortest paths to the same destination do not cross, 
the shortest path from $u$ to $y$ goes through $x$.  Similarly, the shortest path from $x$ to the edge $F(u)$ goes through $y$.  The union of these two shortest paths is a geodesic (locally shortest) path from $u$ to $F(u)$ and must therefore be the shortest path from $u$ to $F(u)$.  Thus $\pi(u,F(u))$ uses the edge $f$. 
\end{proof}


This completes the proof of Lemma~\ref{Total_Size_of_All_Hourglasses_Lemma}.
\end{proof}
