\subsection{Details for Section~\ref{section:largeQ}, Stage 1: Algorithm for Large $Q$}
\label{appendix:Algorithm-Stage1}

In this section we give an algorithm to handle a subproblem corresponding to a
subpolygon $Q$ 
\newchanged{(a simple 3-anchor hull)} with $|Q| >6$ 
and its associated sets ${\cal T}(Q)$, ${\cal K}_T(Q)$, and ${\cal K}(Q)$. 
By Lemma~\ref{lem:floating_cells_constant_size}, no triangle of ${\cal T}(Q)$
contains $Q$, so every triangle of ${\cal T}(Q)$ has a chord in ${\cal K}(Q)$.
The algorithm either finds the edge center or reduces to a subproblem with $|Q| \le 6$ which is handled in 
Appendix~\ref{section:constantQ}.
The idea was described in the main text.

\begin{enumerate}
\squeezelist

\item For 
$\epsilon = \frac{1}{80}$,
construct an $\epsilon$-net $N$ for the 
\newchanged{$3$-anchor}
range space
with ground set ${\cal K}(Q)$.  The range space is defined with respect to $3$-anchor hulls of $P$.

\item Compute the arrangement $A$ of the chords $N$
inside $Q$, and
use the Chord Oracle of Lemma~\ref{lem:generalized-chord-oracle} to find the face $F$ of $A$ that contains the edge center.

\item 
Partition  face $F$ into a constant number of
\newchanged{$3$-anchor hulls}
of $P$.

\item
Use the Geodesic Oracle (Lemma~\ref{lem:geodesic-oracle}) to find
\newchanged{which of these $3$-anchor hulls contains the edge center, and to reduce it to a simple $3$-anchor hull $Q'$.}

\item If $|Q'| \le 6$ then test each triangle of ${\cal T}(Q)$ to find ${\cal T}(Q')$ and ${\cal K}_T(Q')$, and switch to Stage 2 in the next subsection. 


\item Otherwise $|Q'| > 6$.
Find ${\cal K}(Q')$  and ${\cal T}(Q')$, and recurse on the subproblem for $Q'$.

\end{enumerate}

We elaborate on these steps and their run-times below, but first we justify that our choice of $\epsilon$ in Step 2 guarantees that the size of the subproblem we recurse on is reduced by a fraction.
Recall that the size of the subproblem for $Q$ is $|Q| + t(Q)$.

\begin{lem}
For 
$\epsilon = \frac{1}{80}$,
if $|Q'| > 6$, then 
$|Q'| + t(Q') \le \frac{1}{2} (|Q| + t(Q))$.
\end{lem}
\begin{proof} 
Since $|Q'| > 6$, no triangle of ${\cal T}(Q')$ contains $Q'$.
Thus, since ${\cal K}(Q') \cap N = \phi$, the defining property of $\epsilon$-nets (equation~\ref{eqn:epsilon-net}), ensures that
$|{\cal K}(Q')| \le \frac{1}{80}  |{\cal K}(Q)|$,
which we relate to the subproblem sizes as follows.
\begin{align*}
|Q'| + t(Q') & \le 9 t(Q') + t(Q') & \text{by Claim~\ref{observation:cell_triangles}}\\
& = 10 t(Q') \le 20 |{\cal K}(Q')| & \text{by Claim~\ref{observation:chords_triangles} (no triangle of ${\cal T}(Q')$ contains $Q'$)}\\
& \le \tfrac{20}{80} |{\cal K}(Q)| = \tfrac{1}{4}  |{\cal K}(Q)|& \text{by the $\epsilon$-net property}\\
& \le \tfrac{1}{2} t(Q) & \text{by Claim~\ref{observation:chords_triangles}}\\
& \le \tfrac{1}{2}(|Q| + t(Q))\\
\end{align*}
\end{proof}




We now fill in more details of the steps of the algorithm, and justify that the runtime is $O(|Q| + t(Q))$.

\smallskip
\noindent{\bf 1. Construct an $\epsilon$-net.}
Lemma~\ref{lem:constant_shattering_dimension}
proves that the
$3$-anchor 
range space has bounded VC-dimension, and
Lemma~\ref{lem:subspace-oracle} proves that a subspace oracle exists.  This implies (see
Lemma~\ref{lem:constant_size_net})
that we can find a constant
sized $\epsilon$-net for this range space in time proportional to the size of the ground set, which is 
$O(| \mathcal{K}(Q) |)$ in our case.
By Claim~\ref{observation:chords_triangles} this is $O(t(Q))$.

\smallskip
\noindent{\bf 2. Compute the arrangement of $A$ in $Q$ and find the face $F$ that contains the edge center.}
Once the constant sized $\epsilon$-net  $N$ is determined, we can construct the arrangement of the chords in $O(|N|^2) = O(1)$ 
time, using the algorithm of Edelsbrunner et al~\cite{edelsbrunner1986constructing}.
Note that we know the endpoints of each chord of $N$ on $\partial Q$.
We run the chord oracle of Lemma~\ref{lem:generalized-chord-oracle} on each chord of $N$ inside polygon $Q$ to determine the face $F$ that contains the edge center (halting if we find the center on one of the chords).
This takes $O(t(Q))$ time for each chord of $N$. 
Since $N$ has constant size, 
this step takes $O(t(Q))$ time.


\smallskip
\noindent{\bf 3. Partition $F$ into 
\newchanged{$3$-anchor hulls.}
}
The boundary of 
$F$ consists of $O(1)$ segments of chords in $N$, $O(1)$ subchains of the geodesics bounding $Q$, and  $O(1)$ subchains of the polygon $P$. Let 
\newchanged{$V = \{v_0, \ldots, v_t \}$ be the points in order around $\partial F$ that join successive segments/subchains}.
Then $V$ has size $O(1)$.  %
Find shortest paths 
$\gamma_i = \pi(v_0, v_i)$, $i = 1, \ldots, t$
in $F$. 
This takes time $O(|F|)$, which is $O(|Q|)$.

Let $\Gamma$ be the set of these $O(1)$ shortest (geodesic) paths. 
Because $F$ is geodesically convex, each shortest path $\gamma_i \in \Gamma$ is a geodesic path in $P$ (the shortest path in $P$ from $v_0$ to $v_i$ lies inside $F$, and thus is equal to $\gamma$).
\newchanged{We claim that
the paths of $\Gamma$ subdivide $F$ into a constant number of 
$3$-anchor hulls (which need not be simple).  If the boundary of $\partial F$ between $v_i$ and $v_{i+1}$, $i = 1, \ldots, t-1$,  is a segment of a chord of $N$ or a subchain of a geodesic bounding $Q$, then take the $3$-anchor hull that is the geodesic hull of the three point anchors $v_0, v_i, v_{i+1}$.  If the boundary of $\partial F$ between $v_i$ and $v_{i+1}$
is a subchain of $\partial P$, then take the $3$-anchor hull that is the geodesic hull of $v_0$ and the polygon chain. 
Finally, if if the boundary of $\partial F$ between $v_0$ and $v_1$ or between $v_t$ and $v_0$ is a subchain of $\partial P$, then take the $3$-anchor hull of the polygon chain. 
}
 




\smallskip
\noindent{\bf 4. 
\newchanged{Find a simple $3$-anchor hull $Q' \subseteq F$ that contains the edge center.}}
Call the Geodesic Oracle (Lemma~\ref{lem:geodesic-oracle}) in $Q$ for each of the $O(1)$ geodesics
of $\Gamma$.
Halt if we find the center on one of the geodesics.
\newchanged{Otherwise, the geodesic oracle tells us which region of the partition by $\Gamma$ contains the edge center in its interior, and this gives us a \defn{simple} $3$-anchor hull $Q'$ with the edge center in its interior.}
Each of the constant number of calls to the geodesic oracle takes time 
$O(|Q| + t(Q))$.

\smallskip
\noindent{\bf 5. If $|Q'| \le 6$, find  ${\cal T}(Q')$ and ${\cal K}_T(Q')$.}
Since $Q'$ has constant size, we can find its intersection with each triangle in ${\cal T}(Q)$ in constant time, so we can find ${\cal T}(Q')$ and ${\cal K}_T(Q')$ in time $O(t(Q))$.


\smallskip
\noindent{\bf 6. If $|Q'| > 6$ find ${\cal K}(Q')$ and ${\cal T}(Q')$.}
We first find ${\cal K}(Q')$ by checking which chords of ${\cal K}(Q)$ cross $Q'$.
By Observation~\ref{obs:3-geo-cell}, the 
$3$-anchor hull
$Q'$ is bounded by at most three polygon chains and three geodesic chains.
A chord of ${\cal K}(Q)$ crosses $Q'$ if and only if it has an endpoint interior to one of polygon chains of $Q'$, or crosses one of the geodesic chains of $Q'$.  
We can test the former in constant time per chord because we know the endpoints of each chord on $\partial P$ (including knowing  which edge of $P$ contains the endpoint). %
We can test the latter by finding the intersections of the chords of ${\cal K}(Q)$ with each of the at most three geodesics bounding $Q'$ using Lemma~\ref{lem:intersection-with-geodesic} in $Q$. The runtime is $O(|{Q}| + |{\cal K} (Q)|)$ = $O(|{Q}| +  t(Q))$.  

Note that these tests also determine the endpoints of each chord of ${\cal K}(Q')$ on $\partial Q'$.


Finally, since each chord of ${\cal K}(Q)$ records the triangles of ${\cal T}(Q)$ that it bounds, we set ${\cal T} (Q')$ to be the triangles that are bounded by a chord of ${\cal K}(Q')$.
Note that this gives all triangles that intersect the interior of $Q'$ since no triangle contains $Q'$
by Lemma~\ref{lem:floating_cells_constant_size}.
This step takes
$O(t(Q))$ 
time.





