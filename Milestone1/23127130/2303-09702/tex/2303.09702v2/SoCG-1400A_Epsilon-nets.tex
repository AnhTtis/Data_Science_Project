\subsection{Details for Section~\ref{section:epsilon-net-results}, $\epsilon$-Net Results for Stage 1}
\label{appendix:epsilon-nets}


\begin{claim}
\label{claim:4-cell}
If $Q$ is a 4-cell, then it is a $4$-anchor hull.
\end{claim}
\begin{proof}
Around the boundary of $Q$, there are four chords (or segments of chords), with two consecutive ones joined by a polygon chain or meeting at a point.  $Q$ is the geodesic hull of these $\le 4$ polygon chains and points.
\end{proof}

\begin{claim}
\label{claim:same-crossing-chords}
Let $Q$ be a 
\newchanged{$3$-anchor hull}
and $\psi(Q)$ be the corresponding expanded 3-anchor hull. Then 
a chord of $\cal{K}$ crosses $Q$ if and only if it crosses $\psi(Q)$,
i.e., ${\cal K}(Q) = {\cal K}(\psi(Q))$.
\label{claim:expansions}
\end{claim}

\begin{proof} 
One direction of the proof is simple: If a chord crosses $Q$, it must cross $\psi(Q)$ since $Q \subseteq \psi(Q)$.

For the other direction we prove
that if a chord $K \in {\cal K}$ does not cross the 
$3$-anchor hull $Q$, 
then it does not cross the 
expanded $3$-anchor hull $\psi(Q)$.
Suppose that a chord $K \in \mathcal{K}$ does not cross $Q$. 
Then $Q$ is contained in one of the closed half-polygons, say $H$, defined by $K$.
\newchanged{
This implies that the anchors of $Q$ are contained in $H$.
Since the corresponding expanded anchors were defined to not cross chords of $\cal K$, they are contained in $H$. Thus $\psi(Q)$, being the geodesic convex hull of sets in $H$, is also in $H$.  So $K$ does not cross $\psi(Q)$. 
} 
\end{proof}



\begin{lem}
\label{lem:subspace-oracle}
The $3$-anchor 
range space 
has a subspace oracle.
\end{lem}

\begin{proof} 
We must provide a deterministic algorithm that, given a subset ${\cal K}' \subseteq {\cal K}$ with $|{\cal K}'| = m$, computes the set
of ranges ${\cal R} = \{ {\cal K}'(Q) \mid Q \text{ is 
a $3$-anchor hull}\}$
in time $O(m^{d+1})$, where $d=6$ is the shattering dimension of the $3$-anchor range space.  

\newchanged{
We use the equivalence of 
the $3$-anchor range space and the expanded $3$-anchor range space (Lemma~\ref{lem:anchor-ranges}).
In Lemma~\ref{lem:constant_shattering_dimension} we proved that 
the number of expanded $3$-anchor hulls, $Q$, 
is $O(m^6)$.  
We must find these, and find, 
for each $Q$, the set of chords of $\cal K'$ that cross it. 

Recall that $A(\cal{K'})$ is the arrangement of the chords of ${\cal K}'$ plus the edges of $P$. 
This is an arrangement of line segments, with the special property that all segment endpoints are on the outer face. 
Recall also that $V({\cal K'})$ denotes the endpoints of the chords in $\cal K'$.
If $\cal{K'}$ has size $m$, then
$A(\cal{K'})$ has $O(m^2)$ faces, $O(m^2)$ internal vertices and edges, and $n+2m$ external vertices and edges on the boundary of $P$. 
In particular, the external vertices are the vertices of $P$ union $V({\cal K'})$.
For the algorithm we will avoid the dependence on $n$ by working with a combinatorial version of $A(\cal{K'})$ in which each minimal chain along $\partial P$ with endpoints in $V({\cal K'})$
is represented by a single ``dummy edge''. 
Note that the number of dummy edges is at most $2m$.  
Let \defn{$G(A({\cal K'}))$}
denote this planar graph, which has $O(m^2)$ vertices, edges, and faces.  


We compute $G(A({\cal K'}))$ as follows. 
Compute 
the arrangement of the $m$ line segments $\cal K'$ in time $O(m^2)$.
Then traverse the outer face of the arrangement, adding dummy edges corresponding to subchains of $\partial P$ between vertices of  $V({\cal K}')$. 
We thus compute $G(A({\cal K'}))$ in time $O(m^2)$.

Next, we enumerate all of the possible expanded anchors: the $O(m^2)$ internal vertices, edges, and faces of $G(A({\cal K'}))$, and the $O(m^2)$ polygon chains, each represented by two endpoints in $V({\cal K'})$. 

For each of the $m$ chords $K$ of $\cal K'$ we enumerate the
anchors that lie in each of the two closed half-polygons $H$ defined by $K$.
In particular, we can traverse $G(A({\cal K'}))$ in time $O(m^2)$ to find the 
vertices, edges, and faces that lie in $H$.  We can also decide which of the $O(m^2)$  polygon chains lie entirely in $H$, based on where the endpoints lie.  This takes time $O(m^2)$ per chord, for a total of $O(m^3)$.


Finally, we can enumerate all the $O(m^6)$ choices of at most three expanded anchors that determine an expanded $3$-anchor hull $Q$.  For each choice we spend $O(m)$ time to find the set of chords crossing $Q$---begin with all of $\cal K'$ and eliminate chords that have all three anchors on the same side, since these are precisely the chords do not cross $Q$.
This gives us the set of chords crossing $Q$.}
\end{proof}

\newchanged{Designing a subspace oracle for the 
$4$-cell range space of Ahn et al.~seems problematic.  However, the above proof can be used to show that the $4$-anchor range space has a subspace oracle.  Thus constant-sized $\epsilon$-nets can be found in deterministic linear time. An $\epsilon$-net for the $4$-anchor range space is an $\epsilon$-net for the $4$-cell range space.  This repairs the approach of Ahn et al., modulo repairing their partition of a cell into $4$-cells (Appendix~\ref{section:counterexample}).}
