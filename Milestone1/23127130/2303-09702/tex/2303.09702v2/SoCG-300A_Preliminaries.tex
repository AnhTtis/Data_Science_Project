\section{Extra Material for Section~\ref{section:prelims}, Preliminaries}


\subsection{\reviewerchange{Details on General Position Assumptions}}
\label{appendix:general-position}

We restate Assumptions~\ref{assumptions} in order to refer to the parts individually.

\begin{assumption}\label{assumption:no_three_points_collinear}
No three vertices of 
$P$ are collinear.
\end{assumption}



\begin{assumption}
\label{assumption:unique_farthest_neighbor}
After imposing the tie-breaking rule, no vertex is equidistant from two or more edges.
\end{assumption}

Figure~\ref{fig:twod_voronoi} shows the reason for Assumption~\ref{assumption:unique_farthest_neighbor}.


\begin{figure}[h!]
  \centering
  \includegraphics[width=.6\textwidth]{RS_figures/twod_voronoi_region-cropped.pdf}
  \caption{
  Vertex $r$ 
  \changed{is equidistant from edges $e$ and $f$ (colored red), and so is every point in the shaded region.
  Assumption~\ref{assumption:unique_farthest_neighbor} forbids this situation.
  }
  }
\label{fig:twod_voronoi}
\end{figure}




\begin{lem}
\label{lem:1D-Voronoi-edges}
Let $D$ be the  set of points in $P$ with more than one farthest edge (after imposing the tie-breaking rule).  Then $D$ does not contain a 2-dimensional ball, does not contain a vertex of $P$, and intersects $\partial P$ in isolated points. 
\end{lem}
\begin{proof}
It suffices to show that the conditions hold for the set of points equidistant from two edges $e$ and $f$.
Let $p$ be a point with $d(p,e) = d(p,f)$. 
The paths $\pi(p,e)$ and $\pi(p,f)$ do not share a vertex other than the terminal point, by Assumption~\ref{assumption:unique_farthest_neighbor}---in particular, $p$ cannot be a vertex.  

If $\pi(p,e)$ and $\pi(p,f)$ share 
a terminal vertex $u$, then the Tie-Breaking Rule would apply unless $p$ is on the bisector of the angle at $u$, which is 1-dimensional and intersects $\partial P$ in a single point because no three vertices are collinear by Assumption~\ref{assumption:no_three_points_collinear}.

Otherwise, the paths $\pi(p,e)$ and $\pi(p,f)$ diverge at $p$. 
Let $s_e$ be the first vertex on the path $\pi(p,e)$---or let $s_e=e$ in case there are no vertices.  Define $s_f$ similarly.  Then $p$ must be on the weighted bisector between $s_e$ and $s_f$, which is 1-dimensional, and and intersects $\partial P$ in isolated points.
\end{proof}


\begin{assumption}
\label{assumption:Voronoi-vertices}
No point on the polygon boundary has more than two farthest edges. No point in the interior of the polygon has more than a constant number of farthest edges.
\end{assumption}


We note that our
assumptions
can be effected by perturbing vertices, since, in the $2n$-dimensional space of allowed vertex perturbations, the configurations we must avoid are lower-dimensional.


