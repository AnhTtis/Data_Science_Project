\subsection{Details for Section~\ref{section:chord_oracle}, Chord Oracles and Coarse Covers}
\label{appendix:oracles}


The algorithms to find the relative center on a chord and to find the center of a polygon depend on a crucial convexity property.
Define the \defn{geodesic radius function}, $r(x)$, for $x \in P$
to be the maximum geodesic distance from $x$ to a site (a vertex or edge). 
Thus the center is the point $x$ that minimizes $r(x)$.
A function is \defn{geodesically convex} on $P$ if the function is convex on every geodesic path in $P$.
The following result was proved for vertex sites by Pollack et al.~\cite{pollack_sharir} and for edge sites by Lubiw and Naredla~\cite{lubiw2021visibility}.

\begin{lem}
\label{lem:geodesically-convex}
The geodesic radius function $r(x)$ is geodesically convex.
\end{lem}

For our extensions of the chord oracle in the following subsection, we need some details of the
$O(n)$ time chord oracle algorithms of Pollack et al.~\cite[Section 3]{pollack_sharir} for the vertex center and of Lubiw and Naredla~\cite[Section 4.1]{lubiw2021visibility} for the edge center. 

\medskip
\noindent
\ \ \textbf{Chord Oracle}

\smallskip
\noindent
\ \ \ \ {\bf Input:} a chord $K$ of polygon $P$ on $n$ vertices.\\
\ \ \ \ {\bf Output:} whether the center of $P$ lies left/right/on $K$.
\vskip -1in
\setlength{\leftmargini}{.5in}
\begin{enumerate}
\squeezelist
\item Find a coarse cover of $K$.
\item Find
the point on $K$ that minimizes the upper envelope of the coarse cover functions---this is the relative center $c_K$. 
\item Examine the 
maximum values of the coarse cover functions
at $c_K$ to determine
whether the center of $P$ lies 
left/right/on $K$.
\end{enumerate}
\setlength{\leftmargini}{\parindent}



The details of these steps (none of which is trivial) can be found in~\cite{pollack_sharir,lubiw2021visibility}.
Step 1 runs in time $O(n)$ and produces a coarse cover $\cal T$ of size $O(n)$.

Step 2 runs in time $O(|{\cal T}|)$ using divide-and-conquer to reduce the search space to a subinterval of $K$ while eliminating elements of the coarse cover.  It uses a basic test of whether the relative center lies to the left or right of a point on $K$.  The correctness of this test depends on 
convexity of the upper envelope function on $K$ (Lemma~\ref{lem:geodesically-convex}), and on the fact that 
the coarse cover captures the first segments of paths to the farthest sites.  
If the first segments pull the test point in opposite directions on $K$, then the point is locally optimum and therefore is the relative center; and otherwise, we know which direction the test point should move.

Step 3 
similarly relies on Lemma~\ref{lem:geodesically-convex} and uses the first segments of paths from $c_K$ to its farthest sites.  From those segments, we can detect if $c_K$ is locally---and hence globally---optimal, and otherwise decide which side of $K$ to move to.
Step 3 takes time $O(|{\cal T}|)$.
