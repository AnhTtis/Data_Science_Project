\subsection{Details on Separators and Funnels}
\label{appendix:separators}

Any geodesic path between two vertices of $P$ separates the boundary of $P$ into two parts, and when we focus on which vertices/edges are in opposite parts,
we call the geodesic path 
a ``separator''.
Separators are a 
main tool for finding all farthest vertices in a polygon. 
They were first introduced by Suri~\cite{suri1989} (although he called them ``connectors'' rather than  ``separators'') in his $O(n \log n)$ time algorithm to find farthest
vertices of all vertices, and then they were used by Hershberger and Suri~\cite{hershberger1997matrix} who improved the runtime to $O(n)$.  
Vertex separators (called ``separating paths'') were also used by Ahn et al.~\cite{linear_time_geodesic}, both when they appealed to Hershberger-Suri, and in more direct ways.  
We need edge separators in similar ways. 

The basic properties that Suri~\cite{suri1989} proved for 
separators for farthest vertices are as follows:
\begin{enumerate}
\item 
If two vertices $x$ and $y$ are separated by a geodesic path $\pi(a,b)$, then the shortest path from $x$ to $y$ is contained, except for one edge, in the shortest path trees of $a$ and $b$~\cite[Lemma 4]{suri1989}.  Thus, after constructing the shortest path trees from $a$ and $b$, it is easy to find shortest paths for any pair $x,y$ that is separated by $\pi(a,b)$.
\item A constant number of separators suffice to separate every vertex from its farthest vertex~\cite[Section 4]{suri1989}.

\end{enumerate}



In this section we develop the analogous theory of separators for farthest edges. 


\begin{definition}
A \defn{farthest edge separator} is a directed geodesic path $\gamma = \pi(a,b)$ from some vertex $a$ to some vertex $b$ of $P$ such that for every point $p \in \delta P$ to the %
right of $\gamma$, all of $p$'s farthest edges lie to the %
left of $\gamma$.
\end{definition}

Note that we define separators via the strong property that \emph{all} points to one side have their farthest edge on the other side.  Although this property is not part of Suri's original definition, his construction produces vertex separators with the property.


In this section we will prove that Suri's two properties hold for our farthest edge separators.
We first note an even more basic property that is the main reason for using separators: 

\begin{claim}
\label{claim:across-separator}
If $\gamma = \pi(a,b)$ is a farthest edge separator and points $p$ and $q$ lie to the right
of $\gamma$ and their farthest edges $F(p)$ and $F(q)$ are distinct, then the paths to their farthest edges cross.
\end{claim}
\begin{proof}
The ordering $p,F(p),q,F(q)$ is excluded by the separator.  The ordering $p,q,F(q),F(p)$ cannot occur by Property~\ref{prop:anti-parallel}.  Thus the ordering must be $p,q,F(p),F(q)$.  See also Figure~\ref{fig:correct_orderings}.
\end{proof}



\subsubsection{Funnels and Shortest Paths Across a Separator}
\label{sec:funnels-paths}

We first address Suri's property (1) by examining how a shortest path crosses a geodesic $\gamma(a,b)$.
In this subsection
the geodesic need not be a farthest edge separator, and the shortest path need not go to a farthest edge.
Hershberger and Suri~\cite{hershberger1997matrix} expanded on Suri's result and showed how a shortest vertex-to-vertex path that crosses $\gamma$ is related to the \emph{funnels} of the vertices.  We follow their analysis.

Suppose that vertex $v$ lies to the 
right of $\gamma$ and edge $e$ lies to the 
left of $\gamma$.  Then $\pi(v,e)$ crosses $\gamma$, either at a single point, or by sharing chords with $\gamma$.  
See Figure~\ref{fig:funnels-and-tangents}.
We show that $\pi(v,e)$ lies in the \emph{funnels} of $v$ and $e$ which are defined in terms of the shortest path trees $T_a$ and $T_b$.
Note that $\gamma$ is in both $T_a$ and $T_b$ since it is the shortest path from $a$ to $b$.

The \defn{funnel of $v$}, denoted \defn{$Y(v)$}, 
is bounded by $\pi(a,v)$,  $\pi(b,v)$ and $\gamma$,
where $\pi(a,v)$ and  $\pi(b,v)$ are called the \defn{walls} of the funnel.
The vertex where $\pi(a,v)$ diverges from $\gamma$ is \defn{$\gamma_a(v)$}, defined to be the lowest common ancestor of $v$ and $b$ in the tree $T_a$.  Similarly, the vertex where $\pi(b,v)$ diverges from $\gamma$ is \defn{$\gamma_b(v)$}, the lowest common ancestor of $v$ and $a$ in the tree $T_b$. 
The vertex where $\pi(v,a)$ diverges from $\pi(v,b)$ is called the \defn{apex} of the funnel. Observe that the path between the apex and $\gamma_a(v)$ [or $\gamma_b(v)$] is reflex.

Similarly, the \defn{funnel of $e$}, \defn{$Y(e)$} is bounded by $\pi(a,e)$, $\pi(b,e)$, $\gamma$, together with the piece of $e$ between the terminals $t(a,e)$ and $t(b,e)$ if those terminals are distinct. The lowest common ancestors \defn{$\gamma_a(e)$} and \defn{$\gamma_b(e)$} and the \defn{apex} can be defined analogously, where we allow the apex to be the piece of edge $e$ between $t(a,e)$ and $t(b,e)$ when those terminals are distinct. 
Funnels have been used in many shortest path algorithms,
and there are variations on how they are defined (as a subpolygon or a set of edges; including the edges common to two paths or not, etc.).

\begin{figure}
    \centering
    \includegraphics[width=\textwidth]{RS_figures/shortest_paths_across_separator-cropped.pdf}
    \caption{A geodesic $\gamma = \pi(a,b)$, the funnels $Y(v_1), Y(v_2)$, and $Y(e)$ (in blue) and the paths $\pi(v_1, e)$ and $\pi(v_2, e)$ (in red).}
    \label{fig:funnels-and-tangents}
\end{figure}

The pair of funnels $Y(v),Y(e)$ is \defn{closed} if 
the paths  $\pi(\gamma_a(v),\gamma_b(v))$ and  $\pi(\gamma_a(e),\gamma_b(e))$ are internally disjoint, see $Y(v_2)$ and $Y(e)$ in Figure~\ref{fig:funnels-and-tangents}.  Otherwise the pair of funnels  is \defn{open}, see $Y(v_1)$ and $Y(e)$ in the figure.
Hershberger and Suri 
dealt with the case where 
$e$ is replaced by a vertex $u$.
They showed that if the pair $Y(v), Y(u)$ is closed, then the edges of $\pi(v,u)$ are edges of the funnels.  In particular, suppose $\gamma_a(u)$ and $\gamma_b(u)$ are closer to $a$ than  $\gamma_a(v)$ and $\gamma_b(v)$ (the other ordering is analogous).  Then $\pi(v,u)$ consists of the paths 
$\pi(v, \gamma_a(v))$, $\pi(\gamma_a(v),\gamma_b(u))$, 
$\pi(\gamma_b(u), u)$.
On the other hand, if the pair $Y(v),Y(u)$ is open, then $\pi(v,u)$ consists of part of a wall of $Y(v)$ and part of a wall of $Y(u)$ joined by  
a \defn{tangent} edge \defn{$\ell(v,u)$} that crosses $\gamma$.

To deal with an edge funnel $Y(e)$, we abuse the notation and say that a segment that meets $e$ at right angles is tangent to $Y(e)$ (this makes   sense if we imagine that the edges that meet $e$ at right angles extend off to infinity). 

The above results are used to prove the following two lemmas and will also be used in
Appendix~\ref{appendix:Hershberger-Suri}
when we show how to extend Hershberger and Suri's algorithm for finding farthest vertices of all vertices to the case of farthest edges.



\begin{lem}
\label{lem:separator-paths}
\changed{Consider a geodesic path $\gamma = \pi(a,b)$ with vertex $v$ to the right and edge $e$ to the left.  If the pair of funnels $Y(v),Y(e)$ is closed then the edges of $\pi(v,e)$ are contained in the shortest path trees $T_a$ and $T_b$.
If the pair of funnels $Y(v),Y(e)$ is open then the edges of $\pi(v,e)$ are contained in the shortest path trees $T_a$ and $T_b$, except for one edge $\ell(v,e)$ that crosses $\gamma$ and is tangent to $Y(v)$ and $Y(e)$.
}

\end{lem}
\begin{proof}
Consider the terminal point $t(v,e)$ of the path $\pi(v,e)$.  If $t(v,e)$ is a vertex of $P$, then the previous results apply. 
Otherwise, 
\changed{let $s$ be the last segment of the path $\pi(v,e)$.  Segment $s$ meets $e$ at a right angle at point $t(v,e)$. Let $u$ be the vertex at the start of $s$. 
If $u$ is to the left of $\gamma$ then $s$ is an edge of $Y(e)$ and the result follows from the previous result for $v$ and $u$.
Otherwise, $u$ is to the right of $\gamma$, the pair of funnels is open, and $s$ is the tangent edge that crosses $\gamma$.}
\end{proof}


\begin{lem}\label{lemma:walls_can_be_constructed_fast}
Let $\gamma = \pi(a,b)$ be a geodesic path.  
After \reviewerchange{linear-time} preprocessing (to compute the trees 
$T_a$ and $T_b$
and preprocess them for answering lowest common ancestor queries in constant time), the shortest path $\pi(v,e)$ from any vertex $v$ to the 
right %
of $\gamma$ to any edge $e$ to the 
left %
of $\gamma$
can be computed in 
time proportional to the number of vertices in $\pi(v,e)$.
\end{lem}
\begin{proof}
Compute the least common ancestors $\gamma_a(v)$, $\gamma_b(v)$, $\gamma_a(e)$, and $\gamma_b(e)$ in constant time. \reviewerchange{(This can be done after linear-time preprocessing using the algorithm of Harel and Tarjan~\cite{harel1984fast})}.
Test if the pair of funnels $Y(v),Y(e)$ is closed or open in constant time using least common ancestor queries. 
If the pair is closed, the path $\pi(v,e)$ consists of subpaths that can be found in time linear in the number of vertices.

Otherwise, we must find the tangent edge $\ell(v,e)$ of the two funnels.
The edge $\ell(v,e)$ 
may meet $e$ at right angles, which is a special case we deal with later.
Note that it suffices to search between the apexes of the funnels---to ease notation, we will just suppose that that those apexes are $v$ and $e$ themselves.
Then $\ell(v,e)$ is tangent to two reflex curves where one is a wall of the funnel $Y(v)$ and one is a wall of the funnel $Y(e)$.  
There are four possible choices for the two reflex curves, and for each choice, we are essentially finding the common tangent of two disjoint convex polygons, a very well-solved problem (see~\cite{abrahamsen2018common} for some history).  A simple search that walks from the two apexes along the chosen paths towards $\gamma$ will find the tangent in time proportional to the number of vertices traversed---and those vertices are part of the output path.  Doing this in parallel over the four choices, we can find $\ell(v,e)$ and $\pi(v,e)$ in time linear in the number of vertices of $\pi(v,e)$.
(This is the same argument as given by Ahn et al.~\cite[Lemma 3.5]{linear_time_geodesic}.)
Finally, to address the possibility that
$\ell(v,e)$ meets $e$ at right angles, we can perform a similar search between $e$ and each of the walls of $v$'s funnel.
\end{proof}




\subsubsection{Constant Number of Separators}
\label{sec:constant-separators}


We now turn to Suri's property (2)---finding a constant number of separators.

\begin{lem}
\label{lem:constant_splits}
\label{lem:constant-separator-set}
There is a set of at  most five farthest edge separators such that 
every  point $p  \in   \partial P$ 
(and consequently, every edge of $P$) lies to the 
\reviewerchange{right}
of at least one  of the separators. 
Furthermore, such  a set of separators can be found in linear time.
\end{lem}


This lemma 
is extremely important because it reduces farthest edge problems to a constant number of ``bipartite'' cases where the source vertices are separated from the target edges.
Lemma~\ref{lem:constant-separator-set} will be used in 
Appendix~\ref{appendix:Hershberger-Suri}
to find farthest edges from all vertices.  It will also be used in
Appendix~\ref{appendix:hourglasses}
to find 
hourglasses in $P$
and in 
Appendix~\ref{appendix:coarse-cover}
to construct the coarse cover of $P$.

\begin{proof}[Proof of Lemma~\ref{lem:constant_splits}]
We first note the 
consequence that for every polygon edge $(x,y)$, one of the five separators has both $x$ and $y$ to its 
\reviewerchange{right.}
This is because separator
endpoints are
vertices so a farthest edge separator for the midpoint of edge $(x,y)$ must have $x$ and $y$ to its 
\reviewerchange{right}.
Thus it suffices to prove that there are five farthest edge separators such that every point $p \in \partial P$ lies to the 
\reviewerchange{right}
of at least one separator.


The plan in Suri's proof for the case of farthest vertices, was to follow a chain $v_1, v_2, v_3,v_4$ where $v_{i+1}$ is the farthest vertex from $v_i$, and argue that $\pi(v_3,v_4)$ crosses $\pi(v_1,v_2)$, and that this provides three separators, namely the %
three paths.  Our plan is similar, but a bit trickier because our paths go from a vertex/point to a farthest edge, so we must then choose a point in the edge to continue the chain.

Take an arbitrary vertex $u$ and find its farthest edge $F(u)$. Note that $F(u)$ is unique by Lemma~\ref{lem:1D-Voronoi-edges}.
This can be achieved in linear time by constructing the shortest path tree
$T_u$
(Lemma~\ref{SPT_lemma}) and finding a leaf furthest from $u$ in this tree.
Suppose $F(u)$ is the  edge $e$ with endpoints $e^-,  e^+$ in  clockwise order.
Find the farthest edges $F(e^-)$ and  $ F(e^+)$ in linear time.

\textbf{Case 1.} First, we suppose that the geodesics $\pi(e^-, F(e^-))$ and $\pi(e^+,F(e^+))$ both cross $\pi(u,e)$. See Figure~\ref{fig:possible_case_2a}.  We claim that 
the geodesics 
$\gamma_1 = \pi(u,e^+)$,  and $\gamma_2 = \pi(e^-,u)$
are farthest edge separators.
To  prove this,  consider a point $p$ to  the 
right
of $\gamma_1$, i.e., a point in the clockwise chain $C$ from  $e^+$ to $u$, and suppose that $p$ has a farthest edge  $F(p)$ on the same chain.  Note that $F(p) \ne e$,  since $e$ is not part of the  chain.   If $F(p)$ occurs before $p$ along the chain $C$, then $p,u,F(u),F(p)$ occur in that clockwise order and violate Property~\ref{prop:anti-parallel}. 
Otherwise $F(p)$  occurs after $p$ along the chain  $C$ in which case $e^+,p,F(p),F(e^+)$  occur in that clockwise order and violate Property~\ref{prop:anti-parallel}.
A symmetric argument shows that 
$\gamma_2$ is a farthest edge separator.

The two geodesics $\gamma_1$ and $\gamma_2$ separate all  points  of $\delta P$ from their farthest edges except the  points  of edge $e$.  We separate those points by  adding  one more  geodesic $\pi(e^+,e^-)$.  Note that this kind of degenerate %
separator is allowed by the definition, and 
is a farthest edge separator 
since every point in  the edge $e$ has its farthest edge outside $e$.

This gives a set of three farthest edge separators.  Note that they can be found in linear time.



\textbf{Case 2.} 
Otherwise at least 
one of the geodesics $\pi(e^-, F(e^-))$ and $\pi(e^+,F(e^+))$ does not cross $\pi(u,e)$.  
We will consider the case when 
the geodesic $\pi(e^-,F(e^-))$ does not cross $\pi(u,e)$---the other case is symmetric.
Suppose $F(e^-)$ is the edge $f = (f^-,f^+)$ in clockwise order. 
Find the shortest path from $u$ to edge $f$, and let
point $p:=t(u,f)$ %
be the terminal of that path.  Find the farthest edge $g := F(p)$ and suppose $g = (g^-, g^+)$ in clockwise order.
We claim that $g$
cannot lie in the 
clockwise chain from $f^+$ to $e^-$. 
Suppose it does.
Then $g \ne e$, which implies that $p \ne u$ (since $u$ has the unique farthest edge $e$). But then $u,p, F(p), F(u)$ violate Property~\ref{prop:anti-parallel}.
Therefore, the edge $g$ lies either: (a) in the clockwise chain
from $e^-$ to $u$, in which case we find separators; or 
(b) in the clockwise chain from $u$ to $f^-$, which we prove is impossible. We consider the two cases (a) and (b).


\textbf{Case 2a.} The edge $g = F(p)$ lies in the clockwise chain from $e^-$ to $u$. 
The situation is depicted in Figure~\ref{fig:possible_case_2a}.
We claim that the geodesics $\gamma_1 =  \pi(u,e^+)$, $\gamma_2 =  \pi(f^-,g^+)$ and $\gamma_3 = (e^-,f^+)$ are 
farthest edge separators. Note that $\gamma_2$ is redundant if $f^- = u$, and  $\gamma_1$ is redundant if $g=e$. 
To prove that $\gamma_1$ is a farthest edge separator, 
note that because of the ``anti-parallel'' pair $\pi(u,e)$ and $\pi(e^-,f)$, no point $p \in \partial P$ to the 
right %
of $\gamma_1$ has a farthest edge to the 
right %
of $\gamma_1$ (otherwise the path from $p$ to  such  a farthest edge 
must go in the same direction as one of 
$\pi(u,e)$ and $\pi(e^-,f)$, thus violating 
Property~\ref{prop:anti-parallel}).
Similarly, $\gamma_2$ is a farthest edge separator because of the anti-parallel pair $\pi(p,g)$ and $\pi(e^-,f)$, and $\gamma_3$  is a farthest edge separator because of the same anti-parallel pair. 

The three geodesics $\gamma_1, \gamma_2$ and $\gamma_3$ separate all points of $\partial P$
from their farthest edges except the points of edges $e$ and $f$.   We can separate those points by adding the geodesics $\pi(e^+,e^-)$ and $\pi(f^+,f^-)$.
This gives a set of five farthest edge separators.  Note that they can be found in linear time.


\begin{figure}[tb]
\centering
\includegraphics[width=0.9\textwidth]{RS_figures/five_separators_1-cropped.pdf}
\caption{
Case 1 and Case 2a from the proof of Lemma~\ref{lem:constant_splits}, showing (schematically) the paths from points to farthest edges (in red) and the separators (in blue).
}
\label{fig:possible_case_2a}
\end{figure}

\textbf{Case 2b.} The edge $g = F(p)$ lies in the clockwise chain from $u$ to $f^-$. 
See Figure~\ref{fig:impossible_case_2b}.
We will prove that this case cannot occur. 
To show this, 
consider the geodesic paths 
$\sigma_1 := \pi(u,f) = \pi(u,p)$, 
$\sigma_2 :=  \pi(e^-,g)$ and $\sigma_3 := \pi(p,e)$.
Note that because $e$ is the unique farthest edge from $u$, $d(u,e) > |\sigma_1|$.
Similarly, $d(e^-, f) > |\sigma_2|$, and $d(p,g) \ge |\sigma_3|$ ($p$ need not have a unique farthest edge).
Adding these together, %
we obtain
\begin{align}
d(u,e) +d(e^-,f) + d(p,g) > |\sigma_1| + |\sigma_2| + |\sigma_3|.
\label{eqn:path-lengths}
\end{align}

\begin{figure}
\centering
\includegraphics[width=0.55\textwidth]{RS_figures/impossible_case_2b_3-cropped.pdf}
\caption{
Case 2b from the proof of  Lemma~\ref{lem:constant_splits}, showing (schematically) the paths from points to farthest edges (in red) and the shortest paths $\sigma_i$ (in blue).
}
\label{fig:impossible_case_2b}
\end{figure}


Recall that for vertex $v$ and edge $h$,  $t(v,h)$ is the terminal point of the path $\pi(v,h)$.
Observe that in clockwise order $p \le t(e^-,f)$ on edge $f$, and $t(e^-,g) \le t(p,g)$ on edge $g$.
Let $x$ be the intersection point of $\sigma_1$ and $\sigma_2$ (possibly at one of their endpoints).  Let $y$ be the intersection point of $\sigma_2$ and $\sigma_3$ (possibly at one of their endpoints). Observe that along $\sigma_2$, $y$ precedes (or is equal to $x$).
See Figure~\ref{fig:impossible_case_2b}.
We get the following inequality from the definition of a terminal and the triangle inequality:
\begin{align*}
  d(u,e) & = d(u,t(u,e)) \leq d(u,t(p,e)) \\
   &\leq d(u,x) + d(x,y) + d(y,t(p,e)) 
\end{align*}
\noindent Reasoning as above, we also get the following two inequalities:
\begin{align*}
& d(e^-,f) \leq d(e^-,y) + d(y,p)\\
& d(p,g) \leq d(p,x) + d(x,t(e^-,g))
\end{align*}
\noindent Adding the three inequalities 
and noting that we have used each subpath of each $\sigma_i, i=1,2,3$ exactly once, we obtain:
$$d(u,e) +d(e^-,f) + d(p,g) \leq |\sigma_1| + |\sigma_2| + |\sigma_3|.$$
which contradicts Equation~\ref{eqn:path-lengths}.
\end{proof}




