
\subsubsection{Overview of $\epsilon$-Nets}
\label{appendix:epsilon-net-overview}
This section contains background results on 
$\epsilon$-nets and their use in geometric divide-and-conquer algorithms.
For more details, we refer to the paper by Haussler and Welzl~\cite{haussler1987},  the survey by Mustafa and Varadarajan~\cite[Chapter 47]{toth2017handbook}, and the book by Mustafa~\cite{mustafa2022sampling}.
A \defn{range space} is a pair $(X,\mathcal{R})$ where $X$ is a \defn{ground set} of elements and $\mathcal{R}$ is a set of subsets of $X$.
We refer to the elements of $\mathcal{R}$ as the \defn{ranges} of the range space.
For any $\epsilon$ between 0 and 1, an \defn{$\epsilon$-net} for the range space $(X,\mathcal{R})$ is a subset $N \subseteq X$ with the following property: for every range $R$ in $\mathcal{R}$ with $|R| \geq \epsilon |X|$, we have $N \cap R \neq \phi$.
We use this as:

\begin{equation}
\text{if $N \cap R = \phi$, then $|R| < \epsilon |X|$}   \label{eqn:epsilon-net} 
\end{equation}


In many geometric settings, the ground set consists of hyperplanes.
In such cases, 
an $\epsilon$-net $N$ determines a hyperplane arrangement that partitions the space and suggests 
a natural divide and conquer approach based on the cells of this partition (called a \emph{cutting}~\cite{chazelle1993cutting,matouvsek1991cutting}).
We follow this approach, although our ground set consists of chords of the polygon rather than hyperplanes.

The size of the $\epsilon$-net 
directly controls the number of subproblems in the divide and conquer algorithm.
Efficient algorithms using this approach require an $\epsilon$-net of small size.
One way to guarantee constant sized $\epsilon$-nets is using combinatorial properties like the VC-dimension or shattering dimension.



Consider the range space $(X,\mathcal{R})$.
For a set $A \subseteq X$, the \defn{restriction of the ranges} to $A$, denoted \defn{${\cal R}_{|A} $}, is defined to be $\{A \cap R : R \in {\cal R} \}$.
A set $A$ is \defn{shattered} by the range space $(X,\mathcal{R})$ if ${\cal R}_{|A} $ is the power set of $A$.
The \defn{VC-dimension of a range space} $(X,\mathcal{R})$ is the maximum size of a set that can be shattered by the range space.
If a range space can shatter sets of arbitrarily large size, it has infinite VC-dimension.





The \defn{shattering dimension} of the range space $(X,\mathcal{R})$ is the minimum number $d$ such that for all $m$ and for all sets $A \subseteq X$ with $|A|=m$, we have $|{\cal R}_{|A}| \in O(m^d)$.
Equivalently, this says
that the number of ranges 
when restricted to
any subset of size $m$ is upper bounded by a polynomial in $m$ of degree equal to the shattering dimension.
Usually, upper bounds for the shattering dimension can be found more readily than those for the VC-dimension,
and upper bounds on the shattering dimension imply upper bounds on the VC-dimension, as expressed by
the following restatement of Lemma 5.14 from Har-Peled~\cite{har2011geometric}:

\begin{lem}
\label{lemma:shatter_vc_related}
If a range space has shattering dimension $d$, its VC-dimension is bounded by $O(d \log d)$, specifically by $12d \ln{(6d)}$.
\end{lem}


In the next lemma, we state the result of Haussler and Welzl~\cite{haussler1987} that a range space $(X,\mathcal{R})$ of  finite VC dimension has constant-sized $\epsilon$-nets.  
For a divide and conquer algorithm we also need an  algorithm to \emph{find} an $\epsilon$-net of constant size. 
A randomized algorithm is easier to obtain
but we need a deterministic algorithm.
Such a deterministic algorithm
was given by 
Matousek~\cite{matousek1991subspace} for any range space of shattering dimension $d$ that has a 
\defn{subspace oracle} which is defined to be a deterministic algorithm that, given a subset $X' \subseteq X$, computes the set $\mathcal{R}_{|X'}$ in time 
$O(|X'|^{(d+1)})$.

We summarize the results of Haussler and Welzl~\cite{haussler1987} and 
Matousek~\cite{matouvsek1989construction} in the following lemma.
Other sources for these results include
the survey  by Mustafa and Varadarajan~\cite[Chapter 47, Theorem 47.4.3]{toth2017handbook}, and the textbook by Mulmuley~\cite{mulmuley1994computational}.


\begin{lem}
\label{lem:constant_size_net}
A range space $(X,\mathcal{R})$ of finite VC-dimension has $\epsilon$-nets of size $O(\frac{1}{\epsilon} \log{\frac{1}{\epsilon}})$.
Furthermore, if the range space has a subspace oracle
then such an $\epsilon$-net can be found in deterministic time $O(|X|)$.
\end{lem}

