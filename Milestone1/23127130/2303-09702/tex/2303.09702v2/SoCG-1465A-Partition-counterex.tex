\subsection{Problem with the Partitioning Scheme of Ahn et al.}
\label{section:counterexample}




In this section we explain the error in the step of the algorithm of Ahn et al.~\cite[Section 6]{linear_time_geodesic} where they take a 4-cell subdivided by chords of  
an $\epsilon$-net $N$ of constant size and partition the resulting faces into a constant number of $4$-cells. 
From the intersection points and endpoints of the chords in $N$, they shoot vertical rays up and down until either a chord of $N$ or the boundary of the outer 4-cell is reached. 
They claim that this subdivides each face into a constant number of 4-cells.  It is true that there are a constant number of regions, but not true that the regions are $4$-cells.
\reviewerchange{A counterexample is shown in Figure~\ref{fig:decomposition-1};}
there are five chords in $N$, and the construction of Ahn et al.~leaves a $5$-cell.

\reviewerchange{We briefly describe a way to fix their approach.
Find a trapezoidization of the faces of the arrangement of $N$ in the $4$-cell.  This can be done in time linear in the size of the $4$-cell.
The dual of the trapezoidization is a tree.  Working from the leaves of the tree, take a union of trapezoids until the resulting region is a $4$-cell, then chop it off and continue.
}

