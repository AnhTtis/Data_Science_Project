\subsection{Details on Properties of Farthest Edges}
\label{appendix:farthest-properties}





In this section we give some  basic properties of shortest paths from points on $\partial P$ to their farthest edges in a polygon, with a focus on when and how such paths cross---more formally, we examine the ordering of the points and their farthest edges around the polygon boundary.




We first prove Lemma~\ref{lem:new_order}.
\begin{proof}[Proof of Lemma~\ref{lem:new_order}]

\begin{figure}
\centering
\includegraphics[trim={0 0cm 0 0cm}, width=.6\textwidth]{RS_figures/RS_total_monotonicity_lemma_1-cropped.pdf}
\caption{Illustration for Lemma~\ref{lem:new_order}.
}
\label{fig:total_monotonicity_lemma}
\end{figure}
Suppose points
$p$, $q$ and edges $e$, $f$ occur in the order $p,q,e,f$ along the polygon boundary $\partial P$.
See Figure~\ref{fig:total_monotonicity_lemma}. 
We must prove that $d(p,e) + d(q,f) \geq d(p,f) + d(q,e)$.

Due to the ordering of $p,q,e,f$ on $\partial P$, the paths $\pi(p,e)$ and $\pi(q,f)$ must have a common point which we label $x$.   Then:
\begin{align*}
&d(p,e) + d(q,f) \\
  & = d(p,t(p,e)) + d(q,t(q,f)) \text{\quad [Distance to an edge is distance to the terminal]}\\
  & = d(p,x) + d(x,t(p,e)) + d(q,x) + d(x,t(q,f)) \\
  & = (d(p,x) + d(x,t(q,f))) + (d(q,x) + d(x,t(p,e))) \\
  & \geq d(p, t(q,f)) + d(q,t(p,e)) \text{\quad [Triangle Inequality]}\\
  & \geq d(p,t(p,f)) + d(q, t(q,e))  \text{\quad [Definition of a terminal]}\\
  & = d(p,f) + d(q,e) \text{\quad\quad\quad\quad\quad\ [Distance to an edge is distance to the terminal]}
\end{align*}
\end{proof}

We often use Lemma~\ref{lem:new_order} in the following form.

\begin{coro}
\label{cor:ordering-property-paths}
Under the same assumptions, if $d(p,f) > d(p,e)$, then $d(q,f) > d(q,e)$.
\end{coro}


We now return to farthest paths.
Let $p$ and $q$ be two  points on  the polygon boundary.
Let $F(p)$ be a farthest edge from  $p$ and let $F(q)$ be a farthest edge from $q$.
Note that these farthest edges need  not be unique since there are (isolated) points on the polygon boundary with two farthest edges.
Let $\pi_p = \pi(p,F(p))$ and let $\pi_q = \pi(q,F(q))$.
Let $t_p$ be the terminal point of the path $\pi_p$ and let $t_q$ be the terminal point of the path $\pi_q$.
If $F(p) \ne F(q)$, we 
 say that $\pi_p$ and $\pi_q$ \defn{cross} if the ordering around the polygon boundary (in either clockwise or counterclockwise order) is $p,q,F(p),F(q)$.  
Note that this allows the possibility that $t_p = t_q$ at a reflex vertex. 
If $F(p) = F(q)$, we say that $\pi_p$ and $\pi_q$ \defn{cross} if the ordering around the polygon boundary (in either clockwise or counterclockwise order) is $p,q,t_p,t_q$, with $t_p \ne t_q$.

\begin{lem}
\label{lem:ordering-properties}
With the above setup,
the paths $\pi_p$ and $\pi_q$ have the  following properties.

\leavevmode
\begin{enumerate}[label=\bf{(P\arabic*)}]

\setlength{\itemindent}{.15in}

\item \label{prop:path-to-same-edge}
If $F(p) = F(q)$, then 
$\pi_p$ and $\pi_q$ do not cross, 
i.e., 
the ordering of points around the boundary of $P$ is $p,q,t_q,t_p$, possibly with $t_q = t_p$ if the paths merge.
(See Figure~\ref{fig:correct_orderings}(a)).

\item \label{prop:anti-parallel}
If $F(p) \ne F(q)$
then the possible orderings are: $p,F(p), q,F(q)$ (see
Figure~\ref{fig:correct_orderings}(b));
or $p, q$, $F(p), F(q)$, i.e., the paths cross 
(see Figure~\ref{fig:correct_orderings}(c)).
Equivalently, 
the only other ordering, $p,q, F(q),F(p)$, 
cannot occur (see Figure~\ref{fig:correct_orderings}(d)).


\item 
\label{prop:ordering-property}
{\bf The Ordering Property.} 
As $p$ moves clockwise around $\partial  P$, so does $F(p)$.
 


\item 
\label{prop:no-shared-chord}

If the  paths $\pi_p$ and $\pi_q$ cross,
then they  
do  not share a directed 
 polygon chord. 
The paths may cross at a vertex or at internal points of chords.  They  may  share  a chord  in opposite directions.

\end{enumerate}
\end{lem}


The Ordering Property \ref{prop:ordering-property} was proved by Aronov et al.~\cite{aronov1993farthest} for the case of farthest vertices.
To prove Lemma~\ref{lem:ordering-properties}, we use
(as they did) the  ``triangle inequality'' Lemma~\ref{lem:new_order}.
Later on in our paper, we will appeal not only to the Ordering Property and the other parts of Lemma~\ref{lem:ordering-properties}, but also to the triangle inequality, since it  applies more generally to paths that do not go to farthest edges.
In fact, even for the basic problem of finding the farthest vertex from each vertex in a convex polygon, the triangle property is the key to \reviewerchange{linear-time algorithm}s.  The first such \reviewerchange{linear-time algorithm} was given by Aggarwal et al.~\cite{aggarwal1987geometric} using a technique called matrix searching in a totally monotone matrix.  They point out that assuming only the Ordering Property, there is a super-linear lower bound on the time to find all farthest vertices.  However, the triangle inequality implies a ``totally monotone'' matrix and allows a \reviewerchange{linear-time algorithm}.  The matrix searching technique is discussed further in 
Appendix~\ref{appendix:Hershberger-Suri}.


\begin{proof}[Proof of Lemma~\ref{lem:ordering-properties}]
\ 

\noindent
\ref{prop:path-to-same-edge}.
Suppose the ordering is $p,q,t_p,t_q$.  Then the paths must have a common point $x$. 
The shortest path from $x$ to the edge $F(p)=F(q)$ is unique, so the paths $\pi_p$ and $\pi_q$ are the same after $x$. 
Thus the ordering is $p,q,t_q,t_p$, possibly with $t_q=t_p$.

\smallskip\noindent
\ref{prop:anti-parallel}. 
We must show that 
the ordering $p,q, F(q),F(p)$ cannot occur.  Suppose  it does.  
First note that if $t_p = t_q$ then the tie-breaking rule would not allow the ordering $p,q,F(q),F(p)$.  Thus we may assume that $t_p \ne t_q$.

Since $F(p)$ is a farthest edge from $p$, $d(p,F(p))  \ge d(p,F(q))$.  Since $F(q)$ is a farthest edge from $q$, 
$d(q,F(q)) \ge d(q,F(p))$. 
By  Lemma~\ref{lem:new_order}, $d(p,F(q)) + d(q,F(p) \ge d(p,F(p)) + d(q,F(q))$.   
Therefore $d(p,F(p)) = d(p,F(q))$ and $d(q,F(q)) = d(q,F(p))$.


\begin{claim}
Both $p$ and $q$ have $F(p)$ and $F(q)$ as tied farthest edges.
\end{claim}
\begin{proof}
The distances are the same but we must be careful about the tie-breaking rule.  
If the tie-breaking rule applies to $\pi(p,F(p))$ and $\pi(p,F(q))$, then these two paths both terminate at a reflex vertex $u$ common to $F(p)$ and $F(q)$, but in this case $\pi(q,F(p))$ must also terminate at $u$ (since it cannot cross $\pi(p,F(p))$).
Then $\pi(q,F(q))$ also terminates at $u$, since we cannot have two equal-length paths from $q$ to different points on edge $F(q)$.  Thus the original paths $\pi_p$ and $\pi_q$ terminate at the same point, which we already ruled out.
\end{proof}



We claim that any  point $r$ that lies on $\partial P$ between $p$ and $q$ also has $F(p)$ and  $F(q)$ as farthest edges. 
Consider  any  edge $e$ that lies 
on the polygon chain from $r$ to $F(q)$ (the part containing $p$).
Note that $F(p)$ is one  such edge.  Applying Lemma~\ref{lem:new_order} to $q,r,e,F(q)$, gives $d(r,F(q)) + d(q,e) \ge d(r,e) + d(q,F(q))$.  Since $d(q,F(q)) \ge  d(q,e)$ this implies $d(r,F(q)) \ge d(r,e)$.  In particular, $d(r,F(q)) \ge d(r, F(p))$.  A symmetric argument shows that $d(r,F(p)) \ge d(r,e)$ for any  edge $e$ 
that lies on the polygon chain from $r$ to $F(p)$ (the part containing $q$).
In particular, $d(r,F(p)) \ge  d(r,F(q))$.  Since all edges $e$ are included in the  two ranges, this proves that $r$ has $F(p)$ and $F(q)$ tied for farthest edge.
(We can again show that the tie-breaking rule does not apply.)
By Lemma~\ref{lem:1D-Voronoi-edges},
only isolated points on $\partial  P$ can have 
$F(p)$ and $F(q)$ tied for farthest edge.
Therefore the ordering $p,q,F(p),F(q)$ cannot occur.

\smallskip\noindent
\ref{prop:ordering-property}. 
Consider a point $p$ with a farthest edge $F(p)$ and let $q$ be the first point after $p$ moving clockwise around $\partial P$ that has a farthest edge $F(q)$ 
that is not a farthest edge from $p$. 
Note that $q$ comes before $F(p)$.  Since the ordering $p,q, F(q), F(p)$ is prohibited, $F(q)$ must lie after $F(p)$ in clockwise order.


\smallskip\noindent
\ref{prop:no-shared-chord}.  Suppose $\pi_p$ and $\pi_q$ cross.
We first suppose that
$t_p = t_q$. Then the terminal point is a reflex vertex $u$ common to $F(p)$ and $F(q)$.  If the paths share a directed chord $(a,b)$, then the paths are identical after vertex $a$ and therefore identical on their last segment which is a chord from some vertex $v$ to $u$.  
The tie-breaking rule would not allow $p$ and $q$ to have farthest edges $F(p)$ and $F(q)$ unless $v$ lies on the bisector of $u$
which is excluded by Assumption~\ref{assumption:unique_farthest_neighbor}. 

Thus we may assume that
$t_p \ne t_q$ and the paths share the directed 
chord $(a,b)$.  
Consider the portion of $\pi_p$ from $b$ to  $F(p)$ and the  portion of $\pi_q$ from $b$ to $F(q)$.   Those are both shortest paths. 
Because of the \reviewerchange{general position} Assumption~\ref{assumption:unique_farthest_neighbor} (note that $b$ is a vertex), one of the paths must be longer, say the one to $F(q)$.  Now we claim that $d(p,F(q)) > d(p,F(p))$, contradiction to $F(p)$ being   a farthest edge from $p$.   
To show this, construct a path $\sigma$ from $p$ to  $F(q)$ by following $\pi_p$ from $p$ to $a$, then traversing chord $(a,b)$, then  following $\pi_q$ from $b$ to $F(q)$.  Then $\sigma$ is longer than $\pi_p$.  We are done if  we  can show  $\sigma$ is a shortest  path.  But the part  up  to $b$ is locally shortest and the  part after $a$ is locally shortest,  thus none of the  bends in the path  can be shortened, so $\sigma$ is  a geodesic  path  and thus a  shortest path.   
\end{proof}




