\section{Algorithm Guarantees and Invariants}


We will establish two main guarantees for our algorithm and a set of invariants the algorithm will maintain to establish these guarantees. The most challenging and interesting algorithmic property is that the algorithm has bounded cost.  The second is that the algorithm is feasible. The theorem below gives the cost guarantees; its proof is in Section~\ref{sec:cost}. We state our results assuming $B = \textsf{OPT}$, but all results still hold by replacing $\textsf{OPT}$ with $B$, as long as $\textsf{OPT} \leq B$. 

\begin{theorem} \label{thm: main_thm}
The cost of the algorithm is $O(k^5\cdot 3^k \cdot \textsf{OPT})$. 
\end{theorem}

The next theorem states that the algorithm never uses more than $k$ labels and therefore produces a feasible solution. The proof of this theorem is in Section~\ref{sec:invariant}.

\begin{theorem} \label{thm: under_k_labels}
The algorithm uses at most $k$ labels. 
\end{theorem}
 

\subsection{Notation and Definitions} \label{sec: analysis_defs}
We now define some notation and definitions used in the analysis. 
\begin{itemize}
    \item \textbf{Phase $t$} refers to the set of time steps during which there are exactly $t$ pivots. 
    \item $p_1^t, \dots, p_t^t$ denote the pivots for labels 1 through $t$, respectively, during phase $t$.
    \item $w^t$ denotes the natural weights at the end of phase $t$. 
    \item $X(t)$ is the set of points assigned a label before or during phase $t$.  
    \item For $j \in [t]$, let $C_j^t$ denote denote the points labelled $j$ in phases 1 through $t$. 
    \item An \textbf{intermediate phase} $t$ is a phase during which no points are given labels. This is a phase solely used to reset pivots. 
    \item A \textbf{non-intermediate phase} $T$ is a phase in which at least one point is given a label. 
\end{itemize}

For a non-intermediate phase $T$, 
    \begin{itemize}
        \item Let $T^-$ to denote the most recent non-intermediate phase before $T$, and $T^+$ to denote the first non-intermediate phase after $T$ (when these exist).\footnote{Using the notation in Section \ref{sec: algo}, $x_{i-1}$ is the last point labelled during phase $T$, and $x_i$ is the first point labelled during phase $T^+$.}
        \item For $j \in [T]$, $c_j^T$ is the estimated center (\ref{eq: good_center}) computed at the end of phase $T$. 
        \item Let $y_1, \dots, y_k$ be the optimal collection of $k$ centers computed at the end of phase $T$ in The Estimated Center Subroutine. Let $P_T = \{p_1^T, \dots, p_T^T, y_1, \dots, y_k\}$ and call this set the \textbf{offline centers} for phase $T$.\footnote{Note that $c_j^T$ and $P_T$ are only defined for non-intermediate phases $T$.}
        \item The \textbf{attachment digraph} $D(T)$ is a bipartite  digraph with vertex set $P_T$, plus $c_j^{T^-}$ if $T>1$, partitioned as $(\{p_1^T, \dots, p_T^T\}, \{y_1, \dots, y_k, c_j^{T^-}\})$. There is a directed arc $(y_i, p(y_i))$ if $w^T(y_i) \leq w^T(p(y_i))$ and a directed arc $(y_i, p(y_i))$ otherwise. If $c_j^{T^-}$ and $p_j^T$ are $\beta_{T+1}$-attached w.r.t. $w^T$, add the arc $(c_j^{T^-}, p_j^T)$ if $w^T(c_j^{T^-}) \leq w(p_j^T)$ and the arc $(p_j^T, c_j^{T^-})$ otherwise. $\delta^+(p_j^T)$ and $\delta^-(p_j^T)$ denote the out-degree and in-degree of $p_j^T$, respectively.
\end{itemize}

\subsection{Invariants} \label{sec:invariants}

The first property of our algorithm is that it maintains pivots that are sufficiently far apart with respect to their natural weights. This is at the heart of our analysis for both controlling the number of labels used and the algorithm's cost.

\begin{lemma} \label{lem: well-sep-invariant}
Let $t \in [k]$. The algorithm maintains the invariant that $p_1^t, \dots, p_t^t$ are $\beta_t$-well-separated w.r.t. the natural weights at the start of phase $t$ (and thereafter).\footnote{Two points well-separated at one time step will also be well-separated at a later time step, since their natural weights can only increase and the well-separation parameter $\beta_t$ can only decrease.}
\end{lemma}



The next lemma is a key technical lemma. It states that the estimated center (\ref{eq: good_center}) for the points given label $j$ \textit{before} phase $T$ is close, in a weighted sense, to the pivot for label $j$ in phase $T$. This is key to showing that points in cluster $j$ that are labelled \textit{before} phase $T$ can be combined with those that are labelled \textit{during} phase $T$ at bounded cost.  This lemma is in tension with the prior lemma because a pivot must be placed in a location where it is both well-separated from other pivots and is close to center of mass of the points of a given label.  

\begin{lemma} \label{lem: well_attached_centers}
Let $T$ be a non-intermediate phase and let $j \in [T]$. Let $w^t$ denote the natural weights at the end of phase $t$. If $T>1$, then at least one of the following holds:
\begin{enumerate}[label=(\alph*)]
    \item  $w^{T^-}(c_j^{T^-}) \leq w^T(p_j^T)$ and $w^{T^-}(c_j^{T^-}) \cdot d(c_j^{T^-}, p_j^T) \leq \beta_{T^-}(T-T^-) \cdot \textsf{OPT}$, \underline{or}
    \item $c_j^{T^-}$ is $\beta_{T+1}$-attached to $p_j^T$ w.r.t. $w^T$. 
\end{enumerate}
\end{lemma}


\subsection{Proof of the Algorithm's Feasibility and the Invariants}
\label{sec:invariant}

We begin by showing a bound on the number of well-separated points in the entire point set.    Lemma \ref{lem: well-sep-invariant} along with Proposition~\ref{prop: less_than_k_centers} below will immediately imply Theorem \ref{thm: under_k_labels}, which states that the algorithm uses at most $k$ labels. 

\begin{proposition}\label{prop: less_than_k_centers}
Let $X$ be a set of points whose optimal $k$-median cost using $k$ centers is $\textsf{OPT}$. Let $\{x_1, \dots, x_l\}$ be a set of points in $X$, and let $w_X$ denote their natural weights in $X$. Let $\beta > 8$. If $\{x_1, \dots, x_l\}$ is $\beta$-well-separated w.r.t. $w_X$, then $l \leq k$.
\end{proposition}

\begin{proof}[Proof of Proposition \ref{prop: less_than_k_centers}]
For shorthand, let $w_i = w_X(x_i)$. By Markov's inequality, for each $i \in [l]$ there must be at least $w_i/2$ points from $X$ inside $B(x_i, 2\textsf{OPT}/w_i)$. Consider a clustering on $X$ with cost $\textsf{OPT}$ using $k$ centers. Then each ball $B(x_i, 4\textsf{OPT}/w_i)$ must contain at least one of these $k$ centers; for, if not, then at least $w_i/2$ points inside $B(x_i, 2\textsf{OPT}/w_i)$ must each pay strictly more than $2\textsf{OPT}/w_i$ to reach a center. This contradicts that the optimal $k$-median cost is $\textsf{OPT}$. 

Now it remains to show, using the well-separation assumption, that these balls are disjoint; this will imply that $l \leq k$, since each ball must contain a center. Suppose to the contrary that there exist $i,j \in [l]$, $i \neq j$, such that $p \in B(x_i, 4\textsf{OPT}/w_i) \cap B(x_j, 4\textsf{OPT}/w_j)$. Applying the triangle inequality gives 
\begin{align*}
d(x_i, x_j) \leq d(p,x_i) + d(p,x_j) &\leq \frac{4\textsf{OPT}}{w_i} + \frac{4\textsf{OPT}}{w_j} \leq \frac{8\textsf{OPT}}{\min\{w_i, w_j\}}
\end{align*}
which contradicts that $\{(x_i, w_i), (x_j, w_j)\}$ is $\beta$-well-separated, i.e., that $\min\{w_i, w_j\} \cdot d(x_i,x_j) \geq \beta\textsf{OPT}$, since $\beta > 8$. 
\end{proof}
 
\medskip

The next two propositions will be used to aid the proofs of Lemmas \ref{lem: well-sep-invariant} and \ref{lem: well_attached_centers}. Recall that for each non-intermediate phase $T$, we defined a set of offline centers $P_T$ that has cost at most $2\textsf{OPT}$ on $X(T)$ (Section \ref{sec: analysis_defs}). In order to compare the (low-cost) offline clustering induced by $P_T$ to our online algorithm's clustering, we relate the offline set of centers $P_T$ (which we \textit{know} have bounded cost on $X(T)$) to the pivots in phase $T$ (which are used to make the greedy online choices) in the next proposition. 

\medskip

\begin{proposition} \label{prop: digraph_attachment}
Let $P_T = \{p_1^T, \dots, p_T^T, y_1, \dots, y_k\}$ be as in Section \ref{sec: analysis_defs}. Then $y_i$ and $p(y_i)$ are $\beta_{T+1}$-attached w.r.t. the natural weights $w^T$ at the end of phase $T$. 
\end{proposition}

\begin{proof}[Proof of Proposition \ref{prop: digraph_attachment}]
First we show that $y_i$ is $\beta_{T+1}$-attached to at least one of $p_1^T, \cdots, p_T^T$ w.r.t. $w^T$. Suppose not. Then an Add Operation would have been executed instead, and phase $T$ would have terminated in the previous time step, which is a contradiction. That $y_i$ is $\beta_{T+1}$-attached to $p(y_i)$ in particular follows directly from the definition (\ref{eq: p_arc}) of $p(y_i)$ as the pivot with the minimum weighted distance to $y_i$. 
\end{proof}

\medskip

 Observe that the proof of Proposition \ref{prop: digraph_attachment} is a direct consequence of the fact that we always execute an Add Operation or an Exchange Operation when one is available, so during a phase none are available. Each attached pair in Proposition \ref{prop: digraph_attachment} is encoded in the digraph $D(T)$ by a directed arc. Due to Proposition \ref{prop: digraph_attachment}, we can now think of this directed arc as representing the direction in which we could move a certain number of points sitting near one endpoint to the other endpoint at bounded cost.

 Algorithmically, we used $P_T$ to define the estimated centers $c_j^T$ (\ref{eq: good_center}). Next we show that the estimated center for a cluster at the end of a phase is attached to the pivot for that cluster in that phase. Thus, while the pivot itself may not be a good center for the cluster, the pivot is close to the estimated center (in at least one direction, in a weighted sense). 

 \medskip


\begin{proposition} \label{prop: attached_estimated_center}
    The estimated center $c_j^T$ is $\beta_{T+1}$-attached to $p_j^T$ w.r.t. the natural weights $w^T$ at the end of phase $T$. Further, $w^T(c_j^T) \geq w^T(p_j^T)$, with equality if and only if $c_j^T = p_j^T$. 
\end{proposition} 



The proof of Proposition \ref{prop: attached_estimated_center} is a straightforward consequence of Proposition \ref{prop: digraph_attachment} and the definition (\ref{eq: good_center}) of estimated center; for completeness, it can be found in Appendix \ref{appendix: omitted_proofs_invariant}.

Using these propositions, we establish Lemma~\ref{lem: well-sep-invariant}, used heavily in our analysis. The full proof is involved, so we defer it to Appendix \ref{appendix: omitted_proofs_invariant} and provide a sketch of the key ideas here. 

\medskip

\begin{proof}[Proof sketch of Lemma \ref{lem: well-sep-invariant}.]
The proof is by induction. However, we need to couple the induction with a statement about the relative position of the estimated center for a cluster (which stays fixed between intermediate phases) to that cluster's pivot, which may change often as we consecutively reset the pivots between intermediate phases. Roughly, we prove below that if the estimated center for cluster $j$ has not separated entirely from the present set of pivots, then it must be close (in a weighted sense) to the present pivot for label $j$. 
\begin{restatable}{proposition}{innerinduct}\label{prop: intermediate_attachment}
Let $w_{i-1}$, $w_i$, and $w_t$ be as Section \ref{sec: add_op}. For each $j \in [T]$ and $t \in [T, T^+]$ such that $p_1^t, \cdots, p_t^t$ are defined,\footnote{Recall in Case 4 of the Add Operation and Case 5 of the Exchange Operation, we go directly from $t$ to $t+2$ pivots, skipping phase $t+1$.} 
\begin{equation} \label{eq: well-sep-inner-induction}
p_1^t, \dots, p_t^t \text{ are } \beta_{t}\text{-well-separated w.r.t. } w_t. \tag{$\Diamond$}
\end{equation}
Moreover, at least one of the following properties holds:
\begin{enumerate}[label=(\alph*)]
\item $c_j^T$ is $\beta_{t+1}$-well-separated from $p_1^t, \dots, p_t^t$ w.r.t. $w_i$.
\item $c_j^T$ is $\beta_{t+1}$-attached to $p_j^t$ w.r.t. $w_t$. 
\item $c_j^T$ is $f(t,T)$-attached to $p_j^t$ w.r.t. $w_t$ and $w_t(c_j^T) < w_t(p_j^t)$, where $f(t,T) = {\beta_T \cdot (t-T)}$.
\end{enumerate}
\end{restatable}

For the proof sketch we focus on Case 4 of the Add Operation, which will give a flavor of the arguments. This is a concerning case a priori; for, if we were to add $x_{\alpha}$ to the set of pivots as in Cases 2 and 3, it is ambiguous as to whether $x_{\alpha}$ should be associated with label $f$ or $g$, as both $c_f^T$ and $c_g^T$ are close to $x_{\alpha}$ (see Appendix \ref{appendix: label_conflicts} for a diagram). We maneuver around the issue by making $c_f^T$ and $c_g^T$ new pivots and excluding $x_{\alpha}$. However, it is not immediately clear that such a step will preserve the desired invariants. To give intuition, we suppress the separation parameters and the precise weights used, though emphasize both are brittle (e.g., the arguments rely heavily on $\beta_t$ decreasing with $t$, see also Appendix \ref{appendix: decreasing_well_sep}). The directions of attachment between points (arrows in Figure \ref{fig: pro3}) are also crucial.  We will also see why we need to couple the induction with (a)---(c). 

To prove the inductive step for (\ref{eq: well-sep-inner-induction}) when Case 4 of the Add Operation is performed,  we need to show (i) $c_f^T$ and $c_g^T$ are well-separated, (ii), WLOG, $c_f^T$ is well-separated from $p_f^t$, and (iii), WLOG, $c_f$ is well-separated from $p_l^t$, $l \neq f$. See Figure \ref{fig: pro3}. When we say ``close'' or ``far'' below, we always mean in a weighted sense. For (i), because $p_f^T$ is close to $c_f^T$ (Proposition \ref{prop: attached_estimated_center}) and likewise for $p_g^T$, $c_g^T$, then $c_f^T$ and $c_g^T$ cannot be close, since this would violate that $p_f^T$ and $p_g^T$ are (inductively) far. To prove (ii), note $x_{\alpha}$ is far from $p_f^t$ by assumption of the Add Operation, and $c_f^T$ is close to $x_{\alpha}$ by assumption of Case 4, so $p_f^t$ and $c_f^T$ must be far. Finally for (iii), one can (inductively) deduce that (b) must hold when $j=f$, so $c_f^T$ and $p_f^t$ are close; but, since $p_f^t$ and $p_l^t$ are (inductively) far, $c_f^t$ and $p_l^t$ must be far. 

Proving the inductive step for (a)---(c) involves detailed casework. The Add and Exchange Operations are engineered so that, loosely speaking, an estimated center is either attached to the corresponding present pivot, or else breaks off to form its own pivot. A main subtlety is the direction and strength of attachment, e.g., property (c). Another is the sequence of operations, specifically, the Add Operation taking precedence over the Exchange Operation. 
\end{proof}

\begin{figure}[ht]
\centering
\includegraphics[width=0.7\textwidth]{figures/prop3.png}
\captionsetup{width=.9\linewidth}
\caption{Cases (i)---(iii) in the proof sketch of Lemma \ref{lem: well-sep-invariant}. Dashed lines indicate well-separation and solid lines indicate attachment, labelled with the appropriate parameters. Arrows go from smaller to larger natural weights.}
\label{fig: pro3}
\end{figure}




 


Theorem \ref{thm: under_k_labels} follows from Lemma \ref{lem: well-sep-invariant} and Proposition \ref{prop: less_than_k_centers} once we observe that we have set $\beta_1$ sufficiently large. For completeness, we include the proof below. 


\begin{proof}[Proof of Theorem \ref{thm: under_k_labels}]
The number of labels used by the algorithm is the number of pivots in the last phase. By Lemma \ref{lem: well-sep-invariant}, we maintain the invariant that pivots $p_1^t, \dots, p_t^t$ are $\beta_t$-well-separated w.r.t. the natural weights at every time step in phase $t$. Suppose to the contrary that the final number of pivots is strictly more than $k$. Then at some point there are $t = k+1$ or $t = k+2$\footnote{The algorithm may skip a phase, hence we consider both cases.} pivots that are $\beta_t$-well-separated w.r.t. the natural weights throughout phase $t$. But $\beta_{k+2} = 8$, and it is impossible for $k+2$ points to be 8-well-separated, by Proposition \ref{prop: less_than_k_centers}. We conclude the final number of pivots is at most $k$, so the algorithm uses at most $k$ labels.
\end{proof}

\medskip


As the proof of Proposition \ref{prop: intermediate_attachment} shows,  the Add and Exchange operations are engineered so that the estimated center $c_j^{T^-}$ is close to $p_j^T$ at the beginning of phase $T$. Lemma \ref{lem: well_attached_centers} states that this property is maintained through the end of phase $T$. In essence, this is because no Add or Exchange operations are executed during phase $T$, so we can show that the attachment between $c_j^{T^-}$ and $p_j^T$ is static---even as natural weights increase. The proof is below. 

\medskip

\begin{proof}[Proof of Lemma \ref{lem: well_attached_centers}]
We need to show that: either $w^T(c_j^T) \leq w^{T^+}(p_j^{T^+})$ and $w^{T}(c_j^{T}) \cdot d(c_j^{T}, p_j^{T^+}) \leq \beta_T(T^+ - T) \cdot \textsf{OPT}$, \underline{or} $c_j^{T}$ is $\beta_{T^+ + 1}$-attached to $p_j^{T^+}$ w.r.t. $w^{T^+}$.

We need the second part of Proposition \ref{prop: intermediate_attachment}. When we reach the start of phase $T^+$, there are no Add Operations or Exchange Operations available, so (a) in the statement of Proposition \ref{prop: intermediate_attachment} cannot hold when $t = T^+$. Thus either (b) or (c) must hold.  

Let $w_{T^+}$ denote the natural weights at the \textit{start} of phase $T^+$. Note this notation is consistent with taking $t=T^+$ in $w_t$ in Proposition \ref{prop: intermediate_attachment}.  

\setcounter{case}{0}

\begin{case} \label{case: cs1-attachment-lemma}
In Proposition \ref{prop: intermediate_attachment}, (c) holds. 
\end{case}
If (c) holds, then 
\[ w^T(c_j^T) \leq w_{T^+}(c_j^T) < w_{T^+}(p_j^{T^+}) \leq w^{T^+}(p_j^{T^+}), \hspace{0.3cm} \text{and}\] 
\[w^T(c_j^T) \cdot d(c_j^T, p_j^{T^+}) \leq w_{T^+}(c_j^T) \cdot d(c_j^T, p_j^{T^+}) \leq \beta_T(T^+ - T) \cdot \textsf{OPT}\]
where the second inequality in the first line and the last inequality in the second line follow from (c) holding in Proposition \ref{prop: intermediate_attachment}. 

\begin{case} 
In Proposition \ref{prop: intermediate_attachment}, (c) does not hold. 
\end{case}
If (c) does not hold, then $w_{T^+}(c_j^T) \geq w_{T^+}(p_j^{T^+})$ and $c_j^T$ is $\beta_{{T^+}+1}$-attached to $p_j^{T^+}$ w.r.t. $w_{T^+}$, i.e., the weights at the \textit{beginning} of phase $T^+$ (since $\beta_{T^+ +1} < \beta_T(T^+ - T)$). We need to show that $c_j^T$ is $\beta_{{T^+}+1}$-attached to $p_j^{T^+}$ w.r.t. $w^{T^+}$, i.e., the weights at the \textit{end} of phase $T^+$. 
Recall that through the end of phase $T^+$, $c_j^T$ remains $\beta_{{T^+}+1}$-attached to at least one of the ${T^+}$ pivots (otherwise, the phase would terminate and an Add Operation would be executed). So it just remains to show that at the end of phase ${T^+}$, $c_j^T$ is still $\beta_{{T^+}+1}$-attached to $p_j^{T^+}$ in particular, w.r.t. $w^{T^+}$. Suppose to the contrary that $c_j^T$ is $\beta_{{T^+}+1}$-attached to $p_{j'}^{T^+}$ w.r.t. $w^{T+}$, where $j' \neq j$. Then it must be the case that $c_j^T$ is $\beta_{{T^+}+1}$-attached to $p_{j'}^{T^+}$ w.r.t. $w_{T+}$. But we also know that $w_{T^+}(c_j^T) \geq w_{T^+}(p_j^{T^+})$ and $c_j^T$ is $\beta_{{T^+}+1}$-attached to $p_j^{T^+}$ w.r.t. $w_{T^+}$. By Proposition \ref{prop: meta-prop}, this contradicts that $p_{j}^{T^+}$ and $p_{j'}^{T^+}$ are $\beta_{T^+}$-well-separated w.r.t. $w_{T^+}$.

\end{proof}

Having established Lemma \ref{lem: well_attached_centers}, we are ready to bound the cost of the algorithm, using the present pivot $p_j^T$ as the ``bridge'' between old and newly arriving points given label $j$. We will show inductively that, at bounded cost, we can move the old points to $c_j^{T^-}$, which is in some sense close to $p_j^T$ by Lemma \ref{lem: well_attached_centers}. In turn, $p_j^T$ dictates the greedy choices for the new points. So we will combine the cost of old and new points via $p_j^T$. 







\section{Bounding the Algorithm's Cost} \label{sec:cost}

Throughout, let $cost(S; c) = \sum_{p \in S} d(p,c)$ for $S \subseteq X$ and $c \in X$.



As a first step, we need to bound the cost contribution of points that arrive during a single phase. The strategy is to compare the online greedy choices with the offline optimal choices, and to show these are sufficiently similar. More specifically, we know that in $D(T)$ each offline optimal center $y_i$ in $P_T$ is in the neighborhood of exactly one pivot, namely $p(y_i)$, and $y_i$ and $p(y_i)$ are close in a weighted sense, i.e., attached (Proposition \ref{prop: digraph_attachment}).  We further know that  since no Exchange Operations are executed during a phase, we can show that if $y_{i_1}$ and $y_{i_2}$ are in the neighborhood of the same pivot ($p(y_{i_1}) = p(y_{i_2})$), then $y_{i_1}$ and $y_{i_2}$ are also close in a weighted sense. 

Using these facts, we would be in good shape if we could show that for every point arriving during phase $T$, it is the case that if the point is assigned to $y_i$ in the offline optimal solution, then it receives the label of pivot $p(y_i)$ online. While this is not quite true, we can instead show that the number of points that do not satisfy this condition is small relative to the natural weights of their pivot, owing to the well-separated invariant (Lemma \ref{lem: well-sep-invariant}). Further, we show that these points can still be moved to their pivots at bounded cost, due to the greedy labelling rule. In effect, we will \textit{charge} the cost of these ``far'' points to their pivot. Lemma \ref{lem: far_points} summarizes this argument, and is used to prove the main theorem, Theorem \ref{thm: main_thm}. 


\begin{lemma} \label{lem: far_points}
Let $T$ be a non-intermediate phase. For any $j \in [T]$, let $C_j$ be the points given label $j$ during phase $T$, i.e., $C_j = C_j^T \setminus C_j^{T^-}$. Define $S_{ji}$ to be be the set of elements in $C_j$ assigned to $y_i$ in the clustering of $X(T) \setminus X(T^-)$ induced by $P_T$. Define $S_{far, j} = \bigcup_{i : p(y_i) \neq p_j^T} S_{ji}$. Then 
\begin{enumerate}
\item $cost(S_{far,j}; p_j^T) \leq k \cdot (\beta_{T+1} + 2) \cdot \textsf{OPT}$, and 
\item $|S_{far,j}| \leq k \cdot w^T(p_j^T)$, where $w^T$ denotes the natural weights at the end of phase $T$.
\end{enumerate}
\end{lemma}

The following lemma states that the number of points in a cluster by the end of any phase is a bounded factor away from the natural weight (at the end of the phase) of the estimated center for that cluster at the end of the phase. 

\begin{lemma} \label{lem: sufficiently_weighted_centers}
Let $T$ be a non-intermediate phase and $j \in [T]$. Let $w^T(c_j^T)$ denote the natural weight of $c_j^T$ at the end of phase $T$ and let $C_j^T$ denote the set of points in cluster $j$ by the end of phase $T$. Then 
$$|C_j^T| \leq (2k+1)\cdot T \cdot w^T(c_j^T).$$


\end{lemma}


\begin{lemma}\label{lem: cross_phase_bounded_cost} 
Let $T$ be a non-intermediate phase and $j \in [T]$. Then  $cost(C_j^T)$ is bounded against center $c_j^T$, i.e., 
\[ \sum_{x \in C_j^T} d(x, c_j^T) \leq g(T,k) \cdot \textsf{OPT}, \hspace{0.2cm}
g(T,k) = T \cdot g(k),  \hspace{0.1cm} g(k) = \beta_1(2k^3 + 3k^2 +5k+1) + 2k + 4.\] 
\end{lemma}

As a corollary to Lemma \ref{lem: cross_phase_bounded_cost}, we have Theorem~\ref{thm: main_thm}, the main theorem. 




Broadly, here is how the Lemmas \ref{lem: well_attached_centers}, \ref{lem: far_points}, and \ref{lem: sufficiently_weighted_centers} will tie together to prove Lemma \ref{lem: cross_phase_bounded_cost}. The proof of Lemma \ref{lem: cross_phase_bounded_cost} is by induction. Inductively, the points in cluster $j$ that arrived before phase $T$, which we call $C_j^{T^-}$, can be moved to their estimated center $c_j^{T^-}$ at bounded cost. This estimated center is close to $p_j^T$ by Lemma \ref{lem: well_attached_centers}. For instance, (a) in Lemma \ref{lem: well_attached_centers} says that once the points in $C_j^{T^-}$ are moved to $c_j^{T^-}$, they can also be moved to $p_j^T$ at bounded cost, since $w^{T^-}(c_j^{T^-}) \cdot d(c_j^{T^-}, p_j^T)$ is bounded and $|C_j^{T^-}| \leq h(T^-, k) \cdot w^{T^-}(c_j^{T^-})$ (Lemma \ref{lem: sufficiently_weighted_centers}). Finally, since $w^{T^-}(c_j^{T^-}) \leq w^T(p_j^T)$, we will be able to \textit{charge} the points in $C_j^{T^-}$ to $p_j^T$ in order to move them to $c_j^T$, which is attached to $p_j^T$ (Proposition \ref{prop: attached_estimated_center}). By similar logic, the cost of the far points $S_{far, j}$ from phase $T$ can be charged to $p_j^T$ and then moved to $c_j^T$ (Lemma \ref{lem: far_points}). Finally, the remaining points given label $j$ in phase $T$, call them $S_{near,j}$, are close to offline centers in $P_T$ that are in turn close to $p_j^T$. Crucially, no Exchange Operation is executed during a phase, so these offline centers are also close to $c_j^T$ in a weighted sense. 


\subsection{Proofs for bounding cost} \label{sec: proofs_bounded_cost}

We start with the proof of Lemmas \ref{lem: far_points}, which is a key step where the greedy rule for assigning labels to points is used. 

\begin{proof}[Proof of Lemma \ref{lem: far_points}]
 WLOG, let $j = T$. For $c \in P_T$, let $m(c)$ be the number of points assigned to $c$ in the clustering of $X(T) \setminus X(T^-)$ induced by the centers $P_T$, i.e., in this clustering every point is assigned to the \textit{nearest} point in $P_T$. 
 
 For shorthand, let $w$ denote the natural weights $w^T$ of points at the end of phase $T$. 


\begin{observation} \label{obs1}
For $c \in P_T$, $w(c) \geq m(c)$. 
\end{observation}

This follows from the definition of $w(c)$ and the fact that there are $m(c)$ points whose movement cost to $c$ is at most $2\textsf{OPT}$, by construction of $P_T$. 


\begin{observation} \label{obs2}
If $(p(y_i), y_i)$ is a directed edge in $D(T)$, then $w(p(y_i)) \cdot d(p(y_i), y_i) < \beta_{T+1} \cdot \textsf{OPT}$. Likewise, if $(y_i, p(y_i))$ is a directed edge in $D(T)$, then $w(y_i) \cdot d(p(y_i), y_i) < \beta_{T+1} \cdot  \textsf{OPT}$. 
\end{observation}
This follows from the definition of $D(T)$ and Proposition \ref{prop: digraph_attachment}. 

 Call the points in $S_{far,T}$ \textit{far} points. In the claims below, we show that the far points can be moved to $p_T^T$ at bounded cost (Claims \ref{clm: phase-wise_bdd_clm1} and \ref{clm: phase-wise_bdd_clm2}), and that there are not too many far points relative to the weight of $p_T^T$ (Claim \ref{clm: phase-wise_bdd_clm3}). In turn, we will be able to \textit{charge} the cost of the far points to $p_T^T$.   

\setcounter{claim}{0}
\begin{claim} \label{clm: phase-wise_bdd_clm1}
Let $p(y_i) \neq p_T^T$. Suppose $w(y_i) > w(p(y_i))$. Then $cost(S_{Ti};p_T^T) \leq {(\beta_{T+1} + 2)}\textsf{OPT}$.  
\end{claim}



\begin{proof}
WLOG, let $p(y_i) = p_1^T$. We consider two cases. 

\setcounter{case}{0}

\begin{case} \label{cs1}
$|S_{Ti}| \geq w(p_1^T)$. We will show this case cannot happen.
\end{case}
In this case, we know that $w(y_i) \geq m(y_i) \geq |S_{Ti}|\geq w(p_1^T)$. We know by Observation \ref{obs2} that ${w(p_1^T) \cdot d(p_1^T, y_i) < \beta_{T+1}\cdot \textsf{OPT}}$. By Proposition \ref{prop: meta-prop}, this implies $w(p_1^T) \cdot d(y_i, p_T^T) \geq 2\beta_{T+1} \cdot \textsf{OPT}$. 

Since $|S_{Ti}| \geq w(p_1^T)$, there exists $S_{Ti}' \subseteq S_{Ti}$ such that $|S_{Ti}'| = w(p_1^T)$. In turn, $cost(S_{Ti}'; p_1^T) \leq cost(S_{Ti}'; y_i) + w(p_1^T) \cdot d(y_i,p_1^T) <  (\beta_{T+1} + 2)\cdot \textsf{OPT}$, since $P_T$ is a clustering with cost at most $2\textsf{OPT}$. On the other hand, 
$$cost(S_{Ti}'; p_T^T) \geq \sum_{p \in S_{Ti}'} d(y_i, p_T^T) - \sum_{p \in S_{Ti}'} d(p, y_i) = w(p_1^T) \cdot d(y_i, p_T^T) - \sum_{p \in S_{Ti}'} d(p,y_i) \geq (2\beta_{T+1} - 2)\textsf{OPT}.$$ 

Since $\beta_{T+1} \geq 4$, $\beta_{T+1} +2 \leq 2\beta_{T+1} - 2$, so $cost(S_{Ti}';p_1^T) < cost(S_{Ti}'; p_T^T)$, which violates that $T = \arg \min_{j \in [T]} d(p, p_j^T)$ for all $p \in S_{Ti}' \subseteq C_T$. 


\begin{case}
$|S_{Ti}| \leq w_t(p_1^T)$. 
\end{case}
In this case, we know that since $w(p_1^T) \cdot d(y_i, p_1^T) < \beta_{T+1} \cdot \textsf{OPT}$, we also have $|S_{Ti}| \cdot d(y_i, p_1^T) < \beta_{T+1} \cdot \textsf{OPT}$. By the triangle inequality, 
\[cost(S_{Ti};p_1^T) \leq cost(S_{Ti};y_i) + |S_{Ti}|\cdot d(y_i, p_1^T) \leq 2\textsf{OPT} + \beta_{T+1} \cdot \textsf{OPT}. \]
Since $cost(S_{Ti};p_T^T) \leq cost(S_{Ti};p_1^T)$ by the greedy procedure, this proves Claim \ref{clm: phase-wise_bdd_clm1}. 
\end{proof}


\begin{claim} \label{clm: phase-wise_bdd_clm2}
Let $p(y_i) \neq p_T^T$. Suppose that $w(y_i) \leq w(p(y_i))$. Then $cost(S_{Ti};p_T^T) \leq {(\beta_{T+1} +1)}\textsf{OPT}$.
\end{claim}

\begin{proof}
WLOG, let $p(y_i) = p_1^T$. By Observation \ref{obs2}, $w(y_i)\cdot d(y_i, p_1^T) < \beta_{T+1} \cdot \textsf{OPT}$. Further, $|S_{Ti}| \leq m(y_i) \leq w(y_i)$, so $|S_{Ti}|\cdot d(y_i,p_1^T) < \beta_{T+1} \cdot \textsf{OPT}$. So:
$$cost(S_{Ti};p_T^T) \leq cost(S_{Ti};p_1^T) \leq cost(S_{Ti}; y_i) + |S_{Ti}| \cdot d(y_i, p_1^T) \leq 2 \textsf{OPT} + \beta_{T+1} \cdot \textsf{OPT}.$$
\end{proof}

\begin{claim} \label{clm: phase-wise_bdd_clm3}
 Let $p(y_i) \neq p_T^T$. Then $|S_{Ti}| \leq w(p_T^T)$.
\end{claim}

\begin{proof}

\setcounter{case}{0}

As before, assume WLOG that $p(y_i) = p_1^T$.
\begin{case} \label{cs3}
$w(y_i) > w(p_1^T)$. 
\end{case}
We know from the proof of Claim \ref{clm: phase-wise_bdd_clm1}, Case \ref{cs1} that this implies $|S_{Ti}| < w(p_1^T)$. We have 
\begin{align*}
    |S_{Ti}| \cdot d(p_T^T, y_i) &= \sum_{p \in S_{Ti}}d(y_i, p_T^T) \\
    &\leq \sum_{p \in S_{Ti}} d(p,p_T^T) + \sum_{p \in S_{Ti}} d(p, y_i) \\
    &\leq (\beta_{T+1} + 2)\textsf{OPT} + 2\textsf{OPT} \tag{\text{Claim \ref{clm: phase-wise_bdd_clm1}}}\\
    &\leq 2\beta_{T+1} \cdot \textsf{OPT} \\ 
    &\leq w(p_T^T) \cdot d(p_T^T, y_i) 
\end{align*}
where in the last line we have applied Proposition \ref{prop: meta-prop}, using that $w(y_i) > w(p_1^T)$, Observation \ref{obs2}, and $p_1^T$ and $p_T^T$ are $\beta_T$-well-separated w.r.t. $w$. Finally, dividing both ends of the chain of inequalities by $d(p_T^T, y_j)$ gives $|S_{Tj}| \leq w(p_T^T)$, as desired. 

\begin{case}
$w(y_i) \leq w(p_1^T)$.
\end{case}
First consider when $w(p_T^T) \geq w(y_i)$. Then $w(p_T^T) \geq w(y_i) \geq m(y_i) \geq |S_{Ti}|$, so the claim follows. 

So the last case to consider is when $w(p_T^T) < w(y_i)$. It suffices to show that $w(p_T^T) \cdot d(p_T^T, y_i) \geq 2\beta_{T+1} \cdot \textsf{OPT}$; then, we can just apply the argument in Case \ref{cs3}. Suppose to the contrary that $w(p_T^T) \cdot d(p_T^T, y_i) < 2\beta_{T+1} \cdot \textsf{OPT}$. Then 
\begin{align*}
    \beta_T \cdot \textsf{OPT} &\leq  w(p_T^T) \cdot d(p_T^T, p_1^T) \\
    &\leq w(p_T^T) \cdot d(p_T^T, y_i) + w(p_T^T) \cdot d(y_i, p_1^T) \\
    &\leq 2\beta_{T+1} \cdot \textsf{OPT} +  w(p_T^T) \cdot d(y_i, p_1^T) \\
    &< 2\beta_{t+1} \cdot \textsf{OPT} + w(y_i) \cdot d(y_i, p_1^T) \\
    &< 2\beta_{T+1} \cdot \textsf{OPT} + \beta_{T+1} \cdot \textsf{OPT} \\
    &= \beta_T \cdot \textsf{OPT}
\end{align*}
where the second-to-last line follows from Observation \ref{obs2}. The left-hand and right-hand sides give a contradiction, concluding the proof of the case and the claim. 
\end{proof}


\begin{claim} \label{clm4}
$cost(S_{far,T};p_T^T) \leq k \cdot (\beta_{T+1} + 2)\textsf{OPT}$ and $|S_{far,T}| \leq k \cdot w(p_T^T)$. 
\end{claim}

\begin{proof}
By Claims \ref{clm: phase-wise_bdd_clm1} and \ref{clm: phase-wise_bdd_clm2}, 
$$cost(S_{far,T};p_T^T) = \sum_{i:p(y_i) \neq p_T^T} cost(S_{Ti};p_T^T) \leq k \cdot (\beta_{T+1} + 2)\textsf{OPT}$$
By Claim \ref{clm: phase-wise_bdd_clm3}, 
$$|S_{far,T}| = \sum_{i: p(y_i) \neq p_T^T} |S_{Ti}| \leq k\cdot w(p_T^T).$$
\end{proof}
\end{proof}


The proof of Lemma \ref{lem: sufficiently_weighted_centers} can be found in Appendix \ref{appendix: proofs_bounded_cost}. The argument is by induction and is similar in flavor to the proof of Lemma \ref{lem: cross_phase_bounded_cost}, which we give below. Lemma \ref{lem: cross_phase_bounded_cost} implies Theorem \ref{thm: main_thm}. 
\begin{proof}[Proof of Lemma \ref{lem: cross_phase_bounded_cost}]
 The proof is by induction. Let $C_j$, $S_{ji}$, and $S_{far, j}$ be as in Lemma \ref{lem: far_points}. Define $S_{near, j} = \bigcup_{i : p(y_i) = p_j^T} S_{ji}$ and $S_j$ to be the elements in $C_j$ that are assigned to $p_j^T$ in the clustering of $X(T) \setminus X(T^-)$ induced by $P_T$. Let $w^t$ denote the natural weights at the end of phase $t$. First we need the following key claim. 

 \setcounter{claim}{0}
\begin{claim} \label{clm: main_lem_clm1}
For any $x,y \in \delta^+(p_j^T) \cup \delta^-(p_j^T) \cup \{p_j^T\}$, $x$ and $y$ are $2\beta_{T+1}$-attached w.r.t. $w^T$. 
\end{claim}

\begin{proof}[Proof of Claim \ref{clm: main_lem_clm1}]
If $x$ or $y$ is $p_j^T$, then the claim automatically holds by Proposition \ref{prop: digraph_attachment}. There are two other cases. The first case is, WLOG, $x \in \delta^-(p_j^T)$. Regardless of whether $y$ is in $\delta^{-}(p_j^T)$ or $\delta^{+}(p_j^T)$, the claim holds by Propositions \ref{prop: digraph_attachment} and \ref{prop: meta-prop}. The second case is that $x,y \in \delta^{+}(p_j^T)$. We prove the stronger statement that $x$ and $y$ are $\beta_{T+1}$-attached w.r.t. $w^T$. Suppose to the contrary that $x$ and $y$ are $\beta_{T+1}$-well-separated. We claim that this implies 
\begin{equation}\label{add_swap_contr}
\{p_1^T, \dots, p_T^T\} \cup \{x, y \} \setminus \{p_j^T\}
\end{equation}
is $\beta_{T+1}$-well-separated w.r.t. $w^T$; this would give a contradiction, since if an Exchange Operation were available, it would have been executed. Now suppose that (\ref{add_swap_contr}) does not hold. Then WLOG $p_{j'}^T$ and $x$ are $\beta_{T+1}$-attached w.r.t. $w^T$, for some $j' \neq j$. Since $x \in \delta^+(p_j^T)$ and since $x$ and $p_j^T$ are $\beta_{T+1}$-attached w.r.t. $w^T$, by Proposition \ref{prop: meta-prop}, $p_j^T$ and $p_{j'}^T$ are $2\beta_{T+1}$-attached w.r.t. $w^T$. This contradicts that $p_j^T$ and $p_{j'}^T$ are $\beta_T$-well-separated w.r.t. $w^T$, since $2\beta_{T+1} < \beta_T$. This concludes the proof of the case and the claim.
\end{proof}
To bound the cost contribution of $C_j^{T^-}$, we case on which statement holds in Lemma \ref{lem: well_attached_centers}.

\setcounter{case}{0}
\begin{case} \label{cs: main_lem_cs1}
$c_j^{T^-}$ is $\beta_{T+1}$-attached to $p_j^T$ w.r.t. $w^T$ (i.e., (b) holds in Lemma \ref{lem: well_attached_centers}). 
\end{case}


Since in Case \ref{cs: main_lem_cs1}, $c_j^{T^-}$ is $\beta_{T+1}$-attached to $p_j^T$ w.r.t. $w^T$, $c_j^{T^-} \in \delta^+(p_j^T) \cup \delta^-(p_j^T)$. Also, $c_j^T$ by definition is in $\delta^+(p_j^T) \cup \{p_j^T\}$. So by Claim \ref{clm: main_lem_clm1}, $c_j^{T^-}$ is $2\beta_{T+1}$-attached to $c_j^T$ w.r.t. $w^T$. Using this, we bound $cost(C_j^{T^-}; c_j^T)$:
\begin{align}
cost(C_j^{T^-}; c_j^T) &\leq cost(C_j^{T^-}; c_j^{T^-}) + |C_j^{T^-}| \cdot d(c_j^{T^-}, c_j^T) \notag \\
&\leq g(T^-, k) \cdot \textsf{OPT} + |C_j^{T^-}| \cdot d(c_j^{T^-}, c_j^T) \notag \\
&\leq  g(T^-, k) \cdot \textsf{OPT} + (2k+1) \cdot T^- \cdot w^{T^-}(c_j^{T^-}) \cdot d(c_j^{T^-}, c_j^T)  \notag \\
&\leq  g(T^-, k) \cdot \textsf{OPT} + (2k+1) \cdot T^- \cdot w^{T}(c_j^{T^-}) \cdot d(c_j^{T^-}, c_j^T)  \notag \\
&\leq g(T^-, k) \cdot \textsf{OPT} + (2k+1) \cdot T^- \cdot 2 \beta_{T+1} \cdot \textsf{OPT} \label{prev_cost_cs1}
\end{align}
where the third inequality is due to Lemma \ref{lem: sufficiently_weighted_centers}.



\begin{case} \label{cs: main_lem_cs2}
(b) does not hold in Lemma \ref{lem: well_attached_centers}, so (a) holds, i.e., $w^{T^-}(c_j^{T^-}) \leq w^T(p_j^T)$ and $w^{T^-}(c_j^{T^-}) \cdot d(c_j^{T^-}, p_j^T) \leq \beta_{T^-}(T-T^-) \cdot \textsf{OPT}$. 
\end{case}

We bound $cost(C_j^{T^-}; c_j^T)$:
\begin{align}
cost(C_j^{T^-}; c_j^T) &\leq cost(C_j^{T^-}; c_j^{T^-}) + |C_j^{T^-}| \cdot d(c_j^{T^-}, c_j^T) \notag \\
&\leq g(T^-, k) \cdot \textsf{OPT} + |C_j^{T^-}| \cdot d(c_j^{T^-}, p_j^T) + |C_j^{T^-}| \cdot d(p_j^T, c_j^T) \label{two_bounds}
\end{align}
and now we use the assumptions of the case to continue bounding from (\ref{two_bounds}):
\begin{align}
|C_j^{T^-}| \cdot d(c_j^{T^-}, p_j^T) &\leq (2k+1) \cdot T^- \cdot w^{T^-}(c_j^{T^-}) \cdot d(c_j^{T^-}, p_j^T) \notag \\
&\leq (2k+1) \cdot T^- \cdot \beta_{T^-} (T-T^{-}) \cdot \textsf{OPT} \label{bound_1}
\end{align}
where the first inequality is due to Lemma \ref{lem: sufficiently_weighted_centers}. Next, 
\begin{align}
|C_j^{T^-}| \cdot d(p_j^T, c_j^T) \leq (2k+1) T^- \cdot w^{T^-}(c_j^{T^-}) \cdot d(p_j^T, c_j^T) &\leq (2k+1)  T^- \cdot w^{T}(p_j^{T}) \cdot d(p_j^T, c_j^T) \notag \\
&\leq (2k+1) T^- \cdot \beta_{T+1} \cdot \textsf{OPT} \label{bound_2}
\end{align}
where the first inequality is due to Lemma \ref{lem: sufficiently_weighted_centers} and the last inequality is due to Proposition \ref{prop: attached_estimated_center}. So combining (\ref{two_bounds}), (\ref{bound_1}), (\ref{bound_2}) gives 
\begin{equation} \label{prev_cost_cs2}
    cost(C_j^{T^-}; c_j^T) \leq g(T^-, k) \cdot \textsf{OPT} + (2k+1) \cdot T^- \cdot (\beta_{T^-}(T-T^-) + \beta_{T+1}) \cdot \textsf{OPT}. 
\end{equation}
Now we have bounds (\ref{prev_cost_cs1}) and (\ref{prev_cost_cs2}) for $cost(C_j^{T^-}; c_j^T)$. Recall that  $C_j^T = C_j^{T^-} \cup S_{far, j} \cup S_{near, j} \cup S_j$. The following bounds will hold regardless of whether we are in Case \ref{cs: main_lem_cs1} or \ref{cs: main_lem_cs2}. We have
\begin{equation} \label{pivot_cost}
cost(S_j; c_j^T) \leq cost(S_j; p_j^T) + |S_j| \cdot d(p_j^T, c_j^T) \leq 2\textsf{OPT} + w^T(p_j^T) \cdot d(p_j^T, c_j^T) \leq (2 + \beta_{T+1})\textsf{OPT}
\end{equation}
\begin{align}
cost(S_{near,j}; c_j^T) &= \sum_{i: p(y_i) = p_j^T} cost(S_{ji}; c_j^T)
\leq \sum_{i: p(y_i) = p_j^T} \sum_{p \in S_{ji}} d(p, c_j^T) \notag \\
&\leq 2\textsf{OPT} + \sum_{i: p(y_i) = p_j^T} w^T(y_i)\cdot d(y_i, c_j^T)  \leq (2k\beta_{T+1} +2)\textsf{OPT} \label{near_cost}
\end{align}
where we have used Claim \ref{clm: main_lem_clm1} and that $|S_{ji}| \leq w^T(y_i)$. Finally, by Lemma \ref{lem: far_points},
\begin{align}
    cost(S_{far,j}; c_j^T) \leq cost(S_{far,j}; p_j^T) + |S_{far,j}| \cdot d(p_j^T, c_j^T) \leq k(2\beta_{T+1} +2) \textsf{OPT} \label{far_cost}
\end{align}
 Combining (\ref{pivot_cost}), (\ref{near_cost}), (\ref{far_cost}) with (\ref{prev_cost_cs1}) or (\ref{prev_cost_cs2}) gives the sought bound:
$$cost(C_j^T; c_j^T) \leq [g(T^-, k) + g(k)] \textsf{OPT} \leq g(T,k) \cdot \textsf{OPT}.$$
\end{proof}





