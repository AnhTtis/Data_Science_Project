%
\section{Preliminaries}
\label{sec_prelim}
In this section, we present the assumptions that we make on plant, controller and observer and recap some preliminaries that are needed for the proposed STC approach. 

\subsection{Hybrid Lyapunov functions}

In this subsection, we recap assumptions on the plant and the controller that are needed to construct a hybrid Lyapunov function and to derive a bound on it. We use the following assumption \cite[Assumption~1 for $w = 0$]{hertneck21robust_arxiv}, that is based on \cite[Assumption~1]{carnevale2007lyapunov}.
\begin{asum}
	\label{asum_hybrid_lyap}
	There exist a locally Lipschitz function $W:\mathbb{R}^{n_e} \rightarrow \mathbb{R}_{\geq0}$, a locally Lipschitz function $V:\mathbb{R}^{n_x} \rightarrow \mathbb{R}_{\geq0}$, a continuous function $H:\mathbb{R}^{n_x}\times\mathbb{R}^{n_e} \rightarrow \mathbb{R}_{\geq0}$, constants $L, \gamma\in \mathbb{R}_{>0}$, $\epsilon\in \mathbb{R}$, and  $\underline{\alpha}_W$, $\overline{\alpha}_W, \underline{\alpha}_V, \overline{\alpha}_V\in \mathcal{K}_\infty$  such that for all $e\in\mathbb{R}^{n_e}$,
	\begin{equation}
		\label{eq_w_bound}
		\underline{\alpha}_W(\abs{e}) \leq W(e) \leq \overline{\alpha}_W(\abs{e}),
	\end{equation}
	for all $x_p \in\mathbb{R}^{n_x}$,
	\begin{equation}
		\label{eq_V_bound_K}
		\underline{\alpha}_V(\abs{x_p}) \leq V(x_p) \leq \overline{\alpha}_V(\abs{x_p}),
	\end{equation}
	and for all  $x_p \in \mathbb{R}^{n_x} $ and almost all $e\in\mathbb{R}^{n_e},$ 
	\begin{equation}
		\left\langle \frac{\partial W(e)}{\partial e},g(x_p,e)\right\rangle \leq L W(e) + H(x_p,e).  \label{eq_w_est}
	\end{equation}
	Moreover, for all $e \in \mathbb{R}^{n_e}$ and almost all $x_p \in \mathbb{R}^{n_x}$,  
	\begin{equation}
		\begin{split}
			&\left\langle \nabla V(x_p),f(x_p,e) \right\rangle\\
			\leq &- \epsilon V(x_p) -H^2(x_p,e)
			+ \gamma^2 W^2(e).
		\end{split}
		\label{eq_v_desc_hybrid}
	\end{equation}
\end{asum}
%

A discussion of this assumption can be found in \cite{carnevale2007lyapunov}. Note that it involves only the plant and the controller and does not depend on the observer. We will specify conditions on the observer in the next subsection. 
%
Note also that Assumption~\ref{asum_hybrid_lyap} can hold simultaneously for different choices of $\epsilon, \gamma$ and $L$. If we can find one set of parameters for which the assumption holds, then we will typically also be able to find many different parameter sets.
%

To determine transmission times, the proposed STC framework will use a bound on the evolution of $V(x)$, that is adapted from \cite{hertneck21robust_arxiv}. This bound is based on the function
%
%
%
\begin{equation}
	T_{\max}(\gamma,\lvar ) \coloneqq \begin{cases}\vspace{1mm}
		\frac{1}{\lvar r} \arctan(r) & \gamma > \lvar \\ \vspace{1mm}
		\frac{1}{\lvar } & \gamma = \lvar \\
		\frac{1}{\lvar r} \arctanh(r) &\gamma < \lvar 
	\end{cases}
\end{equation}
where
%
%
	$r\coloneqq\sqrt{\abs{
			\left(\frac{\gamma}{\lvar }\right)^2-1}},$
%
that was originally used in \cite{nesic2009explicit} to determine the maximum allowable sampling interval for sampled-data systems. We use the following result from \cite{hertneck21robust_arxiv}.
%

\begin{prop}{\cite[Proposition~1 for $w = 0$]{hertneck21robust_arxiv}}.
	%
	\label{prop_hybrid}
	Consider any maximal solution $\xi$ to \eqref{eq_sys_hyb} at transmission time $\tvar_j^+$ for $j \in \change{\mathbb{N}_0}{\dom~\xi}$.  Let Assumption~\ref{asum_hybrid_lyap} hold 
	%
	for some $\gamma, \epsilon$ and $L$.
	%
	Moreover, let $0 < \auxvar(\tvar_j^+) < T_{\max} (\gamma,\max\left\lbrace L+\frac{\epsilon}{2},1-\delta\right\rbrace )$  for $\delta\in\left(0,1\right)$.	
	Consider 
	\begin{equation}
		\label{eq_def_u}
		U(\xi) \coloneqq V(x)+\gamma \phi(\tau) W^2(e),
	\end{equation}
	where 	 $\phi : [0,\auxvar(\tvar_j^+)] \rightarrow \mathbb{R}$ is the solution to
	\begin{equation}
		\label{eq_def_phi}
		\dot{\phi} = -2\max\left\lbrace L+\frac{\epsilon}{2},1-\delta\right\rbrace \phi-\gamma(\phi^2+1),~ \phi(0) = \lambda^{-1}
	\end{equation}
	for some sufficiently small $\lambda \in \left(0,1\right)$.
	Then, \change{}{$t_{j+1} =  \svar_j+\auxvar(\tvar_j^+) \in \dom~\xi$} and for all $\svar_j \leq t \leq \change{\svar_j+\auxvar(\tvar_j^+)}{t_{j+1}}$, it holds that	
	\begin{equation}
		\begin{split}
			V(x_p(\svar,j\change{+1}{}))
			\leq& U(\xi(\svar,j\change{+1}{}))
			\leq	e^{ -\epsilon (t-\svar_j)}U(\xi(\tvar_j^+)).\\
		\end{split}
		\label{eq_prop_hybrid1}
	\end{equation}
\end{prop}	
		
Proposition~\ref{prop_hybrid} yields an upper bound on the evolution of $U(\xi)$ for the parameters $\epsilon,\gamma$ and  $L$ that satisfy Assumption~\ref{asum_hybrid_lyap}. Thus, it also provides an upper bound on $V(x)$ due to $U(\xi) \geq V(x)$.
%
  This bound is valid, if the time between two transmissions is bounded by $T_{\max}(\gamma,\max\left\lbrace L-\frac{\epsilon}{2},1-\delta\right\rbrace)$. The actual bound depends on the parameters from Assumption~\ref{asum_hybrid_lyap}. Particularly, if $\epsilon > 0$, then the bound is exponentially decreasing over time. In contrast, if $\epsilon < 0$, then the bound is increasing. However, the admissible time between transmissions $T_{\max}(\gamma,L+\frac{\epsilon}{2})$ decreases when $\gamma$ or $\epsilon$ are increased.  
%
We thus observe in Proposition~\ref{prop_hybrid} a trade-off between the admissible time between transmissions and the growth of the bound on $V(x)$. Particularly, if the time between two successive transmissions is small, then we will be able to choose $\epsilon$ 
%
large and thus obtain an exponentially decreasing bound on $V(x)$. In contrast, if the time between two successive transmissions is large, then we need to choose
%
 $\epsilon$ small to be able to derive a bound on $V(x)$, which has the effect that this bound may be increasing. 
%
%
%
%

Note that $U(\xi(\tvar_j^+))$ depends on $e(\tvar_j^+)$ and thus on the observer error (whilst the network induced error has been reset to $0$ at time $s_j^+$). However, the observer error is not available to the STC mechanism and can thus not  be used to determine the next transmission time. Instead, the STC mechanism will use the value of $V(x_o)$ to determine transmission times. To obtain still stability guarantees, we will need an assumption on the observer which we introduce in the next subsection. 


\subsection{Bound on the observer error}
We assume that the observer is designed such, that the observer error $e_o = x_o-x_p$ is asymptotically stable. More formally, we make the following assumption.
\begin{asum}
	\label{as_obs_er}
	The observer~\eqref{eq_obs} is such that 
	\begin{equation}
		\label{eq_obs_er}
		\abs{x_o(t,j)-x_p(t,j)} = \abs{e_o(t,j)}  \leq \beta_o(\abs{e_o(0,0)},t)
	\end{equation}
	holds for $\beta_o\in\mathcal{K}\mathcal{L}$. 
\end{asum}
%

Assumption~\ref{algo_trig_window} requires the observer to be exponentially stable. Note that the design of observers for nonlinear continuous-time systems is not trivial but widely studied in the literature (see, e.g., \cite{bernard2022observer}) and thus beyond the scope of this paper. 
%
%
%
%
%
%
%
%
%
%
%
%
%
%
%
%
%
%
%
%
%
%
%
%
%
%
%
%
%
%
%
%
%
%