%
\section{Setup}
\label{sec_setup}
We consider a setup where the sensors and actuators of a continuous plant are connected through a communication network and exchange information only at discrete transmission times.
%
%
%
%
The plant is described by
\begin{equation}
	\label{eq_plant}
	\begin{split}
		\dot{x}_p &= f_p(x_p,\hat{u}),\\
		y &= g_p(x_p)
	\end{split}	
\end{equation}
where $x_p(t) \in \R^{n_x}$ is the plant state with initial condition  $x_p(0)$, $y(t)\in\R^{n_y}$ is the plant output and $\hat{u}(t) \in \R^{n_u}$ is the last input that has been received by the plant.

Only the output $y$ but not the whole plant state $x_p$ can be measured. Thus, an observer that is connected to the plant's sensors is used to reconstruct the plant state based on $y$. The observer is described\footnote{The observer in \eqref{eq_obs} uses continuous feedback. Note that for suitable system classes, a sampled-data observer as, e.g., in \cite{raff2008observer} could be used as well.} by 
\begin{equation}
	\label{eq_obs}
	\dot{x}_o = f_o(x_o,\hat{u},y),
\end{equation}
where $x_o(t)\in\R^{n_x}$ is the observer state.  

 The input is generated by the static state-feedback controller
 \begin{equation*}
%
 %
 	u = g_c(x_o)
 	%
 \end{equation*}
 using the observer state $x_o$.
In this paper, we consider an emulation approach, i.e., we suppose that the controller has been designed for state feedback for plant~\eqref{eq_plant} ignoring the network effects using any method for the design of nonlinear continuous state-feedback controllers with some robustness to the input errors.
%
%
%
%
The functions $f_p$ and $f_o$ are assumed to be continuous and the functions $g_p$ and $g_c$ are assumed to be continuously differentiable.  


The transmission times $(\svar_j)_{j\in\mathbb{N}_0}$ are determined by an STC mechanism to be specified later. 
At each transmission time, the input $\hat{u}$ is updated based on the current values of $g_c(x_o),$ i.e., $\hat{u}(\svar_j) = u(\svar_j) = g_c(x_o(t_j))$. Further, we denote by $\hat{x}_o$ the observer state associated to the last transmission time, i.e., $\hat{x}_o(\svar_j) = x_o(\svar_j)$.   Between transmission times, we assume that $\hat{u}$ (and thus $\hat{x}_o$) is kept constant, which resembles a zero-order-hold (ZOH) scenario. 
%
%
We introduce $e\coloneqq \hat{x}_o - x_p$ with $e(t) \in \R^{n_x}$ to describe  the sum of the observer induced error and the network induced error.
%
%
%
%

Similar as in \cite{hertneck21robust_arxiv}, we consider in this paper a dynamic STC mechanism that determines at transmission times $\svar_j$ the next transmission time $\svar_{j+1}$ based on information that is available to the STC mechanism at $t_j$ including an internal state $\eta$. The internal state  $\eta(t)\in\R^{n_\eta}$ is used to incorporate the past system behavior when deciding about the next transmission time. Since we consider an output feedback scenario, 
the plant state cannot be directly used to determine transmission times.
%
%
Instead, the proposed STC mechanism uses the observer state $x_o$ to determine transmission times.
%
It can thus be described by 
%
$\svar_{j+1} \coloneqq \svar_j + \Gamma(x_o(\svar_j),\eta(\svar_j)),$
%
where $\Gamma:\R^{n_x}\times\R^{n_\eta} \rightarrow \left[\svar_{\min},\infty\right)$ for some $\svar_{\min} > 0$. We will later provide an explicit value for $t_{\min}$ for the mechanism. 

The dynamic variable $\eta$ is updated at  transmission times based on its current value and the current observer output, and remains constant in between  transmission times. Thus, $\eta$ evolves according to 
\begin{equation}
	\begin{split}
		\eta(\svar_{j+1}) = S(\eta(\svar_j),x_o(\svar_j)),
		\\
		\dot{\eta}(t) = 0, t\in\left[\svar_j,\svar_{j+1}\right)
	\end{split}
\end{equation}
for some $\eta(0)$, where $S:\R^{n_\eta}\times\R^{n_x} \rightarrow \R^{n_\eta}.$ 


In order to model the overall networked control systems as a hybrid system, we introduce the timer variable $\tau$ which keeps track of the elapsed time since the last  transmission time and the auxiliary variable $\auxvar$ which encodes the next  transmission time. Using this, we obtain
\begin{equation}
\label{eq_sys_hyb}
	\begin{cases}
		\dot{\xi} = F(\xi), & \xi \in C,\\
		\xi^+ = G(\xi), & \xi \in D,
	\end{cases}
\end{equation}
with $\xi \coloneqq \left[x_p^\top,x_o^\top,e^\top,\eta^\top,\tau,\auxvar\right]^\top,$ 
%
	$F(\xi) \coloneqq \linebreak\left(f(x_p,e)^\top,f_o(x_o,g_c(x_p+e),g_p(x_p)),g(x,e)^\top,0,1,0\right)^\top,$
%
\linebreak where
%
$	f(x_p,e) = f_p(x_p,g_c(x_p+e)) $
%
%
%
%
%
and $g(x,e) = -f(x,e)$,
%
%
%
	$G(\xi) \coloneqq \left(x_p^\top,x_o^\top,0,S(\eta,x)^\top,0,\Gamma(x,\eta)\right)^\top,$
%
and with \linebreak
%
%
$	C := \left\lbrace \xi \in \R^{2n_x+n_e+n_\eta+2} | \tau \leq \auxvar \right\rbrace$ and $
	D := \linebreak \left\lbrace \xi \in \R^{2n_x+n_e+n_\eta+2} | \tau = \auxvar \right\rbrace.$
%
%
%

Jumps of the hybrid system~\eqref{eq_sys_hyb} correspond for any solution~$\xi$ exactly to transmission times of the STC mechanism. Hence the transmission sequence $(t_j,j)\in\dom~\xi$ corresponds exactly to the indices when \eqref{eq_sys_hyb} jumps.  We thus describe by $\tvar_j \coloneqq  (\svar_j,j-1)$ the hybrid time before the transmission at time $\svar_j$ and by $\tvar_j^+ \coloneqq (\svar_j,j)$ the hybrid time directly after the transmission at time $\svar_j$. 
%
We assume that the STC mechanism is executed at the initial time $t_0 = 0$. This corresponds to a restriction of the initial conditions for the hybrid system for $e(0,0), \tau(0,0)$ and $\auxvar(0,0)$ to $e(0,0) = x_o(0,0)-x_p(0,0), \tau(0,0) = 0$ and $\auxvar(0,0) = \Gamma(x_o(0,0),\eta(0,0))$. Otherwise the first transmission time might not be well-defined. 


%

%

Subsequently, our goal will be to design functions $\Gamma$ and $S$ that ensure input-to-state stability of the origin of \eqref{eq_sys_hyb} according to the following definition. 

\begin{defi}
	\label{def_asym_stab}
	For the hybrid system~$\mathcal{H}_{STC}$ with initial condition $x_p(0,0) \in \R^{n_x} , x_o(0,0)\in\R^{n_x}, e(0,0)  = x_o(0,0)-x_p(0,0), \eta(0,0) \in \R^{n_\eta}$ and $\auxvar(0,0) \linebreak = \Gamma(x_o(0,0),\eta(0,0))$, the set
	\begin{equation*}
		\left\lbrace\left(x_p,x_o,e,\eta,\tau,\auxvar\right):x_p = 0, x_o = 0, e= 0, \eta = 0 \right\rbrace
	\end{equation*} is uniformly globally asymptotically stable (UGAS), if there exists $\beta \in \mathcal{K}\mathcal{L}\mathcal{L}$ such that all corresponding maximal solutions $\xi$ are $t-$complete and satisfy for all $(t,j)\in\dom~\xi$
	\begin{equation*}
		%
		\abs{\begin{bmatrix}
				x_p(t,j)\\
%
				e(t,j)\\
				\eta(t,j)
		\end{bmatrix}} \leq \beta\left(\abs{\begin{bmatrix}
				x_p(0,0)\\
%
				e(0,0)\\
				\eta(0,0)
		\end{bmatrix}},t, j\right).
	\end{equation*}
	%
\end{defi} 
%
%
%
%
%
%
%
%
%
%
%
%
%
%
%
%
%
%
%
%

%
%
%
%
%
%
%
%
%
%
%
%
%
%
%
%
%
%
%
%
%
%
%
%
%
%
%
%
%
%
%


%
%
%
%
%
%
%