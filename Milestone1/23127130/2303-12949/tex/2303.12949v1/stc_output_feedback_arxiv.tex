%
%
%

%
%
%
%
%
\documentclass{ifacconf}

\usepackage{graphicx}      %
\usepackage{natbib}        %
%

\makeatletter
\let\old@ssect\@ssect %
\makeatother


%
\usepackage{amsmath,amssymb,amstext}
\let\proof\relax \let\endproof\relax 
%

\usepackage{resizegather}

\usepackage{pgfplots} %
\pgfplotsset{compat=newest} 
\pgfplotsset{plot coordinates/math parser=false}

\usepackage{physics}
\usepackage{siunitx}

\usepackage{array}
\usepackage{algorithm}
\usepackage{etoolbox}


%
\let\classAND\AND
\let\AND\relax
%
\usepackage{algorithmic}
%
\let\algoAND\AND
\let\AND\classAND
%
\AtBeginEnvironment{algorithmic}{\let\AND\algoAND}

\usepackage{xcolor}
\usepackage{mathtools}

\usepackage[hyperindex,breaklinks]{hyperref}
\hypersetup{
	colorlinks=true,
	linkcolor=blue,
	filecolor=magenta,      
	urlcolor=blue,
	citecolor=black,
}

%
%
%
%
\let\labelindent\relax
\usepackage{enumitem}

\makeatletter
\def\@ssect#1#2#3#4#5#6{%
	\NR@gettitle{#6}%
	\old@ssect{#1}{#2}{#3}{#4}{#5}{#6}%
}
\makeatother
%
%
%


\usepackage{algorithm}
\usepackage{algorithmic}

%

\newcommand{\mati}{{\text{MATI}} }
\newcommand{\ivar}{{i_j}}
\newcommand{\maxvar}{k_{\max}}

%
\newcommand{\change}[2]{{#2}}

\newcommand{\auxvar}{\tau_{\max}}
\newcommand{\auxvari}{\tau_{\max,i}}
\newcommand{\lvar}{\Lambda}
%
\newcommand{\tpar}{p}
\newcommand{\refer}{\text{\normalfont ref}}

%
\newcommand{\itilde}{1}
\newcommand{\tnn}{0,0}

\newcommand{\samplevar}{k}
\newcommand{\mad}{\text{mad}}

\newcommand{\svar}{t}
\newcommand{\tvar}{s}
\newcommand{\timevar}{}
\newcommand{\Cvar}{\change{C}{K}}

\newcommand{\E}{\mathcal{E}}
\newcommand{\X}{\mathcal{X}}
\newcommand{\R}{\mathbb{R}}
\newcommand{\N}{\mathbb{N}}

\newcommand{\MH}[1]{{\color{red}\textbf{#1}}}
%

%
\newcommand{\arxiv}[2]{#2}

\DeclareMathOperator\dom{dom}
\DeclareMathOperator\arctanh{arctanh}

%
\def\QEDclosed{\mbox{\rule[0pt]{1.3ex}{1.3ex}}} %

\def\qed{\QEDclosed} %

\def\proof{\noindent\hspace{2em}{\itshape Proof: }}
\def\endproof{\hspace*{\fill}~\qed}


\makeatletter
\newcommand{\pushright}[1]{\ifmeasuring@#1\else\omit\hfill$\displaystyle#1$\fi\ignorespaces}
\newcommand{\pushleft}[1]{\ifmeasuring@#1\else\omit$\displaystyle#1$\hfill\fi\ignorespaces}
\makeatother


\newtheorem{defi}{Definition}
\newtheorem{theo}{Theorem}
\newtheorem{rema}{Remark}
%
\newtheorem{coro}{Corollary}
%
\newtheorem{sasum}{Standing Assumption}
\newtheorem{asum}{Assumption}
%
\newtheorem{cond}{Condition}

%
%
%
%
%
%
%
%
%
%
%


%
\begin{document}
	\begin{frontmatter}
		\title{	Self-triggered output feedback control for nonlinear networked control systems based on hybrid Lyapunov functions\thanksref{footnoteinfo}
		}
		
		\thanks[footnoteinfo]{		
			\change{
			Funded by Deutsche
			Forschungsgemeinschaft (DFG, German Research Foundation) under Germany’s
			Excellence Strategy - EXC 2075 - 390740016 and under grant
			AL 316/13-2 - 285825138. We acknowledge the support by the Stuttgart
			Center for Simulation Science (SimTech).}{$\copyright$ 2023 the authors. This work has been accepted to IFAC for publication under a Creative Commons Licence CC-BY-NC-ND.\\
			F. Allgöwer is thankful that this work was funded by the Deutsche Forschungsgemeinschaft (DFG, German Research Foundation)  under Germany’s Excellence Strategy -- EXC 2075 -- 390740016 and under grant AL 316/13-2 - 285825138.}
		}
		
		\author[first]{Michael Hertneck} 
		\author[first]{Frank Allg\"ower} 
		
		\address[first]{University of Stuttgart, Institute for Systems Theory and Automatic Control, Stuttgart, Germany (email: $\{$hertneck, allgower$\}$@ist.uni-stuttgart.de)}

		\begin{abstract}                %
			Most approaches for self-triggered control (STC) of nonlinear networked control systems (NCS) require measurements of the full system state to determine transmission times. However, for most control systems only a lower dimensional output is available. To bridge this gap, we present in this paper an output-feedback STC approach for nonlinear NCS. An asymptotically stable observer is used to reconstruct the plant state and transmission times are determined based on the observer state. 
%
			The approach employs hybrid Lyapunov functions and a dynamic variable to encode past state information and to maximize the time between transmissions. 
%
			 It is non-conservative in the sense that the assumptions on plant and controller are the same as for dynamic STC based on hybrid Lyapunov functions with full state measurements and any asymptotically stabilizing observer can be used.  We conclude that the proposed STC approach guarantees asymptotic stability of the origin for the closed-loop system.
%
		\end{abstract}
		
		\begin{keyword}
			Event-triggered and self-triggered control, Control under communication constraints, Control over networks
		\end{keyword}
	
	\end{frontmatter}


%
%
%

\section{Introduction}

The increasing complexity of source code poses a key challenge to the reliability of large-scale software systems. Software bugs in these systems can lead to safety issues~\cite{bug_safety} for users around the world as well as cause non-negligible financial losses~\cite{bug_loss}. As such, developers have to spend a large amount of time and effort on bug fixing. Consequently, \aprfull (\apr), designed to automatically generate patches to fix software bugs, has attracted wide attention from both academia and industry~\cite{long2016prophet, legoues2012genprog, long2015spr, lou2020can, tufano2018empstudy}. 


To achieve \apr, one popular approach is known as Generate-and-Validate (G\&V)~\cite{qi2015gv, ghanbari2019prapr, lou2020can, le2016hdrepair, legoues2012genprog, wen2018capgen, hua2018sketchfix, martinez2016astor, koyuncu2020fixminder, liu2019tbar, liu2019avatar}, which is typically based on the following pipeline: First, fault localization techniques~\cite{wong2016fl, abreu2007ochiai, zhang2013injecting, papadakis2015metallaxis, li2019deepfl, li2017transforming} are applied to determine the suspicious locations in programs where bugs are likely to exist. Then, the buggy locations are used by the \apr tools to generate a list of patches that replace buggy lines with correct lines. Afterward, each patch is validated against the original test suite to identify any \emph{plausible patches} (i.e., passing all tests in the test suite). Finally, to determine the \emph{correct patches}, developers examine the list of plausible patches to see if any of them can correctly fix the bug. 

Traditional \apr tools can mainly be categorized into heuristic-based~\cite{legoues2012genprog, le2016hdrepair, wen2018capgen}, constraint-based~\cite{mechtaev2016angelix, le2017s3, demacro2014nopol, long2015spr} and \template~\cite{ghanbari2019prapr, hua2018sketchfix, martinez2016astor, liu2019tbar, liu2019avatar}. Among these traditional tools, \template \apr tools~\cite{ghanbari2019prapr, liu2019tbar, benton2020effectiveness} have been able to achieve state-of-the-art results. \Template \apr tools typically leverage pre-defined templates (e.g., adding a nullness check) for bug fixing. However, since these fix templates are typically handcrafted, the number and types of bugs they are able to fix can be limited. 



To address the limitations of traditional \apr, researchers have proposed various \learning \apr tools~\cite{li2020dlfix, chen2018sequencer, jiang2021cure, lutellier2020coconut, zhu2021recoder, ye2022rewardrepair} based on the \nmtfull (\nmt) architecture~\cite{sutskever2014mt} where the input is the buggy code snippets and the goal is to translate the buggy code snippets into a fixed version. To accomplish this, \learning \apr tools require supervised training datasets with pairs of both buggy and fixed code snippets in order to learn how to perform this translation step. These training data are usually obtained by mining historical bug fixes using heuristics/keywords~\cite{dallmeier2007benchmark}, which can be imprecise for identifying bug-fixing commits; even the actual bug-fixing commits can include irrelevant code changes, leading to further pollution in the dataset~\cite{xia2022alpharepair}.
% 
Moreover, it can be hard for such \apr tools to generalize and fix bug types unseen during training. 



To better leverage recent advances in \plmfull{s} (\plm{s}), researchers~\cite{xia2022alpharepair, xia2023repairstudy, kolak2022patch, prenner2021codexws} have directly applied \plm{s} to generate patches without bug-fixing datasets. These \llm-based \apr tools work by either directly generating a complete code function~\cite{prenner2021codexws, xia2023repairstudy} or predict/infill the correct code snippet given its surrounding context~\cite{xia2022alpharepair, xia2023repairstudy}. By directly using \llm{s} that are pre-trained on billions of open-source code snippets, \llm-based \apr tools can achieve state-of-the-art performance on many repair datasets~\cite{xia2022alpharepair}. 


% 
%
%

Traditional \apr tools have long used the insight of the \emph{plastic surgery hypothesis}~\cite{barr2014plastic} where it states that the code ingredients to fix a bug already exist within the same project. Traditional \apr tools have manually designed pattern-~\cite{ghanbari2019prapr, saha2017elixir} or heuristic-based~\cite{jiang2018simfix, legoues2012genprog} approaches to finding and using such relevant code ingredients to generate fixes for bugs. However, the plastic surgery hypothesis has been largely ignored in \llm-based \apr. In fact, \llm provides a unique opportunity to fully automate the plastic surgery hypothesis idea via fine-tuning (learning project-specific information via model updates from the buggy project) and prompting (directly providing relevant code ingredients to the model), and make it directly applicable to different languages (since the \llm{s} are typically multi-lingual).%
Moreover, despite the intensive manual efforts involved, traditional \apr tools still cannot fully leverage project-specific information due to large search space for leveraging/composing existing code ingredients. In contrast, the project-specific information can effectively leveraged by \llm{s} due to their power in code understanding/vectorization, e.g., even partial/imprecise information may still guide \llm{s} in correct patch generation!
 To this end, we ask the question: \emph{How useful is the plastic surgery hypothesis in the era of \plm{s}}?








\mypara{Our Work.} To answer the question, we present \ourtech{\xspace} -- a \llm-based approach that automatically utilizes the plastic surgery hypothesis by systematically combining multiple fine-tuning and prompting strategies for \apr. \ourtech fine-tunes \plm{s} using two novel domain-specific training strategies: \textbf{\epfinetune} -- we fine-tune using the original buggy project by aggressively masking out a high percentage of tokens, which allows \plm to learn project-specific code tokens and programming styles; and \textbf{\rofinetune} -- which only masks out a single continuous code sequence per training sample, allowing the model to get used to the final \csapr task of predicting a single continuous code sequence. Furthermore, we directly leverage the ability for \plm{s} to understand natural language instructions and introduce a novel prompting strategy, \textbf{\idprompting}, which uses information retrieval and static analysis to obtain a list of relevant identifiers for the buggy lines. While such relevant identifiers are critical for fixing some difficult bugs, they may not be seen by the \llm during inference due to limited context window size. Through the use of prompting, we directly tell the model to use these extracted identifiers (relevant code ingredients) to generate the correct code. Finally, to perform repair, we combine all four model variants (including the base model, both fine-tuned models and the base model with prompting) for the final repair.





While our insight of leveraging the plastic surgery hypothesis for \llm-based \apr is generalizable across different types of \plm{s}, to implement \ourtech, we choose a recent \plm{\xspace}, \ctfive~\cite{wang2021codet5}, which is pre-trained on millions of open-source code snippets. \ctfive is an encoder-decoder model trained using \mspfull (\msp) objective where a percentage of tokens are masked out and each continuous masked token sequence is referred to as a masked span. Also, although we only extract relevant identifiers from the current buggy project (since this paper focuses on the plastic surgery hypothesis), our work can be easily extended to obtain other code information (such as relevant statements or functions) from other sources, such as  the massive pre-training corpora~\cite{husain2020codesearchnet} or historical bug-fixing datasets~\cite{jiang2019infer}, which can provide more coding knowledge for \llm{s}. Besides, although we mainly focus on using traditional string comparison algorithms for information retrieval in this paper, these techniques can be easily replaced by other frequency-based retrieval~\cite{robertson2009probabilistic} and neural search (or embedding-based search)~\cite{reimers2019sentence}.
  In summary, this paper makes the following contributions:


%


\begin{itemize}[noitemsep, leftmargin=*, topsep=0pt]
    \item \textbf{Dimension.} This paper is the first to revisit the important plastic surgery hypothesis in the era of \llm{s}. It opens up a new dimension for \llm-based \apr to incorporate previously neglected information from the buggy project itself to boost \apr performance. Furthermore, it demonstrates the promising future of retrieval-based prompting for modern \llm-based \apr.
    \item \textbf{Implementation.} We implement \ourtech based on the recent \ctfive model. We augment the model using two novel fine-tuning strategies: \epfinetune and \rofinetune, along with a novel prompting strategy based on information retrieval and static analysis: \idprompting. We combine the patches generated by all four models together and perform patch ranking to speed up \apr.% 
    \item \textbf{Evaluation Study.} We conduct an extensive evaluation against state-of-the-art \apr tools. On the widely studied \dfj 1.2 and 2.0 datasets~\cite{just2014dfj}, \ourtech is able to achieve the new state-of-the-art results of 89 and 44 correct bug fixes (15 and 8 more than best baseline) respectively.  Furthermore, we perform a broad ablation study to justify our design. \ourtech demonstrates for the first time that the plastic surgery hypothesis can substantially boost \llm-based \apr and advance state-of-the-art \apr, while being fully automated and general. Moreover, even partial/imprecise code ingredients may still effectively guide \llm{s} for \apr!
\end{itemize}


%% Macro setup
\definecolor{purple}{rgb}{1, 0, 1}

\newcommand{\ie}{\emph{i.e.,}\xspace}
\newcommand{\eg}{\emph{e.g.,}\xspace}
\newcommand{\abr}{\emph{abbr.}\xspace}
\newcommand{\ea}{\emph{et al.}\xspace}
\newcommand{\gensync}{\emph{GenSync}\xspace}
\newcommand{\colosseum}{\emph{Colosseum}\xspace}
\newcommand{\srep}{\emph{SREP}\xspace} % Set Reconciliation Enhances
\newcommand{\srepsim}{\emph{SREPSim}\xspace}
% Propagation
\newcommand{\esrep}{\emph{E-SREP}\xspace}
\newcommand{\epsrep}{\emph{EP-SREP}\xspace}
\newcommand{\mesrep}{\emph{ME-SREP}\xspace}
\newcommand{\mempoolsync}{\emph{MempoolSync}}

\newcommand{\fref}[1]{Fig.~\ref{#1}}
\newcommand{\tref}[1]{Table~\ref{#1}}
\newcommand{\aref}[1]{Algorithm~\ref{#1}}
\newcommand{\procref}[1]{Procedure~\ref{#1}}
\newcommand{\sref}[1]{Section~\ref{#1}}
\newcommand{\lineref}[1]{line~\ref{#1}}
\newcommand{\appref}[1]{Appendix~\ref{#1}}

% Change \eqref
\LetLtxMacro{\originaleqref}{\eqref}
\renewcommand{\eqref}{Eq.~\originaleqref}

% Theorems and corollaries
\newcounter{theoremcount}
\setcounter{theoremcount}{0}
\DeclareRobustCommand{\theorem}[1]{%
  \refstepcounter{theoremcount}%
  \noindent\textit{\textbf{Theorem \thetheoremcount\label{theorem:#1}: }}%
}
\DeclareRobustCommand{\theoremref}[1]{Theorem~\ref{theorem:#1}}

\DeclareRobustCommand{\proof}{\emph{Proof:}\xspace}
\DeclareRobustCommand{\qqed}{\hfill$\blacksquare$}

\newcounter{corollcount}
\setcounter{corollcount}{0}
\DeclareRobustCommand{\coroll}[1]{%
  \refstepcounter{corollcount}%
  \noindent\textit{\textbf{Corollary \thecorollcount\label{coroll:#1}: }}%
}
\DeclareRobustCommand{\corollref}[1]{Corollary~\ref{coroll:#1}}

\newcounter{lemmacount}
\setcounter{lemmacount}{0}
\DeclareRobustCommand{\lemma}[1]{%
  \refstepcounter{lemmacount}%
  \noindent\textit{\textbf{Lemma \thelemmacount\label{lemma:#1}: }}%
}
\DeclareRobustCommand{\lemmaref}[1]{Lemma~\ref{lemma:#1}}

\newcounter{definitioncount}
\setcounter{definitioncount}{0}
\DeclareRobustCommand{\definition}[1]{%
  \refstepcounter{definitioncount}%
  \noindent\textit{\textbf{Definition \thedefinitioncount\label{definition:#1}: }}%
}
\DeclareRobustCommand{\defref}[1]{Definition~\ref{definition:#1}}

%notes of different authors
\newif\ifnotes
\notestrue
\notesfalse

\newif\ifdiff
\difftrue
\difffalse

\newcommand{\anote}[1]{\ifnotes $\ll$\textsf{\textcolor{purple}{Ari: {#1}}}$\gg$ \fi}
\newcommand{\nnote}[1]{\ifnotes $\ll$\textsf{\textcolor{orange}{Novak: {#1}}}$\gg$ \fi}
\newcommand{\diff}[1]{\ifdiff\textcolor{orange}{#1}\else#1\fi}

%%% Local Variables:
%%% mode: latex
%%% TeX-master: "main"
%%% End:


%%% Redefining theorem-like environments
\newcounter{environments}

\newcounter{theoremCounter}
\newcounter{lemmaCounter}
\newcounter{definitionCounter}
\newcounter{propositionCounter}
\newcounter{corollaryCounter}
\newcounter{exampleCounter}
\newcounter{remarkCounter}
\newcounter{propertyCounter}
\newcounter{assumptionCounter}
\newcounter{proofCounter}

%\theorempreskip{1pt}
%\theorempostskip{1pt}

\let\proposition\relax
\let\theorem\relax
\let\lemma\relax
\let\definition\relax
\let\corollary\relax
\theoremseparator{.}
\theorembodyfont{\itshape}
\theoremsymbol{$\triangleleft$}
\newtheorem{theorem}[theoremCounter]{Theorem}
\newtheorem{lemma}[lemmaCounter]{Lemma}
\newtheorem{definition}[definitionCounter]{Definition}
\newtheorem{proposition}[propositionCounter]{Proposition}
\newtheorem{corollary}[corollaryCounter]{Corollary}

\let\remark\relax
\let\example\relax
\let\assumption\relax

\theorembodyfont{\normalfont}
\newtheorem{example}[exampleCounter]{Example}
\newtheorem{remark}[remarkCounter]{Remark}

\theoremheaderfont{\itshape}
\theoremsymbol{}
\renewtheorem{property}[remarkCounter]{Property}

\theoremheaderfont{\bfseries}
\theorembodyfont{\itshape}
\newtheorem{assumption}[assumptionCounter]{Assumption}
\theoremheaderfont{\itshape}


\theoremstyle{plain}
\theoremheaderfont{\itshape}
\theorembodyfont{\normalfont}
\let\proof\relax
\theoremseparator{.}
\theoremsymbol{\qedfull}
\newtheorem*{proof}{Proof}
\qedsymbol{\qedfull}

% Reset equation counters for each property!
\makeatletter
\@addtoreset{equation}{property}
\makeatother

% Tikz stuff
\newcommand{\seqarr}
{\begin{tikzpicture}
		\draw[-{Triangle[scale=.7]}] (0,0) --  (.3,0); 
\end{tikzpicture}}

\newcommand{\looparr}
{\begin{tikzpicture}[scale=0.7,baseline=-1.55ex]
		\draw[arrows = {-Stealth[inset=0pt, length=2pt, angle'=60]}] (0,0) arc (102:437:.2cm);
\end{tikzpicture}}

\definecolor{blue-violet}{rgb}{0.54, 0.17, 0.89}
\definecolor{cadmiumorange}{rgb}{0.93, 0.53, 0.18}
\definecolor{yellow-green}{rgb}{0.6, 0.8, 0.2}
\definecolor{green1}{rgb}{0.12, 0.3, 0.17}
\definecolor{byzantium}{rgb}{0.44, 0.16, 0.39}

\section{Preliminaries}

\subsection{Lithography Simulation Model} \label{litho_model}
During the lithography process, an input mask $\mathbf{M}$ is projected through layers of optical lens onto a wafer plane. The intensity after optical system $\mathbf{I}$, namely the aerial image, leaves a coating on the wafer with photoresist to form the resulting pattern $\mathbf{Z}$. The conventional simulation of the lithography process is composed of 2 consecutive components: optical projection model and photoresist model. 

For the projection process, Hopkins diffraction model \cite{hopkins1951concept} has been widely used to analyze coherent imaging system mathematically. To avoid the computation complexity of the Hopkins model, a singular value decomposition model (SVD)-based approximation has been proposed by \cite{cobb1998fast} and became the mainstream fashion. In the SVD model, the Hopkins diffraction model can be decomposed into a sum of coherent systems based on eigenvalue decomposition:
\begin{equation}
    \mathbf{I}(x,y) = \sum_{k = 1}^{ N^{2}} w_{k} | \mathbf{M}(x,y) \otimes h_{k}(x,y) |^{2}, \quad x,y = 1,2,...N
\end{equation}
where $h_{k}$ is the $k$-th kernel and $w_{k}$ is the corresponding weight of the coherent system. "$\otimes$" denotes the convolution operator. \cite{gao2014mosaic} indicates the $K$-th order approximation:
\begin{equation}
    \mathbf{I}(x,y) \approx \sum_{k = 1}^{K} w_{k} | \mathbf{M}(x,y) \otimes h_{k}(x,y) |^{2},
\end{equation}

We pick $K = 24$ in our experiment. After optical simulation, the lithography intensity $\textbf{I}$ is sent to the photoresist model to generate the final binary pattern $\mathbf{Z}$ with an exposure resist threshold $I_{th}$:
\begin{equation}
    \mathbf{Z}(x,y) = 
    \begin{cases}
        1, & \text{if} \quad \mathbf{I} (x,y) \geq I_{th}, \\
        0, & \text{if} \quad \mathbf{I} (x,y) < I_{th}, 
    \end{cases}
\end{equation}

Several machine learning-based lithography simulation methods have been proposed. 
\cite{watanabe2017accurate} utilized a CNN network to perform a function model determination for resist model simulation. 
\cite{ye2019lithogan} developed a GAN-based LithoGAN, to map the input mask and output resist pattern. 
\cite{shao2020ic} proposed a two-stage DNN-based framework, solving the mask-to-SEM prediction as a domain-transfer problem and using CycleGAN \cite{zhu2017unpaired} to learn the transferring process.

Although DNN models usually have the comparative speed advantage, we choose the Hopkins model for the reason of analyzability. A white box model enables us to analyze the pattern shift equivariance property mathematically during the lithography process.

\subsection{OPC Evaluation Criteria}
\begin{figure}
    \subfloat[]{ \label{fig:epe} \includegraphics[width=.45\linewidth]{figs/epe} } 
    \subfloat[]{ \label{fig:pvband} \includegraphics[width=.45\linewidth]{figs/pvband} }
    \caption{OPC evluation creteria: (a) Visualization of EPE measurement (b) Visualization of PVBand.}
    % \label{fig:epe_pvband}
\end{figure}

\subsubsection{Edge placement error (EPE).}
\enspace After the lithography process, the printed image on the wafer has an inevitable geometric distortion from the design target. Edge placement error (EPE) is a common criterion to quantify distortion level. 
Measurement of EPE is visualized in \Cref{fig:epe}: A series of measuring points are sampled along the boundary of the target design pattern, including vertical edges and horizontal edges. If the distance $D$ between printed image and target is larger than threshold $th_{EPE}$ at a sample point, we label it as a EPE violation.
\begin{equation}
    EPE\_violation(x,y) = 
    \begin{cases}
        1, & D(x,y) \geq th_{EPE}, \\
        0, & D(x,y) \leq th_{EPE},
    \end{cases}
\end{equation}

\subsubsection{Process Variation Band (PV Band).}
\enspace In real lithography applications, process variation may cause deviation in the final printed images, which possibly leads to printing failure. Given different lithography conditions such as focus/defocus depth and incident light intensity, printed images have various contour results. Process Variation Band (PV Band) is defined as discrepant (XOR) region of innermost and outermost contours as shown in \Cref{fig:pvband} to evaluate printing robustness.
\begin{equation}
    PVBand = \sum_{x,y}^{N^2} | \textbf{Z}_{out} - \textbf{Z}_{in} | ,
\end{equation}
where $N$ is the size of pattern. $\textbf{Z}_{out}$ denotes the printed pattern of outer contour and $\textbf{Z}_{in}$ denotes the inner contour. 


 \section{Main results}\label{sec:Main}
 \subsection{Continuous-time flow}\label{subsec:cont_time}
 In this section, we analyze the ordinary differential equation $\dot{\sx}(t) = \OF(\sx(t))$.  In all the remainder, we fix $r_1 >0$ and $K \subset \bbR^{n}$ with $ K = \{ x \in \bbR^n : \norm{h(x)} \leq r_1\}$.
Consider the following assumption:
 \begin{assumption}\label{hyp:cont_model}
   \begin{enumerate}[label=\roman*), nosep,leftmargin=15pt]
     \item\label{hyp:k_comp} The set $K$ is compact and $\nabla h$ is of full rank on $K$.
     \item\label{hyp:k_fullr}  It holds that $\nabla h^{\top} \nabla h A \in \bbR^{n_h \times n_h}$ is symmetric positive definite on $K$.
     \item\label{hyp:A_loclip} The function $A : K \rightarrow \bbR^{n_h \times n_h}$ can be extended to a locally Lipschitz continuous function on some neighborhood of $K$.
     \item\label{hyp:hA_eigenv}  There is $\alpha_m >0$ such that  $\inf_{x \in K} \lambda_m(x) > \alpha_m$, where $\lambda_m(x)$ is the minimal eigenvalue of $\nabla h^{\top}(x) \nabla h(x) A(x)$
   \end{enumerate}
 \end{assumption}
Note that as soon as $\cM$ is compact, there is always some $r_1 > 0$ such that \Cref{hyp:cont_model}-\ref{hyp:k_comp} holds. Moreover, \Cref{hyp:cont_model}-\ref{hyp:k_fullr}--\ref{hyp:A_loclip} are satisfied for the matrices $A$ given in~\Cref{ex:vanil,ex:mj_flow}.   As is often the case, to analyze the trajectory of an ordinary differential equation we need to find an energy (or Lyapunov) function. For $M > 0$, we define $\Lambda_M :\bbR^{n} \rightarrow \bbR$ as:
\begin{equation}\label{eq:def_LambdaM} \Lambda_M = f + M \norm{h} \, .
\end{equation}
The following theorem is our first main result, it shows that for $M$ large enough, $\Lambda_M$ decreases along any trajectory. This observation immediately implies the convergence of any bounded trajectory to the set of critical points. 
\begin{theorem}\label{th:cont_time}
  Assume \Cref{hyp:cont_model}.
  For any $x_0$ such that $\norm{h(x_0)} \leq r_1$ there is $\sx:\bbR_{+} \rightarrow \bbR^n$ a unique solution to 
  \begin{equation}
      \label{eq:orth_flow}
      \dot{\sx}(t) = \OF(\sx(t))
  \end{equation}
  starting at $x_0$. In addition, it holds that:
  \begin{enumerate}[nosep]
    \item For any $t \geq 0$, $\norm{h(\sx(t))} \leq \rme^{-\alpha_m t} \norm{h(x_0)}$,  where $\alpha_m$ is defined in \Cref{hyp:cont_model}-\ref{hyp:hA_eigenv}.
  \item For all $M \geq \overline{M}= M_1/\alpha_m$, with $M_1 = \sup_{x \in K} \norm{A^{\top} \nabla h^{\top}(\nabla f - \nabla h A h)}$, we get
  \begin{equation*}
    \inf_{0 \leq t \leq T} \norm{\OF(\sx(t))}^2= \inf_{0 \leq t \leq T} \norm{\dot{\sx}(t)}^2 \leq \frac{1}{T} \int_{0}^{T} \norm{\dot{\sx}(t)}^2 \rmd t \leq \frac{\Lambda_{M}(\sx(0)) - \Lambda_{M}(\sx(T))}{T} \, .
  \end{equation*}
  \item Let $x^*$ be in the limit set of $\sx$, i.e. there is $t_n \rightarrow + \infty$ such that $\sx(t_n) \rightarrow x^*$. Then $x^*$ is a critical point of \eqref{eq:main_opt_prob}.
  \end{enumerate}
\end{theorem}

\begin{proof}
 The existence and uniqueness of a local solution of \eqref{eq:orth_flow} follows from the fact that $\OF$ is locally Lipschitz continuous. As we shall see, such a solution must lie in $K$, which is compact by \Cref{hyp:cont_model}. This implies that the domain of a local solution can be extended to $\bbR_{+}$. Indeed, let $\sx$ be such a solution. Since for all $v \in V$, it holds that $\nabla h^{\top} v = 0$, we get using \Cref{hyp:cont_model}-\ref{hyp:hA_eigenv}:
\begin{equation}\label{eq:h_decrease_intm}
\frac{\dif}{ \dif t} \norm{h(\sx)}^2 = - 2 h^{\top}(\sx) \nabla h^{\top}(\sx) \nabla h(\sx) A(\sx) h(\sx)  \leq -2 \alpha_m \norm{h(\sx)}^2 \, ,
\end{equation}
 and Grönwall's lemma implies that $\norm{h(\sx(t))} \leq \rme^{-\alpha_m t} \norm{h(\sx(0))} $, for $t \geq 0$.
 Therefore, any local solution stays away from the boundary of $K$ and can be extended to a global solution for which the first claim holds. We now prove the second claim. Denote $D_h = (\nabla h^{\top} \nabla h)^{-1}$. In order to simplify the notations we omit the dependence on $x$ (see Lemma~\ref{lm:aff_proj}), and get
\begin{equation}\label{eq:interm_OFA}
\OF= - \nabla f +  \nabla h \left(  D_h \nabla h^{\top} \nabla f - A h \right)  \, ,
\end{equation}
where $D_h := (\nabla h^{\top} \nabla h)^{-1}$. This implies $\nabla h^{\top} \OF = - \nabla h^{\top} \nabla hA h$. Therefore, we have
\begin{align}\label{eq:err_f}
    \begin{split}
      \norm{(\OF + \nabla f)^{\top} \OF} &= \norm{\left(  D_h\nabla h^{\top} \nabla f - A h \right)^{\top} \nabla h^{\top} \OF}
      \\
      &\leq \norm{ h^{\top} A^{\top} \nabla h^{\top} \nabla h Ah - \nabla f^{\top} \nabla h A h}  \leq M_1 \norm{h} \, .
    \end{split}
\end{align}
Finally, if $\sx \not \in \cM$, we have
\begin{equation}\label{eq:f_decrease}
  \frac{\dif}{\dif t}f(\sx) = \nabla f(\sx)^{\top} \dot{\sx} = - \norm{\dot{\sx}}^2 + (\dot{\sx} + \nabla f(\sx))^{\top}\dot{\sx}(t) \leq - \norm{\dot{\sx}}^2 + M_1 \norm{h(\sx)} \, .
\end{equation}
Therefore, using~\eqref{eq:h_decrease_intm} and \eqref{eq:f_decrease} we obtain
\begin{equation}\label{eq:strict_lyap}
  \frac{\dif}{\dif t} \Lambda_M(\sx) \leq - \norm{\dot{\sx}}^2  \leq - \norm{\nabla_{V}f(\sx)}^2\, ,
\end{equation}
where the last inequality comes from the fact that the projection of $\dot{\sx}(t)$ onto $V$ is  $\nabla_V f$.
Integrating the last inequality we obtain the second claim for $\sx$.

To establish the third claim, we notice that $\OF \neq 0$ as soon as $x \notin \cM$ or $x \in \cM$ and $\Grad (f) \neq 0$. Equation~\eqref{eq:strict_lyap} then shows that $\Lambda_M$ is a strict Lyapunov function for the ODE~\eqref{eq:orth_flow} and the set of critical points of \eqref{eq:main_opt_prob}. In particular, LaSalle's invariance principle (see e.g. \cite[Theorem 2.17]{har_dynsyst91}) then implies that any limit point of $\sx$ must be contained in the set of critical points of \eqref{eq:main_opt_prob}.
\end{proof}
\vspace{-10pt}
\subsection{Algorithm}\label{sec:det_alg}
In this section we analyze the algorithms provided by the discretization of ODE~\eqref{eq:orth_flow} both in the deterministic and stochastic settings.
  Consider a filtered probability space $(\Omega, \mcF, \{\mcF_k, k >0\},  \bbP)$. Fix $x_0 \in K$ and let $(\eta_{k})_{k \geq 1}$ be a sequence of random variables adapted to $(\mcF_k)$. Our method, \algo, produces iterates as follows:
 \begin{equation}\label{eq:orth_alg}
     x_{k+1} = x_k + \gamma_k v_k + \gamma_k \eta_{k+1} , \quad{} \textrm{ with } v_k = \OF(x_k)
 \end{equation} 
 and with $(\gamma_k)$ a sequence of positive step sizes. The perturbation $(\eta_k)$ allows to capture the case where $\nabla f(x)$ (and hence $\nabla_V f(x)$) is unknown. This covers both streaming data and finite-sum problems in machine learning; see \citep{lan2020first}.
Recall that $\bbE_k$ denotes the conditional expectation given $\mcF_k$ and consider the following assumptions.
\begin{assumption}\label{hyp:disc_model}
\begin{enumerate}[label=\roman*), nosep]
    \item\label{hyp:fh_Lipgrad}The function $f$ (respectively $h$) has $L_f$ (respectively $L_h$) Lipschitz gradients on $K$.
    \item\label{hyp:iter_bound}The iterates $(x_k)$ remain in $K$, $\bbP$-almost surely.
    \item\label{hyp:zer_mean} For every $k \in \bbN$, it holds that $\eta_{k+1} \in V(x_k)$ and $\bbE_k[\eta_{k+1}] = 0$.
    \item\label{hyp:var_bound} There is a constant $\sigma \geq 0$ such that for all $k \in \bbN$, $\bbE_k[\norm{\eta_{k+1}}^2] \leq \sigma^2$.
\end{enumerate}
\end{assumption}
 \begin{example}\label{ex:SA}
 In the stochastic approximation framework, it is assumed that there is a probability space $(\Xi, \mcT, \mu)$ and a $\mu$-integrable function $g: \bbR^n \times \Xi \rightarrow \bbR^n$ such that for each $x \in \bbR^n$, $\int g(x, s) \mu(\rmd s)  = \nabla f(x)$. Let $(\xi_k)_{k \geq 1}$ be a sequence of i.i.d random variables defined on $(\Omega, \mcF, \bbP)$, taking values in $\Xi$ and such that the distribution of $\xi_k$ is $\mu$. We consider the following recursion
 \begin{equation*} 
 x_{k+1} = x_k - \gamma_k \nabla h(x_k) A(x_k) h(x_k) - \gamma_k g_V(x_k, \xi_{k+1}) \, , \end{equation*}
where $g_V(x, \xi)$ denotes the orthogonal projection of $g(x, \xi)$ onto $V(x)$. Thus, if we denote $\eta_{k+1} :=\nabla_V f(x_k) - g_V(x_k, \xi_{k+1})$ and $\mcF_k:= \sigma(\xi_1, \dots, \xi_k)$, we obtain \eqref{eq:orth_alg}. Note also that in this case $\eta_{k+1} \in V(x_k)$, $\bbE_k[\eta_{k+1}] = 0$, and if for some $\sigma >0$, it holds that $\sup_{x \in \bbR^n} \bbE[\norm{g(x,\xi) - \nabla f(x)}^2] \leq \sigma^2$, then $\bbE_k[\norm{\eta_{k+1}}^2] \leq \sigma^2$. 
 \end{example}
The deterministic setting is recovered by setting $\sigma = 0$. If $A$ is defined only on $K$ (see  \Cref{ex:mj_flow}), then \Cref{hyp:disc_model}-\ref{hyp:iter_bound} is required for the recursions to be properly defined. However, for $A$ as in~\Cref{ex:vanil}, this assumption is not needed. Nevertheless, it is necessary for our convergence analysis, and we show in \Cref{proof:safe_step}, that, under mild assumptions, if the step-sizes are small enough \Cref{hyp:disc_model}-\ref{hyp:iter_bound} is automatically satisfied.

The following theorem is the discrete counterpart of \Cref{th:cont_time}. It shows that \algo\ converges to the set of the critical points essentially at the same rate than (unconstrained) gradient descent. 
%Convergence is measured through $\OF$ which is meaningful thanks to Lemma~\ref{lm:OF_crit}.


\begin{theorem}\label{th:gen_rates}
  Assume \Cref{hyp:cont_model}--\ref{hyp:disc_model}. For any $M \geq \overline{M}$, where $\overline{M}$ is defined in \Cref{th:cont_time}, denote $D_M := \Lambda_M(x_0) - \inf_{x \in K} \Lambda_M(x)$ and let $\gamma \leq \gamma_{\max}= \min\left(\alpha_m^{-1}, (L_f + M L_h)^{-1} \right)$. Then, the following holds.
  \begin{enumerate}[nosep,leftmargin=15pt]
      \item If $\sigma = 0$, and for all $k$, $\gamma_k \equiv \gamma$, then:
      \begin{equation}\label{eq:det_rates}
  \inf_{ 0 \leq k \leq N-1}  \norm{\OF(x_k)}^2= \inf_{0\leq k \leq N-1} \norm{v_k}^2 \leq \frac{2 D_M}{N\gamma} \, .
  \end{equation}
   Furthermore, it holds that $\OF(x_k) \rightarrow 0$ and any accumulation point $x^*$ of $(x_k)$ is a critical point of Problem~\eqref{eq:main_opt_prob}.
  \item Otherwise, fix some constant $\bar D >0$, $N >0$ and $\gamma := \min(\gamma_{\max}, \bar D(\sigma \sqrt{N})^{-1})$. If $\gamma_k \equiv \gamma$, and $\hat k$ is uniformly sampled in $\{0, \dots, N-1\}$, then:
  \begin{equation}\label{eq:sto_rate}
  \bbE\left[\norm{\OF(x_{\hat k})}^2\right] \leq \frac{2D_M(L_f + M L_h+ \alpha_m)}{N} + \frac{\sigma}{\sqrt{N}}\left( \bar D (L_f + M L_h) + \frac{2 D_M}{\bar D}\right) \, .
  \end{equation}
  \end{enumerate}
\end{theorem}
\begin{proof}
Using a Taylor expansion of $\Lambda_M$ and using the upper-bound on $\gamma_k$, we obtain
    \begin{equation}\label{eq:rem_decr}
    2\left(\bbE_k[\Lambda_M(x_{k+1})] - \Lambda_M(x_k)\right) \leq - \gamma\norm{v_k}^2 + L_f + M L_h \sigma^2 \gamma^2 \, .
  \end{equation}
  Our claims then follow by telescoping this inequality and applying a standard proof technique (see e.g. \citet[Chapter~6]{lan2020first}) both in the deterministic and stochastic framework. Further details are given in \Cref{proof:gen_rates}.
\end{proof}


The preceding theorem shows that the rate of convergence of our algorithm, measured through $\OF$, is identical to the one obtained by gradient descent in a non-convex framework: $\cO(\varepsilon^{-2})$ in the deterministic setting and $\cO(\varepsilon^{-4})$ in the stochastic setting. As recently shown in \cite{CarmonLowerBF, CarmonLowerBF_sto}, these rates are tight, which makes our algorithm near-optimal in both cases.
  
   The term $(L_f + M L_h)$ in the definition of $\gamma_{\max}$ is  the Lipschitz constant of $\nabla f + M \nabla h$, hence our bound on the step sizes is reminiscent of the $L_f^{-1}$ bound required for convergence of standard gradient descent.
Note also that only an upper bound on $\overline{M}$ is required to achieve such rates. Indeed, in the deterministic setting, we can combine our method with line search; see \Cref{rm:safe_step}.
%below, we can estimate this threshold on the way: We start with one candidate and double it, e.g., if equation~\eqref{eq:rem_decr} is not satisfied (i.e., $\Lambda_M$ does not decrease). In this way, our candidate for the upper bound of $\overline{M}$ is modified only a finite number of times, so our convergence rate is preserved.
In the stochastic framework, performing line search is not an option, but we note that the discussion of \citet[Corollary~2.2.]{gha_lan13} applies here as well. In particular, we can make an error of the order of $\sqrt{N}$ in estimating $(L_f + M L_h)$ while maintaining our rate of convergence of $\cO(\varepsilon^{-4})$. If all constants are known, then the optimal $\overline{D}$ in equation~\eqref{eq:sto_rate} is $\sqrt{2D_M/(L_f + M L_h)}$. Finally, a nonconstant choice of step sizes is possible without affecting the final results; see \cite[Chapter~6]{lan2020first}. The choice of step size is further discussed in \Cref{proof:safe_step}.
%Thanks to the last proposition and remark, we can now made the following assumption.
%\begin{assumption}\label{hyp:iterates_bounded}
 % The iterates $(x_k)$ produced by Algorithm~\eqref{eq:orth_alg} remain in $K$.
%\end{assumption}

 \begin{figure}[t]
        \centering
            \begin{subfigure}{0.48\linewidth}
            \centering
            {\includegraphics[width=0.95\linewidth]{figures/qual_wo.png}}
            \caption{\method w/o TA}\label{fig:quala}
           \end{subfigure}
           \begin{subfigure}{0.48\linewidth}
            \centering
            {\includegraphics[width=0.95\linewidth]{figures/qual_w.png}}
            \caption{\method w/ TA}\label{fig:qualb}
           \end{subfigure}\vspace{-7pt}
        \caption{Visualization of attention maps (a) without temporal adaptation (TA) and (b) with temporal adaptation for the action 'Spinning [something] that quickly stops spinning' in SSv2~\cite{ssv2}.}\vspace{-5pt}
    \label{fig:qual}
    \end{figure}
\section{Conclusion}\label{sec:conclusion}
In this work, we focus on addressing the fundamental challenge of OOD detection tasks, which is how to fully understand the semantic discrepancy between the ID/OOD samples. We reveal that the key to success in the realistic SCOOD task is to allocate as many ID samples in the unlabeled set correctly as possible. To this end, we propose a novel uncertainty-aware optimal transport scheme that introduces class-specific energy scores as guidance for effective label assignment. Experimental results show that our method achieves better performance than previous state-of-the-art methods on SCOOD benchmarks.

\textbf{Limitations.} In addition to temperature scaling, other techniques such as feature clipping applied in ReAct~\cite{sun2021react} also enhance the performance of energy score, so how to obtain an OOD score that best fits the SCOOD task can be further explored. Moreover, a setting highly related to SCOOD has been proposed in \cite{katz2022training} and formulated as a constrained optimization problem. We will also theoretically analyze these practical OOD settings in our feature work.

% \section*{Acknowledgments}
\textbf{Acknowledgments.} 
This work is supported by National Key R\&D Program of China under Grant 2020AAA0105701, National Natural Science Foundation of China (NSFC) under Grants 61872327, Major Special Science and Technology Project of Anhui, National Natural Science Foundation of China (62033012) and Ant Group through Ant Research Intern Program.

%
\bibliography{../../../../Literatur/literature}
\arxiv{}{\appendix
\section{}
%
\label{app_a}
%
	\textbf{Proof of Theorem~\ref{prop_fir}}. Consider any maximal solution $\xi$ to \eqref{eq_sys_hyb}. 
%
	Note that $\hat{x}_o(\tvar_j^+) = x_o(\tvar_j^+) = x_o(\tvar_j)$ and thus $e(\tvar_j^+) = e_o(\tvar_j^+) = e_o(\tvar_j)$.
%
	Obviously, $\bar{h} \geq \delta T_{\max}(\gamma_1,L_1+\frac{\epsilon_1}{2}) = t_{\min}$ in Algorithm~\ref{algo_trig_window} and thus $\svar_{j+1}-\svar_j = \Gamma(x_o(\tvar_j),\eta(\tvar_j)) \geq t_{\min}$ holds for all $j\in\dom~\xi$, i.e. the minimum time between two triggering instants is strictly positive. 
%
%
	Because of the update of $\bar{h}$ in Algorithm~\ref{algo_trig_window}, for each $j\in\dom~\xi$, there is  an $\ivar \in\left\lbrace 1,\dots, n_\tpar \right\rbrace$ such that  $\svar_{j+1}-\svar_j \leq T_{\max}\left(\gamma_\ivar,\max\left\lbrace L_\ivar+\frac{\epsilon_\ivar}{2}, 1-\delta \right\rbrace\right)$ and $\svar_{j+1}-\svar_j \leq t_{\max} \coloneqq \underset{i\in\left\lbrace 1,\dots,n_{\tpar}\right\rbrace}{\max} \delta T_{\max}(\gamma_i,\max\left\lbrace L_i+\frac{\epsilon_i}{2}, 1-\delta \right\rbrace)$. 
%
	Proposition~\ref{prop_hybrid} thus implies for $\svar_j\leq t \leq \svar_{j+1}$ with $x_p(\tvar_{j}^+) = x_p(\tvar_{j})$ and $e(\tvar_j^+) = e_o(\tvar_j)$ that 
	\begin{equation}
		\label{eq_V_dec_fir}
		\begin{split}
			&V(x_p(\svar,j)) \leq U_\ivar(\xi(t,j\change{+1}{}))	
			\leq 	e^{ -\epsilon_\ivar (t-\svar_j)}U_\ivar(\xi(\tvar_j^+))\\ %
			=& e^{ -\epsilon_\ivar (t-\svar_j)}\left(V(x_p(\tvar_j)) + \gamma_\ivar \lambda_\ivar^{-1} W^2(e_o(\tvar_j)) \right).	
		\end{split}
	\end{equation}
%
%
	Here $U_\ivar(\xi)$ is the respective function according to \eqref{eq_def_u} from Proposition~\ref{prop_hybrid} for $\gamma = \gamma_\ivar$, $L = L_\ivar, \epsilon = \epsilon_\ivar$ and some sufficiently small $\lambda\in\left(0,1\right)$.
%

	We will now use this to investigate the evolution of $V(x_p)$ depending on the time between sampling instants. We distinguish between two possible outcomes for $\ivar$ in Algorithm~\ref{algo_trig_window}.
%
	Suppose first $\ivar = 1$, i.e., the fallback strategy is used. Then
	\begin{equation*}
		e^{-\epsilon_\ivar(\svar_{j+1}-\svar_{j})} V(x_p(\tvar_j)) \leq e^{-\epsilon_\refer(\svar_{j+1}-\svar_{j})} V(x_p(\tvar_j)) 
	\end{equation*}
	holds since $\epsilon_1 \geq \epsilon_\refer > 0$ and $\svar_{j+1}-\svar_j < T_{\max}(\gamma_1,L_1+\frac{\epsilon_1}{2})$. This implies with \eqref{eq_V_dec_fir} that 
	\begin{equation}
		\label{eq_bound_i1}
		\begin{split}
			V(x_p(\tvar_{j+1})) 
			\leq& e^{-\epsilon_\refer(\svar_{j+1}-\svar_{j})} V(x_p(\tvar_j))  + \alpha_1(\abs{e_o(\tvar_j)})	
%
		\end{split}
	\end{equation}
	for some $\alpha_1\in\mathcal{K}$. Here we used that $\underset{i\in\left\lbrace 1,\dots,n_{\tpar}\right\rbrace}{\max}e^{ -\epsilon_i (\svar_{j+1}-\svar_j)} \in \R_{>0}$ due to the upper bound $t_{\max}$ on $t_{j+1}-t_j$. 
		
	Next, suppose that $\ivar > 1$. In this case, Algorithm~\ref{algo_trig_window} chooses $\svar_{j+1}-\svar_j$ such that $\svar_{j+1}-\svar_j < T_{\max}(\gamma_\ivar,L_\ivar+\frac{\epsilon_\ivar}{2})$ and
	\begin{equation}
		\label{eq_Vo_bound}
		\begin{split}
			e^{-\epsilon_\ivar(\svar_{j+1}-\svar_{j})} V(x_o(\tvar_j)) \leq& e^{-\epsilon_\refer(\svar_{j+1}-\svar_j)} C(x_o(\tvar_j),\eta(\tvar_j))\\
%
		\end{split}		
	\end{equation}
	hold. Further note that $\ivar > 1$ is only possible if $V(x_o(\tvar_j)) \leq C(x_o(\tvar_j),\eta(\tvar_j)) \leq V_{\max}$ due to Line~\ref{line_for_start} of Algorithm~\ref{algo_trig_window} and our choice of $C(x_p,\eta)$ according to \eqref{eq_C_def}.
%
%
%
%
%
%
%
%
Observe that
	\begin{equation*}
		\begin{split}
			V(x_p(\tvar_j)) =& V(x_o(\tvar_j) - e_o(\tvar_j))\\
%
			\leq& V(x_o(\tvar_j)) +\mathfrak{L}(\abs{e_o(\tvar_j)}) \abs{e_o(\tvar_j)}
		\end{split}
	\end{equation*}
	where $\mathfrak{L}(p)$ is a (local) Lipschitz constant of $V$ that satisfies
	\begin{equation}
		\label{eq_def_L}
	\abs{V(x_o(\tvar_j)-e_o(\tvar_j)) - V(x_o(\tvar_j))}\leq \mathfrak{L}(p) \abs{e_o(\tvar_j)}
	\end{equation}
	for all $e_o(\tvar_j)$ with $\abs{e_o(\tvar_j)} \leq p$ and $x_o$ with $V(x_o) \leq V_{\max}$. Note that $\mathfrak{L}(p)$ is non-decreasing and bounded for all $p \geq 0$ since $V$ is locally Lipschitz. We thus obtain for some  $\alpha_{2}\in\mathcal{K}$
	\begin{equation}
		\label{eq_bound_op}
		V(x_p(\tvar_j)) \leq V(x_o(\tvar_j)) + \alpha_{2}(\abs{e_o(\tvar_j)}).
	\end{equation}
	Using \eqref{eq_Vo_bound} and \eqref{eq_bound_op} we obtain from \eqref{eq_V_dec_fir} for $t = t_{j+1}$
	\begin{equation}
		\label{eq_bound_ij}
		\begin{split}
			&V(x_p(\tvar_{j+1}))	\\
			\leq& e ^{-\epsilon_\refer(t_{j+1}-t_j)} C(x_o(\tvar_j),\eta(\tvar_j))\\
			 &+ e^{ -\epsilon_\ivar (\svar_{j+1}-\svar_j)}\left( \alpha_{2}(\abs{e_o(\tvar_{j})}) + \gamma_\ivar \lambda_\ivar^{-1} W^2(e_o(\tvar_j)) \right)	\\
%
			\leq& e ^{-\epsilon_\refer(t_{j+1}-t_j)} C(x_o(\tvar_j),\eta(\tvar_j)) + \alpha_3(\abs{e_o(\tvar_j)})\\
%
		\end{split}
	\end{equation}
for some $\alpha_3\in\mathcal{K}$, where we used again that \linebreak $\underset{i\in\left\lbrace 1,\dots,n_{\tpar}\right\rbrace}{\max}e^{ -\epsilon_i (\svar_{j+1}-\svar_j)} \in \R_{>0}$ due to the upper bound $t_{\max}$ on $t_{j+1}-t_j$. 
%
From \eqref{eq_C_def}, we obtain with the update of $\eta$ according to \eqref{eq_S_window_2} that\footnote{Note that the second sum is only relevant for $j < m-1$ to capture the effect of the initial condition on $\eta$.}
\begin{equation}
	\label{eq_C_decomp}
	\begin{split}
		&C(x_o(\tvar_{{j}}),\eta(\tvar_{{j}})) = \frac{1}{m} V(x_o(\tvar_{{j}})) + \sum_{k=1}^{m-1} \eta_k(\tvar_{{j}}) \\
		\leq & \frac{1}{m}V(x_o(\tvar_{{j}}))\\
		 &+ \frac{1}{m}\sum_{k=1}^{\min\left\lbrace m-1 , j \right\rbrace} e^{-\epsilon_\refer(t_{{j}}-t_{{j}-k})} \min\left\lbrace V(x_o(\tvar_{{{j}}-k})), V_{\max} \right\rbrace\\
		 &+ \frac{1}{m}\sum_{k = \min\left\lbrace j,m-1\right\rbrace+1}^{m-1} e^{-\epsilon_\refer(t_{{j}}-t_{0})}\eta_{m-k}.
	\end{split}
\end{equation}

If $V(x_o(\tvar_{j-k})) \leq V_{\max},$ we obtain similar as in \eqref{eq_bound_op} that
\begin{equation*}
	\begin{split}
			&\min\left\lbrace V(x_o(\tvar_k)), V_{\max} \right\rbrace\\
			 =& V(x_p(\tvar_{j-k})) + V(x_o(\tvar_{j-k})) - V(x_o(\tvar_{j-k}) - e_o(\tvar_{j-k}))\\
		\leq& V(x_p(\tvar_{j-k})) + \alpha_{2}(\abs{e_o(\tvar_{j-k})}).
	\end{split}
\end{equation*} 
If $V(x_o(\tvar_{j-k})) > V_{\max},$ then either $V(x_p(\tvar_{j-k})) > V_{\max}$ and $\min\left\lbrace V(x_o(\tvar_{j-k})), V_{\max} \right\rbrace \leq V(x_p(\tvar_{j-k})) + \alpha_{2}(\abs{e_o(\tvar_{j-k})})$ trivially holds or $V(x_p(\tvar_{j-k})) \leq V_{\max}$. In the latter case, we can again use the same argumentation that precedes \eqref{eq_bound_op} and obtain
\begin{equation}
	\label{eq_bound_op2}
	\min\left\lbrace V(x_o(\tvar_{j-k})), V_{\max} \right\rbrace \leq V(x_p(\tvar_{j-k})) + \alpha_{2}(\abs{e_o(\tvar_{j-k})}).
\end{equation}
Thus \eqref{eq_bound_op2} holds in all cases.
%
%
%
%
%
%
%
%
%
%
%
%
%
%
%
%
%
%
%
%
%
%
%
%
Using it in \eqref{eq_C_decomp}, we obtain 
\begin{equation}
	\label{eq_C_bound_pre}
	\begin{split}
		&C(x_o(\tvar_{{j}}),\eta(\tvar_{{j}}))\\
		 \leq& \frac{1}{m}\sum_{k=1}^{\min\left\lbrace m-1 , j \right\rbrace} e^{-\epsilon_\refer(t_{{j}}-t_{{j}-k})} V(x_p(\tvar_{{{j}}-k}))\\
		  &+\frac{1}{m}\sum_{k=1}^{\min\left\lbrace m-1 , j \right\rbrace} e^{-\epsilon_\refer(t_{{j}}-t_{{j}-k})}  \alpha_{2}(\abs{e_o(\tvar_{j-k})})\\
		  &+ \frac{1}{m}\sum_{k = \min\left\lbrace j,m-1\right\rbrace+1}^{m-1} e^{-\epsilon_\refer(t_{{j}}-t_{0})}\eta_{m-k}.
	\end{split}
\end{equation}		  
With \eqref{eq_obs_er} from Assumption~\ref{as_obs_er} in the second sum of \eqref{eq_C_bound_pre}, and $t_{k+1}-t_k \leq t_{\min}~\forall k\in\dom\xi$, we further obtain
\begin{equation}
	\label{eq_C_bound}
	\begin{split}		  
		 &C(x_o(\tvar_{{j}}),\eta(\tvar_{{j}}))\\
		   \leq & \frac{1}{m}\sum_{k=1}^{\min\left\lbrace m-1 , j \right\rbrace} e^{-\epsilon_\refer(t_{{j}}-t_{{j}-k})} V(x_p(\tvar_{{{j}}-k}))\\
		   	   &+ \frac{1}{m}\sum_{k = \min\left\lbrace j,m-1\right\rbrace+1}^{m-1} e^{-\epsilon_\refer(t_{{j}}-t_{0})}\eta_{m-k}\\
		   	   &+    \alpha_{2}\left(\beta_o\left(\abs{e_o(\tnn)},\max\left\lbrace (j-m+1)t_{\min},0\right\rbrace\right)\right).
	\end{split}
\end{equation}

	
	
	Now we show by induction based on \eqref{eq_bound_i1} and \eqref{eq_bound_ij} that
	\begin{equation}
		\label{eq_ind_as}
		\begin{split}
			&V(x_p(\tvar_j)) 
			\\ \leq& e^{-\epsilon_\refer t_j} \max\left\lbrace V(x_p(0,0)),\abs{\eta(0,0)}\right\rbrace\\
%
			&+ \sum_{k=0}^{j-1}\left(e^{-\epsilon_\refer t_{\min}(j- k -1)}\right. \\
			&\cdot \left.\alpha_5\left(\beta_o\left(\abs{e_o(\tnn)},\max\left\lbrace (k-m+1)t_{\min},0\right\rbrace\right)\right) \vphantom{e^{-\epsilon_\refer t_{\min}(j- k)}}\right)
		\end{split}		
	\end{equation}
%
%
%
%
%
%
%
%
%
%
	holds for all $j\in\dom~\xi$ with $\alpha_5(\cdot) \coloneqq \alpha_1(\cdot)+\alpha_{2}(\cdot)+\alpha_3(\cdot) \in\mathcal{K}$. It trivially holds for $j = 0$. Suppose it holds for all $j \leq \tilde{j}$ for some $\tilde{j}\in\dom~\xi$. 
	
	We first consider the case $i_{\tilde{j}} \geq 1$. Note that \eqref{eq_ind_as} for $j\leq \tilde{j}$ implies with \eqref{eq_C_bound} that
	\begin{equation*}
		\begin{split}
			&C(x_o(\tvar_{\tilde{j}}),\eta(\tvar_{\tilde{j}}))\\
			\leq&  e^{-\epsilon_\refer t_{\tilde{j}}} \max\left\lbrace V(x_p(0,0)),\abs{\eta(0,0)}\right\rbrace\\
%
			&+ \sum_{k=0}^{{\tilde{j}-1}}\left(e^{-\epsilon_\refer t_{\min}({\tilde{j}}- k-1)}\right. \\
			&~\cdot \left.\alpha_5\left(\beta_o\left(\abs{e_o(\tnn)},\max\left\lbrace (k-m+1)t_{\min},0\right\rbrace\right)\right) \vphantom{e^{-\epsilon_\refer t_{\min}(j- k)}}\right)\\
			&+ \alpha_{2}\left(\beta_o\left(\abs{e_o(\tnn)},\max\left\lbrace (\tilde{j}-m+1)t_{\min},0\right\rbrace\right)\right).
		\end{split}
	\end{equation*}
	Plugging this in \eqref{eq_bound_ij}, we obtain
	\begin{equation}
		\label{eq_ind_ii}
		\begin{split}
			&V(x_p(\tvar_{\tilde{j}+1})) \\
			\leq& e^{-\epsilon_\refer (t_{\tilde{j}+1}-t_{\tilde{j}})} e^{-\epsilon_\refer t_{\tilde{j}}} \max\left\lbrace V(x_p(0,0)),\abs{\eta(0,0)}\right\rbrace\\
%
			& + \alpha_3\left(\abs{e_o(\tvar_{\tilde{j}})}\right)
			+ e^{-\epsilon_\refer (t_{\tilde{j}+1}-t_{\tilde{j}})} \sum_{k=0}^{{\tilde{j}-1}}\left(e^{-\epsilon_\refer t_{\min}({\tilde{j}}- k-1)}\right. \\
			&~\cdot \left.\alpha_5\left(\beta_o\left(\abs{e_o(\tnn)},\max\left\lbrace (k-m+1)t_{\min},0\right\rbrace\right)\right) \vphantom{e^{-\epsilon_\refer t_{\min}(j- k)}}\right)\\
			&+  e^{-\epsilon_\refer (t_{\tilde{j}+1}-t_{\tilde{j}})}\alpha_{2}\left(\beta_o\left(\abs{e_o(\tnn)},\max\left\lbrace (\tilde{j}-m+1)t_{\min},0\right\rbrace\right)\right).
		\end{split}
	\end{equation}
%
	Note that
	\begin{equation*}
%
		\begin{split}
			& \alpha_3\left(\abs{e_o(\tvar_{\tilde{j}})}\right)\\
			 &+ e^{-\epsilon_\refer (t_{\tilde{j}+1}-t_{\tilde{j}})}\alpha_{2}\left(\beta_o\left(\abs{e_o(\tnn)},\max\left\lbrace (\tilde{j}-m+1)t_{\min},0\right\rbrace\right)\right)\\
			\leq & \alpha_5(\beta_o\left(\abs{e_o(\tnn)},\max\left\lbrace (\tilde{j}-m+1)t_{\min},0\right\rbrace\right))
		\end{split}
	\end{equation*}
holds since 
\begin{equation*}
	\begin{split}
		\abs{e_o(\tvar_{\tilde{j}})} &\leq \beta_o\left(\abs{e_o(\tnn)},\svar_{\tilde{j}}\right)\\
		 &\leq \beta_o\left(\abs{e_o(\tnn)},\max\left\lbrace (\tilde{j}-m+1)t_{\min},0\right\rbrace\right),
	\end{split}
\end{equation*} since $e^{-\epsilon_\refer (t_{\tilde{j}+1}-t_{\tilde{j}})} < 1$ and due to the definition of $\alpha_5$. 
	
	Using this and  $e^{-\epsilon_\refer(\svar_{\tilde{j}+1}-\svar_{\tilde{j}})} \leq e^{-\epsilon_\refer t_{\min}} $ in \eqref{eq_ind_ii}, we obtain 
	\begin{equation*}
		\begin{split}
			&V(x_p(\tvar_{\tilde{j}+1})) \\
			\leq & e^{-\epsilon_\refer t_{\tilde{j}+1}} \max\left\lbrace V(x_p(0,0)),\abs{\eta(0,0)}\right\rbrace\\
			%
			&+ \sum_{k=0}^{{\tilde{j}}}\left(e^{-\epsilon_\refer t_{\min}({\tilde{j}}- k)}\right. \\
			&~\cdot \left.\alpha_5\left(\beta_o\left(\abs{e_o(\tnn)},\max\left\lbrace (k-m+1)t_{\min},0\right\rbrace\right)\right) \vphantom{e^{-\epsilon_\refer t_{\min}(j- k)}}\right),
		\end{split}
	\end{equation*}
	i.e., \eqref{eq_ind_as} also holds for $j=\tilde{j}+1$ in this case.
	
	
%
%
	
%
%
%
%
%
%
%
%
%
	
	
	
	Now we consider the remaining case $i_{\tilde{j}} = 1$. In this case, \eqref{eq_bound_i1} holds and hence with \eqref{eq_ind_as} for $j=\tilde{j}$, we obtain
	\begin{equation*}
		\begin{split}
			&V(x(\tvar_{\tilde{j}+1}))\\ \leq& e^{-\epsilon_\refer(\svar_{\tilde{j}+1}-\svar_{\tilde{j}})} V(x_p(\tvar_{\tilde{j}}))  + \alpha_1\left(\abs{e_o(\tvar_{\tilde{j}})}\right)	\\
%
%
%
%
%
%
%
			\leq& e^{-\epsilon_\refer t_{\tilde{j}+1}} \max\left\lbrace V(x_p(0,0)),\abs{\eta(0,0)}\right\rbrace\\
%
			&+ \sum_{k=0}^{\tilde{j}}\left(e^{-\epsilon_\refer t_{\min}(\tilde{j}- k)}\right. \\
			&\cdot \left.\alpha_5\left(\beta_o\left(\abs{e_o(\tnn)},\max\left\lbrace (k-m+1)t_{\min},0\right\rbrace\right)\right) \vphantom{e^{-\epsilon_\refer t_{\min}(j- k)}}\right)
		\end{split}
	\end{equation*}
	where we used that $ e^{-\epsilon_\refer t_{\tilde{j}+1}} \leq e^{-\epsilon_\refer t_{\min}}$ and that
	\begin{equation*}
		\begin{split}
			&\alpha_1\left(\abs{e(\tvar_{\tilde{j}})}\right) \leq \alpha_1\left(\beta_o\left(\abs{e_o(\tnn)},\svar_{\tilde{j}}\right)\right)  \\
			 \leq& \alpha_1\left(\beta_o\left(\abs{e_o(\tnn)},\max\left\lbrace (\tilde{j}-m+1)t_{\min},0\right\rbrace\right)\right)\\
			 \leq&  \alpha_5\left(\beta_o\left(\abs{e_o(\tnn)},\max\left\lbrace (\tilde{j}-m+1)t_{\min},0\right\rbrace\right)\right).
		\end{split}
	\end{equation*}
%
%
%
	 As a result, we can conclude that $\eqref{eq_ind_as}$ holds also for $j = \tilde{j}+1$ if $i_{\tilde{j}} = 1$.
	
	It thus follows by induction that \eqref{eq_ind_as} holds for all $j\in\dom~\xi$. Together with the fact that $t_{\min}\leq t_{j+1}-t_j\leq t_{\max}$ this further implies that $\xi$ is $t$-complete.  
	
	Next, we discuss that \eqref{eq_ind_as} implies that $U_{\ivar}$ is bounded by a $\mathcal{K}\mathcal{L}\mathcal{L}$ function. 
%
%
	Observe 
	\newlength\mylen
	\settoheight\mylen{$+ \frac{1}{1-e^{-\epsilon_\refer t_{\min}}} \beta_o\left(\abs{e_o(\tnn)},\max\left\lbrace \left(\frac{t}{2t_{\max}}-m+1\right)t_{\min},0\right\rbrace\right)$}
	\begin{equation}
		\label{eq_sum_bound_1}
		\begin{split}
			&\sum_{k=0}^{j-1}e^{-\epsilon_\refer t_{\min}(j-k-1)}\\
			&\cdot \alpha_5\left(\beta_o\left(\abs{e_o(\tnn)},\max\left\lbrace (k-m+1)t_{\min},0\right\rbrace\right)\right) \\			
%
%
%
%
%
%
%
%
%
%
		\leq&  e^{-\epsilon_\refer t_{\min} \frac{j}{2}} \alpha_5({\beta_o(\abs{e(\tnn)},0)}) \sum_{k=0}^{\frac{j}{2}}e^{-\epsilon_\refer t_{\min}({\frac{j}{2}}- k-1)}\\
		&+ \alpha_5\left(\beta_o\left(\abs{e(\tnn)},\max\left\lbrace \left(\frac{j}{2}-m+1\right)t_{\min},0\right\rbrace\right)\right)\\
		&\cdot \sum_{k=\frac{j}{2}}^{j-1}e^{-\epsilon_\refer t_{\min}({j}- k-1)}\\
		\leq& e^{-\epsilon_\refer \frac{t_{\min}}{t_{\max}} \frac{t}{2}} \frac{1}{1-\epsilon_\refer t_{\min}}\beta_o(\abs{e_o(\tnn)},0)\\
		&\resizebox{\linewidth}{.95\mylen}{$+ \frac{1}{1-e^{-\epsilon_\refer t_{\min}}} \beta_o\left(\abs{e_o(\tnn)},\max\left\lbrace \left(\frac{t}{2t_{\max}}-m+1\right)t_{\min},0\right\rbrace\right)$}\\
		\eqqcolon& \beta_2(\abs{e_o(\tnn)},t) \in \mathcal{K}\mathcal{L},
		\end{split}
	\end{equation} 
	where we used the geometric series and the fact that $j \geq \frac{t}{t_{\max}}$ since $t_{j+1}-t_j \leq t_{\max}$. 
	
	Plugging this into \eqref{eq_ind_as}, we can conclude that for some $\beta_3\in\mathcal{K}\mathcal{L}$,
%
	$	V(x(\tvar_j)) \leq \beta_3\left(\abs{
			\begin{bmatrix} 				
				x_p(0,0)\\ 
				e(0,0)\\ 				
				\eta(0,0) 		\end{bmatrix}},\svar_j\right)$
%
	holds for all $j\in\R_{>0}$. Using again \eqref{eq_V_dec_fir}, this implies together with Assumption~\ref{as_obs_er} for $k_1 = \underset{i\in\left\lbrace 1,\dots,n_{\tpar}\right\rbrace}{\max}e^{ -\epsilon_i (\svar_{j+1}-\svar_j)} \in \R_{>0}$ for some $\beta_4 \in\mathcal{K}\mathcal{L}\mathcal{L}$ that
	\begin{equation*}
		\begin{split}
			U_\ivar(t,j) \leq& k_1 \left(\beta_3\left(\abs{
				\begin{bmatrix} 				
					x_p(0,0)\\ 
					e(0,0)\\ 				
					\eta(0,0) 		\end{bmatrix}},\max\left\lbrace t - t_{\max}, 0 \right\rbrace\right)\right.\\
				&\left.+ \gamma_\ivar\lambda_\ivar^{-1} W^2(\beta_0(\abs{e_0(0,0)}, \max\left\lbrace t - t_{\max}, 0 \right\rbrace)  \vphantom{\begin{bmatrix} 				
						x_p(0,0)\\ 
						e(0,0)\\ 				
						\eta(0,0) 		\end{bmatrix}}\right)\\
					&\leq \beta_4\left(\begin{bmatrix} 				
						x_p(0,0)\\ 
						e(0,0)\\ 				
						\eta(0,0) 		\end{bmatrix},t,j\right),
		\end{split}		
	\end{equation*}
	where we used that $t \geq \frac{t}{2} + jt_{\min}$ holds for all $(t,j)\in\dom~\xi$. 
	
	Finally, with the bounds on $V$ and $W$ from Assumption~\ref{asum_hybrid_lyap}, the fact that $\phi_\ivar \in \left[\lambda,\lambda^{-1}\right]$ for sufficient small $\lambda\in\left(0,1\right)$, the definition of $U_\ivar$ according to \eqref{eq_def_u} for the respective $\gamma_\ivar$, $L_\ivar$ and $\epsilon_\ivar$ and since the observer state converges to the plant state according to Assumption~\ref{as_obs_er}, UGAS of the set $\left\lbrace \left(x_p,x_c,e,\eta,\tau,\auxvar\right): x_p = 0, x_c = 0, e= 0, \eta = 0 \right\rbrace$ follows similar as in \cite{hertneck21robust_arxiv}. \hfill\hfill\qed
	


%
%
%
%
%
%
%
%
%
%
%
%
%
%
%
%
%
%
%
%
%
%
%
%
%
%
%
%
%
%
%
%
%
%
%
%
%
%
%
%



%
%
%
%
%
%
%

	
	
	
	
	
%
%
%
%
%
%
%
%
%
%
%
%
%
%
%
%
%
%
%
%
%
%
%
	
%}






\end{document}
