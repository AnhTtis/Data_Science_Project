%
\section{Introduction}
\label{sec_intro}
For networked control systems (NCS) with limited communication bandwidth, event-triggered control (ETC) and self-triggered control (STC) have emerged as key techniques to trade-off control performance and the usage of communication resources. In ETC, a state-dependent trigger rule is monitored continuously and a transmission 
%
 is triggered as soon as the trigger rule is violated. In STC, the controller determines at each transmission time based on available state information when the next transmission should take place.
%
%

In \cite{mazo2009self,anta2010sample}, it was demonstrated that the network load for NCS can be significantly reduced by STC compared to classical periodic sampling. For linear systems, a wide variety of STC strategies has been proposed, see, e.g., \cite{heemels2012introduction} and the references therein. For nonlinear systems, a smaller but increasing number of approaches is available. In \cite{benedetto2013digital,tiberi2013simple,theodosis2018self}, Lipschitz continuity properties are exploited to determine transmission times such that a decrease of a Lyapunov function can be guaranteed. Small gain techniques are used in \cite{tolic2012self,liu2015small}.  In \cite{anta2010sample,delimpaltadakis2020isochronous,delimpaltadakis2020region}, isochroneity properties of homogeneous systems are leveraged. In \cite{hertneck21robust_arxiv}, hybrid Lyapunov functions and a dynamic variable that encodes the past system behavior are used to determine transmission times.



Whilst there are several approaches for the design of STC mechanisms for linear systems with output feedback regulators, see, e.g., \cite{almeida2014self,gleizer2020self}, the aforementioned STC approaches for nonlinear systems all require information about the full plant state and are thus limited to state feedback. However, in most practical scenarios only a lower dimensional output can be used for control. 

To bridge this gap, we present in this paper an output feedback STC approach for nonlinear NCS. The approach is deduced from the dynamic STC approach based on hybrid Lyapunov functions with state-feedback from \cite{hertneck21robust_arxiv}. A continuous observer that is located at the sensor node is employed to reconstruct the plant state based on the plant output. The observer can, e.g., be designed using one of the methods from \cite{bernard2022observer}. Transmission times and plant inputs are determined based on the observer state. The proposed approach is non-conservative in the sense that the assumptions on plant and controller are the same as in the state-feedback case from \cite{hertneck21robust_arxiv} and any nonlinear observer that ensures asymptotic stability of the origin for the observer error can be used. We present the modifications that are required for the proposed dynamic output feedback STC approach and model the overall NCS as a hybrid system. For this system, we prove asymptotic stability of the origin. We illustrate the proposed approach with a numerical example.

The remainder of this paper is structured as follows.  In Section~\ref{sec_setup}, we present the setup of the paper and specify our control objective. Some preliminaries are discussed in Section~\ref{sec_prelim}. In Section~\ref{sec_main}, we detail the proposed STC approach and derive stability guarantees. A numerical example is given in Section~\ref{sec_ex}. Section~\ref{sec_conc} concludes the paper. 
%
%
%

\subsection*{Notation and definitions}
The nonnegative real numbers are denoted by  $\mathbb{R}_{\geq 0} $. The natural numbers are denoted by $\mathbb{N}$, and we define $\mathbb{N}_0:=\mathbb{N}\cup  \left\lbrace 0 \right\rbrace $. 
%
We denote the Euclidean norm by $\abs{\cdot}$.
%
 A continuous function $\alpha: \mathbb{R}_{\geq 0} \rightarrow \mathbb{R}_{\geq 0}$ is a class $ \mathcal{K}$ function if it is strictly increasing and $\alpha(0) = 0$. It is a class $\mathcal{K}_\infty$ function if it is a class $\mathcal{K}$ function and it is unbounded. A continuous function $\beta:\mathbb{R}_{\geq 0}\times \mathbb{R}_{\geq 0} \rightarrow \mathbb{R}_{\geq 0}$ is a class $\mathcal{K}\mathcal{L}$ function, if $\beta(\cdot,t)$ is a class $\mathcal{K}$ function for each $t \in\mathbb{R}_{\geq 0}$ and $\beta(q,\cdot)$ is nonincreasing and satisfies $\lim\limits_{t \rightarrow \infty} \beta(q,t) = 0$ for each $q\in\mathbb{R}_{\geq 0}$. A function $\beta:\mathbb{R}_{\geq 0}\times \mathbb{R}_{\geq 0} \times \mathbb{R}_{\geq 0} \rightarrow \mathbb{R}_{\geq 0}$ is a class $\mathcal{K}\mathcal{L}\mathcal{L}$ function if for each $r \geq 0$, $\beta(\cdot,r,\cdot)$ and $\beta(\cdot,\cdot,r)$ are class $\mathcal{K}\mathcal{L}$ functions.

We use \cite[Definitions 1-3]{carnevale2007lyapunov}, that are originally taken from \cite{goebel2006solutions}, to characterize a hybrid model of the considered NCS and corresponding hybrid time domains, trajectories and solutions. Moreover, we adapt the definitions of maximal solutions
 and $t-$completeness from \cite{goebel2006solutions}.