% !TeX spellcheck = en_US
\documentclass[letterpaper, 10 pt, conference]{ieeeconf} 
\pdfoutput=1 
\IEEEoverridecommandlockouts
\overrideIEEEmargins
\usepackage[english]{babel}
\usepackage{cite}
\usepackage{amsmath,amssymb,amsfonts}
\usepackage{algorithmic}
\usepackage{graphicx}
\usepackage{subcaption}
\usepackage{textcomp}
\usepackage{amsmath}
\usepackage{xcolor}
\usepackage{units}
\usepackage{booktabs}
\usepackage{comment}
\usepackage[hidelinks]{hyperref}
\usepackage[normalem]{ulem}

\let\proof\relax
\let\endproof\relax
\let\theorem\relax
\let\endtheorem\relax
\usepackage{amsthm}
\usepackage{flushend}

%\newcounter{thm}
%\newtheorem{prob}[thm]{Problem}

\newtheorem{theorem}{Theorem}[section]
\newtheorem{lemma}{Lemma}[section]
\newtheorem{proposition}[theorem]{Proposition}
\newtheorem{corollary}[theorem]{Corollary}
\newtheorem{definition}[theorem]{Definition}
\newtheorem{remark}[theorem]{Remark}
\newtheorem{assumption}{Assumption}[section]
\newtheorem{problem}{Problem}[section]


\def\BibTeX{{\rm B\kern-.05em{\sc i\kern-.025em b}\kern-.08em
		T\kern-.1667em\lower.7ex\hbox{E}\kern-.125emX}}
\pdfobjcompresslevel=0

\newif\ifmargincomments %A quick way of turning off margin comments for, say, arXiv submission
\margincommentstrue

\ifmargincomments
\newcommand{\lpmargin}[2]{{\color{blue}#1}\marginpar{\color{blue}\raggedright\footnotesize [LP]:\\ #2}}
\newcommand{\msmargin}[2]{{\color{red}#1}\marginpar{\color{red}\raggedright\footnotesize [MaS]:\\ #2}}
\else
\newcommand{\lpmargin}[2]{#1}
\newcommand{\msmargin}[2]{#1}
\fi

\newif\ifextendedversion
%\extendedversionfalse
\extendedversiontrue

\begin{document}
	
	\title{\Large \bf Urgency-aware Routing in Single Origin-destination Itineraries\\ through Artificial Currencies%\\
		%\thanks{Identify applicable funding agency here. If none, delete this.}
	}
	
	\author{Leonardo Pedroso, W.P.M.H. (Maurice) Heemels, Mauro Salazar % <-this % stops a space
		%\thanks{Eindhoven University of Technology}% <-this % stops a space
		\thanks{Control Systems Technology section, Eindhoven University of Technology, the Netherlands {\tt \footnotesize \{l.pedroso,w.p.m.h.heemels, m.r.u.salazar\}@tue.nl}}%
	}
	
	\maketitle
	\thispagestyle{plain}
	\pagestyle{plain}
	
	\begin{abstract}
		Within mobility systems, the presence of self-interested users can lead to aggregate routing patterns that are far from the societal optimum which could be achieved by centrally controlling the users' choices. In this paper, we design a fair incentive mechanism to steer the selfish behavior of the users to align with the societally optimal aggregate routing. The proposed mechanism is based on an artificial currency that cannot be traded or bought, but only {spent or received} when traveling. Specifically, we consider a parallel-arc network with a single origin and destination node within a repeated game setting whereby each user chooses from one of the available arcs to reach their destination on a daily basis.
		In this framework, taking faster routes comes at a cost, whereas taking slower routes is incentivized by a reward. The users are thus playing against their future selves when choosing their present actions. To capture this complex behavior, we assume the users to be rational and to minimize an urgency-weighted combination of their immediate and future discomfort.
		To design the optimal pricing, we first derive a closed-form expression for the best individual response strategy. Second, we formulate the pricing design problem for each arc to achieve the societally optimal aggregate flows, and reformulate it so that it can be solved with gradient-free optimization methods.
		Our numerical simulations show that it is possible to achieve a near-optimal routing whilst significantly reducing the users’ perceived discomfort when compared to a centralized optimal but urgency-unaware policy.
	\end{abstract}
	
	%\begin{IEEEkeywords}
	%\textcolor{red}{[to add: copy EJC paper kws probably]}
	%\end{IEEEkeywords}


\section{Introduction}
\label{sec:introduction}
% \begin{itemize}
%     % Diffusion of FL
%     \item {\st{Diffusion of FL}}
%     % Security threats to FL
%     \item {\st{Security threats to FL with particular focus on model poisoning}}
%     % Limitations of existing countermeasures
%     \item {\st{Current countermeasures (e.g., KRUM) and their limitations}}
%     % Proposed method and its advantages
%     \item {\st{Intuitive description of the proposed method and its difference (i.e., advantages) w.r.t. state of the art}}
%     % Main contributions
%     \item {\st{Summary of the main contributions of this work}}
%     % Paper's structure and organization
%     \item {\st{Paper's structure and organization}}
% \end{itemize}

% Diffusion of FL
Recently, {\em federated learning} (FL) has emerged as the leading paradigm for training distributed, large-scale, and privacy-preserving machine learning (ML) systems~\cite{mcmahan2017googleai,mcmahan2017aistats}. 
The core idea of FL is to allow multiple edge clients to collaboratively train a shared, global model without disclosing their local private training data.
%Specifically, an FL system consists of a central server and many edge clients; 
A typical FL round involves the following steps: {\em(i)} the server randomly picks some clients and sends them the current, global model; {\em(ii)} each selected client locally trains its model with its own private data; then, it sends the resulting local model to the server;\footnote{Whenever we refer to global/local model, we mean global/local model {\em parameters}.} {\em(iii)} the server updates the global model by computing an \emph{aggregation function}, usually the average (FedAvg), on the local models received from clients.
% \begin{enumerate}
%     \item[{\em(i)}] the server sends the current, global model to the clients and appoints some of them for training;
%     \item[{\em(ii)}] each selected client locally trains its copy of the global model with its own private data; then, it sends the resulting local model back to the server;\footnote{Whenever we refer to global/local model, we mean global/local model {\em parameters}.}
%     \item[{\em(iii)}] the server updates the global model by computing an \emph{aggregation function} on the local models received from clients (by default, the average, also referred to as FedAvg~\cite{mcmahan2017aistats}).
% \end{enumerate}
This process goes on until the global model converges. %(e.g., after a certain number of rounds or other similar stopping criteria).
%\\
% The advantages of FL over the traditional, centralized learning paradigm are undoubtedly clear in terms of flexibility/scalability (clients can join/disconnect from the FL network dynamically), network communications (only model weights\footnote{We will use \textit{parameters} and \textit{weights} interchangeably.} are exchanged between clients and server), and privacy (each client's private training data is kept local at the client's end and not uploaded to the server).
\\
% Security threats to FL
%However, the growing adoption of FL also raises security concerns~\cite{costa2022covert}, particularly about its confidentiality, integrity, and availability.
Although its advantages over standard ML, FL also raises security concerns~\cite{costa2022covert}. %, particularly about its confidentiality, integrity, and availability~\cite{costa2022covert}.
% OLD, LONG VERSION
% Indeed, some work deals with privacy leakage that may expose the local data of some clients~\cite{melis2019sp}. 
% A large body of work, instead, investigates attacks that usually aim to detriment the predictive accuracy of the learned global model. For instance, \emph{data poisoning} attacks achieve this goal by letting an adversary pollute the training set of some corrupt FL clients with maliciously crafted examples~\cite{jagielski2018sp}.
% Similarly, in \emph{model poisoning} the attacker attempts to tweak the global model weights~\cite{bhagoji2019pmlr} by directly perturbing the local model's weights of some infected FL clients before these are sent to the central server for aggregation, usually via so-called Byzantine attacks. 
% It turns out that Byzantine model poisoning attacks severely impact standard FedAvg; therefore, more robust aggregation functions must be designed to make FL systems secure.
Here, we focus on \emph{untargeted model poisoning} attacks~\cite{bhagoji2019pmlr}, where an adversary attempts to tweak the global model weights %\footnote{We will use the terms \textit{parameters} and \textit{weights} interchangeably.} 
by directly perturbing the local model's parameters of some infected clients before these are sent to the central server for aggregation.
In doing so, the adversary aims to jeopardize the global model \textit{indiscriminately} at inference time.
Such model poisoning attacks severely impact standard FedAvg; therefore, more robust aggregation functions must be designed to secure FL systems.
\\
% In this paper, we focus on designing a novel robust aggregation scheme at the server's end to contrast the effect of Byzantine model poisoning attacks.
%
% Current countermeasures and their limitations
%Several countermeasures have been proposed in the literature to combat model poisoning attacks on FL systems.
% Some methods use simple statistics more robust than plain average to smooth the impact of malicious updates (e.g., Trimmed Mean and FedMedian~\cite{yin2018icml}). 
% Other defenses implement outlier detection techniques to discard malicious updates from the aggregation performed at the server's end. Those are either based on heuristics (e.g., Krum/Multi-Krum~\cite{blanchard2017nips} and Bulyan~\cite{mhamdi2018pmlr}) or data-driven approaches (e.g., K-means clustering~\cite{shen2016acm} or DnC via spectral analysis~\cite{shejwalkar2021ndss}). 
% Finally, some strategies rely on a centralized ``source of trust'' to spot potential malicious updates (e.g., FLTrust~\cite{cao2020fltrust}).
% Several countermeasures have been proposed in the literature to combat model poisoning attacks on FL systems, i.e., to discard possible malicious local updates from the aggregation performed at the server's end. 
% These techniques range from simple statistics more robust than plain average (e.g., Trimmed Mean and FedMedian~\cite{yin2018icml}) to outlier detection heuristics (e.g., Krum/Multi-Krum~\cite{blanchard2017nips} and Bulyan~\cite{mhamdi2018pmlr}) or data-driven approaches (e.g., spectral analysis via K-means clustering~\cite{shen2016acm} or spectral analysis), or methods based on ``source of trust'' (e.g., FLTrust~\cite{cao2020fltrust}).
% OLD, LONG VERSION
%Several countermeasures have been proposed in the literature to combat Byzantine model poisoning attacks on FL systems.
% Descriptive statistics
% For example, Trimmed Mean and FedMedian aggregate local model updates using more robust statistics than standard average~\cite{yin2018icml}.
%
% % Heuristics for outlier detection
% Many existing Byzantine-resilient strategies implement some outlier detection heuristics to discard the model updates sent by potentially malicious clients from the input of the aggregation function.
% One of the most popular heuristics is Krum~\cite{blanchard2017nips}.
% This strategy tries to mitigate the impact of Byzantine attacks by selecting as a global model the local model with the smallest sum of Euclidean distances to {\em all} the other local models.
% Although powerful, Krum requires the server to know (or, at least, estimate) the number of malicious FL clients upfront, which is generally impossible in a realistic attack scenario. %
% Moreover, Krum may become ineffective for complex, high-dimensional model parameter spaces due to the curse of dimensionality.
% Bulyan~\cite{mhamdi2018pmlr} tries to overcome this issue by combining Krum with a variant of Trimmed Mean.
% % Data-driven outlier detection
% Other strategies use data-driven outlier detection techniques -- e.g., via K-means clustering~\cite{shen2016acm} -- to spot potential malicious local model updates. 
% %For instance, Shen et al. propose to cluster local model updates with K-means and thus identify outliers.
%
% % Other techniques
% As far as the server is concerned, any local model received can be from a potential malicious client. 
% FLTrust~\cite{cao2020fltrust} assumes the server acts as a client, i.e., trains a local model on an additional {\em trustworthy} dataset at the server's end and compares it against all the local models from other clients. 
% This way, the server can rely on some ``source of trust'' when discarding potentially malicious clients.
%\\
% Limitations of existing Byzantine-resilient strategies
Unfortunately, existing defense mechanisms either rely on simple heuristics (e.g., Trimmed Mean and FedMedian by~\cite{yin2018icml}) or need strong and unrealistic assumptions to work effectively (e.g., foreknowledge or estimation of the number of malicious clients in the FL system, as for Krum/Multi-Krum~\cite{blanchard2017nips} and Bulyan~\cite{mhamdi2018pmlr}, which, however, cannot exceed a fixed threshold).
Furthermore, outlier detection methods using K-means clustering~\cite{shen2016acm} or spectral analysis like DnC~\cite{shejwalkar2021ndss} do not directly consider the temporal evolution of local model updates received.
Finally, strategies like FLTrust~\cite{cao2020fltrust} require the server to collect its own dataset and act as a proper client, thereby altering the standard FL protocol.
\\
% OLD, LONG VERSION
% Overall, existing Byzantine-resilient strategies are either simple heuristics (e.g., FedMedian) or, if they are more complex, they rely on strong and unrealistic assumptions to work effectively (e.g., knowing the number of malicious clients in the FL system in advance, as for Krum and alike).
% Furthermore, data-driven outlier detection methods do not consider the temporary evolution of local model updates received (e.g., K-means clustering). 
% Finally, strategies like FLTrust requires the server to collect its own dataset and act as a proper client, thereby altering the standard FL protocol.
%
% Description of the proposed method
This work introduces a novel pre-aggregation \textit{filter} robust to untargeted model poisoning attacks. Notably, this filter $(i)$ operates without requiring prior knowledge or constraints on the number of malicious clients and $(ii)$ inherently integrates temporal dependencies. 
The FL server can employ this filter as a preprocessing step before applying \textit{any} aggregation function, be it standard like FedAvg or robust like Krum or Bulyan.
Specifically, we formulate the problem of identifying corrupted updates as a multidimensional (i.e., matrix-valued) time series anomaly detection task. 
The key idea is that legitimate local updates, resulting from well-calibrated iterative procedures like stochastic gradient descent (SGD) with an appropriate learning rate, show \textit{higher predictability} compared to malicious updates. This hypothesis stems from the fact that the sequence of gradients (thus, model parameters) observed during legitimate training exhibit regular patterns, as validated in Section~\ref{subsec:intuition}. %until convergence. 
%This regularity may be more pronounced for smooth convex loss functions, but it can still be captured within an appropriate time window, even for more complex and convoluted loss surfaces. 
%We provide evidence of this claim in Appendix~B, where we show that the average mutual information (i.e., ``predictability''), calculated over pairs of legitimate model updates sent at different FL rounds, is significantly higher than the corresponding computation for a malicious client.
\\
Inspired by the matrix autoregressive (MAR) framework for multidimensional time series forecasting~\cite{chen2021je}, we propose the FLANDERS ({\em \textbf{F}ederated \textbf{L}earning meets \textbf{AN}omaly \textbf{DE}tection for a \textbf{R}obust and \textbf{S}ecure}) filter.
The main advantages of FLANDERS over existing strategies like FLDetector~\cite{zhao2020multivariate} are its resilience to large-scale attacks, where $50\%$ or more FL participants are hostile, and the capability of working under realistic non-iid scenarios.
We attribute such a capability to two key factors: $(i)$ FLANDERS works without knowing a priori the ratio of corrupted clients, and $(ii)$ it embodies temporal dependencies between intra- and inter-client updates, quickly recognizing local model drifts caused by evil players. Below, we summarize our main contributions:

\begin{itemize}
\item[{\em(i)}]
We provide empirical evidence that the sequence of models sent by legitimate clients is more predictable than those of malicious participants performing untargeted model poisoning attacks.
\\
\item[{\em(ii)}] 
We introduce FLANDERS, the first pre-aggregation filter for FL robust to untargeted model poisoning based on multidimensional time series anomaly detection.
\\
\item[{\em(iii)}] 
We integrate FLANDERS into Flower,\footnote{\scriptsize{\url{https://flower.dev/}}} a popular FL simulation framework for reproducibility.
\\
\item[{\em(iv)}] 
We show that FLANDERS improves the robustness of the existing aggregation methods under multiple settings: different datasets, client's data distribution (non-iid), models, and attack scenarios.
\\
\item[{\em(v)}] 
We publicly release all the implementation code of FLANDERS along with our experiments.\footnote{\scriptsize{\url{https://anonymous.4open.science/r/flanders_exp-7EEB}}}
\end{itemize}

% Paper's structure and organization
The remainder of the paper is structured as follows. %some related work and the current state-of-the-art solutions to security issues that FL entails. 
Section~\ref{sec:background} covers background and preliminaries. 
In Section~\ref{sec:related}, we discuss related work.
Section~\ref{sec:problem} and Section~\ref{sec:method} describe the problem formulation and the method proposed. % to tackle it. 
Section~\ref{sec:experiments} gathers experimental results. %, and Section~\ref{sec:limitations} discusses some limitations of this work.
Finally, we conclude in Section~\ref{sec:conclusion}.
 %discusses the limitations of this work and draws future research directions.
%reports conclusions and draws perspectives for future research directions.

%%%%%%% OLD %%%%%%%
%to overcome the resilience of Byzantine failures in distributed Stochastic Gradient Descent computations. 
% The strength of Krum is its time complexity, which is linear in the gradient dimension. 
% However, the robustness of the approach is guaranteed for gradient-based learning applications only when the majority of the clients are not compromised. 
% Besides, the aggregation mechanism of Krum, as well as that of similar methods, is robust from a coarse-grained perspective and does not provide solutions to errors and perturbations that may occur at inference time.
%A related approach to~\cite{blanchard2017nips} is the work of Su et al.~\cite{su2016dc}. Here, the authors propose an iterated approximate agreement to tackle a multi-layer scenario attacked by Byzantine agents. 
%However, the method works efficiently on the sole discrete context and it is inapplicable to continuous state environments.
%\gabri{Maybe, we should just talk about the main limitations of existing countermeasures without digging into their details (or, we can just mention Krum as this is the most popular one). I will move the description of all these methods to the Related Work section.}
% !TeX spellcheck = en_US
\section{Problem Statement}\label{sec:prob_statement}

This section states the mechanism design problem, which closely follows the formulation in \cite{SalazarPaccagnanEtAl2021}. Three different perspectives are of interest, each corresponding to a subsection: i)~the macroscopic perspective of the central operator, aiming to minimize the societal costs that result from the routing patterns; ii)~the microscopic perspective of each self-interested user, who desires to minimize their daily perceived discomfort;  and iii)~the mesoscopic overarching perspective of the incentive mechanism design framework to align these two seemingly opposing objectives.

%\subsection{Incentive mechanism framework}

Consider the mobility network  with a single origin and destination node connected by $n\in \mathbb{N}$ distinct itineraries depicted in Fig.~\ref{fig:network_example}. We consider a repeated game setting whereby each user chooses from one of the available arcs to reach their destination at each discrete time $t\in \mathbb{N}$.

The incentive mechanism that is employed is based on an artificial currency---Karma. In this framework, taking a particular itinerary comes at a cost of Karma. Let $\mathbf{p}\in \mathbb{R}^n$ denote the prices of the itineraries, i.e., choosing arc $j$ comes at a cost $\mathbf{p}_j$. Users are not allowed to buy or trade Karma, and they can only select arcs that maintain their Karma-level non-negative. It is, hence, crucial that certain (more uncomfortable) arcs are assigned negative prices, which means that users are awarded Karma for taking them.

From a microscopic perspective, denote the route choice of a user $i$ at time instant $t$ by the binary vector $\mathbf{y}^i(t) \in \{0,1\}^n$, whose entry $j$ is given by $\mathbf{y}^i_j(t) = 1$ if the user $i$ chooses arc $j$ at time $t$, and $\mathbf{y}^i_j(t) = 0$ otherwise. Since each user may not travel at time $t$ we can have $\mathbf{y}^i(t) = \mathbf{0}$. Therefore, it follows that $\mathbf{1}^\top\mathbf{y}^i(t) \leq 1$. Let $k^i(t) \in \mathbb{R}_{\geq0}$ denote the amount of Karma that a user $i$ owns at time $t$. Following a routing choice the Karma level is updated according to $k^i(t+1) = k^i(t)-\mathbf{p}^\top\mathbf{y}^i(t)$. 

From a mesoscopic perspective, let $\mathbf{x}(t) \in [0,1]^n$ denote the fraction of users crossing each arc at time $t$. Given a scenario with $M$ users, it is defined as $\mathbf{x}(t) = \frac{1}{M}\sum_{i=1}^M\mathbf{y}^i(t)$. To account for non-traveling users, it is assumed that each user has a constant probability $P_{\text{home}} \in [0,1]$  of not traveling.  Conversely, the probability for a user to travel is $P_{\text{go}}= 1-P_{\text{home}}$ and $\mathbb{E}\left[\mathbf{1}^\top\mathbf{x}(t)\right] = P_{\text{go}}$.


\subsection{Central Operator's Problem}
From the macroscopic perspective of the central operator of the mobility network, the flows across each arc cause a {societal} cost. Let $\mathbf{c}:[0,1]^n \to \mathbb{R}_{\geq 0}^n$ denote the {societal} cost function. It models the {societal} cost of each arc $j$ per user, $\mathbf{c}_j(\mathbf{x}_j(t))$, which is assumed to be monotonically increasing with the fraction  $\mathbf{x}_j(t)$ of users taking it. The desire of the central operator is that the aggregate flows minimize the total {societal} cost {$C(\mathbf{x}):=\mathbf{c}(\mathbf{x})^\top\mathbf{x}$}, which is formulated in the following problem:

\begin{problem}[Central Operator's Problem]\label{prb:design}\color{black}
	The central operator aims at routing customers so that the aggregate flows are
	%$\lim\limits_{t\to \infty} \mathbf{x}(t) = \mathbf{x}^{\star}$, where
	\begin{equation*}
		\begin{split}
			\mathbf{x}^{\star} \in &\arg \min \limits_{\mathbf{x}\in [0,1]^n}  {C(\mathbf{x})}\\
			& \;\mathrm{s.t.} \phantom{ \min \limits_{x\in [0,1]^n} } \mathbf{1}^\top\mathbf{x} = P_{\mathrm{go}}.
		\end{split}
	\end{equation*}
\end{problem}

\subsection{Individual User's Problem}
From the microscopic perspective of each user, taking an itinerary $j$ comes with a discomfort. Let ${\mathbf{d}:[0,1]^n \to \mathbb{R}_{\geq 0}^n}$ denote the user's discomfort function. It  models the discomfort that stems from taking arc $j$ per user, $\mathbf{d}_j(\mathbf{x}_j(t))$, which is assumed to be monotonically increasing with the fraction of users taking it, $\mathbf{x}_j(t)$.
In contrast to well-known monetary tolling schemes~\cite{BergendorffHearnEtAl1997}, the individual users' complex behavior cannot be captured within a static setting:
From their self-interested view-point, the users are assumed to make choices in order to minimize their traveling discomfort without reaching a negative level of Karma. Thus, the users are playing against their future selves when deciding whether to spend or receive Karma. Furthermore, the perception of discomfort of a user varies daily. The sensitivity to discomfort of a user $i$ at time $t$ is denoted by $s^i(t)$, which is a weighting factor of the daily discomfort.  The sensitivities $s^i(t)$ are assumed to be i.i.d. extractions (w.r.t. $i$ and $t$) of a common probability density function $\rho:[s_{\mathrm{min}},s_{\mathrm{max}}] \to [0,1]$ with support set  $[s_{\mathrm{min}},s_{\mathrm{max}}] \subseteq  \mathbb{R}_{\geq 0}$ and expected value $\bar{s} \in \mathbb{R}_{\geq0}$.  To capture this complex behavior, the users are assumed to be rational and to minimize a combination of their immediate discomfort, weighted by their immediate urgency, and the discomfort encountered for a time period $T$ into the future, weighted by their average urgency. Additionally, we assume each user $i$ to be conservative in terms of Karma, i.e., they will make the route decisions so that their Karma at the end of the horizon will not fall below a reference value ${k_{\mathrm{ref}}^i \in \mathbb{R}_{\geq0}}$. For example, a user may choose a reference value of $\mathbf{p}_1$ to ensure that {they} can still afford to travel in arc $1$ at the end of the horizon. It, thus, depends on the pricing policy $\mathbf{p}$.  Herein, we will assume it to be time-invariant and randomly distributed among the users according to the distribution $\theta_{\mathbf{p}}:\mathbb{R}_{\geq 0} \to\mathbb{R}_{\geq0}$.
%\msmargin{Combining everything together}{more technical way of saying it? E.g., Formally, we obtain...} 
{Formally, we obtain} the following individual user's problem:
%Herein, we will assume it to be time-invariant and randomly distributed among the users according to the distribution $\theta:[k_{\mathrm{ref}_{\mathrm{min}}},k_{\mathrm{ref}_{\mathrm{max}}}] \to  \mathbb{R}_{\geq0}$ with support set  $[k_{\mathrm{ref}_{\mathrm{min}}},k_{\mathrm{ref}_{\mathrm{max}}}] \subseteq  \mathbb{R}_{\geq 0}$.
%\textcolor{red}{[Change the support set of kref consistently]}
%, for example, a user may have a greater urgency when traveling to an important meeting than when commuting.  

%\lpmargin{$\theta_{\mathbf{p}}:\mathbb{R}_{\geq 0} \to\mathbb{R}_{\geq0}$}{more intuitive symbol? - let's discuss this}.
%\msmargin{he/she}{use \textbf{they} everywhere}

%\msmargin{In contrast to well-known monetary tolling \msmargin{schemes}{add ref}, the individual users' complex behavior cannot be captured within a static setting.}{perhaps this statement should be earlier} \msmargin{Combining everything together yields the following individual user's problem.}{a bit abrupt wrt the previous sentence...}
\begin{problem}[Individual User's Problem]\label{prb:individual}
	At time $t \in \mathbb{N}$, given the flows $\mathbf{x}$ and prices $\mathbf{p}$ a traveling user with Karma level $k\geq 0$, reference $k_{\mathrm{ref}}$, and sensitivity $s$ will choose {their} route as $\mathbf{y}^{\star}$ resulting from
	%		\par\nobreak\vspace{-10pt}
	%		\begingroup
	%		\allowdisplaybreaks
	%		\begin{small}
		\begin{subequations}\label{eq:singleAgentAverage}
			\begin{align}\label{eq:singleAgentAverage_cost}
				(\mathbf{y}^\star,\bar{\mathbf{y}}^\star) \in \mathop{\mathrm{argmin}}_{\mathbf{y} \in \mathcal{Y},\;\bar{\mathbf{y}} \in \bar{\mathcal{Y}}} \;&s\, \mathbf{d}(\mathbf{x})^\top\mathbf{y}+ T\,\bar{s}\, \mathbf{d}(\mathbf{x})^\top \bar{\mathbf{y}}\\  \label{eq:singleAgentAverage_c1}
				\mathrm{s.t.}\;\; &k-\mathbf{p}^\top \mathbf{y} - T\mathbf{p}^\top\bar{\mathbf{y}} \geq k_\mathrm{ref}\\ \label{eq:singleAgentAverage_c2}
				&\mathbf{p}^\top \mathbf{y} \leq k,
			\end{align}
		\end{subequations}
		%			\end{small}
	%		\endgroup
	with $ \mathcal{Y} = \{\mathbf{y}\in\{0,1\}^n: \mathbf{1}^\top \mathbf{y} = 1\}$ and $\bar{\mathcal{Y}} =\{\mathbf{y}\in[0,1]^n:\mathbf{1}^\top \mathbf{y}= 1\}$.
	%\textcolor{black}{Thus, the user chooses arc $j^\star$ if $\mathbf{y}^\star = \mathbf{e_{j^\star}}$.}
	Non-traveling users have $\mathbf{y}^\star = \mathbf{0}$.
\end{problem}


\subsection{Mechanism Design Problem}

Similar to~\cite{SalazarPaccagnanEtAl2021} and following the notation in~\cite{GorokhBanerjeeEtAl2019}, we consider a non-atomic game framework, which corresponds to the limit case where users form a continuum with ${M\to \infty}$. To describe an infinite-user population, let ${\eta_t : \mathbb{R}_{\geq 0}  \times \mathbb{R}_{\geq 0} \to \mathbb{R}_{\geq 0}}$ denote the instantaneous distribution of the Karma level and reference in the population at time $t$, where $\int_0^{\infty} \!\!\int_0^\infty \eta_t (k,k_\mathrm{ref}) \,\mathrm{d}k \, \mathrm{d}k_\mathrm{ref} = 1$. For the infinite-user setting, the Nash and Wardrop Equilibrium (WE) are identical~\cite{PaccagnanGentileEtAl2018} and can be defined as follows:
% let $\eta_t(k,k_\mathrm{ref}) : \mathbb{R}_{\geq0} \times \mathbb{R}_{\geq 0} \to \mathbb{R}_{\geq 0}$ be the instantaneous distribution of the Karma level $k$ and reference $k_\mathrm{ref}$ in the population.

\begin{definition}[Wardrop Equilibrium]\label{def:WE} \color{black}
	$\mathbf{x}^\mathrm{WE}(t)\in[0,1]^n$ satisfying $\mathbf{1}^\top \mathbf{x}^\mathrm{WE}(t) = P_\mathrm{go}$ is a WE at time $t$, if %there exists $\mathbf{y}^\star(\mathbf{x}^\mathrm{WE}(t),s,k,k_\mathrm{ref})\in \mathcal{Y}^\star(\mathbf{x}^\mathrm{WE}(t),s,k,k_\mathrm{ref})$ so that
	{\small
		\begin{equation*}
			\begin{split}
				&\mathbf{x}^\mathrm{WE}(t)  =\\
				& \int_{0}^\infty \!\! \!\! \int_{0}^\infty \!\!\!\! 	\int^{s_\mathrm{max}}_{s_\mathrm{min}}\!\!\!\! \mathbf{y}^{\star}\!(\mathbf{x}^\mathrm{WE}(t),s,k,k_\mathrm{ref})\,\rho(s)\,\eta_t(k,k_{\mathrm{ref}})\, \mathrm{d}s\,\mathrm{d}k\,\mathrm{d}k_{\mathrm{ref}},
			\end{split}
	\end{equation*}}%
	where $\mathbf{y}^\star$ is a best response strategy that follows from Problem~\ref{prb:individual}.
\end{definition}



%\msmargin{Fig.~\ref{fig:scheme_karma}}{where is the mechanism? It looks more like the simulation model}

At each time $t$ the aggregate choices of the users are modeled by the WE $\mathbf{x}^{\mathrm{WE}}(t)$. Fig.~\ref{fig:scheme_karma} depicts a scheme of a time-step of the overall model. The mechanism design problem is then to select the arc prices, so that the daily WE converges to the system optimum $\mathbf{x}^\star$, as stated in the problem below:

\begin{figure}[ht]
	\centering
	\includegraphics[width = 0.8\linewidth]{fig/scheme_karma.pdf}
	\caption{Schematic representation of one time-step of the overall model.}
	\label{fig:scheme_karma}
\end{figure}

\begin{problem}[Pricing Problem]\label{prb:prices}
	Given a desired system optimum $\mathbf{x}^\star$, select $\mathbf{p}\in\mathbb{R}^n$ so that $\lim_{t\to\infty} \mathbf{x}^\mathrm{WE}(t) = \mathbf{x}^\star$.
\end{problem}

To ensure the well-posedness of the pricing problem, the following key assumptions are made on the existence, uniqueness, and convergence of a WE. Given that the best response strategy cannot be formulated in a static setting, and a mixed user strategy is not meaningful, these assumptions are, by no means, obvious. Future research endeavors will focus on the intricacies of the game-theoretic framework, whereas, in this paper, the focus is on the pricing design framework.

%which are reasonable in the context of the problem. \textcolor{red}{[add better explanation here or (even better) a proof :) - check continuity of $\gamma_j$ as a function of $d$ (existence) - contractivity of eq in Def II.1 for a fixed $\eta$ (uniqueness and convergence)]}.

\begin{assumption}[Existence and uniqueness of WE]\label{ass:WE}
	Given a Karma level distribution $\eta_t : \mathbb{R}_{\geq 0}  \times \mathbb{R}_{\geq 0} \to \mathbb{R}_{\geq 0}$, a WE $\mathbf{x}^\mathrm{WE}(t)$ exists and is unique.
\end{assumption}

\begin{assumption}[Convergence of WE]\label{ass:WE_convergence}
	For a given pricing strategy $\mathbf{p}$, a stationary WE exists, i.e., ${\mathbf{x}^\mathrm{WE}_{\infty} := \lim_{t\to\infty} \mathbf{x}^\mathrm{WE}(t)}$, irrespective of the initial Karma level distribution $\eta_0 : \mathbb{R}_{\geq 0}  \times \mathbb{R}_{\geq 0} \to \mathbb{R}_{\geq 0}$.
\end{assumption}


\section{Best Response Strategy}\label{sec:bestresponse}

In this section, we focus on the individual user's problem. Specifically, we examine its properties and derive a closed-form solution of the best response strategy, which we will prove to be of paramount importance to the pricing design procedure proposed in this paper. The following result details necessary and sufficient conditions for the feasibility of Problem~\ref{prb:individual}.

\begin{lemma}\label{lem:feasibility}
	Consider a traveling user with Karma $k$, sensitivity $s$, Karma reference $k_\mathrm{ref}$, and prices $\mathbf{p}$. Problem~\ref{prb:individual} is feasible if and only if {\small $k \geq \max(0,k_\mathrm{ref}+(\min_j\mathbf{p}_j)(T+1))$}.
\end{lemma}
\begin{proof}
	The proof can be found in
	\ifextendedversion
	Appendix~\ref{app:proof_feasibility}.
	\else
	the extended version of this paper~\cite{extendedversion}.
	\fi
\end{proof}

%It is \msmargin{challenging}{avoid judging articles: Simply start with ``To derive a closed-form..., we follow...''} 
{To derive a closed-form solution to Problem~\ref{prb:individual}, we} follow a divide-and-conquer approach. {In a first instance}, in the following theorem, we establish an equivalence between the solutions of Problem~\ref{prb:individual} and a reduced best response problem whose discomforts and prices can be strictly ordered. {More specifically, we make three statements about Problem~\ref{prb:individual}. First, if, for a given arc $j$, there exists an arc $i$ with strictly lower discomfort and cost, then arc $j$ is unreasonable, in the sense that it is never chosen. Second, the discomforts and cost of the reduced set of arcs that are not unreasonable and have distinct discomfort values can be strictly ordered. Third, all integer solutions of Problem~\ref{prb:individual} can be obtained by the solutions to a reduced best response problem, whose discomforts and prices can be strictly ordered. These statements are presented with rigor in the following theorem.}


% \msmargin{the of}{?}

%\msmargin{feasibility}{are you referring to Lemma III.1? }

\begin{theorem}\label{thm:brs_general}
	Consider a traveling user with Karma $k$, sensitivity $s$, and Karma reference $k_\mathrm{ref}$, aggregate flow $\mathbf{x}$, and prices $\mathbf{p}$. Assume, without loss of generality, that the itineraries are numbered, so that $\mathbf{d}_1(\mathbf{x}_1) \leq  \ldots \leq \mathbf{d}_n(\mathbf{x}_n)$ is satisfied. Then, {under the feasibility conditions of Lemma~\ref{lem:feasibility}}:\\
	i) $j^\star \notin \mathcal{J}_u$, where $\mathbf{y}^\star = \mathbf{e_{j^\star}}$ and {\small 	$\mathcal{J}_u := \{ j\in \{1,\ldots,n\} \;|\; \exists i \in  \{1, \ldots n\}: \mathbf{p}_i  \leq  \mathbf{p}_j \land \mathbf{d}_i(\mathbf{x}_i) < \mathbf{d}_j(\mathbf{x}_j) \};$}\\
	ii) {\small $\!\forall i,j  \!\in\!\! \{1,\ldots,n\} \!\setminus \!( \mathcal{J}_u \cup \mathcal{J}_e) \,j\!<\! i  \Longrightarrow \mathbf{p}_j\! > \!\mathbf{p}_i \land \mathbf{d}_j(\mathbf{x}_j)\! > \!\mathbf{d}_i(\mathbf{x}_i)$}, where 	
	{\small \begin{equation*}
			\begin{split}
				&\mathcal{J}_e \!:= \!\left\{ j\!\in \!\{1,\ldots,n\}  \,|\, \exists i \! \in \! \{1, \ldots n\} : \right.\\&\quad \quad \quad \quad\quad 
				\left. \mathbf{d}_i(\mathbf{x}_i) \!= \!\mathbf{d}_j(\mathbf{x}_j) \land (\mathbf{p}_i <\mathbf{p}_j \lor (\mathbf{p}_i  = \mathbf{p}_j  \land i<j) )\right\}.
			\end{split}
	\end{equation*}}\\
	iii) if $(\mathbf{e_{q}}, \mathbf{\bar{y}^{q}})$, $q = 1,\ldots, Q$ are all the solutions to Problem~\ref{prb:individual} for {reduced} aggregate flows $ \{\mathbf{x}_j\}_{j\in \{1,\ldots,n\} \setminus ( \mathcal{J}_u \cup  \mathcal{J}_e) }$, and {reduced} prices $\{\mathbf{p}_j\}_{j\in \{1,\ldots,n\} \setminus ( \mathcal{J}_u \cup  \mathcal{J}_e) }$, then $\mathbf{y}^\star = \mathbf{e_{j^\star}}$ with
	{\small \begin{equation*}
			\begin{split}
				j^\star \in \bigg\{ j\in \{1,\ldots,n\}| \exists q \in \{1,\ldots,Q\} : \mathbf{d}_j(\mathbf{x}_j) = \mathbf{d}_{q^{\star}}(\mathbf{x}_{q^{\star}}) \land  \\ k \geq \mathbf{p}_j \land 
				\left .k - \mathbf{p}_j -T\sum \nolimits_{i\in \{1,\ldots,n\} \setminus ( \mathcal{J}_u \cup  \mathcal{J}_e)} \mathbf{p}_i  \mathbf{\bar{y}^{q}}_i \geq k_{\mathrm{ref}}\right\}
			\end{split}
	\end{equation*}}%
	are all the integer solutions to Problem~\ref{prb:individual}.
\end{theorem}
\begin{proof}
	The proof can be found in
	\ifextendedversion
	Appendix~\ref{app:proof_brs_general}.
	\else
	the extended version of this paper~\cite{extendedversion}.
	\fi
\end{proof}




%if two arcs have the same discomfort, albeit possibly different prices, we show that both are optimal solutions for a sufficiently high level of Karma. Furthermore, 
%
%\textcolor{red}{[Do this before - what we saying is ...... whic is condendensed in teh following the]}
%Three \lpmargin{remarks}{I would somehow explain what the three subresults of the Thm mean (...) - that's exactly what the First, second, and third are - I made it clearer} 

{A few remarks are in order regarding Theorem~\ref{thm:brs_general}. First, depending on the prices and discomforts at a given time, there may be arcs that are not chosen for any sensitivity or Karma level. 
Second, if two arcs have the same discomfort, albeit possibly different prices, both are {equally fit} for a sufficiently high level of Karma. Third, even though similar equivalence conditions could have been stated for the non-integer component of the solutions, they were omitted for the sake of brevity. In a second instance, in the following theorem, a closed-form solution is presented for the aforementioned reduced problems.}

%\msmargin{Let}{use definition environment before the Thm for all these items?} 

%Afterwards, we derive a closed-form solution to such reduced problem.


\begin{theorem}\label{thm:brs}
	Consider a traveling user with Karma $k$, sensitivity $s$, and Karma reference $k_\mathrm{ref}$, an aggregate flow $\mathbf{x}$, and prices $\mathbf{p}$.   Assume that  $\mathbf{d}_1(\mathbf{x}_1) <  \ldots < \mathbf{d}_n(\mathbf{x}_n)$ and $\mathbf{p}_1 > \ldots >\mathbf{p}_n$. Let $k(j_1,j_2):= k_{\mathrm{ref}} + \mathbf{p}_{j_1} + T\mathbf{p}_{j_2}$,
	\ifextendedversion
	\begin{equation}\label{eq:j_hat}
		\hat{j}_a := \mathop{\mathrm{argmin}}_{\substack{i\in \{1,\ldots,n\} \setminus \{a\}\\  k \geq \min(k(j,a),k(j,i)) \\ k \leq \max(k(j,a),k(j,i)) }}\frac{\mathbf{d}_i(\mathbf{x}_i)-\mathbf{d}_a(\mathbf{x}_a)}{\mathbf{p}_a-\mathbf{p}_i},
	\end{equation}
	\else
	\begin{equation*}
		\hat{j}_a := \mathop{\mathrm{argmin}}_{\substack{i\in \{1,\ldots,n\} \setminus \{a\}\\  k \geq \min(k(j,a),k(j,i)) \\ k \leq \max(k(j,a),k(j,i)) }}\frac{\mathbf{d}_i(\mathbf{x}_i)-\mathbf{d}_a(\mathbf{x}_a)}{\mathbf{p}_a-\mathbf{p}_i},
	\end{equation*}
	\fi
	\ifextendedversion
	\begin{equation}\label{eq:y_j_a_star}
		\mathbf{\bar{y}}^\star(j,a) := \frac{(k-k(j,\hat{j}_a))\mathbf{e_a} - (k-k(j,a))\mathbf{e_{\hat{j}_a}}}{T(\mathbf{p}_a-\mathbf{p}_{\hat{j}_a})},
	\end{equation}
	\else
	\begin{equation*}
		\mathbf{\bar{y}}^\star(j,a) := \frac{(k-k(j,\hat{j}_a))\mathbf{e_a} - (k-k(j,a))\mathbf{e_{\hat{j}_a}}}{T(\mathbf{p}_a-\mathbf{p}_{\hat{j}_a})},
	\end{equation*}
	\fi
	\ifextendedversion
	\begin{equation}\label{eq:a_hat}
		\hat{a} := \mathop{\mathrm{argmin}} \limits_{a\in \{i,\ldots,n\}} \mathbf{d}(\mathbf{x})^\top	\mathbf{\bar{y}}^\star(j,a),
	\end{equation}
	\else
	\begin{equation*}
		\hat{a} := \mathop{\mathrm{argmin}} \limits_{\substack{a\in \{i,\ldots,n\} \\ k \geq \min(k(j,a),k(j,\hat{j}_a)) \\  k \leq \max(k(j,a),k(j,\hat{j}_a))}} \mathbf{d}(\mathbf{x})^\top	\mathbf{\bar{y}}^\star(j,a),
	\end{equation*}
	\fi
	\begin{equation*}
		\mathbf{\bar{y}_j}^\star := \begin{cases}
			\mathbf{\bar{y}}^\star(j,\hat{a}), \, & k < k(j,1)\\
			\mathbf{e_1}, \, & k \geq k(j,1)
		\end{cases},
	\end{equation*}
	\ifextendedversion
	\begin{equation}\label{eq:gamma_ij}
		\gamma_{i,j} := \begin{cases}
			\frac{T\mathbf{d}^\top (\mathbf{x})\left(	\mathbf{\bar{y}_i}^\star - 	\mathbf{\bar{y}_j}^\star\right)}{\mathbf{d}_j(\mathbf{x}_j)-\mathbf{d}_i(\mathbf{x}_i)}, \; & k \geq \max(0,\mathbf{p}_i,k(i,n))\\
			\infty, & \text{otherwise}
		\end{cases},
	\end{equation}
	\else
	\begin{equation*}
		\gamma_{i,j} := \begin{cases}
			\frac{T\mathbf{d}^\top(\mathbf{x})\left(	\mathbf{\bar{y}_i}^\star - 	\mathbf{\bar{y}_j}^\star\right)}{\mathbf{d}_j(\mathbf{x}_j)-\mathbf{d}_i(\mathbf{x}_i)}, \; & k \geq \min(0,k(i,n))\\
			+\infty, & \text{otherwise}
		\end{cases},
	\end{equation*}
	\fi
\begin{equation*}
		\underline{\gamma}_j := \begin{cases}
			\max_{ i\in \{j+1,\ldots,n\}} \gamma_{j,i} & j <n \\
			-\infty & j = n
		\end{cases} ,
	\end{equation*}
\begin{equation*}
		\bar{\gamma}_j := \begin{cases}
			\min_{i\in \{1, \ldots,j-1\}}  \gamma_{i,j} & j >1 \\
			+\infty & j = 1
		\end{cases},
	\end{equation*}
	\begin{equation*}
		\gamma_{j}(k,k_{\mathrm{ref}},\mathbf{p},\mathbf{d(x)}) := \begin{cases}
			s_{\mathrm{min}}/\bar{s}, \;  & j = n\\
			[\bar{\gamma}_{j+1}]_{s_{\mathrm{min}}/\bar{s}}^{s_{\mathrm{max}}/\bar{s}} ,  \;  &  \bar{\gamma}_{j+1} \geq \underline{\gamma}_{j+1}\\
			\gamma_{j+1},  \; &  \bar{\gamma}_{j+1} < \underline{\gamma}_{j+1}\\
			s_{\mathrm{max}}/\bar{s}, \;  & j = 0,
		\end{cases}
	\end{equation*}
	where the dependence on $k$, $k_{\mathrm{ref}}$, $\mathbf{p}$, and $\mathbf{d(x)}$ was dropped to alleviate the notation. Then,  {under the feasibility conditions of Lemma~\ref{lem:feasibility}}, an optimal response strategy that follows from Problem~\ref{prb:individual} is $\mathbf{y}^\star = \mathbf{e_{j^\star}}${, if and only if} $\bar{\gamma}_{j^\star} \geq \underline{\gamma}_{j^\star}$ and  ${\gamma_{j^{\star}} \leq s/\bar{s} \leq \gamma_{j^\star-1}}$.
\end{theorem}
\begin{proof}
	The proof can be found in
	\ifextendedversion
	Appendix~\ref{app:proof_brs}.
	\else
	the extended version of this paper~\cite{extendedversion}.
	\fi
\end{proof}

\begin{figure}[ht]
	\centering
	\includegraphics[width = 1\linewidth]{fig/decision_landscape-eps-converted-to.pdf}
	\caption{Best response strategy of Problem~\ref{prb:individual} for aggregate flows $\mathbf{x}^\star$, prices $\mathbf{p}^\star$, and $k_\mathrm{ref} = 0$.}
	\label{fig:decision_landscape}
\end{figure}

A few remarks are in order. First, an example of a decision landscape generated by the closed-form solution in Theorems~\ref{thm:brs_general} and \ref{thm:brs} is depicted in Fig.~\ref{fig:decision_landscape}. Second, note that there is an attractive invariant Karma set contained in {\small $[0,  \; k_\mathrm{ref}^i + (T+1)\max_{j}\mathbf{p}_j - \min_j \mathbf{p}_j]$}. Third, the best response strategy is invariant on a positive scaling of $k$, $k_\mathrm{ref}$, and $\mathbf{p}$, i.e., $\mathbf{e_{j^\star}}$ is a best response strategy for $k$, $k_\mathrm{ref}$, and $\mathbf{p}$ {if and only if} it is also for  $\alpha k$, $\alpha k_\mathrm{ref}$, and $\alpha \mathbf{p}$, with $\alpha \in \mathbb{R}_{>0}$. Finally, notice that in contrast to the $n=2$ arcs problem analyzed in \cite{SalazarPaccagnanEtAl2021}, the best response strategy explicitly depends also on the quantitative discomfort levels. 
% !TeX spellcheck = en_US
\section{Mesoscopic Average Behavior}\label{sec:mesoscopic}

Now that we have analyzed the behavior of the individual user's response, we can step back and take a mesoscopic point of view, i.e., model the aggregate behavior resulting from the microscopic decisions. 

\subsection{Aggregate Decision}\label{sec:mesoscopic_aggregate}

At each time $t$, given Karma levels and reference probability distribution $\eta_t(k,k_\mathrm{ref})$, the probability of a traveling user with Karma level $k$ choosing arc $j \in \{1,\ldots,n\}$ is denoted by {$P(j | k,k_\mathrm{ref},\mathbf{p},\mathbf{x}^\mathrm{WE}(t))$} and is, under the conditions of Theorem~\ref{thm:brs}, given by
\begin{equation*}
	P(j|k,k_\mathrm{ref},\mathbf{p},\mathbf{x}^\mathrm{WE}(t)) = \!  \int\limits_{\gamma_{j}\left(k,k_\mathrm{ref},\mathbf{p},\mathbf{d}\left(\mathbf{x}^\mathrm{WE}(t)\right)\right)}^{\gamma_{j-1}\left(k,k_\mathrm{ref},\mathbf{p},\mathbf{d}\left(\mathbf{x}^\mathrm{WE}(t)\right)\right)} \rho(s)\,  \mathrm{d}s\,.
\end{equation*}
%\begin{equation*}
%	P_t(j|k,\mathbf{p},\mathbf{d}(\mathbf{x})) = \!  \int_{0}^{+\infty} \!\!\! \int_{\gamma_j(k,k_\mathrm{ref},\mathbf{p},\mathbf{d}(\mathbf{x}))}^{\gamma_{j-1}(k,k_\mathrm{ref},\mathbf{p},\mathbf{d(x)})} \!\!\!\!\!\!\!\!\!\!\!\!\\\rho(s)\eta_t(k,k_\mathrm{ref})\,  \mathrm{d}s\, \mathrm{d}k_{\mathrm{ref}} 
%\end{equation*}
Thus, the discrete-time evolution of the Karma level density function can be written as
\begin{equation*}
	\begin{split}
		&\eta_{t+1}(k,k_\mathrm{ref}) =  P_{\mathrm{home}}\eta_t(k,k_\mathrm{ref}) \\
		&+ P_{\mathrm{go}} \sum_{j = 1}^n P\left(j|k+\mathbf{p}_j,k_\mathrm{ref},\mathbf{p},\mathbf{x}^\mathrm{WE}(t)\right) \eta_t(k+\mathbf{p}_j,k_\mathrm{ref}),
	\end{split}
\end{equation*}
for $k, k_\mathrm{ref} \in \mathbb{R}_{\geq 0}$, and $t\in \mathbb{N}$. Moreover, the definition of the WE equilibrium in Definition~\ref{def:WE} can be rewritten as
\begin{equation}\label{eq:WE_cf} \small 
		\mathbf{x}^\mathrm{WE}_j(t) \!=\!
		P_\mathrm{go}\!\!\int\limits_{0}^\infty\! \!\int\limits_{0}^\infty \! \!P(j|k,k_\mathrm{ref},\mathbf{p},\mathbf{x}^\mathrm{WE}(t)) \eta_t(k,k_{\mathrm{ref}})\, \mathrm{d}k\,\mathrm{d}k_{\mathrm{ref}},\!\!
\end{equation}%
$j = 1,\ldots,n$.

It is important to point out two key aspects. First, note that only in the strict ordering conditions of Theorem~\ref{thm:brs}, it is possible to write a closed-form deterministic expression for $P_t(j|k,k_\mathrm{ref},\mathbf{p},\mathbf{d}(\mathbf{x}^\mathrm{WE}(t)))$. If they are not satisfied, it is only known that $P_t(j|k,k_\mathrm{ref},\mathbf{p},\mathbf{d}(\mathbf{x}^\mathrm{WE}(t)))$ is such that the aggregate decisions reconstruct the aggregate flows at the WE, which is portrayed in \eqref{eq:WE_cf}. Second, remark the discrete nature of the evolution of the Karma level density function, which is a linear combination of the previous density function shifted by $n$ fixed values that correspond to the arcs' prices. Thus, although continuous Karma levels were considered up to this point, a user $i$ with a given initial Karma level $k_0$, can only evolve to Karma levels that are of the form $k =k_0 + \sum_{j=1}^n m_j\mathbf{p}_j$ with $m_j \in \mathbb{N}_0$. This observation suggests that modeling the Karma level evolution of a single user as a Markov chain is appropriate.

{\color{black} Although there is a bounded attractive Karma level set, as mentioned earlier, the number of distinct Karma levels cannot be bounded even if $||\mathbf{p}||$ is bounded. Henceforth, to prevent that we consider that {$\mathbf{p}$ is a vector of integers, i.e.,} $\mathbf{p}\in \mathbb{Z}^n$. Nevertheless, it is important to recall that due to the positive scaling invariance of the prices and Karma levels on the user's decision, pointed out in Section~\ref{sec:bestresponse}, the precision of the prices can be chosen to be as high as desired by increasing $||\mathbf{p}||$, amounting to enforce $\mathbf{p}\in \mathbb{Q}^n$ in a computationally tractable manner.}

%However, the number of distinct Karma levels may grow infinitely large on the atractive Karma level set, depending on $\mathbf{p}$. To prevent that, henceforth, we consider that $\mathbf{p}\in \mathbb{N}^n$. 



\subsection{Stationary Markov Chain Model}

Consider a single user $i$ and assume that we are in the strict ordering conditions of Theorem~\ref{thm:brs}. Starting at a Karma level $k_0$, if $\mathbf{d}(\mathbf{x})$ is held constant, it is possible to propagate the possible Karma transitions and generate a finite Markov chain.  Let $\mathcal{K}(k_0,k^i_\mathrm{ref},\mathbf{p},\mathbf{x}) = \{k^i_1, \ldots, k^i_{|\mathcal{K}(k_0,k^i_\mathrm{ref},\mathbf{p},\mathbf{x})|}\}$ denote the state space of the chain and $\mathbf{A}(k_0,k^i_\mathrm{ref},\mathbf{p},\mathbf{x}) \in \mathbb{R}^{|\mathcal{K}(k_0,k^i_\mathrm{ref},\mathbf{p},\mathbf{x})| \times |\mathcal{K}(k_0,k^i_\mathrm{ref},\mathbf{p},\mathbf{x})| }$ the corresponding transition matrix in column-stochastic form, whereby the states are ordered by their corresponding Karma level. For the remainder of this subsection, the dependence of $\gamma_j$, $\mathcal{K}$, and $\mathbf{A}$ on $k_0$, $k^i_\mathrm{ref}$, $\mathbf{p}$, and $\mathbf{x}$ are dropped to alleviate the notation.
%Note that $k^i(0)$ need not be included in $\mathcal{K}(k^i(0),\mathbf{x})$. 

%For simplicity, assume, without loss of generality, that there is an unique communicating class in the Markov chain and consider the corresponding Markov subchain henceforth\footnote{If there is more than one communication class in the Markov chain, one can just consider multiple $k^i_\mathrm{ref}$, with the same value, but each with a distinct unique communication class,  each with a probability of occurrence }. Let $\mathcal{K}(k_0,k^i_\mathrm{ref},\mathbf{p},\mathbf{x}) = \{k^i_1, \ldots, k^i_{|\mathcal{K}(k_0,k^i_\mathrm{ref},\mathbf{p},\mathbf{x})|}\}$ denote the state space of the subchain and $\mathbf{A}(k_0,k^i_\mathrm{ref},\mathbf{p},\mathbf{x}) \in \mathbb{R}^{|\mathcal{K}(k_0,k^i_\mathrm{ref},\mathbf{p},\mathbf{x})| \times |\mathcal{K}(k_0,k^i_\mathrm{ref},\mathbf{p},\mathbf{x})| }$ the corresponding transition matrix in column-stochastic form, whereby the states are ordered by their corresponding Karma level. For the remainder of this subsection, the dependence of $\gamma_j$, $\mathcal{K}$, and $\mathbf{A}$ on $k_0$, $k^i_\mathrm{ref}$, $\mathbf{p}$, and $\mathbf{x}$ are dropped to alleviate the notation.
%Note that $k^i(0)$ need not be included in $\mathcal{K}(k^i(0),\mathbf{x})$. 

The entries of $\mathbf{A}$ can be expressed in closed-form by
\begin{equation*}
	\mathbf{A}_{uv}  = P_\mathrm{home}\mathbf{I} +  \begin{cases}
		0,  &\nexists j: k^i_v\!-\!k^i_u = \mathbf{p}_j\\
		P_\mathrm{go}\int\limits_{\gamma_{j}}^{\gamma_{j-1}} \rho(s)\,  \mathrm{d}s, & \exists j: k^i_v \!-\!k^i_u = \mathbf{p}_j\\
	\end{cases}.
\end{equation*}
Since $P_\mathrm{home}>0$,  the Markov chain is aperiodic. Note, however, that it is not necessarily irreducible, since there may exist more than one communication class. By the Perron-Frobenius Theorem \cite[Theorem~2.12]{Bullo2018}, it follows that the eigenvalue $\lambda=1$ is dominant but not necessarily simple. Denote the eigenvector associated with the eigenvalue $\lambda=1$ that corresponds to the stationary Karma distribution over  $\mathcal{K}$ of the Markov chain initialized in $k_0$ by $\boldsymbol{\pi}_{\infty}(k_0,k^i_\mathrm{ref},\mathbf{p},\mathbf{x})$. Notice that it corresponds to the limit of the power iteration of $\mathbf{A}$ initialized at the Karma level distribution with all probability concentrated in $k_0$. Finally, define the stationary arc selection matrix $\mathbf{P}(k_\mathrm{ref},\mathbf{p},\mathbf{x}) \in \mathbb{R}^{n \times |\mathcal{K}|}$ as the matrix whose entry $(u,v)$ is  given by $\mathbf{P}_{uv}(k_\mathrm{ref},\mathbf{p},\mathbf{x}) = P(u|k^i_v,k_\mathrm{ref},\mathbf{p},\mathbf{x})$. 

%Which is , is, thus, any column of the limit of the power iteration of $\mathbf{A}$  \cite[Theorem~2.13]{FB-LNS}. Notice that $\boldsymbol{\pi}_{\infty}(k_0,k^i_\mathrm{ref},\mathbf{p},\mathbf{x})$ is the stationary Karma distribution over  $\mathcal{K}$. 

%If the initial Karma level, $k_0$, is in $\mathcal{K}$, then it can reach every Karma level in that set. Furthermore, given that that Markov chain is a communicating class, it is irreducible. On top of that, if $P_\mathrm{home}>0$, then it is also aperiodic, and thus $\mathbf{A}$ is primitive. By the Perron-Frobenius Theorem \cite[Theorem~2.12]{FB-LNS}, it follows that the eigenvalue $\lambda=1$ is simple and dominant. The eigenvector associated with $\lambda=1$, denoted by $\boldsymbol{\pi}_{\infty}(k_0,k^i_\mathrm{ref},\mathbf{p},\mathbf{x})$, is, thus, any column of the limit of the power iteration of $\mathbf{A}$  \cite[Theorem~2.13]{FB-LNS}. Notice that $\boldsymbol{\pi}_{\infty}(k_0,k^i_\mathrm{ref},\mathbf{p},\mathbf{x})$ is the stationary Karma distribution over  $\mathcal{K}$. Finally define the stationary arc selection matrix $\mathbf{P}(k_\mathrm{ref},\mathbf{p},\mathbf{x}) \in \mathbb{R}^{n \times |\mathcal{K}|}$ as the matrix whose entry $(u,v)$ is  given by $\mathbf{P}_{uv}(k_\mathrm{ref},\mathbf{p},\mathbf{x}) = P(u|k^i_v,k_\mathrm{ref},\mathbf{p},\mathbf{x})$. 


\subsection{WE as an Aggregate Markov Chain}

In the previous subsection, we modeled the stationary behavior of a single user under the conditions of Theorem~\ref{thm:brs} as a Markov chain. Now, we analyze the aggregate of the Markov chains that model the stationary behavior of each user. More specifically, given that this model is distinct only for distinct $k_\mathrm{ref}$, the aggregate over the Karma reference distribution is taken. In that regard, on the Assumption~\ref{ass:WE_convergence}, in steady-state, \eqref{eq:WE_cf} can be rewritten as 
\begin{equation}\label{eq:WE_cf_mc}
	\begin{split}
		&\mathbf{x}^\mathrm{WE}_\infty \!=  \\
		&P_\mathrm{go} \!\int\limits_{0}^\infty \! \mathbf{P}(k_\mathrm{ref},\mathbf{p},\mathbf{x}^\mathrm{WE}_\infty)  \boldsymbol{\pi}_{\infty}(k_0,k_\mathrm{ref},\mathbf{p},\mathbf{x}^\mathrm{WE}_\infty)\theta_{\mathbf{p}}(k_{\mathrm{ref}})\, \mathrm{d}k_{\mathrm{ref}}.\!\!\!\!\!\!
	\end{split}
\end{equation}

\section{Pricing Design Problem}\label{sec:pricing_design}

The pricing design problem, formulated in Problem~\ref{prb:prices}, is now tackled on the following assumption:

\begin{assumption}\label{ass:ordering}
	Assume that, at the system optimum, there is an arc ordering such that  $\mathbf{d}_1(\mathbf{x}_1^\star) < \ldots < \mathbf{d}_n(\mathbf{x}_n^\star)$ is satisfied.
\end{assumption}

Under Assumptions~\ref{ass:WE}, \ref{ass:WE_convergence}, and \ref{ass:ordering}, the problem amounts to finding $\mathbf{p} = \mathbf{p}^\star$ such that \eqref{eq:WE_cf_mc} is satisfied for $\mathbf{x}^\mathrm{WE}_\infty = \mathbf{x}^\star$. Notice that without Assumption~\ref{ass:ordering}, neither $\mathbf{P}(k_\mathrm{ref},\mathbf{p},\mathbf{x}^\mathrm{WE}_\infty)$ nor  $\boldsymbol{\pi}_{\infty}(k_0,k_\mathrm{ref},\mathbf{p},\mathbf{x}^\mathrm{WE}_\infty)$ would be deterministic, which follows from the analysis in Section~\ref{sec:mesoscopic_aggregate}.
{It is important to} point out that first, the integer nature of $\mathbf{p}$, i.e. $\mathbf{p} \in \mathbb{Z}^n$, makes it challenging to solve \eqref{eq:WE_cf_mc}.
Second, the Karma reference distribution $\theta_\mathbf{p}(k_\mathrm{ref})$ depends on the pricing policy $\mathbf{p}$.
Third, not only do the entries of $\mathbf{P}(k_\mathrm{ref},\mathbf{p},\mathbf{x}^\mathrm{WE}_\infty)$ and  $\boldsymbol{\pi}_{\infty}(k_0,k_\mathrm{ref},\mathbf{p},\mathbf{x}^\mathrm{WE}_\infty)$ in  \eqref{eq:WE_cf_mc} depend nonlinearly on $\mathbf{p}$, but also the dimensions of the matrix and vector themselves change with $\mathbf{p}$.

To find the optimal prices, we enforce $\mathbf{1}^\top\mathbf{x}^{\star} = P_\mathrm{go}$, which via \eqref{eq:WE_cf_mc} only enforces {one constraint} on $\mathbf{p}$. The additional constraints stem from the fact that, at steady-state, the expected Karma level remains constant, hence $\mathbf{p}^{\star \top}\mathbf{x}^{\star} = 0$, and from the fact that the best response strategy is invariant on a positive scaling of prices and Karma distributions. Whilst these constraints were sufficient to design the optimal static prices for the 2-arc setting~\cite{SalazarPaccagnanEtAl2021}, for the general $n$-arc case under consideration we still need to find the optimal $\mathbf{p}^\star$ satisfying~\eqref{eq:WE_cf_mc} with $\mathbf{x}^\mathrm{WE}_\infty = \mathbf{x}^\star$, which, as mentioned above, is highly nonlinear and non-smooth. To the best of the authors' knowledge, these features make the derivation of a closed-form solution not feasible.


\subsection{Numerical Design Method}

{We leverage the structure of the problem to overcome the aforementioned difficulties and reframe the pricing design problem thoughtfully so that it can be solved efficiently. In this regard, we introduce three considerations to enable the numerical solution of~\eqref{eq:WE_cf_mc} for $\mathbf{p}$ with $\mathbf{x}^\mathrm{WE}_\infty = \mathbf{x}^\star$.} First, $\theta_\mathbf{p}(k_\mathrm{ref})$ has to be bounded and discrete to be numerically tractable. Note that this is a reasonable assumption since there is an attractive invariant Karma set and the Karma levels are discrete because $\mathbf{p}\in \mathbb{Z}^n$. Second, since $\mathbf{p} \in \mathbb{Z}^n$, the equality in \eqref{eq:WE_cf_mc} will not be achieved exactly. Instead, one may attempt to minimize {the deviation of the cost of the right-hand term w.r.t. the optimal aggregate flows.} Nevertheless, the larger $||\mathbf{p}||$ is allowed to be, the closer is the equality. Third, the constraint $(\mathbf{p}^\star)^\top\mathbf{x}^{\star} = 0$ may not be satisfied exactly if the entries of $\mathbf{x}^\star$ are irrational or if $||\mathbf{p}||$ is bounded. Thus, one can substitute it with a quantized  approximation $\mathbf{x}^\star_\mathrm{quant}$.

%\textcolor{red}{Maybe rephrase: remove greadient freee here / frame the probelmm ia way that it can  - bui d a story we çevaerage the structure of the probelm - an the analytical analysis - the probelm is nonsmooth and nolinear -> one could leverage learning algorithms -> but scability - we fraem inb sucgh a way that instead of just say that we find the op+timum follows we find thw precise that results }Given the aforementioned difficulties, we employ a gradient-free numerical optimization method to thoughtfully solve~\eqref{eq:WE_cf_mc} for $\mathbf{p}$ with $\mathbf{x}^\mathrm{WE}_\infty = \mathbf{x}^\star$. 

Therefore, the proposed pricing design optimization problem {becomes:}

\begin{problem}[Numerical Pricing Design]\label{prb:prices_num}
	Given a desired system optimum $\mathbf{x}^\star$, select $\mathbf{p}$ as the solution to
\begin{equation}\label{eq:num_prob}
	\begin{split}
		%		&\mathop{\mathrm{minimize}}\limits_{\mathbf{p}\in \mathbb{N}^n}\\
		&	\!\!\!\!\!\!\!{\small \mathop{\mathrm{min}}\limits_{\mathbf{p}\in \mathbb{Z}^n} {C}\!\left(\!\!P_\mathrm{go}\!\!\!\!\!\sum\limits_{k_\mathrm{ref} = k_{\mathrm{ref}_{\mathrm{min}}}}^{k_{\mathrm{ref}_{\mathrm{max}}}}\!\!\!\!\!\!\! \mathbf{P}(k_\mathrm{ref},\mathbf{p},\mathbf{x}^\star) \boldsymbol{\pi}_{\infty}(k_0,k_\mathrm{ref},\mathbf{p},\mathbf{x}^\star)\theta_{\mathbf{p}}(k_{\mathrm{ref}})\! \!\right)}\\
		& \!\!\!\!\mathop{\mathrm{s.t.}}  \quad \mathbf{p}^\top\mathbf{x}^{\star}_\mathrm{quant} = 0\\
		& \!\!\!\! \phantom{\mathop{\mathrm{s.t.}}}\;\quad \mathbf{p}_j > \mathbf{p}_{j+1}, \, j = 1,\ldots, n-1\\
		& \!\!\!\!\phantom{\mathop{\mathrm{s.t.}}}\;\quad \mathbf{p}_1 > 0\\
		& \!\!\!\!\phantom{\mathop{\mathrm{s.t.}}}\;\quad \mathbf{p}_n < 0,
	\end{split}\!\!\!\!\!\!\!\!\!
\end{equation}
where $k_{\mathrm{ref}_{\mathrm{min}}}$ and $k_{\mathrm{ref}_{\mathrm{max}}}$ are the minimum and maximum values of the support of $\theta_\mathbf{p}$, respectively.
\end{problem}

Such a problem can be efficiently solved with gradient-free methods, as shown in Section~\ref{sec:num_res} below. Furthermore, a useful particularity of Problem~\ref{prb:prices_num} is that the minimum of the objective function is known and given by $C(\mathbf{x}^\star)$. Thus, it is easy to evaluate the suboptimality bound and stop the numerical method whenever it reaches a given threshold.
	
%	 the deviation of the cost of an iteration w.r.t. the minimum falling below a threshold can be employed as an effective stopping criterion.}
% 
% too hard setendce suboptimality bound

%\textcolor{blue}{A precise optimum criteria can be use to stop the iterations - also try x shot}

% !TeX spellcheck = en_US
\section{Numerical Results}\label{sec:num_res}

In this section, numerical results are presented for an illustrative case study with $n = 5$. We consider $M = 1000$ users of which, on average, $P_\mathrm{home} = 5\%$ do not travel every day. Their daily sensitivity is sampled from a uniform distribution on the interval $[0,2]$ and their prediction horizon is $T=4$. We model the discomfort as a travel-time Bureau of Public Roads (BPR) function \cite{BPR1964}
\begin{equation*}
	\mathbf{d}_j(\mathbf{x}_j) = \mathbf{d^0}_{\!\!j}\left(1+ \alpha (\mathbf{x}_j/\boldsymbol{\kappa}_j)^\beta\right),
\end{equation*}
with $\alpha = 0.15$, $\beta = 4$, and $\mathbf{d^0}$ and $\boldsymbol{\kappa}$ were generated randomly which, rounded to four decimal places, are given by $\mathbf{d^0} = [0.5001 \; 0.5734 \; 0.7085 \; 0.6512 \; 0.8602]^\top$ and $\boldsymbol{\kappa} = [0.0923 \; 0.1863 \; 0.3968 \; 0.3456 \;0.5388]^\top$, ordered according to the arc ordering in Assumption~\ref{ass:ordering}. We consider distribution of the reference values $\theta_\mathbf{p}$ to be a discrete uniform distribution with support $\{k_\mathrm{ref}\in \mathbb{N} \,|\, k_\mathrm{ref} = 0 \lor k_\mathrm{ref} = \mathbf{p}_j \}$, which corresponds to users having the possibility of saving Karma to afford traveling through an arc with a positive price at the end of the horizon.  The system's cost is considered to be a weighted sum of the travel-time in each link, i.e., $\mathbf{c}_j(\mathbf{x}):=  \mathbf{c^0}_{\!\!j}\mathbf{d}_j(\mathbf{x}_j)$, whose weights were randomly generated and, rounded to four decimal places, are given by $\mathbf{c^0}_{\!\!\!j} = [0.7096 \; 0.8426 \; 0.9391 \; 0.6022 \; 0.5137]$. This can correspond to the weighted minimization of, for example, sound pollution.



Rounded to four decimal places, {employing \cite{Loefberg2004},} $\mathbf{x}^\star = [0.0877 \; 0.1309 \; 0.0000 \; 0.3053 \; 0.4261]^\top$  and ${\mathbf{d}(\mathbf{x}^\star) = [0.5611 \; 0.5943 \; 0.7085 \; 0.7107 \; 0.9106]^\top}$, which is in accordance with Assumption~\ref{ass:ordering}. The optimization problem \eqref{eq:num_prob} is solved using a standard genetic algorithm method subject to $||\mathbf{p}||_\infty \leq 100$ {in less than 500 wall-clock seconds in a standard laptop}, whose solution is ${\mathbf{p}^\star = [79\;63\;39\;13\;-45 ]^\top}$. We considered $k_0 = \mathbf{p}_1$ and $\mathbf{x}^{\star}_\mathrm{quant}$ {resulting from rounding $\mathbf{x}^\star$ to three decimal places}.



\begin{figure}[ht!]
	\begin{subfigure}{\linewidth}
		\centering
		\includegraphics[width = \linewidth]{fig/decision-eps-converted-to.pdf}
		\caption{Evolution of aggregate flows.}
		\label{fig:decision}
	\end{subfigure}\\%
	\begin{subfigure}{\linewidth}
		\raggedleft
		\includegraphics[width = 0.99\linewidth]{fig/karma-eps-converted-to.pdf}
		\caption{Evolution of Karma level.}
		\label{fig:karma}
	\end{subfigure}\\%
	\begin{subfigure}{\linewidth}
		\raggedleft
		\includegraphics[width = 0.99\linewidth]{fig/cost-eps-converted-to.pdf}
		\caption{Evolution of the relative cost difference.}
		\label{fig:cost}
	\end{subfigure}\\%
	\begin{subfigure}{\linewidth}
		\raggedleft
		\includegraphics[width = \linewidth]{fig/sensitivity-eps-converted-to.pdf}
		\caption{Evolution of the relative sensitivity and discomfort deviation.}
		\label{fig:sensitivity}
	\end{subfigure}%
	\caption{Numerical simulation results.}
	\label{fig:num_sim}
\end{figure}

The daily simulations are carried out by computing the Nash equilibrium that follows from the decisions of each user to Problem~\ref{sec:bestresponse}, which approximate the WE as $M\to\infty$. The Karma values were initialized randomly according to  a discrete uniform distribution with support $\{25\mathbf{p}_1,25\mathbf{p}_1,+1,\ldots,50\mathbf{p}_1\}$.  Figs.~\ref{fig:decision}--\ref{fig:cost} depict the evolution of the aggregate flows, Karma level, and relative cost difference in relation to the system optimum, respectively, throughout the simulation. We denote the average and the standard deviation of the users' Karma level at time $t$ by $\hat{k}(t)$ and $\sigma_k(t)$, respectively. First, since the initial Karma levels are very high, the users act as if the pricing scheme were not implemented. This can be seen in the initial plateau in Fig.~\ref{fig:cost} which is associated with the constant aggregate flows visible in Fig.~\ref{fig:decision}. Nevertheless, as the users' Karma is depleted, as shown in Fig.~\ref{fig:karma}, the users can no longer afford every link and the pricing mechanism drives the aggregate flows to the system-optimal flows. Second, it is important to point out that, despite all the assumptions made to tackle the intractability of the pricing design problem and enable a numerical solution, the prices that were designed get very close to the system optimum, as visible in Fig.~\ref{fig:cost}, with an average relative difference in relation to the theoretical optimum of  $0.15\,\%$ only over the last $50$ instants of the simulation. In fact, the steady-state aggregate flows of the numerical simulation closely match $\mathbf{x}^\star$, as visible in Fig.~\ref{fig:decision}. Third, we analyze  i)~the relative difference of the average perceived discomfort w.r.t. a scenario in which the users are centrally allocated to the optimal flows randomly, i.e. without taking into account their sensitivity, which is given by
%the average relative discomfort perceived at each time that would be perceived if the users were allocated to the same flows randomly without taking into account the sensitivity of each user as
\begin{equation*}
	\Delta \bar{d}(t) := \frac{\sum_{i = 1}^M s^i(t) \mathbf{d}(\mathbf{x})^\top\mathbf{y}^i(t) + \bar{s}\, \mathbf{d}(\mathbf{x})^\top\mathbf{y}^i(t) }{\sum_{i = 1}^M \bar{s}\, \mathbf{d}(\mathbf{x})^\top\mathbf{y}^i(t)};
\end{equation*}
and ii)~the relative deviation of the average sensitivity in relation to the expected sensitivity, i.e., ${\Delta \bar{s}(t) := (1/M)\sum_{i = 1}^M (s^i(t)-\bar{s})/\bar{s}}$. Fig.~\ref{fig:sensitivity} depicts the evolution of these two quantities. It is noticeable that, at steady-state, the perceived discomfort is roughly $8\,\%$ lower in comparison to an optimal but urgency-unaware policy.

%\begin{figure}[ht]
%	\centering
%	\includegraphics[width = 1\linewidth]{fig/decision.eps}
%	\caption{Evolution of aggregate flows.}
%	\label{fig:decision}
%\end{figure}
%
%\begin{figure}[ht]
%	\centering
%	\includegraphics[width = 1\linewidth]{fig/karma.eps}
%	\caption{Evolution of Karma level.}
%	\label{fig:karma}
%\end{figure}
%
%\begin{figure}[ht]
%	\centering
%	\includegraphics[width = 1\linewidth]{fig/cost.eps}
%	\caption{Evolution of the relative cost difference in relation to the theoretical system's optimum.}
%	\label{fig:cost}
%\end{figure}
%
%\begin{figure}[ht]
%	\centering
%	\includegraphics[width = 1\linewidth]{fig/sensitivity.eps}
%	\caption{Evolution of the relative average sensitivity and discomfort deviation.}
%	\label{fig:sensitivity}
%\end{figure}



Due to space limitations, some details regarding the numerical pricing design and the simulation were omitted.  Nevertheless, a MATLAB implementation as well as additional simulation results,  is openly available in an open source repository at  {\small\texttt{\url{https://fish-tue.github.io/single-origin-destination-routing}}}.
%. For more information refer to the Code Availability Statement section.
\section{Conclusion}
We propose \modelname for real-time instance segmentation. Built on a query-based segmentation framework~\cite{cheng2021mask2former} and three designed efficient components, \ie, instance activation-guided queries, dual-path update strategy, and ground truth mask-guided learning, \modelname achieves excellent performance on the popular COCO dataset while maintaining a fast inference speed. Extensive experiments demonstrate the effectiveness of core ideas and the superiority of \modelname over previous state-of-the-art real-time counterparts. We hope this work can serve as a new baseline for real-time instance segmentation and promote the development of query-based image segmentation algorithms. 

\noindent\textbf{Limitations.} (1) Like general query-based models~\cite{detr, cheng2021mask2former,li2021panopticsegformer}, \modelname is not good at small targets. Even though using stronger pixel decoders or larger feature maps improves it, it introduces heavier computational burdens, and the result is still unsatisfactory. We look forward to an essential solution to handle this problem. (2) although GT mask-guided learning improves the performance of masked attention, it increases training costs. We hope a more elegant method can be proposed to replace it.

% !TeX spellcheck = en_US
%\section*{Code Availability Statement}
%A MATLAB implementation of the methods and simulations presented in this paper are openly available in an open-source repository available at {\small\texttt{\url{https://fish-tue.github.io/single-origin-destination-routing}}}.


\section*{Acknowledgment}
We thank Dr.\ I.\ New and F.\ Paparella for proofreading the~paper.


\appendices
\section{Appendix for Proofs}

\paragraph{Proof of Theorem \ref{thm:main}.}

\begin{proof}
\label{proof:main}
Our proof has two steps. In Step 1, we will show that SimCLR is equivalent to minimizing the cross entropy loss defined in Eqn.~(\ref{eqn:cross-entropy}). 
In Step 2, we will show  that minimizing the cross-entropy loss 
is equivalent to spectral clustering on $\bfpi$. 
Combining the two steps together, we have proved our theorem. 

\textbf{Step 1: } SimCLR is equivalent to minimizing the cross entropy loss.

The cross-entropy loss takes expectation over 
$\bfW_\bfX\sim \mathbb{P}(\cdot ; \bfpi)$, 
which means $\bfW_\bfX$ has exactly one non-zero entry in each row $i$. By Lemma~\ref{lem:multinomial}, we know every row $i$ of $\bfW_\bfX$ is independent of other rows. Moreover, 
$\bfW_{\bfX,i}\sim \mathcal{M}(1, \bfpi_i/\sum_j \bfpi_{i,j})=\mathcal{M}(1, \bfpi_i)$, because $\bfpi_i$ itself is a probability distribution.
Similarly, we know $\bfW_\bfZ$ also has the row-independent property by sampling over $\mathbb{P}(\cdot;\bfK_\bfZ)$.
Therefore, by Lemma~\ref{lem:cross_split}, we know Eqn.~(\ref{eqn:cross-entropy}) is equivalent to:
\[
 -\sum_{i=1}^n \mathbb{E}_{\bfW_{\bfX,i}}[\log \mathbb{P}(\bfW_{\bfZ,i}=\bfW_{\bfX,i};\bfK_\bfZ)],
\]

This expression takes expectation over $\bfW_{\bfX,i}$ for the given row $i$. Notice that 
$\bfW_{\bfX,i}$ has exactly one non-zero entry, which equals $1$ (same for $\bfW_{\bfZ,i}$). 
As a result
we expand the above expression to be:
\begin{equation}
 -\sum_{i=1}^n \sum_{j\neq i} \Pr(\bfW_{\bfX,i,j}=1)\log \Pr(\bfW_{\bfZ,i,j}=1).
\label{eqn:detailed-expansion}    
\end{equation}


By Lemma~\ref{lem:multinomial}, $\Pr(\bfW_{\bfZ,i,j}=1)=\bfK_{\bfZ,i,j}/\|\bfK_{\bfZ,i}\|_1$ for $j\neq i$. Recall that $\bfK_\bfZ=(k(\bfZ_i-\bfZ_j))_{(i,j)\in[n]^2}$, which means 
$\bfK_{\bfZ,i,j}/\|\bfK_{\bfZ,i}\|_1=\frac{\exp(-\|\bfZ_i-\bfZ_j\|^2/{2\tau})}{\sum_{k\neq i}
\exp(-\|\bfZ_i-\bfZ_k\|^2/{2\tau})
}$ for $j\neq i$, when $k$ is the Gaussian kernel with variance $\tau$. 

Notice that $\bfZ_i=f(\bfX_i)$, so we know
\begin{equation}
-\log \Pr(\bfW_{\bfZ,i,j}=1)=
-\log \frac{\exp(-\|f(\bfX_i)-f(\bfX_j)\|^2/{2\tau})}{\sum_{k\neq i}
\exp(-\|f(\bfX_i)-f(\bfX_k)\|^2/{2\tau}),
}
\label{eqn:infonce-equivalence}    
\end{equation}


The right hand side is exactly the InfoNCE loss defined in Eqn.~(\ref{eqn:infonce}).
Inserting Eqn.~(\ref{eqn:infonce-equivalence}) into Eqn.~(\ref{eqn:detailed-expansion}), we get the SimCLR algorithm, which first samples augmentation pairs $(i,j)$ with $\Pr(\bfW_{\bfX,i,j}=1)$ for each row $i$, and then optimize the InfoNCE loss. 

\textbf{Step 2: } minimizing the cross entropy loss 
is equivalent to spectral clustering on $\bfpi$.


By Lemma~\ref{lem:convert_to_spectral}, we may further convert the loss to 
\begin{equation}
\label{eqn:main-theorem-repul-attr}
\min_{\bfZ}
-\sum_{(i,j)\in [n]^2} \mathbf{P}_{i,j}
\log k (\bfZ_i-\bfZ_j)+\log \mathbf{R}(\bfZ).
\end{equation}
Since $k$ is the Gaussian kernel, this reduces to \[
\min_\bfZ \mathrm{tr}(\bfZ^\top \mathbf{L}(\bfpi) \bfZ)
+\log \mathbf{R}(\bfZ),
\]

where we use the fact that $\mathbb{E}_{\bfW_\bfX\sim \mathbb{P}(\cdot; \bfpi)}[\mathbf{L}(\bfW_\bfX)]
=\mathbf{L}(\bfpi)
$, because the Laplacian operator is linear and $
\mathbb{E}_{\bfW_\bfX\sim \mathbb{P}(\cdot; \bfpi)}(\bfW_\bfX)=\bfpi
$.
\end{proof}

\paragraph{Proof of Theorem \ref{thm:clip}.}
\begin{proof}
Since $\bfW_\bfX\sim \mathbb{P}(\cdot;\bfpi_{\mathbf{A}, \mathbf{B}})$, we know 
$\bfW_\bfX$ has exactly one non-zero entry in each row, denoting the pair that got sampled. 
A notable difference compared to the previous proof is we now have $n_\mathcal{A}+n_\mathcal{B}$ objects in our graph. CLIP deals with this by taking a mini-batch of size $2N$, 
such that $n_\mathcal{A}=n_\mathcal{B}=N$, and adding the $2N$ InfoNCE losses together. We label the objects in $\mathcal{A}$ as $[n_\mathcal{A}]$, and the objects in $\mathcal{B}$ as $\{n_\mathcal{A}+1, \cdots, n_\mathcal{A}+n_\mathcal{B}\}$. 

Notice that $\bfpi_{\mathbf{A}, \mathbf{B}}$ is a bipartite graph, so the edges of objects in $\mathcal{A}$ will only connect to object in $\mathcal{B}$ and vice versa. We can define the similarity matrix in $\cZ$ as $\bfK_\bfZ$, 
where $\bfK_\bfZ(i, j+n_\mathcal{A})=\bfK_\bfZ(j+n_\mathcal{A},i)= k(\bfZ_i-\bfZ_j)$ for $i\in [n_\mathcal{A}], j\in [n_\mathcal{B}]$, and otherwise we set $\bfK_\bfZ(i,j)=0$. 
The rest is same as the previous proof. 
\end{proof}

\paragraph{Proof of Theorem \ref{thm:exponential}.}

\begin{proof}
\label{proof:exponential}
Since the objective function consists of a linear term combined with an entropy regularization, which is a strongly concave function, the maximization problem is a convex optimization problem. Owing to the implicit constraints provided by the entropy function, the problem is equivalent to having only the equality constraint. We then introduce the Lagrangian multiplier $\lambda$ and obtain the following relaxed problem:

$$
\widetilde{E}(\boldsymbol{\alpha})=\psi_{1}-\sum_{i=1}^n \alpha_{i} \psi_{i}+\tau \sum_{i=1}^n \alpha_{i}\log \alpha_{i}+\lambda\left(\boldsymbol{\alpha}^{\top} \mathbf{1}_n-1\right).
$$

As the relaxed problem is unconstrained, taking the derivative with respect to $\alpha_{i}$ yields

$$
\frac{\partial \widetilde{E}(\boldsymbol{\alpha})}{\partial \alpha_{i}}=-\psi_{i}+\tau\left(\log \alpha_{i}+\alpha_{i} \frac{1}{\alpha_{i}}\right)+\lambda=0.
$$

Solving the above equation implies that $\alpha_{i}$ takes the form
$
\alpha_{i}=\exp \left(\frac{1}{\tau} \psi_{i}\right) \exp \left(\frac{-\lambda}{\tau}-1\right).
$ Since $\alpha_{i}$ lies on the probability simplex, the optimal $\alpha_{i}$ is explicitly given by
$
\alpha^{*}_{i}=\frac{\exp \left(\frac{1}{\tau} \psi_{i}\right)}{\sum_{i^{\prime}=1}^n \exp \left(\frac{1}{\tau} \psi_{i^{\prime}}\right)} .
$ Substituting the optimal point into the objective function, we obtain
$$
\begin{aligned}
E\left(\boldsymbol{\alpha}^*\right)  &=\psi_1-\sum_{i=1}^n \frac{\exp \left(\frac{1}{\tau} \psi_{i}\right)}{\sum_{i^{\prime}=1}^n \exp \left(\frac{1}{\tau} \psi_{i^{\prime}}\right)} \psi_{i}+\tau \sum_{i=1}^n \frac{\exp \left(\frac{1}{\tau} \psi_{i}\right)}{\sum_{i^{\prime}=1}^n \exp \left(\frac{1}{\tau} \psi_{i^{\prime}}\right)}\log \frac{\exp \left(\frac{1}{\tau} \psi_{i}\right)}{\sum_{i^{\prime}=1}^n \exp \left(\frac{1}{\tau} \psi_{i^{\prime}}\right)} \\
& =\psi_1 - \tau \log \left(\sum_{i=1}^n \exp \left(\frac{1}{\tau} \psi_{i}\right)\right).
\end{aligned}
$$
Thus, the Lagrangian dual function is given by
\begin{equation*}
-E\left(\boldsymbol{\alpha}^*\right)= -\tau \log \frac{\exp \left(\frac{1}{\tau} \psi_{1}\right)}{\sum_{i=1}^n \exp \left(\frac{1}{\tau} \psi_{i}\right)}.\qedhere
\end{equation*}
\end{proof}



\section{More on Experiments} \label{section: experiment_details}

\paragraph{CIFAR-10 and CIFAR-100} CIFAR-10 ~\citep{krizhevsky2009learning} and CIFAR-100 ~\citep{krizhevsky2009learning} are well-known classic image classification datasets. Both CIFAR-10 and CIFAR-100 contain a total of 60k $32 \times 32$ labeled images of different classes, with 50k for training and 10k for testing. CIFAR-10 is similar to CIFAR-100, except there are 10 different classes in CIFAR-10 and 100 classes in CIFAR-100.

\paragraph{TinyImageNet} TinyImageNet ~\citep{le2015tiny} is a subset of ImageNet ~\citep{deng2009imagenet}. There are 200 different object classes in TinyImageNet, with 500 training images, 50 validation images, and 50 test images for each class. All the images in TinyImageNet are colored and labeled with a size of $64 \times 64$.

\textbf{Pseudo-code.} Algorithm \ref{alg:Training Procedure} presents the pseudo-code for our empirical training procedure.

\begin{algorithm}[!htbp]
\caption{Training Procedure}
\label{alg:Training Procedure}
\begin{algorithmic}[1]
\REQUIRE trainable encoder network $f$, batch size $N$, augmentation strategy \textit{aug}, loss function $L$ with hyperparameters \textit{args}
\FOR {sampled minibatch ${x_i}_{i=1}^N$}
\FORALL{$i \in { 1, ..., N }$}
\STATE draw two augmentations $t_i = \textit{aug}\left(x_i\right) $, $t_i' = \textit{aug}\left(x_i\right) $
\STATE $z_i = f\left(t_i\right)$, $z_i' = f\left(t_i'\right)$
\ENDFOR
\STATE compute loss $\mathcal{L} = L(N, z, z', \textit{args})$
\STATE update encoder network $f$ to minimize $\mathcal{L}$
\ENDFOR
\STATE \textbf{Return} encoder network $f$
\end{algorithmic}
\end{algorithm}

We also provide the pseudo-code for our core loss function used in the training procedure in Algorithm \ref{alg:Core loss}. The pseudo-code is almost identical to SimCLR's loss function, with the exception of an extra parameter $\gamma$.

\begin{algorithm}[!htbp]
\caption{Core loss function $\mathcal{C}$}
\label{alg:Core loss}
\begin{algorithmic}[1]
\REQUIRE batch size $N$, two encoded minibatches $z_1, z_2$, $\gamma$, temperature $\tau$
\STATE $z = \textit{concat}\left(z_1, z_2\right)$
\FOR {$i \in {1, ..., 2N }, j \in {1, ..., 2N}$ }
\STATE $s_{i,j} = \Vert z_i - z_j \Vert_2^{\gamma}$
\ENDFOR
\STATE \textbf{define} $l(i, j)$ \textbf{as} $l(i, j) = - \log \frac{exp\left(s_{i,j}/\tau \right)}{\sum_{k=1}^{2N} \mathbf{1}{[k \ne i]} exp\left(s{i, j} / \tau \right)} $
\STATE \textbf{Return} $\frac{1}{2N} \sum_{k=1}^N\left[l(i, i+N) + l(i+N, i)\right]$
\end{algorithmic}
\end{algorithm}

Utilizing the core loss function $\mathcal{C}$, we can define all kernel loss functions used in our experiments in Table \ref{table: loss definition}. For all $z_i \in z$ with even dimensions $n$, we define $z_{L_i} = z_i\left[0:n/2\right]$ and $z_{R_i} = z_i\left[n/2:n\right]$.

\begin{table}[ht]
\centering
\begin{tabular}{{@{}l|l@{}}}
Kernel  &  Loss function \\ \midrule
Laplacian & $\mathcal{C}\left(N, z, z', \gamma=1, \tau\right)$\\ \midrule
Sum       & $\lambda * \mathcal{C}\left(N, z, z', \gamma=1, \tau_1\right) + (1-\lambda) * \mathcal{C}\left(N, z, z', \gamma=2, \tau_2\right)$  \\ \midrule
Concatenation Sum&$\lambda * \mathcal{C}\left(N, z_L, z'_L, \gamma=1, \tau_1\right) + (1-\lambda) * \mathcal{C}\left(N, z_R, z'_R, \gamma=2, \tau_2\right)$\\ \midrule
$\gamma = 0.5$ & $\mathcal{C}\left(N, z, z', \gamma=0.5, \tau\right)$          \\ 

\end{tabular}

\caption{Definition of kernel loss functions in our experiments}
\label {table: loss definition}
\end{table}

\textbf{Baselines.} We reproduce the SimCLR algorithm using PyTorch Lightning~\citep{PytorchLightning}.

\textbf{Encoder details.}
The encoder $f$ consists of a backbone network and a projection network. We employ ResNet50~\citep{ResNet} as the backbone and a 2-layer MLP (connected by a batch normalization~\citep{ioffe2015batch} layer and a ReLU \cite{nair2010rectified} layer) with hidden dimensions 2048 and output dimensions 128 (or 256 in the concatenation kernel case).

\textbf{Encoder hyperparameter tuning.}
For each encoder training case, we randomly sample 500 hyperparameter groups (sample details are shown in Table \ref{table: Hyperparameter sample}) and train these samples simultaneously using Ray Tune ~\citep{RayTune}, with the ASHA scheduler~\citep{li2018massively}. Ultimately, the hyperparameter group that maximizes the online validation accuracy (integrated in PyTorch Lightning) within 5000 validation steps is chosen for the given encoder training case.

\begin{table}[ht]
\centering

\begin{tabular}{@{}l|l|l@{}}
\midrule
Hyperparameter  & Sample Range & Sample Strategy \\ \midrule
start learning rate & $\left[10^{-2}, 10\right]$ & log uniform \\ \midrule
$\lambda$       & $\left[0, 1\right]$ & uniform \\ \midrule
$\tau$, $\tau_1$, $\tau_2$ & $\left[0, 1\right]$ & log uniform \\ \midrule
\end{tabular}

\caption{Hyperparameters sample strategy}
\label {table: Hyperparameter sample}
\end{table}

\textbf{Encoder training.} 
We train each encoder using the LARS optimizer~\citep{LARSOptimizer}, LambdaLR Scheduler in PyTorch, momentum 0.9, weight decay $10^{-6}$, batch size 256, and the aforementioned hyperparameters for 400 epochs on a single A-100 GPU.

\textbf{Image transformation.} The image transformation strategy, including augmentation, is identical to the default transformation strategy provided by PyTorch Lightning.

\textbf{Linear evaluation.}
The linear head is trained using the SGD optimizer with a cosine learning rate scheduler, batch size 64, and weight decay $10^{-6}$ for 100 epochs. The learning rate starts at $0.3$ and ends at $0$.

\textbf{Moco Experiments.} We also tested our method based on MoCo~\citep{he2019moco}. The results are summarized in Table \ref{tab:results-moco}. Here we choose ResNet18~\citep{ResNet} as the backbone and set a temperature of $0.1$ as default. For our simple sum kernel, we set $\lambda=0.8$. The results show that our method outperforms the original MoCo method.

\begin{table}[thb]
\centering
\caption{MoCo Experiment Results on CIFAR-10 and CIFAR-100.}
\label{tab:results-moco}
\resizebox{\textwidth}{!}{%
\begin{tabular}{@{}c|ccc|ccc@{}}
\toprule
\multirow{3}{*}{Method} & \multicolumn{3}{c|}{CIFAR-10} & \multicolumn{3}{c}{CIFAR-100} \\ \cmidrule(lr){2-4} \cmidrule(lr){5-7} 
                        & 200 epochs & 400 epochs    & 1000 epochs   & 200 epochs & 400 epochs & 1000 epochs         \\ \midrule
MoCo (repro.)         & $76.41 \pm 0.12$    & $80.01 \pm 0.15$          & $84.45 \pm 0.08$    & $\mathbf{47.02 \pm 0.11}$ & $52.50 \pm 0.07$ & $57.62 \pm 0.15$            \\
\midrule
Laplacian Kernel        & ${78.09 \pm 0.10}$    & $\mathbf{83.85 \pm 0.09}$          & $\mathbf{88.34 \pm 0.16}$    & $46.12 \pm 0.22$   & $53.44 \pm 0.17$ & $59.10 \pm 0.14$        \\
Simple Sum Kernel & $\mathbf{78.12 \pm 0.15}$   & $83.23 \pm 0.18$ & $87.50 \pm 0.20$ & $46.65 \pm 0.06$ & $\mathbf{53.62 \pm 0.19}$ & $\mathbf{59.83 \pm 0.12}$\\
\bottomrule
\end{tabular}
}
\end{table}



\section{More Experiments on Synthetic Data}


Consider a scenario with $n$ clusters, each containing $k$ vertices. Let the probability of vertices $u$ and $v$ from the same cluster belonging to $\bfpi$ be $p$. Conversely, for vertices $u$ and $v$ from different clusters, let the probability of belonging to $\pi$ be $q$. We generate the graph $\bfpi$ randomly, based on $p$ and $q$. We experiment with values of $k=100$ and $n=6$ for ease of visualization, embedding all points in a two-dimensional space. Each vertex's initial position originates from a normal distribution. In each iteration, we sample a subgraph of $\bfpi$ uniformly, ensuring each vertex has an out-degree of $1$. We then optimize the corresponding vectors using InfoNCE loss with an SGD optimizer and iterate until convergence. Our experimental setup consists of an SGD learning rate of $1$, an InfoNCE loss temperature of $0.5$, and a batch size of $50$. We evaluate two scenarios with different $p$ and $q$ values: $p=1$, $q=0$, and $p=0.75$, $q=0.2$. The results of these experiments are visualized in Figure \ref{fig:vis-spectral-cluster}. The obtained embeddings exhibit the hallmark pattern of spectral clustering of graph $\bfpi$.

\begin{figure}[!tb]
\centering
\subfigure{
\includegraphics[width=1\textwidth]{Figures/cluster_pi.png}
\label{fig:vis-cluster}
}
\subfigure{
\includegraphics[width=1\textwidth]{Figures/noised_cluster_pi.png}
\label{fig:vis-noised-cluster}
}
\caption{Visualizations of the optimization process using InfoNCE Loss on the vectors corresponding to $\bfpi$. Points of identical color belong to the same cluster within $\bfpi$. To showcase the internal structure of $\bfpi$, we randomly select 10 vertices from each cluster to display the edge distribution of $\bfpi$.}
\label{fig:vis-spectral-cluster}
\end{figure}





\bibliographystyle{IEEEtran}
\bibliography{bib/main.bib,bib/SML_papers.bib}
%\bibliography{references}


\end{document}
