% !TeX spellcheck = en_US
\section{Conclusion}\label{sec:concl}
In this paper, we explored a fair incentive mechanism based on artificial currencies to tackle routing problems whilst accounting for the daily urgency of the users. We modeled the system as a repeated game and we obtained a closed-form solution for the user's daily strategy, which enables a numerical solution of the {arc-pricing} design problem. We showed that by employing a simple static payment-transaction scheme, our approach steers the aggreggate flows towards the societally optimal flows, achieving the minimum societal cost. On top of that, the simulation results indicated that the proposed scheme allows for a significant reduction of the users' perceived discomfort in relation to an optimal but urgency-unaware policy.
	
%	
%	\textcolor{red}{}
%	First, despite the considerations made to enable a numerical solution of the pricing design problem, we showed that the solution steers the aggregate flows towards the societally-optimal flows achieving the minimum societal cost. On top of that, the simulation results indicated that the proposed scheme allows for a significant reduction of the perceived discomfort in relation to an optimal but urgency-unaware policy.}
%
%
%\textcolor{blue}{stabilit, convergence , prmissing judiciosly ddesigning the prices is sufficient to reach a  - you dont need the bidding}

{In the future, we {aim} to apply this scheme to an intermodal mobility network and to electric vehicle charging problems.}

%\msmargin{}{In the conclusion you assume readers to have read your paper, and try to answer the ``so what?'' question, whilst limiting the review of what you proposed to a few lines (and adding a bit more of why when you do so).}

%\textcolor{blue}{future work intermodal systems, charging }