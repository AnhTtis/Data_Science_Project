% !TeX spellcheck = en_US
\section{Numerical Results}\label{sec:num_res}

In this section, numerical results are presented for an illustrative case study with $n = 5$. We consider $M = 1000$ users of which, on average, $P_\mathrm{home} = 5\%$ do not travel every day. Their daily sensitivity is sampled from a uniform distribution on the interval $[0,2]$ and their prediction horizon is $T=4$. We model the discomfort as a travel-time Bureau of Public Roads (BPR) function \cite{BPR1964}
\begin{equation*}
	\mathbf{d}_j(\mathbf{x}_j) = \mathbf{d^0}_{\!\!j}\left(1+ \alpha (\mathbf{x}_j/\boldsymbol{\kappa}_j)^\beta\right),
\end{equation*}
with $\alpha = 0.15$, $\beta = 4$, and $\mathbf{d^0}$ and $\boldsymbol{\kappa}$ were generated randomly which, rounded to four decimal places, are given by $\mathbf{d^0} = [0.5001 \; 0.5734 \; 0.7085 \; 0.6512 \; 0.8602]^\top$ and $\boldsymbol{\kappa} = [0.0923 \; 0.1863 \; 0.3968 \; 0.3456 \;0.5388]^\top$, ordered according to the arc ordering in Assumption~\ref{ass:ordering}. We consider distribution of the reference values $\theta_\mathbf{p}$ to be a discrete uniform distribution with support $\{k_\mathrm{ref}\in \mathbb{N} \,|\, k_\mathrm{ref} = 0 \lor k_\mathrm{ref} = \mathbf{p}_j \}$, which corresponds to users having the possibility of saving Karma to afford traveling through an arc with a positive price at the end of the horizon.  The system's cost is considered to be a weighted sum of the travel-time in each link, i.e., $\mathbf{c}_j(\mathbf{x}):=  \mathbf{c^0}_{\!\!j}\mathbf{d}_j(\mathbf{x}_j)$, whose weights were randomly generated and, rounded to four decimal places, are given by $\mathbf{c^0}_{\!\!\!j} = [0.7096 \; 0.8426 \; 0.9391 \; 0.6022 \; 0.5137]$. This can correspond to the weighted minimization of, for example, sound pollution.



Rounded to four decimal places, {employing \cite{Loefberg2004},} $\mathbf{x}^\star = [0.0877 \; 0.1309 \; 0.0000 \; 0.3053 \; 0.4261]^\top$  and ${\mathbf{d}(\mathbf{x}^\star) = [0.5611 \; 0.5943 \; 0.7085 \; 0.7107 \; 0.9106]^\top}$, which is in accordance with Assumption~\ref{ass:ordering}. The optimization problem \eqref{eq:num_prob} is solved using a standard genetic algorithm method subject to $||\mathbf{p}||_\infty \leq 100$ {in less than 500 wall-clock seconds in a standard laptop}, whose solution is ${\mathbf{p}^\star = [79\;63\;39\;13\;-45 ]^\top}$. We considered $k_0 = \mathbf{p}_1$ and $\mathbf{x}^{\star}_\mathrm{quant}$ {resulting from rounding $\mathbf{x}^\star$ to three decimal places}.



\begin{figure}[ht!]
	\begin{subfigure}{\linewidth}
		\centering
		\includegraphics[width = \linewidth]{fig/decision-eps-converted-to.pdf}
		\caption{Evolution of aggregate flows.}
		\label{fig:decision}
	\end{subfigure}\\%
	\begin{subfigure}{\linewidth}
		\raggedleft
		\includegraphics[width = 0.99\linewidth]{fig/karma-eps-converted-to.pdf}
		\caption{Evolution of Karma level.}
		\label{fig:karma}
	\end{subfigure}\\%
	\begin{subfigure}{\linewidth}
		\raggedleft
		\includegraphics[width = 0.99\linewidth]{fig/cost-eps-converted-to.pdf}
		\caption{Evolution of the relative cost difference.}
		\label{fig:cost}
	\end{subfigure}\\%
	\begin{subfigure}{\linewidth}
		\raggedleft
		\includegraphics[width = \linewidth]{fig/sensitivity-eps-converted-to.pdf}
		\caption{Evolution of the relative sensitivity and discomfort deviation.}
		\label{fig:sensitivity}
	\end{subfigure}%
	\caption{Numerical simulation results.}
	\label{fig:num_sim}
\end{figure}

The daily simulations are carried out by computing the Nash equilibrium that follows from the decisions of each user to Problem~\ref{sec:bestresponse}, which approximate the WE as $M\to\infty$. The Karma values were initialized randomly according to  a discrete uniform distribution with support $\{25\mathbf{p}_1,25\mathbf{p}_1,+1,\ldots,50\mathbf{p}_1\}$.  Figs.~\ref{fig:decision}--\ref{fig:cost} depict the evolution of the aggregate flows, Karma level, and relative cost difference in relation to the system optimum, respectively, throughout the simulation. We denote the average and the standard deviation of the users' Karma level at time $t$ by $\hat{k}(t)$ and $\sigma_k(t)$, respectively. First, since the initial Karma levels are very high, the users act as if the pricing scheme were not implemented. This can be seen in the initial plateau in Fig.~\ref{fig:cost} which is associated with the constant aggregate flows visible in Fig.~\ref{fig:decision}. Nevertheless, as the users' Karma is depleted, as shown in Fig.~\ref{fig:karma}, the users can no longer afford every link and the pricing mechanism drives the aggregate flows to the system-optimal flows. Second, it is important to point out that, despite all the assumptions made to tackle the intractability of the pricing design problem and enable a numerical solution, the prices that were designed get very close to the system optimum, as visible in Fig.~\ref{fig:cost}, with an average relative difference in relation to the theoretical optimum of  $0.15\,\%$ only over the last $50$ instants of the simulation. In fact, the steady-state aggregate flows of the numerical simulation closely match $\mathbf{x}^\star$, as visible in Fig.~\ref{fig:decision}. Third, we analyze  i)~the relative difference of the average perceived discomfort w.r.t. a scenario in which the users are centrally allocated to the optimal flows randomly, i.e. without taking into account their sensitivity, which is given by
%the average relative discomfort perceived at each time that would be perceived if the users were allocated to the same flows randomly without taking into account the sensitivity of each user as
\begin{equation*}
	\Delta \bar{d}(t) := \frac{\sum_{i = 1}^M s^i(t) \mathbf{d}(\mathbf{x})^\top\mathbf{y}^i(t) + \bar{s}\, \mathbf{d}(\mathbf{x})^\top\mathbf{y}^i(t) }{\sum_{i = 1}^M \bar{s}\, \mathbf{d}(\mathbf{x})^\top\mathbf{y}^i(t)};
\end{equation*}
and ii)~the relative deviation of the average sensitivity in relation to the expected sensitivity, i.e., ${\Delta \bar{s}(t) := (1/M)\sum_{i = 1}^M (s^i(t)-\bar{s})/\bar{s}}$. Fig.~\ref{fig:sensitivity} depicts the evolution of these two quantities. It is noticeable that, at steady-state, the perceived discomfort is roughly $8\,\%$ lower in comparison to an optimal but urgency-unaware policy.

%\begin{figure}[ht]
%	\centering
%	\includegraphics[width = 1\linewidth]{fig/decision.eps}
%	\caption{Evolution of aggregate flows.}
%	\label{fig:decision}
%\end{figure}
%
%\begin{figure}[ht]
%	\centering
%	\includegraphics[width = 1\linewidth]{fig/karma.eps}
%	\caption{Evolution of Karma level.}
%	\label{fig:karma}
%\end{figure}
%
%\begin{figure}[ht]
%	\centering
%	\includegraphics[width = 1\linewidth]{fig/cost.eps}
%	\caption{Evolution of the relative cost difference in relation to the theoretical system's optimum.}
%	\label{fig:cost}
%\end{figure}
%
%\begin{figure}[ht]
%	\centering
%	\includegraphics[width = 1\linewidth]{fig/sensitivity.eps}
%	\caption{Evolution of the relative average sensitivity and discomfort deviation.}
%	\label{fig:sensitivity}
%\end{figure}



Due to space limitations, some details regarding the numerical pricing design and the simulation were omitted.  Nevertheless, a MATLAB implementation as well as additional simulation results,  is openly available in an open source repository at  {\small\texttt{\url{https://fish-tue.github.io/single-origin-destination-routing}}}.
%. For more information refer to the Code Availability Statement section.