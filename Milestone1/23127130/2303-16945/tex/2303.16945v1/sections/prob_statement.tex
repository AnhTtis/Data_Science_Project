% !TeX spellcheck = en_US
\section{Problem Statement}\label{sec:prob_statement}

This section states the mechanism design problem, which closely follows the formulation in \cite{SalazarPaccagnanEtAl2021}. Three different perspectives are of interest, each corresponding to a subsection: i)~the macroscopic perspective of the central operator, aiming to minimize the societal costs that result from the routing patterns; ii)~the microscopic perspective of each self-interested user, who desires to minimize their daily perceived discomfort;  and iii)~the mesoscopic overarching perspective of the incentive mechanism design framework to align these two seemingly opposing objectives.

%\subsection{Incentive mechanism framework}

Consider the mobility network  with a single origin and destination node connected by $n\in \mathbb{N}$ distinct itineraries depicted in Fig.~\ref{fig:network_example}. We consider a repeated game setting whereby each user chooses from one of the available arcs to reach their destination at each discrete time $t\in \mathbb{N}$.

The incentive mechanism that is employed is based on an artificial currency---Karma. In this framework, taking a particular itinerary comes at a cost of Karma. Let $\mathbf{p}\in \mathbb{R}^n$ denote the prices of the itineraries, i.e., choosing arc $j$ comes at a cost $\mathbf{p}_j$. Users are not allowed to buy or trade Karma, and they can only select arcs that maintain their Karma-level non-negative. It is, hence, crucial that certain (more uncomfortable) arcs are assigned negative prices, which means that users are awarded Karma for taking them.

From a microscopic perspective, denote the route choice of a user $i$ at time instant $t$ by the binary vector $\mathbf{y}^i(t) \in \{0,1\}^n$, whose entry $j$ is given by $\mathbf{y}^i_j(t) = 1$ if the user $i$ chooses arc $j$ at time $t$, and $\mathbf{y}^i_j(t) = 0$ otherwise. Since each user may not travel at time $t$ we can have $\mathbf{y}^i(t) = \mathbf{0}$. Therefore, it follows that $\mathbf{1}^\top\mathbf{y}^i(t) \leq 1$. Let $k^i(t) \in \mathbb{R}_{\geq0}$ denote the amount of Karma that a user $i$ owns at time $t$. Following a routing choice the Karma level is updated according to $k^i(t+1) = k^i(t)-\mathbf{p}^\top\mathbf{y}^i(t)$. 

From a mesoscopic perspective, let $\mathbf{x}(t) \in [0,1]^n$ denote the fraction of users crossing each arc at time $t$. Given a scenario with $M$ users, it is defined as $\mathbf{x}(t) = \frac{1}{M}\sum_{i=1}^M\mathbf{y}^i(t)$. To account for non-traveling users, it is assumed that each user has a constant probability $P_{\text{home}} \in [0,1]$  of not traveling.  Conversely, the probability for a user to travel is $P_{\text{go}}= 1-P_{\text{home}}$ and $\mathbb{E}\left[\mathbf{1}^\top\mathbf{x}(t)\right] = P_{\text{go}}$.


\subsection{Central Operator's Problem}
From the macroscopic perspective of the central operator of the mobility network, the flows across each arc cause a {societal} cost. Let $\mathbf{c}:[0,1]^n \to \mathbb{R}_{\geq 0}^n$ denote the {societal} cost function. It models the {societal} cost of each arc $j$ per user, $\mathbf{c}_j(\mathbf{x}_j(t))$, which is assumed to be monotonically increasing with the fraction  $\mathbf{x}_j(t)$ of users taking it. The desire of the central operator is that the aggregate flows minimize the total {societal} cost {$C(\mathbf{x}):=\mathbf{c}(\mathbf{x})^\top\mathbf{x}$}, which is formulated in the following problem:

\begin{problem}[Central Operator's Problem]\label{prb:design}\color{black}
	The central operator aims at routing customers so that the aggregate flows are
	%$\lim\limits_{t\to \infty} \mathbf{x}(t) = \mathbf{x}^{\star}$, where
	\begin{equation*}
		\begin{split}
			\mathbf{x}^{\star} \in &\arg \min \limits_{\mathbf{x}\in [0,1]^n}  {C(\mathbf{x})}\\
			& \;\mathrm{s.t.} \phantom{ \min \limits_{x\in [0,1]^n} } \mathbf{1}^\top\mathbf{x} = P_{\mathrm{go}}.
		\end{split}
	\end{equation*}
\end{problem}

\subsection{Individual User's Problem}
From the microscopic perspective of each user, taking an itinerary $j$ comes with a discomfort. Let ${\mathbf{d}:[0,1]^n \to \mathbb{R}_{\geq 0}^n}$ denote the user's discomfort function. It  models the discomfort that stems from taking arc $j$ per user, $\mathbf{d}_j(\mathbf{x}_j(t))$, which is assumed to be monotonically increasing with the fraction of users taking it, $\mathbf{x}_j(t)$.
In contrast to well-known monetary tolling schemes~\cite{BergendorffHearnEtAl1997}, the individual users' complex behavior cannot be captured within a static setting:
From their self-interested view-point, the users are assumed to make choices in order to minimize their traveling discomfort without reaching a negative level of Karma. Thus, the users are playing against their future selves when deciding whether to spend or receive Karma. Furthermore, the perception of discomfort of a user varies daily. The sensitivity to discomfort of a user $i$ at time $t$ is denoted by $s^i(t)$, which is a weighting factor of the daily discomfort.  The sensitivities $s^i(t)$ are assumed to be i.i.d. extractions (w.r.t. $i$ and $t$) of a common probability density function $\rho:[s_{\mathrm{min}},s_{\mathrm{max}}] \to [0,1]$ with support set  $[s_{\mathrm{min}},s_{\mathrm{max}}] \subseteq  \mathbb{R}_{\geq 0}$ and expected value $\bar{s} \in \mathbb{R}_{\geq0}$.  To capture this complex behavior, the users are assumed to be rational and to minimize a combination of their immediate discomfort, weighted by their immediate urgency, and the discomfort encountered for a time period $T$ into the future, weighted by their average urgency. Additionally, we assume each user $i$ to be conservative in terms of Karma, i.e., they will make the route decisions so that their Karma at the end of the horizon will not fall below a reference value ${k_{\mathrm{ref}}^i \in \mathbb{R}_{\geq0}}$. For example, a user may choose a reference value of $\mathbf{p}_1$ to ensure that {they} can still afford to travel in arc $1$ at the end of the horizon. It, thus, depends on the pricing policy $\mathbf{p}$.  Herein, we will assume it to be time-invariant and randomly distributed among the users according to the distribution $\theta_{\mathbf{p}}:\mathbb{R}_{\geq 0} \to\mathbb{R}_{\geq0}$.
%\msmargin{Combining everything together}{more technical way of saying it? E.g., Formally, we obtain...} 
{Formally, we obtain} the following individual user's problem:
%Herein, we will assume it to be time-invariant and randomly distributed among the users according to the distribution $\theta:[k_{\mathrm{ref}_{\mathrm{min}}},k_{\mathrm{ref}_{\mathrm{max}}}] \to  \mathbb{R}_{\geq0}$ with support set  $[k_{\mathrm{ref}_{\mathrm{min}}},k_{\mathrm{ref}_{\mathrm{max}}}] \subseteq  \mathbb{R}_{\geq 0}$.
%\textcolor{red}{[Change the support set of kref consistently]}
%, for example, a user may have a greater urgency when traveling to an important meeting than when commuting.  

%\lpmargin{$\theta_{\mathbf{p}}:\mathbb{R}_{\geq 0} \to\mathbb{R}_{\geq0}$}{more intuitive symbol? - let's discuss this}.
%\msmargin{he/she}{use \textbf{they} everywhere}

%\msmargin{In contrast to well-known monetary tolling \msmargin{schemes}{add ref}, the individual users' complex behavior cannot be captured within a static setting.}{perhaps this statement should be earlier} \msmargin{Combining everything together yields the following individual user's problem.}{a bit abrupt wrt the previous sentence...}
\begin{problem}[Individual User's Problem]\label{prb:individual}
	At time $t \in \mathbb{N}$, given the flows $\mathbf{x}$ and prices $\mathbf{p}$ a traveling user with Karma level $k\geq 0$, reference $k_{\mathrm{ref}}$, and sensitivity $s$ will choose {their} route as $\mathbf{y}^{\star}$ resulting from
	%		\par\nobreak\vspace{-10pt}
	%		\begingroup
	%		\allowdisplaybreaks
	%		\begin{small}
		\begin{subequations}\label{eq:singleAgentAverage}
			\begin{align}\label{eq:singleAgentAverage_cost}
				(\mathbf{y}^\star,\bar{\mathbf{y}}^\star) \in \mathop{\mathrm{argmin}}_{\mathbf{y} \in \mathcal{Y},\;\bar{\mathbf{y}} \in \bar{\mathcal{Y}}} \;&s\, \mathbf{d}(\mathbf{x})^\top\mathbf{y}+ T\,\bar{s}\, \mathbf{d}(\mathbf{x})^\top \bar{\mathbf{y}}\\  \label{eq:singleAgentAverage_c1}
				\mathrm{s.t.}\;\; &k-\mathbf{p}^\top \mathbf{y} - T\mathbf{p}^\top\bar{\mathbf{y}} \geq k_\mathrm{ref}\\ \label{eq:singleAgentAverage_c2}
				&\mathbf{p}^\top \mathbf{y} \leq k,
			\end{align}
		\end{subequations}
		%			\end{small}
	%		\endgroup
	with $ \mathcal{Y} = \{\mathbf{y}\in\{0,1\}^n: \mathbf{1}^\top \mathbf{y} = 1\}$ and $\bar{\mathcal{Y}} =\{\mathbf{y}\in[0,1]^n:\mathbf{1}^\top \mathbf{y}= 1\}$.
	%\textcolor{black}{Thus, the user chooses arc $j^\star$ if $\mathbf{y}^\star = \mathbf{e_{j^\star}}$.}
	Non-traveling users have $\mathbf{y}^\star = \mathbf{0}$.
\end{problem}


\subsection{Mechanism Design Problem}

Similar to~\cite{SalazarPaccagnanEtAl2021} and following the notation in~\cite{GorokhBanerjeeEtAl2019}, we consider a non-atomic game framework, which corresponds to the limit case where users form a continuum with ${M\to \infty}$. To describe an infinite-user population, let ${\eta_t : \mathbb{R}_{\geq 0}  \times \mathbb{R}_{\geq 0} \to \mathbb{R}_{\geq 0}}$ denote the instantaneous distribution of the Karma level and reference in the population at time $t$, where $\int_0^{\infty} \!\!\int_0^\infty \eta_t (k,k_\mathrm{ref}) \,\mathrm{d}k \, \mathrm{d}k_\mathrm{ref} = 1$. For the infinite-user setting, the Nash and Wardrop Equilibrium (WE) are identical~\cite{PaccagnanGentileEtAl2018} and can be defined as follows:
% let $\eta_t(k,k_\mathrm{ref}) : \mathbb{R}_{\geq0} \times \mathbb{R}_{\geq 0} \to \mathbb{R}_{\geq 0}$ be the instantaneous distribution of the Karma level $k$ and reference $k_\mathrm{ref}$ in the population.

\begin{definition}[Wardrop Equilibrium]\label{def:WE} \color{black}
	$\mathbf{x}^\mathrm{WE}(t)\in[0,1]^n$ satisfying $\mathbf{1}^\top \mathbf{x}^\mathrm{WE}(t) = P_\mathrm{go}$ is a WE at time $t$, if %there exists $\mathbf{y}^\star(\mathbf{x}^\mathrm{WE}(t),s,k,k_\mathrm{ref})\in \mathcal{Y}^\star(\mathbf{x}^\mathrm{WE}(t),s,k,k_\mathrm{ref})$ so that
	{\small
		\begin{equation*}
			\begin{split}
				&\mathbf{x}^\mathrm{WE}(t)  =\\
				& \int_{0}^\infty \!\! \!\! \int_{0}^\infty \!\!\!\! 	\int^{s_\mathrm{max}}_{s_\mathrm{min}}\!\!\!\! \mathbf{y}^{\star}\!(\mathbf{x}^\mathrm{WE}(t),s,k,k_\mathrm{ref})\,\rho(s)\,\eta_t(k,k_{\mathrm{ref}})\, \mathrm{d}s\,\mathrm{d}k\,\mathrm{d}k_{\mathrm{ref}},
			\end{split}
	\end{equation*}}%
	where $\mathbf{y}^\star$ is a best response strategy that follows from Problem~\ref{prb:individual}.
\end{definition}



%\msmargin{Fig.~\ref{fig:scheme_karma}}{where is the mechanism? It looks more like the simulation model}

At each time $t$ the aggregate choices of the users are modeled by the WE $\mathbf{x}^{\mathrm{WE}}(t)$. Fig.~\ref{fig:scheme_karma} depicts a scheme of a time-step of the overall model. The mechanism design problem is then to select the arc prices, so that the daily WE converges to the system optimum $\mathbf{x}^\star$, as stated in the problem below:

\begin{figure}[ht]
	\centering
	\includegraphics[width = 0.8\linewidth]{fig/scheme_karma.pdf}
	\caption{Schematic representation of one time-step of the overall model.}
	\label{fig:scheme_karma}
\end{figure}

\begin{problem}[Pricing Problem]\label{prb:prices}
	Given a desired system optimum $\mathbf{x}^\star$, select $\mathbf{p}\in\mathbb{R}^n$ so that $\lim_{t\to\infty} \mathbf{x}^\mathrm{WE}(t) = \mathbf{x}^\star$.
\end{problem}

To ensure the well-posedness of the pricing problem, the following key assumptions are made on the existence, uniqueness, and convergence of a WE. Given that the best response strategy cannot be formulated in a static setting, and a mixed user strategy is not meaningful, these assumptions are, by no means, obvious. Future research endeavors will focus on the intricacies of the game-theoretic framework, whereas, in this paper, the focus is on the pricing design framework.

%which are reasonable in the context of the problem. \textcolor{red}{[add better explanation here or (even better) a proof :) - check continuity of $\gamma_j$ as a function of $d$ (existence) - contractivity of eq in Def II.1 for a fixed $\eta$ (uniqueness and convergence)]}.

\begin{assumption}[Existence and uniqueness of WE]\label{ass:WE}
	Given a Karma level distribution $\eta_t : \mathbb{R}_{\geq 0}  \times \mathbb{R}_{\geq 0} \to \mathbb{R}_{\geq 0}$, a WE $\mathbf{x}^\mathrm{WE}(t)$ exists and is unique.
\end{assumption}

\begin{assumption}[Convergence of WE]\label{ass:WE_convergence}
	For a given pricing strategy $\mathbf{p}$, a stationary WE exists, i.e., ${\mathbf{x}^\mathrm{WE}_{\infty} := \lim_{t\to\infty} \mathbf{x}^\mathrm{WE}(t)}$, irrespective of the initial Karma level distribution $\eta_0 : \mathbb{R}_{\geq 0}  \times \mathbb{R}_{\geq 0} \to \mathbb{R}_{\geq 0}$.
\end{assumption}


\section{Best Response Strategy}\label{sec:bestresponse}

In this section, we focus on the individual user's problem. Specifically, we examine its properties and derive a closed-form solution of the best response strategy, which we will prove to be of paramount importance to the pricing design procedure proposed in this paper. The following result details necessary and sufficient conditions for the feasibility of Problem~\ref{prb:individual}.

\begin{lemma}\label{lem:feasibility}
	Consider a traveling user with Karma $k$, sensitivity $s$, Karma reference $k_\mathrm{ref}$, and prices $\mathbf{p}$. Problem~\ref{prb:individual} is feasible if and only if {\small $k \geq \max(0,k_\mathrm{ref}+(\min_j\mathbf{p}_j)(T+1))$}.
\end{lemma}
\begin{proof}
	The proof can be found in
	\ifextendedversion
	Appendix~\ref{app:proof_feasibility}.
	\else
	the extended version of this paper~\cite{extendedversion}.
	\fi
\end{proof}

%It is \msmargin{challenging}{avoid judging articles: Simply start with ``To derive a closed-form..., we follow...''} 
{To derive a closed-form solution to Problem~\ref{prb:individual}, we} follow a divide-and-conquer approach. {In a first instance}, in the following theorem, we establish an equivalence between the solutions of Problem~\ref{prb:individual} and a reduced best response problem whose discomforts and prices can be strictly ordered. {More specifically, we make three statements about Problem~\ref{prb:individual}. First, if, for a given arc $j$, there exists an arc $i$ with strictly lower discomfort and cost, then arc $j$ is unreasonable, in the sense that it is never chosen. Second, the discomforts and cost of the reduced set of arcs that are not unreasonable and have distinct discomfort values can be strictly ordered. Third, all integer solutions of Problem~\ref{prb:individual} can be obtained by the solutions to a reduced best response problem, whose discomforts and prices can be strictly ordered. These statements are presented with rigor in the following theorem.}


% \msmargin{the of}{?}

%\msmargin{feasibility}{are you referring to Lemma III.1? }

\begin{theorem}\label{thm:brs_general}
	Consider a traveling user with Karma $k$, sensitivity $s$, and Karma reference $k_\mathrm{ref}$, aggregate flow $\mathbf{x}$, and prices $\mathbf{p}$. Assume, without loss of generality, that the itineraries are numbered, so that $\mathbf{d}_1(\mathbf{x}_1) \leq  \ldots \leq \mathbf{d}_n(\mathbf{x}_n)$ is satisfied. Then, {under the feasibility conditions of Lemma~\ref{lem:feasibility}}:\\
	i) $j^\star \notin \mathcal{J}_u$, where $\mathbf{y}^\star = \mathbf{e_{j^\star}}$ and {\small 	$\mathcal{J}_u := \{ j\in \{1,\ldots,n\} \;|\; \exists i \in  \{1, \ldots n\}: \mathbf{p}_i  \leq  \mathbf{p}_j \land \mathbf{d}_i(\mathbf{x}_i) < \mathbf{d}_j(\mathbf{x}_j) \};$}\\
	ii) {\small $\!\forall i,j  \!\in\!\! \{1,\ldots,n\} \!\setminus \!( \mathcal{J}_u \cup \mathcal{J}_e) \,j\!<\! i  \Longrightarrow \mathbf{p}_j\! > \!\mathbf{p}_i \land \mathbf{d}_j(\mathbf{x}_j)\! > \!\mathbf{d}_i(\mathbf{x}_i)$}, where 	
	{\small \begin{equation*}
			\begin{split}
				&\mathcal{J}_e \!:= \!\left\{ j\!\in \!\{1,\ldots,n\}  \,|\, \exists i \! \in \! \{1, \ldots n\} : \right.\\&\quad \quad \quad \quad\quad 
				\left. \mathbf{d}_i(\mathbf{x}_i) \!= \!\mathbf{d}_j(\mathbf{x}_j) \land (\mathbf{p}_i <\mathbf{p}_j \lor (\mathbf{p}_i  = \mathbf{p}_j  \land i<j) )\right\}.
			\end{split}
	\end{equation*}}\\
	iii) if $(\mathbf{e_{q}}, \mathbf{\bar{y}^{q}})$, $q = 1,\ldots, Q$ are all the solutions to Problem~\ref{prb:individual} for {reduced} aggregate flows $ \{\mathbf{x}_j\}_{j\in \{1,\ldots,n\} \setminus ( \mathcal{J}_u \cup  \mathcal{J}_e) }$, and {reduced} prices $\{\mathbf{p}_j\}_{j\in \{1,\ldots,n\} \setminus ( \mathcal{J}_u \cup  \mathcal{J}_e) }$, then $\mathbf{y}^\star = \mathbf{e_{j^\star}}$ with
	{\small \begin{equation*}
			\begin{split}
				j^\star \in \bigg\{ j\in \{1,\ldots,n\}| \exists q \in \{1,\ldots,Q\} : \mathbf{d}_j(\mathbf{x}_j) = \mathbf{d}_{q^{\star}}(\mathbf{x}_{q^{\star}}) \land  \\ k \geq \mathbf{p}_j \land 
				\left .k - \mathbf{p}_j -T\sum \nolimits_{i\in \{1,\ldots,n\} \setminus ( \mathcal{J}_u \cup  \mathcal{J}_e)} \mathbf{p}_i  \mathbf{\bar{y}^{q}}_i \geq k_{\mathrm{ref}}\right\}
			\end{split}
	\end{equation*}}%
	are all the integer solutions to Problem~\ref{prb:individual}.
\end{theorem}
\begin{proof}
	The proof can be found in
	\ifextendedversion
	Appendix~\ref{app:proof_brs_general}.
	\else
	the extended version of this paper~\cite{extendedversion}.
	\fi
\end{proof}




%if two arcs have the same discomfort, albeit possibly different prices, we show that both are optimal solutions for a sufficiently high level of Karma. Furthermore, 
%
%\textcolor{red}{[Do this before - what we saying is ...... whic is condendensed in teh following the]}
%Three \lpmargin{remarks}{I would somehow explain what the three subresults of the Thm mean (...) - that's exactly what the First, second, and third are - I made it clearer} 

{A few remarks are in order regarding Theorem~\ref{thm:brs_general}. First, depending on the prices and discomforts at a given time, there may be arcs that are not chosen for any sensitivity or Karma level. 
Second, if two arcs have the same discomfort, albeit possibly different prices, both are {equally fit} for a sufficiently high level of Karma. Third, even though similar equivalence conditions could have been stated for the non-integer component of the solutions, they were omitted for the sake of brevity. In a second instance, in the following theorem, a closed-form solution is presented for the aforementioned reduced problems.}

%\msmargin{Let}{use definition environment before the Thm for all these items?} 

%Afterwards, we derive a closed-form solution to such reduced problem.


\begin{theorem}\label{thm:brs}
	Consider a traveling user with Karma $k$, sensitivity $s$, and Karma reference $k_\mathrm{ref}$, an aggregate flow $\mathbf{x}$, and prices $\mathbf{p}$.   Assume that  $\mathbf{d}_1(\mathbf{x}_1) <  \ldots < \mathbf{d}_n(\mathbf{x}_n)$ and $\mathbf{p}_1 > \ldots >\mathbf{p}_n$. Let $k(j_1,j_2):= k_{\mathrm{ref}} + \mathbf{p}_{j_1} + T\mathbf{p}_{j_2}$,
	\ifextendedversion
	\begin{equation}\label{eq:j_hat}
		\hat{j}_a := \mathop{\mathrm{argmin}}_{\substack{i\in \{1,\ldots,n\} \setminus \{a\}\\  k \geq \min(k(j,a),k(j,i)) \\ k \leq \max(k(j,a),k(j,i)) }}\frac{\mathbf{d}_i(\mathbf{x}_i)-\mathbf{d}_a(\mathbf{x}_a)}{\mathbf{p}_a-\mathbf{p}_i},
	\end{equation}
	\else
	\begin{equation*}
		\hat{j}_a := \mathop{\mathrm{argmin}}_{\substack{i\in \{1,\ldots,n\} \setminus \{a\}\\  k \geq \min(k(j,a),k(j,i)) \\ k \leq \max(k(j,a),k(j,i)) }}\frac{\mathbf{d}_i(\mathbf{x}_i)-\mathbf{d}_a(\mathbf{x}_a)}{\mathbf{p}_a-\mathbf{p}_i},
	\end{equation*}
	\fi
	\ifextendedversion
	\begin{equation}\label{eq:y_j_a_star}
		\mathbf{\bar{y}}^\star(j,a) := \frac{(k-k(j,\hat{j}_a))\mathbf{e_a} - (k-k(j,a))\mathbf{e_{\hat{j}_a}}}{T(\mathbf{p}_a-\mathbf{p}_{\hat{j}_a})},
	\end{equation}
	\else
	\begin{equation*}
		\mathbf{\bar{y}}^\star(j,a) := \frac{(k-k(j,\hat{j}_a))\mathbf{e_a} - (k-k(j,a))\mathbf{e_{\hat{j}_a}}}{T(\mathbf{p}_a-\mathbf{p}_{\hat{j}_a})},
	\end{equation*}
	\fi
	\ifextendedversion
	\begin{equation}\label{eq:a_hat}
		\hat{a} := \mathop{\mathrm{argmin}} \limits_{a\in \{i,\ldots,n\}} \mathbf{d}(\mathbf{x})^\top	\mathbf{\bar{y}}^\star(j,a),
	\end{equation}
	\else
	\begin{equation*}
		\hat{a} := \mathop{\mathrm{argmin}} \limits_{\substack{a\in \{i,\ldots,n\} \\ k \geq \min(k(j,a),k(j,\hat{j}_a)) \\  k \leq \max(k(j,a),k(j,\hat{j}_a))}} \mathbf{d}(\mathbf{x})^\top	\mathbf{\bar{y}}^\star(j,a),
	\end{equation*}
	\fi
	\begin{equation*}
		\mathbf{\bar{y}_j}^\star := \begin{cases}
			\mathbf{\bar{y}}^\star(j,\hat{a}), \, & k < k(j,1)\\
			\mathbf{e_1}, \, & k \geq k(j,1)
		\end{cases},
	\end{equation*}
	\ifextendedversion
	\begin{equation}\label{eq:gamma_ij}
		\gamma_{i,j} := \begin{cases}
			\frac{T\mathbf{d}^\top (\mathbf{x})\left(	\mathbf{\bar{y}_i}^\star - 	\mathbf{\bar{y}_j}^\star\right)}{\mathbf{d}_j(\mathbf{x}_j)-\mathbf{d}_i(\mathbf{x}_i)}, \; & k \geq \max(0,\mathbf{p}_i,k(i,n))\\
			\infty, & \text{otherwise}
		\end{cases},
	\end{equation}
	\else
	\begin{equation*}
		\gamma_{i,j} := \begin{cases}
			\frac{T\mathbf{d}^\top(\mathbf{x})\left(	\mathbf{\bar{y}_i}^\star - 	\mathbf{\bar{y}_j}^\star\right)}{\mathbf{d}_j(\mathbf{x}_j)-\mathbf{d}_i(\mathbf{x}_i)}, \; & k \geq \min(0,k(i,n))\\
			+\infty, & \text{otherwise}
		\end{cases},
	\end{equation*}
	\fi
\begin{equation*}
		\underline{\gamma}_j := \begin{cases}
			\max_{ i\in \{j+1,\ldots,n\}} \gamma_{j,i} & j <n \\
			-\infty & j = n
		\end{cases} ,
	\end{equation*}
\begin{equation*}
		\bar{\gamma}_j := \begin{cases}
			\min_{i\in \{1, \ldots,j-1\}}  \gamma_{i,j} & j >1 \\
			+\infty & j = 1
		\end{cases},
	\end{equation*}
	\begin{equation*}
		\gamma_{j}(k,k_{\mathrm{ref}},\mathbf{p},\mathbf{d(x)}) := \begin{cases}
			s_{\mathrm{min}}/\bar{s}, \;  & j = n\\
			[\bar{\gamma}_{j+1}]_{s_{\mathrm{min}}/\bar{s}}^{s_{\mathrm{max}}/\bar{s}} ,  \;  &  \bar{\gamma}_{j+1} \geq \underline{\gamma}_{j+1}\\
			\gamma_{j+1},  \; &  \bar{\gamma}_{j+1} < \underline{\gamma}_{j+1}\\
			s_{\mathrm{max}}/\bar{s}, \;  & j = 0,
		\end{cases}
	\end{equation*}
	where the dependence on $k$, $k_{\mathrm{ref}}$, $\mathbf{p}$, and $\mathbf{d(x)}$ was dropped to alleviate the notation. Then,  {under the feasibility conditions of Lemma~\ref{lem:feasibility}}, an optimal response strategy that follows from Problem~\ref{prb:individual} is $\mathbf{y}^\star = \mathbf{e_{j^\star}}${, if and only if} $\bar{\gamma}_{j^\star} \geq \underline{\gamma}_{j^\star}$ and  ${\gamma_{j^{\star}} \leq s/\bar{s} \leq \gamma_{j^\star-1}}$.
\end{theorem}
\begin{proof}
	The proof can be found in
	\ifextendedversion
	Appendix~\ref{app:proof_brs}.
	\else
	the extended version of this paper~\cite{extendedversion}.
	\fi
\end{proof}

\begin{figure}[ht]
	\centering
	\includegraphics[width = 1\linewidth]{fig/decision_landscape-eps-converted-to.pdf}
	\caption{Best response strategy of Problem~\ref{prb:individual} for aggregate flows $\mathbf{x}^\star$, prices $\mathbf{p}^\star$, and $k_\mathrm{ref} = 0$.}
	\label{fig:decision_landscape}
\end{figure}

A few remarks are in order. First, an example of a decision landscape generated by the closed-form solution in Theorems~\ref{thm:brs_general} and \ref{thm:brs} is depicted in Fig.~\ref{fig:decision_landscape}. Second, note that there is an attractive invariant Karma set contained in {\small $[0,  \; k_\mathrm{ref}^i + (T+1)\max_{j}\mathbf{p}_j - \min_j \mathbf{p}_j]$}. Third, the best response strategy is invariant on a positive scaling of $k$, $k_\mathrm{ref}$, and $\mathbf{p}$, i.e., $\mathbf{e_{j^\star}}$ is a best response strategy for $k$, $k_\mathrm{ref}$, and $\mathbf{p}$ {if and only if} it is also for  $\alpha k$, $\alpha k_\mathrm{ref}$, and $\alpha \mathbf{p}$, with $\alpha \in \mathbb{R}_{>0}$. Finally, notice that in contrast to the $n=2$ arcs problem analyzed in \cite{SalazarPaccagnanEtAl2021}, the best response strategy explicitly depends also on the quantitative discomfort levels. 