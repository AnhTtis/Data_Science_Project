% CVPR 2023 Paper Template
% based on the CVPR template provided by Ming-Ming Cheng (https://github.com/MCG-NKU/CVPR_Template)
% modified and extended by Stefan Roth (stefan.roth@NOSPAMtu-darmstadt.de)

\documentclass[10pt,twocolumn,letterpaper]{article}

%%%%%%%%% PAPER TYPE  - PLEASE UPDATE FOR FINAL VERSION
%\usepackage[review]{cvpr}      % To produce the REVIEW version
\usepackage{cvpr}              % To produce the CAMERA-READY version
%\usepackage[pagenumbers]{cvpr} % To force page numbers, e.g. for an arXiv version

% Include other packages here, before hyperref.
\usepackage{graphicx}
\usepackage{amsmath}
\usepackage{amssymb}
\usepackage{booktabs}
\DeclareMathOperator*{\argmax}{arg\,max}
\DeclareMathOperator*{\argmin}{arg\,min}
\newtheorem{theorem}{Theorem}

%\newtheorem{theorem}{Theorem}[section]
\newcommand{\BW}[1]{{\bf{[\textcolor{blue}{BW:} \textcolor{blue}{#1}]}}}
\newcommand{\RY}[1]{{\bf{[\textcolor{red}{RY:} \textcolor{red}{#1}]}}}

% It is strongly recommended to use hyperref, especially for the review version.
% hyperref with option pagebackref eases the reviewers' job.
% Please disable hyperref *only* if you encounter grave issues, e.g. with the
% file validation for the camera-ready version.
%
% If you comment hyperref and then uncomment it, you should delete
% ReviewTempalte.aux before re-running LaTeX.
% (Or just hit 'q' on the first LaTeX run, let it finish, and you
%  should be clear).
\usepackage[pagebackref,breaklinks,colorlinks]{hyperref}


% Support for easy cross-referencing
\usepackage[capitalize]{cleveref}
\crefname{section}{Sec.}{Secs.}
\Crefname{section}{Section}{Sections}
\Crefname{table}{Table}{Tables}
\crefname{table}{Tab.}{Tabs.}

%addition package
\usepackage{multirow}
\usepackage{multicol}
\usepackage{algorithm}
%\usepackage{amsthm}
\usepackage{algorithmicx}
\usepackage{algpseudocode}
\usepackage{caption}
\usepackage{mathtools, cuted}
\usepackage{lipsum, color}
%%%%%%%%% PAPER ID  - PLEASE UPDATE
\def\cvprPaperID{7271} % *** Enter the CVPR Paper ID here
\def\confName{CVPR}
\def\confYear{2023}


\begin{document}

%%%%%%%%% TITLE - PLEASE UPDATE
\title{IDGI: A Framework to Eliminate  Explanation Noise from Gradients Integration (Appendix)}

\maketitle
\section{Theorem}
\begin{theorem}
\label{theorem_1}
    Given a function $f_c(x): R^n \rightarrow R$, points $x_j, x_{j+1}, x_{j_p} \in R^n$, then the gradient of the function with respect to each point in the space $R^n$ forms the conservative vector fields $\overrightarrow{F}$ and further define the hyperplane $h_j=\{x: f_c(x)=f_c(x_j)\}$ in $\overrightarrow{F}$. Assume the Riemann Integration accurately estimates the line integral of the vector field $\overrightarrow{F}$ from points $x_j$ to $x_{j+1}$ and $x_{j_p}$ e.g. $\int_{x_j}^{x_{j_p}}\frac{\partial f_c(x)}{\partial x} dx \approx \frac{\partial f_c(x_j)}{\partial x_j}{(x_{j_p} - x_j)}$, and $x_j \in h_j$, $x_{j_p}, x_{j+1} \in h_{j+1}$. Then:
    \[\int_{x_j}^{x_{j+1}}\frac{\partial f_c(x)}{\partial x} dx \approx \int_{x_j}^{x_{j_p}}\frac{\partial f_c(x)}{\partial x} dx.\]
\end{theorem}
\textit{Proof.}
\begin{align}
    \int_{x_j}^{x_{j+1}}\frac{\partial f_c(x)}{\partial x} dx &= f_c(x_{j+1}) - f_c(x_{j}) \nonumber \\
    & \approx \frac{\partial f_c(x_j)}{\partial x_j}{(x_{j+1} - x_j)} \nonumber \\
    &= f_c(x_{j_p}) - f_c(x_{j}) \nonumber \\
    &=\frac{\partial f_c(x_j)}{\partial x_j}{(x_{j_p} - x_j)} \nonumber \\
    &=\int_{x_j}^{x_{j_p}}\frac{\partial f_c(x)}{\partial x} dx
\end{align}



\section{IG with IDGI}

The Integrated Gradients algorithm requires a specified reference image to compute the attribution. One often selects a black or white picture as a reference point, resulting in the zero attribution value to pixels with the same value as the reference. This is due to the fact that these pixel values do not change while traveling from the reference image to the original image. However, these pixels may still be crucial for the classifier to make the decision, and merit attribution differs from zero. For example, as shown in Figure \ref{fig_black_image}, the \textit{Xception} model makes the prediction correctly on the given image has a black dog. When utilizing IG for providing an explanation, the body of the dog will be assigned zero attributions since the reference (black) image has the same pixel value as the dog. Intuitively, the explanation method should give non-zero values to these black pixels, since they represent the dog's body and are assumed to be significant characteristics. Alternatively, if the attribution value is zero, it is likely because the feature is insignificant and not because of the explanation method's design. In contrast to the original IG, IG with IDGI might potentially assign non-zero values to pixels with the same value as the reference picture, a desirable trait for a superior explanation technique.

\begin{figure}[h]
     \centering
     \begin{subfigure}[h]{0.15\textwidth}
         \includegraphics[width=\textwidth]{figs/black_image_examples/original_Xception.png}
         \caption{Original Image}
         \label{original_black}
     \end{subfigure}
     \begin{subfigure}[h]{0.15\textwidth}
         \includegraphics[width=\textwidth]{figs/black_image_examples/Xception_ig_original.png}
         \caption{IG}
         \label{id_black}
     \end{subfigure}
     \begin{subfigure}[h]{.15\textwidth}
         \includegraphics[width=\textwidth]{figs/black_image_examples/Xception_ig_update.png}
         \caption{IG+IDGI}
         \label{id_idgi_black}
     \end{subfigure}
\caption{Original image is predicted \textit{Tibetan terrier} from Xception classifier. Both \ref{id_black} and \ref{id_idgi_black} are attributions from IG and IG+IDGI with the black image as reference. Since the pixels are also black for the original image on the dog region, by design, IG is not able to assign important values to those pixels, however, ID+IDGI overcomes the issue.}
\label{fig_black_image}
\end{figure}

\section{Visual Examples}
We present more visual examples in \cref{q1,q2,q3,q4,q5}.
\begin{figure*}[h]
     \centering
     \begin{subfigure}[b]{\textwidth}
         \centering
         \includegraphics[width=\textwidth]{figs/appendix_examples/bucket.png}
     \end{subfigure}
\caption{Predicted Label for all models: \textit{bucket}
}
\label{q1}
\end{figure*}

\begin{figure*}[h]
     \centering
     \begin{subfigure}[b]{\textwidth}
         \centering
         \includegraphics[width=\textwidth]{figs/appendix_examples/crane.png}
     \end{subfigure}
\caption{Predicted Label for all models: \textit{crane}
}
\label{q2}
\end{figure*}

\begin{figure*}[h]
     \centering
     \begin{subfigure}[b]{\textwidth}
         \centering
         \includegraphics[width=\textwidth]{figs/appendix_examples/mergus serrator.png}
     \end{subfigure}
\caption{Predicted Label for all models: \textit{mergus serrator}
}
\label{q3}
\end{figure*}

\begin{figure*}[h]
     \centering
     \begin{subfigure}[b]{\textwidth}
         \centering
         \includegraphics[width=\textwidth]{figs/appendix_examples/partridge.png}
     \end{subfigure}
\caption{Predicted Label for all models: \textit{partridge}
}
\label{q4}
\end{figure*}

\begin{figure*}[h]
     \centering
     \begin{subfigure}[b]{\textwidth}
         \centering
         \includegraphics[width=\textwidth]{figs/appendix_examples/quail.png}
     \end{subfigure}
\caption{Predicted Label for all models: \textit{quail}
}
\label{q5}
\end{figure*}

\section{Distribution by Normalized Entropy and MS-SSIM}
We present more distribution that compares Normalized Entroy and MS-SSIM in \cref{Dense121_f,Dense201_f,Dense169_f,Resnet101V2_f,Resnet151V2_f,Resnet50V2_f,Xception_f,InceptionV3_f,MobileNetV2_f,vgg16_f,vgg19_f}.
\begin{figure*}[!t]
     \centering
     \begin{subfigure}[b]{.9\textwidth}
         \centering
         \includegraphics[width=\textwidth]{figs/appendix_examples/distribution_folder/Dense121.png}
     \end{subfigure}
\caption{Modified distribution of bokeh images over MS-SSIM and Normalized Entropy  \cite{kapishnikov2019xrai}. Model: \textit{DenseNet121} 
}
\label{Dense121_f}
\end{figure*}

\begin{figure*}[!t]
     \centering
     \begin{subfigure}[b]{.9\textwidth}
         \centering
         \includegraphics[width=\textwidth]{figs/appendix_examples/distribution_folder/Dense169.png}
     \end{subfigure}
\caption{Modified distribution of bokeh images over MS-SSIM and Normalized Entropy  \cite{kapishnikov2019xrai}. Model: \textit{DenseNet169} 
}
\label{Dense169_f}
\end{figure*}

\begin{figure*}[!t]
     \centering
     \begin{subfigure}[b]{.9\textwidth}
         \centering
         \includegraphics[width=\textwidth]{figs/appendix_examples/distribution_folder/Dense201.png}
     \end{subfigure}
\caption{Modified distribution of bokeh images over MS-SSIM and Normalized Entropy  \cite{kapishnikov2019xrai}. Model: \textit{DenseNet201} 
}
\label{Dense201_f}
\end{figure*}

\begin{figure*}[!t]
     \centering
     \begin{subfigure}[b]{.9\textwidth}
         \centering
         \includegraphics[width=\textwidth]{figs/appendix_examples/distribution_folder/Resnet50V2.png}
     \end{subfigure}
\caption{Modified distribution of bokeh images over MS-SSIM and Normalized Entropy  \cite{kapishnikov2019xrai}. Model: \textit{Resnet50V2} 
}
\label{Resnet50V2_f}
\end{figure*}

\begin{figure*}[!t]
     \centering
     \begin{subfigure}[b]{.9\textwidth}
         \centering
         \includegraphics[width=\textwidth]{figs/appendix_examples/distribution_folder/Resnet101V2.png}
     \end{subfigure}
\caption{Modified distribution of bokeh images over MS-SSIM and Normalized Entropy  \cite{kapishnikov2019xrai}. Model: \textit{Resnet101V2} 
}
\label{Resnet101V2_f}
\end{figure*}

\begin{figure*}[!t]
     \centering
     \begin{subfigure}[b]{.9\textwidth}
         \centering
         \includegraphics[width=\textwidth]{figs/appendix_examples/distribution_folder/Resnet151V2.png}
     \end{subfigure}
\caption{Modified distribution of bokeh images over MS-SSIM and Normalized Entropy  \cite{kapishnikov2019xrai}. Model: \textit{Resnet151V2} 
}
\label{Resnet151V2_f}
\end{figure*}

\begin{figure*}[!t]
     \centering
     \begin{subfigure}[b]{.9\textwidth}
         \centering
         \includegraphics[width=\textwidth]{figs/appendix_examples/distribution_folder/InceptionV3.png}
     \end{subfigure}
\caption{Modified distribution of bokeh images over MS-SSIM and Normalized Entropy  \cite{kapishnikov2019xrai}. Model: \textit{InceptionV3} 
}
\label{InceptionV3_f}
\end{figure*}

\begin{figure*}[!t]
     \centering
     \begin{subfigure}[b]{.9\textwidth}
         \centering
         \includegraphics[width=\textwidth]{figs/appendix_examples/distribution_folder/Xception.png}
     \end{subfigure}
\caption{Modified distribution of bokeh images over MS-SSIM and Normalized Entropy  \cite{kapishnikov2019xrai}. Model: \textit{Xception} 
}
\label{Xception_f}
\end{figure*}

\begin{figure*}[!t]
     \centering
     \begin{subfigure}[b]{.9\textwidth}
         \centering
         \includegraphics[width=\textwidth]{figs/appendix_examples/distribution_folder/MobileNetV2.png}
     \end{subfigure}
\caption{Modified distribution of bokeh images over MS-SSIM and Normalized Entropy  \cite{kapishnikov2019xrai}. Model: \textit{MobileNetV2} 
}
\label{MobileNetV2_f}
\end{figure*}

\begin{figure*}[!t]
     \centering
     \begin{subfigure}[b]{.9\textwidth}
         \centering
         \includegraphics[width=\textwidth]{figs/appendix_examples/distribution_folder/VGG16.png}
     \end{subfigure}
\caption{Modified distribution of bokeh images over MS-SSIM and Normalized Entropy  \cite{kapishnikov2019xrai}. Model: \textit{VGG16} 
}
\label{vgg16_f}
\end{figure*}

\begin{figure*}[!t]
     \centering
     \begin{subfigure}[b]{.9\textwidth}
         \centering
         \includegraphics[width=\textwidth]{figs/appendix_examples/distribution_folder/VGG19.png}
     \end{subfigure}
\caption{Modified distribution of bokeh images over MS-SSIM and Normalized Entropy  \cite{kapishnikov2019xrai}. Model: \textit{VGG19} 
}
\label{vgg19_f}
\end{figure*}

\section{AIC and SIC with XRAI}
\cref{AUC_of_AIC} and \cref{AUC_of_SIC} show the result of AIC and SIC for all methods and its version with XRAI. Similarly, \cref{AUC_of_AIC_SSIM} and \cref{AUC_of_SIC_SSIM} show the result of AIC and SIC with MS-SSIM for all methods and its version with XRAI. 
\begin{table}[!t]
\centering
\resizebox{\columnwidth}{!}{
\begin{tabular}{|c||c|c||c|c||c|c||c|}
\hline
\multicolumn{8}{|c|}{\textbf{AUC of AIC}} \\
\hline
\textbf{Models}&\multicolumn{6}{|c|}{\textbf{IG-based Methods}}&{\textbf{Other}}\\
\cline{1-8}
&IG&+Ours&GIG&+Ours&BlurIG&+Ours&VG\\
\hline

\textit{DenseNet121}&.161&\textbf{.300}&.141&\textbf{.252}&.192&\textbf{.230}&.087\\
\textit{DenseNet169}&.160&\textbf{.288}&.154&\textbf{.254}&.181&\textbf{.216}&.089\\
\textit{DenseNet201}&.185&\textbf{.307}&.182&\textbf{.269}&.213&\textbf{.246}&.110\\
\textit{InceptionV3}&.203&\textbf{.343}&.189&\textbf{.338}&.266&\textbf{.301}&.127\\
\textit{MobileNetV2}&.098&\textbf{.233}&.114&\textbf{.204}&.145&\textbf{.197}&.068\\
\textit{ResNet50V2}&.162&\textbf{.253}&.162&\textbf{.248}&.189&\textbf{.210}&.108\\
\textit{ResNet101V2}&.177&\textbf{.268}&.163&\textbf{.253}&.198&\textbf{.215}&.116\\
\textit{ResNet151V2}&.186&\textbf{.281}&.165&\textbf{.258}&.205&\textbf{.229}&.112\\
\textit{VGG16}&.145&\textbf{.244}&.141&\textbf{.199}&.181&\textbf{.222}&.108\\
\textit{VGG19}&.153&\textbf{.263}&.150&\textbf{.219}&.204&\textbf{.240}&.117\\
\textit{Xception}&.238&\textbf{.404}&.239&\textbf{.381}&.309&\textbf{.355}&.174\\

\hline
&\multicolumn{7}{|c|}{\textbf{With XRAI}} \\
\hline
\textit{DenseNet121}&.438&\textbf{.479}&.460&\textbf{.460}&.437&\textbf{.452}&.434\\
\textit{DenseNet169}&.468&\textbf{.508}&.483&\textbf{.492}&.466&\textbf{.480}&.462\\
\textit{DenseNet201}&.439&\textbf{.476}&.460&\textbf{.468}&.442&\textbf{.461}&.449\\
\textit{InceptionV3}&.477&\textbf{.506}&.472&\textbf{.513}&.479&\textbf{.503}&.496\\
\textit{MobileNetV2}&.407&\textbf{.442}&.437&\textbf{.435}&.410&\textbf{.436}&.424\\
\textit{ResNet50V2}&.402&\textbf{.433}&.428&\textbf{.438}&.409&\textbf{.410}&.417\\
\textit{ResNet101V2}&.415&\textbf{.447}&.433&\textbf{.445}&.416&\textbf{.422}&.424\\
\textit{ResNet151V2}&.410&\textbf{.443}&.421&\textbf{.435}&.406&\textbf{.416}&.412\\
\textit{VGG16}&.393&\textbf{.423}&.422&\textbf{.418}&.402&\textbf{.413}&.396\\
\textit{VGG19}&.386&\textbf{.416}&.417&\textbf{.414}&.396&\textbf{.408}&.393\\
\textit{Xception}&.486&\textbf{.521}&.507&\textbf{.525}&.492&\textbf{.520}&.511\\

\hline
\end{tabular}}
\caption{AUC of AIC}
\label{AUC_of_AIC}
%\vspace{+2mm}
\end{table}


\begin{table}[!t]
\centering
\resizebox{\columnwidth}{!}{
\begin{tabular}{|c||c|c||c|c||c|c||c|}
\hline
\multicolumn{8}{|c|}{\textbf{AUC of SIC}} \\
\hline
\textbf{Models}&\multicolumn{6}{|c|}{\textbf{IG-based Methods}}&{\textbf{Other}}\\
\cline{1-8}
&IG&+Ours&GIG&+Ours&BlurIG&+Ours&VG\\
\hline
\textit{DenseNet121}&.054&\textbf{.228}&.036&\textbf{.157}&.085&\textbf{.134}&.015\\
\textit{DenseNet169}&.052&\textbf{.230}&.045&\textbf{.170}&.083&\textbf{.130}&.016\\
\textit{DenseNet201}&.068&\textbf{.241}&.058&\textbf{.183}&.109&\textbf{.155}&.019\\
\textit{InceptionV3}&.087&\textbf{.294}&.061&\textbf{.286}&.171&\textbf{.232}&.029\\
\textit{MobileNetV2}&.020&\textbf{.145}&.023&\textbf{.111}&.043&\textbf{.103}&.011\\
\textit{ResNet50V2}&.077&\textbf{.210}&.067&\textbf{.201}&.099&\textbf{.158}&.025\\
\textit{ResNet101V2}&.095&\textbf{.231}&.070&\textbf{.201}&.117&\textbf{.165}&.026\\
\textit{ResNet151V2}&.101&\textbf{.249}&.065&\textbf{.212}&.122&\textbf{.177}&.025\\
\textit{VGG16}&.046&\textbf{.166}&.039&\textbf{.104}&.082&\textbf{.141}&.021\\
\textit{VGG19}&.046&\textbf{.177}&.041&\textbf{.115}&.098&\textbf{.151}&.023\\
\textit{Xception}&.119&\textbf{.363}&.107&\textbf{.336}&.218&\textbf{.296}&.054\\
\hline
&\multicolumn{7}{|c|}{\textbf{With XRAI}} \\
\hline
\textit{DenseNet121}&.407&\textbf{.464}&.435&\textbf{.445}&.403&\textbf{.428}&.404\\
\textit{DenseNet169}&.450&\textbf{.496}&.465&\textbf{.475}&.439&\textbf{.458}&.435\\
\textit{DenseNet201}&.427&\textbf{.473}&.449&\textbf{.462}&.419&\textbf{.449}&.432\\
\textit{InceptionV3}&.450&\textbf{.493}&.449&\textbf{.499}&.441&\textbf{.481}&.477\\
\textit{MobileNetV2}&.351&\textbf{.398}&.391&\textbf{.394}&.353&\textbf{.393}&.374\\
\textit{ResNet50V2}&.401&\textbf{.439}&.430&\textbf{.445}&.404&\textbf{.412}&.418\\
\textit{ResNet101V2}&.424&\textbf{.463}&.445&\textbf{.464}&.419&\textbf{.428}&.433\\
\textit{ResNet151V2}&.413&\textbf{.453}&.423&\textbf{.445}&.401&\textbf{.424}&.414\\
\textit{VGG16}&.343&\textbf{.382}&.381&\textbf{.376}&.352&\textbf{.368}&.347\\
\textit{VGG19}&.337&\textbf{.376}&.374&\textbf{.373}&.347&\textbf{.362}&.344\\
\textit{Xception}&.458&\textbf{.502}&.486&\textbf{.508}&.465&\textbf{.503}&.488\\

\hline
\end{tabular}}
\caption{AUC of SIC}
\label{AUC_of_SIC}
%\vspace{+2mm}
\end{table}


\begin{table}[!t]
\centering
\resizebox{\columnwidth}{!}{
\begin{tabular}{|c||c|c||c|c||c|c||c|}
\hline
\multicolumn{8}{|c|}{\textbf{AUC of AIC with MS-SSIM}} \\
\hline
\textbf{Models}&\multicolumn{6}{|c|}{\textbf{IG-based Methods}}&{\textbf{Other}}\\
\cline{1-8}
&IG&+Ours&GIG&+Ours&BlurIG&+Ours&VG\\
\hline
\textit{DenseNet121}&.229&\textbf{.305}&.231&\textbf{.280}&.216&\textbf{.277}&.186\\
\textit{DenseNet169}&.241&\textbf{.314}&.249&\textbf{.297}&.218&\textbf{.289}&.205\\
\textit{DenseNet201}&.254&\textbf{.323}&.262&\textbf{.303}&.237&\textbf{.303}&.216\\
\textit{InceptionV3}&.264&\textbf{.333}&.268&\textbf{.333}&.264&\textbf{.323}&.228\\
\textit{MobileNetV2}&.179&\textbf{.259}&.197&\textbf{.238}&.186&\textbf{.241}&.150\\
\textit{ResNet50V2}&.225&\textbf{.277}&.239&\textbf{.274}&.209&\textbf{.260}&.198\\
\textit{ResNet101V2}&.235&\textbf{.284}&.243&\textbf{.277}&.215&\textbf{.265}&.206\\
\textit{ResNet151V2}&.247&\textbf{.302}&.250&\textbf{.292}&.227&\textbf{.284}&.212\\
\textit{VGG16}&.205&\textbf{.271}&.212&\textbf{.245}&.204&\textbf{.259}&.179\\
\textit{VGG19}&.211&\textbf{.275}&.220&\textbf{.252}&.214&\textbf{.266}&.188\\
\textit{Xception}&.281&\textbf{.362}&.293&\textbf{.356}&.284&\textbf{.345}&.254\\
\hline
&\multicolumn{7}{|c|}{\textbf{With XRAI}} \\
\hline
\textit{DenseNet121}&.342&\textbf{.376}&.360&\textbf{.367}&.336&\textbf{.369}&.351\\
\textit{DenseNet169}&.375&\textbf{.407}&.386&\textbf{.397}&.368&\textbf{.398}&.382\\
\textit{DenseNet201}&.354&\textbf{.388}&.370&\textbf{.380}&.355&\textbf{.387}&.370\\
\textit{InceptionV3}&.357&\textbf{.384}&.355&\textbf{.386}&.348&\textbf{.390}&.373\\
\textit{MobileNetV2}&.310&\textbf{.339}&.333&\textbf{.334}&.310&\textbf{.339}&.329\\
\textit{ResNet50V2}&.302&\textbf{.326}&.320&\textbf{.330}&.302&\textbf{.322}&.317\\
\textit{ResNet101V2}&.316&\textbf{.342}&.329&\textbf{.342}&.312&\textbf{.334}&.327\\
\textit{ResNet151V2}&.314&\textbf{.341}&.321&\textbf{.334}&.308&\textbf{.335}&.321\\
\textit{VGG16}&.314&\textbf{.339}&.334&\textbf{.334}&.319&\textbf{.336}&.319\\
\textit{VGG19}&.309&\textbf{.333}&.330&\textbf{.329}&.315&\textbf{.332}&.315\\
\textit{Xception}&.370&\textbf{.402}&.391&\textbf{.406}&.372&\textbf{.408}&.396\\
\hline
\end{tabular}}
\caption{AUC of SIC with MS-SSIM}
\label{AUC_of_AIC_SSIM}
%\vspace{+2mm}
\end{table}

\begin{table}[!t]
\centering
\resizebox{\columnwidth}{!}{
\begin{tabular}{|c||c|c||c|c||c|c||c|}
\hline
\multicolumn{8}{|c|}{\textbf{AUC of SIC with MS-SSIM}} \\
\hline
\textbf{Models}&\multicolumn{6}{|c|}{\textbf{IG-based Methods}}&{\textbf{Other}}\\
\cline{1-8}
&IG&+Ours&GIG&+Ours&BlurIG&+Ours&VG\\
\hline
\textit{DenseNet121}&.184&\textbf{.263}&.188&\textbf{.239}&.172&\textbf{.236}&.139\\
\textit{DenseNet169}&.205&\textbf{.282}&.214&\textbf{.263}&.182&\textbf{.256}&.166\\
\textit{DenseNet201}&.212&\textbf{.286}&.221&\textbf{.265}&.194&\textbf{.266}&.170\\
\textit{InceptionV3}&.211&\textbf{.287}&.215&\textbf{.285}&.214&\textbf{.276}&.179\\
\textit{MobileNetV2}&.126&\textbf{.204}&.144&\textbf{.187}&.130&\textbf{.188}&.096\\
\textit{ResNet50V2}&.196&\textbf{.254}&.213&\textbf{.250}&.177&\textbf{.236}&.167\\
\textit{ResNet101V2}&.210&\textbf{.265}&.221&\textbf{.256}&.188&\textbf{.244}&.180\\
\textit{ResNet151V2}&.221&\textbf{.282}&.227&\textbf{.270}&.197&\textbf{.261}&.186\\
\textit{VGG16}&.163&\textbf{.234}&.174&\textbf{.210}&.166&\textbf{.224}&.137\\
\textit{VGG19}&.173&\textbf{.240}&.186&\textbf{.219}&.177&\textbf{.233}&.149\\
\textit{Xception}&.223&\textbf{.312}&.233&\textbf{.304}&.229&\textbf{.293}&.194\\
\hline
&\multicolumn{7}{|c|}{\textbf{With XRAI}} \\
\hline
\textit{DenseNet121}&.290&\textbf{.332}&.309&\textbf{.324}&.282&\textbf{.324}&.306\\
\textit{DenseNet169}&.327&\textbf{.364}&.338&\textbf{.356}&.314&\textbf{.353}&.338\\
\textit{DenseNet201}&.311&\textbf{.349}&.326&\textbf{.345}&.301&\textbf{.350}&.333\\
\textit{InceptionV3}&.300&\textbf{.334}&.295&\textbf{.342}&.291&\textbf{.343}&.323\\
\textit{MobileNetV2}&.238&\textbf{.270}&.264&\textbf{.270}&.239&\textbf{.273}&.262\\
\textit{ResNet50V2}&.273&\textbf{.305}&.294&\textbf{.308}&.270&\textbf{.299}&.295\\
\textit{ResNet101V2}&.291&\textbf{.323}&.305&\textbf{.322}&.283&\textbf{.312}&.306\\
\textit{ResNet151V2}&.286&\textbf{.322}&.294&\textbf{.313}&.277&\textbf{.314}&.302\\
\textit{VGG16}&.256&\textbf{.285}&.280&\textbf{.280}&.260&\textbf{.284}&.264\\
\textit{VGG19}&.252&\textbf{.278}&.276&\textbf{.277}&.258&\textbf{.279}&.262\\
\textit{Xception}&.311&\textbf{.345}&.331&\textbf{.352}&.311&\textbf{.353}&.341\\
\hline
\end{tabular}}
\caption{AUC of SIC with MS-SSIM}
\label{AUC_of_SIC_SSIM}
%\vspace{+2mm}
\end{table}
%%%%%%%%% REFERENCES
\newpage
{\small
\bibliographystyle{ieee_fullname}
\bibliography{egbib}
}

\end{document}
