\begin{figure*}
    \centering
    \includegraphics[width=\linewidth]{fig/syn3x10.pdf}
    \vspace{-0.5cm}
    \caption{Visualizing HOI synthesis from our method and three
baselines~\cite{ldm,pix2pix,kingma2013auto} on HOI4D (left) and EPIC-KITCHEN dataset (right).}
    \label{fig:syn2d}
\vspace{-0.2cm}
 \end{figure*} 
\begin{table*}[!t]
    \caption{\textbf{Quantitative evaluation} using FID and KID for all methods at $256^2$ pixel resolution on \threedfront bedrooms, \threedfront living rooms, and \kitti.}
    \label{table:main}  
    \vskip 0.15in
    \begin{center}
    \begin{tabularx}{0.785\linewidth}{@{}lcccccccc}
        \toprule
        \multirow{2}{*}{Method} & \multirow{2}{*}{Representation} &\multicolumn{2}{c}{Bedrooms} & \multicolumn{2}{c}{Living Rooms} & \multicolumn{2}{c}{\kitti} \\
        \cmidrule(lr){3-4} \cmidrule(lr){5-6} \cmidrule(lr){7-8}
       & & FID ($\downarrow$) & KID ($\downarrow$) & FID ($\downarrow$) & KID ($\downarrow$) & FID ($\downarrow$) & KID ($\downarrow$) \\
        \midrule
        GIRAFFE & Pure MLP&{141.5} & {127.3}  & {155.7} & {157.5} & {189.0} & {238.3} \\
        GSN & 2D Floor Plan&{73.6} & {43.8}  & {175.4} & {164.9}  & {256.7} & {323.0} \\
        EG3D & Tri-plane&{49.0} & {35.7} & {90.9} & {84.3}  & {78.2} & {82.2} \\
        %Ours (uncond.) & 2D-3D Reshape&{38.3} & {29.6} & {-} & {-} & {-} & {-} & {-} & {-} \\
        \midrule
        %Ours w/ Tri-plane& Tri-plane&{-} & {-} & {-} & {-} & {-} & {-} & {-} & {-} \\
        Ours  & 2D-3D Extrusion&\textbf{28.5} & \textbf{21.3} & \textbf{40.3} & \textbf{34.5} & \textbf{65.6} & \textbf{70.5} \\

        \bottomrule
    \end{tabularx}
    \end{center}
    \vskip -0.1in
\end{table*}


We train our model on the contructed HO3Pairs dataset, evaluate it on the HOI4D~\cite{liu2022hoi4d} dataset and show zero-shot generalization to the EPIC-KITCHEN~\cite{epic} dataset.  
% 
We evaluate both the generated HOI images and the extracted 3D poses. 
%
For image synthesis, we compare with conditional image synthesis baselines and show that our method generates more plausible hands in interaction.
%
Beyond 2D HOI image synthesis, we compare the extracted 3D poses with prior works that directly predict 3D hand poses.
% and show that our method is favored by users on both datasets. 
%
%We further analyze the benefits from the proposed data augmentation to prevent overfitting to artifacts and the benefits from DDPM loss in image space to train LayoutNet.
% We also carry out ablation studies to demonstrate the contributions of the data augmentation process in Sec~\ref{sec:data} and the DDPM loss in Sec~\ref{sec:layout}.
%
Furthermore, we show several applications enabled by the proposed HOI synthesis method, including few-shot adaptation, image editing by layout, heatmap-guided prediction and integrating object affordance with the scene. 

% Furthermore, we show that the proposed method can be quickly adapted to hand-object-interaction of new categories with very few samples. 
%
% Lastly, we demonstrate that the proposed hand proxy provides an editing interface to control the interaction layout. The layout generation can be conditioned on additional constraints at test time, which enables reusing the heatmap prediction from prior works and generates consistent hand sizes across different objects in one scene. 

\noindent\textbf{Datasets}
% We evaluate on two egocentric datasets: HOI4D~\cite{liu2022hoi4d} and EPIC-KITCHEN~\cite{epic}. 
%
Instead of testing with inpainted object images, we evaluate our model on the real object-only images cropped from the frames without hands. The goal is to prevent models from cheating by overfitting to the inpainting artifacts, as justified in the ablations below.

% HOI4D dataset is an egocentric video dataset depicting human interacting with various objects under lab environment. It consists of xxxk  20-second short clips spanning xx common object categories. The object categories include large objects like chair, portable objects like bottle, articulated object like laptop. The dataset provides manual annotations of segmentation masks for active objects and hands, action labels, object class categories, object instance identity, and ground truth 3D hand poses. We train and evaluate on 10 categories where full annotations are released.
% We hold out 5 object instances in each categories for testing, and crop out the objects where their states are labeled as rest. In total this results in 126 test images.
The HOI4D dataset is an egocentric video dataset recording humans in a lab environment interacting with various objects such as kettles, bottles, laptops, \etc. The dataset provides manual annotations of hand and object masks, action labels, object categories, instance ID, and ground truth 3D hand poses. We train and evaluate on 10 categories where full annotations are released. For each category, we hold out 5 object instances for evaluation. 
% We crop these objects from frames where their states are labeled as rest (i.e., there is no hand in the frame). 
In total, we collect 126 testing images.

% EPIC-KITCHEN dataset displays more diverse and cluttered scenes. We construct our test set from the recently released subset VISOR~\cite{VISOR2022} which features hand-object-interaction videos. We randomly select 10 frames from each video clips  and detect objects~\cite{detectron} to crop and filter out the object crops with hands. In total, we collect 500 object-only images for testing.
The EPIC-KITCHEN dataset displays more diverse and cluttered scenes. We construct our test set by randomly selecting 10 frames from each video clip. We detect and crop out objects without hands~\cite{detectron2}. In total, we collect 500 object-only images for testing.


\subsection{Evaluating Image Synthesis}
\noindent\textbf{Evaluation Metrics. }
We evaluate HOI generation using three metrics. First, we report the FID score~\cite{Seitzer2020FID,heusel2017gans}, which is widely used for image synthesis that measures the distance between two image sets. We generate 10 samples for every input and calculate FID  with 1000 HOI images extracted from the test sets. 
%
% Besides the general image realism, we also evaluate the quality of interaction for the generated hand -- whether the generated hands make contact with the objects or not. We propose to measure the interaction quality by contact recall, the ratio of the generated hands are detected as "in-contact" state by an off-the-shelf hand detector~\cite{dandan}.
We further evaluate the physical feasibility of the generated hands by the contact recall metric --- it computes the ratio of the generated hands that are in the ``in-contact'' state by an off-the-shelf hand detector~\cite{shan2020understanding}.
%
We also carry out user studies to evaluate their perceptual plausibility. Specifically, we present two images from two randomly selected methods to users and ask them to select the more plausible one. We collect 200 (for HOI4D) and 240 (for EPIC-KITCHEN) answers and report the likelihood of the methods being chosen.

\noindent\textbf{Baselines. }
We compare our method with three strong image-conditional synthesis baselines. 
% 
1) \noindent\textit{Latent Diffusion Model (LDM)}~\cite{ldm} is one of the state-of-the-art generic image generation models that is pre-trained with large-scale image data.  We condition the model on the object image and finetune it on HO3Pair dataset. This baseline jointly generates both layout and appearance with one network. 
% 
2) \textit{Pix2Pix}~\cite{pix2pix} is commonly used for pose-conditioned human/hand synthesis~\cite{chan2019everybody,ganerated_hand}. We modify the model to condition on the generated layout masks that are predicted from our LayoutNet. 
% 
3) \textit{VAE}~\cite{kingma2013auto} is a widely applied generative model in recent affordance literature~\cite{fouhey2012people,li2019putting,PLACE:3DV:2020}. This baseline uses a VAE with ResNet~\cite{he2016deep} as backbone to predict a layout parameter. The layout is then passed to our ContentNet to generate images. 
% \end{enumerate}
 
 \noindent\textbf{Results.} We visualize the generated HOI images in Fig~\ref{fig:syn2d}. Pix2Pix typically lacks detailed finger articulation. While LDM and VAE generate more realistic hand articulations than Pix2Pix,  the generated hands sometimes do not make contact with the objects. The hand appearance near the contact region is less realistic. In some cases, LDM does not add hands at all to the given object images. In contrast, our model can generate hands with more plausible articulation and the synthesized contact regions are more realistic. 
This is consistent with the quantitative results in Tab~\ref{tab:syn2d}. While we perform comparably to the baselines in terms of the FID score, we achieve the best in terms of contact recall. The user study shows that our results are favored the most. This may indicate that humans perceive interaction quality as a more important factor than general image synthesis quality.

\noindent\textbf{Generalizing to EPIC-KITCHEN. }
Although our model is trained only on the HOI4D dataset with limited scenes and relatively clean backgrounds, our model can generalize to the EPIC-KITCHEN dataset without any finetuning. In Fig~\ref{fig:syn2d}, the model also generalizes to interact with unseen categories such as scissors and cabinet handles. Tab~\ref{tab:syn2d} reports similar trends: performing best in  contact recall, comparably well in image synthesis and is favored the most by users. 

\begin{figure*}
    \centering
    \includegraphics[width=\linewidth]{fig/3dpose_fig.pdf}
    \caption{Visualizing 3D affordance prediction from our method,
GANHand~\cite{ganhand} and diffusion model~\cite{ldm} that directly predicts 3D pose on HOI4D (left) and EPIC-KITCHEN dataset (right).}
\vspace{-0.5cm}
    \label{fig:3d}
\end{figure*}
\begin{table}[t]
\centering
\tablestyle{2pt}{1}
\caption{\textbf{Analysis of data augmentation}: contact recall (CR\%) and FID score on the real and the inpainted object image set of HOI4D and comparisons of ours with the ablations of excluding aggressive common data augmentation (CmnAug) or SDEdit~\cite{sdedit}.}
\label{tab:artifact}
\vspace{-0.3em}
\begin{tabular}{llcccc}
\toprule
        &  & \multicolumn{2}{c}{Real Obj Img} & \multicolumn{2}{c}{Inpainted Img}   \\
\cmidrule(l{2pt}r{2pt}){3-4} \cmidrule(l{2pt}r{2pt}){5-6}         
CmnAug & SDEdit& CR & FID & CR & FID   \\
\midrule
          &            &	39.37&	113.93	&	89.05	& 89.38 \\
\checkmark &            &	79.52&	99.12	&	93.81	& 89.01 \\
\checkmark & \checkmark & 87.14  & 	99.00	&	94.29	& 88.50 \\
\bottomrule
\end{tabular}
\vspace{-0.5cm}
\end{table}
\noindent\textbf{Ablation: Data Augmentation. } Tab \ref{tab:artifact} shows the benefits of data augmentation to prevent overfitting. Without any data augmentation, the model performs well on the inpainted object images but catastrophically fails on the real ones. When we add aggressive common data augmentations like Gaussian blur and Gaussian noise, the performance improves. Training on SDEdited images further boosts the performance. The results also justify the use of real object images as test set since evaluating on the inpainted object images may not reflect the real performance. 


\noindent\textbf{Ablation: LayoutNet Design. }
% \begin{table}[t]
\centering
\tablestyle{3pt}{1}
\caption{
\textbf{Analysis of LayoutNet design}: quantitative results of contact recall (CR\%) on the HOI4D dataset and comparisons to ours with the ablations that it is passed to the network without splatting (w/o 2D reasoning) and trains without $\mathcal{L}_{\text{mask}}$ (w/o 2D loss). }
\label{tab:layout}
\begin{tabular}{lcc}
\toprule
 & Contact Recall (\%)  \\
\midrule
Ours	        & 87.14 \\
w/o 2D loss     &	83.96 \\
w/o 2D reasoning & 78.10 \\
\bottomrule
\end{tabular}
\vspace{.5em}
\vspace{-1em}
\end{table}
We analyze the benefits from our LayoutNet design by reporting contact recall. The LayoutNet predicts more physically feasible hands by taking in the splatted layout masks instead of the 5-parameter layout vector (87.14\% vs 78.10\%). Moreover, the contact recall drops to 83.96\% when the diffusion loss in Sec~\ref{sec:layout} is removed, verifying its contribution to the LayoutNet.

\subsection{Evaluating Extracted 3D Hand Poses}
\label{sec:3d}
Thanks to the realism of the generated HOI images, 3D hand poses can be directly extracted from them by an off-the-shelf hand pose estimator~\cite{frankmocap}. We conduct a user study to compare the 3D poses extracted from our HOI images against methods that directly predict 3D pose from object images. We present the rendered hand meshes overlaid on the object images to users and are asked to select the more plausible one. In total, we collected 400 and 380 answers from users for HOI4D and EPIC-KITCHEN, respectively. 

% The 3D poses can be potentially transferred to an excutable actions for robot hands. 

\begin{table}[t]
\centering
\tablestyle{3pt}{1}
% \vspace{-0.5cm}
\caption{User study for 3D affordance prediction on HOI4D and EPIC-KITCHEN dataset. We compare our method with GANHand \cite{ganhand} and a diffusion model that directly predicts 3D poses.
\label{tab:3d}
}
% \vspace{-0.3em}
\begin{tabular}{l cc}
\toprule
% & \multicolumn{12}{c}{HOI4D} &  \multicolumn{3}{c}{EPIC-KITCHEN} \
 Method & HOI4D & EPIC  \\
% \shline
\midrule
GANHand  \cite{ganhand}  & 23.8 & 23.53 \\
3D Pose Diffusion &  27.9 & 34.1 \\
Ours & \textbf{48.2} & \textbf{42.4} \\
% Ours-GLIDE         &90.80	& 121.64	& 42.22 \\
\bottomrule
\end{tabular}
\vspace{-0.3cm}
\end{table}
\noindent\textbf{Baselines.} 
% While most prior works that output 3D hand poses take in a known object 3D meshes, recent works by Corona \etal (GANHand)~\cite{} can solve the same task as ours.
While most 3D hand pose generation works require 3D object meshes as inputs, a recent work by Corona \etal (GANHand)~\cite{ganhand} can hallucinate hand poses from an object image.
Specifically, they first map the object image to a grasp type~\cite{feix2015taxonomy} with the predefined coarse pose and then regress a refinement on top. We finetune their released model on the HO3Pairs datasets with the ground truth 3D hand poses.  We additionally implement a diffusion model baseline that sequentially diffuses 3D hand poses. The architecture is mostly the same as the LayoutNet but the diffused parameter is increased to 51 (48 for hand poses and 3 for scale and location) and the splatting function is replaced by the MANO~\cite{mano} layer that renders hand poses to image. See the appendix for  implementation details.

\noindent\textbf{Results. } As shown in Fig~\ref{fig:3d}, GANHand\cite{ganhand} predicts reasonable hand poses for some objects but fails when the grasp type is not correctly classified. The hand pose diffusion model sometimes generates infeasible hand poses like acute joint angles.  Our model is able to generate hand poses that are compatible with the objects. Furthermore, while previous methods typically assume right hands only, our model can automatically generate  both left and right hands by implicitly learning the correlation between approaching direction and hand sides.  The qualitative performance is also supported by the user study  in Tab~\ref{tab:3d}. 

% \subsection{Generalize better than directly predicting 3D pose??}
\subsection{Application} 
\label{sec:app}
We showcase several applications that are enabled by the proposed method for hand-object-image synthesis. 

\begin{figure*}[htbp]
\vspace{-15pt}
\centering
    \includegraphics[width=17.5cm]{figures/fewshot_new.pdf}
    \vspace{-2.5em}
    \caption{Results of few-shot learning based on supervised and self-supervised pre-training. The \textcolor[rgb]{0.4,0.8,0.4}{green} curves represent supervised pre-training and the \textcolor[rgb]{0,0,1}{blue} curves represent $\rho$MoCo self-supervised pre-training. We illustrate the results of TSM, 3D NonLocal, and VideoSwin for both pre-training methods. Additionally, we add the SOTA self-supervised pre-training method VideoMAE, represented by the \textcolor[rgb]{1,0,0}{red} curves, for comparison. It could be obvious that even the VideoMAE could lag a lot behind in the few-shot setting.}
    \vspace{-1 em}
    \label{fewshot}
\end{figure*}

\section{Few-shot Learning}
\label{sec:few-shot}

% We claim that few-shot learning in video action recognition is still a challenging task.
% Since the annotations of videos can be extremely expensive because a motion concept is much harder to recognize by human annotators compared with images, and in videos which require expertise knowledge, it will become even worse. 

Compared with standard finetuning where abundant annotations can be utilized, few-shot learning is of more practical significance since annotating massive amounts of videos is notoriously expensive. To extend the investigation mentioned in Section \ref{sec:finetune},  we thoroughly investigate the capability of the selected 6 models on BEAR under few-shot setting given both supervised and self-supervised pre-trained weights. Specifically, we consider (2,4,8,16)-shot settings, and for each setting, we randomly generate 3 splits and report the mean and standard deviation. 
% It should be noted that, in the same split, the training samples in a smaller shot are included in a bigger one, \eg the samples in 2-shot are also the samples in 4-shot. We expect this could provide a consistent observation  w.r.t. number of training samples and the performance. 
Due to space constraints, we only select TSM, 3D NonLocal, and Video Swin to represent each model type for illustration as they perform generally better.  Complete few-shot results and the training details are in \textcolor{blue}{Appendix~\ref{appendix_fewshot}}.

\vspace{-1em}
\paragraph{Model comparison.} 

The rankings of the six models in few-shot finetuning exhibit distinct variations compared to the standard finetuning. In contrast to the dominance of TSM in standard finetuning across both pre-training settings, the most effective models differ significantly across datasets in few-shot finetuning. Figure~\ref{fewshot} demonstrates that TSM no longer clearly outperforms other models in most datasets, and the two composite metrics (which are presented in the Supplementary due to space limitations) support this conclusion. Specifically, TSM and TimeSformer exhibit similar performance in supervised pre-training, whereas I3D and VideoSwin perform better in self-supervised learning. These findings further reveal the limitations of previous simple evaluation protocols, which may not provide a fair assessment of video models.
% such as the previous finetuning on a limited set of downstream datasets, which cannot provide a fair assessment. 
These results also confirm the necessity of BEAR, which emphasizes the importance of diverse downstream datasets and various settings for unbiased evaluation. 
% Furthermore, the performance of few-shot learning can be significantly affected by the number of classes in each dataset, as evidenced by the lower micro-average-precision compared to the macro-average-precision. This observation aligns with the fact that the total training data in few-shot learning has positive correlation to the number of classes.

\begin{adjustwidth}{-1.5em}{}
\begin{itemize}
    \item \emph{The ranking relations between models could exhibit differently between standard and few-shot finetuning even within the same datasets. This finding further emphasizes the importance of our proposed BEAR benchmark, which advocates for a comprehensive evaluation approach that considers both dataset diversity and finetuning settings.}
\end{itemize}
\end{adjustwidth}
\vspace{-1em}

\paragraph{Impact of viewpoint change} 
% As can be seen from the green curves in Figure~\ref{fewshot}, compared with the standard finetuning,
As in standard finetuning, viewpoint change also has a severe impact when it comes to few-shot learning.
Comparing the results in Figure \ref{fewshot} with those in Table \ref{tab:finetune}, we can see that
the few-shot learning performance decreases drastically in general, especially in datasets that have less in common with Kinetics-400, such as UAV-Human, which is constructed by videos captured from unmanned aerial vehicles, FineGym, which contains fine-grained gym-related videos, and PETRAW and MISAW, which are simulated medical operations in the 1st person view. Conversely, in datasets that are more similar to Kinetics-400, these performance gaps are notably reduced. For example, even the 2-shot performance on Mini-HACS and MOD20 can reach approximately 60\% and 85\%, and the models achieve satisfying performance on the 16-shot setting on COIN.
In previous works, the homogeneity of the pre-training and downstream data hindered the timely identification of such phenomena in few-shot learning. Our investigation highlights the challenge of few-shot learning and underscores the importance of bridging the gap (as aforementioned, introducing extra data, such as Ego4D) between pre-training and the target data. 

Moreover, in the few-shot setting, self-supervised pre-training is more susceptible to viewpoint change. In challenging datasets such as UAV-Human and WLASL, few-shot learning can hardly obtain satisfying results based on self-supervised pre-trained weights, while in the 16-shot setting, supervised pre-training could provide comparable performance compared with standard finetuning. Similarly, in MOD20, the performance experiences a sharp decline in few-shot settings with self-supervised pre-training, while supervised pre-trained TSN and TSM can achieve accuracy exceeding 90\% in the 16-shot.
% Moreover, the curves of different datasets show different trends w.r.t. the shot numbers. For instance, in Mini-HACS and MOD20, the performance gaps between 2-shot and 16-shot remain relatively indistinct, while in UAV-Human and FineGym, the curves rapidly go downward when the shot number decrease. Coincidently, these datasets are also influenced by viewpoint change as aforementioned. This indicates that, in few-shot finetuning, domain gap between pre-training and target data can also determine the sensitivity towards shot number.

\begin{adjustwidth}{-1.5em}{}
\begin{itemize}
    \item \emph{
    % Few-shot finetuning remains a significant challenge in real-world scenarios. However, the performance drop is primarily influenced by the domain gap between pre-training and target data. Previous studies overlooked this phenomenon due to the lack of diverse downstream datasets, but we highlight the importance of incorporating extra pre-training data to improve the performance across a wide range of downstream datasets.
    Few-shot finetuning remains a significant challenge in real-world scenarios.  The performance drops dramatically compared to standard finetuning especially when there is a large domain gap between pre-training and target data. However, when downstream datasets are similar to source data, the performance drop could be mitigated.
    }
    \item \emph{In few-shot learning, self-supervised pre-training is more vulnerable to viewpoint shift, while supervised pre-trained models can achieve favorable performance compared with standard finetuning on the 16-shot setting.}
\end{itemize}
\end{adjustwidth}
\vspace{-1em}


\paragraph{Self-supervised vs. supervised pre-training.} 
Comparing the blue curves to the green curves in Figure~\ref{fewshot}, we can see that self-supervised pre-training is generally less effective than supervised pre-training, which is consistent with the conclusion in Sec.~\ref{sec:finetune}. The performance gaps are pronounced in gesture datasets and are less significant in Mini-Sports1M, ToyotaSmarthome, etc. The performance gap is also different across different models. The largest gap appears in TSN and TimeSformer (the complete results are provided in \textcolor{blue}{Appendix Table~\ref{tab:sup res1}-~\ref{tab:ssl res3}}).
One reason for the poor performance of self-supervised learning may be the limitation of $\rho$MoCo. Therefore,
to consolidate our conclusion,
% on few-shot learning based on the difference between supervised and self-supervised pretrained models, 
we further consider VideoMAE~\cite{tong2022videomae}, which is the SoTA self-supervised method and has demonstrated even better performance than supervised models on multiple datasets. Here, we use the officially released VideoMAE ViT-B model, which achieves 81.5\% Top-1 accuracy on Kinetics-400. However, comparing the results with our 6 supervised pre-trained models in Figure \ref{fewshot} (red vs. green curves), we show that VideoMAE could only be comparable with the best supervised pre-trained models in less than half of the datasets. 
% This reveals that self-supervised pre-training still lags a lot behind supervised pre-training in few-shot learning with diverse domains. 
% Therefore,  we claim that few-shot learning in video action recognition is still a challenging task that requires further effort for more effective framework.

\begin{adjustwidth}{-1.5em}{}
\begin{itemize}
    \item \emph{Supervised pre-training shows consistent advantages over self-supervised ones in few-shot finetuning. Even the SoTA VideoMAE can hardly outperform simple supervised pre-trained models in diverse domains.}
\end{itemize}
\end{adjustwidth}

% This indicates that, besides Kinetics 400, a large-scale  dataset that focuses on specific viewpoints is required for pre-training. For instance, for egocentric pre-training, the recent Ego4D~\cite{grauman2022ego4d} could be a possible choice. 
\noindent\textbf{Few-shot Adaptation. } In Tab~\ref{tab:fewshot}, we show that our model can be quickly adapted to a new HOI category with as few as 32 training samples. We initialize both LayoutNet and ContentNet from our HOI4D-pretrained checkpoints and compare it with the baseline model that was pre-trained for inpainting on a large-scale image dataset~\cite{ldm}. We finetune both models on 32 samples from three novel categories in HOI4D and test with novel instances. The baseline model adapts quickly on some classes, justifying our reasons to finetune our model from them---generic large-scale image pretraining indeed already learns good priors of HOI. Furthermore, our HOI synthesis model performs even better  than the baseline.


\begin{figure}
    \centering
    \includegraphics[width=\linewidth]{fig/interpolation_fig.pdf}
    \caption{\textbf{Layout Editing}: Visualizing HOI synthesis when the conditioned layouts gradually change location and orientation. }
    \label{fig:interpolate}
    \vspace{-0.1cm}
\end{figure}
\noindent\textbf{Layout Editing. } The layout representation allows users to edit and control the generated hand's structure. As shown in Fig~\ref{fig:interpolate}, while we gradually change the layout's location and orientation, the synthesized hand's appearance changes accordingly. As the approaching direction to the mug changes from right to left, the synthesized fingers change accordingly from pinching around the handle to a wider grip around the mug's body. 


\begin{figure}
    \centering
    % \vspace{-1.5em}
    \includegraphics[width=\linewidth]{fig/heatmap_fig.pdf}
    \caption{\textbf{Heatmap-guided synthesis: } Given a heatmap, LayoutNet is guided to generate layout at the sampled location, from which HOI images are synthesized and 3D poses are extracted.}
    \label{fig:heatmap}
    \vspace{-0.3cm}
\end{figure}
\noindent\textbf{Heatmap-Guided Synthesis.} As shown in Sec~\ref{sec:layout}, our synthesized HOI images can be conditioned on a specified location without any retraining. 
% We hijack the specified location after each layout diffusion step, guiding the generated layout to center at the given location.  
This not only allows users to edit with just keypoints, but also enables our model to utilize contact heatmap predictions from prior works~\cite{nagarajan2019iccv:hotspots,fang2018demo2vec}. In Fig~\ref{fig:heatmap}, we sample points from the heatmaps and conditionally generate layouts and HOI images which further specifies \textit{how} to interact at the sampled location. 


\begin{figure}
    \centering
    % \vspace{-0.5cm}
    \includegraphics[width=\linewidth]{fig/scene-3d.png}
    \caption{\textbf{Scene-level Integration: } Given a cluttered scene, we detect each object and synthesize its interactions individually. Each object's layout scale is guided to appear in the same size when transferred back to the scene.  }
    \label{fig:scene}
    \vspace{-0.5cm}
\end{figure}
\noindent\textbf{Integration to scene.} We integrate our object-centric HOI synthesis to scene-level affordance prediction. While the layout size is predicted relative to each object, hands for different objects in one scene should exhibit consistent scale. To do so, we first specify one shared hand size for each scene and calculate the corresponding relative sizes in each crops (we assume objects at similar depth and thus sizes can be transformed by crop sizes, although more comprehensive view conversions can be used). The LayoutNet is conditioned to generate these specified sizes with guided generation techniques (Sec~\ref{sec:layout}). Fig~\ref{fig:scene} shows the extracted hand meshes from each crops transferred back to the scene. 
