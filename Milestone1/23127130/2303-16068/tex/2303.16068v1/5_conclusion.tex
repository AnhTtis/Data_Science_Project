\section{Conclusion and Future Work}
\label{sec:conclusion}

In this work, we formulated the preference shifts from a causal view and inspected the underlying causal relations from the perspective of multiple environments. Based on the causal relations, we proposed the CDR framework, which captures the preference shifts across environments via a temporal VAE and learns the sparse structure from user preference to interactions. In particular, CDR leverages two learnable matrices to disentangle user representations and formulate the sparse structure. We optimized CDR by a multi-objective loss with variance and sparsity regularization. 
During the training and inference stages, we could flexibly adjust the number of environments to balance the learning of shifted and invariant preference on different datasets. Extensive experiments validate the effectiveness of CDR in capturing preference shifts and achieving superior generalization performance than the baselines. Furthermore, the in-depth analysis demonstrates the rationality of using multiple environments, the sparse structure, and the multi-objective loss.


This work attempts to learn user preference from multiple environments for handling user preference shifts. In this light, there are many promising directions in future work. In particular, 1) as discussed in Section~\ref{sec:environment}, we equally split user interactions into $T$ environments to simplify the data pre-processing. Nevertheless, it can be improved by developing more effective but complex methods to divide environments, \eg clustering interactions according to the time interval. 
2) It is meaningful to discover more fine-grained causal relations in Figure \ref{fig:causal_graph}, such as the mutual impact between the category-level user preference. Besides, we might utilize observed user features (\eg age) to study more fine-grained relations between $E_t$ and $Z_t$. And 3) it is non-trivial to incorporate item features for the disentangled representation learning, which may help CDR to distinguish the category-level user preference. 



% 3) CDR can be extended to more 

% 1. more methods to divide environments.
% 2. more fine-grained causal relations and incorporate more info user feature scocial media
% 3. extend CDR to more sequential prediction tasks, such as time-series prediction.

