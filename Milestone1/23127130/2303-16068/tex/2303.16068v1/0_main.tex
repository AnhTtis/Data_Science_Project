
\documentclass[acmsmall]{acmart}
\AtBeginDocument{%
  \providecommand\BibTeX{{%
    \normalfont B\kern-0.5em{\scshape i\kern-0.25em b}\kern-0.8em\TeX}}}

\setcopyright{acmcopyright}
\copyrightyear{2021}
\acmYear{2021}
\acmDOI{10.nn/nnnnnnn.nnnnnnn}


%%
%% These commands are for a JOURNAL article.
\acmJournal{JACM}
\acmVolume{1}
\acmNumber{1}
\acmArticle{111}
\acmMonth{1}

%%
%% Submission ID.
%% Use this when submitting an article to a sponsored event. You'll
%% receive a unique submission ID from the organizers
%% of the event, and this ID should be used as the parameter to this command.
%%\acmSubmissionID{123-A56-BU3}

%%
%% The majority of ACM publications use numbered citations and
%% references.  The command \citestyle{authoryear} switches to the
%% "author year" style.
%%
%% If you are preparing content for an event
%% sponsored by ACM SIGGRAPH, you must use the "author year" style of
%% citations and references.
%% Uncommenting
%% the next command will enable that style.
%%\citestyle{acmauthoryear}
\usepackage{booktabs}
\usepackage{array}
\usepackage{balance} 
\usepackage{lipsum}
\usepackage{multirow}
\usepackage{booktabs}
\usepackage{amsmath}
\usepackage{subfigure}
% \usepackage{subcaption}
\usepackage{algorithm}  
\usepackage{algorithmicx}  
\usepackage{algpseudocode}  
\usepackage{amsmath}  
\usepackage{enumitem}
\usepackage{tabularx}
\usepackage[utf8]{inputenc}
\usepackage[english]{babel}
\usepackage{amsthm}
\usepackage{bm}
\newcommand{\ie}{\emph{i.e., }}
\newcommand{\eg}{\emph{e.g., }}
\newcommand{\etal}{\emph{et al. }}
\newcommand{\st}{\emph{s.t. }}
\newcommand{\etc}{\emph{etc.}}
\newcommand{\wrt}{\emph{w.r.t. }}
\newcommand{\cf}{\emph{cf. }}
\newcommand{\aka}{\emph{a.k.a. }}

% for algorithm
\newlength\myindent
\setlength\myindent{2em}
\newcommand\bindent{%
    \begingroup
    \setlength{\itemindent}{\myindent}
    \addtolength{\algorithmicindent}{\myindent}
}
\newcommand\eindent{\endgroup}

\newtheorem{theorem}{Theorem}[section]
\newtheorem{proposition}{Proposition}[section]
% \newtheorem{lemma}[theorem]{Lemma}

\usepackage[normalem]{ulem}
\useunder{\uline}{\ul}{}

\floatname{algorithm}{Algorithm}
\renewcommand{\algorithmicrequire}{\textbf{Input:}} 
\renewcommand{\algorithmicensure}{\textbf{Output:}}

% \newcommand{\wenjie}[1]{\textcolor{blue}{[Wenjie: {#1}]}}
\newcommand{\edit}[1]{\textcolor{blue}{#1}}
\newcommand{\editsec}[1]{\textcolor{orange}{#1}}

\clubpenalty=10000
\widowpenalty = 10000
\hyphenpenalty=7000
\tolerance=7000
% \usepackage[algoruled]{algorithm2e}
% \usepackage{algorithmic}
% \setlength{\interspacetitleruled}{8pt}
% \usepackage{listings}
% \usepackage{fancyvrb}
% \fvset{fontsize=\small}



%%
%% end of the preamble, start of the body of the document source.
\begin{document}

% \fancyhead{}
% \title{Invariant Structure Learning for Out-of-Distribution Recommendation}
% \title{Invariant Structure Learning for Generalizable Recommendation}
\title{Causal Disentangled Recommendation Against User Preference Shifts}

% \subtitle{Submission ID: 470}

\author{Wenjie Wang}
\email{wenjiewang96@gmail.com}
\affiliation{%
  \institution{National University of Singapore}
  \country{Singapore}
}
\author{Xinyu Lin}
\email{xylin1028@gmail.com}
\affiliation{%
  \institution{National University of Singapore}
  \country{Singapore}
}
\author{Liuhui Wang}
\email{wangliuhui0401@pku.edu.cn}
\affiliation{%
  \institution{Peking University}
  \country{China}
}
\author{Fuli Feng}
\authornote{Corresponding author: Fuli Feng (fulifeng93@gmail.com).}
\email{fulifeng93@gmail.com}
\affiliation{%
  \institution{University of Science and Technology of China}
  \country{China}
}

\author{Yunshan Ma}
\email{yunshan.ma@u.nus.edu}
\affiliation{%
  \institution{National University of Singapore}
  \country{Singapore}
}

\author{Tat-Seng Chua}
\email{dcscts@nus.edu.sg}
\affiliation{%
  \institution{National University of Singapore}
  \country{Singapore}
}

% \thanks{$*$ Corresponding author: Fuli Feng (fulifeng93@gmail.com). This research is supported by the Sea-NExT Joint Lab, and the National Natural Science Foundation of China (62121002).}

\begin{abstract}

% Recommender systems easily face the issue of user preference shifts. User representations will become out-of-date and lead to inappropriate recommendations if user preference has shifted over time.
% % To solve the issue, existing work focuses on learning the representations of invariant preference or recognizing the pattern of shifts. 
% To solve the issue, existing work focuses on learning robust representations or predicting the shifting pattern. 
% % There lacks an overall consideration from both perspectives, \ie ingeniously modeling both the invariant and shifted parts of user preference to achieve better generalization. 
% % In this work, we analyze and address this issue from a causal view. 
% There lacks a comprehensive view to discover the underlying reasons for user preference shifts. 
% To understand the preference shift, we abstract a causal graph to describe the generation procedure of user interaction sequences. 
% Assuming user preference is stable within a short period, we abstract the interaction sequence as a set of chronological environments. From the causal graph, we find that the changes of some unobserved factors (\eg becoming pregnant) cause preference shifts between environments. Besides, the fine-grained user preference over categories sparsely affects the interactions with different items.
% % Inspired by the causal graph, our key considerations to solving this issue lie in modeling the interaction generation procedure by:
% % 1) learning the invariant preference; 2) capturing the preference shifts across environments; and 3) learning robust disentangled representations for interaction generation. 
% Inspired by the causal graph, our key considerations to handle preference shifts lie in modeling the interaction generation procedure by: 1) capturing the preference shifts across environments for accurate preference prediction, and 2) disentangling the sparse influence from user preference to interactions for accurate effect estimation of preference.
% To this end, we propose a Causal Disentangled Recommendation (CDR) framework, which captures preference shifts via a temporal variational autoencoder and learns the sparse influence from multiple environments. 
% Specifically, an encoder is adopted to infer the unobserved factors from user interactions while a decoder is to model the interaction generation process. 
% Besides, we introduce two learnable matrices to disentangle the sparse influence from user preference to interactions. Lastly, we devise a multi-objective loss to optimize CDR. 
% Extensive experiments on three datasets show the superiority of CDR on enhancing the generalization ability under user preference shifts. 

Recommender systems easily face the issue of user preference shifts. User representations will become out-of-date and lead to inappropriate recommendations if user preference has shifted over time. To solve the issue, existing work focuses on learning robust representations or predicting the shifting pattern. There lacks a comprehensive view to discover the underlying reasons for user preference shifts. To understand the preference shift, we abstract a causal graph to describe the generation procedure of user interaction sequences. Assuming user preference is stable within a short period, we abstract the interaction sequence as a set of chronological environments. From the causal graph, we find that the changes of some unobserved factors (\eg becoming pregnant) cause preference shifts between environments. Besides, the fine-grained user preference over categories sparsely affects the interactions with different items. Inspired by the causal graph, our key considerations to handle preference shifts lie in modeling the interaction generation procedure by: 1) capturing the preference shifts across environments for accurate preference prediction, and 2) disentangling the sparse influence from user preference to interactions for accurate effect estimation of preference. To this end, we propose a Causal Disentangled Recommendation (CDR) framework, which captures preference shifts via a temporal variational autoencoder and learns the sparse influence from multiple environments. Specifically, an encoder is adopted to infer the unobserved factors from user interactions while a decoder is to model the interaction generation process. Besides, we introduce two learnable matrices to disentangle the sparse influence from user preference to interactions. Lastly, we devise a multi-objective loss to optimize CDR. Extensive experiments on three datasets show the superiority of CDR in enhancing the generalization ability under user preference shifts. 

\end{abstract}

%%
%% The code below is generated by the tool at http://dl.acm.org/ccs.cfm.
%% Please copy and paste the code instead of the example below.
%%
\begin{CCSXML}
<ccs2012>
% <concept>
% <concept_id>10002951.10003260.10003261.10003271</concept_id>
% <concept_desc>Information systems~Personalization</concept_desc>
% <concept_significance>500</concept_significance>
% </concept>
<concept>
<concept_id>10002951.10003317.10003347.10003350</concept_id>
<concept_desc>Information systems~Recommender systems</concept_desc>
<concept_significance>500</concept_significance>
</concept>
</ccs2012>
\end{CCSXML}
% \ccsdesc[500]{Information systems~Personalization}
\ccsdesc[500]{Information systems~Recommender systems}

\keywords{Causal Disentangled Recommendation, Preference Shifts, Generalizable Recommendation, Out-of-Distribution Generalization}


\maketitle

\section{Introduction}
\label{sec:intro}
\begin{figure}[t]
\begin{center}
    \includegraphics[width=1\linewidth]{figures/teaser.pdf}
\end{center}
\vspace{-0.1in}
\caption{\textbf{{\em Foggy} vs {\em Clear} NeRF.} Our \ournerf gets rid of reconstruction errors manifested as foggy ``floaters" in the density volume without additional input or significant computational overhead. 
%
Below are density profiles along a given ray before and after our geometry correction procedure, where we discard density peaks corresponding to floaters.
}
\label{fig:teaser}
\vspace{-0.2in}
\end{figure}



%The emergence of 
Neural Radiance Fields (NeRFs)~\cite{mildenhall2020nerf}  %and its variants 
have made revolutionary contributions in %photo-realistic 
novel view synthesis~\cite{barron2021mip,barron2022mip}, 
autonomous driving~\cite{rematas2022urban,tancik2022block}, digital human~\cite{hong2022headnerf,zhao2022humannerf}, and 3D content generation~\cite{eg3d,poole2022dreamfusion,lin2022magic3d}.
%by leveraging a multi-layer perceptron (MLP) to implicitly model the mapping from input 5D coordinates (i.e., 3D coordinates $\mathbf{x} = (x,y,z)$ and 2D viewing directions $\mathbf{d}=(\theta,\phi)$) to volume density $\sigma$ and view-dependent emitted radiance color $\mathbf{c} = (r,g,b)$. 
%
%They then use traditional volume rendering mechanisms on the obtained continuous 5D function (i.e., MLP) to generate novel views. 
To date, unfortunately, most NeRF-based methods encounter challenges when tackling large-scale cluttered scenes (e.g., Fig.~\ref{fig:teaser}):
\begin{enumerate}[leftmargin=0.16in, topsep=2pt,itemsep=-1ex,partopsep=1ex,parsep=1ex]
\item Input observations used for NeRF are often too sparse  compared to forward-facing or synthetic looking-inward scenes;
%\item Recovering fine-grained objects within a large volume is challenging for NeRF; %in capturing details accurately.
\item View-dependent visual effects give rise to ambiguity, resulting in a ``foggy" density field as shown in Fig.~\ref{fig:teaser}. 
%
Such artifacts are particularly pronounced in indoor scenes strewn with view-dependent appearances, such as specular highlights, glossy surface reflections from man-made objects. 
\end{enumerate}

Despite attempts to enhance NeRF's rendering quality given suboptimal input, such as using 3D conical frustums~\cite{barron2021mip,barron2022mip}, physically-grounded augmentations~\cite{chen2022aug}, and misalignment correction~\cite{jiang2022alignerf},  these challenges have yet to be fully resolved.
%
Depth supervision~\cite{deng2022depth, wei2021nerfingmvs} or proxy geometry~\cite{xu2021scalable,wu2022scalable} images can help alleviate the challenges in handling large-scale with sparse input, at the expense of %but they come at the cost of requiring 
expensive pre-processing or additional input.
%
Another line of work~\cite{wang2021neus, oechsle2021unisurf, wang2022neuris} achieves better reconstruction of surface geometry by using signed distances instead of volume density as scene representation. However, they sacrifice the ability to synthesize photo-realistic novel views.

%We observe that NeRF has been suffering from foggy ``floater" artifacts in large-scale cluttered scenes.
%
%Such artifacts are particularly pronounced in indoor scenes strewn with view-dependent appearances from man-made objects. 
%
To address the above issues, we propose an extension to NeRF, dubbed as {\bf \ournerf}, which enforces effective {\em appearance} and {\em geometry} constraints conducive to accurate colors and 3D densities estimation. We believe \ournerf can contribute beyond novel view synthesis, such as NeRF object detection~\cite{hu2022nerf}, NeRF object segmentation~\cite{zhi2021place, liu2022unsupervised, fan2022nerf,ren2022neural}, and NeRF registration~\cite{goli2022nerf2nerf}, where the rooms for improvement are substantial if more accurate color and density estimation are available.

Correspondingly, there are two steps in \ournerf. First, for appearance correction, the view-independent and view-dependent color components are predicted from the underlying 3D scene, which is combined to produce the final color estimation (Fig.~\ref{fig:toaster}).
%
The view-independent component (diffuse color and shading) captures the overall scene color, while the view-dependent component (highlights or reflections) captures color variations due to changes in viewing angle.
%
\ournerf then discards these view-dependent appearances in the training views to prevent them from interfering with the density estimation.
%
Second, a simple and effective geometry correction procedure will be performed to further eliminate the foggy ``floaters" or density errors. This geometry correction procedure is based on an assumption in line with traditional ray tracing in computer graphics.
\begin{comment}
% xh: basically copying method
On the other hand, ClearNeRF performs a geometric correction procedure performed on each traced ray during inference to refine the density estimation and better tackle the floater artifacts. 
%
The geometry correction procedure assumes that there should only be one salient peak along each traced ray during NeRF inference. 
Only the salient peak closest to the ray origin (the camera center) corresponds to  true geometry while the others will be manifested as foggy floaters hovering in the density volume. 
%
This assumption is in line with traditional ray tracing in computer graphics where in the absence of noise, only one intersection per ray should be returned to indicate the closest ray-object intersection.
%
\end{comment}
%%%%%%%%%%%
%As shown in Fig.~\ref{fig:teaser}, when reconstructing an indoor scene with sparse input and highly view-dependent objects, NeRF produces severe floating artifacts due to its attempt to explain view-dependent appearances.
%
Experiments verify that our proposed \ournerf can effectively get rid of floater artifacts without additional input.% or significant computational overhead. 


In summary, our contributions include the following:
\begin{itemize}[leftmargin=0.16in, topsep=2pt,itemsep=-1ex,partopsep=1ex,parsep=1ex]
    \item We propose a concise method for decomposing view-independent and view-dependent appearance during NeRF training and eliminate the interference of view-dependent appearance.
    \item We propose a geometric correction procedure performed on each traced ray during inference to refine the density estimation and better tackle the floater artifacts.
    \item Extensive experiments and ablations verify the effectiveness of our core designs and results in improvements over the vanilla NeRF and other state-of-the-art alternatives.
    %without additional computational resources or other inputs.
\end{itemize}




\section{Method}
In this section, we present the causal mechanism and task formulation of recommendation with consideration of preference shifts in Section \ref{sec:problem}. 
% inspect the causal relations behind user preference shifts via a causal graph, and then formulate generalizable recommendation 
Thereafter, we detail the proposed CDR framework in Section \ref{sec:method}. 

\subsection{Recommendation with Preference Shifts}
\label{sec:problem}


\begin{figure}[t]
\setlength{\abovecaptionskip}{0.2cm}
\setlength{\belowcaptionskip}{0cm}
\centering
\includegraphics[scale=0.7]{figures/causal_graph.pdf}
\caption{{Causal graph behind the interaction generation procedure with multiple environments. We assume that the observed user preference is affected by some hidden user features, and various user preference sparsely controls the interactions with different items.}}
\label{fig:causal_graph}
\end{figure}

Existing recommender models usually rely on the IID interaction distributions from training to testing stages. Without considering user preference shifts over time, 
these recommender models will encounter significant performance drop in OOD environments. 
To improve the generalization ability, we build recommender models with considering the preference shifts. 
In this subsection, we first scrutinize the causal relations regarding user preference shifts, and then formulate the task of generalizable recommendation to evaluate the generalization ability under preference shifts. 

% \vspace{3pt}
% % \noindent$\bullet\quad$\textbf{Causal View of Preference Shifts.}
% \noindent\textbf{Causal View of Preference Shifts.}
\subsubsection{\textbf{Causal View of Preference Shifts}}
We present the causal relations in Figure \ref{fig:causal_graph} and explain its rationality as follows:
\begin{itemize}[leftmargin=*]
    \item $E_t$ denotes unobserved user features (\eg pregnancy) or environmental factors (\eg hot events) in the environment $t$; $Z_t$ and $X_t$ represent the user preference and interactions, respectively. Because of the privacy restriction~\cite{wang2018toward}, we seldom utilize user features for recommendation, and thus we ignore the modeling of observed user features in Figure \ref{fig:causal_graph}, which can be easily incorporated as the input of the CDR framework if necessary. 
    
    \item $E_t \rightarrow Z_t$: user features and various environmental factors affect user preference.
    
    \item $Z_t \rightarrow X_t$: user interactions are determined by current user preference. {In particular, $Z_t$ covers the preference over multiple item categories (\eg seafood and toy). Some factors in $Z_t$ may represent the preference over an item category (\eg seafood), which sparsely affects a category of interactions as shown in Figure \ref{fig:causal_graph}}.
    
    \item $Z_{t-1} \rightarrow Z_t$: the user preference $Z_t$ in the environment $t$ is updated from previous $Z_{t-1}$, which exhibits the preference shifts over time. {From the causal graph, we find that various factors in $E_t$ can affect user preference $Z_t$ and cause the preference shifts $Z_{t-1} \rightarrow Z_t$, leading to the variation of user interaction distributions.
    % Such user preference shifts naturally exist in the real world, and thus we should capture the shifts for better recommendation instead of solving it. 
    Besides, the preference shifts between environments only influence partial interactions due to the sparse influence from $Z_t$ to $X_t$.}
    {\item $E_{t-1} \dashrightarrow E_t$ and $X_{t-1} \dashrightarrow Z_t$: $E_{t-1}$ might affect $E_{t}$ because user features might have conditional relations, \eg pregnancy $\rightarrow$ having child. Besides, user preference $Z_t$ can be influenced by previous interactions $X_{t-1}$. However, these conditional relations are not easy to be inferred from pure interactions, and the effects of these conditional relations on $Z_t$ and $X_t$ are relatively weaker than $(E_t, Z_{t-1})\rightarrow Z_t$ and $Z_t \rightarrow X_t$. As such, we omit the modeling of $E_{t-1} \dashrightarrow E_t$ and $X_{t-1} \dashrightarrow Z_t$ in this work to pursue a simple model with fewer parameters. Empirical evidence in Section~\ref{sec:condition_rel} also validates the superiority of our choice.}
\end{itemize}




% \vspace{3pt}
% % \noindent$\bullet\quad$\textbf{Generalizable Recommendation.}
% \noindent\textbf{Generalizable Recommendation.}
\subsubsection{\textbf{Task Formulation}}\label{sec:task}
To evaluate the generalization ability under preference shifts, we formulate the task of generalizable recommendation. Formally, we utilize $u \in \{1,2,...,U\}$, $i\in \{1,2,...,I\}$, and $t\in \{1,2,...,T\}$ to index the user, item, and environment, respectively. The interactions of user $u$ in $T$ environments are denoted as $\bm{x}_{1:T}$\footnote{For notation brevity, we omit the subscript $u$ for $\bm{x}_{1:T}$ and $\bm{z}_{1:T}$ of user $u$.}, where $\bm{x}_{t}\in \{0,1\}^I$ is a multi-hot vector, and $x_{t,i}$ implies that user $u$ likes item $i$ ($x_{t,i}=1$) or not ($x_{t,i}=0$). 
Generally, given the observed $\bm{x}_{1:T}$ of user $u$, \textbf{generalizable recommendation} aims to capture the hidden preference shifts in $\bm{z}_{1:T}$ and estimates the latest user preference $\bm{z}_T$. 

% generate satisfying recommendations. 
% even if $\bm{z}_T$ is shifted from $\bm{z}_{1:T-1}$.


\vspace{3pt}
\noindent$\bullet\quad$\textbf{Environment division.}
We can divide the environments by time, for instance, equally splitting the user interaction sequence into $T$ pieces, or clustering adjacent interactions according to the time interval. In this work, we choose the first one to simplify the data pre-processing.

\vspace{3pt}
\noindent$\bullet\quad$\textbf{{Inference for future environments.}}
{To evaluate the generalization ability of recommender models, we can utilize the interactions in the environment $T+1$ for testing. To infer the interaction probability in this unknown environment, we consider three strategies:
1) using the latest user preference $\bm{z}_T$ for prediction; 2) uniformly averaging the predictions in $T$ training environments; 3) considering the average user features $\bm{e}_{T+1}=\frac{1}{T}\sum^T_{t=1}\bm{e}_t$ and $\bm{z}_T$ to predict $\bm{z}_{T+1}$, and then using $\bm{z}_{T+1}$ for interaction prediction. Because the testing environment is unknown, these inference strategies inevitably make some assumptions. 
The first strategy requires the small preference shifts from environment $T$ to $T+1$. Meanwhile, the second and third strategies need the average over $T$ training environments, losing some temporal shifting patterns. In practice, we set the first strategy as the default due to its better performance on real-world datasets (refer to Section~\ref{sec:infer_unk}).
}


\vspace{3pt}
\noindent$\bullet\quad$\textbf{Difference from sequential recommendation.}
The main difference between generalizable and sequential recommendations is that generalizable recommendation emphasizes the preference shifts across environments and the invariant preference within an environment. Moreover, generalizable recommendation focuses on the predictions of multiple interactions in an OOD environment, which differs from the next-item prediction in sequential recommendation. 

\subsection{CDR Framework}
\label{sec:method}


% \subsection{Invariant Structure Learning}
In this subsection, we present the CDR framework to model the interaction generation procedure under multiple environments. In particular, we utilize a novel temporal VAE to capture the preference shifts ($Z_{t-1}\rightarrow Z_{t}$) and conduct sparse structure learning to disentangle the sparse influence from user preference to interactions ($Z_{t}\rightarrow X_t$). 

We construct the recommender model by following the causal relations in Figure \ref{fig:causal_graph}. Specifically, for each user $u$ in the environment $t$, we first sample a $K$-dimensional latent representation $\bm{e}_t$ from the standard Gaussian prior $\mathcal{N}\left(0, \mathbf{I}_{K}\right)$~\cite{liang2018variational, yang2021causalvae}, where the covariance $\mathbf{I}_{K}$ is an identity matrix. We then obtain user preference $\bm{z}_t \in \mathbb{R}^{H}$ based on $\bm{e}_t$ and the previous $\bm{z}_{t-1}$. Thereafter, $\bm{z}_t$ is used to predict the interaction probability over $I$ items and the historical interactions $\bm{x}_t\in \mathbb{R}^{I}$ are assumed to be drawn from the interaction probability distribution. In this work, we assume that $\bm{z}_t$ and $\bm{x}_t$ follow the factorized Gaussian and multinomial priors due to their superiority shown in previous work~\cite{liang2018variational, ma2019learning}.
Formally,
\begin{equation}
\label{eqn:all_prior}
\left\{
\begin{aligned}
& \bm{e}_t \sim \mathcal{N}\left(0, \mathbf{I}_{K}\right), \\
& \bm{z}_t \sim \mathcal{N}\left(\bm{\mu}_{\theta_1}(\bm{e}_t, \bm{z}_{t-1}), \text{diag}\{\bm{\sigma}^2_{\theta_1}(\bm{e}_t, \bm{z}_{t-1})\}\right), \\
& \bm{x}_t \sim \text{Mult}\left(N_t, \pi\left(f_{\theta_2}(\bm{z}_t)\right)\right). \\
\end{aligned}
\right.
\end{equation}
Specifically, we explain the generative process in Eq. (\ref{eqn:all_prior}), which is consistent with the causal relations in Figure \ref{fig:causal_graph}: 
\begin{itemize}[leftmargin=*]
    \item $(\bm{e}_t, \bm{z}_{t-1}) \rightarrow \bm{z}_t$: $\bm{\mu}_{\theta_1}(\bm{e}_t, \bm{z}_{t-1})$ and $\bm{\sigma}^2_{\theta_1}(\bm{e}_t, \bm{z}_{t-1})$ denote the \textit{mean} and \textit{diagonal covariance} of the Gaussian distribution of $\bm{z}_t$, which are estimated from $\bm{e}_t$ and $\bm{z}_{t-1}$ via a network $f_{\theta_1}(\cdot)$.
    \item $\bm{z}_t \rightarrow \bm{x}_t$: $\bm{x}_t$ is sampled from a multinomial distribution affected by $\bm{z}_t$, where $N_t=\sum_{i=1}^{I}x_{t,i}$ represents the interaction number of user $u$ in the environment $t$, $\pi(\cdot)$ is the \textit{softmax} function, and the network $f_{\theta_2}(\bm{z}_t)$ transforms $\bm{z}_t$ to produce the interaction probability over $I$ items. 
\end{itemize}

To train the recommender model, we aim to optimize the parameters $\{\theta_1, \theta_2\}$ by maximizing the generative probability of observed user interactions $\bm{x}_{1:T}$ in $T$ environments. {Formally, following~\cite{chung2015recurrent}, we can factorize the joint distribution $p(\bm{x}_{1:T})$ and maximize the log-likelihood as follows:} 
\begin{equation}
\label{eqn:log_likelihood}
% \small
\begin{aligned}
{\log p(\bm{x}_{1:T})} & = {\log \int p(\bm{x}_{1:T}|\bm{e}_{1:T})p(\bm{e}_{1:T})d\bm{e}_{1:T}} \\
                    &= {\log \int \prod_{t=1}^T p(\bm{x}_t|\bm{x}_{1:t-1},\bm{e}_{1:t})p(\bm{e}_{1:T})d\bm{e_{1:T}}},
\end{aligned}
\end{equation}
{where $p(\bm{x}_t|\bm{x}_{1:t-1},\bm{e}_{1:t})$ aligns with the generation procedure in Eq. (\ref{eqn:all_prior}) and we will further factorize it with $\bm{z}_{1:t}$ in the decoding process \ie Eq. (\ref{eqn:decoder}).
Nevertheless, maximizing Eq. (\ref{eqn:log_likelihood}) is intractable because it involves the integral over unobserved $\bm{e}_{1:T}$. 
To solve the problem, we embrace \textit{variational inference}~\cite{liang2018variational} to approximate $\log p(\bm{x}_{1:T})$ by using a variational distribution $q(\bm{e}_{1:T}|\cdot)$. Formally,}
\begin{subequations}
\label{eqn:ELBO}
% \small
\begin{align}
{\log p(\bm{x}_{1:T})} &= {\log \int \prod_{t=1}^T         p(\bm{x}_t|\bm{x}_{1:t-1},\bm{e}_{1:t})p(\bm{e}_{1:T})\frac{q(\bm{e}_{1:T}|\cdot)}{q(\bm{e}_{1:T}|\cdot)}d\bm{e_{1:T}}}\\
                    &\geq {\mathbb{E}_{q(\bm{e}_{1:T}|\cdot)} \left[\log \frac{\prod_{t=1}^Tp(\bm{x}_t|\bm{x}_{1:t-1},\bm{e}_{1:t})p(\bm{e}_{1:T})}{q(\bm{e}_{1:T}|\cdot)}\right] \quad (\text{ELBO})}\\ 
                    &= {\mathbb{E}_{q(\bm{e}_{1:T}|\cdot)}\left[\sum_{t=1}^T\left(\log p(\bm{x}_t|\bm{x}_{1:t-1},\bm{e}_{1:t})-\text{KL}\left[q(\bm{e}_t|\cdot)\|p(\bm{e}_{t})\right]\right)\right]},
\end{align}
\end{subequations}
where variational inference introduces the Evidence Lower BOund (ELBO) of Eq. (\ref{eqn:ELBO}a) by using $q(\bm{e}_{1:T}|\cdot)=\prod_{t=1}^Tq(\bm{e}_t|\cdot)$. Meanwhile, the first term in Eq. (\ref{eqn:ELBO}c) represents the probability of collecting observed $\bm{x}_t$ conditioned on $\bm{x}_{1:t-1}$ and $\bm{e}_{1:t}$ while the second term denotes the Kullback-Leibler (KL) divergence between the variational distribution $q(\bm{e}_t|\cdot)$ and the prior of $\bm{e}_t$. By maximizing the ELBO in Eq. (\ref{eqn:ELBO}c), we are able to increase the log-likelihood $\log p(\bm{x}_{1:T})$. 
Note that we avoid factorizing {$p(\bm{x}_{t}|\bm{x}_{1:t-1},\bm{e}_{1:t})$} with $\bm{z}_{1:t}$ in Eq. (\ref{eqn:ELBO}), and then we do not estimate the distribution of unobserved $\bm{z}_t$ and $\bm{e}_t$ simultaneously by variational inference. This is because we choose the alternative Monte Carlo (MC) sampling to efficiently approximate the posterior distribution of $\bm{z}_t$~\cite{chen2012monte}. {MC sampling constructs a random sampling of $\bm{z}_t$ (\ie draw samples from $p(\bm{z}_t|\bm{e}_t,\bm{z}_{t-1})$) to estimate its distribution, which avoids unnecessary prior hypothesis over the mean and covariance of $\bm{z}_t$ (see Eq. (\ref{eqn:decoder})).}

So far, the key of calculating the ELBO in Eq. (\ref{eqn:ELBO}c) lies in estimating $q(\bm{e}_t|\cdot)$ and {$p(\bm{x}_t|\bm{x}_{1:t-1},\bm{e}_{1:t})$}, which can be obtained by the encoder and decoder networks, respectively. We present the intuitive illustration of CDR with the encoder and decoder networks in Figure \ref{fig:CDR}.




\begin{figure}[t]
\setlength{\abovecaptionskip}{0.2cm}
\setlength{\belowcaptionskip}{0cm}
\centering
\includegraphics[scale=0.6]{figures/CDR.pdf}
\caption{Illustration of the CDR framework, where the encoder network predicts the hidden user features $\bm{e}_t$ in the environment $t$, and then the decoder reconstructs the interaction generation procedure from $\bm{e}_t$ and $\bm{z}_{t-1}$ to the interaction probability $f_{\theta_2}(\bm{z}_t)$. The entire encoder-decoder process repeats T times while $\bm{W}_z$ and $\bm{W}_x$ are shared across $T$ environments.} % , consisting of the encoder and decoder networks
\label{fig:CDR}
\end{figure}

\subsubsection{\textbf{Encoder Network}}

To estimate $q(\bm{e}_t|\cdot)$, we incorporate an encoder network $g_{\phi}(\cdot)$, which predicts $\bm{e}_t$ by the user interaction $\bm{x}_t$. The underlying motivation is that unobserved factors (\eg income) can be inferred from users' behaviors (\eg purchasing expensive products). In particular, 
\begin{equation}
\label{eqn:encoder}
\begin{aligned}
q(\bm{e}_t|\cdot) = q(\bm{e}_t|\bm{x}_t) = \mathcal{N}\left(\bm{e}_t; \bm{\mu}_{\phi}(\bm{x}_t), \text{diag}\{\bm{\sigma}^2_{\phi}(\bm{x}_t)\}\right),
\end{aligned}
\end{equation}
where $\bm{\mu}_{\phi}(\bm{x}_t)$ and $\bm{\sigma}^2_{\phi}(\bm{x}_t)$ denote the mean and diagonal covariance of $\bm{e}_t$, respectively. They are estimated by the encoder network $g_\phi(\cdot)$ parameterized by $\phi$. Formally, we have $g_{\phi}(\bm{x}_t) = [\bm{\mu}_{\phi}(\bm{x}_t), \bm{\sigma}_{\phi}(\bm{x}_t)]\in \mathbb{R}^{2K}$. In this work, we instantiate $g_\phi(\cdot)$ by a Multi-Layer Perceptron (MLP), which outputs the Gaussian parameters of $\bm{e}_t$. Note that the encoder ignores the temporal interaction sequence in $\bm{x}_t$ and fairly encodes every interaction. 



\subsubsection{\textbf{Decoder Network}}
We factorize {$p(\bm{x}_t|\bm{x}_{1:t-1},\bm{e}_{1:t})$} by following the causal relations in the interaction generation process:
\begin{equation}
\label{eqn:decoder}
% \small
\begin{aligned}
% p(\bm{x}_t|\bm{e}_t) &= \int p(\bm{x}_t|\bm{z}_t)p(\bm{z}_t|\bm{e}_t, \bm{z}_{t-1})p(\bm{z}_{t-1}|\cdot)d\bm{z}_t d\bm{z}_{t-1},
{p(\bm{x}_t|\bm{x}_{1:t-1},\bm{e}_{1:t})} &= {\int p(\bm{x}_t|\bm{z}_t)\prod_{a=1}^tp(\bm{z}_a|\bm{z}_{a-1},\bm{e}_a)d\bm{z}_{1:t}},
\end{aligned}
\end{equation}
{where $p(\bm{z}_a|\bm{z}_{a-1},\bm{e}_a)$ denotes the probability distribution of the user preference $\bm{z}_a$ in the environment $a$; and when $a=1$, $\bm{z}_{0}$ is set as the constant vector $\bm{0}$. Besides, to approximate the distribution of $\bm{z}_{t}$, we use MC sampling~\cite{chen2012monte} to draw samples from $p(\bm{z}_t|\bm{z}_{t-1}, \bm{e}_t)$. Then we can calculate $p(\bm{x}_t|\bm{x}_{1:t-1},\bm{e}_{1:t})$ based on $p(\bm{x}_t|\bm{z}_t)$ while marginalizing over $\bm{z}_{1:t}$ via the samples from MC sampling. 
To iteratively calculate $p(\bm{z}_t|\bm{z}_{t-1}, \bm{e}_t)$, we adopt an MLP model $f_{\theta_1}(\cdot)$ to output the $\bm{\mu}_{\theta_1}(\cdot)$ and $\bm{\sigma}_{\theta_1}(\cdot)$ of $\bm{z}_t$. Formally, we have $f_{\theta_1}(\bm{z}_{t-1}, \bm{e}_t) = \left[\bm{\mu}_{\theta_1}(\bm{z}_{t-1}, \bm{e}_t), \bm{\sigma}_{\theta_1}(\bm{z}_{t-1}, \bm{e}_t)\right] \in \mathbb{R}^{2H}$. Thereafter, the remaining challenge is estimating $p(\bm{x}_t|\bm{z}_t)$ in Eq. (\ref{eqn:decoder}).} 






\vspace{10pt}
% \noindent$\bullet\quad$\textbf{Estimation of $p(\bm{x}_t|\bm{z}_t)$.}
\noindent$\bullet\quad$\textbf{Sparse structure learning.} 
To estimate $p(\bm{x}_t|\bm{z}_t)$, we incorporate $f_{\theta_2}(\cdot)$ to transform $\bm{z}_t$ into the interaction probability over $I$ items. However, to align with the causal relations in Figure \ref{fig:causal_graph} and learn the sparse influence from user preference to interactions, we do not simply use an MLP model for the implementation of $f_{\theta_2}(\cdot)$. We instead resort to \textit{sparse structure learning} in multiple environments, which aims to discover a sparse structure from user preference representations to interactions and requires the structure is robust across all the environments with distribution shifts. Consequently, 1) the sparse structure learned from multiple environments instead of one environment will encode the robust relations between user representations and interactions, which are likely to be reliable in future environments with preference shifts; and 2) if partial user preference has shifted, only a subset of interactions are affected due to the sparse structure. Such characteristics will improve the generalization ability of CDR under preference shifts. 

% Following~\cite{ma2019learning}, \edit{we assume that users have the category-level preference over items, and thus we disentangle the representations of $\bm{z}_t\in \mathbb{R}^{H}$ into category-level preference.}
Following~\cite{ma2019learning}, {the user preference representation $\bm{z}_t\in \mathbb{R}^{H}$ can cover the preference over multiple item categories, and we disentangle  $\bm{z}_t$ into several category-specific preference representations.}
Specifically, to implement the sparse structure, we introduce a matrix $\bm{W}_z \in \mathbb{R}^{H\times C}$ to factorize the user representation into the preference over $C$ item categories. In particular, ${W}_z[h,c]\in \bm{W}_z$ denotes the probability of the $h$-th factor in $\bm{z}_t$ belonging to the preference over the $c$-th category. 
Correspondingly, we leverage a matrix $\bm{W}_x \in \mathbb{R}^{I\times C}$ to classify items into $C$ categories. Inspired by~\cite{yamada2020feature, liu2021heterogeneous}, we draw $\bm{W}_z$ and $\bm{W}_x$ from the clipped Gaussian distributions parameterized by $\bm{\alpha}\in \mathbb{R}^{H\times C}$ and $\bm{\beta}\in \mathbb{R}^{I\times C}$, respectively. Formally, for each ${W}_z[h,c]\in \bm{W}_z$ and ${W}_x[i,c]\in \bm{W}_x$, we have 
\begin{equation}
\label{eqn:W_z_W_x}
% \small
\left\{
\begin{aligned}
{W}_z[h,c] &= \min\left(\max\left(\alpha_{h,c} + \epsilon, 0\right), 1\right), \\
{W}_x[i,c] &= \min\left(\max\left(\beta_{i,c} + \epsilon, 0\right), 1\right), \\
\end{aligned}
\right .
\end{equation}
where the noise $\epsilon$ is drawn from $\mathcal{N}\left(0, \sigma_{\epsilon}^2\right)$. We clip the values of $\bm{W}_z$ and $\bm{W}_x$ into $[0,1]$ to ensure a valid range for the probabilities. Besides, to encourage each factor or item belonging to one category, we add a softmax function at the dimension of $C$ categories in $\bm{W}_z$ and $\bm{W}_x$. 
As illustrated in Figure \ref{fig:theta_2}, we then implement $f_{\theta_2}(\cdot)$ to estimate the parameters of $\bm{x}_t$ in Eq. (\ref{eqn:all_prior}) by
\begin{equation}
\label{eqn:f_theta_2}
% \small
\begin{aligned}
f_{\theta_2}(\bm{z}_t) = \sum_{c=1}^C {W}_x[:,c] \odot f_{\gamma}({W}_z[:,c] \odot \bm{z}_t),
\end{aligned}
\end{equation}
where $\theta_2=\{\bm{\alpha}, \bm{\beta}, \gamma\}$, $\odot$ denotes the element-wise multiplication, and $f_{\gamma}(\cdot)$ can be any function transforming $\bm{z}_t$ to the interaction probability distribution over $I$ items. Following~\cite{liang2018variational}, we implement $f_{\gamma}(\cdot)$ by an MLP model.

\begin{figure}[t]
\setlength{\abovecaptionskip}{0.2cm}
% \setlength{\belowcaptionskip}{0cm}
\centering
\includegraphics[scale=0.8]{figures/theta_2.pdf}
\caption{Illustration of the calculation of $f_{\theta_2}(\bm{z}_t)$ in Eq. (\ref{eqn:f_theta_2}). Similar to masking mechanisms, $\bm{W}_z$ and $\bm{W}_x$ disentangle the user preference representations, leading to sparse connection from user preference to interactions. Note that we simplify $\bm{W}_z$ and $\bm{W}_x$ as discrete matrices with $\{0,1\}$ for better understanding.}
\label{fig:theta_2}
% \vspace{-0.2cm}
\end{figure}


\vspace{10pt}
\noindent$\bullet\quad$\textbf{Likelihood estimation.} 
As shown in Figure~\ref{fig:CDR}, given the user interactions $\bm{x}_{1:T}$ of user $u$, we feed them into the encoder network to sample $\bm{e}_{1:T}$, and then iteratively pass $\bm{e}_{1:T}$ to the decoder network to obtain the parameters of the multinomial distribution for $\bm{x}_{1:T}$ (\ie $f_{\theta_2}(\bm{z}_{1:T})$). Thereafter, {the log-likelihood $\log p(\bm{x}_t|\bm{z}_{t})$ can be calculated by} 
\begin{equation}
\label{eqn:likelihood}
\begin{aligned}
{\log p(\bm{x}_t|\bm{z}_{t})} &\overset{c}{=} \sum_{i=1}^{I}x_{t,i}\log \pi_i\left(f_{\theta_2}(\bm{z}_{t})\right),
\end{aligned}
\end{equation}
% where $x_{t,i} \in \{0,1\}$ indicates whether user $u$ has interacted with item $i$ in the environment $t$ ($x_{t,i}=1$) or not ($x_{t,i}=0$). 
where $x_{t,i} \in \{0,1\}$ indicates whether user $u$ has interacted with item $i$ in the environment $t$ or not. 
Besides, the softmax function $\pi(\cdot)$ is used to normalize $f_{\theta_2}(\cdot)$ and $\pi_i(f_{\theta_2}(\cdot))$ denotes the normalized prediction score for item $i$. {Intuitively, the log-likelihood $p(\bm{x}_t|\bm{z}_{t})$ estimates the probability of drawing observed $\bm{x}_t$ from the multinomial distribution by sampling $N_t$ times, where $N_t$ is the interaction number of user $u$ in the environment $t$.}
% Furthermore, here $p(\bm{x}_t|\bm{z}_{t})$ is an approximate estimation of $p(\bm{x}_t|\bm{x}_{1:t-1},\bm{e}_{1:t})$ by using MC sampling to marginalize over $\bm{z}_{1:t}$.
% , \ie $N_t = \sum_{i=1}^I x_{t,i}$


% \begin{algorithm}
% \DontPrintSemicolon \KwIn{Click matrix $\mbX \in \mathbb{R}^{U \times I}$}
% Randomly initialize $\theta$, $\phi$\;
% \While{not converged}{
%   Sample a batch of users $\mathcal{U}$\;
%   \ForAll{$u\in\mathcal{U}$}{
%     Sample $\mb\epsilon\sim \cN(0, \mbI_K)$ and compute $\mbz_u$ via reparametrization trick\;
%     Compute noisy gradient $\nabla_\theta \cL$ and $\nabla_\phi \cL$ with $\mbz_u$\;
%   }
%   Average noisy gradients from batch\;
%   Update $\theta$ and $\phi$ by taking stochastic gradient steps\;
% }
% \Return{$\theta$, $\phi$}\;
% \caption{{\sc VAE-SGD} Training collaborative filtering \gls{VAE} with stochastic gradient descent.}
% \label{alg:vae}
% \end{algorithm}

\begin{algorithm}[t]
	\caption{Training of CDR under Multiple Environments}  
	\label{algo:training}
	\begin{algorithmic}[1]
		\Require $X_{1:T}$ of all $U$ users; $g_\phi(\cdot)$, $f_{\theta_1}(\cdot)$, and $f_{\theta_2}(\cdot)$ with initialized $\phi$, $\theta_1$, and $\theta_2$, respectively.
		\While{\textit{not converged}}
		    \State Sample a batch of users $\mathcal{U}$
		    \ForAll{$u \in \mathcal{U}$}
		        \ForAll{$t \in \{1,2,...,T\}$}
		            \State Sample $\bm{e}_t$ by feeding $\bm{x}_t$ into $g_\phi(\bm{x}_t)$;
    		        \State Sample $\bm{z}_t$ by feeding $\bm{z}_{t-1}$ and $\bm{e}_t$ into $f_{\theta_1}(\bm{z}_{t-1}, \bm{e}_t)$;
    		        \State Calculate $f_{\theta_2}(\bm{z}_t)$ via Eq. (\ref{eqn:f_theta_2});
    		        \State Obtain the probability of drawing $\bm{x}_t$ by Eq. (\ref{eqn:likelihood});
    		        \State Calculate the gradients \wrt the loss in Eq. (\ref{eqn:overall_loss});
                \EndFor
		    \EndFor
		    \State Average the gradients over $|\mathcal{U}|$ users and $T$ environments;
		    \State Update $\phi$, $\theta_1$, and $\theta_2$ via Adam;
		\EndWhile
		\Ensure $g_\phi(\cdot)$, $f_{\theta_1}(\cdot)$, and $f_{\theta_2}(\cdot)$.
	\end{algorithmic}
\end{algorithm}
\vspace{-0.1cm}
% \setlength{\textfloatsep}{0.2cm}


\subsubsection{\textbf{CDR Optimization}}
We maximize the ELBO to increase the log-likelihood in Eq. (\ref{eqn:log_likelihood}) by optimizing the parameters (\ie $\phi$ and $\theta=\{\theta_1, \theta_2\} $) in CDR. The parameters are updated by stochastic gradient descent. However, we conduct the sampling of $\bm{e}_t$ in Eq. (\ref{eqn:encoder}) and $\bm{z}_t$ in Eq. (\ref{eqn:decoder}), which prevents the back-propagation of gradients. To solve this problem, we utilize the \textit{reparameterization trick}~\cite{Kingma2014Auto, liang2018variational}. Besides, we leverage the \textit{KL annealing trick}~\cite{liang2018variational} to control the effect of the KL divergence, which introduces an additional hyper-parameter $\lambda_1$ into Eq. (\ref{eqn:ELBO}c). To summarize, the optimization objective for user $u$ is to minimize the following loss:
\begin{equation}
\label{eqn:rec_loss}
\notag
\small
\begin{aligned}
{\mathcal{L}^u} = {-\mathbb{E}_{q_{\phi}(\bm{e}_{1:T}|\cdot)} \left[\sum_{t=1}^T\left(\log p(\bm{x}_t|\bm{x}_{1:t-1},\bm{e}_{1:t})-\lambda_1\text{KL}\left[q(\bm{e}_t|\bm{x}_t)\|p(\bm{e}_t)\right]\right)\right]},
\end{aligned}
\end{equation}
{which becomes negative timestamp-wise ELBO~\cite{chung2015recurrent} over the $T$ environments.} 

\vspace{10pt}
\noindent$\bullet\quad$\textbf{Sparsity and variance regularization.}
In addition to the ELBO objective, we additionally consider two regularization terms: 1) the sparsity of $\bm{W}_z$ and $\bm{W}_x$, and 2) the variance of the gradients across $T$ environments. 
In Eq. (\ref{eqn:f_theta_2}), we expect that the structure implemented by $\bm{W}_z$ and $\bm{W}_x$ is sparse because the sparse connection between user representations and interactions is more robust under preference shifts. Therefore, we introduce an $L_0$ regularization term $\|\bm{W}_z\|_0 + \|\bm{W}_x\|_0$ to restrict the number of non-zero values. 

As to the variance regularization, it can facilitate the sparse structure learning across multiple environments. Training over multiple environments easily leads to imbalanced optimization: the performance in some environments is good while in other environments has inferior results. Consequently, disentangled preference representations via $\bm{W}_z$ and $\bm{W}_x$ might not be reliable across multiple environments. 
% probably causing unreasonable disentanglement. 
As such, we incorporate the variance penalty regularizer used in invariant learning~\cite{koyama2021invariance, liu2021heterogeneous}, which regulates the variance of the gradients under $T$ environments. Specifically, we calculate the variance regularization for user $u$ by $\sum_{t=1}^T\left\| \nabla_\theta \mathcal{L}^u_t - \nabla_\theta \mathcal{L}^u\right\|^2$,
% \begin{equation}
% \label{eqn:variance}
% \begin{aligned}
% % \text{trace}\left(\text{Var}_{1:T}(\nabla_\theta \mathcal{L}^u)\right) 
% \sum_{t=1}^T\left\| \nabla_\theta \mathcal{L}^u_t - \nabla_\theta \mathcal{L}^u\right\|^2,
% \end{aligned}
% \end{equation}
where $\mathcal{L}^u_t$ is the optimization loss for the environment $t$ in Eq. (\ref{eqn:rec_loss}), $\nabla_\theta$ denotes the gradients \wrt the learnable parameters $\theta$, and $\nabla_\theta \mathcal{L}^u$ represents the average gradients over $T$ environments.

Intuitively, the variance regularizer will restrict the gradient difference among $T$ environments, and thus update the parameters $\theta$ synchronously for multiple environments. This will alleviate the problem that the parameters are unfairly optimized to improve the performance of few environments~\cite{liu2021heterogeneous}. To sum up, we have the final optimization loss for user $u$ as follows:
\begin{equation}
\label{eqn:overall_loss}
% \notag
\begin{aligned}
\mathcal{L}^u + \lambda_2 \cdot (\|\bm{W}_z\|_0 + \|\bm{W}_x\|_0) + \lambda_3 \cdot \sum_{t=1}^T\left\| \nabla_\theta \mathcal{L}^u_t - \nabla_\theta \mathcal{L}^u\right\|^2,
\end{aligned}
\end{equation}
where two hyper-parameters $\lambda_2$ and $\lambda_3$ control the strength of sparsity and variance regularization terms, respectively. 

% \vspace{10pt}
% \noindent$\bullet\quad$\textbf{Summary.}




% \vspace{5pt}
% \noindent$\bullet\quad$\textbf{ Penalty}


% \vspace{5pt}
% \noindent$\bullet\quad$\textbf{Environment division.}
\subsubsection{\textbf{Environment division}}\label{sec:environment}
For the temporal interaction sequence of user $u$, we split it into $T$ pieces according to the equal interaction number in every environment. 
The choice of $T$ is essential because it balances the learning of shifted and invariant preference. CDR only considers the cross-environment preference shifts and assumes the intra-environment preference is invariant by ignoring the temporal information of interactions. 
Therefore, a larger $T$ will expose more sequential information to CDR. 
% because CDR only considers the temporal changes between environments and ignores the chronological order of interactions to capture the invariant preference 
% within an environment (\cf Eq. (\ref{eqn:encoder}) and (\ref{eqn:decoder})). 
Nevertheless, the large $T$ value will increase the sparsity of interactions in each environment, hurting the learning of the encoder and decoder networks. To alleviate the dilemma, we choose a relatively small $T$ during training to ensure the interaction density of each environment. Once the parameters are well learned, we adopt a larger $T$ to fully utilize the sequential information in the inference stage. 


\subsubsection{\textbf{Summary}}\label{sec:summary}
The detailed training procedure can be found in Algorithm~\ref{algo:training}.
To train the encoder and decoder networks in CDR, we divide the training interactions into multiple environments and utilize them to minimize the loss function in Eq. (\ref{eqn:overall_loss}) over all users. 
During the inference stage, we use the latest user preference $\bm{z}_T$ to calculate $f_{\theta_2}(\bm{z}_T)$ for the ranking of item candidates, and then recommend top-ranked items to each user.

To summarize, the encoder network infers unobserved $E_t$ from users' interactions. Thereafter, the decoder network leverages the inferred $E_t$ to iteratively update $Z_t$ for better preference estimation. Besides, the decoder network conducts sparse structure learning to model the sparse influence from $Z_t$ to $X_t$ for better effect estimation of user preference. As compared to traditional VAE-based methods~\cite{liang2018variational, wang2021personalized, xia2021Collaborative}, CDR is more robust in OOD environments because it constructs the encoder and decoder networks by following the causal relations in Figure \ref{fig:causal_graph}. 
Besides, thanks to modeling causal relations, CDR supports the intervention over the causal graph. {As illustrated in Section~\ref{sec:case_do_E}, we can estimate the counterfactual user preference $Z_t$ and the corresponding recommendations by intervening on $E_t$, \ie changing $E_t={\bm{e}}_t$ to $do(E_t=\hat{\bm{e}}_t)$~\cite{pearl2009causality}.} 

% \vspace{6pt}
% \begin{center}
% \fcolorbox{black}{gray!8}{\parbox{0.98\linewidth}{\noindent$\bullet$ \textbf{Summary.}
% We minimize the loss function in Eq. (\ref{eqn:overall_loss}) over all users for CDR training. The detailed training procedure can be found in Algorithm~\ref{algo:training}.
% During the inference stage, we use the latest user preference $\bm{z}_T$ to calculate $f_{\theta_2}(\bm{z}_T)$ for the ranking of item candidates, and then recommend top-ranked items to each user.
% }}
% \end{center}
% \vspace{6pt}
\section{Experiments and Discussion}
\subsection{Datasets}
We assessed Prompt-MIL using three histopathological WSI datasets: TCGA-BRCA \cite{tcga_brca}, TCGA-CRC~\cite{cancer2012comprehensive}, and BRIGHT~\cite{bright}.
These datasets were utilized for both the self-supervised feature extractor pretraining and the end-to-end fine-tuning (with or without prompts), including the MIL component. 
Note that the testing data were not used in the SSL pretraining. \datasection{TCGA-BRCA} contains 1034 diagnostic digital slides of two breast cancer subtypes: invasive ductal carcinoma (IDC) and invasive lobular carcinoma (ILC). 
We used the same training, validation, and test split as that in the first fold cross validation in~\cite{chen2022scaling_hipt}. 
The cropped patches (790K training, 90K test) were extracted at 5$\times$  magnification. 
\datasection{TCGA-CRC} contains 430 diagnostic digital slides of colorectal cancer for a binary classification task: chromosomal instability (CIN) or genome stable (GS). 
Following the common 4-fold data split~\cite{bilal2021development,liu2018comparative}, we used the first three folds for training (236 GS, 89 CIN), and the fourth for testing (77 GS, 28 CIN). 
We further split 20\% (65 slides) training data as a validation set. The cropped patches (1.07M training, 370K test) were extracted at 10$\times$  magnification. 
\datasection{BRIGHT} contains 503 diagnostic slides of breast tissues. 
We used the official training (423 WSIs) and test (80 WSIs) splits. 
The task involves classifying non-cancerous (196 training, 25 test) vs. pre-cancerous (66 training, 23 test) vs. cancerous (161 training, 32 test). 
We further used 20\% (85 slides) training slides for validation. 
The cropped patches (1.24M training, 195K test) were extracted at 10$\times$ magnification. 

\begin{table}[t]
\caption{Comparison of accuracy and AUROC on three datasets. Reported metrics (in $\%$age) are the average across 3 runs. "Num. of Parameters" represents the number of optimized parameters}
\label{table:result:accuracy}
\begin{center}
\setlength{\tabcolsep}{0.9mm}{

\begin{tabular}{l |c c c c c c |c}
\toprule
\multicolumn{1}{c|}{Dataset} 
    & \multicolumn{2}{c}{TCGA-BRCA} 
        & \multicolumn{2}{c}{TCGA-CRC}
            & \multicolumn{2}{c}{BRIGHT}
    & \multicolumn{1}{|c}{Num. of}
\\
% \midrule
\multicolumn{1}{c|}{Metric} 
    & \multicolumn{1}{c}{Accuracy} & \multicolumn{1}{c}{AUROC} 
        & \multicolumn{1}{c}{Accuracy} & \multicolumn{1}{c}{AUROC}
            & \multicolumn{1}{c}{Accuracy} & \multicolumn{1}{c}{AUROC}
    & \multicolumn{1}{|c}{Parameters} 
\\
\midrule
Conventional MIL 
    & $92.10$          & $96.65$       
        & $73.02$         & $69.24$
            & $62.08$         & $80.96$
    & 70k
\\
Full fine-tuning 
    & $88.14$ & $93.78$
        & $74.53$ & $56.63$
            & $56.13$ & $75.87$
    & 5.6M
\\ 
Prompt-MIL (ours) 
    & $\bm{93.47}$ & $\bm{96.89}$
        & $\bm{75.47}$  & $\bm{75.45}$ 
            & $\bm{64.58}$  & $\bm{81.31}$ 
            
    & 70k+192
\\
\bottomrule
\end{tabular}
}
\end{center}
\end{table}


\subsection{Implementation Details}
We cropped non-overlapping 224 $\times$ 224 sized patches in all our experiments and used ViT-Tiny (ViT-T/16)~\cite{vit} for feature extraction.
For SSL pretraining, we leveraged the DINO framework~\cite{dino} with the default hyperparameters, but adjusted the batch size to 256 and employed the global average pooling for token aggregation. 
We pretrained separate ViT models on the TCGA-CRC datasets for 50 epochs, on the BRIGHT dataset for 50 epochs, and on the BRCA dataset for 30 epochs. 
For TCGA-BRCA, we used the AdamW~\cite{loshchilov2017adamW} optimizer with a learning rate of $1e-4$, $1e-2$ weight decay, and trained for 40 epochs.
For TCGA-CRC, we also used the AdamW optimizer with a learning rate of $5e-4$ and trained for 40 epochs.
For Bright, we used the Adam~\cite{adam} optimizer with a learning rate of $1e-4$, $5e-2$ weight decay and trained for 40 epochs.
We applied a cosine annealing learning rate decay policy in all our experiments.
For the MIL baselines, we employed the same hyperparameters as above.
For all full fine-tuning experiments, we used the learning rate in the corresponding prompt experiment as the base learning rate. For parameters in the feature model $F(\cdot)$, which are SSL pretrained, we use 1/10 of the base learning rate. For parameters in the Classifier $G(\cdot)$, which are randomly initialized, we use the base learning rate. We train the full tuning model for 10 more epochs than our prompt training to allow full convergence. This training strategy is optimized using the validation datasets.
All model implementations were in PyTorch~\cite{paszke2019pytorch} on a NVIDIA Tesla V100 or a Nvidia Quadro RTX 8000.

\subsection{Results}
% We tested the effectiveness of our proposed method on the three downstream datasets comprising prostate, colon, and breast cancer. 
% \KM{may add some details about the methods you compared to here if space allows.}
We chose overall accuracy and Area Under Receiver Operating Characteristic curve (AUROC) as the evaluation metrics. 


% \noindent\textbf{Evaluation of prompt tuning performance:} \label{sec:result_prompt}
\resultsection{Evaluation of prompt tuning performance:} \label{sec:result_prompt}
We compared the proposed Prompt-MIL with two baselines: 1) a conventional MIL model with a frozen feature extractor~\cite{li2021dual_dsmil}, 2) fine-tuning all parameters in the feature model (full fine-tuning).
Table~\ref{table:result:accuracy} highlights that our Prompt-MIL consistently outperformed both.
%the conventional MIL method and the full fine-tuning method. 
%added another 192 para, which is less than 0.3\% of the total parameters of the conventional MIL competitor.
% With such a negligible parameter overhead, our Prompt-MIL 
Compared to the conventional MIL method, Prompt-MIL added negligible parameters (192, less than 0.3\% of the total parameters), 
achieving a relative improvement of 1.49\% in accuracy and 0.25\% in AUROC on TCGA-BRCA, 3.36\% in accuracy and 8.97\% in AUROC on TCGA-CRC, and 4.03\% in accuracy and 0.43\% in AUROC on BRIGHT.
The observed improvement can be attributed to a more optimal alignment between the feature representation learned during the SSL pretraining and the downstream task, i.e., the prompt explicitly calibrated the features toward the downstream task.  


The computationally intensive full fine-tuning method under-performed conventional MIL and Prompt-MIL. 
Compared to the full fine-tuning method, our method achieved a relative improvement of 1.29\% to 13.61\% in accuracy and 3.22\% to 27.18\% in AUROC on the three datasets.
Due to the relatively small amount of slide-level labels (few hundred to a few thousands) fully fine tuning 5M parameters in the feature model might suffer from overfitting. 
In contrast, our method contained less than 1.3\% of parameters compared to full fine-tuning, leading to robust training.


\begin{table}[t]
\caption{Comparison of GPU memory consumption and training speed per slide benchmarked on the BRIGHT dataset between the full fine-tuning and our prompt tuning on four slides with different sizes. Our prompt method requires far less memory and is significantly faster.}  
\begin{center}
\setlength{\tabcolsep}{1.6mm}{
\begin{tabular}{clcccc}
\toprule
\multicolumn{2}{c}{WSI size}  
    & $44k\times21k$ & $26k\times21k$ 
        & $22k\times17k$ & $11k\times16k$ \\
\multicolumn{2}{c}{\#Tissue patches} 
    & 9212               & 4765               
        & 2307               & 1108               \\ 
\midrule
GPU        & Full fine-tuning 
    & 21.81G             & 18.22G             
        & 16.37G             & 12.71G             \\
Mem.     & Prompt (ours)    
    & $\bm{12.04}$G             & $\bm{10.66}$G             
        & $\bm{10.00}$G              & $\bm{7.90}$G              \\ 
\cmidrule{2-6}
        & Reduction percentage 
    & 44.79\%  & 41.50\% & 38.92\% & 37.84\% \\
\midrule
Time      & Full fine-tuning 
    & 17.73s             & 8.92s              
        & 4.37s              & 2.15s              \\
per slide  & Prompt (ours)    
    & $\bm{13.92}$s             & $\bm{7.09}$s              
        & $\bm{3.35}$s              & $\bm{1.56}$s              \\
\cmidrule{2-6}
& Reduction percentage 
    & 21.49\%  & 20.51\% & 23.32\% & 27.27\% \\
\bottomrule
\end{tabular}
}
\end{center}
\label{table:result:gpu}
\end{table}

\begin{table}[b]
\caption{Comparison of accuracy and AUROC on three datasets for a pathological foundation model.
}
\label{table:result:universal_models}
\begin{center}
% \setlength{\tabcolsep}{1.6mm}{

\begin{tabular}{l c c c c }
\toprule
\multicolumn{1}{c}{Dataset} & \multicolumn{2}{c}{TCGA-BRCA} & \multicolumn{2}{c}{BRIGHT} \\
% \midrule
\multicolumn{1}{c}{Metric} & \multicolumn{1}{c}{Accuracy} & \multicolumn{1}{c}{AUROC} & \multicolumn{1}{c}{Accuracy} & \multicolumn{1}{c}{AUROC}  \\
\midrule
ViT-small~\cite{wang2022transformer}
    & $91.75$          & $97.03$       
        & $54.17$  & $76.76$ \\
ViT-small w/ Prompt-MIL
    & $\bm{92.78} $ & $\bm{97.53}$
        & $\bm{57.50}$  & $\bm{78.29}$ \\
\bottomrule
\end{tabular}
% }
\end{center}
\end{table}


\resultsection{Evaluation of time and GPU memory efficiency:} Prompt-MIL is an efficient method requiring less GPU memory to train and running much faster than full fine-tuning methods. 
We evaluated the training speed and memory consumption of our method and compared to the full fine-tuning baseline on four different sized WSIs in the BRIGHT dataset.
% As sizes and areas of tissue regions varies among slides in a dataset, instead of evaluating the memory consumption and training speed on three datasets, 
% We evaluated it on four different sized WSIs in the BRIGHT dataset.
As shown in Table~\ref{table:result:gpu}, our method consumed around 38\% to 45\% less GPU memory compared to full fine-tuning and was 21\% to 27\% faster. 
As we scaled up the WSI size (i.e. WSIs with more number of patches), the memory cost difference between Prompt-MIL and full fine-tuning further widened. 
%Therefore our method is particularly even more crucial for processing large WSIs or for processing at higher magnification (eg. 20X, 40X) which contains much more number of patches per WSI. 


\resultsection{Evaluation on the pathological foundation models:} 
% The effectiveness of our prompt-MIL in augmenting the performance of foundational pathology models has been also evaluated. 
We demonstrated our Prompt-MIL also had a better performance when used with the pathological foundation model.
Foundational models refer to those trained on large-scale pathology datasets (e.g. the entire TCGA Pan-cancer dataset~\cite{weinstein2013cancer}). 
We utilized the publicly available~\cite{wang2022transformer,transpath} ViT-Small network pretrained using MoCo v3~\cite{mocov3} on all the slides from TCGA~\cite{weinstein2013cancer} and PAIP~\cite{paip}. 
In Table~\ref{table:result:universal_models}, we showed that our method robustly boosted the performance on both TCGA (the same domain as the foundation model trained on) and BRIGHT (a different domain). 
The improvement is more prominent in BRIGHT, which further confirmed that Prompt-MIL aligns the feature extractor to be more task-specific.

\begin{table}[h]
\caption{Performance with a different number of prompt tokens. For two different WSI classification tasks, one token was enough to boost the performance of the conventional MIL schemes.}
\label{table:result:abalation}
\begin{center}
\setlength{\tabcolsep}{1.6mm}{

\begin{tabular}{ccccc}
\toprule
\multicolumn{1}{c}{Dataset} 
    & \multicolumn{2}{c}{TCGA-BRCA}        
        & \multicolumn{2}{c}{BRIGHT}           \\
\#prompt tokens $k$             
    & \multicolumn{1}{l}{Accuracy} & AUROC 
        & \multicolumn{1}{l}{Accuracy} & AUROC \\ 
\midrule
$k=1$                           
    & $\bm{93.47}$  & $\bm{96.89}$ 
        & $\bm{64.58}$  & $\bm{81.31}$ \\
$k=2$                       
    & 93.13   & $\bm{96.93}$ 
        & 60.41 & 79.74 \\
$k=3$                           
    &  $\bm{93.47}$  &  $96.86$     
        &  59.17   &  76.75     \\ 
\bottomrule
\end{tabular}
}
\end{center}
\end{table}

\resultsection{Ablation study:}
An ablation was performed to study the effect of the number of trainable prompt tokens on downstream tasks. 
%We used the same pretrained feature models as that in section~\ref{sec:result_prompt}.
Table~\ref{table:result:abalation} shows the accuracy and AUROC of our Prompt-MIL model with 1, 2 and 3 trainable prompt tokens ($k=1, 2, 3$) on the TCGA-BRCA and the BRIGHT datasets.
On the TCGA-BRCA dataset, our Prompt-MIL model with 1 to 3 prompt tokens reported similar performance.
On the BRIGHT dataset, the performance of our model dropped with the increased number of prompt tokens. 
% This drop is consistant with
% Such a performance drop is consistent with the large performance drop of our method and full fine-tuning method in Table~\ref{table:result:accuracy}.
Empirically, this ablation study shows that for classification tasks, one prompt token is sufficient to boost the performance of conventional MIL methods.
% It is probably because the number of training samples in the BRIGHT dataset is only around half of that in the TCGA-BRCA dataset, and thus the increased trainable tokens may cause over-fitting.
% Overall, this ablation study showed that for classification tasks, one prompt token is sufficient to boost the performance of conventional MIL methods.




\section{Related Works}
\label{related_works}

\subsection{Model compression}

The simplest and one of the most effective form of compression involves sharing weights within a layer. Deep Compression \cite{HanMD15} introduced k-means clustering based weight sharing for compression. Initially, all weights belonging to the same cluster are share weight of the cluster centroid. During forward pass loss is calculated using the shared weights which are then updated during backward pass. This leads to a loss of accuracy and model train-ability because the weight to cluster assignment is intractable during weight update \cite{yinl2019}. DKM \cite{cho2021dkm} introduces differentiable k-means clustering, therefore making cluster assignments tractable. During forward clustered weights are used, however during backward the gradient is applied on the original weights.

\subsection{Model quantization}
Model quantization reduces the memory footprint of a model by reducing the representative bits per weight for a given model. In this paper we have compared our results with various training time quantization algorithms like EWGS \cite{lee2021network}, LSQ \cite{esser2019learned} and DoReFa \cite{zhou2016dorefa} used in PACT \cite{choi2018pact}. PACT clips activation values with a trainable parameter for activation quantization and uses DoReFa for weight quantization. LSQ quantizes weights and activations with learnable step size (scale or bin size). EWGS applies gradient scaling with respect to position difference in between original full precision weights and quantized weights based on LSQ.

\subsection{Regularization for quantization}
As we mentioned earlier, range regularization is not a quantization or compression technique. It helps compression and quantization by removing outliers in the weight distribution. \cite{kure} show that uniform distribution of weights are more robust to quantization than normally-distributed weights. To this they propose KURE (KUrtosis REgularization) to minimize the kurtosis of weights and ensure a uniform distribution. This method is independent of the quantization bit-width, therefore supports Post-Training Quantization in addition to QAT (Quantization Aware Training). However, this method is best suited for generic models which need to be specifically tuned for a given bit precision use case. To reduce the accuracy drop due to quantization \cite{binregularization} proposes to constrain weights to predefined bins based on the quantization bit-width. However, selecting these bins is a difficult process and the output of the models in very sensitive to the bin selection. In addition to that these methods ignore the effect of quantizing the first and last layers of the models.
\section{Discussion and Limitations}

Although we can ablate concepts efficiently for a wide range of object instances, styles, and memorized images, our method is still limited in several ways. First, while our method overwrites a target concept, this does not guarantee that the target concept cannot be generated through a different, distant text prompt. We show an example in \reffig{limitation} (a), where after ablating {\menlo Van Gogh}, the model can still generate {\menlo starry night painting}. However, upon discovery, one can resolve this by explicitly ablating the target concept {\menlo starry night painting}. Secondly, when ablating a target concept, we still sometimes observe slight degradation in its surrounding concepts, as shown in \reffig{limitation} (c). 

\nupur{Our method does not prevent a downstream user with full access to model weights from re-introducing the ablated concept~\cite{ruiz2022dreambooth,kumari2022multi,gal2022image}. Even without access to the model weights, one may be able to iteratively optimize for a text prompt with a particular target concept. Though that may be much more difficult than optimizing the model weights, our work does not guarantee that this is impossible.}

Nevertheless, we believe every creator should have an ``opt-out'' capability. We take a small step towards this goal, creating a computational tool to remove copyrighted images and artworks from large-scale image generative models.


% \clearpage



{
\tiny
\bibliographystyle{ACM-Reference-Format}
\balance
\bibliography{bibtex}
}
\appendix
% \input{6_SI}


\end{document}
\endinput