% mnras_template.tex 
%
% LaTeX template for creating an MNRAS paper.
%
% v3.0 released 14 May 2015
% (version numbers match those of mnras.cls)
%
% Copyright (C) Royal Astronomical Society 2015
% Authors:
% Keith T. Smith (Royal Astronomical Society)
% Change log.
%
% v3.0 May 2015
%    Renamed to match the new package name
%    Version number matches mnras.cls
%    A few minor tweaks to wording.
% v1.0 September 2013
%    Beta testing only - never publicly released.
%    First version: a simple (ish) template for creating an MNRAS paper.
%%%%%%%%%%%%%%%%%%%%%%%%%%%%%%%%%%%%%%%%%%%%%%%%%%
% Basic setup. Most papers should leave these options alone.
\documentclass[fleqn,usenatbib]{mnras}

% MNRAS is set in Times font. If you don't have this installed (most LaTeX
% installations will be fine) or prefer the old Computer Modern fonts, comment
% out the following line
% Depending on your LaTeX fonts installation, you might get better results with one of these:
%\usepackage{mathptmx}
%\usepackage{txfonts}

% Use vector fonts, so it zooms properly in on-screen viewing software
% Don't change these lines unless you know what you are doing
\usepackage[T1]{fontenc}

% Allow "Thomas van Noord" and "Simon de Laguarde" and alike to be sorted by "N" and "L" etc. in the bibliography.
% Write the name in the bibliography as "\VAN{Noord}{Van}{van} Noord, Thomas"
\DeclareRobustCommand{\VAN}[3]{#2}
\let\VANthebibliography\thebibliography
\def\thebibliography{\DeclareRobustCommand{\VAN}[3]{##3}\VANthebibliography}


%%%%% AUTHORS - PLACE YOUR OWN PACKAGES HERE %%%%%

% Only include extra packages if you really need them. Common packages are:
\usepackage{graphicx}	% Including figure files
\usepackage{amsmath}	% Advanced maths commands
\usepackage{amssymb}	% Extra maths symbols

%%%%%%%%%%%%%%%%%%%%%%%%%%%%%%%%%%%%%%%%%%%%%%%%%%

%%%%% AUTHORS - PLACE YOUR OWN COMMANDS HERE %%%%%

% Please keep new commands to a minimum, and use \newcommand not \def to avoid
% overwriting existing commands. Example:
%\newcommand{\pcm}{\,cm$^{-2}$}	% per cm-squared
\newcommand{\OM}[1]{\textcolor{magenta}{[OM: #1]}}
\newcommand{\om}[1]{\textcolor{magenta}{#1}}

\newcommand{\RMC}[1]{\textcolor{blue}{[RMC: #1]}}
\newcommand{\rmc}[1]{\textcolor{blue}{#1}}

\newcommand{\mkpc}{\text{M}_{\odot}^2\text{kpc}^{-5}}
\newcommand{\mkpcdeg}{\ \text{M}_{\odot}^2\text{kpc}^{-5}\text{deg}}
\newcommand{\rhosquaredmkpcdeg}{\ \text{M}_{\odot}^2\text{kpc}^{-3}\text{deg}}

\usepackage{orcidlink}
\usepackage{newtxtext,newtxmath}
%%%%%%%%%%%%%%%%%%%%%%%%%%%%%%%%%%%%%%%%%%%%%%%%%%

%%%%%%%%%%%%%%%%%%% TITLE PAGE %%%%%%%%%%%%%%%%%%%

% Title of the paper, and the short title which is used in the headers.
% Keep the title short and informative.
\title[DM Annihilation Signals From the Sagittarius Dwarf]{Prospective Dark Matter Annihilation Signals From the Sagittarius Dwarf Spheroidal}
% The list of authors, and the short list which is used in the headers.
% If you need two or more lines of authors, add an extra line using \newauthor
\author[T. A. A. Venville et. al.]{Thomas A. A. Venville,$^{\orcidlink{0000-0003-0278-9933}}$$^{1,2}$\thanks{101615311@student.swin.edu.au}
Alan R. Duffy,$^{1,2}$
Roland M. Crocker$^{\orcidlink{0000-0002-2036-2426}}$$^{3}$,
Oscar Macias$^{\orcidlink{0000-0001-8867-2693}}$$^{4,5}$
\newauthor and Thor Tepper-García$^{\orcidlink{0000-0002-1081-883X}}$$^{6,7}$
\\
% List of institutions
$^{1}$Centre for Astrophysics and Supercomputing, Swinburne University of Technology, PO Box 218, Hawthorn, Victoria, 3122, Australia \\
$^{2}$ARC Centre of Excellence for Dark Matter Particle Physics, Australia \\
$^{3}$Research School of Astronomy and Astrophysics, Australian National University, Canberra 2611, A.C.T., Australia \\
$^{4}$GRAPPA $-$ Gravitational and Astroparticle Physics Amsterdam, University of Amsterdam, Science Park 904, 1098 XH Amsterdam, The Netherlands\\
$^{5}$Institute for Theoretical Physics Amsterdam and Delta Institute for Theoretical Physics, University of Amsterdam, Science Park 904,\\ 1098 XH Amsterdam, The Netherlands.\\ 
$^{6}$Sydney Institute for Astronomy, School of Physics, The University of Sydney, NSW 2006, Australia\\
$^{7}$Centre of Excellence for All Sky Astrophysics in Three Dimensions (ASTRO-3D), Australia
}
% These dates will be filled out by the publisher
\date{Accepted XXX. Received YYY; in original form ZZZ}

% Enter the current year, for the copyright statements etc.
\pubyear{2023}

% Don't change these lines
\begin{document}
\label{firstpage}
\pagerange{\pageref{firstpage}--\pageref{lastpage}}
\maketitle

% Abstract of the paper.
\begin{abstract}
The Sagittarius Dwarf Spheroidal galaxy (Sgr) is investigated as a target for DM annihilation searches utilizing J-factor distributions calculated directly from a high-resolution hydrodynamic simulation of the infall and tidal disruption of Sgr around the Milky Way. In contrast to past studies, the simulation incorporates DM, stellar and gaseous components for both the Milky Way and the Sgr progenitor galaxy. The simulated distributions account for significant tidal disruption affecting the DM density profile. Our estimate of the J-factor value for Sgr, $J_{\text{Sgr}}=1.48\times 10^{10}\ \mkpc$ ($6.46\times10^{16}\ \text{GeV}\ \text{cm}^{-5}$), is significantly lower than found in prior studies. This value, while formally a lower limit, is likely close to the true J-factor value for Sgr. It implies a DM cross-section incompatibly large in comparison with existing constraints would be required to attribute recently observed $\gamma$-ray emission from Sgr \citep{Crocker_and_Macias_et_al_2022} to DM annihilation. We also calculate a J-factor value using an NFW profile fitted to the simulated DM density distribution to facilitate comparison with past studies. This NFW J-factor value supports the conclusion that most past studies have overestimated the dark matter density of Sgr on small scales. This, together with the fact that the Sgr has recently been shown to emit $\gamma$-rays of astrophysical origin, complicate the use of Sgr in indirect DM detection searches.
%Thanks to their relatively large dark matter (DM) to baryonic matter ratio, dwarf spheroidal (dSph) galaxies present a promising target for the detection of DM annihilation products such as $\gamma$-rays, providing valuable insight on DM physics. Naively, the Sagittarius Dwarf spherodial galaxy (Sgr) is an attractive target amongst dSph galaxies for indirect DM detection experiments due to its relatively high mass and nearness. Here the Sagittarius Dwarf Spheroidal galaxy (Sgr) is investigated as a target for DM annihilation searches utilizing J-factor distributions calculated directly from a high-resolution, N-body/hydrodynamic simulation of the infall and tidal disruption of Sgr around the Milky Way. In contrast to past studies, the simulation incorporates both collisionless (DM, stars) and gaseous components for both the Milky Way and the Sgr progenitor galaxy. The simulated distributions account for significant tidal disruption affecting the DM density profile, allowing for a conservative estimate of the Sgr DM density and J-factor distributions. The Sgr J-factor distribution is calculated directly from particle properties. The calculated J-factor value for Sgr, $J_{\text{Sgr}}=1.48\times 10^{10}\ \mkpc$ ($6.46\times10^{16}\ \text{GeV}\ \text{cm}^{-5}$), is significantly lower than found in prior studies and implies a DM cross-section incompatibly large in comparison with existing constraints would be required to attribute recently observed $\gamma$-ray emission from Sgr to DM annihilation. A small observed spatial offset ($\sim1.4^\circ$) between the dark matter and simulated stellar density peaks supports an astrophysical explanation for recently observed $\gamma$-ray emission from Sgr detailed in \citet{Crocker_and_Macias_et_al_2022}. A J-factor value is also calculated from an NFW profile fitted to the simulated DM density distribution to facilitate comparison with pas studies. This NFW J-factor value is significantly lower than found in most past studies yet significantly overestimates the J-factor value calculated from the simulated particle density profile at small radii. The fitted profile supports the conclusion that most past studies have overestimated the dark matter density of Sgr on these scales. This, together with the fact that the Sgr has recently been shown to emit $\gamma$-rays of astrophysical origin, complicate the use of this particular galaxy in indirect DM detection searches.
\end{abstract}

% Select between one and six entries from the list of approved keywords.
% Don't make up new ones.
\begin{keywords}
gamma-rays: galaxies - dark matter - galaxies:individual:Sagittarius Dwarf - astroparticle physics 
\end{keywords}

%%%%%%%%%%%%%%%%%%%%%%%%%%%%%%%%%%%%%%%%%%%%%%%%%%
%%%%%%%%%%%%%%%%% BODY OF PAPER %%%%%%%%%%%%%%%%%%
\section{Introduction: dark matter annihilation signals from dwarf spheroidal Galaxies}
\label{sec:I1}
%% Oscar's first logical point: big problem in science! The lack of detection of  dark matter.
In the $\Lambda \text{CDM}$ cosmological model of the Universe, approximately $83\%$ of the total mass density of the universe consists of dark matter (DM), 
a massive particle species that primarily interacts with baryonic matter through gravitational interactions \citep{Garrett_2011}. The hierarchical gravitational formation of structure in this cosmological model results in galaxies contained in more massive DM haloes. These DM haloes are thus often traced by stellar populations. Diverse experiments have attempted to detect particle DM candidates, targeting a wide range of DM masses and velocity averaged annihilation cross sections \citep[e.g.][]{Bertone_2005}. These experimental searches include monitoring for direct detection of DM interaction with target materials and `indirect' searches for Standard Model products of DM self-annihilation and decay, for example $\gamma$-rays, neutrinos and charged cosmic rays. These experiments have, thus far, not (definitively) detected DM particle candidates.
\\[10pt]
%%Oscar's second logical point: narrowing and identifying the specific knowledge gap: I am targeting the Sgr dSph and Stream
Dwarf spheroidal galaxies are promising targets for DM searches due to their high mass to light ratios (indicating an abundance of dark matter). Indirect dark matter searches  for products of dark matter annihilation in dwarf spheroidal galaxies and the Galactic Centre have been conducted with a variety of observational facilities targeting different areas of particle parameter space. Claims of detection of a $\gamma$-ray spectral line signature \citep{Bringmann_and_Weinger_2012} and potential continuum emission \citep{Goodenough_and_Hooper_2009,Gordon_and_Macias_2013, Abramowski_et_al_2014} due to Weakly Interacting Massive Particles (WIMP) dark matter annihilation from dwarf spheroidal galaxies have been made. However, the fact that such claimed 
$\gamma$-ray signatures
have been, at best,
similar in magnitude to astrophysical $\gamma$-ray backgrounds -- 
which are
themselves somewhat uncertain -- has so far precluded conclusive identification of  observed $\gamma$-ray fluxes as products of dark matter annihilation \citep{Abramowski_et_al_2014,Geringer_Sameth_et_al_2015,Calore_et_al_2015b,Geringer_Sameth_et_al_2018, Macias_et_al_2018,Macias_et_al_2019,Abazajian_et_al_2020,Pohl:2022nnd}.
\\[10pt]
Compared to the Galactic Centre region and `classical' dwarf spheroidal (dSph) galaxies, which have been investigated extensively for indirect signatures of dark matter annihilation, few prior studies have previously investigated the Sagittarius Dwarf Spheroidal Galaxy \citep[hereafter Sgr;]{Ibata_et_al_1997, HESS:2007ora} as a target for dark matter annihilation searches. This is due to the location of Sgr near the Galactic plane and Galactic Centre region\footnote{$(l_{\rm Sgr},b_{\rm  Sgr})\approx(6^\circ,-14^\circ)$; \citet{Majewski_et_al_2003}.}, uncertain astrophysical background sources \citep{Viana_et_al_2012,Crocker_and_Macias_et_al_2022}, and large systematic uncertainties in the dark matter distribution (and thus the spatial morphology of any dark matter annihilation signature) of Sgr due to ongoing tidal disruption \citep{Rico_et_al_2020}. The most significant continuum detection of $\gamma$-ray emission from Sgr was made by \citet{Crocker_and_Macias_et_al_2022}, who detect Sgr with a $8.1\ \sigma$ significance 
in Fermi Large Area Telescope (LAT; \citealt{Atwood_et_al_2009}) data using their standard analysis pipeline. \citet{Crocker_and_Macias_et_al_2022} find the emitted Sgr $\gamma$-ray distribution spatially traces the stellar distribution of Sgr. The spectral distribution of Sgr $\gamma$-ray photons detected by \citet{Crocker_and_Macias_et_al_2022} strongly favour milli-second pulsar (hearafter MSP) $\gamma$-ray emission due to a combination of inverse Compton scattering of CMB photons by high-energy electron-positron pairs escaping from the Sgr MSP population and magnetospheric MSP $\gamma$-ray emission. Additionally, the H.E.S.S collaboration 
(marginally)
detected Sgr with a $2.05\ \sigma$ significance \citep{Abramowski_et_al_2014}; however, they conclude that the results are `well compatible' with a Gaussian significance distribution centred on zero. \citet{Viana_et_al_2012} also analyse Sgr for sources of potential $\gamma$-ray emission detectable with Cherenkov telescopes, concluding that predicted $\gamma$-ray emission from millisecond pulsars outshines the $\gamma$-ray signal due to dark matter annihilation by several orders of magnitude.
\\[10pt]
This study seeks to predict the spatial and quantitative properties of the J-factor distribution of Sgr, providing a template for further searches for annihilation products from diverse particle physics models. As detailed in section~\ref{sec:M3}, in an attempt to provide self-consistent templates of these structures, the N-body/hydrodynamic simulations of \citet{Tepper_Garcia_and_Bland_Hawthorn_2018} were utilized to derive the dark matter density distribution for Sgr. Simulated particle distributions provide an advantage over profile modelling utilizing stellar tracers through more accurate modelling of the significant tidal disruption of Sgr. This has likely resulted in considerable changes to the Sgr internal dynamics, DM density distribution and stellar density profile during the satellite's infall and tidal disruption \citep[Ch. 2]{Kazantzidis_et_al_2011,Newberg_and_Carlin_2016} In section~\ref{sec:M3}, we will also adapt the methodology of \citet{Stoehr_et_al_2003} and \citet{Charbonnier_et_al_2011}, utilizing the density distribution of the simulated dark matter particles to produce J-factor magnitude distributions for use in indirect dark matter searches. The results of our J-factor estimates are presented in section~\ref{sec:R1}, with further details of the J-factor and $\rho$ profiles of Sgr presented in sections~\ref{sec:R2} and \ref{sec:R3}. We then explore the implications of the J-factor value we derive in Section~\ref{sec:D1}, before concluding in section~\ref{sec:C1}.
\section{Methodology}
\subsection{Overview}
\label{sec:M1}
Numerous simulations of the infall and tidal disruption of the progenitor galaxy of Sgr have been performed \citep[e.g.][]{Law_and_Majewski_2010,Lokas_et_al_2010,Law_et_al_2004,Dierickx_and_Loeb_2017}, differing in the initial position, mass and velocity of the Sgr progenitor and differing in the distribution of stellar and dark matter components. These difference in initial parameters has been shown to produce marked variations in the inferred orbit and evolution of the Sgr remnant \citep{Jiang_and_Binney_2000,Law_and_Majewski_2010,Lokas_et_al_2010}.
\\[10pt]
In contrast to prior simulations, the simulation of \citet{Tepper_Garcia_and_Bland_Hawthorn_2018} included a comprehensive treatment of gas in the Sgr progenitor. Realistic treatment of this gaseous component was shown to have a considerable effect on the orbital decay of Sgr, and also successfully reproduced other key features of the Sagittarius Dwarf/Stream system such as the approximate final position of stellar particles and the approximate angular size of the final Sagittarius Dwarf. Furthermore, in contrast to most previous simulations, this simulation includes initial conditions placing the Sgr progenitor at the virial radius of the Milky Way, which facilitates a more accurate treatment of the tidal disruption process during infall \citep{Dierickx_and_Loeb_2017}. This produced a more realistic evolution, including tidal stripping, of the Sgr Progenitor dark matter and stellar particles than simulations where the dwarf is artificially stripped and placed within the virial radius of the Milky Way. 
\\[10pt]
Here, we exploit this simulation to produce an estimation of the expected dark matter distribution of Sgr. This simulation included a Sagittarius Dwarf progenitor of total mass $11\times 10^{9}$ M$_\odot$ modelled with three live components. These components were a dark matter sub-halo of total mass $M_{DM} = 10^{10}$ M$_\odot$ (with a mass per particle of $10^5$ M$_\odot$), a stellar bulge of mass $M_S = 4\times 10^8$ M$_\odot$ (with a mass per particle of $4\times 10^3$ M$_\odot$) and a gaseous halo of mass $M_G = 6\times 10^8$ M$_\odot$ (and a mass per particle of $6\times 10^3$ M$_\odot$). Each of these components consisted of $10^5$ particles, with the initial mass distribution of these three components governed by a spherical Hernquist profile. The Sgr progenitor was placed at a location of $\vec{r}_0=(125,0,0)$ kpc relative to the initial centre of the simulated Milky Way halo and was simulated in infall for a total duration of $3.6$ Gyr, undertaking three pericentric passages. For further details of the simulation and resulting distribution of particles, see \citet{Tepper_Garcia_and_Bland_Hawthorn_2018}.
\\[10pt]
However, despite the successes of the \citet{Tepper_Garcia_and_Bland_Hawthorn_2018}
model, it, in common with all previous simulations, fails to exactly reproduce the observed distribution of the Sagittarius Dwarf stars \citep{Majewski_et_al_2003,Belokurov_et_al_2014}, cf.~Figure~\ref{fig:F1_}. Accordingly, as detailed in \ref{sec:M2}, we first translate the simulated particle distribution to produce a template model for Sgr spatially congruent with the Sgr core position determined by \citet{Majewski_et_al_2003}.
\begin{figure}
    \centering
    \includegraphics[width=\columnwidth]{Images/Method/Comparison_to_observational_data.png}
    \caption{The simulated stellar particle population and observed stars \citep{Majewski_et_al_2003,Law_et_al_2005} in the Sagittarius Stream, reproduced from Figure 2 of \citet{Tepper_Garcia_and_Bland_Hawthorn_2018}.}
    \label{fig:F1_}
\end{figure}
\\[10pt]
We used the translated projected dark matter particle distribution to calculate the Sgr J-factor distribution from the resulting projected Sagittarius Dwarf density profile. These calculations (detailed in section~\ref{sec:M3}) followed the method of \citet{Stoehr_et_al_2003} to calculate the relevant distribution directly from simulated particle properties (also see \citealt{Kuhlen_2009,Charbonnier_et_al_2011}). The J-factor value of the Sgr dark matter population was calculated by summing the J-factor values of particles within the estimated bound radius of the core of Sgr, $(3.7\pm 0.2)^\circ$ \citep{Majewski_et_al_2003}.
%%%%%%%%%%%%%%%%%%%%%%%%%%%%%%%%%
\subsection{Translation of the simulated Sagittarius Stream.}
\label{sec:M2}
An accurate density distribution of dark matter in the area of Sgr, informed by simulations, is of crucial importance to informing indirect dark matter searches. For spatial likelihood searches the position of Sgr is also crucial; however, at the time of writing no past simulation has accurately reproduced the observed position of the Sgr remnant following simulated infall of the Sgr progenitor into the Milky Way halo. The initial projected location of the simulated Sgr, defined as the position of the greatest projected stellar number density\footnote{The location of Sgr reported in \citet{Tepper_Garcia_and_Bland_Hawthorn_2018}, $(\alpha,\delta) = (285,-36.6)^\circ$, is defined utilizing the position of maximum 3D mass density.} occurred at approximately $(\alpha,\delta) \simeq (282.77,-35.23)^\circ$. This differs slightly from the observed location of Sgr reported in \citet{Majewski_et_al_2003}, $(l_{\rm Sgr},b_{\rm  Sgr})\approx(6^\circ,-14^\circ)$.
\\[10pt]
Accordingly, to produce a dark matter and stellar template concordant with the observed Sagittarius Dwarf position, the simulated stars and dark matter particles from Sgr in the simulation of \citet{Tepper_Garcia_and_Bland_Hawthorn_2018} were translated in RA and DEC such that the position of maximum simulated stellar projected number density of Sgr (in the centre of the simulated Sgr) was located at the observed \citep{Majewski_et_al_2003} location of Sgr. Specifically, this translated the simulated stellar and dark matter particle distributions by $(\Delta \alpha,\Delta \delta) = (1.23,4.73)^\circ$; the location of the simulated Sgr was moved from a projected location of $(\alpha,\delta) \simeq (282.77,-35.23)^\circ$ to $(\alpha',\delta') \simeq (284,-30.5)^\circ$. Note that this translation preserves the 3D structure of the simulated stream, given this translation preserves the relative position of all Sgr particles and does not result in a change in heliocentric distance to Sgr. The translated simulated stellar and dark matter particle distributions are shown in Figure\ref{fig:F4} and Figure\ref{fig:F5}, respectively. The offsets between the initial and translated particle distributions are illustrated in Figure\ref{fig:F6} (simulated stellar) and~\ref{fig:F7} (dark matter).
\begin{figure}
    \includegraphics[width=\columnwidth]{Images/Method/Final_simulated_stellar_distribution.pdf}
    \caption{The translated number density distribution of Sagittarius Stream simulated stellar particles. The position of these particles have been uniformly translated by $(\Delta \alpha,\Delta \delta) = (1.23,4.73)^\circ$, and the projected location of the simulated Sgr is now $(\alpha,\delta) \simeq (284,-30.5)$. This is the location of Sgr reported by \citet{Majewski_et_al_2003}, as indicated by the yellow cross.}
    \label{fig:F4}
\end{figure}
\begin{figure}
    \includegraphics[width=\columnwidth]{Images/Method/Final_dark_matter_distribution.pdf}
    \caption{The translated Sagittarius Stream dark matter number density distribution. As per the simulated stellar translation, all particles are translated by $(\Delta \alpha,\Delta \delta) = (1.23,4.73)^\circ$. The observed location of Sgr reported in \citet{Majewski_et_al_2003} is indicated with a yellow cross, which is coincident with the simulated Sgr dark matter population}
    \label{fig:F5}
\end{figure}
%%% Spot for initial and final comparison figures.
\begin{figure}
    \includegraphics[width=\columnwidth]{Images/Method/Sim_stars_distribution_comparison_plot.pdf}
    \caption{The initial and translated Sagittarius Dwarf/Stream simulated stellar particle number density distributions, illustrating the offset. The translated simulated stellar core is clearly visible at $(\alpha,\delta) \simeq (284,-30.5)$.}
    \label{fig:F6}
\end{figure}
\begin{figure}
    \includegraphics[width=\columnwidth]{Images/Method/DM_distribution_comparison_plot.pdf}
    \caption{The initial and translated Sagittarius Dwarf/Stream dark matter particle number density distributions. The width of the dark matter particle distribution is far wider than the applied translation.}
    \label{fig:F7}
\end{figure}
%%% Subsection M2 complete final corrections (aside from comments) implemented, 05/10/2022! 
%%%%%%%%%%%%%%%%%%%%%%%%%%%%%%%%%%%%%%%%%%%%%%%%%
\subsection{Production and analysis of projected density and projected J-factor distributions}
\label{sec:M3}
The self-annihilation or decay of several families of particle dark matter candidates is generically expected to produce $\gamma$-ray emission \citep[e.g.,][]{Bertone_2005}. In the case of WIMPs \citep[][]{Jungman_1996}, primary and secondary $\gamma$-rays are produced as products of self annihilation through multiple production channels (in addition to other annihilation products), with the volumetric annihilation rate scaling with the square of mass density. For astrophysical targets such as dwarf spheroidal galaxies and the Galactic centre the expected $\gamma$-ray flux from WIMP self-annihilation, as a function of energy $E$ per unit energy per solid angle $\Omega$, can be modelled using an equation of the form \citep{Charbonnier_et_al_2011}:
\begin{equation}
    \frac{d\Phi_{\gamma}}{dE_\gamma}(E,\Delta \Omega)= \frac{1}{4\pi}\frac{\langle\sigma\nu\rangle}{2m^2}\frac{dN_\gamma}{dE_\gamma}(E)\times J(\Delta \Omega)
    \label{EQ1}
\end{equation}
where the velocity averaged DM annihilation cross section $\langle\sigma\nu\rangle$, DM particle mass $m$ and spectral energy distribution of emitted $\gamma$-rays $\frac{dN_\gamma}{dE}(E)$ are model dependent parameters. Together, these terms detail the spectral distribution of the $\gamma$-ray annihilation products.
In contrast, the `J-factor' $J(\Delta \Omega)$ specifies the spatial dependence of the $\gamma$-ray flux. Explicitly in the case of WIMP annihilation \citep{Charbonnier_et_al_2011},
\begin{equation}
    J = \int_{\Delta \Omega}\frac{dJ}{d\Omega} = \int_{\Delta \Omega}\int \rho^2(r,\Omega) \text{d}r\text{d}\Omega
    \label{EQ}
\end{equation}
where $\rho$ is the density distribution of dark matter along the line of sight radii $r$ to the object of angular size $\Delta \Omega$.
\\[10pt]
%%%Paragraph addressing Oscar's second point: computing the J-factor for Sgr is hard because it is gravitationally disrupted.
However, accurately constraining the J-factor value for the dark matter component of Sgr is difficult due to ongoing tidal disruption (and uncertain foreground/background $\gamma$-ray sources). This results in a dark matter density distribution subject to significant uncertainties, comparatively difficult (compared to classical dSph systems) to constrain with stellar tracers. Simulations are often used to model the dark matter density distribution of Sgr; however, to the knowledge of the authors no simulation has been utilized to model the J-factor distribution and compute an integrated J-factor value for Sgr.
\\[10pt]
The translated particle distributions were utilized to produce projected simulated stellar density distributions, dark matter density distributions and dark matter J-factor distributions, as follows. The quantity $\rho^2$, the square of the dark matter density, was selected to trace the density profile of Sgr. This quantity correlates with the self-annihilation luminosity of WIMP dark matter particle families (e.g., \citealt{Ullio_et_al_2002}, their equation 35). The projected spatial density squared $\rho^2$ at position $r$, integrated over the volume of Sgr, was estimated directly from the simulation output particle density $\rho_i$ in accordance with the equation \citep{Stoehr_et_al_2003}:
\begin{equation}
    \int_{V} \rho_{DM}^2\ d\text{V} = \sum_i \rho_i m_i
\label{EQ8}
\end{equation}
where $m_i$ is the (constant) simulated particle mass. Note that whilst \citet{Stoehr_et_al_2003} utilize equation~\ref{EQ8} to calculate the J-factor value, in this study J-factor values for substructures within the simulation were calculated including an inverse-square distance dependence given the extended nature of Sgr. Specifically, we include a factor of $1/(4\pi r_i^2)$ for each particle $i$ to account for the size of the flux sphere for each simulation particle. With this distance dependence, the J-factor definition adopted in this study is equivalent to a flux, not absolute brightness. The contribution of each particle $i$ to the J-factor value for each volume element $V_b$ at a heliocentric distance $r$ of the considered structure was thus calculated as:
\begin{equation}
    J_b = \int_{V_b} \rho_{DM}^2/(4\pi r^2)\ d\text{V} = \sum_i \rho_i m_i/(4\pi r_i^2)
\label{eq:EQ6}
\end{equation}
where $\rho_i$ is the dark matter density, $m_i$ is the mass of particle $i$ and $r_i$ is the distance to particle $i$. As shown in appendix~\ref{sec:A1}, this definition is equivalent to the J-factor definition provided by \citet{Charbonnier_et_al_2011}, though differs by a factor of $4\pi$ due to the flux sphere surface area normalization factor. The total summed J-factor value for an extended substructure was calculated by summing over the constituent volume elements. 
\begin{multline}
    J = \Sigma_b \int_{V_b} \rho_{DM}^2/(4\pi r^2)\ d\text{V} = \Sigma_b\Sigma_i \rho_i m_i/(4\pi r_i^2) \\ = \Sigma_{i'}\rho_{i'} m_{i'}/(4\pi r_{i'}^2) 
\label{EQ9}
\end{multline}
for all particles $i'$ in the extended substructure (potentially of multiple volume elements).
\\[10pt]
Projected distributions of the line of sight integrated $\rho^2$ and J-factor values were produced using the HEALPix pixelisation scheme \citep{Gorski_et_al_2005} to sum the particle values in spatial pixels (volume elements). The integrated $\rho^2$ value for each HEALPix pixel was calculated utilizing equation~\ref{EQ8}, summing all particles $i$ in the pixel. Similarly, the integrated J-factor value for extended structures (e.g. Sgr) was calculated utilizing equation~\ref{EQ9}, summing the contributions of all particles in the constituent pixels of the extended structure. Figure~\ref{fig:DM_Rho_squared_final_distribution_plot} shows the resulting pixelized distributions of the projected squared density profiles for the dark matter population of Sgr. Similarly, Figure\ref{fig:Simulated_stellar_rho_squared_final_distribution_plot} shows the projected $\rho$ distribution for simulated stellar population. In these figures, we use the $(\Lambda_\odot,\beta_\odot)$ coordinate system of \citet{Belokurov_et_al_2014}, which is a variant of the coordinate system utilized in \citet{Majewski_et_al_2003}. In the system of \citet{Belokurov_et_al_2014}, the $\Lambda_\odot$ axis proceeds along the Sagittarius Stream in the direction of motion of Sgr, whilst the $\beta_\odot$ axis points toward the North Galactic Pole. These figures detail the region $\Lambda_\odot \in [\Lambda_{0}\pm 50^\circ], \beta_\odot \in [\beta_{0} \pm 50\circ]$, where $(\beta_{0},\Lambda_{0}) = (-1.48,-2.22)$. Sgr is located at $(\beta_{\text{Sgr}},\Lambda_{\text{Sgr}})= (-1.48,-0.22)^\circ$ in this coordinate system.
\begin{figure}
    \centering
    \includegraphics[width=\columnwidth]{Images/Results/Rho_squared_binning_DM.pdf}
    \caption{The predicted dark matter $\rho^2$ distribution for Sgr in the coordinate system of \citet{Belokurov_et_al_2014}. The increased dark matter density within the core radius of Sgr is clearly visible, with the projected $\rho^2$ peak located at $(\beta_\odot,\Lambda_{\odot}) \simeq (-1.0,-1.3)^\circ$}
    \label{fig:DM_Rho_squared_final_distribution_plot}
\end{figure}
\begin{figure}
    \centering
    \includegraphics[width=\columnwidth]{Images/Results/Rho_binning_sim_stars.pdf}
    \caption{The predicted simulated stellar $\rho$ distribution for Sgr, again in the coordinate system of \citet{Belokurov_et_al_2014}. The projected $\rho^2$ peak is located at $(\beta_\odot,\Lambda_{\odot}) \simeq (-2.4,-0.9)^\circ$, an offset of approximately $1.4^\circ$ from the projected dark matter $\rho^2 peak$} 
    \label{fig:Simulated_stellar_rho_squared_final_distribution_plot}
\end{figure}
%%%%%%%%%%%%%%%%%%%%%%%%%%
\\[10pt]
A quantity of potential interest to observational analysis of Sgr is the projected offset in maximum density between the simulated stellar and dark matter populations. To calculate this offset, the projected $\rho^2$ distributions were smoothed, such that the value of a pixel $a$ in the smoothed distribution was the mean value of the density distribution in pixel $a$ and the immediate 8 surrounding pixels. This smoothing was applied to minimize the impact of small-scale density fluctuations on determining the position of maximum density.
\\[10pt]
The pixel corresponding to the projected maximum squared density was then located for both simulated stellar and dark matter populations. In the coordinate system of \citet{Belokurov_et_al_2014}, these pixels were located at $(\beta_\odot,\Lambda_{\odot}) = (-1.02,-1.31)^\circ$ and $(\beta_\odot,\Lambda_{\odot}) = (-2.40,-0.85)^\circ$ for the dark matter and simulated stellar particle populations, respectively (see section~\ref{sec:R2} for a detailed explanation of this coordinate system). The projected offset between these pixel positions was thus $1.44^\circ$.
\\[10pt]
In their $\gamma$-ray analysis employing a Sgr stellar template, \citet{Crocker_and_Macias_et_al_2022} found moderate evidence ($4.5\sigma$ statistical significance) for a shift in the best-fit position of the template 180$^\circ$ away from the dwarf galaxy's travel direction ($\sim 4^\circ$ towards the Galactic south). This is evidence for an offset between the peak of the $\gamma$-ray emission and the centre of Sgr defined by its stars. The much smaller offset, found here, between the simulated DM population and the simulated stellar population of Sgr does not seem to provide an explanation of the offset tentatively found by \citet{Crocker_and_Macias_et_al_2022}. Furthermore, the offset between the simulated stellar and dark matter density peaks we find is in the wrong direction to explain the offset found by \citet{Crocker_and_Macias_et_al_2022}. This in turn tentatively indicates that the observed $\gamma$-ray emission is not due to $\gamma$-ray emission from DM annihilation in Sgr, consistent with \citet{Crocker_and_Macias_et_al_2022}.
\\[10pt]
In addition to this paper analyzing the J-factor distribution of Sgr, an upcoming study (Venville, in prep, hereafter Paper II) will discuss the all-sky J-factor distribution of the Sagittarius Stream and investigate the possibility of detecting $\gamma$-ray emission from the Sagittarius Stream utilizing Fermi LAT data.
\section{Results: predicted dark matter density and J-factor distributions from the Sagittarius Dwarf}
\subsection{Predicted J-factor magnitude of the Sagittarius Dwarf}
\label{sec:R1}
In accordance with equation~\ref{EQ9}, the integrated J-factor value for Sgr was calculated by summing the pixels of the projected J-factor distribution within a `core radius' (corresponding to the approximate bound radius) of $\alpha_{\text{Sgr}} \simeq 3.7^\circ$ \citep{Majewski_et_al_2003} of the centre of the simulated Sagittarius Dwarf, defined as $(\alpha,\delta) \simeq (284,-30.5)^\circ$ in section~\ref{sec:M2}. This radius corresponds to a physical distance of 1.6 kpc at the location of Sgr and a circular integration region of solid angle of $\Omega_{\text{Sgr}} \simeq 3.6\times 10^{-3}$ sr. This integrated J-factor value of Sgr was $1.48\times 10^{10}\ \mkpc$.
\\[10pt]
Prior determinations of the J-factor value for Sgr are consistently higher than value derived in this paper. \citet{Abramowski_et_al_2014} estimate a J-factor value of $J_{\text{Sgr}} = 2.88\times 10^{12}\ \mkpc$ integrated across a solid angle of $\Delta \Omega = 10^{-5}\ \text{sr} \simeq 0.032\ \text{deg}^2$, whereas \citet{Viana_et_al_2012} calculate a J-factor value of $J_{\text{Sgr}} = 1.5\times 10^{15}\ \mkpc$ for an integration solid angle of $\Delta \Omega = 2\times10^{-5}\ \text{sr} = 0.07\ \text{deg}^2$. These results shall be discussed in Section~\ref{sec:D1}.
\\[10pt]
The dark matter J-factor distribution for Sgr produced in this study has been explored using the methods outlined in \citet{Crocker_and_Macias_et_al_2022}. The normalized DM J-factor distribution was detected with a maximum
significance of $<1\sigma$, including during evaluation of transitional and rotational tests. This DM template is thus significantly less favoured by the data than templates tracing the Sgr stellar density distribution (detected with an $8.1\sigma$ significance). Therefore, \citet{Crocker_and_Macias_et_al_2022} concluding that millisecond pulsars are the likely cause of $\gamma$-
ray emission from the direction of Sgr is consistent both with the finding of \citet{Viana_et_al_2012} and the predicted J-factor distribution for Sgr calculated in this study. To provide further support for this conclusion, we utilize the J-factor value calculated here and the $\gamma$-ray photon flux attributed to Sgr in \citet{Crocker_and_Macias_et_al_2022} to calculate a lower limit on the required WIMP cross-section $\langle \sigma \nu \rangle$ to produce this $\gamma$-ray photon flux. In accordance with equation~\ref{EQ1} and \citet{Mazziotta_et_al_2012}, the photon flux $\Phi_\gamma$ can be expressed as
\begin{equation}
    \Phi_\gamma(E,\Delta\Omega) = J(\Delta\Omega)\times\frac{1}{2}\frac{\langle\sigma\nu\rangle}{4\pi m^2_\chi}\Sigma_f N_f(E,m_\chi)B_f
\end{equation}
where $m_\chi$ is the mass of the WIMP and $N_f(E,m_\chi)$ is the differential photon spectrum produced by pair annihilation into a final state $f$ with a branching fraction $B_f$. Setting $N_f=B_f=1$ allows the calculation of a lower limit on the cross section value for an assumed WIMP mass and J-factor value $J(\Delta\Omega)$. For 
%a WIMP mass of 
%$m_\chi=100$ GeV and 
the J-factor of magnitude $J_{\text{Sgr}} \sim 10^{10} \ \mkpc$ as determined in this study, the lower limit on the cross section required to produce the observed GeV-band $\gamma$-ray number  flux $\Phi_\gamma%(3.8\pm0.6)\times10^{36}\ \text{erg}/\text{s}
\sim 10^{-8}\ \text{cm}^{-2}\ \text{s}^{-1}$ observed in \citet{Crocker_and_Macias_et_al_2022} is $\langle \sigma \nu \rangle \sim 6 \times10^{-20} \text{cm}^{3}\ \text{s}^{-1} \left(\frac{m_\chi}{\rm 100 \ GeV}\right)^2$. 
%Similarly, for a WIMP mass of $1$ GeV, the lower limit on the required cross-section is $\langle \sigma \nu \rangle \simeq 5.6\times10^{-24}\ \text{cm}^{3}\ \text{s}^{-1}$.  
%
Such a velocity averaged cross-section is inconsistent  with existing constraints on WIMP cross sections \citep[e.g.,][figure 1]{Abazajian_et_al_2020b}. This reinforces the conclusions of \citet{Crocker_and_Macias_et_al_2022} and strongly disfavours WIMP DM as the source of the observed $\gamma$-ray emission from the Sgr.
\subsection{J-factor profiles}
\label{sec:R2}
As the J-factor value for Sgr in this study is low compared to prior estimations \citep{Abramowski_et_al_2014,Viana_et_al_2012} and significant $\gamma$-ray emission from MSPs is present at the location of Sgr \citep{Crocker_and_Macias_et_al_2022}, morphological features of the Sgr J-factor distribution may aid detection of any $\gamma$-ray emission due to DM annihilation. Thus, the following section shall detail morphological characteristics of the DM J-factor distribution inferred from the simulations. This has the potential advantage, compared to DM distributions derived from stellar distribution data \citep[e.g.][]{Viana_et_al_2012}, of more accurate treatment of the tidal disruption of the DM component and avoiding the need to assume dynamical equilibrium or spherical symmetry in the DM (or stellar tracer) population.
\\[10pt]
Figure~\ref{fig:J_factor_DM_ROI} illustrates the dark matter J-factor distribution at the location of the Sagittarius core. %%% Up to here in removing references to the stream J-factor.
\begin{figure}
    \centering 
    \includegraphics[width=\columnwidth]{Images/Results/J_factor_binning_DM.pdf}
    \caption{The J-factor distribution centred on the location of the simulated Sagittarius Dwarf, in the coordinate system of \citet{Belokurov_et_al_2014}.}
    \label{fig:J_factor_DM_ROI}
\end{figure}
\begin{figure}
    \centering
    \includegraphics[width=\columnwidth]{Images/Results/DM_J_factor_profile_Beta.pdf}
    \caption{The J-factor profile for the dark matter particles as a function of $\beta_\odot$, within the region of interest defined in Section~\ref{sec:R2} and depicted in Figure~\ref{fig:J_factor_DM_ROI}. This region of interest spans the range $-50^\circ<\Lambda_\odot<50^\circ$.}
    \label{fig:DM_J_factor_Beta_profile}
\end{figure}
\\[10pt]
The J-factor distribution integrated in bins of $\beta_\odot$ in this region for the dark matter particles is illustrated in Figure~\ref{fig:DM_J_factor_Beta_profile}. As is evident in Figure~\ref{fig:DM_J_factor_Beta_profile}, there is a marked increase in the magnitude of the predicted J-factor  at the location of Sgr; however, $\gamma$-ray emission from stellar populations or millisecond pulsars are also expected to increase in magnitude at this location \citep{Viana_et_al_2012}.
\\[10pt]
The peak magnitude of the $\beta_\odot$ J-factor profile depicted in Figure~\ref{fig:DM_J_factor_Beta_profile} is $4.55\times10^9\mkpcdeg$; this is approximately $2.4$ times the J-factor value of $1.91\times10^9\mkpcdeg$ at the core radius of Sgr ($\beta_\odot,\Lambda_\odot)=(-3.73,0)^\circ$. The difference between these values indicates the relative J-factor contrast between the core and outskirts of Sgr; this contrast is significantly lower than the stellar number density contrast across similar angular scales in the fit templates to the observed $\gamma$-ray data \citep[Supplementary Figure 1]{Crocker_and_Macias_et_al_2022}. Accordingly, the DM J-factor templates are likely insufficiently peaked to fit the spatial $\gamma$-ray emission distribution observed for Sgr.
\\[10pt]
Similarly, the J-factor profile for the DM particles binned in bins of $\Lambda_\odot$ is illustrated in Figure~\ref{fig:DM_J_factor_Lambda_profile}. The value of this distribution at the observed core central position is $1.3\times10^9\mkpcdeg$, whilst the J-factor value at $(\beta_\odot,\Lambda_\odot)=(0,-3.73)^\circ$ is $8.21\times10^8\mkpcdeg$. However, the $\Lambda_\odot$ profile detailed in Figure~\ref{fig:DM_J_factor_Lambda_profile} shows significant asymmetry and variance on small angular scales; the J-factor value at $(\beta_\odot,\Lambda_\odot)=(0,3.73)^\circ$ is $5.27\times10^8\mkpcdeg$. Again, these J-factor profiles are likely insufficiently peaked to fit the spatial $\gamma$-ray distribution attributed to Sgr in \citet{Crocker_and_Macias_et_al_2022}.
\begin{figure}
    \centering
    \includegraphics[width=\columnwidth]{Images/Results/DM_J_factor_profile_Lambda.pdf}
    \caption{The J-factor profile for the dark matter particles as a function of $\Lambda_\odot$, within the region of interest defined in Section~\ref{sec:R2} and depicted in Figure~\ref{fig:J_factor_DM_ROI}. This region of interest spans the range $-50^\circ<\beta_\odot<50^\circ$.}
    \label{fig:DM_J_factor_Lambda_profile}
\end{figure}
To compare the width of the J-factor profiles to observed stellar distributions and simulated DM/stellar density distributions, distribution functions were fitted to the scaled J-factor profiles, as illustrated in Figure~\ref{fig:DM_J_factor_Beta_profile_normalized} for the dark matter J-factor $\beta_\odot$ profile and Figure~\ref{fig:DM_J_factor_Lambda_profile_normalized} for the $\Lambda_\odot$ profile. A \citet{Moffat_1969} function  of the form
\begin{equation}
    f(x) = A \left(1 + \frac{\left(x - x_{0}\right)^{2}}{\gamma^{2}}\right)^{-\alpha}
    \label{EQ:Moffat}
\end{equation}
best fit the $\beta_\odot$ profile and an offset Gaussian of the form
\begin{equation}
    f(x) = \frac{a}{\sigma \sqrt{2\pi}}exp{\left(\frac{-(x-\mu)^2}{2\sigma^2}\right)} +c
    \label{EQ:Offset_Gaussian}
\end{equation}
best fit the $\Lambda_\odot$ profile. The parameters of these fits are detailed in table~\ref{tab:Fits_summary_table}. The full width half maximum (FWHM) of the Moffat function fitted to the DM J-factor profile along the $\beta_\odot$ axis is $3.15^\circ$, comparable to the observed `core radius' of Sgr reported in \citet{Majewski_et_al_2003}. The FWHM of the offset Gaussian fitted to the scaled dark matter $\Lambda_\odot$ J-factor profile is $8.1^\circ$. This illustrates the known tidal disruption of Sgr along the $\Lambda_\odot$ axis \citep{Majewski_et_al_2003,Lokas_et_al_2010}.  The Sgr J-factor distribution is thus notably extended along both $\beta_\odot$ and $\Lambda_\odot$ axes, providing a valuable morphological discriminant to point-like sources of emission present in analysis procedures given any $\gamma$-ray emission from the Sgr DM population would follow a similarly extended distribution.
%%% How does a gaussian distribution provide a valid comparison to the observed stellar distributions?
\begin{figure}
    \includegraphics[width=\columnwidth]{Images/Results/DM_J_factor_profile_Beta_normalized.pdf}
    \caption{The dark matter $\beta_\odot$ J-factor profile, scaled such that the maximum value is equal to $1.0$. Also shown is the fitted Moffat distribution. The FWHM of this distribution is $3.15^\circ$.}
    \label{fig:DM_J_factor_Beta_profile_normalized}
\end{figure}
\begin{figure}
    \includegraphics[width=\columnwidth]{Images/Results/DM_J_factor_profile_Lambda_normalized.pdf}
    \caption{The dark matter $\Lambda_\odot$ J-factor profile. Again, the profile is scaled such that the maximum value is equal to $1.0$. The FWHM of the fitted offset Gaussian distribution is $8.1^\circ$.}
    \label{fig:DM_J_factor_Lambda_profile_normalized}
\end{figure}
\begin{table*}
    \centering
    \caption{The parameters of distributions fitted to the normalized simulated stellar and dark matter particle profiles.}
    \begin{tabular}{c|c|c}
        Profile & Fitted distribution & Fitted parameters \\
        \hline
        \hline
        DM $\rho^2$ - $\beta_\odot$ & Moffat & $A=0.887$ \\
         & &  $x_0 = -1.31$ \\ 
         & & $\gamma=1.47$ \\
         & & $\alpha=0.639$ \\
        \hline
        Simulated stellar $\rho$ - $\beta_\odot$ & Voigt & $x_0=-1.63$ \\
         & &  $A_L=1.71$ \\ 
         & & $FWHM_L=1.34$ \\
         & & $FWHM_G=2.45$ \\
        \hline
         DM $\rho^2$ - $\Lambda_\odot$ & Offset Gaussian & $a=4.88$ \\
         & &  $mu=-0.922$ \\ 
         & & $\sigma=3.51$ \\
         & & $c=0.11$ \\
        \hline 
        Simulated stellar $\rho$ - $\Lambda_\odot$ & Moffat & $A=0.650$ \\
         & &  $x_0 = -1.27$ \\ 
         & & $\gamma=3.39$ \\
         & & $\alpha=1.09$ \\
        \hline
        DM J-factor - $\beta_\odot$ & Moffat & $A=0.884$ \\
         & &  $x_0 = -1.42$ \\
         & & $\gamma=1.44$ \\
         & & $\alpha=0.67$ \\
        \hline
         DM J-factor - $\Lambda_\odot$ & Offset Gaussian & $a=4.87$ \\
         & &  $\mu=-1.01$ \\ 
         & & $\sigma=3.43$ \\
         & & $c=0.10$ \\
    \end{tabular}
    \label{tab:Fits_summary_table}
\end{table*}
\\[10pt]
These fitted profiles also facilitate an estimation of the percentage contribution of substructure in the J-factor profiles to the overall J-factor value. This is important to calculate given the significant tidal disruption of Sgr could cause substructure not accurately modelled by conventional profiles and, on the other hand, as the simulation could cause spurious substructure. The percentage difference between the total integrated J-factor value of the scaled $\beta_\odot$ J-factor profile and the integrated value of the fitted Moffat distribution (both depicted in Figure\ref{fig:DM_J_factor_Beta_profile_normalized}) over a range $-20^\circ<\beta_\odot<20^\circ$ was calculated as $5.51$ \%. Similarly, the percentage difference in integrated J-factor values between the scaled $\Lambda_\odot$ J-factor profile and fitted Gaussian distribution (depicted in Figure\ref{fig:DM_J_factor_Lambda_profile_normalized}) was calculated as $0.81$ \% over the range $-20^\circ<\Lambda_\odot<20^\circ$. These values imply a similarly low contribution from substructure within these intervals to the J-factor values reported in this study.
\subsection{$\rho^2$ profiles}
\label{sec:R3}
\begin{figure*}
    \centering
    \includegraphics[width=2\columnwidth]{Images/Results/Rho_squared_profiles_Beta.pdf}
    \caption{The DM $\rho^2$ profile (in black) and the simulated stellar $\rho$ profile (in red) as a function of $\beta_\odot$. As detailed in equation~\ref{EQ9}, the DM J-factor (and thus $\gamma$-ray luminosity) is proportional to the DM $\rho^2$, whilst the expected $\gamma$-ray luminosity from stellar associated sources (such as MSP) is proportional to the stellar density \citet{Crocker_and_Macias_et_al_2022}.}
    \label{fig:Rho_squared_beta_profile}
\end{figure*}
The $\rho^2$ profile for the simulated DM distribution of Sgr as a function of $\beta_\odot$ is shown in Figure\ref{fig:Rho_squared_beta_profile}, with the simulated stellar $\rho$ profile. Both the simulated stellar particle population and the dark matter particle population are concentrated within the Sgr core radius, with the simulated stellar $\rho$ profile of higher magnitude than the DM $\rho^2$ at small radii. The magnitude of the simulated stellar $\rho$ distribution at the observed core position of Sgr is $\rho_{ss,\beta_\odot}=1.9\times10^{10}\ $M$_\odot^2$ kpc$^{-3}\ $deg, whilst the magnitude of the DM $\rho^2$ distribution at this location is $\rho_{DM,\beta_\odot}=0.27\times10^{10} \ \rhosquaredmkpcdeg$. This  indicates Sgr is dominated by stars at small radii as expected.
 \begin{figure*}
    \centering
    \includegraphics[width=2\columnwidth]{Images/Results/Rho_squared_profiles_Lambda.pdf}
    \caption{The DM $\rho^2$ profile (in black) and the simulated stellar $\rho$ profile (in red) as a function of $\Lambda_\odot$. As per Figure\ref{fig:Rho_squared_beta_profile}, the simulated stellar density profile exceeds the DM $\rho^2$ within the Sgr core radii.}
    \label{fig:Rho_squared_Lambda_profile}
 \end{figure*}
\\[10pt]
Scaled variants of these $\beta_\odot$ profiles with fitted functions are illustrated in Figures~\ref{fig:DM_rho_squared_beta_profile_normalized} and~\ref{fig:Sim_stars_rho_squared_beta_profile_normalized} for the DM $\rho^2$ and simulated stellar $\rho$ distributions, respectively. As per the J-factor distributions detailed in section~\ref{sec:R2}, these functions are fitted to illustrate the width of the profiles and facilitate comparison between different profiles. The FWHM of the Voigt \citep{Schreier_2018} profile fitted to the scaled simulated stellar $\rho$ distribution is 3.0 degrees, whilst the Moffatt function fitted to the DM $\rho^2$ distribution is 4.96 degrees. Beyond the core radius, the simulated stellar particles decrease in density rapidly with increasing radius (Figure~\ref{fig:Rho_squared_beta_profile}), consistent with observations of the Sagittarius Steam width \citep{Belokurov_et_al_2014}. The greater width of the comparatively flat dark matter density profile, if consistent at higher $\Lambda_\odot$ values, could provide a valuable morphological discriminant between $\gamma$-ray emission from stellar emission channels and dark matter.
\\[10pt]
The DM $\rho^2$ profile as a function of $\Lambda_\odot$  shown in Figure~\ref{fig:Rho_squared_Lambda_profile} along with the corresponding simulated stellar $\rho$ profile. The DM $\rho^2$ value at the observed stellar core position is $\rho^2_{DM,\Lambda_\odot}=7.3\times10^8\ \rhosquaredmkpcdeg$, whilst the simulated stellar $\rho$ profile has a value of $\rho_{ss,\Lambda_\odot}=7.3\times10^9\ $M$_\odot^2$ kpc$^{-3}\ $deg.
\\[10pt]
Functions were fitted to the scaled $\Lambda_\odot$ profiles. A Gaussian with an additional nonzero constant (equation~\ref{EQ:Offset_Gaussian}) was fitted to the scaled dark matter $\rho^2$ profile to account for the nonzero density of the Sagittarius Stream, as shown in Figure~\ref{fig:DM_rho_squared_Lambda_profile_normalized}. This offset Gaussian has a FWHM value of $8.26^\circ$, again illustrating strong tidal disruption along the $\Lambda_\odot$ axis. The Moffat fit to the scaled simulated stellar $\rho$ profile shown in Figure~\ref{fig:Sim_stars_rho_squared_Lambda_profile_normalized} has a FWHM of $10.4^\circ$. This is significantly larger than the `core radius' reported in \citet{Majewski_et_al_2003}, indicating the likely extended nature of any potential $\gamma$-ray emission from the Sgr stellar population. The extended simulated stellar $\rho$ distribution is consistent with the extended stellar-associated $\gamma$-ray source template detected with a $8.1\sigma$ significance in the analysis of \citet{Crocker_and_Macias_et_al_2022}. At angular distances exceeding $\approx 20^\circ$ from Sgr, both the simulated stellar and dark matter distributions have approximately constant density. This is consistent with the projected stellar density for the Sagittarius trailing arm (beyond the `hump' of tidal disruption) reported in \citet{Majewski_et_al_2003,Niederste_Ostholt_et_al_2010}. Extending this density distribution to analyse the Sagittarius Stream will be the subject of Paper II.
%%%%%%%%%%%%%%%%%%%%%
%TTG: You could shorten the title to 'Discussion'
\section{Discussion: Comparison of J-factor magnitude with past studies of the Sagittarius Dwarf}
%\subsection{Comparison of J-factor magnitude with past studies of the Sagittarius Dwarf}
\label{sec:D1}
 Only a few studies have estimated the J-factor value of the Sgr dark matter component; these estimates are consistently larger than the J-factor value $J_{\text{Sgr}}=1.48\times 10^{10}\ \mkpc$ calculated in this work. We will explore several of these works and compare them to our own calculations in turn.
\\[10pt]
Firstly, \citet{Abramowski_et_al_2014} calculate a J-factor value of $J_{\text{Sgr},\text{Abramowski}} \simeq 2.9\times 10^{12}\ \ \mkpc$, estimating the dark matter density profile utilizing a NFW profile fit to stellar velocity measurements. They then utilize a maximum-likelihood process to determine the most probable contribution of this dark matter distribution toward $\gamma$-ray counts observed with the H.E.S.S. telescope. \citet{Abramowski_et_al_2014} marginally detect Sgr with a $2.05\ \sigma$ significance, however conclude that $\gamma$-ray counts due to the dark matter population of Sgr (as opposed to astrophysical sources) are likely negligible.
\\[10pt]
Secondly, \citet{Viana_et_al_2012} fit an NFW profile to Sloan Digital Sky Survey (SDSS) stellar velocity dispersion data. This profile is integrated within an integration area of $\Delta \Omega = 2\times10^{-5}\ \text{sr} = 0.07\ \text{deg}^2$ to derive a J-factor value of $J_{\text{Sgr},\text{Viana}} = 2.0\times10^{16}\ \mkpc$, with uncertainties of a factor of 2.
\begin{figure*}
    \centering
    \includegraphics[width=2\columnwidth]{Images/Discussion/NFW_profile_comparison_plot_relaxed_radii.pdf}
    \caption{The simulation particle densities, fitted NFW profile to the simulation particle densities (in blue) and the NFW profile fitted by \citet{Viana_et_al_2012} to stellar velocity dispersion data (in red). The \citet{Power_et_al_2003} convergence radius $r_c=0.73$ kpc of the simulated particle halo is indicated; the radial density profile internal to this radius was not utilized when fitting the NFW profile indicated in blue. Also indicated is the radius $r=0.044$ kpc corresponding to a circular area of $2\times10^{-5}$ sr, utilized to calculate the J-factor in \citet{Viana_et_al_2012}. Clearly, the fitted NFW profile will provide a significantly lower density estimate at this radius than the NFW profile detailed in \citet{Viana_et_al_2012}, but still overestimates the radial density profile at small radii.}
    \label{fig:Viana_comparison_plot}
\end{figure*}
\\[10pt]
 Given the lack of detection of higher predicted dark matter J-factor fluxes in \citet{Viana_et_al_2012,Abramowski_et_al_2014} and significantly stronger observed $\gamma$-ray emission from MSP candidates \citep{Crocker_and_Macias_et_al_2022}, the derived J-factor value for Sgr in this study is unlikely to improve current constraints on $\gamma$-ray emission from the Sgr dark matter distribution.
\\[10pt]
%%%Why is it different paragraph!
To investigate the discrepancy between this study and the larger J-factor value for Sgr reported in \citet{Viana_et_al_2012} we fitted an NFW profile of the form \citep{Viana_et_al_2012}
\begin{equation}
    \rho(r) = \frac{\delta_c}{(r/r_s)(1+r/r_s)^2}
    \label{NFW_density_profile}
\end{equation}
to the simulated radial DM mass density profile (calculated as the mass density of concentric shells), where $\rho(r)\ \text{M}_{\odot}\text{kpc}^{-3}$ is the density at a radial distance of $r$ kpc, $\delta_c$ is the characteristic density of the halo and $r_s$ is the scale radius of the halo. As a precaution, only particles at a radius of greater than the convergence radius $r_c=0.73$ kpc \citep{Power_et_al_2003} from the centre of Sgr were utilized in the fitting procedure to avoid potentially unrealistic density profiles due to simulation artefacts. The radial density profile at smaller radii is significantly overestimated by the fitted NFW profile, as detailed in Figure\ref{fig:Viana_comparison_plot}. Thus, relaxing this fitting constraint and fitting the radial density profile at smaller radii would %likely 
result in shallower inner NFW profile slope and inner profile density values, with corresponding downward revision of any J-factor value estimated from the fitted NFW profile. The fitted NFW parameters and the equivalent NFW profile parameters utilized in \citet{Viana_et_al_2012} are detailed in table~\ref{tab:NFW_parameters_table}.
\begin{table*}
    \centering
    \caption{The NFW profile parameters and J-factor values from integrating the corresponding projected J-factor distributions given by equation~\ref{Sigma_M_squared_dist}. The area of integration was $\Delta \Omega=2\times10^{-5}$ sr in both cases. The errors quoted are one standard deviation on the fitted parameters.}
    \begin{tabular}{|c|c|c|c|}
        \hline
        Parameters & $r_s$ (kpc) & $\delta_c$ ($\text{M}_{\odot}\text{kpc}^{-3}$) & J-factor value ($ \mkpc$)  \\
        \hline
        \citet{Viana_et_al_2012} & $1.3$ & $1.1\times10^7$ & $1.1\times10^{16}$\\
        Fitted & $6\pm2$ & $(3\pm2)\times10^4$ &  $2.5\times10^{14}$\\
        \hline
    \end{tabular}
    \label{tab:NFW_parameters_table}
\end{table*}
A J-factor value was calculated from the fitted NFW profile utilizing the equation
\begin{equation}
    J(R) = 2\int_R^{\sqrt{r_d^2+R^2}} \frac{r\rho^2(r)}{\sqrt{r^2-R^2}}\ \text{d}r
    \label{Sigma_M_squared_dist}
\end{equation}
where $\rho(r)$ was the NFW density profile given by equation~\ref{NFW_density_profile} with `Fitted' parameters detailed in table~\ref{tab:NFW_parameters_table} and $r_d=4\ \text{kpc}$. Following \citet{Viana_et_al_2012}, a radius of $R=0.044$ kpc was utilized, equivalent to a projected circular area of $\Delta \Omega = 2\times10^{-5}\ \text{sr}$ at the distance of Sgr. This corresponds to a projected angular radius of $0.1^\circ$. The resulting J-factor value was $J_{\text{Sgr},\text{NFW}} = 2.5\times10^{14}\ \mkpc$. It is important to note that this J-factor value is independent of the assumed particle-based J-factor definition.
\\[10pt]
Repeating the calculation detailed in section~\ref{sec:R1}, the lower limit on the WIMP particle annihilation cross section required to explain the $\gamma$-ray photon flux attributed to Sgr in \citet{Crocker_and_Macias_et_al_2022} assuming this NFW J-factor value is also inconsistent with current constraints \citep{Abazajian_et_al_2020b,Evans_et_al_2022} for WIMP masses $\gtrsim 10$ GeV.
%Repeating the calculation detailed in section~\ref{sec:R1}, the lower limit on the WIMP particle annihilation cross section required to explain the $\gamma$-ray photon flux attributed to Sgr in \citet{Crocker_and_Macias_et_al_2022}, assuming this NFW J-factor value, is $\langle \sigma \nu \rangle \simeq 2\times10^{-24}\ \text{cm}^3\ \text{s}^{-1}$ for a $100$ GeV WIMP particle. Similarly, for a WIMP mass of 20 GeV the corresponding lower limit on the required particle annihilation cross section is $\langle \sigma \nu \rangle \simeq 9\times10^{-26}\ \text{cm}^3\ \text{s}^{-1}$. Both of these values are inconsistent with existing constraints detailed in \citet{Abazajian_et_al_2020b,Evans_et_al_2022}, further disfavouring WIMP DM as the source of the observed $\gamma$-ray emission from Sgr detailed in \citet{Crocker_and_Macias_et_al_2022}.
\\[10pt]
Calculating a J-factor value using equation~\ref{Sigma_M_squared_dist} and an identical radius from the NFW profile $\rho(r)$ defined by the parameters of \citet{Viana_et_al_2012} yielded a J-factor value of $1.1\times10^{16}\ \mkpc$. This is consistent within errors of the value $J_{\text{Sgr},\text{Viana}} = 2\times10^{16}\ \mkpc$ reported in \citet{Viana_et_al_2012}, though our calculation uses a different J-factor definition.
\\[10pt]
This in turn implies that the difference between the total J-factor value calculated from the fitted NFW profile $J_{\text{Sgr},\text{NFW}} = 2.5\times10^{14}\ \mkpc$ and the result of \citet{Viana_et_al_2012} can be partially explained by the relatively lower density of our fitted NFW density profile at small radii, as evident in Figure\ref{fig:Viana_comparison_plot} (noting that the J-factor scales as density squared).
\\[10pt]
To further explore the effect of differing J-factor definitions on the calculated J-factor value for Sgr, the value of the J-factor as a function of radius was calculated utilizing the analytic definition detailed in equation~\ref{Sigma_M_squared_dist} and the J-factor calculated from the pixel summation process detailed in equation~\ref{EQ9}. These are detailed in Figure\ref{fig:J_factor_values_comparison}.
\begin{figure*}
    \centering
    \includegraphics[width=2\columnwidth]{Images/Discussion/J_factor_values_comparison.pdf}
    \caption{J-factor values for Sgr, as a function of radius, computed utilizing various definitions. `Ring' J-factor values are computed utilizing equation~\ref{EQ9}, summing particles along the line of sight in a thin ring of radius $r$ kpc centred on Sgr. `Circular region' J-factor values are also computed utilizing equation~\ref{EQ9}, summing over all particles along the line of sight within a circular region of radius $r$ kpc centred on Sgr. The analytic J-factor value is computed utilizing equation~\ref{Sigma_M_squared_dist}. Also shown in green  is the volume integral of the NFW $\rho^2$ profile within a spherical volume of radius $r$ kpc centred on Sgr, computed utilizing equation~\ref{Volume_integral}.}
    \label{fig:J_factor_values_comparison}
\end{figure*}
The `circular region' J-factor values (indicated in solid red) are calculated utilizing equation~\ref{EQ9}, summing pixel contributions within a circular-based conical area of radius $r$ at the distance of Sgr. The `ring' J-factor values (indicated in solid black) are again calculated utilizing equation~\ref{EQ9} but summing pixels along the circumference of a thin ring of radius $r$. As would be expected, this approaches the circular region J-factor value for small radii. The average J-factor value per pixel for both the circular summation regions (in red dashes) and the ring summation regions (in black dashes) are also displayed. In blue on Figure\ref{fig:J_factor_values_comparison} is the J-factor value calculated with equation~\ref{Sigma_M_squared_dist}, implemented utilizing a modified variant of the \texttt{\_surfaceDensity} method from the \scriptsize COLOSSUS \normalsize package \citep{Diemer_2018} and the NFW density profile $\rho(r)$ fitted to the simulated dark matter particle density profile (Figure\ref{fig:Viana_comparison_plot}). Lastly, in green the volume integral of density squared
\begin{equation}
     \int \rho^2\ \text{d}V = 4\pi\int_0^r \rho^2(r)r^2\ \text{d}r
     \label{Volume_integral}
\end{equation}
is displayed, again calculated utilizing the NFW density profile $\rho(r)$ fitted to the dark matter particle density profile. Note that, in contrast to this study, the volume integral of density squared is also defined as the J-factor value in \citet{Stoehr_et_al_2003,Charbonnier_et_al_2011}.
\\[10pt]
As previously detailed, the `total circular region J-factor value' detailed in Figure \ref{fig:J_factor_values_comparison}, as calculated with equation~\ref{EQ9}, was selected to calculate the Sgr J-factor value in this study. The following considerations motivated this selection over other definitions. Firstly, utilizing a simulation based particle summation J-factor definition likely more accurately accounts for the extended, irregular shape of the Sgr dark matter halo due to tidal disruption \citep{Lokas_et_al_2010}. Calculating the J-factor value directly from the simulated particle distribution rather than fitting stellar tracers or a profile also avoids assumptions of dynamical equilibrium or spherical symmetry, which are clearly invalid in the case of Sgr. Given the extrapolated NFW profile clearly overestimates the particle density profile in the inner region of the simulated Sgr halo, the particle-based J-factor value is a conservative lower limit accounting for both potential underestimation due to simulation resolution effects and overestimation due to a potentially overestimated fitted NFW profile density.
In contrast, the J-factor value calculated from the fitted NFW profile likely exceeds the true J-factor value for Sgr, given the overestimate of the fitted NFW profile to the simulated particle density profile at small radii, in addition the divergent behaviour of the NFW density profile and analytic J-factor definition at small radii.
\\[10pt]
Secondly, in accordance with equation~\ref{EQ7} and equation~\ref{EQ9} it is clear that the J-factor definition should follow a similar behaviour as a function of radius as the volume integral of density squared, particularly for small radii, as all particles are at a similar heliocentric distance. The pixel averaged ring and pixel averaged circular definitions do not follow this behaviour, nor does the analytic definition detailed in equation~\ref{Sigma_M_squared_dist} which shows a nonphysical divergence at small radii. Lastly, the ring J-factor value does not serve to calculate the total integrated J-factor value within an extended structure, as required by this study.
\\[10pt]
%%Paragraph commenting on Viana result or Abramowski result - could some of the discrepancy be accounted for?
Consideration of these different J-factor definitions also provides additional insight into potential causes for the discrepancy between the Sagittarius Dwarf J-factor value calculated in this study and prior results such as that reported in \citet{Viana_et_al_2012}. The analytic J-factor definition detailed in equation~\ref{Sigma_M_squared_dist}, which reproduces the result of \citet{Viana_et_al_2012} at a radius of $r=0.044$ kpc within the margin of error, is clearly divergent at such small radii when calculated utilizing a fitted NFW profile. This is a source of discrepancy between the simulation J-factor value calculated from simulated particle data and the J-factor value for Sgr calculated in \citet{Viana_et_al_2012} (and other prior works) in addition to the aforementioned discrepancy resulting from the differing density profiles.
\\[10pt]
 Similarly to \citet{Viana_et_al_2012}, an estimate of the dark matter density derived through Jeans analysis is utilized by \citet{Evans_et_al_2022} to compute a J-factor value of $J_{\text{Sgr},\text{Evans}}=9.13\times10^{12}\ \mkpc$ ($10^{19.6}\ \text{GeV}\ \text{cm}^{-5}$). The analysis of \citet{Evans_et_al_2022} does not explore the possibility of extended emission beyond a radius of $2^\circ$ from the centre of the Sgr and assumes an NFW profile of small scale radius ($r_s=1$ kpc). We have shown that the DM density distribution of Sgr is significantly extended and find the NFW profile fitted to the Sgr DM density profile has a significantly larger scale radius than assumed in \citet{Evans_et_al_2022} and \citet{Viana_et_al_2012}. The simulated DM radial density profile indicates  a significantly lower core DM density than utilized in \citet{Evans_et_al_2022,Viana_et_al_2012} is appropriate for Sgr. This, in addition to the significant tidal disruption of the simulated Sgr halo over a large radial range, reinforces the conclusion of \citet{Evans_et_al_2022} that their DM density profile derived using Jeans analysis likely significantly overestimates the J-factor value of Sgr.
\\[10pt]
Computing the J-factor value of Sgr utilizing the simulated particle density profile, as detailed in this study, results in a significantly lower J-factor value for Sgr than found in \citet{Evans_et_al_2022} and implies a DM cross-section incompatible with the DM mass/annihilation cross section constraints illustrated in \citet{Evans_et_al_2022}, further demonstrating that the Sgr halo should not be utilized for indirect DM detection searches.
%
%
\section{Summary and Conclusions}
\label{sec:C1}
%%Method summary - how is ours better?
An N-body/hydrodynamic simulation of the infall and tidal disruption of the Sagittarius Dwarf Galaxy (Sgr) around the Milky Way was utilized to investigate the expected integrated J-factor value of Sgr and explore the morphological characteristics of the projected J-factor distribution. The simulation of \citet{Tepper_Garcia_and_Bland_Hawthorn_2018} was utilized to produce $\rho^2$ and J-factor distributions from the dark matter and simulated stellar particle populations through a summation of particle density and mass values in line of sight pixels. Utilizing this methodology provides a more accurate model of the Sgr DM halo than derivations utilizing stellar tracers through more accurately accounting for the strong tidal disruption of the Sgr DM halo.
\\[10pt]
%Section - significance of J-factor values, cross section.
The J-factor value for Sgr, $J_{\text{Sgr}}=1.48\times 10^{10}\ \mkpc$ ($6.46\times10^{16}\ \text{GeV}\ \text{cm}^{-5}$), was calculated by summing the pixel J-factor values within the core radius of Sgr reported in \citet{Majewski_et_al_2003}. To explain the recently observed $\gamma$-ray emission from Sgr documented in \citet{Crocker_and_Macias_et_al_2022} with the derived J-factor value for Sgr would require a DM annihilation cross section incompatible with existing constraints. In conjunction with the recently observed $\gamma$-ray signal from MSP sources in \citet{Crocker_and_Macias_et_al_2022}, this J-factor distribution militates against the use of Sgr for indirect DM detection experiments.
\\[10pt]
%Significance of projected distributions and results from projected distributions.
J-factor and $\rho^2$ distributions were derived to provide insight into the morphology of potential dark matter annihilation signatures as a potential discriminant to astrophysical sources. Within the Sgr core radius, the density of the simulated stellar markedly exceeds the DM density distribution, whilst both simulated stellar $\rho$ and DM $\rho^2$ distributions show unsurprisingly strong evidence of tidal disruption. The markedly extended nature of the J-factor distributions imply the extended nature of any $\gamma$-ray source associated with the Sgr DM halo. However, the relative magnitude of the J-factor distributions at the centre of the Sgr halo and at the core radius of \citet{Majewski_et_al_2003} indicate that the DM J-factor distribution is insufficiently peaked to explain the $\gamma$-ray emission from Sgr observed in \citet{Crocker_and_Macias_et_al_2022}, further indicating that DM annihilation is unable to explain the $\gamma$-ray emission observed in \citet{Crocker_and_Macias_et_al_2022}. Comparison of these distributions with fitted functions indicate low contamination from potentially spurious substructure in the simulated J-factor distributions.
\\[10pt]
%% Comparison to past works.
This calculated J-factor value for Sgr is lower than the J-factor values reported in prior studies (e.g \citealt{Viana_et_al_2012,Abramowski_et_al_2014}). To provide an indicative comparison between different J-factor definitions, analytic and various particle-based definitions were computed based on the simulated DM radial density profile. We show that the low J-factor value is due to differing density profiles for Sgr and a different J-factor definition, motivated by more accurate modelling of the tidally disrupted Sgr DM density profile.
\\[10pt]
%% Future work.
Whilst the computed J-factor distribution militates against the use of Sgr in indirect DM detection searches, in future work we plan %an upcoming paper will 
to 
determine the overall magnitude of the Sagittarius Stream J-factor and
investigate its morphological characteristics  with a view
to probing Fermi-LAT data for potential $\gamma$-ray products of annihilation from the Stream.
\section*{Acknowledgements}
T.A.A.V and RMC would like to 
thank D. Mackey
for his contributions to the initial stages of this project.
T.A.A.V would like to 
acknowledge C. Blake and A. Viana for helpful discussions. T.A.A.V. acknowledges the support of the Australian National University Research School of Astronomy and Astrophysics (RSAA) and the Centre for Astrophysics and Supercomputing at Swinburne University of Technology. 
%
RMC acknowledges support from the Australian Research Council through its \textit{Discovery Projects} funding scheme, awards DP190101258 and DP230101055.
%
O.M. was supported by the GRAPPA Prize Fellowship. T.T.G. acknowledges partial financial support from the Australian Research Council (ARC) through an Australian Laureate Fellowship awarded to J.~Bland-Hawthorn. This research was partially supported by the Australian Government through the Australian Research Council Centre of Excellence for Dark Matter Particle Physics (CDM, CE200100008). We acknowledge the facilities, and the scientific and technical assistance of the Sydney Informatics Hub (SIH) at the University of Sydney and, in particular, access to the high-performance computing facility Artemis and additional resources on the National Computational Infrastructure (NCI), which is supported by the Australian Government, through the University of Sydney’s Grand Challenge Program – the Astrophysics Grand Challenge: From Large to Small (CIs: G.~F.~Lewis and J.~Bland-Hawthorn). This research has made use of NASA’s Astrophysics Data System Bibliographic Services\footnote{https://ui.adsabs.harvard.edu}, the COLOSSUS\footnote{https://bdiemer.bitbucket.io/colossus/index.html} package \citep{Diemer_2018}, the Pyccl package and Astropy\footnote{http://www.astropy.org}, a community-developed core Python package and an ecosystem of tools and resources for astronomy \citep{astropy:2013, astropy:2018, astropy:2022}.
\subsection*{Data Availablity Statement}
No new data were generated or analysed in support of this research.
%%%%%%%%%%%%%%%%%%%%%%%%%%%%%%%%%%%%%%%%%%%%%%%%%%

%%%%%%%%%%%%%%%%%%%% REFERENCES %%%%%%%%%%%%%%%%%%

% The best way to enter references is to use BibTeX:

\bibliographystyle{mnras}
\bibliography{Sgr_paper_1} % if your bibtex file is called example.bib


% Alternatively you could enter them by hand, like this:
% This method is tedious and prone to error if you have lots of references
%\begin{thebibliography}{99}
%\bibitem[\protect\citeauthoryear{Author}{2012}]{Author2012}
%Author A.~N., 2013, Journal of Improbable Astronomy, 1, 1
%\bibitem[\protect\citeauthoryear{Others}{2013}]{Others2013}
%Others S., 2012, Journal of Interesting Stuff, 17, 198
%\end{thebibliography}

%%%%%%%%%%%%%%%%%%%%%%%%%%%%%%%%%%%%%%%%%%%%%%%%%%

%%%%%%%%%%%%%%%%% APPENDICES %%%%%%%%%%%%%%%%%%%%%
%%%%%%%%%%%%%%%%%%%%%%%%%%%%%%%%%%%%%%%%%%%%%%%%%%
\appendix
\section{Equivalence of the J-factor definition}
\label{sec:A1}
In section ~\ref{sec:M3}, the calculation of the dark matter J-factor distribution from the simulated dark matter particle distribution was detailed in equation~\ref{eq:EQ6}. This definition is equivalent to the J-factor definition of \citet{Charbonnier_et_al_2011}, which can be seen by converting equation~\ref{eq:EQ6} into spherical coordinates:
\begin{multline}
    J_b = \int_{V_b} \rho_{DM}^2/(4\pi r^2)\ d\text{V} = \int \int \rho_{DM}(l,\Omega)^2/(4\pi r^2) r^2\ d\text{r}d\text{$\Omega$} \\ = \int \int \rho_{DM}(r,\Omega)^2/(4\pi)\ d\text{r}d\text{$\Omega$}
    \label{EQ7}
\end{multline}
As detailed in section~\ref{sec:M3}, the differing factor of $4\pi$ from \citet{Charbonnier_et_al_2011} accounts for the surface area of the flux sphere for each particle.
\section{Scaled 1D $\rho^2$ profiles.}
\label{sec:A2}
As detailed in section~\ref{sec:R3}, $\rho^2$ and $\rho$ profiles were calculated for both DM and simulated stellar particle populations, respectively. These profiles were computed to illustrate morphological features of the simulated particle distributions to inform observational searches. This appendix shows normalized variants for these profiles with the fitted functions detailed in table~\ref{tab:Fits_summary_table}. Implications of these profiles are discussed in section~\ref{sec:R3}.
\\[10pt]
Figure~\ref{fig:DM_rho_squared_beta_profile_normalized} illustrates the Sgr DM $\rho^2$ profile as a function of $\beta_{\odot}$ with a fitted Moffat profile function. Figure~\ref{fig:DM_rho_squared_Lambda_profile_normalized} shows the DM $\rho^2$ profile as a function of $\Lambda_\odot$ with a fitted Offset Gaussian profile. Figure~\ref{fig:Sim_stars_rho_squared_beta_profile_normalized} details the simulated stellar $\rho$ profile as a function of $\beta_\odot$, whilst Figure~\ref{fig:Sim_stars_rho_squared_Lambda_profile_normalized} shows the simulated stellar $\rho$ profile as a function of $\Lambda_\odot$.
\begin{figure}
    \centering
    \includegraphics[width=\columnwidth]{Images/Results/DM_Rho_squared_profile_Beta_normalized.pdf}
    \caption{The DM $\rho^2$ profile as a function of $\beta_\odot$, scaled such that the maximum value is 1.0. The FWHM of the fitted Moffat profile is $4.96^\circ$.}
    \label{fig:DM_rho_squared_beta_profile_normalized}
\end{figure}
\begin{figure*}
    \centering 
    \includegraphics[width=2\columnwidth]{Images/Results/Sim_stars_Rho_profile_Beta_normalized.pdf}
    \caption{The simulated stellar $\rho$ profile as a function of $\beta_\odot$, scaled such that the maximum value is $1.0$. The FWHM of the fitted Voigt profile is $3.0^\circ$.}
    \label{fig:Sim_stars_rho_squared_beta_profile_normalized}
\end{figure*}
\begin{figure}
    \centering
    \includegraphics[width=\columnwidth]{Images/Results/DM_Rho_squared_profile_Lambda_normalized.pdf}
    \caption{The DM $\rho^2$ profile as a function of $\Lambda_\odot$, scaled with a maximum value of $1.0$, with the fitted offset Gausssian distribution. The FWHM of this offset Gaussian is $8.26^\circ$, with a vertical offset of 0.11 to account for the dark matter density of the Sagittarius Stream outside of Sgr core radius.}
    \label{fig:DM_rho_squared_Lambda_profile_normalized}
\end{figure}
\begin{figure*}
    \centering
    \includegraphics[width=2\columnwidth]{Images/Results/Sim_stars_Rho_profile_Lambda_normalized.pdf}
    \caption{The simulated stellar $\rho$ profile as a function of $\Lambda_\odot$, again scaled with the maximum value set as $1.0$. Also shown is the fitted Moffat distribution. The FWHM of this fitted Moffat distribution is $10.4^\circ$, significantly wider than the simulated stellar $\rho$ profile as a function of $\beta$ depicted in Figure\ref{fig:Sim_stars_rho_squared_beta_profile_normalized}. This is due to tidal disruption along the major axis ($\Lambda$) direction of Sgr.}
    \label{fig:Sim_stars_rho_squared_Lambda_profile_normalized}
\end{figure*}
% Don't change these lines
\bsp	% typesetting comment
\label{lastpage}
\end{document}

% End of mnras_template.tex