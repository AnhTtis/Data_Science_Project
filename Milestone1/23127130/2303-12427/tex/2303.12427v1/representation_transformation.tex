\documentclass[]{article}
%\documentclass[preprint,aps,prd,showpacs,superscriptaddress,nofootinbib,tightenlines]{revtex4}
%\documentclass[preprint,aps,prd,showpacs,superscriptaddress,nofootinbib]{revtex4}

\usepackage{fullpage}
\usepackage{authblk}
\usepackage{amsfonts}
\usepackage{multirow}
\usepackage{mathrsfs}
\usepackage{graphicx}
\usepackage{amsmath}
\usepackage{amssymb}
\usepackage{bm}
\usepackage{bbm}
\usepackage{color}
\usepackage{slashed}
\usepackage{diagbox}
\usepackage{ulem}
\usepackage{slashbox}
\usepackage{braket}
\usepackage{wrapfig}
\usepackage[hidelinks,colorlinks=true,linkcolor=blue]{hyperref}
\usepackage{cleveref}
\usepackage{fancyhdr}
\usepackage{subcaption}
\numberwithin{equation}{section}
\newcommand{\calF}{\mathcal{F}}
\newcommand{\calG}{\mathcal{G}}

%%%%%%%%%%%%%%%%%%%%%%%%%%%%%%%%%%%%%%%%%%
%Put your definitions here
%%%%%%%%%%%%%%%%%%%%%%%%%%%%%%%%%%%%%%%%%%
%Put your definitions here


\begin{document}
%%%%%%%%%%%%%%%%%%%%%%%%%%%%%%%%%%%%%%%%%%%%%%%%%%%%%%%%%%%%%%%%%%%%%%%%%%%%%%

%\fancypagestyle{plain}{
%    \fancyhead[R]{preprint number}
%    \renewcommand{\headrulewidth}{0pt}
%}


\title{\bf{A Representation Transformation of Parametric Feynman Integrals}}

\author[a]{\bf{Wen Chen}\footnote{chenwenphy@gmail.com}}
\affil[a]{\small{\it{School of Physics, Zhejiang University, Hangzhou, Zhejiang 310027, China}}}

\date{\small{\today}}

\maketitle


%%%%%%%%%%%%%%%%%%%%%%%%%%%%%%%%%%%%%%%%%%%%%%%%%%%%%%%%%%%%%%%%%%%%%%%%%%%%%%
\begin{abstract}

A transformation is applied to the parametric representation of Feynman integrals. The obtained representation and the original parametric representation are dual to each other. For integrals with momentum-space correspondences, the ``new'' representation is just the Baikov's representation.

\end{abstract}
%%%%%%%%%%%%%%%%%%%%%%%%%%%%%%%%%%%%%%%%%%%%%%%%%%%%%%%%%%%%%%%%%%%%%%%%%%%%%%
%%\pacs{\it}

%\tableofcontents

\section{Introduction}

Calculation of Feynman integrals is essential in high-energy physics. There are several representations of Feynman integrals. The most famous one is the Feynman-parameter representation~\cite{Feynman:1949zx,Nambu:1957shl,Nakanishi:1957,Symanzik:1958}. Some variants of the Feynman-parameter representation are the Lee-Pomeransky representation~\cite{Lee2013,Lee:2014tja} and the one used by the author in refs.~\cite{Chen:2019mqc,Chen:2019fzm,Chen:2020wsh}. Another now widely used representation is the Baikov's representation~\cite{Baikov:1996rk,Baikov:1996iu}. Obviously these two representations should be equivalent, since both of them are derived from the momentum representation. But it was not clear how to directly derive one representation from another. The first step towards solving this problem is made in this article. The representation used in refs.~\cite{Chen:2019mqc,Chen:2019fzm,Chen:2020wsh} and the Baikov's representation are characterized by two polynomials respectively. It is shown that these two polynomials are related to each other through a transformation. The transformation is described in \cref{sec:transf}, and the equivalence of the obtained representation with the Baikov's representation is proved in \cref{sec:Baik}.

\section{Representation transformation}\label{sec:transf}
%
As was shown in refs.~\cite{Chen:2019mqc,Chen:2019fzm,Chen:2020wsh}, a Feynman integral can be parametrized by an integral of the following form.
%
\begin{equation}\label{eq:ParInt}
\begin{split}
I(\lambda_0,\lambda_1,\ldots,\lambda_n)=&\frac{\Gamma(-\lambda_0)}{\prod_{i=m+1}^{n+1}\Gamma(\lambda_i+1)}\int \mathrm{d}\Pi^{(n+1)}\calF^{\lambda_0}\prod_{i=1}^{m}x_i^{-\lambda_i-1}\prod_{i=m+1}^{n+1}x_i^{\lambda_i}\\
\equiv&\int \mathrm{d}\Pi^{(n+1)}\mathcal{I}^{(-n-1)}.
\end{split}
\end{equation}
%
Here $\calF$ is a homogeneous polynomial of the Feynman parameters $x_i$. The integration measure $\mathrm{d}\Pi^{(n+1)}$ is defined by $\mathrm{d}\Pi^{(n+1)}=\prod_{i=1}^{n+1}\mathrm{d}x_i~\delta(1-\mathcal{E}^{(1)}(x))$, with $\mathcal{E}^{(i)}(x)$ a positive definite homogeneous function of $x$ of degree $i$. The parametric integrals satisfy the following identities.
%
\begin{subequations}\label{eq:IBP}
\begin{align}
0=&\int \mathrm{d}\Pi^{(n+1)}\frac{\partial}{\partial x_i}\mathcal{I}^{(-n-1)},&& i=1, 2,\ldots, m,\\
0=&\int \mathrm{d}\Pi^{(n+1)}\frac{\partial}{\partial x_i}\mathcal{I}^{(-n-1)}+\delta_{\lambda_i0}\int \mathrm{d}\Pi^{(n)}\left.\mathcal{I}^{(-n)}\right|_{x_i=0},&& i=m+1, m+2,\ldots, n+1.
\end{align}
\end{subequations}


We apply the following transformation to the $\mathcal{F}$ polynomial.
%
\begin{subequations}\label{eq:transf}
\begin{align}
&u_i\equiv\frac{1}{\calF}\frac{\partial\calF}{\partial x_i},\label{eq:transf1}\\
&\calG(u)\equiv\left.\frac{1}{\calF}\right|_{x=x(u)}.\label{eq:transf2}
\end{align}
\end{subequations}
%
$\calF$ is homogeneous in $x_i$. We assume that the degree of $\calF$ is $L+1$. Then obviously
%
\begin{equation}
\sum_{i=1}^{n+1}u_ix_i=L+1.
\end{equation}
%
Taking the derivative of both sides with respect to $u_i$, we get
%
\begin{equation*}
x_i+\sum_{j=1}^{n+1}u_j\frac{\partial x_j}{\partial u_i}=0.
\end{equation*}
%
On the other hand,
%
\begin{equation*}
\begin{split}
\frac{1}{\calG}\frac{\partial\calG}{\partial u_i}=\calF\frac{\partial}{\partial u_i}\left(\frac{1}{\calF}\right)=-\frac{1}{\calF}\sum_{j=1}^{n+1}\frac{\partial\calF}{\partial x_j}\frac{\partial x_j}{\partial u_i}=-\sum_{j=1}^{n+1}u_j\frac{\partial x_j}{\partial u_i}.
\end{split}
\end{equation*}
%
A combination of the above two equations leads to
%
\begin{equation}
x_i=\frac{1}{\calG}\frac{\partial\calG}{\partial u_i}.
\end{equation}
%
Thus we see that the transformation defined by \cref{eq:transf} is reciprocal.

Generally speaking, the transformation defined by \cref{eq:transf1} may not be invertible. Even if it is, the obtained function $\calG(u)$ may not be a rational function. However, for a loop integral with a complete set of propagators, $\calG$ is indeed a rational function, as will be clear in the next section.

\section{Baikov's representation}\label{sec:Baik}
%
We consider a loop integral with a complete set of propagators
%
\begin{equation}
v_i=\sum_{j,k}A_{i,jk}l_j\cdot l_k+2\sum_{j,k} B_{i,jk}l_j\cdot p_k+C_i,\quad i=1,~2,\dots,n,
\end{equation}
%
where $l$ are the loop momenta and $p$ are the external momenta. For simplicity, I assume that all the propagators are linearly independent and the Gram determinant of the external momenta $p$ does not vanish. For future convenience, we collectively denote $l$ and $p$ by $q$. Scalar integrals can be parametrized by integrals of the form~\cite{Baikov:1996rk,Baikov:1996iu}
%
\begin{equation}\label{eq:Baik}
J(\lambda_0,\lambda_1,\ldots,\lambda_n)=\int\prod_{i=1}^n\mathrm{d}v_n~P^{-\lambda_0-\frac{1}{2}(L+E+1)}\prod_{i=1}^nu^{-\lambda_i-1},
\end{equation}
%
where $E$ is the number of the external momenta $p$. Here I have omitted some irrelevant prefactors for brevity. The polynomial $P(v)$ is the Gram determinant of the momenta $q$. That is
%
\begin{equation}
P(v)\equiv\det(q_i\cdot q_j).
\end{equation}
%
In this section, I will prove that
%
\begin{equation}\label{eq:G2P}
P(v_1,v_2,\dots,v_n)=-\det(p_i\cdot p_j)\calG(v_1,v_2,\dots,v_n,-1),
\end{equation}
%
where $\calG$ is the one defined in \cref{eq:transf2}. The proof of \cref{eq:G2P} is as follows.

We define
%
\begin{align*}
A_{ij}\equiv&\sum_{k=1}^nA_{k,ij}x_k,\\
B_{ij}\equiv&\sum_{k=1}^nB_{k,ij}x_k,\\
C\equiv&\sum_{i=1}^nC_ix_i.
\end{align*}
%
Then we have~\cite{Chen:2019mqc}
%
\begin{equation}\label{eq:Fforloop}
\calF=\sum_{i,j,k,l}p_k\cdot p_l\tilde{A}_{ij}B_{ik}B_{jl}+(x_{n+1}-C)U,
\end{equation}
%
where
\begin{align*}
U\equiv&\det(A_{ij}),\\
\tilde{A}_{ij}\equiv&\left(A^{-1}\right)_{ij}U.
\end{align*}
%
Let $a_{ij}$, $b_{ij}$, $\alpha_{ij,k}$, and $\beta_{ij}$ be those defined in ref.~\cite{Chen:2019fzm}. That is, they are solutions to the following equations.
%
\begin{align*}
&\sum_la_{il}B_{l,jk}=0,\\
&\sum_lb_{il}A_{l,jk}=0,\\
&\sum_{kl}\alpha_{ij,k}a_{kl}A_{l,mn}=\frac{1}{2}\left(\delta^{im}\delta^{jn}+\delta^{in}\delta^{jm}\right),\\
&\sum_{kl}\beta_{ij,k}b_{kl}B_{l,mn}=\delta^{im}\delta^{jn}.
\end{align*}
%
Then it can be proven that (see \cref{app:proof1})
%
\begin{equation}\label{eq:Finu}
\calG=u_{n+1}\det\left[\alpha_{ij,k}a_{kl}(u_l+u_{n+1}C_l)+ u_{n+1}g_{mn}B^\prime_{im}B^\prime_{jn}\right],
\end{equation}
%
where $u_i$ are those appeared in \cref{eq:transf}, $g_{ij}$ is the inverse of the Gram matrix $p_i\cdot p_j$, and
%
\begin{equation*}
B^\prime_{ij}(u)\equiv\frac{1}{2u_{n+1}}\sum_{k,l}\beta_{ij,k}b_{kl}\left(u_l+C_lu_{n+1}\right).
\end{equation*}

On the other hand, $q_i\cdot q_j$ can be understood as a block matrix. Then we have
%
\begin{align*}
P=&\det\begin{pmatrix}
l_i\cdot l_j&l_i\cdot p_j\\
l_i\cdot p_j&p_i\cdot p_j
\end{pmatrix}\\
=&\det(p_i\cdot p_j)\det\left(l_i\cdot l_j-\sum_{mn}g_{mn}l_i\cdot p_ml_j\cdot p_n\right).
\end{align*}
%
It is easy to get
\begin{align*}
l_i\cdot p_j=&\frac{1}{2}\sum_{k,l}\beta_{ij,k}b_{kl}(v_l-C_l)\equiv-B^{\prime\prime}_{ij},\\
l_i\cdot l_j=&\alpha_{ij,k}a_{kl}(v_l-C_l).
\end{align*}
%
Thus
%
\begin{equation}\label{eq:Pinu}
P=\det(p_i\cdot p_j)\det\left(\alpha_{ij,k}a_{kl}(v_l-C_l)-\sum_{mn}g_{mn}B^{\prime\prime}_{im}B^{\prime\prime}_{jn}\right).
\end{equation}
%
Comparing \cref{eq:Finu} with \cref{eq:Pinu}, we get \cref{eq:G2P}.

%Notices that the constraints $u_{n+1}=1$ can naturally arise if we choose $\mathcal{E}^{(1)}=u_{n+1}^{-1}$ for the measure $\mathrm{d}\Pi^{(n+1)}$ in \cref{eq:ParInt}.

\section{Discussion}\label{sec:disc}
%
The parametric representation is characterized by a polynomial $\calF(x)$. A transformation is applied to $\calF$ and we get a function $\calG(u)$ (cf. \cref{eq:transf}). This transformation is reciprocal. Thus $\calF$ and $\calG$ are dual to each other. Similarly, the Baikov's representation is characterized by a polynomial $P$. It is shown that $P$ equals $G|_{u_{n+1}=-1}$ up to a constant (cf. \cref{eq:G2P}).

In order to fully prove the equivalence of the parametric representation in \cref{eq:ParInt} with the Baikov's representation in \cref{eq:Baik}, it remains to express $x^{\lambda}$ in terms of $u^{-\lambda-1}$. I believe that this should be a consequence of \cref{eq:IBP}, but currently I can not figure out the details. So I prefer to leave it for the future.

\section*{\normalsize{Acknowledgments}}

This work is supported by the Natural Science Foundation of China (NSFC) under the contract No. 11975200.

\appendix
\section{Derivation of eq.~(\ref{eq:Finu})}\label{app:proof1}
%
By virtue of \cref{eq:Fforloop}, we have (In this section, summations over repeated indices are always implied.)
%
\begin{align*}
\beta_{ij,k}b_{kl}u_l\calF=&\beta_{ij,k}b_{kl}\frac{\partial\calF}{\partial x_l}\\
=&2\tilde{A}_{ik}B_{kl}p_l\cdot p_j-\beta_{ij,k}b_{kl}C_lU\\
=&2\left(A^{-1}\right)_{ik}B_{kl}p_l\cdot p_ju_{n+1}\calF-\beta_{ij,k}b_{kl}C_lu_{n+1}\calF.
\end{align*}
%
Thus we have
%
\begin{equation*}
B_{ij}=A_{ik}g_{jl}B^\prime_{kl}.
\end{equation*}
%
Similarly,
%
\begin{align*}
\alpha_{ij,k}a_{kl}u_l\calF=&\alpha_{ij,k}a_{kl}\frac{\partial\calF}{\partial x_l}\\
=&-U\left[p_m\cdot p_n\left(A^{-1}\right)_{ik}\left(A^{-1}\right)_{jl}B_{km}B_{ln}+\alpha_{ij,k}a_{kl}C_l\right]\\
&+\tilde{A}_{ij}\left[p_k\cdot p_l\left(A^{-1}\right)_{mn}B_{mk}B_{nl}+x_{n+1}-C\right]\\
=&-\calF u_{n+1}\left(g_{mn}B^\prime_{im}B^\prime_{jn}+\alpha_{ij,k}a_{kl}C_l\right)+\left(A^{-1}\right)_{ij}\calF.
\end{align*}
%
That is
%
\begin{equation*}
\left(A^{-1}\right)_{ij}=\alpha_{ij,k}a_{kl}(u_l+u_{n+1}C_l)+ u_{n+1}g_{mn}B^\prime_{im}B^\prime_{jn}.
\end{equation*}
%
Taking the determinant of both sides, we get
%
\begin{equation*}
\calG=\calF^{-1}=u_{n+1}U^{-1}=u_{n+1}\det\left[\alpha_{ij,k}a_{kl}(u_l+u_{n+1}C_l)+ u_{n+1}g_{mn}B^\prime_{im}B^\prime_{jn}\right].
\end{equation*}

\bibliographystyle{JHEP}
\bibliography{representation_transformation}

\end{document}




