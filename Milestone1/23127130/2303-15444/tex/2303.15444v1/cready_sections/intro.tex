\section{Introduction}\label{sec:intro} 

Since the data volumes that AI technologies are required to process are continuously growing every year, the
pressure to devise more powerful hardware solutions is also increasing. To face this challenge, a promising direction currently pursued both in academic research labs and in leading companies is to exploit the potential
of quantum computing.
Such paradigm leverages quantum mechanical effects for computations and optimization, accelerating several important problems such as prime number factorization, database search \cite{Nielsen2010} and combinatorial optimization \cite{mcgeoch2014adiabatic}. %
The reason for such acceleration is that quantum computers leverage quantum parallelism of qubits, \textit{i.e.}, the property that a quantum system can be in a superposition of multiple (exponentially many) states and perform calculations simultaneously on all of them. 
\begin{figure}[t!]
    \centering 
    \includegraphics[width=\linewidth]{imgs/teaser.pdf} 
    \caption{
    \emph{Left:} Differences between our method and \suter \cite{Doan_2022_CVPR}. While \suter considers only a single model and is tested on quantum hardware with synthetic data, \ourmethod is also evaluated on real data on real quantum hardware. Although devised for multiple models, our method supports a single model likewise.  \emph{Right:} Qualitative results of \ourmethod on motion segmentation on the AdelaideRMF dataset \cite{wong2011dynamic}.
    \vspace{-2em}
    } 
 \label{fig_suter}
    \label{fig:teaser} 
\end{figure} 

Among the two quantum computing models, \textit{i.e.}, 
gate-based and Adiabatic Quantum Computers (AQCs), the latter recently gained attention in the computer vision community thanks to advances in experimental hardware realizations \cite{QuantumSync2021, SeelbachBenkner2021, Meli_2022_CVPR, Doan_2022_CVPR, Zaech_2022_CVPR, Yang_2022_CVPR, Arrigoni2022}. 
{At the present, AQCs provide sufficient resources in terms of the number of qubits, qubit connectivity and admissible problem sizes which they can tackle \cite{Boothby2020}, to be applied to a wide range of problems in computer vision.}
The AQC model is based on the adiabatic theorem of quantum mechanics \cite{BornFock1928} and designed for combinatorial problems (including $\mathcal{NP}$-hard) that are notoriously difficult to solve on classical hardware. 
Modern AQCs operate by optimizing objectives in the \emph{quadratic unconstrained binary optimization} (QUBO) form (see Sec.~\ref{sec:background}). However, many relevant tasks in computer vision cannot be trivially expressed in this form. 
Hence, currently, two prominent research questions in the field are: 1) \textit{Which problems in computer vision could benefit from an AQC?}, and 2) \textit{How can these problems be mapped to a QUBO form in order to use an AQC?} 

Several efforts have been recently undertaken to bring classical vision problems in this direction. 
Notable examples are works on graph matching \cite{SeelbachBenkner2020,SeelbachBenkner2021}, multi-image matching \cite{QuantumSync2021}, point-set registration \cite{golyanik2020quantum, Meli_2022_CVPR}, object detection \cite{LiGhosh2020},  multi-object tracking \cite{Zaech_2022_CVPR}, motion segmentation \cite{Arrigoni2022} and robust fitting \cite{Doan_2022_CVPR}.
Focusing on geometric model fitting, Doan \textit{et al.}~\cite{Doan_2022_CVPR} proposed an iterative consensus maximization approach to robustly fit a \emph{single} geometric model to noisy data. 
This work was the first to demonstrate the advantages of 
quantum hardware in robust single-model fitting with error bounds, which is an important and challenging problem with many applications in computer vision (\textit{e.g.,} template  recognition in a point set). 
The authors proposed to solve a series of linear programs on an AQC and demonstrated promising results on synthetic data. 
They also showed experiments for fundamental matrix estimation and point triangulation using simulated annealing (SA) \cite{Kirkpatrick1983}. 
SA is a classical global optimization approach, that, in contrast to AQCs, can optimize arbitrary objectives and is a frequent choice when evaluating quantum approaches (see Sec.~\ref{sec:background}). 
\suter \cite{Doan_2022_CVPR} takes advantage of the hypergraph formalism to robustly fit a single model. It is not straightforward to extend it to the scenario where \emph{multiple} models are required to explain the data.

Multi-model fitting (MMF) is a relevant problem in many applications, such as 3D reconstruction, where it is employed to fit multiple rigid moving objects to initialize multi-body Structure from Motion \cite{OzdenSchindlerAl10,ArrigoniRicciAl22}, or to produce intermediate interpretations of reconstructed 3D point clouds by fitting geometric primitives\cite{MagriLeveniAl21}. 
Other scenarios include face clustering, body-pose estimation, augmented reality and image stitching, to name a few.

This paper proposes \ourmethod, \textit{i.e.,} the first quantum 
MMF approach. 
We propose to leverage the advantages of AQCs in optimizing combinatorial QUBO objectives to explain the data with \emph{multiple} and \emph{disjoint} geometric models. 
Importantly, \ourmethod does not assume the number of disjoint models to be known in advance.
Note that the potential benefit from AQCs for MMF is higher than in the single-model case: when considering multiple models the search space scales exponentially with their number, making the combinatorial nature of the problem even more relevant.
Furthermore, we show that \ourmethod can be easily applied to single-model fitting even though not explicitly designed for this task. 
We perform an extensive experimental evaluation on quantum hardware with many large-scale real datasets and obtain competitive results with respect to %
both classical and quantum methods. 
Figure \ref{fig:teaser} depicts a visual comparison between \suter \cite{Doan_2022_CVPR} and \ourmethod. 

\smallskip
\textbf{Contributions.} In summary, the primary technical contributions of this paper are the following: 
\begin{itemize}[topsep=0.1em, itemsep=0.05em] 
    \item We bring multi-model fitting, a fundamental computer vision problem with combinatorial nature, into AQCs; 
    \item We introduce \ourmethod, demonstrating that it can be successfully used both for single and multiple models;
    \item We propose \ourmethoddec, a decomposition policy allowing our method to scale to large-scale problems, %
    overcoming the limitations of modern quantum hardware. 
\end{itemize}
The following section provides the background on AQCs and how to use them to solve QUBO problems. 
After introducing \ourmethod and \ourmethoddec in Sec.~\ref{sec:proposed-methodology}, we discuss related work in Sec.~\ref{sec:related-work}. Experiments are given in Sec.~\ref{sec:experimental-results}. Limitations and Conclusion are reported in Sec.~\ref{sec:limitations} and \ref{sec:conclusion}, respectively.
