\section{Related Work} 
\label{sec:related-work} 

The proposed approach is the first to address multi-model fitting using AQC. The most related work is \suter \cite{Doan_2022_CVPR} that leverages quantum computing as well for the scenario of single-model fitting.

\subsection{Multi-Model Fitting} 
\label{sec-related-mmf}

Robust fitting is a central problem in computer vision, and the case of multiple models counts several works that can be organized across two main directions, namely \emph{clustering-based}  and \emph{optimization-based} methods. 

Clustering-based methods cast multi-model fitting as a clustering task where %
points belonging to the same model must be grouped together. Several procedural algorithms have been presented leveraging different schemes: 
hierarchical clustering \cite{toldo2008robust,MagriFusiello14,MagriFusiello19,ZhaoZhangAl20,MagriLeveniAl21},  kernel fitting \cite{ChinWangAl09,ChinSuterAl10}, robust matrix factorization \cite{MagriFusiello17,TepperSapiro17}, biclustering \cite{TepperSapiro14,DenittoMagriAl16}, higher order clustering \cite{AgarwalJongowooAl05,Govindu05, JainGovindu13,ZassShashua05} and hypergraph partitioning \cite{PurkaitChinAl14, WangXiaoAl15,XiaoWangAl16,WangGupbao18,LinXiao19}.

Our approach falls instead in the category of optimization-based methods. Compared with procedural approaches based on clustering, methods based on the optimization of a precise objective function 
provide a quantitative criterion for evaluating the quality of the retrieved solution. 
In particular, our \ourmethod belongs to consensus-based approaches that generalize \textsc{RanSaC} by maximizing the consensus of the sought models.
Sequential \textsc{RanSaC} \cite{VincentLaganiere01}, Multi-\textsc{RanSaC}  \cite{zuliani2005multiransac} and \ransacov \cite{magrifusiello16} belong to this category as well. Other sophisticated techniques, that integrate consensus with additional priors, have also been proposed: Pearl \cite{IsackBoykov12}, 
Multi-X \cite{BarathMatas17} and  Prog-X\cite{BarathMatas19}. 

Our approach is related to \ransacov \cite{magrifusiello16} as we consider the same data structure, \textit{i.e.,} the preference matrix, and a similar optimization objective\footnote{The same objective is considered but constraints are different: \ransacov \cite{magrifusiello16} allows for overlapping models, resulting in an inequality constraint, see Problem \eqref{eq:set-cover}; instead, we consider disjoint models, using an equality constraint, see Problem \eqref{eq:disjoint-set-cover}. 
For a fair comparison, we slightly modify \ransacov, replacing the inequality with an equality constraint.}. 
However, there are substantial differences. 
\ransacov addresses the task via integer linear programming, as Magri and Fusiello \cite{magrifusiello16} use the \textsc{Matlab} function \texttt{intlinprog}, which employs linear programming relaxation followed by rounding the found solution, in a combination with some heuristics.
In case of failure it reduces to branch and bound. 
Our approach, instead, leverages quantum effects to optimize the objective directly in the space of qubits, where global optimality is expected with high probability after multiple anneals. 

\subsection{Quantum Computer Vision}
\label{sec:related_quantum}

Recently, many vision problems have been formulated for an AQC in QUBO forms, hence giving rise to the new field of Quantum Computer Vision (QCV).
AQC are especially promising for problems involving combinatorial optimization (see Sec.~\ref{sec:background}) such as 
graph matching \cite{SeelbachBenkner2020,SeelbachBenkner2021}, multi-image matching via permutation synchronization \cite{QuantumSync2021}, single-model fitting \cite{Doan_2022_CVPR}, multi-object tracking \cite{Zaech_2022_CVPR}, removing redundant boxes in object detection \cite{LiGhosh2020}, and motion segmentation \cite{Arrigoni2022}. 
Moreover, solving non-combinatorial problems 
on quantum hardware is of high general interest as well, since new formulations have unique properties, can be compact and would allow seamless integration with other quantum approaches in future. 
Thus, several methods address point set alignment  \cite{golyanik2020quantum, Meli_2022_CVPR} and approximate rotation  matrices by the exponential map with power series. 

Depending on the problem, QUBO formulation can be straightforward  \cite{LiGhosh2020,SeelbachBenkner2020,Zaech_2022_CVPR} or require multiple analytical steps 
bringing 
the initial objective to the quadratic form \cite{QuantumSync2021,Doan_2022_CVPR,Arrigoni2022}. 
In most cases, problem-specific constraints must be included in a QUBO as soft linear regularizers, 
as done in~\eqref{eq:qubo_soft}. 
This strategy is followed by the majority of works in the literature, including \cite{SeelbachBenkner2020,QuantumSync2021,Zaech_2022_CVPR,Arrigoni2022,Doan_2022_CVPR} and our approach.
Alternatively, it has been recently shown that it is possible to tackle linearly-constrained QUBO problems via the Frank-Wolfe algorithm \cite{YurtseverBirdalAl22}, which alleviates the need for hyper-parameter tuning, at the price of iteratively solving multiple QUBO problems and sub-linear convergence. 

All available approaches can be classified into single QUBO (``one sweep'') and iterative methods.
The first category prepares a single QUBO that is subsequently sampled on an AQC to obtain the final solution  \cite{LiGhosh2020,golyanik2020quantum,SeelbachBenkner2020, QuantumSync2021, Arrigoni2022}. These methods can solve comparably small problems \emph{on current hardware}: in the recent QuMoSeg \cite{Arrigoni2022}, the maximum problem size solved with high accuracy has 200 qubits, which correspond to five images with two motions having ten points each. 
In QuantumSync\cite{QuantumSync2021}, real experiments involve matching four keypoints in four images only. 
The second category of methods, instead, alternates between QUBO preparation on a CPU and QUBO sampling on an AQC, until convergence or a maximum number of iterations is reached \cite{SeelbachBenkner2021, Meli_2022_CVPR, Doan_2022_CVPR}. 
This is done either to overcome the limit on the maximum problem size solvable on current quantum hardware \cite{SeelbachBenkner2021, Meli_2022_CVPR} or because the QUBO represents only a small portion of the whole problem, hence requiring additional (classical) steps as in Q-Match \cite{SeelbachBenkner2021} and \suter  \cite{Doan_2022_CVPR}. Closely related to iterative methods are decompositional approaches (e.g., \cite{Zaech_2022_CVPR}) that partition the original objective into QUBO subproblems. 
This paper proposes both a single QUBO method (see  Sec.~\ref{sec:mmf-qubo-formulation}) and a decompositional, iterative pruning approach (see Sec.~\ref{sec:id-qmmf}). 

\textbf{Differences to \suter.} 
Most related to ours is a hybrid quantum-classical approach for robust fitting, \suter \cite{Doan_2022_CVPR}, that can be regarded as a pioneer in bringing robust fitting into a QUBO-admissible form, thus paving the way to the adoption of AQCs for such a class of hard combinatorial problems.
\suter and our approach have substantial differences. 
First, although being both based on linear programs, the two formulations do not share common steps: \suter considers the hypergraphs formalism, relying on multiple QUBOs in an iterative framework; our formulation, instead, is more compact as it involves a single QUBO in its one-sweep version. 
Second, \suter considers a \emph{single} model, whereas we consider the more general case of \emph{multiple} models, which is relevant in practical applications and more challenging in terms of the search space and the number of unknowns involved. 
Extending Doan \textit{et al.}~\cite{Doan_2022_CVPR} to MMF is not  straightforward (it is not clear so far whether it is possible). Finally, \suter is evaluated with an AQC on \emph{synthetic} data only, whereas we test our method on several {synthetic and} \emph{real} datasets on real quantum hardware, as shown in Sec.~\ref{sec:experimental-results}. Fig.~\ref{fig_suter} summarizes the core differences between \suter \cite{Doan_2022_CVPR} and our approach. 

{%
\textbf{Remark.} Finally, it is also worth mentioning the technique for robust fitting (single model) by Chin \textit{et  al.}~\cite{ChinSuterAl20} 
evaluated on a simulator and designed for a different type of quantum hardware than AQC, \textit{i.e.,} a gate-based quantum computer (GQC). 
GQCs operate differently from AQCs, \textit{i.e.,} they execute unitary transformations on quantumly-encoded data, and currently possess a much smaller number of qubits compared to AQCs \cite{Jurcevic2021}. 
We focus on methods that can run on real quantum hardware.
Unfortunately, modern AQCs cannot run algorithms designed for GQCs. 
}
