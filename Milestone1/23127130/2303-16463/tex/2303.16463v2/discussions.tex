\subsection{Case Studies}

\paragraph{Full disk encryption}
Full disk encryption~(FDE) protects the confidentiality and integrity of data
at-rest.
To prevent accidental disclosure of the secret key (e.g., disk encryption
key), it is a standard practice to encrypt the secret key~(\emph{wrap}
operation) such that it can be decrypted only by the
TPM~(\emph{unwrapping}).
The wrapping key~(i.e., the key which wraps the secret) is often the
storage root key (SRK) present in the TPM.

However, in our ephemeral \vtpm{}, there are no persistent storage keys in
the TPM to support unwrapping of keys.
~\autoref{fig:fde} shows the steps involved in supporting FDE on an
ephemeral \vtpm{}.
To support FDE, we create an intermediary
storage key $K_{iSK}$~(
a restricted decryption key with
sensitiveDataOrigin~\cite{tpm:sensitive-data-origin}).
Now, we perform a TPM \emph{seal} operation on the disk encryption key by
parenting it to the storage key ($K_{iSK}$) we just created, outputting a
sealed blob which can be unsealed only by a TPM with the same key.
On platform boot, the \vtpm{} would generate an ephemeral endorsement key
($e_{Ek}$) and an ephemeral storage root key~($eSRK$).
By retrieving the public part of the eSRK~(~\circlednum{1} in
~\autoref{fig:fde}), we can wrap the intermediary key
$K_{iSK}$ with $eSRK_{pub}$ to create a wrapped key that can be decrypted
only by our \vtpm{}~(~\circlednum{2} in \autoref{fig:fde}).
It has to be noted that all the above operations can be performed on any
TPM, i.e., the user need not necessarily perform these on the \vtpm{} of
the \cvm{}.
Now, the disk encryption key is wrapped to the parent key and the parent is
in turn wrapped to the eSRK, forming a hierarchy under the ephemeral storage
root key~(\circlednum{3} in \autoref{fig:fde}).
It is also possible to wrap the parent key with $EK_{pub}$ instead to
create a hierarchy under the ephemeral endorsement key.

As both the disk encryption key and its parent key are wrapped for our
specific \vtpm{}, they are no longer a secret and can be delivered to the
\cvm{} in the clear.
Since the sealed disk encryption key is invariant, we can embed this into
the initrd.
Finally, we can deliver the wrapped parent key~(\circlednum{2} in \autoref{fig:fde})
to the \cvm{} once we have performed the initial attestation of the
platform to ensure its trustworthiness.




\begin{figure}
  \centering
  \includegraphics[width=0.89\columnwidth]{figures/fde}
  \caption{Full disk encryption in an ephemeral vTPM}
  \label{fig:fde}
  \vspace{-4mm}
\end{figure}

\paragraph{Storing secrets}
We cannot store secrets directly by wrapping the keys on our ephemeral
\svtpm{} as the EK and SRK would be newly generated on every boot.
One could use a similar technique we used for FDE to form a hierarchy of
keys under an intermediary storage key.
Once the system is booted, we can parent the intermediary key to the
ephemeral SRK or EK forming a hierarchy under the chosen key.
Using this technique, one could store a hierarchy of keys, as we do with a
regular persistent TPM.

