
\section{Introduction}



Over the last two decades, public clouds have become an inescapable building block
of virtually every modern application. 
The move to the cloud created a unique security challenge. 
Both application vendors and end-users are required to trust the cloud 
infrastructure that is often in charge of handling security and privacy-sensitive data. 
Such trust is fragile as multi-tenant cloud environments are operated by third
party providers and include a large and complex virtualization and storage
stacks optimized for a wide variety of hardware and software execution
scenarios. 
Unfortunately, vulnerabilities in critical cloud software and infrastructure
are unavoidable.


In the last decade, three widely deployed virtual machine monitors (VMMs) --
Xen, KVM, and VMware -- that provide the foundation of isolation and security in the
cloud suffered from 428~\cite{cves:xen}, 111~\cite{cves:kvm}
and 154~\cite{cves:vmware} vulnerabilities each.
Cloud software stacks like Openstack and Cloudstack suffer from several
vulnerabilities, some resulting in total information disclosure and rendering
resources unusable~\cite{cves:openstack, cves:cloudstack}.
Moreover, physical access to the system opens the door for a range of hardware
attacks, e.g., memory extraction such as cold-boot~\cite{cold-boot-attack},
RAMBleed~\cite{rambleed, rambleed-openssl}, etc.






In an effort to minimize the TCB of cloud applications,
hardware vendors and some cloud providers have introduced support for
hardware-protected trusted execution environments
(TEEs)~\cite{intel-sgx-explained, wp:amd-sev, spec:intel-tdx, wp:arm-cca, ibm-pef}.
TEEs protect data in use from the host software stack including the hypervisor
and even the physical attacker. 
In effect, TEEs remove the cloud provider from the TCB, even though the
provider still manages the lifecycle of an application. 



Isolation alone, however, is not sufficient to protect a workload or sensitive
data. 
To ensure integrity, modern systems rely on a combination of \emph{measured
boot}~\cite{mboot:uefi, mboot:bootstrapping-trust} and \emph{runtime
attestation}~\cite{ra:principles, ra:semantic}.
A measured boot protocol performs measurement of all binaries involved in the
boot of the system to ensure the integrity of all boot-time components, i.e., the platform
firmware, bootloader(s), and the operating system kernel.
Runtime attestation combines measured boot with integrity measurement architecture
(IMA) that ensures integrity measurements of all binaries loaded and executed by
the system after it booted, i.e., dynamic kernel extensions, system binaries, etc.
Attestation works by comparing entries in the measured boot and
IMA logs with a pre-defined set of acceptable values (called an
\emph{attestation policy}) and exposing any measurements that do not
conform to policy expectations.



Support for attestation requires a \emph{root-of-trust device}, i.e.,
an integrity-protected location that can store measurements in a
trustworthy manner, extend them, and authenticate the measurement logs to the user
(remote attestation).
On a physical machine, a trusted platform module (TPM) chip
can be used as the root of trust.
Some cloud providers offer virtual machines with virtual TPMs (vTPMS)
attached to them ~\cite{vtpm:gcp, vtpm:azure, vtpm:aws-nitro,vtpm:alibaba}.
These vTPMs, however, are emulated by the host
virtualization stack. Using this kind of emulated device requires
trusting the service provider, which is at odds with confidential computing.
In this paper, we show how to implement a confidential vTPM emulated inside
a TEE, isolated from both host and guest, linked to the root of trust of
the enclave, and providing similar properties to a physical TPM.

In this work, we design and implement a new virtual trusted platform module
(\vtpm{}) that virtualizes the hardware root-of-trust without requiring trust
in the cloud provider.  
To ensure the security of a \vtpm{} in a provider-controlled environment, we
leverage unique isolation properties of the \snp{} hardware that allows us to
execute secure services (such as \vtpm{}) as part of the enclave environment
protected from the cloud provider. 
We further develop a novel approach to the vTPM state management where the vTPM
state is not preserved across reboots.
Specifically, we develop a stateless \emph{ephemeral} \vtpm{} that supports
remote attestation without a persistent state on the host.
This allows us to pair each confidential VM with a private instance of a \vtpm{} that 
is completely isolated from the provider-controlled environment and other VMs. 


We design our \vtpm{} around the following security requirements: 

\begin{itemize} 
	
\item 
\textbf{Isolation}: 
Physical TPMs are isolated at the hardware level.
Typical vTPMs emulated on the host are isolated from the guest via
		virtualization, but exposed to the trusted host.
In addition, the \vtpm{} also needs isolation from the guest operating system,
		since it acts as a root-of-trust device for attestation.  
A \vtpm{} should be isolated from both the host and the guest system.

\item 
\textbf{Secure communication}: In a physical TPM, communication is isolated at
		the hardware level, although these assurances can sometimes be
		subverted~\cite{tpm-genie, pr:linux-tpm2-hmac}.
In a typical \vtpm{}, the TPM commands and responses are transmitted through
		the untrusted hypervisor~\cite{vtpm:berger, vtpm:xen-doma,
		vtpm:xen-domb, vtpm:gvtpm, vtpm:xen-libos}.
An attacker can interpose on the channel and alter the request or response
		defeating the security guarantees offered by a
		TPM~\cite{tpm-genie}.
Communication with \vtpm{} should be secure. 

\item 
\textbf{Persistent state}: Physical TPMs have a persistent identity that is set
		when the device is manufactured. 
Maintaining persistent state in a virtualized environment usually requires a
		centralized management system to propagate and store \vtpm{}
		state. 
The management system is part of the TCB and is usually managed by the cloud
		provider. 
\vtpm{}'s state should be managed by the client and protected from 
		the cloud provider. 		


\end{itemize}

To implement isolation, we leverage unique properties of the \snp{} execution environment. 
Our confidential vTPM is emulated inside the \snp{} enclave (hence it is 
isolated from the host and the cloud provider). 
Moreover, we leverage Virtual Machine Privilege Levels (VMPLs) to isolate vTPM 
from the guest and hence ensure the integrity of remote attestation. 
Since our confidential vTPM is emulated inside the guest security context, the
guest and vTPM can communicate in plaintext without information being exposed
to the untrusted host.
Moreover, we ensure that neither the guest nor the hypervisor can tamper 
with the communication. 
To avoid exposing sensitive vTPM state to a complex management system, we
develop a new  ephemeral approach to vTPM state management, in which the 
state of the \vtpm{} never leaves the protected enclave.

The above security properties allow us to implement a vTPM that is
comparable in security and functionality to a physical TPM. Our vTPM
does not violate the trust model of confidential computing and extends
existing measurement capabilities to support sophisticated attestation
flows, enabling the creation of cloud-native workloads with a small
TCB that can be rigorously audited.

Our work leverages the unique architectural properties of the AMD \snp{} execution
environment; however, we will discuss how to generalize this solution at the end
of the paper. We will also expand on the properties of the ephemeral vTPM,
which does have certain restrictions. The limitations of an ephemeral vTPM
do not affect the attestation usecases described here.




Our contributions are as follows:
\begin{itemize}[leftmargin=*, nosep, labelindent=\parindent]
\item We propose using an ephemeral \vtpm{} to remove attacks to the \vtpm{} state.
\item We are the first to leverage the new features of AMD \sev{} to provide a secure implementation of a \vtpm{}.
\item We demonstrate a complete remote attestation workflow for our \svtpm{} solution, implicitly proving that remote attestation frameworks can provide
measured boot and remote attestation with an ephemeral \vtpm{}.

\end{itemize}











