\documentclass[11pt,a4paper]{article}
%\usepackage{epsfig,jouni,amsmath,chicago}
\usepackage{epsfig,amsmath,amssymb}
\usepackage{url,color}
\usepackage[english]{babel}
\usepackage{natbib}
\usepackage{authblk}
%\usepackage{apacite}
\usepackage{hyperref}
\usepackage{geometry}
\usepackage{comment}

\setlength{\oddsidemargin}{10mm}
\topmargin 3mm
\headheight 2mm
\headsep 3mm
\textheight 232mm
\textwidth 158mm
%\footheight 10mm
\parindent 0mm
\parskip 12pt
% \emergencystretch=.5em
\def\indep{\perp\kern-0.50em \perp}
\def\var{\mbox{var}}
\def\E{\mbox{E}}
\def\P{\mbox{P}}
\def\cov{\mbox{cov}}

\title{Two-step estimation of latent trait models}
\author[1]{Jouni Kuha}
\affil[1]{Department of Statistics, London School of Economics and
Political Science, London, UK. \texttt{j.kuha@lse.ac.uk}}

\author[2]{Zsuzsa Bakk}
\affil[2]{Department of Methodology and Statistics, Leiden University,
Leiden, The~Netherlands. \texttt{z.bakk@fsw.leidenuniv.nl}}

\begin{document}

\maketitle

\begin{abstract}
We consider two-step estimation of latent variable models, in which just
the measurement model is estimated in the first step and the measurement
parameters are then fixed at their estimated values in the second step
where the structural model is estimated. We show how this approach can
be implemented for latent trait models (item response theory models)
where the latent variables are continuous and their measurement
indicators are categorical variables. The properties of two-step
estimators are examined using simulation studies and applied examples.
They perform well, and have attractive practical and conceptual
properties compared to the alternative one-step and three-step approaches.
These results are in line with previous findings for other families of
latent variable models. This provides strong evidence that
two-step estimation is a flexible and useful general method of
estimation for
different types of latent variable models.
\end{abstract}

\vspace*{2ex}
\emph{Key words}: Item response theory models; latent variable
models; structural equation models; pseudo-maximum likelihood estimation
%\newpage

\section{Introduction}

Latent variable models have normally two distinct parts: a
\emph{measurement model} which describes how the latent variables that
appear in the model are measured by observed indicators of them, and a
\emph{structural model} which describes the associations among the
latent variables and any observed explanatory and response variables
which are not treated as measurement indicators. For instance, in the
illustrative example considered in Section \ref{s_example} of this paper
the structural model specifies how individuals' extrinsic and intrinsic
work value orientations are associated with characteristics of the
individuals, and the measurement model how these value orientations are
measured by a set of survey questions. Here, as in many applications,
the structural model is the focus of substantive interest, but the
measurement model also needs to be included and estimated in order for
the structural model to be estimable.

Estimation of these elements can be organised in different ways. In
joint or \emph{one-step estimation}, both parts of the model are
estimated together. When this is done by maximizing the joint
likelihood, one-step estimates are the maximum likelihood (ML) estimates
of the model parameters. In contrast, ``stepwise'' approaches divide the
estimation of the two parts into separate steps. The most familiar of
them is \emph{three-step estimation}. In its first step, the measurement
model is estimated from the simplest specification that allows this,
omitting all or most of the structural model. In the second step, this
estimated measurement model is used to assign predicted values of the
latent variables to the units of analysis, such as factor scores for
continuous latent variables. In the third step, these values are used in
place of the latent variables to estimate the structural model. There
are ``naive'' and ``adjusted'' versions of three-step estimation,
depending on whether it attempts to account for the measurement error
that results from using the predicted values (this will be discussed
further in Section \ref{ss_estimation_onethree}).

The focus of this paper is on a different stepwise approach,
\emph{two-step estimation} of latent variable models. Its first step is
estimation of the measurement model, as in the three-step method. In the
second step, instead of being used to calculate explicit predictions for
the latent variables, the parameters of the measurement model are simply
fixed at their estimated values. In other words, the second step takes
the same form as one-step estimation, except that all the measurement
parameters are treated as known numbers rather than unknown estimands.
Likelihood-based two-step estimation derives its justification and
properties from the general theory of pseudo maximum likelihood
estimation, as discussed in Section \ref{ss_estimation_variance} below.
Its large-sample properties are very similar to those of one-step ML
estimation.

Two-step estimates avoid the measurement error bias of naive three-step
estimates and are typically comparable in performance, but practically
simpler than, adjusted three-step estimates. Compared to one-step
estimation, the two-step approach has in principle attractive practical
and conceptual advantages. Because it is split into two steps, both of
them will be computationally less demanding than the single step of
one-step estimation. It is also easy to estimate the measurement and
structural models using fully or partially different sets of data, which
is necessary in some applications. The conceptual advantage arises from
an inherent difference in how the two methods use the available data.
Because the one-step approach fits all parts of the model at once, the
estimated measurement model is informed not only by the measurement
indicators but also by the observed covariates and responses in the
structural model. This can cause what \citeauthor{burt73}
(\citeyear{burt73}, \citeyear{burt76}) terms ``interpretational
confounding'', a situation where the implied definition of a latent
variable is in effect partly determined by variables that should be
conceptually separate from it. A related problem is that the estimated
measurement model will change whenever the structural model is changed, and
misspecification of the structural model may also distort the
measurement parameters. These risks are naturally avoided by the
two-step approach because it estimates the measurement model only once
and using only the measurement indicators.

Two-step estimation is a general idea, and one of the points that we
want to make in this paper is that it can in principle be applied to any
latent variable models. However, there may be differences in performance
of the estimates, ease of implementation or other points to consider
when it is used for different broad types of models.
In the literature so far, two-step estimation has accordingly been
described for specific types of models. It was proposed for latent class
analysis, that is for models where both the latent variables and their
indicators are categorical, by \cite{bandeenrocheetal97},
\cite{xue+bandeen-roche02} and \cite{bakk-kuha}, and has since been
extended to further versions of them (e.g.\ multilevel latent class
analysis in \citealt{dimarietal23}). For structural equation models
(SEMs), where the latent variables and indicators are both continuous,
the idea was introduced already by \citeauthor{burt73}
(\citeyear{burt73}, \citeyear{burt76}), but detailed exploration of it
is much more recent. In particular, \cite{rosseel+loh22} proposed a
general two-step approach for SEMs,  with the title of
``structural-after-measurement'' (SAM) estimation. They describe two
(usually equivalent) forms of it: ``local SAM'' which summarises the
first step in the form of the estimated means and covariance matrix of
the variables in the structural model, and ``global'' SAM which is
analogous to the implementation that we describe in this paper for
latent trait models (the moment-based ``local'' approach is not really applicable to
them). A method for SEMs is also proposed by
\cite{levy23}, using Bayesian (MCMC) estimation for both steps. Another
closely related paper is \cite{skrondal+kuha12}, who use two-step
estimation to correct for covariate measurement error in regression
models. Two-step methods for these classes of models are now also being
implemented in general-purpose software for latent variable modelling,
in Latent Gold for latent class analysis \citep{vermunt+magidson21B} and
the R package \emph{lavaan} for SAM \citep{rosseel12, rosseel+loh22}.

In this paper we propose and examine two-step estimation for another
family of latent variable models, one where the latent variables are
continuous but their indicators are categorical (in our presentation
they are all dichotomous for simplicity, but the extension to polytomous
items is immediate). These are commonly known as \emph{Item Response
Theory (IRT) models}. Here we refer to them also as \emph{latent trait
models}. They resemble SEMs in having continuous latent variables, and
latent class models in having categorical indicators. Like SEMs, they
are often used with complex structural models which include multiple
latent variables, whereas latent class analysis more often includes only
one latent class variable. Computationally, latent trait models are
typically the most demanding of these families of models, because their
log likelihood cannot be expressed in a closed form.

We describe the theory and implementation of two-step estimation of
latent trait models. The presentation of this broadly parallels that of
\cite{bakk-kuha} for latent class models, but with the changes that are
needed now that the latent variables are continuous. We then use
simulation studies and an application example to explore the properties
of the method in this context. Here one question of interest is whether its relative
performance is similar to what it is for previously
considered types of models. The conclusion is that it is, with some
differences of emphasis (these findings are discussed in more detail in
Sections \ref{s_simulation}--\ref{s_discussion}). Two-step estimates
of the structural model again perform very well compared to both
one-step and (naive) three-step estimates. In complex models their
computational advantage compared to one-step estimation can be even
larger than for latent class models and SEMs. On the other hand,
interpretational confounding of one-step estimates may be less worrying
here than for latent class models, because shifts in interpretation tend
to be less abrupt for continuous latent variables than for categorical
ones.

The latent trait model specifications that we consider are introduced in Section
\ref{s_models}. The definition, implementation and properties of
two-step estimation for them are described in Section~\ref{s_estimation}.
Simulation studies are reported in Section \ref{s_simulation} and the
real-data example in Section \ref{s_example}, and concluding discussion
is given in Section \ref{s_discussion}. Extended tables of some of the results
from the simulations and data analysis are given in a supplementary
appendix. [Computer code for the estimation in R and Mplus (both of
which are needed) and replication code for the applied example are also
provided as supplementary materials.]

\section{Models and variables}
\label{s_models}

To begin with a general formulation of a latent variable model, let
$\boldsymbol{\eta}_{i}$ be a vector of latent variables, and
$\mathbf{Y}_{i}$, $\mathbf{X}_{i}$ and $\mathbf{Z}_{i}$ distinct vectors
of observed variables, for a unit of analysis $i$. Consider a model of
the form $p(\mathbf{Y}_{i},\boldsymbol{\eta}_{i},\mathbf{Z}_{i}|
\mathbf{X}_{i}; \boldsymbol{\theta}) =
p(\mathbf{Y}_{i}|\boldsymbol{\eta}_{i},\mathbf{Z}_{i}, \mathbf{X}_{i};
\boldsymbol{\theta}_{1})\, \times
p(\boldsymbol{\eta}_{i},\mathbf{Z}_{i}|\mathbf{X}_{i};
\boldsymbol{\theta}_{2})$, where $p(\cdot|\cdot)$ denotes a conditional
distribution and
$\boldsymbol{\theta}=(\boldsymbol{\theta}_{1}',\boldsymbol{\theta}_{2}')'$
are parameters. Here
$p(\mathbf{Y}_{i}|\boldsymbol{\eta}_{i},\mathbf{Z}_{i}, \mathbf{X}_{i};
\boldsymbol{\theta}_{1})$ is the \emph{measurement model} for
$\boldsymbol{\eta}_{i}$ and
$p(\boldsymbol{\eta}_{i},\mathbf{Z}_{i}|\mathbf{X}_{i};
\boldsymbol{\theta}_{2})$ the \emph{structural model}. Their parameters
$\boldsymbol{\theta}_{1}$ and $\boldsymbol{\theta}_{2}$ are assumed to
be distinct and variation-independent of each other. The observed
variables $\mathbf{Y}_{i}$ (the measurement \emph{items}) are regarded
as measures of the latent $\boldsymbol{\eta}_{i}$, while
$\mathbf{X}_{i}$ are exogenous explanatory variables and
$\mathbf{Z}_{i}$ endogenous variables in the structural model.

In two-step estimation, the measurement parameters
$\boldsymbol{\theta}_{1}$ are estimated first and then fixed at their
estimated values in the second step where the structural parameters
$\boldsymbol{\theta}_{2}$ are estimated. This is a general approach
which can in principle be used for any instance of the general model
defined above. For specificity, however, in this paper we focus on its
use for a particular type of latent variable models. In it,
$\boldsymbol{\eta}_{i}$ are continuous and normally distributed and the
structural model is recursive. The items in $\mathbf{Y}_{i}$ are taken
to be binary, and the measurement models are specified separately for
each variable in $\boldsymbol{\eta}_{i}$, with conditional independence
for different items in $\mathbf{Y}_{i}$ and without direct effects from
$\mathbf{Z}_{i}$ and $\mathbf{X}_{i}$. This specification is described
in this section. Possible variants and extensions of it are discussed in
Section~\ref{ss_estimation_extensions}.

Consider first the measurement model. We assume here that
$p(\mathbf{Y}_{i}|\boldsymbol{\eta}_{i},\mathbf{Z}_{i}, \mathbf{X}_{i};
\boldsymbol{\theta}_{1}) =p(\mathbf{Y}_{i}|\boldsymbol{\eta}_{i};
\boldsymbol{\theta}_{1})$, so that there are no direct effects
between $(\mathbf{Z}_{i},\mathbf{X}_{i})$ and $\mathbf{Y}_{i}$,
i.e.\ no
differential item functioning or non-equivalence of measurement.
The model can then be written as
\begin{equation}
p(\mathbf{Y}_{i},\boldsymbol{\eta}_{i},\mathbf{Z}_{i}|\mathbf{X}_{i}; \boldsymbol{\theta}) =
p(\mathbf{Y}_{i}|\boldsymbol{\eta}_{i};
\boldsymbol{\theta}_{1})\,
p(\boldsymbol{\eta}_{i},\mathbf{Z}_{i}|\mathbf{X}_{i}; \boldsymbol{\theta}_{2}),
\label{general_model1}
\end{equation}
which also implies a model for the observed variables as
\begin{equation}
p(\mathbf{Y}_{i},\mathbf{Z}_{i}|\mathbf{X}_{i}; \boldsymbol{\theta}) =
\int \, p(\mathbf{Y}_{i}|\boldsymbol{\eta}_{i};
\boldsymbol{\theta}_{1})\,
p(\boldsymbol{\eta}_{i},\mathbf{Z}_{i}|\mathbf{X}_{i}; \boldsymbol{\theta}_{2})\,
d\boldsymbol{\eta}_{i}.
\label{general_model2}
\end{equation}
Let $\boldsymbol{\eta}_{i}=(\eta_{i1},\dots,\eta_{iK})'$, for $K\ge 1$.
We assume that $\mathbf{Y}_{i}$ can be partitioned correspondingly as
$\mathbf{Y}_{i}=(\mathbf{Y}_{i1}',\dots,\mathbf{Y}_{iK}')'$, in such a
way that only items in $\mathbf{Y}_{ik}= (Y_{ik1},\dots,Y_{ikp_{k}})'$
are measures of the latent variable $\eta_{ik}$, for each $k=1,\dots,K$.
We also make the common assumption that the items within and between
$\boldsymbol{Y}_{ik}$ are conditionally independent of each other given
$\boldsymbol{\eta}_{i}$. The measurement model can then be written as
\begin{equation}
p(\mathbf{Y}_{i}|\boldsymbol{\eta}_{i};
\boldsymbol{\theta}_{1}) = \prod_{k=1}^{K} \,
p(\mathbf{Y}_{ik}|\eta_{ik}; \boldsymbol{\theta}_{1k})
=\prod_{k=1}^{K}\prod_{j=1}^{p_{k}} \, p(Y_{ikj}|\eta_{ik};
\boldsymbol{\theta}_{1k})
\label{measurement_model1}
\end{equation}
so that
$\boldsymbol{\theta}_{1}=(\boldsymbol{\theta}_{11}',\dots,\boldsymbol{\theta}_{1K}')'$,
with the different $\boldsymbol{\theta}_{1k}$ taken to be distinct from
each other. We consider the situation where the items are
Bernoulli-distributed binary variables, with values 0 and 1, and specify
the measurement model for each item in (\ref{measurement_model1}) as the
logit model \begin{equation} \text{logit}[P(Y_{ikj}=1|\eta_{ik};
\boldsymbol{\theta}_{1k})] = \tau_{kj}+\lambda_{kj}\eta_{ik}
\label{logit} \end{equation} so that $\boldsymbol{\theta}_{1k}=
(\tau_{k1},\dots,\tau_{kp_{k}}, \lambda_{k1},\dots, \lambda_{kp_{k}})'$.
In the language of item response theory modelling, this is a
two-parameter logistic (2-PL) model. There can be parameter constraints
within $\boldsymbol{\theta}_{1k}$, such as taking all $\lambda_{kj}$ for
the same $k$ to be equal.

Consider now the structural model
$p(\boldsymbol{\eta}_{i},\mathbf{Z}_{i}|\mathbf{X}_{i};\boldsymbol{\theta}_{2})$. In
general, we take it to be specified as a non-recursive chain of
conditional distributions for elements of $\mathbf{Z}_{i}$ and
$\boldsymbol{\eta}_{i}$ in some pre-specified order. Let
$\boldsymbol{\eta}_{i}=
(\boldsymbol{\eta}_{i(1)}',\dots,\boldsymbol{\eta}_{i(M)}')'$
be partitioned
into $1\le M\le K$ blocks of variables $\boldsymbol{\eta}_{i(m)}$
according to this order.
For notational convenience, we limit the presentation to
the case where $\mathbf{Z}_{i}$ appear only at the
end of this chain, as response variables to $\mathbf{X}_{i}$ and
$\boldsymbol{\eta}_{i}$.
The structural model can then be written as
\begin{eqnarray}
%\lefteqn{
p(\boldsymbol{\eta}_{i},\mathbf{Z}_{i}|\mathbf{X}_{i};\boldsymbol{\theta}_{2})
&=&
p(\boldsymbol{\eta}_{i(1)}|\mathbf{X}_{i};\boldsymbol{\theta}_{21})\,
p(\boldsymbol{\eta}_{i(2)}|\mathbf{X}_{i},\boldsymbol{\eta}_{i(1)};\boldsymbol{\theta}_{22})
%}
\label{structural_model}\\
\hspace*{5em}
&
%\hspace*{-1em}
%\hspace*{-3em}
&
%\hspace*{-1em}
\hspace*{1em}\times
\cdots
\times
p(\boldsymbol{\eta}_{i(M)}|\mathbf{X}_{i},\boldsymbol{\eta}_{i(1)},
\dots,\boldsymbol{\eta}_{i(M-1)};\boldsymbol{\theta}_{2M})\,
p(\mathbf{Z}_{i}|\mathbf{X}_{i},\boldsymbol{\eta}_{i};\boldsymbol{\theta}_{2,M+1})
\nonumber
\end{eqnarray}
so that
$\boldsymbol{\theta}_{2}=(\boldsymbol{\theta}_{21}',\dots,\boldsymbol{\theta}_{2,M+1}')'$.
The model for $\mathbf{Z}_{i}$ will be a regression model of an appropriate kind, for example as a
normal linear model when $\mathbf{Z}_{i}$ is continuous or a
logit model when it is a binary and univariate.
For the latent $\boldsymbol{\eta}_{i}$, we assume that
all of their conditional distributions in (\ref{structural_model}) are
(multivariate or univariate) normal, and consider
the linear models
\begin{eqnarray}
p(\boldsymbol{\eta}_{i(1)}|\mathbf{X}_{i};\boldsymbol{\theta}_{21})
&\sim& N(
\boldsymbol{\beta}_{10}
+\boldsymbol{\beta}_{1x}\mathbf{X}_{i}, \, \boldsymbol{\Psi}_{1}
)
\label{eta_model1}
\\
p(\boldsymbol{\eta}_{i(m)}
|\mathbf{X}_{i},\boldsymbol{\eta}_{i(1)},\dots,\boldsymbol{\eta}_{i(m-1)};\boldsymbol{\theta}_{2m})
&\sim&\hspace*{-.5em} N(
\boldsymbol{\beta}_{m0}
+\boldsymbol{\beta}_{mx}\mathbf{X}_{i}
+\sum_{l=1}^{m-1}\boldsymbol{\beta}_{ml}\boldsymbol{\eta}_{i(l)}, \,
\boldsymbol{\Psi}_{m}
)
%\hspace*{.2em}\text{ for }\hspace*{.2em} m=2,\dots,M,
\label{eta_model2}
\end{eqnarray}
for $m=2,\dots,M$, so that each of
$\boldsymbol{\theta}_{21},\dots,\boldsymbol{\theta}_{2M}$ consists of
the corresponding $\boldsymbol{\beta}$ and $\boldsymbol{\Psi}$
parameters. Constraints on elements of $\boldsymbol{\theta}_{2}$ can be
included, most often zero constraints on the $\boldsymbol{\beta}$
parameters which  omit some of the explanatory variables from some parts
of the model and thus yield different non-saturated choices for the
overall structural model~(\ref{structural_model}). If observed
covariates $\mathbf{X}_{i}$ and/or responses $\mathbf{Z}_{i}$ are not
present, (\ref{structural_model})--(\ref{eta_model2}) simplify
accordingly. For example, in the simulations of Section
\ref{s_simulation} we consider three situations, where the variables in
the structural model are $(X_{i},\eta_{i})$, $(\eta_{i}, Z_{i})$ and
$(\eta_{i(1)}, \eta_{i(2)})= (\eta_{i1}, \eta_{i2})$ (path diagrams for
these are shown in Figure A.1 of the supplementary appendix), and in the
applied example of Section \ref{s_example} we have models for a
bivariate latent response $\boldsymbol{\eta}_{i(1)}=(\eta_{i1},\eta_{i2})$ given
covariates $\mathbf{X}_{i}$.

\section{Two-step estimation of the structural model}
\label{s_estimation}

\subsection{Two-step point estimates}
\label{ss_estimation_twostep}

Suppose that we observe $(\mathbf{Y}_{i}, \mathbf{Z}_{i},
\mathbf{X}_{i})$
for units $i=1,\dots,n$, assumed to be independent of each other.
We use likelihood-based estimation in a parametric framework.
The log-likelihood function for the model described in Section
\ref{s_models} is
\begin{eqnarray}
\lefteqn{\ell(\boldsymbol{\theta}) =
\sum_{i=1}^{n}
\, \log p(\mathbf{Y}_{i},\mathbf{Z}_{i}|\mathbf{X}_{i};
\boldsymbol{\theta})
=
\sum_{i=1}^{n}\, \log \,
\int p(\mathbf{Y}_{i}|\boldsymbol{\eta}_{i};
\boldsymbol{\theta}_{1})\,
p(\boldsymbol{\eta}_{i},\mathbf{Z}_{i}|\mathbf{X}_{i};
\boldsymbol{\theta}_{2}) \, d\boldsymbol{\eta}_{i}}
\nonumber \\
&=&
\sum_{i=1}^{n}\, \log \int \,
\left[
\prod_{k=1}^{K}
\prod_{j=1}^{p_{k}}
p(Y_{ikj}|\eta_{ik}; \boldsymbol{\theta}_{1k})
\right] \,
p(\boldsymbol{\eta}_{i},\mathbf{Z}_{i}|\mathbf{X}_{i};
\boldsymbol{\theta}_{2}) \, d\boldsymbol{\eta}_{i}
\label{loglik1}
\end{eqnarray}
where
the measurement  models
for individual items are given by
(\ref{logit}) and
the structural model
by (\ref{structural_model})--(\ref{eta_model2}).
If any variables in $\mathbf{Y}_{i}$ or
$\mathbf{Z}_{i}$ are missing, they are assumed to be missing at random
and their contributions are omitted from (\ref{loglik1}). We assume that
there is no missing data in $\mathbf{X}_{i}$.

In step 1 of two-step estimation, the measurement parameters
$\boldsymbol{\theta}_{1}$ are estimated from a simpler specification. We do this separately for each
latent variable $\eta_{ik}$, using
the log-likelihoods
\begin{equation}
\ell(\boldsymbol{\psi}_{k})=
\sum_{i=1}^{n}\, \log
\int
\left[ \prod_{j=1}^{p_{k}} \,
p(Y_{ikj}|\eta_{ik}; \boldsymbol{\theta}_{1k})
\right]
\,
p(\eta_{ik};
\mu_{k}, \sigma^{2}_{k})\,
d\eta_{ik}
\label{step1loglik}
\end{equation}
for $k=1,\dots,K$, where $p(\eta_{ik}; \mu_{k},\sigma^{2}_{k})\sim
N(\mu_{k},\sigma^{2}_{k})$
and $\boldsymbol{\psi}_{k}=(\boldsymbol{\theta}_{1k}', \mu_{k},
\sigma^{2}_{k})'$. Maximizing (\ref{step1loglik}) gives the
estimate $\tilde{\boldsymbol{\psi}}_{k}$, from which
$\tilde{\boldsymbol{\theta}}_{1k}$ are
the step-1 estimates of $\boldsymbol{\theta}_{1k}$ and
$\tilde{\mu}_{k}$ and $\tilde{\sigma}_{k}^{2}$ are discarded. The
integrals in (\ref{loglik1}) and (\ref{step1loglik}) are not available
in a closed form, so numerical integration needs to be used to evaluate them in
the estimation.

Compared with latent class models (as discussed in
\citealt{bakk-kuha}), there
are some additional considerations here
where the latent variables
$\boldsymbol{\eta}_{i}$ are
continuous:
\begin{itemize}
\item
Some parameter constraints are needed to fix
the scales of $\boldsymbol{\eta}_{i}$ and thus identify the
parameters
$\boldsymbol{\theta}$.
In two-step estimation they are imposed in the first step, and the scale
implied by them then carries over to the second step where no further
constraints are required. We identify the scale of each $\eta_{ik}$ by
fixing in (\ref{logit}) $\tau_{kj}=0$ and $\lambda_{kj}=1$ for one $j$.
This anchors their interpretation in
the items $\mathbf{Y}_{i}$, leaving the distribution of
$\boldsymbol{\eta}_{i}$ free.
An alternative first-step constraint would be to set
$\mu_{k}=0$ and $\sigma^{2}_{k}=1$ for all $k$.
This, however, gives a slightly less straightforward
interpretation of the latent scales because the marginal (as opposed to
conditional) distributions
of $\eta_{ik}$  do not
appear in the structural model (\ref{structural_model}) which is the
focus of the second step of estimation.
\item
The likelihood in (\ref{step1loglik}) is obtained by integrating the
likelihood (\ref{general_model1}) of the joint model over all the
variables other than $\mathbf{Y}_{ik}$ and $\eta_{ik}$. This implies one
inconsistency between the two expressions, in that $\eta_{ik}$ cannot be
exactly normally distributed both marginally and conditionally on
covariates $\mathbf{X}_{i}$, unless $\mathbf{X}_{i}$ is absent or itself normally
distributed. We will nevertheless take $\eta_{ik}$ to be normal also in
(\ref{step1loglik}). The same approximation occurs also in one-step
estimation when we consider different choices of $\mathbf{X}_{i}$, as
well as in likelihood-based one- and two-step estimation of structural
equation models.
\item
Unlike in typical latent class models, in applications of latent trait models
the latent $\boldsymbol{\eta}_{i}$ is often multivariate. In step
1, we estimate the measurement models of
each element $\eta_{ik}$ of it separately. This keeps this step as
simple as possible, which is a key benefit of two-step estimation. An
alternative would be to estimate all of the measurement models at once,
using a likelihood with contributions $\int
[\prod_{k}\prod_{j}p(Y_{ikj}|\eta_{ik};\boldsymbol{\theta}_{1k})]p(\boldsymbol{\eta}_{i};\boldsymbol{\mu}_{\eta},\boldsymbol{\Sigma}_{\eta})
d\boldsymbol{\eta}_{i}$ where $\boldsymbol{\eta}_{i}$ is taken to be
multivariate normal with mean $\boldsymbol{\mu}_{\eta}$ and variance
matrix $\boldsymbol{\Sigma}_{\eta}$. In other words, this would be
obtained be omitting just $\mathbf{X}_{i}$ and $\mathbf{Z}_{i}$ from the
structural model, and ignoring any parameter constraints in it. This,
however, would make the first step more complex, often with no clear benefit.
\cite{rosseel+loh22} give a good discussion of the considerations of how
to organise step 1 in this respect. They too recommend estimating the
measurement model of each latent variable separately in most cases.
\end{itemize}

In step 2 of two-step estimation, the parameters
$\boldsymbol{\theta}_{2}$ of the structural model are then estimated,
holding $\boldsymbol{\theta}_{1}$ fixed at their estimates
$\tilde{\boldsymbol{\theta}}_{1}$ from step 1. Here the log-likelihood
is $\ell(\tilde{\boldsymbol{\theta}}_{1},\boldsymbol{\theta}_{2})$,
which is of the same form as the one-step log-likelihood
$\ell(\boldsymbol{\theta})=\ell(\boldsymbol{\theta}_{1},\boldsymbol{\theta}_{2})$
in (\ref{loglik1}) but with the fixed values
$\tilde{\boldsymbol{\theta}}_{1}$ substituted for
$\boldsymbol{\theta}_{1}$. Maximizing this with respect to
$\boldsymbol{\theta}_{2}$ gives the two-step estimate of the parameters
of the structural model, which we denote by
$\tilde{\boldsymbol{\theta}}_{2}$.

The step-1 estimate $\tilde{\boldsymbol{\theta}}_{1}$ can be used for
any structural models for the same $\boldsymbol{\eta}_{i}$ in step 2,
for example if we want to compare models with different choices of the
covariates $\mathbf{X}_{i}$. Note also that
$\tilde{\boldsymbol{\theta}}_{1}$ can be obtained from a partially or
completely different set of data than what is used for step 2, as long
as we can assume that the same measurement model holds in both.

\subsection{Variance estimation}
\label{ss_estimation_variance}

As the step-2 estimates $\tilde{\boldsymbol{\theta}}_{2}$ are obtained
by maximizing the log-likelihood $\ell(\tilde{\boldsymbol{\theta}}_{1},
\boldsymbol{\theta}_{2})$, their properties follow from the general
theory of pseudo-maximum likelihood (PML) estimation
\citep{gong+samaniego81}. Much of the presentation here is similar to
that of \cite{bakk-kuha} for latent class models, because the general
form of these results is the same irrespective of the details of the
model specification.

PML estimators are consistent and asymptotically normally distributed
under very general regularity conditions. In our situation these
require, broadly, that the joint model is such that the one-step ML
estimator $\hat{\boldsymbol{\theta}}$ is consistent for
$\boldsymbol{\theta}$, that the values of $\boldsymbol{\theta}_{1}$ and
$\boldsymbol{\theta}_{2}$ can vary independently of each other, and that
the step-1 estimator $\tilde{\boldsymbol{\theta}}_{1}$ is consistent for
$\boldsymbol{\theta}_{1}$. Here these conditions are satisfied, apart
from some approximation which may arise (as discussed in Section
\ref{ss_estimation_twostep}) from taking $\boldsymbol{\eta}_{i}$ to be
normally distributed both marginally (in step 1) and conditionally on
covariates $\mathbf{X}_{i}$ (in step 2). In our simulations in
Section~\ref{s_simulation} this approximation does not have any
noticeable effect on the accuracy of the estimates.

To derive the variance matrix of $\tilde{\boldsymbol{\theta}}_{2}$,
let the Fisher information matrix for $\boldsymbol{\theta}$ in
the full model be
\[
\boldsymbol{{\cal I}}(\boldsymbol{\theta}^{*})=
\begin{bmatrix}
\boldsymbol{{\cal I}}_{11} & \\
\boldsymbol{{\cal I}}'_{12} &
\boldsymbol{{\cal I}}_{22}
\end{bmatrix}
\]
where $\boldsymbol{\theta}^{*}$ denotes the true value of
$\boldsymbol{\theta}$ and the partitioning corresponds to
$\boldsymbol{\theta}_{1}$ and $\boldsymbol{\theta}_{2}$. The asymptotic
variance matrix of the joint (one-step) maximum likelihood estimate
of $\boldsymbol{\theta}$ is
thus $\mathbf{V}_{ML}=\boldsymbol{{\cal
I}}^{-1}(\boldsymbol{\theta}^{*})/n$.
Let $\mathbf{\Sigma}_{11}/n$
be the asymptotic variance matrix of the step-1 estimator
$\tilde{\boldsymbol{\theta}}_{1}$.
The asymptotic variance matrix of the two-step estimator
$\tilde{\boldsymbol{\theta}}_{2}$ is then $\mathbf{V}/n$, where
\begin{equation}
\mathbf{V} =
\boldsymbol{{\cal I}}_{22}^{-1}
+
\boldsymbol{{\cal I}}_{22}^{-1}\,
\boldsymbol{{\cal I}}_{12}'\,
\mathbf{\Sigma}_{11}\,
\boldsymbol{{\cal I}}_{12}\,
\boldsymbol{{\cal I}}_{22}^{-1}
\equiv \mathbf{V}_{2} + \mathbf{V}_{1}
\label{Vmat}
\end{equation}
(if $\tilde{\boldsymbol{\theta}}_{1}$ was obtained using a different
number $n_{1}\ne n$ of observations, $\mathbf{\Sigma}_{11}$ is
multiplied by ($n/n_{1}$) in this; see \cite{xue+bandeen-roche02} and
\cite{bakk-kuha}). The estimated variance matrix of the two-step
estimator $\tilde{\boldsymbol{\theta}}_{2}$ is then
$\hat{\mathbf{V}}/n$, where
$\hat{\mathbf{V}}=\hat{\mathbf{V}}_{2}+\hat{\mathbf{V}}_{1}$ and the
estimates of these matrices are obtained by substituting
$\tilde{\boldsymbol{\theta}}=(\tilde{\boldsymbol{\theta}}_{1},
\tilde{\boldsymbol{\theta}}_{2})$ for $\boldsymbol{\theta}^{*}$ in
$\boldsymbol{{\cal I}}_{22}$ and $\boldsymbol{{\cal I}}_{12}$ and an
estimate $\tilde{\boldsymbol{\Sigma}}_{11}$ from the first step of
estimation for $\boldsymbol{\Sigma}_{11}$.

In (\ref{Vmat}), $\mathbf{V}_{2}$ describes the variability in
$\tilde{\boldsymbol{\theta}}_{2}$ if $\boldsymbol{\theta}_{1}$ were
known, and $\mathbf{V}_{1}$ the additional variability arising from
estimating $\boldsymbol{\theta}_{1}$ by
$\tilde{\boldsymbol{\theta}}_{1}$. One possible approach in data
analysis would be to choose to omit $\mathbf{V}_{1}$ from the variance.
This would mean, in effect, that when we examine the structural model we
regard the measurement model, and thus the definition of the latent
variables $\boldsymbol{\eta}_{i}$, as known and fixed a priori rather
than as an estimable characteristic. This is close in spirit to the
approach that is almost always adopted in naive three-step estimation,
where factor scores are treated as fixed variables in its last step.
This may be appropriate in some applications. When it is not,
however, omitting the contribution from $\mathbf{V}_{1}$ may result in
serious underestimation of the variance of
$\tilde{\boldsymbol{\theta}}_{2}$. We explore this further in the
simulations of Section \ref{s_simulation}.

What remains to be specified is the estimator of the step-1 variance matrix
$\boldsymbol{\Sigma}_{11}/n$. Here some new questions
arise when $\boldsymbol{\eta}_{i}=(\eta_{i1},\dots,\eta_{iK})'$ is a
vector, with corresponding measurement parameters
$\boldsymbol{\theta}_{1}=(\boldsymbol{\theta}_{11}',\dots,\boldsymbol{\theta}_{1K}')'$.
Consider $\boldsymbol{\Sigma}_{11}$ divided into
diagonal blocks $\boldsymbol{\Sigma}_{11(kk)}=n\,
\text{var}(\tilde{\boldsymbol{\theta}}_{1k})$ and off-diagonal blocks
$\boldsymbol{\Sigma}_{11(kl)}=n\, \text{cov}(
\tilde{\boldsymbol{\theta}}_{1k},\tilde{\boldsymbol{\theta}}_{1l})$ for
each $k,l=1,\dots,K$, and denote the Fisher information matrices for
$\boldsymbol{\psi}_{k}$ in (\ref{step1loglik}) by $\boldsymbol{{\cal I}}_{*11(k)}$. The
$\boldsymbol{\Sigma}_{11(kk)}$ are obtained by extracting the
elements corresponding to $\boldsymbol{\theta}_{1k}$ from
$\boldsymbol{{\cal I}}^{-1}_{*11(k)}$, and estimates
$\tilde{\boldsymbol{\Sigma}}_{11(kk)}$ by substituting
$\tilde{\boldsymbol{\psi}}_{k}$ in them. The off-diagonal blocks,
however, cannot be obtained from these step-1 information matrices. What
we do here is to set $\tilde{\boldsymbol{\Sigma}}_{11(kl)}=\mathbf{0}$
for $k\ne l$. This is a simplifying approximation, because even though
$\tilde{\boldsymbol{\theta}}_{1k}$ and
$\tilde{\boldsymbol{\theta}}_{1l}$ are estimated from separate models,
they can still be correlated because they use the data on
$\mathbf{Y}_{ik}$ and $\mathbf{Y}_{il}$ for the same units $i$. The same
approach is taken by \cite{rosseel+loh22} for their SAM implementation.
It seems quite justified at least in our simulations in Section
\ref{s_simulation}, where it has no effect on the accuracy of the
variance estimation.

Alternatively, these cross-block covariances would be obtained if we
estimated all of $\boldsymbol{\theta}_{1}$ together in step 1.
This, however, would
defeat the purpose of keeping this step as simple as possible. Another
way of estimating them can be derived from general
results on estimation equations \citep[see][Chapter
5]{cameron+trivedi05}. Let $\mathbf{s}_{ik}(\boldsymbol{\psi}_{k})$ denote the contribution of unit
$i$ to the score function for the step-1 log-likelihood
(\ref{step1loglik}), define $\mathbf{B}_{kl}=\sum_{i=1}^{n}
\mathbf{s}_{ki}(\tilde{\boldsymbol{\psi}}_{1k})
\mathbf{s}_{li}(\tilde{\boldsymbol{\psi}}_{1l})'$, and let
$\tilde{\boldsymbol{{\cal I}}}_{*11(k)}$ be obtained by
substituting $\tilde{\boldsymbol{\psi}}_{k}$ for $\boldsymbol{\psi}_{k}$
in $\boldsymbol{{\cal I}}_{*11(k)}$. Then $\boldsymbol{\Sigma}_{11(kl)}$
is estimated by the submatrix of
$ \tilde{\boldsymbol{{\cal
I}}}^{-1}_{*11(k)}\, \mathbf{B}_{kl}\, \tilde{\boldsymbol{{\cal
I}}}^{-1}_{*11(l)}$
for the rows corresponding to $\tilde{\boldsymbol{\theta}}_{1k}$ and the
columns for $\tilde{\boldsymbol{\theta}}_{1l}$. However,
the
$\mathbf{s}_{ik}(\boldsymbol{\psi}_{k})$ may not be easily available
from standard software. Bootstrap methods could also be considered for
the variance estimation. Determining the best way to implement them for
two-step estimation may raise interesting further
questions, which remain to be investigated in future research. Another
approach that is comparable in spirit to bootstrapping is that proposed
by \cite{levy23}. In it, the combined uncertainty from the two steps is accounted for by
drawing values of the measurement parameters from their posterior
distribution from step 1 and then running (Bayesian) estimation for step
2 given each of these draws.

\subsection{Alternatives: One- and three-step estimation}
\label{ss_estimation_onethree}

Two well-known alternatives to two-step estimation are the one- and
three-step approaches. In \emph{one-step estimation}, maximum likelihood
estimates
$\hat{\boldsymbol{\theta}}=(\hat{\boldsymbol{\theta}}_{1},\hat{\boldsymbol{\theta}}_{2})$
of all the parameters are obtained together, by maximizing the
log-likelihood (\ref{loglik1}) with respect to $\boldsymbol{\theta}$.
The estimated variance matrix of $\hat{\boldsymbol{\theta}}_{1}$ is
obtained from the information matrix for this in the standard way.

\emph{Three-step estimation} is a stepwise approach. Its first step is
the same as in the two-step method, in our case estimation of the
parameters in (\ref{step1loglik}). In the second step, these are used to
calculate predicted values $\tilde{\eta}_{ik}$ for the latent variables,
for each $i=1,\dots,n$; $k=1,\dots,K$. We use for this purpose the
conditional expected values (empirical Bayes predictions)
$\E(\eta_{k}|\mathbf{Y}_{ik}; \tilde{\boldsymbol{\psi}}_{k})$, the forms
of which are derived from the model behind (\ref{step1loglik}). In the
third step, the structural model is estimated with
$\tilde{\eta}_{ik}$ substituted for $\eta_{ik}$ and treated as observed
variables. We refer to this as ``naive'' three-step estimation. For the
recursive model (\ref{structural_model})--(\ref{eta_model2}) it means
simply fitting the $M+1$ regression models for
$\tilde{\boldsymbol{\eta}}_{i(1)},\dots,\tilde{\boldsymbol{\eta}}_{i(M)}$
and $\mathbf{Z}_{i}$ separately.

The predictions $\tilde{\eta}_{ik}$ can be regarded as erroneously
measured versions of the corresponding $\eta_{ik}$. Because of this
measurement error naive three-step estimates are generally biased, but there are some
exceptions. \cite{skrondal_01} consider this in the
context of factor analysis measurement models for continuous items
$\mathbf{Y}_{i}$, and \cite{lu+thomas08} extend their results to the
case of categorical items. They show that if $\eta_{ik}$ is in the role
of a response variable in a structural model, regression coefficients
for it are consistently estimated when $\tilde{\eta}_{ik}$ are
asymptotically unbiased (i.e.\ $\E(\tilde{\eta}_{ik})=\eta_{ik}$), as in
``Bartlett factor scores'' for continuous $\eta_{ik}$. Conversely, if
$\eta_{ik}$ is the only explanatory variable in a structural model, its
coefficient is consistently estimated if we use as $\tilde{\eta}_{ik}$
the empirical Bayes predictions (``regression factor scores'') which
shrink them toward $\mu_{k}$ in (\ref{step1loglik}). This could
be generalised to structural models with more explanatory variables by
predicting each explanatory factor with its expected value given
covariates $\mathbf{X}_{i}$ and the indicators of all the explanatory
factors in that model; this is known as ``regression calibration'' in
general measurement error modelling \citep{carrolletal06}.

Measurement error bias in the naive estimates can be corrected by
``adjusted'' methods of three-step estimation (see e.g.\
\citealt{bakk+kuha20} and \citealt{rosseel+loh22} for reviews of these
appoaches). Some of them are explicit measurement error corrections of
the naive estimates (e.g.\ \citealt{croon, Bolck:04}), while others
are in effect two-step estimation but with the predicted values
$\tilde{\eta}_{ik}$ taking the place of the items $\mathbf{Y}_{i}$
(\citealt{vermunt:10,wangetal19}). We will not explore such adjusted
methods in this paper, but use only the naive version of three-step
estimation. Its role is then to serve as the starting point and point of
comparison, as a pragmatically simple approach which could be employed
by an analyst in lieu of the consistent but more demanding two-step or
one-step methods.

Full variance estimation for three-step estimates would again include
accounting for the variability from the first step. For the naive
estimates, however, this is usually omitted. Estimated variances of the
estimates of $\boldsymbol{\theta}_{2}$ from its third step are then
obtained as in standard regression, in effect treating
$\tilde{\eta}_{ik}$ like any observed variables.

\subsection{Implementation of the estimation}
\label{ss_estimation_implementation}

Software that can do one-step estimation of a model can also be
used for two-step point estimation of it.
We have used Mplus 6.12
software \citep{muthen+muthen10} for the simulation studies and
application example of this paper. We have, furthermore, supplemented it
with functions in R \citep{r2022} to manage the estimation process. This
automatically sequences the two steps of estimation, passing the
estimated parameter values from the first step to the code for
the second and the estimates back to R. It is also needed for
calculation of the term $\hat{\mathbf{V}}_{1}$ of the estimate of the variance
matrix (\ref{Vmat}) which is not included directly in the
Mplus output. We have used the \emph{MplusAutomation} package in R
\citep{hallquist+wiley18} to control Mplus from R, and the \emph{brew}
package \citep{horner11} to automatically edit the input files.
[Examples of the estimation code are included in supplementary materials
for the paper.]

Because the log likelihood functions of these models can have multiple
local maxima, it is desirable to use multiple starting values for the
estimation algorithm. This is a standard feature in software such as
Mplus.

Estimation of $\boldsymbol{\mathcal{I}}_{12}$ in (\ref{Vmat}) requires
separate attention. It is not produced directly by one- or two-step
estimation, because it is part of the joint-model information matrix but
evaluated at the two-step estimates $\tilde{\boldsymbol{\theta}}$. We
have calculated it separately by a further call to Mplus, where one-step
estimation is started from $\tilde{\boldsymbol{\theta}}
=(\tilde{\boldsymbol{\theta}}_{1}',\tilde{\boldsymbol{\theta}}_{2}')'$
and an estimate of $\boldsymbol{\mathcal{I}}_{12}$ is taken from the
information matrix of this after one iteration. This is only
approximately correct because that one iteration means that
$\boldsymbol{\mathcal{I}}_{12}$ is in the end evaluated at a value which
differs to some extent from $\tilde{\boldsymbol{\theta}}$. Even with it,
however, the estimated standard errors perform well in our simulations.

\subsection{Extensions and variants}
\label{ss_estimation_extensions}

Two-step estimation could in principle be applied to any instance of the
general latent variable model, as defined at the start of Section
\ref{s_models}. For notational simplicity and to provide a focus for
this paper, we concentrate here on the class of models that was
introduced later in Section~\ref{s_models}. Before proceeding to
empirical investigations of them, in this section we comment
briefly on some further settings where the two-step method could be
used. Some of them are obvious extensions or variants of the models
considered here, while others might involve more consequential
differences. The implementation and small-sample performance of two-step
estimation in many of these other cases remain to be examined.

We made the assumption
$p(\mathbf{Y}_{i}|\boldsymbol{\eta}_{i},\mathbf{Z}_{i}, \mathbf{X}_{i};
\boldsymbol{\theta}_{1}) =p(\mathbf{Y}_{i}|\boldsymbol{\eta}_{i};
\boldsymbol{\theta}_{1})$ of measurement equivalence with respect to the
observed $(\mathbf{X}_{i},\mathbf{Z}_{i})$. Two-step
estimation of models which do not assume it may introduce new issues,
and deserves further investigation. For latent class models, it has been
studied by \cite{dimari+bakk18}, \cite{janssenetal19} and
\cite{vermunt+magidson21}.

When we do assume measurement equivalence, a general form of the model
is (\ref{general_model1}). We further specified its structural model as
in (\ref{structural_model})--(\ref{eta_model2}). We included
$\mathbf{Z}_{i}$ just at the end of the chain purely for notational
convenience, and this could be immediately relaxed by allowing such
endogenous observed variables to appear also as explanatory variables
for each other or the latent $\eta_{ik}$. The types and distributions of
$\eta_{ik}$ in (\ref{eta_model1})--(\ref{eta_model2}) and the forms of
the models for them could also be different. This would allow, for
example, models which contained both continuous latent factors and
latent class variables. It seems plausible that this would not
fundamentally change the behaviour of two-step estimators. Larger
changes to the structural model would include such possibilities as
interactions or nonlinear terms of the latent variables, or
non-recursive models. Two-step estimates could behave differently in
these cases, if only because such models are inherently complex and
difficult even for standard one-step estimation.

Considering then the measurement model, we took it to be of the form
(\ref{measurement_model1}). Within this, the forms of the items
$Y_{ikj}$ and the models $p(Y_{ikj}|\eta_{ik};
\boldsymbol{\theta}_{1k})$ for them could easily be varied, with only
corresponding changes to the details of the implementation of two-step
estimation. For binary items this would allow, for example, probit
rather than logit measurement models. More generally, combinations of
other types of items and models could be included, such as multinomial
models for polytomous items and factor analysis models for continuous
ones. Another apparently simple extension would be to allow the latent
variables measured by an item set $\mathbf{Y}_{ik}$ to be a vector
$\boldsymbol{\eta}_{ik}$ (at least as long as all of
$\boldsymbol{\eta}_{ik}$ belonged in the same block
$\boldsymbol{\eta}_{i(m)}$ in the structural model, so as not to
introduce a non-recursive element to it through the measurement model).
Conditional associations between some items within a $\mathbf{Y}_{ik}$
given the latent variables could also be included,
requiring only the corresponding
modifications of the log likelihoods (\ref{loglik1}) and
(\ref{step1loglik}).

\section{Simulations}
\label{s_simulation}

We have used simulation studies to examine the properties of
two-step estimates of the models considered in this paper. In every
simulation, the structural model for unit $i$ is of the form
\begin{equation}
V_{i2} = \beta_{0}+\beta_{1}V_{i1} +\epsilon_{i}.
\label{sim_model}
\end{equation}
We consider three cases, with different choices for
the types of the variables $V_{i1}$, $V_{i2}$ and $\epsilon_{i}$:
\begin{itemize}
\item \emph{\textbf{Case 1}: Observed covariate, latent response.} Here
$V_{i1}=X_{i}$ and $V_{i2}=\eta_{i}$, and $\eta_{i}$ is normally
distributed given $X_{i}$. Two distributions for $X_{i}$ are considered,
$X_{i}\sim N(0,1)$ [Case 1(a)] and $X_{i} \sim
2\,\text{Bernoulli}(0.5)-1$ [Case 1(b)].
\item \emph{\textbf{Case 2}: Latent covariate, observed response.} Here
$V_{i1}=\eta_{i}\sim N(0,1)$ and $V_{i2}=Z_{i}$.
Two conditional distributions for $Z_{i}$ given $\eta_{i}$ are considered,
a normal distribution [Case 2(a)] and
a skew-normal distribution \citep{azzalini+capitanio14}
set to be
positively skewed with skewness coefficient of just under 1 [Case 2(b)].
\item \emph{\textbf{Case 3}: Latent covariate, latent response.} Here
$V_{i1}=\eta_{i1}\sim N(0,1)$ and $V_{i2}=\eta_{i2}$ is normally
distributed given $\eta_{i1}$.
\end{itemize}
Path
diagrams for these cases are shown in Figure A.1 in the
supplementary appendix.
In each setting, the parameters are set so that the marginal
distributions of $V_{i1}$ and $V_{i2}$ have means $\E(V_{i1})=\E(V_{i2})=0$
and variances $\var(V_{i1})=\var(V_{i2})=1$.
In each case we focus on estimates of the regression coefficient
$\beta_{1}$. The true value of $\beta_{1}$ is set so that the $R^{2}$
statistic for the response $V_{i2}$ (i.e.\ $\eta_{1}$, $Z_{i}$ or
$\eta_{2i}$) is $R^{2}=0, 0.2$ or $0.4$.

Each latent variable $\eta_{i}$ (which in case 3 means either
$\eta_{i1}$ or $\eta_{i2}$) is measured by $p$ conditionally independent
binary items $Y_{ij}$ with values 0 and 1, and with $p=4$ or $p=8$. The
measurement model is of the form $\text{logit}[P(Y_{ij}=1|\eta_{i})] =
\tau_{j} + \lambda_{j} \eta_{i}$. The $(\tau_{j},\lambda_{j})$ are
estimated as distinct parameters for different items $j$, but their
true values $\tau_{j}=\tau$ and $\lambda_{j}=\lambda$ in the
data-generating model
are set to be equal for all items
in a given simulation. The loading parameter $\lambda$ is set with reference to the
linear model for a notional continuous latent variable $Y_{ij}^{*}$
which can be used to motivate the logistic model for $Y_{ij}$. Here
$R^{2}$ for $Y_{ij}^{*}$ given $\eta_{i}$ is
$R^{2}_{Y}=\lambda^{2}\var(\eta_{i})/(\lambda^{2}\var(\eta_{i})+\pi^{2}/3)$
where $\var(\eta_{i})=1$, and $\lambda$ is set so that $R^{2}_{Y}=0.4$
or 0.6. The values of $\tau$ are then set so that the marginal
probability $\pi_{Y}=P(Y_{ij}=1)$ is 0.5 or (approximately) 0.8. When
the model is estimated, we constrain $\tau_{1}=0$ and $\lambda_{1}=1$
for one item $Y_{i1}$ per latent variable, leaving $\tau_{j}$ and
$\lambda_{j}$ for the other $p-1$ items estimable. This affects the
implied scale of the latent variable(s), so that the value of
$\beta_{1}$ that the estimators should be estimating will depend also on
the value of $\lambda$ in the data-generating model. These true values
of $\beta_{1}$ are shown in the results tables.

In each setting we generate 1000 simulated datasets with $n=200$ or
$n=1000$ independent observations $i$. All combinations of $n$, $R^{2}$,
$p$, $\pi_{Y}$ and $R^{2}_{Y}$ are considered, resulting in 48 settings
in each of cases 1(a), 1(b), 2(a), 2(b) and 3. We calculate the two-step
estimator, and for comparison also the one-step and naive three-step
estimators. Two-step estimation was implemented as described in Section
\ref{ss_estimation_implementation} above. The same R functions we wrote
for it (combined with Mplus) also produced the one-step and three-step
estimates.

For the point estimates of the parameter of interest $\beta_{1}$ the
tables of results include their mean bias, root mean squared error
(RMSE) and median absolute error (MAE). To examine the estimated
standard errors, we report the standard deviation of the
point estimates of $\beta_{1}$ and the mean of their estimated standard errors
across the simulations (these two for just the one-step and two-step estimates), and
simulation coverage of the 95\% confidence intervals. For the two-step
estimates we also report two quantities for the case where the
uncertainty from step 1 is ignored (i.e.\ when we include only the
contribution from $\mathbf{V}_{2}$ in (\ref{Vmat}) in the estimated
standard error): the average proportion of the overall estimated
variance of the two-step estimate of $\beta_{1}$ that this accounts for,
and the coverage of the resulting 95\% confidence interval.

The first key finding is that the results are essentially similar in all
the five cases. In other words, the relative behaviour of the estimates in the
different settings is broadly the same irrespective of whether the latent
variable appears as covariate or response variable (or both) in
(\ref{sim_model}), and does not depend on the distribution of an observed response
$Z_{i}$ or covariate $X_{i}$. Because of this we focus on the most
complex case 3 where both the covariate and the response are latent. The
results for it are shown in Table \ref{t_sim_beta} for the point
estimates and Table \ref{t_sim_se} for the standard errors. The results
for all the simulations are shown in the same form in Tables A.1--A.10 of the
supplementary appendix.

We focus first on the comparison between the two-step and one-step
estimates. Consider the results for the point estimates of $\beta_{1}$,
in Table \ref{t_sim_beta}. There is no meaningful difference between the
two estimators for the larger sample size of $n=1000$. They also behave
very similarly in almost all settings when $n=200$, except for some
differences  when the number of items is small ($p=4$),
the measurement model is weak ($R^{2}_{Y}=0.4$) and one of the values of
the items is rare ($\pi_{Y}=0.8$). This is the setting where the items
provide the least information about the latent variables. When
$\beta_{1}$ is 0, the two-step estimates have in these cases a slightly
better RMSE and MAE than the one-step estimates. With larger values of
$\beta_{1}$, on the other hand, in a few simulations these cases
yield an extreme value of the two-step estimate, inflating its RMSE.
These extremes are reduced for the one-step estimates, most likely
because the measurement models of $\eta_{1}$ and $\eta_{2}$ stabilise
each other when they are estimated together. Essentially similar results
are obtained for cases 1 and 2, as shown in supplementary Tables
A.1--A.4. A small difference is that in case 2 the two
estimators perform very similarly even in the most difficult settings.
A possible explanation of this is that when the
latent $\eta$ is only an explanatory variable, a weak estimated
measurement model for it tends to attenuate rather than exaggerate the
estimate of~$\beta_{1}$.

Simulation results for the estimated standard errors of the estimates of
$\beta_{1}$ for case 3 are reported in Table \ref{t_sim_se}. They are
very satisfactory when $n=1000$: for both the two-step and one-step methods the
standard errors are good estimates of the simulation standard deviations
of estimates of $\beta_{1}$, and coverages of confidence intervals are
close to the nominal 95\%. With the smaller sample size of $n=200$ this
coverage is less correct, ranging from 89\% to 100\% in different
settings. The estimated standard errors are even then reasonably
accurate when the estimated measurement model is sufficiently
informative. In the most difficult settings, with a small sample, weak
measurement model and small number of items, the standard errors tend
to underestimate the true variability. These results are again broadly
similar in the other cases, with Case 2 being the easiest (see
supplementary tables A.6--A.9).

The last two columns of Table \ref{t_sim_se} (and the corresponding
supplementary tables) examine what happens if we ignore the uncertainty
from step 1 of two-step estimation, i.e.\ if we calculate its standard
errors using only the term $\mathbf{V}_{2}$ in the variance formula
(\ref{Vmat}). Here this step-2 variance accounts for almost all of the
variance of the estimate of $\beta_{1}$ when the true $\beta_{1}$ is 0,
but much less otherwise, down to less than a third of the overall
variance in the settings with the largest $\beta_{1}$. The coverage of
confidence intervals is correspondingly reduced, down to less than 70\%.
When the association in the structural model is strong, the uncertainty
in its estimates is thus dominated by the uncertainty in the estimated
measurement parameters from step 1.

Consider then the naive three-step estimates. The key determinant of
how well they behave is the role in which the latent variable(s) appear
in the model. In case 2, where $\eta$ is only an explanatory variable,
the three-step point estimates behave essentially identically with one-
and two-step estimates, but in cases 1 and 3 where $\eta$ (or
$\eta_{1}$) appears as a response variable they are clearly biased when
the true $\beta_{1}$ is not 0. This result was expected based on
previous literature and general results for measurement error models, as
discussed in Section~\ref{ss_estimation_onethree}. The standard errors
of three-step estimates are clearly underestimated when $\beta_{1}$
is not 0. The situation here is essentially the same as for two-step
standard errors which ignore the step-1 uncertainty, in that naive
three-step estimation takes the factor scores $\tilde{\eta}_{ik}$ ---
and thus the estimated measurement parameters which are used to
calculate them --- as known, resulting in underestimation of the full
uncertainty. It could be argued, however, that it would be more in the
spirit of three-step estimation to regard the factor scores as observed
variables, fixed before the estimation of the structural model starts.
If we adopt this view, the structural model of interest is the
regression model involving the factor scores. Both the measurement error
bias of the point estimates and the underestimation of their standard
errors would then be absent by definition.

The conclusions from these simulations are broadly similar to previous
ones for other types of models by \cite{bakk-kuha} and
\cite{rosseel+loh22}. In all of these contexts two-step and
one-step estimates mostly perform very similarly. Differences
between them emerge in difficult situations where the
measurement models are weak and sample sizes are small. Here there is
some variation between the findings of the different studies. In the
latent class examples of \cite{bakk-kuha}, two-step estimation does
relatively poorly in difficult settings, possibly because an
observed covariate or response provides particularly useful
stabilisation for one-step estimates there. In contrast, in the
simulations of  structural equation models by
\cite{rosseel+loh22}, it is the one-step estimates that perform much
less well in the most difficult settings. This may be in part because
the models considered in their simulations were more complex (with five latent factors)
and included no observed covariates or responses, which may be
particularly challenging for the small-sample performance of one-step
estimation. In their more extensive simulations,
\cite{rosseel+loh22} also considered various cases where the analysis
models were misspecified (a question that was not included here). There
too they found that this had a more deleterious effect on one-step
estimates, especially in the more demanding settings.
In our examples here the differences between one- and
two-step estimates in the hardest situations are somewhat smaller than
in these previous studies, but
it should be noted that the exact simulation settings are not easily
comparable.

\clearpage
\newgeometry{top=10mm}
%\thispagestyle{empty}
\begin{table}[ht]
\caption{Simulation results for point estimates of structural regression
coefficient $\beta_{1}$ of
a latent covariate $\eta_{1}$ for a conditionally normally distributed
latent response $\eta_{2}$ (simulation Case 3).}
%\centering
{\small{
\begin{tabular}{|rrrrr|rrr|rrr|rrr|}
  \hline
  & & & &
  & \multicolumn{3}{|c|}{Bias}
  & \multicolumn{3}{|c|}{RMSE}
  & \multicolumn{3}{|c|}{MAE} \\
$p$ & $\pi_{Y}$ & $R^{2}_{Y}$ & $R^{2}_{\eta}$ & $\beta_1$ & 2-step & 1-step & 3-step & 2-step & 1-step & 3-step & 2-step & 1-step & 3-step \\
  \hline \multicolumn{14}{|l|}{$n=200$}\\ \hline
4 & 0.5 & 0.4 & 0.0 & 0.000 & -0.004 & -0.003 & -0.002 & 0.133 & 0.141 & 0.081 & 0.077 & 0.080 & 0.046 \\
  4 & 0.8 & 0.4 & 0.0 & 0.000 & -0.007 & -0.004 & -0.000 & 0.240 & 0.276 & 0.110 & 0.101 & 0.113 & 0.044 \\
  4 & 0.5 & 0.6 & 0.0 & 0.000 & 0.007 & 0.007 & 0.005 & 0.115 & 0.117 & 0.080 & 0.069 & 0.070 & 0.048 \\
  4 & 0.8 & 0.6 & 0.0 & 0.000 & 0.002 & 0.002 & 0.003 & 0.153 & 0.154 & 0.091 & 0.081 & 0.084 & 0.048 \\
  8 & 0.5 & 0.4 & 0.0 & 0.000 & -0.001 & -0.001 & -0.001 & 0.104 & 0.104 & 0.078 & 0.063 & 0.062 & 0.047 \\
  8 & 0.8 & 0.4 & 0.0 & 0.000 & 0.008 & 0.009 & 0.006 & 0.147 & 0.151 & 0.087 & 0.081 & 0.083 & 0.047 \\
  8 & 0.5 & 0.6 & 0.0 & 0.000 & -0.003 & -0.003 & -0.002 & 0.097 & 0.095 & 0.077 & 0.062 & 0.059 & 0.050 \\
  8 & 0.8 & 0.6 & 0.0 & 0.000 & -0.002 & -0.002 & -0.000 & 0.114 & 0.116 & 0.080 & 0.072 & 0.073 & 0.050 \\
  \hline
  4 & 0.5 & 0.4 & 0.2 & 0.447 & 0.016 & 0.035 & -0.162 & 0.236 & 0.237 & 0.221 & 0.132 & 0.132 & 0.202 \\
  4 & 0.8 & 0.4 & 0.2 & 0.447 & 0.031 & 0.070 & -0.208 & 0.404 & 0.385 & 0.300 & 0.183 & 0.177 & 0.267 \\
  4 & 0.5 & 0.6 & 0.2 & 0.447 & 0.019 & 0.025 & -0.116 & 0.182 & 0.189 & 0.176 & 0.117 & 0.116 & 0.148 \\
  4 & 0.8 & 0.6 & 0.2 & 0.447 & 0.030 & 0.039 & -0.144 & 0.257 & 0.244 & 0.225 & 0.135 & 0.134 & 0.183 \\
  8 & 0.5 & 0.4 & 0.2 & 0.447 & 0.013 & 0.010 & -0.104 & 0.177 & 0.188 & 0.170 & 0.109 & 0.110 & 0.137 \\
  8 & 0.8 & 0.4 & 0.2 & 0.447 & 0.023 & 0.040 & -0.152 & 0.249 & 0.250 & 0.222 & 0.144 & 0.136 & 0.197 \\
  8 & 0.5 & 0.6 & 0.2 & 0.447 & 0.005 & -0.008 & -0.081 & 0.143 & 0.159 & 0.143 & 0.092 & 0.093 & 0.116 \\
  8 & 0.8 & 0.6 & 0.2 & 0.447 & 0.024 & 0.028 & -0.100 & 0.183 & 0.181 & 0.170 & 0.114 & 0.111 & 0.137 \\
  \hline
  4 & 0.5 & 0.4 & 0.4 & 0.632 & 0.028 & 0.047 & -0.223 & 0.309 & 0.285 & 0.301 & 0.176 & 0.162 & 0.268 \\
  4 & 0.8 & 0.4 & 0.4 & 0.632 & 0.042 & 0.081 & -0.284 & 0.608 & 0.457 & 0.436 & 0.228 & 0.199 & 0.363 \\
  4 & 0.5 & 0.6 & 0.4 & 0.632 & 0.048 & 0.052 & -0.143 & 0.275 & 0.254 & 0.247 & 0.140 & 0.147 & 0.199 \\
  4 & 0.8 & 0.6 & 0.4 & 0.632 & 0.038 & 0.048 & -0.195 & 0.326 & 0.283 & 0.295 & 0.174 & 0.161 & 0.255 \\
  8 & 0.5 & 0.4 & 0.4 & 0.632 & 0.020 & 0.022 & -0.142 & 0.213 & 0.222 & 0.216 & 0.131 & 0.134 & 0.176 \\
  8 & 0.8 & 0.4 & 0.4 & 0.632 & 0.033 & 0.054 & -0.204 & 0.344 & 0.346 & 0.309 & 0.177 & 0.166 & 0.258 \\
  8 & 0.5 & 0.6 & 0.4 & 0.632 & 0.033 & 0.006 & -0.087 & 0.196 & 0.227 & 0.182 & 0.127 & 0.124 & 0.138 \\
  8 & 0.8 & 0.6 & 0.4 & 0.632 & 0.024 & 0.030 & -0.138 & 0.223 & 0.218 & 0.220 & 0.137 & 0.135 & 0.180 \\
  \hline \multicolumn{14}{|l|}{$n=1000$}\\ \hline
  4 & 0.5 & 0.4 & 0.0 & 0.000 & 0.000 & 0.000 & 0.000 & 0.053 & 0.054 & 0.032 & 0.036 & 0.036 & 0.021 \\
  4 & 0.8 & 0.4 & 0.0 & 0.000 & -0.001 & -0.001 & -0.000 & 0.072 & 0.073 & 0.032 & 0.049 & 0.049 & 0.021 \\
  4 & 0.5 & 0.6 & 0.0 & 0.000 & -0.001 & -0.001 & -0.001 & 0.045 & 0.045 & 0.031 & 0.030 & 0.030 & 0.021 \\
  4 & 0.8 & 0.6 & 0.0 & 0.000 & 0.001 & 0.001 & 0.001 & 0.054 & 0.055 & 0.032 & 0.036 & 0.036 & 0.021 \\
  8 & 0.5 & 0.4 & 0.0 & 0.000 & 0.000 & 0.000 & 0.000 & 0.045 & 0.045 & 0.033 & 0.031 & 0.031 & 0.023 \\
  8 & 0.8 & 0.4 & 0.0 & 0.000 & 0.001 & 0.001 & 0.001 & 0.059 & 0.059 & 0.034 & 0.038 & 0.038 & 0.022 \\
  8 & 0.5 & 0.6 & 0.0 & 0.000 & -0.001 & -0.001 & -0.000 & 0.040 & 0.040 & 0.032 & 0.026 & 0.026 & 0.021 \\
  8 & 0.8 & 0.6 & 0.0 & 0.000 & 0.000 & 0.000 & 0.000 & 0.045 & 0.045 & 0.032 & 0.030 & 0.030 & 0.021 \\
  \hline
  4 & 0.5 & 0.4 & 0.2 & 0.447 & 0.006 & 0.012 & -0.174 & 0.092 & 0.090 & 0.183 & 0.061 & 0.059 & 0.178 \\
  4 & 0.8 & 0.4 & 0.2 & 0.447 & 0.011 & 0.021 & -0.227 & 0.124 & 0.124 & 0.236 & 0.078 & 0.077 & 0.234 \\
  4 & 0.5 & 0.6 & 0.2 & 0.447 & 0.001 & 0.003 & -0.131 & 0.075 & 0.074 & 0.142 & 0.053 & 0.052 & 0.135 \\
  4 & 0.8 & 0.6 & 0.2 & 0.447 & 0.011 & 0.013 & -0.160 & 0.091 & 0.090 & 0.171 & 0.057 & 0.059 & 0.168 \\
  8 & 0.5 & 0.4 & 0.2 & 0.447 & 0.004 & 0.005 & -0.112 & 0.073 & 0.073 & 0.125 & 0.047 & 0.047 & 0.116 \\
  8 & 0.8 & 0.4 & 0.2 & 0.447 & 0.010 & 0.015 & -0.163 & 0.094 & 0.093 & 0.174 & 0.062 & 0.061 & 0.166 \\
  8 & 0.5 & 0.6 & 0.2 & 0.447 & 0.002 & 0.003 & -0.084 & 0.062 & 0.062 & 0.098 & 0.039 & 0.039 & 0.087 \\
  8 & 0.8 & 0.6 & 0.2 & 0.447 & 0.008 & 0.009 & -0.113 & 0.074 & 0.074 & 0.126 & 0.048 & 0.048 & 0.118 \\
  \hline
  4 & 0.5 & 0.4 & 0.4 & 0.632 & 0.009 & 0.014 & -0.244 & 0.118 & 0.112 & 0.255 & 0.076 & 0.076 & 0.250 \\
  4 & 0.8 & 0.4 & 0.4 & 0.632 & 0.003 & 0.014 & -0.319 & 0.156 & 0.150 & 0.329 & 0.103 & 0.099 & 0.329 \\
  4 & 0.5 & 0.6 & 0.4 & 0.632 & 0.006 & 0.007 & -0.177 & 0.093 & 0.090 & 0.190 & 0.061 & 0.058 & 0.182 \\
  4 & 0.8 & 0.6 & 0.4 & 0.632 & 0.009 & 0.010 & -0.218 & 0.117 & 0.109 & 0.232 & 0.074 & 0.071 & 0.223 \\
  8 & 0.5 & 0.4 & 0.4 & 0.632 & 0.003 & 0.005 & -0.158 & 0.088 & 0.086 & 0.172 & 0.059 & 0.057 & 0.164 \\
  8 & 0.8 & 0.4 & 0.4 & 0.632 & -0.001 & 0.007 & -0.231 & 0.114 & 0.111 & 0.244 & 0.078 & 0.072 & 0.237 \\
  8 & 0.5 & 0.6 & 0.4 & 0.632 & 0.007 & 0.008 & -0.110 & 0.080 & 0.079 & 0.128 & 0.054 & 0.053 & 0.112 \\
  8 & 0.8 & 0.6 & 0.4 & 0.632 & 0.006 & 0.008 & -0.152 & 0.088 & 0.086 & 0.167 & 0.060 & 0.058 & 0.156 \\
   \hline
\multicolumn{14}{l}{\emph{Note:} $p$ denotes number of measurement
items $Y_{j}$, $\pi_{Y}$ marginal proportion of $Y_{j}=1$, }\\
\multicolumn{14}{l}{
\hspace*{2em}$R_{Y}^{2}$ and $R^{2}_{\eta}$ the $R^{2}$ statistics in models for
$Y_{j}$ and $\eta_{2}$, and $\beta_{1}$ true value of $\beta_{1}$.}
\end{tabular}
}}
\label{t_sim_beta}
\end{table}
\clearpage


\thispagestyle{empty}
\begin{table}[ht]
\caption{Simulation results for standard error estimates of estimated structural regression
coefficient $\beta_{1}$ of a latent covariate $\eta_{1}$ for a conditionally
normally distributed latent response~$\eta_{2}$ (simulation Case 3).}
%\centering
{\small{
\begin{tabular}{|rrrrr|rrrr|rrr|rr|}
  \hline
  &&&&
  & \multicolumn{4}{c|}{Simulation standard deviation}
  &&&
  & \multicolumn{2}{c|}{2-step}
  \\
  &&&&
  & \multicolumn{4}{c|}{vs.\ mean of est.\ std.\ error}
  & \multicolumn{3}{c|}{Coverage of}
  & \multicolumn{2}{c|}{without}
  \\
  &&&&
  & \multicolumn{2}{|c}{2-step}
  & \multicolumn{2}{c|}{1-step}
  & \multicolumn{3}{c|}{95\% conf.\ interval}
  & \multicolumn{2}{c|}{step-1 var.$^{\dagger}$}
\\
$p$ & $p_{Y}$ & $R^{2}_{Y}$ & $R^{2}_{\eta}$ & $\beta_1$
& s.d.\ & m(se) & s.d.\ & m(se) & 2-st.\ & 1-st.\ & 3-st.\ & \%var. &
{\footnotesize{cover.}} \\
\hline \multicolumn{14}{|l|}{$n=200$}\\ \hline
4 & 0.5 & 0.4 & 0.0 & 0.000 & 0.133 & 0.137 & 0.141 & 0.138 & 99.5 & 99.3 & 95.6 & 87.3 & 96.0 \\
  4 & 0.8 & 0.4 & 0.0 & 0.000 & 0.240 & 0.224 & 0.276 & 0.235 & 99.9 & 100.0 & 94.7 & 78.0 & 96.0 \\
  4 & 0.5 & 0.6 & 0.0 & 0.000 & 0.115 & 0.114 & 0.116 & 0.114 & 98.9 & 98.5 & 94.3 & 91.4 & 94.9 \\
  4 & 0.8 & 0.6 & 0.0 & 0.000 & 0.153 & 0.146 & 0.154 & 0.145 & 99.2 & 98.6 & 94.2 & 88.2 & 95.1 \\
  8 & 0.5 & 0.4 & 0.0 & 0.000 & 0.105 & 0.105 & 0.105 & 0.102 & 98.5 & 98.5 & 95.7 & 92.5 & 95.8 \\
  8 & 0.8 & 0.4 & 0.0 & 0.000 & 0.147 & 0.143 & 0.151 & 0.143 & 99.3 & 99.0 & 95.0 & 87.1 & 95.9 \\
  8 & 0.5 & 0.6 & 0.0 & 0.000 & 0.097 & 0.095 & 0.095 & 0.090 & 98.4 & 98.1 & 95.5 & 94.3 & 95.7 \\
  8 & 0.8 & 0.6 & 0.0 & 0.000 & 0.115 & 0.112 & 0.116 & 0.112 & 99.1 & 99.0 & 94.5 & 92.1 & 95.0 \\
\hline
  4 & 0.5 & 0.4 & 0.2 & 0.447 & 0.236 & 0.223 & 0.235 & 0.217 & 90.5 & 90.5 & 36.8 & 41.3 & 77.9 \\
  4 & 0.8 & 0.4 & 0.2 & 0.447 & 0.403 & 0.333 & 0.378 & 0.331 & 88.5 & 90.3 & 28.2 & 41.6 & 75.4 \\
  4 & 0.5 & 0.6 & 0.2 & 0.447 & 0.181 & 0.183 & 0.187 & 0.178 & 92.8 & 92.7 & 48.6 & 43.3 & 82.8 \\
  4 & 0.8 & 0.6 & 0.2 & 0.447 & 0.256 & 0.230 & 0.241 & 0.219 & 91.5 & 91.9 & 38.4 & 42.8 & 79.3 \\
  8 & 0.5 & 0.4 & 0.2 & 0.447 & 0.176 & 0.162 & 0.188 & 0.160 & 90.7 & 89.3 & 48.7 & 39.7 & 77.0 \\
  8 & 0.8 & 0.4 & 0.2 & 0.447 & 0.248 & 0.226 & 0.247 & 0.226 & 91.1 & 91.8 & 36.5 & 39.7 & 75.1 \\
  8 & 0.5 & 0.6 & 0.2 & 0.447 & 0.143 & 0.143 & 0.159 & 0.137 & 92.1 & 90.0 & 57.4 & 42.3 & 78.9 \\
  8 & 0.8 & 0.6 & 0.2 & 0.447 & 0.181 & 0.175 & 0.179 & 0.171 & 92.1 & 92.6 & 51.4 & 40.9 & 78.5 \\
\hline
  4 & 0.5 & 0.4 & 0.4 & 0.632 & 0.308 & 0.295 & 0.281 & 0.263 & 91.9 & 92.7 & 28.6 & 27.3 & 72.2 \\
  4 & 0.8 & 0.4 & 0.4 & 0.632 & 0.607 & 0.439 & 0.450 & 0.385 & 89.7 & 92.8 & 19.9 & 27.9 & 69.6 \\
  4 & 0.5 & 0.6 & 0.4 & 0.632 & 0.271 & 0.254 & 0.248 & 0.229 & 94.0 & 94.6 & 36.7 & 28.8 & 73.8 \\
  4 & 0.8 & 0.6 & 0.4 & 0.632 & 0.324 & 0.316 & 0.279 & 0.267 & 93.2 & 94.5 & 29.5 & 28.3 & 71.2 \\
  8 & 0.5 & 0.4 & 0.4 & 0.632 & 0.212 & 0.208 & 0.221 & 0.203 & 92.5 & 92.5 & 40.7 & 24.9 & 66.1 \\
  8 & 0.8 & 0.4 & 0.4 & 0.632 & 0.342 & 0.284 & 0.342 & 0.280 & 91.0 & 92.7 & 27.6 & 24.6 & 65.1 \\
  8 & 0.5 & 0.6 & 0.4 & 0.632 & 0.193 & 0.189 & 0.227 & 0.175 & 94.6 & 90.2 & 47.7 & 25.9 & 66.8 \\
  8 & 0.8 & 0.6 & 0.4 & 0.632 & 0.221 & 0.218 & 0.216 & 0.211 & 93.2 & 94.1 & 38.4 & 26.2 & 68.5 \\
\hline \multicolumn{14}{|l|}{$n=1000$}\\ \hline
  4 & 0.5 & 0.4 & 0.0 & 0.000 & 0.053 & 0.054 & 0.054 & 0.054 & 95.9 & 95.7 & 95.1 & 97.1 & 95.1 \\
  4 & 0.8 & 0.4 & 0.0 & 0.000 & 0.072 & 0.076 & 0.073 & 0.076 & 97.9 & 97.2 & 96.3 & 95.1 & 96.3 \\
  4 & 0.5 & 0.6 & 0.0 & 0.000 & 0.045 & 0.046 & 0.045 & 0.046 & 96.5 & 96.4 & 95.2 & 98.3 & 95.4 \\
  4 & 0.8 & 0.6 & 0.0 & 0.000 & 0.054 & 0.056 & 0.055 & 0.056 & 96.8 & 96.8 & 95.9 & 97.6 & 96.2 \\
  8 & 0.5 & 0.4 & 0.0 & 0.000 & 0.045 & 0.044 & 0.045 & 0.044 & 95.3 & 95.2 & 94.5 & 98.2 & 94.5 \\
  8 & 0.8 & 0.4 & 0.0 & 0.000 & 0.059 & 0.056 & 0.059 & 0.056 & 95.0 & 94.8 & 93.7 & 96.9 & 94.0 \\
  8 & 0.5 & 0.6 & 0.0 & 0.000 & 0.040 & 0.040 & 0.040 & 0.040 & 96.1 & 95.9 & 95.2 & 98.8 & 95.2 \\
  8 & 0.8 & 0.6 & 0.0 & 0.000 & 0.045 & 0.046 & 0.046 & 0.046 & 96.3 & 96.2 & 95.1 & 98.4 & 95.3 \\
\hline
  4 & 0.5 & 0.4 & 0.2 & 0.447 & 0.092 & 0.091 & 0.089 & 0.089 & 94.4 & 95.4 & 5.9 & 41.6 & 79.5 \\
  4 & 0.8 & 0.4 & 0.2 & 0.447 & 0.124 & 0.123 & 0.122 & 0.120 & 93.6 & 94.1 & 3.2 & 42.6 & 80.2 \\
  4 & 0.5 & 0.6 & 0.2 & 0.447 & 0.075 & 0.076 & 0.074 & 0.074 & 94.8 & 94.6 & 12.6 & 43.8 & 79.3 \\
  4 & 0.8 & 0.6 & 0.2 & 0.447 & 0.090 & 0.090 & 0.089 & 0.089 & 96.0 & 95.8 & 8.4 & 43.8 & 81.3 \\
  8 & 0.5 & 0.4 & 0.2 & 0.447 & 0.073 & 0.070 & 0.073 & 0.069 & 93.9 & 94.6 & 18.8 & 39.3 & 77.7 \\
  8 & 0.8 & 0.4 & 0.2 & 0.447 & 0.093 & 0.091 & 0.092 & 0.091 & 93.9 & 94.6 & 8.5 & 38.7 & 76.6 \\
  8 & 0.5 & 0.6 & 0.2 & 0.447 & 0.062 & 0.062 & 0.062 & 0.062 & 94.6 & 94.7 & 28.6 & 41.9 & 79.2 \\
  8 & 0.8 & 0.6 & 0.2 & 0.447 & 0.074 & 0.073 & 0.073 & 0.072 & 95.1 & 94.8 & 17.9 & 41.4 & 80.1 \\
\hline
  4 & 0.5 & 0.4 & 0.4 & 0.632 & 0.118 & 0.118 & 0.112 & 0.109 & 94.3 & 94.9 & 2.3 & 28.9 & 70.9 \\
  4 & 0.8 & 0.4 & 0.4 & 0.632 & 0.156 & 0.155 & 0.149 & 0.143 & 93.4 & 93.8 & 1.6 & 29.1 & 70.4 \\
  4 & 0.5 & 0.6 & 0.4 & 0.632 & 0.093 & 0.097 & 0.089 & 0.092 & 95.7 & 95.9 & 6.8 & 30.5 & 73.9 \\
  4 & 0.8 & 0.6 & 0.4 & 0.632 & 0.116 & 0.115 & 0.109 & 0.108 & 94.7 & 94.7 & 4.0 & 30.4 & 73.9 \\
  8 & 0.5 & 0.4 & 0.4 & 0.632 & 0.088 & 0.089 & 0.086 & 0.087 & 95.5 & 96.2 & 7.9 & 24.6 & 68.0 \\
  8 & 0.8 & 0.4 & 0.4 & 0.632 & 0.114 & 0.114 & 0.110 & 0.111 & 93.3 & 94.5 & 2.6 & 24.5 & 65.3 \\
  8 & 0.5 & 0.6 & 0.4 & 0.632 & 0.080 & 0.080 & 0.078 & 0.077 & 95.1 & 94.6 & 19.8 & 26.3 & 69.2 \\
  8 & 0.8 & 0.6 & 0.4 & 0.632 & 0.088 & 0.092 & 0.086 & 0.089 & 95.5 & 95.8 & 10.5 & 26.3 & 70.8 \\
   \hline
\multicolumn{14}{l}{\emph{Note:} $p$ denotes number of measurement
items $Y_{j}$, $\pi_{Y}$ marginal proportion of $Y_{j}=1$, }\\
\multicolumn{14}{l}{
\hspace*{2em}$R_{Y}^{2}$ and $R^{2}_{\eta}$ the $R^{2}$ statistics in models for
$Y_{j}$ and $\eta_{2}$, and $\beta_{1}$ true value of $\beta_{1}$.}\\
\multicolumn{14}{l}{
$\dagger$ Average \% of the variance of 2-step estimator accounted
for by step-2 variance alone,}\\
\multicolumn{14}{l}{
\hspace*{1em} and coverage of 95\% confidence interval if only this variance is
included.}\\
\end{tabular}
}}
\label{t_sim_se}
\end{table}
\clearpage
\restoregeometry

\section{Application: Extrinsic and intrinsic work values}
\label{s_example}

Here we apply the methods to a real-data example. We use it,
in particular, to further explore two topics which were not the focus
of the simulations in Section \ref{s_simulation}: the possibility of interpretative
confounding and computational demands of the different estimators.

The example concerns individuals' \emph{work value orientations},
conceptualised in two dimensions: intrinsic and extrinsic work values
(for more information about substantive research in this area, see for
example \cite{gesthuizenetal19} and \cite{bacheretal22}, and references
cited therein). We use data from the fifth wave of the European Values
Study (EVS), conducted in 2017--20 \citep{EVS22}. The survey
includes six items on work values, introduced (in the English-language
version) by \emph{Here are some aspects of a job that people say are
important. Please look at them and tell me which ones you personally
think are important in a job?} The responses are binary, coded as 1 if
the respondent mentioned that aspect and 0 if they did not (and with
``Don't know'' responses coded as missing). Three of the items are
treated as measures of extrinsic work values --- ``Good pay'' (labelled
\emph{Pay} in the tables below), ``Good hours'' (\emph{Hours}), and
``Generous holidays'' (\emph{Holidays}) --- and three of intrinsic work
values --- ``An opportunity to use initiative'' (\emph{Initiative}), ``A
job in which you feel you can achieve something'' (\emph{Achievement}),
and ``A responsible job'' (\emph{Responsibility}). For more discussion
of the measurement of work values in the EVS, see \cite{gesthuizenetal19}.

To provide a focus for this illustrative application, we consider
specifically gender differences in work value orientations between men
and women (in EVS, gender is coded as a binary variable). This question
was also examined, for example, by \cite{bacheretal22}, using data for
Austria from a different survey (The Social Survey of Austria).

We carry out two analyses. The second of them will be a cross-sectional
analysis of all 36 countries in the EVS data. But first we consider just
one country, the Netherlands. In addition to the work value items and
gender, a set of covariates are included: the respondent's age (in
categories: 15--29, 30--49 or 50-- years), whether they are in a
registered partnership and/or live with a partner (yes/no), the age of the youngest
person in the household (0--5, 6-17, or older), respondent's highest
level of education (less than upper secondary, upper secondary, or higher), occupation-based
social class (European Socioeconomic Classification [ESeC08]; grouped by
type of employment regulation of the occupation as Service relationship
[ESeC08 classes 1,2], Mixed [3,6], Labour contract [7,8,9] or
Self-employed [4,5]), whether they are currently in paid employment
(yes/no), and the highest education level of the respondent's parents
(coded as for own education). These parallel, as far as possible with
the EVS data, the covariates used by \cite{bacheretal22}. This analysis
dataset for the Netherlands includes $n=1374$
respondents with observed values of all the covariates and at least
one of the six value items.

In the notation of Section \ref{s_models}, we consider here the
structural model $p(\boldsymbol{\eta}_{i(1)}|\mathbf{X}_{i};
\boldsymbol{\theta}_{21})$ where
$\boldsymbol{\eta}_{i(1)}=(\eta_{i1},\eta_{i2})'$ and $\eta_{i1}$ stands
for extrinsic and $\eta_{i2}$ for intrinsic work values. They are
responses to covariates $\mathbf{X}_{i}$, and there are no observed
response variables $\mathbf{Z}_{i}$. As defined in (\ref{eta_model1}),
the model includes separate linear models for $\eta_{i1}$ and
$\eta_{i2}$, plus their conditional association which we report as the
correlation $\rho=\text{corr}(\eta_{i1},\eta_{i2}|\boldsymbol{X}_{i})$.
The two latent variables are measured by the corresponding 3 items each,
with logistic measurement models (\ref{logit}).

\newpage
We consider three models, with different sets of covariates: Model (1)
which includes only gender, Model (2) which adds respondent's age,
partnership status and age of the youngest person in the household, and Model
(3) which includes all the covariates. Estimates
are shown in Table \ref{t_example_NL}. This includes only the
loading parameters ($\lambda_{kj}$) of the measurement models, the
regression coefficient of gender (as dummy variable for men) and the
conditional correlation $\rho$. The full sets of structural coefficients
are shown in supplementary Tables B.1 and B.2.

\begin{table}
\caption{Models for extrinsic and intrinsic work values given gender,
controlling for different
sets of covariates,
estimated from EVS2017 data for Netherlands. The table shows two-step
(``2-st.''), one-step
(``1-st.'')
and naive three-step
(``3-st.'')
estimates. ``Step 1'' denotes the
estimated measurement loadings from step 1 of two-step estimation, which
are used for two-step and (for factor scores of) three-step estimation of all three models.
}
{\small{\hspace*{-3em}
\begin{tabular}{|l|r|rrr|rrr|rrr|}\hline
& &
\multicolumn{3}{|c|}{Model (1)$^{*}$} &
\multicolumn{3}{|c|}{Model (2)$^{*}$} &
\multicolumn{3}{|c|}{Model (3)$^{*}$} \\
 & Step 1 &
2-st.\ & 1-st.\ & 3-st. &
2-st.\ & 1-st.\ & 3-st. &
2-st.\ & 1-st.\ & 3-st. \\ \hline
\multicolumn{11}{|l|}{\emph{Model for Extrinsic work values}} \\ \hline
\multicolumn{11}{|l|}{Measurement loadings:} \\[.3ex]
\emph{Pay}      & 1.000 &  & 1.000 &  &  & 1.000 &  &  & 1.000 &  \\
\emph{Hours}    & 1.161 &  & 1.101 &  &  & 1.148 &  &  & 1.190 &  \\
\emph{Holidays} & 3.080 &  & 3.123 &  &  & 2.586 &  &  & 2.573 &  \\
\hline
\multicolumn{11}{|l|}{Structural model: coefficient of Gender (male)} \\[.3ex]
  Male &  & -0.042 & -0.033 & -0.033 & 0.025 & 0.022 & 0.024 & 0.018 & 0.009 & 0.018 \\
  (s.e.) &  & (0.103) & (0.104) & (0.082) & (0.101) & (0.104) &
  (0.080) &
  (0.102) & (0.104) & (0.081) \\ \hline\hline
\multicolumn{11}{|l|}{\emph{Model for Intrinsic work values}} \\ \hline
\multicolumn{11}{|l|}{Measurement loadings:} \\[.3ex]
\emph{Initiative} & 1.000 &  & 1.000 &  &  & 1.000 &  &  & 1.000 &  \\
\emph{Achievement} & 0.431 &  & 0.453 &  &  & 0.500 &  &  & 0.556 &  \\
\emph{Responsibility} & 0.563 &  & 0.589 &  &  & 0.631 &  &  & 0.804 &  \\
\hline
\multicolumn{11}{|l|}{Structural model: coefficient of Gender (male)} \\[.3ex]
  Male &  & 0.607 & 0.603 & 0.460 & 0.657 & 0.638 & 0.500 & 0.516 & 0.506 & 0.400 \\
  (s.e.) &  & (0.193) & (0.186) & (0.142) & (0.194) & (0.176) & (0.142)
  & (0.188) & (0.151) & (0.139) \\
\hline\hline
\multicolumn{11}{|l|}{\emph{Correlation of Extrinsic and
Intrinsic work values given covariates}} \\[.3ex]
  &  & 0.550 & 0.551 & 0.345 & 0.543 & 0.549 & 0.334 & 0.559 & 0.567 & 0.334 \\
\hline
\multicolumn{11}{l}{\footnotesize{* Model (1) includes only gender as
covariate, Model (2) also
respondent's age, partnership status and age of
}}\\
\multicolumn{11}{l}{\footnotesize{\hspace*{1em}
the youngest person
in the household, and Model (3) also
respondent's education, social class and work status,}} \\
\multicolumn{11}{l}{\footnotesize{\hspace*{1em}
and their parents'
education.}} \\
\multicolumn{11}{l}{\footnotesize{
The full sets of estimated coefficients of the structural model are
shown in the supplementary appendix.}}
\end{tabular}
}}
\label{t_example_NL}
\end{table}

Consider first the estimated measurement models in Table
\ref{t_example_NL}. The column ``Step 1'' shows the measurement loadings
that are obtained from step 1 of the two-step method. The measurement
parameters from it are used for step 2 for all choices of the set of
covariates $\mathbf{X}_{i}$. They are also used to calculate the factor
scores that are used for three-step estimation of all the models.

In contrast, the one-step method estimates the measurement model anew
whenever the structural model changes; these estimated loadings are
shown under the ``1-st.'' columns of the table. This raises the
possibility of interpretational confounding, where the definition of a
latent variable is affected not only by its indicators but also by its
predictors. Previously, \cite{bakk-kuha} examined this for latent class
models. In one of their examples, different choices of covariates
flipped the measurement model between two qualitatively quite different
configurations of the classes, so that comparisons of one-step estimates
between different structural models became effectively meaningless. The
differences are not similarly dramatic in our example here. All of the
one-step estimates of the loadings are positive, indicating that for
both the extrinsic and intrinsic factors higher values correspond to a
respondent placing higher importance on that dimension of work values,
and the loadings of different items are in the same order of size in
each case. This is perhaps not surprising, in that we might expect that
for continuous latent variables such changes in the measurement model
will be quantitative shifts rather than qualitative jumps. Nevertheless,
in this more subtle way the one-step estimates for the three models in
Table \ref{t_example_NL} are still for three slightly different response
variables which differ both from each other and from the response in all
of the two-step estimates.

Consider then the estimated structural models for the Netherlands.
In terms of the settings of the simulations in Section
\ref{s_simulation}, these data have $n=1374$ and $R^{2}_{\eta}$
around 0.1 for both latent factors, each of them measured by $p=3$
items with $R^{2}_{Y}$ of around 0.3--0.8 and $\pi_{Y}$ of 0.4--0.7
(different for different items). These place this example around the
middle of the simulation settings in terms of the strength of the
measurement model, but with a somewhat larger sample size than even the
larger $n=1000$ considered there. The simulations suggest that
the two-step and one-step estimates should both behave well here.
They are also similar to each other, with somewhat higher standard errors for
the two-step estimates. For the naive three-step estimates, for better
comparability we first re-scaled the factor scores so that their
variances equal the estimated marginal variances of the corresponding
latent variables from step 1 of two-step estimation. Even after this,
the three-step estimates of regression coefficients are attenuated toward zero, as would be
expected given the theoretical and simulation results above.

Substantively, the results in Table \ref{t_example_NL} show that
individuals' intrinsic and extrinsic work values, as measured by these
survey items, are strongly positively correlated even given covariates.
Between genders, there is essentially no difference in extrinsic values,
but for intrinsic ones men indicate on average significantly (at the 5\%
level) higher level of importance on them. Some other significant
associations with covariates are also found, as shown by the estimated
coefficients in supplementary Tables B.1 and B.2. The importance of
extrinsic rewards of a job (pay, hours and holidays) tends to be higher
for respondents who are younger, have young children at home, or are
currently in paid employment. For intrinsic values (related to the scope
of initiative, achievement and responsibility), the expected levels are
higher for individuals who have (or whose parents have) higher levels
education and who are in professional, administrative or managerial
occupations.

Summarising previous literature on gender differences in work values,
\cite{bacheretal22} observed that ``[...]intrinsic work values,
especially altruistic or social work values, seem still to be more
important for women than men, at least in some countries and/or in
combination with other factors[...]'' and that ``Gender differences in
extrinsic work value orientations are absent or smaller than those for
intrinsic work value orientations''. In their own analysis in Austria,
they also found higher levels of intrinsic values for women. Our
conclusion from the Netherlands is the opposite. This may be because the
three intrinsic items in EVS do not include ones of ``altruistic or
social work'' kind. But it could also be an instance of variation
between different country contexts. To examine this further, we can use
the EVS to carry out a comparative cross-national analysis. Here we
include as covariates the respondents' gender, age, education, social
class and work status; the other covariates were omitted from this
illustrative example because they had large proportions of missing data
in some countries. The list of the 36 countries and their sample sizes
in this analysis are shown in Tables B.3 and B.4 of the supplementary
appendix.

\begin{figure}[t]
\caption{Estimates of the coefficient (with 95\% confidence
intervals) of
dummy variable for men in models for intrinsic work values (circles and
solid lines) and extrinsic work values (triangles and dashed lines).
Two-step estimates obtained separately for each country in the 2017
European Values Survey, but with the same measurement model for each
country estimated from pooled data in step 1.}

\includegraphics[width=.95\textwidth]{countryplot}

\vspace*{-1ex}
{\scriptsize{\emph{Note}:
The coefficients are expressed on a scale where the residual
variance of both factors for the Netherlands is 1.
The models also include as covariates the respondent's age, education,
social class and employment status (working
vs.\ not).
}}
\label{f_countries}
\end{figure}

We allow all parameters of the structural model (regression coefficients
and residual variances and covariance) to have different values in each
country. Every parameter of the measurement model, in contrast, is
constrained to have the same value in all countries, to specify full
measurement equivalence across the countries. We note that because the
step-1 estimates are here based on much more data than the step-2
estimates (with nearly 59,000 observations in the pooled data but
1000--2000 in each country), step 1 contributes essentially nothing to
the standard errors of the estimated structural coefficients. So in this
example we could safely treat the measurement parameters from step 1 as
known and omit the term for them from the estimated standard errors.

Two-, one- and three-step estimates of gender differences in the work values in each country are
shown in supplementary Tables B.3 and B.4. We focus here on the two-step
estimates, which are also shown in Figure \ref{f_countries}. For
extrinsic values, most of the point estimates indicate higher average
importance of them for women, but these differences are small and mostly
not statistically significant (at the 5\% level). For intrinsic values
there are a few more countries with significant differences, but they
include ones in both directions and have no very clear geographic
regularities in the variation across countries. In other words, these results
do not show strong evidence of consistent gender differences in work
value orientations, suggesting at best small differences with much
cross-national variation in them.

As mentioned at the start of this this section, this application example
also allowed us to examine two topics in the relative performance of
two-step estimation. From the analysis for the Netherlands
above, we discussed the observed level of interpretative confounding in
one-step estimates. The second, cross-national analysis provides an
extreme illustration of the difference in computational convenience
between two- and one-step estimation. The model has 836 estimable
parameters. One-step estimates have to be calculated for all of them at
once, from the pooled dataset of all the countries. On our computer this
took 4.5 hours, using just one starting value. Two-step estimation, in
contrast, splits the task in a way which considerably simplifies it. In
its step 1, the measurement models are estimated from the pooled
dataset, as two one-factor models for three binary items each and
without any covariates. Step 2 can then be done separately for each
country, because their structural parameters are distinct. Even with
multiple starting values (\texttt{Starts=30 10} in Mplus syntax) this
took just 1-2 minutes for both step 1 and each country in step 2, for a
total of a little over 0.5 hours for the whole model.

\section{Conclusions and discussion}
\label{s_discussion}

Two-step estimation is a general approach which is in principle
applicable to any types of latent variable models where the focus of
interest is on the structural model. In this paper we have described it
for the broad family of latent trait (or item response theory) models,
where continuous latent variables are measured by observed categorical
indicators. In our simulations and examples, two-step estimation
performed very well for them, as was also expected on theoretical
grounds. This is also in line with previous findings on two-step
estimation for other families of models, in particular latent class
analysis of categorical variables and structural equation modelling of
continuous variables. Together these studies provide increasingly strong
evidence and reassurance that the two-step method is a useful and
convenient approach to estimation of latent variable models in general,
and an attractive alternative to the existing one-step and three-step
methods.

We also considered whether the different potential advantages of
two-step estimation were more or less prominent here than for other
families of models. One of them is practical and computational
convenience, compared to one-step estimation of all parts of the model
at once. This advantage may be particularly important for latent trait
models, because they are more often used with complex structural models
than are latent class models, and are also inherently more
computationally demanding than structural equation models. Another
attractive characteristic of the two-step approach is that it avoids
interpretational confounding, where the effective definition of the
latent variables is affected by the specification of the structural
model. This may be more consequential for latent class models, where the
interpretation of a categorical latent class variable can change in a
discontinuous way, than for latent trait models and structural equation
models which have continuous latent variables. However, interpretational confounding is
also an instance of the broader problem of biases that can be induced
when the models are misspecified. Two-step estimation is also expected to
provide some protection against them in any types of models.

To build on these promising results so far, different directions of
further research on different elements of two-step estimation could be
pursued. The accessibility of the approach would be much improved by
integrated implementation of it (including calculation of correct standard
errors of the estimates) in general-purpose software for latent variable
models, including latent trait models. This would
extend and complement the procedures that have already become available
for latent class and structural equation modelling. For better
understanding of the scope of two-step estimation, its implementation
and performance could be studied for other categories of models,
for example ones which combine different types of latent variables and
measurement items, or for extensions such as multilevel models and ones
with non-equivalent measurement models. These and other areas of
further research remain to be explored.

\bibliographystyle{chicago}
\bibliography{irt}

\clearpage
\appendix

\begin{center}
{\huge{
\textbf{Two-step estimation of latent trait models}

\textbf{Supplementary appendices}
}}
\end{center}

\vspace*{3ex}

These supplementary materials provide the following information:

\section{Results of the simulations (Section 4 of the paper)}
\label{s_a_sims}

\begin{itemize}
\item
Path diagrams that represent the cases 1, 2, and 3 that
are considered in the simulations of Section 4 of the paper are shown in Figure A.1.
\item
Full results of the simulations are
given Tables A.1--A.10:
\begin{itemize}
\item Case 1 (Observed covariate, latent response)
in Tables A.1--A.2 (point estimates) and A.6--A.7 (standard errors and
confidence intervals).
\item Case 2 (Latent covariate, observed response)
in Tables A.3--A.4 (point estimates) and A.8--A.9 (standard errors and
confidence intervals).
\item Case 3 (Latent covariate, latent response)
in Tables A.5 (point estimates) and A.10 (standard errors and
confidence intervals).
\begin{itemize}
\item
The tables for Case 3 (A.5 and A.10) are also shown in Section
4, as Tables 1 and 2 respectively.
\end{itemize}
\end{itemize}
\end{itemize}


\section{Results of the example (Section 5 of the paper)}
\label{s_a_example}

Further results of the applied example discussed in Section 5 of the
paper are given Tables B.1--B.4:

\begin{itemize}
\item
The full sets of estimated coefficients for the structural models for
Netherlands in Tables B.1--B.2.
\begin{itemize}
\item
Of these, the results for the coefficient of the dummy variable for Male
are also shown in Section
5, as part of Table 3.
\end{itemize}
\item
The list of countries in the cross-national analysis, their sample
sizes, and estimated coefficients of the dummy variable for Male in each
of them in Tables B.3--B.4.
\begin{itemize}
\item
Of these, the two-step estimates of the coefficients and their 95\%
confidence intervals are also displayed in Figure 1 in Section 5
(with the further
standardisation that there the coefficients are expressed on a scale where the residual
variance of both intrinsic and exterinsic values for the Netherlands is 1).
\end{itemize}
\end{itemize}


\newpage
\setcounter{figure}{0}
\renewcommand{\thefigure}{A.\arabic{figure}}

\begin{figure}
\caption{Path diagrams representing the three cases that are considered
in the simulation studies of Section 4 of the paper. Here the dashed
arrows represent the measurement model that is estimated in step 1 of
two-step estimation, and the solid arrow the structural model
(regression model for $\eta$ given $X$, $Z$ given $\eta$, or $\eta_{2}$
given $\eta_{1}$) that is estimated in its step 2.}
\begin{center}
\vspace*{3ex}
\includegraphics[width=.7\textwidth]{case1}

\vspace*{3ex}
\includegraphics[width=.7\textwidth]{case2}

\vspace*{3ex}
\includegraphics[width=.7\textwidth]{case3}
\end{center}
\label{f_a_cases}
\end{figure}

\clearpage
\newpage
\setcounter{table}{0}
\renewcommand{\thetable}{A.\arabic{table}}

\thispagestyle{empty}
\newgeometry{top=10mm}
\begin{table}[ht]
\caption{Simulation results for point estimates of structural regression
coefficient $\beta_{1}$ of a normally distributed
covariate $X$ for a latent response $\eta$
(simulation Case 1(a))
.}
\centering
{\small{
\begin{tabular}{|rrrrr|rrr|rrr|rrr|}
  \hline
  & & & &
  & \multicolumn{3}{|c|}{Bias}
  & \multicolumn{3}{|c|}{RMSE}
  & \multicolumn{3}{|c|}{MAE} \\
$p$ & $\pi_{Y}$ & $R^{2}_{Y}$ & $R^{2}_{\eta}$ & $\beta_1$ & 2-step & 1-step & 3-step & 2-step & 1-step & 3-step & 2-step & 1-step & 3-step \\
  \hline \multicolumn{14}{|l|}{$n=200$}\\ \hline
4 & 0.5 & 0.4 & 0.0 & 0.000 & -0.003 & -0.004 & -0.002 & 0.146 & 0.151 & 0.090 & 0.084 & 0.086 & 0.052 \\
  4 & 0.8 & 0.4 & 0.0 & 0.000 & 0.007 & 0.006 & 0.003 & 0.196 & 0.208 & 0.094 & 0.100 & 0.102 & 0.045 \\
  4 & 0.5 & 0.6 & 0.0 & 0.000 & 0.008 & 0.008 & 0.006 & 0.207 & 0.208 & 0.145 & 0.125 & 0.126 & 0.086 \\
  4 & 0.8 & 0.6 & 0.0 & 0.000 & 0.002 & 0.002 & 0.002 & 0.251 & 0.253 & 0.151 & 0.144 & 0.145 & 0.085 \\
  8 & 0.5 & 0.4 & 0.0 & 0.000 & -0.005 & -0.005 & -0.004 & 0.132 & 0.132 & 0.098 & 0.085 & 0.085 & 0.063 \\
  8 & 0.8 & 0.4 & 0.0 & 0.000 & 0.003 & 0.003 & 0.002 & 0.150 & 0.152 & 0.088 & 0.096 & 0.097 & 0.056 \\
  8 & 0.5 & 0.6 & 0.0 & 0.000 & 0.004 & 0.004 & 0.003 & 0.187 & 0.185 & 0.150 & 0.118 & 0.118 & 0.095 \\
  8 & 0.8 & 0.6 & 0.0 & 0.000 & -0.013 & -0.013 & -0.009 & 0.203 & 0.204 & 0.141 & 0.133 & 0.135 & 0.093 \\
  \hline
  4 & 0.5 & 0.4 & 0.2 & 0.662 & 0.030 & 0.039 & -0.239 & 0.251 & 0.229 & 0.292 & 0.137 & 0.132 & 0.267 \\
  4 & 0.8 & 0.4 & 0.2 & 0.662 & 0.065 & 0.054 & -0.326 & 0.527 & 0.367 & 0.423 & 0.179 & 0.166 & 0.382 \\
  4 & 0.5 & 0.6 & 0.2 & 0.993 & 0.045 & 0.048 & -0.270 & 0.320 & 0.300 & 0.355 & 0.182 & 0.176 & 0.319 \\
  4 & 0.8 & 0.6 & 0.2 & 0.993 & 0.085 & 0.071 & -0.359 & 0.520 & 0.394 & 0.475 & 0.221 & 0.210 & 0.418 \\
  8 & 0.5 & 0.4 & 0.2 & 0.662 & 0.023 & 0.026 & -0.153 & 0.183 & 0.180 & 0.208 & 0.119 & 0.115 & 0.175 \\
  8 & 0.8 & 0.4 & 0.2 & 0.662 & 0.031 & 0.040 & -0.254 & 0.255 & 0.266 & 0.298 & 0.151 & 0.142 & 0.277 \\
  8 & 0.5 & 0.6 & 0.2 & 0.993 & 0.042 & 0.042 & -0.163 & 0.256 & 0.253 & 0.262 & 0.161 & 0.162 & 0.205 \\
  8 & 0.8 & 0.6 & 0.2 & 0.993 & 0.052 & 0.056 & -0.264 & 0.315 & 0.312 & 0.345 & 0.188 & 0.190 & 0.302 \\
  \hline
  4 & 0.5 & 0.4 & 0.4 & 0.937 & 0.031 & 0.034 & -0.348 & 0.305 & 0.251 & 0.401 & 0.185 & 0.148 & 0.382 \\
  4 & 0.8 & 0.4 & 0.4 & 0.937 & 0.143 & 0.074 & -0.443 & 0.924 & 0.365 & 0.632 & 0.254 & 0.213 & 0.533 \\
  4 & 0.5 & 0.6 & 0.4 & 1.405 & 0.083 & 0.071 & -0.367 & 0.404 & 0.364 & 0.466 & 0.233 & 0.207 & 0.412 \\
  4 & 0.8 & 0.6 & 0.4 & 1.405 & 0.093 & 0.075 & -0.527 & 0.642 & 0.449 & 0.663 & 0.269 & 0.252 & 0.611 \\
  8 & 0.5 & 0.4 & 0.4 & 0.937 & 0.035 & 0.042 & -0.216 & 0.234 & 0.233 & 0.280 & 0.136 & 0.131 & 0.245 \\
  8 & 0.8 & 0.4 & 0.4 & 0.937 & 0.040 & 0.051 & -0.362 & 0.309 & 0.297 & 0.408 & 0.183 & 0.167 & 0.389 \\
  8 & 0.5 & 0.6 & 0.4 & 1.405 & 0.051 & 0.051 & -0.239 & 0.317 & 0.316 & 0.351 & 0.198 & 0.197 & 0.286 \\
  8 & 0.8 & 0.6 & 0.4 & 1.405 & 0.051 & 0.054 & -0.391 & 0.365 & 0.356 & 0.470 & 0.214 & 0.207 & 0.434 \\
  \hline \multicolumn{14}{|l|}{$n=1000$}\\ \hline
  4 & 0.5 & 0.4 & 0.0 & 0.000 & 0.000 & 0.000 & 0.000 & 0.062 & 0.062 & 0.037 & 0.041 & 0.041 & 0.025 \\
  4 & 0.8 & 0.4 & 0.0 & 0.000 & 0.005 & 0.005 & 0.002 & 0.073 & 0.074 & 0.033 & 0.046 & 0.046 & 0.020 \\
  4 & 0.5 & 0.6 & 0.0 & 0.000 & 0.002 & 0.002 & 0.001 & 0.086 & 0.086 & 0.060 & 0.058 & 0.058 & 0.040 \\
  4 & 0.8 & 0.6 & 0.0 & 0.000 & 0.007 & 0.007 & 0.004 & 0.095 & 0.095 & 0.055 & 0.065 & 0.065 & 0.038 \\
  8 & 0.5 & 0.4 & 0.0 & 0.000 & 0.000 & 0.000 & 0.000 & 0.055 & 0.055 & 0.041 & 0.036 & 0.036 & 0.026 \\
  8 & 0.8 & 0.4 & 0.0 & 0.000 & -0.003 & -0.003 & -0.002 & 0.065 & 0.065 & 0.038 & 0.044 & 0.044 & 0.026 \\
  8 & 0.5 & 0.6 & 0.0 & 0.000 & 0.002 & 0.002 & 0.002 & 0.080 & 0.079 & 0.064 & 0.052 & 0.051 & 0.041 \\
  8 & 0.8 & 0.6 & 0.0 & 0.000 & -0.001 & -0.001 & -0.000 & 0.085 & 0.085 & 0.059 & 0.057 & 0.057 & 0.040 \\
  \hline
  4 & 0.5 & 0.4 & 0.2 & 0.662 & -0.004 & -0.001 & -0.268 & 0.098 & 0.092 & 0.276 & 0.063 & 0.059 & 0.273 \\
  4 & 0.8 & 0.4 & 0.2 & 0.662 & 0.009 & 0.014 & -0.365 & 0.117 & 0.108 & 0.369 & 0.078 & 0.072 & 0.369 \\
  4 & 0.5 & 0.6 & 0.2 & 0.993 & 0.015 & 0.016 & -0.292 & 0.127 & 0.123 & 0.307 & 0.084 & 0.079 & 0.300 \\
  4 & 0.8 & 0.6 & 0.2 & 0.993 & 0.006 & 0.007 & -0.409 & 0.143 & 0.137 & 0.419 & 0.095 & 0.093 & 0.413 \\
  8 & 0.5 & 0.4 & 0.2 & 0.662 & 0.003 & 0.004 & -0.171 & 0.077 & 0.076 & 0.181 & 0.054 & 0.052 & 0.174 \\
  8 & 0.8 & 0.4 & 0.2 & 0.662 & 0.003 & 0.003 & -0.273 & 0.095 & 0.093 & 0.280 & 0.066 & 0.063 & 0.277 \\
  8 & 0.5 & 0.6 & 0.2 & 0.993 & -0.001 & -0.002 & -0.199 & 0.108 & 0.108 & 0.218 & 0.072 & 0.073 & 0.201 \\
  8 & 0.8 & 0.6 & 0.2 & 0.993 & 0.001 & 0.001 & -0.299 & 0.116 & 0.115 & 0.310 & 0.078 & 0.077 & 0.304 \\
  \hline
  4 & 0.5 & 0.4 & 0.4 & 0.937 & 0.008 & 0.011 & -0.371 & 0.123 & 0.107 & 0.380 & 0.080 & 0.070 & 0.378 \\
  4 & 0.8 & 0.4 & 0.4 & 0.937 & 0.012 & 0.014 & -0.517 & 0.158 & 0.133 & 0.523 & 0.096 & 0.086 & 0.524 \\
  4 & 0.5 & 0.6 & 0.4 & 1.405 & 0.013 & 0.014 & -0.420 & 0.157 & 0.151 & 0.436 & 0.099 & 0.097 & 0.426 \\
  4 & 0.8 & 0.6 & 0.4 & 1.405 & 0.011 & 0.011 & -0.579 & 0.184 & 0.170 & 0.591 & 0.120 & 0.116 & 0.586 \\
  8 & 0.5 & 0.4 & 0.4 & 0.937 & 0.006 & 0.007 & -0.241 & 0.097 & 0.093 & 0.252 & 0.062 & 0.060 & 0.243 \\
  8 & 0.8 & 0.4 & 0.4 & 0.937 & 0.009 & 0.010 & -0.385 & 0.121 & 0.114 & 0.392 & 0.082 & 0.075 & 0.388 \\
  8 & 0.5 & 0.6 & 0.4 & 1.405 & 0.007 & 0.007 & -0.275 & 0.125 & 0.123 & 0.293 & 0.079 & 0.079 & 0.278 \\
  8 & 0.8 & 0.6 & 0.4 & 1.405 & 0.020 & 0.018 & -0.410 & 0.153 & 0.150 & 0.425 & 0.100 & 0.096 & 0.412 \\
   \hline
\multicolumn{14}{l}{\emph{Note:} $p$ denotes number of measurement
items $Y_{j}$, $\pi_{Y}$ marginal proportion of $Y_{j}=1$, }\\
\multicolumn{14}{l}{
\hspace*{2em}$R_{Y}^{2}$ and $R^{2}_{\eta}$ the $R^{2}$ statistics in models for
$Y_{j}$ and $\eta$, and $\beta_{1}$ true value of $\beta_{1}$.}
\end{tabular}
}}
\label{sres_1_48}
\end{table}

\clearpage

\thispagestyle{empty}
\begin{table}[ht]
\caption{Simulation results for point estimates of structural regression
coefficient $\beta_{1}$ of a binary
covariate $X$ for a latent response $\eta$
(Case 1(b)).}
\centering
{\small{
\begin{tabular}{|rrrrr|rrr|rrr|rrr|}
  \hline
  & & & &
  & \multicolumn{3}{|c|}{Bias}
  & \multicolumn{3}{|c|}{RMSE}
  & \multicolumn{3}{|c|}{MAE} \\
$p$ & $\pi_{Y}$ & $R^{2}_{Y}$ & $R^{2}_{\eta}$ & $\beta_1$ & 2-step & 1-step & 3-step & 2-step & 1-step & 3-step & 2-step & 1-step & 3-step \\
  \hline \multicolumn{14}{|l|}{$n=200$}\\ \hline
4 & 0.5 & 0.4 & 0.0 & 0.000 & 0.001 & 0.001 & 0.000 & 0.144 & 0.147 & 0.090 & 0.090 & 0.092 & 0.055 \\
  4 & 0.8 & 0.4 & 0.0 & 0.000 & 0.007 & 0.008 & 0.003 & 0.191 & 0.213 & 0.091 & 0.099 & 0.106 & 0.046 \\
  4 & 0.5 & 0.6 & 0.0 & 0.000 & -0.001 & -0.002 & -0.001 & 0.209 & 0.210 & 0.147 & 0.129 & 0.129 & 0.091 \\
  4 & 0.8 & 0.6 & 0.0 & 0.000 & 0.002 & 0.001 & 0.001 & 0.259 & 0.254 & 0.160 & 0.138 & 0.141 & 0.081 \\
  8 & 0.5 & 0.4 & 0.0 & 0.000 & -0.002 & -0.002 & -0.002 & 0.125 & 0.126 & 0.093 & 0.086 & 0.086 & 0.063 \\
  8 & 0.8 & 0.4 & 0.0 & 0.000 & 0.004 & 0.005 & 0.003 & 0.160 & 0.162 & 0.095 & 0.091 & 0.092 & 0.053 \\
  8 & 0.5 & 0.6 & 0.0 & 0.000 & -0.005 & -0.005 & -0.004 & 0.185 & 0.184 & 0.149 & 0.126 & 0.126 & 0.100 \\
  8 & 0.8 & 0.6 & 0.0 & 0.000 & 0.005 & 0.005 & 0.003 & 0.204 & 0.206 & 0.143 & 0.135 & 0.135 & 0.094 \\
  \hline
  4 & 0.5 & 0.4 & 0.2 & 0.662 & 0.016 & 0.026 & -0.231 & 0.236 & 0.221 & 0.285 & 0.138 & 0.135 & 0.261 \\
  4 & 0.8 & 0.4 & 0.2 & 0.662 & 0.077 & 0.050 & -0.318 & 0.567 & 0.316 & 0.439 & 0.194 & 0.171 & 0.383 \\
  4 & 0.5 & 0.6 & 0.2 & 0.993 & 0.035 & 0.037 & -0.236 & 0.299 & 0.297 & 0.332 & 0.191 & 0.182 & 0.290 \\
  4 & 0.8 & 0.6 & 0.2 & 0.993 & 0.082 & 0.083 & -0.345 & 0.465 & 0.428 & 0.444 & 0.218 & 0.211 & 0.393 \\
  8 & 0.5 & 0.4 & 0.2 & 0.662 & 0.021 & 0.023 & -0.137 & 0.192 & 0.189 & 0.208 & 0.126 & 0.124 & 0.169 \\
  8 & 0.8 & 0.4 & 0.2 & 0.662 & 0.018 & 0.026 & -0.257 & 0.239 & 0.232 & 0.298 & 0.147 & 0.140 & 0.277 \\
  8 & 0.5 & 0.6 & 0.2 & 0.993 & 0.044 & 0.044 & -0.123 & 0.256 & 0.255 & 0.254 & 0.157 & 0.158 & 0.191 \\
  8 & 0.8 & 0.6 & 0.2 & 0.993 & 0.046 & 0.048 & -0.248 & 0.309 & 0.303 & 0.336 & 0.180 & 0.178 & 0.290 \\
  \hline
  4 & 0.5 & 0.4 & 0.4 & 0.937 & 0.042 & 0.039 & -0.280 & 0.308 & 0.236 & 0.361 & 0.171 & 0.156 & 0.323 \\
  4 & 0.8 & 0.4 & 0.4 & 0.937 & 0.062 & 0.065 & -0.479 & 0.587 & 0.388 & 0.563 & 0.234 & 0.182 & 0.525 \\
  4 & 0.5 & 0.6 & 0.4 & 1.405 & 0.050 & 0.045 & -0.254 & 0.398 & 0.351 & 0.425 & 0.223 & 0.198 & 0.337 \\
  4 & 0.8 & 0.6 & 0.4 & 1.405 & 0.079 & 0.072 & -0.495 & 0.508 & 0.505 & 0.594 & 0.253 & 0.241 & 0.558 \\
  8 & 0.5 & 0.4 & 0.4 & 0.937 & 0.015 & 0.018 & -0.175 & 0.217 & 0.208 & 0.255 & 0.137 & 0.131 & 0.204 \\
  8 & 0.8 & 0.4 & 0.4 & 0.937 & 0.025 & 0.037 & -0.361 & 0.290 & 0.279 & 0.406 & 0.181 & 0.164 & 0.381 \\
  8 & 0.5 & 0.6 & 0.4 & 1.405 & 0.047 & 0.044 & -0.114 & 0.321 & 0.308 & 0.320 & 0.185 & 0.175 & 0.229 \\
  8 & 0.8 & 0.6 & 0.4 & 1.405 & 0.070 & 0.071 & -0.318 & 0.359 & 0.352 & 0.419 & 0.216 & 0.213 & 0.365 \\
  \hline \multicolumn{14}{|l|}{$n=1000$}\\ \hline
  4 & 0.5 & 0.4 & 0.0 & 0.000 & -0.000 & -0.000 & 0.000 & 0.061 & 0.062 & 0.037 & 0.040 & 0.040 & 0.024 \\
  4 & 0.8 & 0.4 & 0.0 & 0.000 & 0.000 & 0.000 & 0.000 & 0.073 & 0.074 & 0.032 & 0.046 & 0.046 & 0.020 \\
  4 & 0.5 & 0.6 & 0.0 & 0.000 & -0.002 & -0.002 & -0.001 & 0.082 & 0.083 & 0.058 & 0.055 & 0.056 & 0.039 \\
  4 & 0.8 & 0.6 & 0.0 & 0.000 & 0.003 & 0.003 & 0.002 & 0.094 & 0.094 & 0.055 & 0.060 & 0.060 & 0.035 \\
  8 & 0.5 & 0.4 & 0.0 & 0.000 & 0.001 & 0.001 & 0.001 & 0.056 & 0.056 & 0.042 & 0.037 & 0.037 & 0.027 \\
  8 & 0.8 & 0.4 & 0.0 & 0.000 & 0.002 & 0.002 & 0.001 & 0.064 & 0.064 & 0.037 & 0.044 & 0.043 & 0.025 \\
  8 & 0.5 & 0.6 & 0.0 & 0.000 & -0.002 & -0.002 & -0.001 & 0.081 & 0.081 & 0.065 & 0.056 & 0.055 & 0.045 \\
  8 & 0.8 & 0.6 & 0.0 & 0.000 & 0.005 & 0.005 & 0.004 & 0.087 & 0.087 & 0.061 & 0.055 & 0.055 & 0.038 \\
  \hline
  4 & 0.5 & 0.4 & 0.2 & 0.662 & 0.004 & 0.005 & -0.249 & 0.093 & 0.086 & 0.257 & 0.058 & 0.055 & 0.253 \\
  4 & 0.8 & 0.4 & 0.2 & 0.662 & 0.001 & 0.009 & -0.366 & 0.128 & 0.114 & 0.371 & 0.081 & 0.072 & 0.370 \\
  4 & 0.5 & 0.6 & 0.2 & 0.993 & 0.008 & 0.011 & -0.261 & 0.118 & 0.118 & 0.278 & 0.077 & 0.078 & 0.263 \\
  4 & 0.8 & 0.6 & 0.2 & 0.993 & -0.002 & -0.001 & -0.401 & 0.145 & 0.140 & 0.412 & 0.093 & 0.093 & 0.408 \\
  8 & 0.5 & 0.4 & 0.2 & 0.662 & 0.007 & 0.007 & -0.152 & 0.079 & 0.078 & 0.165 & 0.053 & 0.052 & 0.156 \\
  8 & 0.8 & 0.4 & 0.2 & 0.662 & 0.008 & 0.009 & -0.266 & 0.096 & 0.094 & 0.273 & 0.061 & 0.059 & 0.268 \\
  8 & 0.5 & 0.6 & 0.2 & 0.993 & 0.003 & 0.002 & -0.161 & 0.109 & 0.109 & 0.187 & 0.069 & 0.070 & 0.164 \\
  8 & 0.8 & 0.6 & 0.2 & 0.993 & 0.008 & 0.009 & -0.276 & 0.118 & 0.118 & 0.290 & 0.082 & 0.079 & 0.279 \\
  \hline
  4 & 0.5 & 0.4 & 0.4 & 0.937 & 0.001 & 0.005 & -0.322 & 0.119 & 0.103 & 0.334 & 0.077 & 0.069 & 0.330 \\
  4 & 0.8 & 0.4 & 0.4 & 0.937 & 0.005 & 0.011 & -0.520 & 0.165 & 0.131 & 0.526 & 0.105 & 0.085 & 0.525 \\
  4 & 0.5 & 0.6 & 0.4 & 1.405 & 0.007 & 0.012 & -0.296 & 0.154 & 0.141 & 0.325 & 0.105 & 0.097 & 0.308 \\
  4 & 0.8 & 0.6 & 0.4 & 1.405 & 0.023 & 0.020 & -0.531 & 0.182 & 0.166 & 0.545 & 0.122 & 0.102 & 0.541 \\
  8 & 0.5 & 0.4 & 0.4 & 0.937 & 0.001 & 0.003 & -0.191 & 0.095 & 0.089 & 0.208 & 0.062 & 0.057 & 0.194 \\
  8 & 0.8 & 0.4 & 0.4 & 0.937 & 0.006 & 0.010 & -0.376 & 0.114 & 0.111 & 0.383 & 0.072 & 0.072 & 0.377 \\
  8 & 0.5 & 0.6 & 0.4 & 1.405 & 0.014 & 0.014 & -0.148 & 0.129 & 0.124 & 0.191 & 0.085 & 0.083 & 0.161 \\
  8 & 0.8 & 0.6 & 0.4 & 1.405 & 0.007 & 0.005 & -0.364 & 0.148 & 0.143 & 0.382 & 0.097 & 0.094 & 0.372 \\
   \hline
\multicolumn{14}{l}{\emph{Note:} $p$ denotes number of measurement
items $Y_{j}$, $\pi_{Y}$ marginal proportion of $Y_{j}=1$, }\\
\multicolumn{14}{l}{
\hspace*{2em}$R_{Y}^{2}$ and $R^{2}_{\eta}$ the $R^{2}$ statistics in models for
$Y_{j}$ and $\eta$, and $\beta_{1}$ true value of $\beta_{1}$.}
\end{tabular}
}}
\label{sres_201_132}
\end{table}

\clearpage
\thispagestyle{empty}
\begin{table}[ht]
\caption{Simulation results for point estimates of structural regression
coefficient $\beta_{1}$ of
a latent covariate $\eta$ for a conditionally normally distributed
response $Z$
(Case 2(a)).}
\centering
{\small{
\begin{tabular}{|rrrrr|rrr|rrr|rrr|}
  \hline
  & & & &
  & \multicolumn{3}{|c|}{Bias}
  & \multicolumn{3}{|c|}{RMSE}
  & \multicolumn{3}{|c|}{MAE} \\
$p$ & $\pi_{Y}$ & $R^{2}_{Y}$ & $R^{2}_{Z}$ & $\beta_1$ & 2-step & 1-step & 3-step & 2-step & 1-step & 3-step & 2-step & 1-step & 3-step \\
  \hline \multicolumn{14}{|l|}{$n=200$}\\ \hline
4 & 0.5 & 0.4 & 0.0 & 0.000 & -0.003 & -0.003 & -0.003 & 0.067 & 0.070 & 0.067 & 0.041 & 0.042 & 0.040 \\
  4 & 0.8 & 0.4 & 0.0 & 0.000 & -0.003 & -0.003 & -0.003 & 0.089 & 0.104 & 0.088 & 0.050 & 0.057 & 0.050 \\
  4 & 0.5 & 0.6 & 0.0 & 0.000 & -0.000 & -0.000 & -0.000 & 0.039 & 0.040 & 0.039 & 0.024 & 0.025 & 0.024 \\
  4 & 0.8 & 0.6 & 0.0 & 0.000 & -0.000 & -0.000 & -0.000 & 0.044 & 0.046 & 0.044 & 0.027 & 0.029 & 0.027 \\
  8 & 0.5 & 0.4 & 0.0 & 0.000 & 0.001 & 0.001 & 0.001 & 0.059 & 0.060 & 0.059 & 0.038 & 0.038 & 0.038 \\
  8 & 0.8 & 0.4 & 0.0 & 0.000 & -0.002 & -0.002 & -0.002 & 0.068 & 0.071 & 0.068 & 0.042 & 0.043 & 0.042 \\
  8 & 0.5 & 0.6 & 0.0 & 0.000 & 0.002 & 0.002 & 0.002 & 0.038 & 0.038 & 0.037 & 0.025 & 0.025 & 0.025 \\
  8 & 0.8 & 0.6 & 0.0 & 0.000 & 0.001 & 0.001 & 0.001 & 0.040 & 0.040 & 0.040 & 0.025 & 0.025 & 0.025 \\
  \hline
  4 & 0.5 & 0.4 & 0.2 & 0.302 & 0.003 & 0.012 & 0.003 & 0.111 & 0.113 & 0.111 & 0.072 & 0.071 & 0.071 \\
  4 & 0.8 & 0.4 & 0.2 & 0.302 & 0.008 & 0.027 & 0.008 & 0.159 & 0.159 & 0.156 & 0.088 & 0.087 & 0.090 \\
  4 & 0.5 & 0.6 & 0.2 & 0.201 & -0.000 & 0.003 & 0.000 & 0.060 & 0.061 & 0.061 & 0.041 & 0.040 & 0.041 \\
  4 & 0.8 & 0.6 & 0.2 & 0.201 & -0.005 & -0.000 & -0.004 & 0.065 & 0.063 & 0.066 & 0.044 & 0.042 & 0.044 \\
  8 & 0.5 & 0.4 & 0.2 & 0.302 & -0.004 & 0.001 & -0.004 & 0.083 & 0.085 & 0.083 & 0.056 & 0.057 & 0.056 \\
  8 & 0.8 & 0.4 & 0.2 & 0.302 & 0.005 & 0.018 & 0.006 & 0.114 & 0.119 & 0.114 & 0.070 & 0.068 & 0.069 \\
  8 & 0.5 & 0.6 & 0.2 & 0.201 & 0.000 & 0.001 & 0.000 & 0.051 & 0.051 & 0.051 & 0.035 & 0.034 & 0.035 \\
  8 & 0.8 & 0.6 & 0.2 & 0.201 & -0.002 & 0.001 & -0.002 & 0.055 & 0.056 & 0.056 & 0.037 & 0.038 & 0.037 \\
  \hline
  4 & 0.5 & 0.4 & 0.4 & 0.427 & 0.002 & 0.012 & 0.003 & 0.134 & 0.129 & 0.134 & 0.083 & 0.082 & 0.084 \\
  4 & 0.8 & 0.4 & 0.4 & 0.427 & -0.001 & 0.028 & 0.006 & 0.181 & 0.191 & 0.188 & 0.101 & 0.091 & 0.104 \\
  4 & 0.5 & 0.6 & 0.4 & 0.285 & 0.001 & 0.002 & 0.002 & 0.073 & 0.070 & 0.074 & 0.049 & 0.045 & 0.050 \\
  4 & 0.8 & 0.6 & 0.4 & 0.285 & -0.005 & 0.003 & -0.003 & 0.086 & 0.082 & 0.087 & 0.057 & 0.056 & 0.058 \\
  8 & 0.5 & 0.4 & 0.4 & 0.427 & 0.005 & 0.013 & 0.005 & 0.107 & 0.107 & 0.107 & 0.067 & 0.070 & 0.070 \\
  8 & 0.8 & 0.4 & 0.4 & 0.427 & 0.008 & 0.024 & 0.010 & 0.137 & 0.146 & 0.139 & 0.083 & 0.083 & 0.084 \\
  8 & 0.5 & 0.6 & 0.4 & 0.285 & 0.002 & 0.003 & 0.002 & 0.061 & 0.061 & 0.061 & 0.039 & 0.040 & 0.039 \\
  8 & 0.8 & 0.6 & 0.4 & 0.285 & -0.003 & 0.002 & -0.003 & 0.067 & 0.068 & 0.067 & 0.044 & 0.045 & 0.046 \\
  \hline \multicolumn{14}{|l|}{$n=1000$}\\ \hline
  4 & 0.5 & 0.4 & 0.0 & 0.000 & -0.000 & -0.000 & -0.000 & 0.028 & 0.029 & 0.028 & 0.019 & 0.019 & 0.019 \\
  4 & 0.8 & 0.4 & 0.0 & 0.000 & -0.001 & -0.001 & -0.001 & 0.034 & 0.034 & 0.034 & 0.022 & 0.022 & 0.022 \\
  4 & 0.5 & 0.6 & 0.0 & 0.000 & 0.000 & 0.000 & 0.000 & 0.017 & 0.018 & 0.017 & 0.012 & 0.012 & 0.012 \\
  4 & 0.8 & 0.6 & 0.0 & 0.000 & -0.000 & -0.000 & -0.000 & 0.020 & 0.020 & 0.020 & 0.012 & 0.012 & 0.012 \\
  8 & 0.5 & 0.4 & 0.0 & 0.000 & -0.001 & -0.001 & -0.001 & 0.026 & 0.026 & 0.025 & 0.016 & 0.016 & 0.016 \\
  8 & 0.8 & 0.4 & 0.0 & 0.000 & 0.002 & 0.002 & 0.002 & 0.030 & 0.030 & 0.030 & 0.019 & 0.020 & 0.020 \\
  8 & 0.5 & 0.6 & 0.0 & 0.000 & 0.001 & 0.001 & 0.001 & 0.016 & 0.016 & 0.016 & 0.011 & 0.011 & 0.011 \\
  8 & 0.8 & 0.6 & 0.0 & 0.000 & 0.000 & 0.000 & 0.000 & 0.017 & 0.017 & 0.017 & 0.011 & 0.011 & 0.011 \\
  \hline
  4 & 0.5 & 0.4 & 0.2 & 0.302 & 0.003 & 0.003 & 0.002 & 0.046 & 0.044 & 0.046 & 0.030 & 0.029 & 0.030 \\
  4 & 0.8 & 0.4 & 0.2 & 0.302 & 0.006 & 0.008 & 0.007 & 0.060 & 0.057 & 0.061 & 0.041 & 0.039 & 0.040 \\
  4 & 0.5 & 0.6 & 0.2 & 0.201 & -0.001 & 0.000 & -0.001 & 0.026 & 0.025 & 0.026 & 0.017 & 0.017 & 0.017 \\
  4 & 0.8 & 0.6 & 0.2 & 0.201 & -0.002 & -0.001 & -0.002 & 0.031 & 0.030 & 0.031 & 0.022 & 0.021 & 0.021 \\
  8 & 0.5 & 0.4 & 0.2 & 0.302 & -0.000 & 0.001 & -0.001 & 0.035 & 0.035 & 0.035 & 0.023 & 0.023 & 0.023 \\
  8 & 0.8 & 0.4 & 0.2 & 0.302 & -0.000 & 0.003 & -0.000 & 0.045 & 0.046 & 0.046 & 0.031 & 0.031 & 0.031 \\
  8 & 0.5 & 0.6 & 0.2 & 0.201 & -0.002 & -0.001 & -0.002 & 0.022 & 0.022 & 0.022 & 0.015 & 0.015 & 0.015 \\
  8 & 0.8 & 0.6 & 0.2 & 0.201 & 0.000 & 0.001 & -0.000 & 0.025 & 0.025 & 0.025 & 0.016 & 0.017 & 0.017 \\
  \hline
  4 & 0.5 & 0.4 & 0.4 & 0.427 & 0.002 & 0.003 & 0.002 & 0.058 & 0.052 & 0.058 & 0.039 & 0.034 & 0.039 \\
  4 & 0.8 & 0.4 & 0.4 & 0.427 & -0.001 & 0.002 & -0.002 & 0.074 & 0.065 & 0.074 & 0.049 & 0.043 & 0.050 \\
  4 & 0.5 & 0.6 & 0.4 & 0.285 & 0.001 & 0.002 & 0.001 & 0.031 & 0.030 & 0.031 & 0.021 & 0.020 & 0.021 \\
  4 & 0.8 & 0.6 & 0.4 & 0.285 & 0.000 & 0.002 & 0.000 & 0.037 & 0.034 & 0.037 & 0.025 & 0.022 & 0.026 \\
  8 & 0.5 & 0.4 & 0.4 & 0.427 & -0.001 & 0.001 & -0.001 & 0.044 & 0.043 & 0.044 & 0.029 & 0.028 & 0.029 \\
  8 & 0.8 & 0.4 & 0.4 & 0.427 & -0.001 & 0.003 & -0.001 & 0.053 & 0.052 & 0.053 & 0.035 & 0.034 & 0.035 \\
  8 & 0.5 & 0.6 & 0.4 & 0.285 & -0.000 & -0.000 & -0.000 & 0.026 & 0.026 & 0.026 & 0.017 & 0.017 & 0.018 \\
  8 & 0.8 & 0.6 & 0.4 & 0.285 & 0.000 & 0.001 & 0.000 & 0.028 & 0.029 & 0.029 & 0.019 & 0.019 & 0.020 \\
   \hline
\multicolumn{14}{l}{\emph{Note:} $p$ denotes number of measurement
items $Y_{j}$, $\pi_{Y}$ marginal proportion of $Y_{j}=1$, }\\
\multicolumn{14}{l}{
\hspace*{2em}$R_{Y}^{2}$ and $R^{2}_{Z}$ the $R^{2}$ statistics in models for
$Y_{j}$ and $Z$, and $\beta_{1}$ true value of $\beta_{1}$.}
\end{tabular}
}}
\label{sres_49_96}
\end{table}

\clearpage
\thispagestyle{empty}
\begin{table}[ht]
\caption{Simulation results for point estimates of structural regression
coefficient $\beta_{1}$ of
a latent covariate $\eta$ for a conditionally skew-normally distributed
response $Z$ (Case~2(b)).}
\centering
{\small{
\begin{tabular}{|rrrrr|rrr|rrr|rrr|}
  \hline
  & & & &
  & \multicolumn{3}{|c|}{Bias}
  & \multicolumn{3}{|c|}{RMSE}
  & \multicolumn{3}{|c|}{MAE} \\
$p$ & $\pi_{Y}$ & $R^{2}_{Y}$ & $R^{2}_{Z}$ & $\beta_1$ & 2-step & 1-step & 3-step & 2-step & 1-step & 3-step & 2-step & 1-step & 3-step \\
  \hline \multicolumn{14}{|l|}{$n=200$}\\ \hline
4 & 0.5 & 0.4 & 0.0 & 0.000 & 0.003 & 0.003 & 0.003 & 0.068 & 0.070 & 0.067 & 0.040 & 0.042 & 0.040 \\
  4 & 0.8 & 0.4 & 0.0 & 0.000 & 0.005 & 0.008 & 0.003 & 0.085 & 0.113 & 0.084 & 0.043 & 0.048 & 0.043 \\
  4 & 0.5 & 0.6 & 0.0 & 0.000 & 0.000 & 0.000 & 0.000 & 0.039 & 0.040 & 0.039 & 0.025 & 0.026 & 0.025 \\
  4 & 0.8 & 0.6 & 0.0 & 0.000 & 0.001 & 0.001 & -0.000 & 0.044 & 0.046 & 0.044 & 0.026 & 0.027 & 0.026 \\
  8 & 0.5 & 0.4 & 0.0 & 0.000 & 0.002 & 0.002 & 0.002 & 0.056 & 0.057 & 0.056 & 0.037 & 0.038 & 0.037 \\
  8 & 0.8 & 0.4 & 0.0 & 0.000 & 0.001 & 0.001 & -0.000 & 0.064 & 0.067 & 0.064 & 0.041 & 0.043 & 0.041 \\
  8 & 0.5 & 0.6 & 0.0 & 0.000 & -0.000 & -0.000 & -0.000 & 0.036 & 0.036 & 0.036 & 0.024 & 0.025 & 0.024 \\
  8 & 0.8 & 0.6 & 0.0 & 0.000 & 0.004 & 0.004 & 0.003 & 0.040 & 0.040 & 0.040 & 0.026 & 0.026 & 0.026 \\
  \hline
  4 & 0.5 & 0.4 & 0.2 & 0.302 & 0.007 & 0.016 & 0.007 & 0.114 & 0.113 & 0.113 & 0.071 & 0.067 & 0.070 \\
  4 & 0.8 & 0.4 & 0.2 & 0.302 & 0.022 & 0.043 & 0.008 & 0.173 & 0.193 & 0.160 & 0.093 & 0.090 & 0.089 \\
  4 & 0.5 & 0.6 & 0.2 & 0.201 & -0.002 & 0.001 & -0.002 & 0.059 & 0.057 & 0.059 & 0.038 & 0.036 & 0.038 \\
  4 & 0.8 & 0.6 & 0.2 & 0.201 & 0.008 & 0.014 & 0.001 & 0.076 & 0.076 & 0.071 & 0.048 & 0.047 & 0.048 \\
  8 & 0.5 & 0.4 & 0.2 & 0.302 & 0.004 & 0.009 & 0.004 & 0.086 & 0.088 & 0.085 & 0.057 & 0.058 & 0.057 \\
  8 & 0.8 & 0.4 & 0.2 & 0.302 & 0.019 & 0.035 & 0.010 & 0.119 & 0.128 & 0.111 & 0.069 & 0.069 & 0.066 \\
  8 & 0.5 & 0.6 & 0.2 & 0.201 & -0.001 & 0.001 & -0.001 & 0.050 & 0.050 & 0.050 & 0.034 & 0.034 & 0.034 \\
  8 & 0.8 & 0.6 & 0.2 & 0.201 & 0.005 & 0.010 & 0.000 & 0.059 & 0.061 & 0.055 & 0.037 & 0.037 & 0.035 \\
  \hline
  4 & 0.5 & 0.4 & 0.4 & 0.427 & -0.000 & 0.014 & -0.000 & 0.142 & 0.133 & 0.142 & 0.088 & 0.082 & 0.087 \\
  4 & 0.8 & 0.4 & 0.4 & 0.427 & 0.020 & 0.052 & 0.009 & 0.193 & 0.206 & 0.194 & 0.107 & 0.097 & 0.109 \\
  4 & 0.5 & 0.6 & 0.4 & 0.285 & -0.003 & 0.002 & -0.003 & 0.072 & 0.068 & 0.072 & 0.050 & 0.046 & 0.049 \\
  4 & 0.8 & 0.6 & 0.4 & 0.285 & 0.006 & 0.015 & -0.001 & 0.089 & 0.088 & 0.087 & 0.056 & 0.053 & 0.055 \\
  8 & 0.5 & 0.4 & 0.4 & 0.427 & 0.003 & 0.012 & 0.002 & 0.112 & 0.112 & 0.111 & 0.071 & 0.074 & 0.071 \\
  8 & 0.8 & 0.4 & 0.4 & 0.427 & 0.020 & 0.034 & 0.010 & 0.145 & 0.150 & 0.140 & 0.087 & 0.083 & 0.085 \\
  8 & 0.5 & 0.6 & 0.4 & 0.285 & 0.001 & 0.003 & 0.001 & 0.062 & 0.063 & 0.062 & 0.040 & 0.040 & 0.040 \\
  8 & 0.8 & 0.6 & 0.4 & 0.285 & 0.003 & 0.009 & -0.002 & 0.069 & 0.070 & 0.067 & 0.046 & 0.046 & 0.045 \\
  \hline \multicolumn{14}{|l|}{$n=1000$}\\ \hline
  4 & 0.5 & 0.4 & 0.0 & 0.000 & 0.001 & 0.001 & 0.001 & 0.029 & 0.029 & 0.029 & 0.019 & 0.019 & 0.019 \\
  4 & 0.8 & 0.4 & 0.0 & 0.000 & 0.000 & 0.000 & -0.000 & 0.033 & 0.033 & 0.032 & 0.021 & 0.021 & 0.021 \\
  4 & 0.5 & 0.6 & 0.0 & 0.000 & 0.000 & 0.000 & 0.000 & 0.018 & 0.018 & 0.018 & 0.012 & 0.012 & 0.012 \\
  4 & 0.8 & 0.6 & 0.0 & 0.000 & 0.001 & 0.001 & 0.001 & 0.018 & 0.018 & 0.018 & 0.013 & 0.013 & 0.013 \\
  8 & 0.5 & 0.4 & 0.0 & 0.000 & -0.000 & -0.000 & -0.000 & 0.024 & 0.025 & 0.024 & 0.016 & 0.016 & 0.016 \\
  8 & 0.8 & 0.4 & 0.0 & 0.000 & 0.001 & 0.001 & 0.001 & 0.028 & 0.029 & 0.028 & 0.018 & 0.019 & 0.019 \\
  8 & 0.5 & 0.6 & 0.0 & 0.000 & 0.000 & 0.000 & 0.000 & 0.017 & 0.017 & 0.017 & 0.011 & 0.011 & 0.011 \\
  8 & 0.8 & 0.6 & 0.0 & 0.000 & 0.001 & 0.001 & 0.001 & 0.017 & 0.017 & 0.017 & 0.011 & 0.011 & 0.011 \\
  \hline
  4 & 0.5 & 0.4 & 0.2 & 0.302 & 0.001 & 0.002 & 0.000 & 0.045 & 0.043 & 0.045 & 0.030 & 0.029 & 0.030 \\
  4 & 0.8 & 0.4 & 0.2 & 0.302 & 0.017 & 0.021 & 0.003 & 0.065 & 0.064 & 0.058 & 0.040 & 0.037 & 0.038 \\
  4 & 0.5 & 0.6 & 0.2 & 0.201 & 0.000 & 0.002 & 0.000 & 0.026 & 0.025 & 0.026 & 0.017 & 0.016 & 0.017 \\
  4 & 0.8 & 0.6 & 0.2 & 0.201 & 0.007 & 0.007 & 0.000 & 0.033 & 0.032 & 0.030 & 0.021 & 0.021 & 0.020 \\
  8 & 0.5 & 0.4 & 0.2 & 0.302 & 0.001 & 0.003 & 0.001 & 0.035 & 0.036 & 0.035 & 0.023 & 0.023 & 0.023 \\
  8 & 0.8 & 0.4 & 0.2 & 0.302 & 0.008 & 0.013 & -0.000 & 0.048 & 0.050 & 0.045 & 0.030 & 0.032 & 0.029 \\
  8 & 0.5 & 0.6 & 0.2 & 0.201 & -0.001 & -0.001 & -0.001 & 0.022 & 0.022 & 0.022 & 0.015 & 0.015 & 0.014 \\
  8 & 0.8 & 0.6 & 0.2 & 0.201 & 0.004 & 0.005 & -0.000 & 0.025 & 0.025 & 0.024 & 0.017 & 0.017 & 0.016 \\
  \hline
  4 & 0.5 & 0.4 & 0.4 & 0.427 & 0.005 & 0.007 & 0.004 & 0.059 & 0.053 & 0.058 & 0.039 & 0.033 & 0.038 \\
  4 & 0.8 & 0.4 & 0.4 & 0.427 & 0.025 & 0.029 & 0.004 & 0.084 & 0.076 & 0.076 & 0.051 & 0.047 & 0.049 \\
  4 & 0.5 & 0.6 & 0.4 & 0.285 & 0.001 & 0.001 & 0.000 & 0.032 & 0.031 & 0.032 & 0.022 & 0.021 & 0.022 \\
  4 & 0.8 & 0.6 & 0.4 & 0.285 & 0.011 & 0.014 & 0.002 & 0.040 & 0.039 & 0.037 & 0.025 & 0.026 & 0.023 \\
  8 & 0.5 & 0.4 & 0.4 & 0.427 & -0.000 & 0.002 & -0.000 & 0.044 & 0.042 & 0.044 & 0.029 & 0.028 & 0.029 \\
  8 & 0.8 & 0.4 & 0.4 & 0.427 & 0.012 & 0.015 & 0.001 & 0.059 & 0.057 & 0.055 & 0.038 & 0.035 & 0.035 \\
  8 & 0.5 & 0.6 & 0.4 & 0.285 & -0.001 & -0.001 & -0.001 & 0.025 & 0.025 & 0.025 & 0.017 & 0.017 & 0.017 \\
  8 & 0.8 & 0.6 & 0.4 & 0.285 & 0.006 & 0.007 & -0.000 & 0.030 & 0.031 & 0.029 & 0.019 & 0.019 & 0.019 \\
   \hline
\multicolumn{14}{l}{\emph{Note:} $p$ denotes number of measurement
items $Y_{j}$, $\pi_{Y}$ marginal proportion of $Y_{j}=1$, }\\
\multicolumn{14}{l}{
\hspace*{2em}$R_{Y}^{2}$ and $R^{2}_{Z}$ the $R^{2}$ statistics in models for
$Y_{j}$ and $Z$, and $\beta_{1}$ true value of $\beta_{1}$.}
\end{tabular}
}}
\label{sres_133_180}
\end{table}

\clearpage
\thispagestyle{empty}
\begin{table}[ht]
\caption{Simulation results for point estimates of structural regression
coefficient $\beta_{1}$ of
a latent covariate $\eta_{1}$ for a conditionally normally distributed
latent response $\eta_{2}$ (Case~3).}
\centering
{\small{
\begin{tabular}{|rrrrr|rrr|rrr|rrr|}
  \hline
  & & & &
  & \multicolumn{3}{|c|}{Bias}
  & \multicolumn{3}{|c|}{RMSE}
  & \multicolumn{3}{|c|}{MAE} \\
$p$ & $\pi_{Y}$ & $R^{2}_{Y}$ & $R^{2}_{\eta}$ & $\beta_1$ & 2-step & 1-step & 3-step & 2-step & 1-step & 3-step & 2-step & 1-step & 3-step \\
  \hline \multicolumn{14}{|l|}{$n=200$}\\ \hline
4 & 0.5 & 0.4 & 0.0 & 0.000 & -0.004 & -0.003 & -0.002 & 0.133 & 0.141 & 0.081 & 0.077 & 0.080 & 0.046 \\
  4 & 0.8 & 0.4 & 0.0 & 0.000 & -0.007 & -0.004 & -0.000 & 0.240 & 0.276 & 0.110 & 0.101 & 0.113 & 0.044 \\
  4 & 0.5 & 0.6 & 0.0 & 0.000 & 0.007 & 0.007 & 0.005 & 0.115 & 0.117 & 0.080 & 0.069 & 0.070 & 0.048 \\
  4 & 0.8 & 0.6 & 0.0 & 0.000 & 0.002 & 0.002 & 0.003 & 0.153 & 0.154 & 0.091 & 0.081 & 0.084 & 0.048 \\
  8 & 0.5 & 0.4 & 0.0 & 0.000 & -0.001 & -0.001 & -0.001 & 0.104 & 0.104 & 0.078 & 0.063 & 0.062 & 0.047 \\
  8 & 0.8 & 0.4 & 0.0 & 0.000 & 0.008 & 0.009 & 0.006 & 0.147 & 0.151 & 0.087 & 0.081 & 0.083 & 0.047 \\
  8 & 0.5 & 0.6 & 0.0 & 0.000 & -0.003 & -0.003 & -0.002 & 0.097 & 0.095 & 0.077 & 0.062 & 0.059 & 0.050 \\
  8 & 0.8 & 0.6 & 0.0 & 0.000 & -0.002 & -0.002 & -0.000 & 0.114 & 0.116 & 0.080 & 0.072 & 0.073 & 0.050 \\
  \hline
  4 & 0.5 & 0.4 & 0.2 & 0.447 & 0.016 & 0.035 & -0.162 & 0.236 & 0.237 & 0.221 & 0.132 & 0.132 & 0.202 \\
  4 & 0.8 & 0.4 & 0.2 & 0.447 & 0.031 & 0.070 & -0.208 & 0.404 & 0.385 & 0.300 & 0.183 & 0.177 & 0.267 \\
  4 & 0.5 & 0.6 & 0.2 & 0.447 & 0.019 & 0.025 & -0.116 & 0.182 & 0.189 & 0.176 & 0.117 & 0.116 & 0.148 \\
  4 & 0.8 & 0.6 & 0.2 & 0.447 & 0.030 & 0.039 & -0.144 & 0.257 & 0.244 & 0.225 & 0.135 & 0.134 & 0.183 \\
  8 & 0.5 & 0.4 & 0.2 & 0.447 & 0.013 & 0.010 & -0.104 & 0.177 & 0.188 & 0.170 & 0.109 & 0.110 & 0.137 \\
  8 & 0.8 & 0.4 & 0.2 & 0.447 & 0.023 & 0.040 & -0.152 & 0.249 & 0.250 & 0.222 & 0.144 & 0.136 & 0.197 \\
  8 & 0.5 & 0.6 & 0.2 & 0.447 & 0.005 & -0.008 & -0.081 & 0.143 & 0.159 & 0.143 & 0.092 & 0.093 & 0.116 \\
  8 & 0.8 & 0.6 & 0.2 & 0.447 & 0.024 & 0.028 & -0.100 & 0.183 & 0.181 & 0.170 & 0.114 & 0.111 & 0.137 \\
  \hline
  4 & 0.5 & 0.4 & 0.4 & 0.632 & 0.028 & 0.047 & -0.223 & 0.309 & 0.285 & 0.301 & 0.176 & 0.162 & 0.268 \\
  4 & 0.8 & 0.4 & 0.4 & 0.632 & 0.042 & 0.081 & -0.284 & 0.608 & 0.457 & 0.436 & 0.228 & 0.199 & 0.363 \\
  4 & 0.5 & 0.6 & 0.4 & 0.632 & 0.048 & 0.052 & -0.143 & 0.275 & 0.254 & 0.247 & 0.140 & 0.147 & 0.199 \\
  4 & 0.8 & 0.6 & 0.4 & 0.632 & 0.038 & 0.048 & -0.195 & 0.326 & 0.283 & 0.295 & 0.174 & 0.161 & 0.255 \\
  8 & 0.5 & 0.4 & 0.4 & 0.632 & 0.020 & 0.022 & -0.142 & 0.213 & 0.222 & 0.216 & 0.131 & 0.134 & 0.176 \\
  8 & 0.8 & 0.4 & 0.4 & 0.632 & 0.033 & 0.054 & -0.204 & 0.344 & 0.346 & 0.309 & 0.177 & 0.166 & 0.258 \\
  8 & 0.5 & 0.6 & 0.4 & 0.632 & 0.033 & 0.006 & -0.087 & 0.196 & 0.227 & 0.182 & 0.127 & 0.124 & 0.138 \\
  8 & 0.8 & 0.6 & 0.4 & 0.632 & 0.024 & 0.030 & -0.138 & 0.223 & 0.218 & 0.220 & 0.137 & 0.135 & 0.180 \\
  \hline \multicolumn{14}{|l|}{$n=1000$}\\ \hline
  4 & 0.5 & 0.4 & 0.0 & 0.000 & 0.000 & 0.000 & 0.000 & 0.053 & 0.054 & 0.032 & 0.036 & 0.036 & 0.021 \\
  4 & 0.8 & 0.4 & 0.0 & 0.000 & -0.001 & -0.001 & -0.000 & 0.072 & 0.073 & 0.032 & 0.049 & 0.049 & 0.021 \\
  4 & 0.5 & 0.6 & 0.0 & 0.000 & -0.001 & -0.001 & -0.001 & 0.045 & 0.045 & 0.031 & 0.030 & 0.030 & 0.021 \\
  4 & 0.8 & 0.6 & 0.0 & 0.000 & 0.001 & 0.001 & 0.001 & 0.054 & 0.055 & 0.032 & 0.036 & 0.036 & 0.021 \\
  8 & 0.5 & 0.4 & 0.0 & 0.000 & 0.000 & 0.000 & 0.000 & 0.045 & 0.045 & 0.033 & 0.031 & 0.031 & 0.023 \\
  8 & 0.8 & 0.4 & 0.0 & 0.000 & 0.001 & 0.001 & 0.001 & 0.059 & 0.059 & 0.034 & 0.038 & 0.038 & 0.022 \\
  8 & 0.5 & 0.6 & 0.0 & 0.000 & -0.001 & -0.001 & -0.000 & 0.040 & 0.040 & 0.032 & 0.026 & 0.026 & 0.021 \\
  8 & 0.8 & 0.6 & 0.0 & 0.000 & 0.000 & 0.000 & 0.000 & 0.045 & 0.045 & 0.032 & 0.030 & 0.030 & 0.021 \\
  \hline
  4 & 0.5 & 0.4 & 0.2 & 0.447 & 0.006 & 0.012 & -0.174 & 0.092 & 0.090 & 0.183 & 0.061 & 0.059 & 0.178 \\
  4 & 0.8 & 0.4 & 0.2 & 0.447 & 0.011 & 0.021 & -0.227 & 0.124 & 0.124 & 0.236 & 0.078 & 0.077 & 0.234 \\
  4 & 0.5 & 0.6 & 0.2 & 0.447 & 0.001 & 0.003 & -0.131 & 0.075 & 0.074 & 0.142 & 0.053 & 0.052 & 0.135 \\
  4 & 0.8 & 0.6 & 0.2 & 0.447 & 0.011 & 0.013 & -0.160 & 0.091 & 0.090 & 0.171 & 0.057 & 0.059 & 0.168 \\
  8 & 0.5 & 0.4 & 0.2 & 0.447 & 0.004 & 0.005 & -0.112 & 0.073 & 0.073 & 0.125 & 0.047 & 0.047 & 0.116 \\
  8 & 0.8 & 0.4 & 0.2 & 0.447 & 0.010 & 0.015 & -0.163 & 0.094 & 0.093 & 0.174 & 0.062 & 0.061 & 0.166 \\
  8 & 0.5 & 0.6 & 0.2 & 0.447 & 0.002 & 0.003 & -0.084 & 0.062 & 0.062 & 0.098 & 0.039 & 0.039 & 0.087 \\
  8 & 0.8 & 0.6 & 0.2 & 0.447 & 0.008 & 0.009 & -0.113 & 0.074 & 0.074 & 0.126 & 0.048 & 0.048 & 0.118 \\
  \hline
  4 & 0.5 & 0.4 & 0.4 & 0.632 & 0.009 & 0.014 & -0.244 & 0.118 & 0.112 & 0.255 & 0.076 & 0.076 & 0.250 \\
  4 & 0.8 & 0.4 & 0.4 & 0.632 & 0.003 & 0.014 & -0.319 & 0.156 & 0.150 & 0.329 & 0.103 & 0.099 & 0.329 \\
  4 & 0.5 & 0.6 & 0.4 & 0.632 & 0.006 & 0.007 & -0.177 & 0.093 & 0.090 & 0.190 & 0.061 & 0.058 & 0.182 \\
  4 & 0.8 & 0.6 & 0.4 & 0.632 & 0.009 & 0.010 & -0.218 & 0.117 & 0.109 & 0.232 & 0.074 & 0.071 & 0.223 \\
  8 & 0.5 & 0.4 & 0.4 & 0.632 & 0.003 & 0.005 & -0.158 & 0.088 & 0.086 & 0.172 & 0.059 & 0.057 & 0.164 \\
  8 & 0.8 & 0.4 & 0.4 & 0.632 & -0.001 & 0.007 & -0.231 & 0.114 & 0.111 & 0.244 & 0.078 & 0.072 & 0.237 \\
  8 & 0.5 & 0.6 & 0.4 & 0.632 & 0.007 & 0.008 & -0.110 & 0.080 & 0.079 & 0.128 & 0.054 & 0.053 & 0.112 \\
  8 & 0.8 & 0.6 & 0.4 & 0.632 & 0.006 & 0.008 & -0.152 & 0.088 & 0.086 & 0.167 & 0.060 & 0.058 & 0.156 \\
   \hline
\multicolumn{14}{l}{\emph{Note:} $p$ denotes number of measurement
items $Y_{j}$, $\pi_{Y}$ marginal proportion of $Y_{j}=1$, }\\
\multicolumn{14}{l}{
\hspace*{2em}$R_{Y}^{2}$ and $R^{2}_{\eta}$ the $R^{2}$ statistics in models for
$Y_{j}$ and $\eta_{2}$, and $\beta_{1}$ true value of $\beta_{1}$.}
\end{tabular}
}}
\label{sres_301_348}
\end{table}

\clearpage
\thispagestyle{empty}
\begin{table}[ht]
\caption{Simulation results for standard error estimates of structural regression
coefficient $\beta_{1}$ of a normally distributed
covariate $X$ for a latent response $\eta$
(Case 1(a)).}
\centering
{\small{
\begin{tabular}{|rrrrr|rrrr|rrr|rr|}
  \hline
  &&&&
  & \multicolumn{4}{c|}{Simulation standard deviation}
  &&&
  & \multicolumn{2}{c|}{2-step}
  \\
  &&&&
  & \multicolumn{4}{c|}{vs.\ mean of est.\ std.\ error}
  & \multicolumn{3}{c|}{Coverage of}
  & \multicolumn{2}{c|}{without}
  \\
  &&&&
  & \multicolumn{2}{|c}{2-step}
  & \multicolumn{2}{c|}{1-step}
  & \multicolumn{3}{c|}{95\% conf.\ interval}
  & \multicolumn{2}{c|}{step-1 var.$^{\dagger}$}
\\
$p$ & $p_{Y}$ & $R^{2}_{Y}$ & $R^{2}_{\eta}$ & $\beta_1$
& s.d.\ & m(se) & s.d.\ & m(se) & 2-st.\ & 1-st.\ & 3-st.\ & \%var. &
{\footnotesize{cover.}} \\
\hline \multicolumn{14}{|l|}{$n=200$}\\ \hline
4 & 0.5 & 0.4 & 0.0 & 0.000 & 0.146 & 0.149 & 0.151 & 0.149 & 98.4 & 98.0 & 95.3 & 92.6 & 95.3 \\
  4 & 0.8 & 0.4 & 0.0 & 0.000 & 0.196 & 0.204 & 0.208 & 0.201 & 99.2 & 98.1 & 94.8 & 86.5 & 95.0 \\
  4 & 0.5 & 0.6 & 0.0 & 0.000 & 0.207 & 0.205 & 0.208 & 0.204 & 96.1 & 95.8 & 93.2 & 95.2 & 93.4 \\
  4 & 0.8 & 0.6 & 0.0 & 0.000 & 0.251 & 0.240 & 0.253 & 0.243 & 97.4 & 96.8 & 94.0 & 92.8 & 94.2 \\
  8 & 0.5 & 0.4 & 0.0 & 0.000 & 0.132 & 0.129 & 0.132 & 0.129 & 96.8 & 96.8 & 95.3 & 95.6 & 95.3 \\
  8 & 0.8 & 0.4 & 0.0 & 0.000 & 0.150 & 0.150 & 0.152 & 0.150 & 98.2 & 98.0 & 95.1 & 92.6 & 95.0 \\
  8 & 0.5 & 0.6 & 0.0 & 0.000 & 0.187 & 0.186 & 0.185 & 0.185 & 95.4 & 95.6 & 94.2 & 96.8 & 94.3 \\
  8 & 0.8 & 0.6 & 0.0 & 0.000 & 0.203 & 0.202 & 0.203 & 0.202 & 96.5 & 96.2 & 94.7 & 95.9 & 94.7 \\
\hline
  4 & 0.5 & 0.4 & 0.2 & 0.662 & 0.250 & 0.240 & 0.226 & 0.218 & 94.1 & 95.4 & 27.5 & 40.4 & 80.9 \\
  4 & 0.8 & 0.4 & 0.2 & 0.662 & 0.523 & 0.386 & 0.363 & 0.301 & 92.3 & 93.7 & 13.5 & 34.6 & 73.7 \\
  4 & 0.5 & 0.6 & 0.2 & 0.993 & 0.317 & 0.310 & 0.296 & 0.294 & 95.9 & 96.1 & 38.1 & 46.6 & 83.3 \\
  4 & 0.8 & 0.6 & 0.2 & 0.993 & 0.513 & 0.400 & 0.387 & 0.351 & 93.4 & 93.8 & 25.4 & 44.6 & 79.2 \\
  8 & 0.5 & 0.4 & 0.2 & 0.662 & 0.182 & 0.181 & 0.178 & 0.178 & 95.7 & 95.8 & 47.6 & 45.3 & 82.6 \\
  8 & 0.8 & 0.4 & 0.2 & 0.662 & 0.253 & 0.232 & 0.263 & 0.227 & 94.9 & 94.7 & 26.0 & 39.3 & 76.2 \\
  8 & 0.5 & 0.6 & 0.2 & 0.993 & 0.252 & 0.252 & 0.250 & 0.249 & 95.3 & 95.3 & 61.4 & 51.0 & 85.2 \\
  8 & 0.8 & 0.6 & 0.2 & 0.993 & 0.310 & 0.289 & 0.307 & 0.285 & 95.5 & 95.6 & 42.1 & 47.5 & 80.6 \\
\hline
  4 & 0.5 & 0.4 & 0.4 & 0.937 & 0.303 & 0.297 & 0.249 & 0.243 & 94.7 & 96.3 & 17.5 & 24.3 & 66.0 \\
  4 & 0.8 & 0.4 & 0.4 & 0.937 & 0.913 & 0.566 & 0.357 & 0.323 & 94.0 & 95.4 & 8.3 & 20.3 & 61.8 \\
  4 & 0.5 & 0.6 & 0.4 & 1.405 & 0.396 & 0.393 & 0.357 & 0.347 & 96.6 & 96.8 & 26.9 & 30.6 & 75.4 \\
  4 & 0.8 & 0.6 & 0.4 & 1.405 & 0.635 & 0.636 & 0.442 & 0.407 & 93.7 & 94.6 & 14.5 & 28.6 & 71.2 \\
  8 & 0.5 & 0.4 & 0.4 & 0.937 & 0.231 & 0.223 & 0.229 & 0.214 & 96.1 & 94.7 & 29.9 & 27.3 & 71.5 \\
  8 & 0.8 & 0.4 & 0.4 & 0.937 & 0.307 & 0.291 & 0.293 & 0.270 & 94.4 & 95.0 & 13.9 & 23.2 & 67.2 \\
  8 & 0.5 & 0.6 & 0.4 & 1.405 & 0.313 & 0.300 & 0.312 & 0.294 & 95.2 & 95.0 & 43.0 & 32.3 & 74.0 \\
  8 & 0.8 & 0.6 & 0.4 & 1.405 & 0.361 & 0.349 & 0.352 & 0.337 & 95.9 & 95.8 & 22.8 & 30.3 & 72.7 \\
\hline \multicolumn{14}{|l|}{$n=1000$}\\ \hline
  4 & 0.5 & 0.4 & 0.0 & 0.000 & 0.062 & 0.061 & 0.062 & 0.061 & 95.4 & 95.3 & 94.3 & 98.4 & 94.3 \\
  4 & 0.8 & 0.4 & 0.0 & 0.000 & 0.073 & 0.072 & 0.074 & 0.072 & 95.5 & 95.2 & 94.8 & 97.3 & 94.8 \\
  4 & 0.5 & 0.6 & 0.0 & 0.000 & 0.086 & 0.086 & 0.086 & 0.086 & 95.5 & 95.4 & 95.0 & 99.1 & 95.2 \\
  4 & 0.8 & 0.6 & 0.0 & 0.000 & 0.095 & 0.093 & 0.095 & 0.093 & 95.1 & 95.0 & 94.7 & 98.7 & 94.9 \\
  8 & 0.5 & 0.4 & 0.0 & 0.000 & 0.055 & 0.055 & 0.055 & 0.055 & 94.7 & 94.6 & 94.4 & 99.1 & 94.4 \\
  8 & 0.8 & 0.4 & 0.0 & 0.000 & 0.065 & 0.062 & 0.065 & 0.062 & 95.4 & 95.2 & 94.2 & 98.4 & 94.3 \\
  8 & 0.5 & 0.6 & 0.0 & 0.000 & 0.080 & 0.079 & 0.079 & 0.079 & 95.0 & 95.2 & 94.8 & 99.4 & 94.9 \\
  8 & 0.8 & 0.6 & 0.0 & 0.000 & 0.085 & 0.085 & 0.085 & 0.085 & 95.4 & 95.2 & 94.9 & 99.2 & 95.2 \\
\hline
  4 & 0.5 & 0.4 & 0.2 & 0.662 & 0.098 & 0.094 & 0.092 & 0.089 & 93.9 & 94.0 & 0.8 & 41.8 & 76.7 \\
  4 & 0.8 & 0.4 & 0.2 & 0.662 & 0.117 & 0.121 & 0.107 & 0.112 & 95.9 & 96.1 & 0.0 & 36.5 & 77.8 \\
  4 & 0.5 & 0.6 & 0.2 & 0.993 & 0.127 & 0.125 & 0.122 & 0.122 & 95.0 & 95.7 & 3.8 & 47.7 & 81.9 \\
  4 & 0.8 & 0.6 & 0.2 & 0.993 & 0.143 & 0.142 & 0.137 & 0.138 & 94.7 & 95.0 & 0.4 & 45.6 & 81.4 \\
  8 & 0.5 & 0.4 & 0.2 & 0.662 & 0.077 & 0.078 & 0.076 & 0.077 & 94.6 & 95.6 & 7.0 & 45.9 & 81.7 \\
  8 & 0.8 & 0.4 & 0.2 & 0.662 & 0.095 & 0.095 & 0.093 & 0.093 & 95.2 & 95.0 & 0.1 & 40.4 & 78.7 \\
  8 & 0.5 & 0.6 & 0.2 & 0.993 & 0.108 & 0.106 & 0.108 & 0.105 & 93.4 & 93.5 & 17.8 & 51.4 & 82.5 \\
  8 & 0.8 & 0.6 & 0.2 & 0.993 & 0.116 & 0.119 & 0.115 & 0.119 & 96.6 & 96.4 & 2.5 & 48.7 & 83.4 \\
\hline
  4 & 0.5 & 0.4 & 0.4 & 0.937 & 0.122 & 0.119 & 0.107 & 0.104 & 94.0 & 94.3 & 0.2 & 25.7 & 67.3 \\
  4 & 0.8 & 0.4 & 0.4 & 0.937 & 0.158 & 0.155 & 0.133 & 0.128 & 95.3 & 94.8 & 0.0 & 22.4 & 65.3 \\
  4 & 0.5 & 0.6 & 0.4 & 1.405 & 0.157 & 0.155 & 0.150 & 0.144 & 94.4 & 94.1 & 1.4 & 31.5 & 75.2 \\
  4 & 0.8 & 0.6 & 0.4 & 1.405 & 0.184 & 0.178 & 0.170 & 0.164 & 95.3 & 94.1 & 0.1 & 30.2 & 69.4 \\
  8 & 0.5 & 0.4 & 0.4 & 0.937 & 0.096 & 0.095 & 0.093 & 0.092 & 94.8 & 95.0 & 2.1 & 27.5 & 69.4 \\
  8 & 0.8 & 0.4 & 0.4 & 0.937 & 0.121 & 0.120 & 0.114 & 0.113 & 95.2 & 96.0 & 0.1 & 24.0 & 66.8 \\
  8 & 0.5 & 0.6 & 0.4 & 1.405 & 0.124 & 0.128 & 0.123 & 0.126 & 94.3 & 94.2 & 6.3 & 32.7 & 75.6 \\
  8 & 0.8 & 0.6 & 0.4 & 1.405 & 0.152 & 0.148 & 0.149 & 0.144 & 95.5 & 95.2 & 1.3 & 30.8 & 72.1 \\
   \hline
\multicolumn{14}{l}{\emph{Note:} $p$ denotes number of measurement
items $Y_{j}$, $\pi_{Y}$ marginal proportion of $Y_{j}=1$, }\\
\multicolumn{14}{l}{
\hspace*{2em}$R_{Y}^{2}$ and $R^{2}_{\eta}$ the $R^{2}$ statistics in models for
$Y_{j}$ and $\eta$, and $\beta_{1}$ true value of $\beta_{1}$.}\\
\multicolumn{14}{l}{
$\dagger$ Average \% of the variance of 2-step estimator accounted
for by step-2 variance alone,}\\
\multicolumn{14}{l}{
\hspace*{1em} and coverage of 95\% confidence interval if only this variance is
included.}\\
\end{tabular}
}}
\label{sres_1_48_se}
\end{table}

\clearpage
\thispagestyle{empty}
\begin{table}[ht]
\caption{Simulation results for standard error estimates of structural regression
coefficient $\beta_{1}$ of a binary covariate $X$ for a latent response
$\eta$ (Case 1(b)).}
\centering
{\small{
\begin{tabular}{|rrrrr|rrrr|rrr|rr|}
  \hline
  &&&&
  & \multicolumn{4}{c|}{Simulation standard deviation}
  &&&
  & \multicolumn{2}{c|}{2-step}
  \\
  &&&&
  & \multicolumn{4}{c|}{vs.\ mean of est.\ std.\ error}
  & \multicolumn{3}{c|}{Coverage of}
  & \multicolumn{2}{c|}{without}
  \\
  &&&&
  & \multicolumn{2}{|c}{2-step}
  & \multicolumn{2}{c|}{1-step}
  & \multicolumn{3}{c|}{95\% conf.\ interval}
  & \multicolumn{2}{c|}{step-1 var.$^{\dagger}$}
\\
$p$ & $p_{Y}$ & $R^{2}_{Y}$ & $R^{2}_{\eta}$ & $\beta_1$
& s.d.\ & m(se) & s.d.\ & m(se) & 2-st.\ & 1-st.\ & 3-st.\ & \%var. &
{\footnotesize{cover.}} \\
\hline \multicolumn{14}{|l|}{$n=200$}\\ \hline
4 & 0.5 & 0.4 & 0.0 & 0.000 & 0.144 & 0.147 & 0.147 & 0.147 & 97.9 & 97.7 & 95.5 & 92.7 & 95.6 \\
  4 & 0.8 & 0.4 & 0.0 & 0.000 & 0.191 & 0.239 & 0.213 & 0.214 & 99.2 & 98.8 & 94.4 & 85.2 & 94.7 \\
  4 & 0.5 & 0.6 & 0.0 & 0.000 & 0.209 & 0.204 & 0.210 & 0.204 & 95.8 & 95.6 & 93.7 & 95.2 & 93.7 \\
  4 & 0.8 & 0.6 & 0.0 & 0.000 & 0.260 & 0.243 & 0.254 & 0.243 & 97.5 & 97.3 & 94.1 & 92.9 & 94.4 \\
  8 & 0.5 & 0.4 & 0.0 & 0.000 & 0.126 & 0.129 & 0.126 & 0.129 & 97.1 & 97.1 & 96.0 & 95.8 & 96.0 \\
  8 & 0.8 & 0.4 & 0.0 & 0.000 & 0.160 & 0.151 & 0.162 & 0.151 & 98.4 & 98.1 & 96.4 & 93.2 & 96.3 \\
  8 & 0.5 & 0.6 & 0.0 & 0.000 & 0.185 & 0.185 & 0.184 & 0.184 & 96.5 & 96.5 & 95.3 & 96.9 & 95.4 \\
  8 & 0.8 & 0.6 & 0.0 & 0.000 & 0.204 & 0.201 & 0.206 & 0.201 & 96.9 & 96.8 & 94.7 & 95.7 & 94.8 \\
\hline
  4 & 0.5 & 0.4 & 0.2 & 0.662 & 0.235 & 0.228 & 0.219 & 0.209 & 94.2 & 94.0 & 27.6 & 38.6 & 76.3 \\
  4 & 0.8 & 0.4 & 0.2 & 0.662 & 0.562 & 0.486 & 0.312 & 0.288 & 93.3 & 94.1 & 13.7 & 33.8 & 73.1 \\
  4 & 0.5 & 0.6 & 0.2 & 0.993 & 0.297 & 0.296 & 0.295 & 0.285 & 94.4 & 94.5 & 44.1 & 45.9 & 82.1 \\
  4 & 0.8 & 0.6 & 0.2 & 0.993 & 0.459 & 0.378 & 0.420 & 0.354 & 94.3 & 95.5 & 27.6 & 42.9 & 80.2 \\
  8 & 0.5 & 0.4 & 0.2 & 0.662 & 0.191 & 0.178 & 0.188 & 0.174 & 93.2 & 93.5 & 51.0 & 44.3 & 78.5 \\
  8 & 0.8 & 0.4 & 0.2 & 0.662 & 0.239 & 0.226 & 0.231 & 0.218 & 92.8 & 93.1 & 25.0 & 38.7 & 74.9 \\
  8 & 0.5 & 0.6 & 0.2 & 0.993 & 0.252 & 0.247 & 0.251 & 0.244 & 95.2 & 95.2 & 64.9 & 49.3 & 83.5 \\
  8 & 0.8 & 0.6 & 0.2 & 0.993 & 0.305 & 0.285 & 0.299 & 0.281 & 95.1 & 95.0 & 42.1 & 46.5 & 79.6 \\
\hline
  4 & 0.5 & 0.4 & 0.4 & 0.937 & 0.305 & 0.297 & 0.233 & 0.230 & 95.0 & 95.6 & 21.5 & 22.6 & 66.7 \\
  4 & 0.8 & 0.4 & 0.4 & 0.937 & 0.584 & 0.444 & 0.382 & 0.349 & 93.1 & 95.8 & 6.5 & 20.5 & 63.2 \\
  4 & 0.5 & 0.6 & 0.4 & 1.405 & 0.395 & 0.372 & 0.348 & 0.322 & 94.8 & 94.4 & 34.0 & 28.1 & 69.8 \\
  4 & 0.8 & 0.6 & 0.4 & 1.405 & 0.502 & 0.470 & 0.501 & 0.402 & 94.6 & 94.1 & 15.6 & 27.8 & 72.1 \\
  8 & 0.5 & 0.4 & 0.4 & 0.937 & 0.217 & 0.213 & 0.207 & 0.201 & 94.6 & 93.9 & 38.1 & 26.0 & 68.0 \\
  8 & 0.8 & 0.4 & 0.4 & 0.937 & 0.289 & 0.285 & 0.277 & 0.265 & 94.8 & 94.9 & 14.8 & 22.3 & 64.9 \\
  8 & 0.5 & 0.6 & 0.4 & 1.405 & 0.318 & 0.294 & 0.305 & 0.281 & 94.6 & 94.7 & 51.7 & 29.7 & 70.7 \\
  8 & 0.8 & 0.6 & 0.4 & 1.405 & 0.352 & 0.348 & 0.345 & 0.335 & 96.6 & 96.3 & 32.8 & 28.5 & 72.4 \\
\hline \multicolumn{14}{|l|}{$n=1000$}\\ \hline
  4 & 0.5 & 0.4 & 0.0 & 0.000 & 0.061 & 0.061 & 0.062 & 0.061 & 94.9 & 94.6 & 93.8 & 98.5 & 93.9 \\
  4 & 0.8 & 0.4 & 0.0 & 0.000 & 0.073 & 0.073 & 0.074 & 0.073 & 95.2 & 95.0 & 94.4 & 97.2 & 94.4 \\
  4 & 0.5 & 0.6 & 0.0 & 0.000 & 0.082 & 0.086 & 0.083 & 0.085 & 95.9 & 95.9 & 95.8 & 99.1 & 95.9 \\
  4 & 0.8 & 0.6 & 0.0 & 0.000 & 0.094 & 0.093 & 0.094 & 0.093 & 94.6 & 94.6 & 94.4 & 98.7 & 94.4 \\
  8 & 0.5 & 0.4 & 0.0 & 0.000 & 0.056 & 0.055 & 0.056 & 0.055 & 94.3 & 94.4 & 94.1 & 99.1 & 94.2 \\
  8 & 0.8 & 0.4 & 0.0 & 0.000 & 0.064 & 0.062 & 0.064 & 0.062 & 95.7 & 95.7 & 95.1 & 98.5 & 95.1 \\
  8 & 0.5 & 0.6 & 0.0 & 0.000 & 0.081 & 0.079 & 0.081 & 0.079 & 95.1 & 95.6 & 94.8 & 99.3 & 94.7 \\
  8 & 0.8 & 0.6 & 0.0 & 0.000 & 0.087 & 0.085 & 0.087 & 0.085 & 94.1 & 94.0 & 93.6 & 99.2 & 93.6 \\
\hline
  4 & 0.5 & 0.4 & 0.2 & 0.662 & 0.093 & 0.094 & 0.085 & 0.088 & 95.4 & 95.8 & 1.7 & 40.3 & 79.2 \\
  4 & 0.8 & 0.4 & 0.2 & 0.662 & 0.128 & 0.120 & 0.114 & 0.111 & 92.4 & 93.9 & 0.0 & 36.9 & 73.4 \\
  4 & 0.5 & 0.6 & 0.2 & 0.993 & 0.118 & 0.122 & 0.118 & 0.119 & 95.8 & 95.1 & 7.1 & 46.3 & 83.8 \\
  4 & 0.8 & 0.6 & 0.2 & 0.993 & 0.145 & 0.140 & 0.140 & 0.135 & 93.8 & 94.5 & 0.7 & 44.8 & 79.3 \\
  8 & 0.5 & 0.4 & 0.2 & 0.662 & 0.079 & 0.077 & 0.078 & 0.076 & 93.9 & 93.8 & 13.8 & 44.8 & 79.3 \\
  8 & 0.8 & 0.4 & 0.2 & 0.662 & 0.096 & 0.095 & 0.094 & 0.093 & 94.8 & 94.6 & 0.7 & 39.8 & 79.0 \\
  8 & 0.5 & 0.6 & 0.2 & 0.993 & 0.109 & 0.105 & 0.109 & 0.104 & 93.7 & 93.2 & 31.0 & 50.3 & 80.7 \\
  8 & 0.8 & 0.6 & 0.2 & 0.993 & 0.118 & 0.119 & 0.117 & 0.118 & 95.7 & 95.5 & 4.3 & 47.2 & 82.4 \\
\hline
  4 & 0.5 & 0.4 & 0.4 & 0.937 & 0.119 & 0.115 & 0.103 & 0.098 & 94.0 & 94.1 & 1.1 & 24.2 & 66.4 \\
  4 & 0.8 & 0.4 & 0.4 & 0.937 & 0.165 & 0.154 & 0.131 & 0.128 & 94.0 & 94.3 & 0.0 & 22.5 & 63.0 \\
  4 & 0.5 & 0.6 & 0.4 & 1.405 & 0.154 & 0.150 & 0.141 & 0.136 & 95.2 & 94.4 & 8.1 & 29.0 & 69.3 \\
  4 & 0.8 & 0.6 & 0.4 & 1.405 & 0.181 & 0.177 & 0.165 & 0.161 & 95.5 & 94.8 & 0.6 & 28.9 & 71.0 \\
  8 & 0.5 & 0.4 & 0.4 & 0.937 & 0.095 & 0.092 & 0.089 & 0.088 & 94.1 & 94.6 & 7.8 & 26.2 & 67.2 \\
  8 & 0.8 & 0.4 & 0.4 & 0.937 & 0.114 & 0.119 & 0.111 & 0.112 & 96.4 & 95.3 & 0.0 & 23.2 & 69.5 \\
  8 & 0.5 & 0.6 & 0.4 & 1.405 & 0.128 & 0.125 & 0.123 & 0.121 & 95.0 & 94.9 & 32.9 & 30.1 & 69.2 \\
  8 & 0.8 & 0.6 & 0.4 & 1.405 & 0.148 & 0.144 & 0.143 & 0.140 & 94.2 & 94.9 & 2.7 & 29.4 & 72.2 \\
   \hline
\multicolumn{14}{l}{\emph{Note:} $p$ denotes number of measurement
items $Y_{j}$, $\pi_{Y}$ marginal proportion of $Y_{j}=1$, }\\
\multicolumn{14}{l}{
\hspace*{2em}$R_{Y}^{2}$ and $R^{2}_{\eta}$ the $R^{2}$ statistics in models for
$Y_{j}$ and $\eta$, and $\beta_{1}$ true value of $\beta_{1}$.}\\
\multicolumn{14}{l}{
$\dagger$ Average \% of the variance of 2-step estimator accounted
for by step-2 variance alone,}\\
\multicolumn{14}{l}{
\hspace*{1em} and coverage of 95\% confidence interval if only this variance is
included.}\\
\end{tabular}
}}
\label{sres_201_132_se}
\end{table}

\clearpage
\thispagestyle{empty}
\begin{table}[ht]
\caption{Simulation results for standard error estimates of structural regression
coefficient $\beta_{1}$ of a latent covariate $\eta$ for a conditionally
normally distributed response $Z$ (Case 2(a)).}
\centering
{\small{
\begin{tabular}{|rrrrr|rrrr|rrr|rr|}
  \hline
  &&&&
  & \multicolumn{4}{c|}{Simulation standard deviation}
  &&&
  & \multicolumn{2}{c|}{2-step}
  \\
  &&&&
  & \multicolumn{4}{c|}{vs.\ mean of est.\ std.\ error}
  & \multicolumn{3}{c|}{Coverage of}
  & \multicolumn{2}{c|}{without}
  \\
  &&&&
  & \multicolumn{2}{|c}{2-step}
  & \multicolumn{2}{c|}{1-step}
  & \multicolumn{3}{c|}{95\% conf.\ interval}
  & \multicolumn{2}{c|}{step-1 var.$^{\dagger}$}
\\
$p$ & $p_{Y}$ & $R^{2}_{Y}$ & $R^{2}_{Z}$ & $\beta_1$
& s.d.\ & m(se) & s.d.\ & m(se) & 2-st.\ & 1-st.\ & 3-st.\ & \%var. &
{\footnotesize{cover.}} \\
\hline \multicolumn{14}{|l|}{$n=200$}\\ \hline
4 & 0.5 & 0.4 & 0.0 & 0.000 & 0.067 & 0.065 & 0.070 & 0.066 & 97.9 & 97.6 & 94.7 & 92.5 & 95.2 \\
  4 & 0.8 & 0.4 & 0.0 & 0.000 & 0.089 & 0.085 & 0.105 & 0.093 & 98.6 & 98.4 & 91.8 & 86.4 & 93.3 \\
  4 & 0.5 & 0.6 & 0.0 & 0.000 & 0.039 & 0.039 & 0.040 & 0.040 & 97.7 & 97.7 & 96.1 & 95.7 & 96.3 \\
  4 & 0.8 & 0.6 & 0.0 & 0.000 & 0.045 & 0.044 & 0.046 & 0.045 & 98.1 & 97.8 & 95.0 & 93.7 & 95.6 \\
  8 & 0.5 & 0.4 & 0.0 & 0.000 & 0.059 & 0.058 & 0.060 & 0.059 & 96.5 & 96.2 & 94.6 & 95.9 & 94.7 \\
  8 & 0.8 & 0.4 & 0.0 & 0.000 & 0.068 & 0.067 & 0.072 & 0.069 & 97.2 & 96.6 & 94.3 & 92.9 & 94.5 \\
  8 & 0.5 & 0.6 & 0.0 & 0.000 & 0.038 & 0.036 & 0.038 & 0.037 & 96.0 & 95.9 & 95.4 & 96.9 & 95.3 \\
  8 & 0.8 & 0.6 & 0.0 & 0.000 & 0.040 & 0.039 & 0.040 & 0.040 & 96.9 & 96.7 & 95.4 & 96.0 & 95.3 \\
\hline
  4 & 0.5 & 0.4 & 0.2 & 0.302 & 0.111 & 0.105 & 0.112 & 0.102 & 92.5 & 92.7 & 72.9 & 46.8 & 79.6 \\
  4 & 0.8 & 0.4 & 0.2 & 0.302 & 0.159 & 0.145 & 0.157 & 0.141 & 89.6 & 91.9 & 67.6 & 43.1 & 74.7 \\
  4 & 0.5 & 0.6 & 0.2 & 0.201 & 0.060 & 0.058 & 0.060 & 0.057 & 93.5 & 92.9 & 73.2 & 50.2 & 79.4 \\
  4 & 0.8 & 0.6 & 0.2 & 0.201 & 0.065 & 0.066 & 0.063 & 0.064 & 93.6 & 93.9 & 75.7 & 47.8 & 82.0 \\
  8 & 0.5 & 0.4 & 0.2 & 0.302 & 0.083 & 0.080 & 0.085 & 0.081 & 92.1 & 92.5 & 76.9 & 48.1 & 80.0 \\
  8 & 0.8 & 0.4 & 0.2 & 0.302 & 0.113 & 0.105 & 0.118 & 0.107 & 91.1 & 92.3 & 72.7 & 42.1 & 77.1 \\
  8 & 0.5 & 0.6 & 0.2 & 0.201 & 0.051 & 0.048 & 0.051 & 0.048 & 93.5 & 93.7 & 77.7 & 51.8 & 80.6 \\
  8 & 0.8 & 0.6 & 0.2 & 0.201 & 0.055 & 0.055 & 0.056 & 0.054 & 93.0 & 93.2 & 77.0 & 48.1 & 80.4 \\
\hline
  4 & 0.5 & 0.4 & 0.4 & 0.427 & 0.134 & 0.131 & 0.128 & 0.119 & 92.7 & 92.7 & 59.9 & 33.4 & 73.5 \\
  4 & 0.8 & 0.4 & 0.4 & 0.427 & 0.181 & 0.182 & 0.189 & 0.163 & 93.5 & 94.6 & 58.4 & 31.4 & 73.9 \\
  4 & 0.5 & 0.6 & 0.4 & 0.285 & 0.073 & 0.072 & 0.070 & 0.067 & 94.1 & 93.9 & 60.8 & 35.9 & 74.5 \\
  4 & 0.8 & 0.6 & 0.4 & 0.285 & 0.086 & 0.083 & 0.082 & 0.078 & 93.4 & 93.7 & 59.0 & 34.1 & 71.9 \\
  8 & 0.5 & 0.4 & 0.4 & 0.427 & 0.107 & 0.101 & 0.106 & 0.100 & 93.5 & 94.2 & 63.5 & 30.3 & 69.7 \\
  8 & 0.8 & 0.4 & 0.4 & 0.427 & 0.137 & 0.133 & 0.144 & 0.131 & 92.8 & 93.9 & 61.0 & 26.8 & 67.9 \\
  8 & 0.5 & 0.6 & 0.4 & 0.285 & 0.061 & 0.059 & 0.061 & 0.058 & 93.9 & 93.7 & 65.9 & 33.3 & 71.0 \\
  8 & 0.8 & 0.6 & 0.4 & 0.285 & 0.067 & 0.067 & 0.068 & 0.066 & 93.8 & 93.5 & 65.6 & 30.8 & 71.2 \\
\hline \multicolumn{14}{|l|}{$n=1000$}\\ \hline
  4 & 0.5 & 0.4 & 0.0 & 0.000 & 0.028 & 0.028 & 0.029 & 0.028 & 95.8 & 95.5 & 94.9 & 98.4 & 95.1 \\
  4 & 0.8 & 0.4 & 0.0 & 0.000 & 0.034 & 0.033 & 0.034 & 0.033 & 96.0 & 95.9 & 95.1 & 97.3 & 95.1 \\
  4 & 0.5 & 0.6 & 0.0 & 0.000 & 0.017 & 0.017 & 0.018 & 0.017 & 95.2 & 95.2 & 94.9 & 99.1 & 95.0 \\
  4 & 0.8 & 0.6 & 0.0 & 0.000 & 0.020 & 0.019 & 0.020 & 0.019 & 95.4 & 95.4 & 94.6 & 98.7 & 94.7 \\
  8 & 0.5 & 0.4 & 0.0 & 0.000 & 0.026 & 0.025 & 0.026 & 0.025 & 95.0 & 95.0 & 94.8 & 99.1 & 94.8 \\
  8 & 0.8 & 0.4 & 0.0 & 0.000 & 0.030 & 0.028 & 0.030 & 0.028 & 93.9 & 93.8 & 93.3 & 98.4 & 93.3 \\
  8 & 0.5 & 0.6 & 0.0 & 0.000 & 0.016 & 0.016 & 0.016 & 0.016 & 94.5 & 94.5 & 94.3 & 99.4 & 94.1 \\
  8 & 0.8 & 0.6 & 0.0 & 0.000 & 0.017 & 0.017 & 0.017 & 0.017 & 95.8 & 95.8 & 95.1 & 99.2 & 95.3 \\
\hline
  4 & 0.5 & 0.4 & 0.2 & 0.302 & 0.046 & 0.045 & 0.043 & 0.043 & 94.7 & 93.4 & 74.7 & 46.9 & 82.1 \\
  4 & 0.8 & 0.4 & 0.2 & 0.302 & 0.060 & 0.058 & 0.057 & 0.055 & 93.8 & 94.2 & 70.5 & 43.9 & 78.2 \\
  4 & 0.5 & 0.6 & 0.2 & 0.201 & 0.026 & 0.025 & 0.025 & 0.025 & 93.1 & 92.8 & 76.4 & 50.5 & 82.2 \\
  4 & 0.8 & 0.6 & 0.2 & 0.201 & 0.031 & 0.029 & 0.030 & 0.028 & 92.5 & 92.6 & 73.2 & 48.4 & 78.4 \\
  8 & 0.5 & 0.4 & 0.2 & 0.302 & 0.035 & 0.035 & 0.035 & 0.035 & 95.2 & 95.6 & 80.5 & 47.5 & 83.8 \\
  8 & 0.8 & 0.4 & 0.2 & 0.302 & 0.045 & 0.044 & 0.046 & 0.044 & 94.4 & 94.4 & 73.7 & 42.8 & 78.0 \\
  8 & 0.5 & 0.6 & 0.2 & 0.201 & 0.022 & 0.021 & 0.022 & 0.021 & 94.4 & 94.1 & 79.8 & 52.2 & 82.6 \\
  8 & 0.8 & 0.6 & 0.2 & 0.201 & 0.025 & 0.024 & 0.025 & 0.024 & 95.3 & 95.2 & 77.9 & 48.7 & 81.5 \\
\hline
  4 & 0.5 & 0.4 & 0.4 & 0.427 & 0.058 & 0.057 & 0.052 & 0.051 & 94.0 & 94.9 & 60.9 & 33.7 & 74.6 \\
  4 & 0.8 & 0.4 & 0.4 & 0.427 & 0.074 & 0.074 & 0.065 & 0.064 & 93.4 & 94.0 & 55.1 & 32.2 & 74.0 \\
  4 & 0.5 & 0.6 & 0.4 & 0.285 & 0.031 & 0.032 & 0.030 & 0.030 & 95.9 & 95.3 & 64.1 & 35.7 & 77.9 \\
  4 & 0.8 & 0.6 & 0.4 & 0.285 & 0.037 & 0.036 & 0.034 & 0.034 & 94.2 & 94.8 & 61.9 & 34.7 & 75.7 \\
  8 & 0.5 & 0.4 & 0.4 & 0.427 & 0.044 & 0.044 & 0.043 & 0.043 & 93.9 & 93.9 & 65.5 & 30.4 & 71.3 \\
  8 & 0.8 & 0.4 & 0.4 & 0.427 & 0.053 & 0.055 & 0.052 & 0.053 & 95.4 & 95.8 & 62.7 & 27.8 & 70.3 \\
  8 & 0.5 & 0.6 & 0.4 & 0.285 & 0.026 & 0.026 & 0.026 & 0.025 & 95.3 & 94.9 & 66.8 & 33.7 & 72.0 \\
  8 & 0.8 & 0.6 & 0.4 & 0.285 & 0.028 & 0.030 & 0.029 & 0.029 & 95.6 & 94.9 & 68.7 & 31.5 & 74.7 \\
   \hline
\multicolumn{14}{l}{\emph{Note:} $p$ denotes number of measurement
items $Y_{j}$, $\pi_{Y}$ marginal proportion of $Y_{j}=1$, }\\
\multicolumn{14}{l}{
\hspace*{2em}$R_{Y}^{2}$ and $R^{2}_{Z}$ the $R^{2}$ statistics in models for
$Y_{j}$ and $Z$, and $\beta_{1}$ true value of $\beta_{1}$.}\\
\multicolumn{14}{l}{
$\dagger$ Average \% of the variance of 2-step estimator accounted
for by step-2 variance alone,}\\
\multicolumn{14}{l}{
\hspace*{1em} and coverage of 95\% confidence interval if only this variance is
included.}\\
\end{tabular}
}}
\label{sres_49_96_se}
\end{table}

\clearpage
\thispagestyle{empty}
\begin{table}[ht]
\caption{Simulation results for standard error estimates of structural regression
coefficient $\beta_{1}$ of a latent covariate $\eta$ for a conditionally
skew-normally distributed response~$Z$ (Case 2(b)).}
\centering
{\small{
\begin{tabular}{|rrrrr|rrrr|rrr|rr|}
  \hline
  &&&&
  & \multicolumn{4}{c|}{Simulation standard deviation}
  &&&
  & \multicolumn{2}{c|}{2-step}
  \\
  &&&&
  & \multicolumn{4}{c|}{vs.\ mean of est.\ std.\ error}
  & \multicolumn{3}{c|}{Coverage of}
  & \multicolumn{2}{c|}{without}
  \\
  &&&&
  & \multicolumn{2}{|c}{2-step}
  & \multicolumn{2}{c|}{1-step}
  & \multicolumn{3}{c|}{95\% conf.\ interval}
  & \multicolumn{2}{c|}{step-1 var.$^{\dagger}$}
\\
$p$ & $p_{Y}$ & $R^{2}_{Y}$ & $R^{2}_{Z}$ & $\beta_1$
& s.d.\ & m(se) & s.d.\ & m(se) & 2-st.\ & 1-st.\ & 3-st.\ & \%var. &
{\footnotesize{cover.}} \\
\hline \multicolumn{14}{|l|}{$n=200$}\\ \hline
4 & 0.5 & 0.4 & 0.0 & 0.000 & 0.068 & 0.066 & 0.070 & 0.067 & 97.9 & 97.6 & 95.7 & 93.1 & 95.9 \\
  4 & 0.8 & 0.4 & 0.0 & 0.000 & 0.085 & 0.085 & 0.113 & 0.101 & 99.4 & 99.2 & 95.7 & 87.8 & 96.6 \\
  4 & 0.5 & 0.6 & 0.0 & 0.000 & 0.039 & 0.039 & 0.040 & 0.040 & 96.5 & 96.3 & 94.9 & 95.6 & 95.0 \\
  4 & 0.8 & 0.6 & 0.0 & 0.000 & 0.044 & 0.044 & 0.046 & 0.046 & 97.2 & 97.0 & 94.3 & 93.7 & 95.0 \\
  8 & 0.5 & 0.4 & 0.0 & 0.000 & 0.056 & 0.057 & 0.057 & 0.058 & 96.7 & 96.6 & 94.8 & 95.8 & 95.1 \\
  8 & 0.8 & 0.4 & 0.0 & 0.000 & 0.064 & 0.067 & 0.067 & 0.069 & 98.8 & 98.3 & 95.6 & 93.1 & 95.8 \\
  8 & 0.5 & 0.6 & 0.0 & 0.000 & 0.036 & 0.037 & 0.036 & 0.037 & 97.5 & 97.3 & 96.5 & 97.2 & 96.5 \\
  8 & 0.8 & 0.6 & 0.0 & 0.000 & 0.040 & 0.039 & 0.040 & 0.040 & 96.8 & 96.5 & 95.2 & 96.0 & 94.9 \\
\hline
  4 & 0.5 & 0.4 & 0.2 & 0.302 & 0.113 & 0.106 & 0.112 & 0.103 & 93.1 & 92.6 & 72.9 & 46.2 & 80.1 \\
  4 & 0.8 & 0.4 & 0.2 & 0.302 & 0.172 & 0.705 & 0.188 & 0.150 & 91.5 & 93.7 & 69.4 & 42.8 & 78.9 \\
  4 & 0.5 & 0.6 & 0.2 & 0.201 & 0.059 & 0.057 & 0.057 & 0.056 & 92.6 & 94.0 & 75.3 & 49.9 & 80.1 \\
  4 & 0.8 & 0.6 & 0.2 & 0.201 & 0.075 & 0.069 & 0.075 & 0.068 & 93.2 & 93.3 & 73.6 & 47.6 & 79.3 \\
  8 & 0.5 & 0.4 & 0.2 & 0.302 & 0.086 & 0.082 & 0.088 & 0.083 & 93.6 & 93.9 & 76.5 & 47.5 & 79.7 \\
  8 & 0.8 & 0.4 & 0.2 & 0.302 & 0.117 & 0.109 & 0.123 & 0.112 & 93.7 & 95.1 & 73.5 & 41.4 & 77.4 \\
  8 & 0.5 & 0.6 & 0.2 & 0.201 & 0.050 & 0.048 & 0.050 & 0.048 & 93.0 & 93.0 & 79.5 & 52.2 & 81.9 \\
  8 & 0.8 & 0.6 & 0.2 & 0.201 & 0.059 & 0.056 & 0.060 & 0.056 & 93.7 & 93.7 & 78.6 & 47.5 & 80.3 \\
\hline
  4 & 0.5 & 0.4 & 0.4 & 0.427 & 0.142 & 0.132 & 0.132 & 0.119 & 91.1 & 92.6 & 57.6 & 32.7 & 70.2 \\
  4 & 0.8 & 0.4 & 0.4 & 0.427 & 0.193 & 0.192 & 0.199 & 0.170 & 94.4 & 94.8 & 57.8 & 30.5 & 74.1 \\
  4 & 0.5 & 0.6 & 0.4 & 0.285 & 0.072 & 0.071 & 0.068 & 0.068 & 93.2 & 94.7 & 62.5 & 35.6 & 75.2 \\
  4 & 0.8 & 0.6 & 0.4 & 0.285 & 0.089 & 0.085 & 0.087 & 0.080 & 94.5 & 94.0 & 61.9 & 33.7 & 74.3 \\
  8 & 0.5 & 0.4 & 0.4 & 0.427 & 0.112 & 0.101 & 0.112 & 0.100 & 92.7 & 92.9 & 61.5 & 30.1 & 68.0 \\
  8 & 0.8 & 0.4 & 0.4 & 0.427 & 0.143 & 0.136 & 0.146 & 0.133 & 93.4 & 95.3 & 60.5 & 26.3 & 67.6 \\
  8 & 0.5 & 0.6 & 0.4 & 0.285 & 0.062 & 0.059 & 0.063 & 0.058 & 92.9 & 93.1 & 65.7 & 33.1 & 69.9 \\
  8 & 0.8 & 0.6 & 0.4 & 0.285 & 0.069 & 0.068 & 0.069 & 0.067 & 93.4 & 94.1 & 65.3 & 30.1 & 70.9 \\
\hline \multicolumn{14}{|l|}{$n=1000$}\\ \hline
  4 & 0.5 & 0.4 & 0.0 & 0.000 & 0.029 & 0.028 & 0.029 & 0.028 & 95.1 & 95.0 & 94.0 & 98.4 & 94.1 \\
  4 & 0.8 & 0.4 & 0.0 & 0.000 & 0.033 & 0.033 & 0.033 & 0.033 & 95.8 & 95.7 & 94.3 & 97.4 & 94.5 \\
  4 & 0.5 & 0.6 & 0.0 & 0.000 & 0.018 & 0.017 & 0.018 & 0.017 & 95.0 & 95.0 & 94.7 & 99.0 & 94.7 \\
  4 & 0.8 & 0.6 & 0.0 & 0.000 & 0.018 & 0.019 & 0.018 & 0.019 & 96.5 & 96.5 & 96.3 & 98.8 & 96.4 \\
  8 & 0.5 & 0.4 & 0.0 & 0.000 & 0.024 & 0.025 & 0.025 & 0.025 & 96.3 & 96.1 & 95.7 & 99.2 & 95.7 \\
  8 & 0.8 & 0.4 & 0.0 & 0.000 & 0.028 & 0.028 & 0.029 & 0.029 & 95.3 & 95.3 & 95.0 & 98.5 & 94.9 \\
  8 & 0.5 & 0.6 & 0.0 & 0.000 & 0.017 & 0.016 & 0.017 & 0.016 & 94.3 & 94.2 & 94.1 & 99.4 & 94.1 \\
  8 & 0.8 & 0.6 & 0.0 & 0.000 & 0.017 & 0.017 & 0.017 & 0.017 & 96.6 & 96.6 & 96.5 & 99.2 & 96.3 \\
\hline
  4 & 0.5 & 0.4 & 0.2 & 0.302 & 0.045 & 0.045 & 0.043 & 0.043 & 94.8 & 95.8 & 74.9 & 46.8 & 83.3 \\
  4 & 0.8 & 0.4 & 0.2 & 0.302 & 0.063 & 0.060 & 0.060 & 0.057 & 94.5 & 94.5 & 70.9 & 43.2 & 78.4 \\
  4 & 0.5 & 0.6 & 0.2 & 0.201 & 0.026 & 0.025 & 0.025 & 0.025 & 95.1 & 94.5 & 76.2 & 50.5 & 82.0 \\
  4 & 0.8 & 0.6 & 0.2 & 0.201 & 0.032 & 0.030 & 0.031 & 0.029 & 94.6 & 94.9 & 75.1 & 48.2 & 78.9 \\
  8 & 0.5 & 0.4 & 0.2 & 0.302 & 0.035 & 0.036 & 0.036 & 0.036 & 94.6 & 94.6 & 79.5 & 47.6 & 82.1 \\
  8 & 0.8 & 0.4 & 0.2 & 0.302 & 0.047 & 0.045 & 0.049 & 0.045 & 94.8 & 94.5 & 74.1 & 42.3 & 77.9 \\
  8 & 0.5 & 0.6 & 0.2 & 0.201 & 0.022 & 0.021 & 0.022 & 0.021 & 94.1 & 93.5 & 79.2 & 52.0 & 81.8 \\
  8 & 0.8 & 0.6 & 0.2 & 0.201 & 0.025 & 0.024 & 0.025 & 0.024 & 94.9 & 94.6 & 79.8 & 48.3 & 81.4 \\
\hline
  4 & 0.5 & 0.4 & 0.4 & 0.427 & 0.059 & 0.057 & 0.053 & 0.051 & 95.5 & 94.8 & 59.1 & 33.7 & 74.5 \\
  4 & 0.8 & 0.4 & 0.4 & 0.427 & 0.081 & 0.078 & 0.071 & 0.067 & 95.7 & 95.3 & 56.7 & 31.8 & 72.1 \\
  4 & 0.5 & 0.6 & 0.4 & 0.285 & 0.032 & 0.032 & 0.031 & 0.030 & 95.0 & 94.3 & 61.0 & 35.8 & 74.5 \\
  4 & 0.8 & 0.6 & 0.4 & 0.285 & 0.038 & 0.038 & 0.037 & 0.035 & 94.3 & 93.3 & 64.3 & 34.0 & 75.7 \\
  8 & 0.5 & 0.4 & 0.4 & 0.427 & 0.044 & 0.044 & 0.042 & 0.043 & 94.1 & 94.7 & 67.6 & 30.4 & 72.4 \\
  8 & 0.8 & 0.4 & 0.4 & 0.427 & 0.057 & 0.057 & 0.055 & 0.054 & 95.0 & 95.5 & 61.8 & 27.0 & 69.1 \\
  8 & 0.5 & 0.6 & 0.4 & 0.285 & 0.025 & 0.026 & 0.025 & 0.025 & 94.8 & 94.6 & 69.7 & 33.7 & 76.5 \\
  8 & 0.8 & 0.6 & 0.4 & 0.285 & 0.030 & 0.030 & 0.030 & 0.029 & 95.5 & 94.1 & 69.5 & 30.8 & 74.5 \\
   \hline
\multicolumn{14}{l}{\emph{Note:} $p$ denotes number of measurement
items $Y_{j}$, $\pi_{Y}$ marginal proportion of $Y_{j}=1$, }\\
\multicolumn{14}{l}{
\hspace*{2em}$R_{Y}^{2}$ and $R^{2}_{Z}$ the $R^{2}$ statistics in models for
$Y_{j}$ and $Z$, and $\beta_{1}$ true value of $\beta_{1}$.}\\
\multicolumn{14}{l}{
$\dagger$ Average \% of the variance of 2-step estimator accounted
for by step-2 variance alone,}\\
\multicolumn{14}{l}{
\hspace*{1em} and coverage of 95\% confidence interval if only this variance is
included.}\\
\end{tabular}
}}
\label{sres_133_180_se}
\end{table}

\clearpage
\thispagestyle{empty}
\begin{table}[ht]
\caption{Simulation results for standard error estimates of structural regression
coefficient $\beta_{1}$ of a latent covariate $\eta_{1}$ for a conditionally
normally distributed latent response~$\eta_{2}$ (Case 3).}
\centering
{\small{
\begin{tabular}{|rrrrr|rrrr|rrr|rr|}
  \hline
  &&&&
  & \multicolumn{4}{c|}{Simulation standard deviation}
  &&&
  & \multicolumn{2}{c|}{2-step}
  \\
  &&&&
  & \multicolumn{4}{c|}{vs.\ mean of est.\ std.\ error}
  & \multicolumn{3}{c|}{Coverage of}
  & \multicolumn{2}{c|}{without}
  \\
  &&&&
  & \multicolumn{2}{|c}{2-step}
  & \multicolumn{2}{c|}{1-step}
  & \multicolumn{3}{c|}{95\% conf.\ interval}
  & \multicolumn{2}{c|}{step-1 var.$^{\dagger}$}
\\
$p$ & $p_{Y}$ & $R^{2}_{Y}$ & $R^{2}_{\eta}$ & $\beta_1$
& s.d.\ & m(se) & s.d.\ & m(se) & 2-st.\ & 1-st.\ & 3-st.\ & \%var. &
{\footnotesize{cover.}} \\
\hline \multicolumn{14}{|l|}{$n=200$}\\ \hline
4 & 0.5 & 0.4 & 0.0 & 0.000 & 0.133 & 0.137 & 0.141 & 0.138 & 99.5 & 99.3 & 95.6 & 87.3 & 96.0 \\
  4 & 0.8 & 0.4 & 0.0 & 0.000 & 0.240 & 0.224 & 0.276 & 0.235 & 99.9 & 100.0 & 94.7 & 78.0 & 96.0 \\
  4 & 0.5 & 0.6 & 0.0 & 0.000 & 0.115 & 0.114 & 0.116 & 0.114 & 98.9 & 98.5 & 94.3 & 91.4 & 94.9 \\
  4 & 0.8 & 0.6 & 0.0 & 0.000 & 0.153 & 0.146 & 0.154 & 0.145 & 99.2 & 98.6 & 94.2 & 88.2 & 95.1 \\
  8 & 0.5 & 0.4 & 0.0 & 0.000 & 0.105 & 0.105 & 0.105 & 0.102 & 98.5 & 98.5 & 95.7 & 92.5 & 95.8 \\
  8 & 0.8 & 0.4 & 0.0 & 0.000 & 0.147 & 0.143 & 0.151 & 0.143 & 99.3 & 99.0 & 95.0 & 87.1 & 95.9 \\
  8 & 0.5 & 0.6 & 0.0 & 0.000 & 0.097 & 0.095 & 0.095 & 0.090 & 98.4 & 98.1 & 95.5 & 94.3 & 95.7 \\
  8 & 0.8 & 0.6 & 0.0 & 0.000 & 0.115 & 0.112 & 0.116 & 0.112 & 99.1 & 99.0 & 94.5 & 92.1 & 95.0 \\
\hline
  4 & 0.5 & 0.4 & 0.2 & 0.447 & 0.236 & 0.223 & 0.235 & 0.217 & 90.5 & 90.5 & 36.8 & 41.3 & 77.9 \\
  4 & 0.8 & 0.4 & 0.2 & 0.447 & 0.403 & 0.333 & 0.378 & 0.331 & 88.5 & 90.3 & 28.2 & 41.6 & 75.4 \\
  4 & 0.5 & 0.6 & 0.2 & 0.447 & 0.181 & 0.183 & 0.187 & 0.178 & 92.8 & 92.7 & 48.6 & 43.3 & 82.8 \\
  4 & 0.8 & 0.6 & 0.2 & 0.447 & 0.256 & 0.230 & 0.241 & 0.219 & 91.5 & 91.9 & 38.4 & 42.8 & 79.3 \\
  8 & 0.5 & 0.4 & 0.2 & 0.447 & 0.176 & 0.162 & 0.188 & 0.160 & 90.7 & 89.3 & 48.7 & 39.7 & 77.0 \\
  8 & 0.8 & 0.4 & 0.2 & 0.447 & 0.248 & 0.226 & 0.247 & 0.226 & 91.1 & 91.8 & 36.5 & 39.7 & 75.1 \\
  8 & 0.5 & 0.6 & 0.2 & 0.447 & 0.143 & 0.143 & 0.159 & 0.137 & 92.1 & 90.0 & 57.4 & 42.3 & 78.9 \\
  8 & 0.8 & 0.6 & 0.2 & 0.447 & 0.181 & 0.175 & 0.179 & 0.171 & 92.1 & 92.6 & 51.4 & 40.9 & 78.5 \\
\hline
  4 & 0.5 & 0.4 & 0.4 & 0.632 & 0.308 & 0.295 & 0.281 & 0.263 & 91.9 & 92.7 & 28.6 & 27.3 & 72.2 \\
  4 & 0.8 & 0.4 & 0.4 & 0.632 & 0.607 & 0.439 & 0.450 & 0.385 & 89.7 & 92.8 & 19.9 & 27.9 & 69.6 \\
  4 & 0.5 & 0.6 & 0.4 & 0.632 & 0.271 & 0.254 & 0.248 & 0.229 & 94.0 & 94.6 & 36.7 & 28.8 & 73.8 \\
  4 & 0.8 & 0.6 & 0.4 & 0.632 & 0.324 & 0.316 & 0.279 & 0.267 & 93.2 & 94.5 & 29.5 & 28.3 & 71.2 \\
  8 & 0.5 & 0.4 & 0.4 & 0.632 & 0.212 & 0.208 & 0.221 & 0.203 & 92.5 & 92.5 & 40.7 & 24.9 & 66.1 \\
  8 & 0.8 & 0.4 & 0.4 & 0.632 & 0.342 & 0.284 & 0.342 & 0.280 & 91.0 & 92.7 & 27.6 & 24.6 & 65.1 \\
  8 & 0.5 & 0.6 & 0.4 & 0.632 & 0.193 & 0.189 & 0.227 & 0.175 & 94.6 & 90.2 & 47.7 & 25.9 & 66.8 \\
  8 & 0.8 & 0.6 & 0.4 & 0.632 & 0.221 & 0.218 & 0.216 & 0.211 & 93.2 & 94.1 & 38.4 & 26.2 & 68.5 \\
\hline \multicolumn{14}{|l|}{$n=1000$}\\ \hline
  4 & 0.5 & 0.4 & 0.0 & 0.000 & 0.053 & 0.054 & 0.054 & 0.054 & 95.9 & 95.7 & 95.1 & 97.1 & 95.1 \\
  4 & 0.8 & 0.4 & 0.0 & 0.000 & 0.072 & 0.076 & 0.073 & 0.076 & 97.9 & 97.2 & 96.3 & 95.1 & 96.3 \\
  4 & 0.5 & 0.6 & 0.0 & 0.000 & 0.045 & 0.046 & 0.045 & 0.046 & 96.5 & 96.4 & 95.2 & 98.3 & 95.4 \\
  4 & 0.8 & 0.6 & 0.0 & 0.000 & 0.054 & 0.056 & 0.055 & 0.056 & 96.8 & 96.8 & 95.9 & 97.6 & 96.2 \\
  8 & 0.5 & 0.4 & 0.0 & 0.000 & 0.045 & 0.044 & 0.045 & 0.044 & 95.3 & 95.2 & 94.5 & 98.2 & 94.5 \\
  8 & 0.8 & 0.4 & 0.0 & 0.000 & 0.059 & 0.056 & 0.059 & 0.056 & 95.0 & 94.8 & 93.7 & 96.9 & 94.0 \\
  8 & 0.5 & 0.6 & 0.0 & 0.000 & 0.040 & 0.040 & 0.040 & 0.040 & 96.1 & 95.9 & 95.2 & 98.8 & 95.2 \\
  8 & 0.8 & 0.6 & 0.0 & 0.000 & 0.045 & 0.046 & 0.046 & 0.046 & 96.3 & 96.2 & 95.1 & 98.4 & 95.3 \\
\hline
  4 & 0.5 & 0.4 & 0.2 & 0.447 & 0.092 & 0.091 & 0.089 & 0.089 & 94.4 & 95.4 & 5.9 & 41.6 & 79.5 \\
  4 & 0.8 & 0.4 & 0.2 & 0.447 & 0.124 & 0.123 & 0.122 & 0.120 & 93.6 & 94.1 & 3.2 & 42.6 & 80.2 \\
  4 & 0.5 & 0.6 & 0.2 & 0.447 & 0.075 & 0.076 & 0.074 & 0.074 & 94.8 & 94.6 & 12.6 & 43.8 & 79.3 \\
  4 & 0.8 & 0.6 & 0.2 & 0.447 & 0.090 & 0.090 & 0.089 & 0.089 & 96.0 & 95.8 & 8.4 & 43.8 & 81.3 \\
  8 & 0.5 & 0.4 & 0.2 & 0.447 & 0.073 & 0.070 & 0.073 & 0.069 & 93.9 & 94.6 & 18.8 & 39.3 & 77.7 \\
  8 & 0.8 & 0.4 & 0.2 & 0.447 & 0.093 & 0.091 & 0.092 & 0.091 & 93.9 & 94.6 & 8.5 & 38.7 & 76.6 \\
  8 & 0.5 & 0.6 & 0.2 & 0.447 & 0.062 & 0.062 & 0.062 & 0.062 & 94.6 & 94.7 & 28.6 & 41.9 & 79.2 \\
  8 & 0.8 & 0.6 & 0.2 & 0.447 & 0.074 & 0.073 & 0.073 & 0.072 & 95.1 & 94.8 & 17.9 & 41.4 & 80.1 \\
\hline
  4 & 0.5 & 0.4 & 0.4 & 0.632 & 0.118 & 0.118 & 0.112 & 0.109 & 94.3 & 94.9 & 2.3 & 28.9 & 70.9 \\
  4 & 0.8 & 0.4 & 0.4 & 0.632 & 0.156 & 0.155 & 0.149 & 0.143 & 93.4 & 93.8 & 1.6 & 29.1 & 70.4 \\
  4 & 0.5 & 0.6 & 0.4 & 0.632 & 0.093 & 0.097 & 0.089 & 0.092 & 95.7 & 95.9 & 6.8 & 30.5 & 73.9 \\
  4 & 0.8 & 0.6 & 0.4 & 0.632 & 0.116 & 0.115 & 0.109 & 0.108 & 94.7 & 94.7 & 4.0 & 30.4 & 73.9 \\
  8 & 0.5 & 0.4 & 0.4 & 0.632 & 0.088 & 0.089 & 0.086 & 0.087 & 95.5 & 96.2 & 7.9 & 24.6 & 68.0 \\
  8 & 0.8 & 0.4 & 0.4 & 0.632 & 0.114 & 0.114 & 0.110 & 0.111 & 93.3 & 94.5 & 2.6 & 24.5 & 65.3 \\
  8 & 0.5 & 0.6 & 0.4 & 0.632 & 0.080 & 0.080 & 0.078 & 0.077 & 95.1 & 94.6 & 19.8 & 26.3 & 69.2 \\
  8 & 0.8 & 0.6 & 0.4 & 0.632 & 0.088 & 0.092 & 0.086 & 0.089 & 95.5 & 95.8 & 10.5 & 26.3 & 70.8 \\
   \hline
\multicolumn{14}{l}{\emph{Note:} $p$ denotes number of measurement
items $Y_{j}$, $\pi_{Y}$ marginal proportion of $Y_{j}=1$, }\\
\multicolumn{14}{l}{
\hspace*{2em}$R_{Y}^{2}$ and $R^{2}_{\eta}$ the $R^{2}$ statistics in models for
$Y_{j}$ and $\eta_{2}$, and $\beta_{1}$ true value of $\beta_{1}$.}\\
\multicolumn{14}{l}{
$\dagger$ Average \% of the variance of 2-step estimator accounted
for by step-2 variance alone,}\\
\multicolumn{14}{l}{
\hspace*{1em} and coverage of 95\% confidence interval if only this variance is
included.}\\
\end{tabular}
}}
\label{sres_301_348_se}
\end{table}

\clearpage
\restoregeometry
\setcounter{table}{0}
\renewcommand{\thetable}{B.\arabic{table}}
\begin{table}[ht]
\caption{
Estimated coefficients of structural models for extrinsic work values given
different sets of covariates (with standard errors in parentheses), estimated for EVS2017 data for Netherlands. The table shows two-step
(``2-st.''), one-step (``1-st.'') and naive three-step (``3-st.'') estimates.
This example is considered in Section 5 of the paper, and the
coefficients of the dummy variable for male gender of respondent are
also shown in Table 3 there.
}

\vspace*{1ex}
\centering
{\small{
\begin{tabular}{|l|rrr|rrr|rrr|}
  \hline
 &
\multicolumn{3}{|c|}{Model (1)} &
\multicolumn{3}{|c|}{Model (2)} &
\multicolumn{3}{|c|}{Model (3)} \\
&
2-st.\ & 1-st.\ & 3-st. &
2-st.\ & 1-st.\ & 3-st. &
2-st.\ & 1-st.\ & 3-st. \\ \hline
Intercept & 1.403 & 1.398 & 1.699 & 1.385 & 1.404 & 1.694 & 1.155 & 1.160 & 1.518 \\
    & (0.113) & (0.116) & (0.057) & (0.231) & (0.238) & (0.172) &
    (0.287) & (0.294) &    (0.224) \\[1ex]
Male
       & -0.042 & -0.033 & -0.033 & 0.025 & 0.022 & 0.024 & 0.018 & 0.009 & 0.018 \\
      & (0.103) & (0.104) & (0.082) & (0.101) & (0.104) & (0.080) &
      (0.102) & (0.104) & (0.081) \\[1ex]
\multicolumn{4}{|l|}{Age (vs.\ 30--49)} & & & & & & \\
\hspace*{2em}15--29
&  &  &  & 0.472 & 0.489 & 0.380 & 0.448 & 0.462 & 0.364 \\
      &  &  &  & (0.207) & (0.214) & (0.166) & (0.208) & (0.213) &
      (0.167) \\[.5ex]
      \hspace*{2em}50--
          &  &  &  & -0.391 & -0.408 & -0.326 & -0.191 & -0.196 & -0.162 \\
      &  &  &  & (0.150) & (0.155) & (0.119) & (0.152) & (0.155) &
      (0.123) \\[1ex]
Has partner
   &  &  &  & -0.019 & -0.022 & -0.017 & -0.052 & -0.056 & -0.048 \\
      &  &  &  & (0.163) & (0.169) & (0.133) & (0.163) & (0.167) &
      (0.133) \\[1ex]
\multicolumn{7}{|l|}{Age of youngest person in household (vs.\ older)} &&&\\
\hspace*{2em}0--5
&  &  &  & 0.542 & 0.570 & 0.437 & 0.477 & 0.499 & 0.390 \\
      &  &  &  & (0.194) & (0.201) & (0.151) & (0.191) & (0.195) &
      (0.150) \\[.5ex]
\hspace*{2em}6--17
   &  &  &  & 0.367 & 0.381 & 0.299 & 0.272 & 0.279 & 0.225 \\
      &  &  &  & (0.150) & (0.156) & (0.120) & (0.149) & (0.153) &
      (0.121) \\[1ex]
\multicolumn{4}{|l|}{Education level (vs.\ upper secondary)} &&&&&&\\
\hspace*{2em}  Lower
  &  &  &  &  &  &  & -0.104 & -0.105 & -0.080 \\
      &  &  &  &  &  &  & (0.144) & (0.147) & (0.117) \\[.5ex]
\hspace*{2em}  Higher
  &  &  &  &  &  &  & 0.034 & 0.031 & 0.039 \\
      &  &  &  &  &  &  & (0.133) & (0.135) & (0.107) \\[1ex]
\multicolumn{7}{|l|}{Occupation-based social class (vs.\
Managerial/service contract)} &&&\\
\hspace*{1em}Mixed
   &  &  &  &  &  &  & -0.074 & -0.074 & -0.069 \\
      &  &  &  &  &  &  & (0.144) & (0.147) & (0.116) \\[.5ex]
\hspace*{1em}Self-empl.\
 &  &  &  &  &  &  & -0.350 & -0.358 & -0.273 \\
       &  &  &  &  &  &  & (0.243) & (0.248) & (0.192) \\[.5ex]
\hspace*{1em}Labour
  &  &  &  &  &  &  & 0.231 & 0.229 & 0.185 \\
       &  &  &  &  &  &  & (0.144) & (0.147) & (0.116) \\[1ex]
\multicolumn{4}{|l|}{Currently in paid employment}
&&&& 0.449 & 0.468 & 0.358 \\
       &  &  &  &  &  &  & (0.123) & (0.125) & (0.092) \\[1ex]
\multicolumn{7}{|l|}{Parents' education level (highest, vs.\ upper secondary)} &&&\\
\hspace*{2em}Lower
   &  &  &  &  &  &  & -0.122 & -0.124 & -0.108 \\
       &  &  &  &  &  &  & (0.146) & (0.149) & (0.118) \\[.5ex]
\hspace*{2em}Higher
   &  &  &  &  &  &  & -0.092 & -0.099 & -0.092 \\
       &  &  &  &  &  &  & (0.159) & (0.162) & (0.128) \\
   \hline
\end{tabular}
}}
\label{NL_mods_ext}
\end{table}

\clearpage
\begin{table}[ht]
\caption{
Estimated coefficients of structural models for intrinsic work values given
different sets of covariates (with standard errors in parentheses), estimated for EVS2017 data for Netherlands. The table shows two-step
(``2-st.''), one-step (``1-st.'') and naive three-step (``3-st.'') estimates.
This example is considered in Section 5 of the paper, and the
coefficients of the dummy variable for male gender of respondent are
also shown in Table 3 there.
}

\vspace*{1ex}
\centering
{\small{
\begin{tabular}{|l|rrr|rrr|rrr|}
  \hline
 &
\multicolumn{3}{|c|}{Model (1)} &
\multicolumn{3}{|c|}{Model (2)} &
\multicolumn{3}{|c|}{Model (3)} \\
&
2-st.\ & 1-st.\ & 3-st. &
2-st.\ & 1-st.\ & 3-st. &
2-st.\ & 1-st.\ & 3-st. \\ \hline
Intercept & 0.624 & 0.611 & 0.955 & 0.474 & 0.449 & 0.847 & 1.311 & 1.087 & 1.480 \\
    & (0.190) & (0.165) & (0.099) & (0.415) & (0.376) & (0.305) &
    (0.569) & (0.430) &    (0.387) \\[1ex]
  Male
  & 0.607 & 0.603 & 0.460 & 0.657 & 0.638 & 0.500 & 0.516 & 0.506 & 0.400 \\
      & (0.193) & (0.186) & (0.142) & (0.194) & (0.176) & (0.142) &
      (0.188) & (0.151) &      (0.139) \\[1ex]
\multicolumn{4}{|l|}{Age (vs.\ 30--49)} & & & & & & \\
\hspace*{2em}15--29
 &  &  &  & 0.783 & 0.749 & 0.573 & 0.568 & 0.507 & 0.417 \\
      &  &  &  & (0.405) & (0.365) & (0.293) & (0.390) & (0.312) &
      (0.288) \\[.5ex]
      \hspace*{2em}50--
  &  &  &  & -0.584 & -0.545 & -0.452 & -0.183 & -0.162 & -0.159 \\
      &  &  &  & (0.293) & (0.261) & (0.210) & (0.283) & (0.227) &
      (0.213) \\[1ex]
Has partner
 &  &  &  & 0.412 & 0.371 & 0.311 & 0.168 & 0.108 & 0.132 \\
      &  &  &  & (0.328) & (0.293) & (0.235) & (0.311) & (0.248) &
      (0.229) \\[1ex]
\multicolumn{7}{|l|}{Age of youngest person in household (vs.\ older)} &&&\\
\hspace*{2em}0--5
    & &  &  & -0.081 & -0.072 & -0.067 & -0.276 & -0.221 & -0.228 \\
      &  &  &  & (0.355) & (0.324) & (0.267) & (0.349) & (0.278) &
      (0.259) \\[.5ex]
\hspace*{2em}6--17
   &  &  &  & 0.028 & 0.010 & 0.029 & -0.132 & -0.130 & -0.098 \\
      &  &  &  & (0.282) & (0.257) & (0.213) & (0.276) & (0.221) &
      (0.208) \\[1ex]
\multicolumn{4}{|l|}{Education level (vs.\ upper secondary)} &&&&&&\\
\hspace*{2em}  Lower
   &  &  &  &  &  &  & -0.459 & -0.374 & -0.378 \\
      &  &  &  &  &  &  & (0.279) & (0.218) & (0.202) \\[.5ex]
\hspace*{2em}  Higher
   &  &  &  &  &  &  & 0.446 & 0.365 & 0.342 \\
      &  &  &  &  &  &  & (0.257) & (0.201) & (0.185) \\[1ex]
\multicolumn{7}{|l|}{Occupation-based social class (vs.\
Managerial/service contract)} &&&\\
\hspace*{1em}Mixed
   &  &  &  &  &  &  & -0.588 & -0.503 & -0.454 \\
      &  &  &  &  &  &  & (0.279) & (0.216) & (0.200) \\[.5ex]
\hspace*{1em}Self-empl.\
   &  &  &  &  &  &  & -0.452 & -0.402 & -0.337 \\
       &  &  &  &  &  &  & (0.444) & (0.355) & (0.332) \\[.5ex]
\hspace*{1em}Labour
   &  &  &  &  &  &  & -0.857 & -0.748 & -0.652 \\
       &  &  &  &  &  &  & (0.289) & (0.220) & (0.200) \\[1ex]
\multicolumn{4}{|l|}{Currently in paid employment}
  &  &  &  & 0.304 & 0.252 & 0.237 \\
       &  &  &  &  &  &  & (0.219) & (0.172) & (0.160) \\[1ex]
\multicolumn{7}{|l|}{Parents' education level (highest, vs.\ upper secondary)} &&&\\
\hspace*{2em}Lower
   &  &  &  &  &  &  & -1.009 & -0.796 & -0.758 \\
       &  &  &  &  &  &  & (0.344) & (0.239) & (0.203) \\[.5ex]
\hspace*{2em}Higher
   &  &  &  &  &  &  & -0.238 & -0.180 & -0.167 \\
       &  &  &  &  &  &  & (0.306) & (0.242) & (0.221) \\
   \hline
\end{tabular}
}}
\label{NL_mods_int}
\end{table}

\clearpage

\begin{table}[ht]
\caption{
Estimated coefficients of gender (as dummy variable for men) in
structural models for extrinsic work values, for each country
in EVS2017 data. The table shows two-step, one-step and naive three-step estimates.
The models also include as covariates the respondent's age, education,
occupation-based social class and current employment status (working
vs.\ not).
All the coefficients of the structrural model are estimated separately
for each country, but the measurement model for work values is the same
in all countries.
The coefficients and confidence intervals from two-step estimation are
also also shown in Figure 1 in Section 5 of the paper
(with the further
standardisation that they are expressed on a scale where the residual
variance of extrinsic values for the Netherlands is 1).
}

\vspace*{2ex}
\centering
{\small{
\begin{tabular}{|llrrrrrr|}
  \hline
country & $n$&\hspace*{2em} 2-step & (s.e.) & 1-step & (s.e.) & 3-step & (s.e.) \\
  \hline
Albania    &1054& -0.180 & (0.276) & -0.195 & (0.320) & -0.016 & (0.069) \\
  Armenia  &1157& -0.133 & (0.076) & -0.149 & (0.089) & -0.124 & (0.074) \\
  Austria  &1450& -0.066 & (0.089) & -0.045 & (0.103) & -0.056 & (0.079) \\
Azerbaijan &1205& -0.300 & (0.099) & -0.346 & (0.116) & -0.240 & (0.081) \\
  Belarus  &1452& -0.092 & (0.109) & -0.096 & (0.127) & -0.069 & (0.078) \\
  Bosnia \& Herzegovina
           &1002& -0.022 & (0.123) & -0.023 & (0.142) & -0.019 & (0.098) \\
  Bulgaria &1327& -0.085 & (0.137) & -0.091 & (0.156) & -0.046 & (0.081) \\
  Croatia  &1328& -0.313 & (0.117) & -0.352 & (0.135) & -0.214 & (0.086) \\
  Czechia  &1546& -0.108 & (0.114) & -0.113 & (0.131) & -0.078 & (0.076) \\
  Denmark  &3024& -0.143 & (0.087) & -0.118 & (0.100) & -0.106 & (0.063) \\
  Estonia  &1272& -0.088 & (0.088) & -0.081 & (0.104) & -0.076 & (0.084) \\
  Finland  &1035& -0.130 & (0.127) & -0.111 & (0.146) & -0.104 & (0.099) \\
  France   &1729& -0.035 & (0.095) & -0.027 & (0.106) & -0.026 & (0.078) \\
  Georgia  &1756& 0.032  & (0.077) & 0.051  & (0.090) & 0.035  & (0.072) \\
  Germany  &1890& -0.159 & (0.095) & -0.155 & (0.109) & -0.128 & (0.076) \\
Great Britain
           &1673& -0.154 & (0.110) & -0.154 & (0.127) & -0.105 & (0.082) \\
  Hungary  &1360& -0.041 & (0.100) & -0.041 & (0.117) & -0.023 & (0.078) \\
  Iceland  &1547& -0.156 & (0.075) & -0.172 & (0.089) & -0.139 & (0.072) \\
  Italy    &1782& -0.141 & (0.068) & -0.134 & (0.077) & -0.123 & (0.064) \\
  Latvia   &1204& -0.389 & (0.130) & -0.441 & (0.149) & -0.292 & (0.091) \\
 Lithuania &1232& -0.274 & (0.138) & -0.313 & (0.159) & -0.142 & (0.069) \\
Montenegro &\hspace*{.5em}716& -0.283 & (0.135) & -0.320 & (0.153) & -0.237 & (0.112) \\
Netherlands
           &2068& -0.003 & (0.108) & 0.032  & (0.123) & -0.010 & (0.077) \\
North Macedonia
           &\hspace*{.5em}748& 0.017  & (0.092) & 0.022  & (0.109) & 0.014  & (0.081) \\
  Norway   &1079& -0.322 & (0.113) & -0.334 & (0.129) & -0.266 & (0.095) \\
  Poland   &1200& -0.213 & (0.087) & -0.244 & (0.098) & -0.202 & (0.083) \\
  Portugal &1092& 0.169  & (0.226) & 0.215  & (0.258) & 0.077  & (0.097) \\
  Romania  &1098& -0.205 & (0.135) & -0.232 & (0.158) & -0.139 & (0.080) \\
  Russia   &1582& -0.397 & (0.086) & -0.435 & (0.097) & -0.363 & (0.078) \\
  Serbia   &1168& 0.117  & (0.121) & 0.133  & (0.139) & 0.082  & (0.085) \\
  Slovakia &1280& -0.493 & (0.118) & -0.560 & (0.134) & -0.391 & (0.093) \\
  Slovenia &\hspace*{.5em}954& -0.212 & (0.117) & -0.247 & (0.140) & -0.169 & (0.088) \\
  Spain    &1002&  0.007 & (0.140) & 0.020  & (0.164) & -0.003 & (0.084) \\
  Sweden   &1053& -0.307 & (0.146) & -0.340 & (0.169) & -0.225 & (0.106) \\
Switzerland&2841& 0.023  & (0.072) & 0.066  & (0.082) & 0.029  & (0.057) \\
  Ukraine  &1455& -0.423 & (0.125) & -0.476 & (0.143) & -0.253 & (0.073) \\
   \hline
\end{tabular}
}}
\label{country_mods_ext}
\end{table}


\clearpage

\begin{table}[ht]
\caption{
Estimated coefficients of gender (as dummy variable for men) in
structural models for intrinsic work values, for each country
in EVS2017 data. The table shows two-step, one-step and naive three-step estimates.
The models also include as covariates the respondent's age, education,
occupation-based social class and current employment status (working
vs.\ not).
All the coefficients of the structural model are estimated separately
for each country, but the measurement model for work values is the same
in all countries.
The coefficients and confidence intervals from two-step estimation are
also also shown in Figure 1 in Section 5 of the paper
(with the further
standardisation that they are expressed on a scale where the residual
variance of intrinsic values for the Netherlands is 1).
}

\vspace*{2ex}
\centering
{\small{
\begin{tabular}{|llrrrrrr|}
  \hline
country & $n$&\hspace*{2em} 2-step & (s.e.) & 1-step & (s.e.) & 3-step & (s.e.) \\
  \hline
Albania    &1054&
0.152 & (0.237) & 0.141 & (0.235) & 0.092 & (0.122) \\
  Armenia  &1157&
-0.051 & (0.201) & -0.052 & (0.196) & -0.039 & (0.142) \\
  Austria  &1450&
  0.010 & (0.122) & 0.014 & (0.119) & 0.006 & (0.119) \\
Azerbaijan &1205&
  0.010 & (0.229) & 0.003 & (0.222) & 0.024 & (0.160) \\
  Belarus  &1452&
  0.235 & (0.191) & 0.239 & (0.184) & 0.164 & (0.141) \\
Bosnia \& Herzegovina &1002&
  -0.269 & (0.161) & -0.268 & (0.153) & -0.229 & (0.145) \\
  Bulgaria &1327&
  0.277 & (0.249) & 0.269 & (0.242) & 0.185 & (0.137) \\
  Croatia  &1328&
  -0.204 & (0.184) & -0.204 & (0.178) & -0.145 & (0.137) \\
  Czechia  &1546&
  0.463 & (0.196) & 0.454 & (0.190) & 0.295 & (0.130) \\
  Denmark  &3024&
  0.004 & (0.095) & -0.007 & (0.092) & 0.017 & (0.079) \\
  Estonia  &1272&
  0.474 & (0.175) & 0.449 & (0.169) & 0.407 & (0.146) \\
  Finland  &1035&
  0.105 & (0.147) & 0.096 & (0.143) & 0.112 & (0.136) \\
  France   &1729&
  -0.024 & (0.116) & -0.035 & (0.113) & -0.026 & (0.107) \\
  Georgia  &1756&
  0.854 & (0.188) & 0.813 & (0.181) & 0.558 & (0.128) \\
  Germany  &1890&
  0.287 & (0.116) & 0.284 & (0.114) & 0.252 & (0.101) \\
Great Britain &1673&
  0.087 & (0.125) & 0.077 & (0.121) & 0.076 & (0.110) \\
  Hungary  &1360&
  0.268 & (0.186) & 0.253 & (0.181) & 0.203 & (0.134) \\
  Iceland  &1547&
  -0.155 & (0.121) & -0.165 & (0.118) & -0.137 & (0.109) \\
  Italy    &1782&
  0.214 & (0.081) & 0.192 & (0.077) & 0.248 & (0.093) \\
  Latvia   &1204&
  0.049 & (0.228) & 0.058 & (0.219) & 0.033 & (0.156) \\
 Lithuania &1232&
  -0.033 & (0.312) & -0.036 & (0.304) & -0.047 & (0.147) \\
Montenegro &\hspace*{.5em}716&
  -1.196 & (0.475) & -1.146 & (0.454) & -0.556 & (0.213) \\
Netherlands &2068&
  0.366 & (0.117) & 0.355 & (0.114) & 0.334 & (0.105) \\
North Macedonia &\hspace*{.5em}748&
  -0.063 & (0.260) & -0.068 & (0.254) & -0.007 & (0.148) \\
  Norway   &1079&
  -0.190 & (0.130) & -0.196 & (0.126) & -0.174 & (0.126) \\
  Poland   &1200&
  0.101 & (0.131) & 0.084 & (0.125) & 0.098 & (0.131) \\
  Portugal &1092&
  0.501 & (0.311) & 0.478 & (0.302) & 0.239 & (0.147) \\
  Romania  &1098&
  -0.865 & (0.277) & -0.849 & (0.271) & -0.462 & (0.134) \\
  Russia   &1582&
  0.054 & (0.135) & 0.048 & (0.128) & 0.039 & (0.107) \\
  Serbia   &1168&
  -0.101 & (0.238) & -0.101 & (0.230) & -0.078 & (0.156) \\
  Slovakia &1280&
  -0.117 & (0.180) & -0.105 & (0.173) & -0.104 & (0.145) \\
  Slovenia &\hspace*{.5em}954&
  0.262 & (0.320) & 0.253 & (0.312) & 0.103 & (0.106) \\
  Spain    &1002&
  0.147 & (0.285) & 0.134 & (0.277) & 0.085 & (0.159) \\
  Sweden   &1053&
  -0.681 & (0.156) & -0.689 & (0.153) & -0.495 & (0.116) \\
Switzerland&2841&
  0.084 & (0.078) & 0.077 & (0.076) & 0.086 & (0.077) \\
  Ukraine  &1455&
  0.266 & (0.220) & 0.262 & (0.213) & 0.169 & (0.138) \\
   \hline
\end{tabular}
}}
\label{country_mods_int}
\end{table}


\end{document}

