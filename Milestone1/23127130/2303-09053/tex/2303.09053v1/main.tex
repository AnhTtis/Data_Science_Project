\documentclass[journal]{IEEEtran}
\usepackage{amsmath, amssymb,amsthm}
\usepackage{graphicx, float, adjustbox, multirow, array}
\usepackage[caption=false, font=footnotesize, labelfont=sf,textfont=sf]{subfig}
\usepackage{authblk}
\usepackage{enumerate,enumitem, hyperref, siunitx}
\usepackage[numbers,sort,square,compress]{natbib}
\renewcommand{\bibfont}{\footnotesize}

\usepackage{cuted}
\usepackage{algorithm,algpseudocode}

\newcommand{\MB}[1]{\mathbb{#1}}
\newcommand{\MR}[1]{\mathrm{#1}}
\newcommand{\MC}[1]{\mathcal{#1}}
\newcommand{\norm}[1]{\left\lVert#1\right\rVert}
\newcommand{\sinc}[1]{\text{ sinc}(#1)}
\newcommand{\e}[1]{\mathrm{e}^{#1}}
\newcommand{\dif}{\mathrm{d}}
\newcommand{\del}{\partial}
\newcommand{\eqsize}[1]{{\normalsize{#1}}}
\newcommand{\ceil}[1]{\left\lceil #1 \right\rceil}
\newcommand{\floor}[1]{\left\lfloor #1 \right\rfloor}
\newcommand{\imgwidth}{0.72\columnwidth}
\newcommand{\twidth}{0.47\columnwidth}
\newcommand{\sref}[1]{\protect\subref{#1}}
\renewcommand{\vec}[1]{{\mathbf{#1}}}
\newcommand{\vecsym}[1]{{\boldsymbol{#1}}}
\newtheorem{theorem}{Theorem}
\def\db{\si{\deci\bel}}
% \setlength{\belowdisplayskip}{1pt} \setlength{\belowdisplayshortskip}{1pt}
% \setlength{\abovedisplayskip}{1pt} \setlength{\abovedisplayshortskip}{1pt}
% \setlength{\belowcaptionskip}{2pt} \setlength{\abovecaptionskip}{1pt}
% \setlength{\dblfloatsep}{0pt} \setlength{\textfloatsep}{0pt}

%
\title{Robust Direction-of-Arrival Estimation using Array Feedback
  Beamforming in Low SNR Scenarios}
\author{Parth Mehta$^{\dagger}$, Kumar Appaiah~\IEEEmembership{Member,~IEEE}, and Rajbabu Velmurugan~\IEEEmembership{Member,~IEEE}
%
\thanks{~}\thanks{Manuscript received \today.}}
\markboth{Journal of \LaTeX\ Class Files,~Vol.~14, No.~8, August~2022}{~}
% \IEEEpubid{0000--0000/00\$00.00~\copyright~2022 IEEE}{\IEEEpubidadjcol}
% Remember, if you use this you must call \IEEEpubidadjcol in the second
% column for its text to clear the IEEEpubid mark.
%
\begin{document}
% \onecolumn
% {\bf 1) What is the problem being addressed by the manuscript and why is it important to the Antennas \& Propagation community?}
% 
% This work proposes a new spatial IIR beamformer-based direction of arrival estimation 
% method based on a retransmission strategy using a uniform linear array at both the
% transmitter and receiver. This is a particularly useful method to achieve low 
% angle-estimation error in low SNR scenarios. We propose a retransmission based 
% minimum variance distortionless response beamformer that is able to provide
% beamforming output and can be used to estimate the target directions, simultaneously.
% 
% 
% {\bf 2) What is the novelty of your work over the existing work?}.
% 
% The existing work in the domain of spatial IIR beamforming method considers an 
% extension of conventional beamforming i.e. FIR beamforming. Among the most recent work in this field is an approach that considers a retransmission method with a single transmitting antenna so as to the benefit from this feedback on the beamforming output.
% 
% We propose an enhancement of this approach by introducing an array at the transmitter, thus considering directional retransmission instead of the existing method that
% uses an omnidirectional single transmitting antenna. We show that the proposed beamforming
% architecture achieves higher directivity, narrower beamwidths and improved side-lobe
% suppression when compared with past methods.
% We also propose a method to estimate signal directions
% by incorporating the minimum variance distortionless response method to the 
% retransmission based feedback framework, and show that the proposed method outperforms
% the existing works in terms of low estimation error for the low SNR scenarios.
% 
% 
% {\bf 3) Provide up to three references, published or under review, (journal papers, conference papers, technical reports, etc.) done by the authors/coauthors that are closest to the present work. Upload them as supporting documents if they are under review or not available in the public domain. Enter “N.A.” if it is not applicable.}
% 
% N.A.
% 
% 
% {\bf 4) Provide at least three references (journal papers, conference papers, technical reports, etc.) done by other authors that are most important to the present work. These references should also be discussed in the submitted manuscript and listed among its references. Please include the citation numbers used in the manuscript for easy reference.}
% 
% [17] Duan, Huiping, et al. "Broadband beamforming using TDL-form IIR filters." IEEE Transactions on Signal Processing 55.3 (2007): 990-1002.
% 
% [22] Wen, Fuxi, Boon Poh Ng, and Vinod Veera Reddy. "Extending the concept of IIR filtering to array processing using approximate spatial IIR structure." Multidimensional Systems and Signal Processing 24 (2013): 157-179.
% 
% [23] Itay Yehezkel Karo, Tsvi G Dvorkind, and Israel Cohen, “Source Localization with Feedback  Beamforming,”
% IEEE Trans. Signal Process., vol. 69, pp. 631–640, 2020.
% 
% \newpage
% \twocolumn
\maketitle
% ---abstract--- %
\begin{abstract}
A new spatial IIR beamformer based direction-of-arrival (DoA) estimation
method is proposed in this paper. We propose a retransmission based
spatial feedback method for an array of transmit and receive antennas
that improves the performance parameters of a beamformer viz. half-power
beamwidth (HPBW), side-lobe suppression, and directivity.
Through quantitative comparison we show that our approach
outperforms the previous feedback beamforming approach with single 
transmit antenna, and the conventional beamformer. We then incorporate 
a retransmission based minimum variance distortionless response (MVDR) 
beamformer with the feedback beamforming setup. We propose two approaches 
and show that 
% performance comparison between them. We conclude that
one 
approach is superior in terms of lower estimation error, and use that 
as DoA estimation method. We then compare this approach with Multiple
Signal Classfication (MUSIC) and Estimation of Parameters using Rotation
Invariant Technique (ESPRIT) methods. While these previous methods perform poorly in low 
signal-to-noise-ratio (SNR) regime, we show that our method outperforms both at very low SNR levels. 
The results show that at SNR levels of -80~\db\ to -10~\db, the error 
is 80\% less compared to that of MUSIC and ESPRIT.
\end{abstract}
\begin{IEEEkeywords}
Spatial IIR, Feedback, Array, Beamforming, Direction-of-Arrival (DoA)
\end{IEEEkeywords}
% ---abstract--- %
%
\IEEEpeerreviewmaketitle
%
% ---section--- %
\section{Introduction}
\label{sec:intro}
Beamforming is a well studied concept in the field of spatial signal
processing using antenna arrays. Spatial beamforming, along with
direction estimation of the incoming signals and interference
mitigation has received significant attention lately, especially for
systems wherein radar and communication systems need to
coexist~\cite{li2014robust,liu2017robust}. Such coexistence
necessitates robust beamforming with strong interference suppression,
and accurate direction-of-arrival (DoA) estimation capabilities. The
MVDR~\cite{van2002optimum}
beamformer is one candidate that offers simultaneous beamforming and
interference suppression. Apart from beamforming, DoA estimation
becomes an inherent part of the overall architecture.  Typically, the
signal that is received using spatial sensors is considered a
sum-of-complex-exponentials along with additive white Gaussian
noise. This particular model appears in a variety of applications,
such as finite rate of innovation (FRI) sampling and
denoising~\cite{vetterli2002sampling}, target identification and
classification~\cite{sarkar1995using,sathe2022automatic}, and
estimation of graph-dynamics~\cite{venkitaraman2022annihilation}. With
this model, the problem reduces to estimating the ``frequency''
parameter from the observed vector signal, as we discuss in
Section~\ref{sec:model}. Depending on the context of the problem, the
notion of the ``frequency'' parameter changes. Extensive prior work
has been done to estimate this frequency, using the MVDR method and
other approaches. This includes subspace based methods for DoA
estimation such as MUSIC~\cite{1143830}, ESPRIT~\cite{roy1989esprit}
and their
variants~\cite{khan2008analysis,elbir2020deepmusic,hwang2008direction,gao2005generalized,steinwandt2017generalized},
filter based
methods~\cite{duan2005new,yan2006optimal,duan2007broadband}, and
machine learning based
methods~\cite{salvati2016use,alkhateeb2018deep}. One aspect that is
common among all these prior works is that the frequency estimation is
based on a finite impulse response (FIR) model.
%There is generally no concept of active feedback.

It is well-known that infinite impulse response (IIR) filters provide
some advantages over FIR filters for several applications. Unlike FIR
filters, the transfer function of an IIR filter has both a numerator
that accounts for the zeros of the transfer function and a denominator
that accounts for poles of the transfer function. Due to the combined
effect of both poles and zeros, IIR filters address some key issues
with FIR filters~\cite{oppenheim2001discrete} such as:
\begin{itemize}
	\item Potentially offering a lower filter order for comparable performance.
	\item Offering a sharper roll-off for passband-stopband transitions.
	%\item Avoiding overshoots in the response.
\end{itemize}
While IIR filters offer these advantages, there are also some
trade-offs that are involved when using IIR filters. These include a
non-linear phase response, more complexity involved in optimizing
coefficients etc. To this end, we ensure that the designs presented
here are not affected by these shortcomings~\cite{antonion1993digital}.

Prior work related to IIR filter based beamforming
include~\cite{duan2005new,duan2007broadband,wen2013extending,karo2020source}. Most
prior works consider direct implementation of the IIR filter by
replacing the delay-and-sum FIR structure. Restructuring the
delay-and-sum to individual IIR filters is explained
in \cite{duan2005new,duan2007broadband} wherein it is proposed to
replace the delays of the delay-and-sum with tap-delay IIR filters,
and the filter coefficients are designed using the least-mean-square
method iteratively. Extending this work in \cite{wen2013extending},
the ``spatial'' delay
elements are estimated using the recursive-least-squares method,
which then can be used to estimate the spatial frequencies
present in the signal. These architectures are
largely an approximation, and use time-domain IIR filtering to achieve
required desired spatial frequency response. To get an analogous spatial IIR structure,
the concept of ``spatial feedback'' has to be implemented, as
discussed in~\cite{karo2020source}.  Here, the authors propose to
achieve this feedback by continuously retransmitting the beamformed
signal to achieve IIR like performance.

In our work, we consider a similar setting, with the modification that
(a) the retransmission is performed using an array instead of a single
element, and (b) propose a method to incorporate the MVDR beamforming with
this feedback structure to further improve the performance.  This is
an effective way to exploit the larger number of antennas available on
MIMO wireless communication systems for more efficient DoA
estimation. We first develop an optimal retransmission strategy that
effectively utilizes the multitude of antennas by maximizing the
Fischer information for the radar transceiver, and use this to infer
the DoA. Through simulations, we first compare the performance
parameters of the proposed method with conventional FIR beamformer and
the single-element feedback method proposed in~\cite{karo2020source},
and show that our method outperforms both, achieving higher
directivity, narrower beamwidth and improved side-lobe suppression. We
then compare the DoA estimation using the proposed method with
MUSIC and ESPRIT methods, and show that our method
works better even in low SNR scenarios. The proposed method is able to
provide beamforming output, and can be used to estimate target directions,
simultaneously.

The rest of the paper is organised as follows: Section~\ref{sec:model}
discusses the system model in the context of array
signal processing using a uniform linear array
(ULA). Section~\ref{sec:feedbackbf} explains the proposed feedback
beamforming structure using an array. Section~\ref{sec:perform}
discusses the performance parameters, and Section~\ref{sec:fbmvdr}
discusses the use of MVDR with feedback beamforming for DoA
estimation. Section~\ref{sec:results} shows the comparison of
beam-pattern and its performance parameters, and performance of DoA estimation
method with prior methods. Finally, Section~\ref{sec:conclude}
concludes the discussion.

% ---section--- %
\section{System Model}
\label{sec:model}
Consider the narrowband DoA estimation problem, where the reflected 
signal from $L$ number of targets is captured using an $N$-element 
uniform linear array (ULA). At any time $n$, the received
signal vector can be written as
\begin{equation}
	\vec{r}[n] = \sum_{k=1}^{L} a_k[n] \vec{v}(\psi_k) + \vec{w}[n].
	\label{eq1:model}
\end{equation}
Here, $\vec{v}(\psi_k) = \left[
1~\e{-j\psi_k}~\dots~\e{-j(N-1)\psi_k}\right]$, $\psi_k=\frac{2\pi
d}{\lambda}\cos{\theta_k}$ is the spatial frequency, $\lambda$ is the
operating wavelength, $d$ is the inter-element spacing of ULA, and
$\vec{w}$ is complex additive white Gaussian noise with zero mean and
variance $\sigma^2$. The spatial frequency $\psi_k$ depends on the
angle of arrival $\theta_k$ of the returned signal, and is measured
from the array axis. The targets considered here are assumed to be
stationary, and the reflected signals from each target are assumed to
be uncorrelated with each other.

The model shown in \eqref{eq1:model} is closely related to the sum-of-complex-exponential model
that has been used in several other problems,
cf.~\cite{vetterli2002sampling,roy1989esprit,sarkar1995using,sathe2022automatic}. 
Therefore, the solution discussed here can be generalised to different problems,
when presented in this form.

Our goal is to estimate $\psi_k$ from $\vec{r}$. As mentioned in
Section~\ref{sec:intro}, several methods exist to solve such
systems. In this work, we take the approach of estimating the
parameters $\psi_k$ using the
MVDR~\cite{van2002optimum} beamforming method, and incorporate it
with spatial IIR feedback beamforming~\cite{karo2020source}. We
compare the performance parameters of the beam-pattern, viz. beamwidth, first
side-lobe level and directivity, with the feedback beamformer and
conventional FIR beamformers that have been used in past work. We then
use the MVDR IIR structure to estimate $\psi_k$, and show that our
approach outperforms the existing methods in terms of requiring a
lower SNR, and possessing a smaller beamwidth, thus resulting in finer
resolution and higher side-lobe suppression.

% ---section--- %
\section{Feedback Beamforming}
\label{sec:feedbackbf}
There are multiple approaches for constructing spatial IIR beamformers. We
enumerate them below.
\begin{itemize}
	\item Case 1: No feedback (FIR)
	\item Case 2: Feedback without retransmission
	\item Case 3: Feedback with retransmission using single antenna
	\item Case 4: Feedback with retransmission using antenna array
\end{itemize}
Case 1 is the conventional beamformer, wherein the beamformer response
can be modelled similar to that of a discrete-time FIR filter. Case 2
was introduced in~\cite{wen2013extending}, where the delay-like
elements in the feedback depend on an initial estimate of $\psi_k$,
and are iteratively refined over time using approaches such like
recursive least-squares (RLS). Cases 3 and 4 involve retransmission
based methods, which is the closest to a discrete-time IIR filter
based approach.  Case 3 was proposed in~\cite{karo2020source}, where
the captured signal is retransmitted using a single transmitting
antenna. The approach in~\cite{karo2020source} is specific to the case
of a single transmit antenna, and does not extend directly to the case
of multiple antennas. The case of multiple antennas is, however,
interesting from a practical point of view, owing to the fact that
several recent radar applications involve situations where
multiple antennas exist and can be exploited to enhance
performance. Therefore, we present Case 4, which is the multi-antenna retransmission
concept, wherein the retransmission is performed using an
array, as opposed to a single element.

Overall, the goal is to design the beamformer weights based on an appropriate
optimality criteria. One way to achieve this is to maximise the Fischer
information of the beamformer output. Given \eqref{eq1:model} as the
signal captured by the antenna array, the beamformer output is
\begin{equation}
	y[n] = \vecsym{\beta}^H \vec{r}[n]
	\label{eq1:feedbackbf}
\end{equation}
where $\vecsym{\beta} = [\beta_0 ~\ldots ~\beta_{N-1}]^T$ is the beamformer
weight vector. The Fischer information in the output is
\begin{equation}
	J(\psi) = \Re\left \{\frac{1}{2\pi \sigma^2} \int_{-\omega_s/2}^{\omega_s/2} \left| \frac{\del Y(\e{j\omega})}{\del \psi} \right|^2 \del \omega \right\}
	\label{eq2:feedbackbf}
\end{equation}
where $Y(\e{j\omega})$ is the discrete-time Fourier transform of $y[n]$,
$\omega_s$ is the bandwidth of interest, $\sigma^2$ is noise spectral density, and $\Re\{\cdot\}$
denotes real part. Maximising $J(\psi)$ with respect to $\vecsym{\beta}$ 
yields optimum beamformer weights.
We will use this approach to obtain 
the beamformer weights of the proposed method as well.

\subsection{Feedback with retransmission using single antenna}
\label{subsec:singlefb}
The feedback beamforming as explained in \cite{karo2020source} is shown in \figurename{\ref{fig:singlefb_blk}}.
\begin{figure}[!hbt]
	\centering
	\includegraphics[width=\imgwidth]{figures/singlefb_block.pdf}
	\caption{Feedback beamforming block diagram with single transmitting antenna.}
	\label{fig:singlefb_blk}
\end{figure}
The signal in the feedback path is retransmitted using only a single isotropic
transmitting antenna.  The diagram shown here is a simplified version
of what is shown in~\cite{karo2020source}, since the optimum feedback
coefficients obtained in the work by maximising the Fischer
information matrix (FIM) is same as the beamformer weights
$\vecsym{\beta}$.  Therefore, combining both the paths results in the
structure shown in
\figurename{\ref{fig:singlefb_blk}}. The overall feedback beamformer response
is given as:
\begin{equation}
	H_{\text{single}}(\psi) = \frac{\vecsym{\beta}^H \vec{v}(\psi) }{1 - \vecsym{\beta}^H \vec{v}(\psi) }
	\label{eq1:singlefb}
\end{equation}
where $\vecsym{\beta} = \left[ \beta_0 ~\ldots ~\beta_{N-1}\right]^T$. 
The beamformer response in the case of a conventional FIR
beamformer is $H_{\text{FIR}}(\psi) = \vecsym{\beta}^H\vec{v}(\psi)$.
From \eqref{eq1:singlefb}, it can be seen that because of the feedback,
the beamformer response also has the denominator term when 
compared to $H_{\text{FIR}}(\psi)$. This
is explained in detail in ~\cite{karo2020source}. In the proposed work, we
focus on obtaining the angle ($\theta$) information only, and hence
we discard the parameter $\phi$ used in~\cite{karo2020source} that is 
associated with range estimation.

\subsection{Retransmitting feedback with Array}
\label{subsec:arrayfb}
Extending the previous concept, we propose a retransmission method using an array instead
of a single element. The modified block diagram is shown in \figurename{\ref{fig:arrayfb_blk}}.
\begin{figure}[!hbt]
	\centering
	\includegraphics[width=\imgwidth]{figures/arrayfb_block.pdf}
	\caption{Feedback beamforming block diagram with transmitting antenna array.}
	\label{fig:arrayfb_blk}
\end{figure}

Unlike in the previous case, the transmit signal is directional,
since we are using an array instead of a single isotropic element. The
overall beamformer transfer function is given as:
\begin{equation}
	H_{\text{array}}(\psi) = \frac{\vecsym{\beta}^H \vec{v}(\psi) }{1 - \vecsym{\alpha}^H \vec{v}(\psi) \vecsym{\beta}^H \vec{v}(\psi) }.
	\label{eq1:arrayfb}
\end{equation}
Comparing \eqref{eq1:arrayfb} with \eqref{eq1:singlefb}, the
denominator possesses both $\vecsym{\alpha}$ and $\vecsym{\beta}$,
which contribute to the higher directivity. The optimal weights
$\vecsym{\alpha}$ and $\vecsym{\beta}$ are obtained by maximising the
FIM:
\begin{equation}
	\begin{split}
	\vecsym{\beta} &= \frac{k\vec{v(\psi)}}{N},\\
	~\text{and}~\vecsym{\alpha} &= \frac{\vec{v(\psi)}}{kN}, ~k \in \MB{C}-\{0\}.
	\end{split}
	\label{eq2:arrayfb}
\end{equation}
Both \eqref{eq1:arrayfb} and \eqref{eq2:arrayfb} are derived 
in Appendix~\ref{apdix:FIMderiv}.

% ---section--- %
\section{Performance parameters}
\label{sec:perform}
In this section, we derive the performance parameters of the proposed beamformer,
viz. Half-Power beamwidth (HPBW), First Side-Lobe Level (FSLL) and
directivity using the generalised beam-pattern expression,
\begin{equation}
	H_{\text{array}}(\psi) = \frac{g\vecsym{\beta}^H \vec{v}(\psi) }{1 - g\vecsym{\beta}^H \vec{v}(\psi) \vecsym{\alpha}^H \vec{v}(\psi) }
	\label{eq1:perform}
\end{equation}
where $g$ is the received signal gain. In the following
sections, we refer to $H_{\text{array}}(\cdot)$ as $H(\cdot)$ only. The beamformer weights 
$\vecsym{\alpha}$ and $\vecsym{\beta}$ are dependent on the
spatial frequency $\psi$ as shown in \eqref{eq2:arrayfb}.
Assuming that the target spatial frequency is $\psi_0$, the
beam-pattern can be obtained as a function of $\psi$:
\begin{equation}
	B(\psi) = |H(\psi)| = \left|\frac{\vecsym{\beta}^H \vec{v}(\psi_0)}{1 - \vecsym{\beta}^H \vec{v}(\psi_0) \vecsym{\alpha}^H \vec{v}(\psi_0)}\right|.
	\label{eq2:perform}
\end{equation}
Using \eqref{eq2:arrayfb} in \eqref{eq2:perform},
\begin{equation}
	B(\psi) = \frac{\frac{\sin{\left(N(\psi-\psi_0)/2\right)}}{\sin{\left((\psi-\psi_0)/2\right)}}}{1 - \left(\frac{\sin{\left(N(\psi-\psi_0)/2\right)}}{\sin{\left((\psi-\psi_0)/2\right)}}\right)^2}
	\label{eq3:perform}
\end{equation}
which is a standard beam-pattern expression. If the feedback filter 
$\vecsym{\alpha}$ is removed, \eqref{eq2:perform} reduces to the conventional FIR
beamformer response $B_{\text{FIR}}(\psi) = |H_{\text{FIR}}(\psi)| = |\vecsym{\beta}^H\vec{v}(\psi)|$.

\subsection{Half-Power Beamwidth (HPBW)}
\label{subsec:hpbw}
Assuming the filter gain $\hat{g} \neq g$, we have
\begin{equation}
	\vecsym{\beta} = \frac{\vec{v}}{\hat{g}N}~\text{and}~\vecsym{\alpha} = \frac{\vec{v}}{kN}.
\end{equation}
Substituting values in \eqref{eq1:perform}, we obtain the maximum gain as
\begin{equation}
	H(\psi) = \frac{\frac{g}{\hat{g}}}{1 - \frac{g}{k\hat{g}}} = \frac{r}{1 - \frac{r}{k}} = H(\psi)|_{\text{max}}
\end{equation}
where $k$ is tunable to ensure that the denominator can always be made
zero. Hence, ideally the beamwidth remains zero regardless of the gain
$\hat{g}$, compared to the HPBW of single element feedback
~\cite{karo2020source} $\frac{1.4}{f(r)N}$, which is a
function of the gain mismatch $r = \frac{g}{\hat{g}}$.

\subsection{First Side-lobe Level}
\label{subsec:fsll}
The side-lobe (secondary) peaks of $H(\psi)$ are at
\begin{equation}
	\psi - \psi_0 = \frac{2m+1}{N}\pi,~\forall m \in \MB{Z}-\{0\}.
\end{equation}
Hence the first side-lobe is at $m=1$, which is $\Delta \psi
= \frac{3\pi}{N}$. For a large enough $N$, from \eqref{eq3:perform} we have
\begin{equation}
	\begin{split}
	\text{FSLL}(N) &= \left| r\frac{\frac{\sin{3\pi/2}}{\sin{3\pi/2N}}}{1 - \frac{r}{k}\left(\frac{\sin{3\pi/2}}{\sin{3\pi/2N}}\right)^2} \right|^2 \\ ~ & \approx \frac{9\pi^2k^2}{4 N^2} \left( 1+\frac{9\pi^2}{2 N^2} \right) \to \MC{O}(N^{-2})				% between step: \\ ~ & \approx \left| \frac{r2N/3\pi}{1 - \frac{r}{k}(2N/3\pi)^2} \right|^2
	\end{split}
	\label{eq:fsll_formula}
\end{equation}
which is independent of the gain mismatch $r$, and
\begin{equation}
	\lim_{N \to \infty} \text{FSLL}(N) \to 0.
\end{equation}

\subsection{Directivity}
\label{subsec:dir}
The directivity $D(\psi)$ is defined as
\begin{equation}
	D = \frac{H(\psi)|_{\text{max}}}{\frac{1}{2\pi}\int_{0}^{2\pi} H(\psi) \dif \psi} = \frac{\frac{2\pi r}{1 - \frac{r}{k}}}{\int_{0}^{2\pi} H(\psi) \dif \psi}.
\end{equation}
In this case, $H(\psi)|_{\text{max}}$ is tunable, hence directivity is a function of the tuning parameter $k$.

\section{Array feedback beamforming for DoA estimation using MVDR}
\label{sec:fbmvdr}
We now present the proposed approach to estimate the $\psi$ parameters using
MVDR with feedback beamforming for the model shown
in \eqref{eq1:model}. \figurename{\ref{fig:fbmvdr_block}} shows the
block diagram depicting how MVDR can be used with a feedback
beamforming structure.
\begin{figure}[!hbt]
	\centering
	\includegraphics[width=\imgwidth]{figures/feedbackbf_mvdr_block.pdf}
	\caption{Block diagram of MVDR DoA estimation using feedback beamforming.}
	\label{fig:fbmvdr_block}
\end{figure}

As shown in \figurename{\ref{fig:fbmvdr_block}}, the signal is first captured using
a ULA, and the received signal $\vec{r}$ as in \eqref{eq1:model} is used to compute
the MVDR coefficients. Given the autocorrelation matrix 
$\vec{R}_{\vec{rr}} = \frac{1}{N_{\text{samples}}}\sum_{n=1}^{N_{\text{samples}}}\vec{r}[n]\vec{r}^H[n]$ and a
linear constraint $\vecsym{\beta}^H\vec{c} = 1$, the MVDR coefficients can be computed as:
\begin{equation}
	\vecsym{\beta}_{\text{MVDR}} = \frac{\vec{R}^{-1}_{\vec{rr}}\vec{c}}{\vec{c}^H \vec{R}^{-1}_{\vec{rr}} \vec{c}}
	\label{eq1:fbmvdr}
\end{equation}
where $N_{\text{samples}}$ are the available number of time-samples.

Next, the feedback (retransmission) coefficients $\vecsym{\alpha}$ are
inferred from $\vecsym{\beta}$, and over multiple such
retransmissions, the coefficients get stabilised.  At this point,
$\psi$ parameters are estimated using $\vecsym{\alpha}$ and
$\vecsym{\beta}$.  Depending on the approach taken to infer
$\vecsym{\alpha}$ from $\vecsym{\beta}$, there are two
possible methods to estimate $\vecsym{\alpha}$, as shown in Algorithm~\ref{alg:method1} and 
Algorithm~\ref{alg:method2}.

The classical MVDR method was primarily designed to mitigate/suppress
interference, but here we utilise the same functionality to find the
target directions instead. Typically, the MVDR weights are computed in the
absence of the targets and in presence of interference and jammers, so
that the algorithm steers the beamformer nulls in the directions where
the interferers and jammers are present while maintaining
distortionless response in a desired direction. However, in the
presence of targets, the MVDR method computes the beamformer weights such
that nulls are placed along the target directions. The constraint vector in \eqref{eq1:fbmvdr}
is taken as $\vec{c} = [1 ~\ldots ~\e{-j(N-1)\psi}]^T$ that
allows an undistorted response from the desired direction. This constraint
is used for Algorithm~\ref{alg:method1}.
\begin{algorithm}[!hbt]
	\caption{}
	\begin{algorithmic}[1]
	\State{Start with $\vecsym{\beta}$ as in \eqref{eq1:fbmvdr}.}
	\State{Formulate $\vecsym{\alpha} = \vec{c} = [1 ~\e{-j\psi} ~\dots ~\e{-j(N-1)\psi}]^T$}
	\State{Sweep $\psi$ in the visible region (typically $[-\pi,\pi)$). If a target angle
	coincides with the sweep angle, the response at that angle peaks while suppressing
	contribution of all the other targets.}
	\State{Transmit vector = $\vecsym{\alpha}(\psi)$} 	
	\end{algorithmic}
	\label{alg:method1}
\end{algorithm}

The same constraint vector $\vec{c}$ can be
changed to any other linear constraint vector, as long as the algorithm
produces computable beamformer weights. A simple choice is
$\vec{c} = [1 ~0 ~\ldots ~0]^T$, that ensures
the first element of $\vecsym{\beta}$, that is $\beta_0$,
is always $1$. This constraint is used for Algorithm~\ref{alg:method2}.
\begin{algorithm}[!hbt]
	\caption{}
	\begin{algorithmic}[1]
	\State{Start with $\vecsym{\beta}$ as in \eqref{eq1:fbmvdr} using $\vec{c} = [1 ~\ldots ~0]^T$. 
	The nulls of $\vecsym{\beta}$ appear along the target directions.}
	\State{Formulate $\vecsym{\alpha}$ as impulse response of $1/\vecsym{\beta}(\psi)$}
	\State{The nulls in $\vecsym{\beta}(\psi)$ result as peaks in $\vecsym{\alpha}(\psi)$}
	\State{Transmit vector = $\vecsym{\alpha}(\psi)$}
	\end{algorithmic}
\label{alg:method2}
\end{algorithm}

Both the methods yield iteratively better estimates, even if the initial estimates 
are inaccurate. We discuss and compare the performance of both methods in the next section.

% ---section--- %
\section{Results and Discussion}
\label{sec:results}
In this section, via simulations, we first show the beam-pattern and
FSLL as discussed in Section~\ref{sec:perform} for the proposed beamforming
method. We then show the
performance of the DoA estimation using proposed method and compare with
previous methods, viz. 
MUSIC~\cite{1143830}, and ESPRIT~\cite{roy1989esprit}.

\subsection{Beam-pattern and performance parameters}
We consider a 3-element ULA, with inter-element spacing $\lambda/2$, and assume
a target present at $\theta = \pi/3$. We plot the beamformer response for
all angles $\theta \in [0,\pi)$ and compare the beam-pattern parameters of the proposed
method with that of existing methods.
\begin{figure}[!hbt]
	\centering
	\includegraphics[width=\imgwidth]{figures/beampattern_compare.pdf}
	\caption{Comparing beam-pattern of array feedback beamformer with single feedback and conventional FIR beamformers}
	\label{fig:beamcompare}
\end{figure}

In \figurename{~\ref{fig:beamcompare}}, we compare the 
beam-pattern of the proposed feedback beamforming with existing 
feedback beamforming methods~\cite{karo2020source} and the conventional FIR method.
The HPBW of the FIR beamformer is limited by the number 
of elements, and that of the feedback beamformer is a
function of gain mismatch. Compared to the single element feedback case,
the HPBW of the proposed approach is 50\% less, which is consistent
with the reduction predicted in theory.
The side-lobe level of the feedback beamformer is 50~\db\
below conventional FIR beamformer, while that for the proposed method,
it is even lower, at $110$~\db. The proposed method outperforms 
both the conventional FIR beamformer and feedback beamformer~\cite{karo2020source} 
in terms of side-lobe suppression and beamwidth.

\begin{figure}[!hbt]
	\centering
	\includegraphics[width=\imgwidth]{figures/fb_beam.pdf}
	\caption{Beam-pattern after finite (100) retransmission time-stamps}
	\label{fig:beamfinite}
\end{figure}
\figurename{~\ref{fig:beamfinite}} shows the comparison of an ideal
feedback beamformer with that of a realisable one. In practice, the
IIR-like performance can be achieved only approximately, because of
the finite number of snapshots that are available. Even then, the
proposed method achieves a HPBW that is 90\% lower than that obtained
using FIR beamformers.

\begin{figure}[!hbt]
	\centering
	\includegraphics[width=\imgwidth]{figures/sidelobe_trend.pdf}
	\caption{First Side-lobe level trend with number of array elements.}
	\label{fig:fsll}
\end{figure}
\figurename{~\ref{fig:fsll}} shows the absolute value of FSLL for 
different number of array elements. According to 
\eqref{eq:fsll_formula}, the FSLL decreases as 
$\MC{O}(N^{-2})$, and is independent of the gain mismatch $r$, as opposed
to the FIR beamformer, where the FSLL remains constant at 
$\sim -13.5$~\db. Even with as few as 16 elements, the level
difference is $7$~\db, and it decreases fast with increasing number
of elements. At 1024 elements, the difference is around $100$~\db.

\subsection{Direction Estimation}
As discussed in Section~\ref{sec:fbmvdr}, we can use the MVDR with 
feedback beamforming to estimate $\psi$ from
$\vec{r}$ in \eqref{eq1:model}. We use Algorithm~\ref{alg:method1}
and Algorithm~\ref{alg:method2} shown in
Section~\ref{sec:fbmvdr} and compare the resultant estimation error
with MUSIC and ESPRIT for various SNR levels. We
also consider inter-element spacing in the ULA $d
= \lambda/2$, hence $\psi = \pi\cos{\theta}$.
We take an 8-element ULA, and assume 4 targets at distinct angles
$\theta_1, \theta_2, \theta_3, \theta_4$. The signals reflected from
these targets are considered uncorrelated with each other. The received
signal at the array is assumed to follow the model in \eqref{eq1:model}.

\begin{figure}[!hbt]
	\centering
	\subfloat[Algorithm~\ref{alg:method1}]{\includegraphics[width=\twidth]{figures/doaerrsnr2.pdf}\label{sfig:fbmvdr_err2}}
	\subfloat[Algorithm~\ref{alg:method2}]{\includegraphics[width=\twidth]{figures/doaerrsnr1.pdf}\label{sfig:fbmvdr_err1}}
	\caption{RMS error of MVDR DoA estimation using feedback beamforming with respect to number of retransmissions for SNR
	-10,~0,~and 10 \db}
	\label{fig:fbmvdr_err}
\end{figure}
\figurename{~\ref{fig:fbmvdr_err}} shows the root-mean-squared error (RMSE) for 
three different levels of SNR. Since our goal is to target the low
SNR region, we consider SNR = -10~\db, 0~\db,
and 10~\db. \figurename{~\ref{sfig:fbmvdr_err1}}
and \figurename{~\ref{sfig:fbmvdr_err2}} show the performance for
Algorithm~\ref{alg:method1} and Algorithm~\ref{alg:method2} discussed in Section~\ref{sec:fbmvdr},
respectively. From \figurename{~\ref{fig:fbmvdr_err}} it can be seen that, for no 
retransmission, the estimation error is large, since feedback is absent.
As the number of retransmissions increases, the error reduces rapidly for
both methods. Comparing \figurename{~\ref{sfig:fbmvdr_err1}} and
\figurename{~\ref{sfig:fbmvdr_err2}}, it is evident that Algorithm~\ref{alg:method1}
yields better estimates than Algorithm~\ref{alg:method2}, since the error reduces faster
for Algorithm~\ref{alg:method1}.

\begin{figure}[!hbt]
	\centering
	\subfloat[SNR=40~\db]{\includegraphics[width=\twidth]{figures/fbmvdr_compare40.pdf}\label{sfig:fbmvdr_compare1}}
	\subfloat[SNR=0~\db]{\includegraphics[width=\twidth]{figures/fbmvdr_compare0.pdf}\label{sfig:fbmvdr_compare2}}
	\caption{Comparison of feedback MVDR with 2 retransmissions with MUSIC and ESPRIT}
	\label{fig:fbmvdr_compare}
\end{figure}
\figurename{\ref{fig:fbmvdr_compare}} shows the angle estimation
performance as compared with MUSIC and ESPRIT.  For this, we use Algorithm~\ref{alg:method1}.
The array steering direction $\theta$ is varied from $0$ to $\pi$,
and the transmit weight vector $\vecsym{\alpha}$ varies
accordingly with the constraint $\vec{c}$. 
The beamformer response is computed as $y_{\psi}$, and
the pseudo-spectrum is computed as
$P_{\text{out}}(\psi) = \frac{1}{N_{\text{samples}}}\sum_{n}y_{\psi}[n]y_{\psi}^*[n],~\forall \psi$. 
\figurename{\ref{fig:fbmvdr_compare}} shows the normalised 
pseudo-spectrum as a function of $\theta$.

It can be seen that, with as low as just two retransmissions, the
feedback beamformer is able to achieve angle estimates at par with
that of ESPRIT at high SNR. However, for both MUSIC and ESPRIT, the number
of targets $L$ is known, which is not a constraint for the
proposed feedback MVDR method. Even at $0$~\db\ SNR, the proposed method shows
better estimates than both MUSIC and ESPRIT.

Finally, in \figurename{\ref{fig:doa_snr_compare}} we compare the
DoA estimation error of the proposed method
with MUSIC and ESPRIT for different SNR values.
\begin{figure}[!hbt]
	\centering
	\includegraphics[width=\imgwidth]{figures/doa_snr_compare.pdf}
	\caption{Comparison of RMSE with respect to SNR of feedback MVDR, MUSIC and ESPRIT}
	\label{fig:doa_snr_compare}
\end{figure}

From \figurename{\ref{fig:doa_snr_compare}} it is evident that feedback MVDR 
yields better error performance than both MUSIC and ESPRIT. While these methods
primarily work better at higher SNR levels as seen for SNR values beyond $60$~\db,
our method works even at the low SNR range of $-80$~\db\ to $-10$~\db. The 
RMSE is 80\% lower compared to that of MUSIC and ESPRIT
in the low SNR region.

% ---section--- %
\section{Conclusion}
\label{sec:conclude}
We propose a novel approach to the problem of spatial filtering, beamforming
and DoA estimation using feedback beamforming method with retransmitting array.
We apply this approach for a ULA to derive the beam-pattern performance parameters.
Through extensive simulations, we show that our method outperforms the
conventional FIR beamformer and previously presented IIR beamforming structures in terms
of better side-lobe suppression, higher directivity, and narrower beamwidth. Then, we propose
a method to incorporate MVDR into this feedback structure and show that the combined
architecture is able to provide beamforming output, and can be used to estimate
the target directions. 
%We propose two different architectures, and show that one performs better than the other in terms of low angle-estimation error.
The proposed method outperforms  conventional DoA estimations methods
MUSIC and ESPRIT  in terms of achieving better accuracy and better 
target separation. The proposed method is arguably better in the low-SNR regime, and yields 
80\% lower estimation error, and hence is useful for applications with stringent 
SNR constraints.


% ---section--- %
\appendix
%\small
\label{apdix:FIMderiv}
Here we will derive the beamformer weights $\vecsym{\alpha}$ and $\vecsym{\beta}$
by maximising the Fischer information matrix. Consider the signal
$s[n]$ that is transmitted
using the $N_t$-element transmitting array, along with the feedback component.
The transmitted signal from the $k$th element of the array at time $n$ is given as
\begin{equation*}
	t_k[n] = \delta_k s[n] + \alpha_k^* y[n - \tau_k], \quad\delta_k = \begin{cases}
	                                                   	1&:~k=0 \\
	                                                   	0&:~\text{otherwise}
	                                                   \end{cases}
\end{equation*}
where $k=0,\ldots,N_t-1$, and $(\cdot)^*$ denotes complex conjugate. Similarly, 
the  signal received at the $l$th element of the $N_r$-element receiving array is given as
\begin{align*}
	r_l[n] &= \sum_k t_k[n-\tau_l - \tau_R] \\
	 ~ &= s[n- \tau_l - \tau_R] + \sum_k \alpha_k^* y[n - \tau_k-\tau_l - \tau_R]
\end{align*}
where $k=0,\ldots,N_t-1$ and $l=0,\ldots,N_r -1$. $\tau_l$ and
$\tau_k$ are the delays due to the array
geometry, and $\tau_R$ is the delay due to the target range. Converting $r_l$ to frequency domain,
\begin{align*}
	R_l(\e{j\omega}) = \left( S(\e{j\omega}) + \sum_k \alpha_k^* \e{-j\omega\tau_k}Y(\e{j\omega})\right)\e{-j\omega (\tau_l+\tau_R)}
\end{align*}
where $R_l(\e{j\omega})$, $S(\e{j\omega})$, and $Y(\e{j\omega})$ are
the discrete-time Fourier transforms of
$r_l[n]$, $s[n]$ and $y[n]$, respectively.

The beamformer output is given as
\begin{align*}
	Y(\e{j\omega}) &= \sum_l \beta_l^* R_l(\e{j\omega}) \\
				~&= \left( S(\e{j\omega}) + \sum_k \alpha_k^* \e{-j\omega\tau_k}Y(\e{j\omega})\right) \sum_l \beta_l^* \e{-j\omega \tau_l}\e{-j\omega\tau_R}.
\end{align*}

Hence, the transfer function at $\omega$ is given as
\begin{align*}
	H(\e{j\omega}) = \frac{Y(\e{j\omega})}{S(\e{j\omega})} = \frac{\sum_l \beta_l^* \e{-j\omega\tau_l} \e{-j\omega\tau_R}}{1 - \sum_l \beta_l^* \e{-j\omega\tau_l} \sum_k \alpha_k^* \e{-j\omega\tau_k}\e{-j\omega\tau_R}}.
\end{align*}

Considering the narrowband signal model, for ULA, $\tau_l = l \tau$ and 
$\tau_k = k \tau$ for $k,l = 0,1,\dots, N-1$
assuming $N_t = N_r = N$. Taking 
$\omega\tau = \frac{2\pi}{\lambda}d\cos{\theta} = \psi$, and single frequency $\omega_0$
\begin{equation*}
	H(\psi,\phi) = \frac{\vecsym{\beta}^H \vec{v}(\psi) \e{-j\phi}}{1 - \vecsym{\beta}^H \vec{v}(\psi) \vecsym{\alpha}^H \vec{v}(\psi) \e{-j\phi}}
\end{equation*}
where $\vec{v}(\psi) = [1 ~\e{-j\psi} ~\e{-2j\psi} ~\dots ~\e{-j(N-1)\psi}]^T$
and $\phi = \omega\tau_R$.
This expression is shown in \eqref{eq1:arrayfb}.

% Dropping the range parameter $\phi$
% \begin{equation}
% 	H(\psi) = \frac{\vecsym{\beta}^H \vec{v}_r(\psi)}{1 - \vecsym{\beta}^H \vec{v}_r(\psi) \vecsym{\alpha}^H \vec{v}_t(\psi)}
% \end{equation}
% 
% Constructing Fischer information considering $\vec{v}_t = \vec{v}_r$,
% \begin{equation}
% 	J(\psi) = \Re\left\{\frac{1}{2\pi \sigma^2} \int_{-\frac{\omega_s}{2}}^{\frac{\omega_s}{2}} \left| \frac{\dif y(\e{j\omega})}{\dif \psi} \right|^2 \dif \omega \right\}
% \end{equation}
% 
% Finding the derivartive
% \begin{align}
% 	\frac{\dif y(\e{j\omega})}{\dif \psi} &= \frac{\begin{matrix}(1 - \vecsym{\beta}^H \vec{v}(\psi) \vecsym{\alpha}^H \vec{v}(\psi))(\vecsym{\beta}^H \vec{A}\vec{v}(\psi)) \\
% 	- (\vecsym{\beta}^H \vec{v}(\psi))(-\vecsym{\beta}^H \vec{A}\vec{v}(\psi)\vecsym{\alpha}^H \vec{v}(\psi) - \vecsym{\beta}^H \vec{v}(\psi) \vecsym{\alpha}^H \vec{A}\vec{v}(\psi))\end{matrix}}{(1 - \vecsym{\beta}^H \vec{v}(\psi) \vecsym{\alpha}^H \vec{v}(\psi))^2} S(\e{j\omega}) \\
% 	~ &= \frac{\vecsym{\beta}^H (\vec{I} + \vec{v}(\psi)\vecsym{\beta}^H\vec{v}(\psi)\vecsym{\alpha}^H) \vec{A}\vec{v}(\psi)}{(1 - \vecsym{\beta}^H \vec{v}(\psi) \vecsym{\alpha}^H \vec{v}(\psi))^2}S(\e{j\omega}) \\
% \end{align}
% where $\vec{A} = \text{diag}([0 ~-j ~-2j ~\dots ~-j(N-1)])$.
% 
% Hence the Fischer information
% \begin{equation}
% 	J(\psi) = \Re\left\{\frac{1}{2\pi \sigma^2} \int_{-\frac{\omega_s}{2}}^{\frac{\omega_s}{2}} \left| \frac{\vecsym{\beta}^H (\vec{I} + \vec{v}(\psi)\vecsym{\beta}^H\vec{v}(\psi)\vecsym{\alpha}^H) \vec{A}\vec{v}(\psi)}{(1 - \vecsym{\beta}^H \vec{v}(\psi) \vecsym{\alpha}^H \vec{v}(\psi))^2}S(\e{j\omega}) \right|^2 \dif \omega \right\}
% \end{equation}
% 
% Considering singal energy to be 1,
% \begin{equation}
% 	J(\psi) = \frac{1}{2\pi \sigma^2} \left| \frac{\vecsym{\beta}^H (\vec{I} + \vec{v}(\psi)\vecsym{\beta}^H\vec{v}(\psi)\vecsym{\alpha}^H) \vec{A}\vec{v}(\psi)}{(1 - \vecsym{\beta}^H \vec{v}(\psi) \vecsym{\alpha}^H \vec{v}(\psi))^2} \right|^2
% \end{equation}
% 
% Maximising this means
% \begin{align}
% 	1 - \vecsym{\beta}^H \vec{v}(\psi) \vecsym{\alpha}^H \vec{v}(\psi) = 0
% \end{align}
% 
% which results in
% \begin{align}
% 	\vecsym{\beta} = \frac{k\vec{v}}{\norm{\vec{v}}^2},~\text{and}~\vecsym{\alpha} = \frac{\vec{v}}{k\norm{\vec{v}}^2}, ~k \in \MB{C}-\{0\}
% \end{align}
% 
% Because of the structure of $\vec{v}(\psi)$, $\norm{\vec{v}(\psi)}^2 = N$, where $N$ is the number of array elements. Hence
% \begin{align}
% 	\vecsym{\beta} = \frac{k\vec{v}}{N},~\text{and}~\vecsym{\alpha} = \frac{\vec{v}}{kN}, ~k \in \MB{C}-\{0\}
% \end{align}
% 
% Taking $k=1, \vecsym{\beta} = \vecsym{\alpha} = [1 ~\e{-j\psi_0} ~\dots ~\e{-j(N-1)\psi_0}]$, \eqref{eq1:arrayfb} can be obtained.

We formulate the Fischer Information Matrix as below:
\begin{equation*}
	\vec{J} = \begin{bmatrix} J_{\psi\psi} & J_{\psi\phi} \\ J_{\phi\psi} & J_{\phi\phi} \end{bmatrix}
\end{equation*}
where $J_{\psi\phi} = J_{\phi\psi}^*$. Each term $J_{pq}$ can be computed as
\begin{equation*}
	J_{pq} = \Re\left\{\frac{1}{2\pi \sigma^2} \int_{-\frac{\omega_s}{2}}^{\frac{\omega_s}{2}} \left( \frac{\del Y(\e{j\omega})}{\del p} \right)^* \left( \frac{\del Y(\e{j\omega})}{\del q} \right) \dif \omega \right\}
\end{equation*}
where $\Re\{\cdot\}$ denotes real part, $p,q \in \{\phi,\psi \}$, $\omega_s$ is the bandwidth, 
and $\sigma^2$ is the noise spectral density. Hence,
\begin{footnotesize}
\begin{align*}
	\frac{\del Y(\e{j\omega})}{\del \psi} &= \frac{\vecsym{\beta}^H (\vec{I} + \vec{v}(\psi)\vecsym{\beta}^H\vec{v}(\psi)\vecsym{\alpha}^H \e{-j\phi}) \vec{A}\vec{v}(\psi)\e{-j\phi}}{(1 - \vecsym{\beta}^H \vec{v}(\psi) \vecsym{\alpha}^H \vec{v}(\psi)\e{-j\phi})^2}S(\e{j\omega}) \\
	\frac{\del Y(\e{j\omega})}{\del \phi} &= \frac{-j\vecsym{\beta}^H \vec{v}(\psi) \e{-j\phi}}{(1 - \vecsym{\beta}^H \vec{v}(\psi) \vecsym{\alpha}^H \vec{v}(\psi)\e{-j\phi})^2}S(\e{j\omega})
\end{align*}
\end{footnotesize}

All the terms of $\vec{J}$ can be computed as:
\begin{subequations}
\begin{equation*}
	\resizebox{\hsize}{!}{$
	J_{\psi\psi} = \frac{1}{2\pi \sigma^2} \int_{-\frac{\omega_s}{2}}^{\frac{\omega_s}{2}} \left| \frac{\vecsym{\beta}^H (\vec{I} + \vec{v}(\psi)\vecsym{\beta}^H\vec{v}(\psi)\vecsym{\alpha}^H \e{-j\phi}) \vec{A}\vec{v}(\psi)}{(1 - \vecsym{\beta}^H \vec{v}(\psi) \vecsym{\alpha}^H \vec{v}(\psi)\e{-j\phi})^2} \right|^2 |S(\e{j\omega})|^2 \dif \omega
	$}
\end{equation*}

\begin{equation*}
\resizebox{\hsize}{!}{$
	J_{\phi\phi} = \frac{1}{2\pi \sigma^2} \int_{-\frac{\omega_s}{2}}^{\frac{\omega_s}{2}} \left| \frac{\vecsym{\beta}^H \vec{v}(\psi)}{(1 - \vecsym{\beta}^H \vec{v}(\psi) \vecsym{\alpha}^H \vec{v}(\psi)\e{-j\phi})^2}\right|^2 |S(\e{j\omega})|^2 \dif \omega
	$}
\end{equation*}

\begin{equation*}
\resizebox{\hsize}{!}{$
	J_{\psi\phi} = \Re\left\{\frac{j}{2\pi \sigma^2} \int_{-\frac{\omega_s}{2}}^{\frac{\omega_s}{2}} \frac{\vecsym{\beta}^H \left(\vec{I} + \vec{v}(\psi)\vecsym{\beta}^H\vec{v}(\psi)\vecsym{\alpha}^H \e{-j\phi}\right)\vec{A} \vec{v}(\psi) \vec{v}(\psi)^H \vecsym{\beta}}{(1 - \vecsym{\beta}^H \vec{v}(\psi) \vecsym{\alpha}^H \vec{v}(\psi)\e{-j\phi})^4}\right\} |S(\e{j\omega})|^2  \dif \omega
	$}
\end{equation*}
\end{subequations}
where $\vec{A} = \text{diag}(\begin{bmatrix} 0 &-j &-2j &\dots &-j(N-1)] \end{bmatrix})$.
Maximising this FIM results in the weights to be
\begin{align*}
	\vecsym{\beta} = \frac{k\vec{v}}{N}\e{-j(1-l)\phi},~\text{and}~\vecsym{\alpha} = \frac{\vec{v}}{kN}\e{-jl\phi}, ~k \in \MB{C}-\{0\}, ~l \in [0,1].
\end{align*}
This expression is shown in \eqref{eq2:arrayfb} in Section~\ref{sec:feedbackbf}
without the parameter $\phi$.

\bibliography{IEEEabrv,bibly}
\bibliographystyle{IEEEbib}

% \begin{IEEEbiographynophoto}{Parth Mehta}
% received his B.Tech. degree in electronics and communication
% cngineering from Nirma University, India in 2016 and M.Tech degree in electronics 
% engineering under the specialization of radar and communication at Defence Institute
% of Advanced Technology (DIAT), Pune, India in 2018. He is currently pursuing Ph.D. 
% at department of electrical engineering, Indian Institute of Technology Bombay (IITB), India.
% \end{IEEEbiographynophoto}
% 
% \begin{IEEEbiographynophoto}{Kumar Appaiah}
% received the B.Tech. and M.Tech. degrees in electrical engineering from 
% IIT Madras, India in 2008, and the Ph.D. degree in electrical and computer engineering
% from the University of Texas at Austin, Austin, TX, USA, in 2013. From 2013 to 2014, 
% he was a Senior Engineer with Qualcomm Flarion Technologies, Bridgewater, NJ. Since 
% 2014, he has been an Assistant Professor of Electrical Engineering with the IIT Bombay,
% Mumbai, India. His research interests include signal processing for optical communication,
% and multiplexing in wireless and fiber-optic communication systems.
% \end{IEEEbiographynophoto}
% 
% \begin{IEEEbiographynophoto}{Rajbabu Velmurugan}
% is currently an Associate Professor at the Department of Electrical 
% Engineering, Indian Institute of Technology Bombay (IITB), India. He received his Ph.D. 
% degree in electrical engineering from Georgia Institute of Technology, Atlanta. 
% His research interests are in signal processing, statistical signal processing, 
% audio, speech, and image processing. His current focus is on source separation, 
% blind deconvolution, and target tracking problems.
% \end{IEEEbiographynophoto}

\end{document}

