The use of medical research data for various statistical tasks has been done from a prolonged period of time. The data after having consistent number of records can be utilized for deriving inference using prognostication methods. The methods that fall under the area of inferential statistics \cite{Marshall2011AnIT} which are extended with applied areas of statistics can be used, one of which can be used is Machine Learning \cite{sarker2021machine}. Task of prognostication based on data can be done by learning patterns from the data using machine learning. The dataset that we have used in this paper is pertaining to polycystic ovary syndrome \cite{ndefo2013polycystic} diagnosis, which falls under the classification \cite{10.2307/2344237} category. Classification has 2 sub-divisions, viz. Binary and Multi-Class where the data used in this paper falls under binary classification \cite{10.5120/ijca2017913083} precisely. The methods using Machine Learning have already been performed on polycystic ovary syndrome abbreviated as PCOS using logistic regression \cite{9510128}, bagging ensemble methods  \cite{kanvinde2022binary}, discriminant analysis \cite{gupta2022discriminant}, stacked generalization \cite{nair2022combining}, boosting ensemble methods \cite{gupta2021succinct} and deep neural networks \cite{gupta2022residual}. The use of deep learning is evident and we want to focus on the more variations that can be brought into the current state-of-the-art system. Deep Learning \cite{lecun2015deep,goodfellow2016deep} provides more depth of learning as compared to machine learning and is used when the amount of parameters in dimensions are high. The PCOS dataset has over 41 dimensions which are enough for instating the use of deep learning. The variation we wanted to perform was related to some machine learning algorithm that can be leveraged with power of deep learning. The implementation of any machine learning algorithm using a library like scikit-learn \cite{scikit-learn,sklearn_api} meets limitations in terms of utilization with GPU processing power. For the same reason, using a mature framework like Tensorflow \cite{tensorflow2015-whitepaper} can definitely bring change to the working. This is where we decided to work with parametric learning method which works with simple and definite procedures. The parametric learning method we focused on using was discriminant analysis \cite{10.2307/143150,10.1007/978-1-4612-6079-0_16} which has variations in it where we focused on Linear Discriminant Analysis \cite{Tharwat2017LinearDA}. This actually accounted for an idea that training the linear discriminant analysis with deep learning style will yield us Deep Linear Discriminant Analysis \cite{dorfer2015deep,8014812} which has been already been discovered and we decided to proceed with out implementation using it. The variations that we brought in the network will be explained in further sections of this paper. 