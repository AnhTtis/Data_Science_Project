\rev{\section{Run Times in Dependence of the Number of Data Points}}
\label{sec:performancepersize}

\rev{In this section, we complement
  Section~\ref{sec:numerical-results} by presenting the run times in
  dependence of the number of points in the data set in order to shed
  some light on the scalability of our approaches.
  To this end, we split the entire data set in three subsets.}

\begin{figure}
    \centering
    \includegraphics[width=0.7\textwidth]{figures/PerformanceProfileTimewithSVMuntil500points.pdf}
    \includegraphics[width=0.7\textwidth]{figures/PerformanceProfileTimewithSVM500-1500points.pdf}
    \includegraphics[width=0.7\textwidth]{figures/PerformanceProfileTimewithSVMMmorethan1500points.pdf}
    \caption{\rev{ECDFs for run time (in seconds).
        Top: Instances with $N \geq 500$.
        Middle: Instances with $N \in  (500,1500]$.
      Bottom: Instances with $N > 1500$.}}
    \label{fig:mainfig}
\end{figure}

\rev{The first subset only considers those 46 data sets with $N \leq
  500$.
  As can be seen in Figure~\ref{fig:mainfig} (top), IRCM
  solves more than \SI{75}{\percent} of the instances while CS$^3$VM
  and WIRCM solve more than \SI{50}{\percent}.
  The second subset contains 11 data sets with $N \in (500,1500]$.
  Figure~\ref{fig:mainfig} (middle) shows that for these test sets, IRCM solves
  about \SI{40}{\percent} of the instances while CS$^3$VM and WIRCM
  solve more than \SI{10}{\percent}.
  The last subset contains those 21 data sets with $N > 1500$.
  Figure~\ref{fig:mainfig} (bottom) shows that CS$^3$VM and WIRCM do not solve any of
  these instances and IRCM solves about \SI{20}{\percent}.
  As expected, the larger the number of points and, thus, the larger
  the number of binary variables, the more challenging it is to solve
  the instances.
  Besides that, SVM solves all instances, which is expected since it
  does not include any binary variables.}


%%% Local Variables:
%%% mode: latex
%%% TeX-master: "constrained-svm-preprint"
%%% End:
