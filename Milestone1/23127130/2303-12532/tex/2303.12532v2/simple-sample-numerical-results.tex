\section{Numerical Results for Simple random samples}
\label{sec:num-results-simple-sample}

In Section~\ref{sec:numerical-results}, we focused our
computational study on non-representative, biased samples.
The common baseline scenario to check the performance of estimators is
to apply them on simple random samples. Hence, for completeness, we
also present the results under simple random sampling. That is, each
unit in the data set has the same probability $\pi_i = n/N$ to be
included into the sample of size $n$.
The instances are the same as described in
Section~\ref{subsection-test-sets}.
The computational setup follows the description in
Section~\ref{subsection-comp-setup}.
As before, the used evaluation criteria are
$\widehat{\AC},$ $\widehat{\PR}$ as in~\eqref{compartrue}
and $\widehat{\RE},$ $\widehat{\FPR}$ as in~\eqref{compartrue2}.

\begin{figure}
  \centering
  \includegraphics[width=0.495\textwidth]{figures/ACbasedontrueALLss}
  \includegraphics[width=0.495\textwidth]{figures/ACbasedontrueUNss}
  \caption{Relative accuracy $\widehat{\AC}$ w.r.t.\ the true
    hyperplane; see~\eqref{compartrue}, for the simple random samples.
    Left: Comparison for all data points.
    Right: Comparison only for unlabeled data points.}
  \label{ACtru2}
\end{figure}

Figure~\ref{ACtru2} and~\ref{PRtru2} show similar accuracy and
precision performance for all approaches. This is as expected, as the
sample is not biased and hence the cardinality constraint does not
contribute relevant additional information to the problem.
Therefore, the SVM does not tend to classify the points as positive as
it is the case for the biased samples.
The outliers, mainly present for CS$^3$VM, are due those instances
that are not solved within the time limit.
As can be seen in Figure~\ref{REtru2} and~\ref{FPRtru2}, recall and
false positive rate are also similar for all approaches.

Hence, for the simple random samples our approaches have almost the
same results as the SVM.
Note that for the biased samples, they outperformed the SVM.
Hence, in cases for which the type of sample is not known,
it is ``safe'' to use the newly proposed approaches for
classification.

\begin{figure}
  \centering
  \includegraphics[width=0.495\textwidth]{figures/PRbasedontrueALLss}
  \includegraphics[width=0.495\textwidth]{figures/PRbasedontrueUNss}
  \caption{Relative precision $\widehat{\PR}$ w.r.t.\ the true
    hyperplane; see~\eqref{compartrue}, for the simple random samples.
    Left: Comparison for all data points.
    Right: Comparison only for unlabeled data points.}
  \label{PRtru2}
\end{figure}
%
\begin{figure}
  \centering
  \includegraphics[width=0.495\textwidth]{figures/REbasedontrueALLss}
  \includegraphics[width=0.495\textwidth]{figures/REbasedontrueUNss}
  \caption{Relative recall $\widehat{\RE}$ w.r.t.\ the true hyperplane;
    see~\eqref{compartrue2}, for the simple random samples.
    Left: Comparison for all data points.
    Right: Comparison only for unlabeled data points.}
  \label{REtru2}
\end{figure}
%
\begin{figure}
  \centering
  \includegraphics[width=0.495\textwidth]{figures/FPRbasedontrueALLss}
  \includegraphics[width=0.495\textwidth]{figures/FPRbasedontrueUNss}
  \caption{Relative false positive rate $\widehat{\FPR}$ w.r.t.\ the
    true hyperplane; see~\eqref{compartrue2}, for the simple random samples.
    Left: Comparison for all data points.
    Right: Comparison only for unlabeled data points.}
  \label{FPRtru2}
\end{figure}


%%% Local Variables:
%%% mode: latex
%%% TeX-master: "constrained-svm-preprint"
%%% End:
