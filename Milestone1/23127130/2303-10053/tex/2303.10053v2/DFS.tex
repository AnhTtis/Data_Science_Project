\documentclass[14pt,reprint,twocolumn,longbibliography,
%groupedaddress,
%unsortedaddress,
%runinaddress,
%frontmatterverbose,
%preprint,
%showpacs,preprintnumbers,
%nofootinbib,
%nobibnotes,
%bibnotes,
superscriptaddress,
amsmath,
amssymb,
aps,
pra
%prb,
%rmp,
%prstab,
%prstper,
%floatfix,
]{revtex4-1}
\usepackage{graphicx,amsmath,subfigure}%
\usepackage{color,xcolor}
\usepackage[bookmarks=false]{hyperref}
\hypersetup{colorlinks=true,citecolor=cyan,linkcolor=blue,urlcolor=blue,pdfstartview=FitH,bookmarksopen=true}
\usepackage{epstopdf}

\begin{document}

\title{Dynamical-decoupling protected nonadiabatic geometric quantum computation with silicon-vacancy centers}

\author{M.-R. Yun}
\affiliation{ School of Physics and Microelectronics, Key Laboratory of Materials Physics of Ministry of Education,
	Zhengzhou University, Zhengzhou 450052, China}
 \author{J.-L. Wu}
\affiliation{ School of Physics and Microelectronics, Key Laboratory of Materials Physics of Ministry of Education,
	Zhengzhou University, Zhengzhou 450052, China}
\author{L.-L. Yan}
\email{llyan@zzu.edu.cn}
\affiliation{ School of Physics and Microelectronics, Key Laboratory of Materials Physics of Ministry of Education,
	Zhengzhou University, Zhengzhou 450052, China}
\author{Yu Jia}
\email{jiayu@zzu.edu.cn}
 \affiliation{ Key Laboratory for Special Functional Materials of Ministry of Education, and School of Materials and Engineering, Henan University, Kaifeng 475001, China}	
 \affiliation{ School of Physics and Microelectronics, Key Laboratory of Materials Physics of Ministry of Education,
	Zhengzhou University, Zhengzhou 450052, China}
\author{S.-L. Su}
\email{slsu@zzu.edu.cn}
\affiliation{ School of Physics and Microelectronics, Key Laboratory of Materials Physics of Ministry of Education,
	Zhengzhou University, Zhengzhou 450052, China}
\author{C. X. Shan}
\email{cxshan@zzu.edu.cn}
\affiliation{ School of Physics and Microelectronics, Key Laboratory of Materials Physics of Ministry of Education,
	Zhengzhou University, Zhengzhou 450052, China}
%$^{3}$ State Key Laboratory of Magnetic Resonance and Atomic and Molecular Physics,
%Wuhan Institute of Physics and Mathematics, Innovation Academy of Precision Measurement Science and Technology, Chinese Academy of Sciences, Wuhan, 430071, China\\
%$^{4}$Research Center for Quantum Precision Measurement, Institute of Industry Technology, Guangzhou and
%Chinese Academy of Sciences, Guangzhou 511458, China


\begin{abstract}
The negatively charged silicon-vacancy center in diamond has great potential for quantum information processing due to its strong zero-phonon line emission, narrow inhomogeneous broadening, and stable optical transition frequencies. Developing universal quantum computation in silicon-vacancy centers is highly expected. Here, we propose a scheme for nonadiabatic geometric quantum computation in the system, in which silicon-vacancy centers are placed in a one-dimensional phononic waveguide. To improve the performance of the scheme, dynamical decoupling pulse sequences are used to eliminate the impact of the environment on its system. 
This scheme has the feature of geometric quantum computation that is robust to control errors and has the advantage of dynamical decoupling that is insensitive to environmental impact. Moreover, the feature that qubits are encoded in long-lifetime ground states of silicon-vacancy centers can reduce the decoherence caused by decay. Numerical simulation shows the effectiveness of the silicon-vacancy center system for quantum computation and the improvement of dynamic decoupling pulse in quantum system immunity to environmental noise.
 Our scheme may provide a promising way toward high-fidelity geometric quantum computation in the solid-state system.
\end{abstract}



\maketitle

\section{Introduction}

Color centers system in diamond has great potential in many applications, such as quantum computation~\cite{doi:10.1063/5.0007444,PhysRevA.105.012611,PhysRevA.100.052332,Zhou:15,PhysRevA.85.042306}, state detection~\cite{PhysRevLett.92.076401}, coherent manipulation~\cite{PhysRevA.96.032342}, nanoscale sensing~\cite{Rondin_2014}, and so on.
  As one of the most widely studied solid defects in diamond, nitrogen-vacancy~(NV) center~\cite{DOHERTY20131,doi:10.1063/PT.3.2549} in diamond, due to its bright and stable luminescence properties and long electron spin coherence time, it is widely used in quantum computation~\cite{PhysRevA.105.012611,PhysRevA.100.052332,Zhou:15,PhysRevA.85.042306}, state detection~\cite{PhysRevLett.92.076401}, coherent manipulation~\cite{PhysRevA.96.032342}, nanoscale sensing applications~\cite{Rondin_2014}. However, these developments are limited, and the characteristics of weak and unstable optical transitions hinder the further development of the NV centers~\cite{Bernien2013}. Recently, the negatively charged silicon-vacancy~(SiV$^-$)~(abbreviated as SiV) center has triggered great attention. The SiV center is formed by replacing two adjacent carbon atoms in the diamond lattice with a silicon atom, and the silicon atom is located between two vacancies. Due to its $D_{3d}$ point group symmetry that can protect it from optical inhomogeneity~\cite{doi:10.1126/science.aah6875,Dietrich_2014,https://doi.org/10.1002/pssa.201700586,PhysRevB.98.035306,PhysRevB.97.205444,PhysRevApplied.11.044022,PhysRevLett.128.203603,PhysRevLett.112.036405}. The SiV center can realize practicable initialization and readout of qubits~\cite{PhysRevLett.113.263602,Pingault2017}. The scalability of SiV centers can be achieved by one-dimensional~(1D) phononic waveguide~\cite{PhysRevLett.120.213603,PhysRevA.105.032415} and photonic crystal cavity~\cite{Li2020}. 
In addition, the narrow width~(around 5 nm)~\cite{PhysRevX.8.021063}, the zero-phonon line emission~\cite{HAuBler_2017}, and optical coherence properties of SiV centers in diamond lay the foundation for quantum computation~\cite{Ladd2010}. So far, the SiV center has been very much in favor of different quantum information processes, including entanglement generation~\cite{Li2020,PhysRevA.101.042313}, spin-squeezed states preparation~\cite{PhysRevA.103.013709,https://doi.org/10.1002/qute.202000034}, topological phase simulation~\cite{PhysRevResearch.2.013121}, etc. The SiV center could have important applications in quantum computation.


  %Quantum computation provides an effective method to resolve some hard problems for classical computers, such as factoring large integers~\cite{doi:10.1137/S0097539795293172} and searching unsorted data~\cite{PhysRevLett.79.325} due to its intrinsic entanglement and superposition principle. % The realization of quantum computation requires a general single-qubit quantum logic gate and a non-trivial two-qubit quantum logical gate~\cite{PhysRevLett.89.247902}.
   %The practical implementation of quantum computation depends on high-fidelity quantum logic gates. However, there are two obstacles to the realization of high-fidelity quantum logic gates. One of them is the control error caused by imperfect operations of the quantum system, and the other one is the decoherence caused by the interaction between the quantum system and the environment. At present, quantum computation in SiV center has not been studied.


  The geometric phase only depends on the global characteristics of the evolution instead of the evolution details, it has built-in noise-resilience features against certain local noises, providing a great resource of fault-tolerant quantum computation~\cite{PhysRevA.70.042316,PhysRevA.72.020301}. From the initial adiabatic Abelian phase~(Berry phase)~\cite{doi:10.1098/rspa.1984.0023} to the adiabatic non-Abelian phase~\cite{PhysRevLett.52.2111}, and then to the nonadiabatic Abelian phase~\cite{ANANDAN1988171} and nonadiabatic non-Abelian phase~(Aharonov-Anandan phase)~\cite{PhysRevLett.58.1593}. Based on these, adiabatic Abelian geometric quantum computation~(GQC)~\cite{Jones2000}, adiabatic non-Abelian GQC~\cite{ZANARDI199994}, nonadiabatic Abelian GQC~(NGQC)~\cite{PhysRevLett.87.097901,PhysRevLett.89.097902}, and nonadiabatic non-Abelian GQC~\cite{Sj_qvist_2012,PhysRevLett.109.170501} have been proposed successively.
  In recent years, various optimization methods, such as optimal control~\cite{PhysRevA.102.042607,Yun:21,PhysRevA.103.062607}, the time-optimal technique~\cite{PhysRevApplied.10.054051,PhysRevApplied.14.064009}, the shortened path method~\cite{doi:10.1063/5.0071569,https://doi.org/10.1002/qute.202000140}, the noncyclic shceme~\cite{PhysRevResearch.2.043130}, the reverse engineering scheme~\cite{PhysRevResearch.4.013233,PhysRevResearch.2.023295,Li_2021,PhysRevA.103.032609} and so on~\cite{PhysRevLett.123.100501,PhysRevResearch.3.023104,PhysRevResearch.3.L032066,https://doi.org/10.1002/qute.201900013,https://doi.org/10.1002/andp.202100057,PhysRevA.103.032616}, have been considered to combine with NGQC.  Some of them are demonstrated in different platforms, including superconducting circuits~\cite{PhysRevLett.124.230503,PhysRevLett.122.080501}, trapped ions~\cite{PhysRevLett.127.030502,PhysRevApplied.14.054062}, NV centers in diamond~\cite{PhysRevApplied.16.024060,PhysRevLett.122.010503}. These works demonstrate the meaning and robustness of geometric quantum computation with respect to control error.


On the other hand, the quantum system will inevitably be influenced by its environment, destroying the quantum information therein. A great of methods have been proposed to protect quantum information processing, including decoherence-free subspace~(DFS)~\cite{PhysRevLett.81.2594,PhysRevLett.95.130501,PhysRevLett.85.1762,doi:10.1126/science.290.5491.498,Lidar2003,PhysRevLett.85.1758,PhysRevLett.109.170501,PhysRevLett.97.140501}, noiseless subsystems~\cite{PhysRevLett.96.050501,doi:10.1126/science.1064460,PhysRevA.89.042302,PhysRevA.67.062303}, and dynamical decoupling~(DD)~\cite{2014,PhysRevLett.82.2417,doi:10.1098/rsta.2011.0355,PhysRevLett.95.180501,Biercuk2009}. Among these methods, DD is attractive due to its low resource consumption and excellent results. DD counteracts the interaction between the system and the environment by using some suitable external instantaneous intense pulse sequences, which can effectively improve the immunity of the quantum system to the external environment. These rapid intense pulse sequences can be regarded as a generalization of spin-echo experiment~\cite{PhysRev.80.580} that approximately eliminated the effect of unwanted interaction. By choosing appropriate external local control operations, the effect of unwanted interaction can be eliminated approximately. So far, DD has attracted widespread attention in both experiment and theory~\cite{PhysRevA.90.022323,PhysRevA.103.012205,Wu2021,PhysRevA.102.032627,PhysRevA.93.022304,PhysRevA.91.042325,PhysRevA.86.050301}. In other words, DD~\cite{PhysRevLett.82.2417,PhysRevLett.83.4888} can average the decoherence caused by the interaction between qubits and its environment to zero, which is an effective tool to protect quantum logic gates~\cite{PhysRevA.103.012205,PhysRevA.102.032627}. In summary, combining NGQC with DD method to realize quantum computation may be a promising way.


In this work, we show dynamical-decoupling-protected NGQC in a hybrid system, where SiV centers are placed in a 1D phononic diamond waveguide. The distance between two adjacent SiV centers is fixed, and SiV centers are coupled by strong strain. The Hamiltonian of the system is simplified to two-level and the qubits are encoded in long-lifetime ground states of SiV centers, which is convenient for realizing quantum gates. To further improve the system's immunity to the environment, we apply DD pulse sequences, which use a series of rapid pulses to eliminate the impact of the environment on the system. This work may present a new possibility for realizing NGQC in SiV centers, due to the following advantages and interests. First, SiV centers are stable and easy to operate, and qubits are insensitive to spontaneous emission. Second, the geometric phase is used to realize NGQC, which has built-in robustness against certain local noises. Finally, DD pulse sequences almost eliminate the impact of the environment on the system, that is to say, our scheme is immunity to decoherence caused by the environment.



This paper is organized as follows. In Sec.~\ref{siv}, the effective Hamiltonian of SiV centers placed in a 1D phononic waveguide is discussed. 
   The general form of dynamical-decoupling-protected Hamiltonian satisfying nonadiabatic geometric conditions is given in Sec.~\ref{ham}. The single-qubit gate and the nontrivial two-qubit gate are constructed in Sec.~\ref{sec4b} and Sec.~\ref{sec4c}, respectively, and decoupling sequences are added to improve the performance of quantum gates. 

\section{The physical model}
\label{siv}
We consider $N$ SiV centers in a 1D phononic waveguide~[see Fig.~\ref{fig1}~(a)], where the distance between two adjacent SiV centers is fixed, and the atomic structure of the SiV center is shown in Fig.~\ref{fig1}~(b). 

\begin{figure}[htbp]
	\centering \includegraphics[width=\linewidth]{model}
	\caption{ (a) Illustrative schematic. $N$ SiV centers in a phononic waveguide at fixed positions and the distance between two adjacent SiV centers is fixed. The upper right corner is a diagram of the direction of the external magnetic field. (b) The energy level configuration of the $j$th SiV center. (c) The effective energy level structure.}
	\label{fig1}
\end{figure}
The structure of negatively charged SiV we consider is determined by the spin-orbit interaction, the Jahn-Teller~(JT) effect, and the Zeeman splittings. The Hamilton of the SiV center can be written as 
\begin{eqnarray}
H_{\rm SiV}=-\lambda_{\rm SO}L_z S_z+H_{JT}+f\gamma_L \vec{B}\cdot \vec{L}+\gamma_s \vec{B}\cdot \vec{S},
\end{eqnarray}
where $\lambda_{\rm SO}$ represents the strength of the spin orbit coupling, $\gamma_L$ and $\gamma_s$ are the orbital and spin gyromagnetic ratio, respectively. Suppose the external magnetic field is tilted from the positive direction along the $z$-axis, so the Zeeman splitting of Hamiltonian can be expressed as $\gamma_sB_zS_z+\gamma_sB_xS_x$. The strength of the JT effect coupling along $x$~$(y)$ can be denoted as $\Upsilon_x$~$(\Upsilon_y)$.
And the orbital Zeeman effect is too small to neglect.
In the present system, only four ground states and the first excited state are considered. The Hamiltonian of a single SiV center can be expressed as 
\begin{eqnarray}
H_{\rm SiV} = \sum_{i=1}^4\omega_i |i\rangle |\langle i|+\omega_a|A\rangle\langle A|,
\end{eqnarray}

where \begin{eqnarray}
|1\rangle & \equiv&|e_-\searrow\rangle \approx|e_-\downarrow\rangle-\eta_+|e_-\uparrow\rangle,\ \ \notag\\
|2\rangle &\equiv &|e_+\nearrow\rangle\approx |e_+\uparrow\rangle-\eta_-|e_+\downarrow\rangle,\ \ \notag\\
|3\rangle &\equiv&|e_+\searrow\rangle \approx |e_+\downarrow\rangle+\eta_-|e_+\uparrow\rangle,\ \   \notag\\
|4\rangle &\equiv& |e_-\nearrow\rangle\approx |e_-\uparrow\rangle+\eta_+|e_-\downarrow\rangle,\ \   \notag
\end{eqnarray}
with $\omega_1\approx-\frac{\Delta+\omega_B}{2}-\frac{\eta_+\omega_x}{2},\ \omega_2\approx-\frac{\Delta-\omega_B}{2}-\frac{\eta_-\omega_x}{2},\ \omega_3\approx\frac{\Delta-\omega_B}{2}+\frac{\eta_+\omega_x}{2},\ \omega_4\approx\frac{\Delta+\omega_B}{2}+\frac{\eta_+\omega_x}{2},\ \eta_{\pm}=\frac{1}{2}\frac{\omega_x}{\Delta\pm\omega_B},\ \Delta=\sqrt{\lambda^2_{\rm SO}+4(\Upsilon_x^2+\Upsilon_y^2)},\ |A\rangle=|E\downarrow\rangle\langle E\downarrow|$ The $|e_\pm\rangle$ are the eigenstates of the angular momentum operator and the up and down arrow denote the spin-up and spin-down state of the spin projections. In addition, the spin-orbit interaction in the excited state $|E\downarrow\rangle$ is too large to neglect the effect of $B_x$ on the excited state.

We can find that each basis vector in the ground state subspace contains spin-up and spin-down components when the magnetic field and SiV axes are not oriented in the same direction, so the ground state basis vectors can be coupled to all energy levels. 
 

 
 Two lasers coupling  $|2\rangle\leftrightarrow|A\rangle$ and  $|3\rangle\leftrightarrow|A\rangle$ with Rabi frequencies of $\Omega_{A2}$ and $\Omega_{A3}$ are used simultaneously, and the frequency are $\omega_{A2}$ and $\omega_{A3}$, respectively, as illustrated in Fig.~\ref{fig1}~(b).
 The driving Hamiltonian can be written as 
\begin{eqnarray}
H_{\rm d}=\frac{\Omega_{A2}}{2}|A\rangle\langle2|e^{i\omega_{A2}t}+\frac{\Omega_{A3}}{2}|A\rangle\langle 3|e^{i\omega_{A3}t}+\rm H.c.\ .
\end{eqnarray}
The detuning is $\delta_1=\omega_a-\omega_{A3}-\omega_3=\omega_a-\omega_{A2}-\delta$.

As for the strain coupling caused by the small displacement of the defect atoms, the interaction in the Born-Oppenheimer approximation~\cite{PhysRevB.97.205444,PhysRevB.94.214115,PhysRevLett.120.213603} can be approximated as
\begin{eqnarray}
H_s=\sum_{n,j,k}g_{j,k,n}\hat{a}_{j,k}\hat{J}_+^n e^{ikx_n}+\rm H.c.,
\end{eqnarray}
where $\hat{J}_-=\hat{J}_+^\dagger=|1\rangle\langle3|+|2\rangle\langle4|$ means the lowering operator of the $k$ the mode of the $j$th  branch of the $n$th SiV center, $g$ is the coupling strength, $a_{j,k}$ ($a^\dagger_{j,k}$) is the annihilation (creation) operator of the $k$th mode of the $j$th branch at frequency of $\omega_{j,k}$, and $x_n$ denotes the position of the $n$th SiV center. So the Hamiltonian of phonon modes can be described by
\begin{eqnarray}
\hat{H}_{ph}=\sum_{j,k} \omega_{j,k}\hat{a}^\dagger_{j,k}\hat{a}_{j,k}.
\end{eqnarray}
Then, the full Hamiltonian of the system is 
\begin{eqnarray}
H_{\rm full}=H_{\rm SiV}+H_{\rm ph} +H_d +H_s
\label{eqfull}
\end{eqnarray}

In the interaction picture, by considering the rotating-wave approximation and using effective Hamiltonian theory~\cite{doi:10.1139/p07-060,Brion_2007} with the condition of $\delta_1,\delta\gg \Omega_{A2},\Omega_{A3}$, the Hamiltonian of the $n$th SiV center can be rewritten as
\begin{small}
\begin{eqnarray}
H_n(t)=\frac{\Omega}{2} |3\rangle\langle 2|e^{i\delta t} +ga(|3\rangle\langle 1|e^{i\Delta_1 t}+|4\rangle\langle 2|e^{i\Delta_2 t}) +\rm H.c.,\notag\\
\label{eq18}
\end{eqnarray} 
\end{small}
where $\Omega=-\Omega_{A2}^*\Omega_{A3}(2\delta_1+\delta)/ 4\delta_1(\delta_1+\delta)$, and the Stark shift can be compensated by lasers~\cite{PhysRevLett.97.083002,PhysRevLett.108.206401}.  

Then, in the condition of $\Delta_1, \Delta_2\gg g, \Omega$, the effective Hamiltonian can be denoted as 
\begin{eqnarray}
H_{n \rm eff}= \frac{\Omega_{\rm eff}}{2} a^\dagger |1\rangle \langle 2|e^{i(\delta-\Delta_1)t}+\rm H.c.
\label{eq7}
\end{eqnarray}
with $\Omega_{\rm eff}=\Omega g(\Delta_1+\delta)/4\Delta_1\delta$.
We can find that its form is consistent with the Jaynes-Cummings model that can realize quantum computation~\cite{doi:10.1080/09500349314551321}.

\section{Dynamical-decoupling-protected nonadiabatic geometric Hamiltonian}
\label{ham}
In this section, we will use the reverse-engineering scheme to give the target Hamiltonian for the two-level system satisfying nonadiabatic geometric conditions~\cite{PhysRevResearch.2.023295}.
 For the two-level system, the orthogonal auxiliary bases can be chosen as
\begin{eqnarray}
\begin{split}
 |\varphi_1(t)\rangle&=\sin{\frac{\theta(t)}{2}}e^{-i\phi(t)}|0\rangle-\cos{\frac{\theta(t)}{2}}|1\rangle,&\\
|\varphi_2(t)\rangle&=\cos{\frac{\theta(t)}{2}}|0\rangle+\sin{\frac{\theta(t)}{2}e^{i\phi(t)}}|1\rangle,&
\end{split}
\end{eqnarray}
where the $|0\rangle~(|1\rangle)$ is the logical qubit, and $\theta(t)$ and $\phi(t)$ ate the time-dependent parameters.
Then, the Hamiltonian of the two-level system can be expressed as
\begin{eqnarray}
\label{eq2}
H(t)&=&i\sum_{k\neq l}^{2}\langle \varphi_l(t)|\dot{\varphi_k}(t)\rangle|\varphi_l(t)\rangle\langle\varphi_k(t)|\notag\\
&=&\Delta(t)(|1\rangle\langle1|-|0\rangle\langle0|)+[\frac{\Omega(t)}{2}|1\rangle\langle0|+\rm H.c.].
\end{eqnarray}
Then, we have $\Omega(t)=i e^{i\phi(t)} [\dot{\theta}(t)+i\cos{\theta}(t)\sin{\theta(t)}\dot{\phi}(t)]$ and $\Delta(t)=-\frac{1}{2}\sin^2{\theta(t)}\dot{\phi}(t)$.

Corresponding to the auxiliary basis vectors, the initial states of the system are
\begin{eqnarray}
\begin{split}
|\psi_1(0)\rangle&= |\varphi_1(0)\rangle=\sin{\frac{\theta(0)}{2}}e^{-i\phi(0)}|0\rangle-\cos{\frac{\theta(0)}{2}}|1\rangle,&\\
     |\psi_2(0)\rangle&=|\varphi_2(0)\rangle=\cos{\frac{\theta(0)}{2}}|0\rangle+\sin{\frac{\theta(0)}{2}e^{i\phi(0)}}|1\rangle.
\end{split}
\end{eqnarray}
After a cyclic evolution, the evolution operator $U(T)=e^{-i\gamma(T)}|\psi_1(0)\rangle\langle\psi_1(0)|+e^{i\gamma(T)}|\psi_2(0)\rangle\langle\psi_2(0)|$ can be written as
\begin{eqnarray}
&&\left(
\begin{array}{cc}
\cos{\gamma(T)}+i \cos{\theta_0}\sin{\gamma(T)}                           &ie^{-i\phi_0}\sin{\gamma(T)}\sin{\theta_0}    \\
ie^{i\phi_0}\sin{\gamma(T)}\sin{\theta_0}                &\cos{\gamma(T)}-i \cos{\theta_0}\sin{\gamma(T)}
\end{array}
\right)\nonumber\\
&=&e^{-i\gamma(T)\bf n\cdot \boldsymbol{\sigma}},
\end{eqnarray}
where $\theta_0\equiv \theta(0)$, $\phi_0\equiv \phi(0)$, $\gamma(T)=i \int_0^T\langle \varphi_1(t)|\dot{\varphi_1}(t)\rm dt$$= -i \int_0^T\langle \varphi_2(t)|\dot{\varphi_2}(t)\rm dt$ $=\frac{1}{2}\int_0^T [1-\cos\theta(t)\dot{\phi}(t)]\rm dt$ $=\frac{1}{2}\oint_C [1-\cos\theta(t)] d\phi$, it can be clearly seen that the $\gamma(T)$ is half of the solid angle, independent of the evolution details, and $\bf{n}=$ $[\sin\theta(0)\cos\phi(0),\ \sin\theta(0)\sin\phi(0),\ \cos\theta(0)]$, $\boldsymbol{\sigma}$ = $(\sigma_x,\ \sigma_y,\ \sigma_z)$. The schematic diagram of the evolution path is shown in Fig.~\ref{bloch}(a). General single-qubit gates can be constructed by selecting different parameters. Universal quantum computation can be realized by a non-trivial two-qubit quantum gate assist with an arbitrary single-qubit gate~\cite{PhysRevLett.89.247902}. 

For the two-qubit gate, the auxiliary bases can be chosen as 
\begin{eqnarray}
&&|\varphi_1(t)\rangle=|00\rangle,\notag\\
&&|\varphi_2(t)\rangle=\cos{\frac{\theta(t)}{2}}|01\rangle+\sin{\frac{\theta(t)}{2}e^{i\phi(t)}}|10\rangle,\notag\\
&&|\varphi_3(t)\rangle=\sin{\frac{\theta(t)}{2}}e^{-i\phi(t)}|01\rangle-\cos{\frac{\theta(t)}{2}}|10\rangle,\notag\\
&&|\varphi_4(t)\rangle=|11\rangle.
\end{eqnarray}
The general form of Hamiltonian of the two-qubit gate is 
\begin{eqnarray}
H_{\rm two}(t)&=&i\sum_{k\neq l}^{4}\langle \varphi_l(t)|\dot{\varphi}_k(t)\rangle|\varphi_l(t)\rangle\langle\varphi_k(t)|\notag\\
&=&\frac{\Omega_{\rm two}(t)}{2}|01\rangle\langle10|+\rm H.c..
\label{eq6}
\end{eqnarray}


The evolution operator of this system $U_2$ can be expressed as 
\begin{widetext}
	\begin{eqnarray}
U_2(T)&=&|\varphi_1(T)\rangle\langle\varphi_1(0)|+e^{-i\gamma(T)}|\varphi_2(0)\rangle\langle\varphi_2(0)|+e^{i\gamma(T)}|\varphi_3(0)\rangle\langle\varphi_3(0)|+\varphi_4(T)\rangle\langle\varphi_4(0)|\notag 
\\
&=&\begin{pmatrix}
    1&0&0&0\\    0&\cos{\gamma(T)}-i\cos{\theta(0)}\sin{\gamma(T)}&-ie^{-i\varphi(0)}\sin{\theta(0)}\sin{\gamma(T)}&0\\
    0&-ie^{i\varphi(0)}\sin{\theta(0)}\sin{\gamma(T)}&\cos{\gamma(T)}+i\cos{\theta(0)}\sin{\gamma(T)}&0\\
    0&0&0&1\\
 \end{pmatrix}.\notag\\
	\end{eqnarray}
\end{widetext}

%\begin{eqnarray}
%U_2(T)&=&|\varphi_1(T)\rangle\langle\varphi_1(0)|+e^{-i\gamma(T)}|\varphi_2(0)\rangle\langle\varphi_2(0)|+e^{i\gamma(T)}|\varphi_3(0)\rangle\langle\varphi_3(0)|+\varphi_4(T)\rangle\langle\varphi_4(0)|\notag\\
%&=&\begin{pmatrix}
 %   1&0&0&0\\    0&\cos{\gamma(\tau)}-i\cos{\theta(0)}\sin{\gamma(\tau)}&-ie^{-i\varphi(0)}\sin{\theta(0)}\sin{\gamma(\tau)}&0\\
  %  0&-ie^{i\varphi(0)}\sin{\theta(0)}\sin{\gamma(\tau)}&\cos{\gamma(\tau)}+i\cos{\theta(0)}\sin{\gamma(\tau)}&0\\
  %  0&0&0&1\\
 %\end{pmatrix}.\notag\\
%\end{eqnarray}
 By selecting appropriate parameters, we can build a nontrivial two-qubit quantum geometric gate. The general geometric quantum computation can be achieved by combining universal single-qubit gates.
  \begin{figure}[htbp]
	\centering \includegraphics[width=\linewidth]{bloch.pdf}
	\caption{The evolution paths for the single-loop NGQC scheme (a) without and (b) with impaction of the environment and decoupled pulses~(take the basic pulse as an example).}
\label{bloch}
\end{figure}

In the process of gate implementation, evolution inevitably receives the influence of the surrounding environment, which can lead to a reduction in fidelity. Next, a set of general pulse sequences that can cancel the system and environment are derived.

Here we consider the system and environment coupling Hamiltonian~\cite{2014,doi:10.1098/rsta.2011.0355,PhysRevLett.95.180501}
 \begin{eqnarray}
 H_{\rm se}=H_{\rm s}\otimes H_{\rm e}=\sum_{\alpha=x,y,z}\sigma^\alpha \otimes B^\alpha,
 \end{eqnarray}
 where $\sigma^\alpha $ denotes the system Hamiltonian and $B^\alpha$ expresses the environment operator.
The system evolution operator expressed as $f_\tau=e^{-i\tau H_{\rm se}}$, we assume these pulses last for a period of time $\mathcal{D}$ with strength $\lambda$, and $\mathcal{D}\lambda=\frac{\pi}{2}$, in the ideal case, $\mathcal{D}\rightarrow 0$, $\lambda\rightarrow \infty$. Now, we apply two X-pulse $X=e^{i\mathcal{D}\lambda\sigma^x}\otimes  I_B=e^{-i\frac{\pi}{2}\sigma^x}\otimes I_B=-i\sigma^x\otimes I_B$ on the system, we have
\begin{equation}
\sigma^xH_{\rm se}\sigma^{x}=\sigma^x\otimes B^x -\sigma^y\otimes B^y-\sigma^z\otimes B^z,
\end{equation}
which means that $Xf_\tau Xf_\tau$ pluse sequence can cancel both the $y$ and $z$ contributions. Evolution can be described as 
\begin{small}
\begin{eqnarray}
Xf^\prime_{2\tau}=Xf_\tau Xf_\tau=e^{-2i\tau(\sigma^x\otimes B^x+H_e)}+\mathcal{O}(\tau^2).
\end{eqnarray}
\end{small}
Then we apply Y-pulse $Y=e^{i\mathcal{D}\lambda\sigma^y}\otimes  I_B=e^{-i\frac{\pi}{2}\sigma^y}\otimes I_B=-i\sigma^y\otimes I_B$ on the system, 
\begin{small}
	\begin{eqnarray}
	Yf^\prime_{2\tau} Yf^\prime_{2\tau}&=&YXf_\tau Xf_\tau YX f_\tau Xf_\tau\\\notag
	&=&Zf_\tau Xf_\tau Zf_\tau Xf_\tau\\\notag
	&=&e^{-i4\tau H_e}+\mathcal{O}(\tau^2),
	\end{eqnarray}
\end{small}
 so when evolution time $t=4\tau$, the system can decouple from the environment, i.e., $ZXZX$ are universal decoupling pulse sequences that can reduce the impact of the environment on the system. Here, we can repeat the basic DD sequence periodically named periodic dynamical decoupling~\cite{PhysRevA.90.022323} to further enhance the performance of the system, the basic pulse sequence (four-beam pulse), the pulse sequence with a period of two (eight-beam pulse), and the pulse sequence with a period of three (twelve-beam pulse) are shown in Fig.~\ref{pluse} (a), (b), and (c), respectively. 
 \begin{figure}[htbp]
	\centering \includegraphics[width=\linewidth]{pluse.pdf}
	\caption{Schematic diagram of period decoupled pulses. (a) The base four-beam pulse. (b) The eight-beam pulse with period two. (c) The twelve-beam pulse with period three.}
\label{pluse}
\end{figure}

 Then, the decoherence caused by the interaction between the quantum system and the environment can be inhibited greatly. The schematic diagram of the evolution with the impaction of the environment and decoupled pulses is shown in Fig.~\ref{bloch}(b). Because of the interaction of the environment, the trajectory of the evolution state will have a slight deviation from the ideal situation, i.e., the evolution shown in Fig.~\ref{bloch}(a). This deviation can be eliminated by the dynamic decoupling pulses (represented by orange and blue arrows with solid lines), so as to achieve the ideal evolution.
Therefore, the dynamical-decoupling-protected nonadiabatic geometric quantum computation can be realized in the SiV centers by constructing the Hamiltonian in the form of Eq~.(\ref{eq2}) and Eq.~(\ref{eq6}) and adding the dynamic decoupling pulses.

\section{Implementation}
By combining a set of universal single-qubit gates and a nontrivial two-qubit gate, universal quantum can be realized, in this section, we will show how to use SiV centers to realize general geometric quantum logic gates.
\subsection{The single-qubit gate}
\label{sec4b}

Based on the effective Hamiltonian of the SiV center system Eq.~\ref{eq7}, we construct a single-qubit quantum gate, with only one SiV center placed on the 1D phononic waveguide, i.e. $n=1$.
For simplicity, we consider the single excitation mode, regard $|11\rangle$ and $|20\rangle$ as logical qubits $|0\rangle_L$ and $|1\rangle_L$, where $|11\rangle$~($|20\rangle$) is the abbreviation $|1\rangle\otimes|1\rangle$~($|2\rangle\otimes|0\rangle$) and the first qubit means the energy level of SiV center, the second qubit means the Fork state of phonon.  In this case, the Hamiltonian of the single qubit is in the same form as Eq.~(\ref{eq2}).
The degree of conformity between the Hamiltonian in Eq.~(\ref{eqfull}) and Eq.~(\ref{eq7}) is shown in Fig.~(\ref{fig2}), in which the red~(blue) dotted line represents the evolution of the population of $|0\rangle_L$~($|1\rangle_L$) in Eq.~(\ref{eqfull}), and the red~(blue) solid line indicates the evolution of the population of $|0\rangle_L$~($|1\rangle_L$) in Eq.~(\ref{eq7}). It can be seen that the effective Hamiltonian and the original Hamiltonian agree very well. 

\begin{figure}[htbp]
	\centering \includegraphics[width=\linewidth]{Fig2.eps}
	\caption{The comparison of the population of $|0\rangle_L$ and $|1\rangle_L$ from Eq.~(\ref{eqfull}) with
the Eq.~(\ref{eq7}), where $\Omega/2\pi=10\rm MHz$, $g/2\pi=5\rm MHz$, $\Delta_1/2\pi= 100\rm MHz$, $\Delta_2/2\pi=500\rm MHz$, $\delta=\Delta_1$. The red~(black) points and solid lines represent the population of $|0\rangle_L$~($|1\rangle_L$) evolved under the full and effective Hamiltonian, respectively.}
\label{fig2}
\end{figure}


We take a single qubit phase gate and a SWAP gate as  examples. For the phase gate, the polar angle~($\theta(t)$) and the azimuth angle~($\phi(t)$) starting from the north pole $(\theta(0)=0, \phi(0))$, passing through the south pole, a phase sudden change occurs~$(\theta(t)=\pi, \ \phi(t)=\phi(0)+\pi/4)$, and then returning to the north pole. 
The Hamiltonian can be expressed as
\begin{align}
\begin{split}
\left \{
\begin{array}{ll}
\frac{\dot{\theta}(t)}{2}[-\sin{\phi(0)\sigma_x}+\cos{\phi(0)}\sigma_y],                    & t\in [0, \frac{T}{2}],\\
    \frac{\dot{\theta}(t)}{2}[-\sin({\phi(0)+\frac{\pi}{4})\sigma_x}+\cos({\phi(0)}+\frac{\pi}{4})\sigma_y],     & t\in (\frac{T}{2}, T],
\end{array}
\right.
\end{split}
\label{eq19}
\end{align}
and the parameters meet $\int_0^{T/2}\dot{\theta}(t)\rm dt=\pi/2$,  $\int_{T/2}^{T}\dot{\theta}(t)\rm dt=-\pi/2$.

For the SWAP gate~(ignore the global phase), the polar angle and the azimuth angle starting from $(\theta(0)=\frac{\pi}{2},\  \phi(0)=0)$, passing the south pole and the north pole and back to the starting point. The parameters in the Hamiltonian quantity satisfy
\begin{align}
\left \{
\begin{array}{ll}
\int_0^{T_1} \dot{\theta}(t) dt=\theta(0), \ \ \ \ \phi(t)=0,  & t\in [0, T_1],\\[2mm]
 \int_{T_1}^{T_2} \dot{\theta}(t) dt=\pi, \ \ \ \ \phi(t)=-\frac{\pi}{2}  ,     & t\in (T_1, T_2],\\[2mm]
\int_{T_2}^{T} \dot{\theta}(t) dt=\pi-\theta(0), \ \ \  \phi(t)=0 ,     & t\in (T_2, T].
\end{array}
\right.
\label{eqH}
\end{align} 

Then, add the decoupling sequence in Fig.~\ref{pluse} to Hamiltonian to further improve the performance of the gate. We use the fidelity defined as 
\begin{eqnarray}
F=\langle \psi_{\rm ideal}|\rho(T)|\psi(T)\rangle
\label{fidelity}
\end{eqnarray}
to evaluate the performance of the gate, where $|\psi_{ideal}\rangle =U(T)|\psi(0)\rangle$ is the ideal final state, the $\psi(T)$ is the actual final state, and the $\rho(T)$ is the density matrix. 

\begin{figure}[htbp]
	\centering \includegraphics[width=\linewidth]{Fig6.eps}
	\caption{ The fidelity of the SWAP gate driven by Eq.~(\ref{eq19}). (a) The fidelity without being protected by dynamical decoupling. (b) The fidelity with being protected by four decoupled pulse sequences dynamical decoupling. (c) The fidelity with being protected by eight decoupled pulse sequences dynamical decoupling. (d) The fidelity with being protected by twelve decoupled pulse sequences dynamical decoupling. The control parameters are set the same as Fig.~\ref{fig2}.}
	\label{figH}
\end{figure}

According to Ref.~\cite{PhysRevLett.105.230503}, the Liouville equation $\dot{\rho}(t)=-i[H_{\rm sum}(t),\rho(t)]$ can be used to calculate the density matrix, where $H_{\rm sum}(t)=H(t)+H_E(t)+H_I(t)$ is the total Hamiltonian, $H_E(t)$ is the Hamiltonian of the environment, and the $H_I(t)$ is the Hamiltonian of the interaction between the system and the environment. In the numerical simulations, we use the single-qubit SWAP (phase) gate $U(T)=\rm exp(-i\pi \sigma_x /2)$ ($U(T)=\rm exp(-i\pi \sigma_z /2)$), $\psi (0)=|0\rangle_L$, and the environment Hamiltonian $H_E(t)=G_1[I_{\rm sys}\otimes \sum _{k=x,y,z}\sigma_k]$ with $I_{\rm sys} $ means the identity matrix of the system, the $H_I(t)=G_2[H(t)\otimes \sum _{k=x,y,z}\sigma_k]$, where $\Omega_{\rm eff}/G_1=\Omega_{\rm eff}/G_2=3\times10^{3}$. From Fig.~\ref{figH}~(Fig.~\ref{fig3}), we can see the fidelity of the SWAP (phase) gate is significantly improved, when the decoupling pulse sequences are four, eight, and twelve, the fidelity can reach 0.9976~(0.9974), 0.999~(0.9989), and 0.9991~(0.9991), respectively. 

We further demonstrate the robustness of our scheme against the decoherence and dephasing coefficients impacted by environment~($G_1$ and $G_2$), which adjust the value of coefficients in numerical simulation. The fidelity of the SWAP gate~(phase gate) changing with the strength of the environmental interaction is shown in Fig.~\ref{figrobust}(a)((b)), where the black line represents the fidelity not protected by dynamic decoupling pulse, and the red line represents the fidelity protected by a group of the basic dynamic decoupling pulse (four decoupling pulses). From this, we can see that the dynamical decoupling pulse greatly protects the nonadiabatic geometric quantum gates from environment-induced decoherence and decay.


\begin{figure}[htbp]
	\centering \includegraphics[width=\linewidth]{Fig3.eps}
	\caption{The fidelity of the phase gate driven by Eq.~(\ref{eq19}). (a) The fidelity without being protected by dynamical decoupling. (b) The fidelity with being protected by four decoupled pulse sequences dynamical decoupling. (c) The fidelity with being protected by eight decoupled pulse sequences dynamical decoupling. (d) The fidelity with being protected by twelve decoupled pulse sequences dynamical decoupling. The control parameters are set the same as Fig.~\ref{fig2}.}
\label{fig3}
\end{figure}


\begin{figure}[htbp]
	\centering \includegraphics[width=\linewidth]{Fig7.eps}
	\caption{The fidelity of the phase gate~(a) and the SWAP gate~(b) varying with the strength of environmental interaction. The control parameters are set the same as Fig.~\ref{fig2}.}
\label{figrobust}
\end{figure}

\subsection{The two-qubit quantum gate}
\label{sec4c}
Here, we consider two SiV centers in the phononic waveguide, we define the effective detuning  $\delta-\Delta_1=\Lambda_1$, and the  Hamiltonian 
\begin{eqnarray}
    H_n=\frac{\Omega_{\rm eff}}{2}a^\dagger|1\rangle\langle2|e^{i\Lambda_n t}
    \label{eq22}
\end{eqnarray}
 of the $n$th SiV center  when $\Lambda_n\gg \Omega_{\rm eff}$, the effective Hamiltonian of the system can be written as
\begin{eqnarray}
H_{\rm two}=-\Omega_{\rm eff}^2\frac{\Lambda_1+\Lambda_2}{8\Lambda_1\Lambda_2}\sigma_1^-\sigma_2^+ e^{i (\Lambda_1-\Lambda_2) t}+\rm H.c.,
\label{eq23}
\end{eqnarray}
where $\sigma_1^-$ denotes the energy level decline of the first SiV center, $\sigma_2^+$ means the energy level rise of the second SiV center. Now, similar to the single qubit gate, we code $|1\rangle_{\rm SiV}$~($|2\rangle_{\rm SiV}$) as logical qubit $|0\rangle_L$~($|1\rangle_L$), so the Hamiltonian in Eq.~(\ref{eq23}) can be written in the following form in the basic $|00\rangle_L\ |01\rangle_L\ |10\rangle_L\ |11\rangle_L $,
\begin{eqnarray}
H_2=-
\begin{pmatrix}
    0&0&0&0\\
    0&\Lambda_2-\Lambda_1&\frac{\Omega^2_{\rm eff}(\Lambda_1+\Lambda_2)}{8\Lambda_1 \Lambda_2}&0\\
    0&\frac{\Omega^2_{\rm eff}(\Lambda_1+\Lambda_2)}{8\Lambda_1 \Lambda_2}&\Lambda_1-\Lambda_2&0\\
    0&0&0&0\\
\end{pmatrix}.
\end{eqnarray}
The reduced Hamiltonian of $|01\rangle_L$ and $|10\rangle_L$ is in  the same form as Eq.~(\ref{eq6}), and it can be seen from Fig.~\ref{fig4} that the original Hamiltonian and the effective Hamiltonian are highly consistent.
\begin{figure}[htbp]
	\centering \includegraphics[width=\linewidth]{Fig4.eps}
	\caption{The comparison of the population of $|01\rangle_L$ and $|10\rangle_L$ from Eq.~(\ref{eq18}) with
    the Eq.~(\ref{eq23}), where $\Omega/2\pi=10\rm~MHz$, $g/2\pi=5\rm~MHz$, $\Delta_1/2\pi= 80\rm~MHz$, $\Delta_2/2\pi=150\rm~MHz$, $\delta=20\rm~MHz$. The red~(black) points and solid lines represent the population of $|01\rangle_L$~($|10\rangle_L$) evolved under the full and effective Hamiltonian, respectively, where $\mathop{O}_{\rm eff}=\Omega^2_{\rm eff}\frac{\Lambda_1+\Lambda_2}{4\Lambda_1\Lambda_2}$.}
\label{fig4}
\end{figure}

Here, we choose $\Lambda_1=\Lambda_2$, and $\gamma(T)=\pi/2$, $\theta(0)=\pi/2$, and $\varphi(0)=\pi$, the iSWAP gate can be achieved.
The parameters in Hamiltonian meet 
\begin{align}
\left \{
\begin{array}{ll}
\int_0^{T_1} \mathop{O}_{\rm eff} dt=\theta(0), \ \ \ \ \phi(t)=\pi,  & t\in [0, T_1],\\[2mm]
 \int_{T_1}^{T_2} \mathop{O}_{\rm eff} dt=\pi, \ \ \ \ \phi(t)=\pi+\frac{\pi}{2}  ,     & t\in (T_1, T_2],\\[2mm]
\int_{T_2}^{T} \mathop{O}_{\rm eff} dt=\pi-\theta(0), \ \ \  \phi(t)=\pi  ,     & t\in (T_2, T].
\end{array}
\right.
\label{eq26}
\end{align} 
The azimuth angle $\phi$ changes twice at the south pole and north pole.

The system Hamiltonian is $H_2(t)$ in Eq.~(\ref{eq18}), and the environment Hamiltonian and interaction Hamiltonian is $H_E(t)=G_1[I_{\rm sys}\otimes \sum _{k=x,y,z}\sigma_k]$ and $H_I(t)=G_2[H_2(t)\otimes \sum _{k=x,y,z}\sigma_k]$, the parameters setting are the same as Sec.~\ref{sec4b}.
\begin{figure}[htbp]
	\centering \includegraphics[width=\linewidth]{Fig5.eps}
	\caption{The fidelity of the gate driven by Eq.~(\ref{eq26}). (a) The fidelity without being protected by dynamical decoupling. (b) The fidelity with being protected by four decoupled pulse sequences dynamical decoupling. (c) The fidelity with being protected by eight decoupled pulse sequences dynamical decoupling. (d) The fidelity with being protected by twelve decoupled pulse sequences dynamical decoupling. The control parameters are set the same as Fig.~\ref{fig4}.}
\label{fig9}
\end{figure}
The initial state is set as $|01\rangle_L $, the fidelity $F$ defined as Eq.~(\ref{fidelity}) is shown as Fig.~\ref{fig9}. Due to environmental impaction, the decoherence and dephasing coefficients ($G_1$ and $G_2$) compared with the effective Rabi frequency $\mathcal{O}_{\rm eff}$ of the model are too large to maintain high fidelity, as shown in Fig.~\ref{fig9}(a). So, add the decoupling sequence in Fig.~\ref{pluse} to Hamiltonian to further improve the performance of the gate. When the decoupling pulse sequences are four, eight, and twelve, the fidelity can reach 0.9595, 0.996, and 0.9992 respectively, as shown in Fig.~\ref{fig9}(b), (c), and (d). From the result of numerical simulations, the dynamical decoupling can protect the quantum system against the decoherence and dephasing coming from the environment.
The fidelity of the iSWAP gate varying with the strength of environmental interaction $G_1$ and $G_2$ is shown in Fig.~\ref{fig10} in which the black line with square marks represents the fidelity without dynamical decoupling sequences, and the red line with circle marks means fidelity protected by a set of basic dynamical decoupling sequence, i.e. the pulse sequences shown in Fig.~\ref{pluse}(a), we can see the fidelity can be protected greatly. 
\begin{figure}[htbp]
	\centering \includegraphics[width=7.5cm]{Fig10.eps}
	\caption{The fidelity of the iSWAP gate varies with the strength of environmental interaction $G_1$ and $G_2$. The control parameters are set the same as Fig.~\ref{fig9}.}
\label{fig10}
\end{figure}


\section{Conclusion}\label{sec5}
We use the SiV centers to realize universal NGQC for the first time, and the dynamical decoupling pulse sequences are used to improve the immunity of the quantum system to environmental impaction. The scalability can be achieved by a 1D phononic waveguide. Besides the scalability, our scheme is easy to initialize and read out. In summary, our scheme can protect quantum gates from both the control error caused by actual operation due to the intrinsic robustness of geometric quantum logic gates and the dephasing caused by the environment, which are two obstacles to the realization of high-fidelity quantum logic gates. In addition, logical qubits are encoded in long-lifetime ground states of SiV centers, which can reduce the decoherence caused by spontaneous emission. Numerical simulations show the superiority of our scheme in preparing quantum logic gates. Therefore, our scheme may be promising for implementing high-fidelity geometric quantum computation in the solid-state system. 


%In summary, we put forward a set of universal nonadiabatic geometric quantum gates where the logical qubits encoding on SiV centers are placed in a one-dimensional phononic waveguide, and the dynamical decoupling pulse sequences are used to improve the immunity of the system to environmental impaction. So, our scheme can protect quantum gates from both the control error caused by actual operation due to the intrinsic robustness of geometric quantum logic gates and the dephasing caused by the environment, which are two obstacles to the realization of high-fidelity quantum logic gates. Numerical simulations show the superiority of our scheme in preparing quantum logic gates. Therefore, our scheme may be promising for implementing high-fidelity geometric quantum computation in the solid-state system.

 \section{Acknowledgement}
This work was supported by the National Natural Science Foundation of China under Grants (No. 12274376, No. U21A20434, No. 12074346), and a major science and technology project of Henan Province under Grant No. 221100210400, and the Natural Science Foundation of Henan Province under Grant No.  232300421075 and 212300410085.



\bibliography{DFS}






\end{document}
