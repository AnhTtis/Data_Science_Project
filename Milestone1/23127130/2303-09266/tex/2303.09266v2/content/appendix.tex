\subsection*{A. Implementation Details}
\subsubsection{Algorithm}
The inference processes of \our~ can be described as Algorithm~\ref{algorithm1}.
\begin{algorithm}
	\renewcommand{\algorithmicrequire}{\textbf{Require:}}
	\caption{Inference ($Input: \mathbf{X}_0$)}
	\label{algorithm1}
	\begin{algorithmic}[1]
	\FOR{$i=1$ to $L$}
        \IF{$\mathcal{G}^i(\mathbf{X}^{i-1}) \geq 0.5$} 
        \STATE $\mathbf{\mathbf{X}^{i}} = \mathbf{\mathbf{X}^{i-1}}$ \\ 
        \IF{i $\neq$ L}
        \STATE{continue} 
        \ENDIF
        \ELSE
        \STATE $\mathbf{X}^{i} = \mathcal{E}^i(\mathbf{X}^{i-1})$
        \ENDIF \\
        $\mathbf{z}^i = \mathcal{C}^i(\mathbf{X}^i)$
        \IF{$\operatorname{Entropy}(\mathbf{z}^i)<S$}
        \STATE{return $\mathbf{z}^i$ }
        \ENDIF 
    \ENDFOR \\
    return $\mathbf{z}^L$
	\end{algorithmic}  
\end{algorithm}

\subsection*{B. The Impact of Different Entropy Thresholds}
We counted the number of skipping and exiting on each layer in different entropy thresholds \emph{S}. From the Fig \ref{fig:frequence}, We intuitively see that the early exiting classifiers and skipping gates can play their role in the inference stage.

\begin{figure}[htb]
    \centering
    \includegraphics[scale = 0.4]{picture/frequence.eps}
    \vspace{-1em}
    \caption{Statistics about how often the model uses skipping and early exiting in different entropy threshold \emph{S}.}
    \vspace{-1em}
    \label{fig:frequence}
\end{figure}

\subsection*{C. Motivation of Cross-Layer Contrastive Learning}
In cross-layer contrastive learning(CCL), we assume that each token should have similar semantics across the consecutive layers because their representations do not change drastically. Empirically, we find the representations of all tokens change slightly in the consecutive layers(i.e., the cosine similarity of the same token in the consecutive layer is over 0.9) according to Fig~\ref{fig:similarity}. This phenomenon verifies our motivation. As for [CLS] token which only changes drastically in the last layer, we will not use this layer to do CCL.

\begin{figure}[htb]
    \centering
    \includegraphics[scale = 0.4]{picture/similarity.eps}
    \vspace{-1em}
    \caption{Token representations' similarity of two consecutive layers. CLS represents similarity of [CLS] token representations in the two consecutive layers. AVG represents the average similarity of all tokens(except [CLS]).}
    \vspace{-1em}
    \label{fig:similarity}
\end{figure}

\subsection*{D. Ablation Study on More Datasets}
Table \ref{tab:table3} and Figure \ref{fig:contrast} only show ablation study on partial datasets due to the limited space. There, we provide more experiments in Table \ref{tab:table4} and Figure \ref{fig:supplement}. 

\begin{table}[htb]
\centering
{
\begin{tabular}{c|cc|cc|cc}
\hline
\makecell{Dataset/\\ Model} & \multicolumn{2}{c|}{MRPC}                                      & \multicolumn{2}{c|}{CoLA}   
& \multicolumn{2}{c}{QNLI}\\
\hline                                                                          
& F1  & \makecell{FLOPs\\ (cost)}
& MCC  & \makecell{FLOPs\\ (cost)}
& ACC  & \makecell{FLOPs\\ (cost)}  
   \\
\hline 
BERT                                                                      
& 88.5 & \makecell{21744\\ (100\%)} 
& 50.8 & \makecell{21744\\ (100\%)}
& 88.1 & \makecell{21744\\ (100\%)}
 \\
\hline
Skip                                                                      
& 88.4 & \makecell{14894\\ (68\%)}   
& 50.1 & \makecell{16400\\ (74\%)}
& 87.3 & \makecell{16568\\ (74\%)}
\\  
\hline
\end{tabular}}

\caption{Ablation study of skipping mechanism on other datasets.}

\label{tab:table4}
\end{table}

\begin{figure}[tbp]
    \centering
    \includegraphics[scale = 0.60]{picture/Supplement.eps}
    \vspace{-0.8em}
    \caption{Ablation study of CCL on other datasets.}
    \vspace{-0.8em}
    \label{fig:supplement}
\end{figure}






