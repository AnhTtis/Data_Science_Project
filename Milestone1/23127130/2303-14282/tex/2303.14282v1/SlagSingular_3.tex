\documentclass{amsart}
\usepackage{graphicx}
\usepackage{amssymb}
\usepackage{enumitem}

% THEOREMS -------------------------------------------------------
\newtheorem{thm}{Theorem}[section]
\newtheorem{cor}[thm]{Corollary}
\newtheorem{lem}[thm]{Lemma}
\newtheorem{prop}[thm]{Proposition}
\theoremstyle{definition}
\newtheorem{defn}[thm]{Definition}
\theoremstyle{remark}
\newtheorem{rem}[thm]{Remark}

% ----------------------------------------------------------------

\begin{document}

\title[non $C^1$ solutions]{Non $C^1$ solutions to the special Lagrangian equation}
\author{Connor Mooney}
\address{Department of Mathematics, UC Irvine}
\email{\tt mooneycr@math.uci.edu}
\author{Ovidiu Savin}
\address{Department of Mathematics, Columbia University}
\email{\tt savin@math.columbia.edu}
%\subjclass[2010]{}
%\keywords{special Lagrangian equation, regularity}

% ----------------------------------------------------------------
\begin{abstract}
We construct viscosity solutions to the special Lagrangian equation that are Lipschitz but not $C^1$.
\end{abstract}
\maketitle
% ----------------------------------------------------------------


%%%%%%%%%%%%%%%%%%%%%%%%%%%%%%%%%%%%%%%%%%%%%%%%%%%%%%%%%%%%%%%%%%%%%%%%%%%%%%%%%%%%%%%%%%%%%%%%%
\section{Introduction}
For a symmetric $n \times n$ matrix $M$ with eigenvalues $\{\lambda_i\}_{i = 1}^n$, we let
$$F(M) = \sum_{i = 1}^n \arctan(\lambda_i).$$
The special Lagrangian equation for a function $u$ on a domain in $\mathbb{R}^n$ is
\begin{equation}\label{sLag}
F(D^2u) = c \in (-n \frac \pi 2,\,n \frac \pi 2).
\end{equation}
Here $c$ is a constant. Equation (\ref{sLag}), introduced in the seminal work \cite{HL}, is the potential equation for area-minimizing Lagrangian graphs of dimension $n$ in $\mathbb{R}^{2n}$. 

The regularity of solutions to (\ref{sLag}) is a delicate issue. It is known that viscosity solutions are real analytic when $|c| \geq (n-2)\frac{\pi}{2}$ \cite{WY1}, or when $u$ is convex \cite{CSY}. In these cases, (\ref{sLag}) is a concave equation. When $|c| < (n-2)\frac{\pi}{2}$ the equation is not concave, and there are examples of viscosity solutions to (\ref{sLag}) which are $C^1$ but not $C^2$ (\cite{NV}, \cite{WY2}). However, the gradient graphs of these examples are analytic and area-minimizing as geometric objects. It remained open whether all viscosity solutions to (\ref{sLag}) are $C^1$, and whether they have minimal gradient graphs (see e.g. the conjecture at the end of the introduction in \cite{NV}). In this paper we answer these questions in the negative:

\begin{thm}\label{Main}
There exist $c \in [0,\,\pi/2)$, a smooth bounded domain $\Omega \subset \mathbb{R}^3$, an analytic embedded surface $\Gamma \subset \subset \Omega$ with boundary, and a Lipschitz function $u$ on $\Omega$ that is analytic in $\Omega \backslash \Gamma$, such that 
\begin{enumerate}
\item $F(D^2u) = c$ in the viscosity sense in $\Omega$,
\item $\nabla u$ is discontinuous on $\Gamma \backslash \partial \Gamma$, and
\item The graph of $\nabla u$ is of regularity $C^{1,\,1}$ but not $C^2$, thus it is not minimal.
\end{enumerate}
\end{thm}

\noindent Our approach is to first construct a $C^{2,\,1}$ solution $w$ to the degenerate Bellman equation
\begin{equation}\label{Bellman}
\text{max}\{F(D^2w) - c^*,\,\det D^2w\} = 0
\end{equation}
which has a compact free boundary between the operators, and then take the Legendre transform.

We will prove Theorem \ref{Main} in the following section. In the last section we will discuss related examples of singular solutions to (\ref{sLag}) that can be viewed as small perturbations of the singular solutions in \cite{NV} and \cite{WY2}. The examples in the last section have non-minimal gradient graphs, and the singularities appear near the center of a ball. We expect that the examples in the last section are not $C^1$, and that their singularities are modeled locally by examples like the one from Theorem \ref{Main}. In particular, we expect that the degenerate Bellman equation (\ref{Bellman}) with compact free boundaries plays an important role in the formation of Lipschitz singularities in solutions to (\ref{sLag}), and the examples in the last section suggest that this mechanism of singularity formation is stable.
 
%%%%%%%%%%%%%%%%%%%%%%%%%%%
\section*{Acknowledgments}
C. Mooney was supported by NSF CAREER Grant DMS-2143668 and a Sloan Fellowship.
O. Savin was supported by the NSF grant DMS-2055617.


%%%%%%%%%%%%%%%%%%%%%%%%%%%%%%%%%%%%%%%%%%%%%%%%%%%%%%%%%%%%%%%%%%%%%%%%%%%%%%%%%%%%%%%%%%%%%%%%%
\section{Proof of Theorem \ref{Main}}\label{MainExample}
For $\lambda > 0$ to be chosen shortly, let
$$\Phi(x) = \frac{\lambda x_1^2}{1+x_3} + \frac{\lambda x_2^2}{1-x_3}.$$
The function $\Phi$ is convex and analytic in $\{|x_3| < 1\}$. Each term in $\Phi$ is a translation of a one-homogeneous function whose Hessian has rank $1$, so $D^2\Phi$ has rank $2$. It follows that
$$\det D^2\Phi = 0.$$
Note also that the image of $\nabla \Phi$ is contained in the paraboloid
$$\Sigma := \left\{y_3 = \frac{1}{4\lambda}(y_2^2 - y_1^2)\right\}.$$

\begin{lem}\label{LocalMin}
The analytic function $\Theta(x) = F(D^2\Phi(x))$ has a non-degenerate local minimum at $x = 0$ for $\lambda > 0$ small.
\end{lem}


\begin{proof}
Denote by $f(\lambda):= \arctan \lambda$ and then $F(D^2 u)= \sum f(\lambda_i)$.

We use the expansion of order 4 for $\Phi$ at $0$,
$$ \Phi:= \lambda \left(x_1^2 + x_2^2 + x_3(x_2^2-x_1^2) + x_3^2 (x_1^2 + x_2^2) + O(|x|^5) \right),$$
hence $D^2 \Phi(0)$ is diagonal with eigenvalues $2\lambda$, $2\lambda$ and $0$. 

 We compute $ D \Theta$ and $D^2 \Theta$ at $x=0$ and find
$$ \Theta_k= F_{ij} \,  \Phi_{ijk}=f'(2\lambda)\Phi_{11k} + f'(2\lambda) \Phi_{22k} + f'(0) \Phi_{33k} =0,$$
and
\begin{align*}
\Theta_{kl}& = F_{ij}\,  \Phi_{ijkl} + F_{ij,mn} \Phi _{ijk} \Phi_{mnl} \\
& =  f'(2\lambda)\Phi_{11kl} + f'(2\lambda) \Phi_{22kl} + f'(0) \Phi_{33kl} + O(\lambda^2) \\
&= \Phi_{11kl} +  \Phi_{22kl} + \Phi_{33kl} + O(\lambda^2)\\
& = c_k \lambda \, \delta_k^l + O(\lambda^2),
\end{align*}
with $c_1=c_2=4$ and $c_3=8$. In the computation above the derivatives of $F$ are evaluated at $D^2 \Phi(0)$, and we have used that $$D^3 \Phi = O(\lambda), \quad D^4 \Phi = O(\lambda), \quad f'(2 \lambda) = f'(0) + O(\lambda), \quad f'(0)=1.$$ 

Hence, if $\lambda >0$ is chosen small, then $D^2\Theta(0)$ is positive definite, and the lemma is proved.

\end{proof}



Lemma \ref{LocalMin} implies that for $\epsilon > 0$ small, the connected component $K$ of the set $\{\Theta \leq \Theta(0) + \epsilon^2\}$ containing the origin is compact, analytic, uniformly convex, and contained in $B_{C\epsilon}$. Here and below $C$ denotes a large constant, which may change from line to line. As a result, $D^2\Phi$ is within $C\epsilon$ of the diagonal matrix $D^2\Phi(0) = 2\lambda(I - e_3 \otimes e_3)$ in $K$. Later we will use the map
$$\Psi(x) := (\Phi_1(x),\,\Phi_2(x),\,x_3),$$
which is an analytic diffeomorphism in a neighborhood of $0$. Since
$$D^2(\Theta \circ \Psi^{-1})(0) = (D\Psi^{-1}(0))^TD^2\Theta(0)(D\Psi^{-1}(0))$$
is positive, we also have for $\epsilon$ sufficiently small that $\Psi(K),$ the connected component of $\{\Theta \circ \Psi^{-1} \leq \Theta(0) + \epsilon^2\}$ containing the origin, is analytic and uniformly convex.

Now, for $c^* := \Theta(0) + \epsilon^2$, let $v$ be the solution in a small neighborhood of $\partial K$ to
$$F(D^2v) = c^*, \quad v|_{\partial K} = \Phi,\, \quad v_{\nu}|_{\partial K} = \Phi_{\nu}.$$
Here $\nu$ is the outer unit normal to $\partial K$ and we obtain $v$ using Cauchy–Kovalevskaya. Since $\Phi$ and $v$ solve the same equation on $\partial K$ and have the same Cauchy data there, we have
$$D^2v = D^2\Phi \text{ on } \partial K.$$ 
As a result, for $x_0 \in \partial K$, all third derivatives of $\Phi$ and $v$ that involve a differentiation in a direction tangent to $\partial K$ at $x_0$ agree. Since $\Theta_{\nu} > 0$ on $\partial K$ by construction, we conclude on $\partial K$ that
$$0 < \partial_{\nu}(\Theta - F(D^2v)) = F_{ij}(\Phi_{ij\nu}-v_{ij\nu}) = F_{\nu\nu}(\Phi_{\nu\nu\nu}-v_{\nu\nu\nu}),$$
which implies that
\begin{equation}\label{Ineq1}
v_{\nu\nu\nu} < \Phi_{\nu\nu\nu} \text{ on } \partial K.
\end{equation}

We let $K_{\mu}$ denote the set of points a distance less than $\mu$ from $K$.
\begin{lem}\label{DetSign}
We have $\det D^2v < 0$ on $K_{\mu} \backslash K$, for $\mu > 0$ small.
\end{lem}
\begin{proof}
Let $G$ denote determinant. Since $G(D^2v) = G(D^2\Phi) = 0$ on $\partial K$, it suffices to show that 
$$\mbox{$\partial_{\nu}(G(D^2v)) \leq 0$ on $\partial K$, and where equality holds, that $\partial_{\nu}^2(G(D^2v)) < 0.$}$$ To that end we fix $x_0 \in \partial K$, and we let $\xi$ denote the eigendirection at $x_0$ corresponding to the $0$ eigenvalue of $D^2 \Phi(x_0)$. We distinguish two cases.

The first case is that $\xi$ is not tangent to $\partial K$. Then at $x_0$ we pick a system of coordinates with $\nu$ being a coordinate direction, and at $x_0$ we compute
\begin{align*}
\partial_{\nu}(G(D^2v)) &= \partial_{\nu}(G(D^2v) - G(D^2\Phi)) \\
&= G_{ij}(v_{ij\nu} - \Phi_{ij \nu}) \\
&= G_{\nu\nu}(v_{\nu\nu\nu} - \Phi_{\nu\nu\nu}),
\end{align*}
using that $v_{ijk}(x_0) = \Phi_{ijk}(x_0)$ unless $i = j = k = \nu$. Since $\nu$ is not perpendicular to $\xi$, we have that $G_{\nu\nu}(D^2(\Phi(x_0))) > 0$, and we obtain the desired (strict) inequality using (\ref{Ineq1}).

The second case is that $\xi$ is tangent to $\partial K$. Choose coordinates at $x_0$ such that both $\xi$ and $\nu$ are coordinate directions. In these coordinates the only nonzero derivative of $G$ is $G_{\xi\xi} > 0$. In particular, $G_{\nu\nu} = 0$, so the previous calculation implies that $\partial_{\nu}(G(D^2v)) = 0$. Combining these observations we have
$$0 = \partial_{\nu}(G(D^2v)) = \partial_{\nu}(G(D^2\Phi)) = G_{\xi\xi}v_{\xi\xi\nu} = G_{\xi\xi}\Phi_{\xi\xi\nu},$$
hence
\begin{equation}\label{Vanishing}
v_{\xi\xi\nu} = \Phi_{\xi\xi\nu} = 0.
\end{equation}
Now we calculate the second normal derivative:
\begin{align*}
\partial_{\nu}^2(G(D^2v)) &= \partial_{\nu}^2(G(D^2v) - G(D^2\Phi)) \\
&= G_{\xi\xi}(v_{\xi\xi\nu\nu} - \Phi_{\xi\xi\nu\nu}) + G_{ij,\,kl}(v_{ij\nu}v_{kl\nu} - \Phi_{ij\nu}\Phi_{kl\nu}) \\
&= I + II.
\end{align*}
Since all third-order derivatives of $v$ and $\Phi$ involving a tangential direction agree, the only possible nonzero terms in $II$ are those with $i=j=\nu$ or $k = l = \nu$. Using (\ref{Vanishing}), we further reduce $II$ to terms involving $G_{\nu\nu,\,kl}$ where $k$ and $l$ are not both $\xi$. Finally, using that $D^2\Phi(x_0)$ vanishes in the $\xi$ column and row, we see that $G_{\nu\nu,\,kl} = 0$ when $(k,\,l) \neq (\xi,\,\xi)$, thus the term $II$ vanishes.

To estimate the term $I$ note that
$$(v_{\nu\nu} - \Phi_{\nu\nu})_{\xi\xi} = \kappa(v_{\nu\nu\nu} - \Phi_{\nu\nu\nu}),$$
where $\kappa > 0$ is the curvature of $\partial K$ in the direction $\xi$. Using (\ref{Ineq1}) we conclude that
$$\partial_{\nu}^2(G(D^2v)) = \kappa G_{\xi\xi}(v_{\nu\nu\nu} - \Phi_{\nu\nu\nu}) < 0,$$
completing the proof.
\end{proof}

Now, we let
$$w = \begin{cases}
\Phi \text{ in } K, \\
v \text{ in } K_{\mu} \backslash K.
\end{cases}$$
Note that $w \in C^{2,\,1}$ and $D^2w$ is within $C\epsilon$ of the matrix $2\lambda(I - e_3 \otimes e_3)$ in $K_{\mu}$ (we assume $\mu$ was taken small). We let
$$\Gamma := \nabla w(K),$$
and we note that $\Gamma$ is the piece of the paraboloid $\Sigma$ that lies over the projection of $\Psi(K)$ to the horizontal plane.

\begin{lem}\label{Injective}
For $\mu' > 0$ small, the map $\nabla w$ is one-to-one on $K_{\mu'} \backslash K$ and maps $K_{\mu'} \backslash K$ diffeomorphically to a neighborhood of $\Gamma$.
\end{lem}
\begin{proof}
Let $y = H(x) := (w_1(x),\,w_2(x),\,x_3)$. Similar calculations to those in Proposition 3.1 from \cite{WY2} imply that $\det DH > 0$ and that $H$ is distance-expanding, up to a factor depending on $\lambda$. In particular, $H$ is a global diffeomorphism of $K_{\mu}$. As noted above, the set $H(K) = \Psi(K)$ is an analytic uniformly convex set. Thus, for $\mu' > 0$ sufficiently small, $H(K_{\mu'})$ is contained in a convex neighborhood $D \subset H(K_{\mu})$ of $H(K)$. We will show that $\nabla w$ is injective in $K_{\mu'} \backslash K$.

Because $H$ is a diffeomorphism, it suffices to check that $T := \nabla w \circ H^{-1}$ is injective on $D \backslash H(K)$. We have
$$T(y) = (y_1,\, y_2,\, w_3(H^{-1}(y))).$$
Since $D$ is convex, every vertical line intersects it in a connected segment, so it is enough to show that $\partial_3T^3 = \partial_{y_3}(w_3(H^{-1}(y))) \leq 0$, with strict inequality when $y \in D \backslash H(K)$. This follows directly from the identity
\begin{equation}\label{Monotone}
\partial_{y_3}(w_3(H^{-1}(y))) = \det DT(y) = \det D^2w(H^{-1}(y))\det DH^{-1}(y)
\end{equation}
and Lemma \ref{DetSign}.

It just remains to show that $\nabla w$ maps $K_{\mu'}$ into a neighborhood of $\Gamma$. Equivalently, $T$ maps $H(K_{\mu'})$ into a neighborhood of $\Gamma$. Using the monotonicity $\partial_3T^3 < 0$ away from $H(K)$ we see that the image of $T$ contains a small vertical segment through every point in $\Gamma$, and the result follows from the continuity of $T$.
\end{proof}


We recall that for a $C^2$ function $w$, its Legendre transform $w^*$ is defined on the image of the gradient of $w$ by the formula
$$ w^*(\nabla w) = x \cdot \nabla w - w(x),$$ 
with $w^*$ being possibly multivalued. If $\det D^2 w (x_0)\ne 0$, then in a neighborhood of $x_0$, $w^*$ is single-valued and $D^2 w^* = (D^2 w)^{-1}$, hence
$$F(D^2 w^*) + F(D^2 w) = (n- 2 l) \frac \pi 2,$$
where $l$ denotes the number of negative eigenvalues of $D^2w$.


Using Lemma \ref{Injective} we conclude that there exists a neighborhood of $\Gamma$ on which the Legendre transform 
$u = w^*$
of $w$ is single-valued. Away from $\Gamma$, the function $u$ is analytic and has two positive Hessian eigenvalues and one negative Hessian eigenvalue, thus it solves
\begin{equation}\label{finaleq}
F(D^2u) = \frac{\pi}{2} - c^* := c
\end{equation}
classically away from $\Gamma$. We also calculate away from $\Gamma$ that
$$u_{33} = \frac{1}{\det(D^2w)}\text{cof}(D^2w)_{33} < 0,$$
and $u_{33}$ tends to $-\infty$ on $\Gamma$. On $\Gamma \backslash \partial \Gamma$, the function $u$ has a ``downward" Lipschitz singularity. Indeed, from the identity
$$u_3(y_1,\,y_2,\,w_3(H^{-1}(y))) = u_3(\nabla w(H^{-1}(y))) = y_3$$
we infer that $u_3^+$ and $u_3^-$, the limits of $u_3$ from above and below along vertical segments through $\Gamma \backslash \partial \Gamma$, satisfy that 
$$(u_3^+ - u_3^-)(y_1,\,y_2,\,w_3(H^{-1}(y))) = -L(y_1,\,y_2) < 0,$$
where $L(y_1,\,y_2)$ is the length of the intersection between $\Psi(K)$ and the vertical line through $(y_1,\,y_2,\,0)$.

We conclude from this discussion that $u$ is concave along vertical lines, and on $\Gamma$ it cannot be touched from below by any $C^2$ function. As a consequence, $u$ is a viscosity super-solution to (\ref{finaleq}).
Note also that $F(D^2w) \leq c^*$. It follows that $w_k := w - |x|^2/k$ satisfies $F(D^2w_k) < c^*$ for all $k > 0$. For $k$ large $D^2w_k$ has two positive eigenvalues and one negative eigenvalue, and by similar considerations to those above, $w_k^*$ is single-valued and solves
$$F(D^2w_k^*) = \pi/2 - F(D^2w_k) > c$$
classically in a neighborhood of $\Gamma$. Since $w_k^*$ converge uniformly to $u$, we conclude that $u$ is also a viscosity sub-solution to (\ref{finaleq}), completing the construction. 

\begin{rem}
By combining the proofs of Lemmas \ref{DetSign} and \ref{Injective} one can show that $u$ is $C^{1,\,1/2}$ up to $\Gamma$ from each side at points on $\Gamma \backslash \partial \Gamma$, and $u$ is $C^{1,\,1/5}$ on $\partial \Gamma$. Indeed, for $z_0 \in \Gamma \backslash \partial \Gamma$, let $x_0$ be either pre-image under $\nabla w$ of $z_0$ in $\partial K$. Lemma \ref{DetSign} shows that $\partial_{\nu}(G(D^2w)) < 0$ at $x_0$. Using this in (\ref{Monotone}) one can conclude that $|\nabla w(x) - \nabla w(x_0)| \geq C^{-1}|x - x_0|^2$ for $x \in B_r(x_0) \cap (K_{\mu'} \backslash K)$ and $r$ small, giving $C^{1/2}$ regularity of $\nabla u$ on each side of $\Gamma$ at $z_0$. Likewise, if $z_0 \in \partial \Gamma$, then one has $\partial_{\nu}^2(G(D^2w)) < 0$ at $x_0$. Using the uniform convexity of $\partial K$ one concludes in a similar way using (\ref{Monotone}) that $|\nabla w(x) - \nabla w(x_0)| \geq C^{-1}|x-x_0|^5$ for $x \in K_{\mu'} \backslash K$, corresponding to $C^{1/5}$ regularity of $\nabla u$ at $z_0$.
\end{rem}

%%%%%%%%%%%%%%%%%%%%%%%%%%%%%%%%%%%%%%%%%%%%%%%%%%%
\section{Related Examples}
The examples in \cite{NV} and \cite{WY2} are obtained by starting with an analytic solution to the special Lagrangian equation with singular Hessian at the origin and injective gradient. The gradient graph can then be rotated so it has a ``vertical" tangent direction at the origin, and the new potential (the Legendre transform of the original one) is $C^1$ but not $C^2$. Rotating the gradient graphs a tiny bit further gives rise to a potential that is multi-valued in a small neighborhood of the origin. By solving the Dirichlet problem for (\ref{sLag}) with boundary data given by that of the multi-valued potential, one obtains solutions that cannot have minimal gradient graph. In this section we outline a proof, and we discuss the relationship between these examples and the one from the previous section.

\vspace{3mm}   

{\bf Step 1:} Calculations in Section $2$ of \cite{WY2} show that there is a solution $w$ to the special Lagrangian equation with $c = \pi/2$ in $B_{2\kappa} \subset \mathbb{R}^3$ such that
\begin{align*}
w &= \frac{1}{2}\left(x_1^2 + x_2^2\right) + x_3(x_1^2-x_2^2) \\
&+ \frac{1}{12}x_3^2\left(18 x_1^2 + 18 x_2^2 - x_3^2\right) - \frac{1}{8}(x_1^2 + x_2^2)^2 + O(|x|^5).
\end{align*}
It is shown that the two largest eigenvalues of $D^2w$ are close to $1$, and that the smallest eigenvalue $\lambda_3$ of $D^2w$ is analytic near the origin and satisfies
\begin{equation}\label{QuadSep}
\lambda_3 = -|x|^2 + O(|x|^3).
\end{equation}
Let $\epsilon > 0$ be small and let $\tan\theta = \epsilon$. Let $(x,\,y)$ be coordinates of $\mathbb{R}^6$ with $x$ and $y$ in $\mathbb{R}^3$. Rotating the gradient graph of $w$ by an angle $\theta$, that is, representing the graph in new coordinates
$$\tilde{x} = \cos\theta x - \sin\theta y, \quad \tilde{y} = \sin\theta x + \cos\theta y,$$ 
we get a new potential $\tilde{w}$ which satisfies
$$\nabla \tilde{w}(\cos\theta x - \sin\theta \nabla w(x)) = \sin\theta x + \cos\theta \nabla w(x),$$
$$D^2\tilde{w}(\cos\theta x - \sin\theta \nabla w(x)) = (I-\epsilon D^2w(x))^{-1}(\epsilon I + D^2w(x)),$$
$$F(D^2\tilde{w}) = \frac{\pi}{2} + 3\theta \text{ in } B_{\kappa}.$$
Letting $\tilde{\lambda}_3$ be the smallest eigenvalue of $D^2\tilde{w}$, we conclude using (\ref{QuadSep}) that
$$\tilde{\lambda}_3(\cos\theta x - \sin\theta \nabla w(x)) = \epsilon - (1+\epsilon^2)|x|^2 + O(|x|^3),$$
hence $\tilde{\lambda}_3(0) = \epsilon,\, \nabla \tilde{\lambda}_3(0) = 0$, and 
$$D^2\tilde{\lambda}_3(0) = -\frac{2(1+\epsilon^2)}{\cos^{2}\theta}\left((1-\epsilon)^{-2}(e_1 \otimes e_1 + e_2 \otimes e_2) + e_3 \otimes e_3\right).$$
For $\epsilon$ small, the connected component $Z$ of the set $\{\tilde{\lambda}_3 > 0\}$ containing the origin is thus an analytic uniformly convex set contained in $B_{C\sqrt{\epsilon}}$. Furthermore, for 
$$\Psi(x) := (\tilde{w}_1,\, \tilde{w}_2,\, x_3),$$
the same is true for the set $\Psi(Z)$, that is, the connected component of $\{\tilde{\lambda}_3 \circ \Psi^{-1} > 0\}$ containing the origin.

\vspace{3mm}

{\bf Step 2:} The Legendre transform $\tilde{w}^*$ of $\tilde{w}$ is defined in a ball $B_{2d}$, and for $\epsilon$ small it is analytic and single-valued in $B_{2d} \backslash B_{d/8}$. Let $u$ be the solution to the Dirichlet problem
$$F(D^2u) = -3\theta \text{ in } B_{3d/2}, \quad u|_{\partial B_{3d/2}} = \tilde{w}^*.$$
We claim that 
$$\|u - \tilde{w}^*\|_{C^2(B_{d} \backslash B_{d/2})} < C\epsilon^{2}.$$

First, using the convexity of $\Psi(Z)$ and arguments similar to those in Lemma \ref{Injective}, one can show that the preimages under $\nabla \tilde{w}$ of vertical lines are nearly vertical curves that have connected intersection with $Z$. As one follows one of these curves upwards, $\tilde{w}_3$ decreases when the curve lies outside of $Z$ and increases when it is inside of $Z$. This means that $\tilde{w}^*$ is multivalued in a simple way: the graph of $\tilde{w}^*$ along a vertical line is either single valued and concave, or it consists of two crossing concave pieces that lie below and are connected by a convex piece (see Figure \ref{Fig1}). Since $\tilde{w}^*$ solves the dual equation $F(D^2\tilde{w}^*) = -3\theta$ where it is concave in the vertical direction, we conclude that the function $\text{min}(\tilde{w}^*)$ given by the minimum of the possible values of $\tilde{w}^*$ is a super-solution to the equation solved by $u$. In particular, $u \leq \text{min}(\tilde{w}^*)$.

\begin{figure}
 \begin{center}
    \includegraphics[scale=0.4, trim={0mm 100mm 0mm 20mm}, clip]{Fig1.pdf}
\caption{The graph of $\tilde{w}^*$ restricted to a vertical line.}
\label{Fig1}
\end{center}
\end{figure}

Second, since $a_{ij} := F_{ij}(D^2\tilde{w})$ is nearly constant in $B_{\kappa}$, we can build a positive super-solution $\varphi$ to the linearized equation $a_{ij}\varphi_{ij} < 0$ in $B_{\kappa}$ that behaves like the fundamental solution to the Laplace equation outside $B_{C\sqrt{\epsilon}}$, is glued to a paraboloid with Hessian eigenvalues smaller than $-10$ in $B_{C\sqrt{\epsilon}}$, and satisfies $|\varphi| \leq C\epsilon$. Then for $\epsilon$ small $\tilde{w} + \epsilon \varphi$ is a super-solution to the equation solved by $\tilde{w}$. Moreover, it has a single-valued Legendre transform that lies below $\tilde{w}^*$ on $\partial B_{3d/2}$, is $C\epsilon^{2}$ close to $\tilde{w}^*$ in $B_{3d/2} \backslash B_{d/4}$ and is a sub-solution of the dual equation solved by $u$. We conclude using the maximum principle that $|u - \tilde{w}^*| < C\epsilon^{2}$ in $B_{3d/2} \backslash B_{d/4}$, and the claim follows from the Schauder estimates.

\vspace{3mm}

{\bf Step 3:} Assume by way of contradiction that $\Gamma_u$ is minimal. By the above considerations, the graphs $\Gamma_u := \{(\nabla u(y),\, y) : y \in B_d\}$ and $\Gamma_{\tilde{w}} = \{(x,\,\nabla \tilde{w}(x)) : x \in B_{\kappa}\}$ are $\epsilon^{2}$-close in $C^1$ outside of a cylinder. Note that $\Gamma_{\tilde{w}}$ is graphical over its tangent $3$-plane $P$ at the origin provided $\kappa$ is small, and lies within a cylinder of radius $C\kappa^2$ around $P$. The same is thus true of $\Gamma_u$ near its boundary. By the maximum principle we conclude that $\Gamma_u$ lies in a cylinder of radius $C\kappa^2$ around $P$. Provided $\kappa$ is small the Allard regularity theorem implies that $\Gamma_u$ is graphical over $P$, and with respect to coordinates in $P$, $\Gamma_u$ and $\Gamma_{\tilde{w}}$ are the graphs of maps that are $\epsilon^{2}$-close in $C^1$ in an annulus, have gradient of size $\kappa$ and solve the minimal surface system. Using the standard regularity theory of systems arising as Euler-Lagrange equations of functionals with uniformly convex integrands (the area element is uniformly convex for maps with small gradient) we infer that $\Gamma_u$ and $\Gamma_{\tilde{w}}$ are everywhere $\epsilon^{2}$-close in the $C^1$ sense. In particular, $\Gamma_u$ is graphical in the $x$ variable as well, so $u^*$ is single-valued and satisfies
$$D^2u^*(0) = D^2\tilde{w}(0) + O(\epsilon^{2}) > C^{-1}\epsilon I$$
for $\epsilon$ small. This implies that $D^2u(\nabla u^*(0))$ is a positive matrix, contradicting the equation for $u$.

\vspace{3mm}

One feature of the examples in this section is that the singularities occur near the center of a ball, in contrast with the examples in the previous section, which are only constructed in a small neighborhood of a singularity. Another feature is that the singularities of the examples in this section exist for all choices of $\epsilon$ small, illustrating their stable nature. 

The argument above shows that $u$ is not well-approximated by $\tilde w^*$ near the origin. Instead, we need to consider the Legendre transform of a solution $v$ to the modified equation 
$$ \max \{ F(D^2 v)- \frac \pi 2 - 3 \theta, \det \, D^2 v\} =0, \quad \quad v= \tilde w \quad \mbox{on $\partial B_\kappa$.}$$
This is in fact the starting point of our construction in Theorem \ref{Main}, in which we exhibit a $C^{2,1}$ solution of \eqref{Bellman} with an analytic compact free boundary between the two operators. 

We expect that the examples constructed in the previous section are in fact local models for the singularities appearing in this section. More precisely, we conjecture that for all $\epsilon > 0$ small, the examples $u$ constructed in this section exhibit Lipschitz singularities, and moreover that their Legendre transforms $u^*$ solve degenerate Bellman equations with compact free boundaries. The main difficulty consists in showing that solutions $v$ to the equation above are of class $C^2$ and have injective gradient.
On the other hand, after an appropriate rescaling, as $\epsilon \to 0$ the equation linearizes  to a model equation of the type
$$ \max \{ \triangle v, v_{33} + 1 - |x|^2\} =0, \quad \quad v \to 0 \quad \mbox{as $|x| \to \infty$.}$$
This problem has a compact free boundary which seems to have good regularity properties. We intend to analyze these questions further in a subsequent work. 

%%%%%%%%%%%%%%%%%%%%%%%%%%%%%%%%%%%%%%%%%%%%%%%%%%%%%%%%%%%%%%%%%%%%%%%%%%%%%%%%%%%
\begin{thebibliography}{9999}
\bibitem[CSY]{CSY} Chen, J. Y.; Shankar, R.; Yuan, Y. Regularity for convex viscosity solutions of special Lagrangian equation. {\it Comm. Pure Appl. Math.}, to appear.
\bibitem[HL]{HL} Harvey, R.; Lawson, H. B. Calibrated geometries. {\it Acta Math.} {\bf 148} (1982), 47-157.
\bibitem[NV]{NV} Nadirashvili, N.; Vladut, S. Singular solution to special Lagrangian equations. {\it Ann. Inst. H. Poincar\'{e} Anal. Non Lin\'{e}aire} {\bf 27} (2010), 1179-1188.
\bibitem[WY1]{WY1} Wang. D. K.; Yuan, Y. Hessian estimates for special Lagrangian equations with critical and supercritical phases in general dimensions. {\it Amer. J. Math.} {\bf 136} (2014), 481-499.
\bibitem[WY2]{WY2} Wang. D. K.; Yuan, Y. Singular solutions to special Lagrangian equations with subcritical phases and minimal surface systems. {\it Amer. J. Math.} {\bf 135} (2013), 1157-1177.
\end{thebibliography}
%%%%%%%%%%%%%%%%%%%%%%%%%%%%%%%%%%%%%%%%%%%%%%%%%%%%%%%%%%%%%%%%%%%%%%%%%%%%%%%%%%%%


\end{document}