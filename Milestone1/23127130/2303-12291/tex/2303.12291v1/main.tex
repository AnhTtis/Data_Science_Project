\documentclass{article}
\usepackage{arxiv}
\usepackage[utf8]{inputenc} 

\usepackage{microtype}
\usepackage{graphicx}
\usepackage{subfigure}
\usepackage{booktabs}
\usepackage{hyperref}
% \newcommand{\theHalgorithm}{\arabic{algorithm}}

\usepackage{amsmath}
\usepackage{amssymb}
\usepackage{mathtools}
\usepackage{amsthm}

% \usepackage[capitalize,noabbrev]{cleveref}

%%%%%%%%%%%%%%%%%%%%%%%%%%%%%%%%
% THEOREMS
%%%%%%%%%%%%%%%%%%%%%%%%%%%%%%%%
\theoremstyle{plain}
\newtheorem{theorem}{Theorem}[section]
\newtheorem{proposition}[theorem]{Proposition}
\newtheorem{lemma}[theorem]{Lemma}
\newtheorem{corollary}[theorem]{Corollary}
\theoremstyle{definition}
\newtheorem{definition}[theorem]{Definition}
\newtheorem{assumption}[theorem]{Assumption}
\theoremstyle{remark}
\newtheorem{remark}[theorem]{Remark}
\usepackage[textsize=tiny]{todonotes}


\usepackage[utf8]{inputenc} % allow utf-8 input
\usepackage[T1]{fontenc}    % use 8-bit T1 fonts     
\usepackage{url}            % simple URL typesetting
\usepackage{booktabs}       % professional-quality tables
\usepackage{amsfonts}       % blackboard math symbols
\usepackage{nicefrac}       % compact symbols for 1/2, etc.
\usepackage{microtype}      % microtypography
\usepackage{xcolor}         % colors
\usepackage{graphicx}
\usepackage{subfigure}
\usepackage{booktabs}
% \newcommand{\theHalgorithm}{\arabic{algorithm}}
\usepackage{amsmath}
\usepackage{amssymb}
\usepackage{mathtools}
\usepackage{amsthm,bm}
\usepackage{url}
% \usepackage[english]{babel}
\usepackage{mathrsfs}
\usepackage{times}

\usepackage{wrapfig}
% \usepackage{latexsym}
% \usepackage{natbib}
\usepackage{hyperref}
\def\given{\,|\,}
\def\biggiven{\,\big{|}\,}
\def\Biggiven{\,\Big{|}\,}
\def\bigggiven{\,\bigg{|}\,}
\def\tr{\mathop{\text{tr}}\kern.2ex}
\def\tZ{{\tilde Z}}
\def\tX{{\tilde X}}
\def\tR{{\tilde r}}
\def\tU{{\tilde U}}
\def\P{{\mathbb P}}
\def\E{{\mathbb E}}
\def\B{{\mathbb B}}
\def\nD{{\mathcal{\widetilde{D}}}}
\def\sign{\mathop{\text{sign}}}
\def\supp{\mathop{\text{supp}}}
\def\card{\mathop{\text{card}}}
\def\rank{\mathrm{rank}}
\long\def\comment#1{}
\def\skeptic{{\sc skeptic}}
\def\NPN{\bPiox{\it NPN}}
\def\vec{\mathop{\text{vec}}}
\def\tr{\mathop{\text{Tr}}}
\def\trc{\mathop{\text{TRC}}}
\def\cS{{\mathcal{S}}}
\def\dr{\displaystyle \rm}
\def\skeptic{{\sc skeptic}}
\providecommand{\bnorm}[1]{\Big\vvvert#1\Big\vvvert}
\providecommand{\norm}[1]{\vvvert#1\vvvert}

\usepackage{multirow}
\usepackage{xcolor}
\newsavebox\MBox
\newcommand\Cline[2][red]{{\sbox\MBox{$#2$}%
  \rlap{\usebox\MBox}\color{#1}\rule[-1.2\dp\MBox]{\wd\MBox}{0.5pt}}}
\newcommand{\bel}{\begin{eqnarray}\label}
\newcommand{\eel}{\end{eqnarray}}
\newcommand{\bes}{\begin{eqnarray*}}
\newcommand{\ees}{\end{eqnarray*}}
\def\bess{\bes\small }
\def\Shat{{\widehat S}}
\def\lam{\rho}
\def\real{{\mathbb{R}}}
\def\R{{\real}}
\def\Ti{T_{\rm init}}
\def\Tm{T_{\rm main}}
\def\bcdot{{\bm \cdot}}
\def\ud{\, \text{d}}
\newcommand{\red}{\color{red}}
\newcommand{\blue}{\color{blue}}
\newcommand{\la}{\langle}
\newcommand{\ra}{\rangle}
\newcommand{\cIs}{\cI_{\hat{s}}}
\newcommand{\F}{{\rm F}}
\newcommand{\iip}{I_{\text{i}}^+}
\newcommand{\iim}{I_{\text{i}}^-}
\newcommand{\zip}{z_{\text{i}}^+}
\newcommand{\zim}{z_{\text{i}}^-}
\newcommand{\Zip}{Z_{\text{i}}^+}
\newcommand{\Zim}{Z_{\text{i}}^-}
\newcommand{\Xxi}{X_{\text{I}}}
\newcommand{\hth}{\hat{\theta}_{\text{H}}}
\newcommand{\htt}{\hat{\theta}_{\text{T}}}
\newcommand{\rh}{r_{\text{H}}}
\newcommand{\rt}{r_{\text{T}}}
\newcommand{\Rp}{r_+}
\newcommand{\Rm}{r_-}
\newcommand{\nS}{\widetilde{S}}
\newcommand{\Xip}{X_{\text{I}}^+}
\newcommand{\Xim}{X_{\text{I}}^-}
\newcommand{\nSib}{\nS_{\text{I}}^{\pm}}
\newcommand{\nSip}{\nS_{\text{I}}^+}
\newcommand{\Sip}{S_{\text{I}}^+}
\newcommand{\nSim}{\nS_{\text{I}}^-}
\newcommand{\Sim}{S_{\text{I}}^-}
\newcommand{\Xhb}{X_{\text{H}}^{\pm}}
\newcommand{\Xhp}{X_{\text{H}}^+}
\newcommand{\Xhm}{X_{\text{H}}^-}
\newcommand{\Xtb}{X_{\text{T}}^{\pm}}
\newcommand{\Xtp}{X_{\text{T}}^+}
\newcommand{\Xtm}{X_{\text{T}}^-}
\newcommand{\nShb}{\nS_{\text{H}}^{\pm}}
\newcommand{\Shb}{S_{\text{H}}^{\pm}}
\newcommand{\nShp}{\nS_{\text{H}}^+}
\newcommand{\Shp}{S_{\text{H}}^+}
\newcommand{\nShm}{\nS_{\text{H}}^-}
\newcommand{\Shm}{S_{\text{H}}^-}
\newcommand{\nStb}{\nS_{\text{T}}^{\pm}}
\newcommand{\Stb}{S_{\text{T}}^{\pm}}
\newcommand{\nStp}{\nS_{\text{T}}^+}
\newcommand{\Stp}{S_{\text{T}}^+}
\newcommand{\nStm}{\nS_{\text{T}}^-}
\newcommand{\Stm}{S_{\text{T}}^-}
\newcommand{\ebt}{\rho^{\pm}_{\text{T}}}
\newcommand{\ept}{\rho^+_{\text{T}}}
\newcommand{\emt}{\rho^-_{\text{T}}}
\newcommand{\ebh}{\rho^{\pm}_{\text{H}}}
\newcommand{\eph}{\rho^+_{\text{H}}}
\newcommand{\emh}{\rho^-_{\text{H}}}
\newcommand{\tran}{{t}}
\let\oldemptyset\emptyset
\let\emptyset\varnothing


\newcommand{\op}{o_{\raisemath{-1.5pt}\PP}}
\newcommand{\Op}{O_{\raisemath{-1.5pt}\PP}}


\newcommand{\dist}{\textsf{Dist}}


% \usepackage[]{mathtools} % loads amsmath as well
\def\##1\#{\begin{align}#1\end{align}}
\def\$#1\${\begin{align*}#1\end{align*}}
\usepackage{enumitem}
% https://www.overleaf.com/learn/latex/Theorems_and_proofs
% \usepackage[]{amsthm} % provides proof, theorem, lemma, etc. environments
% \usepackage[]{amssymb}
\usepackage{bbm}
\usepackage{xcolor}
\usepackage{tcolorbox}
\newtcolorbox{mybox}{colback=gray!10!white,colframe=gray!10!white,left=1mm,top=1mm,right=1mm,boxsep=0mm,width=15cm,before=\par\smallskip\centering,after=\par,
height=1cm}

\newtcolorbox{mybox2}{colback=gray!10!white,colframe=gray!10!white,left=1mm,top=1mm,right=1mm,boxsep=0mm,width=15cm,before=\par\smallskip\centering,after=\par,
height=1cm}

\newtcolorbox{mybox1}{colback=pink!30!white,colframe=pink!30!white,left=0mm,top=0mm,right=0mm,boxsep=0mm,width=13cm,before=\par\smallskip\centering,after=\par,
height=1.6cm}


\newtcolorbox{mybox3}{colback=pink!30!white,colframe=pink!30!white,left=0.5mm,top=0mm,right=0.5mm,boxsep=0mm,width=11cm,before=\par\smallskip\centering,after=\par,
height=1cm}

\newtheorem{observation}[theorem]{Observation}
\newcommand{\squishlist}{
\begin{list}{{{\small{$\bullet$}}}}
{\setlength{\itemsep}{3pt}      \setlength{\parsep}{1pt}
\setlength{\topsep}{1pt}       \setlength{\partopsep}{0pt}
\setlength{\leftmargin}{1em} \setlength{\labelwidth}{1em}
\setlength{\labelsep}{0.5em} } }
\newcommand{\squishend}{  \end{list}  }
\usepackage{xcolor}
\usepackage{colortbl}
\definecolor{best}{HTML}{BAFFCD}
\definecolor{issue}{HTML}{FFC8BA}
\definecolor{bad}{HTML}{FFC87C}
\newcommand{\good}[1]{\cellcolor{best}#1} 
\newcommand{\bad}[1]{\cellcolor{issue}#1} 
\newcommand{\better}[1]{\cellcolor{issue}#1} 
\usepackage[textsize=tiny]{todonotes}

\DeclareMathOperator*{\argmax}{argmax}
\DeclareMathOperator*{\argmin}{argmin}

\usepackage[capitalize,noabbrev]{cleveref}
\newcommand{\BR}{\mathbbm{1}}
% \newcommand{\bias}{\text{Bias}}
\newcommand{\bias}{\mathrm{Bias}}
% \newcommand{\E}{\mathbb E}
\newcommand{\PP}{\mathbb P}
\newcommand{\QQ}{\mathbb Q}
\newcommand{\RR}{\mathbb R}
\newcommand{\Hy}{\mathcal{H}}
\newcommand{\A}{\textsf{A}}
\newcommand{\FNR}{\textsf{FNR}}
\newcommand{\FPR}{\textsf{FPR}}
\newcommand{\TPR}{\textsf{TPR}}
\newcommand{\f}{f^*}
\newcommand{\rf}{\hat{f}}
\newcommand{\D}{\mathcal D}

\newcommand{\p}{\mathbb P}
\newcommand{\x}{\mathbf{x}}
\newcommand{\ny}{\tilde{y}}
\newcommand{\nY}{\widetilde{Y}}

\newcommand{\aph}{A^+_{\text{H}}}
\newcommand{\apt}{A^+_{\text{T}}}
\newcommand{\amh}{A^-_{\text{H}}}
\newcommand{\amt}{A^-_{\text{T}}}
\newcommand{\fr}{\textbf{FR}}

\newcommand{\wjh}[1]{\textbf{\color{blue}(Jiaheng: #1)}}


\ifodd 1
\newcommand{\rev}[1]{{\color{black}#1}}
\newcommand{\com}[1]{\textbf{\color{red}(YANG: #1)}}
\newcommand{\yl}[1]{\textbf{\color{red}(YANG: #1)}}
\newcommand{\clar}[1]{\textbf{\color{green}(NEED CLARIFICATION: #1)}}
\newcommand{\response}[1]{\textbf{\color{magenta}(RESPONSE: #1)}}
\newcommand{\zzw}[2]{\textbf{\color{blue}(Zhaowei: #1)}{\color{blue}~#2}}
\else
\newcommand{\rev}[1]{#1}
\newcommand{\com}[1]{}
\newcommand{\clar}[1]{}
\newcommand{\response}[1]{}
\newcommand{\yl}[1]{}
\newcommand{\zzw}[2]{}
\fi
\newcommand{\RNum}[1]{\uppercase\expandafter{\romannumeral #1\relax}}
\newcommand{\fix}{\marginpar{FIX}}
\newcommand{\new}{\marginpar{NEW}}

\title{Fairness Improves Learning from \\Noisily Labeled Long-Tailed Data}


\author{Jiaheng Wei \\
UC Santa Cruz 
\And
Zhaowei Zhu\\
UC Santa Cruz \\
\And
Gang Niu\\
Riken \\
\And
Tongliang Liu \\
University of Sydney
\And
Sijia Liu\\
Michigan State University \\
\And 
Masashi Sugiyama\\
Riken \& University of Tokyo\\
\And
Yang Liu \thanks{Correspondence to yang.liu01@bytedance.com. (Paper under review)}\\
UC Santa Cruz \& ByteDance Research\\
}


\begin{document}


\maketitle


\begin{abstract}
Both long-tailed and noisily labeled data frequently appear in real-world applications and impose significant challenges for learning. Most prior works treat either problem in an isolated way and do not explicitly consider the coupling effects of the two. Our empirical observation reveals that such solutions fail to consistently improve the learning when the dataset is long-tailed with label noise. Moreover, with the presence of label noise, existing methods do not observe universal improvements across different sub-populations; in other words, some sub-populations enjoyed the benefits of improved accuracy at the cost of hurting others. Based on these observations, we introduce the \textbf{F}airness \textbf{R}egularizer (\fr), inspired by regularizing the performance gap between any two sub-populations. We show that the introduced fairness regularizer improves the performances of sub-populations on the tail and the overall learning performance. Extensive experiments demonstrate the effectiveness of the proposed solution when complemented with certain existing popular robust or class-balanced methods.
\end{abstract}


\section{Introduction}
\label{sec:introduction}
% \begin{itemize}
%     % Diffusion of FL
%     \item {\st{Diffusion of FL}}
%     % Security threats to FL
%     \item {\st{Security threats to FL with particular focus on model poisoning}}
%     % Limitations of existing countermeasures
%     \item {\st{Current countermeasures (e.g., KRUM) and their limitations}}
%     % Proposed method and its advantages
%     \item {\st{Intuitive description of the proposed method and its difference (i.e., advantages) w.r.t. state of the art}}
%     % Main contributions
%     \item {\st{Summary of the main contributions of this work}}
%     % Paper's structure and organization
%     \item {\st{Paper's structure and organization}}
% \end{itemize}

% Diffusion of FL
Recently, {\em federated learning} (FL) has emerged as the leading paradigm for training distributed, large-scale, and privacy-preserving machine learning (ML) systems~\cite{mcmahan2017googleai,mcmahan2017aistats}. 
The core idea of FL is to allow multiple edge clients to collaboratively train a shared, global model without disclosing their local private training data.
%Specifically, an FL system consists of a central server and many edge clients; 
A typical FL round involves the following steps: {\em(i)} the server randomly picks some clients and sends them the current, global model; {\em(ii)} each selected client locally trains its model with its own private data; then, it sends the resulting local model to the server;\footnote{Whenever we refer to global/local model, we mean global/local model {\em parameters}.} {\em(iii)} the server updates the global model by computing an \emph{aggregation function}, usually the average (FedAvg), on the local models received from clients.
% \begin{enumerate}
%     \item[{\em(i)}] the server sends the current, global model to the clients and appoints some of them for training;
%     \item[{\em(ii)}] each selected client locally trains its copy of the global model with its own private data; then, it sends the resulting local model back to the server;\footnote{Whenever we refer to global/local model, we mean global/local model {\em parameters}.}
%     \item[{\em(iii)}] the server updates the global model by computing an \emph{aggregation function} on the local models received from clients (by default, the average, also referred to as FedAvg~\cite{mcmahan2017aistats}).
% \end{enumerate}
This process goes on until the global model converges. %(e.g., after a certain number of rounds or other similar stopping criteria).
%\\
% The advantages of FL over the traditional, centralized learning paradigm are undoubtedly clear in terms of flexibility/scalability (clients can join/disconnect from the FL network dynamically), network communications (only model weights\footnote{We will use \textit{parameters} and \textit{weights} interchangeably.} are exchanged between clients and server), and privacy (each client's private training data is kept local at the client's end and not uploaded to the server).
\\
% Security threats to FL
%However, the growing adoption of FL also raises security concerns~\cite{costa2022covert}, particularly about its confidentiality, integrity, and availability.
Although its advantages over standard ML, FL also raises security concerns~\cite{costa2022covert}. %, particularly about its confidentiality, integrity, and availability~\cite{costa2022covert}.
% OLD, LONG VERSION
% Indeed, some work deals with privacy leakage that may expose the local data of some clients~\cite{melis2019sp}. 
% A large body of work, instead, investigates attacks that usually aim to detriment the predictive accuracy of the learned global model. For instance, \emph{data poisoning} attacks achieve this goal by letting an adversary pollute the training set of some corrupt FL clients with maliciously crafted examples~\cite{jagielski2018sp}.
% Similarly, in \emph{model poisoning} the attacker attempts to tweak the global model weights~\cite{bhagoji2019pmlr} by directly perturbing the local model's weights of some infected FL clients before these are sent to the central server for aggregation, usually via so-called Byzantine attacks. 
% It turns out that Byzantine model poisoning attacks severely impact standard FedAvg; therefore, more robust aggregation functions must be designed to make FL systems secure.
Here, we focus on \emph{untargeted model poisoning} attacks~\cite{bhagoji2019pmlr}, where an adversary attempts to tweak the global model weights %\footnote{We will use the terms \textit{parameters} and \textit{weights} interchangeably.} 
by directly perturbing the local model's parameters of some infected clients before these are sent to the central server for aggregation.
In doing so, the adversary aims to jeopardize the global model \textit{indiscriminately} at inference time.
Such model poisoning attacks severely impact standard FedAvg; therefore, more robust aggregation functions must be designed to secure FL systems.
\\
% In this paper, we focus on designing a novel robust aggregation scheme at the server's end to contrast the effect of Byzantine model poisoning attacks.
%
% Current countermeasures and their limitations
%Several countermeasures have been proposed in the literature to combat model poisoning attacks on FL systems.
% Some methods use simple statistics more robust than plain average to smooth the impact of malicious updates (e.g., Trimmed Mean and FedMedian~\cite{yin2018icml}). 
% Other defenses implement outlier detection techniques to discard malicious updates from the aggregation performed at the server's end. Those are either based on heuristics (e.g., Krum/Multi-Krum~\cite{blanchard2017nips} and Bulyan~\cite{mhamdi2018pmlr}) or data-driven approaches (e.g., K-means clustering~\cite{shen2016acm} or DnC via spectral analysis~\cite{shejwalkar2021ndss}). 
% Finally, some strategies rely on a centralized ``source of trust'' to spot potential malicious updates (e.g., FLTrust~\cite{cao2020fltrust}).
% Several countermeasures have been proposed in the literature to combat model poisoning attacks on FL systems, i.e., to discard possible malicious local updates from the aggregation performed at the server's end. 
% These techniques range from simple statistics more robust than plain average (e.g., Trimmed Mean and FedMedian~\cite{yin2018icml}) to outlier detection heuristics (e.g., Krum/Multi-Krum~\cite{blanchard2017nips} and Bulyan~\cite{mhamdi2018pmlr}) or data-driven approaches (e.g., spectral analysis via K-means clustering~\cite{shen2016acm} or spectral analysis), or methods based on ``source of trust'' (e.g., FLTrust~\cite{cao2020fltrust}).
% OLD, LONG VERSION
%Several countermeasures have been proposed in the literature to combat Byzantine model poisoning attacks on FL systems.
% Descriptive statistics
% For example, Trimmed Mean and FedMedian aggregate local model updates using more robust statistics than standard average~\cite{yin2018icml}.
%
% % Heuristics for outlier detection
% Many existing Byzantine-resilient strategies implement some outlier detection heuristics to discard the model updates sent by potentially malicious clients from the input of the aggregation function.
% One of the most popular heuristics is Krum~\cite{blanchard2017nips}.
% This strategy tries to mitigate the impact of Byzantine attacks by selecting as a global model the local model with the smallest sum of Euclidean distances to {\em all} the other local models.
% Although powerful, Krum requires the server to know (or, at least, estimate) the number of malicious FL clients upfront, which is generally impossible in a realistic attack scenario. %
% Moreover, Krum may become ineffective for complex, high-dimensional model parameter spaces due to the curse of dimensionality.
% Bulyan~\cite{mhamdi2018pmlr} tries to overcome this issue by combining Krum with a variant of Trimmed Mean.
% % Data-driven outlier detection
% Other strategies use data-driven outlier detection techniques -- e.g., via K-means clustering~\cite{shen2016acm} -- to spot potential malicious local model updates. 
% %For instance, Shen et al. propose to cluster local model updates with K-means and thus identify outliers.
%
% % Other techniques
% As far as the server is concerned, any local model received can be from a potential malicious client. 
% FLTrust~\cite{cao2020fltrust} assumes the server acts as a client, i.e., trains a local model on an additional {\em trustworthy} dataset at the server's end and compares it against all the local models from other clients. 
% This way, the server can rely on some ``source of trust'' when discarding potentially malicious clients.
%\\
% Limitations of existing Byzantine-resilient strategies
Unfortunately, existing defense mechanisms either rely on simple heuristics (e.g., Trimmed Mean and FedMedian by~\cite{yin2018icml}) or need strong and unrealistic assumptions to work effectively (e.g., foreknowledge or estimation of the number of malicious clients in the FL system, as for Krum/Multi-Krum~\cite{blanchard2017nips} and Bulyan~\cite{mhamdi2018pmlr}, which, however, cannot exceed a fixed threshold).
Furthermore, outlier detection methods using K-means clustering~\cite{shen2016acm} or spectral analysis like DnC~\cite{shejwalkar2021ndss} do not directly consider the temporal evolution of local model updates received.
Finally, strategies like FLTrust~\cite{cao2020fltrust} require the server to collect its own dataset and act as a proper client, thereby altering the standard FL protocol.
\\
% OLD, LONG VERSION
% Overall, existing Byzantine-resilient strategies are either simple heuristics (e.g., FedMedian) or, if they are more complex, they rely on strong and unrealistic assumptions to work effectively (e.g., knowing the number of malicious clients in the FL system in advance, as for Krum and alike).
% Furthermore, data-driven outlier detection methods do not consider the temporary evolution of local model updates received (e.g., K-means clustering). 
% Finally, strategies like FLTrust requires the server to collect its own dataset and act as a proper client, thereby altering the standard FL protocol.
%
% Description of the proposed method
This work introduces a novel pre-aggregation \textit{filter} robust to untargeted model poisoning attacks. Notably, this filter $(i)$ operates without requiring prior knowledge or constraints on the number of malicious clients and $(ii)$ inherently integrates temporal dependencies. 
The FL server can employ this filter as a preprocessing step before applying \textit{any} aggregation function, be it standard like FedAvg or robust like Krum or Bulyan.
Specifically, we formulate the problem of identifying corrupted updates as a multidimensional (i.e., matrix-valued) time series anomaly detection task. 
The key idea is that legitimate local updates, resulting from well-calibrated iterative procedures like stochastic gradient descent (SGD) with an appropriate learning rate, show \textit{higher predictability} compared to malicious updates. This hypothesis stems from the fact that the sequence of gradients (thus, model parameters) observed during legitimate training exhibit regular patterns, as validated in Section~\ref{subsec:intuition}. %until convergence. 
%This regularity may be more pronounced for smooth convex loss functions, but it can still be captured within an appropriate time window, even for more complex and convoluted loss surfaces. 
%We provide evidence of this claim in Appendix~B, where we show that the average mutual information (i.e., ``predictability''), calculated over pairs of legitimate model updates sent at different FL rounds, is significantly higher than the corresponding computation for a malicious client.
\\
Inspired by the matrix autoregressive (MAR) framework for multidimensional time series forecasting~\cite{chen2021je}, we propose the FLANDERS ({\em \textbf{F}ederated \textbf{L}earning meets \textbf{AN}omaly \textbf{DE}tection for a \textbf{R}obust and \textbf{S}ecure}) filter.
The main advantages of FLANDERS over existing strategies like FLDetector~\cite{zhao2020multivariate} are its resilience to large-scale attacks, where $50\%$ or more FL participants are hostile, and the capability of working under realistic non-iid scenarios.
We attribute such a capability to two key factors: $(i)$ FLANDERS works without knowing a priori the ratio of corrupted clients, and $(ii)$ it embodies temporal dependencies between intra- and inter-client updates, quickly recognizing local model drifts caused by evil players. Below, we summarize our main contributions:

\begin{itemize}
\item[{\em(i)}]
We provide empirical evidence that the sequence of models sent by legitimate clients is more predictable than those of malicious participants performing untargeted model poisoning attacks.
\\
\item[{\em(ii)}] 
We introduce FLANDERS, the first pre-aggregation filter for FL robust to untargeted model poisoning based on multidimensional time series anomaly detection.
\\
\item[{\em(iii)}] 
We integrate FLANDERS into Flower,\footnote{\scriptsize{\url{https://flower.dev/}}} a popular FL simulation framework for reproducibility.
\\
\item[{\em(iv)}] 
We show that FLANDERS improves the robustness of the existing aggregation methods under multiple settings: different datasets, client's data distribution (non-iid), models, and attack scenarios.
\\
\item[{\em(v)}] 
We publicly release all the implementation code of FLANDERS along with our experiments.\footnote{\scriptsize{\url{https://anonymous.4open.science/r/flanders_exp-7EEB}}}
\end{itemize}

% Paper's structure and organization
The remainder of the paper is structured as follows. %some related work and the current state-of-the-art solutions to security issues that FL entails. 
Section~\ref{sec:background} covers background and preliminaries. 
In Section~\ref{sec:related}, we discuss related work.
Section~\ref{sec:problem} and Section~\ref{sec:method} describe the problem formulation and the method proposed. % to tackle it. 
Section~\ref{sec:experiments} gathers experimental results. %, and Section~\ref{sec:limitations} discusses some limitations of this work.
Finally, we conclude in Section~\ref{sec:conclusion}.
 %discusses the limitations of this work and draws future research directions.
%reports conclusions and draws perspectives for future research directions.

%%%%%%% OLD %%%%%%%
%to overcome the resilience of Byzantine failures in distributed Stochastic Gradient Descent computations. 
% The strength of Krum is its time complexity, which is linear in the gradient dimension. 
% However, the robustness of the approach is guaranteed for gradient-based learning applications only when the majority of the clients are not compromised. 
% Besides, the aggregation mechanism of Krum, as well as that of similar methods, is robust from a coarse-grained perspective and does not provide solutions to errors and perturbations that may occur at inference time.
%A related approach to~\cite{blanchard2017nips} is the work of Su et al.~\cite{su2016dc}. Here, the authors propose an iterated approximate agreement to tackle a multi-layer scenario attacked by Byzantine agents. 
%However, the method works efficiently on the sole discrete context and it is inapplicable to continuous state environments.
%\gabri{Maybe, we should just talk about the main limitations of existing countermeasures without digging into their details (or, we can just mention Krum as this is the most popular one). I will move the description of all these methods to the Related Work section.}




\section{Preliminary}

We go through the preliminaries in this section.

\subsection{Sub-Populations of Features}\label{sec:formulation}
 
In a $K$-class classification task, denote a set of data samples with clean labels  as $S:=\{(x_i, y_i)\}_{i=1}^n$, given by random variables $(X, Y)$, which is assumed to be drawn from $\mathcal{D}$. In this work, we are interested in how sub-populations intervene with learning. Formally, we denote $G \in \{1,2,...,N\}$ as the random variable for the index of sub-population, and each sample $(x_i, y_i)$ is further associated with a $g_i$. The set of sub-population $k$ could then be denote as $\mathcal G_k := \{i: g_i = k\}$. We consider a long-tail scenario where the head population and the tail population differ significantly in their sizes, i.e., $\max_k |\mathcal G_k| \gg \min_{k'} |\mathcal G_{k'}|$. 

Consider Figure \ref{fig:dis_longtail} for an example of sub-population separations using the CIFAR-100 dataset \cite{krizhevsky2009learning}: images are grouped into 20 coarse classes, and each coarse class could be further categorized into 5 fine classes. For example, the coarse class "aquatic mammals" was further split into "beaver", "dolphin", "otter", "seal", 
 and "whale". From Figure \ref{fig:dis_longtail}, we observe a strong imbalanced distribution of different sub-populations and a long-tailed pattern. In Section \ref{sec:exp_detail}, we provide more details on long-tail data generation models for our synthetic experiments. 

\paragraph{Clarification}
Throughout this work, the saying of sub-populations is a generalized definition of the separation of samples, which includes many popular settings as special cases, i.e.,
\squishlist
    \item \emph{Class-relevant}: the class name is actually a natural separation of samples, such separations could be more fine-grained class-related (such as further splitting the class “cat” into finer separations by referring to the breed of cats);
    \item \emph{Class-irrelevant}: such population could also be class-irrelevant, for example, in image classification tasks where the gender information is the (hidden) attribute information of each image while the class/label does not disclose this information.
\squishend

\begin{figure}[!t]
    \centering
  \begin{center}
\includegraphics[width=0.5\textwidth]{figures/influence/longtail_sub.pdf}
  \end{center}
  \caption{Count plot of a synthetic long-tailed CIFAR-100 train dataset: $x$-axis denotes the sub-population index; $y$-axis indicates the number of samples in each sub-population.}\label{fig:dis_longtail}
\end{figure}


\subsection{Our Task}
In practice, obtaining "clean" labels from human annotators is both time-consuming and expensive. The obtained human-annotated labels usually consist of certain noisy labels \cite{xiao2015learning,lee2018cleannet,jiang2020beyond,wei2021learning}. The flipping from clean labels to noisy labels is usually described by the noise transition matrix $T(X)$, with its element denoted by
$T_{ij}(X)=\p(\nY=j|Y=i, X)$. We denote the obtained noisy training dataset as $\widetilde{S}:=\{(x_i, \tilde{y}_i)\}_{i=1}^n$, given by random variables $(X, \nY)$, which is assumed to be drawn from $\widetilde{\mathcal{D}}$. 

Though we only have access to noisily labeled long-tailed data $\widetilde{S}$, 
our goal remains to obtain the optimal classifier with respect to a clean and balanced distribution $\mathcal D$:
   $ \min_{f\in\mathcal{F}} ~\mathbb{E}_{(X, Y)\thicksim \mathcal D} \left[\ell(f(X), Y)\right]
$,
where $f$ is the classifier chosen from the hypothesis space $\mathcal{F}$, and $\ell(\cdot)$ is a calibrated loss function (e.g., CE)
Furthermore, we do not assume the knowledge of the sub-population information during training. We are interested in how sub-populations intervene with the learning performance and how we could improve by treating the sub-populations with special care. 



\section{Disparate Influences of Sub-Populations: An Empirical Study}
\label{sec:emp_analysis}
In this section, we empirically illustrate the disparate influence of sub-populations when learning with noisily labeled data. Inspired by the literature on using the influence function to capture the impact of a subset of training data, we define influence metrics at the sub-population level and perform a multi-faceted evaluation of how the imbalanced sub-populations affect the learning performance. 
We take the long-tailed sub-populations for illustration and defer the results of head populations to Appendix \ref{app:exp}. 

\textbf{Influences:}
In the literature of explainable deep learning, the notions of influence can be different, e.g., the influences of features on an individual sample prediction \cite{ribeiro2016should,sundararajan2017axiomatic, lundberg2017unified, feldman2020neural}, the influences of features on the loss/accuracy of the model \cite{owen2017shapley,owen2014sobol}, the influences of training samples on the loss/accuracy of the model \cite{jia2019towards}. In this section, we focus on the influence of a sub-population on both the sub-population level and the individual sample level.

We now empirically demonstrate the role of sub-populations when measuring the test accuracy, and the prediction of model confidence on test samples. For the synthetic long-tailed noisy training dataset, we first flip clean labels of the class-balanced CIFAR-10 dataset to any other classes, and there exist 20\% wrong labels in all. We then adopt the class-imbalanced \cite{cui2019class} CIFAR-10 dataset to select a long-tailed distributed amount of samples for each class (by referring to clean labels). As for the separation of sub-populations, we adopt the $k$-means clustering to categorize the extracted features of each feature given by the Image-Net pre-trained model. Since sub-population information sometimes may not be available for training use, understanding the influences of such division of sub-populations is beneficial. More separation details and experiment designs could be found at \ref{sec:exp_detail}. 

We explore the influences of tail sub-populations on performances of cross-entropy (ce) loss, the forward loss correction (fw) \cite{patrini2017making}, label smoothing (ls) \cite{lukasik2020does}, and the peer loss (pl) \cite{liu2020peer}. There are 17 sub-populations (train) with less than 50 instances considered as the tail section.

We illustrate observations on several randomly selected tail sub-populations. Results of more sub-populations are deferred to Appendix  \ref{app:more}.


\subsection{Influences on Sub-Population Level (Test Accuracy)}
We start with the influence of sub-populations in the test set. We adopt the (population-level) test accuracy changes when removing all samples in the sub-population $\mathcal G_i$ during the training procedure to capture the influences of a sub-population on each sub-population at the test set: 
\begin{mybox1}
\begin{align*}
&\text{Acc}_{\text{p}}(\mathcal{A}, \widetilde{S}, i, j)=\P_{\substack{f\leftarrow \mathcal{A}(\widetilde{S})\\
  (X', Y', G=j)}} (f(X')=Y') - \P_{\substack{f\leftarrow \mathcal{A}(\widetilde{S}^{\backslash i})\\
   (X', Y', G=j)}} (f(X')=Y'),
\end{align*}
\end{mybox1}
where in the above two quantities, $f\leftarrow\mathcal{A}(\widetilde{S})$ indicates that the classifier $f$ is trained from the whole noisy training dataset $\widetilde{S}$ via Algorithm $\mathcal{A}$, $f\leftarrow \mathcal{A}(\widetilde{S}^{\backslash i})$ means $f$ is trained on $\widetilde{S}$ without samples in the sub-population $\mathcal{G}_i$. $(X', Y', G=j)$ denotes the test data distribution given that the samples are from the $j$-th sub-population. 

In Figure \ref{fig:clean_class}, the $x$-axis denotes the loss function for training, and the $y$-axis visualized the distribution of $\{\text{Acc}_{\text{p}}(\mathcal{A}, S, i, j)\}_{j\in[100]}$ (top) and $\{\text{Acc}_{\text{p}}(\mathcal{A}, \widetilde{S}, i, j)\}_{j\in[100]}$ (bottom) for several long-tailed sub-populations ($i=52, 70, 91$) under each robust method, where "$S$" refers to the clean training samples and "$\widetilde{S}$" denotes the noisy version. The blue zone shows the 25-th percentile ($Q_1$) and 75-th percentile ($Q_3$) accuracy changes, and the orange line indicates the median value. Accuracy changes that are drawn as circles are viewed as outliers. Note that all sub-figures have the same limits for $y$-axis. It is clear to observe the top three figures have lower variance than the bottom ones, indicating that: \begin{mybox}
\begin{observation}\label{obs1}
Compared with clean training, tail sub-populations in noisy training tend to have a more significant influence on the test accuracy. 
\end{observation}
\end{mybox}

\begin{figure}[!htb]
    \centering
    \includegraphics[width=0.48\textwidth]{figures/influence/acc_fine_00_part_new.pdf}\hspace{0.2in}
    \includegraphics[width=0.48\textwidth]{figures/influence/acc_fine_02_part_new.pdf}
    \caption{Box plot of the population-level test accuracy changes when removing all samples of a selected long-tailed sub-population during the training w.r.t. 4 methods. (Left: trained on clean labels; Right: trained on noisy labels.)}\label{fig:clean_class}
    \vspace{-0.1in}
    \end{figure}

\subsection{Influences on Sample Level (Prediction Confidence)}
Note that grouping testing samples into classes/sub-populations for analysis may ignore some individual behavior changes, we next consider the influence of sub-populations on the individual test samples. Instead of insisting on the accuracy measure, we adopt the model prediction confidence as a proxy, to see how each test sample got influenced. And we introduce $\text{Infl}(\mathcal{A}, \widetilde{S}, i, j)$ to quantify the influence of a sub-population on a specific test sample:
\vspace{-5pt}
\begin{mybox3}
\vspace{-5pt}
\begin{align*}
    &\text{Infl}(\mathcal{A}, \widetilde{S}, i, j)=\P_{f\leftarrow \mathcal{A}(\widetilde{S})} (f(x'_j)=y'_j) -\P_{f\leftarrow \mathcal{A}(\widetilde{S}^{\backslash i})} (f(x'_j)=y'_j).
\end{align*}
\end{mybox3}

As shown in Figure \ref{fig:influence_clean_part}, we visualize $\text{Infl}(\mathcal{A}, S, i, j)$ (Top) and $\text{Infl}(\mathcal{A}, \widetilde{S}, i, j)$ (Bottom), where $j\in [10000]$ means 10K test samples. For example, $\text{Infl}(\mathcal{A}, \widetilde{S}, i, j)=-1$ means the model prediction confidence on test sample $x'_j$ changed from $0$ to $1$. With the presence of label noise, we observe:
\begin{mybox2}\label{obs2}
\begin{observation}
    Compared with clean training, removing certain tail sub-populations lead to significant changes/influences on the model prediction confidence of more test samples. 
\end{observation} 
\end{mybox2}



\begin{figure}[!htb]
    \centering
\includegraphics[width=0.48\textwidth]{figures/influence/influence_00_part_new.pdf}\hspace{0.2in}
    \includegraphics[width=0.48\textwidth]{figures/influence/influence_02_part_new.pdf}
    \caption{Distribution plot w.r.t. the changes of model confidence on the test data samples using CE loss (Left: trained on clean labels; Right: trained on noisy labels). See Appendix~\ref{app:more} for more details. Removing population 91 appears to result in much worse model prediction confidence on test samples.}\label{fig:influence_clean_part}
    \end{figure}




% !TEX root = ./CauchyCombination.tex
\section{Multiple Hypothesis Testing} \label{secPrelims}

This section introduces the notation on multiple hypothesis testing and the benchmark procedures for addressing the multiple testing problem. 

\subsection{Setting}
Let $H_{i}$ denote the $i^{\text{th}}$ null hypothesis of interest, with $i=1,...,d$,  
and $d$ being the total number of individual hypotheses. To test the $d$ hypotheses, we can use the associated vector of test statistics $\bm{X}=(X_{1},X_{2},\ldots,X_{d})^{^{\prime }}$, one for each hypothesis being tested, or the corresponding raw $p$-values $p_{1},\ldots ,p_{d}$. The test statistics can be independent or % corrected.
correlated. 

%In some cases, like in Section \ref{secApplDriftBurst}, the test statistics are constructed from rolling windows and are extremely serially correlated. 

The first task is to test the global null hypothesis. Let $\mathcal{H}_{0}$ be the collection of null hypotheses of interest. 
The strategy of a classical global test is to abandon the multiplicity issue altogether and replace multiple tests with the global null hypothesis that all elementary hypotheses are true.  The alternative is that at least one elementary hypothesis is false. For example, in high-frequency financial econometrics,  we often need to monitor the presence of certain events (e.g., jumps or drift bursts) within a fixed time period (e.g., within a day). The global null is that there is no occurrence of such an event at all (e.g., 
% in Example 2, none of the stocks has a significant alpha
in Example 1, there is no drift burst within the day or in Example 2, none of the stocks has a significant alpha).
The goal is to get $\alpha$-level control under this global null, i.e., $P_{\mathcal{H}_0} [\text{reject}\, \mathcal{H}_0] \leq \alpha$. The test is conservative when $P_{\mathcal{H}_0} [\text{reject}\, \mathcal{H}_0]$ is strictly less than the theoretical upper bound $\alpha$ and ideal when it is equal to  $\alpha$. 

When, by any  test, the global null $\mathcal{H}_0$ is rejected, the second task is to identify which of the elementary hypotheses $H_{i}$ should be rejected. The set of true hypotheses $\mathcal{T}$, the set of false hypotheses $\mathcal{F}$ and the set of rejected hypotheses $\mathcal{R}$ are defined as: 
\begin{align}
	\begin{split}  \label{eqLocalHypothesis}
		\mathcal{T} &= \{ H_{i}\in \mathcal{H}_0: H_{i} \, \text{is true}\}, \\
		\mathcal{F} &= \{ H_{i}\in \mathcal{H}_0: H_{i} \, \text{is false}\}, \text{and} \\
		\mathcal{R} &= \{H_{i}\in \mathcal{H}_0: H_{i} \, \text{is rejected}\}.
	\end{split}%
\end{align}
The set of true and false hypotheses are unknown. We choose a set of hypotheses to reject. 
on the basis of our data. 
The set of discoveries $\mathcal{R}$ 
should coincide with the set of false hypotheses $\mathcal{F}$ as much as possible.

The goal of various multiple testing corrections is to control the familywise error rate (FWER), defined as the probability of at least one false
rejection in the family, $P[\mathcal{T} \cap \mathcal{R} \neq \varnothing]$,
while retaining the reasonable power in detecting false hypotheses. We want procedures for which the FWER is less than or equal to the upper bound $\alpha$ and ideally as close as possible to the upper bound. We focus on strong control of the FWER, meaning that some of the hypotheses we are testing can be false ($\mathcal{F} \neq \varnothing$), as opposed to the weak FWER control where all hypotheses of interest are true, i.e., $\mathcal{H}_0=\mathcal{T}$.

The probability of falsely rejecting a single hypothesis that is true (i.e., false positive or Type I error) is usually controlled at a nominal $\alpha$-level. However, when the number of tested hypotheses is large, the problem of multiplicity arises: the probability of having at least one false positive conclusion rises well above $\alpha$ if the Type I error of each individual test is controlled at the $\alpha$-level. Numerous controlling procedures have been proposed to deal with this problem. 
We review two classes of controlling procedures: one based on statistical inequalities (Section \ref{ssecOrderdPvals}) and one based on the maximum of the test statistics (Section \ref{ssecMaxTest}).


\subsection{Procedures based on statistical inequalities}
\label{ssecOrderdPvals}

Let us denote by $0 < p_{(1)}\leq p_{(2)}\leq \ldots\leq p_{(d)} < 1$ the set of $d$  ordered (in ascending order) raw $p$-values and $H_{(1)},H_{(2)},\ldots,H_{(d)}$ their corresponding null hypotheses. A single-stage method uses the same rejection 
criterion for all individual hypotheses, like the conservative Bonferroni threshold, while a multi-stage method examines the ordered $p$-values sequentially and adjusts the rejection criterion for each of the individual tests  \citep[e.g.,][]{holm1979simple,hochberg1988sharper,hommel1988stagewise}. 

The Bonferroni method rejects the elementary null hypothesis $H_{(i)}$ if 
$p_{(i)}\leq\alpha/d$ 
for $i=1,\ldots,d$. \citet{holm1979simple} and \citet{hochberg1988sharper} use the same critical values $ \alpha / (d - i + 1)$ depending on the rank of the $p$-value, but reject differently depending on whether they ``step up" or ``step down". The terminologies (``step up" or ``step down") were originally formulated in terms of test statistics which can be confusing when discussing $p$-values.  \citet{holm1979simple} proposes a step-down method that ``steps up" from the smallest $p$-value to the largest one. It is a pessimistic approach: it scans forward and stops as soon as a $p$-value fails to clear its threshold. \citet{hochberg1988sharper} suggests a step-up method that ``steps down" from the largest $p$-value to the	smallest one. It is an optimistic approach: it scans backward and stops as soon as a $p$-value succeeds in clearing its threshold. By construction, Hochberg's procedure will reject as many hypotheses as Holm's procedure. 

\cite{hommel1988stagewise} proposes a more complicated procedure which applies  \citet{simes1986improved}' global test to the $p$-value subset $\left\{ p_{\left(k\right) }\right\} _{ k = i }^{d}$, instead of relying  only on $p_{\left(i\right)}$ to draw inference on $H_{(i)}$ and thus borrows power across hypotheses. 
Hommel's procedure is shown to have higher power than Hochberg's method \citep{hommel1989}. We refer to Appendix \ref{AppOrderedPVals} for more details on the practical implementation of these procedures.


Bonferroni and \citet{holm1979simple}'s method are based on the first-order Bonferroni inequality, which states that given any set of events, the probability of their union is smaller than or equal to the sum of their probabilities. 
Under the null hypothesis, 
the probability that there is at least one hypothesis $H_{(i)}$  for which its raw $p$-value $p_{(i)} \leq \alpha / d$ 
% is not greater than $\alpha$ 
is bounded by $\alpha$: 
\begin{align}  \label{eqIneqPval}
	\Pr\left( \min_{i} p_{(i)} \leq \frac{\alpha}{d} \right) 
	= \Pr\left(\bigcup_{i =
		1}^{d} \left\{ p_{(i)} \leq \frac{\alpha}{d} \right\} \right) 
	&\leq
	\sum^{d}_{i = 1} \Pr \left( p_{(i)} \leq \frac{\alpha}{d}\right) \leq d
	\frac{\alpha}{d} \leq \alpha.
\end{align}
The Bonferroni \eqref{eqIneqPval} inequality 
makes no specific assumption on the dependence between the $p$-values, but protects against the so-called ``worst-case", in which all events are independent and the rejection regions are disjoint (the right half of Equation \eqref{eqIneqPval}) . 
The inequality becomes an equality when all test statistics are independent, and a strict inequality when the hypotheses are dependent. 
In other words, the Bonferroni correction is  conservative when the $p$-values are correlated. 

The methods of \cite{hochberg1988sharper} and \cite{hommel1988stagewise} are based on  \cite{simes1986improved}'s inequality. If a set of hypotheses $H_{(1)}, ...,H_{(d)}$ are all true, the probability of the joint event is: 
\begin{equation}  \label{eqSimes}
	\Pr\left( p_{\left( i\right) }> \frac{i\alpha}{d}, \text{ for all } i=1,\ldots
	,d\right) \geq 1-\alpha.
\end{equation}
\citet{simes1986improved}' inequality was developed for independent uniform $p$-values, and it is applicable for a large family of multivariate distributions. The simulations of \citet{simes1986improved} do show, however, that the test is very conservative for highly correlated multivariate normal statistics, but less so than the classical Bonferroni correction. 



\subsection{Procedures based on the maximum of test statistics}
\label{ssecMaxTest}

Another class of controlling procedures uses the maximum in a group of test statistics: $X_m = \max_{i} \abs{X_{i}}$, with $i = 1, \ldots, d$, to set a stringent critical value. The same critical value can be used for each elementary hypothesis and will control the familywise error rate. In particular, when the individual test statistics are independent and follow the standard normal distribution under the null, the maximum of the test statistics follows a Gumbel distribution when $d$ is large. Quantiles of the Gumbel distribution were used as critical values of the individual tests as a multiple testing correction when, for example, conducting jump tests in high-frequency asset returns \citep[][]{lee2007jumps}. Unfortunately, if the sequence of test statistics exhibits strong correlation, the number of tests severely overstates the effective number of independent copies in a given sample, which makes the Gumbel critical values too conservative \citep[see e.g.,][]{christensen2018drift}. We refer to Appendix \ref{AppOrderedPVals} for more details on the Gumbel distribution. 

Resampling-based methods account for the dependence structure that is specific to the considered dataset, leading to less conservative testing outcomes than the Gumbel-based methods and the inequality-based procedures. Depending on the empirical problem of interest, the resampling can be carried out by bootstrap, permutation, simulation, or randomization \citep[see e.g.,][for detailed discussions on resampling methods and testing procedures]{white2000,romano2005exact,romano2005stepwise,lehmann2005testing}. We refer to Section \ref{secApplDriftBurst} for an example of the resampling-based approach for the drift burst test. 


\section{Cauchy Combination Tests}
\label{secSeqCauchy}

In this section, we first review the global Cauchy combination (GCC) test of \citet{liu2020cauchy} and present our sequential Cauchy combination (SCC) test. While the global test tests the global null hypothesis $\mathcal{H}_0 = \bigcap_{i=1}^{d} H_{i}$ against the alternative hypothesis that at least one of the elementary null hypotheses is false, the sequential test aims at identifying the violations of the elementary null hypotheses while controlling the global error rate. 

\subsection{Global Cauchy combination test}
\label{sec:CC}

The GCC test statistic is constructed from the raw $p$-values of the test statistics $X_i$, which follow a uniform distribution between $0$ and $1$ under the  null hypothesis. The idea of this test is first to transform the uniformly distributed $p$-values into standard Cauchy variates using the formula $\tan \{(0.5-p_{i})\pi \}$ and then construct a new test statistic as the weighted sum of these transformed $p$-values. The new test statistic is denoted by $\tilde{T}$ and defined as: 
\begin{equation}
	\label{eqCauchyStatistic}
	{\normalsize \tilde{T}=\sum_{i=1}^{d}w_{i}\tan \{(0.5-p_{i})\pi \},} 
\end{equation}
in which the $w_{i}$'s are non-negative weights summing to 1. Throughout the paper, the weights $w_{i}$ are set to $1/d$ for $i=1,\ldots,d$ as in \citet{liu2020cauchy}. 

When the raw and hence the transformed $p$-values are independent (resp. perfectly correlated), % under the null hypothesis, 
the new test statistic $\tilde{T}$ is a linear combination of independent (resp. perfectly correlated) Cauchy variates and therefore follows a standard Cauchy distribution because the family of Cauchy densities is closed under convolutions. Although the correlation structure can affect the null distribution of $\tilde{T}$ in the case of general dependence, \citet{liu2020cauchy} show that the impact on the tail is very limited because of the heaviness of the Cauchy tail. Specifically, they prove that: 
\begin{equation}
	\lim_{h\rightarrow \infty }\frac{\Pr\left( \tilde{T}>h\right) }{\Pr\left(
		C>h\right) }=1,  \label{eq:tail}
\end{equation}
in which $C$ is a standard Cauchy random variable, under the null hypothesis $\mathcal{H}_0$ and Assumption \ref{ass1} which requires the test statistics to follow a bivariate zero mean normal distribution.
\begin{assumption}
	\label{ass1} (1) The original test statistics $(X_{i},X_{j})$, for any $1\leq
	i<j\leq d$, follow a bivariate normal distribution; (2) $E\left( \bm{X}%
	\right) =0$, with $\bm{X}=(X_{1},X_{2},\ldots,X_{d})^{^{\prime }}$. 
\end{assumption}
The bivariate normal requirement of Assumption \ref{ass1} is a condition weaker than joint normality, making the procedure applicable for high-dimensional settings. When the dimension $d$ increases at a certain rate with the sample size, the test statistics $\bm{X}$ may not jointly converge to a multivariate normal distribution due to its slower rate of convergence \citep[see][and references therein]{liu2020cauchy} and thus a joint normality assumption is not realistic for those settings. In contrast, the bivariate normality assumption is much weaker and more realistic. There are, of course, applications for which the test statistics are not normally distributed. Through simulations, \citet[][]{liu2020cauchy} show the Cauchy approximation is still accurate when the normality assumption is violated and follows a multivariate Student-$t$ distribution (with 4 degrees of freedom) instead. 
We refer to Section \ref{secApplFan} for a showcase example in finance with test statistics being Student-$t$ distributed. 

The result in  \eqref{eq:tail} suggests that, under the null hypothesis $\mathcal{H}_0$, the tail of the Cauchy combination test statistic is approximately Cauchy under arbitrary dependence structures, so that a $p$-value of the Cauchy combination test, denoted 
$\widetilde{p}$, can  be calculated from the standard Cauchy distribution. Suppose that we observe $\tilde{T}=t_{0}$, then: 
\begin{equation}
	\label{eqCauchyPval}
	\widetilde{p}=\frac{1}{2}-\frac{\arctan t_{0}}{\pi }. 
\end{equation}

Using the GCC $p$-values, the tail result in \eqref{eq:tail} can be equivalently stated as the actual size converging to the nominal size $\alpha$ as the significance level tends to zero:  % \textit{i.e.}, 
\begin{equation}
	\lim_{\alpha\rightarrow 0 }\frac{\Pr\left( \widetilde{p} \leq \alpha\right) }{\alpha}=1, \label{eq:wFWER}
\end{equation} 
The approximation should be particularly accurate for small $\alpha$'s, which are of particular interest in large-scale problems as in Examples 1 and 2. The simulations in \citeauthor{liu2020cauchy} show that when the significance level is moderately small ($\alpha = 10^{-1}, 10^{-2},10^{-3},10^{-4},10^{-5}$), the $p$-value calculation is accurate:  the ratio of the empirical size to the significance level is close to 1 for different types of correlations. 
Put differently, the GCC test achieves the weak familywise error rate control as the empirical size is very close to the nominal size $\alpha$ regardless of the correlation structure. 


Figure \ref{figCauchyPvalsAR} illustrates the fact that while the dependence between the individual test statistics $X_i$ can affect the null distribution of the GCC test statistic, the impact of the dependence is marginal on the tail. We simulate a vector of $d$ test statistics  $\bm{X}$ from a $d$-variate normal distribution with correlation matrix $\bm{\Sigma}$, \textit{i.e.}, $N_d(\bm{0}, \bm{\Sigma})$ with $\bm{\Sigma} = (\sigma_{ij})$ and $d = 300$. The diagonal elements $\sigma_{ii}=1$ for all $i=1,\ldots,d$ and the off-diagonal elements $\sigma_{ij} = \theta^{\abs{i-j}}$ for $i \neq j$, with $\theta = 0.2, 0.4, 0.6,$ $0.8, 0.90, 0.95$. The simulation is repeated $10^7$ times. For each draw, we calculate the GCC test statistic \eqref{eqCauchyStatistic} and its corresponding $p$-value \eqref{eqCauchyPval}. The histogram of the $10^7$ GCC $p$-values is displayed in Figure \ref{figCauchyPvalsAR}. For a low level of autocorrelation (i.e., $\theta=0.2$), the distribution of the $p$-values is close to a uniform distribution. When the level of autocorrelation is higher, there is a pothole in the middle and a bump at the end of the histogram, but whatever the strength of the autoregressive parameter, the percentage of the GCC $p$-values in the first bin is always around $5$\% as is ensured by the limit result in \eqref{eq:wFWER}. 


\begin{figure}[p]
	\caption{The impact of dependence on the tail of the GCC test statistic}
	\label{figCauchyPvalsAR}
	\centering
	
	\par
	
	\subfloat[${\theta} = 0.2$ ]{{\includegraphics[width=.40\textwidth,angle =
			-90,scale=0.70]{Sample_pval_dim_300_rho_0.2_alpha0.05.eps} }} 
	\subfloat[$\theta = 0.4$ ]{{\includegraphics[width=.40\textwidth,angle =
			-90,scale=0.70]{Sample_pval_dim_300_rho_0.4_alpha0.05.eps} }} 
	
	\vspace{0.4cm}
	
	\subfloat[$\theta = 0.6$ ]{{\includegraphics[width=.40\textwidth,angle =
			-90,scale=0.70]{Sample_pval_dim_300_rho_0.6_alpha0.05.eps} }} 
	\subfloat[$\theta = 0.8$ ]{{\includegraphics[width=.40\textwidth,angle =
			-90,scale=0.70]{Sample_pval_dim_300_rho_0.8_alpha0.05.eps} }} 
	
	\vspace{0.4cm}	
	
	\subfloat[$\theta = 0.90$]{{\includegraphics[width=.40\textwidth,angle =
			-90,scale=0.70]{Sample_pval_dim_300_rho_0.9_alpha0.05.eps} }} 
	% 
	\subfloat[$\theta = 0.95$]{{\includegraphics[width=.40\textwidth,angle =
			-90,scale=0.70]{Sample_pval_dim_300_rho_0.95_alpha0.05.eps} }} 
	
	\par
	\begin{minipage}{1.0\linewidth}
		\begin{tablenotes}
			\small
			\item {
				\medskip
				Note: We plot histograms of GCC $p$-values \eqref{eqCauchyPval} for various correlation strengths. The individual test statistics are drawn from a $d$-variate normal distribution $N_d(\bm{0}, \bm{\Sigma})$ with $\bm{\Sigma}= (\sigma_{ij})$ and $d=300$. The diagonal elements of the covariance matrix $\sigma_{ii}=1$ for all $i=1,\ldots,d$ and the off-diagonal elements  $\sigma_{ij} = \theta^{\vert i-j \vert}$ for $i\neq j$, with $\theta = 0.2, 0.4, 0.6,$ $0.8, 0.90, 0.95$. We compute the GCC $p$-value from the test statistic sequence. The simulation is repeated $10^7$ times. The simulated GCC $p$-values are sorted into bins with the bin edges being a sequence of edges from 0 to 1 with a width of  0.05. 				Each bin includes the right edge (right-closed) but does not include the left edge (left-open). We highlight the first bin in black and we
				also add a text note with the probability of $p$-values being in the first bin. }
		\end{tablenotes}
	\end{minipage}
\end{figure}

Interestingly,  \citet{liu2020cauchy} show that the tail property \eqref{eq:tail}  also holds when the number of hypotheses $d$ diverges to infinity at a rate of $o\left(h^{\eta }\right) \ $with $0<\eta <1/2$ and the following additional assumption is satisfied.
\begin{assumption}
	\label{ass2} Let $\mathbf{\Sigma }=corr\left( \bm{X}\right) $. 
	(1) The	largest eigenvalue of the correlation matrix $\lambda _{\max}\left( \mathbf{%
		\Sigma }\right) \leq C_0$ for some constant $C_0>0$; 
	(2) $\max_{1\leq i<j\leq
		d}\left\{ \sigma _{i,j}^{2}\right\} \leq \sigma _{\max }^{2}<1$ for some
	constant $0<\sigma _{\max }^{2}<1$, where $\sigma _{i,j}$ is the $\left(
	i,j\right) $ element of $\mathbf{\Sigma }$.
\end{assumption}
The additional assumptions on the correlation matrix are mild and standard in high dimensional settings and are general enough to incorporate a large class of tests. 


\subsection{Sequential Cauchy Combination Test}

The main contribution of this paper is the sequential Cauchy combination (SCC) test, which unravels the GCC test to make statements on the elementary hypotheses. The raw $p$-values are sorted in ascending order so that  $p_{(1)}\leq p_{(2)}\leq \ldots \leq p_{(d)}$, which is standard for step-down and step-up sequential procedures (see Section \ref{ssecOrderdPvals}). For the inference on hypothesis $H_{\left( i\right) }$ we compute a Cauchy combination test statistic ${\normalsize \tilde{T}}_{\left( i\right) }$ from a subset of $p$-values, running from $p_{(i)}$ to $p_{(d)}$ as:  
\begin{equation}
{\normalsize \ \tilde{T}%
		_{\left( i\right) }=\sum_{j=i}^{d}w_{j}\tan \{(0.5-p_{(j)})\pi \}
	}.
	\label{eq:CC_mt}
\end{equation}
The corresponding $p$-value is: $$ \widetilde{p}_{(i)}=\frac{1}{2}-\frac{\arctan \tilde{T}_{\left(i\right) }}{\pi }.$$ We reject the null hypothesis $H_{(i)}$ if  $\widetilde{p}_{(i)}\leq\alpha$. Like the step-up procedure of  \citet{hommel1988stagewise}, the SCC test also borrows power across hypotheses: the test statistic $\tilde{T}_{(i)}$ is computed from the raw $p$-values associated with $\mathcal{H}_0^{(i)}=\bigcap_{j=i}^{d} H_{(j)}$.


\subsubsection*{Theoretical Properties}
The SCC testing procedure can be viewed as a sequential rejection procedure. Let $\mathcal{R}^{(s)}$ be the collection of rejected hypothesis after step $s$, with $s=\left\{1,2,\ldots,d\right\}$. The hypothesis of interest and decision rules in each step  are illustrated in Table \ref{tabDecisionRule}.
\begin{table}[H]
	\caption{Decision rule in the sequential Cauchy combination test}
	\label{tabDecisionRule}
	\centering
	\begin{tabular}{p{1.cm}p{3.8cm}p{10.5cm}}
		\hline
		Step & Hypothesis & Decision\\ 
		$s=1$ & $\mathcal{H}_0^{\left( d\right) }=H_{(d)}$ & 
		If $\widetilde{p}_{(d)}\leq\alpha$ then reject $\mathcal{H}_0^{\left( d\right) }$ and include $H_{(d)}$ in $\mathcal{R}^{(1)}$
		\\
		$s=2$ & $\mathcal{H}_0^{\left( d-1\right) }=\bigcap_{j=d-1}^d H_{(j)}$  
		& 
		If $\widetilde{p}_{(d-1)}\leq\alpha$ then reject $\mathcal{H}_0^{\left( d-1\right) }$  and include $H_{(d-1)}$ in $\mathcal{R}^{(2)}$  
		\\
		$\ldots$  & $\ldots$  & $\ldots$  \\ 
		$s=d$ & $\mathcal{H}_0^{\left( 1\right) }=\bigcap_{j=1}^d H_{(j)}$ & 
		If $\widetilde{p}_{(1)}\leq\alpha$ then reject $\mathcal{H}_0^{\left( 1\right) }$ and include $H_{(1)}$ in $\mathcal{R}^{(d)}$ \\
		\hline
	\end{tabular}
\end{table}
Let $\mathcal{N}\left(\mathcal{R}^{(s)}\right)$ be the successor function, representing hypotheses to be rejected in the next step given that $\mathcal{R}^{(s)}$ has been rejected. For the SCC test, the successor function is defined as: 
\[
\mathcal{N}\left(\mathcal{R}^{(s)}\right)=\left\{H_{(d-s)} :  \widetilde{p}_{(d-s)} \leq \alpha_{\mathcal{R}^{(s)}}=\alpha\right\}.
\]
The cut-off value is fixed (i.e., $\alpha_{\mathcal{R}^{(s)}}=\alpha$) instead of depending on the rejection set $\mathcal{R}^{(s)}$ like in many other sequential procedures. According to the sequential rejection principle of \cite{goeman2010sequential},  the SCC test  achieves a strong family-wise error rate control if the following two conditions are satisfied. 
\begin{condition}[Monotonicity]
	For every $\mathcal{R}^{(s)}\subseteq \mathcal{R}^{(l)} \subset \mathcal{H}_{0}$, 
	\[
	\mathcal{N}(\mathcal{R}^{(s)}) \subseteq \mathcal{N}(\mathcal{R}^{(l)}) \cup \mathcal{R}^{(l)}
	\]
	almost surely. 
\end{condition}
% By construction, 
The transformed $p$-values of the SCC test are monotonic by construction, with $\widetilde{p}_{(d)}$ being the largest for the smallest set of global null hypotheses $\mathcal{H}_0^{(d)} = H_{(d)}$ and $\widetilde{p}_{(1)}$ being the smallest for the largest set of global nulls $\mathcal{H}_0^{(1)} = \bigcap_{j=1}^{d} H_{(j)}$ (see Figure \ref{figSequentialCauchyIllustration}(e) for an illustration of the monotonic $p$-values). Note that the largest set of global null hypotheses has the same null specification as the GCC test \eqref{eqCauchyStatistic}. It follows that that $\widetilde{p}_{(s)}\geq \widetilde{p}_{(l)}$. Since the cut-off value is fixed, the monotonicity condition of the successor function is satisfied.

\begin{condition}[Single-step condition] \label{SS} 
	When $\mathcal{H}_{0}^{(i)} =\mathcal{T}$, 
	$\Pr\left( \widetilde{p}_{(i)} \leq \alpha\right) \leq \alpha. $
\end{condition}
Condition \ref{SS} requires FWER control of the underlying test of SCC (i.e., the Cauchy combination test) at the ``critical case" in which all hypotheses of interest are true: $\mathcal{H}_{0}^{(i)} =\mathcal{T}$. The condition can be rewritten as $\Pr{\mathcal{N}(\mathcal{F})\subseteq \mathcal{F}} \geq 1-\alpha$ and has been shown to be satisfied by \cite{liu2020cauchy}. In fact,  when $\alpha$ is very small, the familywise false rejection probability of the GCC test under the null is not only bounded by $\alpha$ but also approaches the nominal size $\alpha$, as stated in \eqref{eq:wFWER}, which implies that it is less conservative than tests based on statistical inequalities or tests which impose independence in the presence of correlation. 

The theorem below follows directly from \cite[Theorem 1]{goeman2010sequential} for general sequential rejection procedures, so that Type I control in the critical case is sufficient for overall familywise error control of the sequential procedure. 
\begin{theorem}\label{thm}
	The SCC testing procedure satisfies both the monotonicity and the single-step condition and achieves the strong FWER control:  
	\[
	\lim_{\alpha\rightarrow 0}\Pr\left\{\mathcal{R}^{(d)} \subseteq \mathcal{F} \right\} \geq 1-\alpha, 
	\]
	under Assumption \ref{ass1} if $d$ is fixed and under Assumptions \ref{ass1} and \ref{ass2} if $d\rightarrow \infty$.
\end{theorem}


\subsubsection*{An Illustration}

A more prescriptive description of the SCC testing procedure is as follows: 
\begin{enumerate}
	
	\item Obtain raw $p$-values $p_1, p_2,\ldots, p_d$ corresponding to the null hypotheses $H_{1}, H_{2},\ldots, H_{d} $;%
	
	\item Order the raw $p$-values in ascending order, 	$p_{(1)},p_{(2)},\ldots,p_{(d)}$, with corresponding null ordered hypotheses $H_{(1)},H_{(2)},\ldots,H_{(d)}$;
	
	\item Calculate the SCC test statistic $\tilde T_{(i)}$ and the transformed Cauchy $p$-values $\widetilde{p}_{(i)}$ from a subset of the ordered $p$-values $\left\{p_{(j)}\right\} _{j=i}^{d}$ using \eqref{eq:CC_mt} for $i=1,\ldots,d$;
	
	\item Obtain the rejection set $\mathcal{R}=\left\{H_{\left(i\right)} : \widetilde{p}_{(i)}\leq \alpha\right\}$. 
\end{enumerate}

Figure \ref{figSequentialCauchyIllustration} illustrates the sequential Cauchy combination procedure on a simulated sequence of test statistics. The top row shows the simulated test statistics and their corresponding $p$-values, of which some hypotheses are under the null (grey dots) and some are under the alternative (black dots). The data-generating process is the same as that in Figure \ref{figCauchyPvalsAR} with $\theta=0.9$ and $d=100$. We add constant signals for $5$ out of 100 hypotheses,  with a signal strength equal to $\pm2.806$. The sign of the signal is the same as the sign of the test statistic under the null, such that the signal always amplifies the magnitude of the test statistic. 
The GCC test rejects the global null at $\alpha = 5\%$ for this sequence of $p$-values, which tells us there is at least one signal in the sequence.  
%The estimated first-order autocorrelation of the simulated test statistics is equal to $0.7910$ under the null and is equal to $0.4987$ under the alternative. 

\begin{figure}[p]
	\caption{Rejection procedure of the sequential Cauchy Combination test}
	\label{figSequentialCauchyIllustration}\centering

	\par
	
	\subfloat[Raw test statistics]{{\includegraphics[width=.31\textwidth,angle =
			-90]{1_tstat_d_100_rho_0.9_signal_5_5} }} 
	\subfloat[Raw
	$p$-values]{{\includegraphics[width=.31\textwidth,angle = -90]{2_pvals_d_100_rho_0.9_signal_5_5} }}
	
	\vspace{0.4cm}	
	
	\subfloat[Ordered raw $p$-values]{{\includegraphics[width=.31\textwidth,angle =
			-90]{3_spvals_d_100_rho_0.9_signal_5_5} }} 
	
	\vspace{0.4cm}	
	
	\subfloat[SCC test statistics
	]{{\includegraphics[width=.31\textwidth,angle = -90]{4_ctstats_d_100_rho_0.9_signal_5_5}}}
	\subfloat[SCC 
	$p$-values]{{\includegraphics[width=.31\textwidth,angle = -90]{4_cpvals_d_100_rho_0.9_signal_5_5}}}
	
	
	\begin{minipage}{1.0\linewidth}
		\begin{tablenotes}
			\small
			\item {
				\medskip
				Note: We illustrate the mechanics of the SCC procedure on a simulated test statistic sequence with sparse signals. The top row shows raw test statistics and $p$-values of which some hypotheses are under the null and some are under the alternative. The test statistics are simulated from $N_{d}(\bm{0},\bm{\Sigma})$ as in Figure \ref{figCauchyPvalsAR}. We set $d=100$, $\theta=0.9$ and add $5\%$ signals. The strength of the signal is $\pm2.806$, with its sign identical to that of the test statistic under the null. The horizon line in panel (e) is the 5\% significance level.
			}
		\end{tablenotes}
	\end{minipage}
\end{figure}

The SCC test can tell us which individual $p$-values trigger the rejection of the GCC test. The middle row plots the raw $p$-values in ascending order and the bottom row plots its sequential Cauchy combination test statistics and $p$-values. Specifically, the bottom right panel shows that the SCC $p$-values $\widetilde{p}_{(i)}$ decrease as $i$ moves from $d$ to $1$. In this example, the SCC test rejects three out of the five alternative hypotheses and does not reject under the null hypothesis. The rejections correspond to the 4$^\text{th}$, 29$^\text{th}$ and  46$^\text{th}$ hypotheses in the top row. Note that the smallest SCC $p$-value corresponds to the $p$-value of the GCC test of \citet{liu2020cauchy}, which performs the test on the largest set of hypotheses. 

\section{Experiments}
\label{sec:exp}

In this section, we demonstrate the wide range of applications and the high capabilities of Uni-Fusion. 
First, we evaluate Uni-Fusion in application 1) Incremental surface and color reconstruction, comparing its performance with SOTAs.
%
For applications 2) and 5), which are new topics, no specific benchmarks are available. 
Therefore, we showcase the performance on existing results.
%
Next, we implement application 3) and compare it with SOTA zero-shot semantic segmentation models.
%
Finally, for application 4), since infrared data is not commonly used, we collect our own dataset containing infrared values and show all applications on this data.

\subsection{Implementation Details}
\label{sec:exp:details}

In the experiments, we use our sample-based GPIS for local geometry encoding.
For each point, two additional points are sampled along normal direction, one positive and one negative, with distance $d_s=0.1$ in the local voxel's normalized space. 
Compared to derivative-based GPIS, our sample-based GPIS is more efficient in both space and time. 
For the encoder, we randomly sample $256$ anchor points from the range $[-0.5,0.5]^3$.
We utilize the first $20$ eigenpairs, resulting in a feature dimension of $20$.
The model selection process is discussed in the ablation study.

Different latent maps use different granularities.
For the surface LIM, we use a voxel size of $5\si{\centi\meter}$. 
For color which requires later comparison to NeRF, we use a voxel size of $2\si{\centi\meter}$.
For other property LIM and feature LIM, we use a voxel size of $10\si{\centi\meter}$.

For smooth reconstruction, the encoded voxel is designed overlapped following~\cite{huang2021di}.
The encoded voxel uses twice the voxel size, resulting in a half-space overlap with each neighboring voxel.
During meshing, SDFs are retrieved and interpolated from the overlapped voxels~\cite{huang2021di}.
While for the remaining properties, we sample only from its own voxel part.

The implementation runs on a PC with AMD Ryzen 9 5950X 16-core CPU and an Nvidia Geforce RTX 3090 GPU (24 GB).

\subsection{Datasets}

We evaluate incremental reconstruction on the ScanNet dataset~\cite{dai2017scannet}, TUM RGB-D dataset~\cite{sturm2012benchmark}, and Replica dataset~\cite{sucar2021imap}.
Using MSG-Net~\cite{zhang2018multi}'s material set, we transfer styles to the 3D canvas.
For open-vocabulary scene understanding, we evaluate on ScanNet segmentation data~\cite{qi2017pointnet++} and S3DIS dataset~\cite{armeni20163d}.

\subsubsection{ScanNet~\cite{dai2017scannet}}

ScanNet is a densely annotated RGB-D video dataset.
It is captured with the structure sensor~\cite{occipital} and contains 1513 scenes for training and validation.
For each scene, both images and a 3D mesh is provided, along with their 2D and 3D semantic annotations. 

ScanNet provides 312 scenes for validation, which contains a wide range of different room structures.
It has now been widely used in the thorough evaluation of the performance of reconstruction and semantic segmentation.

\subsubsection{TUM RGB-D~\cite{sturm2012benchmark}}

TUM RGB-D is a benchmark to mainly evaluate the tracking performance.
It is captured with Microsoft Kinect sensor together with ground-truth trajectory from the sensor.

\subsubsection{Replica~\cite{sucar2021imap}}

The Replica dataset refers to iMAP's pre-processed dataset~\cite{sucar2021imap}.
It is a synthetic rendered RGB-D dataset from given 3D models.
The advantage of including this dataset is that Replica does not have motion blur. 
This is better to evaluate the capability of the algorithms on reconstructing surface color.

\subsubsection{MSG-Net Style~\cite{zhang2018multi}}

MSG-Net provides material images for transfering the styles.
We select 21style fold for demonstration.
These images are given in \cref{fig:style} together with our result.

\subsubsection{ScanNet Point Cloud Segmentation Data~\cite{qi2017pointnet++}}

For point cloud semantic segmentation benchmarking, PointNet++~\cite{qi2017pointnet++} preprocesses the original ScanNet~\cite{dai2017scannet} and generates subsampled point clouds and corresponding annotations for each scene.

\subsubsection{S3DIS~\cite{armeni20163d} and 2D-3D-S~\cite{armeni2017joint}}

S3DIS is a semantic segmentation dataset for 3D point clouds.
Which is also a subset of the 2D-3D-S dataset.
The 2D-3D-S dataset is a multi-modality dataset containing 2D, 2.5D and 3D domains. 
This dataset is densely annotated with semantic labels.

Note that 2D-3D-S's 2D captures is not a RGB-D video as ScanNet.
2D-3D-S's images only have small overlap. 
Therefore, it is only suitable for semantic segmentation and not for incremental reconstruction.

\subsection{Baselines}

For online surface mapping evaluation, we select TSDF-Fusion~\cite{curless1996volumetric}, iMAP~\cite{sucar2021imap}, SOTA DI-Fusion~\cite{huang2021di} and BNV-Fusion~\cite{li2022bnv} as four baseline methods.

For the color field, we choose TSDF-Fusion~\cite{curless1996volumetric}, $\sigma$-Fusion~\cite{rosinol2023probabilistic}, iMAP~\cite{sucar2021imap}, NICE-SLAM~\cite{zhu2022nice} and even the recent hot Neural Radiance Fields model NeRF-SLAM~\cite{rosinol2022nerf} as five baselines.
While including NeRF in the comparison may not be entirely fair, we want to show how Uni-Fusion narrows the performance gap.

For the scene understanding application, we evaluate generalized zero-shot point cloud semantic segmentation with ZSLPC~\cite{cheraghian2019zero}, DeViSe~\cite{frome2013devise} and SOTA 3DGenZ~\cite{michele2021generative} for comparison.

\subsection{Metrics}

For incremental reconstruction, we evaluate the geometric reconstruction using \textbf{Accuracy}, \textbf{Completeness}, and \textbf{F1 score} according to SOTA BNV-Fusion. It firstly uniformly samples $100,000$ points from the reconstruction and ground truth meshes respectively.
Then \textbf{Accuracy} (\textbf{Completeness}) measures the percentage of reconstruction-to-groundtruth (groundtruth-to-reconstruction) distances that are lower than $2.5\si{\centi\meter}$ threshold. \textbf{F1 score} is the harmonic mean of accuracy and completeness.
For tracking performance, we use \textbf{ATE RMSE}.

To evaluate color reconstruction, we follow SOTA on this topic, NeRF to render both depth and RGB images to compare the image level \textbf{Depth L1} and \textbf{RGB PSNR}.

To compare scene understanding, we follow zero-shot point cloud semantic segmentation SOTA 3DGenZ to evaluate the \textbf{Intersection-of-Union (IoU)} and \textbf{Accuracy}.


\subsection{Reconstruction Results}

For evaluation, we first use the ScanNet validation set with 312 sequences to thoroughly test the geometric reconstruction on a large variant of scenes.
%
Then, we use TUM RGB-D to compare our modified tracking model with related works.
Because this part is not the main contribution, we give a rough overview of the tracking results.
%
To fairly evaluate the color reconstruction, we use the high quality rendered Replica dataset to compare with related works, including NeRF.

%\subsubsection{Object}
% on instance-gp
% Objective data usually has more fine detail
% 1. for detail precision
% A: no, object reconstruction is not as good as instance-ngp, so cancelled.
\begin{table*}[!]
	\centering
	\caption{Comparison to ScanNet~\cite{dai2017scannet}.
       Our method generalizes better to various scenes.
       $^*$ indicates the result from our runs of the official BNV-Fusion code.}
	\small
	%\setlength{\tabcolsep}{5mm}
	\setlength{\tabcolsep}{0.9em}
		%\resizebox{\textwidth}{!}{
		\begin{tabular}{l  c c c| c c c }
			\toprule
			Method & \begin{tabular}{@{}c@{}}Pre-Train\\ with extra dataset\end{tabular} & \begin{tabular}{@{}c@{}}Train \\ with sequences\end{tabular} & Real-time & Accuracy (\%) $\uparrow$ & Completeness (\%) $\uparrow$ & F1 score $\uparrow$ \\
			\midrule
			TSDF Fusion~\cite{zhou2018open3d} & None & None & $\checkmark$ &73.83 & 85.85 & 78.84 \\
			iMAP~\cite{sucar2021imap} & None & Online train& &68.96 & 82.12 & 74.96 \\
			DI-Fusion~\cite{huang2021di} &Object Pretrain & None & $\checkmark$&66.34 & 79.65 & 72.97 \\
			BNV-Fusion~\cite{li2022bnv} &Object Pretrain &  Post Optimization& &{74.90} & \textbf{88.12} & {80.56} \\
			BNV-Fusion$^{*}$~\cite{li2022bnv} &Object Pretrain & Post Optimization &&{73.42} & {81.75} & {77.18} \\
			\textbf{Uni-Fusion (Ours)} &None &None &$\checkmark$&\textbf{80.43} & {84.91} & \textbf{82.44} \\
			\bottomrule
		\end{tabular}
	  %}
	\label{tab:scannet}
	\vspace{-.6cm}
\end{table*}
\begin{figure*}[t]
	\subfloat[width=.33\textwidth][Accuracy]{
		\centering
		\includegraphics[width=.22\linewidth]{im/exp/recons/scannet/scannet_acc.png}
		\includegraphics[width=.1\linewidth]{im/exp/recons/scannet/scannet_acc_box.png}
	}
	\subfloat[width=.33\textwidth][Completeness]{
		\centering
		\includegraphics[width=.22\linewidth]{im/exp/recons/scannet/scannet_comp.png}
		\includegraphics[width=.1\linewidth]{im/exp/recons/scannet/scannet_comp_box.png}
	}
	\subfloat[width=.33\textwidth][F1 score]{
		\centering
		\includegraphics[width=.22\linewidth]{im/exp/recons/scannet/scannet_F1.png}
		\includegraphics[width=.1\linewidth]{im/exp/recons/scannet/scannet_F1_box.png}
	}
	\label{fig:recon:scannet:elementwise}
	\caption{Quantitative comparison on 312 scenes of the ScanNet validation set.
       We demonstrate the performance of SOTA BNV-Fusion and our Uni-Fusion.
       We sort our evaluation value and reordered all of the scores.
       The zigzag pink is the BNV-Fusion result;
       we also plot a deep-pink smoothed curve for better visualization.}
\end{figure*}

\subsubsection{Evaluation on ScanNet Dataset~\cite{dai2017scannet}}
\label{sec:exp:scannet}

We use the 312 diversified scenes from the ScanNet validation set to evaluate the performance of surface reconstruction. 
We follow the pure mapping SOTA BNV-Fusion to take every 10th posed frame as input. 
%
Without using any learning (in contrast iMAP, DI-Fusion, and BNV-Fusion do) or any post optimization (as BNV-Fusion does), our Uni-Fusion is capable to achieve precise continuous mapping performance. 

As shown in~\cref{tab:scannet}, our Uni-Fusion achieves \textbf{$+6$ higher accuracy} than the incremental surface reconstruction SOTA BNV-Fusion.
Our model does not exceed on completeness comparing to BNV-Fusion that support completion in post-optimization.
Though, Uni-Fusion's completion is still much higher than one other optimization based model iMAP.
%We consider it because our model does not support hole-completion as the optimization based models iMap and BNV-Fusion.
Overall, our Uni-Fusion model achieves higher F1-scores that quantifies the overall quality.

Please note that, SOTA BNV-Fusion is not real-time capable, since it requires post optimization of all fed frames.
Without the post-optimization, the real-time model Di-Fusion shows much worse results.
However, our \textbf{real-time} model \textbf{Uni-Fusion} is able to achieves \textbf{much better} reconstruction quality than these approaches even without post-optimization. 

\newcommand{\scannetImSize}{.16}
\begin{figure*}[t!]
	\centering
	\setlength{\tabcolsep}{0.1em}
	\renewcommand{\arraystretch}{.1}
	\begin{tabular}{|c | c |c |||c |c | c|}
		\hline
		{\Large{BNV-Fusion}} & {\Large{Uni-Fusion}} &{\Large{Ground Truth}} & {\Large{BNV-Fusion}} &{\Large{Uni-Fusion}} & {\Large{Ground Truth}} \\ \hline \hline
		
\includegraphics[width=\scannetImSize\linewidth]{im/exp/recons/scannet_qualifi/scene0568_00_bnv.png}
		&\includegraphics[width=\scannetImSize\linewidth]{im/exp/recons/scannet_qualifi/scene0568_00_mine.png}
		&\includegraphics[width=\scannetImSize\linewidth]{im/exp/recons/scannet_qualifi/scene0568_00_gt.png}
		&		\includegraphics[width=\scannetImSize\linewidth]{im/exp/recons/scannet_qualifi/scene0164_00_bnv.png}
		&\includegraphics[width=\scannetImSize\linewidth]{im/exp/recons/scannet_qualifi/scene0164_00_mine.png}
		&\includegraphics[width=\scannetImSize\linewidth]{im/exp/recons/scannet_qualifi/scene0164_00_gt.png}\\
		
\includegraphics[width=\scannetImSize\linewidth]{im/exp/recons/scannet_qualifi/scene0249_00_bnv.png}
		&\includegraphics[width=\scannetImSize\linewidth]{im/exp/recons/scannet_qualifi/scene0249_00_mine.png}
		&\includegraphics[width=\scannetImSize\linewidth]{im/exp/recons/scannet_qualifi/scene0249_00_gt.png}
		&		\includegraphics[width=\scannetImSize\linewidth]{im/exp/recons/scannet_qualifi/scene0435_00_bnv.png}
		&\includegraphics[width=\scannetImSize\linewidth]{im/exp/recons/scannet_qualifi/scene0435_00_mine.png}
		&\includegraphics[width=\scannetImSize\linewidth]{im/exp/recons/scannet_qualifi/scene0435_00_gt.png}\\
		
\includegraphics[width=\scannetImSize\linewidth]{im/exp/recons/scannet_qualifi/scene0046_00_bnv.png}
		&\includegraphics[width=\scannetImSize\linewidth]{im/exp/recons/scannet_qualifi/scene0046_00_mine.png}
		&\includegraphics[width=\scannetImSize\linewidth]{im/exp/recons/scannet_qualifi/scene0046_00_gt.png}
		&		\includegraphics[width=\scannetImSize\linewidth]{im/exp/recons/scannet_qualifi/scene0050_00_bnv.png}
		&\includegraphics[width=\scannetImSize\linewidth]{im/exp/recons/scannet_qualifi/scene0050_00_mine.png}
		&\includegraphics[width=\scannetImSize\linewidth]{im/exp/recons/scannet_qualifi/scene0050_00_gt.png}\\
		\hline
	\end{tabular}
	%\captionof{figure}
	\caption{Surface reconstruction on ScanNet dataset.}
	\label{fig:recons:scannet_demo}
	\vspace{-.5cm}
\end{figure*}

We additionally run BNV-Fusion's official implementation (emphasized with $^*$) on the 312 videos of ScanNet and do a post element-wise comparison in \cref{fig:recon:scannet:elementwise}. 
Our result is the {\color{Cyan}light blue} curve, BNV-Fusion's result is colored with {\color{Lavender}pink}.
Scene index is sorted corresponding to the score value of Uni-Fusion.
For better visualization, we smooth BNV-Fusion's curve and plot it with dark pink.
It is obvious that the score of Uni-Fusion is overall higher than BNV-Fusion's. 
Moreover, we use box-plot to conclude the statistics besides the curve plot. Uni-Fusion's scores are distributed in a higher region. For completeness which is less obvious better, Uni-Fusion's box is smaller while in a relative higher position. This means that Uni-Fusion has more stable completeness result while BNV-Fusion is more likely to get low completeness in some cases.

To summarize, our model is almost better on all 312 scenes on all accuracy, completeness and F1-score.
Which is also revealed in \cref{tab:scannet} with BNV-Fusion$^*$, that the BNV-Fusion's official implementation does not exceed Uni-Fusion on all metrics.

We plot reconstruction on selected scenes from ScanNet in~\cref{fig:recons:scannet_demo}. 
Both BNV-Fusion and our Uni-Fusion are able to produce high quality reconstruction.
We see that BNV-Fusion gives lots of small meshes on walls, which are shown as small particles in the reconstruction. 
We consider it is because BNV-Fusion use very small voxel size ($0.02\si{\meter}$) to get a high score.
This is also revealed by their \textbf{\SI{247}{MB}} mesh in average, while ours is only \textbf{\SI{54}{Mb}} in average.
Furthermore, our Uni-Fusion's mesh is more smooth and
%Both BNV-Fusion and Uni-Fusion demonstrate high quality result.
also provides high-precise color to the mesh which is not available for the Surface SOTA.

%In this test, we purely evaluate the surface reconstruction capacity with SOTAs. 
%While reconstruction is not merely surface.  
%Thus in the following, we find benchmarks for both surface and color.


\subsubsection{Tracking Evaluation on TUM RGB-D Dataset~\cite{sturm2012benchmark}}
% follow nice-slam

In the above test, we compare the performance of pure mapping.
While tracking is not the contribution focus in our paper, it is part of the reconstruction model. We follow the novel reconstruction model NICE-SLAM~\cite{zhu2022nice} to evaluate the camera tracking on the small-scale TUM RGB-D dataset.
Our Uni-Fusion uses a coarse-to-fine strategy for 3D reconstruction tracking.
From~\cref{tab:tum_rmse}, it demonstrates overall better ATE RMSE than other implicit representation models.

\begin{table}[]
		\caption{Tracking on TUM RGB-D~\cite{sturm2012benchmark}.
		ATE RMSE [$\si{\centi\meter}$] ($\downarrow$) is used as the evaluation metric.
	}
	\centering
	\footnotesize
	\setlength{\tabcolsep}{0.7em}
	\resizebox{\linewidth}{!}{
		\begin{tabular}{l|ccc}
			\hline
			& \tt{fr1/desk} &  \tt{fr2/xyz} &  \tt{fr3/office} \\
			
			\hline
			{iMAP}~\cite{sucar2021imap}      & 4.9 & 2.0 & 5.8  \\
			{iMAP$^*$}~\cite{sucar2021imap} & 7.2 & 2.1  & 9.0 \\
			{DI-Fusion~\cite{huang2021di}} & 4.4 & 2.3 & 15.6 \\
			NICE-SLAM~\cite{zhu2022nice}           & 2.7 & 1.8 & 3.0 \\
			Ours& 1.8& 0.5& 2.1 \\
			\hline
			{BAD-SLAM}\cite{schops2019bad} & 1.7  & 1.1  & 1.7 \\
			{Kintinuous}\cite{whelan2012kintinuous} & 3.7  &  2.9  & 3.0 \\
			{ORB-SLAM2}\cite{mur2017orb} & \bf 1.6  & \bf 0.4  & \bf 1.0 \\
			\hline
	\end{tabular}}
	\vspace{2pt}

	\label{tab:tum_rmse}
\end{table}

On the other hand, there also exist high accuracy algorithms from SLAM. 
By additional using Bundle Adjustment and Loop-closing techniques, their tracking quality is much better than all of the implicit based models.

%But it is dangerous to directly apply SLAM result on reconstruction. Please find our demonstration in Fig [?]. Which explains the more widely used frame-to-model strategy in 3D reconstruction.
Even though, our coarse-to-fine strategy firstly ensure not easy to lose track. Secondly, it is more suitable for surface fitting.

Which further support our test in Replica dataset.



\begin{table*}[t!]
	\centering
	\caption{Geometric (L1) and Photometric (PSNR) evaluation on the Replica dataset~\cite{sucar2021imap}.}
	\footnotesize
	\setlength{\tabcolsep}{0.36em}
	\renewcommand{\arraystretch}{1.2}
	\begin{tabular}{clcccccccccccccccccc}
		\toprule
		& & \multicolumn{1}{c}{\makecell{\tt{office-0}}} & \multicolumn{1}{c}{\makecell{\tt{office-1}}} & \multicolumn{1}{c}{\makecell{\tt{office-2}}}& \multicolumn{1}{c}{\makecell{\tt{office-3}}} & \multicolumn{1}{c}{\makecell{\tt{office-4}}} & \multicolumn{1}{c}{\makecell{\tt{room-0}}} & \multicolumn{1}{c}{\makecell{\tt{room-1}}} &  \multicolumn{1}{c}{\makecell{\tt{room-2}}} & Avg. \\
		\midrule
		\multicolumn{5}{l}{\textit{Non-continuous mapping method}}\\
		\multirow{2}{*}{\makecell{\textbf{TSDF-Fusion}~\cite{curless1996volumetric}}}
		& {\bf Depth L1} [$\si{\centi\meter}$] $\downarrow$
	 & 14.11 & 10.50 & 30.89 & 28.92 & 22.83	& 23.51 & 20.94 & 23.34 & 21.88 \\
		& {\bf PSNR } [$\si{\dB}$] $\uparrow$
		& 11.16 & 15.92 & 4.86 & 5.68 & 5.46 & 3.43 & 4.51 & 5.57 & 7.07 \\
		
		\midrule
		\multirow{2}{*}{\makecell{\textbf{$\sigma$-Fusion}\cite{rosinol2023probabilistic} }}
		& {\bf Depth L1} [$\si{\centi\meter}$] $\downarrow$
		 & 13.80 & 10.21 & 22.27 & 28.70 & 22.21& 21.92 & 19.28 & 22.40 & 20.10 \\
		& {\bf PSNR } [$\si{\dB}$] $\uparrow$
		 & 11.16 & 15.92 & 4.86 & 5.69 & 5.46& 3.45  & 4.51 & 5.57 & 7.08 \\
		
		
		
		
		
		\midrule
		\midrule
		\multicolumn{5}{l}{\textit{Continuous mapping method}}\\
		\multirow{2}{*}{\makecell{\textbf{iMAP$^*$}~\cite{sucar2021imap}}}
		& {\bf Depth L1} [$\si{\centi\meter}$] $\downarrow$
		 & 6.43 & 7.41 & 14.23 & 8.68 & 6.80& 5.70 & 4.93 & 6.94 & 7.64\\
		& {\bf PSNR } [$\si{\dB}$] $\uparrow$
		& 7.39 & 11.89 & 8.12 & 5.62 & 5.98& 5.66 & 5.31 & 5.64  & 6.95\\
		\midrule
		\multirow{2}{*}{{\makecell{\textbf{Nice-SLAM}~\cite{zhu2022nice} }}}
		& {\bf Depth L1} [$\si{\centi\meter}$] $\downarrow$
		& { 1.51 } & { 0.93 } & { 8.41 } & { 10.48 } & {2.43} & { 2.53 } & { 3.45 } & { 2.93 }  & { 4.08 } \\
		& {\bf PSNR } [$\si{\dB}$] $\uparrow$
		 & { 22.44 } & { 25.22 } & { 22.79 } & { 22.94 } & { 24.72 } & \textbf{ 29.90 } & \textbf{ 29.12 } & { 19.80 }& { 24.61 } \\
		
		
		
		
		\midrule	
		\multirow{2}{*}{{\makecell{\textbf{Uni-Fusion} (Ours) }}}
		% using abs(diff)
		%	& {\bf Depth L1} [$\si{\centi\meter}$] $\downarrow$ &\textbf{1.98}&\textbf{1.18}&\textbf{1.64}&\textbf{1.23}&\textbf{0.84}&\textbf{1.61}&\textbf{3.01}&\textbf{1.60} &\textbf{1.64}
		% follow nerf-slam to remove outlier gt first
		& {\bf Depth L1} [$\si{\centi\meter}$] $\downarrow$
		& \textbf{0.79}&\textbf{0.56}&\textbf{1.59}&\textbf{2.71}&\textbf{1.66}&\textbf{1.94}&\textbf{0.69}&\textbf{1.80}& \textbf{1.47}
		\\
		& {\bf PSNR } [$\si{\dB}$] $\uparrow$ &\textbf{33.88}&\textbf{33.31}&\textbf{25.84}&\textbf{26.01}&\textbf{28.14}&24.02&26.20&\textbf{27.17} &\textbf{28.07}
		\\
		
		\midrule
		\midrule
		\multicolumn{5}{l}{\textit{Neural radiance field method}}\\
		\multirow{2}{*}{{\makecell{\textbf{NeRF-SLAM}~\cite{rosinol2022nerf} }}}
		& {\bf Depth L1} [$\si{\centi\meter}$] $\downarrow$
	 & {2.49}   & {1.98}  & {9.13}  & {10.58} & {3.59}	& {2.97}  & {2.63}  & {2.58}  & {4.49} \\
		& {\bf PSNR } [$\si{\dB}$] $\uparrow$
	 & \textbf{48.07}  & \textbf{53.44} & \textbf{39.30} & \textbf{38.63} & \textbf{39.21} 	& \textbf{34.90} & \textbf{36.95} & \textbf{40.75}& \textbf{41.40} \\
		
		\bottomrule
	\end{tabular}%
	
	\label{tab:replica_per_scene}
\end{table*}


\begin{table*}[t!]
	\centering
	\caption{Differences among different Surface \& Color reconstruction models.}
	\small
	\setlength{\tabcolsep}{.6em}
	%{
		%\resizebox{\textwidth}{!}{
			\begin{tabular}{l | c c c c c c }
				\toprule
				Method & 
				\begin{tabular}{@{}c@{}}Pre-Train\\ with extra dataset\end{tabular}
				& \begin{tabular}{@{}c@{}}Train\\ with sequences\end{tabular}
				& Real-time	
				& Direct Output &  \begin{tabular}{@{}c@{}}Light\\ direction\end{tabular} 
				&Render\\
				\hline 
				TSDF-Fusion & None & None & $\checkmark$& Discrete TSDF &  &Ray Rasterization\\\hline
				$\sigma$-Fusion & None & None &$\checkmark$&Discrete TSDF  && Ray Rasterization\\\hline
				iMAP & None & Online Train && MLPs  & &Volumetric Rendering\\\hline
				NICE-SLAM & \begin{tabular}{@{}c@{}}Pretrain\\ with indoor dataset\end{tabular} & Online Train&& Neural Implicit Grid&  & Volumetric Rendering\\\hline
				
				NeRF-SLAM & None & Train hundred epochs &-&NeRF & $\checkmark$ &Volumetric Rendering \\\hline
				
				\textbf{Uni-Fusion} & None & None&$\checkmark$& Latent Implicit Map && Ray Rasterization\\				
				\hline
			\end{tabular}
		%}
	%}
	\label{tab:replica_diff}
\end{table*}
\subsubsection{Evaluation on Replica RGB-D Dataset~\cite{sucar2021imap}}
In this evaluation, we compare with implicit reconstruction (TSDF-Fusion, $\sigma$-Fusion) and latent implicit reconstruction models (iMAP, NICE-SLAM) that support colors. 
We also add a large-scale NeRF model, NeRF-SLAM in to the table.  
NeRF is SOTA in view-synthesis task, which is unfair to direct compare with the rest. As the rest model does not even model light directions.
We add NeRF in this part to demonstrate that Uni-Fusion strongly reduce the gap.
Note that, NeRF-SLAM embeds external tracking model ~\cite{teed2021droid} to provide poses while using SOTA NeRF implementation Instance-ngp~\cite{muller2022instant} for NeRF construction.
%Therefore it is considered the SOTA to model the colors.

Uni-Fusion track and follow our previous setting in ScanNet test to take every 10 frames for mapping.
NICE-SLAM and NeRF-SLAM produce depth and color by rendering,
To obtain result from Uni-Fusion, we cast rays from virtual camera to our result surface mesh for depth image. 
Then Uni-Fusion infer the cast points in Uni-Fusion's color LIM for color result.

From~\cref{tab:replica_per_scene}, Uni-Fusion demonstrate
best Depth L1 on all scenes with an average of \textbf{$\pmb{1.47}$$\si{\centi\meter}$ depth L1}. Which is \textbf{$\pmb{177\%}$ boost} comparing to the second best.

Moreover, excluding NeRF, our Uni-Fusion also shows the best performance to model the colors with an average of $28.07$$\si{\dB}$ PSNR.

However, it is strange that NICE-SLAM lost details while in two cases, it shows better PSNR than Uni-Fusion. 
To highlight the true result,
we plot the rendered image in \cref{fig:replica_render}.
It is obvious that our Uni-Fusion models the details of painting, carpet and quilt well. 
While NICE-SLAM just roughly models the average color.

Moreover, from the  \cref{fig:replica_render}, our Uni-Fusion's rendering quality is as precise as NeRF. 
Please also find the painting, carpet and quilt, Uni-Fusion recovered the original appearance well.
Please find the {\color{green} green window} for the emphasized region.
Uni-Fusion reproduce the high-quality appearance which is very close to NeRF on qualitative evaluation.
%It can hardly find difference between the results from NeRF-SLAM, Uni-Fusion and Ground Truth.
%
But, Uni-Fusion still has a quantitative score gap to the color rending of NeRF ($41.4$$\si{\dB}$).
Though the Uni-Fusion's rendering result is highly close to NeRF and ground truth.
%
We consider the main reasons are that \textbf{1.} Uni-Fusion does not model the light directions to points, which is essential to NeRF.
\textbf{2.} NeRF optimizes on the rendering image quality by focussing mainly on color while less on depth.
It can be revealed by the higher color rendering score with much worse depth rendering than our Uni-Fusion.
\textbf{3.} our Uni-Fusion does not support hole filling.
This directly leads to black holes in our rendered images.

We summarize the differences in \cref{tab:replica_diff}.
Similar to TSDF-Fusion and $\sigma$-Fusion, our Uni-Fusion is a forward method which, does not need any training, i.e., pre- or online training. 
Uni-Fusion also produces similar to NICE-SLAM and NeRF-SLAM an implicit map with set of latent that outputs results at arbitrary resolution.
However, we differ on the extracting of the signed distance field.
%FIXME: I do not understand the next sentence.
Uni-Fusion's latent feature rule its own region independently.
Each query value is directly inferred with the corresponding ruling latent.
While NICE-SLAM and NeRF-SLAM use a much denser grid to interpolate query features. This requires volumetric rendering for inference.

Similar to TSDF-Fusion, $\sigma$-Fusion, our Uni-Fusion is also a real-time algorithm.
iMAP, NICE-SLAM and NeRF-SLAM run hardly in real-time.
NeRF-SLAM is claiming to be real-time, which is questionable as they still need hundreds of epochs training after feeding the data.

Nevertheless, optimization with backpropagation learns pixel-to-pixel well.
It is theoretically advanced for a regression-and-fusion strategy. 
Though Uni-Fusion demonstrates its high capability to model the color, NeRF-like post-optimization would still be a good direction for further improvements of Uni-Fusion.

\newcommand{\replicaImSize}{.24}
\begin{figure*}[t]
	\centering
	\setlength{\tabcolsep}{0.1em}
	\renewcommand{\arraystretch}{.1}
	\begin{tabular}{|c | c |c |c| }
		 \hline
		{\Large{NICE-SLAM}} &{\Large{NeRF-SLAM}}&\textbf{\Large{Uni-Fusion}}&\Large{Ground Truth}\\
		%		\hline
		%		\includegraphics[width=\replicaImSize\linewidth]{im/exp/recons/replica/nice-slam/of2_1286.png} &
		%		\includegraphics[width=\replicaImSize\linewidth]{im/exp/recons/replica/mine/of2_1286.jpg} &
		%		\includegraphics[width=\replicaImSize\linewidth]{im/exp/recons/replica/mine/of2_1286.jpg} &
		%		\includegraphics[width=\replicaImSize\linewidth]{im/exp/recons/replica/gt/of2_1286.jpg} \\
		
		\hline
		\includegraphics[width=\replicaImSize\linewidth]{im/exp/recons/replica/nice-slam/rm0_769_window.png} &
		\includegraphics[width=\replicaImSize\linewidth]{im/exp/recons/replica/nerf-slam/rm0_769_window.jpg} &
		\includegraphics[width=\replicaImSize\linewidth]{im/exp/recons/replica/mine/rm0_769_window.jpg} &
		\includegraphics[width=\replicaImSize\linewidth]{im/exp/recons/replica/gt/rm0_769_window.jpg} \\
		\hline
		\includegraphics[width=\replicaImSize\linewidth]{im/exp/recons/replica/nice-slam/of3_575_window.png} &
		\includegraphics[width=\replicaImSize\linewidth]{im/exp/recons/replica/nerf-slam/of3_575_window.jpg} &
		\includegraphics[width=\replicaImSize\linewidth]{im/exp/recons/replica/mine/of3_575_window.jpg} &
		\includegraphics[width=\replicaImSize\linewidth]{im/exp/recons/replica/gt/of3_575_window.jpg} \\
		\hline
		\includegraphics[width=\replicaImSize\linewidth]{im/exp/recons/replica/nice-slam/rm1_425_window.png} &
		\includegraphics[width=\replicaImSize\linewidth]{im/exp/recons/replica/nerf-slam/rm1_425_window.jpg} &
		\includegraphics[width=\replicaImSize\linewidth]{im/exp/recons/replica/mine/rm1_425_window.jpg} &
		\includegraphics[width=\replicaImSize\linewidth]{im/exp/recons/replica/gt/rm1_425_window.jpg} \\
		\hline
		\includegraphics[width=\replicaImSize\linewidth]{im/exp/recons/replica/nice-slam/rm2_1085_window.png} &
		\includegraphics[width=\replicaImSize\linewidth]{im/exp/recons/replica/nerf-slam/rm2_1085_window.jpg} &
		\includegraphics[width=\replicaImSize\linewidth]{im/exp/recons/replica/mine/rm2_1085_window.jpg} &
		\includegraphics[width=\replicaImSize\linewidth]{im/exp/recons/replica/gt/rm2_1085_window.jpg} \\		
		
	\end{tabular}
	%\captionof{figure}
	\caption{Demonstration of color rendering on the Replica dataset. Fine appearances are highlighted in {\color{green}green window}. Small flaws are in a {\color{red}red} box.}
	\label{fig:replica_render}
	\vspace{-.5cm}
\end{figure*}

%(2) NeRF model learning radiance field that model the light on different direction on surface. While Uni-Fusion naturally treat different directional light the same color.

%\begin{table*}[t!]
%	\centering
%	\setlength{\tabcolsep}{0.1em}
%	\renewcommand{\arraystretch}{.1}
%	\begin{tabular}{c | c |c |c |c }
%		\hline 
%		\rotatebox{90}{\large{NICE-SLAM}} &
%		\includegraphics[width=\replicaImSize\linewidth]{im/exp/recons/replica/nice-slam/of3_575.png} &
%		\includegraphics[width=\replicaImSize\linewidth]{im/exp/recons/replica/nice-slam/rm0_769.png} &
%		\includegraphics[width=\replicaImSize\linewidth]{im/exp/recons/replica/nice-slam/rm1_425.png} &
%		\includegraphics[width=\replicaImSize\linewidth]{im/exp/recons/replica/nice-slam/rm2_1085.png} \\
%		\hline
%		\rotatebox{90}{\large{NeRF-SLAM}} &
%		\includegraphics[width=\replicaImSize\linewidth]{im/exp/recons/replica/mine/of3_575.jpg} &
%		\includegraphics[width=\replicaImSize\linewidth]{im/exp/recons/replica/mine/rm0_769.jpg} &
%		\includegraphics[width=\replicaImSize\linewidth]{im/exp/recons/replica/mine/rm1_425.jpg} &
%		\includegraphics[width=\replicaImSize\linewidth]{im/exp/recons/replica/mine/rm2_1085.jpg} \\	
%		\hline
%		\rotatebox{90}{\textbf{\Large{Uni-Fusion}}} &
%		\includegraphics[width=\replicaImSize\linewidth]{im/exp/recons/replica/mine/of3_575.jpg} &
%		\includegraphics[width=\replicaImSize\linewidth]{im/exp/recons/replica/mine/rm0_769.jpg} &
%		\includegraphics[width=\replicaImSize\linewidth]{im/exp/recons/replica/mine/rm1_425.jpg} &
%		\includegraphics[width=\replicaImSize\linewidth]{im/exp/recons/replica/mine/rm2_1085.jpg} \\
%		\hline
%		\rotatebox{90}{\large{Ground Truth}} &
%		\includegraphics[width=\replicaImSize\linewidth]{im/exp/recons/replica/gt/of3_575.jpg} &
%		\includegraphics[width=\replicaImSize\linewidth]{im/exp/recons/replica/gt/rm0_769.jpg} &
%		\includegraphics[width=\replicaImSize\linewidth]{im/exp/recons/replica/gt/rm1_425.jpg} &
%		\includegraphics[width=\replicaImSize\linewidth]{im/exp/recons/replica/gt/rm2_1085.jpg} \\
%		\hline		
%		
%	\end{tabular}
%	\captionof{figure}{Demonstration of color rendering on Replica dataset.}
%\end{table*}

\subsection{Ablation study}
\label{exp:surface:ablation}

\begin{figure}[]
	\centering
%		\subfloat[width=\textwidth][Sample based]{
%		\centering
%		\includegraphics[width=.7\linewidth]{im/exp/ablation/GPIS/seq3_sample_color.png}
%	}\\
%	\subfloat[width=\textwidth][Derivative based]{
%		\centering
%		\includegraphics[width=.7\linewidth]{im/exp/ablation/GPIS/seq3_derivative_color.png}
%	}
		\includegraphics[width=.7\linewidth]{im/exp/ablation/GPIS/seq3_sample_color_a.png}
		\includegraphics[width=.7\linewidth]{im/exp/ablation/GPIS/seq3_derivative_color_b.png}
	\caption{Ablation study on surface construction basis. (a) Sample based. (b) Derivative based.}
	\label{fig:ablation:GPIS}
\end{figure}


\begin{table}[]
	\caption{Ablation study on tracking.
	}
	\centering
	\footnotesize
	\setlength{\tabcolsep}{0.7em}
	\resizebox{\linewidth}{!}{
		\begin{tabular}{l|ccc}
			\hline
			& \tt{fr1/desk} &  \tt{fr2/xyz} &  \tt{fr3/office} \\
			\hline
			External& 2.1& 0.5& 2.5 \\
			External+Internal&1.8& 0.5& 2.1 \\
			\hline
	\end{tabular}}
	\vspace{-2pt}
	%\vspace{-1cm}
	\label{tab:tum_rmse2}
\end{table}

\begin{figure}
	\centering
	\includegraphics[width=.7\linewidth]{im/exp/ablation/voxel_size/seq_voxel_size.png}
	%	\subfloat[width=.33\textwidth][0.1]{
		%		\centering
		%		\includegraphics[width=.33\linewidth]{im/exp/ablation/voxel_size/seq3_0_1_color.png}
		%	}
	%	\subfloat[width=.3\textwidth][0.05]{
		%		\centering
		%		\includegraphics[width=.33\linewidth]{im/exp/ablation/GPIS/seq3_sample_color.png}
		%	}
	%	\subfloat[width=.33\textwidth][0.02]{
		%	\centering
		%	\includegraphics[width=.33\linewidth]{im/exp/ablation/voxel_size/seq3_0_02_color.png}
		%	}
	\caption{Ablation study on voxel size.}
	\label{fig:ablation:voxel_size}
\end{figure}




 \newcommand{\styleImSize}{.2}
\begin{figure*}[b!]
	\vspace{-.5cm}
	\centering
	\setlength{\tabcolsep}{0.1em}
	\renewcommand{\arraystretch}{.1}
	\resizebox{\textwidth}{!}{\begin{tabular}{ccccc}
			%		\includegraphics[width=\styleImSize\line]{im/exp/style/style/0} &
			%		\includegraphics[width=\styleImSize\linewidth]{im/exp/style/style/1} &
			%		\includegraphics[width=\styleImSize\linewidth]{im/exp/style/style/2} &
			%		\includegraphics[width=\styleImSize\linewidth]{im/exp/style/style/3} &
			%		\includegraphics[width=\styleImSize\linewidth]{im/exp/style/style/4} &
			%		\includegraphics[width=\styleImSize\linewidth]{im/exp/style/style/5} &
			%		\includegraphics[width=\styleImSize\linewidth]{im/exp/style/style/6} \\
			\hline\hline
			\includegraphics[width=\styleImSize\linewidth]{im/exp/style/processed/office0_0.png} &
			\includegraphics[width=\styleImSize\linewidth]{im/exp/style/processed/office0_1.png} &
			\includegraphics[width=\styleImSize\linewidth]{im/exp/style/processed/office0_2.png} &
			\includegraphics[width=\styleImSize\linewidth]{im/exp/style/processed/office0_3.png} &
			\includegraphics[width=\styleImSize\linewidth]{im/exp/style/processed/office0_4.png} \\
			\includegraphics[width=\styleImSize\linewidth]{im/exp/style/processed/office0_5.png} &
			\includegraphics[width=\styleImSize\linewidth]{im/exp/style/processed/office0_6.png} &
			\includegraphics[width=\styleImSize\linewidth]{im/exp/style/processed/office0_7.png} &
			\includegraphics[width=\styleImSize\linewidth]{im/exp/style/processed/office0_8.png} &
			\includegraphics[width=\styleImSize\linewidth]{im/exp/style/processed/office0_9.png} \\
			\includegraphics[width=\styleImSize\linewidth]{im/exp/style/processed/office0_10.png} &
			\includegraphics[width=\styleImSize\linewidth]{im/exp/style/processed/office0_11.png} &
			\includegraphics[width=\styleImSize\linewidth]{im/exp/style/processed/office0_12.png} &
			\includegraphics[width=\styleImSize\linewidth]{im/exp/style/processed/office0_13.png} &
			\includegraphics[width=\styleImSize\linewidth]{im/exp/style/processed/office0_14.png} \\
			\includegraphics[width=\styleImSize\linewidth]{im/exp/style/processed/office0_15.png} &
			%\includegraphics[width=\styleImSize\linewidth]{im/exp/style/processed/office0_16.png} &
			\includegraphics[width=\styleImSize\linewidth]{im/exp/style/processed/office0_17.png} &
			\includegraphics[width=\styleImSize\linewidth]{im/exp/style/processed/office0_18.png} &
			\includegraphics[width=\styleImSize\linewidth]{im/exp/style/processed/office0_19.png} &
			\includegraphics[width=\styleImSize\linewidth]{im/exp/style/processed/office0_20.png} 
			\\	\hline
		\end{tabular}
	}
	%\captionof{figure}
	\caption{Style transfer on 3D canvas.}
	\label{fig:style}
\end{figure*}


\subsubsection{ Sample-based or Derivative-based}

We select the surface model with our own captured sequences. 
All settings are detailed in \cref{sec:exp:details}.
As shown in~\cref{fig:ablation:GPIS}, reconstruction of Yijun's office is demonstrated. 
Both models are able to construct, but the derivative-based model produces a lot of noise on the surface.
This is because for smoothness purpose, we build voxels that are overlapped to its neighbor, which causes redundant voxels near the surface.
For those redundant voxels, no center sample is provided and thus the derivative based surface construction builds bad SDFs on unknow region of the voxels.


Instead, sample-based surface construction does not have this problem as it adds more points in voxels and is able to construct highly-smooth surfaces.
From which, we find well constructed and colored white board, chair, school bag and even the oranges.

\subsubsection{Tracking}


Our Uni-Fusion use a coarse-to-fine strategy for tracking. 
An external tracking model is running in one thread aside from the mapping thread.
In the mapping thread, it takes pose result from the external tracking and applies the internal tracking for colored point cloud.

The result is demonstrated in~\cref{tab:tum_rmse2}. 
The coarse-to-fine is relatively better on trajectory estimation.

\subsubsection{Voxel size}

Testing the office scene, we vary the voxel size from low to high. 
From~\cref{fig:ablation:voxel_size}, when low voxel size $0.02$m is used, the surface is rough.
Then when voxel size goes larger, the smoothness is improved.
However, when we use $0.1$m voxel size, the surface color is blur. 
Considering Uni-Fusion produces a surface color field, the quality of surface directly affect the coloring.
Thus, continuing enlarging the voxel size also results in worse surface results.

Therefore, in the above experiments, $0.05$m voxel size is utilized for surface construction.
In addition, each voxel for encoding are actually with size $0.1$m, since we use overlapped voxel.

%\subsubsection{Anchor number and feature dimension}


\subsection{Application: 2D-to-3D Transfer}
\label{sec:fabircated_prop}

Applications such as 2) and 4) can be easily integrated with application 1) incremental reconstruction (\cref{sec:incremental_reconstruction}) by incorporating the fabricated result together with the point cloud.
%
For instance, given RGB-D frames, we detect saliency or transfer image styles to generate a fabricated $X$ image. Here, $X$ represents saliency, style, or other properties. 
By combining $X$ with depth information through unprojection,
we assign
the fabricated values to corresponding points, resulting in point pairs ($\V X$, $\V Q_{X}$).

Similar to the reconstruction pipeline in~\cref{fig:recons_and_scene_understanding}, we employ encoding (\cref{sec:encoder}) and fusion (\cref{eq:fuse}) to construct a global LIM for the fabricated properties $X$.
This global LIM represents a surface $X$ fields that is utilized for subsequent inference.

While it is possible to similarly transfer a 2D semantic image to 3D,
it may not be feasible in practice due to the need for multiple passes of different categories of semantic information 
 on the same dataset (such as object, usability, etc.).
Therefore, in the following section, we demonstrate the construction of a surface feature field for scene understanding application that satisfies various 
requirements through a single mapping pass.

\begin{table*}[b!]
	%\renewcommand{\arraystretch}{0.9}
	%\setlength{\tabcolsep}{3pt}
	\caption{GZSL semantic segmentation results. Scores are in \%.
	  $^\dagger$ indicate 3DGenZ's adaption of the method.
       Note that, Uni-Fusion-SU does not even train with the seen classes.}
	\centering
	\begin{tabular}{l|c|c |c ||ccc|ccc}
		\toprule
		\multicolumn{1}{c}{}& \multicolumn{2}{c|}{Training set} & Inference input &\multicolumn{3}{c|}{ScanNet } & \multicolumn{3}{c}{S3DIS}\\
		& Backbone & Classifier & &$Seen$& $Unseen$ & $All$&$Seen$& $Unseen$ & $All$
%		\multicolumn{3}{c|}{mIoU} & 
%		\multicolumn{3}{c|}{mIoU} \\ 
%		&&&& $Seen$& $Unseen$ & $All$&$Seen$& $Unseen$ & $All$
		%\cellcolor{white}{}  
		%\cellcolor{white}{}
		\\
		\midrule
		
		\multicolumn{5}{l}{\textit{Supervised methods with different levels of supervision}}\\
		
		Full supervision & $seen \cup unseen$ & $seen \cup unseen$ & Point Cloud &43.3&51.9 &45.1&74.0&50.0&66.6 \\
		
		ZSL backbone & $seen$ & $seen \cup unseen$  &Point Cloud&41.5&39.2 & 40.3&60.9& 21.5&  48.7 \\
		
		ZSL-trivial & $seen$ & $seen$ &Point Cloud&39.2&0.0&31.3&70.2 &0.0&48.6  \\
		\midrule
		\multicolumn{5}{l}{\textit{Generalized zero-shot-learning methods}}\\
		
		ZSLPC-Seg~\cite{cheraghian2019zero}$^\dagger$ & $seen$ & $unseen$  &Point Cloud&28.2&0.0& 22.6&65.6 &0.0& 45.3\\
		
		DeViSe-3DSeg~\cite{frome2013devise}$^\dagger$ & $seen$ & $unseen$   &Point Cloud &20.0&0.0&16.0&70.2&0.0& 48.6\\ 
		%ZSLPC-Seg~\cite{cheraghian2019zero}$^\dagger$ & $seen$ & $unseen$  &  4.0&13.9\\
		%DeViSe-3DSeg~\cite{frome2013devise}$^\dagger$ & $seen$ & $unseen$   &  3.0&10.9\\
		3DGenZ~\cite{michele2021generative} & $seen$ & $seen \cup \hat{unseen}$  &Point Cloud &32.8&7.7& {27.8}&53.1&7.3&   \textbf{39.0} \\
		\midrule
		\multicolumn{5}{l}{\textit{Zero-shot learning + map fusion}}\\
		Uni-Fusion-SU (Ours) &None&None&Sparse Frames&31.0&\textbf{41.9}&\textbf{32.9} &31.3&\textbf{24.0}&29.0\\
		\bottomrule
		\multicolumn{1}{l}{}\\[-7pt]
	\end{tabular}

	\label{tab:sem_seg_overview}
\end{table*}

\begin{figure*}[t!]
	\centering
	\setlength{\tabcolsep}{0.1em}
	\renewcommand{\arraystretch}{.1}
	\begin{tabular}{|c | c |c |||c |c | c|}
		\toprule
		{\Large{3DGenZ}} & {\Large{Uni-Fusion}} &{\Large{Ground Truth}} & {\Large{3DGenZ}} &{\Large{Uni-Fusion-SU}} & {\Large{Ground Truth}} \\ \midrule
		
		\includegraphics[width=\scannetImSize\linewidth]{im/exp//ss/gen3dz_0568.png}
		&\includegraphics[width=\scannetImSize\linewidth]{im/exp//ss/mine_0568.png}
		&\includegraphics[width=\scannetImSize\linewidth]{im/exp//ss/gt_0568.png}
		&		\includegraphics[width=\scannetImSize\linewidth]{im/exp//ss/gen3dz_0164.png}
		&\includegraphics[width=\scannetImSize\linewidth]{im/exp//ss/mine_0164.png}
		&\includegraphics[width=\scannetImSize\linewidth]{im/exp//ss/gt_0164.png}\\
		
		
		\includegraphics[width=\scannetImSize\linewidth]{im/exp//ss/gen3dz_0249.png}
		&\includegraphics[width=\scannetImSize\linewidth]{im/exp//ss/mine_0249.png}
		&\includegraphics[width=\scannetImSize\linewidth]{im/exp//ss/gt_0249.png}
		&		\includegraphics[width=\scannetImSize\linewidth]{im/exp//ss/gen3dz_0435.png}
		&\includegraphics[width=\scannetImSize\linewidth]{im/exp//ss/mine_0435.png}
		&\includegraphics[width=\scannetImSize\linewidth]{im/exp//ss/gt_0435.png}\\
		
		
		\includegraphics[width=\scannetImSize\linewidth]{im/exp//ss/gen3dz_0046.png}
		&\includegraphics[width=\scannetImSize\linewidth]{im/exp//ss/mine_0046.png}
		&\includegraphics[width=\scannetImSize\linewidth]{im/exp//ss/gt_0046.png}
		&		\includegraphics[width=\scannetImSize\linewidth]{im/exp//ss/gen3dz_0050.png}
		&\includegraphics[width=\scannetImSize\linewidth]{im/exp//ss/mine_0050.png}
		&\includegraphics[width=\scannetImSize\linewidth]{im/exp//ss/gt_0050.png}\\
		\bottomrule
		
	\end{tabular}
	\includegraphics[width=\linewidth]{im/ss_colorbar}
	%\captionof{figure}
	\caption{Demonstration of semantic segmentation on the ScanNet dataset.
       Selected scenes are consistent with~\cref{fig:recons:scannet_demo}}
	\label{fig:segmentation_demo}
	
\end{figure*}

\subsection{Scene Understanding Results}

Saliency detection effectively highlights the objects of interest.
This is also considered part of 3D semantic understanding.
However, as the semantics categories vary, fusing different categories of semantics into multiple LIMs can be inefficient.
%
Therefore, in this section, we utilize Uni-Fusion to fuse and construct a surface field for high-dimensional CLIP embeddings.
With a single LIM, we can generate different semantic results based on corresponding commands.
%
Since now our Uni-Fusion works with OpenSeg for scene understanding purposes, we call it Uni-Fusion-SU.

\subsubsection{Semantic Segmentation}
\label{sec:exp:semantic}

We first evaluate our model on generalized zero-shot point cloud semantic segmentation application.
Generalized Zero-Shot Learning (GZSL) differs from Zero-Shot Learning (ZSL) in that ZSL only predicts classes unseen during training, while GZSL predicts both unseen and seen classes~\cite{michele2021generative}.
Therefore, comparing our results with GZSL SOTAs provides a better understanding of the potential of Uni-Fusion-SU, as it does not train on both seen and unseen. 

This test uses ScanNet and S3DIS datasets for benchmarking. 
It is important to note that the \textbf{compared baselines are trained on the corresponding datasets}.
Our Uni-Fusion-SU uses OpenSeg to provide the 2D image level feature ebmedding.
Although \textbf{Uni-Fusion-SU} is also zero-shot, \textbf{it does not touch any ScanNet or S3DIS annotations}.

We demonstrate the mIoU scores in~\cref{tab:sem_seg_overview}.
In particular, our model achieves best results among the zero-shot learning methods on the ScanNet dataset and remains competitive with fully supervised methods.

Furthermore, we provide results specifically for the unseen classes in~\cref{sup:tab:sn_acc_miou}.
Although not as good as the fully supervised approach, Uni-Fusion-SU performs much better than 3DGenZ.
In addition, our Uni-Fusion-SU demonstrates high precision in classes such as sofa and Toilet, even when compared to the fully supervised model.

\begin{table}[htbp]
		\caption{Classwise GZSL semantic segmentation performance (\%) on the ScanNet unseen split.}
	\centering
	\newcommand*\rotext{\multicolumn{1}{R{45}{1em}}}
	\setlength{\tabcolsep}{1.7pt}
	\begin{tabular}{@{}l@{~}c|rrrr|r@{}}
		\toprule		
		& &
		{\textbf{Bookshelf}} & {\textbf{Desk}} & {\textbf{Sofa}} & {\textbf{Toilet}} & \stackbox{mean} \\
		
		\midrule
		FSL (Fully supervise) & IoU & 	56.9&	30.0&	57.4&	63.4 & 51.9
		\\ 
		3DGenZ (Zero-shot) & IoU & 	6.3&	3.3&	13.1&	8.1 & 7.7
		\\
		Uni-Fusion-SU (Ours) & IoU &38.3&16.8&51.7&60.9&41.9
	\\ \midrule 
	3DGenZ (Zero-shot)& Acc. & 	13.4&	5.9&	49.6&	26.3 &23.8
	\\
	Uni-Fusion-SU (Ours) & Acc. &61.9&29.6&67.4&91.6& 62.6
		\\
		\bottomrule
	\end{tabular}

	\label{sup:tab:sn_acc_miou}
\end{table}

However, in the S3DIS dataset, our model does not outperform 3DGenZ and other methods as shown in~\cref{tab:sem_seg_overview}.

Even in the result of unsceened data, as presented in \cref{sup:tab:s3dis_acc_miou}, we observe that Uni-Fusion-SU hardly finds some classed, e.g. Beam and Column, which are not commonly annotated objects. 
However, for common objects like sofa and window, our model performs much better.

\begin{table}[htbp]
		\caption{Classwise GZSL semantic segmentation performance (\%) on the S3DIS unseen split.}
	\centering
	\newcommand*\rotext{\multicolumn{1}{R{45}{1em}}}
	\setlength{\tabcolsep}{1.7pt}
	\begin{tabular}{@{}l@{~}c|rrrr|r@{}}
		\toprule		
		& &
		{\textbf{Beam}} & {\textbf{Column}} & {\textbf{Sofa}} & {\textbf{Window}} & \stackbox{mean} \\
		
		\midrule
		FSL (Fully supervise) & IoU & 	63.1&	10.2&	54.1&	72.4 & 50.0
		\\ 
		3DGenZ (Zero-shot) & IoU & 	13.9&	2.4&4.9&	8.1 &7.3
		\\
		Uni-Fusion-SU (Ours) & IoU &5.5&0.02&57.4&32.9&	24.0
		\\ \midrule 
		3DGenZ (Zero-shot) & Acc. & 	20.0&	9.1&	62.4&	23.7 &28.8
		\\
		Uni-Fusion-SU (Ours) & Acc. &41.5&0.02&78.3&42.1& 40.5
		\\	
		\bottomrule
	\end{tabular}

	\label{sup:tab:s3dis_acc_miou}
\end{table}

We present the results of the semantic segmentation in~\cref{fig:segmentation_demo}. 
It is evident that, 3DGenZ's result contains more noise, as seen in the spotted sofa, bed and other objects.
Conversely, Uni-Fusion-SU's results are generally smoother and more precise.

%
%\begin{figure*}[htbp]
%	\centering
%	\includegraphics[width=.3\linewidth]{example-image-golden}
%	\includegraphics[width=.3\linewidth]{example-image-golden}
%	\includegraphics[width=.3\linewidth]{example-image-golden}
%	\\
%	\includegraphics[width=.3\linewidth]{example-image-golden}
%	\includegraphics[width=.3\linewidth]{example-image-golden}
%	\includegraphics[width=.3\linewidth]{example-image-golden}
%	
%	\caption{Semantic segmentation result on ScanNet.}
%\end{figure*}
%
%\begin{figure*}[htbp]
%	\centering
%	\includegraphics[width=.3\linewidth]{example-image-golden}
%	\includegraphics[width=.3\linewidth]{example-image-golden}
%	\includegraphics[width=.3\linewidth]{example-image-golden}
%	\\
%	\includegraphics[width=.3\linewidth]{example-image-golden}
%	\includegraphics[width=.3\linewidth]{example-image-golden}
%	\includegraphics[width=.3\linewidth]{example-image-golden}
%	
%	\caption{Semantic segmentation result on S3DIS.}
%\end{figure*}

\subsubsection{Scene Understanding with Different Properties}

\begin{figure*}[t!]
	\centering
	\setlength{\tabcolsep}{0.1em}
	\renewcommand{\arraystretch}{.1}
	\resizebox{\textwidth}{!}{\begin{tabular}{|c | c | c | c | c | c|}
			\toprule 
			& \textbf{scene0568\_00} & \textbf{scene0249\_00} & \textbf{scene0435\_00} & \textbf{office3} & \textbf{room0}\\
			\midrule
			{} &
			\raisebox{-.5\height}{\includegraphics[width=\fabImSize\linewidth]{im/exp/fab/scannet/0568_color.png}} & %\raisebox{-.5\height}{\includegraphics[width=\fabImSize\linewidth]{im/exp/fab/scannet/0164_color.png}} &
			\raisebox{-.5\height}{\includegraphics[width=\fabImSize\linewidth]{im/exp/fab/scannet/0249_color.png}} & \raisebox{-.5\height}{\includegraphics[width=\fabImSize\linewidth]{im/exp/fab/scannet/0435_color.png}}
			&
			\raisebox{-.5\height}{\includegraphics[width=\fabImSize\linewidth]{im/exp/fab/replica/office3_color.png}}
			&
			\raisebox{-.5\height}{\includegraphics[width=\fabImSize\linewidth]{im/exp/fab/replica/room0_color.png}}\\ %\raisebox{-.5\height}{\includegraphics[width=\fabImSize\linewidth]{im/exp/fab/scannet/0050_color.png}} %\includegraphics[width=\fabImSize\linewidth]{im/exp/fab/replica/office3_color.png}
			\\
			\textbf{Desk}  &
			\raisebox{-.5\height}{\includegraphics[width=\fabImSize\linewidth]{im/exp/fab/scannet/0568_lt_desk.png}}&
			%\raisebox{-.5\height}{\includegraphics[width=\fabImSize\linewidth]{im/exp/fab/scannet/0164_lt_desk.png}}&
			\raisebox{-.5\height}{\includegraphics[width=\fabImSize\linewidth]{im/exp/fab/scannet/0249_lt_desk.png}}&
			\raisebox{-.5\height}{\includegraphics[width=\fabImSize\linewidth]{im/exp/fab/scannet/0435_lt_desk.png}}
			&
			\raisebox{-.5\height}{\includegraphics[width=\fabImSize\linewidth]{im/exp/fab/replica/office3_lt_desk.png}}
			&
			\raisebox{-.5\height}{\includegraphics[width=\fabImSize\linewidth]{im/exp/fab/replica/room0_lt_desk.png}}\\
			%\raisebox{-.5\height}{\includegraphics[width=\fabImSize\linewidth]{im/exp/fab/scannet/0050_lt_desk.png}}
			%\includegraphics[width=\fabImSize\linewidth]{im/exp/fab/replica/office3_saliency.png}
			\\
			
			\textbf{Sofa} &
			\raisebox{-.5\height}{\includegraphics[width=\fabImSize\linewidth]{im/exp/fab/scannet/0568_lt_sofa.png}} &
			%\raisebox{-.5\height}{\includegraphics[width=\fabImSize\linewidth]{im/exp/fab/scannet/0164_lt_sofa.png}} &
			\raisebox{-.5\height}{\includegraphics[width=\fabImSize\linewidth]{im/exp/fab/scannet/0249_lt_sofa.png}} &
			\raisebox{-.5\height}{\includegraphics[width=\fabImSize\linewidth]{im/exp/fab/scannet/0435_lt_sofa.png}}&
			\raisebox{-.5\height}{\includegraphics[width=\fabImSize\linewidth]{im/exp/fab/replica/office3_lt_sofa.png}}
			&
			\raisebox{-.5\height}{\includegraphics[width=\fabImSize\linewidth]{im/exp/fab/replica/room0_lt_sofa.png}}\\
			%\raisebox{-.5\height}{\includegraphics[width=\fabImSize\linewidth]{im/exp/fab/scannet/0050_lt_sofa.png}}
			%\includegraphics[width=\fabImSize\linewidth]{im/exp/fab/replica/office3_style.png}
			\\
			\textbf{Work} &
			\raisebox{-.5\height}{\includegraphics[width=\fabImSize\linewidth]{im/exp/fab/scannet/0568_lt_work.png}} &
			%\raisebox{-.5\height}{\includegraphics[width=\fabImSize\linewidth]{im/exp/fab/scannet/0164_lt_work.png}} &
			\raisebox{-.5\height}{\includegraphics[width=\fabImSize\linewidth]{im/exp/fab/scannet/0249_lt_work.png}} &
			\raisebox{-.5\height}{\includegraphics[width=\fabImSize\linewidth]{im/exp/fab/scannet/0435_lt_work.png}}&
			\raisebox{-.5\height}{\includegraphics[width=\fabImSize\linewidth]{im/exp/fab/replica/office3_lt_work.png}}
			&
			\raisebox{-.5\height}{\includegraphics[width=\fabImSize\linewidth]{im/exp/fab/replica/room0_lt_work.png}}\\
			%\raisebox{-.5\height}{\includegraphics[width=\fabImSize\linewidth]{im/exp/fab/scannet/0050_lt_work.png}}
			%\includegraphics[width=\fabImSize\linewidth]{im/exp/fab/replica/office3_style.png}
			\\
			\textbf{Sittable} &
			\raisebox{-.5\height}{\includegraphics[width=\fabImSize\linewidth]{im/exp/fab/scannet/0568_lt_sit.png}} &
			%\raisebox{-.5\height}{\includegraphics[width=\fabImSize\linewidth]{im/exp/fab/scannet/0164_lt_sit.png}} &
			\raisebox{-.5\height}{\includegraphics[width=\fabImSize\linewidth]{im/exp/fab/scannet/0249_lt_sit.png}} &
			\raisebox{-.5\height}{\includegraphics[width=\fabImSize\linewidth]{im/exp/fab/scannet/0435_lt_sit.png}}&
			\raisebox{-.5\height}{\includegraphics[width=\fabImSize\linewidth]{im/exp/fab/replica/office3_lt_sit.png}}
			&
			\raisebox{-.5\height}{\includegraphics[width=\fabImSize\linewidth]{im/exp/fab/replica/room0_lt_sit.png}}\\
			%\raisebox{-.5\height}{\includegraphics[width=\fabImSize\linewidth]{im/exp/fab/scannet/0050_lt_sit.png}}
			%\includegraphics[width=\fabImSize\linewidth]{im/exp/fab/replica/office3_style.png}
			\\
			\textbf{Wood} &
			\raisebox{-.5\height}{\includegraphics[width=\fabImSize\linewidth]{im/exp/fab/scannet/0568_lt_wood.png}} &
			%\raisebox{-.5\height}{\includegraphics[width=\fabImSize\linewidth]{im/exp/fab/scannet/0164_lt_wood.png}} &
			\raisebox{-.5\height}{\includegraphics[width=\fabImSize\linewidth]{im/exp/fab/scannet/0249_lt_wood.png}} &
			\raisebox{-.5\height}{\includegraphics[width=\fabImSize\linewidth]{im/exp/fab/scannet/0435_lt_wood.png}}&
			\raisebox{-.5\height}{\includegraphics[width=\fabImSize\linewidth]{im/exp/fab/replica/office3_lt_wood.png}}
			&
			\raisebox{-.5\height}{\includegraphics[width=\fabImSize\linewidth]{im/exp/fab/replica/room0_lt_wood.png}}\\
			%\raisebox{-.5\height}{\includegraphics[width=\fabImSize\linewidth]{im/exp/fab/scannet/0050_lt_wood.png}}
			%\includegraphics[width=\fabImSize\linewidth]{im/exp/fab/replica/office3_style.png}
			\\
			
			
			\bottomrule
		\end{tabular}
	}
	%\captionof{figure}
	\caption{Demonstration of the original mesh, highlighted semantic mesh given various queries.}
	\label{fig:fab_lt}
	\vspace{-.5cm}
\end{figure*}

The main contribution of this application is that, Uni-Fusion is the first model to construct a continuous mapping of high-dimensional embeddings onto the surface without the need for any training of the map representation.
%
In the previous experiment (\cref{sec:exp:semantic}), we evaluate the performance of generalized zero-shot semantic segmentation.
However, the potential of Uni-Fusion goes beyond semantic segmentation.
%
By constructing a LIM, we obtain a surface CLIP feature field.
This enables us to query various semantic categories such as 
%without the need of multiple LIMs or rerun for other properties, we query 
\textbf{Object, Room Type, Material, Affordance and Activity} without requiring multiple LIMs or re-running the model.

We present the results in \cref{fig:fab_lt}, 
where we query object (desk, sofa), activity (work), affordance (sittable), and material (wood).
Uni-Fusion-SU accurately identifies and highlights the object and material regions.
However, for less specific commands such as work or sittable, the model provides a wider range of results with less confidence (indicated by dull yellow).
Nevertheless, the suggested options are also roughly correct.









\subsection{Time}

We run all of the applications in a single pass using our captured office sequences and evaluate the time cost of construction and fusion of each LIM. 
The average time cost across frames is shown in~\cref{tab:time}.

\begin{table}[htbp]
	\caption{Time required for each frame.
	}
	\centering
	\footnotesize
	\setlength{\tabcolsep}{0.7em}
	\resizebox{\linewidth}{!}{
		\begin{tabular}{l|ccccccc}
			\toprule
			&Surface & Color & Infrared & Style & Saliency & Latent&Internal Track \\ \midrule
			Time ($\si{\second}$)&0.100 & 0.038 & 0.045 & 0.048 & 0.045 &0.011 &0.225 \\ \bottomrule
	\end{tabular}}
	
	\label{tab:time}
\end{table}

\newcommand{\mineImSize}{.32}
%\begin{table*}[t!]
%	\centering
%	\setlength{\tabcolsep}{0.1em}
%	\renewcommand{\arraystretch}{.1}
%	\begin{tabular}{|c | c |c |}
%		\hline 
%		{Color} &{Infrared} & {Saliency} \\
%		\includegraphics[width=\mineImSize\linewidth]{im/exp/fab/mine/office/seq3_color.png} &
%		\includegraphics[width=\mineImSize\linewidth]{im/exp/fab/mine/office/seq3_color.png} &
%		\includegraphics[width=\mineImSize\linewidth]{im/exp/fab/mine/office/seq3_saliency.png} \\
%		{Style 1}&{Style 1}&{Style 1}\\
%		\includegraphics[width=\mineImSize\linewidth]{im/exp/fab/mine/office/seq3_style.png}&
%		\includegraphics[width=\mineImSize\linewidth]{im/exp/fab/mine/office/seq3_style.png}&
%		\includegraphics[width=\mineImSize\linewidth]{im/exp/fab/mine/office/seq3_style.png}\\
%		{Sofa}&{Desk}&{Soft}\\
%		\includegraphics[width=\mineImSize\linewidth]{im/exp/fab/mine/office/seq3_lt_sofa.png}&
%		\includegraphics[width=\mineImSize\linewidth]{im/exp/fab/mine/office/seq3_lt_desk.png}&
%		\includegraphics[width=\mineImSize\linewidth]{im/exp/fab/mine/office/seq3_lt_soft.png}\\		
%		\hline
%	\end{tabular}
%	\captionof{figure}{Demonstration on captured office data.}
%	\label{fig:mine_demo}
%\end{table*}
%\begin{figure*}[t!]
%	\centering
%	\setlength{\tabcolsep}{0.1em}
%	\renewcommand{\arraystretch}{.1}
%	\begin{tabular}{|c | c |c |}
%		\hline \hline
%		\includegraphics[width=\mineImSize\linewidth]{im/exp/fab/mine/office/seq3_w_slam_color.png}&	\includegraphics[width=\mineImSize\linewidth]{im/exp/fab/mine/office/seq3_w_slam_ir.png}&	\includegraphics[width=\mineImSize\linewidth]{im/exp/fab/mine/office/seq3_w_slam_saliency.png}\\
%		{Color} &{Infrared} & {Saliency}\\
%<<<<<<< HEAD


%=======
%		
%		\includegraphics[width=\mineImSize\linewidth]{im/exp/fab/mine/office/seq3_w_slam_style.png}
%		&\includegraphics[width=\mineImSize\linewidth]{im/exp/fab/mine/office/seq3_w_slam_lt_desk.png}
%		&\includegraphics[width=\mineImSize\linewidth]{im/exp/fab/mine/office/seq3_w_slam_lt_wood.png}\\
%		{Style} & {Object-desk} & {Material-wood} \\\hline
%>>>>>>> e014bc950c14dec9ffa1d2d7a6de9b7abfefabdd
%	\end{tabular}
%	%\captionof{figure}
%	\caption{Demonstration on captured Office data.}
%	\label{fig:office}
%\end{figure*}

\begin{figure*}[]
	\centering
	\setlength{\tabcolsep}{0.1em}
	\renewcommand{\arraystretch}{.1}
	\begin{tabular}{|c | c |c |}
	\hline \hline
	\includegraphics[width=\mineImSize\linewidth]{im/exp/fab/mine/appartment2/appartment2_color.png}&	\includegraphics[width=\mineImSize\linewidth]{im/exp/fab/mine/appartment2/appartment2_ir.png}&	\includegraphics[width=\mineImSize\linewidth]{im/exp/fab/mine/appartment2/appartment2_saliency.png}\\
		{Color} &{Infrared} & {Saliency}\\
	\includegraphics[width=\mineImSize\linewidth]{im/exp/fab/mine/appartment2/appartment2_style.png}
&\includegraphics[width=\mineImSize\linewidth]{im/exp/fab/mine/appartment2/appartment2_lt_sofa.png}
&\includegraphics[width=\mineImSize\linewidth]{im/exp/fab/mine/appartment2/appartment2_lt_desk.png}
\\
{Style} & {Object-sofa} & {Object-desk}\\
\includegraphics[width=\mineImSize\linewidth]{im/exp/fab/mine/appartment2/appartment2_lt_coat.png}
&\includegraphics[width=\mineImSize\linewidth]{im/exp/fab/mine/appartment2/appartment2_lt_sit.png}
&\includegraphics[width=\mineImSize\linewidth]{im/exp/fab/mine/appartment2/appartment2_lt_wood.png}\\
{Object-coat} & {Affordance-sit} & {Material-wood} \\\hline
	\end{tabular}
%\captionof{figure}
\caption{Demonstration on the captured apartment data.}
\label{fig:appartment}
%\vspace{-.5cm}
\end{figure*}


Using depth and property images of size $720\times1280$ as input, it is evident from the table, that our model operates at a frequency of $\sim10\si{\hertz}$ for  surface (sample mode) LIM construction and integration. 
It alse achieves a frequency of over $20\si{\hertz}$ for color, infrared, style, and saliency.
These results demonstrate the suitability of Uni-Fusion for real-time applications.

However, our internal tracking process takes around $0.225\si{\second}$ per frame, which is relatively slower compared to the mapping module. 
Nevertheless, Uni-Fusion uses external tracking to prevent tracking loss, enabling our internal tracking and mapping to operate at a lower frequency.
As a result, the entire model can be effectively applied in real-time in various scenarios.

\section{Extensive experiment on our own dataset}

In previous experiments, we evaluate the capabilities of Uni-Fusion in different applications. 
To further demonstrate its effectiveness in robotic environmental understanding, we capture our own dataset to show all applications together.

We capture two scenes: The office and apartment of the first author using a Microsoft Kinect Azure. 
%
RGB-D and infrared video are captured. After calibration, RGB, depth, infrared inputs have resolution of $720\times1280$.
Uni-Fusion tracks and reconstructs all applications in one pass.
%
While office data has been involved in ablation study (\cref{exp:surface:ablation}), we showcase all applications using the apartment dataset, as depicted in~\cref{fig:appartment}.

For better visualization, the ceiling of reconstruction is removed.
The top row of images presents the colored mesh with room details, the infrared mesh revealing the lighting effect, and the saliency reconstruction highlighting objects crucial for navigation.
Additionally, we select the second style from~\cref{fig:style} for style transfer to the apartment canvas.
%
As a result, the wooden floor in the room is colored with dark green.
The whole apartment is in a warm style.

The remaining results are generated from the surface field of the CLIP embeddings. 
We issue commands to locate objects, e.g., where is the sofa, desk and coat.
In addition, it easily identifies affordances such as being sittable.
For material, it successfully detects the wooden floor in each room.


\section{Conclusions}
In this paper, we qualitatively and quantitatively analyzed the influence of sub-populations under various metrics, where we observed disparate impacts incurred by sub-populations,
especially when the label noise presents. What is more, our experiment results also reveal that existing robust solutions improve the performance of certain sub-populations at
the cost of hurting others, hence leading to unfair performances among sub-populations. We then propose \textbf{F}airness \textbf{R}egularizer (\fr), which encourages the learned classifier to achieve fair performances across sub-populations. Extensive experiment results demonstrate the effectiveness of \fr, indicating that fairness constraints improve the learning from noisily labeled long-tailed data. 

\section*{Acknowledgement}
This work is partially supported by the National Science Foundation (NSF) under grants IIS-2007951, IIS-2143895, and IIS-2040800 (FAI program in collaboration with Amazon).


\newpage

\bibliography{library, myref, noise_learning, longtail}
\bibliographystyle{plain}


\newpage
\appendix
\onecolumn

\section{Proof of Proposition \ref{pro:scaling_matrix}}
\begin{proof}
This can be shown by $\boldsymbol{D}_R\widetilde{\mathbfsbilow{u}}_{\Xi,c} =\widetilde{\mathbfsbilow{u}}_{\Xi,c} \boldsymbol{D}_L$ and and the property of determinant:
\begin{subequations}
\begin{align}
    &\text{det}(\mathsfbi{I}-\mathbfsbilow{\widetilde{H}}_{\nabla}\mathbfsbilow{\widetilde{u}}_{\Xi,c})\\
    = &\text{det}(\mathsfbi{I}-\mathbfsbilow{\widetilde{H}}_{\nabla}\boldsymbol{D}_R^{-1}\mathbfsbilow{\widetilde{u}}_{\Xi,c}\boldsymbol{D}_L)\\
    =&\text{det}(\boldsymbol{D}_L^{-1}-\mathbfsbilow{\widetilde{H}}_{\nabla}\boldsymbol{D}_L^{-1}\mathbfsbilow{\widetilde{u}}_{\Xi,c})\text{det}(\boldsymbol{D}_L)\\
    =&\text{det}(\boldsymbol{D}_L)\text{det}(\boldsymbol{D}_L^{-1}-\mathbfsbilow{\widetilde{H}}_{\nabla}\boldsymbol{D}_R^{-1}\mathbfsbilow{\widetilde{u}}_{\Xi,c})\\
    =&\text{det}(\mathsfbi{I}-\boldsymbol{D}_L\mathbfsbilow{\widetilde{H}}_{\nabla}\boldsymbol{D}_R^{-1}\mathbfsbilow{\widetilde{u}}_{\Xi,c})
\end{align}
\end{subequations}
Using the definition of structured singular value in \eqref{eq:mu}, we have $\mu_{\widetilde{\mathbfsbilow{U}}_{\Xi,c}}[\mathbfsbilow{\widetilde{H}}_{\nabla}]=\mu_{\widetilde{\mathbfsbilow{U}}_{\Xi,c}}(\boldsymbol{D}_L \mathbfsbilow{\widetilde{H}}_{\nabla} \boldsymbol{D}_R^{-1})$. 
\end{proof}




We firstly provide the relation between structured singular value and the large singular value of a scaled spatio-temporal frequency response operator based on \cite[theorem 3.8]{packard1993complex}. 

\begin{pro}
\label{pro:scaling_matrix}
Given $(k_x,k_z,\omega)$ pair and $\mathbfsbilow{\widetilde{H}}_{\nabla}\in \mathbb{C}^{9N_y\times 9N_y}$, uncertainty set $\widetilde{\mathbfsbilow{U}}_{\Xi,c}$, and scaling positive definite matrix set $\underline{\boldsymbol{D}}_L$ ($\boldsymbol{D}_L^*=\boldsymbol{D}_L\succ 0$) and $\underline{\boldsymbol{D}}_R$ ($\boldsymbol{D}_R^*=\boldsymbol{D}_R\succ 0$) such that for $\boldsymbol{D}_L\in \underline{\boldsymbol{D}}_L$  $\boldsymbol{D}_R\in \underline{\boldsymbol{D}}_R$ and $\widetilde{\mathbfsbilow{u}}_{\Xi,c} \in \widetilde{\mathbfsbilow{U}}_{\Xi,c}$, they satisfy
\begin{align}
\boldsymbol{D}_R\widetilde{\mathbfsbilow{u}}_{\Xi,c} =\widetilde{\mathbfsbilow{u}}_{\Xi,c} \boldsymbol{D}_L.
\label{eq:D_L_D_R_relation}
\end{align}
Then, 
\begin{align}
\mu_{\widetilde{\mathbfsbilow{U}}_{\Xi,c}}[\mathbfsbilow{\widetilde{H}}_{\nabla}]=\mu_{\widetilde{\mathbfsbilow{U}}_{\Xi,c}}(\boldsymbol{D}_L \mathbfsbilow{\widetilde{H}}_{\nabla} \boldsymbol{D}_R^{-1}).
\end{align}
\label{thm:chapter2_mu_lmi}
\end{pro}
However, scaling the matrix $\mathbfsbilow{\widetilde{H}}_{\nabla}$ as $\boldsymbol{D}_L \mathbfsbilow{\widetilde{H}}_{\nabla} \boldsymbol{D}_R^{-1}$ may change the largest singular value, which gives the following inequality to compute the upper bound of structured singular value \cite[equation (11.14)]{zhou1996robust}:
\begin{align}
    \mu_{\widetilde{\mathbfsbilow{U}}_{\Xi,c}}[ \mathbfsbilow{\widetilde{H}}_{\nabla}]=&\mu_{\widetilde{\mathbfsbilow{U}}_{\Xi,c}}[\boldsymbol{D}_L \mathbfsbilow{\widetilde{H}}_{\nabla} \boldsymbol{D}_R^{-1}]\\
    \leq& \underset{\boldsymbol{D}_L\in \underline{\boldsymbol{D}}_L,\boldsymbol{D}_R\in \underline{\boldsymbol{D}}_R }{\text{inf}}\bar{\sigma}[ \boldsymbol{D}_L \mathbfsbilow{\widetilde{H}}_{\nabla} \boldsymbol{D}_R^{-1}].
    \label{eq:chapter2_mu_DMD}
\end{align}
\begin{remark}
The scaling matrix sets $\underline{\boldsymbol{D}}_L$ and $\underline{\boldsymbol{D}}_R$ associated with $\widetilde{\mathbfsbilow{U}}_{\Xi,c}$ in \eqref{eq:uncertain_set} satisfying \eqref{eq:D_L_D_R_relation} can be chosen as:
\begin{align}
    \underline{\boldsymbol{D}}_L=\underline{\boldsymbol{D}}_L=\{&\text{diag}[d_1 \mathsfbi{I}_{N_y\times N_y},d_2\mathsfbi{I}_{N_y\times N_y},... ,d_9\mathsfbi{I}_{N_y\times N_y}]\nonumber\\
    &:d_1, d_2, ...,d_9\in \mathbb{R}\}
\end{align}
\end{remark}

We then use \texttt{mussvextract} command in MATLAB to obtain corresponding $\boldsymbol{D}_L$ and $\boldsymbol{D}_R$ scaling matrices and  

\textbf{Mention the relation of rank-1}



\begin{align}
    \boldsymbol{P}:=&\text{diag}\left(\mathcal{I}_{1\times 3},\mathcal{I}_{1\times 3},\mathcal{I}_{1\times 3}\right),\label{eq:P}\\ \boldsymbol{\widehat{u}}_{\Xi,c}:=&\mathcal{I}_{3\times 3}\otimes\text{diag}\left(-\widehat{u}_\xi,-\widehat{v}_\xi,-\widehat{w}_\xi \right),
    \label{eq:uncertainty_u_Xi_c}
\end{align}

\iffalse
\begin{figure}
    \centering
    \includegraphics[width=0.49\textwidth]{feedback_interconnection_detail_tilde.png}
    \caption{Illustration of structured input--output analysis: (a) a componentwise description,  where blocks inside of ({\color{blue}$\dashed$}, blue) represent the modeled forcing in equation \eqref{eq:feedback_structured_uncertainty} with $\widehat{\boldsymbol{u}}_{\Xi,c}$ being relaxed as $\widetilde{\boldsymbol{u}}_{\Xi,c}$ in \eqref{eq:uncertainty_u_Xi_c_non_repeated}.}
    \label{fig:feedback_detail}
\end{figure}

\begin{figure}
    \centering
    \includegraphics[width=0.1\textwidth]{feedback_interconnection_abstract_tilde.png}
    \caption{ Panel (b) redraws panel (a) after discretization with the top block corresponding to the combination of the three top blocks in panel (a) and the bottom block corresponding to the bottom block of panel (a).}
    \label{fig:feedback_discretized}
\end{figure}

\fi




%In the present work we first modify the originally proposed SIOA feedback interconnection in order to  We then demonstrate that the structural features associated with these correlations are consistent with results of nonlinear optimal perturbation analysis and the secondary stability analysis associated with streamwise streaks. These results provide further evidence that behavior associated with nonlinear effects can be captured through using SIOA based approaches. 
%The structured uncertainty is then the most destabilizing perturbations associated with the given feedback interconnection, which we interpret as being associated with flow structures most likely to be amplified or predominant in the transitional flow.


% \dennice{we could say secondary stability analysis or something here but the term secondary analysis is unclear, even secondary stability analysis is a bit strange for a controls audience but it is fine because people who work on this problem will know what we mean}


%Recently proposed structured input-output analysis (SIOA) employs a feedback interconnection between a spatio-temporal response operator and structured uncertainty that allows us to include a model of the nonlinearity to identify the  characteristic features of transitional wall-bounded shear flows. In the present work we first slightly modify the originally proposed SIOA feedback interconnection in order to decompose the structured uncertainty into three components associated with the streamwise, spanwise and wall-normal velocity correlations. We then demonstrate that the structural features associated with velocity correlations are consistent with results of nonlinear optimal perturbation analysis and the secondary stability analysis associated with streamwise streaks. These results provide further evidence that behavior associated with nonlinear effects can be captured through using SIOA based approaches. 


%Recently, structured input-output analysis (SIOA) has been introduced into transitional wall-bounded shear flows to identify characteristic scales in translational invariant directions. This work aims to further develop SIOA to provide insight into inhomogeneous (wall-normal) directions. We introduce a feedback interconnection between a modified spatio-temporal response operator and structured uncertainty to model the nonlinearity, where structured uncertainty isolates destabilizing velocity vectors into each component. This work then interprets the destabilizing structured uncertainty as destabilizing velocity correlation. The most destabilizing velocity correlation shows absolute value maximized near the channel center consistent with secondary instability of streamwise streaks and shows real part reversing sign near the channel center consistent with nonlinear optimal perturbations. 



%\cliu{It is also observed that nonlinear mechanisms disadvantage the growth of streamwise elongated streaks \cite{duguet2013minimal,Brandt2014}. }

%demonstrated that the input mode is associated with cross-stream forcing resembling streamwise vortices while the output mode corresponds to streamwise streaks \cite{Jovanovic2005,schmid2007nonmodal,jovanovic2020bypass}.


%is able to capture the non-normality of the linearized operator and associated transient growth that is inherent in wall-bounded shear flows. The streamwise elongated structures are identified as the most amplified flow structures, where the input mode is associated with cross-stream forcing resembling streamwise vortices while the output mode corresponds to streamwise streaks \cite{Jovanovic2005,schmid2007nonmodal,jovanovic2020bypass}. This observation captures the important lift-up mechanism that cross-stream forcing redistributes background mean shear across channel height to form streamwise streaks \cite{Brandt2014,jovanovic2020bypass}. However, input-output analysis obscures streamwise dependent structures, which are also known to be important from observations in experiments \cite{prigent2003long}, direct numerical simulation (DNS) \cite{reddy1998stability} and nonlinear optimal perturbations (NLOP) \cite{Rabin2012}. 

%\dennice{I am not sure the point here, singular values are known to provide this type of information and this is a controls paper so people know the difference between eigenvalues and singular values. I think you want to start with input output analysis and then mention that traditional analysis emphasizes the streamwise elongated while nonlinear shows importance of other structures to lead into why we are using SIOA} \cliu{I have edited this paragraph starting from input-output analysis.}




%\dennice{To me the idea here is to further explore the nature of the structure not just the wall normal extent. Before we did not look at the actual forcing so I think that needs to be the focus here. }  \cliu{Chang: Actually this work does not look at the forcing but the structured uncertainty. This structured uncertainty maps the output from $\widetilde{\mathcal{H}}_\nabla$ to become the forcing. Edited this paragraph as below. }\dennice{Forcing in the sense that we say these are optimal perturbations in a sense}

%Previous SIOA based studies have focused primarily on the streamwise and spanwise characteristics of the flow \cite{liu2021structuredJournal,liu2021feedback,liu2022structured}. 


%the wall-normal variation of destabilizing structured uncertainty, which is important within feedback interconnection to generate the structured input forcing. \dennice{Again I do not see the wall normal as the focus here as we are digging into what the singular value looks like and not really providing a wall normal analog of what we had before but we can discuss} 

%\dennice{I think we can say these have been useful but is that really what motivates this interpretation?}



% \dennice{from figure 3 it looks like this is true only for the autocorrelation} \dennice{I added auto to correlation as the profile is $y=y'$} \cliu{Thanks, autocorrelation looks good to me. These statement here is also true for correlation matrix with $y\neq y'$, although we only show one component.  Plotting $y=y'$ is mainly trying to save space otherwise it is not easy to show totally nine components} \dennice{I am a bit confused here maybe we should have  brief talk once you are up. Do you mean the full contour plots look like this? I changed the statement a bit to reflect this}


% The nonlinear term in \eqref{eq:NSDecompf1} is then written as \dennice{formatting makes this hard to read as it looks like one long equation rather than a matrix. I am also not sure why this equation is necessary.  I would just say we model the nonlinearity $u\cdot\nabla u $ as and move right to (3)} \cliu{I agree}
% \begin{subequations}
%      \label{eq:f_nonlinear}
% \begin{align}
% \boldsymbol{f}:=&-\boldsymbol{u}\!\cdot\! \boldsymbol{\nabla }\boldsymbol{u}\\
% =&\begin{bmatrix}-u\partial_x u-v\partial_y u-w\partial_z u\\ -u\partial_x v-v\partial_y v-w\partial_z v\\
% -u\partial_x w-v\partial_y w-w\partial_z w\end{bmatrix}=:\begin{bmatrix}f_x\\
% f_y\\
% f_z\end{bmatrix}. 
% \end{align}
% \end{subequations}
% This expression of the nonlinearity as forcing terms makes \eqref{eq:NS_All} into a set of forced linear evolution equations. 

%\\=:&\begin{bmatrix}f_{xu}+f_{xv}+f_{xw}\\ f_{yu}+f_{yv}+f_{yw}\\f_{zu}+f_{zv}+f_{zw}\end{bmatrix}


%The linear form of \eqref{eq:f_uncertain_model}


% \dennice{I think this sentence belosongs with equation (6) because the is where you are defining the forcing and the structure but we can discuss} \cliu{I agree} \sout{The structured uncertainty $\widetilde{\boldsymbol{u}}_{\Xi,c}$ in \eqref{eq:uncertainty_u_Xi_c_non_repeated} has a block-diagonal structure such that the resulting feedback interconnection leads to a forcing model that retains the componentwise structure of the nonlinearity, which is demonstrated to be important to uncover streamwise dependent flow structures \cite[\S 3.3]{liu2021structuredJournal}.}


%\dennice{I would move this paragraph up before discusing the modification of the formulation. There are a few options which we can discuss}


    % 0, \,\text{if}\;\;\forall \mathbfsbilow{\widetilde{u}}_{\Xi,c}\in \mathbfsbilow{\widetilde{U}}_{\Xi,c}, \text{det}(\mathsfbi{I}-\mathbfsbilow{\widetilde{H}}_{\nabla}\mathbfsbilow{\widetilde{u}}_{\Xi,c})\neq 0
    % \end{cases}.
    
    
    
% \end{widetext}
% \noindent where $N_y$ denotes the number of grid points in $y$. 


% \dennice{This needs a bit more discussion. }

% In order to directly use the \texttt{mussv} command in the Robust Control Toolbox of MATLAB, we relax $\boldsymbol{\widehat{u}}_{\Xi,c}$ in \eqref{eq:uncertainty_u_Xi_c} as:
% \begin{align}
%     \boldsymbol{\widetilde{u}}_{\Xi,c}:=&\text{diag}(-\widehat{u}_{\xi,1}-\widehat{v}_{\xi,1},-\widehat{w}_{\xi,1},\nonumber\\
%     &-\widehat{u}_{\xi,2}-\widehat{v}_{\xi,2},-\widehat{w}_{\xi,2},-\widehat{u}_{\xi,3}-\widehat{v}_{\xi,3},-\widehat{w}_{\xi,3}),
%     \label{eq:uncertainty_u_Xi_c_non_repeated}
% \end{align}
% where $\widehat{u}_{\xi,j}$, $\widehat{v}_{\xi,j}$, $\widehat{w}_{\xi,j}$ ($j=1,2,3$) are not necessarily repeated for three different indices $j=1,2,3$. This decomposition of the forcing function in \eqref{eq:feedback_structured_uncertainty} with relaxation in \eqref{eq:uncertainty_u_Xi_c_non_repeated} is illustrated in the three blocks inside the blue dashed line ({\color{blue}$\dashed$}) in Fig. \ref{fig:feedback_detail}(a).

% \dennice{This should be a different section as it is really results}

% \dennice{Please add number of points, I do not see this elsewhere} \cliu{add the number here.}



%\dennice{there was some repetition here so I edited, please check}


% The upper bound of structured singular value in definition \ref{def:mu} can be obtained by the largest singular value \cite{packard1993complex,zhou1996robust}. We define $\|\widetilde{\mathcal{H}}_{\nabla}\|_\infty(k_x,k_z):=\underset{\omega \in \mathbb{R}}{\text{sup}}\;\bar{\sigma}\left[\mathbfsbilow{\widetilde{H}}_{\nabla}(k_x,k_z,\omega)\right]$, which similarly provides $\|\widetilde{\mathcal{H}}_{\nabla}\|_{\mu,c}(k_x,k_z)\leq \|\widetilde{\mathcal{H}}_{\nabla}\|_\infty(k_x,k_z)$.



% \[
% \mathcal{H}_{pqj}=\mathcal{\hat{C}}_p(i\omega\mathcal{I}-\hat{A})^{-1}\mathcal{\hat{B}}_q,\quad j=1,2,3
% \]
% where 
% BLAH BLAH

%\dennice{I would define this here since we have a bit of space since you are using the notation below and it is not defined you can use the version in (3.3) of the original paper with the modification here. } \cliu{I move the definition from theorem to above here. }

% Quantities similar to a velocity correlation have been widely analyzed in related analysis; see e.g., \cite[Fig. 9]{Zare2017}, \cite[Fig. 8]{liu2020input}, \cite[Fig. 14]{towne2020resolvent} and \cite[Fig. 6]{nogueira2021forcing}.We note that $\widetilde{\mathbfsbilow{u}}_{\Xi,c}(y,y';k_x, k_z, \omega)$ is in spectrum space so in general it can be a complex function.


% \dennice{This notation does not match equation (12) it should be a capitol u throughout this part right?}

%\dennice{I removed the note about it being complex.  You can move that to after equation (12) but it does not make sense here sine (12) says they are complex objects}  

%\dennice{I do not see another place where number of grid points is indicated}


%that imaginary parts $\widehat{u}_{\xi,2}$, $\widehat{v}_{\xi,2}$ and $\widehat{w}_{\xi,2}$ are closely matching the absolute values over wall-normal direction $y$, a behavior different from that associated with $x$ momentum equation shown in Fig. \ref{fig:u_xi_abs_real_imag_diag_1_cou}. For the  $\widehat{u}_{\xi,3}$, $\widehat{v}_{\xi,3}$ and $\widehat{w}_{\xi,3}$, their shape is similar to  $\widehat{u}_{\xi,1}$, $\widehat{v}_{\xi,1}$ and $\widehat{w}_{\xi,1}$ as shown in Fig. \ref{fig:u_xi_abs_real_imag_diag_1_cou}, while their absolute values on $\widehat{u}_{\xi,3}$ and $\widehat{w}_{\xi,3}$ is smaller than their counterpart by $\widehat{u}_{\xi,1}$ an $\widehat{w}_{\xi,1}$. This suggests that different components of structured uncertainty that destabilizes the feedback interconnection are showing different behavior associated with each momentum equation within the current formulation. 



%We also find that the absolute values of $\widehat{u}_{\xi,1}$ and $\widehat{w}_{\xi,1}$ have a similar order of magnitude and both of them are larger than $\widehat{v}_{\xi,1}$. Furthermore, the $\widehat{u}_{\xi,1}$ and $\widehat{w}_{\xi,1}$ show almost the same shape over $(y,y')$. 



\iffalse
\begin{widetext}
\begin{figure}
    \centering
    %u1
 (a) $|10^3\,\widehat{u}_{\xi,1}(y,y')|$ \hspace{0.12\textwidth} (b) $\mathcal{R}e[10^3\,\widehat{u}_{\xi,1}(y,y')]$ \hspace{0.09\textwidth} (c) $\mathcal{I}m[10^3\,\widehat{u}_{\xi,1}(y,y')]$ \hspace{0.04\textwidth} (d) $10^3\,\widehat{u}_{\xi,1}(y,y'=y)$
    
    \includegraphics[width=0.27\textwidth]{figure/rotating_laminar_cou_Re_358rotation_0_ssvd_20220804_N122u1_xiNy=120_abs.png}
    \includegraphics[width=0.27\textwidth]{figure/rotating_laminar_cou_Re_358rotation_0_ssvd_20220804_N122u1_xiNy=120_real.png}
    \includegraphics[width=0.27\textwidth]{figure/rotating_laminar_cou_Re_358rotation_0_ssvd_20220804_N122u1_xiNy=120_imag.png}
    \includegraphics[width=0.135\textwidth,trim=-0 -0.7in 0 0]{figure/rotating_laminar_cou_Re_358rotation_0_ssvd_20220804_N122u1_xiNy=120_diag_abs_real_imag.png}

    %v1
     (e) $|10^3\,\widehat{v}_{\xi,1}(y,y')|$ \hspace{0.12\textwidth} (f) $\mathcal{R}e[10^3\,\widehat{v}_{\xi,1}(y,y')]$ \hspace{0.09\textwidth} (g) $\mathcal{I}m[10^3\,\widehat{v}_{\xi,1}(y,y')]$ \hspace{0.04\textwidth} (h) $10^3\,\widehat{v}_{\xi,1}(y,y'=y)$
     
    \includegraphics[width=0.27\textwidth]{figure/rotating_laminar_cou_Re_358rotation_0_ssvd_20220804_N122v1_xiNy=120_abs.png}
    \includegraphics[width=0.27\textwidth]{figure/rotating_laminar_cou_Re_358rotation_0_ssvd_20220804_N122v1_xiNy=120_real.png}
    \includegraphics[width=0.27\textwidth]{figure/rotating_laminar_cou_Re_358rotation_0_ssvd_20220804_N122v1_xiNy=120_imag.png}
    \includegraphics[width=0.135\textwidth,trim=-0 -0.7in 0 0]{figure/rotating_laminar_cou_Re_358rotation_0_ssvd_20220804_N122v1_xiNy=120_diag_abs_real_imag.png}
    
    %w1
     (i) $|10^3\,\widehat{w}_{\xi,1}(y,y')|$ \hspace{0.11\textwidth} (j) $\mathcal{R}e[10^3\,\widehat{w}_{\xi,1}(y,y')]$ \hspace{0.08\textwidth} (k) $\mathcal{I}m[10^3\,\widehat{w}_{\xi,1}(y,y')]$ \hspace{0.03\textwidth} (l) $10^3\,\widehat{w}_{\xi,1}(y,y'=y)$
     
    \includegraphics[width=0.27\textwidth]{figure/rotating_laminar_cou_Re_358rotation_0_ssvd_20220804_N122w1_xiNy=120_abs.png}
    \includegraphics[width=0.27\textwidth]{figure/rotating_laminar_cou_Re_358rotation_0_ssvd_20220804_N122w1_xiNy=120_real.png}
    \includegraphics[width=0.27\textwidth]{figure/rotating_laminar_cou_Re_358rotation_0_ssvd_20220804_N122w1_xiNy=120_imag.png}
    \includegraphics[width=0.135\textwidth,trim=-0 -0.7in 0 0]{figure/rotating_laminar_cou_Re_358rotation_0_ssvd_20220804_N122w1_xiNy=120_diag_abs_real_imag.png}

    % % %u2
    %  (a) $|10^3\,\widehat{u}_{\xi,2}(y,y')|$ \hspace{0.12\textwidth} (b) $\mathcal{R}e[10^3\,\widehat{u}_{\xi,2}(y,y')]$ \hspace{0.09\textwidth} (c) $\mathcal{I}m[10^3\,\widehat{u}_{\xi,2}(y,y')]$ \hspace{0.04\textwidth} (d) $10^3\,\widehat{u}_{\xi,2}(y,y'=y)$

    % \includegraphics[width=0.27\textwidth]{figure/rotating_laminar_cou_Re_358rotation_0_ssvd_20220804_N122u2_xiNy=120_abs.png}
    % \includegraphics[width=0.27\textwidth]{figure/rotating_laminar_cou_Re_358rotation_0_ssvd_20220804_N122u2_xiNy=120_real.png}
    % \includegraphics[width=0.27\textwidth]{figure/rotating_laminar_cou_Re_358rotation_0_ssvd_20220804_N122u2_xiNy=120_imag.png}
    % \includegraphics[width=0.135\textwidth,trim=-0 -0.7in 0 0]{figure/rotating_laminar_cou_Re_358rotation_0_ssvd_20220804_N122u2_xiNy=120_diag_abs_real_imag.png}
    
    % %v2
    % (e) $|10^3\,\widehat{v}_{\xi,2}(y,y')|$ \hspace{0.12\textwidth} (f) $\mathcal{R}e[10^3\,\widehat{v}_{\xi,2}(y,y')]$ \hspace{0.09\textwidth} (g) $\mathcal{I}m[10^3\,\widehat{v}_{\xi,2}(y,y')]$ \hspace{0.04\textwidth} (h) $10^3\,\widehat{v}_{\xi,2}(y,y'=y)$

    % \includegraphics[width=0.27\textwidth]{figure/rotating_laminar_cou_Re_358rotation_0_ssvd_20220804_N122v2_xiNy=120_abs.png}
    % \includegraphics[width=0.27\textwidth]{figure/rotating_laminar_cou_Re_358rotation_0_ssvd_20220804_N122v2_xiNy=120_real.png}
    % \includegraphics[width=0.27\textwidth]{figure/rotating_laminar_cou_Re_358rotation_0_ssvd_20220804_N122v2_xiNy=120_imag.png}
    % \includegraphics[width=0.135\textwidth,trim=-0 -0.7in 0 0]{figure/rotating_laminar_cou_Re_358rotation_0_ssvd_20220804_N122v2_xiNy=120_diag_abs_real_imag.png}
    
    % %w2
    % (i) $|10^3\,\widehat{w}_{\xi,2}(y,y')|$ \hspace{0.11\textwidth} (j) $\mathcal{R}e[10^3\,\widehat{w}_{\xi,2}(y,y')]$ \hspace{0.08\textwidth} (k) $\mathcal{I}m[10^3\,\widehat{w}_{\xi,2}(y,y')]$ \hspace{0.03\textwidth} (l) $10^3\,\widehat{w}_{\xi,2}(y,y'=y)$

    % \includegraphics[width=0.27\textwidth]{figure/rotating_laminar_cou_Re_358rotation_0_ssvd_20220804_N122w2_xiNy=120_abs.png}
    % \includegraphics[width=0.27\textwidth]{figure/rotating_laminar_cou_Re_358rotation_0_ssvd_20220804_N122w2_xiNy=120_real.png}
    % \includegraphics[width=0.27\textwidth]{figure/rotating_laminar_cou_Re_358rotation_0_ssvd_20220804_N122w2_xiNy=120_imag.png}
    % \includegraphics[width=0.135\textwidth,trim=-0 -0.7in 0 0]{figure/rotating_laminar_cou_Re_358rotation_0_ssvd_20220804_N122w2_xiNy=120_diag_abs_real_imag.png}

    % %u3
    % (a) $|10^3\,\widehat{u}_{\xi,3}(y,y')|$ \hspace{0.12\textwidth} (b) $\mathcal{R}e[10^3\,\widehat{u}_{\xi,3}(y,y')]$ \hspace{0.09\textwidth} (c) $\mathcal{I}m[10^3\,\widehat{u}_{\xi,3}(y,y')]$ \hspace{0.04\textwidth} (d) $10^3\,\widehat{u}_{\xi,3}(y,y'=y)$

    % \includegraphics[width=0.27\textwidth]{figure/rotating_laminar_cou_Re_358rotation_0_ssvd_20220804_N122u3_xiNy=120_abs.png}
    % \includegraphics[width=0.27\textwidth]{figure/rotating_laminar_cou_Re_358rotation_0_ssvd_20220804_N122u3_xiNy=120_real.png}
    % \includegraphics[width=0.27\textwidth]{figure/rotating_laminar_cou_Re_358rotation_0_ssvd_20220804_N122u3_xiNy=120_imag.png}
    % \includegraphics[width=0.135\textwidth,trim=-0 -0.7in 0 0]{figure/rotating_laminar_cou_Re_358rotation_0_ssvd_20220804_N122u3_xiNy=120_diag_abs_real_imag.png}
    
    % %v3
    % (e) $|10^3\,\widehat{v}_{\xi,3}(y,y')|$ \hspace{0.12\textwidth} (f) $\mathcal{R}e[10^3\,\widehat{v}_{\xi,3}(y,y')]$ \hspace{0.09\textwidth} (g) $\mathcal{I}m[10^3\,\widehat{v}_{\xi,3}(y,y')]$ \hspace{0.04\textwidth} (h) $10^3\,\widehat{v}_{\xi,3}(y,y'=y)$

    % \includegraphics[width=0.27\textwidth]{figure/rotating_laminar_cou_Re_358rotation_0_ssvd_20220804_N122v3_xiNy=120_abs.png}
    % \includegraphics[width=0.27\textwidth]{figure/rotating_laminar_cou_Re_358rotation_0_ssvd_20220804_N122v3_xiNy=120_real.png}
    % \includegraphics[width=0.27\textwidth]{figure/rotating_laminar_cou_Re_358rotation_0_ssvd_20220804_N122v3_xiNy=120_imag.png}
    % \includegraphics[width=0.135\textwidth,trim=-0 -0.7in 0 0]{figure/rotating_laminar_cou_Re_358rotation_0_ssvd_20220804_N122v3_xiNy=120_diag_abs_real_imag.png}
    
    % %w3
    % (i) $|10^3\,\widehat{w}_{\xi,3}(y,y')|$ \hspace{0.11\textwidth} (j) $\mathcal{R}e[10^3\,\widehat{w}_{\xi,3}(y,y')]$ \hspace{0.08\textwidth} (k) $\mathcal{I}m[10^3\,\widehat{w}_{\xi,3}(y,y')]$ \hspace{0.03\textwidth} (l) $10^3\,\widehat{w}_{\xi,3}(y,y'=y)$

    % \includegraphics[width=0.27\textwidth]{figure/rotating_laminar_cou_Re_358rotation_0_ssvd_20220804_N122w3_xiNy=120_abs.png}
    % \includegraphics[width=0.27\textwidth]{figure/rotating_laminar_cou_Re_358rotation_0_ssvd_20220804_N122w3_xiNy=120_real.png}
    % \includegraphics[width=0.27\textwidth]{figure/rotating_laminar_cou_Re_358rotation_0_ssvd_20220804_N122w3_xiNy=120_imag.png}
    % \includegraphics[width=0.135\textwidth,trim=-0 -0.7in 0 0]{figure/rotating_laminar_cou_Re_358rotation_0_ssvd_20220804_N122w3_xiNy=120_diag_abs_real_imag.png}
    
    \caption{$\widehat{u}_{\xi,1}$ (panels (a)-(d)), $\widehat{v}_{\xi,1}$ (panels (e)-(h)) and $\widehat{w}_{\xi,1}$ (panels (i)-(l)) of plane Couette flow at $Re=358$, $k_x=0.22$, $k_z=0.67$ and $\omega=0$. }
    \label{fig:u_xi_abs_real_imag_diag_1}
\end{figure}
\end{widetext}
\fi

%{\color{blue}shouldn't the symbols be bold in the bottom loop? this should mirror (11) and (12) right?}




% \widetilde{\mathcal{H}}_{\nabla,i}:=&\text{diag}\left(\widehat{\boldsymbol{\nabla}},\widehat{\boldsymbol{\nabla}},\widehat{\boldsymbol{\nabla}}\right)\mathcal{H}\boldsymbol{P}_i,\;\;i=1,2,3 \;\;\text{with}\\
%     \boldsymbol{P}_1=&\text{diag}(\mathcal{I}_{1\times 3}, \boldsymbol{0}_{1\times 3},\boldsymbol {0}_{1,\times 3}),\\
%     \boldsymbol{P}_2=&\text{diag}(\boldsymbol{0}_{1\times 3}, \mathcal{I}_{1\times 3}, \boldsymbol{0}_{1,\times 3}),\\
%     \boldsymbol{P}_3=&\text{diag}(\boldsymbol{0}_{1\times 3}, \boldsymbol{0}_{1,\times 3}, \mathcal{I}_{1\times 3}),




\begin{figure}
    \centering
    (a) $10^3\,\widehat{\mathbfsbilow{u}}_{\xi,1}$\hspace{0.05\textwidth} (b) $10^3\,\widehat{\mathbfsbilow{v}}_{\xi,1}$ \hspace{0.05\textwidth} (c) $10^3\,\widehat{\mathbfsbilow{w}}_{\xi,1}$
    
    \includegraphics[width=0.135\textwidth]{figure/rotating_laminar_poi_Re_690rotation_0_ssvd_20220804_N122u1_xiNy=120_diag_abs_real_imag.png}
    \includegraphics[width=0.135\textwidth]{figure/rotating_laminar_poi_Re_690rotation_0_ssvd_20220804_N122v1_xiNy=120_diag_abs_real_imag.png}
    \includegraphics[width=0.135\textwidth]{figure/rotating_laminar_poi_Re_690rotation_0_ssvd_20220804_N122w1_xiNy=120_diag_abs_real_imag.png}
    
    % (d) $10^3\,\widehat{\mathbfsbilow{u}}_{\xi,2}$\hspace{0.05\textwidth} (e) $10^3\,\widehat{\mathbfsbilow{v}}_{\xi,2}$ \hspace{0.05\textwidth} (f) $10^3\,\widehat{\mathbfsbilow{w}}_{\xi,2}$
    
    % \includegraphics[width=0.135\textwidth]{figure/rotating_laminar_poi_Re_690rotation_0_ssvd_20220804_N122u2_xiNy=120_diag_abs_real_imag.png}
    % \includegraphics[width=0.135\textwidth]{figure/rotating_laminar_poi_Re_690rotation_0_ssvd_20220804_N122v2_xiNy=120_diag_abs_real_imag.png}
    % \includegraphics[width=0.135\textwidth]{figure/rotating_laminar_poi_Re_690rotation_0_ssvd_20220804_N122w2_xiNy=120_diag_abs_real_imag.png}

    % (g) $10^3\,\widehat{\mathbfsbilow{u}}_{\xi,3}$\hspace{0.05\textwidth} (h) $10^3\,\widehat{\mathbfsbilow{v}}_{\xi,3}$ \hspace{0.05\textwidth} (i) $10^3\,\widehat{\mathbfsbilow{w}}_{\xi,3}$

    % \includegraphics[width=0.135\textwidth]{figure/rotating_laminar_poi_Re_690rotation_0_ssvd_20220804_N122u3_xiNy=120_diag_abs_real_imag.png}
    % \includegraphics[width=0.135\textwidth]{figure/rotating_laminar_poi_Re_690rotation_0_ssvd_20220804_N122v3_xiNy=120_diag_abs_real_imag.png}
    % \includegraphics[width=0.135\textwidth]{figure/rotating_laminar_poi_Re_690rotation_0_ssvd_20220804_N122w3_xiNy=120_diag_abs_real_imag.png}

    \caption{First three components of $\widetilde{\mathbfsbilow{u}}_{\Xi,c}$; i.e.,  $10^3\,\widehat{\mathbfsbilow{u}}_{\xi,1}$, $10^3\,\widehat{\mathbfsbilow{v}}_{\xi,1}$, and $10^3\,\widehat{\mathbfsbilow{w}}_{\xi,1}$ for plane Poiseuille flow at $Re=690$, $k_x=0.65$, $k_z=1.56$ and $c=-\omega/k_x=0.53$. 
    %Each component of $\widetilde{\mathbfsbilow{u}}_{\Xi,c}$ for plane Poiseuille flow at $Re=690$, $k_x=0.65$, $k_z=1.56$ and $c=-\omega/k_x=0.53$. 
    }
    \label{fig:u_xi_diag_2_3_poi}
\end{figure}



\end{document}
