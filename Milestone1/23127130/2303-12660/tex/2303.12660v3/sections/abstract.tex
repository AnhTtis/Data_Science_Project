We investigate the structural factors that drive cascading failures in production networks, focusing on quantifying these risks with a topological resilience metric corresponding to the largest exogenous systemic shock that the production network can withstand, such that almost all of the network survives with high probability. We model failures using a node percolation process where systemic shocks cause suppliers to fail, leading to further breakdowns. We classify networks into two categories -- resilient and fragile -- based on their ability to handle shocks as the network grows large, and give bounds on their resilience. We show that the main factors affecting resilience are the number of raw products (primary sector), the number of final goods (final sector), and the source and supply dependencies. Further, we give methods to lower bound resilience based on bounding the cascade size with a linear program that can be efficiently calculated. We establish connections between our model, the independent cascade model, the Risk Exposure Index, and the Eisenberg-Noe contagion model. We give an almost linear-time deterministic algorithm to approximate the cascade size, which matches known lower bounds up to logarithmic factors. Finally, we design intervention algorithms and show that under reasonable assumptions, targeting nodes based on Katz centrality in the edge-reversed network is optimal. Finally, we account for network heterogeneities and validate our findings with real-world data.

