\section{Managerial Insights} \label{sec:discussion}



\xpar{Increased risk from raw products} Our results show that \bi{production networks with even a sublinear number of raw products are fragile}. \cref{theorem:resilience_graph_statistics} implies that $\Omega (K^{2/3})$ raw products suffice for the network to be fragile. Two direct corollaries of this are the tree network, where the number of raw products is $m^D$, for $m$-ary tree with $D$ layer and a complex product at the root, and the trellis network with width at least $K^{2/3}$ and depth at most $K^{1/3}$, which both constitute fragile families of structures. This increased risk from raw products is also verified empirically and we show that the average resilience decreases as the number of raw products increase using Monte Carlo estimates of resilience on real-word supply chain networks (\cref{tab:resilience_auc}). This agrees with our theoretical results and prior works that state networks with higher interdependence in the tree model (which correspond to a higher number of raw products) are less resilient \citep{elliott2022supply}. Qualitatively, these graphs look like ``cowboy hats'' (see \cref{fig:qualitative_representation}), with a large number of raw products on the base. 

\xpar{Increased resilience from lower dependencies, fewer raw products, and fewer final goods} Our lower bounds on the resilience of the complex products (\cref{theorem:parallel_products}), and general networks (\cref{theorem:resilience_graph_statistics}) show that \bi{networks with low dependencies ($\mu, m$) and few raw products (small $r$) and few final goods (small $c$) are resilient}. For example, the sparse random trellis network (\cref{informaltheorem:trellis}) with constant width is resilient. Qualitatively, such graphs look like ``rolling pins'' (see \cref{fig:qualitative_representation}), while there are small primary and final sectors and sparse connections in the layers between them. 

\xpar{Connections to Risk Exposure Index (REI)} We adopt the definition of REI in \cite{levi2016identifying} to our setup by considering the product that has the maximum potential impact on the size of the cascading failures and we show that under reasonable assumptions the \bi{Risk Exposure Index of a network is determined by the node with the highest Katz centrality in the edge-reversed graph}. Our empirical results verify the inverse relationship between REI and our resilience metric. 

% Even though our measure is static, it can still be used to characterize the REI of a network in its steady state as the failure probability $x$ would correspond to the stationary probability of a supplier being unfunctional, which is directly related to the time-to-recover model presented in \cite{levi2016identifying}. 

% Our insights so far show that more interconnected networks (with increased interdependencies among products) are less resilient. Specifically, in the case of tree networks, no more than a $(1/m)$-fraction of the products survive as we increase the node degrees $(m \to \infty)$, making the structure fragile --- see \cref{eq:quick} and the discussion in its preceding paragraph. Similarly, in the case of parallel product networks, we observe that resilience metrics converge to zero as the source dependency increases (in-degree $m\to\infty$; see \cref{theorem:parallel_products}).  Furthermore, in the trellis case, the network becomes less resilient when the interdependency -- which corresponds to the average in-degree $pw$ -- is large. This agrees with our empirical observations. Firstly, in the case of \citet{willems2008data} supply-chain networks, the networks with higher sourcing dependencies and higher average degrees were less resilient. Additionally, our results using Input-Output networks of world economies show that networks that are denser (with higher average degrees) --- which generally correspond to more developed economies --- are less resilient since shocks can spread more easily in denser networks (see \cref{tab:statistics-world} and \cref{fig:wolrd-io}). 




% \xpar{Raw Materials vs. Complex Products} A second potential point of discussion is whether for an economy to be resilient -- through the lens of our model -- should it focus on producing more raw materials or more complex products. 
% First, \cref{theorem:tree_resilience} shows that as we increase the depth $D$ of the network, the network becomes less resilient. Similarly, in the case of trellis architectures, we show that denser and deeper, rather than sparser and wider, production networks are less resilient (see \cref{theorem:trellis_resilience}). Experimentally, this is also verified in \cref{subfig:willems_resilience} since Network \#30, which is more concentrated towards complex products (i.e., later tiers), is found to be less resilient experimentally. In contrast, Network \#10 is mostly concentrated on raw materials and is found to be the most resilient network.  

\xpar{Optimal interventions} We show that \bi{targeting nodes with highest Katz centrality in the edge-reversed graph is optimal} under regularity assumptions --- these nodes represent less complex products. This argument agrees with our intuition and recent observations from the COVID-19 pandemic, where the failure of chip manufacturing industries (which are relatively close to the raw materials, compared e.g., to more complicated electronic devices) caused very large global cascades. 


% \xpar{Resilience, Cascades, and Systemic Risk} \bi{Our work also has strong connections to the financial contagion  \citep{eisenberg2001systemic}  independent cascades \citep{kempe2003maximizing},  and systemic risk \citep{chen2013axiomatic}}. Specifically, the linear program we give which can be used to give an $(1 + \varrho + O(\varrho^2))$-approximation to the cascade size ($\ev {} {F_y}$) which can be computed in almost linear-time $\tilde O(K + M)$ time for small infection probabilities $y = \varrho / m$ for $\varrho \in (0, 1)$ (which is in general \#P-hard to compute; see \cite{chen2010scalable}). This matches the $\Omega(K + M)$ lower bound for influence maximization (up to log factors); see \cite{borgs2014maximizing}. This constitutes a novel result and connection to the independent cascade literature, perhaps of independent interest, to the best of our knowledge. Finally, the dual program to maximizing a weighted cascade is a systemic risk measure, as shown in \cite{chen2013axiomatic}. 

\begin{figure}[t]
    \centering
    \includegraphics[width=\linewidth]{figures/resilience_slide_2.pdf}
    \caption{Qualitative representation of fragile and resilient networks}
    \label{fig:qualitative_representation}
\end{figure}

\xpar{Role of heterogeneity and correlations} We extend our resilience metric to capture supply heterogeneities and correlations among suppliers and products. Our results show that while restricting correlations to suppliers of the same product have limited effect, \bi{correlation across suppliers of different product can significantly impact resilience, rendering all network structures fragile}. Low dimensional representations based on lower order moments of the joint distribution opens a way to efficiently estimate expected cascade size using measurable failure statistics on historical supply chain data. Our empirical results on real-word data set of multi-echelon supply chain networks \citep{willems2008data} also verify that higher supplier correlation reduces resilience. 

\section{Concluding Remarks and Future Directions}\label{sec:conc}

\xpar{Conclusions} Through our paper, we aim to devise a systematic way to test which supply chain networks are \emph{resilient}, i.e., can withstand ``large enough'' shocks while sourcing almost all their productions. Through this metric of resilience, we classify some common supply chain architectures as resilient or fragile and identify structural factors that affect resiliency. We then generalize our analysis and provide tools for studying the resilience of general supply networks, as well as motivating methods that can be used for the design of optimal interventions. 
% Overall, we make some general observations that are consistent with and expand prior works \citep{elliott2022supply}: as multisourcing (number of suppliers) increases, supply chains become more resilient, whereas dependencies across products can cause vulnerabilities in the supply network.  

\xpar{Future directions} There are various ways that our model can be extended. The first interesting direction is to fit our model of the production network to economic and financial networks that are encountered in practice and real-world settings. Our model can be extended to contain costs, quantities, and link capacities, and the desired objective would be to optimize the network total profits subject to production shocks, extending, thus, a long-standing line of work on economic networks \citep{hallegatte2008adaptive,acemoglu2016networks,acemoglu2012network}. Another promising avenue is to give theoretical bounds on networks with varying interdependency and multi-sourcing. So far, we have assumed that products have the same number $n$ of suppliers. However, this can be extended to cases where each industry has random size $n_i \sim \cD_{\mathsf{industry}}$, where $\cD_{\mathsf{industry}}$ is a distribution (e.g., power law, as in \citet{gabaix2011granular}). It would be interesting to study this heterogeneity in the size of different industries and how it affects $R_{\cG}(\varepsilon)$; see also \citet{gabaix2011granular}.  Moreover, as we saw in \cref{sec:general_supply_chains}, there is a direct connection between cascading failures in supply chains and systemic risk in financial networks. Therefore, we can readily adapt tools from the study of financial risk to supply chain resiliency and risk. 


