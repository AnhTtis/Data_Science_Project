
\section{Introduction} \label{sec:introduction}

The global economy consists of many interconnected entities that are responsible for the supply and sourcing of products \citep{carvalho2019production}. These \emph{production networks} consist of \emph{products}, each of which has some required \emph{inputs} that can be obtained from a group of \emph{suppliers} and play a critical role in the day-to-day operations of the world economy \citep{long1983real}. This interdependence between products leads to \emph{cascading failures} once parts of the supply chain are disrupted \citep{horvath1998cyclicality,carvalho2019production,lucas1995understanding,gabaix2011granular,acemoglu2012network,hallegatte2008adaptive}. For example, the recent COVID-19 pandemic and the war in Ukraine disrupted parts of global supply, whose failures then spread to other parts of the supply chain network, causing \emph{bottlenecks} and \emph{choke points} \citep{ergun2022structured,elliott2022supply,guan2020global,walmsley2021impacts,levi2016identifying}. In light of such problems, there is an ever-emerging need to study the resilience of supply chain networks to identify vulnerabilities, such as bottlenecks and choke points, and take steps to mitigate their effects \citep{kleindorfer2005managing,gurnani2012supply,simchi2014superstorms,bimpikis2018multisourcing,simchi2015identifying}. A motivating example is a tree production network in which various raw materials lead to the production of more specialized products in several layers. There, it is easy to observe that the failure to produce any of the raw materials, because all the suppliers that make them have failed, has a devastating effect on the network and almost nothing can be produced. 

Various methods and approaches can be used to study the resilience of supply chain networks \citep{perera2017network,artime2024robustness,elliott2023supply}. One direction is to use network analysis tools to identify critical components of the network and assess their vulnerability to disruption. This can include identifying key suppliers, products, and transportation routes and evaluating their importance to the overall network. Another approach is to use simulation modeling to analyze the impact of different disruptions on the network and evaluate the network's ability to recover from these disruptions \citep{simchi2015identifying}. This can include analyzing the effect of different disturbances (e.g., natural disasters, supply chain failures, etc.) and scenarios where disruptions occur (e.g., simultaneous disruptions, sequential disruptions, etc.). In addition to these technical approaches, it is important to consider organizational and strategic approaches to building resilience in supply chain networks \citep{braunscheidel2009organizational,simchi2018increasing}. This can include implementing contingency plans, developing relationships with alternative suppliers, and diversifying the network to reduce the impact of a single point of failure. Building resilience in supply chain networks requires a multifaceted approach that includes technical analysis, strategic planning, and organizational preparedness. By understanding the structure and dynamics of these networks, it is possible to build foundational resilience into systems that can evolve to withstand disruptions and recover more quickly when they occur.

\begin{figure}[t]
    \centering
    \includegraphics[width=0.75\linewidth]{figures/resilience_slide.pdf}
    \caption{High-level graphical overview of our main results. Exact bounds are located in \cref{tab:resiliences}.}
    \label{fig:resiliences_graphic}
\end{figure}

\subsection{Overview of the Results} \label{sec:contribution}

We model the product requirements as a digraph, including raw material with no incoming edges and only outgoing edges, and more complex products whose sets of incoming edges indicate requirements for their productions. We refer to this digraph as the production network and also assume that each product has a number of suppliers that operate and can fail with probability $x$. According to our model, a product can be produced when all its requirements are met and its suppliers do not all fail. We consider two models for percolation: \textit{(i)} a \bi{homogeneous model} according to which each product has $n$ suppliers and suppliers fail i.i.d., and \textit{(ii)} a \bi{heterogeneous model} according to which each sector has a varying size (modeled by a heterogeneous number of suppliers) and supplier failures can be correlated. 

For a high-level summary of managerial insights, see \cref{sec:discussion}. 

\xpar{Emergence of Power Laws due to Cascading Failures (\cref{sec:preliminaries})} To start with, we state that the size of cascading failures in a production network follows a power law when the underlying production network is a random directed acyclic graph (DAG) -- representing a simple topological hierarchy among $K$ ordered products whereby the production of more complex products is contingent on the supply of simpler products and raw materials.  Our first result follows:

% \begin{quote}
\begin{informaltheorem}
    Consider the production network of $K$ products that is realized according to a random DAG model with edge probability $p$, and consider the node percolation model with failure probability $x$ for each of the $n$ suppliers of each product. Let $F$ be the total number of products that cannot be produced due to the unavailability of their suppliers or other required products. Then $F$ follows a power law distribution and admits a lower tail bound of $\;\Pr [F \ge f] \ge {C}/{f}$ for some constant $C > 0$ which depends on $K$, $p$, and $x$.
\end{informaltheorem}
% \end{quote}

Random DAGs provide a simple and clear representation of our resilience measure. These and other stylized models of production networks allow us to gain a foundational understanding before analyzing the resilience of real-world production networks (see, e.g., \cite{willems2008data,bimpikis2019supply}).

\xpar{The Resilience Metric (\cref{sec:architectures})} We study the drivers of resilience in several stylized models of production networks and identify resilient and non-resilient architectures. Recall that we use $x$ to denote the probability of suppliers failing randomly. Consider a production network ${\cG}$ with $K$ products. Our notion of resilience, denoted by $R_{\cG} (\varepsilon)$, determines the maximum value of $x$ such that at least $(1 - \varepsilon)$ fraction of the products in the production network ${\cG}$ are produced with probability at least $1 -1/K$. In particular, we are interested in the behavior of $R_{\cG}(\varepsilon)$ for fixed $\varepsilon$ as $K\to\infty$. For resilient production networks, this limit is bounded away from zero. We show that this asymptotic notion of resilience is non-trivial, as resilient and fragile families of networks both exist. In the sequel, we provide a non-asymptotic characterization of resilience, i.e., for a production network of finite size with a finite number of suppliers, suitable for analyzing real-world network architectures.

\xpar{Characteristics of Fragile and Resilient Networks (\cref{sec:architectures})} We characterize the resilience of networks based on their structural characteristics. We find that there are \bi{four unifying characteristics/global graph features that determine resilience}: \textit{(i)} $r$: the number of raw products, which corresponds to the size of the primary sector; \textit{(ii)} $c$: the number of final goods; \textit{(iii)} $m$: the sourcing dependency, which is the maximum number of inputs a product sources from; and \textit{(iv)} $\mu$: the supply dependency, which is the maximum number of outputs a specific product supplies. 

In the warm-up results, we study the resilience of stylized production networks that help demonstrate the interplay between these parameters. These structures and their parameterization are as follows.


\begin{table}[t]
    \centering
    \scriptsize
    \begin{tabular}{llll}
    \toprule
        \textbf{Topology} &\textbf{ Is the Network Resilient?} & \textbf{Main Result} & \textbf{Resilience} \\
    \midrule
        Random DAG & No & \cref{theorem:rdag_resilience} & $\Oeps { \left ( \frac 1 K \right )^{1/n}}$ \\ 
        Parallel Products & Yes & \cref{theorem:parallel_products} & $\Omegaeps {\left ( \frac {1} {\mu m} \right )^{1/n}}$ \\

        Backward network & No & \cref{theorem:tree_resilience,eq:quick} & $\Oeps { \left ( \frac {m \log K} {K}\right )^{1/n} }$ for $m \ge 2$ \\
        &  No & & $\Oeps {\left ( \frac {1} {K} \right )^{1/n}}$ for $m = 1$\\
        Forward network & Yes, with probability $\eta^*$ & \cref{theorem:gw_resilience} & See \cref{theorem:gw_resilience} \\
        Random Trellis & Yes, when $r = m = \mu = O(1)$ & \cref{informaltheorem:trellis} & $\Omegaeps {\left ( \frac {1} {r(1 + p)} \right )^{2/n}}$ \\
        & No, when $K / r = o(K^{1/3})$ & & $\Oeps {\left ( \frac {1} {\sqrt K} \right )^{1/n}}$ \\ \midrule
        Any Network $\cG$ & No if $r = \omega(K^{2/3})$ & \cref{theorem:resilience_graph_statistics} & $\Oeps {\left ( \frac {K} {r^{3/2}} \right )^{1/n}}$ \\
         & Yes, for $\mu$, $r$, and $c$ fixed & & $\Omegaeps {\left ( \frac {1} {(m + r) (\mu + c)}\right )^{1/n}}$ \\
     \bottomrule 
    \end{tabular}
    \caption{Summary of Results for the Homogeneous Cascade Model (see \cref{sec:contribution} for the definition of parameters for each network). The notation $\Omegaeps {\cdot}, \Oeps {\cdot}$ suppresses the dependence on $\varepsilon$.}
    \shrink
    \shrink
    \label{tab:resiliences}
\end{table}

\begin{compactenum}
    \item \emph{Random DAGs:} $K$ products ordered as $1,2,\ldots, K$ and connected with directed edge probability $p$ that respects their ordering; see \cref{sec:motivation} and
 \cref{fig:random_dag_2}. 
    \item \emph{Parallel products with dependencies:} A set of $r$ raw materials that are used to produce $K - r$ final goods, each complex product requires $m$ raw inputs (source dependencies), and each raw material is sourced by $\mu$ final goods (supply dependency); see \cref{sec:parallel_products} and \cref{fig:parallel_products}.
    \item \emph{Hierarchical production networks:} \emph{(i) Backward hierarchical production network:}  A tree production network with depth $D$ and fanout $m \ge 1$, i.e., each product has $m$ inputs independently of the other products. In this supply chain, the failures start from the raw materials we position at the tree's leaves, and the cascades grow from the leaves to the root (see \cref{sec:forward_backward_tree} and \cref{subfig:forward_backward}), and \emph{(ii) Forward hierarchical production network:} A tree production network generated by a branching process (also known as the Galton-Watson process) with branching distribution $\cD$ with mean $\mu$, which has (random) extinction time $\tau$, and extinction probability $\eta^*(\cD) = \Pr [\tau < \infty]$. In this regime, the percolation starts from the root node (raw material), and proceeds to the leaves (see \cref{sec:forward_forward_tree} and \cref{subfig:forward_forward}). 

\end{compactenum} 
 
\cref{tab:resiliences} and \cref{fig:resiliences_graphic} summarize our results for the resilience of the above architectures and the corresponding theorems where the results are proved. Namely, we show:

\begin{informaltheorem}
    Random DAG, backward production network, and random trellis with $m \ge 1$ and constant width are always fragile with $R_{\cG} (\varepsilon) \to 0$ as $K \to \infty$ for all $\varepsilon$. In contrast, parallel products with dependencies are resilient. The forward production network satisfies $\Pr \left [ R_{\cG}(\varepsilon) = 0 \right ] > 0$.
\end{informaltheorem}

In addition to identifying fragile architectures, our non-asymptotic analysis of $R_{\cG}(\varepsilon)$ reveals how various factors such as the number of raw materials $r$, the number of final goods $c$, and the sourcing dependency $\mu$ affect resilience. These nuances are fleshed out in \cref{sec:architectures}, as we present our results for each architecture in \cref{tab:resiliences}. Specifically, we show:

\begin{informaltheorem}
    For any network $\cG$, even $r > K^{2/3}$ raw products cause it to be fragile. In contrast, a network that has a fixed number of raw products $r$, source dependency $m$, supply dependency $\mu$, and number of final goods $c$ is always resilient. 
\end{informaltheorem}

Intuitively, fragile networks look like ``cowboy hats'', and resilient networks look like ``rolling pins'' (cf. \cref{fig:resiliences_graphic}). As a corollary of the previous result, we see that a trellis network with width $r$ and $D = K / r$ tiers with random edges generated independently with probability $p$ between tiers $d$ and $d + 1$ satisfies the following: 

\begin{informaltheorem} \label{informaltheorem:trellis}
    If the trellis has $D = o(K^{1/3})$ tiers, then it is fragile with resilience that goes to zero at the rate $\Oeps {\left ( \frac {1} {\sqrt K} \right )^{1/n}}$ as $K\to\infty$. On the other hand, if the number of raw products and the edge probability for the trellis are fixed ($r$ and $p$ are both $O(1)$), then the trellis is resilient and its resilience can be lower bounded by $\Omegaeps {\left ( \frac {1} {r(1 + p)} \right )^{2/n}}$ as $K\to\infty$.
\end{informaltheorem}

\xpar{Homogeneous Model with Inventory} In the sequel, we consider the capability of suppliers to hold inventories. In the simple case of the homogeneous model, if the suppliers of product $i$ have enough stock of product $j$, then they drop the dependency of $i$ to $j$ with probability $1 - y$ where $y \in (0, 1)$ is a quantity related to the available inventory. The question comes to evaluating the cascade size when inventories are available. As a first remark, this is equivalent to decreasing the sourcing dependency from $m$ to $my$.

However, if done with Monte Carlo methods, calculating the cascade size $F_y$ in a network is \#P-hard to compute \citep{chen2010scalable,borgs2014maximizing}. A way to efficiently estimate the size of the cascade $\ev {} {F_y}$ is through linear programming (LP). Using the union bound, we bound the cascade size $\ev {} {F_y}$ by a linear program that takes into account the failure probabilities $x$, the number of suppliers, the network structure, and the inventory (cf. \cref{theorem:general_ub}). We show that if $y$ is sufficiently small, then the cascade size is well approximated by a linear program:

\begin{informaltheorem}\label{informal_theorem:LP}
    If $y = \varrho/m$ for some $\varrho \in (0, 1)$, then there exists an LP, whose optimal value $p_\cG^*$ satisfies $\ev {} {F_y} \le p_\cG^* \le \left ( 1 + \varrho + O(\varrho^2) \right ) \cdot \ev {} {F_y}$. 
\end{informaltheorem}


The linear program we give can be used to give an $(1 + \varrho + O(\varrho^2))$ approximation to the cascade size ($\ev {} {F_y}$) which can be computed in almost linear time $\tilde O(K + M)$ for small infection probabilities $y = \varrho / m$ for $\varrho \in (0, 1)$. This matches the lower bound $\Omega(K + M)$ in runtime for maximum influence (up to logarithmic factors); see \cite{borgs2014maximizing}. This constitutes a novel result and connection to the independent cascade literature, perhaps of independent interest, to the best of our knowledge. 

Our LP resembles those used in the literature on systemic risk and financial contagion (see, e.g., \cite{eisenberg2001systemic,glasserman2015likely}), and it is also related to the literature on influence maximization \citep{borgs2014maximizing,kempe2003maximizing}. With the LP framework, we establish a promising connection between cascading failures in supply chains and financial contagion, as well as the independent cascade model. On the other hand, we show that if the inventory is small ($y > 1/m$), there exist instances where the LP can be as far away as an $\Omega(K)$ additive factor from $\ev {} {F}$ (Appendix \ref{app:innaproximability}). Moreover, the dual LP of $p_\cG^*$ corresponds to a systemic risk measure, as axiomatized in \cite{chen2013axiomatic} and subsequent works. 

Finally, the LP formulation can be used to identify ``vulnerable'' products -- i.e., products that are more likely to fail than others, and subsequently design interventions in supply chains (see \cref{sec:interventions}) based on Katz centrality. We show that under certain assumptions, the probability that each node fails is related to its Katz centrality \citep{katz1953new}, which is consistent with the existing literature on financial networks \citep{siebenbrunner2018clearing,bartesaghi2020risk}. Furthermore, we establish a generic lower bound on $R_{\cG}(\varepsilon)$ and an optimal intervention policy (under assumptions) as follows:
 
 \begin{informaltheorem} \label{informal_theorem:intervention}
     \begin{compactenum}
         \item For $y < 1/m$, $R_{\cG}(\varepsilon) \ge \left (\frac {\varepsilon} {\one^T \beta_\cG^\katz (y)} \right )^{1/n}$, where $\beta_\cG^\katz (y) = (I - y A^T)^{-1} \one$ is the Katz centrality of $\cG$. 
         \item If a planner can protect up to $T$ products, then the optimal intervention of the planner is to protect the $T$ products in decreasing order of Katz centrality in $\cG^R$, i.e., $\gamma_\cG^\katz (y) = (I - y A)^{-1} \one$ for $0 < x < (1 - \mu y)^{1/n}$. If $x \ge (1 - \mu y)^{1/n}$ the optimization problem can be solved in poly-time via leveraging the Lov\'asz extension of $p^*_{\cG}$ subject to the intervention constraints. 
     \end{compactenum}
     
 \end{informaltheorem}
 
 Therefore, with sufficiently large inventories ($y < 1/m$) and sufficiently small shocks ($x \ge (1 - \mu y)^{1/n}$), the Katz centrality of each node in the production network can inform the social planner's effort to design contagion mitigation strategies to improve supply chain resilience. In the latter case, our approximation algorithm leverages the LP's submodularity and cascade size approximation. 

\xpar{Connections to Risk Exposure Index (\cref{sec:rei})} To establish connections with existing measures of supply chain risk in the literature, we adopt the definition of Risk Exposure Index (REI) by \citet{levi2016identifying} to our model. Accordingly, REI for a product $i$ is defined as the maximum potential impact of a given product (node), measured by the change (derivative) of the cascade size subject to a shock at node $i$. We show that 

\begin{informaltheorem}
    Under reasonable assumptions, a network's REI is proportional to its nodes' highest Katz centrality.
\end{informaltheorem}

\xpar{Empirical Results (\cref{sec:experiments})} We experiment with real-world supply chain networks from \citet{willems2008data} and the World Input-Output database from \citet{timmer2015illustrated}. In the former case, the production networks are DAGs, whereas in the latter case, the networks correspond to countries' economies and can have cycles. We use Monte Carlo simulations to numerically determine the average resilience ($\Ravgsamp {\cG} = \int_0^1 \hat R_{\cG}(\varepsilon) d \varepsilon$) in the multi-echelon networks of \cite{willems2008data}, and show that the average resilience agrees with the derived upper bound of \cref{theorem:resilience_graph_statistics}: 

\begin{observation}
    $\Ravgsamp {\cG} \propto \frac {K^{\alpha_1}} {r^{\beta_1}}$ for some $\alpha_1, \beta_1 > 0$ empirically determined (p-values $< 0.01$) by regressing $\Ravgsamp {\cG}$ calculated from the supply chain data set of \cite{willems2008data}.
\end{observation}

We also relate the average REI, that is, $\reiavg {\cG} = \int_0^1 \rei_{\cG}(x) dx$, to $\Ravg {\cG}$, $m$, and $K$ using regression. We show: 

\begin{observation}
    In the large inventory regime of Informal Theorems \ref{informal_theorem:LP} and \ref{informal_theorem:intervention}, $\reiavg {\cG} \propto \frac {K^{\alpha_2} m^{\beta_2}} {(\Ravg {\cG})^{\gamma_2}}$ where $\alpha_2$, $\beta_2$, and $\gamma_2 > 0$ are empirically determined (p-values $< 0.01$) using regression from values calculated in the supply chain data set of \cite{willems2008data}.
\end{observation}

\xpar{Heterogeneous Model (\cref{sec:heterogeneities})} We generalize our analysis to a heterogeneous model, where each product has a different number of suppliers and supplier failures may be correlated. We generalize the definition of $R_{\cG}(\varepsilon)$ as the largest value $x$ such that under the worst-case joint distribution of failures, $\nu$, with the average number of failures upper bounded by $x$, at least $(1 - \varepsilon)$ fraction of the products survive with probability at least $1/K$ over the supplier failures sampled from their joint distribution $\nu$.

We give lower bounds by extending the aforementioned LP technique and, moreover, show how to have low-dimensional representations of the marginals of $\nu$ by leveraging the Bahadur representation \citep{bahadur1961representation}, according to which we express the probability of failure of a product $i$ as a function of higher-order correlations. This allows us to establish upper and lower bounds for resilience with correlations (\cref{prop:correlation}). Finally, we empirically show how resilience is affected by correlations in real-world data sets of \cite{willems2008data}.

\subsection{Main Related Work} We model the production network as a graph that undergoes a percolation process on its nodes, representing products. Each product has some suppliers, who, if they all fail, then the product cannot be produced. Subsequently, a cascade is caused by such a failure. Our modeling decision to study the supply network as a graph that undergoes percolation has its roots in previous work (see the referenced works in Appendix \ref{app:further_related_work}); the most closely related is the work of \citet{elliott2022supply}. 

\xpar{Comparison with the results of \texorpdfstring{\citet{elliott2022supply}}{elliott2022supply}}\label{sec:comparison_golub} 
\citet{elliott2022supply} consider a \emph{hierarchical production network} that undergoes a \emph{link percolation process}. Specifically, firms are operational if all their inputs have operating links to at least one firm producing the specific inputs. \citet{elliott2022supply} study the \emph{reliability of the supply network} as the probability that the root product is produced. They show that three important factors affect reliability: \emph{(i)} the depth/size of the supply chain, \emph{(ii)} the number of inputs that each product requires, and \emph{(iii)} the number of suppliers (which corresponds to the ability of firms to multisource their required inputs). We investigate the effect of similar structural factors -- i.e., topology, number of inputs, and number of suppliers -- under a fundamentally expanded setup by \emph{(i)} considering production shocks at the level of individual suppliers rather than links; \emph{(ii)} considering more general architectures than the hierarchical production networks of \citet{elliott2022supply}, and \emph{(iii)} studying a novel theory-informed resilience metric that guarantees that almost all products can be produced in the event of a supply chain shock. One of our main results shows that production networks are either resilient or fragile. \citet{elliott2022supply} show that when the magnitude of the systemic shock is greater than some critical value, the reliability goes to zero, corresponding to \emph{fragile} networks. When the magnitude of the shocks is below this critical value, the reliability attains a strictly positive value corresponding to the \emph{resilient} networks. They observe that reliability increases with multisourcing (having many suppliers)  and decreases with interdependency (requiring a multitude of inputs). Our analysis differs from \citet{elliott2022supply}. It focuses on determining the largest possible shock that a network can withstand so that a significant fraction of products are produced (survive) in the event of such a shock as the size of the network increases. Our networks are similarly parametrized by size, multi-sourcing, and interdependency parameters. In agreement with \citet{elliott2022supply}, we show that networks become more resilient when multi-sourcing increases, and when interdependencies among products increase, networks become less resilient.

Our model and resilience metric complement the results of \cite{elliott2022supply} as our metric, model, and results provide tools for studying any production network (in contrast to the tree model presented in \cite{elliott2022supply}). Moreover, we show two stronger premises for determining the resilience of supply chain networks with global features: Our first result says that \textit{even a sublinear number of raw materials on any graph (not necessarily tree-like) can cause fragility}, which agrees and generalizes the observations on the tree model of \cite{elliott2022supply}. Moreover, \textit{ spare networks with a few raw and final goods are resilient}, as our theory suggests, complementing their theory. In addition, we model the scenario where suppliers have inventories and show that under sufficient inventory capacity, the cascade size is closely connected to financial contagion models and experiences a phase transition. Specifically, we show that there is a critical value $y_{\mathrm{crit}}$ where for large enough inventories, which correspond to $y < y_{\mathrm{crit}})$, the cascade can be well approximated by solving a linear program; however, above this critical threshold (i.e., $y > y_{\mathrm{crit}}$), there are instances where approximating the cascade size becomes impossible. Conceptually, this phase transition is complementary to the phase transition of \cite{elliott2022supply}, as $y$ would be related to their notion of the strength of the relationship between suppliers. In contrast, our results focus on characterizing the maximum shock probability (resilience) by approximating the cascade size via an LP. 

We observe that certain types of networks are resilient, i.e., they can withstand sufficiently large shocks without experiencing catastrophic cascading failures, and others are fragile, i.e., non-trivial shocks are enough to render a significant proportion of their products unproducible. We provide additional evidence in the form of the emergence of power laws for the size of cascading failures, which further motivates the study of resilience in complex production networks. Compared to existing indices that combine many temporal, financial, and economic risk factors \citep{gao2019disruption}, our resilience metric identifies topological risk factors and focuses on cascading failures to identify fragile and resilient supply chain architectures. \citet{perera2017network} systematically reviews the literature on modeling the topology and robustness of production networks with applications to real-world data. Their first observation is that many real-world production networks follow a power-law degree distribution with an exponent of around two.

Moreover, the resilience (called ``robustness''  in their work) of such networks is defined as the size of the largest connected component or the average/max path length in the presence of random failures. In contrast, we propose a novel, theory-informed resilience metric that complements the existing measures in \citet{perera2017network} and references therein and supplement their empirical work with novel theories about how topological attributes contribute to the increased risk of cascading failures in production networks. We also test our proposed metric on real-world data and provide empirical insights comparable to \citet{perera2017network}. Finally, we propose interventions to improve resilience with theoretical guarantees similar to optimal allocations in the presence of shocks and financial risk contagion \citep{eisenberg2001systemic, papachristou2021allocating,blume2013network,erol2022network,bimpikis2019supply}, and complementary to other supply chain interventions that, for example, increase visibility and traceability \citep{blaettchen2021traceability}, or design optimal investments \citep{chen2023when}. Appendix \ref{app:further_related_work} provides an extensive review of the literature.


% \subsection{Paper Organization} \label{sec:paper_organization}

% In the next section, we use the context described above to summarize our main contributions. Next, in \cref{sec:preliminaries}, we introduce some preliminaries that we use to motivate our proposed measure of resilience in \cref{sec:motivation}. In \cref{sec:architectures}, we present our new resilience metric formally and study it in various architectures -- see also \cref{tab:resiliences} in the next subsection, which lists our main results for different architectures, under a homogeneous model of disruption. In \cref{sec:general_supply_chains}, we present a general computational framework for lower bounding the proposed metric in general graphs and make additional connections to financial networks, financial contagions and centrality measures, as well as propose algorithms for optimal interventions in supply chains. In \cref{sec:experiments}, we provide empirical results using real-world supply chain data, where we numerically compute resilience and design intervention policies. Then, in \cref{sec:heterogeneities}, we generalize the homogeneous model to a heterogeneous model, which encodes heterogeneous supply relationships and correlations among sectors and suppliers. Finally, in \cref{sec:discussion}, we summarize our main managerial insights and provide conclusions and potential avenues for future work in \cref{sec:conc}.