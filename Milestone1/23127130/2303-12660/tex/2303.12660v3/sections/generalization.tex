\section{The Heterogeneous Model} \label{sec:heterogeneities}

% \subsection{General Definition} \label{sec:general_resilience_defintion}

So far, for simplicity of exposition and to be able to derive meaningful bounds, we have assumed that each product has $n$ suppliers and that each supplier fails i.i.d. with probability $x$. In this section, we extend our proposed model and the resilience metric to account for heterogeneities, such as a different number of suppliers and different failure probabilities. 


In our heterogeneous model, each product has $n_i = |\cS(i)| = O(1)$ suppliers. We define the universe of suppliers as $\cU = \bigcup_{i \in \cK} \cS(i)$ and $N = |\cU|$, and each supplier $s \in \cU$ fails with probability $x_s = \Pr [X_s = 1]$. Note that here suppliers' failures are not assumed to be i.i.d. but are correlated according to a joint distribution $\nu$ with marginals $\{ x_s \}_{s \in \cU}$. 

The resilience metric for a joint distribution $\nu$ is defined as the maximum shock $x \in (0, 1)$ such that: \emph{(i)} there are on average $\sum_{s \in \cU} x_{s} = x N$ failures, \emph{(ii)} at least $1 - \varepsilon$ of the products survive with high probability assuming that the failures of the suppliers ($F_\cS$) are distributed according to $\nu$ (i.e., $F_\cS \sim \nu$). The resilience of the graph is taken to be the resilience over the worst possible such joint distribution, i.e.: 

\begin{tcolorbox}
\vspace{-20pt}
\begin{align} \label{eq:resilience_general}
    R_{\cG}(\varepsilon) & = \inf_{\nu} \sup_{x} x  \quad \text{s.t.} \quad  \sum_{s \in \cU} x_{s} \le x N\; \text{and} \; \Pr_{F_\cS \sim \nu} \left [ F \ge \varepsilon K \right ] \le \frac {1} {K}. 
\end{align}
\end{tcolorbox}


First, we can show that if $x_s \in \{ 0, 1 \}$, i.e., the marginals are deterministic, then calculating $R_\cG(\varepsilon)$ is NP-Hard and is also hard to approximate (see \cref{app:generalized_resilience}). Even when the marginals are fractional, numerically evaluating resilience is computationally intractable, as one has to search for all possible joint distributions and calculate the resilience for each joint distribution. Therefore, we are interested in efficiently computable upper and lower bounds. 

\subsection{The Bahadur Representation} \label{sec:general_resilience_lower_bound}

Regarding the upper bound, since the definition of resilience according to \cref{eq:resilience_general} considers the worst possible joint distribution, the generalized resilience value is upper bounded by resilience in the case of i.i.d. failures, and the upper bound of \cref{theorem:resilience_graph_statistics} holds for the general definition of resilience.

To devise a lower bound, since Markov's inequality depends on $\ev {} {F}$, maximizing this quantity implies a lower bound on the resilience. However, such an optimization program has $O(K)$ variables and $O(2^N)$ constraints, making it impractical to solve. 


% As we show in \cref{sec:general_supply_chains}, we can devise lower bounds to the resilience by considering an upper bound on the number of failures that resembles the optimization problem defined in \cref{eq:dual_upper_bound_lp}. In the sequel, we can use the upper bound devised in \cref{app:generalized_resilience} and LP duality to devise a lower bound on the resilience: 
 
% \begin{theorem} \label{theorem:general_resilience}
%     If the resilience is defined as in \cref{eq:resilience_general}, then a lower bound to the resilience can be found by solving the following optimization problem with $O(K)$ variables and $O(2^N)$ constraints: 

%     \begin{align*}
%     \underline R_{\cG}(\varepsilon; y)  = \max_{\gamma, \kappa, \xi \ge \zero}  \quad & \frac {\varepsilon} {\xi N} \\
%     \text{s.t.} \quad & (I - yA) \gamma \ge \one \nonumber,\; - \kappa + \one \xi \ge \zero, \;- \Psi^T \gamma + \Phi^T \kappa \ge \zero. \nonumber
%     \end{align*}

%     Where $\Psi, \Phi$ are matrices given in \cref{app:generalized_resilience}. 

% \end{theorem}

% \subsection{Low-Dimensional Representations of the Failure Distribution}

In the sequel, we focus on low-dimensional representations of $\nu$. Our analysis is based on the Bahadur decomposition \citep{bahadur1961representation,yuan2021community}, which gives a way to calculate the joint probability distribution of failures as a sum of multi-way correlations. According to the Bahadur representation, the joint distribution $\nu$ of active suppliers can be written. 

{\small
\begin{align} \label{eq:bahadur_expansion}
    \nu(T) & = \prod_{s \in T} x_s \prod_{s \notin T} (1 - x_s) \left \{ \sum_{j = 1}^N \sum_{\{ s_1, \dots, s_j \} \in \binom {\cU} j} \rho_{s_1, \dots, s_j} (T) \prod_{j' = 1}^j \hat x_{s_j'}(T) \right \}
\end{align}}

where $\hat x_{s_j} = \frac {\one \{ s_j \in T \} - x_{s_j}} {\sqrt {x_{s_j} (1 - x_{s_j})}}$, $\rho_{s_1, \dots, s_j}(T)$ denotes the $j$-th order correlation between $s_1, \dots, s_j$, and $T \subseteq \cU$ denotes the set of active suppliers. From this we can obtain the marginal for each product $i \in \cK$ as $u_i = \sum_{T \subseteq \cU \setminus \cS(i)} \nu(\cS(i) \cup T)$. 

To obtain a low-dimensional representation of $\nu$, we can make some assumptions that simplify our calculations and provide meaningful ways to calculate resilience. In the following, we give some simplified models that account for correlations between suppliers within each product, across products, and across all suppliers. 

\xpar{Homogeneous failures, uncorrelated products, and correlated suppliers} The first simplification of \cref{eq:bahadur_expansion} considers the case of homogeneous failures and correlated suppliers within a product (but not across products), which requires $O(N)$ parameters. This model accounts for failures within a product or a sector in the economy; for instance, a failure of a supplier of a specific scarce raw material would harm the failure of other suppliers of the same raw product since the failure of a supplier can increase the demand of the material from another supplier, and, thus, increase its failure probability. 

Thus, if $x_s = x$ for all suppliers $s \in \cU$, and the $j$-th order correlation coefficient for product $i$ is $\rho_{ij} \in [0, 1]$ for all $j$-tuples, we can write the joint probability of failure of a product as 


{\small
\begin{align} \label{eq:correlated_suppliers}
   u_i = x^{n_i} + \sum_{j = 2}^{n_i} \binom {n_i} {j} \rho_{ij} (1 - x)^{j/2} x^{n_i - j/2}
\end{align}}

As we expect, increasing $\rho_{ij}$ to $+1$ increases the probability of failure of one product and therefore decreases the resilience. Similarly, bringing $\rho_{ij}$ equal to 0 makes the failure probability equal to $x^{n_i}$, which reduces to the previously studied case of independent failures. When $\rho_{ij} = 1$, the problem is equivalent to each product having one supplier instead of $n_i$ suppliers and thus retains its resilience properties; that is, for network classes that are indexed by their number of products $K$ their asymptotic in $K\to\infty$ classification as resilient or fragile (i.e., their resilience taxonomy) remains the same with the introduction correlations between suppliers of the same product (recall \Cref{def:resilience-taxonomy}). This is no longer the same when correlations are introduced between products; then all networks are, in general, fragile.  

\xpar{Homogeneous failures, correlated products, and uncorrelated suppliers} The second simplification considers the case of homogeneous failures and correlated products but uncorrelated suppliers within a product, which requires $O(K)$ parameters. This case models a scenario where a natural disaster can affect multiple sectors, even if they are not dependent on each other by import/export relations. 

Thus, if we assume that all suppliers $s \in \cU$ fail with probability $x_s = x$ independently within each product, but products are correlated with each other, with the $j$-th order correlation coefficient being $\rho_j \in [0, 1]$. Then, we can show that the marginal probability of one product's failure is

{\small
\begin{align} \label{eq:correlated_products}
    u_i = \sum_{b = 0}^{K - 1} \binom {K - 1} {b} \left ( x^{n_i} \right )^{b + 1} \left ( 1 - x^{n_i} \right )^{K - 1 - b} \left \{ 1 + \sum_{j = 2}^K \sum_{r = 0}^{\min \{j, b + 1 \}} \binom j r \binom {K - j} {j - r} \rho_j \left (1 - x^{n_i} \right )^{r - j/2} \left  ( x^{n_i} \right )^{j/2 - r}  \right \}
\end{align}}

When $\rho_j = 1$, the failure of a single product is enough to have $F \ge \varepsilon K$ for $\varepsilon \in (0, 1 - 1 / K)$. By the union bound, this happens with probability at most $\sum_{i = 1}^K x^{n_i} \le Kx$, and setting the probability of this event to be at most $1 / K$, yields the result that the average resilience is dropping at an $\Oeps {1 / K^2}$ rate, and therefore the network becomes fragile. 

\xpar{Homogeneous failures, correlated suppliers and products} 
The final simplification considers the case of homogeneous failures and correlations among all suppliers, which requires $O(N)$ parameters. This case models the scenario where a firm produces more than one product; for example, an automotive manufacturer produces engines (which can be exported to others) and chips using the same manufacturing facilities. Then a disaster in the manufacturing facility (e.g., a fire) would damage both products.

By extending \cref{eq:correlated_products}, we can state a similar result for the case where all suppliers are correlated, and the $j$-th order correlation coefficient is $\rho_j \in [0, 1]$. Similarly to \cref{eq:correlated_products}, we can show that the marginal probability of failure of one product is 

{\small
\begin{align} \label{eq:joint_correlations}
        u_i = \sum_{b = 0}^{N - n_i} \binom {N - n_i} {b} x^{b + n_i} \left ( 1 - x \right )^{N - n_i - b} \left \{ 1 + \sum_{j = 2}^N \sum_{r = 0}^{\min \{j, b + n_i \}} \binom j r \binom {N - j} {j - r} \rho_j \left (1 - x \right )^{r - j/2}  x^{j/2 - r}  \right \}
\end{align}}

We can verify that setting $n_i = 1$ (which implies $N = K$) yields \cref{eq:correlated_products}, and setting $K = 1$ yields \cref{eq:correlated_suppliers}. Furthermore, in the extreme case where $\rho_j = 1$, the failure of a single supplier is enough to make $F \ge (1 - \varepsilon) K$. By the union bound, this happens with probability at most $Nx$, and setting this to be at most $1 / K$, shows that the average resilience is dropping as $\Oeps {1 / K^2}$. Therefore, the network becomes fragile. 

\xpar{Bounds based on the Bahadur representation} To obtain bounds on the resilience assuming the simplified marginals (e.g.  \cref{eq:joint_correlations,eq:correlated_suppliers,eq:correlated_products}) we can directly apply the results of \cref{theorem:resilience_graph_statistics} and \cref{theorem:lp_duality_resilience} and invert the corresponding marginals (cf. \cref{fig:inverse_marginal}):

\begin{proposition} \label{prop:correlation}
    Let $\cG$ be a network where suppliers fail with a correlation vector $\rho$, which corresponds to the marginal failure probabilities $u_i = u(x; \rho)$ which are monotone, increasing in $x$, let $y < 1/m$, and $n_i = n$. Then, the resilience $R_{\cG}(\varepsilon; y, \rho)$ satisfies $R_{\cG}(\varepsilon; y, \rho) \ge u^{-1} \left ( \frac {\varepsilon} {\one^T \beta_{\cG}^{\katz}(y)} ; \rho \right )^{1/n}$ and $R_{\cG}(\varepsilon; y, \rho) \le u^{-1} \left ( \frac {(1 - \varepsilon) K} {2r^{3/2} + \sqrt {r \log K}} ; \rho \right )^{1/n}$. 
\end{proposition}

\xpar{An example from empirical networks} \cref{fig:correlations} shows the impact of introducing correlations to the resilience. Specifically, we see that introducing correlations decreases resilience, with the biggest impact in the case where correlations is assumed among all pairs of suppliers (correlated products and suppliers). 


\begin{figure}
    \centering
    \includegraphics[width=\linewidth]{figures/results_correlation_msom_willems.pdf}
    \caption{Effect of correlation on $R_\cG(\varepsilon)$ for three networks from \cite{willems2008data}. We consider the cases studied in \cref{eq:correlated_products,eq:correlated_suppliers,eq:joint_correlations}. The 2n-th order correlation coefficient has been set to $\rho_2 = \rho \in \{ 0, 0.25, 0.5, 0.75, 1 \}$. The higher-order correlations have been set to zero.}
    \label{fig:correlations}
\end{figure}

\begin{figure}
    \centering
    \includegraphics[width=\linewidth]{figures/bahadur_correlated.pdf}
    \caption{Value of inverted marginal $u^{-1}(x; \rho)$ for the cases studied in \cref{eq:correlated_products,eq:correlated_suppliers,eq:joint_correlations}. All correlations have been set equal to $\rho \in \{ 0.25, 0.5, 0.75, 1 \}$. The network is assumed to have $K = 2$ products and each product has $n_i = n = 2$ suppliers.}
    \label{fig:inverse_marginal}
\end{figure}


