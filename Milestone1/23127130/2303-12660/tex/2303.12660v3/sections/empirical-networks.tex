
\subsection{Resilience of Empirical Production Networks} \label{sec:experiments}

% \begin{figure}
%     \centering
%     \includegraphics[width=0.5\textwidth]{figures/statistics.pdf}
%     \caption{Network statistics for the three networks studied in \citet{willems2008data}}
%     \label{fig:msom_willems_statistics}
% \end{figure}

With the definition of resilience and the theoretical results developed in the previous sections, we study the resilience of production networks in practice. We examine the resilience of networks contained in the following two data sets of production networks (\cref{tab:statistics-willems,tab:statistics-world}):  

\begin{compactenum}
    \item \emph{Multi-echelon Supply-chain Networks} from \citet{willems2008data}. The dataset contains 38 different multi-echelon (see also \cref{fig:supply_chain}) supply chain networks, from which we select three networks to run simulations on.  Similar supply chain networks have been used in prior literature, see, e.g., \citet{blaettchen2021traceability} and \citet{perera2017network}. 
    \item \emph{Country Economy Production Networks} derived from \emph{World Input-Output Tables} taken from the World Input-Output Database \citep{timmer2015illustrated}. We focus on the economies of six countries in 2014: the USA, Japan, China, Great Britain, Indonesia, and India. We consider any non-zero amount cell at the input-output tables as an edge between two industries in a country. For space considerations, the results have been deferred to Appendix \ref{app:experiments_addendum}. 
\end{compactenum}

% \begin{table}[H]
%     \footnotesize
%     \centering
%     \begin{tabular}{lllllll}
%     \toprule
%          Network ID  & Size ($K$) & Avg. Degree & Density & Min/Max In-degree & Min/Max Out-degree  & AUC  \\
%     \midrule 
%     \multicolumn{7}{c}{Networks from \citet{willems2008data}} \\
%     \midrule
%          \#10 & 58 & 3.03 & 0.053 & 0 -- 27 & 0 -- 13  & 0.136 \\
%          \#20 & 156 & 1.08 & 0.006 & 0 -- 29 & 0 -- 3  & 0.117 \\
%          \#30 & 626 & 1.00 & 0.001 & 0 -- 2 & 0 -- 48  & 0.357 \\ 
%     \midrule
%     \multicolumn{7}{c}{World I-O Tables} \\
%     \midrule
%          USA & 55 & 54.00 & 1.000 & 54 & 54 & 0.052 \\
%          Japan & 56 & 45.33 & 0.824 & 0 -- 50 & 0 -- 50  & 0.058 \\
%          G. Britain & 56 & 52.05 & 0.946 & 0 -- 54 & 0 -- 54  & 0.052  \\
%          China & 56 & 37.89 & 0.688 & 0 -- 46 & 0 -- 46  & 0.078 \\
%          Indonesia & 56 & 35.75 & 0.65 & 0 -- 46 & 0 -- 46  & 0.078 \\
%          India & 56 & 29.57 & 0.537 & 0 -- 41 & 0 -- 43 & 0.095 \\
%     \bottomrule
%     \end{tabular}
%     \caption{Network Statistics and AUC. The edge density is computed as $\frac {|\cE(\cG)|} {K^2 - K}$.}
%     \label{tab:statistics}
% \end{table}


\begin{table}[!htbp] \centering
\scriptsize
\begin{tabular}{@{\extracolsep{5pt}}lc}
\\[-1.8ex]\hline
\hline \\[-1.8ex]
& \multicolumn{1}{c}{\textit{Dependent variable: log-$\Ravg {\cG} (y)$}} \
\cr \cline{2-2}
\hline \\[-1.8ex]
 const & -1.757$^{***}$ (0.045) \\
 log-$K$ & 0.146$^{***}$ (0.012) \\
 log-$r$ & -0.029$^{***}$ (0.010) \\
\hline \\[-1.8ex]
 Observations & 34 \\
 $R^2$ & 0.871 \\
 Adjusted $R^2$ & 0.862 \\
 Residual Std. Error & 0.067 ($df=31$) \\
 $F$ Statistic & 104.385$^{***}$ ($df=2; 31$) \\
\hline
\hline \\[-1.8ex]
\textit{Note:} & \multicolumn{1}{r}{$^{*}p<0.1; ^{**}p<0.05; ^{***}p<0.01$} \\
\hline \\
\end{tabular}
\caption{Relation between the $\Ravg {\cG} (y)$, $K$ and $r$ for the multi-echelon networks of \cite{willems2008data}. For each network, we set $y = 1 / (m + 10^{-5})$.}
\label{tab:resilience_auc}
\shrink
\end{table}


\xpar{Resilience metrics} First, we study the resilience of the above networks as a function of $\varepsilon$ for $\varepsilon$ ranging from 0 to 1 numerically. To achieve this, we run $1000$ Monte Carlo (MC) simulations where we sample the (spontaneous) state of $n$  suppliers and then propagate the state of each product to the adjacent ones, based on \cref{eq:dynamics}. To calculate resilience, we estimate the probability $\Pr [S \ge (1 - \varepsilon) K]$ for various values of $x \in (0, 1)$ with MC simulation and find the maximum value of $x$ for which the estimate is at least $1 - 1 / K$. We plot the estimated resilience $\hat R_{\cG}(\varepsilon)$ as a function of $\varepsilon$. and present the results in \cref{subfig:willems_resilience,subfig:world_io_tables_resilience}. As an additional resilience metric that's independent of $\varepsilon$, we also report the average resilience $\Ravgsamp {\cG}(y) = \int_0^1 \hat R_{\cG}(\varepsilon; y) d \varepsilon$.  

Furthermore, we perform a regression to relate $\Ravgsamp {\cG}(y)$  with the number of raw products ($r$) and the size of the network ($K$), and report the results in \cref{tab:resilience_auc}. We find that $\Ravgsamp {\cG} (y)$ increases in $K$ ($p < 0.01$) and decreases with $r$ ($p < 0.01$). This agrees with our theoretical results (e.g., \Cref{theorem:resilience_graph_statistics}) which highlight that networks with more raw products have lower resilience.

\xpar{Optimal interventions} To study the effect of targeted interventions in real-world supply chain networks, we apply the results of \cref{prop:intervention} for the networks from the two datasets. More specifically, for a value of $y = \frac {1} {10^{-5} + \mu}$, and $\varepsilon = 0.2$, we plot the lower bound for the resilience devised by \cref{prop:intervention} as a function of the total intervention budget $T$. This involves calculating the Katz centralities for the reverse graph $\cG^R$, sorting them in decreasing order, and plotting the cumulative sum for the first $T$ entries, for $T$ ranging from $0$ to $K$. In \cref{subfig:willems_interventions,subfig:world_io_tables_interventions}, we report the lower bound on the resilience as a function of the normalized intervention budget $T / K$.


\xpar{Empirical relation between REI and our resilience metric} In \cref{tab:resilience_ttr}, we show the relationship between $\reiavg {\cG}(y) = \int_0^1 \rei_{\cG}(x; y) dx \approx \left \| \gamma_{\cG}^\katz (y) \right \|_{\infty}$) and $\Ravg {\cG}(y)$ and the maximum degree $m$ of the network. We observe that $\reiavg {\cG} (y)$ is inversely proportional to $\Ravg {\cG}(y)$ ($p < 0.01$) showing that less resilient networks (on average) are ``riskier'' (on average) and proportional to $m$ ($p < 0.01$) and $K$ ($p < 0.01$), showing that bigger and denser networks are ``riskier'' (on average).

\begin{table}[!htbp] \centering
\scriptsize
\begin{tabular}{@{\extracolsep{5pt}}lc}
\\[-1.8ex]\hline
\hline \\[-1.8ex]
& \multicolumn{1}{c}{\textit{Dependent variable: log-$\reiavg {\cG} (y)$}} \
\cr \cline{2-2}
\hline \\[-1.8ex]
 const & -14.782$^{***}$ (2.817) \\
 log-$m$ & 0.540$^{***}$ (0.167) \\
 log-$K$ & 0.932$^{***}$ (0.259) \\
 log-$\Ravg {\cG} (y)$ & -8.828$^{***}$ (1.589) \\
\hline \\[-1.8ex]
 Observations & 34 \\
 $R^2$ & 0.678 \\
 Adjusted $R^2$ & 0.646 \\
 Residual Std. Error & 0.629 ($df=30$) \\
 $F$ Statistic & 21.089$^{***}$ ($df=3; 30$) \\
\hline
\hline \\[-1.8ex]
\textit{Note:} & \multicolumn{1}{r}{$^{*}p<0.1; ^{**}p<0.05; ^{***}p<0.01$} \\
\hline
\end{tabular}
\caption{Relation between $\log \reiavg {\cG} (y)$ and $\log m$, $\log K$ and $\log \Ravg {\cG} (y)$ for the multi-echelon networks of \cite{willems2008data}. For each network, we set $y = 1 / (m + 10^{-5})$.} 
\label{tab:resilience_ttr}
\shrink
\end{table}



% \paragraph{Insights.} In the multi-echelon networks, we observe that intervening in less resilient networks (see \cref{fig:monte_carlo_resilience}), the more the interventions increase, the lower bound in the resilience. In the World I-O tables dataset, we observe that the interventions mostly improve the lower bound on the resilience for the sparser networks (which correspond to less developed economies).  


\begin{table}
    \scriptsize
    \centering
    \begin{tabular}{lllllll}
    \toprule
         Network ID  & Size ($K$) & Avg. Degree & Density ($\frac {|\cE(\cG)|} {K^2 - K}$) & $\mu$ & $m$  & $\Ravgsamp {\cG}$  \\
    \midrule 
    % \multicolumn{7}{c}{Networks from \citet{willems2008data}} \\
    % \midrule
         \#10 & 58 & 3.03 & 0.053 & 27 & 13  & 0.136 \\
         \#20 & 156 & 1.08 & 0.006 & 29 & 3  & 0.117 \\
         \#30 & 626 & 1.00 & 0.001 & 2 & 48  & 0.357 \\ 
    \midrule
    \end{tabular}
    \caption{Network Statistics and $\Ravgsamp {\cG}(y)$ from \citet{willems2008data}. }
    \shrink
    \label{tab:statistics-willems}
\end{table}
\begin{figure}
    \centering
    \subfigure[Estimating $\hat R_{\cG}(\varepsilon)$ and $\Ravgsamp {\cG} (\varepsilon)$ \label{subfig:willems_resilience}]{\includegraphics[width=0.49\textwidth]{figures/resilience_monte_carlo_vs_eps_msom_willems.pdf}}
    \subfigure[Optimal Interventions\label{subfig:willems_interventions}]{\includegraphics[width=0.49\textwidth]{figures/resilience_lb_vs_key_msom_willems.pdf}}
    \caption{Resilience estimation and optimal interventions for three networks from \citet{willems2008data}. We set the number of suppliers for each product to $n = 1$.}
    \shrink
    \label{fig:willems}
\end{figure}



