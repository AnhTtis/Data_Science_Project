
\section{Resilience of General Graphs and Optimal Interventions} \label{sec:general_supply_chains}

% \mpcomment{move discussion about circular economies to the discussion section, talk only about cycles in supply chains}

So far, we have focused our attention on the cases where $\cG$ is a DAG and under the assumption that the network undergoes a node percolation process. It is certainly possible for a production network to have cycles, e.g., when complex products are used in the production of simpler products. It is also possible that production networks have cycles which represent recycling or other forms of circular flows of materials or resources in modern economies; cf., circular economies  \citep{geissdoerfer2017circular}.
% such as in the case of circular economies \citep{geissdoerfer2017circular,kirchherr2017conceptualizing}. In some cases, production networks can have cycles, which can represent recycling or other forms of circular flows of materials or resources. 
In such cases, it is important to consider the impact of percolation on these cycles and the overall structure of the network. One way to analyze the effect of percolation on a network with cycles is to use tools from graph theory, such as connectivity and centrality measures, to study how the removal of nodes affects the overall structure of the network. It is also possible to extend percolation models to account for the presence of cycles in the network, although this can be more complex to analyze mathematically. Thus, the following questions arises naturally in this setting:
\begin{compactenum}
\item[\emph{Question 1.}] \emph{How is the resiliency $R_{\cG}(\varepsilon)$ affected if we relax the DAG assumption?}
\item[\emph{Question 2.}] \emph{How do we identify the most vulnerable nodes?}
\item[\emph{Question 3.}] \emph{How can we design interventions to minimize the (expected) size of failure cascades?}
\end{compactenum}

As one would expect, the most vulnerable nodes are the ones that are most ``central'' to the network since lots of other materials would have (potentially higher-order) connections. As it turns out, such nodes can be identified by computing the Katz centrality of the network \citep{katz1953new}, which arise naturally under certain assumptions in our model. Moreover, minimizing the cascade size can be achieved (partly) by safeguarding the most vulnerable nodes, which is again related to the Katz centrality, as we discuss in \cref{sec:interventions}.

A surprising way to identify such nodes involves extending the percolation process to account for some link percolation phenomena, creating a noisy version of the node percolation process introduced in \cref{sec:node_percolation}. More specifically, for each edge $(i, j) \in \cE(\cG)$ of the production network, we flip a coin of bias $y \in (0, 1)$, independently of the other edges and suppliers, and decide to keep the edge with probability $y$. This creates a subsampled graph $\cG_y \subseteq \cG$, which under reasonable assumptions for $x$ and $y$, can be used to identify ``likely to fail'' products. The above process can also be viewed as a \emph{joint percolation} on both nodes and edges or a node percolation on the noisy subnetwork $\cG_y$, where in order for a product to function, on average it only needs a $y$-fraction of its inputs to operate.

Below, we answer the questions raised above and provide a systematic way of treating general graphs that undergo a joint percolation process. More specifically, we devise a systematic way to bound the expected number of failures and subsequently derive bounds for the resilience metric. Finally, we show that our analysis has deep connections to financial networks. To put our analysis into a mathematical framework, Markov's inequality states that $\Pr [F \ge \varepsilon K] \le \frac {\ev {} {F}} {\varepsilon K}$, so, limiting a failure of at least $\varepsilon K$ products requires upper bounding $\ev {} {F} = \sum_{i \in \cK} \Pr [Z_i = 0]$. At first, glance, bounding $\ev {} {F}$ looks to be instance-specific since different architectures behave differently under the percolation process; cf. \cref{sec:architectures}. The following Theorem gives a way to characterize an upper bound on $\ev {} {F}$ as the solution to a linear program.

\begin{theorem} \label{theorem:general_ub}
    Let $\cG$ be a graph of maximum outdegree $\Delta \ge 1$, that undergoes a joint percolation process with node failure probability $x \in (0, 1)$ and edge survival probability $y \in (0, 1)$. If $A$ is the adjacency matrix of $\cG$, and $\beta = (\beta_i)_{i \in \cK} = \left ( \Pr[Z_i = 0] \right )_{i \in \cK}$, then an upper bound to $\ev {} {F}$ can be found via solving the following linear program
    \begin{align}
	\ev {} {F} \le \mathsf{OPT}_1 = \max_{\beta \in [\zero, \one]} \quad & \sum_{i \in \cK} \beta_i \label{eq:upper_bound_lp} &
	\text{s.t.} \quad & \beta \le y A^T \beta + x^n \one.
    \end{align}
    Moreover, if $y \le \frac {1} {\Delta}$, then the solution to the LP can be found as the greatest fixed point $\beta^* \in [\zero, \one]$ of the non-expansive operator $\Phi(\beta) = \one \wedge (y A^T \beta + x^n \one)$.
    
\end{theorem}

\begin{proofsketch}
    The proof uses a union bound on $\beta_i = \Pr [Z_i = 0]$ for every $i \in \cK$ and maximizes $\sum_{i \in \cK} \Pr [Z_i = 0]$ under the union bound constraint and the fact that $\{ \Pr [Z_i = 0] \}_{i \in \cK}$ are valid probabilities. The second part of the proof applies \citep[Lemma 4]{eisenberg2001systemic}. 
\end{proofsketch}


\begin{algorithm}[t]
    \scriptsize
    \captionof{algorithm}{Solution to \cref{eq:upper_bound_lp} for DAGs\label{alg:dag_upper_bound}}
    \begin{flushleft}
    \textbf{Input:} DAG $\cG$, node percolation probability $x$, number of suppliers $n$, edge sampling probability $y$. \\
    \textbf{Output:} The solution ${\beta^*}$ to the linear program in \cref{theorem:general_ub} ---  \cref{eq:upper_bound_lp}. 
    \medskip
    \begin{compactenum}
        \item Use depth-first search to create a topological ordering of DAG $\cG$, $\pi : \cK \to \cK$, and let ${\beta^*}_{\pi(1)} = x^n$
        \item For $2 \le i \le K$, set ${\beta^*}_{\pi(i)} = \min \left \{ 1, y \sum_{j \in \cN(\pi(i))} \beta^*_j + x^n \right \}$
    \end{compactenum}
    \end{flushleft}
\end{algorithm}

\cref{theorem:general_ub} shows that there is a systematic way of bounding $\ev {} {F}$ via the solution of a linear program or a fixed-point equation (if the edge survival probability is less than $1/{\Delta}$). It is surprising to note that an elegant upper bound on the expected number of failures becomes possible by introducing shocks at the edge level. However, \cref{theorem:general_ub} gives a polynomial time algorithm to compute an \emph{upper bound} on $\ev {} {F}$ by solving the aforementioned LP. \cref{alg:dag_upper_bound} solves this LP when $\cG$ is a DAG in $O(K + |\cE(\cG)|)$ time. 
% \cref{fig:optimization2} shows the solution of the optimization problem in a DAG with 3 nodes. 
When we allow cycles in $\cG$, the calculation becomes more complicated and \cref{alg:dag_upper_bound} does not produce a valid solution; this case is treated by the following proposition %Below, we show how we can treat the case of finding the optimal solution to \cref{theorem:general_ub}, when $\cG$ can have cycles.
where we prove that under certain conditions, the fixed-point solution in \cref{theorem:general_ub} corresponds to the Katz centrality of $\cG$. Recall $\gamma_{\mathsf{Katz}}(\cG,y) =  (I - yA^T)^{-1} \one$, where $A$ is the adjacency matrix of $\cG$.  %as we formalize by the following Proposition,

\begin{proposition} \label{prop:katz}
     Consider the joint percolation process on graph $\cG$ with maximum outdegree $\Delta \ge 1$, adjacency matrix $A$, node failure probability $x \in (0, 1)$, and edge survival probability $y \in (0, 1)$, and let ${\gamma_{\mathsf{Katz}}(\cG,y)}$ be its Katz centrality vector.  If $0 < y < \frac 1 {\Delta}$ and $0 < x < \left (1 - y\Delta \right )^{1/n}$, then the fixed point solution ($\hat{\beta}$) to the problem of \cref{theorem:general_ub} is the Katz centrality measure of $\cG$ scaled by $x^n$: $\beta^* = {x^n} \gamma_{\mathsf{Katz}}(\cG,y) = {x^n}  (I - yA^T)^{-1} \one$. Subsequently, $R_{\cG}(\varepsilon) \ge \left ( \frac {\varepsilon} {\| \mathsf{vec} ((I - yA^T)^{-1} \|_1} \right )^{1/n} = \left ( \frac {\varepsilon} {\one^T \gamma_{\mathsf{Katz}}(\cG,y)} \right )^{1/n}$.
\end{proposition}

\begin{proofsketch}
    We show that for these values of $x$ and $y$ the union bound constraints are enclosed in the box constraints $[0, 1]^K$, and since $\beta^*$ which solves the linear program in \cref{eq:upper_bound_lp} is also the unique solution to the fixed point problem in \cref{theorem:general_ub} (due to Banach's theorem because $\Phi(\beta)$ is a contraction), we have $\beta^* = {x^n} \gamma_{\mathsf{Katz}}(\cG,y)$.
\end{proofsketch}

In the case of DAGs, we can get a special lower bound for $R_{\cG}(\varepsilon)$, in the large-shock regime for edges, i.e., when edges fail with high probability $1-y$. 

\begin{proposition} \label{prop:dag_lb_sparse}
    Let $\cG$ be a DAG, and consider the joint percolation process with node percolation probability $x \in (0, 1)$, and edge survival probability $y \in (0, 1)$. The number of failures on $\cG$ obeys $\ev {} {F} \le \frac {x^n e^{Ky}} {y}$, and therefore the resilience of any DAG obeys $R_{\cG}(\varepsilon) \ge \left ( \frac {\varepsilon y} {e^{Ky}} \right )^{1/n}$. When $y = \frac 1 K$, the lower bound on the resilience is maximized and equals $\left ( \frac {\varepsilon} {e K} \right )^{1/n}$.
\end{proposition}

\begin{proofsketch}
    We consider a topological order of the DAG and prove that the solution $\beta^*$ to the linear program of \cref{theorem:general_ub} --- \cref{eq:upper_bound_lp} --- satisfies $\beta_i^* \le (1 + y)^{i - 1} x^n$ for all $i \in \cK$. Therefore, we can prove that the expected number of failures is at most ${x^n e^{K y}}/ {y}$ which is minimized for $y = 1/K$, yielding the maximized lower bound on  $R_{\cG}(\varepsilon)$. 
\end{proofsketch}

An immediate consequence of \cref{{prop:dag_lb_sparse}} is that whenever there is an over-diversification of suppliers, i.e., $n \ge Ky$, then every DAG is resilient --- in agreement with the results of \citet{elliott2022supply}. This idea of maximizing $\ev {} {F}$ has its roots in financial networks, specifically the Eisenberg-Noe model of financial contagion among agents with assets and liabilities \citep{eisenberg2001systemic, glasserman2015likely}. Similar connections between the Eisenberg-Noe clearing problem and network centrality have been made by \citet{siebenbrunner2018clearing,bartesaghi2020risk} and references therein. %, which has a similar functional form. In this model, we have a collection of agents with liabilities to one another as well as external assets and liabilities. In the equilibrium, the agents should pay their debts in full, and if they are not able to do it, they must redirect all of their cash flows to their neighbors using a rationing scheme. 
%Identifying the equilibria in the Eisenberg-Noe problem corresponds to solving a fixed-point equation or equivalently, as it is proven in \citep[Lemma 4]{eisenberg2001systemic}, a linear program, which we utilize in our proof as well in the case of $y \le \frac {1} {\Delta}$. 
Moreover, the solution to \cref{eq:upper_bound_lp} gives a measure of \emph{``vulnerability''} of each node, i.e., how likely they are to be affected by cascading failures. In case of DAGs, we can relate this ranking of the node vulnerabilities to their topological ordering as follows:

\begin{corollary}
    If $\cG$ is a DAG, $\pi : \cK \to \cK$ is a topological ordering of the nodes in $\cG$, and $\beta^*$ is the solution to the linear program in \cref{eq:upper_bound_lp}, then $\beta^*_{\pi(i)} \le \beta^*_{\pi(j)}$ for all $1 \le i \le j \le K$. 
\end{corollary}

The preceding corollary, which is a direct consequence of \cref{theorem:general_ub} and \cref{alg:dag_upper_bound}, states that when $\cG$ is a DAG, the least vulnerable nodes are the raw materials that precede more complex products in their topological ordering. On the other hand, the failure of raw materials in a DAG will cause large cascading failures, affecting all the complex products that succeed the raw materials in their topological ordering. Hence, to prevent large cascading failures, it seems intuitive to intervene to protect nodes that rank highest in the reversed graph. Our results in the next section formalize this intuition, giving an explicit solution for optimal interventions in terms of the Katz centrality of the reversed graph (\cref{prop:intervention}).    

%  .... This gives a clear way to design interventions on a DAG, which would involve intervening on $\pi(1), \dots, \pi(T)$

% \subsection{Relationship to Input-Output Matrices}

% \mpcomment{talk about Leonfieff inverse, input output matrices, etc. see the production networks primer review for ideas}


% \begin{figure}[t]
%     \centering
%     \begin{tikzpicture}[transform shape]
%         \Vertex[x=-1, y=0, label=$1$]{u1}
%         \Vertex[x=1, y=0, label=$2$]{u2}
%         \Vertex[x=0, y=1, label=$3$]{u3}
%         \Edge[Direct, color=black](u2)(u1)
%         \Edge[Direct, color=black](u2)(u3)
%         \Edge[Direct, color=black](u3)(u1)
        
%     \end{tikzpicture}
    
    
%     \caption{Example for \cref{theorem:general_ub}. The optimization of \cref{theorem:general_ub} yields $\beta = ((1 + y)^2 x^n, x^n, (1 + y) x^n)$ and a bound $\ev {} {F} \le \frac {(1 + y)^3 x^n} {y}$.}
%     \label{fig:optimization2}
% \end{figure}

\subsection{Intervention Design} \label{sec:interventions}

Safeguarding supply chains is an ever-important issue in supply-chain management. Many potential risks can disrupt the supply chain and impact a business's operations and bottom line. These risks can include natural disasters, transportation issues, supplier bankruptcy, cyber-attacks, geopolitical turmoils, etc. It is important for businesses to have contingency plans to address potential supply chain disruptions and minimize their impact. In our model, we start by building intuition behind designing interventions to safeguard the supply chain. Our problem involves a global planner that can cure a maximum of $T$ suppliers in the network. Their aim is to minimize failures, which can be achieved in many ways, such as  diversifying the supplier base, building inventory buffers, and implementing robust risk management and monitoring systems. In mathematical terms, since each product can be produced if it has at least one functional supplier, this is equivalent to selecting a maximum of $T$ products to intervene. The decision variables are set to be $t_i \in \{ 0, 1 \}$, which implies that the budget constraint is $\sum_{i \in \cK} t_i \le T$ and a random number of failures $F=  F(t)$. Also, every product's failure probability $(x(1 - t_i))^n = x^n (1 - t_i)$, $t_i \in \{ 0, 1 \}$. A ``good'' problem for the planner would be to try to minimize the upper bound on the worst-case expected damage, i.e.,
\begin{align} \label{eq:interventions}
    \min_{t \in \{0, 1 \}^n} \ev {} {F(t)}  \le  \min_{t \in \{ 0, 1 \}^n}  \max_{\beta \in [\zero, \one]} \quad & \one^T \beta &
    \text{s.t.} \quad & \beta \le y A^T \beta + (\one - t) x^n,\; \one^T t \le T. 
\end{align}
% In \cref{theorem:general_ub}, we showed that we can construct a rigorous upper bound on $\ev {} {F}$ as the solution to an LP. Following that, it was easy to observe that when $\cG$ is a DAG, then this upper bound can be computed efficiently using \cref{alg:dag_upper_bound}. This directly leads to the following corollary, which states that the  optimal vulnerabilities are increasing in the topological order of the DAG: 

% At first glance, it is not obvious how to resolve the problem in general graphs. However, 
Note that if $0 < y < \frac 1 {\Delta}$ and $0 < x < (1 - y\Delta)^{1/n}$, then \cref{prop:katz} implies that the solution to the internal maximization in \eqref{eq:interventions} is $\hat \beta (t) = x^n (I - yA^T)^{-1} (\one - t)$. This yields the following Proposition:

\begin{proposition} \label{prop:intervention}
    Let $\cG$ be a graph of maximum outdegree $\Delta \ge 1$, let $\cG^R$ be the graph where the direction of edges in $\cG$ are reversed, and let $\Delta_R$ be the maximum outdegree of $\cG^R$. Let $0 < y < \frac 1 {\max \{ \Delta, \Delta_R \}}$, $0 < x < (1 - y \Delta )^{1/n}$. Consider $\gamma_{\mathsf{Katz}}(\cG^R,y) = (I - yA)^{-1} \one$, the Katz centrality of $\cG^R$, and let $\pi : \cK \to \cK$ be a decreasing ordering on the entries of $\gamma_{\mathsf{Katz}}(\cG^R,y)$. Then, the optimal policy $\hat t$ sets $\hat t_{\pi(i)} = 1$ for $i \in [T]$ and sets it to zero otherwise. Subsequently, the resilience is at least $\left ( \frac {\varepsilon} {\gamma_{\mathsf{Katz}}^T(\cG^R,y) (\one - \hat t)} \right )^{1/n}$.
\end{proposition}
%\mar{reversed graph $\Delta$ and $\Delta_R$ undefined}


So, one can think of the ``riskiest'' to be the ones with high Katz centrality in $\cG^R$  --- the reversed graph representing  the sourcing relationships between products. This agrees with our intuition on DAGs which says to intervene starting from the raw materials and progressing in the topological order of the DAG until the budget is exhausted. 