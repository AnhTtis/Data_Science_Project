
\section{Introduction} \label{sec:introduction}

The global economy consists of many interconnected entities which are responsible for the supply and sourcing of products \citep{carvalho2019production}. These \emph{production networks} consist of \emph{products}, each of which has some required \emph{inputs} that can be sourced from a group of \emph{suppliers}, and they play a critical role in day-to-day operations of the world economy \citep{long1983real}. Such interdependence between products yields \emph{cascades} once parts of the supply chain are disrupted \citep{horvath1998cyclicality,carvalho2019production,lucas1995understanding,gabaix2011granular,acemoglu2012network,hallegatte2008adaptive} For instance, the recent COVID-19 pandemic and the war in Ukraine disrupted parts of the global supply, whose failures then propagated to other parts of the supply chain network, causing \emph{bottlenecks} and \emph{choke points} \citep{ergun2022structured,elliott2022supply,guan2020global,walmsley2021impacts}. In light of such problems, there is an ever-emerging need to study the resilience of supply chain networks to identify  vulnerabilities, such as bottlenecks and choke points, and take steps to mitigate their effects \citep{kleindorfer2005managing,gurnani2012supply,simchi2014superstorms}. A motivating example is a tree production network where a variety of raw materials leads to the production of a specialized product. There, it is easy to observe that the failure to produce any of the raw materials, because all of the suppliers that make it failed, has a devastating effect on the network, and almost nothing can be produced. 

%Disruptions in these networks can have significant impacts, and it is essential to study their resilience to identify . 
There are various methods and approaches that can be used to study the resilience of supply chain networks \citep{perera2017network}. One direction is to use network analysis tools to identify critical components of the network and assess their vulnerability to disruption. This can include identifying key suppliers, products, and transportation routes and evaluating their importance to the overall network. Another approach is to use simulation modeling to analyze the impact of different disruptions on the network and assess the ability of the network to recover from these disruptions \citep{simchi2015identifying}. This can include analyzing the effect of different types of disturbances (e.g., natural disasters, supply chain failures, etc.) and different scenarios in which the disruptions occur (e.g., simultaneous disruptions, sequential disruptions, etc.). In addition to these technical approaches, it is important to consider organizational and strategic approaches to building resilience in supply chain networks \citep{braunscheidel2009organizational,simchi2018increasing}. This can include implementing contingency plans, developing relationships with alternative suppliers, and diversifying the network to reduce the impact of a single point of failure. Overall, building resilience in supply chain networks requires a combination of technical analysis, strategic planning, and organizational preparedness. By understanding the structure and dynamics of these networks, it is possible to build systems that can withstand disruptions and recover quickly when they occur.

\subsection{Main Related Work} We model the production network as a graph that undergoes a percolation process on its nodes, representing products. Each product has some suppliers that, if they all fail, then the product cannot be produced. Subsequently, a cascade is caused because of such a failure. Our modeling decision to study the supply network as a graph that undergoes percolation has its roots in prior work (see the referenced works in Appendix \ref{app:further_related_work}), the most closely related is the recent work of \citet{elliott2022supply}. They consider a \emph{hierarchical production network} which undergoes a \emph{link percolation process}. Specifically, firms are operational if all of their inputs have operating links to at least one firm producing the specific inputs. \citet{elliott2022supply} study the \emph{reliability of the supply network} as the probability that the root product is produced. They show that three important factors affect reliability: \emph{(i)} the depth/size of the supply chain, \emph{(ii)} the number of inputs that each product requires, and \emph{(iii)} the number of suppliers (which corresponds to the ability of firms to multi-source their required inputs). 

In our model, we investigate the effect of similar structural factors -- i.e., topology, number of inputs, and number of suppliers -- however, under a {fundamentally expanded setup} by \emph{(i)} considering production shocks at the level of individual suppliers -- rather than links; \emph{(ii)} by considering more general architectures than the hierarchical production networks of  \citet{elliott2022supply}, and \emph{(iii)} by studying a novel, theory-informed resilience metric that ensures almost all products can be produced in the event of a supply chain shock. One of our main results shows that production networks are either resilient or fragile. \citet{elliott2022supply} show that when the systemic shock magnitude is above some critical value, then reliability goes to zero corresponding to \emph{fragile} networks, and when the shocks' magnitude is below this critical value, then reliability attains a strictly positive value corresponding to \emph{resilient} networks. Briefly, they observe that if multi-sourcing (having a multitude of suppliers) increases, then reliability increases, and as interdependency (requiring a multitude of inputs) increases, reliability decreases. Our analysis differentiates from \citet{elliott2022supply} and focuses on determining the largest possible shock that a network can withstand so that a significant fraction of the products are producible (survive) in the event of such a shock as the network size increases. Our networks are parametrized similarly by size, multi-sourcing, and interdependency parameters. In agreement with \citet{elliott2022supply}, we show that networks become more resilient when multi-sourcing increases, and when interdependencies among products increase, networks become less resilient. Additionally, we observe that certain types of networks are resilient, i.e., they can withstand sufficiently large shocks without experiencing catastrophic, cascading failures, and others are fragile -- i.e., arbitrarily small shocks are enough to render {a significant proportion of} their products unproducible. Furthermore, we provide additional evidence in the form of the emergence of power laws for the size of cascading failures, which further motivates the study of resilience in complex production networks. Compared to existing indices that combine multitudes of temporal, financial, and economic risk factors \citep{gao2019disruption}, our resilience metric identifies topological risk factors and focuses on cascading failures to identify fragile and resilient supply chain architectures. 

\citet{perera2017network} systematically reviews the literature on modeling the topology and robustness of production networks with applications to real-world data. Their first observation is that many real-world production networks follow a power law degree distribution with an exponent of around two. Moreover, resilience (called ``robustness''  in their work) of such networks is defined as the size of the largest connected component or the average/max path length in presence of random failures. On the contrary, we propose a novel, theory-informed resilience metric that complements the existing measures in \citet{perera2017network}, and references therein, and supplement their empirical works with novel theories about how topological attributes contribute to increased risk of cascading failures in production networks. We also test our proposed metric on real-world data and provide empirical insights comparable to \citet{perera2017network}. Finally, we propose interventions for improving resilience with theoretical guarantees similar to optimal allocations in the presence of shocks and financial risk contagion  \citep{eisenberg2001systemic,papachristou2022dynamic,blume2013network,erol2022network,bimpikis2019supply}, and complementary to other supply chain interventions that, e.g., increase visibility and traceability \citep{blaettchen2021traceability}. An extended literature review is provided in Appendix \ref{app:further_related_work}.


In the next Subsection, we use the above context to summarize our main contributions. Next, in \cref{sec:preliminaries}, we introduce some preliminaries, which we use to motivate our proposed measure of resilience in \cref{sec:motivation}. In \cref{sec:architectures}, we present our new resilience metric formally and study it in various architectures  -- see also \cref{tab:resiliences} in the next subsection, which lists our main results for different architectures. In \cref{sec:general_supply_chains}, we present a general computational framework for lower-bounding the proposed metric in general graphs and make additional connections to financial networks, financial contagions, and centrality measures, as well as propose algorithms for optimal interventions in supply chains. In \cref{sec:experiments}, we supply our theoretical analysis with experiments with real-world supply-chain data, where we numerically calculate resilience and design intervention policies. Finally, in \cref{sec:discussion}, we summarize our main insights and conclusions and provide potential avenues for future work.

\subsection{Main Contributions} \label{sec:contribution}

To motivate our investigation of resiliency in supply chain networks, we first present a model of why supply chain networks are prone to catastrophic failures and why a measure of resilience is needed to distinguish different supply chain architectures. Our model is based on a node percolation process in which the possible suppliers of a product fail independently with probability $x$, and a product can be produced when its requirements are met and its suppliers do not fail. 

\noindent \textbf{Cascading Failures and Emergence of Power Laws (\cref{sec:preliminaries}).} To start with, we state that the size of cascading failures in a supply chain follows a power law when the underlying supply graph is a random DAG --- representing a natural topological hierarchy among products whereby the production of more complex products is contingent on the supply of simpler products and raw materials:

% \begin{quote}
\begin{informaltheorem}
    Consider the production network of $K$ products that is realized according to a random DAG model with edge probability $p$, and consider the node percolation model with failure probability $x$ for each of the $n$ suppliers of each product. Let $F$ be the total number of products that cannot be produced due to the unavailability of their supplier or other requisite products. Then $F$ follows a power-law distribution and admits a tail lower-bound of $\;\Pr [F \ge f] \ge {C}/{f}$ for some constant $C > 0$ is a constant dependent on $K$, $p$, and $x$.
\end{informaltheorem}
% \end{quote}

\noindent \textbf{Study of Resilient Architectures (\cref{sec:architectures}).} In the sequel, we study the resilience of many production networks and identify resilient and non-resilient architectures. Recall that we use $x$ to denote the probability of suppliers failing randomly. Consider a production network ${\cG}$ with $K$ products. Our notion of resilience, denoted by $R_{\cG} (\varepsilon)$, determines the maximum value of $x$ such that at least $(1 - \varepsilon)$ fraction of the products in the production network ${\cG}$ are produced with probability at least $1 -1/K$. In particular, we are interested in the limit of $R_{\cG} (\varepsilon)$ for different structures as $K\to\infty$, and for resilient structures, this limit is bounded away from zero. 



\begin{table}[b]
    \centering
    \scriptsize
    \begin{tabular}{lll}
    \toprule
        Topology & Resilience ($R_{\cG}(\varepsilon)$) & Main Result \\
    \midrule
        Random DAG & No & \cref{theorem:rdag_resilience} \\ 
        Parallel Products w/ dependency $d$ & Yes, with $R_{\cG}(\varepsilon) = \Omega \left ( \left ( \frac {\varepsilon} {dm} \right )^{1/n} \right )$ & \cref{theorem:parallel_products}  \\
        Backward network & No, with $R_{\cG}(\varepsilon) = O \left (  \left ( \frac {m \log K} {K} \right )^{1/n}  \right )$ for $m \ge 2$ & \cref{theorem:tree_resilience,eq:quick}  \\
        &  No, with $R_{\cG}(\varepsilon) = O \left (  \left ( \frac {2} {(1 - \varepsilon) K} \right )^{1/n}  \right )$ for $m = 1$ \\
        Forward network & Yes, with probability $\eta^*$ & \cref{theorem:gw_resilience} \\
        Random Trellis & Yes, when $pw \le 1$, $K / w \to D < \infty$ & \cref{theorem:trellis_resilience} \\
        & with $R_{\cG}(\varepsilon) = \Omega ((\varepsilon/D^2)^{1/n})$ & \\
        & No, when $pw \ge 1, K / D \to w < \infty$ \\ 
        
        & with $R_{\cG}(\varepsilon) = O((w / K)^{1/n})$. & \\ 
     \bottomrule 
    \end{tabular}
    \caption{Summary of Results (see \cref{sec:contribution} for the definition of parameters for each network).}
    \shrink
    \shrink
    \label{tab:resiliences}
\end{table}
We study the resilience of the following production networks, whose parameters we state below:

\begin{compactenum}
    \item \emph{Random DAGs:} $K$ products ordered as $1,2,\ldots, K$ and connected with directed edge probability $p$ that respects their ordering (see \cref{sec:motivation} and
 \cref{fig:random_dag_2}). 
    \item \emph{Parallel products with dependencies:} A set of raw materials produces $K$ complex products, each complex  product requires $m$ raw inputs (source dependencies), and each raw material is used by $d$ complex products ($d$ is called supply dependency) (see \cref{sec:parallel_products} and \cref{fig:parallel_products}).
    \item \emph{Hierarchical production networks:} \emph{(i) Backward hierarchical production network:}  A tree production network with depth $D$ and fanout $m \ge 1$, i.e., each product has $m$ inputs independently of the other products. In this supply chain, the failures start from the raw materials we position at the tree's leaves, and the cascades grow from the leaves to the root (see \cref{sec:forward_backward_tree} and \cref{subfig:forward_backward}), and \emph{(ii) Forward hierarchical production network:} A tree production network generated by a branching process (also known as the Galton-Watson process) with branching distribution $\cD$ with mean $\mu$, which has (random) extinction time $\tau$, and extinction probability $\eta^*(\cD) = \Pr [\tau < \infty]$. In this regime, the percolation starts from the root node (raw material), and proceeds to the leaves (see \cref{sec:forward_forward_tree} and \cref{subfig:forward_forward}). 
    \item \emph{Random width-$w$ trellis:} A trellis network with $D$ tiers, with each tier having width $w$. Random edges are generated independently with probability $p$ between (only) tier $d$ and $d + 1$. The network has $K = wD$ nodes (see Appendix \ref{app:random_trellis} and \cref{fig:trellis}). It serves as a generalization of the parallel products and is a special instance of the random DAG.
\end{compactenum} 

\cref{tab:resiliences} summarizes our results for the resilience of the above architectures and the corresponding theorems where the results are proved. Namely, we show:

% \begin{quote}
\begin{informaltheorem}
    The random DAG, backward production network, and random trellis with $pw \ge 1$ and constant width are always fragile with $R_{\cG} (\varepsilon) \to 0$ as $K \to \infty$ for all $\varepsilon$. In contrast, parallel products with dependencies and the random trellis with constant depth and $pw \le 1$ are resilient. The forward production network is resilient with a positive probability.
\end{informaltheorem}
% \end{quote}

In addition to identifying fragile architectures, our asymptotic analyses of $R_{\cG}(\varepsilon)$ reveal how various factors such as increased 
 dependency ($d$ and $m$) affect how fast $R_{\cG}(\varepsilon) \to 0$ in case of fragile networks, or how far it is bounded away from zero, in case of resilient structures. These nuances are expanded upon in \cref{sec:architectures}, as we present our results for each architecture in \cref{tab:resiliences}.

\noindent \textbf{Resilience of General Graphs \& Optimal Interventions (\cref{sec:general_supply_chains}).} Our results in \cref{tab:resiliences} address the various instances of the DAG architecture. Our next set of results concentrates on generalizing the lower-bound analysis of the resilience metric, $R_{\cG}(\varepsilon)$, to any production network. We do this by upper-bounding the expected number of failures for a given graph. The latter is, in general, \#P-hard to compute, but its upper-bound can be solved in polynomial time as an LP:

\begin{informaltheorem}
    An upper bound to the expected number of failures for any production network $\cG$ 
 can be found by solving the LP presented in \cref{eq:upper_bound_lp}. 
\end{informaltheorem}

In \cref{theorem:general_ub}, we give this LP for a more general percolation process than the one we study for DAG production networks. In particular, we allow links to also fail independently at random with probability $y \in (0, 1)$. Moreover, in \cref{theorem:general_ub}, we show that under specific conditions on $y$, our problem is a special instance of the financial contagion model of \citet{eisenberg2001systemic}, thus establishing a promising connection between financial network contagion and our percolation analysis. 

Finally, the formulation of \cref{theorem:general_ub} can be used to  identify ``vulnerable'' products -- i.e., products that are more likely to fail than others -- and, subsequently, to design interventions in supply chains (see \cref{sec:interventions}) based on Katz centrality. In \cref{prop:katz}, we show that under certain assumptions, the probability that each node is failing is related to its Katz centrality \citep{katz1953new}, which is in agreement with the existing literature in financial networks \citep{siebenbrunner2018clearing,bartesaghi2020risk}. Additionally, \cref{prop:katz} allows us to establish a generic lower bound on $R_{\cG}(\varepsilon)$ and an optimal intervention policy (under assumptions) as follows:
 
 \begin{informaltheorem} \label{informal_theorem:intervention}
     Consider a production network $\cG$ with adjacency matrix $A$ and max outdegree $\Delta \ge 1$, and its reverse network $\cG^R$ with maximum outdegree $\Delta_R \ge 1$. If $0<y < \frac 1 {\max \{ \Delta, \Delta_R \}}$ and $0<x < (1 - y \max \{ \Delta, \Delta_R\})^{1/n}$, then the following are true: 
     \begin{compactenum}
         \item $R_{\cG}(\varepsilon) \ge \left (\frac {\varepsilon} {\one^T \gamma_{\mathsf{Katz}}(\cG,y)} \right )^{1/n}$, where $\gamma_{\mathsf{Katz}}(\cG,y) = (I - y A^T)^{-1} \one$ is the Katz centrality. 
         \item If a planner can protect up to $T$ suppliers (which corresponds to protecting $T$ products), then the optimal intervention of the planner is to protect the $T$ products in decreasing order of Katz centrality in $\cG^R$, i.e., $\gamma_{\mathsf{Katz}}(\cG^R,y) = (I - y A)^{-1} \one$.
     \end{compactenum}
     
 \end{informaltheorem}
 
 Therefore, the Katz centrality of each node in the production network can inform the social planner's effort to design contagion mitigation strategies to improve supply chain resiliency. Furthermore, we show that the resilience of any DAG is $\Omega \left ( \left ( \frac {\varepsilon y} {e^{Ky}} \right )^{1/n} \right )$ which is maximized when $y = 1/K$; thence, giving an $\Omega \left ( \left ( \frac {\varepsilon} {K} \right )^{1/n} \right )$ lower bound on $R_{\cG}(\varepsilon)$ for any  DAG; a result which is consistent with our earlier analysis of the proposed resilience metric in DAGs. 

\noindent \textbf{Empirical Results (\cref{sec:experiments}).} We experiment with real-world production networks from \citet{willems2008data} and the World I-O database from \citet{timmer2015illustrated}. In the former case, the production networks are DAGs, whereas in the latter case, the networks correspond to countries' economies and can also have cycles. We do Monte Carlo simulations to numerically determine the resilience of such real-world networks and show that overall more densely connected networks and those with more concentration of products in their early tiers are less resilient. Then, we design interventions based on the policy devised in \cref{informal_theorem:intervention} and numerically show different resilience lower bounds between the different networks. In both cases, interventions make less resilient networks have a higher resilience lower bound. %In the latter dataset, intervening in sparser production networks increases this lower bound.   


% \noindent \textbf{Paper Organization.} This paper is organized as follows: \cref{sec:related_work} provides a brief overview of related works in the supply chain, fragility, and resilience. Next, \cref{sec:preliminaries} introduces some preliminaries and motivates the need for the resilience metric in \cref{sec:motivation} through the analysis of the random DAG model. In \cref{sec:architectures}, the resilience metric is presented, and various architectures are studied (see also \cref{tab:resiliences}). In \cref{sec:general_supply_chains}, we present the lower bound analysis for general graphs and its connections to financial networks and centrality measures. Finally, in \cref{sec:discussion}, we discuss insights into this work and potential avenues for future work. 
%moved ot sthe end of the inroduction section


% 1) a comprehensive review of the methodologies
% adopted in literature for modelling the topology and robustness of SCNs; (2) a summary
% of topological features of the real world SCNs, as reported in various data driven studies;
% and (3) a discussion on the limitations of existing network growth models to realistically
% represent the observed topological characteristics of SCNs.

% Also, the work most related to ours, as noted above, is the work of \citet{elliott2022supply}. \citet{elliott2022supply} study supply-chain fragility through an edge percolation model lens. Their analysis contains firms that can source essential inputs from other firms, and links in the network undergo percolation. In such environments, the endogenous topology of supply networks suffices to amplify shocks. The physics of their model resembles the branching process. Even though our work shares similar parametrizations (interdependency of products, ability to multisource), our model, metrics, and analysis differ from the one studied in \citet{elliott2022supply}. First, \citet{elliott2022supply} aims to measure the reliability of a hierarchical production network as the probability that the end product is produced. In contrast, our resiliency metric is fundamentally different from theirs in two axes: \emph{(i)} it measures the largest possible shock such that a large fraction of products are produced with high probability, and \emph{(ii)} it considers all the nodes in the network which makes it suitable to study more complex architectures such as random DAGs (see \cref{sec:motivation}) and general graphs (see \cref{sec:general_supply_chains}). In closing, we ought to highlight that the high-level conclusions of \citet{elliott2022supply} are similar (see \cref{sec:comparison_golub}) to ours.


