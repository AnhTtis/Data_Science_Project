\section{Managerial Insights and Concluding Remarks} \label{sec:discussion}


\noindent \textbf{Increased Risk from Interdependencies} Our insights so far show that more interconnected networks (with increased interdependencies among products) are less resilient. Specifically, in the case of tree networks, no more than a $(1/m)$-fraction of the products survive as we increase the node degrees $(m \to \infty)$, making the structure fragile --- see \cref{eq:quick} and the discussion in its preceding paragraph. Similarly, in the case of parallel product networks, we observe that resilience metrics converge to zero as the source dependency increases (in-degree $m\to\infty$; see \cref{theorem:parallel_products}).  Furthermore, in the trellis case, the network becomes less resilient when the interdependency -- which corresponds to the average in-degree $pw$ -- is large. This agrees with our empirical observations. Firstly, in the case of \citet{willems2008data} supply-chain networks, the networks with higher sourcing dependencies and higher average degrees were less resilient. Additionally, our results using Input-Output networks of world economies show that networks that are denser (with higher average degrees) --- which generally correspond to more developed economies --- are less resilient since shocks can spread more easily in denser networks (see \cref{tab:statistics-world} and \cref{fig:wolrd-io}). This agrees with our theoretical results and prior works that state networks with higher interdependence are less resilient \citep{elliott2022supply}. 

% \mpcomment{resilience decreases are raw materials increase, do not talk about complexity}

\noindent \textbf{Raw Materials vs. Complex Products.} A second potential point of discussion is whether for an economy to be resilient -- through the lens of our model -- should it focus on producing more raw materials or more complex products. 
First, \cref{theorem:tree_resilience} shows that as we increase the depth $D$ of the network, the network becomes less resilient. Similarly, in the case of trellis architectures, we show that denser and deeper, rather than sparser and wider, production networks are less resilient (see \cref{theorem:trellis_resilience}). Experimentally, this is also verified in \cref{subfig:willems_resilience} since Network \#30, which is more concentrated towards complex products (i.e., later tiers), is found to be less resilient experimentally. In contrast, Network \#10 is mostly concentrated on raw materials and is found to be the most resilient network.  

\noindent \textbf{Supplier Heterogeneity \& Optimal Interventions.} \cref{prop:intervention} shows that targeting the nodes that have high Katz centrality in the reversed (source) graph is optimal --- these nodes represent less complex products. This argument agrees with recent observations from the COVID-19 pandemic, where the failure of chip manufacturing industries (which are relatively close to the raw materials, compared, e.g., to more complicated electronic devices) caused very large cascades on a global scale. Another interesting insight that we can harvest from the above results is the effect of heterogeneity in multisourcing. Following our design of optimal interventions in \cref{sec:interventions}, we can attempt to increase the number of suppliers of product $i$, from $n$ to $n + \nu_i$ for some $0 \le \nu_i \le \overline \nu_i$, subject to a budget $\sum_{i \in \cK} \nu_i \le N$, to decrease its self-percolation probability from $x^n$ to $x^{n + \nu_i}$. The optimal intervention policy would be, again, given an ordering $\pi$ of the Katz centralities in the reverse production network as in \cref{prop:intervention} and under the same conditions, to set $\hat \nu_{\pi(i)} = \left ( \overline \nu_{\pi(i)} \wedge \left ( N - \sum_{j < i} \hat \nu_{\pi(j)} \right ) \right )^+$; see Appendix \ref{app:supplier_heterogeneity}. 

%We argue that this insight through our model is in agreement with existing arguments that there is a positive correlation between the complexity of the structure of an economy and how developed it is (see, e.g., \citet{hidalgo2009building,hidalgo2021economic,nguyen2022countries,saad2023economic}). In addition, our analysis also suggests that the Katz centrality can be used as a reasonable measure of economic complexity.

\noindent \textbf{Future Directions.} There are various ways that our model can be extended. The first interesting extension is to fit our model of the production network to actual economic and financial networks. After that, our model can be extended to contain costs, quantities, and link capacities, and the desired objective would be to optimize the network total profits subject to production shocks, extending, thus, a long-standing line of work on economic networks \citep{hallegatte2008adaptive,acemoglu2016networks,acemoglu2012network}. Another promising avenue is to study networks with varying interdependency and multi-sourcing. So far, we have assumed that products have the same number $n$ of suppliers. However, this can be extended to cases where each industry has random size $n_i \sim \cD_{\mathsf{industry}}$, where $\cD_{\mathsf{industry}}$ is a distribution (e.g., power law, as in \citet{gabaix2011granular}). It would be interesting to study this heterogeneity in the size of different industries and how it affects $R_{\cG}(\varepsilon)$; see also \citet{gabaix2011granular}. Yet another interesting roadmap would be to study $R_{\cG}(\varepsilon)$ under heterogeneous hierarchical networks, where the (potentially random) fanout changes at each tier. Moreover, as we saw in \cref{sec:general_supply_chains}, there is a direct connection between cascading failures in supply chains and systemic risk in financial networks. Therefore, we can readily adapt tools from the study of financial risk to supply chain resiliency and risk. 
% Finally, an interesting question is establishing generative models for supply chains and fitting real-world data to them. 

\noindent \textbf{Conclusions.} Through our paper, we aim to devise a systematic way to test which supply chain networks are \emph{resilient}, i.e., can withstand ``large enough'' shocks while sourcing almost all their productions. Through this metric of resilience, we classify some common supply chain architectures -- represented by DAGs -- as resilient or fragile and identify structural factors that affect resiliency. We then generalize our analysis and provide tools for studying the resilience of general supply networks, as well as motivating methods that can be used for the design of optimal interventions. 
% Overall, we make some general observations that are consistent with and expand prior works \citep{elliott2022supply}: as multisourcing (number of suppliers) increases, supply chains become more resilient, whereas dependencies across products can cause vulnerabilities in the supply network.  

