
\section{More Details on Dataset Features}


Figure~\ref{fig:dataset-overview} and Table~\ref{table:features} provide more information on the feature set used.

\begin{figure}[H]
	\centering
	\includegraphics[width=0.85\textwidth]{figures/dataset-overview}
	\caption{Overview of dataset.}
	\label{fig:dataset-overview}
\end{figure}




Table~\ref{table:features} describes the full feature variable set used at a high level, including basic network features, DeFi features from Uniswap, and features informed from economic models.


\begin{table}
	\centering
	\begin{tabular}{c | l}
		\textbf{Feature type} & \textbf{Feature (high level description)} \\
		\hline
		Network & Number of blocks\\
		 & Number of transactions \\
		 & \% change in accumulated ETH supply \\
		 & Avg gas limit \\
		 & Avg gas used \\
		 & Avg gas price \\
		 & Hash rate \\
		Uniswap & Liquidity in ETH/stablecoin pools \\
		& Trade volume in ETH \\
		Economic & Mining pay-off factors \\
		& Computational burden measures \\
		& Congestion factors \\
		& Social cost factors \\
		& Spreading factor \\
		\hline
	\end{tabular}
	\caption{Data features (high level).}
	\label{table:features}
\end{table}

Table~\ref{table:feature-defs} defines the variables used in the main text figures.

\begin{table}
\centering
\begin{tabular}{c | l}
    \textbf{Feature} & \textbf{Definition} \\
    \hline
    \texttt{eth\_gaslimit} & hourly average gas limit on Ethereum \\
    \texttt{eth\_gasused} & hourly average gas used per block \\
    \texttt{eth\_blocksize} & hourly average Ethereum blocksize (gas) \\
    \texttt{eth\_n\_from\_address} & hourly average sender addresses per block \\
    \texttt{eth\_n\_to\_address} & hourly average receiver addresses per block \\
    \texttt{eth\_supply\_growth} & hourly change in ETH supply \\
    \texttt{eth\_hashrate} & hourly average hashrate on Ethereum \\
    \texttt{eth\_gasprice} & hourly average tx gas price \\
    \texttt{eth\_weighted\_gasprice} & hourly avg tx gas price, weighted by gas used in tx / total gas used \\
    \texttt{eth\_congestion\_1} & hourly gas used / hourly gas limit \\
    \texttt{eth\_congestion\_2} &  square of \texttt{eth\_congestion\_1} \\
    \texttt{eth\_spreading} & hourly \# receiving addresses / hourly \# sender addresses \\
    \texttt{eth\_close} & hourly ETH/USD closing price \\
    \texttt{btc\_difficulty} & hourly average difficulty on Bitcoin \\
    \texttt{btc\_minter\_reward} & hourly average miner reward on Bitcoin \\
    \texttt{btc\_n\_from\_address} & hourly average sender addresses per block \\
    \texttt{btc\_block\_size} & hourly average Bitcoin block size (bytes) \\
    \texttt{btc\_n\_tx\_per\_block} & hourly average \# txs per block \\
    \texttt{btc\_to\_address} & hourly average receiver addresses per block \\
    \texttt{btc\_daily\_hashrate} & hourly Bitcoin difficulty / hourly average block time \\
    \end{tabular}
    \caption{Definitions of variables used in figures.}
    \label{table:feature-defs}
\end{table}


Online documentation in the project github repo provide further details of the underlying economic models and calculation of the economic factors (as well as calculation of other factors from the raw data): \url{https://github.com/tamamatammy/oracle-counterpoint}.


\subsection{Economic features}
A brief overview of the features informed by fundamental economic models is as follows along with citations for the relevant models that influenced the choice of these features.

\begin{itemize}
	\item Mining payoff factor 1: $(R (\text{blockReward} + \text{blockFees}))^{-1}$ \cite{Kroll2013TheAdversaries,Prat2017AnMining}
	\begin{itemize}
		\item R = block rate (/s), eth\_n\_blocks = \# blocks in the last hour
	\end{itemize}
	\item Previous high hash rate / current hash rate
	\item previous high $(R (\text{blockReward} + \text{blockFees}))^{-1} / \text{current}$
	\item Excess block space (block limit - gas used)
	\item Social value: D(W) is the social value of the level of decentralization = D(W) = - log(W) $\implies$ D(W) = - log(gas\_used) for ethereum, = - log(bytes); gas used as the measure of the weight of a block (W) \cite{Buterin2018BlockchainPricing}
	\item Social cost: Marginal cost = 1/gas\_used or 1/bytes \cite{Buterin2018BlockchainPricing}
	\item Computational burden on nodes: use block\_size as bandwidth $\implies$ $\text{block\_size} * \log^2(\text{block\_size})$ \cite{Buterin2018BlockchainPricing}
	\item Congestion factors: rho = gas used/gas limit, and $\text{rho}^2$; (in economic model, rho is defined as average number of transaction per block / number of transactions per block) \cite{Huberman2019AnSystem}
	\item Congestion factor: Indicator\_\{$\text{rho} > x$\}, heuristic use x = 0.8 \cite{Huberman2019AnSystem}
	\item Congestion pricing term 1: F(rho) / tx\_fees\_eth, where F describes relationship between USD tx fees and congestion \cite{Huberman2019AnSystem}
	\begin{itemize}
		\item Heuristic: use F = congestion factor 1 or 2 above
	\end{itemize}
	\item Congestion pricing term 2: max number of transactions in a block / fees in block \cite{Nicolas2014TheFees}
	\item Congestion pricing term 3: max number of transactions squared in a block / fees in block \cite{Nicolas2014TheFees}
	\item Spreading factor: number of unique output addresses / number of unique input addresses \cite{SusanAtheyIvaParashkevovVishnuSarukkai2016BitcoinUsage}
\end{itemize}


%%%%%%%%%%%%%%%%%%%%%%%%%%%%%%%%%%%%%%%%%%%%%%%%%%%%%%%%%%%%%%%%%%%
\section{Further Information on Ethereum Analysis}

Sparse inverse covariance estimation was performed with the implementation in SciPy using an alpha parameter of 4, convergence tolerance of 5e-4, 5 folds for cross valiation, 4 grid refinements, and 1000 max iterations.

\begin{figure}[H]
	\centering
	\includegraphics[width=\textwidth]{figures/partial_corr.png}
	\caption{Partial correlation matrix from sparse inverse covariance estimation.}
	\label{fig:partial-corr}
\end{figure}


%%%%%%%%%%%%%%%%%%%%%%%%%%%%%%%%%%%%%%%%%%%%%%%%%%%%%%%%%%%%%%%%%%%
\subsection{Performance of Price Recovery}\label{appendix:performance}

We continue the analysis from Section~\ref{sec:performance} with a few additional charts that give a wider view on performance for the 30-day rolling retraining model. Figure~\ref{fig:DSEs} shows the difference in squared error of the benchmark minus the model over time. When $DSE>0$, the model is performing better than the martingale benchmark of the last observed price in the rolling retraining. Of note is that while the performance improves in the final year (May 2021-May 2022), one small period represents most of the performance size.

\begin{figure}[H]
	\centering
	\includegraphics[width=\textwidth]{figures/DSE-30-day}
	\caption{}
	\label{fig:DSEs}
\end{figure}

Figure~\ref{fig:RMSEs} shows the RMSE evaluated from each different starting point on the x-axis to the end of the dataset (May 2022). As we move to later starting points on the x-axis, it is worth noting that there is more training data incorporated into the model before the test set for RMSE. Toward the end of the dataset, the model becomes more competitive with the benchmark, surpassing it when measured over the final year of data.

\begin{figure}[H]
	\centering
	\includegraphics[width=\textwidth]{figures/RMSE-30-day}
	\caption{}
	\label{fig:RMSEs}
\end{figure}

Figure~\ref{fig:RMSEs-vol} shows a similar plot of RMSE evaluated from different starting points on the x-axis, but with the calculation restricted to the top 10\% times of volatility. Here volatility is calculated as 24 hour rolling volatility of hourly returns. The model is overall more competitive with the benchmark for top 10\% volatility times compared to all times, and surpassing it by a sizable amount measured over the final year of data. Note that as suggested in Figure~\ref{fig:DSEs}, the outperformance of the model in the final year largely rests on a short period of high outperformance.

\begin{figure}[H]
	\centering
	\includegraphics[width=\textwidth]{figures/RMSE-30-day-vol}
	\caption{}
	\label{fig:RMSEs-vol}
\end{figure}


%%%%%%%%%%%%%%%%%%%%%%%%%%%%%%%%%%%%%%%%%%%%%%%%%%%%%%%%%%%%%%%%%%%
\section{Analysis of Celo PoS Data}

In addition to Ethereum data, we also analyse data on the Celo PoS network. This analysis involves some further features involving PoS systems as well as Celo's dual token model.
This additionally serves as a first look at the analysis of a PoS system with historical data spanning longer than a year. In comparison, a similar analysis of Ethereum's new PoS system does not yet have enough history at the current time to perform a good analysis.



\begin{figure}[H]
	\centering
	\includegraphics[width=\textwidth]{figures/celo_full}
	\caption{Graphical network visualization from sparse inverse covariance estimation.}
	\label{fig:partial-corr-graph-celo}
\end{figure}

\begin{figure}[H]
	\centering
	\includegraphics[width=\textwidth]{figures/celo-heatmap}
	\caption{Partial correlation matrix from sparse inverse covariance estimation.}
	\label{fig:partial-corr-matrix-celo}
\end{figure}

\begin{figure}[H]
	\centering
	\includegraphics[width=\textwidth]{figures/mutual-info-celo}
	\caption{Mutual information of price data and features, with smoothing $\alpha$.}
	\label{fig:mutual-info-celo}
\end{figure}


The price recovery is generally poorer than for the ETH/USD price explored earlier. This is likely explained by the higher volatility of Celo compared to Ethereum as well as the smaller size of historical data available.

\begin{figure}[H]
	\centering
	\includegraphics[width=\textwidth]{figures/celo-rolling-pred}
	\caption{Recovered price vs actual for random forest with given retraining periods.}
	\label{fig:pred-retrainings-celo}
\end{figure}