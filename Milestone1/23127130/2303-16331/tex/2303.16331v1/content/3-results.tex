\section{Results}

We focus on Ethereum data analysis under PoW in this section. Analysis of Celo data is included in the appendix as a first look at a PoS system. There is not yet enough historical data to analyze Ethereum PoS but would be a next step.



%%%%%%%%%%%%%%%%%%%%%%%%%%%%%%%%%%%%%%%%%%%%%%%%%%%%%%%%%%%%%%%%%%%
\subsection{Feature Analysis}

We find that a large amount of off-chain pricing information is contained in on-chain data and that the various features are connected in some strong but complicated ways.

Figure~\ref{fig:graphical-model} and Appendix Figure~\ref{fig:partial-corr} show the results of sparse inverse covariance modeling for a selection of the feature set. 
The graphical structure depicted is the consistent structure over time as smoothed over the outputs of many $k$-fold subsets. The partial correlation matrix shows the graphical structure in matrix form.
In the graphical model, the features that are most connected with ETH/USD price include number of active to and from addresses sending transactions, block difficulty, and number of transactions per block. Many other strong relationships are also exhibited among the various other features, potentially indirectly connected to price.



\begin{figure}
	\centering
	\includegraphics[width=0.8\textwidth]{figures/graphical_model.png}
	\caption{Graphical network visualization.}
	\label{fig:graphical-model}
\end{figure}


%The graphical structure (if consistent over varying training sets) tells us about the connections between different on-chain markets and helps to pick features that are most directly connected with price and also which features may be replicating similar information.




Figure~\ref{fig:mutual-info} shows the mutual information between ETH/USD prices and other features, meaning the amount of information (reduction of uncertainty) obtained about price by observing each other variable individually. We find that across the top 10 features, a large amount of information about off-chain price is contained in on-chain data. We also find that the mutual information decreases with $\alpha$, the exponential moving average memory factor for smoothing, indicating that the smoothed data is generally less informative than the most up-to-date data.


\begin{figure}
	\centering
	\includegraphics[width=\textwidth]{figures/mutual-info-better}
	\caption{Mutual information of price data and features, with smoothing $\alpha$.}
	\label{fig:mutual-info}
\end{figure}


We also analyze the full feature set, including the transformed economic factors and Uniswap pool liquidity factors. Perhaps unsurprisingly, since the transformed features contain the same underlying information, they do not exhibit stronger relationships than the raw features. More surprising is that the Uniswap pool factors also did not present strong relationships with price. We then arrived at the above version of the analysis excluding Uniswap factors enabling us to use the entire data history (as Uniswap was launched later than the start of the dataset).


%%%%%%%%%%%%%%%%%%%%%%%%%%%%%%%%%%%%%%%%%%%%%%%%%%%%%%%%%%%%%%%%%%%
\subsection{Recovering Off-chain Prices from On-chain Data}

Random forest and gradient boost both outperformed the other two simpler ML algorithms. We selected Random Forest as the candidate model in the end as it is in principle simpler to be implemented on-chain compared to the gradient boost model (theoretically, a random forest model could be implemented as one big mapping table in a smart contract).

We tested the model performance over different lengths of period - the length of time duration between time t and time t+c. As would be expected with nonstationary time series, we observed that the longer the time duration that a single trained model is used for price estimation, the less accurate is price estimation. The degree to which time between retrainings affects accuracy is informative, however.

Figure~\ref{fig:pred-retrainings} shows the random forest model performances, Estimated vs Actual ETH/USD price, for 1-day ahead, 1-week ahead and 1-month ahead of retrainings. While none of the models provide high accuracy of recovering ETH prices, they do demonstrate that a good signal of the general price level can be recovered, particularly in the 1-day and somewhat in the 1-week retraining cases.

\begin{figure}
	\centering
	\includegraphics[width=\textwidth]{figures/pred-retrainings-better}
	\caption{Recovered price vs actual for random forest with given retraining periods.}
	\label{fig:pred-retrainings}
\end{figure}

The deviation between estimated price and actual price is bigger for higher ETH prices. This is a combination of both having less data in the dataset for these prices and the fact that the same relative error scales with the absolute price, and so deviations measured absolutely are expected to be greater.



We run the models on the full feature set, including transformed economic factors and Uniswap pool factors. The economic factors provide little new information vs the raw features, perhaps a consequence of the flexibility of the tree models. Uniswap pool factors similarly do not improve accuracy. The final analysis excludes Unsiwap factors enabling the entire data history to be used.













