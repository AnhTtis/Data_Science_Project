
\section{More Details on Dataset Features}


Figure~\ref{fig:dataset-overview} and Table~\ref{table:features} provide more information on the feature set used.

\begin{figure}[H]
	\centering
	\includegraphics[width=0.85\textwidth]{figures/dataset-overview}
	\caption{Overview of dataset.}
	\label{fig:dataset-overview}
\end{figure}




[table of features, including the economic ones, refer to online appendix that will be provided with more details of underlying economic models]


\begin{table}
	\centering
	\begin{tabular}{c | l}
		\textbf{Feature type} & \textbf{Feature (high level description)} \\
		\hline
		Network & Number of blocks\\
		 & Number of transactions \\
		 & \% change in accumulated ETH supply \\
		 & Avg gas limit \\
		 & Avg gas used \\
		 & Avg gas price \\
		 & Hash rate \\
		Uniswap & Liquidity in ETH/stablecoin pools \\
		& Trade volume in ETH \\
		Economic & Mining pay-off factors \\
		& Computational burden measures \\
		& Congestion factors \\
		& Social cost factors \\
		& Spreading factor \\
		\hline
	\end{tabular}
	\caption{Data features.}
	\label{table:features}
\end{table}

Online documentation in the project github repo will provide further details of the underlying economic models and calculation of the economic factors (as well as calculation of other factors from the raw data).
A brief overview is as follows along with citations for the relevant models that influenced the choice of these features.

\begin{itemize}
	\item Mining payoff factor 1: $(R (\text{blockReward} + \text{blockFees}))^{-1}$ \cite{Kroll2013TheAdversaries,Prat2017AnMining}
	\begin{itemize}
		\item R = block rate (/s), eth\_n\_blocks = \# blocks in the last hour
	\end{itemize}
	\item Previous high hash rate / current hash rate
	\item previous high $(R (\text{blockReward} + \text{blockFees}))^{-1} / \text{current}$
	\item Excess block space (block limit - gas used)
	\item Social value: D(W) is the social value of the level of decentralization = D(W) = - log(W) $\implies$ D(W) = - log(gas\_used) for ethereum, = - log(bytes); gas used as the measure of the weight of a block (W) \cite{Buterin2018BlockchainPricing}
	\item Social cost: Marginal cost = 1/gas\_used or 1/bytes \cite{Buterin2018BlockchainPricing}
	\item Computational burden on nodes: use block\_size as bandwidth $\implies$ $\text{block\_size} * \log^2(\text{block\_size})$ \cite{Buterin2018BlockchainPricing}
	\item Congestion factors: rho = gas used/gas limit, and $\text{rho}^2$; (in economic model, rho is defined as average number of transaction per block / number of transactions per block) \cite{Huberman2019AnSystem}
	\item Congestion factor: Indicator\_\{$\text{rho} > x$\}, heuristic use x = 0.8 \cite{Huberman2019AnSystem}
	\item Congestion pricing term 1: F(rho) / tx\_fees\_eth, where F describes relationship between USD tx fees and congestion \cite{Huberman2019AnSystem}
	\begin{itemize}
		\item Heuristic: use F = congestion factor 1 or 2 above
	\end{itemize}
	\item Congestion pricing term 2: max number of transactions in a block / fees in block \cite{Nicolas2014TheFees}
	\item Congestion pricing term 3: max number of transactions squared in a block / fees in block \cite{Nicolas2014TheFees}
	\item Spreading factor: number of unique output addresses / number of unique input addresses \cite{SusanAtheyIvaParashkevovVishnuSarukkai2016BitcoinUsage}
\end{itemize}



\section{Further figures on Ethereum Analysis}

\begin{figure}[H]
	\centering
	\includegraphics[width=\textwidth]{figures/partial_corr.png}
	\caption{Partial correlation matrix from sparse inverse covariance estimation.}
	\label{fig:partial-corr}
\end{figure}



%%%%%%%%%%%%%%%%%%%%%%%%%%%%%%%%%%%%%%%%%%%%%%%%%%%%%%%%%%%%%%%%%%%
\section{Analysis of Celo PoS Data}

In addition to Ethereum data, we also analyse data on the Celo PoS network. This analysis involves some further features involving PoS systems as well as Celo's dual token model.
This additionally serves as a first look at the analysis of a PoS system with historical data spanning longer than a year. In comparison, a similar analysis of Ethereum's new PoS system does not yet have enough history at the current time to perform a good analysis.



\begin{figure}[H]
	\centering
	\includegraphics[width=\textwidth]{figures/celo_full}
	\caption{Graphical network visualization from sparse inverse covariance estimation.}
	\label{fig:partial-corr-graph-celo}
\end{figure}

\begin{figure}[H]
	\centering
	\includegraphics[width=\textwidth]{figures/celo-heatmap}
	\caption{Partial correlation matrix from sparse inverse covariance estimation.}
	\label{fig:partial-corr-matrix-celo}
\end{figure}

\begin{figure}[H]
	\centering
	\includegraphics[width=\textwidth]{figures/mutual-info-celo}
	\caption{Mutual information of price data and features, with smoothing $\alpha$.}
	\label{fig:mutual-info-celo}
\end{figure}


The price recovery is generally poorer than for the ETH/USD price explored earlier. This is likely explained by the higher volatility of Celo compared to Ethereum as well as the smaller size of historical data available.

\begin{figure}[H]
	\centering
	\includegraphics[width=\textwidth]{figures/celo-rolling-pred}
	\caption{Recovered price vs actual for random forest with given retraining periods.}
	\label{fig:pred-retrainings-celo}
\end{figure}