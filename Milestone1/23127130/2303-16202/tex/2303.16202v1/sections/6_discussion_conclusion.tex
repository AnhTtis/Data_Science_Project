\section{Conclusion}\label{sec:conclusion} 
\label{sec:discussion}


The proposed method achieves our main goal: improving mesh alignment w.r.t.~the quantum state of the art. 
Furthermore, it is even highly competitive among classical state-of-the-art methods. %
This suggests that the proposed approach can be used as a reference for comparisons and extensions of classical mesh-alignment works in the future. 
(For such cases, classical SA is a viable alternative when access to quantum computers is lacking.) 
Our results show that ignoring certain higher-order terms still allows for high-quality matchings, which is promising for future quantum approaches that could use similar approximations. 
Finally, unlike classical work, we designed our method within the constraints of contemporary quantum hardware. 
We found that iteratively considering shape triplets is highly effective, perhaps even for classical methods. 

{\small 
\noindent\textbf{Acknowledgements}. 
This work was partially supported by the ERC Consolidator Grant 4DReply (770784). 
ZL is funded by the Ministry of Culture and Science of the State of NRW. %
}




