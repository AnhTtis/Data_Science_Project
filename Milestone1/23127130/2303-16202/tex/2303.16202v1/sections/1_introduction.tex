\section{Introduction}\label{sec:intro} 




Recently, there has been a growing interest in applying quantum computers in computer vision  \cite{QuantumSync2021,golyanik2020quantum,Meli_2022_CVPR}. 
Such quantum computer vision methods rely on quantum annealing (QA) that allows to solve $\mathcal{NP}$-hard quadratic unconstrained binary optimisation problems (QUBOs). 
While having to formulate a problem as a QUBO is rather 
inflexible, QA is, in the future, widely expected to solve QUBOs at speeds not achievable with classical hardware.  
Thus, casting a problem as a QUBO promises to outperform more unrestricted formulations in terms of tractable problem sizes and attainable accuracy through sheer speed. 

\begin{figure}
    \centering
    \includegraphics[width=1\linewidth]{figures/daft_teaser.png}
    \caption{\textbf{Our quantum-hybrid method matches all $\boldsymbol{100}$ shapes of the FAUST collection \cite{Bogo:CVPR:2014} with guaranteed cycle consistency (white arrows).} 
    Here, we visualise the matchings via texture transfer between all shapes. 
    Our method scales linearly in the number of shapes. 
    See the full figure in the supplement. 
    }
    \label{fig:teaser}
\end{figure}

A recent example for such a problem is shape matching, where the goal is to estimate correspondences between two shapes. 
Accurate shape matching is a core element of many computer vision and graphics applications (\textit{i.e.,} texture transfer and statistical shape modelling). 
If non-rigid deformations are allowed, even pairwise matching is $\mathcal{NP}$-hard, leading to a wide area of research that approximates this problem, as a recent survey shows \cite{Deng2022}. 
Matching two shapes is one of the problems that was shown to benefit from quantum hardware: Q-Match~\cite{SeelbachBenkner2021} iteratively updates a subset of point correspondences using QA. 
Specifically, its cyclic $\alpha$-expansion allows to parametrise changes to permutation matrices without relaxations. 


The question we ask in this work is: How can we design a \emph{multi-shape} matching algorithm in the style of Q-Match that has the same benefits? 
As we show in the experiments, where we introduce several na\"ive multi-shape extensions of Q-Match, this is a highly non-trivial task. 
Despite tweaking them, our proposed method significantly outperforms them. 



If $N{>}2$ shapes have to be matched, the computational complexity of na\"ive exhaustive pairwise matching increases quadratically with $N$, which %
does not scale to large $N$. 
Furthermore, these pairwise matchings can easily turn out to be inconsistent with each other, thereby violating cycle consistency. 
For example, chaining the matchings $P_{\mathcal{X}\mathcal{Y}}$ from shape $\mathcal{X}$ to $\mathcal{Y}$ and $P_{\mathcal{Y}\mathcal{Z}}$ from $\mathcal{Y}$ to $\mathcal{Z}$ can give very different correspondences between $\mathcal{X}$ and $\mathcal{Z}$ than the direct, pairwise matching $P_{\mathcal{X}\mathcal{Z}}$ of $\mathcal{X}$ and $\mathcal{Z}$: $P_{\mathcal{X}\mathcal{Z}} \neq P_{\mathcal{X}\mathcal{Y}}P_{\mathcal{Y}\mathcal{Z}}$. 
(We apply the permutation matrix $P_{\mathcal{X}\mathcal{Y}}$ to the one-hot vertex index vector $x\in\mathcal{X}$ as $x^\top P_{\mathcal{X}\mathcal{Y}}=y\in\mathcal{Y}$.) 
Thus, how can we achieve cycle consistency by design? 
A simple solution %
would be to match a few pairs in the collection to create a spanning tree covering all shapes and infer the remaining correspondences by chaining along the tree. 
Despite a high accuracy of methods for matching two shapes, this correspondence aggregation policy 
is prone to error accumulation \cite{sahillioglu2014multiple}. 
A special case of this policy is pairwise matching against a single anchor shape, which also guarantees cycle-consistent solutions by construction \cite{Gao2021}. 
We build on this last option in our method as it avoids error accumulation. 







\begin{figure}
    \centering
    \includegraphics[width=\linewidth]{figures/draft_main_5.png}  
    \caption{We match $N$ shapes by iteratively matching triplets. %
    }
    \label{fig:main_figure}
\end{figure}

This paper, in contrast to purely classical methods, leverages the advantages of quantum computing for multi-shape matching 
and introduces a new method for simultaneous alignment of multiple meshes with guaranteed cycle consistency; see Fig.~\ref{fig:teaser}. 
It makes a significant step forward compared to Q-Match and other methods utilising adiabatic quantum computing (AQC), the basis for QA. 
Our \underline{c}ycle-\underline{c}onsistent q\underline{uantu}m-hybrid \underline{m}ulti-shape \underline{m}atching (CCuantuMM; pronounced ``quantum'') 
approach relies on the computational power of modern quantum hardware. 
Thus, our main challenge lies in casting our problem in QUBO form, which is necessary for compatibility with AQC. 
To that end, two design choices are crucial: 
(1) Our method reduces the $N$-shapes case to a series of three-shape matchings; see Fig.~\ref{fig:main_figure}. 
Thus, CCuantuMM is iterative and hybrid, \textit{i.e.,} it alternates in every iteration between preparing a QUBO problem on the CPU and sampling a QUBO solution on the AQC. 
(2) It discards negligible higher-order terms, which makes mapping the three-shape objective to quantum hardware possible. 
In summary, the core technical contributions of this paper are as follows: 
\vspace{-3pt}
\begin{itemize}
    \setlength{\itemsep}{1pt}
    \setlength{\parskip}{0pt}
    \setlength{\parsep}{0pt}
    \item CCuantuMM, \textit{i.e.,} a new quantum-hybrid method for shape  multi-matching relying on cyclic $\alpha$-expansion. 
    CCuantuMM produces cycle-consistent matchings and scales linearly with the number of shapes $N$. 
    \item A new formulation of the optimisation objective for the three-shapes case that is mappable to modern QA. 
    \item A new policy in shape multi-matching to address the $N$-shape case relying on a three-shapes formulation and adaptive choice of an anchor shape. 
\end{itemize} 

Our experiments show that CCuantuMM significantly outperforms several variants of the previous quantum-hybrid method Q-Match \cite{SeelbachBenkner2021}. 
It is even competitive with several non-learning-based classical state-of-the-art shape methods \cite{melzi2019zoomout, Gao2021} and can match more shapes than them. 
In a broader sense, this paper demonstrates the very high potential of applying (currently available and future) quantum hardware in computer vision. 
