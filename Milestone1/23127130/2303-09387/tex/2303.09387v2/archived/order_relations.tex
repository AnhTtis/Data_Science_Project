%don't worry. maybe this section is just for my own enjoyment
\section{Order relations and preferences}

An foundational mathematics a relation $R$ is an operator which takes two arguments $a\in A$ $b\in B$ for sets $A$ and $B$ and evaluates to either True or False. By convention we write $aRb=True$ as $aRb$ and $aRb=False$ as $aRb$. The set of all positive answers can be thought of as a subset of the Cartesian product of $A$ and $B$. Normally relations are defined on the same set, that is to say $A=B$.

An order relation on set $A$ is one that has the following properties:

\begin{enumerate}
    \item \textbf{Reflexivity}: $aRa$ for $a$ in $A$ 
    \item \textbf{Anti-Symmetry}: If $aRb$ and $bRa$ then $a=b$ for $a,b\in A$
    \item \textbf{Transitivity}: If $aRb$ and $bRc$ then $aRc$ for $a,b,c \in A$
\end{enumerate}

An order relation $R$ on a set $A$ is total (aka linear) if any two elements in the set can be compared. 
An example of a total order relation on a set is the $\geq$ operator on the real numbers $\mathbb{R}$. Any attempt to translate preferences into real numbers via a utility function necessarily inherits these three properties. This is formalised by the Von Neumann-Morgenstern utility theorem. In this light, the four 'rationality' requirements that it makes (Completeness, Transitivity, Continuity and Independence) seem more like tools necessary to prove the theorem rather than deep statements about how preferences should work.

A partial-order relation on set $A$ is one where there exist $a,b \in A$ such that neither $aRb$ or $bRa$ is true. $A$ is said to be a partially ordered set or poset for short. Preference mappings to $\mathbb{R}^n$  for $n\geq2$ are posets (assuming there isn't some subsequent mapping onto $\mathbb{R}$). Typically we define an order relation $\leq$ as $(a,b)\leq (c,d)$ iff $a \leq c$ and $b\leq d$, for $a,b,c,d\in \mathbb{R}$. This means that certain pairs cannot be compared, for example in the  situations where $a<c$ and $b>d$ neither $(a,b)\leq(c,d)$ or $(c,d)\leq(a,b)$ is true.

