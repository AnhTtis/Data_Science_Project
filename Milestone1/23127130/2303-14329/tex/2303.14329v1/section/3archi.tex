\section{Architecture}\label{sec_archi}
This section specifies and compares several types of existing edge-based video analytics system architectures. Different architectures to the edge-based video analytics may prioritize different aspects and have distinct optimization approaches. In this regard, we describe the components and features of edge-based video analytics systems and provide examples of typical systems for each architecture type.


\subsection{Edge/Fog-based Architecture}
Edge computing or fog computing, as an extension of cloud computing, allows for computing tasks to be performed at the edge of the network with low latency and real-time computing capabilities. With the help of edge computing, video analytics tasks can be conducted on the edge servers instead of only being executed on end devices with relatively limited computing resources. The edge/fog-based architecture generally consists of two tiers, the end device tier and the edge computing tier, in which the near-site edge tier is essential for achieving  real-time video analytics. Generally, a wide range of devices can serve as edge nodes, e.g., smart phones, laptops and drones. The edge computing platforms (e.g., ParaDrop \cite{Liu2016SEC_ParaDrop}) provide application program interfaces (APIs) to manage  edge nodes as well as running edge services.


\begin{figure}[h]
\centering
  \includegraphics[width=0.45\textwidth]{edge.eps}\\
 \caption{Two types of the edge/fog-based video analytics architecture.}\label{edge}
\end{figure}
Based on the operation modes, the edge-based architecture can be divided into two types, i.e., the dedicated edge-based architecture and the shared edge-based architecture.
As shown in Fig. \ref{edge} (a), each camera is equipped with a dedicated server for the dedicated edge-based architecture. While for the shared edge-based architecture, cameras will share the resource of an individual edge server as illustrated in Fig. \ref{edge} (b).


\subsubsection{Dedicated Edge-based Architecture}
The dedicated edge-based architecture is a simple and reliable way to build a video analytics system.
One example of an edge-based system that leverages real-time video surveillance is Vigil \cite{Zhang2015MobiCom}. It utilizes edge computing to allow for wireless video surveillance to scale to multiple cameras. The Vigil architecture is designed to enable basic vision analytics tasks to be performed at the edge nodes, which are connected to camera devices. To reduce the transmission overhead, only relevant portions of the video feed are uploaded to a controller. 
Grassi \emph{et al.} \cite{Grassi2017SEC} presented the ParkMaster architecture for road sign detection and utilized smartphone cameras with camera calibration for video processing. 
King \emph{et al.} \cite{King2020EdgeSum} designed the EdgeSum framework for video summarization and compression of dash cam (a.k.a. drive recorder) videos using mobile devices as edge servers before uploading to the cloud.


However, the number of edge servers will increase with the number of deployed cameras. With the increase of installed cameras, it will impose a rather heavy burden on installing too many edge servers. In view of  this challenge,  the shared edge-based architecture is preferred in many studies.


\subsubsection{Shared Edge-based Architecture}
Different from the dedicated architecture, the shared edge-based video analytics architecture is built on the virtualization technology.
For example, Jang \emph{et al.} \cite{Jang2018SEC} proposed an edge camera virtualization architecture that leverages an ontology-based application description model to virtualize the camera. They used container technology to decouple the physical camera and support multiple applications on board, thus improving resource utilization and flexibility in edge computing environments.
Wang \emph{et al.} \cite{Wang2017SmartIOT} proposed a smart surveillance system that leverages edge computing and application program interface technologies to enable flexible monitoring of security events in urban regions where have a dense network of cameras, with low latency and minimal backbone bandwidth consumption.
Similarly, a real-time video analytics system called EdgeEye~\cite{Liu2018EdgeSys_EdgeEye} was proposed. EdgeEye offered a high-level abstraction of partial video analytics functions through DNNs and provided tools to deploy and execute DNN models on edge servers.



Specially, most surveillance and rescue applications with cameras on  drones are realized on the shared edge server.
In \cite{Wang2018SEC}, the edge server is connected directly to the LTE base station and packets transmitted from the drones are directed to the edge server without traveling through the Internet backbone.
George \emph{et al.} \cite{George2019HotMobile} proposed a system architecture that leverages edge computing resources for drone-sourced video analytics in live building inspection. The high computational demands are met by using substantial edge computing resources, while the ability to virtualize on an edge server allows for the deployment of a virtual machine that contains the engineering and architectural drawings for the construction site.


\subsection{P2P-based Architecture}

In order to improve the analytics performance by utilizing cross-camera correlations, the analytics pipeline must have the capability to access inference results from related video streams and enable peer-triggered inference at runtime. It means that any relevant camera can assign an analytics task to process a video stream regardless of the time, which divide the logical analytics pipeline from its execution.

In order to achieve this, the inference results must be shared between pipelines in real-time. Although prior research has explored task offloading across cameras and between the edge and cloud, Jain \emph{et al.} \cite{Jain2019HotMobile} argued that the video streams of other related cameras should be considered in such dynamic triggering.


At present, the execution of video analytics pipelines is typically predetermined in terms of resource allocation and video selection. However, to leverage cross-camera correlations, a pipeline should have knowledge of the inference results of other relevant video streams and support real-time triggering based on this information. This enables the compute resources of related cameras to handle the analytic tasks dynamically according to the video streams.


Stone \emph{et al.} \cite{Stone2019SECON} proposed Tetris system that focuses on scalable video analytics at the edge. 
The system identifies active regions across all video feeds and compresses them into a compressed volume which are then passed a Convolutional Neural Networks (CNNs) layer and carefully organized system pipelines to achieve a high parallelism.
Luo \emph{et al.} \cite{Luo2018SEC_EdgeBox} proposed an EdgeBox solution that a group of cameras are managed by an edge device and deployed on the same local area network. This approach is suitable for covering relatively small areas. However, when edge nodes are connected and collaborate to perform complex activity detection utilizing deep learning and computer vision, they can cover larger areas such as a building or a factory.


\begin{table*}
\renewcommand\arraystretch{1.35}
%\normalsize
\caption{Summary of the existing architectures for edge-based video analytics}\label{QoE_table}
\centering
\linespread{1}\selectfont
\begin{tabular}{|p{2cm}<{\centering}|p{2cm}<{\centering}|p{3cm}<{\centering}|p{3cm}<{\centering}|p{2.5cm}<{\centering}|p{3cm}<{\centering}|}
\hline \bf{Type} & \bf{Literature} & \bf{End Device Layer} & \bf{Edge/Fog Layer} & \bf{Controller} & \bf{Cloud Layer} \\
\hline \multirow{7}{*}{\shortstack{Edge/Fog-based \\ Architecture}} 
& Vigil \cite{Zhang2015MobiCom} & video recording and offloading & simple vision analytics & Internet & $\times$ \\
\cline{2-6} & Jang \emph{et al.} \cite{Jang2018SEC} & IoT camera which has specific functionalities (e.g., video recording) & edge-based video analytics & accepts video requests from applications via the cloud (or the user) & $\times$ \\
\cline{2-6} & Wang \emph{et al.} \cite{Wang2017SmartIOT} & video recording and offloading & edge-based video analytics & $\times$ & $\times$ \\
\cline{2-6} & EdgeEye \cite{Liu2018EdgeSys_EdgeEye} & video recording and offloading & an easy and efficient way to execute DNN models & ParaDrop & $\times$ \\
\cline{2-6} & Wang \cite{Wang2018SEC} & drone-sourced cameras & edge servers connected to LTE base stations & $\times$ & without traversing the Internet backbone \\
\cline{2-6} & George \cite{George2019HotMobile} & drone-sourced cameras & edge servers are used to meet the high computation and virtualization demands & $\times$ & $\times$ \\
\cline{2-6} & Dao \emph{et al.} \cite{Dao2017ICDCS} & extract and upload features when environmental changes are detected & installed in each camera & detection metadata collection & $\times$ \\

\hline \multirow{2}{*}{\shortstack{P2P-based \\ Architecture}} 
& Jain \emph{et al.} \cite{Jain2019HotMobile} & video recording and offloading & a video analytics pipeline about resources to use and video to analyze & $\times$ & $\times$ \\
\cline{2-6} & Tetris \cite{Stone2019SECON} & video feeds & a solution for large-scale video analytics in edge & identifies active regions across all video feeds and compresses them into a compressed volume & \checkmark \\

\hline \multirow{8}{*}{\shortstack{Hierarchical \\ Architecture}} 
& LAVEA \cite{Yi2017SEC_LAVEA} & cameras & clients are one-hop away from edge server via wire or wireless links & $\times$ & run heavy tasks on resource rich cloud node to improve response time or energy cost \\
\cline{2-6} & Anveshak \cite{Khochare2019CCGRID} & cameras & edge-based video analytics & automates application deployment and orchestration across edge and cloud & \checkmark \\
\cline{2-6} & Ali \emph{et al.} \cite{Ali2018ICFEC} & cameras & deep learning at the edge & deep learning across resources &  set to achieve an improved performance for object inference \\
\cline{2-6} & Ananthanarayanan \emph{et al.} \cite{Ananthanarayanan2017computer} & cameras & decode video, detect objects, and perform video analytics tasks & $\times$ & \checkmark \\
\cline{2-6} & Chen \emph{et al.} \cite{Chen2016SEC} & surveillance application layer &  process end users data and return the results back on time & $\times$ & \checkmark \\
\cline{2-6} & Perala \emph{et al.} \cite{Perala2018ISCAS} & capture videos in different resolution based on their configuration & computing devices directly connected to the cameras & the gateway on which the devices are connected and share & the cloud server to which the gateway is connected \\
\cline{2-6} & Drolia \emph{et al.} \cite{Drolia2017ICDCS_Cachier} & cameras & use an edge server as a cache with compute resources & $\times$ & over the Internet\\
\cline{2-6} & CloudSeg \cite{Wang2019HotCloud} & camera sensor & send a downsampled high-resolution video adaptively over Internet & $\times$ & process the video with DNN inference, and return the inference results to the edge \\

\hline
\end{tabular}
\end{table*}


\subsection{Hierarchical Architecture}

\begin{figure}[h]
\centering
  \includegraphics[width=0.45\textwidth]{edge-cloud.eps}\\
 \caption{The three-layer hierarchical video analytics architecture consists of camera layer (also called user layer), edge layer, and cloud layer.}\label{edge-cloud}
\end{figure}
It is a challenging task to coordinate highly heterogeneous computing nodes to work as homogeneous computing nodes, which is shown in Fig. \ref{edge-cloud}. The three-layer hierarchical video analytics architecture consists of an end device layer (also called application layer or user layer), an edge layer, and a cloud layer. The camera, edge, and public cloud clusters differ in the available hardware types. For instance, GPUs are commonly found in some clusters (including the cameras), while other types of hardware, such as field-programmable gate arrays (FPGAs) and application-specific integrated circuits (ASICs), are typically found in public clouds \cite{Ananthanarayanan2017computer}.


FilterForward \cite{Canel2019SysML} is a platform that enables multi-tenant video filtering for edge nodes with limited bandwidth. While purely edge-based approaches are limited by static compute and storage resources, datacenter-only analytics require heavy video compression for transport. It addresses these challenges by allowing applications to split the work flexibly into edge and cloud. By leveraging high-fidelity data available at the edge, this approach allows for relevant video sequences to be made available in the cloud.


Similarly, Yi \emph{et al.} \cite{Yi2017SEC_LAVEA} utilized the cloud layer to execute computationally intensive tasks on powerful cloud nodes in order to reduce response time or improve energy efficiency.
Khochare \emph{et al.} \cite{Khochare2019CCGRID} built Anveshak framework, which automates the application deployment and orchestration across edge and cloud resources.
Ali \emph{et al.}  \cite{Ali2018ICFEC} proposed a deep learning pipeline that utilizes resources at the edge, in-transit, and cloud to achieve low latency, reduced bandwidth costs, and improved performance of video analytics tasks.
Ananthanarayanan \emph{et al.} \cite{Ananthanarayanan2017computer} proposed a hierarchical geo-distributed infrastructure which consists of edge clusters and private clusters with heterogeneous hardware for video decoding, object detection, and other video analytics tasks.
Ran \emph{et al.} \cite{Ran2018INFOCOM_DeepDecision} proposed a framework that integrates front-end devices with more powerful backend ``helpers" (such as home servers) to enable local or remote execution of deep learning in the edge/cloud. 
In \cite{Chen2016SEC}, the system comprises of three different layers, where the edge layer is made up of different on-site smart devices that act as both data producers and computing nodes.
Besides the above mentioned works, most edge-based video analytics systems, e.g., \cite{Perala2018ISCAS, Drolia2017ICDCS_Cachier, Wang2019HotCloud, Guo2019TMM, Zhang2016SEC, Long2017TMM}, consist of three parts, including end devices, edge servers, and the cloud. To  authors' best knowledge, the hierarchical architecture has been verified as the most efficient and scalable one for edge-based video analytics.

