\section{Issues and Future Research Directions}\label{sec_issue}
This section presents the essential issues and future research directions in the area of edge-based video analytics.


\subsection{Efficient Video Compression Methods}
In the design of video transmission, it has been proved that an efficient video compression can tremendously reduce the bandwidth consumption.
However, it has not been effectively revealed how to describe the relationship between the video transmission efficiency and the video analytics accuracy.
For example, when the video is coded via the super resolution technology that can be transmitted without consuming much bandwidth, how we can quantitatively obtain the video analytics performance, e.g., accuracy.



\subsection{5G/6G Empowered Video Analytics}
In the video analytics applications, the focused parts are always the mobile objects on the captured frames. However, it has not been given enough attention on the use cases that the cameras themselves are in motion. Although some preliminary works focused on the scenarios that cameras are installed on the drones, challenges still remain, especially for the emerging 6G technologies that support satellite services.


In the future 6G service scenario, the video analytics applications will not only process the video streaming captured by the cameras deployed at stores, crossroads, and places all around the cities, but also by the cameras installed on the satellite. 
Denby \emph{et al.} \cite{Denby2020ASPLOS_OEC} proposed an orbital edge computing system to enable on-board edge computing at each nano-satellite with camera, allowing for local processing of sensed data when downlinking is not feasible. It is urgently needed to take as a coupled consideration with all orbit parameters, physical models, and ground station positions to trigger data collection, predict energy availability, and conduct task offloading, and execute video analytics tasks.


Current works generally focused on the video analytics use cases and service scenarios for the civil use. Most researchers developed the systems by balancing the trade off between accuracy and latency. However, some dedicated industries and applications put high requirements for accuracy and latency, e.g., the video analytics of track safety monitoring for high-speed railway. Thus, it is urgently required to design a dedicated edge-based video analytics architecture for these kinds of applications.



\subsection{Interactive Video Analytics System}
Augmented reality technology allows for interactive elements to be added over real-world views for specific purposes. However, AR overlays digital is overlaid onto the real world views and the digital content is not directly anchored to real-world elements, which means that it cannot interact with real-world elements. To overcome this limitation, AR researchers must develop techniques that allow digital content to interpret and respond to users' head movements and body gestures dynamically, enabling a more interactive and immersive AR experience.

