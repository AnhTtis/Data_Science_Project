\section{Introduction}
\begin{figure*}
  \centering
  \includegraphics[width=0.65\textwidth]{./images/structure.eps}
  \caption{The structure of this article.}\label{structure}
\end{figure*}
In the past few decades, we have witnessed the explosive growth of data particularly with video data. Video cameras, such as closed-circuit television (CCTV) cameras, webcams and dashboard cameras, are everywhere. They are used for various purposes, such as surveillance, security and safety. 
% With the expansion of surveillance and security applications, cameras are installed at stores, crossroads, and places all around today's cities. 
They record ``everyone" and ``everything" all the time. In a 2015 Information Handling Services report~\cite{ihs2015}, authors estimated 245 million security cameras installed globally meaning one camera for every 29 people. However, it is not simply the video data these cameras produce, but insights from such data. In other words, the timely processing and analysis of such data is of great practical importance.

% As a result, video analytics applications have attracted more and more attention. 
As reported by Fortune Business in Sights, the global video analyitcs market size is projected to reach USD\$ 12 billion by the end of 2026, exhibiting a compound annual growth rate of 22.67\% \cite{fortune}. Current video analytics applications are generally built on deep neural network models, which will bring great computational pressure from both model training and inference. Up until recently, these applications have run mostly in clouds~\cite{Yi2017SEC_LAVEA, Hung2018SEC}. However, the ``distant" clouds are facing serious challenges in meeting real-time processing requirements due to network bottleneck, i.e., high bandwidth consumption and high latency as video data has to travel back and forth over the internet.

% However, end devices (e.g., cameras or smartphones) are always constrained by computing resources, thus, the local computation at the end devices might not meet the demand of a low task processing latency. Originally, the complex deep-learning tasks can be executed at the resource-abundant cloud servers.  However, the cloud data centers are always a greater distance apart from the end devices. 
Recently, the edge computing paradigm has emerged as a solution to that. It is rather a complementary approach to cloud computing making use of computing resources at the edge of the network, close to data sources. These resources are called edge servers. They include small-scale on-premises server clusters, server-grade computing resources at mobile base stations and even mobile devices like smartphones and tablets. While running video anayltics application at the edge with these computing resources cannot be perceived as a similar case to that in the cloud because there are some unique challenges in edge-based video analytics.

In this paper, we first provide a survey of edge-based video analytics applications and use cases. We then identify and discuss those challenges with a comprehensive survey of state-of-the-art works on edge-based video analytics. The following are four categories of these challenges.
\begin{itemize}
    \item \emph{Architecture Design}. Different architectures for edge-based video analytics may emphasize on various aspects and can be applied to different use cases and service scenarios.
    \item \emph{Video processing and analysis techniques}. Real-time processing requirements are a key challenge with edge servers that are often static with lower resource capacity compared to cloud servers. In-situ data analytics and collaborative processing are of particular interests for edge-based video analytics.
    \item \emph{Resource management}. With constraints on the quality of experience (QoE) metrics (e.g., accuracy, latency and energy consumption) and resources (hardware and software), it is of great importance to develop efficient resource management policies. It is largely unknown how effective traditional heuristic and optimization methods are for edge-based video analytics.
    \item \emph{Security and privacy}. The use of highly heterogeneous and decentralized edge servers is subject to security vulnerabilities. Besides, the processing, storage and caching of video data on these servers have serious privacy implications. 
\end{itemize}

% \begin{itemize}
%     \item \textbf{Architecture Design}. There are several types of  edge-based video analytic system  architectures. Generally, the architecture design for edge-based video analytics should choose a proximate server to serve the whole or part of analytics tasks. Different architectures to the edge-based video analytic may emphasize on various aspects and can be applied to different use cases and service scenarios. %The architecture design issues will be summarized in Setion \ref{sec_archi}.
%     \item \textbf{In-situ Video Analytics}. For the analytics on real-time video streaming, it is quite challenging to process all the video frames in a timely manner. Without enough processing capacity, the most efficient way is to preprocess the video before conducting the analyzing operations. The pre-processing procedures mainly include frame sampling, cropping, super resolution, compressing, feature extraction, classification, etc. %These techniques and methods will be summarized in Sections \ref{subsec_preprocess} and \ref{subsec_DNN}.
%     \item \textbf{Computation Offloading}. For real-time video analytics, computation tasks can be offloaded to edge servers to reduce latency and increase throughput. A task can be offloaded to a nearby edge server, which returns the results back when the computation task is completed. However, edge servers are heterogeneous in nature, which have different computation capacities. Moreover, the cameras or the edge server can also collaborate to improve the system performance. It is essential to decide which edge server should the computation tasks be offloaded to and when should the offloading procedure begin. %This part will be discussed in Sections \ref{subsec_offload} and \ref{subsec_collab}.
%     \item \textbf{Resource Management}. With constraints on the quality of experience (QoE) metrics (e.g., accuracy, latency, energy) and the resources (e.g., hardware, software), it is of great importance to develop efficient resource management policies. Traditional heuristic and optimization based methods might fail to achieve this goal as many of the factors are coupled with each other. %This attracts tremendous attentions, and will be discussed in Section~\ref{sec_resource}
%     \item \textbf{Security and Privacy}. People are concerned about privacy leakage extremely, especially in the portrait related applications, which are coherent to the images or videos captured by cameras. Privacy allows users to have specific control over their sensitive information, preventing them from being abused by the third parties. %The current privacy-preserving video analytics schemes will be illustrated in Section \ref{sec_privacy}.
% \end{itemize}

While there have been several surveys on video analytics at the edge, they are not as comprehensive or in-depth as our survey in this paper. 
Shi \emph{et al.} \cite{Shi2016IEEEInternet} surveyed the works related to edge computing and its use cases. 
Zhou \emph{et al.} \cite{Zhou2019IEEE} summarized the research efforts on artificial intelligence (AI) from the perspective of edge computing and its corresponding intelligent applications. 
These works mainly discussed the technologies and applications with edge computing, however, the discussion on video analytics-related applications is limited.
Vega \emph{et al.} \cite{Vega2018IEEE} reviewed state-of-the-art QoE management methods for video-related services based on machine learning. Their goal is to optimize the QoE from the perspective of clients (i.e., devices), while efficiently utilizing network resources from the perspective of providers.
Barakabitze \emph{et al.} \cite{Barakabitze2019IEEE} studied QoE management solutions for edge-based multimedia applications. They mainly discussed about how to provide users with better video services. They considered this aspect as well as efficient video processing and analytics. 
As the most similar survey to our work, Jedari \emph{et al.} \cite{Jedari2020surveys} studied the state-of-the-art researches on edge video caching, edge computing, and communication. However, they mainly considered the combined use of resources at the edge to support several video-oriented applications, while only a small part of the content discussed the studies on video analytics. 

This paper provides a comprehensive and detailed review on what works, what doesn't work and why. These findings give insights and suggestions for next generation edge-based video analytics. We also identify open issues and research directions.

The structure of this paper is shown in Fig. \ref{structure}. In particular, Section~\ref{sec_scena} presents the general use cases and service scenarios for edge-based video analytics.
Section~\ref{sec_archi} summarizes the proposed architecture for edge-based video analytics.
Section~\ref{sec_tech} illustrates the technologies and methods for edge-based video analytics.
Section~\ref{sec_resource} reveals relation between resources and performance for edge-based video analytics, and discusses the scheduling strategies in resource-limited video analytics scenarios.
Section~\ref{sec_privacy} discusses the security and privacy issues occurred in edge-based video analytics system design.
Section~\ref{sec_issue} summarizes the challenging issues and outlooks future research directions.


