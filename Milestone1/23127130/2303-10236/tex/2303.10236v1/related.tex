% $Id: related.tex 
% !TEX root = main.tex

\section{Related Work}
\label{sec:related}

%ivana notes: i'd move this section before - so can pick up on it in conclusions of our work, that it could just be lack of training

%\subsection{Quality Analysis of \ac{ML} Projects}

The increasing interest in \ac{ML} generated interest to apply software 
engineering to it~\cite{menzies19}, giving rise to different quality 
analysis proposals, which we discuss in the following.

A general empirical study on the quality requirements from \ac{AI} 
practitioners~\cite{golendukhina22} highlights the 12 most pressing 
issues identified from interviews conducted with industry \ac{AI} 
practitioners. This study is motivated by the identification the lack 
of proper training, practices, and processes in \ac{AI} based projects. 
From the 12 categories identified, three of them are related to code 
quality (\ie non-deterministic Python behavior, independencies between 
projects/modules, and lack of clear code). This reinforces our 
hypothesis that code quality is problematic for developing \ac{RL} 
systems.

The analysis of code smells for general \ac{ML} projects highlights 20 
code smells recurrent in \ac{ML} projects as extracted from a standard 
Python linter~\cite{vanoort21}. Additionally, they highlight 22 
specific code smells mined from the literature, bug databases, forums, 
and blogs for \ac{ML} applications~\cite{zhang22}. The \ac{LM}, 
\ac{LPL}, \ac{LMC}, and \ac{LLF} code smells all feature in the 
identified lists, suggesting a common complexity for accessing and 
expressing both \ac{RL} and \ac{ML} algorithms.

Assuring quality has focused on testing \ac{AI} 
systems~\cite{murphy06}. Different proposals exist for bug detection, 
identification, and fixing in \ac{ML} projects. There are 7 major bug 
types identified for \ac{ML} projects, extracted from the empirical 
analysis of 3 \ac{ML} projects~\cite{sun17}.  
Four of the bug types are directly related to the code quality (\ie 
variable bugs, design defects, performance optimization, and memory 
overflow). A similar study analyzes TensorFlow-related bugs in deep 
learning applications~\cite{zhang18}, with 175 bugs related to code 
quality detected. Similarly, common bug fixing patterns in \ac{DNN}
~\cite{islam20} include errors in the dimensionality of data or the 
connectivity of the layers in the network, which lead to  
application crashes. Such bugs, are directly associated with the 
expressiveness of the tools to develop \ac{DNN} projects.

Closely related to our study,~\citet{wang20} analyze the  quality of 
Jupyter Notebooks code, identifying issues in their design and 
maintainability. As a conclusion, there is a 
need for analysis tools to ensure the quality of Jupyter notebooks 
code. However, another argument over the results could be to improve 
the expressiveness of Jupyter notebooks so that there is no accidental 
complexity when structuring \ac{ML} programs. 

%%

%ivana notes: this probably doesnt need to be submission - maybe paragraphs, since they're so small subsections

%\subsection{Programming Languages for \ac{ML}}

%Identifying the problems raised by code level bugs and lack of quality 
%of \ac{AI}/\ac{ML} projects, highlights the need for new abstractions 
%and tools assuring code quality during the development 
%process~\cite{golendukhina22}. A possibility to address this issues is 
%tackle them from the programming language perspective.

There is a study on the impact of programming languages 
to \ac{ML} projects~\cite{sztwiertnia21}, identifying that programming 
languages can add to the bugs detected for \ac{ML} 
projects. The results show that 15\% of code quality bugs in average
can be attributed to the programming language. This suggests, that 
the use of more appropriate programming languages can help reduce the 
bugs in \ac{ML} projects, hence improving their quality.

Our preliminary code quality analysis of \ac{RL} projects, is inline 
with existing analyses for general \ac{ML}. We reach a similar 
conclusion that code smells hinder the quality 
of \ac{RL} projects. This highlights the need for code analysis tools 
specific to mange the characteristics of \ac{RL} projects, as well as 
the need for better programming abstractions.


\endinput

