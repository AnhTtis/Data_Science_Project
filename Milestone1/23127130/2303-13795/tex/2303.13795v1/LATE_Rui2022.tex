\pdfoutput=1
\documentclass[12pt, oneside]{article}   	% use "amsart" instead of "article" for AMS\TVX format

\usepackage[svgnames]{xcolor}
\usepackage[colorlinks]{hyperref}

\usepackage{geometry}                		% See geometry.pdf to learn the layout options. There are lots.
\geometry{letterpaper}                   		% ... or a4paper or a5paper or ... 
%\geometry{landscape}                		% Activate for for rotated page geometry
%\usepackage[parfill]{parskip}    		% Activate to begin paragraphs with an empty line rather than an indent
\usepackage{graphicx}				% Use pdf, png, jpg, or eps with pdflatex; use eps in DVI mode
							% TeX will automatically convert eps --> pdf in pdflatex		

\usepackage{amssymb}
\usepackage{bm}
\usepackage{amsmath}
\usepackage{bbm}
\usepackage{amsthm}	
\usepackage{natbib}
\usepackage{tikz}
\usepackage{capt-of}
\usepackage{extarrows}
\usepackage{enumitem}  
\usepackage{multirow}
\usepackage{booktabs} 
\usepackage{caption}
\captionsetup[table]{skip=10pt}


\hypersetup{
	citecolor=DarkBlue,
	bookmarksnumbered=true,
	urlcolor=Indigo,linkcolor=blue
}

\bibliographystyle{abbrvnat}
\setcitestyle{authoryear,open={(},close={)}}


\newcommand{\thatsall}{\hfill$\square$}

\newtheorem{thm}{Theorem}
\newtheorem{prop}{Proposition}
\newtheorem{ass}{Assumption}
\newtheorem{rem}{Remark}
\newtheorem{lemma}{Lemma}
\newtheorem{col}{Corollary}
\newtheorem{example}{Example}                                              
\newtheorem{defn}{Definition}

\usepackage{setspace}
\usepackage{csquotes}
\onehalfspacing

 \def\Var{\mathop{\rm Var}} 
\def\LATE{\mathrm{LATE}}
\def\IV{\mathrm{IV}}
\def\AT{\mathrm{AT}}
\def\NT{\mathrm{NT}}
\def\CP{\mathrm{CP}}
\def\DF{\mathrm{DF}}
\newcommand\independent{\protect\mathpalette{\protect\independenT}{\perp}}
\def\independenT#1#2{\mathrel{\rlap{$#1#2$}\mkern5mu{#1#2}}}
\def\sgn{\mathop{\rm sgn}}

%\newcommand{\thatsall}{\hfill$\square$}

\title{Point Identification of LATE with Two Imperfect Instruments\thanks { I am grateful to Valentina Corradi, Marc Henry, Keisuke Hirano, Sung Jae Jun, Matt Masten,  Joris Pinkse,  Adam Rosen, and Jo\~ao Santos Silva for their many valuable comments.} %‹
}
\author{Rui Wang%
\thanks{Department of Economics, The Ohio State University.  Email: \texttt{wang.16498@osu.edu}. }
}
%\date{\today}							% Activate to display a given date or no date
\date{December 30, 2022}
\begin{document}
\maketitle
%\section{}
%\subsection{}
\begin{abstract}
%The paper characterizes new point identification results of the local average treatment effect by using two instruments but requiring weaker assumptions on both instruments compared to \cite{imbens1994}. \cite{imbens1994} require an instrument to satisfy the conditions of exclusion, monotonicity, and independence, while their results do not hold if one of the conditions fails. My paper uses two instruments; however, the first instrument is allowed to violate the exclusion restriction and the second instrument does not need to satisfy the monotonicity condition. Therefore, the first instrument can affect the outcome via both direct effects and a shift in the treatment status. My method can identify the direct effects of the first instrument via exogenous variation in the second instrument and consequently identify the local average treatment effect. An estimator for the local average treatment effect is developed, and using Monte Carlo simulations, it is shown to perform more robustly than the instrumental variable estimand.


This paper characterizes point identification results of the local average treatment effect (LATE) using two imperfect instruments. The classical approach (\cite{imbens1994}) establishes the identification of LATE via an instrument that satisfies exclusion, monotonicity, and independence. However, it may be challenging to find a single instrument that satisfies all these assumptions simultaneously. My paper uses two instruments but imposes weaker assumptions on both instruments. The first instrument is allowed to violate the exclusion restriction and the second instrument does not need to satisfy monotonicity. Therefore, the first instrument can affect the outcome via both direct effects and a shift in the treatment status. The direct effects can be identified via exogenous variation in the second instrument and therefore the local average treatment effect is identified. An estimator is proposed, and using Monte Carlo simulations, it is shown to perform more robustly than the instrumental variable estimand.


%This paper studies heterogeneous treatment effects with endogeneity. The classical approach (\cite{imbens1994}) establishes the identification of local average treatment effects (LATE) via an instrument that satisfies exclusion, monotonicity, and independence. However, it may be challenging to find a single instrument that satisfies all these assumptions simultaneously. My paper provides point identification results of LATE using two imperfect instruments but imposes weaker assumptions on both instruments. The first instrument is allowed to violate the exclusion restriction and the second instrument does not need to satisfy monotonicity. Therefore, the first instrument can affect the outcome via both direct effects and a shift in the treatment status. The direct effects can be identified via exogenous variation in the second instrument and therefore the local average treatment effect is identified. An estimator is proposed, and using Monte Carlo simulations, it is shown to perform more robustly than the instrumental variable estimand.
\hspace{10pt} 

%TC:ignore
\textbf{Keywords}: local average treatment effect, instrumental variables, exclusion restriction, monotonicity, point identification

%JEL: C31, C36

\end{abstract}

\newpage

\section{Introduction} \label{sec:intro}
Instrumental variables are widely used to estimate causal effects with endogenous treatment. When treatment effects are heterogeneous, \cite{imbens1994} and \cite*{angrist1996} show that the local average treatment effect (LATE) is identified as the instrumental variable (IV) estimand with a valid instrument. The instrumental variable is required to exert no direct effects on the outcome (exclusion), weakly increase treatment status (monotonicity), and be independent of the potential outcome and potential treatment (independence).


In practice, it is a challenging task to find a valid instrument that satisfies all the assumptions simultaneously. In estimating returns to schooling, several papers such as
\cite{uusitalo1999} use family background variables as an instrument; however, these variables may have direct effects on people's earnings via family education  and thus violate the exclusion restriction. 
% and thus violates the exclusion restriction
% including \cite{angrist1991} and \cite{staiger1997}, use an individual's quarter of birth as an instrument. However, \cite{bound1996} raise the question of whether the individual's quarter of birth is correlated with their family background or unobserved ability. 
The exclusion assumption may also fail in encouragement experiments. For example, \cite{hirano2000} study the effects of the flu vaccine on  the prevalence of influenza and use random encouragement to take the vaccine as an instrument. Their paper  shows that the encouragement to take the vaccine has direct effects on the outcome since it may remind people to take other actions to prevent the flu. 

%allows for different violations of the exclusion restriction and

Studies such as \cite{kitagawa2015}, \cite{huber2015}, \cite{kedagni2016}, and \cite{mourifie2017} develop different methods to test the assumptions of the instrumental variable and reject the validity of some instruments. For example, college proximity is used as an instrument in \cite{card1993}, \cite{kling2001}, and \cite*{carneiro2011}, but its validity is rejected by \cite{huber2015} and \cite{mourifie2017}.

Motivated by these findings, this paper proposes a new approach to identify LATE. This approach uses two instruments while imposing weaker assumptions on both instruments compared to the standard IV assumptions. The first instrument is allowed to violate the exclusion restriction, and the second instrument does not need to satisfy monotonicity. Therefore, the first instrument can affect the outcome through both direct effects and treatment effects.
The second instrument is introduced to separate the two effects. By exploiting exogenous variation in the second instrument, the direct effects of the first instrument are identified, and therefore LATE defined by the first instrument is point identified. 


The paper develops estimators for LATE and the direct effects, and establishes their asymptotic properties. I compare this approach by using two imperfect instruments with the IV estimand using a single instrument via Monte Carlo simulations. The results show that the IV estimand can have a large bias with nonzero direct effects and that the bias increases when the direct effects increase.
The method with two instruments presented in this paper performs uniformly regardless of the direct effects; thus, it has a more robust performance concerning violations of the exclusion restriction. 

Here I discuss some potential choices of the two instruments in various applications. One example is the effect of participation in the food stamp program on health outcomes. \footnote{See, e.g., \cite*{debono2012}, \cite{kreider2012}, and \cite*{gundersen2017}.} The first instrument can be an increase in the benefits of the program. Such an increase is likely to affect participation monotonically, but the increased benefit may directly affect health outcomes. The second instrument can be whether the benefit of the program is issued electronically or in hard copy. The delivery method is unlikely to have a direct effect on health outcomes, but it may not satisfy monotonicity,  as some people prefer electronic delivery but others prefer hard-copy delivery. 


%I present some potential choices of the two instruments in various applications.
%One example is studying the effects of the food stamp program on health outcomes. The effects of assistance programs have been explored by \cite*{debono2012}, \cite*{kreider2012}, and \cite*{gundersen2017}. The first instrument can be an increase in the benefits of the program, and the increased benefits may directly affect health outcomes. The second instrument can be whether the benefit of the program is issued electronically or in hard copy.
%This instrument may not satisfy the monotonicity condition since some people prefer electronic delivery but others prefer hard-copy delivery.

The second example is the study of the effect of having a third child on mothers' labor supply, as studied in \cite*{angrist1998}. The first instrument can be whether there is a financial subsidy for having a third child. Financial support is likely to encourage a third child, but may directly affect people's incentive for labor participation. The second instrument could be whether the first two children have the same sex. This instrument seems unlikely to directly affect labor participation, but it may violate monotonicity since different people may prefer different sibling-sex composition.

%Another example is estimating the effect of early achievement for children (e.g., kindergarten performance) on subsequent outcomes (e.g., earnings). This effect has been studied by \cite{chetty2011} and \cite{kolesar2015} with the STAR experiment that randomly assigns teachers and classrooms to students. The first instrument can be whether students are assigned with more experienced teachers. Experienced teachers are likely to improve kindergarten performance, but may also have direct effects on children' future earnings. The second instrument could be the class size, which is shown to have no significant effects on earnings in \cite{chetty2011}. However, this instrument may not satisfy monotonicity as class size effects could be heterogeneous. For example, \cite*{maasoumi2005} show opposite signs of class size effects for students below and above the median.

Another example is estimating the effect of early achievement for children (e.g., kindergarten performance) on subsequent outcomes (e.g., earnings), which has been  studied by \cite{chetty2011}. One potential instrument is the random assignment of teachers/classrooms. But as argued by \cite{kolesar2015}, the exclusion restriction might be violated because an experienced teacher may not only improve kindergarten performance, but also have direct effects on children's subsequent outcomes. Nevertheless, we can use a second instrument to separate the direct effects. One potential choice is the class size, which is shown to have no significant effects on earnings in \cite{chetty2011}, but may not satisfy monotonicity as class size effects could be heterogeneous.\footnote{For example, \cite*{maasoumi2005} show opposite signs of class size effects for students below and above the median.}


%This has been studied by \cite{chetty2011}  the STAR experiment that randomly assigns teachers and classrooms to students. As noted by \cite{kolesar2015}, the assignm



%\footnote{See, e.g., \cite*{ding2005}, \cite*{maasoumi2005}, and \cite*{bandiera2010}.} 
% For example, \cite*{maasoumi2005} show that reductions in class size increase test scores for students below the median and decrease test scores above the median.



%The second example is the study of returns to college.\footnote{This has been broadly studied in the literature, such as \cite{angrist1991}, \cite{card1993}, \cite{brunello1999}, and \cite{card2001}.} The first instrument can be parental education, which may violate the exclusion restriction as parental education is likely to have direct affect on potential earnings via family education. The second instrument can be birth order. The relationship between birth order and education seems ambiguous and could vary across individuals so  the monotonicity condition may fail.

%
%The second example is estimating returns to college, which has been broadly studied in the literature including \cite{angrist1991}, \cite{card1993}, \cite{brunello1999}, and \cite{card2001}. The first instrument can be parental education, which may violate the exclusion restriction since parental education may directly affect people's earnings via family education and family environment. The second instrument can be birth order. The relationship between birth order and education seems ambiguous and could vary across individuals so  the monotonicity condition may fail.


%Another example is the effects of inoculation on influenza, as studied in \cite*{mcdonald1992} and \cite*{hirano2000}. The first instrument can be the random encouragement to take the vaccine. This instrument may remind people to take other preventive actions for influenza so it may have a nonzero effect on the outcome. The second instrument could be whether an individual is religious. This instrument may affect individuals to take flu shots differently depending on their religion so the monotonicity condition could fail.
 %may not satisfy the exclusion restriction



%The second example is studying the effects of alcohol abuse on employment, as discussed in \cite{terza2002} and \cite{auld2005}. Similar to the example of returns to college, the first instrument can be whether parents have alcohol abuse problems, which may violate the exclusion restriction. 
%The second instrument can be the state cigarette tax. This instrument may not satisfy the monotonicity condition because some people are likely to consume more alcohol with a higher cigarette tax if alcohol and cigarettes are substitutes for them, while others may consume less if alcohol and cigarettes are complements for them.





\subsection{Related Literature}
This paper contributes to the literature studying heterogeneous treatment effects with endogeneity using instrumental variables.
 \cite{imbens1994}, \cite*{angrist1996}, and \cite{heckman2005} identify treatment effects using one instrument that satisfies the conditions of exclusion, monotonicity, and independence simultaneously. My paper complements the literature by using two instruments while relaxing one assumption on each instrument. 

%\cite{imbens1994} use one instrument to identify $\LATE$ but requires this instrument to satisfy exclusion restriction, monotonicity, and independence with potential treatments. My method needs two instruments but imposes weaker assumptions for the two instruments. One instrument is allowed to have direct effects on outcomes so it can violate exclusion restrictions. And another instrument does not need to satisfy monotonicity and the independence with potential treatments. In some empirical  applications, finding two invalid instruments is more practical  than finding one valid instrument.

Several papers relax the exclusion restriction under the heterogeneous treatment effects framework. \cite{hirano2000} study different violations of exclusion restrictions for subgroups and apply parametric models and the Bayesian approach for inference.
\cite{flores2013} derive partial identification for LATE by employing a weak monotonicity assumption 
of mean potential outcomes within or across subgroups. \cite{mealli2013} study partial identification of  intention-to-treat effects through a secondary outcome. %My paper achieves point identification for the local average treatment effect by using an additional instrument. 

%Some papers discuss other assumptions of the instrumental variable. For example, \cite{de2017} drops the monotonicity condition and shows that the $\IV$ estimand is the average treatment effect for a subgroup of compliers under particular assumptions. 

\cite{de2017} relaxes the monotonicity assumption and shows that the IV estimand estimates the local average treatment for a subgroup of compliers under a ``compliers-defiers'' condition. He further provides sufficient conditions for the ``compliers-defiers'' condition to hold. \cite{kedagni2021} relaxes the independence assumption of the instrument with potential treatment. His paper establishes partial identification results by using an additional instrument that serves as a proxy of the first instrument.



%Another important assumption, monotonicity, has been studied by \cite{de2017}. The paper relaxes monotonicity and shows that the IV estimand estimates the local average treatment for a subgroup of compliers under a ``compliers-defiers'' (CD) condition. \cite{de2017} further derives sufficient conditions for the CD condition. 
%\cite{kedagni2021} relaxes the independence condition of the instrument with potential treatment. His paper establishes partial identification results by using an additional instrument that serves as a proxy of the first instrument.
% In this paper, I relax the exclusion restriction of the first instrument and do not require monotonicity for the second instrument. I show that point identification of treatment effects is achieved by using the two instruments.


The paper is also related to the literature that relaxes the exclusion restriction of the instrument in a linear regression model. \cite{hahn2005} derive the bias of different estimators with direct effects. \cite{nevo2012} provide partial identification for the model parameter under the assumptions of a correlation between the instrument and the error term.  \cite*{conley2012} employ different assumptions on the effect of the instrument on the outcome to conduct inference for the model parameter. 
\cite{kolesar2015}  allow for direct effects and develop an estimator under an orthogonality condition of the direct effects.
%My paper focuses on identifying heterogeneous treatment effects. 






% and derives the bias when the exclusion restriction or the monotonicity condition fails
The remainder of the paper is organized as follows. Section \ref{sec:treat} presents the heterogeneous treatment effect framework. Section \ref{sec:iden} derives the identification result. Section \ref{sec:esti} develops an estimator for LATE and examines its finite sample performance via simulations. Section \ref{sec:exte} studies an extension and Section \ref{sec:conc} concludes. 



\section{Heterogeneous Treatment Effects Model} \label{sec:treat}
The analysis focuses on the heterogeneous treatment effects framework introduced in \cite{imbens1994} and  \cite*{angrist1996}. Let $Y\in \mathcal{Y}$ denote an outcome, $D \in\{0 ,1\}$ denote a binary treatment, and $Z\in \{0, 1\}$ denote a binary instrument. The objective is to learn the effects of treatment $D$ on outcome $Y$, but the treatment can be endogenous. Instrument $Z$ is used to address the endogeneity issue. Observed variables are $(Y, D, Z)$.
%The covariate is suppressed for simplicity of notation, while the analysis can be conducted conditional on covariates.
 
Following \cite{imbens1994} and \cite*{angrist1996}, I use counterfactual variables to describe the data generating process. Let $D_z$ denote the potential treatment given the instrument $Z=z$ and $Y_{d, z}$ denote the potential outcome given the instrument and treatment $Z=z, D_z=d$. Let $Y_d$ denote the potential outcome given $D=d$, which is given as $Y_d=Y_{d, 1}Z+Y_{d, 0}(1-Z)$. Observed variables $(Y, D)$ are generated by
\begin{equation*}
\begin{aligned}
D&=D_1Z+D_0(1-Z), \\
Y&=Y_1 D+Y_0 (1-D).
\end{aligned}
\end{equation*}


The population can be divided into four subgroups based on the potential treatments $(D_1, D_0)$: always takers (AT): $D_1=D_0=1$; compliers (CP): $D_z=z$ for $z\in \{0, 1\}$; never takers (NT):  $D_1=D_0=0$; and defiers (DF): $D_z=1-z$ for $z\in \{0, 1\}$.

The following summarizes the assumptions in \cite{imbens1994} and \cite*{angrist1996}.

\begin{ass}[IV Validity] \label{ass:IV}

\
%(i) Exclusion: $Y_{d,1}=Y_{d, 0}\equiv Y_d$ for any $d\in \{0, 1\}$;  
%
%(ii) Monotonicity: $D_1\geq D_0$;
%
%
%(iii) Independence: $Z\independent (Y_{1,1}, Y_{1,0}, Y_{0, 1}, Y_{0, 0}, D_1, D_0)$;
%
%
%(iv) Relevance: $\Pr(D_1>D_0)>0$; 
%\vspace{-0.2cm}
% $0<\Pr(Z=1)<1$.
%
%
\begin{enumerate}[label=(\roman*)]

\item Exclusion: $Y_{d,1}=Y_{d, 0}\equiv Y_d$ for any $d\in \{0, 1\}$;  
\vspace{-0.2cm}
\item Monotonicity: $D_1\geq D_0$;
\vspace{-0.2cm}
\item Independence: $Z\independent (Y_{1,1}, Y_{1,0}, Y_{0, 1}, Y_{0, 0}, D_1, D_0)$;
\vspace{-0.2cm}
\item Relevance: $\Pr(D_1>D_0)>0$; 
\vspace{-0.2cm}
\item $0<\Pr(Z=1)<1$.

\end{enumerate}

\end{ass}

Assumption \ref{ass:IV} (i) requires instrument $Z$ to have no direct effects on the potential outcome, (ii) indicates that the instrument weakly increases the potential treatment (no defier), (iii) refers to the independence of the instrument with all potential variables, (iv) guarantees the existence of compliers, and (v) needs nonzero variation in instrument $Z$. 


Under Assumption \ref{ass:IV}, \cite{imbens1994} show that the average treatment effect for compliers can be identified as the IV estimand:
\begin{equation}\label{equ:late}
\begin{aligned}
\LATE\equiv E[Y_1-Y_0\mid \CP]=\frac{E[Y\mid Z=1]-E[Y\mid Z=0]}{E[D\mid Z=1]-E[D\mid Z=0]}\equiv \IV.
\end{aligned}
\end{equation}

This identification result for LATE relies on the validity of instrument $Z$ in Assumption \ref{ass:IV}. However, it may be difficult to find a single instrument satisfying all conditions in Assumption \ref{ass:IV}. The identification result may fail if one of the assumptions does not hold. 

 \cite*{angrist1996} (Section 5) discuss the sensitivity of the IV estimand to deviations from the IV validity assumptions. The IV estimand involves both direct effects and treatment effects when the exclusion restriction fails, and it is the combination of treatment effects for compliers and defiers when the monotonicity condition is violated. This paper proposes a new approach to identify LATE by using two instruments jointly but requiring weaker assumptions on both instruments. The first instrument is allowed to violate the exclusion restriction, and the second instrument does not need to satisfy the monotonicity condition. 



%I first derive the bias between the IV estimand and the local average treatment effect when the exclusion restriction or the monotonicity condition violates. It shows how the invalidity of the instrument can affect the identification of LATE. The paper then proposes a new approach to identify LATE by using two instruments jointly but requiring weaker assumptions on both instruments. The first instrument is allowed to violate the exclusion restriction, and the second instrument does not need to satisfy the monotonicity condition. 

%
%\subsection{Bias with Invalid Instrument}\label{subsec:bias}
%
%This section derives the bias between the IV estimand and LATE when the exclusion restriction or the monotonicity condition violates. When the exclusion restriction fails, instrument $Z$ will affect the outcome through both direct effects and treatment effects. The direct effect is $Y_{d, 1}-Y_{d, 0}$ for each individual with $D=d$. Let $\rho_{G, d}=E[Y_{d,1}-Y_{d, 0}\mid G]$ denote the average direct effect for group $G\in\{\AT, \NT, \CP\}$ given treatment $D=d$. 
%For the next proposition, I consider the average of the direct effects to be the same across subgroups and under different treatments: $\rho_{G, d}=\rho$ for any $G\in\{\AT, \NT, \CP\}$ and $d\in \{0, 1\}$, where $\rho$ is an unknown constant.
% 
%\begin{prop}\label{prop:bias1}
%Under Assumptions \ref{ass:IV} (ii)-(v) and $\rho_{G, d}=\rho$ for any $G\in\{\AT, \NT, \CP\}$ and $d\in \{0, 1\}$, the following holds:
%\begin{equation*}
%\IV-\LATE=\frac{\rho}{\Pr(\CP)}.
%\end{equation*}
%\end{prop}
%
%Proposition \ref{prop:bias1} shows that the IV estimand is biased for LATE with nonzero direct effects $\rho\neq 0$. The bias depends on two components: the direct effect $\rho$ and the size of the compliers $\Pr(\CP)$. A smaller direct effect and larger-sized compliers lead to smaller bias. The value of LATE can be bounded if the direct effect $\rho$ is bounded by auxiliary information since the size of the compliers is identified.
% When the direct effect is zero, $\rho=0$, then LATE is identified as the IV estimand, as shown in \cite{imbens1994}. 
%Appendix \ref{proof:prop1} also derives the bias when the average of the direct effects varies under different treatments, $\rho_{G, 1}\neq \rho_{G, 0}$.
%
%
%Now we look at the case where the monotonicity condition fails but other conditions hold. Without monotonicity, there may be defiers who switch from the treatment group to the control group when $Z$ increases. 
%Let $\LATE_{\DF}=E[Y_1-Y_0\mid \DF]$ denote the average treatment effect for the defiers, and let $\tau=\Pr(\CP)/\Pr(\DF)$ denote the size of the compliers relative to the defiers. 
%
%\begin{prop}\label{prop:bias2}
%Under Assumptions \ref{ass:IV} (i) and (iii)-(v), the following holds:
%\begin{equation*}
%\IV-\LATE=\frac{\LATE-\LATE_{\DF}}{\tau-1}.
%\end{equation*}
%\end{prop}
%
%The bias in Proposition \ref{prop:bias2} depends on the degree of heterogeneity in the average treatment effect between the compliers and the defiers as well as the size of the compliers relative to the defiers. Smaller heterogeneity in treatment effects between the two groups implies smaller bias, and point identification is achieved under homogeneous treatment effects without monotonicity. Moreover, the bias is smaller when the ratio of the compliers to the defiers is larger, and the bias goes to zero when the size of the defiers goes to zero ($\tau=\infty$).
%
%
%The two propositions show the bias between the IV estimand and LATE when using an invalid instrument. The paper next presents an approach to identify the direct effects of the first instrument when the exclusion restriction fails and thus identify LATE. My main strategy  is to use an additional instrument, while I require weaker assumptions on both instruments compared to Assumption \ref{ass:IV}.

\section{Identifying LATE with Two Invalid Instruments} \label{sec:iden}

This section focuses on identifying the local average treatment effect defined by instrument $Z$, while allowing instrument $Z$ to have nonzero direct effects on the outcome. I introduce an additional binary instrument $W\in \{0, 1\}$ to identify the direct effects and LATE defined by instrument $Z$. 
My approach uses two instruments $(Z, W)$ but relaxing one main assumption on both instruments. Instrument $Z$ is allowed to have direct effects on the outcome, and instrument $W$ does not need to satisfy monotonicity.  Instrument $W$ is assumed to have no direct effects on the outcome such that the potential outcome is not indexed by instrument $W$.

When the exclusion restriction is relaxed, instrument $Z$ will affect the outcome through both direct effects and treatment effects. The direct effect is $Y_{d, 1}-Y_{d, 0}$ for each individual with $D=d$. Let $\rho_{G, d}=E[Y_{d,1}-Y_{d, 0}\mid G]$ denote the average direct effect for group $G\in\{\AT, \NT, \CP\}$ given treatment $D=d$. 

Next, I present assumptions on the two instruments $(Z, W)$.



\begin{ass}[Instruments $Z \& W$] \label{ass:z} 
\

\begin{enumerate}[label=(\roman*)]
\item Direct effects: $\rho_{G, d}=\rho_d$ for any $G\in \{\AT, \NT, \CP\}$, where $\rho_d$ is a unknown constant for $d\in \{1, 0\}$; 
\vspace{-0.2cm}
\item Monotonicity: $D_1\geq D_0$;  
\vspace{-0.2cm}
\item Independence: $(Z, W)\independent Y_{d, z} \mid (D_1, D_0)$ for any $d, z\in \{0, 1\}$, and $Z\independent (D_1, D_0) \mid W$;
\vspace{-0.2cm}
\item Relevance: $\Pr(D_1>D_0)>0$; 
\vspace{-0.2cm}
\item $0<\Pr(Z=z, W=w)<1$ for any $z, w\in\{0, 1\}$.

\end{enumerate}
\end{ass}

Assumption \ref{ass:z} (i) relaxes the exclusion restriction in Assumption \ref{ass:IV} and allows instrument $Z$ to have nonzero direct effects on the outcome, $\rho_d\neq 0$. The average of the direct effects is assumed to be homogeneous across different subgroups but may vary given different treatments, $\rho_1\neq \rho_0$. One example for potential outcome $Y_{d, z}$ is given as $Y_{d, z}=\rho_d z+u_d$, where $\rho_d$ is a unknown constant. Then the direct effect is $\rho_d$ for each individual given treatment $D=d$. Section \ref{sec:exte} studies an extension allowing for different direct effects across subgroups and provides partial identification of LATE. The rest of the conditions in Assumption \ref{ass:z} are similar to Assumption \ref{ass:IV} except for introducing the independence condition of instrument $W$ with the potential outcome.

Under the monotonicity condition of instrument $Z$, we can divide the population into three subgroups $\{\AT, \NT, \CP\}$ as described in Section \ref{sec:treat}. The objective is to identify the average treatment effect for the compliers defined by instrument $Z$:
\begin{equation*}
\LATE=E[Y_1-Y_0\mid \CP].
\end{equation*}

Under the assumption of no direct effects ($\rho_1=\rho_0=0$), LATE is identified as
\begin{equation*}
\LATE=\IV_1\equiv \frac{E[Y\mid Z=1, W=1]-E[Y\mid Z=0, W=1]}{E[D\mid Z=1, W=1]-E[D\mid Z=0, W=1]}.
\end{equation*}

The $\IV_1$ estimand is similar to the $\IV$ estimand except it is conditional on the additional variable $W$. However, the above identification result does not apply if $Z$ has nonzero direct effects on the outcome. 
My paper uses the additional instrument $W\in \{0, 1\}$ to help identify the direct effects $(\rho_1, \rho_0)$ and LATE.  

The following describes a relevance condition of instrument $W$.

\begin{ass}[Instrument W]\label{ass:w}
Relevance: $\Pr(G \mid W=1)\Pr(\CP\mid W=0)\neq \Pr(G \mid W=0)\Pr(\CP\mid W=1)$ for any $G \in \{\AT, \NT\}$.
%(i) Independence: $W \independent (Y_{1,1}, Y_{1,0}, Y_{0, 1}, Y_{0, 0}) \mid (D_1, D_0)$; (ii) 
%\begin{equation*}
%\frac{\Pr(G \mid W=1)}{\Pr(\CP\mid W=1)}\neq \frac{\Pr(G \mid W=0)}{ \Pr(\CP\mid W=0)}.
%\end{equation*}
\end{ass}


Assumption \ref{ass:w} is a relevance condition of instrument $W$. It requires that instrument $W$ is correlated with the potential treatment such that the size of always takers (or never takers) relative to compliers varies when instrument $W$ changes. Assumption \ref{ass:w} does not impose monotonicity on instrument $W$, so instrument $W$ can affect treatment in any direction. This assumption is testable since the conditional probability of all types $G\in \{\AT, \NT, \CP\}$ is identified, as shown in Appendix \ref{proof:thm1}. 

%The independence condition of instrument $W$ is stated in Assumption \ref{ass:z}. 

Section \ref{sec:intro} provides some examples of the two instruments in various applications. When estimating the effects of the food stamp program on health outcomes, the first instrument can be an increase in the benefits of the program. The increased benefits may directly affect health outcomes so the exclusion restriction fails. The second instrument can be whether the benefit of the program is issued electronically or in hard copy.
This instrument may not satisfy the monotonicity condition since some people prefer electronic delivery but others prefer hard-copy delivery.
%{\color{red} 
%In the example of estimating returns to schooling, instrument $Z$ can be parental education. This instrument may have direct effects on individuals' wages since it may be correlated with unobserved ability. Instrument $W$ can be proximity to college, which may not satisfy monotonicity since individuals who reside close to college may not be able to afford the college tuition. 
%}


\begin{thm} \label{thm:point}
Under Assumptions \ref{ass:z}-\ref{ass:w},  direct effects $(\rho_1, \rho_0)$ and local average treatment effect $\LATE$ are point identified.
\end{thm}

When the exclusion restriction is relaxed, instrument $Z$ can induce both treatment effects by switching the treatment status and direct effects on the outcome. We are unable to distinguish treatment effects and direct effects without further information, so the standard identification result for treatment effects no longer applies. This paper uses additional instrument $W$, which can serve as an instrument for the imperfect instrument $Z$, to address instrument $Z$'s violation of the exclusion restriction. By exploiting variation in instrument $W$, this approach can identify the direct effects of instrument $Z$ on outcome $Y$. Then the local average treatment effect is identified by subtracting the direct effects.

Theorem \ref{thm:point} provides an alternative approach to identify treatment effects. This approach uses two instruments while relaxing one of the assumptions on the two instruments. This method can be applied to scenarios where there are multiple options of instruments but the instruments are imperfect in different dimensions. The availability of multiple instruments is discussed in the literature including \cite{card2001}, \cite{hausman2012}, and \cite{kolesar2015}. Moreover, instrument $W$ can help identify the direct effects of instrument $Z$ on the treated and untreated outcome, meaning that this approach can be used to test whether instrument $Z$ has direct effects on the outcome. 

%The standard $\LATE$ framework in \cite{imbens1994} needs a valid instrument $Z$ to satisfy all conditions in Assumption \ref{ass:IV}. As a complement, my paper allows the instrument $Z$ to violate the exclusion restriction, while needs an additional instrument $W$ for identification. The paper imposes weaker assumptions on both instruments compared to Assumption \ref{ass:IV}: the instrument $Z$ can violate the exclusion restriction and the instrument $W$ can violate the monotonicity. 
%

\section{Estimation} \label{sec:esti}
This section provides an estimation method for direct effects $(\rho_1, \rho_0)$ and local average treatment effect $\LATE$. Suppose that we have an $i.i.d$ sample $(Y_i, D_i, Z_i, W_i)_{i=1}^N$. 
As shown in Appendix \ref{proof:thm1}, LATE is identified as the $\IV_1$ estimand minus a weighted average of the two direct effects $(\rho_1, \rho_0)$:
\begin{equation*}
\LATE=\IV_1-\rho_1w^{\rho}_{1}-\rho_0 w^{\rho}_0,
\end{equation*}
where $w^{\rho}_1=\frac{\Pr(\AT\mid W=1)}{\Pr(\CP\mid W=1)}+\Pr(Z=0)$, $w^{\rho}_0=\frac{\Pr(\NT\mid W=1)}{\Pr(\CP\mid W=1)}+\Pr(Z=1)$, and the formula for the two direct effects $(\rho_1, \rho_0)$ is described below. 

To estimate $\LATE$, we need to estimate all terms in the above expression of $\LATE$. The first term $\IV_1$ is given as
\begin{equation*}
\begin{aligned}
\IV_1&=\frac{E[Y\mid Z=1, W=1]-E[Y\mid Z=0, W=1]}{E[D\mid Z=1, W=1]-E[D\mid Z=0, W=1]} \\
&=\frac{E[YZW]E[W]-E[YW] E[ZW]}{E[DZW] E[W]-E[DW]E[ZW]}.
\end{aligned}
\end{equation*}

The estimator for the $\IV_1$ estimand can be developed by replacing the population expectation with the sample mean:
\begin{equation*}
\widehat{\IV}_1=\frac{\sum_i(Y_iZ_iW_i) \sum W_i-\sum_i(Y_iW_i)\sum_i(Z_iW_i)}{\sum_i(D_iZ_iW_i)  \sum W_i-\sum_i(D_iW_i)\sum_i(Z_iW_i)}.
\end{equation*}
%where $\bar{Y}=\frac{1}{N}\sum_iY_i$ and $\bar{D}=\frac{1}{N}\sum_i D_i$.

Now I construct estimators for the two weights $(w_1^{\rho}, w_0^{\rho})$. 
Let $\widehat{\Pr}(G\mid w)$ denote the estimator for the conditional probability $\Pr(G\mid w)$ of the three subgroups $G\in\{\AT, \NT, \CP\}$ given $W=w$, constructed as follows: 
\begin{equation*}
\begin{aligned}
\widehat{\Pr}(\AT\mid w)&=\frac{\sum_i D_i(1-Z_i)\mathbbm{1}\{W_i=w\} }{\sum_i (1-Z_i)\mathbbm{1}\{W_i=w\}}, \\
\widehat{\Pr}(\NT\mid w)&=\frac{\sum_i (1-D_i)Z_i \mathbbm{1}\{W_i=w\} } {\sum_i Z_i\mathbbm{1}\{W_i=w\}}, \\
\widehat{\Pr}(\CP\mid w)&=\frac{\sum_i D_iZ_i\mathbbm{1}\{W_i=w\} }{\sum_i Z_i\mathbbm{1}\{W_i=w\}}-\frac{\sum_i D_i(1-Z_i)\mathbbm{1}\{W_i=w\} }{\sum_i (1-Z_i)\mathbbm{1}\{W_i=w\}}.
%1-\widehat{\Pr}(\AT\mid w)-\widehat{\Pr}(\NT\mid w).
\end{aligned}
\end{equation*}


%\begin{equation*}
%\begin{aligned}
%\widehat{\Pr}(\AT)&=\frac{\sum_i D_i(1-Z_i)}{\sum_i (1-Z_i)}, \\
%\widehat{\Pr}(\NT)&=\frac{\sum_i (1-D_i)Z_i}{\sum_i Z_i}, \\
%\widehat{\Pr}(\CP)&=1-\widehat{\Pr}(\AT)-\widehat{\Pr}(\NT).
%\end{aligned}
%\end{equation*}

Then the two weights $(w^{\rho}_1, w^{\rho}_0)$ can be estimated as
\begin{equation*}
\begin{aligned}
\hat{w}^{\rho}_1&=\frac{\widehat{\Pr}(\AT \mid 1)}{\widehat{\Pr}(\CP\mid 1)}+1-\bar{Z}, \qquad
\hat{w}^{\rho}_0&=\frac{\widehat{\Pr}(\NT\mid 1)}{\widehat{\Pr}(\CP\mid 1)}+\bar{Z},
\end{aligned}
\end{equation*}
where $\bar{Z}=\frac{1}{N}\sum_i Z_i$.

We only need to develop estimators for the two direct effects $(\rho_1, \rho_0)$. I focus on the estimator for direct effect $\rho_1$, and the idea also applies to $\rho_0$. As shown in Appendix \ref{proof:thm1}, direct effect $\rho_1$ is identified as
\begin{equation*}
\rho_1=\frac{r_1(1)\Pr(\CP\mid 0)-r_1(0)\Pr(\CP\mid 1) }{\Pr(\AT\mid 1)\Pr(\CP\mid 0)-\Pr(\AT\mid 0)\Pr(\CP\mid 1) },
\end{equation*}
where $r_1(w)$ is defined as
\begin{equation*}
r_1(w)\equiv E[YD\mid Z=1, w]-E[YD\mid Z=0, w].
\end{equation*}

The estimator for $r_1(w)$ is developed as follows:
\begin{equation*}
\hat{r}_1(w)=\frac{\sum_i Y_iD_iZ_i\mathbbm{1}\{W_i=w\}}{\sum_i Z_i\mathbbm{1}\{W_i=w\}}-\frac{\sum_i Y_iD_i(1-Z_i)\mathbbm{1}\{W_i=w\}}{\sum_i (1-Z_i)\mathbbm{1}\{W_i=w\}}.
\end{equation*}

Then estimator $\hat{\rho}_1$ for direct effect $\rho_1$ can be established by replacing all terms with their estimators:
\begin{equation*}
\hat{\rho}_1=\frac{\hat{r}_1(1)\widehat{\Pr}(\CP\mid 0)-\hat{r}_1(0)\widehat{\Pr}(\CP\mid 1) }{\widehat{\Pr}(\AT\mid 1)\widehat{\Pr}(\CP\mid 0)-\widehat{\Pr}(\AT\mid 0)\widehat{\Pr}(\CP\mid 1) }. 
\end{equation*}

Estimator $\hat{\rho}_0$ for $\rho_0$ can be established similarly, so it is omitted here. 
Local average treatment effect $\LATE$ is estimated as
\begin{equation*}
\widehat{\LATE}=\widehat{\IV}_1-\hat{\rho}_1 \hat{w}^{\rho}_1-\hat{\rho}_0 \hat{w}^{\rho}_0.
\end{equation*}


The asymptotic properties of $\widehat{\LATE}$ and the two direct effects $(\hat{\rho}_1, \hat{\rho}_0)$ are derived in Appendix \ref{asymptotic}. The estimators $\widehat{\LATE}$ and $(\hat{\rho}_1, \hat{\rho}_0)$ are $\sqrt{N}$ consistent assuming all variances and covariances are finite, and the expressions of asymptotic variances are shown in Appendix \ref{asymptotic}. % while the asymptotic variances can also be calculated using bootstrap.



 \subsection{Simulation Study}
 
This section examines the finite sample performance of estimator $\widehat{\LATE}$ by using two instruments $(Z, W)$ via Monte Carlo simulations. I compare the estimator in this paper with the IV estimator in \cite{imbens1994} and show that the method in this paper works more robustly when the direct effects are nonzero. The $\IV_1$ estimator (conditional on $W$) performs slightly worse than the $\IV$ estimator, so I display the results of the $\IV$ estimator for comparison.


The simulation setup is as follows. Instrument $Z$ follows the Bernoulli distribution with probability $p=0.5$. Potential treatment $D_z$ is $D_z=\mathbbm{1}\{z\geq \epsilon \}$, and observed treatment $D$ is $D=D_1 Z+D_0(1-Z)$.
Potential outcome $Y_d$ is given $Y_d=a_d+\rho_d Z+u_d$ for $d\in\{0, 1\}$, where $a_1=1$, $a_0=0$. Observed outcome $Y$ is given $Y=Y_1D+Y_0(1-D)$. I consider four different cases of direct effects: $\rho_1=\rho_0=\rho\in \{0, 0.5, 1, -1\}$. 
Instrument $W$ is given as $W=\mathbbm{1}\{v\leq D_1+D_0\}$. 

Latent variables $(\epsilon, u_1)$ follow a standard multivariate distribution with correlation $c=0.5$, and this correlation captures the endogeneity of treatment $D$. Error terms $u_0$ and $v$ follow a standard normal distribution and are independent of all other variables. I consider sample size $N=\{1000, 4000, 16000\}$, and the repetition number is $B=5000$.

The assumptions of the two instruments $(Z, W)$ are satisfied under this setup. Instrument $Z$ is allowed to have nonzero direct effects on outcome $\rho_d\neq 0$, and the direct effects are the same across subgroups. The independence condition of $Z$ is satisfied since it is independent of all variables. The monotonicity and relevance conditions are satisfied by the definition of potential treatment $D_z$. Instrument $W$ is independent of the potential outcome given the potential treatment since error term $v$ is independent of all variables. The relevance of instrument $W$ is satisfied because $W$ depends on the potential outcome.

Let $\hat{\theta}_{zw}$ denote the estimator by using the two instruments $(Z, W)$ in this paper, and let $\hat{\theta}_z$ denote the IV estimator by using only instrument $Z$ in \cite{imbens1994}. Let $(\hat{\rho}_1, \hat{\rho}_0)$ denote the estimators for the direct effects $(\rho_1, \rho_0)$ in this paper. To compare the two estimators $(\hat{\theta}_{zw}, \hat{\theta}_{z})$, I report four evaluations of the two estimators: bias, standard deviation (SD), root mean-squared error (rMSE), and median of absolute deviation (MAD). Under the simulation setup, the true local average treatment effect $\theta_0$ is given as
\begin{equation*}
\begin{aligned}
\theta_0=E[Y_1-Y_0\mid \CP]=&E[a_1-a_0+(\rho_1-\rho_0)Z+u_1-u_0\mid \epsilon\in(0, 1)] \\
=&1+\frac{(\phi(0)-\phi(1))}{2(\Phi(1)-\Phi(0))},
\end{aligned}
\end{equation*}
where $\phi$ and $ \Phi$ denote the PDF and CDF of the standard normal distribution, respectively.

Table \ref{table:est} presents the performance of the two estimators $\hat{\theta}_{zw}$ and $\hat{\theta}_z$ under different direct effects $\rho\in \{0, 0.5, 1, -1\}$ and sample size $N\in\{1000, 4000, 16000\}$. Estimator $\hat{\theta}_z$ performs better when the direct effect is zero but can have a large bias with nonzero direct effects. The bias increases when the direct effect is larger, and it does not decrease even when the sample size increases. When the direct effect is negative $\rho<0$, the bias becomes negative and estimator $\hat{\theta}_z$ may have the wrong signs of the true treatment effect. 
Estimator $\hat{\theta}_{zw}$ by using two instruments performs uniformly under different direct effects, and it performs better than estimator $\hat{\theta}_z$ with nonzero direct effects. This pattern becomes more significant when the sample size increases. 
The comparison in Table \ref {table:est} shows that estimator $\hat{\theta}_{zw}$ by using two instruments has a more robust performance for nonzero direct effects. 


\begin{table}[!htbp]
\centering
\caption{Performance Comparisons of $\hat{\theta}_{zw}$ and $\hat{\theta}_z$ }
\label{table:est}
\begin{tabular}{cc |cccc|cccc}
\hline
\hline
 \multirow{2}{*}{$N$}&\multirow{2}{*}{$\rho$}&\multicolumn{4}{c|}{$\hat{\theta}_{zw}$}  &  \multicolumn{4}{c}{$\hat{\theta}_z$}  \\
\cline{3-10} 
&  & Bias & SD & rMSE & MAD  & Bias & SD  & rMSE & MAD \\
\hline
\multirow{5}{*}{1000} & $0$ \ & 0.025 & 0.504 & 0.505 & 0.396 & 0.006 &0.183 &0.184 &0.146  \\ [1.0ex]       
& $0.5$ \ & 0.025 & 0.504 & 0.505 & 0.396 &1.481 & 0.245 & 1.501 &1.481  \\ [1.0ex]                                                                                                                                                                                                                                                                                                                                                                                                                                                                                                                                                                                                                                                                    
&  $1$ \  & 0.025 & 0.504 & 0.505 & 0.396  & 2.956 & 0.342 & 2.976 & 2.956 \\ [1.0ex]  
& $-1$ \  & 0.025 & 0.504 & 0.505 & 0.396  & -2.944 & 0.269 &2.956 &2.944  \\ [1.0ex] 
\hline
\multirow{5}{*}{4000} & $0$ \ &0.009 &0.230 &0.230 & 0.183 &-0.001 &0.092 &0.092 &0.073  \\ [1.0ex] 
&  $0.5$ \  &0.009 &0.230 &0.230 & 0.183 & 1.466 &0.119 & 1.470 & 1.466   \\ [1.0ex] 
& $1$ \  &0.009 &0.230 &0.230 & 0.183&  2.932 & 0.165 &2.936 &2.932      \\ [1.0ex] 
& $-1$ \  &0.009 &0.230 &0.230 & 0.183 & -2.933 &0.133 &2.936 &2.933     \\   [1.0ex] 
\hline
\multirow{5}{*}{16000} & $0$ \ &0.001 &0.113 &0.113 &0.090 &0.001 &0.046 &0.046 &0.036  \\ [1.0ex]       
&  $0.5$ \  &0.001 &0.113 &0.113 &0.090 & 1.465 & 0.060 &1.466 &1.465  \\ [1.0ex]                                                                                                                                                                                                                                                                                                                                                                                                                                                                                                                                                                                                                                                                 
& $1$ \ &0.001 &0.113 &0.113 &0.090& 2.930 & 0.083 & 2.931  & 2.930 \\ [1.0ex]   
&  $-1$ \ &0.001 &0.113 &0.113 &0.090 &  -2.929 & 0.066 &2.930 &2.929 \\ [1.0ex] 
\hline
\end{tabular}
\end{table}
\newpage
Table \ref{table:direct} presents the performance of the estimators $(\hat{\rho}_1, \hat{\rho}_0)$ for the direct effects. The performance of $(\hat{\rho}_1, \hat{\rho}_0)$ does not depend on the true direct effects, so I only report the results under different sample sizes. When the sample size increases, the bias and deviation of  $(\hat{\rho}_1, \hat{\rho}_0)$ shrink dramatically. 

Appendix \ref{simu} presents more simulation results about the two estimators $(\hat{\theta}_{zw}, \hat{\theta}_z)$  under different probabilities of the three subgroups.






\begin{table}[!htbp]
\centering
\caption{Performance of Direct Effects $\hat{\rho}_1$ and $\hat{\rho}_0$}
\label{table:direct}
\begin{tabular}{c |cccc|cccc}
\hline
\hline
\rule{0pt}{20pt} \multirow{2}{*}{$N$}&\multicolumn{4}{c|}{$\hat{\rho}_1$}  &  \multicolumn{4}{c}{$\hat{\rho}_0$}  \\
\cline{2-9} 
 \rule{0pt}{20pt}& Bias & SD & rMSE & MAD  & Bias & SD  & rMSE & MAD \\
\hline
\rule{0pt}{20pt}1000 & -0.015 & 0.210 & 0.210 & 0.160 & 0.006 &0.351 &0.351 &0.275  \\ [1.7ex]       
\hline
\rule{0pt}{20pt}4000  & -0.004 &0.095 &0.095 &0.075 &-0.003 & 0.168 &0.168 &0.133  \\ [0.7ex] 
\hline
\rule{0pt}{20pt}16000 & 0.000 &0.047 &0.047 &0.038 &-0.001 & 0.083 &0.083 &0.066  \\ [0.7ex]       
\hline
\end{tabular}
\end{table}




% 
%\section{Empirical Illustration}
%
%I plan to study the example of returns to schooling and use the NLS76 data in Card (1995). The treatment $D$ is whether to attend a college, and the outcome $Y$ is log(wage). I use parental education as the first instrument $Z$, which equals to one if either people's father or mother has received education larger than 9 years. This instrument may not satisfy the exclusion restriction since parental education can have direct effects on people's wage.
%I use proximity to college as the instrument $W$, which is whether people grew up near four-year college. This instrument can violate the monotonicity since people who stay near college may not afford the tuition. 
%
%\begin{table}[!htbp]
%\centering
%\caption{Results}
%\label{table:direct}
%\begin{tabular}{c| cccc}
%\hline
%\hline
% \rule{0pt}{20pt}  & $\hat{\theta}_{zw}$  & $\hat{\rho}_1$  & $\hat{\rho}_0$  & $\hat{\theta}_{z}$\\
%\hline
%\rule{0pt}{20pt}  Estimators & 0.285  & 0.000 & 0.133 &  0.936 \\
%\hline
%\end{tabular}
%\end{table}
%
%
%If I incorporate covariates such as whether is black, then the sample size given $(Z, W, X)$ becomes small and the results become weird.
%\begin{table}[!htbp]
%\centering
%\caption{Direct Effects}
%\label{table:direct}
%\begin{tabular}{c|ccc}
%\hline
%\hline
% \rule{0pt}{20pt}  & Black  & Non-Black  & Average  \\
%\hline
%\rule{0pt}{20pt}  $\hat{\rho}_1$ &-0.217 &  -0.395 & -0.353 \\
%\hline
%\rule{0pt}{20pt}  $\hat{\rho}_0$ & -0.044 & 0.305  & 0.223 \\
%\hline
%\rule{0pt}{20pt}  $\hat{\theta}_{zw}$ &  2.158  & -0.189 & 0.359 \\
%\hline
%\rule{0pt}{20pt}  $\hat{\theta}_{z}$ & 1.208  & 0.76774 & 0.870 \\
%\hline
%\end{tabular}
%\end{table}


\section{Extension} \label{sec:exte}

Section \ref{sec:iden} establishes point identification results of LATE when the average of direct effects is assumed to be homogeneous across subgroups. This assumption can allow for heterogeneous direct effects within a subgroup but assumes same direct effects across subgroups. This section further relaxes this assumption and provides partial identification results under heterogeneous direct effects across subgroups. 

Recall that $\rho_{G, d}=E[Y_{d,1}-Y_{d, 0}\mid G]$ denote the average direct effect for group $G\in\{\AT, \NT, \CP\}$ given treatment $D=d$. I consider that the heterogeneity in direct effects between subgroups can be bounded by a known number.
%Let $\rho_{G, d}=E[Y_{d, 1}-Y_{d, 0}\mid G]$ denote the average of direct effect for subgroup $G\in\{\AT, \NT, \CP\}$ under treatment $D=d$. 

\begin{ass} \label{ass:dir}
Direct effects: $|\rho_{\CP, 1}-\rho_{\AT, 1}|\leq k_1$ and $|\rho_{\CP, 0}-\rho_{\NT, 0}|\leq k_0$, where $k_1, k_0\geq 0$ are known constants.
\end{ass}

Assumption \ref{ass:dir} relaxes Assumption \ref{ass:z} (i) and allows heterogeneous direct effects across different subgroups. This assumption requires that the difference in direct effects between subgroups is not too large and can be bounded by a known number $k_d$. The information about $k_d$ depends on specific applications. For example, when the instrument is whether there is an increase (or decrease) in the benefits of a social program, the direct effect on people who did not participate in the program is zero $\rho_{\NT, 0}=\rho_{\CP, 0}=0$ so that $k_0=0$. The value of $k_1$ can be developed when the support of the outcome is bounded such as binary outcomes.

%Also, it may be reasonable to assume that the heterogeneity in direct effects is smaller than the absolute value of the direct effect of always takers and never takers: $k_1=|\rho_{AT, 1}|$ and  $k_0=|\rho_{NT, 0}|$. As shown in Appendix \ref{proof:exte}, $\rho_{AT, 1}$ and $\rho_{NT, 0}$ are identified. 

%may come from data, by assumption, or it can be established when the support of the outcome is bounded such as binary outcome. 

As shown in Section \ref{sec:esti}, when the average of direct effects is the same across different groups $k_1=k_0=0$, $\LATE$ is identified as 
\begin{equation*}
\LATE=\IV_1-\rho_1 w^{\rho}_1-\rho_0 w^{\rho}_0\equiv \widetilde{\IV}. 
\end{equation*}

The next proposition derives sharp bounds on $\LATE$.

\begin{prop} \label{prop:exte}
Under Assumptions \ref{ass:z} (ii)-(iv) and Assumptions \ref{ass:w}-\ref{ass:dir}, the sharp bounds for $\LATE$ are given as follows:
\[ \LATE\in \left[\widetilde{\IV}-k_1 \Pr(Z=0)-k_0 \Pr(Z=1),  \widetilde{\IV}+k_1 \Pr(Z=0)+k_0 \Pr(Z=1)  \right]. \]
%\begin{equation*}
%\begin{aligned}
%\LATE \geq \widetilde{\IV}-k_1 \Pr(Z=0)-k_0 \Pr(Z=1), \\
%\LATE \leq \widetilde{\IV}+k_1 \Pr(Z=0)+k_0 \Pr(Z=1). \\
%\end{aligned}
%\end{equation*}

\end{prop}

Proposition \ref{prop:exte} shows that $\LATE$ can be still bounded by using two instruments under heterogeneous direct effects. The bounds are tighter when the heterogeneity $(k_1, k_0)$ in direct effects between different subgroups is smaller, and point identification is achieved when the heterogeneity is zero. The bounds on LATE can be established similarly and be further tightened if the direction of the difference in direct effects is known such as $0\leq \rho_{\CP, 1}-\rho_{\AT, 1}\leq k_1$ and $0\leq \rho_{\CP, 0}-\rho_{\NT, 0}\leq k_0$.





\section{Conclusion} \label{sec:conc}

This paper proposes a new approach to point identify LATE by using two instruments while imposing weaker assumptions on both instruments compared to \cite{imbens1994}. The first instrument is allowed to violate the exclusion restriction, so it can have nonzero direct effects on the outcome. Then the $\IV$ estimand involves both treatment effects and direct effects, so the standard identification result does not apply. This paper uses an additional imperfect instrument, which does not need to satisfy monotonicity. By exploiting variation in the second instrument, we can identify the direct effects of the first instrument and then identify LATE.
Based on the identification results, an estimator for LATE is developed and it is shown to perform more robustly than the $\IV$ estimator with nonzero direct effects.


This paper relaxes different assumptions for the two instruments to achieve point identification of treatment effects. It would be worthwhile to investigate the identifying power of multiple instruments when they violate the same assumption such as the exclusion restriction. Moreover, the paper considers two instruments and relaxes one main assumption for each of the two instruments. It would be interesting to explore whether point identification can be achieved under weaker conditions on instruments when there are more than two instruments available. 





\bibliography{LATE}
\bibliographystyle{apalike}





\appendix

\section{Appendix} \label{sec:appen}



%\subsection{Proof of Proposition \ref{prop:bias1}} \label{proof:prop1}
%
%\begin{proof}
%I consider a more general case $\rho_{G, d}=\rho_d$, where $\rho_1$ is allowed to be different from $\rho_0$. The results for the case $\rho_1=\rho_0=\rho$ follow directly.
%%The results under this more general case will be used in \ref{proof:thm1} to prove Theorem \ref{thm1}, and 
%
%I first look at the formula of the IV estimand:  
%\begin{equation*}
%\IV=\frac{E[Y\mid Z=1]-E[Y\mid Z=0]}{E[D\mid Z=1]-E[D\mid Z=0]}.
%\end{equation*}
%
%We can divide $E[Y\mid Z=z]$ into three subgroups: always takers (AT), never takers (NT), and compliers (CP). When $Z=1$, the conditional expectation $E[Y\mid Z=1]$ can be expressed as 
%\begin{equation*}
%\begin{aligned}
%&E[Y\mid Z=1] \\
%%=&E[Y\mid \AT, Z=z]\Pr(\AT\midZ=1)+E[Y|N, Z=z]\Pr(N|Z=z)+E[Y|C, Z=z]\Pr(C|Z=z)\\
%=&E[Y_{1, 1} \mid \AT ]\Pr(\AT)+E[Y_{0, 1}\mid \NT]\Pr(\NT)+E[Y_{1,1}\mid \CP]\Pr(\CP).
%\end{aligned}
%\end{equation*}
%The above condition holds by the independence condition of instrument $Z$ in Assumption \ref{ass:IV} (iii).
%We can divide $E[Y\mid Z=0]$ into the three subgroups similarly. Then under the assumption $E[Y_{d, 1}-Y_{d, 0}\mid G]=\rho_d$ , the following condition holds:
%\begin{equation*}
%E[Y\mid Z=1]-E[Y\mid Z=0]=\rho_1\Pr(\AT)+\rho_0 \Pr(\NT)+E[(Y_{1,1}-Y_{0, 0} )\mid \CP]\Pr(\CP).
%\end{equation*}
%
%For always takers and never takers, instrument $Z$ only has direct effects $\rho_d$ on the outcome because the treatment does not change. For compliers, $Z$ induces both direct effects and treatment effects by shifting the treatment status.
%
%The denominator of the $\IV$ estimand is equal to the size of compliers under Assumption \ref{ass:IV} (ii)-(v):
%\begin{equation*}
%E[D\mid Z=1]-E[D\mid Z=0]=[\Pr(\AT)+\Pr(\CP)]-\Pr(\AT)=\Pr(\CP).
%\end{equation*}
%
%Therefore, the $\IV$ estimand is given as
%\begin{equation*} \label{IV}
%\begin{aligned}
%\IV=\rho_1 \frac{\Pr(\AT)}{\Pr(\CP)} +\rho_0 \frac{\Pr(\NT)}{\Pr(\CP)}+ E[(Y_{1,1}-Y_{0, 0} )\mid \CP].
%\end{aligned}
%\end{equation*}
%
%Now we look at the expression of $\LATE$ to build its relationship with the IV estimand. I divide $\LATE$ into two groups: $Z=1$ and $Z=0$,
%\begin{equation*}
%\begin{aligned}
%\LATE=&E[Y_1-Y_0\mid \CP] \\
%=&E[Y_{1, 1}-Y_{0, 1}\mid \CP, Z=1]\Pr(Z=1)+E[Y_{1, 0}-Y_{0, 0}\mid \CP, Z=0]\Pr(Z=0) \\
%=&E[Y_{1, 1}-Y_{0, 1}\mid \CP]\Pr(Z=1)+E[Y_{1, 0}-Y_{0, 0}\mid \CP]\Pr(Z=0).
%\end{aligned}
%\end{equation*}
%
%Using the condition $E[Y_{d, 1}-Y_{d, 0}\mid \CP]=\rho_d$ to substitute $E[Y_{1, 0}\mid \CP]$ and $E[Y_{0, 1}\mid \CP]$ has the following implication:
%\begin{equation*}
%\LATE=E[(Y_{1,1}-Y_{0, 0} )\mid \CP]-\rho_1\Pr(Z=0)-\rho_0\Pr(Z=1).
%\end{equation*}
%
%Taking the difference between $\IV$ and $\LATE$ can cancel out the term $E[(Y_{1,1}-Y_{0, 0} )\mid \CP]$ and has the following implication:
%\begin{equation*}
%\IV-\LATE=\rho_1\left[\frac{\Pr(\AT)}{\Pr(\CP)}+\Pr(Z=0)\right]+\rho_0 \left[\frac{\Pr(\NT)}{\Pr(\CP)}+\Pr(Z=1)\right].
%\end{equation*}
%
%When the direct effects satisfy $\rho_1=\rho_0=\rho$, the above condition becomes
%\begin{equation*}
%\IV-\LATE=\frac{\rho}{\Pr(\CP)}.
%\end{equation*}
%
%
%
%\end{proof}





\subsection{Proof of Theorem \ref{thm:point}} \label{proof:thm1}
                  
\begin{proof}

I first look at the expression of LATE, and divide it into two groups $Z=1$ and $Z=0$ as follows:
\begin{equation*}
\begin{aligned}
\LATE=&E[Y_1-Y_0\mid \CP]\\
=&E[Y_{1, 1}-Y_{0, 1}\mid \CP, Z=1]\Pr(Z=1)+E[Y_{1, 0}-Y_{0, 0}\mid \CP, Z=0]\Pr(Z=0)\\
=&E[Y_{1, 1}-Y_{0, 1}\mid \CP]\Pr(Z=1)+E[Y_{1, 0}-Y_{0, 0}\mid \CP]\Pr(Z=0).
\end{aligned}
\end{equation*}
The above condition holds by the independence condition of instrument $Z$ in Assumption \ref{ass:z}. Using the condition $E[Y_{d, 1}-Y_{d, 0}\mid \CP]=\rho_d$ to substitute $E[Y_{0, 1} \mid \CP]$ and $E[Y_{1, 0} \mid \CP]$ leads to the following implication:
\begin{equation*}
\LATE=E[Y_{1, 1}-Y_{0, 0}\mid \CP]-\rho_1\Pr(Z=0)-\rho_0\Pr(Z=1).
\end{equation*}

To prove LATE is identified, we need to show that the two direct effects $(\rho_1, \rho_0)$ and $(E[Y_{11}\mid \CP], E[Y_{0, 0}\mid \CP])$ are identified.  To prove it, I first show that the conditional probability of the three subgroups $\{\AT, \NT, \CP\}$ given $W=w$ is identified. When $Z=0$, only always takers (AT) are treated so that the probability of always takers is identified as
\begin{equation*}
\begin{aligned}
\Pr(D=1\mid Z=0,w)&=\Pr(D=1\mid \AT, Z=0,w)\Pr(\AT\mid Z=0, w)\\
&=\Pr(\AT\mid w). \\
\end{aligned}
\end{equation*}
The last equality holds since the probability of being treated conditional on always takers is one and instrument $Z$ is independent of potential treatments given $W$ in Assumption \ref{ass:z}.

Similarly, the conditional probability of never takers and compliers given $W=w$ can be derived as

\begin{equation*}
\begin{aligned}
\Pr(\NT\mid w)=&\Pr(D=0\mid Z=1,w);   \\
\Pr(\CP\mid w)=&\Pr(D=1\mid Z=1,w)-\Pr(D=1\mid Z=0,w).
\end{aligned}
\end{equation*}

%Then the unconditional probability of the three subgroups $\Pr(G)$ is identified as follows: for $G\in \{\AT, \NT, \CP\}$,
%\begin{equation*}
%\Pr(G)=\Pr(G\mid W=1)\Pr(W=1)+\Pr(G\mid W=0)\Pr(W=0).
%\end{equation*}

Now we are ready to show that the two direct effects $(\rho_1, \rho_0)$ and $(E[Y_{11}\mid \CP], E[Y_{0, 0}\mid \CP])$ are identified by using variation in instrument $W$.
The expectation of $YD$ conditional on $(Z=1, W=w)$ can be the expressed as a mixture of always takers and compliers:
\begin{equation} \label{z1}
\begin{aligned}
&E[YD\mid Z=1, w]\\
=&E[Y_{1,1}\mid \AT, Z=1, w]\Pr(\AT\mid Z=1, w)+E[Y_{1, 1}\mid \CP, Z=1, w]\Pr(\CP\mid Z=1, w) \\
=&E[Y_{1, 1}\mid \AT]\Pr(\AT\mid w)+E[Y_{1,1}\mid \CP]\Pr(\CP\mid w). 
\end{aligned}
\end{equation}
The above condition holds by the independence conditions of the two instruments $(Z, W)$ in Assumptions \ref{ass:z}.

Similarly, the expectation of $YD$ given $(Z=0, W=w)$ can be expressed as
\begin{equation}\label{z0}
E[YD\mid Z=0, w]=E[Y_{1, 0}\mid \AT]\Pr(\AT\mid w).
\end{equation}

Taking the difference between \eqref{z1} and \eqref{z0} leads to the following condition: for any $w\in\{0, 1\}$,
\begin{equation}\label{eq:dif}
\begin{aligned}
r_1(w)&\equiv E[YD\mid Z=1, w]-E[YD\mid Z=0, w]\\
&=\rho_1\Pr(\AT\mid w)+E[Y_{1,1}\mid \CP]\Pr(\CP\mid w).
\end{aligned}
\end{equation}

Equation \eqref{eq:dif} comes from the condition $E[Y_{d, 1}-Y_{d, 0}\mid \AT]=\rho_d$ in Assumption \ref{ass:z} (i). Since condition \eqref{eq:dif} holds for any $w\in \{0 ,1\}$, using variation in  instrument $W$ can identify direct effect $\rho_1$ and $E[Y_{1, 1}\mid \CP]$ as follows:
\begin{equation*}
\begin{aligned}
&\rho_1=\frac{r_1(1)\Pr(\CP\mid 0)-r_1(0)\Pr(\CP\mid 1) }{\Pr(\AT\mid 1)\Pr(\CP\mid 0)-\Pr(\AT\mid 0)\Pr(\CP\mid 1) }, \\
&E[Y_{1,1}\mid \CP]=\frac{r_1(1)-\rho_1\Pr(\AT\mid 1) }{\Pr(\CP\mid 1 ) }. \\
\end{aligned}
\end{equation*}
The relevance condition of instrument $W$ in Assumption \ref{ass:w} guarantees that the denominator of  direct effect $\rho_1$ is nonzero: $\Pr(\AT\mid 1)\Pr(\CP\mid 0)-\Pr(\AT\mid 0)\Pr(\CP\mid1)\neq 0$.  The relevance condition of instrument $Z$ in Assumption \ref{ass:z} implies that there exists $w\in \{0, 1\}$ such that $\Pr(\CP\mid w)>0$. For simplicity, I assume that $\Pr(\CP\mid 1)>0$.

Similarly, I use the expectation of $Y(1-D)$ under different values of $(Z, W)$ to identify direct effect $\rho_0$ and $E[Y_{0, 0}\mid \CP]$. Let $r_0(w)$ be defined as
\begin{equation*}
r_0(w)=E[Y(1-D)\mid Z=1, w]-E[Y(1-D)\mid Z=0, w].
\end{equation*}

By using variation in $r_0(w)$ with respect to $w$, direct effect  $\rho_0$ and $E[Y_{0, 0}\mid \CP]$ can be identified as follows:
\begin{equation*}
\begin{aligned}
&\rho_0=\frac{r_0(1)\Pr(\CP\mid 0)-r_0(0)\Pr(\CP\mid 1) }{\Pr(\NT\mid 1)\Pr(\CP\mid 0)-\Pr(\NT\mid 0)\Pr(\CP\mid 1) }, \\
&E[Y_{0, 0}\mid \CP]=-\frac{r_0(1)-\rho_0\Pr(\NT\mid 1) }{\Pr(\CP\mid 1) }.\\
\end{aligned}
\end{equation*}


Therefore, LATE is identified since the two direct effects $(\rho_1, \rho_0)$ and $(E[Y_{1, 1}\mid \CP], E[Y_{0, 0}\mid \CP] )$ are shown to be identified. Plugging into the expression of $(E[Y_{1, 1}\mid \CP], E[Y_{0, 0}\mid \CP] )$ into the formula of LATE leads to following expression:
\begin{equation*}
\begin{aligned}
\LATE&=E[Y_{1, 1}-Y_{0, 0}\mid \CP]-\rho_1\Pr(Z=0)-\rho_0\Pr(Z=1) \\
&=\frac{r_1(1)+r_0(1)}{\Pr(\CP\mid 1)}-\rho_1 \left(\frac{\Pr(\AT \mid 1)}{\Pr(\CP\mid 1)} +\Pr(Z=0) \right)-\rho_0\left( \frac{\Pr(\NT \mid 1)}{\Pr(\CP\mid 1)}+ \Pr(Z=1) \right) \\
%&=\frac{E[Y\mid Z=1, W=1]-E[Y\mid Z=0, W=1]} {E[D\mid Z=1, W=1]-E[D\mid Z=0, W=1]}-\rho_1 w_1^{\rho}-\rho_0 w^{\rho}_0\\
&\equiv \IV_1-\rho_1 w_1^{\rho}-\rho_0 w^{\rho}_0,
\end{aligned}
\end{equation*}
where $\IV_1=\frac{E[Y\mid Z=1, W=1]-E[Y\mid Z=0, W=1]}{E[D\mid Z=1, W=1]-E[D\mid Z=0, W=1]}$, $w_1^{\rho}= \frac{\Pr(\AT \mid 1)}{\Pr(\CP\mid 1)} +\Pr(Z=0)$, and $w_0^{\rho}= \frac{\Pr(\NT \mid 1)}{\Pr(\CP\mid 1)}+ \Pr(Z=1)$. Therefore, LATE is identified as all terms in the above expression are shown to be identified.
%
%Now we are ready to identify the local average treatment effect. The local average treatment effect can be divided into two groups: $Z=1$ and $Z=0$,
% \begin{equation*}
% \begin{aligned}
%\LATE&=E[Y_{1}-Y_{0}\mid \CP] \\
%&=E[Y_{1, 1}-Y_{0, 1}\mid \CP]\Pr(Z=1)+ E[Y_{1, 0}-Y_{0, 0}\mid \CP]\Pr(Z=0). \\
%\end{aligned}
%\end{equation*}
%
%Plugging into the equation $E[Y_{d, 1}-Y_{d, 0}\mid \CP]=\rho_d$ has the following implication:
%\begin{equation*}
%\LATE=E[Y_{1, 1}-Y_{0, 0}\mid \CP]-\rho_0\Pr(Z=1)-\rho_1\Pr(Z=0).
%\end{equation*}


\end{proof}



\subsection{Asymptotic Properties of $\widehat{\LATE}$ and $(\hat{\rho}_1, \hat{\rho}_0)$} \label{asymptotic}

This section derives the asymptotic properties of estimator $\widehat{\LATE}$ for the local average treatment effect and the two estimators $(\hat{\rho}_1, \hat{\rho}_0)$ for the two direct effects. Suppose that the variances of all variables $(Y, D, Z, W)$ are finite, and the variances of the product of any two, three, and four variables are finite. 

As shown in Section \ref{sec:esti} , the expression for estimator $\widehat{\LATE}$ is given as follows:
\begin{equation*}
\widehat{\LATE}=\widehat{\IV}_1-\hat{\rho}_1 \hat{w}^{\rho}_1-\hat{\rho}_0 \hat{w}^{\rho}_0.
\end{equation*}

The analysis proceeds by deriving the asymptotic properties of the above terms $(\widehat{\IV}_1, \hat{w}^{\rho}_1, \hat{w}^{\rho}_0, \hat{\rho}_1, \hat{\rho}_0)$ respectively, then the asymptotic property of $\widehat{\LATE}$ can be derived accordingly.

 Let $\mu_Y$ denote the expectation of the random variable $Y$. Let $\phi_i^{Y}=Y_i-\mu_Y$ for any random variable $Y$, let $\phi_i^{YZ}=Y_iZ_i-E[YZ]$ for any two random variables $(Y, Z)$, and let $\phi_i^{YZW}=Y_iZ_iW_i-E[YZW]$ for any three random variables $(Y, Z, W)$.

I first look at estimator $\widehat{\IV}_1$. The numerator of estimator $\widehat{\IV}_1$ is given as
\begin{equation*}
\begin{aligned}
&\frac{1}{N}\sum_i(Y_iZ_iW_i) \bar{W}-\frac{1}{N}\sum_i(Y_iW_i)\frac{1}{N}\sum_i(Z_iW_i)-(E[YZW]E[W]-E[YW]E[ZW])\\
=&\frac{1}{N}\sum_i  (E[YZW]\phi_i^{W}+\mu_W\phi_i^{YZW}-E[YW]\phi_i^{ZW}-E[ZW]\phi_i^{YW})+O_p\left(\frac{1}{N} \right).
\end{aligned}
\end{equation*}


%\begin{equation*}
%\begin{aligned}
%\frac{1}{N}\sum_i Z_i(Y_i-\bar{Y})-E[Z(Y-\mu_Y)] =&\frac{1}{N}\sum_i (Z_iY_i-E[ZY])-(\bar{Z}\bar{Y}-\mu_Z\mu_Y)\\
%=&\frac{1}{N}\sum_i \left( \phi_i^{ZY}-\mu_Z\phi_i^{Y}-\mu_Y\phi_i^{Z} \right)+O_p\left(\frac{1}{N} \right).
%\end{aligned}
%\end{equation*}
The last equality holds by applying Taylor expansion and the fact that $\frac{1}{N}\sum_i(Y_iZ_iW_i)-E[YZW]=O_p(1/\sqrt{N})$, $\bar{W}-\mu_W=O_p(1/\sqrt{N})$, $\frac{1}{N}\sum_i(Y_iW_i)-E[YW]=O_p(1/\sqrt{N})$, and $ \frac{1}{N}\sum_i(Z_iW_i)-E[ZW]=O_p(1/\sqrt{N})$.

Let $\phi_i^{\IV_N}$ denote the influence function of the numerator term $\frac{1}{N}\sum_i(Y_iZ_iW_i) \bar{W}-\frac{1}{N}\sum_i(Y_iW_i)\frac{1}{N}\sum_i(Z_iW_i)$, defined as
\begin{equation*}
\phi_i^{\IV_N}=E[YZW]\phi_i^{W}+\mu_W\phi_i^{YZW}-E[YW]\phi_i^{ZW}-E[ZW]\phi_i^{YW}.
\end{equation*}

Similarly, the influence function of the denominator term $\frac{1}{N}\sum_i(D_iZ_iW_i) \bar{W}-\frac{1}{N}\sum_i(D_iW_i)\frac{1}{N}\sum_i(Z_iW_i)$ is derived as 
\begin{equation*}
\phi_i^{\IV_D}=E[DZW]\phi_i^{W}+\mu_W\phi_i^{DZW}-E[DW]\phi_i^{ZW}-E[ZW]\phi_i^{DW}.
\end{equation*}

Then by applying Taylor expansion, the asymptotic property of $\widehat{\IV}_1$  is derived as follows:
\begin{equation*}
\begin{aligned}
\widehat{\IV}_1-\IV_1&=\frac{1}{N}\sum_i \left\{  \frac{\phi_i^{\IV_N}-\IV_1 \phi_i^{\IV_D} }{E[YZW]E[W]-E[YW]E[ZW]} \right\}+O_p\left(\frac{1}{N} \right) \\
&\equiv \frac{1}{N}\sum_i \phi_i^{\IV_1}+O_p\left(\frac{1}{N} \right).
\end{aligned}
\end{equation*}


The approach for deriving influence functions of the two weights and direct effects is similar. I focus on the properties of $\hat{w}^{\rho}_1$ and $\hat{\rho}_1$, and the analysis applies to $(\hat{w}^{\rho}_0, \hat{\rho}_0)$.  Estimator $\hat{w}_1^{\rho}$ is given as
\begin{equation*}
\begin{aligned}
\hat{w}^{\rho}_1=\frac{\widehat{\Pr}(\AT\mid 1)} {\widehat{\Pr}(\CP\mid 1)}+\widehat{\Pr}(Z=0).%=\frac{\sum_i(D_i\tilde{Z}_i)\bar{Z} }{\sum_i[Z_i(D_i-\bar{D})]}+1-\bar{Z}.
\end{aligned}
\end{equation*}

To derive the asymptotic property of $\hat{w}^{\rho}_1$, I need to show the asymptotic property of the conditional probability of the two subgroups. 
Let $\phi_i^{\AT1}$, $\phi_i^{\AT0}$, $\phi_i^{\CP1}$, $\phi_i^{\CP0}$ denote the influence function for the four conditional probabilities $\widehat{\Pr}(\AT\mid  1)$, $\widehat{\Pr}(\AT\mid 0)$, $\widehat{\Pr}(\CP\mid 1)$, $\widehat{\Pr}(\CP\mid 0)$ respectively, which are derived as follows:
\begin{equation*}
\begin{aligned}
\phi_i^{\AT1}&=\frac{\phi_i^{D\tilde{Z}W}} {E[\tilde{Z}W]}- \frac{E[D\tilde{Z}W]\phi_i^{\tilde{Z}W}}{(E[\tilde{Z}W])^2}, \qquad 
\phi_i^{\AT0}=\frac{\phi_i^{D\tilde{Z}\tilde{W}}} {E[\tilde{Z}\tilde{W}]}- \frac{E[D\tilde{Z}\tilde{W}]\phi_i^{\tilde{Z}\tilde{W}}}{(E[\tilde{Z}\tilde{W}])^2}, \\
\phi_i^{\CP1}&=\frac{\phi_i^{DZW}}{E[ZW]}-\frac{E[DZW]\phi_i^{ZW}}{(E[ZW])^2}-\left\{ \frac{\phi_i^{D\tilde{Z}W}}{E[\tilde{Z}W]}-\frac{E[D\tilde{Z}W]\phi_i^{\tilde{Z}W}}{(E[\tilde{Z}W])^2} \right\}, \\
\phi_i^{\CP0}&=\frac{\phi_i^{DZ\tilde{W}}}{E[Z\tilde{W}]}-\frac{E[DZ\tilde{W}]\phi_i^{Z\tilde{W}}}{(E[Z\tilde{W}])^2}-\left\{ \frac{\phi_i^{D\tilde{Z}\tilde{W}}}{E[\tilde{Z}\tilde{W}]}-\frac{E[D\tilde{Z}\tilde{W}]\phi_i^{\tilde{Z}\tilde{W}}}{(E[\tilde{Z}\tilde{W}])^2} \right\},
\end{aligned}
\end{equation*}
where $\tilde{Z}=1-Z$ and $\tilde{W}=1-W$.

Then the influence function for $\hat{w}_1^{\rho}$ is derived as follows by applying Taylor expansion:
\begin{equation*}
\begin{aligned}
\hat{w}^{\rho}_{1}-w^{\rho}_1&=\frac{1}{N}\sum_i\left\{ \frac{ \phi_i^{\AT1}}{\Pr(\CP\mid 1)}-\frac{\Pr(\AT\mid 1)\phi_i^{\CP1}}{(\Pr(\CP\mid 1))^2}-\phi_i^Z \right\}+ O_p\left(\frac{1}{N}\right)  \\
&\equiv\frac{1}{N}\sum_i \phi_i^{w^{\rho}_1}+O_p\left(\frac{1}{N}\right).
\end{aligned}
\end{equation*}

Now we only need to derive the influence function of estimator $\hat{\rho}_1$, given as
\begin{equation*}
\hat{\rho}_1=\frac{\hat{r}_1(1)\widehat{\Pr}(\CP\mid 0)-\hat{r}_1(0)\widehat{\Pr}(\CP\mid 1) }{\widehat{\Pr}(\AT\mid 1)\widehat{\Pr}(\CP\mid 0)-\widehat{\Pr}(\AT\mid 0)\widehat{\Pr}(\CP\mid 1) }. 
\end{equation*}

%I first look at the denominator of $\hat{\rho}_1$, which involves four conditional probabilities of subgroups.  


Let $\phi_i^{d1}$ denote the influence function for the denominator of $\hat{\rho}_1$, derived as
\begin{equation*}
\phi_i^{d1}=\Pr(\AT\mid 1)\phi_i^{\CP0}+\Pr(\CP\mid 0) \phi_i^{\AT1}-\Pr(\AT\mid 0)\phi_i^{\CP1}-\Pr(\CP\mid 1)\phi_i^{\AT0}.
\end{equation*}

 I look at the numerator of estimator $\hat{\rho}_1$. Let $\phi_i^{YDZW}=Y_iD_iZ_iW_i-E[YDZW]$ for any random variables $(Y, D, Z, W)$.  According to the definition of $\hat{r}_1(1), \hat{r}_1(0)$, their influence functions $\phi_i^{r_{11}}$ and $\phi_i^{r_{10}}$ are shown as
\begin{equation*}
\begin{aligned}
\phi_i^{r_{11}}&=\frac{\phi_i^{YDZW}}{E[ZW]}-\frac{E[YDZW]\phi_i^{ZW}}{(E[ZW])^2}-\left\{\frac{\phi_i^{YD\tilde{Z}W}}{E[\tilde{Z}W]}-\frac{E[YD\tilde{Z}W]\phi_i^{\tilde{Z}W}}{(E[\tilde{Z}W])^2} \right\}, \\
\phi_i^{r_{10}}&=\frac{\phi_i^{YDZ\tilde{W} }}{E[Z\tilde{W} ]}-\frac{E[YDZ\tilde{W} ]\phi_i^{Z\tilde{W} }}{(E[Z\tilde{W} ])^2}-\left\{ \frac{\phi_i^{YD\tilde{Z}\tilde{W} }}{E[\tilde{Z}\tilde{W} ]}-\frac{E[YD\tilde{Z}\tilde{W} ]\phi_i^{\tilde{Z}\tilde{W} }}{(E[\tilde{Z}\tilde{W} ])^2} \right\}.
\end{aligned}
\end{equation*}

Let $\phi_i^{n1}$ denote the influence function of the numerator of $\hat{\rho}_1$, derived as
\begin{equation*}
\phi_i^{n1}=r_1(1)\phi_i^{\CP0}+\Pr(\CP\mid 0)\phi_i^{r_{11}}-r_1(0)\phi_i^{\CP1}-\Pr(\CP\mid 1)\phi_i^{r_{10}}.
\end{equation*}

Now we are ready to derive the influence function for estimator $\hat{\rho}_1$. Let $\phi_i^{\rho_1}$ denote the influence function for estimator $\hat{\rho}_1$, derived as follows:
\begin{equation*}
\phi_i^{\rho_1}=\frac{ \phi_i^{n1}-\rho_1 \phi_i^{d1} }{ \Pr(\AT\mid 1)\Pr(\CP\mid 0)-\Pr(\AT\mid 0)\Pr(\CP\mid 1) }.
\end{equation*}

Therefore, the asymptotic property of estimator $\hat{\rho}_1$ is derived as
\begin{equation*}
\sqrt{N}(\hat{\rho}_1-\rho_1)\rightarrow \mathcal{N}(0, \Var(\phi_i^{\rho_1} ) ).
\end{equation*}

The analysis for the estimators $\hat{\rho}_0$ and $\hat{w}^{\rho}_0$ can be shown similarly, so it is omitted here. Let $\phi_i^{\rho_0}$ and $\phi_i^{w^{\rho}_0}$ denote the influence function for $\hat{\rho}_0$ and $\hat{w}^{\rho}_0$ respectively. Then the influence function $\phi_i^{\LATE}$ for estimator $\widehat{\LATE}$ is expressed as
\begin{equation*}
\phi_i^{\LATE}=\phi_i^{\IV_1}-\rho_1\phi_i^{w^{\rho}_1}-w^{\rho}_1\phi_i^{\rho_1}-\rho_0\phi_{i}^{w^{\rho}_0}-w^{\rho}_0\phi_i^{\rho_0}.
\end{equation*}

The asymptotic property for $\widehat{\LATE}$ is shown as
\begin{equation*}
\sqrt{N}(\widehat{\LATE}-\LATE) \rightarrow \mathcal{N}(0, \Var(\phi_i^{\LATE})).
\end{equation*}



\subsection{Proof of Proposition \ref{prop:exte}} \label{proof:exte}

\begin{proof}

Following the proof in \ref{proof:thm1}, LATE can be divided into two groups $Z=1$ and $Z=0$ as follows:
\begin{equation*}
\begin{aligned}
\LATE=&E[Y_1-Y_0\mid \CP]\\
=&E[Y_{1, 1}-Y_{0, 1}\mid \CP, Z=1]\Pr(Z=1)+E[Y_{1, 0}-Y_{0, 0}\mid \CP, Z=0]\Pr(Z=0)\\
=&E[Y_{1, 1}-Y_{0, 1}\mid \CP]\Pr(Z=1)+E[Y_{1, 0}-Y_{0, 0}\mid \CP]\Pr(Z=0).
\end{aligned}
\end{equation*}

Using $\rho_{\CP, d}=E[Y_{d, 1}-Y_{d, 0}\mid \CP]$ to substitute $E[Y_{0, 1} \mid \CP]$ and $E[Y_{1, 0} \mid \CP]$ has the following implication:
\begin{equation*}
\LATE=E[Y_{1, 1}-Y_{0, 0}\mid \CP]-\rho_{\CP, 1}\Pr(Z=0)-\rho_{\CP, 0} \Pr(Z=1).
\end{equation*}


As shown in \ref{proof:thm1}, using variation in $r_1(w)$ with respect to $w$ can identify $\rho_{\AT, 1}$ and $E[Y_{1,1}\mid \CP]$ as follows:
\begin{equation*}
\begin{aligned}
&\rho_{\AT, 1}=\frac{r_1(1)\Pr(\CP\mid 0)-r_1(0)\Pr(\CP\mid 1) }{\Pr(\AT\mid 1)\Pr(\CP\mid 0)-\Pr(\AT\mid 0)\Pr(\CP\mid 1) }, \\
&E[Y_{1,1}\mid \CP]=\frac{r_1(1)-\rho_1\Pr(\AT\mid 1) }{\Pr(\CP\mid 1 ) }. \\
\end{aligned}
\end{equation*}

Similarly, the direct effect  $\rho_{\NT, 0}$ and $E[Y_{0 ,0}\mid \CP]$ are identified by exploiting variation in $r_0(w)$:
\begin{equation*}
\begin{aligned}
&\rho_{\NT, 0}=\frac{r_0(1)\Pr(\CP\mid 0)-r_0(0)\Pr(\CP\mid 1) }{\Pr(\NT\mid 1)\Pr(\CP\mid 0)-\Pr(\NT\mid 0)\Pr(\CP\mid 1) }, \\
&E[Y_{0, 0}\mid \CP]=-\frac{r_0(1)-\rho_0\Pr(\NT\mid 1) }{\Pr(\CP\mid 1) }.\\
\end{aligned}
\end{equation*}

Under Assumption \ref{ass:dir} about the difference in direct effects across subgroups, $\rho_{\CP, d}$ for $d\in\{0, 1\}$ can be bounded as
\begin{equation*}
\begin{aligned}
\rho_{\AT, 1}-k_1\leq \rho_{\CP, 1}\leq \rho_{\AT, 1}+k_1, \\
\rho_{\NT, 0}-k_0\leq \rho_{\CP, 0}\leq \rho_{\NT, 0}+k_0. \\
\end{aligned}
\end{equation*}

Therefore, the bounds on $\LATE$ is established as follows:
\begin{equation*}
\begin{aligned}
\LATE \geq E[Y_{1, 1}-Y_{0, 0}\mid \CP]-(\rho_{\AT, 1}+k_1)\Pr(Z=0)-(\rho_{\NT, 0}+k_0) \Pr(Z=1), \\
\LATE \leq E[Y_{1, 1}-Y_{0, 0}\mid \CP]-(\rho_{\AT, 1}-k_1)\Pr(Z=0)-(\rho_{\NT, 0}-k_0) \Pr(Z=1).
\end{aligned}
\end{equation*}

Plugging into the formulas for $(\rho_{\AT, 1}, \rho_{\NT, 0})$ and $(E[Y_{1, 1}\mid \CP], E[Y_{0 ,0}\mid \CP])$ lead to the results in Proposition \ref{prop:exte}. The lower bound is achieved when $\rho_{CP, 1}=\rho_{\AT, 1}+k_1$ and $\rho_{CP, 0}=\rho_{\NT, 0}+k_0$, and the upper bound is achieved when $\rho_{CP, 1}=\rho_{\AT, 1}-k_1$ and $\rho_{CP, 0}=\rho_{\NT, 0}-k_0$. Therefore, the bounds are sharp.

\end{proof}


\subsection{More Simulation Results} \label{simu}

This section presents the simulation results of the two estimators $(\hat{\theta}_{zw}, \hat{\theta}_z)$ when the size of the subgroups changes. Consider that the potential treatments $(D_1, D_0)$ is given as follows:
\begin{equation*}
\begin{aligned}
D_1&=\mathbbm{1}\{\epsilon\leq 1   \}, \\
D_0&=\mathbbm{1}\{\epsilon\leq k   \}.
\end{aligned}
\end{equation*}

The value of $k$ determines the size of the three subgroups, and the size of compliers $\Pr(k<\epsilon<1)$ decreases when the value of $k$ increases. 
I consider three specifications of $k$: $k\in \{-0.25, 0, 0.25\}$. 

Table \ref{table3} and \ref{table4} show the results of the two estimators $(\hat{\theta}_{zw}, \hat{\theta}_z)$ under different values of $k$ and different values of direct effects $\rho$ with the sample size $N=1000$ and $N=4000$, respectively. The two estimators  $(\hat{\theta}_{zw}, \hat{\theta}_z)$ both perform better when the size of compliers increases, while the estimator $\hat{\theta}_{zw}$ uniformly performs better than the estimator $\hat{\theta}_z$ with nonzero direct effects regardless of the size of compliers. Therefore, the robustness feature of the estimator $\hat{\theta}_{zw}$ to nonzero direct effects still holds under different probabilities of subgroups. 


\begin{table}[!htbp]
\centering
\caption{Performance Comparisons of $\hat{\theta}_{zw}$ and $\hat{\theta}_z$ $(N=1000)$  }
\label{table3}
\begin{tabular}{c |cccc|cccc}
\hline
\hline
\multirow{2}{*}{$\rho$}&\multicolumn{4}{c|}{$\hat{\theta}_{zw}$}  &  \multicolumn{4}{c}{$\hat{\theta}_z$}  \\
\cline{2-9} 
  & Bias & SD & rMSE & MAD  & Bias & SD  & rMSE & MAD \\
\hline
\multicolumn{9}{c}{$k=-0.25$} \\
\hline
 $0$ \ & 0.016 & 0.459 & 0.459 & 0.362  & 0.004 &0.142 &0.142 &0.113  \\ [1.0ex]       
$0.5$ \ & 0.016 & 0.459 & 0.459 & 0.362 &1.145 & 0.173 & 1.158 &1.145  \\ [1.0ex]                                                                                                                                                                                                                                                                                                                                                                                                                                                                                                                                                                                                                                                                     $1$ \  & 0.016 & 0.459 & 0.459 & 0.362  & 2.285 & 0.225 & 2.296 & 2.285 \\ [1.0ex]  
$-1$ \  & 0.016 & 0.459 & 0.459 & 0.362  & -2.277 & 0.178 &2.284 &2.277  \\ [1.0ex] 
\hline
\multicolumn{9}{c}{$k=0$} \\
\hline
 $0$ \ & 0.025 & 0.504 & 0.505 & 0.396 & 0.006 &0.183 &0.184 &0.146  \\ [1.0ex]       
$0.5$ \ & 0.025 & 0.504 & 0.505 & 0.396 &1.481 & 0.245 & 1.501 &1.481  \\ [1.0ex]                                                                                                                                                                                                                                                                                                                                                                                                                                                                                                                                                                                                                                                                     $1$ \  & 0.025 & 0.504 & 0.505 & 0.396  & 2.956 & 0.342 & 2.976 & 2.956 \\ [1.0ex]  
$-1$ \  & 0.025 & 0.504 & 0.505 & 0.396  & -2.944 & 0.269 &2.956 &2.944  \\ [1.0ex] 
\hline
\multicolumn{9}{c}{$k=0.25$} \\
\hline
 $0$ \ &0.057 &0.915 &0.917 & 0.508 & 0.010 &0.262 &0.262 &0.207  \\ [1.0ex] 
 $0.5$ \  &0.057 &0.915 &0.917 & 0.508 & 2.100 &0.403 & 2.139 & 2.100   \\ [1.0ex] 
 $1$ \  &0.057 &0.915 &0.917 & 0.508 &  4.190 & 0.615 &4.235 & 4.190     \\ [1.0ex] 
 $-1$ \  &0.057 &0.915 &0.917 & 0.508 & -4.169 &0.498 &4.199 &4.169     \\   [1.0ex] 

\hline
\end{tabular}
\end{table}


\begin{table}[!htbp]
\centering
\caption{Performance Comparisons of $\hat{\theta}_{zw}$ and $\hat{\theta}_z$ $(N=4000)$  }
\label{table4}
\begin{tabular}{c |cccc|cccc}
\hline
\hline
\multirow{2}{*}{$\rho$}&\multicolumn{4}{c|}{$\hat{\theta}_{zw}$}  &  \multicolumn{4}{c}{$\hat{\theta}_z$}  \\
\cline{2-9} 
  & Bias & SD & rMSE & MAD  & Bias & SD  & rMSE & MAD \\
\hline
\multicolumn{9}{c}{$k=-0.25$} \\
\hline
 $0$ \  & 0.007 & 0.214 & 0.214 & 0.170  & -0.000 &0.072 &0.072 &0.057  \\ [1.0ex]       
$0.5$ \ & 0.007 & 0.214 & 0.214 & 0.170 &1.136 & 0.086 & 1.139 &1.136  \\ [1.0ex]                                                                                                                                                                                                                                                                                                                                                                                                                                                                                                                                                                                                                                                                     $1$ \  & 0.007 & 0.214 & 0.214 & 0.170   & 2.273 & 0.110 & 2.275 & 2.273 \\ [1.0ex]  
$-1$ \  & 0.007 & 0.214 & 0.214 & 0.170  & -2.274 & 0.089 &2.275 &2.274  \\ [1.0ex] 
\hline
\multicolumn{9}{c}{$k=0$} \\
\hline
 $0$ \ &0.009 &0.230 &0.230 & 0.183 &-0.001 &0.092 &0.092 &0.073  \\ [1.0ex] 
 $0.5$ \  &0.009 &0.230 &0.230 & 0.183 & 1.466 &0.119 & 1.470 & 1.466   \\ [1.0ex] 
 $1$ \  &0.009 &0.230 &0.230 & 0.183&  2.932 & 0.165 &2.936 &2.932      \\ [1.0ex] 
$-1$ \  &0.009 &0.230 &0.230 & 0.183 & -2.933 &0.133 &2.936 &2.933     \\   [1.0ex] 
\hline
\multicolumn{9}{c}{$k=0.25$} \\
\hline
 $0$ \   &0.013 &0.273 &0.273 & 0.217 & -0.002 &0.130  &0.130 &0.104  \\ [1.0ex] 
 $0.5$ \  &0.013 &0.273 &0.273 & 0.217 & 2.064 &0.193 & 2.073 & 2.064   \\ [1.0ex] 
 $1$ \  &0.013 &0.273 &0.273 & 0.217 &  4.129 & 0.291 &4.140 & 4.129     \\ [1.0ex] 
 $-1$ \  &0.013 &0.273 &0.273 & 0.217 & -4.133 &0.242 &4.140 &4.133     \\   [1.0ex] 

\hline
\end{tabular}
\end{table}

%From the above two tables, the two estimators both have better performance when the size of compliers increases (smaller $k$). The bias of the estimator $\hat{\theta}_z$ depends on the size of compliers, and it increases when the size of compliers decreases. The estimator $\hat{\theta}_{zw}$ depends on the probability of the three subgroups in a more complicated way, while it has better performance when the size of compliers increases in this setup.
%The comparisons in the above two tables show that the estimator $\hat{\theta}_{zw}$ still has better performance than the estimator $\hat{\theta}_z$ with nonzero direct effects regardless of the value of $k$. Therefore, the robustness feature of the estimator $\hat{\theta}_{zw}$ to nonzero direct effects also holds under different probabilities of subgroups. 




%\begin{table}[!htbp]
%\centering
%\caption{Performance of Direct Effects $\hat{\rho}_1$ and $\hat{\rho}_0$}
%\label{table:direct}
%\begin{tabular}{cc |cccc|cccc}
%\hline
%\hline
%\rule{0pt}{20pt} \multirow{2}{*}{$N$} &  \multirow{2}{*}{$k$}&\multicolumn{4}{c|}{$\hat{\rho}_1$}  &  \multicolumn{4}{c}{$\hat{\rho}_0$}  \\
%\cline{3-10} 
% \rule{0pt}{20pt}& & Bias & SD & rMSE & MAD  & Bias & SD  & rMSE & MAD \\
%\hline
%\rule{0pt}{20pt}\multirow{3}{*}{1000} & -0.25 & -0.015 & 0.259 & 0.259 & 0.200 & 0.005 &0.361 &0.361 &0.284 \\
%\rule{0pt}{20pt}  & 0 & -0.015 & 0.210 & 0.210 & 0.160 & 0.006 &0.351 &0.351 &0.275  \\       
%\rule{0pt}{20pt}  & 0.25 & -0.020 & 0.281 & 0.282 & 0.146 & 0.004 &0.339 &0.339 &0.264  \\       
%\hline
%\rule{0pt}{20pt}4000 & -0.25 & -0.004  & 0.120 &0.120 &0.094  &-0.003 & 0.175 &0.175 &0.139  \\
%\rule{0pt}{20pt}  & 0 &   \\       
%\rule{0pt}{20pt}  & 0.25 & -0.004 & 0.079 & 0.079 & 0.062 & -0.003 &0.163 &0.163 &0.129  \\       
%\hline
%\end{tabular}
%\end{table}





\end{document}  

%Now I look at the IV estimand. I divide $E[Y\mid Z=1, w]$ into the three subgroups:
%\begin{equation*}
%\begin{aligned}
%&E[Y\mid Z=1, w] \\
%%=&E[Y_{11}\mid \AT, Z=1, w]\Pr(\AT\mid Z=1, w)+E[Y_{11}\mid \CP, Z=1, w]\Pr(\CP\mid Z=1, w)+E[Y_{01}\mid \NT, Z=1, w]\Pr(\NT\mid Z=1, w) \\
%=&E[Y_{11}\mid \AT]\Pr(\AT\mid w)+E[Y_{11}\mid \CP]\Pr(\CP\mid w)+E[Y_{01}\mid \NT]\Pr(\NT\mid w). \\
%\end{aligned}
%\end{equation*}
%
%This condition holds by the independence condition of the two instruments $(Z, W)$. By taking expectation over $w$, the following condition holds:
%\begin{equation*}
%E[Y\mid Z=1]=E[Y_{11}\mid \AT]\Pr(\AT)+E[Y_{11}\mid \CP]\Pr(\CP)+E[Y_{01}\mid \NT]\Pr(\NT).
%\end{equation*}
%
%We can express $E[Y\mid Z=0]$ similarly, then the difference $E[Y\mid Z=1]-E[Y\mid Z=0]$ is written as
%\begin{equation*}
%E[Y\mid Z=1]-E[Y\mid Z=0]=\rho_1\Pr(\AT)+\rho_0\Pr(\NT)+E[(Y_{11}-Y_{00})\mid \CP]\Pr(\CP).
%\end{equation*}
%
%By the independence condition of the two instruments $(Z, W)$, the denominator of the IV estimand is given as
%\begin{equation*}
%E[D\mid Z=1]-E[D\mid Z=0]=\Pr(\CP).
%\end{equation*}
%
%Then, the IV estimand can be expressed as
%\begin{equation*}
%\IV=E[(Y_{11}-Y_{00})\mid \CP]+\rho_1\frac{\Pr(\AT)}{\Pr(\CP)}+\rho_0\frac{\Pr(\NT)}{\Pr(\CP)}.
%\end{equation*}
%
%Therefore, the relationship between the local average treatment effect and the IV estimand still holds:
%\begin{equation*}
%\LATE=\IV-\rho_1w^{\rho}_1-\rho_0w^{\rho}_0.
%\end{equation*}
%where $w^{\rho}_1=\frac{\Pr(\AT)}{\Pr(\CP)}+\Pr(Z=0)$ and $w^{\rho}_0=\frac{\Pr(\NT)}{\Pr(\CP)}+\Pr(Z=1)$.


%
%\section{Testable Implications for Two Instruments}
%My identification result relies on an additional instrument and requires this instrument to satisfy exclusion restriction assumption (assumption \ref{IV2}). This section provides some testable implications for this assumption when the additional instrument is binary. The idea is that we can point and partially  identify some conditional expectations of  the outcome at different values of $w$ using observed variables $(Y, D, Z, W)$. Under the exclusion restriction of the instrument $W$, the expectations should be the same under different values of $w$. Therefore their identified sets at different values of $w$ should have interactions. If not, then we know the instrument violates the exclusion restriction assumption and we can reject this instrument.
%%Also we can provide sharp bounds for expectations for compliers and their bounds should have overlaps at different values of $w$.
%
%When $Z=0$, always takers are the only subpopulation who are treated. Therefore we can point identify $E[Y_1|A, Z=0, W=w]$ under the assumptions of the instrument $Z$ (assumption \ref{treat2}),
%\begin{equation*}
%\begin{aligned}
%E[Y|D=1, Z=0, W=w]&=E[Y_1|A, Z=0, W=w] \\
%%\Pr(Y\leq y, D=0|Z=1, W=w)&=\Pr(Y_0\leq y|N, Z=1, W=w)\Pr(N|W=w)\\
%\end{aligned}
%\end{equation*}
%Similarly we can also identify the expectation $E[Y_0|N, Z=1, W=w]$ for every $w\in\{0, 1\}$. The expectations should be the same under different values of $w$ under the exclusion restriction of the instrument $W$ (assumption \ref{IV2}).  
%
%When $Z=1$, both always takers and compliers are treated. Let $q_{\text{type}, w}=\Pr(\text{type}|W=w)/\Pr(D=1|Z=1, W=w)$ denotes the ratio of the type$\in\{A, C\}$ among the treated when $Z=1$. Then $E[Y|D=1, Z=1, W=w]$ can be expressed as a mixture of always takers and compliers,
%\begin{multline*}
%E[Y|D=1, Z=1, W=w]=E[Y_1|A, Z=1, W=w] q_{A, w}+\\E[Y_1|C, Z=1, W=w] q_{C, w}
%\end{multline*}
%Let $F_{Y|D, Z, W}(y|d,z,w)$ denotes the distribution of $Y$ conditional on $D=d, Z=z, W=w$ and 
%$y_{q_{\text{type}, w}}=F_{Y|D, Z, W}^{-1}(q_{\text{type}, w}|1,1,w)$ be the $q_{\text{type}, w}$th conditional quantile of the distribution when $D=1, Z=1, W=w$.
%Then the sharp bounds for $E[Y_1|\text{type}, Z=1, W=w]$ can be derived as follows (see \cite{huber2015}) for type$\in\{A, C\}$,
%\begin{equation*}
%\begin{aligned}
%E[Y_1|\text{type}, Z=1, W=w]&\geq E[Y|D=1, Z=1, W=w, Y\leq y_{q_{\text{type}, w}}]  \\
%E[Y_1|\text{type}, Z=1, W=w] &\leq E[Y|D=1, Z=1, W=w, Y\geq y_{1-q_{\text{type}, w}}] 
%\end{aligned}
%\end{equation*}
%The lower (upper) bound is achieved when the type is concentrated on the lower (upper) tail of the distribution. Since their value  at different values of $w$ should be the same under assumption \ref{IV2} , the sharp bounds at different values of $w$ should have intersections. Let $r_{\text{type},w}=\Pr(\text{type}|W=w)/\Pr(D=0|Z=0, W=w)$ denote the ratio of type among the untreated when $Z=0$ and $y_{r_{\text{type}, w}}=F_{Y|D, Z, W}^{-1}(r_{\text{type}, w}|0,0,w)$ denote the $r_{\text{type}, w}$th quantile of the distribution of  $F_{Y|D, Z, W}(y|0,0,w)$ for type$\in \{N, C\}$.  We can derive sharp bounds for $E[Y_0|\text{type}, Z=0, W=w]$ for type$\in \{N, C\}$ similarly. The next proposition summarizes the testable implications for the exclusion restriction assumption for the instrument $W$.
%
%\begin{thm}
%Under assumption \ref{treat2}, the testable implications for assumption \ref{IV2} are summarized in the following moment conditions:
%\begin{equation*}
%\left\{
%\begin{aligned}
%E[Y|D=1, Z=0, W=1]&=E[Y|D=1, Z=0, W=0] \\
%E[Y|D=0, Z=1, W=1]&=E[Y|D=0, Z=1, W=0] \\
%\min_w\left\{E[Y|D=1, Z=1, W=w, Y\geq y_{1-q_{A, w}}]\right\} &\geq 
%\max_w \left\{E[Y|D=1, Z=1, W=w, Y\leq y_{q_{A, w}}] \right\} \\
%\min_w\left\{E[Y|D=1, Z=1, W=w, Y\geq y_{1-q_{C, w}}]\right\} &\geq 
%\max_w \left\{E[Y|D=1, Z=1, W=w, Y\leq y_{q_{C, w}}] \right\} \\
%\min_w\left\{E[Y|D=0, Z=0, W=w, Y\geq y_{1-r_{\text{N}, w}}]\right\}  
%&\geq \max_w \left\{E[Y|D=0, Z=0, W=w, Y\leq y_{r_{\text{N}, w}}] \right\} \\
%\min_w\left\{E[Y|D=0, Z=0, W=w, Y\geq y_{1-r_{\text{C}, w}}]\right\}  
%&\geq \max_w \left\{E[Y|D=0, Z=0, W=w, Y\leq y_{r_{\text{C}, w}}] \right\} \\
%\end{aligned}
%\right.
%\end{equation*}
%\end{thm}

%When the instrument $W$ takes more than three values, there will be more testable implications. We can use different pairs of $W$ to identify the expectations of compliers and their value should be the same under different pairs.
%\begin{equation*}
%\begin{aligned}
%\frac{\Pr(Y\leq y, D=1|Z=0, W=1)}{\Pr(D=1|Z=0,W=1)}=\frac{\Pr(Y\leq y, D=1|Z=0, W=0)}{\Pr(D=1|Z=0,W=0)} \\
%\frac{\Pr(Y\leq y, D=0|Z=1, W=1)}{\Pr(D=0|Z=1,W=1)}=\frac{\Pr(Y\leq y, D=0|Z=1, W=0)}{\Pr(D=0|Z=1,W=0)}
%\end{aligned}
%\end{equation*}
