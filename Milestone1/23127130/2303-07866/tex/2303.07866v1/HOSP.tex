\documentclass[main]{siamart220329}
%\documentclass[review]{siamart220329}

\usepackage{amsfonts}
\usepackage{graphicx}
\usepackage{epstopdf}
\usepackage{algorithmic}
\usepackage{hyperref}
 \hypersetup{
     colorlinks=true,
     linkcolor=black,
     filecolor=black,
     citecolor = black,      
     urlcolor=black,
     }
\usepackage{amsmath, amssymb}
\usepackage{relsize}
\usepackage{pifont}
\usepackage{enumitem}
\usepackage{oubraces}
\usepackage{listings}
\usepackage{booktabs}
\usepackage{xfrac}
\usepackage{cleveref}
\usepackage{caption}
\usepackage{tabularx}
\usepackage{array}
\usepackage{multirow}
\usepackage{mathrsfs}
\usepackage{microtype}
% \usepackage[square,sort,comma,numbers]{natbib}
\usepackage{natbib}
    
% Tikz/PGF packages
\usepackage{tikz}
\usetikzlibrary{arrows,backgrounds,patterns}
\usetikzlibrary{matrix,shapes,fit,calc,shadows,plotmarks}
\usetikzlibrary{decorations}
\usetikzlibrary{decorations.markings}
\RequirePackage{pgfplotstable}
\RequirePackage{pgfplots}
\usepackage{pgfplotstable}
\usepackage{xparse}

\newcommand\eps{\epsilon}
\renewcommand*{\i}{\mathrm{i}}
\newcommand*{\e}{\mathrm{e}}
\newcommand*{\im}{\mathrm{i}}
\newcommand*{\ep}{\epsilon}
\newcommand*{\de}{\operatorname{d\!}{}} % Do this
\newcommand{\dd}[2]{\frac{\de#1}{\de#2}}
\newcommand{\pd}[2]{\frac{\partial#1}{\partial#2}}
\newcommand{\pD}[2]{\dfrac{\partial#1}{\partial#2}}
\newcommand{\pdd}[2]{\frac{\partial^2\!#1}{\partial#2^2}}
\newcommand{\myblue}[1]{{\color{blue}#1}}

\def\Xint#1{\mathchoice
   {\XXint\displaystyle\textstyle{#1}}%
   {\XXint\textstyle\scriptstyle{#1}}%
   {\XXint\scriptstyle\scriptscriptstyle{#1}}%
   {\XXint\scriptscriptstyle\scriptscriptstyle{#1}}%
   \!\int}
\def\XXint#1#2#3{{\setbox0=\hbox{$#1{#2#3}{\int}$}
     \vcenter{\hbox{$#2#3$}}\kern-.5\wd0}}
\def\ddashint{\Xint=}
\def\dashint{\Xint-}

\headers{J. Shelton, S. Crew, P. H. Trinh}{The higher-order Stokes phenomenon}
\title{Exponential asymptotics and higher-order Stokes phenomenon in singularly perturbed ODEs
\thanks{Submitted to the editors 01/03/2023. JS and SC are joint lead authors.}
\funding{PHT is supported by the Engineering and Physical Sciences Research Council EP/V012479/1. This work was supported by EPSRC grant EP/R014604/1.}
}

\author{Josh Shelton\thanks{Department of Mathematical Sciences, University of Bath, BA2 7AY, UK (\email{j.shelton@bath.ac.uk}, \email{p.trinh@bath.ac.uk}).}
\and Samuel Crew\footnotemark[2] \thanks{Faculty of Computer Science, Ruhr University, Bochum, Germany. (\email{samuel.crew@ruhr-uni-bochum.de}),\\
Max Planck Institute for Security and Privacy, Bochum, Germany. (samuel.crew@mpi-sp.org)}
\and Philippe H. Trinh\footnotemark[2]}



\begin{document}

\maketitle
\begin{abstract} 
Outside the area of exponential asymptotics, the concept of the higher-order Stokes phenomenon remains somewhat esoteric. The intention of this work is to provide several examples of relatively simple ordinary differential equations where the phenomenon arises, and to develop additional practical methodologies for its study.
%
In particular, we show how the higher-order Stokes phenomenon may be derived through the hyperterminant representation of the late-term divergence of an asymptotic expansion developed by Olde Daalhuis ({\it J. Comp. Appl. Math.} vol. 76, 1996, pp. 255-264). Lower-order components of the factorial-over-power divergence are considered, for which late-late-term divergence arises. Borel resummation of these lower-order components reveals the higher-order Stokes phenomenon, in which new components of the late-terms of the expansion are smoothly switched on with an error function dependence. The techniques are firstly demonstrated with a second-order linear inhomogeneous ODE that exemplifies the simplest example of higher-order Stokes phenomena. Further examples studied include higher-order equations and eigenvalue problems. 
%Connections between matched asymptotic expansions and Borel plane analysis are discussed.
\end{abstract}

\begin{keywords} 
Stokes phenomenon, exponential asymptotics, beyond-all-orders
analysis
\end{keywords}

\section{Introduction} \label{sec:intro}

Outside the exponential asymptotics community, the concept of the \emph{higher-order Stokes phenomenon} (HOSP) is still considered to be relatively obscure. Yet,  as we shall explain in this work, it occurs generically and routinely in divergent asymptotic expansions. Deep understanding of the phenomenon seems to be necessary in order to perform beyond-all-orders analysis of multi-dimensional problems (such as in partial differential equations) arising in many physical applications \citep{howls_2004,chapman_2005,body2005exponential}. The emphasis of this work is to provide some relatively simple examples of the HOSP in the context of singularly perturbed ordinary differential equations. For example, we show that it occurs in the asymptotic analysis of the seemingly innocuous equation,
\begin{equation} \label{eq:themainone_intro}
 \eps^2 y''+ \eps (1+z)y' + z y = 1
\end{equation}
with $\ep \to 0$ and $y = y(z)$. We shall explain why the above is likely the simplest example of HOSP, and we develop applicable techniques for its study. Finally, our work serves to survey perspectives of the phenomenon across disciplines in asymptotic analysis.

We first review the regular Stokes phenomenon. In applications to singularly perturbed ordinary differential equations, the solution is typically expanded as an asymptotic expansion. For example, we approximate the solution as
\begin{equation} \label{eq:base}
y(z; \ep) \sim y^{(0)}(z; \epsilon) \equiv y_0(z) + \ep y_1(z) + \cdots + \ep^n y_n(z) + \cdots
\end{equation}
with $\ep \to 0$ and $z \in \mathbb{C}$, and refer to such an expansion as the {\itshape base series}. The late terms, $y_n$, with $n\to\infty$ diverge and this divergence is coupled to the existence of Stokes lines in $z\in\mathbb{C}$. Across such lines, exponentially-small terms, say $\sigma (z,\ep)\e^{-\chi(z)/\ep}$, are switched-on within a small boundary layer via the (regular) Stokes phenomenon. The quantity $\sigma$ is called the \emph{Stokes multiplier}. By now, such ideas are well established [cf. reviews in e.g. \cite{dingle_book} and \cite{berry_1989}], and there exists a plethora of tools in asymptotics beyond-all-orders to study these effects in the context of ordinary and partial differential equations, difference equations, integral equations, and beyond.

Typically, the divergence of $y_n$ is approximated by a factorial-over-power growth as $n \to \infty$. However, this is generally only the leading order term in an asymptotic expansion in decreasing powers of $n$. If $\delta = 1/n$ is now considered to be the perturbative parameter, then it is interesting to consider if $y_n$ can, itself, exhibit Stokes phenomena when analytically continued in $z$. Indeed this is the higher-order Stokes phenomenon associated with {\itshape higher-order Stokes lines} (HOSL). The standard Stokes phenomenon described above is linked to the divergence in $y_n$, but now, since these approximations may be non-uniform in $\mathbb{C}_{z}$, the regular Stokes line structure is consequently altered. 

For the practitioner interested in the prediction of exponentially-small terms, two significant changes can occur as a consequence of HOSP:
\begin{enumerate}[label=(\roman*),leftmargin=*, align = left, labelsep=\parindent, topsep=3pt, itemsep=2pt,itemindent=0pt]
\item Stokes lines can be truncated or there can be new Stokes lines; and 
\item there can be atypical values of the Stokes multiplier(s). 
\end{enumerate}
%
We shall explore both of these consequences in this work.



\subsection{Prior literature and ongoing work}

\noindent There are a number of seminal works that study HOSP, though not necessarily in the same context of our work here. The origins of the phenomenon are generally attributed to the work of \cite{berk1982new}. In their paper, titled ``\emph{New Stokes' line in WKB theory}", the altered structure of Stokes lines was discovered in the context of an integral equation (Pearcey's integral). The theory of such ``new Stokes lines" was then extensively studied in the exact WKBJ (Liouville-Green) communities (cf. \citealt{aoki1994new} and references therein). Further WKBJ examples with exact integral solutions were studied by \cite{aoki1998exact, aoki2001exact}; a singularly perturbed Painlev\'e equation appears in the work of \cite{honda2007stokes}; and HOSP appears as well in the analysis of the Henon map in quantum chaos by \cite{shudo2007role}. We refer readers to reviews by \emph{e.g.} \cite{aoki2001exact}, \cite{honda2015virtual}, and \cite{takei2017wkb} and references therein for a historical survey. 

We note an important line of research in the early 2000s, initiated by a triplet of works by \cite{howls_2004}, \cite{chapman_2005}, and \cite{body2005exponential}, which characterise some important connections between beyond-all-orders analysis of partial differential equations and the HOSP. These three works apply a number of different methodologies to beyond-all-order asymptotics, and present complementary views of similar problems. In particular, the work of \cite{howls_2004} returned to analyse the Pearcey integral, discussed above, and established a connection between HOSP and the prior theory of hyperterminants developed by \cite{daalhuis1996hyperterminants, daalhuis1998hyperterminants}. A particularly simple example of the HOSP was presented by \cite{daalhuis2004higher} in the context of an inhomogeneous linear second-order ODE. 

There have been a number of recent exciting works concerning HOSP in both applications and theoretical studies. For example, we highlight the work of \cite{nemes2022dingle} who studied atypical Stokes multipliers arising in the asymptotic representation of the Gamma function. Further applications have been discovered in fluid dynamics by \cite{lustri2019three} for three dimensional water waves, and \cite{shelton2022Hermite,shelton2023kelvin} for geophysical instabilities.
 Most recently, we note unpublished investigations by \cite{smoothingAdri} into smoothing of HOSP, and \cite{iniKing} on the occurrence of HOSP in linear PDEs.

 
\subsection{Outline}

In section~\ref{sec:background}, we present a unified view of three different perspectives for understanding HOSP: the Borel plane considered by {\it e.g.} \cite{howls_2004}, hyperasymptotics and analysis of the transseries \citep{daalhuis1996hyperterminants,daalhuis1998hyperterminants}, and the application of a complete late-term ansatz to study lower-order components (as $n \to \infty$) of the late terms. Thus for example, when our minimal example \eqref{eq:themainone_intro} is studied in the Borel plane, the transformed solution exhibits a square root branch point and a simple pole; we shall explain why these are essentially the minimal ingredients for developing HOSP. 

In the context of \eqref{eq:themainone_intro}, we shall consider corrections to the late-term divergence in the form of a $1/n$ transseries expansion (this is analogous to the hyperterminant representation by \citealt{daalhuis1996hyperterminants,daalhuis1998hyperterminants}). In section \ref{sec:modelprob}, it is shown that a new divergent series emerges within the late-terms themselves, which we denote the {\it late-late-terms} of the asymptotic expansion. Borel summation of these late-late-terms reveals the higher-order Stokes lines, across which new components of the late-terms are seen to smoothly switch on with an error function dependence.


Finally, in section \ref{sec:otherexamples}, we apply our general techniques to detect the higher-order Stokes lines in three previously encountered ODEs. These problems had previously been studied through methods that cannot be easily generalised to arbitrary nonlinear differential equations---such as steepest descent analysis on an integral representation, or evaluation of a known recurrence relation. Together, the methods studied in this work will provide additional practical tools for the derivation of HOSP and HOSL.


\section{Transseries, divergence, and the Borel transform} \label{sec:background}


\subsection{Transseries relations}\label{sec:transseriesrelations}

 We begin with a review of the relationship between exponential asymptotic expansions (the {\it transseries}) and the late terms of an asymptotic series. 
 %
 Note that in general, the base series, written in \eqref{eq:base}, represents only one of many possible contributions to the formal parametric transseries of $y(z; \epsilon)$. The full transseries can be written as 
\begin{equation}\label{eq:MainTransseries}
\begin{split}
    y(z;\eps) &= \biggl[ y_0(z) + \ep y_1(z) + \cdots + \ep^n y_n(z) + \cdots \biggr]   \\
    &\quad+\sum_{j=1
    }\sigma_{j}(z; \ep)\biggl[y^{(j)}_0(z) + \ep y^{(j)}_1(z) + \cdots +\ep^n y^{(j)}_n(z) + \cdots\biggr] \e^{-\chi_j(z)/\ep},
     \end{split}
\end{equation}
where the components $y_n^{(j)}$ and $\sigma_j$ are to be determined, and following Dingle, the functions $\chi_j(z)$ are known as \textit{singulants}. We focus on factorially divergent asymptotic series; this structure frequently arises in solutions to singularly perturbed ordinary differential equations \citep{dingle_book}.

As we have introduced previously, it is possible to re-sum the divergent series \eqref{eq:base} using Borel resummation, or to equivalently study the remainder after optimal truncation. This procedure gives rise to the Stokes phenomenon, in which exponentially subdominant terms switch on across Stokes lines. For the case above, the associated Stokes lines of the asymptotic expansion are curves in the $\mathbb{C}_z$ plane specified by the loci
\begin{equation} \label{l_chij}
    l_{\chi_j}: \quad \text{Im}[\chi_j(z)] = 0 \qquad \text{and} \qquad \text{Re}[\chi_j(z)] \geq 0.
\end{equation}
Across such curves, terms proportional to $\e^{-\chi_j(z)/\ep}$ are switched on. Indeed, these additional components of the asymptotic solution are contained within the transseries \eqref{eq:MainTransseries}. Analysis of the Stokes phenomenon also determines the multipliers,  $\sigma_{j}$, which rapidly and smoothly transition across Stokes lines \citep{berry_1989}.

Let us assume that the components, $y_n(z)$, of the base series are meromorphic functions with singular behaviour at a point, say $z=z_{1}$. The leading-order divergence associated with this point is given by
\begin{equation}\label{eq:MainDivergence}
    y_{n}(z) \sim  y^{(1)}_0(z)\frac{\Gamma(n+\alpha_1)}{\chi_1(z)^{n+\alpha_1}} \qquad \text{as $n \to \infty$},
\end{equation}
where $\chi_1(z_1)=0$ and typically $\alpha_1$ is constant. We denote \eqref{eq:MainDivergence} as the \emph{late-term divergence}.
Note that the singulant, $\chi_1$, and the amplitude function, $y_0^{(1)}$, in \eqref{eq:MainDivergence} above are those same functions that appear in the transseries \eqref{eq:MainTransseries}, about the exponential $\e^{-\chi_1/\ep}$. This connection between late terms and exponentials is fundamental in exponential asymptotics.

Surprisingly, the relationship between the late-terms, $y_n$, of the base expansion, and each of the amplitudes $y_p^{(j)}$ from the $\e^{-\chi_j/\ep}$ transseries component can be found through the consideration of the lower-order components of $y_n$ as $n \to \infty$ (typically in inverse powers of $n$). More specifically, we may include additional $1/n$ corrections to the late-terms of \eqref{eq:MainDivergence} and organise $y_n$ as follows:
\begin{equation}\label{eq:factorial123}
    y_{n}(z) \sim \sum_{j=1} \left(\sum_{p=0}^{\infty} y^{(j)}_{p}(z) \frac{\Gamma(n-p+\alpha_j)}{\chi_j(z)^{n-p+\alpha_j}}\right) \qquad \text{as $n \to \infty$}.
\end{equation}
Above, the outer sum over $j$ is over the singulants, $\chi = \chi_j$. The inner sum, indexed by $p$, includes the lower-order correction terms. 

We note that lower-order corrections to the late-term divergence have previously been calculated by {\it e.g.} \cite{king2001asymptotics} and \cite{shelton2022Hermite}, but these prior studies considered a series in powers of $n^{-1}$, and thus the connection with the transseries \eqref{eq:MainTransseries} was not clear. We refer the reader to e.g. \cite{crew2023resurgent} for a more thorough review and explanation of this connection between the $\epsilon$-transseries expansion \eqref{eq:MainTransseries} and the $1/n$ expansion of the late terms \eqref{eq:factorial123}.


\subsection{The Borel plane}\label{sec:BorelIntro}

Borel resummation encodes the parametric transseries \eqref{eq:MainTransseries} as a single function $y_B(w,z)$, which is holomorphic in $w$. The location of singularities of $y_B$ encode the singulants, $\chi_j(z)$, and the power of the singularity encodes the constants $\alpha_{j}$. To see this, consider the formal base series \eqref{eq:base}. The Borel transform with respect to $\epsilon$ is a power series on the \textit{Borel plane}, $w \in \mathbb{C}_w$, defined by\footnote{The $\delta$ term here is a formal symbol that allows us to consider the Borel transform of perturbative series beginning at $O(1)$. It is defined such that $\mathcal{B}^{-1}[\delta] = 1$. In the following we study analytic properties of the series $y_1(z) + w y_2(z) + \cdots$ where the leading term plays little role.}
\begin{equation}
    y_B(w,z) := \delta y_0(z) + \sum_{n=0}^{\infty} \frac{y_{n+1}(z)}{n!} w^n = y_0(z)\delta  + y_1(z) + y_2(z) w + \frac{y_3(z)}{2!} w^2 + \cdots.
\end{equation}
We assume throughout this work that $y_B$ is endlessly analytically continuable (cf. \citealt{mitschi2016divergent}) except at a finite number of points $z \in \mathbb{C}_z$, and extends to a function with at most exponential growth along any ray emanating from the origin. Under these conditions we may then define the inverse Borel transform of $y_B$ via 
\begin{equation}\label{eq:InverseBorel}
    y(z; \ep)=\mathcal{B}^{-1}[y_B]=\int_0^{\infty}y_B(w,z)\e^{-w/\ep}\de{w} = y_0(z) + \epsilon y_1(z) + \cdots.
\end{equation}

We now explain how Stokes phenomena can be understood via the Borel plane. Consider the asymptotic evaluation of the integral \eqref{eq:InverseBorel} for small $\epsilon$. Provided singularities of $y_B$ do not lie on the integration contour, we may demonstrate using the integral formula of the Gamma function that
$\mathcal{B}^{-1}[y_B] = y_0(z) + \epsilon y_1(z) + \cdots$. However, if we suppose now that $y_B(w,z)$ has an algebraic singularity in the analytic continuation, of the form
\begin{equation}
    y_B(w,z) = \frac{1}{(w-\chi_1(z))^{\alpha_1}}\Big[y_0^{(1)}(z) + \big(w-\chi_1(z)\big)y_1^{(1)} + \cdots\Big],
\end{equation}
then whenever the singularity at $w=\chi_1(z)$ crosses the integration contour $(0,\infty)$, we pick up an extra contribution to the asymptotics of $\mathcal{B}^{-1}[y_B]$ from the Hankel contour integral, denoted $\mathcal{H}_\chi$, 
\begin{equation}
    \int_{\mathcal{H}_{\chi}} \frac{y_0^{(1)}(z)}{\big(w-\chi_1(z)\big)^{\alpha_1}} \e^{-w/\epsilon} \,  \de{w}\sim \sigma_1(z,\ep) y_0^{(1)}(z)\e^{-\chi_1(z)/\epsilon},
\end{equation}
where $\sigma_1$ is the Stokes multiplier and can be derived via an inner-matching procedure. The above transition occurs across the locus $l_{\chi_1}$ in the physical $\mathbb{C}_z$ plane, which corresponds to the Stokes lines in \eqref{l_chij}; thus switching-on an exponentially small contribution in the transseries of $y(z,\epsilon)$. Further details of the above can be found in \cite{crew2023resurgent}. 


\subsection{Higher-order Stokes phenomena} \label{sec:HOSP_Borel}
Consider a scenario in which the Borel transform, $y_B$, has two singularities, say at $w = \chi_1(z)$ and $w = \chi_2(z)$, in the Borel plane $\mathbb{C}_w$. In the following, we present a geometrical explanation of HOSP. 

\begin{enumerate}[label=(\roman*),leftmargin=*, align = left, labelsep=\parindent, topsep=3pt, itemsep=2pt,itemindent=0pt]
\item One singularity, say $w = \chi_1$, is assumed to be a branch point in $\mathbb{C}_w$, with a zero $\chi_1(z_1)=0$. Then, according to the correspondence elucidated in \S\ref{sec:BorelIntro}, this singularity generates divergence in the base series, with $y^{(0)}_n = y_n$ diverging as $n \to \infty$. The result is a standard $\text{B}>1$ Stokes line at $l_{\chi_1}$, where $\chi_1$ is real and positive. 

\item Note that there may also exist a point, $z = z_2$, where $\chi_2(z_2) = 0$. While this location is also associated with a singularity in the analytic continuation of the Borel transform of the transseries \eqref{eq:MainTransseries}, it is not necessarily present in the base series $y^{(0)}(z,\ep)$.

\item Consider a \textit{Stokes crossing point} in $\mathbb{C}_z$, $z=z_{*}$, where $\chi_1 = \chi_2$. In the Borel plane this is where the two singularities in $\mathbb{C}_w$ coalesce and therefore the Borel power series $y_B^{(1)}$, defined by the Borel transform of the transseries component around $\e^{-\chi_1/\epsilon}$, is singular at $z=z_2$. In the perturbative language it means that the higher transseries components, $y^{(1)}_n(z)$, are singular at $z=z_2$. This yields a $1>2$ Stokes line at $l_{\chi_2}$, where $\chi_2$ is real and positive.

\item However, since crucially our Borel plane is ramified, $w = \chi_2$ may not always be on the same Riemann sheet as the integration contour $(0,\infty)$ emanating from the distinguished origin (relative to $y_n^{(0)}$); then even if the Stokes line $l_{\chi_2}$ is crossed in $\mathbb{C}_z$, the inversion contour of the Borel transform may not produce a Hankel contribution. The critical locus where $\chi_2$ is visible from rays emanating from the distinguished origin, $w = 0$, is the condition that the two singularities are co-linear on a half line:
\begin{equation}\label{eq:higherorderintroline}
  \text{Im}\bigg[ \frac{\chi_2-\chi_1}{\chi_1}\bigg] = 0  \qquad \text{and} \qquad  \text{Re}\bigg[ \frac{\chi_2-\chi_1}{\chi_1}\bigg] \geq 0.
\end{equation}
\end{enumerate}
The above is a geometrical explanation of HOSP from the perspective of the Borel plane, which produced condition \eqref{eq:higherorderintroline} for a higher-order Stokes line. Thus, it is only in the correct region of the $\mathbb{C}_z$ plane, on one or the other side of \eqref{eq:higherorderintroline}, that the Stokes line $l_{\chi_2}$ is active (which side depends on the problem setup). The Stokes line and higher-order Stokes line is illustrated in figure \ref{fig:BorelIntro}. Our explanation, here, is similar and complementary to those given in \cite{howls_2004} and \cite{chapman_2005}. We shall provide a concrete example of the above geometry in section \ref{sec:modelprob}.

\begin{figure}
    \centering
    \includegraphics[scale=1]{figures/BorelIntro.pdf}
    \caption{In order for the Borel singularity at $w=\chi_2$ to induce the Stokes phenomenon by crossing the integration contour $(0,\infty)$ of the inverse Borel transform, it must be present on the principle Riemann sheet when viewed from the origin, $O$. In ($b$), as the singularity at $w=\chi_2$ moves towards the real axis, it is on the principal sheet and results in a $B>2$ Stokes phenomenon. However, in ($c$), the singularity at $w=\chi_2$ crosses the real-axis on a higher Riemann sheet and thus the $B>2$ Stokes line is truncated. The geometrical colinearity condition \eqref{eq:higherorderintroline} divides these two illustrative scenarios.}
    \label{fig:BorelIntro}
\end{figure}

  One of the basic principles of resurgence is that the singularity structure of $y_B$ is encoded in the asymptotics (in $n$) of the late-terms of the power series coefficients, i.e. $y_n(z)$ as $n \to \infty$. This follows by Darboux's lemma and extensions thereof. Throughout the rest of the work we shall study the HOSP from the perspective of the large-$n$ asymptotics of the late terms, $y_n$. Then, the visibility of $\chi_2$ from the origin may be realised as a Stokes phenomena (in $1/n$) in the expansion of the late terms near $w=\chi_1$, which occurs on the colinear locus \eqref{eq:higherorderintroline}. In other words, the geometric condition \eqref{eq:higherorderintroline} is equivalent to which divergent components are present in the late-term asymptotics of $y_n$.

In fact, from the late terms perspective, we typically only learn about the presence of an additional singularity, $w = \chi_2(z)$, by noting that the terms $y_n^{(1)}$ contain an extra physical singularity at some $z=z_2$ (that is not necessarily a zero of $\chi_1$). In this sense, it is appropriate to refer to the truncated Stokes line associated to $\chi_2$ as a \textit{new Stokes line}. We now turn to this late terms perspective in more detail.

\subsection{Late-late-term divergence and optimal truncation} \label{sec:latelateintro}
The HOSP may also be detected purely through the study of the late-terms of the asymptotic expansion. The presence of an additional singularity in the Borel plane indicates that the component of the transseries around $\e^{-\chi_1/\epsilon}$,
\begin{equation}
    \e^{-\chi_1(z)/\epsilon}\left(y^{(1)}_0 + \epsilon y^{(1)}_1 + \cdots \right),
\end{equation}
forms a divergent asymptotic expansion. By the correspondence discussed in section \ref{sec:transseriesrelations}, we have that the $1/n$ late-term representation,
\begin{equation}
    y_{n}(z) \sim \sum_{p=0}^{\infty} y^{(1)}_{p}(z) \frac{\Gamma(n-p+\alpha)}{\chi_1(z)^{n-p+\alpha}} \qquad \text{as $n \to \infty$},
\end{equation}
is also divergent as $p \to \infty$. We express this divergence of $y_p^{(1)}$ in the factorial-over-power form of
\begin{equation}
    y_{p}^{(1)}(z) \sim y^{(2)}_0(z) \frac{\Gamma(p + \beta)}{\tilde{\chi}(z)^{p+\beta}} \qquad \text{as $p \to \infty$}.
\end{equation}
Here, $\beta$ is a constant, and $\tilde{\chi}$ is the late-late-term singulant for which $\tilde{\chi}(z_2)=0$. Consistent with the simplicity of the Borel plane assumed in section~\ref{sec:HOSP_Borel}, we assume that $y_0^{(2)}$ contains no further singular point. 

Due to the divergence of $y_p^{(1)}$ as $p \to \infty$ appearing in the infinite series \eqref{eq:factorial123}, optimal truncation is required at $p=P-1$, where consecutive terms are of the same order. The remainder to this optimally truncated series then displays the Stokes phenomenon across a HOSL, which are contours satisfying the Dingle conditions
\begin{equation}\label{eq:intro-hosl}
 \text{Im}\left[\frac{\tilde{\chi}}{\chi_1}\right] = 0  \qquad \text{and} \qquad \text{Re}\left[\frac{\tilde{\chi}}{\chi_1}\right] \geq 0.
\end{equation}
Thus the geometrical condition \eqref{eq:higherorderintroline} is now realised precisely as a Stokes line, but in the $1/n$ asymptotic series.
This transition across the HOSL is smooth, and follows error function dependence over a boundary layer of width $O(n^{-1/2})$. We demonstrate this in section~\ref{sec:smoothinghosp} through Borel resumming the remainder, which is analogous to the method used by \cite{berry_1989} to smooth the regular Stokes phenomenon.
Furthermore, the term switched on across the higher-order Stokes line \eqref{eq:intro-hosl} is another factorial-over-power component of the late-term divergence, with a singulant given by $\chi+\tilde{\chi}$. This is the HOSP, which results in a modified late-term divergence of the base series, given by
\begin{equation}\label{eq:latetermshospmodified}
    y_{n}(z) \sim \tilde{\sigma}(z,\ep) y_0^{(2)}(z)\frac{\Gamma(n+\alpha+\beta)}{(\chi_1+\tilde{\chi})^{n+\alpha+\beta}}+\sum_{p=0}^{P-1} y^{(1)}_{p}(z) \frac{\Gamma(n-p+\alpha)}{\chi_1(z)^{n-p+\alpha}}.
\end{equation}
In the above, $\tilde{\sigma}$ is the higher-order Stokes multiplier, which rapidly transitions across the HOSL \eqref{eq:intro-hosl}. Thus, in addition to the Stokes phenomenon induced by the $\chi_1$ divergence, there will be an additional Stokes phenomenon induced by $\chi_1+\tilde{\chi}$, which only occurs where $\tilde{\sigma} \neq 0$.

This yields the $l_{\chi_1}$ and $l_{\chi_2}$ Stokes lines previously specified in section \ref{sec:HOSP_Borel} with $\chi_2=\chi_1+\tilde{\chi}$. Following \cite{berk1982new}, contours that satisfy $\text{Im}[\chi_2]=0$, $\text{Re}[\chi_2] \geq 0$, and $\tilde{\sigma} \neq 0$ are often referred to as {\it new Stokes lines}. Thus, we find the following transseries components with singulants of $\chi_1$ and $\chi_1+\tilde{\chi}$,
\begin{equation}
\begin{split}\label{eq:transseriesintronewlines}
    y(z,\epsilon) &\sim y_0(z) + \epsilon y_1(z) + \epsilon^2 y_2(z) + \cdots +\sigma_1(z,\ep) y^{(1)}_0(z)\e^{-\chi_1/\epsilon} \\
    &\quad + \sigma(z,\ep)y^{(2)}_0(z)\e^{-(\chi_1+\tilde{\chi})/\epsilon}.
    \end{split}
\end{equation}
The additional exponential present in \eqref{eq:transseriesintronewlines}, generated by the new Stokes line, is often necessary to avoid contradictions arising from the intersection of regular Stokes lines.

\subsection{Summary}
The HOSP may be viewed from the two perspectives outlined in this section. These are:
\begin{enumerate}[label=(\roman*),leftmargin=*, align = left, labelsep=\parindent, topsep=3pt, itemsep=2pt,itemindent=0pt ]
\item {\it The late-late-term divergence.} The leading-order component of the late-terms exhibits factorial-over-power divergence of $O(\Gamma(n+\alpha)/\chi^{n+\alpha})$. Lower order components of this divergence, of $O(\Gamma(n+\alpha-p)/\chi^{n+\alpha-p})$, were considered in section~\ref{sec:latelateintro}. The amplitude functions at each of these orders diverge in the limit of $p \to \infty$; this is referred to as late-late-term divergence. Borel resummation of the optimally truncated late-late-term expansion, performed in section~\ref{sec:smoothinghosp}, reveals the HOSP.
\item {\it The Borel transform.} In the Borel plane, $\mathbb{C}_w$, HOSP occurs when there are at least two singularities, one of which is a branch point with a non-trivial Riemann sheet structure. The HOSL is then a condition for a particular singularity to be `visible' from the origin and therefore cause Stokes phenomena. This leads to a truncated `new Stokes line'.
\end{enumerate}



\section{A model problem}\label{sec:modelprob}
We now introduce a new model example whose Stokes line structure contains the simplest possible realisation of the HOSP, due to the presence of one branch point and a simple pole in the Borel plane.
Let us consider the following singularly perturbed second order linear inhomogeneous differential equation:
\begin{equation}\label{eq:themainone}
    \eps^2 y''+ \eps (1+z)y' + z y = 1,
\end{equation}
where $y(z,\ep)$ is the unknown function and $0<\ep \ll 1$ is a small parameter. 

\subsection{The transseries solution}
First, we seek the formal transseries expansion by writing
\begin{equation}\label{eq:transseriesansatz}
    y(z,\epsilon) = \sum_{i=0}\e^{-\chi_i(z)/\epsilon}y^{(i)}(z,\epsilon),
\end{equation}
where $\chi_i(z)$ is the singulant function, and $y^{(i)}(z,\ep)$ is the solution corresponding to the singulant $\chi_i$.
Substitution of transseries ansatz \eqref{eq:transseriesansatz} into the differential equation \eqref{eq:themainone} yields the three possible solutions for the singulant, which are given by
\begin{equation}\label{eq:chisolutions}
 \chi_0(z)=0, \qquad   \chi_1^{\prime}(z) = z, \qquad \chi^{\prime}_2(z) = 1.
\end{equation}
The first of these, $\chi_0$, corresponds to the base solution, whose asymptotic expansion will be a standard power series in $\ep$. Furthermore, equations for each of the amplitude solutions, $y^{(i)}(z,\ep)$, are obtained as
\begin{subequations}\label{eq:amplitudeequations}
\begin{align}
\label{eq:amplitudeeq0}
    \ep^2 y^{(0) \prime \prime} + \ep (1+z) y^{(0)\prime} + z y^{(0)} =1,\\
\label{eq:amplitudeeq1}
    \ep y^{(1) \prime \prime} + (1-z) y^{(1)\prime} -y^{(1)}=0,\\
\label{eq:amplitudeeq2}
    \ep y^{(2) \prime \prime} +(z-1) y^{(2)\prime}  =0.
\end{align}
\end{subequations}

We begin by expanding the base solution, $y^{(0)}(z,\ep)$, in powers of $\ep$ as
\begin{equation}\label{eq:baseasymptexp}
   y^{(0)}(z,\ep) = \sum_{n=0}^{\infty} \ep^{n} y_{n}(z),
\end{equation}
for which substitution into the base equation \eqref{eq:amplitudeeq0} yields the early orders of the solution expansion as
\begin{equation} \label{eq:y0earlyorders}
   y^{(0)}(z,\ep) = \frac{1}{z} + \ep\frac{(1+z)}{z^3} + \ep^2 \frac{(2z^2+3z+3)}{z^5}+\cdots.
\end{equation}
Note that there is a singularity at $z=0$, near to which the asymptotic series \eqref{eq:y0earlyorders} reorders. The resultant inner analysis in this region is performed in Appendix \ref{sec:AppInner}. 
At $O(\ep^n)$ in the base equation \eqref{eq:amplitudeeq0}, we find the equation
\begin{equation} \label{eq:y0lateorders}
y_{n-2}^{\prime \prime} +(1+z) y_{n-1}^{\prime} + z y_{n} =0,
\end{equation}
which holds for $n \geq 2$.
As successive orders of the asymptotic expansion are determined by the differentiation of previous orders, the solution $y_n$ will diverge as $n \to \infty$. This divergence is studied in the next section.

\subsection{Late-term divergence}\label{sec:lateanalysisexample}
To capture the factorial-over-power divergence of $y_n$, we consider a late-term ansatz of the form
\begin{equation} \label{eq:fullfactorialpower}
y_n^{(0)} \sim \sum_{p=0}^{\infty} B_p(z)\frac{\Gamma(n-p+\alpha)}{[\chi(z)]^{n-p+\alpha}} \qquad \text{as $n \to \infty$},
\end{equation}
where $\chi$ and $B_p$ are the late-term singulant and amplitude functions respectively, and $\alpha$ is a constant.
In \eqref{eq:fullfactorialpower}, in addition to the leading-order divergence specified by $p=0$, we have also included lower-order components. In fact, there will be additional components of the late-term divergence not captured by the ansatz \eqref{eq:fullfactorialpower}.
These will occur due to the HOSP, in which they are smoothly switched on across a boundary-layer, of diminishing width as $n \to \infty$, that surrounds a HOSL. The key result of this paper is that this phenomenon may be derived through the study of the late-late-terms, $B_p$, as $p \to \infty$, which is performed in section~\ref{sec:smoothinghosp}.

Substitution of the factorial-over-power ansatz \eqref{eq:fullfactorialpower} into the $O(\ep^n)$ equation \eqref{eq:y0lateorders} yields at each order of $\Gamma(n-p+\alpha)/\chi^{n-p+\alpha}$ the equations
\begin{subequations}\label{eq:amplitudeequationsfactorial}
\begin{align}
\label{eq:amplitudeeq0factorial}
p=0: & \qquad (\chi^{\prime}-1)(\chi^{\prime}-z)=0,\\
\label{eq:amplitudeeq1factorial}
p=1: & \qquad   (1+z-2\chi^{\prime})B_0^{\prime}-\chi^{\prime \prime}B_0=0,\\
\label{eq:amplitudeeq2factorial}
p \geq 2: & \qquad     B_{p-2}^{\prime \prime} + (1+z-2 \chi^{\prime})B_{p-1}^{\prime}-\chi^{\prime \prime}B_{p-1}=0.
\end{align}
\end{subequations}

Integration from the singularity at $z=0$, in order to satisfy the matching criteria of $\chi(0)=0$, yields the two solutions
\begin{equation}\label{eq:basesingulants}
\chi_1(z) =z^2/2 \qquad \text{and} \qquad \chi_2(z)=z.
\end{equation}

Since the latter of these two singulants yields $B_p(z)=\text{const}$ for $p\geq 0$, it will not generate any late-term Stokes switching. We therefore focus on the first late-term singulant, $\chi=z^2/2$. The first of the amplitude equations, \eqref{eq:amplitudeeq1factorial}, may then be solved to find
\begin{equation} \label{eq:latetermB0}
B_0(z)=\frac{\Lambda_0}{1-z},
\end{equation}
where $\Lambda_0$ is a constant of integration. Note that solution \eqref{eq:latetermB0} is singular at $z=1$, and that equation \eqref{eq:amplitudeeq1factorial} for $B_1$ requires differentiation of $B_0$. Hence, $B_1$ will have stronger singular behaviour at $z=1$. This pattern continues and leads to factorial-over-power divergence of $B_p$ as $p \to \infty$, which we denote the late-late-terms of the asymptotic expansion. This is considered next in section~\ref{sec:latelatediv}.

\subsection{Late-late-term divergence}\label{sec:latelatediv}
In this section, we determine the divergence of $B_p$ as $p \to \infty$, which arises as a consequence of the singularity at $z=1$ in solution \eqref{eq:latetermB0} for $B_0$. The power of this singularity will grow as we proceed into the asymptotic series. For instance, equation \eqref{eq:amplitudeeq1factorial} may be solved for $B_1$, yielding
\begin{equation} \label{eq:latetermB1}
B_1(z)=-\frac{\Lambda_0}{(1-z)^3}+\frac{\Lambda_1}{1-z},
\end{equation}
and this growing singular behaviour generates factorial-over-power divergence of $B_p$ as $p \to \infty$.
We begin by considering the factorial-over-power ansatz
\begin{equation} \label{eq:latelatetermBp}
B_p(z) \sim C(z)\frac{\Gamma(p+\beta)}{[\tilde{\chi}(z)]^{p+\beta}} \qquad \text{as $p \to \infty$},
\end{equation}
where $\tilde{\chi}$ and $C$ are the late-late-term singulant and amplitude functions, and $\beta$ is a constant.
In order to match with an inner solution for $y^{(0)}_n$ near $z=1$, we require that $\tilde{\chi}(1)=0$. Substitution of ansatz
\eqref{eq:latelatetermBp} into equation \eqref{eq:amplitudeeq2factorial} yields the equations $\tilde{\chi}^{\prime}=1-z$ and $C^{\prime}=0$, which we integrate to find
\begin{equation}\label{eq:latelatesolutions}
\tilde{\chi}=-\frac{(z-1)^2}{2} \qquad \text{and} \qquad C(z)=\tilde{\Lambda}.
\end{equation}
In \eqref{eq:latelatesolutions} above, we have used the matching condition $\tilde{\chi}(1)=0$, and $\tilde{\Lambda}$ is a constant, determined by an inner matching procedure as $\tilde{\Lambda}=\i/(2\pi)$ in \eqref{eq:latelateconstantapp} of Appendix~\ref{sec:AppInner}.

\subsection{Smoothing the higher-order Stokes phenomenon}\label{sec:smoothinghosp}
In this section, we demonstrate that optimal truncation of the late-term ansatz \eqref{eq:fullfactorialpower} at $p=P-1$ yields an exponentially-small (in $n$) remainder that displays the Stokes phenomenon. This remainder corresponds to an additional factorial-over-power component of the late-terms with a singulant given by $\chi + \tilde{\chi}$. Furthermore, we show how this change occurs smoothly across a HOSL, which is located where $\text{Im}[\tilde{\chi}/\chi]=0$ and $\text{Re}[\tilde{\chi}/\chi] \geq 0$. This analysis closely follows the method used by \cite{berry_1989}, in which the tail of the divergent expansion is re-summed, and examination of the resultant integral reveals error function behaviour localised across the Stokes line.

We begin by truncating the late-term ansatz \eqref{eq:fullfactorialpower} at $p=P-1$ by writing
\begin{equation} \label{eq:fullfactorialpowertruncated}
y_n^{(0)} \sim \sum_{p=0}^{P-1} B_p(z)\frac{\Gamma(n-p+\alpha)}{[\chi(z)]^{n-p+\alpha}} + \underbrace{\sum_{p=P}^{\infty} B_p(z)\frac{\Gamma(n-p+\alpha)}{[\chi(z)]^{n-p+\alpha}}}_{R_P},
\end{equation}
and consider $P$ to be large. Note that later in equation \eqref{eq:optimalPvalue}, we specify $P$ in terms of $n$ by minimising $R_p$. In substituting for both the late-late-term divergent ansatz of $B_p$ from \eqref{eq:latelatetermBp} and the integral definition of the Gamma function (since $n-P$ is also large), we also swap the order of summation and integration. This yields a geometric series, which may be evaluated exactly to find
\begin{equation}
\begin{split}
R_p&=\frac{C}{\tilde{\chi}^{\beta}\chi^{n+\alpha}} \int_0^{\infty} \int_0^{\infty}w^{\beta-1}t^{n+\alpha-1}\e^{-w-t} \sum_{p=P}^{\infty}\bigg(\frac{\chi w}{\tilde{\chi} t}\bigg)^P\de{w} \de{t},\\
&=\frac{C}{\tilde{\chi}^{\beta}\chi^{n+\alpha}} \frac{\chi^P}{\tilde{\chi}^P}\dashint_0^{\infty} \dashint_0^{\infty}w^{P+\beta-1}t^{n-P+\alpha-1}\e^{-w-t} \frac{1}{1-\frac{\chi w}{\tilde{\chi} t}}\de{w} \de{t}.
 \end{split}
\end{equation}
In the above, we have used the principle value integral due to the simple pole at $\chi w=\tilde{\chi}t$ introduced by evaluating the infinite geometric series. 
Next, we change integration variables from $(w,t)$ to $(s,f)$ where $s=\chi w /(\tilde{\chi}t)$ and $f=t+w$. This substitution allows for the independent evaluation of the two integrals, one of which is recognised as the definition of the Gamma function. Performing this substitution, and simplifying, yields
\begin{equation}
\begin{split}
R_p&=\frac{C}{\chi^{n+\alpha+\beta}} \dashint_0^{\infty} \frac{s^{P+\beta-1}}{\big(1+\frac{s \tilde{\chi}}{\chi}\big)^{n+\alpha+\beta}(1-s)} \de{s} \int_0^{\infty}f^{n+\alpha+\beta-1} \e^{-f} \de{f}, \\
&=C\frac{\Gamma(n+\alpha+\beta)}{\chi^{n+\alpha+\beta}} \dashint_0^{\infty} \frac{s^{P+\beta-1}}{\big(1+\frac{s \tilde{\chi}}{\chi}\big)^{n+\alpha+\beta}(1-s)} \de{s}.
 \end{split}
\end{equation}
Next we substitute for $x=s-1$, which shifts the location of the pole from $s=1$ to $x=0$. As the dominant component of the integrand at $x=0$ lies inside the new range of integration, $-1 \leq x<\infty$, we may also extend the lower range to $-\infty$ without changing the dominant asymptotic behaviour. This gives
\begin{equation}\label{eq:smoothingintegral5}
\begin{split}
R_p&=-C\frac{\Gamma(n+\alpha+\beta)}{\chi^{n+\alpha+\beta}} \dashint_{-1}^{\infty} \frac{{(1+x)}^{P+\beta-1}}{x\big[1+(1+x){\tilde{\chi}}/{\chi}\big]^{n+\alpha+\beta}} \de{x},\\
&\sim -C\frac{\Gamma(n+\alpha+\beta)}{(\chi+\tilde{\chi})^{n+\alpha+\beta}} \dashint_{-\infty}^{\infty} \frac{1}{x}\e^{(P+\beta-1)\log{(1+x)}-(n+\alpha+\beta)\log{[1+x\tilde{\chi}/ (\chi+\tilde{\chi})]}} \de{x}.
 \end{split}
\end{equation}

We now seek to approximate the integral in \eqref{eq:smoothingintegral5} above, both near the HOSL and close to the pole at $x=0$. In writing
\begin{equation}
 \chi = r \e^{\i \vartheta} \qquad \text{and} \qquad \tilde{\chi} = \tilde{r} \e^{\i \tilde{\vartheta}},
\end{equation}
the HOSL, for which $\text{Im}[\tilde{\chi}/\chi]=0$ is located at $\tilde{\vartheta}- \vartheta=0$. Thus, in the vicinity of this, and the pole at $x=0$, both $x$ and $\tilde{\vartheta}-\vartheta$ are small. We may therefore Taylor expand these components in \eqref{eq:smoothingintegral5} to find 
\begin{equation}\label{eq:smoothingintegralexpansions}
\begin{split}
\e^{(P+\beta-1)\log{(1+x)}} &\sim \e^{(P+\beta-1)\big(x-\tfrac{x^2}{2}+ O(x^3)\big)},\\
\e^{-(n+\alpha+\beta)\log{\big[1+\tfrac{x\tilde{\chi}}{(\chi+\tilde{\chi})}\big]}} &\sim \e^{-(n+\alpha+\beta)\big[\tfrac{\tilde{r}x}{r+\tilde{r}} -\tfrac{\tilde{r}^2x^2}{2(r+\tilde{r})^2}+\i\tfrac{r \tilde{r}x(\tilde{\vartheta}-\vartheta)}{(r+\tilde{r})^2}+O(x (\tilde{\vartheta}-\vartheta)^2;x^3)\big]} .
 \end{split}
\end{equation}
The remainder, $R_P$, is minimised when the largest components of \eqref{eq:smoothingintegralexpansions}, $\exp{(Px)}$ and $\exp{(- \frac{\tilde{r}x n}{r+\tilde{r}})}$, cancel upon multiplication. This motivates the specification of our optimal truncation point by
\begin{equation}\label{eq:optimalPvalue}
P=\frac{\tilde{r}n}{r+\tilde{r}}+\tilde{\rho},
\end{equation}
where $0 \leq \tilde{\rho} <1$ ensures that $P$ takes integer values. We may now verify {\it a priori} that our prior assumption of $n-P$ being large is self consistent, since $n-P  \sim r n / (r+\tilde{r}) \gg 0$. Substitution of the optimal value of $P$ from \eqref{eq:optimalPvalue} into the integral \eqref{eq:smoothingintegral5}, and using expansions \eqref{eq:smoothingintegralexpansions}, yields real and imaginary components in the integrand. Analogous to the methodology used by \cite{berry_1989}, it may be verified through direct evaluation that the real component of this integral is subdominant to the imaginary component. The leading order imaginary component then simplifies to give
\begin{equation}\label{eq:smoothingintegral7}
\begin{split}
R_P&\sim \i C\frac{\Gamma(n+\alpha+\beta)}{(\chi+\tilde{\chi})^{n+\alpha+\beta}} \dashint_{-\infty}^{\infty} \frac{1}{x}\e^{-\tfrac{r \tilde{r} n x^2}{2(r+\tilde{r})^2}} \sin{\bigg(\frac{r \tilde{r} (\tilde{\vartheta}-\vartheta)n}{(r+\tilde{r})^2}x\bigg)}\de{x},\\
&\sim \sqrt{2 \pi} \i C\frac{\Gamma(n+\alpha+\beta)}{(\chi+\tilde{\chi})^{n+\alpha+\beta}} \int_0^{\tfrac{(r \tilde{r})^{1/2}(\tilde{\vartheta}-\vartheta)n^{1/2}}{(r+\tilde{r})}} \e^{-t^2/2}\de{t}.
 \end{split}
\end{equation}

Equation \eqref{eq:smoothingintegral5} is the main result of this section: an explicit solution for the smooth transition of $R_P$ across a boundary layer of width $O(n^{-1/2})$ surrounding the HOSL. Thus, we have that
\begin{equation}\label{eq:smoothingintegral6}
R_P \sim 2 \pi \i C(z)\frac{\Gamma(n+\alpha+\beta)}{(\chi+\tilde{\chi})^{n+\alpha+\beta}}
\end{equation}
switches on in the late-terms of the asymptotic solution across the HOSL, which is given by
\begin{equation}\label{eq:mainsmoothing-hosl}
 \text{Im}\left[\frac{\tilde{\chi}}{\chi_1}\right] = 0  \qquad \text{and} \qquad \text{Re}\left[\frac{\tilde{\chi}}{\chi_1}\right] \geq 0.
\end{equation}

We have focused in the above on a concrete calculation that applies to the simplest form of HOSP occurring in practice in singularly perturbed differential equations. In fact, it is possible to recast the above calculation in the hyperterminant language of \cite{daalhuis1996hyperterminants,daalhuis1998hyperterminants} where the integral \eqref{eq:smoothingintegral7} may be understood as a hyperterminant. We refer the reader to the recent conference talk by \cite{smoothingAdri} for a treatment of the higher-order Stokes smoothing in this hyperterminant setting.



\subsection{Stokes and higher-order Stokes line structure} \label{sec:stokesnot}
The late-terms of the base expansion contain three distinct divergent components. Firstly, there are the two factorial-over-power terms with singulants $\chi_1=z^2/2$ and $\chi_2=z$ from \eqref{eq:basesingulants}. Secondly, there is the component from \eqref{eq:smoothingintegral6} switched on across the HOSL, with $\chi_3=\chi_1+\tilde{\chi}=z-1/2$. The Stokes phenomenon associated with these three components produce Stokes lines, which we customarily denote as $B>1$, $B>2$, and $B>3$,  respectively, in order to express the exponential dominance of one component over another as switching occurs. Furthermore, examination of the $O(\e^{-\chi_1/\ep})$ transseries equation \eqref{eq:amplitudeeq1} for $y^{(1)}(z,\ep)$ reveals a divergent series whose solutions at each order of $\ep$ are the same as the $B_p$ functions found in the late-late-term divergence of the base expansion from section~\ref{sec:latelatediv}. This relationship was discussed in section~\ref{sec:transseriesrelations}, and this switching results in a term of $O(\e^{-(\chi_1+\tilde{\chi})/\ep})$. Thus, there will also be a $1>3$ Stokes line generated by the divergence of $y^{(1)}$.

Combined, the Stokes lines for this problem are given by
\begin{subequations}\label{eq:allthestokeslines}
\begin{align}
(B>1)&:\qquad \text{Im}[\chi_1]=0, \qquad \text{Re}[\chi_1] \geq 0,\\
(B>3)&: \qquad  \text{Im}[\chi_3]=0, \qquad \text{Re}[\chi_3] \geq 0,\\
(1>3)&: \qquad \text{Im}[\tilde{\chi}]=0, \qquad ~ \text{Re}[\tilde{\chi}] \geq 0,
\end{align}
\end{subequations}
where the $(B>3)$ Stokes line is active only when the higher-order Stokes multiplier is non-zero. Note that we have not included the $B>2$ Stokes line here, as it can be shown through a more detailed matching procedure (to all orders in $\ep$) than Appendix~\ref{sec:AppInnerz0} that the constant prefactor of the $\chi_2=z$ late-term component must equal zero. 
These are shown in figure \ref{fig:OurEquation}. For this example, the new $B>3$ Stokes line generated by the HOSP is necessary to avoid a contradiction occurring from rotating about the simple pole at $z=1$.
\begin{figure}
    \centering
    \includegraphics[scale=1]{figures/OurEquation.pdf}
    \caption{The Stokes line structure for our model problem \eqref{eq:themainone} is shown in $z \in \mathbb{C}$. HOSL are shown with thin lines, and regular Stokes lines with Bold lines.}
    \label{fig:OurEquation}
\end{figure}

\subsection{Borel-Pad\'e and the higher-order Stokes phenomenon}

The above calculation may also be interpreted from the Borel plane perspective. As discussed in section \ref{sec:BorelIntro}, singularities in the Borel plane, say at $w = \chi(z)$, correspond to the singulant functions that characterise the divergence of late terms in the transseries. Moreover, the local behaviour of $y_B$ near such singularities is determined from the constant, $\alpha$, that appears in divergent representation \eqref{eq:MainDivergence}. For instance, from the late-term analysis in \S\ref{sec:lateanalysisexample}, the divergence associated with $\chi_1$ has $\alpha=1/2$, and that associated with $\chi_3$ in \eqref{eq:smoothingintegral6} has $\alpha+\beta=1$. These constants were determined in Appendix \ref{sec:AppInner}. Thus, the parametric Borel transform has a branch point at $w=\chi_1(z)$ and a pole at $w=\chi_3(z)$. The HOSL is then the condition for the pole at $\chi_3(z)$ to be hidden (with respect to rays from the distinguished origin) behind the branch cut from $\chi_1(z)$. This geometric condition \eqref{eq:higherorderintroline} was realised precisely as a Stokes line in the late-terms from \eqref{eq:mainsmoothing-hosl}.

 The singularity structure of $y_B$ can also be explored via numerical analytic continuation techniques. The Pad\'e approximant is a powerful method to analytically continue truncated power series, and we refer the reader to the work by \cite{costin2022uniformization} (and references therein) for the state of the art of using Pad\'e approximants as analytic continuation tool. Here, we shall demonstrate that Pad\'e approximants can be used as a useful heuristic to understand the parametric Borel plane singularity structure for a given parametric series.

In our present example, we compute $2N$ perturbative terms and consider the following truncated power series, for which $z$ is a parameter,
\begin{equation}
    y^{2N}_B(w;z) = y_0(z) + y_1(z)w + \cdots + \frac{y_{2N}(z)}{(2N)!}w^{2N}.
\end{equation}
We may then compute the off-diagonal $[N-1:N]$ Pad\'e approximant for the fixed values of $z \in \mathbb{C}_z$ of interest. That is, we write
\begin{equation}
    P_{[N-1:N]}(w;z) = \frac{P_{N-1}(w;z)}{Q_{N}(w;z)},
\end{equation}
where $P_{N-1}(w;z)$ and $Q_{N}(w;z)$ are polynomials, whose coefficients depend on $z$, chosen to agree with the truncated power series at a given order. The interesting region to explore is near the HOSL where one may see poles disappearing behind branch cuts, realising the HOSP.
\begin{figure}
    \centering
    \includegraphics[scale=1]{figures/BorelExp.pdf}
    \caption{The Borel plane, $\mathbb{C}_w$, is shown in ($a$) for $z=1/2+\i$, and in ($b$) for $z=1/2$. The singularities (nodes) correspond to $w = \chi_1(z) = z^2/2$ and $w = \chi_3(z) = z - 1/2$, and are computed from a Pad\'{e} approximant with $N = 250$. Note that the situation $(a)$ corresponds to the region `above' the HOSL in figure~\ref{fig:OurEquation} while $(b)$ corresponds to `below' the HOSL.}
    \label{fig:BorelExp}
\end{figure}

  We demonstrate this in figure \ref{fig:BorelExp} for our example equation. In this figure we note the characteristic coalescing poles that the rational approximation assigns to branch points. We plot the Borel plane for two different values of $z$: inside the HOSL for $z=1/2 + \i$ in $(a)$, and outside for $z=1/2$ in $(b)$. One interesting fact to note is that in $(b)$ when $z=1/2$, the Borel singularity $w=\chi_3(1/2)=0$; however the series $y_B$ is convergent here as the singularity $w=0$ lies on a higher sheet above the distinguished origin. The full parametric Alien calculus structure of this example will be explored in future work \cite{crew2023}.

\section{Further examples of the higher-order Stokes phenomenon}\label{sec:otherexamples}
We now consider three previously studied differential equations whose asymptotic solutions display the HOSP. We demonstrate how this phenomenon may be resolved in these problems through the application of the methods developed in this paper, in which the late-late-term divergence of the asymptotic solution is derived.

The singularly perturbed equations considered in this section are:
\begin{enumerate}[label=(\roman*),leftmargin=*, align = left, labelsep=\parindent, topsep=3pt, itemsep=2pt,itemindent=0pt]
\item The Pearcey equation in \S\ref{sec:FirstEx}. This is a linear third-order homogeneous equation, previously studied by \cite{howls_2004} through consideration of an integral representation of the solution and a hyperterminant analysis.
Our results are summarised in \ref{sec:FirstEx} and full details appear in Appendix~\ref{sec:FirstExApp}.

\item A linear second-order inhomogeneous differential equation in \S\ref{sec:SecondEx}. This equation was proposed by \cite{trinh2013new}, who considered an exact series representation for the late-terms of the asymptotic solution.

\item A linear second-order differential equation with an eigenvalue in \S\ref{sec:AppKelvin}, which arises in geophysical fluid dynamics. This equation, and the reusltant eigenvalue divergence, was studied by \cite{shelton2022Hermite,shelton2023kelvin}.
\end{enumerate}
The first two examples contain new Stokes lines which emanate from intersecting Stokes lines. In the third example the higher-order and regular Stokes lines coincide, which changes the value of the regular Stokes multiplier.
 
\subsection{The Pearcey equation}\label{sec:FirstEx}

The Pearcey function is perhaps the most recognisable subject in connection to the HOSP; it is discussed in the work of \cite{howls_2004}, primarily with respect to the integral representation
\begin{equation} \label{eq:Pearceyintegral}
    I(z) = \int_C \e^{-f(s; z)/\ep} \, \de{s} \quad \text{with} \quad f(s; z) = -\im\left(\frac{1}{4}s^4 + \frac{1}{2} s^2 + sz\right).
\end{equation}
Here, $\ep \to 0$ and $C$ is a contour from $\infty \exp(-3\pi\im/8)$ to $\infty \exp(\pi\im/8)$. Alternatively, manipulation of \eqref{eq:Pearceyintegral} yields a linear, third-order, homogeneous differential equation:
\begin{equation}\label{eq:pearceymain}
\ep^3 I^{\prime \prime \prime}(z)-\ep I^{\prime}(z) - \i z I(z)=0,
\end{equation}
with decay conditions in the same limits as specified above. The relationship between these two systems is discussed in section 7.2 of \cite{takei2017wkb}, who considers the BNR equation from \cite{berk1982new}, which is obtained by rescaling \eqref{eq:pearceymain}. The general Pearcey integral \eqref{eq:Pearceyintegral} contains a second parameter, assumed to be fixed in the above formulation \citep{takei2017wkb}.

Aided by the integral representation \eqref{eq:Pearceyintegral}, the HOSP can be understood more easily via a steepest descent analysis and this is presented by \cite{howls_2004}. Interestingly, we could not find a direct asymptotic analysis of the HOSP -- as connected to the late terms -- of the Pearcey function in its singularly-perturbed differential equation form. This we do now using the techniques from the previous section. 

Three different WKBJ solutions are permitted to \eqref{eq:pearceymain}, each indexed by a leading exponential, $A_i(z,\ep) \e^{-S_i(z)/\ep}$ for $i = 1, 2, 3$, which yields divergent expansions of the form
\begin{equation}\label{eq:PearceyTransseirs}
\begin{aligned}
I(z) &= \e^{-S_1/\ep}\bigg[A_1^{(0)}+ \cdots + \ep^n \bigg(B_1^{(0)}\frac{\Gamma(n+\alpha)}{(S_2-S_1)^{n+\alpha}} +\tilde{\sigma}_1C_1\frac{\Gamma(n+\alpha)}{(S_3-S_1)^{n+\alpha}}\bigg) + \cdots \bigg]\\
&\phantom{=}+\e^{-S_2/\ep}\bigg[A_2^{(0)}+ \cdots + \ep^n \bigg(B_{2a}^{(0)}\frac{\Gamma(n+\alpha)}{(S_3-S_2)^{n+\alpha}} +B^{(0)}_{2b}\frac{\Gamma(n+\alpha)}{(S_1-S_2)^{n+\alpha}}\bigg) + \cdots \bigg]\\
&\phantom{=}+\e^{-{S_3}/{\ep}}\bigg[A_3^{(0)}+ \cdots + \ep^n \bigg(B_3^{(0)}\frac{\Gamma(n+\alpha)}{(S_2-S_3)^{n+\alpha}} +\tilde{\sigma}_3 C_3\frac{\Gamma(n+\alpha)}{(S_1-S_3)^{n+\alpha}}\bigg) + \cdots \bigg].
\end{aligned}
\end{equation}
The derivation of each of these components is given in Appendix~\ref{sec:FirstExApp}. Each of the six factorial-over-power components of \eqref{eq:PearceyTransseirs} induces the Stokes phenomenon on another exponential, when the relevant singulant is real and positive. However, two of these divergences are present only when the corresponding higher-order Stokes multiplier, $\tilde{\sigma}_1$ or $\tilde{\sigma}_3$, is non-zero. The higher-order Stokes lines, across which these higher-order multipliers change, are also shown in figure~\ref{fig:Pearcey}.

\begin{figure}
    \centering
    \includegraphics[scale=1]{figures/Pearcey.pdf}
    \caption{The Stokes line structure is shown in the $z$-plane for the Pearcey equation \eqref{eq:pearceymain}. Higher-order Stokes lines (HOSL), across which the late-terms switch-on a new divergent component, are shown by thin lines. Stokes lines, across which the regular Stokes phenomenon occurs on an exponentially-small component of the asymptotic expansion, are shown as bold lines. On account of the HOSP, two Stokes lines are born (or truncated) at the Stokes crossing point. The notation of $i > j$ corresponds to exponential $\e^{S_i/\ep}$ switching on $\e^{S_j/\ep}$.}
    \label{fig:Pearcey}
\end{figure}
The resultant Stokes line structure is shown in figure~\ref{fig:Pearcey}. Note that there are two singular point at $z_{-}$ and $z_{+}$, which correspond both to singularities in the leading-order asymptotic solution $A_i^{(0)}$, and zeros of the singulant functions, $S_1(z_{-})=S_2(z_-)=S_2(z_+)=S_3(z_{+})=0$. In total, four Stokes lines emanate from $z_{-}$ and $z_{+}$ (shown thick). The Stokes lines intersect at two Stokes crossing points on the real axis. These crossing points also lie along the respective higher-order Stokes lines (shown thin), across which the respective multiplier $\tilde{\sigma}$ switches on to a non-zero value. Thus, beyond the Stokes crossing point, an additional divergences are present in the late-terms of transseries \eqref{eq:PearceyTransseirs}, which leads to two new Stokes lines. These new Stokes lines are marked $1 > 3$ and $3 > 1$ in the figure.


\subsection{An inhomogeneous second-order equation}\label{sec:SecondEx}
We now study the following second-order singularly perturbed differential equation for $A(z)$:
\begin{equation}\label{eq:secondmaineq2}
\ep^2 A^{\prime \prime} + 2 \ep A^{\prime} + (1-z)A=\frac{1}{z-a}.
\end{equation}
 Equation \eqref{eq:secondmaineq2} was considered as a model problem by \cite{trinh2013new} due to the turning point at $z=1$ and the singularity at $z=a$. 
In fact, each order of the asymptotic solution to \eqref{eq:secondmaineq2} may be written exactly as a polynomial in powers of $(1-z)^{-1}$ and $(z-a)^{-1}$, for which the constant coefficients are determined as the solution to a recurrence relation. \cite{trinh2013new} conjectured that the analysis of the late-orders behaviour of the recurrence relations could yield details of the HOSP, but this was left as an open challenge.

We now apply our general techniques to detect HOSP in this problem. In expanding the solution to equation \eqref{eq:secondmaineq2} as $A(z,\ep)=A_0(z) + \ep A_1(z) + \cdots$, the first two orders of this asymptotic expansion may be determined as
\begin{equation}\label{eq:secondearlysolutions}
A_0=\frac{1}{(1-z)(z-a)} \quad \text{and} \quad A_1= \frac{2}{(1-z)^2(z-a)^2} -\frac{2}{(1-z)^3(z-a)}.
\end{equation}
At $O(\ep^n)$ in equation \eqref{eq:secondmaineq2}, we find $A_{n-2}^{\prime \prime} +2A_{n-1}^{\prime} +(1-z)A_n=0$, which holds for $n \geq 2$.
The late-terms, $A_n$, of the asymptotic expansion will diverge as $n \to \infty$ on account of the singularities at $z=1$ and $z=a$ in \eqref{eq:secondearlysolutions}. 

\subsubsection*{Late-term divergence}
In order to study the divergence of $A_n$, we consider a factorial-over-power ansatz of the form
\begin{equation}\label{eq:secondexpansion1}
A_n(z) \sim \sum_{p=0}^{\infty} B_p(z) \frac{\Gamma(n-p+\alpha)}{[\chi(z)]^{n-p+\alpha}} \qquad \text{as $n \to \infty$},
\end{equation}
where $\alpha$ is a constant, and $\chi$ and $B_p$ are the singulant and amplitude functions, respectively. Substitution of ansatz \eqref{eq:secondexpansion1} into the $O(\ep^n)$ equation yields at each order of $\Gamma(n-p+\alpha)/\chi^{n-p+\alpha}$ the equations
\begin{subequations}\label{eq:secondeachorderofn}
\begin{align}
\label{eq:secondlateO1}
(\chi^{\prime})^2-2\chi^{\prime}+(1-z)&=0,\\
\label{eq:secondlateOn}
2(1-\chi^{\prime})B_0^{\prime}- \chi^{\prime \prime}B_0&=0,\\
\label{eq:secondlateOnp}
B_{p-2}^{\prime \prime}+2(1-\chi^{\prime})B_{p-1}^{\prime}-\chi^{\prime \prime}B_{p-1}&=0.
\end{align}
\end{subequations}
Here, \eqref{eq:secondlateOnp} holds for $p \geq 2$.
The quadratic \eqref{eq:secondlateO1} for $\chi^{\prime}$, and \eqref{eq:secondlateOn} for $B_0$ may be solved to find
\begin{equation}\label{eq:secondsingulantsol}
\chi_{\pm}^{\prime}(z) = 1 \pm z^{1/2} \qquad \text{and} \qquad B_0(z) = \frac{\Lambda_0}{(1-\chi_{\pm}^{\prime})^{1/2}},
\end{equation}
where $\pm$ denotes each of the two solutions for $\chi^{\prime}$, and $\Lambda_0$ is a constant of integration.  From the singulant solution in \eqref{eq:secondsingulantsol} above we have $1-\chi^{\prime}_{\pm}=\mp z^{1/2}$, and thus the solution for $B_0$ is singular at $z=0$. This new singularity at $z=0$ will result in the divergence of $B_p$ as $p \to \infty$, which is now studied.

\subsubsection*{Late-late-term divergence}
To capture the divergence of $B_p$ as $p \to \infty$, we consider a typical factorial-over-power ansatz [cf. \eqref{eq:latelatetermBp}]. Substitution of this into equation \eqref{eq:secondlateOnp} yields $\tilde{\chi}^{\prime}_{\pm}(z) = 2(1-\chi_{\pm}^{\prime})$, and hence $\tilde{\chi}_{\pm}(z) = \mp 4 z^{3/2}/3$ upon imposing the condition $\tilde{\chi}_{\pm}(0)=0$. Combined with the solution for the prefactor equation, we find a late-late-term divergence of the form
\begin{equation}\label{eq:secondlatelatediverg}
B_p(z) \sim  \frac{\tilde{\Lambda}}{(\chi^{\prime}_{\pm}-1)^{1/2}} \frac{\Gamma(p+\beta)}{[\mp 4 z^{3/2}/3]^{p+\beta}} \qquad \text{as $p \to \infty$},
\end{equation}
where $\tilde{\Lambda}$ is a constant of integration. This will induce the HOSP in which a new late-term component with singulant $\chi_{\pm} \mp 4 z^{3/2}/3 $ switches on when $\mp 4 z^{3/2}/(3 \chi_{\pm})$ is real and positive.

\subsubsection*{Stokes line structure}
The naive Stokes lines may be determined by integrating the solution $\chi_{\pm}^{\prime}(z) = 1 \pm z^{1/2}$ from \eqref{eq:secondsingulantsol}. As there are two singular points, $z=a$ and $z=1$, that generate divergence, we will have four initial singulants. Only the two generated by the boundary condition $\chi(a)=0$ are now considered to rectify the issue of crossing Stokes lines found by \cite{trinh2013new}. This yields
\begin{equation}\label{eq:secondchisolutions2}
\chi_1(z) =z+\frac{2}{3}z^{3/2}-a-\frac{2}{3}a^{3/2} \quad \text{and} \quad \chi_2(z)=z-\frac{2}{3}z^{3/2}-a+\frac{2}{3}a^{3/2}.
\end{equation}

Furthermore, the expansion in $n$ for the amplitude functions of the above four singulants diverges in the manner specified by \eqref{eq:secondlatelatediverg}, on account of a singularity at $z=0$. This new divergent series induces the HOSP in which a new factorial-over-power contribution to $A_n$, with a singulant given by $\chi+\tilde{\chi}$, switches on across the HOSL $\text{Im}[\tilde{\chi}/\chi]=0$ and $\text{Re}[\tilde{\chi}/\chi]>0$. The two new singulants switched on in $A_n$ are therefore
\begin{equation}\label{eq:secondchisolutions3}
\chi_{3}(z)=z-\frac{2}{3}z^{3/2}-a-\frac{2}{3}a^{3/2} \quad \text{and} \quad \chi_{4}(z)= z+\frac{2}{3}z^{3/2}-a+\frac{2}{3}a^{3/2}.
\end{equation}

The new singulants in \eqref{eq:secondchisolutions3}, generated through the HOSP, are equal to zero at locations that correspond to regular points of the differential equation \eqref{eq:secondmaineq2}. This was commented upon by \cite[p.~418]{trinh2013new}:
\vspace{1mm}
\begin{quotation}\noindent{\emph{The curiosity, however, is that these singularities do not appear anywhere in the base series, and so they do not seem to be associated with any eventual divergence.}}
\end{quotation}
\vspace{1mm}
As we know now, the divergent contributions of $A_n$ with singulants \eqref{eq:secondchisolutions3} will switch-off across the HOSL; consequently, they do not generate unexpected singular behaviour at the apparent singularity, itself. However, the corresponding singulants still generate divergence in the asymptotic expansions, within other regions where they are indeed present.

\begin{figure}
    \centering
    \includegraphics[scale=1]{figures/TrinhChap.pdf}
    \caption{The Stokes line and HOSL structure is shown near the singularity (circle) at $z=a$ for $a=-1+\i$. The turning point at $z=0$ (circle), generates a $1>3$ Stokes line as it is a singularity in the $O(\e^{-\chi_1/\ep})$ transseries equation. Thick lines represent Stokes lines, across which exponentially-small terms in the asymptotic solution switch on. The thin line is a HOSL, across which a new divergent contribution switches on in the late-terms, $A_n$.}
    \label{fig:TrinhChap}
\end{figure}
In figure~\ref{fig:TrinhChap}, we plot the Stokes lines and HOSL associated with the singularity at $z=a$. Two naive Stokes lines, which we denote by $\text{B}>1$ and $\text{B}>2$, are generated by \eqref{eq:secondchisolutions2}. The Airy Stokes line found by evaluating Dingles conditions on $\tilde{\chi}=\mp 4 z^{3/2}/3$, for which $1>3$, is also shown. This results in two Stokes crossing points. The first, between the $1>3$ and $\text{B}>2$ Stokes lines, is self consistent. However the second, between $1>3$ and $\text{B}>1$, results in a contradiction. The new divergent component with $\chi_3$ generates a $\text{B}>3$ new Stokes line emanating from the SCP, which resolves this inconsistency.

\subsection{The equatorial Kelvin wave}\label{sec:AppKelvin}

Our last example of the HOSP concerns the study of travelling equatorial Kelvin waves with small latitudinal shear. Here, the system is governed by a second-order singularly perturbed differential equation:
\begin{equation}\label{eq:threeVequation}
\begin{aligned}
\ep^2\bigg[(Y-c-1)(Y-c+1)^2+(Y-c-1)^2(Y-c+1)\bigg] V'' \\
-2Y\bigg[(Y-c-1)(Y-c+1)^2+(Y-c+1)(Y-c-1)^2 +\frac{2 \ep^2(Y-c)^2}{Y} \bigg]V' \\
+\bigg[\big[2Y(Y-c-1)+\ep^2\big](Y-c+1)^2 +(2c-2-\ep^2)(Y-c-1)^2\bigg] V=0.
\end{aligned}
\end{equation}
The unknown is $V(Y)$, representing the eigenfunction amplitude, corresponding to the vertical velocity, and $\epsilon$ is a non-dimensional shear. The wavespeed is $c$, an eigenvalue whose value is obtained from enforcing the boundary condition $V(0)=1$. For further details of this model, see \cite{shelton2023kelvin} and \cite{griffiths2008limiting}. We consider a solution expansion of the form $V(Y)=V_0 + \ep^2 V_1 + \ep^4 V_2+\cdots$. Enforcing the boundary condition at each order of $\ep$ is only possible if the eigenvalue is also expanded as $c=c_0 + \ep^2 c_1 + \ep^4 c_2+\cdots$.

At leading order, once the boundary condition $V_0(0)=1$ is satisfied, the solutions are given by
\begin{equation}\label{eq:threeO1soleig}
V_0(Y)= (1-Y)^{\frac{1}{2}}\e^{Y/2}  \qquad \text{and} \qquad c_0=1.
\end{equation}
This singularity at $Y=1$ results in a divergent series expansion. Furthermore, the eigenvalue expansion $c=1+\frac{1}{2}\ep^2 - \frac{1}{8}\ep^4 + \cdots$ will also diverge.

\subsubsection{Late-term divergence}
Due to the nonlinearity between the solution, $V$, and eigenvalue, $c$, in equation \eqref{eq:threeVequation}, the $O(\ep^{2n})$ equation will contain a countably infinite number of terms as $n \to \infty$. However, only those appearing at leading order as $n \to \infty$ will be required to derive the HOSP. We do not provide this here, and refer to reader to \cite{shelton2023kelvin} for this.

The $O(\ep^{2n})$ equation contains both homogeneous terms, for instance $V_n$ and $V_{n-1}^{\prime}$ and inhomogenous terms such as $c_n$. Both of these contribute to the divergent solution, which contains one term associated with the divergence of $c_n$, and another generated by the singular behaviour at $Y=1$. This yields
\begin{equation}\label{eq:A0sol}
V_n(Y) \sim B_0^{(c_n)}(Y)\Gamma(n+\alpha)+ B_0(Y)\frac{\Gamma(n+\gamma)}{(1-Y^2)^{n+\gamma}},
\end{equation}
where $\alpha$ and $\gamma$ are constants, and the singulant function $\chi=1-Y^2$ satisfies the condition $\chi(1)=0$. Similar to the example from \S\ref{sec:SecondEx}, there is an apparent paradox in that a singularity is seemingly present at $Y=-1$ in the factorial-over-power representation \eqref{eq:A0sol}, which is absent from the early orders of the expansion. This may be resolved by considering lower-order components of $B_0^{(c_n)}$, for which late-late-term divergence and the HOSP emerges. 

\subsubsection{Late-late-term divergence}
In the study by \cite{shelton2023kelvin}, it was assumed that the HOSP occurred, and that expression \eqref{eq:A0sol} would switch off across a HOSL across the imaginary axis. We now demonstrate how this may be determined through the application of the methods developed in this present paper. 

The late-terms of the eigenvalue expansion will diverge in the factorial-over-power manner of $c_n \sim \delta\Gamma(n+\alpha)$, where $\delta$ and $\alpha$ are constants. The particular solution generated by this eigenvalue divergence can then be determined with a late-term ansatz of the form
\begin{equation}\label{eq:thirdexpansion2}
V_n(Y) \sim \sum_{p=0}^{\infty} B^{(c_n)}_p(Y) {\Gamma(n-p+\alpha)} \qquad \text{as $n \to \infty$}.
\end{equation}
In substituting ansatz \eqref{eq:thirdexpansion2} into the $O(\ep^{2n})$ equation and solving the resultant equations, we have $B_0^{(c_n)} \sim 1/Y$ as $Y \to 0$. This singularity at $Y=0$ will result in late-late-term divergence as $p \to \infty$, for which we consider an ansatz of the form
\begin{equation}\label{eq:thirdlatelatediverg}
B_p^{(c_n)}(Y) \sim C(Y) \frac{\Gamma(p+\beta)}{[\tilde{\chi}(Y)]^{p+\beta}} \qquad \text{as $p \to \infty$}.
\end{equation}
The requirement that $\tilde{\chi}(0)=0$ then yields $\tilde{\chi}(Y)=-Y^2$. Thus, across the HOSL $\text{Im}[-Y^2/1]=0$ and $\text{Re}[-Y^2/1]>0$, a new factorial-over-power component of $V_n$ with a singulant given by $\chi=1-Y^2$ will switch on. This is the same singulant as that generated by the singularity at $Y=1$. Thus, the apparent contradiction of $\chi=1-Y^2$ containing an undesired singularity at $Y=-1$ is resolved, since this expression switches off as we cross the HOSL.
\begin{figure}
    \centering
    \includegraphics[scale=1]{figures/Eigenvalue.pdf}
    \caption{The Stokes line structure is shown for the Kelvin wave equation \eqref{eq:threeVequation}. The regular Stokes line for $\chi=1-Y^2$, shown bold, lies along the imaginary axis and between $0$ and $1$ on the real-axis. There is also a HOSL along the imaginary axis, which causes the regular Stokes line on the negative real axis to be irrelevant.}
    \label{fig:Eigenvalue}
\end{figure}


\section{Discussion}\label{sec:Discussion}

Although somewhat obscure, the higher-order Stokes phenomena is generic in the analysis of singular perturbation problems. As we have noted, it accompanies situations where the late-term approximation of a $\ep$-expansion, say $y_n$, is itself divergent in inverse powers of $n$. Consequently Stokes phenomena occurs (in $n$), switching on or off late terms and hence Stokes lines themselves. Alternatively, the genericity of HOSP can be interpreted as a consequence of the Borel plane possessing a branched structure. The HOSP often arises in situations where:  (i) Stokes lines cross; (ii) additional singularities, not present in early orders, are predicted in the late terms; or (iii) a naive divergent series is unable to satisfy necessary boundary conditions.

The methods developed in this paper have been presented for linear ODEs. However, they are also applicable to more general nonlinear problems and PDEs. Our approach is similar, and in some cases more general, than those of \cite{chapman_2005, body2005exponential, howls_2004}; in ours, we have focused on the analysis of the $1/n$ divergent series of the late terms and its subsequent Stokes phenomena. Additionally, we have shown that the HOSP and second-generation Stokes smoothing are both caused by the same type of divergence, either in the late-late-terms, or found in the subsequent transseries expansion. Thus, any HOSP that \emph{e.g.} generates a $B>3$ Stokes line with $\chi_3=\tilde{\chi}+\chi_1$ must necessarily accompany a second-generation $1>3$ Stokes line that switches on $\chi_3$.


One of the main complications with nonlinearity is the determination of the constants of integration, $\Lambda$ and $\tilde{\Lambda}$, of the amplitude functions from the late- and late-late-term ansatzes. For our linear examples, the series expansion for the outer limit of the inner solution could be determined explicitly. This will not be the case, in general, for nonlinear problems, where the associated recurrence relation is more complicated. In the study of PDEs, integration of the singulant and amplitude equations from the late- and late-late-term ansatzes can be challenging \citep{chapman_2005, body2005exponential}. The singularity structure in the Borel plane will also be more complicated and one expects a more intricate Riemann sheet structure.

In addition to the development of methodology, another contribution of our work has been to highlight the beautifully simple example \eqref{eq:themainone_intro}. Similar to the example given in \cite{daalhuis2004higher}, such cases are reminders of the subtle and complicated nature of divergence and transseries, even in apparently benign problems. There are a number of exciting ongoing works related to HOSP that have emerged from a recent Isaac Newton Institute programme, notably by \cite{smoothingAdri} on smoothing of HOSP, \cite{iniKing} on coincident Stokes lines and HOSP, and \cite{nemes2022dingle} on atypical Stokes multipliers due to HOSP. This is an exciting and advanced area of research with still more to discover.



\mbox{}\par
{\bf \noindent Acknowledgements}.
We thank the Isaac Newton Institute for Mathematical Sciences, Cambridge, for the programme ``Applicable resurgent asymptotics: towards a universal theory", during which many insightful discussions took place. In particular, we are thankful for discussions with S. J. Chapman (Oxford), G. Nemes (Tokyo Metropolitan University), A. Olde Daalhuis (Edinburgh), and many other participants of the programme.


%\bibliographystyle{agsm}
% \bibliographystyle{agsmModified}
% \bibliographystyle{siam}
\bibliographystyle{jfm}
%\bibliography{mybib}

\providecommand{\noopsort}[1]{}
\begin{thebibliography}{29}
\expandafter\ifx\csname natexlab\endcsname\relax\def\natexlab#1{#1}\fi
\def\au#1{#1} \def\ed#1{#1} \def\yr#1{#1}\def\at#1{#1}\def\jt#1{\textit{#1}}
  \def\bt#1{#1}\def\bvol#1{\textbf{#1}} \def\vol#1{#1} \def\pg#1{#1}
  \def\publ#1{#1}\def\arxiv#1{#1}\def\org#1{#1}\def\st#1{\textit{#1}}

\bibitem[Aoki(1994)]{aoki1994new}
{\sc \au{Aoki, T.}} \yr{1994}  \at{New turning points in the exact {WKB}
  analysis for higher-order ordinary differential equations}.  \jt{Analyse
  alg{\'e}brique des perturbations singuli{\`e}res. I} .

\bibitem[Aoki {\em et~al.\/}(1998)Aoki, Kawai \& Takei]{aoki1998exact}
{\sc \au{Aoki, T.}, \au{Kawai, T.} \& \au{Takei, Y.}} \yr{1998}  \at{On the
  exact {WKB} analysis for the third order ordinary differential equations with
  a large parameter}.  \jt{Asian J. Math.}  \bvol{2}~(4),  \pg{625--640}.

\bibitem[Aoki {\em et~al.\/}(2001)Aoki, Kawai \& Takei]{aoki2001exact}
{\sc \au{Aoki, T.}, \au{Kawai, T.} \& \au{Takei, Y.}} \yr{2001}  \at{On the
  exact steepest descent method: {A} new method for the description of {S}tokes
  curves}.  \jt{J. Math. Phys.}  \bvol{42}~(8),  \pg{3691--3713}.

\bibitem[Berk {\em et~al.\/}(1982)Berk, Nevins \& Roberts]{berk1982new}
{\sc \au{Berk, H.~L.}, \au{Nevins, W.~M.} \& \au{Roberts, K.~V.}} \yr{1982}
  \at{New {S}tokes’ line in {WKB} theory}.  \jt{J. Math. Physics}
  \bvol{23}~(6),  \pg{988--1002}.

\bibitem[Berry(1989)]{berry_1989}
{\sc \au{Berry, M.~V.}} \yr{1989}  \at{Uniform asymptotic smoothing of {S}tokes
  discontinuities}.  \jt{Proc. R. Soc. Lond. A}  \bvol{422},  \pg{7--21}.

\bibitem[Body {\em et~al.\/}(2005)Body, King \& Tew]{body2005exponential}
{\sc \au{Body, G.~L.}, \au{King, J.~R.} \& \au{Tew, R.~H.}} \yr{2005}
  \at{Exponential asymptotics of a fifth-order partial differential equation}.
  \jt{Eur. J. Appl. Math.}  \bvol{16}~(5),  \pg{647--681}.

\bibitem[Chapman \& Mortimer(2005)]{chapman_2005}
{\sc \au{Chapman, S.~J.} \& \au{Mortimer, D.~B.}} \yr{2005}  \at{Exponential
  asymptotics and {S}tokes lines in a partial differential equation}.
  \jt{Proc. R. Soc. Lond. A}  \bvol{461}~(2060),  \pg{2385--2421}.

\bibitem[Costin \& Dunne(2022)]{costin2022uniformization}
{\sc \au{Costin, O.} \& \au{Dunne, G.~V.}} \yr{2022}  \at{Uniformization and
  constructive analytic continuation of {T}aylor series}.  \jt{Commun. Math.
  Phys.}  \bvol{392}~(3),  \pg{863--906}.

\bibitem[Crew \& Sauzin(2023)]{crew2023}
{\sc \au{Crew, S.} \& \au{Sauzin, D.}} \yr{2023}  \at{Parametric resurgence and
  piecewise stokes constants}.  \jt{In preparation} .

\bibitem[Crew \& Trinh(2023)]{crew2023resurgent}
{\sc \au{Crew, S.} \& \au{Trinh, P.~H.}} \yr{2023}  \at{Resurgent aspects of
  applied exponential asymptotics}.  \jt{arXiv preprint arXiv:2208.07290} .

\bibitem[Dingle(1973)]{dingle_book}
{\sc \au{Dingle, R.~B.}} \yr{1973} {\em Asymptotic Expansions: Their Derivation
  and Interpretation\/}.  \publ{Academic Press, London}.

\bibitem[Griffiths(2008)]{griffiths2008limiting}
{\sc \au{Griffiths, S.~D.}} \yr{2008}  \at{The limiting form of inertial
  instability in geophysical flows}.  \jt{J. Fluid Mech.}  \bvol{605},
  \pg{115--143}.

\bibitem[Honda(2007)]{honda2007stokes}
{\sc \au{Honda, N.}} \yr{2007}  \at{On the {S}tokes geometry of the
  {N}oumi-{Y}amada system}.  \jt{Algebraic, Analytic and Geometric Aspects of
  Complex Differential Equations and their Deformations. Painleve Hierarchies}
  \bvol{2},  \pg{45--72}.

\bibitem[Honda {\em et~al.\/}(2015)Honda, Kawai \& Takei]{honda2015virtual}
{\sc \au{Honda, N.}, \au{Kawai, T.} \& \au{Takei, Y.}} \yr{2015} {\em Virtual
  turning points\/}, ,  \vol{vol.~4}.  \publ{Springer}.

\bibitem[Howls {\em et~al.\/}(2004)Howls, Langman \& Daalhuis]{howls_2004}
{\sc \au{Howls, C.~J.}, \au{Langman, P.~J.} \& \au{Daalhuis, A. B.~Olde}}
  \yr{2004}  \at{On the higher-order {S}tokes phenomenon}.  \jt{Proc. R. Soc.
  Lond. A}  \bvol{460},  \pg{2285--2303}.

\bibitem[King(2022)]{iniKing}
{\sc \au{King, J.}} \yr{2022} Exponential asymptotics, the {S}tokes phenomenon
  and the higher-order {S}tokes phenomenon in some linear partial differential
  equations. Issac Newton Institute Applicable resurgent asymptotics: summary
  workshop presentation 16th Dec.

\bibitem[King \& Chapman(2001)]{king2001asymptotics}
{\sc \au{King, J.~R.} \& \au{Chapman, S.~J.}} \yr{2001}  \at{Asymptotics beyond
  all orders and {S}tokes lines in nonlinear differential-difference
  equations}.  \jt{Eur. J. Appl. Math.}  \bvol{12}~(4),  \pg{433--463}.

\bibitem[Lustri {\em et~al.\/}(2019)Lustri, Pethiyagoda \&
  Chapman]{lustri2019three}
{\sc \au{Lustri, C.~J.}, \au{Pethiyagoda, R.} \& \au{Chapman, S.~J.}} \yr{2019}
   \at{Three-dimensional capillary waves due to a submerged source with small
  surface tension}.  \jt{J. Fluid Mech.}  \bvol{863},  \pg{670--701}.

\bibitem[Mitschi {\em et~al.\/}(2016)Mitschi, Sauzin, Loday-Richaud \&
  Delabaere]{mitschi2016divergent}
{\sc \au{Mitschi, C.}, \au{Sauzin, D.}, \au{Loday-Richaud, M.} \&
  \au{Delabaere, {\'E}.}} \yr{2016} {\em Divergent series, summability and
  resurgence\/}.  \publ{Springer}.

\bibitem[Nemes(2022)]{nemes2022dingle}
{\sc \au{Nemes, G.}} \yr{2022}  \at{Dingle’s final main rule, {B}erry’s
  transition, and {H}owls’ conjecture}.  \jt{J. Phys. A-Math. Theor.}
  \bvol{55}~(49),  \pg{494001}.

\bibitem[Olde~Daalhuis(1996)]{daalhuis1996hyperterminants}
{\sc \au{Olde~Daalhuis, A.~B.}} \yr{1996}  \at{Hyperterminants {I}}.  \jt{J.
  Comp. Appl. Math.}  \bvol{76}~(1-2),  \pg{255--264}.

\bibitem[Olde~Daalhuis(1998)]{daalhuis1998hyperterminants}
{\sc \au{Olde~Daalhuis, A.~B.}} \yr{1998}  \at{Hyperterminants {II}}.  \jt{J.
  Comp. Appl. Math.}  \bvol{89}~(1),  \pg{87--95}.

\bibitem[Olde~Daalhuis(2004)]{daalhuis2004higher}
{\sc \au{Olde~Daalhuis, A.~B.}} \yr{2004}  \at{On higher-order {S}tokes
  phenomena of an inhomogeneous linear ordinary differential equation}.  \jt{J.
  Comp. Appl. Math.}  \bvol{169}~(1),  \pg{235--246}.

\bibitem[Olde~Daalhuis(2022)]{smoothingAdri}
{\sc \au{Olde~Daalhuis, A.~B.}} \yr{2022} Smoothing for the higher-order
  {S}tokes phenomenon. Issac Newton Institute Applicable resurgent asymptotics:
  summary workshop presentation 16th Dec.

\bibitem[Shelton {\em et~al.\/}(2023{\natexlab{{\em a\/}}})Shelton, Chapman \&
  Trinh]{shelton2022Hermite}
{\sc \au{Shelton, J.}, \au{Chapman, S.~J.} \& \au{Trinh, P.~H.}}
  \yr{2023{\natexlab{{\em a\/}}}}  \at{Pathological exponential asymptotics for
  a model problem of an equatorially trapped {R}ossby wave}.  \jt{In
  Preperation} .

\bibitem[Shelton {\em et~al.\/}(2023{\natexlab{{\em b\/}}})Shelton, Griffiths,
  Chapman \& Trinh]{shelton2023kelvin}
{\sc \au{Shelton, J.}, \au{Griffiths, S.}, \au{Chapman, S.~J.} \& \au{Trinh,
  P.~H.}} \yr{2023{\natexlab{{\em b\/}}}}  \at{On the exponentially-small
  instability of the equatorial {K}elvin wave}.  \jt{In preparation} .

\bibitem[Shudo(2008)]{shudo2007role}
{\sc \au{Shudo, A.}} \yr{2008} {\em A role of virtual turning points and new
  {S}tokes curves in {S}tokes geometry of the quantum H{\'e}non map. Algebraic
  analysis of differential equations\/}.  \publ{Springer}.

\bibitem[Takei(2017)]{takei2017wkb}
{\sc \au{Takei, Y.}} \yr{2017}  \at{{WKB} analysis and {S}tokes geometry of
  differential equations}.  \bt{In {\em Analytic, algebraic and geometric
  aspects of differential equations\/}},  \pg{pp. 263--304}.  \publ{Springer}.

\bibitem[Trinh \& Chapman(2013)]{trinh2013new}
{\sc \au{Trinh, P.~H.} \& \au{Chapman, S.~J.}} \yr{2013}  \at{New
  gravity--capillary waves at low speeds. {P}art 2. {N}onlinear geometries}.
  \jt{J. Fluid Mech.}  \bvol{724},  \pg{392--424}.

\end{thebibliography}


\appendix

\section{Inner analyses near $z=0$ and $z=1$}\label{sec:AppInner}
The purpose of the section is to determine the unknown constants appearing in the late-terms, $y_n$, of the asymptotic solution. We note that, in addition to $\alpha$ and $\beta$ appearing in the late- and late-late-term ansatzes
\eqref{eq:fullfactorialpower} for $y_n$ as $n \to \infty$ and \eqref{eq:latelatetermBp} for $B_p$ as $p \to \infty$, each of $B_p$ for $p \geq 0$ contains an unknown constant of integration, $\Lambda_p$. However, explicit knowledge of all of these constants, $\Lambda_p$, is not required. Firstly, we require $\Lambda_0$ which appears in the Stokes switching induced by the $\chi_1=z^2/2$ late-term component. Secondly, there is the Stokes switching induced by the $\chi_1 + \tilde{\chi}=z-1/2$ late-term component generated through the HOSP; this requires knowledge of
 $B_p$ as $p \to \infty$, for which the amplitude solution \eqref{eq:latelatesolutions} contains the constant of integration $\tilde{\Lambda}$.

\subsection{Inner analysis near $z=0$}\label{sec:AppInnerz0}
We begin by deriving the inner equation near $z=0$, near to which the early orders of expansion \eqref{eq:baseasymptexp} reorder. This is due to a turning point in the original differential equation that generates a singularity in the leading order solution, $y_0^{(0)}(z)=z^{-1}$. Since $y_0^{(0)} \sim z^{-1}$ and $\ep y_1^{(0)} \sim \ep z^{-3}$ as $z \to 0$ from equation \eqref{eq:y0earlyorders}, the early orders of the expansion reorder when $z \sim \ep^{1/2}$. To study the solution in this region, we introduce the inner variable, $\hat{z}$, and the inner solution, $\hat{y}$, by
\begin{equation}\label{eq:AppInnerVar}
  z= \ep^{1/2} \hat{z} \qquad \text{and} \qquad   y_{\text{outer}}= \frac{1}{ (\ep^{1/2} \hat{z})} \hat{y}_{\text{inner}}(\hat{z}).
\end{equation}
The inner region is defined by $\hat{z}=O(1)$, and the precise definition of the inner variable in \eqref{eq:AppInnerVar} ensures that $\hat{y}_{\text{inner}} =O(1)$ in the inner region. Next, we derive the inner equation by substituting equations \eqref{eq:AppInnerVar} into the outer equation \eqref{eq:amplitudeeq0}. As the inner limit of the outer divergent solution, $\ep^n y_n$, will be seen in \eqref{eq:innerInnerLim} to reorder into the leading-order inner solution, only the leading-order, in $\ep$, inner equation need be considered. This yields
\begin{equation}\label{eq:AppInnerEq}
 \frac{1}{\hat{z}} \dd{\hat{y}}{\hat{z}} +\bigg(1-\frac{1}{\hat{z}^2} \bigg) \hat{y}= 1.
\end{equation}

\subsubsection{Inner solution}\label{sec:AppInnerSolz0}
\mbox{}\par\noindent
To motivate the correct form to take for the inner expansion of $\hat{y}(\hat{z})$, we take the inner limit of the early orders of the outer expansion from \eqref{eq:y0earlyorders} by substituting for $z=\ep^{1/2}\hat{z}$ from \eqref{eq:AppInnerVar} and expanding as $\ep \to 0$. This yields a 1-term inner limit of 3-terms of the outer-solution, given by $\hat{y} \sim  1+{\hat{z}^{-2}} + 3{\hat{z}^{-4}}+\cdots$. Furthermore, as knowledge of the inner solution of equation \eqref{eq:AppInnerEq} is required only in the limit of $\hat{z} \to \infty$ in order to match with the outer solution, we will consider a series solution for $\hat{y}(\hat{z})$ of the form
\begin{equation}\label{eq:AppInnerSol3}
\hat{y}(\hat{z}) = \sum_{n=0}^{\infty} \frac{\hat{a}_{n}}{\hat{z}^{2n}}.
\end{equation}

Substitution of series solution \eqref{eq:AppInnerSol3} into the inner equation \eqref{eq:AppInnerEq} yields at each order of $\hat{z}$ the equations
\begin{equation}\label{eq:AppInnerSol4}
\hat{a}_0=1 \qquad \text{and} \qquad \hat{a}_n = (2n-1) \hat{a}_{n-1},
\end{equation}
for $n \geq 1$. Typically, and usually for nonlinear problems, this inner recurrence relation must be solved numerically to a reasonably large value of $n \approx 100$, for which matching with the outer solution yields an approximate value for the constants of integration in the amplitude function of the factorial-over-power ansatz. 
However,  recurrence relation \eqref{eq:AppInnerSol4} has the exact solution
\begin{equation}\label{eq:AppInnerSol5}
\hat{a}_{n} = \frac{\Gamma(2n)}{2^{n-1}\Gamma(n)}.
\end{equation}

\subsubsection{Inner-outer matching}\label{sec:AppInnerMatchingz0}
We now match with the outer divergent solution by writing the inner solution \eqref{eq:AppInnerSol3} in terms of the outer variable, $z$, and outer solution, $y$, from equation \eqref{eq:AppInnerVar}. This outer limit of the inner solution is given by
\begin{equation}\label{eq:innerOuterLim}
y=\frac{1}{\ep^{1/2}\hat{z}}\sum_{n=0}^{\infty} \frac{\hat{a}_n}{\hat{z}^{2n}} = \sum_{n=0}^{\infty} \ep^n\frac{\hat{a}_n}{z^{2n+1}}.
\end{equation}
We match the $O(\ep^n)$ component of this outer limit with the $O(\ep^n)$ component of the inner limit (as $z \to 0$) of the outer divergent solution, which is given by
\begin{equation}\label{eq:innerInnerLim}
y_n = \frac{\ep^n \Lambda_0}{1-z} \frac{\Gamma(n+\alpha)}{(z^2/2)^{n+\alpha}} \sim  \Lambda_0 \frac{\Gamma(n+\alpha)}{(z^2/2)^{n+\alpha}}.
\end{equation}
Matching the $O(\ep^n)$ component of \eqref{eq:innerOuterLim} with \eqref{eq:innerInnerLim} requires $\alpha=1/2$, which then gives an expression for the constant, $\Lambda_0$. Thus, we have found
\begin{equation}\label{eq:appendixalpha}
\alpha=1/2 \qquad \text{and} \qquad \Lambda_0 = \lim_{n \to \infty}\bigg( \frac{\Gamma(2n)}{2^{2n-1/2}\Gamma(n)\Gamma(n+1/2)}\bigg)= \frac{1}{\sqrt{2 \pi}}.
\end{equation}



\subsection{Inner analysis near $z=1$}\label{sec:AppInnerz1}
The constants $\beta$ and $\tilde{\Lambda}$ are now determined by matching with an inner solution at $z=1$. These constants appear in the divergent form of $B_p$ as $p \to \infty$. One method is to consider a boundary layer of diminishing width as $n \to \infty$ on account of the late-term reordering $B_0 \Gamma(n+\alpha)/\chi^{n+\alpha} \sim B_1 \Gamma(n+\alpha-1) / \chi^{n+\alpha-1}$. However, we note that a power series expansion,
\begin{equation}\label{eq:newtransseriesexp}
y^{(1)}(z,\ep) = \sum_{p=0}^{\infty} \ep^p B_p(z),
\end{equation}
in the $\chi=z^2/2$ transseries equation \eqref{eq:amplitudeeq1} yields at each order of $\ep$ the same equations as those satisfied by $B_p$ in \eqref{eq:amplitudeeq1factorial} and \eqref{eq:amplitudeeq2factorial}. Thus, we may determine the unknown constants in the divergent form of $B_p$ through the consideration of a boundary layer, of diminishing width as $\ep \to 0$, at $z=1$ in the governing equation \eqref{eq:amplitudeeq1} for $y^{(1)}$.

Since the early orders of expansion \eqref{eq:newtransseriesexp} reorder when $\Lambda_0 (1-z)^{-1} \sim -\ep \Lambda_0 (1-z)^{-3}$, and $y^{(1)} \sim \Lambda_0 / (1-z)$, we introduce the inner variable, $\hat{z}$ and inner solution, $\hat{y}$, via
\begin{equation}\label{eq:newtranseriesinnerthings}
(1-z) = \ep^{1/2} \hat{z} \qquad \text{and} \qquad y^{(1)}=\frac{\Lambda_0}{\ep^{1/2}\hat{z}}\hat{y}(\hat{z}).
\end{equation}
The leading-order inner equation for $\hat{y}$ may be found by substituting relations \eqref{eq:newtranseriesinnerthings} into the transseries equation \eqref{eq:amplitudeeq1} for $y^{(1)}$, and retaining the dominant terms as $\ep \to 0$. This yields
\begin{equation}\label{eq:AppInnerEqz1}
\frac{1}{\hat{z}}\dd{^2\hat{y}}{\hat{z}^2}-\bigg(1+\frac{2}{\hat{z}^2}\bigg)\dd{\hat{y}}{\hat{z}}+\frac{2}{\hat{z}^3}\hat{y}=0,
\end{equation}
for which a series solution as $\hat{z} \to \infty$ may be considered. The coefficients of this series solution may be determined exactly, yielding
\begin{equation}\label{eq:AppInnerEqz1_again}
\hat{y}(\hat{z}) = \sum_{p=0}^{\infty} \frac{(-1)^p\Gamma(2p)}{2^{p-1}\Gamma(p)} \hat{z}^{-{2p}}.
\end{equation}

\subsubsection{Inner-outer matching}\label{sec:AppInnerMatchingz1}
Now that the outer limit of the inner solution near $z=1$ is known from \eqref{eq:AppInnerEqz1_again} we may match, in terms of the outer variable $z$, with the inner limit of the outer solution. Writing \eqref{eq:AppInnerEqz1_again} in terms of the outer variables $z$ and $y$ from \eqref{eq:newtranseriesinnerthings} gives the outer limit of the inner solution as
\begin{equation}\label{eq:Appouterlimz1}
y= \sum_{p=0}^{\infty} \frac{\Lambda_0(-1)^p\Gamma(2p)}{2^{p-1}\Gamma(p)} \frac{\ep^{p}}{(1-z)^{2p+1}}.
\end{equation}
We now match the $O(\ep^p)$ component of \eqref{eq:Appouterlimz1} with that of the outer solution \eqref{eq:newtransseriesexp}, 
\begin{equation}
B_p \sim \tilde{\Lambda}(-1)^{p+\beta}2^{p+\beta}\frac{\Gamma(p+\beta)}{(1-z)^{2p+2\beta}},
\end{equation}
which yields the constants $\beta$ and $\tilde{\Lambda}$ as
\begin{equation}\label{eq:latelateconstantapp}
\beta=1/2 \qquad \text{and} \qquad \tilde{\Lambda}= \frac{\i}{2 \pi}.
\end{equation}

\section{Details for the Pearcey equation}\label{sec:FirstExApp}
Equation \eqref{eq:pearceymain} permits three WKBJ solutions of the form $A_m(z,\ep) \exp{(-S_{m}(z) / \ep)}$, where $m=1$, $2$, and $3$. Substitution of this ansatz into \eqref{eq:pearceymain} yields the following equations for the singulants, $S_m^{\prime}$, and amplitude function, $A_m$:
\begin{subequations}\label{eq:pearcyafterWKB}
\begin{align}
\label{eq:pearceyexpsol}
\big[S_m'(z)\big]^3-S_m'(z)+\i z=0, \\
\label{eq:pearceyampeq}
\ep^2 A_m^{\prime \prime \prime}-3 \ep S_m^{\prime} A_m^{\prime \prime}+\Big(3(S_m^{\prime})^2-3 \ep S_m^{\prime \prime}-1\Big)A_m^{\prime}+\Big(3S_m^{\prime}S_m^{\prime \prime}-\epsilon S_m^{\prime \prime \prime}\Big)A_m=0.
\end{align}
\end{subequations}

It will be seen that the asymptotic expansions for the amplitude functions, $A_m$, diverge. The Stokes phenomenon that occurs as a consequence of this divergence will switch on one of the other initial singulants. For instance, across a $1>2$ Stokes line, we will have $ A_1 \exp{(-S_1/\ep)} \mapsto A_1 \exp{(-S_1/\ep)} + A_2 \exp{(-S_2/\ep)}$. However, this $1>2$ Stokes line will intersect a $2>3$ Stokes line at a Stokes crossing point, for which a new Stokes line is required in order to avoid a contradiction. We now demonstrate how this new Stokes line, and the HOSL that generates it, may be derived through determination of the late-late divergence.

In expanding $A_m(z,\ep) = A_m^{(0)} + \ep A_m^{(1)} + \cdots$ in equation \eqref{eq:pearceyampeq}, we find the following equations at each order of $\ep$:
\begin{subequations}\label{eq:pearcyeachorderofamp}
\begin{align}
\label{eq:pearceyexpeqO1}
O(1):& \qquad \mathcal{D}\big[A_m^{(0)}\big] \equiv \Big[3(S_m^{\prime})^2-1\Big]A^{(0)\prime}_m+3 S_m^{\prime} S_m^{\prime \prime} A^{(0)}_m=0,\\
\label{eq:pearceyexpeqOep}
O(\ep):& \qquad \mathcal{D}\big[A_{m}^{(1)}\big] =3 S_m^{\prime}A_m^{(0)\prime \prime}+3 S_m^{\prime \prime}A_m^{(0)\prime}+S_m^{\prime \prime \prime}A_m^{(0)},\\
\label{eq:pearceyexpeqOepn}
O(\ep^n):& \qquad \mathcal{D}\big[A_{m}^{(n)}\big] =3 S_m^{\prime}A_{m}^{(n-1)\prime \prime}+3 S_m^{\prime \prime}A_{m}^{(n-1)\prime}+S_m^{\prime \prime \prime}A_{m}^{(n-1)}-A_{m}^{(n-2)\prime \prime \prime},
\end{align}
\end{subequations}
where $\mathcal{D}$ is the linear operator defined in the leading-order equation \eqref{eq:pearceyexpeqO1}.
The first of these, equation \eqref{eq:pearceyexpeqO1}, yields the solution
\begin{equation}\label{eq:pearceyO1sola}
A^{(0)}_m(z)=\frac{C^{(0)}_m}{(3[S_m^{\prime}(z)]^2-1)^{1/2}},
\end{equation}
where $C_m^{(0)}$ is a constant of integration. The singular behaviour of \eqref{eq:pearceyO1sola} may be determined by solving the cubic equation \eqref{eq:pearceyexpsol} for $S_m^{\prime}$, which yields the three solutions
\begin{equation}\label{eq:valuesofS123}
S^{\prime}_1(z)= \frac{12+p^2(z)}{6p(z)} \quad \text{and} \quad S^{\prime}_{2,3}(z)=-\frac{(1 +a \i \sqrt{3})p^2(z)+12(1 -a \i \sqrt{3})}{12p(z)}.
\end{equation}
Here, $a=1$ for $S_2^{\prime}$ and $a=-1$ for $S_3^{\prime}$, and $p(z)=\big[12 \i \big(-9z+\sqrt{81z^2+12}\big)\big]^{1/3}$.

We note that $A_m^{(0)}$ is singular whenever $(S_m^{\prime})^2=1/3$, for which substitution into \eqref{eq:pearceyexpsol} yields two singular points, given by
\begin{equation}\label{eq:actuallocations}
z_{+}=\frac{ 2\sqrt{3}\i}{9} \qquad \text{and} \qquad z_{-}=- \frac{ 2\sqrt{3}\i}{9}.
\end{equation}
The singularities in $A_m^{(0)}$ lie at $z_{+}$ for $m=2$ and $m=3$, and at $z_{-}$ for $m=1$ and $m=2$. All of these singularities are of order $1/4$. As an example, we have that $A_1^{(0)}(z) \sim 6  C_1^{(0)}/ (-36\sqrt{3}\i[z-z_{-}])^{1/4}$.


\subsection{Integrating the singulant solutions}
In order to determine the Stokes line structure in the next section, each of \eqref{eq:valuesofS123} must be integrated to find $S_m$. We now determine these explicitly through examination of the Hamiltonian system for the Borel operator of the Pearcy equation, much like that performed by \cite{honda2015virtual}.

We consider the Borel operator, $\mathscr{P}_B$, associated with the Pearcy equation \eqref{eq:pearceymain}. 
The symbol, $\sigma$, of the operator $\mathscr{P}_B$ is a complex Hamiltonian on the phase space $T^*(\mathbb{C}_w \times \mathbb{C}_z)$, where $\mathbb{C}_z$ is the physical plane and $\mathbb{C}_w$ is the Borel plane. These are given by 
\begin{equation}\label{eq:Appborelop}
  \mathscr{P}_B = \partial^3_z- \partial_w^2 \partial_z- \i z\partial_w^3 \qquad \text{and}\qquad \sigma = \xi^3 - \eta^2 \xi - \i z \eta^3,
\end{equation}
where $\xi$ and $\eta$ are local coordinates on the cotangent directions. The Hamiltonian equations arising from the natural K\"ahler structure on $T^*(\mathbb{C}_w \times \mathbb{C}_z)$,
\begin{equation}
\label{eq:hamiltonianequations}
 \dot{z}(\tau) = \partial_\xi \sigma, 
 \quad \dot{w}(\tau) = \partial_{\eta} \sigma, 
 \quad \dot{\xi}(\tau) = -\partial_{z}\sigma, 
 \quad  \dot{\eta}(\tau) = -\partial_w \sigma, 
 \quad  \sigma (\tau) = 0,
\end{equation}
may be solved to find the solutions
\begin{subequations}\label{eq:AppPartialSols}
\begin{align}
 \label{eq:AppPartialSols1}  {\eta}(\tau) &= \eta_0,\\
  \label{eq:AppPartialSols2}   {\xi}(\tau) &= \i \eta_0^3 \tau+\xi_0,\\
  \label{eq:AppPartialSols3}  {z}(\tau) &=- \eta_0^6 \tau^3+ 3 \i \xi_0 \eta_0^3 \tau^2+(3\xi_0^2-\eta_0^2)\tau +z_0,\\
  \label{eq:AppPartialSols4}  {w}(\tau) &=\frac{3 \i \eta_0^8 }{4} \tau^4+3 \eta_0^5 \xi_0 \tau^3+ \frac{\i \eta_0^4-9 \i \eta_0^2 \xi_0^2}{2}\tau^2 -(2 \eta_0\xi_0 + 3 \i \eta_0^2 z_0) \tau + w_0.
\end{align}
\end{subequations}
Here, $\eta_0$, $\xi_0$, $z_0$, and $w_0$ are constants of integration. The desired solution, $S_m$, equals $w(\tau)$ from \eqref{eq:AppPartialSols4}, which may be expressed as a function of $z$ through knowledge of $z(\tau)$ from \eqref{eq:AppPartialSols3}. The four constants of integration in \eqref{eq:AppPartialSols} are determined through imposition of the conditions
\begin{equation}\label{eq:Appboundaryconds}
\eta(0)=1, \qquad w(0)=0, \qquad z(0)=\pm \frac{2 \sqrt{3}\i}{9} ,  \qquad \sigma(0)=0.
\end{equation}
The first three of these yield $\eta_0=1$, $w_0=0$, and $z_0=\pm 2 \sqrt{3}\i/9$. The last of conditions \eqref{eq:Appboundaryconds} gives the relationship $\xi_0^3 - \xi_0 - \i z_0=0$. Since $z_0$ is known, this equation has the four solutions of $\xi_0=\pm \sqrt{3}/3$ and $\xi_0=\pm 2 \sqrt{3}/3$. The last two of these solutions give $S_m\neq 0$ at the singular point of the leading-order outer solution $A_m^{(0)}$ from \eqref{eq:pearceyO1sola}. Thus, we consider only $\xi_0=\pm \sqrt{3}/3$ for which the plus sign may be taken without loss of generality as the resultant expression for $w$ involves $\xi_0^2$ only. In defining $\hat{\tau}= - \i ( \tau - \i \xi_0)$, we substitute $\eta_0=1$, $w_0=0$, $z_0=\pm 2 \sqrt{3}\i/9$, and $\xi_0= \sqrt{3}/3$ into equations \eqref{eq:AppPartialSols3} and \eqref{eq:AppPartialSols4} to find
\begin{equation}\label{eq:AppPartialSolsnext}
   {z}(\hat{\tau}) =\i \hat{\tau}(\hat{\tau}^2-1) \qquad \text{and} \qquad {w}(\hat{\tau}) =\frac{3 \i}{4}\left(\hat{\tau}^2- 1/3 \right)^2.
\end{equation}
We recognise the first of equations \eqref{eq:AppPartialSolsnext} as the same cubic equation that governs $S_m^{\prime}$ in \eqref{eq:pearceyexpsol}, and hence $S^{\prime}_m=\hat{\tau}$. Furthermore, we may calculate $\mathrm{d}w/\mathrm{d}z = (\mathrm{d}w/\mathrm{d}\hat{\tau})(\mathrm{d}\hat{\tau}/\mathrm{d}z)=\hat{\tau}=S_m^{\prime}(z)$, which verifies that $S_m^{\prime}=w^{\prime}$. Since we have $S^{\prime}_m=\hat{\tau}$, it is not necessary to invert $z(\hat{\tau})$ to find $\hat{\tau}(z)$. Instead, we simply write our expression for $S_m$ in terms of the previously determined function $S_m^{\prime}$, which yields
\begin{equation}\label{eq:AppcheckS5}
S_m(z) =\frac{3 \i}{4}\left(\big[S_m^{\prime}(z)\big]^2- \frac{1}{3}\right)^2.
\end{equation}
Equation \eqref{eq:AppcheckS5} is the main result of this section: an explicit solution for the singulant, $S_m$, specified in terms of the function $S^{\prime}_m$ previously determined in equation \eqref{eq:valuesofS123}.

\subsection{Late-term divergence}
We consider a late-term ansatz of the form
\begin{equation}\label{eq:pearceyOnansatzfull}
A^{(n)}_m(z) \sim \sum_{p=0}^{\infty}B^{(p)}_m(z) \frac{\Gamma(n+\alpha-p)}{[{\chi_{m}^{}}(z)]^{n+\alpha-p}},
\end{equation}
where $\alpha$ is a constant, $\chi_m$ is the singulant, and $B_m^{(p)}$ are amplitude functions. The HOSP may be detected through analysis of the late-late term divergence of $B_m^{(p)}$ as $p \to \infty$. Substitution of \eqref{eq:pearceyOnansatzfull} into the $O(\ep^n)$ equation \eqref{eq:pearceyexpeqOepn} yields at each order of $\Gamma(n+\alpha-p)/\chi_m^{n+\alpha-p}$ the equations
\begin{subequations}\label{eq:latelateamp}
\begin{align}
\label{eq:pearcychieq}
&{\chi}_{m}^{\prime}\Big( ({\chi}_{m}^{\prime})^2+3 S_{m}^{\prime}{\chi}_{m}^{\prime}+3(S_{m}^{\prime})^2-1\Big)=0,\\
\label{eq:latelateampO1}
&\mathcal{L}\Big[B^{(0)}_m\Big] \equiv \left(3[S_m^{\prime}+\chi_m^{\prime}]^2-1\right)B^{(0)\prime}_m
+3(S_m^{\prime}+\chi_m^{\prime})(S_m^{\prime \prime}+\chi_m^{\prime \prime})B^{(0)}_m=0,\\
\label{eq:latelateampO2}
&\mathcal{L}\Big[B^{(1)}_m\Big] =3\left(S_m^{\prime}+\chi_m^{\prime} \right)B^{(0)\prime \prime}_m+3 \left(S_m^{\prime \prime}+\chi_m^{\prime \prime}\right)B^{(0)\prime}_m
+\left(S_m^{\prime \prime \prime}+\chi_m^{\prime \prime \prime}\right)B^{(0)}_m,\\
\label{eq:latelateampOp}
&\begin{aligned}
\mathcal{L}\Big[B^{(p)}_m\Big] &=3\left(S_m^{\prime}+\chi_m^{\prime} \right)B^{(p-1)\prime \prime}_{m}+3 \left(S_m^{\prime \prime}+\chi_m^{\prime \prime}\right)B^{(p-1)\prime}_{m}\\
& \quad +\left(S_m^{\prime \prime \prime}+\chi_m^{\prime \prime \prime}\right)B^{(p-1)}_{m}-B^{(p-2)\prime \prime \prime}_{m},
\end{aligned}
\end{align}
\end{subequations}
where \eqref{eq:latelateampOp} holds for $p \geq 2$, and $\mathcal{L}$ is the differential operator defined in \eqref{eq:latelateampO1}.

We begin by solving equation \eqref{eq:pearcychieq} to find $\chi_m$. Six solutions, ${\chi}_{m}^{\prime}=-3S_{m}^{\prime}/2 \pm \sqrt{4-3(S_{m}^{\prime})^2}/2$, are found in total, but two of these are unable to match with inner solutions near the associated singular points. It may be verified that these solutions for $\chi_m^{\prime}$ yield the difference between another $S_j^{\prime}$ and $S_m^{\prime}$ (for instance $\chi_1^{\prime}=S_2^{\prime}-S_1^{\prime}$).
Integration of the expressions for $\chi_m^{\prime}$ then yields the four initial singulants
\begin{equation}
\label{eq:chisolnewes2}
\left\{ \quad 
\begin{aligned}
\chi_1(z)&=S_2(z)-S_1(z), \qquad \chi_3(z)=S_2(z)-S_3(z),\\
\chi_{2a}(z) &=S_1(z)-S_2(z),\qquad  \chi_{2b}(z) =S_3(z)-S_2(z),
\end{aligned}\right.
\end{equation}
which satisfy $\chi_1(z_{-})=0$, $\chi_3(z_{+})=0$, $\chi_{2a}(z_{-})=0$, and $\chi_{2b}(z_{+})=0$. 

\subsection{Late-late-term divergence}

We now consider equations \eqref{eq:latelateamp} for the late-term amplitude functions, $B_m^{(p)}$. The first of these may be solved to find the solution
\begin{equation}\label{eq:B0sol}
B_m^{(0)}(z)=\frac{\Lambda_m^{(0)}}{\Big(3(S_m^{\prime}+\chi_m^{\prime})^2-1\Big)^{1/2}},
\end{equation}
where $\Lambda_m^{(0)}$ is a constant of integration. For $m=2$, $B_2^{(0)}$ in \eqref{eq:B0sol} is singular at the same locations as that predicted by the factorial-over-power form with $\chi_2$. However, for $m=1$ and $m=3$, \eqref{eq:B0sol}
 is singular at additional locations beyond that where the factorial-over-power form has $\chi_m(z)=0$. For example, when $m=1$, while $\chi_1$ predicts a singularity at $z_{-}$ only, the amplitude function \eqref{eq:B0sol} is also singular at $z_{+}$. This will lead to a new divergent series as $p \to \infty$, which we now study for $m=1$ and $m=3$.

In substituting a late-late-term factorial-over-power ansatz, of the form
\begin{equation}\label{eq:Bnsol}
B_m^{(p)}(z)\sim C_{m}(z) \frac{\Gamma(p+\beta)}{[\tilde{\chi}_m(z)]^{p+\beta}},
\end{equation}
into equation \eqref{eq:latelateampOp}, we find the equation $(\tilde{\chi}_m^{\prime})^2+3(S_m^{\prime}+\chi_m^{\prime})\tilde{\chi}^{\prime}_m+3(S_m^{\prime}+\chi_m^{\prime})^2-1=0$. For each value of $m$, only one solution of this quadratic equation has the correct singular behaviour after the boundary condition $\tilde{\chi}_1(z_{+})=0$ or $\tilde{\chi}_3(z_{-})=0$ is applied. This yields our late-late-term singulant as
\begin{equation}
\label{eq:chisollatelate}
\tilde{\chi}_m(z)=\left\{
\begin{aligned}
 S_{3}(z)-S_{2}(z)  \qquad \text{if $m=1$}, \\
  S_{1}(z)-S_{2}(z)  \qquad \text{if $m=3$}.
\end{aligned}\right.
\end{equation}
From the method developed in section \ref{sec:smoothinghosp}, these will induce the HOSP, in which a new factorial-over-power component of $A_m^{(n)}$ with a singulant of $\chi_m+\tilde{\chi}_m$ will switch on across a HOSL given by $\text{Im}[\tilde{\chi}_m/\chi_m]=0$ and $\text{Re}[\tilde{\chi}_m/\chi_m] \geq 0$. The dominant components of the late-terms of the initial amplitude functions, $A_m^{(n)}$, are therefore given by
\begin{subequations}\label{eq:finalfacoverpow}
\begin{align}
\label{eq:finalfacoverpow1}
A_1^{(n)}(z) &\sim B_1^{(0)}(z) \frac{\Gamma(n+\alpha)}{(S_{2}-S_1)^{n+\alpha}} + \tilde{\sigma}_1(z,n)C_1(z)\frac{\Gamma(n+\alpha)}{(S_{3}-S_1)^{n+\alpha}} ,\\
\label{eq:finalfacoverpow2}
A_2^{(n)}(z) &\sim B_{2a}^{(0)}(z)\frac{\Gamma(n+\alpha)}{(S_{3}-S_2)^{n+\alpha}}+B_{2b}^{(0)}(z)\frac{\Gamma(n+\alpha)}{(S_{1}-S_2)^{n+\alpha}},\\
\label{eq:finalfacoverpow3}
A_3^{(n)}(z) &\sim B_3^{(0)}(z)\frac{\Gamma(n+\alpha)}{(S_{2}-S_3)^{n+\alpha}} + \tilde{\sigma}_3(z,n)C_3(z)\frac{\Gamma(n+\alpha)}{(S_{1}-S_3)^{n+\alpha}},
\end{align}
\end{subequations}
where $\tilde{\sigma}_1$ and $\tilde{\sigma}_3$ are the higher-order Stokes multipliers.
The resultant Stokes lines structure was shown in figure~\ref{fig:Pearcey}.

\end{document} 





