\documentclass[a4paper,11pt]{article}
\usepackage{pos}
\usepackage{hyperref}
\usepackage{subcaption}
\usepackage{dsfont}
\usepackage{slashed}



%%%%%%%%%%%%%%%%%%%%%%%%%%%%%%%%%%%%%%%%%%%%%%%%


\newcommand{\m}{{\bf m}}
\newcommand{\h}{{\bf h}}
\newcommand{\Hnu}{{\bf H}^\nu}
\newcommand{\eq}{{\rm eq}}
\newcommand{\wh}{\widehat}
\newcommand{\Id}{{\bf 1}}



\definecolor{purple}{rgb}{1, 0, 1}

\newcommand{\ie}{\emph{i.e.,}\xspace}
\newcommand{\eg}{\emph{e.g.,}\xspace}
\newcommand{\abr}{\emph{abbr.}\xspace}
\newcommand{\ea}{\emph{et al.}\xspace}
\newcommand{\gensync}{\emph{GenSync}\xspace}
\newcommand{\colosseum}{\emph{Colosseum}\xspace}
\newcommand{\srep}{\emph{SREP}\xspace} % Set Reconciliation Enhances
\newcommand{\srepsim}{\emph{SREPSim}\xspace}
% Propagation
\newcommand{\esrep}{\emph{E-SREP}\xspace}
\newcommand{\epsrep}{\emph{EP-SREP}\xspace}
\newcommand{\mesrep}{\emph{ME-SREP}\xspace}
\newcommand{\mempoolsync}{\emph{MempoolSync}}

\newcommand{\fref}[1]{Fig.~\ref{#1}}
\newcommand{\tref}[1]{Table~\ref{#1}}
\newcommand{\aref}[1]{Algorithm~\ref{#1}}
\newcommand{\procref}[1]{Procedure~\ref{#1}}
\newcommand{\sref}[1]{Section~\ref{#1}}
\newcommand{\lineref}[1]{line~\ref{#1}}
\newcommand{\appref}[1]{Appendix~\ref{#1}}

% Change \eqref
\LetLtxMacro{\originaleqref}{\eqref}
\renewcommand{\eqref}{Eq.~\originaleqref}

% Theorems and corollaries
\newcounter{theoremcount}
\setcounter{theoremcount}{0}
\DeclareRobustCommand{\theorem}[1]{%
  \refstepcounter{theoremcount}%
  \noindent\textit{\textbf{Theorem \thetheoremcount\label{theorem:#1}: }}%
}
\DeclareRobustCommand{\theoremref}[1]{Theorem~\ref{theorem:#1}}

\DeclareRobustCommand{\proof}{\emph{Proof:}\xspace}
\DeclareRobustCommand{\qqed}{\hfill$\blacksquare$}

\newcounter{corollcount}
\setcounter{corollcount}{0}
\DeclareRobustCommand{\coroll}[1]{%
  \refstepcounter{corollcount}%
  \noindent\textit{\textbf{Corollary \thecorollcount\label{coroll:#1}: }}%
}
\DeclareRobustCommand{\corollref}[1]{Corollary~\ref{coroll:#1}}

\newcounter{lemmacount}
\setcounter{lemmacount}{0}
\DeclareRobustCommand{\lemma}[1]{%
  \refstepcounter{lemmacount}%
  \noindent\textit{\textbf{Lemma \thelemmacount\label{lemma:#1}: }}%
}
\DeclareRobustCommand{\lemmaref}[1]{Lemma~\ref{lemma:#1}}

\newcounter{definitioncount}
\setcounter{definitioncount}{0}
\DeclareRobustCommand{\definition}[1]{%
  \refstepcounter{definitioncount}%
  \noindent\textit{\textbf{Definition \thedefinitioncount\label{definition:#1}: }}%
}
\DeclareRobustCommand{\defref}[1]{Definition~\ref{definition:#1}}

%notes of different authors
\newif\ifnotes
\notestrue
\notesfalse

\newif\ifdiff
\difftrue
\difffalse

\newcommand{\anote}[1]{\ifnotes $\ll$\textsf{\textcolor{purple}{Ari: {#1}}}$\gg$ \fi}
\newcommand{\nnote}[1]{\ifnotes $\ll$\textsf{\textcolor{orange}{Novak: {#1}}}$\gg$ \fi}
\newcommand{\diff}[1]{\ifdiff\textcolor{orange}{#1}\else#1\fi}

%%% Local Variables:
%%% mode: latex
%%% TeX-master: "main"
%%% End:



\title{Tri-Resonant Leptogenesis}
\author{P. Candia da Silva}
\author{D. Karamitros}
\author{T. McKelvey}
\author*{A. Pilaftsis}

\affiliation{Department of Physics and Astronomy, University of Manchester,\newline Manchester, M13 9PL, United Kingdom}

\emailAdd{pablo.candiadasilva@manchester.ac.uk}
\emailAdd{dimitrios.karamitros@manchester.ac.uk}
\emailAdd{thomas.mckelvey@manchester.ac.uk}
\emailAdd{apostolos.pilaftsis@manchester.ac.uk}

\abstract{We present a class of leptogenesis models where the light neutrinos acquire their observed mass through a symmetry-motivated construction. We consider an extension of the Standard Model, which includes three singlet neutrinos which have mass splittings comparable to their decay widths. We show that this tri-resonant structure leads to an appreciable increase in the observed CP asymmetry over that found previously in typical bi-resonant models. To analyse such tri-resonant scenarios, we solve a set of coupled Boltzmann equations, crucially preserving the variations in the relativistic degrees of freedom. We highlight the fact that small variations at high temperatures can have major implications for the evolution of the baryon asymmetry when the singlet neutrino mass scale is below $100 \; \GeV$. We then illustrate how this variation can significantly affect the ability to find successful leptogenesis at these low masses. Finally, the parameter space for viable leptogenesis is delineated, and comparisons are made with current and future experiments.}

\FullConference{Corfu Summer Institute 2022 ``School and Workshops on Elementary Particle Physics and Gravity'', August 28 - September 8, 2022, Corfu, Greece.}



\begin{document}
%
\maketitle

%\newpage
%
\section{Introduction}\label{sec:Intro}
% Importance and appeal of children's drawings
Children's depictions of the human figure are highly expressive and varied.
As one of the very first subjects children attempt to draw, the representation begins as an almost unintelligible cloud of scribbles. 
As the child grows, their representation of the human figure becomes more developed and is extended to graphically represent many different types of characters: people, animals, and even personified objects (see Figure 1).

Who among us has not wished, either as a child or as an adult, to see such figures come to life and move around on the page?
Sadly, while it is relatively fast to produce a single drawing, creating the sequence of images necessary for animation is a much more tedious endeavor, requiring discipline, skill, patience, and sometimes complicated software.
As a result, most of these figures remain static upon the page.

% We built a system to animate them.
Inspired by the importance and appeal of the drawn human figure, we design and build a system to automatically animate it given an in-the-wild photograph of a child's drawing. 
Our system is fast, intuitive, and robust to much of the variation present in these types of drawings, making it well-suited to allow our target audience--children--to see their own characters coming to life.
The system is comprised of four stages: figure detection, segmentation masking, pose estimation/rigging, and animation. 
We describe each stage and identify common causes of failure in each. 
For object detection and pose estimation, we make use of existing computer vision models designed to detect human figures and joints in photographs; we fine-tune these models for use with children's drawings.
For segmentation, we present a straightforward, image processing-based method that, for animation purposes, is more useful and accurate than segmentation masks obtained from a fine-tuned object detection model.
During the animation step, we take advantage of the \textit{twisted perspective} commonly seen in children’s drawings to retarget motion capture data onto the character in a novel and appealing way.

% We use existing machine learning models. However, given the wide domain gap it's not clear how much fine-tuning data was needed. So we ran some experiments to find out and report it.
While our system leverages existing models and techniques, most are not directly applicable to the task due to the many differences between photographic images and simple pen and paper representations. 
To this end, we couple the presentation of our system with a set of experiments exploring the relationship between fine-tuning training set size and success rates.
We also include a perceptual study validating viewer preference for incorporating \textit{twisted perspective} into the motion retargeting step.

We validate the desirability and appeal of our system by building and publicly releasing a version of it as the \AD Demo \,\cite{animateddrawings}.
Launched in December 2021, this demo has been used by millions of people around the world to animate their children's drawings.
Inspired by this reception, our second contribution is The Amateur Drawings Dataset: \hjs{180,000 drawings and user-accepted annotations collected, with consent, through the demo. See Section \ref{sec:UI} for a description of how the annotations were generated.}
We believe this dataset will be a resource to researchers from various fields seeking to better understand the space of amateur drawings, evaluate new algorithms in this domain, or develop new drawing-based tools in general.

To summarize, our contributions are as follows:
\begin{enumerate}
    \item 
    We explore the problem of automatic sketch-to-animation for children's drawings of human figures and present a framework that achieves this effect. We also present a set of experiments determining the amount of training data necessary to achieve high levels of success and a perceptual study validating the usefulness of our motion retargeting technique.
    \item To encourage additional research in the domain of amateur drawings, we present a first-of-its-kind dataset of 180,000 user-submitted amateur drawings, along with user-accepted bounding box, segmentation mask, and joint location annotations.
\end{enumerate}

Upon acceptance of this paper, we plan to publicly release the Amateur Drawings Dataset, project code, and fine-tuned model weights.


\section{Flavour Symmetric Model}\label{sec:Model}
We utilise a minimal extension of the SM, with the inclusion of three right-handed neutrinos, which are singlets of the SM gauge group: $\textrm{SU}(3)_c \times \textrm{SU}(2)_L \times \textrm{U}(1)_Y$, and have lepton number $\textrm{L} = 1$. Given this additional particle content and quantum number assignment, the SM is extended through the additional Lagrangian terms
\begin{align}
    \mathcal{L}_{\nu_R} =i\overline{\nu}_R\slashed{\partial}\nu_R -\left( \overline{L}\,\h^\nu\tilde{\Phi}\,\nu_R + \frac{1}{2}\overline{\nu}^C_R\,\m_M\nu_R + {\rm H.c.\, }\right)\label{eq:seesaw_lagrangian}.
\end{align}
Here, $L_i = \left(\nu_{L, i}, e_{L, i} \right)^{\sf T}$, with $i=1,2,3$, are left-handed lepton doublets; $\nu_{R, \alpha}$, with $\alpha=1,2,3$, are right-handed neutrino singlet fields; and $\tilde{\Phi}$ is the isospin conjugate Higgs doublet. The matrices $\mathbf{h}_{i\alpha}^\nu$ and $(\mathbf{m}_M)_{\alpha\beta}$ are the neutrino Yukawa couplings and Majorana mass matrix, respectively. It is worth pointing out that the inclusion of the Majorana mass matrix explicitly breaks lepton number conservation by two units, $\Delta L = 2$, satisfying one of the three Sakharov conditions for the generations of appreciable lepton asymmetry.

Without loss of generality, we may select a basis for the singlet neutrino sector such that the Majorana mass term is diagonalised, \textit{i.e} $\m_M = \textrm{diag}(m_{N_1}, m_{N_2}, m_{N_3})$. In this basis, the Lagrangian in the unbroken phase takes the form
\begin{align}
\mathcal{L}_{\nu_R} =i\overline{N}\slashed{\partial}N-\left( \overline{L}\,\h^\nu\tilde{\Phi}\,P_R N + {\rm H.c.}\right) - \frac{1}{2}\overline{N}\,\m_MN.\label{eq:seesaw_lagrangian_mass}
\end{align}
In this expression, $N_\alpha = \nu_{R, \alpha} + \nu_{R, \alpha}^C$, and $P_{R/L} = \frac{1}{2}\left(\mathds{1}_4 \pm \gamma^5 \right)$ are right/left-chiral projection operators.

In the broken phase, the addition of a Dirac mass term from the Yukawa sector results in the mixing between left- and right-chiral neutrinos, with the mass basis in the broken phase a particular combination of left- and right-chiral neutrinos
\begin{align}
P_R \begin{pmatrix}
\nu \\
N
\end{pmatrix}=
\begin{pmatrix}
U_{\nu\nu_L^C} & U_{\nu \nu_R}\\
U_{N\nu_L^C} & U_{N \nu_R}
\end{pmatrix}
\begin{pmatrix}
\nu_L^C\\
\nu_R
\end{pmatrix}\;.
\end{align}
In the above, we have defined $\nu_i$ as light neutrino mass eigenstates and $N_i$ as heavy neutrino mass eigenstates. Furthermore, the unitary matrix, $U$, diagonalises the full neutrino mass matrix. To leading order in the quantity $\xi_{i\alpha} = (\mathbf{m}_D \mathbf{m}_M^{-1})_{i\alpha}$~\cite{Pilaftsis:1991ug}, the light neutrino mass matrix may be written as
\begin{align}
\m^\nu = -\m_D \m^{-1}_M \m^{\sf T}_D \; ,\label{eq:tree_level_mass}
\end{align}
with $\m_D = \mathbf{h}^\nu v /\sqrt{2}$ the Dirac mass matrix, with Higgs VEV, $v \simeq 246\; \GeV$~\cite{GellMann:1980vs}. By virtue of this relation, it is clear that to satisfy observed neutrino data, the Majorana mass matrix would have to be GUT scale if the Dirac matrix is of electroweak scale ($|\!|\m_D|\!| \sim v$), and of general structure. As a result, the impact of singlet neutrinos on experimental signatures would be minimal as the charged current interactions are suppressed through the mixing parameter $B_{i\alpha} = \xi_{i\alpha}$~\cite{Pilaftsis:1991ug}
\begin{align}
\mathcal{L}^W_{\rm int} = -\frac{g_w}{\sqrt{2}}W^-_\mu
  \overline{e}_{iL} B_{i\alpha}\gamma^\mu P_L N_\alpha + {\rm H.c.}\,.
\end{align}
Consequently, there is a motivation to identify models which allow for low-scale heavy neutrino masses whilst remaining in alignment with the observed neutrino data.

One approach which may be taken to address this problem is to assume the existence of a symmetry on the flavour structure of the Yukawa couplings, $\mathbf{h}_0^\nu$, which would render the light neutrino eigenstates massless
\begin{equation}
    -\m_D \m^{-1}_M \m^{\sf T}_D = -\frac{v^2}{2}\mathbf{h}_0^\nu \m^{-1}_M (\mathbf{h}_0^\nu)^{\sf T} = \mathbf{0}_3.\label{eq:zeromassconstraint}
\end{equation}
From this, small neutrino masses may be generated through perturbations about the symmetric Yukawa couplings
\begin{align}
    (\h^\nu_0 + \delta\h^\nu)\,\m_M^{-1}\,(\h^\nu_0 + \delta\h^\nu)^{\sf
  T}\, =\, \frac{2}{v^2}\,\m^\nu\;. 
    \label{eq:numassconstraint}
\end{align}

In the case of a near degenerate heavy neutrino mass spectrum, the condition on the symmetric Yukawa couplings given in (\ref{eq:zeromassconstraint}) may be approximately satisfied by
\begin{equation}
    \mathbf{h}_0^\nu (\mathbf{h}_0^\nu)^{\sf T} =\mathbf{0}_3.\label{eq:NilPotent}
\end{equation}
This motivates a nil-potent structure of the Yukawa matrix. In particular, we have identified the structure
\begin{align}
    \h^\nu_0=\begin{pmatrix}
        a & a\,\omega & a\,\omega^2\\
        b & b\,\omega & b\,\omega^2\\
        c & c\,\omega & c\,\omega^2
    \end{pmatrix}\;,
    \label{eq:z6yukawa}
\end{align}
with $a, \, b, \,c \in \mathbb{C}$, and $\omega = \exp\left( \frac{2\pi i}{6} \right)$ the generator of the $\mathbb{Z}_6$ group. This structure is not unique in satisfying the constraint given in (\ref{eq:NilPotent}). Other similar structures, such as $\mathbb{Z}_3$, with generators $\omega^\prime = \exp\left( \frac{2\pi i}{3} \right)$ would also produce a vanishing light neutrino mass spectrum at leading order. Most interestingly, this symmetry-motivated structure offers large CP-violating phases which contribute significantly to the generation of appreciable BAU.
Instead, this possibility is not easily achievable in bi-resonant models, where the CP-odd phases are strongly correlated to the light-neutrino masses.

As a further insight, since this symmetry exists within the flavour structure, any additional contributions to the light neutrino mass matrix with an identical flavour structure will vanish. In particular, the first-order loop correction to the light neutrino mass matrix~\cite{Pilaftsis:1991ug} may be incorporated into the zero mass condition of the symmetric Yukawa matrix
\begin{align}
    \frac{v^2}{2}\h^\nu_0\left[\m^{-1}_M - \frac{\alpha_w}{16\pi M^2_W}\m^{\dagger}_Mf(\m_M\m^\dagger_M)\right]\h^{\nu\sf
  T}_0=\,\mathbf{0}_3\;,
    \label{eq:one_loop_zero_mass}
\end{align}
where
\begin{align}
    f(\m_M\m^\dagger_M)=\frac{M^2_H}{\m_M\m^\dagger_M -
M^2_H\mathds{1}_3}\ln\left(\frac{\m_M\m^\dagger_M}{M^2_H}\right) +
  \frac{3M^2_Z}{\m_M\m^\dagger_M -
M^2_Z\mathds{1}_3}\ln\left(\frac{\m_M\m^\dagger_M}{M^2_Z}\right)\;. 
    \label{eq:loop_factor_f_def}
\end{align}
In the above, $\alpha_w\equiv g_w^2/(4\pi)^2$ is the electroweak gauge-coupling parameter, and $M_W$, $M_Z$, and $M_H$ are the masses of the $W$, $Z$, and Higgs bosons, respectively.

\section{Leptonic Asymmetries}\label{sec:CPA}
In models of thermal leptogenesis, CP-violating effects enter through the difference in the decay rates of heavy neutrinos into leptons and Higgs bosons $(N \rightarrow L \Phi)$, and the conjugate process $(N \rightarrow L^c \Phi^\dagger)$~\cite{FukYan:1986,Buchmuller:2004nz}. This difference appears at the loop level, with the wavefunction contribution particularly dominant in models of RL, where mass splittings are of a similar size to the decay widths of the heavy neutrinos (for a review, see~\cite{Pilaftsis:1998pd}). To aid the discussion of analytic results regarding the leptonic CP asymmetries, we introduce the coefficients~\cite{Pilaftsis:1997jf,Pilaftsis:2003gt,Pilaftsis:2005rv}
%
\begin{align}
    A_{\alpha \beta} &= \sum_{l=1}^3 \frac{\h^\nu_{l \alpha}\h_{l
                       \beta}^{\nu*}}{16\pi} =
                       \frac{(\h^{\nu\dagger}\h^\nu)^*_{\alpha\beta}}{16\pi},\\ 
    %
    V_{l \alpha} &= \sum_{k=1}^3 \sum_{\gamma \neq \alpha}
                   \frac{\h^{\nu*}_{k\alpha}\h^\nu_{k
                   \gamma}\h^\nu_{l\gamma}}{16\pi} f\left(
                   \frac{m^2_{N_\gamma}}{m^2_{N_\alpha}} \right),
    %
    \label{eq:V_def}
\end{align}
%
which correspond to absorptive transition rates for the wavefunction and vertex, respectively. In~(\ref{eq:V_def}), ${f(x) = \sqrt{x}\left[1-(1+x)\ln \left( \frac{1+x}{x} \right) \right]}$ is the Fukugita-Yanagida 1-loop function~\cite{FukYan:1986,Buchmuller:2004nz}.

Completing a full re-summation of the loop corrections, including all three Majorana neutrinos, generates an effective $NL\tilde{\Phi}$ coupling~\cite{Pilaftsis:2003gt, Pilaftsis:2005rv, Deppisch:2010fr}
\begin{align}
\label{eq:Eff_Yuk}
    (\bar{\mathbf{h}}^\nu_+)_{l\alpha} =&\; \h^\nu_{l\alpha} +
                                          iV_{l\alpha} - i
                                          \sum_{\beta,\gamma = 1}^3
                                          |\varepsilon_{\alpha\beta\gamma}|\,\h^\nu_{l\beta}\nonumber\\&\times
  \frac{m_{N_\alpha}\left(M_{\alpha\alpha\beta}+M_{\beta\beta\alpha}\right)-i
  R_{\alpha\gamma}
  \left[M_{\alpha\gamma\beta}\left(M_{\alpha\alpha\gamma}+M_{\gamma\gamma\alpha}\right)
  +
  M_{\beta\beta\gamma}\left(M_{\alpha\gamma\alpha}+M_{\gamma\alpha\gamma}\right)\right]}{m_{N_\alpha}^2-m_{N_\beta}^2
  + 2i m^2_{N_\alpha} A_{\beta\beta} + 2i\,\Im m
  R_{\alpha\gamma}\left(m_{N_\alpha}^2 |A_{\beta\gamma}|^2 +
  m_{N_\beta} m_{N_\gamma} \Re e A_{\beta\gamma}^2 \right)}\;, 
\end{align}
%
where $\epsilon_{\alpha\beta\gamma}$ is the anti-symmetric Levi-Civita
symbol, $M_{\alpha\beta\gamma}\equiv m_{N_\alpha}A_{\beta\gamma}$ and 
%
\begin{equation}
    R_{\alpha\beta} \equiv \frac{m_{N_\alpha}^2}{m_{N_\alpha}^2-m_{N_\beta}^2+2i m_{N_\alpha}^2 A_{\beta\beta}}\;.
\end{equation}
The conjugate $NL^c\tilde{\Phi}^\dagger$ couplings, denoted by $(\bar{\mathbf{h}}^\nu_-)_{l\alpha}$, are found through the replacement of $\mathbf{h}^\nu_{l\alpha}$ by $(\mathbf{h}^\nu)^*_{l\alpha}$ in (\ref{eq:Eff_Yuk}). These effective couplings capture both \textit{bi-resonant} and \textit{tri-resonant} effects, corresponding to maximal CP asymmetries through the mixing of two and three singlet neutrinos, respectively. In particular, one may recover the bi-resonant expressions by simply taking $R_{\alpha\gamma}$ to~zero.

Utilising these re-summed effective couplings, we may calculate the partial decay widths of the heavy neutrinos as
\begin{equation}
    \Gamma (N_\alpha \rightarrow L_l \Phi) =
    \frac{m_{N_\alpha}}{8\pi}\left| (\bar{\h}^\nu_+)_{l \alpha}
    \right|^2,\qquad 
    \Gamma (N_\alpha \rightarrow L^C_l \Phi^\dagger) =
    \frac{m_{N_\alpha}}{8\pi}\left| (\bar{\mathbf{h}}^\nu_-)_{l
        \alpha} \right|^2\;. 
    \label{eq:Gammas_def}
\end{equation}

\begin{figure}[t!]
\hspace*{-0.6cm}
%\centering
    \includegraphics[width=\linewidth]{Images/delta_vs_M3.pdf}
    \caption{\textit{Left panel:} CP asymmetries generated by the decays of
      $N_1$, $N_2$ and $N_3$, together with the total CP asymmetry
      $\delta_T = \sum_\alpha \delta_\alpha$, as a function of the
      mass of $N_3$. \textit{Centre panel:} Comparison of the CP asymmetry in the decay
      of $N_2$ vs. $m_{N_3}$ as calculated from two neutrino mixing ($\delta^{(2)}_2$) and three-neutrino mixing
      ($\delta^{(3)}_2$). \textit{Right panel:} CP asymmetry in the
      decay of $N_3$ vs. $m_{N_3}$ calculated from two-neutrino mixing ($\delta^{(2)}_3$) and
      three-neutrino mixing ($\delta^{(3)}_3$). We indicate the
      values of $m_{N_1}$, $m_{N_2}$ and the tri-resonant value of
      $m_{N_3}$ with grey dashed lines.} 
    \label{fig:cp_asymmetry}
\end{figure}

From this, we identify the size of the CP asymmetries within the model using the dimensionless quantity
\begin{equation}
    \delta_{\alpha l} \equiv \frac{\Gamma (N_\alpha \rightarrow L_l
      \Phi) - \Gamma (N_\alpha \rightarrow L^C_l \Phi^\dagger) }{
      \sum_{k=e,\mu,\tau} \Gamma (N_\alpha \rightarrow L_k \Phi)
      +\Gamma (N_\alpha \rightarrow L^C_k \Phi^\dagger)} =
    \frac{\left| (\bar{\mathbf{h}}^\nu_+)_{l \alpha} \right|^2 -
      \left| (\bar{\mathbf{h}}^\nu_-)_{l \alpha}
      \right|^2}{(\bar{\mathbf{h}}^{\nu\dagger}_+\bar{\mathbf{h}}^\nu_+)_{\alpha\alpha}
      +
      (\bar{\mathbf{h}}^{\nu\dagger}_-\bar{\mathbf{h}}^\nu_-)_{\alpha\alpha}}. 
    \label{eq:CP-deltas_def}
\end{equation}
Furthermore, we may define the total CP asymmetry associated with each neutrino species by summing over the lepton families
\begin{equation}
    \delta_\alpha = \sum_{l} \delta_{\alpha l} = 
    \frac{(\bar{\mathbf{h}}^{\nu\dagger}_+\bar{\mathbf{h}}^\nu_+)_{\alpha\alpha}
      -
      (\bar{\mathbf{h}}^{\nu\dagger}_-\bar{\mathbf{h}}^\nu_-)_{\alpha\alpha}}{(\bar{\mathbf{h}}^{\nu\dagger}_+\bar{\mathbf{h}}^\nu_+)_{\alpha\alpha}
      +
      (\bar{\mathbf{h}}^{\nu\dagger}_-\bar{\mathbf{h}}^\nu_-)_{\alpha\alpha}}.
\end{equation}
At this point, it is important to mention that the existence of non-zero CP asymmetries is only possible in the event that the CP-odd invariant
\begin{align}
    \Delta_{\rm CP} &= \Im m \left\{ \textrm{Tr} \left[
                  (\mathbf{h}^\nu)^\dagger \mathbf{h}^\nu
                  \mathbf{m}_M^\dagger \mathbf{m}_M
                  \mathbf{m}_M^\dagger (\mathbf{h}^\nu)^{\sf T}
                  (\mathbf{h}^\nu)^* \mathbf{m}_M\right] \right\}\\ 
    &= \sum_{\alpha<\beta} m_{N_\alpha} m_{N_\beta}
      \left(m_{N_\alpha}^2 - m_{N_\beta}^2 \right)\, \Im m \Big[
      \left(\mathbf{h}^{\nu\dagger}\mathbf{h}^\nu
      \right)_{\beta\alpha}^2 \Big]
\end{align}
does not vanish~\cite{Pilaftsis:1997dr, Pilaftsis:2003gt,
Branco:1986gr, Yu:2020gre} When all neutrinos are exactly degenerate, this quantity is trivially zero, and hence CP asymmetries are not possible. However, if mass splittings are permitted, we see that in the $\mathbb{Z}_6$ model presented earlier, this CP-odd quantity is proportional to the $\mathbb{Z}_6$ element $\omega^2$
\begin{align}
    \Delta_{\rm CP} &\approx \left(|a|^2 + |b|^2 + |c|^2 \right)^2
                  \sum_{\alpha<\beta} m_{N_\alpha} m_{N_\beta}
                  \left(m_{N_\alpha}^2 - m_{N_\beta}^2 \right)\, \Im m
                  \Big( \omega^{2(\alpha-\beta)} \Big)\; . 
\end{align}
Accordingly, the $\mathbb{Z}_6$ structure we have proposed offers both naturally light SM neutrino masses and produce significant levels of CP asymmetry due to the large CP-violating phases present.

In the literature, there are several examples of the bi-resonant approximation being used in RL scenarios to enhance the contribution to the CP asymmetry through the mixing of two singlet neutrinos whilst permitting the third neutrino to decouple, either through suppressed couplings or a higher mass scale. However, in a model where all three neutrinos satisfy the resonance condition
\begin{equation}
    |m_{N_\alpha} - m_{N_\beta}| \sim \frac{\Gamma_{N_{\alpha,\beta}}}{2},
\end{equation}
the contributions to CP asymmetries may be enhanced through constructive interference between all three neutrinos~\cite{Pilaftsis:1997dr}. Figure~\ref{fig:cp_asymmetry} shows the variation in the generated CP asymmetry through the decay of singlet neutrinos, as well as the total CP asymmetry $\delta_T = \sum_\alpha \delta_\alpha$. In this figure, $m_{N_1}$ and $m_{N_2}$ are fixed to satisfy the resonance condition, and $m_{N_3}$ is permitted to vary. As is expected, the total CP asymmetry is seen to vanish in the case that $m_{N_3}$ is equal to either $m_{N_1}$ or $m_{N_2}$; however, is maximised when $m_{N_3} = m_{N_2} + \frac{1}{2}\Gamma_{N_2}$. This maximum of the CP asymmetry is $35\%$ larger than what can be produced in models with only two neutrino mixing. Furthermore, at this maximum, it can be seen that $\delta_1 \simeq \delta_3$, while $\delta_2$ is significantly enhanced. Consequently, $\delta_2$ is the dominant contributor to $\delta_T$.

The latter two panels in Figure~\ref{fig:cp_asymmetry} highlight the difference between two neutrino mixing, $\delta^{(2)}_\alpha$, and three neutrino mixing $\delta^{(3)}_\alpha$. It is clear from the second panel that the proper inclusion of three neutrino mixing is important in the resonant region, as a sizeable difference becomes apparent in the CP asymmetry of $N_2$. A similar effect is present in the CP asymmetry of $N_3$, shown in the final panel, although to a lesser extent.

In general, Figure~\ref{fig:cp_asymmetry} highlights the importance of full and proper accounting for the mixing of three neutrinos when these neutrinos are in consecutive resonance. As a consequence, this tri-resonant structure saturates the available CP asymmetry and maximises the generated BAU at a given mass scale with specified couplings. This is in contrast to the bi-resonant models commonly studied in the literature, which neglect contributions to the CP asymmetry from the mixing of a third neutrino species. 

\section{Boltzmann Equations}\label{sec:BEs}
The generation of appreciable BAU requires not only significant CP asymmetries but also a departure from equilibrium and baryon number violation. Here, we will introduce a complete set of Boltzmann equations which describe the out-of-equilibrium dynamics in the early universe, which allows for a dynamical generation of appreciable lepton asymmetry. This lepton asymmetry may be reprocessed into a baryon asymmetry through $(B+L)$-violating sphaleron transitions~\cite{KUZMIN198536}.

At temperature scales pertinent to leptogenesis, it is assumed that the Universe is in the radiation-domination era, with energy and entropy densities
\begin{equation}
    \rho(T) = \frac{\pi^2}{30} g_\textrm{eff}(T)\, T^4, \qquad s(T) = \frac{2\pi^2}{45} h_\textrm{eff}(T)\, T^3,
\end{equation}
respectively. Here, $T$ is the temperature of the Universe, with $g_\textrm{eff}$ and $h_\textrm{eff}$ relativistic degrees of freedom of the SM plasma. We include the variations in the relativistic degrees of freedom since these are not constant, even when the temperature is well above $100\; \GeV$. These variations are small in magnitude but may have drastic implications for the generation of appreciable BAU with low-scale neutrino masses. For our numerical simulations, we utilise the data set labelled `EOS C' provided in \cite{Hindmarsh:2005ix}.

The evolution of the neutrino and lepton asymmetry number densities are described by their respective Boltzmann equations, written as a function of the dimensionless parameter $z_\alpha = m_{N_\alpha}/T$. To align with previously used conventions, we define $z=z_1$ to be the dynamical evolution parameter. In addition, we normalise the number density of a species, $i$, to the photon density,
\begin{equation}
    n_{\gamma}(\za) = \frac{2\zeta(3)T^3}{\pi^2} = \frac{2\zeta(3)}{\pi^2} \lrb{\dfrac{\mNa}{\za}}^3.
\end{equation}
This normalisation simplifies the Boltzmann equations and relates the number density to an observable quantity,
\begin{equation}
    \eta_i(\za) = \frac{n_i(\za)}{n_\gamma(\za)}.
\end{equation}
In the case of the neutrino Boltzmann equations, it is convenient to express the evolution in terms of a departure-from-equilibrium quantity
\begin{equation}
    \delta\eta_\alpha(\za) = \frac{\eta_\alpha(\za)}{\eta_\alpha^{\textrm{eq}}(\za)} - 1.
\end{equation}
In this definition, we have used the equilibrium value of $\eta_\alpha$, which may be explicitly calculated to be
\begin{equation}
    \eta_\alpha^{\textrm{eq}} \approx \frac{\za^2}{2\zeta(3)} K_2(\za),
\end{equation}
with $\zeta(3)$ Ap\'ery's constant, and $K_n(\za)$ a modified Bessel function of the second kind.

With these considerations, we may write a set of coupled Boltzmann equations, including decay terms, $\Delta L = 1$ and $\Delta L = 2$ scattering processes, as well as the running of the degrees of freedom,
\begin{align}
    %
    \frac{d\dNa}{d \ln \za} =&-\dfrac{ \delta_h(\za)}{H(\za) \ \eta_{N_\alpha}^{\rm eq}(\za) }\lrsb{\dNa  \lrb{\Gamma^{D(\alpha)} + \Gamma^{S(\alpha)}_Y + \Gamma^{S(\alpha)}_G} +\frac{2}{9}\, \etaL\, \delta_\alpha \lrb{\tilde\Gamma^{D(\alpha)} + \hat{\Gamma}^{S(\alpha)}_Y + \hat{\Gamma}^{S(\alpha)}_G} } \nonumber\\
    & +\lrb{\dNa+1} \, \left[\za \frac{K_1(\za)}{K_2(\za)} - 3(\delta_h(\za) -1) \right]\;,
    \label{eq:BEN}\\
    %
    \frac{d\eta_L}{d \ln z} =& -  \frac{\delta_h(z)}{H(z)} \lrBiggcb {\sum_{\alpha=1}^3 \dNa \delta_\alpha \lrb{ \Gamma^{D(\alpha)} + \Gamma^{S(\alpha)}_Y + \Gamma^{S(\alpha)}_G} \nonumber \\
    & +\frac{2}{9}\eta_L \left[\sum_{\alpha=1}^3\left(\tilde\Gamma^{D(\alpha)} + \tilde{\Gamma}^{S(\alpha)}_Y + \tilde\Gamma^{S(\alpha)}_G + 
    \Gamma^{W(\alpha)}_Y + \Gamma^{W(\alpha)}_G\right) + \Gamma^{\Delta L = 2} \right] \nn\\ 
    & + \dfrac{2}{27} \etaL \, \sum_{\alpha=1}^3 \delta_\alpha^2 \, \lrb{ \Gamma^{W(\alpha)}_Y + \Gamma^{W(\alpha)}_G } }
    - 3\etaL(\delta_h(z) -1) \label{eq:BEL}\;.
\end{align}
In these equations, we have utilised the well-known expression for the temperature-dependent Hubble parameter
\begin{equation}
    H(\za) = \sqrt{\frac{4\pi^3 g_{\textrm{eff}}(\za)}{45}}\frac{m_{N_\alpha}^2}{M_{\textrm{Pl}}} \frac{1}{\za^2},
\end{equation}
with $M_{\textrm{Pl}} \approx 1.221 \times 10^{19} \; \GeV$ the Planck mass. The relevant collision terms, denoted by $\Gamma^X_Y$, are readily available in the literature \cite{Pilaftsis:2005rv}.

As presented here, the Boltzmann equations are complete up to $\Delta L = 2$ scattering processes. Moreover, the terms included take into account the subtraction of real intermediate states (RIS). Such terms may contribute negatively to the Boltzmann equations due to the lack of an on-shell contribution to the scattering amplitude.

As briefly alluded to earlier, the lepton asymmetry generated is partially re-processed into a baryon asymmetry through $(B+L)$-violating sphaleron transitions. As may be found in the literature \cite{PhysRevD.42.3344}, the generated BAU from a lepton asymmetry is given by
\begin{equation}
    \eta_B = -\frac{28}{51} \eta_L.
\end{equation}
However, sphaleron transitions become suppressed once the temperature of the Universe falls below the critical temperature $T_\textrm{sph} \approx 132 \; \GeV$. Consequently, no leptons are re-processed once the Universe cools to a temperature below $T_\textrm{sph}$. 

Moreover, observations of the BAU produce the value at the recombination epoch; however, due to the expansion of the Universe, this results in a dilution of the overall baryon asymmetry present at $T_\textrm{sph}$. To compare these two values meaningfully, we assume that as the Universe cools from $T_\textrm{sph}$ to $T_\textrm{rec}$, there are no entropy-releasing processes. Consequently, the entropy of the Universe is constant, and we may calculate the ratio
\begin{equation}
    \frac{\eta_B(T_\textrm{rec})}{\eta_B(T_\textrm{sph})} =\frac{n_\gamma(T_\textrm{sph}) s(T_\textrm{rec})}{n_\gamma(T_\textrm{rec}) s(T_\textrm{sph})} = \frac{h_{\textrm{eff}}(T_\textrm{rec})}{h_{\textrm{eff}}(T_\textrm{sph})}.
\end{equation}
For this ratio, we take the approximate value $1/27$~\cite{Pilaftsis:2003gt, Buchmuller:2004nz}, and as a result, the observed baryon asymmetry is related to the generated lepton asymmetry by
\begin{equation}
    \eta_B^\textrm{obs} = - \frac{1}{27}\frac{28}{51} \eta_L(T_\textrm{sph}).
\end{equation}

\section{Variations in the Relativistic Degrees of Freedom of the SM Plasma}
 \begin{figure}[t!]
    \centering
    \begin{subfigure}[]{0.48\textwidth}
    \includegraphics[width=1.\textwidth]{Images/etaB_negative.pdf}
    \caption{}
    \label{fig:etaB_negative}
    \end{subfigure}
    \begin{subfigure}[]{0.48\textwidth}
    \includegraphics[width=1.\textwidth]{Images/etaB_positive.pdf}
    \caption{}
    \label{fig:etaB_positive}
    \end{subfigure}
    \caption{Evolution of $\delta_{N_1}$ (red) and $\etaL$ (black) for
      $m_{N_1} = 35~\GeV$ (a) and $m_{N_1} = 45~\GeV$ (b), with
      $|\h^{\nu}_{ij}| \approx 4.5 \times 10^{-5}$. The black and red
      solid (dashed) lines indicate where $\eta_B$ and $\delta_{N_1}$ are
      positive (negative). Grey dotted lines indicate the sphaleron freeze-out value, $z_\textrm{sph}$, and the observed Baryon asymmetry at $z_\textrm{sph}$.} 
    \label{fig:etaB_sign}
\end{figure}

In this section, we consider the effect of variations in the relativistic degrees of freedom of the SM plasma. In the Boltzmann equations, we introduce this effect through the inclusion of the factor
\begin{equation}
    \delta_h(z) = 1 - \frac{1}{3}\frac{d\ln h_{\textrm{eff}}}{d \ln z},\label{eq:DoFs}
\end{equation}
which is greater than 1 and has the limiting value $1$ for constant degrees of freedom.

While the size of this quantity does not differ significantly from unity, the final term of (\ref{eq:BEN}) may be dominant for low values of $z$. Consequently, early in the evolution of $\delta\eta_{N_\alpha}$, negative values may be observed. Accordingly, the dependence of $\eta_L$ on $\delta\eta_{N_\alpha}$ will result in negative values for the baryon asymmetry, $\eta_B$. In the case that the singlet neutrino mass scale is sufficiently low, the sphaleron freeze-out at $T_\textrm{sph}$ may preserve this feature.

In Figure~\ref{fig:etaB_sign}, we can see the impact of the variations in the relativistic degrees of freedom. In these figures, the dashed lines represent regions where the relevant quantity takes on negative values, typically $z \lesssim 10^{-1}$, and solid lines represent positive values, typically $z \gtrsim 10^{-1}$. Whilst these two figures are qualitatively similar, the importance of the mass scale becomes clear when we consider the values at the sphaleron freeze-out temperature. In Figure~\ref{fig:etaB_negative}, we see that for singlet neutrinos of mass $m_{N_1} = 35 \; \GeV$, the sphaleron freeze-out occurs prior to the evolution `bouncing back' to positive values, resulting in an overall negative sign for the generated BAU. Conversely, in Figure~\ref{fig:etaB_positive}, we consider singlet neutrinos of mass $m_{N_1} = 45 \; \GeV$, the baryon asymmetry has had enough time to return to positive values before the sphaleron freeze-out, and we find the expected positive values for the generated BAU.

We highlight the impact of the variations in the relativistic degrees of freedom in Figure~\ref{fig:heff_var}. From these figures, it is clear that when we utilise models with sub-TeV masses, the variations in the relativistic degrees of freedom may result in drastically different values for the observed BAU. In particular, we call attention to the fact that for singlet neutrino masses below $100 \;\GeV$, the observed BAU may take on negative values once the variations in the degrees of freedom are accounted for.

As a concluding remark, we note that this effect is stable under perturbations of the initial conditions since the solutions of the Boltzmann equations quickly reach attractor solutions. Moreover, we expect that this behaviour would pervade even with the inclusion of additional CP-violating phenomena, such as coherent heavy neutrino oscillations.
\begin{figure}[t!]
\centering
	\begin{subfigure}[]{0.48\textwidth}
		\includegraphics[width=1.\textwidth]{Images/heff_effect.pdf}
		\caption{}
		\label{fig:heff_effect}
	\end{subfigure}
	\begin{subfigure}[]{0.48\textwidth}
		\includegraphics[width=1.\textwidth]{Images/heff_ratios.pdf}
		\caption{}
		\label{fig:heff_ratios}
	\end{subfigure}
	\centering
    \caption{\textit{Left panel:} The generated $\eta_B$ for
      $|\h^{\nu}_{ij}| \approx 3 \times 10^{-4}$ in the tri-resonant
      scenario as a function of $m_{N_1}$ for $\heff$ as given
      in~\cite{Hindmarsh:2005ix} (black),~\cite{Gondolo:1990dk}
      (blue), and taking $\heff = {\rm const.} \approx 105$ (red). The
      grey dotted line shows the observed baryon asymmetry, $\eta_B^{\rm CMB} = 6.104 \times
      10^{-10}$. \textit{Right panel:} The ratio of $|\eta_B|$ with
      varying $\heff$. The black (blue) line corresponds
      to~\cite{Hindmarsh:2005ix} (\hspace{1sp}\cite{Gondolo:1990dk}),
      with respect to $\heff = {\rm const.}$} 
    \label{fig:heff_var}
\end{figure}

\section{Results}\label{sec:Res}
\section*{Results}
We started by assembling a dataset derived from public hikes. This process included an iterative data cleaning process to remove erroneous/false data, identify and remove breaks (e.g. Fig \ref{Fig2}) to give us a final usable dataset containing 7,636 GPS tracks, with over 1.4 million individual data points and covering almost 88,000 km of travel in the U.K. 

Our curated hike dataset allowed us to create a data-driven model which we can directly compare with existing walking speed algorithms. The model formulation was selected using a small-scale exploratory study which considered data from Scotland (see \nameref{S3_Appendix}). In this exploratory study, multiple different model types were explored which could fit the data, and which matched existing knowledge about walking speeds. Cross-validation methods showed that there was very little difference in performance of the best models, therefore the final model was a Generalised Linear Model (GLM), which was chosen as it was the simplest of those tested (we had no evidence that a more complex model would be superior). This choice also meant that our model was both easy to interpret, and simple to apply to future work.

This final GLM model included all three of the variables suggested by Arnet \cite{Arnet2009ArithmeticalJapan}:

\begin{equation}
    v = exp(a+b\phi+c\theta+d\theta^2)
\end{equation}
where
\begin{quote}
$v = \text{walking speed (km/h)}$\\
$\phi = \text{hill slope angle (degrees)}$\\
$\theta = \text{walking slope angle (degrees)}$
\end{quote}

Terrain obstruction level was included as a factor variable, while we considered the road types as both factor variables and interaction terms. Not all terms had a significant effect on all variables; we therefore created a model with all possible terms, and removed them one at a time (in order of least significance) until all remaining terms were significant to at least 95\% confidence  level (using Wald test). The final values for a, b, c and d are given in Table \ref{tab:2ROUK model variable values} for each of the terrain obstruction levels and road types. The critical gradient for this model is between 14 -- 16 degrees when walking uphill and -16 -- -18 degrees when walking downhill (depending on road and obstruction conditions), which is in line with previous findings. 

Fig \ref{Fig3} shows the predicted walking speeds under different conditions. The importance of including both the hill slope and terrain obstruction variables can be clearly seen when looking at the Off Road Light Obstruction speed predictions. When directly ascending or descending a slope, the walking speed is comparable to walking on a road. However, when traversing a slope while off road, the walking speed is comparable to traversing a slope of double the gradient while on a road or path. Similarly, comparing the walking speed predictions of Off Road Light Obstruction and Off Road Heavy Obstruction reveals that just 10 cm of vegetation (our cutoff point for heavy obstruction) can reduce the walking speed by more than 0.5 km/h.

\begin{table}[!ht]
\begin{adjustwidth}{-0.5in}{0in}
    \centering
    \caption{Final walking speed model variable coefficients}
    \begin{tabular}{|l+c|c|c|c|}
    \hline
    & $a$ & $b$ & $c$  & $d$ \\ 
    \thickhline
    Paved road & 1.580 & -0.00389 & -0.00726 & -0.00218 \\ 
    \hline
    Unpaved road & 1.580 & -0.00389 & -0.00965 & -0.00248 \\
    \hline
    Off-road (obstruction unknown) & 1.536 & -0.00731 & -0.00965 & -0.00187 \\
    \hline
    Off-road (light obstruction) & 1.580 & -0.00731 & -0.00965 & -0.00187 \\ 
    \hline
    Off-road (heavy obstruction) & 1.400 & -0.00731 & -0.00965 & -0.00187 \\ 
    \hline
    \end{tabular}
    \label{tab:2ROUK model variable values}
\end{adjustwidth}
\end{table}

\begin{figure}[!h]
\begin{adjustwidth}{-2.25in}{0in} 
    \includegraphics[width=\linewidth]{Images/Paper/Fig3.eps}
    \captionsetup{width=1\linewidth}
    \caption[width=\textwidth]{{\bf Walking speed predictions under different terrain conditions.}  When: (A) travelling directly up or down hills of varying slope, (B) traversing across hills of varying slope.}
    \label{Fig3}
    \end{adjustwidth}
\end{figure}

Fig \ref{Fig4} compares the Paved Road and Off Road Heavy Obstruction speed predictions from our model against the existing functions from Naismith, Tobler and Campbell et al. When looking at the walking slope, the largest areas of deviation between our model and Naismith's rule occurs when descending a slope, as Naismith's rule does not predict a reduced speed in this scenario. For both Tobler's and Campbell et al.'s functions, the shape of the walking slope component is relatively similar to our new model, with the main distinction being the peak predicted speed on flat ground. None of the existing functions account for the hill slope, which leads to large disparities when predicting the walking speed for slope traversals. A further example of this can be seen in \nameref{S6_Appendix}, which shows the walking speeds for a simulated off-road route which encounters the full range of hill and walking slopes.

\begin{figure}[!h]
\begin{adjustwidth}{-2.25in}{0in} 
    \includegraphics[width=\linewidth]{Images/Paper/Fig4.eps}
    \captionsetup{width=1\linewidth}
    \caption[width=\textwidth]{{\bf Comparison of new model and existing hiking functions.}  Predicted walking speeds of the new model, Naismith's rule, Tobler's function and Campbell et al.'s function when: (A, C, E) travelling directly up or down hills of varying slope, (B, D, F) traversing across hills of varying slope.}
    \label{Fig4}
\end{adjustwidth}
\end{figure}

When comparing the performances of each of the models (Table \ref{tab:2comparison}), the predicted speeds for individual 50 m sections had a lower RMSE and percentage error, and a higher R squared value using our new model than in the existing ones. To isolate the impact of each of the slope variables, we filtered the results to look at the data where a slope was being directly climbed or traversed. Figs \ref{Fig5}A, B and \ref{Fig6}A, B show the RMSE and mean residuals for each of the models, for data which was within 5 degrees of directly climbing (A) or traversing (B) hills of varying slope. From this we can clearly see that Naismith's rule consistently overestimates walking speeds when descending a slope, and underestimates speeds when climbing a slope. When ascending or descending a slope, the RMSE of our GLM is similar to that of Tobler's hiking function. However, one of the main areas where we see an improvement using our model is on slight declines. Tobler's hiking function suggests that walking speed increases on mild descents up to a maximum of 6 km/h. It is clear from Fig \ref{Fig5}A, that Tobler's function overestimates the walking speed in this region. Campbell et al.'s function has a slightly lower RMSE value than our new model on the steepest walking slopes, however it underestimates the walking speeds on flat ground and mild slopes. Previous research has found that most walking takes place on low walking slopes \cite{Proffitt1995PerceivingSlant}, and this is evidenced by our data ($\sim$98\% of our data was from walking slopes of under 10 degrees). Improved walking speed predictions in this region therefore have the greatest impact in real-world situations. Within this region our model consistently has a lower RMSE than the existing functions, and a mean residual error close to 0 km/h. 

\begin{table}[!ht]
\centering
\caption{Comparison of new model against existing methods to calculate walking speeds.}
\begin{tabular}{|l|c|c|c|c|}
\hline
& New Model & Naismith & Tobler & Campbell\\
\hline
Average \% error & 23.68 & 26.36 & 26.17 & 25.33\\
\hline
MSE & 1.20 & 1.61 & 1.53 & 1.58\\
\hline
RMSE & 1.10 & 1.27 & 1.24 & 1.26\\
\hline
R\textsuperscript{2}  & 0.09 & -0.22 & -0.16 & -0.19\\
\hline
\end{tabular}
\label{tab:2comparison}  
\end{table}

\begin{figure}[!h]    
\begin{adjustwidth}{-2.25in}{0in} 
    \includegraphics[width=\linewidth]{Images/Paper/Fig5.eps}
    \captionsetup{width=1\linewidth}
    \caption[width=\textwidth]{{\bf Comparing RMSE values for the new model, Naismith's rule, Tobler's function and Campbell et al.'s function.} When: (A) travelling directly up or down hills of varying slope (all data), (B) traversing across hills of varying slope (all data), (C) travelling directly up or down hills of varying slope (off-road data only), (D) traversing across hills of varying slope (off-road data only). Campbell et al.'s function does not provide off-road speed estimates, so was not included in the off-road data comparisons.}
    \label{Fig5}
\end{adjustwidth}
\end{figure}

\begin{figure}[!h]
    \begin{adjustwidth}{-2.25in}{0in} 
    \includegraphics[width=\linewidth]{Images/Paper/Fig6.eps}
    \captionsetup{width=1\linewidth}
    \caption[width=\textwidth]{{\bf Comparing mean residual values for the new model, Naismith's rule, Tobler's function and Campbell et al.'s function.} When: (A) travelling directly up or down hills of varying slope, (B) traversing across hills of varying slope, (C)  travelling directly up or down hills of varying slope (off-road data only), (D) traversing across hills of varying slope (off-road data only). Campbell et al.'s function does not provide off-road speed estimates, so was not included in the off-road data comparisons.}
    \label{Fig6}
\end{adjustwidth}
\end{figure}

 We also see an improvement in RMSE when using our model to predict speeds for hill traversals (Fig \ref{Fig5}B). We can note from Fig \ref{Fig6}B that both Naismith's rule and Tobler's hiking function consistently overestimate the walking speed when traversing a slope, as they do not take into account the impact that the hill slope has on reducing walking speeds. The performance of Campbell et al's model improves as the hill slope increases, although we suggest this is more due to it underestimating the speed on shallow slopes. We do see that the average error in our model increases as the hill slope increases, but we believe that this is due to limited volumes of data at high hill slopes ($\sim$0.5\% of our data occurs on hill slopes steeper than 40 degrees). 

As well as looking at the overall performance of our new model, we looked to explore how well our model performed in off-road conditions, compared to the off-road adjustments for the existing functions (Naismith's reduced base speed of 4 km/h, and Tobler's correction factor of 0.6). Figs \ref{Fig5}C, D and \ref{Fig6}C, D show the RMSE and mean residuals, only considering data which was recorded in off-road conditions. From Figs \ref{Fig5}C and \ref{Fig6}C it is clear that Tobler's function consistently underestimates the walking speed when off-road. The factor of 0.6 is a larger reduction in walking speed than is observed in practice. As we found when looking at our data as a whole, Naismith's rule underestimates the walking speed when climbing a slope and overestimates when descending a slope. Our new model does not suffer from these problems, with both a lower RMSE and lower absolute mean residual value across all walking slopes. Both of these existing models also consistently underestimate walking speeds when traversing a slope, unlike our new model which has a mean residual of less than 0.4 km/h on slopes of up to 35 degrees. The error in predictions of our new model does increase as the hill slope increases, though the RMSE is generally lower than seen in the existing models. On the steepest hill slopes our model appears to perform less well than the existing ones, though only 0.2\% of our off-road data occurred on a hill slope steeper than 40 degrees. 

Although we have shown an improvement in walking speed predictions over short sections of routes, this did not translate to similar results when looking at predicted walking times for routes as a whole. Our model and all of the existing models which we have explored here had an average percentage error of 13.5\% - 15.5\% when predicting the time taken for a complete route. However, based on the errors seen in Figs \ref{Fig5} and \ref{Fig6}, we believe that this is a result of errors cancelling out over the course of a hike. For example while ascending a hill, Naismith's rule will underestimate the walking speed (and thus overestimate the walking time), but it will then overestimate the walking speed on the subsequent descent, leading to a relatively accurate total time estimate. The results here suggest that Naismith's rule, and other existing functions, are still a good rule of thumb to calculate route times as a whole, but time estimates for individual sections of a route will be less accurate than when using the new model found here.




\section{Conclusions}
\section{Conclusions}
\label{sec:conclusions}

We have demonstrated that the fraction of negative event weights in
existing large high-multiplicity samples can be reduced by more than
an order of magnitude, whilst preserving predictions for observables
within statistical uncertainties. Concretely, we have employed the cell
resampling method proposed in~\cite{Andersen:2021mvw} with NLO event
samples for Z boson production with up to three jets
and W boson production with five jets produced with \textsc{Sherpa}
and \textsc{BlackHat}.

For the first time, cell resampling has been applied to samples with
up to several billions of events. This was made possible by
algorithmic improvements leading to a speed-up by several orders of
magnitude. Our updated implementation can be retreived from
\url{https://cres.hepforge.org/}.

The advances in the development of the cell resampling method
presented in this work pave the way for future applications to processes with
high-multiplicities, in particular including parton showered
predictions. It will be necessary to quantify the uncertainty
introduced by the weight smearing. Variations in the maximum cell size
parameter and different prescriptions for weight redistribution within
a cell can serve as handles to assess this uncertainty. Another
promising avenue for further exploration is the analysis of the
information on weight distribution within phase space collected during
cell resampling. Regions with insufficient Monte Carlo statistics
could be identified by their accumulated negative weight, thereby
guiding the event generation. We leave the investigation of these
questions to future work.

\section*{Acknowledgements}

AM thanks Zahari Kassabov for encouragement to reconsider the use of nearest
neighbour search trees. The work of JRA and DM is supported by the STFC under
grant ST/P001246/1.

%%% Local Variables:
%%% mode: latex
%%% TeX-master: "main"
%%% End:


\section{Acknowledgements}
\noindent The work of AP and DK is supported in part by the Lancaster-Manchester-Sheffield Consortium for Fundamental Physics under STFC Research Grant ST/T001038/1. The work of PCdS is funded by Agencia Nacional de Investigaci\'{o}n y Desarrollo (ANID) through the Becas Chile Scholarship No. 72190359. TM acknowledges support from the STFC Doctoral Training Partnership under STFC training grant 
 ST/V506898/1.


\bibliography{bibs-refs}{}
\bibliographystyle{JHEP}
\end{document}