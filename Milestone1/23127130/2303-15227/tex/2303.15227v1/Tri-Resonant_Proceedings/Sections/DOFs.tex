 \begin{figure}[t!]
    \centering
    \begin{subfigure}[]{0.48\textwidth}
    \includegraphics[width=1.\textwidth]{Images/etaB_negative.pdf}
    \caption{}
    \label{fig:etaB_negative}
    \end{subfigure}
    \begin{subfigure}[]{0.48\textwidth}
    \includegraphics[width=1.\textwidth]{Images/etaB_positive.pdf}
    \caption{}
    \label{fig:etaB_positive}
    \end{subfigure}
    \caption{Evolution of $\delta_{N_1}$ (red) and $\etaL$ (black) for
      $m_{N_1} = 35~\GeV$ (a) and $m_{N_1} = 45~\GeV$ (b), with
      $|\h^{\nu}_{ij}| \approx 4.5 \times 10^{-5}$. The black and red
      solid (dashed) lines indicate where $\eta_B$ and $\delta_{N_1}$ are
      positive (negative). Grey dotted lines indicate the sphaleron freeze-out value, $z_\textrm{sph}$, and the observed Baryon asymmetry at $z_\textrm{sph}$.} 
    \label{fig:etaB_sign}
\end{figure}

In this section, we consider the effect of variations in the relativistic degrees of freedom of the SM plasma. In the Boltzmann equations, we introduce this effect through the inclusion of the factor
\begin{equation}
    \delta_h(z) = 1 - \frac{1}{3}\frac{d\ln h_{\textrm{eff}}}{d \ln z},\label{eq:DoFs}
\end{equation}
which is greater than 1 and has the limiting value $1$ for constant degrees of freedom.

While the size of this quantity does not differ significantly from unity, the final term of (\ref{eq:BEN}) may be dominant for low values of $z$. Consequently, early in the evolution of $\delta\eta_{N_\alpha}$, negative values may be observed. Accordingly, the dependence of $\eta_L$ on $\delta\eta_{N_\alpha}$ will result in negative values for the baryon asymmetry, $\eta_B$. In the case that the singlet neutrino mass scale is sufficiently low, the sphaleron freeze-out at $T_\textrm{sph}$ may preserve this feature.

In Figure~\ref{fig:etaB_sign}, we can see the impact of the variations in the relativistic degrees of freedom. In these figures, the dashed lines represent regions where the relevant quantity takes on negative values, typically $z \lesssim 10^{-1}$, and solid lines represent positive values, typically $z \gtrsim 10^{-1}$. Whilst these two figures are qualitatively similar, the importance of the mass scale becomes clear when we consider the values at the sphaleron freeze-out temperature. In Figure~\ref{fig:etaB_negative}, we see that for singlet neutrinos of mass $m_{N_1} = 35 \; \GeV$, the sphaleron freeze-out occurs prior to the evolution `bouncing back' to positive values, resulting in an overall negative sign for the generated BAU. Conversely, in Figure~\ref{fig:etaB_positive}, we consider singlet neutrinos of mass $m_{N_1} = 45 \; \GeV$, the baryon asymmetry has had enough time to return to positive values before the sphaleron freeze-out, and we find the expected positive values for the generated BAU.

We highlight the impact of the variations in the relativistic degrees of freedom in Figure~\ref{fig:heff_var}. From these figures, it is clear that when we utilise models with sub-TeV masses, the variations in the relativistic degrees of freedom may result in drastically different values for the observed BAU. In particular, we call attention to the fact that for singlet neutrino masses below $100 \;\GeV$, the observed BAU may take on negative values once the variations in the degrees of freedom are accounted for.

As a concluding remark, we note that this effect is stable under perturbations of the initial conditions since the solutions of the Boltzmann equations quickly reach attractor solutions. Moreover, we expect that this behaviour would pervade even with the inclusion of additional CP-violating phenomena, such as coherent heavy neutrino oscillations.
\begin{figure}[t!]
\centering
	\begin{subfigure}[]{0.48\textwidth}
		\includegraphics[width=1.\textwidth]{Images/heff_effect.pdf}
		\caption{}
		\label{fig:heff_effect}
	\end{subfigure}
	\begin{subfigure}[]{0.48\textwidth}
		\includegraphics[width=1.\textwidth]{Images/heff_ratios.pdf}
		\caption{}
		\label{fig:heff_ratios}
	\end{subfigure}
	\centering
    \caption{\textit{Left panel:} The generated $\eta_B$ for
      $|\h^{\nu}_{ij}| \approx 3 \times 10^{-4}$ in the tri-resonant
      scenario as a function of $m_{N_1}$ for $\heff$ as given
      in~\cite{Hindmarsh:2005ix} (black),~\cite{Gondolo:1990dk}
      (blue), and taking $\heff = {\rm const.} \approx 105$ (red). The
      grey dotted line shows the observed baryon asymmetry, $\eta_B^{\rm CMB} = 6.104 \times
      10^{-10}$. \textit{Right panel:} The ratio of $|\eta_B|$ with
      varying $\heff$. The black (blue) line corresponds
      to~\cite{Hindmarsh:2005ix} (\hspace{1sp}\cite{Gondolo:1990dk}),
      with respect to $\heff = {\rm const.}$} 
    \label{fig:heff_var}
\end{figure}