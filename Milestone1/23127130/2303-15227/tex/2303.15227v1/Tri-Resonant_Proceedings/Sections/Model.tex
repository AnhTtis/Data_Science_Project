We utilise a minimal extension of the SM, with the inclusion of three right-handed neutrinos, which are singlets of the SM gauge group: $\textrm{SU}(3)_c \times \textrm{SU}(2)_L \times \textrm{U}(1)_Y$, and have lepton number $\textrm{L} = 1$. Given this additional particle content and quantum number assignment, the SM is extended through the additional Lagrangian terms
\begin{align}
    \mathcal{L}_{\nu_R} =i\overline{\nu}_R\slashed{\partial}\nu_R -\left( \overline{L}\,\h^\nu\tilde{\Phi}\,\nu_R + \frac{1}{2}\overline{\nu}^C_R\,\m_M\nu_R + {\rm H.c.\, }\right)\label{eq:seesaw_lagrangian}.
\end{align}
Here, $L_i = \left(\nu_{L, i}, e_{L, i} \right)^{\sf T}$, with $i=1,2,3$, are left-handed lepton doublets; $\nu_{R, \alpha}$, with $\alpha=1,2,3$, are right-handed neutrino singlet fields; and $\tilde{\Phi}$ is the isospin conjugate Higgs doublet. The matrices $\mathbf{h}_{i\alpha}^\nu$ and $(\mathbf{m}_M)_{\alpha\beta}$ are the neutrino Yukawa couplings and Majorana mass matrix, respectively. It is worth pointing out that the inclusion of the Majorana mass matrix explicitly breaks lepton number conservation by two units, $\Delta L = 2$, satisfying one of the three Sakharov conditions for the generations of appreciable lepton asymmetry.

Without loss of generality, we may select a basis for the singlet neutrino sector such that the Majorana mass term is diagonalised, \textit{i.e} $\m_M = \textrm{diag}(m_{N_1}, m_{N_2}, m_{N_3})$. In this basis, the Lagrangian in the unbroken phase takes the form
\begin{align}
\mathcal{L}_{\nu_R} =i\overline{N}\slashed{\partial}N-\left( \overline{L}\,\h^\nu\tilde{\Phi}\,P_R N + {\rm H.c.}\right) - \frac{1}{2}\overline{N}\,\m_MN.\label{eq:seesaw_lagrangian_mass}
\end{align}
In this expression, $N_\alpha = \nu_{R, \alpha} + \nu_{R, \alpha}^C$, and $P_{R/L} = \frac{1}{2}\left(\mathds{1}_4 \pm \gamma^5 \right)$ are right/left-chiral projection operators.

In the broken phase, the addition of a Dirac mass term from the Yukawa sector results in the mixing between left- and right-chiral neutrinos, with the mass basis in the broken phase a particular combination of left- and right-chiral neutrinos
\begin{align}
P_R \begin{pmatrix}
\nu \\
N
\end{pmatrix}=
\begin{pmatrix}
U_{\nu\nu_L^C} & U_{\nu \nu_R}\\
U_{N\nu_L^C} & U_{N \nu_R}
\end{pmatrix}
\begin{pmatrix}
\nu_L^C\\
\nu_R
\end{pmatrix}\;.
\end{align}
In the above, we have defined $\nu_i$ as light neutrino mass eigenstates and $N_i$ as heavy neutrino mass eigenstates. Furthermore, the unitary matrix, $U$, diagonalises the full neutrino mass matrix. To leading order in the quantity $\xi_{i\alpha} = (\mathbf{m}_D \mathbf{m}_M^{-1})_{i\alpha}$~\cite{Pilaftsis:1991ug}, the light neutrino mass matrix may be written as
\begin{align}
\m^\nu = -\m_D \m^{-1}_M \m^{\sf T}_D \; ,\label{eq:tree_level_mass}
\end{align}
with $\m_D = \mathbf{h}^\nu v /\sqrt{2}$ the Dirac mass matrix, with Higgs VEV, $v \simeq 246\; \GeV$~\cite{GellMann:1980vs}. By virtue of this relation, it is clear that to satisfy observed neutrino data, the Majorana mass matrix would have to be GUT scale if the Dirac matrix is of electroweak scale ($|\!|\m_D|\!| \sim v$), and of general structure. As a result, the impact of singlet neutrinos on experimental signatures would be minimal as the charged current interactions are suppressed through the mixing parameter $B_{i\alpha} = \xi_{i\alpha}$~\cite{Pilaftsis:1991ug}
\begin{align}
\mathcal{L}^W_{\rm int} = -\frac{g_w}{\sqrt{2}}W^-_\mu
  \overline{e}_{iL} B_{i\alpha}\gamma^\mu P_L N_\alpha + {\rm H.c.}\,.
\end{align}
Consequently, there is a motivation to identify models which allow for low-scale heavy neutrino masses whilst remaining in alignment with the observed neutrino data.

One approach which may be taken to address this problem is to assume the existence of a symmetry on the flavour structure of the Yukawa couplings, $\mathbf{h}_0^\nu$, which would render the light neutrino eigenstates massless
\begin{equation}
    -\m_D \m^{-1}_M \m^{\sf T}_D = -\frac{v^2}{2}\mathbf{h}_0^\nu \m^{-1}_M (\mathbf{h}_0^\nu)^{\sf T} = \mathbf{0}_3.\label{eq:zeromassconstraint}
\end{equation}
From this, small neutrino masses may be generated through perturbations about the symmetric Yukawa couplings
\begin{align}
    (\h^\nu_0 + \delta\h^\nu)\,\m_M^{-1}\,(\h^\nu_0 + \delta\h^\nu)^{\sf
  T}\, =\, \frac{2}{v^2}\,\m^\nu\;. 
    \label{eq:numassconstraint}
\end{align}

In the case of a near degenerate heavy neutrino mass spectrum, the condition on the symmetric Yukawa couplings given in (\ref{eq:zeromassconstraint}) may be approximately satisfied by
\begin{equation}
    \mathbf{h}_0^\nu (\mathbf{h}_0^\nu)^{\sf T} =\mathbf{0}_3.\label{eq:NilPotent}
\end{equation}
This motivates a nil-potent structure of the Yukawa matrix. In particular, we have identified the structure
\begin{align}
    \h^\nu_0=\begin{pmatrix}
        a & a\,\omega & a\,\omega^2\\
        b & b\,\omega & b\,\omega^2\\
        c & c\,\omega & c\,\omega^2
    \end{pmatrix}\;,
    \label{eq:z6yukawa}
\end{align}
with $a, \, b, \,c \in \mathbb{C}$, and $\omega = \exp\left( \frac{2\pi i}{6} \right)$ the generator of the $\mathbb{Z}_6$ group. This structure is not unique in satisfying the constraint given in (\ref{eq:NilPotent}). Other similar structures, such as $\mathbb{Z}_3$, with generators $\omega^\prime = \exp\left( \frac{2\pi i}{3} \right)$ would also produce a vanishing light neutrino mass spectrum at leading order. Most interestingly, this symmetry-motivated structure offers large CP-violating phases which contribute significantly to the generation of appreciable BAU.
Instead, this possibility is not easily achievable in bi-resonant models, where the CP-odd phases are strongly correlated to the light-neutrino masses.

As a further insight, since this symmetry exists within the flavour structure, any additional contributions to the light neutrino mass matrix with an identical flavour structure will vanish. In particular, the first-order loop correction to the light neutrino mass matrix~\cite{Pilaftsis:1991ug} may be incorporated into the zero mass condition of the symmetric Yukawa matrix
\begin{align}
    \frac{v^2}{2}\h^\nu_0\left[\m^{-1}_M - \frac{\alpha_w}{16\pi M^2_W}\m^{\dagger}_Mf(\m_M\m^\dagger_M)\right]\h^{\nu\sf
  T}_0=\,\mathbf{0}_3\;,
    \label{eq:one_loop_zero_mass}
\end{align}
where
\begin{align}
    f(\m_M\m^\dagger_M)=\frac{M^2_H}{\m_M\m^\dagger_M -
M^2_H\mathds{1}_3}\ln\left(\frac{\m_M\m^\dagger_M}{M^2_H}\right) +
  \frac{3M^2_Z}{\m_M\m^\dagger_M -
M^2_Z\mathds{1}_3}\ln\left(\frac{\m_M\m^\dagger_M}{M^2_Z}\right)\;. 
    \label{eq:loop_factor_f_def}
\end{align}
In the above, $\alpha_w\equiv g_w^2/(4\pi)^2$ is the electroweak gauge-coupling parameter, and $M_W$, $M_Z$, and $M_H$ are the masses of the $W$, $Z$, and Higgs bosons, respectively.