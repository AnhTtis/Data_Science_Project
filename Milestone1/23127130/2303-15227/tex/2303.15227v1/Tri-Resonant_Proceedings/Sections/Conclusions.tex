We have showcased a class of leptogenesis models characterised by their $\mathbb{Z}_6$ or $\mathbb{Z}_3$ symmetries. These models offer naturally light SM neutrino masses with large CP-violating phases. When utilised in a tri-resonant framework, this model can fully saturate the available CP asymmetry. We have highlighted how the full and consistent incorporation of a third singlet neutrino can lead to significantly higher scales of CP asymmetry when compared with the bi-resonant approximation commonly utilised in the literature.

Furthermore, we have presented a complete set of Boltzmann equations, accounting for various effects such as varying degrees of freedom and chemical potential corrections. In addition, we have included scattering terms up to $\Delta L =2$ processes with proper RIS subtraction. In our analysis, we have explicitly demonstrated the importance of proper implementation of the variation of the degrees of freedom since this feature can have a significant impact on the generated BAU.

In addition, we have illustrated that an enhanced parameter space is possible when a tri-resonant mass spectrum is considered when compared with the expectations from a typical seesaw model. While the parameter space found is out of range for many current experiments, it is still possible for certain mass ranges to be probed, particularly through $\mu \to e$ conversion in Titanium at PRISM or through collision experiments at the FCC. Due to the democratic structure of the Tri-RL models, flavour effects will not be significant, and hence the results we provide give an upper bound on the scale of the neutrino Yukawa couplings. However, in principle, we may expand this parameter space through the inclusion of additional phenomena, such as coherent oscillations and supersymmetry (SUSY).