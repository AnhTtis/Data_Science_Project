Observations done by the Wilkinson Microwave Anisotropy Probe (WMAP)
and the Planck observatory indicate that the extent of the Baryon
Asymmetry of the Universe (BAU) amounts
to~\cite{Planck:2018vyg,Fields:2019pfx} 
\begin{equation}
\eta_B^{\rm CMB} = 6.104\pm 0.058 \times 10^{-10}.
\end{equation}
Hence, explaining the observed BAU has been one of the central themes
of Particle Cosmology for decades. The existence of this non-zero BAU
is one of the greatest pieces of evidence for physics beyond the
Standard Model (SM). Moreover, the observation of neutrino oscillation phenomena \cite{Ahmad:2001an,Ahmad:2002jz,Fukuda:1998mi} indicates the existence of non-zero neutrino masses in contradiction with the SM prediction. A minimal resolution to both of these problems is to introduce additional neutrinos, which are singlets of the SM gauge group: $\textrm{SU}(3)_c \times \textrm{SU}(2)_L \times \textrm{U}(1)_Y$. The inclusion of a lepton number violating Majorana mass term permits these additional neutrinos to have large masses whilst suppressing the masses of the SM neutrinos. This mechanism is aptly referred to as the seesaw mechanism \cite{Minkowski:1977sc,GellMann:1980vs,Yanagida:1979as,Mohapatra:1979ia}. The violation of lepton number by two units satisfies one of the famous Sakharov conditions~\cite{Sakharov:1967dj} for the generation of appreciable particle asymmetries. Further to this, the expansion of the FRW Universe provides a cosmic arrow of time as well as satisfying the out-of-equilibrium condition, again provided by Sakharov. In combination with the CP violation present in the Yukawa sector, these properties allow for the generation of large lepton asymmetries, which may then be re-processed into a baryon asymmetry through equilibrium $(B+L)$-violating sphaleron transitions. This mechanism of generating appreciable BAU is widely known as \textit{leptogenesis}~\cite{FukYan:1986,Buchmuller:2004nz}.

In \cite{daSilva:2022mrx}, we consider a class of leptogenesis models which can provide naturally light SM neutrino masses as well as generate appreciable levels of BAU to match the observed CMB data. We assume these models to contain three singlet neutrinos, which have mass splittings comparable to their decay widths, permitting maximal mixing between the heavy eigenstates. This framework is commonly referred to as Resonant Leptogenesis (RL). To this end, we compute the CP asymmetries associated with this tri-resonant model and follow this up with a set of complete Boltzmann equations to calculate the generated BAU. These Boltzmann equations describe the evolution of the neutrino and lepton number densities prior to the sphaleron freeze-out when the temperature of the universe falls below $T_\textrm{sph}\approx 132 \; \GeV$~\cite{DOnofrio:2014rug}. A particular highlight of our study is the preservation of the variations in the relativistic degrees of freedom. In much of the literature, the degrees of freedom are taken to be constant due to the high-temperature scales. However, we show that the small variations, which pervade even above temperatures of $T = 100\;\GeV$, can have a significant impact on the generated BAU.

Finally, we present results for the allowed regions of parameter space which can achieve successful leptogenesis and compare these results with current and future experiments. In particular, we make comparisons with coherent flavour-changing processes within nuclei from experiments such as MEG~\cite{MEG:2016leq,MEGII:2018kmf} and PRISM~\cite{BARLOW201144}, as well as collider experiments such as the LHC and FCC.