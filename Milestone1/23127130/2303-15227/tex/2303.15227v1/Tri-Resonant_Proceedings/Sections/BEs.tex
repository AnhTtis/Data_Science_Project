The generation of appreciable BAU requires not only significant CP asymmetries but also a departure from equilibrium and baryon number violation. Here, we will introduce a complete set of Boltzmann equations which describe the out-of-equilibrium dynamics in the early universe, which allows for a dynamical generation of appreciable lepton asymmetry. This lepton asymmetry may be reprocessed into a baryon asymmetry through $(B+L)$-violating sphaleron transitions~\cite{KUZMIN198536}.

At temperature scales pertinent to leptogenesis, it is assumed that the Universe is in the radiation-domination era, with energy and entropy densities
\begin{equation}
    \rho(T) = \frac{\pi^2}{30} g_\textrm{eff}(T)\, T^4, \qquad s(T) = \frac{2\pi^2}{45} h_\textrm{eff}(T)\, T^3,
\end{equation}
respectively. Here, $T$ is the temperature of the Universe, with $g_\textrm{eff}$ and $h_\textrm{eff}$ relativistic degrees of freedom of the SM plasma. We include the variations in the relativistic degrees of freedom since these are not constant, even when the temperature is well above $100\; \GeV$. These variations are small in magnitude but may have drastic implications for the generation of appreciable BAU with low-scale neutrino masses. For our numerical simulations, we utilise the data set labelled `EOS C' provided in \cite{Hindmarsh:2005ix}.

The evolution of the neutrino and lepton asymmetry number densities are described by their respective Boltzmann equations, written as a function of the dimensionless parameter $z_\alpha = m_{N_\alpha}/T$. To align with previously used conventions, we define $z=z_1$ to be the dynamical evolution parameter. In addition, we normalise the number density of a species, $i$, to the photon density,
\begin{equation}
    n_{\gamma}(\za) = \frac{2\zeta(3)T^3}{\pi^2} = \frac{2\zeta(3)}{\pi^2} \lrb{\dfrac{\mNa}{\za}}^3.
\end{equation}
This normalisation simplifies the Boltzmann equations and relates the number density to an observable quantity,
\begin{equation}
    \eta_i(\za) = \frac{n_i(\za)}{n_\gamma(\za)}.
\end{equation}
In the case of the neutrino Boltzmann equations, it is convenient to express the evolution in terms of a departure-from-equilibrium quantity
\begin{equation}
    \delta\eta_\alpha(\za) = \frac{\eta_\alpha(\za)}{\eta_\alpha^{\textrm{eq}}(\za)} - 1.
\end{equation}
In this definition, we have used the equilibrium value of $\eta_\alpha$, which may be explicitly calculated to be
\begin{equation}
    \eta_\alpha^{\textrm{eq}} \approx \frac{\za^2}{2\zeta(3)} K_2(\za),
\end{equation}
with $\zeta(3)$ Ap\'ery's constant, and $K_n(\za)$ a modified Bessel function of the second kind.

With these considerations, we may write a set of coupled Boltzmann equations, including decay terms, $\Delta L = 1$ and $\Delta L = 2$ scattering processes, as well as the running of the degrees of freedom,
\begin{align}
    %
    \frac{d\dNa}{d \ln \za} =&-\dfrac{ \delta_h(\za)}{H(\za) \ \eta_{N_\alpha}^{\rm eq}(\za) }\lrsb{\dNa  \lrb{\Gamma^{D(\alpha)} + \Gamma^{S(\alpha)}_Y + \Gamma^{S(\alpha)}_G} +\frac{2}{9}\, \etaL\, \delta_\alpha \lrb{\tilde\Gamma^{D(\alpha)} + \hat{\Gamma}^{S(\alpha)}_Y + \hat{\Gamma}^{S(\alpha)}_G} } \nonumber\\
    & +\lrb{\dNa+1} \, \left[\za \frac{K_1(\za)}{K_2(\za)} - 3(\delta_h(\za) -1) \right]\;,
    \label{eq:BEN}\\
    %
    \frac{d\eta_L}{d \ln z} =& -  \frac{\delta_h(z)}{H(z)} \lrBiggcb {\sum_{\alpha=1}^3 \dNa \delta_\alpha \lrb{ \Gamma^{D(\alpha)} + \Gamma^{S(\alpha)}_Y + \Gamma^{S(\alpha)}_G} \nonumber \\
    & +\frac{2}{9}\eta_L \left[\sum_{\alpha=1}^3\left(\tilde\Gamma^{D(\alpha)} + \tilde{\Gamma}^{S(\alpha)}_Y + \tilde\Gamma^{S(\alpha)}_G + 
    \Gamma^{W(\alpha)}_Y + \Gamma^{W(\alpha)}_G\right) + \Gamma^{\Delta L = 2} \right] \nn\\ 
    & + \dfrac{2}{27} \etaL \, \sum_{\alpha=1}^3 \delta_\alpha^2 \, \lrb{ \Gamma^{W(\alpha)}_Y + \Gamma^{W(\alpha)}_G } }
    - 3\etaL(\delta_h(z) -1) \label{eq:BEL}\;.
\end{align}
In these equations, we have utilised the well-known expression for the temperature-dependent Hubble parameter
\begin{equation}
    H(\za) = \sqrt{\frac{4\pi^3 g_{\textrm{eff}}(\za)}{45}}\frac{m_{N_\alpha}^2}{M_{\textrm{Pl}}} \frac{1}{\za^2},
\end{equation}
with $M_{\textrm{Pl}} \approx 1.221 \times 10^{19} \; \GeV$ the Planck mass. The relevant collision terms, denoted by $\Gamma^X_Y$, are readily available in the literature \cite{Pilaftsis:2005rv}.

As presented here, the Boltzmann equations are complete up to $\Delta L = 2$ scattering processes. Moreover, the terms included take into account the subtraction of real intermediate states (RIS). Such terms may contribute negatively to the Boltzmann equations due to the lack of an on-shell contribution to the scattering amplitude.

As briefly alluded to earlier, the lepton asymmetry generated is partially re-processed into a baryon asymmetry through $(B+L)$-violating sphaleron transitions. As may be found in the literature \cite{PhysRevD.42.3344}, the generated BAU from a lepton asymmetry is given by
\begin{equation}
    \eta_B = -\frac{28}{51} \eta_L.
\end{equation}
However, sphaleron transitions become suppressed once the temperature of the Universe falls below the critical temperature $T_\textrm{sph} \approx 132 \; \GeV$. Consequently, no leptons are re-processed once the Universe cools to a temperature below $T_\textrm{sph}$. 

Moreover, observations of the BAU produce the value at the recombination epoch; however, due to the expansion of the Universe, this results in a dilution of the overall baryon asymmetry present at $T_\textrm{sph}$. To compare these two values meaningfully, we assume that as the Universe cools from $T_\textrm{sph}$ to $T_\textrm{rec}$, there are no entropy-releasing processes. Consequently, the entropy of the Universe is constant, and we may calculate the ratio
\begin{equation}
    \frac{\eta_B(T_\textrm{rec})}{\eta_B(T_\textrm{sph})} =\frac{n_\gamma(T_\textrm{sph}) s(T_\textrm{rec})}{n_\gamma(T_\textrm{rec}) s(T_\textrm{sph})} = \frac{h_{\textrm{eff}}(T_\textrm{rec})}{h_{\textrm{eff}}(T_\textrm{sph})}.
\end{equation}
For this ratio, we take the approximate value $1/27$~\cite{Pilaftsis:2003gt, Buchmuller:2004nz}, and as a result, the observed baryon asymmetry is related to the generated lepton asymmetry by
\begin{equation}
    \eta_B^\textrm{obs} = - \frac{1}{27}\frac{28}{51} \eta_L(T_\textrm{sph}).
\end{equation}