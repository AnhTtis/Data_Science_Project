\begin{figure}[t!]
	\begin{subfigure}[]{0.48\textwidth}
		\includegraphics[width=1.\textwidth]{Images/evolution_high.pdf}
		\caption{}
		\label{fig:evolution_high}
	\end{subfigure}
	\begin{subfigure}[]{0.48\textwidth}
		\includegraphics[width=1.\textwidth]{Images/evolution_low.pdf}
		\caption{}
		\label{fig:evolution_low}
	\end{subfigure}
	\centering
    \caption{Evolution plots for the deviation from equilibrium of the neutrino
      densities $\delta\eta_{N_\alpha}$ (blue and red solid
      lines) and the baryon asymmetry $\eta_B$ (black
      solid line). Input parameters for the mass of the lightest
      singlet neutrino, and the scale of the Yukawa coupling can be
      seen on each panel, with the grey (dotted) line indicating the
      value $z=\zsph$ at which the sphaleron processes freeze out. The
      orange dot-dashed line indicates the observed value of the
      baryon asymmetry of $\eta^{\rm CMB}_B = 6.104\times 10^{-10}$.} 
    \label{fig:evol_plots}
\end{figure}

We now present the numerical solutions of the Boltzmann equations given in (\ref{eq:BEN}) and (\ref{eq:BEL}). We assume a tri-resonant mass spectrum for the singlet neutrinos and a democratic $a = b = c$ structure for the Yukawa couplings. We consider masses in the range $40 \; \GeV$ to $1 \; \TeV$, as numerical solutions below $40 \; \GeV$ are more limited due to the neglect of thermal masses, which may cause phase space suppression. In addition, a study at these low masses may require the inclusion of additional CP-violating effects, such as coherent oscillations of singlet neutrinos.

The inclusion of scattering processes generates a delay in the maximum of the baryon asymmetry evolution. As a result, the evolution of the baryon asymmetry becomes dependent on the mass scale of the singlet neutrinos. In Figure~\ref{fig:evol_plots}, we analyse this phenomenon. In both panels, we take the initial conditions $\delta \eta_{N_\alpha}(z_0) = 0$ and $\eta_L(z_0) = 0$, with $z_0 = 10^{-2}$, although due to the heavily attractive nature of the solution, the results remain unchanged for any other sensible choice of the initial conditions. Figure~\ref{fig:evolution_high} considers neutrinos of mass scale $m_{N_1} = 1 \; \TeV$ and Yukawa couplings $|\mathbf{h}_{ij}^\nu| = 3\times 10^{-4}$. It can be seen in this figure that for $\TeV$ scale neutrinos, the baryon asymmetry reaches a maximum value before a significant amount is washed out prior to the sphaleron freeze-out at $T_\textrm{sph}$. Conversely, Figure~\ref{fig:evolution_low} shows the evolution for neutrinos of mass $m_{N_1} = 120 \; \GeV$ and Yukawa couplings of size $|\mathbf{h}_{ij}^\nu| = 2\times 10^{-4}$. In this figure, we see that the generation of the BAU occurs at the maximum of the evolution. In general, light singlet neutrino masses results in the generated BAU freezing out earlier in the evolution. Finally, we observe that in both of the panels in Figure~\ref{fig:evol_plots}, at high values of $z$, there is a significant difference between the evolution of $\delta \eta_{N_2}$ and $\delta \eta_{N_{1,3}}$. This is due to the significantly higher CP asymmetry associated with $N_2$, as highlighted in Figure~\ref{fig:cp_asymmetry}.

\begin{figure}[t!]
\centering
\begin{subfigure}[]{0.49\textwidth}
	\includegraphics[width=1.0\textwidth]{Images/leptogenesis_param_space_DK_LFV_2.pdf}
	\caption{}
	\label{fig:lfv_lims}
\end{subfigure}
\begin{subfigure}[]{0.49\textwidth}
	\includegraphics[width=1.0\textwidth]{Images/leptogenesis_param_space_DK_collider_2.pdf}
	\caption{}
	\label{fig:collider_lims}
\end{subfigure}
\caption{Parameter space for the TRL model, including current limits
  (solid lines) and projected sensitivities of future experiments
  (dashed lines). \textit{Left panel:} Projected sensitivities of cLFV
  searches for $\mu\rightarrow e\gamma$ (orange dashed line),
  $\mu\rightarrow eee$ (dashed red line), coherent $\mu\rightarrow e$
  conversion in titanium (dashed blue line), and in gold (solid
  blue line). \textit{Right panel:} Projected
  sensitivities for collider searches at LHC$14$ (blue dashed line),
  FCC-ee (red dashed line), and current limits from DELPHI (orange
  solid line). The green region in these panels
  indicates points in the parameter space where successful leptogenesis is possible, and the green solid line corresponds to the points that reproduce exactly the
  observed value for a tri-resonant model. 
  The upper and lower yellow dot-dashed lines were obtained by scaling the total CP asymmetry $\delta_T$ by a factor of 2 and 0.1, respectively, then matching the observed baryon asymmetry. These lines represent an uncertainty estimate in the calculation of the solid green line due to the oscillations of singlet neutrinos.}
\label{fig:parameter_space}
\end{figure}

Figure~\ref{fig:parameter_space} shows the parameter space on the $\sum_\alpha B_{l\alpha} B^*_{k\alpha}$ vs. $m_{N_1}$ plane. As before, we assume a democratic flavour structure with $a=b=c$ and a tri-resonant mass spectrum. Moreover, we take the initial conditions $\delta \eta_{N_\alpha}(z_0) = 0$ and $\eta_L(z_0) = 0$, with $z_0 = 10^{-2}$. We highlight the region in which successful generation of the BAU is possible, with the solid green line indicating points in the parameter space where the generated BAU is equal to the observed value $\eta_B^\textrm{CMB} = 6.104 \times 10^{-10}$. Points within the green-shaded region may be made to match $\eta_B^\textrm{CMB}$ by softly relaxing the tri-resonant condition, and hence this region also permits the successful generation of the BAU.

In Figure~\ref{fig:parameter_space}, the yellow dashed lines represent bounds when additional sources of CP asymmetry are included. The upper dashed line is obtained by scaling up the total CP asymmetry by a factor of 2, and the lower dashed line is obtained by scaling down the CP asymmetry by a factor of 10. These represent the theoretical uncertainty due to the neglect of coherent oscillation effects between heavy neutrinos. These estimates were generated by assuming that the CP asymmetry from coherent oscillations is additive to the CP asymmetry arising through singlet neutrino mixing. However, there may be constructive or destructive interference between these two effects, highlighting the current lack of consensus regarding whether mixing and oscillations are distinct phenomena or whether mixing is contained within the oscillation formalism. Consequently, the numerical results from a detailed study containing both effects may not be as extreme as the bounds presented in Figure~\ref{fig:parameter_space}.

Figure~\ref{fig:lfv_lims} compares the available parameter space with sensitivity limits on current cLFV experiments involving muons. In particular, we consider coherent muon to electron transitions within nuclei, as well as $\mu \to e\gamma$ and $\mu \to eee$ experiments. As may be seen in this figure, the only experiment which may probe the parameter space of successful leptogenesis is the coherent $\mu\to e$ transition in Titanium at PRISM~\cite{BARLOW201144}

Figure~\ref{fig:collider_lims} considers the projected and current limits for various collider experiments. The estimate denoted by LHC$14$
(blue dashed line) presents conservative projections LHC
with $300~{\rm fb}^{-1}$ data operating at $\sqrt{s} =
14~\TeV$ for the sensitivity to the process $pp\rightarrow N\ell^\pm jj$~\cite{Deppisch:2015qwa,Dev:2013wba}. The orange solid line represents 95\% C.L. limits found by comparing LEP data with
the prediction for signals of decaying heavy neutrinos that are
produced via $Z\rightarrow N\nu_L$~\cite{DELPHI:1996qcc} at DELPHI. Similar
limits have been derived by the L3
collaboration~\cite{L3:1992xaz}. The red dashed line shows the Future Circular Collider (FCC)
sensitivity to the same signals
for electron-positron collisions assuming the normal order of the light
neutrino spectrum, and considering the lifetime of the heavy
neutrinos~\cite{Blondel:2014bra}.