In models of thermal leptogenesis, CP-violating effects enter through the difference in the decay rates of heavy neutrinos into leptons and Higgs bosons $(N \rightarrow L \Phi)$, and the conjugate process $(N \rightarrow L^c \Phi^\dagger)$~\cite{FukYan:1986,Buchmuller:2004nz}. This difference appears at the loop level, with the wavefunction contribution particularly dominant in models of RL, where mass splittings are of a similar size to the decay widths of the heavy neutrinos (for a review, see~\cite{Pilaftsis:1998pd}). To aid the discussion of analytic results regarding the leptonic CP asymmetries, we introduce the coefficients~\cite{Pilaftsis:1997jf,Pilaftsis:2003gt,Pilaftsis:2005rv}
%
\begin{align}
    A_{\alpha \beta} &= \sum_{l=1}^3 \frac{\h^\nu_{l \alpha}\h_{l
                       \beta}^{\nu*}}{16\pi} =
                       \frac{(\h^{\nu\dagger}\h^\nu)^*_{\alpha\beta}}{16\pi},\\ 
    %
    V_{l \alpha} &= \sum_{k=1}^3 \sum_{\gamma \neq \alpha}
                   \frac{\h^{\nu*}_{k\alpha}\h^\nu_{k
                   \gamma}\h^\nu_{l\gamma}}{16\pi} f\left(
                   \frac{m^2_{N_\gamma}}{m^2_{N_\alpha}} \right),
    %
    \label{eq:V_def}
\end{align}
%
which correspond to absorptive transition rates for the wavefunction and vertex, respectively. In~(\ref{eq:V_def}), ${f(x) = \sqrt{x}\left[1-(1+x)\ln \left( \frac{1+x}{x} \right) \right]}$ is the Fukugita-Yanagida 1-loop function~\cite{FukYan:1986,Buchmuller:2004nz}.

Completing a full re-summation of the loop corrections, including all three Majorana neutrinos, generates an effective $NL\tilde{\Phi}$ coupling~\cite{Pilaftsis:2003gt, Pilaftsis:2005rv, Deppisch:2010fr}
\begin{align}
\label{eq:Eff_Yuk}
    (\bar{\mathbf{h}}^\nu_+)_{l\alpha} =&\; \h^\nu_{l\alpha} +
                                          iV_{l\alpha} - i
                                          \sum_{\beta,\gamma = 1}^3
                                          |\varepsilon_{\alpha\beta\gamma}|\,\h^\nu_{l\beta}\nonumber\\&\times
  \frac{m_{N_\alpha}\left(M_{\alpha\alpha\beta}+M_{\beta\beta\alpha}\right)-i
  R_{\alpha\gamma}
  \left[M_{\alpha\gamma\beta}\left(M_{\alpha\alpha\gamma}+M_{\gamma\gamma\alpha}\right)
  +
  M_{\beta\beta\gamma}\left(M_{\alpha\gamma\alpha}+M_{\gamma\alpha\gamma}\right)\right]}{m_{N_\alpha}^2-m_{N_\beta}^2
  + 2i m^2_{N_\alpha} A_{\beta\beta} + 2i\,\Im m
  R_{\alpha\gamma}\left(m_{N_\alpha}^2 |A_{\beta\gamma}|^2 +
  m_{N_\beta} m_{N_\gamma} \Re e A_{\beta\gamma}^2 \right)}\;, 
\end{align}
%
where $\epsilon_{\alpha\beta\gamma}$ is the anti-symmetric Levi-Civita
symbol, $M_{\alpha\beta\gamma}\equiv m_{N_\alpha}A_{\beta\gamma}$ and 
%
\begin{equation}
    R_{\alpha\beta} \equiv \frac{m_{N_\alpha}^2}{m_{N_\alpha}^2-m_{N_\beta}^2+2i m_{N_\alpha}^2 A_{\beta\beta}}\;.
\end{equation}
The conjugate $NL^c\tilde{\Phi}^\dagger$ couplings, denoted by $(\bar{\mathbf{h}}^\nu_-)_{l\alpha}$, are found through the replacement of $\mathbf{h}^\nu_{l\alpha}$ by $(\mathbf{h}^\nu)^*_{l\alpha}$ in (\ref{eq:Eff_Yuk}). These effective couplings capture both \textit{bi-resonant} and \textit{tri-resonant} effects, corresponding to maximal CP asymmetries through the mixing of two and three singlet neutrinos, respectively. In particular, one may recover the bi-resonant expressions by simply taking $R_{\alpha\gamma}$ to~zero.

Utilising these re-summed effective couplings, we may calculate the partial decay widths of the heavy neutrinos as
\begin{equation}
    \Gamma (N_\alpha \rightarrow L_l \Phi) =
    \frac{m_{N_\alpha}}{8\pi}\left| (\bar{\h}^\nu_+)_{l \alpha}
    \right|^2,\qquad 
    \Gamma (N_\alpha \rightarrow L^C_l \Phi^\dagger) =
    \frac{m_{N_\alpha}}{8\pi}\left| (\bar{\mathbf{h}}^\nu_-)_{l
        \alpha} \right|^2\;. 
    \label{eq:Gammas_def}
\end{equation}

\begin{figure}[t!]
\hspace*{-0.6cm}
%\centering
    \includegraphics[width=\linewidth]{Images/delta_vs_M3.pdf}
    \caption{\textit{Left panel:} CP asymmetries generated by the decays of
      $N_1$, $N_2$ and $N_3$, together with the total CP asymmetry
      $\delta_T = \sum_\alpha \delta_\alpha$, as a function of the
      mass of $N_3$. \textit{Centre panel:} Comparison of the CP asymmetry in the decay
      of $N_2$ vs. $m_{N_3}$ as calculated from two neutrino mixing ($\delta^{(2)}_2$) and three-neutrino mixing
      ($\delta^{(3)}_2$). \textit{Right panel:} CP asymmetry in the
      decay of $N_3$ vs. $m_{N_3}$ calculated from two-neutrino mixing ($\delta^{(2)}_3$) and
      three-neutrino mixing ($\delta^{(3)}_3$). We indicate the
      values of $m_{N_1}$, $m_{N_2}$ and the tri-resonant value of
      $m_{N_3}$ with grey dashed lines.} 
    \label{fig:cp_asymmetry}
\end{figure}

From this, we identify the size of the CP asymmetries within the model using the dimensionless quantity
\begin{equation}
    \delta_{\alpha l} \equiv \frac{\Gamma (N_\alpha \rightarrow L_l
      \Phi) - \Gamma (N_\alpha \rightarrow L^C_l \Phi^\dagger) }{
      \sum_{k=e,\mu,\tau} \Gamma (N_\alpha \rightarrow L_k \Phi)
      +\Gamma (N_\alpha \rightarrow L^C_k \Phi^\dagger)} =
    \frac{\left| (\bar{\mathbf{h}}^\nu_+)_{l \alpha} \right|^2 -
      \left| (\bar{\mathbf{h}}^\nu_-)_{l \alpha}
      \right|^2}{(\bar{\mathbf{h}}^{\nu\dagger}_+\bar{\mathbf{h}}^\nu_+)_{\alpha\alpha}
      +
      (\bar{\mathbf{h}}^{\nu\dagger}_-\bar{\mathbf{h}}^\nu_-)_{\alpha\alpha}}. 
    \label{eq:CP-deltas_def}
\end{equation}
Furthermore, we may define the total CP asymmetry associated with each neutrino species by summing over the lepton families
\begin{equation}
    \delta_\alpha = \sum_{l} \delta_{\alpha l} = 
    \frac{(\bar{\mathbf{h}}^{\nu\dagger}_+\bar{\mathbf{h}}^\nu_+)_{\alpha\alpha}
      -
      (\bar{\mathbf{h}}^{\nu\dagger}_-\bar{\mathbf{h}}^\nu_-)_{\alpha\alpha}}{(\bar{\mathbf{h}}^{\nu\dagger}_+\bar{\mathbf{h}}^\nu_+)_{\alpha\alpha}
      +
      (\bar{\mathbf{h}}^{\nu\dagger}_-\bar{\mathbf{h}}^\nu_-)_{\alpha\alpha}}.
\end{equation}
At this point, it is important to mention that the existence of non-zero CP asymmetries is only possible in the event that the CP-odd invariant
\begin{align}
    \Delta_{\rm CP} &= \Im m \left\{ \textrm{Tr} \left[
                  (\mathbf{h}^\nu)^\dagger \mathbf{h}^\nu
                  \mathbf{m}_M^\dagger \mathbf{m}_M
                  \mathbf{m}_M^\dagger (\mathbf{h}^\nu)^{\sf T}
                  (\mathbf{h}^\nu)^* \mathbf{m}_M\right] \right\}\\ 
    &= \sum_{\alpha<\beta} m_{N_\alpha} m_{N_\beta}
      \left(m_{N_\alpha}^2 - m_{N_\beta}^2 \right)\, \Im m \Big[
      \left(\mathbf{h}^{\nu\dagger}\mathbf{h}^\nu
      \right)_{\beta\alpha}^2 \Big]
\end{align}
does not vanish~\cite{Pilaftsis:1997dr, Pilaftsis:2003gt,
Branco:1986gr, Yu:2020gre} When all neutrinos are exactly degenerate, this quantity is trivially zero, and hence CP asymmetries are not possible. However, if mass splittings are permitted, we see that in the $\mathbb{Z}_6$ model presented earlier, this CP-odd quantity is proportional to the $\mathbb{Z}_6$ element $\omega^2$
\begin{align}
    \Delta_{\rm CP} &\approx \left(|a|^2 + |b|^2 + |c|^2 \right)^2
                  \sum_{\alpha<\beta} m_{N_\alpha} m_{N_\beta}
                  \left(m_{N_\alpha}^2 - m_{N_\beta}^2 \right)\, \Im m
                  \Big( \omega^{2(\alpha-\beta)} \Big)\; . 
\end{align}
Accordingly, the $\mathbb{Z}_6$ structure we have proposed offers both naturally light SM neutrino masses and produce significant levels of CP asymmetry due to the large CP-violating phases present.

In the literature, there are several examples of the bi-resonant approximation being used in RL scenarios to enhance the contribution to the CP asymmetry through the mixing of two singlet neutrinos whilst permitting the third neutrino to decouple, either through suppressed couplings or a higher mass scale. However, in a model where all three neutrinos satisfy the resonance condition
\begin{equation}
    |m_{N_\alpha} - m_{N_\beta}| \sim \frac{\Gamma_{N_{\alpha,\beta}}}{2},
\end{equation}
the contributions to CP asymmetries may be enhanced through constructive interference between all three neutrinos~\cite{Pilaftsis:1997dr}. Figure~\ref{fig:cp_asymmetry} shows the variation in the generated CP asymmetry through the decay of singlet neutrinos, as well as the total CP asymmetry $\delta_T = \sum_\alpha \delta_\alpha$. In this figure, $m_{N_1}$ and $m_{N_2}$ are fixed to satisfy the resonance condition, and $m_{N_3}$ is permitted to vary. As is expected, the total CP asymmetry is seen to vanish in the case that $m_{N_3}$ is equal to either $m_{N_1}$ or $m_{N_2}$; however, is maximised when $m_{N_3} = m_{N_2} + \frac{1}{2}\Gamma_{N_2}$. This maximum of the CP asymmetry is $35\%$ larger than what can be produced in models with only two neutrino mixing. Furthermore, at this maximum, it can be seen that $\delta_1 \simeq \delta_3$, while $\delta_2$ is significantly enhanced. Consequently, $\delta_2$ is the dominant contributor to $\delta_T$.

The latter two panels in Figure~\ref{fig:cp_asymmetry} highlight the difference between two neutrino mixing, $\delta^{(2)}_\alpha$, and three neutrino mixing $\delta^{(3)}_\alpha$. It is clear from the second panel that the proper inclusion of three neutrino mixing is important in the resonant region, as a sizeable difference becomes apparent in the CP asymmetry of $N_2$. A similar effect is present in the CP asymmetry of $N_3$, shown in the final panel, although to a lesser extent.

In general, Figure~\ref{fig:cp_asymmetry} highlights the importance of full and proper accounting for the mixing of three neutrinos when these neutrinos are in consecutive resonance. As a consequence, this tri-resonant structure saturates the available CP asymmetry and maximises the generated BAU at a given mass scale with specified couplings. This is in contrast to the bi-resonant models commonly studied in the literature, which neglect contributions to the CP asymmetry from the mixing of a third neutrino species. 