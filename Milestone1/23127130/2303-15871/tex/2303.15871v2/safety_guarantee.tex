Having described the Collision Cone CBF candidate, we will see their application on quadrotors in this section. Based on the shape of the obstacle we can divide the proposed candidates into two cases: 
% \subsubsection{3D CBF candidate}
% When the dimensions of the obstacle are comparable to each other, we can assume the obstacle as a sphere with radius $r = max(c_1, c_2, c_3) + \frac{w}{2}$, where $w$ is the max width of the quadrotor absorbed in the obstacle width (shown in Fig. \ref{fig:3D CBF}). We call the CBF candidate so formed in this case as \textbf{3D CBF} candidate (see Fig. \ref{fig:3D CBF}).

% \subsubsection{Projection CBF candidate}
%  


\subsection{3D CBF candidate}
\label{section: 3D-CBF}
In scenarios where the dimensions of the obstacle are roughly equal, we can model the obstacle as a sphere, as illustrated in Fig. \ref{fig:3D CBF}. The resulting CBF in this context is referred to as the \textbf{3D CBF}. The relative position vector between the body center of the quadrotor and the center of the obstacle is as follows:
\begin{align}\label{eq:pos-vec-3D}
    \prel := \begin{bmatrix}
        c_x \\
        c_y \\
        c_z
    \end{bmatrix}
    - \left (
    \begin{bmatrix}
        x_p \\
        y_p \\
        z_p
    \end{bmatrix}
    + \textbf{R} \begin{bmatrix}
        0 \\
        0 \\
        l
    \end{bmatrix}
    \right )
\end{align}
Here $l$ is the distance of the body center from the base (see Fig. \ref{fig:models}). $c_x,c_y,c_z$ represents the obstacle location as a function of time. Also, since the obstacles are of constant velocity, we have $\Ddot{c}_x= \Ddot{c}_y= \Ddot{c}_z = 0$. We obtain its relative velocity as
\begin{align}\label{eq:vel-vec-3D}
    \vrel := \dot{p}_{rel}
\end{align}


% Having defined the $p_{rel}$ and $v_{rel}$
% relative position and velocity of the moving obstacle
% Now, we calculate the $\vreldot$ term which contains our inputs i.e. $(f_1, f_2, f_3, f_4)$, as follows:
% \begin{align*}
%     \vreldot = - (\frac{1}{m}\textbf{R}
%             \begin{bmatrix}
%                 0 & 0 & 0 & 0 \\
%                 0 & 0 & 0 & 0 \\
%                 1 & 1 & 1 & 1 
%             \end{bmatrix} + \\
%             \textbf{R} lL \hat{k} I_b^{-1}
%             \begin{bmatrix}
%                 1 & 0 & -1 & 0 \\
%                 0 & 1 & 0 & -1 \\
%                 0 & 0 & 0 & 0 
%             \end{bmatrix})
%             \begin{bmatrix}
%     		f_{1} \\
%     		f_{2} \\
%                 f_{3} \\
%                 f_{4} 
%     	\end{bmatrix} \\
%         + some other terms. \nonumber
% \end{align*}
% where

% \begin{align*}
%     \hat{k} = 
%             \begin{bmatrix}
%                 0 & -1 & 0 \\
%                 1 & 0 & 0  \\
%                 0 & 0 & 0 
%             \end{bmatrix}, 
% % \end{align*}
% % \begin{align*}
%     I_b^{-1} =
%             \begin{bmatrix}
%                 \frac{1}{I_{xx}} & 0 & 0 \\
%                 0 & \frac{1}{I_{yy}} & 0 \\
%                 0 & 0 & \frac{1}{I_{zz}}
%             \end{bmatrix}
%          \nonumber
% \end{align*}
% Therefore, 
% \begin{equation}\label{eqn:vrel-dot-3D}
%     \vreldot = -\textbf{R}
%             \begin{bmatrix}
%                 0 & \frac{Ll}{I_{yy}} & 0 & \frac{-Ll}{I_{yy}} \\
%                 \frac{-Ll}{I_{xx}} & 0 & \frac{Ll}{I_{xx}} & 0 \\
%                 \frac{1}{m_Q} & \frac{1}{m_Q} & \frac{1}{m_Q} & \frac{1}{m_Q} 
%             \end{bmatrix}
%             \begin{bmatrix}
%     		f_{1} \\
%     		f_{2} \\
%                 f_{3} \\
%                 f_{4} 
%     	\end{bmatrix} \\
%         + \rm{additional} \: \rm{terms}. \nonumber
% \end{equation}

% \subsubsection{Ellipse CBF}
% Consider the following CBF candidate:
% % \begin{tcolorbox}
% \begin{equation}
% \begin{aligned}
%     h_{El}(\state,t) = \left(\frac{c_x(t) - x_p}{c_1}\right)^2 + \left(\frac{c_y(t) - y_p}{c_2}\right)^2 \\
%     + \left(\frac{c_z(t) - z_p}{c_3}\right)^2 - 1,
%     \label{eqn:Ellipse-CBF}
% \end{aligned}
% \end{equation}
% % \end{tcolorbox}
% Since $h_{El}$ in \eqref{eqn:Ellipse-CBF} is dependent on time (e.g. moving obstacles), the resulting set $\mathcal{C}$ is also dependent on time. To analyze this class of sets, time dependent versions of CBFs can be used \cite{IGARASHI2019735}. Alternatively, we can reformulate our problem to treat the obstacle position $c_x,c_y$ as states, with their derivatives being constants. This will allow us to continue using the classical CBF given by Definition \ref{definition: CBF definition} including its properties on safety. The derivative of \eqref{eqn:Ellipse-CBF} is
% \begin{align}
% \frac{2(c_x-x)(\dot c_x - \dot x)}{c_1^2} +\frac{2(c_y-y)(\dot c_y - \dot y)}{c_2^2} +\frac{2(c_z-z)(\dot c_z - \dot z)}{c_3^2},
% \end{align}
% which has no dependency on the inputs $(f_1, f_2, f_3, f_4)$, since there is no $\vreldot$ term. Hence, $h_{El}$ will not be a valid CBF for the quadrotor model \eqref{eqn:quadrotor_model}. 


% \subsubsection{Higher Order CBF}
% Having introduced HO-CBFs in \ref{subsection: HO-CBF}, we will try to give guarantees for Form 1  \eqref{eqn:HO-CBF-1} for the simple reason that it is similar to C3BF. So, the effective Higher Order CBF candidate looks like this:

% \begin{equation}
%     h_{HO}(\state, t) = < \prel, \vrel> + \gamma \sqrt{(\|\prel\|^2 - r^2)}
%     \label{eqn:HO-CBF}
% \end{equation}

% where $\gamma$ is the penalty term. We now show that HO CBF candidate \eqref{eqn:HO-CBF} is indeed a valid CBF. 

% \begin{theorem}\label{thm:HO-CBF-3D}{\it
% Given the quadrotor model \eqref{eqn:quadrotor_model}, the proposed CBF candidate \eqref{eqn:HO-CBF} with $\prel,\vrel$ defined by \eqref{eq:pos-vec-3D}, \eqref{eq:vel-vec-3D} is a valid CBF defined for the set $\mathcal{D}$.}
% \end{theorem}
% \begin{proof}
% Taking the derivative of \eqref{eqn:HO-CBF} yields
% \begin{align}
% \dot h_{HO} = &  < \preldot, \vrel > + < \prel, \vreldot >  \nonumber \\
%  & +  \frac{\gamma}{\sqrt{(\|\prel\|^2 - r^2)}} < \prel, \preldot >.
%  \label{eqn:HO_h_d}
% \end{align}


% Given $\vreldot$ (which contains the input) from \eqref{eqn:vrel-dot-3D} and $\dot h$ \eqref{eqn:HO_h_d}, we have the following expression for $\mathcal{L}_g h_{HO}$:
% \begin{align}
%     \mathcal{L}_g h_{HO} = \begin{bmatrix}
%         < \prel, 
%             \textbf{R}\begin{bmatrix}
%                 0  \\
%                 \frac{-Ll}{I_{xx}}\\
%                 \frac{1}{m}
%             \end{bmatrix}>\\
%                 < \prel, 
%             \textbf{R}\begin{bmatrix}
%                 \frac{Ll}{I_{yy}} \\
%                 0\\
%                 \frac{1}{m}
%             \end{bmatrix}>\\
%          < \prel, 
%             \textbf{R}\begin{bmatrix}
%                 0  \\
%                 \frac{Ll}{I_{xx}}\\
%                 \frac{1}{m}
%             \end{bmatrix}>\\
%          < \prel, 
%             \textbf{R}\begin{bmatrix}
%                 \frac{-Ll}{I_{yy}} \\
%                 0 \\
%                 \frac{1}{m}
%             \end{bmatrix}>
%     \end{bmatrix}^T,
% \end{align}
% It can be verified that for $\mathcal{L}_gh_{HO}$ to be zero, we can have the following scenarios:
% \begin{itemize}
%     \item $\prel =0$, which is not possible, because this indicates that the vehicle is already inside the obstacle. 
%     \item $\prel$ is perpendicular to all  $\textbf{R}\begin{bmatrix}
%                     0  \\
%                     \frac{-Ll}{I_{xx}}\\
%                     \frac{1}{m}
%                 \end{bmatrix}$, 
%     $\textbf{R}\begin{bmatrix}
%                 \frac{Ll}{I_{yy}} \\
%                 0\\
%                 \frac{1}{m}
%             \end{bmatrix} $,
%     $\textbf{R}\begin{bmatrix}
%                 0  \\
%                 \frac{Ll}{I_{xx}}\\
%                 \frac{1}{m}
%             \end{bmatrix} $
%             and  $\textbf{R}\begin{bmatrix}
%                 \frac{-Ll}{I_{yy}} \\
%                 0 \\
%                 \frac{1}{m}
%             \end{bmatrix} $, which is also not possible. (Because these vectors form basis vectors for $\mathbb{R}^{3}$)
% \end{itemize}
% This implies that $\mathcal{L}_gh_{HO}$ is always a non-zero matrix, implying that $h_{HO}$ is a valid CBF.
% \end{proof}
% \begin{remark}
% {\it
% Since $\mathcal{L}_g h \neq 0$, we can infer from \cite[Theorem 8]{XU201554} that the resulting QP given by \eqref{eq:CBF-QP} is Lipschitz continuous. Hence, we can construct CBF-QPs with the CBF \eqref{eqn:HO-CBF} for the quadrotor model and guarantee collision avoidance. In addition,  if $h(x(0))<0$, then we can construct a class $\mathcal{K}$ function $\kappa$ in such a way that the magnitude of $h$ exponentially decreases over time, thereby minimizing the violation. We will demonstrate these scenarios in Section \ref{section: Results and Discussions}.
% }
% \end{remark}

\begin{figure}[t]
    % \centering
    \includegraphics[width=0.9\linewidth]{images/3D_CBF.pdf}
\caption{\textbf{3D CBF} candidate: The dimensions of the obstacle are comparable to each other, it can be assumed as a sphere}
\label{fig:3D CBF}
\end{figure}



% \subsubsection{Collision Cone CBF}
Having introduced Collision Cone CBF candidates in \ref{subsection: C3BF}, the next step is to formally verify that they are, indeed, valid CBFs. 
% The 3D C3BF candidate is as follows: 
% 
% \begin{equation}
%     h_{3D}(\state, t) = < \prel, \vrel> + \| \prel\|\| \vrel\|\cos\phi
%     \label{eqn:CC-CBF-3D}
% \end{equation}
% 
% where, $\phi$ is the half angle of the cone, the expression of $\cos\phi$ is given by $\frac{\sqrt{\|\prel\|^2 - r^2}}{\|\prel\|}$ (see Fig. \ref{fig:3D CBF}).  
  % 
% We now show that the proposed CBF candidate \eqref{eqn:CC-CBF-3D} is indeed a valid CBF.
We have the following result.

\begin{theorem}\label{thm:CC-CBF-3D}{\it
Given the quadrotor model \eqref{eqn:quadrotor_model}, the proposed CBF candidate \eqref{eqn:CC-CBF} with $\prel,\vrel$ defined by \eqref{eq:pos-vec-3D}, \eqref{eq:vel-vec-3D} is a valid CBF defined for the set $\mathcal{D}$.}
\end{theorem}
Please refer to \cite[Thm 3]{C3BF_tac} for the proof of Theorem.
% \begin{proof}
% Since we are considering the 3D case, we will denote the resulting CBF candidate given by \eqref{eqn:CC-CBF} by $h_{3D}$. We have the following derivative:
% %Taking the derivative of \eqref{eqn:CC-CBF} yields
% \begin{align}
% \dot h_{3D} = &  < \preldot, \vrel > + < \prel, \vreldot >  \nonumber \\
%  & + < \vrel, \vreldot > \frac{\sqrt{\|\prel\|^2 - r^2}}{\|\vrel\|} \nonumber \\
%  & + < \prel, \preldot > \frac{\|\vrel\| }{\sqrt{\|\prel\|^2 - r^2}}.
%  \label{eqn:CC_h_d}
% \end{align}
% % 
% By substituting for $\vreldot$ (which contains the input) in $\dot h_{3D}$ \eqref{eqn:CC_h_d}, we have the following expression for $\mathcal{L}_g h_{3D}$:
% \begin{align}
%     \mathcal{L}_g h_{3D} = \begin{bmatrix}
%         < \prel + \vrel \frac{\sqrt{\|\vrel\|^2 - r^2}}{\|\vrel\|}, 
%             \textbf{R}\begin{bmatrix}
%                 0  \\
%                 \frac{-Ll}{I_{xx}}\\
%                 \frac{1}{m_Q}
%             \end{bmatrix}>\\
%                 < \prel + \vrel \frac{\sqrt{\|\vrel\|^2 - r^2}}{\|\vrel\|}, 
%             \textbf{R}\begin{bmatrix}
%                 \frac{Ll}{I_{yy}} \\
%                 0\\
%                 \frac{1}{m_Q}
%             \end{bmatrix}>\\
%          < \prel + \vrel \frac{\sqrt{\|\vrel\|^2 - r^2}}{\|\vrel\|}, 
%             \textbf{R}\begin{bmatrix}
%                 0  \\
%                 \frac{Ll}{I_{xx}}\\
%                 \frac{1}{m_Q}
%             \end{bmatrix}>\\
%          < \prel + \vrel \frac{\sqrt{\|\vrel\|^2 - r^2}}{\|\vrel\|}, 
%             \textbf{R}\begin{bmatrix}
%                 \frac{-Ll}{I_{yy}} \\
%                 0 \\
%                 \frac{1}{m_Q}
%             \end{bmatrix}>
%     \end{bmatrix}^T,
% \end{align}

% It can be verified that for $\mathcal{L}_gh_{3D}$ to be zero, we can have the following scenarios:
% \begin{itemize}
%     \item $\prel + \vrel \frac{\sqrt{\|\prel\|^2 - r^2}}{\|\vrel\|}=0$, which is not possible. Firstly, $\prel=0$ indicates that the vehicle is already inside the obstacle. Secondly, if the above equation were to be true for a non-zero $\prel$, then $\vrel/\|\vrel\| = - \prel/\sqrt{\|\prel\|^2 - r^2}$. This is also not possible as the magnitude of LHS is $1$, while that of RHS is $>1$.
%     \item $\prel + \vrel \frac{\sqrt{\|\vrel\|^2 - r^2}}{\|\vrel\|}$ is perpendicular to all  $\textbf{R}\begin{bmatrix}
%                     0  \\
%                     \frac{-Ll}{I_{xx}}\\
%                     \frac{1}{m_Q}
%                 \end{bmatrix}$, 
%     $\textbf{R}\begin{bmatrix}
%                 \frac{Ll}{I_{yy}} \\
%                 0\\
%                 \frac{1}{m_Q}
%             \end{bmatrix} $,
%     $\textbf{R}\begin{bmatrix}
%                 0  \\
%                 \frac{Ll}{I_{xx}}\\
%                 \frac{1}{m_Q}
%             \end{bmatrix} $
%             and  $\textbf{R}\begin{bmatrix}
%                 \frac{-Ll}{I_{yy}} \\
%                 0 \\
%                 \frac{1}{m_Q}
%             \end{bmatrix} $, which is also not possible. (Because three of these vectors form basis vectors for $\mathbb{R}^{3}$)
% \end{itemize}
% This implies that $\mathcal{L}_gh_{3D}$ is always a non-zero matrix, implying that $h_{3D}$ is a valid CBF.
% \end{proof}
% \begin{remark}
% {\it
% Since $\mathcal{L}_g h \neq 0$, we can infer from \cite[Theorem 8]{XU201554} that the resulting QP given by \eqref{eq:CBF-QP} is Lipschitz continuous. Hence, we can construct CBF-QPs with the proposed CBF \eqref{eqn:CC-CBF} for the quadrotor model and guarantee collision avoidance. In addition,  if $h(x(0))<0$, then we can construct a class $\mathcal{K}$ function $\kappa$ in such a way that the magnitude of $h$ exponentially decreases over time, thereby minimizing the violation. We will demonstrate these scenarios in 
% %This would imply either braking or steering leading to collision with the periphery of the obstacle. %This will be shown in 
% Section \ref{section: Results and Discussions}.
% }
% \end{remark}

\subsection{Projection CBF candidate}
\label{section: proj-CBF}

\begin{figure}[t]
    \centering
    \includegraphics[width=0.8\linewidth]{images/Projection_CBF.pdf}
\caption{\textbf{Projection CBF} candidate: One of the dimensions, of the obstacle, is bigger than the other dimensions, it can be assumed as a cylinder.}
\label{fig:Projection CBF}
\end{figure}
When an obstacle has significantly disparate dimensions, it can be approximated as a cylinder, giving rise to the \textbf{Projection CBF} (refer to Fig. \ref{fig:Projection CBF}). To derive this, we calculate the relative position vector between the quadrotor's body center and the intersection point of the obstacle's axis with the projection plane, which is perpendicular to the axis. Thus, we obtain:
\begin{align}\label{eqn:pos-vec-proj}
    (\prel)_{proj} := \mathcal{P}\left (\begin{bmatrix}
        c_x \\
        c_y \\
        c_z
    \end{bmatrix}
    - \left (
    \begin{bmatrix}
        x_p \\
        y_p \\
        z_p
    \end{bmatrix}
    + \textbf{R} \begin{bmatrix}
        0 \\
        0 \\
        l
    \end{bmatrix}
    \right ) \right ).
\end{align}
Here $l$ is the distance of the body center from the base (see Fig. \ref{fig:models}). $\mathcal{P}: \mathbb{R}^3 \to \mathbb{R}^3 $ is the projection operator, which can be assumed to be a constant\footnote{Note that the obstacles are always translating and not rotating. In addition, it is not restrictive to assume that the translation direction is always perpendicular to the cylinder axis. This makes the projection operator a constant.}. 
%But in this case $\Ddot{c}_x , \Ddot{c}_y, \& \Ddot{c}_z \neq 0$.
Now, since the relative position lies on the projection plane, we have one more condition to satisfy:
\begin{align}\label{eqn:p_in_plane}
    < (\prel)_{proj}, \hat{n} > = 0,   
\end{align}
% 
where, $\hat{n}$ is the normal to the plane. Also, the relative velocity is given by:
\begin{align}\label{eqn:vel-vec-proj}
    (\vrel)_{proj} := \frac{d({p}_{rel})_{proj}}{dt} = (\preldot)_{proj}
\end{align}
% Having defined the $p_{rel}$ and $v_{rel}$
% relative position and velocity of the moving obstacle
% Now, we calculate the $\frac{d}{dt}(\vrel)_{proj}$ term which contains our inputs i.e. $(f_1, f_2, f_3, f_4)$, as follows:
% \begin{align*}
%     (\vreldot)_{proj} = \begin{bmatrix}
%             \ddot{c_x} \\
%             \ddot{c_y} \\
%             \ddot{c_z}
%         \end{bmatrix}
%         - 
%         (\frac{1}{m}\textbf{R}
%             \begin{bmatrix}
%                 0 & 0 & 0 & 0 \\
%                 0 & 0 & 0 & 0 \\
%                 1 & 1 & 1 & 1 
%             \end{bmatrix} + \\
%             \textbf{R} lL \hat{k} I_b^{-1}
%             \begin{bmatrix}
%                 1 & 0 & -1 & 0 \\
%                 0 & 1 & 0 & -1 \\
%                 0 & 0 & 0 & 0 
%             \end{bmatrix})
%             \begin{bmatrix}
%     		f_{1} \\
%     		f_{2} \\
%                 f_{3} \\
%                 f_{4} 
%     	\end{bmatrix} \\
%         + some other terms. \nonumber
% \end{align*}
% where

% \begin{align*}
%     \hat{k} = 
%             \begin{bmatrix}
%                 0 & -1 & 0 \\
%                 1 & 0 & 0  \\
%                 0 & 0 & 0 
%             \end{bmatrix}, 
% % \end{align*}
% % \begin{align*}
%     I_b^{-1} =
%             \begin{bmatrix}
%                 \frac{1}{I_{xx}} & 0 & 0 \\
%                 0 & \frac{1}{I_{yy}} & 0 \\
%                 0 & 0 & \frac{1}{I_{zz}}
%             \end{bmatrix}
%          \nonumber
% \end{align*}
% Therefore, 
% \begin{equation}%\label{eqn:vrel_dot}
%     \frac{d}{dt}(\vrel)_{proj} =
%     \mathcal{P}(
%             -  
%             \textbf{R}
%             \begin{bmatrix}
%                 0 & \frac{Ll}{I_{yy}} & 0 & \frac{-Ll}{I_{yy}} \\
%                 \frac{-Ll}{I_{xx}} & 0 & \frac{Ll}{I_{xx}} & 0 \\
%                 \frac{1}{m_Q} & \frac{1}{m_Q} & \frac{1}{m_Q} & \frac{1}{m_Q} 
%             \end{bmatrix}
%             \begin{bmatrix}
%     		f_{1} \\
%     		f_{2} \\
%                 f_{3} \\
%                 f_{4} 
%     	\end{bmatrix} \\
%         + \rm{additional} \: \rm{terms}). \nonumber
% \end{equation}
% or, from \eqref{eqn:vrel-dot-3D}, we have
% \begin{equation}\label{eqn:vrel_dot}
%     \frac{d}{dt}(\vrel)_{proj} =
%     \mathcal{P}( 
%             \vreldot )
% \end{equation}
% $\frac{d}{dt}(\vrel)_{proj}$  is the projection of $\vreldot$ in \eqref{eqn:vrel-dot-3D} on the projection plane, that is:
% \begin{equation}
% \begin{aligned}\label{eqn:vrel-dot-proj}
%     (\vreldot)_{proj} = \vreldot - <\vreldot, \hat{n}> \hat{n}.
% \end{aligned}
% \end{equation}

% Thus, from \eqref{eqn:p_in_plane} and \eqref{eqn:vrel-dot-proj}, we have the following:
% \begin{equation}
% \begin{aligned}\label{eqn:pdot-vreldot}
%     % <(\prel)_{proj}, (\vreldot)_{proj}> = <(\prel)_{Proj}, \vreldot - <\vreldot, \hat{n}> \hat{n}> \\
%     <(\prel)_{proj}, (\vreldot)_{proj}> = <(\prel)_{proj}, \vreldot>.
% \end{aligned}
% \end{equation}

% Similarly,
% \begin{equation}
% \begin{aligned}\label{eqn:vdot-vreldot}
%     % <(\vrel)_{proj}, (\vreldot)_{Proj}> = <(\vrel)_{proj}, \vreldot - <\vreldot, \hat{n}> \hat{n}> \\
%     <(\vrel)_{proj}, (\vreldot)_{proj}> = <(\vrel)_{proj}, \vreldot>
% \end{aligned}
% \end{equation}


% \subsubsection{Ellipse CBF}
% Consider the following CBF candidate:
% % \begin{tcolorbox}
% \begin{equation}
%     h_{El}(\state,t) = \left(\frac{c_x(t) - x_p}{c_1}\right)^2 + \left(\frac{c_y(t) - y_p}{c_2}\right)^2 + \left(\frac{c_z(t) - z_p}{c_3}\right)^2 - 1,
%     \label{eqn:Ellipse-CBF}
% \end{equation}
% % \end{tcolorbox}
% Since $h_{El}$ in \eqref{eqn:Ellipse-CBF} is dependent on time (e.g. moving obstacles), the resulting set $\mathcal{C}$ is also dependent on time. To analyze this class of sets, time dependent versions of CBFs can be used \cite{IGARASHI2019735}. Alternatively, we can reformulate our problem to treat the obstacle position $c_x,c_y$ as states, with their derivatives being constants. This will allow us to continue using the classical CBF given by Definition \ref{definition: CBF definition} including its properties on safety. The derivative of \eqref{eqn:Ellipse-CBF} is
% \begin{align}
% \frac{2(c_x-x)(\dot c_x - \dot x)}{c_1^2} +\frac{2(c_y-y)(\dot c_y - \dot y)}{c_2^2} +\frac{2(c_z-z)(\dot c_z - \dot z)}{c_3^2},
% \end{align}
% which has no dependency on the inputs $(f_1, f_2, f_3, f_4)$, since there is no $\vreldot$ term. Hence, $h_{El}$ will not be a valid CBF for the quadrotor model \eqref{eqn:quadrotor_model}. 


% \subsubsection{Higher Order CBF}

% We will try to give guarantees for Form 1  \eqref{eqn:HO-CBF-1} for the projection case. So, the effective Higher Order CBF candidate looks like:

% \begin{equation}
%     h_{HO}(\state, t) = < (\prel)_{proj}, (\vrel)_{proj}> + \gamma \sqrt{(\|(\prel)_{proj}\|^2 - r^2)}
%     \label{eqn:HO-CBF-proj}
% \end{equation}

% where, $\gamma$ is the penalty term. 

% We now show that HO CBF candidate \eqref{eqn:HO-CBF-proj} is indeed a valid CBF. 

% \begin{theorem}\label{thm:HO-CBF-Proj}{\it
% Given the quadrotor model \eqref{eqn:quadrotor_model}, the proposed CBF candidate \eqref{eqn:HO-CBF} with $(\prel)_{proj}, (\vrel)_{proj}$ defined by \eqref{eqn:pos-vec-proj}, \eqref{eqn:vel-vec-proj} is a valid CBF defined for the set $\mathcal{D}$.}
% \end{theorem}

% \begin{proof}
% Taking the derivative of \eqref{eqn:HO-CBF-proj} yields
% \begin{align}
% \dot h_{HO} = &  < (\preldot)_{proj}, (\vrel)_{proj} > + < (\prel)_{proj}, (\vreldot)_{proj} >  \nonumber \\
%  & +  \frac{\gamma}{\sqrt{(\|(\prel)_{proj}\|^2 - r^2)}} < (\prel)_{proj}, (\preldot)_{proj} >.
%  \label{eqn:HO_h_d-proj}
% \end{align}

% Given $(\vreldot)_{proj}$ (which contains the input) from \eqref{eqn:vrel-dot-proj}, equation \eqref{eqn:pdot-vreldot} and $\dot h$ from \eqref{eqn:HO_h_d-proj}, we have the following expression for $\mathcal{L}_g h_{HO}$:
% \begin{align}
%     \mathcal{L}_g h_{HO} = \begin{bmatrix}
%         < (\prel)_{proj}, 
%             \textbf{R}\begin{bmatrix}
%                 0  \\
%                 \frac{-Ll}{I_{xx}}\\
%                 \frac{1}{m}
%             \end{bmatrix}>\\
%                 < (\prel)_{proj}, 
%             \textbf{R}\begin{bmatrix}
%                 \frac{Ll}{I_{yy}} \\
%                 0\\
%                 \frac{1}{m}
%             \end{bmatrix}>\\
%          < (\prel)_{proj}, 
%             \textbf{R}\begin{bmatrix}
%                 0  \\
%                 \frac{Ll}{I_{xx}}\\
%                 \frac{1}{m}
%             \end{bmatrix}>\\
%          < (\prel)_{proj}, 
%             \textbf{R}\begin{bmatrix}
%                 \frac{-Ll}{I_{yy}} \\
%                 0 \\
%                 \frac{1}{m}
%             \end{bmatrix}>
%     \end{bmatrix}^T,
% \end{align}
% Now, we can give the same arguments on why $\mathcal{L}_gh_{HO}$ cannot be zero as given in the 3D CBF case.
% % It can be verified that for $\mathcal{L}_gh_{HO}$ to be zero, we can have the following scenarios:
% % \begin{itemize}
% %     \item $\prel =0$, which is not possible, because this indicates that the vehicle is already inside the obstacle. 
% %     \item $\prel$ is perpendicular to all  $\textbf{R}\begin{bmatrix}
% %                     0  \\
% %                     \frac{-Ll}{I_{xx}}\\
% %                     \frac{1}{m}
% %                 \end{bmatrix}$, 
% %     $\textbf{R}\begin{bmatrix}
% %                 \frac{Ll}{I_{yy}} \\
% %                 0\\
% %                 \frac{1}{m}
% %             \end{bmatrix} $,
% %     $\textbf{R}\begin{bmatrix}
% %                 0  \\
% %                 \frac{Ll}{I_{xx}}\\
% %                 \frac{1}{m}
% %             \end{bmatrix} $
% %             and  $\textbf{R}\begin{bmatrix}
% %                 \frac{-Ll}{I_{yy}} \\
% %                 0 \\
% %                 \frac{1}{m}
% %             \end{bmatrix} $, which is also not possible. (Because these vectors form basis vectors for $\mathbb{R}^{3}$
% % \end{itemize}
% % This implies that $\mathcal{L}_gh_{HO}$ is always a non-zero matrix, implying that $h_{HO}$ is a valid CBF.
% \end{proof}
% % \begin{remark}
% % {\it
% % Since $\mathcal{L}_g h \neq 0$, we can infer from \cite[Theorem 8]{XU201554} that the resulting QP given by \eqref{eq:CBF-QP} is Lipschitz continuous. Hence, we can construct CBF-QPs with the CBF \eqref{eqn:HO-CBF} for the quadrotor model and guarantee collision avoidance. In addition,  if $h(x(0))<0$, then we can construct a class $\mathcal{K}$ function $\kappa$ in such a way that the magnitude of $h$ exponentially decreases over time, thereby minimizing the violation. We will demonstrate these scenarios in Section \ref{section: Results and Discussions}.
% % }
% % \end{remark}


% \subsubsection{Collision Cone CBF}
% We now provide the formal results for \eqref{eqn:CC-CBF} for the Projection case in this subsection. The candidate is given as follows:
% \begin{equation}
%     h_{proj}(\state, t) = < (\prel)_{proj}, (\vrel)_{proj}> + \| (\prel)_{proj}\|\| (\vrel)_{proj}\|\cos\phi
%     \label{eqn:CC-CBF-proj}
% \end{equation}
% where, $\phi$ is the half angle of the cone, the expression of $\cos\phi$ is given by $\frac{\sqrt{\|(\prel)_{proj}\|^2 - r^2}}{\|(\prel)_{proj}\|}$ (see Fig. \ref{fig:Projection CBF}).  We now show that the proposed CBF candidate \eqref{eqn:CC-CBF-proj} is indeed a valid CBF. 

\begin{theorem}\label{thm:CC-CBF-proj}{\it
Given the quadrotor model \eqref{eqn:quadrotor_model}, the proposed CBF candidate \eqref{eqn:CC-CBF} with $\prel,\vrel$ defined by \eqref{eqn:pos-vec-proj}, \eqref{eqn:vel-vec-proj} is a valid CBF defined for the set $\mathcal{D}$.}
\end{theorem}
Please refer to \cite[Thm 4]{C3BF_tac} for the proof of Theorem.
% \begin{proof}
% % Since we are considering the projection case, we will denote the resulting CBF candidate given by \eqref{eqn:CC-CBF} by $h_{proj}$. 
% We have the following derivative of $h_{proj}$:
% \begin{align}
% \dot h_{proj} = &  < (\preldot)_{proj}, (\vrel)_{proj} > + < (\prel)_{proj}, (\vreldot)_{proj} >  \nonumber \\
%  & + < (\vrel)_{proj}, (\vreldot)_{proj} > \frac{\sqrt{\|(\prel)_{proj}\|^2 - r^2}}{\|(\vrel)_{proj}\|} \nonumber \\
%  & + < (\prel)_{proj}, (\preldot)_{proj} > \frac{\|(\vrel)_{proj}\| }{\sqrt{\|(\prel)_{proj}\|^2 - r^2}}.
%  \label{eqn:CC_h_d-proj}
% \end{align}
% % 
%  $\vreldot$ (which contains the input) from \eqref{eqn:vrel-dot-proj}, equations \eqref{eqn:pdot-vreldot}, \eqref{eqn:vdot-vreldot} and $\dot h_{proj}$ \eqref{eqn:CC_h_d-proj}, we have the following expression for $\mathcal{L}_g h_{proj}$:
% \begin{align}
%     \mathcal{L}_g h_{proj} = \begin{bmatrix}
%         < \prel + \vrel \frac{\sqrt{\|\vrel\|^2 - r^2}}{\|\vrel\|}, 
%             \textbf{R}\begin{bmatrix}
%                 0  \\
%                 \frac{-Ll}{I_{xx}}\\
%                 \frac{1}{m_Q}
%             \end{bmatrix}>\\
%                 < \prel + \vrel \frac{\sqrt{\|\vrel\|^2 - r^2}}{\|\vrel\|}, 
%             \textbf{R}\begin{bmatrix}
%                 \frac{Ll}{I_{yy}} \\
%                 0\\
%                 \frac{1}{m_Q}
%             \end{bmatrix}>\\
%          < \prel + \vrel \frac{\sqrt{\|\vrel\|^2 - r^2}}{\|\vrel\|}, 
%             \textbf{R}\begin{bmatrix}
%                 0  \\
%                 \frac{Ll}{I_{xx}}\\
%                 \frac{1}{m_Q}
%             \end{bmatrix}>\\
%          < \prel + \vrel \frac{\sqrt{\|\vrel\|^2 - r^2}}{\|\vrel\|}, 
%             \textbf{R}\begin{bmatrix}
%                 \frac{-Ll}{I_{yy}} \\
%                 0 \\
%                 \frac{1}{m_Q}
%             \end{bmatrix}>
%     \end{bmatrix}^T,
% \end{align}
% Using the same arguments we gave in the proof of theorem. \ref{thm:CC-CBF-3D}, we can infer that $\mathcal{L}_gh_{proj}$ cannot be zero. This implies that $h_{proj}$ is a valid CBF.
% % Now, we can give the same arguments on why $\mathcal{L}_gh_{CC}$ cannot be zero as given in the 3D CBF case.
% % It can be verified that for $\mathcal{L}_gh_{CC}$ to be zero, we can have the following scenarios:
% % \begin{itemize}
% %     \item $\prel + \vrel \frac{\sqrt{\|\prel\|^2 - r^2}}{\|\vrel\|}=0$, which is not possible. Firstly, $\prel=0$ indicates that the vehicle is already inside the obstacle. Secondly, if the above equation were to be true for a non-zero $\prel$, then $\vrel/\|\vrel\| = - \prel/\sqrt{\|\prel\|^2 - r^2}$. This is also not possible as the magnitude of LHS is $1$, while that of RHS is $>1$.
% %     \item $\prel + \vrel \frac{\sqrt{\|\vrel\|^2 - r^2}}{\|\vrel\|}$ is perpendicular to all  $\textbf{R}\begin{bmatrix}
% %                     0  \\
% %                     \frac{-Ll}{I_{xx}}\\
% %                     \frac{1}{m}
% %                 \end{bmatrix}$, 
% %     $\textbf{R}\begin{bmatrix}
% %                 \frac{Ll}{I_{yy}} \\
% %                 0\\
% %                 \frac{1}{m}
% %             \end{bmatrix} $,
% %     $\textbf{R}\begin{bmatrix}
% %                 0  \\
% %                 \frac{Ll}{I_{xx}}\\
% %                 \frac{1}{m}
% %             \end{bmatrix} $
% %             and  $\textbf{R}\begin{bmatrix}
% %                 \frac{-Ll}{I_{yy}} \\
% %                 0 \\
% %                 \frac{1}{m}
% %             \end{bmatrix} $, which is also not possible. (Because these vectors form basis vectors for $\mathbb{R}^{3}$
% % \end{itemize}
% % This implies that $\mathcal{L}_gh_{CC}$ is always a non-zero matrix, implying that $h_{CC}$ is a valid CBF.
% \end{proof}

% \begin{remark}
% {\it
% % Since $\mathcal{L}_g h \neq 0$ in Theorems \eqref{thm:CC-CBF-3D} \& \eqref{thm:CC-CBF-proj}, we can infer from \cite[Theorem 8]{XU201554} that the resulting QP given by \eqref{eqn: CBF QP} is Lipschitz continuous. Hence, we can construct CBF-QPs with the proposed CBF \eqref{eqn:CC-CBF} for the quadrotor model and guarantee collision avoidance.

% Based on Theorems \eqref{thm:CC-CBF-3D} and \eqref{thm:CC-CBF-proj}, where $\mathcal{L}_g h \neq 0$, we can utilize the conclusion from \cite[Theorem 8]{XU201554} to deduce that the control inputs obtained from the resulting CBF-QP \eqref{eqn: CBF QP} are Lipschitz continuous. As a result, 
% %it is feasible to create CBF-QPs using the suggested CBF \eqref{eqn:CC-CBF} for the quadrotor model, ensuring collision avoidance.
% the resulting solutions guarantee forward invariance of the safe set generated by the proposed C3BF candidates.
% }
% \end{remark}

\subsection{Comparison with Higher Order CBFs}
We introduce the state-of-the-art Higher Order Control Barrier Functions (HO-CBFs) and compare them with the proposed C3BF in this section. Given that the collision constraints are with respect to position, the associated CBF has a relative degree of two. Therefore, it is necessary to establish a higher-order CBF with $m = 2$ as outlined in \cite[Eq. 16]{9516971}, which is expressed as: 
\begin{equation}
\begin{aligned}
%\label{eqn:HO-CBF}
\psi_{1}(x,t) &= \dot{b}(x,t) + p\alpha_{1} (b(x,t))\\
\psi_{2}(x,t) &= \dot{\psi_{1}}(x,t) + p \alpha_{2} (\psi_{1}(x,t)) ,\\
\end{aligned}
\end{equation}
where $b(\state,t) = (c_x(t) - x_p)^2 + (c_y(t) - y_p)^2 + (c_z(t) - z_p)^2 - r^2$, and r is the encompassing radius given by $r = max(c_1, c_2, c_3)$. $\alpha_1, \alpha_2$ are both class $\mathcal{K}$ functions, and $p$ is a tunable constant. As explained previously, $c_x,c_y,c_z$ is the center location of the obstacle as a function of time. Let us examine the form of HO-CBF where $\alpha_{1}$ is a square root function (which is also strictly increasing), and $\alpha_{2}$ is a linear function, due to its similarity to C3BF. Consequently, the resulting Higher Order CBF candidate takes the following form:
\begin{equation}
    h_{HO}(\state, t) = < \prel, \vrel> + \gamma \sqrt{(\|\prel\|^2 - r^2)}.
    \label{eqn:HO-CBF}
\end{equation}
% 
% If we try to understand C3BF, it tries to avoid the $\vrel$ vector between the quadrotor and the obstacle from going in the collision cone region given by the half angle $\phi' = \phi$ as shown in figures \ref{fig:3D CBF} and \ref{fig:Projection CBF}. Now, rewriting HO-CBF given in \eqref{eqn:HO-CBF} in C3BF form will result the following $\phi'$:
We can show that the above-mentioned HO-CBF is also a valid CBF for quadrotors. We will now compare it with the proposed C3BF.

The C3BF concept aims to prevent the $\vrel$ vector, which represents the relative velocity between the quadrotor and the obstacle, from entering the collision cone region defined by the half-angle $\phi$. Figures \ref{fig:3D CBF}, \ref{fig:Projection CBF} and \ref{fig:HO-C3-CBF} illustrate this idea. We can rewrite the HO-CBF formula presented in \eqref{eqn:HO-CBF} in the following form: 
% of C3BF, we obtain the following expression for $\phi'$:
\begin{align}
h_{HO}(\state, t) = < \prel, \vrel> + \|\prel\| \|\vrel\| cos(\phi')
    \label{eqn:HO-CBF-CC}
\end{align}
where, $cos(\phi') = \frac{\gamma}{\|\vrel\|}cos(\phi)$.
% Now if we search and find an appropriate $\gamma$ (penalty term) for the given HOCBF, it will result in a valid CBF as per \cite{9516971}. However, $\gamma$ will still be a constant in that case, resulting in an overestimation of the cone. On the other hand, in the case of C3BF, since we are allowing the penalty term to change with time i.e. keeping $\gamma = \|\vrel\|$, it ends up giving us a more accurate estimate of the collision cone as compared to HO-CBF case. This is even evident from the simulation results of both the CBFs (as shown in the next section).
If we are able to identify a suitable $\gamma$ (penalty term) for the given HO-CBF, it would result in a valid CBF as per \cite{9516971}. Nonetheless, in such a scenario where $\gamma$ remains constant and $\|\vrel\|$ goes on increasing, it leads to an increase in $\phi'$, thus, overestimating the cone as can be seen in Fig.\ref{fig:HO-C3-CBF}. Conversely, with the C3BF approach, we permit the penalty term to vary over time, i.e., $\gamma = \|\vrel\|$, resulting in a more precise estimation of the collision cone compared to the HO-CBF case. 
This also shows that C3BF is not a special case of Higher Order CBF.
% 
This is also evident from the simulation outcomes of both CBFs, as demonstrated in Section \ref{section: Simulation Results}.

\begin{figure}[t]
    % \centering
    \includegraphics[width=0.9\linewidth]{images/HO-3D_CBF.pdf}
\caption{Comparison of HO-CBF with C3BF. Here we are trying to compare the $\phi'$ and $\phi$ obtained from the two CBF formulations. It can be observed that $\phi'$ (pink cone) is dependent on $v_{rel}$, while $\phi$ (yellow cone) is a constant. The HO-CBF guarantees safety for a set that is not only smaller but also dependent on $v_{rel}$ as shown by the pink cone. Hence, HO-CBF is more conservative compared to C3BF.}
\label{fig:HO-C3-CBF}
\end{figure}