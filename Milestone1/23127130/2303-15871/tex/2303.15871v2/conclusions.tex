We presented the extension of a novel collision cone CBF formulation for quadrotors to avoid collision with static and moving obstacles of various shapes and sizes.
% \par The combination of collision cones and CBFs gives an ability to handle moving obstacles and guarantees collision avoidance. 
% The minimization objective of the optimization problem translates to minimal modification of input control. The quadratic program results in smooth trajectories. 
% Existing works in literature were not able to circumnavigate / brake in the presence of obstacles with non-zero velocity values. The proposed QP formulation (C3BF-QP) allows the vehicle to safely maneuver under different scenarios presented in the paper. 
We successfully constructed CBF-QPs with the proposed CBF for the quadrotor model and guaranteed safety by avoiding moving obstacles. This includes collision avoidance with spherical and cylindrical obstacles.
We also showed that the 
% proposed Collision Cone CBF is better than the 
current state-of-the-art Higher Order CBFs is more conservative and fails in certain scenarios (shown in the video). 
Finally, we demonstrated the robustness of the proposed CBF-QP controller for safe navigation in a cluttered environment consisting of multiple obstacles and agents with the same safety filters.
% In general, CC-CBF is able to either circumnavigate moving obstacles with constant velocity or come to a halt if it isn't able to find a way.
% \par The obstacles considered in this project are assumed to have a constant velocity which isn't a slick assumption by itself. Much dynamic models of the car can also be considered for the application of CC-CBF. The idea of $\vec{v}_{rel}$ falling into collision cone as when the obstacle is perceived i.e. $h(s, t) < 0$ at $t_0$ will result in asymptotically pulling $\vec{v}_{rel}$ out of collision cone which is desirable but, there can't be time guarantees on $h(s, t) \geq 0$. 
% It is worth mentioning that the results presented for the bicycle model are preliminary, and its modifications will be a subject of future work. 
% As a part of future work, the focus will be more on autonomous cars with a complex dynamic model for avoiding common obstacles on road.
In our future work, we plan to implement the controller on quadrotors in real-world situations, interacting with a variety of obstacles. We also intend to explore applications such as the safe teleoperation of quadrotors. Additionally, we aim to investigate the potential of applying the C3BF formulation to legged robots walking in confined spaces. %An idea of the safety-critical stack can also be explored where CC-CBF is restricted to stay on the road else break.
