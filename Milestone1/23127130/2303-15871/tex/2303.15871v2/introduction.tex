\par Quadrotors are used in a wide range of applications, including search and rescue, environmental monitoring, agriculture, transportation, and entertainment \cite{kumar2015future}. In many of these applications, quadrotors operate in complex and dynamic environments, where they must navigate around obstacles such as trees, buildings, and other drones. The literature presents a variety of methods such as artificial potential field \cite{8022685}, reachability analysis \cite{8263977} \cite{RA-UAV}, and nonlinear model predictive control \cite{8442967} to address the problem of obstacle avoidance in UAVs.

In recent years, the Control Barrier Functions (CBFs) based approach \cite{7040372}\cite{Ames_2017} has emerged as a promising strategy for ensuring safe operation of autonomous systems. This is a model-based control design method, which provides a computationally efficient solution that can handle complex situations while guaranteeing safety. CBFs can be formulated as a Quadratic Problem (QP) and can be solved online, making them well-suited for real-time safety-critical applications. % where strong safety guarantees are essential. 
CBFs are specifically designed to enforce safety constraints and provide hard constraints on the system's trajectory, making them superior to Nonlinear MPC (NMPC) in terms of safety guarantees. NMPC, on the other hand, provides soft constraints on the system's trajectory, with the degree of constraint satisfaction dependent on the optimization algorithm's performance.

\begin{figure}[t]
    \centering
    \includegraphics[width=0.50\linewidth]{images/Crazyflie.pdf}
\caption{World coordinates and body fixed coordinates of Crazyflie and Euler's angles defined in these coordinates}
\label{fig:models}
\end{figure}

In situations that involve complex interactions between subsystems or where safety requirements are highly dynamic and subject to frequent changes, reachability analysis may be limited. In such cases, CBFs are more suitable due to their ability to handle highly dynamic safety requirements \cite{https://doi.org/10.48550/arxiv.2106.13176}. While the artificial potential field approach is easy to implement, it suffers from limitations such as the possibility of getting stuck in local minima and difficulties in handling complex environments with multiple obstacles and \cite{Singletary2021ComparativeAO} shows that CBFs offer a viable, and arguably improved alternative to APFs for real-time obstacle avoidance.  Thus, the Control Barrier Functions approach provides a better solution for obstacle avoidance in UAVs, particularly in safety-critical scenarios. \cite{7525253} has shown collision avoidance using CBFs in planar quadrotor case.
% 
In a recent work, \cite{DBLP:journals/corr/abs-1903-04706, 9516971} showed that Higher order CBFs are a generalized form of the exponential CBFs, thus it also addresses this problem of obstacle avoidance. However, a major challenge with this approach is the need to identify suitable penalty parameters (p's) and class $\mathcal{K}$ functions ($\alpha$'s) that can yield optimal results.

With regards to obstacle avoidance in dynamic environments, another class of approaches that is widely used is the method of collision cones \cite{Fiorini1993}, \cite{ doi:10.1177/027836499801700706}, \cite{709600}
% The Collision Cone \cite{Fiorini1993}, \cite{ doi:10.1177/027836499801700706}, \cite{709600} approach is a widely used method for collision avoidance in robotics and autonomous vehicles. 
It involves defining a cone-shaped region between two objects to represent the potential area of collision, which can be avoided by adjusting the object's trajectory to prevent the relative velocity vector from falling within the cone. This approach has several advantages, including its simplicity, efficiency, and adaptability to different environments. The method can be easily integrated into existing motion planning algorithms, takes into account the speed and direction of moving objects and the shape and size of potential obstacles, and can work in dynamic and unpredictable environments. As a result, the Collision Cone approach provides a reliable and flexible means of avoiding collisions in various robotic and autonomous systems. 

Despite its simplicity and effectiveness, the collision cone method has largely been restricted to offline motion planning/navigation problems, and its extensions for real-time implementations have been limited. However, by exploiting the CBF-QP formulations, we can synthesize a new class of CBFs through the notion of collision cones, which can then be implemented in real-time. This will be the main objective of this paper. This idea was originally proposed in \cite{C3BF} for the planar case (2D) and for wheeled robots, while we aim to extend this for 3D and for quadrotors, which are underactuated and have higher degrees of freedom (DoF).

\subsection{Contribution and Paper Structure}
The main idea is to realize a CBF-QP formulation for the quadrotor dynamics and for obstacles with non-zero velocity values.
% We consider the quadrotor dynamics and the relative position \& velocity of the obstacle to develop a safe controller, which avoids a kinematic obstacle. 
The main contributions of our work are:

\begin{itemize}
    \item We formulate a direct method for safe trajectory tracking of quadrotors based on collision cone control barrier functions expressed through a quadratic program.
    \item We consider static and constant velocity obstacles of various dimensions and provide mathematical guarantees for collision avoidance.
    \item We compare the collision cone CBF with the state-of-the-art higher-order CBF (HO-CBF), and show how the former is better in terms of feasibility and safety guarantees. 
\end{itemize}



\subsection{Organisation}
The rest of this paper is organized as follows. Preliminaries explaining the quadrotor model, the concept of control barrier functions (CBFs), collision cone CBFs, and controller design are introduced in section \ref{section: Background}. The application of the above CBFs on the quadrotor to avoid obstacles of various shapes is discussed in section \ref{section: Safety Guarantee}. The simulation setup and results will be discussed in section \ref{section: Simulation Results}. Finally, we present our conclusion in section \ref{section: Conclusions}. 
