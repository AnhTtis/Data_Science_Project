In this section, first, we will describe the dynamics of quadrotor. Next, we will formally introduce Control Barrier Functions (CBFs) and their importance for real-time safety-critical control. Finally, we will introduce  Collision Cone Control Barrier Function (C3BF) approach.

% \subsection{Fixed Wing UAV model}
% % The Fixed Wing UAV , (see Fig. \ref{fig:models}). 
% The dynamics of the Fixed Wing UAV is as follows:
% \begin{equation}
% \label{eqn:fixed_wing_model}
% 	\underbrace{\begin{bmatrix}
% 		\dot{x}_p \\
% 		\dot{y}_p \\
%             \dot{z}_p \\
%             \ddot{x}_p \\
% 		\ddot{y}_p \\
%             \ddot{z}_p \\
% 		\dot{\phi} \\
%             \dot{\theta} \\
%             \dot{\psi} \\
% 		\dot{\omega}_{1} \\
%             \dot{\omega}_{2} \\
%             \dot{\omega}_{3}
% 	\end{bmatrix}}_{\dot{\state}}
% 	=
% 	\underbrace{\begin{bmatrix}
% 		\dot{x}_p \\
% 		\dot{y}_p \\
%             \dot{z}_p \\
%             0 \\
%             0 \\
%             -g \\
%             \textbf{W}^{-1}
%             \begin{bmatrix}
%                 \omega_{1} \\
%                 \omega_{2} \\
%                 \omega_{3}
%             \end{bmatrix}\\
%             \\
% 		  -I^{-1} \vec{\omega} \times I \vec{\omega}  \\
%             .
% 	\end{bmatrix}}_{f(\state)}
% 	+
% 	\underbrace{\begin{bmatrix}
% 		\\
%             \begin{bmatrix}
%                 0 & 0 & 0 & 0 \\
%                 0 & 0 & 0 & 0 \\
%                 0 & 0 & 0 & 0 
%             \end{bmatrix}\\
%             \\
%             \frac{1}{M}\textbf{R}
%             \begin{bmatrix}
%                 0 & 0 & 0 & 0 \\
%                 0 & 0 & 0 & 0 \\
%                 1 & 1 & 1 & 1 
%             \end{bmatrix}\\
%             \\
%             \begin{bmatrix}
%                 0 & 0 & 0 & 0 \\
%                 0 & 0 & 0 & 0 \\
%                 0 & 0 & 0 & 0 
%             \end{bmatrix}\\
%             \\
% 		I^{-1}L
%             \begin{bmatrix}
%                 1 & 0 & -1 & 0 \\
%                 0 & 1 & 0 & -1 \\
%                 0 & 0 & 0 & 0 
%             \end{bmatrix}
% 	\end{bmatrix}}_{g(\state)}
% 	\underbrace{\begin{bmatrix}
% 		f_{1} \\
% 		f_{2} \\
%             f_{3} \\
%             f_{4} 
% 	\end{bmatrix}}_{u}.
% \end{equation}
% $x_p$, $y_p$ and $z_p$ denote the coordinates of the vehicle’s centre of the base of the quadrotor in an inertial frame. $\phi$, $\theta$ and $\psi$ represents the (roll, pitch \& yaw) orientation of the quadrotor.  (see Fig. \ref{fig:models}). \textbf{R} is the rotation matrix (from the body frame to the inertial frame), M is the mass of the quadrotor, $\textbf{W}$ is the transformation matrix for angular velocities from the inertial frame to the body frame, I is the inertia matrix and L is the diagonal length of quadrotor.


\subsection{Quadrotor model}
The quadrotor model has four propellers, which provides upward thrusts of $(f_1, f_2, f_3, f_4)$ (see Fig. \ref{fig:models}) and the states needed to describe the quadrotor system is given by $x = [x_p, y_p, z_p, \dot{x}_p, \dot{y}_p, \dot{z}_p,  \phi, \theta, \psi, {\omega}_{1}, {\omega}_{2}, {\omega}_{3}]$ . The quadrotor dynamics is as follows~\cite{quadfolk}:
\begin{equation}
\label{eqn:quadrotor_model}
	\underbrace{\begin{bmatrix}
		\dot{x}_p \\
		\dot{y}_p \\
            \dot{z}_p \\
            \ddot{x}_p \\
		\ddot{y}_p \\
            \ddot{z}_p \\
		\dot{\phi} \\
            \dot{\theta} \\
            \dot{\psi} \\
		\dot{\omega}_{1} \\
            \dot{\omega}_{2} \\
            \dot{\omega}_{3}
	\end{bmatrix}}_{\dot{\state}}
	=
	\underbrace{\begin{bmatrix}
		\dot{x}_p \\
		\dot{y}_p \\
            \dot{z}_p \\
            0 \\
            0 \\
            -g \\
            \textbf{W}^{-1}
            \begin{bmatrix}
                \omega_{1} \\
                \omega_{2} \\
                \omega_{3}
            \end{bmatrix}\\
            \\
		  -I^{-1} \vec{\omega} \times I \vec{\omega}  \\
            .
	\end{bmatrix}}_{f(\state)}
	+
	\underbrace{\begin{bmatrix}
		\\
            \begin{bmatrix}
                0 & 0 & 0 & 0 \\
                0 & 0 & 0 & 0 \\
                0 & 0 & 0 & 0 
            \end{bmatrix}\\
            \\
            \frac{1}{m_Q}\textbf{R}
            \begin{bmatrix}
                0 & 0 & 0 & 0 \\
                0 & 0 & 0 & 0 \\
                1 & 1 & 1 & 1 
            \end{bmatrix}\\
            \\
            \begin{bmatrix}
                0 & 0 & 0 & 0 \\
                0 & 0 & 0 & 0 \\
                0 & 0 & 0 & 0 
            \end{bmatrix}\\
            \\
		I^{-1}L
            \begin{bmatrix}
                1 & 0 & -1 & 0 \\
                0 & 1 & 0 & -1 \\
                c_{\tau} & -c_{\tau} & c_{\tau} & -c_{\tau} 
            \end{bmatrix}
	\end{bmatrix}}_{g(\state)}
	\underbrace{\begin{bmatrix}
		f_{1} \\
		f_{2} \\
            f_{3} \\
            f_{4} 
	\end{bmatrix}}_{u}
\end{equation}
$x_p$, $y_p$, and $z_p$ denote the coordinates of the vehicle’s center of the base of the quadrotor in an inertial frame. $\phi$, $\theta$ and $\psi$ represents the (roll, pitch \& yaw) orientation of the quadrotor.  (see Fig. \ref{fig:models}). \textbf{R} is the rotation matrix (from the body frame to the inertial frame), $m_Q$ is the mass of the quadrotor, $\textbf{W}$ is the transformation matrix for angular velocities from the inertial frame to the body frame, $I$ is the inertia matrix and $L$ is the diagonal length of the quadrotor. $c_{\tau}$ is the constant
that determines the torque produced by each propeller. Note that even though we restrict our study to quadrotors in this paper, the extension of the proposed CBF-QPs for different multi-rotor UAVs is straightforward.

%  \subsection{Trajectory tracking controller}
% \label{subsection: track_controller}
% We have to design a controller that tracks a desired trajectory with a safety filter, which modifies the input in a minimal way around the obstacles.  
% % 
% We use a PD controller to generate a desired controller ($u_{des}$), which tracks the desired trajectory. The PD controller formulated for the quadrotor is given as follows:

% \begin{align*}
%     \begin{bmatrix}
%         \ddot{x} \\
%         \ddot{y} \\
%         \ddot{z}
%     \end{bmatrix}
%     := 
%     \left (
%     \begin{bmatrix}
%         \ddot{x_d} \\
%         \ddot{y_d} \\
%         \ddot{z_d}
%     \end{bmatrix} +
%     \begin{bmatrix}
%         K_{xd}(\dot{x_d}- \dot{x}) \\
%         K_{yd}(\dot{y_d}- \dot{y}) \\
%         K_{zd}(\dot{z_d}- \dot{z})
%     \end{bmatrix} +
%     \begin{bmatrix}
%         K_{xp}(x_d - x) \\
%         K_{yp}(y_d - y) \\
%         K_{zp}(z_d - z)
%     \end{bmatrix}
%     \right )
% \end{align*}

% \begin{align*}
%     \begin{bmatrix}
%         \ddot{r} \\
%         \ddot{p} 
%     \end{bmatrix}
%     := 
%     % \begin{bmatrix}
%     %     \ddot{r_d} \\
%     %     \ddot{p_d} 
%     % \end{bmatrix} +
%     \left (
%     \begin{bmatrix}
%         K_{rd}(\dot{r_d}- \dot{r}) \\
%         K_{pd}(\dot{p_d}- \dot{p}) 
%     \end{bmatrix} +
%     \begin{bmatrix}
%         K_{rp}(r_d - r) \\
%         K_{pp}(p_d - p) 
%     \end{bmatrix}
%     \right )
% \end{align*}

% where, $r_d = - \frac{\ddot{y}}{g+ \ddot{z}}, p_d = \frac{\ddot{x}}{g+ \ddot{z}} $

% \begin{align*}
%     \begin{bmatrix}
%         f_1 \\
%         f_2 \\
%         f_3 \\
%         f_4  
%     \end{bmatrix}
%     := 
%     \begin{bmatrix}
%         1 & 1 & 1 & 1 \\
%         0 & 1 & 0 & -1 \\
%         1 & 0 & -1 & 0 \\
%         c_{\tau} & -c_{\tau} & c_{\tau} & -c_{\tau} 
%     \end{bmatrix}^{-1}
%     \begin{bmatrix}
%         m \sqrt{(\ddot{x}^{2}+\ddot{y}^{2}+(g+\ddot{z})^{2})} \\
%         I_{xx} \ddot{r}/L \\
%         -I_{xx} \ddot{p}/L\\
%         0
%     \end{bmatrix}
% \end{align*}
% where, $K_p's$ and $K_d's$ are the proportional and derivative coefficients of respective variables. 

\subsection{Control barrier functions (CBFs)}
Having described the vehicle models, we now formally introduce Control Barrier Functions (CBFs) and their applications in the context of safety. 
% 
% \textbf{Safety interpreted as forward invariance of a set:} 
Given the quadrotor model, we have the nonlinear control system in affine form:
\begin{equation}
	\dot{\state} = f(\state) + g(\state)u
	\label{eqn: affine control system}
\end{equation}
where $\state \in \mathcal{D} \subseteq \mathbb{R}^n$ is the state of system, and $u \in \mathbb{U} \subseteq \mathbb{R}^m$ the input for the system. Assume that the functions $f: \mathbb{R}^n \rightarrow \mathbb{R}^n$ and $g: \mathbb{R}^n \rightarrow \mathbb{R}^{n \times m}$ are continuously differentiable. Specific formulation of $f,g$ for the quadrotor were described in \eqref{eqn:quadrotor_model}. Given a Lipschitz continuous control law $u = k(\state)$, the resulting closed loop system $\dot{\state} = f_{cl}(\state) = f(\state) + g(\state)k(\state)$ yields a solution $\state(t)$, with initial condition $\state(0) = \state_0$.
% 
% \textbf{Definition 1 Forward Invariance:} The set $\mathcal{C}$ for every $x_0 \in \mathcal{C}$, $x(t) \in \mathcal{C}$ and all $t \in I(x_0)$. Then the set $\mathcal{C}$ is forward invariant \\
% \\
% \textbf{Definition 2 set $\mathcal{C}$:} 
Consider a set $\mathcal{C}$ defined as the \textit{super-level set} of a continuously differentiable function $h:\mathcal{D}\subseteq \mathbb{R}^n \rightarrow \mathbb{R}$ yielding,
\begin{align}
\label{eq:setc1}
	\mathcal{C}                        & = \{ \state \in \mathcal{D} \subset \mathbb{R}^n : h(\state) \geq 0\} \\
\label{eq:setc2}
	\partial\mathcal{C}                & = \{ \state \in \mathcal{D} \subset \mathbb{R}^n : h(\state) = 0\}\\
\label{eq:setc3}
	\text{Int}\left(\mathcal{C}\right) & = \{ \state \in \mathcal{D} \subset \mathbb{R}^n : h(\state) > 0\}
\end{align}
It is assumed that $\text{Int}\left(\mathcal{C}\right)$ is non-empty and $\mathcal{C}$ has no isolated points, i.e. $\text{Int}\left(\mathcal{C}\right) \neq \phi$ and $\overline{\text{Int}\left(\mathcal{C}\right)} = \mathcal{C}$. %We refer $\mathcal{C}$ as a \underline{safe set}.\\
The system is safe w.r.t. the control law $u = k(\state)$ if
% \begin{equation}
	$\forall \: \state(0) \in \mathcal{C} \implies \state(t) \in \mathcal{C} \;\;\; \forall t \geq 0$.
% \end{equation}
We can mathematically verify if the controller $k(\state)$ is safeguarding or not by using Control Barrier Functions (CBFs), which is defined next.

\begin{definition}[Control barrier function (CBF)]{\it
\label{definition: CBF definition}
% \textbf{Definition 3 Control barrier function - CBF}
Given the set $\mathcal{C}$ defined by \eqref{eq:setc1}-\eqref{eq:setc3}, with $\frac{\partial h}{\partial \state}(\state) \neq 0\; \forall \state \in \partial \mathcal{C}$, the function $h$ is called the control barrier function (CBF) defined on the set $\mathcal{D}$, if there exists an extended \textit{class} $\mathcal{K}$ function $\kappa$ such that for all $\state \in \mathcal{D}$:

\begin{equation}
\begin{aligned}
    \underbrace{\text{sup}}_{ u \in \mathbb{U}}\! \left[\underbrace{\mathcal{L}_{f} h(\state) + \mathcal{L}_g h(\state)u} \iffalse+ \frac{\partial h}{\partial t}\fi_{\dot{h}\left(\state, u\right)} \! + \kappa\left(h(\state)\right)\right] \! \geq \! 0
% \underbrace{\text{sup}}_{ u \in \mathbb{U} }\! \left[\underbrace{\mathcal{L}_{f} h(\state, t) + \mathcal{L}_g h(\state, t) + \frac{\partial h(\state, t)}{\partial t}}_{\dot{h}\left(\state, t, u\right)} \! + \kappa\left(h(\state, t)\right)\right] \! \geq \! 0
\end{aligned}
\end{equation}
where $\mathcal{L}_{f} h(\state) = \frac{\partial h}{\partial \state}f(\state)$ and $\mathcal{L}_{g} h(\state)= \frac{\partial h}{\partial \state}g(\state)$ are the Lie derivatives. 
% $\kappa : [0,\infty) \to [0,\infty)$ a strictly increasing continuous function with $\kappa(0)=0$. Formally, $\kappa$ is known as a class $\mathcal{K}$ function.
%Then we say that $h$ is a control barrier function (CBF) defined on the set $\mathcal{D}$.
% where $\mathcal{L}_{f} h(\state, t) = \frac{\partial h}{\partial \state}f(\state, t)$ and $\mathcal{L}_{g} h(\state, t) = \frac{\partial h}{\partial \state}g(\state, t)$. Then we say that h is a control barrier function (CBF) on $\mathcal{C}$
}
\end{definition}
% \textbf{Safety Lemma (extended to time varying case)} \cite{IGARASHI2019735}
% The set $\mathcal{C} \subset \mathbb{R}^n$ be a set defined on the super-level set of a continuously differentiable function $h : \mathcal{D} \subset \mathbb{R}^n \rightarrow \mathbb{R}$. 

% For time varying CBFs, an additional term $\frac{\partial h}{\partial t}$ would be added as mentioned in \cite{IGARASHI2019735}. 
Given this definition of a CBF, we know from \cite{Ames_2017} and \cite{8796030} that any Lipschitz continuous control law $k(\state)$ satisfying the inequality: $\dot{h} + \kappa( h )\geq 0$ ensures safety of $\mathcal{C}$ if $x(0)\in \mathcal{C}$, and asymptotic convergence to $\mathcal{C}$ if $x(0)$ is outside of $\mathcal{C}$. %The conditions derived for the CBF are, in fact, necessary and sufficient for forward invariance.
% \par It can also be interpreted that the existence of a control barrier function implies that the control system is safe.





\subsection{Safety Filter Design}
\label{subsection: safe_controller}
Having described the CBF, we can now describe the Quadratic Programming (QP) formulation of CBFs. CBFs act as \textit{safety filters} which take the desired input $u_{des}(\state,t)$ and modify this input in a minimal way: 

\begin{equation}
\begin{aligned}
\label{eqn: CBF QP}
u^{*}(x,t) &= \argmin_{u \in \mathbb{U} \subseteq \mathbb{R}^m} \norm{u - u_{des}(x,t)}^2\\
\quad & \textrm{s.t. } \mathcal{L}_f h(x) + \mathcal{L}_g h(x)u + \iffalse \frac{\partial h}{\partial t} +\fi \kappa \left(h(x)\right) \geq 0\\
\end{aligned}
\end{equation}
This is called the Control Barrier Function based Quadratic Program (CBF-QP). The CBF-QP control $u^{*}$ can be obtained by solving the above optimization problem using KKT conditions.

% \subsubsection{Ellipse-CBF Candidate}
% Consider the following CBF candidate:
% % \begin{tcolorbox}
% \begin{equation}
%     h(\state,t) = \left(\frac{c_x(t) - x_p}{c_1}\right)^2 + \left(\frac{c_y(t) - y_p}{c_2}\right)^2 + \left(\frac{c_z(t) - z_p}{c_3}\right)^2 - 1,
%     \label{eqn:Ellipse-CBF}
% \end{equation}
% % \end{tcolorbox}
% which approximates an obstacle with an ellipse with center $(c_x(t), c_y(t), c_z(t))$ and axis lengths $c_1,c_2,c_3$. 
% Since $h$ in \eqref{eqn:Ellipse-CBF} is dependent on time (e.g. moving obstacles), the resulting set $\mathcal{C}$ is also dependent on time. To analyze this class of sets, time dependent versions of CBFs can be used \cite{IGARASHI2019735}. Alternatively, we can reformulate our problem to treat the obstacle position $c_x,c_y,c_z$ as states, with their derivatives being constants. This will allow us to continue using the classical CBF given by Definition \ref{definition: CBF definition} including its properties on safety. The derivative of \eqref{eqn:Ellipse-CBF} is
% \begin{align}
%     \dot h = & {2 (c_x - x_p) (\dot c_x - v_x)/c_1^2} + {2 (c_y - y_p) ( \dot c_y - v_y)/c_2^2} \nonumber \\
%     & + {2 (c_z - z_p) ( \dot c_z - v_z )/c_3^2},
% \end{align}
% which has no dependency on the inputs $f_1,f_2,f_3,f_4$. Hence, $h$ will not be a valid CBF for the acceleration based model \eqref{eqn:quadrotor_model}. 



% We consider three different forms of class $\mathcal{K}$ functions in \cite[Definition 8]{DBLP:journals/corr/abs-1903-04706} (with a penalty p $>$ 0): 

% Form 1: $\alpha_{1}$ is square root, $\alpha_{2}$ is linear:
% % \begin{tcolorbox}
% \begin{equation}
% \begin{aligned}
% \label{eqn:HO-CBF-1}
% \psi_{1}(x,t) &= \dot{b}(x,t) + p\sqrt{b(x,t)} \\
% \psi_{2}(x,t) &= \dot{\psi_{1}}(x,t) + p(\psi_{1}(x,t))\\
% \end{aligned}
% \end{equation}

% Form 2: $\alpha_{1}$ and $\alpha_{2}$ are linear:
% % \begin{tcolorbox}
% \begin{equation}
% \begin{aligned}
% \label{eqn:HO-CBF-2}
% \psi_{1}(x,t) &= \dot{b}(x,t) + p{(b(x,t))} \\
% \psi_{2}(x,t) &= \dot{\psi_{1}}(x,t) + p(\psi_{1}(x,t))\\
% \end{aligned}
% \end{equation}

% Form 3: $\alpha_{1}$ is square, $\alpha_{2}$ is linear:
% % \begin{tcolorbox}
% \begin{equation}
% \begin{aligned}
% \label{eqn:HO-CBF-3}
% \psi_{1}(x,t) &= \dot{b}(x,t) + p{(b(x,t))}^2 \\
% \psi_{2}(x,t) &= \dot{\psi_{1}}(x,t) + p(\psi_{1}(x,t))\\
% \end{aligned}
% \end{equation}
% \begin{figure}
%        \centering
%         \begin{subfigure}[b]{0.33\textwidth}
%         \includegraphics[width=\textwidth]{images/C3BF_1.pdf}
%         \caption{}
%         \end{subfigure}
%         %
%         \begin{subfigure}[b]{0.33\textwidth}
%         \includegraphics[width=\textwidth]{images/C3BF_2.pdf}
%         \caption{}
%         \end{subfigure}
%         %
%         \begin{subfigure}[b]{0.33\textwidth}
%         \includegraphics[width=\textwidth]{images/C3BF_3.pdf}
%         \caption{}
%         \end{subfigure}
%         \caption{Construction of collision cone for an elliptical obstacle considering the quadrotor's dimensions (width: $w$).}
%         \label{fig:C3BF}
% \end{figure}
\subsection{Collision Cone CBF (C3BF) candidate for quadrotors}
\label{subsection: C3BF}
We now formally introduce the proposed CBF candidate for quadrotors. Let us assume that the obstacle is centered at $(c_x(t), c_y(t), c_z(t))$ and with dimensions $c_1,c_2,c_3$. We assume that $c_x(t),c_y(t), c_z(t)$ are differentiable and their derivatives are piece-wise constants.
%The relative degree of the safety constraint in our case is two. 
The proposed approach combines the idea of potential unsafe directions given by collision cone (Fig. \ref{fig:3D CBF}, \ref{fig:Projection CBF}) as an unsafe set to formulate a CBF as in \cite{C3BF}.
Consider the following CBF candidate:
% \begin{tcolorbox}
\begin{equation}
    h(\state, t) = < \prel, \vrel> + \| \prel\|\| \vrel\|\cos\phi ,
    \label{eqn:CC-CBF}
\end{equation}
% \end{tcolorbox}
where $\prel$ is the relative position vector between the body center of the quadrotor and the center of the obstacle, $\vrel$ is the relative velocity, $<\cdot , \cdot>$ is the dot product of 2 vectors and $\phi$ is the half angle of the cone, the expression of $\cos\phi$ is given by $\frac{\sqrt{\|\prel\|^2 - r^2}}{\|\prel\|}$ (see Fig. \ref{fig:3D CBF}, \ref{fig:Projection CBF}). Precise mathematical definitions for $\prel, \vrel$ will be given in the next section. The proposed constraint simply ensures that the angle between $\prel, \vrel$ is less than $180^\circ - \phi$.

In \cite{C3BF}, it was shown that the proposed candidate \eqref{eqn:CC-CBF} is valid CBF for wheeled mobile robots, i.e., the unicycle and bicycle. With this result, CBF-QPs were constructed that yielded collision-avoiding behaviors in these models. We aim to extend this to the class of quadrotors. %This is described next.
% \begin{figure}
%     \centering
%     % \includegraphics[width=0.7\linewidth]{images/screenshot049}
% %     \begin{tikzpicture}[
% %       collisioncone/.style={shape=rectangle, fill=red, line width=2, opacity=0.30},
% %       obstacleellipse/.style={shape=rectangle, fill=blue, line width=2, opacity=0.35},
% %     ]
        
% %         \def\r{1.32003}; % radius
% %         % \def\r{sqrt{1.75^2 + 0.75^2}} % radius
% %         \def\q{-3.5}; % distance center-external point q = |OQ|
% %         \def\x{{\r^2/\q}}; % Q x coordinate
% %         \def\y{{\r*sqrt(1-(\r/\q)^2}}; % Q y coordinate
% %         \def\z{{\q - abs(\q - (\r^2/\q))}};
% %         \coordinate (Q) at (\q,0); % external point Q
% %         \coordinate (P) at (\x,\y); % point of tangency, P
% %         \coordinate (O) at (0.0, 0); % center of circles and ellipse
% %         \coordinate (E) at (\q, 0); % Ego COM
% %         \coordinate (K) at (\x, {-\y}); % Other tangent
% %         \coordinate (H) at (\z, \y);
% %         \coordinate (I) at (\z, {-\y});
        
% %         \draw[name path = aux, red!60, very thick, dashed] (O) circle (1.32003);
% %         %\draw[name path global = circle P] (O) circle(\r);
% %         \draw[blue!50, thick, fill=blue!20] (O) ellipse (1.20 and 0.55);
% %         \draw[black, thick] (E) -- (O) node [midway, below] {$\|\prel\|$};
% %         \node[anchor=south west,inner sep=0] at (-4.4,-1) {\includegraphics[width=0.20\linewidth] {images/crazyflie_2.1.jpg}};
        
% %         %\draw[->] (0,-1.3*\r) -- (0,1.5*\r); % Axes
% %         %\draw[->] (-1.3*\r,0) -- (\q+0.4*\r,0);
        
% %         \draw[black, thick, name path = tangent] ($(Q)!-0.0!(P)$) -- ($(Q)!1.3!(P)$); % tangent
% %         % \draw[name path = tangent, black, thick] (Q) -- (P);
% %         \draw[black, thick, name path = normal] ($(O)!-0.0!(P)$) -- ($(O)!1.4!(P)$);
% %         % \draw[name path = normal, black, thick] (O) -- (P);
% %         \draw[black, thick] ($(Q)!-0.0!(K)$) -- ($(Q)!1.3!(K)$);
        
% %         % Reflected Cone
% %         \draw[black, thick, name path = tangent, dashed] ($(Q)!-0.0!(H)$) -- ($(Q)!1.1!(H)$);
% %         \draw[black, thick, dashed] ($(Q)!-0.0!(I)$) -- ($(Q)!1.1!(I)$);
        
% %         % \path [name intersections={of = aux and normal}];
% %         % \coordinate[label=below left:X] (X) at (intersection-1);
% %         % \fill [black] (X) circle (2pt);
        
% %         \tkzMarkRightAngle[draw=black,size=.2](O,P,Q);
% %         \tkzMarkAngle[draw=black, size=0.75](O,Q,P);
% %         \tkzLabelAngle[dist=1.0](O,Q,P){$\phi$};
        
% %         \path[shade, left color=red, right color = red, opacity=0.2] (E) -- (H) -- (I) -- cycle;
        
% %         \fill [black] (E) circle (1pt) node[anchor=north, black] (n1) {$(x,y,z)$};
% %         \fill [blue] (O) circle (2pt) node[anchor=north, blue] (n1) {$(c_x, c_y)$} node[anchor=south east, blue] (n1) {O}; 
% %         \fill [black] (P) circle (2pt) node[anchor=south, black] (n1) {$P$};
% %         \fill [black] (K) circle (2pt) node[anchor=north, black] (n1) {$K$};
% %         \fill [black] (H) circle (2pt) node[anchor=south, black] (n2) {$H$};
% %         \fill [black] (I) circle (2pt) node[anchor=north, black] (n1) {$I$};
        
% %         % Arrow labelling lines
% %         % \draw [<->, color=black, thin, dashed] ([xshift=5 pt, yshift=0 pt]X) -- ([xshift=5 pt, yshift=0 pt]O) node [midway, right] {$a$};
% %         % \draw [<->, color=black, thin, dashed] ([xshift=5 pt, yshift=0 pt]X) -- ([xshift=5 pt, yshift=0 pt]P) node [midway, right] {$\frac{w}{2}$};
% %         \draw [<->, color=black, thick, dashed] ([xshift=5 pt, yshift=0 pt]O) -- ([xshift=5 pt, yshift=0 pt]P) node [midway, right] {$r = a+\frac{w}{2}$};
% %         \draw [<->, color=black, thick, dashed] (O) -- (1.20, 0) node [midway, above] {$a$};
        
% %         % Legend
% %         \matrix [above right,nodes in empty cells, matrix of nodes, column sep=0.5cm, inner sep=6pt] at (current bounding box.north west) {
% %           \node [collisioncone,label=right:{\footnotesize Collision Cone}] {}; &
% %           \node [obstacleellipse,label=right:{\footnotesize Obstacle Ellipse}] {}; \\
% %         };
% %     \end{tikzpicture}
% %     \caption{Construction of collision cone for an elliptical obstacle considering the quadrotor's dimensions (width: $w$).} %The goal is to ensure that the velocity vector does not fall in the red shaded region EHI.}
% %     \label{Fig:Construction of Collision Cone}
% % \end{figure}

