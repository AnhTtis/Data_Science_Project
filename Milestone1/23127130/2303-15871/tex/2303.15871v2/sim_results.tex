\par We have validated the C3BF-QP based controller on quadrotors for both 3D and Projection CBF cases. We have used a PD controller as a reference controller to track the desired path with constant target velocities. Note that the reference controller can be replaced by any existing trajectory tracking/path-planning like the MPC \cite{9483029}. For the class $\mathcal{K}$ function in the CBF inequality, we chose $\kappa(h) = \gamma h$, where $\gamma=1$.

\begin{figure}
       \centering
        \begin{subfigure}[b]{0.46\textwidth}
        \includegraphics[width=\textwidth]{images/st-side.pdf}
        \caption{}
        \end{subfigure}
        %
        \begin{subfigure}[b]{0.20\textwidth}
        \includegraphics[width=\textwidth]{images/st-up-si.pdf}
        \caption{}
        \end{subfigure}
        %
        \begin{subfigure}[b]{0.235\textwidth}
        \includegraphics[width=\textwidth]{images/st-up-sa.pdf}
        \caption{}
        \end{subfigure}
        % %
        % \begin{subfigure}[b]{0.48\textwidth}
        % \includegraphics[width=\textwidth]{images/Pybullet.pdf}
        % \end{subfigure}
        \caption{Interaction with static obstacles: overtaking (a), (b), and braking (c) behavior of the quadrotor, Section \ref{section: 3D-CBF}.In all these cases the reference velocity of the quadrotor is 1m/s.}
        \label{fig:static-obs}
    \end{figure}

\begin{table}[b]

\begin{tabular}{|l|l|l|}
\hline
\textbf{Variables} & \textbf{Definition}         & \textbf{Value} \\ \hline
g                  & Gravitational acceleration  & $9.81 kg \cdot m/s^2$   \\ \hline
m                  & Mass of quadrotor           & $0.027 kg$        \\ \hline
L                  & Distance between two opp. rotors & 0.130 $m$         \\ \hline
l                  & Distance of center from base & $0.014 m$         \\ \hline
Ix, Iy             & Inertia about x, y-axis  & $2.39\cdot 10^{-5} kg \cdot m^2$    \\ \hline
Iz                 & Inertia about z-axis       & $3.23\cdot 10^{-5} kg \cdot m^2$    \\ \hline
kf                 & Motor’s thrust constant     & $3.16 \cdot 10^{-10}$          \\ \hline
km                 & Motor’s torque constant     & $7.94 \cdot 10^{-12}$          \\ \hline
\end{tabular}
\caption{Modelling parameters of Crazyflie}
\label{table:quadrotor_parameters}

\end{table}

\subsection{Simulation setup}
The simulations were conducted using the multi-drone environment \cite{pybullet-drones} on Pybullet \cite{coumans2019}, a Python-based physics simulation engine. The parameters of Crazyflie are tabulated in \ref{table:quadrotor_parameters}. Having presented our proposed control method design, we now test our framework under three different scenarios to illustrate the controller performance. These scenarios include the interaction of a quadrotor with (1) a static obstacle (3D case) Fig. \ref{fig:static-obs}, (2) a moving obstacle (3D case) Fig. \ref{fig:moving-obs} and (3) an elongated obstacle (Projection case) Fig. \ref{fig:long-obs}.

% \subsubsection{Interaction with static obstacles}
% Fig. \ref{fig:static-obs} shows the overtaking (a, b), and braking (c) behavior of the quadrotor while interacting with the static obstacle (which is another quadrotor).  

% \subsubsection{Interaction with moving obstacles}
% Fig. \ref{fig:moving-obs} shows the overtaking (a), (b), slowing (c), and reversing (d) behavior of the quadrotor while interacting with the moving obstacle (which is another quadrotor). 
\begin{figure}
       \centering
        \begin{subfigure}[b]{0.33\textwidth}
        \includegraphics[width=\textwidth]{images/mov-side.pdf}
        \caption{}
        \end{subfigure}
        %
        \begin{subfigure}[b]{0.14\textwidth}
        \includegraphics[width=\textwidth]{images/mov-up-si.pdf}
        \caption{}
        \end{subfigure}
        %
        \begin{subfigure}[b]{0.235\textwidth}
        \includegraphics[width=\textwidth]{images/mov-up-sa.pdf}
        \caption{}
        \end{subfigure}
        \begin{subfigure}[b]{0.235\textwidth}
        \includegraphics[width=\textwidth]{images/mov-dn-sa.pdf}
        \caption{}
        \end{subfigure}
        % %
        % \begin{subfigure}[b]{0.48\textwidth}
        % \includegraphics[width=\textwidth]{images/Pybullet.pdf}
        % \end{subfigure}
        \caption{Interaction with moving obstacles: overtaking (a), (b), slowing (c), and reversing (d) behavior of the quadrotor, section \ref{section: 3D-CBF}. In all these cases the reference velocity of the quadrotor is 1m/s and the obstacle quadrotor speed is 1m/s in case (a) and 0.1 m/s in (b),(c),(d).}
        \label{fig:moving-obs}
    \end{figure}

% \subsubsection{Interaction with long obstacles}
% Fig. \ref{fig:long-obs} shows the quadrotor moving from side and top in (a), (b) respectively while interacting with an elongated obstacle. 

    \begin{figure}
       \centering
        \begin{subfigure}[b]{0.22\textwidth}
        \includegraphics[width=\textwidth]{images/proj-ver.pdf}
        \caption{}
        \end{subfigure}
        %
        \begin{subfigure}[b]{0.22\textwidth}
        \includegraphics[width=\textwidth]{images/proj-hor.pdf}
        \caption{}
        \end{subfigure}
        % %
        % \begin{subfigure}[b]{0.48\textwidth}
        % \includegraphics[width=\textwidth]{images/Pybullet.pdf}
        % \end{subfigure}
        \caption{Interaction with longer obstacles: moving from side (a) and top (b), Section \ref{section: proj-CBF}. In all these cases the reference velocity of the quadrotor is 1m/s. }
        \label{fig:long-obs}
    \end{figure}

\subsection{ Experimental Results}
% \begin{figure*}
%        \centering
%         \begin{subfigure}[b]{0.32\textwidth}
%         \includegraphics[width=\textwidth]{images/cylinder_path.pdf}
%         \caption{}
%         \label{fig:single_cyl_path}
%         \end{subfigure}
%         %
%         \begin{subfigure}[b]{0.32\textwidth}
%         \includegraphics[width=\textwidth]{images/cylinder_vel.pdf}
%         \caption{}
%         \label{fig:single_cyl_vel}
%         \end{subfigure}
%         %
%         \begin{subfigure}[b]{0.32\textwidth}
%         \includegraphics[width=\textwidth]{images/cylinder_thr.pdf}
%         \caption{}
%         \label{fig:single_cyl_inp}
%         \end{subfigure}
%         % %
%         % \begin{subfigure}[b]{0.48\textwidth}
%         % \includegraphics[width=\textwidth]{images/Pybullet.pdf}
%         % \end{subfigure}
%         \caption{Experimental results with a single cylindrical obstacle. a) Traced Path. b) Evolution of linear velocity. c) Evolution of control inputs.}
%         \label{fig:single_cyl_exp}
%     \end{figure*}


%     \begin{figure*}
%        \centering
%         \begin{subfigure}[b]{0.32\textwidth}
%         \includegraphics[width=\textwidth]{images/double_cylinder_path.pdf}
%         \caption{}
%         \label{fig:double_cyl_path}
%         \end{subfigure}
%         %
%         \begin{subfigure}[b]{0.32\textwidth}
%         \includegraphics[width=\textwidth]{images/double_cylinder_vel.pdf}
%         \caption{}
%         \label{fig:double_cyl_vel}

%         \end{subfigure}
%         %
%         \begin{subfigure}[b]{0.32\textwidth}
%         \includegraphics[width=\textwidth]{images/double_cylinder_thr.pdf}
%         \caption{}
%         \label{fig:double_cyl_input}
%         \end{subfigure}
%         % %
%         % \begin{subfigure}[b]{0.48\textwidth}
%         % \includegraphics[width=\textwidth]{images/Pybullet.pdf}
%         % \end{subfigure}
%         \caption{Experimental results with a two cylindrical obstacle. a) Traced Path. b) Evolution of linear velocity. c) Evolution of control inputs.}
%         \label{fig:double_cyl_exp}
%     \end{figure*}

    
%     \begin{figure*}
%        \centering
%         \begin{subfigure}[b]{0.32\textwidth}
%         \includegraphics[width=\textwidth]{images/moving_obs_path.pdf}
%         \caption{}
%         \label{fig:moving_obs_path}
%         \end{subfigure}
%         %
%         \begin{subfigure}[b]{0.32\textwidth}
%         \includegraphics[width=\textwidth]{images/moving_obs_vel.pdf}
%         \caption{}
%         \label{fig:moving_obs_vel}
%         \end{subfigure}
%         %
%         \begin{subfigure}[b]{0.32\textwidth}
%         \includegraphics[width=\textwidth]{images/moving_obs_thr.pdf}
%         \caption{}
%         \label{fig:moving_obs_inp}
%         \end{subfigure}
%         % %
%         % \begin{subfigure}[b]{0.48\textwidth}
%         % \includegraphics[width=\textwidth]{images/Pybullet.pdf}
%         % \end{subfigure}
%         \caption{Experimental results with a moving obstacle. a) Traced Path. b) Evolution of linear velocity. c) Evolution of control inputs.}
%         \label{fig:moving_obs_expt}
%     \end{figure*}

The experimental results with Bitcraze\textsuperscript{\texttrademark} Crazyflie 2.1 aerial drone are presented to demonstrate the efficacy of the C3BF controller framework. The global position of the drone as well as the obstacle is measured using Qualisys\textsuperscript{\texttrademark} Miqus M3 motion capture system with a tracking frequency of 100 $Hz$. Further, for the drone, the global position from the motion capture system is fused with the onboard IMU data via the Extended Kalman Filter to get the filtered state. The control commands are generated by an off-board computer and transmitted to the drone via a radio link. The communication with the drone is facilitated through the Crazyflie Python library \cite{cfclient}. Experiments are performed for the cases with a single static obstacle, multiple static obstacles, and moving obstacles. The graphs and videos of hardware experiments are available here\footnote{\label{note: exp videos link} \url{https://tayalmanan28.github.io/C3BF-UAV/}}.
% The corresponds results are shown in Figs.~{\ref{fig:single_cyl_exp}-\ref{fig:moving_obs_expt}}. As is observed from Figs.~{\ref{fig:single_cyl_path}-\ref{fig:moving_obs_path}}, the quadrotor is able to successfully evade the obstacles. The corresponding command inputs and the resultant linear velocities are shown in Figs.~{\ref{fig:single_cyl_inp}-\ref{fig:moving_obs_inp}} and Figs.~{\ref{fig:single_cyl_vel}-\ref{fig:moving_obs_vel}}, respectfully. 
Hence, the experimental results verify the efficacy of the proposed scheme for obstacle avoidance.

\subsection{Comparison between C3BF and HO-CBF}
% Fig. \ref{fig:cc-ho-comp} shows the comparison of trajectories of the quadrotor when following C3BF and HO CBF with a static obstacle. 
All the aforementioned cases were tested with the HO-CBF to compare its performance against C3BF. We observe that the HO-CBF could not avoid a high-speed approaching obstacle. Moreover, it is not able to properly avoid the longer obstacles in the projection CBF case. These shortcomings of the Higher Order CBF are also demonstrated in the simulation video available here$^{\ref{note: exp videos link}}$.


\subsection{Robustness of C3BF}
Without changing the above control framework we can observe that the C3BF is robust in the following two cases: 

\subsubsection{Multiple Obstacles}
The quadrotor successfully navigates through a series of obstacles (both Spherical and Long obstacles) avoiding collisions and showcasing robustness as in Fig. \ref{fig:robustness} (a) \& (b). 

\subsubsection{Multiple quadrotors with C3BF-QPs}
In multi-agent scenarios, where multiple quadrotors employ the collision cone CBF-QP (Fig. \ref{fig:robustness} (c)), both the ego-quadrotor and the approaching quadrotor are able to avoid collision in different configurations (static or moving), thus demonstrating robustness with respect to obstacles following the same Collision Cone CBF controller. 
% The supplementary video$^{\ref{note: exp videos link}}$ shows the simulation video of all the scenarios shown in Fig \ref{fig:robustness}.

\begin{figure}
       \centering
        \begin{subfigure}[b]{0.46\textwidth}
        \includegraphics[width=\textwidth]{images/MO-3D.pdf}
        \caption{}
        \end{subfigure}
        %
        \begin{subfigure}[b]{0.155\textwidth}
        \includegraphics[width=\textwidth]{images/MO-proj.pdf}
        \caption{}
        \end{subfigure}
        %
        \begin{subfigure}[b]{0.29\textwidth}
        \includegraphics[width=\textwidth]{images/Robustness.pdf}
        \caption{}
        \end{subfigure}
        % %
        % \begin{subfigure}[b]{0.48\textwidth}
        % \includegraphics[width=\textwidth]{images/Pybullet.pdf}
        % \end{subfigure}
        \caption{Robustness in scenarios with multiple obstacles (a), (b) and with obstacle also following Collision Cone CBF(c).}
        \label{fig:robustness}
    \end{figure}
