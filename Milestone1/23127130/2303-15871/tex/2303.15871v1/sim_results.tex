\par We have validated the C3BF-QP based controller on quadrotors for both 3D and Projection CBF cases. The simulations were conducted using the multi-drone environment \cite{pybullet-drones} on Pybullet \cite{coumans2019}, a python-based physics simulation engine. The parameters of Crazyflie are tabulated in \ref{table:quadrotor_parameters}. PD Controller is used as a reference controller to track the desired path, and the safety controller deployed is given by Sections \ref{subsection: track_controller} \ref{subsection: safe_controller}. We chose constant target velocities for verifying the C3BF-QP. For the class $\mathcal{K}$ function in the CBF inequality, we chose $\kappa(h) = \gamma h$, where $\gamma=1$.

\begin{figure}
       \centering
        \begin{subfigure}[b]{0.46\textwidth}
        \includegraphics[width=\textwidth]{images/st-side.jpg}
        \caption{}
        \end{subfigure}
        %
        \begin{subfigure}[b]{0.20\textwidth}
        \includegraphics[width=\textwidth]{images/st-up-si.jpg}
        \caption{}
        \end{subfigure}
        %
        \begin{subfigure}[b]{0.235\textwidth}
        \includegraphics[width=\textwidth]{images/st-up-sa.jpg}
        \caption{}
        \end{subfigure}
        % %
        % \begin{subfigure}[b]{0.48\textwidth}
        % \includegraphics[width=\textwidth]{images/Pybullet.jpg}
        % \end{subfigure}
        \caption{Interaction with static obstacles: overtaking (a), (b), and braking (c) behavior of the quadrotor, Section \ref{section: 3D-CBF}.}
        \label{fig:static-obs}
    \end{figure}


\begin{table}[b]

\begin{tabular}{|l|l|l|}
\hline
\textbf{Variables} & \textbf{Definition}         & \textbf{Value} \\ \hline
g                  & Gravitational acceleration  & $9.81 kg \cdot m/s^2$   \\ \hline
m                  & Mass of quadrotor           & $0.027 kg$        \\ \hline
L                  & Distance between two opp. rotors & 0.130 $m$         \\ \hline
l                  & Distance of center from base & $0.014 m$         \\ \hline
Ix, Iy             & Inertia about x, y-axis  & $2.39\cdot 10^{-5} kg \cdot m^2$    \\ \hline
Iz                 & Inertia about z-axis       & $3.23\cdot 10^{-5} kg \cdot m^2$    \\ \hline
kf                 & Motor’s thrust constant     & $3.16 \cdot 10^{-10}$          \\ \hline
km                 & Motor’s torque constant     & $7.94 \cdot 10^{-12}$          \\ \hline
\end{tabular}
\caption{Modelling parameters of Crazyflie}
\label{table:quadrotor_parameters}

\end{table}

\subsection{Simulation setup}
Having presented our proposed control method design, we now test our framework under three different scenarios to illustrate the performance of the controller. These scenarios include the interaction of quadrotor with: (1) a static obstacle (3D case) Fig. \ref{fig:static-obs}, (2) a moving obstacle (3D case) Fig. \ref{fig:moving-obs} and (3) an elongated obstacle (Projection case) Fig. \ref{fig:long-obs}.

\subsubsection{Interaction with static obstacles}
Fig. \ref{fig:static-obs} shows the overtaking (a, b), and braking (c) behavior of the quadrotor while interacting with the static obstacle (which is another quadrotor). In all these cases the reference velocity of the quadrotor is 1m/s. 

\subsubsection{Interaction with moving obstacles}
Fig. \ref{fig:moving-obs} shows the overtaking (a), (b), slowing (c), and reversing (d) behavior of the quadrotor while interacting with the moving obstacle (which is another quadrotor). In all these cases the reference velocity of the quadrotor is 1m/s and the obstacle quadrotor speed is 1m/s in case (a) and 0.1 m/s in (b),(c),(d).
\begin{figure}
       \centering
        \begin{subfigure}[b]{0.33\textwidth}
        \includegraphics[width=\textwidth]{images/mov-side.jpg}
        \caption{}
        \end{subfigure}
        %
        \begin{subfigure}[b]{0.14\textwidth}
        \includegraphics[width=\textwidth]{images/mov-up-si.jpg}
        \caption{}
        \end{subfigure}
        %
        \begin{subfigure}[b]{0.235\textwidth}
        \includegraphics[width=\textwidth]{images/mov-up-sa.jpg}
        \caption{}
        \end{subfigure}
        \begin{subfigure}[b]{0.235\textwidth}
        \includegraphics[width=\textwidth]{images/mov-dn-sa.jpg}
        \caption{}
        \end{subfigure}
        % %
        % \begin{subfigure}[b]{0.48\textwidth}
        % \includegraphics[width=\textwidth]{images/Pybullet.jpg}
        % \end{subfigure}
        \caption{Interaction with moving obstacles: overtaking (a), (b), slowing (c), and reversing (d) behavior of the quadrotor, section \ref{section: 3D-CBF}}
        \label{fig:moving-obs}
    \end{figure}



\subsubsection{Interaction with long obstacles}
Fig. \ref{fig:long-obs} shows the quadrotor moving from side and top in (a), (b) respectively while interacting with an elongated obstacle. In all these cases the reference velocity of the quadrotor is 1m/s. 

    \begin{figure}
       \centering
        \begin{subfigure}[b]{0.22\textwidth}
        \includegraphics[width=\textwidth]{images/proj-ver.jpg}
        \caption{}
        \end{subfigure}
        %
        \begin{subfigure}[b]{0.22\textwidth}
        \includegraphics[width=\textwidth]{images/proj-hor.jpg}
        \caption{}
        \end{subfigure}
        % %
        % \begin{subfigure}[b]{0.48\textwidth}
        % \includegraphics[width=\textwidth]{images/Pybullet.jpg}
        % \end{subfigure}
        \caption{Interaction with longer obstacles: moving from side (a) and top (b), Section \ref{section: proj-CBF}. }
        \label{fig:long-obs}
    \end{figure}

\subsection{Comparison between C3BF and HO-CBF}
% Fig. \ref{fig:cc-ho-comp} shows the comparison of trajectories of the quadrotor when following C3BF and HO CBF with a static obstacle. 
All the aforementioned cases were tested with the HO-CBF to compare its performance against C3BF. We observe that the HO-CBF could not avoid a high-speed approaching obstacle. Moreover, it is not able to properly avoid the longer obstacles in the projection CBF case. These shortcomings of the Higher Order CBF are demonstrated in the supplementary video.


\subsection{Robustness of C3BF}
Without changing the above control framework we can observe that the C3BF is robust in the following two cases: 

\subsubsection{Multiple Obstacles}
We have considered the scenario where the quadrotor is made to move through a series of obstacles (both Spherical and Long obstacles) as in Fig. \ref{fig:robustness} (a) \& (b). We observe that the quadrotor is able to successfully navigate through this complex environment by avoiding all the obstacles, thus demonstrating robustness with respect to multiple obstacles.

\subsubsection{Multiple quadrotors with C3BF-QPs}
We have also considered the multi-agent scenarios where the multiple quadrotors have the collision cone CBF-QP operational as shown in Fig. \ref{fig:robustness} (c). We observe that both the ego-quadrotor and the approaching quadrotor are able to avoid collision in different configurations (static or moving), thus demonstrating robustness with respect to obstacles following the same Collision Cone CBF controller. 

The supplementary video shows the simulation video of all the scenarios shown in Fig \ref{fig:robustness}.

\begin{figure}
       \centering
        \begin{subfigure}[b]{0.46\textwidth}
        \includegraphics[width=\textwidth]{images/MO-3D.jpg}
        \caption{}
        \end{subfigure}
        %
        \begin{subfigure}[b]{0.155\textwidth}
        \includegraphics[width=\textwidth]{images/MO-proj.jpg}
        \caption{}
        \end{subfigure}
        %
        \begin{subfigure}[b]{0.29\textwidth}
        \includegraphics[width=\textwidth]{images/Robustness.jpg}
        \caption{}
        \end{subfigure}
        % %
        % \begin{subfigure}[b]{0.48\textwidth}
        % \includegraphics[width=\textwidth]{images/Pybullet.jpg}
        % \end{subfigure}
        \caption{Robustness in scenarios with multiple obstacles (a), (b) and with obstacle also following Collision Cone CBF(c).}
        \label{fig:robustness}
    \end{figure}
