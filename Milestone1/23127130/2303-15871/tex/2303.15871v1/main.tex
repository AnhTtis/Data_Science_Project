% \documentclass[conference]{IEEEtran}
\documentclass[letterpaper, 10 pt, conference]{ieeeconf}
\IEEEoverridecommandlockouts
% The preceding line is only needed to identify funding in the first footnote. If that is unneeded, please comment it out.

% \usepackage{hyperref}       % hyperlinks
\usepackage[colorlinks]{hyperref}
\hypersetup{
    colorlinks,
    linkcolor=black,
    citecolor=black,
    filecolor=magenta,
    urlcolor=cyan,
}

\usepackage{resizegather}

\usepackage{cite}
\usepackage{amsmath,amssymb,amsfonts,mathrsfs}
\usepackage{algorithmic}
\usepackage{graphicx}
\usepackage{textcomp}
\usepackage{xcolor}
\usepackage{tcolorbox}

\usepackage{tablefootnote}
\usepackage{threeparttable}
\usepackage{caption, subcaption}

\usepackage{tikz}
\usepackage{graphicx}
\usepackage{tkz-euclide}

\usetikzlibrary{positioning}
\usetikzlibrary{shapes}
\usetikzlibrary{shapes.misc}
\usetikzlibrary{shapes.geometric}
\usetikzlibrary{plotmarks}
\usetikzlibrary{intersections}
\usetikzlibrary{calc}
\usetikzlibrary{fit}
\usetikzlibrary{patterns,tikzmark}
\usetikzlibrary{matrix,decorations.pathreplacing,calc}


\tikzset{cross/.style={cross out, draw, 
         minimum size=2*(#1-\pgflinewidth), 
         inner sep=0pt, outer sep=0pt}}
\definecolor{purple}{rgb}{1, 0, 1}

\newcommand{\ie}{\emph{i.e.,}\xspace}
\newcommand{\eg}{\emph{e.g.,}\xspace}
\newcommand{\abr}{\emph{abbr.}\xspace}
\newcommand{\ea}{\emph{et al.}\xspace}
\newcommand{\gensync}{\emph{GenSync}\xspace}
\newcommand{\colosseum}{\emph{Colosseum}\xspace}
\newcommand{\srep}{\emph{SREP}\xspace} % Set Reconciliation Enhances
\newcommand{\srepsim}{\emph{SREPSim}\xspace}
% Propagation
\newcommand{\esrep}{\emph{E-SREP}\xspace}
\newcommand{\epsrep}{\emph{EP-SREP}\xspace}
\newcommand{\mesrep}{\emph{ME-SREP}\xspace}
\newcommand{\mempoolsync}{\emph{MempoolSync}}

\newcommand{\fref}[1]{Fig.~\ref{#1}}
\newcommand{\tref}[1]{Table~\ref{#1}}
\newcommand{\aref}[1]{Algorithm~\ref{#1}}
\newcommand{\procref}[1]{Procedure~\ref{#1}}
\newcommand{\sref}[1]{Section~\ref{#1}}
\newcommand{\lineref}[1]{line~\ref{#1}}
\newcommand{\appref}[1]{Appendix~\ref{#1}}

% Change \eqref
\LetLtxMacro{\originaleqref}{\eqref}
\renewcommand{\eqref}{Eq.~\originaleqref}

% Theorems and corollaries
\newcounter{theoremcount}
\setcounter{theoremcount}{0}
\DeclareRobustCommand{\theorem}[1]{%
  \refstepcounter{theoremcount}%
  \noindent\textit{\textbf{Theorem \thetheoremcount\label{theorem:#1}: }}%
}
\DeclareRobustCommand{\theoremref}[1]{Theorem~\ref{theorem:#1}}

\DeclareRobustCommand{\proof}{\emph{Proof:}\xspace}
\DeclareRobustCommand{\qqed}{\hfill$\blacksquare$}

\newcounter{corollcount}
\setcounter{corollcount}{0}
\DeclareRobustCommand{\coroll}[1]{%
  \refstepcounter{corollcount}%
  \noindent\textit{\textbf{Corollary \thecorollcount\label{coroll:#1}: }}%
}
\DeclareRobustCommand{\corollref}[1]{Corollary~\ref{coroll:#1}}

\newcounter{lemmacount}
\setcounter{lemmacount}{0}
\DeclareRobustCommand{\lemma}[1]{%
  \refstepcounter{lemmacount}%
  \noindent\textit{\textbf{Lemma \thelemmacount\label{lemma:#1}: }}%
}
\DeclareRobustCommand{\lemmaref}[1]{Lemma~\ref{lemma:#1}}

\newcounter{definitioncount}
\setcounter{definitioncount}{0}
\DeclareRobustCommand{\definition}[1]{%
  \refstepcounter{definitioncount}%
  \noindent\textit{\textbf{Definition \thedefinitioncount\label{definition:#1}: }}%
}
\DeclareRobustCommand{\defref}[1]{Definition~\ref{definition:#1}}

%notes of different authors
\newif\ifnotes
\notestrue
\notesfalse

\newif\ifdiff
\difftrue
\difffalse

\newcommand{\anote}[1]{\ifnotes $\ll$\textsf{\textcolor{purple}{Ari: {#1}}}$\gg$ \fi}
\newcommand{\nnote}[1]{\ifnotes $\ll$\textsf{\textcolor{orange}{Novak: {#1}}}$\gg$ \fi}
\newcommand{\diff}[1]{\ifdiff\textcolor{orange}{#1}\else#1\fi}

%%% Local Variables:
%%% mode: latex
%%% TeX-master: "main"
%%% End:


\def\BibTeX{{\rm B\kern-.05em{\sc i\kern-.025em b}\kern-.08em
    T\kern-.1667em\lower.7ex\hbox{E}\kern-.125emX}}
    
\begin{document}

\title{Control Barrier Functions in Dynamic UAVs for Kinematic Obstacle Avoidance: A Collision Cone Approach}

\author{Manan Tayal, Shishir Kolathaya% <-this % stops a space
\thanks{This research was supported by the Pratiksha Young Investigator Fellowship and the SERB grant CRG/2021/008115.
}
% \thanks{*This work was not supported by any organization}% <-this % stops a space
\thanks{$^{1}$Robert Bosch Center for Cyber-Physical Systems (RBCCPS), Indian Institute of Science (IISc), Bengaluru.
{\tt\scriptsize \{manantayal, shishirk\}@iisc.ac.in}
.
}%
}

\maketitle
\begin{abstract}
 % In this paper, we propose a method that allows a highly dynamic system like quadrotor to follow a set path while avoiding highly kinematic obstacles (of various sizes) with safety guarantees. The safe region is constructed using concepts of control barrier functions (CBF) and collision cone while explicitly considering the nonlinear underactuated dynamics of the quadrotor. We also compare the performance of this controller with the state of the art Higher Order CBF based controller. We demonstrate the feasibility of our method on a quadrotor in simulation with static and moving obstacles of various sizes.
Unmanned aerial vehicles (UAVs), specifically quadrotors, have revolutionized various industries with their maneuverability and versatility, but their safe operation in dynamic environments heavily relies on effective collision avoidance techniques. 
% To this end, this paper presents a novel approach for enabling a quadrotor to navigate a desired route while avoiding kinematic obstacles with safety guarantees. The proposed technique incorporates control barrier functions by using the concept of collision cones. The main idea behind the collision cones (CC) is to realize a constraint inequality that ensures that the velocity of the quadrotor and the obstacle are always pointing away from each other. In other words, the proposed constraint ensures that the relative velocity always avoids assuming a cone of vectors (set of possible directions leading to collision). We formally show that for the quadrotor model, this constraint inequality is a valid control barrier function (CBF). We incorporate a real-time controller by using this CBF, called the CBF-QP and demonstrate collision avoidance in a simulated environment. 
This paper introduces a novel technique for safely navigating a quadrotor along a desired route while avoiding kinematic obstacles. The proposed approach employs control barrier functions and utilizes collision cones to ensure that the quadrotor's velocity and the obstacle's velocity always point away from each other. In particular, we propose a new constraint formulation that ensures that the relative velocity between the quadrotor and the obstacle always avoids a cone of vectors that may lead to a collision. 
By showing that the proposed constraint is a valid control barrier function (CBFs) for quadrotors, we are able to leverage on its real-time implementation via Quadratic Programs (QPs), called the CBF-QPs.
We validate the effectiveness of the proposed CBF-QPs by demonstrating collision avoidance with moving obstacles under multiple scenarios. This is shown in the pybullet simulator.
% We validate the proposed approach by demonstrating its effectiveness through the CBF-QP real-time controller in a simulated environment. 
Furthermore we compare the proposed approach with CBF-QPs shown in literature, especially the well-known higher order CBF-QPs (HO-CBF-QPs), where in we show that it is more conservative compared to the proposed approach. This comparison also shown in simulation in detail.
% safe set  Despite the non-linear and underactuated dynamics of the quadrotor, to form a real-time safety filter that guarantees safety in the dynamic environment. %to construct a safe region that is forward invariant. 
% The effectiveness of the controller is evaluated by comparing it to the existing state-of-the-art CBF formulations including the Higher Order CBFs. The approach is then demonstrated by implementing it on a quadrotor in a simulated environment with stationary and non-stationary obstacles of varying shapes and sizes, highlighting its potential for real-world applications.
% In this paper, we propose a novel Quadratic Program (QP) based control approach for quadrotor to navigate a cluttered environment, avoiding a range of kinematic obstacles safely. By incorporating concepts such as control barrier functions (CBF) \& collision cones and accounting for the nonlinear underactuated dynamics of the quadrotor, we can construct a safe region, which allows us to tackle these challenges. We assess the efficacy of our controller by comparing it to the state of the art  Higher Order CBF based controller. Finally, we demonstrate the potential of our approach by implementing it on a quadrotor in a simulated environment featuring obstacles of varying shapes, both stationary and nonstationary and showing its robustness in multiple obstacle environments.
\end{abstract}

% \begin{IEEEkeywords}
%     Kinematic obstacle avoidance, control barrier function, collision cone
% \end{IEEEkeywords}

\section{Introduction}
\label{section: Introduction}
\section{Introduction}
\label{sec:introduction}
% \begin{itemize}
%     % Diffusion of FL
%     \item {\st{Diffusion of FL}}
%     % Security threats to FL
%     \item {\st{Security threats to FL with particular focus on model poisoning}}
%     % Limitations of existing countermeasures
%     \item {\st{Current countermeasures (e.g., KRUM) and their limitations}}
%     % Proposed method and its advantages
%     \item {\st{Intuitive description of the proposed method and its difference (i.e., advantages) w.r.t. state of the art}}
%     % Main contributions
%     \item {\st{Summary of the main contributions of this work}}
%     % Paper's structure and organization
%     \item {\st{Paper's structure and organization}}
% \end{itemize}

% Diffusion of FL
Recently, {\em federated learning} (FL) has emerged as the leading paradigm for training distributed, large-scale, and privacy-preserving machine learning (ML) systems~\cite{mcmahan2017googleai,mcmahan2017aistats}. 
The core idea of FL is to allow multiple edge clients to collaboratively train a shared, global model without disclosing their local private training data.
%Specifically, an FL system consists of a central server and many edge clients; 
A typical FL round involves the following steps: {\em(i)} the server randomly picks some clients and sends them the current, global model; {\em(ii)} each selected client locally trains its model with its own private data; then, it sends the resulting local model to the server;\footnote{Whenever we refer to global/local model, we mean global/local model {\em parameters}.} {\em(iii)} the server updates the global model by computing an \emph{aggregation function}, usually the average (FedAvg), on the local models received from clients.
% \begin{enumerate}
%     \item[{\em(i)}] the server sends the current, global model to the clients and appoints some of them for training;
%     \item[{\em(ii)}] each selected client locally trains its copy of the global model with its own private data; then, it sends the resulting local model back to the server;\footnote{Whenever we refer to global/local model, we mean global/local model {\em parameters}.}
%     \item[{\em(iii)}] the server updates the global model by computing an \emph{aggregation function} on the local models received from clients (by default, the average, also referred to as FedAvg~\cite{mcmahan2017aistats}).
% \end{enumerate}
This process goes on until the global model converges. %(e.g., after a certain number of rounds or other similar stopping criteria).
%\\
% The advantages of FL over the traditional, centralized learning paradigm are undoubtedly clear in terms of flexibility/scalability (clients can join/disconnect from the FL network dynamically), network communications (only model weights\footnote{We will use \textit{parameters} and \textit{weights} interchangeably.} are exchanged between clients and server), and privacy (each client's private training data is kept local at the client's end and not uploaded to the server).
\\
% Security threats to FL
%However, the growing adoption of FL also raises security concerns~\cite{costa2022covert}, particularly about its confidentiality, integrity, and availability.
Although its advantages over standard ML, FL also raises security concerns~\cite{costa2022covert}. %, particularly about its confidentiality, integrity, and availability~\cite{costa2022covert}.
% OLD, LONG VERSION
% Indeed, some work deals with privacy leakage that may expose the local data of some clients~\cite{melis2019sp}. 
% A large body of work, instead, investigates attacks that usually aim to detriment the predictive accuracy of the learned global model. For instance, \emph{data poisoning} attacks achieve this goal by letting an adversary pollute the training set of some corrupt FL clients with maliciously crafted examples~\cite{jagielski2018sp}.
% Similarly, in \emph{model poisoning} the attacker attempts to tweak the global model weights~\cite{bhagoji2019pmlr} by directly perturbing the local model's weights of some infected FL clients before these are sent to the central server for aggregation, usually via so-called Byzantine attacks. 
% It turns out that Byzantine model poisoning attacks severely impact standard FedAvg; therefore, more robust aggregation functions must be designed to make FL systems secure.
Here, we focus on \emph{untargeted model poisoning} attacks~\cite{bhagoji2019pmlr}, where an adversary attempts to tweak the global model weights %\footnote{We will use the terms \textit{parameters} and \textit{weights} interchangeably.} 
by directly perturbing the local model's parameters of some infected clients before these are sent to the central server for aggregation.
In doing so, the adversary aims to jeopardize the global model \textit{indiscriminately} at inference time.
Such model poisoning attacks severely impact standard FedAvg; therefore, more robust aggregation functions must be designed to secure FL systems.
\\
% In this paper, we focus on designing a novel robust aggregation scheme at the server's end to contrast the effect of Byzantine model poisoning attacks.
%
% Current countermeasures and their limitations
%Several countermeasures have been proposed in the literature to combat model poisoning attacks on FL systems.
% Some methods use simple statistics more robust than plain average to smooth the impact of malicious updates (e.g., Trimmed Mean and FedMedian~\cite{yin2018icml}). 
% Other defenses implement outlier detection techniques to discard malicious updates from the aggregation performed at the server's end. Those are either based on heuristics (e.g., Krum/Multi-Krum~\cite{blanchard2017nips} and Bulyan~\cite{mhamdi2018pmlr}) or data-driven approaches (e.g., K-means clustering~\cite{shen2016acm} or DnC via spectral analysis~\cite{shejwalkar2021ndss}). 
% Finally, some strategies rely on a centralized ``source of trust'' to spot potential malicious updates (e.g., FLTrust~\cite{cao2020fltrust}).
% Several countermeasures have been proposed in the literature to combat model poisoning attacks on FL systems, i.e., to discard possible malicious local updates from the aggregation performed at the server's end. 
% These techniques range from simple statistics more robust than plain average (e.g., Trimmed Mean and FedMedian~\cite{yin2018icml}) to outlier detection heuristics (e.g., Krum/Multi-Krum~\cite{blanchard2017nips} and Bulyan~\cite{mhamdi2018pmlr}) or data-driven approaches (e.g., spectral analysis via K-means clustering~\cite{shen2016acm} or spectral analysis), or methods based on ``source of trust'' (e.g., FLTrust~\cite{cao2020fltrust}).
% OLD, LONG VERSION
%Several countermeasures have been proposed in the literature to combat Byzantine model poisoning attacks on FL systems.
% Descriptive statistics
% For example, Trimmed Mean and FedMedian aggregate local model updates using more robust statistics than standard average~\cite{yin2018icml}.
%
% % Heuristics for outlier detection
% Many existing Byzantine-resilient strategies implement some outlier detection heuristics to discard the model updates sent by potentially malicious clients from the input of the aggregation function.
% One of the most popular heuristics is Krum~\cite{blanchard2017nips}.
% This strategy tries to mitigate the impact of Byzantine attacks by selecting as a global model the local model with the smallest sum of Euclidean distances to {\em all} the other local models.
% Although powerful, Krum requires the server to know (or, at least, estimate) the number of malicious FL clients upfront, which is generally impossible in a realistic attack scenario. %
% Moreover, Krum may become ineffective for complex, high-dimensional model parameter spaces due to the curse of dimensionality.
% Bulyan~\cite{mhamdi2018pmlr} tries to overcome this issue by combining Krum with a variant of Trimmed Mean.
% % Data-driven outlier detection
% Other strategies use data-driven outlier detection techniques -- e.g., via K-means clustering~\cite{shen2016acm} -- to spot potential malicious local model updates. 
% %For instance, Shen et al. propose to cluster local model updates with K-means and thus identify outliers.
%
% % Other techniques
% As far as the server is concerned, any local model received can be from a potential malicious client. 
% FLTrust~\cite{cao2020fltrust} assumes the server acts as a client, i.e., trains a local model on an additional {\em trustworthy} dataset at the server's end and compares it against all the local models from other clients. 
% This way, the server can rely on some ``source of trust'' when discarding potentially malicious clients.
%\\
% Limitations of existing Byzantine-resilient strategies
Unfortunately, existing defense mechanisms either rely on simple heuristics (e.g., Trimmed Mean and FedMedian by~\cite{yin2018icml}) or need strong and unrealistic assumptions to work effectively (e.g., foreknowledge or estimation of the number of malicious clients in the FL system, as for Krum/Multi-Krum~\cite{blanchard2017nips} and Bulyan~\cite{mhamdi2018pmlr}, which, however, cannot exceed a fixed threshold).
Furthermore, outlier detection methods using K-means clustering~\cite{shen2016acm} or spectral analysis like DnC~\cite{shejwalkar2021ndss} do not directly consider the temporal evolution of local model updates received.
Finally, strategies like FLTrust~\cite{cao2020fltrust} require the server to collect its own dataset and act as a proper client, thereby altering the standard FL protocol.
\\
% OLD, LONG VERSION
% Overall, existing Byzantine-resilient strategies are either simple heuristics (e.g., FedMedian) or, if they are more complex, they rely on strong and unrealistic assumptions to work effectively (e.g., knowing the number of malicious clients in the FL system in advance, as for Krum and alike).
% Furthermore, data-driven outlier detection methods do not consider the temporary evolution of local model updates received (e.g., K-means clustering). 
% Finally, strategies like FLTrust requires the server to collect its own dataset and act as a proper client, thereby altering the standard FL protocol.
%
% Description of the proposed method
This work introduces a novel pre-aggregation \textit{filter} robust to untargeted model poisoning attacks. Notably, this filter $(i)$ operates without requiring prior knowledge or constraints on the number of malicious clients and $(ii)$ inherently integrates temporal dependencies. 
The FL server can employ this filter as a preprocessing step before applying \textit{any} aggregation function, be it standard like FedAvg or robust like Krum or Bulyan.
Specifically, we formulate the problem of identifying corrupted updates as a multidimensional (i.e., matrix-valued) time series anomaly detection task. 
The key idea is that legitimate local updates, resulting from well-calibrated iterative procedures like stochastic gradient descent (SGD) with an appropriate learning rate, show \textit{higher predictability} compared to malicious updates. This hypothesis stems from the fact that the sequence of gradients (thus, model parameters) observed during legitimate training exhibit regular patterns, as validated in Section~\ref{subsec:intuition}. %until convergence. 
%This regularity may be more pronounced for smooth convex loss functions, but it can still be captured within an appropriate time window, even for more complex and convoluted loss surfaces. 
%We provide evidence of this claim in Appendix~B, where we show that the average mutual information (i.e., ``predictability''), calculated over pairs of legitimate model updates sent at different FL rounds, is significantly higher than the corresponding computation for a malicious client.
\\
Inspired by the matrix autoregressive (MAR) framework for multidimensional time series forecasting~\cite{chen2021je}, we propose the FLANDERS ({\em \textbf{F}ederated \textbf{L}earning meets \textbf{AN}omaly \textbf{DE}tection for a \textbf{R}obust and \textbf{S}ecure}) filter.
The main advantages of FLANDERS over existing strategies like FLDetector~\cite{zhao2020multivariate} are its resilience to large-scale attacks, where $50\%$ or more FL participants are hostile, and the capability of working under realistic non-iid scenarios.
We attribute such a capability to two key factors: $(i)$ FLANDERS works without knowing a priori the ratio of corrupted clients, and $(ii)$ it embodies temporal dependencies between intra- and inter-client updates, quickly recognizing local model drifts caused by evil players. Below, we summarize our main contributions:

\begin{itemize}
\item[{\em(i)}]
We provide empirical evidence that the sequence of models sent by legitimate clients is more predictable than those of malicious participants performing untargeted model poisoning attacks.
\\
\item[{\em(ii)}] 
We introduce FLANDERS, the first pre-aggregation filter for FL robust to untargeted model poisoning based on multidimensional time series anomaly detection.
\\
\item[{\em(iii)}] 
We integrate FLANDERS into Flower,\footnote{\scriptsize{\url{https://flower.dev/}}} a popular FL simulation framework for reproducibility.
\\
\item[{\em(iv)}] 
We show that FLANDERS improves the robustness of the existing aggregation methods under multiple settings: different datasets, client's data distribution (non-iid), models, and attack scenarios.
\\
\item[{\em(v)}] 
We publicly release all the implementation code of FLANDERS along with our experiments.\footnote{\scriptsize{\url{https://anonymous.4open.science/r/flanders_exp-7EEB}}}
\end{itemize}

% Paper's structure and organization
The remainder of the paper is structured as follows. %some related work and the current state-of-the-art solutions to security issues that FL entails. 
Section~\ref{sec:background} covers background and preliminaries. 
In Section~\ref{sec:related}, we discuss related work.
Section~\ref{sec:problem} and Section~\ref{sec:method} describe the problem formulation and the method proposed. % to tackle it. 
Section~\ref{sec:experiments} gathers experimental results. %, and Section~\ref{sec:limitations} discusses some limitations of this work.
Finally, we conclude in Section~\ref{sec:conclusion}.
 %discusses the limitations of this work and draws future research directions.
%reports conclusions and draws perspectives for future research directions.

%%%%%%% OLD %%%%%%%
%to overcome the resilience of Byzantine failures in distributed Stochastic Gradient Descent computations. 
% The strength of Krum is its time complexity, which is linear in the gradient dimension. 
% However, the robustness of the approach is guaranteed for gradient-based learning applications only when the majority of the clients are not compromised. 
% Besides, the aggregation mechanism of Krum, as well as that of similar methods, is robust from a coarse-grained perspective and does not provide solutions to errors and perturbations that may occur at inference time.
%A related approach to~\cite{blanchard2017nips} is the work of Su et al.~\cite{su2016dc}. Here, the authors propose an iterated approximate agreement to tackle a multi-layer scenario attacked by Byzantine agents. 
%However, the method works efficiently on the sole discrete context and it is inapplicable to continuous state environments.
%\gabri{Maybe, we should just talk about the main limitations of existing countermeasures without digging into their details (or, we can just mention Krum as this is the most popular one). I will move the description of all these methods to the Related Work section.}

\section{Preliminaries}
\label{section: Background}
\section{Background on Network Calculus}
\label{sec: background}


\begin{figure*}[tbh]
\centering
\begin{subfigure}[b]{0.3\textwidth}
    \centering
    \includegraphics[width=\linewidth]{images/in-out.png}
    \caption{Arrival and departure data and their relation with delay $d(t)$ and backlog $b(t)$. For a FIFO system, the delay is the horizontal distance between $R(t)$ and $R^*(t)$ but some other multiplexing techniques may shift the data to a later priority, causing a longer delay.}
    \label{fig: data in-out}
\end{subfigure}
\hfill
\begin{subfigure}[b]{0.35\textwidth}
    \centering
    \includegraphics[width=\linewidth]{images/arrival-service.png}
    \caption{Characteristics of an arrival curve and a service curve. From any point of observation, the arriving data never exceeds its arrival curve; the departure data is also never less than the service curve with respect to the data arrival.}
    \label{fig: arrival-service curves}
\end{subfigure}
\hfill
\begin{subfigure}[b]{0.33\textwidth}
    \centering
    \includegraphics[width=\linewidth]{images/bound.png}
    \caption{Delay and backlog bounds of a system. Backlog is the maximum vertical distance between $\alpha(t)$ and $\beta(t)$; FIFO delay is their maximum horizontal distance; but for arbitrary multiplexing, the delay guarantee is when the system clears its buffer, thus it's the intersection of $\alpha(t)$ and $\beta(t)$.}
    \label{fig: system bounds}
\end{subfigure}
\caption{Network calculus framework. We let $R(t)$ and $R^*(t)$ be the arrival and departure data flow of a system; $\alpha(t)$ be the piecewise linear concave arrival curve and $\beta(t)$ be the piecewise linear convex service curve of a system.}
% \hossein{Better to show piece-wise linear concave arrival curve and piece-wise linear convex service curve instead of token-bucket and rate-latency.}}
\end{figure*}

We recall some of the network calculus essentials for a better understanding of the framework used in Saihu. In the following context, we use the following notation: $\mbb{R}^+$ is the set of non-negative real numbers; $[x]_+$ denotes $\max(0, x)$

The data flow is by convention modeled as a left-continuous wide-sense increasing function $R(t): \mbb{R}^+ \mapsto \mbb{R}^+$ with respect to time $t$~\cite{ncbook2001leboudec}. 

A system $\mcal{S}$ receives arrival data described as a cumulative function $R(t)$ and delivers departure data as another cumulative function $R^*(t)$. Figure~\ref{fig: data in-out} illustrates such a system $\mcal{S}$. The benefit of representing a system like this is that we can observe system backlog and delay with such a model. 

\begin{definition}[Backlog and Delay~\cite{ncbook2001leboudec}]
    The backlog of a system at time~$t$ is
    \begin{equation}
        b(t) = R(t) - R^*(t)
    \end{equation}
    
    The virtual delay of a FIFO system at time $t$ is
    \begin{equation}
        d_{FIFO}(t) = \inf \lbp \tau \geq 0 : R(t) \leq R^*(t+\tau) \rbp
    \end{equation}
\end{definition}



The backlog of a system can be viewed as the vertical distance between $R$ and $R^*$. The FIFO (\textit{First-in First-out}) delay is the horizontal distance between $R$ and $R^*$. One may obtain other delay values if the multiplexing technique is not FIFO.

% \begin{figure}
%     \centering
%     \includegraphics[width=0.9\linewidth]{images/in-out.png}
%     \caption{In/out data flow; delay and backlog}
%     \label{fig: data in-out}
% \end{figure}

Since we are interested in the system guarantee instead of a single instance of data flow, we would like to have general bounds to the arrival and departure data flows. Therefore, we define \textit{arrival curve} and \textit{service curve} as the bounds of arrival and departure data flows.

\begin{definition}[Arrival Curve~\cite{ncbook2001leboudec}]
    Given a wide-sense increasing function $\alpha: \mbb{R}^+ \mapsto \mbb{R}^+$, we say that a flow $R(t)$ is $\alpha$-constrained if and only if for all $s \leq t$:
    \begin{equation}
        R(t) - R(s) \leq \alpha(t-s)
    \end{equation}
    We say $R(t)$ has $\alpha$ as an arrival curve.
\end{definition}

\begin{definition}[Service Curve~\cite{ncbook2001leboudec}]
    Given a wide-sense increasing function $\beta: \mbb{R}^+ \mapsto \mbb{R}^+$ and $\beta(0) = 0$. A system $\mcal{S}$ having $R(t)$ and $R^*(t)$ as its arrival and departure flows. We say $\mcal{S}$ offers a service curve $\beta$ if and only if
    \begin{equation}
        R^*(t) \geq (R \otimes \beta)(t) =: \inf_{s \leq t} \lbp R(s) + \beta(t-s) \rbp
    \end{equation}
    where $\otimes$ denotes the min-plus convolution
\end{definition}

Figure~\ref{fig: arrival-service curves} illustrates the arrival and service curves. Any segment of arrival flow $R(t)$ is constrained by arrival curve $\alpha$ and the output curve $R^*(t)$ is always no less than the curve $R\otimes\beta$. As a result, an arrival curve upper bounds the incoming traffic, and a service curve lower bounds the outgoing traffic.

% \begin{figure}
%     \centering
%     \includegraphics[width=\linewidth]{images/arrival-service.png}
%     \caption{Arrival/Service curve}
%     \label{fig: arrival-service curves}
% \end{figure}

We consider 2 special types of curves throughout this paper, \textit{token-bucket} (or sometimes called \textit{leaky-bucket}) curve and \textit{rate-Latency} curve.

\begin{definition}[Token-bucket and Rate-latency~\cite{ncbook2001leboudec}]
    A token-bucket curve $\gamma_{r,b}$ with arrival rate $r$ and burst $b$ is defined as
    \begin{equation}
        \gamma_{r,b}(t) = b + rt
    \end{equation}

    A rate-latency curve $\beta_{R,T}$ with service rate $R$ and latency $T$ is defined as
    \begin{equation}
        \beta_{R,T}(t) = R \lb t - T \rb_+
    \end{equation}
\end{definition}

A token-bucket curve is determined by a burst $b$ and an arrival rate~$r$. Burst represents the maximum possible data volume that can arrive simultaneously, and arrival rate represents the maximum long-term data rate~\cite{bouillard2022tradeoff}.
A rate-latency curve is determined by a latency~$T$ and a service rate~$R$. Latency represents the time a server needs before starting to process the incoming data, and service rate represents the minimum rate to process data after the initial latency.

With the help of arrival and service curves, we can derive delay and backlog bounds for a system $\mcal{S}$ illustrated in Figure~\ref{fig: system bounds}. Suppose a system $\mcal{S}$ has arrival curve $\alpha$ and service curve~$\beta$, its worst-case backlog $b^*$ is the maximum vertical distance between~$\alpha$ and~$\beta$. Similarly, depending on the multiplexing technique applied to the system, its worst-case delay bound $d^*$ is the maximum horizontal distance between $\alpha$ and $\beta$ if $\mcal{S}$ is a FIFO system. If we don't have any information about its multiplexing technique, referred to as arbitrary multiplexing, the best we can say is that when $\alpha$ and $\beta$ intersect each other, where all data has been delivered out of the system. Consequently, the worst-case delay bound for arbitrary multiplexing is the time required for $\mcal{S}$ to clear its buffer.

% \begin{figure}
%     \centering
%     \includegraphics[width=\linewidth]{images/bound.png}
%     \caption{System delay/backlog bounds}
%     \label{fig: system bounds}
% \end{figure}

While a service curve captures the slowest possible output speed of a system, a link's transmission capacity limits the speed as well. Hence, we model this phenomenon using a \textit{greedy shaper} with a sub-additive function $\sigma: \mbb{R}^+ \mapsto \mbb{R}^+$ concatenated with a server. We consider a concatenation as shown in Figure \ref{fig: system}. By convention we assume $\sigma(0) = 0$ and $\beta(t) \leq \sigma(t), \forall t \in \mbb{R}^+$, meaning that the buffer is cleared at the beginning and the service never exceed its physical limitation. With the above definition, such greedy shaper conserves the service provided by the system due to theorem \ref{thm: shaping}.

\begin{figure}[thb]
    \centering
    \includegraphics[width=0.7\linewidth]{images/system.png}
    \caption{Shaping of departure data. A flow that has an arrival curve $\alpha$ feeds into a server with an arrival data flow $R(t)$. The server having service curve $\beta$ takes $R(t)$ and gives a departure data flow $R^*(t)$ to a shaper with shaping function $\sigma$. The shaper takes $R^*(t)$ and shape the data flow as another departure $D(t)$.}
    \label{fig: system}
\end{figure}


\begin{theorem}[Shaping conserves service \cite{ncbook2001leboudec}]
\label{thm: shaping}
Following the system shown in Figure \ref{fig: system}, we have
\begin{equation}
     D = R^* \otimes \sigma \geq \lp R \otimes \beta \rp \otimes \sigma = R \otimes \lp \beta \otimes \sigma \rp = R \otimes \beta
\end{equation}
\end{theorem}

In the following context, we model the shaping function $\sigma$ as a token-bucket curve $\gamma_{C,L}$ with transmission capacity $C$ and the packet size $L$ to capture the link capacity and packetization~\cite{bouillard2022tradeoff}.


% \section{Collision Cone CBF (C3BF)}
% \label{section: Controller}
% % \acrfull{mpc} is an advanced form of closed-loop control that calculates the optimal input sequence for a dynamic-sytem to reach a desired state.
% \acrshort{mpc} works really well with complex, non-linear, \acrshort{mimo} systems that may have interactions between their inputs and outputs.
% It is also able to handle constraints on the system's input, state and output.
% This enables the controller to simulate the bounded inputs of a realistic model as well as implement a collision avoidance system inside the controller itself.

% \acrshort{mpc} works by using a mathematical model of the plant to predict the system's behavior in the future.
% At time $t$, the system's state is sampled by means of various sensors and a control strategy is calculated to minimize a cost function over a short horizon $[t, t+T]$.
% An optimization algorithm is used to calculate a sequence of control inputs $u_k$ that will drive the system to a state that minimizes the cost function.
% Only the first step of the control strategy, $u_0$, is used to drive the system.
% Then, in the next control iteration, the plant's state is sampled again and a new control strategy is calculated.
% This continuous shifting of the horizon to the future, eventually drives the system to the desired state.

% \insertfig{1_mpc}{Block diagram of a \acrshort{mpc} controller}

% In this project, the chasers need to maintain a fixed pose relative to the \gls{target} and avoid collision with the rest of the chasers and the target.
% To achieve this, both the future states of the chasers and the target need to be predicted and their relative pose needs to be maintained in every time-step of the horizon.
% Details about the system dynamics of both a \gls{chaser} and the target in can be found in Section \ref{ch:chap3:sec2} and the cost function that implements this concept in Section \ref{ch:chap3:sec3}.
% Additional constraints are added to the optimization problem to ensure collision avoidance.
% Details about the constraints enforced on the optimizer can be found in Section \ref{ch:chap3:sec4}.

% To sum up, an \acrshort{mpc} controller uses the following principles:
% \begin{itemize}
%     \item A model of the controlled dynamic system.
%     \item A cost function $J$ over the horizon $[t, t+T]$.
%     \item An optimization algorithm that minimizes $J$ using the control input $u$ while obeying a set of constraints.
% \end{itemize}

As previously mentioned, the control architecture uses a MPC backbone.
In this section, the cost function and constraints used to formulate the MPC will be presented.




% \subsection{Euler Method} \label{ch:chap3:euler}
% Now the system dymanics need to be descretized and solved to be used with the MPC controller.
% For that purpose, the Euler method is used (Equation \ref{eq:3_euler_method}).
% The local error of this method is proportinal to the square of the step size, and the global error proportional to the step size.
% This means that a smaller step size will result in a more accurate solution but for the same horizon duration, smaller step size will result in higher computational complexity and thus an increase on processing time.
% A good balance between time-step and horizon length is necessary to obtain a good \acrshort{mpc} prediction and keep the calculation time low.

% \begin{equation} \label{eq:3_euler_method}
%     \begin{array}{c}
%         \text{For a dynamic system defined as: } \dot{x} = f(x,u)\\
%         x_{k+1} = x_k + T_s f(x_k,u_k)
%     \end{array}
% \end{equation}

% From Equation \ref{eq:3_chaser_dynamics}, it is obvious that the control inputs that will be used in the descritized equations for each chaser are:

% \begin{equation}
%     \begin{split}
%         u   &= [F_x, F_y, F_z, \tau_x, \tau_y, \tau_z]^T\\
%             &= [\bm{F}, \bm{\tau}]^T
%     \end{split}
% \end{equation}

\subsection{Cost Function} \label{ch:chap3:sec3}

% As mentioned in Section \ref{ch:chap3}, a cost function $J$ must be determined for the \acrshort{mpc} controller to work.
% This cost function will be minimized by the optimization algorithm using $u$ to drive the plant to the desired state.

% Since the mission requires each chaser $i$ to be at a pose ${x_{ref}^i}$ relative to the target, the cost function could be defined as the distance between the desired pose ${x_{ref}^i}$ and the chaser's pose $p_{ch} = [\bm{x}, \bm{q}]$.
% But $ref^i$ changes while the target is moving during the time horizon.
% Therefore, the MPC controller predicts the movement of the target as well, and uses these predicted future poses to calculate what the desired pose $ref^i_k$ will be in every time-step $k$ of the horizon.
% An appropriate cost function, that uses these predictions to calculate an optimal control sequence is $J$ in \eqref{eq:3_cost_function}.

The chasers state vectors are defined as $x^i = [p^i, q^i]^\top$ and the corresponding control action as $u^i = [F^i, \tau^i]^\top$.
The system dynamics are discretized with a sampling time of $dt$ using the forward Euler method to obtain $x^i_{k+1} = \zeta(x^i_k, u^i_k)$. 
The target's state vector is defined as $x^{tar} = [p^{tar}, q^{tar}]$, where $p^{tar}$ is the position and $q^{tar}$ the quaternion attitude represention of the target.
They are descritized in a manner similar to the chaser's states.
The reference poses for the chasers are calculated by transfering the target's body frame using $x^{{off}_i} = [p^{{off}_i}, q^{{off}_i}]$, where $p^{{off}_i}$ is a translation and $q^{{off}_i}$ is a rotation quaternion.
These calculations are performed as such: $p^{{ref}_i}_k = p^{tar}_k + q^{tar}_k \otimes p^{{off}_i} \otimes {q^{tar}_k}^\ast$ and $q^{{ref}_i}_k = q^{tar}_k q^{{off}_i}$.
This discrete model is used as the prediction model of the NMPC.
The prediction is performed over a receding horizon of $D = T/dt$ steps, where $T$ is the horizon duration in seconds.

A cost function is defined such that, when minimized in the current time and the predicted horizon, an optimal set of control actions $u^i_k$ will be calculated.
Let $x_{k+j|k}$ and $x^{tar}_{k+j|k}$ be the predicted chaser and target states at time step $k+j$ respectively, calculated in time step $k$.
The corresponding control actions are $u_{k+j|k}$ and reference states $x^{{ref}_i}_{k+j|k}$.
Also, let $\bm{x}_k$ and $\bm{u}_k$ be the predicted states and control actions for the whole horizon duration calculated at time step $k$.
The cost function is formulated as follows:

\begin{multline} \label{eq:3_cost_function}
    J(\bm{x}_k, \bm{u}_k, u_{k-1|k}) = \sum_{i=0}^{N-1} \{ \sum_{j=0}^{D} \{ \\ 
    \underbrace{(1 - \frac{\alpha \cdot k}{D} )}_\text{Falloff \%} \cdot [\underbrace{{\| p^{{ref}_i}_{k+j|k} - p^i_{k+j|k} \|}^2 Q_p}_\text{Position cost} + \underbrace{{\| q^i_{k+j|k} \otimes {q^{{ref}_i}_{k+j|k}}^\ast \|}^2 Q_q}_\text{Orientation cost}] \\
    + \underbrace{\| u^i_{ref} - u^i_{k+j|k} \|^2 Q_u \}}_\text{Input cost} \\ 
    + \underbrace{{\| p^{{ref}_i}_{k+D|k} - p^i_{k+D|k} \|}^2 Q^f_p + {\| q^i_{k+D|k} \otimes {q^{{ref}_i}_{k+D|k}}^\ast \|}^2 Q^f_q}_\text{Final state cost} \}
\end{multline}

where $Q_p, Q^f_p \in \mathbb{R}^{3\times3}$, $Q_q, Q^f_q \in \mathbb{R}^{4\times4}$ and $Q_u \in \mathbb{R}^{6\times6}$ are positive definite weight matrices for the position and orientation states and inputs respectively. 
In \eqref{eq:3_cost_function}, the first term represents the state cost which penalizes deviation from the reference state at time-step $k$, $x^{{ref}_i}_k$.
The Falloff term $\alpha$, is an adaptive weight to penalize overshoot errors.
% The idea was that error propagation of velocity measurements can accumulate and become proportional to the horizon duration.
The second term represents the input cost which penalizes deviation from the steady-state input $u_{ref} = 0$ that describes constant-velocity movement.
Finally, the final state-cost applies an extra penalty for deviation of the state from the reference at the end of the horizon period.

% where %\footnote{Powers in these equations are element-wise operations}: 
% \begin{itemize}
%     % \item $N =$ The number of chasers
%     % \item $D = T / dt$ The horizon length
%     % \item $J_{p_{i,k}} = Q_p \cdot \left\{ \begin{array}{c} 
%     %     ({ref}_k - p_{i_k})^2 \text{ , for $x$, $\dot{x}$ and $\omega$}\\ 
%     %     (p_{i_k} \otimes {ref}_k^*)^2 \text{ , for $\bm{q}$}
%     % \end{array} \right. $ \\The cost representing the position error of chaser $i$ on time $k$
%     \item $Q_p, Q_{fp} \in \mathbb{R}^{3\times3}$, $Q_q, Q_{fq} \in \mathbb{R}^{4\times4}$ and $Q_u \in \mathbb{R}^{6\times6}$ are positive definite weight matrices for the position and orientation states and inputs respectively.
%     \item $J_{u_{i,k}} = Q_u \cdot u_{i_k}$ The control cost on time $k$ for chaser $i$
%     \item $J_{f_{i}} = Q_f \cdot J_{i_{T}}$ The cost representing the final state error of chaser $i$
%     \item $Q_p, Q_u, Q_f =$ Weight matrices for the cost functions
%     \item ${ref}_k =$ The predicted pose of the \gls{target} at time $k$
%     \item $\alpha$ = Falloff percentage
% \end{itemize}

\subsection{Constraints} \label{ch:chap3:sec4}

% Appropriate optimization constraints must force the controller to respect the hardware limits for the control inputs \eqref{eq:3_constr_max}.
% They can also force the optimizer to limit the plant output to avoid collisions.
A minimum chaser-chaser and chaser-target distance $d_{min}$ is enforced by \eqref{eq:3_constr_dist1} and \eqref{eq:3_constr_dist2}.
\eqref{eq:3_constr_dist3} prevents the chasers from moving too far away from the target by setting their maximum distance to $d_{max}$.
The implemented constraints are the following:
\begin{subequations}
    \begin{equation}\label{eq:3_constr_max}
        u^i_{k+j|k} = [\bm{F}^i_{k+j|k}, \bm{\tau}^i_{k+j|k}] \in [-F_{max}, F_{max}] \times [-\tau_{max}, \tau_{max}]
    \end{equation}
    
    \begin{equation}\label{eq:3_constr_dist1}
        \dist{(p^{tar}_{k+j|k}, p^m_{k+j|k})} \geq d_{min}\; \forall\; j \in [0, D]
    \end{equation}
    
    \begin{equation}\label{eq:3_constr_dist2}
        \dist{(p^m_{k+j|k}, p^n_{k+j|k})} \geq d_{min} \forall\; j \in [0, D],\; m \neq n
    \end{equation}
    
    \begin{equation}\label{eq:3_constr_dist3}
        \dist{(p^{{ref}_m}_{k+j|k}, p^m_{k+j|k})} \leq d_{max}\; \forall\; j \in [0, D]
    \end{equation}
\end{subequations}
where $dist$ is a function of Eucledean distance.

\subsection{Optimizer} \label{ch:chap3:sec5}

% An optimization problem is defined as the calculation of the extrema of an objective function $f(x)$ over a set of real variables $x$.
% The problem might be subject to a set of conditions defined as a system of equalities and inequalities called constraints.
% \eqref{eq:3_optimization} is the mathematic representation of the problem where $m,p \in \mathbb{Z^+}$ and $f, g_i, h_j$ are real functions on $X \subseteq \mathbb{R}^n$.
% \acrfull{nlp} is the process of solving an optimization problem where at least one of $f, g_i, h_j$ is non-linear \cite{math_programming}.

% \begin{equation} \label{eq:3_optimization}
%     \begin{split}
%         \min_{x \in X \subseteq \mathbb{R}^n} \quad & f(x)\\
%         \text{subject to} \quad & g_i(x) \ge 0 \; \forall i \in \{1, \dots, m\}\\
%         & h_j(x) = 0 \; \forall j \in \{1, \dots, p\}
%     \end{split}
% \end{equation}

\acrfull{open} was used as the MPC cost function optimizer.
\acrshort{open} is a framework developed by \cite{opengen} and is based on the PANOC \citep{panoc} optimization algorithm.
PANOC is an extremely fast, the state-of-the-art optimizer for real-time, embedded applications.
A powerful symbolic mathematics library called CasADi \citep{casadi} is used to define the optimization problem and perform under-the-hood operations.
% OpEn runs in Rust, a programming language that is ideal for embeded, real-time applications and can be interfaced for use with many high level programming languages.
% These make it the state-of-the-art option for real-time applications such as robotics, autonomous vehicles and \acrshort{uav}s where running an optimizer is necessary to close a control loop in every time-step.

% In contrast to other popular libraries, \acrshort{open} uses an algorithm called PANOC.
% PANOC is an algorithm introduced by \cite{panoc} that involves only very simple iterations making it extremely fast.
% It is fundamentally different from other popular, iterative algorithms as it implements a method called proximal averaged Newton-type method.
% PANOC is capable of solving \acrshort{nlp} problems in the form described by \eqref{eq:3_nlp_panoc}.
% (As described in the library's \href{https://alphaville.github.io/optimization-engine/docs/open-intro}{documentation page}).
% \begin{equation}\label{eq:3_nlp_panoc}
%     \begin{split}
%         \mathbb{P}(p) \; \text{:} \; \min_{u \in \mathbb{R}^{n_u}} \quad & f(u, p)\\
%         \text{subject to} \quad & u \; \in \; U\\
%         & F_1(u, p) \; \in \; C\\
%         & F_2(u, p) \; = \; 0
%     \end{split}
% \end{equation}
% Where $u \in \mathbb{R}^{n_u}$ is the vector decision variables of the problem, $p \in \mathbb{R}^{n_p}$ is the parameter vector and $F_1$ and $F_2$ describe two types of constraints handled with different methods in the algorithm.

\section{Collision Cone CBFs on Quadrotor}
\label{section: Safety Guarantee}
Having described the Collision Cone CBF candidate, we will see their application on quadrotors in this section. Based on the shape of the obstacle we can divide the proposed candidates into two cases: 
% \subsubsection{3D CBF candidate}
% When the dimensions of the obstacle are comparable to each other, we can assume the obstacle as a sphere with radius $r = max(c_1, c_2, c_3) + \frac{w}{2}$, where $w$ is the max width of the quadrotor absorbed in the obstacle width (shown in Fig. \ref{fig:3D CBF}). We call the CBF candidate so formed in this case as \textbf{3D CBF} candidate (see Fig. \ref{fig:3D CBF}).

% \subsubsection{Projection CBF candidate}
%  


\subsection{3D CBF candidate}
\label{section: 3D-CBF}
In scenarios where the dimensions of the obstacle are roughly equal, we can model the obstacle as a sphere, as illustrated in Fig. \ref{fig:3D CBF}. The resulting CBF in this context is referred to as the \textbf{3D CBF}. The relative position vector between the body center of the quadrotor and the center of the obstacle is as follows:
\begin{align}\label{eq:pos-vec-3D}
    \prel := \begin{bmatrix}
        c_x \\
        c_y \\
        c_z
    \end{bmatrix}
    - \left (
    \begin{bmatrix}
        x_p \\
        y_p \\
        z_p
    \end{bmatrix}
    + \textbf{R} \begin{bmatrix}
        0 \\
        0 \\
        l
    \end{bmatrix}
    \right )
\end{align}
Here $l$ is the distance of the body center from the base (see Fig. \ref{fig:models}). $c_x,c_y,c_z$ represents the obstacle location as a function of time. Also, since the obstacles are of constant velocity, we have $\Ddot{c}_x= \Ddot{c}_y= \Ddot{c}_z = 0$. We obtain its relative velocity as
\begin{align}\label{eq:vel-vec-3D}
    \vrel := \dot{p}_{rel}
\end{align}


% Having defined the $p_{rel}$ and $v_{rel}$
% relative position and velocity of the moving obstacle
% Now, we calculate the $\vreldot$ term which contains our inputs i.e. $(f_1, f_2, f_3, f_4)$, as follows:
% \begin{align*}
%     \vreldot = - (\frac{1}{m}\textbf{R}
%             \begin{bmatrix}
%                 0 & 0 & 0 & 0 \\
%                 0 & 0 & 0 & 0 \\
%                 1 & 1 & 1 & 1 
%             \end{bmatrix} + \\
%             \textbf{R} lL \hat{k} I_b^{-1}
%             \begin{bmatrix}
%                 1 & 0 & -1 & 0 \\
%                 0 & 1 & 0 & -1 \\
%                 0 & 0 & 0 & 0 
%             \end{bmatrix})
%             \begin{bmatrix}
%     		f_{1} \\
%     		f_{2} \\
%                 f_{3} \\
%                 f_{4} 
%     	\end{bmatrix} \\
%         + some other terms. \nonumber
% \end{align*}
% where

% \begin{align*}
%     \hat{k} = 
%             \begin{bmatrix}
%                 0 & -1 & 0 \\
%                 1 & 0 & 0  \\
%                 0 & 0 & 0 
%             \end{bmatrix}, 
% % \end{align*}
% % \begin{align*}
%     I_b^{-1} =
%             \begin{bmatrix}
%                 \frac{1}{I_{xx}} & 0 & 0 \\
%                 0 & \frac{1}{I_{yy}} & 0 \\
%                 0 & 0 & \frac{1}{I_{zz}}
%             \end{bmatrix}
%          \nonumber
% \end{align*}
% Therefore, 
% \begin{equation}\label{eqn:vrel-dot-3D}
%     \vreldot = -\textbf{R}
%             \begin{bmatrix}
%                 0 & \frac{Ll}{I_{yy}} & 0 & \frac{-Ll}{I_{yy}} \\
%                 \frac{-Ll}{I_{xx}} & 0 & \frac{Ll}{I_{xx}} & 0 \\
%                 \frac{1}{m_Q} & \frac{1}{m_Q} & \frac{1}{m_Q} & \frac{1}{m_Q} 
%             \end{bmatrix}
%             \begin{bmatrix}
%     		f_{1} \\
%     		f_{2} \\
%                 f_{3} \\
%                 f_{4} 
%     	\end{bmatrix} \\
%         + \rm{additional} \: \rm{terms}. \nonumber
% \end{equation}

% \subsubsection{Ellipse CBF}
% Consider the following CBF candidate:
% % \begin{tcolorbox}
% \begin{equation}
% \begin{aligned}
%     h_{El}(\state,t) = \left(\frac{c_x(t) - x_p}{c_1}\right)^2 + \left(\frac{c_y(t) - y_p}{c_2}\right)^2 \\
%     + \left(\frac{c_z(t) - z_p}{c_3}\right)^2 - 1,
%     \label{eqn:Ellipse-CBF}
% \end{aligned}
% \end{equation}
% % \end{tcolorbox}
% Since $h_{El}$ in \eqref{eqn:Ellipse-CBF} is dependent on time (e.g. moving obstacles), the resulting set $\mathcal{C}$ is also dependent on time. To analyze this class of sets, time dependent versions of CBFs can be used \cite{IGARASHI2019735}. Alternatively, we can reformulate our problem to treat the obstacle position $c_x,c_y$ as states, with their derivatives being constants. This will allow us to continue using the classical CBF given by Definition \ref{definition: CBF definition} including its properties on safety. The derivative of \eqref{eqn:Ellipse-CBF} is
% \begin{align}
% \frac{2(c_x-x)(\dot c_x - \dot x)}{c_1^2} +\frac{2(c_y-y)(\dot c_y - \dot y)}{c_2^2} +\frac{2(c_z-z)(\dot c_z - \dot z)}{c_3^2},
% \end{align}
% which has no dependency on the inputs $(f_1, f_2, f_3, f_4)$, since there is no $\vreldot$ term. Hence, $h_{El}$ will not be a valid CBF for the quadrotor model \eqref{eqn:quadrotor_model}. 


% \subsubsection{Higher Order CBF}
% Having introduced HO-CBFs in \ref{subsection: HO-CBF}, we will try to give guarantees for Form 1  \eqref{eqn:HO-CBF-1} for the simple reason that it is similar to C3BF. So, the effective Higher Order CBF candidate looks like this:

% \begin{equation}
%     h_{HO}(\state, t) = < \prel, \vrel> + \gamma \sqrt{(\|\prel\|^2 - r^2)}
%     \label{eqn:HO-CBF}
% \end{equation}

% where $\gamma$ is the penalty term. We now show that HO CBF candidate \eqref{eqn:HO-CBF} is indeed a valid CBF. 

% \begin{theorem}\label{thm:HO-CBF-3D}{\it
% Given the quadrotor model \eqref{eqn:quadrotor_model}, the proposed CBF candidate \eqref{eqn:HO-CBF} with $\prel,\vrel$ defined by \eqref{eq:pos-vec-3D}, \eqref{eq:vel-vec-3D} is a valid CBF defined for the set $\mathcal{D}$.}
% \end{theorem}
% \begin{proof}
% Taking the derivative of \eqref{eqn:HO-CBF} yields
% \begin{align}
% \dot h_{HO} = &  < \preldot, \vrel > + < \prel, \vreldot >  \nonumber \\
%  & +  \frac{\gamma}{\sqrt{(\|\prel\|^2 - r^2)}} < \prel, \preldot >.
%  \label{eqn:HO_h_d}
% \end{align}


% Given $\vreldot$ (which contains the input) from \eqref{eqn:vrel-dot-3D} and $\dot h$ \eqref{eqn:HO_h_d}, we have the following expression for $\mathcal{L}_g h_{HO}$:
% \begin{align}
%     \mathcal{L}_g h_{HO} = \begin{bmatrix}
%         < \prel, 
%             \textbf{R}\begin{bmatrix}
%                 0  \\
%                 \frac{-Ll}{I_{xx}}\\
%                 \frac{1}{m}
%             \end{bmatrix}>\\
%                 < \prel, 
%             \textbf{R}\begin{bmatrix}
%                 \frac{Ll}{I_{yy}} \\
%                 0\\
%                 \frac{1}{m}
%             \end{bmatrix}>\\
%          < \prel, 
%             \textbf{R}\begin{bmatrix}
%                 0  \\
%                 \frac{Ll}{I_{xx}}\\
%                 \frac{1}{m}
%             \end{bmatrix}>\\
%          < \prel, 
%             \textbf{R}\begin{bmatrix}
%                 \frac{-Ll}{I_{yy}} \\
%                 0 \\
%                 \frac{1}{m}
%             \end{bmatrix}>
%     \end{bmatrix}^T,
% \end{align}
% It can be verified that for $\mathcal{L}_gh_{HO}$ to be zero, we can have the following scenarios:
% \begin{itemize}
%     \item $\prel =0$, which is not possible, because this indicates that the vehicle is already inside the obstacle. 
%     \item $\prel$ is perpendicular to all  $\textbf{R}\begin{bmatrix}
%                     0  \\
%                     \frac{-Ll}{I_{xx}}\\
%                     \frac{1}{m}
%                 \end{bmatrix}$, 
%     $\textbf{R}\begin{bmatrix}
%                 \frac{Ll}{I_{yy}} \\
%                 0\\
%                 \frac{1}{m}
%             \end{bmatrix} $,
%     $\textbf{R}\begin{bmatrix}
%                 0  \\
%                 \frac{Ll}{I_{xx}}\\
%                 \frac{1}{m}
%             \end{bmatrix} $
%             and  $\textbf{R}\begin{bmatrix}
%                 \frac{-Ll}{I_{yy}} \\
%                 0 \\
%                 \frac{1}{m}
%             \end{bmatrix} $, which is also not possible. (Because these vectors form basis vectors for $\mathbb{R}^{3}$)
% \end{itemize}
% This implies that $\mathcal{L}_gh_{HO}$ is always a non-zero matrix, implying that $h_{HO}$ is a valid CBF.
% \end{proof}
% \begin{remark}
% {\it
% Since $\mathcal{L}_g h \neq 0$, we can infer from \cite[Theorem 8]{XU201554} that the resulting QP given by \eqref{eq:CBF-QP} is Lipschitz continuous. Hence, we can construct CBF-QPs with the CBF \eqref{eqn:HO-CBF} for the quadrotor model and guarantee collision avoidance. In addition,  if $h(x(0))<0$, then we can construct a class $\mathcal{K}$ function $\kappa$ in such a way that the magnitude of $h$ exponentially decreases over time, thereby minimizing the violation. We will demonstrate these scenarios in Section \ref{section: Results and Discussions}.
% }
% \end{remark}

\begin{figure}[t]
    % \centering
    \includegraphics[width=0.9\linewidth]{images/3D_CBF.pdf}
\caption{\textbf{3D CBF} candidate: The dimensions of the obstacle are comparable to each other, it can be assumed as a sphere}
\label{fig:3D CBF}
\end{figure}



% \subsubsection{Collision Cone CBF}
Having introduced Collision Cone CBF candidates in \ref{subsection: C3BF}, the next step is to formally verify that they are, indeed, valid CBFs. 
% The 3D C3BF candidate is as follows: 
% 
% \begin{equation}
%     h_{3D}(\state, t) = < \prel, \vrel> + \| \prel\|\| \vrel\|\cos\phi
%     \label{eqn:CC-CBF-3D}
% \end{equation}
% 
% where, $\phi$ is the half angle of the cone, the expression of $\cos\phi$ is given by $\frac{\sqrt{\|\prel\|^2 - r^2}}{\|\prel\|}$ (see Fig. \ref{fig:3D CBF}).  
  % 
% We now show that the proposed CBF candidate \eqref{eqn:CC-CBF-3D} is indeed a valid CBF.
We have the following result.

\begin{theorem}\label{thm:CC-CBF-3D}{\it
Given the quadrotor model \eqref{eqn:quadrotor_model}, the proposed CBF candidate \eqref{eqn:CC-CBF} with $\prel,\vrel$ defined by \eqref{eq:pos-vec-3D}, \eqref{eq:vel-vec-3D} is a valid CBF defined for the set $\mathcal{D}$.}
\end{theorem}
Please refer to \cite[Thm 3]{C3BF_tac} for the proof of Theorem.
% \begin{proof}
% Since we are considering the 3D case, we will denote the resulting CBF candidate given by \eqref{eqn:CC-CBF} by $h_{3D}$. We have the following derivative:
% %Taking the derivative of \eqref{eqn:CC-CBF} yields
% \begin{align}
% \dot h_{3D} = &  < \preldot, \vrel > + < \prel, \vreldot >  \nonumber \\
%  & + < \vrel, \vreldot > \frac{\sqrt{\|\prel\|^2 - r^2}}{\|\vrel\|} \nonumber \\
%  & + < \prel, \preldot > \frac{\|\vrel\| }{\sqrt{\|\prel\|^2 - r^2}}.
%  \label{eqn:CC_h_d}
% \end{align}
% % 
% By substituting for $\vreldot$ (which contains the input) in $\dot h_{3D}$ \eqref{eqn:CC_h_d}, we have the following expression for $\mathcal{L}_g h_{3D}$:
% \begin{align}
%     \mathcal{L}_g h_{3D} = \begin{bmatrix}
%         < \prel + \vrel \frac{\sqrt{\|\vrel\|^2 - r^2}}{\|\vrel\|}, 
%             \textbf{R}\begin{bmatrix}
%                 0  \\
%                 \frac{-Ll}{I_{xx}}\\
%                 \frac{1}{m_Q}
%             \end{bmatrix}>\\
%                 < \prel + \vrel \frac{\sqrt{\|\vrel\|^2 - r^2}}{\|\vrel\|}, 
%             \textbf{R}\begin{bmatrix}
%                 \frac{Ll}{I_{yy}} \\
%                 0\\
%                 \frac{1}{m_Q}
%             \end{bmatrix}>\\
%          < \prel + \vrel \frac{\sqrt{\|\vrel\|^2 - r^2}}{\|\vrel\|}, 
%             \textbf{R}\begin{bmatrix}
%                 0  \\
%                 \frac{Ll}{I_{xx}}\\
%                 \frac{1}{m_Q}
%             \end{bmatrix}>\\
%          < \prel + \vrel \frac{\sqrt{\|\vrel\|^2 - r^2}}{\|\vrel\|}, 
%             \textbf{R}\begin{bmatrix}
%                 \frac{-Ll}{I_{yy}} \\
%                 0 \\
%                 \frac{1}{m_Q}
%             \end{bmatrix}>
%     \end{bmatrix}^T,
% \end{align}

% It can be verified that for $\mathcal{L}_gh_{3D}$ to be zero, we can have the following scenarios:
% \begin{itemize}
%     \item $\prel + \vrel \frac{\sqrt{\|\prel\|^2 - r^2}}{\|\vrel\|}=0$, which is not possible. Firstly, $\prel=0$ indicates that the vehicle is already inside the obstacle. Secondly, if the above equation were to be true for a non-zero $\prel$, then $\vrel/\|\vrel\| = - \prel/\sqrt{\|\prel\|^2 - r^2}$. This is also not possible as the magnitude of LHS is $1$, while that of RHS is $>1$.
%     \item $\prel + \vrel \frac{\sqrt{\|\vrel\|^2 - r^2}}{\|\vrel\|}$ is perpendicular to all  $\textbf{R}\begin{bmatrix}
%                     0  \\
%                     \frac{-Ll}{I_{xx}}\\
%                     \frac{1}{m_Q}
%                 \end{bmatrix}$, 
%     $\textbf{R}\begin{bmatrix}
%                 \frac{Ll}{I_{yy}} \\
%                 0\\
%                 \frac{1}{m_Q}
%             \end{bmatrix} $,
%     $\textbf{R}\begin{bmatrix}
%                 0  \\
%                 \frac{Ll}{I_{xx}}\\
%                 \frac{1}{m_Q}
%             \end{bmatrix} $
%             and  $\textbf{R}\begin{bmatrix}
%                 \frac{-Ll}{I_{yy}} \\
%                 0 \\
%                 \frac{1}{m_Q}
%             \end{bmatrix} $, which is also not possible. (Because three of these vectors form basis vectors for $\mathbb{R}^{3}$)
% \end{itemize}
% This implies that $\mathcal{L}_gh_{3D}$ is always a non-zero matrix, implying that $h_{3D}$ is a valid CBF.
% \end{proof}
% \begin{remark}
% {\it
% Since $\mathcal{L}_g h \neq 0$, we can infer from \cite[Theorem 8]{XU201554} that the resulting QP given by \eqref{eq:CBF-QP} is Lipschitz continuous. Hence, we can construct CBF-QPs with the proposed CBF \eqref{eqn:CC-CBF} for the quadrotor model and guarantee collision avoidance. In addition,  if $h(x(0))<0$, then we can construct a class $\mathcal{K}$ function $\kappa$ in such a way that the magnitude of $h$ exponentially decreases over time, thereby minimizing the violation. We will demonstrate these scenarios in 
% %This would imply either braking or steering leading to collision with the periphery of the obstacle. %This will be shown in 
% Section \ref{section: Results and Discussions}.
% }
% \end{remark}

\subsection{Projection CBF candidate}
\label{section: proj-CBF}

\begin{figure}[t]
    \centering
    \includegraphics[width=0.8\linewidth]{images/Projection_CBF.pdf}
\caption{\textbf{Projection CBF} candidate: One of the dimensions, of the obstacle, is bigger than the other dimensions, it can be assumed as a cylinder.}
\label{fig:Projection CBF}
\end{figure}
When an obstacle has significantly disparate dimensions, it can be approximated as a cylinder, giving rise to the \textbf{Projection CBF} (refer to Fig. \ref{fig:Projection CBF}). To derive this, we calculate the relative position vector between the quadrotor's body center and the intersection point of the obstacle's axis with the projection plane, which is perpendicular to the axis. Thus, we obtain:
\begin{align}\label{eqn:pos-vec-proj}
    (\prel)_{proj} := \mathcal{P}\left (\begin{bmatrix}
        c_x \\
        c_y \\
        c_z
    \end{bmatrix}
    - \left (
    \begin{bmatrix}
        x_p \\
        y_p \\
        z_p
    \end{bmatrix}
    + \textbf{R} \begin{bmatrix}
        0 \\
        0 \\
        l
    \end{bmatrix}
    \right ) \right ).
\end{align}
Here $l$ is the distance of the body center from the base (see Fig. \ref{fig:models}). $\mathcal{P}: \mathbb{R}^3 \to \mathbb{R}^3 $ is the projection operator, which can be assumed to be a constant\footnote{Note that the obstacles are always translating and not rotating. In addition, it is not restrictive to assume that the translation direction is always perpendicular to the cylinder axis. This makes the projection operator a constant.}. 
%But in this case $\Ddot{c}_x , \Ddot{c}_y, \& \Ddot{c}_z \neq 0$.
Now, since the relative position lies on the projection plane, we have one more condition to satisfy:
\begin{align}\label{eqn:p_in_plane}
    < (\prel)_{proj}, \hat{n} > = 0,   
\end{align}
% 
where, $\hat{n}$ is the normal to the plane. Also, the relative velocity is given by:
\begin{align}\label{eqn:vel-vec-proj}
    (\vrel)_{proj} := \frac{d({p}_{rel})_{proj}}{dt} = (\preldot)_{proj}
\end{align}
% Having defined the $p_{rel}$ and $v_{rel}$
% relative position and velocity of the moving obstacle
% Now, we calculate the $\frac{d}{dt}(\vrel)_{proj}$ term which contains our inputs i.e. $(f_1, f_2, f_3, f_4)$, as follows:
% \begin{align*}
%     (\vreldot)_{proj} = \begin{bmatrix}
%             \ddot{c_x} \\
%             \ddot{c_y} \\
%             \ddot{c_z}
%         \end{bmatrix}
%         - 
%         (\frac{1}{m}\textbf{R}
%             \begin{bmatrix}
%                 0 & 0 & 0 & 0 \\
%                 0 & 0 & 0 & 0 \\
%                 1 & 1 & 1 & 1 
%             \end{bmatrix} + \\
%             \textbf{R} lL \hat{k} I_b^{-1}
%             \begin{bmatrix}
%                 1 & 0 & -1 & 0 \\
%                 0 & 1 & 0 & -1 \\
%                 0 & 0 & 0 & 0 
%             \end{bmatrix})
%             \begin{bmatrix}
%     		f_{1} \\
%     		f_{2} \\
%                 f_{3} \\
%                 f_{4} 
%     	\end{bmatrix} \\
%         + some other terms. \nonumber
% \end{align*}
% where

% \begin{align*}
%     \hat{k} = 
%             \begin{bmatrix}
%                 0 & -1 & 0 \\
%                 1 & 0 & 0  \\
%                 0 & 0 & 0 
%             \end{bmatrix}, 
% % \end{align*}
% % \begin{align*}
%     I_b^{-1} =
%             \begin{bmatrix}
%                 \frac{1}{I_{xx}} & 0 & 0 \\
%                 0 & \frac{1}{I_{yy}} & 0 \\
%                 0 & 0 & \frac{1}{I_{zz}}
%             \end{bmatrix}
%          \nonumber
% \end{align*}
% Therefore, 
% \begin{equation}%\label{eqn:vrel_dot}
%     \frac{d}{dt}(\vrel)_{proj} =
%     \mathcal{P}(
%             -  
%             \textbf{R}
%             \begin{bmatrix}
%                 0 & \frac{Ll}{I_{yy}} & 0 & \frac{-Ll}{I_{yy}} \\
%                 \frac{-Ll}{I_{xx}} & 0 & \frac{Ll}{I_{xx}} & 0 \\
%                 \frac{1}{m_Q} & \frac{1}{m_Q} & \frac{1}{m_Q} & \frac{1}{m_Q} 
%             \end{bmatrix}
%             \begin{bmatrix}
%     		f_{1} \\
%     		f_{2} \\
%                 f_{3} \\
%                 f_{4} 
%     	\end{bmatrix} \\
%         + \rm{additional} \: \rm{terms}). \nonumber
% \end{equation}
% or, from \eqref{eqn:vrel-dot-3D}, we have
% \begin{equation}\label{eqn:vrel_dot}
%     \frac{d}{dt}(\vrel)_{proj} =
%     \mathcal{P}( 
%             \vreldot )
% \end{equation}
% $\frac{d}{dt}(\vrel)_{proj}$  is the projection of $\vreldot$ in \eqref{eqn:vrel-dot-3D} on the projection plane, that is:
% \begin{equation}
% \begin{aligned}\label{eqn:vrel-dot-proj}
%     (\vreldot)_{proj} = \vreldot - <\vreldot, \hat{n}> \hat{n}.
% \end{aligned}
% \end{equation}

% Thus, from \eqref{eqn:p_in_plane} and \eqref{eqn:vrel-dot-proj}, we have the following:
% \begin{equation}
% \begin{aligned}\label{eqn:pdot-vreldot}
%     % <(\prel)_{proj}, (\vreldot)_{proj}> = <(\prel)_{Proj}, \vreldot - <\vreldot, \hat{n}> \hat{n}> \\
%     <(\prel)_{proj}, (\vreldot)_{proj}> = <(\prel)_{proj}, \vreldot>.
% \end{aligned}
% \end{equation}

% Similarly,
% \begin{equation}
% \begin{aligned}\label{eqn:vdot-vreldot}
%     % <(\vrel)_{proj}, (\vreldot)_{Proj}> = <(\vrel)_{proj}, \vreldot - <\vreldot, \hat{n}> \hat{n}> \\
%     <(\vrel)_{proj}, (\vreldot)_{proj}> = <(\vrel)_{proj}, \vreldot>
% \end{aligned}
% \end{equation}


% \subsubsection{Ellipse CBF}
% Consider the following CBF candidate:
% % \begin{tcolorbox}
% \begin{equation}
%     h_{El}(\state,t) = \left(\frac{c_x(t) - x_p}{c_1}\right)^2 + \left(\frac{c_y(t) - y_p}{c_2}\right)^2 + \left(\frac{c_z(t) - z_p}{c_3}\right)^2 - 1,
%     \label{eqn:Ellipse-CBF}
% \end{equation}
% % \end{tcolorbox}
% Since $h_{El}$ in \eqref{eqn:Ellipse-CBF} is dependent on time (e.g. moving obstacles), the resulting set $\mathcal{C}$ is also dependent on time. To analyze this class of sets, time dependent versions of CBFs can be used \cite{IGARASHI2019735}. Alternatively, we can reformulate our problem to treat the obstacle position $c_x,c_y$ as states, with their derivatives being constants. This will allow us to continue using the classical CBF given by Definition \ref{definition: CBF definition} including its properties on safety. The derivative of \eqref{eqn:Ellipse-CBF} is
% \begin{align}
% \frac{2(c_x-x)(\dot c_x - \dot x)}{c_1^2} +\frac{2(c_y-y)(\dot c_y - \dot y)}{c_2^2} +\frac{2(c_z-z)(\dot c_z - \dot z)}{c_3^2},
% \end{align}
% which has no dependency on the inputs $(f_1, f_2, f_3, f_4)$, since there is no $\vreldot$ term. Hence, $h_{El}$ will not be a valid CBF for the quadrotor model \eqref{eqn:quadrotor_model}. 


% \subsubsection{Higher Order CBF}

% We will try to give guarantees for Form 1  \eqref{eqn:HO-CBF-1} for the projection case. So, the effective Higher Order CBF candidate looks like:

% \begin{equation}
%     h_{HO}(\state, t) = < (\prel)_{proj}, (\vrel)_{proj}> + \gamma \sqrt{(\|(\prel)_{proj}\|^2 - r^2)}
%     \label{eqn:HO-CBF-proj}
% \end{equation}

% where, $\gamma$ is the penalty term. 

% We now show that HO CBF candidate \eqref{eqn:HO-CBF-proj} is indeed a valid CBF. 

% \begin{theorem}\label{thm:HO-CBF-Proj}{\it
% Given the quadrotor model \eqref{eqn:quadrotor_model}, the proposed CBF candidate \eqref{eqn:HO-CBF} with $(\prel)_{proj}, (\vrel)_{proj}$ defined by \eqref{eqn:pos-vec-proj}, \eqref{eqn:vel-vec-proj} is a valid CBF defined for the set $\mathcal{D}$.}
% \end{theorem}

% \begin{proof}
% Taking the derivative of \eqref{eqn:HO-CBF-proj} yields
% \begin{align}
% \dot h_{HO} = &  < (\preldot)_{proj}, (\vrel)_{proj} > + < (\prel)_{proj}, (\vreldot)_{proj} >  \nonumber \\
%  & +  \frac{\gamma}{\sqrt{(\|(\prel)_{proj}\|^2 - r^2)}} < (\prel)_{proj}, (\preldot)_{proj} >.
%  \label{eqn:HO_h_d-proj}
% \end{align}

% Given $(\vreldot)_{proj}$ (which contains the input) from \eqref{eqn:vrel-dot-proj}, equation \eqref{eqn:pdot-vreldot} and $\dot h$ from \eqref{eqn:HO_h_d-proj}, we have the following expression for $\mathcal{L}_g h_{HO}$:
% \begin{align}
%     \mathcal{L}_g h_{HO} = \begin{bmatrix}
%         < (\prel)_{proj}, 
%             \textbf{R}\begin{bmatrix}
%                 0  \\
%                 \frac{-Ll}{I_{xx}}\\
%                 \frac{1}{m}
%             \end{bmatrix}>\\
%                 < (\prel)_{proj}, 
%             \textbf{R}\begin{bmatrix}
%                 \frac{Ll}{I_{yy}} \\
%                 0\\
%                 \frac{1}{m}
%             \end{bmatrix}>\\
%          < (\prel)_{proj}, 
%             \textbf{R}\begin{bmatrix}
%                 0  \\
%                 \frac{Ll}{I_{xx}}\\
%                 \frac{1}{m}
%             \end{bmatrix}>\\
%          < (\prel)_{proj}, 
%             \textbf{R}\begin{bmatrix}
%                 \frac{-Ll}{I_{yy}} \\
%                 0 \\
%                 \frac{1}{m}
%             \end{bmatrix}>
%     \end{bmatrix}^T,
% \end{align}
% Now, we can give the same arguments on why $\mathcal{L}_gh_{HO}$ cannot be zero as given in the 3D CBF case.
% % It can be verified that for $\mathcal{L}_gh_{HO}$ to be zero, we can have the following scenarios:
% % \begin{itemize}
% %     \item $\prel =0$, which is not possible, because this indicates that the vehicle is already inside the obstacle. 
% %     \item $\prel$ is perpendicular to all  $\textbf{R}\begin{bmatrix}
% %                     0  \\
% %                     \frac{-Ll}{I_{xx}}\\
% %                     \frac{1}{m}
% %                 \end{bmatrix}$, 
% %     $\textbf{R}\begin{bmatrix}
% %                 \frac{Ll}{I_{yy}} \\
% %                 0\\
% %                 \frac{1}{m}
% %             \end{bmatrix} $,
% %     $\textbf{R}\begin{bmatrix}
% %                 0  \\
% %                 \frac{Ll}{I_{xx}}\\
% %                 \frac{1}{m}
% %             \end{bmatrix} $
% %             and  $\textbf{R}\begin{bmatrix}
% %                 \frac{-Ll}{I_{yy}} \\
% %                 0 \\
% %                 \frac{1}{m}
% %             \end{bmatrix} $, which is also not possible. (Because these vectors form basis vectors for $\mathbb{R}^{3}$
% % \end{itemize}
% % This implies that $\mathcal{L}_gh_{HO}$ is always a non-zero matrix, implying that $h_{HO}$ is a valid CBF.
% \end{proof}
% % \begin{remark}
% % {\it
% % Since $\mathcal{L}_g h \neq 0$, we can infer from \cite[Theorem 8]{XU201554} that the resulting QP given by \eqref{eq:CBF-QP} is Lipschitz continuous. Hence, we can construct CBF-QPs with the CBF \eqref{eqn:HO-CBF} for the quadrotor model and guarantee collision avoidance. In addition,  if $h(x(0))<0$, then we can construct a class $\mathcal{K}$ function $\kappa$ in such a way that the magnitude of $h$ exponentially decreases over time, thereby minimizing the violation. We will demonstrate these scenarios in Section \ref{section: Results and Discussions}.
% % }
% % \end{remark}


% \subsubsection{Collision Cone CBF}
% We now provide the formal results for \eqref{eqn:CC-CBF} for the Projection case in this subsection. The candidate is given as follows:
% \begin{equation}
%     h_{proj}(\state, t) = < (\prel)_{proj}, (\vrel)_{proj}> + \| (\prel)_{proj}\|\| (\vrel)_{proj}\|\cos\phi
%     \label{eqn:CC-CBF-proj}
% \end{equation}
% where, $\phi$ is the half angle of the cone, the expression of $\cos\phi$ is given by $\frac{\sqrt{\|(\prel)_{proj}\|^2 - r^2}}{\|(\prel)_{proj}\|}$ (see Fig. \ref{fig:Projection CBF}).  We now show that the proposed CBF candidate \eqref{eqn:CC-CBF-proj} is indeed a valid CBF. 

\begin{theorem}\label{thm:CC-CBF-proj}{\it
Given the quadrotor model \eqref{eqn:quadrotor_model}, the proposed CBF candidate \eqref{eqn:CC-CBF} with $\prel,\vrel$ defined by \eqref{eqn:pos-vec-proj}, \eqref{eqn:vel-vec-proj} is a valid CBF defined for the set $\mathcal{D}$.}
\end{theorem}
Please refer to \cite[Thm 4]{C3BF_tac} for the proof of Theorem.
% \begin{proof}
% % Since we are considering the projection case, we will denote the resulting CBF candidate given by \eqref{eqn:CC-CBF} by $h_{proj}$. 
% We have the following derivative of $h_{proj}$:
% \begin{align}
% \dot h_{proj} = &  < (\preldot)_{proj}, (\vrel)_{proj} > + < (\prel)_{proj}, (\vreldot)_{proj} >  \nonumber \\
%  & + < (\vrel)_{proj}, (\vreldot)_{proj} > \frac{\sqrt{\|(\prel)_{proj}\|^2 - r^2}}{\|(\vrel)_{proj}\|} \nonumber \\
%  & + < (\prel)_{proj}, (\preldot)_{proj} > \frac{\|(\vrel)_{proj}\| }{\sqrt{\|(\prel)_{proj}\|^2 - r^2}}.
%  \label{eqn:CC_h_d-proj}
% \end{align}
% % 
%  $\vreldot$ (which contains the input) from \eqref{eqn:vrel-dot-proj}, equations \eqref{eqn:pdot-vreldot}, \eqref{eqn:vdot-vreldot} and $\dot h_{proj}$ \eqref{eqn:CC_h_d-proj}, we have the following expression for $\mathcal{L}_g h_{proj}$:
% \begin{align}
%     \mathcal{L}_g h_{proj} = \begin{bmatrix}
%         < \prel + \vrel \frac{\sqrt{\|\vrel\|^2 - r^2}}{\|\vrel\|}, 
%             \textbf{R}\begin{bmatrix}
%                 0  \\
%                 \frac{-Ll}{I_{xx}}\\
%                 \frac{1}{m_Q}
%             \end{bmatrix}>\\
%                 < \prel + \vrel \frac{\sqrt{\|\vrel\|^2 - r^2}}{\|\vrel\|}, 
%             \textbf{R}\begin{bmatrix}
%                 \frac{Ll}{I_{yy}} \\
%                 0\\
%                 \frac{1}{m_Q}
%             \end{bmatrix}>\\
%          < \prel + \vrel \frac{\sqrt{\|\vrel\|^2 - r^2}}{\|\vrel\|}, 
%             \textbf{R}\begin{bmatrix}
%                 0  \\
%                 \frac{Ll}{I_{xx}}\\
%                 \frac{1}{m_Q}
%             \end{bmatrix}>\\
%          < \prel + \vrel \frac{\sqrt{\|\vrel\|^2 - r^2}}{\|\vrel\|}, 
%             \textbf{R}\begin{bmatrix}
%                 \frac{-Ll}{I_{yy}} \\
%                 0 \\
%                 \frac{1}{m_Q}
%             \end{bmatrix}>
%     \end{bmatrix}^T,
% \end{align}
% Using the same arguments we gave in the proof of theorem. \ref{thm:CC-CBF-3D}, we can infer that $\mathcal{L}_gh_{proj}$ cannot be zero. This implies that $h_{proj}$ is a valid CBF.
% % Now, we can give the same arguments on why $\mathcal{L}_gh_{CC}$ cannot be zero as given in the 3D CBF case.
% % It can be verified that for $\mathcal{L}_gh_{CC}$ to be zero, we can have the following scenarios:
% % \begin{itemize}
% %     \item $\prel + \vrel \frac{\sqrt{\|\prel\|^2 - r^2}}{\|\vrel\|}=0$, which is not possible. Firstly, $\prel=0$ indicates that the vehicle is already inside the obstacle. Secondly, if the above equation were to be true for a non-zero $\prel$, then $\vrel/\|\vrel\| = - \prel/\sqrt{\|\prel\|^2 - r^2}$. This is also not possible as the magnitude of LHS is $1$, while that of RHS is $>1$.
% %     \item $\prel + \vrel \frac{\sqrt{\|\vrel\|^2 - r^2}}{\|\vrel\|}$ is perpendicular to all  $\textbf{R}\begin{bmatrix}
% %                     0  \\
% %                     \frac{-Ll}{I_{xx}}\\
% %                     \frac{1}{m}
% %                 \end{bmatrix}$, 
% %     $\textbf{R}\begin{bmatrix}
% %                 \frac{Ll}{I_{yy}} \\
% %                 0\\
% %                 \frac{1}{m}
% %             \end{bmatrix} $,
% %     $\textbf{R}\begin{bmatrix}
% %                 0  \\
% %                 \frac{Ll}{I_{xx}}\\
% %                 \frac{1}{m}
% %             \end{bmatrix} $
% %             and  $\textbf{R}\begin{bmatrix}
% %                 \frac{-Ll}{I_{yy}} \\
% %                 0 \\
% %                 \frac{1}{m}
% %             \end{bmatrix} $, which is also not possible. (Because these vectors form basis vectors for $\mathbb{R}^{3}$
% % \end{itemize}
% % This implies that $\mathcal{L}_gh_{CC}$ is always a non-zero matrix, implying that $h_{CC}$ is a valid CBF.
% \end{proof}

% \begin{remark}
% {\it
% % Since $\mathcal{L}_g h \neq 0$ in Theorems \eqref{thm:CC-CBF-3D} \& \eqref{thm:CC-CBF-proj}, we can infer from \cite[Theorem 8]{XU201554} that the resulting QP given by \eqref{eqn: CBF QP} is Lipschitz continuous. Hence, we can construct CBF-QPs with the proposed CBF \eqref{eqn:CC-CBF} for the quadrotor model and guarantee collision avoidance.

% Based on Theorems \eqref{thm:CC-CBF-3D} and \eqref{thm:CC-CBF-proj}, where $\mathcal{L}_g h \neq 0$, we can utilize the conclusion from \cite[Theorem 8]{XU201554} to deduce that the control inputs obtained from the resulting CBF-QP \eqref{eqn: CBF QP} are Lipschitz continuous. As a result, 
% %it is feasible to create CBF-QPs using the suggested CBF \eqref{eqn:CC-CBF} for the quadrotor model, ensuring collision avoidance.
% the resulting solutions guarantee forward invariance of the safe set generated by the proposed C3BF candidates.
% }
% \end{remark}

\subsection{Comparison with Higher Order CBFs}
We introduce the state-of-the-art Higher Order Control Barrier Functions (HO-CBFs) and compare them with the proposed C3BF in this section. Given that the collision constraints are with respect to position, the associated CBF has a relative degree of two. Therefore, it is necessary to establish a higher-order CBF with $m = 2$ as outlined in \cite[Eq. 16]{9516971}, which is expressed as: 
\begin{equation}
\begin{aligned}
%\label{eqn:HO-CBF}
\psi_{1}(x,t) &= \dot{b}(x,t) + p\alpha_{1} (b(x,t))\\
\psi_{2}(x,t) &= \dot{\psi_{1}}(x,t) + p \alpha_{2} (\psi_{1}(x,t)) ,\\
\end{aligned}
\end{equation}
where $b(\state,t) = (c_x(t) - x_p)^2 + (c_y(t) - y_p)^2 + (c_z(t) - z_p)^2 - r^2$, and r is the encompassing radius given by $r = max(c_1, c_2, c_3)$. $\alpha_1, \alpha_2$ are both class $\mathcal{K}$ functions, and $p$ is a tunable constant. As explained previously, $c_x,c_y,c_z$ is the center location of the obstacle as a function of time. Let us examine the form of HO-CBF where $\alpha_{1}$ is a square root function (which is also strictly increasing), and $\alpha_{2}$ is a linear function, due to its similarity to C3BF. Consequently, the resulting Higher Order CBF candidate takes the following form:
\begin{equation}
    h_{HO}(\state, t) = < \prel, \vrel> + \gamma \sqrt{(\|\prel\|^2 - r^2)}.
    \label{eqn:HO-CBF}
\end{equation}
% 
% If we try to understand C3BF, it tries to avoid the $\vrel$ vector between the quadrotor and the obstacle from going in the collision cone region given by the half angle $\phi' = \phi$ as shown in figures \ref{fig:3D CBF} and \ref{fig:Projection CBF}. Now, rewriting HO-CBF given in \eqref{eqn:HO-CBF} in C3BF form will result the following $\phi'$:
We can show that the above-mentioned HO-CBF is also a valid CBF for quadrotors. We will now compare it with the proposed C3BF.

The C3BF concept aims to prevent the $\vrel$ vector, which represents the relative velocity between the quadrotor and the obstacle, from entering the collision cone region defined by the half-angle $\phi$. Figures \ref{fig:3D CBF}, \ref{fig:Projection CBF} and \ref{fig:HO-C3-CBF} illustrate this idea. We can rewrite the HO-CBF formula presented in \eqref{eqn:HO-CBF} in the following form: 
% of C3BF, we obtain the following expression for $\phi'$:
\begin{align}
h_{HO}(\state, t) = < \prel, \vrel> + \|\prel\| \|\vrel\| cos(\phi')
    \label{eqn:HO-CBF-CC}
\end{align}
where, $cos(\phi') = \frac{\gamma}{\|\vrel\|}cos(\phi)$.
% Now if we search and find an appropriate $\gamma$ (penalty term) for the given HOCBF, it will result in a valid CBF as per \cite{9516971}. However, $\gamma$ will still be a constant in that case, resulting in an overestimation of the cone. On the other hand, in the case of C3BF, since we are allowing the penalty term to change with time i.e. keeping $\gamma = \|\vrel\|$, it ends up giving us a more accurate estimate of the collision cone as compared to HO-CBF case. This is even evident from the simulation results of both the CBFs (as shown in the next section).
If we are able to identify a suitable $\gamma$ (penalty term) for the given HO-CBF, it would result in a valid CBF as per \cite{9516971}. Nonetheless, in such a scenario where $\gamma$ remains constant and $\|\vrel\|$ goes on increasing, it leads to an increase in $\phi'$, thus, overestimating the cone as can be seen in Fig.\ref{fig:HO-C3-CBF}. Conversely, with the C3BF approach, we permit the penalty term to vary over time, i.e., $\gamma = \|\vrel\|$, resulting in a more precise estimation of the collision cone compared to the HO-CBF case. 
This also shows that C3BF is not a special case of Higher Order CBF.
% 
This is also evident from the simulation outcomes of both CBFs, as demonstrated in Section \ref{section: Simulation Results}.

\begin{figure}[t]
    % \centering
    \includegraphics[width=0.9\linewidth]{images/HO-3D_CBF.pdf}
\caption{Comparison of HO-CBF with C3BF. Here we are trying to compare the $\phi'$ and $\phi$ obtained from the two CBF formulations. It can be observed that $\phi'$ (pink cone) is dependent on $v_{rel}$, while $\phi$ (yellow cone) is a constant. The HO-CBF guarantees safety for a set that is not only smaller but also dependent on $v_{rel}$ as shown by the pink cone. Hence, HO-CBF is more conservative compared to C3BF.}
\label{fig:HO-C3-CBF}
\end{figure}

\section{Simulation Results}
\label{section: Simulation Results}
\par We have validated the C3BF-QP based controller on quadrotors for both 3D and Projection CBF cases. The simulations were conducted using the multi-drone environment \cite{pybullet-drones} on Pybullet \cite{coumans2019}, a python-based physics simulation engine. The parameters of Crazyflie are tabulated in \ref{table:quadrotor_parameters}. PD Controller is used as a reference controller to track the desired path, and the safety controller deployed is given by Sections \ref{subsection: track_controller} \ref{subsection: safe_controller}. We chose constant target velocities for verifying the C3BF-QP. For the class $\mathcal{K}$ function in the CBF inequality, we chose $\kappa(h) = \gamma h$, where $\gamma=1$.

\begin{figure}
       \centering
        \begin{subfigure}[b]{0.46\textwidth}
        \includegraphics[width=\textwidth]{images/st-side.jpg}
        \caption{}
        \end{subfigure}
        %
        \begin{subfigure}[b]{0.20\textwidth}
        \includegraphics[width=\textwidth]{images/st-up-si.jpg}
        \caption{}
        \end{subfigure}
        %
        \begin{subfigure}[b]{0.235\textwidth}
        \includegraphics[width=\textwidth]{images/st-up-sa.jpg}
        \caption{}
        \end{subfigure}
        % %
        % \begin{subfigure}[b]{0.48\textwidth}
        % \includegraphics[width=\textwidth]{images/Pybullet.jpg}
        % \end{subfigure}
        \caption{Interaction with static obstacles: overtaking (a), (b), and braking (c) behavior of the quadrotor, Section \ref{section: 3D-CBF}.}
        \label{fig:static-obs}
    \end{figure}


\begin{table}[b]

\begin{tabular}{|l|l|l|}
\hline
\textbf{Variables} & \textbf{Definition}         & \textbf{Value} \\ \hline
g                  & Gravitational acceleration  & $9.81 kg \cdot m/s^2$   \\ \hline
m                  & Mass of quadrotor           & $0.027 kg$        \\ \hline
L                  & Distance between two opp. rotors & 0.130 $m$         \\ \hline
l                  & Distance of center from base & $0.014 m$         \\ \hline
Ix, Iy             & Inertia about x, y-axis  & $2.39\cdot 10^{-5} kg \cdot m^2$    \\ \hline
Iz                 & Inertia about z-axis       & $3.23\cdot 10^{-5} kg \cdot m^2$    \\ \hline
kf                 & Motor’s thrust constant     & $3.16 \cdot 10^{-10}$          \\ \hline
km                 & Motor’s torque constant     & $7.94 \cdot 10^{-12}$          \\ \hline
\end{tabular}
\caption{Modelling parameters of Crazyflie}
\label{table:quadrotor_parameters}

\end{table}

\subsection{Simulation setup}
Having presented our proposed control method design, we now test our framework under three different scenarios to illustrate the performance of the controller. These scenarios include the interaction of quadrotor with: (1) a static obstacle (3D case) Fig. \ref{fig:static-obs}, (2) a moving obstacle (3D case) Fig. \ref{fig:moving-obs} and (3) an elongated obstacle (Projection case) Fig. \ref{fig:long-obs}.

\subsubsection{Interaction with static obstacles}
Fig. \ref{fig:static-obs} shows the overtaking (a, b), and braking (c) behavior of the quadrotor while interacting with the static obstacle (which is another quadrotor). In all these cases the reference velocity of the quadrotor is 1m/s. 

\subsubsection{Interaction with moving obstacles}
Fig. \ref{fig:moving-obs} shows the overtaking (a), (b), slowing (c), and reversing (d) behavior of the quadrotor while interacting with the moving obstacle (which is another quadrotor). In all these cases the reference velocity of the quadrotor is 1m/s and the obstacle quadrotor speed is 1m/s in case (a) and 0.1 m/s in (b),(c),(d).
\begin{figure}
       \centering
        \begin{subfigure}[b]{0.33\textwidth}
        \includegraphics[width=\textwidth]{images/mov-side.jpg}
        \caption{}
        \end{subfigure}
        %
        \begin{subfigure}[b]{0.14\textwidth}
        \includegraphics[width=\textwidth]{images/mov-up-si.jpg}
        \caption{}
        \end{subfigure}
        %
        \begin{subfigure}[b]{0.235\textwidth}
        \includegraphics[width=\textwidth]{images/mov-up-sa.jpg}
        \caption{}
        \end{subfigure}
        \begin{subfigure}[b]{0.235\textwidth}
        \includegraphics[width=\textwidth]{images/mov-dn-sa.jpg}
        \caption{}
        \end{subfigure}
        % %
        % \begin{subfigure}[b]{0.48\textwidth}
        % \includegraphics[width=\textwidth]{images/Pybullet.jpg}
        % \end{subfigure}
        \caption{Interaction with moving obstacles: overtaking (a), (b), slowing (c), and reversing (d) behavior of the quadrotor, section \ref{section: 3D-CBF}}
        \label{fig:moving-obs}
    \end{figure}



\subsubsection{Interaction with long obstacles}
Fig. \ref{fig:long-obs} shows the quadrotor moving from side and top in (a), (b) respectively while interacting with an elongated obstacle. In all these cases the reference velocity of the quadrotor is 1m/s. 

    \begin{figure}
       \centering
        \begin{subfigure}[b]{0.22\textwidth}
        \includegraphics[width=\textwidth]{images/proj-ver.jpg}
        \caption{}
        \end{subfigure}
        %
        \begin{subfigure}[b]{0.22\textwidth}
        \includegraphics[width=\textwidth]{images/proj-hor.jpg}
        \caption{}
        \end{subfigure}
        % %
        % \begin{subfigure}[b]{0.48\textwidth}
        % \includegraphics[width=\textwidth]{images/Pybullet.jpg}
        % \end{subfigure}
        \caption{Interaction with longer obstacles: moving from side (a) and top (b), Section \ref{section: proj-CBF}. }
        \label{fig:long-obs}
    \end{figure}

\subsection{Comparison between C3BF and HO-CBF}
% Fig. \ref{fig:cc-ho-comp} shows the comparison of trajectories of the quadrotor when following C3BF and HO CBF with a static obstacle. 
All the aforementioned cases were tested with the HO-CBF to compare its performance against C3BF. We observe that the HO-CBF could not avoid a high-speed approaching obstacle. Moreover, it is not able to properly avoid the longer obstacles in the projection CBF case. These shortcomings of the Higher Order CBF are demonstrated in the supplementary video.


\subsection{Robustness of C3BF}
Without changing the above control framework we can observe that the C3BF is robust in the following two cases: 

\subsubsection{Multiple Obstacles}
We have considered the scenario where the quadrotor is made to move through a series of obstacles (both Spherical and Long obstacles) as in Fig. \ref{fig:robustness} (a) \& (b). We observe that the quadrotor is able to successfully navigate through this complex environment by avoiding all the obstacles, thus demonstrating robustness with respect to multiple obstacles.

\subsubsection{Multiple quadrotors with C3BF-QPs}
We have also considered the multi-agent scenarios where the multiple quadrotors have the collision cone CBF-QP operational as shown in Fig. \ref{fig:robustness} (c). We observe that both the ego-quadrotor and the approaching quadrotor are able to avoid collision in different configurations (static or moving), thus demonstrating robustness with respect to obstacles following the same Collision Cone CBF controller. 

The supplementary video shows the simulation video of all the scenarios shown in Fig \ref{fig:robustness}.

\begin{figure}
       \centering
        \begin{subfigure}[b]{0.46\textwidth}
        \includegraphics[width=\textwidth]{images/MO-3D.jpg}
        \caption{}
        \end{subfigure}
        %
        \begin{subfigure}[b]{0.155\textwidth}
        \includegraphics[width=\textwidth]{images/MO-proj.jpg}
        \caption{}
        \end{subfigure}
        %
        \begin{subfigure}[b]{0.29\textwidth}
        \includegraphics[width=\textwidth]{images/Robustness.jpg}
        \caption{}
        \end{subfigure}
        % %
        % \begin{subfigure}[b]{0.48\textwidth}
        % \includegraphics[width=\textwidth]{images/Pybullet.jpg}
        % \end{subfigure}
        \caption{Robustness in scenarios with multiple obstacles (a), (b) and with obstacle also following Collision Cone CBF(c).}
        \label{fig:robustness}
    \end{figure}


\section{Conclusions}
\label{section: Conclusions}
\section{Conclusions}
We consider the phase-extraction problem, and we showed that, given a unitary $U = e^{i\pi H}$ and its inverse $U^{\dag}$, we could implement a block-encoding of $\phi(H)$ for some smooth function $\phi(x)$. The word `smooth' here means existence and continuity of the derivatives: the higher the number of continuous derivatives that a function has, the faster its Fourier sum (and thus the Laurent polynomial on the eigenphases) uniformly converges to that function. We are confident this can have many more applications beyond what is shown in this work. It is also worth remarking that Jackson showed that the convergence rate of a Fourier series is almost-optimal, in the sense that no trigonometric (or, equivalently, complex exponential) series can approximate the desired function faster, up to that $\log d$ factor~\cite[p.\ 21]{jacksonTheoryApproximation1930a}. Also remember that `smoothing' a function, i.e., replacing its derivative with a continuous function, does not give faster convergence for free in general, as its derivative will become steep in the points where we smooth out discontinuities, and this translates to a high Lipschitz constant: a~clear example is given by Eq.~\ref{eq:lipschitz-constant-recurrence-solution}, but in that case, fortunately, nothing depends on the size of the input $N$, and thus does not influence the asymptotic query complexity of Algorithm~\ref{alg:prop-sampling-qsp}, although the constant factor can become large even for $p = 20$. From a theoretical point of view, this work shows that, for any $\eta > 0$, there is an algorithm with query complexity 
$$\Tilde{\bigO}\left(\frac{1}{\bar{c}^{\frac{1}{2} + \eta}} \frac{1}{\epsilon^\eta} \right)$$
solving the proportional-sampling problem. This statement seems to suggest there exists an algorithm which directly solves the problem with $\eta = 0$, and an open question would be to find such algorithm.


It is also interesting to remark that Theorems~\ref{thm:haah-construction},~\ref{thm:haah-completion} indeed allow the construction for any $\phi$, even complex-valued, provided that its absolute value is reciprocal.

One could think that, in Section~\ref{sec:prop-sampling}, instead of using the linear function in the phase-extraction subroutine, we could approximate the square root and then apply the transformation directly on $e^{i \pi c(x)}$. However, in the case of proportional sampling this would be inconvenient, as the derivative of the square root function has a discontinuity with an infinite jump around 0, and we could not choose a constant $\delta$ if we had values of the oracle that are too close to $0$.


\label{section: References}
%%% -*-BibTeX-*-
%%% Do NOT edit. File created by BibTeX with style
%%% ACM-Reference-Format-Journals [18-Jan-2012].

\begin{thebibliography}{52}

%%% ====================================================================
%%% NOTE TO THE USER: you can override these defaults by providing
%%% customized versions of any of these macros before the \bibliography
%%% command.  Each of them MUST provide its own final punctuation,
%%% except for \shownote{}, \showDOI{}, and \showURL{}.  The latter two
%%% do not use final punctuation, in order to avoid confusing it with
%%% the Web address.
%%%
%%% To suppress output of a particular field, define its macro to expand
%%% to an empty string, or better, \unskip, like this:
%%%
%%% \newcommand{\showDOI}[1]{\unskip}   % LaTeX syntax
%%%
%%% \def \showDOI #1{\unskip}           % plain TeX syntax
%%%
%%% ====================================================================

\ifx \showCODEN    \undefined \def \showCODEN     #1{\unskip}     \fi
\ifx \showDOI      \undefined \def \showDOI       #1{#1}\fi
\ifx \showISBNx    \undefined \def \showISBNx     #1{\unskip}     \fi
\ifx \showISBNxiii \undefined \def \showISBNxiii  #1{\unskip}     \fi
\ifx \showISSN     \undefined \def \showISSN      #1{\unskip}     \fi
\ifx \showLCCN     \undefined \def \showLCCN      #1{\unskip}     \fi
\ifx \shownote     \undefined \def \shownote      #1{#1}          \fi
\ifx \showarticletitle \undefined \def \showarticletitle #1{#1}   \fi
\ifx \showURL      \undefined \def \showURL       {\relax}        \fi
% The following commands are used for tagged output and should be
% invisible to TeX
\providecommand\bibfield[2]{#2}
\providecommand\bibinfo[2]{#2}
\providecommand\natexlab[1]{#1}
\providecommand\showeprint[2][]{arXiv:#2}

\bibitem[\protect\citeauthoryear{Albrecht and Stone}{Albrecht and
  Stone}{2017}]%
        {Albrecht2017ReasoningAH}
\bibfield{author}{\bibinfo{person}{Stefano~V. Albrecht} {and}
  \bibinfo{person}{P. Stone}.} \bibinfo{year}{2017}\natexlab{}.
\newblock \showarticletitle{Reasoning about Hypothetical Agent Behaviours and
  their Parameters}. In \bibinfo{booktitle}{\emph{AAMAS}}.
\newblock


\bibitem[\protect\citeauthoryear{Andrejczuk, Berger, Rodriguez-Aguilar, Sierra,
  and Mar{\'\i}n-Puchades}{Andrejczuk et~al\mbox{.}}{2018}]%
        {andrejczuk2018composition}
\bibfield{author}{\bibinfo{person}{Ewa Andrejczuk}, \bibinfo{person}{Rita
  Berger}, \bibinfo{person}{Juan~A Rodriguez-Aguilar}, \bibinfo{person}{Carles
  Sierra}, {and} \bibinfo{person}{V{\'\i}ctor Mar{\'\i}n-Puchades}.}
  \bibinfo{year}{2018}\natexlab{}.
\newblock \showarticletitle{The composition and formation of effective teams:
  computer science meets organizational psychology}.
\newblock \bibinfo{journal}{\emph{The Knowledge Engineering Review}}
  \bibinfo{volume}{33} (\bibinfo{year}{2018}), \bibinfo{pages}{e17}.
\newblock


\bibitem[\protect\citeauthoryear{Arjona-Medina, Gillhofer, Widrich,
  Unterthiner, Brandstetter, and Hochreiter}{Arjona-Medina
  et~al\mbox{.}}{2019}]%
        {arjona2019rudder}
\bibfield{author}{\bibinfo{person}{Jose~A Arjona-Medina},
  \bibinfo{person}{Michael Gillhofer}, \bibinfo{person}{Michael Widrich},
  \bibinfo{person}{Thomas Unterthiner}, \bibinfo{person}{Johannes
  Brandstetter}, {and} \bibinfo{person}{Sepp Hochreiter}.}
  \bibinfo{year}{2019}\natexlab{}.
\newblock \showarticletitle{Rudder: Return decomposition for delayed rewards}.
\newblock \bibinfo{journal}{\emph{NeurIPS}}  \bibinfo{volume}{32}
  (\bibinfo{year}{2019}).
\newblock


\bibitem[\protect\citeauthoryear{Beal, Changder, Norman, and Ramchurn}{Beal
  et~al\mbox{.}}{2020}]%
        {beal2020learning}
\bibfield{author}{\bibinfo{person}{Ryan Beal}, \bibinfo{person}{Narayan
  Changder}, \bibinfo{person}{Timothy Norman}, {and} \bibinfo{person}{Sarvapali
  Ramchurn}.} \bibinfo{year}{2020}\natexlab{}.
\newblock \showarticletitle{Learning the value of teamwork to form efficient
  teams}. In \bibinfo{booktitle}{\emph{Proceedings of the AAAI Conference on
  Artificial Intelligence}}, Vol.~\bibinfo{volume}{34}.
  \bibinfo{pages}{7063--7070}.
\newblock


\bibitem[\protect\citeauthoryear{Beetz, Hoyningen-Huene, Bandouch,
  Kirchlechner, Gedikli, and Maldonado}{Beetz et~al\mbox{.}}{2006}]%
        {beetz2006camera}
\bibfield{author}{\bibinfo{person}{Michael Beetz}, \bibinfo{person}{Nico~v
  Hoyningen-Huene}, \bibinfo{person}{Jan Bandouch}, \bibinfo{person}{Bernhard
  Kirchlechner}, \bibinfo{person}{Suat Gedikli}, {and} \bibinfo{person}{Alexis
  Maldonado}.} \bibinfo{year}{2006}\natexlab{}.
\newblock \showarticletitle{Camera-based observation of football games for
  analyzing multi-agent activities}. In \bibinfo{booktitle}{\emph{Proceedings
  of the fifth international joint conference on Autonomous agents and
  multiagent systems}}. \bibinfo{pages}{42--49}.
\newblock


\bibitem[\protect\citeauthoryear{Bialkowski, Lucey, Carr, Yue, Sridharan, and
  Matthews}{Bialkowski et~al\mbox{.}}{2014}]%
        {bialkowski2014large}
\bibfield{author}{\bibinfo{person}{Alina Bialkowski}, \bibinfo{person}{Patrick
  Lucey}, \bibinfo{person}{Peter Carr}, \bibinfo{person}{Yisong Yue},
  \bibinfo{person}{Sridha Sridharan}, {and} \bibinfo{person}{Iain Matthews}.}
  \bibinfo{year}{2014}\natexlab{}.
\newblock \showarticletitle{Large-scale analysis of soccer matches using
  spatiotemporal tracking data}. In \bibinfo{booktitle}{\emph{2014 IEEE
  international conference on data mining}}. IEEE, \bibinfo{pages}{725--730}.
\newblock


\bibitem[\protect\citeauthoryear{Bouveret and Lang}{Bouveret and Lang}{2014}]%
        {bouveret2014manipulating}
\bibfield{author}{\bibinfo{person}{Sylvain Bouveret} {and}
  \bibinfo{person}{J{\'e}r{\^o}me Lang}.} \bibinfo{year}{2014}\natexlab{}.
\newblock \showarticletitle{Manipulating picking sequences.}. In
  \bibinfo{booktitle}{\emph{ECAI}}, Vol.~\bibinfo{volume}{14}.
  \bibinfo{pages}{141--146}.
\newblock


\bibitem[\protect\citeauthoryear{Brams and Straffin~Jr}{Brams and
  Straffin~Jr}{1979}]%
        {brams1979prisoners}
\bibfield{author}{\bibinfo{person}{Steven~J Brams} {and}
  \bibinfo{person}{Philip~D Straffin~Jr}.} \bibinfo{year}{1979}\natexlab{}.
\newblock \showarticletitle{Prisoners' dilemma and professional sports drafts}.
\newblock \bibinfo{journal}{\emph{The American Mathematical Monthly}}
  \bibinfo{volume}{86}, \bibinfo{number}{2} (\bibinfo{year}{1979}),
  \bibinfo{pages}{80--88}.
\newblock


\bibitem[\protect\citeauthoryear{Bransen and Van~Haaren}{Bransen and
  Van~Haaren}{2020}]%
        {bransen2020player}
\bibfield{author}{\bibinfo{person}{Lotte Bransen} {and} \bibinfo{person}{Jan
  Van~Haaren}.} \bibinfo{year}{2020}\natexlab{}.
\newblock \showarticletitle{Player chemistry: Striving for a perfectly balanced
  soccer team}.
\newblock \bibinfo{journal}{\emph{Sports Analytics Conference}}
  (\bibinfo{year}{2020}).
\newblock


\bibitem[\protect\citeauthoryear{Dafoe, Bachrach, Hadfield, Horvitz, Larson,
  and Graepel}{Dafoe et~al\mbox{.}}{2021}]%
        {DafoeNature2021}
\bibfield{author}{\bibinfo{person}{Allan Dafoe}, \bibinfo{person}{Yoram
  Bachrach}, \bibinfo{person}{Gillian Hadfield}, \bibinfo{person}{Eric
  Horvitz}, \bibinfo{person}{Kate Larson}, {and} \bibinfo{person}{Thore
  Graepel}.} \bibinfo{year}{2021}\natexlab{}.
\newblock \showarticletitle{Cooperative {AI}: machines must learn to find
  common ground}.
\newblock \bibinfo{journal}{\emph{Nature}}  \bibinfo{volume}{593}
  (\bibinfo{year}{2021}), \bibinfo{pages}{33--36}.
\newblock


\bibitem[\protect\citeauthoryear{Derks and Peters}{Derks and Peters}{1993}]%
        {derks1993shapley}
\bibfield{author}{\bibinfo{person}{Jean Derks} {and} \bibinfo{person}{Hans
  Peters}.} \bibinfo{year}{1993}\natexlab{}.
\newblock \showarticletitle{A Shapley value for games with restricted
  coalitions}.
\newblock \bibinfo{journal}{\emph{International Journal of Game Theory}}
  \bibinfo{volume}{21}, \bibinfo{number}{4} (\bibinfo{year}{1993}),
  \bibinfo{pages}{351--360}.
\newblock


\bibitem[\protect\citeauthoryear{Durugkar, Liebman, and Stone}{Durugkar
  et~al\mbox{.}}{2020}]%
        {Durugkar2020BalancingIP}
\bibfield{author}{\bibinfo{person}{Ishan Durugkar}, \bibinfo{person}{E.
  Liebman}, {and} \bibinfo{person}{P. Stone}.} \bibinfo{year}{2020}\natexlab{}.
\newblock \showarticletitle{Balancing Individual Preferences and Shared
  Objectives in Multiagent Reinforcement Learning}. In
  \bibinfo{booktitle}{\emph{IJCAI}}.
\newblock


\bibitem[\protect\citeauthoryear{Elitzur}{Elitzur}{2020}]%
        {elitzur2020data}
\bibfield{author}{\bibinfo{person}{Ramy Elitzur}.}
  \bibinfo{year}{2020}\natexlab{}.
\newblock \showarticletitle{Data analytics effects in major league baseball}.
\newblock \bibinfo{journal}{\emph{Omega}}  \bibinfo{volume}{90}
  (\bibinfo{year}{2020}), \bibinfo{pages}{102001}.
\newblock


\bibitem[\protect\citeauthoryear{Ellis}{Ellis}{1983}]%
        {ellis1983similarities}
\bibfield{author}{\bibinfo{person}{M Ellis}.} \bibinfo{year}{1983}\natexlab{}.
\newblock \showarticletitle{Similarities and differences in games: A system for
  classification}. In \bibinfo{booktitle}{\emph{International association for
  physical education in higher education Conference}}.
\newblock


\bibitem[\protect\citeauthoryear{Fern{\'a}ndez, Bornn, and
  Cervone}{Fern{\'a}ndez et~al\mbox{.}}{2021}]%
        {fernandez2021framework}
\bibfield{author}{\bibinfo{person}{Javier Fern{\'a}ndez}, \bibinfo{person}{Luke
  Bornn}, {and} \bibinfo{person}{Daniel Cervone}.}
  \bibinfo{year}{2021}\natexlab{}.
\newblock \showarticletitle{A framework for the fine-grained evaluation of the
  instantaneous expected value of soccer possessions}.
\newblock \bibinfo{journal}{\emph{Machine Learning}} \bibinfo{volume}{110},
  \bibinfo{number}{6} (\bibinfo{year}{2021}), \bibinfo{pages}{1389--1427}.
\newblock


\bibitem[\protect\citeauthoryear{Fisac, Bronstein, Stefansson, Sadigh, Sastry,
  and Dragan}{Fisac et~al\mbox{.}}{2019}]%
        {fisac2019hierarchical}
\bibfield{author}{\bibinfo{person}{Jaime~F Fisac}, \bibinfo{person}{Eli
  Bronstein}, \bibinfo{person}{Elis Stefansson}, \bibinfo{person}{Dorsa
  Sadigh}, \bibinfo{person}{S~Shankar Sastry}, {and} \bibinfo{person}{Anca~D
  Dragan}.} \bibinfo{year}{2019}\natexlab{}.
\newblock \showarticletitle{Hierarchical game-theoretic planning for autonomous
  vehicles}. In \bibinfo{booktitle}{\emph{ICRA}}. IEEE,
  \bibinfo{pages}{9590--9596}.
\newblock


\bibitem[\protect\citeauthoryear{Garner, Humphrey, and Simkins}{Garner
  et~al\mbox{.}}{2016}]%
        {garner2016business}
\bibfield{author}{\bibinfo{person}{Jacqueline Garner},
  \bibinfo{person}{Phillip~R Humphrey}, {and} \bibinfo{person}{Betty Simkins}.}
  \bibinfo{year}{2016}\natexlab{}.
\newblock \showarticletitle{The business of sport and the sport of business: A
  review of the compensation literature in finance and sports}.
\newblock \bibinfo{journal}{\emph{International Review of Financial Analysis}}
  \bibinfo{volume}{47} (\bibinfo{year}{2016}), \bibinfo{pages}{197--204}.
\newblock


\bibitem[\protect\citeauthoryear{Goes, Kempe, Meerhoff, and Lemmink}{Goes
  et~al\mbox{.}}{2019}]%
        {goes2019not}
\bibfield{author}{\bibinfo{person}{Floris~R Goes}, \bibinfo{person}{Matthias
  Kempe}, \bibinfo{person}{Laurentius~A Meerhoff}, {and}
  \bibinfo{person}{Koen~APM Lemmink}.} \bibinfo{year}{2019}\natexlab{}.
\newblock \showarticletitle{Not every pass can be an assist: a data-driven
  model to measure pass effectiveness in professional soccer matches}.
\newblock \bibinfo{journal}{\emph{Big data}} \bibinfo{volume}{7},
  \bibinfo{number}{1} (\bibinfo{year}{2019}), \bibinfo{pages}{57--70}.
\newblock


\bibitem[\protect\citeauthoryear{Hu, Xie, Liang, and Chang}{Hu
  et~al\mbox{.}}{2022}]%
        {hu2022policy}
\bibfield{author}{\bibinfo{person}{Siyi Hu}, \bibinfo{person}{Chuanlong Xie},
  \bibinfo{person}{Xiaodan Liang}, {and} \bibinfo{person}{Xiaojun Chang}.}
  \bibinfo{year}{2022}\natexlab{}.
\newblock \showarticletitle{Policy diagnosis via measuring role diversity in
  cooperative multi-agent {RL}}. In \bibinfo{booktitle}{\emph{ICML}}.
  \bibinfo{pages}{9041--9071}.
\newblock


\bibitem[\protect\citeauthoryear{Le, Yue, Carr, and Lucey}{Le
  et~al\mbox{.}}{2017}]%
        {le2017coordinated}
\bibfield{author}{\bibinfo{person}{Hoang~M Le}, \bibinfo{person}{Yisong Yue},
  \bibinfo{person}{Peter Carr}, {and} \bibinfo{person}{Patrick Lucey}.}
  \bibinfo{year}{2017}\natexlab{}.
\newblock \showarticletitle{Coordinated multi-agent imitation learning}. In
  \bibinfo{booktitle}{\emph{International Conference on Machine Learning}}.
  PMLR, \bibinfo{pages}{1995--2003}.
\newblock


\bibitem[\protect\citeauthoryear{Ledezma, Aler, Sanchis, and Borrajo}{Ledezma
  et~al\mbox{.}}{2009}]%
        {ledezma2009ombo}
\bibfield{author}{\bibinfo{person}{Agapito Ledezma}, \bibinfo{person}{Ricardo
  Aler}, \bibinfo{person}{Araceli Sanchis}, {and} \bibinfo{person}{Daniel
  Borrajo}.} \bibinfo{year}{2009}\natexlab{}.
\newblock \showarticletitle{OMBO: An opponent modeling approach}.
\newblock \bibinfo{journal}{\emph{{AI} Communications}} \bibinfo{volume}{22},
  \bibinfo{number}{1} (\bibinfo{year}{2009}), \bibinfo{pages}{21--35}.
\newblock


\bibitem[\protect\citeauthoryear{Lewis}{Lewis}{2004}]%
        {lewis2004moneyball}
\bibfield{author}{\bibinfo{person}{Michael Lewis}.}
  \bibinfo{year}{2004}\natexlab{}.
\newblock \bibinfo{booktitle}{\emph{Moneyball: The art of winning an unfair
  game}}.
\newblock \bibinfo{publisher}{WW Norton \& Company}.
\newblock


\bibitem[\protect\citeauthoryear{Liemhetcharat and Luo}{Liemhetcharat and
  Luo}{2015}]%
        {liemhetcharat2015applying}
\bibfield{author}{\bibinfo{person}{Somchaya Liemhetcharat} {and}
  \bibinfo{person}{Yicheng Luo}.} \bibinfo{year}{2015}\natexlab{}.
\newblock \showarticletitle{Applying the Synergy Graph Model to Human
  Basketball.}. In \bibinfo{booktitle}{\emph{AAMAS}}.
  \bibinfo{pages}{1695--1696}.
\newblock


\bibitem[\protect\citeauthoryear{Liu, Schulte, Poupart, Rudd, and Javan}{Liu
  et~al\mbox{.}}{2020}]%
        {liu2020learning}
\bibfield{author}{\bibinfo{person}{Guiliang Liu}, \bibinfo{person}{Oliver
  Schulte}, \bibinfo{person}{Pascal Poupart}, \bibinfo{person}{Mike Rudd},
  {and} \bibinfo{person}{Mehrsan Javan}.} \bibinfo{year}{2020}\natexlab{}.
\newblock \showarticletitle{Learning agent representations for ice hockey}.
\newblock \bibinfo{journal}{\emph{Advances in Neural Information Processing
  Systems}}  \bibinfo{volume}{33} (\bibinfo{year}{2020}),
  \bibinfo{pages}{18704--18715}.
\newblock


\bibitem[\protect\citeauthoryear{Ljung, Carlsson, and Lambrix}{Ljung
  et~al\mbox{.}}{2018}]%
        {Ljung2018PlayerPV}
\bibfield{author}{\bibinfo{person}{Dennis Ljung}, \bibinfo{person}{Niklas
  Carlsson}, {and} \bibinfo{person}{P. Lambrix}.}
  \bibinfo{year}{2018}\natexlab{}.
\newblock \showarticletitle{Player Pairs Valuation in Ice Hockey}. In
  \bibinfo{booktitle}{\emph{MLSA@PKDD/ECML}}.
\newblock


\bibitem[\protect\citeauthoryear{Lucey, Bialkowski, Carr, Foote, and
  Matthews}{Lucey et~al\mbox{.}}{2012}]%
        {lucey2012characterizing}
\bibfield{author}{\bibinfo{person}{Patrick Lucey}, \bibinfo{person}{Alina
  Bialkowski}, \bibinfo{person}{Peter Carr}, \bibinfo{person}{Eric Foote},
  {and} \bibinfo{person}{Iain Matthews}.} \bibinfo{year}{2012}\natexlab{}.
\newblock \showarticletitle{Characterizing multi-agent team behavior from
  partial team tracings: Evidence from the english premier league}. In
  \bibinfo{booktitle}{\emph{Proceedings of the AAAI Conference on Artificial
  Intelligence}}, Vol.~\bibinfo{volume}{26}. \bibinfo{pages}{1387--1393}.
\newblock


\bibitem[\protect\citeauthoryear{Pourmehr and Dadkhah}{Pourmehr and
  Dadkhah}{2011}]%
        {pourmehr2011overview}
\bibfield{author}{\bibinfo{person}{Shokoofeh Pourmehr} {and}
  \bibinfo{person}{Chitra Dadkhah}.} \bibinfo{year}{2011}\natexlab{}.
\newblock \showarticletitle{An overview on opponent modeling in RoboCup soccer
  simulation 2D}.
\newblock \bibinfo{journal}{\emph{Robot Soccer World Cup}}
  (\bibinfo{year}{2011}), \bibinfo{pages}{402--414}.
\newblock


\bibitem[\protect\citeauthoryear{Raabe, Nabben, and Memmert}{Raabe
  et~al\mbox{.}}{2022}]%
        {raabe2022graph}
\bibfield{author}{\bibinfo{person}{Dominik Raabe}, \bibinfo{person}{Reinhard
  Nabben}, {and} \bibinfo{person}{Daniel Memmert}.}
  \bibinfo{year}{2022}\natexlab{}.
\newblock \showarticletitle{Graph representations for the analysis of
  multi-agent spatiotemporal sports data}.
\newblock \bibinfo{journal}{\emph{Applied Intelligence}}
  (\bibinfo{year}{2022}), \bibinfo{pages}{1--21}.
\newblock


\bibitem[\protect\citeauthoryear{Radke, Brecht, and Radke}{Radke
  et~al\mbox{.}}{2022a}]%
        {radke2022identifying}
\bibfield{author}{\bibinfo{person}{David Radke}, \bibinfo{person}{Tim Brecht},
  {and} \bibinfo{person}{Daniel Radke}.} \bibinfo{year}{2022}\natexlab{a}.
\newblock \showarticletitle{Identifying Completed Pass Types and Improving
  Passing Lane Models}. In \bibinfo{booktitle}{\emph{Link{\"o}ping Hockey
  Analytics Conference}}. \bibinfo{pages}{71--86}.
\newblock


\bibitem[\protect\citeauthoryear{Radke, Larson, and Brecht}{Radke
  et~al\mbox{.}}{2022b}]%
        {Radke2022Exploring}
\bibfield{author}{\bibinfo{person}{David Radke}, \bibinfo{person}{Kate Larson},
  {and} \bibinfo{person}{Tim Brecht}.} \bibinfo{year}{2022}\natexlab{b}.
\newblock \showarticletitle{Exploring the Benefits of Teams in Multiagent
  Learning}. In \bibinfo{booktitle}{\emph{IJCAI}}.
\newblock


\bibitem[\protect\citeauthoryear{Radke, Larson, and Brecht}{Radke
  et~al\mbox{.}}{2022c}]%
        {radke2022importance}
\bibfield{author}{\bibinfo{person}{David Radke}, \bibinfo{person}{Kate Larson},
  {and} \bibinfo{person}{Tim Brecht}.} \bibinfo{year}{2022}\natexlab{c}.
\newblock \showarticletitle{The Importance of Credo in Multiagent Learning}.
\newblock \bibinfo{journal}{\emph{ALA Workshop at AAMAS}}
  (\bibinfo{year}{2022}).
\newblock


\bibitem[\protect\citeauthoryear{Radke, Radke, Brecht, and Pawelczyk}{Radke
  et~al\mbox{.}}{2021}]%
        {Radke2021Passing}
\bibfield{author}{\bibinfo{person}{D.~T. Radke}, \bibinfo{person}{D.~L. Radke},
  \bibinfo{person}{T. Brecht}, {and} \bibinfo{person}{A. Pawelczyk}.}
  \bibinfo{year}{2021}\natexlab{}.
\newblock \showarticletitle{Passing and Pressure Metrics in Ice Hockey}.
\newblock \bibinfo{journal}{\emph{Workshop of AI for Sports Analytics}}
  (\bibinfo{year}{2021}).
\newblock


\bibitem[\protect\citeauthoryear{Rahimian and Toka}{Rahimian and Toka}{2022}]%
        {rahimian2022optical}
\bibfield{author}{\bibinfo{person}{Pegah Rahimian} {and}
  \bibinfo{person}{Laszlo Toka}.} \bibinfo{year}{2022}\natexlab{}.
\newblock \showarticletitle{Optical tracking in team sports}.
\newblock \bibinfo{journal}{\emph{Journal of Quantitative Analysis in Sports}}
  \bibinfo{volume}{18}, \bibinfo{number}{1} (\bibinfo{year}{2022}),
  \bibinfo{pages}{35--57}.
\newblock


\bibitem[\protect\citeauthoryear{Rahwan, Michalak, Wooldridge, and
  Jennings}{Rahwan et~al\mbox{.}}{2015}]%
        {rahwan2015coalition}
\bibfield{author}{\bibinfo{person}{Talal Rahwan}, \bibinfo{person}{Tomasz~P
  Michalak}, \bibinfo{person}{Michael Wooldridge}, {and}
  \bibinfo{person}{Nicholas~R Jennings}.} \bibinfo{year}{2015}\natexlab{}.
\newblock \showarticletitle{Coalition structure generation: A survey}.
\newblock \bibinfo{journal}{\emph{Artificial Intelligence}}
  \bibinfo{volume}{229} (\bibinfo{year}{2015}), \bibinfo{pages}{139--174}.
\newblock


\bibitem[\protect\citeauthoryear{Rashid, Samvelyan, Schroeder, Farquhar,
  Foerster, and Whiteson}{Rashid et~al\mbox{.}}{2018}]%
        {rashid2018qmix}
\bibfield{author}{\bibinfo{person}{Tabish Rashid}, \bibinfo{person}{Mikayel
  Samvelyan}, \bibinfo{person}{Christian Schroeder}, \bibinfo{person}{Gregory
  Farquhar}, \bibinfo{person}{Jakob Foerster}, {and} \bibinfo{person}{Shimon
  Whiteson}.} \bibinfo{year}{2018}\natexlab{}.
\newblock \showarticletitle{Qmix: Monotonic value function factorisation for
  deep multi-agent reinforcement learning}. In
  \bibinfo{booktitle}{\emph{ICML}}. \bibinfo{pages}{4295--4304}.
\newblock


\bibitem[\protect\citeauthoryear{Rein and Memmert}{Rein and Memmert}{2016}]%
        {rein2016big}
\bibfield{author}{\bibinfo{person}{Robert Rein} {and} \bibinfo{person}{Daniel
  Memmert}.} \bibinfo{year}{2016}\natexlab{}.
\newblock \showarticletitle{Big data and tactical analysis in elite soccer:
  future challenges and opportunities for sports science}.
\newblock \bibinfo{journal}{\emph{SpringerPlus}} \bibinfo{volume}{5},
  \bibinfo{number}{1} (\bibinfo{year}{2016}), \bibinfo{pages}{1--13}.
\newblock


\bibitem[\protect\citeauthoryear{Ritchie, Harell, and Shreeves}{Ritchie
  et~al\mbox{.}}{2022}]%
        {ritchie2022pass}
\bibfield{author}{\bibinfo{person}{Robyn Ritchie}, \bibinfo{person}{Alon
  Harell}, {and} \bibinfo{person}{Phillip Shreeves}.}
  \bibinfo{year}{2022}\natexlab{}.
\newblock \showarticletitle{Pass Evaluation in Women's Olympic Ice Hockey}. In
  \bibinfo{booktitle}{\emph{Proceedings of the 5th International ACM Workshop
  on Multimedia Content Analysis in Sports}}. \bibinfo{pages}{65--73}.
\newblock


\bibitem[\protect\citeauthoryear{Sampaio, McGarry, Calleja-Gonz{\'a}lez,
  Jim{\'e}nez~S{\'a}iz, Schelling i~del Alc{\'a}zar, and Balciunas}{Sampaio
  et~al\mbox{.}}{2015}]%
        {sampaio2015exploring}
\bibfield{author}{\bibinfo{person}{Jaime Sampaio}, \bibinfo{person}{Tim
  McGarry}, \bibinfo{person}{Julio Calleja-Gonz{\'a}lez},
  \bibinfo{person}{Sergio Jim{\'e}nez~S{\'a}iz}, \bibinfo{person}{Xavi
  Schelling i~del Alc{\'a}zar}, {and} \bibinfo{person}{Mindaugas Balciunas}.}
  \bibinfo{year}{2015}\natexlab{}.
\newblock \showarticletitle{Exploring game performance in the National
  Basketball Association using player tracking data}.
\newblock \bibinfo{journal}{\emph{PloS one}} \bibinfo{volume}{10},
  \bibinfo{number}{7} (\bibinfo{year}{2015}), \bibinfo{pages}{e0132894}.
\newblock


\bibitem[\protect\citeauthoryear{Santos, Santos, Pacheco, and Levin}{Santos
  et~al\mbox{.}}{2021}]%
        {Santos2021SocialNI}
\bibfield{author}{\bibinfo{person}{F. Santos}, \bibinfo{person}{F.~C. Santos},
  \bibinfo{person}{J. Pacheco}, {and} \bibinfo{person}{S. Levin}.}
  \bibinfo{year}{2021}\natexlab{}.
\newblock \showarticletitle{Social Network Interventions to Prevent
  Reciprocity-driven Polarization}. In \bibinfo{booktitle}{\emph{AAMAS}}.
\newblock


\bibitem[\protect\citeauthoryear{Schr{\"o}der, Hoey, and Rogers}{Schr{\"o}der
  et~al\mbox{.}}{2016}]%
        {schroder2016modeling}
\bibfield{author}{\bibinfo{person}{Tobias Schr{\"o}der}, \bibinfo{person}{Jesse
  Hoey}, {and} \bibinfo{person}{Kimberly~B Rogers}.}
  \bibinfo{year}{2016}\natexlab{}.
\newblock \showarticletitle{Modeling dynamic identities and uncertainty in
  social interactions: Bayesian affect control theory}.
\newblock \bibinfo{journal}{\emph{American Sociological Review}}
  \bibinfo{volume}{81}, \bibinfo{number}{4} (\bibinfo{year}{2016}),
  \bibinfo{pages}{828--855}.
\newblock


\bibitem[\protect\citeauthoryear{Schuckers}{Schuckers}{2011}]%
        {schuckers2011s}
\bibfield{author}{\bibinfo{person}{Michael~E Schuckers}.}
  \bibinfo{year}{2011}\natexlab{}.
\newblock \showarticletitle{What's An NHL Draft Pick Worth? A Value Pick Chart
  for the National Hockey League}.
\newblock \bibinfo{journal}{\emph{Statistical Sports Consulting}}
  (\bibinfo{year}{2011}).
\newblock


\bibitem[\protect\citeauthoryear{Schulte, Khademi, Gholami, Zhao, Javan, and
  Desaulniers}{Schulte et~al\mbox{.}}{2017}]%
        {schulte2017markov}
\bibfield{author}{\bibinfo{person}{Oliver Schulte}, \bibinfo{person}{Mahmoud
  Khademi}, \bibinfo{person}{Sajjad Gholami}, \bibinfo{person}{Zeyu Zhao},
  \bibinfo{person}{Mehrsan Javan}, {and} \bibinfo{person}{Philippe
  Desaulniers}.} \bibinfo{year}{2017}\natexlab{}.
\newblock \showarticletitle{A Markov Game model for valuing actions, locations,
  and team performance in ice hockey}.
\newblock \bibinfo{journal}{\emph{Data Mining and Knowledge Discovery}}
  \bibinfo{volume}{31}, \bibinfo{number}{6} (\bibinfo{year}{2017}),
  \bibinfo{pages}{1735--1757}.
\newblock


\bibitem[\protect\citeauthoryear{Schwind, Demirovic, Inoue, and
  Lagniez}{Schwind et~al\mbox{.}}{2021}]%
        {schwind2021partial}
\bibfield{author}{\bibinfo{person}{Nicolas Schwind}, \bibinfo{person}{Emir
  Demirovic}, \bibinfo{person}{Katsumi Inoue}, {and}
  \bibinfo{person}{Jean-Marie Lagniez}.} \bibinfo{year}{2021}\natexlab{}.
\newblock \showarticletitle{Partial Robustness in Team Formation: Bridging the
  Gap between Robustness and Resilience.}. In
  \bibinfo{booktitle}{\emph{AAMAS}}, Vol.~\bibinfo{volume}{21}.
  \bibinfo{pages}{20th}.
\newblock


\bibitem[\protect\citeauthoryear{Simon}{Simon}{1990}]%
        {simon1990bounded}
\bibfield{author}{\bibinfo{person}{Herbert~A Simon}.}
  \bibinfo{year}{1990}\natexlab{}.
\newblock \showarticletitle{Bounded rationality}.
\newblock In \bibinfo{booktitle}{\emph{Utility and probability}}.
  \bibinfo{publisher}{Springer}, \bibinfo{pages}{15--18}.
\newblock


\bibitem[\protect\citeauthoryear{Spearman}{Spearman}{2018}]%
        {spearman2018beyond}
\bibfield{author}{\bibinfo{person}{William Spearman}.}
  \bibinfo{year}{2018}\natexlab{}.
\newblock \showarticletitle{Beyond expected goals}. In
  \bibinfo{booktitle}{\emph{Proceedings of the 12th MIT sloan sports analytics
  conference}}. \bibinfo{pages}{1--17}.
\newblock


\bibitem[\protect\citeauthoryear{Stone, Riley, and Veloso}{Stone
  et~al\mbox{.}}{2000}]%
        {stone2000defining}
\bibfield{author}{\bibinfo{person}{Peter Stone}, \bibinfo{person}{Patrick
  Riley}, {and} \bibinfo{person}{Manuela Veloso}.}
  \bibinfo{year}{2000}\natexlab{}.
\newblock \showarticletitle{Defining and using ideal teammate and opponent
  agent models}. In \bibinfo{booktitle}{\emph{AAAI/IAAI}}.
  \bibinfo{pages}{1040--1045}.
\newblock


\bibitem[\protect\citeauthoryear{Tuyls, Omidshafiei, Muller, Wang, Connor,
  Hennes, Graham, Spearman, Waskett, Steel, et~al\mbox{.}}{Tuyls
  et~al\mbox{.}}{2021}]%
        {tuyls2021game}
\bibfield{author}{\bibinfo{person}{Karl Tuyls}, \bibinfo{person}{Shayegan
  Omidshafiei}, \bibinfo{person}{Paul Muller}, \bibinfo{person}{Zhe Wang},
  \bibinfo{person}{Jerome Connor}, \bibinfo{person}{Daniel Hennes},
  \bibinfo{person}{Ian Graham}, \bibinfo{person}{William Spearman},
  \bibinfo{person}{Tim Waskett}, \bibinfo{person}{Dafydd Steel},
  {et~al\mbox{.}}} \bibinfo{year}{2021}\natexlab{}.
\newblock \showarticletitle{Game Plan: What {AI} can do for Football, and What
  Football can do for AI}.
\newblock \bibinfo{journal}{\emph{Journal of Artificial Intelligence Research}}
   \bibinfo{volume}{71} (\bibinfo{year}{2021}), \bibinfo{pages}{41--88}.
\newblock


\bibitem[\protect\citeauthoryear{Van Der~Hoek, Jamroga, and Wooldridge}{Van
  Der~Hoek et~al\mbox{.}}{2005}]%
        {van2005logic}
\bibfield{author}{\bibinfo{person}{Wiebe Van Der~Hoek},
  \bibinfo{person}{Wojciech Jamroga}, {and} \bibinfo{person}{Michael
  Wooldridge}.} \bibinfo{year}{2005}\natexlab{}.
\newblock \showarticletitle{A logic for strategic reasoning}. In
  \bibinfo{booktitle}{\emph{Proceedings of the fourth international joint
  conference on Autonomous agents and multiagent systems}}.
  \bibinfo{pages}{157--164}.
\newblock


\bibitem[\protect\citeauthoryear{Vats, Fani, Clausi, and Zelek}{Vats
  et~al\mbox{.}}{2022}]%
        {vats2022evaluating}
\bibfield{author}{\bibinfo{person}{Kanav Vats}, \bibinfo{person}{Mehrnaz Fani},
  \bibinfo{person}{David~A Clausi}, {and} \bibinfo{person}{John~S Zelek}.}
  \bibinfo{year}{2022}\natexlab{}.
\newblock \showarticletitle{Evaluating deep tracking models for player tracking
  in broadcast ice hockey video}.
\newblock \bibinfo{journal}{\emph{arXiv preprint arXiv:2205.10949}}
  (\bibinfo{year}{2022}).
\newblock


\bibitem[\protect\citeauthoryear{Visser, Dr{\"u}cker, H{\"u}bner, Schmidt, and
  Weland}{Visser et~al\mbox{.}}{2000}]%
        {visser2000recognizing}
\bibfield{author}{\bibinfo{person}{Ubbo Visser}, \bibinfo{person}{Christian
  Dr{\"u}cker}, \bibinfo{person}{Sebastian H{\"u}bner}, \bibinfo{person}{Esko
  Schmidt}, {and} \bibinfo{person}{Hans-Georg Weland}.}
  \bibinfo{year}{2000}\natexlab{}.
\newblock \showarticletitle{Recognizing formations in opponent teams}. In
  \bibinfo{booktitle}{\emph{Robot Soccer World Cup}}. Springer,
  \bibinfo{pages}{391--396}.
\newblock


\bibitem[\protect\citeauthoryear{Williamson and Cox}{Williamson and
  Cox}{2014}]%
        {williamson2014distributed}
\bibfield{author}{\bibinfo{person}{Kellie Williamson} {and}
  \bibinfo{person}{Rochelle Cox}.} \bibinfo{year}{2014}\natexlab{}.
\newblock \showarticletitle{Distributed cognition in sports teams: Explaining
  successful and expert performance}.
\newblock \bibinfo{journal}{\emph{Educational Philosophy and Theory}}
  \bibinfo{volume}{46}, \bibinfo{number}{6} (\bibinfo{year}{2014}),
  \bibinfo{pages}{640--654}.
\newblock


\bibitem[\protect\citeauthoryear{Yan, Kroer, and Peysakhovich}{Yan
  et~al\mbox{.}}{2020}]%
        {Yan2020EvaluatingAR}
\bibfield{author}{\bibinfo{person}{Tom Yan}, \bibinfo{person}{Christian Kroer},
  {and} \bibinfo{person}{A. Peysakhovich}.} \bibinfo{year}{2020}\natexlab{}.
\newblock \showarticletitle{Evaluating and Rewarding Teamwork Using Cooperative
  Game Abstractions}.
\newblock \bibinfo{journal}{\emph{NeurIPS}} (\bibinfo{year}{2020}).
\newblock


\end{thebibliography}


\end{document}
