% \par Quadrotors are used in a wide range of applications, including search and rescue, environmental monitoring, agriculture, transportation, and entertainment \cite{kumar2015future}. In many of these applications, quadrotors operate in complex and dynamic environments, where they must navigate around obstacles such as trees, buildings, and other vehicles. Collision avoidance is essential for enabling quadrotors to carry out their tasks effectively and safely, while minimizing the risk of accidents and damage. There exist a plethora of methods in the literature, such as the potential field \cite{8022685} , reachability analysis \cite{8263977} \cite{RA-UAV}  and model predictive control \cite{8442967} to solve the problem of obstacle avoidance in UAVs.

% In recent years, the Control Barrier Functions \cite{Ames_2017} (CBFs) approach has gained considerable attention as a promising strategy for ensuring safe operation of autonomous systems. This approach provides a computationally efficient method that is able to handle complex situations while guaranteeing provable safety. CBF is a model-based control design method that can be formulated as a Quadratic Problem (QP) and solved online, similar to Model Predictive Control (MPC). However, unlike MPC, feasible state sets under CBF-based control possess a forward invariance guarantee property, CBF does not rely on convex programming, sequential convex programming or mixed-integer programming for non-convex constraints. Additionally, CBF does not require future state predictions, making it a more flexible and adaptable method for ensuring safe operation in a wide range of scenarios.

% CBFs are generally considered to be more suited for safety-critical applications as they provide stronger safety guarantees than Nonlinear MPC (NMPC). CBFs are specifically designed to enforce safety constraints and provide hard constraints on the system's trajectory. In contrast, NMPC provides soft constraints on the system trajectory, and the degree of constraint satisfaction depends on the optimization algorithm's performance.

% Reachability analysis may be limited in situations where safety-critical issues involve complex interactions between subsystems, or where safety requirements are highly dynamic and subject to frequent changes. CBFs are more suitable for applications where safety requirements are highly dynamic and subject to frequent changes, while reachability analysis is more suitable for applications where safety-critical issues involve complex interactions between subsystems. \cite{https://doi.org/10.48550/arxiv.2106.13176}

% APF is easy to implement, it suffers from several limitations, such as the possibility of getting stuck in local minima, and difficulties in handling complex environments with multiple obstacles. \cite{Singletary2021ComparativeAO}


\par Quadrotors are used in a wide range of applications, including search and rescue, environmental monitoring, agriculture, transportation, and entertainment \cite{kumar2015future}. In many of these applications, quadrotors operate in complex and dynamic environments, where they must navigate around obstacles such as trees, buildings, and other drones. The literature presents a variety of methods such as artificial potential field \cite{8022685}, reachability analysis \cite{8263977} \cite{RA-UAV}, and nonlinear model predictive control \cite{8442967} to address the problem of obstacle avoidance in UAVs.

In recent years, the Control Barrier Functions (CBFs) based approach \cite{7040372}\cite{Ames_2017} has emerged as a promising strategy for ensuring safe operation of autonomous systems. This is a model-based control design method, which provides a computationally efficient solution that can handle complex situations while guaranteeing safety. CBFs can be formulated as a Quadratic Problem (QP) and can be solved online, making them well-suited for real-time safety-critical applications. % where strong safety guarantees are essential. 
CBFs are specifically designed to enforce safety constraints and provide hard constraints on the system's trajectory, making them superior to Nonlinear MPC (NMPC) in terms of safety guarantees. NMPC, on the other hand, provides soft constraints on the system's trajectory, with the degree of constraint satisfaction dependent on the optimization algorithm's performance.

In situations that involve complex interactions between subsystems or where safety requirements are highly dynamic and subject to frequent changes, reachability analysis may be limited. In such cases, CBFs are more suitable due to their ability to handle highly dynamic safety requirements \cite{https://doi.org/10.48550/arxiv.2106.13176}. While the artificial potential field approach is easy to implement, it suffers from limitations such as the possibility of getting stuck in local minima and difficulties in handling complex environments with multiple obstacles and \cite{Singletary2021ComparativeAO} shows that CBFs offer a viable, and arguably improved alternative to APFs for real-time obstacle avoidance.  Thus, the Control Barrier Functions approach provides a better solution for obstacle avoidance in UAVs, particularly in safety-critical scenarios. \cite{7525253} has shown collision avoidance using CBFs in planar quadrotor case.
% 
In a recent work, \cite{DBLP:journals/corr/abs-1903-04706, 9516971} showed that Higher order CBFs are a generalised form of the exponential CBFs, thus it also addresses this problem of obstacle avoidance. However, a major challenge with this approach is the need to identify suitable penalty parameters (p's) and class $\mathcal{K}$ functions ($\alpha$'s) that can yield optimal results.

% The Collision Cone \cite{Fiorini1993, doi:10.1177/027836499801700706, 709600} approach is a popular method for collision avoidance planning in robotics and autonomous vehicles. It involves defining a cone-shaped region between a pair of objects, which represents the potential area of collision if the relative velocity vector is pointing inside of it. By continuously monitoring this cone and adjusting the object's trajectory to avoid the relative velocity vector from falling within the cone, collision avoidance can be achieved. 

% The Collision Cone approach has several advantages over other collision avoidance approaches. One of the main advantages is its simplicity and efficiency. The method can be easily integrated into existing motion planning algorithms and can be quickly implemented in various robotic and autonomous systems. Additionally, it provides a reliable and flexible means of avoiding collisions, as it takes into account the speed and direction of the moving object, as well as the shape and size of any potential obstacles in the environment. Another advantage of this approach is that the Collision Cone approach can be adapted to work in various environments, including dynamic and unpredictable ones, making it a versatile and widely applicable method for collision avoidance. 

With regards to obstacle avoidance in dynamic environments, another class of approaches that is widely used is the method of collision cones \cite{Fiorini1993}, \cite{ doi:10.1177/027836499801700706}, \cite{709600}
% The Collision Cone \cite{Fiorini1993}, \cite{ doi:10.1177/027836499801700706}, \cite{709600} approach is a widely used method for collision avoidance in robotics and autonomous vehicles. 
It involves defining a cone-shaped region between two objects to represent the potential area of collision, which can be avoided by adjusting the object's trajectory to prevent the relative velocity vector from falling within the cone. This approach has several advantages, including its simplicity, efficiency, and adaptability to different environments. The method can be easily integrated into existing motion planning algorithms, takes into account the speed and direction of moving objects and the shape and size of potential obstacles, and can work in dynamic and unpredictable environments. As a result, the Collision Cone approach provides a reliable and flexible means of avoiding collisions in various robotic and autonomous systems. 

The method of collision cones, despite its simplicity and effectiveness, have largely been restricted to offline motion planning/navigation problems, and their extensions for real-time implementations have been limited. However, by exploiting the CBF-QP formulations, we can synthesize a new class of CBFs through the notion of collision cones, which can then be implemented in real-time. This will be the main objective of this paper. This idea was originally proposed in \cite{C3BF} for the planar case (2D) and for wheeled robots, while we aim to extend this for 3D and for quadrotors, which are underactauted and have higher degrees of freedom (DoF).
% Therefore, to provide safe navigation in cluttered dynamic scenarios in real-time, we exploit the collision cone approach combined with control barrier functions. 
% In this paper, we extend the previous work \cite{C3BF} on Collision Cone CBFs by including the dynamics of the quadrotor and exploring its interaction with obstacles of various shapes.

\begin{figure}[t]
    \centering
    % \begin{tikzpicture}
    %     \node[anchor=south west,inner sep=0] at (0,0) {\includegraphics[width=0.90\linewidth] {images/crazyflie.png}};
        
    %     \draw [red, thick,dashed] (3,3.3) -- (4.5,2.4) node [midway, right] {. $x_p$};
    %     \draw [green, thick,dashed] (2.5,1) -- (4.5,2.4) node [midway, right] {. $y_p$};
    %     \draw [<->,blue, thick,dashed] (5,2.2) -- (5,2.8) node [midway, right] {. $l$};
        
    % \end{tikzpicture}
    \includegraphics[width=0.90\linewidth]{images/Crazyflie.jpg}
\caption{World coordinates and body fixed coordinates of Crazyflie and Euler's angles defined in these coordinates}
\label{fig:models}
\end{figure}


\subsection{Contribution and Paper Structure}
Main idea is to realize a CBF-QP formulation for the quadrotor dynamics and for obstacles with non-zero velocity values.
% We consider the quadrotor dynamics and the relative position \& velocity of the obstacle to develop a safe controller, which avoids a kinematic obstacle. 
Main contributions of our work are:

\begin{itemize}
    \item We formulate a direct method for safe trajectory tracking of quadrotors based on collision cone control barrier functions expressed through a quadratic program.
    \item We consider static and constant velocity obstacles of various dimensions and provide mathematical guarantees for collision avoidance.
    \item We compare the collision cone CBF with the state-of-the-art higher-order CBF (HO-CBF), and show how the former is better in terms of feasibility and safety guarantees. 
\end{itemize}



\subsection{Organisation}
The rest of this paper is organized as follows. Preliminaries explaining the quadrotor model, the concept of control barrier functions (CBFs), collision cone CBFs and controller design are introduced in section \ref{section: Background}. The application of the above CBFs on the quadrotor to avoid obstacles of various shapes is discussed in section \ref{section: Safety Guarantee}. The Simulation setup and results will be discussed in section \ref{section: Simulation Results}. Finally, we present our conclusion in section \ref{section: Conclusions}. 
