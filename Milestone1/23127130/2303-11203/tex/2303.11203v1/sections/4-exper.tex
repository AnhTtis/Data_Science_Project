%%%%%%%%% Experiments
\vspace{-0.3cm}   
\section{Evaluation}
\label{sec:experiments}
\noindent
We evaluate our proposed \textit{Less is More 3D} ({\ourmodel}) approach against
state-of-the-art 3D point cloud semantic segmentation approaches using the SemanticKITTI~\cite{behley2019semantickittia} and ScribbleKITTI~\cite{Unal_2022_CVPR} benchmark datasets.



\begin{table*}[htp]
    \scriptsize
    \setlength{\abovecaptionskip}{0.05cm}
    \centering
    \caption{Comparative mIoU for Range- and Voxel-based methods using uniform sampling (U), sequential partition (P) and {\samplshort} sampling (S): \textbf{bold}/\underline{underlined} = \textbf{best}/\underline{2nd best}; $^*$ denotes reproduced result; -- denotes missing result due to unavailability from original authors.}
    \resizebox{\textwidth}{!}
    {
    \begin{tabular}{@{}C{0.5cm}|C{0.4cm}|R{1.55cm}L{1.05cm}|C{0.35cm}C{0.35cm}C{0.35cm}C{0.35cm}C{0.35cm}C{0.35cm}C{0.35cm}|C{0.35cm}C{0.35cm}C{0.35cm}C{0.35cm}C{0.35cm}C{0.35cm}C{0.35cm}}
    \topr
    \multirow{2}{*}{Repr.} & \multirow{2}{*}{Samp.} & \multirow{2}{*}{Method} &  & \multicolumn{7}{c|}
    {SemanticKITTI~\cite{behley2019semantickittia}} & \multicolumn{7}{c}{ScribbleKITTI~\cite{Unal_2022_CVPR}} \\
    &  &  &  & \hspace{0.5pt}1\% & \hspace{0.5pt}5\% & 10\% & 20\% & 40\% & 50\% & 100\% & \hspace{0.5pt}1\% & \hspace{0.5pt}5\% & 10\% & 20\% & 40\% & 50\% & 100\% \\
    \toprr

    \multirow{1}{*}{{Range}} & U & LaserMix~\cite{kong2022lasermix} & (2022) & 43.4 & -- & 58.8  & 59.4  &  \hspace{4.5pt}-- & 61.4 &\hspace{4.5pt}-- & 38.3 & \hspace{4.5pt}-- & 54.4  & 55.6 & \hspace{4.5pt}-- & 58.7 &\hspace{4.5pt}-- \\
    \midrule

    \multirow{5}{*}{{Voxel}} & U & Cylinder3D~\cite{zhu2021cylindrical} & (CVPR'21) & \hspace{4.5pt}-- & 45.4 & 56.1 & 57.8 & 58.7 & \hspace{4.5pt}-- & 67.8 & \hspace{4.5pt}-- & 39.2 & 48.0 & 52.1 & 53.8 & \hspace{4.5pt}-- &56.3 \\

    & U & LaserMix~\cite{kong2022lasermix} & (2022) & 50.6 & \hspace{4.5pt}-- & 60.0 & \underline{61.9} & \hspace{4.5pt}-- & 62.3 &\hspace{4.5pt}-- & 44.2 & \hspace{4.5pt}-- & 53.7 & 55.1 & \hspace{4.5pt}-- & 56.8 &\hspace{4.5pt}-- \\

    & P & Jiang \etal~\cite{jiang2021guided} & (ICCV'21)  & \hspace{4.5pt}-- & 41.8 & 49.9 & 58.8 & 59.9 & \hspace{4.5pt}-- & 65.8  &\hspace{4.5pt}--&\hspace{4.5pt}--&\hspace{4.5pt}--&\hspace{4.5pt}--&\hspace{4.5pt}-- &\hspace{4.5pt}--&\hspace{4.5pt}-- \\

    & U & Unal \etal~\cite{Unal_2022_CVPR} & (CVPR'22)  & \hspace{4.5pt}-- &49.9$^*$&58.7$^*$&59.1$^*$&60.9& \hspace{4.5pt}-- &\underline{68.2}$^*$& \hspace{4.5pt}-- &46.9$^*$&54.2$^*$&56.5$^*$&58.6$^*$& \hspace{4.5pt}-- &\underline{61.3} \\

    & S & {\ourmodelsdsc} & (ours) 
    & \underline{57.2}  %
    & \underline{57.6}  %
    & \underline{61.0}  %
    & 61.7  %
    & \underline{62.1}  %
    & \underline{62.7}  %
    & 67.5  %
    & \underline{55.8}  %
    & \underline{56.1}  %
    & \underline{56.9}  %
    & \underline{57.2}  %
    & \underline{58.9}  %
    & \underline{59.3}  %
    & 60.7  %
    \\

    & S & {\ourmodel} & (ours) 
    & \textbf{58.4}  %
    & \textbf{59.5}  %
    & \textbf{62.2}  %
    & \textbf{63.1}  %
    & \textbf{63.3}  %
    & \textbf{63.6}  %
    & \textbf{69.5}  %
    & \textbf{57.0}  %
    & \textbf{58.1}  %
    & \textbf{61.0}  %
    & \textbf{61.2}  %
    & \textbf{62.0}  %
    & \textbf{62.1}  %
    & \textbf{62.4}  %
    \\
    \bottomr
    \end{tabular}}
    \label{tab:benchmark_2}
\end{table*}

\subsection{Experimental Setup}

\bdtitle{SemanticKITTI}~\cite{behley2019semantickittia} is a large-scale 3D point cloud dataset for semantic scene understanding with 20 semantic classes consisting of 22 sequences - [\texttt{00} to \texttt{10} as \textit{training}-split (of which \texttt{08} as \textit{validation}-split) + \texttt{11} to \texttt{21} as \textit{test}-split].

\bdtitle{ScribbleKITTI}~\cite{Unal_2022_CVPR} is the first scribble (\ie sparsely) annotated dataset for LiDAR semantic segmentation providing sparse annotations for the \textit{training} split of SemanticKITTI for 19 classes, with only 8.06\% of points from the full SemanticKITTI dataset annotated.

\bdtitle{Evaluation Protocol:} Following previous work~\cite{zhu2021cylindrical,jiang2021guided,Unal_2022_CVPR,kong2022lasermix}, we report performance on the SemanticKITTI and ScribbleKITTI {\trainset} for intermediate training steps, as this metric provides an indication of the pseudo-labeling quality, and on the {\validset} to assess the performance benefits of each individual component. Performance is reported using the mean Intersection over Union (mIoU, as \%) metric. For semi-supervised training, we report over both the benchmarks using the SemanticKITTI and ScribbleKITTI {\validset} under 5\%, 10\%, 20\%, and 40\% partitioning. We further report the relative performance of semi-supervised or scribble-supervised for ScribbleKITTI (SS) training to the fully supervised upper-bound (FS) in percentages (SS/FS) to further analyze semi-supervised performance and report the results for the fully-supervised training on both {\validset}s for reference. The trainable parameter count and number of multiply-adds (multi-adds) are additionally provided as a metric of computational cost.

\bdtitle{Implementation Details:} Training is performed using 4$\times$ NVIDIA A100 80GB GPU without pre-trained weights with a DDP shared training strategy~\cite{FairScale2021} to maintain GPU scaling efficiency, whilst reducing memory overhead significantly. Specific hyper-parameters are set as follows - Mean Teacher: $\kappa=0.99$; unreliable pseudo-labeling: $\lambda_C=0.3$, $\tau=0.5$; {\samplshort}: $\beta = \{7.45$, $5.72$, $4.00$, $2.28$, $0$\} for sampling \{5\%, 10\%, 20\%, 40\%, 100\%\} labeled training frames, assuming the remainder as unlabeled; Reflec-TTA: $N_b=10$, $s=3$ various Reflec-TTA bin sizes, following~\cite{Unal_2022_CVPR}, we set each bin $b_i = \left(\rho, \phi\right) \in \{ (20, 40), (40, 80), (80, 120)$\}.


\begin{figure}[thp]
    \centering
    \includegraphics[width=0.478\textwidth]{figures/visual_results_big.pdf}
    \caption{Comparing the 10\% sampling split of SemanticKITTI (SeK, first row) and ScribbleKITTI (ScK, second row) {\validset} with ground-truth (left), our approach (middle) and Unal \etal~\cite{Unal_2022_CVPR} (right) with areas of improvement highlighted.}
    \label{fig:visual_results}
    \vspace{-0.4cm}
\end{figure}
\begin{table}[htp]
    \setlength{\abovecaptionskip}{0.05cm}
    \centering
    \caption{Component-wise ablation of {\ourmodel} (mIoU as \%, and \#parameters in millions, M) on SemanticKITTI~\cite{behley2019semantickittia} \textit{training} and \textit{validation} sets where UP, RF, RT, ST, SD denote Unreliable Pseudo-labeling, Reflectivity Feature, Reflec-TTA, {\samplshort}, and SDSC module respectively.}
    \resizebox{0.48\textwidth}{!}{
        \begin{tabular}{@{}C{0.25cm}C{0.25cm}C{0.25cm}C{0.25cm}C{0.25cm}|C{0.45cm}C{0.45cm}C{0.45cm}C{0.45cm}|C{0.45cm}C{0.45cm}C{0.45cm}C{0.45cm}|c@{}}
            \topr 
            \multirow{2}{*}{UP} & \multirow{2}{*}{RF} & \multirow{2}{*}{RT} & \multirow{2}{*}{ST} & \multirow{2}{*}{SD} & \multicolumn{4}{c|}{Training mIoU (\%)} & \multicolumn{4}{c|}{Validation mIoU (\%)} & {\#Params} \\
            &   &   &   &   & 5\% & 10\% & 20\% & 40\% & 5\% & 10\% & 20\% & 40\% & (M) \\
    \midr

    &  &  &  &  
    & 82.8    %(RF+RT+MR+SD) 05%: libuntu/uem3vnpf
    & 87.5    %(RF+RT+MR+SD) 10%: libuntu/c9rviemw
    & 87.8    %(RF+RT+MR+SD) 20%: libuntu/4kvuxe3j
    & 88.2    %(RF+RT+MR+SD) 40%: libuntu/bua3pvi9
    & 54.8    %(RF+RT+MR+SD) 05%: libuntu/8vew0bhr
    & 58.1    %(RF+RT+MR+SD) 10%: libuntu/qvkl3pbe
    & 59.3    %(RF+RT+MR+SD) 20%: libuntu/7bfkjp9n
    & 60.8    %(RF+RT+MR+SD) 40%: libuntu/vo5ie7yc
    & 49.6   \\

    \checkmark &  &  &  &  & -- & -- & -- & -- 
    & 55.9    %(RF+RT+MR+SD) 05%: libuntu/8bu0rmge
    & 58.8    %(RF+RT+MR+SD) 10%: libuntu/vec3kbl6
    & 59.9    %(RF+RT+MR+SD) 20%: libuntu/bys2mw7r
    & 61.2    %(RF+RT+MR+SD) 40%: libuntu/1cn3bezo
    & 49.6   \\
    
    \midr
    \checkmark & \checkmark &  &  & 
    & 83.6    %(UP+RT+MR+SD) 5%:  libuntu/c8vi2xpw
    & 88.3    %(UP+RT+MR+SD) 10%: libuntu/2bmcu0zr
    & 88.7    %(UP+RT+MR+SD) 20%: libuntu/nvuel2d9
    & 89.1    %(UP+RT+MR+SD) 40%: libuntu/x3vwpdbo 
    & 56.8    %(UP+RF+MR+SD) 05%: libuntu/vp8dx9gr
    & 59.6    %(UP+RF+MR+SD) 10%: libuntu/2cirgn4x
    & 60.5    %(UP+RF+MR+SD) 20%: libuntu/tw4u5vns
    & 61.4    %(UP+RF+MR+SD) 40%: libuntu/ufs4vdwh
    & 49.6   \\

    \checkmark &  & \checkmark &  &  & -- & -- & -- & -- 
    & 57.5    %(RF+RT+MR+SD) 05%: libuntu/8bu0rmge
    & 59.8    %(RF+RT+MR+SD) 10%: libuntu/vec3kbl6
    & 61.2    %(RF+RT+MR+SD) 20%: libuntu/bys2mw7r
    & 62.6    %(RF+RT+MR+SD) 40%: libuntu/1cn3bezo
    & 49.6   \\
    
    \checkmark & \checkmark & \checkmark &  &  & -- & -- & -- & -- 
    & 58.7    %(UP+RT+MR+SD) 5%:  libuntu/9vi3xevm
    & 61.3    %(UP+RT+MR+SD) 10%: libuntu/cusm6bp2
    & 62.4    %(UP+RT+MR+SD) 20%: libuntu/4viep5mz
    & 62.8    %(UP+RT+MR+SD) 40%: libuntu/cux0mvb3
    & 49.6   \\


    \midrule
    \checkmark & \checkmark & \checkmark & \checkmark & 
    & \textbf{85.2}    %(UP+RF+RT+MR) 5%:  qfmk61/ci9gv5ln
    & \textbf{89.1}    %(UP+RF+RT+MR) 10%: mznv82/38bd52r6
    & \textbf{89.5}    %(UP+RF+RT+MR) 20%: mznv82/3nd08qs9
    & \textbf{89.7}    %(UP+RF+RT+MR) 40%: mznv82/1fzqfi4s
    & \textbf{59.5}    %(UP+RF+RT+MR) 5%:  qfmk61/4nx8bmsb
    & \textbf{62.2}    %(UP+RF+RT+MR) 10%: mznv82/2l6mcqv8
    & \textbf{63.1}    %(UP+RF+RT+MR) 20%: mznv82/3ez9oave
    & \textbf{63.3}    %(UP+RF+RT+MR) 40%: mznv82/spavmo89
    & 49.6   \\

    \checkmark & \checkmark & \checkmark & \checkmark & \checkmark
    & 83.8    % all components 05%: qfmk61/cie3mvi1
    & 88.6    % all components 10%: qfmk61/vuemxq4n
    & 89.0    % all components 20%: qfmk61/4vjke8xr
    & 89.2    % all components 40%: qfmk61/vb8emci4
    & 57.6    % all components 05%: qfmk61/j3bve6cs
    & 61.0    % all components 10%: qfmk61/x82cfs0t
    & 61.7    % all components 20%: qfmk61/6sdxe9c5
    & 62.1    % all components 40%: qfmk61/c8xs6q0s
    & \textbf{21.5}   \\
    \bottomr
    \end{tabular}}
    \label{tab:ablation}
    \vspace{-10pt}
\end{table}

%%%%%%%%%%%%%%%%%%%%%%%%%%%%%%%%%%%%%%%%%%%%%%%%%%%%%%%%%%%
\vspace{-0.2cm}
\subsection{Experimental Results}
\vspace{-0.2cm}
\noindent
In~\cref{tab:benchmark_2}, we present the performance of our \textit{Less is More} 3D ({\ourmodel}) point cloud semantic segmentation approach both with ({\ourmodelsdsc}) and without ({\ourmodel}) SDSC in a side-by-side comparison with leading contemporary state-of-the-art approaches on the SemanticKITTI and ScribbleKITTI benchmark {\validset}s to illustrate our approach offers superior or comparable (within 1\% mIoU) performance across all sampling ratios. Furthermore, we present supporting qualitative results in~\cref{fig:visual_results}.

On SemanticKITTI, with a lack of available supervision, {\ourmodel} shows a relative performance (SS/FS) from $85.6\%$ ($5\%$-fully-supervised) to $91.1\%$ ($40\%$-fully-supervised), and {\ourmodelsdsc} from $85.3\%$ to $92.0\%$, compared to their respective fully supervised upper-bound. {\ourmodel}/{\ourmodelsdsc} performance is also less sensitive to reduced labeled data sampling compared with other methods.

Our model significantly outperforms on small ratio sampling splits, \eg, $5\%$ and $10\%$. {\ourmodel} shows up to $19.8\%$ and $18.9\%$ mIoU improvements whilst, with a smaller model size {\ourmodelsdsc} again shows significant mIoU improvements by up to $16.4\%$ and $15.5\%$ when compared with other range and voxel-based methods respectively.


%%%%%%%%%%%%%%%%%%%%%%%%%%%%%%%%%%%%%%%%%%%%%%%%%%%%%%%%%%%
\vspace{-0.15cm}
\subsection{Ablation Studies}
\vspace{-0.15cm}
\bdtitle{Effectiveness of Components.} In~\cref{tab:ablation} we ablate each component of {\ourmodel} step by step and report the performance on the SemanticKITTI {\trainset} at the end of training as an overall indicator of pseudo-labeling quality in addition to the corresponding {\validset}.

As shown in~\cref{tab:ablation}, adding unreliable pseudo-labeling (UP) in the distillation stage, we can increase the $valid$ mIoU by $+0.7\%$ on average in {\validset}. Appending reflectivity features (RF) in the training stage, we further improve the mIoU on the {\trainset} by $+0.7\%$ on average. Due to the improvements in training, the model generates a higher quality of pseudo-labels, which results to a $+0.5\%$ increase in mIoU in the {\validset}. If we disable reflectivity features in the training stage, applying Reflec-TTA in the distillation stage alone, we then get an average improvement of $+1.3\%$ compared with pseudo-labeling only. On the whole, enabling all reflectivity-based components (RF+RT) shows great improvements of up to $+2.8\%$ in $validation$ mIoU.
%%%%%%%%%%%%%%%%%%%% Train on 5% sampling %%%%%%%%%%%%%%%%%%%%
\begin{table}[htp]
    \scriptsize
    \vspace{-3pt}
    \setlength{\abovecaptionskip}{0.05cm}
    \centering
    \caption{The computation cost and mIoU (in percentage) under 5\%-labeled training results on SemanticKITTI (SeK) and ScribbleKITTI (ScK) {\validset}.}
    %\resizebox{0.48\textwidth}{!}
{\begin{tabular}{lcrcc}
\toprule
Method & \# Parameters & \hspace{-6pt} \# Mult-Adds & SeK~\cite{behley2019semantickittia} & ScK~\cite{Unal_2022_CVPR} \\
\midrule
Cylider3D~\cite{zhu2021cylindrical} & 56.3 & 476.9M & 45.4 & 39.2 \\
Unal~\etal~\cite{Unal_2022_CVPR} & 49.6 & 420.2M & 49.9 & 46.9 \\
2DPASS~\cite{yan20222dpass} & 26.5 & \underline{217.4M} & 51.7 & 45.1 \\
MinkowskiNet~\cite{choy20194d} & 21.7 & 114.0G & 42.4 & 35.8 \\
SPVNAS~\cite{tang2020searching} & \textbf{12.5} & 73.8G & 45.1 & 38.9 \\
{\ourmodelsdsc} (ours) & \underline{21.5} & \textbf{182.0M} & \underline{57.6} & \underline{54.7} \\
{\ourmodel} (ours) & 49.6 & 420.2M & \textbf{59.5} & \textbf{58.1} \\
\bottomrule
\end{tabular}}
\label{tab:computation_cost}
\vspace{-10pt}
\end{table}

\begin{table}[htp]
    \scriptsize
    \vspace{-3pt}
    \centering
    \setlength{\abovecaptionskip}{0.05cm}
    \caption{Effects of {\samplshort} sampling on SemanticKITTI and ScribbleKITTI {\validset} (mIoU as \%).}
{\begin{tabular}{c|cccc|cccc}
\toprule
\multirow{2}{*}{Sampling} & \multicolumn{4}{c|}{SemanticKITTI~\cite{behley2019semantickittia}} & \multicolumn{4}{c}{ScribbleKITTI~\cite{Unal_2022_CVPR}} \\
& 5\% & 10\% & 20\% & 40\% & 5\% & 10\% & 20\% & 40\% \\
\midrule 

Random 
& 58.5   %
& 61.6   %
& 62.6   %
& 62.7   %
& 57.1   %
& 60.3   %
& \underline{60.5}   %
& 60.9\\ %

Uniform  
& 58.7   %
& 61.3   %
& 62.4   %
& 62.8   %
& 56.9   %
& 60.6   %
& 60.3   %
& 61.0 \\% ScribbleKITTI 40: qfmk61/7ti6gdfe

{\samplshort}-R
& \underline{59.1}                %
& \textbf{62.4}       %
& \underline{62.9}                %
& \textbf{63.4}       %
& \underline{58.0}                %
& \underline{60.7}                %
& \textbf{61.2}       %
& \underline{61.8} \\             %

{\samplshort}  & \textbf{59.5}  & \underline{62.2} & \textbf{63.1} & \underline{63.3} & \textbf{58.1} & \textbf{61.0}  & \textbf{61.2} & \textbf{62.0} \\

\bottomrule
\end{tabular}}
\vspace{-10pt}
\label{tab:sampl}
\end{table}
 
\begin{table}[H]
    \scriptsize
    \vspace{-3pt}
    \setlength{\abovecaptionskip}{0.05cm}
    \centering
\caption{Effects of differing reliability using pseudo voxels on SemanticKITTI {\validset}, measured by the entropy of voxel-wise prediction. %
\textit{Unreliable} and \textit{Reliable}: selecting negative candidates %
with top $20 \%$ highest entropy scores and bottom $20 \%$ counterpart respectively. \textit{Random}: sampling randomly regardless of entropy.
}
{
\begin{tabular}{c|cc|cc|cc}
\toprule 
 \multirow{2}{*}{Ratio} & \multicolumn{2}{c|}{Unreliable} & \multicolumn{2}{c|}{Reliable} & \multicolumn{2}{c}{Random} \\
 & mIoU & SS/FF & mIoU & SS/FF & mIoU & SS/FF \\
\midrule 
 5\% & \textbf{59.5} & \textbf{85.6} & 57.2 & 82.3 & 56.4 & 81.2 \\
10\% & \textbf{62.2} & \textbf{89.5} & 60.8 & 87.5 & 59.7 & 85.9 \\
20\% & \textbf{63.1} & \textbf{90.8} & 61.4 & 88.3 & 60.5 & 87.1 \\
40\% & \textbf{63.3} & \textbf{91.1} & 62.8 & 90.4 & 61.3 & 88.2 \\
 \bottomrule
\end{tabular}
\label{tab:pseudo}
}
\vspace{-10pt}
\end{table}
\begin{table}[!h]
    \scriptsize
    \vspace{-3pt}
    \centering
    \setlength{\abovecaptionskip}{0.05cm}
    \caption{Reflectivity (Reflec-TTA) vs. Intensity (intensity-based TTA) on \textls[-45]{SemanticKITTI and ScribbleKITTI} {\validset} (mIoU, \%).}
{\begin{tabular}{c|cccc|cccc}
\toprule
\multirow{2}{*}{TTA} & \multicolumn{4}{c|}{SemanticKITTI~\cite{behley2019semantickittia}} & \multicolumn{4}{c}{ScribbleKITTI~\cite{Unal_2022_CVPR}} \\
& 5\% & 10\% & 20\% & 40\% & 5\% & 10\% & 20\% & 40\% \\
\midrule 
Intensity 
& 56.2   %
& 59.1   %
& 59.8   %
& 60.9   %
& 55.7   %
& 57.5   %
& 57.9   %
& 59.2 \\% ScribbleKITTI 40: libuntu/qwaf4b6v       
Reflectivity  & \textbf{59.5}  & \textbf{62.2} & \textbf{63.1} & \textbf{63.3} & \textbf{58.1} & \textbf{61.0}  & \textbf{61.2} & \textbf{62.0} \\
\bottomrule
\end{tabular}}
\label{tab:ref_vs_inten}
\vspace{-10pt}
\end{table}


Substituting the uniform sampling with our {\samplshort} strategy, we observe further average improvements of $+1.0\%$ and $+0.8\%$ on $training$ and $validation$ respectively (\cref{tab:ablation}).

Our SDSC module reduces the trainable parameters of our model by $57\%$, with a performance cost of $-0.7\%$ and $-1.4\%$ mIoU on $training$ and $validation$ respectively (\cref{tab:ablation}). Finally, we provide two models, one without SDSC ({\ourmodel}) and one with ({\ourmodelsdsc}), corresponding to the bottom two rows of~\cref{tab:ablation}.

%%%%%%%%%%%%%%%%%%%%%%%%%%%%%%%%%%%%%%%%%%%%%%%%%%%%%%%%%%%
\bdtitle{Effectiveness of SDSC module.} In ~\cref{tab:computation_cost}, we compare our {\ourmodel} and {\ourmodelsdsc} with recent state-of-the-art methods under 5\%-labeled semi-supervised training on the SemanticKITTI and ScribbleKITTI {\validset}s. {\ourmodelsdsc} outperforms the voxel-based methods~\cite{zhu2021cylindrical,Unal_2022_CVPR} with at least a \textbf{2.3}$\times$ reduction in model size. Similarly, with comparable model size~\cite{choy20194d,tang2020searching,yan20222dpass}, {\ourmodelsdsc} has higher mIoU in both datasets and up to \textbf{641}$\times$ fewer multiply-add operations.



%%%%%%%%%%%%%%%%%%%%%%%%%%%%%%%%%%%%%%%%%%%%%%%%%%%%%%%%%%%
\bdtitle{Effectiveness of {\samplshort} strategy.} In~\cref{tab:sampl}, we illustrate the effectiveness of our {\samplshort} strategy by comparing  {\ourmodel} with two widely-used strategies in semi-supervised training, \ie, random sampling and uniform sampling on SemanticKITTI~\cite{behley2019semantickittia} and ScribbleKITTI~\cite{Unal_2022_CVPR} {\validset}. Whilst uniform and random sampling have comparable results on both {\validset}s, simply applying our {\samplshort} strategy improves the baseline by $+0.90\%$, $+0.75\%$, $+0.60\%$ and $+0.55\%$ on SemanticKITTI under $5\%$, $10\%$, $20\%$ and $40\%$ sampling protocol respectively.
Furthermore, using corresponding range images of point cloud, rather than RGB images to compute the spatio-temporal redundancy within {\samplshort} (see {\samplshort}-R in~\cref{tab:sampl}), has no significant difference on the performance.



%%%%%%%%%%%%%%%%%%%%%%%%%%%%%%%%%%%%%%%%%%%%%%%%%%%%%%%%%%%
% \vspace{-5pt}
\bdtitle{Effectiveness of Unreliable Pseudo-Labeling.} In~\cref{tab:pseudo}, we evaluate selecting negative candidates with different reliability to illustrate the improvements of using unreliable pseudo-labels in semi-supervised semantic segmentation. The \textit{“Unreliable”} selecting of negative candidates outperforms other alternative methodologies, showing the positive performance impact of unreliable pseudo-labels.



%%%%%%%%%%%%%%%%%%%%%%%%%%%%%%%%%%%%%%%%%%%%%%%%%%%%%%%%%%%
\bdtitle{Effectiveness of Reflec-TTA.} In~\cref{tab:ablation}, we compare {\ourmodel} performance with and without Reflec-TTA and further experiment on the SemanticKITTI and ScribbleKITTI {\validset} in~\cref{tab:ref_vs_inten}. This demonstrates that the LiDAR point-wise intensity feature $I^\circledast$, in place of the distance-normalized reflectivity feature $R^\circledast$, offers inferior on-task performance.