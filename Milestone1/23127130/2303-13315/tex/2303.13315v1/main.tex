%%%%%%%%%%%%%%%%%%%%%%%%%%%%%%%%%%%%%%%%%%%%%%%%%%%%%%%%%%%%%%%%%%%%%%%%%%%%%%%%
%2345678901234567890123456789012345678901234567890123456789012345678901234567890
%        1         2         3         4         5         6         7         8

%\documentclass[letterpaper, 10 pt, conference,onecolumn]{IEEEconf}  % Comment this line out if you need a4paper
\documentclass[letterpaper, 10 pt, conference]{ieeeconf}
%\documentclass[letterpaper, 10 pt, times, conference]{ieeeconf}
%\documentclass[journal,twoside,web]{ieeecolor}
%\usepackage{lcsys}
%\documentclass[a4paper, 10pt, conference]{ieeeconf}      % Use this line for a4 paper

\IEEEoverridecommandlockouts                              % This command is only needed if 
                                                          % you want to use the \thanks command

\usepackage{dsfont}
%For indicator




           
\usepackage{amssymb}
\usepackage{eqnarray} %temp

\usepackage{amsmath}
\usepackage{mathalfa} % add for greek aclligraphic
%\usepackage[dvips]{epsfig}    % or this line, depending on which you prefer.
%\usepackage{hyperref} 
%\usepackage{amsthm}			% for theorems
\usepackage{pdfpages}		% for including pdf 
\usepackage{dsfont}   		%for using  \mathds{numerals}
%
\usepackage{algorithm}
\usepackage{algpseudocode}

\usepackage{bbm}
                                               
\usepackage{graphicx}
\usepackage[small]{caption}
\usepackage[utf8]{inputenc}
\usepackage{psfrag}
%\usepackage{amsthm}
\usepackage{amsmath}
\usepackage{amssymb}
\usepackage{mathtools}


\usepackage{comment}


\usepackage{psfrag}
%\usepackage{auto-pst-pdf}
\usepackage{url}

%\usepackage{bbold}                                     % Questo pacchetto serve per \mathbb{1}
\usepackage{cite}

\newcommand{\sD}{\mathcal{D}}
\newcommand{\sS}{\mathcal{S}}
\newcommand{\sA}{\mathcal{A}}
\newcommand{\sL}{\mathcal{L}}
\newcommand{\sM}{\mathcal{M}}
\newcommand{\sP}{\mathbb{P}}
\newcommand{\sN}{\mathcal{N}}
\newcommand{\sV}{\mathcal{V}}
\newcommand{\sE}{\mathcal{E}}
\newcommand{\sF}{\mathcal{F}}
\newcommand{\sQ}{\mathcal{Q}}
\newcommand{\sT}{\mathcal{T}}
%\newcommand{\Nh}{N_h}
\newcommand{\Nh}{n}

\newcommand{\R}{\mathbb{R}}
\newcommand{\E}{\mathbb{E}}
\newcommand{\Z}{\mathbb{Z}}

\newcommand{\bv}[1]{\mathbf{#1}}
\newcommand{\policy}[2]{\pi^{#2}(#1)}

\newcommand{\abs}[1]{\left\vert #1 \right\vert}

\newcommand {\GR}[1]{\mbox{ }\medskip \\ {{\bf GR:} {\color{\colGRU} \em #1}} \mbox{ }\medskip \\}

% new commands
\newcommand{\KL}{\text{KL}}
\newcommand{\Ipdf}{g}
\newcommand{\realone}{\mathbb{R}}
\newcommand{\realn}[1]{\mathbb{R}^{#1}}
\newcommand{\upto}[1]{_{#1}}
\newcommand{\DKL}{\mathcal{D}_{\KL}}

% constraints name
\newcommand{\constrEqSymb}{\mathbf{E}}
\newcommand{\constrEq}[2]{\constrEqSymb_{#1,#2}}
\newcommand{\constrVecSymb}{\mathbf{C}}
\newcommand{\constrVec}[2]{\constrVecSymb_{#1}^{#2}}

%for review 
%----------------------------
% colors def
\newcommand{\colDGA}{red}
\newcommand{\colGRU}{blue}
\newcommand{\colALL}{violet}

%input % 
\newcommand{\revRED}[1]       {{ \color{\colDGA} #1 } }   	% #2 updated version	
\newcommand{\revDGA}[1]       {{ \color{\colDGA} $\triangleright$ #1 $\triangleleft$} }   	% #2 updated version	
% \newcommand{\revDGA}[1]       {{ \color{\colDGA}  - #1 - } }   	% #1 updated version	
\newcommand{\revGRU}[1]       {{ \color{\colGRU}  $\triangleright$ #1 $\triangleleft$ } }   		
\newcommand{\revALL}[1]        {{ \color{\colALL}  $\triangleright$ #1 $\triangleleft$ } }   

\newcommand{\eqlmarg}       {\hspace{3mm}}   % alternatively  "\;" "\;\;" "\quad"


\newtheorem{Example}{Example}
\newtheorem{Remark}{Remark}
\newtheorem{Theorem}{Theorem}
\newtheorem{Definition}{Definition}
%\newtheorem{Lemma}{Lemma}
\newtheorem{Corollary}{Corollary}
\newtheorem{Assumption}{Assumption}
\newtheorem{Proposition}{Proposition}
\newtheorem{Problem}{Problem}

%\usepackage{draftwatermark}
%\SetWatermarkText{Confidential}
%\SetWatermarkScale{1}

\overrideIEEEmargins                                      % Needed to meet printer requirements.

%In case you encounter the following error:
%Error 1010 The PDF file may be corrupt (unable to open PDF file) OR
%Error 1000 An error occurred while parsing a contents stream. Unable to analyze the PDF file.
%This is a known problem with pdfLaTeX conversion filter. The file cannot be opened with acrobat reader
%Please use one of the alternatives below to circumvent this error by uncommenting one or the other
%\pdfobjcompresslevel=0
%\pdfminorversion=4

% See the \addtolength command later in the file to balance the column lengths
% on the last page of the document

% The following packages can be found on http:\\www.ctan.org
%\usepackage{graphics} % for pdf, bitmapped graphics files
%\usepackage{epsfig} % for postscript graphics files
%\usepackage{times} % assumes new font selection scheme installed
%\usepackage{amsmath} % assu tmes amsmath package installed
%\usepackage{amssymb}  % assumes amsmath package installed


\title{\LARGE \bf
Optimal Decision-Making for Autonomous Agents via Data Composition \\
%On Action Composition for Autonomous Sequential Decision-Making
}

\author{\'Emiland Garrab\'e, Martina Lamberti and Giovanni Russo$^\ast$% <-this % stops a space
\thanks{$^\ast$ emails: {\tt\small \{egarrabe,giovarusso\}@unisa.it}, {\tt\small martinalamberti3@gmail.com}. E. Garrab\'e and G. Russo with the DIEM at the University of Salerno, 84084, Salerno, Italy.} 
%M. Lamberti with TMC Europe, 20126, Milan, Italy.}%
}

\begin{document}


\maketitle
\thispagestyle{empty}
\pagestyle{empty}

\begin{abstract}
We consider the problem of designing agents able to compute optimal decisions by composing data from multiple sources to tackle tasks involving: (i) tracking a desired behavior while minimizing an agent-specific cost; (ii) satisfying safety constraints. After formulating the control problem, we show that this is convex under a suitable assumption and find the optimal solution. The effectiveness of the results, which are turned in an algorithm, is illustrated on a {\em connected cars} application via in-silico and in-vivo experiments with real vehicles and drivers. All the experiments confirm our theoretical predictions and the deployment of the algorithm on a real vehicle shows its suitability for in-car operation.
\end{abstract}

% This paper is concerned with the design of data-driven agents able to merge data from different sources to form policies. We cast this process as a data-driven, optimal control problem and, through a convexity result, give an explicit strategy to find its optimal solution. This strategy is formulated as an algorithmic procedure. Finally, we use such a procedure to implement an in-car service, which is benchmarked on a routing task.

\section{Introduction}
%A key challenge, transversal to learning and control, is that of devising mechanisms allowing autonomous decision-makers  \cite{GARRABE202281} to emulate our unique ability of adapting and re-using knowledge gathered from a variety of sources, abstracting what we know to make the appropriate decisions when solving new tasks \cite{doi:10.1126/science.aab3050,lake_ullman_tenenbaum_gershman_2017}.

We often make decisions by composing knowledge gathered from {\em others}  \cite{doi:10.1126/science.aab3050}. In this context, a challenge transversal to control and learning is to devise mechanisms allowing autonomous decision-makers to emulate these abilities. Systems based on sharing data \cite{sharing_eco} are  examples where agents need to make decisions based on some form of information crowdsourcing \cite{russo2020crowdsourcing}. Similar mechanisms can also be useful for the data-driven control paradigm when e.g., one needs to re-use policies synthesized on plants for which data are available in order to solve a new control task on a new plant, for which data are instead scarcely available \cite{russo2020crowdsourcing,designof,https://doi.org/10.48550/arxiv.2211.05536}.

Motivated by this, we consider the problem of designing decision-making mechanisms that enable autonomous agents to compute optimal decisions by composing information from third parties to solve tasks that involve: (i) tracking a desired behavior while minimizing an agent-specific cost; (ii) satisfying  safety constraints. Our results enable computation of the optimal behavior and are turned into an algorithm. This is  experimentally validated on a {\em connected car} application. 


\subsubsection*{Related works} we briefly survey a number of works related to the results and methodological framework of this paper. The design of context-aware switches between multiple datasets for autonomous agents has been recently considered in  \cite{russo2020crowdsourcing,designof}, where the design problem,  formalized as a data-driven control (DDC) problem, did not take into account safety requirements. Results in DDC include \cite{8795639,behav2,8933093}, which take a behavioral systems \cite{behav_review} perspective, \cite{9763859}, which finds data-driven formulas towards a model-free theory of geometric control. We also recall e.g., \cite{8039204,DBLP:conf/l4dc/WabersichZ20,9109670} for results inspired from MPC, \cite{https://doi.org/10.48550/arxiv.2211.05536} that considers data-driven control policies transfer and \cite{8703172} that tackles he problem of computing data-driven minimum-energy control for linear systems. In our control problem (see Section \ref{sec:control_problem}) we formalize the tracking of a given behavior via Kullback-Leibler (KL) divergence minimization and we refer to e.g., \cite{GARRABE202281,9029512} for examples across learning and control that involve minimizing this functional. Further, the study of mechanisms enabling agents to re-use data, also arises in the design of prediction algorithms from experts \cite{books/daglib/0016248} and of learning algorithms from multiple simulators \cite{7106543}. In a yet broader context, studies in neuroscience \cite{Mountcastle97} hint that our neocortex might implement a mechanism composing the output of the cortical columns and this might be the basis of our ability to re-use knowledge.



%In the context of an agent that directly specifies its state transition probabilities, we also recall \cite{KLC}, where a value function based on a Kullback-Leibler\cite{KL_51} component and a state-based reward signal is directly minimized. Such value functions can be calculated by casting the problem as an eigenvector problem \cite{todorov_linearly_mdps}, or through graphical inference \cite{KL_as_inference}. 

%Situations where multiple sources provide data arise in e.g. the Sharing Economy \cite{sharing_eco}, Multi-Fidelity Reinforcement Learning \cite{multifidelity}, Multi-Agent learning \cite{multi_agent_rl} or data sharing platforms for robots \cite{roboearth}, and crowdsourcing mechanisms have been linked to human cognitive function\cite{1000brains}.\\
\emph{Contributions:} we consider the problem of designing agents that dynamically combine data from heterogeneous sources to fulfill tasks that involve tracking a target behavior while optimizing a cost and satisfying safety requirements expressed as box constraints. By leveraging a probabilistic framework, we formulate the problem as a data-driven optimal control problem and, for this problem, we: (i) prove convexity under a suitable condition; (ii) find the optimal solution; (iii) turn our results into an algorithm, using it to  design an intelligent parking system for connected vehicles. Validations are performed both {\em in-silico} and {\em in-vivo}, with real cars. In-vivo validations were performed via an hardware-in-the-loop platform allowing to involve real cars and drivers in the experiments. Using the platform, we deploy our algorithm on a real vehicle showing its suitability for in-car operation. All experiments confirm the effectiveness of our approach ({documented code and data to replicate our simulations are given at \url{https://tinyurl.com/3ep4pknh})}.

While the results of this paper are inspired by the ones in \cite{russo2020crowdsourcing,designof}, our paper extends these in several ways. First, the results in \cite{russo2020crowdsourcing,designof} cannot consider box constraints and hence cannot tackle the control problem of this paper. Second, we find that, even when there are no box constraints, the solutions from \cite{russo2020crowdsourcing,designof} cannot get a better cost than the one obtained with the results of this paper. Third, the algorithm obtained from the results in this paper is deployed, and validated, on a real car and this was not done in \cite{russo2020crowdsourcing,designof}.

\subsubsection*{Notation}
sets are in {\em calligraphic} and vectors in {\bf bold}. Given the measurable space $(\mathcal{X},\mathcal{F}_x)$, with $\mathcal{X}\subseteq\R^{d}$ ($\mathcal{X}\subseteq\Z^{d}$) and $\mathcal{F}_x$ being a $\sigma$-algebra on $\mathcal{X}$, 
a random variable on $(\mathcal{X},\mathcal{F}_x)$ is denoted by $\mathbf{X}$ and its realization by $\mathbf{x}$. The \textit{probability density (resp. {mass})  function} or \textit{pdf} (\textit{pmf}) of a continuous (discrete) $\mathbf{X}$ is denoted by $p(\mathbf{x})$. The convex subset of such probability functions (pfs) is $\sD$.  The  expectation of a function $\mathbf{h}(\cdot)$ of the continuous  variable $\mathbf{X}$ is  $\E_{{p}}[\mathbf{h}(\mathbf{X})]:=\int\mathbf{h}(\mathbf{x})p(\mathbf{x})d\mathbf{x}$, where the integral (in the sense of Lebesgue) is over the support of $p(\mathbf{x})$. The joint pf of $\bv{X}_1$, $\bv{X}_2$ is $p(\mathbf{x}_1,\mathbf{x}_2)$ and the conditional pf of $\mathbf{X}_1$ given $\mathbf{X}_2$ is $p\left( \mathbf{x}_1\mid \mathbf{x}_2 \right)$. Countable sets are denoted by $\lbrace w_k \rbrace_{k_1:k_n}$, where $w_k$ is the generic element of the set and $k_1:k_n$ is the closed set of consecutive integers between $k_1$ and $k_n$. The KL divergence between $p(\mathbf{x})$ and $q(\mathbf{x})$, where $p$ is absolutely continuous w.r.t. $q$, is  $\mathcal{D}_{\KL}\left(p \mid\mid q \right):= \int p \; \ln\left( {p}/{q}\right)\,d\mathbf{x}$: it is non-negative and $0$ if and only if $p(\mathbf{x}) = q(\mathbf{x})$. In the  expressions for the expectation and KL divergence, the integral is replaced by the sum if the variables are discrete. Finally: (i) we make use of $\mathds{1}_\mathcal{A}(\bv{x})$ to denote the indicator function being equal to $1$ if $\bv{x}\in\mathcal{A}\subseteq \mathcal{X}$ and $0$ otherwise; (ii) set exclusion is instead denoted by $\setminus$.
%\GR{Specify in which sense we have the integrals (when you introduce the KL divergence and the expectation) + in the proof of Lemma 1, make sure that the integration can be swapped with differentiation + you need to define the indicator function as used in the problem statement}

\section{The Setup}

The agent seeks to craft its behavior by combining a number of sources to fulfill a task that involves tracking a target/desired behavior while maximizing an agent-specific reward over the time horizon $\sT:= 0:T$, $T>0$. The agent's state at time step $k \in \sT$ is $\bv{x}_k \in \mathcal{X}$ and the target behavior that the agent seeks to track is $p_{0:T} := p_0(\bv{x}_0)\prod_{k\in 1:T} p(\bv{x}_k\mid\bv{x}_{k-1})$. As in \cite{russo2020crowdsourcing,designof}, we design the behavior of the agent by designing its joint pf $\pi_{0:T} := \pi(\bv{x}_0,\ldots,\bv{x}_T)$ and we have:
\begin{equation}\label{eqn:agent}
    \pi_{0:T} =\pi_0(\bv{x}_0)\prod_{k\in 1:T} \pi(\bv{x}_k\mid\bv{x}_{k-1}).
\end{equation}
That is, the behavior of the agent can be designed by shaping the pfs $\pi(\bv{x}_k\mid\bv{x}_{k-1})$, which are the transition probabilities for the agent. To do so, the agent has access to $S$ sources and we denote by $\pi^{(i)}(\bv{x}_k\mid\bv{x}_{k-1})$ the behavior made available by source $i$, $i\in\mathcal{S}:= 1:S$, at time step $k-1$. We also let $r_k(\bv{x}_k)$ be the agent's reward for being in state $\bv{x}_k$ at time $k$. Finally, throughout the paper we make the following:
%\begin{Assumption}\label{asn:DKL_sources}
%\forall k, \forall i \in \mathcal{S}, \DKL\left(\pi^{(i)}_{1:N\textcolor{black}{|0}}||p_{1:N\textcolor{black}{|0}}\right) < +\infty$
%\end{Assumption}
\begin{Assumption}\label{asn:int_def} $\forall i \in \mathcal{S}$,  %the pf $\policy{\bv{x}_k|\bv{x}_{k-1}}{(i)}$ is Lebesgue integrable in $\mathcal{X}$ and 
there exist some constants, say $m$ and $M$, with $0<m\le M < +\infty$, such that $m \le \policy{\bv{x}_k\mid\bv{x}_{k-1}}{(i)} \le M$, $\forall \bv{x}_k, \bv{x}_{k-1}\in\mathcal{X}$.
\end{Assumption}
%\begin{Assumption}\label{asn:sources}
%the pfs for the sources are such that $\policy{\bv{x}_k|\bv{x}_{k-1}}{(i)} > 0$, $\forall \bv{x}_k, \bv{x}_{k-1}$.
%\end{Assumption}
%We conclude this section by making the following:
%\begin{Remark}
%Assumption \ref{asn:int_def} formalizes the fact that $\int_{\bv{x}_k\in\mathcal{X}}\abs{\policy{\bv{x}_k|\bv{x}_{k-1}}{(i)}}d\bv{x}_k < +\infty$; 
%\textcolor{red}{By definition the sources are Lebesgue integrable.} For discrete variables, the integral is replaced with the sum. The upper/lower bounds give a mild condition on the pfs from the sources. 
%Indeed, as we shall see, the agent receives the sources as an input. 
%Hence, upon receiving the $\policy{\bv{x}_k|\bv{x}_{k-1}}{(i)}$'s the agent can: (i) check that the pfs assumption and only use the sources that satisfy this assumption; or (ii) inject some (e.g., uniform) noise in the sources to satisfy the assumption. 
%\end{Remark}
%\begin{Remark}
%Assumption \ref{asn:bounded} is not restrictive in practice. The upper bound formalizes the fact that the pf is bounded Indeed, as we shall see, the agent receives the sources as an input. Hence, upon receiving the $\policy{\bv{x}_k|\bv{x}_{k-1}}{(i)}$'s the agent can: (i) check Assumption \ref{asn:sources} and only use the sources that satisfy this assumption; or (ii) inject some (e.g., uniform) noise in the sources to satisfy the assumption. 
%\end{Remark}

%\begin{Remark}
%The sources do not have access to the agent's target/reward. These can e.g., be pfs  from multiple simulators as in multi-fidelity RL, or from a pool of experts/recommenders.
%\end{Remark}

\section{Formulation of the Control Problem}\label{sec:control_problem}

Let $\alpha_k^{(1)},\ldots,\alpha_k^{(S)}$ be weights and $\boldsymbol{\alpha}_k$ be their stack. Then, the control problem we consider can be formalized as: 
\begin{Problem}\label{prob:problem_merge}
find the sequence ${\left\{\boldsymbol{\alpha}_k^\ast\right\}_{1:T}}$ solving
%\begin{equation}
    \begin{equation*}
    \begin{aligned}
    \underset{\left\{ \boldsymbol{\alpha}_k\right\}_{1:T}}{\text{min}}
    &\DKL\left(\pi_{0:T}\mid\mid p_{0:T}\right) - \sum_{k=1}^T\E_{\pi(\bv{x}_{k-1})}\left[\tilde{r}_k(\bv{X}_{k-1})\right]\\
    s.t. & \ \mathbb{E}_{\pi(\bv{x}_k\mid\bv{x}_{k-1})}\left[\mathds{1}_{\mathcal{X}_k}(\bv{x}_k)\right]\geq 1-\epsilon_k, \ \ \ \forall k,\\
    & \ \pi(\bv{x}_k\mid\bv{x}_{k-1}) = \sum_{i\in\sS}\alpha_k^{(i)}\policy{\bv{x}_k\mid\bv{x}_{k-1}}{(i)}, \ \ \ \forall k, \\
    & \sum_{i\in\sS}\alpha_k^{(i)} = 1, \ \ \alpha_k^{(i)} \in [0, 1],   \ \ \ \forall k. 
    \end{aligned}
    \end{equation*}
%\end{equation}
\end{Problem}
In Problem \ref{prob:problem_merge}, $\tilde{r}_k(\bv{x}_{k-1}) := \E_{\pi(\bv{x}_k\mid\bv{x}_{k-1})}\left[r_k(\bv{X}_k)\right]$ and we note  that $\E_{\pi(\bv{x}_{k-1})}\left[\tilde{r}_k(\bv{X}_{k-1})\right] = \E_{\pi(\bv{x}_{k})}\left[r_k(\bv{X}_k)\right]$ is the expected reward for the agent when the behavior in \eqref{eqn:agent} is followed.  The problem is a finite-horizon optimal control problem with the $\boldsymbol{\alpha}_k^\ast$'s being decision variables. These are generated as feedback based on $\bv{x}_{k-1}$ (see Section \ref{sec:algorithm}). We say that $\left\{\pi^\ast\left(\bv{x}_k\mid\bv{x}_{k-1}\right)\right\}_{1:T}$, with  $\pi^\ast\left(\bv{x}_k\mid\bv{x}_{k-1}\right) = \sum_{i\in\sS}\alpha_k^{(i),\ast}\policy{\bv{x}_k\mid\bv{x}_{k-1}}{(i)}$, is the optimal behavior for the agent. In the problem, the cost formalizes the fact that the agent seeks to maximize its reward, while tracking (in the KL divergence sense) the target behavior. Minimizing the KL term amounts at minimizing the discrepancy between $\pi_{0:T}$ and $p_{0:T}$. This term can also be thought as a divergence regularizer and, when $p_{0:T}$ is uniform, it becomes an entropic regularizer.
The second and third constraints formalize the fact that, at each $k$, $\pi^\ast\left(\bv{x}_k\mid\bv{x}_{k-1}\right)$ is obtained by composing the pfs from the sources. The fulfillment of these constraints also guarantees that $\pi^\ast\left(\bv{x}_k\mid\bv{x}_{k-1}\right)\in\sD$, $\forall k$.  The first constraint is a box constraint and models the fact that the probability that the agent behavior is, at each $k$, inside some (e.g., safety) measurable set $\mathcal{X}_k \subseteq \mathcal{X}$ is greater than some $\varepsilon_k \ge 0$. Finally, we make the following standard assumption
\begin{Assumption}\label{asn:bounded}
the optimal cost of Problem \ref{prob:problem_merge} is bounded.
\end{Assumption}

\begin{algorithm}[b!]
    \caption{{Pseudo-code}}\label{alg}
\begin{algorithmic}[1]
\State \textbf{Input:} time horizon $T$, target behavior $p_{0:T}$, reward $r_k(\cdot)$, sources $\pi^{(i)}(\bv{x}_k\mid\bv{x}_{k-1})$, box constraints  (optional)
\State \textbf{Output:} optimal agent behavior $\left\{\pi^\ast\left(\bv{x}_k\mid\bv{x}_{k-1}\right)\right\}_{1:T}$
\State $\hat{r}_T(\bv{x}_T)\gets0$
%\State $\left\{\boldsymbol{\alpha}_{k}^\ast\right\}_{1:N}\gets0$
\State {\bf for} {$k = T:1$} {\bf do}
    \State $\bar{r}_k(\bv{x}_k) \gets r_k(\bv{x}_k) - \hat{r}_k(\bv{x}_k)$
    \State  $\boldsymbol{\alpha}^{\ast}_{k} (\bv{x}_{k-1}) \gets \text{minimizer of the problem in \eqref{eqn:subproblem_alg}}$;
    \State $\pi^\ast(\bv{x}_k\mid\bv{x}_{k-1})\gets\sum_{i\in\mathcal{S}}\boldsymbol{\alpha}^{(i),\ast}_{k}(\bv{x}_{k-1})\pi^{(i)}(\bv{x}_k\mid\bv{x}_{k-1})$ 
    \State $\hat{r}_{k-1}(\bv{x}_{k-1}) \gets c_k(\pi^\ast(\bv{x}_k\mid\bv{x}_{k-1}))$
    %\text{minimum of the problem in \eqref{eqn:subproblem_alg}}$
    %\State $\boldsymbol{\alpha}_{k}(\bv{x}_k)\gets$solve$(\bv{x}_k,\bar{r}_k)$
\State {\bf end for}
\end{algorithmic}
\end{algorithm}


\section{Main Results}\label{sec:algorithm}

We propose an algorithm to tackle Problem \ref{prob:problem_merge}. The algorithm takes as input $T$, the target behavior, the reward, the behaviors of the sources and the box constraints of Problem \ref{prob:problem_merge} (if any). Given the input, the algorithm returns the optimal behavior for the agent. The key steps of the algorithm are given as pseudo-code in Algorithm \ref{alg}. An agent that follows Algorithm \ref{alg} computes  $\left\{\boldsymbol{\alpha}_k\right\}_{1:N}$ via backward recursion (lines $4-9$). At each $k$, the $\boldsymbol{\alpha}_k$'s are obtained as the minimizers of the following optimization problem
\begin{equation}\label{eqn:subproblem_alg}
\begin{aligned}
    \underset{\boldsymbol{\alpha}_k}{\min}
    & \ c_k(\pi(\bv{x}_k\mid\bv{x}_{k-1}))\\
     s.t. & \ \mathbb{E}_{\pi(\bv{x}_k\mid\bv{x}_{k-1})}\left[\mathds{1}_{\mathcal{X}_k}(\bv{x}_k)\right]\geq 1-\epsilon_k\\
    & \ \pi(\bv{x}_k\mid\bv{x}_{k-1}) = \sum_{i\in\sS}\alpha_k^{(i)}\policy{\bv{x}_k\mid\bv{x}_{k-1}}{(i)} \\
     & \sum_{i\in\sS}\alpha_k^{(i)} = 1, \ \ \alpha_k^{(i)} \in [0, 1],
    \end{aligned}
\end{equation}
where
\begin{equation}\label{eqn:cost_step}
\begin{split}
c_k(\pi(\bv{x}_k\mid\bv{x}_{k-1})) & := \DKL\left(\pi(\bv{x}_k\mid\bv{x}_{k-1})\mid\mid p(\bv{x}_k\mid\bv{x}_{k-1})\right) \\
& - \E_{\pi(\bv{x}_k\mid\bv{x}_{k-1})}\left[\bar{r}_k(\bv{X}_{k})\right],
\end{split}
\end{equation}
with $\bar{r}_k(\cdot)$ iteratively built within the recursion (lines $5$, $8$). The weights are used (line $7$) to compute $\pi^\ast(\bv{x}_k\mid\bv{x}_{k-1})$. Next, we characterize convexity of the above problem and optimality of the solution returned by Algorithm \ref{alg}.

\begin{Remark}results are stated for continuous variables. The proofs for discrete variables, which would follow the  reasoning of the proofs given below, are  omitted here for brevity.
\end{Remark}
\subsection{Properties of Algorithm \ref{alg}}\label{sec:properties}

We first characterize convexity of the problems recursively solved in Algorithm \ref{alg} and then optimality. 

\subsubsection*{Convexity}
we are now ready to prove the following
\begin{Proposition}\label{lem:convexity}
let Assumption \ref{asn:int_def} hold. Then, the problem in \eqref{eqn:subproblem_alg} is convex.
\end{Proposition}



    %\begin{equation*}
    %\begin{split}
    %c_k(\pi(\bv{x}_k\mid\bv{x}_{k-1})) & \gets \DKL\left(\pi(\bv{x}_k\mid\bv{x}_{k-1})\mid|p(\bv{x}_k\mid\bv{x}_{k-1})\right)\\
    %& - \E_{\pi(\bv{x}_k\mid\bv{x}_{k-1})}\left[\bar{r}_k(\bv{X}_{k})\right]\end{split}\end{equation*}
    


%In Algorithm \ref{alg}, the operator solve$()$ refers to numerically solving the following optimization problem:
%\begin{equation}\label{eqn:subproblem_alg}
%    \begin{aligned}
%    \underset{\boldsymbol{\alpha}_k}{\text{min}}
%    &\DKL\left(\pi(\bv{x}_k|\bv{x}_{k-1})||p(\bv{x}_k|\bv{x}_{k-1})\right) - \E_{\pi(\bv{x}_k|\bv{x}_{k-1})}\left[\bar{r}_k(\bv{X}_{k})\right]\\
%    s.t. & \ \mathbb{E}_{\pi(\bv{x}_k\mid\bv{x}_{k-1})}\left[\mathds{1}_{\mathcal{X}_k}(\bv{x}_k)\right]\leq 1-\epsilon_k,\\
%    & \ \pi(\bv{x}_k|\bv{x}_{k-1}) = \sum_{i\in\sS}\alpha_k^{(i)}\policy{\bv{x}_k|\bv{x}_{k-1}}{(i)}, \\
%    & \sum_{i\in\sS}\alpha_k^{(i)} = 1, \ \ \alpha_k^{(i)} \in [0, 1].
%    \end{aligned}
%\end{equation}



%This yields the weight vector $\boldsymbol{\alpha}^\ast_k$. On the other hand, once the optimal weight vector is computed, cost$()$ is then used to compute the cost $c_k(\pi(\bv{x}_k\mid\bv{x}_{k-1}))$, with $\pi(\bv{x}_k\mid\bv{x}_{k-1})$ being obtained from the optimal weights computed previously. Together, these two functionals return the optimal cost for a given initial state and reward signal.\\
%Starting from the last time step, the algorithm iterates over each state and calculates the optimal cost at this state. Then, this information is passed down through the cost-to-go $\hat{r}$, until time step zero, where the policy $\pi(\bv{x}_1\mid\bv{x}_0)$ is computed. This policy is sampled to obtain a new state, and the algorithm is run again to obtain a new policy.


%\textcolor{red}{EG: The proof, algorithm and problem statement check out. I added a proposed clarification on the sampling mechanism for receding horizons in the paragraph above, but it's not vital if we can't spare the lines.}




\begin{proof}
we start with showing convexity of the constraints set. Clearly, the second and third constraint in the problem are convex. For the first constraint, we get
    \begin{align*}
        \mathbb{E}_{\pi(\bv{x}_k\mid\bv{x}_{k-1})}\left[\mathds{1}_{\mathcal{X}_k}(\bv{x}_k)\right] &= \int  \pi(\bv{x}_k\mid\bv{x}_{k-1})\mathds{1}_{\mathcal{X}_k}(\bv{x}_k)d\bv{x}_k\\
        &= \int_{\mathcal{X}_k}\sum_{i \in \mathcal S}\alpha_k^{(i)}\pi^{(i)}(\bv{x}_k\mid\bv{x}_{k-1})d\bv{x}_k\\
        &= \sum_{i \in \mathcal S}\alpha_k^{(i)}\int_{\mathcal{X}_k}\pi^{(i)}(\bv{x}_k\mid\bv{x}_{k-1})d\bv{x}_k,
        \end{align*}
which is therefore convex in the decision variables. Now, we show that the cost is also convex in these variables and we do so by explicitly computing, for each $\bv{x}_{k-1}$, its Hessian, say $\bv{H}(\bv{x}_{k-1})$. Specifically, after embedding the second constraint of the problem in (\ref{eqn:subproblem_alg}) in the cost and differentiating with respect to the decision variables we get, for each $j\in\sS$:
    \begin{equation*}
    \begin{split}
       & \frac{\partial c_k}{\partial \alpha_k^{(j)}}  := \frac{\partial}{\partial \alpha_k^{(j)}}\int_{\mathcal{X}} \sum_{i\in\sS}\alpha_k^{(i)}\policy{\bv{x}_k\mid\bv{x}_{k-1}}{(i)}\cdot \\
&\cdot\left(\log\left(\frac{\sum_{i\in\sS}\alpha_k^{(i)}\policy{\bv{x}_k\mid\bv{x}_{k-1}}{(i)}}{p(\bv{x}_k\mid\bv{x}_{k-1})}\right) - \bar{r}_k(\bv{x}_{k})\right)d\bv{x}_k\\
        &= \int_{\mathcal{X}} \frac{\partial}{\partial \alpha_k^{(j)}} \sum_{i\in\sS}\alpha_k^{(i)}\policy{\bv{x}_k\mid\bv{x}_{k-1}}{(i)}\cdot \\
&\cdot\left(\log\left(\frac{\sum_{i\in\sS}\alpha_k^{(i)}\policy{\bv{x}_k\mid\bv{x}_{k-1}}{(i)}}{p(\bv{x}_k\mid\bv{x}_{k-1})}\right) - \bar{r}_k(\bv{x}_{k})\right)d\bv{x}_k\\
        &= \int_{\mathcal{X}} \policy{\bv{x}_k\mid\bv{x}_{k-1}}{(j)}\Bigg( \Bigg.\log\left(\sum_{i\in\sS}\alpha_k^{(i)}\policy{\bv{x}_k\mid\bv{x}_{k-1}}{(i)}\right) \nonumber \\
        &- \log\left(p(\bv{x}_k\mid\bv{x}_{k-1})\right) - \bar{r}_k(\bv{x}_k) + 1\Bigg. \Bigg) d\bv{x}_k . \\
    \end{split}
    \end{equation*}
The above chain of identities was obtained by swapping integration and differentiation, leveraging the fact that the cost is smooth in the decision variables. Similarly, we  get
\begin{align*}
    \frac{\partial^2 c_k}{\partial \alpha_k^{(j)^2}} &=  \int \frac{\policy{\bv{x}_k\mid\bv{x}_{k-1}}{(j)}^2}{\sum_{i\in\sS}\alpha_k^{(i)}\policy{\bv{x}_k\mid\bv{x}_{k-1}}{(i)}}d\bv{x}_k  := h_{ii}(\bv{x}_{k-1}),
\end{align*}
and, for each $m \ne j$, $m\in\sS$,
\begin{align*}
    \frac{\partial^2 c_k}{\partial \alpha_k^{(j)}\partial \alpha_k^{(m)}} &= \int \frac{\policy{\bv{x}_k\mid\bv{x}_{k-1}}{(j)}\policy{\bv{x}_k\mid\bv{x}_{k-1}}{(m)}}{\sum_{i\in\sS}\alpha_k^{(i)}\policy{\bv{x}_k\mid\bv{x}_{k-1}}{(i)}}d\bv{x}_k \\&:= h_{im}(\bv{x}_{k-1}).
\end{align*}

%{\color{red}
%(for internal discussion, I'll only write within the colored segment so we can remove everything without perturbing the original layout)\\

%\begin{align*}
%    \frac{\partial^2 c_k}{\partial \alpha_k^{(j)}\partial \alpha_k^{(m)}} 
%    &= \int_{\bv{x}_k\in\mathcal{X}} \frac{\policy{\bv{x}_k|\bv{x}_{k-1}}{(j)}\policy{\bv{x}_k|\bv{x}_{k-1}}{(m)}}{\sum_{i\in\sS}\alpha_k^{(i)}\policy{\bv{x}_k|\bv{x}_{k-1}}{(i)}}d\bv{x}_k; \\
%    &\leq \int_{\bv{x}_k\in\mathcal{X}} \frac{ \policy{\bv{x}_k|\bv{x}_{k-1}}{(j)}\policy{\bv{x}_k|\bv{x}_{k-1}}{(m)}}{\policy{\bv{x}_k|\bv{x}_{k-1}}{(p)}}d\bv{x}_k;\\
%    &\leq \frac{1}{\gamma}  \int_{\bv{x}_k\in\mathcal{X}} \policy{\bv{x}_k|\bv{x}_{k-1}}{(j)}\policy{\bv{x}_k|\bv{x}_{k-1}}{(m)}d\bv{x}_k;\\
%    \policy{\bv{x}_k|\bv{x}_{k-1}}{(p)} &:= \underset{i}{min}\ \policy{\bv{x}_k|\bv{x}_{k-1}}{(i)} ;\\
%    \gamma &:= \underset{\bv{x}_k}{min}\ \policy{\bv{x}_k|\bv{x}_{k-1}}{(p)}.
%\end{align*}
%$\gamma$ is strictly positive (Asn 2), hence the hessian element is defined if Asn 1 hold.
%}
Also, following Assumption \ref{asn:int_def}, $\forall j,m\in \sS$ we have:
\begin{equation*}
\begin{split}
& \int\abs{ \frac{\policy{\bv{x}_k\mid\bv{x}_{k-1}}{(j)}\policy{\bv{x}_k\mid\bv{x}_{k-1}}{(m)}}{\sum_{i\in\sS}\alpha_k^{(i)}\policy{\bv{x}_k\mid\bv{x}_{k-1}} {(i)}}}d\bv{x}_k\\
& \le \frac{M}{m}\int\abs{\policy{\bv{x}_k\mid\bv{x}_{k-1}}{(j)}}d\bv{x}_k,
\end{split}
\end{equation*}
where we used the third constraint (in the above expressions we omitted the integration domain $\mathcal{X}$). That is, the above integrals are well defined and thus we can conclude the proof by computing $\bv{v}^T\bv{H}(\bv{x}_{k-1})\bv{v}$ for some non-zero $\bv{v}\in\mathbb{R}^S$:
\begin{align*}
    \bv{v}^T\bv{H}(\bv{x}_{k-1})\bv{v} &= \sum_{j,m}v_jv_mh_{jm} (\bv{x}_{k-1})\\
    &= \sum_{j,m}v_jv_m\int_{\mathcal{X}}a_{jm}(\bv{x}_k,\bv{x}_{k-1})d\bv{x}_k\\
    &= \int_{\mathcal{X}}\sum_{j,m}v_jv_ma_{jm}(\bv{x}_k,\bv{x}_{k-1})d\bv{x}_k,
\end{align*}
where the $a_{jm}$'s are the elements of the matrix
\begin{equation*}
\begin{split}
&A(\bv{x}_k,\bv{x}_{k-1}) := \bar{\pi} (\bv{x}_{k},\bv{x}_{k-1})\left[\begin{array}{*{20}c}
\policy{\bv{x}_k\mid\bv{x}_{k-1}}{(1)}\\ \vdots\\ \policy{\bv{x}_k\mid\bv{x}_{k-1}}{(S)}
\end{array}\right] \cdot \\
& \cdot\left[\begin{array}{*{20}c}
\policy{\bv{x}_k\mid\bv{x}_{k-1}}{(1)} & \ldots & \policy{\bv{x}_k\mid\bv{x}_{k-1}}{(S)}
\end{array}\right],
\end{split}
\end{equation*}
with $\bar{\pi} (\bv{x}_{k},\bv{x}_{k-1}) := {1}/\left({\sum_{i\in\sS}\alpha_k^{(i)}\policy{\bv{x}_k\mid\bv{x}_{k-1}}{(i)}}\right)$. The above expression is indeed positive semi-definite for each $\bv{x}_k$, $\bv{x}_{k-1}$ and from this we can draw the desired conclusion.
\end{proof}



\subsubsection*{Optimality} we can now prove the following

\begin{Proposition}\label{thm:1}
let Assumption \ref{asn:int_def} and Assumption \ref{asn:bounded} hold. Then, Algorithm \ref{alg} gives an optimal solution for Problem \ref{prob:problem_merge}.    
\end{Proposition}
\begin{proof}
the chain rule for the KL divergence and the linearity of expectation imply that the cost can be written as
\begin{equation}\label{eqn:cost_split}
\begin{split}
& \DKL\left(\pi_{0:T-1}\mid\mid p_{0:T-1}\right)  - \sum_{k=1}^{T-1}\E_{\pi(\bv{x}_{k-1})}\left[\tilde{r}_k(\bv{X}_{k-1})\right] \\
& + \E_{\pi(\bv{x}_{T-1})}\left[c_T(\pi(\bv{x}_T\mid\bv{x}_{T-1}))\right],
\end{split}
\end{equation}
where $c_T(\pi(\bv{x}_T\mid\bv{x}_{T-1}))$ is defined as in \eqref{eqn:cost_step} with $\bar{r}_T(\bv{x}_T)$ given by Algorithm \ref{alg} -- see lines $3$ and $5$ and note that, at time step $T$, $\bar{r}_T(\bv{x}_T) = {r}_T(\bv{x}_T)$. To obtain the above expression, the fact that $c_T(\pi(\bv{x}_T\mid\bv{x}_{T-1}))$ only depends on $\bv{x}_{T-1}$ was also used. Hence, Problem \ref{prob:problem_merge} can be split into the sum of two sub-problems: a first problem over $k\in 0:T-1$ and the second for $k=T$. For this last time step, the problem can be solved independently on the others and is given by:
\begin{equation}\label{eqn:subproblem_N}
    \begin{aligned}
    \underset{\boldsymbol{\alpha}_T}{\text{min}}
    &\E_{\pi(\bv{x}_{T-1})}[c_T(\pi(\bv{x}_T\mid\bv{x}_{T-1}))]\\
    s.t. & \ \mathbb{E}_{\pi(\bv{x}_T\mid\bv{x}_{T-1})}\left[\mathds{1}_{\mathcal{X}_T}(\bv{x}_T)\right]\geq 1-\epsilon_T,\\
    & \ \pi(\bv{x}_T\mid\bv{x}_{T-1}) = \sum_{i\in\sS}\alpha_T^{(i)}\policy{\bv{x}_T\mid\bv{x}_{T-1}}{(i)}, \\
    & \sum_{i\in\sS}\alpha_T^{(i)} = 1, \ \ \alpha_T^{(i)} \in [0, 1].
    \end{aligned}
\end{equation}
Using linearity of the expectation and the fact that the decision variable is independent on $\pi(\bv{x}_{T-1})$, we have that the minimizer of the problem in \eqref{eqn:subproblem_N} is the same as the problem in \eqref{eqn:subproblem_alg} with $k=T$. Following Proposition \ref{lem:convexity}, such a problem is convex and we denote its optimal solution as $\boldsymbol{\alpha}^{\ast}_{T} (\bv{x}_{T-1})$ -- see line $6$ of Algorithm \ref{alg} -- and the optimal cost of the problem, which is bounded by Assumption \ref{asn:bounded},  is $c_T(\pi^{\ast}(\bv{x}_T\mid\bv{x}_{T-1}))$, where: 
$$\pi^{\ast}(\bv{x}_T\mid\bv{x}_{T-1}) = \sum_{i\in\mathcal{S}}\boldsymbol{\alpha}^{(i),\ast}_{T}(\bv{x}_{T-1})\pi^{(i)}(\bv{x}_T\mid\bv{x}_{T-1}).
$$ 
This gives $\hat{r}_{T-1}(\bv{x}_{T-1})$ in Algorithm \ref{alg} (lines $7-8$), thus yielding the steps for the backward recursion of the Algorithm \ref{alg} at time step $T$. Now, the minimum value of the problem in \eqref{eqn:subproblem_N} is given by $\E_{\pi(\bv{x}_{T-1})}\left[\hat{r}_{T-1}(\bv{X}_{T-1})\right]$. Hence, the cost of Problem \ref{prob:problem_merge} becomes $
\DKL\left(\pi_{0:T-1}\mid\mid p_{0:T-1}\right) - \sum_{k=1}^{T-1}\E_{\pi(\bv{x}_{k-1})}\left[\tilde{r}_k(\bv{X}_{k-1})\right]  + \E_{\pi(\bv{x}_{T-1})}\left[\hat{r}_{T-1}(\bv{X}_{T-1})\right]$. Then, following the same reasoning used to obtain \eqref{eqn:cost_split} and by noticing that 
$\E_{\pi(\bv{x}_{T-1})}\left[\hat{r}_{T-1}(\bv{X}_{T-1})\right] = \E_{\pi(\bv{x}_{T-2})}\left[\E_{\pi(\bv{x}_{T-1}\mid\bv{x}_{T-2})}\left[\hat{r}_{T-1}(\bv{X}_{T-1})\right]\right]$, we get:
%(see \cite{designof} for the details):
\begin{equation*}
\begin{split}
& \DKL\left(\pi_{0:T-2}\mid\mid p_{0:T-2}\right)  - \sum_{k=1}^{T-2}\E_{\pi(\bv{x}_{k-1})}\left[\tilde{r}_k(\bv{X}_{k-1})\right] \\
& + \E_{\pi(\bv{x}_{T-2})}\left[c_{T-1}(\pi(\bv{x}_{T-1}\mid\bv{x}_{T-2}))\right],
\end{split}
\end{equation*}
where $c_{T-1}(\pi(\bv{x}_{T-1}\mid\bv{x}_{T-2}))$ is again given in \eqref{eqn:cost_step} with $\bar{r}_{T-1}(\bv{x}_{T-1})$ again defined as in Algorithm \ref{alg}. 
By iterating the arguments above, we find that at each time step Problem \ref{prob:problem_merge} can always be split as the sum of two sub-problems, where the last sub-problem can be solved independently on the previous ones. Moreover, the minimizer of this last sub-problem is always the solution of a problem of the form
\begin{equation}\label{eqn:subproblem}
    \begin{aligned}
    \underset{\boldsymbol{\alpha}_k}{\text{min}}
    &\DKL\left(\pi(\bv{x}_k\mid\bv{x}_{k-1})\mid\mid p(\bv{x}_k\mid\bv{x}_{k-1})\right) - \E_{\pi(\bv{x}_k\mid\bv{x}_{k-1})}\left[\bar{r}_k(\bv{X}_{k})\right]\\
    s.t. & \ \mathbb{E}_{\pi(\bv{x}_k\mid\bv{x}_{k-1})}\left[\mathds{1}_{\mathcal{X}_k}(\bv{x}_k)\right]\geq 1-\epsilon_k,\\
    & \ \pi(\bv{x}_k\mid\bv{x}_{k-1}) = \sum_{i\in\sS}\alpha_k^{(i)}\policy{\bv{x}_k\mid\bv{x}_{k-1}}{(i)}, \\
    & \sum_{i\in\sS}\alpha_k^{(i)} = 1, \ \ \alpha_k^{(i)} \in [0, 1],
    \end{aligned}
\end{equation}
where $\bar{r}_k(\bv{x}_k) := r_k(\bv{x}_k) - \hat{r}_k(\bv{x}_k)$, with $\hat{r}_k(\bv{x}_k) := c_{k+1}({\pi}^\ast(\bv{x}_{k+1}\mid\bv{x}_{k}))$ and $\pi^\ast(\bv{x}_k\mid\bv{x}_{k-1})=\sum_{i\in\mathcal{S}}\boldsymbol{\alpha}^{(i),\ast}_{k}(\bv{x}_{k-1})\pi^{(i)}(\bv{x}_k\mid\bv{x}_{k-1})$. This yields the desired conclusions.
\end{proof}
\begin{Remark}\label{rem:comparison}
consider the special case where Problem \ref{alg} has no box constraints    so that the results from \cite{russo2020crowdsourcing,designof}  can be used. Then,  a direct comparison between the behaviors obtained in \cite{russo2020crowdsourcing,designof} and the one computed via Algorithm \ref{alg} shows that the optimal solutions from \cite{russo2020crowdsourcing,designof} cannot obtain a better cost than the one achieved by Algorithm \ref{alg}. 
\end{Remark}



\section{Using Algorithm \ref{alg} to Design an Intelligent Parking System}\label{sec:application}

We now use Algorithm \ref{alg} to design an intelligent parking system for {connected cars} and validate the results via in-silico  and in-vivo experiments. For the latter set of experiments, Algorithm \ref{alg} is deployed on a real car and  validation is performed via an hardware-in-the-loop (HiL) platform inspired from \cite{Griggs2019}. Before reporting the results, we describe the validation scenarios and the experimental set-up. Code, maps and parameters with instructions to replicate the simulations are given at \url{https://tinyurl.com/3ep4pknh}. 

\begin{figure}[b!]
    \centering
    \includegraphics[width=0.8\columnwidth]{campus_map_new.png}
    \caption{campus map. The magnified areas show the obstructed road link (in blue) and links used within the validations. Colors online.}
    \label{fig:unisa_map}
\end{figure}

\subsection{Validation Scenarios and Experimental Set-up}

We consider the problem of routing vehicles in a given geographic area to find parking. In all experiments we consider a morning rush scenario at the University of Salerno campus (see Figure \ref{fig:unisa_map}). Specifically, cars arrive to the campus through a highway exit and, from here, users seek to park in one of the parking locations: {\em Biblioteca} and {\em Terminal}.

In this context, vehicles are agents equipped with Algorithm \ref{alg}. The set of road links within the campus is $\mathcal{X}$ and time steps are associated to the time instants when the vehicle changes road link. The state of the agent, $x_k$, is the road link occupied by the vehicle  at time step $k$. Given this set-up, at each $k$ Algorithm \ref{alg} outputs the turning probability for the car given the current car link,  $\pi^{\ast}({x}_k\mid {x}_{k-1})$. The next direction for the car is then obtained by sampling from this pf. Agents have access to a set of sources, each providing different routes.  As discussed in \cite{crawling}, sources might be third parties navigation services, routes collected from other cars/users participating to some sharing service. Agents wish to track their target/desired behavior (driving them to the preferred parking -- Terminal in our experiments) and the reward depends on the actual road conditions within the campus. Links adjacent to a parking lot are assigned a constant reward of: (i) $3.8$ if the parking has available spaces; (ii) {0 when it becomes full}. {Unparked cars already on full parking lots are assigned a target behavior leading them to another parking.} In normal road conditions, the reward for the other links is $0$ and becomes $-20$ when there is an obstruction. In the first scenario (Scenario $1$) there are no box constraints: this is done to benchmark Algorithm \ref{alg} with \cite{russo2020crowdsourcing,designof}. To this aim, we use the campus map from \cite{crawling} in which \cite{russo2020crowdsourcing,designof} were thoroughly validated via simulations. Then, we show that by introducing box constraints Algorithm \ref{alg} can effectively regulate access of vehicles through selected road links. This is Scenario $2$ and we considered three situations: (A) the road towards the {\em Biblioteca} parking lot is forbidden. To account for this, we set in Algorithm \ref{alg} $\mathcal{X}_k = \mathcal{X}\setminus {{l_2}}$, where the link ${l_2}$ is shown in Figure \ref{fig:unisa_map}, and $\epsilon_k = 0.027$;  (B) the set $\mathcal{X}_k$ as before but now $\epsilon_k = 0.5$; (C) the road towards the {\em Terminal} parking lot is forbidden and in this case we set $\mathcal{X}_k = \mathcal{X}\setminus {l_1}$, $\epsilon_k = 0.027$ (see Figure \ref{fig:unisa_map} for link ${l_1}$). For this last scenario, Algorithm \ref{alg} is validated both in-silico and in-vivo. Next, we describe the simulation and HiL experimental set-ups.

\subsubsection*{Simulation set-up}  simulations were performed in SUMO \cite{SUMO2018}; {see also \cite{crawling} for a description of the pipeline to import maps and traffic demands.} In our simulations, each parking lot can accommodate up to $50$ cars and we  generated the traffic demand so that $100$ cars would arrive on campus at $5$-second intervals. All the cars seek to park and, by doing so, the parking capacity is saturated. Given this setting, we simulated a road obstruction on the main link (in blue in Figure \ref{fig:unisa_map}) from the highway exit to the campus entrance. This was done by restricting, in SUMO, the speed of the link to less than one kilometer per hour. Information on the cars in the simulation are contained in the stand-alone file {\tt agent.npy}. Instead, the pfs associated with the sources (described below)  are all stored in {\tt behaviors.npy}. 

\subsubsection*{HiL set-up} the platform embeds a real vehicle into a SUMO simulation. By doing so, performance of the algorithm deployed on a real car can be assessed under arbitrary synthetic traffic conditions generated via SUMO. The high-level architecture of the platform is shown in Figure \ref{fig:HIL_archi}. The platform creates an avatar of the real car in the SUMO simulation. Namely, as shown in Figure \ref{fig:HIL_archi}, the position of the real car is embedded in SUMO by using a standard smartphone to collect its GPS coordinates. These coordinates are then sent via bluetooth to a computer, also hosted on the car in the current  implementation. The connection is established via an off-the-shelf app, which writes the coordinates in a {\tt rfcomm} file. By using the {\tt pySerial} library, an interface was developed to read data in Python. Here, a script was designed leveraging {\tt pynmeaGPS} to translate the data in the {\tt NMEA} format for longitude/latitude coordinates. With this data format, a Python script was created to place the avatar of the real car in the position given by the coordinates. A GUI is also included to highlight the trajectory of the real car on the map and an off-the-shelf text-to-speech routine is used to provide audio feedback to the driver on the vehicle.
%\begin{Remark}
%For both set-ups, all interfaces between SUMO and our algorithm are handled through TraCI in the main Python file. Logs are stored as {\tt .npy} files.
%\end{Remark}



\begin{figure}[b!]
    \centering
    \includegraphics[width=0.8\linewidth]{Presentazione_senza_titolo.png}
    \caption{HiL functional architecture. The smartphone acts as a GPS sensor, while the NMEA data is read and converted in Python.}
    \label{fig:HIL_archi}
\end{figure}

\subsection{In-car Implementation of the Algorithm}
For both the in-silico and in-vivo experiments, Algorithm \ref{alg} was implemented in Python as a stand-alone class so that each car equipped with the algorithm could function as a stand-alone agent. The class has methods accepting all the input parameters of Algorithm \ref{alg} and providing as output the car behavior computed by the algorithm. The optimization problems within the algorithm loop were solved via the Python library {\tt scipy.optimize}. Additionally, the class also implements  methods to compute the cost and to support receding horizon implementations of Algorithm \ref{alg}. In our experiments, the width of the receding horizon window is $T=5$ and every time the car enters in a new link/state computations are triggered. Following \cite{crawling}, we also implemented an utility function that restricts calculations of the agent only to the road links that can be reached in $T$ time steps (rather than through the whole state space/map). With this feature, the algorithm  takes on average approximately half a second to output a behavior (less than the typical time taken by cars to drive through a road link). Finally, the pfs of the sources were obtained via built-in routing functions in SUMO and we added noise to the routes so that Assumption \ref{asn:int_def} would be fulfilled. See our github for the details.

%a set of target destinations are selected (in this work, we chose both parking lots and the south of the campus). For a given road link and source, the next move towards the corresponding destination is assigned a probability of $0.94$, while the other accessible road links share the remaining probability. Such stochasticity accounts for driver error and for privacy requirements, with users being able to corrupt their historical data with e.g. laplacian noise. Stochastic behaviors can also arise when several actions fulfil the user's goal.\\
%Finally, if a connected car is on a parking lot with no vacant spaces (this happens if the parking lot became full after the car engaged on the corresponding road link but before it was able to reach a parking space), it is assigned a new target behavior and queries the decision-maker again.\\
%This, and all utility functions (such as computing the sufficient state space for the decision-making process, assigning parking spaces to cars when possible, or logging results), is done in the main Python file.

\subsection{Experimental Results}

%We now report the results from our experiments.

\subsubsection*{Simulation results}
first, we benchmarked the performance obtained by Algorithm \ref{alg}  against these from the algorithm in \cite{russo2020crowdsourcing,designof}, termed as crowdsourcing algorithm in what follows. To this aim, we considered Scenario $1$ and performed two sets of $10$ simulations. In the first set of experiments, Algorithm \ref{alg} was used to determine the behavior of cars on campus (note that Assumption \ref{asn:bounded} is fulfilled). In the second set of simulations, the cars instead used the crodwourcing algorithm. Across the simulations, we recorded the number of cars that the algorithms were able to park. The results are illustrated in Figure \ref{fig:unparked_plot}. The figure shows that the crowdsourcing algorithm was not able to park all the cars within the simulation. This was instead achieved by Algorithm \ref{alg}, which outperformed the algorithm from \cite{russo2020crowdsourcing,designof}. To further quantify the performance, we also computed the average time spent by a car looking for a parking space after it enters the simulation (ATTP: average time-to-parking). Across the simulations, the ATTP for the algorithm in \cite{russo2020crowdsourcing} was of $224.74 \pm 19.67$, while for Algorithm \ref{alg} it was of $151.32 \pm 30.59$ (first quantities are means, second quantities are  standard deviations). That is, Algorithm \ref{alg} yielded an average improvement of $32.7\%$ in the ATTP. Then, we simulated the three cases of Scenario $2$ to verify that Algorithm \ref{alg} can  effectively regulate access through specific links. The constraints for the three cases of Scenario $2$ were given as an input to the algorithm and in Figure \ref{fig:proba} the optimal solution $\pi^{\ast}({x}_k\mid x_{k-1}=l_r)$ is shown. The figure clearly illustrates that the optimal solution indeed fulfills the constraints (the road link $l_r$ is  in Figure \ref{fig:unisa_map}).


\begin{figure}[b!]
    \centering
    \includegraphics[width=0.65\columnwidth]{plot_merge.png}
    \caption{unparked cars over time for crowdsourcing  and for Algorithm \ref{alg}. Solid lines are means across the simulations, shaded areas are confidence intervals corresponding to the standard deviation.}
    \label{fig:unparked_plot}
\end{figure}



\begin{figure}[b!]
    \centering
    \includegraphics[width=0.65\columnwidth]{turning_probs.png}
    \caption{Turning probabilities $\pi^{\ast}({x}_k\mid x_{k-1}=l_r)$ from Algorithm \ref{alg} for the three cases of Scenario $2$. In all three cases, the pfs satisfy the constraints. Link definitions are in Figure \ref{fig:unisa_map}.}
    \label{fig:proba}
\end{figure}


\subsubsection*{HiL results} we deployed Algorithm \ref{alg} on a real vehicle using the HiL platform and validated its effectiveness using Scenario $2$ (C): the target behavior of the agent would make the real car reach the {\em Terminal} parking but this route is forbidden. What we observed in the experiment was that, once the car entered in the campus, this was re-routed towards the {\em Biblioteca} parking. The re-routing was an effect of Algorithm \ref{alg} computing the rightmost pf in Figure \ref{fig:proba}. A video of the HiL experiment is available on our github. The video shows that the algorithm is suitable for real car operation: Algorithm \ref{alg} would run smoothly during the drive, providing feedback to the driver on the vehicle. Figure \ref{fig:HIL} shows the car's route recorded during the experiment.

\begin{figure}[b!]
    \centering
    \includegraphics[width=0.7\columnwidth]{gps_track.png}\\
    \caption{Route of the real vehicle. The continuous line shows the  GPS position during the HiL experiment (map from OpenStreetMaps).}
    \label{fig:HIL}
\end{figure}



\section{Conclusions}

We considered the problem of designing agents able to compute optimal decisions by re-using data from multiple sources to solve tasks involving: (i) tracking a desired behavior while minimizing an agent-specific cost; (ii) satisfying certain safety  constraints. After formulating the control problem, we showed that this is convex under a mild condition and computed the optimal solution. We turned the results in an algorithm and used it to design an intelligent parking system. We experimentally evaluated the algorithm via in-silico and in-vivo validations with real vehicles/drivers. All experiments confirmed the effectiveness of the algorithm and its suitability for in-car operation. Besides embedding learning in the algorithm, our future research will involve devising mechanisms for the composition of policies to tackle the tasks with actuation constraints considered in \cite{gagliardi2020probabilistic}. 

\begin{thebibliography}{10}
\providecommand{\url}[1]{#1}
\csname url@rmstyle\endcsname
\providecommand{\newblock}{\relax}
\providecommand{\bibinfo}[2]{#2}
\providecommand\BIBentrySTDinterwordspacing{\spaceskip=0pt\relax}
\providecommand\BIBentryALTinterwordstretchfactor{4}
\providecommand\BIBentryALTinterwordspacing{\spaceskip=\fontdimen2\font plus
\BIBentryALTinterwordstretchfactor\fontdimen3\font minus
  \fontdimen4\font\relax}
\providecommand\BIBforeignlanguage[2]{{%
\expandafter\ifx\csname l@#1\endcsname\relax
\typeout{** WARNING: IEEEtran.bst: No hyphenation pattern has been}%
\typeout{** loaded for the language `#1'. Using the pattern for}%
\typeout{** the default language instead.}%
\else
\language=\csname l@#1\endcsname
\fi
#2}}

\bibitem{doi:10.1126/science.aab3050}
B.~M. Lake \emph{et~al.}, ``Human-level concept learning through probabilistic
  program induction,'' \emph{Science}, vol. 350, no. 6266, pp. 1332--1338,
  2015.

\bibitem{sharing_eco}
E.~Crisostomi \emph{et~al.}, \emph{Analytics for the sharing economy:
  Mathematics, Engineering and Business perspectives}.\hskip 1em plus 0.5em
  minus 0.4em\relax Springer, 2020.

\bibitem{russo2020crowdsourcing}
G.~Russo, ``On the crowdsourcing of behaviors for autonomous agents,''
  \emph{IEEE Cont. Sys. Lett.}, vol.~5, pp. 1321--1326, 2020.

\bibitem{designof}
{\'E}.~Garrab{\'e} and G.~Russo, ``On the design of autonomous agents from
  multiple data sources,'' \emph{IEEE Cont. Sys. Lett.}, vol.~6, pp. 698--703,
  2021.

\bibitem{https://doi.org/10.48550/arxiv.2211.05536}
\BIBentryALTinterwordspacing
L.~Li, C.~De~Persis, P.~Tesi, and N.~Monshizadeh, ``Data-based transfer
  stabilization in linear systems,'' 2022. [Online]. Available:
  \url{https://arxiv.org/abs/2211.05536}
\BIBentrySTDinterwordspacing

\bibitem{8795639}
J.~Coulson, J.~Lygeros, and F.~Dörfler, ``Data-enabled predictive control: In
  the shallows of the {DeePC},'' in \emph{European Control Conference}, 2019,
  pp. 307--312.

\bibitem{behav2}
H.~J. van Waarde, J.~Eising, H.~L. Trentelman, and M.~K. Camlibel, ``Data
  informativity: a new perspective on data-driven analysis and control,''
  \emph{IEEE Trans. Automatic Control}, vol.~65, pp. 4753--4768, 2020.

\bibitem{8933093}
C.~{De Persis} and P.~{Tesi}, ``Formulas for data-driven control:
  Stabilization, optimality, and robustness,'' \emph{IEEE Trans. Automatic
  Control}, vol.~65, pp. 909--924, 2020.

\bibitem{behav_review}
I.~Markovsky and F.~Dörfler, ``Behavioral systems theory in data-driven
  analysis, signal processing, and control,'' \emph{Annual Rev. in Control},
  vol.~52, pp. 42--64, 2021.

\bibitem{9763859}
F.~Celi and F.~Pasqualetti, ``Data-driven meets geometric control: Zero
  dynamics, subspace stabilization, and malicious attacks,'' \emph{IEEE Cont.
  Sys. Lett.}, vol.~6, pp. 2569--2574, 2022.

\bibitem{8039204}
U.~{Rosolia} and F.~{Borrelli}, ``Learning model predictive control for
  iterative tasks. {A} data-driven control framework,'' \emph{IEEE Trans.
  Automatic Control}, vol.~63, pp. 1883--1896, 2018.

\bibitem{DBLP:conf/l4dc/WabersichZ20}
K.~P. Wabersich and M.~N. Zeilinger, ``Bayesian model predictive control:
  Efficient model exploration and regret bounds using posterior sampling,''
  ser. Proc. of ML Research, vol. 120, 2020, pp. 455--464.

\bibitem{9109670}
J.~Berberich, J.~Kohler, M.~A. Muller, and F.~Allgower, ``Data-driven model
  predictive control with stability and robustness guarantees,'' \emph{IEEE
  Trans. Aut. Contr.}, vol.~66, pp. 1702--1707, 2021.

\bibitem{8703172}
G.~Baggio, V.~Katewa, and F.~Pasqualetti, ``Data-driven minimum-energy controls
  for linear systems,'' \emph{IEEE Cont. Sys. Lett.}, vol.~3, pp. 589--594,
  2019.

\bibitem{GARRABE202281}
Émiland Garrabé and G.~Russo, ``Probabilistic design of optimal sequential
  decision-making algorithms in learning and control,'' \emph{Annual Rev, in
  Control}, vol.~54, pp. 81--102, 2022.

\bibitem{9029512}
N.~Cammardella, A.~Busic, Y.~Ji, and S.~P. Meyn, ``{Kullback-Leibler-Quadratic}
  optimal control of flexible power demand,'' in \emph{IEEE 58th Conference on
  Decision and Control}, 2019, pp. 4195--4201.

\bibitem{books/daglib/0016248}
N.~Cesa-Bianchi and G.~Lugosi, \emph{Prediction, learning, and games.}\hskip
  1em plus 0.5em minus 0.4em\relax Cambridge University Press, 2006.

\bibitem{7106543}
M.~Cutler, T.~J. Walsh, and J.~P. How, ``Real-world reinforcement learning via
  multifidelity simulators,'' \emph{IEEE Trans. on Robotics}, vol.~31, pp.
  655--671, 2015.

\bibitem{Mountcastle97}
V.~B. Mountcastle, ``\BIBforeignlanguage{eng}{The columnar organization of the
  neocortex.}'' \emph{\BIBforeignlanguage{eng}{Brain}}, vol. 120, pp. 701--722,
  Apr 1997.

\bibitem{Griggs2019}
W.~Griggs \emph{et~al.}, \emph{A Vehicle-in-the-Loop Emulation Platform for
  Demonstrating Intelligent Transportation Systems}.\hskip 1em plus 0.5em minus
  0.4em\relax Cham: Springer International Publishing, 2019, pp. 133--154.

\bibitem{crawling}
\BIBentryALTinterwordspacing
{\'E}.~Garrabé and G.~Russo, ``{CRAWLING}: a {C}rowdsourcing {A}lgorithm on
  {W}heels for {S}mart {P}arking,'' 2022, preprint submitted to Scientific
  Reports. [Online]. Available: \url{https://arxiv.org/abs/2212.02467}
\BIBentrySTDinterwordspacing

\bibitem{SUMO2018}
Pablo Alvarez Lopez and others,
  ``Microscopic traffic simulation using {SUMO},'' in
  \emph{21st IEEE International Conference on Intelligent Transportation Systems},
  2018, pp. 2575--2582.

\bibitem{gagliardi2020probabilistic}
D.~Gagliardi and G.~Russo, ``On a probabilistic approach to synthesize control
  policies from example datasets,'' \emph{Automatica}, vol. 137, p. 110121,
  2022.

\end{thebibliography}


\end{document}
