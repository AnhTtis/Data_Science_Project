\section{Learning Quadrotor Dynamics}\label{sec:quad_dynamics}
We validate our approach by considering a learning residual quadrotor dynamics problem. In the following section, we describe the formulation of our problem for both a Single input single output (SISO) GP implementation and MISO GP formulation. For our SISO GP, we follow a similar modeling approach to \cite{Guillem_datadriveMPC}. We consider the following dynamics
\begin{equation}\label{eq:nominal_dynamic}
m\mathbf{a} = \mathbf{R}^W_B\tau \mathbf{e}_3 - mg\mathbf{e}_3, 
\end{equation}
where $\mathbf{e}_3 = [0~0~1]^\top$, $\mathbf{a}$ is the acceleration in the world frame of the quadrotor, $g$ the gravitational value, and $\mathbf{R}^W_B$ the rotation converting points from the quadrotor's body frame to the world frame. The vehicle's velocity in the body frame, $\mathbf{v}_b$ is used to estimate the acceleration error on the body of the vehicle $\bm{\delta}_a$. This is based on standard fluid dynamic approximations of the air resistance for a moving body when evolving at different speeds. In our dataset, we calculate the term of $\bm{\delta}_a$ as
\begin{equation}
     \bm{\delta}_a=\mathbf{R}^W_B\left(\frac{\mathbf{v}_b - \hat{\mathbf{v}}_b}{\delta t_k}\right),
\end{equation}
where $\hat{\mathbf{v}}_b$ is the predicted velocity from integrating eq.~(\ref{eq:nominal_dynamic}) from the previous time step, and $\delta t_k$ is the length of that time step. We can then construct our SISO GP using $\mathbf{v}_b$ as the input and $\bm{\delta}_a$ as the output, acceleration error, we would like to predict. To handle multiple outputs, we consider each body axis independent and apply an independent GP individually to each axis component. This makes our augmented GP dynamic formulation of eq.~(\ref{eq:nominal_dynamic}) as
\begin{equation}\label{eq:siso_drag}
        \mathbf{a} = \frac{1}{m}\mathbf{R}^W_B(\tau \mathbf{e_3} +f_{GP}(\mathbf{v}_b))+g\mathbf{e_3}.
\end{equation}
Fundamentally, the GP term $f_{GP}$ in eq.~(\ref{eq:siso_drag}) infers a first order propeller drag term~\cite{Guillem_datadriveMPC,svacha}. However, the vehicle is generally subject to additional types of disturbances. These disturbances are a function of motor thrust which is a function of $\omega_i$ of the motor's angular velocity through a $k_f$ coefficient, obtaining the total thrust as
\begin{equation}\label{eqn:motor_dynamics}
        \begin{aligned}
    \tau &=T_0 + T_1 + T_2 + T_3,~T_{i} = \omega_i^2*k_f,
   \end{aligned}
\end{equation}
where $T_{i}$ is the motor thrust generated by the motor $i$.  
However, the thrust mapping is only an approximation and does not account for other external factors or changes in the propeller over time~\cite{DAI_propeller_analysis}. Our system augments this term by using a MISO GP to calculate the disturbances related to motor thrust from both the fluid resistance from high body rate changes and inaccuracies in the thrust mapping. Therefore, our full system augmented model of eq.~(\ref{eq:nominal_dynamic}) for MISO becomes
\begin{equation}
        \mathbf{a} = \frac{1}{m}\mathbf{R}^W_B(\tau \mathbf{e_3} +f_{GP}(\bm{v}_b, \omega_0,\omega_1,\omega_2,\omega_3))+g\mathbf{e_3}.
\end{equation}

