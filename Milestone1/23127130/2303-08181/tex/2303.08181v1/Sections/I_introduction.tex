\section{Introduction}\label{sec_I}
Gaussian Processes (GPs) have become popular models in a wide range of robotics inference~\cite{Guillem_datadriveMPC,hewing2019cautious,Deisenroth_GP_robotics}, planning~\cite{barfoot2014batch}, and control problems~\cite{Anayo_reachabilityonlineGP,WangonlineGP2018,loquercio2022}. 
However, their inference time scales cubically with respect to the dataset size, thus making these models unsuitable for many real-time robotics problems, especially on Size, Weight, and Power (SWaP) constrained aerial robots. 
This motivates the need to develop novel approximation techniques that can perform inference at a much higher rate without sacrificing the accuracy of the original model.
In this paper, we present \projectName, a GP Toolkit that converts a GP into a linear state space form. This form processes information sequentially to learn the data trend allowing the computation to scale linearly with the dataset instead. 
\projectName~can easily instantiate any kernel or combination of them in C++ with minor or no user effort. 
The approach does not perform the expensive matrix inversions or multiplications of a typical GP and is easily portable to  SWaP-constrained robots.

%we can freely apply any kernel to C++ without the need to implement the kernel function and perform expensive matrix inversions as is typical for many embedded systems.
%The approach is easily portable to several optimizers used in robotics. We achieve this by converting the Gaussian process inference model from its typical form to a linear state space model which processes data sequentially to learn the data trend for data inference.

%This linear state space model allows our library to be easily portable to any other system by simply swapping the transition and update matrix we can freely apply any kernel to C++ without the need to implement the kernel function and perform expensive matrix inversions as is typical for many embedded systems.
% are expressive non-parametric models capturing signal statistics and expressing prediction uncertainty~\cite{RasmussenW06}. 
%As a result, the robotics community has applied these methods to a variety of tasks including:  learning residual dynamics \cite{hewing2019cautious}, learning state transitions models \cite{Deisenroth_GP_robotics}, or safety regions for repeated tasks \cite{WangonlineGP2018}.  Unfortunately being a non-parameteric model, the Gaussian Process inference time is  $O(n^3)$ therefore typically scaling with the cube of the dataset size $n$. This motivates us to develop quicker inference approximations of the GP.

 \begin{figure}
     \centering
     \vspace{0.7em}
     \includegraphics[width=\linewidth, trim=0 0 0 0, clip]{images/First_Page/quadrotor.png}\\
     \includegraphics[width=1.3\linewidth, trim=350 500 0 0, clip]{images/First_Page/legend_gp.pdf}\\
     \includegraphics[width=\linewidth, trim=0 0 0 0, clip]{images/First_Page/siso_KF_pred_test_1.png}
     \caption{
     Comparison between GaPT and a classical GP inference approach (GPyTorch) for learning quadrotor residual dynamics during agile flight on $6000$ test data points.
    %  The histogram shows the superior computational advantages of GaPT which requires $\times 55$ less time
    %  compared to the GP baseline, while also producing similar inference results when estimating external disturbances.
    %Moved the above caption into text to keep the sam eidea
     }
     \label{fig:intro_fig}
     \vspace{-2em}
\end{figure}

Our main contributions are the following. First, we design and present \projectName, a GP Toolkit that can perform inference at a much faster rate than the standard inference model used in popular libraries such as~\cite{gardner2018gpytorch} by translating GPs into a linear state space form. 
The linear state space form or State Space Gaussian Process (SSGP) scales linearly with respect to the dataset size. This allows quicker data processing and overcomes one of the major hurdles related to real-time robotics applications, especially when employing large datasets on SWaP constrained aerial robots. In fact, the histogram in Fig.~\ref{fig:intro_fig} shows the superior computational advantages of GaPT which requires $\times 37$ less time compared to the GP baseline, while also producing similar inference results when estimating external disturbances.
Second, \projectName~ includes the state space realization for Single Input Single Output (SISO) as well as for Multiple Input Single Output (MISO) robotics systems. 
By expanding the input space to multi-dimensions, \projectName~ better captures the high-frequency characteristic of the true signal, outperforming single input GPs on unfiltered real-world data, as experimentally validated in our results. % where for unfiltered real-world data where the MISO inference is substantially better than the SISO. 
Third, we provide a variety of ready-to-use kernel implementations such as RBF, Periodic, and Matern kernels. Fourth, we validate our approach on real-world learning quadrotor dynamics inference problems
to capture the system behavior in multiple flight regimes and operating conditions, including those producing highly nonlinear effects such as aerodynamic forces and torques, and rotor interactions. Finally, our method includes a quick training routine and can easily export C++ ready functions that can be integrated into several optimization libraries (e.g., \cite{acados,FORCESPro,mastalli20crocoddyl}) often used on small-scale aerial~\cite{Guillem_datadriveMPC} and ground~\cite{hewing2019cautious} robots. Our approach and its open source implementation will contribute to reducing the entry barrier to the adoption of these approaches avoiding the complex integration and computation time required by several libraries, especially for SWaP robots.


%In many aerial robotics use cases~\cite{Guillem_datadriveMPC, Mehndiratta2020}, an RBF kernel is the only option considered despite real-world data being often too complex for an RBF kernel to completely model the task. 
%that is highly noisy to show the robustness of our approach. Here we show that \projectName~ is able still to extract trends and have a strong fit over complex patterns that only improve when expanding the input.


%We validate the proposed approach in several learning dynamics inference problems.
%However, real-world data is often too complex for an RBF kernel to completely model the task, and better results can be obtained by considering different kernels 
%The aforementioned derivation addresses the challenges related to transitioning this class of processes from theory to practice by showing a clear hierarchical formulation that can run on real robots.The proposed approach addresses the challenges related to transitioning this class of processes from theory to practice for robotics problems by showing a clear hierarchical derivation of the state space representation of the GP. 
%

%
%such pytorch/tensorflow, or gpytorch~\cite{gardner2018gpytorch} 

%Second, a greater variety of kernels than RBF is significant because often times in literature such as \cite{Guillem_datadriveMPC,park2020gaussian} a RBF kernel is the only kernel considered. Real world data is often too complex for an RBF Kernel to completely model, and better results can be obtained from combining multiple of the above kernels as we  demonstrate.



%Although there are tools such as GPyTorch and Gpflow that are optimized for probabilistic inference through Gaussian processes, the computational complexity is dominated by the inversion of covariance matrices, which is a bottleneck in the case of large datasets.
