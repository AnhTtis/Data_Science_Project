%  This file is part of the APS files in the REVTeX 4.2 distribution.
%  Version 4.2a of REVTeX, December 2014
%
%  Copyright (c) 2014 The American Physical Society.
%
%  See the REVTeX 4 README file for restrictions and more information.
%
% TeX'ing this file requires that you have AMS-LaTeX 2.0 installed
% as well as the rest of the prerequisites for REVTeX 4.2
%
% See the REVTeX 4 README file
% It also requires running BibTeX. The commands are as follows:
%
% 1) latex apssamp.tex
% 2) bibtex apssamp
% 3) latex apssamp.tex
% 4) latex apssamp.tex
%
\documentclass[%
 reprint,
%superscriptaddress,
%groupedaddress,
%unsortedaddress,
%runinaddress,
%frontmatterverbose, 
%preprint,
%preprintnumbers,
%nofootinbib,
%nobibnotes,
%bibnotes,
 amsmath,amssymb,
 aps,
 prl,
%pra,
%prb,
%rmp,
%prstab,
%prstper,
floatfix,
]{revtex4-2}

\usepackage{comment}
\usepackage{graphicx}% Include figure files
\usepackage{dcolumn}% Align table columns on decimal point
\usepackage{bm}% bold math
\usepackage{hyperref}% add hypertext capabilities
\DeclareUnicodeCharacter{2009}{\,}

\begin{document}

\preprint{APS/123-QED}

\title{Analytic understanding of the resonant nature of Kozai Lidov Cycles
 with a precessing quadruple potential}

\author{Ygal Y. Klein}
\email{ygalklein@gmail.com}
 
\author{Boaz Katz}
 
\affiliation{
 Dept. of Particle Phys. \& Astrophys., Weizmann Institute of Science,
 Rehovot 76100, Israel
}

% \date{\today}% It is always \today, today,
       % but any date may be explicitly specified
\date{22/03/2023}

\begin{abstract}
 The very long-term evolution of the hierarchical restricted three-body problem with a precessing quadruple potential is studied analytically. This problem describes the evolution of a star and a planet which are perturbed either by a (circular and not too inclined) binary star system or by one other star and a second more distant star, as well as a perturbation by one distant star and the host galaxy or a compact-object binary system orbiting a massive black hole in non-spherical nuclear star clusters \citep{hamers2017,petrovich2017}. Previous numerical experiments have shown that when the precession frequency is comparable to the Kozai-Lidov time scale, long term evolution emerges that involves extremely high eccentricities with potential applications for a broad scope of astrophysical phenomena including systems with merging black holes, neutron stars or white dwarfs. We show that a central ingredient of the dynamics is a resonance between the perturbation frequency and the precession frequency of the eccentricity vector in the regime where the eccentricity vector, the precession axis and the quadruple direction are closely aligned. By averaging the secular equations of motion over the Kozai-Lidov Cycles we solve the problem analytically in this regime. 
\end{abstract}

\maketitle

In this letter we study analytically the dynamics of a test particle
orbiting a central mass $M$ on a keplerian orbit with semi major
axis $a$ which is perturbed by an external time dependent quadruple potential given by: 
\begin{equation}
\Phi_{outer}=\frac{\Phi_0}{a^2}\left[3\left(\mathbf{\hat{j}}_{outer}\cdot\mathbf{r}\right)^{2}-r^{2}\right]\label{eq:potential}
\end{equation}
where $\Phi_0$ is constant and $\mathbf{\hat{j}}_{outer}$ is a unit vector which precesses around the $z$ axis at a constant rate $\beta$ with a constant inclination $\alpha$:
\begin{equation}
 \mathbf{\hat{j}}_{outer}=\left(\begin{array}{c}
   \sin\alpha\cos\left(\beta\tau\right) \\
   \sin\alpha\sin\left(\beta\tau\right) \\   
   \cos\alpha
  \end{array}\right)\label{eq:jOuter_as_a_function_of_tau}
\end{equation}
where
\begin{gather}
 \tau\equiv\left(\frac{4\Phi_0}{\sqrt{GMa}}\right)t\label{eq:tau}.
\end{gather}

This problem describes the evolution of a star and a planet which are perturbed either by a (circular and not too inclined) binary star system or by one other star and a second more distant star \citep{hamers2017}, as well as a perturbation by one distant star and the host galaxy or a compact-object binary system orbiting a massive black hole in non-spherical nuclear star clusters \citep{petrovich2017}. Previous numerical experiments have shown that when the precession frequency is comparable to the Kozai-Lidov time scale, long term evolution emerges that involves extremely high eccentricities \citep{hamers2017} with potential applications for the formation of planets around white dwarfs \citep{munoz20,oconnor21,stephan21} and hot planets \citep{fabrycky2007,katz2011,naoz2011,grishin18}. If the test particle assumption is relaxed, the system exhibit similar dynamics and the description is applicable to a broader scope of astrophysical phenomena, including Type Ia supernovae through the merger or collision of white dwarfs in multiple systems \citep{thompson2011,katz2012,pejcha2013,fang2018,grishin22}, gravitational wave emission through the merger of black holes or neutron stars in quadruple systems \citep{liu2019,safarzadeh2020,hamers2020} and the formation of close binaries \citep{antonini2012,petrovich2017,bub2020,grishin22}.

The case of $\alpha=0$ is the periodic analytically solved
Kozai-Lidov cycles (KLCs) \citep{lidov1962,kozai62}.

\paragraph{Equations of motion}

The dynamics of the test particle can be parameterized
by two orthogonal vectors $\mathbf{j}=\mathbf{J}/\sqrt{GMa}$ and
$\mathbf{e}$ a vector pointing in the direction of the pericenter
with magnitude $e$. The Kozai-Lidov equations for $\mathbf{j}$ and
$\mathbf{e}$ are given by (as Eq. 10a-b in \citep{hamers2017})
\begin{align}
\frac{d\mathbf{j}}{d\tau}&=-\frac{3}{4}\left(\left(\mathbf{j}\cdot\mathbf{\hat{j}}_{outer}\right)\mathbf{j}-5\left(\mathbf{e}\cdot\mathbf{\hat{j}}_{outer}\right)\mathbf{e}\right)\times\mathbf{\hat{j}}_{outer} \cr
\frac{d\mathbf{e}}{d\tau}&=\frac{3}{4}\left(5\left(\mathbf{e}\cdot\mathbf{\hat{j}}_{outer}\right)\mathbf{j}-\left(\mathbf{j}\cdot\mathbf{\hat{j}}_{outer}\right)\mathbf{e}\right)\times\mathbf{\hat{j}}_{outer}-\frac{3}{2}\left(\mathbf{j}\times\mathbf{e}\right) \cr
 \label{eq:secular_equations}
\end{align}
with $\mathbf{\hat{j}}_{outer}$ given by Eq. \ref{eq:jOuter_as_a_function_of_tau}. A numerical integration of Eqs. \ref{eq:secular_equations} for $\alpha=0.01$ is shown as blue lines in the top two panels of Fig. \ref{fig:delta_and_jz_one_minus_e_alpha_0.01_ez_0.98}.

\begin{figure}
 \begin{centering}
 \includegraphics[scale=0.3]{assets/figures/alpha_0.01_ez_0.98/one_minus_e}
  \par\end{centering}
 \begin{centering}
  \includegraphics[scale=0.3]{assets/figures/alpha_0.01_ez_0.98/jz_zoomed}
  \par\end{centering}
 \begin{centering}
  \includegraphics[scale=0.3]{assets/figures/alpha_0.01_ez_0.98/delta}
  \par\end{centering}
 \caption{Results of numerical integrations for $\alpha=0.01$,
 with initial conditions $e_{x}=j_{x}=-j_{y}=10^{-5},e_{z}=0.98$.
 The blue solid lines are the result of the integration of the full
 secular equations, Eqs. \ref{eq:secular_equations} (with \ref{eq:jOuter_as_a_function_of_tau}),
 while the red dashed lines are the result of the averaged equations,
 Eqs. \ref{eq:splusDot}-\ref{eq:a_minus_b_dot}, and using the Kozai-Lidov
 relations to determine $e_{min}$ and $e_{max}$ using Eqs. \ref{eq:C4}
 and \ref{eq:Ck}. The two green horizontal lines in the bottom panel
 represent the extremum values of $\delta$ as determined from initial
 conditions and in the middle panel the maximal and minimal values
 of $j_{z}$ as determined from initial conditions using Eqs. \ref{eq:C4}
 and the extremums of $\delta$.\label{fig:delta_and_jz_one_minus_e_alpha_0.01_ez_0.98}}
\end{figure}

We restrict the analysis to the regime where $\alpha\ll1$ (i.e $\alpha$ being a
small parameter around which $\alpha=0$ is already analytically solved)
and $\left|\mathbf{e}\cdot\mathbf{\hat{j}}_{outer}\right|\sim1$ (i.e
$\mathbf{j}\cdot\mathbf{\hat{j}}_{outer} \approx j_z \ll1$). In this regime, the eccentricity vector precesses around the $z$ axis. When the frequencies of the precession of $\mathbf{e}$ and $\mathbf{\hat{j}}_{outer}$ are far from each other - the precession of the quadruple potential has a minor effect on the KLCs. On the other hand, when these two frequencies are close, long-term resonant dynamics are obtained and are the focus of this letter.

\paragraph{Approximated Equations}

In this regime and up to first order in $\alpha$ one obtains the
following 6 equations (neglecting $j_{z}$ in this regime for the
derivatives):
\begin{align}
 \frac{d}{d\tau}j_{z}&=\frac{15}{4}e_{z}\alpha\left(e_{x}\sin\left(\beta\tau\right)-e_{y}\cos\left(\beta\tau\right)\right)\label{eq:jzdot} \\
 \frac{d}{d\tau}e_{z}&=-\frac{3}{4}\left(2\left(j_{x}e_{y}-j_{y}e_{x}\right)-5e_{z}\alpha\left(j_{x}\sin\left(\beta\tau\right)-j_{y}\cos\left(\beta\tau\right)\right)\right)\label{eq:ezdot}
\end{align}
\begin{align}
 \frac{d}{d\tau}e_{x} & =+\frac{9}{4}e_{z}j_{y}\label{eq:exdot} \\
 \frac{d}{d\tau}e_{y} & =-\frac{9}{4}e_{z}j_{x}\label{eq:eydot}
\end{align}
\begin{align}
 \frac{d}{d\tau}j_{x} & =\frac{15}{4}e_{z}\left(e_{y}-e_{z}\alpha\sin\left(\beta\tau\right)\right)\label{eq:jxdot} \\
 \frac{d}{d\tau}j_{y} & =\frac{15}{4}e_{z}\left(e_{z}\alpha\cos\left(\beta\tau\right)-e_{x}\right)\label{eq:jydot}
\end{align}
In the lowest order approximation,
$\frac{d}{d\tau}e_{z}=0$, resulting with a forced harmonic
oscillator for the vector $\mathbf{e}$ in the $x-y$ plane with $\ddot{e}_{x}=\omega_{0}^{2}\left(L\cos\left(\omega\tau\right)-e_{x}\right)$
where $\omega=\beta,L=e_{z}\alpha$ and $\omega_{0}=\sqrt{\frac{135}{16}}e_{z}$.
Below we solve the next level of approximation where $e_{z}$ is slowly changing.

\paragraph*{Averaged Equations\label{sec:Averaged Equations}}

Since $\alpha$ is small the dynamics on short time scales follow
the known (test particle triple system) Kozai-Lidov Cycles, which
have two constants of motion: $j_{z}$ and
\begin{equation}
C_{K}=e^{2}-\frac{5}{2}e_{z}^{2}=e^{2}\left(1-\frac{5}{2}\sin^{2}i\sin^{2}\omega\right).
\end{equation}
On longer time scales the parameters of the KLC, $j_{z}$ and $C_{K}$,
evolve.

Consider the following ansatz for the vector $\mathbf{e}$
in the $x-y$ plane: At any time $\tau$, the projection of the vector
$\mathbf{e}$ on the $x-y$ plane can be presented as a point moving
on a slowly evolving ellipse with semi major axis $a$ inclined with an angle $\theta$ with respect to the $x$ axis and semi minor axis $b$ centered at the origin, i.e
\begin{equation}
 \mathbf{e}_{x-y}=\alpha^{\frac{1}{3}}\left(\begin{array}{cc}
   \cos\theta, & -\sin\theta \\
   \sin\theta, & \cos\theta
  \end{array}\right)\left(\begin{array}{c}
   a\cos\left(\hat{\beta}\hat{\tau}+\phi\right) \\
   b\sin\left(\hat{\beta}\hat{\tau}+\phi\right)
  \end{array}\right)\label{eq:exy-ansatz}
\end{equation}
where $\phi$ is a slowly dynamically evolving phase and
\begin{gather}
 \hat{\tau}=\frac{1}{2}\alpha^{\frac{2}{3}}\tau\label{eq:tau_hat}\\
 \hat{\beta}=2\alpha^{-\frac{2}{3}}\beta\label{eq:beta_hat}.
\end{gather}
See note after Eq. \ref{eq:a_minus_b_dot} regarding the choice of normalization prefactors: $\alpha^{\frac{1}{3}},\alpha^{\frac{2}{3}}$ and $\alpha^{-\frac{2}{3}}$.
The ansatz in Eq. \ref{eq:exy-ansatz} has a symmetry under the following transformation (both changes together)
\begin{equation}
\begin{cases}
\begin{array}{c}
\left(a+b\right)\rightarrow-\left(a+b\right)\\
\left(\theta+\phi\right)\rightarrow\left(\theta+\phi+\pi\right)
\end{array}
\end{cases}
\end{equation}
meaning that without loss of generality $\left(a+b\right)$ is non negative.

Using Eqs. \ref{eq:exdot}-\ref{eq:eydot} in the limit $e_{z}=1$ and neglecting the time derivatives of the slowly varying functions, the projection of the angular momentum on the $x-y$ plane is correspondingly given by
\begin{equation}
 \mathbf{j}_{x-y}=-\frac{4}{9}\alpha^{\frac{1}{3}}\beta\left(\begin{array}{cc}
   \cos\theta, & -\sin\theta \\
   \sin\theta, & \cos\theta
  \end{array}\right)\left(\begin{array}{c}
   b\cos\left(\hat{\beta}\hat{\tau}+\phi\right) \\
   a\sin\left(\hat{\beta}\hat{\tau}+\phi\right)
  \end{array}\right).\label{eq:jxy}
\end{equation}
Note the ansatz includes four slowly evolving variables, $a,b,\theta,\phi$, which describe the averaged evolution of the four components $e_x,e_y,j_x,j_y$.

Since the frequency of the precession of $\mathbf{\hat{j}}_{outer}$ is $\beta$ and the driving frequency of the Kozai oscillations is $\sqrt{\frac{135}{16}}e_{z}$ a resonance is obtained between the two perturbations when the two frequencies approach each other and it is useful to quantify the distance from resonance by a dynamical parameter,
\begin{equation}
 \delta=\alpha^{-\frac{2}{3}}\frac{1}{\beta_{0}}\left(\left(\beta_{0}\bar{e}_{z}\right)^{2}-\beta^{2}\right)\label{eq:delta}
\end{equation}
where $\bar{e}_{z}$ is the averaged value of $e_{z}$ over KLC which satisfies (using Eq. \ref{eq:ezdot})
\begin{equation}
 e_{z}=\bar{e}_{z}+\frac{\alpha^{\frac{2}{3}}}{6}\left(a^{2}-b^{2}\right)\cos\left(2\left(\hat{\beta}\hat{\tau}+\phi\right)\right)\label{eq:instantanous_ez}
\end{equation}
and
\begin{equation}
 \beta_{0}=\sqrt{\frac{135}{16}}\label{eq:beta0}.
\end{equation}
Using the following slow variables 
\begin{gather}
 s=\frac{45}{2}\left(a+b\right)\sin\left(\theta+\phi\right)\\
 c=\frac{45}{2}\left(a+b\right)\cos\left(\theta+\phi\right)
\end{gather}
and focusing on the resonant limit of $\omega=\omega_{0}$ in the forced harmonic oscillator mentioned above, i.e $\beta=\beta_{0}$, the following set of ODEs is obtained:
\begin{gather}
 \dot{\delta}=s\label{eq:deltaDot}\\
 \dot{s}=\delta c-45\beta_{0}\label{eq:splusDot}\\
 \dot{c}=-\delta s\label{eq:cPlusDot}
\end{gather}
and 
\begin{align}
 \frac{d\left(\theta-\phi\right)}{d\hat{\tau}} & =-\delta\label{eq:theta_minus_phi_dot} \\
 \frac{d\left(a-b\right)}{d\hat{\tau}}     & =0\label{eq:a_minus_b_dot},
\end{align}
where $\dot{}$ denotes a derivative with respect to $\hat{\tau}$.

Several notes are in order: (1) The parameters $s,c,a-b,\theta-\phi$ uniquely determine all the slow variables: $a,b,\theta,\phi$. (2) The evolution of $\delta,s$ and $c$ can be obtained by solving the closed subset of Eqs. \ref{eq:deltaDot}-\ref{eq:cPlusDot}. (3) The equations obtained have no explicit dependence on the small parameter $\alpha$. In fact, the $\alpha$ dependent prefactors in Eqs. \ref{eq:exy-ansatz}-\ref{eq:delta} were chosen for this reason. (4) $j_{z}$ can be obtained from the demand that $\mathbf{j}\cdot\mathbf{e}=0$. In fact, the following combination of $j_z$ and $\delta$ is constant:
\begin{equation}
j_z+\frac{\delta}{6}\alpha^{\frac{2}{3}}=\text{const.},\label{eq:C4}
\end{equation}
allowing $j_z$ to be readily obtained using the initial conditions and the time evolution of $\delta$.
The resulting slow evolution of $\delta$ for the example in Fig. \ref{fig:delta_and_jz_one_minus_e_alpha_0.01_ez_0.98} is shown in the bottom panel and is used for the solution of $j_z$ plotted as a dashed red line in the middle panel. As can be seen, the slow evolution of $j_z$ agrees to an excellent approximation with the numerical result. 

\paragraph{Analytic Solution\label{sec:Analytic-Solution}}

The averaged equations, Eqs. \ref{eq:deltaDot}-\ref{eq:cPlusDot}, admit two constants of the motion, denoted $C_{1}$ and $C_{2}$,
\begin{gather}
 C_{1}=\frac{1}{2}\left(\delta^{2}+45\left(a+b\right)\cos\left(\theta+\phi\right)\right)\label{eq:C1}\\
 C_{2}=\left(a+b\right)^{2}+\frac{2\delta}{\sqrt{15}}\label{eq:C2}
\end{gather}
implying that the evolution of the three variables $\delta,a+b$ and $\theta+\phi$ is periodic.
Using Eq. \ref{eq:a_minus_b_dot} we define a third constant
\begin{equation}
 C_{3}=\left(a-b\right)^{2}\label{eq:C3}
\end{equation}
which together with $C_1$ and $C_2$ determine the long term
evolution of the entire system.

The resulting evolution of $\delta$ is equivalent to the dynamics of a particle moving in a one dimensional potential with a constant energy 
\begin{equation}
 E=\frac{1}{2}\dot{\delta}^{2}+V=\frac{1}{2}\left(\left(\frac{45}{2}\right)^{2}C_{2}-C_{1}^{2}\right)\label{eq:energy}
\end{equation}
where (using $\beta_{0}=\sqrt{\frac{135}{16}}$,
see Eq. \ref{eq:beta0})
\begin{equation}
 V=45\beta_{0}\delta-\frac{1}{2}C_{1}\delta^{2}+\frac{1}{8}\delta^{4}\label{eq:Potential}.
\end{equation}

This potential has two distinct shapes depending on whether $C_1$ is smaller or larger than the critical value
\begin{equation}
 C_{1}^{\text{crit}}=15\left(\frac{3}{2}\right)^{\frac{7}{3}}.
\end{equation}
If $C_{1}<C_{1}^{\text{crit}}$ the potential has no maxima and one
minimum (see example in the left upper panel of Fig. \ref{fig:potential_energy_and_libration_rotation_maps}).
If $C_{1}>C_{1}^{\text{crit}}$ the potential has a maxima and two minima (see example in the right upper panel of Fig. \ref{fig:potential_energy_and_libration_rotation_maps} showing the case that is solved in Fig. \ref{fig:delta_and_jz_one_minus_e_alpha_0.01_ez_0.98}) \footnote{For analysis of the different features we equip the reader with a
 link to a visualization of Eqs. \ref{eq:Potential}, \ref{eq:energy}: \url{https://www.desmos.com/calculator/ubgicqtddn}}. The extremum values of $\delta$ determined from the potential and energy are marked as red circles in the top panels of Fig. \ref{fig:potential_energy_and_libration_rotation_maps} and are plotted as green lines in the bottom panel of Fig. \ref{fig:delta_and_jz_one_minus_e_alpha_0.01_ez_0.98}. The extremums of $\left(a+b\right)$ are readily given using Eq. \ref{eq:C2} and are marked as red circles in the bottom panels of Fig. \ref{fig:potential_energy_and_libration_rotation_maps}.

\begin{figure}
 \begin{centering}
  \includegraphics[scale=0.22]{assets/figures/C1_2_C2_5/V_and_E_vs_delta}\includegraphics[scale=0.22]{assets/figures/alpha_0.01_ez_0.98/V_and_E_vs_delta}
  \par\end{centering}
 \begin{centering}
  \includegraphics[scale=0.22]{assets/figures/C1_2_C2_5/aPlusb_vs_Theta_C1_contours}\includegraphics[scale=0.22]{assets/figures/alpha_0.01_ez_0.98/aPlusb_vs_Theta_C1_contours}
  \par\end{centering}
 \caption{Upper panel: the potential $V$ (Eq. \ref{eq:Potential}) in blue
 and the constant energy $E$ (Eq. \ref{eq:energy}) in black for the
 two optional shapes dependent on the constants $C_{1}$ (Eq. \ref{eq:C1})
 and $C_{2}$ (Eq. \ref{eq:C2}). Lower panel: Trajectories in the
 $a+b$ vs. $\theta+\phi$ plane for different values of $C_{1}$ at
 some $C_{2}$. Dashed black line mark the minimal value of $C_{1}$
 for rotations. Red dashed lines mark the value of $C_{1}$ of the
 potentials in the upper panels. The left plots show a case where $C_{1}<C_{1}^{\text{crit}}$
 and $\theta+\phi$ is librating. The right plots are the case that
 is shown in Fig. \ref{fig:delta_and_jz_one_minus_e_alpha_0.01_ez_0.98}
 and show a case where $C_{1}>C_{1}^{\text{crit}}$ and $\theta+\phi$
 is rotating. Red circles (in all panels) mark the extremums of $\delta$
 (which are also $a+b$ extremums, see Eq. \ref{eq:C2}).\label{fig:potential_energy_and_libration_rotation_maps}}
\end{figure}

The slow angle $\left(\theta+\phi\right)$ can either librate or rotate depending on the constants of motion $C_1$ and $C_2$. Examples of trajectories of both cases are plotted as equi-$C_1$ curves in the bottom panels of Fig. \ref{fig:potential_energy_and_libration_rotation_maps}. For rotations, $\cos\left(\theta+\phi\right)$ must reach both $1$ and $-1$. Using Eqs. \ref{eq:C1}-\ref{eq:C2}, we have 
\begin{equation}
    \cos\left(\theta+\phi\right)=\frac{1}{\left(a+b\right)}\left(\frac{2}{45}C_{1}-\frac{1}{12}\left(C_{2}-\left(a+b\right)^{2}\right)^{2}\right).\label{eq:cosThetaPlusPhi_vs_aPlusb}
\end{equation}
Given $C_{1}$ and $C_{2}$ the rhs. of Eq. \ref{eq:cosThetaPlusPhi_vs_aPlusb} has a global maximal value (in the $\left(a+b\right)>0$ regime) denoted $M\left(C_1, C_2\right)$. If $M\left(C_1, C_2\right)<-1$ - Eq. \ref{eq:cosThetaPlusPhi_vs_aPlusb} cannot be satisfied for any $\left(\theta+\phi\right)$ and so the pair $\left(C_1, C_2\right)$ do not represent any set of initial conditions. If $M\left(C_1, C_2\right)<1$ the slow angle $\left(\theta+\phi\right)$ is librating. If $M\left(C_1, C_2\right)>1$ both $\cos\left(\theta+\phi\right)=-1$ and $\cos\left(\theta+\phi\right)=1$ can be reached (because at $\left(a+b\right)\rightarrow\infty$ the rhs. of Eq. \ref{eq:cosThetaPlusPhi_vs_aPlusb} approaches $-\infty$) and $\left(\theta+\phi\right)$ is rotating \footnote{For analysis of Eq. \ref{eq:cosThetaPlusPhi_vs_aPlusb} we equip the reader with a
 link to a visualization: \url{https://www.desmos.com/calculator/wiitg5elt6}}. Since the rhs. of Eq. \ref{eq:cosThetaPlusPhi_vs_aPlusb} is monotonically increasing with $C_1$, for each $C_2$ there is therefore a minimal permitted $C_1$ and a higher minimal $C_1$ above which $\left(\theta+\phi\right)$ is rotating. The latter threshold is shown as a black dashed curve in the lower panels of Fig. \ref{fig:potential_energy_and_libration_rotation_maps}.

For the regime we solve, $e^2_z$ close to $1$, $C_k<0$ and the minimum and maximum values of the eccentricity during any such Kozai cycle are obtained at $\omega=\pm\frac{\pi}{2}$. As a result, these can be calculated using the constants $j_z$ and $C_k$ through
\begin{equation}
 3e^4_{extremum}+\left(5j^2_z-3+2C_k\right)e^2_{extremum}-2C_k=0.
\end{equation}
The long-term evolution of $j_z$ is obtained via Eq. \ref{eq:C4} and the evolution of $C_k$ follows 
\begin{equation}
 C_{k}=-\frac{3}{2}+\frac{1}{2}\alpha^{\frac{2}{3}}\left(\frac{1}{2}\left(C_{2}+C_{3}\right)-\sqrt{\frac{5}{3}}\delta\right).\label{eq:Ck}
\end{equation}
The extremal values of the eccentricity obtained from $j_z$ and $C_k$ are plotted (on a semi-log $1-e$ plot) as red dashed curves in the upper panel of Fig. \ref{fig:delta_and_jz_one_minus_e_alpha_0.01_ez_0.98}. As can be seen, the analytical solution captures the long term evolution of the oscillations to an excellent approximation compared to the numerical integration of Eqs. \ref{eq:secular_equations}.
\paragraph{Discussion}

Although the analytical model presented in this letter is directly applicable only to a small region of the parameter space (i.e test particle and small perturbation) and only for the final stages of the evolution (i.e at high eccentricity) - it serves as a basis for understanding the more complex phenomena, when the two bodies have comparable mass, and hints for the evolution farther from resonance (i.e starting with low eccentricity).

In the future, we plan to explore the validity and relevance of this model to the different astrophysical phenomena involving KLCs. In addition, we plan to relax some of the assumptions especially starting with low eccentricity or relaxing the test particle assumption in order to expand the range of parameters for which this analysis is applicable. 

We thank Ido Barth for a useful discussion pointing the connection to coordinate moving in a potential well.

\bibliography{precessing_potential_KLC}% Produces the bibliography via BibTeX.

\end{document}

