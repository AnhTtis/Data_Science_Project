%% Beginning of file 'sample631.tex'
%%
%% Modified 2022 May  
%%
%% This is a sample manuscript marked up using the
%% AASTeX v6.31 LaTeX 2e macros.
%%
%% AASTeX is now based on Alexey Vikhlinin's emulateapj.cls 
%% (Copyright 2000-2015).  See the classfile for details.

%% AASTeX requires revtex4-1.cls and other external packages such as
%% latexsym, graphicx, amssymb, longtable, and epsf.  Note that as of 
%% Oct 2020, APS now uses revtex4.2e for its journals but remember that 
%% AASTeX v6+ still uses v4.1. All of these external packages should 
%% already be present in the modern TeX distributions but not always.
%% For example, revtex4.1 seems to be missing in the linux version of
%% TexLive 2020. One should be able to get all packages from www.ctan.org.
%% In particular, revtex v4.1 can be found at 
%% https://www.ctan.org/pkg/revtex4-1.

%% The first piece of markup in an AASTeX v6.x document is the \documentclass
%% command. LaTeX will ignore any data that comes before this command. The 
%% documentclass can take an optional argument to modify the output style.
%% The command below calls the preprint style which will produce a tightly 
%% typeset, one-column, single-spaced document.  It is the default and thus
%% does not need to be explicitly stated.
%%
%% using aastex version 6.3
\documentclass[twocolumn, longauthor]{aastex631}

%% The default is a single spaced, 10 point font, single spaced article.
%% There are 5 other style options available via an optional argument. They
%% can be invoked like this:
%%
%% \documentclass[arguments]{aastex631}
%% 
%% where the layout options are:
%%
%%  twocolumn   : two text columns, 10 point font, single spaced article.
%%                This is the most compact and represent the final published
%%                derived PDF copy of the accepted manuscript from the publisher
%%  manuscript  : one text column, 12 point font, double spaced article.
%%  preprint    : one text column, 12 point font, single spaced article.  
%%  preprint2   : two text columns, 12 point font, single spaced article.
%%  modern      : a stylish, single text column, 12 point font, article with
%% 		  wider left and right margins. This uses the Daniel
%% 		  Foreman-Mackey and David Hogg design.
%%  RNAAS       : Supresses an abstract. Originally for RNAAS manuscripts 
%%                but now that abstracts are required this is obsolete for
%%                AAS Journals. Authors might need it for other reasons. DO NOT
%%                use \begin{abstract} and \end{abstract} with this style.
%%
%% Note that you can submit to the AAS Journals in any of these 6 styles.
%%
%% There are other optional arguments one can invoke to allow other stylistic
%% actions. The available options are:
%%
%%   astrosymb    : Loads Astrosymb font and define \astrocommands. 
%%   tighten      : Makes baselineskip slightly smaller, only works with 
%%                  the twocolumn substyle.
%%   times        : uses times font instead of the default
%%   linenumbers  : turn on lineno package.
%%   trackchanges : required to see the revision mark up and print its output
%%   longauthor   : Do not use the more compressed footnote style (default) for 
%%                  the author/collaboration/affiliations. Instead print all
%%                  affiliation information after each name. Creates a much 
%%                  longer author list but may be desirable for short 
%%                  author papers.
%% twocolappendix : make 2 column appendix.
%%   anonymous    : Do not show the authors, affiliations and acknowledgments 
%%                  for dual anonymous review.
%%
%% these can be used in any combination, e.g.
%%
%% \documentclass[twocolumn,linenumbers,trackchanges]{aastex631}
%%
%% AASTeX v6.* now includes \hyperref support. While we have built in specific
%% defaults into the classfile you can manually override them with the
%% \hypersetup command. For example,
%%
%% \hypersetup{linkcolor=red,citecolor=green,filecolor=cyan,urlcolor=magenta}
%%
%% will change the color of the internal links to red, the links to the
%% bibliography to green, the file links to cyan, and the external links to
%% magenta. Additional information on \hyperref options can be found here:
%% https://www.tug.org/applications/hyperref/manual.html#x1-40003
%%
%% Note that in v6.3 "bookmarks" has been changed to "true" in hyperref
%% to improve the accessibility of the compiled pdf file.
%%
%% If you want to create your own macros, you can do so
%% using \newcommand. Your macros should appear before
%% the \begin{document} command.
%%
\newcommand{\vdag}{(v)^\dagger}
\newcommand\aastex{AAS\TeX}
\newcommand\latex{La\TeX}

%% Reintroduced the \received and \accepted commands from AASTeX v5.2
%\received{March 1, 2021}
%\revised{April 1, 2021}
%\accepted{\today}

%% Command to document which AAS Journal the manuscript was submitted to.
%% Adds "Submitted to " the argument.
%\submitjournal{PSJ}

%% For manuscript that include authors in collaborations, AASTeX v6.31
%% builds on the \collaboration command to allow greater freedom to 
%% keep the traditional author+affiliation information but only show
%% subsets. The \collaboration command now must appear AFTER the group
%% of authors in the collaboration and it takes TWO arguments. The last
%% is still the collaboration identifier. The text given in this
%% argument is what will be shown in the manuscript. The first argument
%% is the number of author above the \collaboration command to show with
%% the collaboration text. If there are authors that are not part of any
%% collaboration the \nocollaboration command is used. This command takes
%% one argument which is also the number of authors above to show. A
%% dashed line is shown to indicate no collaboration. This example manuscript
%% shows how these commands work to display specific set of authors 
%% on the front page.
%%
%% For manuscript without any need to use \collaboration the 
%% \AuthorCollaborationLimit command from v6.2 can still be used to 
%% show a subset of authors.
%
%\AuthorCollaborationLimit=2
%
%% will only show Schwarz & Muench on the front page of the manuscript
%% (assuming the \collaboration and \nocollaboration commands are
%% commented out).
%%
%% Note that all of the author will be shown in the published article.
%% This feature is meant to be used prior to acceptance to make the
%% front end of a long author article more manageable. Please do not use
%% this functionality for manuscripts with less than 20 authors. Conversely,
%% please do use this when the number of authors exceeds 40.
%%
%% Use \allauthors at the manuscript end to show the full author list.
%% This command should only be used with \AuthorCollaborationLimit is used.

%% The following command can be used to set the latex table counters.  It
%% is needed in this document because it uses a mix of latex tabular and
%% AASTeX deluxetables.  In general it should not be needed.
%\setcounter{table}{1}

%%%%%%%%%%%%%%%%%%%%%%%%%%%%%%%%%%%%%%%%%%%%%%%%%%%%%%%%%%%%%%%%%%%%%%%%%%%%%%%%
%%
%% The following section outlines numerous optional output that
%% can be displayed in the front matter or as running meta-data.
%%
%% If you wish, you may supply running head information, although
%% this information may be modified by the editorial offices.
%\shorttitle{AASTeX v6.3.1 Sample article}
%\shortauthors{Schwarz et al.}
%%
%% You can add a light gray and diagonal water-mark to the first page 
%% with this command:
%% \watermark{text}
%% where "text", e.g. DRAFT, is the text to appear.  If the text is 
%% long you can control the water-mark size with:
%% \setwatermarkfontsize{dimension}
%% where dimension is any recognized LaTeX dimension, e.g. pt, in, etc.
%%
%%%%%%%%%%%%%%%%%%%%%%%%%%%%%%%%%%%%%%%%%%%%%%%%%%%%%%%%%%%%%%%%%%%%%%%%%%%%%%%%
%\graphicspath{{./}{figures/}}
%% This is the end of the preamble.  Indicate the beginning of the
\usepackage{xcolor}
\usepackage{comment}
\usepackage{graphicx}% Include figure files
\usepackage{dcolumn}% Align table columns on decimal point
\usepackage{bm}% bold math
\usepackage{hyperref}% add hypertext capabilities
\DeclareUnicodeCharacter{2009}{\,}

%% manuscript itself with \begin{document}.

\begin{document}

\title{Analytic understanding of the resonant nature of Kozai Lidov Cycles
 with a precessing quadrupole potential}

%% LaTeX will automatically break titles if they run longer than
%% one line. However, you may use \\ to force a line break if
%% you desire. In v6.31 you can include a footnote in the title.

%% A significant change from earlier AASTEX versions is in the structure for 
%% calling author and affiliations. The change was necessary to implement 
%% auto-indexing of affiliations which prior was a manual process that could 
%% easily be tedious in large author manuscripts.
%%
%% The \author command is the same as before except it now takes an optional
%% argument which is the 16 digit ORCID. The syntax is:
%% \author[xxxx-xxxx-xxxx-xxxx]{Author Name}
%%
%% This will hyperlink the author name to the author's ORCID page. Note that
%% during compilation, LaTeX will do some limited checking of the format of
%% the ID to make sure it is valid. If the "orcid-ID.png" image file is 
%% present or in the LaTeX pathway, the OrcID icon will appear next to
%% the authors name.
%%
%% Use \affiliation for affiliation information. The old \affil is now aliased
%% to \affiliation. AASTeX v6.31 will automatically index these in the header.
%% When a duplicate is found its index will be the same as its previous entry.
%%
%% Note that \altaffilmark and \altaffiltext have been removed and thus 
%% can not be used to document secondary affiliations. If they are used latex
%% will issue a specific error message and quit. Please use multiple 
%% \affiliation calls for to document more than one affiliation.
%%
%% The new \altaffiliation can be used to indicate some secondary information
%% such as fellowships. This command produces a non-numeric footnote that is
%% set away from the numeric \affiliation footnotes.  NOTE that if an
%% \altaffiliation command is used it must come BEFORE the \affiliation call,
%% right after the \author command, in order to place the footnotes in
%% the proper location.
%%
%% Use \email to set provide email addresses. Each \email will appear on its
%% own line so you can put multiple email address in one \email call. A new
%% \correspondingauthor command is available in V6.31 to identify the
%% corresponding author of the manuscript. It is the author's responsibility
%% to make sure this name is also in the author list.
%%
%% While authors can be grouped inside the same \author and \affiliation
%% commands it is better to have a single author for each. This allows for
%% one to exploit all the new benefits and should make book-keeping easier.
%%
%% If done correctly the peer review system will be able to
%% automatically put the author and affiliation information from the manuscript
%% and save the corresponding author the trouble of entering it by hand.

\correspondingauthor{Ygal Y. Klein}
\email{ygalklein@gmail.com}

\author[0009-0004-1914-5821]{Ygal Y. Klein}
\affiliation{Dept. of Particle Phys. \& Astrophys., Weizmann Institute of Science,
 Rehovot 76100, Israel}

\author[0000-0003-0584-2920
]{Boaz Katz}
\affiliation{Dept. of Particle Phys. \& Astrophys., Weizmann Institute of Science,
 Rehovot 76100, Israel}

% \collaboration{20}{(AAS Journals Data Editors)}

% \author{F.X Timmes}
% \affiliation{Arizona State University}
% \affiliation{AAS Journals Associate Editor-in-Chief}

% \author{Amy Hendrickson}
% \altaffiliation{AASTeX v6+ programmer}
% \affiliation{TeXnology Inc.}

% \author{Julie Steffen}
% \affiliation{AAS Director of Publishing}
% \affiliation{American Astronomical Society \\
% 1667 K Street NW, Suite 800 \\
% Washington, DC 20006, USA}

%% Note that the \and command from previous versions of AASTeX is now
%% depreciated in this version as it is no longer necessary. AASTeX 
%% automatically takes care of all commas and "and"s between authors names.

%% AASTeX 6.31 has the new \collaboration and \nocollaboration commands to
%% provide the collaboration status of a group of authors. These commands 
%% can be used either before or after the list of corresponding authors. The
%% argument for \collaboration is the collaboration identifier. Authors are
%% encouraged to surround collaboration identifiers with ()s. The 
%% \nocollaboration command takes no argument and exists to indicate that
%% the nearby authors are not part of surrounding collaborations.

%% Mark off the abstract in the ``abstract'' environment. 
\begin{abstract}

 The very long-term evolution of the hierarchical restricted three-body problem with a slightly aligned precessing quadrupole potential is studied analytically. This problem describes the evolution of a star and a planet which are perturbed either by a (circular and not too inclined) binary star system or by one other star and a second more distant star, as well as a perturbation by one distant star and the host galaxy or a compact-object binary system orbiting a massive black hole in non-spherical nuclear star clusters \citep{hamers2017,petrovich2017}. Previous numerical experiments have shown that when the precession frequency is comparable to the Kozai-Lidov time scale, long term evolution emerges that involves extremely high eccentricities with potential applications for a broad scope of astrophysical phenomena including systems with merging black holes, neutron stars or white dwarfs. By averaging the secular equations of motion over the Kozai-Lidov Cycles (KLCs) we solve the problem analytically in the neighborhood of the KLC fixed point where the eccentricity vector is close to unity and aligned with the quadrupole axis and for a precession rate similar to the Kozai Lidov time scale. In this regime the dynamics is dominated by a resonance between the perturbation frequency and the precession frequency of the eccentricity vector. While the quantitative evolution of the system is not reproduced by the solution far away from this fixed point, it sheds light on the qualitative behaviour.

\end{abstract}

%% Keywords should appear after the \end{abstract} command. 
%% The AAS Journals now uses Unified Astronomy Thesaurus concepts:
%% https://astrothesaurus.org
%% You will be asked to selected these concepts during the submission process
%% but this old "keyword" functionality is maintained in case authors want
%% to include these concepts in their preprints.
% \keywords{Classical Novae (251) --- Ultraviolet astronomy(1736) --- History of astronomy(1868) --- Interdisciplinary astronomy(804)}

%% From the front matter, we move on to the body of the paper.
%% Sections are demarcated by \section and \subsection, respectively.
%% Observe the use of the LaTeX \label
%% command after the \subsection to give a symbolic KEY to the
%% subsection for cross-referencing in a \ref command.
%% You can use LaTeX's \ref and \label commands to keep track of
%% cross-references to sections, equations, tables, and figures.
%% That way, if you change the order of any elements, LaTeX will
%% automatically renumber them.
%%
%% We recommend that authors also use the natbib \citep
%% and \citet commands to identify citations.  The citations are
%% tied to the reference list via symbolic KEYs. The KEY corresponds
%% to the KEY in the \bibitem in the reference list below. 

\section{Introduction} \label{sec:intro}

In this letter we study analytically the dynamics of a test particle
orbiting a central mass $M$ on a Keplerian orbit with semimajor
axis $a$ which is perturbed by an external
quadrupole potential given by: 
\begin{equation}
\Phi_{outer}=\frac{\Phi_0}{a^2}\left[3\left(\mathbf{\hat{j}}_{outer}\cdot\mathbf{r}\right)^{2}-r^{2}\right]\label{eq:potential}
\end{equation}
where $\Phi_0$ is constant. In the periodic analytically solved
Kozai-Lidov cycles (KLCs) \citep{lidov1962,kozai62} the external quadrupole potential is constant in time (i.e 
 $\mathbf{\hat{j}}_{outer}$ is a constant unit vector) (for a recent review on KLCs see \citep{naoz2016}). We study the case where the quadrupole potential is time dependent and $\mathbf{\hat{j}}_{outer}$ is a unit vector which precesses around the $z$ axis at a constant rate $\beta$ with a constant inclination $\alpha$:
\begin{equation}
 \mathbf{\hat{j}}_{outer}=\left(\begin{array}{c}
   \sin\alpha\cos\left(\beta\tau\right) \\
   -\sin\alpha\sin\left(\beta\tau\right) \\   
   \cos\alpha
  \end{array}\right)\label{eq:jOuter_as_a_function_of_tau}
\end{equation}
where $\tau\equiv\frac{t}{t_{sec}}$ and $t_{sec}=\frac{\sqrt{GMa}}{\Phi_0}$ is the secular timescale.

This problem describes the evolution of a star and a planet which are perturbed either by a (circular and not too inclined) binary star system or by one other star and a second more distant star \citep{hamers2017}, as well as a perturbation by one distant star and the host galaxy or a compact-object binary system orbiting a massive black hole in non-spherical nuclear star clusters \citep{petrovich2017}. Previous numerical experiments have shown that when the precession frequency is comparable to the Kozai-Lidov time scale, long term evolution emerges that involves extremely high eccentricities \citep{hamers2017} with potential applications for the formation of planets around white dwarfs \citep{munoz20,oconnor21,stephan21} and hot planets \citep{fabrycky2007,katz2011,naoz2011,grishin18}. If the test particle assumption is relaxed, the system exhibits similar dynamics and the description is applicable to a broader scope of astrophysical phenomena, including Type Ia supernovae through the merger or collision of white dwarfs in multiple systems \citep{thompson2011,katz2012,pejcha2013,fang2018,grishin22}, gravitational wave emission through the merger of black holes or neutron stars in quadruple systems \citep{liu2019,safarzadeh2020,hamers2020} and the formation of close binaries \citep{antonini2012,petrovich2017,bub2020,grishin22}.

As mentioned, the case of $\alpha=0$ is the periodic analytically solved
Kozai-Lidov cycles (KLCs) \citep{lidov1962,kozai62}.

\section{Equations of motion} \label{sec:equations_of_motion}

The dynamics of the test particle can be parameterized
by two dimensionless orthogonal vectors $\mathbf{j}=\mathbf{J}/\sqrt{GMa}$, where $\mathbf{J}$ is the specific angular momentum vector, and
$\mathbf{e}$ a vector pointing in the direction of the pericenter
with magnitude $e$. In the secular approximation, $a$ is constant with time while $\mathbf{j}$ and
$\mathbf{e}$ evolve according to the Kozai-Lidov equations (as Eq. 10a-b in \citep{hamers2017})
\begin{eqnarray}
\frac{d\mathbf{j}}{d\tau}=&\frac{3}{4}\left(\left(\mathbf{j}\cdot\mathbf{\hat{j}}_{outer}\right)\mathbf{j}-5\left(\mathbf{e}\cdot\mathbf{\hat{j}}_{outer}\right)\mathbf{e}\right)\times\mathbf{\hat{j}}_{outer} \cr
\frac{d\mathbf{e}}{d\tau}=&\frac{3}{2}\left(\mathbf{j}\times\mathbf{e}\right)-\frac{3}{4}\left(5\left(\mathbf{e}\cdot\mathbf{\hat{j}}_{outer}\right)\mathbf{j}-\left(\mathbf{j}\cdot\mathbf{\hat{j}}_{outer}\right)\mathbf{e}\right)\times\mathbf{\hat{j}}_{outer} \cr
 \label{eq:secular_equations}
\end{eqnarray}
with $\mathbf{\hat{j}}_{outer}$ given by Eq. \ref{eq:jOuter_as_a_function_of_tau} (itself a solution of Eq. 10c in \citep{hamers2017}). A numerical integration of Eqs. \ref{eq:secular_equations} is shown as blue lines in the top two panels of Fig. \ref{fig:delta_and_jz_one_minus_e_alpha_0.01_ez_0.98} for $\alpha=0.01^{\circ}$ and $\beta\approx2.9$ (left panel) and $\alpha=5^{\circ}$ and $\beta=2.5$ (right panel). We remind that in the $\alpha=0$ case (pure KLCs), $j_z$ (middle panel of Fig. \ref{fig:delta_and_jz_one_minus_e_alpha_0.01_ez_0.98}) is constant and $e$ (top panel of Fig. \ref{fig:delta_and_jz_one_minus_e_alpha_0.01_ez_0.98}) is periodically oscillating but with a constant $e_{max}$ (which for $e_0\ll1$ can be approximated with $e_{max}\approx\sqrt{1-\frac{5}{3}j^2_z}$). As can be seen in the top and middle panels of Fig.  \ref{fig:delta_and_jz_one_minus_e_alpha_0.01_ez_0.98} the times of zero crossing of $j_z$ correspond to the times of extremely high eccentricities, as expected from KLCs.

\begin{figure}
 \begin{centering}
 \includegraphics[scale=0.21]{alpha_0.01_beta_beta0_ez_0.98_one_minus_e.png}\includegraphics[scale=0.21]{alpha_5_beta_2_5_slow_as_beta0_ez_0.83_one_minus_e_broken_line.png}
  \par\end{centering}
 \begin{centering}
  \includegraphics[scale=0.2]{alpha_0.01_beta_beta0_ez_0.98_jz_zoomed.png}\includegraphics[scale=0.2]{alpha_5_beta_2_5_slow_as_beta0_ez_0.83_jz_zoomed.png}
  \par\end{centering}
 \begin{centering}
  \includegraphics[scale=0.21]{alpha_0.01_beta_beta0_ez_0.98_delta.png}\includegraphics[scale=0.21]{alpha_5_beta_2_5_slow_as_beta0_ez_0.83_delta.png}
  \par\end{centering}
 \caption{Results of numerical integrations for: Left panel: $\alpha=0.01^{\circ}$ and $\beta\approx2.9$,
 with initial conditions $e_{x}=j_{x}=-j_{y}=10^{-5},e_{z}=0.98$ and right panel: $\alpha=5^{\circ}$ and $\beta=2.5$,
 with initial conditions $e_{x}\sim-10^{-5}, j_{x}\sim-0.28,j_{y}\sim-0.186,e_{z}\sim0.825$.
 The blue solid lines are the result of the integration of the full
 secular equations, Eqs. \ref{eq:secular_equations} (with \ref{eq:jOuter_as_a_function_of_tau}),
 while the red dashed lines are the result of the averaged equations,
 Eqs. \ref{eq:splusDot}-\ref{eq:a_plus_b_dot}, and using Eq. \ref{eq:polynom_for_extremal_eccentricity} (where it has real roots) to determine $e_{min}$ and $e_{max}$  using Eqs. \ref{eq:C4}
 and \ref{eq:Ck}. The two green horizontal lines in the bottom panel
 represent the extremum values of $\delta$ as determined from initial
 conditions and in the middle panel the maximal and minimal values
 of $j_{z}$ as determined from initial conditions using Eq. \ref{eq:C4}
 and the extremums of $\delta$. $\hat{\tau}$ is defined in Eq. \ref{eq:tau_hat}.\label{fig:delta_and_jz_one_minus_e_alpha_0.01_ez_0.98}}
\end{figure}

We restrict the analysis to the regime where $\alpha\ll1$ (i.e $\alpha$ being a
small parameter around which $\alpha=0$ is already analytically solved)
and $\left|\mathbf{e}\cdot\mathbf{\hat{j}}_{outer}\right|\sim1$ (i.e
$\mathbf{j}\cdot\mathbf{\hat{j}}_{outer} \approx j_z \ll1$, $e\sim1$ and inclination close to $90^\circ$, which is close to the KLC fixed point of $e=1$, $i=90^\circ$ and $j=0$). In this regime, the eccentricity vector precesses around the $z$ axis. When the frequencies of the precession of $\mathbf{e}$ and $\mathbf{\hat{j}}_{outer}$ are far from each other - the precession of the quadrupole potential has a minor effect on the KLCs. On the other hand, when these two frequencies are close, long-term resonant dynamics are obtained and are the focus of this letter.

\section{Approximated Equations} \label{sec:Approximated Equations}

In this regime and up to first order in $\alpha$ one obtains the
following 6 equations (neglecting $j_{z}$ in this regime in the rhs of the
derivatives of $e_x$, $e_y$ and $e_z$):
\begin{eqnarray}
 \frac{d}{d\tau}j_{z}&=\frac{15}{4}e_{z}\alpha\left(e_{x}\sin\left(\beta\tau\right)+e_{y}\cos\left(\beta\tau\right)\right)\label{eq:jzdot} \\
 \frac{d}{d\tau}e_{z}&=\frac{3}{4}\left(2\left(j_{x}e_{y}-j_{y}e_{x}\right)+5e_{z}\alpha\left(j_{x}\sin\left(\beta\tau\right)+j_{y}\cos\left(\beta\tau\right)\right)\right)\label{eq:ezdot}
\end{eqnarray}
\begin{eqnarray}
 \frac{d}{d\tau}e_{x} & =-\frac{9}{4}e_{z}j_{y}\label{eq:exdot} \\
 \frac{d}{d\tau}e_{y} & =+\frac{9}{4}e_{z}j_{x}\label{eq:eydot}
\end{eqnarray}
\begin{eqnarray}
 \frac{d}{d\tau}j_{x} & =-\frac{15}{4}e_{z}\left(e_{y}+e_{z}\alpha\sin\left(\beta\tau\right)\right)\label{eq:jxdot} \\
 \frac{d}{d\tau}j_{y} & =+\frac{15}{4}e_{z}\left(e_{x}-e_{z}\alpha\cos\left(\beta\tau\right)\right)\label{eq:jydot}
\end{eqnarray}
In the lowest order approximation,
$\frac{d}{d\tau}e_{z}=0$, resulting with a forced harmonic
oscillator for the vector $\mathbf{e}$ in the $x-y$ plane with $\ddot{e}_{x}=\omega_{0}^{2}\left(L\cos\left(\omega\tau\right)-e_{x}\right)$
where $\omega=\beta,L=e_{z}\alpha$ and $\omega_{0}=\sqrt{\frac{135}{16}}e_{z}$.
Below we solve the next level of approximation where $e_{z}$ is slowly changing.

\section{Averaged Equations\label{sec:Averaged Equations}}

Since $\alpha$ is small the dynamics on short time scales follow
the known (test particle triple system) Kozai-Lidov Cycles, which
have two constants of motion: $j_{z}$ and
\begin{equation}
C_{K}=e^{2}-\frac{5}{2}e_{z}^{2}=e^{2}\left(1-\frac{5}{2}\sin^{2}i\sin^{2}\omega\right).
\end{equation}
On longer time scales the parameters of the KLC, $j_{z}$ and $C_{K}$,
evolve.

Consider the following ansatz for the vector $\mathbf{e}$
in the $x-y$ plane: At any time $\tau$, the projection of the vector
$\mathbf{e}$ on the $x-y$ plane can be presented as a point moving
on a slowly evolving ellipse with semimajor axis $a$ inclined with an angle $\theta$ with respect to the $x$ axis and semiminor axis $b$ centered at the origin, i.e
\begin{equation}
 \mathbf{e}_{x-y}=\alpha^{\frac{1}{3}}\left(\begin{array}{cc}
   \cos\theta, & -\sin\theta \\
   \sin\theta, & \cos\theta
  \end{array}\right)\left(\begin{array}{c}
   a\cos\left(\hat{\beta}\hat{\tau}+\phi\right) \\
   b\sin\left(\hat{\beta}\hat{\tau}+\phi\right)
  \end{array}\right)\label{eq:exy-ansatz}
\end{equation}
where $\phi$ is a slowly dynamically evolving phase and
\begin{equation}
 \hat{\tau}=\frac{1}{2}\alpha^{\frac{2}{3}}\tau
 \label{eq:tau_hat}
\end{equation}
and
\begin{equation}
 \hat{\beta}=2\alpha^{-\frac{2}{3}}\beta.
 \label{eq:beta_hat}
\end{equation}
See note after Eq. \ref{eq:a_plus_b_dot} regarding the choice of normalization prefactors: $\alpha^{\frac{1}{3}},\alpha^{\frac{2}{3}}$ and $\alpha^{-\frac{2}{3}}$.
The ansatz in Eq. \ref{eq:exy-ansatz} has a symmetry under the following transformation (both changes together)
\begin{eqnarray}
% \begin{cases}
% \begin{array}{c}
\left(a-b\right)\rightarrow-\left(a-b\right)\nonumber\\
\left(\theta-\phi\right)\rightarrow\left(\theta-\phi+\pi\right)\nonumber
% \end{array}
% \end{cases}
\end{eqnarray}
meaning that without loss of generality $\left(a-b\right)$ is non negative.

Using Eqs. \ref{eq:exdot}-\ref{eq:eydot} in the limit $e_{z}=1$ and neglecting the time derivatives of the slowly varying functions, the projection of the angular momentum on the $x-y$ plane is correspondingly given by
\begin{equation}
 \mathbf{j}_{x-y}=\frac{4}{9}\alpha^{\frac{1}{3}}\beta\left(\begin{array}{cc}
   \cos\theta, & -\sin\theta \\
   \sin\theta, & \cos\theta
  \end{array}\right)\left(\begin{array}{c}
   b\cos\left(\hat{\beta}\hat{\tau}+\phi\right) \\
   a\sin\left(\hat{\beta}\hat{\tau}+\phi\right)
  \end{array}\right).\label{eq:jxy}
\end{equation}
Note the ansatz includes four slowly evolving variables, $a,b,\theta,\phi$, which describe the averaged evolution of the four components $e_x,e_y,j_x,j_y$.

Since the frequency of the precession of $\mathbf{\hat{j}}_{outer}$ is $\beta$ and the driving frequency of the Kozai oscillations is $\sqrt{\frac{135}{16}}e_{z}$ a resonance is obtained between the two perturbations when the two frequencies approach each other and it is useful to quantify the distance from resonance by a dynamical parameter,
\begin{equation}
 \delta=\alpha^{-\frac{2}{3}}\frac{1}{\beta_{0}}\left(\left(\beta_{0}\bar{e}_{z}\right)^{2}-\beta^{2}\right)\label{eq:delta}
\end{equation}
where $\bar{e}_{z}$ is the averaged value of $e_{z}$ over KLC which satisfies (using Eq. \ref{eq:ezdot})
\begin{equation}
 e_{z}=\bar{e}_{z}+\frac{\alpha^{\frac{2}{3}}}{6}\left(a^{2}-b^{2}\right)\cos\left(2\left(\hat{\beta}\hat{\tau}+\phi\right)\right)\label{eq:instantanous_ez}
\end{equation}
and
\begin{equation}
 \beta_{0}=\sqrt{\frac{135}{16}}\approx2.9\label{eq:beta0}.
\end{equation}
Using the following slow variables 
\begin{eqnarray}
 s=-\frac{45}{2}\left(a-b\right)\sin\left(\theta-\phi\right)\\
 c=-\frac{45}{2}\left(a-b\right)\cos\left(\theta-\phi\right)
\end{eqnarray}
and focusing on the resonant limit of $\omega=\omega_{0}$ in the forced harmonic oscillator mentioned above, i.e $\beta=\beta_{0}$, the following set of ODEs is obtained:
\begin{eqnarray}
 \dot{\delta}=&s\label{eq:deltaDot}\\
 \dot{s}=&-\left(45\beta_{0}+\delta c\right)\label{eq:splusDot}\\
 \dot{c}=&\delta s\label{eq:cPlusDot}
\end{eqnarray}
and 
\begin{eqnarray}
 \frac{d\left(\theta+\phi\right)}{d\hat{\tau}} & =\delta\label{eq:theta_minus_phi_dot} \\
 \frac{d\left(a+b\right)}{d\hat{\tau}}     & =0\label{eq:a_plus_b_dot},
\end{eqnarray}
where $\dot{}$ denotes a derivative with respect to $\hat{\tau}$.

Several notes are in order: (1) The parameters $s,c,a+b,\theta+\phi$ uniquely determine all the slow variables: $a,b,\theta,\phi$. (2) The evolution of $\delta,s$ and $c$ can be obtained by solving the closed subset of Eqs. \ref{eq:deltaDot}-\ref{eq:cPlusDot}. (3) The equations obtained have no explicit dependence on the small parameter $\alpha$. In fact, the $\alpha$ dependent prefactors in Eqs. \ref{eq:exy-ansatz}-\ref{eq:delta} were chosen for this reason. (4) $j_{z}$ can be obtained from the demand that $\mathbf{j}\cdot\mathbf{e}=0$. In fact, the following combination of $j_z$ and $\delta$ is constant:
\begin{equation}
j_z+\frac{\delta}{6}\alpha^{\frac{2}{3}}=\text{const.},\label{eq:C4}
\end{equation}
allowing $j_z$ to be readily obtained using the initial conditions and the time evolution of $\delta$.
The resulting slow evolution of $\delta$ for the examples in Fig. \ref{fig:delta_and_jz_one_minus_e_alpha_0.01_ez_0.98} is shown in the bottom panel and is used for the solution of $j_z$ plotted as a dashed red line in the middle panel. As can be seen in the left panel, the slow evolution of $j_z$ agrees to an excellent approximation with the numerical result.

\section{Analytic Solution\label{sec:Analytic-Solution}}

The averaged equations, Eqs. \ref{eq:deltaDot}-\ref{eq:cPlusDot}, admit two constants of the motion, denoted $C_{1}$ and $C_{2}$,
\begin{eqnarray}
 C_{1}=&\frac{1}{2}\left(\delta^{2}+45\left(a-b\right)\cos\left(\theta-\phi\right)\right)\label{eq:C1}\\
 C_{2}=&\left(a-b\right)^{2}+\frac{2\delta}{\sqrt{15}}\label{eq:C2}
\end{eqnarray}
implying that the evolution of the three variables $\delta,a-b$ and $\theta-\phi$ is periodic.
Using Eq. \ref{eq:a_plus_b_dot} we define a third constant
\begin{equation}
 C_{3}=\left(a+b\right)^{2}\label{eq:C3}
\end{equation}
which together with $C_1$ and $C_2$ determine the long term
evolution of the entire system.

The resulting evolution of $\delta$ is equivalent to the dynamics of a particle moving in a one dimensional potential with a constant energy 
\begin{equation}
 E=\frac{1}{2}\dot{\delta}^{2}+V=\frac{1}{2}\left(\left(\frac{45}{2}\right)^{2}C_{2}-C_{1}^{2}\right)\label{eq:energy}
\end{equation}
where (using $\beta_{0}=\sqrt{\frac{135}{16}}$,
see Eq. \ref{eq:beta0})
\begin{equation}
 V=45\beta_{0}\delta-\frac{1}{2}C_{1}\delta^{2}+\frac{1}{8}\delta^{4}\label{eq:Potential}.
\end{equation}

This potential has two distinct shapes depending on whether $C_1$ is smaller or larger than the critical value
\begin{equation}
 C_{1}^{\text{crit}}=15\left(\frac{3}{2}\right)^{\frac{7}{3}}.
\end{equation}
If $C_{1}<C_{1}^{\text{crit}}$ the potential has no maxima and one
minimum (see example in the left upper panel of Fig. \ref{fig:potential_energy_and_libration_rotation_maps}).
If $C_{1}>C_{1}^{\text{crit}}$ the potential has a maxima and two minima (see example in the right upper panel of Fig. \ref{fig:potential_energy_and_libration_rotation_maps} showing the case that is solved in the left panel of Fig. \ref{fig:delta_and_jz_one_minus_e_alpha_0.01_ez_0.98}) \footnote{For analysis of the different features we equip the reader with a link to a visualization of Eqs. \ref{eq:Potential}, \ref{eq:energy}: \url{https://www.desmos.com/calculator/ubgicqtddn}}. The extremum values of $\delta$ determined from the potential and energy are marked as red circles in the top panels of Fig. \ref{fig:potential_energy_and_libration_rotation_maps} and are plotted as green lines in the bottom panel of Fig. \ref{fig:delta_and_jz_one_minus_e_alpha_0.01_ez_0.98}. The extremums of $\left(a-b\right)$ are readily given using Eq. \ref{eq:C2} and are marked as red circles in the bottom panels of Fig. \ref{fig:potential_energy_and_libration_rotation_maps}.

\begin{figure}
 \begin{centering}
  \includegraphics[scale=0.22]{C1_2_C2_5_V_and_E_vs_delta.png}\includegraphics[scale=0.22]{alpha_0.001_ez_0.98_V_and_E_vs_delta.png}
  \par\end{centering}
 \begin{centering}
  \includegraphics[scale=0.22]{C1_2_C2_5_aMinusb_vs_Theta_C1_contours.png}\includegraphics[scale=0.22]{alpha_0.001_ez_0.98_aMinusb_vs_Theta_C1_contours.png}
  \par\end{centering}
 \caption{Upper panel: the potential $V$ (Eq. \ref{eq:Potential}) in blue
 and the constant energy $E$ (Eq. \ref{eq:energy}) in black for the
 two optional shapes dependent on the constants $C_{1}$ (Eq. \ref{eq:C1})
 and $C_{2}$ (Eq. \ref{eq:C2}). Lower panel: Trajectories in the
 $a-b$ vs. $\theta-\phi$ plane for different values of $C_{1}$ at
 some $C_{2}$. Dashed black line mark the minimal value of $C_{1}$
 for rotations. Red dashed lines mark the value of $C_{1}$ of the
 potentials in the upper panels. The left plots show a case where $C_{1}<C_{1}^{\text{crit}}$
 and $\theta-\phi$ is librating. The right plots are the case that
 is shown in the left panel of Fig. \ref{fig:delta_and_jz_one_minus_e_alpha_0.01_ez_0.98}
 and show a case where $C_{1}>C_{1}^{\text{crit}}$ and $\theta-\phi$
 is rotating. Red circles (in all panels) mark the extremums of $\delta$
 (which are also $a-b$ extremums, see Eq. \ref{eq:C2}).\label{fig:potential_energy_and_libration_rotation_maps}}
\end{figure}

The slow angle $\left(\theta-\phi\right)$ can either librate or rotate depending on the constants of motion $C_1$ and $C_2$. Examples of trajectories of both cases are plotted as equi-$C_1$ curves in the bottom panels of Fig. \ref{fig:potential_energy_and_libration_rotation_maps}. For rotations, $\cos\left(\theta-\phi\right)$ must reach both $1$ and $-1$. Using Eqs. \ref{eq:C1}-\ref{eq:C2}, we have 
\begin{equation}
    \cos\left(\theta-\phi\right)=\frac{1}{\left(a-b\right)}\left(\frac{2}{45}C_{1}-\frac{1}{12}\left(C_{2}-\left(a-b\right)^{2}\right)^{2}\right).\label{eq:cosThetaMinusPhi_vs_aMinusb}
\end{equation}
Given $C_{1}$ and $C_{2}$ the rhs. of Eq. \ref{eq:cosThetaMinusPhi_vs_aMinusb} has a global maximal value (in the $\left(a-b\right)>0$ regime) denoted $M\left(C_1, C_2\right)$. If $M\left(C_1, C_2\right)<-1$ - Eq. \ref{eq:cosThetaMinusPhi_vs_aMinusb} cannot be satisfied for any $\left(\theta-\phi\right)$ and so the pair $\left(C_1, C_2\right)$ do not represent any set of initial conditions. If $M\left(C_1, C_2\right)<1$ the slow angle $\left(\theta-\phi\right)$ is librating. If $M\left(C_1, C_2\right)>1$ both $\cos\left(\theta-\phi\right)=-1$ and $\cos\left(\theta-\phi\right)=1$ can be reached (because at $\left(a-b\right)\rightarrow\infty$ the rhs. of Eq. \ref{eq:cosThetaMinusPhi_vs_aMinusb} approaches $-\infty$) and $\left(\theta-\phi\right)$ is rotating \footnote{For analysis of Eq. \ref{eq:cosThetaMinusPhi_vs_aMinusb} we equip the reader with a link to a visualization: \url{https://www.desmos.com/calculator/wiitg5elt6}}. Since the rhs. of Eq. \ref{eq:cosThetaMinusPhi_vs_aMinusb} is monotonically increasing with $C_1$, for each $C_2$ there is therefore a minimal permitted $C_1$ and a higher minimal $C_1$ above which $\left(\theta-\phi\right)$ is rotating. The latter threshold is shown as a black dashed curve in the lower panels of Fig. \ref{fig:potential_energy_and_libration_rotation_maps}.

For the regime we solve, $e^2_z$ close to $1$, $C_K<0$ and the minimum and maximum values of the eccentricity during any such Kozai cycle are obtained at $\omega=\pm\frac{\pi}{2}$. As a result, these can be calculated using the constants $j_z$ and $C_K$ through
\begin{equation}
 3e^4_{extremum}+\left(5j^2_z-3+2C_K\right)e^2_{extremum}-2C_K=0\label{eq:polynom_for_extremal_eccentricity}.
\end{equation}
The long-term evolution of $j_z$ is obtained via Eq. \ref{eq:C4} and the evolution of $C_K$ follows 
\begin{equation}
 C_K=-\frac{3}{2}+\frac{1}{2}\alpha^{\frac{2}{3}}\left(\frac{1}{2}\left(C_{2}+C_{3}\right)-\sqrt{\frac{5}{3}}\delta\right).\label{eq:Ck}
\end{equation}
The extremal values of the eccentricity obtained from $j_z$ and $C_K$ are plotted (on a semi-log $1-e$ plot) as red dashed curves in the upper panel of Fig. \ref{fig:delta_and_jz_one_minus_e_alpha_0.01_ez_0.98}. As can be seen in the left panel, the analytical solution captures the long term evolution of the oscillations to an excellent approximation compared to the numerical integration of Eqs. \ref{eq:secular_equations}.

\section{Discussion}

In this letter we provide a concise presentation of the analysis and analytic solution for the dynamics of a test particle in a Keplerian orbit perturbed by a precessing quadrupole potential. We explicitly demonstrate the success of the solution vs. full numerical solution of the secular equations for a case that is very close to the assumptions made, i.e extremely small $\alpha$, $\beta=\beta_0$ and $e_{z}\sim1$.

Exploring the validity of the solution for wider scopes of the parameters is beyond the scope of this work, but as an example we present in the right panel of Fig. \ref{fig:delta_and_jz_one_minus_e_alpha_0.01_ez_0.98} the capability of the analytic solution to describe and approximately reconstruct the long term evolution even for higher values of $\alpha$ ($5^{\circ}$ as is numerically explored in \citep{hamers2017}) and for a value of $\beta$ slightly different from $\beta_0$.

Although the analytical model presented in this letter is directly applicable only to a small region of the parameter space (i.e test particle and small perturbation) and only for the final stages of the evolution (i.e at high eccentricity) - it serves as a basis for understanding the more complex phenomena, when the two bodies have comparable mass, and hints for the evolution farther from resonance (i.e starting with low eccentricity).

In the future, we plan to explore the validity and relevance of this model to the different astrophysical phenomena involving KLCs. In addition, we plan to relax some of the assumptions especially starting with low eccentricity or relaxing the test particle assumption. 

%% IMPORTANT! The old "\acknowledgment" command has be depreciated. It was
%% not robust enough to handle our new dual anonymous review requirements and
%% thus been replaced with the acknowledgment environment. If you try to 
%% compile with \acknowledgment you will get an error print to the screen
%% and in the compiled pdf.
%% 
%% Also note that the akcnowlodgment environment does not support long amounts of text. If you have a lot of people and institutions to acknowledge, do not use this command. Instead, create a new \section{Acknowledgments}.
\begin{acknowledgments}
We thank the anonymous referee for helpful comments improving this letter. We thank Ido Barth for a useful discussion pointing the connection to coordinate moving in a potential well.
\end{acknowledgments}

%% To help institutions obtain information on the effectiveness of their 
%% telescopes the AAS Journals has created a group of keywords for telescope 
%% facilities.
%
%% Following the acknowledgments section, use the following syntax and the
%% \facility{} or \facilities{} macros to list the keywords of facilities used 
%% in the research for the paper.  Each keyword is check against the master 
%% list during copy editing.  Individual instruments can be provided in 
%% parentheses, after the keyword, but they are not verified.

% \vspace{5mm}
% \facilities{HST(STIS), Swift(XRT and UVOT), AAVSO, CTIO:1.3m,
% CTIO:1.5m,CXO}

%% Similar to \facility{}, there is the optional \software command to allow 
%% authors a place to specify which programs were used during the creation of 
%% the manuscript. Authors should list each code and include either a
%% citation or url to the code inside ()s when available.

% \software{astropy \citep{2013A&A...558A..33A,2018AJ....156..123A},  
%           Cloudy \citep{2013RMxAA..49..137F}, 
%           Source Extractor \citep{1996A&AS..117..393B}
%           }

%% Appendix material should be preceded with a single \appendix command.
%% There should be a \section command for each appendix. Mark appendix
%% subsections with the same markup you use in the main body of the paper.

%% Each Appendix (indicated with \section) will be lettered A, B, C, etc.
%% The equation counter will reset when it encounters the \appendix
%% command and will number appendix equations (A1), (A2), etc. The
%% Figure and Table counter will not reset.

%% For this sample we use BibTeX plus aasjournals.bst to generate the
%% the bibliography. The sample631.bib file was populated from ADS. To
%% get the citations to show in the compiled file do the following:
%%
%% pdflatex sample631.tex
%% bibtext sample631
%% pdflatex sample631.tex
%% pdflatex sample631.tex

\bibliography{precessing_potential_KLC}{}
\bibliographystyle{aasjournal}

%% This command is needed to show the entire author+affiliation list when
%% the collaboration and author truncation commands are used.  It has to
%% go at the end of the manuscript.
%\allauthors

%% Include this line if you are using the \added, \replaced, \deleted
%% commands to see a summary list of all changes at the end of the article.
%\listofchanges

\end{document}

% End of file `sample631.tex'.
