\documentclass[11pt, a4paper]{amsart}
\usepackage{amssymb}
\usepackage{amsmath}
\usepackage{youngtab}
\usepackage{young}
\usepackage{xfrac}
\usepackage{ytableau}
\usepackage{young}
\usepackage{youngtab}
\usepackage{lmodern}
\usepackage{enumitem}
\usepackage{graphicx} 
\usepackage[T1]{fontenc}
\usepackage{amsthm}
\usepackage{tikz-cd}
\usepackage[maxbibnames=99,
backend=biber,
style=alphabetic,
doi=false,
url=false,giveninits=true
]{biblatex}
\usepackage{fullpage}
\usepackage{subfig}




\addbibresource{library.bib}

 \usepackage{amsfonts,mathrsfs}
\usepackage{bbm,amsmath,amssymb,amsfonts,graphicx,color,subfig,mathtools, verbatim}
   \usepackage[bookmarksopen=false,pdftex=true,breaklinks=true,%
      %backref=page,%pagebackref=true,
      plainpages=false,%
      hyperindex=true,pdfstartview=FitH,colorlinks=true,%
      pdfpagelabels=true,colorlinks=true,linkcolor=blue,%
      citecolor=red,urlcolor=green,hypertexnames=false%
      ]%
   {hyperref}
   
 \newcommand{\abb}[5]{%
\setlength{\arraycolsep}{0.4ex}%
\begin{array}{rcccc}%
#1 &:\,& #2 & \,\,\longrightarrow\,\, & #3 \\[0.5ex]%
     & & #4 & \longmapsto & #5%
\end{array}%
}
\newcommand{\F}{\mathcal{F}}   
\renewcommand{\vartheta}{\theta}
\renewcommand{\H}{\mathcal{H}}   
\newcommand{\A}{\mathcal{A}}
\newcommand{\K}{\mathbb{K}}
\newcommand{\N}{\mathbb{N}}
\newcommand{\Z}{\mathbb{Z}}
\newcommand{\R}{\mathbb{R}}
\newcommand{\C}{\mathbb{C}}
\newcommand{\x}{\mathbf{x}}
\newcommand{\Q}{\mathcal{Q}}
\renewcommand{\leq}{\leqslant}
\newcommand{\YT}{\Tab}
\usepackage{faktor}
\newcommand{\fto}{\longrightarrow}
\usepackage{MnSymbol} 
\numberwithin{equation}{section}
\newtheorem{theorem}{Theorem}[section]
\newtheorem{lemma}[theorem]{Lemma}
\newtheorem{remark}[theorem]{Remark}
\newtheorem{notation}[theorem]{Notation}
\newtheorem{corollary}[theorem]{Corollary}
\newtheorem{proposition}[theorem]{Proposition}

\newtheorem{example}[theorem]{Example}
\newtheorem{question}[theorem]{Question}
\newtheorem{conjecture}[theorem]{Conjecture}
    \newtheoremstyle{TheoremNum}
        {\topsep}{\topsep}              %%% space between body and thm
        {\itshape}                      %%% Thm body font
        {}                              %%% Indent amount (empty = no indent)
        {\bfseries}                     %%% Thm head font
        {.}                             %%% Punctuation after thm head
        { }                             %%% Space after thm head
        {\thmname{#1}\thmnote{ \bfseries #3}}%%% Thm head spec
    \theoremstyle{TheoremNum}
    \newtheorem{thmn}{Theorem}
  \newtheorem{corn}{Corollary}
    \newtheorem{propn}{Proposition}
\theoremstyle{definition}
\newtheorem{definition}[theorem]{Definition}
    


\DeclareMathOperator{\Stab}{Stab}
\DeclareMathOperator{\spe}{sp}
\DeclareMathOperator{\spn}{span}
\DeclareMathOperator{\Tr}{tr}
\DeclareMathOperator{\Sym}{\mathcal{S}}
\DeclareMathOperator{\trop}{trop}
\DeclareMathOperator{\inti}{int}
\DeclareMathOperator{\clo}{cl}
\DeclareMathOperator{\bd}{bd}
\DeclareMathOperator{\conv}{conv}
\DeclareMathOperator{\cone}{cone}
\DeclareMathOperator{\Ko}{\mathcal{K}}
\DeclareMathOperator{\M}{\mathcal{M}}
\DeclareMathOperator{\rank}{rk}
\DeclareMathOperator{\E}{\mathcal{E}}
\DeclareMathOperator{\Img}{Im}
\DeclareMathOperator{\Jac}{J}

\newcommand{\sebastian}[1]{{\color{blue} S: #1}}
\newcommand{\greg}[1]{{\color{red} GB: #1}}

\newcommand{\cordian}[1]{{\color{green} C: #1}}

 \newenvironment{case}{%
 \let\olditem\item% 
 \renewcommand\item[1][]{\olditem \textbf{##1} \\}%
 \begin{enumerate}[label=\textbf{Case \arabic*:},itemindent=*,leftmargin=0em]}{\end{enumerate}%
 }
 

%% code from mathabx.sty and mathabx.dcl
\DeclareFontFamily{U}{mathx}{\hyphenchar\font45}
\DeclareFontShape{U}{mathx}{m}{n}{
      <5> <6> <7> <8> <9> <10>
      <10.95> <12> <14.4> <17.28> <20.74> <24.88>
      mathx10
      }{}
\DeclareSymbolFont{mathx}{U}{mathx}{m}{n}
\DeclareFontSubstitution{U}{mathx}{m}{n}
\DeclareMathAccent{\widecheck}{0}{mathx}{"71}

\newcommand{\defi}[1]{\textit{#1}}

\def\spliteq#1#2{\setbox0=\hbox{$\displaystyle
    #1$}\hskip\wd0\setbox0=\hbox{$\displaystyle
    #2$}\hskip-\wd0\qquad #2}
\ytableausetup{centertableaux}


\usepackage{listings}
\lstdefinelanguage{Sage}[]{Python}
{morekeywords={False,sage,True},sensitive=true}
\lstset{
  frame=none,
  showtabs=False,
  showspaces=False,
  showstringspaces=False,
  commentstyle={\ttfamily\color{dgreencolor}},
  keywordstyle={\ttfamily\color{dbluecolor}\bfseries},
  stringstyle={\ttfamily\color{dgraycolor}\bfseries},
  language=Sage,
  basicstyle={\fontsize{9pt}{9pt}\ttfamily},
  aboveskip=0.3em,
  belowskip=0.1em,
  numbers=left,
  numberstyle=\footnotesize
}
\definecolor{dblackcolor}{rgb}{0.0,0.0,0.0}
\definecolor{dbluecolor}{rgb}{0.01,0.02,0.7}
\definecolor{dgreencolor}{rgb}{0.2,0.4,0.0}
\definecolor{dgraycolor}{rgb}{0.30,0.3,0.30}
\newcommand{\dblue}{\color{dbluecolor}\bf}
\newcommand{\dred}{\color{dredcolor}\bf}
\newcommand{\dblack}{\color{dblackcolor}\bf}




\begin{document}

\title[]
{
The Wonderful Geometry of the Vandermonde map}


\author{Jose Acevedo}
\address{School of Mathematics, Georgia Institute of Technology, 686 Cherry Street Atlanta, GA 30332, USA}
\email{jacevedo@gatech.edu}

\author{Grigoriy Blekherman}
\address{School of Mathematics, Georgia Institute of Technology, 686 Cherry Street Atlanta, GA 30332, USA}
\email{greg@math.gatech.edu}

\author{Sebastian Debus}
\address{Fakultät für Mathematik, Institut für Algebra und Geometrie, Otto-von-Guericke-Universität Magdeburg, 39106 Magdeburg,
Germany}
\email{sebastian.debus@ovgu.de}


\author{Cordian Riener}
\address{Department of Mathematics and Statistics, UiT - the Arctic University of Norway, 9037 Troms\o, Norway}
\email{cordian.riener@uit.no}


\thanks{The first and second author were partially supported by NSF grant DMS-1901950. The third and fourth author have been supported by European Union's Horizon 2020 research and innovation programme under the Marie Sk\l{}odowska-Curie grant agreement 813211 (POEMA) and the Troms\o~ Research foundation grant agreement 17matteCR. The third author was additionally supported by the Deutsche Forschungsgemeinschaft (DFG, German Research Foundation) – 314838170, GRK 2297 MathCoRe}.


\keywords{Vandermonde map, cyclic polytopes, trace polynomials, copositivity, undecidability}




\begin{abstract}
We study the geometry of the image of the nonnegative orthant under the power-sum map and the  elementary symmetric polynomials map. After analyzing the image in finitely many variables, we concentrate on the limit as the number of variables approaches infinity. We explain how the geometry of the limit plays a crucial role in undecidability results in nonnegativity of symmetric polynomials, deciding validity of trace inequalities in linear algebra, and extremal combinatorics - recently observed by  Blekherman, Raymond, and F. Wei \cite{blekherman2022undecidability}. We verify the experimental observation that the image has the combinatorial geometry of a cyclic polytope made by Melánová, Sturmfels, and  Winter \cite{melanova2022recovery}, and generalize results of Choi,  Lam, and Reznick \cite{choi1987even} on nonnegative even symmetric polynomials. We also show that undecidability does not hold for the normalized power sum map. 
\end{abstract}
\maketitle
\markboth{J.~Acevedo, G.~Blekherman, S.~Debus, and C.~Riener}{Symmetric forms}

\section{Introduction} The main object of this article is the so  called Vandermonde map which appears quite naturally in various contexts and thus providing connections between different mathematical domains. Our interest in this object is motivated by the following  problem. Suppose that we are given a polynomial expression in traces of powers of symmetric matrices, such as \[2\operatorname{tr} (A^2)\operatorname{tr} (B^6) - \operatorname{tr} (A^4) \operatorname{tr} (B^4),\] is there an algorithm to decide whether this expression is nonnegative for all symmetric matrices $A$, $B$ of all sizes? What happens if we replace trace by normalized trace $\widetilde{\operatorname{tr}} (A)=\frac{\operatorname{tr} A}{n}$, where $n$ is the size of the matrix?

As one of the results of our work we show that the first (unnormalized) problem is \emph{undecidable}, while the second one is \emph{decidable}. The key to the hardness of the unnormalized problem is the fascinating geometry of \emph{the image of the probability simplex under the Vandermonde map}. As we explain below, some geometric properties of this set were observed in different areas of mathematics making it an important and beautiful object to study.


For any $n\times n$ matrix $A$ recall that $\operatorname{tr} (A^d)=\lambda_1^d+\dots+\lambda_n^d$, where $\lambda_i$ are the eigenvalues of $A$. We use $p_d$ to denote the $d$-th power sum polynomial: $p_d(x)=x_1^d+\dots+x_n^d$. We see that testing whether $2\operatorname{tr} (A^2)\operatorname{tr} (B^6) - \operatorname{tr} (A^4) \operatorname{tr} (B^4) $ is nonnegative on all symmetric matrices of all sizes is equivalent to understanding whether $2p_2(x)p_6(y)-p_4(x)p_4(y)$ is nonnegative on all real vectors $x$ and $y$ of any dimension. %We will only consider even exponents $k$. 
Define the \textit{$d$-th Vandermonde map $\nu_{n,d}$} by sending a point in $\mathbb{R}^n$ to its image under the first $d$ power sums:
$$\nu_{n,d}(x)=(p_1(x),\dots,p_{d}(x)).$$
Let $\Delta_{n-1}$ be the probability simplex in $\mathbb{R}^n$: $\Delta_{n-1}$ consists of all vectors with nonnegative coordinates with the sum of coordinates equal to $1$. We call the image $\nu_{n,d}(\Delta_{n-1})$ of the probability simplex under the Vandermonde map \textit{the $(n,d)$-Vandermonde cell} and denote it by $\Pi_{n,d}$. Observe that the first coordinate of $\Pi_{n,d}$ is identically $1$, and so we may project it out, and see $\Pi_{n,d}$ as the subset of $\mathbb{R}^{d-1}$, which is the image of $\Delta_{n-1}$ under $(p_2,\dots, p_d)$. 


Since $2p_2(x)p_6(y)-p_4(x)p_4(y)$ is an even homogeneous polynomial, deciding whether it is nonnegative for all $x,y \in \mathbb{R}^n$ is equivalent to deciding whether the polynomial $2a_1b_3-a_2b_2$ is nonnegative on the product $\Pi_{n,3}\times \Pi_{n,3}$, where $a_i=p_i(x)$ and $b_i=p_i(y)$.

We reach two important conclusions: first, we are interested in nonnegativity of polynomials on (products of) Vandermonde cells $\Pi_{n,d}$, and second, to consider matrices of all sizes we need to take the \emph{limit of the Vandermonde cell $\Pi_{n,d}$ as $n$ goes to infinity}.


The Vandermonde cell $\Pi_{n,d}$ is a compact subset of $\mathbb{R}^{d-1}$, and our first main result is that $\Pi_{n,d}$ has \emph{the combinatorial structure of a cyclic polytope}, verifying an experimental observation of \cite{melanova2022recovery}. 

For a fixed $d$ the sets $\Pi_{n,d}$ form an increasing sequence of sets in $\mathbb{R}^{d-1}$. Let $\Pi_d$ be the closure of the union of $\Pi_{n,d}$.
We show that the set $\Pi_d$ has the combinatorial structure of an \textit{infinite cyclic polytope}, and that $\Pi_d$ is not semialgebraic for all $d \geq 3$. The sets $\Pi_{n,3}$ and $\Pi_3$ are depicted in Figure \ref{fig: N33 and N53}. 
Reduction needed to show undecidability of the unnormalized trace problem is borrowed from the one used by Hatami and Norine in \cite{hatami2011undecidability} in the context of homomorphism density inequalities in graph theory. The set used by Hatami and Norine is essentially a linear transformation of the set $\Pi_3$, and the reduction is based on the geometry of $\Pi_3$. In particular this shows that deciding validity of matrix power trace inequalities is already undecidable if we only use second, fourth and sixth matrix powers, and we need at most 11 matrix variables for the problem to become undecidable. We note that the geometry of $\Pi_3$ was also used directly by Blekherman, Raymond and Wei \cite{blekherman2022undecidability} to show undecidability of homomorphism density inequalities with arbitrary edge weights.

We also consider the image of $\Delta_{n-1}$ under elementary symmetric polynomials. Our previous results on the boundary structure transfer over by using Newton's identities. We write $E_{n,d} := (e_1,\ldots,e_d)(\Delta_{n-1})$ and denote the limit image by $E_d$. We show that the convex hull of $E_{n,d}$ is an actual cyclic polytope. This helps us reprove and slightly generalize the result of Choi, Lam and Reznick \cite{choi1987even} on test sets for nonnegativity of even symmetric sextics. We note that the convex hull result can be traced to the work of Bollobás in extremal graph theory \cite{bollobas1976relations}. 


Testing nonnegativity of univariate normalized trace polynomials was considered  by Klep, Pascoe and Vol{\v{c}}i{\v{c}} \cite{klep2021positive} where the authors proved
a Positivstellensatz in the univariate case. Geometrically, such normalized trace polynomials correspond to power means. Nonnegativity of polynomials in power means was investigated by Blekherman and Riener in degree $4$ \cite{blekherman2021symmetric} and more generally by Acevedo and Blekherman \cite{acevedo2022}. We briefly illustrate the connection with the Vandermonde map. Decidability of the normalized trace problem follows quickly from the work in \cite{blekherman2021symmetric}. As before we can consider the image of the normalized Vandermonde map, and fixing $d$ take the (closure of the) limit as $n$ goes to infinity.  As explained in \cite{acevedo2022} the geometry of the limit is drastically different. For instance, the limit of the normalized Vandermonde map of the unit simplex $\Delta_{n-1}$ corresponds to the set of the first $d$ moments of a probability measure supported on $\mathbb{R}_{\geq 0}$, and it is well-known that this set can be described by linear matrix inequalities \cite{MR3729411}. In particular, the limit is semialgebraic for all $d$.

\subsection{Previous Work and Main Results in Detail}
The Vandermonde map has been studied from several different perspectives. Originating from the question of understanding univariate hyperbolic polynomials, Arnold, Givental and Kostov investigated the sets $(e_1,\ldots,e_d)(\R^n)$ \cite{arnol1986hyperbolic,givental1987moments,kostov1989geometric,kostov1989geometric,kostov1999hyperbolicity}. A detailed examination can also be found in \cite{meguerditchian1992theorem}. Kostov investigated the limit of the images of $\R^n$ for $d=4$. The authors observed, that one can also allow positive \textit{weights} in the definition of the Vandermonde map. Their description of the boundary of the image of the Vandermonde map and of fibers generalizes to the map \[\R^n \longrightarrow \R^d, x \mapsto (a_1x_1+\ldots+a_nx_n,\ldots,a_1x_1^d+\ldots+a_nx_n^d)\]for any positive weights $a_1,\ldots,a_n > 0$. This is mainly due to the fact that Jacobians of the weighted and unweighted maps differ only by positive constant multiples. %\greg{This sentence is very unclear. Do we mean Jacobian?} \sebastian{yes.}

The restriction of the Vandermonde map to the nonnegative orthant was investigated by Ursell \cite{ursell1959inequalities}. The paper contains several important results some of which we reprove. Ursell observed that the geometry of the Vandermonde map restricted to the nonnegative orthant generalizes further to arbitrary real positive exponents, i.e. to maps \[\R^n \longrightarrow \R^d, x \mapsto (x_1^{\alpha_1}+\ldots+x_n^{\alpha_1},\ldots,x_1^{\alpha_d}+\ldots+x_n^{\alpha_d})\] for which $0 < \alpha_1 < \ldots < \alpha_d$. Ursell's original motivation came from studying valid inequalities in $\ell_p$-norms.

 Recently, there has been an interest in describing fibers and the image of the Vandermonde map using computational algebraic geometry \cite{bik2021semi,melanova2022recovery}. Bik, Czapliński and Wageringe derived semialgebraic description of $\nu_{n,3}([0,1]^n)$ for all $n \geq 3$ which has applications in the study of $L$-functions and their zeros. Melánová, Sturmfels and Winter explored fibers and the image  of the Vandermonde map over the complex numbers and real numbers. \medskip


Our first theorem is a result initially found by Ursell \cite{ursell1959inequalities}. We provide a different proof by adapting the techniques in Arnold's and Givental's work. \medskip

\begin{thmn}[\ref{thm:1}] 
For integers $n \geq d$ the set $\bd \Pi_{n,d}$ is the set of evaluations of $\nu_{n,\alpha}$ at all points in $\Delta_{n-1}$ of the following two types:
\begin{itemize}
 \item[\emph{(1)}] $(\underbrace{0,\ldots,0}_{r_0},\underbrace{x_1}_{r_1},\underbrace{x_2,\ldots,x_2}_{r_2},\ldots,\underbrace{x_{d-1},\ldots,x_{d-1}}_{r_{d-1}})$ with $r_{2k-1} = 1$ and $r_0 \geq 0$, $r_{2k} \geq 1$ for all $k$,
 \item[\emph{(2)}] $(\underbrace{x_1,\ldots,x_1}_{r_1},\underbrace{x_2}_{r_2},\ldots,\underbrace{x_{d-1},\ldots,x_{d-1}}_{r_{d-1}})$ with $r_{2k} = 1$ and $r_{2k} \geq 1$ for all $k$.
\end{itemize} 
and $ 0 \leq x_1 \leq x_2 \leq \ldots \leq x_{d-1}$
\end{thmn} \medskip

We then investigate concretely the planar boundaries of $\Pi_{n,3}$ and the limit set $\Pi_3$ and derive consequences for all $d\geq 3$. \medskip

\begin{corn}[\ref{cor:not semialgebraic}]
The sets $\Pi_{d}$ and $E_d$ are not semialgebraic for all $d \geq 3$.
\end{corn} \medskip


 Let $\mu_d : \R \to \R^d, t \mapsto (t,t^2,\ldots,t^d)$ denote the $d$-dimensional moment curve and let $t_1 < \ldots < t_n$. For $n > d$ the \textit{cyclic polytope} $C(n,d)$ is the convex polytope with vertices $\mu_d(t_i)$ for $1 \leq i \leq n$. The combinatorial structure of the cyclic polytope is independent of the chosen $n$ points on the moment curve. Cyclic polytopes are the polytopes with maximal $f$-vector among all convex polytopes of given dimension and number of vertices \cite{mcmullen1970maximum,stanley1975upper}. The facets of $C(n,d)$ are characterized by \textit{Gale's evenness condition} \cite{gale1963neighborly}. A subset $\{\mu_d(t_{i_1}),\ldots,\mu_d(t_{i_d})\}$ with $i_j < i_{j+1}$ for all $1 \leq j <d$ spans a facet if and only if any two elements in $\{t_1,\ldots,t_n\} \setminus \{t_{i_1},\ldots,t_{i_d}\}$ are separated by an even number of elements $\{t_{i_1},\ldots,t_{i_d}\}$. Answering a question in \cite{melanova2022recovery} we prove that the set $\Pi_{n,d}$ has the combinatorial structure of a cyclic polytope, in the sense that the boundary is a gluing of patches where each patch is a curved simplex and the vertices of the patches are characterized by Gale's evenness condition. A set $S$ is a curved simplex if it is the image of $\Delta_m$ under a continuous map $f$, such that $f$ is a diffeomorphism when restricted to the relative interior of any face of $\Delta_m$. \medskip

\begin{thmn} [\ref{thm:combinatorially cyclic polytope}]
The set $\Pi_{n,d}$ has the combinatorial structure of the cyclic polytope $C(n,d-1)$, i.e. there is a homeomorphism $ \bd C(n,d-1) \rightarrow \bd\Pi_{n,d} $ that is a diffeomorphism when restricted to the relative interior of any face of $\bd C(n,d-1)$.
\end{thmn} \medskip

We provide an explicit map $ \bd C(n,d-1) \rightarrow \bd\Pi_{n,d} $ in Section \ref{sec:comb prop}. \\
For $n \geq d$ it follows from Newton's identities 
\begin{align} \label{eq:Newton's identities}
    p_k & =(-1)^{k-1}ke_k+\sum _{i=1}^{k-1}(-1)^{k-1+i}e_{k-i}p_i \hspace{2cm}\text{for all } 1 \leq k \leq n
\end{align}
that the image of the even Vandermonde map is diffeomorphic to the image on the first $d$ elementary symmetrics, i.e.
$E_{n,d} \simeq \Pi_{n,d}$ under a polynomial diffeomorphism. We show that the convex set $\mathcal{E}_{n,d} := \conv E_{n,d}$ has nice properties which $ \conv \Pi_{n,d}$ does not have. \medskip

\begin{thmn} [\ref{thm:Convex hull Image of Elementary}]
For $d\geq 3$ 
  the set $\mathcal{E}_{n,d} $ is a cyclic polytope, and it is
 the convex hull of the following finite set of points $
    \left( {k \choose 2} \frac{1}{k^2},\ldots,{k \choose d} \frac{1}{k^d} \right):  k \in [n]. 
$
\end{thmn} \medskip

Recall that any symmetric polynomial can be written as a polynomial expression in elementary symmetrics or power sums. The vertices of $\mathcal{E}_{n,d}$ relate to copositivity of homogeneous symmetric polynomials in which only $e_1$ occurs nonlinearly, i.e. symmetric forms which can be written as $$f=c_1e_1^d+c_2e_1^{d-2}e_2+\ldots+c_me_d$$  for some $c_1,\ldots,c_m \in \R$. We call such forms \emph{hook-shaped symmetric polynomials}. \medskip

\begin{thmn}[\ref{thm:discrete test set in elementarys}]
Let $f$
be a hook-shaped symmetric polynomial in $n \geq d$ variables. Then $f$ is copositive if and only if  $f\left(1,{k \choose 2}\frac{1}{k^2},\ldots,{k \choose d} \frac{1}{k^d}\right) \geq 0$ for all $k \in[n]$.
\end{thmn} \medskip


For $d=3$ this test set was found by Choi, Lam and Reznick \cite{choi1987even} but formulated for even symmetric sextics. \medskip

\begin{thmn}[\ref{thm:choi-lam-reznick}] \cite{choi1987even}
Let $f(p_2,p_4,p_6)$ be an even symmetric form of degree $6$ in $n \geq 3$ variables. Then, $f$ is nonnegative if and only if $f\left(1,\frac{1}{k},\frac{1}{k^2}\right)$ is nonnegative for all $k \in [n]$.
\end{thmn} \medskip

Using Newton's identities and the fact that we can restrict to $p_2 =1$ due to homogenicity and the nonnegative orthant, the test sets in elementary symmetrics and power sums are equivalent for $d \leq 3$ due to the linear relation of those families of polynomials on the probability simplex. Surprisingly, we show that the test set for copositivity in power sums does not generalize to any higher degree. \medskip

\begin{propn}[\ref{prop:not power sums}]
Let $n \geq d \geq 4$. Then the set $\conv \left\{ \left( \frac{1}{k},\frac{1}{k^2},\cdots,\frac{1}{k^{d-1}} \right) : k \in [n] \right\}$ does not contain the set $\Pi_{n,d}$ and $\Pi_d \not \subset \conv \left\{(0,\ldots,0), \left( \frac{1}{k},\frac{1}{k^2},\cdots,\frac{1}{k^{d-1}} \right) : k \in \N \right\}$.    
\end{propn} \medskip

Finally, we prove undecidability of testing validity of inequalities of trace polynomials in all symmetric matrices of all sizes. \medskip

\begin{thmn} [\ref{thm:undecidable traces}]
The following decision problem is undecidable.
\begin{itemize}
    \item[{\footnotesize Instance:}] A positive integer $k$ and a trace polynomial $f(X_1,\ldots,X_k)$.
    \item[{\footnotesize Question:}] Is $f(M_1,\ldots,M_k)$ nonnegative for all real symmetric matrices $M_1,\ldots,M_k$ of all sizes for all $1 \leq i \leq k$?
\end{itemize}
\end{thmn}\medskip

When we replace the usual trace by the \text{normalized trace}, i.e. $\frac{\Tr (A)}{n}$ for a symmetric matrix $A$ of size $n \times n$, the problem becomes decidable. \medskip

\begin{thmn}[\ref{thm:normalized decidable}]
 The following decision problem is decidable.
\begin{itemize}
    \item[{\footnotesize Instance:}] A positive integer $k$ and a normalized trace polynomial $f(X_1,\ldots,X_k)$.
    \item[{\footnotesize Question:}] Is $f(M_1,\ldots,M_k)$ nonnegative for all symmetric matrices $M_1,\ldots,M_k $ of all sizes?
\end{itemize}     
\end{thmn} \medskip




\section{The Vandermonde map} \label{sec:Vandermonde map} 


In this section we study the geometry of the boundary of the Vandermonde cell. We start with some definitions.
\begin{definition} \label{def:Vandermonde}
\begin{enumerate}[leftmargin=*]
\item For $n,k\in \mathbb{N}$, $n\geq k$ we write \[e_k := \sum_{I \subset [n], |I|=k}\prod_{i \in I}x_i\] for the \emph{$k$-th elementary symmetric polynomial} in $n$ variables.
\item For $a\in\R_{>0}$ we consider  \[p_{a}:=\sum_{i=1}^n x_i^a\]
the \emph{power sum function}, which for $a \in \N$ is called a \emph{power sum polynomial}.
\item Given a sequence of strictly increasing positive real  numbers  $\alpha = (\alpha_1,\ldots,\alpha_d) \in \R_{>0}^d$ we consider the \emph{$\alpha$-Vandermonde map} in $n$ variables to be  the function
$$ \abb{\nu_{n,\alpha}}{\R_{\geq 0}^n}{\R^d}{x}{(p_{\alpha_1}(x),p_{\alpha_2}(x),\dots,p_{\alpha_d}(x))}$$  
\end{enumerate}
\end{definition}
In the sequel we will restrict our study of Vandermonde maps to the probability simplex and power sum polynomials. This can be done without loss of generality which  can be seen as follows.
\begin{remark}\label{rem:wlog}
Given a sequence $\alpha$ as in Definition \ref{def:Vandermonde}. We  obtain  the following normalized  sequence  \[\beta = ( 1, \frac{\alpha_2}{\alpha_1},\ldots,\frac{\alpha_d}{\alpha_1})\] for which we find $\nu_{n,\alpha}(x)=\nu_{n,\beta} (x_1^{\alpha_1},\ldots,x_d^{\alpha_1})$ for all $x \in \R_{\geq 0}^n$. Therefore, given rational positive $\alpha\in\mathbb{Q}^d_{>0}$, i.e., the components of $\alpha$ are of the form  $\alpha_1=\frac{s_1}{t_1},\ldots,\alpha_d=\frac{s_d}{t_d}$ for some integers $s_i,t_i$ one can set  $q := t_1\cdots t_d$ and $\beta:=(q\alpha_1,\ldots,q\alpha_d)$ to obtain $\nu_{n,\alpha}(x^q) = \nu_{n,\beta}(x)$. Taking into consideration that every real power sum function with irrational exponents  can be be approximated by rational power sum functions, we can conclude that it is sufficient to   study  the image of $\nu_{n,\alpha}$ for integer exponents and $\alpha_1=1$ in order to explore the geometry of  general Vandermonde mappings.
\end{remark}

\begin{definition}
Let $\Delta_{n-1}:=\left\{x\in\R^{n}_{\geq 0}\,:\,x_1+\ldots+x_{n}=1\right\}$ and  $\alpha = (\alpha_1,\ldots,\alpha_{d-1})\in\R^{d-1}_{>1}$ be a  strictly increasing sequence of real numbers larger than $1$. The  \emph{$(n,\alpha)$-Vandermonde cell} $\Pi_{n,\alpha}$ is the set   $\nu_{n,\alpha}(\Delta_{n-1})$. Note for $\alpha = (2,\ldots,d)$ we also write  $ \Pi_{n,\alpha} = \Pi_{n,d} $.
\end{definition}

In Subsection \ref{subsection boundary of the vandermonde cell} we investigate generally the boundary of the Vandermonde cell.  In Subsection \ref{subsection power sum and elementary} we parameterize planar boundaries of the sets $\Pi_{n,3}$ and $\Pi_3$ and we show that the limit set $\Pi_d$ is not semialgebraic for all $d \geq 3$. 




\subsection{The boundary the  Vandermonde cell} \label{subsection boundary of the vandermonde cell}
In the sequel we will prove the following statement on points on the boundary of a Vandermonde cell.

\begin{theorem}\label{thm:1}
For integers $n \geq d$ the set $\bd \Pi_{n,\alpha}$ is the set of evaluations of $\nu_{n,\alpha}$ at all points in $\Delta_{n-1}$ of the following two types:
\begin{itemize}
 \item[\emph{(1)}] $(\underbrace{0,\ldots,0}_{r_0},\underbrace{x_1}_{r_1},\underbrace{x_2,\ldots,x_2}_{r_2},\ldots,\underbrace{x_{d-1},\ldots,x_{d-1}}_{r_{d-1}})$ with $r_{2k-1} = 1$ and $r_0 \geq 0$, $r_{2k} \geq 1$ for all $k$,
 \item[\emph{(2)}] $(\underbrace{x_1,\ldots,x_1}_{r_1},\underbrace{x_2}_{r_2},\ldots,\underbrace{x_{d-1},\ldots,x_{d-1}}_{r_{d-1}})$ with $r_{2k} = 1$ and $r_{2k} \geq 1$ for all $k$.
\end{itemize} 
and $ 0 \leq x_1 \leq x_2 \leq \ldots \leq x_{d-1}$
\end{theorem}


\begin{definition}
The  vector $r=(r_0,\ldots,r_{d-1}) \in \N^d$ defined via  Theorem \ref{thm:1} for every $x\in\Delta_{n-1}$ is called  the \emph{multiplicity vector} of $x$, where we set  $r_0=0 $ if the associated point is of type $(2)$.
\end{definition}
In order to understand the boundary of the Vandermonde cell we are considering the following notion of generalized positive Vandermonde variety. The study of such varieties goes back to work of Arnold, Givental and Kostov \cite{arnol1986hyperbolic,givental1987moments,kostov1989geometric}
  who had considered  these in their study of hyperbolic polynomials. Our setup is a bit more evolved, as in contrast to the above named authors, we will consider general exponents and  consider only the positive part. We fix $2\leq d\leq n$ and a vector of integer exponents $\alpha = (\alpha_1,\ldots,\alpha_{d-1})$ with $1 < \alpha_1 < \ldots < \alpha_{d-1}$ for the remaining part of this subsection. 
 \begin{definition}
For $2 \leq k \leq d$ and $c \in \{1\}\times\R^{k-1}_{\geq 0}$ we define the associated  \emph{generalized positive $\alpha$-Vandermonde variety} to be the fiber over $c$ of the corresponding Vandermonde map, i.e.,  \[V_k^\alpha(c):=\nu_{n,(1,\alpha_1,\ldots,\alpha_{k-1})}^{-1}(c) \cap\R^n_{\geq 0}.\]
 \end{definition}
 Some  fundamental properties of these varieties had been shown already by the mentioned authors. We show here that their proofs almost directly can be generalized to the more general setup presented above. To this end  we follow the proofs presented in \cite{meguerditchian1992theorem} (see also \cite{rainer2004perturbation} Section 3). 

We begin with the following observation for the tangent space of a Vandermonde variety.

\begin{lemma}\label{lem:d-1 points} Let $c \in \{1\}\times\R^{k-1}_{\geq 0}$. Then a point $x \in V_{k}^\alpha(c)\cap\R^{n}_{\geq 0}$ with  more than $k$ distinct  need non-zero coordinates is a smooth point. 
\end{lemma}
\begin{proof} 
First, note that $\frac{\partial{p_{a}}}{\partial x_j} =ax_j^{a-1}$. Therefore,  the Jacobian of the map $(p_1,p_{\alpha_1},\ldots,p_{\alpha_{k-1}})$ equals
\[ 
\left( \begin{array}{ccccc}
    1 &   1 &  \ldots &   1 \\
      \alpha_1 x_1^{\alpha_1-1} &   \alpha_1 x_2^{\alpha_1-1} &  \ldots &   \alpha_1 x_n^{\alpha_1-1} \\
      \vdots & \vdots & & \vdots \\
        \alpha_{k-1} x_1^{\alpha_{k-1}-1} &   \alpha_{k-1} x_2^{\alpha_{k-1}-1} &  \ldots &   \alpha_{k-1} x_n^{\alpha_{k-1}-1}
\end{array} \right)
\]
Now suppose that the rows of the Jacobi matrix are linearly dependent, so  there exist $l_0,\ldots,l_{k-1}$ such then every column satisfies $l_0\cdot 1+l_2\cdot \alpha_1 x_j^{\alpha_1-1}+\ldots l_{k-1}\cdot \alpha_{k-1}x_j^{\alpha_k-1}=0$. Therefore, every coordinate $x_j$ is a solution to the same univariate polynomial  \[f(t)= b_0+b_1t^{\alpha_1-1}+\ldots+b_{k-1}t^{\alpha_{k-1}-1}.\] However, by Descarte's rule of signs $f$ can have at most $k-1$ different roots in $\R^n_{> 0}$. Therefore, in case $x$ has more distinct non-zero absolute values of coordinates the Jacobi matrix has full rank. 
\end{proof}


\begin{lemma} \label{lem:critical points}
 Let $c \in \{1\}\times\R^{d-2}_{\geq 0}$  be generic. The critical points of $p_{\alpha_{d-1}}$ on  $V_{d-1}^\alpha(c)\cap\R^n_{\geq 0}$ are exactly the points with precisely $d-1$ distinct non-zero coordinates. 
\end{lemma}
\begin{proof}
For generic $c$ the Vandermonde variety $V_k^\alpha(c)$ is smooth and by the Jacobian criterion (\cite[Thm 16.19]{eisenbud2013commutative}) $(n-d-1)$ equidimensional (or empty), therefore by the previous Lemma every point in $x \in V_{d-1}^\alpha(c)$ will have strictly more than $d-2$ distinct non-zero coordinates. Let $z=(z_1,\ldots,z_n) \in \R^n_{\geq 0}$ be a critical point of $p_{\alpha_{d-1}}$ on  $V_{d-1}^\alpha(c)$.

Then, there exists Lagrange multipliers $\lambda_1^*,\ldots,\lambda_{d-1}^* \in \R$ such that all partial derivatives of the Lagrangian function \begin{equation}\label{eq:largange} 
L(X) := p_{\alpha_{d-1}}(X)+\lambda_0(p_1-a_1)+ \sum_{i=1}^{d-2}\lambda_i^* (p_{\alpha_i}(X)-a_{i+1})\end{equation} vanish at $x$. This yields
\[ 0=\nabla p_{\alpha_{d-1}} (z)  + \lambda_1^* \nabla p_{1}(z) + \ldots + \lambda_{d-1}^* \nabla p_{\alpha_{d-2}}(z). \]
Again noting that $\frac{\partial{p_{a}}}{\partial x_j} =ax_j^{a-1}$ we  observe that there exists a univariate polynomial \[f(t) = \alpha_{d-1}t^{\alpha_{d-1}-1} -  \lambda_1  t - \ldots - \lambda_{d-1} {\alpha_{d-2}}t^{\alpha_{d-2}-1}\] such  that  $f(z_i) = 0$ for all $1 \leq i \leq n$. However, by Descarte's rule of signs the polynomial $f$ can have at most $d-1$ positive roots. Since $z \in \R^n$ is regular we have $z$ must have exactly $d-1$ distinct absolute values of non-zero coordinates. Conversely, if $z \in V_{d-1}^\alpha(c)\cap\R^{n}_{\geq 0}$ has precisely $d-1$ distinct absolute values of non-zero coordinates the existence of the Lagrange multipliers follows from the observation that 
the rank of the $d-1 \times d$ matrix 
\[ 
A := \left( 
\begin{array}{cccc}
    z_1 & z_1^{\alpha_1-1} & \ldots & z_1^{\alpha_{d-1}-1}\\
    z_2 & z_2^{\alpha_1-1} & \ldots & z_2^{\alpha_{d-1}-1}\\
   \vdots & \vdots & & \vdots \\
    z_{d-1} & z_{d-1}^{\alpha_1-1} & \ldots & z_{d-1}^{\alpha_{d-1}-1}
\end{array}
\right)
\]
is $d-1$. Thus, the columns of this matrix being linearly dependent yields the existence of the Lagrange multipliers.
\end{proof}


\begin{definition}
To $x\in\R_{\geq 0}^n$ we associate the multiplicity vector $m=(m_0,\ldots,m_n)\in\Z_{\geq 0}^n$ as follows: The number $m_0$ counts the number of zero coordinates, and $m_i$ corresponds to the number of times the $i-th$ smallest coordinate appears. This means, assuming that $x$ has $k$ positive coordinates which are ordered increasingly, we have 
 \[x= (\underbrace{0,\ldots,0}_{m_0},\underbrace{x_1,\ldots,x_1}_{m_1},\ldots,\underbrace{x_{k},\ldots,x_{k}}_{m_{k}}).\]
\end{definition}

\begin{proposition}\label{prop2}
Let $V_{d-1}^\alpha(c)$ be a smooth Vandermonde variety and $x\in\R_{\geq 0}^n$ be a critical point of $p_{\alpha_{d-1}}$ on $V_{d-1}^\alpha(c)$. Further set $r_i = m_i-1$. Then, if $d$ is odd (even), the Hessian of $p_{\alpha_{d-1}}$ on $V_{d-1}^\alpha(c)$ at the point $x$ is the sum of a negative (positive) definite quadratic form on $\R^{a}$ and a positive (negative) definite form on $\R^{b}$, where $a = \sum_{i < d, i \not \in 2 \N} r_i$ and $b = m_0 +\sum_{i < d, i \in 2\N} r_i$. $p_{\alpha_{d-1}}$ is a Morse function on $V_{d-1}^\alpha(a)$ with Morse index $b$.
\end{proposition}
\begin{proof}
Let $x$ be a critical point which we assume without loss of generality to have only nonnegative coordinates. By Lemma \ref{lem:critical points} we can assume
$$ x= (\underbrace{0,\ldots,0}_{m_0},\underbrace{x_1,\ldots,x_1}_{r_1},\ldots,\underbrace{x_{d-1},\ldots,x_{d-1}}_{r_{d-1}},x_1,\ldots,x_{d-1})$$ for some positive pairwise distinct $x_i$'s. Let $$\tilde{x} = (\underbrace{0,\ldots,0}_{m_0},\underbrace{x_1,\ldots,x_1}_{r_1},\ldots,\underbrace{x_{d-1},\ldots,x_{d-1}}_{r_{d-1}}) \in \R^{n-d+1}$$ denote the point consisting of the first $n-d+1$ coordinates of $x$. Notice  that since $x_1,\ldots, x_{d-1}$ are pairwisely distinct, the last $d-1$ columns of the associated Jacobian will be of full rank.  Thus, the first $n-d+1$ coordinates can be used as a system of local coordinates for $V_{d-1}^\alpha(c)$ in a neighborhood of $x$.
%
Since $x$ is a critical point we know from the proof of Lemma \ref{lem:critical points} that every coordinate of $x$ satisfies the same  univariate polynomial equation  $f(x_i)=0$.
By  the intermediate value theorem  the roots of the  derivative $f'(t)$ interlace the roots of  $f$. Noting that the leading coefficient of $g$ is positive, we find that the function values of the derivative $f'$ at the roots of $f$ satisfy 
\[f'(x_{d-1}) > 0, f'(x_{d-2}) < 0,f'(x_{d-3}) >0, \ldots, (-1)^{q} f'(x_1) < 0, (-1)^{q} f'(0) > 0,\]
where $q=d-1 \mod 2$. Now, since we have that the  Hessian of Lagrange function in \eqref{eq:largange}  \[\frac{\partial^2 L}{\partial X_i \partial X_j} = 0 \text{ for }i \neq j,\text{ and  }\frac{\partial^2 L}{\partial X_i \partial X_i} = f'(x_i),\]we can conclude that  the Hessian of $p_{\alpha_{d-1}}$ on $V_{d-1}^\alpha(c)$ at $x$ has indeed the claimed form.
\end{proof}



Thus, we immediately obtain:
\begin{corollary} \label{cor:pap3 1}
Let $x$ be a critical point of $p_{\alpha_{d-1}}$ on $V_{d-1}^\alpha(c)$. Then $x$ is a strict local minimum/ maximum if $x$ is of type (1)/(2) if $d$ is odd and of type (2)/(1) if $d$ is even.
 \end{corollary}



\begin{lemma}\label{lem:interval}
Let $n \geq d \geq 2$. The image of the function $p_{\alpha_{d-1}} : \R^n \rightarrow \R$ on the set $V_{d-1}^\alpha(c)$ is either empty or an interval for all $c \in \R^{d-1}$. 
\end{lemma}
 The proof of this Lemma essentially follows from the following statement, which originally had been shown bu Givental \cite[p. 275]{givental1987moments} for the case of Vandermonde varieties $V_{d-1}^{\alpha}(c)$, with $\alpha=(2,\ldots,d)$.

\begin{proposition}\label{prop:1}
For generic  $c$ the set  \[\left\{0\leq x_1\leq\ldots\leq x_n\right\}\,\bigcap\, V_{d-1}^{\alpha}(c)\] is either  contractible or empty.
    \end{proposition} 
Although the statement in Givental's article \cite{givental1987moments} is not stated in this generality the proof follows verbatim using  Proposition \ref{prop2} and the following general insight the  on the local topology of functions with non-degenerate Hessians.
\begin{proposition}\cite[Lemma 2]{givental1987moments}
The reconstruction of the topology of a level set of a function
$f$ on $\R^a_+\times \R^b_-$ in the neighborhood of the critical point $(0, 0)$ with non-degenerate Hessian $F = Q_{+} + Q_{-}$ is trivial if $a, b > 0$ and consists of the birth (death) of a
simplex otherwise.
\end{proposition}
A reader interested in more details about the above statements might also consult \cite{rainer2004perturbation,meguerditchian1992theorem} for more detailed proofs. With these preparations the proof of Lemma \ref{lem:interval} follows almost directly.
\begin{proof}[Proof of Lemma \ref{lem:interval}]
We remark first that it is sufficient to show the claimed statement for generic $c \in \R^{d-1}$ since it follows from \cite[Lemma~2.6]{kostov1989geometric} that from the generic case the statement follows for all $c$. Thus, we can assume that $c$ is generic.
 Then, following Proposition \ref{prop:1} the image of $p_{\alpha_{d-1}}$ on the restriction is connected and compact and thus, if it is  non-empty,  an interval. 
\end{proof}




We are now in the position to give the  proof of Theorem \ref{thm:1}. 
\begin{proof}[Proof of Theorem \ref{thm:1}] We conduct the proof in three steps starting with integer exponents and then deal with positive rational and general positive exponents.  
\begin{enumerate}[leftmargin=*]
    \item First, we suppose $\alpha \in \Z_{\geq 2}^{d-1}$ is an integer exponent vector. 
Any point of type (1) or (2) with $d-1$ distinct non-zero coordinates is indeed mapped to the boundary by Corollary \ref{cor:pap3 1}. However, any point of type (1) or (2) with less than $d-1$ distinct non-zero coordinates is then also mapped to the boundary by continuity. 
Now, we assume that $c := (p_{\alpha_1}(x),\ldots,p_{\alpha_{d-1}}(x))$ is contained in the boundary of the set $\Pi_{n,\alpha}$. If $c$ is non-singular the set $p_{\alpha_{d-1}}(V_{d-1}^\alpha (c) )$ is an interval by Lemma \ref{lem:interval}. We observe that $p_{\alpha_{d-1}}$ is either minimized or maximized at $x$. We can apply Corollary \ref{cor:pap3 1} and obtain that $x$ must be of type (1) or (2). If $c$ is a singular point then $x$ can be obtained as the limit of a sequence of such points. 


    \item Second, we suppose $\alpha \in \mathbb{Q}_{> 1}^{d-1}$ is a rational exponent vector. There exists $q \in \N$ such that $\beta := q \alpha \in \Z_{\geq 2}^{d-1}$ is an integer exponent vector and $\nu_{n,\alpha}(x)=\nu_{n,\beta}(x^{1/q})$ for all $x \in \R_{\geq 0}^n$, where $x^{1/q} = (x_1^{1/q},\ldots,x_n^{1/q})$. We already know that the claim for the Vandermonde cell $\Pi_{n,\beta}$. However, since $\nu_{n,\beta}$ is weighted homogeneous, we have an analogous description of $\bd \nu_{n,\beta}(\R_{\geq 0}^n) = \nu_{n,\alpha}(\R_{\geq 0}^n)$. This shows the claim for $\Pi_{n,\alpha}$ since also $\nu_{n,\alpha}$ is weighted homogeneous with rational weights.

    \item Third, we suppose $\alpha \in \mathbb{R}_{> 1}^{d-1}$ is a real exponent vector. Then there exists a sequence of rational exponents $\alpha^{m} \in \mathbb{Q}_{> 1}^{d-1}$ which converges to $\alpha$. We have $c \in \bd \Pi_{n,\alpha}$ if and only if there exists a sequence $c_m \in \bd \Pi_{n,\alpha^n}$ with $c_m \to c$ for $m \to \infty$. Then the claim follows by continuity. 
\end{enumerate} 
\end{proof}


\begin{remark}
The description of the boundary of the $(n,\alpha)$-Vandermonde cell in Theorem \ref{thm:1} transfers to $\alpha$-Vandermonde maps with positive weight vector $w \in \R_{>0}^n$, i.e. to maps $$x \mapsto (w_1x_1^{\alpha_1}+\ldots+ w_nx_n^{\alpha_1},\ldots,w_1x_1^{\alpha_{d-1}}+\ldots+ w_nx_n^{\alpha_{d-1}})$$
This is since the Jacobi matrix of a weighted $\alpha$-Vandermonde map differs only by positive scalars of the Jacobi matrix of an $\alpha$-Vandermonde map and thus of a generalized Vandermonde matrix.
\end{remark}



%\subsection{Boundary of the Vandermonde Cell $\Pi_{n,3}$.} \label{subsection power sum and elementary}
\subsection{Boundary of the Vandermonde Cell $\Pi_{n,3}$.} \label{subsection power sum and elementary}
In this subsection we investigate planar parametrizations of $\bd \Pi_{n,3}$. 
 Newton's identities imply $\Pi_{n,d} \simeq E_{n,d}$ up to a polynomial diffeomorphism for $n \geq d$. Thus, the parametrization transfers to the image in elementary symmetric polynomials. 



\begin{theorem} \label{thm:Parametrization Nnd}
For $n \geq 3$, a parametrization of
$\bd \Pi_{n,3}$ is given by the following $n$ arcs. The upper part of the boundary is parametrized by the arc 
\begin{align}\label{eq:upper part}
   \left(\frac{(1-t)^2}{n-1}+t^2,\frac{(1-t)^3}{(n-1)^2}+t^3\right) : \frac{1}{n} \leq t \leq 1
\end{align}
while the lower part is parameterized by the $n-1$ arcs
\begin{align}\label{eq:lower part}
      \left( \left(\frac{(1-t)^2}{n-k-1}+t^2,\frac{(1-t)^3}{(n-k-1)^2}+t^3\right) : 0 \leq t \leq \frac{1}{n-k}\right)_{0 \leq k \leq n-2}~.
\end{align}
\end{theorem}
\begin{proof}
We apply Theorem \ref{thm:1} to determine the boundary. The boundary consists of the closure of the set of all point evaluations in $(p_2,p_3)$ at all points $(0,\ldots,0,x_1,\ldots,x_1,x_2,\ldots,x_2) \in \Delta_{n-1}$ of the form $0 < x_1 < x_2$ of type (1) or (2). \\
Note that any point of type (2) must be of the form $(a,\ldots,a,b)$ with $0 < a < b$ and $(n-1)a +b=1$. Thus, $a = \frac{1-b}{n-1}$, $\frac{1}{n} < b < 1$ and we observe that the upper part of the boundary is indeed parameterized by the curve in (\ref{eq:upper part}).


We note that there are essentially $n-1$ points of type (1). Namely, points of the form $$(\underbrace{0,\ldots,0}_{\# = k},a,\underbrace{b,\ldots,b}_{\# = n-k-1})$$
for $0 \leq k \leq n-2$ satisfying  $b=\frac{1-a}{n-k-1}$ and $a \leq \frac{1}{n-k}$. We obtain precisely the parametrizations (\ref{eq:lower part}) of the lower part of the boundary.
\end{proof}


\begin{figure}[h!]%
    \centering
    \subfloat%[\centering $\mathcal{E}_{2,3}$]
    {{\includegraphics[width=4cm]{graphics/p4p6n=3.jpeg} }}%
    \qquad
    \subfloat%[\centering $\mathcal{E}_{2,3}$]
    {{\includegraphics[width=4cm]{graphics/p4p6n=4.jpeg} }}%
    \qquad
    \subfloat%[\centering $\mathcal{E}_{2,6}$]
    {{\includegraphics[width=4cm]{graphics/p4p6nis5.jpeg} }}%
    \caption{The sets $\bd \Pi_{n,3}$ for $3 \leq n \leq 5$}%
    \label{fig: N33 and N53}%
\end{figure}


We immediately obtain parametrizations of the boundary of the sets $E_{n,3}$ by Newton's identities. See Figure \ref{fig: E33 and E53} for a visualisation of these boundaries. \begin{figure}[h!]%
    \centering
    \subfloat%[\centering $\mathcal{E}_{2,6}$]
    {{\includegraphics[width=4cm]{graphics/e33.jpeg} }}%
    \qquad
    \subfloat%[\centering $\mathcal{E}_{2,3}$]
    {{\includegraphics[width=4cm]{graphics/e53.jpeg} }}%
    \caption{The sets $\bd E_{n,3}$ for $n \in \{3,5\}$ }%
    \label{fig: E33 and E53}%
\end{figure}


Theorem \ref{thm:Parametrization Nnd} generalizes to a parametrization of the boundary of the set $$\{ (p_{k}(x),p_{m}(x)) : x \in \Delta_{n-1}\}$$ for $2 \leq k < m$. However, we note that the upper part of the boundary cannot be described by just one smooth parametrizitation. This is since there are essentially more points of type \emph{(1)} resp. \emph{(2)} than for $k=2$ and $m=3$, where there is only $1$. However, a careful analysis can lead to a description of the boundary.


\begin{example}
For $2 \leq k \leq 3$, the lower part of the boundary of the set $$\{(p_{k}(x),p_4(x)): x \in \Delta_3\}$$ is the union of the images of the following two parametrizations  
\begin{align}
\left(2s^k+t^k+(1-2s-t)^k,2s^4+t^4+(1-2s-t)^4\right) & : 0 \leq s \leq t\leq \frac{1}{2}-s ~, \label{eq:1first}\\
\left(s^k+t^k+\frac{1}{2^{k-1}}(1-s-t)^k,s^4+t^4+\frac{1}{8}(1-s-t)^4\right) & : 0 \leq s\leq t<\frac{1}{3}-\frac{1}{3}s ~. \label{eq:1second}
\end{align}
Parametrization \emph{(\ref{eq:1first})} comes from the points with multiplicity vector $(x_1,x_1,x_2,x_3)$ and \emph{(\ref{eq:1second})} from the points $(x_1,x_2,x_3,x_3)$. \\
The upper part of the boundary is the union of the images of the following two parametrizations  
\begin{align}
    \left(s^k+2t^k+(1-s-2t)^k,s^4+2t^4+(1-s-2t)^4\right) & : 0 \leq s \leq t \leq \frac{1}{3}-\frac{1}{3}s ~, \label{eq:2first} \\
    \left( s^k+t^k+(1-s-t)^k,s^4+t^4+(1-s-t)^4 \right))& : 0 \leq s \leq t \leq    \frac{1}{2}-\frac{1}{2}t ~. \label{eq:2second}
\end{align}
Parametrization \emph{(\ref{eq:2first})} comes from the points with multiplicity vector $(x_1,x_2,x_2,x_3)$ and \emph{(\ref{eq:2second})} from those with $(0,x_1,x_2,x_3)$.
\end{example}


In the transition from $\Pi_{n,3}$ to $\Pi_{n+1,3}$ in Theorem \ref{thm:Parametrization Nnd} the arc describing the upper part of the boundary grows and converges. Its limit has the parametrization $(t,t^{3/2})$, $0 \leq t \leq 1$. Moreover, any point on the lower part of the boundary for $n$ remains on the boundary for $n+1$, but a single new smooth curve is added. Namely, the smooth curve with parametrization 
$$ \left(\frac{(1-t)^2}{n}+t^2,\frac{(1-t)^3}{n^2}+t^3\right) ~, ~0 \leq t \leq \frac{1}{n+1}~. $$



\begin{figure}[h!]%
    \centering
    {{\includegraphics[width=4cm]{graphics/p2p4nis20.jpeg} }}%
    \caption{The boundary of the set $\Pi_{20,3} $}%
    \label{fig:2 N203}%
\end{figure}



\begin{corollary} \label{cor:boundaryN3}
The boundary of the set $\Pi_{3} $ equals 
$$\left\{(t,t^{3/2}) : 0 \leq t \leq 1\right\} \cup \bigcup_{k \in \N_{>1}} \left\{ \left(\frac{(1-t)^2}{k}+t^2,\frac{(1-t)^3}{k^2}+t^3\right) : 0 \leq t \leq \frac{1}{k+1} \right\} ~.$$
\end{corollary}
We note that two different parametrizations of the lower part of the boundary
\begin{align*}
    &\left(\frac{(1-t)^2}{k}+t^2,\frac{(1-t)^3}{k^2}+t^3\right)~ : ~0 \leq t \leq \frac{1}{k+1} \mbox{ and } \\ 
    &\left(\frac{(1-s)^2}{l}+s^2,\frac{(1-s)^3}{l^2}+s^3\right) ~:~ 0 \leq s \leq \frac{1}{l+1}
\end{align*}
intersect if and only if $k = l-1$ or $k=l+1$. Without loss of generality be $k=l-1$. The intersection is a point and the curves meet at $\left( \frac{1}{l},\frac{1}{l^2}\right)$ for $t = \frac{1}{l}$ and $s=0$. Moreover, the gradients $(0,0)$ and $(\frac{-2}{l},\frac{-3}{l})$ differ at this point which shows that $\left( \frac{1}{l},\frac{1}{l^2}\right)$ is a singular point of $\bd \Pi_3$.   

\begin{corollary} \label{cor:N3 infinite many }
The limit Vandermonde cell $\Pi_{3}$ has countably infinitely many isolated singular points which are the points of the form $$\left(\frac{1}{k},\frac{1}{k^2}\right), ~k \in \N_{>0} \mbox{ and } (0,0)~.$$ 
\end{corollary}

\begin{proof}
It follows from the discussion above that only neighboring parametrizations of the lower part of the boundary intersect and their intersection point is a singular point of the boundary. The intersection points are all of the form $ \left(\frac{1}{k+1},\frac{1}{(k+1)^2}\right)$ for all $k \in \N$. However, $(1,1)$ is an intersection of the parametriztation $(t,t^{3/2}), 0 \leq t \leq 1$ of the upper part of the boundary and $((1-s)^2+s^2,(1-s)^3+s^3) : 0 \leq s \leq 1/2$ of the lower part. For $t = 1$ and $s=0$, but again the gradients are different which shows that $(1,1)$ is a singular point.  \\
Moreover, any singular point must be an intersection of two parametrizations. But the intersection points are precisely the points of the claimed form and the limit point $(0,0)$. \\
Since all the singular points lie in the rational moment curve $(t,t^2)$ the points are indeed isolated (see e.g. (\cite[Chapter II.9.]{barvinok2002course})).
\end{proof}


\begin{corollary}\label{cor:not semialgebraic}
The sets $\Pi_{d}$ and $E_d$ are not semialgebraic for all $d \geq 3$.
\end{corollary}
\begin{proof}
We show that $\Pi_3$ is not semialgebraic. Then for $d \geq 4$ the set $\Pi_d$ is not semialgebraic, since for $d\geq 3$ we have $\Pi_3 = \pi (\Pi_d)$, where $\pi : \R^d \to \R^2$ denotes the projection onto the first $2$ coordinates. Moreover, $E_d$ is a polynomial image of the set $\Pi_d$ which must also be non semialgebraic.\\
\indent Suppose hat the set $\Pi_3$ is semialgebraic. However, by Corollary \ref{cor:N3 infinite many } the semialgebraic set $\Pi_3$ has countably infinitely many isolated singular points. Let $T$ denote the union of these singular points. 
The union of all singular points of a semialgebraic set is again semialgebraic since this condition can be formalized as the vanishing and non-vanishing of certain polynomial equalities. Thus, $T$ is semialgebraic. By (\cite[Theorem 2.4.4]{bochnak2013real}) every semialgebraic set is the disjoint union of a finite number of semialgebraically connected semialgebraic sets. However, there are countable infinitely isolated points in $T$ which contradicts $T$ being semialgebraic. In particular, $\Pi_3$ cannot be semialgebraic.
\end{proof}



\section{Combinatorial properties of the boundary of $\Pi_{n,d}$} \label{sec:comb prop}


Our main result in this section is the following theorem, which provides a combinatorial description of the boundary of the Vandermonde cell. 

\begin{theorem} \label{thm:combinatorially cyclic polytope}
The set $\Pi_{n,d}$ has the combinatorial structure of the cyclic polytope $C(n,d-1)$. 
\end{theorem}

Cyclic polytopes are well studied objects in polyhedral combinatorics.

\begin{definition}
For $n > d \geq 2$ the \emph{cyclic polytope} $C(n,d)$ is the convex polytope with $n$ vertices which are points on the real $d$-dimensional moment curve $(t,t^2,\ldots,t^d)$. 
\end{definition}
The combinatorial structure, e.g. the $f$-vector, of $C(n,d)$ is independent of the chosen points and its boundary is a $d-1$-dimensional \textit{simplicial polytope}. Thus, we can speak about \textit{the} cyclic polytope $C(n,d)$. 
Cyclic polytopes are interesting objects. For instance, the upper bound theorem says that $C(n,d)$ has the component-wise maximal $f$-vector among all $d$-dimensional convex polytopes with $n$ vertices \cite{mcmullen1970maximum,stanley1975upper}. We refer to (\cite[Section 0]{ziegler2012lectures}) for more background on cyclic polytopes. For all $n \geq d\geq 3$ we have $\conv \left\{ \left( \frac{1}{k},\frac{1}{k^2},\ldots,\frac{1}{k^{d-1}} \right) : k \in [n] \right\}$ is the cyclic polytope $C(n,d-1)$ and this is the choice of vertices of $C(n,d-1)$ we will usually use.

\begin{definition}
   A set $S \subset \R^d$ has the \emph{combinatorial structure} of the cyclic polytope $C(n,d)$ if there exists a homeomorphism $\Phi : \bd C(n,d) \to \bd S$ which is a diffeomorphism when restricted to the relative interior of any face of $\bd C(n,d)$. The \emph{vertices} of $S$ are the images of the vertices of $C(n,d)$. \\
 We call a set $S \subset \mathbb{R}^n$ a \emph{curved $m$-simplex} if $S$ is the image of $\Delta_m$ under a continuous map $f$, such that $f$ is a diffeomorphism when restricted to the relative interior of any face of $\Delta_m$. The \emph{vertices} of a curved $m$-simplex are the images of the vertices of the simplex $\Delta_m$. 
\end{definition}


Note that the boundary of a set which has the combinatorial structure of a cyclic polytope is the gluing of $\# \{\text{facets of } C(n,d)\}$ many patches and each patch is a curved $d$-simplex such that the vertices of the patches are labelled by Gale's evenness condition. Moreover, patches of the boundary intersect if and only if the intersection of their sets of vertices is non-empty. \\




 The facets of a cyclic polytope $C(n,d)$ are characterized by \textit{Gale's evenness condition}. For an integer $k$ we write $\hat{k} := \left( \frac{1}{k},\frac{1}{k^2},\ldots,\frac{1}{k^{d}} \right)$.
 
\begin{theorem}[\cite{gale1963neighborly}] \label{thm:facets of cyclic polytope}
The facets of $C(n,d)$ are precisely given by all $\{ \hat{k} : k \in S\}$, where $S \subset [n]$ is any set of size $d$ satisfying 
\begin{enumerate}
    \item[i)] If $d$ is even, then $S$ is either a disjoint union of consecutive pairs $\{i,i+1\}$, or a disjoint union of consecutive pairs $\{i,i+1\}$ and $\{1,n\}$.
    \item[ii)] If $d$ is odd, then $S$ is a disjoint union of consecutive pairs $\{i,i+1\}$ and either the singleton $\{1\}$ or $\{n\}$.
\end{enumerate}
\end{theorem}
The standard formulation of Gale's evenness condition is the following.
Let $n > d$, $k_1 < \ldots < k_n \in \mathbb{R}$ and $T = \{\hat{k}_1,\ldots,\hat{k}_n\}$ be the vertices of $\conv \{ \hat{k}_i : 1 \leq i \leq n\}$. Then a set $T_d \subset T$ of size $d$ spans a facet of $C(n,d)$ if and only if any two elements in $T \setminus T_d$ are separated by an even number of elements from $T_d$ in the sequence $(k_1,k_2,\ldots,k_n)$. \smallskip


We briefly present an outline of our proof of Theorem \ref{thm:combinatorially cyclic polytope}. By Theorem \ref{thm:1} we have $\bd \Pi_{n,d} = \{ (p_2,\ldots,p_d)(x) : x \in \Delta_{n-1}, x_i \leq x_{i+1}, \forall i, x \text{ is of type } (1)\text{ or }(2)\}$. We associate the multiplicity vectors $r$ of type (1) and (2) with $d-2$-dimensional simplices $\Delta_*^r$ and the simplices correspond to the facets of $C(n,d-1)$ (see Proposition \ref{prop:simplices and cyclic polytopes}). We show that there are natural homeomorphisms $\Psi_r : \Delta_*^r \to \bd \Pi_{n,d}$ which are diffeomorphisms when restricted to the relative interior of any face of $\Delta_*^r$ in Lemma \ref{lem:homeom}. Then we show that there is a one-to-one map between the union of all the associated simplices $\Delta_*^r$ and $\bd \Pi_{n,d}$ in Theorem \ref{thm:one-to-one}. \smallskip


For a sequence $r = (r_0,r_1,\ldots,r_{d-1}) \in \N^d$ we define the $d-2$-dimensional simplex $$\Delta_{*}^r := \{ (\underbrace{0,\ldots,0}_{r_0},\underbrace{x_1,\ldots,x_1}_{r_1},\ldots,\underbrace{x_{d-1},\ldots,x_{d-1}}_{r_{d-1}}) \in \R_{\geq 0}^n : x_i \leq x_{i+1} \forall i, \sum_{i=1}^{d-1}r_ix_i = 1\}$$  and the map $$ \abb{\psi_{r}}{ \Delta_*^r}{ \R_{\geq 0}^{d-1}}{ z }{(p_{2}(z),\ldots,p_{d}(z))}\, .$$



\begin{lemma} \label{lem:homeom}
Let $r \in \N^{d}$ with $\sum_{i = 0}^{d-1} r_i = n$. Then the map $ \psi_{r} : { \Delta_*^r} \to { \Img (\psi_{r})}$ is a homeomorphism and a diffeomorphism when restricted to the relative interior of any face of $\Delta_*^r$.
\end{lemma}
\begin{proof}[Proof of Lemma \ref{lem:homeom}]
First, we want to show that $\psi_r$ is a homeomorphism.  It follows from (\cite[Theorem~1]{mas1979homeomorphisms}) that we only need to verify that the Jacobian of $\psi_r$ is positive on the interior of $\Delta_*^r$ and positive on the restriction of $\psi_{r}$ to any face. Note, this is true since $$ \det (\operatorname{Jac } \psi_{r} ) = r_1r_2 \cdots r_{d-1} \cdot (d-1)! \prod_{1 \leq i <  j \leq d-1} (x_i-x_j)$$ is a positive scalar of the Vandermonde determinant and the Vandermonde matrix is totally positive \cite{stembridge2002concise}, i.e. all the minors are positive. \\
Second, we show that $\psi_r$ restricted to the relative interior of any face is a diffeomorphism. 
Since the Jacobian of $\psi_r$ restricted to the relative interior of any face of $\Delta_*^r$ is always non-singular a local inverse of the Jacobi matrix exists, by the inverse function theorem. However, the Jacobi matrix at any point in the interior of $\Delta_*^r$ is a one-to-one map and differentiable in any local neighborhood. Thus, the local inverse must be a global inverse and $\psi_r$ is a diffeomorphism when restricted to the relative interior of any face.
\end{proof}

\begin{corollary} \label{cor:deformed facets}
Each simplex $\Delta_*^r$ is mapped to a curved $d-2$-simplex in $\R^d$ via $\psi_{r}$ and the vertices of the curved simplex $\psi_r(\Delta_*^r)$ are the points $(\frac{1}{k},\frac{1}{k^2},\ldots,\frac{1}{k^{d-1}})$ for all vertices $\overline{k}$ of $\Delta_*^r$. 
\end{corollary}


It follows from Theorem \ref{thm:1} that the set $\bd \Pi_{n,d}$ is the union of the curved simplices $\psi_{r}( \Delta_*^r)$ for all multiplicity vectors of type (1) and (2). We still have to show that the vertices of each simplex satisfy Gale's evenness condition and the curved simplices are indeed correctly arranged patches of $\bd \Pi{n,d}$.\\

Recall, a multiplicity vector $r$ of type (1) has the form $r_0 \geq 0$, $r_{2k-1}=1$ and $r_{2k} \geq 1$ for all $1 \leq k \leq \lfloor \frac{d-1}{2}\rfloor$ and a multiplicity vector of type (2) is of the form $r_0=0, r_{2k-1} \geq 1$ and $r_{2k}=1$ for all $1 \leq k \leq \lfloor \frac{d-1}{2} \rfloor $. For $1 \leq k \leq n$ we write $\overline{k} := (\underbrace{0,\ldots,0}_{n-k},\underbrace{\frac{1}{k},\ldots,\frac{1}{k}}_k)$.



\begin{lemma}  \label{lem:gale}
Let $r \in \N^d$ with $\sum_{i=0}^{d-1}=n$ be a multiplicity vector of type (1) or (2). The vertices of $\Delta^r_*$ are 
\[ 
\left\{ \begin{array}{cc}
 \overline{n-r_0},\overline{n-r_0-1},\overline{n-r_0-r_2-1},\overline{n-r_0-r_2-2},\ldots     &  \text{ if $r$ is of type $(1)$} \\
 \overline{n},\overline{n-r_1},\overline{n-r_1-1},\overline{n-r_1-r_3-1},\overline{n-r_1-r_3-2},\ldots     &   \text{ if $r$ is of type $(2)$}
\end{array} \right.
\]
\end{lemma}
\begin{proof}
The vertices are the points where all but one of the defining inequalities of the simplex $\Delta_*^r$ are tight. Thus, the vertices are the $d-1$ points for which $x_i = 0, 1 \leq i \leq k-1$ and $x_k = \ldots = x_{d-1}$ for all $0 \leq k \leq d-2$.    
\end{proof}

In the following Proposition we observe that multiplicity vectors of type (1) and (2) encode Gale's evenness condition with the identification $k \leftrightarrow \overline{k}$.

\begin{proposition} \label{prop:simplices and cyclic polytopes}
Let $r=(r_0,\ldots,r_{d-1}) \in \N^d$ with $\sum_{i=0}^{d-1}r=n$ be a multiplicity vector of type $(1)$ or $(2)$. Then, the set \[\{k \in [n] : \overline{k} \text{ vertex of } \Delta_*^r\}\]satisfies Gale's evenness condition. 
Moreover, any set $S \subset [n]$ of size $d-1$ satisfying Gale's evenness condition gives rise to a simplex $\Delta_*^s$ such that $|s|=n$ and the multiplicity vector $s$ is of type \emph{(1)} or \emph{(2)}.
\end{proposition}
\begin{proof}
We distinguish between $d-1$ odd and even. 
We first consider the case $d-1$ is odd. In this situation we have  by Lemma \ref{lem:gale} that the vertices of $\Delta_*^r$ are 
        \[ 
\left\{ \begin{array}{cc}
\overline{n}  \text{ and } \overline{r_{d-1}},\overline{r_{d-1}+1},\ldots,\overline{n-r_1-1},\overline{n-r_1}     &  \text{ if $r$ is of type $(1)$} \\
\overline{1} \text{ and } \overline{1+r_{d-1}},\overline{2+r_{d-1}},\ldots,\overline{n-r_0-1},\overline{n-r_0}  &   \text{ if $r$ is of type $(2)$}
\end{array} \right.
\]   
    The corresponding set of integers satisfies Gale's evenness condition in both cases. 
Conversely, we suppose $J \subset [n]$ with $|J| = d-1$ and $J$ satisfies Gale's evenness condition. Given a set $S \subset [n]$ we construct the associated multiplicity vector $s$.
\begin{enumerate}[leftmargin=*]
    \item First, we suppose $J = \biguplus_{j=1}^{\frac{d}{2}-1}\{i_j,i_j+1\} \uplus\{1\}$ and $1 < i_1 < \ldots < i_{\frac{d}{2}-1}<n$. We note $d-2$ is even and define $s_{d-2}:=i_1-1 \geq 1$ and $s_{d-2j} := i_j-i_{j-1}-1 \geq 1$ for all $1 < j \leq \frac{d}{2}-1$. Then, $$s_2+s_4+\ldots+s_{d-2} = i_{\frac{d}{2}-1}-\frac{d}{2}+1 \leq n-\frac{d}{2}.$$ We set $s_1 = s_3 = \ldots = s_{d-1} :=1$ and $ 0 \leq s_0 := n - s_1+s_2+\ldots+s_{d-1}.$ We note $s_1+\ldots+s_{d-1}  \leq n -\frac{d}{2} + \lceil \frac{d-1}{2}\rceil \leq n$. Thus, the vector $s$ is indeed of type (2) and the simplex $\Delta_*^s$ has vertex set $\{\overline{k} : k \in J\}$
    \item Second, we suppose $J = \biguplus_{j=1}^{\frac{d}{2}-1}\{i_j,i_j+1\} \uplus\{n\}$ and $1 \leq i_1 < \ldots < i_{\frac{d}{2}-1}<n$. We define $s_{d-1} := i_1$, $s_{d-2j+1}:=i_j-i_{j-1}-1 \geq 1$ for all $2 \leq j \leq \frac{d}{2}-1$. Then, $$s_{d-1}+s_{d-3}+\ldots+s_3 = i_{\frac{d}{2}-1}-\frac{d}{2}+2 \leq n-\frac{d}{2}.$$ We define $s_2 = s_4 =\ldots = s_{d-2} := 1$ and $s_1 := n-(s_{d-1}+s_{d-1}+\ldots+s_2) \geq 1$. We have $s$ is of type (2) and the simplex $\Delta_*^s$ has vertex set $\{\overline{k} : k \in J\}$. 
\end{enumerate}   

 
We now turn to the   case with $d-1$  even. 
Again by Lemma \ref{lem:gale} the vertices of $\Delta_*^r$ are         \[ 
\left\{ \begin{array}{cc}
\overline{r_{d-1}},\overline{r_{d-1}+1},\ldots,\overline{n-r_0-1},\overline{n-r_0}    &  \text{ if $r$ is of type $(1)$} \\
\overline{1},\overline{n} \text{ and } \overline{r_{d-2}+1},\overline{r_{d-2}+2},\ldots,\overline{n-r_1-1},\overline{n-r_1} &   \text{ if $r$ is of type $(2)$}
\end{array} \right.
\]   
    Also in this case  the corresponding set of integers satisfies Gale's evenness condition. Conversely, let $J \subset [n]$ be of size $d-1$ satisfying Gale's evenness condition. Given a set $S \subset [n]$ we construct the associated multiplicity vector $s$. 
\begin{enumerate}[leftmargin=*]
    \item First, we suppose \[J = \{1,n\} \uplus \biguplus_{j=1}^{\frac{d-3}{2}} \{ i_j,i_j+1\} \text{ with } < i_1 < \ldots < i_{\frac{d-3}{2}} < n-1.\] We define $s_{2} = s_4 = \ldots = s_{d-1} := 1$, $s_{d-2}:=i_1-1 \geq 1$, $s_{d-{2k}} := i_k-i_{k-1} -1 \geq 1$ for all $2 \leq k \leq \frac{d-3}{2}$, and $s_1 :=n-(s_2+\ldots+s_{d-1})\geq 1$. The vector $s$ is of type (2) and the simplex $\Delta_*^s$ has the vertex set $\{\overline{k} : k \in J\}$. 
    \item Second, we suppose \[J = \biguplus_{j=1}^{\frac{d-1}{2}}\{i_j,i_j+1\}\text{ and }1 \leq i_1 < \ldots < i_{\frac{d-1}{2}} < n.\] We set $s_1 = \ldots = s_{d-2}:=1$, $s_{d-1} := i_1 \geq 1$, $s_{d-2k+1} := i_k-i_{k-1}-1 \geq 1$ for all $2 \leq k \leq \frac{d-1}{2}$ and $s_0 := n-(s_1+\ldots+s_{d-1}) \geq 0$. Then, the vector $s$ is of type (1) and the vertex set of the simplex $\Delta_*^s$ is $\{\overline{k} : k \in J$\}. 
\end{enumerate}

\end{proof}



\begin{corollary}
    \label{cor:3.9}
    The map $$ \abb{\kappa_{n,d}}{ \bd \left( \conv \{ \left(\frac{1}{i},\ldots,\frac{1}{i^{d-1}}\right) : 1 \leq i\leq n \}\right)}{\bigcup_{r \text{ has type }(1),(2)}\Delta_*^r}{\sum_{j=1}^{d}\lambda_{i_j} \left( \frac{1}{i_j},\ldots,\frac{1}{i^{d-1}_j}\right) }{\sum_{j=1}^{d}\lambda_{i_j} \left(0,\ldots,0,\frac{1}{i_j},\ldots,\frac{1}{i_j}\right)} $$ is a homeomorphism and a diffeomorphism when restricted to the relative interior of any face of $\bd \left( \conv \{ \left(\frac{1}{i},\ldots,\frac{1}{i^{d-1}}\right) : 1 \leq i\leq n \}\right)$.
\end{corollary} 
The map $\bd C(n,d-1) \to \bd \Pi_{n,d}$ in Theorem \ref{thm:combinatorially cyclic polytope} will be the composition $\nu_{n,d} \circ \kappa_{n,d} $.
\begin{proof}
Since any facet of the cyclic polytope $\conv \{ \left(\frac{1}{i},\ldots,\frac{1}{i^{d-1}}\right) : 1 \leq i\leq n \}$ is the convex hull of $d-1$ points on the moment curve, these points are convexly independent. Moreover, the facet defining sets of vertices correspond to the multiplicity vectors $r$ of type (1) and (2) by Proposition \ref{prop:simplices and cyclic polytopes}. Thus the map $\kappa_{n,d}$ is well-defined. However, the map is clearly a homeomorphism and a diffeomorphism when restricted to the relative interior of any face of $\bd \left( \conv \{ \left(\frac{1}{i},\ldots,\frac{1}{i^{d-1}}\right) : 1 \leq i\leq n \}\right)$ since it is a linear map on any facet of $C(n,d-1)$.
\end{proof}


The following Theorem is used to show that the curved simplices $\psi_r(\Delta_*^r)$ can be arranged according to Gale's evenness condition as patches of $\bd \Pi_{n,d}$.

\begin{theorem}\label{thm:one-to-one}
    The Vandermonde map maps $\nu_{n,d}^{-1}(\bd \Pi_{n,d}) \cap \{x \in \R_{\geq 0}^n : 0 \leq x_1 \leq \ldots \leq x_n\}$ one-to-one to $\bd \Pi_{n,d}$.
\end{theorem}
Thus, any point in $\bd \Pi_{n,d}$ has a unique preimage in $\{x \in \Delta_{n-1} : 0 \leq x_1 \leq \ldots \leq x_n\}$.
Theorem \ref{thm:one-to-one} is actually an adaption of (\cite{kostov1989geometric},~Theorem~1.12.) and follows from restricting the domain of the Vandermonde map from $\{(x_1,\ldots,x_n)\in \R^n : x_1 \leq \ldots \leq x_n\}$ to its intersection with the nonnegative orthant. 


\begin{proof}[Proof of Theorem \ref{thm:one-to-one}]
The proof follows exactly the same steps as Kostov's proof of (\cite{kostov1989geometric}, Theorem 1.12.). Since the Vandermonde map is weighted homogeneous we actually can prove the claim for the domain $\{x \in \R^n : 0 \leq x_1 \leq x_2 \ldots \leq x_n\}$. %\cordian{This seems not to be consistent}
%\sebastian{Should be fine now}
Instead of considering $(p_1,\ldots,p_d)(\{x \in \R_{\geq 0}^n : x_i \leq x_{i+1}\})$ we consider $(p_2,\ldots,p_{2d})( \{x \in \R^n : 0 \leq x_1 \leq x_2 \ldots \leq x_n\})$. 

All statements in \cite{kostov1989geometric}  needed to prove an adapted version of (\cite{kostov1989geometric},~Theorem~1.12) are already proved in Subsection \ref{subsection boundary of the vandermonde cell} with the exception of (\cite{kostov1989geometric}, Theorem~1.8), i.e. the image of the closure of any $k \leq d$ dimensional stratum of $\{ x \in \R_{\geq 0}^n : 0 \leq x_1 \leq \ldots x_n\}$ under $(p_2,\ldots,p_{2d})$ is a stratified manifold and the graph of a $d-k$ dimensional vector function. However, working with even power sums restricted to $\{ x \in \R^n : 0 \leq x_1 \leq \ldots \leq x_n\}$ instead of the Vandermonde map on $\{x \in \R^n : x_1 \leq \ldots \leq x_n\}$, the determinant occurring in Kostov's proof of (\cite{kostov1989geometric}, Theorem~1.8) must be replaced by $\prod_{i=1}^kx_i \prod_{1 \leq q < r \leq k} (x_q^2-x_r^2)$ which is up to a positive scalar equal to the determinant of $(\frac{\partial p_{2i}}{\partial x_j})_{1 \leq i, j \leq k}$. This determinant vanishes at $a \in \{ x \in \R^n : 0 \leq x_1 \leq x_2 \leq \ldots \leq x_n\}$ if and only if $a$ is contained in the boundary of a stratum of $\{ x \in \R_{\geq 0}^n : 0 \leq x_1 \leq \ldots \leq x_n\}$. \\
This way we obtain an adapted version of Theorem (\cite{kostov1989geometric}, Theorem~1.8).
\end{proof}

We are ready to give a proof of Theorem 
\ref{thm:combinatorially cyclic polytope}. 

\begin{proof}[Proof of Theorem \ref{thm:combinatorially cyclic polytope}]
We suppose $C(n,d-1) = \conv \{ \left(\frac{1}{i},\ldots,\frac{1}{i^{d-1}}\right) : 1 \leq i\leq n \}$. By Corollary \ref{cor:3.9} the map $\kappa_{n,d} : \bd C(n,d-1) \to \bigcup_{r \text{ has type }(1),(2)}\Delta_*^r$ is a homeomorphism and a diffeomorphism when restricted to the relative interior of any face of $\bd C(n,d-1)$. \\
We consider the map $\nu_{n,d} \circ \kappa_{n,d} : \bd C(n,d-1) \to \bd \Pi_{n,d}$. The map $\nu_{n,d}$ is surjective by Theorem \ref{thm:1}.
On each simplex $\Delta_*^r$ the map $\psi_r$ is a homeomorphism and a diffeomorphism when restricted to the relative interior of any face by Lemma \ref{lem:homeom}. Thus, $\nu_{n,d} \circ \kappa_{n,d}$ is a diffeomorphism on the restriction of any face of $\bd C(n,d-1)$. The claim follows, since $\nu_{n,d}^{-1}(\bd \Pi_{n,d}) \to \bd \Pi_{n,d}$ is one-to-one by Theorem \ref{thm:one-to-one}. 
\end{proof}
We conclude the section with two observations.
\begin{remark}
    Although the set $\Pi_{n,d}$ has the combinatorial structure of a cyclic polytope the natural extension $\kappa$ of $\kappa_{n,d}$ to the interior of $C(n,d-1)$
    $$ \abb{\kappa}{\conv \{ \left(\frac{1}{i},\ldots,\frac{1}{i^{d-1}}\right) : 1 \leq i\leq n \}}{\Pi_{n,d}}{\sum_{j=1}^{d}\lambda_{i_j} \left( \frac{1}{i_j},\ldots,\frac{1}{i^{d-1}_j}\right) }{\nu_{n,d}\left(\sum_{j=1}^{d}\lambda_{i_j} \left(0,\ldots,0,\frac{1}{i_j},\ldots,\frac{1}{i_j}\right) \right)} $$
    is not well defined. For instance, $\frac{1}{13}(\frac{1}{2},\frac{1}{4})+\frac{12}{13}(\frac{1}{4},\frac{1}{16})= \frac{27}{52}(\frac{1}{3},\frac{1}{9})+\frac{25}{52}(\frac{1}{5},\frac{1}{25})$ but 
    \begin{align*}
    \begin{array}{llll}
        &\nu_{5,3}\left(1/13 \left(0,0,0,1/2,1/2\right)+12/13\left(0,1/4,1/4,1/4,1/4\right)\right) & =& \left(1, 85/338\right) \\
        &\nu_{5,3}\left(27/52\left(0,0,1/3,1/3,1/3\right)+25/52\left(1/5,1/5,1/5,1/5,1/5\right) \right) & =& \left(1,319/1352\right).
    \end{array}
    \end{align*}
    \end{remark}
    
        Similarly, we obtain from (\cite[Theorem 1.14]{kostov1989geometric}) that $\bd (\nu_{n,d}(\R^n))$ is a gluing of patches $$\nu_{n,d}(\{(\underbrace{x_1,\ldots,x_1}_{t_1},\ldots,\underbrace{x_{d-1},\ldots,x_{d-1}}_{t_{d-1}}) \in \R^n : x_1 \leq x_2 \leq \ldots \leq x_{d-1}\})$$ where $t_{2i} \geq 1, t_{2i-1} =1$ or $t_{2i}=1, t_{2i-1}  \geq 1$ for all $1 \leq i \leq d-1$. Moreover, $$ \bigcup_{t \text{ eligible}} \{(\underbrace{x_1,\ldots,x_1}_{t_1},\ldots,\underbrace{x_{d-1},\ldots,x_{d-1}}_{t_{d-1}}) \in \R^n : x_1 \leq x_2 \leq \ldots \leq x_{d-1}\} \to \bd \nu_{n,d}(\R^n)$$ is one-to-one. Recall that the patches on the boundary of the Vandermonde cell are of type (1) or (2), while the patches on $\bd (\nu_{n,d}(\R^n))$ are all of type (2). In general there are less patches than facets of the cyclic polytope $C(n,d-1)$. For instance, the facets of $C(4,3)$ correspond to $\{1,2,3\},\{1,3,4\},\{1,2,4\},\{2,3,4\}$. But the multiplicity patterns of points on the boundary of $\mu_{4,4}(\R^4)$ are $(x_1,x_1,x_2,x_3),$  $(x_1,x_2,x_3,x_3)$ and $(x_1,x_2,x_2,x_3)$ 
 
    


\section{The boundary at infinity} \label{sec:Vandermonde map limit image}

We show that the set $\bd \Pi_d$ is a gluing of countably infinitely many patches and each patch is a curved $(d-2)$-simplex. The vertices of any patch satisfy Gale's evenness condition. We begin with investigating properties of $\Pi_d$. \\
We write $\mathfrak{p}_k := \sum_{i \in \N} x_i^k$ for the power sum function in countably many variables. If $x \in \R^\N$ contains only finitely many non-zero coordinates we could also write $p_k(x)$ instead of $\mathfrak{p}_k(x)$ for a power sum polynomial in sufficiently many variables.  

 
\begin{lemma} \label{lem:bound}
For $x =(x_1,\ldots,x_{d-1})\in \Pi_d$ we have $(t^2x_1,\ldots,t^d x_{d-1}) \in \Pi_d$ for all $0 \leq t \leq 1$. 
\end{lemma}
\begin{proof}
Let $x = (p_2\ldots,p_d)(z)$ for a point $z \in \Delta_{n-1}$ and consider the point $z'$: $$z'=\left(tz,\frac{1-t}{n},\ldots,\frac{1-t}{n}\right) \in \Delta_{2n-1}.$$ We see that $$ \nu_{n,d}(z')= (t^{2}x_1,\ldots,t^{d}x_{d-1})+\left(\frac{(1-t)^{2}}{n},\ldots,\frac{(1-t)^{d}}{n^{{d-2}}}\right) \in \Pi_d,$$ which implies $(t^2x_1,\ldots,t^dx_{d-1}) \in \Pi_d$ for $n \to \infty$.
\end{proof}

Let $\Delta'_{n-1}$ be the convex hull of $\Delta_{n-1}$ and the origin, i.e. $\Delta'_{n-1}$ consists of all points with nonnegative coordinates, with the sum of coordinates at most $1$.


\begin{lemma}
The set $\Pi_d$ is the closure of the limit of the sets $\nu_{n,d}(\Delta'_{n-1})$ as $n$ goes to infinity, i.e. $\Pi_d = \clo \left(\bigcup_{n\geq d}\nu_{n,d}(\Delta'_{n-1})\right)$.
\end{lemma}
\begin{proof}
Clearly, we have $\Pi_d \subset \clo (\bigcup_{n\geq d}\nu_{n,d}(\Delta'_{n-1}))$. If $x=(x_1,\ldots,x_n) \in  \Delta'_{n-1}$, then %$x = \sum_{i=1}^{n+1}\lambda_i v_i$ for some $v_i \in \Delta_{n-1}$ and scalars $0 \leq \lambda_i$ with $\sum_{i=1}^{n+1}\lambda_i \leq1$. In particular, 
$\theta := \sum_{j =1}^n x_j \leq 1$ % = \sum_{j=1}^n\sum_{i=1}^{n+1} \lambda_i v_{ij} = \sum_{i=1}^{n+1} \lambda_i \leq 1$$ 
and thus $x^{[m]} := (x,(1-\theta)/m,\ldots,(1-\theta)/m) \in \Delta_{n+m-1}$. We have $$\Pi_d \ni \lim_{m \to \infty} \nu_{n,d}(x^{[m]}) = \nu_{n,d}(x)$$ and the remaining inclusion follows since $\Pi_d$ is closed.
\end{proof}

\begin{lemma}\label{lem:boundary as limit of sequences}
Let $ d \geq 3$. Then $p \in \bd \Pi_d$ if and only if there exists a sequence $(p_n)$ such that $p_n \in \bd \Pi_{n,d}$ and $p_n \to p$ as $n \to \infty$.
\end{lemma}
For a set $A \subset \R^n$ and a point $x \in \R^n$ we denote the \textit{distance} from $x$ to $A$ by $\operatorname{d}(x,A)$, i.e. $\operatorname{d}(x,A)  = \inf \{ ||x-a|| : a \in A\}$. 
\begin{proof}
First, we suppose that $p \in \bd \Pi_d$. Then, since the sets $\Pi_{n,d}$ are nested increasingly, we have $\operatorname{d}(p,\Pi_{n,d})$ is a decreasing sequence in $n$. However, $p \in \bd \Pi_d$ implies that $p \not \in \inti \Pi_{n,d}$ for all $n$, and thus $\operatorname{d}(p,\Pi_{n,d}) = \operatorname{d}(p,\bd \Pi_{n,d})$. Hence $\operatorname{d}(p,\bd \Pi_{n,d}) \to 0$ which implies that there exists a sequence $(p_n)_n$ with $p_n \in  \bd \Pi_{n,d}$ and $p_n \to p$. 

Now suppose that  $p \not \in \bd \Pi_d$. If $p \notin \Pi_d$, then clearly there does not exist a sequence of points in $\Pi_{n,d}$ approaching $p$. The only remaining case is $p \in \inti \Pi_d$. Suppose that for some $\varepsilon >0$ and some $n'$ we have $B_{\varepsilon}(p) \subset \Pi_{n',d}$. %Then we have $p \in \inti \Pi_d$.
Then, $ \varepsilon \leq \operatorname{d}(p,\bd \Pi_n) $ for all $n\geq n'$, which shows that there cannot exist a sequence $p_n \in \bd \Pi_{n,d}$ with $p_n \to p$. Since we have $\operatorname{d}(p,\Pi_{n,d})\to 0$, it follows that for any $\varepsilon >0$, the ball $B_{\varepsilon}(p)$ contains a boundary point of $\Pi_{n,d}$ for all $n$ sufficiently large.
\end{proof}



Recall that $\bd \Pi_{n,d}$ is a gluing of patches, where each patch is a curved $(d-2)$-simplex whose vertices satisfy Gale's evenness condition for the set $[n]$ by Theorem \ref{thm:combinatorially cyclic polytope}. We aim to show in Theorem \ref{thm:limit bd Gale} that the boundary of $\Pi_d$ consists of two types of patches. The first type comes from a finite subset $J$ of $\mathbb{N}$ which satisfies Gale's evenness condition. We call such patches \emph{stable patches}. Let $m$ be the maximal element of $J$. Note that the patch corresponding to $J$ is a boundary patch of $\Pi_{n,d}$ for all $n\geq m$, and therefore it lies on the boundary of $\Pi_d$. The second type comes from limits of patches of $\Pi_{n,d}$, and we call such patches \emph{limit patches}.\\

\indent For $n \geq d \geq 3$, we say that a set $J \subset [n]$ of size $d-1$ satisfying Gale's evenness condition contains $n$ as an \textit{end point}, if $J = \biguplus_{i \in \tilde{J}}\{i,i+1\} \uplus \{n\}$ or $J = \biguplus_{i \in \tilde{J}}\{i,i+1\} \uplus \{1,n\}$. Note that $d$ must be even in the first case and odd in the second case. %Analogously, we say that $J \subset [n]$ satisfying Gale's evenness condition contains $1$ as an end point.
Let $I \subset [m]$ be a set of size $d-1$ satisfying Gale's evenness condition which contains $m$ as an end point. Then, for $n \geq m$ we define $I_n := I \uplus \{n\} \setminus \{m\}$. The set $I_n \subset [n]$ also satisfies Gale's evenness condition and contains $n$ as an end point. We can take the limit of the patches $I_n$, and we say that the resulting limit patch corresponds to the set $I_\infty \subset \mathbb{N} \cup \{\infty\} $, where $I_\infty := I \uplus \{ \infty\} \setminus\{m\}$. We denote this limit patch by $P_{I_{\infty}}$. 
We note that a point $q$ belongs to $P_{I_\infty}$ if and only if there exists a set $I \subset [m]$  of size $d-1$ satisfying Gale's evenness condition which contains $m$ as an end point, and a sequence $(q_n)_{n \geq m}$ such that $q_n \in P_{I_n}$ for all $n\geq m$ and $q_n \to q$ for $n \to \infty$.

More formally, we can naturally extend Gale's evenness condition to subsets of  $\mathbb{N} \cup \{\infty\}$, and we see that $I_\infty$ does indeed satisfy Gale's evenness condition. Recall that the vertices of the patch of $\Pi_{n,d}$ corresponding to $J\subset [n]$ have the form $(\frac{1}{j},\frac{1}{j^2},\dots,\frac{1}{j^{d-1}})$ for $j\in J$. Therefore the endpoints of $I_n$ converge to $0\in \mathbb{R}^d$ as $n$ goes to $\infty$. 
For a finite finite set $J \subset \N \cup \{\infty\}$ we define 
\[P_J := (\mathfrak{p}_2,\ldots,\mathfrak{p}_d)( \conv \{ (0,\ldots,0,\underbrace{1/j,\ldots,1/j}_{j \text{ times}}) : j \in J \} ) 
\]
where we say that $1/j=0$ if $j=\infty$.

\begin{theorem}\label{thm:limit bd Gale}
The set $\bd \Pi_{d}$ is the union of all stable and limit patches. It consists of curved $d-2$-simplices $P_I$, where the index set $I \subset \N \cup\{\infty\}$ ranges over all sets of size $d-1$ which satisfy Gale's evenness condition %with respect to the end points $1,\infty$,
i.e. all sets $I \subset \N \cup \{\infty\}$ of size $d-1$ of the form 
\begin{align*}
    I = \biguplus_{i} \{i,i+1\} \mbox{ or } I = \{1,\infty\} \uplus \biguplus_{i} \{i,i+1\} \mbox{ or } I = \{\infty\} \uplus \biguplus_{i} \{i,i+1\} \mbox{ or } I = \{1\} \uplus \biguplus_{i} \{i,i+1\}.
\end{align*}
\end{theorem}
\begin{proof}
By Lemma \ref{lem:boundary as limit of sequences} the boundary of $\Pi_d$ consists of the points $q$ for which there exists a sequence $(q_n)_n$ with $q_n \in \bd \Pi_{n,d}$ and $q_n \to q$. In particular, if $I \subset \N \cup \{\infty\}$ satisfies Gale's evenness condition, we have $P_{I} \subset \bd \Pi_d$. To prove the theorem we need to show that any limit of a converging sequence $(q_n)_n$ with $q_n \in \bd \Pi_{n,d}$ is contained in a patch $P_I$ for a set $I \subset \N \cup \{\infty\}$ of size $d-1$ which satisfies Gale's evenness condition. 


Suppose that $(q_n)$ is a sequence with limit $q$, and $q_n \in P_{J_n}$ where $J_n \subset [n]$ satisfies Gale's evenness condition. We proceed by a case distinction.
\begin{enumerate}[leftmargin=*]
    \item There exists an integer $N$ and a subsequence of index sets $(J_{n_k})$ such that $J_{n_k} \subset [N]$ for all $k$. Then, by the pigeonhole principle $(J_n)$ contains a constant subsequence $(J)$. Since $P_J$ is closed, we have $q \in P_J $ and $J \subset \N \cup \{\infty\}$ satisfies Gale's evenness condition.
    \item There does not exist a subsequence of bounded index sets. Then there are two options: \\
    (a) The sequence $(\alpha_n)$, where $\alpha_n$ is the smallest element of $J_n$, has a subsequence which monotonously diverges to $\infty$. \\ % i.e. for all $N \in \N$ there exists an integer $n_N \geq N$ with $j \geq N$ for all $j \in J_{n_N}$. 
    (b) There exists an integer $K \in \N$ and a subsequence $(J_{n_k})$ of $(J_n)$ with $|[K] \cap J_{n_k}|=m$ is equal for all $k$ and the sequence $(\alpha_{n_k})$, where $\alpha_{n_k}$ is the smallest element of $J_{n_k} \cap [K]^c$, monotonously diverges to $\infty$. \\
    We investigate the cases 2) (a) and (b) below.
   % {\color{red} I am lost on what the distinction is above. We should explain this better. Is the first case, where all indices diverge to $\infty$?} \sebastian{yes that's supposed to be the first case. I reformulated (2)}
    \begin{enumerate}[leftmargin=*]
        \item We must have $q_{n_k} \to 0$ for $k \to \infty$. Recall that $0 \in P_I$ for any set $I \subset \N \cup \{\infty\}$ of size $d-1$ which contains $\infty$. In particular, there exists a set $I \subset \N \cup \{\infty\}$ of size $d-1$ of the form $$ I=\{\infty\} \uplus \biguplus_{i}\{i,i+1\} \mbox{ or } I = \{1,\infty\} \uplus \biguplus_i \{i,i+1\}$$ and $0 \in P_I \subset \bd \Pi_d$.
        \item In the second case we can restrict to a subsequence $(J_{n_\ell})$ of $(J_n)$ which intersection with $[K]$ is the same set for all $n_\ell \in \N$. This follows from the pigeonhole principle. We claim that we can extend the set $J_{n_\ell} \cap [K]$ to a set $I \subset \N \cup \{\infty\}$ of size $d-1$ satisfying Gale's evenness condition and $q \in P_I$. It must be $$ J_{n_\ell} \cap [K] = \biguplus \{i,i+1\} \mbox{ or } J_{n_\ell} \cap [K] = \{1\} \uplus \biguplus \{i,i+1\}$$ since $J_{n_\ell}$ satisfies Gale's evenness condition and $K+1 \not \in J_{n_\ell}$. We have to distinguish between $d$ even and odd. 
        \begin{enumerate}
            \item If $d-1$ is odd and $J_{n_\ell} \cap [K] = \biguplus \{i,i+1\}$ we consider $$I := \biguplus \{i,i+1\} \uplus \{\infty\} \uplus \biguplus \{j,j+1\}$$ for some (possibly none) large integers $j$ which is possible since $|J_{n_\ell} \cap [K]|$ is even and cannot satisfy Gale's evenness condition.
            \item If $d-1$ is even and $J_{n_\ell} \cap [K] = \biguplus \{i,i+1\}$ we know that $d-1 > |J_{n_\ell} \cap [K]|$. Let $k \leq K+1$ be minimal with $k \not \in J_{n_\ell} \cap [K]$. 
            \begin{enumerate}
            \item If $k=1$ we set $$I := \{1,\infty\} \uplus \biguplus \{i,i+1\} \uplus \biguplus \{j,j+1\}$$ for some (possibly none) large integers $j$.
            \item Otherwise 
            $$ I:= \{1,\infty\} \uplus \{2,3\} \uplus \ldots \uplus \{k-1,k\} \uplus \biguplus_{k < i  \leq K-1,\, i,i+1 \in J_n} \{i,i+1\} \uplus \biguplus\{j,j+1\}$$ for some (possibly none) large integers $j$. 
            \end{enumerate}
            In (A) and (B) we added an even number of integers to $J_{n_\ell} \cap [K]$. This is possible since $|J_{n_\ell}|$ is even and contains at least two integers larger than $K$.
      
       \item If $d-1$ is even and $J_{n_\ell} \cap [K] = \{1\} \biguplus \{i,i+1\}$ one proceeds analogously to (ii).
       \item If $d-1$ is odd and $J_{n_\ell} \cap [K] = \{1\} \biguplus \{i,i+1\}$ one proceeds analogously to (i).
       \end{enumerate}
    \end{enumerate}
    Finally, we point out that $q$ is indeed contained in $P_I$ since the limit of the sequence $q_n$ equals $(\mathfrak{p}_2,\ldots,\mathfrak{p}_d)(y)$ for an $y \in \conv \{ (0,\ldots,0),(0,\ldots,0,1/j,\ldots,1/j) : j \in J_{n_\ell} \cap [K]\}$. 
\end{enumerate}
\end{proof}


\begin{remark}
Sequences of patches of $\bd \Pi_{n,d}$ indexed by sets $I_n \subset [n]$ satisfying Gale's evenness condition which do not contain $n$ as a boundary point but contain $n$, i.e $\{n-1,n\} \subset I$ is a disjoint part of $I_n$, converge to lower dimensional cells in $\bd \Pi_d$. Any such lower dimensional cell is contained in a patch $P_I$ of $\bd \Pi_d$. 
\end{remark}






\section{Convex hull for elementary symmetrics and test sets for copositivity} \label{sec:Convex hull}
In this section we analyze the convex hulls of the sets $E_{n,d},\Pi_{n,d},E_d$ and $\Pi_d$. Although $ E_{n,d} \simeq \Pi_{n,d}$ and $E_d \simeq \Pi_d$ are diffeomorphic, we show that $\conv E_{n,d}$ has nice properties which are not shared by $\conv \Pi_{n,d}$. We relate the study of the convex hulls to copositivty of certain symmetric forms. The vertex representation of $\conv E_{n,d}$ can be reformulated in terms of test sets which geometrically explains and slightly generalizes the case $d=3$ investigated by Choi, Lam and Reznick \cite{choi1987even}.
\medskip

We embed $\Delta_{n-1} \subset \Delta_{n}$ via $a \mapsto (a,0)$, and denote by $\Delta :=  \clo \left( \bigcup_{n \in \N} \Delta_n \right)$ the \textit{infinite probability simplex} which can be viewed as the limit of the $\Delta_n$'s. For $n \geq d$ we write $\mathcal{E}_{n,d} := \conv E_{n,d} \text{ and }\mathcal{E}_{d} := \clo ( \bigcup_{n \geq d} \mathcal{E}_{n,d}  ).$

 The following observation about extreme points of $\mathcal{E}_{n,d}$ appeared for the first time in the context of extremal combinatorics. In the planar setting it was proven by Bollobás to give a description of the convex hull of the range of edge versus triangle densities of graphs \cite{bollobas1976relations}. The result was extended to larger dimensions shortly afterwards and new proofs appeared for instance also in \cite{foregger1987relative,kovavcec2012note,riener2012degree,linear}. The cyclic polytope observation appears to be new. %\greg{is that correct?} \sebastian{I did not see it anywhere in the literature}
\begin{theorem} \label{thm:Convex hull Image of Elementary}
The set $\mathcal{E}_{n,d} $ is a cyclic polytope and it
 is the convex hull of the following finite set of points $
  \left( {k \choose 2} \frac{1}{k^2},\ldots,{k \choose d} \frac{1}{k^d} \right):  k \in [n].
$

\end{theorem}
We present a short proof using the following two Lemmas. The following short proof is a formalization of Bollobás's original argument, which we borrow from \cite{zhao2022} and provide for completeness. %\greg{should we give the short proof of 5.2 here?} \sebastian{ok}
\begin{lemma}[\cite{zhao2022},~Lemma~5.4.3] \label{lem:zhao}
For $n \geq d$ a non-constant symmetric map of the form $ c_1e_1+c_2e_2+\ldots +c_de_d : \R^n \rightarrow \R$ attains its extremal values on $\Delta_{n-1}$ at points of the form $(0,\ldots,0,\underbrace{1/k,\ldots,1/k}_{k \text{ times}})$ for $1 \leq k \leq n$.
\end{lemma}

\begin{proof}
    Let $n \geq d$ and $\phi(e_2,\ldots,e_d) = c_1 + c_2e_2 + \ldots +c_n e_d :\R^n \rightarrow \R$ be an affine non-constant linear map on $\mathcal{E}_{n,d}$ and let $x^*$ be a mininizer of $\phi^* = \phi (e_2,\ldots,e_d)$ on $\Delta_{n-1}$. We show that $x^* = (1/k,\ldots,1/k,0,\ldots,0)$ up to permutation for a $1 \leq k \leq n$. If $x^*$ is not the vector containing only $0$'s and one $1$ we suppose without loss generality that $x_1,x_2 > 0$ and write $\phi^*(x) = x_1A+x_2B+x_1x_2C +D$, where $A,B,C,D$ are functions in $x_3,\ldots,x_n$. Then, since $\phi^*$ is symmetric we have $A = B$ and by fixing $x_1+x_2 = x_1^*+x_2^*$ we obtain $\phi^*(x)=x_1x_2C+D'$. If $C(x^*) \geq 0$ we set either $x_1 = 0$ or $x_2 = 0$ with holding $x_1+x_2 = x_1^* + x_2^*$ fixed and obtain that $x^*$ was not a minimum. If $C (x^*) < 0$ we obtain $\phi^*(x^*)$ is minimized at $x_1^*=x_2^*$. Iteratively, we must have $x^* = (1/k,\ldots,1/k,0,\ldots,0)$.
\end{proof}


\begin{lemma} \label{prop:affine isomorphism}
For $n \geq d$, the map $$\abb{\Phi_{d}}{\left\{ \left( {k \choose 2} \frac{1}{k^2},\ldots,{k \choose d} \frac{1}{k^d} \right):  k \in [n] \right\}}{\left\{ \left( \frac{1}{k},\frac{1}{k^2},\ldots,\frac{1}{k^{d-1}} \right) : k \in [n] \right\}}{\left( {k \choose 2} \frac{1}{k^2},\ldots,{k \choose d} \frac{1}{k^d} \right)}{\left( \frac{1}{k},\frac{1}{k^2},\ldots,\frac{1}{k^{d-1}} \right)} $$ induces an affine isomorphism $\R^{d-1} \to \R^{d-1}$.
\end{lemma}
\begin{proof}
Let $m \geq 2$ and $k \in [n]$ be integers and $z_m := (1,2,\ldots,m-1,0,\ldots,0) \in \R^n$. Then by Vieta's formula we have 
\begin{align*}
   {k \choose m} \frac{1}{k^m} %& = \frac{k!}{d!(k-d)!} \frac{1}{k^d} \\
   & = \frac{\prod_{i=1}^{m-1}(k-i)}{m!\cdot k^{m-1}} \\
   & = \frac{1}{m!\cdot k^{m-1}}(k^{m-1}-e_1(z_m)k^{m-2}\pm \cdots + (-1)^{m-1}(e_{m-1}(z_m)) \\
   & = \frac{1}{m!}-\frac{1}{2(m-2)!}\frac{1}{k}+\cdots+\frac{(-1)^{m-1}}{m}\frac{1}{k^{m-1}}
\end{align*}
which shows that for all $k \in [n]$ the same affine linear relation of the $m$-th coordinates of points in the sets ${\left\{ \left( {k \choose 2} \frac{1}{k^2},\ldots,{k \choose d} \frac{1}{k^d} \right):  k \in [n] \right\}}$ and ${\left\{ \left( \frac{1}{k},\frac{1}{k^2},\ldots,\frac{1}{k^{d-1}} \right) : k \in [n] \right\}}$ is satisfied.
\end{proof}

\begin{proof}[Proof of Theorem \ref{thm:Convex hull Image of Elementary}]
By Minkowski's theorem a compact, convex set is the convex hull of its extreme points. Extreme points of $\mathcal{E}_{n,d}$ are precisely the minima of affine linear maps on $\mathcal{E}_{n,d}$. \\
Through evaluating we obtain $(e_2,\ldots,e_d)(0,\ldots,0,1/k,\ldots,1/k)=\left( {k \choose 2} \frac{1}{k^2},\ldots,{k \choose d} \frac{1}{k^d} \right)$ for all $1 \leq k \leq n$. It follows from Lemma \ref{prop:affine isomorphism} that all the claimed points are indeed vertices of $\mathcal{E}_{n,d}$, since points on the moment curve are in convex position and that $\mathcal{E}_{n,d}$ is a cyclic polytope.
\end{proof}

Since $$ \lim_{k \to \infty}  \left(  {k \choose 2} \frac{1}{k^2},\ldots,{k \choose d} \frac{1}{k^d} \right) = \left(\frac{1}{2!},\ldots,\frac{1}{d!}\right)$$ we immediately obtain a description of the limit set $\mathcal{E}_d$. Figure \ref{fig:2 E220} visualizes how the additional vertices eventually accumulate around the point $\left( \frac{1}{2!},\frac{1}{3!}\right) $.


\begin{proposition} \label{cor:image infinite probability simplex}
$\mathcal{E}_{d} = \conv \left\{  \left\{ \left( {k \choose 2} \frac{1}{k^2},\ldots,{k \choose d} \frac{1}{k^d} \right) : k \in \N \right\}  \uplus \{ \left( \frac{1}{2!},\frac{1}{3!},\ldots,\frac{1}{d!} \right) \} \right\}. $
\end{proposition}
\begin{proof}
We observe that the set $\bigcup_{n \geq d} \mathcal{E}_{n,d}$ is convex: if $v,w \in \bigcup_{n \geq d} \mathcal{E}_{n,d}$ then for some integer $N$ we have $v,w$ are contained in the convex set $\mathcal{E}_{n,d}$, because the sets $\mathcal{E}_{n,d}$ are nested. Thus, $\mathcal{E}_d$ is convex as the closure of the convex set $\bigcup_{n \geq d} \mathcal{E}_{n,d}$. We note $$\left( {k \choose 2} \frac{1}{k^2},\ldots,{k \choose d} \frac{1}{k^d} \right), \left( \frac{1}{2!},\frac{1}{3!},\ldots,\frac{1}{d!} \right) \in \mathcal{E}_{d}$$ per definition and since $\mathcal{E}_d$ is closed. Thus, the set on the right hand side is contained in $\mathcal{E}_d$. Moreover, we have $\mathcal{E}_{n,d} \subset \conv \left\{  \left\{ \left( {k \choose 2} \frac{1}{k^2},\ldots,{k \choose d} \frac{1}{k^d} \right) : k \in \N \right\}  \uplus \{ \left( \frac{1}{2!},\frac{1}{3!},\ldots,\frac{1}{d!} \right) \} \right\}$ for all $n \geq d $ and thus $\clo \left( \bigcup_{n \geq d} \mathcal{E}_{n,d} \right) \subset \conv \left\{  \left\{ \left( {k \choose 2} \frac{1}{k^2},\ldots,{k \choose d} \frac{1}{k^d} \right) : k \in \N \right\}  \uplus \{ \left( \frac{1}{2!},\frac{1}{3!},\ldots,\frac{1}{d!} \right) \} \right\}$, since the set on the right-hand side is closed.
\end{proof}




\begin{figure}[h!]%
    \centering
    \subfloat%[\centering $\mathcal{E}_{2,3}$]
    {{\includegraphics[width=4cm]{graphics/e23.pdf} }}%
    \qquad
    \subfloat%[\centering $\mathcal{E}_{2,6}$]
    {{\includegraphics[width=4cm]{graphics/e26.pdf} }}%
    \caption{The sets $\mathcal{E}_{3,3}$ (left) and $\mathcal{E}_{6,3}$ (right)}%
    \label{fig:1 E23 and E26}%
\end{figure}



\begin{figure}[h!]%
    \centering
    {{\includegraphics[width=4cm]{graphics/e220.pdf} }}%
    \caption{The set $\mathcal{E}_{20,3}$}%
    \label{fig:2 E220}%
\end{figure}

We note
$\left( p_i \left( 0,\ldots,0,\frac{1}{k},\ldots,\frac{1}{k}\right)\right)_{2 \leq i \leq d} =  \left(\frac{1}{k},\frac{1}{k^2},\ldots,\frac{1}{k^{d-1}}\right)  $ 
and recall that also $\bd \Pi_{n,d}$ has isolated singularities at $(p_2,\ldots,p_d)(0,\ldots,0,1/k,\ldots,1/k)$ for all $1 \leq k \leq n$.


For $n \geq d \geq 4$ Newton's identities (\ref{eq:Newton's identities}) provide polynomial (but not linear) transition maps between $E_{n,d}$ and $\Pi_{n,d}$. Already the power sum $p_4$ is quadratic in $e_2$. However, Lemma \ref{prop:affine isomorphism} shows that for any degree $d \geq 2$ there still exists an isomorphism between the isolated singularities of $E_{n,d}$ and $\Pi_{n,d}$, i.e $$(e_2,\ldots,e_d)(0,\ldots,0,1/k,\ldots,1/k) \mapsto (p_2,\ldots,p_d)(0,\ldots,0,1/k,\ldots,1/k) \quad \text{ for } 1 \leq k \leq n$$ is an isomorphism.


We observed in Corollary \ref{cor:N3 infinite many } that $\Pi_{n,3} \subset \conv \{ (\frac{1}{k},\frac{1}{k^2}) : k \in [n]\}$.
So it seems natural to ask whether an analogous result to Theorem \ref{thm:Convex hull Image of Elementary} in terms of the power sums and the rational points on the moment curve generalizes to $d \geq 4$. We provide a negative answer.

\begin{figure}[h!]%
    \centering
    \subfloat%[\centering $\mathcal{E}_{2,3}$]
    {{\includegraphics[width=4cm]{graphics/p23.pdf} }}%
    \qquad
    \subfloat%[\centering $\mathcal{E}_{2,6}$]
    {{\includegraphics[width=4cm]{graphics/p26.pdf} }}%
    \caption{The convex polytopes $\conv \Pi_{n,3}$ for $n=3$ (left) and $n=6$ (right)}%
    \label{fig:3 P23 and P26}%
\end{figure}

\begin{proposition} \label{prop:not power sums}
Let $n \geq d \geq 4$. Then the set $\conv \left\{ \left( \frac{1}{k},\frac{1}{k^2},\cdots,\frac{1}{k^{d-1}} \right) : k \in [n] \right\}$ does not contain the set $\Pi_{n,d}$. Moreover, $\Pi_d \not \subset \conv \left\{(0,\ldots,0), \left( \frac{1}{k},\frac{1}{k^2},\cdots,\frac{1}{k^{d-1}} \right) : k \in \N \right\}$
\end{proposition}
\begin{proof}
We consider $f(p_1,\ldots,p_4) = 2p_4-3p_{(3,1)}+p_{(2,1^2)}$. For $n=m+1$ we have \[  g_m(a) := f(p_1,\ldots,p_4)(a,\underbrace{1,\ldots,1}_{\# 1's = m}) = 
-ma^3 + a^2 m^2 + a^2 m + 2 a m^2 - 3 a m + m^3 - 3 m^2 + 2 m.\] Thus, for fixed $m$ we observe that the univariate polynomial $g_m(a)$ has a negative leading coefficient which shows that for sufficiently large $a >0$ we must have $f(p_1,\ldots,p_4)(a,1,\ldots,1) < 0$. Therefore, $f$ cannot be nonnegative on $\R_{\geq 0}^n$ and since $f$ is homogeneous $f$ cannot be nonnegative on $\Delta_{n-1}$. \\
However, the form $f(1,p_2,p_3,p_4)$ is nonnegative on the rational points on the moment curve of the form $(1/k,1/k^2,1/k^3)$, since \[ f(1,1/k,1/k^2,1/k^3) = \frac{(k - 3/2)^2  - 1/4}{k^3} \geq 0\] for all $k \in \N$. \\ 
To conclude the proof we suppose $$  \Pi_{n,d} \subset \conv \left\{ \left( \frac{1}{k},\frac{1}{k^2},\cdots,\frac{1}{k^{d-1}} \right) : k \in [n] \right\}.$$ But since $f(1,p_2,p_3,p_4)$ is linear in the $p_i$'s and $f(1,1/k,1/k^2,1/k^3) \geq 0$ we have $f$ is nonnegative on $\conv \left\{ \left( 1,\frac{1}{k},\frac{1}{k^2},\cdots,\frac{1}{k^{d-1}} \right) : k \in [n] \right\}$ and thus nonnegative on $\Pi_{n,d}$ which is a contradiction. 
\end{proof}

\subsection{Test sets for copositivity}
Choi, Lam and Reznick investigated in their paper \cite{choi1987even} nonnegative even symmetric sextics in any number of variables $\geq 3$. They found finite test sets for nonnegativity. Note that any even symmetric sextic is of the form $f(p_2,p_4,p_6)=c_1p_2^3+c_2p_2p_{4}+c_3p_6$ for some $c_1,c_2,c_3$ $\in \R$. 
\begin{theorem}[\cite{choi1987even},~Theorem~3.7] \label{thm:choi-lam-reznick}
Let $f(p_2,p_4,p_6)$ be an even symmetric sextic in $n \geq 3$ variables. Then $f$ is nonnegative if and only if $f\left(1,\frac{1}{k},\frac{1}{k^2}\right)$ is nonnegative for all $k \in [n]$.
\end{theorem}
Choi, Lam and Reznick derived their result by induction and using trigonometric functions.
We want to geometrically explain 
 and slightly expand Theorem \ref{thm:choi-lam-reznick} to a certain set of symmetric polynomials. We call a symmetric polynomial \emph{hook-shaped} if it can be written as a linear combination of elementary symmetrics of the form $e_{(d-i,1^i)}$, i.e. $f = \sum_{i=1}^d c_ie_{(d-i,1^i)}$ for some scalars $c_i \in \R$. A hook-shaped symmetric polynomial is homogeneous and thus copositive if and only if it is nonnegative on the probability simplex $\Delta_{n-1}$. By setting $e_1=1$ we obtain linear polynomials in $e_2,\ldots,e_d$. Hence nonnegativity of $c_1+\sum_{i=2}^dc_ie_{d-i}$ on $E_{n,d}$ is equivalent to nonnegativity on the vertices of $\mathcal{E}_{n,d}$. Alternatively, we can consider even symmetric polynomials of the form $\sum_{i=1}^dc_ie_i(x^2)e_1(x^2)^{d-i}$ and present test sets for global nonnegativity.


\begin{theorem}\label{thm:discrete test set in elementarys}
Let $f$
be a hook-shaped symmetric form in $n \geq d$ variables. Then $f$ is copositive if and only if  $f\left(1,{k \choose 2}\frac{1}{k^2},\ldots,{k \choose d} \frac{1}{k^d}\right) $ is nonnegative for all $k \in[n]$. 
\end{theorem}
\begin{proof}
Since $f(e_1,e_2,\ldots,e_d)$ is homogeneous we can restrict to the domain $\Delta_{n-1}$ where $e_1$ is the constant $1$ function. As $f(1,e_2,\ldots,e_d)$ is linear in the remaining elementary symmetrics $e_i$ for $2 \leq i \leq d$, we observe that $f(e_1,\ldots,e_d)$ is copositive if and only if $f(1,x)$ is nonnegative $\mathcal{E}_{n,d}$. In particular, $f(1,x)$ is nonnegative on $\mathcal{E}_{n,d}$ if and only if $f(1,x)$ is nonnegative on the vertices of $\mathcal{E}_{n,d}$, which are precisely the claimed points by Theorem \ref{thm:Convex hull Image of Elementary}.
\end{proof}

 

\begin{corollary} \label{cor:limit test set}
Let $\mathfrak{f}(e_1,e_2,\ldots,e_d) = \sum_{i=1}^d c_ie_ie_1^{d-i} $
be a symmetric form. Then $\mathfrak{f}(e_1,\ldots,e_d)$ is nonnegative in any number of variables $\geq d$ if and only if $\mathfrak{f}$ is nonnegative on the discrete set $\left\{ \left(1,{k \choose 2}\frac{1}{k^2},\ldots,{k \choose d} \frac{1}{k^d}\right) \, : \, k \in \N \right\}$.
\end{corollary}





\begin{remark} \label{rmk:choi-lam-reznick}
A generalization of the discrete test sets in power sums to degrees $\geq 4$ cannot be given. This follows from Proposition \ref{prop:not power sums}.
\end{remark}


 Recall that for $d \geq 3$ the sets $\Pi_d$ and $E_d$ are not semialgebraic by Corollary \ref{cor:not semialgebraic}. Thus $\mathcal{E}_d$ cannot be semialgebraic as the convex hull of the set $E_{d}$. 




\subsection{The convex body $\mathcal{E}_d$}
We show that the convex body $\mathcal{E}_d$ behaves like an infinite cyclic polytope also from the dual point of view: it has countably infinitely many facets which can be described by Gale's evenness condition. 

The following results on $\bd \mathcal{E}_d$ can be derived completely analogously to our examination of $\bd \Pi_d$ in Section \ref{sec:Vandermonde map limit image}. We omit the proofs as they are even simpler, since we work with convex polytopes instead of curved simplices.



\begin{lemma} \label{lem:convex hull bd 1}
 $\bd \mathcal{E}_d = \{ q \in \R^{d-1} : \forall n \geq d ~\exists q_n \in \bd \mathcal{E}_{n,d} \emph{ with } q_n \to \infty \}. $
\end{lemma}
Let $\mathcal{I} \subset \mathcal{P}(\N \cup \{\infty\})$ denote the set of all sets $I \subset \N \cup \{\infty\}$ of size $d-1$ satisfying Gale's evenness condition with respect to the end points $1$ and $\infty$. Furthermore, for $I \in \mathcal{I}$ let $F_I$ denote the convex hull of $\{ \left( {k \choose 2} \frac{1}{k^2},\ldots,{k \choose d} \frac{1}{k^d} \right) : k \in I \}$, where we use the limit point $(\frac{1}{2!},\ldots,\frac{1}{d!})$ if $k = \infty$.
\begin{theorem} 
    $\bd \mathcal{E}_d = \bigcup_{I \in \mathcal{I}} F_I$.
\end{theorem}
\begin{corollary}
 The convex set $\mathcal{E}_d$ contains countably infinitely many facets indexed by Gale's evenness condition. 
\end{corollary}
\begin{proof}
The sets $F_I$ can only intersect on their boundary and not in their interior since they are cuts of hyperplanes.
\end{proof}




To conclude the section we briefly present a $H$-representation of $\mathcal{E}_{n,d}$ and show that the convex body $\mathcal{E}_d$ can be defined as the intersection of countably many halfspaces. We follow (\cite[Page~14]{ziegler2012lectures}) where the $H$-representation of a cyclic polytope is given. For $S=\{k_1,\ldots,k_{d-1}\} \subset [n]$ of size $d-1$ we define the linear map \[
\tilde{\ell}_S : \R^{d-1} \to \R, X \mapsto \det \left( \begin{array}{cccc}
    1 & 1 & \ldots & 1  \\
    X_1 & k_1 & \ldots & k_{d-1} \\
    \vdots & \vdots &   & \vdots \\
   X_{d-1} & k_1^{d-1} & \ldots & k_{d-1}^{d-1}
\end{array} \right)~.\]
By properties of the Vandermonde determinant we have $\tilde{\ell}_S(k,k^2,\ldots,k^{d-1}) = 0$ if and only if $k \in S$. Thus, the kernel of $\tilde{\ell}_S$ equals $\langle (k,k^2,\ldots,k^{d-1}) : k \in S \rangle_\R$ and the $H$-representation of $C(n,d-1)$ is given by inequalities of the form $\pm \tilde{\ell}_S (X) \leq r_S$ for all facet defining sets $S \subset [n]$ and some $r_S \in \R$. We write $\ell_S$ for $\tilde{\ell}_S$ multiplied by $-1$ to the correct power such that the inequality reads $\ell_S (X) \leq r_S$. 

\begin{proposition} \label{prop:H-repr. of Edn}
Let $n \geq d \geq 3$ be nonnegative integers and let $\mathcal{C}_{d-1}$ denote the collection of facet defining sets of $C(n,d-1)$. Then the $H$-representation of $\mathcal{E}_{n,d}$ is $\{ \ell_S \circ \Phi_{d} (X) \leq r_S ~: ~S \in \mathcal{C}_{d-1} \}.$ 
\end{proposition}
\begin{proof}
The claim follows from the discussion above and since $$\mathcal{E}_{n,d} = \Phi_{d}^{-1} (\conv \{ (1/k,\ldots,1/k^{d-1}) : k \in [n]\}) = \Phi_{d}^{-1} (\{ x \in \R^{d-1} : \ell_S (x) \leq r_S, S \in \mathcal{C}_{d-1}\}) $$ by Proposition \ref{prop:affine isomorphism}. We have
$\mathcal{E}_{n,d} = \{ x \in \R^{d-1} :\ell_S \circ \Phi_{d} (x) \leq r_S, S \in \mathcal{C}_{d-1} \}. $
\end{proof}



\begin{example}
Using Sage \emph{(\cite{stein2007sage})\emph} we calculate the H-representations of $\mathcal{E}_{n,3}$ for $ 3 \leq n \leq 5$.
\begin{align*}
    \mathcal{E}_{3,3} & = \{ x \in \R^2 : x_2 \geq 0,~ x_1-9x_2 \geq 0,~-4x_1+9x_2\geq -1 \}, \\
     \mathcal{E}_{4,3} & = \{ x \in \R^2 : x_2 \geq 0,~ x_1-6x_2 \geq 0,~ -11x_1+18x_3\geq-3, ~-4x_1+9x_2 \geq -1 \}, \\
      \mathcal{E}_{5,3} & = \{ x \in \R^2 : x_2 \geq 0,~ -4x_1+9x_2 \geq -1, ~-11x_1+18x_2 \geq -3,~ -7x_1+10x_2\geq -2,~ x_1-5x_2 \geq 0 \}
\end{align*}
\end{example}

We recall that for integers $n > m$ and a set $I \subset [m]$ we write $I_n = I \setminus\{m\} \uplus \{n\}$.
\begin{lemma}\label{lem:limit inequalities}
Let $m \geq d \geq 3$ and $I \subset [m]$ be a facet defining set of indices of $\mathcal{E}_{m,d}$ containing $m$ as an end point. Then, for all $n \geq m$ the inequalities $\ell_{I_n} \circ \Phi_{d} \leq r_{I_n}$ corresponding to a facet of $\mathcal{E}_{n,d}$ converge to an inequality $\ell_{I_\infty} \circ \Phi_{d} \leq r_{I_\infty}$ defining a facet of $\mathcal{E}_{d}$.
\end{lemma}
This is to be understood in the sense that the sequence $(a_{n,1},\ldots,a_{n,d-1},r_{I_n})$ containing the coefficients of $\ell_{I_n}$ and $r_{I_n}$ converges to a limit inequality $(a_1,\ldots,a_{d-1},r_{I_\infty})$. 
\begin{proof}
 Since all but one of the vertices of the facets corresponding to $I_n$ are equal, the remaining sequence of changing vertices converges to the limit vertex $$\left( {n \choose 2} \frac{1}{n^2},\ldots,{n \choose d} \frac{1}{n^d} \right) \to \left( \frac{1}{2!},\frac{1}{3!},\ldots,\frac{1}{d!} \right) \in \mathcal{E}_{d},~n \to \infty.$$ Thus, the facets corresponding to $I_n$ in $\mathcal{E}_{n,d}$ converge to the facet indexed by $I_\infty$ in $\mathcal{E}_d$. By continuity the defining linear inequalities must also converge which was to show.
\end{proof}
It follows that $\mathcal{E}_d$ can be defined as an intersection of countably infinitely many halfspaces. 

\begin{proposition} \label{prop:half of conjecture}
Let $d \geq 3$, then $\mathcal{E}_{d} = \left\{ x \in \R^{d-1} \,:\, \ell_I \circ \Phi_d (x) \leq r_I : I \in \mathcal{I}_d \right\}.$
\end{proposition}




\section{Undecidability of nonnegativity of trace polynomials} \label{sec:undecidability}

In this Section we show that the problem of deciding nonnegativity of trace polynomials in symmetric matrices of all sizes is undecidable (see Theorem \ref{thm:undecidable traces}). This result stays in sharp contrast to the case of finitely many variables. Surprisingly, we then prove that the analogous problem defined with normalized traces is decidable (see Theorem \ref{thm:normalized decidable}). The key for the undecidability lies in the geometry of $\Pi_3$. To prove Theorem \ref{thm:undecidable traces} we show that deciding copositivity of homogeneous product symmetric polynomials in any number of variables is an undecidable problem (see Theorem \ref{thm:undecidable}) which proof follows from \cite{hatami2011undecidability,blekherman2022undecidability} on undecidability in graph homomorphism densities.


\begin{definition}
For a variable $X$ we denote by $\Tr (X)$ the \emph{formal trace symbol} on $X$. A \emph{trace polynomial} in the variables $X_1,\ldots,X_k$ is a polynomial expression in formal trace symbols of powers of the variables $X_1,\ldots,X_k$. A trace polynomial is \emph{univariate} if $k=1$.
\end{definition}

For instance, $2\Tr (X_1^4)\Tr (X_2)-\Tr(X_2^5)^3$ is a trace polynomial in the variables $X_1,X_2$, while $4\Tr(X_1X_2^2)-4\Tr(X_1^3)$ % or $\Tr(X_1)^4+X_1$ 
is not a trace polynomial. A trace polynomial can naturally be evaluated on square matrices of all sizes. 
We call a trace polynomial $f(X_1,\ldots,X_k)$ \textit{nonnegative} if $f(A_1,\ldots,A_k) \geq 0$ for all symmetric matrices $A_1,\ldots,A_k$ of all sizes. We show that establishing nonnegativity of a trace polynomial is an undecidable problem. 
\begin{theorem} \label{thm:undecidable traces}
The following decision problem is undecidable.
\begin{itemize}
    \item[{\footnotesize Instance:}] A positive integer $k$ and a trace polynomial $f(X_1,\ldots,X_k)$.
    \item[{\footnotesize Question:}] Is $f(M_1,\ldots,M_k)$ nonnegative for all real symmetric matrices $M_1,\ldots,M_k$  of all sizes with $\Tr (M_i^2) = 1$ for all $1 \leq i \leq k$?
\end{itemize}
\end{theorem}

We now give an intuitive explanation of undecidability, and relate trace nonnegativity to the geometry of the limit Vandermonde cell. Recall that for any matrix $A \in \R^{n \times n}$ with eigenvalues $\lambda_1,\ldots,\lambda_n$ we have $\Tr (A^m) = \sum_{i=1}^n \lambda_i^m$. The problem of establishing trace nonnegativity is already undecidable when we consider only trace polynomials with traces in the first three even powers of symmetric matrices, i.e. we only consider even power sums $\sum \lambda_i^2, \sum \lambda_i^4$ and $\sum \lambda_i^6$ of eigenvalues. If we restirct to matrices $A$ such that $\Tr A^2=1$, then the image of all symmetric matrices of all sizes is simply the limit Vandermonde cell $\Pi_3$. The key to the hardness of the problem is the geometry of $\Pi_3$ which we investigated in Section \ref{sec:Vandermonde map}. Recall that the set $\bd \Pi_3$ contains countably infinitely many isolated singularities on the rational moment curve $(t,t^2)$ (see Corollary \ref{cor:N3 infinite many }). A $k$-variate trace polynomial can be viewed as a polynomial on $k$-fold direct product $(\Pi_3)^k$.
We can reduce testing nonnegativity of certain integer polynomials on $(\Pi_3)^k$ to just testing nonnegativity on products of the isolated points on the moment curve. 
Then deciding nonnegativity of such trace polynomials reduces to deciding nonnegativity of $k$-variate polynomials on $\N^k$ which is known to be undecidable \cite{hatami2011undecidability}.



\begin{remark}
We deduce from Theorem \ref{thm:undecidable traces} that there cannot exist a unified algorithm or an effective certificate to determine the validity of polynomial inequalities in traces of powers of symmetric matrices of all sizes. Note that for a finite number of variables it follows by Artin's solution to Hilbert's 17th problem \emph{\cite{artin1927zerlegung}} that validity of polynomial inequalities on semialgebraic sets is decidable. 
\end{remark}

Nonnegativity of trace polynomials is investigated in the context of non-commutative real algebraic geometry. There one usually considers normalized trace polynomials. In \cite{klep2021positive} the authors prove a Positivstellensatz for univariate normalized trace polynomials. 

\begin{definition}
For a variable $X$ we denote by $\widetilde{\operatorname{tr}}(X)$ the \emph{normalized formal trace symbol} on $X$. A \emph{normalized trace polynomial} in the variables $X_1,\ldots,X_k$ is a polynomial expression in normalized formal trace symbols of powers of the variables $X_1,\ldots,X_k$. A normalized trace polynomial is univariate if $k=1$.
\end{definition}
As the name normalized trace operator indicates, for a matrix $A \in \R^{n \times n}$ we define the evaluation $\widetilde{\operatorname{tr}}(A) := \frac{1}{n}\Tr (A)$. A normalized trace polynomial is \textit{nonnegative} if its evaluation on all symmetric matrices of all sizes is nonnegative. 

\begin{theorem}\label{thm:normalized decidable}
 The following decision problem is decidable.
\begin{itemize}
    \item[{\footnotesize Instance:}] A positive integer $k$ and a normalized trace polynomial $f(X_1,\ldots,X_k)$.
    \item[{\footnotesize Question:}] Is $f(M_1,\ldots,M_k)$ nonnegative for all symmetric matrices $M_1,\ldots,M_k $ of all sizes?
\end{itemize}   
\end{theorem}

For matrices of fixed size deciding nonnegativity of normalized trace polynomials and trace polynomials is equivalent. The sharp contrast appears when we ask about nonnegativity for matrices of all sizes. Geometrically, the limit of the normalized Vandermonde map of the unit simplex $\Delta_{n-1}$ corresponds to the set of the first $d$ moments of a probability measure supported on $\mathbb{R}_{\geq 0}$, and it is well-know that this set can be described by linear matrix inequalities \cite{MR3729411}. In particular, the limit is semilagebraic for all $d$. The phenomenon of decidability for normalized trace can also be explained with the half-degree principle (\cite{timofte2003positivity}, Corollary 2.1), and we follow this direction in our proof.


\subsection{Proof of Theorem \ref{thm:undecidable traces}} %\greg{Can we shorten this? It seems that we are repeting a few things...} \sebastian{I tried to shorten the subsection}

We show that the subproblem of deciding copositivity of polynomial expressions in $p_1(X_i)$, $p_2(X_i)$ and $p_3(X_i)$ for all $1 \leq i \leq k$ on $\Delta_{n-1}$ for all $n$ is undecidable. Recall, we can also work with the first $3$ elementary symmetric polynomials on the probability simplex. 
Nonnegativity of a symmetric polynomial in any number of variables can also be formulated as nonnegativity of an associated symmetric function. A \textit{symmetric function} $f$ is a formal power series in countably infinitely many variables which is invariant under the action of the group $S_\infty = \bigcup_{n \in \N} S_n$ and for which the set of degrees of the monomials in $f$ is finite (see e.g. \cite[§I.2]{macdonald1998symmetric} for details). The ring of symmetric functions $\R[x]^{\mathcal{S}_{\infty}} := \R[x_1,x_2,\ldots]^{\mathcal{S}_\infty}$ can be constructed as the inverse limit of the rings of symmetric polynomials with respect to the transition maps \begin{align}\label{intro:transisition maps}
    \R[x]^{\mathcal{S}_{n+1}} \to \R[x]^{\mathcal{S}_{n}}, ~f(x_1,\ldots,x_{n+1}) \mapsto f(x_1,\ldots,x_n,0).
\end{align} 
 For $n \geq d$ the transition map implies $$f(x_1,\ldots,x_{n+1})=g(p_1,\ldots,p_d) \mapsto   f(x_1,\ldots,x_n,0)=g(p_1,\ldots,p_d),$$ where the power sums are polynomials in a different number of variables. The analogous to elementary symmetric and power sum polynomials in $\R[x]^{\mathcal{S}_\infty}$ are the \textit{elementary symmetric function}
$ \mathfrak{e}_k := \sum_{I \subset \N, |I|=k} \prod_{i \in I}X_i$ and the \textit{power sum function} $ \mathfrak{p}_k := \sum_{i \in \N}X_i^k.$ 

A homogeneous symmetric polynomial $f = \sum_{\alpha} c_\alpha e_1^{\alpha_1}\cdots e_{d}^{\alpha_d}$ of degree $d$ is nonnegative in any number of variables $n \geq d$ if and only if the symmetric function $ \mathfrak{f} = \sum_{\alpha} c_\alpha \mathfrak{e}_1^{\alpha_1}\cdots \mathfrak{e}_{d}^{\alpha_d} $ is nonnegative on the infinite probability simplex $\Delta$. 

To prove Theorem \ref{thm:undecidable} which follows from \cite{hatami2011undecidability}, we require access to polynomials with domain $E_d^k$. Therefore, we need \textit{product symmetric functions}, i.e. symmetric functions in several groups of countably infinitely many variables which are invariant under the diagonal action of $\mathcal{S}_\infty^k$. We denote the by $\Delta^k$ the $k$-copies of the infinite probability simplex.
%i.e. $$\Delta^k = \left\{(x_{1,1},\ldots,x_{k,1},x_{1,2},\ldots,x_{k,2},\ldots) : \sum_{j \geq 1}x_{i,j} = 1, x_{i,j} \geq 0, \forall i,j\right\}$$ 
\begin{theorem}\label{thm:undecidable}
 The following problem is undecidable.
\begin{itemize}
    \item[{\footnotesize Instance:}] A positive integer $k$ and a product symmetric function $\mathfrak{f}$ in $k$ groups of variables.
    \item[{\footnotesize Question:}] Does the inequality $\mathfrak{f}(a) \geq 0$ hold for all $a \in \Delta^k$?
\end{itemize}
\end{theorem}
We follow (\cite[§ 5]{hatami2011undecidability}) and use their notation. Hatami and Norin's work concerns undecidability of determining the validity of linear inequalities in graph homomorphism densities for graphons and answers negative a question of Lovász (\cite[Problem 17]{lovasz2008graph}). By adapting only very few parts of Hatami and Norin's proof we show that an undecidable problem can be embedded into the problem of deciding copositivity of product symmetric homogeneous functions in $\mathfrak{e}_1,\mathfrak{e}_2,\mathfrak{e}_3$. We write $\mathfrak{e}_{j,(i)}$ for the $j$-th elementary symmetry functions in the $i$-th group of variables.


\begin{proof}[Proof of Theorem \ref{thm:undecidable}]
By (\cite[Lemma~5.1]{hatami2011undecidability}) it follows from Matiyasevich’s solution to Hilbert’s tenth problem that the following validity problem is undecidable:
\begin{itemize}
    \item[{\footnotesize Instance:}] A positive integer $k$ and a polynomial $p \in \Z[Y_1,\ldots,Y_k]$.
    \item[{\footnotesize Question:}] Do there exist $x_1,\ldots,x_k \in \left\{ 1- \frac{1}{n} : n \in \N\right\}$ with $p(x_1,\ldots,x_k) <0$?
\end{itemize}
We define $ C :=  \conv (2\mathfrak{e}_2,6\mathfrak{e}_3)(\Delta)$,  
 $g(x) := 2x^2-x$ and the piecewise linear function $$L(x):= \frac{3t^2-t-2}{t(t+1)}x-\frac{2(t-1)}{t+1}$$ on the interval $[0,1]$, where $t \in [0,1)$ is chosen such that $x \in \left[ 1 - \frac{1}{t}, 1- \frac{1}{t+1}\right]$ for some $t \in \left\{  1-\frac{1}{n} : n \in \N \right\}$, and $L(1) := 1$. By Corollary \ref{cor:image infinite probability simplex} we have $C = \conv \left\{(1,1),\left( 1-\frac{1}{n}, \frac{(n-1)(n-2)}{n^2}\right) : n \in \N\right\} $. The piecewise linear function $L$ takes the same value as $g$ on all the endpoints of the intervals $\left[ 1 - \frac{1}{t}, 1- \frac{1}{t+1}\right]$ and we have $L(x) \geq g(x)$ for all $x \in [0,1]$. Further, we define $R := \{ (x,y) \in [0,1]^2 : y \geq L(x)\}$. The images of each piecewise linear part of $L$ on $[0,1]$ are precisely the facets of the lower part of the boundary of $C$. \\ 
Let $p \in \R[Y_1,\ldots,Y_k]$ be a polynomial and let $M$ be the sum of the absolute values of its coefficients multiplied by $100 \deg (p)$. We consider the real auxiliary polynomial 
$$ q(Y_1,\ldots,Y_k,Z_1,\ldots,Z_k):= p \prod_{i=1}^k (1-Y_i)^6+M \left( \sum_{i=1}^kZ_i - g(Y_i) \right).$$ Then, by (\cite[Lemma 5.4]{hatami2011undecidability}) and the observation $(1,1) \in R$ the following are equivalent: 
\begin{enumerate}
    \item[{(i)}] $q(x_1,\ldots,x_k,y_1,\ldots,y_k) < 0$ for some $x_1,\ldots,x_k,y_1,\ldots,y_k$ with $(x_i,y_i) \in R$ for all $1 \leq i \leq k$;
    \item[{(ii)}] $p(x_1,\ldots,x_k) < 0$ for some $x_1,\ldots,x_k \in \{ 1,1- \frac{1}{n} : n \in \N\}$.
\end{enumerate}
Now, we consider the map 
$$ \abb{\tau}{\R[Y_1,\ldots,Y_k,Z_1,\ldots,Z_k]}{\R[X^k]^{S^k}}{f(Y_1,\ldots,Y_k,Z_1,\ldots,Z_k)}{\prod_{i=1}^k \mathfrak{e}_{1,(i)}^{3\deg f} \cdot f\left( \frac{\mathfrak{e}_{2,(1)}}{\mathfrak{e}_{1,(1)}^2},\ldots,\frac{\mathfrak{e}_{2,(k)}}{\mathfrak{e}_{1,(k)}^2},\frac{\mathfrak{e}_{3,(1)}}{\mathfrak{e}_{1,(1)}^3},\ldots,\frac{\mathfrak{e}_{3,(k)}}{\mathfrak{e}_{1,(k)}^3}\right)}.$$
For $f \in \R[Y_1,\ldots,Y_k,Z_1,\ldots,Z_k]$ the rational function $\tau (f)$ is actually a homogeneous product symmetric function. This is, since ${\mathfrak{e}_{2,(i)}}$ and ${\mathfrak{e}_{1,(i)}^2}$ (resp. $\mathfrak{e}_{3,(i)}$ and ${\mathfrak{e}_{1,(i)}^3}$) have degree $2$ (resp. $3$) and thus every monomial in the rational product symmetric function $$f\left( \frac{\mathfrak{e}_{2,(1)}}{\mathfrak{e}_{1,(1)}^2},\ldots,\frac{\mathfrak{e}_{2,(k)}}{\mathfrak{e}_{1,(k)}^2},\frac{\mathfrak{e}_{3,(1)}}{\mathfrak{e}_{1,(1)}^3},\ldots,\frac{\mathfrak{e}_{3,(k)}}{\mathfrak{e}_{1,(k)}^3}\right)$$ has degree $0$. Multiplying by $\mathfrak{e}_{1,(i)}^{3\deg f}$ ensures that $\tau (f)$ has always nonnegative exponent in  $\mathfrak{e}_{1,(i)}$ for all $1 \leq i \leq k$.\smallskip

As in (\cite{hatami2011undecidability}, Claims 5.7 \& 5.8) we claim that the following assertions are equivalent 
\begin{itemize}
    \item[(a)] $q(x_1,\ldots,x_k,y_1,\ldots,y_k) < 0$ for some $x_1,\ldots,x_k,y_1,\ldots,y_k$ with $(x_i,y_i) \in R$ for all $1 \leq i \leq k$;
    \item[(b)] $\tau (q)$ attains a negative value on $\Delta^k$.
\end{itemize}
First, we suppose (a). Hatami and Norine show in the proof of (\cite[Lemma~5.4]{hatami2011undecidability}) that if \\
$q(x_1,\ldots,x_k,y_1,\ldots,y_k) < 0$ for some $x_1,\ldots,x_k,y_1,\ldots,y_k$ with $(x_i,y_i) \in R$ for all $1 \leq i \leq k$ then the $x_i$'s can be chosen as $x_1,\ldots,x_k \in \{ 1, 1-\frac{1}{n} : n \in \N\}$, and $y_i = L(x_i)$. Thus, $\tau(q)$ is negative on $\Delta^k$ by Corollary \ref{cor:image infinite probability simplex}. More precisely, $\mathfrak{e}_{1,(i)}=1, 2\mathfrak{e}_{2,(i)}=x_i$ and $6\mathfrak{e}_{3,(i)}=y_i$ for all $1 \leq i \leq k$ is feasible and thus $\tau(q)$ is not nonnegative. \\
Second, we suppose $q(x_1,\ldots,x_k,y_1,\ldots,y_k) \geq 0$ for all $x_i,y_i$ with $(x_i,y_i) \in R$ for all $1 \leq i \leq k$, then $\tau(q)$ is nonnegative on $\Delta^k$, since $C^k \subset R^k$.  \smallskip

So the assertions (ii) and (b) are equivalent. In particular, the question to determine if a given Diophantine set is non-empty was reformulated as asking whether a product symmetric polynomial in $\mathfrak{e}_{1,(i)}$, $\mathfrak{e}_{2,(i)}$ and $\mathfrak{e}_{3,(i)}$ for $1 \leq i \leq k$ is nonnegative on $\Delta^k$. This proves the Theorem.
\end{proof}




We are ready to prove the main theorem on undecidability of nonnegativity of trace polynomials.

\begin{proof}[Proof of Theorem \ref{thm:undecidable traces}]
    For a symmetric matrix $A$ with trace $1$ of size $n \times n$ with eigenvalues $\lambda_1,\ldots,\lambda_n$ we have $\Tr (A^k) = \sum_{i=1}^n \lambda_i^k$. We identify a subproblem which is already known to be undecidable. Thus the general problem must also be undecidable.\\
    Consider the subproblem of determining validity of nonnegativity of homogeneous trace polynomials $f(X_1,\ldots,X_k)$ in which any formal trace symbol is in an even square of a variable up to degree $6$, i.e. $f$ is a polynomial expression in $\Tr (X_i^{2}),\Tr (X_i^4), \Tr (X_i^6)$ for $1 \leq i \leq k$. Then deciding nonnegativity of $f$ for all symmetric matrices $M_1,\ldots,M_k$ of all sizes is equivalent to deciding nonnegativity of the product symmetric function $f(\mathfrak{p}_{2,(1)},\mathfrak{p}_{4,(1)},\mathfrak{p}_{6,(1)},\ldots,\mathfrak{p}_{2,(k)},\mathfrak{p}_{4,(k)},\mathfrak{p}_{6,(k)}).$ Its nonnegativity is equivalent to copositivity of $f (\mathfrak{p}_{1,(1)},\mathfrak{p}_{2,(1)},\mathfrak{p}_{3,(1)},\ldots,\mathfrak{p}_{1,(k)},\mathfrak{p}_{2,(k)},\mathfrak{p}_{3,(k)})$. However, this problem is undecidable by Theorem \ref{thm:undecidable} since Newton's identities provide a linear relation between the power sums and elementary symmetrics up to degree $3$ when $\mathfrak{p}_1 =1$. 
\end{proof}


\subsection{Proof of Theorem \ref{thm:normalized decidable}}
The small adjustment of using normalized traces makes the problem decidable. An important role is played by Timofte's half degree principle. The decidability was implicitly observed by Blekherman and Riener in \cite{blekherman2021symmetric}. 

\begin{theorem}[\cite{timofte2003positivity}] \label{thm:half degree princ}
A symmetric polynomial $f \in \R[x]^{S_n}$ is nonnegative if and only if $f(a) \geq 0$ for any $a \in \R^n$ with $\# \{a_1,\ldots,a_n\} \leq \max \{ \lfloor \frac{\deg f}{2} \rfloor ,2\}$.    
\end{theorem}

We briefly illustrate why the normalized problem is decidable. 
Suppose we are given a power sum $p_d=x_1^d+\ldots+x_n^d$ in $n$ variables of degree $d \geq 4$. To test nonnegativity we can equivalently test nonnegativity of the $\lfloor \frac{\deg f}{2} \rfloor$-variate polynomials $(p_d)_\alpha = \alpha_1x_1^d+\ldots+\alpha_{\lfloor \frac{\deg f}{2} \rfloor}x_{\lfloor \frac{\deg f}{2} \rfloor}^d$ for all sequences $\alpha \in \N^{\lfloor \frac{\deg f}{2} \rfloor}$ with $\sum_{i=1}^{\lfloor \frac{\deg f}{2} \rfloor}\alpha_i = n$ by Theorem \ref{thm:half degree princ}. Thus, testing nonnegativity of normalized $\frac{p_d}{n}$ power sums is equivalent to testing nonnegativity of $(\frac{p_d}{n})_\alpha = \frac{\alpha_1}{n}x_1^d+\ldots+\frac{\alpha_{\lfloor \frac{\deg f}{2} \rfloor}}{n}x_{\lfloor \frac{\deg f}{2} \rfloor}^d$ for all $\alpha$'s. We observe that nonnegativity of $\frac{p_d}{n}$ for all $n$ is equivalent to nonnegativity of $\beta_1x_1^d+\ldots+\beta_{\lfloor \frac{\deg f}{2} \rfloor}x_{\lfloor \frac{\deg f}{2} \rfloor}^d$ for all $(\beta_1,\ldots,\beta_\frac{d}{2}) \in \Delta_{\lfloor \frac{\deg f}{2} \rfloor-1} \times \R^d$ due to the density of $\mathbb{Q}$ in $\R$.




\begin{definition}
    Let $\mathfrak{f}=(\sum_\lambda c_\lambda \frac{p_{\lambda_1}\cdots p_{\lambda_l}}{n^{l}})_{n \in \N}$ be a sequence of symmetric polynomials of degree $2d$ where $\mathfrak{f}_n$, the $n$-th element in the sequence, is a polynomial in $n$ variables. We define the associated $2d$-variate function $\Phi_\mathfrak{f}$ as 
    $$ \Phi_\mathfrak{f}(s,t) =  \sum_\lambda c_\lambda \prod_{i=1}^{l} (s_1t_1^{\lambda_i}+\ldots + s_d t_d^{\lambda_i}) $$
\end{definition}


The following Lemma generalizes the application of Timofte's half degree principle from the discussion above to arbitrary normalized symmetric polynomials. 
\begin{lemma}[\cite{blekherman2021symmetric} Theorem~3.4] \label{lem:gregcordian}
 $\mathfrak{f}=(\sum_\lambda c_\lambda \frac{p_{\lambda_1}\cdots p_{\lambda_l}}{n^{l}})_{n \in \N}$ be a sequence of symmetric polynomials of degree $2d$ where $\mathfrak{f}_n$ is a polynomial in $n$ variables. Then $\mathfrak{f}_n$ is nonnegative for all $n \in \N$ if and only if $\Phi_{\mathfrak{f}}$ is nonnegative on $\Delta_{d-1} \times \R^d$.
\end{lemma}

We are ready to prove Theorem \ref{thm:normalized decidable}.
\begin{proof}[Proof of Theorem \ref{thm:normalized decidable}]
We note, for a symmetric matrix $M \in \R^{n \times n}$ and eigenvalues $\lambda_1,\ldots,\lambda_n$ we have $$\widetilde{\operatorname{tr} }\left(M^k\right) = \frac{1}{n} \Tr(M^k) = \frac{1}{n} \sum_{i=1}^n \lambda_i^k = \frac{1}{n}p_k(\lambda). $$ 
Thus verifying nonnegativity of a univariate normalized trace polynomial is equivalent to verifying nonnegativity of the associated sequence of normalized symmetric polynomials in any number of variables. By Lemma \ref{lem:gregcordian} this is equivalent to nonnegativity of a polynomial on the semialgebraic set $\Delta_{d-1}\times \R^d$ and thus decidable. \\
For a multivariate normalized trace polynomial we proceed analogously and have that nonnegativity of a normalized trace polynomial in $k$ variables is equivalent to nonnegativity of an associated polynomial on the semialgebraic set $(\Delta_{d-1} \times \R^d)^k$.
\end{proof}




\section{Conclusion and open questions}
In this article we have studied  the wonderful geometry of the Vandermonde map in the finite and infinite setup. In particular, we have shown how a connection to trace polynomials allows to show that the problem to determine if a given multivariate trace polynomial is  nonnegative is undecidable. Our proof inspired by Hatami-Norine's proof \cite{hatami2011undecidability} relied on Matiyasevich work on  Hilbert's tenth problem \cite{matiyasevich1970diophantineness} which showed that it is not possible to computationally decide if a Diophantine equation in several variables has an integer solution. In this context it is worth noticing that asserting that a given univariate polynomial has a root in the integers is a decidable task. Our construction used to prove Theorem \ref{thm:undecidable traces} does not apply if we restrict to univariate trace polynomials and therefor it remains a natural question whether verification of nonnegativity of univariate trace polynomials is decidable. 

\printbibliography
\end{document}