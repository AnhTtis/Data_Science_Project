\section{Applications}
\label{sec:applications}
To demonstrate how to leverage XAIR for XAI design, we present two examples that showcase potential workflows that use XAIR for everyday AR applications (Fig.~\ref{fig:applications}). More details can be found in Appendix~\ref{sub:appendix:details_scenarios:more_details_of_application}.
After determining the key factors for a given scenario, we used the framework (Fig.~\ref{fig:overview_when}-Fig.~\ref{fig:overview_how}) to make design choices based on the factors.

\begin{figure*}[!t]
    \centering
    % \vspace{-0.3cm}
    % \includegraphics[width=1\columnwidth]{figures/application_scenarios.png}
    % \includegraphics[width=0.8\columnwidth]{figures/application_scenarios_2.png}
    \begin{subfigure}[b]{0.49\textwidth}
    \includegraphics[width=1\columnwidth]{figures/application_scenario1.png}
    \caption{Scenario 1: Route Suggestion when Jogging}
    \label{subfig:applications:jogging}
    \end{subfigure}
    \hfill
    \begin{subfigure}[b]{0.49\textwidth}
    \centering
    \includegraphics[width=1\columnwidth]{figures/application_scenario2.png}
    \caption{Scenario 2: Plant Fertilization Reminder.}
    \label{subfig:applications:plant}
    \end{subfigure}
    % \vspace{-0.3cm}
    \caption{Application of XAIR on Two Everyday AR Scenarios. In the second scenario, the hand icon indicates that explanations are manually triggered (the same below). Figures only present the default, concise explanations. Detailed explanations are described in the main text of Sec.~\ref{sec:applications}.}
    \label{fig:applications}
    \Description[Application of XAIR on Two Everyday AR Scenarios.]{The figure illustrates two everyday AR scenarios of how XAIR could be applied. The left subfigure shows a scenario described in section 6.1 in the paper. In the scenario, Nanci is jogging on a quiet trial. Her AR glasses display a map beside her view and recommend a detour. Nancy is curious to know the reason why the new route is recommended. The AR glasses prompt explanations with the user goal of resolving surprise, and the system goal of user intent discovery. The right subfigure shows a scenario described in section 6.2 in the paper. In the scenario, Sarah is recommended by her AR glasses a plant fertilization reminder after chatting with her neighbor on relevant topics. She is concerned about her privacy being invaded. In this case, the system displays an explanation with the user goal of privacy awareness and the system goal of trust building. Both scenarios demonstrate how XAIR could be used to support the design of explainable interfaces in AR glasses.}
\end{figure*}

\subsection{Scenario 1: Route Suggestion while Jogging}
\label{sub:applications:scenario_jogging}
\textit{Scene}. Nancy (AI expert, high AI literacy) is jogging in the morning on a quiet trail. Since it is the cherry-blossom season and Nancy loves cherries, her AR glasses display a map beside her and recommend a detour. Nancy is surprised since this route is different from her regular one, but she is happy to explore it. She is also curious to know the reason this new route was recommended.

\colorwhen{\textit{\textbf{When}}}.
\colorwhen{Delivery}. Nancy has enough cognitive capacity in this scenario. Her \textit{User Goal} is Resolving Surprise. Therefore, an explanation is automatically triggered because the two conditions are met (\gthree).

\colorwhat{\textit{\textbf{What}}}.
\colorwhat{Content}. Other than the \textit{User Goal}, the \textit{System Goal} is User Intent Discovery (exploring a new route to see cherry blossom). Considering Nancy's \textit{User Profile}, she is an expert in AI, so the appropriate explanation content types (\gfour) are Input/Output (\eg ``This route is recommended based on seasons, your routine, and preferences.'') and Why/Why-Not (\eg ``The route has cherry blossom trees that you can enjoy. The length of the route is appropriate and fits your morning schedule.'').
Examples for all seven explanation content types can be found in Appendix~\ref{sub:appendix:details_scenarios:more_details_of_application}.
\\
\colorwhat{Detail}. The AR interface shows the Why as default (\gfive), and can be expanded to show both types in detail (\gsix). Nancy can slow down and click the ``More'' button to see more detailed explanations while standing or walking.

\colorhow{\textit{\textbf{How}}}.
\colorhow{Modality}. The explanation is presented visually, the same as the recommendation (\gseven).
\\
\colorhow{Format}. The default explanation uses text, while the detailed explanation contains cherry-blossom pictures of the new route to help explain the Why (\geight).\\
\colorhow{Pattern}. The explanation is shown explicitly within the route recommendation window (\gnine).

\subsection{Scenario 2: Plant Fertilization Reminder}
\label{sub:applications:scenario_reminder}
\textit{Scene}. Sarah (general end-user, low AI literacy) was chatting with her neighbor about gardening. After she returned home and sat on the sofa, her AR glasses recommended instructions about plant fertilization by showing a care icon on the plant. Sarah is concerned about technology invading her privacy, and wants to know the reason behind the recommendation.

\colorwhen{\textit{\textbf{When}}}.
\colorwhen{\textit{Delivery}}. Although Sarah has enough cognitive capacity, none of the three cases in the second condition of \gthree\ are met (\ie she was familiar with the recommendation and not confused, and the model didn't make a mistake). Therefore, the explanation needs to be manually triggered (\gtwo).

\colorwhat{\textit{\textbf{What}}}.
\colorwhat{Content}. In this case, the \textit{System Goal} is Trust Building (clarifying the usage of data), and the \textit{User Goal} is Privacy Awareness. Sarah's \textit{User Profile} indicates that she is not an expert in AI. According to \gfour, the explanation content type list contains Input/Output, Why/Why-Not, and How.\\
\colorwhat{Detail}. Considering Sarah's concern, the default explanation merges Why and How: ``The system scans the plant's visual appearance. It has abnormal spots on the leaves, which indicate fungi or bacteria infection.'' (\gfive).
For the detailed explanation, the full content of the three types is presented in a drop-down list upon her request (\gsix).

\colorhow{\textit{\textbf{How}}}.
\colorhow{Modality}. Following \gseven, the visual modality is used for both the explanation and the manual trigger (a button beside the plant care icon).\\
\colorhow{Format}. Other than using text as the primary format, the abnormal spots on the leaves are also highlighted via circles to provide an in-situ explanation (\geight).\\
\colorhow{Pattern}. Since the highlighting of spots is compatible with the environment (shown on leaves), it adopts the implicit pattern (\gnine). The rest of the texts of the explanation uses the explicit pattern.

Our two examples demonstrate XAIR's ability to guide XAI design in AR in various scenarios. In Appendix~\ref{sub:appendix:details_scenarios:more_scenarios}, we provide additional everyday AR scenarios to further illustrate its practicality.