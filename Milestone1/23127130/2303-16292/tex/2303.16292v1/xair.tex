%%
%% This is file `sample-manuscript.tex',
%% generated with the docstrip utility.
%%
%% The original source files were:
%%
%% samples.dtx  (with options: `manuscript')
%% 
%% IMPORTANT NOTICE:
%% 
%% For the copyright see the source file.
%% 
%% Any modified versions of this file must be renamed
%% with new filenames distinct from sample-manuscript.tex.
%% 
%% For distribution of the original source see the terms
%% for copying and modification in the file samples.dtx.
%% 
%% This generated file may be distributed as long as the
%% original source files, as listed above, are part of the
%% same distribution. (The sources need not necessarily be
%% in the same archive or directory.)
%%
%% The first command in your LaTeX source must be the \documentclass command.
%%%% Small single column format, used for CIE, CSUR, DTRAP, JACM, JDIQ, JEA, JERIC, JETC, PACMCGIT, TAAS, TACCESS, TACO, TALG, TALLIP (formerly TALIP), TCPS, TDSCI, TEAC, TECS, TELO, THRI, TIIS, TIOT, TISSEC, TIST, TKDD, TMIS, TOCE, TOCHI, TOCL, TOCS, TOCT, TODAES, TODS, TOIS, TOIT, TOMACS, TOMM (formerly TOMCCAP), TOMPECS, TOMS, TOPC, TOPLAS, TOPS, TOS, TOSEM, TOSN, TQC, TRETS, TSAS, TSC, TSLP, TWEB.
% \documentclass[acmsmall]{acmart}

%%%% Large single column format, used for IMWUT, JOCCH, PACMPL, POMACS, TAP, PACMHCI
% \documentclass[acmlarge,screen]{acmart}

%%%% Large double column format, used for TOG
% \documentclass[acmtog, authorversion]{acmart}

%%%% Generic manuscript mode, required for submission
%%%% and peer review
% \documentclass[acmsmall,review, anonymous]{acmart}
\documentclass[sigconf]{acmart}
% \documentclass[manuscript,review,anonymous]{acmart}

\begin{table}[t]
    \footnotesize
    \NineColors{saturation=high}
    \caption[Variables list]{Selection of variables used in this paper, grouped in their respective categories.}
    \begin{adjustbox}{width=\linewidth}
    \begin{tblr}
    	{
    		%caption = {Selection of variables used in this paper, grouped in categories.},
    		colspec = {cll|ll},
    		vlines = {white},
    		hlines = {white},
    		vline{4} = {2}{-}{white},
    		cell{1}{2-5} = {red3, fg=yellow9, font=\bfseries},
    		cell{2-14}{1} = {red3, fg=yellow9, font=\bfseries},
    		%cell{1-10}{2-5} = {red3, fg=yellow9, font=\bfseries},
    		cell{3,5,7,9,11,13,15}{2-5} = {red9},
    		%cell{3-6}{1}={red3, fg=yellow9},
    		%cell{3-6}{1} = {font=\bfseries},
    	}
    	&Symbol&Description&Symbol&Description\\
    	\multirow{4}{*}{\rotatebox{90}{Acq. model}}&
    	$\sigma$ & Wavenumbers &$\mathbf{s}$&Focal plane coordinates\\
    	&$\bm{\omega}=(\theta^{[i]},\phi^{[i]})$& Incident angle&$\{\Omega_j\}_{\range{j}{1}{N_p}}$& Solid angle of incidence\\
   		&$\mathcal{L}(\sigma,\bm{\omega})$ & Input spectral radiance&$\{\Phi_{jk}\}_{\range{j}{1}{N_p},\range{k}{1}{N_i}}$&Received flux\\
   		&$\{S_k\}_{\range{k}{1}{N_i}}$&Entrance pupil surface&$\{d_k\}_{\range{k}{1}{N_i}}$&Interferometer thickness\\
    	\multirow{4}{*}{\rotatebox{90}{Parameters}}&
    	$\bm{\delta}=\{\delta_i\}_{\range{i}{1}{N_i}}$& \glsentryshortpl{opd} & $\varphi^{ }_0$&Phase shift\\
    	&$\mathcal{A}(\sigma)$&Gain &$\mathbf{a}=\{a_m\}_{\range{m}{0}{N_d}}$& Gain coefficients\\
    	&$\mathcal{R}(\sigma)$&Surface reflectivity &$\mathbf{r}=\{r_m\}_{\range{m}{0}{N_d}}$& Reflectivity coefficients\\
    	&$\bm{\beta}=\{\beta_m\}_{\range{m}{1}{N_m}}$& Vector of parameters &$\hat{\bm{\beta}}=\{\hat{\beta}_m\}_{\range{m}{1}{N_m}}$& Estimated parameters \\
    	\multirow{3}{*}{\rotatebox{90}{Acq. vectors}} &$\bm{\sigma}=\{\sigma_i\}_{\range{i}{1}{N_a}}$& Central wavenumbers& $T_{\bm{\beta}}(\sigma_i)=\{t_i\}_{\range{i}{1}{N_a}}$ & Transfer function\\
    	&$\mathbf{y}=\{y_i\}_{\range{i}{1}{N_a}}$ & Single pixel acquisition& $\mathbf{w}=\{w_i\}_{\range{i}{1}{N_a}}$ & Flat field pixel statistic \\
    	& $\mathbf{u}=\{u_i\}_{\range{i}{1}{N_a}}$ & Neighborhood mean&$\mathbf{v}=\{v_i\}_{\range{i}{1}{N_a}}$ & Scaled neighborhood mean\\
    	\multirow{3}{*}{\rotatebox{90}{Amount}} &$N_a$ & Acquisitions & $N_i$ & Interferometers\\
    	& $N_p$ & Pixels per interferometer & $W$ & Waves\\
    	& $N_d$ & Degree & $N_m$ & Parameters\\
%    	
 	\end{tblr}
 	\end{adjustbox}
    \label{tab:variables}
\end{table}


%%
%% \BibTeX command to typeset BibTeX logo in the docs
\AtBeginDocument{%
  \providecommand\BibTeX{{%
    \normalfont B\kern-0.5em{\scshape i\kern-0.25em b}\kern-0.8em\TeX}}}

%% Rights management information.  This information is sent to you
%% when you complete the rights form.  These commands have SAMPLE
%% values in them; it is your responsibility as an author to replace
%% the commands and values with those provided to you when you
%% complete the rights form.

\copyrightyear{2023}
\acmYear{2023}
\setcopyright{acmlicensed}
\acmConference[CHI '23]{Proceedings of the 2023
CHI Conference on Human Factors in Computing Systems}{April 23--28,
2023}{Hamburg, Germany}
\acmBooktitle{Proceedings of the 2023 CHI Conference on Human Factors in
Computing Systems (CHI '23), April 23--28, 2023, Hamburg, Germany}
\acmPrice{15.00}
\acmDOI{10.1145/3544548.3581500}
\acmISBN{978-1-4503-9421-5/23/04}

%%
%% Submission ID.
%% Use this when submitting an article to a sponsored event. You'll
%% receive a unique submission ID from the organizers
%% of the event, and this ID should be used as the parameter to this command.
%%\acmSubmissionID{123-A56-BU3}

%%
%% The majority of ACM publications use numbered citations and
%% references.  The command \citestyle{authoryear} switches to the
%% "author year" style.
%%
%% If you are preparing content for an event
%% sponsored by ACM SIGGRAPH, you must use the "author year" style of
%% citations and references.
%% Uncommenting
%% the next command will enable that style.
%%\citestyle{acmauthoryear}

%%
%% end of the preamble, start of the body of the document source.
\begin{document}

%%
%% The "title" command has an optional parameter,
%% allowing the author to define a "short title" to be used in page headers.
\title{XAIR: A Framework of Explainable AI in Augmented Reality}

%%
%% The "author" command and its associated commands are used to define
%% the authors and their affiliations.
%% Of note is the shared affiliation of the first two authors, and the
%% "authornote" and "authornotemark" commands
%% used to denote shared contribution to the research.

\author{Xuhai Xu}
\affiliation{%
  \institution{Meta Reality Labs \& UW}
  \city{} \state{} \country{}
}
\email{xuhaixu@uw.edu}

\author{Mengjie Yu}
\affiliation{%
  \institution{Meta Reality Labs}
  \city{} \state{} \country{}
}
\email{annaymj@meta.com}

\author{Tanya Jonker}
\affiliation{%
  \institution{Meta Reality Labs}
  \city{} \state{} \country{}
}
\email{tanya.jonker@meta.com}

\author{Kashyap Todi}
\affiliation{%
  \institution{Meta Reality Labs}
  \city{} \state{} \country{}
}
\email{kashyap.todi@gmail.com}

\author{Feiyu Lu}
\affiliation{%
  \institution{Meta Reality Labs \& VT}
  \city{} \state{} \country{}
}
\email{feiyulu@vt.edu}

\author{Xun Qian}
\affiliation{%
  \institution{Meta Reality Labs \& Purdue}
  \city{} \state{} \country{}
}
\email{qian85@purdue.edu}

\author{João Belo}
\affiliation{%
  \institution{Meta Reality Labs \& Aarhus Univ}
  \city{} \state{} \country{}
}
\email{joaobelo@cs.au.dk}

\author{Tianyi Wang}
\affiliation{%
  \institution{Meta Reality Labs}
  \city{} \state{} \country{}
}
\email{tianyiwang@meta.com}

\author{Michelle Li}
\affiliation{%
  \institution{Meta Reality Labs}
  \city{} \state{} \country{}
}
\email{michelleli@meta.com}

\author{Aran Mun}
\affiliation{%
  \institution{Meta Reality Labs}
  \city{} \state{} \country{}
}
\email{aranmun@meta.com}

\author{Te-Yen Wu}
\affiliation{%
  \institution{Meta Reality Labs \& Dartmouth}
  \city{} \state{} \country{}
}
\email{Te-yen.Wu.GR@dartmouth.edu}

\author{Junxiao Shen}
\affiliation{%
  \institution{Meta Reality Labs \& Cambridge}
  \city{} \state{} \country{}
}
\email{js2283@cam.ac.uk}

\author{Ting Zhang}
\affiliation{%
  \institution{Meta Reality Labs}
  \city{} \state{} \country{}
}
\email{tingzhang@meta.com}

\author{Narine Kokhlikyan}
\affiliation{%
  \institution{Meta Reality Labs}
  \city{} \state{} \country{}
}
\email{narine@meta.com}

\author{Fulton Wang}
\affiliation{%
  \institution{Meta Reality Labs}
  \city{} \state{} \country{}
}
\email{fultonwang@meta.com}

\author{Paul Sorenson}
\affiliation{%
  \institution{Meta Reality Labs}
  \city{} \state{} \country{}
}
\email{pfsorenson52@meta.com}

\author{Sophie Kahyun Kim}
\affiliation{%
  \institution{}
  \city{} \state{} \country{Meta Reality Labs}
}
\email{sophiekkim@meta.com}

\author{Hrvoje Benko}
\affiliation{%
  \institution{Meta Reality Labs}
  \city{} \state{} \country{}
}
\email{benko@meta.com}

% Mengjie Yu
% Tanya R. Jonker
% Kashyap Todi
% Feiyu Lu
% Xun Qian
% João Marcelo Evangelista Belo
% Tianyi Wang
% Michelle Li
% Aran Mun
% Te-Yen Wu
% Junxiao Shen
% Ting Zhang
% Narine Kokhlikyan
% Fulton Wang
% Paul Sorenson
% Sophie Kahyun Kim
% Hrvoje Benko


%%
%% By default, the full list of authors will be used in the page
%% headers. Often, this list is too long, and will overlap
%% other information printed in the page headers. This command allows
%% the author to define a more concise list
%% of authors' names for this purpose.
\renewcommand{\shortauthors}{Xu et al.}
% \renewcommand{\shorttitle}{XAIR}

%%
%% The abstract is a short summary of the work to be presented in the
%% article.
\begin{abstract}
\begin{abstract}
As models continue to grow in size, the development of memory optimization methods (MOMs) has emerged as a solution to address the memory bottleneck encountered when training large models. To comprehensively examine the practical value of various MOMs, we have conducted a thorough analysis of existing literature from a systems perspective. 
% Furthermore, we have evaluated the most widely adopted MOMs employed in mainstream frameworks for both vision and language models.
Our analysis has revealed a notable challenge within the research community: the absence of standardized metrics for effectively evaluating the efficacy of MOMs. The scarcity of informative evaluation metrics hinders the ability of researchers and practitioners to compare and benchmark different approaches reliably. Consequently, drawing definitive conclusions and making informed decisions regarding the selection and application of MOMs becomes a challenging endeavor.
To address the challenge, this paper summarizes the scenarios in which MOMs prove advantageous for model training. We propose the use of distinct evaluation metrics under different scenarios. By employing these metrics, we evaluate the prevailing MOMs and find that their benefits are not universal. We present insights derived from experiments and discuss the circumstances in which they can be advantageous.

\end{abstract}
\end{abstract}

%%
%% The code below is generated by the tool at http://dl.acm.org/ccs.cfm.
%% Please copy and paste the code instead of the example below.
%%
\begin{CCSXML}
<ccs2012>
<concept>
<concept_id>10003120.10003121.10011748</concept_id>
<concept_desc>Human-centered computing~Empirical studies in HCI</concept_desc>
<concept_significance>500</concept_significance>
</concept>
<concept>
<concept_id>10003120.10003121.10003128</concept_id>
<concept_desc>Human-centered computing~Interaction techniques</concept_desc>
<concept_significance>500</concept_significance>
</concept>
</ccs2012>
\end{CCSXML}

% \ccsdesc[500]{Human-centered computing~Empirical studies in HCI}
% \ccsdesc[300]{Human-centered computing~Interaction techniques}

%%
%% Keywords. The author(s) should pick words that accurately describe
%% the work being presented. Separate the keywords with commas.
\keywords{Explainable AI, Augmented Reality, Design Framework}

% \begin{figure*}[t]
% 
\renewcommand{\arraystretch}{-0.5}
\begin{tabular}{l}
\begin{overpic}[width=1\linewidth]{fig/Overview_v4.pdf}
\put (1, 1)  {\footnotesize (a) Append frame}
\put (16, 0.8) {\footnotesize \centeredtab{(b) Optimize RF \\ and poses}}
\put (32.3, 0.8) {\footnotesize \centeredtab{(c) Refine RF \\ and poses}}
\put (46.5, 0.3) {\footnotesize (d) Allocate another radiance field}
\put (75, -1.5) {\footnotesize (e) Complete reconstruction}
\end{overpic} \\
$\underbrace{\hspace{0.29\linewidth}}_{\text{\footnotesize Repeat until out of uncontracted bound}}$  \\
$\underbrace{\hspace{0.70\linewidth}}_{\text{\footnotesize Repeat until all frames are registered}}$
\end{tabular}
\vspace{\figcapmargin}
\vspace{-1mm}
\caption{\textbf{Method overview.} 
The squares represent the uncontracted space of each local radiance field and the triangles are camera poses.
The color of each camera pose indicates to which radiance fields it is linked and serves as supervision.
We show as insert the renders intermediate at the intermediate optimization step for each local radiance field. (a) We add a frame at the end of the trajectory before (b) jointly estimating poses and the corresponding local radiance field. After the pose reaches the boundary of the high-resolution uncontracted space, (c) we run the optimization without adding frames to refine the poses and the radiance field. Then, to dynamically extend the representation, (d) we allocate a new radiance field. We repeat this process until we cover the full trajectory to produce (e) a complete reconstruction.  
}
\label{fig:overview}
\vspace{-3mm}
\end{figure*}

%%
%% This command processes the author and affiliation and title
%% information and builds the first part of the formatted document.
\maketitle

\section{Introduction}


Recent years have witnessed the rise of human digitization~\cite{habermannDeepCapMonocularHuman2020,alexanderCREATINGPHOTOREALDIGITAL,pengNeuralBodyImplicit2021,alldieckDetailedHumanAvatars2018, rajANRArticulatedNeural2020}. This technology greatly impacts the entertainment, education, design, and engineering industry.
There is a well-developed industry solution for this task.
High-fidelity reconstruction of humans can be achieved either with full-body laser scans~\cite{saitoSCANimateWeaklySupervised2021}, dense synchronized multi-view cameras~\cite{xiangModelingClothingSeparate2021a,xiangDressingAvatarsDeep2022a}, or light stages~\cite{alexanderCREATINGPHOTOREALDIGITAL}.
However, these settings are expensive and tedious to deploy and consist of a complex processing pipeline, preventing the technology's democratization.

Another solution is to view the problem as inverse rendering and learn digital humans directly from custom-collected data.
Traditional approaches directly optimize explicit mesh representation~\cite{loperSMPLSkinnedMultiperson2015, fangRMPERegionalMultiperson2018, pavlakosExpressiveBodyCapture2019} which suffers from the problems of smooth geometry and coarse textures~\cite{prokudinSMPLpixNeuralAvatars2020,alldieckVideoBasedReconstruction2018}. Besides, they require professional artists to design human templates, rigging, and unwrapped UV coordinates.
Recently, with the help of volumetric-based implicit representations~\cite{mildenhallNeRFRepresentingScenes2020, parkDeepSDFLearningContinuous2019, meschederOccupancyNetworksLearning2019} and neural rendering~\cite{laineModularPrimitivesHighPerformance2020, liuSoftRasterizerDifferentiable2019, thiesDeferredNeuralRendering2019}, 
one can easily digitize a quality-plausible human avatar from video footage~\cite{jiangNeuManNeuralHuman2022,wengHumanNeRFFreeviewpointRendering}.
Particularly, volumetric-based implicit representations~\cite{mildenhallNeRFRepresentingScenes2020, pengNeuralBodyImplicit2021} can reconstruct scenes or objects with much higher fidelity against previous neural renderer~\cite{thiesDeferredNeuralRendering2019,prokudinSMPLpixNeuralAvatars2020}, and is more user-friendly as it does not need any human templates, pre-set rigging, or UV coordinates.
Captured visual footage and corresponding skeleton tracking are enough for training.
However, better reconstructions and more friendly usability are at the expense of the following factors.
1) \textbf{Inefficiency:}
They require longer optimization times (typically tens of hours or days) and inference slowly.
Volume rendering~\cite{mildenhallNeRFRepresentingScenes2020,lombardiNeuralVolumesLearning2019} formulates images by querying the densities and colors of millions of spatial coordinates. 
In the training stage, due to memory constraints, only a small fraction of points are sampled which leads to slow convergence speed.
2) \textbf{Entangled representations}:
The geometry, materials, and motion dynamics are entangled in the neural networks. 
Due to the implicit nature of neural nets, one can hardly edit one property without touching the others~\cite{yuanNeRFEditingGeometryEditing2022a,liuEditingConditionalRadiance2021}.
3) \textbf{Graphics incompatibility}:
Volume rendering is incompatible with the current popular graphic pipeline,
which renders triangular/quadrilateral meshes efficiently with the rasterization technique.
Many downstream applications require mesh rasterization in their workflow (\eg, editing~\cite{foundationBlenderOrgHome}, simulation~\cite{benderPositionBasedSimulationMethods2015}, real-time rendering~\cite{akenine2019real}, ray-tracing~\cite{waldRTXRayTracing}).
Although there are approaches~\cite{lorensenMarchingCubesHigh,labelleIsosurfaceStuffingFast2007} can convert volumetric fields into meshes, the gaps from discrete sampling degrade the output quality in terms of both meshes and textures.


To address these issues, we present \textbf{EMA}, a method based on \textbf{E}fficient \textbf{M}eshy neural fields to reconstruct animatable human \textbf{A}vatars.
Our method enjoys flexibility from implicit representations and efficiency from explicit meshes, yet still maintains high-fidelity reconstruction quality.
Given video sequences and the corresponding pose tracking, our method digitizes humans in terms of canonical triangular meshes, physically-based rendering (PBR) materials, and skinning weights \textit{w.r.t.} skeletons.
We jointly learn the above components via inverse rendering~\cite{laineModularPrimitivesHighPerformance2020,chenDIBRLearningPredict2021,chenLearningPredict3D2019} in an end-to-end manner.
Each of them is derived from a separate neural field, which relaxes the requirements of a preset human template, rigging, or UV coordinates.
Specifically, we predict a canonical mesh out of a signed distance field (SDF) by differentiable marching tetrahedra~\cite{shenDeepMarchingTetrahedra2021,gaoGET3DGenerativeModel,gaoLearningDeformableTetrahedral2020,munkbergExtractingTriangular3D2022}, then we extend the marching tetrahedra~\cite{shenDeepMarchingTetrahedra2021} for spatial-varying materials by utilizing a neural field to predict PBR materials \textit{on the mesh surfaces} after rasterization~\cite{munkbergExtractingTriangular3D2022,hasselgrenShapeLightMaterial2022,laineModularPrimitivesHighPerformance2020}.
To make the canonical mesh animatable, we take another neural field to model the forward linear blend skinning for the meshes. 
Given a posed skeleton, the canonical mesh is then transformed into the corresponding poses.
Finally, we shade the mesh with a rasterization-based differentiable renderer~\cite{laineModularPrimitivesHighPerformance2020} and train our models with a photo-metric loss.
After training, we export the mesh with materials and discard the neural fields.

\looseness=-1
There are several merits of our method design.
1) \textbf{Efficiency}:
Powered by efficient mesh rendering, our method can render in real-time.
Besides, the training speed is boosted as well, 
since we compute loss holistically on the whole image and the gradients only flow on the mesh surface. In contrast, volume rendering takes limited pixels for loss computation and back-propagates the gradients in the whole space.
Our method only needs about an hour of training and minutes of optimization are enough for plausible avatar reconstruction.
2) \textbf{Disentangled representations}:
Our shape, materials, and motion modules are disentangled naturally by design, which facilitates editing. 
Besides, Canonical meshes with forward skinning modeling handle the out-of-distribution poses better.
3) \textbf{Graphics compatibility}:
Our derived mesh representation is compatible with 
the prominent graphic pipeline, which leads to instant downstream applications (\eg, the shape and materials can be edited directly in design software~\cite{foundationBlenderOrgHome}).
To further improve reconstruction quality, we additionally optimize image-based environment lights and non-rigid motions.


We conduct extensive experiments on standards benchmarks H36M~\cite{ionescuHuman36MLarge2014b} and ZJU-MoCap~\cite{pengNeuralBodyImplicit2021}.
Our method achieves very competitive performance for novel view synthesis, generalizes better for novel poses, 
and significantly improves both training time and inference speed against previous arts.
Our research-oriented code reaches real-time inference speed (100+ FPS for rendering $512\times512$ images).
We in addition showcase applications including novel pose synthesis, material editing, and relighting.

\section{Background}

\subsection{Graph Neural Network}

 GNNs \cite{kipf2016semi, hamilton2017inductive} are proposed for representation learning on graphs $  \mathcal{G}(\mathcal{V},\mathcal{E})$, and 
 follow the  message-passing paradigm (Algorithm \ref{alg:GNN-computation-abstraction}) in which the vertices recursively aggregate information from the neighbors. $\bm{h}^{L}_{v}$ denotes the last-layer embedding of the target vertex $v$. The Update() is usually a Multi-Layer Perceptron that transforms the vertex features.  
%  The outputs of GNN are the embeddings of target vertices that 
An element-wise activation function is applied to the feature vectors after the Aggregate() and Update() in each layer. 
% In some GNN models (e.g., GCN \cite{kipf2016semi}),  Update() can be performed before Aggregate().
 The output embedding $\bm{h}^{L}_{v}$
 can be used for many downstream tasks, such as node classification (\cite{hamilton2017inductive,kipf2016semi}), link prediction, etc. GCN \cite{kipf2016semi}, GraphSAGE \cite{hamilton2017inductive}, GIN \cite{xu2018powerful}, and SGC \cite{wu2019simplifying} are some representative GNN models. Table \ref{tab:notations} summarizes the notations used in this paper.
 
% \noindent \textbf{GCN}: GCN \cite{kipf2016semi} has the layer definition: 
% \begin{equation}
%     \begin{split}
%         \bm{a}_{i}^{l} & = \text{Sum}\left( \left\{ \alpha_{ji} \cdot \bm{h}_{j}^{l-1}:j\in \mathcal{N}(i)\cup  \{i\}\right\}\right)\\
%         \bm{z}_{i}^{l} & = \bm{a}_{i}^{l}\bm{W}^{l}, \text{ } \bm{h}_{i}^{l}  = \text{ReLU}(\bm{z}_{i}^{l}) 
%     \end{split}
%     \label{label:gcn}
% \end{equation}
% where $l$ denotes $l^{\text{th}}$ layer, $\alpha_{ji}=\frac{1}{ \sqrt{D(j)\cdot D(i)}}$ ($D(j)$ is the degree of $v_{j}$). Using matrix representation, a GCN layer can be expressed as: 
% $\bm{H}^{l} = \text{ReLU}(\bm{A}\bm{H}^{l-1}\bm{W}^{l})$, where $\bm{A}$ denotes graph adjacency matrix, $\bm{H}$ denotes vertex feature matrix, and $\bm{W}$ denotes weight matrix. 



% \noindent \textbf{GraphSAGE}: GraphSAGE \cite{hamilton2017inductive} is proposed for inductive representation learning on graphs, where each layer is:
% \begin{equation}
%     \begin{split}
%         \bm{a}_{i}^{l} & = \text{Mean} \left( \left\{ \bm{h}_{j}^{l-1}:j\in\mathcal{N}(i) \cup \{i\} \right\} \right) \\
%         \bm{z}_{i}^{l} & =  \bm{a}_{i}^{l}\bm{W}_{\text{neighbor}}^{l} || \bm{h}_{i}^{l-1}\bm{W}_{\text{self}}^{l}, \text{ } \bm{h}_{i}^{l} = \text{ReLU}(\bm{z}_{i}^{l})
%     \end{split}
%      \label{label:graphsage}
% \end{equation}
% Using matrix representation, the GraphSAGE layer can be expressed as: $
%     \bm{H}^{l} = \text{ReLU}(\bm{A}\bm{H}^{l-1}\bm{W}^{l}_{\text{neighbor}}||\bm{H}^{l-1}\bm{W}^{l}_{\text{self}})
% $


\begin{table}[h]
\centering
\caption{Notations}
\begin{adjustbox}{max width=0.48\textwidth}
\begin{tabular}{cc|cc}
\toprule
 \textbf{{Notation}} & \textbf{{Description}}  & \textbf{{Notation}}  & \textbf{{Description}} \\
 \midrule
\midrule
{$  \mathcal{G}(\mathcal{V},\mathcal{E})$ }& {input graph}  &  $ v_{i}$ & {$i^{\text{th}}$ vertex} \\ \midrule
$ \mathcal{V}$ &  {set of vertices} &  $ e_{ij}$ & {edge from $ v_{i}$ to $  v_{j}$} \\ \midrule
$ \mathcal{E}$& {set of edges} &  $ L$&{number of GNN layers} \\ \midrule
$\bm{A}$& graph adjacency matrix &  $ \mathcal{N}(i)$& the set of neighbors of $ v_{i}$ \\ \midrule
$ \bm{h}_{i}^{l-1}$& input feature vector of $ v_{i}$
at layer $l$    & $\bm{W}^{l}$  &  weight matrix of layer $l$  \\  \midrule
$\bm{H}^{l-1}$ & input feature matrix to layer $l$ & $\sigma()$ & activation function \\
 \bottomrule
\end{tabular}
\end{adjustbox}
\label{tab:notations}
\end{table}


\begin{algorithm}
\caption{GNN Computation Abstraction}
\label{alg:GNN-computation-abstraction}
\begin{small}
\begin{algorithmic}[1]
 \renewcommand{\algorithmicrequire}{\textbf{Input:}}
\renewcommand{\algorithmicensure}{\textbf{Output:}}
 \Require Input graph: $\mathcal{G}(\mathcal{V},\mathcal{E})$; vertex features: $\left\{\bm{h}^{0}_{1}, \bm{h}^{0}_{2}, \bm{h}^{0}_{3}, ..., \bm{h}^{0}_{|\mathcal{V}|}\right\}$;
 \Ensure Output vertex features $\left\{\bm{h}^{L}_{1}, \bm{h}^{L}_{2}, \bm{h}^{L}_{3}, ..., \bm{h}^{L}_{|\mathcal{V}|}\right\}$;
\For{$l=1...L$}
\For{each vertex $v \in \mathcal{V}$}
\State{$\bm{a}^l_{v} = {\text{Aggregate}(}\bm{h}_{u}^{l-1}: u\in \mathcal{N}(v))$}
\State{$\bm{z}_{v}^l = {\text{Update}(}\bm{a}_{v}^{l}, \bm{W}^{l} \textbf{)}$, $ \bm{h}_{v}^l = \sigma(\bm{z}_{v}^l )$}
\EndFor
\EndFor
\end{algorithmic}
\end{small}
\end{algorithm}

% \subsection{GNN Accelerator}
% Introduce the GNN accelerator

\subsection{Data Sparsity in GNN inference}
\label{subsec:GNN-sparsity}






The \emph{density} of a matrix is defined as the total number of non-zero elements divided by the total number of elements. Note that, the \emph{sparsity} is given by $(1 - \text{\emph{density}})$. The computation kernels in GNNs involve three types of matrices: graph adjacency matrix $\bm{A}$, vertex feature matrix $\bm{H}$, and weight matrix $\bm{W}$. 
% For example, the feature aggregation involves the multiplication of $\bm{A}$ and $\bm{H}$. The feature update involves the multiplication of $\bm{H}$ and $\bm{W}$. 
% As shown in Figure \ref{fig:density-of-adjacency-matrix},
The adjacency matrix $\bm{A}$ of different graph datasets \cite{pyg-dataset} can have different densities. For a given adjacency matrix, different parts of the matrix have different densities. Figure \ref{fig:density-of-feature-matrix}
shows the densities of feature matrices in GCN \cite{kipf2016semi}. For different graphs, the input feature matrices have different densities. The feature matrices of different layers also have different densities.  For the weight matrices, prior works (\cite{rahman2022triple, chen2021unified}) have  proposed various pruning techniques to reduce the density of the weight matrices. 
% As shown in  \cite{rahman2022triple}, 75-98\% of weight entries can be pruned without affecting the accuracy (See Figure 2 of \cite{rahman2022triple}). 

\begin{figure}[h]
     \centering
     \includegraphics[width=4cm]{pic/density-of-adjacency-matrix-a.pdf}
     \includegraphics[width=4cm]{pic/density-of-adjacency-matrix.pdf}
     \caption{The density and the visualization of graph adjacency matrix $\bm{A}$ of various graphs \cite{pyg-dataset}}
     \label{fig:density-of-adjacency-matrix}
\end{figure}

\begin{figure}[ht]
     \centering
     \includegraphics[width=8.5cm]{pic/density-of-feature-matrix.pdf}
     \caption{Density of the feature matrices in the GCN model \cite{kipf2016semi}}
     \label{fig:density-of-feature-matrix}
     \vspace{-0.2cm}
\end{figure}

\begin{figure*}[h]
     \centering
     \includegraphics[width=18cm]{pic/workflow.pdf}
     \caption{Proposed workflow}
     \label{fig:workflow}
\end{figure*}

\subsection{GNN Acceleration based on Data Sparsity}

Although there are various data sparsities in GNNs, no prior work has systematically studied exploiting the data sparsity for GNN inference acceleration.
HyGCN \cite{yan2020hygcn} and BoostGCN \cite{zhang2021boostgcn}  map Aggregate() to SpDMM and map update() to GEMM, ignoring the data sparsity in feature matrices and weight matrices.  AWB-GCN \cite{geng2020awb} maps both Aggregate() and update() to SpDMM. Then, they propose an accelerator to efficiently execute SpDMM. However, they do not exploit the data sparsity in weight matrices.  DeepBurning-GL \cite{liang2020deepburning} is a design automation framework that generates the optimized hardware accelerator given the information of the input graph and the GNN model. However, their framework needs to regenerate the optimized accelerator if the sparsity of the data is changed.  To summarize, prior GNN accelerators do not fully exploit the data sparsity in GNNs, or are not flexible to exploit data sparsity in GNN inference. 



% views both feature aggregation  and feature transformation as SpDMM based on the profiling of the data sparsity using a specific GCN model on five datasets. Then, they propose an accelerator that can efficiently execute SpDMM. However, they does not exploit the data sparsity in weight matrix, and the accelerator is not efficient for GEMM when there is no data sparsity in feature update. Moreover, the accelerator of AWB-GCN does not support SPMM which is important when involved two matrices have high sparsity. 
\section{XAIR Problem Space and Key Factors}
\label{sec:problem_space_factors}

Determining the way to create effective XAI experiences in AI is a complex challenge. Thus, it is important to first identify the problem space to bound the scope of our investigation.
We first summarize over 100 papers from the ML and HCI literature to identify the problem space and the main dimensions within each problem (Sec.~\ref{sub:problem_space_factors:problem_space}).
Then, we outline the key factors that determine the answers to the problems (Sec.~\ref{sub:problem_space_factors:factors}).

The problem space and key factors define the structure of XAIR (Fig.~\ref{fig:xair} middle).
In Sec.~\ref{sec:methodology}, we present two studies conducted to obtain insights from end-users and expert stakeholders about how to design XAI in AR.
Then, combining the structure and insights, we show how these factors are connected with the problem space, and provide design guidelines in Sec.~\ref{sec:framework}.

% We first identify the problem space of our research question about creating effective XAI experience in AR (Sec.~\ref{sub:problem_space_factors:problem_space}).
% We then spotlight the key factors that determine the answer to the research question (Sec.~\ref{sub:problem_space_factors:factors}).
% Fig.~\ref{fig:overview} presents a detailed overview of the problem space and the factors.
% After introducing the overall picture, the rest of the paper follows Fig.~\ref{fig:pipeline} to develop and evaluate XAIR.

% % \begin{teaserfigure}
\begin{figure}
    \centering
    \includegraphics[width=1\columnwidth]{figures/overview-details.png}
    \caption{The Detailed Overview of XAIR framework. (Left) The main problem space and important dimensions within each space.
    (Middle) Key factors that determine the design choice of each dimension.
    (Right) AR-specific sensors that have the potential to be employed to detect key factors.}
    \label{fig:overview}
\end{figure}
% \end{teaserfigure}

\subsection{Problem Space}
\label{sub:problem_space_factors:problem_space}

Following the design space analysis method~\cite{qoc_1999}, the research question was divided into three sub-questions: when to explain, what to explain, and how to explain~\cite{elliott2017living,nahum-shani_just--time_2018}.

\subsubsection{\colorwhen{When to Explain?}}
\label{subsub:problem_space_factors:problem_space:when}
The literature review revealed two aspects of ``when'' that were important to consider: the \textit{availability} of explanations (\ie whether to prepare explanations?), and the timing of the explanation's \textit{delivery} (\ie when to show explanations?).

\colorwhen{\textbf{\textit{Availability}}}.
% It is important to determine whether AI models should generate any explanation available for users to access.
Previous research has found that to maintain a positive user experience, supporting user agency and control is important during human-AI interaction~\cite{cai_impacts_2022,lee1992trust}.
Having explanations that are available and accessible is in line with the goal of supporting user agency.

\colorwhen{\textbf{\textit{Delivery}}}.
With the ability to show information at any time, AR can employ various timing strategies to present explanations. Thus, it is important to find the appropriate method to deliver explanations to users. Generally, there are two approaches, manual-trigger (\ie initiated by users) and auto-trigger (\ie initiated by the system)~\cite{cimolino_two_2022,yeh2022guide}.
On the one hand, researchers have found that explanations should not always be presented to users, because they can introduce unnecessary cognitive load and become overwhelming for non-expert end-users~\cite{chazette2019end,wagner2020regulating,bunt2012explanations,robbins2019misdirected,stumpf2016explanations}. This is especially important in AR, as users' cognitive capacity tends to be limited~\cite{buchner2022impact}.
Moreover, adopting manual triggers would enable users to choose to see explanations as needed, thus enabling them to exercise agency over their experience~\cite{roy_automation_2019,lu_exploring_2022}.
On the other hand, existing findings on just-in-time intelligent systems (\eg just-in-time recommendations~\cite{kapoor2015just,ma2020temporal} and just-in-time interventions~\cite{nahum-shani_just--time_2018,Sarker2014, xu_typeout_2022}) have suggested that automatically delivering explanations at the right time based on user intent and need (as detected via AR sensing that identifies a user's state and context) can provide a better user experience~\cite{bhattacharya2017intent,mehrotra2019jointly}.

% In Sec.~\ref{sec:methodology}, we will draw insights obtained from both end-users and expert stakeholders and propose our framework and guidelines to answer the two parts in Sec.~\ref{sec:framework}.

\subsubsection{\colorwhat{What to Explain?}}
\label{subsub:problem_space_factors:problem_space:what}
The literature review also found two important aspects of ``what'' to consider: First, the \textit{content} of the explanations (\ie what type of content to include?). Second, the level of \textit{detail} of the explanations (\ie how much detail should be explained?).

\colorwhat{\textbf{\textit{Content}}}.
Previous literature in XAI has identified several explanation content types~\cite{barredo_arrieta_explainable_2020,mohseni_multidisciplinary_2021}. 
% On include input/output, conceptual model (why/why not, how, what else, what if) and non-functional types (certainty, how-to).
\review{
The seven types are:
\begin{s_enumerate}
\item Input/Output. This type explains the details of input (\eg data sources, coverage, capabilities) or output (\eg additional details, options that the system could produce) of a model~\cite{lim_assessing_2009,lim_toolkit_2010}.
\item Why/Why-Not. This type explains the features in the input data~\cite{Ribeiro2016} or the model logic~\cite{ribeiro_anchors_2018} that have led or not led to a given AI outcome~\cite{myers2006answering} (also known as contrastive explanations). Showing feature importance is another commonly used technique to generate these explanations~\cite{schlegel2019towards,casalicchio2018visualizing}.
\item How. This type provides a holistic view to explain the overall logic of an algorithm or a model and illustrate how the AI model works. Typical techniques include model graphs~\cite{lakkaraju2016interpretable}, decision boundaries~\cite{lombrozo2009explanation}, or natural language explanations~\cite{berkovsky2017recommend}. 
\item Certainty. This type describes the confidence level of the model with its input (\eg for models whose input is not deterministic, explain how accurate the input of the model is) or output (\eg explain how accurate, or reliable the AI outcomes are)~\cite{google_map_match_rate_2018,schoonderwoerd2021human}. Scores based on softmax~\cite{bridle1989training} or calibration~\cite{platt1998sequential} are commonly used as the confidence/certainty score for ML models.
\item Example. This type presents similar input-output pairs from a model, \eg similar input that lead to the same output or similar output examples given the same input~\cite{cai2019human,keane2019case}. This is also known as the What-Else explanation. Example methods include influence functions~\cite{koh_understanding_2017} and Bayesian case modelling~\cite{kim2014bayesian}.
\item What-If. This type demonstrates how changing input or applying new input can affect model output~\cite{cai2019effects,lim_why_2009}.
\item How-To. In contrast to What-If, this type explains how to change input to achieve a target output~\cite{wang_designing_2019,liao_questioning_2020}, \eg how to change the output from X to Y? Common methods for What-If/How-To content include rule generation~\cite{guidotti2018local}, feature description~\cite{wachter2017counterfactual}, and input perturbation~\cite{zhang2018interpreting}.
\end{s_enumerate}
}
% Previous studies by Lim \etal offered suggestions about subsets of content types for context-aware systems~\cite{lim_toolkit_2010,lim_assessing_2009,lim2019these}.
Moreover, another aspect that is independent of the explanation content type is global \vs local explanations (explaining the general decision-making process \vs a single instance)~\cite{molnar2020interpretable}. In general, non-expert end-users were found to prefer local explanations ~\cite{lakkaraju2019faithful,dhanorkar_who_2021}.


% \begin{figure}[htbp]
\centering
\includegraphics[width=\textwidth]{Figures/Figs/categories.png}        
\caption{Overview of this survey. This survey categorizes approaches based on four labeling scenarios: No label (Section \ref{sec:ssl}), insufficient label (Section \ref{sec:semi}), inexact label (Section \ref{sec:mil}), and label refinement (Section \ref{sec:al}). This figure illustrates the disparity between data growth and annotator scarcity, the core techniques employed in each scenario, and trends in label-efficient learning applications. Detailed survey scope can be referred to Appendix \ref{appendix1}.}
	\label{fig_class}
\end{figure}

\colorwhat{\textbf{\textit{Detail}}}.
Displaying every relevant explanation content type to an end-user can be overwhelming, especially with the limited cognitive capacity they have in AR~\cite{buchner2022impact,baumeister2017cognitive}. Explanations that extend a user's prior knowledge or fulfill their immediate needs should be prioritized~\cite{coppers2018intellingo}.
Moreover, previous research has suggested that presenting detailed and personalized explanations is useful for better understanding AI outcomes ~\cite{schneider2019personalized,kouki2019personalized,esteva2017dermatologist,jahanbakhsh_effects_2020,xu_understanding_2020}.

% We will verify whether these previous findings are transferable to AR scenarios in Sec.~\ref{sec:methodology}.
\review{
Our focus on \textit{content} and \textit{detail} is about choosing appropriate explanation content types and proper levels of detail, but not on picking which techniques to generate explanations.
From a technical perspective, there are interpretable models (\ie the model being transparent, such as linear regression or decision trees) and ad-hoc explainers (\ie generating explanations for complex, black-box models)~\cite{Lundberg2017, Ribeiro2016}. The latter can further be divided into model-specific and model-agnostic explanation methods~\cite{barredo_arrieta_explainable_2020}.
We refer readers to other surveys and toolkits for developing or selecting explanation generation algorithms \cite{arya2019one,liao2021human,dalex,h2oai,adadi_peeking_2018}.
}



\subsubsection{\colorhow{How to Explain?}}
\label{subsub:problem_space_factors:problem_space:how}
The last sub-question \colorhow{\textit{how}} focuses on the visual representation of the content in AR.
% Although Eiband \etal touched on the question of \textit{how to explain} in their participatory XAI design process outside of AR, they only involve a general ``iterative prototyping'' step~\cite{eiband_bringing_2018}.
% We summarize important dimensions that are needed to develop the framework and guidelines.
% The two important dimensions that are needed to develop the framework and guidelines are summarized below.
Two dimensions emerged from the literature review, \ie modality and paradigm.

\colorhow{\textbf{\textit{Modality}}}.
The multi-modal nature of AR enables it to support AI outcomes via various modalities (\eg visual, audio, or haptic)~\cite{chen2017multimodal,nizam2018review}.
Explanations are hard to convey using modalities with limited bandwidth (\eg haptic, olfactory, or even gustatory). Therefore, visual and audio are the two major modalities that should be employed for explanations.

\colorhow{\textbf{\textit{Paradigm}}}.
If explanations are presented using audio, the design space is relatively limited (\eg volume, speed). We refer readers to existing literature on audio design (\eg \cite{frauenberger2007survey, kern2009design}).
The design space of the visual paradigm for explanations, however, is much larger.
First, from a formatting perspective, explanation content can be presented in a textual format (\eg narrative, dialogue)~\cite{lakkaraju2016interpretable,myers2006answering}, graphical format (\eg icons, images, heatmaps)~\cite{zeiler2014visualizing,simonyan2013deep}, or a combination of both.
Second, from a pattern perspective, an explanation can be displayed either in an implicit way (\ie embedded in the environment, such as a boundary highlight of an object) or explicit way (\ie distinguished from the environment, such as a pop-up dialogue window)~\cite{lindlbauer_context-aware_2019,tatzgern_adaptive_2016,diverdi2004level}.
The pattern is closely related to the adaptiveness of the AR interface~\cite{dai2017scannet,wang_designing_2019}. With 3D sensing capabilities, the location of an explanation can be body-based (mostly explicit), object-based (implicit or explicit), or world-based (implicit or explicit) ~\cite{lu_exploring_2022,bonanni2005attention,laviola20173d,xu_hand_2018}. Prior AR research has explored adaptive interface locations~\cite{luo2022should,muller2016taxonomy}, \eg interfaces should be adaptive based on the semantic understanding of the ongoing interaction~\cite{cheng_semanticadapt_2021,qian_scalar_2022,rzayev2020effects} and ergonomic metrics~\cite{evangelista2021xrgonomics}.

% To answer \colorhow{\textit{how}} to explain, we will leverage end-users' preference and experts' knowledge collected from two user studies (see Sec.~\ref{sec:methodology}), and propose our framework to guide the choice of these options.


\subsection{Key Factors}
\label{sub:problem_space_factors:factors}
These three questions, and their dimensions, form the overall problem space of XAIR.
Another important aspect of XAIR is the factors that determine the answers to these questions.
We summarize these factors from two perspectives, one specific to AR platforms (Sec.~\ref{subsub:problem_space_factors:factors:ar_specific}), and the other agnostic to any platform (Sec.~\ref{subsub:problem_space_factors:factors:non_ar_specific}).
% The middle of Fig.~\ref{fig:overview} summarizes these factors.

\subsubsection{AR-Specific Factors}
\label{subsub:problem_space_factors:factors:ar_specific}
Fig.~\ref{fig:uniqueness_ar} summarizes the three main features that distinguish AR from other platforms: \textit{User State}, \textit{Contextual Information}, and \textit{Interface}.
As \textit{Interface} is an integral property of an AR platform, it remains invariant to external changes.
In contrast, the other two aspects are dynamic and would alter the design of XAI in AR.

\textbf{\textit{User State.}}
The sensors that could be integrated within future HMDs would empower an AR system to have a rich, instant understanding of user's state, such as activities (IMU~\cite{gjoreski2021head,windau2016walking}, camera~\cite{fathi2011understanding,singh2016first,schroder2017deep,liang_authtrack_2021}, microphone~\cite{xu_listen2cough_2021,xu_earbuddy_2020,wang_hearcough_2022,jin_earcommand_2022}), cognitive load (eye tracking~\cite{duchowski_index_2018,zagermann2018studying,joseph2020potential}, EEG~\cite{antonenko2010using,xu2018review}), attention (eye tracking~\cite{huang2018predicting,chong2018connecting,stappen2020x,xu_recognizing_2020}, IMU~\cite{leelasawassuk2015estimating}, EEG~\cite{vortmann2019eeg}), emotion (facial tracking~\cite{yong2019emotion, yan2022emoglass}, EEG~\cite{wan2021wearable,soleymani2015analysis}) and potential intent (the fusion of multiple sensors and low-level intelligence~\cite{tsai2018augmented,admoni2016predicting,kim2016understanding}).
Depending on a user's state, the design of explanations could be different. For example, as identified in previous research on ambient interfaces~\cite{pielot2017beyond,fogarty2005predicting}, when users engage in activities with a high cognitive load, explanations should not show up automatically to interrupt them
(related to \colorwhen{\textit{when}}).

\textbf{\textit{Contextual Information.}}
Compared to devices such as smartphones, AR HMDs have more context awareness. Other than having an awareness of location and time~\cite{dey_understanding_2001}, an egocentric camera and LiDAR, combined with other sensors (\eg Bluetooth, WiFi, RFID), can identify details about digital and non-digital objects in the environment~\cite{redmon2016you,liu2020deep,park2018high}, and have a better understanding of the semantics of a scene~\cite{grauman2022ego4d,pan2019content,bolanos2016toward,miech2020end}.
Such contextual information would also influence the design of XAI.
For instance, an explanation visualization about recipe recommendations that appears when users open the fridge may look differently from explanations about podcast suggestions that are shown while driving (related to \colorhow{\textit{how}}).

\subsubsection{Platform-Agnostic Factors}
\label{subsub:problem_space_factors:factors:non_ar_specific}
There are also other factors that are platform agnostic such as the motivation to present explanations (\ie \textit{why explain?}). We view this factor from two perspectives, one from the system side (\ie what are the \textit{system's goals} when presenting explanations?), and the other from the non-expert end-user side (\ie what are \textit{users' goals} when they want to see explanations?)~\cite{samek2019towards}.
The \textit{user profile} (\ie individual details) is another important factor related to personalized explanations~\cite{schneider2019personalized,kouki2019personalized}.

\textbf{\textit{System Goal.}}
Based on prior literature, we summarize four system goals that are desired when an AR system provides explanations for AI outcomes:
\begin{s_enumerate}
\item User Intent Discovery. When an AI model generates suggestions for a new topic, the system seeks to help users discover new intent~\cite{samek2019towards,gotz2009behavior,pu2006trust}.
For example, when a user is traveling in a city, the system recommends several attractions and local restaurants to visit.
Both the recommendation and explanations help the user explore new things that they were not aware of.
\item User Intent Assistance. When the target task has been already initiated by users, then the goal of generating AI outcomes and explanations assists users with existing intent~\cite{brown1998utility,bercher2014plan,das2021explainable}. For instance, when a user is making dinner, intelligent instructions and explanations would suggest alternative ingredients based on what a user has in their space.
\item Error Management. When a system has low confidence about input/output or makes a mistake, explanations can serve as error management and explain the process so that users can understand where an error comes from if it appears~\cite{xu2019explainable,adadi_peeking_2018}, how they might better collaborate with the system~\cite{das2021explainable}, or when to adjust their expectation of the system's intelligence~\cite{bertossi2020data, zhang2020effect}.
\item Trust Building. Various studies have found that explanations can help systems build user trust by offering transparency and increasing intelligibility~\cite{antifakos2005towards,mahbooba2021explainable,shin2021effects}. As a result, users’ trust in models leads them to rely on the system~\cite{berkovsky2017recommend,bussone2015role}.
\end{s_enumerate}
These four types of system goals are not exclusive. A system can seek to achieve multiple goals simultaneously.
Depending on the subset of system goals, the appropriate explanation timing and content types can differ~\cite{lim_assessing_2009,myers2006answering} (related to \colorwhen{\textit{when}} and \colorwhat{\textit{what}}).

\textbf{\textit{User Goal.}}
While a system has varying reasons to provide explanations, end-users also have varying reasons to have explanations. We summarize four types of user goals from literature.

\begin{s_enumerate}
\item Resolving Confusion/Surprise. Expectation mismatch is one of the main reasons to need explanations~\cite{dhanorkar_who_2021,ribera2019can,brennen2020people,langer_what_2021}. Users can become confused or surprised when AI outcomes are different from what users are expecting, and having explanations can help to resolve concerns~\cite{rai2020explainable,gervasio2018explanation}.
\item Privacy Awareness. As AI influences more aspects of daily living, concerns about invasion of one's privacy are also growing~\cite{manheim2019artificial}. Explanations could disclose which data is being used in a model's decision-making process to end-users ~\cite{eslami2015always,rader2018explanations,datta2014automated}. 
\review{
Researchers and designers are recommended to follow an existing privacy framework, such as contextual integrity~\cite{nissenbaum2009privacy}, to make privacy explanations more robust.}
\item Reliability. Ensuring the reliability of AI outcomes is essential for non-trivial decision-making processes so that users can rely on a trustworthy system~\cite{lepri2018fair,jiang_who_2022,Ribeiro2016}, \eg daily activity recommendations for personal health management or automatic emergency service contacting in safety-threatening incidents.
\item Informativeness. End-users can be curious about the reason or process behind an AI outcome~\cite{hoffman2018metrics,li2020survey}. Explanations can fulfill users' curiosity by providing more information~\cite{lage2019human,binns2018s,rader2018explanations}.
\end{s_enumerate}
Similar to the system goals, these user goals are not exclusive and users can have multiple goals at the same time. Different goals can require different explanation timings and content (\colorwhen{\textit{when}} and \colorwhat{\textit{what}}).
% However, it is worth noting that \textit{user goal} is a latent factor that can often be hard to detect by an AR system.

\textbf{\textit{User Profile.}}
This factor covers a range of individual details that influence the design of XAI.
For example, information such as demographics and user preferences is necessary to generate personalized explanations~\cite{schneider2019personalized,kouki2019personalized,gedikli2014should}.
End-users' familiarity with system outcomes is related to the need for explanations and \colorwhen{\textit{when}} to provide them~\cite{coppers2018intellingo}.
Users' digital literacy with AI also affects \colorwhat{\textit{what}} types of explanations are appropriate and would serve users' purposes~\cite{long_what_2020,ttc_labs,ehsan2021explainable}.
Moreover, users may have individual preferences about explanation visualizations, which may be closely related to \colorhow{\textit{how}}.
This factor takes these considerations into account.

It is worth noting that XAIR is proposed as a design framework.
In a context that AR can detect robustly, designers can use the framework to infer end-users' latent factors, such as \textit{User State} and \textit{User Goal}, based on their design expertise~\cite{eiband_bringing_2018}. 
For example, when users are driving (which can be easily detected by AR), designers can assess users' cognitive load to be high (\textit{User State}). For more complex factors such as \textit{User Goal}, designers can propose a set of potential goals in a given scenario and then refer to the framework to propose a set of designs.
% The automatic detection of these factors is not the focus of this paper.
As sensing and AI technology are maturing, the framework could be coupled with the automatic inference of these factors~\cite{tsai2018augmented,admoni2016predicting,kim2016understanding,gjoreski2021head,yan2022emoglass}.


% In the next section, we will leverage the insights acquired from end-users and expert stakeholders, and connect these key factors to answer the three XAI in AR design sub-questions.
\section{Methods}
\label{sec:methodology}
We conducted two studies after outlining the problem space, one from end-users' perspectives (Sec.~\ref{sub:methodology:end_user_surveys}), and the other from XAI/design/AR expert stakeholders' perspectives (Sec.~\ref{sub:methodology:expert_workshops}). The findings from the studies are complementary and provided insights that guided the development of the framework.

\subsection{Study 1: Large-Scale End-User Survey}
\label{sub:methodology:end_user_surveys}
In spite of the existing studies on XAI for end-users, it is unclear whether these findings hold for AR scenarios due to the unique features of AR systems.
Thus, we conducted a large-scale survey with end-users to collect their preferences on various aspects of XAI experiences for everyday AR.
% The survey results reveal the need of XAI in AR scenarios and are generally consistent with previous research outside XAI.

\subsubsection{Participants}
\label{subsub:methodology:end_user_surveys:participants}
\review{We recruited 506 participants from a third-party online user study platform (age 18 - 54, average 37$\pm$10), with a balanced gender distribution (Female 260, Male 241, Non-binary 5).}
Participants' digital literacy with AI varied, thus they were split into six groups: 1) unfamiliar with AI (12.2\%, 62), 2) heard of AI but never used AI-based products (23.5\%, 119), 3) used AI products occasionally a few times (23.1\%, 117), 4) used AI products on a regular basis (12.8\%, 65), 5) used AI products frequently (20.0\%, 101), and 6) worked on AI products (8.3\%, 42).
Participants were familiar with the concept of AR.
Among these groups, we further randomly sampled 20 participants (age 18 - 53, average 37$\pm$9, 11 Female, 9 Male) for a semi-structured interview to collect a more in-depth understanding about their preferences for XAI in AR.

\subsubsection{Design and Procedure}
\label{subsub:methodology:end_user_surveys:materials_design}
We prepared five sets of proof-of-concept descriptions and images with intelligent everyday AR services that represented five scenes in a typical weekday (\ie one set per scene). They included 1) music recommendations for the morning when users would be brushing their teeth, 2) podcast recommendations for when users would be driving to work, 3) music recommendations for when users would be working out, 4) recipe recommendations for when users would be making dinner, and 5) additional spice recommendations for when users would be making dinner.
In this study, we chose recommendations as the main AI service category, since it is arguably one of the most common AI applications in everyday AR~\cite{lam_a2w_2021,chatzopoulos2016readme} and users could easily contextualize these scenes in their mind.

% we pre-determined a set of explanations from different categories
For the AI outcome in each scene, participants were asked whether they wanted explanations (\ie yes, no, neutral). If their answer was yes, they would be directed to answer when they wanted it (\ie always/frequent, contextually dependent, rare/never), their preferred length of explanation (\ie concise \vs detailed) and the presenting modality (\eg visual, audio, neutral).
After viewing these scenes, they were asked to choose the explanation content types that they found useful. Participants were compensated \$5 USD for the task.
% In interviews, the experiment host would follow up with more questions based on participants' answers to collect detailed reasons behind their choices.

% including input/output, why/why-not, how, and certainty. The other three categories are not included as they may not compatible 

% \subsubsection{Procedure}
% \label{subsub:methodology:end_user_surveys:procedure}
% Participants were told to imagine themselves as end-users of AR devices. They then went through the five scenes and answered the questions.
% At the end of the survey, they were asked whether they would be interested in joining a follow-up interview study.
We randomly sampled 20 respondents who were willing to participate in a one-hour interview about the detailed reasons behind their survey responses.
These participants were compensated \$10 for the interview.
The interviews were video-recorded and manually transcribed.
\review{Two researchers collectively summarized and coded the data using a thematic analysis~\cite{braun2012thematic}. Specifically, they first met to establish an agreement on the themes and independently coded all the data. Then, they gathered to discuss and refine the coded data to resolve differences. Their inter-rater reliability ($\kappa$) was over 90\% after the refinement.
}

\subsubsection{Results}
\label{subsub:methodology:end_user_surveys:results}
The survey found that respondents had specific preferences for the timing, content, and modality of explanations.

\begin{figure*}[!b]
    \centering
    \vspace{-0.2cm}
    \begin{subfigure}[b]{.33\textwidth}
    \centering
    \includegraphics[width=1\columnwidth]{figures/percentage_whether.png}
    \caption{Need Explanations?}
    \label{subfig:survey_results:whether}
    \end{subfigure}
    \hfill
    \begin{subfigure}[b]{.34\textwidth}
    \centering
    \includegraphics[width=1\columnwidth]{figures/percentage_when.png}
    \caption{When to Have Explanations?}
    \label{subfig:survey_results:when}
    \end{subfigure}
    \hfill
    \begin{subfigure}[b]{.32\textwidth}
    \centering
    \includegraphics[width=1\columnwidth]{figures/percentage_what.png}
    \caption{What Explanations Are Preferred?}
    \label{subfig:survey_results:what}
    \end{subfigure}
    \vspace{-0.4cm}
    \caption{Highlight of Survey Results with 506 End-Users about Their Needs and Preferences of XAI in everyday AR scenarios.}
    \label{fig:survey_results}
    \Description[Survey Results.]{(a) A stacked percentage bar plot with six bars showing whether users need XAI with their AI experience. The six X-axis stick labels are: unfamiliar with AI, never heard of AI, use AI occasionally, use AI regularly, user AI frequently, and work in AI. The Y-axis is about the percentage. The bar chart shows that around 37\% of the "unfamilar with AI" replied "No". All other types of users have less than 10\% replied "No".
(b) A stacked percentage bar plot with six bars showing when users need XAI with their AI experience. The six X-axis stick labels are the same. And it shows that over 60\% of the users prefer the explanations to be "contextually dependent". As users get more experience with AI, the percentage of choosing "always or frequently" increases, and the percentage of choosing "rare or never" decreases.
(c) A group bar plot with six groups showing when users need XAI with their AI experience. The six X-axis stick labels are the same. Each group has four categories: Input/Output, Why/Why-Not, How, and Certainty. As users get more experience with AI, the percentage of choosing these four categories increases (from around 15\% on average to 45\% on average).}
    \vspace{-0.3cm}
\end{figure*}

\review{\textbf{Finding 1: Most users wanted explanations of AI outputs in AR.}} (related to \colorwhen{\textit{when - availability}}).
A large proportion of respondents wanted explanations (89.7\%), motivating the need for XAI in everyday AR scenarios (see Fig.~\ref{subfig:survey_results:whether}).
Our findings were consistent with previous work on end-users' needs for XAI outside AR~\cite{ehsan2021explainable,ttc_labs}.
The results indicated that if respondents had at least heard of AI, they were more likely to express a need for XAI in AR compared to those who were not familiar with AI.
\review{
% An ANOVA with digital literacy as the main factor indicates statistical significance ($F_{5,500} = 10.8, p < 0.001$).
% Generally, the more familiar respondents were with AI, the more they felt there to be a need for explanations.
Our interviews found that respondents with little knowledge of AI didn't realize what explanations could be used for. 
Interestingly, around 10\% of respondents who worked on AI indicated that they didn't want explanations. Our interviews revealed the main reason being that some users were \textit{``familiar enough... with the algorithm''} (P2).
% For example, \pquote{4}{I can understand why... it's not really needed}. These participants already had knowledge of AI and thus didn't need additional explanations.
% This finding is consistent with previous work outside AR~\cite{ehsan2021explainable,ttc_labs}.
}
% 
% Moreover, our interview results also suggest that when participants were unsatisfied 

\textbf{Finding 2: The majority of users wanted explanations to be occasional and contextual, especially when they saw anomalies} (related to \colorwhen{\textit{when - delivery}}).
Although most respondents wanted explanations, only 13.8\% indicated that they needed explanations all the time.
% , mostly from participants who work on AI (see Fig.~\ref{subfig:survey_results:when}) -- an interesting polarization of XAI needs of users with high digital literacy when comparing \textbf{Finding 1} \& \textit{2}.
The majority of respondents (63.4\%) preferred for explanations to be presented contextually only when they have the need.
% For example, P7 didn't think there was a need for explanations in the morning music recommendation scene, \pquote{7}{I wouldn't be surprised if it knew what I wanted to listen to in the morning. I'm a creature of habit.}
The results of the interviews indicated that the need for explanations was mainly in cases where AI outcomes were new or anomalous to respondents. This finding is also in line with previous studies' findings outside AR~\cite{dhanorkar_who_2021,jiang_who_2022}.

\textbf{Finding 3: Users generally preferred specific types of explanations} (related to \colorwhat{\textit{what - content}}).
Four explanation content types stood out as useful: Input/Output (41.5\%), How (37.1\%), Why/Why-Not (31.6\%), and Certainty (30.6\%).
The first three types were highlighted in previous findings about context-aware systems~\cite{lim_assessing_2009,lim_toolkit_2010}, while the last type has been adopted by industrial practitioners~\cite{google_map_match_rate_2018,spotify_blend_taste}.
As shown in Fig.~\ref{subfig:survey_results:what}, respondents with more knowledge of AI would prefer having these explanation types more than those with less AI knowledge.
% (ANOVA $F_{5,2018} = 15.4, p < 0.001$),
% which aligns with \textbf{Finding 1}.

\textbf{Finding 4: Users found detailed and personalized explanations useful} (related to \colorwhat{\textit{what - detail}}).
Although showing more explanation content can introduce additional cognitive costs, 48.3\% of respondents reported that they would find detailed explanations with multiple content types to be useful.
% \pquote{1}{It gives me more context, substance for why I need to take this suggestion.}
Moreover, respondents indicated that explanations that included personal preferences would be more convincing, \eg \pquote{13}{more personable, more upbeat}.
These results suggest that there is a need to provide options to modulate the level of explanation detail (see Sec.~\ref{subsub:problem_space_factors:problem_space:what}) and the \textit{User Profile} factor in the framework).

\textbf{Finding 5: Users' preferences for modalities depended on the cognitive load in an AR scenario} (related to \colorhow{\textit{how - modality}}).
The five scenes introduced different levels of cognitive load, which led respondents' preferences for XAI modality to vary. We found that for scenes with complex visual stimuli such as driving, respondents tended to prefer audio explanations over visual ones by 40\%, as they were \pquote{8}{more easy and convenient}.
This suggests that it is necessary to take modality bandwidths into account when choosing \colorhow{\textit{how}} to present XAI in different AR scenarios~\cite{buchner2022impact}.

Overall, these findings motivated the need for XAI in AR (\textbf{Finding 1}).
% , and were aligned with previous studies on XAI outside of AR (\textbf{Finding 1, 2} \& \textit{3}).
% This verifies that previous findings are transferable to AR scenarios.
Moreover, these results (\textbf{Finding 2-5}) also provided guidance for design XAI for end-users in AR.

\subsection{Study 2: Iteration with Expert Workshops}
\label{sub:methodology:expert_workshops}
Based on the existing literature and the end-user survey results, we created an early draft of the framework. Since XAIR aims to support designers and researchers during their design process, we utilized our draft within three workshops with expert stakeholders to collect their insights and finalize the framework.

% \footnotetext{By expert stakeholders, we refer to experts related to our research problem (XAI, design, UX, HCI, AR), rather than domain experts as XAI users. We will use the word ``experts'' as this meaning for the rest of the paper.}

\subsubsection{Participants}
\label{subsub:methodology:expert_workshops:participants}
Twelve participants (7 Female, 5 Male, Age 35 $\pm$ 6) from a technology company volunteered to participate in the study. They came from four backgrounds, \ie 3 XAI algorithm developers, 3 designers, 3 UX professionals, and 3 HCI/AR researchers. Participants worked in their domains for at least five years. All participants were familiar with the concept of AI and AR. Participants were randomly assigned into three groups, with each group containing one expert from each domain.

\subsubsection{Design and Procedure}
\label{subsub:methodology:expert_workshops:materials_design}
We proposed a draft of the framework combining the summary of literature and the results of end-user study. It was an early version of XAIR that is introduced in Sec.~\ref{sec:framework} and can be found in Appendix~\ref{sec:appendix:earlier_frameworks}.
We also prepared a set of everyday AR scenarios similar to the ones used in the end-user survey (Sec~\ref{sub:methodology:end_user_surveys}) to provide more context and stimulate more insights from experts.
We utilized a Figma board to show images of the framework and experts could add in-place feedback to different areas of the framework.

We adopted an iterative process using three sequential workshops. \review{All workshops lasted about 90 minutes and were video-recorded. After each workshop, two researchers went through a similar coding and refining process as Sec.~\ref{subsub:methodology:end_user_surveys:materials_design}, to make sure the result achieved a inter-rater reliability ($\kappa$) over 90\%. We summarized experts' feedback, iterated on the framework, and presented the new version in the next workshop.}
% \subsubsection{Procedure}
% \label{subsub:methodology:expert_workshops:procedure}
% After participants signed the consent form, we conducted the workshop with a group of participants. 
% In each workshop, we introduced our framework and walked through an example intelligence everyday AR scenario to show the use case of our framework. We then asked participants to apply the framework on another scene to get a deeper understanding of the framework.
% Participants were encouraged to raise questions or give feedback at any time during the whole process.
% We repeated the process with the other two workshops, each time with an iterated version of the framework.
% met to establish an agreement on the themes. They independently coded all the data, and gathered together to refine the coded data to make sure the result achieved a inter-rater reliability $\kappa$ over 90\%.

% We then update the framework based on the final results.

\subsubsection{Results}
\label{subsub:methodology:expert_workshops:results}
Overall, experts found the framework to be \pquote{2, P6, P7}{useful} and that it would \pquote{11}{serve as a very good reference for design}.
Our framework converged as the workshops proceeded, with us receiving rich feedback during the first workshop, and participants in the last workshop only offering small suggestions. We briefly highlight the major comments that were made.

\textbf{Suggestion 1: Add Missing Pieces.}
Participants found a few factors missing in the early version of the framework.
% For example, they suggested that the \textit{System Goal} error management should also be considered for the automatic delivery (\colorwhen{\textit{when}}) of explanations if the model was uncertain.
% and that the 
\review{
For example, they pointed out that \textit{User Goal} and \textit{User Profile} needed to be considered for the \colorwhat{\textit{what}} part, and that the modality of AI output in AR needed to be taken into account for the \colorhow{\textit{how}} part.
}
% as it would indicate the need for different explanations.
They also provided suggestions on appropriate explanation content types with different system/user goals (\colorwhat{\textit{what - content}}).

\textbf{Suggestion 2: Remove Redundancy.}
Participants also found some parts unnecessarily complex. For example, four experts suggested removing the interface location from \colorhow{\textit{how}} part (\ie where to explain, mentioned in Sec.~\ref{subsub:problem_space_factors:problem_space:how}), because the location needed to be optimized with the whole interface including AI outcomes.

\textbf{Suggestion 3: Add Default Options.}
Participants provided advice for default options of different dimensions.
For instance, they recommended using the manual-trigger as the default delivery method (\colorwhen{\textit{when}}) due to users' limited cognitive capacity in AR.
% We worked with participants to summarize the advice into a set of guidelines.

\textbf{Suggestion 4: Connect across Sub-questions.}
Participants came to the consensus that the three sub-questions were interwoven.
For example, the choice of \colorwhat{\textit{what}} to explain would influence the design of \colorhow{\textit{how}} to explain, and the framework should capture and emphasize such connection.
% The interface design of the \colorhow{\textit{how}} part would also need to consider the manual-/auto-trigger mechanism for the \colorwhen{\textit{when}} part.

\textbf{Suggestion 5: Improve Visual Structure.}
Finally, participants also offered several suggestions about the visual simplification, clarification, and color choices. The figures in Appendix~\ref{sec:appendix:earlier_frameworks} show the evolution of the visual structure.

The results of the end-user study and expert workshops are complementary and guided the final version of the framework.
\section{XAIR Framework}
\label{sec:framework}
We introduced the structure of XAIR framework in Sec.~\ref{sec:problem_space_factors} (\ie problem space and key factors), and summarized insights from end-users and experts in Sec.~\ref{sec:methodology}.
\review{
Connecting the literature survey and studies' results (\textbf{Findings 1-5} in Sec.~\ref{subsub:methodology:end_user_surveys:results}, and \textbf{Suggestions 1-5} in Sec.~\ref{subsub:methodology:expert_workshops:results}), we introduce the details of XAIR, identify how the key factors determine the design choices for each dimension in the when/what/how questions, and present a set of guidelines.}

% Connecting between Sec.~\ref{sec:problem_space_factors} and Sec.~\ref{sec:methodology}, we develop XAIR, a framework for the design of XAI in AR. We further propose a series of guidelines in company with XAIR to answer the questions in the problem space.

\subsection{\colorwhen{When to Explain?}}
\label{sub:framework:when}
We first introduce the \textit{when} part and discuss how to make a choice for delivery options. Fig.~\ref{fig:overview_when} presents an overview.

\subsubsection{\colorwhen{\textcircledmod{A} \textbf{Availability}}}
\label{subsub:framework:when:availability}
\review{The end-user survey results suggested the need for explanations in AR for the majority end-users (\textbf{Finding 1})}. A system should always generate explanations with AI outcomes and make them accessible for users, so that they can have a better sense of agency whenever they need explanations~\cite{liu2021ai,zanzotto_viewpoint_2019,lee1992trust}.

\vspace{0.2cm}\noindent\colorwhen{\textbf{\textit{G1. Make explanations always accessible to provide user agency.}}}

\subsubsection{\colorwhen{\textcircledmod{B} \textbf{Delivery}}}
% \subsubsection{B - Delivery}
\label{subsub:framework:when:delivery}
Aligned with previous work~\cite{ibili2019effect,buchner2022impact,Pielot2017a}, experts also mentioned the risk of cognitive overload in AR \review{(\textbf{Suggestion 3})}. The default option should be to wait until users manually request explanations. An example could be a button with an information icon that enables users to click on it to see an explanation. 

% \colorwhen{\textbf{\textit{G2: By default, don’t trigger explanations automatically, wait until user’s request.}}}

However, there are cases where automatically presenting just-in-time explanations is beneficial~\cite{bhattacharya2017intent,mehrotra2019jointly}. \review{We summarize the three cases based on our two studies (\textbf{Finding 2} about the importance of contextual explanations, and \textbf{Suggestion 1} about the need of considering \textit{User Goal} and \textit{User Profile})}:

1) Cases when users have an expectation mismatch and become surprised/confused about AI outcomes~\cite{dhanorkar_who_2021,brennen2020people}, \review{\ie \textit{User Goal} as Resolving Surprise/Confusion (also reflected by \textit{User State}, which could be detected by AR HMDs using facial expressions and gaze patterns ~\cite{arguel2017inside,umemuro2003detection}).}
An example could be an intelligent reminder to bring umbrella when users are leaving home on a sunny morning (but it will rain in the afternoon). Automatic explanations of the weather forecast could help resolve users' confusion.

\begin{figure*}[!t]
    \centering
    \includegraphics[width=0.83\textwidth]{figures/when-design.png}
    \caption{The "When" Part of XAIR. It contains two major dimensions: (A) Availability and (B) Delivery, highlighted in \textbf{bolded} texts.
    The design choice of dimensions are in \textbf{\textit{italic}} texts (same below for Fig.~\ref{fig:overview_what} and Fig.~\ref{fig:overview_how}).
    For example, \textbf{Delivery} can be either \textbf{\textit{User-trigger}} or \textbf{\textit{Auto-trigger}}.
    Each dimension has factors that should be considered for explanation designs.
    The guidelines G1 and G2 provide advice on these design choices.}
    \label{fig:overview_when}
    \Description[The "When" Part of XAIR.]{The "When" Part of XAIR. It contains two major dimensions: (A) Availability and (B) Delivery. Each dimension has its own design choice. Availability only has "Always Available". Delivery has the choices of "User-trigger" or "Auto-trigger". Each dimension has factors that should be considered for explanation designs. The guidelines G1 and G2 provide advice on these design choices.
G1: Make explanations always accessible to provide user agency.
G2: By default, don't trigger explanations automatically, wait until user's request.
Only trigger explanation automatically when both conditions are met:
(1) Users have enough capacity (e.g., cognitive load, urgency);
(2) Users are surprised/confused, or unfamiliar with the outcome, or the model is uncertain. }
\end{figure*}

2) Cases when users are unfamiliar with new AI outcomes (indicated via history information of \textit{User Profile}), \eg users receive a recommendation of a song that they have never heard before. Just-in-time explanations of the reason can help users to better understand the recommendation.

3) Cases when the model's input or output confidence is low and the model may make mistakes~\cite{bertossi2020data,kenny2021explaining}, \ie \textit{System Goal} as Error Management. \review{For instance, a system turning on a do-not-disturb mode when it detects a user working on a laptop in an office when the AR-based activity recognition confidence was low (\eg 80\%).} Explanations could be a gatekeeper if the detection was wrong and users could calibrate their expectations or adjust the system to improve the detection~\cite{das2021explainable,zhang2020effect}.

All of these cases have the prerequisite that users have enough capacity to consume explanations~\cite{schmidt_transparency_2020,Pielot2015}, \eg users' cognitive load is not high (\review{could be detected via gaze or EEG on wearable AR devices~\cite{xu2018review,zagermann2018studying}}), and users have enough time to do so (inferred based on context).
% We summarize these considerations in the third guideline.

\noindent\colorwhen{\textbf{\textit{G2. By default, don’t trigger explanations automatically, wait until users' request.
Only trigger explanations automatically when both conditions are met:\\
(1) Users have enough capacity (e.g., cognitive load, urgency);\\
(2) Users are surprised/confused, or unfamiliar with the outcome, or the model is uncertain.}}}


\subsection{\colorwhat{What to Explain?}}
\label{sub:framework:what}
In Sec.~\ref{subsub:problem_space_factors:problem_space:what}, we identified that \colorwhat{\textit{content}} and \colorwhat{\textit{detail}} were two dimensions of the \colorwhat{\textit{what}} part of the framework. We introduce how to choose among all explanation content types in Fig.~\ref{fig:overview_what}.

\subsubsection{\colorwhat{\textcircledmod{A} \textbf{Content}}}
\label{subsub:framework:what:content}
In AR systems, the AI outcomes are based on factors such as \textit{User State} (\eg user activity), \textit{Contextual Information} (\eg the current environment), and \textit{User Profile} (\eg user preference). These factors also determine the content of different explanation content types.
\review{
To choose the right types, the framework lists three factors to consider and provides recommendations of personalized explanation content types based on the literature (shown as solid check marks in the top table in Fig.~\ref{fig:overview_what}), end-user survey, or expert advice (based on \textbf{Finding 3} and \textbf{Suggestion 1}, shown as hollow check marks).}

1) \textit{System Goal}. Different system goals need different explanations. For example, when a system recommends that users check out a new clothing store (User Intent Discovery), presenting {Examples} of similar stores that users are interested in and {Why} this store is attractive to users can be helpful. When a system wants to calibrate users' expectations about uncertain recipe recommendations (Error Management), showing {Examples} is less meaningful than presenting {How} and {Why} the system recommended this recipe, and {How To} change output if users want to. We leverage some literature on contextualized explanation content types to support our recommendations in the framework~\cite{lim_toolkit_2010,lim_assessing_2009,das2021explainable}.

2) \textit{User Goal}. Similarly, different user goals also require different explanations. For instance, {Certainty} explanations are helpful when users want to make sure an exercise recommendation fits their health plan (Reliability), while such explanations would be not useful when users want to be more aware of which data an AR system uses (Privacy Awareness). Most of these recommendations are supported by previous studies~\cite{lim_assessing_2009,barredo_arrieta_explainable_2020,mohseni_multidisciplinary_2021,rai2020explainable,langer_what_2021,wang_designing_2019}.
Regarding how to identify the user goals to choose explanations, designers can use their expertise to infer them in the context determined by AR systems. \review{In the future, it is also possible for AR systems to combine a range of sensor signals to detect/predict users' goals}~\cite{tsai2018augmented,admoni2016predicting,kim2016understanding}.

% As we mentioned earlier in Sec.~\ref{sec:problem_space_factors}, designers can infer these goals in a given context based on their expertise. Future AR systems may combine various sensor signals to infer users' goal~\cite{tsai2018augmented,admoni2016predicting,kim2016understanding}.

3) \textit{User Profile}, specifically user literacy with AI. For the majority of end-users who are unfamiliar with the AI techniques, we recommend only considering the four content types that users indicated that they would find useful: Input/Output, Why/Why-Not, How, and Certainty \review{(as shown in \textbf{Findings 3} and Fig.~\ref{subfig:survey_results:what}).} If users have high AI literacy, then all types could be considered~\cite{ttc_labs,ehsan2021explainable}.

Elements in \textit{System/User Goal} are not exclusive to each other. If there is more than one goal, these columns can be merged within each factor section to find the union (\ie content types checked in at least one column). Then, one can find the intersection among the three factors' content type sets (\ie overlapping types in all sets) to ensure that these explanations can fulfill all factors simultaneously. We show complete examples in later sections (Sec.~\ref{sec:applications} and Sec.~\ref{sec:evaluation}).
% These considerations lead to our next guideline.

\begin{figure*}[!t]
    % \vspace{-0.4cm}
    \centering
    \includegraphics[width=1\textwidth]{figures/what-design.png}
    % \vspace{-0.5cm}
    \caption{The "What" Part of XAIR. (A) \textbf{Content} and (B) \textbf{Detail} are the two major dimensions. Combining the literature summary (shown as solid check marks) and findings from Study 1 \& 2 (shown as hollow check marks), G3-G5 provide guidelines on choosing appropriate explanation content types and length.}
    \label{fig:overview_what}
    \Description[The "What" Part of XAIR.]{The "What" Part of XAIR. (A) Content and (B) Detail are the two major dimensions. Content has the choice of seven categories, including Input/Output, Why/Why-Not, How, Certainty, Examples, What-If, and How-To. Detail has two choices: "Concise Explanations" and "Detailed Explanations".
Combining the literature summary and findings from Study 1 & 2, G3-G5 provide guidelines on choosing appropriate explanation categories and length.
G3. To determine personalized explanation content, consider three factors: system goal, user goal, and user profile.
G4. By default, display concise explanations by highlighting top categories. Prioritize Why, and choose other categories based on the context.
G5. Always provide users opportunities for agency with the option to explore more detailed explanations upon request.}
    % \vspace{-0.3cm}
\end{figure*}

\colorwhat{\textbf{\textit{G3. To determine personalized explanation content, consider three factors: system goal, user goal, and user profile.}}}

\subsubsection{\colorwhat{\textcircledmod{B} \textbf{Detail}}}
\label{subsub:framework:what:detail}
After selecting the appropriate content, default explanations need to be concise and can be further simplified by highlighting the most important types~\cite{buchner2022impact,baumeister2017cognitive}. General end-users are primarily interested in Why, \review{which is in line with experts' advice (\textbf{Suggestion 3} about default options) and previous literature~\cite{jiang_who_2022,lim_assessing_2009,lim_toolkit_2010}.}
Designers' can leverage their expertise to determine whether other types should be omitted or combined with Why in a specific context.

\colorwhat{\textbf{\textit{G4. By default, display concise explanations with top types. Prioritize Why, and choose other types based on the context.}}}

\review{As a large proportion of Study 1 participants indicated that detailed explanations could be useful (\textbf{Finding 4}), AR systems need to provide an easy portal (an interface widget such as a button) for end-users to explore more details. This can also provide user agency~\cite{lu_exploring_2022}.}

\colorwhat{\textbf{\textit{G5. Always provide users opportunities for agency with the option to explore more detailed explanations upon request.}}}



% Essentially it is a recommendation system with additional input from AR~\cite{da2020recommendation,manotumruksa2018contextual}
% Mix-initiative~\cite{horvitz_principles_1999}
% Personalization \& Customization~\cite{cui2020personalized,qian2013personalized}

% Focus of local explanations for end-users~\cite{lakkaraju2019faithful} Global for experienced \cite{dhanorkar_who_2021,long_what_2020}

% Different types of explanations~\cite{barredo_arrieta_explainable_2020,langer_what_2021,wang_designing_2019}


% Even after selecting an appropriate category subset, showing all of them can be overwhelming for end users, especially with limited cognitive capacity in AR~\cite{buchner2022impact,baumeister2017cognitive}. A summary with key explanation elements would be needed


% ``Findings include only showing explanations that extend prior knowledge~\cite{coppers2018intellingo}, and not to be too ``creepy'' by disclosing too much information~\cite{esteva2017dermatologist}.''~\cite{wang_designing_2019}

\subsection{\colorhow{How to Explain?}}
\label{sub:framework:how}
Finally, we introduce the \colorhow{\textit{how}} part and elaborate from the \colorhow{\textit{modality}} and \colorhow{\textit{paradigm}} perspective (see Fig.~\ref{fig:overview_how}).

\begin{figure*}[!t]
    \centering
    \vspace{-0.2cm}
    \includegraphics[width=0.85\textwidth]{figures/how-design.png}
    \caption{The "How" Part of XAIR. (A) \textbf{Modality} and (B) \textbf{Paradigm} are the major dimensions. Note that \textbf{Paradigm} is only for the visual modality, and it is further broken down into two perspectives: format and pattern. G6-G8 provides guidelines on making the proper design choices.}
    \Description[The "How" Part of XAIR.]{The "How" Part of XAIR. (A) Modality and (B) Paradigm are the major dimensions. Modality has the choice of "Audio" or "Visual". Paradigm is only for the visual modality, and it is further broken down into two perspectives: format and pattern. Format has the choices of "Textual Format" or "Graphical Format". Pattern has the choices of "Implicit Pattern" or "Explicit Pattern". G6-G8 provides guidelines for proper design choices.
G6. By default, adopt the same explanation modality as that of AI output (except haptic→audio). When one modality's load is high, use the other modality.
G7. [Visual] Use text as the primary format. Only use graphics if they are easy to understand.
G8. [Visual] Use implicit patterns if content can be part of the environment. Otherwise, use explicit patterns.}
    \label{fig:overview_how}
\end{figure*}

\subsubsection{\colorhow{\textcircledmod{A} \textbf{Modality}}}
\label{subsub:framework:how:modality}
Considering channel bandwidth, the visual and audio modalities are the two most feasible modalities for AR.
Since explanations usually come during or after AI outcomes, to maintain consistency, the default modality of an explanation should be the same\review{(\textbf{Finding 5} and \textbf{Suggestion 3}).}
In cases when outcomes use a haptic modality (\eg vibration as a reminder), audio channels could be used as necessary (although this should be rare), since the choice of the haptic channel already conveys the need to be subtle.

However, there are also cases where one modality could be overloaded (based on \textit{User State} and \textit{Contextual Information}). For example, when users are driving and a navigation app suggests an alternate detour route, although the AI outcome is visual, the explanation should be audio to avoid visual overload.
When users are in a loud environment, a vibration-based AI outcome needs to use the visual modality for explanations.
\review{These scenarios can be easily detected by AR HMDs.}

\colorhow{\textbf{\textit{G6. By default, adopt the same explanation modality as that of the AI output (except for haptic$\rightarrow$audio). When one modality’s load is high, use another modality.}}}

Note that the modality choice also applies to the manual-trigger case when explanations are not automatically delivered (\gtwo), \eg a button icon for visual modality, a voice trigger for audio modality.

\subsubsection{\colorhow{\textcircledmod{B} \textbf{Paradigm}}}
\label{subsub:framework:how:paradigm}

Experts agreed that the audio design space does not belong within this framework. 
\review{For visual design, after removing the location from our framework (\textbf{Suggestion 2} of redundancy removal)}, we mainly focused on two aspects: \colorhow{format} and \colorhow{pattern}.
Depending on the content (\gfour), the explanation \colorhow{format} can be textual~\cite{lakkaraju2016interpretable,myers2006answering}, graphical~\cite{zeiler2014visualizing,simonyan2013deep}, or both.
Based on the consensus of experts in Study 2, text should be the primary format.
Experts suggested several reasons for this. Text takes up less space in a limited AR interface, and can introduce relatively less cognitive load. Moreover, the textual format is more universal and can cover all types.
Graphics can be used as the secondary format.
For default explanations (\gfive), in addition to displaying a short and concise textual paragraph, simple graphics such as icons can be used to provide additional information.
For detailed explanations (\gsix), more complex graphical formats (\eg example images or heatmaps) can be used as long as they are easy for end-users to understand.

\colorhow{\textbf{\textit{G7. [Visual] Use text as the primary format. Only use graphics if they are easy to understand.}}}

Independent of the \colorhow{format}, explanations can be presented in an implicit or explicit \colorhow{pattern}~\cite{lindlbauer_context-aware_2019,tatzgern_adaptive_2016}. Given the capability of depth sensing and 3D registration in AR, we recommend using the implicit pattern when the explanation content is compatible with the environment (\ie can be naturally embedded as a part of the environment).
For example, for book recommendations, a text cue or a small icon can float on the book to indicate the book's topic that users like (belonging to the Why explanation content type).
When explanations and the environment are not compatible, using an explicit pattern (\eg a dialogue window) can be the back-up option.
With regard to what explanation content is compatible with the environment, designers can leverage their expertise and intuition to propose appropriate embedding patterns for given a context. \review{Future AR systems may first understand the environment using object detection and context recognition algorithms, and then utilize techniques such as knowledge graphs (\ie networks of real-world entities and their relationships)~\cite{chen2020review} to assess the compatibility between the content and the environment.}

\colorhow{\textbf{\textit{G8. [Visual] Use implicit patterns if content can be embedded in the environment. Otherwise, use explicit patterns.}}}
\\

\noindent XAIR can not only serve as a summary of the study findings and the multidisciplinary literature across XAI and HCI, but also guide effective XAI design in AR.
In the next two sections, we provide examples of XAIR-supported applications (Sec.~\ref{sec:applications}), and evaluate XAIR from both designers' and end-users' perspectives (Sec.~\ref{sec:evaluation}).
% In Sec.~\ref{sec:evaluation}, we evaluate our framework from various perspectives.

\section{Applications}
\label{sec:applications}
To demonstrate how to leverage XAIR for XAI design, we present two examples that showcase potential workflows that use XAIR for everyday AR applications (Fig.~\ref{fig:applications}). More details can be found in Appendix~\ref{sub:appendix:details_scenarios:more_details_of_application}.
After determining the key factors for a given scenario, we used the framework (Fig.~\ref{fig:overview_when}-Fig.~\ref{fig:overview_how}) to make design choices based on the factors.

\begin{figure*}[!t]
    \centering
    % \vspace{-0.3cm}
    % \includegraphics[width=1\columnwidth]{figures/application_scenarios.png}
    % \includegraphics[width=0.8\columnwidth]{figures/application_scenarios_2.png}
    \begin{subfigure}[b]{0.49\textwidth}
    \includegraphics[width=1\columnwidth]{figures/application_scenario1.png}
    \caption{Scenario 1: Route Suggestion when Jogging}
    \label{subfig:applications:jogging}
    \end{subfigure}
    \hfill
    \begin{subfigure}[b]{0.49\textwidth}
    \centering
    \includegraphics[width=1\columnwidth]{figures/application_scenario2.png}
    \caption{Scenario 2: Plant Fertilization Reminder.}
    \label{subfig:applications:plant}
    \end{subfigure}
    % \vspace{-0.3cm}
    \caption{Application of XAIR on Two Everyday AR Scenarios. In the second scenario, the hand icon indicates that explanations are manually triggered (the same below). Figures only present the default, concise explanations. Detailed explanations are described in the main text of Sec.~\ref{sec:applications}.}
    \label{fig:applications}
    \Description[Application of XAIR on Two Everyday AR Scenarios.]{The figure illustrates two everyday AR scenarios of how XAIR could be applied. The left subfigure shows a scenario described in section 6.1 in the paper. In the scenario, Nanci is jogging on a quiet trial. Her AR glasses display a map beside her view and recommend a detour. Nancy is curious to know the reason why the new route is recommended. The AR glasses prompt explanations with the user goal of resolving surprise, and the system goal of user intent discovery. The right subfigure shows a scenario described in section 6.2 in the paper. In the scenario, Sarah is recommended by her AR glasses a plant fertilization reminder after chatting with her neighbor on relevant topics. She is concerned about her privacy being invaded. In this case, the system displays an explanation with the user goal of privacy awareness and the system goal of trust building. Both scenarios demonstrate how XAIR could be used to support the design of explainable interfaces in AR glasses.}
\end{figure*}

\subsection{Scenario 1: Route Suggestion while Jogging}
\label{sub:applications:scenario_jogging}
\textit{Scene}. Nancy (AI expert, high AI literacy) is jogging in the morning on a quiet trail. Since it is the cherry-blossom season and Nancy loves cherries, her AR glasses display a map beside her and recommend a detour. Nancy is surprised since this route is different from her regular one, but she is happy to explore it. She is also curious to know the reason this new route was recommended.

\colorwhen{\textit{\textbf{When}}}.
\colorwhen{Delivery}. Nancy has enough cognitive capacity in this scenario. Her \textit{User Goal} is Resolving Surprise. Therefore, an explanation is automatically triggered because the two conditions are met (\gthree).

\colorwhat{\textit{\textbf{What}}}.
\colorwhat{Content}. Other than the \textit{User Goal}, the \textit{System Goal} is User Intent Discovery (exploring a new route to see cherry blossom). Considering Nancy's \textit{User Profile}, she is an expert in AI, so the appropriate explanation content types (\gfour) are Input/Output (\eg ``This route is recommended based on seasons, your routine, and preferences.'') and Why/Why-Not (\eg ``The route has cherry blossom trees that you can enjoy. The length of the route is appropriate and fits your morning schedule.'').
Examples for all seven explanation content types can be found in Appendix~\ref{sub:appendix:details_scenarios:more_details_of_application}.
\\
\colorwhat{Detail}. The AR interface shows the Why as default (\gfive), and can be expanded to show both types in detail (\gsix). Nancy can slow down and click the ``More'' button to see more detailed explanations while standing or walking.

\colorhow{\textit{\textbf{How}}}.
\colorhow{Modality}. The explanation is presented visually, the same as the recommendation (\gseven).
\\
\colorhow{Format}. The default explanation uses text, while the detailed explanation contains cherry-blossom pictures of the new route to help explain the Why (\geight).\\
\colorhow{Pattern}. The explanation is shown explicitly within the route recommendation window (\gnine).

\subsection{Scenario 2: Plant Fertilization Reminder}
\label{sub:applications:scenario_reminder}
\textit{Scene}. Sarah (general end-user, low AI literacy) was chatting with her neighbor about gardening. After she returned home and sat on the sofa, her AR glasses recommended instructions about plant fertilization by showing a care icon on the plant. Sarah is concerned about technology invading her privacy, and wants to know the reason behind the recommendation.

\colorwhen{\textit{\textbf{When}}}.
\colorwhen{\textit{Delivery}}. Although Sarah has enough cognitive capacity, none of the three cases in the second condition of \gthree\ are met (\ie she was familiar with the recommendation and not confused, and the model didn't make a mistake). Therefore, the explanation needs to be manually triggered (\gtwo).

\colorwhat{\textit{\textbf{What}}}.
\colorwhat{Content}. In this case, the \textit{System Goal} is Trust Building (clarifying the usage of data), and the \textit{User Goal} is Privacy Awareness. Sarah's \textit{User Profile} indicates that she is not an expert in AI. According to \gfour, the explanation content type list contains Input/Output, Why/Why-Not, and How.\\
\colorwhat{Detail}. Considering Sarah's concern, the default explanation merges Why and How: ``The system scans the plant's visual appearance. It has abnormal spots on the leaves, which indicate fungi or bacteria infection.'' (\gfive).
For the detailed explanation, the full content of the three types is presented in a drop-down list upon her request (\gsix).

\colorhow{\textit{\textbf{How}}}.
\colorhow{Modality}. Following \gseven, the visual modality is used for both the explanation and the manual trigger (a button beside the plant care icon).\\
\colorhow{Format}. Other than using text as the primary format, the abnormal spots on the leaves are also highlighted via circles to provide an in-situ explanation (\geight).\\
\colorhow{Pattern}. Since the highlighting of spots is compatible with the environment (shown on leaves), it adopts the implicit pattern (\gnine). The rest of the texts of the explanation uses the explicit pattern.

Our two examples demonstrate XAIR's ability to guide XAI design in AR in various scenarios. In Appendix~\ref{sub:appendix:details_scenarios:more_scenarios}, we provide additional everyday AR scenarios to further illustrate its practicality.
\section{Evaluation}
\label{sec:evaluation}

In addition to showing examples to illustrate the use case of XAIR, we also conducted two user studies to evaluate XAIR.
The first study was from the perspective of designers (as XAIR users) to evaluate XAIR's ability to assist designers during their design processes (Sec.~\ref{sub:evaluation:utility}).
The second study was from an end-user perspective and evaluated XAIR's effectiveness at achieving a user-friendly XAI experience in AR. We measured the usability of the real-time AR experiences that were developed based on the design examples proposed by designers (Sec.~\ref{sub:evaluation:effectiveness}).
% The results indicate positive utility and strong effectiveness of XAIR.

\subsection{Study 3: Design Workshops}
\label{sub:evaluation:utility}
We conducted one-on-one design workshops with designers to investigate whether the framework could support their design processes, inspire them to identify new design opportunities, and achieve effective designs.

\subsubsection{Participants}
\label{subsub:evaluation:utility:participants}
Future XAI and AR designers can come from various backgrounds, so we recruited 10 participants (4 Female, 6 Male, Age 32 $\pm$ 6) from a technology company as volunteers.
Three were XAI algorithm researchers, four were product designers, and three were HCI/AR researchers.
All participants were familiar with AI and AR, and none had participated in previous studies.

\subsubsection{Design and Procedure}
\label{subsub:evaluation:utility:materials}
We prepared two AR scenarios, both related to recipe recommendations while preparing meals.

\textbf{Case 1: Reliable Recipe Recommendation}.
Michael works in a sales company (general end-user, low digital literacy). He recently started a high-protein diet due to his workout routine. He opens the fridge and wants to make lunch. His AR glasses present a window on the fridge door and recommend an option that Michael usually has, but Michael wants to make sure that this option fits his recent diet changes.

\textbf{Case 2: Wrong Recipe Recommendation}.
Mary works in an AI company (high AI literacy) and has friends coming over for dinner, who are beef lovers. She opens the fridge and sees steak. However, her AR glasses mistakenly recognize steak as salmon with a medium level of confidence~\footnote{If the system has low-level confidence, the expected cost of making mistakes will be higher than the cost of asking for users' input, so the system should ask for users' confirmation about the ingredients they have on hand before presenting recommendations (\eg asking ``Is this salmon or steak?''). In this scenario, the confidence is at the medium level, thus the system provides recommendations, but  is still aware of the potential to make mistakes.}, and recommends a few recipes that use salmon. She is confused and wonders how she can correct the recommendations.

Since generating explanations is not the focus of the framework, we prepared examples for the seven explanation content types (Appendix~\ref{sub:appendix:details_scenarios:more_details_of_study}). Participants were free to use our examples, or propose their own (without the need to design how an algorithm could generate them).

% We adopted a process similar to the within-subject design: 
Participants first used their expertise and intuition to propose XAI designs for the two cases before being shown the framework. They spent 10 minutes on each case.
Participants were encouraged to think aloud and describe their design via text and simple sketches.
Then, after XAIR was introduced, they spent another 10 minutes following the three parts and eight guidelines and applied them to the two cases, resulting in another version of the design. The order of the two cases was counterbalanced.

To quantify the utility of XAIR, we employed the Creativity Support Index (CSI, 1-10 Likert scale)~\cite{cherry2014quantifying} and System Usability Scale (SUS)~\cite{bangor2008empirical}.
Since both scales were originally designed for tools or systems, the language was modified from ``tools'' and ``system'' to ``framework'' and ``guidelines''.
At the end of the workshop, we conducted a semi-structured interview that began with the question: ``Do you think the framework and guidelines are helpful? If so, in what aspects they are helpful?''
Each workshop lasted 90 minutes. Two researchers independently coded the qualitative data using thematic analysis and discussed it to reach an agreement.


% \subsubsection{Procedure}
% \label{subsub:evaluation:utility:procedure}
% We introduced the background and two scenarios. Participants first spent around 10 minutes on each task and proposed designs of the three sub-questions from scratch.
% After we presented our framework and guidelines, participants spent another 10 minutes on each task and proposed a new version.
% They completed the questionnaire and did the interview after finishing the designs.

% \begin{figure}[b]
\centering
\begin{minipage}{.49\textwidth}
% \begin{figure}
    % \centering
    \begin{subfigure}{.495\textwidth}
    \centering
    \includegraphics[width=1\columnwidth]{figures/design_task1_example1.png}
    \caption{P2, Designer}
    \label{subfig:design_examples_task1:example1}
    \end{subfigure}
    \hfill
    \begin{subfigure}{.495\textwidth}
    \centering
    \includegraphics[width=1\columnwidth]{figures/design_task1_example2.png}
    \caption{P6, XAI Researcher}
    \label{subfig:design_examples_task1:example2}
    \end{subfigure}
    \caption{Design Examples of Task1: Reliable Recipe Rec}
    \label{fig:design_examples_task1}
\end{minipage}
\begin{minipage}{.49\textwidth}
    \begin{subfigure}{.495\textwidth}
    \centering
    \includegraphics[width=1\columnwidth]{figures/design_task2_example1.png}
    \caption{P5, HCI/AR Researcher}
    \label{subfig:design_examples_task2:example1}
    \end{subfigure}
    \hfill
    \begin{subfigure}{.495\textwidth}
    \centering
    \includegraphics[width=1\columnwidth]{figures/design_task2_example2.png}
    \caption{P9, Designer}
    \label{subfig:design_examples_task2:example2}
    \end{subfigure}
    \caption{Design Examples of Task2: Wrong Recipe Rec}
    \label{fig:design_examples_task2}
\end{minipage}
\end{figure}

\subsubsection{Design Results}
\label{subsub:evaluation:utility:design_results}
After using XAIR, nine out of ten participants modified their designs and preferred the updated version. One participant (P7) liked the design as it was and thought that the framework \textit{``perfectly supported the design''}.
Consistency was found among the designs, which indicated that XAIR could effectively guide users through the design process.
For example, Tab.~\ref{tab:design_example_case1} presents two designers' designs (images are rendered based on their proposals) of the reliable recommendation case. Their designs of the \colorwhen{\textit{when}} part and most of the \colorwhat{\textit{what}} part were the same.
Tab.~\ref{tab:design_example_case2} presents another two designers' designs of the wrong recommendation case (Case 2). Similarly, we also found consistent design choices between the two examples.

\renewcommand{\arraystretch}{1.6}
\begin{table*}[t]
\centering
\resizebox{1\textwidth}{!}{
\begin{tabular}{c|c|m{7.3cm}|m{7.3cm}}
\hline \hline
\multicolumn{2}{c|}{\makecell{\text{ }}} & \includegraphics[width=7.3cm]{figures/design_case1_example1.png} & \includegraphics[width=7.3cm]{figures/design_case1_example2.png}\\
\multicolumn{2}{c|}{\makecell{\textbf{Designer}}} & \makecell{\textbf{P2, Product Designer}} & \makecell{\textbf{P6, XAI Researcher}} \\
 \hline
\multirow{3}{*}{\makecell[c]{\\\textbf{Platform-}\\\textbf{Agnostic}\\\textbf{Key Factors}}}  & \textbf{System Goal} & \multicolumn{2}{c}{\makecell[c]{\textbf{User Intent Assistance} (to find a good recipe)}} \\
 & \textbf{User Goal} & \multicolumn{2}{c}{\makecell[c]{\textbf{Reliability} (to make sure the recipe fits the diet)}} \\
 & \textbf{User Profile} & \multicolumn{2}{c}{\makecell[c]{\textbf{User Preference}: High protein food; \textbf{History}: Know these recommended recipes;\\\textbf{AI Literacy}: General end-user, low}} \\ \hdashline
\multirow{2}{*}{\makecell[c]{\\\textbf{AR-Specific}\\\textbf{Key Factors}}} & \textbf{Contextual Info} & \multicolumn{2}{c}{\makecell[c]{\textbf{Location}: Kitchen; \textbf{Time}: Noon; \textbf{Environment}: Various ingredients in the fridge}} \\
 & \textbf{User State} & \makecell[l]{\textbf{Activity}: Opening the fridge to make lunch;\\\textbf{Cognitive Load}: Low} & \makecell[l]{\textbf{Activity}: Opening the fridge to make lunch,\\possibly holding something\\\textbf{Cognitive Load}: Low} \\ \hline
\multirow{2}{*}{\makecell[c]{\\\\\textbf{XAI Designs}\\\textbf{in AR}: \colorwhen{\textbf{\textit{When}}}}} & \colorwhen{Availability} (\gone) & Always available & Same as P2\\
 & \colorwhen{Delivery} (\gtwo) & Manual-trigger, because the second condition of auto-trigger was not met given the \textit{System Goal}, \textit{User Goal}, and \textit{User Profile}. & Same as P2 \\ \hdashline
\multirow{3}{*}{\makecell[c]{\\\\\textbf{XAI Designs}\\\textbf{in AR}: \colorwhat{\textbf{\textit{What}}}}} & \colorwhat{Content} (\gfour) & Input/Output \& Why/Why-Not based on Fig.~\ref{fig:overview_what}'s table & Same as P2\\
 & \colorwhat{Detail - Concise} (\gfive) & An explanation merging the Why and Input content types, as explaining \textit{``showing ingredients is also important''} & An explanation of the Why part as \textit{``it needs to be prioritized''}\\
 & \colorwhat{Detail - Detailed} (\gsix) & A list of the two explanation types in detail & Same as P2; Cherry flower pictures to support the Why explanation \\ \hdashline
\multirow{3}{*}{\makecell[c]{\\\\\textbf{XAI Designs}\\\textbf{in AR}: \colorhow{\textbf{\textit{How}}}}} & \colorhow{Modality} (\gseven) & Visual modality to ensure consistency with the recommendation interface & \makecell[l]{Visual modality for explanations;\\Audio/visual modality for manual trigger if\\the user is/isn't holding something}\\
 & \colorhow{Paradigm - Format} (\geight) & Textual format & Textual format as the primary format; Graphic format (protein icon) to support explanations\\
 & \colorhow{Paradigm - Pattern} (\gnine) & Explicit pattern, presenting texts in the same window as the recommendations & Same as P2\\
\hline \hline
\end{tabular}
}
\vspace{0.1cm}
\caption{Two Design Examples of Case 1: Reliable Recipe Recommendation. Participants' quotes were presented in italic font. Among key factors, P2 and P6 had different thoughts on \textit{User State}, which leads to different design choices of \colorhow{how - modality}. The comparison indicates both consistency and variance between two designers' examples.}
\label{tab:design_example_case1}
\Description{Two Design Examples of Case 1: Reliable Recipe Recommendation: Michael works in a sales company (general end-user, low digital literacy). He recently started a high-protein diet due to his workout routine. He opens the fridge and wants to make lunch. The AR glasses present a window on the fridge door, and recommend an option that Michael usually has, but Michael wants to make sure that this option fits his recent diet changes.
The examples are from P2 and P6. Among key factors, P2 and P6 had different thoughts on User State, which leads to different design choices of "how - modality". The comparison indicates both consistency and variance between the two designers' examples.
P2's design is as follows:
When. P2 thought that Michael's User Goal was to ensure the recipes went along with his diet (Reliability), and the System Goal was to help Michael find suitable recipes (User Intent Assistant). Moreover, Michael was pretty familiar with these recommendations. Therefore, the second condition of G2 was not met, and P2 agreed that the manual trigger was a better idea for delivery.
What. Using XAIR, he identified the three factors (i.e., the goals of the system and the user, plus Michael had a low AI literacy) and narrowed down the content to be Input/Output, and Why/Why-Not. For detail, besides prioritizing Why, P2 also proposed a short paragraph by merging the Input category as he thought explaining ingredients was also important. As for detailed explanations, he decided to show the list of the two explanation categories in detail.
How. P2's design is to adopt the visual modality, which was in line with the recommendation interface. P2 designed a simple button icon to support the manual trigger. The textual format was appropriate to explain the Why/Why-Not and Input/Output reasons. Since a textual paragraph was not easily compatible with the environment, P2 adopted the explicit pattern that presented texts in the same window as the recommendations.
P6's design is as follows:
When. Same as P2, P6 also examined the conditions and chose the manual-trigger option for delivery.
What. P6 selected the same categories as Input/Output and Why/Why-Not, and adopted the same detailed explanations design. However, for the default concise explanations in the detail dimension, P6 mainly focused on the Why part and proposed to add an icon to indicate the high-protein feature (also about the Why explanation).
How. P6 proposed the same visual modality and explicit pattern. Moreover, he also brought up an interesting case when Michael's hands were busy holding ingredients. In this case, P6 proposed to support an audio trigger as an alternative. As for the format, other than using texts as the primary format, P6 further proposed using a simple icon as a secondary format to support explanations.}
% \vspace{-0.7cm}
\end{table*}
\renewcommand{\arraystretch}{1.0}

Meanwhile, we also found variance across participants' designs.
For instance, in Case 1, P6 had a different consideration of \textit{User State} than P2, in which P6 brought up a case where the user could hold something in their hand. In this case, P6 adopted the audio modality for manual trigger (the rightmost column of Tab.~\ref{tab:design_example_case1}).
Moreover, as shown in the rightmost column of Tab.~\ref{tab:design_example_case2}, P9 proposed an interesting tweak that always highlighted ingredients (Input explanation type). Her reason was that it introduced \textit{``ultra-low cognitive cost''}, thus there was no need to check the second auto-trigger condition. \textit{``I don't think it is a violation of the guideline. Instead, I was inspired by the framework to consider this case.''}
This reveals that XAIR is flexible and can support the diverse creativity of users.

\renewcommand{\arraystretch}{1.6}
\begin{table*}[t]
\vspace{-0.1cm}
\centering
\resizebox{1\textwidth}{!}{
\begin{tabular}{c|c|m{7.3cm}|m{7.3cm}}
\hline \hline
\multicolumn{2}{c|}{\makecell{\text{ }}} & \includegraphics[width=7.3cm]{figures/design_case2_example1.png} & \includegraphics[width=7.3cm]{figures/design_case2_example2.png}\\
\multicolumn{2}{c|}{\makecell{\textbf{Designer}}} & \makecell{\textbf{P5, HCI/AR Researcher}} & \makecell{\textbf{P9, Product Designer}} \\
 \hline
\multirow{3}{*}{\makecell[c]{\\\textbf{Platform-}\\\textbf{Agnostic}\\\textbf{Key Factors}}}  & \textbf{System Goal} & \multicolumn{2}{c}{\makecell[c]{\textbf{User Intent Assistance} (to find a good recipe for friends)\\\textbf{Error Management} (to calibrate the user's trust for mid-level recognition confidence)}} \\
 & \textbf{User Goal} & \multicolumn{2}{c}{\makecell[c]{\textbf{Resolve Confusion} (to understand why the recommendations are wrong)}} \\
 & \textbf{User Profile} & \multicolumn{2}{c}{\makecell[c]{\textbf{User Preference}: Meet-lovers (friends); \textbf{AI Literacy}: Expert, high}} \\ \hdashline
\multirow{2}{*}{\makecell[c]{\textbf{AR-Specific}\\\textbf{Key Factors}}} & \textbf{Contextual Info} & \multicolumn{2}{c}{\makecell[c]{\textbf{Location}: Kitchen; \textbf{Time}: Evening; \textbf{Environment}: Various ingredients in the fridge}} \\
 & \textbf{User State} & \multicolumn{2}{c}{\makecell[c]{\textbf{Activity}: Opening the fridge to make dinner; \textbf{Cognitive Load}: Low}} \\ \hline
\multirow{2}{*}{\makecell[c]{\\\\\textbf{XAI Designs}\\\textbf{in AR}: \colorwhen{\textbf{\textit{When}}}}} & \colorwhen{Availability} (\gone) & Always available & Same as P5\\
 & \colorwhen{Delivery} (\gtwo) & Auto-trigger, because both conditions were met given the \textit{System Goal} and \textit{User Goal} & Auto-trigger; Besides, a new tweak to always spotlight ingredients automatically, since it introduced \textit{``ultra-low cognitive cost''} \\ \hdashline
\multirow{3}{*}{\makecell[c]{\\\\\\\textbf{XAI Designs}\\\textbf{in AR}: \colorwhat{\textbf{\textit{What}}}}} & \colorwhat{Content} (\gfour) & Five Types: Input/Output, Why/Why-Not, How-To, Certainty, and How & Same as P5\\
 & \colorwhat{Detail - Concise} (\gfive) & An explanation merging Why, Input, Certainty (color-coding to show ingredient with a mid-level confidence), and How-To (selecting ingredients to change) & An explanation Why and How-To; Besides, Input explanations were shown by spotlighting ingredients, which can be selected and changed (How-To)\\
 & \colorwhat{Detail - Detailed} (\gsix) & A drop down menu of the five types & Same as P5\\ \hdashline
\multirow{3}{*}{\makecell[c]{\\\\\textbf{XAI Designs}\\\textbf{in AR}: \colorhow{\textbf{\textit{How}}}}} & \colorhow{Modality} (\gseven) & Visual modality & Same as P5\\
 & \colorhow{Paradigm - Format} (\geight) & Textual format & Textual format as the primary format; Graphic format (spotlighting boundaries) to denote ingredients\\
 & \colorhow{Paradigm - Pattern} (\gnine) & Explicit pattern, presenting texts in the same window as the recommendations & Explicit pattern for texts (same as P5); Implicit pattern for graphic spotlights\\
\hline \hline
\end{tabular}
}
\vspace{0.1cm}
\caption{Two Design Examples of Case 2: Wrong Recipe Recommendation.}
\label{tab:design_example_case2}
\Description{Two Design Examples of Case 2: Wrong Recipe Recommendation: Mary works in an AI company (high AI literacy) and has friends coming over for dinner, who are beef lovers. She opens the fridge and sees steak. However, the AR glasses mistakenly recognize steak as salmon with mid-level confidence, and recommend a few recipes using salmon. She is confused and wonders how she can correct them.
The examples are from P5 and P9. The comparison again indicates both consistency and variance between two designers' examples.
P5's design is as follows:
When. P5 speculated that the major changes between the two tasks include: the System Goal (User Intent Assistance for finding a good recipe for friends, and Error Management for mid-level confidence), the User Goal (Resolve Confusion), and the User Profile (Mary's high AI literacy, friends' food preference). Both conditions of G3 were fulfilled. Thus P5 chose to deliver explanations automatically.
What. According to the table in the what part of the framework, the three factors led to five categories, including Input/Output, Why/Why-Not, How-To, Certainty, and How. For default explanations, in addition to Why, P5 proposed to color-code the Input to emphasize the ingredient with the mid-level confidence (Certainty), and to add a simple selection-based way to allow Mary to change the salmon (How-To). For detailed explanations, P5 proposed using a drop-down menu to show the five categories.
How. P5 chose to use visual modality, textual format, and explicit pattern to present explanations.
P9's design is as follows:
When. P9 had a similar analysis of the System Goal and User Goal as P5. She further proposed an interesting tweak of delivery: always spotlighting ingredients by showing simple information around them (Input category, G4), since it introduced "ultra-low cognitive cost, thus didn't need to follow G2".
What. P9 proposed the consistent category list following G3, and thus suggested the same detailed explanations design as P5. She decided to present Why and How-To as the default explanations. Moreover, as mentioned in the when part, P9 also proposed to show names and recognition certainty on ingredients as "low-cost" explanations.
How. The three sub-questions in P9's design are closely related. Besides displaying main textual explanations explicitly with recommendations, P9's proposed to use simple graphics in an implicit pattern for low-cost explanations}
\vspace{-0.7cm}
% \vspace{-0.7cm}
\end{table*}
\renewcommand{\arraystretch}{1.0}

% We present design examples with participants' thinking processes.

% \noindent\textbf{Example 1 of Case 1 - P2}.
% Fig.~\ref{subfig:design_examples_task1:example1} presents the overall design of Case 1 proposed by P2 after using XAIR.

% \colorwhen{\textit{\textbf{When}}}.
% At first, P2 proposed to trigger explanations together with recommendations automatically. However, he changed the design after using XAIR. P2 thought that Michael's \textit{User Goal} was to ensure the recipes went along with his diet (Reliability), and the \textit{System Goal} was to help Michael find suitable recipes (User Intent Assistant). Moreover, Michael was pretty familiar with these recommendations. Therefore, the second condition of \gthree\ was not met, and P2 agreed that the manual-trigger was a better idea for \colorwhen{delivery
% } (\gtwo).

% \colorwhat{\textit{\textbf{What}}}.
% Before using XAIR, P2 suggested using Why/Why-Not category as the explanation content. He did not think of the second \colorwhat{detail} dimension.
% Using XAIR, he identified the three factors (\ie the goals of the system and the user, plus Michael had a low AI literacy), referred to the table in Fig.~\ref{fig:overview_what}, and narrowed down the \colorwhat{content} to be Input/Output, and Why/Why-Not (\gfour).
% P2 liked \gfive\ and \gsix\ as they reminded him about this missing part.
% Besides prioritizing Why, P2 also proposed a short paragraph by merging the Input category as he thought explaining ingredients was also important.
% As for detailed explanations, he decided to show the list of the two explanation categories in detail.

% \colorhow{\textit{\textbf{How}}}.
% P2's original design was to display explanations in the text under the recipe recommendation. This design was supported and formalized by XAIR.
% The visual \colorhow{modality} was in line with the recommendation interface, \ie a window on the fridge door (\gseven). Considering \gtwo, P2 designed a simple button icon to support the manual trigger.
% Following \geight, the textual \colorhow{format} was appropriate to explain the Why/Why-Not and Input/Output reasons.
% Since a textual paragraph was not easily compatible with the environment, P2 adopted the explicit \colorhow{pattern} that presented texts in the same window as the recommendations (\gnine).
% P2 was satisfied with the design and glad that the framework could confirm his design.
% \\

% \noindent\textbf{Example 2 of Case 1 - P6}.
% We present another design example of Case 1 by P6. A large proportion of P6's design of Case 1 is consistent with that of P2. We mainly highlight the differences. Fig.~\ref{subfig:design_examples_task1:example2} presents P6's design.

% \colorwhen{\textit{\textbf{When}}}.
% Same as P2, P6 also examined the conditions and chose the manual-trigger option for \colorwhen{delivery} (\gtwo).

% \colorwhat{\textit{\textbf{What}}}.
% P6 selected the same categories as Input/Output and Why/Why-Not, and adopted the same detailed explanations design. However, for the default concise explanations in the \textit{detail} dimension, P6 mainly focuses on the Why part and proposed to add an icon to indicate the high-protein feature (also about the Why explanation, \gfive).

% \colorhow{\textit{\textbf{How}}}.
% P6 proposed the same visual \colorhow{modality} and explicit \colorhow{pattern}.
% Moreover, he also brought up an interesting case when Michael's hands were busy holding ingredients. In this case, P6 proposed to support an audio trigger as an alternative (\gseven).
% As for the \colorhow{format}, in addition to using texts as the primary format, P6 further proposed to use a simple icon as a secondary format to support explanations (\gfive\ and \geight).
% \\



% \noindent\textbf{Example 1 of Case 2 - P5}.
% We then present two examples of Case 2. Fig.~\ref{subfig:design_examples_task2:example1} shows P5's design after using XAIR.

% \colorwhen{\textit{\textbf{When}}}.
% P5 speculated that the major changes between the two cases include: the \textit{System Goal} (User Intent Assistance for finding a good recipe for friends, and Error Management for mid-level confidence), the \textit{User Goal} (Resolve Confusion), and the \textit{User Profile} (Mary's high AI literacy, friends' food preference).
% Both conditions of \colorwhen{G3} were fulfilled. Thus P5 chose to \colorwhen{deliver} explanations automatically. This was in line with P5's design before using XAIR.

% \colorwhat{\textit{\textbf{What}}}.
% Prior to knowing XAIR, P5 picked How-To as the explanation. This \colorwhat{content} design changed after referring to \gfour's table. The three factors led to five categories, including Input/Output, Why/Why-Not, How-To, Certainty, and How.
% For default explanations (\gfive), in addition to Why, P5 proposed to color-code the Input to emphasize the ingredient with the mid-level confidence (Certainty), and to add a simple selection-based way to allow Mary to change the salmon (How-To).
% For detailed explanations (\gsix), P5 proposed to use a drop-down menu to show the five categories.

% \colorhow{\textit{\textbf{How}}}.
% P5 followed \gseven\ to \gnine and chose to use visual \colorhow{modality}, textual \colorhow{format}, and explicit \textit{pattern} to present explanations. 
% These choices were consistent with his original design.
% \\


% \noindent\textbf{Example 2 of Case 2 - P9}.
% We show another example of Case 2 designed by P9, as visualized in Fig.~\ref{subfig:design_examples_task2:example2}.

% \colorwhen{\textit{\textbf{When}}}.
% P9 had a similar analysis of the \textit{System Goal} and \textit{User Goal} as P5, and loved how \gone-\gthree\ led to the choice of auto-trigger of explanations. She further proposed an interesting tweak of \colorwhen{delivery}: always spotlighting ingredients by showing simple information around them (Input category, \gfour), since it introduced \textit{``ultra-low cognitive cost, thus didn't need to follow G3''}.

% \colorwhat{\textit{\textbf{What}}}.
% P9 proposed the consistent category list following \gfour, and thus suggested the same detailed explanations design as P5 (\gsix). She decided to present Why and How-To as the default explanations. Moreover, as mentioned in the \colorwhen{\textit{when}} part, P9 also proposed to show names and recognition certainty on ingredients as ``\textit{low-cost}'' explanations.

% \colorhow{\textit{\textbf{How}}}.
% The three sub-questions in P9's design are closely related. Besides displaying main textual explanations explicitly with recommendations, P9's proposed to use simple \colorhow{graphics} in an implicit \colorhow{pattern} for low-cost explanations.

\begin{figure}[b]
    \centering
% \begin{minipage}[t]{.49\columnwidth}
    \centering
    \includegraphics[width=1\columnwidth]{figures/utility_study_csi.png}
    \vspace{-0.3cm}
    \label{subfig:design_examples_task1:example2}
    \caption{CSI Scores of Design Workshops in Study 3}
    \label{fig:workshop_utility_csi}
    \vspace{-0.3cm}
    \Description[CSI Scores of Design Workshops in Study 3]{The figure shows a bar chart of the mean ratings from experts on Creativity SUpport Index CSI scores in study 3. The Y-axis is the mean score on a 10-point Likert scale. The X-axis contains six categories, including Exploration, Expressiveness, Enjoyment, Results in Worth Effort, Collaboration, and Immersion. The Exploration category received a mean rating of 7.9, and the enjoyment category received a mean rating of 8.0, which indicates the good utility of XAIR framework in supporting creative designs by experts in the domain. The immersion category received a lower score with a mean of 4.4, which is due to the comprehensiveness of the framework.}
% \end{minipage}
% \hfill
% \begin{minipage}[t]{.49\columnwidth}
%     \centering
%     \includegraphics[width=0.8\columnwidth]{figures/enduser_study_xai.png}
%     \vspace{0pt}
%     \caption{End-user Evaluation of The AR System in Study 4. Users had positive experience in both tasks. Note that tasks were evaluated separately and not meant to compare against each other.}
%     \label{fig:enduser_usability_xai}
% \end{minipage}
\end{figure}

\subsubsection{Feedback Results}
\label{subsub:evaluation:utility:feedback_results}
Participants provided positive feedback about the framework. Eight participants explicitly commented that XAIR was \textit{``useful/helpful''}.
The results of the CSI scores (Fig.~\ref{fig:workshop_utility_csi}) and the SUS scores (74 $\pm$ 6 out of 100, indicating good usability) both illustrate the good utility of XAIR.
Four themes emerged in participants' feedback.

\textit{The Framework as a Useful and Comprehensive Reference.}
Consistent with the feedback from the experts in Study 2 (Sec.~\ref{sub:methodology:expert_workshops}), participants also found that the framework was a valuable handbook. For example, \pquote{4}{This framework is an excellent reference point for people getting started designing XAI experiences... to check if they have missed things} and \pquote{7}{I may not use it for every design decision, but I would refer to it when I want to make sure that I have considered everything.}
The comprehensiveness of XAIR thus helped participants perform a sanity check of their designs.

\textit{Design Opportunity Inspiration.}
Participants also leveraged XAIR to inspire new ideas.
P6's original design did not consider the case where users' hands could be busily holding ingredients. But the modality in the \colorhow{how} part inspired him, \ie
\textit{``The framework reminded me to realize potential alternatives. It inspired me to think about not just one design, but a set of designs.''}
Moreover, participants found that XAIR could help generate baseline designs. \pquote{8}{I could then further customize it for various scenarios.}
The high scores for exploration (7.9$\pm$0.4 out of 10) and expressiveness (7.1$\pm$0.6) on the CSI also support this observation.

\textit{Backing Up Design Intuitions.}
Some participants also found that the guidelines in XAIR could support their intuition.
For instance, P7 did not change her design after using XAIR, but was very excited to see the alignment, \eg \textit{``Sometimes I am not sure whether my design intuition is right. It feels great that the framework can support it.''}
This could be part of the reason for the positive enjoyment score on the CSI (8.0$\pm$0.4).

\textit{Time to Learn The Framework.}
Participants also commented that XAIR incorporates a lot of information and that they needed time to digest it, \eg \pquote{10}{I need to go back and forth between the visual diagrams} and \pquote{4}{the table \text{[in Fig.~\ref{fig:overview_what}]} is useful but also pretty complex}.
This may explain the relatively low immersion score (4.4$\pm$0.5) on the CSI. Moreover, six participants Agreed or Strongly Agreed in response to the question \textit{``Need to learn a lot...''} on the SUS.
On the one hand, this shows XAIR's comprehensiveness (covering multiple research domains), whereas on the other hand, this illuminates future directions to convert XAIR into a design tool.

% \textit{Potential of An Automatic Framework.}
% Interestingly, several participants asked about the potential of using XAIR as an automatic toolkit. 
% For example, P3 was thinking aloud when using XAIR in the study, \textit{``If this framework is described as an algorithm, the five key factors can be viewed as the input of the algorithm... and the output is the design of the three questions.''}
% We elaborate more on this topic in the discussion section.

\subsection{Study 4: Intelligent AR System Evaluation}
\label{sub:evaluation:effectiveness}
To demonstrate XAIR's effectiveness, we show that the designers' proposals using XAIR could achieve a positive XAI user experience in AR for end-users.
Based on the designs proposed in Study 3, we took one example from each case and implemented a real-time intelligent AR system. We then evaluated the system's usability.

\subsubsection{System Implementation}
We selected one reliable recipe recommendation example from the left of Tab.~\ref{tab:design_example_case1} and one wrong recipe recommendation example from the left of Tab.~\ref{tab:design_example_case2}. We then instantiated the examples by implementing a real-time system on a Microsoft Hololens V2.
The system had three major modules: a recognition module, a recommendation module, and an interface module.

For ingredient recognition, we trained a vision-based object detection model that was a variant of the Vision Transformer from CLIP~\cite{radford2021learning} on the LVIS~\cite{gupta2019lvis} and Objects365~\cite{shao2019objects365} datasets.
We then added ImageNet22k and performed weakly-supervised training with both box and image level annotations~\cite{zhou2022detecting}.
The top 50 ingredient-related classes from LVIS were retained, with an average F1 score of 81.1\%.
The model was run on Hololens' egocentric camera stream at 5 FPS to recognize ingredients.
The model was used in Case 1, while in Case 2, misrecognition (\ie recognizing steak as salmon) was manually inserted to create the designed experience.

For recipe recommendation, the Spoonacular Food API~\cite{spoonacular} was used to obtain potential recipes given a set of ingredients. We then implemented an algorithm to rank the recipes based on user preference and recommend the top recipes (\eg if a user prefers food that is fast to prepare, the recipes are sorted based on the cooking time). For the explanations, we developed a template-based explanation generation technique~\cite{zhang_explainable_2020} to cover different 
types.

Finally, the interface followed the designs in Tab.~\ref{tab:design_example_case1} and Tab.~\ref{tab:design_example_case2}.
Clicking on one recipe's image would show the detailed instructions. An icon button under each recipe could be triggered to present short default explanations, followed by another button to display detailed explanations as a list of content types.

\begin{figure}
    \vspace{-0.1cm}
    \centering
    \begin{subfigure}[b]{1\columnwidth}
    \centering
    \includegraphics[width=0.7\columnwidth]{figures/enduser_study_setup.png}
    % \vspace{0.3cm}
    \caption{Evaluation Setup}
    \label{subfig:study_end_user_eval:setup}
    \end{subfigure}
    \begin{subfigure}[b]{1\columnwidth}
    \centering
    \includegraphics[width=1\columnwidth]{figures/enduser_study_xai.png}
    \caption{Evaluation Scores}
    \label{subfig:study_end_user_eval:scores}
    \end{subfigure}
    \vspace{-0.6cm}
    \caption{
    \review{End-user Evaluation of The AR System in Study 4. (a) Study Setup. (b) Evaluation Scores. Users had positive experience in both tasks. Note that tasks were evaluated separately and not meant to compare against each other.}
    }
    \label{fig:study_end_user_eval}
    \Description[End-user Evaluation of The AR System in Study 4.]{(a) The figure shows the study setup. A man was wearing a Hololens, looking at a shelf, with his hand reaching toward the shelf. There are a lot of vegetables and fruits on the shelf.
    (b) The figure shows a bar chart illustrating the result of our evaluation of the AR system derived from the XAIR framework in section 7.2. The Y-axis is the mean ratings from the end users on a 7-point Likert scale. The X-axis is the seven evaluation categories: intelligibility, transparency, trustworthiness, design of when, design of what, and design of how. In both task 1, reliable recipe recommendation, and task 2, recipe recommendation with error, our system obtained positive ratings from end users in terms of all categories. Note that among these categories, intelligibility, transparency, and trustworthiness obtained especially high ratings, with a mean of 6 and above. The figure demonstrates the strong usability of the XAIR framework in providing satisfying XAI experiences in AR, as described in section 7.4.1.
}
    \vspace{-0.4cm}
\end{figure}

\subsubsection{Participants and Apparatus}
Twelve participants (5 Female, 7 Male, Age 32 $\pm$ 3) volunteered to join the study. None of the them had participated in previous studies. The two cases had the same setup (except for the recognition error). \review{We prepared a number of food ingredients on a shelf (including steak, but no salmon) to simulate the opening-a-fridge moment, as shown in Fig.~\ref{subfig:study_end_user_eval:setup}.}

\subsubsection{Design and Procedure}
% In each task, we had the availability of explanation as the main factor. 
Since there is no existing XAI design for AR systems, we compared the design examples with a baseline condition that only presented recommendations without explanations.
Note that for Case 2's baseline condition, participants could still change the output by clicking a button that said ``Doesn't seem right? Click to see the next batch.'' to ensure a fair comparison \footnote{Another baseline could have been to compare against designers' old designs before using XAIR. However, we did not include this baseline since designers already explicitly preferred the new version that they created after using XAIR.}.

We used a within-subject design. Participants started with one case and completed both conditions. They took a break and completed a questionnaire to compare the two conditions. Then, they completed the two conditions in the second case and completed a similar questionnaire. The case order was counterbalanced. The study took about 30 minutes and ended with a brief interview.

The questionnaire contained six questions (1-7 Likert scale) comparing the two conditions. Three were from the XAI literature and measured the explanations' effect on the system's intelligibility, transparency, and trustworthiness. The other three questions asked about participants' preferences towards the design choices of \colorwhen{\textit{when}}, \colorwhat{\textit{what}}, and \colorhow{\textit{how}}\footnote{Since a factorial study design to compare all XAIR design options would involve a large number of conditions (\ie 2 options of \colorwhen{\textit{when}} $\times$ at least 2 options of \colorwhat{\textit{what}} $\times$ 2 options of \colorhow{\textit{how}}), asking participants to undergo several scenarios would be too costly. Order effects would also be hard to counterbalance. So the three questions about when/what/how described other design choices by showing examples and asked about participants' preferences. For instance, in Case 1's \colorwhen{\textit{when}} part, participants rated how much they agreed with the claim ``I prefer to have explanations triggered manually by me, compared to being triggered automatically.'', or vice-versa in Case 2.}.
The SUS was also administered to measure the usability of the system with explanations.

\subsubsection{Results}
Participants strongly preferred the condition with explanations in both cases, especially Case 2, \eg
\pquote{2}{Seeing the explanation automatically when the AR system makes mistakes is very helpful. It lets me know when I should adjust my expectation} and \pquote{9}{the mistake \text{[in Case 2]} is understandable... salmon and steak can have similar colors and shapes. But if I didn't see the explanation, I would be very confused.}
This sentiment was also reflected in participants' high rating of the system's intelligibility, transparency, and trustworthiness with the explanation (Fig.~\ref{subfig:study_end_user_eval:scores}).
Moreover, the AR system received high SUS scores: 86 $\pm$ 3 in Case 1, and 80 $\pm$ 3 in Case 2, both indicating excellent usability of the system.
Participants also liked the design of the system, which was supported by the positive ratings for the when/what/how questions (see Fig.~\ref{subfig:study_end_user_eval:scores}).
\review{
These results demonstrated that compared to the baseline, XAI design using XAIR can effectively improve the transparency and trustworthiness of AR systems for end-users.
}
% \input{sections/7-applications}
\section{Discussion}
\label{sec:discussion}
XAIR defines the problem space structure of XAI design in AR and details the relationship that exists between the factors and the problem space. By highlighting the key factors that designers need to consider and providing a set of design guidelines for XAI in AR, XAIR not only serves as a reference for researchers, but also assists designers by helping them propose more effective XAI designs in AR scenarios.
The two evaluation studies in Sec.~\ref{sec:evaluation} illustrated that XAIR can inspire designers with more design opportunities and lead to transparent and trustworthy AR systems.
\review{In this section, we discuss how researchers and designers can apply XAIR, as well as potential future directions of the framework inspired by our studies. We also summarize the limitations of this work.}

\review{
\subsection{Applying XAIR to XAI Design for AR}
\label{sub:discussion:procedure}
Researchers and designers can make use of XAIR in their XAI design for AR scenarios by initially using their intuition to propose an initial set of designs.
Then, they can follow the framework to identify five key factors: \textit{User State}, \textit{Contextual Information}, \textit{System Goal}, \textit{User Goal}, and \textit{User Profile}. The example scenarios in Sec.~\ref{sec:applications} and Sec.~\ref{sec:evaluation} indicate how these factors can be specified.
Based on these factors, they would then work through the eight guidelines of \colorwhen{\textit{when}}, \colorwhat{\textit{what}}, and \colorhow{\textit{how}}, using Fig.~\ref{fig:overview_when}-Fig.~\ref{fig:overview_how} to inspect their initial design and make modifications if there is anything inappropriate or missing. Low-fidelity storyboards or prototypes of the designs can be tested via small-scale end-user evaluation studies. This would be an iterative process.
In the future, when sensing and AI technologies are more advanced, it is promising that the procedures of identifying factors and checking guidelines could be automated.
}

\subsection{Towards An Automatic Design Recommendation Toolkit}
\label{sub:discussion:toolkit}
In Study 3, more than one user mentioned the possibility of converting the framework into an automatic toolkit. 
For example, P3 was thinking aloud when using XAIR in the study, \textit{``If this framework is described as an algorithm, the five key factors can be viewed as the input of the algorithm... and the output is the design of the three questions.''}
There are a few decision-making steps in the current framework that involve human intelligence. For example, when designing the default explanations in \colorwhat{what - detail}, designers need to consider users' priority under a given context to determine which explanation content type to highlight. When picking the appropriate visual \colorhow{paradigm}, designers need to determine whether the explanation content is more appropriate in a textual or graphical format, as well as whether the content can be naturally embedded within the environment.
Assuming future intelligent models can assist with these decisions, XAIR could be transformed into a design recommendation tool that could enable designers and researchers to experiment with a set of \textit{User State}, \textit{Contexts}, \textit{System/User Goals}, and so on. 
This could achieve a more advanced version of XAIR, where XAIR are fully automated as an end-to-end model: determining the optimal XAI experience by inferring the five key factors in real time.
This is an appealing direction. However, although factors such as \textit{Context} and \textit{System Goal} are easier to predict with a system, the inference of \textit{User State/Goal} is still at an early research stage~\cite{arguel2017inside,duchowski_index_2018,huang2018predicting}. Moreover, extensive research is needed to validate the adequacy and comprehensiveness of the end-to-end algorithm.
This also introduces the challenge of nested explanations in XAIR (\ie explaining explanations)~\cite{mittelstadt2019explaining}, which calls for further study.

\subsection{The Customized Configuration of XAI Experiences in AR}
\label{sub:discussion:config}
The experts in Study 2 and the designers in Study 3 brought up the need for end-user to control XAI experiences in AR, \eg
\pquote{12, Study 2}{XAIR can provide a set of default design solutions, and users could further customize the system} and
\pquote{8, Study 3}{I personally agree with the guidelines, but I can also imagine some users may want different design options. So there should be some way that allows them to select when/what/how... For example, a user may want the interface to be in an explicit dialogue window all the time \text{[related to \colorhow{how}]}. We should support this.}
This need for control suggests that to achieve a personalized AR system, designers should provide users with methods to configure their system, so that they can set up specific design choices to customize their XAI experience.
Such personalization capabilities may also be used to support people with accessibility needs (also mentioned by P2 in Study 3), \eg visually impaired users can choose to always use the audio \colorhow{modality}.

% ``To this date, there is no agreement in prior work on whether to include all details of system logic in explanation interfaces. Moreover, there might not be a universal answer to this question, but it might rather depend on the product domain, user groups, and the context of use.''~\cite{eiband_bringing_2018}

\subsection{User-in-The-Loop and Co-Learning}
\label{sub:discussion:co_learning}
During the iterative expert workshops (Study 2, Sec.~\ref{sub:methodology:expert_workshops}), experts mentioned an interesting long-term co-learning process between the AR system and a user.
On the one hand, based on a user's reactions to AI outcomes and explanations, a system can learn from the data and adapt to the user. Ideally, as the AR system better understands the user, the AI models would be more accurate, thus reducing the need for mistake-related explanations (\eg cases where \textit{System Goal} as Error Management).
On the other hand, the user is also learning from the system. \pquote{4, Study 2}{Users' understanding of the system and AI literacy may change as they learn from explanations}. This may also affect the user's need for explanations. For example, the user may have less confusion (\textit{User Goal} as Resolving Surprise/Confusion) as they become more familiar with the system. Meanwhile, they may become more interested in exploring additional explanation types (\textit{User Goal} as Informativeness).
Such a long-term and co-learning process is an interesting research question worth more exploration.

\subsection{Limitations}
\label{sub:discussion:limitations}
There are a few limitations to this research.
First, although we highlighted promising technical paths within the framework in Sec.~\ref{sec:framework}, XAIR does not involve specific AR techniques. The real-time AR system in Study 4 implemented the ingredient recognition and recipe recommendation modules, but the detection of user state/goal was omitted.
\review{
Second, our studies might have some intrinsic biases. For example, Study 1 only involved AR recommendation cases. Since everyday AR HMDs are still not widely adopted in daily life, we grouped 500+ participants only based on AI experience instead of AR experience. The experts and designers of our studies were all employees of a technology company. Study 4 only evaluated two specific proposals from designers. Moreover, as there is no previous XAI design in AR, we were only able to compare our XAIR-based system against a baseline without explanation.
}
Third, other than when, what, and how, there could be more aspects in the problem space, \eg who and where to explain. Moreover, XAIR mainly focuses on non-expert end-users. Other potential users, such as developers or domain experts, were not included.
% There are other AR-specific information presentation frameworks that can be further incorporated~\cite{muller2016taxonomy,luo2022should}.
The scope of the five key factors may also not be comprehensive. For example, we do not consider user trust in AI, which is a part of \textit{User Profile} that may be dynamic along with user-system interaction.
These could limited the generalizability of our framework, but also suggests a few potential future work directions to expand and enhance XAIR.
\section{Conclusion and Future Work}

In this paper, we propose an approach for bagging by singulating layers using interactive perception. Experiments show that \bagging achieves significantly higher success rates over baselines for opening a bag, inserting items, and lifting the bag. In future work, we plan to apply this approach to related tasks such as packing and wrapping. 

%We hope that this work will lead to an exciting era in robotic manipulation of bags and 3D deformable objects more generally.

\bibliographystyle{ACM-Reference-Format}
% \bibliography{
% bib/Augmented_Reality,
% bib/Adaptive_UI,
% bib/Behavior_Intervention,
% bib/Human-AI_Interaction,
% bib/Interpretable_ML,
% bib/Modeling_Behavior-General,
% bib/Orson_Publication,
% bib/other
% }

\bibliography{bib/merged.bib}

\section*{appendix}

\section{Theoretical Analysis}
\label{section:theory}

%study the accuracy and efficiency of our randomized algorithms from theoretical perspective. Specifically, we analyze the failure probability of a query and expected evaluated dimensionality of a non-KNN object. 

% \subsection{Failure Probability}
% Let's continue with our analysis in Section 3.2. ARSearch algorithm fails only when some positive objects of a search are wrongly rejected at some level $l$. Let $\mathcal O_+$ be the set of positive objects and $r $ be the radius of distance comparison of positive object $i$. We have
% \begin{align}
%     \mathbb{P} \left\{ fail \right\}  = \mathbb{P} \left\{ \exists l \le L, \exists i \in \mathcal O_{+}, dis ' > r  \cdot \gamma (d_l) \right\}  
% \end{align}
% Note that $r  > dis $ because object $i$ is positive.
% \begin{align}
%     &\le \mathbb{P} \left\{ \exists l \le L, \exists i\in \mathcal O_{+}, dis ' > dis  \cdot \gamma(d_l) \right\}  
% \end{align}
% We then plugin the definition of $dis , dis '$ and $\gamma(d_l)$:
% \begin{align}
%     =\mathbb{P} \left\{ \exists l \le L, \exists i \in \mathcal O_{+},  \sqrt {\frac{D}{d_l} } \left\|\mathbf{y} |_{[1,2,...,d_l]} \right\|
%     %\frac{\left\| \mathbf{y} |_{[1,2,...,d_l]} \right\|}{\gamma(d_l)}   
%     > \left( 1 + \frac{\epsilon _1}{\sqrt {d_l} }  \right) \| \mathbf{y}  \|  \right\} \label{eq:plugin} 
% \end{align}
% Applying union bound, i.e. the probability of the union of events is no greater than the sum of their individual probability, we have:
% \begin{align}
%     \le \sum_{l=1}^{L} \sum_{i\in \mathcal O_+} \mathbb{P} \left\{ \sqrt {\frac{D}{d_l} } \left\|\mathbf{y} |_{[1,2,...,d_l]} \right\|
%     > \left( 1 + \frac{\epsilon _1}{\sqrt {d_l} }  \right)  \| \mathbf{y}  \|  \right\} 
% \end{align}
% With the equivalence between random projection and row sampling with random transformation and the distance-preserving property of random orthogonal transformation, we have:
% \begin{align}
%     &= \sum_{l=1}^{L} \sum_{i\in \mathcal O_+} \mathbb{P} \left\{ \sqrt {\frac{D}{d_{l}} } \left\| P'|_{[1,2,...,d_l]}\mathbf{x}  \right\|   > \left( 1 + \frac{\epsilon _1}{\sqrt {d_l} }  \right)\| \mathbf{x}  \| \right\} 
% \end{align}
% Note that $P'|_{[1,2,...,d_l]}$ is a $d_l \times D$ random projection matrix. Applying Lemma~\ref{eq:concen}, we finally have
% \begin{align}
%     \mathbb{P} \left\{ fail \right\}  &\le \sum_{l=1}^{L} \sum_{i\in \mathcal O_+} \exp(-c_0 \epsilon_1^2) = L \cdot N_{pos} \cdot \exp(-c_0 \epsilon_1^2)  \\
%     &\le D \cdot N_{pos} \cdot \exp (-c_0 \epsilon _{1}^{2})  \label{eq:superexp}
%     %\\ &= \exp (-c_0 \epsilon_1^2 + \log |\mathcal O_+| + \log L)  
% \end{align}
% Letting the failure probability to be no greater than $\delta$ to solve $\epsilon_1$, then we finally have Theorem~\ref{theorem:eps}.
% In terms of Corollary~\ref{corollary:eps}, it simply shifts our concern from all positive objects to KNN objects only. %Note that the corollary is not at all related with the cardinality of a database $N$. %As a result, as a database scales up, there's no need to tune the parameters of our adaptive algorithms. 
% %when $L$ and $|\mathcal{O}_+|$ are not changed.

% \begin{comment}

% For a fixed query, suppose that $N_{vis}$ vectors are visited. For object $i$, our algorithm depends on the fact that $dis^*  \le dis $ with high probability. A failure happens if there exists an object $i$ fails at a level $l$. Thus, the overall failure probability is given as:
% \begin{align}
%     \mathbb{P}\left\{ \exists i \le N_{vis}, l < L, dis ^* > dis  \right\}
% \end{align}
% We plug-in the definiton of $dis $ and $dis^* $: 
% \begin{align}
%     =\mathbb{P} \left\{ \exists i\le N_{vis}, l < L, \sqrt {\frac{D}{d_l} } \frac{\left\| \mathbf{y} |_{[1,2,...,d_l]} \right\|}{\gamma(d_l)}   > \| \mathbf{y}  \|  \right\} \label{eq:plugin}  
% \end{align}
% Applying union bound, i.e. the probability of the union of events is no greater than the sum of their individual probability, we have:
% \begin{align}
%     \le \sum_{l=1}^{L-1} \sum_{i=1}^{N_{vis}} \mathbb{P} \left\{ \sqrt {\frac{D}{d_l} } \frac{\left\| \mathbf{y} |_{[1,2,...,d_l]} \right\|}{\gamma(d_l)}   > \| \mathbf{y}  \|  \right\}  
% \end{align}
% With the equivalence between random projection and row sampling with random transformation and preset constant significance, we finally obtain: 
% \begin{align}
%     &= \sum_{l=1}^{L-1} \sum_{i=1}^{N_{vis}} \mathbb{P} \left\{ \sqrt {\frac{D}{d_{l}} } \left\| P\mathbf{x}  \right\|   > (1 + \frac{\epsilon _1}{\sqrt {d_l} } )\| \mathbf{x}  \| \right\}  
%     \\ &\le \sum_{l=1}^{L-1} \sum_{i=1}^{N_{vis}} \exp(-c_0 \epsilon_1^2) < L \cdot N_{vis} \cdot \exp(-c_0 \epsilon_1^2)  
%     \\ &= \exp (-c_0 \epsilon_1^2 + \log N_{vis} + \log L)  
% \end{align}
% Letting the failure probability be no greater than $\delta$, then $\epsilon_1$ should be $\Theta \left( \sqrt {\log \frac{L\cdot N_{vis}}{\delta}}  \right)$.
% \end{comment}

%{\color{red} To avoid distration, I commented out the discussion about the tightness of the theoretical results.}
%{\color{red} There is a gap between theory and practice. Note that the theory only provides the correct order of $\epsilon_1$. The gap is due to 1) The tail bound of random projection is not tight in constant (but tight in order). 2) Union bound guarantees the failure probability for the worst case among all the datasets, so it highly overestimates the failure probability for real world datasets which have some good properties (they are embeddings produced by models, so they cannot be some arbitrary things.). 3) Failures not necessarily affect the correctness of result. For example, failures happened at negative objects (Non-KNN objects) even benefit the algorithm for it stops unnecessary sampling earlier. //I think it might need discussion.} 


%{\color{red} Note that improving the number of testing number and also overhead from extra priority queue operations. We should be careful about the settings of layers.}

%\subsection{Expected Terminate Dimensionality}
%We next investigate the expected dimensionality of negative objects. For a negative object $i$, supposing that it's compared with radius $r $, let $(1+\alpha )$ be the ratio between $dis $ and $r $ (though we don't know $dis $ and $\alpha $ in prior). Let random variable $\hat D $ be the terminate dimensionality of object $i$. We assume that we do hypothesis testing every time after sampling a new dimension. 
In this section, we prove Lemma~\ref{theorem:ADSampling efficiency} in detail. 
% For a negative object, let $\alpha = (dis - r) / r$ be the gap between $dis$ and $r$. Let random variable $\hat D$ be the terminate dimension (corresponds to time complexity). 
We assume that {\CHENGB we sample one additional dimension of $\mathbf{y}$ each time.}
% we do hypothesis testing each time after sampling one dimension. 
Let $\gamma(d) = (1 + \epsilon_0 / \sqrt {d} )$. 
% \begin{proof}
% Note that $\hat D$ is a non-negative random variable. 
We have
\begingroup
\allowdisplaybreaks
\begin{align}
    \mathbb{E} \left[ \hat D \right]  
    % =& \sum_{d=1}^{+\infty} \mathbb{P} \left\{ \hat D \ge d \right\}
    =& \sum_{d=1}^{D} d\cdot \mathbb{P} \left\{ \hat D = d \right\}
    = \sum_{d=1}^{D} \mathbb{P} \left\{ \hat D \ge d \right\}
% \end{align}
% {\CHENGB Here,} $\hat D \ge d$ {\CHENGB represents the event that} all the previous hypothesis testing cannot reject the hypothesis, i.e.,
% \begin{align}
    \\=& \sum_{d=1}^{D} \mathbb{P} \left\{  \forall p<d,  \sqrt {\frac{D}{p}}  \| \mathbf{y}|_{1,2,...,p}\| \le \gamma(p) \cdot r  \right\}      \label{reduction:interpret}
    \\=& \sum_{d=1}^{D} \mathbb{P} \left\{  \forall p<d, \sqrt {\frac{D}{p}} \| \mathbf{y}|_{1,2,...,p}\| \le \gamma(p) \cdot \frac{\| \mathbf{y}\| }{1+\alpha} \right\} 
% \end{align}
% We relax the condition of all previous hypothesis testing to the last testing:
% %because the filter-out probability decays exponentially. The last testing dominates the filter-out probability. 
% \begin{align}
    \\\le& 1 + \sum_{d=1}^{D-1} \mathbb{P} \left\{ \sqrt {\frac{D}{d}} \| \mathbf{y}|_{1,2,...,d}\| \le \gamma(d) \cdot \frac{\| \mathbf{y}\| }{1+\alpha}  \right\}    \label{reduction: relax hypothesis testing}  
    % \\\le&  1 + d_0   + \sum_{d=d_0 +1 }^{D} \mathbb{P} \left\{ \sqrt {\frac{D}{d}} \| \mathbf{y}|_{1,2,...,d}\| \le \frac{\gamma (d)}{1+\alpha } \| \mathbf{y} \| \right\}  \label{reduction:separate analyze}
\end{align}
\endgroup
where (\ref{reduction:interpret}) is because $\hat D \ge d$ {\CHENGB represents the event that} all the previous hypothesis testings cannot reject the hypothesis and (\ref{reduction: relax hypothesis testing}) relaxes the event corresponding to all the testings (i.e., $\forall p < d$) to that corresponding to the last testing (i.e., $p=d-1$). 
% Let $\tilde d= \epsilon _{0}^{2} /\alpha^2$ and $d_0 = \mathrm{ceil} (\tilde d)$. (\ref{reduction:separate analyze}) relaxes the probability for $d \le d_0$ to 1. 

We denote $\tilde d:= \epsilon _{0}^{2} / \alpha^2, d_0 := \mathrm{ceil} (\tilde d) $ and relax the probability for $d \le d_0$ to $1$:
\begin{align}
    \mathbb{E} \left[ \hat D \right]  \le  1 + d_0   + \sum_{d=d_0 +1 }^{D} \mathbb{P} \left\{ \sqrt {\frac{D}{d}} \| \mathbf{y}|_{1,2,...,d}\| \le \frac{\gamma (d)}{1+\alpha } \| \mathbf{y} \| \right\}  \label{reduction:separate analyze}
\end{align}
% Then we separately analyze the cases when 1) $\gamma(d) \ge 1+\alpha$ and 2) when $\gamma(d) < 1+\alpha $ for different $d$.
% %, because for $d$ whose $\gamma(d) \ge \alpha + 1$, the bound provided by Lemma~\ref{eq:concen} is trivial. 
% We denote
% \begin{align}
%     \tilde d = \lceil \frac{1}{\alpha^2} \cdot \epsilon_0^2  \rceil 
%     %= \Theta \left( \frac{1}{\alpha ^2} \cdot \log \frac{D \cdot K}{\delta} \right)
%     %\max \left( \log \frac{D}{\delta} ,\log \frac{K}{\delta}  \right) 
% \end{align}
% Note that we have $\gamma(d) \ge \alpha + 1$ for $d \le \tilde d$ and $\gamma(d) < \alpha+1$ for $d > \tilde{d}$. 
%Without loss of generality we assume that $l_0$ is an integer (otherwise, we let $l_0$ be $\lfloor \epsilon_1^2/\alpha ^2 \rfloor $). 
% With relaxing the probability for $d \le \tilde d$ to $1$, we have
% \begin{align}
%     \mathbb{E} \left[ \hat D  \right]  \le  1 + d_0   + \sum_{d=d_0 +1 }^{D} \mathbb{P} \left\{ \sqrt {\frac{D}{d}} \| \mathbf{y}|_{1,2,...,d}\| \le \frac{\gamma (d)}{1+\alpha } \| \mathbf{y} \| \right\}  \label{reduction:separate analyze}
% \end{align}
% Note again that now for $d > \tilde{d}$, $\gamma(d) /(1+\alpha) < 1$. 
% Note that for $d>d_0$, $\gamma(d) < 1 + \alpha$. 
Let's focus on the last term of (\ref{reduction:separate analyze}) and deduce from it as follows,
\begingroup
\allowdisplaybreaks
\begin{align}
    &\sum_{d=d_0 +1 }^{D} \mathbb{P} \left\{ \sqrt {\frac{D}{d}} \| \mathbf{y}|_{1,2,...,d}\| \le \frac{\gamma (d)}{1+\alpha } \| \mathbf{y} \| \right\}   \\
    =&\sum_{d=d_0 +1}^{D} \mathbb{P} \left\{ \sqrt {\frac{D}{d}} \| \mathbf{y}|_{1,2,...,d}\| \le \left[ 1- \left( 1-\frac{\gamma(d)}{1+\alpha}  \right)  \right]   \| \mathbf{y} \| \right\}    \label{reduction:rewrite}
% \end{align}
% Then with Lemma~\ref{eq:concen}, we have 
% \begin{align}
    \\\le& \sum_{d=d_0+1}^{D} \exp \left[ -c_0 d \left( 1 - \frac{\gamma(d)}{1+\alpha}  \right)^2 \right]  
    % \qquad \text{(Lemma~\ref{eq:concen})}  
    \label{reduction:lemma3.1}
    \\=&\sum_{d=d_0+1}^{D} \exp \left[ -\frac{c_0 \alpha^2}{(1+\alpha)^2}\left( \sqrt {d} - \sqrt{\tilde{d}}  \right) ^2  \right]  \label{reduction:expand and sort out}
% \end{align}
% Note that this expression is monotonically decreasing with respect to $d$, so the summation is bounded by the integration:
% \begin{align}
    \\\le& \int_{d_0}^{D} \exp \left[ -\frac{c_0 \alpha^2}{(1+\alpha)^2}\left( \sqrt {x} - \sqrt {\tilde{d}}  \right) ^2  \right] \mathrm{d} x  \label{reduction:integration}
    \\=& \int_{d_0}^{D} \exp \left[ -\frac{c_0 \epsilon_0^2}{(1+\alpha)^2}\left( \sqrt {\frac{x}{\tilde{d}} } - 1 \right) ^2  \right] \mathrm{d} x  
% \end{align}
% Let $u = x / \tilde{d}$, then we have: 
% \begin{align}
    \\=&\tilde{d} \int_{d_0 /\tilde d}^{D/\tilde{d}} \exp \left[ - \frac{c_0 \epsilon_0^2}{(1+\alpha)^2} \left( \sqrt {u} -1 \right)^2  \right]  \mathrm{d} u  
    % \quad \text{(Let}~u= x/\tilde d) 
    \label{reduction:substitute u}
    \\\le& \tilde{d} \int_{1}^{+\infty} \exp \left[ - \frac{c_0 \epsilon_0^2}{(1+\alpha)^2} \left( \sqrt {u} -1 \right)^2  \right]  \mathrm{d} u \label{reduction:relaxing to inf}
\end{align}
\endgroup
{\CHENGC where
(\ref{reduction:rewrite}) rewrites it to fit the format of Lemma~\ref{eq:concen}, 
(\ref{reduction:lemma3.1}) applies Lemma~\ref{lemma:concen}, 
(\ref{reduction:expand and sort out}) plugs in $\gamma(d)$, 
(\ref{reduction:integration}) relaxes (\ref{reduction:expand and sort out}) to an integration,
(\ref{reduction:substitute u}) substitutes $u = x / \tilde d$, and
(\ref{reduction:relaxing to inf}) relaxes the integration to $[1,+\infty)$.}
{\JIANYANG 
Next we first analyze the case of $\alpha \le \epsilon_0$ as follows.
% Under the region of reasonable accuracy, we can assume $\epsilon_0 \ge 1$ and yield (\ref{reduction:eps >= 1}). 
}
% Without loss of generality, we assume that $\tilde{d} \ge 1$ because when $\tilde{d} < 1$ the distance gap $\alpha$ is larger than the error bound $\epsilon_0$, in which constant dimensions are enough to provide firmed comparison results. Then we have $\epsilon_0 \ge \alpha $. 
\begin{align}
    \text{(\ref{reduction:relaxing to inf})}\le& \tilde{d} \int_1^{+\infty} \exp \left[ - \frac{c_0 \epsilon_0^2}{(1+\epsilon_0)^2} \left( \sqrt {u} -1 \right)^2  \right]  \mathrm{d} u  \label{reduction:alpha <= eps}
% \end{align}
% Under the region of reasonable accuracy, we can assume that $\epsilon_0\ge 1$. 
% %With {\color{red} Lévy's Concentration Inequality~\cite{wainwright_2019}}, the constant $c_0$ is no smaller than $1/2$. 
% Then we have:
% \begin{align}
    %&\le \tilde{d} \int_1^{+\infty} \exp \left[ - \frac{1}{8} \left( \sqrt {u} -1 \right)^2  \right]  \mathrm{d} u
    \\\le& \tilde{d} \int_1^{+\infty} \exp \left[ - \frac{c_0}{4} \left( \sqrt {u} -1 \right)^2  \right]  \mathrm{d} u \label{reduction:eps >= 1}
\end{align}
{\CHENGC where (\ref{reduction:alpha <= eps}) is because $\alpha \le \epsilon_0$ and (\ref{reduction:eps >= 1}) is yielded when setting $\epsilon_0 \ge 1$ for reasonable accuracy.}
Note that the integration is convergent so as to be bounded by a constant. For the case of $\alpha \le \epsilon_0$, we have 
\begin{align}
    \mathbb{E} \left[ \hat D  \right]  = 1 + d_0 + O(\tilde{d}) = O(\tilde{d}) = O \left( \alpha^{-2} \cdot \epsilon_0^2 \right)  \label{eq:quadratic}
\end{align}
{\JIANYANG
{\CHENGB For the case of} $\alpha > \epsilon_0$, its expected dimensionality must be no greater than the case of $\alpha =\epsilon_0$ because its distance gap $\alpha$ is larger. Thus, its expected dimensionality is upper bounded by $O(1)$.
% , i.e., a constant.
}
%Applying the $\epsilon_0$ given in Theorem~\ref{theorem:eps}, we have Theorem~\ref{theorem:efficiency}.
% \end{proof}

% \begin{comment}

% \begin{theorem}
% For a KNN query in $D$-dimensional space, let $(1+\alpha )$ be the ratio between negative object $i$ and its corresponding largest positive object. Then the expected evaluated dimensionality of $i$ is 
% \begin{align}
%     %\mathbb{E} \left[ \hat d  \right]  = O \left( \frac{1}{\alpha ^2} \cdot \log \frac{L \cdot |\mathcal O_+|}{\delta} \right)  
%     %\mathbb{E} \left[ \hat d  \right]  = O \left( \min ( D,  \frac{1}{\alpha ^2}\log D + \frac{1}{\alpha ^2}\log \frac{K}{\delta} )  \right)   
%     \mathbb{E} \left[ \hat d  \right]  = O \left(   \frac{1}{\alpha ^2}\log D + \frac{1}{\alpha ^2}\log \frac{K}{\delta} \right)   
% \end{align}
% \end{theorem}
% \end{comment}



% \section{Product Quantization}

% \begin{figure}[hbt]
%     \centering
%     \vspace{-4mm}
%     \subfigure[\texttt{IVFPQ 64}]{
%         \includegraphics[width=0.45\linewidth]{revision experimental result/IVFPQ_64.pdf}
%        \label{fig:cost IVFPQ64}
%     }
%     \subfigure[\texttt{IVFPQ 128}]{
% 	    \includegraphics[width=0.45\linewidth]{revision experimental result/IVFPQ_128.pdf}
% 	    \label{fig:cost IVFPQ128}
%     }
%     \vspace{-4mm}
%     \caption{{\CHENG Breakdown of Running Times of} IVFPQ.}
%     % \vspace{-4mm}
%     \label{fig:cost statistics-IVF}
% \end{figure}

% \begin{figure}
%     \centering
%     \vspace{-4mm}
%     \includegraphics[width=\linewidth]{revision experimental result/QPS_linear.png}
%     \vspace{-4mm}
%     \caption{Efficiency of DCOs.}
%     \label{fig:qps linear}
% \end{figure}
{\JIANYANGREVISION
\section{Results of Tree-based and Hashing-based Methods}
\label{appendix:section tree and hashing}
\begin{figure*}[thb]
% \vspace*{-4mm}
  \centering 
  % \includesvg[width=17cm]{experimental result/time-accuracy.svg}
  % \includesvg[width=17cm]{revision experimental result/time-accuracy-hash.svg}
    \includegraphics[width=17cm]{revision experimental result/time-accuracy-hash.pdf}
  \vspace*{-4mm}
  \caption{{\JIANYANGREVISION Time-Accuracy Tradeoff (\texttt{PMLSH} and \texttt{Annoy}).}}
  \vspace*{-4mm}
  \label{figure:time-accuracy-tree-and-hashing}
\end{figure*}
For \texttt{Annoy}, following \cite{li2019approximate}, we set the number of trees $N_{tree} = 50$. 
{\JIANYANGREVISION During the \underline{index phase}, we feed the raw data vectors into the indexing algorithm of \texttt{Annoy} (note that \texttt{Annoy}, \texttt{Annoy+} and \texttt{Annoy}* have the same index structure). Then during the \underline{query phase}, for \texttt{Annoy}/\texttt{Annoy}*, we load the index and the raw data vectors into main memory, generate candidates by feeding the raw query vector into the the query algorithm of \texttt{Annoy} and re-rank the candidates with \texttt{FDScanning}/\texttt{PDScanning}. For \texttt{Annoy+}, we load the index and the transformed data vectors into main memory, generate candidates by feeding the raw query vector into the the query algorithm of \texttt{Annoy} and re-rank the candidates with \texttt{ADSampling}. }
For \texttt{PMLSH}, following \cite{zheng2020pm}, we set the dimensionality of random projection as 15, the size of internal and leaf nodes of the PM-Tree as 16. Similar to \texttt{Annoy}, during the \underline{index phase}, we build the indexes based on the raw data vectors. During the \underline{query phase}, we generate candidates by feeding the raw query vectors to the search algorithm of \texttt{PMLSH} and re-rank them with \texttt{FDScanning}, \texttt{PDScanning} and \texttt{ADSampling} based on raw vectors, raw vectors and transformed vectors respectively. 
For both methods, we vary the number of accessed candidates to control the time-accuracy tradeoff.  
{\chengr We exclude the optimization of data layout for this experiment since it is not applicable for index ensembles (e.g., tree ensembles of \texttt{Annoy}).}
% 2) according to Section~\ref{subsec:main result}, the \texttt{ADSampling} method contributes the most improvement for \texttt{IVF}, (3) as shown in Figure~\ref{figure:time-accuracy-tree-and-hashing}, \texttt{Annoy+} and \texttt{PMLSH+} have already brought consistent and significant improvement 
% and (4) \texttt{Annoy} and \texttt{PMLSH} are not the focus of our paper due to their suboptimal performance,
% {\chengr we } we exclude the optimization of data layout.
We plot the QPS-recall and QPS-average distance ratio curves {\chengr of the compared algorithms} in Figure~\ref{figure:time-accuracy-tree-and-hashing}. It shows that \texttt{AKNN+} outperforms the \texttt{AKNN}* and \texttt{AKNN} algorithms consistently and significantly. 



}

\end{document}
\endinput
%%
%% End of file `sample-manuscript.tex'.
