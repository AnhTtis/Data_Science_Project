\renewcommand{\arraystretch}{1.4}
\begin{table}[hbtp]
\centering
\resizebox{1\textwidth}{!}{
\begin{tabular}{c|c|m{7cm}|m{7cm}}
\hline \hline
\multicolumn{2}{c|}{\makecell{\text{ }}} & \includegraphics[width=7cm]{figures/application_scenario1.png} & \includegraphics[width=7cm]{figures/application_scenario2.png}\\
\multicolumn{2}{c|}{\makecell{\textbf{Scenario Info}}} & \makecell{\textbf{Scenario 1: Route Suggestion}} & \makecell{\textbf{Scenario 2: Plant Fertilization Reminder}} \\
 \hline
\multicolumn{2}{c|}{\makecell{\textbf{Scenario}}} & Nancy (AI expert, high AI literacy) is jogging in the morning on a quiet trail. Since it is the cherry-blossom season and Nancy loves cherries, her AR glasses display a map beside her and recommend a detour. Nancy is surprised since this route is different from her regular one, but she is happy to explore it. She is also curious to know the reason this new route was recommended. & Sarah (general end-user, low AI literacy) was chatting with her neighbor about gardening. After she returned home and sat on the sofa, her AR glasses recommended instructions about plant fertilization by showing a care icon on the plant. Sarah is concerned about technology invading her privacy, and wants to know the reason behind the recommendation. \\ \hline
\multirow{3}{*}{\makecell[c]{\\\textbf{Platform-}\\\textbf{Agnostic}\\\textbf{Key Factors}}}  & \textbf{System Goal} & \textbf{User Intent Discovery (new route)} & \textbf{Trust Building (clarification)} \\
 & \textbf{User Goal} & \textbf{Resolve Surprise} & \textbf{Privacy Awareness} \\
 & \textbf{User Profile} & \makecell[l]{\textbf{User Preference}: Cherry blossom tree lover;\\\textbf{History}: Regular jogging in the morning;\\\textbf{AI Literacy}: Expert, high} & \makecell[l]{\textbf{User Preference}: Plant enthusiast;\\\textbf{History}: Did not care of the plant for a while;\\\textbf{AI Literacy}: General end-user, low} \\ \hdashline
\multirow{2}{*}{\makecell[c]{\\\textbf{AR-Specific}\\\textbf{Key Factors}}} & \textbf{Contextual Info} & \makecell[l]{\textbf{Location}: Outdoor; \textbf{Time}: Morning;\\\textbf{Environment}: Trails, streets} & \makecell[l]{\textbf{Location}: Home; \textbf{Time}: Afternoon;\\\textbf{Environment}: Living room furniture, the plant} \\
 & \textbf{User State} & \makecell[l]{\textbf{Activity}: Jogging;\\\textbf{Cognitive Load}: Low} & \makecell[l]{\textbf{Activity}: Sitting on the sofa;\\\textbf{Cognitive Load}: Low} \\ \hline
\multirow{7}{*}{\makecell[c]{\\\\\\\textbf{Explanation}\\ \textbf{Content Type}\\\textbf{Examples}}} & \textbf{Input/Output} & This route is recommended based on seasons, your routine and preferences. & The system checks the plant’s current status by visually scanning the plant. \\
 & \textbf{Why/Why-Not} & The route has cherry blossom trees that you can enjoy. The length of the route is appropriate and fits your morning schedule. & The plant has abnormal spots on the leaves, which indicates fungi or bacteria infection. \\
 & \textbf{How} & This algorithm finds and ranks possible routes based your location and other people who share similar preferences to you. & The system checks the plant's visual appearance, then searches online to find ways to cure it.\\
 & \textbf{Certainty} & Match rate between this route’s condition and your preference: 93\% & The chance of the plant having disease is high (94\%). \\
 & \textbf{Example} & These photos captured memories about jogging during cherry blossom season. & These are some images of other plants with similar symptoms. \\
 & \textbf{What-If} & The recommended route will be a shorter one if you jog in the evening. & N/A \\
 & \textbf{How-To} & Disable the “season option” to return to the old route. & N/A \\
\hline \hline
\end{tabular}
}
\caption{Details of The Two Application Scenarios in Sec.~\ref{sec:applications}. ``N/A'' indicates that this particular explanation type is not applicable for this case. The same below.}
\label{tab:application_scenario_1}
\end{table}
\renewcommand{\arraystretch}{1.0}