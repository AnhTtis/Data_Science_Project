\vspace{-0.1cm}
\renewcommand{\arraystretch}{1.6}
\begin{table}[hbtp]
\centering
\resizebox{0.86\textwidth}{!}{
\begin{tabular}{c|c|m{7cm}|m{7cm}}
\hline \hline
\multicolumn{2}{c|}{\makecell{\text{ }}} & \includegraphics[width=7cm]{figures/scenario_moviedinner.png} & \includegraphics[width=7cm]{figures/scenario_driving.png}\\
\multicolumn{2}{c|}{\makecell{\textbf{Scenario Info}}} & \makecell{\textbf{Scenario 3: Food Rec for A Movie Night}} & \makecell{\textbf{Scenario 4: Podcast Rec while Driving}} \\
 \hline
\multicolumn{2}{c|}{\makecell{\textbf{Scenario}}} & Emma (general end-user, low AI literacy) has a few friends over for a small party. They decide to watch a Bollywood movie and now they are about to order food. The AR glasses recommends ordering from an Indian restaurant. Mary never heard of this restaurant before, but she loves this idea. She is also curious about the reason of this recommendation. & Jeff (general end-user, low AI literacy) is about to drive to work. The AR glasses recommends a new podcast ``TEDx Shorts'' that Jeff is unfamiliar with. However, the topic is interesting and Jeff wants to give it a try. Meanwhile, Jeff is curious to know the reason for this recommendation. \\ \hline
\multirow{3}{*}{\makecell[c]{\\\textbf{Platform-}\\\textbf{Agnostic}\\\textbf{Key Factors}}}  & \textbf{System Goal} & \textbf{User Intent Assistance} (find good food) & \textbf{User Intent Discovery} (new podcast) \\
 & \textbf{User Goal} & \textbf{Reliability, Informativeness} & \textbf{Informativeness} \\
 & \textbf{User Profile} & \makecell[l]{\textbf{User Preference}: Everyone's food preferences;\\\textbf{History}: Just decided to watch a Bollywood movie;\\\textbf{AI Literacy}: General end-user, low} & \makecell[l]{\textbf{User Preference}: Topic interests;\\\textbf{History}: Morning driving routine;\\\textbf{AI Literacy}: General end-user, low} \\ \hdashline
\multirow{2}{*}{\makecell[c]{\\\textbf{AR-Specific}\\\textbf{Key Factors}}} & \textbf{Contextual Info} & \makecell[l]{\textbf{Location}: Indoor; \textbf{Time}: Evening;\\\textbf{Environment}: Home with a group of friends} & \makecell[l]{\textbf{Location}: Outdoor; \textbf{Time}: Morning;\\\textbf{Environment}: Street conditions}\\
 & \textbf{User State} & \makecell[l]{\textbf{Activity}: Hanging out with friends;\\\textbf{Cognitive Load}: Low to Medium} & \makecell[l]{\textbf{Activity}: About to Start Driving to Work;\\\textbf{Cognitive Load}: High} \\ \hline
\multirow{7}{*}{\makecell[c]{\\\\\\\textbf{Explanation}\\ \textbf{Content Type}\\\textbf{Examples}}} & \textbf{Input/Output} & This restaurant is recommended based on everyone's food preference and movie genre. & The recommendation takes your playlist history and driving routine into account. \\
 & \textbf{Why/Why-Not} & The food fits everyone's food preferences and matches the genre of the movie you are watching. & This podcast's topic is in line with your interest, and its length fits your expected driving time. \\
 & \textbf{How} & The algorithm filters the restaurants by food preferences and then finds the best match between the food and the related activity. & The algorithm detects that it’s morning and you are driving to work, then recommends the new podcast whose topic may be of interest to you. \\
 & \textbf{Certainty} & Match score between the restaurant and the food preference and the movie: 90\% & The podcast was liked by 85\% of people with similar interests as you. \\
 & \textbf{Example} & Last time, everyone enjoyed Chinese food while watching a Chinese movie.  & ``The Daily'' and ``Fresh Air'' are other appropriate  examples when you drove to work \\
 & \textbf{What-If} & If movie genre is disabled, other cuisines would be recommended. & If the commute is longer, there are other episodes that may be of interest to you. \\
 & \textbf{How-To} & N/A & To listen to previous podcasts, you can set history as the main recommendation factor.\\
\hline
\multirow{2}{*}{\makecell[c]{\\\\\textbf{XAI Designs}\\\textbf{in AR}: \colorwhen{\textbf{\textit{When}}}}} & \colorwhen{Availability} (\gone) & Always available & Always available\\
 & \colorwhen{Delivery} (\gtwo) & Auto-trigger as both conditions is met (enough capacity and the user is not familiar with the recommendation). & Manual-trigger (high cognitive load during driving). \\ \hdashline
\multirow{3}{*}{\makecell[c]{\\\\\textbf{XAI Designs}\\\textbf{in AR}: \colorwhat{\textbf{\textit{What}}}}} & \colorwhat{Content} (\gfour) & Input/Output \& Why/Why-Not & Input/Output \& Why/Why-Not\\
 & \colorwhat{Detail - Concise} (\gfive) & The Why part of the explanation examples. & The Why part of the explanation examples.\\
 & \colorwhat{Detail - Detailed} (\gsix) & A list of the two explanation content types, plus images of the movie and food to support the Why part. & A list of the two explanation types.\\ \hdashline
\multirow{3}{*}{\makecell[c]{\\\\\textbf{XAI Designs}\\\textbf{in AR}: \colorhow{\textbf{\textit{How}}}}} & \colorhow{Modality} (\gseven) & Visual modality. & \makecell[l]{Audio modality}\\
 & \colorhow{Paradigm - Format} (\geight) & Textual format, plus graphical format in the detailed explanations. & N/A\\
 & \colorhow{Paradigm - Pattern} (\gnine) & Explicit pattern, presenting texts in the same window as the recommendations. & N/A\\
\hline \hline
\end{tabular}
}
\caption{Additional Application Examples of XAIR on Two Recommendation Scenarios.}
\label{tab:more_scenario_1}
\end{table}
\renewcommand{\arraystretch}{1.0}


\renewcommand{\arraystretch}{1.6}
\begin{table}[hbtp]
\centering
\resizebox{0.86\textwidth}{!}{
\begin{tabular}{c|c|m{7cm}|m{7cm}}
\hline \hline
\multicolumn{2}{c|}{\makecell{\text{ }}} & \includegraphics[width=7cm]{figures/scenario_cooking.png} & \includegraphics[width=7cm]{figures/scenario_donotdisturb.png}\\
\multicolumn{2}{c|}{\makecell{\textbf{Scenario Info}}} & \makecell{\textbf{Scenario 5: Cooking Instructions}} & \makecell{\textbf{Scenario 6: Automatic Do-Not-Disturb Mode}} \\
 \hline
\multicolumn{2}{c|}{\makecell{\textbf{Scenario}}} & Lisa (general end-user, low AI literacy) has recently been learning how to cook. She wants to try out a new recipe for today's lunch. She picks ``Poached Egg on Avacado Toast'' and starts to follow the instructions. After she takes the eggs out of the fridge, the AR glasses prompts to boil the egg for 1 min. Lisa is curious about the time recommendation and wants to understand what the prompt is based on. & Jeff (general end-user, low AI literacy) is about to drive to work. The AR glasses recommends a new podcast ``TEDx Shorts'' that Jeff is unfamiliar with. However, the topic is interesting and Jeff wants to give it a try. Meanwhile, Jeff is curious to know the reason for this recommendation. \\ \hline
\multirow{3}{*}{\makecell[c]{\\\textbf{Platform-}\\\textbf{Agnostic}\\\textbf{Key Factors}}}  & \textbf{System Goal} & \textbf{User Intent Assistance} (learn the recipe) & \textbf{User Intent Discovery} (new podcast) \\
 & \textbf{User Goal} & \textbf{Reliability} & \textbf{Informativeness} \\
 & \textbf{User Profile} & \makecell[l]{\textbf{User Preference}: The purpose of learning how to cook;\\\textbf{History}: Following this instruction for the first time;\\\textbf{AI Literacy}: General end-user, low} & \makecell[l]{\textbf{User Preference}: Topic interests;\\\textbf{History}: Morning driving routine;\\\textbf{AI Literacy}: General end-user, low} \\ \hdashline
\multirow{2}{*}{\makecell[c]{\\\textbf{AR-Specific}\\\textbf{Key Factors}}} & \textbf{Contextual Info} & \makecell[l]{\textbf{Location}: Kitchen; \textbf{Time}: Noon;\\\textbf{Environment}: Cookwares and ingredients } & \makecell[l]{\textbf{Location}: Outdoor; \textbf{Time}: Morning;\\\textbf{Environment}: Street conditions}\\
 & \textbf{User State} & \makecell[l]{\textbf{Activity}: Cooking;\\\textbf{Cognitive Load}: High} & \makecell[l]{\textbf{Activity}: About to Start Driving to Work;\\\textbf{Cognitive Load}: High} \\ \hline
\multirow{7}{*}{\makecell[c]{\\\\\\\textbf{Explanation}\\ \textbf{Content Type}\\\textbf{Examples}}} & \textbf{Input/Output} & The guidance is based on the instruction and your current stage. & Last week you turned on smart do-not-disturb mode. The mode is based on your location, time, and your ongoing activity. \\
 & \textbf{Why/Why-Not} & Boiling eggs for one minute will result in soft-boiled eggs with slightly firm whites and a runny egg yolk. This is how people prefer soft-boiled eggs with toast. & This setting automatically blocks notifications when you are at the office during the working hour and working on the laptop. \\
 & \textbf{How} & The algorithm detects your activity and recognizes which stage you are in, then it provides the guidance for the next step. & The system detects your current context and activity, and checks whether they meet your authored settings. If so, the Do-Not-Disturb mode will be turned on. \\
 & \textbf{Certainty} & The recognition of activity has a high certainty (88\%). & The recognition of the time, location and current activity has a high certainty of 92\%. \\
 & \textbf{Example} & N/A  & N/A \\
 & \textbf{What-If} & Other possible ways of cooking eggs, such as scrambled eggs, if you want to explore other recipe instructions. & When you are not in the office, or it is out of working hours, or you are not working in front of the laptop, the setting will not be turned on. \\
 & \textbf{How-To} & N/A & You can update any of the three conditions to change the moment the setting it’s activated.\\
\hline
\multirow{2}{*}{\makecell[c]{\\\\\textbf{XAI Designs}\\\textbf{in AR}: \colorwhen{\textbf{\textit{When}}}}} & \colorwhen{Availability} (\gone) & Always available & Always available\\
 & \colorwhen{Delivery} (\gtwo) & Manual-trigger (high cognitive load during cooking). & Manual-trigger (limited capacity in office). \\ \hdashline
\multirow{3}{*}{\makecell[c]{\\\\\textbf{XAI Designs}\\\textbf{in AR}: \colorwhat{\textbf{\textit{What}}}}} & \colorwhat{Content} (\gfour) & Input/Output \& Why/Why-Not & Input/Output, Why/Why Not, How to, Confidence, How\\
 & \colorwhat{Detail - Concise} (\gfive) & The Why part of the explanation examples. & A summary of Input, Why, and How-To as the user gets confused and wants to change the output.\\
 & \colorwhat{Detail - Detailed} (\gsix) & A list of the two explanation types, plus images of the soft-boiled eggs to support the Why part. & A list of the five types.\\ \hdashline
\multirow{3}{*}{\makecell[c]{\\\\\textbf{XAI Designs}\\\textbf{in AR}: \colorhow{\textbf{\textit{How}}}}} & \colorhow{Modality} (\gseven) & Visual modality. & \makecell[l]{Visual modality}\\
 & \colorhow{Paradigm - Format} (\geight) & Textual format, plus graphical format in the detailed explanations. & Textual format\\
 & \colorhow{Paradigm - Pattern} (\gnine) & Explicit pattern, presenting texts in the window besides the timer. & Explicit pattern, presenting texts in front of the user.\\
\hline \hline
\end{tabular}
}
\caption{Additional Application Examples of XAIR on Intelligent Instructions and Automation.}
\label{tab:more_scenario_2}
\end{table}
\renewcommand{\arraystretch}{1.0}