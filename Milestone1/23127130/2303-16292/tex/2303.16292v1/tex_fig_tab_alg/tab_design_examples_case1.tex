\renewcommand{\arraystretch}{1.6}
\begin{table*}[t]
\centering
\resizebox{1\textwidth}{!}{
\begin{tabular}{c|c|m{7.3cm}|m{7.3cm}}
\hline \hline
\multicolumn{2}{c|}{\makecell{\text{ }}} & \includegraphics[width=7.3cm]{figures/design_case1_example1.png} & \includegraphics[width=7.3cm]{figures/design_case1_example2.png}\\
\multicolumn{2}{c|}{\makecell{\textbf{Designer}}} & \makecell{\textbf{P2, Product Designer}} & \makecell{\textbf{P6, XAI Researcher}} \\
 \hline
\multirow{3}{*}{\makecell[c]{\\\textbf{Platform-}\\\textbf{Agnostic}\\\textbf{Key Factors}}}  & \textbf{System Goal} & \multicolumn{2}{c}{\makecell[c]{\textbf{User Intent Assistance} (to find a good recipe)}} \\
 & \textbf{User Goal} & \multicolumn{2}{c}{\makecell[c]{\textbf{Reliability} (to make sure the recipe fits the diet)}} \\
 & \textbf{User Profile} & \multicolumn{2}{c}{\makecell[c]{\textbf{User Preference}: High protein food; \textbf{History}: Know these recommended recipes;\\\textbf{AI Literacy}: General end-user, low}} \\ \hdashline
\multirow{2}{*}{\makecell[c]{\\\textbf{AR-Specific}\\\textbf{Key Factors}}} & \textbf{Contextual Info} & \multicolumn{2}{c}{\makecell[c]{\textbf{Location}: Kitchen; \textbf{Time}: Noon; \textbf{Environment}: Various ingredients in the fridge}} \\
 & \textbf{User State} & \makecell[l]{\textbf{Activity}: Opening the fridge to make lunch;\\\textbf{Cognitive Load}: Low} & \makecell[l]{\textbf{Activity}: Opening the fridge to make lunch,\\possibly holding something\\\textbf{Cognitive Load}: Low} \\ \hline
\multirow{2}{*}{\makecell[c]{\\\\\textbf{XAI Designs}\\\textbf{in AR}: \colorwhen{\textbf{\textit{When}}}}} & \colorwhen{Availability} (\gone) & Always available & Same as P2\\
 & \colorwhen{Delivery} (\gtwo) & Manual-trigger, because the second condition of auto-trigger was not met given the \textit{System Goal}, \textit{User Goal}, and \textit{User Profile}. & Same as P2 \\ \hdashline
\multirow{3}{*}{\makecell[c]{\\\\\textbf{XAI Designs}\\\textbf{in AR}: \colorwhat{\textbf{\textit{What}}}}} & \colorwhat{Content} (\gfour) & Input/Output \& Why/Why-Not based on Fig.~\ref{fig:overview_what}'s table & Same as P2\\
 & \colorwhat{Detail - Concise} (\gfive) & An explanation merging the Why and Input content types, as explaining \textit{``showing ingredients is also important''} & An explanation of the Why part as \textit{``it needs to be prioritized''}\\
 & \colorwhat{Detail - Detailed} (\gsix) & A list of the two explanation types in detail & Same as P2; Cherry flower pictures to support the Why explanation \\ \hdashline
\multirow{3}{*}{\makecell[c]{\\\\\textbf{XAI Designs}\\\textbf{in AR}: \colorhow{\textbf{\textit{How}}}}} & \colorhow{Modality} (\gseven) & Visual modality to ensure consistency with the recommendation interface & \makecell[l]{Visual modality for explanations;\\Audio/visual modality for manual trigger if\\the user is/isn't holding something}\\
 & \colorhow{Paradigm - Format} (\geight) & Textual format & Textual format as the primary format; Graphic format (protein icon) to support explanations\\
 & \colorhow{Paradigm - Pattern} (\gnine) & Explicit pattern, presenting texts in the same window as the recommendations & Same as P2\\
\hline \hline
\end{tabular}
}
\vspace{0.1cm}
\caption{Two Design Examples of Case 1: Reliable Recipe Recommendation. Participants' quotes were presented in italic font. Among key factors, P2 and P6 had different thoughts on \textit{User State}, which leads to different design choices of \colorhow{how - modality}. The comparison indicates both consistency and variance between two designers' examples.}
\label{tab:design_example_case1}
\Description{Two Design Examples of Case 1: Reliable Recipe Recommendation: Michael works in a sales company (general end-user, low digital literacy). He recently started a high-protein diet due to his workout routine. He opens the fridge and wants to make lunch. The AR glasses present a window on the fridge door, and recommend an option that Michael usually has, but Michael wants to make sure that this option fits his recent diet changes.
The examples are from P2 and P6. Among key factors, P2 and P6 had different thoughts on User State, which leads to different design choices of "how - modality". The comparison indicates both consistency and variance between the two designers' examples.
P2's design is as follows:
When. P2 thought that Michael's User Goal was to ensure the recipes went along with his diet (Reliability), and the System Goal was to help Michael find suitable recipes (User Intent Assistant). Moreover, Michael was pretty familiar with these recommendations. Therefore, the second condition of G2 was not met, and P2 agreed that the manual trigger was a better idea for delivery.
What. Using XAIR, he identified the three factors (i.e., the goals of the system and the user, plus Michael had a low AI literacy) and narrowed down the content to be Input/Output, and Why/Why-Not. For detail, besides prioritizing Why, P2 also proposed a short paragraph by merging the Input category as he thought explaining ingredients was also important. As for detailed explanations, he decided to show the list of the two explanation categories in detail.
How. P2's design is to adopt the visual modality, which was in line with the recommendation interface. P2 designed a simple button icon to support the manual trigger. The textual format was appropriate to explain the Why/Why-Not and Input/Output reasons. Since a textual paragraph was not easily compatible with the environment, P2 adopted the explicit pattern that presented texts in the same window as the recommendations.
P6's design is as follows:
When. Same as P2, P6 also examined the conditions and chose the manual-trigger option for delivery.
What. P6 selected the same categories as Input/Output and Why/Why-Not, and adopted the same detailed explanations design. However, for the default concise explanations in the detail dimension, P6 mainly focused on the Why part and proposed to add an icon to indicate the high-protein feature (also about the Why explanation).
How. P6 proposed the same visual modality and explicit pattern. Moreover, he also brought up an interesting case when Michael's hands were busy holding ingredients. In this case, P6 proposed to support an audio trigger as an alternative. As for the format, other than using texts as the primary format, P6 further proposed using a simple icon as a secondary format to support explanations.}
% \vspace{-0.7cm}
\end{table*}
\renewcommand{\arraystretch}{1.0}