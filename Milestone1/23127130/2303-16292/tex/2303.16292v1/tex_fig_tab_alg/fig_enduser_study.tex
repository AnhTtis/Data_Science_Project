\begin{figure}
    \vspace{-0.1cm}
    \centering
    \begin{subfigure}[b]{1\columnwidth}
    \centering
    \includegraphics[width=0.7\columnwidth]{figures/enduser_study_setup.png}
    % \vspace{0.3cm}
    \caption{Evaluation Setup}
    \label{subfig:study_end_user_eval:setup}
    \end{subfigure}
    \begin{subfigure}[b]{1\columnwidth}
    \centering
    \includegraphics[width=1\columnwidth]{figures/enduser_study_xai.png}
    \caption{Evaluation Scores}
    \label{subfig:study_end_user_eval:scores}
    \end{subfigure}
    \vspace{-0.6cm}
    \caption{
    \review{End-user Evaluation of The AR System in Study 4. (a) Study Setup. (b) Evaluation Scores. Users had positive experience in both tasks. Note that tasks were evaluated separately and not meant to compare against each other.}
    }
    \label{fig:study_end_user_eval}
    \Description[End-user Evaluation of The AR System in Study 4.]{(a) The figure shows the study setup. A man was wearing a Hololens, looking at a shelf, with his hand reaching toward the shelf. There are a lot of vegetables and fruits on the shelf.
    (b) The figure shows a bar chart illustrating the result of our evaluation of the AR system derived from the XAIR framework in section 7.2. The Y-axis is the mean ratings from the end users on a 7-point Likert scale. The X-axis is the seven evaluation categories: intelligibility, transparency, trustworthiness, design of when, design of what, and design of how. In both task 1, reliable recipe recommendation, and task 2, recipe recommendation with error, our system obtained positive ratings from end users in terms of all categories. Note that among these categories, intelligibility, transparency, and trustworthiness obtained especially high ratings, with a mean of 6 and above. The figure demonstrates the strong usability of the XAIR framework in providing satisfying XAI experiences in AR, as described in section 7.4.1.
}
    \vspace{-0.4cm}
\end{figure}