\begin{figure*}[!b]
    \centering
    \vspace{-0.2cm}
    \begin{subfigure}[b]{.33\textwidth}
    \centering
    \includegraphics[width=1\columnwidth]{figures/percentage_whether.png}
    \caption{Need Explanations?}
    \label{subfig:survey_results:whether}
    \end{subfigure}
    \hfill
    \begin{subfigure}[b]{.34\textwidth}
    \centering
    \includegraphics[width=1\columnwidth]{figures/percentage_when.png}
    \caption{When to Have Explanations?}
    \label{subfig:survey_results:when}
    \end{subfigure}
    \hfill
    \begin{subfigure}[b]{.32\textwidth}
    \centering
    \includegraphics[width=1\columnwidth]{figures/percentage_what.png}
    \caption{What Explanations Are Preferred?}
    \label{subfig:survey_results:what}
    \end{subfigure}
    \vspace{-0.4cm}
    \caption{Highlight of Survey Results with 506 End-Users about Their Needs and Preferences of XAI in everyday AR scenarios.}
    \label{fig:survey_results}
    \Description[Survey Results.]{(a) A stacked percentage bar plot with six bars showing whether users need XAI with their AI experience. The six X-axis stick labels are: unfamiliar with AI, never heard of AI, use AI occasionally, use AI regularly, user AI frequently, and work in AI. The Y-axis is about the percentage. The bar chart shows that around 37\% of the "unfamilar with AI" replied "No". All other types of users have less than 10\% replied "No".
(b) A stacked percentage bar plot with six bars showing when users need XAI with their AI experience. The six X-axis stick labels are the same. And it shows that over 60\% of the users prefer the explanations to be "contextually dependent". As users get more experience with AI, the percentage of choosing "always or frequently" increases, and the percentage of choosing "rare or never" decreases.
(c) A group bar plot with six groups showing when users need XAI with their AI experience. The six X-axis stick labels are the same. Each group has four categories: Input/Output, Why/Why-Not, How, and Certainty. As users get more experience with AI, the percentage of choosing these four categories increases (from around 15\% on average to 45\% on average).}
    \vspace{-0.3cm}
\end{figure*}