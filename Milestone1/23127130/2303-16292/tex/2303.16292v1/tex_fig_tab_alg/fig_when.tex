\begin{figure*}[!t]
    \centering
    \includegraphics[width=0.83\textwidth]{figures/when-design.png}
    \caption{The "When" Part of XAIR. It contains two major dimensions: (A) Availability and (B) Delivery, highlighted in \textbf{bolded} texts.
    The design choice of dimensions are in \textbf{\textit{italic}} texts (same below for Fig.~\ref{fig:overview_what} and Fig.~\ref{fig:overview_how}).
    For example, \textbf{Delivery} can be either \textbf{\textit{User-trigger}} or \textbf{\textit{Auto-trigger}}.
    Each dimension has factors that should be considered for explanation designs.
    The guidelines G1 and G2 provide advice on these design choices.}
    \label{fig:overview_when}
    \Description[The "When" Part of XAIR.]{The "When" Part of XAIR. It contains two major dimensions: (A) Availability and (B) Delivery. Each dimension has its own design choice. Availability only has "Always Available". Delivery has the choices of "User-trigger" or "Auto-trigger". Each dimension has factors that should be considered for explanation designs. The guidelines G1 and G2 provide advice on these design choices.
G1: Make explanations always accessible to provide user agency.
G2: By default, don't trigger explanations automatically, wait until user's request.
Only trigger explanation automatically when both conditions are met:
(1) Users have enough capacity (e.g., cognitive load, urgency);
(2) Users are surprised/confused, or unfamiliar with the outcome, or the model is uncertain. }
\end{figure*}