\section{Improved Search Based on Locality-Sensitive Hashing}
\label{sec:LSH_search}

Locality-sensitive hashing (LSH) is a method for approximate
nearest-neighbour search where points are inserted into a number of
hash tables, with hashes that are calculated from the coordinates in
such a way that nearby points end up inside the same hash table
buckets with high probability. To search for a point's nearest
neighbour, one only checks points that share at least a given number
of hash table buckets. An equivalent formulation in the language of
particle physics is to consider a number of one-dimensional
histograms, where the observables are chosen such that similar events
end up in the same histogram bins. To find events that are nearby in
phase space, one only checks those events that share a large number of
histogram bins. A first LSH-based search algortihm for cell resampling
was proposed in~\cite{Andersen:2021mvw}. In the following, we discuss
an improved version, where the histogram observables have a closer
relation to the exact distance measure.

The first step in defining the locality-sensitive observables is the
same as in the exact distance calculation: we cluster the outgoing
particles in each event into infrared-safe physics objects and group
them according to their types. As usual, we add
objects with vanishing momentum to ensure that all events have the
same number of objects for each type. For each object type $t$, we then choose a
random axis $a_t$ in three-dimensional Euclidean space. We choose a final
axis $A$ in a Euclidean space whose dimension is equal to the total number
of infrared-safe physics objects in an event.

For a given event, we then calculate the observable as follows. For
each object type $t$, we project the spatial momentum of each object
onto the previously chosen axis $a_t$ and sort the resulting
coordinates. We concatenate all coordinates obtained in this way into
a single vector. Finally, we obtain the observable by projecting this
vector onto the axis $A$.

We find that the LSH-based search based on the present observables
performs significantly better than the original
version~\cite{Andersen:2021mvw}. However, it still suffers from the
same problem. For constant (or at most logarithmically growing)
numbers of histograms and bin sizes we observe that the typical
distance between an event and the approximate nearest neighbour fails
to decrease with a growing sample size. Hence, we mainly focus on the
exact tree-based search presented in
section~\ref{sec:nearest-neighbour_search}.


%%% Local Variables:
%%% mode: latex
%%% TeX-master: "main"
%%% End:
