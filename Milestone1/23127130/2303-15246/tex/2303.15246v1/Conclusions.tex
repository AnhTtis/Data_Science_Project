\section{Conclusions}
\label{sec:conclusions}

We have demonstrated that the fraction of negative event weights in
existing large high-multiplicity samples can be reduced by more than
an order of magnitude, whilst preserving predictions for observables
within statistical uncertainties. Concretely, we have employed the cell
resampling method proposed in~\cite{Andersen:2021mvw} with NLO event
samples for Z boson production with up to three jets
and W boson production with five jets produced with \textsc{Sherpa}
and \textsc{BlackHat}.

For the first time, cell resampling has been applied to samples with
up to several billions of events. This was made possible by
algorithmic improvements leading to a speed-up by several orders of
magnitude. Our updated implementation can be retreived from
\url{https://cres.hepforge.org/}.

The advances in the development of the cell resampling method
presented in this work pave the way for future applications to processes with
high-multiplicities, in particular including parton showered
predictions. It will be necessary to quantify the uncertainty
introduced by the weight smearing. Variations in the maximum cell size
parameter and different prescriptions for weight redistribution within
a cell can serve as handles to assess this uncertainty. Another
promising avenue for further exploration is the analysis of the
information on weight distribution within phase space collected during
cell resampling. Regions with insufficient Monte Carlo statistics
could be identified by their accumulated negative weight, thereby
guiding the event generation. We leave the investigation of these
questions to future work.

\section*{Acknowledgements}

AM thanks Zahari Kassabov for encouragement to reconsider the use of nearest
neighbour search trees. The work of JRA and DM is supported by the STFC under
grant ST/P001246/1.

%%% Local Variables:
%%% mode: latex
%%% TeX-master: "main"
%%% End:
