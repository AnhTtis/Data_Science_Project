\section{Cell Sizes}
\label{sec:cell_sizes}

Larger cell sizes naturally lead to stronger smearing effects. Ideally,
all cells should be small compared to the experimental resolution,
which is limited both by the detector and by statistics.

In the left pane of figure~\ref{fig:cell_stats} we show the
distribution of cell radii obtained for the \mbox{Z + 1 jet} sample
\texttt{Z1}, cf. table~\ref{tab:samples}. We have omitted cells where
aside from the seed no further event is found within the maximum cell
radius of 10\,GeV. The shape of the distribution is similar to the one
found for W + 2 jets~\cite{Andersen:2021mvw}. The median cell radius
is 3.4\,GeV.

The cell diameter imposes an upper limit on the spread in any single
direction. However, especially in a higher-dimensional phase space, the smearing
range in one-dimensional distributions will be typically much smaller,
as pointed out in our earlier work~\cite{Andersen:2021mvw}. To
illustrate this point, we compute the difference $\Delta
p_\perp(\text{jet})$ between the transverse momenta of the softest jet
and the hardest jet among all events within a cell. The distribution
is shown in the right pane of figure~\ref{fig:cell_stats}. We
observe a steep decline with a median of 0.4\,GeV. There is a notable
drop where $\Delta p_\perp(\text{jet})$ reaches the maximum cell
radius of 10\,GeV. While the theoretical upper limit is given by the
maximum cell diameter of 20\,GeV, we find that the largest transverse
momentum spread in the considered sample is approximately 15\,GeV.
\begin{figure}[htb]
  \centering
  \includegraphics{cell_radii}
  \includegraphics{delta_pt}
  \caption{%
    Cell size characteristics for the sample \texttt{Z1}. The left
    pane shows the distribution of cell radii. The right pane displays
    the differences between the transverse momenta of the hardest and
    softest jet within a cell.
  }
  \label{fig:cell_stats}
\end{figure}

%%% Local Variables:
%%% mode: latex
%%% TeX-master: "main"
%%% End:
