%%%% macros.tex starts here %%%%

\usepackage{ifthen}

\newcommand{\warning}[1]{{\it\color{red} #1}}
\newcommand{\toremove}[1]{{\it\color{red} (To remove) #1}}
\newcommand{\note}[1]{{\it\color{blue} #1}}
\newcommand{\nothing}[1]{}
\newcommand{\revision}[1]{{\color{black} #1}}

\definecolor{FanColor}{rgb}{0.8,0,0.8}
\newcommand{\fan}[1]{{\color{FanColor}[Fan: #1]}}

\newcommand{\chongyang}[1]{{\it\color{red} Chongyang: #1}}

\newcommand{\yuxin}[1]{{\it\color{green} Yuxin: #1}}

{

\renewcommand{\warning}[1]{}
\renewcommand{\toremove}[1]{}
\renewcommand{\note}[1]{}
\renewcommand{\nothing}[1]{}
}{}

\hyphenpenalty=1000 

\newcommand{\inputsection}[1]{\input{Sections/#1}}
\newcommand{\inputfigure}[1]{\input{Figs/#1}}
\newcommand{\inputtable}[1]{\input{Tables/#1}}

\newcommand{\etal}{\textit{et al.}}
\newcommand{\figtextnormal}[1]{{#1}}
\newcommand{\figtext}[1]{{\footnotesize #1}}

% \newcommand{\changed}[2]{#2} 


%%%% macros.tex ends here %%%%