\section{Data-Driven Crash-Safety Analysis}
\label{sec:crash_robust}

This section motivates crash-safety in the context of data-driven analysis. This section will remove the restriction that the performance function satisfies $J(0)=0$, but will retain the property that the level sets of $J$ are connected.

\subsection{Data-Driven Overview}
In this section, we will assume that $N_s$ time-state-derivative data  records $\mathcal{D} = \{(t_k, x_k, y_k)\}_{k=1}^{N_s}$ are provided for the true system $\dot{x} = F(t, x)$. The data records in $\dc$ are corrupted by $L_\infty$-bounded noise of intensity $\epsilon$ with
\begin{align}  \label{eq:linf_corrupt}
    \forall k = 1..N_s & & \norm{y_k - F(t_k, x_k)}_\infty \leq \epsilon.
\end{align}

We are given a dictionary of functions $(f_0, \{f_\ell\}_{\ell=1}^L)$ that are Lipschitz in $[0, T] \times X$ (e.g. monomials). We are also given the knowledge that there exists at least one ground-truth choice of parameters $w^* \in \R^L$ with 
\begin{align}
   \textstyle  F(t, x) = f_0(t, x) + \sum_{\ell=1}^L w^*_\ell f_\ell(t, x). \label{eq:dynamics_affine_true}
\end{align}

% The crash-safety problem for the $L_\infty$ data-driven case has the following performance function $J_\infty$ based on \eqref{eq:linf_corrupt} and \eqref{eq:dynamics_affine}:
% \begin{align}
%     J_\infty(w) = \sup
% \end{align}



In the $L_\infty$-bounded polytopic framework, the crash-safety problem \eqref{eq:crash_traj_z} finds an infimal upper bound on the data corruption needed to crash into the unsafe set:
\begin{subequations}
    \label{eq:crash_base}
    \begin{align}
    Z^* = & \inf_{t, \, x_0, \, z, \, w} z\\
    & \forall t' \in [0, T] : \nonumber \\
    & \qquad \dot{x}(t') =  f_0(t', x) + \textstyle \sum_{\ell=1}^L w_\ell f_\ell(t', x(t')) \\
    & \qquad \dot{z}(t')=0  \\
    & x_0 \in X_0, \ x(t \mid x_0, w) \in X_u \\ 
    &  \forall k = 1..N_s:  \\
    & \qquad z \geq \norm{f_0(t_k, x_k) + \textstyle \sum_{\ell=1}^L w_\ell f_\ell(t_k, x_k) - y_k}_\infty \nonumber& &  \label{eq:crash_base_poly}\\
    & z \in Z, \ w \in \R^L, \ t \in [0, T].
    \end{align}
\end{subequations}
If the returned value of \eqref{eq:crash_base} is $Z^* = 0$, then there exists some choice of model parameters $w$ that exactly fit the data $\mathcal{D}$ by \eqref{eq:dynamics_affine_true}. Additionally, this choice $w$ renders at least one trajectory $x(\cdot)$ starting from $X_0$ is unsafe (crashes into $X_0$). Values of $Z^*$ greater than 0 are a certificate of safety in the model structure. A larger value of $Z^*$ indicates that the data must be increasingly corrupted in order to render any trajectory unsafe. Safety is certified if $Z^* > \epsilon$, though we note that the true value of $\epsilon$ may be a-priori unknown.

\subsection{Robust Data-Driven Program}

% Let $\Gamma \in \R^{n T \times L}, \ h \in \R^{n T}$ be matrices 
We will use the input-affine structure of dynamics and polytopic form of  \eqref{eq:crash_base_poly} to form an \ac{LP} that eliminates the uncertainty $w$. This elimination leads to increasingly tractable \ac{SOS} \acp{SDP}.
For each $k=1..N_s$, define the data-record matrices $\Gamma_k, \ h_k$ by
\begin{subequations}
\begin{align}
    \Gamma_k &= \begin{bmatrix}f_1(t_k, x_k), \cdots, f_L(t_k, x_k) \end{bmatrix}  \label{eq:gamma_matrix}\\
    h_k &= f_0(t_k, x_k) - y_k.
\end{align}
\end{subequations}
Letting $\Gamma$ and $h$ be the vertical concatenations of $\{\Gamma_k\}$ and $\{h_k\}$ respectively, we can define the $L_\infty$ performance function and support set as 
\begin{align}
J(w) &= \norm{\Gamma w - h}_\infty = \theta(t, x, w), \quad Z = [0, J_{\max}], \\
\intertext{and the support set for $(w, z)$ from \eqref{eq:crash_base_poly} as}
\label{eq:support_set_crash_data}
    \Omega &= \left\{(w, z) \in \R^L \times Z: \begin{matrix}\Gamma w \leq z\1 - h \\ -\Gamma w \leq z \1 + h\end{matrix}\right\}.
\end{align}

We will eliminate the $w$ variable from \eqref{eq:crash_cont_lie} by introducing new nonnegative multiplier functions $\{\zeta^+, \zeta^-_j\}_{j=1}^{2nT}$. This elimination proceeds using the infinite-dimensional robust counterpart method of \cite{miller2023robustcounterpart}, which requires that \eqref{eq:crash_cont_lie} hold strictly (with a $>0$) constraint.
\begin{thm}
\label{thm:crash_robust_lie}
A strict version of Lie constraint in \eqref{eq:crash_cont_lie} may be robustified  (will have the same feasibility/infeasibility conditions) into
\begin{subequations}
\label{eq:crash_robust_lie}
\begin{align}
& \forall (t, x, z) \in [0, T] \times X \times Z: \nonumber\\
    & \qquad \Lie_{f_0} v - (z\1-h)^T \zeta^+ - (z\1 + h)^T \zeta^- > 0 \label{eq:sub_lie_decomp}\\
    & \forall \ell=1..L: \quad \  (\Gamma^T)_\ell (\zeta^+ - \zeta^-) + f_\ell \cdot \nabla_x v = 0 & &   \\
    & \forall j=1..2nT: \ \zeta^+_j, \zeta^-_j \in C_+([0, T] \times X \times Z).
    \end{align}
    \end{subequations}
\end{thm}
\begin{proof}
% See Theorem \ref{thm:lie_robust} for the robustification. 
% We perform the following assignments to the robust inequality in \eqref{eq:lin_robust}:
% We express the strict version of \eqref{eq:crash_cont_lie} using 
We define the following variables 
\begin{subequations}
\label{eq:crash_robust_assign}
\begin{align}
    b_0 &= \Lie_{f_0} v & b_\ell &= f_\ell \cdot \nabla_x v & \forall \ell=1..L \\
    A&= [-\Gamma; \Gamma] & e &= [z-h; z+h]\\
     K &= \textstyle \prod_{s=1}^{2 n T}\R_{\geq 0}.
\end{align}
\end{subequations}
to express the strict version of \eqref{eq:crash_cont_lie} into the form
\begin{align}
         \textstyle b_0 + \sum_{\ell=1}^L w_\ell b_\ell & > 0 & \forall A w + e \in K. \label{eq:lin_robust_strict}
\end{align}
The parameters of \eqref{eq:lin_robust_strict} are $(t, x, z) \in [0, T] \times X \times Z$.
By Theorems 4.2 and 4.3 of \cite{miller2023robustcounterpart}, sufficient conditions for \eqref{eq:crash_robust_lie} to equal the strict version of \eqref{eq:crash_cont_lie} are that:
\begin{itemize}
   \item[R1] $K$ is a convex pointed cone.
    \item[R2] $[0, T] \times X \times Z$ is compact.
    \item[R3] $A$ is constant in $(t, x, z)$.
    \item[R4] $(e, b_0, \{b_\ell\})$ are continuous in   $(t, x, z)$.
\end{itemize}

R1 holds because $\R_{\geq0}^{2nT}$ is a convex and pointed cone. Compactness of $[0, T] \times X$ holds by A1, and compactness of $Z$ holds by A2. R3 is true because $\Gamma$ is a constant matrix computed from the data in $\dc$
from \eqref{eq:gamma_matrix}. R4 is satisfied because $e$ is continuous (affine) in $z$, and $( b_0, \{b_\ell\})$ are continuous given that $v \in C^1$ \eqref{eq:crash_cont_v} and $(f_0, \{f_\ell\})$ are Lipschitz (A3). The theorem is proven because R1-R4 are all fullfilled.
%  The elements $e, b_0, b_\ell$ are all continuous functions of the parameters, and the constraint matrix $A$ is constant in the parameters.
\end{proof}

\begin{rmk}
Strictness in \eqref{eq:sub_lie_decomp} is required to ensure that $\zeta^\pm$ may be chosen to be continuous while not adding conservatism. A nonstrict inequality for \eqref{eq:sub_lie_decomp} may be developed  using possibly discontinuous multipliers $\zeta^\pm$.
\end{rmk}

\begin{rmk}
This paper discussed $L_\infty$-bounded uncertainty, resulting in polytopic decomposition of the Lie constraint by Theorem \ref{thm:crash_robust_lie}. Theorems 4.2 and 4.4 of \cite{miller2023robustcounterpart} may be applied when $\Omega$ is a more general semidefinite representable set parameterized by $z$, such as an intersection of ellipsoids for $L_2$-bounded noise, or a projection of spectahedra for semidefinite bounded noise. 
\end{rmk}
