\section{Conclusion}

\label{sec:conclusion}

% \urg{Conclude the paper. Summarize the problem and important findings. Add some extensions and future work.}
This paper utilized peak minimizing control in order to perform safety analysis. The returned values from \ac{SOS} programs are lower-bounds on the maximal control effort needed to crash into the unsafe set.  Crash-safety adds a new perspective on the safety of trajectories, covering some of the blind spots of distance estimation and safety margins. Crash-safety may be applied in the context of data-driven systems analysis, by quantifying the minimum tolerable corruption in an \rw{uncertainty} model before a trajectory is at risk of being unsafe.

Future work involves attempting to reduce computational burden of the Crash programs \eqref{eq:crash_sos} by identifying new kinds of structure (in addition to robust decompositions) to hopefully allow for real-time computation. 
% $L_2$-bounds on the data observations in $\mathcal{D}$ could be used to represent stochastic certificates, in which crash-safety mi
Other extensions could include applying these methods to other classes of systems (e.g., discrete-time, hybrid), and creating a stochastic interpretation of crash-safety.
% An extended Arxiv version of this paper is available at \urg{[Arxiv link goes here] what will the arxiv version have that this work does not?}.
