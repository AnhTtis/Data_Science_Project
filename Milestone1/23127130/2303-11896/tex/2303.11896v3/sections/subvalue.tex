\section{Subvalue Map}
\label{sec:crash_subvalue}
Program \eqref{eq:crash_cont} returns the worst-case crash safety over a set of initial conditions $X_0$.
We briefly discuss an extension of the crash-safety technique to assessing the safety of arbitrary initial conditions. In the data-driven framework, this could be interpreted as lower-bounding the minimum data corruption needed for a $\rw{\mathcal{D}}$-consistent system to crash when starting at any initial point.

\subsection{Value Functions}
% Define $Q(x')$ as the optimum value (value function) of \eqref{eq:crash_traj_z} when starting at the initial condition $X_0 = x'$. 

We define the fixed-$z$ value function of \eqref{eq:crash_traj_z} (when starting at $X_0 = x'$) as
\begin{align}
\label{eq:value_v}
    V(x', z) = \begin{cases} z & z \in [0, J_{\max}], \ \exists t\in [0, T], w(\cdot) \in \mathcal{W}:  \\
    & \qquad x(t \mid x_0, w(\cdot)) \in X_u, \ J(w(t')) \leq z\  \forall t'\in [0,t] \\
    \infty & \textrm{otherwise.}\end{cases}
\end{align}

The value function $V(x', z)$ is infinite if the control problem of steering a point from $x'$ to $X_u$ is infeasible within the performance budget $J(w)\leq z$.
The value function of \eqref{eq:crash_traj_z} when restricted to the single initial condition $x'$ is
\begin{align}
\label{eq:value_q}
    Q(x') = \inf_{z \in [0, J_{\max}]} V(x', z).
\end{align}

The value function $Q(x')$ will have an upper bound of $J_{\max}$ if $Q(x')$ is finite, and otherwise will have a value of $\infty$. We make no assumptions of continuity or boundedness of $Q(x')$, beyond A1's assurance that $J_{\max}$ is finite.

\subsection{Subvalue Approximations}

We now use the moment-\ac{SOS} hierarchy to develop subvalue maps to lower-bound $Q(x')$ from \eqref{eq:value_q}.


\begin{prop}
\label{prop:v_subvalue}
Any function $v(t, x, z)$ that satisfies \eqref{eq:crash_cont_p} and \eqref{eq:crash_cont_lie} obeys $v(0, x, z) \leq V(x', z)$ from \eqref{eq:value_v} at all $(x, z) \in X \times Z$.
\end{prop}
\begin{proof}
Equations \eqref{eq:crash_cont_p} and \eqref{eq:crash_cont_lie} are inequality constrained versions of the \ac{HJB} equality constraints for an optimal value function $v^*$ \cite{liberzon2011calculus}:
\begin{subequations}
\begin{align}
    &v^*(t, x', z) = z       & & \forall (t, x', z) \in [0, T] \times X_u \times Z \\
    &\min_{w \mid (w, z) \in \Omega } \Lie_f v^*(t, x', z, w) = 0 & & \forall (t, x', z) \in [0, T] \times X \times Z.
\end{align}
\end{subequations}
Refer to the Section 4 of \cite{henrion2008nonlinear} and the proof of Proposition 1 of \cite{jones2021polynomial} for the establishment of subvalue relations.
\end{proof}


Let $\varphi \in \Mp{X}$ be a probability distribution with easily computable moments (e.g., uniform distribution over $X$ when $X$ is a ball or a box),  and $Q_{\max} \geq J_{\max}$ be a finite control cap. 
\begin{thm}
The following program provides a subvalue function $q(x) \leq Q(x)$:
\begin{subequations}
\label{eq:q_joint}
\begin{align}
    J^* &= \sup \int_X q(x) d\varphi(x) \label{eq:q_joint_obj}\\
    & q(x) \leq v(0, x, z) & & \forall (x, z) \in X \times [0, Z_{\max}] \label{eq:q_joint_qv} \\
    & q(x) \leq Q_{\max} & & \forall x \in \supp{\varphi} \label{eq:q_joint_cap} \\
    & z \geq v(t,x, z) & & \forall (t, x, z) \in [0, T] \times X_u \times Z  \label{eq:q_joint_z}\\
    & \Lie_{f} v(t, x, z, w) \geq 0 & & \forall (t, x, z, w) \in [0, T] \times X \times \Omega \label{eq:subvalue_lie_v} \\
    & v \in C^1([0, T] \times X \times Z) \label{eq:q_joint_v} \\
    & q \in C(X).
\end{align}
\end{subequations}
\end{thm}
\begin{proof}
Proposition \ref{prop:v_subvalue} proves that $v(0, x, z) \leq V(x, z)$ from \eqref{eq:value_v}. Constraint  \eqref{eq:q_joint_qv} imposes that $q(x) \leq v(0, x, z) \leq V(x, z)$ for all $x \in X$, which implies that $q(x) \leq \inf_z v(0, x, z)$ for all $x \in X$. From the definition of $Q(x')$ in \eqref{eq:value_q} with $Q(x') \leq \inf_z V(x', z)$, it therefore holds that $q(x) \leq Q(x)$ for all $x \in X$.
\end{proof}

\begin{cor}
The objective $J^*$ from \eqref{eq:q_joint} is finite and is bounded above by $J^* \leq Q_{\max}$.
\end{cor}
\begin{proof}
Constraint \eqref{eq:q_joint_cap} requires that $q(x)$ is upper-bounded by $Q_{\max}$. The objective \eqref{eq:q_joint_obj} is therefore upper-bounded by 
\begin{equation}
    \int_X q(x) d \varphi(x) \leq \int_X Q_{\max} d \varphi(x) \leq  Q_{\max} \int_{X} d \varphi(x) = Q_{\max},
\end{equation} given that $\varphi$ is a probability distribution.
\end{proof}

\begin{rmk}
\label{rmk:subvalue_exceed}
Let $v$ be a subvalue solution to \eqref{eq:q_joint_z}-\eqref{eq:q_joint_v}. Any point $x' \in X$ such that $\inf_{z \in Z} v(0, x', z) > J_{\max}$ implies that $Q(x') = \infty$.
\end{rmk}


\begin{rmk}
Without the $Q_{\max}$ cap on the value of $q$ in constraint \eqref{eq:q_joint_cap}, the optimal value of \eqref{eq:q_joint} could be $J^* = \infty$ if $\exists x' \in X: Q(x') = \infty$.
\end{rmk}

\begin{rmk}
    The Lie constraint in \eqref{eq:subvalue_lie_v} may be robustified through the methods in Section \ref{sec:crash_robust} when $f$ is input-affine and $W$ is a semidefinite representable set (e.g., a polytope from the $L_\infty$ data-driven case).
\end{rmk}

% \begin{rmk}
% The objective value $J^*$ from \eqref{eq:q_joint} is upper-bounded by $J_{\max} \text{vol}(X)$.
% \end{rmk}
% In practice, we will choose $\varphi$ to be the uniform probability measure over $X$.
Define $q_d\in \R[x]_{\leq 2d}, v_d \in \R[t, x, z]_{\leq 2d}$ as the polynomials obtained by solving the degree-$d$ \ac{SOS} tightening of \eqref{eq:q_joint}. 
Let $I_u(x)$ be the indicator function
\begin{equation}
    I_u(x) \rw{=} \begin{cases}
    0 & x \in X_u \\
    -\infty & x \not \in X_u
    \end{cases}.
\end{equation}

For a sequence of orders $d' = 1..d$, a parametric function $q_{1:d}$ may be defined as
\begin{align}
\label{eq:parametric_subvalue}
    q_{1:d}(x) = \max(I_u(x), \max_{d' \in 1..d} q_{d'}(x)).
\end{align}

\begin{defn}[\cite{lasserre2010joint}] A sequence of continuous functions $\{q_k(x)\}$ converges \textbf{almost uniformly} to $Q(x)$ with respect to a measure $\varphi \in \Mp{X}$ if $\epsilon > 0: \exists  A \subseteq X$, such that $q_k \rightarrow Q$ uniformly on $X \setminus A$  and $\varphi(A) < \epsilon$.
% \begin{align}
%     \lim_{k \rightarrow \infty} \int_{X} \abs{Q(x) - q_k(x)}d\varphi(x) = 0. \label{eq:almost_uniform}
% \end{align}
\end{defn}

\begin{thm}
The function $q_{1:d}(x)$ will converge almost uniformly to \\ $\min(Q_{\max}, Q(x))$ on the state-space $\supp{\varphi} \in X$ as $d\rightarrow \infty$.
\end{thm}
\begin{proof}
Let $\tilde{v} \in \R[t, x, z]$ be a polynomial subvalue function that obeys \eqref{eq:q_joint_z}-\eqref{eq:q_joint_v}. 
Corollary 2.5 of \cite{lasserre2010joint} proves that the parameterized program $q_{1:d}$ will converge $\varphi$-almost uniformly to $\min(Q_{\max}, \min_{z} \tilde{v}(0, x, z))$, resulting in 
\begin{align}
    \lim_{k \rightarrow \infty} \int_{X} \abs{Q(x) - q_k(x)}d\varphi(x) = 0. \label{eq:almost_uniform}
\end{align} 
Increasing the degree of sublevel polynomials $\tilde{v}$ allows for the choice of admissible $\tilde{v}$ such that $\tilde{v}(0, x, z)$ converges in an $L^1$-sense to $V(x, z)$ whenever $V(x, z) \leq Q_{\max}$ \cite[Propositions 5 and 6]{jones2021polynomial}, thus proving the theorem.
\end{proof}

% \begin{rmk}
% The subvalue approximation $q_{1:d}$ in \eqref{eq:parametric_subvalue} is vulnerable to a Gibbs phenomenon


% \rw{Gibbs}  which is common among all polynomial optimization methods \cite{gottlieb1997gibbs, tacchi2022stokes}. 
% % Choosing $Q_{\max} = J_{\max}$ may lead to $q_{1:d}$
% Remark \ref{rmk:subvalue_exceed} is vital in establishing infeasibility of reaching $X_u$, but choosing $Q_{\max} = J_{\max}$ may lead to Gibbs phenomena that distort the infeasibility $Q(x') = \infty$ into $q(x') \leq J_{\max}$. 
% Picking $Q_{\max} > J_{\max}$ (such as $Q_{\max} = 4 J_{\max}$) allows for slack in the range of $[Q_{\max} - J_{\max}, Q_{\max}]$, which hopefully could contain the Gibbs phenomena when establishing infeasibility (safety up to $J_{\max}$).
% % Allowing for slack to develop with $Q_{\max} > J_{\max}$ (such as $Q_{\max} > J_{\max}$) 
% \end{rmk}

% \begin{rmk}
%     
% \end{rmk}