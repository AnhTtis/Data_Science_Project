\section{Examples}
\label{sec:crash_examples}

% \urg{TODO: report timings of experiments.}

This section demonstrates the utility of the crash-safety framework. Robust decompositions of the Lie constraint are applied in all examples. MATLAB R2021a code to generate examples is available at \url{https://github.com/Jarmill/crash-safety}. All \ac{SDP} are generated using YALMIP \cite{lofberg2004yalmip} and solved using Mosek \cite{mosek92}. Finite-degree crash-bounds from \eqref{eq:crash_sos_robust} are compared against \ac{OCP} bounds found using the solver CasADi \cite{andersson2019casadi}. 

The examples in \ref{sec:crash_subvalue_demo} perform crash safety with respect to applied inputs $w$ when the ground truth system is known. The examples in 
\ref{sec:crash_subvalue_data} perform data-driven crash-safety analysis.


\subsection{Single-Input Subvalue Comparison }
\label{sec:crash_subvalue_demo}
This example demonstrates the computation of crash-bounds and the creation of crash-subvalue functionals for system     \eqref{eq:flow_w1} with $J_{\max}=1$ and $Q_{\max} = 4$. This subvalue is constructed by solving \ac{SOS} tightenings of \eqref{eq:q_joint} in the space $X = [-2, 2]^2$ and in the time horizon $t \in [0, 5]$ 

\subsubsection{Half-Circle}
The first part of this example involves the half-circle  respect to the unsafe set $X_u = \{x \mid (x_1+0.25)^2 + (x_2+0.7)^2 \leq 0.5^2, \ (0.95+x_1+x_2)/\sqrt{2 }\leq 0\}.$
Figure \ref{fig:crash_subvalue} draws the unsafe set $X_u$ in red. The color shading (colorbar) plots $q_{1:5}(x)$ clamped to the range $[0, J_{\max}]=[0, 1]$. The integral objective values of \ac{SOS} tightening \eqref{eq:q_joint} at degrees $1..5$ are $J^*_{1:5} = [1.934\times 10^{-7}, 4.864\times 10^{-7}, 3.0794, 5.992, 8.260]$.

% 
\begin{figure}[!h]
    \centering
    \includegraphics[width=0.7\linewidth]{fig/flow_crash_subvalue_4cap_5.png}
    \caption{Subvalue function for Flow system \eqref{eq:flow_w1} between degrees $1..5$.}
    \label{fig:crash_subvalue}
\end{figure}
The black dot in Figure \ref{fig:crash_subvalue} is the specific initial point $X_0 = [1; 0]$. Table \ref{tab:crash_subvalue} lists crash-bounds on \eqref{eq:flow_w1} starting at $X_0$. The subvalue 
bound \eqref{eq:q_joint} is lower than the corresponding degree bounds at the $X_0$-specific program \eqref{eq:crash_cont}.

\begin{table}[h]
\centering
\caption{\label{tab:crash_subvalue} Crash-bounds at $X_0 = [1; 0]$ under \ac{SOS} tightenings}
\begin{tabular}{llllll}
order    & 1                     & 2      & 3      & 4   & 5  \\
subvalue \eqref{eq:q_joint}& $1.089\times10^{-9}$ & $1.607\times10^{-9}$  &  0.1473 & 0.3392 & 0.4053 \\
specific \eqref{eq:crash_cont} & $1.117\times10^{-7}$                & 0.1843 & 0.4369 & 0.5092 & 0.5118
\end{tabular}
\end{table}

We now consider worst-case crash-bounds for the half-circle set with respect to the perturbed flow system \eqref{eq:flow_w1} and the circular initial set $X_0 = \{x \mid 0.4^2 \geq (x_1-1)^2+x_2\}$. Crash-bounds as computed by \eqref{eq:crash_sos_robust} (\ac{SOS} tightenings to \eqref{eq:crash_cont}) in degrees $1..5$ are $[8.101\times 10^{-8}, 6.590\times 10^{-2}, 0.4054, 0.4631, 0.4638].$ The degree-5 lower-bound of $0.4638$ should be compared against the numerical bound of $0.4639$ produced by CasADi. The numerically solved trajectory (blue curve) is plotted in Figure \ref{fig:crash_circ_circ}, along with the unsafe set $X_u$ (red half-circle) and the initial set $X_0$ (black circle). The initial point of the controlled trajectory (blue dot) is $x_0 \approx [1.3424; 0.2069]$.

\begin{figure}[h]
    \centering
    \includegraphics[width=0.7\linewidth]{fig/crash_flow_casadi_circ_circ.png}
    \caption{Numerical optimal control yields worst-case $Q^* \approx 0.4639$ for the half-circle $X_u$}
    \label{fig:crash_circ_circ}
\end{figure}

\subsubsection{Moon}
The second part of this example has a nonconvex moon-shaped unsafe set 
\begin{equation}
X_u = \{x \mid 0.8^2 - (x_1 - 0.4)^2 - (x_2 + 0.4)^2 \geq 0, \ (x_1 - 0.6596)^2 + (x_2 - 0.3989)^2-1.16^2 \geq 0\}. \label{eq:crash_moon}    
\end{equation}
Figure \ref{fig:casadi_moon} displays a controlled trajectory (blue curve)  starting from $X_0 = [0; 0]$ (black circle) and terminating in the $X_u$ (red moon), as computed by CasADi.

\begin{figure}[h]
    \centering
    \includegraphics[width=0.6\linewidth]{fig/crash_flow_casadi_moon_single.png}
    \caption{Numerical optimal control yields $Q^* \approx 0.3232$ for the moon $X_u$}
    \label{fig:casadi_moon}
\end{figure}

Table \ref{tab:crash_subvalue_moon} lists subvalue \eqref{eq:q_joint} and specific \eqref{eq:crash_cont} crash-bounds for $X_0 = [0; 0]$ between degrees $1..5$. The objectives of the \ac{SOS} tightenings to \eqref{eq:q_joint} are 
\[J^*_{1..5} = [1.973\times10^{-7}, 1.323\times10^{-7}, 1.027, 3.188, 4.502].\]

\begin{table}[h]
\centering
\caption{\label{tab:crash_subvalue_moon} Crash-bounds at $X_0 = [0; 0]$ for the moon \eqref{eq:crash_moon} under \ac{SOS} tightenings}
% \begin{tabular}{llllll}
% order    & 1                     & 2      & 3      & 4   & 5  \\
% subvalue \eqref{eq:q_joint}& $8.770\times10^{-9}$  &$4.652\times10^{-10}$  &$-7.861\times10^{-2}$  &$-5.692\times10^{-3}$  &$7.721\times10^{-2}$ \\
% specific \eqref{eq:crash_cont} & $2.723\times10^{-8}$                & 0.1010 & 0.2912 & 0.3216 & 0.3224
% \end{tabular}
\begin{tabular}{llllll}
order                         & 2      & 3      & 4   & 5  \\
subvalue \eqref{eq:q_joint}   &$4.652\times10^{-10}$  &$-7.861\times10^{-2}$  &$-5.692\times10^{-3}$  &$7.721\times10^{-2}$ \\
specific \eqref{eq:crash_cont}     & 0.1010 & 0.2912 & 0.3216 & 0.3224
\end{tabular}
\end{table}
The data from Table \ref{tab:crash_subvalue_moon} at order 1 is subvalue: $8.770\times10^{-9}$, specific: $2.723\times10^{-8}$ (suppressed for layout/formatting purposes).

Figure \ref{fig:subvalue_moon} plots the subvalue function $q_{1:5}(x)$ from \eqref{eq:parametric_subvalue} under a cap of $Q_{\max} = 2$ (and $J_{\max} = 1$). All values of $q_{1:5}$ in Figure \ref{fig:subvalue_moon} are clamped to $[0, J_{\max}]$.

\begin{figure}[h]
    \centering
    \includegraphics[width=0.6\linewidth]{fig/flow_crash_subvalue_moon_2cap_5.png}
    \caption{Subvalue map for the moon \eqref{eq:crash_moon} on the flow system \eqref{eq:flow_w1}}
    \label{fig:subvalue_moon}
\end{figure}

\subsection{Data-Driven Flow System}
\label{sec:crash_subvalue_data}

Data $\dc$ is collected for the Flow system \eqref{eq:flow} from $N_s=40$ samples with perfect knowledge in dynamics $\dot{x}_1 = x_2$ and a ground-truth \rw{uncertainty} bound of $\epsilon = 0.5$ in the coordinate $\dot{x}_2$. The noisy derivative data in $\dc$ and ground-truth derivatives are drawn in the orange and blue arrows respectively in Figure \ref{fig:flow_observations}.
It is assumed that $\dot{x}_2$ is described by a cubic polynomial in $(x_1, x_2)$. The parameterized polytope $\{w \mid A w \leq b + z\}$ ($\Omega$ with fixed $z$ value) has $L=10$ dimensions and $m=2 nT = 80$. The minimum possible corruption while obeying \eqref{eq:dynamics_affine_true} under the cubic \rw{uncertainty} model is $\inf_{(w, z) \in \Omega} z = 0.4617$. 

\begin{figure}[h]
    \centering
    \includegraphics[width=0.6\linewidth]{fig/flow_observations.png}
    \caption{Observed data of the Flow system \eqref{eq:flow}}
    \label{fig:flow_observations}
\end{figure}

% The same parameters $(X, X_0, X_u, T)$ are used as in example \ref{sec:crash_subvalue_demo} to solve the crash-safety problem \eqref{eq:crash_traj}.
The crash-safety problem \eqref{eq:crash_cont} and subvalue problem \eqref{eq:q_joint} were solved with the unsafe set $X_u = \{x \mid (x_1+0.25)^2 + (x_2+0.7)^2 \leq 0.5^2, \ (0.95+x_1+x_2)/\sqrt{2 }\leq 0\}$ between $t=[0, 5]$ time units in the space $X = \{x \in \R^2 : \norm{x}_2^2 \leq 8\}$. The subvalue problem \eqref{eq:q_joint} integrates over the uniform measure of the ball $X$.
% in the space $X = [-2, 2]^2$ and in the time horizon $t \in [0, 5]$, with respect to the unsafe set 

% \urg{TODO: run the experiments}
% Robustified crash-safety analysis in \eqref{eq:crash_sos_robust} results in bounds $\tilde{q}^*_{1:4}=[0.0582, 0.4423, 0.4864 0.5499]$. 
Table \ref{tab:crash_subvalue_data} reports bounds for the crash-corruption $Q(X_0)$ by solving Lie-robustified \ac{SOS} tightenings of \eqref{eq:crash_cont} and \eqref{eq:q_joint} from degrees $1..4$ with $J_{\max} = 1, \ Q_{\max}=4$. The objective function (integrals of $q(x)$) for the subvalue  \eqref{eq:q_joint} are $J^*_{1:4} = [0.2193, 3.8185, $ $7.8326, 18.5945]$. The subvalue-estimated control cost at $X_0$ between degrees $1..4$ is $0.3399$ by Equation \eqref{eq:parametric_subvalue}. The subvalue-estimated bound is valid for all $x \in X$, and is therefore lower than the bound $q^*_4 = 0.5499$ from \eqref{eq:crash_cont} that focuses exclusively on the initial point $X_0$.

\begin{table}[h]
\centering
\caption{\label{tab:crash_subvalue_data} Data-Driven Crash-bounds at $X_0 = [1; 0]$ under \ac{SOS} tightenings}
\begin{tabular}{lllll}
order    & 1                     & 2      & 3      & 4   \\
specific \eqref{eq:crash_cont} & 
$0.0582$ & $0.4423$  & 0.4864& 0.5499   \\
subvalue \eqref{eq:q_joint} & $6.180\times 10^{-3}$ & 0.1829 & 0.3399 & \rw{0.3399}\\
\end{tabular}
\end{table}

Figure \ref{fig:flow_data_driven_subvalue} plots the subvalue function from \eqref{eq:parametric_subvalue} on the data-driven flow system. Subvalues in the plot are clamped to the range $[0, J_{\max}] = [0, 1]$.
\begin{figure}[!h]
    \centering
    \includegraphics[width=0.6\linewidth]{fig/flow_crash_subvalue_data_driven_cap4.png}
    \caption{Subvalue for data-driven \eqref{eq:flow} between degrees $1..4$}
    \label{fig:flow_data_driven_subvalue}
\end{figure}


Safety of trajectories starting in $X_0$ is certified because the crash-bound $\tilde{q}^*_4 = 0.5499$ is greater than the ground-truth \rw{uncertainty}-bound $\epsilon = 0.5$. 


\rw{The} CasADi optimal control suite \cite{andersson2019casadi} \rw{was used} to numerically solve the crash program \eqref{eq:crash_traj}, \rw{and the produced trajectory is visualized in Figure \ref{fig:flow_casadi}}. The numerical crash-bound of $q^{\textrm{CasADI}} = 0.5499$ is approximately equal (up to four decimal places) to the crash-bound $q^*_4=0.5499$.

\begin{figure}[h]
    \centering
    \includegraphics[width=0.7\linewidth]{fig/crash_flow_casadi_state_data_driven.png}
    \caption{Numerically computed crash-bound for data-driven Flow \eqref{eq:flow}}
    \label{fig:flow_casadi}
\end{figure}

The left plot of figure \ref{fig:crash_casadi_control} shows the applied control of the $L=10$ inputs. The right plot demonstrates how the polytopic input constraint is obeyed with respect to the crash bound $q^{\textrm{CasADI}} = 0.5499$ (upper and lower black lines).

 \begin{figure}[!ht]
     \centering
     \begin{subfigure}[b]{0.48\linewidth}
         \centering
         \includegraphics[width=\linewidth]{fig/crash_flow_casadi_control.png}
         % \caption{\label{fig:twist_reach_3}Order 3 relaxation}         
     \end{subfigure}
     \;
     \begin{subfigure}[b]{0.48\linewidth}
         \centering
         \includegraphics[width=\linewidth]{fig/crash_flow_casadi_constraints.png}
         % \caption{\label{fig:twist_reach_4}Order 4 relaxation}
     \end{subfigure}
      \caption{\label{fig:crash_casadi_control}Applied control for the data-driven Flow crash system.}
\end{figure}


 These crash-bounds should be compared against the $L_2$ distance estimates of $c^*_{1:5} = [1.698\times 10^{-5}, \ 0.1936, \ $ $0.2003, \ 0.2009, \ 0.2013]$  from Section 6.3 of \cite{miller2023robustcounterpart}. The distance estimates do not indicate that adding an additional budget of $0.0499$ constraint violation will cause at least one trajectory to enter the unsafe set.

% indicates that the system is safe with respect to corruption of maximal extent $$.

% \urg{I've got bugs in my crash code. Fix them.}

