\section{Crash-Safety Program}
\label{sec:crash_lp}

This section applies peak-minimizing control conversion to the crash-safety task in \eqref{eq:crash_traj}.

\subsection{Motivating Example}

This subsection provides an example demonstrating how \eqref{eq:crash_traj} can be used for safety quantification.
The Flow dynamics from \cite{rantzer2004analysis} are
\begin{align}
    \label{eq:flow}
    \dot{x} = \begin{bmatrix}x_2 \\ -x_1 -x_2 + \frac{1}{3}x^3_1\end{bmatrix}.
\end{align}
This example will perturbed \eqref{eq:flow} by an uncertainty process restricted to $\forall t: \ w(t) \in [-1, 1]$:
\begin{align}
    \label{eq:flow_w1}
    \dot{x} = \begin{bmatrix}x_2 \\ -x_1 -x_2 + \frac{1}{3}x^3_1\end{bmatrix} + w \begin{bmatrix}
        0 \\ 1
    \end{bmatrix}.
\end{align}

Trajectories evolve over a time horizon of $T=5$ in the state set $X = [-0.6, 1.75] \times [-1.5,1.5]$ with a maximum corruption of $J_{\max}=2$. 
System dynamics are illustrated by the blue streamlines in Figure \ref{fig:same_dist}. The red half-circle is the unsafe set $X_u = \{x \mid x_2 \leq -0.5, \ (x_1 - 1)^2 + (x_2+0.5)^2 \leq 0.5^2\}$. Two trajectories of this system are highlighted. The green trajectory starts from the top initial point $X^1_0 = [0; 1],$ and the yellow trajectory starts from the bottom initial point $X^2_0 = [1.2966 -1.5]$. The distance of closest approach to $X_u$ is $0.2498$ for both trajectories (matching up to four decimal places). The $0.2498$-contour of constant distance is displayed by the red curve surrounding $X_u$.


\begin{figure}[h]
    \centering
    \includegraphics[width=0.6\linewidth]{fig/same_dist_2.png}
    \caption{Two trajectories with nearly the same distance but different crash-bounds}
    \label{fig:same_dist}
\end{figure}


The \ac{OCP} solver CasADi \cite{andersson2019casadi} returns approximate bounds for \eqref{eq:crash_traj} of $Q^* \approx 0.3160$ for $X_0^1$ (green) and $Q^* \approx 0.6223$ for $X_0^2$ (yellow).  The points $(X^1_0, X^2_0)$ return nearly identical distances of closest approach, but $X^2_0$ may be judged as safer than $X^1_0$ under the disturbance model in \eqref{eq:flow} due to its higher crash-bound value. Degree-4 \ac{SOS} tightenings of \eqref{eq:crash_traj_z} developed in the sequel return lower bounds of $0.3018$ and $0.5273$ respectively.




\subsection{Assumptions}
We will require the following assumptions:
\begin{enumerate}
    \item[A1] The sets $[0,T], [0, J_{\max}], X, W, X_u, X_0$ are all compact.
    \item[A2] The image $f(t, x, W)$ is convex for each fixed $(t, x)$.
    \item[A3] The dynamics function $f(t, x, w)$ is Lipschitz in the compact domain $[0, T] \times X \times W$.
    \item[A4] If $x(t \mid x_0, w) \, \in \partial X$ for some $t \in [0, T], \ x_0 \in X_0, \ w \in \mathcal{W}$, then $x(t' \mid x_0) \not\in X \ \forall t' \in (t, T]$.
\end{enumerate}

A4 is an assumption of non-return used in \cite{miller2021distance} that is weaker than ensuring $X$ is an invariant set.

% \item[A1] The sets $[0, T], \ X , \ X_u\rw{, \ X_0}$ are \rw{all} compact, \rw{$X_0 \subset X$, and $n$ is finite.}.
%     \item[A2] The function $f(t, x)$ is Lipschitz in each argument \rev{in the compact set $[0, T] \times X$}.
%     \item[A3] The cost $c(x, y)$ is $C^0$ \rev{in $X\times X_u$.}
%     % \item[A4] There exists a \rev{pair} $(x_0, y_0) \in X_0 \times X_u$ such that $c(x_0, y_0) < \infty$.
%     \item[A4] \rev{If $x(t \mid x_0) \, \rw{\in \partial X}$ for some $t \in [0, T], \ x_0 \in X_0$, then $x(t' \mid x_0) \not\in X \ \forall t' \in \rw{(}t, T].$

\subsection{Formulation}

We use the peak-minimizing control conversion of \cite{molina2022equivalent} on program \eqref{eq:crash_traj}.

\begin{thm}
\label{thm:same_crash_z}
The following program has the same optimal value as \eqref{eq:crash_traj}:
\begin{subequations}
    \label{eq:crash_traj_z}
    \begin{align}
    Q^*_z = & \inf_{t, \ x_0, \ z, \ w}  \ z\\
    & \dot{x}(t') =  f(t', x(t'), w(t')) & 
    & \forall t' \in [0, T]\\
    & \dot{z}(t') = 0 & 
    & \forall t' \in [0, T] \\
    & J(w(t')) \leq z & &\forall t' \in [0, T]  \label{eq:crash_traj_z_supp}\\
    & x(t \mid x_0, w(\cdot)) \in X_u \\ 
    & w(\cdot) \in W, \ t \in [0, T] \\
    & x_0 \in X_0,  z \in [0, J_{\max}].
    \end{align}
\end{subequations}
\end{thm}
\begin{proof}
This follows from Proposition \ref{prop:peak_min} under the peak-function $\theta(t, x, w) = J(w).$
The following changes are made with respect to     \eqref{eq:peak_min_z}:
\begin{enumerate}
    \item Free terminal time $t \in [0, T]$ with a terminal state constraint in $X_u$.
    \item Free initial condition $x_0 \in X_0$.
\end{enumerate}
The state $z$  upper-bounds the worst-case control $\sup_{t' \in [0, t]} J(w(t'))$, yielding the peak-control-minimized cost $Q^*_z = Q^*$.
\end{proof}


\subsection{Crash-Safety Linear Program}

Define the following compact support  sets involving the input $w$ and peak-bound $z$:
\begin{align}
\label{eq:crash_support}
    Z &= [0, J_{\max}] & \Omega &= \{(w, z) \in W \times Z: J(w) \leq z\}.
\end{align}

Let $\Lie_f$ be the Lie derivative associated with $f$ for $v(t,x, z) \in C^1$ as 
\begin{align}
\label{eq:lie}
\Lie_f v(t,x,z,w) = (\partial_t + f(t, x, w)\cdot \nabla_x) v(t, x, z).
\end{align}

An auxiliary function $v \in C^1$ may be defined to form an \ac{LP} formulation of the crash-safety \ac{OCP} in \eqref{eq:crash_traj}:
\begin{subequations}
\label{eq:crash_cont}
\begin{align}
    q^* = & \sup_{\gamma \in \R, \ v} \ \gamma \label{eq:crash_cont_obj}& & \\
    & \forall (x, z) \in X_0 \times Z: \nonumber \\ 
    & \qquad v(0, x, z) \geq \gamma \ & & \label{eq:crash_cont_0} \\
    & \forall (t, x, z) \in  [0, T] \times X_u \times Z \nonumber \\
    & \qquad v(t, x, z)  \leq z \ & &  \label{eq:crash_cont_p} \\
    & \forall (t, x, z, w)\in [0, T] \times X \times \Omega \nonumber \\
    & \qquad \Lie_{f} v(t, x, z, w) \geq 0 & & \label{eq:crash_cont_lie}\\
    & v(t,x,z) \in C^1([0, T]\times X \times Z). \label{eq:crash_cont_v}& &
\end{align}
\end{subequations} 

\begin{thm}
\label{thm:same_crash_meas}
    Under assumptions A1-A5, programs \eqref{eq:crash_traj} and \eqref{eq:crash_cont} will have equal objectives $q^* = Q^*$.
\end{thm}
\begin{proof}
    Program \eqref{eq:crash_traj_z} with optimum $Q^*_z$ is a standard-form \ac{OCP} with free terminal time and zero running cost. Under assumptions A1-A5, Theorem 2.1 of \cite{lewis1980relaxation} proves that $Q^*_z = q^*$. Section 6.3 of \cite{lewis1980relaxation} specifically discusses state-dependent controls (e.g. $(w, z) \in \Omega$). Theorem \ref{thm:same_crash_z} provides that $Q^* = Q^*_z$, which together implies that $Q^* = q^*$.
\end{proof}


The dual \ac{LP} of \eqref{eq:crash_cont} may be phrased in terms of an initial measure $\mu_0$, terminal measure $\mu_u$, and relaxed occupation measure $\mu$:
\begin{subequations}
\label{eq:crash_meas}
\begin{align}
m^* = & \  \inf_{\mu_0, \mu_p, \mu} \quad 
\inp{z}{\mu_u}
 \label{eq:crash_meas_obj} \\
    & \inp{v(t, x, z)}{\mu_u} = \inp{v(0, x, z)}{\mu_0} + \inp{\Lie_f v(t, x, z, w)}{\mu} \label{eq:crash_meas_flow}\\
    & \inp{1}{\mu_0} =  1 \label{eq:crash_meas_prob}\\
    & \mu_0 \in \Mp{X \times Z} \label{eq:crash_meas_init}\\
    & \mu_u \in \Mp{[0, T] \times X_u \times Z} \label{eq:crash_meas_term}\\
    & \mu \in \Mp{[0, T] \times X \times  \Omega }. \label{eq:crash_meas_occ}
\end{align}
\end{subequations}
Constraint \eqref{eq:crash_meas_obj} is a Liouville equation involving the Young measure $\mu$ \cite{young1942generalized}.

\begin{lem}
\label{lem:feas_meas}
There exists a feasible solution to \eqref{eq:crash_meas_flow}-\eqref{eq:crash_meas_occ} under A1-A4.
\end{lem}
\begin{proof}
    Let $t^* \in [0, T]$ be a stopping time, $x_0 \in X_0$ be an initial condition, and $w(\cdot) \in \mathcal{W}$ be an input such that $x(t^* \mid x_0, w(\cdot)) \in X_u$. Let $z^*$ be a feasible solution to $\forall t \in [0, t^*]: (z^*, w(t)) \in \Omega$. Then the probability measures can be set to  $\mu_0 = \delta_{x=x_0, z=z^*}$ and $\mu_u = \delta_{t = t^*, x=x(t^* \mid x_0, w(\cdot)), \ z=z^*}$, and $\mu$ can be assigned to the occupation measure of $t \mapsto (t, x(t^* \mid x_0, w(\cdot)), w(t))$ in the times $[0, t^*]$.
\end{proof}

\begin{rmk}
The process of \ref{lem:feas_meas} to generate a feasible measure solution may be used when only A1 and A4 are active, thus certifying that $m^* \leq Q^*$.
\end{rmk}

\begin{thm}
\label{thm:crash_duality}
    Strong duality occurs with $q^*=d^*$ between \eqref{eq:crash_meas} and \eqref{eq:crash_cont} under assumptions A1-A4.
\end{thm}
\begin{proof}
    This holds by standard \ac{OCP} \ac{LP} duality arguments from \cite{ lewis1980relaxation, lasserre2008nonlinear}, in which feasibility of a measure solution is demonstrated in Lemma \ref{lem:feas_meas}.
\end{proof}