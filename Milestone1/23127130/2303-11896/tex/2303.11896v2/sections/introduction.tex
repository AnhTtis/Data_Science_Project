\section{Introduction}
\label{sec:introduction}

% \subsection{New}
% The safety of trajectories will be quantified by the maximum control effort (\ac{OCP} cost) needed to crash the agent into the unsafe set. Distance estimation does not tell the full story about safety; a trajectory may lie close to $X_u$ in the sense of distance $c(x; X_u)$, but it could require a severe control effort to render the same trajectory unsafe. An example of this type of safety result is if the tilting of the steering wheel of a car by a maximum extent of  $3^\circ$ over the course of its motion would cause the car to crash. The process of analyzing safety by peak-minimizing-\ac{OCP} cost will be referred to as `crash safety'.
% This perspective will also be used in the data-driven framework, in which a trajectory is labeled safe if it would require a large constraint violation against any of its state-derivative datapoints in $\mathcal{D}$ in order to crash.


% Let $W \subset \R^L$ be a compact input set, and let $\mathcal{W}$ be the class of functions whose graphs satisfy $(t, w(t)) \in [0, T] \times W$.
% Given a control-cost $J(w)$, we can pose the following peak-minimizing free-terminal-time \ac{OCP}:
% \begin{equation}
%     \label{eq:crash_traj}
%     \begin{aligned}
%     Q^* = & \inf_{t, \ x_0, \ w}  \ \sup_{t' \in [0, t]} J(w(t'))\\
%     & \dot{x}(t') =  f(t', x(t'), w(t')) & & \forall t' \in [0, T] \\
%     & x(t \mid x_0, w(\cdot)) \in X_u \\ 
%     & w(\cdot) \in \mathcal{W}, \ t \in [0, T], \ x_0 \in X_0.
%     \end{aligned}
% \end{equation}
% The variables of \eqref{eq:crash_traj} are the stopping time $t$, the initial condition $x_0$, and the input process $w(\cdot)$. 
% Assuming for the purposes of this discussion that $J(0)=0, \ \forall w \neq 0: J(w) > 0$, and that $J$ possesses connected superlevel sets;
% the set $x_0$ is unsafe if $Q^* = 0$ because the process $w(t) = 0$ is sufficient for the trajectory to reach a terminal set of $X_u$. 
% The value of a nonzero $Q^*$ then measures the amount of control effort (perturbation intensity) needed to render the trajectory unsafe. Connected level sets are imposed to add interpretability to $Q^*$; a disconnected choice of $J$ with multiple local minima could yield a large input $w$ with a low $Q^*$.

% A running cost $\int_{0}^{T} J(w(t')) dt$ yielding a standard-form (Lagrange) \ac{OCP} may also be applied, but we elect to use a peak-minimizing cost $\max_{t'} J(w(t'))$ in order to penalize perturbation intensity.
% The running-cost would penalize a low-magnitude control being applied for an extended period of time, while peak-minimizing control reduces the intensity. 

% Peak-minimizing control problems, such as in  \eqref{eq:crash_traj}, are a particular form of robust optimal control in which the minimizing agents are $(t, x_0, w(\cdot))$ and the maximizing agent is $t' \in [0, t]$. Necessary conditions for these robust programs may be found in \cite{vinter2005minimax}. Instances of  peak-minimizing control include minimizing the maximum number of infected persons in an epidemic under budget constraints \cite{molina2022optimal} and choosing flight parameters to minimize the maximum skin temperature during atmospheric reentry \cite{lu1988minimax, kreim1996minimizing}. The work in \cite{molina2022equivalent} outlines conversions between peak-minimizing \acp{OCP} and equivalent Mayer-form \acp{OCP} (terminal cost only). 


% This chapter transforms program \eqref{eq:crash_traj} into the Mayer \ac{OCP} using \cite{molina2022equivalent}, relaxes the nonconvex \ac{OCP} into an infinite-dimensional \ac{LP} with the same objective value \cite{lewis1980relaxation}, and then lower-bounds $Q^*$ by using the moment-\ac{SOS} hierarchy \cite{henrion2008nonlinear, lasserre2009moments}. The robust counterpart method of Chapter \ref{chap:robust} will be used to simplify the infinite-dimensional \ac{LP} when $W$ and the graph of $J$ are both \ac{SDR}.

% \subsection{Old}

A trajectory starting at an initial point $x_0 \in X$ following dynamics $\dot{x}= f_0(t, x)$ is safe with respect to the unsafe set $X_u$ in the time horizon $t \in [0, T] \subset [0, \infty)$ if there does not exist a time $t'$ such that $x(t' \mid x_0)$ is a member of $X_u$. The set $X_0$ is safe with respect to $X_u$ if all initial points $x_0 \in X_0$ generate safe trajectories. 
This paper quantifies the safety of trajectories by maximum control effort (\ac{OCP} cost) needed to crash the agent into the unsafe set. 
% Distance estimation does not tell the full story about safety; a trajectory may lie close to $X_u$ in the sense of distance $c(x; X_u)$, but it could require a severe control effort to render the same trajectory unsafe. 
An example of this type of safety result is if tilting a car's steering wheel by a maximum extent of  $3^\circ$ over the course of its motion would cause the car to crash.
% The distance of closest approach does not tell the full story about safety; a trajectory may lie close to $X_u$ in the sense of distance, but it could require a severe control effort to render the same trajectory unsafe.  
The process of analyzing safety by peak-minimizing-\ac{OCP} cost will be referred to as `crash safety'.
This crash-safety will also be used in the data-driven framework, in which a trajectory is labeled safe if it would require the true system to have a large constraint violation against any of its state-derivative data observations in order to crash.

Let $W \subset \R^L$ be a compact input set, and let $\mathcal{W}$ be the class of functions whose graphs satisfy $(t, w(t)) \in [0, T] \times W$.
Given a control-cost $J(w)$, we can pose the following peak-minimizing free-terminal-time \ac{OCP}:
\begin{equation}
    \label{eq:crash_traj}
    \begin{aligned}
    Q^* = & \inf_{t, \ x_0, \ w}  \ \sup_{t' \in [0, t]} J(w(t'))\\
    & \dot{x}(t') =  f(t', x(t'), w(t')) & & \forall t' \in [0, T] \\
    & x(t \mid x_0, w(\cdot)) \in X_u \\ 
    & w(\cdot) \in \mathcal{W}, \ t \in [0, T], \ x_0 \in X_0.
    \end{aligned}
\end{equation}


% Define $w$ to be a selected perturbation process restricted to a class $\mathcal{W}$ (e.g. $L_\infty$ bounded), producing induced perturbation-dynamics $f(t, x, w)$. Given a  cost $J(w)$ with $J(0) = 0 , \ \forall w \neq 0: \ J(w) > 0$ that also possesses connected sublevel sets, we can pose the following peak-minimizing free-terminal-time \ac{OCP},
% \begin{equation}
%     \label{eq:peak_cost}
%     \begin{aligned}
%     Q^* = & \inf_{t, \ x_0, \ w}  \ \sup_{t' \in [0, t]} J(w(t'))\\
%     & \dot{x}(t') =  f(t', x(t'), w(t')) & & \forall t' \in [0, T] \\
%     & x(t \mid x_0, w(\cdot)) \in X_u \\ 
%     & w(\cdot) \in W, \ t \in [0, T], \ x_0 \in X_0
%     \end{aligned}
% \end{equation}

The variables of \eqref{eq:crash_traj} are the stopping time $t$, the initial condition $x_0$, and the input process $w(\cdot)$. 
Assuming for the purposes of this introduction that $J(0)=0, \ \forall w \neq 0: J(w) > 0$ and that $J$ possesses connected superlevel sets,
the set $x_0$ is unsafe if $Q^* = 0$ because the process $w(t) = 0$ is sufficient for the trajectory to reach a terminal set of $X_u$. 
The value of a nonzero $Q^*$ then measures the amount of control effort (perturbation intensity) needed to render the trajectory unsafe. Connected level sets are imposed to add interpretability to $Q^*$; a disconnected choice of $J$ with multiple local minima could yield a large input $w$ with a low $Q^*$.


A running cost $\int_{0}^{T} J(w(t')) dt$ yielding a standard-form (Lagrange) \ac{OCP} may also be applied, but we elect to use a peak-minimizing cost $\max_{t'} J(w(t'))$ in order to penalize perturbation intensity.
The running-cost would penalize a low-magnitude control being applied for an extended period of time, while peak-minimizing control reduces the intensity. 

Peak-minimizing control problems, such as in  \eqref{eq:crash_traj}, are a particular form of robust optimal control in which the minimizing agents are $(t, x_0, w(\cdot))$ and the maximizing agent is $t' \in [0, t]$. Necessary conditions for these robust programs may be found in \cite{vinter2005minimax}. Instances of  peak-minimizing control include minimizing the maximum number of infected persons in an epidemic under budget constraints \cite{molina2022optimal} and choosing flight parameters to minimize the maximum skin temperature during atmospheric reentry \cite{lu1988minimax, kreim1996minimizing}. The work in \cite{molina2022equivalent} outlines conversions between peak-minimizing \acp{OCP} and equivalent Mayer-form \acp{OCP} (terminal cost only). 

This paper continues a sequence of research about quantifying the safety of trajectories. 
Unsafety can be proven using path-planning by finding a feasible pair $(t', x_0) \in [0, T] \times X_0$ such that $x(t' \mid x_0) \in X_0$.
Barrier \cite{prajna2004safety, prajna2006barrier} and Density functions \cite{rantzer2004analysis} are binary certificates confirming that there does not exist an unsafe trajectory based on the satisfaction of nonnegativity constraints. Safety margins use maximin peak estimation to estimate the $X_u$-representing-inequality-constraint violation \cite{miller2020recovery}. The distance of closest approach between a trajectory starting in $X_0$ and points in $X_u$ is a more interpretable measure of safety than abstract safety margins \cite{miller2021distance}. Even so, distance estimation does not tell the full story; a trajectory may lie close to $X_u$ in the sense of distance, but it could require a large value of $Q^*$ to render the same trajectory unsafe. 
 

Direct solution of \acp{OCP} using the \ac{HJB} equation or the Pontryagin Maximum Principle may be challenging, especially when solutions do not exist in closed form \cite{liberzon2011calculus}. These generically non-convex \acp{OCP} may be lifted into convex infinite-dimensional \acp{LP} in occupation measures \cite{lewis1980relaxation}, whose dual \ac{LP} involve subvalue functions satisfying \ac{HJB} inequalities.
These infinite-dimensional \acp{LP} produce lower-bounds on the true \ac{OCP}, with equality holding under compactness and regularity conditions.
The moment-\ac{SOS} hierarchy of \acp{SDP} may be used to produce a rising sequence of lower bounds to the true \ac{OCP} if the dynamics $f(t, x, w)$ are polynomial and the sets $(W, X, X_0, X_u)$ are \ac{BSA} \cite{henrion2008nonlinear}. This infinite-dimensional \ac{LP} and finite-dimensional \ac{SDP} pattern has also been applied to reachable set estimation \cite{henrion2013convex, korda2013inner}, peak estimation \cite{fantuzzi2020bounding, miller2021uncertain}, and maximum controlled invariant set estimation \cite{korda2014convex}.

This paper transforms program \eqref{eq:crash_traj} into the Mayer \ac{OCP} using \cite{molina2022equivalent}, relaxes the nonconvex \ac{OCP} into an infinite-dimensional \ac{LP} with the same objective value \cite{lewis1980relaxation}, and then lower-bounds $Q^*$ by using the moment-\ac{SOS} hierarchy \cite{henrion2008nonlinear, lasserre2009moments}. The robust counterpart method of  \cite{ben2009robust, miller2023robustcounterpart} will be used to simplify the infinite-dimensional \ac{LP} when $W$ and the graph of $J$ are both polytopic.
