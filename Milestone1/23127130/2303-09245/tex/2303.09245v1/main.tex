% Template for ICASSP-2019 paper; to be used with:
%          spconf.sty  - ICASSP/ICIP LaTeX style file, and
%          IEEEbib.bst - IEEE bibliography style file.
% --------------------------------------------------------------------------
\documentclass{article}
\usepackage{spconf,amsmath,graphicx}
% \usepackage{indentfirst}
\usepackage{multirow}
\usepackage{booktabs}
\usepackage{amssymb}
\usepackage{amsmath}
\usepackage{xspace}
\usepackage{color}
\usepackage{pifont}
\usepackage{hyperref}

% Example definitions.
% --------------------
\def\x{{\mathbf x}}
\def\L{{\cal L}}
\newcommand{\methodname}{CHS-Net\xspace}
\newcommand{\blue}[1]{\textcolor{blue}{#1}}
\newcommand{\shan}[1]{\textcolor{blue}{HM: #1}}
\newcommand{\gao}[1]{\textcolor{green}{JG: #1}}
\newcommand{\mat}[1]{\boldsymbol{#1}}
\newcommand{\ie}{\emph{i.e.}\xspace}
\newcommand{\etal}{\emph{et al.}\xspace}
\newcommand{\eg}{\emph{e.g.}\xspace}

\newcommand{\cmark}{\ding{51}}%
\newcommand{\xmark}{\ding{55}}%

\newlength\savewidth\newcommand\shline{\noalign{\global\savewidth\arrayrulewidth
  \global\arrayrulewidth 1.5pt}\hline\noalign{\global\arrayrulewidth\savewidth}}
% Title.
% ------
\title{cross-head supervision for crowd counting with noisy annotations}
%
% Single address.
% ---------------
\name{Mingliang Dai$^{1}$ \qquad Zhizhong Huang$^{1}$ \qquad Jiaqi Gao$^{1}$ \qquad Hongming Shan$^{2}$ \qquad Junping Zhang$^{1\dagger}$\thanks{$\dagger$: Corresponding author.  jpzhang@fudan.edu.cn. This work was supported in part by the National Natural Science Foundation of China (Nos. 62176059 and 62101136), the Shanghai Municipal Science and Technology
Major Project (No. 2018SHZDZX01), the Zhangjiang Laboratory (ZJLab) and the Shanghai Center for Brain Science and Brain-Inspired Technology.}}
\address{$^1$ Shanghai Key Lab of Intelligent Information Processing, School of Computer Science\\
$^2$ Institute of Science and Technology for Brain-inspired Intelligence\\
Fudan University, Shanghai 200433, China}
%
% For example:
% ------------
%\address{School\\
%	Department\\
%	Address}
%
% Two addresses (uncomment and modify for two-address case).
% ----------------------------------------------------------
%\twoauthors
%  {A. Author-one, B. Author-two\sthanks{Thanks to XYZ agency for funding.}}
%	{School A-B\\
%	Department A-B\\
%	Address A-B}
%  {C. Author-three, D. Author-four\sthanks{The fourth author performed the work
%	while at ...}}
%	{School C-D\\
%	Department C-D\\
%	Address C-D}
%
\begin{document}
% \ninept
%
\maketitle
%
\begin{abstract}
%Crowd counting is a challenging computer vision task due to multi-scale head sizes, high occlusion, etc. 
Noisy annotations such as missing annotations and location shifts often exist in crowd counting datasets due to multi-scale head sizes, high occlusion, etc.
%is a challenging computer vision task due to multi-scale head sizes, high occlusion, etc. 
These noisy annotations severely affect the model training, especially for density map-based methods. 
To alleviate the negative impact of noisy annotations, we propose a novel crowd counting model with one convolution head and one transformer head, in which these two heads can supervise each other in noisy areas, called \textbf{C}ross-\textbf{H}ead \textbf{S}upervision. The resultant model, \methodname, can synergize different types of inductive biases for better counting.
%novel cross-head supervision model, named \methodname, which can supervise each other in noisy areas by leveraging different types of inductive biases.
In addition, we develop a progressive cross-head supervision learning strategy to stabilize the training process and provide more reliable supervision.
%Despite continuous efforts to improve the counting performance, the labeling noise issues have not received sufficient attention. 
%However, the performance of crowd counting suffers from the potential noises in the labeling process, such as missing annotations and location shifts.
%, may have a significant negative impact on the counting performance. 
%To mitigate this issue, therefore, we propose a novel model, named \methodname, consisting of two complementary heads to supervise each other in noisy areas. In addition, we develop a progressive 
%cross-head supervision learning strategy to stabilize the training process and provide more reliable supervision.
Extensive experimental results on ShanghaiTech and QNRF datasets demonstrate superior performance over state-of-the-art methods. Code is available at \url{https://github.com/RaccoonDML/CHSNet}.
\end{abstract}

%\shan{some general suggestions: (1) this paper should be reviewed anonymously; no github url is required. (2) use commands to denote some abbreviations so that you can easily rename them; see \methodname. (3) use a tilde mark to connect the text and cite, ref.; for example, xxxx~\cite{bai2020adaptive}.  }

\begin{keywords}
Crowd counting, noisy annotations
% , cross-head supervision
\end{keywords}

\section{Introduction}
\label{sec:introduction}
% \begin{itemize}
%     % Diffusion of FL
%     \item {\st{Diffusion of FL}}
%     % Security threats to FL
%     \item {\st{Security threats to FL with particular focus on model poisoning}}
%     % Limitations of existing countermeasures
%     \item {\st{Current countermeasures (e.g., KRUM) and their limitations}}
%     % Proposed method and its advantages
%     \item {\st{Intuitive description of the proposed method and its difference (i.e., advantages) w.r.t. state of the art}}
%     % Main contributions
%     \item {\st{Summary of the main contributions of this work}}
%     % Paper's structure and organization
%     \item {\st{Paper's structure and organization}}
% \end{itemize}

% Diffusion of FL
Recently, {\em federated learning} (FL) has emerged as the leading paradigm for training distributed, large-scale, and privacy-preserving machine learning (ML) systems~\cite{mcmahan2017googleai,mcmahan2017aistats}. 
The core idea of FL is to allow multiple edge clients to collaboratively train a shared, global model without disclosing their local private training data.
%Specifically, an FL system consists of a central server and many edge clients; 
A typical FL round involves the following steps: {\em(i)} the server randomly picks some clients and sends them the current, global model; {\em(ii)} each selected client locally trains its model with its own private data; then, it sends the resulting local model to the server;\footnote{Whenever we refer to global/local model, we mean global/local model {\em parameters}.} {\em(iii)} the server updates the global model by computing an \emph{aggregation function}, usually the average (FedAvg), on the local models received from clients.
% \begin{enumerate}
%     \item[{\em(i)}] the server sends the current, global model to the clients and appoints some of them for training;
%     \item[{\em(ii)}] each selected client locally trains its copy of the global model with its own private data; then, it sends the resulting local model back to the server;\footnote{Whenever we refer to global/local model, we mean global/local model {\em parameters}.}
%     \item[{\em(iii)}] the server updates the global model by computing an \emph{aggregation function} on the local models received from clients (by default, the average, also referred to as FedAvg~\cite{mcmahan2017aistats}).
% \end{enumerate}
This process goes on until the global model converges. %(e.g., after a certain number of rounds or other similar stopping criteria).
%\\
% The advantages of FL over the traditional, centralized learning paradigm are undoubtedly clear in terms of flexibility/scalability (clients can join/disconnect from the FL network dynamically), network communications (only model weights\footnote{We will use \textit{parameters} and \textit{weights} interchangeably.} are exchanged between clients and server), and privacy (each client's private training data is kept local at the client's end and not uploaded to the server).
\\
% Security threats to FL
%However, the growing adoption of FL also raises security concerns~\cite{costa2022covert}, particularly about its confidentiality, integrity, and availability.
Although its advantages over standard ML, FL also raises security concerns~\cite{costa2022covert}. %, particularly about its confidentiality, integrity, and availability~\cite{costa2022covert}.
% OLD, LONG VERSION
% Indeed, some work deals with privacy leakage that may expose the local data of some clients~\cite{melis2019sp}. 
% A large body of work, instead, investigates attacks that usually aim to detriment the predictive accuracy of the learned global model. For instance, \emph{data poisoning} attacks achieve this goal by letting an adversary pollute the training set of some corrupt FL clients with maliciously crafted examples~\cite{jagielski2018sp}.
% Similarly, in \emph{model poisoning} the attacker attempts to tweak the global model weights~\cite{bhagoji2019pmlr} by directly perturbing the local model's weights of some infected FL clients before these are sent to the central server for aggregation, usually via so-called Byzantine attacks. 
% It turns out that Byzantine model poisoning attacks severely impact standard FedAvg; therefore, more robust aggregation functions must be designed to make FL systems secure.
Here, we focus on \emph{untargeted model poisoning} attacks~\cite{bhagoji2019pmlr}, where an adversary attempts to tweak the global model weights %\footnote{We will use the terms \textit{parameters} and \textit{weights} interchangeably.} 
by directly perturbing the local model's parameters of some infected clients before these are sent to the central server for aggregation.
In doing so, the adversary aims to jeopardize the global model \textit{indiscriminately} at inference time.
Such model poisoning attacks severely impact standard FedAvg; therefore, more robust aggregation functions must be designed to secure FL systems.
\\
% In this paper, we focus on designing a novel robust aggregation scheme at the server's end to contrast the effect of Byzantine model poisoning attacks.
%
% Current countermeasures and their limitations
%Several countermeasures have been proposed in the literature to combat model poisoning attacks on FL systems.
% Some methods use simple statistics more robust than plain average to smooth the impact of malicious updates (e.g., Trimmed Mean and FedMedian~\cite{yin2018icml}). 
% Other defenses implement outlier detection techniques to discard malicious updates from the aggregation performed at the server's end. Those are either based on heuristics (e.g., Krum/Multi-Krum~\cite{blanchard2017nips} and Bulyan~\cite{mhamdi2018pmlr}) or data-driven approaches (e.g., K-means clustering~\cite{shen2016acm} or DnC via spectral analysis~\cite{shejwalkar2021ndss}). 
% Finally, some strategies rely on a centralized ``source of trust'' to spot potential malicious updates (e.g., FLTrust~\cite{cao2020fltrust}).
% Several countermeasures have been proposed in the literature to combat model poisoning attacks on FL systems, i.e., to discard possible malicious local updates from the aggregation performed at the server's end. 
% These techniques range from simple statistics more robust than plain average (e.g., Trimmed Mean and FedMedian~\cite{yin2018icml}) to outlier detection heuristics (e.g., Krum/Multi-Krum~\cite{blanchard2017nips} and Bulyan~\cite{mhamdi2018pmlr}) or data-driven approaches (e.g., spectral analysis via K-means clustering~\cite{shen2016acm} or spectral analysis), or methods based on ``source of trust'' (e.g., FLTrust~\cite{cao2020fltrust}).
% OLD, LONG VERSION
%Several countermeasures have been proposed in the literature to combat Byzantine model poisoning attacks on FL systems.
% Descriptive statistics
% For example, Trimmed Mean and FedMedian aggregate local model updates using more robust statistics than standard average~\cite{yin2018icml}.
%
% % Heuristics for outlier detection
% Many existing Byzantine-resilient strategies implement some outlier detection heuristics to discard the model updates sent by potentially malicious clients from the input of the aggregation function.
% One of the most popular heuristics is Krum~\cite{blanchard2017nips}.
% This strategy tries to mitigate the impact of Byzantine attacks by selecting as a global model the local model with the smallest sum of Euclidean distances to {\em all} the other local models.
% Although powerful, Krum requires the server to know (or, at least, estimate) the number of malicious FL clients upfront, which is generally impossible in a realistic attack scenario. %
% Moreover, Krum may become ineffective for complex, high-dimensional model parameter spaces due to the curse of dimensionality.
% Bulyan~\cite{mhamdi2018pmlr} tries to overcome this issue by combining Krum with a variant of Trimmed Mean.
% % Data-driven outlier detection
% Other strategies use data-driven outlier detection techniques -- e.g., via K-means clustering~\cite{shen2016acm} -- to spot potential malicious local model updates. 
% %For instance, Shen et al. propose to cluster local model updates with K-means and thus identify outliers.
%
% % Other techniques
% As far as the server is concerned, any local model received can be from a potential malicious client. 
% FLTrust~\cite{cao2020fltrust} assumes the server acts as a client, i.e., trains a local model on an additional {\em trustworthy} dataset at the server's end and compares it against all the local models from other clients. 
% This way, the server can rely on some ``source of trust'' when discarding potentially malicious clients.
%\\
% Limitations of existing Byzantine-resilient strategies
Unfortunately, existing defense mechanisms either rely on simple heuristics (e.g., Trimmed Mean and FedMedian by~\cite{yin2018icml}) or need strong and unrealistic assumptions to work effectively (e.g., foreknowledge or estimation of the number of malicious clients in the FL system, as for Krum/Multi-Krum~\cite{blanchard2017nips} and Bulyan~\cite{mhamdi2018pmlr}, which, however, cannot exceed a fixed threshold).
Furthermore, outlier detection methods using K-means clustering~\cite{shen2016acm} or spectral analysis like DnC~\cite{shejwalkar2021ndss} do not directly consider the temporal evolution of local model updates received.
Finally, strategies like FLTrust~\cite{cao2020fltrust} require the server to collect its own dataset and act as a proper client, thereby altering the standard FL protocol.
\\
% OLD, LONG VERSION
% Overall, existing Byzantine-resilient strategies are either simple heuristics (e.g., FedMedian) or, if they are more complex, they rely on strong and unrealistic assumptions to work effectively (e.g., knowing the number of malicious clients in the FL system in advance, as for Krum and alike).
% Furthermore, data-driven outlier detection methods do not consider the temporary evolution of local model updates received (e.g., K-means clustering). 
% Finally, strategies like FLTrust requires the server to collect its own dataset and act as a proper client, thereby altering the standard FL protocol.
%
% Description of the proposed method
This work introduces a novel pre-aggregation \textit{filter} robust to untargeted model poisoning attacks. Notably, this filter $(i)$ operates without requiring prior knowledge or constraints on the number of malicious clients and $(ii)$ inherently integrates temporal dependencies. 
The FL server can employ this filter as a preprocessing step before applying \textit{any} aggregation function, be it standard like FedAvg or robust like Krum or Bulyan.
Specifically, we formulate the problem of identifying corrupted updates as a multidimensional (i.e., matrix-valued) time series anomaly detection task. 
The key idea is that legitimate local updates, resulting from well-calibrated iterative procedures like stochastic gradient descent (SGD) with an appropriate learning rate, show \textit{higher predictability} compared to malicious updates. This hypothesis stems from the fact that the sequence of gradients (thus, model parameters) observed during legitimate training exhibit regular patterns, as validated in Section~\ref{subsec:intuition}. %until convergence. 
%This regularity may be more pronounced for smooth convex loss functions, but it can still be captured within an appropriate time window, even for more complex and convoluted loss surfaces. 
%We provide evidence of this claim in Appendix~B, where we show that the average mutual information (i.e., ``predictability''), calculated over pairs of legitimate model updates sent at different FL rounds, is significantly higher than the corresponding computation for a malicious client.
\\
Inspired by the matrix autoregressive (MAR) framework for multidimensional time series forecasting~\cite{chen2021je}, we propose the FLANDERS ({\em \textbf{F}ederated \textbf{L}earning meets \textbf{AN}omaly \textbf{DE}tection for a \textbf{R}obust and \textbf{S}ecure}) filter.
The main advantages of FLANDERS over existing strategies like FLDetector~\cite{zhao2020multivariate} are its resilience to large-scale attacks, where $50\%$ or more FL participants are hostile, and the capability of working under realistic non-iid scenarios.
We attribute such a capability to two key factors: $(i)$ FLANDERS works without knowing a priori the ratio of corrupted clients, and $(ii)$ it embodies temporal dependencies between intra- and inter-client updates, quickly recognizing local model drifts caused by evil players. Below, we summarize our main contributions:

\begin{itemize}
\item[{\em(i)}]
We provide empirical evidence that the sequence of models sent by legitimate clients is more predictable than those of malicious participants performing untargeted model poisoning attacks.
\\
\item[{\em(ii)}] 
We introduce FLANDERS, the first pre-aggregation filter for FL robust to untargeted model poisoning based on multidimensional time series anomaly detection.
\\
\item[{\em(iii)}] 
We integrate FLANDERS into Flower,\footnote{\scriptsize{\url{https://flower.dev/}}} a popular FL simulation framework for reproducibility.
\\
\item[{\em(iv)}] 
We show that FLANDERS improves the robustness of the existing aggregation methods under multiple settings: different datasets, client's data distribution (non-iid), models, and attack scenarios.
\\
\item[{\em(v)}] 
We publicly release all the implementation code of FLANDERS along with our experiments.\footnote{\scriptsize{\url{https://anonymous.4open.science/r/flanders_exp-7EEB}}}
\end{itemize}

% Paper's structure and organization
The remainder of the paper is structured as follows. %some related work and the current state-of-the-art solutions to security issues that FL entails. 
Section~\ref{sec:background} covers background and preliminaries. 
In Section~\ref{sec:related}, we discuss related work.
Section~\ref{sec:problem} and Section~\ref{sec:method} describe the problem formulation and the method proposed. % to tackle it. 
Section~\ref{sec:experiments} gathers experimental results. %, and Section~\ref{sec:limitations} discusses some limitations of this work.
Finally, we conclude in Section~\ref{sec:conclusion}.
 %discusses the limitations of this work and draws future research directions.
%reports conclusions and draws perspectives for future research directions.

%%%%%%% OLD %%%%%%%
%to overcome the resilience of Byzantine failures in distributed Stochastic Gradient Descent computations. 
% The strength of Krum is its time complexity, which is linear in the gradient dimension. 
% However, the robustness of the approach is guaranteed for gradient-based learning applications only when the majority of the clients are not compromised. 
% Besides, the aggregation mechanism of Krum, as well as that of similar methods, is robust from a coarse-grained perspective and does not provide solutions to errors and perturbations that may occur at inference time.
%A related approach to~\cite{blanchard2017nips} is the work of Su et al.~\cite{su2016dc}. Here, the authors propose an iterated approximate agreement to tackle a multi-layer scenario attacked by Byzantine agents. 
%However, the method works efficiently on the sole discrete context and it is inapplicable to continuous state environments.
%\gabri{Maybe, we should just talk about the main limitations of existing countermeasures without digging into their details (or, we can just mention Krum as this is the most popular one). I will move the description of all these methods to the Related Work section.}
\section{Method}
\label{s:method}

We consider the 3D euclidean space $\Real^3$ with points $p=(x,y,z)\in\Real^3$. We discretize the unit cube $\gC=[0,1]^3$ with a 3D voxel grid $\gG=\set{p_I}$, with nodes $p_I$ indexed by $I=(i,j,k)$, $i,j,k\in [n]=\set{1,\ldots,n}$, \ie, $p_I=(x_{ijk},y_{ijk},z_{ijk})$. We denote by $h=n^{-1}$, and by $N=n^3$ the total number of nodes.   
We represent our reconstructed surface as a zero level of a scalar function $f$ defined over the cube $\gC$. $f$ is defined by prescribing its values at the grid's nodes $f_I\in\Real$ and trilinear interpolating in each voxel. We will denote by $f(p)$ the interpolated value at point $p$. 

Given an input point cloud consisting of $m$ points $q_k\in\Real^3$ with or without (unit norm) normals $n_k\in \Real^3$, $k\in [m]$, our goal is to compute $f$ so that its zero level set approximates the unknown surface, \ie, 
\begin{equation}
    \gS = \set{p\in\gC \ \vert \ f(p)=0}.
\end{equation}
Our approach to compute $f$ is to minimize a loss function of the form
\begin{equation}
    \gL = \gL_{\text{data}} + \gL_{\text{prior}}
\end{equation}
where 
\begin{equation}\label{e:loss_data}
    \gL_{\text{data}} = \frac{\lambda_{\text{p}}}{m}\sum_{k=1}^m \abs{f(q_k)}^2 + \frac{\lambda_{\text{n}}}{m}\sum_{k=1}^m \norm{\nabla f(q_k) - n_k}^2
\end{equation}
where $\norm{\cdot}$ is the standard euclidean norm in $\Real^3$, $\nabla f(p) \in \Real^3$ is the gradient of $f$ sampled at point $p$. Note that $\nabla f$ is defined in interior of voxels, which is generically where the input points $q_k$ resides. $\gL_{\text{data}}$ is the standard data loss encouraging the zero level to pass through the input points $q_k$, and its normals (defined by gradients of $f$) to coincide with input normals $n_k$. 

The prior, $\gL_{\text{prior}}$, is the main contribution of this work, where we combine two novel losses,
\begin{equation}
    \gL_{\text{prior}} = \lambda_{\text{v}} \gL_{\text{viscosity}} + \lambda_{\text{c}} \gL_{\text{coarea}}
\end{equation}
Intuitively, the viscosity loss optimizes for a smooth Signed Distance Function (SDF) solutions, avoiding auxiliary bad minima of the Eikonal equation, while the coarea loss strives to minimize the area of the zero level surface. Our loss has $4$ hyper-parameters $\lambda_{\text{p}},\lambda_{\text{n}},\lambda_{\text{v}},\lambda_{\text{c}}$. We provide more details on these priors next. 


\subsection{Viscosity Loss}\label{ss:viscosity_loss}
The goal of the viscosity loss is to make $f$ approximate an SDF over $\gC$. Given boundary conditions asking $f$ to vanish on some closed compact surface $\gS$, the SDF solves the Eikonal equation PDE, \ie, $\norm{\nabla f(p)}=1$, in a certain well defined sense (viscosity). This motivated some previous work to directly optimize the Eikonal loss \citep{gropp2020implicit,sitzmann2020implicit}
\begin{equation}\label{e:loss_eikonal}
    \gL_{\text{eikonal}} = \int_\gC \Big (\norm{\nabla f(p)}-1\Big )^2 dp
\end{equation}
\begin{wrapfigure}[14]{r}{0.28\textwidth}\vspace{-15pt}
  \begin{center}
    \includegraphics[width=0.25\textwidth]{figs/illustrations/eikonl_1d.png}
  \end{center}
  \caption{Two global minimizers of the Eikonal loss over a grid in 1D. Top solution is not an SDF. }\label{fig:eikonal_1d}
\end{wrapfigure}
Unfortunately, the Eikonal loss has many undesirable minima which are not SDFs. Figure \ref{fig:eikonal_1d} shows a 1D example: both depicted solutions (denoted $f$) vanish at the input points $q_1,q_2$ (black points) and globally minimize the Eikonal loss over the grid (grid points are shown in blue). The INR works mentioned above use neural networks for representing $f$ which injects an inductive bias avoiding these bad minima, however on grids, minimizing \eqref{e:loss_eikonal} cannot avoid these solutions. See, \eg, middle column in Figure \ref{fig:teaser}. 

More classical Eikonal solvers do work with grids however use mostly fast marching or sweeping methods \citep{osher1988fronts,sethian1996fast,zhao2005fast,chacon2012fast}. Namely, use a special discretization of the Eikonal equation favoring the viscosity  solution of the Eikonal \cite{rouy1992viscosity}, and update node values according to a moving front \cite{sethian1996fast}. Since this discretization is up-wind (will only propagate values in one direction) and requires choosing the maximal among its solution, its success in adaptation to a loss is not clear. 

We use a different approach to build a loss favoring SDF solutions over grids motivated by the vanishing viscosity method \cite{crandall1983viscosity}. Namely, adding to the Eikonal PDE a small perturbation of the Laplacian of $f$ (denoted by $\Delta f$), \ie, $\norm{\nabla f(p)}-1 - \eps\Delta f(p)=0$, makes the PDE semi-linear elliptic \citep{calder2018lecture}, and hence with suitable boundary conditions it is uniquely solvable inside $\gS$ with a smooth solution, approaching the viscosity positive distance function to the boundary as $\eps\too 0$. Similarly, for $1-\norm{\nabla f(p)} - \eps \Delta f(p)=0$ the solution approaches the negative distance function inside the domain. 
Motivated by the vanishing viscosity principle we suggest the following viscosity loss:
\begin{equation}\label{e:loss_viscosity_eikonal}
\gL_{\text{viscosity}} = \int_\gC \Big((\norm{\nabla f (p)}-1)\mathrm{sign}(f(p)) - \eps \Delta f(p)\Big)^2 dp
\end{equation}
We discretize this loss over the grid $\gG$ by replacing the first order derivatives and second order derivatives with symmetric finite  differences, \ie,
\begin{align*}
    D_x f_I=D_x f_{i,j,k} = \frac{f_{i+1,j,k}-f_{i-1,j,k}}{2h}, \quad D^2_x f_I = D^2_x f_{i,j,k}=\frac{f_{i+1,j,k}-2f_{i,j,k}+f_{i-1,j,k}}{h^2}
\end{align*}
and similarly for $D_y$ and $D_z$. We use these discrete operators to approximate the gradient $\widehat{\nabla} f(p_I) = (D_x f_I, D_y f_I, D_z f_I)$ and Laplacian $\widehat{\Delta}f(p_I) = D_x^2f_I + D_y^2 f_I + D_z^2 f_I$. The discretized viscosity loss now takes the form
\begin{equation}
    \widehat{\gL}_{\text{viscosity}} = \frac{1}{N}\sum_{I} \Big((\|\widehat{\nabla} f (p_I)\|-1)\mathrm{sign}(f(p_I)) - \eps \widehat{\Delta} f(p_I)\Big)^2
\end{equation}



\subsection{Coarea loss}\label{ss:coarea_loss}
The coarea loss is approximating the area of the zero level set, and therefore incorporating it in the optimization pushes the reconstructed surface to be economic in area. 

First, similarly to  \citep{yariv2021volume} we use the centered Laplace CDF
\begin{equation}
   \Psi\beta(s)= \begin{cases}
   \frac{1}{2}\exp\parr{\frac{s}{\beta}} & s\leq 0 \\ 1-\frac{1}{2}\exp\parr{-\frac{s}{\beta}} & s\geq  0
   \end{cases}
\end{equation} to transform the SDF $f$ to a smooth approximation of the indicator function:
\begin{equation}
    \chi_\beta(p)=\Psi\beta (-f(p))
\end{equation}
As $\beta\too 0$, $\chi_\beta$ converges to an indicator function leading to $1$ inside $\gS$ and $0$ outside. The coarea loss is defined as 
\begin{equation}
    \gL_{\text{coarea}} = \int_\gC \norm{\nabla \chi_\beta (p)} dp
\end{equation}
To understand why this loss approximates the area of $\gS$ we can use the coarea formula \citep{rindler2018calculus}:
\begin{equation}\label{e:coarea}
    \int_\gC \norm{\nabla \chi_\beta(p)}dp = \int_{-\infty}^{\infty} \mathrm{area}(\chi_\beta^{-1}(s))ds,
\end{equation}
where $\chi_\beta^{-1}(s)=\set{p\ \vert \ \chi_\beta(p)=s}$ is the preimage of the value $s$. Since $\chi_x(p)\in [0,1]$ the r.h.s.~integral can be restricted to the interval $[0,1]$, and therefore the coarea loss averages the area of the level sets of $\chi_\beta$. Next,  $$\chi_\beta^{-1}(s)= \set{p\ \vert \ \Psi\beta (-f(p)) = s } = \{p\ \vert \ f(p) = -\Psi\beta^{-1} (s) \} = f^{-1}(-\Psi\beta^{-1} (s)),$$
\begin{wrapfigure}[11]{r}{0.28\textwidth}\vspace{-20pt}
  \begin{center}
  \includegraphics[width=0.25\textwidth]{figs/semi.png}
  \end{center}
  \caption{Reconstruction of a semisphere point cloud (white dots) without (left) and with (right) coarea loss. }\label{fig:coarea_semisphere}
\end{wrapfigure}

which shows that the level set $s\in (0,1)$ of $\chi_\beta$ is the level set $-\Psi\beta^{-1}(s)$ of the SDF $f$. As $\beta\too 0$, $-\Psi\beta^{-1}(s)\too 0$ for all $s\in (0,1)$ (and uniformly in $(\eps,1-\eps)$ for fixed $\eps>0$). Therefore the average of the level set area (\ie, the r.h.s.~of \eqref{e:coarea}) converges to the area of $f^{-1}(0)=\gS$. Figure \ref{fig:teaser} (right) shows how removing the coarea loss introduces an extraneous zero level set, and hence results in an undesired surface part. Figure \ref{fig:coarea_semisphere} shows a comparison of a reconstruction of semisphere with and without coarea. In the experiments section we provide more ablation tests with the coarea and viscosity losses.

To discretize the coarea loss we let $w_I$ denote the centers of grid's voxels, and note that $\nabla \chi_\beta(w_I) = \Phi_\beta(-f(w_I))\nabla f(w_I)$, where 
\begin{equation*}
    \Phi_\beta(s) = \frac{1}{2\beta}\exp\parr{\frac{\abs{s}}{\beta}}
\end{equation*}
is the PDF of the Laplace distribution, and $\nabla f(w_I)$ is computed as a linear combination of the voxel's corner values $f_{I_1},\ldots,f_{I_8}$, see more details in the Appendix. We end up with the discretized loss:
\begin{equation}
    \widehat{\gL}_{\text{coarea}} = \frac{1}{N}\sum_{I}\Phi_\beta(-f(w_I))\norm{\nabla f(w_I)}
\end{equation}
This loss is usually incorporated with a small hyper-parameter $\lambda_{\text{c}}$ with the purpose of eliminating redundant surface parts.


We present in section~\ref{ssec:faces} an application of PnP-HVAE on face images, using a pretrained state-of-the-art hierarchical VAE. 
Next, we study the application of our framework to natural images. To that end, we introduce  in section~\ref{ssec:patchVDVAE}  a patch hierachical VAE architecture, that is able to model natural images of different resolutions. In section~\ref{ssec:app_nat}, we provide deblurring, super-resolution and inpainting experiments to demonstrate the relevance of the proposed method.

Additional results are presented in Appendix~\ref{app:add}. All experiments can be reproduced using the code available at \url{https://github.com/jprost76/PnP-HVAE}.



\subsection{Face Image restoration (FFHQ)}\label{ssec:faces}
We first demonstrate the effectiveness of PnP-HVAE on highly structured data, by performing face image restoration.
Latent variable generative models can accurately model structured images such as face images \cite{karras2019style,vahdat2020nvae,child2021very,kingma2018glow}, and then be used to produce high quality restoration of such data. 
In our experiments, we use the VDVAE model of~\cite{child2021very}, pre-trained on the FFHQ dataset~\cite{karras2019style}, as our hierarchical VAE prior.
VDVAE has $L=66$ latent variable groups in its hierarchy and generates images at resolution $256\times256$.

We compare PnP-HVAE with the intermediate layer optimization algorithm (ILO)~\cite{daras2021intermediate} that is based on a different class of generative models than HVAE. ILO is a GAN inversion method which optimizes the image latent code along with the intermediate layer representation of a StyleGAN to generate an image consistent with a degraded observation.
We use the official implementation of ILO, along with a StyleGAN2 model~\cite{karras2020analyzing, stylegan2pytorch}, that was trained for 550k iterations on images of resolution $256\times256$ from FFHQ.  
As VDVAE and StyleGAN models are not trained on the same train-test split of FFHQ, we chose to evaluate the methods on a subset of 100 images from the CelebA dataset~\cite{liu2018large}. 
For super-resolution, the degradation model corresponds to the application of a gaussian low-pass filter followed by a $\times 4$ sub-sampling, and the addition of a gaussian white noise with $\sigma=3$.
For the deblurring, we considered motion blur and  gaussian kernels, both with a noise level $\sigma=8$. %

We provide quantitative comparisons in table~\ref{table:comp_ILO}, along with a visual comparison of the results in figure~\ref{fig:face_restoration}.
PnP-HVAE has the best  PSNR and SSIM results for all the considered restoration tasks, while ILO provides better results  for the perceptual distance.
By jointly optimizing the image and its latent variable, PnP-HVAE provides  results that are both realistic and consistent with the degraded observation.
On the other hand,  ILO  only optimizes on an extended latent space. This method generates  sharp and realistic images with better LPIPS scores,   
but the results lack  of consistency with respect to the observation, which explains the overall lower PSNR performance. 






\subsection{PatchVDVAE: a HVAE for natural images}\label{ssec:patchVDVAE}
Available generative models in the literature operate on images of  fixed resolutions and
are either restrained to datasets of limited diversity, or even to registered face images~\cite{kingma2018glow,child2021very, vahdat2020nvae, karras2019style}, or requiring additional class information~\cite{brock2018large, dhariwal2021diffusion, song2020score, luhman2022optimizing}.
Fitting an unconditional model on natural images appears to be a more difficult task, as their resolution can change, and their content is highly diverse.
The complexity of the problem can be reduced by learning a prior model on patches of reduced dimension. 
For image restoration problems, the patch model can be reused on images of higher dimensions~\cite{zoran2011learning,prost2021learning,altekruger2022patchnr}. When the model is a full CNN, the prior on the set of the  patches can  be computed efficiently by applying the network on the full image~\cite{prost2021learning}.

We thus introduce  patchVDVAE, a fully convolutional hierarchical VAE.
Contrary to existing HVAE models whose resolution is constrained by the constant tensor at the input of the top-down block, patchVDVAE can generate images of different resolutions by controlling the dimension of the input latent. 
This amounts to defining a prior on patches whose dimension corresponds to the receptive field of the VAE. A similar model is used for image denoising in~\cite{prakash2021interpretable}.

 
For PatchVDVAE architecture, we use the same bottom-up and top-down blocks as VDVAE~\cite{child2021very}, and replace the constant trainable input in the first top-down block by a latent variable, to make the model fully convolutional (details on the  architecture are given in Appendix~\ref{app:details}). 
The training dataset is composed of $128\times 128$ patches extracted from a combination of DIV2K~\cite{agustsson2017ntire} and Flickr2K~\cite{Lim_2017_CVPR_workshops} datasets.
We perform data augmentation by extracting  patches at $3$ resolutions: HR-images and $\times 2$ and $\times 4$ downscaled images. 
The model is trained for $7.10^5$ iterations with a batch size of $64$. Following the recommendation of~\cite{hazami2022efficient}, we use Adamax optimizer with an exponential moving average and gradient smoothing of the variance.
We set the decoder model to be a gaussian with diagonal covariance, as in~\cite{luhman2022optimizing}.
PatchVDVAE is fully convolutional and can generate images of dimension that are multiples of $64$ as illustrated by
figure~\ref{fig:vdvae}.

\newlength{\patchwidth}
\setlength{\patchwidth}{0.135\columnwidth}
\begin{figure}[!ht]
    \centering
    \begin{subfigure}[t]{.34\columnwidth}\hspace{0.1cm}
        \setlength{\tabcolsep}{0.02pt}
\renewcommand{\arraystretch}{0}
        \begin{tabular}{*{2}{p{1.03\patchwidth}}}
            \includegraphics[width=\patchwidth]{figures_arxiv/patchVDVAE/samples/generated/64x64/setup-5-image-0018.png} &
            \includegraphics[width=\patchwidth]{figures_arxiv/patchVDVAE/samples/generated/64x64/setup-5-image-0016.png} \\
            \includegraphics[width=\patchwidth]{figures_arxiv/patchVDVAE/samples/generated/64x64/setup-5-image-0008.png} &
            \includegraphics[width=\patchwidth]{figures_arxiv/patchVDVAE/samples/generated/64x64/setup-5-image-0019.png}   
        \end{tabular}
    \end{subfigure}\hspace{-0.15cm}
    \begin{subfigure}[t]{.64\columnwidth}
\begin{tabular}{cc}\vspace{-0.1cm}
\includegraphics[width=2\patchwidth]{figures_arxiv/patchVDVAE/samples/generated/256x256/setup-2-image-0009.png}&
        \includegraphics[width=2\patchwidth]{figures_arxiv/patchVDVAE/samples/generated/256x256/setup-2-image-0002.png}\end{tabular}

    \end{subfigure}
    \caption{\label{fig:vdvae} Left: $64\times64$ patches samples from our patchVDVAE model trained on patches from natural images.
    Right: PatchVDVAE is fully convolutional and it can generate images of higher resolution (here: $128\times128$).\vspace{-0.2cm}}
\end{figure}

\subsection{Natural images restoration}\label{ssec:app_nat}
We  evaluate PnP-HVAE on natural image restoration.
For each task, we report the average value of the PSNR, the SSIM, and the LPIPS metrics on $20$ images from the test set of the BSD dataset~\cite{MartinFTM01}.\\


\noindent
{\bf Image deblurring.}
In the experiments, we consider $2$ gaussian kernels and $2$ motion blur kernels from~\cite{levin2009understanding}, with $3$ different noise levels 
$\sigma \in \{2.55, 7.65, 12.75\}$.
As a baseline we consider  EPLL~\cite{zoran2011learning}, which learns a prior on image patches with a gaussian mixture model.
We also compare PnP-HVAE  with PnP-MMO and GS-PnP, $2$ competing convergent Plug-and-Play methods based on CNN denoisers.
PnP-MMO~\cite{pesquet2021learning} restricts the denoiser to be contraction in order to guarantee the convergence of the PnP forward-backard algorithm. GS-PnP~\cite{hurault2022gradient} considers a gradient step denoiser and reaches state-of-the-art performances of non converging methods~\cite{zhang2021plug}.
We set the temperature $\tau$  in our method as $0.95$, $0.8$ and $0.6$ for noise levels $2.55$, $7.65$ and $12.75$ respectively, and we let it run for a maximum of $50$ iterations. 
For the three compared methods we use the official implementations and pre-trained models provided by the respective authors. 
Details on the choice of hyperparameters for the concurrent methods are provided in the Appendix~\ref{app:details}
Figure~\ref{fig:deblurring_bsd} illustrates that our method provides correct deblurring results. 

According to table~\ref{tab:deb}, the performance of PnP-HVAE is between those of EPLL and GS-PnP and it outperforms PnP-MMO for large noise levels.\\

\begin{table}
\begin{center}\footnotesize
    \begin{tabular}{>{\centering}m{.3cm}*{5}{c}}
    $\sigma$ &Method & PSNR$\uparrow$ & SSIM$\uparrow$ & LPIPS$\downarrow$  \\ 
    \hline
    \multirow{4}{*}{\vcell{$2.55$}}
    & PnP-HVAE & $27.75$ & $0.79$ & $0.31$\\
    & GS-PNP \cite{hurault2022gradient} & $\mathbf{29.59}$ & $\mathbf{0.84}$ & $\mathbf{0.22}$\\
    & EPLL \cite{zoran2011learning} & $26.49$ & $0.71$ & $0.36$\\ 
    & PnP-MMO \cite{pesquet2021learning} & $\underbar{29.50}$ & $\underbar{0.83}$ & $\underbar{0.20}$ \\ \hline
    \multirow{4}{*}{\vcell{$7.65$}}
    & PnP-HVAE & $\underbar{26.36}$ & $\underbar{0.72}$ & $\underbar{0.40}$\\
    & GS-PNP \cite{hurault2022gradient} & $\mathbf{27.33}$ & $\mathbf{0.77}$ & $\mathbf{0.31}$\\
    & EPLL \cite{zoran2011learning} & $24.04$ & $0.66$ & $0.45$ \\ 
    & PnP-MMO \cite{pesquet2021learning} & $25.34$ & $0.69$ & $0.34$\\
    \hline
    \multirow{4}{*}{\vcell{$12.75$}}
    & PnP-HVAE & $\underbar{25.12}$ & $\mathbf{0.73}$ & $\underbar{0.47}$\\
    & GS-PNP \cite{hurault2022gradient} & $\mathbf{26.32}$ & $\mathbf{0.73}$ & $\mathbf{0.37}$\\
    & EPLL \cite{zoran2011learning} & $23.28$ & $0.61$ & $0.51$ \\ 
    & PnP-MMO \cite{pesquet2021learning} & $22.42$ & $0.53$& $0.54$ \\
    \hline
    &\vspace*{-.3cm}\\
            \multicolumn{2}{c}{Blur and motion kernels}& \multicolumn{3}{c}{
        \includegraphics*[scale=1]{figures_arxiv/kernels/4.png}\;\includegraphics*[scale=1]{figures_arxiv/kernels/7.png}\;\includegraphics*[scale=1]{figures_arxiv/kernels/9.png}\;\includegraphics*[scale=1]{figures_arxiv/kernels/11.png}} 
    \end{tabular}
        \caption{\label{tab:deb}Comparison  of PnP-HVAE  and other restoration methods on deblurring. Results are averaged on $4$ kernels.\vspace{-0.2cm}}% on image deblurring.}
    \end{center}
\end{table}

\begin{figure}
    
    \begin{subfigure}[h]{\linewidth}
        \centering
        \includegraphics*[width=\columnwidth]{figures_arxiv/deb_s255_k7.pdf}\vspace{-0.1cm}
        \caption{Gaussian blur, $\sigma=2.55$}
    \end{subfigure}
    \begin{subfigure}[h]{\linewidth}
        \centering
        \includegraphics*[width=\columnwidth]{figures_arxiv/deb_s765_k11.pdf}\vspace{-0.1cm}
        \caption{Motion blur, $\sigma=7.65$}
    \end{subfigure}\vspace*{-0.1cm}
    \caption{\label{fig:deblurring_bsd} Natural image deblurring\vspace{-0.1cm}}
\end{figure}

\noindent {\bf Effect of the temperature.}
PnP-HVAE gives control on the temperature of the prior over the latent space.
In figure~\ref{fig:temp_effect}, we illustrate that reducing the temperature increases the strength of the regularization prior. In this example the tuning $\tau=0.7$ produces the best performance.\\
\begin{figure}[!ht]
   
    \includegraphics[width=\columnwidth]{figures_arxiv/demo_temp.pdf}\vspace{-0.15cm}
    \caption{ \label{fig:temp_effect} Effect of the temperature in PnP-VAE on a deblurring problem, with $\sigma=7.65$.\vspace{-0.15cm}}
\end{figure}


\noindent
{\bf Image inpainting.}
Next we consider the task of noisy image inpainting. 
We compose a test-set of 10 images from the validation set of BSD~\cite{MartinFTM01} and we create masks
  by occluding diverse objects of small size in the images. 
A gaussian white noise with $\sigma=3$ is added to the images.
As a comparaison, we still consider GS-PnP and EPLL.
For PnP-HVAE, the temperature is set to $\tau=0.6$, and the algorithm is run for a maximum of $200$ iterations, unless the residual $||\x_{k+1}-\x_k||$ is on a plateau.
We provide on Table~\ref{tab:inpainting_bsd} the distortion metrics with the ground truth, as well as a visual
\begin{table}



\begin{center}
    \begin{tabular}{cccc}
        & PSNR$\uparrow$ & SSIM$\uparrow$ &LPIPS$\downarrow$ \\\hline
        PnP-HVAE  & $\mathbf{29.54}$ & $\mathbf{0.93}$ & $\mathbf{0.06}$\\
        GS-PNP & $28.52$ & $\mathbf{0.93}$ & $0.09$\\
        EPLL & $\underline{29.16}$ & $\mathbf{0.93}$ & $\mathbf{0.06}$\\
    \end{tabular}
    \caption{\label{tab:inpainting_bsd}Quantitative evaluation for inpainting on BSD.}
    \end{center}
\end{table}
comparison on figure~\ref{fig:inpainting_bsd}. 
With its hierarchical structure,  PnP-HVAE outperforms the compared methods. \vspace{0.05cm}



\begin{figure}[!h]
    \includegraphics[width=\columnwidth]{figures_arxiv/demo_inp_bsd2.pdf}\vspace{-0.1cm}
    \caption{\label{fig:inpainting_bsd}Natural image inpainting\vspace{-0.3cm}}
\end{figure}











\vspace{-10pt}
\section{Conclusion}
\vspace{-5pt}
Noisy annotations are common in crowd counting datasets. To alleviate the negative impact of noisy annotations, we propose \methodname, a network with a convolution head and a transformer head to mutually supervise each other in noisy areas. In addition, we develop a progressive cross-head supervision learning strategy to stabilize training process and provide more reliable supervision. Experimental results show the superior performance of our proposed approach.
For future work, we will explore more noise robust loss functions to further utilize the ability of \methodname. Besides, we may consider to enhance the model sustainable learning ability like~\cite{gao2023forget} so that the noise ratio of training samples from different domains can be adaptively adjusted.

\clearpage
% References should be produced using the bibtex program from suitable
% BiBTeX files (here: strings, refs, manuals). The IEEEbib.bst bibliography
% style file from IEEE produces unsorted bibliography list.
% -------------------------------------------------------------------------
\bibliographystyle{IEEEbib}
\bibliography{refs}

\end{document}
