\documentclass[11pt]{article}
\usepackage[utf8]{inputenc}
\usepackage{caption}
\usepackage[T1]{fontenc}
\usepackage{graphicx}
\usepackage[english]{babel}
\usepackage{csquotes}
\usepackage{amssymb}
\usepackage{amsthm}
\usepackage{amsbsy}
\usepackage[version=4]{mhchem} %notacion quimica /ce
\usepackage{hyperref}
\usepackage{placeins} %para el floatbarrier
\usepackage{url}
%\usepackage{dblfloatfix}
\usepackage{multirow}
\usepackage{amsmath}
\usepackage{tikz}
\usepackage[sc]{mathpazo}
\linespread{1.05}
\usepackage[letterpaper,top=2.5cm,bottom=2cm,left=1.3cm,right=1.5cm,marginparwidth=1.75cm]{geometry}
\setlength{\marginparwidth}{2cm}
\usepackage[colorinlistoftodos]{todonotes}
%\usepackage{epstopdf}
\usepackage{units}
%paquete para enumerar con letras
\usepackage[shortlabels]{enumitem}
\usepackage{float}
\usepackage{blindtext}
\usepackage{microtype}
\usepackage{makecell}
\usepackage{lettrine}
\usepackage{titling}
\usepackage{booktabs}
\usepackage{ragged2e}
\usepackage{cuted}
\usepackage{authblk}
\usepackage{adjustbox}
\usepackage{xcolor}
\usepackage{lipsum}
%pa que pueda numerar dentro de align*
%\newcommand\numberthis{\addtocounter{equation}{1}\tag{\theequation}}
\usepackage[normalem]{ulem}
\usepackage{lscape}
\usepackage{tabularx}
\usepackage{cancel}
\usepackage{lmodern}%get scalable font
\usepackage{dsfont}
\usepackage{lipsum}
\usepackage{relsize}
\usepackage{comment}
\usepackage{physics} % para la notacion bra y ket
\usepackage{mathtools} % para hacer la flecha en ambas direcciones
\usepackage{tablefootnote}
%\usepackage{gensymb} % para el simbolo °
\usepackage{derivative}%ecuaciones diferenciales de grado n usar \odv[n]{f}{x} y parciales de grado n \pdv[n]{f}{x}
\setlist[itemize]{noitemsep}
\usepackage{abstract}
\renewcommand{\abstractnamefont}{\normalfont\bfseries}
\renewcommand{\abstracttextfont}{\normalfont\small\itshape}
% Keywords command
\providecommand{\keywords}[1]
{
  \small	
  \textbf{\textit{Keywords---}} #1
}

\usepackage{titlesec}% Allows customization of titles

\titleformat{\section}[hang]{\large\centering}{\thesection.}{1em}{} % Change the look of the section titles

\titleformat{\subsubsection}[block]{\large}{\thesubsubsection.}{1em}{} % Change the look of the subsubsection titles

% Allows customization of titles
\usepackage[labelfont=bf,textfont=it]{caption}
\usepackage{subcaption}

\titleformat{\subsection}[block]{\large}{\thesubsection.}{1em}{}
% Change the look of the section titles
\setlength{\droptitle}{-4\baselineskip}
% Move the title up
\usepackage{hyperref}
\usepackage{listings}
\hypersetup{
    colorlinks=true,
    citecolor=blue,
    filecolor=blue,      
    urlcolor=blue,
    allcolors=blue
}

\usepackage{cleveref}
\usepackage[
	backend = biber,
	bibencoding = utf8,
	sorting = none
]{biblatex}
% \usepackage[backend=bibtex]{biblatex}
\newcommand{\angstrom}{\textup{\AA}}
\addbibresource{bibliogra.bib}

%simbolos del thanks
\makeatletter 	  	 	
\renewcommand*{\@fnsymbol}[1]{\ensuremath{\ifcase#1\or *\or \dagger\or \ddagger\or
    \mathsection\or \mathparagraph\or \|\or **\or \dagger\dagger
    \or \ddagger\ddagger \else\@ctrerr\fi}}
\makeatother

\newcommand{\subtitle}[1]{%
  \posttitle{%
    \par\end{center}
    \begin{center}\large#1\end{center}
    \vskip0.5em}%
}

\usepackage{appendix}
\addto\captionsspanish{%
	\renewcommand\appendixname{Anexos}
	\renewcommand\appendixpagename{Anexos}
}
\raggedbottom

\title{Higher order derivatives of the adjugate matrix and the Jordan form}
%\title{{\huge{Some comments on Riesz projection, the derivatives of the adjugate Matrix and a generalization of Thompson and McEnteggert Theorem}}}
\author[1]{Jorge I. Rubiano-Murcia}
\affil[1]{Departamento de F\'isica, Universidad Nacional de Colombia, Carrera 45 No. 26-85, Edificio Uriel Guti\'errez, Bogot\'a D.C., Colombia. Email {\tt jrubianom@unal.edu.co}}
\author[2]{Juan Galvis}
\affil[2]{Departamento de Matem\'aticas, Universidad Nacional de Colombia, Carrera 45 No. 26-85, Edificio Uriel Guti\'errez, Bogot\'a D.C., Colombia. Email {\tt jcgalvis@unal.edu.co}}

% \renewcommand\Authands{ and }

\renewcommand\Authands{, }
\newtheorem{theorem}{Theorem}[section]
\newtheorem{corollary}{Corollary}[theorem]
\newtheorem{lemma}[theorem]{Lemma}
\theoremstyle{definition}
\newtheorem{definition}{Definition}[section]

%\date{December 12, 2022}
\date{\today}
\begin{document}
\maketitle
\begin{abstract}
In this short note, we show that the higher order derivatives of  the adjugate matrix $\mbox{Adj}(z-A)$, are related to the nilpotent matrices  and projections in  the Jordan decomposition of the matrix $A$. These relations appear as a factorization of the derivative of the adjugate matrix as a product of factors related to the eigenvalues, nilpotent matrices and projectors.  
The novel relations are obtained using the Riesz projector and functional calculus. 
The results presented here can be considered a generalization of the Thompson and McEnteggert theorem that relates the adjugate matrix with the orthogonal projection on the eigenspace of simple eigenvalues for symmetric matrices. They can also be viewed as a complement to some  previous results by B. Parisse, M. Vaughan that related derivatives of the adjugate matrix with the invariant subspaces associated with an eigenvalue. Our results can also be interpreted as a general eigenvector-eigenvalue identity.
Many previous works have dealt with relations between the projectors on the eigenspaces and derivatives of the adjugate matrix with the characteristic spaces but it seems there is no explicit mention in the literature of the factorization of the higher-order derivatives of the 
adjugate matrix as a product involving  nilpotent and projector matrices that appears in the Jordan decomposition theorem.
\end{abstract}\hspace{10pt}


\keywords{Riezs projector, adjugate matrix, Cauchy integral form, Jordan normal form.}

%\tableofcontents
\section{Introduction and main results}
Let $A$ be a $n \times n$ complex matrix. 
Denoted by  $\mbox{Adj}A$ the adjugate matrix of 
$A$. We recall a result of  {Thompson and McEnteggert}  which says that for a hermitian matrix
 $A$ and $\lambda_i$ a simple eigenvalue of $A$ with unit eigenvector $z_i$, it holds,
\begin{equation}
    \mbox{\normalfont Adj}(\lambda_i - A) = \frac{d p(z)}{dz}\Bigr|_{\lambda_i} z_i z_i^*\,.    
    \label{eq:def:TM}
\end{equation}
That is, $\mbox{\normalfont Adj}(\lambda_i - A)$ is a re-scaling of the orthogonal projection on the characteristic (or invariant) subspace associated with $\lambda_i$.   Denote by $n_i$ the (algebraic) multiplicity of the eigenvalue $\lambda_i$, $1\leq i\leq m$. 
It is also known that the matrix
\[
    \frac{d^{n_i-1}  \mbox{\normalfont Adj}(z - A)}{dz^{n_i-1}}\Bigr|_{\lambda_i}
\]
span the characteristic subspace associated with $\lambda_i$. 
See \cite{1parisse:hal-00003444} and references therein. In this short note,  we extend and unify these two observations. In particular, we extend 
\eqref{eq:def:TM} to the case of repeated eigenvalues and general matrices. We obtain the following factorization, 
\[
    \frac{d^{n_i-1}  \mbox{\normalfont Adj}(z - A)}{dz^{n_i-1}}\Bigr|_{\lambda_i} = (n_i-1)! \, \textbf{   } 
    \prod_{j\not =i }^m (N_i+\lambda_i -\lambda_j)^{n_j} P_i\,,
\]
Here $N_i$ and $P_i$ are present in the Jordan form. In particular, the $n\times n$ matrix $N_i$ is a nilpotent matrix of degree less or equal to $n_i$ ($N_i^{n_i}=0 $), and $P_i$ is the oblique projection on the characteristic subspace associated with $\lambda_i$, that is, $P_i^2=P$ and the space  spanned by the columns of $P_i$ coincides with the characteristic subspace. In the cases of diagonalizable matrices, we have $N_i=0$ and recover a generalization of the result of Thompson and McEnteggert that includes the case of repeated eigenvalues. In the general case 
\[
   N_i=N_i P_i = \frac{1}{(n_i-2)!}\frac{d^{n_i-2} }{dz^{n_i-2}} \left( \frac{\mbox{\normalfont Adj}(z - A)}{\prod_{j\not =i }^m (z-\lambda_j)^{n_j}}\right)\Bigr|_{\lambda_i}\,
\]
is a nilpotent matrix of degree less or equal to $n_i$ that is related to the action of $A$ restricted to the invariant subspace associated with 
the eigenvalue $\lambda_i$. Additional related results are also discussed, among them, we prove that for $0\leq s<n_i-1$ we have
    \[
    \frac{d^{s} \mbox{\normalfont Adj}(z - A)}{dz^{s}}\Bigr|_{\lambda_i} = s! N_i^{n_i-1-s}\, \textbf{   } \prod_{j\not =i }^m (N_i+\lambda_i-\lambda_j)^{n_j} P_i.
    \]
Note that, for each $i$, the novel identities above relate the eigenvalues of $A$, $\lambda_j$, the eigenvectors (since $P_i$ span the characteristic spaces associated with $\lambda_i$), and the derivatives of the co-factors evaluated at the eigenvalue $\lambda_i$. Therefore these identities can be interpreted as generalized \emph{eigenvector-eigenvalue identities}. See  \cite{denton2022eigenvectors}  for a survey 
of eigenvector-eigenvalue identities. We observe that the eigenvector-eigenvalue identities in \cite{denton2022eigenvectors} can be obtained, among different methods, using the  {Thompson and McEnteggert} formula  \eqref{eq:def:TM}.



The rest of the manuscript is organized as follows. In Section \ref{sec:prelim} we present some background material and discuss some related previous work. In Section \ref{sec:derivatives} we present our main results in a detailed manner. In Section \ref{sec:examples} we show some illustrative particular examples and in Section \ref{sec:conclusions} we close the manuscript with some final comments and conclusions.


\section{Preliminaries and related works}\label{sec:prelim}
Let $A$ be a $n\times n$ complex matrix.
%$A:\mathbb{C}^n\xrightarrow{} \mathbb{C}^n$, 
$\Gamma \subset \mathbb{C}$ a positively oriented rectifiable Jordan curve (\cite[chapter 3,pp 68]{ahlfors1953complex},) that does not contain eigenvalues of $A$ and encloses a region $D$. Then, the \textit{Riesz projector} is defined as, see \cite[chapter 1,pp 3-6]{gohberg1978introduction}
\begin{align}
    P_\Gamma &= \frac{1}{2\pi i} \oint\limits_{\Gamma} (z-A)^{-1} \, \mathrm{d}z\, .
    \label{eq:RieszProjector}
\end{align}
When $D$ contains only one eigenvalue of $A$, say $\lambda_i$, the Riesz projector is denoted by $P_{\lambda_i}$ or just by $P_i$ if there is no ambiguity. As a reminder, when $A$ is hermitian, $P_\lambda$ is an orthogonal projection onto the eigenspace associated with $\lambda$. The Riesz projector has many interesting properties and it can be defined  on more general Banach spaces, see \cite[chapter 11, pp 418-425]{riesz2012functional}\cite[pp 273--275]{kozlov1999differential}\cite[chapter 2, pp 63-64]{jeribi2015spectral} and references therein.\\


Another important integral that is used in this document is  the \textit{Cauchy} integral formulation for functions of  matrices.
Let $f(z)$ be an analytic function inside and on a closed contour $\Gamma$ which encloses all the spectrum of A, then \cite[chapter 6,pp 427,Theorem 6.2.28]{horn1994topics}\cite[chapter 9,528,529]{golub2013matrix}
\begin{align}
    f(A) = \displaystyle \frac{1}{2\pi i} \oint\limits_{\Gamma} f(z) (z-A)^{-1} \, \mathrm{d}z\,.
    \label{cauchy}
\end{align}
See also \cite[chapter 1, pp 2-4]{higham2008functions}. The matrix $(z-A)^{-1}$ (when exists) is known as the resolvent of $A$ and it is denoted as $R(z)=R(z,A)$, which is closely related to the adjugate of $z-A$. 
Observe also that $z-A$ denotes the matrix $ z\mathbb{1}-A\,.$\\

The \textit{adjugate} matrix  of $A$, denoted by  $\mbox{Adj}A$,  is the matrix with entries $(\mbox{Adj}A)_{ij} = (-1)^{i+j} M_{ji}(A)$, where $M_{ji}(A)$ is the $(j,i)$ minor of $A$ (that is, the determinant of the 
$(n-1)\times (n-1)$ submatrix formed by deleting the $j$-th row and the $i$-th column from A). The adjugate of a matrix satisfies the following relation (\cite[Theorem 2, pp 119]{grossman1994elementary})
\begin{align}
   A  \,(\mbox{Adj}A) = \mbox{det}(A) \mathbb{1}\,.
    \label{adjugateRelation}
\end{align}
Here  $\mathbb{1}$ denotes the Identity matrix. Computing the adjugate from its definition could be expensive because involves the calculation of $n^2$ determinants of order $n-1$. In \cite{stewart1998adjugate} it is discussed an algorithm to compute the adjugate of a matrix $A$ using \eqref{adjugateRelation}, even in some cases when $A^{-1}$ is ill-conditioned.\\ 


 Introduce the characteristic polynomial of $A$ by \begin{equation}\label{eq:charpol}
    p_A(z) = p(z) =  \mbox{det}(z-A) = z^n+\alpha_1z^{n-1}+\dots+\alpha_{n-1}z+\alpha_{n}.
 \end{equation}
 Then we have that when $z-A$ in invertible, it follows from \eqref{adjugateRelation} that
\begin{equation}
    B(z)\equiv \mbox{Adj}(z-A) = (z-A)^{-1} p(z)\,.
    \label{eq:Adjz}
\end{equation}
In the rest of the text $\mbox{Adj}(z-A)$ is denoted as $B(z)$ for short.\\

%


It is well known that  every square complex matrix $A$ admits a Jordan decomposition. When $A$ is diagonalizable, the Jordan decomposition reduces to diagonalization. In  particular, it appears in the solution of linearly coupled physical systems and it helps to find the normal coordinates of such diagonalizable systems \cite{moulton1952h,griffiths2018introduction,georgi1993physics,cohen1986quantum}.   On the other hand, if $A$ is not diagonalizable, intuitively speaking, the theorem says that $A$ is similar to a matrix that is almost diagonal.
There exist many applications and theoretical studies of the Jordan decomposition of a matrix; the Jordan decomposition theorem allows us to define  the  concept of {\it generalized eigenvectors} and leads to the concept of Jordan chains \cite[chapter 7, pp 426-430]{boyce2021elementary}, which is useful for solving systems of linear ordinary differential equations and computing functions of a matrix (e.g., $\mbox{exp}(A)$) \cite[chapter 6.6-6.7]{grossman1994elementary}\cite[chapter 1.2]{ higham2008functions}.

%In addition to that, the Jordan decomposition theorem has a deep connection with the structure theorem for finitely generated modules over a principal ideal domain [Tao blog], in fact, it can be deduced as one of its corollaries.\\


%https://terrytao.wordpress.com/2007/10/12/the-jordan-normal-form-and-the-euclidean-algorithm/


The Jordan canonical form is introduced commonly as follows. Let $A$  be a $n\times n$ complex matrix, let $\{\lambda_i\}_{1\leq i\leq m}$ denote the $m$ different eigenvalues of A with corresponding algebraic multiplicities $\{n_i\}_{1\leq i\leq m}$. Then, the characteristic polynomial is 
\[
p(z)=\prod_{i=1}^m (z-\lambda_i)^{n_i}.
\]
We can define $\{W_i\}_{1\leq i\leq m}$, the invariant sub-spaces $W_{\lambda_i}=W_i = \mbox{ker}(A-\lambda_i)^{n_i}$ and projections $P_i$ onto the spaces $W_i$, then exist a matrix $V$ such that $A = V J V^{-1}$, where $J$ is a block diagonal matrix, whose diagonal blocks are square and of the form (\cite{grossman1994elementary,horn1994topics})
\begin{align}
    J(\lambda)_{rr} &= 
  \begin{pmatrix}
    \lambda & 1 & 0 & 0 & ... & 0 & 0\\
     0 &  \lambda & 1 & 0 & ...& 0 & 0\\
     . & . & . & . & .& . & . \\
    . & . & . & . & .& . & . \\
    0 &  0 & 0 & 0 & ... & \lambda & 1 \\
    0 &  0 & 0 & 0 & ...& 0 & \lambda
  \end{pmatrix}\, ,
  \label{eq:Pmatrix}
\end{align}
for $1\leq r\leq k$ (in general, $k$ depends on $A$), $\lambda$ is an eigenvalue of $A$ and the matrix $V$ contains the generalized eigenvectors associated with the corresponding Jordan blocks, which in turn belong to the corresponding spaces $W_{\lambda}$. There may be more than one Jordan block associated with the same eigenvalue.\\
Specifically, the matrix $J$ is a matrix with $k$ diagonal Jordan blocks $J(\lambda_{j})_{r,r}$ of size $s_r\times s_r$
\begin{align}
    J &=
  \begin{pmatrix}
    J(\lambda_{j_1})_{1,1} & 0 & 0 & 0 & ... & 0 & 0\\
     0 &  J(\lambda_{j_2})_{2,2} & 0 & 0 & ...& 0 & 0\\
     . & . & . & . & .& . & . \\
    . & . & . & . & .& . & . \\
    0 &  0 & 0 & 0 & ... & J(\lambda_{j_{k-1}})_{k-1,k-1} & 0 \\
    0 &  0 & 0 & 0 & ...& 0 & J(\lambda_{j_{k}})_{k,k}
  \end{pmatrix}\,.
  \label{eq:JordanJmatrix}
\end{align}
Let $T_i(\lambda_{j})_{r,r} = \delta_{\lambda_i,\lambda_{j}} \mathbb{1}_{n_r\times n_r}$, where $\delta_{\lambda_i,\lambda_{j}} $ is the Kronecker delta.

Let us define the matrix $I_{\lambda_i}$ of the same size of $J$ as
\begin{align}
    \mathcal{I}_{\lambda_i} &\equiv 
  \begin{pmatrix}
    T_i(\lambda_{j_1})_{1,1} & 0 & 0 & 0 & ... & 0 & 0\\
     0 &  T_i(\lambda_{j_2})_{2,2} & 0 & 0 & ...& 0 & 0\\
     . & . & . & . & .& . & . \\
    . & . & . & . & .& . & . \\
    0 &  0 & 0 & 0 & ... & T_i(\lambda_{j_{k-1}})_{k-1,k-1} & 0 \\
    0 &  0 & 0 & 0 & ...& 0 & T_i(\lambda_{j_{k}})_{k,k}
  \end{pmatrix}\,,
  \label{eq:IImatrix}
\end{align}
that is, $\mathcal{I}_{\lambda_i}$ is a sparse matrix, where only the diagonal indexes corresponding to the Jordan vectors associated with $\lambda_i$ are ones. Then, the projection $P_{\lambda_i} \equiv P_{\lambda_i}$ onto the subspace of the all Jordan chains related to $\lambda$ is
\begin{align}
    P_i &= V\mathcal{I}_{\lambda_i}V^{-1}\,.
    \label{eq:RelationProjectV}
\end{align}
Now, if we denote $S_i(\lambda_{j})_{r,r} \equiv \delta_{\lambda_i,\lambda_{j}} \left(J(\lambda_{j})_{r,r}-\lambda_j\mathbb{1}_{n_r\times n_r}\right)$, then we can express the nilpotent part $N_i$ as $V \hat{N}_i V^{-1}$, where $\hat{N}_i$ is
\begin{align}
    \hat{N}_i &\equiv 
  \begin{pmatrix}
    S_i(\lambda_{j_1})_{1,1} & 0 & 0 & 0 & ... & 0 & 0\\
     0 &  S_i(\lambda_{j_2})_{2,2} & 0 & 0 & ...& 0 & 0\\
     . & . & . & . & .& . & . \\
    . & . & . & . & .& . & . \\
    0 &  0 & 0 & 0 & ... & S_i(\lambda_{j_{k-1}})_{k-1,k-1} & 0 \\
    0 &  0 & 0 & 0 & ...& 0 & S_i(\lambda_{j_{k}})_{k,k}
  \end{pmatrix}\,,
  \label{eq:IImatrix}
\end{align}
That is, $\hat{N}_i$ is a sparse matrix, where only the entries of the upper diagonal associated with the characteristic space $\lambda_i$ are present and coincide with those of  $J$. Thus the  \textit{Jordan matrix decomposition theorem} can be formulated as follows. 

\begin{theorem}\label{thm:TheoJordan}
Given a $n\times n$ complex matrix $A$, let $\{\lambda_i\}_{1\leq i\leq m}$,$\{n_i\}_{1\leq i\leq m}$, $\{W_i\}_{1\leq i\leq m}$ and $\{P_i\}_{1\leq i\leq m}$ be defined as above, then $\mathbb{C}^n = \bigoplus W_i$ and there exist $N_i$, nilpotent operators of degree $d_i \leq n_i$, $1\leq i\leq m$,  such that $P_i N_i 
 = N_i P_i$ for all $i$ (we write the commutator $[N_i,P_i]=0$, for short), we have
\begin{equation}    \label{TheoJordan}
A = \sum_{i=1}^m \left( N_i + \lambda_i \right) P_i
\end{equation}
and the projectors satisfy $ \sum_{i=1}^m P_i = \mathbb{1}$ and 
\begin{align} 
    P_i P_j &= \delta_{ij} P_i\quad \text{for all $1\leq i,j\leq m$}.
\end{align}
\end{theorem}

If $A$ is a hermitian matrix, we then have $N_i=0$ for $1\leq i\leq m$. The interested reader can refer to \cite{PolRingroman2005advanced,PolRingbrown1993matrices,PolRinglezamacuadernos} where the  theory of canonical forms is presented in a more general context considering $F$-vector spaces, where $F$ is any field. We also refer to \cite{Daniel, galantai2013projectors} for properties of the projections $P_i$.\\



Since the set of all diagonalizable complex matrices is dense in the set of all complex matrices, small perturbation in $A$ may cause large changes in $J$, see for example \cite[354]{golub2013matrix}. This issue has inspired authors to search for algorithms capable of computing the matrix $P$ and the blocks $J_{r,r}$ appearing in the Jordan normal form, see \cite{10.2307/2029711}. For instance,  in \cite{4cai1994computing} it is shown that the Jordan normal form of a rational matrix can be computed in polynomial time, other algorithms can be found in \cite{2li1997determining,3fletcher1983algorithmic}.






In \cite{1parisse:hal-00003444}, B. Pairisse and M.Vaughan express 
\begin{equation}\label{PV1aroundzero}
B(z) = z^{n-1}+z^{n-2}C_1+\dots+zC_{n-2}+C_{n-1}    
\end{equation}
and find a method  based on the Faddev Algorithm in order to compute $C_\ell$, $1\leq \ell\leq n-1$, using the recurrence relation
\begin{equation}\label{eq:recurrece}
C_0=\mathbb{1}, \quad C_\ell = AC_{\ell-1} -\frac{1}{\ell}\mbox{tr}(AC_{\ell-1})\mathbb{1}\quad \mbox{ for } \ell=1,2,\dots,n-1.
\end{equation}
Note that $
   \frac{d^{k} B(z)}{dz^{k}}|_{0} =k!C_{n-k-1}.$
A byproduct of this recurrence relation is the coefficients of the characteristic polynomial of $A$ in \eqref{eq:charpol}, that is, 
\begin{equation}
    \alpha_{\ell}=-\frac{1}{\ell}\mbox{tr}(AC_{\ell-1}), \quad \ell=1,2,\dots,n-1.
\end{equation}
They also write the expansion centered around a given eigenvalue, say $\lambda_i$, 
\begin{equation}\label{PV1}
B(z) = \sum_{0\leq k \leq n-1} B_k(\lambda_i) (z-\lambda_i)^k.    
\end{equation}
Note that 
 $$
   \frac{d^{n_i-1} B(z)}{dz^{n_i-1}}\Bigr|_{\lambda_i} =(n_i-1)!B_{n_i-1}(\lambda_i).
 $$
Concerning this expansion they show that $\mbox{Span}(B_{n_i-1})$ is the characteristic space associated with $\lambda_i$, i.e., the subspace $W_i$ in our notation; see \cite[Theorem 2]{1parisse:hal-00003444}. Additionally, they propose an algorithm to compute the Jordan canonical  form that we summarize here as follows. Given a matrix $A$, proceed as follows:
\begin{enumerate}
    \item Compute the matrices $C_1,C_2,\dots,C_{n-1}$ and the coefficients of the characteristic polynomial $\alpha_1,\alpha_2,\dots,\alpha_{m}$ using the Faddev Algorithm for matrix $A$ (see \eqref{eq:recurrece}).
    \item Compute the matrices $B_k(\lambda_i)$, $0\leq k\leq n-1$ and $1\leq i\leq n$. This could be done using the Horner division of $B(z)$ (and its derivatives) by $z-\lambda_i$. Another alternative is to compute them directly from the Faddev algorithm for $A-\lambda_i$.
    \item Compute the corresponding eigenvectors and generalized eigenvectors from the matrices $B_k(\lambda_i)$.
\end{enumerate}
This last step requires joint and careful columnwise Gauss elimination  for all the matrices $B_k(\lambda_i)$ that preserves the following structure, 
\begin{eqnarray*}
    (A-\lambda\mathbb{1}) B(\lambda_i)&=&0  \\
    (A-\lambda\mathbb{1}) B_1(\lambda_i)&=&B(\lambda_i) \\
    \vdots\\
 (A-\lambda\mathbb{1}) B_{n_i-1}(\lambda_i)&=&B_{n_i-2}(\lambda_i)\\  
 (A-\lambda\mathbb{1}) B_{n_i}(\lambda_i)-B_{n_i-1}\lambda_i)&=&
 -\prod_{j\not= i} (\lambda_i-\lambda_j)^{n_i}\mathbb{1}.\\
\end{eqnarray*}
 
In this manuscript, we present a different proof of some of these results using the Riesz projector. Furthermore, it is also found that $B_{n _i-1}$ is proportional to the projection to the characteristic space $W_i$. Additionally, the product $N_iP_i$ is expressed in terms of the derivatives of $B$, that is, it is provided a matrix representation  of the nilpotent matrix of the Theorem \ref{thm:TheoJordan}. As a related result, we mention that, recently, in \cite{denton2023eigenvectors} M. Franchi has shown some relations between the Riesz projection and the Jordan structure of a matrix.\\



\section{Higer order derivatives of the adjugate matrix}\label{sec:derivatives}
%%%%%%%%%%%%%%%%%%%%%%%%%%%%%

In order to prove our main results, first, based on formula \eqref{eq:Adjz}, let us write $(A-z)^{-1}$ is in terms of the matrices $P_i$ and $N_i$, $1\leq i \leq m$. For this write the following general algebraic lemma.

\begin{lemma}
Let $\{X_i\}_i$ and  $\{Y_i\}_i$ be families of matrices such that $[X_i,Y_i] = \mathbf{0}$ and $X_i$ is non singular, for all i.

Assume that $\sum_j Y_j = \mathbb{1}$ and that for all $i,j$ $Y_i Y_j = \delta_{ij}Y_i$. Then if $X=\sum X_i Y_i$  we can write $X^{-1} = \sum X_i^{-1} Y_i.$    
\label{lemma:LemmaInversa}
\end{lemma}
\begin{proof} We show this by direct computations as follows,
\begin{align*}
    X \sum_j X_j^{-1} Y_j &= \sum_i X_i Y_i  \sum_j X_j^{-1} Y_j\,, & & \\
    &= \sum_i \sum_j X_i Y_i X_j^{-1} Y_j\,, & & \\
    &= \sum_i \sum_j X_i Y_i Y_j X_j^{-1}\,, & & \text{because $[X_i,Y_i]=0$, hence $[X_i^{-1},Y_i]=0$,}\\
    &= \sum_i \sum_j X_i \delta_{ij} Y_i X_j^{-1}\,,& & \text{because $Y_i Y_j =\delta_{i,j}Y_i$,}\\
    &= \sum_i Y_i X_i \sum_j \delta_{ij} X_j^{-1}\,,& & \text{because $[X_i,Y_i]=0$,}\\
    &= \sum_i Y_i X_i X_i^{-1}\,,\\
    &= \sum_i Y_i\,,& &\\
    &= \mathbb{1}\,.& & 
\end{align*}
 \end{proof}
 %%%%
 %%%
 %TEXT HABLAR SOBRE FUNIONES DE MATRICES
 %RELAICONAR CON LA FOMRULA INTEGRAL DE CACUHY DE ARRIBA
 %tipo .. NOte que si la integral se toma bajo un simple curve que enciierra todos los gamma_i, se reduce a la formula integral de cauchy y por lo tanto ....

We still need some additional results. The following result relates the complex Cauchy integral of $f(z) R(z)$ with the matrices $N_i$ and $P_i$. Recall that analytic functions evaluated on the matrix can be defined with the integral form of Cauchy \eqref{cauchy}; see \cite{higham2008functions}. Similar results are obtained when the contour $\Gamma$ does not enclose the entire spectrum, as if the function were restricted to the subspaces $W_i$ of the eigenvalues enclosed by $\Gamma$. 
 
\begin{theorem}
     Let $f(z)$ be an analytic function and $\Gamma_r \subset \mathbf{C}$ a positively oriented rectifiable Jordan curve that does not contain eigenvalues of $A$ and encloses a region $D_r$ containing only one eigenvalue $\lambda_r$ of $A$. Then
     \begin{align*}
     \displaystyle \frac{1}{2\pi i} \oint\limits_{\Gamma_r} f(z) (z-A)^{-1} \, \mathrm{d}z =  f(\lambda_r + N_r) P_r\,.
     \end{align*}
     \label{Theo:CauchyProjection}
\end{theorem}
\begin{proof}
First note that if $z$ does not belong to the spectrum of $A$, then
%\sum_i z P_i + \sum_i\left( -\lambda_i - N_i\right) P_i
\begin{align*}
    z-A &= z-\sum_i \left( \lambda_i + N_i\right) P_i\, ,& & \text{By \eqref{TheoJordan},}\\
    z-A &= \sum_i z P_i + \sum_i \left( -\lambda_i - N_i\right) P_i\, ,& & \text{because $\mathbb{1}= \sum_i P_i$ according to theorem \ref{thm:TheoJordan},}\\
    z-A &= \sum_i \left( (z-\lambda_i)-N_i \right) P_i\, ,& & \text{ }\\
    z-A &= \sum_i (z-\lambda_i) \left( \mathbb{1}-\frac{1}{(z-\lambda_i)}N_i \right) P_i\, ,& & \text{}\\
    (z-A)^{-1} &= \sum_i \frac{1}{(z-\lambda_i)}
   \left( \sum_{l=0}^{n_i-1} \frac{N_i^l}{(z-\lambda_i)^l} \right) P_i\, ,& & \text{Applying Lemma \ref{lemma:LemmaInversa} with $X_i = (z-\lambda_i) \left( \mathbb{1}-\frac{1}{(z-\lambda_i)}N_i\right)$. }
   \label{eq:inverseZmA}
\end{align*}
In the last step, since $N_i$ is nilpotente, then $\mathbb{1}-\frac{1}{(z-\lambda_i)}N_i$ is invertible and its inverse is $\sum_{l=0}^{n_i-1} \frac{N_i^l}{(z-\lambda_i)^l}$, since $n_i$ is greater or equal than the degree of $N_i$.\\
%%%%
%%%%%%%%%%
Then multiplying by $f(z)$, integrating over $\Gamma_r$ and applying the Cauchy formula, then:
\begin{align*}
    \frac{1}{2\pi i} \oint\limits_{\Gamma_r} f(z) (z-A)^{-1} \, \mathrm{d}z &=  \sum_i \sum_{l=0}^{n_i-1} 
 \left( \frac{1}{2\pi i} \oint\limits_{\Gamma_r} \frac{f(z)}{(z-\lambda_i)^{l+1}} \, \mathrm{d}z  \right) N_i^l P_i\, ,& & & \\
  &=  \sum_i \sum_{l=0}^{n_i-1} 
 \delta_{r,i} \frac{f^{(l)}(\lambda_i)}{l!} N_i^l P_i\, ,& & \text{since if $r\neq i$, 
 $\frac{f(z)}{(z-\lambda_i)^{l+1}}$ is analytical in $D_r$,} & \\
 &=  \sum_{l=0}^{n_r-1} \frac{f^{(l)}(\lambda_r)}{l!}N_r^l P_r\, ,& & & \\
 &=  \sum_{l=0}^{\infty} \frac{f^{(l)}(\lambda_r)}{l!}N_r^l P_r\, ,& & \text{because $N_r^l = 0$ for $l\geq n_r$,} & \\
 &= f(\lambda_r + N_r) P_r\, .& &  &
\end{align*}
\end{proof}
%%%%%%%%%%%
As a remarkable fact, note that for $0<l$, $N_r^l P_r = (N_r P_r)^l$, because $[P_r,N_r]=0$ and $P_r^l = P_r$ given that $P_r$ is a projection. Then, $f(\lambda_r + N_r) P_r$ can be expressed as a function of $\lambda$,$P_r$ and $N_r P_r$.


  Note that with these results and  \eqref{cauchy}, we have that
\[
    f(A) = \sum_i f(\lambda_i + N_i) P_i\, ,
\]
and 
\[
   f(A)P_r = f(\lambda_r + N_r) P_r\, .
\]

Before stating some results, let us define 
\begin{equation}
q_i(z) = p(z)/(z-\lambda_i)^{n_i} =\prod_{j\not =i }^m (z-\lambda_j)^{n_j}\,,
\label{eq:def:qi}
\end{equation}
for $1\leq i\leq m$, i.e, $q_i$ is the multiplication of the other factors of $p$ with $q_i(\lambda_i) \neq 0 $. The previous theorem can be applied taking into account the equation \eqref{eq:Adjz}, and we can obtain the following result.


\begin{theorem} \label{thmmain} Let $B(z)$ be defined in \eqref{eq:Adjz} and
$q_i$ be defined in \eqref{eq:def:qi}. Then, for $0\leq s<n_i-1$
    \begin{equation}
    \frac{d^{s} B(z)}{dz^{s}}\Bigr|_{\lambda_i} = s! N_i^{n_i-1-s}\, \textbf{   } q_i\left(N_i + \lambda_i\right)  P_i.
        \label{The:BzeqPbefore}
    \end{equation}
In particular, taking $s=n_i-1$, we have 
    \begin{equation}
    \frac{d^{n_i-1} B(z)}{dz^{n_i-1}}\Bigr|_{\lambda_i} = (n_i-1)! \, \textbf{   } q_i\left(N_i + \lambda_i\right)  P_i.\label{The:BzeqP}
    \end{equation}
\label{The:Bzmain}
\end{theorem}
\begin{proof}
We know that 
\begin{equation}
 B(z) = p(z) (z-A)^{-1}= (z-\lambda_i)^{n_i} q_i(z) (z-A)^{-1}  
 \label{eq:step}
\end{equation}
and therefore,
\begin{align}
    \frac{B(z)}{(z-\lambda_i)^{s+1}} =(z-\lambda_i)^{n_i-1-s} q_i(z) (z-A)^{-1}.
\end{align}
Since $B(z) = \mbox{Adj}(z-A)$, each one of its entries is a cofactor of $z-A$ and hence $B(z)$ is an analytic function (see also \eqref{PV1aroundzero}). Then, applying the Cauchy formula to $B(z)$ and the Theorem \ref{Theo:CauchyProjection} to the right hand side, we get
    \begin{equation}
    \frac{d^{s} B(z)}{dz^{s}}\Bigr|_{\lambda_i} = s! N_i^{n_i-1-s}\, \textbf{   } q_i\left(N_i + \lambda_i\right)  P_i.
        \label{eq:Derivativesbefore}
    \end{equation}
%\begin{align}
%  \frac{1}{(n_i-1)!}\frac{d^{n_i-1} B(z)}{dz^{n_i-1}}\Bigr|_{\lambda_i} = q_i\left(N_i + \lambda_i\right)  P_i\,.
%  \label{eq:Derivatives}
%\end{align}
\end{proof}





Recall that if $A$ is diagonalizable (e.g., hermitian matrices) then 
$N_i=0$ for $1\leq i\leq m$.
%Then if we divide in  \eqref{eq:step} by $(z-\lambda_i)^{s+1}$ with $0 \leq s < n_i-1$, we get
%\begin{align*}
%    \frac{B(z)}{(z-\lambda_i)^{s+1}} &= (z-\lambda_i)^{n_i-1-s} q_i(z) (z-A)^{-1}.
%\end{align*}
%Therefore, using and argument similar to the one before we obtain,
%\[
 %   \frac{1}{s!}\frac{d^{s} B(z)}{dz^{s}}\Bigr|_{\lambda_i} = (\lambda_i-\lambda_i)^{n_i-1-s} q_i\left(\lambda_i\right)  P_i = 0 \,.
%\]
%We summarize this previous result in the following theorem.
We have the following result.
\begin{corollary}
Let $A$ a diagonalizable square matrix, $B(z) = \mbox{Ajd}(z-A)$, then for every integer $0\leq s < n_i-1 $, we have
    \begin{align*}  \label{The:TMcRecover}
    \frac{d^{s} B(z)}{dz^{s}}\Bigr|_{\lambda_i} &= 0\,.
    \end{align*}
We also have
    \begin{equation}
    \frac{d^{n_i-1} B(z)}{dz^{n_i-1}}\Bigr|_{\lambda_i} = (n_i-1)! \, \textbf{   } q_i\left(\lambda_i\right)  P_i.
    \end{equation}
In particular, if $A$ is hermitian and $\lambda_i$ is a simple eigenvalue, with a unit eigenvector $z_i$, then
\begin{align*}
    B(z)&=\frac{d p(z)}{dz}\Bigr|_{\lambda_i} z_i z_i^*\,.
\end{align*}
\end{corollary}



%In the next section, it will be shown in Theorems \ref{The:BzeqP} and \ref{The:BzeqNP} of this paper (with the same notation as above in \eqref{eq:Adjz} and \eqref{TheoJordan}), that
%\begin{align}
%    \frac{d^{n_i-1} B(z)}{dz^{n_i-1}}\Bigr|_{\lambda_i} &= (n_i-1)! \, \textbf{   } q_i\left(N_i + \lambda_i\right)  P_i\,,\\
%    N_i P_i &= \frac{1}{(n_i-2)!}\frac{d^{n_i-2} }{dz^{n_i-2}} \left( \frac{B(z)}{q_i(z)}\right)\Bigr|_{\lambda_i}\,,\quad \text{if $n_i>1$}\,.
%    \label{main_results}
%\end{align}
%That is, the derivatives of the adjugate matrix evaluated at the eigenvalue $\lambda_i$ is related to the projection on the subspace $W_i$ and with the respective nilpotent matrices $N_i$.\\


%Note that if $A$ is hermitian, then $N_i = 0$ and our result says that the projection on the respective eigenspace is a multiple of the $(n_i-1) - th$ derivative of $B(z)$ evaluated at the eigenvalue. Moreover, for every integer $0\leq s < n_i-1 $, we have 
 %$\frac{d^{s} B(z)}{dz^{s}}\Bigr|_{\lambda_i} = 0$.\\

 
%\textcolor{red}{In general the form of $P$ depends on the representation of $A$, that is, the entries $P_{ij}$ depend of the entries of $A$, thus, $P_{ij}$ depends on the basis chosen to represent the operator $A$, see example below  ....\tt EXPLAIN THIS MORE PRECISELY}.
%Aqui estaba pensando en operadores, y su representacion elegida una base, pero si la matriz ya se nos es dada, no viene la caso hablar de ello.



As before, for $A$ hermitian and $\lambda_i$  simple, then $n_i = 1$, $N_i = 0$ and $\mbox{rank}(P_i) = 1$, hence $P_i$ has a matrix representation of the operator $z_i z_i^*$, and the formula is reduced to the one of the Thompson and McEnteggert Theorems stated next.
%
%
%
\begin{theorem}[Thompson and McEnteggert]
Suppose that $A$ is a hermitian matrix and $\lambda_i$ a simple eigenvalue with unit eigenvector $z_i$, then
\begin{align}
    \mbox{\normalfont Adj}(\lambda_i - A) = \frac{d p(z)}{dz}\Bigr|_{\lambda_i} z_i z_i^*\,.
\end{align}
\label{eq:TheoremTM}
\end{theorem}
For this result, we refer to \cite{iserles2002acta,parlett1998symmetric,thompson1968principal}.
See also \cite{Castillo2021OnAF} for a generalization of this identity to include any matrix with entries in an arbitrary field that is stated as follows. Let $\lambda_i$ be an eigenvalue of a  matrix $A$ with right and left eigenvectors $v_i$ and $z_i$. Then
\begin{equation}
\mbox{\normalfont Adj}(\lambda_i - A) = \frac{d p(z)}{dz}\Bigr|_{\lambda_i}\frac{1}{z^*_iv_i}v_i z_i^*\,.
\end{equation} 
Note that $\frac{1}{z^*_iv_i}v_i z_i^*$ is an oblique projection on $z_i$ in the direction orthogonal to $v_i$. 
Note that Theorem \ref{The:Bzmain} includes this case since $n_i=1$ implies $N_i=0$ and $\mbox{rank}(P_i)=1$. In this case, $P_i$ is an oblique projection in the space generated by the associated eigenvector $v_i$.  Statements of results hold in a more general algebraic setting (as in \cite{Castillo2021OnAF}) and they will be presented elsewhere.


% $q_i(N_i+\lambda_i)P_i$ can be considered as a operator form $W_i$ onto $W_i$, because
% \begin{align*}
%     q_i(N_i+\lambda_i)P_i = q_i(A)P_i = \Pi_{j\neq i} (A- \lambda_j)^i P_i = \Sigma a_r (A-\lambda_i)^r
% \end{align*}
% with $a_0 \neq 0$ because $\lambda_i$ is not a root of $q_i$. Since $W_i$ is invariant under $(A-\lambda_i)$, then so under $\Sigma a_r (A-\lambda_i)^r = q_i(N_i+\lambda_i)$. Now, it is invertible (as an operator form $W_i$ to $W_i$), because by the Jordan theorem, $q_(A) = V q(J) V^{-1}$, since $J$ is an upper triangular matrix, then $q(J)$ is an upper triangular matrix where its diagonal entries are $q(J)_{t,t} = q(J_{t,t})$. When projector $P_i$ acts on $q_(A)$, then just remains the Jordan blocks associated with $\lambda_i$, therefore $q_(A)P_i=P_iq_(A)P_i$ (seen as an operator from $W_i$ to $W_i$) is an upper triangular matrix, where its diagonal entries all are $q(\lambda_i)\neq 0$, hence, it is invertible. 
% 
% Now, from \ref{The:BzeqP}, it is deduced that $span(B^{(n_i-1)}(\lambda_i)) \subset W_i$, because $B^{(n_i-1)}(\lambda_i)$ is the multiplication of $P_i$ and other matrix. Now, if $y \in W_i$, then, since $q_i(A_i)=q_i(N_i+\lambda_i)P_i$is invertible as an operator from $W_i$ to $W_i$, then, it does exist $x\in W_i$ such that $q_i(A_i)x = y$, then $B^{(n_i-1}(\lambda_i) x = q_i(A_i)x = y$, hence $W_i \subset span(B^{(n_i-1}(\lambda_i))$ and therefore $W_i = span(B^{(n_i-1}(\lambda_i))$, thus we recover the result of \cite{1parisse:hal-00003444}.



To finish this section we express $P_i$ and $N_i$ in terms of the derivatives of $B(z)$ and $q_i(z)$. Considering $f(z) = 1$ and applying Theorem \ref{Theo:CauchyProjection}, we have
\begin{align*}
    P_i=\frac{1}{2\pi i} \oint\limits_{\Gamma_i} (z-A)^{-1} \, \mathrm{d}z = \frac{1}{2\pi i} \oint\limits_{\Gamma_i} \frac{B(z)}{(z-\lambda_i)^{n_i} q_i(z)} \, =\frac{1}{(n_i-1)!}\frac{d^{n_i-1} }{dz^{n_i-1}} \left( \frac{B(z)}{q_i(z)}\right)\Bigr|_{\lambda_i}\,.
\end{align*}
If $n_i = 1$, then $N_i = 0$. However, if $n_i\geq 2$, taking $f(z) = z-\lambda_i$ and applying Theorem \ref{Theo:CauchyProjection}, then,
\begin{align*}
    N_i P_i = \frac{1}{2\pi i} \oint\limits_{\Gamma_i} (z-\lambda_i) (z-A)^{-1} \, \mathrm{d}z &= \frac{1}{2\pi i} \oint\limits_{\Gamma_i} \frac{B(z)}{(z-\lambda_i)^{n_i-1} q_i(z)}=\frac{1}{(n_i-2)!}\frac{d^{n_i-2} }{dz^{n_i-2}} \left( \frac{B(z)}{q_i(z)}\right)\Bigr|_{\lambda_i}\,.
\end{align*}
Hence we have the following result.

\begin{theorem} \label{thmmain} Let $B(z)$ be defined in \eqref{eq:Adjz} and
$q_i$ be defined in \eqref{eq:def:qi}. Then
\begin{equation}
    P_i=\frac{1}{(n_i-1)!}\frac{d^{n_i-1} }{dz^{n_i-1}} \left( \frac{B(z)}{q_i(z)}\right)\Bigr|_{\lambda_i}\,.
    \label{The:BzQzeqP}
\end{equation}
Moreover, if $n_i\geq2$, then
\begin{align}
    N_i P_i &= \frac{1}{(n_i-2)!}\frac{d^{n_i-2} }{dz^{n_i-2}} \left( \frac{B(z)}{q_i(z)}\right)\Bigr|_{\lambda_i}\,.
    \label{The:BzeqNPeq}
\end{align}
\end{theorem}
Therefore, one can recover the matrices $P_i$ and $N_i P_i$ from the derivatives of cofactors terms. Therefore we  can compute functions $f(A)$ with finite polynomials because of the fact that $N_i P_i$ is nilpotent.







%
\section{Illustrative examples}\label{sec:examples}
In this section, we present particular illustrative examples to help the reader to fix ideas related to the results presented in the previous section. 
\subsection{Example 1}
Let A be the following hermitian matrix
\begin{align}
    A &=  
\begin{pmatrix}
  \begin{matrix}
    1 & -1 & 1\\
    -1 & 1 & -1 \\
    1 & -1 & 1
  \end{matrix}
\end{pmatrix}\,.
\end{align}
The characteristic polynomial is $p(z) = z^2 (z-3)$. Then the eigenspace related with $\lambda = 0$  has dimension 2 and the eigenspace related with $\lambda = 3$ has dimension 1. The adjugate matrix is 
$$B(z)=\left(\begin{matrix}z^{2} - 2 z & - z & z\\- z & z^{2} - 2 z & - z\\z & - z & z^{2} - 2 z\end{matrix}\right)$$ and 
its  derivative is written as
$$
B^{\prime}(z)=
\left(\begin{matrix}2 z - 2 & -1 & 1\\-1 & 2 z - 2 & -1\\1 & -1 & 2 z - 2\end{matrix}\right).
$$
Evaluating $B^{\prime}(z)$ at $z=0$ we get
$$
    P_0 =  
    \frac{1}{-3}B'(0)=
\begin{pmatrix}
    \frac{2}{3} & \frac{1}{3} & -\frac{1}{3}\\
    \frac{1}{3} & \frac{2}{3} & \frac{1}{3}\\
    -\frac{1}{3} & \frac{1}{3} & \frac{2}{3}
\end{pmatrix}
$$
and  evaluating $B(z)$ at $z = 3$
$$
P_3 = 
\frac{1}{3^2 \cdot 0!}B(3)=
\begin{pmatrix}
  \begin{matrix}
    \frac{1}{3} & -\frac{1}{3} & \frac{1}{3}\\
    -\frac{1}{3} & \frac{1}{3} & -\frac{1}{3}\\
    \frac{1}{3} & -\frac{1}{3} & \frac{1}{3}
  \end{matrix}
\end{pmatrix}\,.
$$
One can check the following relations
\[    P_0 + P_3 = \mathbb{1}\,, \quad  P_0 P_3 = P_3 P_0 = 0\,,  \quad   P_0^2 = P_0\,, \quad 
    P_3^2 = P_3\,,  \quad AP_0 = 0\,   \quad    AP_3 = 3 P_3    \quad \mbox{ and }  A = 3 P_3 + 0 P_0\,.
\]
These relations are basically the spectral theorem for this specific matrix. Note that $\mbox{Tr}(P_0) = 2$ and $\mbox{Tr}(P_3) = 1$, i.e, the dimensions of the eigenspaces, as it should be (the trace of a projection is the dimension of its range). See \cite{Daniel, galantai2013projectors}.

As a remark related to our results, we observe that the Jordan canonical form for this matrix is given by 
$$
    V =  
\begin{pmatrix}
  \begin{matrix}
    1 & -1 & 1\\
    1 & 0 & -1\\
    0 & 1 &1
  \end{matrix}
\end{pmatrix}
\quad \mbox{ and }  \quad
   J     =
\begin{pmatrix}
  \begin{matrix}
    0 & 0 & 0\\
    0 & 0 & 0\\
    0 & 0 &3
  \end{matrix}
\end{pmatrix}
$$
We verify with the eigenvectors corresponding to the first eigenvalues, $V_0= V(:,[1,2])$ that,
$P_0=V_0*(V_0^TV_0)^{-1}V_0^T$
and with $V_3=V(:,3)$ that 
$
P_3=V_3*(V_3^TV_3)^{-1}V_3^T.
$

\subsection{Example 2}
Consider now the following matrix
$$
   A 
    =
\begin{pmatrix}
  \begin{matrix}
    0 & 1 & 0 & 0\\\\11 & 6 & -4 & -4\\\\22 & 15 & -8 & -9\\\\-3 & -2 & 1 & 2
  \end{matrix}
\end{pmatrix}.
$$
The characteristic polynomial is $p(z) = \left(z - 1\right)^{2} \left(z + 1\right)^{2}$. The characteristic space related to $\lambda = 1$ has dimension 2, the same for $\lambda = -1$. The adjugate matrix is given by
$$
   B(z)
    =
\begin{pmatrix}
  \begin{matrix}
    z^{3} + 9 z - 10 & z^{2} + 6 z - 7 & 4 - 4 z & 4 - 4 z\\\\11 z^{2} - 10 z - 1 & z^{3} + 6 z^{2} - 7 z & - 4 z^{2} + 4 z & - 4 z^{2} + 4 z\\\\22 z^{2} + 16 z - 26 & 15 z^{2} + 10 z - 17 & z^{3} - 8 z^{2} - 7 z + 10 & - 9 z^{2} - 6 z + 11\\\\- 3 z^{2} - 6 z - 3 & - 2 z^{2} - 4 z - 2 & z^{2} + 2 z + 1 & z^{3} + 2 z^{2} + z
  \end{matrix}.
\end{pmatrix}
$$
Since the algebraic multiplicity is $2$ in both cases, we compute the first derivative of $B(z/q_1(z)$ that is given by, 

$$
   \left(\frac{B(z)}{q_1(z)}\right)^{\prime}
    =
\begin{pmatrix}
  \begin{matrix}
    \frac{- 2 z^{3} - 18 z + 3 \left(z + 1\right) \left(z^{2} + 3\right) + 20}{\left(z + 1\right)^{3}} & \frac{4 \left(5 - z\right)}{z^{3} + 3 z^{2} + 3 z + 1} & \frac{4 \left(z - 3\right)}{\left(z + 1\right)^{3}} & \frac{4 \left(z - 3\right)}{\left(z + 1\right)^{3}}\\\frac{8 \left(4 z - 1\right)}{z^{3} + 3 z^{2} + 3 z + 1} & \frac{z^{3} + 3 z^{2} + 19 z - 7}{z^{3} + 3 z^{2} + 3 z + 1} & \frac{4 \left(1 - 3 z\right)}{z^{3} + 3 z^{2} + 3 z + 1} & \frac{4 \left(1 - 3 z\right)}{z^{3} + 3 z^{2} + 3 z + 1}\\\frac{4 \left(7 z + 17\right)}{z^{3} + 3 z^{2} + 3 z + 1} & \frac{4 \left(5 z + 11\right)}{z^{3} + 3 z^{2} + 3 z + 1} & \frac{z^{3} + 3 z^{2} - 9 z - 27}{z^{3} + 3 z^{2} + 3 z + 1} & - \frac{12 z + 28}{z^{3} + 3 z^{2} + 3 z + 1}\\0 & 0 & 0 & 1
  \end{matrix}.
\end{pmatrix}
$$
Evaluating at $z=1$, we have
$$
   P_1 = \left(\frac{B(z)}{q_1(z)}\right)^{\prime}\Bigr|_{z=1}
    =
\begin{pmatrix}
  \begin{matrix}
    3 & 2 & -1 & -1\\\\3 & 2 & -1 & -1\\\\12 & 8 & -4 & -5\\\\0 & 0 & 0 & 1
  \end{matrix}
\end{pmatrix}.
$$
%%%%%%
%%%%%%
%%%%%%%55
%%%%%%%%%%%55
%%%%%%%%%%%%%%%%
%%%%%%%%%%%
%%%%%%%%%%
%
and 
$$
   \left(\frac{B(z)}{q_{-1}(z)}\right)^{\prime}
    =
\begin{pmatrix}
  \begin{matrix}
    \frac{z^{2} - 2 z - 11}{z^{2} - 2 z + 1} & - \frac{8}{\left(z - 1\right)^{2}} & \frac{4}{\left(z - 1\right)^{2}} & \frac{4}{\left(z - 1\right)^{2}}\\- \frac{12}{\left(z - 1\right)^{2}} & \frac{z^{2} - 2 z - 7}{z^{2} - 2 z + 1} & \frac{4}{\left(z - 1\right)^{2}} & \frac{4}{\left(z - 1\right)^{2}}\\\frac{12 \left(3 - 5 z\right)}{z^{3} - 3 z^{2} + 3 z - 1} & \frac{8 \left(3 - 5 z\right)}{z^{3} - 3 z^{2} + 3 z - 1} & \frac{z^{3} - 3 z^{2} + 23 z - 13}{z^{3} - 3 z^{2} + 3 z - 1} & \frac{8 \left(3 z - 2\right)}{z^{3} - 3 z^{2} + 3 z - 1}\\\frac{12 \left(z + 1\right)}{z^{3} - 3 z^{2} + 3 z - 1} & \frac{8 \left(z + 1\right)}{z^{3} - 3 z^{2} + 3 z - 1} & - \frac{4 z + 4}{z^{3} - 3 z^{2} + 3 z - 1} & \frac{z^{3} - 3 z^{2} - 5 z - 1}{z^{3} - 3 z^{2} + 3 z - 1}
  \end{matrix}
\end{pmatrix}.
$$
Evaluating at $z=-1$, we have
$$
   P_{-1} = \left(\frac{B(z)}{q_{-1}(z)}\right)^{\prime}\Bigr|_{z=-1}
    =
\begin{pmatrix}
  \begin{matrix}
    -2 & -2 & 1 & 1\\\\-3 & -1 & 1 & 1\\\\-12 & -8 & 5 & 5\\\\0 & 0 & 0 & 0
  \end{matrix}
\end{pmatrix}.
$$

%%% ME FALTA AÑADIR LOS N_i P_i

The Jordan canonical form for this matrix is given by 
$$
    V =  
\begin{pmatrix}
  \begin{matrix}
    \frac{1}{3} & - \frac{2}{3} & 0 & \frac{1}{4}\\- \frac{1}{3} & 1 & 0 & \frac{1}{4}\\\frac{1}{3} & 0 & \frac{1}{4} & 1\\0 & 0 & - \frac{1}{4} & 0
  \end{matrix}
\end{pmatrix}
\quad \mbox{ and } \quad 
   J 
    =
\begin{pmatrix}
  \begin{matrix}
    -1 & 1 & 0 & 0\\\\0 & -1 & 0 & 0\\\\0 & 0 & 1 & 1\\\\0 & 0 & 0 & 1
  \end{matrix}
\end{pmatrix}.
$$
Thus,
$$
   \mathcal{I}_1
    =
\begin{pmatrix}
  \begin{matrix}
    0 & 0 & 0 & 0\\\\0 & 0 & 0 & 0\\\\0 & 0 & 1 & 0\\\\0 & 0 & 0 & 1
  \end{matrix}
\end{pmatrix}
\quad 
\mbox{ and }
\quad    \mathcal{I}_{-1}    =
\begin{pmatrix}
  \begin{matrix}
    1 & 0 & 0 & 0\\\\0 & 1 & 0 & 0\\\\0 & 0 & 0 & 0\\\\0 & 0 & 0 & 0
  \end{matrix}
\end{pmatrix}
$$

We verify $
    P_1 = V \mathcal{I}_1 V^{-1}$ and $
    P_{-1} = V \mathcal{I}_{-1} V^{-1}.$

Analogously, we obtain
$N_iP_i =  \left( \frac{B(z)}{q_i(z)}\right)\Bigr|_{\lambda_i}$, that is, 

$$
   N_1 P_1
    =
\begin{pmatrix}
  \begin{matrix}
    0 & 0 & 0 & 0\\\\0 & 0 & 0 & 0\\\\3 & 2 & -1 & -1\\\\-3 & -2 & 1 & 1
  \end{matrix}
\end{pmatrix}
\quad 
\mbox{ and }
\quad 
   N_{-1} P_{-1}
    =
\begin{pmatrix}
  \begin{matrix}
    -5 & -3 & 2 & 2\\\\5 & 3 & -2 & -2\\\\-5 & -3 & 2 & 2\\\\0 & 0 & 0 & 0
  \end{matrix}
\end{pmatrix}.
$$
We have verified the following relations, 
\begin{align*}
    &P_{1} + P_{-1} = \mathbb{1}\,,\quad 
    P_{1} P_{-1} = P_{-1} P_1 = 0\,,\quad 
    P_{1}^2 = P_{1}\,,\quad 
    P_{-1}^2 = P_{-1}\,,\quad
    AP_1 = (N_1+1)P_1\,,\\ &AP_{-1} = (N_{-1}-1) P_{-1}\,,\quad 
    (N_1 P_1)^2 = (N_{-1}P_{-1})^2 = 0, \quad \mbox{ and }
    A = (N_1+1)P_1 + (N_{-1}-1) P_{-1}\, .
\end{align*}

As an example of a function of $A$,
$$\mbox{exp}(A) = \mbox{exp}(N_1+1)P_1 + \mbox{exp}(N_{-1}-1)P_{-1} = \mbox{exp}(1) P_1 + \mbox{exp}^\prime(1) N_1 P_1 + \mbox{exp}(-1) P_{-1} + \mbox{exp}^\prime(-1) N_{-1} P_{-1},$$
We verify that this coincides with $\mbox{exp}(A)$
$$
   \mbox{exp}(A)
    =
\begin{pmatrix}
  \begin{matrix}
    - \frac{7}{e} + 3 e & - \frac{5}{e} + 2 e & - e + \frac{3}{e} & - e + \frac{3}{e}\\\frac{2}{e} + 3 e & 4 \cosh{\left(1 \right)} & - 2 \cosh{\left(1 \right)} & - 2 \cosh{\left(1 \right)}\\- \frac{17}{e} + 15 e & - \frac{11}{e} + 10 e & - 5 e + \frac{7}{e} & - 6 e + \frac{7}{e}\\- 3 e & - 2 e & e & 2 e
  \end{matrix}
\end{pmatrix}.
$$




\section{Final Comments}\label{sec:conclusions}
In this manuscript, we obtained explicit formulas relating    higher order derivatives of  the adjugate matrix $\mbox{Adj}(z-A)$ to  the Jordan decomposition of the matrix $A$.  See Theorem \ref{thmmain}.
To obtain these identities we used  the Riesz projector and some results from functional calculus. 
The results presented here can be considered a generalization of the Thompson and McEnteggert theorem that relates the adjugate matrix with the orthogonal projection on the eigenspace of simple eigenvalues for symmetric matrices. See \cite{iserles2002acta,parlett1998symmetric,thompson1968principal}.
It can also be regarded as a generalized form of some identities in \cite{Castillo2021OnAF}.
They can also be viewed as a complement to some  previous results by B. Parisse, M. Vaughan in  \cite{1parisse:hal-00003444} that related derivatives of the adjugate matrix with the invariant subspaces associated with an eigenvalue. Additionally, the formulas can be regarded as a general eigenvector-eigenvalue identity, see \cite{denton2022eigenvectors}.
Although this method for obtaining the eigenvectors from derivatives of the cofactors of $A-z$ and  the nilpotent matrices of the Jordan decomposition (in the base generated by the eigenvectors) may not be efficient from a numerical point of view, it may be useful for symbolic computations or  theoretical purposes.
Further generalization of the main results of this paper over other fields than complex may be carried out in the future. Applications of the techniques developed here to other problems in linear algebra related to functional calculus and invariant subspaces are the subject of current research.

\section*{Authorship contribution statement}
J.R-M. wrote an initial draft of the paper 
with the results and proofs presented here. J.G. Helped write the final version of the manuscript and to establish connections with existing previous results. 
\section*{Acknowledgements}
This work was developed  in the 2022-II Numerical Analysis and Finite Element Seminar at the Universidad Nacional de Colombia - Bogot\'a.
The authors thank Professor Marcus Sarkis for proposing to study the article \cite{frommer2001algebraic} that led them to discuss the spectrum of nonsymmetric matrices. 

\printbibliography
\end{document}