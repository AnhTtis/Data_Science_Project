% CVPR 2023 Paper Template
% based on the CVPR template provided by Ming-Ming Cheng (https://github.com/MCG-NKU/CVPR_Template)
% modified and extended by Stefan Roth (stefan.roth@NOSPAMtu-darmstadt.de)

\documentclass[10pt,twocolumn,letterpaper]{article}

\PassOptionsToPackage{numbers}{natbib}
\newcommand{\citecustom}{\citep}

%%%%%%%%% PAPER TYPE  - PLEASE UPDATE FOR FINAL VERSION
% \usepackage[review]{cvpr}      % To produce the REVIEW version
\usepackage{cvpr}              % To produce the CAMERA-READY version
%\usepackage[pagenumbers]{cvpr} % To force page numbers, e.g. for an arXiv version

\usepackage[accsupp]{axessibility}

% Include other packages here, before hyperref.
\usepackage{graphicx}
\usepackage{amsmath}
\usepackage{amssymb}
\usepackage{booktabs}

\usepackage[utf8]{inputenc} % allow utf-8 input
\usepackage[T1]{fontenc}    % use 8-bit T1 fonts
\usepackage{url}            % simple URL typesetting
\usepackage{amsfonts}       % blackboard math symbols
\usepackage{nicefrac}       % compact symbols for 1/2, etc.
\usepackage{microtype}      % microtypography
\usepackage{xcolor}         % colors
\usepackage{natbib}         % bibliography
\usepackage{mathtools}      % 
\usepackage{comment}
\usepackage[size=scriptsize]{subcaption}
\usepackage{multirow}
\usepackage{graphbox}
\usepackage{nameref}
\usepackage{breqn}


% It is strongly recommended to use hyperref, especially for the review version.
% hyperref with option pagebackref eases the reviewers' job.
% Please disable hyperref *only* if you encounter grave issues, e.g. with the
% file validation for the camera-ready version.
%
% If you comment hyperref and then uncomment it, you should delete
% ReviewTempalte.aux before re-running LaTeX.
% (Or just hit 'q' on the first LaTeX run, let it finish, and you
%  should be clear).
\usepackage[pagebackref,breaklinks,colorlinks]{hyperref}


% Support for easy cross-referencing
\usepackage[capitalize]{cleveref}
\crefname{section}{Sec.}{Secs.}
\Crefname{section}{Section}{Sections}
\Crefname{table}{Table}{Tables}
\crefname{table}{Tab.}{Tabs.}


%%%%%%%%% PAPER ID  - PLEASE UPDATE
\def\cvprPaperID{11751} % *** Enter the CVPR Paper ID here
\def\confName{CVPR}
\def\confYear{2023}

\begin{document}

% \iftrue % use space-saving macro
\iffalse
    \newcommand{\cutabstractup}{\vspace*{-0.1in}}
    \newcommand{\cutabstractdown}{\vspace*{-0.1in}}

    \newcommand{\cutsectionup}{\vspace*{-0.08in}}
    \newcommand{\cutsectiondown}{\vspace*{-0.07in}}

    \newcommand{\cutsubsectionup}{\vspace*{-0.08in}}
    \newcommand{\cutsubsectiondown}{\vspace*{-0.07in}}

    \newcommand{\cutsubsubsectionup}{\vspace*{-0.08in}}
    \newcommand{\cutsubsubsectiondown}{\vspace*{-0.07in}}

    \newcommand{\cutparagraphup}{\vspace*{-0.09in}}
    \newcommand{\cutparagraphdown}{\vspace*{-0in}}

    \newcommand{\cutcaptionup}{\vspace*{-0.1in}}
    \newcommand{\cutcaptiondown}{\vspace*{-0.1in}}

    \newcommand{\cuttablecaptionup}{\vspace*{-0.1in}}
    \newcommand{\cuttablecaptiondown}{\vspace*{-0.1in}}

    \newcommand{\cutequationup}{\vspace*{-0.1in}}
    \newcommand{\cutequationdown}{\vspace*{-0.1in}}

    \newcommand{\cuttableup}{}
    \newcommand{\cuttabledown}{}

    \newcommand{\cut}{{\vspace*{-0.02in}}}
    \newcommand{\cutmore}{{\vspace*{-0.06in}}}
    \newcommand{\negcut}{}
\else % do not use space-saving macro
    \newcommand{\cutsectionup}{}
    \newcommand{\cutsectiondown}{}

    \newcommand{\cutsubsectionup}{}
    \newcommand{\cutsubsectiondown}{}

    \newcommand{\cutsubsubsectionup}{}
    \newcommand{\cutsubsubsectiondown}{}

    \newcommand{\cutparagraphup}{}
    \newcommand{\cutparagraphdown}{}

    \newcommand{\cutcaptionup}{}
    \newcommand{\cutcaptiondown}{}

    \newcommand{\cutequationup}{}
    \newcommand{\cutequationdown}{}

    \newcommand{\cuttableup}{}
    \newcommand{\cuttabledown}{}

    \newcommand{\cut}{}
    \newcommand{\cutmore}{}
    \newcommand{\negcut}{}
\fi

\title{Sequential training of GANs against GAN-classifiers reveals correlated ``knowledge gaps'' present among independently trained GAN instances}

\author{%
  Arkanath Pathak \qquad Nicholas Dufour \\
  Google Research \\ 
  \texttt{\{arkanath,ndufour\}@google.com}
  % examples of more authors
  % \AND
  % Coauthor \\
  % Affiliation \\
  % Address \\
  % \texttt{email} \\
  % \And
  % Coauthor \\
  % Affiliation \\
  % Address \\
  % \texttt{email} \\
  % \And
  % Coauthor \\
  % Affiliation \\
  % Address \\
  % \texttt{email} \\
}

\maketitle
% Abstract -----------------------------------------------------------------------------
\begin{abstract}
The current study investigated possible human-robot kinaesthetic interaction using a variational recurrent neural network model, called PV-RNN, which is based on the free energy principle.
Our prior robotic studies using PV-RNN showed that the nature of interactions between top-down expectation and bottom-up inference is strongly affected by a parameter, called the meta-prior, which regulates the complexity term in free energy.
% The current study examines how the behaviours of robots alter by changing the meta-prior $w$ in human-robot kinaesthetic interaction.
The current study examines how changing the meta-prior $w$ in the interaction phase affects the counter force generated when an experimenter attempts to induce movement pattern transitions familiar to the robot through its prior training.
The study also compares the counter force generated when trained transitions are induced by a human experimenter and when untrained transitions are induced.
Our experimental results indicated that (1) the human experimenter needs more/less force to induce trained transitions when $w$ is set with larger/smaller values, (2) the human experimenter needs more force to act on the robot when he attempts to induce untrained as opposed to trained movement pattern transitions.
Our analysis of time development of essential variables and values in PV-RNN during bodily interaction clarified the mechanism by which gaps in actional intentions between the human experimenter and the robot can be manifested as reaction forces between them.


%% Hiroki writing 2022-11-4
%Current study investigates the dynamics of the latent states during human-robot kinaesthetic interaction using PV-RNN.
%We have achieved to observe and analyse the internal state of an RNN model based on the free energy principle, during real-time human-robot interaction.
%Essential characteristics observed in the previous study of this variational recurrent neural network model, PV-RNN, is that by changing a meta prior $w$, the balance between the top-down intention and the bottom-up perceptual reality changes.
%In the current study, we examined how changing the weighting parameter $w$ between accuracy and complexity in free energy principle affects the humanoid robot's behaviour through human-robot interaction. We have conducted some human-robot kinaesthetic interaction experiments with various $w$ and quantitatively analysed the latent variable and the force applied to the humanoid robot. We have observed that the force required to change the robot's intention has increased, both when the top-down intention was strengthened by changing the $w$ and when corresponding switch of its primitive was against the experience of the RNN during its training. The study confirms through quantitative analysis that by increasing or decreasing the $w$ in PV-RNN, humanoid robot leads or follows the human counterpart during the human-robot kinaesthetic interaction.

\begin{comment}
Comment from Jun #2
・最後にQualitativeな結果(インパクト)が欲しい
・Current study investigates the problem on~と書き出すのが一般的
・最初の一文と最後の一文を対応させる
・最後の一文はもう少しAbstractかつ包括的に
\end{comment}

\begin{comment}
Comment from Jun #1
We investigated how the kinaesthetic human-robot interaction can affect the internal state of a model based on the free energy principle. 
=> how the internal state is affected is not the most important point in this study. This part should be rewritten.

The key function of this variational recurrent neural network model, PV-RNN, is that by changing a meta prior $w$, it takes a balance between the "complexity” term and the ”accuracy” term which corresponds to a top-down intention and a bottom-up perceptual reality in the free energy principle, respectively. 
=> This is not key function of PV-RNN. It is an essential characteristics observed in the previous study. The grammar after $w$ is something strange. Rewrite these.

This research has conducted a human-robot interaction experiment with a robotic agent in a kinaesthetic sense.
=> The sentence is not good. "in a kinaesthetic sense" is grammatically wrong.
MODIFIED => "In the current study human-robot interaction experiments using the kinaesthetic sense were conducted."

We investigated that when human forces the agent to switch primitives from one to another, larger force was required both when the human intention is conflictive against the top-down the intention of the agent and when the agent has a stronger top-down intention by modifying the $w$.
=> You should write the essential results of the experiments rather than what we investigated and also how these results could contribute to the studies on human-robot interaction.
\end{comment}

\end{abstract}
% Introduction -------------------------------------------------------------------------
% \begin{figure}[t]
%     % \begin{subfigure}{1\linewidth}
%     %   \centering
%     % %   \includegraphics[width=1\linewidth]{figs/fig_1_moti_textattn.pdf}  
%     % %   \includegraphics[width=1\linewidth]{figs/fig_1_moti_textattn_v2.pdf}  
%     %   \includegraphics[width=1\linewidth]{figs/fig_1_moti_textattn_v5.pdf}  
%     %   \vspace{-0.5cm}
%     %     \caption{Amount of attention added to each video clip from the source video and query text in the self-attention layers of Moment-DETR encoder.}
%     %     % \caption{Distribution of attention for source and query in Moment-DETR encoder}
%     %     % Visualization of video clip's self-attention score in Moment-DETR encoder.
%     %   \label{fig:fig1_text_attn_ex}
%     % \end{subfigure}%\hfill% or  or \hspace{0.3\textwidth}
%     \vspace{0.2cm}
%     % \begin{subfigure}{1\linewidth}
%       \centering
%     %   \includegraphics[width=1\linewidth]{figs/fig1_moti_negattn.pdf}  
%       \includegraphics[width=1\linewidth]{figs/fig1_moti_negattn_v3.pdf}  
%       \vspace{-0.4cm}
%     %   \caption{Correspondence of saliency scores on the relevance between video clips and the text query.}
%     % \caption{Predicted saliency scores against the video relevant positive query and video irrelevant negative query}
%       \label{fig:fig1_neg_attn_ex}
%     % \end{subfigure}%\hfill% or  or \hspace{0.3\textwidth}
%     \caption{
%     % 원준 원본
%     % (a) Comparison between attention scores of source and query for each video clip~(We sum the attention scores from video and text). 
%     % We observe that the attention scores are dominated by other clips in the source video. 
%     % Text queries do not account for much attention regardless of the relevance to the video clips.
%     % \textbf{(a)} Inspection of the query dependency in Moment-DETR encoder.
%     % % We visualize the attention score of video tokens in the transformer encoder and observe that text query accounts for only a low portion of attention.
%     % % This tendency occurs regardless of the relevance between the text query and video clips. 
%     % We visualize the attention score of video tokens in the transformer encoder and observe 1) text query only accounts for a low portion of attention, and 2) relevance between video-query pair does not affect the attention scores ratio of text.
%     \textbf{(b)} Comparison of highlight-ness when relevant and non-relevant queries are input.
%     As observed in , existing work only uses queries to play an insignificant role, thereby may not be capable of detecting false queries and considering the video-query relevance even when the problem in (a) is resolved. 
%     % \SE{} % 이 부분이 "not capable of" 란 용어가 세다는 피드백이 있는 듯 합니다. 이러한 능력이 없다는 것은 굉장히 강한 어조인거 같기는 하고, 이러한 경우들이 종종 있다거나 좀 약화시킬 필요가 있어보이긴 하네요.
%     On the other hand, our QD-DETR yields a query-dependent representation that the relevance between the source video and query text is updated in the saliency scores.
%     There is a large gap between positive and negative saliency scores, and scores are consistent since the clips are all highly correlated to others.
%     }
%     \label{fig:motivation_ex}
%     % \captionsetup{belowskip=13pt}
%     % \setlength{\belowcaptionskip}{-10pt}
% \end{figure}
\begin{figure}
    \centering
    \includegraphics[width=1\linewidth]{figs/fig1_moti_negattn_1111.pdf}
    % \includegraphics[width=1\linewidth]{figs/fig1_moti_negattn_1109.pdf}
    % \includegraphics[width=1\linewidth]{figs/fig1_moti_negattn_stat.pdf}
    \vspace{-0.6cm}
    \caption{
        % \SE{} % 수정 필요
        Comparison of highlight-ness~(saliency score) when relevant and non-relevant queries are given.
        We found that the existing work only uses queries to play an insignificant role, thereby may not be capable of detecting negative queries and video-query relevance; saliency scores for clips in ground-truth~(GT) moments are low and equivalent for positive and negative queries.
        % This also results in mispredicted moments when ground-truth~(GT) moment is dominated by clips unrelated to GT since their prediction is highly focused on the video.
        % \SE{} % 여기 한번 더 보면 좋을 듯 합니다. GT moment에 unrelated한 clip이 많으면? label이 틀렷을 경우를 말씀하시는건지?
        % As observed in saliency graph, existing work only uses queries to play an insignificant role, thereby may not be capable of detecting false queries and considering the video-query relevance.
        On the other hand, query-dependent representations of QD-DETR result in corresponding saliency scores to the video-query relevance and precisely localized moments.
        % On the other hand, our QD-DETR yields a query-dependent representation that the
        % saliency scores are in accordance with the relevance between the video and query.
        % text is in accordance with the saliency scores.
        % There is a large gap between positive and negative saliency scores, and scores are consistent since the clips are all highly correlated to others.
}
    \label{fig:motivation_ex}
\end{figure}


\section{Introduction}
% 원준 원본
% Along with the advance of digital devices and platforms, video is now one of the most desired data type for consumers. However, although the large information capacity of videos may be beneficial in many aspects, e.g., informative and entertaining, on the contrary perspective, videos are time-consuming, and hard to search for desirable moments. 
% This has led many creators to use extra manpower to crop and edit the video to generate highlight clips to gain the consumer’s attention.
Along with the advance of digital devices and platforms, video is now one of the most desired data types for consumers~\cite{apostolidis2021video,wu2017deep}.
% SE: Video aware deep learning application & survey papers?
Although the large information capacity of videos might be beneficial in many aspects, e.g., informative and entertaining, inspecting the videos is time-consuming, so that it is hard to capture the desired moments~\cite{anne2017localizing,apostolidis2021video}. 
% This has led many creators to use extra manpower to crop and edit the video to generate highlight clips to gain the consumer’s attention.


% On the other side, 
Indeed, the need to retrieve user-requested or highlight moments within videos is greatly raised.
Numerous research efforts were put into the search for the requested moments in the video~\cite{anne2017localizing, gao2017tall, liu2015multi, escorcia2019temporal} and summarizing the video highlights~\cite{zhang2016video, mahasseni2017unsupervised, badamdorj2022contrastive, wei2022learning}.
% Numerous research efforts were put into the search for the requested moments in the video~\cite{anne2017localizing, gao2017tall, liu2015multi, escorcia2019temporal}, summarizing the video to generate highlights was another popular topic~\cite{zhang2016video, mahasseni2017unsupervised, badamdorj2022contrastive, wei2022learning}.
Recently, Moment-DETR~\cite{momentdetr} further spotlighted the topic by proposing a QVHighlights dataset that enables the model to perform both tasks, retrieving the moments with their highlight-ness, simultaneously.

% 원준 원본
% To detect the desired moments, previous works employed transformer encoder-decoder architectural designs to fuse the text query into the video representations. Moment-DETR~\cite{mDETR} modified detection transformer to process capture the moment as a set, and UMT~\cite{umt} implemented transformer decoder as to output clip-wise saliency. 
% Yet to their outstanding breakthroughs in the literature of moment retrieval with the seminal architectures, their limitation is that the role of the given text query is insignificant in representing the query-conditioned video representation; the attention mechanism of moment DETR is not explicitly conditioned on the text query, and the text query is conditioned on multi-modal clips where the differences between the clips are smoothed after encoding process in UMT.



% \begin{figure}[t]
% \centering
%     \begin{subfigure}[l]{0.37\linewidth}
%       \centering
%       \vspace{0.20cm}
%     %   \includegraphics[width=1\linewidth]{figs/fig_1_moti_textattn.pdf}  
%     %   \includegraphics[width=1\linewidth]{figs/fig_1_moti_textattn_v2.pdf}  
%       \includegraphics[width=1\linewidth]{figs/fig1_moti_violin_a.pdf}  
%       \vspace{-0.60cm}
%     %   \caption{text attention}
%         \caption{Importance of queries in video representation}
%       \label{fig:fig1_text_attn}
%     \end{subfigure}%\hfill% or  or \hspace{0.3\textwidth}
%     \vspace{0.2cm}
%     \begin{subfigure}[r]{0.61\linewidth}
%       \centering
%     %   \includegraphics[width=1\linewidth]{figs/fig1_moti_negattn.pdf}  
%       \includegraphics[width=1\linewidth]{figs/fig1_moti_violin_b.pdf}  
%     %   \caption{neg attention}
%         % \caption{Relation between the highlight-ness and the relevance between videos and query texts.}
%         \caption{Highlight-ness~(saliency) histogram of positive and negative video-query pairs\SE{}}
%       \label{fig:fig1_neg_attn}
%     \end{subfigure}%\hfill% or  or \hspace{0.3\textwidth}
%     % \vspace{-0.2cm}
%     \caption{Overall statistics for attention scores in Fig.~\ref{fig:motivation_ex} in QVHighlights dataset. 
%     (a) For the attention scores that measure how much the text query is generally involved in video representation, we use violin plots to show the probability density. We plot the score for each layer in the encoder.
%     % (b) Using the histogram, we compare how the baseline and QD-DETR yield different salient scores given the positive and negative video-text pairs.
%     (b) Saliency histogram shows the distributional gap between positive and negative video-text query pairs of baseline~(Moment-DETR) and proposed QD-DETR.\SE{}
%     }
%     \label{fig:motivation}
%     % \captionsetup{belowskip=13pt}
%     % \setlength{\belowcaptionskip}{-10pt}
% \end{figure}

% \begin{figure}[t]
% \centering

%     \begin{subfigure}[r]{1\linewidth}
%       \centering
%       \hspace{-0.2cm}
%     %   \includegraphics[width=1\linewidth]{figs/fig1_moti_negattn.pdf}  
%       \includegraphics[width=1.1\linewidth]{figs/fig1_moti_violin_a_v2.pdf}  
%     %   \caption{neg attention}
%         % \caption{Relation between the highlight-ness and the relevance between videos and query texts.}
%         \vspace{-0.5cm}
%         % \caption{Saliency histogram of positive and negative video-query pairs}
%         \caption{We plot the histograms and its average value~(dotted line) to compare saliency scores when true and false text queries are given for each method. (left) Since the video representations do not include much textual information, both the true and false queries yield similar saliency scores. (Middle) Even when the video representation is enforced to be updated with the textual information, the issue is not much resolved. (Right) By extracting discriminative features in the text query, distributions are differentiated.
%         % \SE{} % R1@0.5 설명
%         Also, R1@0.5 indicates evaluation metric, Recall at 1 with IoU 0.5 threshold on QVhighlight \textit{val} set.
%         }
%       \label{fig:fig1_neg_attn}
%     \end{subfigure}%\hfill% or  or \hspace{0.3\textwidth}
%     \\
%     \begin{tabular}{cc}
%     \hspace{-0.2cm}
%         \begin{minipage}{.4\linewidth}
%             \begin{subfigure}[l]{1\linewidth}
%               \centering
%             %   \vspace{0.20cm}
%             %   \includegraphics[width=1\linewidth]{figs/fig_1_moti_textattn.pdf}  
%             %   \includegraphics[width=1\linewidth]{figs/fig_1_moti_textattn_v2.pdf}  
%               \includegraphics[width=1\linewidth]{figs/fig1_moti_violin_a.pdf}  
%               \vspace{-0.60cm}
%             %   \caption{text attention}
%                 \caption{Importance of queries in video representation}
%               \label{fig:fig1_text_attn}
%             \end{subfigure}%\hfill% or  or \hspace{0.3\textwidth}
%         \end{minipage}
        
%         \begin{minipage}{.6\linewidth}
%             \vspace{-0.2cm}
%             \caption{Overall statistics of Fig.~\ref{fig:motivation_ex} in QVHighlights dataset. 
%             (a) Saliency histogram shows the distributional gap between positive and negative video-text query pairs.
%             % (a) For the attention scores that measure how much the text query is generally involved in video representation, we use violin plots to show the probability density. We plot the score for each layer in the encoder.
%             % (b) Using the histogram, we compare how the baseline and QD-DETR yield different salient scores given the positive and negative video-text pairs.
%             % (b) Text ratio in self-attention layer to  of Moment-DETR
%             % (b) Ratio of text when representing video tokens in self-attention of Moment-DETR.
%             % (b) Magnitude of attention text query involved.
%             % (b) Attention score of video tokens
%             % (b) Magnitude of text query to refine the video tokens in self-attention layer of Moment-DETR.
%             (b) Probability density depicting the weight of the text query in attention score for video clips. Scores are from the self-attention layers in Moment-DETR encoder.
%             % (b) The text query ratio in attention score of video clips (Self-attention layer in Moment-DETR encoder). We use violin plots to show probability density.
%             % 텍스트 쿼리가, 비디오 피쳐에 얼만큼 attend 하는지
%             }
%         \end{minipage}
    
%     \end{tabular}
%     \vspace{-0.5cm}
%     \label{fig:moti}
%     % \captionsetup{belowskip=13pt}
%     % \setlength{\belowcaptionskip}{-10pt}
% \end{figure}


% \begin{figure}
%     \centering
%     % \includegraphics[width=1\linewidth]{figs/fig1_moti_negattn_1109.pdf}
%     \includegraphics[width=1\linewidth]{figs/fig1_moti_negattn_stat_v2.pdf}
%     \vspace{-0.8cm}
%     \caption{
%         Histogram of saliency when the positive and negative queries are given. We plot the histograms and its average value~(dotted line) to compare saliency scores when relevant~(positive) and irrelevant~(negative) text queries are given for each method. (Left) Since the video representations do not properly reflect textual information, both the positive and negative queries yield similar saliency scores. 
%         % (Middle) Even when the video representation is enforced to be updated with the textual information, the issue is not much resolved. 
%         (Right) By representing video clips in query-dependent manner, distributions are differentiated.
%     }
%     \vspace{-0.6cm}
%     \label{fig:motivation}
% \end{figure}


% One of the demanding task is moment retrieval task, which is detecting the desired moments from the given query, typically the text query.
When describing the moment, one of the most favored types of query is the natural language sentence~(text)\cite{anne2017localizing}. 
While early methods utilized convolution networks~\cite{zhang2020learning, gao2021fast, wang2020temporally}, recent approaches have shown that deploying the attention mechanism of transformer architecture is more effective to fuse the text query into the video representation.
% To handle these modalities, previous works simply employed the attention mechanism of transformer architecture to fuse the text query into the video representation.
For example, Moment-DETR~\cite{momentdetr} introduced the transformer architecture which processes both text and video tokens as input by modifying the detection transformer~(DETR), and UMT~\cite{umt} proposed transformer architectures to take multi-modal sources, e.g., video and audio. 
Also, they utilized the text queries in the transformer decoder.
Although they brought breakthroughs in the field of MR/HD with seminal architectures, they overlooked the role of the text query.
To validate our claim, we investigate the Moment-DETR~\cite{momentdetr} in terms of the impact of text query in MR/HD~(Fig.\ref{fig:motivation_ex}).
Given the video clips with a relevant positive query and an irrelevant negative query, we observe that the baseline often neglects the given text query when estimating the query-relevance scores, i.e., saliency scores, for each video clip.
% the output saliency score, i.e. query-relevance scores.
% Based on the observation, we traced the actual saliency prediction of the model against both the video-relevant query and the irrelevant dummy one where we find that the baseline often neglects the given text query when estimating the query-relevance scores of video clips.
% For example, in Fig.~\ref{fig:motivation_ex}, saliency scores are not affected even when the query is substituted with the dummy.
% % General statistics for Fig.~\ref{fig:motivation_ex} is shown in Fig.~\ref{fig:motivation}. 
% General statistics corresponding to Fig.~\ref{fig:motivation_ex} are also shown in Fig.~\ref{fig:motivation}.



% The limitation of the concrete baseline~\cite{momentdetr} is inspected in two different aspects; 1) Utilization of text-query in the encoding process and 2) the output saliency score, i.e. query-relevance scores.
% Firstly, we visualize the attention score when video clips are given as a query in self-attention. 
% We observe that the text queries have relatively small impacts compared to other video features, as shown in Fig.~\ref{fig:fig1_text_attn_ex}.
% That is, the text does not account for much in representing every video clip, although the goal of MR/HD is to detect query-relevant moments.
% Based on the observation, we traced the actual saliency prediction of the model against both the video-relevant query and the irrelevant dummy one where we find that the baseline often neglects the given text query when estimating the query-relevance scores of video clips.
% For example, in Fig.~\ref{fig:motivation_ex}, saliency scores are not affected even when the query is substituted with the dummy.
% % General statistics for Fig.~\ref{fig:motivation_ex} is shown in Fig.~\ref{fig:motivation}. 
% General statistics are also shown in Fig.~\ref{fig:motivation}.

% Consequently, in Fig.~\ref{fig:fig1_neg_attn_ex}~(b), we found that the baseline often neglects the given text query when estimating the query-relevance scores of video clips; 
% For example, 


% We validate the previous work sometimes neglects the given query when estimating the saliency of video clips.
% For example, there is an example that the saliency scores from positive and negative queries cannot be distinguishable, as shown in Fig.~\ref{fig:fig1_neg_attn_ex}.
% % 우리는 추가로 text attention을 추가도 해봤지만, 효과가 있긴 했으나, still 이슈가 있는 것을 확인하였다?
% % Still, we observe that assuring the high attendance of text queries does not resolve the overlap which motivates us to question the quality of the naive use of task-agnostic text representation~\cite{momentdetr, umt}.
% We found that introducing the text-attention for ensuring the high attendance of text queries relieve the overlap, but there still be a severe overlap.


% To validate their limitations, we inspect the impacts of text queries in the concrete baseline~\cite{momentdetr} with the two different aspects, 1) tendency of attention in self-attention layer and 2) saliency score, i.e. query-relevance scores. \SE{} % attention 이 갑자기 등장하는가?
% Firstly, we visualize the attention score when video clips are given as a query in self-attention. We observe the text queries have relatively low attention scores compared to the video features, as shown in Fig.~\ref{fig:fig1_text_attn_ex}.
% That is, the text does not account for much in representing every video clip, although the goal of MR/HD is to detect query-relevant moments.
% Based on this observation, we trace the actual saliency prediction of the model against both positive and negative text queries.
% We validate the previous work sometimes neglects the given query when estimating the saliency of video clips.
% For example, there is an example that the saliency scores from positive and negative queries cannot be distinguishable, as shown in Fig.~\ref{fig:fig1_neg_attn_ex}.
% % 우리는 추가로 text attention을 추가도 해봤지만, 효과가 있긴 했으나, still 이슈가 있는 것을 확인하였다?
% % Still, we observe that assuring the high attendance of text queries does not resolve the overlap which motivates us to question the quality of the naive use of task-agnostic text representation~\cite{momentdetr, umt}.
% We found that introducing the text-attention for ensuring the high attendance of text queries relieve the overlap, but there still be a severe overlap.



% Thus, we 
% query dependency를 높이기 위해 
% Cross-attention? text-attention? detailed explanation on text-attention should be needed?
% By handling these two issues, we find that more precise retrieval can be achieved.
% 
% 
%
% By projecting video-discriminative text features with high text attendance to source video, we f 
% We also find the need to improve the quality of query features since assuring high text attendance also results in...
% pairs are not finetuned to be discriminative that even the similarity within the pairs does not reflect the relevance between the query and the video clips.
% General statistics for Fig.~\ref{fig:motivation_ex} is shown in Fig.~\ref{fig:motivation}. 
% \SE{} % 이거 ??로 뜨는데, 위처럼 figure 그리면 label이 안되는걸까요
% \SE{}
% 형님 아래 사항 생각 좀 해보는게 좋을 거 같아요.
% fig 1. (a) 그림만 봤을 때 모든 clip에 대해 text attention이 일정이상 존재하긴 하니까, 뭔가 not assured to be conditioned가 와닿지 않는거 같아요.
% + 왜 text가 항상 attend 해야하나?
% not assured to be conditioned --> text shows relatively low affects compared to video 같이 실제 나타난 현상까지 같이 적으면 어떨까 싶어요.
% fig 1. (b) 덜 반영한다?

% \SU{}
% 일단 text가 attend 잘 되어야 한다는 것에 좀 궁금점이 생깁니다. 결국에는 text와 관련있는 frame들을 attend해서 higlight를 찾아야 하는게 아닐까요? 그리고, 현제 저희의 모델 구조상 text query가 Key와 Value로 거의 활용되고 있는데 그렇다면 결국에는 해당 모델은 text에 대한 attention이 전혀 없다고 봐도 무방하지 않을까요? 그런 면에서 text attention을 강조하는게 좀 걸리긴 합니다.

% Specifically, the text query is not assured to be explicitly conditioned on every clip of the video, and as the query texts are evenly treated, discriminative keywords may not be spotlighted.
% attention mechanism of Moment-DETR is not explicitly conditioned on the text query as shown in Fig~\ref{}(d), and in UMT, the text are only used for conditioning the queries while the video representation are refined itself by self-attention.

% \begin{figure}[t]
%     \begin{subfigure}{1\linewidth}
%       \centering
%     %   \includegraphics[width=1\linewidth]{figs/fig_1_moti_textattn.pdf}  
%     %   \includegraphics[width=1\linewidth]{figs/fig_1_moti_textattn_v2.pdf}  
%       \includegraphics[width=1\linewidth]{figs/fig_1_moti_textattn_v4.pdf}  
%       \vspace{-0.5cm}
%     %   \caption{text attention}
%         \caption{Distribution of attention scores in Moment-DETR encoder}
%       \label{fig:fig1_text_attn}
%     \end{subfigure}%\hfill% or  or \hspace{0.3\textwidth}
%     \vspace{0.2cm}
%     \begin{subfigure}{1\linewidth}
%       \centering
%     %   \includegraphics[width=1\linewidth]{figs/fig1_moti_negattn.pdf}  
%       \includegraphics[width=1\linewidth]{figs/fig1_moti_negattn_v2.pdf}  
%       \vspace{-0.5cm}
%     %   \caption{neg attention}
%         \caption{Saliency score against positive and negative text queries}
%       \label{fig:fig1_neg_attn}
%     \end{subfigure}%\hfill% or  or \hspace{0.3\textwidth}
%     \vspace{0.2cm}
%     \begin{subfigure}{1\linewidth}
%       \centering
%     %   \includegraphics[width=1\linewidth]{figs/fig1_moti_violin.pdf}  
%       \includegraphics[width=1\linewidth]{figs/fig1_moti_violin_v2.pdf}  
%       \vspace{-0.5cm}
%       \caption{violin}
%       \label{fig:fig1_violin}
%     \end{subfigure}%\hfill% or  or \hspace{0.3\textwidth}
%     \vspace{-0.2cm}
%     \caption{(a) 1. portion of text attention vs. video attention 2. relation with text query and content (e.g. fg, bg) of clip seems not to affect the attention score
%     (b) 1. high variability even though entire clips are highly correlated with the given text query 2. positive and negative query makes overlaps on saliency score distribution
%     (3) actual distribution on validation dataset.}
%     \label{fig:motivation}
%     % \captionsetup{belowskip=13pt}
%     % \setlength{\belowcaptionskip}{-10pt}
% \end{figure}

To this end, we propose Query-Dependent DETR~(QD-DETR) that produces query-dependent video representation.
% Our key focus is to ensure each clip in predicted moments is explicitly conditioned by the query, particularly on the video-descriptive portion of the text query.
% Our key focus is to ensure that query-relevant clips are predicted by enforcing each clip to be explicitly conditioned by the query.
%Our key focus is to ensure that the model prediction for each clip is highly relevant to the query.
Our key focus is to ensure that the model's prediction for each clip is highly dependent on the query.
% by enforcing each clip to be explicitly conditioned by the query. :)
% hmm...
% \SE {} % "query-relevant clips are predicted" 이 문장이 좀 애매한거 같습니다. relevant 클립을 놓지지 않고 찾는 것을 보장한다? 이런 느낌인지 아니면 높은 saliency 를 주는게 목적이다? model prediction이 query-relevance를 반영하는 것을 보장한다?
% Our key focus is to ensure that the model prediction reflects query-relevance of clips by enforcing each clip to be explicitly conditioned by the query.
First, to fully utilize the contextual information in the query, we revise the transformer encoder to be equipped with cross-attention layers at the very first layers.
% 상익's thought :  single video - query간의 관계만 고려 - 같은 word가 더 많이 쓰이는 것을 보고 
% 교수님's thought : neg pair 를 쓰면 쿼리를 보지 않고서는 video clip간만 고려하는 것이 사라짐. 왜냐면 0으로 내보내야 하기 때문. --> SE: relative difference 만 고려하다가, 
By inserting a video as the query and a text as the key and value of the cross-attention layers, our encoder enforces the engagement of the text query in extracting video representation.
% 원준 교수님 코멘트 반영해서 다시
Then, in order to not only inject a lot of textual information into the video feature but also make it fully exploited, we leverage the negative video-query pairs generated by mixing the original pairs.
Specifically, the model is learned to suppress the saliency scores of such  negative~(irrelevant) pairs.
Our expectation is the increased contribution of the text query in prediction since the videos will be sometimes required to yield high saliency scores and sometimes low ones depending on whether the text query is relevant or not.
% \SE{}
% learns to?
% By suppressing the saliency scores of the irrelevant video-query pairs, the model learns to spotlight only the video-specific discriminative words in the query.
% % \SE{} % ====================== 상익 수정 ========================
% However, this architectural design still lacks the capability of identifying the video-descriptive keywords in the query.
% % However, this architectural design still lacks in identifying proper query relevance.
% This is because the current training scheme only focuses on the interactions of video and clips within a single video while neglecting information shared throughout the entire video.
% % We argue the problem of the current training scheme that only focuses on distinguishing the clips in a single video while neglecting information shared throughout the entire video.
% Therefore, we leverage the negative video-query relationships to enhance the capability of identifying the contextual similarity of query and video clips.
% 
% 원준 원본 
% However, this architectural design heavily relies on the quality of the text query.
% Therefore, we leverage the negative video-query relationships to enable the model to emphasize key corresponding query features.
% By suppressing the saliency scores of the irrelevant video-query pairs, the model learns to spotlight only the video-specific discriminative words in the query.
% =========================================================
Lastly, to apply the dynamic criterion to mark highlights for each instance, we deploy a saliency token to represent the entire video and utilize it as an input-adaptive saliency criterion. 
With all components combined, our QD-DETR produces query-dependent video representation by integrating source and query modalities.
This further allows the use of positional queries~\cite{dabdetr} in the transformer decoder.
% Furthermore, we can exploit the advanced DETR decoder architectures using the positional information, e.g., DAB-DETR, since our encoded tokens consist of identical position representations from a single modality.
% \SE{} % ====================== 상익 수정 ========================
% Furthermore, we can exploit the advanced DETR decoder architectures using the positional information, e.g., DAB-DETR, since our video clip tokens consist of identical position representations from a single modality.
% 원준 원본
% It also enables the use of advanced DETR decoder architectures, e.g., DAB-DETR, for the first time, as these works exploit the position information within a single modality.
% =========================================================
Overall, our superior performances over the existing approaches validate the significance of the role of text query for MR/HD.
% Our extensive experiments on QVHighlights, TVSum, and Charades-STA datasets validate the significance of considering the role and the quality of text query.

% All components combined with dynamic anchor moments for the query of decoder, our FOQUE fosters the query-dependent video representation, thereby making the 
% All components combined, our modified transformer encoding process fosters the query-dependent video representation thereby achieving the state-of-the-art results on various benchmarks of moment-retrieval and highlight detection.
	
% -	Video Platform & Streamer & Consumer의 증가. 
% Video는 다른 데이터 타입보다 정보가 많아 유용하지만, 이는 다른 말로 해석하면 video를 보는 것은 time-consuming 하고, 원하는 것을 찾아보기에는 힘들 수 있음.
% 따라서, 많은 매체에서는 사람들의 더 많은 이목을 끌기 위해 highlight 비디오라는 것을 편집하여 공유도 함.
% 하지만, highlight video를 만들기 위해 사람의 노력이 필요한 현 시점에서, This spotlights the need to retrieve the user-requested / Highlight moments in the video.

% -	이전에도 이러한 문제를 해결하기 위해 (asdfasdf) for moment retrieval, (asdfasdf) for highlight detection 등이 제안 되었지만, 이들은 비디오의 특정 영역을 찾는다는 공통된 목적을 가지고 있으면서도, 데이터 셋의 한계로 인해 따로 연구되었음. 이를 문제 삼으며, 최근에는 두 task를 동시에 학습할 수 있는 dataset이 소개 되었는데, 컴퓨터비전에서 최근 각광을 받고 있는 Transformer 모델 도입과 함께 큰 발전을 거듭하고 있음.

% -	구체적으로, 이 두가지 task를 수행하기 위해서는 transformer를 두가지 방법으로 이용할 수 있는데, moment-DETR 처럼 moment 를 clip의 set 단위로 예측할 수 있고, UMT 처럼 clip-wise prediction을 할 수 있음. 하지만, 이들은 query를 condition이 아닌 video와 동등한 레벨로 취급하거나 [mDETR], 매 클립이 self-attention으로 mixing 된 후에 condition을 걸어주어 clip간의 차이를 확실하지 이용하지 못하였고, 또한, 확실하게 condition으로 주지 못하였고, video와 query 사이의 관계를 한정적으로만 이용하였다.

% -	따라서, we explore three different ways to fully exploit query information. First, we design one-way cross-attention layer to condition every clip with the query features. Then, we utilized the negative video-text pairs to better model the relationships between the video and the text embeddings. Lastly, we define the saliency token to be the video-query dependent saliency estimator.


















% ===================== neg pair 부분 ===========================
% Nevertheless, the current training scheme, only considering the given video-query pair, still disturbs the model from identifying proper query-relevance prediction.
% In detail, the model focus on learning the fine-grained discrepancy between video clips, while neglecting the information they share, which contains significant clues to understand the context of video.
% Therefore, we leverage the negative video-query relationships to enhance the capability of identifying the contextual similarity of query and video clips.
% Therefore, we leverage the negative video-query relationships by suppressing those pairs, so that enhance the capability of identifying the contextual similarity of query and video clips.
% We hypothsize the diversity in query-video pairs are insufficient to learn the general relationship between text query and video.
% Therefore, we leverage the negative video-query relationships by suppressing the saliency scores of the irrelevant video-query pairs.
% However, this architectural design still lacks in identifying proper query relevance.
% We argue that the current training scheme only focuses on learning the fine-grained discrepancy between clips in a single video, while neglecting the information they share, which contains significant clues to understand the context of the video.
% Therefore, we leverage the negative video-query relationships to enhance the capability of identifying the contextual similarity of query and video clips.
% However, this architectural design still lacks in identifying proper query relevance.
% We argue the problem of the current training scheme that only focuses on learning the fine-grained discrepancy between clips in a single video.
% That is, the current design neglects the information shared throughout the video, although it contains significant clues to understand the context of the video.
% Related Work -------------------------------------------------------------------------
\section{Related Work}
\label{sec:related_work}
\subsection{Co-Speech Gesture Synthesis}
The early approaches for generating co-speech gestures often involve creating linguistic rules to translate speech input into a sequence of pre-collected gesture segments, which are typically referred to as rule-based methods \cite{cassell1994rulefullbody,cassell2001beat,kipp2004gesture,kopp2006bml}. \citet{wagner2014rulereview} provide a comprehensive review of these methods. Rule-based methods produce interpretable and controllable results, but creating gesture datasets and rules requires significant effort. To alleviate the manual effort of designing rules in rule-based methods, data-driven approaches have gradually become predominant in this field. \citet{nyatsanga2023data_driven_gesture_survey} offer a thorough survey of these methods. Early data-driven approaches aim to directly learn mapping rules from data through statistical models \cite{neff2008videogesture,levine2009prosodygesture,levine2010gesturecontroller} and combine them with predefined gesture units for gesture generation. Later, the powerful modeling capability of deep neural networks makes it possible to train complex end-to-end models using raw speech-gesture data directly. One option is deterministic models, such as MLP \cite{kucherenko2020gesticulator}, CNN \cite{habibie2021videogesture}, RNN \cite{yoon2019robot,yoon2020trimodalgesture,bhattacharya2021affectivegesture,liu2022hierarchicalgesture}, and Transformer \cite{bhattacharya2021text2gestures}. Another choice is generative models, including flow-based models \cite{alexanderson2020stylegesture,ye2022styleflowgesture}, VAEs \cite{li2021audio2gesture,ghorbani2022zeroeggs}, and VQ-VAE \cite{yi2022talkshow,yazdian2022gesture2vec,liu2022vqgesturevideo}. Due to the inherent many-to-many relationship between speech and gesture, end-to-end models can generate natural-looking gestures but face challenges in ensuring content matching between speech and generated gestures \cite{yoon2022genea}. To address this issue, some neural systems aim to explicitly model both rhythm and semantics from the perspective of model structure \cite{kucherenko2021speech2properties2gestures,ao2022rhythmicgesticulator,liu2022disco} or training supervision strategy \cite{liang2022seeg}. Furthermore, hybrid systems, such as the combination of deep features and motion graphs \cite{zhou2022gesturemaster}, have been proposed to harness the advantages of different approaches. Recently, diffusion models \cite{sohldickstein2015diffusion,song2020improvedscore,ho2020ddpm} have demonstrated impressive results in image synthesis \cite{ramesh2022dalle2} and human motion generation \cite{tevet2022humanmotiondiffusion, zhang2022motiondiffuse}. Inspired by these works, our system adapts the latent diffusion model \cite{rombach2022latentdiffusion} for the co-speech gesture generation task and achieves appealing results.

\subsection{Style Control for Human Motion}
A typical approach to style control for human motion involves specifying a motion clip as a reference and transferring the reference clip's style to the source motion. This task is also known as \emph{style transfer}. Early works in motion style transfer integrate traditional machine learning techniques with manually defined features to infer motion styles \cite{hsu2005motion_style_translation,ma2010motion_style_transfer,xia2015realtime_motion_style_transfer,yumer2016spectral_motion_style_transfer}. Recently, deep learning-based methods have significantly enhanced motion quality. \citet{holden2016deepmotion} first propose a learning framework enabling motion style control through optimization in the motion manifold space. \citet{du2019stylemotioncvae} improve transfer efficiency by training a conditional VAE. \citet{mason2018few-shot_motion_style_transfer} use few-shot learning to generate stylized locomotion. \citet{aberman2020adain} employ a temporally invariant adaptive instance normalization (AdaIN) layer for target style injection, eliminating the need for paired data during training. \citet{wen2021stylemotionflow} achieve unsupervised style transfer using a flow model. \citet{jang2022motionpuzzle} introduce a method capable of controlling styles for individual body parts.

Previous co-speech gesture synthesis systems with style control can be categorized based on whether or not they require style labels. For methods needing labeled data, early works can only learn an individual style for one generator \cite{levine2010gesturecontroller,neff2008videogesture,ginosar2019stylegesture}. \citet{ahuja2022lowresource} propose a strategy that efficiently adapts the source generator to another speaker style using low-resource data. Some works learn a speaker style embedding space with labeled speaker-motion data, enabling gesture style control by sampling from this space \cite{ahuja2020stylegesture,yoon2020trimodalgesture,bhattacharya2021affectivegesture}. \citet{alexanderson2020stylegesture} aimat controlling fine-grained styles, such as gesturing speed and spatial scope, using preprocessed control signal-motion data. Their later work \cite{alexanderson2022diffusiongesture} utilizes a diffusion model for audio-driven motion synthesis, achieving label-based style control by training the model on labeled data. For methods not requiring style labels, \citet{habibie2022motionmatching} propose a motion matching framework to achieve flexible style control. Other studies achieve arbitrary style control by imitating an example given as a video \cite{liu2022hierarchicalgesture} or a motion clip \cite{ghorbani2022zeroeggs,ye2022styleflowgesture,kuriyama2022tokenizedgestures}.  In this work, we utilize a CLIP-based encoder to extract a style embedding from an arbitrary text prompt and incorporate it into the generator via an AdaIN layer, guiding the synthesis of stylized gestures. Our system supports fine-grained multimodal style prompts as opposed to label-based style control. It employs a self-supervised learning scheme and eliminates the need for labeled data. Additionally, we use an autoregressive model rather than a parallel model, making it potentially suitable for real-time applications.
% Methods -------------------------------------------------------------
\cutsectionup
\section{Approach}
\cutsectiondown

We study the phenomena outlined in the introduction by creating and measuring the performance of classifiers trained to detect images sampled from \emph{unseen} generators and subsequently training new generators to fool them, in sequential rounds, forming a chain of generators and classifiers. We do this in one of two settings, first with low dimensional images (MNIST), a simplistic DCGAN, and a basic classifier architecture. In the second setting, we use higher dimensional images (FFHQ), and perform experiments using the unmodified StyleGAN2 (SG2) architecture. Seeking to  minimize sources of variance as much as possible, we limit to a single GAN architecture and a fixed dataset in both settings. We also do not use the ``truncation'' trick \citecustom{karras2019style}, a sample-time heuristic commonly used with the SG2 architecture to improve the output visual quality at the expense of diversity (\emph{see} Supplement for more discussion on this). In the SG2 setting, we test three different widely-used classifier architectures: ResNet-50, Inception-v3, and MobileNetV2. These architectures were chosen for their architectural diversity. All classifiers and generators are trained from scratch, without any pre-training. Supplement provides details about the model architectures and training parameters.

\cutsubsubsectionup
\subsection{A note on terminology} \label{sec:terminology}
\cutsubsubsectiondown

Because our procedure involves both GANs and classifiers, there is potential ambiguity in terminology as GANs themselves are trained with a subnetwork designed to distinguish generated images from natural images, which is commonly called the ``discriminator'', ``adversarial network'', or ``critic'', among others. To keep the text clear, we will refer to subnetworks co-trained with a generator which together comprise a GAN as ``\textbf{discriminators}'', denoted $D$. The networks trained on samples from multiple, independently trained generators are referred to as ``\textbf{classifiers}'', $C$. Each sequential round of training a pool of GANs followed by training classifiers is an ``\textbf{iteration}'' (detailed in Sec. \ref{sec:overview_setup}, and Figs. \ref{fig:experiment_setup_1} and \ref{fig:experiment_setup_2}) and is indexed with a superscript. Iterations are distinct from training steps: during a single iteration, GANs are fully trained, then classifiers are fully trained using those GAN generators. Broadly speaking an ``\textbf{artifact}'' is any property of a generated image that distinguishes it from a real image. By ``\textbf{knowledge gaps}'', we are referring to a specific class of artfacts that reliably occur \emph{across} samples from a generator. Since this class of artifacts is the only one studied in this work, we use artifact and knowledge gap interchangeably.

\cutsubsubsectionup
\subsection{Overview of setup and iterations} \label{sec:overview_setup}
\cutsubsubsectiondown
\begin{figure}[h]
	\centering
    \begin{subfigure}{\linewidth}
        \centering
        \includegraphics[width=0.7\linewidth]{images/gan_training_zeroth.pdf}
        \caption{Stage 1 at iteration 0: GAN training with standard loss function} \label{fig:gan_training_zeroth}
    \end{subfigure}%
    \hspace{0.05\textwidth}
    % \hspace*{\fill}   % maximize separation between the subfigures
    \begin{subfigure}{\linewidth}
    \centering
        \includegraphics[width=0.7\linewidth]{images/classifier_training.pdf}
        \caption{Stage 2 at iteration $i$: Classifier training}\label{fig:classifier_training}
    \end{subfigure}%
	\caption{\textbf{Experimental setup \& training classifiers.} Generators $G$ are \textbf{\textcolor[HTML]{93C47C}{green}}, co-trained discriminators $D$ are \textbf{\textcolor[HTML]{A4C2F4}{blue}} and classifiers $C$ trained using multiple, frozen generators are \textbf{\textcolor[HTML]{C27BA0}{purple}}. Dashed borders indicate that the subnetwork is not being updated during this stage of the iteration. \subref{fig:gan_training_zeroth} Generators trained in iteration 0 are trained in the typical way. \subref{fig:classifier_training} Classifiers are trained in the second stage of all iterations, on samples drawn from subsets of the generators trained in the first stage.}
\label{fig:experiment_setup_1}
% \vspace{-0.05in}
\end{figure}

Our experiments consist of sequential rounds (``iterations''), each with two stages: first, a pool of GAN generators initialized randomly is trained, then classifiers are trained to detect samples from the generators trained in the first stage. In the first stage of the first iteration ($i = 0$), a number of GANs (DCGAN in the first setting, SG2 in the second setting) are trained independently on the training images (MNIST in the first setting, FFHQ in the second setting), as shown in \cref{fig:gan_training_zeroth}. This setup is modified slightly in later iterations (\emph{see} \cref{fig:experiment_setup_2}) as detailed below. Classifier training follows in the second stage (\cref{fig:classifier_training}) as a standard classification task where each classifier is trained on a balanced dataset of real images and images sampled from a subset of generators trained in the first stage. The second stage is the same in every iteration, always sampling images from generators trained in the first stage of the iteration.
The first stage of subsequent iterations ($i > 0$) proceeds like the first stage of the first iteration but with a modified generator loss function: generators are trained to fool not only the discriminator they are co-trained with, but also frozen classifiers from preceding iterations. To do this we modify the ``classical'' GAN generator loss function $\mathbf{\mathcal{L}}$:
\begin{dmath}
\mathbf{\mathcal{L}_{G^{(i)}}} = -\log(D^{(i)}(G^{(i)}(w)))
\label{eq:orig_gan_loss}
\end{dmath}
in one of two ways. In the first, $\mathbf{\mathcal{L}^\Sigma_{G^{(i)}}}$, generators must fool a classifier from \emph{every} preceding iteration: 
\begin{dmath}
\mathbf{\mathcal{L}^\Sigma_{G^{(i)}}} = -[\log(D^{(i)}(G^{(i)}(w))) + \phi \sum_{j=0}^{i-1}\log(C^{(j)}_0(G^{(i)}(w)))]
\label{eq:fool_all_gan_loss}
\end{dmath}
A graphical depiction of a single generator using this loss function is shown in \cref{fig:gan_training_modified}. $\phi$ is a used to weight the relative influence of classifiers. Because a classifier from each previous iteration must be fooled in order to minimize this function, we refer to it as the ``fool-all'' loss function.

The other generator loss function variation, $\mathbf{\mathcal{L}^*_{G^{(i)}}}$, relies purely on a classifier from the iteration immediately preceding the current one, rather than all preceding iterations: 
\begin{align}
\mathbf{\mathcal{L}^*_{G^{(i)}}} = -[\log(D^{(i)}(G^{(i)}(w))) + \phi \log(C^{(i-1)}_0(G^{(i)}(w)))]
\label{eq:memoryless_gan_loss}
\end{align}
This is depicted in \cref{fig:gan_training_modified_memoryless}. Because $\mathbf{\mathcal{L}^*_{G^{(i)}}}$ depends only on the current iteration and the preceding iteration, we refer to this as the ``memoryless'' loss function.

The two modifications result in markedly different training dynamics. Reported results will generally be for the ``fool-all'' $\mathcal{L}^\Sigma$ variation (\cref{fig:gan_training_modified}). When results are based on experiments using the ``memoryless'' variation $\mathcal{L}^*$ (\cref{fig:gan_training_modified_memoryless}), they will be explicitly noted as such. Classifiers are frozen (i.e., their weights are not updated) during the first stage of every iteration.

\begin{figure}[h]
	\centering
    \begin{subfigure}{\linewidth}
        \centering
        \includegraphics[width=0.7\linewidth]{images/gan_training_modified.pdf}
        \caption{Stage 1 at iteration $i$: GAN training with ``fool-all'' modified loss function}
        \label{fig:gan_training_modified}
    \end{subfigure}%
    \hspace{0.05\textwidth}
    % \hspace*{\fill}   % maximize separation between the subfigures
    \begin{subfigure}{\linewidth}
        \centering
        \includegraphics[width=0.7\linewidth]{images/gan_training_modified_memoryless.pdf}
        \caption{Stage 1 at iteration $i$: GAN training with ``memoryless'' modified loss function} \label{fig:gan_training_modified_memoryless}
    \end{subfigure}%
	\caption{\textbf{GANs trained in higher iterations.} In subsequent iterations ($i > 0$), stage 1 GAN training is modified from the first iteration ($i=0$, \emph{see} \cref{fig:gan_training_zeroth}) such that the generator $G^{(i)}_k$ learns to fool not only its co-trained discriminator $D^{(i)}_k$ but also \subref{fig:gan_training_modified} $i$ classifiers, one from each preceding iteration (\cref{eq:fool_all_gan_loss}) or \subref{fig:gan_training_modified_memoryless} a single classifier from the immediately preceding iteration (\cref{eq:memoryless_gan_loss}). At $i=1$, these two approaches are equivalent.}
\label{fig:experiment_setup_2}
\end{figure}
The classifier subscript $0$, used in Figs. \ref{fig:gan_training_modified} and \ref{fig:gan_training_modified_memoryless} (e.g., $C^{(i-1)}_0$), is purely to distinguish classifiers within the same iteration. In each iteration, multiple classifiers are trained that are initialized randomly and trained independently. When testing a GAN trained to fool the previous iteration's classifiers, classifiers used for training and testing are trained on disjoint subsets of generators, to measure generalization. For example, if $G^{(i)}_k$ is trained to fool $C^{(i-1)}_0$, and is evaluated against $C^{(i-1)}_1$, then $C^{(i-1)}_0$ and $C^{(i-1)}_1$ are trained on disjoint subsets of iteration $i-1$ generators.
% Results ------------------------------------------------------
%%%%%%%%%%%%%%%%%%%%%%%%%%%%%%%%%%%%%%%%%%%%%%%
%%%%%%%        4. Results         %%%%%%%
%%%%%%%%%%%%%%%%%%%%%%%%%%%%%%%%%%%%%%%%%%%%%%%

\section{Results}
\label{sec:results}

\subsection{MOS prediction results}
\label{subsec:mos_results}
We first evaluate our MOS-prediction performance in comparison with other approaches. In particular, we compare against NISQA~\cite{mittag2019non}, which we modified to estimate human-accessed MOS. Originally, they estimate perceptual objective listening quality assessment (POLQA)~\cite{beerends2013perceptual} scores using a CNN and BLSTM architecture. We also compare against the PMOS model proposed in~\cite{dong2020pyramid}, which is identical in structure to our PMOS model. Finally, we include our proposed SE+PMOS approach~\cite{nayem2021incorporating} (no joint training), where our PMOS model is held fixed while the SE model is training using the embeddings from the PMOS encoder. 

We use four metrics to evaluate MOS-estimation performance: mean absolute error (MAE), epsilon insensitive root mean squared error (RMSE)~\cite{rec2012p}, Pearson’s correlation coefficient $\gamma$ (PCC), and Spearman’s rank correlation coefficient $\rho$ (SRCC). 

%    Later, both models are jointly-trained for fine tuning. Our proposed PMOS model is similar of \cite{nayem2021incorporating}, however, SE models are different in structure.

%%%%%%%%%%%%%%%%%%%%%%%%%%%%%%%%%%%%%%%%%%%%%%%%%%%%
% Table 1, MOS results
%%%%%%%%%%%%%%%%%%%%%%%%%%%%%%%%%%%%%%%%%%%%%%%%%%%%
\begin{table}[t!]

\centering
\caption{Performance comparison with MOS prediction models {comparing against the ground truth MOS obtained from human subjects}. Best results are shown in \textbf{bold}.}
\label{tab:mos_results}
% \vspace{-0.5em}
\resizebox{\columnwidth}{!}{%
\begin{tabular}{| l | c c c c | }
\cline{2-5}
   \multicolumn{1}{c|}{}         & {MAE}$\downarrow$ & {RMSE}$\downarrow$ & {PCC ($\gamma$)}$\downarrow$ & {SRCC ($\rho$)}$\downarrow$ \\ \hline
   
NISQA~\cite{mittag2019non}    & 0.62 ($\pm$0.18)        & 0.7 ($\pm$0.16)      & 0.71 ($\pm$0.14)           & 0.79 ($\pm$0.15)            \\
PMOS~\cite{dong2020pyramid}                      & 0.51 ($\pm$0.15)         & 0.57 ($\pm$0.12)          & 0.88 ($\pm$0.17)           & 0.88 ($\pm$0.14)           \\
SE+PMOS~\cite{nayem2021incorporating}                     & \textbf{0.45} ($\pm$0.08) & \textbf{0.52} ($\pm$0.09) & \textbf{0.9} ($\pm$0.12) & \textbf{0.91} ($\pm$0.1)           \\
Proposed                     & \textbf{0.45} ($\pm$0.08) & \textbf{0.52} ($\pm$0.09) & \textbf{0.9} ($\pm$0.12) & \textbf{0.91} ($\pm$0.1)         \\
\hline
\end{tabular}
}
% \vspace{-2em}
\end{table}

Table~\ref{tab:mos_results} shows the results, where our proposed approach and SE+PMOS clearly outperform the other MOS prediction models according to all metrics. MAE is minimized by $0.6$ compared to the original PMOS~\cite{dong2020pyramid} approach. There is also a $0.05$ reduction in RMSE. This justifies our proposed approach that combines MOS estimation and speech enhancement tasks. Note, however, that similar results are obtained for our proposed approach and the SE+PMOS approach, which suggests that joint training (e.g., fine tuning) may help speech enhancement more than MOS prediction.  




\subsection{Speech enhancement model}
\label{subsec:se_results}
%%%%%%%%%%%%%%%%%%%%%%%%%%%%%%%%%%%%%%%%%%%%%%%%%%%%
% Table 2, SE comparison results on COSINE & VOiCES
%%%%%%%%%%%%%%%%%%%%%%%%%%%%%%%%%%%%%%%%%%%%%%%%%%%%

% Please add the following required packages to your document preamble:
% \usepackage{multirow}
% \usepackage[table,xcdraw]{xcolor}
% If you use beamer only pass "xcolor=table" option, i.e. \documentclass[xcolor=table]{beamer}
\begin{table*}[t!]
\centering
\caption{Average results of the speech enhancement models in different performance metrics. Best results are shown in \textbf{bold}.}
\label{tab:results_cosineVoices}
\resizebox{\linewidth}{!}{%
\begin{tabular}{ | l | l | c c c c | c c c c | }
\cline{3-10}
\multicolumn{1}{l}{\multirow{2}{*}{}} &                    & \multicolumn{4}{ c |}{{COSINE}}                             & \multicolumn{4}{ c |}{{VOiCES}} 
\\ \hline
\multicolumn{1}{|l|}{{models}}                 & \multicolumn{1}{c|}{{loss func.}} & {PESQ}$\uparrow$ & {SI-SDR}$\uparrow$ & {ESTOI}$\uparrow$ & {MOS-LQO}$\uparrow$ & {PESQ}$\uparrow$ & {SI-SDR}$\uparrow$ & {ESTOI}$\uparrow$ & {MOS-LQO}$\uparrow$ \\ \hline
\multicolumn{1}{|l|}{{Mixture}}                                        & {-}                                                       & {1.46} & {0.53}   & {0.62}  & {4.04}    & {1.26} & {-1.3}   & {0.48}  & {2.74}    \\ \hline
\multicolumn{1}{|l|}{}                                                        & mse                                                              & 2.68          & 2.8             & 0.8            & 3.2              & 2.3           & 1.2             & 0.69           & 3.5              \\ 
\multicolumn{1}{|l|}{}                                                        & mos~\cite{fu2019learning}                                                              & 2.8           & 3.8             & 0.82           & 4.2              & 2.37          & 1.66            & 0.74           & 5.3              \\ 
\multicolumn{1}{|l|}{}                                                        & mse+sa                                                           & 2.72          & 3.1             & 0.82           & 4                & 2.35          & 1.6             & 0.7            & 3.8              \\ 
\multicolumn{1}{|l|}{}                                                        & mos+sa                                                           & 2.89          & 4.1             & 0.85           & 4.4              & 2.42          & 1.72            & 0.77           & 5.7              \\ 
\multicolumn{1}{|l|}{\multirow{-5}{*}{SE}}                                    & sdr~\cite{kawanaka2020stable}                                                              & 2.7           & 4.5             & 0.82           & 3.4                & 2.32          & 2.01            & 0.72           & 3              \\ \hline
\multicolumn{1}{|l|}{{ }}                                 & mse                                                              & 3.1           & 4               & 0.85           & 4.2              & 2.48          & 1.8             & 0.8            & 6                \\ 
\multicolumn{1}{|l|}{{}}                                 & mse+sa                                                           & 3.19          & 4.6             & 0.93           & 4.8              & 2.54          & 2.08            & 0.86           & 6.3              \\  
\multicolumn{1}{|l|}{\multirow{-3}{*}{SE+PMOS~\cite{nayem2021incorporating}}}        & mse+sa+mos                                                       & 3.19          & 4.5             & 0.92           & \textbf{5.1}     & 2.53          & 2.06            & 0.84           & \textbf{6.5}     \\ \hline
\multicolumn{1}{|l|}{}                                                        & pesq                                                             & \textbf{3.28} & 4.4             & 0.9            & 5                & \textbf{2.67} & 2.01            & 0.83           & 6.1              \\ 
\multicolumn{1}{|l|}{\multirow{-2}{*}{MetricGAN~\cite{fu2019metricGAN}} }                             & stoi                                                             & 3.19          & 4.3             & \textbf{0.94}  & 4.8              & 2.5           & 2               & \textbf{0.87}  & 5.8              \\ \hline
\multicolumn{1}{|l|}{SSEMS~\cite{zezario2019specialized}}                                                   & qnet ($\phi=0dB$)                                                       & 2.85          & 2.99            & 0.83           & 3                & 2.4           & 1.8             & 0.7            & 2.8              \\ \hline
\multicolumn{1}{|l|}{{Chi++\textsubscript{fQSM,bS}~\cite{nayem2021towards}}}                     &    dc+cls+sa                                                              & 2.9           & 3.3             & 0.84           & 3.4              & 2.44          & 1.78            & 0.7            & 3                \\ \hline
\multicolumn{1}{|l|}{}                                 & mse+sa                                                           & 3.25          & 4.8             & \textbf{0.94}  & 4.75             & 2.64          & 2.1             & \textbf{0.87}  & 6.2              \\ 
\multicolumn{1}{|l|}{\multirow{-2}{*}{Proposed}} & mse+sa+mos                                                       & 3.25          & \textbf{4.82}   & \textbf{0.94}  & 5.04             & 2.64          & \textbf{2.13}   & \textbf{0.87}  & 6.47             \\ \hline
\end{tabular}
}
\end{table*}
For speech enhancement, we compare against a baseline approach without an attention mechanism \cite{graves2013speech}. We denote this baseline approach as SE. Five separate loss functions are applied to optimize this approach, and they are MSE, MSE plus signal approximation, MOS, signal approximation with MOS, and SDR. To compute the MOS loss function, we utilize the SE loss function from \cite{fu2019learning} which leverages objective-MOS (oMOS) ratings learned from a speech assessment model~\cite{fu2018quality}. SDR~\cite{kawanaka2020stable} loss functions are proposed in literature previously with different enhancement architectures. For the SDR loss function, the SE model is optimized using the following cost function:
\begin{align}
    \mathcal{L}_{SDR} = \sum_{n=1}^N \mathcal{K}_{\theta}  \Big( 10 \log \frac{\Vert s^n\Vert^2}{\Vert s^n-\hat{s}^n\Vert^2} \Big)
\end{align}
where $\mathcal{K}_\theta(a)=\theta\cdot \tanh(\frac{a}{\theta})$, $\theta$ is a clipping parameter, $N$ is the mini-batch size, and $s^n$ and $\hat{s}^n$ are the n\textsuperscript{th} sample of the clean and estimated speech signal in time. We use $\theta=20$ in our training. We also compare against a generative adversarial network (GAN) approach that individually optimizes with PESQ and STOI~\cite{fu2019metricGAN}. We denote this model as MetricGAN. 
% They estimate the IRM conditioned on continuous space of the discriminator label based on either PESQ or STOI target label. 
They estimate the IRM for a speech mixture conditioned on a GAN discriminator that outputs evaluation scores in continuous space (i.e. scores between 0 and 1) based on either normalized PESQ or STOI target metrics. 
We compare our model with the ensemble-based Specialized Speech Enhancement Model Selection (SSEMS) approach~\cite{zezario2019specialized} that uses Quality-Net~\cite{fu2018quality} as its objective function in a black-box manner. Quality-Net is an oMOS approach that estimates the Perceptual Evaluation of Speech Quality (PESQ) score. The SSEMS approach uses an ensemble of enhancement models, each trained on audio at specific SNRs and speaker genders. During inference, it selects the output with the highest PESQ score. SSEMS uses a SNR threshold of $20$ dB, while we use a threshold of $0$ dB for balanced training and better performance. Additionally, we conduct a comparison with our initial approach that integrates MOS embeddings in speech enhancement, as presented in \cite{nayem2021incorporating}. This model is referred to as SE+PMOS, and it does not involve joint training or the QSM language model. We evaluate SE+PMOS with varying combinations of loss functions. %We compare against a quantized speech enhancement model which utilizes a spectral language model~\cite{nayem2021towards}. This model is motivated from chimera++~\cite{wang2018alternative} in structure with BLSTM layers and deep clustering (dc) loss.
%Traditional chimera++ model estimates a phase-sensitive mask which has been applied in the task of speech enhancement in non-speech noisy conditions with multi-talker speech~\cite{wichern2019wham, yang2019improved}. However, in \cite{nayem2021incorporating}, they estimate quantized speech signal, not mask; they use cross-entropy classification (cls) loss, and signal approximation loss altogether. They report best results using per-frequency quantized spectral model (fQSM) as language model for beam search (bS) with beam size $100$. We use this model as our comparison model denoting as Chi++\textsubscript{fQSM,bS}. 
All models are trained using the experimental setup that is previously mentioned. We modify the comparison models using the code provided by the original authors.

We assess speech enhancement performance using PESQ~\cite{rix2001perceptual}, scale-invariant SDR (SI-SDR)~\cite{le2019sdr}, and extended STOI (ESTOI)~\cite{jensen2016algorithm}. In the absence of actual human quality objective, we measure the predicted MOS score of the enhanced speech, using our proposed PMOS model, since we aim to improve human-assessed speech quality. We denote this metric as MOS listener quality objective (MOS-LQO). Table~\ref{tab:results_cosineVoices} shows the average results of the different enhancement models, according to each of the performance metrics on COSINE and VOiCES dataset. As the scores of the unprocessed mixtures show, the VOiCES corpus is  more challenging than the COSINE corpus. 
With the baseline SE model, we experiment with 5 different combination of loss functions. Using the MSE loss only in SE:mse, we see improvements in objective scores, except with MOS-LQO for the COSINE data. Then we apply a MOS loss $\mathcal{L}_{mos}$ as the sole objective criterion, as proposed in \cite{fu2019learning}. Our experimental results show that this approach results in an overall improvement of $1.4$ in MOS-LQO compared to SE:mse. %We apply MOS-LQO scores of enhanced speech to calculate MOS loss $\mathcal{L}_{mos}$ as the only objective criteria as proposed in \cite{fu2019learning}, which gives improves MOS-LQO by $1.4$ overall compared with SE:mse. 
Then we separately combine the signal approximation loss with the mse loss and MOS loss (e.g., mse+sa and mos+sa). In PESQ, we gain an average of $\ge0.05$ and $\ge0.07$ compared to the models that use only the MSE loss and only the MOS loss, respectively. Furthermore, the model trained with the mos+sa loss function achieves the highest MOS-LQO score of $4.4$ and $5.7$ among all five loss functions tested with the SE model in COSINE and VOiCES dataset, respectively. This result is on average $1.15$ MOS-LQO higher than that obtained with the mse+sa loss function. These scores suggest that $\mathcal{L}_{mse}$ and $\mathcal{L}_{sa}$ maximize the overall speech intelligibility, whereas $\mathcal{L}_{mos}$ guides the model towards perceptual speech quality. Note that in all these $\mathcal{L}_{mos}$ calculations, we use a separately trained PMOS model's output without joint learning.
Lastly, we apply the SDR loss function as proposed in \cite{kawanaka2020stable}, which is used as the pre-training stage for model training. We observe an average gain of $0.9$ in SI-SDR, however, it yields a poor score according to other metrics, especially a $0.7$ loss in MOS-LQO compared to SE with mse and sa loss terms. 

SE+PMOS is separately investigated with 3 combinations of loss functions, i.e. mse, mse+sa, and mse+sa+mos. Compared with SE models, SE+PMOS with mse loss achieves $0.9$ SI-SDR and $1.75$ MOS-LQO improvements on average, which shows the benefit of incorporating the PMOS model. The SE+PMOS:mse+sa model improves the performance further with an average of $0.14$ ESTOI gain over the SE:mse+sa model. The inclusion of the mos loss gives the best MOS-LQO scores of $5.1$ and $6.5$ over all the comparison models in noisy and reverberant conditions, respectively.

%%%%%%%%%%%%%%%%%%%%%%%%%%%%%%%%%%%%%%%%%%%%%%%%%%%%
% Table 3, SE comparison test results on CHiME 5+4 
%%%%%%%%%%%%%%%%%%%%%%%%%%%%%%%%%%%%%%%%%%%%%%%%%%%%

% Please add the following required packages to your document preamble:
% \usepackage{multirow}
\begin{table*}[t!]
\centering
\caption{Average testing results of the speech enhancement models on CHiME-5 and CHiME-4 datasets. Best results are shown in \textbf{bold}.}
\label{tab:results_chime}
\resizebox{\linewidth}{!}{%
\begin{tabular}{| l | l | c c c c c | c c c c c |}
\cline{3-12}
\multicolumn{1}{l}{\multirow{2}{*}{}} &                    & \multicolumn{5}{ c |}{{CHiME-5}}                             & \multicolumn{5}{ c |}{{CHiME-4}} 
\\ \hline
\multicolumn{1}{|l|}{models}                          & \multicolumn{1}{c|}{loss func.} & PESQ$\uparrow$          & SI-SDR$\uparrow$       & ESTOI$\uparrow$         & MOS-LQO$\uparrow$      & WER\%$\downarrow$         & PESQ$\uparrow$          & SI-SDR$\uparrow$        & ESTOI$\uparrow$         & MOS-LQO$\uparrow$      & WER\%$\downarrow$         \\ \hline
\multicolumn{1}{|l|}{Mixture}                & -                               & 1.7           & 2.4          & 0.52          & 3.8          & 152.1         & 1.96          & 2.86          & 0.6           & {4.6} & {33.7} \\ \hline
\multicolumn{1}{|l|}{SE}                              & mos+sa                          & {2.25} & {3.9} & {0.62} & {4}   & {96.4} & {2.32} & {5.22} & {0.63} & {5}   & {25.6} \\ \hline
\multicolumn{1}{|l|}{SE+PMOS}                         & mse+sa+mos                      & 2.37          & 6.1          & 0.67          & 4.4          & 84.5          & 2.45          & 7.6           & 0.7           & 5.8          & 22.6          \\ \hline
\multicolumn{1}{|l|}{\multirow{2}{*}{MetricGAN}}      & pesq                            & \textbf{2.44} & {6.3} & {0.65} & {4.1} & {94.8} & \textbf{2.51} & {7}    & {0.68} & {5.3} & {19.7} \\ 
\multicolumn{1}{|l|}{}                                & stoi                            & 2.39          & 6.2          & \textbf{0.71} & 4.1          & 91.3          & 2.45          & {6.45} & \textbf{0.73} & 5.6          & 21.5          \\ \hline
\multicolumn{1}{|l|}{\multirow{2}{*}{Proposed}} & mse+sa                          & 2.41          & 7.1          & {0.68} & 4.7          & \textbf{78.3} & 2.5           & {7.9}  & 0.72          & 5.76         & \textbf{18.1} \\ 
\multicolumn{1}{|l|}{}                                & mse+sa+mos                      & 2.41          & \textbf{7.3} & {0.68} & \textbf{4.9} & 79.4          & {2.5}  & \textbf{8.61} & \textbf{0.73} & \textbf{6}   & 18.9          \\ \hline
\end{tabular}}
\end{table*}
MetricGAN optimizes PESQ or STOI, therefore, it outperforms other comparison models in terms of PESQ and ESTOI, although the scores for the SE+PMOS approaches are higher according to the other evaluation metrics even though these metrics are not leveraged during training. 
SSEMS yields the lowest scores across all metrics compared with SE+PMOS and MetricGAN approaches, though we do parameter tuning for this model.
Chi++\textsubscript{fQSM,bS} estimates quantized speech, and the results show that it affects the traditional objective functions. This performs poorly compared with the SE+PMOS and MetricGAN approaches, however, on average, it outperforms SSEMS in all criteria, and the SE models in terms of PESQ. With the MOS-LQO criteria, it fails to produce good scores. This points out the importance of incorporating perceptual features during enhancement, which Chi++\textsubscript{fQSM,bS} clearly lacks.

We calculate the performance of our proposed model using two combinations of loss functions. 
Using only mse and sa loss terms, we achieve the highest ESTOI scores for both corpora, though these results are nearly identical to the model trained with all three loss terms. Using $\mathcal{L}$ (eq:\ref{eq:loss}) in our proposed model, we obtain the highest SI-SDR scores while maintaining similar PESQ and ESTOI performance as compared to the best-performing model. Specifically, our proposed model achieves the highest ESTOI score and an average PESQ score that is only $0.03$ less than that of the best performing MetricGAN:pesq model.
Contrasting with the Chi++\textsubscript{fQSM,bS} model, which uses spectral language model to estimate quantized speech, our proposed approach outperforms the quantized model according to all metrics, which proves the significance of joint learning.% to direct speech enhancement model towards perceptually better speech using a speech quality assessment model.
When comparing MOS-LQO scores, our proposed:mse+sa+mos model achieves better scores than the other models except the SE+PMOS:mse+sa+mos model with an average of only $0.05$ declination. Thus, the inclusion of a spectral language model helps the model proposed (e.g., mse+sa+mos) to estimate better quality speech according to the overall evaluation criteria. 
It is important to note that our proposed approach performs best according to SI-SDR in both noisy and reverberant environments, where this metric is not used by any of the approaches during optimization.  

We further examine our approaches using completely unseen corpora. We test models with the CHiME-5 and CHiME-4 corpora where the models are trained from the COSINE dataset according to the system setup mentioned in section~\ref{subsec:setup}. Table~\ref{tab:results_chime} shows the performance evaluated according to PESQ, SI-SDR, ESTOI, MOS-LQO, and word error rate (WER). To calculate WER, we use the conventional ASR baseline that is provided with CHiME-5 and CHiME-4 dataset. We investigate WER with both GMM based ASR and end-to-end ASR, however, we find that the end-to-end approach results in a higher error compared to the GMM baseline. This might happen due to larger data requirements of the end-to-end ASR system as mentioned in \cite{barker2018fifth}. Therefore, we use the GMM ASR approach to compare the WER performance of the enhancement models.
From the scores of mixtures, we find that CHiME-5 is more challenging than CHiME-4 with a $118.8\%$ higher WER and a $0.46$ lower SI-SDR. Our proposed approach yields the best MOS-LQO scores with $4.9$ with CHiME-5 and $6$ with CHiME-4 data. The proposed mse+sa model results in the lowest WER of $78.3$ and $18.1$ using CHiME-5 and CHiME-4, respectively. Note that the WER of the GMM baseline ASR for the CHiME-5 challenge is $72.8$ in binaural and $91.7$ in single array conditions. Here our approaches enhance monaural speech, a more challenging condition. Our proposed approach outperforms other comparison models in terms of SI-SDR with a $5.29$ average improvement compared to others. According to PESQ and ESTOI metrics, MetricGAN variants give the best performace, however, proposed model's performance is $0.02$  and $ 0.015$ lower according to PESQ and ESTOI, respectively, for the best performing MetricGAN models. Hence, our proposed approach is effective on out-of-vocabulary scenario trained by a comparable dataset.


% \nayem{*** Possibly add graphs of evaluation metrics vs SNRs.}

%%%%%%%%%%%%%%%%%%%%%%%%%%%%%%%%%%%%%%%%%%%%%%%%%%%%
% Table 3, DNSMOS results
%%%%%%%%%%%%%%%%%%%%%%%%%%%%%%%%%%%%%%%%%%%%%%%%%%%%
% \begin{table}[thb!]

% \centering
% \caption{Average MOS ratings of the speech enhancement modes on CHiME-4 and CHiME-5 datasets using DNSMOS P.835~\cite{reddy2022dnsmos}. Best results are shown in \textbf{bold}.}
% \label{tab:dnsmos_results}
% % \vspace{-0.5em}
% \resizebox{\columnwidth}{!}{%
% \begin{tabular}{| l | c c | }
% \cline{2-3}
%   \multicolumn{1}{c|}{}         & {CHiME-4} & {CHiME-5} \\ \hline
   
% Mixture   & 1.54 ($\pm$0.85)         & 1.3 ($\pm$1.1)                \\
% PMOS+SE                      & 4.28 ($\pm$0.9)       & 3.67 ($\pm$1.3)\\
% MetricGAN                    & 4.26 ($\pm$0.87) & 3.5 ($\pm$1.34)          \\
% Proposed                     & \textbf{4.32} ($\pm$0.8)& \textbf{3.8} ($\pm$1.41)            \\ \hline
% Clean                     & 4.67 ($\pm$1.2) & -      \\
% \hline
% \end{tabular}
% }
% % \vspace{-2em}
% \end{table}

%%%%%%%%%%%%%%%%%%%%%%%%%%%%%%%%%%%%%%%%%%%%%%%%%%%%
% Fig 3, DNSMOS results plot
%%%%%%%%%%%%%%%%%%%%%%%%%%%%%%%%%%%%%%%%%%%%%%%%%%%%

\begin{figure}[b!]
    \centering
\begin{tikzpicture}
	\begin{axis}[
	    cycle list/Dark2-4,
		boxplot/draw direction = y,
		boxplot/box extend=0.8,
% 		x=3em,
% 		x axis line style = {opacity=0.6},
		axis x line* = bottom,
		axis y line = left,
		enlarge y limits,
		ymajorgrids,
		xtick = {1, 2, 3, 4, 5, 6, 7, 8},
		xticklabel style = {align=center, font=\small, rotate=60, alias={xtick-\ticknum}},
		xticklabels = {Mixture, SE+PMOS, MetricGAN, Proposed, Mixture, SE+PMOS, MetricGAN, Proposed},
% 		xtick style = {draw=none}, % Hide tick line
		ylabel = {MOS},
		ytick = {1, 2, 3, 4, 5},
	]
	
	\addplot+[
        boxplot prepared={
        lower whisker=1, lower quartile=1.45,
        median=1.74,
        upper quartile=2.5, upper whisker=4.05, }, fill, draw=black]
        coordinates {}
        node[above, color=black] at
        (boxplot box cs: \boxplotvalue{median},.5)
        {\scriptsize \pgfmathprintnumber{\boxplotvalue{median}}};
    \addplot+[
        boxplot prepared={
        lower whisker=1.38, lower quartile=1.84,
        median=2.28,
        upper quartile=3.1, upper whisker=4.3, }, fill, draw=black]
        coordinates {}
        node[above, color=black] at
        (boxplot box cs: \boxplotvalue{median},.5)
        {\scriptsize \pgfmathprintnumber{\boxplotvalue{median}}};
    \addplot+[
        boxplot prepared={
        lower whisker=1.3, lower quartile=1.75,
        median=2.13,
        upper quartile=3.2, upper whisker=4.1, }, fill, draw=black]
        coordinates {}
        node[above, color=black] at
        (boxplot box cs: \boxplotvalue{median},.5)
        {\scriptsize \pgfmathprintnumber{\boxplotvalue{median}}};
    \addplot+[
        boxplot prepared={
        lower whisker=1.4, lower quartile=1.9,
        median=2.46,
        upper quartile=3.16, upper whisker=4.34, }, fill, draw=black]
        coordinates {}
        node[above, color=black] at
        (boxplot box cs: \boxplotvalue{median},.5)
        {\scriptsize \pgfmathprintnumber{\boxplotvalue{median}}};
        
    \addplot+[
        boxplot prepared={
        lower whisker=1.0, lower quartile=1.35,
        median=1.64,
        upper quartile=2.39, upper whisker=4.18, }, fill, draw=black]
        coordinates {}
        node[above, color=black] at
        (boxplot box cs: \boxplotvalue{median},.5)
        {\scriptsize \pgfmathprintnumber{\boxplotvalue{median}}};
    \addplot+[
        boxplot prepared={
        lower whisker=1.31, lower quartile=1.8,
        median=2.18,
        upper quartile=2.76, upper whisker=4.24, }, fill, draw=black]
        coordinates {}
        node[above, color=black] at
        (boxplot box cs: \boxplotvalue{median},.5)
        {\scriptsize \pgfmathprintnumber{\boxplotvalue{median}}};
    \addplot+[
        boxplot prepared={
        lower whisker=1.26, lower quartile=1.71,
        median=2.06,
        upper quartile=3.17, upper whisker=4.32, }, fill, draw=black]
        coordinates {}
        node[above, color=black] at
        (boxplot box cs: \boxplotvalue{median},.5)
        {\scriptsize \pgfmathprintnumber{\boxplotvalue{median}}};
    \addplot+[
        boxplot prepared={
        lower whisker=1.34, lower quartile=1.85,
        median=2.25,
        upper quartile=3.07, upper whisker=4.48, }, fill, draw=black]
        coordinates {}
        node[above, color=black] at
        (boxplot box cs: \boxplotvalue{median},.5)
        {\scriptsize \pgfmathprintnumber{\boxplotvalue{median}}};
        
	\end{axis}
	
	\path (0,0) coordinate (P);
    \draw [thick,decoration={brace,mirror,raise=5em},decorate] (xtick-0|-P) -- (xtick-3.5|-P) 
        node[midway,yshift=-6em]{CHiME-4};
    \draw [thick,decoration={brace,mirror,raise=5em},decorate] (xtick-4|-P) -- (xtick-7.5|-P) 
        node[midway,yshift=-6em]{CHiME-5};

    % \node[text width=3cm] at (1.54,0.5) 
    % {\scriptsize 1.54};

\end{tikzpicture}

\caption{MOS ratings of the speech enhancement modes on CHiME-4 and CHiME-5 datasets using DNSMOS P.835.}
    % \vspace{-2em}
\label{fig:dnsmos_results}
    % \vspace{-0.4cm}
\end{figure}

\subsection{Perceptual quality evaluation}
\label{subsec:dnsmos}

We finally evaluate our model using P.835 metric~\cite{reddy2022dnsmos} to measure perceptual quality. We calculate the DNSMOS score on a scale of $[1-5]$ ($1$ = worst, $5$ = best) for the mixture, PMOS+SE, MetricGAN, and our proposed models using the CHiME-4~\cite{vincent2017analysis} and CHiME-5~\cite{barker2018fifth} datasets (simulated and real-recording). Figure~\ref{fig:dnsmos_results} shows the scores. With CHiME-4, the original mixture scores range from $1.45$ to $2.5$ with a median of $1.74$. Our proposed model achieves a median MOS of $2.46$, which is higher than the others. Fon CHiME-5, the original mixture scores range from $1.0$ to $4.18$. Our proposed model outperforms the others with a median of $2.25$. Our proposed model and PMOS+SE have smaller standard deviations compared to MetricGAN. Overall, our proposed model improves noisy speech in both the acoustic and perceptual aspects. 




% \subsection{Listening results}
% \label{subsec:listening_results}

% We conduct an IRB-approved listening study using Amazon Mechanical Turk to conceive the perceptual quality of enhanced speech assessed by normal-hearing listeners. 

% This study follows the design structure of \cite{nayem2021towards} and figure~\ref{fig:survey} shows the actual listener study interface of a single question. The study is conducted as follows, the participant will listen to two audio signals, one is enhanced and the other is clean audio as reference.  Then they provide a preference score using a Likert scale. The scale ranges from $-3$ to $+3$, where $-3$ refers to a strong preference towards the first signal, $+3$ refers to a strong preference towards the second signal, and $0$ refers to no preference. Before providing a score, the participant can listen to the signals as many as times they like, where the scores are not limited to integer values. The two signals are randomly selected, and the participant listens to different audio clips in each question. The audio clips are chosen from the CHiME-5 and CHiME-4 corpus spoken by both males and females in equal proportion. Prior to actual survey questions, each participants has to pass eligibility test and make themselves familiar with the upcoming study session by going through a practice session. The structure of this practice session is similar to the actual study, however, speakers' voice and audio clips which participants hear in practice session are not used in the actual study. A tentative feedback is provided in the practice session to give a guideline to the participants, however, to avoid biases and leading answers, the feedback is provided in a form of range where the expected answer should reside.



%  \begin{figure}[thb!]
%     \centering
%     \includegraphics[width = 0.5\linewidth]{IEEEtran/figs/survey.png}
%     % \vspace{-2em}
%     \caption{A question of actual listener study interface conducted on MTurk.}
%     \label{fig:survey}
%     % \vspace{-2em}
% \end{figure}

% \nayem{***One paragraph on the statistics of the conducted study.}
% The study session contains total 30 questions, which is preceded by a practice session of 7 questions. Ten participants (9 male, 1 female) who are native English speakers over the age of 18 participated, where a headset/headphone was required to be worn. On average, participants took 14 minutes to complete the study, they were given $\$3$ monetary incentive.


\section{Discussion}
\label{sec:discuss}

Our proposed model outperforms all comparison models on SI-SDR metrics for both seen and unseen datasets, without optimization of any of the models (Table \ref{tab:results_cosineVoices}, \ref{tab:results_chime}). This means that our approach improves speech quality by minimizing the distortion ratio when separated from the noise component. Additionally, our models yield the best MOS-LQO ratings on real-world captured audios (CHiME datasets, Table \ref{tab:results_chime}). These results are consistent with the findings of \cite{zezario2022deep, nayem2021incorporating} that incorporating embeddings from a speech assessment model improves SE performance, and the results of \cite{braun2022effect} that using MOS loss during model optimization leads to higher MOS-LQO scores. Our proposed approach achieves PESQ and ESTOI scores that are only slightly lower than those of the best-performing model, with a difference of only $0.03$ and $0.01$, respectively. This indicates that speech quality and intelligibility metrics are closely related to the subjective speech quality metric (MOS-LQO), and that these metrics can be improved without explicit optimization. Furthermore, our proposed model achieves the best average DNSMOS scores with low standard deviations on CHiME datasets (Figure \ref{fig:dnsmos_results}), indicating that it is effective in a wide range of real-world noise levels. This is a desirable quality for an effective SE model to be effective not only in high SNRs and limited noisy environments, but also in large SNR ranges and real-world conditions such as those offered by the CHiME dataset.

When comparing our proposed model that uses mse+sa+mos loss to the PMOS+SE model (as shown in Table \ref{tab:results_chime}), we can observe significant improvements in all performance metrics. As both models use the same loss function, the improvements are attributed to the incorporation of LM and the joint learning method. Moreover, we found that these two models exhibit similar performance on the MOS prediction (Table \ref{tab:mos_results}), indicating that the benefits of joint learning mostly impact the enhancement part of the model.

An intriguing finding is that our proposed model shows a decline in WER\% when MOS loss is incorporated, especially for larger real-world recordings such as CHiME-5, with degradation up to $1.1$. Although our study is not primarily concerned with ASR performance, this suggests a potential trade-off between ASR accuracy and subjective speech quality scores. Further investigation is needed to comprehend this relationship.

Our proposed method demonstrates that training a speech enhancement (SE) model and a MOS-based speech assessment model jointly can lead to better speech quality measured by objective metrics such as perceptual quality, intelligibility, and MOS ratings. However, we acknowledge that our study's use of subjective MOS (sMOS) estimation instead of actual human listeners may introduce discrepancies between MOS-LQO and human-rated MOS, which could impact our findings. To address this limitation, we plan to conduct sMOS evaluation by human listeners in future work. Although we used the same MOS prediction model for all comparison models, we believe that incorporating human-rated sMOS evaluations will provide more robust insights into our proposed method's effectiveness.
For computing loss terms, we opt for the MSE loss function along with a bi-gram language model that considers only time-along transitions. Our aim is to keep the model simple and focus on the effectiveness of our approach. However, we acknowledge that using different loss functions for different loss components and employing a more complex language model that considers both temporal and spectral transition levels can be beneficial. We plan to explore these possibilities in our future work.



% Conclusion -----------------------------------------
\section{Discussion and Limitations}

Although we can ablate concepts efficiently for a wide range of object instances, styles, and memorized images, our method is still limited in several ways. First, while our method overwrites a target concept, this does not guarantee that the target concept cannot be generated through a different, distant text prompt. We show an example in \reffig{limitation} (a), where after ablating {\menlo Van Gogh}, the model can still generate {\menlo starry night painting}. However, upon discovery, one can resolve this by explicitly ablating the target concept {\menlo starry night painting}. Secondly, when ablating a target concept, we still sometimes observe slight degradation in its surrounding concepts, as shown in \reffig{limitation} (c). 

\nupur{Our method does not prevent a downstream user with full access to model weights from re-introducing the ablated concept~\cite{ruiz2022dreambooth,kumari2022multi,gal2022image}. Even without access to the model weights, one may be able to iteratively optimize for a text prompt with a particular target concept. Though that may be much more difficult than optimizing the model weights, our work does not guarantee that this is impossible.}

Nevertheless, we believe every creator should have an ``opt-out'' capability. We take a small step towards this goal, creating a computational tool to remove copyrighted images and artworks from large-scale image generative models.

% References ---------------------------------------------------------------------------
\clearpage

% \appendix
% % Appendix -----------------------------------------
% \newpage
\appendix

\section*{\LARGE Appendix}


\section{Dataset Details}
\label{sec:dataset_details}

This section describes the details about the dataset we used in experiments (Section~\ref{sec:experiment}).

We use "tiny" version of Taskonomy dataset provided by \citep{taskonomy2018}, which consists of images and labels collected from 35 different buildings.
We use the train and val split for training and early-stopping, respectively, and use the "muleshoe" building included in the test split for evaluation.

To demonstrate our universal few-shot learner, we use ten dense prediction tasks in Taskonomy dataset~\citep{taskonomy2018}, which are semantic segmentation (SS), surface normal (SN), Euclidean distance (ED), Z-buffer depth (ZD), texture edge (TE), occlusion edge (OE), 2D keypoints (K2), 3D keypoints (K3), reshading (RS), and principal curvature (PC).
All labels are normalized into $[0, 1]$ with task-specific pre-processing.
For details on the pre-processing, we refer readers to \cite{taskonomy2018}.
Based on the annotations provided by Taskonomy, we preprocess some tasks to increase the diversity of tasks.
Specifically, we modify three single-channel tasks that can be easily augmented: Euclidean distance, texture edge, and occlusion edge.
\begin{enumerate}[leftmargin=0.5cm]
    \item 
    \textbf{Texture edge} (TE) labels are generated by applying Sobel edge detector~\citep{kanopoulos1988design} to RGB images, which consists of a Gaussian filter and image gradient computation.
    The Gaussian filter has two hyper-parameters, namely kernel size and the standard deviation, where adjusting those hyper-parameters yield different \emph{thickness} of detected edges.
    We use three different sets of hyper-parameters -- $(3, 1), (11, 2), (19, 3)$ -- to produce $3$-channel labels.
    We give an example of each channel of TE task in Figure~\ref{fig:texture_edge_augmentation}.
    
    \item
    \textbf{Euclidean distance} (ED) labels consists of pixel-wise depth map, where the depth is computed by the Euclidean distance from each image pixel to the camera's optical center.
    As this task is very similar to the Z-buffer depth prediction (ZD) whose label pixels are the distance from each image pixel to the camera plane, we augment the ED task by segmenting the depth range and re-normalizing within each segment.
    Specifically, we compute the $5$-quantiles of the pixel-wise depth labels in the whole dataset, then use each quantile as different channels after re-noramlization into $[0, 1]$.
    Thus the objective of each channel of the augmented ED task is to predict Euclidean distance within a specific range, where the ranges are disjoint for different channels.
    We give an example of each channel of ED task in Figure~\ref{fig:euclidean_distance_augmentation}.
    To visualize 5-channel labels, we average the first and the second channels as "R"-channel, the third and the fourth channels as "G"-channel, and use the fifth channel as "B"-channel.
    
    \item
    \textbf{Occlusion edge} (OE) labels are similar to texture edge, but they are constructed to depend on only the 3D geometry rather than color or lighting~\citep{taskonomy2018}.
    We observe that the channel augmentation by quantiles (that we apply to Euclidean distance task) can fairly diversify the labels.
    Therefore, we augment the OE labels into 5-channel labels, where we visualize them similar to the ED labels.
    We give an example of each channel of OE task in Figure~\ref{fig:occlusion_edge_augmentation}.
\end{enumerate}

Also, for semantic segmentation, we exclude three classes ("bottle", "toilet", "book"), as little images of the classes are included in the Taskonomy dataset.
The 12 classes we used in experiments are: "chair", "couch", "plant", "bed", "dining table", "tv", "mircrowave", "oven", "sink", "fridge", "clock", and "base".

\begin{figure}[ht!]
    \centering
    \includegraphics[width=0.8\textwidth]{figure_files/Texture_Edge_Augmentation.pdf}
    \caption{Channel augmentation on texture edge prediction (TE) task. We apply three different sets of hyper-parameters (kernel size, standard deviation) in Sobel edge detector to generate a 3-channel edge task.
    Second to Fourth columns show the augmented channel with different kernel size and standard deviation, where the last column shows the 3-channel label visualized as RGB.}
    \label{fig:texture_edge_augmentation}
\end{figure}
\begin{figure}[ht!]
    \centering
    \includegraphics[width=\textwidth]{figure_files/Euclidean_Distance_Augmentation.pdf}
    \caption{Channel augmentation on Euclidean distance prediction (ED) task. We compute 5-quantiles of the pixel-wise label distribution, and use each $p$-th 5-quantile as each channel after re-normalizing into $[0, 1]$.
    Second to Fifth columns show the augmented channel with different quantile, where the last column shows the 5-channel label visualized as RGB.}
    \label{fig:euclidean_distance_augmentation}
\end{figure}
\begin{figure}[ht!]
    \vspace{-0.2cm}
    \centering
    \includegraphics[width=\textwidth]{figure_files/Occlusion_Edge_Augmentation.pdf}
    \caption{Channel augmentation on occlusion edge prediction (OE) task. We compute 5-quantiles of the pixel-wise label distribution, and use each $p$-th 5-quantile as each channel after re-normalizing into $[0, 1]$.
    Second to Fifth columns show the augmented channel with different quantile, where the last column shows the 5-channel label visualized as RGB.}
    \label{fig:occlusion_edge_augmentation}
\end{figure}


\clearpage
\section{Implementation Details}
\label{sec:implementation_details}

This section describes the implementation details in our experiments (Section~\ref{sec:experiment}).

\subsection{Architecture Details of VTM}
\label{sec:arch-vtm}
\paragraph{Encoders and Decoders}
We employ BEiT-B architecture~\citep{bao2021beit} pretrained on Imagenet-22k dataset~\citep{deng2009imagenet} with $224 \times 224$ resolution as our image encoder.
For our label encoder and decoder, we follow the DPT-B architecture~\citep{ranftl2021vision}.
Specifically, we use a randomly initialized ViT-B~\citep{dosovitskiy2020image} as label encoder $g$ and extract features from $3, 6, 9, 12$-th layers of the encoder to form multi-level label features (label tokens).
Similarly, we extract multi-level image features (image tokens) from $3, 6, 9, 12$-th layers of the image encoder (BEiT).
As the DPT-B architecture decodes four-level features using RefineNet-based decoder~\citep{lin2017refinenet}, we pass the predicted query label features from matching module at each layer to the decoder.
As the label values of tasks in Taskonomy are normalized to $[0, 1]$, we use a sigmoid activation function at the head of the decoder to produce values in $[0, 1]$.
To predict semantic segmentation task whose label values are discrete (either $0$ or $1$), we discretize the predicted label with threshold $0.1$.

\paragraph{Matching Modules}
In the implementation of the matching module with multihead attention, we adopt three conventions in vision transformer~\citep{dosovitskiy2020image} which slightly modifies the equations described in Section~\ref{sec:architecture}.
Recall that the matching module is computed on three input matrices $\mathbf{q}\in\mathbb{R}^{M\times d}$ and $\mathbf{k},\mathbf{v}\in\mathbb{R}^{NM\times d}$ as follows:
\begin{align}
    \text{MHA}(\mathbf{q},\mathbf{k},\mathbf{v}) &= \text{Concat}(\mathbf{o}_1, ..., \mathbf{o}_H)w^O, \\
    \text{where }\mathbf{o}_h &= \text{Softmax}\left(\frac{\mathbf{q}w_h^Q(\mathbf{k}w_h^K)^\top}{\sqrt{d_H}}\right)\mathbf{v}w_h^V,
\end{align}
where $H$ is number of heads, $d_H$ is head size, and $w_h^Q,w_h^K,w_h^V\in\mathbb{R}^{d\times d_H}$, $w^O\in\mathbb{R}^{Hd_H\times d}$.
First, we perform layer normalization~\citep{ba2016layer} before each input projection matrices $w_h^Q, w_h^K, w_h^V$ and after the output projection matrix $w^O$, where we share the layer normalization parameters for $w_h^Q$ and $w_h^K$.
Second, we add a residual connection with GELU non-linearity~\citep{hendrycks2016gaussian} after gathering the outputs from multiple heads as follows:
\begin{align}
    \text{MHA}(\mathbf{q}, \mathbf{k}, \mathbf{v}) &= \mathbf{o} + \text{GELU}(\mathbf{o}w^O), \\
    \text{where}~\mathbf{o} &= \text{Concat}(\mathbf{o}_1, \mathbf{o}_2, \cdots, \mathbf{o}_H).
\end{align}
Finally, we apply Dropout~\citep{srivastava2014dropout} with rate 0.1 in the attention scores.

\subsection{Architecture Details of Baselines}
\label{sec:arch-baseline}
\paragraph{Encoders and Decoders}
For the supervised learning baselines based on transformer encoder (DPT and InvPT), we use the same encoder backbone with ours (BEiT pretrained on ImageNet-22k).
We use the decoder of DPT-B configuration in \cite{ranftl2021vision} for DPT as ours, and use the original multi-task decoder implementation provided by \cite{ye2022inverted} for InvPT.
For few-shot learning baselines (HSNet, VAT, DGPNet), we use ResNet-101~\citep{he2016deep} pretrained on ImageNet-1k~\citep{deng2009imagenet} as their encoder backbones, which is their best configuration.
For the other architectural details, we follow the original implementation of each method provided by \cite{min2021hypercorrelation} (HSNet), \cite{hong2022cost} (VAT), and \cite{johnander2021dense} (DGPNet).

\paragraph{Modification on Few-shot Baselines}
As HSNet and VAT are designed for semantic segmentation, we slightly modify their architectures to train them on general dense prediction tasks.
Specifically, both models involve a binary masking operation to filter out support image features using their labels (which are assumed to be binary), before computing 4D correlation tensor between support and query feature pixels.
For continuous labels of general dense prediction tasks, the binary masking becomes pixel-wise multiplication with labels.
However, as the correlation is computed by cosine similarity between feature pixels that is norm-invariant, all non-zero feature pixels with the same direction are treated in the same manner.
This make them unable to discriminate different non-zero label values, \emph{e.g.}, correlation between query and support feature pixels would be the same regardless of the assigned support label values. 
Therefore, we move the masking operation to after computing the cosine-similarity, so that the models can recognize different non-zero label values through different norms of the masked features by (non-binary) labels.

We use the DGPNet without modification as it is based on a regression method (Gaussian Processes) which is inherently applicable to general dense prediction tasks with continuous labels.


\subsection{Training Details}

\paragraph{Training}
We train all models with 300,000 iterations using the Adam optimizer~\citep{kingma2015adam}, and use \emph{poly} learning rate schedule~\citep{liu2015parsenet} with base learning rates $10^{-5}$ for pre-trained parameters and $10^{-4}$ for parameters trained from scratch.
The models are early-stopped based on the validation metric.
At each episodic training of iteration, we sample a batch of episodes with size 8.
In each episode, we construct a 5-channel task from the training tasks $\mathcal{T}_\text{train}$ by first splitting all channels of training tasks and randomly sample 5 channels among them.
Then support and query sets are sampled for the selected channels, where we use support and query size of 4 for Ours and DGP, while using 1 for HSNet and VAT as they only supports 1-shot training.
To train DPT, we construct a batch of each target task $\mathcal{T}_\text{test}$, whose channels are given at once, with batch size $64$.
To train InvPT, we construct a batch of all ten tasks, whose channels are all given at once, while using batch size $16$ due to its large memory consumption.

\paragraph{Data Augmentation}
We apply random crop (from $256 \times 256$ resolution to $224 \times 224$) and random horizontal flip to images, where the random horizontal flip is applied except for surface normal labels as their values are sensitive to the horizontal direction (flipping images and labels together changes the semantics of the task).
As we apply random crop during training, the resolution of test images ($256 \times 256$) differs from the training images.
To evaluate the models with consistent resolution, we perform five-crop (cropping the four corners and center of an image) to test query images so that the model also predicts five-cropped labels, then aggregate them by averaging the overlapping regions to produce final prediction for evaluation of resolution $(256 \times 256)$.
For few-shot models, we apply center crop to support images at test-time.

\paragraph{Task Augmentation}
For episodic training of few-shot models, we further apply two kinds of task augmentation.
First, for each channel of $C$-channel labels sampled at each episode ($C=5$ in our experiments), we apply random jittering and gaussian blur on each channel independently.
Then we apply MixUp~\citep{zhang2018mixup} on the augmented channels and auxiliary channels which are additionally sampled from the training tasks $\mathcal{T}_\text{train}$, to create a linearly interpolated label of two channels.
We apply the task augmentation consistently in each episode to preserve the task identity.


\clearpage
\section{Additional Results}
\label{sec:additional_results}

This section provides additional results on our experiments (Section~\ref{sec:experiment}).


\subsection{Additional Results on Ablation Study}
\label{sec:additional_results_on_ablation_study}

\subsubsection{Sensitivity to the Choice of Support Set}
\label{sec:support_set_sensitivity}
As discussed in Section~\ref{sec:experiment}, we evaluate the $10$-shot performance of our VTM with four different support sets that are disjointly sampled from the training data $\mathcal{D}_\text{train}$.
We report the results in Table~\ref{tab:support_set_choice}, which shows that our model is robust to the choice of support set.
We use the first support set ($\#1$) in Table~\ref{tab:support_set_choice} for comparison with other baselines or ablated variants in Section~\ref{sec:experiment}, due to the huge computational cost for evaluating few-shot baselines HSNet and VAT.

\begin{table}[ht]
\caption{Ablation study on the choice of support set. We disjointly sample four different support sets and report the $10$-shot performance on each set, with the mean and standard deviation.}
\label{tab:support_set_choice}
\begin{center}
    \renewcommand{\arraystretch}{1.5}
    \renewcommand{\aboverulesep}{0pt}
    \renewcommand{\belowrulesep}{0pt}
    \setlength\tabcolsep{2pt}
    \small
    \begin{tabular}{c|cc|cc|cc|cc|cc}
        \toprule
        \multirow{4}{*}{Support set} &
        \multicolumn{10}{c}{Tasks} \\
        
        \cmidrule{2-11}
        &
        \multicolumn{2}{c|}{Fold 1} & \multicolumn{2}{c|}{Fold 2} & \multicolumn{2}{c|}{Fold 3} & 
        \multicolumn{2}{c|}{Fold 4} & \multicolumn{2}{c}{Fold 5} \\
        
        \cmidrule{2-11}
        &
        SS & SN & ED & ZD & TE & OE & K2 & K3 & RS & PC \\
        &
        mIoU ↑ & mErr ↓ & RMSE ↓ & RMSE ↓ & RMSE ↓ & RMSE ↓ & RMSE ↓ & RMSE ↓ & RMSE ↓ & RMSE ↓ \\
        
        \midrule
        \# 1 &
        0.4097 & 11.4391 & 0.0741 & 0.0316 & 0.0791 & 
		0.0912 & 0.0639 & 0.0519 & 0.1089 & 0.0420 \\

        \# 2 &
        0.4190 & 11.8860 & 0.0845 & 0.0338 & 0.0839 & 
		0.0926 & 0.0629 & 0.0497 & 0.1131 & 0.0437 \\

        \# 3 &
        0.3781 & 11.7418 & 0.0776 & 0.0343 & 0.0807 & 
		0.0944 & 0.0656 & 0.0494 & 0.1101 & 0.0425 \\

        \# 4 &
        0.4017 & 11.6203 & 0.0794 & 0.0362 & 0.0799 & 
		0.0908 & 0.0672 & 0.0502 & 0.1158 & 0.0424 \\
		
	\midrule
        Mean &
        0.4021 & 11.6718 & 0.0789 & 0.0340 & 0.0809 & 
		0.0922 & 0.0649 & 0.0503 & 0.1120 & 0.0427 \\

        Std. &
        0.0152 & 0.1640 & 0.0038 & 0.0016 & 0.0018 & 
		0.0014 & 0.0016 & 0.0010 & 0.0027 & 0.0006 \\
		
        \bottomrule
        
    \end{tabular}
\end{center}
\end{table}


% \paragraph{Ablation Study on Training Procedure}
\subsubsection{Ablation Study on Training Procedure}
\label{sec:training_procedure}
To understand the source of the generalization performance of our method more clearly, we conduct an ablation study on training procedure.
We compare four models based on DPT architecture with different training procedures as follows.
\begin{itemize}[leftmargin=0.5cm]
    \item \textbf{M1}: Randomly initialized DPT, 10-shot trained.
    \item \textbf{M2}: DPT with BEiT pre-trained encoder, 10-shot fine-tuned.
    \item \textbf{M3} (Ours w/o Matching): DPT with BEiT pre-trained encoder, multi-task trained with task-specific bias tuning, and then 10-shot fine-tuned.
    \item \textbf{M4} (Ours): DPT with BEiT pre-trained encoder, meta-trained with task-specific bias tuning, and then 10-shot fine-tuned.

\end{itemize}

\begin{table}[ht]
\vspace{-0.2cm}
\caption{10-shot learning performance of ablated variants of DPT and Ours.}
\vspace{-0.2cm}
\label{tab:training_procedure_ablation}
\begin{center}
    \renewcommand{\arraystretch}{1.5}
    \renewcommand{\aboverulesep}{0pt}
    \renewcommand{\belowrulesep}{0pt}
    \setlength\tabcolsep{2pt}
    \small
    \begin{tabular}{c|cc|cc|cc|cc|cc}
        \toprule
        \multirow{4}{*}{Model} &
        \multicolumn{10}{c}{Tasks} \\
        
        \cmidrule{2-11}
        &
        \multicolumn{2}{c|}{Fold 1} & \multicolumn{2}{c|}{Fold 2} & \multicolumn{2}{c|}{Fold 3} & 
        \multicolumn{2}{c|}{Fold 4} & \multicolumn{2}{c}{Fold 5} \\
        
        \cmidrule{2-11}
        &
        SS & SN & ED & ZD & TE & OE & K2 & K3 & RS & PC \\
        &
        mIoU ↑ & mErr ↓ & RMSE ↓ & RMSE ↓ & RMSE ↓ & RMSE ↓ & RMSE ↓ & RMSE ↓ & RMSE ↓ & RMSE ↓ \\
        
        \midrule
        M1 &
        0.0644 & 21.0976 & 0.1959 & 0.0711 & 0.0995 & 
		0.1842 & 0.0670 & 0.0600 & 0.2335 & 0.0431 \\
		
        M2 &
        0.0582 & 15.8135 & 0.1615 & 0.0530 & 0.1136 & 
		0.1480 & 0.0948 & 0.0606 & 0.1858 & 0.0431 \\
        
        M3 &
        0.2681 & 13.0704 & 0.1111 & 0.0404 & \textbf{0.0778} & 
		0.1061 & \textbf{0.0613} & 0.0537 & 0.1559 & 0.0445 \\
        
        M4 &
        \textbf{0.4097} & \textbf{11.4391} & \textbf{0.0741} & \textbf{0.0316} & 0.0791 & 
		\textbf{0.0912} & 0.0639 & \textbf{0.0519} & \textbf{0.1089} & \textbf{0.0420} \\
        
        \bottomrule
        
    \end{tabular}
\end{center}
\vspace{-0.1cm}
\end{table}
\begin{figure}[ht]
    \centering
    \vspace{-0.3cm}
    \includegraphics[width=\textwidth]{figure_files/Fine-Tuning_Visualization.pdf}
    \caption{Qualitative comparison of Ours and its ablated variants in training procedure. All models use 10 labeled examples for each target task, where M3 and M4 observe additional labeled examples of training tasks (different from the target task) in each fold.
    }
    \label{fig:training_procedure_qualitative}
    \vspace{-0.3cm}
\end{figure}

We summarize the quantitative result in Table~\ref{tab:training_procedure_ablation} and qualitative comparison in Figure~\ref{fig:training_procedure_qualitative}.
First, as expected, we observe that DPT with naive 10-shot training (M1) fails to generalize to the test examples in most of the tasks, except for two 2D texture-related tasks (TE, K2). We conjecture that TE and K2 are “easy” cases in terms of few-shot learning, as they are defined as low-level computational algorithms on RGB images, while other high-level tasks require knowledge about semantics (SS) or 3D space (SN, ED, ZD, OE, K3, RS, PC).
Second, we note that BEiT pretraining (M2) largely improves the few-shot generalization performance, allowing the model to produce coarse predictions of the dense labels. However, it still cannot capture object-level fine-grained details in many tasks.
Third, we observe that multi-task training and few-shot adaptation, combined with an efficient parameter-sharing strategy of bias tuning (M3, M4), further improves the performance with a clear gap with M2 where the predictions are also qualitatively finer than M2’s.
Finally, as discussed in Section~\ref{sec:ablation_study}, M4 still further improves over M3 with a clear gap. This shows that in a few-shot learning setting, our matching framework and episodic training are more effective than simple multi-task pretraining employed in M3.
In summary, we may conclude that the fast generalization of Ours is benefitted from episodic training of various tasks followed by parameter-efficient few-shot adaptation as well as powerful pre-training of the encoder (BEiT).



\subsubsection{Fine-tuning with Full Supervision}
To further explore how our method scales well when a large labeled dataset is given, we also fine-tuned our VTM with full supervision of test tasks.
For the fine-tuning, we used the same training dataset as the fully-supervised DPT and employed the episodic fine-tuning objective (Section 3.3). For evaluation, since providing the entire training data as the support set for the matching module is infeasible, we provide a random subset of the training data as the support set to the model.
We summarize the result in Figure~\ref{fig:performance_on_shots_with_full}, which extends Figure~\ref{fig:performance_on_shots} in Section~\ref{sec:experiment}.
In most tasks, our model consistently improves when more supervision is given.
With full supervision at test tasks, our model performs slightly worse than the DPT baseline in seven tasks and performs better or similarly in the other three tasks.
We conjecture that the performance degradation comes from two aspects: (1) the absence of direct input-output connection, \emph{i.e.}, the matching module serves as a bottleneck, and (2) negative transfer from meta-training tasks to test tasks.

\begin{figure}[ht!]
    \centering
    \vspace{-0.3cm}
    \includegraphics[width=\textwidth]{figure_files/Performance_on_Shots_v3.pdf}
    \caption{Performance of VTM on various shots.
    In general, VTM consistently improves performance as more supervision is given, and even surpasses fully supervised baselines on many tasks.
    }
    \label{fig:performance_on_shots_with_full}
    \vspace{-0.3cm}
\end{figure}

\subsubsection{Effect of Number of Training Tasks}
\label{sec:number_of_training_tasks}
The amount of meta-training tasks is an important factor that can affect the performance of the universal few-shot learner.
To verify this, we fixed two test tasks (SS, SN) and trained our VTM on five different subsets of the original eight training tasks (three different subsets with two tasks and two different subsets with five tasks).
We summarize the results in the Table~\ref{tab:training_tasks_ablation}.
As expected, the performance consistently improves as we increase the number of training tasks.
We also note that the few-shot performance becomes sensitive to the choice of training tasks when their number is small (two), presumably as the model becomes reliant on training tasks more correlated to test tasks, while the variance decreases substantially when more training tasks are added.
In addition, the experiment with incomplete training data (Appendix~\ref{sec:incomplete_experiment}) shows the potential ability of our methods in more realistic settings where the training dataset is formed by a combination of different task-specific datasets.
From these results, we expect that our model can further enhance its universality on few-shot learning by utilizing a combined training dataset of much more diverse tasks, which we leave as future work.

\begin{table}[ht]
\caption{10-shot learning performance of Ours with various number of training tasks.}
\vspace{-0.2cm}
\label{tab:training_tasks_ablation}
\begin{center}
    \renewcommand{\arraystretch}{1.5}
    \renewcommand{\aboverulesep}{0pt}
    \renewcommand{\belowrulesep}{0pt}
    \setlength\tabcolsep{6pt}
    \small
    \begin{tabular}{c|cc}
        \toprule
        \multirow{4}{*}{Number of Training Tasks} &
        \multicolumn{2}{c}{Tasks} \\
        
        \cmidrule{2-3}
        &
        \multicolumn{2}{c}{Fold 1} \\
        
        \cmidrule{2-3}
        &
        SS & SN \\
        &
        mIoU ↑ & mErr ↓ \\
        
        \midrule
        2 &
        0.2878 ± 0.0565 & 17.3947 ± 4.8742 \\
		
        5 &
        0.3919 ± 0.0132 & 12.6769 ± 0.1235 \\
        
        8 &
        \textbf{0.4097} & \textbf{11.4391} \\
        
        \bottomrule
        
    \end{tabular}
\vspace{-0.2cm}
\end{center}
\end{table}


\subsubsection{Episodic Training with Incomplete Dataset}
\label{sec:incomplete_experiment}
It would make our method more practical if the model could learn from an incomplete dataset where images are not associated with whole training task labels.
To see how our framework extends to such incomplete settings, we conducted an additional experiment.
We simulate the extreme case of incomplete data by partitioning the training images, such that each image is associated with only a single task out of 8 training tasks.
Specifically, we partitioned the buildings in Taskonomy into eight groups – each corresponds to a different training task.
As this reduces the effective size of training data by the number of training tasks (1/8 in our case), we also train a baseline where we use complete data but use only 1/8 of the training images (for each building, we discard 7/8 of the images).
The results are summarized in Table~\ref{tab:incomplete_dataset}.
We can see that the performance degradation is marginal when we give incomplete data, which implies that our method can be promising in handling realistic scenarios where the training data is a collection of heterogeneous datasets with different label annotations.

\begin{table}[ht]
\caption{10-shot learning performance of Ours trained with incomplete and complete multi-task dataset.}
\vspace{-0.2cm}
\label{tab:incomplete_dataset}
\begin{center}
    \renewcommand{\arraystretch}{1.5}
    \renewcommand{\aboverulesep}{0pt}
    \renewcommand{\belowrulesep}{0pt}
    \setlength\tabcolsep{6pt}
    \small
    \begin{tabular}{c|cc}
        \toprule
        \multirow{4}{*}{Training Data} &
        \multicolumn{2}{c}{Tasks} \\
        
        \cmidrule{2-3}
        &
        \multicolumn{2}{c}{Fold 1} \\
        
        \cmidrule{2-3}
        &
        SS & SN \\
        &
        mIoU ↑ & mErr ↓ \\
        
        \midrule
        Incomplete (one task per building) &
        0.3559 & 13.6207 \\

        Complete (1/8 training data) &
        0.3980 & 12.1633 \\
        
        Complete (whole training data) &
        0.4097 & 11.4391 \\
        
        \bottomrule
        
    \end{tabular}
\vspace{-0.2cm}
\end{center}
\end{table}


\subsection{Further Analysis}

\subsubsection{Parameter-Efficiency Analysis}
We report the number of task-specific and shared parameters of our VTM and two supervised baselines, DPT and InvPT, to compare how our task adaptation is parameter-efficient.
As DPT is a single-task learning model, no parameters are shared across tasks and the whole network should be trained independently for every new task.
InvPT, which is a multi-task learning model, shares a large portion of its parameters across tasks (\emph{e.g.}, encoder backbone), still consumes many parameters for each task in the decoder.

Due to the extensive amount of parameter-sharing, our method is also promising in continual learning setting.
As all task-specific knowledge is included in the bias parameters of the image encoder, the knowledge acquired from past tasks can be recalled without forgetting by keeping the corresponding bias parameters and switching to them whenever a past model is needed.
We especially note that the size of bias parameters is fairly small (288 KB, which amounts to keeping about 3 labeled images of 256x256 resolution for each task).
This allows our model to retain past knowledge very efficiently by keeping the tuned bias parameters plus a few-shot support set, whose external memory requirement is far less compared to memory-based approaches in continual learning that keep hundreds of images~\citep{bang2021rainbow,wang2022continual}.
While the continual learning setting is not our main focus, applying our method to a continual learning setting would be an interesting future direction.

\begin{table}[ht]
\caption{Number of task-specific and shared parameters for a single-channel task (in million).}
\vspace{-0.2cm}
\label{tab:number_of_parameters}
\begin{center}
    \renewcommand{\arraystretch}{1.5}
    \small
    \begin{tabular}{cccc}
        \toprule
        Model & Task-Specific & Shared \\
        \midrule
        DPT (supervised learning) & 110.55 & 0 \\
        InvPT (multi-task learning) & 24.57 & 106.75 \\
        Ours (few-shot learning) & 0.0703 & 202.95 \\
        \bottomrule
        
    \end{tabular}
\vspace{-0.2cm}
\end{center}
\end{table}

\subsubsection{Computation Cost Analysis}
To analyze how our method is computationally efficient compared to supervised DPT, we measured the MACs (multiply–accumulate operations) of our model and DPT using an open-source python library thop~\footnote{https://github.com/Lyken17/pytorch-OpCounter}.
We report the results in Table~\ref{tab:computation_cost}.
Having encoded the support set (e.g., 10-shot), we can see that the computational cost of our model’s inference on a single query image is about 30\% larger than the cost of DPT’s, due to the Matching part.

\begin{table}[ht]
\caption{MACs of Ours and DPT on a single-query inference for a single-channel task.}
\vspace{-0.2cm}
\label{tab:computation_cost}
\begin{center}
    \renewcommand{\arraystretch}{1.5}
    \small
    \begin{tabular}{ccc}
        \toprule
        Model & MACs (G) \\
        \midrule
        DPT & 30.15 \\
        Ours after encoding support (10-shot) & 38.79 \\
        \bottomrule
        
    \end{tabular}
\vspace{-0.2cm}
\end{center}
\end{table}

\subsubsection{Role of Attention Heads}
To analyze the role of attention heads, in Figure~\ref{fig:multihead_attention}, we visualized the attention maps for each head over support images for a given query patch, feature level (3rd level in this example), and task (RS in this example).
The figure shows that each head attends to different regions of the support images.
Moreover, we can find some patterns in heads; for example, the first head tends to attend to flat areas of the scene, such as the floor or ceiling (low-frequency features), while the third head tends to attend to objects, such as couch or plant (high-frequency features).
To further verify the benefit of multi-head attention in the matching module, we also trained our VTM with single head in the matching modules.
The result is summarized in the table below and Table~\ref{tab:attention_heads}.
We can see the performance drop in both SS and SN tasks, which supports that exploiting multiple heads benefits our matching framework.

\begin{figure}[ht]
    \centering
    \includegraphics[width=\textwidth]{figure_files/Multihead_Attention_Visualization.pdf}
    \caption{Visualization of multi-head attention maps of VTM. Here we visualize the matching module at 3rd level for reshading (RS) task.
    }
    \label{fig:multihead_attention}
\end{figure}
\begin{table}[ht]
\caption{10-shot learning performance of Ours with different number of attention heads in Matching module.}
\label{tab:attention_heads}
\begin{center}
    \renewcommand{\arraystretch}{1.5}
    \renewcommand{\aboverulesep}{0pt}
    \renewcommand{\belowrulesep}{0pt}
    \setlength\tabcolsep{6pt}
    \small
    \begin{tabular}{c|cc}
        \toprule
        \multirow{4}{*}{Number of Attention Heads} &
        \multicolumn{2}{c}{Tasks} \\
        
        \cmidrule{2-3}
        &
        \multicolumn{2}{c}{Fold 1} \\
        
        \cmidrule{2-3}
        &
        SS & SN \\
        &
        mIoU ↑ & mErr ↓ \\
        
        \midrule
        1 &
        0.3702 & 12.5936 \\
        
        4 &
        \textbf{0.4097} & \textbf{11.4391} \\
        
        \bottomrule
        
    \end{tabular}
\end{center}
\end{table}


\clearpage
\subsection{Additional Qualitative Comparison with Baselines}
\label{sec:additional_qualitative_comparison_with_baselines}

We provide additional results on the qualitative evaluation of our model and the baselines.
Figure~\ref{fig:appendix_comparison_1}-\ref{fig:appendix_comparison_4} show visualizations on different query image and support set, where we vary the class of semantic segmentation task included in each support.
The result shows consistent trends of that we discussed in Section~\ref{sec:experiment}.
Ours is competitive to the fully supervised baselines (DPT and InvPT), while the other few-shot baselines (HSNet, VAT, DGPNet) fail to learn different dense prediction tasks.


In Figure~\ref{fig:appendix_comparison_2}, even the GT label for semantic segmentation ("couch" class) is noisy as it is a pseudo-label generated by a pre-trained segmentation model~\citep{taskonomy2018}, our model successfully segments two couches present in the figure.
This can be attributed to the task-agnostic architecture of VTM based on non-parametric matching.

\begin{figure}[ht]
    \centering
    \includegraphics[width=\textwidth]{figure_files/Appendix_Comparison_1.pdf}
    \caption{Additional results of qualitative comparison between Ours and the baselines.
    }
    \label{fig:appendix_comparison_1}
\end{figure}

\begin{figure}[ht]
    \centering
    \includegraphics[width=\textwidth]{figure_files/Appendix_comparison_2.pdf}
    \caption{Additional results of qualitative comparison between Ours and the baselines.
    }
    \label{fig:appendix_comparison_2}
\end{figure}

\begin{figure}[ht]
    \centering
    \includegraphics[width=\textwidth]{figure_files/Appendix_comparison_3.pdf}
    \caption{Additional results of qualitative comparison between Ours and the baselines.
    }
    \label{fig:appendix_comparison_3}
\end{figure}

\begin{figure}[ht]
    \centering
    \includegraphics[width=\textwidth]{figure_files/Appendix_comparison_4.pdf}
    \caption{Additional results of qualitative comparison between Ours and the baselines.
    }
    \label{fig:appendix_comparison_4}
\end{figure}


\clearpage
\subsection{Additional Qualitative Comparison with Our Variants}
\label{sec:additional_qualitative_comparison_with_our_variants}

We also provide additional results on the qualitative evaluation of our model and our ablated variants, Ours w/o Matching and Ours w/o Adaptation.
Figure~\ref{fig:appendix_ablation_1}-\ref{fig:appendix_ablation_4} show visualizations on different query image and support set. 
The results show a consistent trend with the quantitative results in Table~\ref{tab:main_table}.
Interestingly, our method without adaptation already exhibits some degree of adaptation to the unseen tasks even without fine-tuning and task-specific components, showing that the non-parametric architecture of our model and the parameter sharing derived from is appropriate to learn generalizable knowledge to understand the novel tasks.
On the other hand, adding a task-specific component and adaptation mechanism to the model allows more dramatic improvement in understanding novel tasks from few-shot examples, showing the importance of the adaptation mechanism in our task.
Finally, we observe that equipping the matching mechanism with the adaptation module provides much sharper and fast adaptation to the unseen tasks, which verifies our claims.


\begin{figure}[ht]
    \centering
    \includegraphics[width=\textwidth]{figure_files/Appendix_abaltion_1.pdf}
    \caption{Additional results of qualitative comparison between Ours and its ablated variants.
    }
    \label{fig:appendix_ablation_1}
\end{figure}

\begin{figure}[ht]
    \centering
    \includegraphics[width=\textwidth]{figure_files/Appendix_abaltion_2.pdf}
    \caption{Additional results of qualitative comparison between Ours and its ablated variants.
    }
    \label{fig:appendix_ablation_2}
\end{figure}

\begin{figure}[ht]
    \centering
    \includegraphics[width=\textwidth]{figure_files/Appendix_abaltion_3.pdf}
    \caption{Additional results of qualitative comparison between Ours and its ablated variants.
    }
    \label{fig:appendix_ablation_3}
\end{figure}

\begin{figure}[ht]
    \centering
    \includegraphics[width=\textwidth]{figure_files/Appendix_abaltion_4.pdf}
    \caption{Additional results of qualitative comparison between Ours and its ablated variants.
    }
    \label{fig:appendix_ablation_4}
\end{figure}

%%%%%%%%% REFERENCES
{\small
\bibliographystyle{ieee_fullname}
\bibliography{main}
}

\clearpage

\section*{Supplementary Material}


\subsection*{Data}
We use MNIST or FFHQ (depending on the DCGAN or StyleGAN2 setting) as the dataset of natural images in our experiments. They both consist of 70,000 images.
We use $80\%$ of the data for training, and $20\%$ for evaluation, i.e. we use a fixed sample of 56,000 images for training in all the experiments, and use the rest (14,000) for the evaluation of classifiers. Note that this means that we use the same data of natural images for training both GAN and GAN classifiers, across all iterations. MNIST images are used as $28\times28$ grayscale images, and FFHQ images are used as $256\times256$ RGB images in all our experiments, for both training and evaluation. 
\subsection*{GAN training} \label{sec:supplement_gan_training}
In the DCGAN setting, we trained a simplistic DCGAN architecture well suited for the MNIST generation task (unconditional generation of all digits). Specifically, the generator network is modeled as follows: the random noise variable of $100$ dimensions is passed through a fully connected layer of 12544 units, followed by 3 transposed convolution layers of 128, 64 and 32 units, each with a kernel size of 5, before the final transposed convolution unit for image output. All hidden layers use batchnorm and the LeakyReLU activation.

For the second setting, we have used the unmodified StyleGAN2 as the GAN architecture of choice. StyleGAN2 is generally considered a SOTA GAN model, capable of generating high-resolution, diverse, photo-realistic images, especially of human faces. There are several components and techniques used in its training framework that cause the generated images to be of high quality and of greater diversity. We specifically emphasize the additional inputs to the generator network: a latent code being output by a non-linear mapping network, and the random noise inputs, both are fed to the individual layers of the generator network. These techniques help the outputs capture the stochasticity and variance present in the real world.

We use proprietary implementation of StyleGAN2, which replicates the TensorFlow \citecustom{tensorflow2015-whitepaper} implementation available online (https://github.com/NVlabs/stylegan2). We did not tweak any training parameters. For training a single GAN instance, we use 8 NVIDIA Tesla P100 GPUs and each GAN instance roughly required 1 week to train.

For the modified loss functions, we have used $\phi=0.001$ in all experiments in the StyleGAN2 setting. 

We have not used any pre-training for the GANs in the main paper.

\subsection*{Classifier training}
In the DCGAN setting, we use a basic CNN classifier that performed well for this task. The classifier includes two convolutional layers of 32 and 64 units, each with a kernel size of 3. Both layers use ReLU activation and followed by a max pooling of 2 in both dimensions. We train directly on the grayscale images without any compression.

In the StyleGAN2 setting, we have used ResNet-50 (version 1.5), Inception-v3 and MobileNetV2 architectures when training the classifiers. ResNet-50 is a high-performing CNN architecture, particularly for image classification, and has been shown to be effective for our task in previous research \citecustom{Wang_2020_CVPR,gragnaniello2021gan}. One of the distinguishing features of the ResNet-50 architecture is the use of residual connections, that generally enables efficient learning by sharing information between the hidden layers of a deep network. The two other CNN architectures in our study, Inception-v3 (\cite{szegedy2016rethinking}) and MobileNetV2 (\cite{sandler2018mobilenetv2}), are chosen for their differences to ResNet-50. Like ResNet-50, Inception-v3 is also a large CNN architecture but it does not include residual connections and uses a different "module" that is repeated across the layers. We include MobileNetV2 as a relatively lighter capacity architecture, when compared to the other two architectures.

For the StyleGAN2 classifiers, we use the publicly available implementations as part of the TensorFlow library. When training the classifiers, we pass both the natural and GAN-generated images through JPEG encoding. For each classifier, we train using a single Tesla P100 GPU, and the models roughly require 1 day to train. We do not use any pre-training for the classifiers (unlike \citecustom{Wang_2020_CVPR}), to avoid any external influence in our experiments. We let the classifiers train till they reach convergence, and did not need to finetune the parameters for better performance.

We always train the classifiers on a balanced sample of natural and generated images. Therefore, when we train a classifier using a sample of 15 GAN instances, we train with $15\times56000$ generated and an equal number ($15\times56000$) of natural images, and use 0.5 as the classification decision threshold. For this, the natural images are simply repeated 15 times to obtain a balanced training dataset. When evaluating the fooling ability of GAN generators, the held-out test classifiers are trained using 10 held-out GAN instances. 

\subsection*{Training a StyleGAN2 classifier in the presence of the ``truncation trick''}
The ``truncation trick'' is often used with StyleGAN2 (followed from the StyleGAN model) to avoid generating unrealistic images. The approach shrinks the distribution, in order to remove the regions of low density that might be poorly represented by the GAN model. The expression used to shrink the latent distribution is:
\begin{align}
\mathbf{w'} = \mathbf{\bar{w}} + \psi(\mathbf{w}-\mathbf{\bar{w}})
\end{align}
, where $\mathbf{\bar{w}}$ is the the expected value of the mapped latent space. Here, $\psi$ is the coefficient of truncation: $\psi=1$ implies an absence of truncation and $\psi=0$ would correspond to using the (fixed) expected value of the mapped latent space as the latent input for sample generation. Typically \citecustom{karras2019style,karras2020analyzing}, $\psi=0.5$ is effective in practice.

The truncation trick is used with StyleGAN2 if sampling realistic images, when trained with the FFHQ dataset. We note that we also used this trick when we visually compared the image quality. However, the use of this trick effectively shrinks the diversity and brings the samples closer to the ``average face'' that is learned by the model. And therefore, we don't use the truncation trick in our experiments, since our study is directly measuring the extent of diversity present in the GAN models. 

In our experiments, we have also identified that the FID is negatively affected when employing the truncation trick. The FID (lower is better) for a sample of generated images without truncation is $\sim37$, whereas with truncation ($\psi=0.5$) is $\sim81$.

Since the diversity is significantly reduced if sampling images with truncation, we have also identified that we do not require multiple generators when training a classifier to be able to achieve generalization: i.e., training a classifier using truncated samples from just one SG2 generator instance suffices to detect truncated samples from an independent SG2 generator instance. 

Moreover, we note that by training a classifier which can detect the full range of GAN-generated samples, we also achieve a perfect accuracy when detecting samples generated with the truncation trick.

\subsection*{Finetuning StyleGAN2 to transform to the next iteration}
\begin{figure}[h]
    \centering
    \includegraphics[width=0.6\linewidth]{images/fig_finetuning.pdf}
	\caption{\textbf{Finetuning GAN of iteration $i=\mathbf{0}$.} We finetune a fully trained GAN of iteration 0, using the modified loss in the finetuning steps. As depicted, a held-out classifier of iteration $i=0$ quickly gets fooled, and a held-out classifier of iteration $i=1$ starts to detect the artifacts that were not present before.}
\label{fig:gan_finetuning}
\end{figure}

The experiments mentioned in the paper train GANs from scratch using the modified loss, with random weight initialization. However, we acknowledge that training GANs is an expensive process, where modern GAN models like StyleGAN2 require weeks to train using multiple accelerators. To this point, we have observed that we can ``transform'' a GAN to the succeeding iteration by finetuning a pre-trained GAN, and including a pre-trained detector in the finetuning steps. For instance, using the same modification to the generator loss, finetuning an $i=0$ iteration GAN results in a model that exhibits the same artifacts as what's present in an $i=1$ iteration GAN generator trained from scratch. We arrive at this finding because they are both detected by a held-out $i=1$ iteration detector and they both fool a held-out $i=0$ iteration detector.

\end{document}