\documentclass[%
aip,
jcp,
% jmp,
% bmf,
% sd,
% rsi,
amsmath,amssymb,
%
reprint%
%author-year,%
%author-numerical,%
% Conference Proceedings
]{revtex4-2}
%\documentclass[journal = jpclcd, manuscript=letter,layout=twocolumn]{achemso}
\usepackage{amsmath,amssymb,amsthm, mathtools, enumitem, empheq, gensymb, textcomp, bm, booktabs, multirow}
\usepackage[margin=1in]{geometry}
\usepackage{chemformula}
\usepackage{dsfont}{\tiny }

\def\bea{\begin{eqnarray}}
\def\eea{\end{eqnarray}}
\def\nn{\nonumber}
\def\beq{\begin{equation}}
\def\eeq{\end{equation}}
\def\ba{\begin{eqnarray}}
\def\ea{\end{eqnarray}}
\def\be{\ba\displaystyle}
\def\ee{\ea}
\def\F{{\cal F}}
\def\e{\epsilon}
\def\ve{\varepsilon}
\definecolor{cadmiumgreen}{rgb}{0.0, 0.42, 0.24}
\definecolor{carmine}{rgb}{0.59, 0.0, 0.09}


\newcommand{\la}{\langle}
\newcommand{\ra}{\rangle}

\newcommand{\w}{{\mathsf w}}


\DeclarePairedDelimiter\bra{\langle}{\rvert}
\DeclarePairedDelimiter\ket{\lvert}{\rangle}
\DeclarePairedDelimiterX\braket[2]{\langle}{\rangle}{#1 \delimsize\vert #2}
\DeclarePairedDelimiterX\Braket[3]{\langle}{\rangle}{#1 \delimsize\vert #2 \delimsize\vert #3}
\newcommand{\crtwo}{Cr\textsubscript{2}}
\newcommand{\matr}[1]{\mathbf{#1}}

%\makeatletter
%\def\@email#1#2{%
%	\endgroup
%	\patchcmd{\titleblock@produce}
%	{\frontmatter@RRAPformat}
%	{\frontmatter@RRAPformat{\produce@RRAP{*#1\href{mailto:#2}{#2}}}\frontmatter@RRAPformat}
%	{}{}
%}%
%\makeatother

\begin{document}
	
	\preprint{AIP/123-QED}

	\title{Variational quantum computation with discrete variable representation for ro-vibrational calculations}
	\author{K. Asnaashari}
	\email{kasra.asnaashari@phys.chem.ethz.ch}
	\author{D. Bondarenko}
	\author{R. V. Krems}
	\affiliation{ 
		Department of Chemistry, University of British Columbia, \\Vancouver, B.C. V6T 1Z1, Canada \\
		Stewart Blusson Quantum Matter Institute, \\
		Vancouver, B.C. V6T 1Z4, Canada
	}
	
	\date{\today}
	
	
	
	\begin{abstract}
%		While quantum computing algorithms have been successfully applied to electronic structure of molecules, applications of quantum computing to molecular dynamics remain scarce. The variability of intra-molecular interaction potentials gives rise to vibrational states with a  wide range of properties. It is therefore challenging to obtain a general representation of molecular ro-vibrational states by states of a quantum computer with a limited number of qubits and quantum gates. Another challenge is the exponential scaling of the computation complexity with the dimensionality of molecular configuration space.  
		We demonstrate an approach to computing the vibrational energy levels of molecules that 
		combines the discrete variable representation (DVR) of molecular Hamiltonians with variational quantum eigensolvers (VQE) and a greedy  search of optimal quantum gate sequences.  We show that the structure of the DVR Hamiltonians reduces the quantum measurement complexity scaling from exponential to polynomial, allowing for efficient VQE without second quantization.  
		We then demonstrate that DVR Hamiltonians also lead to very efficient quantum ansatze for representing ro-vibrational states of molecules by states of a quantum computer. To obtain these compact representations, we demonstrate 
				the quantum ansatz search by computing the vibrational energy levels of Cr$_2$ in seven electronic states as well as of van der Waals complexes Ar--HCl and Mg--NH. Our numerical results show that accuracy of 1~cm$^{-1}$ can be achieved by very shallow circuits with 2 to 9 entangling gates. 
		
		
	\end{abstract}
	
	
	
	\maketitle
	
	
	\section{Introduction}
	
	Accurate calculation of molecular properties is considered a promising application of quantum computing. 
	The eigenstates of molecular Hamiltonians can be obtained on quantum computers by variational quantum eigensolvers (VQEs) \cite{vqe-review, vqd} that employ sequences of gates operating on qubits (quantum circuits) to prepare quantum states tailored for specific problems. 
	VQEs have been applied for solving the electronic structure problem for molecules \cite{vqe-1,vqe-2,chem-1,chem-2,chem-3,chem-4,chem-5,chem-6,chem-7} and lattice models \cite{chem-3,vqe-review,lattice-1}. 
	%The quantum circuits for such computations can be informed by the electronic structure method \cite{uccsd,uccgsd,symmetry,hva} or the relevant space of electronic orbitals. 
	However, applications of VQE to computations of ro-vibrational energies and states have been limited \cite{dvr-dynamics, morse, mol-vib, mol-vib-2, mol-vib-3, co2-1, co2-2, co2-3}. 
	%Yet, by comparison with electronic structure calculations that benefit from many accurate classical methods readily available, nuclear dynamics calculations are much more challenging. 
%	The challenge of molecular dynamics calculations is partly due to the complexity of inter- and intra-molecular interaction potentials, which are molecule-specific and exhibit widely varying landscapes of potential energy. 
%	Refs. \cite{dvr-dynamics, morse,  mol-vib} demonstrated VQE computations of ro-vibrational states for specific model problems, while Refs. \cite{co2-1, co2-2} used VQE to compute the Hamiltonian matrix, which was then diagonalized classically. 
	%, such as one-dimensional Morse potential \cite{morse}. 
	%or two-mode model of CO$_2$ \cite{co2-1, co2-2}. 
	Refs. \cite{mol-vib-3, mol-vib-2} demonstrated a general approach for computing ro-vibrational energy levels of polyatomic molecules inspired by previous work  on electronic structure. However, the methods of  Refs. \cite{mol-vib-2, mol-vib-3} require extended quantum circuits including $>200$ \cite{mol-vib-3} or between 44 and 140 292  \cite{mol-vib-2} entangling gates. 
	A key challenge for quantum computing of molecular dynamics is to develop a general approach that can (i) be applied to a wide variety of ro-vibrational states, from deeply bound to van der Waals states; (ii) yield high accuracy with  shallow quantum circuits; (iii) exhibit at most a polynomial scaling with the dimensionality of the molecular configuration space. Within VQE, (i) requires an ansatz for quantum circuits that can represent molecular states in widely varying landscapes of potential energy; (ii) is necessary due to noise and hardware limitations of current quantum computers; and (iii) is required for the quantum advantage. 
	
	%It is therefore not known how to represent ro-vibrational states of molecules for variation computations on a quantum computer for arbitrary molecules governed by general potential energy surfaces
	
	Here, we formulate a general approach to computing the ro-vibrational energy levels of molecules with VQE based on discrete variable representations (DVR) of molecular Hamiltonians \cite{colbert, choi}. 
	We first derive the theoretical bounds on the quantum measurement complexity with VQE based on general DVR.  
	Our analysis shows that 
the structure of the DVR Hamiltonians reduces the quantum measurement complexity scaling from exponential to polynomial.

In the second part of this work, we demonstrate that DVR Hamiltonians also lead to very efficient quantum ansatze for representing ro-vibrational states of molecules by states of a quantum computer. 
To do this, we develop a compositional search algorithm that explores the space of gate permutations to yield quantum circuits suitable for ro-vibrational VQE computations.
To illustrate the generality of this approach and the efficiency of the resulting quantum circuit representations of ro-vibrational states, we consider Cr$_2$ 
	in seven different electronic states \cite{cr2-pot} and van der Waals complexes Ar--HCl($^1\Sigma$) and Mg--NH($^3\Sigma$). These molecular systems exhibit ro-vibrational states with widely different energies (from $-55$ to $-15,000$ cm$^{-1}$ from the dissociation threshold) and spatial variations of wave functions and energy level patterns. 
	Our greedy compositional search yields quantum circuits that produce VQE results with accuracy $<1$ cm$^{-1}$ for ground and excited ro-vibrational energy levels, illustrating the ability of VQE to compute the rotational constants and vibrational anharmonicity, with between 2 and 9 entangling gates.  	
	
	The remainder of this article is organized as follows. After a brief introduction of VQE, Section \ref{efficient-measurements} presents an algorithm to evaluate the DVR Hamiltonains with a polynomial number of measurements. 
Section \ref{anzats-optimization} describes the algorithm for the compositional anzats optimization, which is followed by numerical results illustrating the efficiency and accuracy of the optimized ansatze for Cr$_2$ 
	in seven different electronic states \cite{cr2-pot} and van der Waals complexes Ar--HCl($^1\Sigma$) and Mg--NH($^3\Sigma$) in Section \ref{numerical-results}. 
	The work is summarized in Section \ref{summary}. 
	
	
	
	
	
	
	\section{Theory}
	
	
	 In VQE, 
	 	 a quantum computer calculates the expectation value  $\Braket{\psi(\bm\varphi)}{\hat{H}}{\psi(\bm\varphi)}$, which  is minimized by varying $\bm\varphi$ to yield 
	the lowest eigenvalue $E_{i=0}$ and an approximate representation of the corresponding eigenvector of $\hat H$. 
	%Here,  $\tilde{\bm\varphi}_0 = {\rm argmin}_{\bm\varphi} \Braket{\psi(\bm\varphi)}{\hat H}{\psi(\bm\varphi)}$.
%	with 
%	\begin{align}
%		\Braket{\psi(\tilde{\bm\varphi}_0)}{\hat H}{\psi(\tilde{\bm\varphi}_0)} \approx \tilde{E}_0,
%		\label{H}
%	\end{align}	
%$\tilde{\bm\varphi}_0 = {\rm argmin}_{\bm\varphi} \Braket{\psi(\bm\varphi)}{H}{\psi(\bm\varphi)}$ 
%and $|\psi(\tilde{\bm\varphi}_0) \rangle$ approximates the corresponding eigenvector. 	
	This method can be extended to compute excited states  by optimizing \cite{vqd}
		\begin{align}
		\tilde{\bm \varphi}_v &= {\rm argmin}_{\bm\varphi} \left [ \Braket{\psi(\bm\varphi)}{\hat H}{\psi(\bm\varphi)} 
		+ \sum_{i=0}^{v-1} \beta_i \langle {\psi(\tilde{\bm\varphi}_i)}| {\psi({\bm\varphi})\rangle} \right ]
		\label{Hexcited}
	\end{align}
	where $\beta_i \geq E_{i+1} - E_i$ and $\tilde {\bm \varphi}_i$ denotes an optimal solution for the corresponding quantum state. 


The quantum states $| \psi({\bm \varphi}) \rangle$ are obtained by quantum circuits acting on qubits and the Hamiltonian is expanded in quantum gates. Most generally, 
	 %		In order to evaluate the expectation values  in Eqs. (\ref{H}) and (\ref{Hexcited}), we use the following decomposition of the Hamiltonian: 
	\begin{align}
		\hat H = \sum_{i = 0}^{4^n} A_i K^i_1\otimes K^i_2 \ldots \otimes K^i_n 
		\label{Pauli}
	\end{align}
	where $n$ is the number of qubits, $K^i_j \in \{\sigma_X, \sigma_Y, \sigma_Z, I\}$ acting on qubit $j$, $\{ \sigma_i \}$ are the Pauli matrices, $I$ is the identity matrix, 
	\begin{align}
		A_i = \frac{1}{2^n}{\rm Tr}[(K^i_1\otimes K^i_2 \ldots \otimes K^i_n)\cdot {\bm H}]
	\end{align}
	and $\bm H$ is the Hamiltonian matrix in some basis. 
The computational complexity is determined by the number of non-zero terms in Eq. (\ref{Pauli}), which for a general matrix is $4^n$; 
and the complexity of the quantum circuits yielding $\ket{\psi(\bm\varphi)}$.
We begin by showing that the structure of DVR matrices allows an efficient quantum circuit representation of $\hat H$,
scaling with $n$ as poly($n$). 


\begin{figure}[h]
    \centering
    \includegraphics[width=0.5\columnwidth]{BellCirc.pdf}~~~~~\includegraphics[width=0.4\columnwidth]{kthDiag.pdf}
    \caption{Left: Preparation of state (\ref{plus-l}) from the computational basis. $H$ denotes the Hadamard gate and the circles -- the CNOT gates. Right: Measuring $t^{k[n]}$ that includes matrix elements on the main and $k^{\text{th}}$ diagonals. The open circles $\color{cadmiumgreen} \circ$ are obtained from projections of (\ref{plus-l}),  $\color{carmine} \otimes $ -- via $\mathds{1}_2 \otimes t^{k[m]}$, and $\color{magenta}+$ -- by additional measurements in the entangled basis. }
    \label{fig:BellCirc}
\end{figure}

	\begin{figure*}
		\begin{tabular}{cc}
			\includegraphics[width=0.8\columnwidth]{cr2_pots} & 	\includegraphics[width=1.2\columnwidth]{cr2_ansatz_c1} \\
		\end{tabular}
		\caption{Left: potential energy for Cr$_2$ from Ref. \cite{cr2-pot}. Right:  quantum circuits for VQE yielding the ground state energy with error $\leq 1$ cm${}^{-1}$. The squares represent the $R_Y$ gates and the circles show the entangling CNOT gates.}
		\label{fig:cr2_pot_ansatz}
	\end{figure*}


	
	


		

\subsection{Efficient measurement of DVR Hamiltonians}
\label{efficient-measurements}
% DVR of a Hamiltonian diagonalizes the potential energy and leads to structured matrices representing the kinetic energy. 
 DVR is a finite basis representation, in which the coordinate operators (and consequently the potential energy) are diagonal. 
 The DVR matrix of kinetic energy is not sparse. However, it has specific structure that is exploited here. 
We use DVR introduced by Colbert and Miller \cite{colbert}.  Other DVR bases can be reduced to those in Ref. \cite{colbert} by coordinate transformations. 
% For example, the Hamiltonian of a diatomic molecule in a DVR basis is \cite{colbert}
 %\begin{eqnarray}
%H_{ij} = T_{ij} + V (r_i) \delta_{ij},
 %\end{eqnarray}
 %with the kinetic energy given by
 %	\begin{equation}\label{eq:dvr_0inf}
%		T_{ij} = \frac{\hbar^2}{2m\Delta r^2} (-1)^{i-j}\begin{cases}
%			\pi^2/3 - 1/(2i^2), & i=j \\
%			\frac{2}{(i-j)^2} - \frac{2}{(i+j)^2}, & i\neq j
%		\end{cases},
%	\end{equation}
%	where $\Delta r$ is the distance between grid points $r_i = i\Delta r$ discretizing the continuous inter-atomic distance $r$, $m$ is the reduced mass, $\hbar$ is the reduced Planck constant, and 
%$i \in [1,  N]$, and $V(r)$ is the inter-atomic interaction potential.  
%
As follows from Ref. \cite{colbert}, the Hamiltonian matrix has the following structure: 
\be
H_{ij} =
    \begin{cases}
    d(i), &  i=j,\\
    f\left(|i-j|\right) + g(i+j), & i \neq j
    \end{cases}.
%    \\ f\left(|i-j|\right) \in O\left( |i-j|^{-\alpha} \right), \ \alpha > 0, \\
 %   g\left(i+j\right) \in O\left( (i+j)^{-\beta} \right), \ \beta > 0.
\label{dvr-structure}
\ee	
We use a number encoding to map the DVR basis states onto qubit states, which allows VQE to compute the eigenvalues of $\bm H$ of size
	$2^n\times 2^n$ using $n$ qubits.
It can be shown (see SM \cite{sm}), that the functions $f$ and $g$ in  Eq. (\ref{dvr-structure}) satisfy
\be
    \sum_{k=s}^{2^n-1}\left| f(k) \right| \leq  O(s^{-\alpha}), \ \alpha>0,\\
    \sum_{k=r}^{2^{n+1}-1-r}\left| g(k) \right| \leq O(r^{-\beta}), \ \beta>0,
    \label{off-diagonals}
\ee
for $1 \leq r \ll 2^n$ and $1 \leq s \ll 2^n$.  
%with  (see supplemental material (SM)  \cite{sm}): 
%\be
 %  f\left(|i-j|\right) \in  O\left( |i-j|^{-\alpha} \right), \ \alpha > 0,~~~ \\
%  g\left(i+j\right) \in  O\left( \min\{i+j,2^n-i-j\}^{-\beta} \right), \ \beta > 0.~~~ \label{g}
%\ee  
Given a quantum state $|\psi \rangle$, our aim is to show that
\be
\tau = \la \psi| \hat H | \psi \ra + O(\epsilon) 
\ee
can be measured with the number of quantum circuits that scales polynomially with $n$ and $1/\epsilon$. We seek to transform $|\psi \rangle$ by short-depth $\hat V_i$, 
%where $i$ is $ \leq{\rm Poly}(n, 1/\epsilon)$, 
so that
\be
\tau = \sum_{i=1}^{{\rm poly}(n, 1/\epsilon)} \sum_{j=1}^{2^n} w_{ij} \left|\la \psi |V_i^{\dagger} | j \ra\right|^2
\ee
where $|j\rangle$ is the state of $n$ qubits in the computational $Z$ basis.  % The coefficients $w_{ij}$ encode the elements of the DVR matrix. 

We leverage the structure of the DVR matrix to decompose $\bm \tau$ into contributions from a diagonal matrix ($D$), $s \approx \epsilon^{-1/\alpha}$ diagonal bands ($t^k$) and $r\approx \epsilon^{-1/\beta}$ anti-diagonal components ($a^k$), as follows: 
\be 
\label{the-tau}
    \tau = \langle D \rangle +  \sum_{k=1}^{\left\lceil \epsilon^{- \frac{1}{\alpha}} \right\rceil}  f(k) \la t^{k[n]} \ra +  ~~~~~~~~~~~~~~   \\  \sum_{k=1}^{\left\lceil \epsilon^{- \frac{1}{\beta}} \right\rceil}\left( g(k) \la a^{k[n]} \ra + g(2^n-k) \la a^{(2^n-k)[n]} \ra \right)  + O\left(\epsilon\right).
\nonumber
\ee
The computation of the diagonal contribution $\langle D \rangle$ is classically efficient.

%For the diagonal contribution,  $\{ V_i \} = \{ \mathds{1} \}$ and $w_{1i} = D_{ii}$. 

	
	
To construct $t^{k[n]}$, we note that $ l \equiv \left\lceil \log_2(k+1) \right\rceil$ is the smallest number of qubits that allows $k$-th band.
We consider $l$-qubit entangled states produced from the $Z$ basis by a sequence of CNOT gates shown in Fig. \ref{fig:BellCirc} (left). 
The resulting state is
\begin{eqnarray}
| + l \ra =  \frac{ \bigotimes_i^l |j_i\ra   +  \bigotimes_i^l |p_i\ra }{\sqrt{2}},
\label{plus-l}
\end{eqnarray}
where $j_i$ and $p_i$ are single-qubit states, with at least one $j \neq p$.
%For example, the circuit $\prod_{q=0}^{n-1}{\rm CNOT}(q, q+1)$ shown in Figure \ref{fig:BellCirc} produces the $l$-qubit Greenberger-Horne-Zeilinger (GHZ) state with $j_i = 0$ and $p_i = 1$ for all $l$ qubits. 
The projection of an $n$-qubit state onto the $| + l \rangle$ state 
has 2 non-zero off-diagonal matrix elements of magnitude $1/2$. The binary representation of the row and column positions of these off-diagonal elements differ by $l$ bits. 
The position of the non-zero off-diagonal elements is controlled by the position of the CNOT gates.  
Thus, the expectation value of $t^{k[l]}$ can be obtained by measurements with, at most, $k$ $l$-qubit entangled states.




We now observe that $t^{k[m+1
]} = \mathds{1}_2 \otimes t^{k[m]} + \gamma$, where $\gamma$ represents $k$ pairs of elements missing from the middle of the tensor product matrix (pluses in Fig.~\ref{fig:BellCirc} right). 
These elements can be directly targeted by additional measurements in the entangled basis constructed as described above for these specific elements, requiring at most $k$ measurements if performed element-wise.  


The full algorithm to compute $t^{k[n]}$ thus includes: 
(i) At most $2^l - k \leq k$ expectation values via circuits of depth  $ \leq l+1$ to obtain $t^{k[l]}$ (yielding elements represented by open circles in  Fig.~\ref{fig:BellCirc});
(ii) Filling the gap (pluses in Fig.~\ref{fig:BellCirc}) in $\mathds{1}_2 \otimes t^{k[m]}$ to produce $t^{k[m+1]}$, which requires measurements of at most $k$ projections on entangled states via circuits of depth at most $\left\lceil\log_2 2k \right\rceil$;
(iii) A total of $n-l$ iterations to fill the entire band. 
The total number of measurement circuits for $t^{k[n]}$  is thus
\be
    \textrm{Comp}\left( t^{k[n]} \right) <
\left(n + 1 - \log_2 k \right) k.
\ee
	

To construct the anti-diagonals $a^{k[n]}$, we first note that the expectation values of
$a^{1[1]} = \begin{bmatrix}
1 & 0 \\
0 & 0
\end{bmatrix}$, 
$a^{2[1]} = \begin{bmatrix}
0 & 1 \\
1 & 0
\end{bmatrix}$ and
$a^{3[1]} = \begin{bmatrix}
0 & 0 \\
0 & 1
\end{bmatrix}$ can be obtained by sampling one-qubit measurements of $\psi$ in $Z$ to get $\left\la a^{1[1]} \right\ra_\psi$ and $\left\la a^{3[1]} \right\ra_\psi$ and in $X$ to get $\left\la a^{2[1]} \right\ra_\psi$. 
The elements $a^{k[m+1}]$ can be obtained by combining $\mathds{1}_2 \otimes a^{k[m]}$ and $a^{k[m]} \otimes \mathds{1}_2$, which  at most doubles the number of measurements \cite{sm}.

%When measuring in $Z$, the probability of obtaining $\uparrow$ yields $\left\la a^{1[1]} \right\ra_\psi$, while the probability of obtaining $\downarrow$ yields $\left\la a^{3[1]} \right\ra_\psi$.  Obtaining $a^{k[m+1]}$ from $a^{k[m]}$ at most doubles the number of measurements \cite{sm}. 

We limit the construction to $r = {\log_2 \epsilon}/{\beta}  $ anti-diagonals.  Eq. (\ref{off-diagonals}) ensures that the remaining anti-diagonals will contribute less than $\epsilon$ to the expectation value (\ref{the-tau}).  
The circuits can be constructed from $a^{k[1]}$ by incrementally increasing the number of qubits to  $a^{k[r]}$, which requires $\leq 2^r$ bases in total. The contributions from the anti-diagonals are obtained by measuring $n-r$ qubits in the computational basis and $a^{k[r]}$ thus constructed. 
For the $n-s$ qubits, one needs to take into account only the all-$\uparrow$ and all-$\downarrow$ outputs, yielding the up-most and down-most $2^r$ anti-diagonals. The total number of bases for this protocol is bounded by
\be
    2^{r} < 2^{1-\frac{\log_2 \epsilon}{\beta}} = 2\epsilon^{-\frac{1}{\beta}}.
\ee



	


		\begin{table}[!htbp]
		\centering
		\begin{tabular*}{\columnwidth}{@{\extracolsep{\fill}}cccccc}
			\toprule
			Electronic & \multirow{2}{*}{$v$} & \multirow{2}{*}{BM} & \multicolumn{3}{c}{$E_v^{\rm VQE}$} \\
			\cmidrule{4-6}
			state & & & $\mathcal{C}_1$ & $\mathcal{C}_{0.01}$ & Linear \\
			\midrule
			\multirow{2}{*}{$^1\Sigma_g^+$}  & 0   & -15358.94 & -15358.87 & -15358.99 & -15358.99 \\
			& 1   & -14846.67 & -14838.70 & -14846.75 & -14846.96 \\
			& 2   & -14333.21 & -14310.29 & -14332.80 & -14332.82 \\
			& 3   & -13826.93 & -13797.65 & -13826.87 & -13827.29 \\
			& 4   & -13334.02 & -13275.12 & -13318.77 & -13335.45 \\
			& 5   & -12861.37 & -12897.70 & -12871.32 & -12868.75 \\
			\midrule
			\multirow{2}{*}{$^3\Sigma_u^+$}  & 0   & -9862.07 & -9861.37 & -9862.14 & -9862.14 \\
			& 1   & -9559.46 & -9538.97 & -9556.67 & -9559.41 \\
			& 2   & -9300.92 & -9240.74 & -9266.82 & -9300.88 \\
			& 3   & -9080.60 & 9068.09 & -9079.26 & -9085.40 \\
			& 4   & -8897.58 & -8866.09 & -8870.57 & -8896.82 \\
			& 5   & -8747.53 & -8729.41 & -8750.18 & ... \\
			\midrule
			\multirow{2}{*}{$^5\Sigma_g^+$}  & 0   & -7566.53 & -7565.88 & -7566.50 & -7566.50 \\
			& 1   & -7416.28 & -7397.28 & -7416.28 & -7416.29 \\
			& 2   & -7264.92 & -7181.98 & -7264.60 & -7264.69 \\
			& 3   & -7114.40 & -7075.88 & -7117.83 & -7118.43 \\
			& 4   & -6965.91 & -7001.98 & -6953.74 & -6958.08 \\
			& 5   & -6820.04 & -6873.37 & -6837.02 & -6840.02 \\
			\midrule
			\multirow{2}{*}{$^7\Sigma_u^+$}  & 0   & -6519.01 & \multicolumn{2}{c}{-6519.04} & -6519.04 \\
			& 1   & -6350.36 & \multicolumn{2}{c}{-6350.11} & -6350.11 \\
			& 2   & -6183.38 & \multicolumn{2}{c}{-6185.17} & -6185.17 \\
			& 3   & -6018.09 & \multicolumn{2}{c}{-6018.12} & -6018.28 \\
			& 4   & -5854.50 & \multicolumn{2}{c}{-5848.69} & -5849.10 \\
			& 5   & -5692.63 & \multicolumn{2}{c}{-5728.00} & -5731.33 \\
			\midrule
			\multirow{2}{*}{$^9\Sigma_g^+$}  & 0   & -5348.79 & \multicolumn{2}{c}{-5348.82} & -5348.82 \\
			& 1   & -5175.81 & \multicolumn{2}{c}{-5175.51} & -5175.51 \\
			& 2   & -5005.17 & \multicolumn{2}{c}{-5008.55} & -5008.56 \\
			& 3   & -4836.85 & \multicolumn{2}{c}{-4829.80} & -4829.92 \\
			& 4   & -4670.91 & \multicolumn{2}{c}{-4683.42} & -4683.68 \\
			& 5   & -4507.31 & \multicolumn{2}{c}{-4526.72} & -4612.83 \\
			\midrule
			\multirow{2}{*}{$^{11}\Sigma_u^+$} & 0   & -3677.68 & -3677.00 & -3677.68 & -3677.68 \\
			& 1   & -3507.89 & -3489.51 & -3507.82 & -3507.82 \\
			& 2   & -3341.77 & -3253.24 & -3341.26 & -3341.40 \\
			& 3   & -3180.16 & -3133.59 & -3185.51 & -3186.93 \\
			& 4   & -3023.07 & -3061.61 & -3001.04 & -3012.82 \\
			& 5   & -2870.22 & -2904.46 & -2866.05 & -2874.54 \\
			\midrule
			\multirow{2}{*}{$^{13}\Sigma_g^+$} & 0   & -548.68 & -548.65 & -548.67 & -548.68 \\
			& 1   & -497.16 & -496.47 & -496.84 & -497.15 \\
			& 2   & -449.18 & -443.48 & -448.36 & -449.26 \\
			& 3   & -404.71 & -382.88 & -390.96 & -404.67 \\
			& 4   & -363.58 & -369.09 & -360.62 & -362.99 \\
			& 5   & -325.67 & -315.46 & -310.29 & -325.72 \\
			\bottomrule
		\end{tabular*}
		\caption{Vibrational energy (in cm$^{-1}$) of \ch{Cr2} ($v =0-5$) in different electronic states. The benchmark (BM) results are obtained with a converged DVR basis. VQE uses quantum circuits shown in Fig. \ref{fig:cr2_pot_ansatz}. An extended table for a wider range of molecular states is in SM \cite{sm}.
		}
		\label{tab:diatomic}
	\end{table}

	
\subsection{Numerical ansatz optimization}	

\label{anzats-optimization} 

In this section, we demonstrate that DVR Hamiltonians also lead to very efficient quantum ansatze for representing ro-vibrational states of molecules by states of a quantum computer. 
In order to apply VQE, it is necessary to find a proper ansatz for $| \psi \rangle$. It is not always clear how to select the ansatz for $| \psi \rangle$. 
Previous work on electronic structure proposed various types of ansatzes for VQE with both fixed \cite{vqe-1,chem-3,uccsd,fixed-ansatz,uccgsd,symmetry,hva} and adaptive structure \cite{adaptvqe,adaptvqe-nuc,qubit-adaptvqe,iqcc,cluster-vqe,rotoselect,vans,evo-vqe,mog-vqe,qas}. 
% Fixed-structure ansatzes have a predetermined structure, while adaptive-structure ansatzes can adjust the circuit architecture to the problem. 
Unlike electronic structure problems, where interactions are pairwise additive, ro-vibrational energy calculations are determined by a wide range of potential energy landscapes, which are highly molecule-specific. 
In order to obtain the most efficient quantum circuit representations of $| \psi \rangle$ for VQE with DVR matrices, we develop and illustrate an iterative algorithm for ansatz construction that minimizes the number of entangling gates for each specific molecule. 
This algorithm is inspired by work in Refs. \cite{comp-search-1, comp-search-2, xuyang}. 

%Because the spatial extent and energy of ro-vibrational states are molecule-specific, it is important to develop a general approach to determining the most efficient $| \psi \rangle$ that can be adapted for any molecular species. 
%	We develop adaptive-structure ansatzes by optimizing CNOT gates through the compositional search, inspired by Refs. \cite{comp-search-1, comp-search-2}. 

%In this section, we develop and illustrate an iterative algorithm for ansatz construction that minimizes the number of free parameters for VQE.  							
%We consider several numerical examples of diatomic and tri-atomic molecules. 
	
		
		
	Our starting point is 
	%a hardware-efficient ansatz that consists of repeated alternating blocks of single-qubit and entangling gates 
	\cite{chem-3}: 
	\begin{align}
		\ket{\psi(\bm\varphi)} = &\prod_{d=0}^{k-1}\left[\prod_{q=0}^{n-1} U^{q,d}(\varphi^q_{d})\times U^d_{\rm ent}\right] \nonumber\\
		&\times \prod_{q=0}^{N-1} U^{q, k}(\varphi^q_{k})\ket{0^n}, \label{eq:hea}
	\end{align}
	where $U^{q,d}(\varphi)$ represent $R_Y = \exp(-i\varphi\sigma_Y/2)$ for qubit $q$, and $k$ is the number of repetitions of the ansatz blocks. 
	% In this work, we use $\sigma_\alpha = \sigma_Y$  and $\rm CNOT$ entangling gates. 
	The form of $U^d_{\rm ent}$ is determined by the ansatz optimization algorithm described below. For reference, we also use 
	\begin{eqnarray}
	U_{\rm ent} = \prod_{q=0}^{n-1}{\rm CNOT}(q, q+1),
	\label{linear}
	\end{eqnarray}
	denoted hereafter as linear to reflect the position of CNOT with $q$. 
	%	The two types of commonly used entangling blocks are the fully-entangled ansatz with $U_{\rm ent} = \prod_{q=0}^{n-1}\prod_{p=q+1}^{n}{\rm CNOT}(q, p)$ and 
	% the linearly-entangled ansatz with $U_{\rm ent} = \prod_{q=0}^{n-1}{\rm CNOT}(q, q+1)$.
	
	\begin{figure*}
		%		\begin{tabular}{c}
			\includegraphics[width=\textwidth]{opt_ansatz_triatomic}
			%		\end{tabular}
		\caption{Left: VQE error for lowest energy of Ar--HCl (open symbols) and Mg--NH (full symbols) computed with optimized quantum circuits in the right panel. Circles -- with $\mathcal{C}_1$ circuits; squares -- with  $\mathcal{C}_{0.01}$ circuits; triangles -- with ansatz (\ref{eq:hea}) using (left to right) $k=1,2,3$ and $4$ and Eq. (\ref{linear}). 
		%The squares represent the $R_Y$ gates and the circles show the entangling CNOT gates. 
		The $\mathcal{C}_1$ ansatz for Mg -- NH excludes the gate shown in green.}
		\label{fig:greedyent_const}
	\end{figure*}
	
	
	\begin{table*}
		\begin{tabular*}{\textwidth}{@{\extracolsep{\fill}}cccccccc}
			\toprule
			\multirow{2}{*}{Molecule} & \multirow{2}{*}{$v$} & \multirow{2}{*}{$E_v$ (experiment)} & \multirow{2}{*}{$E_v$ (computed in \cite{arhcl-pot})} & \multirow{2}{*}{$E_v$ (classical, present)} & \multicolumn{3}{c}{$E_{v}$ (VQE)} \\
			\cmidrule{6-8}
			& & & & & $\mathcal{C}_1$ & $\mathcal{C}_{0.01}$ & Linear \\
			\midrule
			\multirow{3}{*}{ArHCl} & \multirow{1}{*}{0} & -114.7~\cite{arhcl-obs-0} & -115.151 & -115.265 & -114.645 & -115.169 & -115.171 \\
			& \multirow{1}{*}{1} & -91.04~\cite{arhcl-obs-1, arhcl-obs-12} & -91.485 & -91.642 & -80.824 & -90.485 & -90.929 \\
			& \multirow{1}{*}{2} & -82.26~\cite{arhcl-obs-2} & -82.717 & -82.825 & -75.900 & -82.986 & -82.650 \\
			\midrule
			\multirow{3}{*}{MgNH} & \multirow{1}{*}{0} & - & - & -88.227 & -87.650 & -88.191 & -88.190 \\
			& \multirow{1}{*}{1} & - & - & -63.603 & -56.050 & -62.730 & -62.664 \\
			& \multirow{1}{*}{2} & - & - & -55.461 & -55.145 & -54.850 & -54.866 \\
			\bottomrule
		\end{tabular*}
		\caption{Vibrational energy (in cm$^{-1}$) of Ar--HCl and Mg--NH by VQE with 32 DVR points and 5-qubit circuits. 
			%displayed in Figure \ref{fig:vqe_partition}.
			%All energies are in cm$^{-1}$. % The zero of energy corresponds to the infinite separation between the atom and the diatomic molecule.
		}
		\label{tab:triatomic}
	\end{table*}
	
	
	
	Our algorithm starts with a non-entangled quantum state given by Eq. (\ref{eq:hea}) with a predetermined number of blocks $k$  and $U_{\rm ent}$ set to identity. The method considers ${\rm CNOT}(q, p)~ \forall q < p$ as candidate gates for $U^d_{\rm ent}$, with each $d$ segment treated independently. In each optimization step, the candidate gate that lowers the VQE energy is added without replacement until convergence. 
	%The search continues over the remaining candidate gates in the subsequent steps, adding one gate at a time until the desired convergence is achieved. 
	Here, we aim to converge the VQE calculation of the ground state either to 1 cm$^{-1}$ or 0.01 cm$^{-1}$, which yields quantum circuits of different complexity denoted $\mathcal{C}_1$ and $\mathcal{C}_{0.01}$. 
	This convergence error is with respect to the lowest eigenstate of the DVR matrix with the same number of DVR bases. 
	%Note that the results reported throughout this work are benchmarked by fully converged DVR calculations and literature results. 	
%	For circuits with more than one entangling block, each block is considered independent. 
	%resulting in a larger number of candidate gates (i.e. an ansatz with 4 repetitions on 5 qubits has 40 possible entangling gates to be added). 

	
	
	
	
	
	
	\section{Numerical Results} 
	
	\label{numerical-results}
	
	We calculate the ro-vibrational energy levels of diatomic (\ch{Cr2}) and triatomic (Ar--HCl and Mg--NH) systems. 
	For diatomic molecules we use the DVR Hamiltonian from Ref. \cite{colbert}. For triatomic complexes, we use the DVR approach by Choi and Light \cite{choi}. 
%		We use two classical constrained optimization methods to optimize the ansatz parameters from Refs. 
%%		: the bounded limited memory Broyden, Fletcher, Goldfarb, and Shanno method 
%		 \cite{lbfgsb-1, lbfgsb-2} 
%and Sequential Least SQuares Programming 
%and		 
%		 \cite{slsqp}. 
		 We benchmark the VQE results by the vibrational levels calculated using direct diagonalization with the converged DVR basis and previous literature results, where available.  
	
	
	
	
	
	Table \ref{tab:diatomic} (extended version in the SM \cite{sm}) demonstrates VQE for vibrational states $v=0$ -- $5$ of seven electronic states 
	%(${}^1\Sigma_g^+, {}^3\Sigma_u^+, {}^5\Sigma_g^+, {}^7\Sigma_u^+, {}^9\Sigma_g^+, {}^{11}\Sigma_u^+, {}^{13}\Sigma_g^+$) 
	of Cr$_2$ with zero rotational angular momentum. We use the interaction potentials from Ref. \cite{cr2-pot}, illustrated in Fig. \ref{fig:cr2_pot_ansatz}.  The VQE calculations use 16 DVR points placed to span the range including
	the minimum of the potential energy. The Hamiltonian is represented by 
the expansion (\ref{Pauli}) including $\approx 130$ Pauli terms. 
	Table \ref{tab:diatomic} displays VQE results obtained with three types of quantum circuits:  the linear ansatz (\ref{linear}) with 3 repetitions, and optimized circuits ${\cal C}_1$ (shown in Fig. \ref{fig:cr2_pot_ansatz}) and ${\cal C}_{0.01}$ (shown in the SM \cite{sm}). 
%	The quantum circuits used for the 
%	${\cal C}_1$ calculations are displayed in Fig. \ref{fig:cr2_pot_ansatz}. The ${\cal C}_{0.01}$ circuits are given in the supplemental material \cite{sm}. 
	%Table  \ref{tab:diatomic} and Fig. \ref{fig:cr2_pot_ansatz} show that VQE can be used to compute the vibrational levels of diatomic molecules with high precision using shallow quantum circuits. 
	
	For tri-atomic complexes, we use either 32 or 64 DVR basis states and accurate atom - molecule potential energy surfaces by Hutson for Ar--HCl \cite{arhcl-pot} and by Sold\'an et al. for Mg--NH \cite{mgnh-pot}. We keep both HCl and NH in the ground ro-vibrational state and compute the vibrational states supported by the atom - molecule interaction potential.  We obtain the DVR points for the triatomic systems by diagonalizing the coordinate representations. 
	%The basis functions used in the FBR control the number and placement of the DVR points. 
	The parameters to generate the DVR points are selected to cover the low-energy regions of the potential energy surface. 
	%(ArHCl: $N_{\theta} = 4, N_R=35, l=1, r_{\rm min} = 3.4$ \r{A}, $r_{\rm max}=5$ \r{A} and MgNH: $N_{\theta} = 4, N_R=20, l=1, r_{\rm min} = 3$ \r{A}, $r_{\rm max}=56$). 
	The 32-point DVR Hamiltonians are represented by 165 (Ar--HCl) and 170 (Mg--NH) Pauli terms in Eq. (\ref{Pauli}) acting on 5 qubits, while the 64-point DVR Hamiltonians are represented by 610 (Ar--HCl) and 621 (Mg--NH) Pauli terms (\ref{Pauli}) acting on 6 qubits. 	
	As above, we construct three types of quantum circuits: ${\cal C}_1$,  ${\cal C}_{0.01}$ and the linear ansatz (\ref{linear}) with $k$ repetitions.
	Figure \ref{fig:greedyent_const}
	%\ref{fig:vqe_excited}  
	and Table \ref{tab:triatomic} show that accurate results can be obtained with very shallow circuits.   
%	We observe that a small number of entangling gates is sufficient to ensure accurate calculations for both the ground  and excited state energy. 
%	Figure  \ref{fig:vqe_excited}  demonstrates the improvement of the computation accuracy with the increasing number of qubits (illustrated by the curves of different color) and the number of quantum gates (illustrated by the shaded area between the curves with the corresponding number of qubits). 
	
%	We also observe that some of the optimized circuits in Figs. \ref{fig:cr2_pot_ansatz} and \ref{fig:greedyent_const} are only partially entangled. To examine the effect of qubit entanglement, we repeated the circuit optimization for Ar -- HCl and Mg -- NH using a sequence of quantum circuits with  unentangled groups of entangled qubits.  The lowest errors achievable with partially entangled circuits for the ground state of Ar -- HCl are displayed in Fig. \ref{fig:vqe_partition}. 
	%We observe similar results for for Mg -- NH. 
%	The results in Fig. \ref{fig:vqe_partition} show that the ro-vibrational energy calculations can be computed with errors $< 1$ cm$^{-1}$ using partially entangled circuits. 
	
\section{Summary}	
	
	\label{summary}
	
	
	In conclusion, we have presented two significant results. First, we have shown that the structure of DVR matrices can be leveraged 
	 to represent molecular Hamiltonians by the polynomial (in the number of qubits $n$) number of quantum circuits. 	 
%	 efficient expansions in quantum gate sequences, which can be used to compute ro-vibrational energies and states with the number of quantum measurements that scales polynomially with the number of qubits.  This results in the exponential reduction of the computation complexity. 
	Second, we have demonstrated that DVR leads to efficient quantum circuits for VQE computations of ro-vibrational energy levels. 
To show this, we have introduced a general approach to constructing the quantum ansatz by combining DVR with VQE and a greedy search of gate sequences.  The results yield compact representations of vibrational states by quantum circuits of a gate-based quantum computer.  
%	The quantum states generated by these circuits can be optimized to compute the ro-vibrational energies of molecules. 
	We have shown that both the ground and excited vibrational energies can be computed with the relative accuracy of $< 1 \%$ using very simple, in some cases, partially entangled circuits. The accuracy of $1$ cm$^{-1}$ can be achieved with $< 20$ ($< 5$ entangling) gates and 4 qubits for diatomic molecules and $< 30$ ($< 9$ entangling) gates with 5 qubits for triatomic van der Waals complexes. 
	%Representing molecular vibrational states by quantum states of a quantum computer can also be used for quantum machine learning with quantum inputs \cite{qml-with-quantum-data}. 	
%	Because any potential is diagonal in a DVR basis, the present approach does not require analytical fits of potential energy surfaces and can be readily combined with VQE calculations of electronic energy or extended beyond molecular dynamics. 
This should be compared with previous VQE calculations of ro-vibrational energy levels that required extended quantum circuits with $>200$ (for CO, COH and O$_3$ molecules  \cite{mol-vib-3}) or between 44 and 140 292 (for CO$_2$, H$_2$CO and HCOOH molecules \cite{mol-vib-2})  entangling gates. 
We note that DVR does not require global fits of potential energy surfaces or integrals over the potential energy for the construction of the Hamiltonian matrix. 
	The implementation of VQE with DVR in first quantization does not require normal-mode analysis for the construction of the quantum ansatz. 
	The present approach can be be readily extended beyond molecular dynamics. For example, this method can be directly applied to designing efficient quantum circuits for variational computations of the eigenspectra of semiconductor quantum dots \cite{demler}.
	
\section{Supplementary Material}

The supplemental material \cite{sm} includes Sections 1 – 3. Section 1 provides detailed proofs that support Eqs. (5)
and (6) of the main text. Sections 2 is a detailed description of the quantum measurement algorithms, including a
summary of the measurement complexity. Section 3 presents the quantum circuits and
optimization details for the numerical computations.

\section{Acknowledgments}
This work was support by NSERC of Canada and the Stewart Blusson Quantum Matter Institute. 	
	
\section{Conflict of Interest Statement}
The authors have no conflicts to disclose.

\section{Author Contributions Statement}
All authors have designed the concept of this work and co-wrote the manuscript. The algorithm presented in Section IIA was developed by DB. The algorithm for ansatz optimization was developed by KA and RVK and the numerical calculations were performed by KA.

\section{Data Availability Statement}
The data that support the findings of this study are available within the article and its supplementary material.

	\bibliography{manuscript}
	
\end{document}

%	\begin{figure}
%		%		\begin{tabular}{c}
%			\includegraphics[width=\columnwidth]{arhcl_vqe_opt_excited64} 
%			%		\end{tabular}
%		\caption{Error of VQE energies for Ar -- HCl computed with optimized circuits ${\cal C}_1$ (upper boundary of shaded area) and ${\cal C}_{0.01}$ (lower boundary of shaded area) built with 5 qubits (corresponding to 32 DVR functions) and 6 qubits (corresponding to 64 DVR functions).}
%		\label{fig:vqe_excited}
%	\end{figure}
%	
%	\begin{figure}
%		%		\begin{tabular}{c}
%			\includegraphics[width=\columnwidth]{arhcl_partition_5} 
%			%		\end{tabular}
%		\caption{Minimum error of ground state energy for ArHCl predicted by VQE with partially entangled 5-qubit circuits. The bar labels indicate the partitioning of qubits.}
%		\label{fig:vqe_partition}
%	\end{figure}
	



	
	%\begin{table}
	%	\centering
	%	\begin{tabular*}{\columnwidth}{@{\extracolsep{\fill}}ccccc}
		%		\toprule
		%		Electronic & \multirow{2}{*}{$n$} & \multirow{2}{*}{$\Delta E_n^{\rm DVR}$} & \multirow{2}{*}{$\Delta E_n^{\rm VQE}$} & \multirow{2}{*}{$\Delta E_n^{\rm tot}$} \\
		%		state &  &  &  &  \\
		%		\midrule
		%		\multirow{4}{*}{$^1\Sigma_g^+$}  & 0   & -0.050               & $6.27\times10^{-8}$    & -0.050               \\
		%		& 1   & -0.294               & $1.85\times10^{-8}$    & -0.294               \\
		%		& 2   & 0.388                & $2.23\times10^{-9}$    & 0.388                \\
		%		& 3   & -0.360               & $-1.96\times10^{-8}$   & -0.360               \\
		%		\midrule
		%		\multirow{3}{*}{$^3\Sigma_u^+$}  & 0   & -0.073               & $5.33\times10^{-8}$    & -0.073               \\
		%		& 1   & 0.044                & $-2.68\times10^{-9}$   & 0.044                \\
		%		& 2   & 0.044                & $-2.36\times10^{-9}$   & 0.044                \\
		%		\midrule
		%		\multirow{3}{*}{$^5\Sigma_g^+$}  & 0   & 0.026                & $5.59\times10^{-5}$    & 0.026                \\
		%		& 1   & -0.009               & $1.94\times10^{-5}$    & -0.009               \\
		%		& 2   & 0.239                & $-1.06\times10^{-7}$   & 0.239                \\
		%		\midrule
		%		\multirow{2}{*}{$^7\Sigma_u^+$}  & 0   & -0.026               & $2.30\times10^{-6}$    & -0.026               \\
		%		& 1   & 0.246                & $1.30\times10^{-6}$    & 0.246                \\
		%		\midrule
		%		\multirow{2}{*}{$^9\Sigma_g^+$}  & 0   & -0.029               & $5.26\times10^{-6}$    & -0.029               \\
		%		& 1   & 0.304                & $7.08\times10^{-9}$    & 0.304                \\
		%		\midrule
		%		\multirow{3}{*}{$^{11}\Sigma_u^+$} & 0   & -0.001               & $7.48\times10^{-6}$    & -0.001               \\
		%		& 1   & 0.071                & $1.08\times10^{-4}$    & 0.071                \\
		%		& 2   & 0.362                & $-7.26\times10^{-7}$   & 0.362                \\
		%		\midrule
		%		\multirow{6}{*}{$^{13}\Sigma_g^+$} & 0   & 0.006                & $4.73\times10^{-8}$    & 0.006                \\
		%		& 1   & 0.005                & $-2.90\times10^{-11}$  & 0.005                \\
		%		& 2   & -0.082               & $-3.63\times10^{-11}$  & -0.082               \\
		%		& 3   & 0.038                & $-9.86\times10^{-11}$  & 0.038                \\
		%		& 4   & 0.590                & $-1.47\times10^{-10}$  & 0.590                \\
		%		& 5   & -0.054               & $-2.82\times10^{-10}$  & -0.054       \\
		%		\bottomrule
		%	\end{tabular*}
	%	\caption{DVR and VQE errors in energies of the lowest eigenstates $n=0,1,2, ..$ of \ch{Cr2} in different electronic states. Excited states with less than $\Delta E^{\rm tot}_{n} < 1$ cm${}^{-1}$  are displayed. DVR energies are calculated using 16 points. VQE calculations use the HEA ansatzes described in text with linear entanglement and 3 repetitions. All energies listed in the table are in cm$^{-1}$.}
	%	\label{tab:diatomic}
	%\end{table}
	
	%\begin{figure}
	%	\begin{tabular}{c}
		%		\includegraphics[width=\columnwidth]{dvr_cr2_16} \\
		%		\includegraphics[width=\columnwidth]{dvr_triatomic}
		%	\end{tabular}
	%	\caption{Error in the ground, first excited and second excited state energies of 16-point DVR for different electronic states of \ch{Cr2} (top) and 32-point DVR for ArHCl and MgNH (bottom). $\Delta E^{\rm DVR}_n = E^{\rm DVR}_n - E_n$, where $E^{\rm DVR}_n$ is the 16-point DVR energy of the $n$th state and $E_n$ is its corresponding converged energy value. \ch{Cr2} states not shown in the graph (${}^5\Sigma_g^+, {}^9\Sigma_g^+, {}^{11}\Sigma_u^+$, and ${}^{13}\Sigma_g^+$) have negligible errors in the 32 point DVR energies.}
	%	\label{fig:dvr_error}
	%\end{figure}
	
	
	%\begin{figure}
	%	\begin{tabular}{c}
		%		\includegraphics[width=\columnwidth]{arhcl_fullin_0} \\
		%		\includegraphics[width=\columnwidth]{mgnh_fullin_0}
		%	\end{tabular}
	%	\caption{Error in the VQE calculated ground state energy of 32 point DVR compared to the converged DVR values for the ArHCl (top) and MgNH (bottom). Black dashed line indicates the 32 point DVR error from converged DVR results. Different circuits correspond to $r=2,3,4,5$ in Eq. \refeq{eq:hea}.}
	%	\label{fig:arhcl_mgnh_vqe}
	%\end{figure}
	
	
	
	
	
	
	
	%\begin{figure}
	%	\begin{tabular}{c}
		%		\includegraphics[width=\columnwidth]{arhcl_vqe_opt_excited} \\
		%		\includegraphics[width=\columnwidth]{mgnh_vqe_opt_excited}
		%	\end{tabular}
	%	\caption{Error in VQE energies of the excited states of ArHCl (top) and MgNH (bottom). Simplest circuits and optimal circuits are displayed in Fig. \ref{fig:opt_ansatz} and are marked by circles and squares respectively, in Fig. \ref{fig:greedyent_const}. The black dashed line indicates 1 cm${}^{-1}$.}
	%	\label{fig:vqe_excited}
	%\end{figure}
	
	
	
%	\nocite{*}



%		\begin{table}
%		\centering
%		\begin{tabular*}{\columnwidth}{@{\extracolsep{\fill}}cccccc}
%			\toprule
%			Electronic & \multirow{2}{*}{$v$} & \multirow{2}{*}{BM} & \multicolumn{3}{c}{$E_v^{\rm VQE}$} \\
%			\cmidrule{4-6}
%			state & & & $\mathcal{C}_1$ & $\mathcal{C}_{0.01}$ & Linear \\
%			\midrule
%			\multirow{6}{*}{$^1\Sigma_g^+$}  & 0   & -15358.94 & -15358.87 & -15358.99 & -15358.99 \\
%			& 1   & -14846.67 & -14838.70 & -14846.75 & -14846.96 \\
%			& 2   & -14333.21 & -14310.29 & -14332.80 & -14332.82 \\
%			& 3   & -13826.93 & -13797.65 & -13826.87 & -13827.29 \\
%			& 4   & -13334.02 & -13275.12 & -13318.77 & -13335.45 \\
%			& 5   & -12861.37 & -12897.70 & -12871.32 & -12868.75 \\
%			\midrule
%			\multirow{6}{*}{$^3\Sigma_u^+$}  & 0   & -9862.07 & -9861.37 & -9862.14 & -9862.14 \\
%			& 1   & -9559.46 & -9538.97 & -9556.67 & -9559.41 \\
%			& 2   & -9300.92 & -9240.74 & -9266.82 & -9300.88 \\
%			& 3   & -9080.60 & 9068.09 & -9079.26 & -9085.40 \\
%			& 4   & -8897.58 & -8866.09 & -8870.57 & -8896.82 \\
%			& 5   & -8747.53 & -8729.41 & -8750.18 & ... \\
%			\midrule
%			\multirow{6}{*}{$^5\Sigma_g^+$}  & 0   & -7566.53 & -7565.88 & -7566.50 & -7566.50 \\
%			& 1   & -7416.28 & -7397.28 & -7416.28 & -7416.29 \\
%			& 2   & -7264.92 & -7181.98 & -7264.60 & -7264.69 \\
%			& 3   & -7114.40 & -7075.88 & -7117.83 & -7118.43 \\
%			& 4   & -6965.91 & -7001.98 & -6953.74 & -6958.08 \\
%			& 5   & -6820.04 & -6873.37 & -6837.02 & -6840.02 \\
%			\midrule
%			\multirow{6}{*}{$^7\Sigma_u^+$}  & 0   & -6519.01 & \multicolumn{2}{c}{-6519.04} & -6519.04 \\
%			& 1   & -6350.36 & \multicolumn{2}{c}{-6350.11} & -6350.11 \\
%			& 2   & -6183.38 & \multicolumn{2}{c}{-6185.17} & -6185.17 \\
%			& 3   & -6018.09 & \multicolumn{2}{c}{-6018.12} & -6018.28 \\
%			& 4   & -5854.50 & \multicolumn{2}{c}{-5848.69} & -5849.10 \\
%			& 5   & -5692.63 & \multicolumn{2}{c}{-5728.00} & -5731.33 \\
%			\midrule
%			\multirow{6}{*}{$^9\Sigma_g^+$}  & 0   & -5348.79 & \multicolumn{2}{c}{-5348.82} & -5348.82 \\
%			& 1   & -5175.81 & \multicolumn{2}{c}{-5175.51} & -5175.51 \\
%			& 2   & -5005.17 & \multicolumn{2}{c}{-5008.55} & -5008.56 \\
%			& 3   & -4836.85 & \multicolumn{2}{c}{-4829.80} & -4829.92 \\
%			& 4   & -4670.91 & \multicolumn{2}{c}{-4683.42} & -4683.68 \\
%			& 5   & -4507.31 & \multicolumn{2}{c}{-4526.72} & -4612.83 \\
%			\midrule
%			\multirow{6}{*}{$^{11}\Sigma_u^+$} & 0   & -3677.68 & -3677.00 & -3677.68 & -3677.68 \\
%			& 1   & -3507.89 & -3489.51 & -3507.82 & -3507.82 \\
%			& 2   & -3341.77 & -3253.24 & -3341.26 & -3341.40 \\
%			& 3   & -3180.16 & -3133.59 & -3185.51 & -3186.93 \\
%			& 4   & -3023.07 & -3061.61 & -3001.04 & -3012.82 \\
%			& 5   & -2870.22 & -2904.46 & -2866.05 & -2874.54 \\
%			\midrule
%			\multirow{6}{*}{$^{13}\Sigma_g^+$} & 0   & -548.68 & -548.65 & -548.67 & -548.68 \\
%			& 1   & -497.16 & -496.47 & -496.84 & -497.15 \\
%			& 2   & -449.18 & -443.48 & -448.36 & -449.26 \\
%			& 3   & -404.71 & -382.88 & -390.96 & -404.67 \\
%			& 4   & -363.58 & -369.09 & -360.62 & -362.99 \\
%			& 5   & -325.67 & -315.46 & -310.29 & -325.72 \\
%			\bottomrule
%		\end{tabular*}
%		\caption{Vibrational energy (in cm$^{-1}$) of \ch{Cr2} ($v =0-5$) in different electronic states. The benchmark (BM) results are obtained with a converged DVR basis. VQE computations use quantum circuits displayed in Fig. \ref{fig:cr2_pot_ansatz}. An extended version including results for a wider range of molecular states is in SM \cite{sm}}
%		\label{tab:diatomic}
%	\end{table}
%	
%
%
%
%In conclusion, we have demonstrated that vibrational states of diatomic and triatomic molecules can be represented by shallow circuits of a gate-based quantum computer. The quantum states generated by these circuits can be optimized to compute the ro-vibrational energies of molecules. 
%We have shown that both the ground and excited vibrational energies can be variationally computed with the relative accuracy of $< 1 \%$ using very simple, in some cases, partially entangled circuits, while the accuracy of $1$ cm$^{-1}$ can be achieved with $< 20$ ($< 5$ entangling) gates acting on 4 qubits for diatomic molecules and $< 30$ ($< 9$ entangling) gates acting on 5 qubits for triatomic van der Waals complexes. Representing molecular vibrational states by quantum states of a quantum computer can be used not only for computing the ro-vibrational energy levels by quantum measurements, but also for quantum machine learning taking quantum states on input \cite{qml-with-quantum-data}. 
%
%	
%	
%	
%	
%
%	For triatomic molecules, we use the DVR by Choi and Light \cite{choi}. This approach constructs the ro-vibrational Hamiltonian in a finite basis of orthogonalized Sturmian functions, associated Legendre functions and parity-adapted Wigner rotation functions with respect to the Jacobi coordinates $(R, r, \theta)$ and the angular momentum projection in the body-fixed frame ($K$). Fixing the total angular momentum of the system and the bond length of the diatomic molecule to its equilibrium value ($r = r_e$), leaves two variables $(R, \theta)$ defining the position of the atom with respect to the molecule. This leads to a three-dimensional ${}^{\rm FBR}H_{ijK}^{i'j'K'}$ with $i$ and $j$ representing the  basis functions associated with $R$, $\theta$. This Hamiltonian is then diagonalized in the $R$ and $\theta$ coordinates leading to a set of DVR points
%	\begin{align}
%		{^R\matr{\Delta}} = &{^R\matr{T}}\cdot\matr{R}\cdot({^R\matr{T}})^\top, \\
%		{^{\theta K}\matr{\Delta}} = &{^{\theta K}\matr{T}}\cdot\matr{X}\cdot({^{\theta K}\matr{T}})^\top, \\
%		\matr{T} =& {^R\matr{T}}\otimes{^{\theta K}\matr{T}}\otimes\matr{I}_K
%	\end{align}
%	where ${}^R\Delta_{ii'}$ and ${}^{K\theta}\Delta_{jj'}$ are the $R$ and $x = \cos\theta$ representations in the FBR and $\matr{R}$ and $\matr{X}$ are diagonal matrices defining the DVR quadrature points $\{R_\alpha\}$ and $\{\theta_{K\beta}\}$. The transformation matrices ${^R\matr{T}}$ and ${^{\theta K}\matr{T}}$ are then applied to the Hamiltonian evaluated in the FBR
%	\begin{align}
%		{^{\rm DVR}\matr{H}} = &\matr{T}^\top\cdot{^{\rm FBR}\matr{H}}\cdot\matr{T}
%	\end{align}
%	resulting in a DVR Hamiltonian ${^{\rm DVR}\matr{H}}$ with a diagonal representation of the potential energy.
%	%	\begin{equation}
%		%		V_{\alpha\beta}^{\alpha'\beta'} = V(R_\alpha,\theta_{K\beta})\delta_{\alpha\alpha'}\delta_{\beta\beta'}.
%		%	\end{equation}
	
