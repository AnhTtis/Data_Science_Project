\documentclass[onecolumn,aps,pra,preprintnumbers,bibnotes10pt,superscriptaddress,longbibliography,groupedaddress,notitlepage]{revtex4-2}
\usepackage[utf8]{inputenc}

\usepackage{amsmath,amsfonts,amssymb,amsthm,mathtools}
%\usepackage{url}\part{title}
\usepackage{graphicx}
\usepackage{epsfig}
%%BeginIpePreamble
\usepackage{dsfont}{\tiny }
\usepackage[dvipsnames]{xcolor}
\usepackage{placeins}
%%BeginIpePreamble

\usepackage{color}
\usepackage{amsmath,amssymb,amsthm, mathtools, enumitem, empheq, gensymb, textcomp, bm, booktabs, multirow}
%\usepackage[margin=1in]{geometry}
%\usepackage{chemformula}
\usepackage{dsfont}{\tiny }


\def\bea{\begin{eqnarray}}
\def\eea{\end{eqnarray}}
\def\nn{\nonumber}
\def\beq{\begin{equation}}
\def\eeq{\end{equation}}
\def\ba{\begin{eqnarray}}
\def\ea{\end{eqnarray}}
\def\be{\ba\displaystyle}
\def\ee{\ea}
\def\F{{\cal F}}
\def\e{\epsilon}
\def\ve{\varepsilon}

\newtheorem{theorem}{Theorem}[section]
\newtheorem{lemma}[theorem]{Lemma}
\newtheorem{proposition}{Proposition}[section]
\newtheorem{postulate}[theorem]{Postulate}
\newtheorem{claim}[theorem]{Claim}
\newtheorem{definition}[theorem]{Definition}
\newtheorem{notation}[theorem]{Notation}
\newtheorem{remark}[theorem]{Remark}
\newtheorem{corollary}[theorem]{Corollary}
\newtheorem{conjecture}[theorem]{Conjecture}


\usepackage{chemformula}


\newcommand{\la}{\langle}
\newcommand{\ra}{\rangle}

\newcommand{\w}{{\mathsf w}}

\newcommand{\comment}{\Comment \textcolor[RGB]{67,115,222}}

\usepackage{algorithm}
\usepackage[noend]{algpseudocode}

\setcounter{equation}{0}
\renewcommand*{\thesection}{\arabic{section}}
\renewcommand{\theequation}{S\arabic{equation}}
\renewcommand{\figurename}{Fig.}
\newcommand{\vts}{\vect{\theta}}

%\setcounter{equation}{0} \setcounter{figure}{0} \renewcommand\thefigure{S\arabic{figure}} \renewcommand\theequation{S\arabic{equation}}

\begin{document}

\title{Supplemental Material for ``Variational quantum computation of vibrational energy using discrete variable representation''}
	\author{K. Asnaashari}
	\author{D. Bondarenko}
	\author{R. V. Krems}
	\affiliation{ 
		Department of Chemistry, University of British Columbia, \\Vancouver, B.C. V6T 1Z1, Canada \\
		Stewart Blusson Quantum Matter Institute, \\
		Vancouver, B.C. V6T 1Z4, Canada
	}
	
\date{\today}

\maketitle
%\section{Sufficiently decaying sequences}
%The discrete variable representation (DVR) of the kinetic energy operator consists of parts with matrix elements that are:
%\begin{enumerate}
%    \item the same on either diagonals or anti-diagonals and
%    \item falling the further they are from either the main diagonal or the left upmost and right downmost corners of the matrix.
%\end{enumerate}
%We exploit the first observation by using tensor products to construct DVR matrices for a large number of qubits. The second observation is leveraged in an approximation scheme that provides a sufficiently small error while constructing a matrix with just a polynomial number of non-zero (anti-)diagonals. For this approximation scheme to work, we need to specify how fast should the matrix elements fall.
%\begin{definition}
%Consider a sequence $\{s_i\}_{i=1}^{L}$. We define two classes of sufficiently decaying sequences $\mathcal{P}_\alpha$ and $\mathcal{S}_\beta$ as
%\be
%    \{s_i\}_{i=1}^{L} \in \mathcal{P}_\alpha \ \mathrm{ if } \ \frac{\sum_{i=1}^k s_i}{\sum_{j=1}^L s_j} \in O(k^{-\alpha}), \ \alpha > 0;\\
%    \{s_i\}_{i=1}^{L} \in \mathcal{S}_\beta \ \mathrm{ if } \ \frac{\sum_{i=1}^k s_i + s_{L-i}}{\sum_{j=1}^L s_j} \in O(k^{-\beta}), \ \beta > 0.
%\ee
%\end{definition}


This supplemental material includes Sections 1 -- 3. Section 1
provides detailed proofs that support Eqs. (5) and (6) of the main text. Sections 2 is a detailed description of the quantum measurement algorithms, including a summary of the measurement complexity. Section 3 presents the quantum circuits and optimization details for the  numerical computations. 



\section{Decay of off-diagonal matrix elements}\label{Question}


We consider a discrete variable representation (DVR) introduced by Colbert and Miller \cite{DVR_Colbert_Miller} with  $N$ basis states. 
As mentioned in the main text, we use a number encoding to map the DVR basis states onto the qubit states, which allows VQE to compute the eigenvalues of the Hamiltonian matrix of size
	$2^n\times 2^n$ using $n$ qubits. Thus, $N = 2^n$. 

The kinetic energy operator in the DVR basis for a particle with mass $m$ on a grid of $x=[a,b]$ with the uniform spacing $\Delta x$  has the matrix elements \cite{DVR_Colbert_Miller}:
\be
T_{ij} = \frac {\hbar^2}{2m \Delta x^2} \times
    \begin{cases}
    d(i) & \ i=j,\\
    f\left(|i-j|\right) + g(i+j), & i \neq j.
    \end{cases}
  %\left\{f\left(|i-j|\right)\right\}_{|i-j|=1}^{2^n-1} \in \mathcal{P}_\alpha,\quad
  %\left\{g\left(i+j\right)\right\}_{i+j=2}^{2^n+2} \in \mathcal{S}_\beta.
\ee
%
Our first goal is to show that
\be
    \sum_{k=s}^{2^n-1}\left| f(k) \right| \leq O\left( s^{-\alpha} \right), \ \alpha>0, \label{Prop_f}\\
    \sum_{k=r}^{2^{n+1}-1-r}\left| g(k) \right| \leq O\left(  r^{-\beta} \right), \ \beta>0, \label{Prop_g}
\ee
where $r \ll N$ and $s \ll N$, and $N$ is assumed to be large.  

\subsection{Expansion in Fourier functions}
%In this subsection we show that the KE DVR from Appendix A of~\cite{Colbert Miller} satisfy the assumptions for $f(\cdot)$ and $g(\cdot)$ specified in the Section~\ref{Question}.

Consider an equally-spaced 1D grid with endpoints $a$ and $b$
\be
    x_j = b + j\Delta x,\quad \Delta x = \frac{b-a}{N}, \quad j = 1,\dots, N-1.
\ee
Given an orthonormal basis $\left\{\phi_n \right\}$, the matrix elements of the kinetic energy can be written as
\be
    T_{ij} =   - \frac {\hbar^2}{2m} \Delta x \sum_{n=0}^{N-1}
    \phi_n (x_i) \phi_n^{''} (x_j).
\ee
For the boundary conditions $\phi_n(x_0 \equiv a) = \phi_n(x_N \equiv b) =0$, the associated Fourier functions are
\be
    \phi_n(x) = \left(\frac{2}{b-a}\right)^{1/2} 
    \sin\left[ \frac{n \pi (x-a)}{b-a} \right].
\ee
For this basis, the sum can be evaluated analytically yielding \cite{DVR_Colbert_Miller}
\be \label{KEDVR}
T_{ij} = 
    \begin{cases}
    \frac {\hbar^2}{2m} \frac{1}{(b-a)^2}\frac{\pi^2}{2}\left[ \frac{2 N^2 +1}{3} - \frac{1}{\sin^2 (\pi j / N) }\right], & \ i=j,\\
    \frac {\hbar^2}{2m} \frac{(-1)^{i-j}}{(b-a)^2}\frac{\pi^2}{2}
    \left\{  \frac{1}{\sin^2 [\pi (i-j) / 2 N] } -  \frac{1}{\sin^2 [\pi (i+j) / 2 N] }\right\}, & \ i \neq j.
    \end{cases}
\ee
%If we aim to represent grid points as an $n$-qubit computational basis states, we have $N= 2^n$.
%

Following Colbert and Miller, we now consider the cases of infinite lattices, relevant for radial molecular coordinates, and a case of finite $a$ and $b$. 


\subsubsection{Infinite lattice $[a= -\infty,b = \infty ]$}
\label{infinite}

The finite grid spacing $\Delta x = \frac{b-a}{N}$ requires $N \rightarrow \infty$. Eq.~(\ref{KEDVR}) becomes
\be \label{infDVR}
    T_{ij} = \frac {\hbar^2}{2m \Delta x^2}(-1)^{i-j}
    \begin{cases}
    \frac {\pi^2}{3},  & \ i=j,\\
    \frac {2}{(i-j)^2}, & i \neq j.
    \end{cases}
\ee
Note that in this case $g(i+j)$ vanishes. It is easy to see that 
\be
    | f(k)| =  \frac {2}{(i-j)^2}
\ee
satisfies Eq.~(\ref{Prop_f}). To prove this, we note that for functions that are (strictly) monotonically decaying in absolute value
\be \label{ineq:int}
    \int_{s}^{u} |f(k)| dk < \sum_{k=s}^{u} |f(k)| < \int_{s-1}^{u-1} |f(k)| dk.
\ee
For our choice of $f(\cdot)$ this yields the bound
\be \label{f:proof}
    \sum_{k=s}^{2^n - 1} |f(k)| < 2  \int_{s-1}^{2^n - 2} \frac{dk}{k^2} = 2\left( \frac{1}{s-1} - \frac{1}{2^n-2} \right) < 3  s^{-1}.
\ee



\subsubsection{Infinite lattice $[a= 0,b = \infty ]$} \label{ss:halfinfinite}

For the lattice with $[a= 0,b = \infty ]$, the matrix elements
\be
    T_{ij} = \frac {\hbar^2}{2m \Delta x^2}(-1)^{i-j}
    \begin{cases}
    \frac {\pi^2}{3} - \frac{1}{2i^2},  & \ i=j,\\
    \frac {2}{(i-j)^2} - \frac {2}{(i+j)^2}, & i \neq j.
    \end{cases}
\ee
include $g(k)$. Since Eq. (\ref{f:proof}) applies to $g(k)$, the arguments in Subsubsection (\ref{infinite}) apply to Eq. (12). 


%However, the proof that $g(k)$ satisfies Equation~(\ref{Prop_g}) is identical to the one presented in Equation~(\ref{f:proof}). 
%As this $g(k)$ is monotonically decreasing, a streamlined construction for the anti-diagonals is possible, see Subsection~\ref{Streamlined}.

\subsubsection{Finite lattice $[a,b]$}
%We show that KE DVR in Equation~(\ref{KEDVR}) obeys Equations~(\ref{Prop_f}) and \ref{Prop_g}) via a slight generalization of the arguments from Equation~(\ref{f:proof}). 
Observe that the prefactor in Eq.~(\ref{KEDVR}) satisfy $\frac {\hbar^2}{2m} \frac{1}{(b-a)^2}\frac{\pi^2}{2} \cdot \frac{2m \Delta x^2}{\hbar^2} = \frac{\pi^2 }{2 N^2}$. Consider the following functions
\be
    |\tilde f(k)| &=& \frac{\pi^2 }{2 N^2}
    \left[  \frac{1}{\sin^2 (\pi k / 2 N) } -  1\right],\\
    |\tilde g(k)| &=& \frac{\pi^2 }{2 N^2}
    \left[ \frac{1}{\sin^2 (\pi k / 2 N) } - 1\right].
\ee
Eq.~(\ref{ineq:int}) yields
\be
    \sum_{k=s}^{N-1}  \frac{1}{\sin^2 (\pi k / 2 N) } &=& \frac{2 N}{\pi} \sum_{k=s} \frac{\pi}{2 N} \frac{1}{\sin^2 (\pi k / 2 N) }\\
    &<& \frac{2 N}{\pi} \int_{\frac{\pi(s-1)}{2 N}}^{\frac{\pi}{2}} \frac{dx}{\sin^2(x)}\\ \label{sum:int}
    %= \left.-\frac{2 N}{\pi} \cot(x)\right|_{\frac{\pi(s-1)}{2 N}}^{\frac{\pi}{2}}\\
    &=& \frac{2 N}{\pi} \cot \left[ \frac{\pi(s-1)}{2 N} \right]
\ee
Using the Laurent series for $\cot(y)$ around $y=0$, we can write
\be
    \frac{2 N}{\pi} \cot \left[ \frac{\pi(s-1)}{2 N} \right] = \frac{4 N^2}{\pi^2 (s-1)} - \frac{(s-1)}{3} + O\left(\frac{s^3}{N^2}\right).
\ee
This yields 
\be
    \sum_{k=s}^{2^n-1} |\tilde f(k)| &<&  \frac{\pi^2 }{2 N^2}
    \left[ \frac{4 N^2}{\pi^2 (s-1)} - \frac{(s-1)}{3} -  (N-1-s)  + O\left(\frac{s^3}{N^2}\right)\right]\\
    &<&  \frac{2}{s-1} + O\left(\frac{s^3}{N^4}\right) \\
    &<& 3 s^{-1}.
\ee
%manifesting the desired property for $f(\cdot)$.

For $g(k)$, we exploit 
the symmetry of $\sin(\cdot)$ to write
\be
    \sum_{k=r}^{2^{n+1}-1-r} \frac{1}{\sin^2 (\pi k / 2 N) } = 1 + 2 \sum_{k=r}^{2^{n}-1} \frac{1}{\sin^2 (\pi k / 2 N) },
\ee
and apply Eqs. (19)-(21).


%After the exploitation of this symmetry, the rest of the proof that $\sum_{k=r}^{2^{n+1}-1-r} |g(k)| < 5 E(T) s^{-1}$ is identical to the proof for $f(\cdot)$.

\section{Quantum measurement of the kinetic energy}

Given a quantum state $|\psi \rangle$ prepared by a quantum computer, our aim is to show that
\be
\tau = \la \psi| \hat H | \psi \ra + O(\epsilon) 
\ee
can be measured with the number of quantum circuits that scales polynomially with $n$ and $1/\epsilon$. We seek to transform $|\psi \rangle$ by short-depth $\hat V_i$, 
%where $i$ is $ \leq{\rm Poly}(n, 1/\epsilon)$, 
so that
\be
\tau = \sum_{i=1}^{{\rm poly}(n, 1/\epsilon)} \sum_{j=1}^{2^n} w_{ij} \left|\la \psi |V_i^{\dagger} | j \ra\right|^2
\ee
where $|j\rangle$ is the state of $n$ qubits in the computational $Z$ basis.

%\begin{figure}[h]
 %   \centering
  %  \includegraphics{kDiagCirc.pdf}
  %  \caption{Can we perform a polynomial in $n$ and $1/\epsilon$ number of experiments shown on this figure to get $\tau$?}
   % \label{fig:kDiagCirc}
%\end{figure}
%

%\section{Stuff}

%Let the matrix elements fall polynomially fast starting from some initial easy to evaluate point $q_0, r_0$. For the examples we consider, the initial point is simply the diagonal matrix, $q_0 = 0$. We consider such $d(\cdot)$, $f(\cdot)$ and $g(\cdot)$ that
%\be
%      \frac{f(k)}{\la\psi | T^{[q_0, r_0]}| \psi \ra } \in O\left(k^{-\alpha}\right),\\ \frac{g(k)}{\la\psi | T^{[q_0, r_0]}| \psi \ra }  \in O\left(\min\left\{k, 2^n-k\right\}^{-\beta}\right),\\
%  {\la\psi | T^{[q_0, r_0]}| \psi \ra }  - \sum_{k=1}^{2^n-1}\left|f(k)\right| - \sum_{k=2}^{2^n+2}\left|g(k)\right|> O(\epsilon).
%\ee

%
\subsection{DVR matrix truncation}
The results of Section 1  suggest that it is possible to truncate the DVR matrix of the kinetic energy as follows: 
\be
    T^{(s,r)}_{ij} = 
     \frac {\hbar^2}{2m \Delta x^2} \times \begin{cases}
    d(i) & \ i=j,\\
    f\left(|i-j|\right) + g(i+j) & \ i \neq j,\ |i-j|< s,\ i+j < r,\\
    0 \ \text{otherwise}.
    \end{cases}
\ee
We now aim to determine the values of $s$ and $r$ such that $\la \psi| T | \psi \ra = \la \psi| T^{(s,r)} | \psi \ra + O(\epsilon)$.

Consider a $k^\text{th}$ (anti)-diagonal matrix 
\be
    K^{k \pm \pm}_{ij} \equiv \delta_{i \pm j, \pm k},
\ee
where only $++$, $+-$ and $--$ combination of signs are used. It is straightforward to check that $\| K \psi \|_2 \leq \| \psi \|_2$ (and the inequality is tight). The Cauchy–Schwarz inequality $|\la \psi, K \psi \ra| \leq \| \psi \|_2 \| K \psi \|_2$ yields the bound for the expectation value of $K$
\be
    |\la \psi, K \psi \ra| \leq \| \psi \|^2_2.
\ee
Due to the triangle inequality, for $\| \psi \|^2_2=1$, the approximation error is bounded by
\be
    \left|\la\psi|\left(T -T^{(s,r)}\right) |\psi\ra \right| &=& 
    \left|\la\psi|\left(\sum_{k=s}^{2^n-1}f(k)\left[K^{k-+} + K^{k--}\right] + \sum_{k=r}^{2^{n+1}-1-r}g(k) K^{k++}\right) |\psi\ra \right|
    \nn\\ &\leq&
    E(T) \cdot \left( 2 \sum_{k=s}^{2^n-1}\left| f(k) \right| + \sum_{k=r}^{2^{n+1}-1-r}\left| g(k) \right| \right)
    \nn\\ &\leq&
    E(T)\cdot\left[ 2 O\left( s^{-\alpha} \right) + O \left(r^{-\beta}\right)\right],
\ee
where $E(T) =  {\hbar^2} \left [ {2m \Delta x^2} \right ]^{-1}$.
Due to Eqs. (\ref{Prop_f})-(\ref{Prop_g}) for $f(\cdot)$ and $g(\cdot)$, $T^{(s,r)}$ is a good approximation of $T$ for $s \sim \epsilon^{-\frac{1}{\alpha}}$ and $r \sim \epsilon^{-\frac{1}{\beta}}$.


%
\subsection{Measuring the expectation value of the truncated operator $T^{(s,r)}$}
We leverage the structure of the matrix of interest $T$ by truncating and decomposing it into a diagonal matrix and and $k^{\text{th}}$ diagonal and anti-diagonal components. 
\be 
    T^{\left(s,r\right)}
    &=& D + \sum_{k=1}^{s-1} f(k) t^{k[n]} + \sum_{k=1}^{r-1} \left( g(k) a^{k[n]} + g(2^n-k) a^{(2^n-k)[n]} \right),
\ee
where
\be
    D_{ij} &=& \delta_{ij}\left( d(i) - g(2i) - \sum_{k=1}^{s-1}f(k) q^{k[n]}(i)  \right)\\
    t^{k[n]}_{ij} &=& \delta_{|i-j|, k} + \delta_{ij}q^{k[n]}(i),\\
    a^{k[n]}_{ij} &=& \delta_{i+j, k}   
\ee
and $q^{k[n]}(i)$ are easy to compute classically. Our strategy is to measure the expectation values of each of the $\left\{D, t^{k[n]}, a^{k[n]} \right\}$ separately using the induction over $n$ and add them together with the corresponding weights.

\subsection{Measuring $D$}
The expectation values of diagonal matrices with classically easy to compute matrix elements can be measured in one step. Indeed, just measure $| \psi \ra$ in $Z$ basis and weight the outcomes, i.e. pick $\{ V_i \} = \{ \mathds{1} \}$ and $w_{1i} = D_{ii}$.

\subsection{Measuring $t^{k[n]}$ and computing $q^{k[n]}(i)$}
\begin{figure}[h]
    \centering
    \includegraphics[scale=1.5]{kthDiagSM.pdf}~~~~~~~~~~~~~~~~~~~~\includegraphics[scale=1]{BellCirc.pdf}
    \caption{Left: Measuring $t^{k[n]}$ that includes $k^{\text{th}}$ diagonals. \textcolor{ForestGreen}{Dark green} matrix elements denoted by \textcolor{ForestGreen}{circles $\circ$} are obtained during the basis step, \textcolor{Maroon}{dark red $\otimes$}-denoted elements---via the tensor product and the \textcolor{Fuchsia}{magenta +} ---by performing extra measurements in a Bell basis. The measured operator might have some non-zero elements on the diagonal that are easy to compute, denoted by the \textcolor{Goldenrod}{golden} line.\\
    Right: Circuit that prepares $|+\ra_{j_1 \dots j_l, p_1 \dots p_l} = \frac{|j_1 \dots j_l\ra + |p_1 \dots p_l\ra}{\sqrt{2}}, \ \exists i: j_i \neq p_i$ states from the computational basis. The identity circuit acts on the subset of qubits for which $\{ j_i = p_i\}$ (\textcolor{Green}{green}); a version of the GHZ circuit acts on the rest of the qubits.
    \label{fig:BellCirc}
    \label{fig:kthDiag}}
\end{figure}
We construct $t^{k[n]}$ inductively. The construction is inspired by~\cite{DVR_off-diags,DVRStackEx}. The basis of construction is measuring $t^{k[l]}$, where $l$ is the smallest number of qubits that allow for the $k$-th diagonal
\be
    l \equiv \left\lceil \log_2(k+1) \right\rceil.
\ee
Consider a matrix element above the main diagonal with coordinates $(j_1 \dots j_l, p_1 \dots p_l)$. Define $l$-qubit ``$+$'' states
\be
    |+\ra_{j_1 \dots j_l, p_1 \dots p_l} = \frac{|j_1 \dots j_l\ra + |p_1 \dots p_l\ra}{\sqrt{2}}, \quad \exists i: j_i \neq p_i,
\ee
that can be obtained from $Z$ basis states by the circuits on Fig.~\ref{fig:BellCirc} right.
%A projection on a ``$+$'' Bell state is
%\be
%    |\mathrm{Bell}_{+j} \ra \la \mathrm{Bell}_{+j}| = \frac{1}{2}
%        \left(
%            | j_1\dots j_l \ra \la j_1\dots j_l | + | \bar{j}_1\dots\bar{j}_l \ra \la \bar{j}_1\dots \bar{j}_l | + \right. \nn\\
%        \left.  | j_1\dots j_l \ra \la \bar{j}_1\dots \bar{j}_l | + | \bar{j}_1\dots\bar{j}_l \ra \la j_1\dots j_l |
%        \right),
%\ee

The  projection $|+\ra_{j_1 \dots j_l, p_1 \dots p_l} \la + |_{j_1 \dots j_l, p_1 \dots p_l}$ has exactly 1 non-zero matrix element above the main diagonal with coordinates $(j_1 \dots j_l, p_1 \dots p_l)$, one bellow, and two on the main diagonal of magnitude $1/2$. The binary representation of the row and column positions of these off-diagonal elements differ by at most $l$ bits. Thus, to get the expectation value of $t^{k[l]}$ we need to measure in at most\footnote{The number of need bases might be lower is different pairs of matrix elements can be measured by mapping different computational states to different ``$+$'' states with the same circuit.} $k$ $l$-qubit bases containing ``$+$'' states. 

The step is adding a new qubit. Observe that $\mathds{1} \otimes t^{k[m]}$ looks almost like $t^{k[m+1
]}$. The only missing elements are at the middle of the tensor product matrix, and there are $k$ pairs of them (see \textcolor{Fuchsia}{magenta +} on the Fig.~\ref{fig:kthDiag} left). How do we get those extra $k$ pairs? Via employing extra measurements in the basis containing ``$+$'' states, just like for the basis of the construction.

The complexity of this algorithm to measure the expectation value of $t^{k[n]}$ consists of: \footnote{This upper bound assumes that we need a separate basis for every pair of off-diagonal elements. The actual complexity may be lower.}
\begin{itemize}
    \item At most $2^l - k \leq k$ expectation value measurements of projection on ``$+$'' states via circuits of depth at most $l+1$ to obtain $t^{k[l]}$ (see \textcolor{ForestGreen}{dark green $\circ$} on the Fig.~\ref{fig:kthDiag}).
    \item Measuring complexity of $\mathds{1} \otimes t^{k[m]}$ is the same as the measuring complexity of $t^{k[m]}$ (see \textcolor{Maroon}{dark red $\otimes$} on the Fig.~\ref{fig:kthDiag} left).
    \item Thus, when adding a new qubit we need to include measurements of at most $k$ projections on ``$+$'' states via circuits of depth at most $\left\lceil\log_2 2k \right\rceil$. Starting with $l$ qubits, we add $n-l$ qubits (see \textcolor{Fuchsia}{magenta +} on Fig.~\ref{fig:kthDiag} left).
\end{itemize}
All together,  the number of measurement bases is
\be
    \textrm{Comp}\left( t^{k[n]} \right) &\leq & 2^l - k + (n-l)k\\
    &<& \left(n + 1 - \log_2 k \right) k.
\ee

In addition to $k^{\text{th}}$ diagonals, $t^{k[n]}$ contains extra contributions to matrix elements on the main diagonal $q^{k[n]}(i)$ that have to be accounted for. Fortunately, it is easy to deduce how to compute these contributions from the measurement protocol for $t^{k[n]}$, see Algorithm~\ref{alg:Q}.
\begin{algorithm}[H]
	\caption{Computing $q^{k[n]}(i)$ from binary representation of $i$}\label{alg:Q}
\begin{algorithmic}
%\Function{$q^{k[n]}$}{$i$}
\State \textbf{define} $l \gets \lceil \log_2 (k+1) \rceil$,
\State $\quad last\_l\_bits \gets i \mod 2^{l}$,
\State $\quad anti\_last\_l\_bits \gets 2^{l} - last\_l\_bits$;
\State \textbf{initialize} $output \gets 1$,
\State $\quad j \gets l+1$;
\While {$j \leq n$}
\If {$j^{\mathrm{th}}$ bit of $i$ is 1}
\If {$last\_l\_bits < k$ or $anti\_last\_l\_bits < l$} 
\State $output \rightarrow out+1$;
\EndIf
\EndIf
\State $j \rightarrow j+1$;
\EndWhile
\Return $output$.
%\EndFunction
	\end{algorithmic}
\end{algorithm}

\subsection{Measuring $a^{k[n]}$}
\begin{figure}[h]
    \centering
    \includegraphics[scale=1.5]{kthAntiDiag.pdf}
    \caption{Measuring the set of anti-diagonals $\left\{a^{k[m+1]}\right\}_k$ given a protocol for each of $\left\{a^{j[m]} \right\}_j$. Larger $k^{\text{th}}$ anti-diagonals can be obtained by taking a tensor product of either
    $\begin{bmatrix}
    1 & 0 \\
    0 & 0
    \end{bmatrix}$
    or
    $\begin{bmatrix}
    0 & 0 \\
    0 & 1
    \end{bmatrix}$ with the $k^{\text{th}}$ smaller ones, and for $k\geq 2^{m}$ additionally combining them with the tensor product of
    $\begin{bmatrix}
    0 & 1 \\
    1 & 0
    \end{bmatrix}$
    with the $2^m-k^{\text{th}}$ smaller anti-diagonal (dashed lines).
    }
    \label{fig:kthAntiDiag}
\end{figure}
We also use inductive construction for the anti-diagonals $a^{k[n]}$. For the basis we need to obtain expectation values of
$a^{1[1]} = \begin{bmatrix}
1 & 0 \\
0 & 0
\end{bmatrix}$, 
$a^{2[1]} = \begin{bmatrix}
0 & 1 \\
1 & 0
\end{bmatrix}$ and
$a^{3[1]} = \begin{bmatrix}
0 & 0 \\
0 & 1
\end{bmatrix}$.
This can be achieved by sampling one-qubit measurements of $\psi$ in $Z$ to get $\left\la a^{1[1]} \right\ra_\psi$ and $\left\la a^{3[1]} \right\ra_\psi$ and in $X$ to get $\left\la a^{2[1]} \right\ra_\psi$. When measuring in $Z$, the probability of obtaining $\uparrow$ yields $\left\la a^{1[1]} \right\ra_\psi$, while the probability of obtaining $\downarrow$ yields $\left\la a^{3[1]} \right\ra_\psi$.

For the step of the construction (see Fig.~\ref{fig:kthAntiDiag}), observe that 
\be
a^{k[m+1]} = 
\begin{cases}
    a^{1[1]} \otimes a^{k[m]}& \text{ for } k\leq 2^m,
    \\
    a^{1[1]} \otimes a^{k[m]} + 
        a^{2[1]} \otimes a^{( - k \mod 2^m)[m]}& \text{ for } 2^m<k<2\cdot2^{m},
    \\
    a^{2[1]} \otimes a^{(2^m)[m]} & \text{ for } k=2\cdot2^{m},
    \\
    a^{3[1]} \otimes a^{(k \mod 2^m)[m]} + 
    a^{2[1]} \otimes a^{(-k \mod 2^m)[m]}& \text{ for } 2\cdot 2^{m} < k\leq 3\cdot2^{m}
    \\
    a^{3[1]} \otimes a^{(k \mod 2^m)[m]} & \text{ for } k < 4\cdot2^{m}.
\end{cases}
\ee
As such, every time we add a qubit, we (at most) double the number of measurements. Indeed, for $m+1$ qubits we measure the first qubit in $Z$ or $X$ basis, just like for the basis of the construction, and perform the protocol for $m$ qubits on the rest. We would like to get the expectation values that are $\epsilon$-close to the target, so we perform $p-1$ steps for
\be
    p = \left\lceil \log_2 r \right\rceil 
    \sim - \left\lceil \frac{\log_2 \epsilon}{\beta} \right\rceil.
\ee

For the conclusion of the construction, we use the rapid decline of the anti-diagonal matrix elements to construct large matrices with only first and last $2^p$ anti-diagonals being non-zero
\be
    a^{k[n]} =
\begin{cases}
    \bigotimes_1^{n - p} a^{1[1]} \otimes a^{k[p]}& \text{ for } k\leq 2^p ,
    \\
    %\text{will be multiplied by } g(k) \in  O(\epsilon) & \text{ for } \epsilon^{\frac{1}{\beta}}<k. %2^{n}-\epsilon^{\frac{1}{\beta}},
    \text{can be neglected}  & \text{ for } 2^p<k<2^n - 2^p,
    \\
    \bigotimes_1^{n - p} a^{3[1]} \otimes a^{(k \mod 2^p)[p]} & \text{ for } 2^{n}-2^p \leq k.
\end{cases}
\ee
Thus, we can measure $n-p$ qubits in computational basis and use the step-wise protocol for the rest. We need to take into account only the all-$\uparrow$ and all-$\downarrow$ outputs for the first $n-p$ qubits, yielding upmost and downmost $2^p$ anti-diagonals. The total number of bases for this protocol is bounded by
\be
    2^{p} < 2^{1-\frac{\log_2 \epsilon}{\beta}} \sim 2\epsilon^{-\frac{1}{\beta}}.
\ee


\subsubsection{Streamlined construction if only upper anti-diagonals are needed}\label{Streamlined}

As discussed above, some lower anti-diagonals can be neglected. In this case we can compute all anti-diagonals for $p-2$ steps and proceed with
\be
a^{k[p]} = 
\begin{cases}
    a^{1[1]} \otimes a^{k[p-1]}& \text{ for } k\leq 2^{p-1},
    \\
    a^{1[1]} \otimes a^{k[p-1]} + 
        a^{2[1]} \otimes a^{( - k \mod 2^{p-1})[p]}& \text{ for } 2^{p-1}<k<2^{p},
    \\
    a^{2[1]} \otimes a^{(2^{p-1})[p-1]} & \text{ for } k=2^{p},
    \\
    \text{can be neglected} & \text{ for } 2^{p}<k;
\end{cases}
\ee
note that measuring the relevant part of $\{ a^{k[p]}\}_{k\leq 2^p}$ requires as many bases as measuring all of $\{ a^{k[p-1]}\}_{k=1}^{2^{p}-1}$. Indeed, we supplement the measurement of first $2^{p-1}$ anti-diagonals by the measurement of an additional qubit in $Z$ basis, and for the other anti-diagonals---in $X$ basis. Finally,
\be
    a^{k[n]} =
\begin{cases}
    \bigotimes_1^{n - p} a^{1[1]} \otimes a^{k[p]}& \text{ for } k\leq 2^p,
    \\
    \text{can be neglected} & \text{ for } 2^p<k. %2^{n}-\epsilon^{\frac{1}{\beta}},
    %\\
    %\bigotimes_1^{n - s} a^{3[1]} \otimes a^{(k \mod 2^s)[s]} & \text{ for } 2^{n}-\epsilon^{\frac{1}{\beta}} \leq k.
\end{cases}
\ee
After measuring $n-p$ qubits in computational basis and using the $a^{k[p]}$ protocol for the rest, we need to take into account only the all-$\uparrow$ outputs for the first $n-p$ qubits. This yields upmost $2^p$ anti-diagonals with the total number of bases for the protocol bounded by
\be
    2^{p-1} < 2^{-\frac{\log_2 \epsilon}{\beta}} = \epsilon^{-\frac{1}{\beta}}.
\ee


\subsection{Complexity of measuring $\tau$}
For the quantity of interest
\be
    \tau = \left\la D \right\ra + \sum_{k=1}^{s}  f(k) \left\la t^{k[n]} \right\ra  + \sum_{k=1}^{r}\left[ g(k) \left\la a^{k[n]} \right\ra  + g(2^n-k) \left\la a^{(2^n-k)[n]} \right\ra\right], 
    \\ \ \text{where} \ 
    s \sim \epsilon^{-\frac{1}{\alpha}} 
    \ \text{and} \
    r \sim \epsilon^{-\frac{1}{\beta}},
\ee
we the we need to measure in at most
\be
    \textrm{Comp}\left( \tau \right) \leq 1 + \sum_{k=1}^{\sim \epsilon^{-\frac{1}{\alpha}}} \left(2^l - k + (n-l)k \right) + 2\epsilon^{-\frac{1}{\beta}} = O\left[\frac{\epsilon^{-\frac{2}{\alpha}} n}{2}+ 2\epsilon^{-\frac{1}{\beta}} \right]
\ee
number of bases. We need circuits depicted on Fig.~\ref{fig:BellCirc} of depth at most $\lceil\log_2(2s)\rceil \sim 1 - \frac{\log_2\left(\epsilon\right)}{\alpha}$ and to sample at most $O\left( \frac{1}{\sqrt{\epsilon}}\right)$ times per circuit.


\section{Additional details of numerical VQE results}

\subsection{Vibrational energy levels of Cr$_2$}

The following table is an extended version of table I in main text, including results for a wider range of molecular states.  
	
		\begin{table}
		\centering
		\begin{tabular*}{\columnwidth}{@{\extracolsep{\fill}}cccccc}
			\toprule
			Electronic & \multirow{2}{*}{$v$} & \multirow{2}{*}{BM} & \multicolumn{3}{c}{$E_v^{\rm VQE}$} \\
			\cmidrule{4-6}
			state & & & $\mathcal{C}_1$ & $\mathcal{C}_{0.01}$ & Linear \\
			\midrule
			\multirow{6}{*}{$^1\Sigma_g^+$}  & 0   & -15358.94 & -15358.87 & -15358.99 & -15358.99 \\
			& 1   & -14846.67 & -14838.70 & -14846.75 & -14846.96 \\
			& 2   & -14333.21 & -14310.29 & -14332.80 & -14332.82 \\
			& 3   & -13826.93 & -13797.65 & -13826.87 & -13827.29 \\
			& 4   & -13334.02 & -13275.12 & -13318.77 & -13335.45 \\
			& 5   & -12861.37 & -12897.70 & -12871.32 & -12868.75 \\
			\midrule
			\multirow{6}{*}{$^3\Sigma_u^+$}  & 0   & -9862.07 & -9861.37 & -9862.14 & -9862.14 \\
			& 1   & -9559.46 & -9538.97 & -9556.67 & -9559.41 \\
			& 2   & -9300.92 & -9240.74 & -9266.82 & -9300.88 \\
			& 3   & -9080.60 & 9068.09 & -9079.26 & -9085.40 \\
			& 4   & -8897.58 & -8866.09 & -8870.57 & -8896.82 \\
			& 5   & -8747.53 & -8729.41 & -8750.18 & ... \\
			\midrule
			\multirow{6}{*}{$^5\Sigma_g^+$}  & 0   & -7566.53 & -7565.88 & -7566.50 & -7566.50 \\
			& 1   & -7416.28 & -7397.28 & -7416.28 & -7416.29 \\
			& 2   & -7264.92 & -7181.98 & -7264.60 & -7264.69 \\
			& 3   & -7114.40 & -7075.88 & -7117.83 & -7118.43 \\
			& 4   & -6965.91 & -7001.98 & -6953.74 & -6958.08 \\
			& 5   & -6820.04 & -6873.37 & -6837.02 & -6840.02 \\
			\midrule
			\multirow{6}{*}{$^7\Sigma_u^+$}  & 0   & -6519.01 & \multicolumn{2}{c}{-6519.04} & -6519.04 \\
			& 1   & -6350.36 & \multicolumn{2}{c}{-6350.11} & -6350.11 \\
			& 2   & -6183.38 & \multicolumn{2}{c}{-6185.17} & -6185.17 \\
			& 3   & -6018.09 & \multicolumn{2}{c}{-6018.12} & -6018.28 \\
			& 4   & -5854.50 & \multicolumn{2}{c}{-5848.69} & -5849.10 \\
			& 5   & -5692.63 & \multicolumn{2}{c}{-5728.00} & -5731.33 \\
			\midrule
			\multirow{6}{*}{$^9\Sigma_g^+$}  & 0   & -5348.79 & \multicolumn{2}{c}{-5348.82} & -5348.82 \\
			& 1   & -5175.81 & \multicolumn{2}{c}{-5175.51} & -5175.51 \\
			& 2   & -5005.17 & \multicolumn{2}{c}{-5008.55} & -5008.56 \\
			& 3   & -4836.85 & \multicolumn{2}{c}{-4829.80} & -4829.92 \\
			& 4   & -4670.91 & \multicolumn{2}{c}{-4683.42} & -4683.68 \\
			& 5   & -4507.31 & \multicolumn{2}{c}{-4526.72} & -4612.83 \\
			\midrule
			\multirow{6}{*}{$^{11}\Sigma_u^+$} & 0   & -3677.68 & -3677.00 & -3677.68 & -3677.68 \\
			& 1   & -3507.89 & -3489.51 & -3507.82 & -3507.82 \\
			& 2   & -3341.77 & -3253.24 & -3341.26 & -3341.40 \\
			& 3   & -3180.16 & -3133.59 & -3185.51 & -3186.93 \\
			& 4   & -3023.07 & -3061.61 & -3001.04 & -3012.82 \\
			& 5   & -2870.22 & -2904.46 & -2866.05 & -2874.54 \\
			\midrule
			\multirow{6}{*}{$^{13}\Sigma_g^+$} & 0   & -548.68 & -548.65 & -548.67 & -548.68 \\
			& 1   & -497.16 & -496.47 & -496.84 & -497.15 \\
			& 2   & -449.18 & -443.48 & -448.36 & -449.26 \\
			& 3   & -404.71 & -382.88 & -390.96 & -404.67 \\
			& 4   & -363.58 & -369.09 & -360.62 & -362.99 \\
			& 5   & -325.67 & -315.46 & -310.29 & -325.72 \\
			\bottomrule
		\end{tabular*}
		\caption{Vibrational energy (in cm$^{-1}$) of Cr$_2$ ($v =0-5$) in different electronic states. The benchmark (BM) results are obtained with a converged DVR basis. VQE computations use quantum circuits displayed in Fig. 1 of main text. }
		\label{tab:diatomic}
	\end{table}

\clearpage
\newpage

\subsection{Optimized quantum circuits for diatomic and triatomic VQE calculations}


	\subsubsection{$\mathcal{C}_{1}$ and $\mathcal{C}_{0.01}$ ansatzes for C\lowercase{r}${}_2$}
	
	Figures \ref{fig:cr2_ansatz_c1} and \ref{fig:cr2_ansatz_c001} display the 4-qubit $\mathcal{C}_1$ and $\mathcal{C}_{0.01}$ ansatzes for the seven electronic states of \ch{Cr2}. These circuits are ansatzes obtained using the greedy compositional search and correspond to the simplest circuits that can estimate the 16-point DVR energy with less than 1 cm${}^{-1}$ and 0.01 cm${}^{-1}$ error respectively. As expected, $\mathcal{C}_{0.01}$ are more complex than the $\mathcal{C}_1$ circuits and include one or more additional entangling gates. It should also be noted that $\mathcal{C}_1$ and $\mathcal{C}_{0.01}$ are the same circuit in the case of ${}^7\Sigma$ and ${}^9\Sigma$ states, meaning that the smallest ansatz found in the compositional search with an error of less than 1 cm${}^{-1}$ from the 16-point DVR ground state energy, also has less than 0.01 cm${}^{-1}$ error from the same value.
	
	\subsubsection{Circuit optimization for triatomics}
	
	Figure \ref{fig:ansatz_opt} demonstrates how the VQE error in calculating the ground state energy changes throughout the compositional search. As described in the main text, the ansatz compositional search adds entangling gates to an un-entangled ansatz one at a time, minimizing the VQE-calculated ground state energy at each step. Each of the curves $k=1, 2, 3, 4$ corresponds to the total number of entangling blocks, as in Eq. (3), considered in the compositional search.
	
	The dashed and dotted lines indicate 1 cm${}^{-1}$ and 0.01 cm${}^{-1}$ error from the 32-point DVR ground state respectively. The optimized circuits $\mathcal{C}_1$ and $\mathcal{C}_{0.01}$ which correspond to the simplest circuits found below the dashed and dotted lines are indicated with circles and squares respectively.

	\subsubsection{VQE optimization}
	
	In this work, we optimized the ansatz parameters in VQE using the bounded limited memory Broyden, Fletcher, Goldfarb, and Shanno method (L-BFGS-B) \cite{lbfgsb-1, lbfgsb-2} and Sequential Least SQuares Programming (SLSQP) \cite{slsqp}. Figure \ref{fig:vqe_opt} shows the expectation value of the Hamiltonian over the output quantum state of the 5-qubit $\mathcal{C}_1$ and $\mathcal{C}_{0.01}$ ansatzes of ArHCl during the optimization. Both ansatzes are displayed in the right panel of Fig. (4).
	
	
	
	\begin{figure*}
		\includegraphics[width=\textwidth]{cr2_ansatz_c1}
		\caption{Quantum circuits for VQE yielding the ground state energy with error $\leq 1$ cm${}^{-1}$. The squares represent the $R_Y$ gates and the circles show the entangling CNOT gates.}
		\label{fig:cr2_ansatz_c1}
	\end{figure*}

	\begin{figure*}
		\includegraphics[width=\textwidth]{cr2_ansatz_c001}
		\caption{Quantum circuits for VQE yielding the ground state energy with error $\leq 0.01$ cm${}^{-1}$. The squares represent the $R_Y$ gates and the circles show the entangling CNOT gates.}
		\label{fig:cr2_ansatz_c001}
	\end{figure*}

	\begin{figure*}
		\begin{tabular}{cc}
			\includegraphics[width=0.5\textwidth]{arhcl32_const} &
			\includegraphics[width=0.5\textwidth]{mgnh32_const} 
		\end{tabular}
		\caption{Error in 5-qubit VQE ground state energy of ArHCl (left) and MgNH (right) in the ansatz composition algorithm with respect to the number of entangling gates added for $k=1,2,3,4$ ansatz repetitions. Circles indicate $\mathcal{C}_1$ circuits and squares indicate $\mathcal{C}_{0.01}$ circuits on the optimization graphs.}
		\label{fig:ansatz_opt}
	\end{figure*}

	\begin{figure*}
		\begin{tabular}{cc}
			\includegraphics[width=0.5\textwidth]{arhcl_opt_1} &
			\includegraphics[width=0.5\textwidth]{arhcl_opt_2} 
		\end{tabular}
		\caption{VQE optimization graphs for 5-qubit $\mathcal{C}_1$ (left) and $\mathcal{C}_{0.01}$ (right) of ArHCl. The dotted line indicates the converged DVR ground state energy and the dashed line indicates the 32-point DVR ground state energy.}
		\label{fig:vqe_opt}
	\end{figure*}
	
%	\begin{figure*}
%		\begin{tabular}{cc}
%			\includegraphics[width=0.5\textwidth]{graphs/arhcl_partition_5} &
%			\includegraphics[width=0.5\textwidth]{graphs/arhcl_partition_6} \\
%			\includegraphics[width=0.5\textwidth]{graphs/mgnh_partition_5} &
%			\includegraphics[width=0.5\textwidth]{graphs/mgnh_partition_6} \\
%		\end{tabular}
%		\caption{Minimum error of ground state energy for ArHCl (top) and MgNH (bottom) predicted by VQE with partially entangled 5-qubit (left) and 6-qubit (right) circuits. The bar labels indicate the partitioning of qubits.}
%		\label{fig:partitions}
%	\end{figure*}





\clearpage
\newpage

%\FloatBarrier
 %\vspace*{-4\baselineskip}
\bibliography{DVRSM}

\end{document}