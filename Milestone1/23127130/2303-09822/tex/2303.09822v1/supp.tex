\documentclass[%
aps,
pra,
% jmp,
% bmf,
% sd,
% rsi,
amsmath,amssymb,
%
preprint%
%author-year,%
%author-numerical,%
% Conference Proceedings
]{revtex4-2}
\usepackage{amsmath,amssymb,amsthm, mathtools, enumitem, empheq, gensymb, textcomp, bm, booktabs, multirow}
\usepackage[margin=1in]{geometry}
\usepackage{chemformula}
\usepackage{qcircuit}
\usepackage{tabularx}

\DeclarePairedDelimiter\bra{\langle}{\rvert}
\DeclarePairedDelimiter\ket{\lvert}{\rangle}
\DeclarePairedDelimiterX\braket[2]{\langle}{\rangle}{#1 \delimsize\vert #2}
\DeclarePairedDelimiterX\Braket[3]{\langle}{\rangle}{#1 \delimsize\vert #2 \delimsize\vert #3}
\newcommand{\crtwo}{Cr\textsubscript{2}}
\newcommand{\matr}[1]{\mathbf{#1}}

\makeatletter
\def\@email#1#2{%
	\endgroup
	\patchcmd{\titleblock@produce}
	{\frontmatter@RRAPformat}
	{\frontmatter@RRAPformat{\produce@RRAP{*#1\href{mailto:#2}{#2}}}\frontmatter@RRAPformat}
	{}{}
}%

\newcommand{\supplementarysection}{%
	\setcounter{figure}{0}% Reset figure counter
	\let\oldthefigure\thefigure% Capture figure numbering scheme
	\renewcommand{\thefigure}{SM\oldthefigure}% Prefix figure
	%	 number with S
	\setcounter{section}{0}% Reset figure counter
	\let\oldthesection\thesection% Capture figure numbering scheme
	\renewcommand{\thesection}{SM\oldthesection}% Prefix figure
	%	 number with S
}

\makeatother
\begin{document}
	
	\title{Supplemental material for ``Compact quantum circuits for variational calculations of ro-vibrational energy levels of molecules on a quantum computer''}
	\author{K. Asnaashari}
	\author{R. V. Krems}
	\affiliation{ 
		Department of Chemistry, University of British Columbia, Vancouver, B.C. V6T 1Z1, Canada \\
		Stewart Blusson Quantum Matter Institute, Vancouver, B.C. V6T 1Z4, Canada
	}
	
	\date{\today}
	
	\maketitle
	
		\supplementarysection
	\section{$\mathcal{C}_{1}$ and $\mathcal{C}_{0.01}$ ansatzes for C\lowercase{r}${}_2$}
	
	Figures \ref{fig:cr2_ansatz_c1} and \ref{fig:cr2_ansatz_c001} display the 4-qubit $\mathcal{C}_1$ and $\mathcal{C}_{0.01}$ ansatzes for seven electronic states of \ch{Cr2}. These circuits are ansatzes obtained using the greedy compositional search and correspond to the simplest circuits that can estimate the 16-point DVR energy with less than 1 cm${}^{-1}$ and 0.01 cm${}^{-1}$ error respectively. As expected, $\mathcal{C}_{0.01}$ are more complex than the $\mathcal{C}_1$ circuits and include one or more additional entangling gates. It should also be noted that $\mathcal{C}_1$ and $\mathcal{C}_{0.01}$ are the same circuit in the case of ${}^7\Sigma$ and ${}^9\Sigma$ states, meaning that the smallest ansatz found in the compositional search with an error of less than 1 cm${}^{-1}$ from the 16-point DVR ground state energy, also has an error of $<0.01$ cm${}^{-1}$.
	
	\section{Circuit optimization for tri-atomic van der Waals complexes}
	
	Figure \ref{fig:ansatz_opt} demonstrates how the VQE error of the ground state energy changes throughout the compositional search. As described in main text, the ansatz compositional search adds entangling gates to an un-entangled ansatz one at a time, minimizing the VQE-calculated ground state energy at each step. Each of the curves $k=1, 2, 3, 4$ corresponds to the total number of entangling blocks in Eq. (3) of main text, considered in the compositional search.
	
	The dashed and dotted lines indicate the 1 cm${}^{-1}$ and 0.01 cm${}^{-1}$ errors from the 32-point DVR ground state respectively. The optimized circuits $\mathcal{C}_1$ and $\mathcal{C}_{0.01}$ which correspond to the simplest circuits found below the dashed and dotted lines are indicated with circles and squares respectively.
	
	\section{VQE optimization}
	
	In this work, we optimized the ansatz parameters in VQE using the bounded limited memory Broyden, Fletcher, Goldfarb, and Shanno method (L-BFGS-B) \cite{lbfgsb-1, lbfgsb-2} and Sequential Least SQuares Programming (SLSQP) \cite{slsqp}. Figure \ref{fig:vqe_opt} shows the expectation value of the Hamiltonian over the output quantum state of the 5-qubit $\mathcal{C}_1$ and $\mathcal{C}_{0.01}$ ansatzes of ArHCl during the optimization. Both ansatzes are displayed in the right panel of Fig. 4.
	
	\section{Partial entanglement}
	
	Figure \ref{fig:partitions} shows the best results obtained for 5- and 6-qubit ArHCl and MgNH using the partially entangled compositional search. We observe a similar pattern between the 5- and 6-qubit results of each molecule, however, the patterns observed for the two complexes are significantly different. While the compositional search can still find ansatzes that can accurately obtain the ground state energy of both ArHCl and MgNH with the final two qubits un-entangled from the rest, un-entangling the last qubit in the compositional search leads to a large increase in the VQE ground state accuracy for MgNH, which is not the case for ArHCl. 
	
	
	\begin{figure*}
		\includegraphics[width=\textwidth]{cr2_ansatz_c1}
		\caption{Quantum circuits for VQE yielding the ground state energy with error $\leq 1$ cm${}^{-1}$. The squares represent the $R_Y$ gates and the circles show the entangling CNOT gates.}
		\label{fig:cr2_ansatz_c1}
	\end{figure*}
	
	\begin{figure*}
		\includegraphics[width=\textwidth]{cr2_ansatz_c001}
		\caption{Quantum circuits for VQE yielding the ground state energy with error $\leq 0.01$ cm${}^{-1}$. The squares represent the $R_Y$ gates and the circles show the entangling CNOT gates.}
		\label{fig:cr2_ansatz_c001}
	\end{figure*}
	
	\begin{figure*}
		\begin{tabular}{cc}
			\includegraphics[width=0.5\textwidth]{arhcl32_const} &
			\includegraphics[width=0.5\textwidth]{mgnh32_const} 
		\end{tabular}
		\caption{Error in 5-qubit VQE ground state energy of ArHCl (left) and MgNH (right) in the ansatz composition algorithm with respect to the number of entangling gates added for $k=1,2,3,4$ ansatz repetitions. Circles indicate $\mathcal{C}_1$ circuits and squares indicate $\mathcal{C}_{0.01}$ circuits on the optimization graphs.}
		\label{fig:ansatz_opt}
	\end{figure*}
	
	\begin{figure*}
		\begin{tabular}{cc}
			\includegraphics[width=0.5\textwidth]{arhcl_opt_1} &
			\includegraphics[width=0.5\textwidth]{arhcl_opt_2} 
		\end{tabular}
		\caption{VQE optimization graphs for 5-qubit $\mathcal{C}_1$ (left) and $\mathcal{C}_{0.01}$ (right) of ArHCl. The dotted line indicates the converged DVR ground state energy and the dashed line indicates the 32-point DVR ground state energy.}
		\label{fig:vqe_opt}
	\end{figure*}
	
	\begin{figure*}
		\begin{tabular}{cc}
			\includegraphics[width=0.5\textwidth]{arhcl_partition_5} &
			\includegraphics[width=0.5\textwidth]{arhcl_partition_6} \\
			\includegraphics[width=0.5\textwidth]{mgnh_partition_5} &
			\includegraphics[width=0.5\textwidth]{mgnh_partition_6} \\
		\end{tabular}
		\caption{Minimum error of ground state energy for ArHCl (top) and MgNH (bottom) predicted by VQE with partially entangled 5-qubit (left) and 6-qubit (right) circuits. The bar labels indicate the partitioning of qubits.}
		\label{fig:partitions}
	\end{figure*}
	
	%	\nocite{*}
	\clearpage
	\bibliography{manuscript}
	
\end{document}
