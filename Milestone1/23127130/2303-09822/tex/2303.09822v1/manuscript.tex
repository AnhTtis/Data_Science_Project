\documentclass[%
aps,
prl,
% jmp,
% bmf,
% sd,
% rsi,
amsmath,amssymb,
%
reprint%
%author-year,%
%author-numerical,%
% Conference Proceedings
]{revtex4-2}
\usepackage{amsmath,amssymb,amsthm, mathtools, enumitem, empheq, gensymb, textcomp, bm, booktabs, multirow}
\usepackage[margin=1in]{geometry}
\usepackage{chemformula}

\DeclarePairedDelimiter\bra{\langle}{\rvert}
\DeclarePairedDelimiter\ket{\lvert}{\rangle}
\DeclarePairedDelimiterX\braket[2]{\langle}{\rangle}{#1 \delimsize\vert #2}
\DeclarePairedDelimiterX\Braket[3]{\langle}{\rangle}{#1 \delimsize\vert #2 \delimsize\vert #3}
\newcommand{\crtwo}{Cr\textsubscript{2}}
\newcommand{\matr}[1]{\mathbf{#1}}

\makeatletter
\def\@email#1#2{%
	\endgroup
	\patchcmd{\titleblock@produce}
	{\frontmatter@RRAPformat}
	{\frontmatter@RRAPformat{\produce@RRAP{*#1\href{mailto:#2}{#2}}}\frontmatter@RRAPformat}
	{}{}
}%

\makeatother
\begin{document}
	
	\preprint{AIP/123-QED}
	
	\title{Compact quantum circuits for variational calculations of ro-vibrational energy levels of molecules on a quantum computer}
	\author{K. Asnaashari}
	\author{R. V. Krems}
	\affiliation{ 
		Department of Chemistry, University of British Columbia, Vancouver, B.C. V6T 1Z1, Canada \\
		Stewart Blusson Quantum Matter Institute, 
		Vancouver, B.C. V6T 1Z4, Canada
	}
	
	\date{\today}
	
	\begin{abstract}
		While quantum computing algorithms have been successfully applied to electronic structure of molecules, applications of quantum computing to molecular dynamics remain scarce. The variability of intra-molecular interaction potentials gives rise to vibrational states with a  wide range of properties. It is therefore challenging to obtain a general representation of molecular ro-vibrational states by states of a quantum computer with a limited number of qubits and quantum gates. We demonstrate a general approach to computing the ro-vibrational energy levels of molecules by combining the discrete variable representation of molecular Hamiltonians with variational quantum eigensolvers and a greedy search of gate permutations, yielding accurate representations of both ground and excited vibrational states by compact quantum circuits with a small number of gates.  To illustrate the generality and adaptability of this approach, we compute the vibrational energy levels of Cr$_2$ in seven electronic states as well as the vibrational energy levels of van der Waals complexes Ar--HCl and Mg--NH, illustrating that accuracy of 1 cm$^{-1}$ can be achieved with between 2 and 9 entangling gates. 
		
		
	\end{abstract}
	
	\maketitle
	
	
	Accurate calculation of molecular properties is considered a promising application of quantum computing. 
	Although current quantum computers are capable of manipulating, measuring, and entangling qubits \cite{qc-1,qc-2,qc-3,qc-4,qc-5,qc-6,qc-7,qc-8,qc-9,qc-10,qc-11,qc-12,qc-13,qc-14,qc-15}, they are limited by noise and hardware restrictions, such as the number of operations and gates that can be executed consecutively \cite{error-1,error-2,error-3}. A key goal of current research is therefore to develop efficient algorithms for accurate quantum computation with as few entangled qubits and gate operations as possible. 
	The eigenstates of molecular Hamiltonians can be obtained on quantum computers by variational quantum eigensolvers (VQE) \cite{vqe-review, vqd} that employ sequences of gates (quantum circuits) operating on qubits to prepare quantum states tailored for specific problems. 
	VQEs have been applied for solving the electronic structure problem for molecules \cite{vqe-1,vqe-2,chem-1,chem-2,chem-3,chem-4,chem-5,chem-6,chem-7} and lattice models \cite{chem-3,vqe-review,lattice-1}. 
	%The quantum circuits for such computations can be informed by the electronic structure method \cite{uccsd,uccgsd,symmetry,hva} or the relevant space of electronic orbitals. 
	However, applications of VQE to molecular dynamics  -- such as computations of ro-vibrational energies and states -- have been limited to Refs. \cite{dvr-dynamics, morse, mol-vib, mol-vib-2, mol-vib-3, co2-1, co2-2}. 
	%Yet, by comparison with electronic structure calculations that benefit from many accurate classical methods readily available, nuclear dynamics calculations are much more challenging. 
	The challenge of molecular dynamics calculations is partly due to the complexity of inter- and intra-molecular interaction potentials, which are molecule-specific and exhibit widely varying landscapes of potential energy. 
	Refs. \cite{dvr-dynamics, morse,  mol-vib} demonstrated VQE computations of ro-vibrational states for specific model problems, while Refs. \cite{co2-1, co2-2} used VQE to compute the Hamiltonian matrix, which was then diagonalized classically. 
	%, such as one-dimensional Morse potential \cite{morse}. 
	%or two-mode model of CO$_2$ \cite{co2-1, co2-2}. 
	Refs. \cite{mol-vib-3, mol-vib-2} demonstrated a general approach for computing ro-vibrational energy levels of polyatomic molecules inspired by previous work  on electronic structure. However, the methods of  Refs. \cite{mol-vib-2, mol-vib-3} require extended quantum circuits including $>200$ \cite{mol-vib-3} or between 44 and 140 292  \cite{mol-vib-2} entangling gates. A current key challenge for application of quantum computing to molecular dynamics is to develop a general approach that can (i) be applied to deeply bound molecules and van der Waals interactions alike and (ii) yield high accuracy with compact quantum circuits with a small number of entangling gates. 
	
	%It is therefore not known how to represent ro-vibrational states of molecules for variation computations on a quantum computer for arbitrary molecules governed by general potential energy surfaces
	
	Here, we address this problem by first formulating a general approach to computing the ro-vibrational energy levels of molecules with VQE based on discrete variable representations (DVR) of molecular Hamiltonians \cite{colbert, choi}. We show that DVRs of molecular Hamiltonian matrices allow accurate 
	VQE computations of both ground and excited vibrational states with shallow quantum circuits. 
	To further decrease the complexity of quantum circuits, we then demonstrate a compositional search algorithm that explores the space of gate permutations to yield simplified quantum circuits suitable for ro-vibrational VQE computations. 
	To illustrate the generality of this approach, we compute the ro-vibrational energy levels of Cr$_2$
	in seven different electronic states \cite{cr2-pot} as well as the ro-vibrational energy levels of van der Waals complexes Ar--HCl($^1\Sigma$) and Mg--NH($^3\Sigma$). These molecular systems exhibit ro-vibrational states with widely different energies and spatial variations of wave functions and energy level patterns. We build quantum circuits that produce VQE results with accuracy $<1$ cm$^{-1}$ for several ro-vibrational energy levels, illustrating the ability of VQE to compute the rotational constants and vibrational anharmonicity, with between 2 and 9 entangling gates.  	
	
	
	{\it Method description: VQE based on DVR.} In VQE, a quantum computer calculates the expectation value of a Hamiltonian matrix over a parameterized quantum state $\Braket{\psi(\bm\varphi)}{\hat{H}}{\psi(\bm\varphi)}$. This value is then minimized with respect to $\bm\varphi$ to yield 
	the lowest eigenvalue $\tilde{E}_0$  of $\hat H$ 
	\begin{align}
		\Braket{\psi(\tilde{\bm\varphi}_0)}{\hat H}{\psi(\tilde{\bm\varphi}_0)} \approx \tilde{E}_0,
		\label{H}
	\end{align}	
	where $\tilde{\bm\varphi}_0 = {\rm argmin}_{\bm\varphi} \Braket{\psi(\bm\varphi)}{H}{\psi(\bm\varphi)}$ and $|\psi(\tilde{\bm\varphi}_0) \rangle$ approximates the corresponding eigenvector. 
	We use a number encoding to map the DVR basis states $\ket{i}$ onto the qubit states, which allows us to compute the lowest eigenvalue of the Hamiltonian matrix of size
	$2^n\times 2^n$ using $n$ qubits. 
	
	This method can be extended to calculate excited states \cite{vqd}. Once the quantum state parameters $\{\tilde{\bm\varphi}_i\}$ are found for $i < v$, the cost function can be modified to penalize the overlaps of $| \psi_v(\bm\varphi) \rangle$ with the previously estimated eigenvectors $\{\ket{\psi(\tilde{\bm\varphi}_i)}\}$ as follows: 
	\begin{align}
		\tilde{\varphi}_v &= {\rm argmin}_{\bm\varphi} \Braket{\psi(\bm\varphi)}{\hat H}{\psi(\bm\varphi)} \nonumber\\
		&+ \sum_{i=0}^{v-1} \beta_i \Braket{\psi(\tilde{\bm\varphi}_i)}{\hat H}{\psi(\tilde{\bm\varphi}_i)}
		\label{Hexcited}
	\end{align}
	where $\beta_i$ should be at least as large as the energy difference between the ro-vibrational states $E_{i+1} - E_i$.
	
	An important part of any quantum variational method is the choice of the quantum circuit (ansatz). 
	Our starting point is a hardware-efficient ansatz that consists of repeated alternating blocks of single-qubit and entangling gates \cite{chem-3}:
	\begin{align}
		\ket{\psi(\bm\varphi)} = &\prod_{d=0}^{k-1}\left[\prod_{q=0}^{n-1} U^{q,d}(\varphi^q_{d})\times U^d_{\rm ent}\right] \nonumber\\
		&\times \prod_{q=0}^{N-1} U^{q, k}(\varphi^q_{k})\ket{0^n}, \label{eq:hea}
	\end{align}
	where $U^{q,d}(\varphi)$ represent single-qubit Pauli rotations $R_\alpha = \exp(-i\varphi\sigma_\alpha/2)$ for Pauli matrices $\sigma_\alpha\in \{ {\sigma_X,\sigma_Y,\sigma_Z} \}$ acting on qubit $q$, and $k$ is the number of repetitions of the rotation and entanglement blocks. In this work, we use $\sigma_\alpha = \sigma_Y$  and $\rm CNOT$ entangling gates. The form of $U^d_{\rm ent}$ is determined by the ansatz optimization algorithm described below. For reference, we also use the linearly-entangled ansatz with $U^d_{\rm ent} = \prod_{q=0}^{n-1}{\rm CNOT}(q, q+1)$.
	%	The two types of commonly used entangling blocks are the fully-entangled ansatz with $U_{\rm ent} = \prod_{q=0}^{n-1}\prod_{p=q+1}^{n}{\rm CNOT}(q, p)$ and 
	% the linearly-entangled ansatz with $U_{\rm ent} = \prod_{q=0}^{n-1}{\rm CNOT}(q, q+1)$.
	
	
	
	
	
	
	\begin{table}
		\centering
		\begin{tabular*}{\columnwidth}{@{\extracolsep{\fill}}cccccc}
			\toprule
			Electronic & \multirow{2}{*}{$v$} & \multirow{2}{*}{BM} & \multicolumn{3}{c}{$E_v^{\rm VQE}$} \\
			\cmidrule{4-6}
			state & & & $\mathcal{C}_1$ & $\mathcal{C}_{0.01}$ & Linear \\
			\midrule
			\multirow{6}{*}{$^1\Sigma_g^+$}  & 0   & -15358.94 & -15358.87 & -15358.99 & -15358.99 \\
			& 1   & -14846.67 & -14838.70 & -14846.75 & -14846.96 \\
			& 2   & -14333.21 & -14310.29 & -14332.80 & -14332.82 \\
			& 3   & -13826.93 & -13797.65 & -13826.87 & -13827.29 \\
			& 4   & -13334.02 & -13275.12 & -13318.77 & -13335.45 \\
			& 5   & -12861.37 & -12897.70 & -12871.32 & -12868.75 \\
			\midrule
			\multirow{6}{*}{$^3\Sigma_u^+$}  & 0   & -9862.07 & -9861.37 & -9862.14 & -9862.14 \\
			& 1   & -9559.46 & -9538.97 & -9556.67 & -9559.41 \\
			& 2   & -9300.92 & -9240.74 & -9266.82 & -9300.88 \\
			& 3   & -9080.60 & 9068.09 & -9079.26 & -9085.40 \\
			& 4   & -8897.58 & -8866.09 & -8870.57 & -8896.82 \\
			& 5   & -8747.53 & -8729.41 & -8750.18 & ... \\
			\midrule
			\multirow{6}{*}{$^5\Sigma_g^+$}  & 0   & -7566.53 & -7565.88 & -7566.50 & -7566.50 \\
			& 1   & -7416.28 & -7397.28 & -7416.28 & -7416.29 \\
			& 2   & -7264.92 & -7181.98 & -7264.60 & -7264.69 \\
			& 3   & -7114.40 & -7075.88 & -7117.83 & -7118.43 \\
			& 4   & -6965.91 & -7001.98 & -6953.74 & -6958.08 \\
			& 5   & -6820.04 & -6873.37 & -6837.02 & -6840.02 \\
			\midrule
			\multirow{6}{*}{$^7\Sigma_u^+$}  & 0   & -6519.01 & \multicolumn{2}{c}{-6519.04} & -6519.04 \\
			& 1   & -6350.36 & \multicolumn{2}{c}{-6350.11} & -6350.11 \\
			& 2   & -6183.38 & \multicolumn{2}{c}{-6185.17} & -6185.17 \\
			& 3   & -6018.09 & \multicolumn{2}{c}{-6018.12} & -6018.28 \\
			& 4   & -5854.50 & \multicolumn{2}{c}{-5848.69} & -5849.10 \\
			& 5   & -5692.63 & \multicolumn{2}{c}{-5728.00} & -5731.33 \\
			\midrule
			\multirow{6}{*}{$^9\Sigma_g^+$}  & 0   & -5348.79 & \multicolumn{2}{c}{-5348.82} & -5348.82 \\
			& 1   & -5175.81 & \multicolumn{2}{c}{-5175.51} & -5175.51 \\
			& 2   & -5005.17 & \multicolumn{2}{c}{-5008.55} & -5008.56 \\
			& 3   & -4836.85 & \multicolumn{2}{c}{-4829.80} & -4829.92 \\
			& 4   & -4670.91 & \multicolumn{2}{c}{-4683.42} & -4683.68 \\
			& 5   & -4507.31 & \multicolumn{2}{c}{-4526.72} & -4612.83 \\
			\midrule
			\multirow{6}{*}{$^{11}\Sigma_u^+$} & 0   & -3677.68 & -3677.00 & -3677.68 & -3677.68 \\
			& 1   & -3507.89 & -3489.51 & -3507.82 & -3507.82 \\
			& 2   & -3341.77 & -3253.24 & -3341.26 & -3341.40 \\
			& 3   & -3180.16 & -3133.59 & -3185.51 & -3186.93 \\
			& 4   & -3023.07 & -3061.61 & -3001.04 & -3012.82 \\
			& 5   & -2870.22 & -2904.46 & -2866.05 & -2874.54 \\
			\midrule
			\multirow{6}{*}{$^{13}\Sigma_g^+$} & 0   & -548.68 & -548.65 & -548.67 & -548.68 \\
			& 1   & -497.16 & -496.47 & -496.84 & -497.15 \\
			& 2   & -449.18 & -443.48 & -448.36 & -449.26 \\
			& 3   & -404.71 & -382.88 & -390.96 & -404.67 \\
			& 4   & -363.58 & -369.09 & -360.62 & -362.99 \\
			& 5   & -325.67 & -315.46 & -310.29 & -325.72 \\
			\bottomrule
		\end{tabular*}
		\caption{Vibrational energy (in cm$^{-1}$) of \ch{Cr2} ($v =0-5$) in different electronic states. The benchmark (BM) results are obtained with a converged DVR basis. VQE computations use quantum circuits displayed in Fig. \ref{fig:cr2_pot_ansatz}. }
		\label{tab:diatomic}
	\end{table}
	
	
	
	\begin{figure*}
		\begin{tabular}{cc}
			\includegraphics[width=0.8\columnwidth]{cr2_pots} & 	\includegraphics[width=1.2\columnwidth]{cr2_ansatz_c1} \\
		\end{tabular}
		\caption{Left: potential energy for Cr$_2$ from Ref. \cite{cr2-pot}. Right:  quantum circuits for VQE yielding the ground state energy with error $\leq 1$ cm${}^{-1}$. The squares represent the $R_Y$ gates and the circles show the entangling CNOT gates.}
		\label{fig:cr2_pot_ansatz}
	\end{figure*}
	
	
	
	
	In order to evaluate the expectation values  in Eqs. (\ref{H}) and (\ref{Hexcited}), we use the following decomposition of the Hamiltonian: 
	\begin{align}
		\hat H = \sum_{i = 0}^{4^n} A_i K^i_1\otimes K^i_2 \ldots \otimes K^i_n 
		\label{Pauli}
	\end{align}
	where $K^i_j \in \{\sigma_X, \sigma_Y, \sigma_Z, I\}$ acting on qubit $j$,
	\begin{align}
		A_i = \frac{1}{2^n}{\rm Tr}[(K^i_1\otimes K^i_2 \ldots \otimes K^i_n)\cdot {\bm H}]
	\end{align}
	and $\bf H$ is the Hamiltonian matrix evaluated in some basis. 
	
	
	We use DVR to discretize the continuous nuclear coordinates of atoms in molecules and construct $\bf H$. A DVR is a finite basis representation (FBR) of a Hamiltonian in which the coordinate operators (and consequently the potential energy operator) are diagonal. 
	A DVR can be derived from any given finite basis representation by diagonalizing the coordinate representations. 
	For diatomic molecules, we use the DVR derivation by Colbert and Miller \cite{colbert}, yielding the kinetic energy 
	%	which defines an infinite number of uniformly spaced DVR points on a radial coordinate $r\in(0, \infty)$. The kinetic energy 
	representation on $r_i = i\Delta r$ for $i \in  \mathbb{Z}$ as
	\begin{equation}\label{eq:dvr_0inf}
		T_{ij} = \frac{\hbar^2}{2m\Delta r^2} (-1)^{i-j}\begin{cases}
			\pi^2/3 - 1/(2i^2), & i=j \\
			\frac{2}{(i-j)^2} - \frac{2}{(i+j)^2}, & i\neq j
		\end{cases},
	\end{equation}
	where $\Delta r$ is the distance between adjacent DVR points, $m$ is the reduced mass, and $\hbar$ is the reduced Planck constant. 
	
	
	
	For triatomic molecules, we use the DVR by Choi and Light \cite{choi}. This approach constructs the ro-vibrational Hamiltonian in a finite basis of orthogonalized Sturmian functions, associated Legendre functions and parity-adapted Wigner rotation functions with respect to the Jacobi coordinates $(R, r, \theta)$ and the angular momentum projection in the body-fixed frame ($K$). Fixing the total angular momentum of the system and the bond length of the diatomic molecule to its equilibrium value ($r = r_e$), leaves two variables $(R, \theta)$ defining the position of the atom with respect to the molecule. This leads to a three-dimensional ${}^{\rm FBR}H_{ijK}^{i'j'K'}$ with $i$ and $j$ representing the  basis functions associated with $R$, $\theta$. This Hamiltonian is then diagonalized in the $R$ and $\theta$ coordinates leading to a set of DVR points
	\begin{align}
		{^R\matr{\Delta}} = &{^R\matr{T}}\cdot\matr{R}\cdot({^R\matr{T}})^\top, \\
		{^{\theta K}\matr{\Delta}} = &{^{\theta K}\matr{T}}\cdot\matr{X}\cdot({^{\theta K}\matr{T}})^\top, \\
		\matr{T} =& {^R\matr{T}}\otimes{^{\theta K}\matr{T}}\otimes\matr{I}_K
	\end{align}
	where ${}^R\Delta_{ii'}$ and ${}^{K\theta}\Delta_{jj'}$ are the $R$ and $x = \cos\theta$ representations in the FBR and $\matr{R}$ and $\matr{X}$ are diagonal matrices defining the DVR quadrature points $\{R_\alpha\}$ and $\{\theta_{K\beta}\}$. The transformation matrices ${^R\matr{T}}$ and ${^{\theta K}\matr{T}}$ are then applied to the Hamiltonian evaluated in the FBR
	\begin{align}
		{^{\rm DVR}\matr{H}} = &\matr{T}^\top\cdot{^{\rm FBR}\matr{H}}\cdot\matr{T}
	\end{align}
	resulting in a DVR Hamiltonian ${^{\rm DVR}\matr{H}}$ with a diagonal representation of the potential energy.
	%	\begin{equation}
		%		V_{\alpha\beta}^{\alpha'\beta'} = V(R_\alpha,\theta_{K\beta})\delta_{\alpha\alpha'}\delta_{\beta\beta'}.
		%	\end{equation}
	
	
	
	
	
	
	\begin{figure*}
		%		\begin{tabular}{c}
			\includegraphics[width=\textwidth]{opt_ansatz_triatomic}
			%		\end{tabular}
		\caption{Left: VQE computation error for ground ro-vibrational energy of Ar--HCl (open symbols) and Mg--NH (full symbols) computed with optimized quantum circuits displayed in the right panel. Circles -- the results obtained with $\mathcal{C}_1$ circuits; squares -- with  $\mathcal{C}_{0.01}$ circuits; triangles -- with linearly-entangled ansatz (\ref{eq:hea}) using (left to right) $k=1,2,3$ and $4$ repetitions. The squares represent the $R_Y$ gates and the circles show the entangling CNOT gates. The $\mathcal{C}_1$ ansatz for Mg -- NH excludes the entangling gate shown in green.}
		\label{fig:greedyent_const}
	\end{figure*}
	
	
	\begin{figure}
		%		\begin{tabular}{c}
			\includegraphics[width=\columnwidth]{arhcl_vqe_opt_excited64} 
			%		\end{tabular}
		\caption{Error of VQE energies for Ar -- HCl computed with optimized circuits ${\cal C}_1$ (upper boundary of shaded area) and ${\cal C}_{0.01}$ (lower boundary of shaded area) built with 5 qubits (corresponding to 32 DVR functions) and 6 qubits (corresponding to 64 DVR functions).}
		\label{fig:vqe_excited}
	\end{figure}
	
	\begin{figure}
		%		\begin{tabular}{c}
			\includegraphics[width=\columnwidth]{arhcl_partition_5} 
			%		\end{tabular}
		\caption{Minimum error of ground state energy for ArHCl predicted by VQE with partially entangled 5-qubit circuits. The bar labels indicate the partitioning of qubits.}
		\label{fig:vqe_partition}
	\end{figure}
	
	
	
	\begin{table*}
		\begin{tabular*}{\textwidth}{@{\extracolsep{\fill}}cccccccc}
			\toprule
			\multirow{2}{*}{Molecule} & \multirow{2}{*}{$v$} & \multirow{2}{*}{$E_v$ (experiment)} & \multirow{2}{*}{$E_v$ (computed in \cite{arhcl-pot})} & \multirow{2}{*}{$E_v$ (classical, present)} & \multicolumn{3}{c}{$E_{v}$ (VQE)} \\
			\cmidrule{6-8}
			& & & & & $\mathcal{C}_1$ & $\mathcal{C}_{0.01}$ & Linearly entangled \\
			\midrule
			\multirow{3}{*}{ArHCl} & \multirow{1}{*}{0} & -114.7~\cite{arhcl-obs-0} & -115.151 & -115.265 & -114.645 & -115.169 & -115.171 \\
			& \multirow{1}{*}{1} & -91.04~\cite{arhcl-obs-1, arhcl-obs-12} & -91.485 & -91.642 & -80.824 & -90.485 & -90.929 \\
			& \multirow{1}{*}{2} & -82.26~\cite{arhcl-obs-2} & -82.717 & -82.825 & -75.900 & -82.986 & -82.650 \\
			\midrule
			\multirow{3}{*}{MgNH} & \multirow{1}{*}{0} & - & - & -88.227 & -87.650 & -88.191 & -88.190 \\
			& \multirow{1}{*}{1} & - & - & -63.603 & -56.050 & -62.730 & -62.664 \\
			& \multirow{1}{*}{2} & - & - & -55.461 & -55.145 & -54.850 & -54.866 \\
			\bottomrule
		\end{tabular*}
		\caption{Vibrational energy levels  of Ar--HCl and Mg--NH. VQE computations use a 32-point DVR Hamiltonian with 5-qubit  quantum circuits displayed in Figure \ref{fig:vqe_partition}. All energies are in cm$^{-1}$. The zero of energy corresponds to the infinite separation between the atom and the diatomic molecule.}
		\label{tab:triatomic}
	\end{table*}
	
	
	{\it Quantum circuit optimization.} 
	Previous work on electronic structure calculations proposed various types of ansatzes, that can generally be divided into fixed-structure ansatzes \cite{vqe-1,chem-3,uccsd,fixed-ansatz,uccgsd,symmetry,hva} and adaptive-structure ansatzes \cite{adaptvqe,adaptvqe-nuc,qubit-adaptvqe,iqcc,cluster-vqe,rotoselect,vans,evo-vqe,mog-vqe,qas}. Fixed-structure ansatzes have a predetermined structure, while adaptive-structure ansatzes can adjust the circuit architecture to the problem. 
	We develop adaptive-structure ansatzes by optimizing 2-qubit entangling gates in Eq. (\ref{eq:hea}) through greedy compositional search, inspired by Refs. \cite{comp-search-1, comp-search-2}. 
	
	Our algorithm starts with a non-entangled quantum state given by Eq. (\ref{eq:hea}) with a predetermined number of blocks $k$  and $U^d_{\rm ent}$ set to identity. The method considers entangling gates of the form ${\rm CNOT}(q, p)~ \forall q < p$ as candidate gates for each entangling block. In each optimization step, the candidate gate that results in the lowest VQE energy is added to the ansatz without replacement. The search continues over the remaining candidate gates in the subsequent steps, adding one gate at a time until the desired convergence is achieved. 
	For circuits with $d>1$, each entangling block is considered independent, resulting in a larger number of candidate gates. 	
	We aim to converge the VQE calculation of the ground state either to 1 cm$^{-1}$ or 0.01 cm$^{-1}$, which yields quantum circuits of different complexity, denoted by $\mathcal{C}_1$ and $\mathcal{C}_{0.01}$, respectively. 
	This convergence error is with respect to the lowest eigenvalue of the corresponding DVR matrix used in Eq.~(\ref{eq:hea}). 
	%Note that the results reported throughout this work are benchmarked by fully converged DVR calculations and literature results. 	
	
	
	
	{\it Results.} We calculate the ro-vibrational energy levels of diatomic (\ch{Cr2}) and triatomic (Ar--HCl and Mg--NH) molecular systems using VQE. We use two classical constrained optimization methods to optimize the ansatz parameters: the bounded limited memory Broyden, Fletcher, Goldfarb, and Shanno method  \cite{lbfgsb-1, lbfgsb-2} and Sequential Least SQuares Programming  \cite{slsqp}. We compare the energies obtained by VQE with the Hamiltonian eigenvalues calculated using direct diagonalization with the converged DVR basis and previous literature results, where available.  
	
	
	
	
	
	Table \ref{tab:diatomic} illustrates the performance of VQE for six vibrational states $v=0$ -- $5$ of seven electronic states (${}^1\Sigma_g^+, {}^3\Sigma_u^+, {}^5\Sigma_g^+, {}^7\Sigma_u^+, {}^9\Sigma_g^+, {}^{11}\Sigma_u^+, {}^{13}\Sigma_g^+$) of the molecule Cr$_2$ with zero rotational angular momentum. The computations are performed with the interaction potentials from Ref. \cite{cr2-pot}, illustrated in Fig. \ref{fig:cr2_pot_ansatz}.  The VQE calculations are based on the Hamiltonian matrix with 16 DVR points placed to span the interatomic distance from before to after
	the minimum of the interaction potential curve. 
	The corresponding decomposition (\ref{Pauli}) includes $\approx 130$ Pauli terms. 
	Table \ref{tab:diatomic} displays VQE results obtained with three types of quantum circuits:  the linearly-entangled ansatz with 3 repetitions (linear), and optimized circuits ${\cal C}_1$ and ${\cal C}_{0.01}$. 
	The quantum circuits used for the 
	${\cal C}_1$ calculations are displayed in Fig. \ref{fig:cr2_pot_ansatz}. The ${\cal C}_{0.01}$ circuits are given in the supplemental material \cite{sm}. 
	%Table  \ref{tab:diatomic} and Fig. \ref{fig:cr2_pot_ansatz} show that VQE can be used to compute the vibrational levels of diatomic molecules with high precision using shallow quantum circuits. 
	
	To represent the Hamiltonian of the tri-atomic systems, we use either 32 or 64 DVR basis states and accurate atom - molecule potential energy surfaces by Hutson for Ar--HCl \cite{arhcl-pot} and by Sold\'an et al. for Mg--NH \cite{mgnh-pot}. We keep both HCl and NH in the ground ro-vibrational state and compute the ro-vibrational states supported by the atom - molecule interaction potential.  We obtain the DVR points for the triatomic systems by diagonalizing the coordinate representations. The basis functions used in the FBR control the number and placement of the DVR points. The parameters to generate the 32 and 64 DVR points are selected to cover the low-energy regions of the potential energy surface. 
	%(ArHCl: $N_{\theta} = 4, N_R=35, l=1, r_{\rm min} = 3.4$ \r{A}, $r_{\rm max}=5$ \r{A} and MgNH: $N_{\theta} = 4, N_R=20, l=1, r_{\rm min} = 3$ \r{A}, $r_{\rm max}=56$). 
	The 32-point DVR Hamiltonians are represented by 165 (Ar--HCl) and 170 (Mg--NH) Pauli terms acting on 5 qubits, while the 64-point DVR Hamiltonians are represented by 610 (Ar--HCl) and 621 (Mg--NH) Pauli terms acting on 6 qubits. 
	
	As above, we construct three types of quantum circuits: ${\cal C}_1$,  ${\cal C}_{0.01}$ as well as the linearly-entangled ansatz (\ref{eq:hea}) with $k$ repetitions.
	Figures \ref{fig:greedyent_const}, \ref{fig:vqe_excited}  and Table \ref{fig:vqe_excited}  illustrate the results, both for the ground and nine excited states. 
	We observe that a small number of entangling gates is sufficient to ensure accurate calculations for both the ground  and excited state energy. 
	Figure  \ref{fig:vqe_excited}  demonstrates the improvement of the computation accuracy with the increasing number of qubits (illustrated by the curves of different color) and the number of quantum gates (illustrated by the shaded area between the curves with the corresponding number of qubits). 
	
	We also observe that some of the optimized circuits in Figs. \ref{fig:cr2_pot_ansatz} and \ref{fig:greedyent_const} are only partially entangled. To examine the effect of qubit entanglement, we repeated the circuit optimization for Ar -- HCl and Mg -- NH using a sequence of quantum circuits with 
	unentangled groups of entangled qubits.  The lowest errors achievable with partially entangled circuits for the ground state of Ar -- HCl are displayed in Fig. \ref{fig:vqe_partition}. 
	%We observe similar results for for Mg -- NH. 
	The results in Fig. \ref{fig:vqe_partition} show that the ro-vibrational energy calculations can be computed with errors $< 1$ cm$^{-1}$ using partially entangled circuits. 
	
	
	
	In conclusion, we have demonstrated a general approach to computing the ro-vibrational energy levels of molecules by combining DVR with VQE and a greedy search of gate sequences. This method yields compact representations of vibrational states by quantum circuits of a gate-based quantum computer. 
	The quantum states generated by these circuits can be optimized to compute the ro-vibrational energies of molecules. 
	We have shown that both the ground and excited vibrational energies can be variationally computed with the relative accuracy of $< 1 \%$ using very simple, in some cases, partially entangled circuits, while the accuracy of $1$ cm$^{-1}$ can be achieved with $< 20$ ($< 5$ entangling) gates acting on 4 qubits for diatomic molecules and $< 30$ ($< 9$ entangling) gates acting on 5 qubits for triatomic van der Waals complexes. Representing molecular vibrational states by quantum states of a quantum computer can also be used for quantum machine learning with quantum inputs \cite{qml-with-quantum-data}. 	
	Because any interaction potential is diagonal in a DVR basis, the present approach does not require analytical fits of potential energy surfaces and can be readily integrated with VQE calculations of electronic energy or extended beyond molecular dynamics. For example, the present approach can be directly applied to designing efficient quantum circuits for variational computations of the eigenspectra of semiconductor quantum dots \cite{demler}.
	
	\bibliography{manuscript}
	
\end{document}