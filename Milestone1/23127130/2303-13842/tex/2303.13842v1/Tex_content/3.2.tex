Outpainting approaches typically employ several layers of deconvolution as the task head, and FishDreamer is no exception.
Given that fisheye semantic completion seeks to accomplish both image completion and semantic completion simultaneously, we opt not to incorporate a more complex outpainting decoder in the design.
Specifically, FishDreamer uses three layers of deconvolution, along with the PCA mechanism described in Sec. \ref{sec:polar_aware_cross_attention}, to upsample and produce the feature map obtained from the feature extractor. This process generates the extrapolated output, which can be calculated as follows:
\begin{equation}
\centering
\begin{split}
    \hat{z_s} = ConvT_{i}(z_s),\\
    z = ConvT_{i+1}(\hat{z_s} +PCA(\hat{z_s},z_c)),\\
    z := ConvT_{i+2}(z),
\end{split}
\end{equation}
where ConvT is the 2D transposed convolution operator with kernel size$=3\times3$ while the stride and padding are set as 2 and 1. $i$ indicates the stage of the ConvT and is chosen as 0 in this work. PCA is the proposed Polar-aware Cross Attention module which will be detailed in Sec.~\ref{sec:polar_aware_cross_attention}, $z_{s}$ and $z_c$ denote the feature extracted from the Swin Transformer-based backbone and the features from semantic completion head, $\hat{z}$ denotes the feature map of the semantic segmentation branch after the first ConvT layer, $z$ denotes the final merged feature map, respectively.
With  additional priors from the semantic completion head we can acquire a semantically coherent outpainting result.

