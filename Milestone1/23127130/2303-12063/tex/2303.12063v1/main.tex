\documentclass[a4paper,11pt]{article}
\usepackage{jheppub} % for details on the use of the package, please see the JINST-author-manual
\usepackage{lineno}
\usepackage{float}
%\linenumbers

\newcommand{\muI}{\mu_I}
\newcommand{\muB}{\mu_B}
\newcommand{\LA}{\left \langle}
\newcommand{\RA}{\right \rangle}
\newcommand{\Ns}{N_{\sigma}}
\newcommand{\Nt}{N_{\tau}}
\newcommand{\Tpc}{T_{pc}}
\newcommand{\N}{\mathcal{N}/T^3}
\newcommand{\Ob}{\mathcal{O}}
\newcommand{\M}{\mathcal{M}}
\newcommand{\hs}{\hspace{.6cm}}
\newcommand{\heq}{\hspace{.5mm}}
\renewcommand{\L}{\mathcal{L}}
\newcommand{\diff}{\Delta_I(\text{NR},0)}
%\arxivnumber{1234.56789} % if you have one
\usepackage{calc}
\newlength{\depthofsumsign}
\setlength{\depthofsumsign}{\depthof{$\sum$}}
\newlength{\totalheightofsumsign}
\newlength{\heightanddepthofargument}
\newcommand{\nsum}[1][1.2]{% only for \displaystyle
    \mathop{%
        \raisebox
            {-#1\depthofsumsign+1\depthofsumsign}
            {\scalebox
                {#1}
                {$\displaystyle\sum$}%
            }
    }
}

\title{\boldmath A new way to identify the breakdown of the unbiased exponential resummation in Lattice QCD at a finite isospin chemical potential}

% Collaborations

%% [A] If main author
%% \collaboration{\includegraphics[height=17mm]{collabroation-logo}\\[6pt]
%%  XXX collaboration}

%% or
%% [B] If "on behalf of"
%% \collaboration[c]{on behalf of XXX collaboration}


% Authors
% The "\note" macro will give a warning: "Ignoring empty anchor...", you can safely ignore it.

%% [A] simple case: 2 authors, same institution
\author[a]{Sabarnya Mitra}
%\author[a]{and Prasad Hegde}
\affiliation[a]{Centre for High Energy Physics, Indian Institute of Science \\
 CV Raman Avenue, Bengaluru-560012, India}

%% or, e.g.
%% [B] more complex case: 4 authors, 3 institutions, 2 footnotes
%% \author[a,b]{F. Irst,\note{Now at another university}}
%% \author[c]{S. Econd,}
%% \author[a,2]{T. Hird\note{Also at Some University.}}
%% \author[c,2]{and Fourth}
%% \affiliation[a]{Institution_1,\\Address, Country}
%% \affiliation[b]{Institution_2,\\Address, Country}
%% \affiliation[c]{Institution_3,\\Address, Country}

% \author{Sabarnya Mitra}
% \affiliation{Indian Institute of Science,\\
% some-street, Country}
% \affiliation{Another University,\\
% different-address, Country}

% E-mail addresses: only for the corresponding author
\emailAdd{sabarnyam@iisc.ac.in}
%\emailAdd{prasadhegde@iisc.ac.in}

\abstract
{
The finite temperature calculations of unbiased exponential resummation at a real, finite isospin chemical potential $\muI$ in lattice QCD do not experience the fermion sign problem. Although this suggests that the computations, in principle can be continued for all real finite values of $\muI$, however recent studies demonstrating the formation of a pion condensate in the isospin phase diagram of QCD at a finite temperature $T$ possibly imply the occurrence of a breakdown of this formalism at some finite value of $\muI$. In this paper, we present results illustrating the singularities of the partition function obtained using Newton-Raphson method in complex $\muI$ plane. We observe a naive non-monotonic behaviour of isospin number density beyond the radius of convergence which is determined by the Newton-Raphson singularity situated closest to the origin. We also introduce a new way of formulating a non-trivial phasefactor in complex $\muI$ which promises to identify reliably the onset of this breakdown along the real $\muI$ axis. We also illuminate increasing severity of the overlap problem with increasing value of physical $\muI$. We find that the relative errorbars of kurtosis as a measure of this overlap problem become significantly large, on going beyond the value of real $\muI$ from which, these phasefactor values averaged over the working gauge ensemble start becoming zero.
}



\begin{document}
\maketitle
\flushbottom

\section{Introduction}
\label{sec:intro}

\hs The strong force as one of the four fundamental forces of Nature is very well described by the quantum field theory of Quantum Chromodynamics (QCD)\,\cite{Gross:2022QCD}. An immensely important and intriguing spectacle in the paradigm of these strong interactions is the QCD phase diagram which features various interesting phases of strongly interacting matter. One of the important aspects of QCD is to explore and map this phase diagram\,\cite{Halasz:1998phasediagram, Rajagopal:1999phasediagram,Stephanov:2006phasediagram,Fukushima:2011phasediagram} as a function of temperature $T$ and baryochemical potential $\muB$ which is pivotal not only for knowing the strong dynamics at various energy scales, but also for illuminating the physics of early universe\,\cite{McGuigan:2008earlyuniverse,Castorina:2015earlyuniverse}. Although this phase diagram looks apparently complete and quite robust by itself, a majority of it being constructed out from mere symmetry arguments and model analyses, continue to remain conjectured and await further conclusive evidences. Because of its ability to successfully predict results to appreciable degree of precision for most of the time, one often resorts to the non-perturbative formulation of thermodynamics crafted by lattice QCD\,\cite{Davies:2005latticeQCD,Boyle:2022latticeQCD}. Besides offering possible signatures of unexplored phases, framing QCD on lattice also enables one to obtain important reliable insights about the phase diagram.  Like at present, lattice simulations can very well explain the manifestations along the vertical $T$ axis ($T \neq 0, \muB=0$) of the diagram and establish that the phase transition between the hadronic and quark-gluon plasma phases at zero $\muB$ is an analytical crossover\,\cite{Steinbrecher:2018QCDcrossover,Borsanyi:2020QCDcrossover,Bazavov:2018QCDcrossover,Li:2020QCDcrossover,Guenther:2021QCDcrossover}.


% In quest of these crucial evidences, one must embrace a  reliable non-perturbative thermodynamic approach, since the system can be explained perturbatively only at high values of $T$ and $\muB$. Out of the many non-perturbative methods being proposed, lattice QCD appears to be the most promising of all at present, mostly because of its successful prediction of results to appreciable precision. Like at present, lattice simulations of QCD can very well explain the dynamics along the vertical $T$ axis ($T \neq 0, \muB=0$) and establish that the phase transition between the hadronic and quark-gluon plasma phases at zero $\muB$ is an analytical crossover.  

However for real $\muB \neq 0$, lattice QCD faces a stiff computational hurdle in form of the infamous sign problem. At finite $\muB$, the path integral\,\cite{Palumbo:2002pathintegral} expressing the QCD partition function $Z$ becomes complex with its measure containing a complex fermion determinant\,\cite{Nakamura:2005complexfermiondeterminant} which gives rise to the problem of complex measure. This complex measure hinders implementation of Monte-Carlo important sampling to estimate this integral form of $Z$. While reweighting\,\cite{Ejiri:2004reweighting,Li:2006reweighting} this complex measure with a real fermion determinant at zero $\muB$ makes the measure real, the observable part of the integral gets complexified which, after Monte-Carlo estimation provides a phaseangle $\Theta$ and a subsequent gauge ensemble averaged phasefactor $\LA \cos \Theta \RA$. This gives rise to a sign problem\,\cite{Gupta:2004signproblem,Danzer:2009signproblem,Goy:2016signproblem,Nagata:2021signproblem}, the severity of which is governed by the magnitude of this phasefactor that reduces to zero with increasing values of $\muB$. This signifies that the sign problem becomes highly severe in the sense, that the integrand exhibits tremendous oscillations across positive and negative real values, each of which is also large in magnitude. This sign problem eventually leads to the breakdown of lattice QCD computations at a finite value of $\muB$ reflected by the non-monotonic behaviour of the calculated observables. This highly restricts our investigation and consequent knowledge of QCD at finite density. 

% This is the infamous complex measure problem which hinders implementation of Monte-Carlo important sampling for estimation of the path integral. This assumes the form of a sign problem eventually after reweighting the integrand with real fermion determinant obtained at $\muB=0$, which makes the integral measure real and re-enables application of Monte-Carlo sampling. The resulting sign problem leads to the breakdown of lattice QCD computations after a finite value of $\muB$ where the integrand becomes highly oscillating and hence cannot be effectively captured using Monte-Carlo importance sampling. This makes the finite density regime of QCD mostly unexplored, which encompass color superconducting phases and neutron stars. 

Several new methods\,\cite{Bilic:1987Langevin,Aarts:2016Langevin,Kogut:2019Langevin,Sinclair:2019Langevin,Cristoforetti:2012Thimbles,Cristoforetti:2013Thimbles,Scorzato:2015Thimbles} have been introduced which can successfully avoid this sign problem, most of which unfortunately have very limited applications in explicit QCD. For QCD, the Taylor expansion around $\muB=0$\,\cite{Ejiri:2003Taylor,Miao:2008Taylor,Falcone:2010Taylor} and analytic continuation of simulations from imaginary to real $\muB$\,\cite{DElia:2002Analytic,Lombardo:2006Analytic,Sakai:2009Analytic} continue to remain the prominent methods for circumventing the sign problem and providing state-of-the-art results for QCD equation of state\,\cite{Fodor:2002EoS,Aoki:2005EoS,Miller:2006EoS,Karsch:2008EoS,Kanaya:2010EoS,Huovinen:2011EoS,Philipsen:2012EoS,Hegde:2014EoS,Bazavov:2017EoS} at finite $\muB$. Resummation approaches like  Pad{\'e}\,\cite{Cvetic:2011Pade,Pasztor:2020Pade,Bollweg:2022Pade} and exponential resummation\,\cite{Mondal:2021exponentialresummation} have been proposed, to improve the slowly convergent Taylor series results. While the former approximates the Taylor coefficients by rational functions where one is interested to find the roots and poles of these functions, the latter provides a direct estimate of $Z$ in the form of an exponential, the argument of which comprises finite contributions of lower order Taylor series. Besides capturing these contributions to all orders in $\muB$, exponential resummation is observed to predict the singularities of $Z$ in complex $\muB$ plane and provide an estimate of radius of convergence to an appreciable extent. Despite these, it is also observed to encounter biased estimates\,\cite{Mitra:2022Bonn,Mitra:2022cumulantexpansion} which can be successfully eliminated to some finite order in $\muB$ by the recently introduced formalism of unbiased exponential resummation\,\cite{Mitra:2023unbiasedexponentialresummation}. This unbiased approach is paramount for recognising the genuine higher order Taylor contributions captured through this approach of resummation.

Unlike $\muB$, there is no sign problem in case of isospin chemical potential $\muI$ in $2+1$ flavor QCD. Although this means that in principle, one can perform unbiased exponential resummation to all real values of $\muI$ extending to $\infty$, studies suggest that there is a genuine phase diagram\,\cite{deForcrand:2007isospin,Moller:2009isospin,Brandt:2016isospin,Brandt:2017isospin,Brandt:2019isospin} in the $T-\muI$ plane which illustrates the formation of a pion condensate starting from some finite value of $\muI$ for a low $T$. This signifies that at a low $T$ surely, this formalism is supposed to have a finite radius of convergence in $\muI$ and is expected to experience a breakdown beyond that. This is because, the calculation relies on extrapolations from $\muI=0$ which remains in a non-condensed phase of the isospin phase diagram. Since the usual phasefactor values remain at unity for all $\muI$ due to no sign problem, this phasefactor unlike $\muB$, cannot locate the initiation of this breakdown. This paper aims to come up with a reliable indicator for this purpose.   

% However, both these methods in QCD remain effective only upto some small values of $\muB/T$, for which they require substantial modifications to explore and probe deeper into regimes with higher density, where one may expect to observe signatures of other interesting and exotic phases which are being proposed.

% Several modification schemes have been put forward, out of which a very important approach is resumming lower order Taylor series in QCD at finite density. Apart from Pad{\'e} resummation using the Pad{\'e} approximants\,\cite{Cvetic:2011Pade,Pasztor:2020Pade,Bollweg:2022Pade}, the method of exponential resummation\,\cite{Mondal:2021exponentialresummation} exponentiating the first few lower order Taylor coefficients to all orders in $\muB$ is another significant approach. This method approximates the partition function $Z$ with an exponential series, the argument of which comprises a finite series in $n$-point correlation functions $D_n$ coupled with appropriate powers of $\muB$. This leads to capturing contributions of $D_n$ to some finite $n$ to all orders in $\muB$ where the $D_n$ are the same correlation functions used for constructing the successive Taylor coefficients in the usual Taylor expansion of thermodynamic observables. This usual formula of exponential resummation unfortunately gets affected by the appearance of stochastically biased estimates\,\cite{Mitra:2022cumulantexpansion} of $D_n$ and therefore warrants elimination for a genuine interpretation of the QCD phase diagram. Recently, a new formalism of unbiased exponential resummation\,\cite{Mitra:2023unbiasedexponentialresummation,Mitra:2022unbiasedexponentialresummation,Mitra:2022Bonn} has been introduced which can eliminate stochastic bias upto a given order in $\muB$ and can successfully capture higher order Taylor contributions for both real and imaginary values of $\muB/T$. 

% Despite this, the results can only be obtained upto the breakdown value of $\muB$ as the sign problem continues to prevail. This sign problem can be efficiently captured by the zeroes of phasefactor results averaged over all the gauge configurations present in the working gauge ensemble and hence, one can know from this, the extent of WRITE HERE. For isospin chemical potential $\muI/T$, there is no sign problem and the phasefactor results remaining at unity, never plummet to zero. This therefore leaves us with no reliable indicators that can identify the breakdown of this unbiased exponential resummation for $\muI$. At the same time, it is certainly known that this formalism will experience a breakdown at some value of $\muI$, since there is a genuine phase transition in the form of a pion condensate in the plane of $\muI-T$. 

 
The paper is organised as follows: In \autoref{sec:Unbiased exp}, we provide a quick overview on the nomenclature of the unbiased exponential resummation formalism. The details of scale setting and setup of lattice including random volume sources and gauge configurations used in constructing the Taylor coefficients for the unbiased formalism are highlighted in \autoref{sec:setup}. In \autoref{sec:Results}, we present a comprehensive picture of the Newton-Raphson singularities of QCD partition function $Z$ and subsequent radius of convergence in the complex $\muI/T$ plane where $Z$ is evaluated as a function of $\muI$ at a finite $T$ using unbiased exponential resummation. In \autoref{sec:Phasefactor formalism}, we introduce a new prescription of a non-trivial phasefactor that can possibly capture the Newton-Raphson singularity nearest to the origin by discerning the radius of convergence. We also show that how the overlap problem becomes increasingly severe as $\muI$ is increased along the real axis of the complex $\muI$ plane, with particular interest being the rapid and striking enlargement of the subsequent errorbars across the radius of convergence, as directed by this new phasefactor. We conclude with a summary of the main results and possible scope in \autoref{sec:Conclusions}. Throughout this paper, we have used relativistic units ($\hbar = c = 1$) and unit Boltzmann constant and have denoted $\muI/T$ as simply $\muI$.           


\section{Unbiased exponential Resummation: A brief overview}
\label{sec:Unbiased exp}


\hs We present a quick discussion on unbiased exponential resummation in this section. For a detailed picture of this formalism, we refer to Ref.\,\cite{Mitra:2023unbiasedexponentialresummation}. The partition function $Z$ for a given $T,\mu$\footnote{$\mu$ denotes any arbitrary chemical potential.} is given as 

\begin{equation}
    Z(T,\mu) = \int \mathcal{D}U \heq e^{-S_G\left[T,U\right]} \heq \det \M(T,\mu,U)
    \label{eq:partition function}
\end{equation}
%
where the $\det \M(T,\mu,U)$\footnote{We have suppressed volume dependence from here on.} is given by 

\begin{equation}
\det \M(T,\mu,U) = \prod_{f=u,d,s} \big[\det \M(T,\mu_f,U)\big]^{1/4} 
\label{eq:staggered fermion}
\end{equation}
%
This equation reflects the staggered signature of the constituent fermion action within the QCD action. In Eqns.\eqref{eq:partition function} and \eqref{eq:staggered fermion}, $U$ represent the gauge field configurations and functional $S_G\left[T,U\right]$ denotes the gluon action. The usual exponential resummation formulae of number density $\mathcal{N}$ and excess pressure defined as $\Delta P(T,\mu) = P(T,\mu)-P(T,0)$ for a thermodynamic system of volume $V$ at temperature $T$ are given as follows:

\begin{equation}
    \frac{\Delta P(T,\mu)}{T^4} = \frac{1}{VT^3} \, \ln \left[\text{Re}\LA \exp \left(\nsum_{n=1}^N \left(\frac{\mu}{T}\right)^n \frac{\overline{D_n}}{n!}\right) \RA \right], \hspace{4mm} \frac{\mathcal{N}}{T^3} = \frac{\partial}{\partial (\mu/T)} \left[\frac{\Delta P}{T^4}\right]
    \label{eq:expoential resum}
\end{equation}
%
In Eqn.\,\eqref{eq:expoential resum}, the excess pressure and number density are scaled with appropriate powers of $T$ to make them dimensionless. The $D_n$ are the usual temperature dependent $n$-point correlation functions given as 

\begin{equation}
    D_n(T,U) = \frac{\partial^n \ln \det \M(T,\mu,U)}{\partial (\mu/T)^n}\bigg|_{\mu=0}
    \label{eq:derivatives}
\end{equation}
%
$\LA \Ob \RA$ represent the average of arbitrary observable $\Ob$ over all gauge field configurations comprising the working gauge ensemble, which is given by 

\begin{equation*}
    \Big \langle \Ob(T,\mu) \Big \rangle = \frac{1}{Z} \int \mathcal{D}U \heq e^{-S_G\left[T,U\right]} \heq \Ob(T,\mu,U) \heq \det \M(T,\mu,U)
\end{equation*}
%
where $Z$ is given in Eqn.\,\eqref{eq:partition function}. The $\overline{D_n}$ in Eqn.\,\eqref{eq:expoential resum} is the average of $D_n$ calculated over all the available Gaussian random volume sources $N_R$ present within a given gauge configuration given as:

\begin{equation}
    \overline{D_n} = \frac{1}{N_R} \nsum_{r=1}^{N_R} D_n^{(r)}
    \label{eq:random volume sources}
\end{equation}
%
This is because the fermion matrix $\M$ cannot be calculated exactly using analytical means\,\cite{Ying:1998fermionmatrix}. We have seen how biased estimates appear in the formula of usual exponential resummation due to Eqn.\,\eqref{eq:random volume sources}. The unbiased exponential resummation retains the original form of exponential resummation with the exception that the argument of the exponential is modified accordingly so that it produces unbiased estimates of $D_n$ to a given order in $\mu$, where $\mu$ is any arbitrary chemical potential. Typically, in our calculations we have considered $\mu \in (B,S,I)$ basis where the symbols have the usual meanings. This formalism is performed in two bases namely,

\subsection{Chemical potential basis}

\hs In chemical potential basis, the unbiased exponential resummation resembles:
        \begin{align}   %  6
     &\frac{\Delta P_{N}^{R(\text{unb})}(T,\mu)}{T^4} = \frac{1}{VT^3} \hspace{1mm}\ln \hspace{1mm} Z_{N}^{R(\text{unb})}(T,\mu) , \notag \\ 
     &Z_{N}^{R(\text{unb})}(T,\mu) = \text{Re} \left \langle \bigg[\exp \Big(A_N\left(T,\mu\right)\Big)\bigg] \right \rangle , \notag \\
    &A_N\left(T,\mu\right) = \nsum_{n=1}^{N} \left(\frac{\mu}{T}\right)^n\,\frac{\mathcal{C}_{n}}{n!} 
     \label{eq:mu basis}
      \end{align}  %  6
      %\\[0.2cm]
      where the first four $\mathcal{C}_n$ are given as follows:

%&\hspace{3mm}
 
 %\begin{widetext}
\begin{align}
    &\mathcal{C}_1 = \overline{D_1}, \notag \\
    &\mathcal{C}_2 = \overline{D_2} + \left[\overline{(D_1)^2} - \overline{(D_1)}^2\right], \notag \\
    &\mathcal{C}_3 = \overline{D_3} + 3\left(\overline{D_2 D_1} - \overline{D_2}\;\overline{D_1}\right) + 
           \left(\overline{D_1^3} - 3\,\overline{(D_1)^2}\;\overline{D_1} + 2\,\left(\overline{D_1}\right)^3\right), \notag \\
    &\mathcal{C}_4 = \overline{D_4} + 3\left(\overline{(D_2)^2} - \left(\overline{D_2}\right)^2\right)+ 4\left(\overline{D_3 D_1} - \overline{D_3}\;\overline{D_1}\right) + 6\left( \overline{D_2 (D_1)^2} - \overline{D_2}\;\overline{(D_1)^2}\right)  \notag \\
    &- 12 \left(\overline{D_2 D_1}\;\overline{D_1} - \overline{D_2}\left(\overline{D_1}\right)^2\right) +
     \Big(\overline{(D_1)^4} - 4\,\,\overline{(D_1)^3}\;\overline{D_1} + 12\,\overline{(D_1)^2}\left(\overline{D_1}\right)^2 \notag \\
     &\hspace{6cm}- 6\left(\overline{D_1}\right)^4 - 3\,(\overline{(D_1)^2})^2\Big)
\label{eq:mu coefficients}
\end{align}
%
 Here in Eqn.\,\eqref{eq:mu coefficients}, $\overline{D_n^p}$ indicates average of $p^{th}$ unbiased power of $D_n$ over all the $N_{R}$ random volume sources contained within a given gauge field configuration. Eqn.\,\eqref{eq:mu basis} along with Eqn.\,\eqref{eq:mu coefficients} eliminates stochastic bias upto $\Ob(\mu^4)$.

\subsection{Cumulant basis}

\hs In cumulant basis with $X_N = \sum_{n=1}^N \left(\frac{\mu}{T}\right)^n \frac{D_n}{n!}$,  this formalism looks as follows:

 \begin{align}   %  6
     &\frac{\Delta P_{N,M}^{R(\text{unb})}(T,\mu)}{T^4} = \frac{1}{VT^3} \hspace{1mm}\ln \hspace{1mm} Z_{N,M}^{R(\text{unb})}(T,\mu) , \notag \\ 
     &Z_{N,M}^{R(\text{unb})}(T,\mu) = \text{Re} \LA \bigg[\exp \Big(W_M\big[X_N\left(T,\mu\right)\big]\Big)\bigg] \RA , \notag \\
     &W_M\big[X_N\left(T,\mu\right)\big] = \nsum_{m=1}^{M} \frac{\mathcal{L}_{m}\left(X_N\right)}{m!}
      \label{eq:cumulant basis}
      \end{align}
%
where $\L_m$ for $1 \leq m \leq 4$ are as follows:

\begin{align}
   \L_1 &= \overline{X_N}, \notag \\
   \L_2 &= \overline{(X_N)^2} - \big(\overline{X_N}\big)^2, \notag \\
   \L_3 &= \overline{(X_N)^3} - 3\,\big(\overline{X_N}\big)\;\overline{(X_N)^2} + 2\,\big(\overline{X_N}\big)^3, \notag \\
   \L_4 &= \overline{(X_N)^4}- 4\,\overline{(X_N)^3}\;\big(\overline{X_N}\big)  
         + 12\,\big(\overline{X_N}\big)^2\;\overline{(X_N)^2} - 6\,\big(\overline{X_N}\big)^4- 3\,\big(\overline{(X_N)^2}\big)^2
\label{eq:cumulant L}
\end{align}
%
Implementing the formalism in this basis following Eqn.\,\eqref{eq:cumulant basis}, using the expressions of Eqn.\,\eqref{eq:cumulant L} leads to reproducing the first $4$ cumulants exactly of unbiased cumulant expansion, where the cumulants are expressed in terms of unbiased powers. We observed that this formalism not only eliminates stochastic bias upto a finite order in $\mu$, it can also capture higher order Taylor contributions successfully. On adding more number of $\L_n$ or $\mathcal{C}_n$, one makes approach towards exponential resummation unbiased to all orders in $\mu$ which, in infinite limit becomes identical to the infinite Taylor series in $\mu$.  


\section{Setup of the calculations}
\label{sec:setup}

\hs In this work, we have extensively used the data generated by the HotQCD collaboration for its ongoing Taylor expansion calculations. In this section, we discuss the setup and other important relevant details of this data. 

With a $2+1$ flavor signature, the QCD action considered for generating the data for these calculations consists of a Symanzik-improved gauge action\,\cite{Symanzik:1983gauge,Symanzik:1983gauge2nd}
and the Highly Improved Staggered Quark (HISQ) fermion action\,\cite{Gabrielli:1990Improvement,MILC:2008HISQ,MILC:2010HISQ}. Gauge field configurations of the order $\Ob(10^4$ - $10^6)$ are generated in the temperature range $135$~MeV~$\lesssim~T~\lesssim$~$176$~MeV with $\Nt=8$, $12$ and $16$ and $\Ns=4\Nt$\footnote{$\Ns$ and $\Nt$ are the number of points in each of the three spatial directions and temporal direction respectively for $\Ns^3 \cdot \Nt$ lattice.}. The lattices considered are isotropic with lattice spacing $a=a_{\sigma}=a_{\tau}$\footnote{$a_{\sigma}$ and $a_{\tau}$ are the lattice spacings along each of the spatial and temporal directions of the lattice.}. Our work is performed using an isotropic lattice of size $32^3 \cdot 8$ in Euclidean four spacetime, which is Wick rotated from the usual $3+1$ relativistic Minkowski spacetime. Following the relation $T=(a\Nt)^{-1}$, the temperature for each $\Nt$ is varied by varying the lattice spacing $a$ through the inverse gauge coupling $\beta$ where $\beta=6/g^2$ and $g$ is the QCD coupling parameter. For each $a$ the bare light and strange quark masses $m_l(a)$ and $m_s(a)$ are also tuned so that the pseudo-Goldstone pion and kaon masses produced become equal to the physical pion ($\pi$) and kaon ($K$) masses respectively. This fixes the line of constant physics for the lattice setup under consideration. The scale setting is determined using both the Sommer parameter $r_1$ and the kaon decay constant $f_K$. A complete description of the gauge ensembles and scale setting is provided in Ref.~\,\cite{Bollweg:2021vqf}.
To calculate the Taylor coefficients for unbiased exponential resummation, on each gauge configuration the correlation functions $D_1^u,\dots,D_4^u$ for up quark are estimated stochastically using $500$ Gaussian volume sources. The mass degeneracy between up and down quarks in $2+1$ flavor nomenclature ensures that these correlation functions have the same value for down quarks, namely $D_1^d,\dots,D_4^d$ for each of these $500$ random sources. Using these correlation functions and consequent basis transformation illustrated in \autoref{sec:appendix}, the isospin correlation functions $D_1^I,\dots,D_4^I$ are calculated for these random sources. The exponential-$\mu$ formalism is used to calculate these four derivatives (see Eqn.\,\eqref{eq:derivatives})\footnote{The higher derivatives $D_n$ for $n>4$ are calculated using the linear-$\mu$ formalism.}. Using this data, we have calculated the isospin number density for real isospin chemical potentials $\muI$, in the range $0 \leqslant \lvert \mu_{I}/T \rvert \leqslant 2.5$, using $20K$ configurations per temperature. Our results have been obtained on $\Nt=8$ lattices for three temperatures namely at $T \sim 135$, $157$ and $176$ MeV. Besides lying in the hadronic, crossover and QGP phases of the QCD phase diagram, these temperatures have been chosen as being approximately equal to $\Tpc$ and $\Tpc\pm20$~MeV, where $\Tpc=156.5(1.5)$~MeV is the chiral crossover temperature at zero baryon chemical potential $\muB$ for physical values of bare quark masses.  For the determination of phasefactor at the nearest Newton-Raphson singularity in the complex $\muI$ plane, we have considered taking $100$ bootstrap samples of the working gauge configuration ensemble. Each of these bootstrapped samples comprises $20K$ configurations which may have repetition due to randomisation of the chosen random number generator in the bootstrap algorithm. 
 
% The Newton-Raphson method has been illustrated briefly in the next section and the singularities have been plotted without the errorbars, because they appear as a sizeable chunk in the region of the complex $\muI$ plane considered. This is the same during the evaluation of the phasefactor values for different values of complex $\muI$.


\section{Results: Newton-Raphson singularities of partition function}
\label{sec:Results}

\hs In this section, we present our results for the singularities of partition function $Z$ and also illustrate the behaviour of number density $\N$ as a function of $\muI$ across the radius of convergence. The partition function is estimated using the formalism of unbiased exponential resummation to $2^{nd}$ and $4^{th}$ orders in both cumulant and chemical potential bases respectively. The singularities of this partition function are obtained using the Newton-Raphson method and the distance of the nearest Newton-Raphson singularity as obtained, from the origin \{$\text{Re}\left(\muI\right)=\text{Im}\left(\muI\right)=0$\} in the complex $\muI$ plane determines the radius of convergence\footnote{Given that one of the Newton-Raphson singularities is a complex $\muI$ which is $\muI=\muI^R+i\,\muI^I$, its radial distance from origin is given by $|\muI| = \sqrt{(\muI^R)^2 + (\muI^I)^2}$. The radius of convergence is the least of all such radial distances corresponding to all such singularities.}\,$\rho$. It is crucial to determine these singularities because the free energy $\sim \ln Z$ diverges at these singularity values of complex $\muI$. Consequently, all the thermodynamic observables being some finite order $\muI$ derivative of free energy will exhibit divergent and ll-defined  behaviour at these values of $\muI$. All these results of singularities of $Z$ for three temperatures, namely $135$, $157$ and $176$ MeV have been demonstrated in Fig.\,\ref{fig:135_sing_and_num_den},  \ref{fig:157_sing_and_num_den} and \ref{fig:176_sing_and_num_den} respectively.



 The Newton-Raphson method having a quadratic rate of convergence has been implemented in this work for determining singularities of $Z$ using a maximum of $\Ob(10^6)$ iterations. Although the perturbed iterative algorithm\,\cite{Dey:1979PIS} provides faster converging results over the Newton-Raphson method, implementing this algorithm here will require one to evaluate exponential of $Z$ which produces an astronomically large number. Consequently, the computations following this algorithm becomes highly unsuitable and unstable. Hence despite its limitations in the form of calculating a non-singular derivative of $Z$, the Newton-Raphson method is preferred here as one only needs to provide the partition function $Z$ and its first $\muI$ derivative for a given value of $\muI$. The naive formula of this method for our work resembles

 \begin{equation}
    \muI^{\left[n+1\right]} = \muI^{\left[n\right]} - \frac{Z\left(\muI^{\left[n\right]}\right)}{Z^{'}\left(\muI^{\left[n\right]}\right)} \hspace{.3cm},\hspace{.7cm} \epsilon = \left|\muI^{\left[n+1\right]} - \muI^{\left[n\right]}\right|
    \label{eq:Newton-Raphson}
\end{equation}
 %
 where $n \geq 0$ is an integer, bearing the label of $n^{th}$ iteration. This method initiates from a random guess of $\muI$ value as a root of $Z$, given by $\muI^{\left[0\right]}$ in the above Eqn.\,\eqref{eq:Newton-Raphson} and subsequently, one keeps performing iterative evaluations of roots till it surpasses the maximum number of allowed iterations or until the magnitude of the difference $\epsilon$ in Eqn.\,\eqref{eq:Newton-Raphson} drops below a given value. In our work, $\epsilon = 0.002$. The initial guesses of $\muI$ i.e. $\muI^{\left[0\right]}$ have been constructed from $0 \leq \text{Re}\left(\muI\right) \leq 3$, $0 \leq \text{Im}\left(\muI\right) \leq 3$ in the steps of $0.1$ between successive guesses along each of the real and imaginary axes in the complex plane of $\muI$. In both the left column plots of Fig.\,\ref{fig:135_sing_and_num_den}, \ref{fig:157_sing_and_num_den} and \ref{fig:176_sing_and_num_den}, the blue points indicate the Newton-Raphson singularities of partition function $Z$ in cumulant basis, whereas the red points represent the same in chemical potential basis. Correspondingly, the radius of convergence $\rho$ in the respective bases are  represented by the blue and red circles respectively. The plots in the right column exhibit the behaviour of isospin number density across the radius of convergence as given by these Newton-Raphson singularities. The top row plots correspond to $2^{nd}$ order whereas the bottom row plots present the $4^{th}$ order results. Following \autoref{sec:Unbiased exp}, $\rho_{N.M}^{R(\text{unb})}$ denotes the radius of convergence in cumulant basis whereas $\rho_N^{R(\text{unb})}$ represents the same in chemical potential basis.
 % We find that the number of roots in the $2^{nd}$ order results are far less than the same for $4^{th}$ order within $0 \leq \text{Re}(\muI) \leq 2$, $0 \leq \text{Im}(\muI) \leq 2$. This is due to the decreasing value of the ratio of $Z$ and $Z^{'}$ in $4^{th}$ order as compared to the $2^{nd}$ order calculations. The denominator $Z^{'}$ of this ratio in Eqn.\,\eqref{eq:Newton-Raphson} which is the $\muI$ derivative of $Z$, comprises non-exponentiated $2$-point and $4$-point isospin correlation functions $D_2^I$ and $D_4^I$ along with other unbiased powers in the $4^{th}$ order, apart from the usual exponential form of the unbiased resummation. The larger values of $D_4^I$ over $D_2^I$ configuration by configuration coupled with higher powers of $\muI$ cause this denominator significantly larger in $4^{th}$ order over $2^{nd}$ order, which leads to smaller values of the ratio and more crowding of the final roots in this regime One also observes much more number of roots at $157$ and $176$ MeV for both $2^{nd}$ and $4^{th}$ order calculations over the same obtained at $T=135$ MeV. This occurs because the values of the temperature dependent $D_n^I$ increases with temperature, as a result of which the ratio between the $Z$ and its $\muI$ derivative decreases with temperature. This causes the successive root approximations to converge and stabilise fast, resulting to more number of roots within the same region of complex $\muI$ plane.


\begin{figure}[htbp]
\centering
\includegraphics[width=.47\textwidth]{figures/135_2nd_order_ISOSPIN_ROOTS.pdf}
\quad
\includegraphics[width=.47\textwidth]{figures/135_2nd_order_ISOSPIN_number_density.pdf}\\
\includegraphics[width=.47\textwidth]{figures/135_4th_order_ISOSPIN_ROOTS.pdf}
\quad
\includegraphics[width=.47\textwidth]{figures/135_4th_order_ISOSPIN_number_density.pdf}
\caption{(Top row) Newton-Raphson singularities of the $2^{nd}$ order partition function $Z$ and the radius of convergence $\rho$ in both cumulant and chemical potential bases at $T=135$ MeV. Plot of $2^{nd}$ order number density as a function of $\muI$ with the radius of convergence $\rho$ in both the cumulant and chemical potential bases. (Bottom row) The same plotted for $4^{th}$ order partition function $Z$ and $4^{th}$ order number density. The blue labels represent results in cumulant basis and the red labels represent same for chemical potential basis.\label{fig:135_sing_and_num_den}}
\end{figure}


% \begin{figure}[htbp]
% \centering
% \includegraphics[width=.47\textwidth]{figures/135_4th_order_ISOSPIN_ROOTS.pdf}
% \qquad
% \includegraphics[width=.47\textwidth]{figures/135_4th_order_ISOSPIN_number_density.pdf}
% \caption{(Left) Singularities of the $4^{th}$ order partition function $Z$ obtained using Newton-Raphson method at $T=135$ MeV and consequently the radius of convergence $\rho$ in both cumulant and chemical potential bases. (Right) Plot of $4^{th}$ order number density as a function of $\muI$ with the radius of convergence $\rho$ in both the cumulant and chemical potential bases. The blue labels represent results in cumulant basis and the red labels represent same for chemical potential basis.\label{fig:135_4}}
% \end{figure}

\begin{figure}[htbp]
\centering
\includegraphics[width=.47\textwidth]{figures/157_2nd_order_ISOSPIN_ROOTS.pdf}
\quad
\includegraphics[width=.47\textwidth]{figures/157_2nd_order_ISOSPIN_number_density.pdf}\\
\includegraphics[width=.47\textwidth]{figures/157_4th_order_ISOSPIN_ROOTS.pdf}
\quad
\includegraphics[width=.47\textwidth]{figures/157_4th_order_ISOSPIN_number_density.pdf}
\caption{(Top row) Newton-Raphson singularities of the $2^{nd}$ order partition function $Z$ and the radius of convergence $\rho$ in both cumulant and chemical potential bases at $T=157$ MeV. Plot of $2^{nd}$ order number density as a function of $\muI$ with the radius of convergence $\rho$ in both the cumulant and chemical potential bases. (Bottom row) The same plotted for $4^{th}$ order partition function $Z$ and $4^{th}$ order number density. The blue labels represent results in cumulant basis and the red labels represent same for chemical potential basis.\label{fig:157_sing_and_num_den}}
\end{figure}
%
%

We find that the number of roots in the $2^{nd}$ order results are far less than the same for $4^{th}$ order within $0 \leq \text{Re}(\muI) \leq 2$, $0 \leq \text{Im}(\muI) \leq 2$ for all the three temperatures. This is due to the decreasing value of the ratio of $Z$ and $Z^{'}$ in $4^{th}$ order as compared to the $2^{nd}$ order calculations. The denominator $Z^{'}$ of this ratio in Eqn.\,\eqref{eq:Newton-Raphson} being the first order $\muI$ derivative of $Z$, comprises non-exponentiated isospin correlation functions $D_2^I$ and $D_4^I$ along with other unbiased powers and usual exponential form of the unbiased resummation in $4^{th}$ order. The larger values of $D_4^I$ over $D_2^I$ for every configuration coupled with higher powers of $\muI$ cause this denominator significantly larger in $4^{th}$ order over $2^{nd}$ order evaluations, which leads to smaller values of the ratio and more crowding of the roots in this region of complex $\muI$ plane, to which this method finally converges to. One also observes the number of roots at $157$ and $176$ MeV for both $2^{nd}$ and $4^{th}$ order calculations is much more within the same part of the complex $\muI$ plane than $T=135$ MeV. This occurs because the values of the temperature dependent $D_n^I$ increases with temperature, as a result of which the ratio between the $Z$ and its $\muI$ derivative decreases with temperature. This causes the successive root approximations to converge and stabilise fast, resulting to more number of roots within the same region of complex $\muI$ plane.


We observe that the crowding of the roots begins from $\text{Re}(\muI) \sim 1.7$, $\text{Im}(\muI) \sim 0.7$ for $T=135$ MeV in Fig.\,\ref{fig:135_sing_and_num_den}. This is because, the values of $Z$ and $Z^{'}$ in Eqn.\,\eqref{eq:Newton-Raphson} become larger with increasing values of complex $\muI$. As a result, the Newton-Raphson method fails to converge properly to actual roots.
More importantly, the degree of convergence of this method becomes worse with higher values of complex $\muI$, due to which these values when considered as initial guesses for the Newton-Raphson method, often get manifested as the Newton-Raphson roots in the figures. This strongly illustrates poor convergent properties of this method for these values of complex $\muI$. It is for this very reason, which leads to large number of roots forming a rectangular grid-like structure in this part of the complex $\muI$ plane at $135$ MeV.

The same starts appearing at a lower value of the radial distance from origin for $157$ and $176$ MeV around $\text{Re}(\muI) = 1.5$, $\text{Im}(\muI) = 0.5$ in Figs.\,\ref{fig:157_sing_and_num_den} and \ref{fig:176_sing_and_num_den} respectively. This is because the isospin correlation functions $D_n^I$ for $n=2,4$ increases with increasing values of $T$, which consequently leads to larger values of $Z$ and $Z^{'}$ for smaller values of complex $\muI$. As a result, due to the aforementioned poor convergence of Newton-Raphson method, the grid-like crowding of Newton-Raphson roots start forming from these smaller values of complex $\muI$ for these higher temperatures. One can also suggest that the initial guesses in this part of complex $\muI$ plane are far off from the actual roots due to which, the Newton-Raphson method becomes inefficient. Despite recent efforts\,\cite{Dey:1979PIS} for improved algorithms, this disadvantage of Newton-Raphson approach will affect the entire anatomy of singularities in the complex $\muI$ plane and will hardly alter the position(s) of singularity(ies) lying closest to the origin of the complex $\muI$ plane, which are instrumental in determining the value of the radius of convergence $\rho$. The increasing differences observed between $\rho_{N,4}^{R(\text{unb})}$ and $\rho_N^{R(\text{unb})}$ for $N=2,4$ with higher values of $T$ also signify that these $D_n^I$ for $n=2,4$ increase as $T$ increases. They also try to emphasise that the excess number of higher-order terms obtained in cumulant basis in addition to the terms in chemical potential basis are making significant contributions in final results.

\begin{figure}[htbp]
\centering
\includegraphics[width=.47\textwidth]{figures/176_2nd_order_ISOSPIN_ROOTS.pdf}
\quad
\includegraphics[width=.47\textwidth]{figures/176_2nd_order_ISOSPIN_number_density.pdf}\\
\includegraphics[width=.47\textwidth]{figures/176_4th_order_ISOSPIN_ROOTS.pdf}
\quad
\includegraphics[width=.47\textwidth]{figures/176_4th_order_ISOSPIN_number_density.pdf}
\caption{(Top row) Newton-Raphson singularities of the $2^{nd}$ order partition function $Z$ and the radius of convergence $\rho$ in both cumulant and chemical potential bases at $T=176$ MeV. Plot of $2^{nd}$ order number density as a function of $\muI$ with the radius of convergence $\rho$ in both the cumulant and chemical potential bases. (Bottom row) The same plotted for $4^{th}$ order partition function $Z$ and $4^{th}$ order number density. The blue labels represent results in cumulant basis and the red labels represent same for chemical potential basis.\label{fig:176_sing_and_num_den}}
\end{figure}


The isospin number density plots have been presented beside the plots of Newton-Raphson singularities. This is done for observing the changing behaviour of the observable if any, across the radius of convergence which is represented by the semicircular red and blue lines similar to the adjacent plots of Newton-Raphson singularities. Although small, there are differences clearly between the monotonicity of number density across the radius of convergence $\rho$. However, this non-monotonicity manifests in different forms. Fig.\,\ref{fig:135_sing_and_num_den} shows that the errorbars in the $4^{th}$ order calculations highly increase for both the bases and the behaviour becomes non-monotonic since the curve attains a local maximum as $\muI \geq \rho$. In the $2^{nd}$ order at the same temperature, there is a flattening of the number density curve which is clearly supposed to increase more and to higher values of number density, if extrapolated beyond the radius of convergence. Although there is hardly any noticeable change in number density at $157$ MeV, the $2^{nd}$ order evaluations at $T=176$ MeV show increasing differences between the results obtained from cumulant and chemical potential bases respectively, as one goes beyond $\rho_2^{R(\text{unb})}$. We also observe a small, yet noticeable change in the monotonicity for $4^{th}$ order calculations at this temperature. It still remains to be seen in this regard, that which observable can reflect a reasonable non-monotonicity across the radius of convergence at the crossover temperature $T=157$ MeV. As mentioned before, all these calculations are performed using unbiased exponential resummation at finite $\muI$, so that stochastic bias up to $2^{nd}$ and $4^{th}$ order are eliminated. This is done in order to ensure that the following results reflect genuine indications of the phase diagram, as much as possible.  

%We already have argued that the difference between these radii of convergence of the respective cumulant and chemical potential bases increase with increasing $T$, for a given order of calculations.

% \begin{figure}[htbp]
% \centering
% \includegraphics[width=.47\textwidth]{figures/157_4th_order_ISOSPIN_ROOTS.pdf}
% \qquad
% \includegraphics[width=.47\textwidth]{figures/157_4th_order_ISOSPIN_number_density.pdf}
% \caption{(Left) Singularities of the $4^{th}$ order partition function $Z$ obtained using Newton-Raphson method at $T=157$ MeV and consequently the radius of convergence $\rho$ in both cumulant and chemical potential bases. (Right) Plot of $4^{th}$ order number density as a function of $\muI$ with the radius of convergence $\rho$ in both the cumulant and chemical potential bases. The blue labels represent results in cumulant basis and the red labels represent same for chemical potential basis.\label{fig:157_4}}
% \end{figure}



% \begin{figure}[htbp]
% \centering
% \includegraphics[width=.47\textwidth]{figures/176_4th_order_ISOSPIN_ROOTS.pdf}
% \qquad
% \includegraphics[width=.47\textwidth]{figures/176_4th_order_ISOSPIN_number_density.pdf}
% \caption{(Left) Singularities of the $4^{th}$ order partition function $Z$ obtained using Newton-Raphson method at $T=135$ MeV and consequently the radius of convergence $\rho$ in both cumulant and chemical potential bases. (Right) Plot of $4^{th}$ order number density as a function of $\muI$ with the radius of convergence $\rho$ in both the cumulant and chemical potential bases. The blue labels represent results in cumulant basis and the red labels represent same for chemical potential basis.\label{fig:176_4}}
% \end{figure}





\section{A new way of identifying singularities through phasefactor}
\label{sec:Phasefactor formalism}

\subsection{Formalism}

\hs In $2+1$ flavor QCD since the $u$ and $d$ quarks are mass degenerate, the odd isospin derivatives are identically zero, following which there is no sign problem in $\muI$. This is clearly described in \autoref{sec:appendix}. The abscence of sign problem also becomes apparent from  the phaseangle $\Theta_N(T,\mu)$ in the two bases which to $\Ob(\mu^N)$ are given by

\begin{align}
    &\Theta_N^{\text{R(unb)}}(T,\muI^C) = \text{Im}\left[\nsum_{n=1}^N \left(\frac{\muI^C}{T}\right)^n \frac{\mathcal{C}_n}{n!}\right] \notag \\
    &\Theta_{N,M}^{\text{R(unb)}}(T,\muI^C) = \text{Im}\left[\nsum_{m=1}^M \frac{\L_m(X_N(T,\muI^C))}{m!}\right]
    \label{eq:unbiased phasefactor}
\end{align}
%
where $\mathcal{C}_n$ for $1 \leq n \leq 4$ are given in Eqn.\,\eqref{eq:mu coefficients} and so are $\L_m$ from $m=1$ to $m=4$ in Eqn.\,\eqref{eq:cumulant L}. In Eqn.\,\eqref{eq:unbiased phasefactor}, the $X_N(T,\muI^C)$ are defined as 

\begin{equation*}
    X_N(T,\muI^C) = \nsum_{n=1}^N \left(\frac{\muI^C}{T}\right)^n \frac{D_n^I}{n!}
\end{equation*}
%
Since odd ordered $D_n^I$ vanishes and even ordered non-vanishing $D_n^I$ are real following the CP symmetry of QCD, the phaseangle $\Theta$ becomes zero following Eqn.\,\eqref{eq:unbiased phasefactor}. Consequently, the phasefactor $ \cos \Theta$ assumes unity for every gauge configuration at finite $\muI$. Hence, although in principle, the lattice QCD calculations do not experience a genuine breakdown due to sign problem and can be continued for all finite values of $\muI$, this phasefactor information does not appear reliable and hence cannot be utilised for capturing the singularities. This is very much unlike $\muB$ or similar type of chemical potentials like $\mu_S$, $\mu_Q$ which otherwise suffer from a sign problem and in which, this problem can be located by the zeroes of this phasefactor $\LA \cos \Theta \RA$ averaged out over the working gauge ensemble. 

Similar to this notion of phasefactor, we propose formulating a non-trivial phasefactor in $\muI$ with the aim of capturing the radius of convergence, set by the closest Newton-Raphson singularity. This phasefactor is introduced by complexifying $\muI$. Given that the phasefactor is an already proven reliable indicator of the sign problem, it will be interesting to observe if this phasefactor at a complex $\muI$ can furnish genuine indications about the manifestation of the singularities and eventually the radius of convergence. At complex $\muI$, the reweighting factor becomes complex which therefore would yield a non-trivial phasefactor, which is expected to differ for different values of complex $\muI$. For a given complex $\muI = \muI^R + i\,\muI^I$, the phaseangle in the chemical potential basis for second and fourth order calculations are given as follows:

\begin{align}
    \Theta_2(T,\muI^R,\muI^I) &= \muI^R \muI^I \, \mathcal{C}_2 \label{eq:2nd order}\\
    \Theta_4(T,\muI^R,\muI^I) &= \muI^R \muI^I \left[\mathcal{C}_2 + 4\left(\left(\muI^R\right)^2 - \left(\muI^I\right)^2\right) \mathcal{C}_4 \right] \label{eq:4th order}
\end{align}
%
Eqn.\,\eqref{eq:2nd order} implies that the phaseangle $\Theta_2$ remains invariant as $(\muI^R, \muI^I) \rightarrow (\muI^I, \muI^R)$, and so will be the corresponding phasefactor. This proves that there are multiple values of complex $\muI$ which can harness the same phaseangle and subsequent phasefactor. Hence, we plot this average phasefactor as a function of the radial distance from origin in the next section. This proves to be beneficial not only by reducing this multiplicity of phasefactor, but also by helping to understand its ability of capturing the radius of convergence in a better and more convenient way.
% A complex $\muI$ can be written as $\muI^C = \muI^R + i\,\muI^I$ where $\muI^R = \text{Re}(\muI^C)$ and $\muI^I = \text{Im}(\muI^C)$. The radius of convergence $\rho_I$ is defined as 

% \begin{equation}
% \rho_I = |\muI^C| = \sqrt{\left(\muI^R\right)^2 + \left(\muI^I\right)^2}    
% \label{eq:radius}
% \end{equation}
% %
%   Following Eqn.\,\eqref{eq:expoential resum}, the phaseangle $\Theta_N(T,\muI^C)$ to $\Ob\left[(\muI^C)^N\right]$ can be defined as  

%  \begin{equation}
%  \Theta_N(T,\muI^C) = \text{Im}\left[\nsum_{n=1}^N \left(\frac{\muI^C}{T}\right)^n \frac{D_n^I}{n!}\right]
%  \end{equation}
%  %
% where $D_n^I$ are the $n$-point isospin correlation functions, which we know evaluates to zero for all odd $n$. 
% In the case of unbiased exponential resummation, this phasefactor in the respective chemical potential (see Eqn.\,\eqref{eq:mu basis}) and cumulant (see Eqn.\,\eqref{eq:cumulant basis}) bases translates to as follows:

% \begin{align}
%     &\Theta_N^{\text{R(unb)}}(T,\muI^C) = \text{Im}\left[\nsum_{n=1}^N \left(\frac{\muI^C}{T}\right)^n \frac{\mathcal{C}_n}{n!}\right] \notag \\
%     &\Theta_{N,M}^{\text{R(unb)}}(T,\muI^C) = \text{Im}\left[\nsum_{m=1}^M \frac{\L_m(X_N(T,\muI^C))}{m!}\right]
%     \label{eq:unbiased phasefactor complex}
% \end{align}
% %
% where $\mathcal{C}_n$ for $1 \leq n \leq 4$ are given in Eqn.\,\eqref{eq:mu coefficients} and so are $\L_m$ from $m=1$ to $m=4$ in Eqn.\,\eqref{eq:cumulant L}. In Eqn.\,\eqref{eq:unbiased phasefactor complex}, the $X_N(T,\muI^C)$ are defined as 

% \begin{equation*}
%     X_N(T,\muI^C) = \nsum_{n=1}^N \left(\frac{\muI^C}{T}\right)^n \frac{D_n^I}{n!}
% \end{equation*}
% \vspace{.5cm}

\subsection{Results}

\hs We present the results of $2^{nd}$ and $4^{th}$ order average phasefactor $\LA \cos \Theta \RA$ as a function of the radial distance $\rho_I$ for $T=135$, $157$ and $176$ MeV in Figs.\ref{fig:135_phasefac}, \ref{fig:157_phasefac} and \ref{fig:176_phasefac} respectively. The definition and explicit determination of radial distance of a singularity point in complex $\muI$ plane has already been illuminated in \autoref{sec:Results}. We have also captured the nearest Newton Raphson singularity along with the measurement of the average phasefactor at the singularity. This is because, we are interested in capturing the onset of these singularities of $Z$ in the complex $\muI$ plane rather than ascertaining the exact coordinates of singularities in the complex plane. This is expected to offer a proper idea about the regime of reliability and also the starting point of breakdown of unbiased exponential resummation in $\muI$. Once this breakdown starts, it is expected to continue for all values of $\muI$ lying beyond this breakdown value. This imply that this formalism hence, needs to be modified for further searches and also to extend the radius of convergence or its domain of application in complex $\muI$ plane, where it can offer reliable genuine results. In all these figures for all the temperatures, the orange points represent the average phasefactor values in cumulant and chemical potential bases respectively with the black vertical line signifies the Newton-Raphson radius of convergence $\rho_I^{\text{NR}}$. The red line indicates the critical value $\rho_I^0$ of the radial distance from origin in the complex $\muI$ plane, from which there is a continuous trail of the $\LA \cos \Theta \RA$ points condensing to zero. The phasefactor results have been obtained using $0 \leq \muI^R \leq 3$, $0 \leq \muI^I \leq 3$ in steps of $0.1$, so that in units of the radial distance, we plot phasefactor values in the range $0 \leq \rho_I \leq 3\,\sqrt{2} \approx 4.25$. 
%We also define a quantity $\diff = \rho_I^{\text{NR}} - \rho_I^0$, the magnitude of which measures the separation between the two vertical lines and and the sign of which determines the relative position of the lines with respect to the vertical $Y$ axis.

\begin{figure}[htbp]
\centering
\includegraphics[width=.47\textwidth]{figures/135_2nd_order_absolute_ISOSPIN_phasefactor_X.pdf}
\quad
\includegraphics[width=.47\textwidth]{figures/135_2nd_order_absolute_ISOSPIN_phasefactor_mu.pdf}\\
\includegraphics[width=.47\textwidth]{figures/135_4th_order_absolute_ISOSPIN_phasefactor_X.pdf}
\quad
\includegraphics[width=.47\textwidth]{figures/135_4th_order_absolute_ISOSPIN_phasefactor_mu.pdf}
\caption{(Top row) Plots of the $2^{nd}$ order average phasefactor $\LA \cos\,\Theta \RA$ as a function of the radial distance $\rho_I$ for isospin ($\muI$) chemical potential in both cumulant basis (left) and chemical potential basis (right) respectively at $T=135$ MeV. (Bottom row) The same is done for $4^{th}$ order average phasefactor. For both $2^{nd}$ and $4^{th}$ orders, the orange points (squares) represent the average phasefactor data and the black point (diamond) illustrate $\LA \cos\,\Theta \RA$ at Newton-Raphson singularity. The black line represents the Newton-Raphson radius of convergence $\rho_I^{NR}$ whereas the red line marks the zero point, from where $\LA \cos\,\Theta \RA=0$ for some value of complex $\muI$. The blue line shows the zero line of phasefactor i.e. $\LA \cos\,\Theta \RA=0$.  \label{fig:135_phasefac}}
\end{figure}

In plotting as a function of the radial distance, there is a two-fold degeneracy of the $2^{nd}$ order phasefactor values in chemical potential basis, in Eqn.\,\eqref{eq:2nd order}. This is because this distance remains invariant under the transformation $(\muI^R, \muI^I) \Rightarrow (\muI^I, \muI^R)$ and hence for every value of the radius, one finds at least two distinct values of this phasefactor. 
 At $135$ MeV as revealed in Fig.\,\ref{fig:135_phasefac}, we find the red line lies left to the black line, where the former is determined from the phasefactor values and latter from the nearest Newton-Raphson singularity. All the plots of Fig.\,\ref{fig:135_phasefac} show that $\diff$\footnote{$\diff$ is defined as $\diff = \rho_I^{\text{NR}} - \rho_I^0$} satisfies $0 < \diff \lesssim 0.2$. This demonstrates that the zeroes of phasefactor efficiently capture all the possible Newton-Raphson singularities at this temperature, since majority of the phasefactor values saturate to a continuous zero before the onset of the radius of convergence $\rho_I^{\text{NR}}$. This is emphasised by the positive sign of $\diff$, which means that this phasefactor successfully puts a bound and can save unbiased exponential resummation from running into the regime of Newton-Raphson singularities and breaking down eventually. We also find that the red and black lines are not positioned drastically apart from each other for calculations in both the bases and for both the orders. We also readily observe that the value of the phasefactor obtained at the nearest Newton-Raphson singularity lying on the circle of convergence given by the black line, is very close to zero with reasonable errorbars. This bodes well with the usual expectation of having highly oscillating phasefactor values when one approaches these Newton-Raphson singularities of partition function $Z$.

% \begin{figure}[htbp]
% \centering
% \includegraphics[width=.47\textwidth]{figures/135_4th_order_absolute_ISOSPIN_phasefactor_X.pdf}
% \qquad
% \includegraphics[width=.47\textwidth]{figures/135_4th_order_absolute_ISOSPIN_phasefactor_mu.pdf}
% \caption{(Left) Plots of the $4^{th}$ order average phasefactor $\LA \cos\,\Theta \RA$ as a function of the radius of convergence $\rho_I$ for isospin ($\muI$) chemical potential in cumulant basis. (Right) The same plot in chemical potential basis. In both cases at $T=135$ MeV, the orange points (squares) represent the average phasefactor data and the black point (diamond) illustrate $\LA \cos\,\Theta \RA$ at Newton-Raphson singularity. The black line represents the Newton-Raphson radius of convergence $\rho_I^{NR}$ whereas the red line marks the zero point, from where $\LA \cos\,\Theta \RA=0$ for some value of complex $\muI$. The blue line shows the zero line of phasefactor i.e. $\LA \cos\,\Theta \RA=0$. \label{fig:135_4_phasefac}}
% \end{figure}


\begin{figure}[htbp]
\centering
\includegraphics[width=.47\textwidth]{figures/157_2nd_order_absolute_ISOSPIN_phasefactor_X.pdf}
\quad
\includegraphics[width=.47\textwidth]{figures/157_2nd_order_absolute_ISOSPIN_phasefactor_mu.pdf}\\
\includegraphics[width=.47\textwidth]{figures/157_4th_order_absolute_ISOSPIN_phasefactor_X.pdf}
\quad
\includegraphics[width=.47\textwidth]{figures/157_4th_order_absolute_ISOSPIN_phasefactor_mu.pdf}
\caption{(Top row) Plots of the $2^{nd}$ order average phasefactor $\LA \cos\,\Theta \RA$ as a function of the radius of convergence $\rho_I$ for isospin ($\muI$) chemical potential in both cumulant basis (left) and chemical potential basis (right) respectively at $T=157$ MeV. (Bottom row) The same is done for $4^{th}$ order average phasefactor. For both $2^{nd}$ and $4^{th}$ orders, the orange points (squares) represent the average phasefactor data and the black point (diamond) illustrate $\LA \cos\,\Theta \RA$ at Newton-Raphson singularity. The black line represents the Newton-Raphson radius of convergence $\rho_I^{NR}$ whereas the red line marks the zero point, from where $\LA \cos\,\Theta \RA=0$ for some value of complex $\muI$. The blue line shows the zero line of phasefactor i.e. $\LA \cos\,\Theta \RA=0$.  \label{fig:157_phasefac}}
\end{figure}


 Compared to $135$ MeV, at the crossover temperature $157$ MeV the degree of capture is more precise for $4^{th}$ order calculations over the $2^{nd}$ order, which is quantified by the positive lower magnitude of $\diff$. This is true in both the cumulant and chemical potential bases as shown in Fig.\,\ref{fig:157_phasefac}, despite the former capturing more number of higher order contribution terms over the latter. In the $2^{nd}$ order computations, the value of $\diff$ is around $0.2$ in cumulant basis, which reduces even further while considering chemical potential basis, with no alteration in the sign. Although the lowering magnitude of $\Delta_I(NR,0)$ may be attributed to the less number of terms present in the chemical potential basis, the positive sign continues to indicate that the upper bound of reliability in the scale pf $\rho_I$ starting from $\rho_I=0$ is imposed by the zeroes of this newly defined phasefactor rather than the nearest Newton-Raphson singularity; the former continues to encapsulate the latter also in the crossover region of the QCD phase diagram. The agreement and the accuracy of this capture is very appreciable for the $4^{th}$ order calculations where the two lines seem to merge with each other signifying $\diff \rightarrow 0^{+}$\footnote{The $+$ sign indicates that the value goes to zero from the positive side of zero in the real number line.}.
% \begin{figure}[htbp]
% \centering
% \includegraphics[width=.47\textwidth]{figures/157_4th_order_absolute_ISOSPIN_phasefactor_X.pdf}
% \qquad
% \includegraphics[width=.47\textwidth]{figures/157_4th_order_absolute_ISOSPIN_phasefactor_mu.pdf}
% \caption{(Left) Plots of the $4^{th}$ order average phasefactor $\LA \cos\,\Theta \RA$ as a function of the radius of convergence $\rho_I$ for isospin ($\muI$) chemical potential in cumulant basis. (Right) The same plot in chemical potential basis. In both cases at $T=157$ MeV, the orange points (squares) represent the average phasefactor data and the black point (diamond) illustrate $\LA \cos\,\Theta \RA$ at Newton-Raphson singularity. The black line represents the Newton-Raphson radius of convergence $\rho_I^{NR}$ whereas the red line marks the zero point, from where $\LA \cos\,\Theta \RA=0$ for some value of complex $\muI$. The blue line shows the zero line of phasefactor i.e. $\LA \cos\,\Theta \RA=0$. \label{fig:157_4_phasefac}}
% \end{figure}


\begin{figure}[htbp]
\centering
\includegraphics[width=.47\textwidth]{figures/176_2nd_order_absolute_ISOSPIN_phasefactor_X.pdf}
\quad
\includegraphics[width=.47\textwidth]{figures/176_2nd_order_absolute_ISOSPIN_phasefactor_mu.pdf}\\
\includegraphics[width=.47\textwidth]{figures/176_4th_order_absolute_ISOSPIN_phasefactor_X.pdf}
\quad
\includegraphics[width=.47\textwidth]{figures/176_4th_order_absolute_ISOSPIN_phasefactor_mu.pdf}
\caption{(Top row) Plots of the $2^{nd}$ order average phasefactor $\LA \cos\,\Theta \RA$ as a function of the radius of convergence $\rho_I$ for isospin ($\muI$) chemical potential in both cumulant basis (left) and chemical potential basis (right) respectively at $T=176$ MeV. (Bottom row) The same is done for $4^{th}$ order average phasefactor. For both $2^{nd}$ and $4^{th}$ orders, the orange points (squares) represent the average phasefactor data and the black point (diamond) illustrate $\LA \cos\,\Theta \RA$ at Newton-Raphson singularity. The black line represents the Newton-Raphson radius of convergence $\rho_I^{NR}$ whereas the red line marks the zero point, from where $\LA \cos\,\Theta \RA=0$ for some value of complex $\muI$. The blue line shows the zero line of phasefactor i.e. $\LA \cos\,\Theta \RA=0$.  \label{fig:176_phasefac}}
\end{figure}

As for $T=176$ MeV, we observe that in Fig.\,\ref{fig:176_phasefac} the $2^{nd}$ order result in cumulant basis exhibit commendable degree of precision in capturing the Newton-Raphson singularity from the zeroes of phasefactor. We find that this phasefactor however fails to capture the singularity for the $4^{th}$ order results as well as the $2^{nd}$ order results in chemical potential basis at $176$ MeV, which can be quantified by the negative sign of $\diff$. This behaviour may be attributed to the flattening of the pion condensate curve for higher values of $T$ in the isospin phase diagram of QCD. We also find that the phasefactor value at the singularity in $4^{th}$ order computations in cumulant basis is significantly lower than the zero (blue) line as compared to the other temperatures at $135$ and $157$ MeV. The correlation functions $D_n^I$ increase with increasing temperature and at $176$ MeV, is the highest among all these three temperatures. This may highlight the importance and need to incorporate even higher order contributions in $\muI$ in chemical potential basis or, more number of cumulants in the cumulant basis of unbiased exponential resummation. Despite this failure, we make an interesting observation which may enlighten the utility of this phasefactor at $176$ MeV in the next section. We investigate overlap problem in the next section and observe how it becomes severe across the radius of convergence $\rho_I^{\text{NR}}$ and also across the critical radial value $\rho_I^0$. 


\subsection{Severity of overlap problem and kurtosis}

\hs As mentioned before, the lattice QCD computations for finite isospin chemical potential $\muI$ does not suffer from a sign problem. As a result, it is important to investigate the overlap problem, which may be one of the many possible reasons for the breakdown of unbiased exponential resummation starting from some threshold value of finite $\muI$ along the real $\muI$ axis. 
This problem arises when the distribution or sample comprising values of the ratio of fermion determinants i.e. $\frac{\det \M(\muI)}{det \M(0)}$ becomes heavily tailed which causes Monte-Carlo importance sampling ineffective. One comes across this ratio while reweighting the integrand of the path integral formulation of $Z$ in Eqn.\,\eqref{eq:partition function}. Being a function of individual gauge configurations, every gauge configuration in the ensemble yields a value of this ratio. In the light of exponential resummation, this ratio for a gauge configuration $U$ can be cast in an exponential form as follows:

\begin{equation}
    \frac{\det \M(\muI,U)}{\det \M(0,U)} = \exp{\left[\nsum_{n=1}^{\infty} \muI^n \frac{D_n^I(U)}{n!}\right]}, \hs D_n^I(U) = \frac{\partial^n }{\partial \muI^n}\ln \det \M(\muI,U)\bigg|_{\muI=0}
    \label{eq:reweighting}
\end{equation}
%
The above exponential representation of the ratio in Eqn.\,\eqref{eq:reweighting} is only possible because the fermion determinant at a finite $\muI$ is positive definite for every gauge configuration $U$. With a fixed positive sign, the distribution formed from the values of this ratio from different configurations gives an idea about the extent of the overlap problem. More precisely, the severity of this problem is characterised by the tail of the distribution. A heavy tailed distribution\footnote{These distributions have large sample variance and often the sample mean is drastically different from the population mean.} is characterised by values which are highly apart from the distribution mean, manifesting with appreciable probability and this trait tends to change the sample statistics by a large extent. Although standard deviation can prove to be a reliable estimate characterising the heavy tail, a better quantitative measure is kurtosis $\kappa$ which is defined to be the standardised fourth order central moment. For a total of $N$ gauge field configurations, this is represented by:

\begin{equation}
    \kappa(\muI) = \frac{M_4^{\bar{x}}(\muI)}{(\sigma(\muI))^4}
\end{equation}
%
where $M_4^{\bar{x}}$ is the $4^{th}$ order central moment and $\sigma$ is the standard deviation of the distribution with mean $\bar{x}$. These are defined as
\begin{equation}
    M_4^{\bar{x}} = \frac{1}{N} \nsum_{i=1}^N (x_i-\bar{x})^4, \hspace{5mm} \sigma = \left[\frac{1}{N} \nsum_{i=1}^N (x_i-\bar{x})^2\right]^{\frac{1}{2}}, \hspace{5mm} \bar{x} = \frac{1}{N} \nsum_{i=1}^N x_i 
\end{equation}

where $x_i$ is the value of reweighting factor or the ratio of the fermion determinants obtained from $i^{th}$ configuration. 
\begin{figure}[htbp]
\centering
\includegraphics[width=.47\textwidth]{figures/135_2nd_order_kurtosis_X_error.pdf}
\quad
\includegraphics[width=.47\textwidth]{figures/135_2nd_order_kurtosis_mu_error.pdf}\\
\includegraphics[width=.47\textwidth]{figures/135_4th_order_kurtosis_X_error.pdf}
\quad
\includegraphics[width=.47\textwidth]{figures/135_4th_order_kurtosis_mu_error.pdf}
\caption{(Top row) Plots of the $2^{nd}$ order kurtosis $\kappa_2$ as a function of the isospin ($\muI$) chemical potential in cumulant basis (left) and chemical potential basis (right) at $T=135$ MeV. (Bottom row) The same plots are made for $4^{th}$ order results. The red and black lines have their usual interpretations as mentioned before. The blue points provide the measure of kurtosis. \label{fig:135_kurtosis}}
\end{figure}

We present results illustrating kurtosis as a function of real $\muI$ at the three temperatures, namely $135$, $157$ and $176$ MeV respectively. The calculations are performed for both $2^{nd}$ and $4^{th}$ orders in the cumulant and chemical potential bases and are demonstrated in Fig.\,\ref{fig:135_kurtosis}, \ref{fig:157_kurtosis} and \ref{fig:176_kurtosis} respectively. The red and black lines indicate the usual $\rho_I^0$ and $\rho_I^{\text{NR}}$ which have been already explained in the previous section. These lines intersect the positive real $\muI$ axis in the following Fig.\,\ref{fig:135_kurtosis}, \ref{fig:157_kurtosis} and \ref{fig:176_kurtosis} at $\muI^0$ and $\muI^{\text{NR}}$, where $\muI^0 = \rho_I^0$ and $\muI^{\text{NR}} = \rho_I^{\text{NR}}$. Although the absolute values of kurtosis are very large of $\Ob(10^3-10^4)$ as shown in the figures, the relative values of this kurtosis for successive values of $\muI$, forms the pivotal piece of interest here. As expected, all these figures feature that the value of the errorbars becomes strikingly large and increases at a rapid pace as the values of $\muI$ are increased beyond  $\muI^{\text{NR}}$ which is the value of Newton-Raphson radius of convergence. This signifies that across the singularity, there is a minimal overlap between the distributions generated at a finite $\muI$ and $\muI=0$. Hence, using an extrapolation from the data at $\muI=0$ to obtain thermodynamics at finite $\muI$, near and beyond these singularity values, is a very poor and ill-favoured idea.  

In Fig.\,\ref{fig:135_kurtosis}, we show that the magnitude of the errorbars start increasing as $\muI$ is increased beyond $\muI^0$. This is from where the average value of the phasefactor goes to zero as indicated by the red line. Besides putting an upper bound on the reliability of unbiased exponential resummation in real $\muI$, we also find that the red line marks the beginning of a significant increase in errorbars at $135$ MeV. Although the errorbars increase even more after crossing $\muI^{\text{NR}}$, the importance of $\muI^0$ as the starting point of these fluctuations cannot be underestimated. We find this behaviour of errorbars in both the $2^{nd}$ and $4^{th}$ order calculations for both the cumulant and chemical potential bases at this temperature. A similar set of observations and conclusions applies also for Fig.\,\ref{fig:157_kurtosis}, where we find this stark increase of errorbars beyond $\muI^0$, even more visibly appreciable. One may hold the reducing gap between $\muI^0$ and $\muI^{\text{NR}}$ responsible for this observation. These observations imply that this measure of phasefactor zeroes, rather than the nearest Newton-Raphson singularities for $135$ and $157$ MeV is a better and more reliable indicator to understand the onset of the severity of overlap problem.  

 The observations performed at $T=176$ MeV, as demonstrated in Fig.\,\ref{fig:176_kurtosis} very much support this implication. Although we find $\muI^{\text{NR}} < \muI^0$ at this temperature which highlights that the zeroes of this new phasefactor do not reliably reflect and capture the Newton-Raphson radius of convergence, we notice that the errorbars significantly increase as one crosses $\muI^0$ rather than $\muI^{\text{NR}}$ at this temperature. This is clearly evident from the $4^{th}$ order plots of cumulant and chemical potential bases. Although there is a commendable difference between $\muI^0$ and $\muI^{\text{NR}}$ and the errorbars start increasing across $\muI^{\text{NR}}$, the degree of increment of these errorbars is distinctly higher as one goes beyond $\muI^0$. This very much indicates that the severity of overlap problem maybe is a reflection of the genuine behaviour of integrand in the path integral of partition function $Z$ rather than its Newton-Raphson singularities.    



%Although there is an appreciable distinction between these two boundary lines in units of $\muI$ specially in the $4^{th}$ order calculations (bottom row) in Fig.\,\ref{fig:176_kurtosis} with negative $\diff$ and the errorbars increasing across the radius of convergence $\rho_I^{\text{NR}}$, but the degree of increment is palpably higher across the $\mu_I^0$. marking the start of rapidly oscillating integrand eventually causing the average phasefactor value to go down to zero.





% \begin{figure}[htbp]
% \centering
% \includegraphics[width=.47\textwidth]{figures/135_4th_order_kurtosis_X.pdf}
% \qquad
% \includegraphics[width=.47\textwidth]{figures/135_4th_order_kurtosis_mu.pdf}
% \caption{(Left) Plots of the $4^{th}$ order kurtosis $\kappa_4$ as a function of the isospin ($\muI$) chemical potential in cumulant basis. (Right) The same plot in chemical potential basis, where both are plotted at $T=135$ MeV. The red line indicate the value of $\muI$ and the circular region in complex $\muI$ plane, from where the average phasefactor becomes zero. The black line illustrates the radius of convergence intersecting the real $\muI$ axis, and the blue points provide the measure of kurtosis. \label{fig:135_4_kurtosis}}
% \end{figure}


\begin{figure}[htbp]
\centering
\includegraphics[width=.47\textwidth]{figures/157_2nd_order_kurtosis_X_error.pdf}
\quad
\includegraphics[width=.47\textwidth]{figures/157_2nd_order_kurtosis_mu_error.pdf}\\
\includegraphics[width=.47\textwidth]{figures/157_4th_order_kurtosis_X_error.pdf}
\quad
\includegraphics[width=.47\textwidth]{figures/157_4th_order_kurtosis_mu_error.pdf}
\caption{(Top row) Plots of the $2^{nd}$ order kurtosis $\kappa_2$ as a function of the isospin ($\muI$) chemical potential in cumulant basis (left) and chemical potential basis (right) at $T=157$ MeV. (Bottom row) The same plots are made for $4^{th}$ order results. The red and black lines have their usual interpretations as mentioned before. The blue points provide the measure of kurtosis.  \label{fig:157_kurtosis}}
\end{figure}

% \begin{figure}[htbp]
% \centering
% \includegraphics[width=.47\textwidth]{figures/157_4th_order_kurtosis_X.pdf}
% \qquad
% \includegraphics[width=.47\textwidth]{figures/157_4th_order_kurtosis_mu.pdf}
% \caption{(Left) Plots of the $4^{th}$ order kurtosis $\kappa_4$ as a function of the isospin ($\muI$) chemical potential in cumulant basis. (Right) The same plot in chemical potential basis, where both are plotted at $T=157$ MeV. The red line indicate the value of $\muI$ and the circular region in complex $\muI$ plane, from where the average phasefactor becomes zero. The black line illustrates the radius of convergence intersecting the real $\muI$ axis, and the blue points provide the measure of kurtosis. \label{fig:157_4_kurtosis}}
% \end{figure}



\begin{figure}[htbp]
\centering
\includegraphics[width=.47\textwidth]{figures/176_2nd_order_kurtosis_X_error.pdf}
\quad
\includegraphics[width=.47\textwidth]{figures/176_2nd_order_kurtosis_mu_error.pdf}\\
\includegraphics[width=.47\textwidth]{figures/176_4th_order_kurtosis_X_error.pdf}
\quad
\includegraphics[width=.47\textwidth]{figures/176_4th_order_kurtosis_mu_error.pdf}
\caption{(Top row) Plots of the $2^{nd}$ order kurtosis $\kappa_2$ as a function of the isospin ($\muI$) chemical potential in cumulant basis (left) and chemical potential basis (right) at $T=176$ MeV. (Bottom row) The same plots are made for $4^{th}$ order results. The red and black lines have their usual interpretations as mentioned before. The blue points provide the measure of kurtosis.  \label{fig:176_kurtosis}}
\end{figure}


\section{Summary and Conclusions}
\label{sec:Conclusions}

\hs In this paper, we have presented a new formalism of evaluating a non-trivial phasefactor at complex $\muI$ which can efficiently encapsulate the Newton-Raphson singularities of partition function $Z$ in the complex $\muI$ plane. These singularities are crucial to determine, because the thermodynamic observables become ill-defined for these values of $\muI$. Hence, they may reflect important signatures of a possible phase transition and manifest important aspects of the isospin phase diagram to some extent. The unbiased exponential resummation is considered since, it eliminates stochastic bias to a given order. The functional form of partition function obtained using this resummation is considered while evaluating the Newton-Raphson singularities. The determination of these singularities also leads to constructing the circle of convergence in the complex $\muI$ plane which is important for identifying the onset of breakdown for unbiased exponential resummation along the real $\muI$ axis. This becomes possible since the circle of convergence in the complex $\muI$ plane intersects the real $\muI$ axis at a finite value of $\muI$, thereby providing $\muI^{\text{NR}}$ beyond which this resummation is expected to break down. In support of this argument, we have illustrated the behaviour of isospin number density in real $\muI$, in which we have observed some degree of non-monotonicity of number density across $\muI^{\text{NR}}$. We have formulated a way of having a non-trivial phasefactor by complexifying $\muI$. This not only captures these singularities reliably, but also manages to put an upper bound of convergence and reliability in real $\muI$ for calculations starting from zero $\muI$. This bound implies that this resummation method fails to produce reliable results and consequently breaks down for real $\muI$ values greater than this bound. We find the value of this bound is less as well as not strikingly different from the point of intersection of the circle of convergence with the real $\muI$ axis, at least for $135$ and $157$ MeV which correspond to the hadronic and crossover regimes in the QCD phase diagram. This implies that at these temperatures, this new phasefactor is a more reliable indicator and by preceding over the value of $\muI^{\text{NR}}$, it can justify the non-monotonicity beyond radius of convergence and consequently can put warning signs which can be very helpful in producing reliable and genuine results.  In addition, we have demonstrated how the overlap problem signified by the tail of distribution of the reweighting factors become increasingly alarming with increasing values of $\muI$. We also observe that they become highly severe while crossing the bound $\muI^0$ from where this new phasefactor points start yielding zero continuously. Although this phasefactor at complex $\muI$ does not efficiently capture the singularities at $T=176$ MeV in the high temperature quark-gluon plasma regime, we observe that the relative kurtosis values of the distribution at different $\muI$ which gives a quantitative measure of the overlap problem, become extremely large with significantly lofty errorbars on crossing the phasefactor bound. Most importantly at this temperature, the errorbars noticeably increase much more across the phasefactor bound $\muI^0$ than the radius of convergence $\muI^{\text{NR}}$, obtained from the Newton-Raphson singularity in the complex $\muI$ plane. This very much imply that it is the rapid oscillations of the integrand of the partition function, rather than its Newton-Raphson singularities in the complex $\muI$ plane that contribute significantly to the severity of this overlap problem, and provide genuine indications of this problem.

% \begin{figure}[htbp]
% \centering
% \includegraphics[width=.47\textwidth]{figures/176_4th_order_absolute_ISOSPIN_phasefactor_X.pdf}
% \qquad
% \includegraphics[width=.47\textwidth]{figures/176_4th_order_absolute_ISOSPIN_phasefactor_mu.pdf}
% \caption{(Left) Plots of the $4^{th}$ order average phasefactor $\LA \cos\,\Theta \RA$ as a function of the radius of convergence $\rho_I$ for isospin ($\muI$) chemical potential in cumulant basis. (Right) The same plot in chemical potential basis. In both cases at $T=176$ MeV, the orange points (squares) represent the average phasefactor data and the black point (diamond) illustrate $\LA \cos\,\Theta \RA$ at Newton-Raphson singularity. The black line represents the Newton-Raphson radius of convergence $\rho_I^{NR}$ whereas the red line marks the zero point, from where $\LA \cos\,\Theta \RA=0$ for some value of complex $\muI$. The blue line shows the zero line of phasefactor i.e. $\LA \cos\,\Theta \RA=0$. \label{fig:176_4_phasefac}}
% \end{figure}








% \begin{figure}[htbp]
% \centering
% \includegraphics[width=.47\textwidth]{figures/176_4th_order_kurtosis_X.pdf}
% \qquad
% \includegraphics[width=.47\textwidth]{figures/176_4th_order_kurtosis_mu.pdf}
% \caption{(Left) Plots of the $4^{th}$ order kurtosis $\kappa_4$ as a function of the isospin ($\muI$) chemical potential in cumulant basis. (Right) The same plot in chemical potential basis, where both are plotted at $T=176$ MeV. The red line indicate the value of $\muI$ and the circular region in complex $\muI$ plane, from where the average phasefactor becomes zero. The black line illustrates the radius of convergence intersecting the real $\muI$ axis, and the blue points provide the measure of kurtosis. \label{fig:176_4_kurtosis}}
% \end{figure}















% \begin{figure}[htbp]
% \centering
% \includegraphics[width=.4\textwidth]{example-image-a}
% \qquad
% \includegraphics[width=.4\textwidth]{example-image-b}
% \caption{Always give a caption.\label{fig:i}}
% \end{figure}



% \paragraph{Up to paragraphs.} We find that having more levels usually
% reduces the clarity of the article. Also, we strongly discourage the
% use of non-numbered sections (e.g.~\texttt{\textbackslash
%   subsubsection*}).  Please also consider the use of
% ``\texttt{\textbackslash texorpdfstring\{\}\{\}}'' to avoid warnings
% from the \texttt{hyperref} package when you have math in the section titles.



% \appendix
% \section{Some title}
% Please always give a title also for appendices.



%\vspace{1cm}

\acknowledgments

\hs I sincerely acknowledge Prasad Hegde for useful discussion and constructive suggestions for this draft. I also thank all the other members of the HotQCD collaboration for their inputs and valuable discussions, as well as for allowing me to use their data from the respective Taylor expansion calculations. The computations in this work have been performed on the GPU cluster at Bielefeld University, Germany. I also heartily thank the Bielefeld HPC.NRW team for their wholehearted support.

%\vspace{4cm}


\appendix
\section{Appendix: Basis transformation}
\label{sec:appendix}

In this appendix, we present the basis transformation formulae from $\left(u,d,s\right)$ basis to $\left(B,S,I\right)$ basis, where the symbols have their usual conventional meanings. The calculations in this paper have been performed after performing the basis transformation of the HotQCD data in $\left(u,d,s\right)$ to $\left(B,S,I\right)$ basis.

We use the usual quantum number conservation formulae as follows

\begin{align}
    B &= \frac{1}{3} \hspace{1mm}\bigg[N_u+N_d+N_s\bigg] \notag\\
    I &= \frac{1}{2} \hspace{1mm}\bigg[N_u-N_d\bigg] \notag\\
    S &= -N_s
    \label{eq:quant conservation}
\end{align}

where $N_u$, $N_d$, $N_s$ are the number of up, down and strange quarks respectively.  

Using the fact that fugacity $= \sum_k \mu_k N_k$ is basis independent, where $k$ characterises basis, we have 

\begin{equation}
    \mu_B B + \mu_S S + \mu_I I = \mu_u N_u + \mu_d N_d + \mu_s N_s
    \label{eq:fuga conser}
\end{equation}

Applying relations of \autoref{eq:quant conservation} in the above \autoref{eq:fuga conser} and equating coefficients of $N_u$, $N_d$ and $N_s$ which are all independent terms in $\left(u,d,s\right)$ basis, we obtain the following differentials for a given semi-positive definite integer $k$


\begin{align}
    \frac{\partial^k}{\partial \mu_B^k} &= \frac{1}{3^k} \left[\frac{\partial^k}{\partial \mu_u^k} + \frac{\partial^k}{\partial \mu_d^k} + \frac{\partial^k}{\partial \mu_s^k}\right] \notag \\
    \frac{\partial^k}{\partial \mu_S^k} &= - \frac{\partial^k}{\partial \mu_s^k} \notag \\
    \frac{\partial^k}{\partial \mu_I^k} &= \frac{1}{2^k} \left[\frac{\partial^k}{\partial \mu_u^k} - \frac{\partial^k}{\partial \mu_d^k}\right]
\end{align}


%\vspace{1cm}


% Bibliography

%% [A] Recommended: using JHEP.bst file
 \bibliographystyle{JHEP}
 %\bibliography{main}
 % This must be in the first 5 lines to tell arXiv to use pdfLaTeX, which is strongly recommended.
\pdfoutput=1
% In particular, the hyperref package requires pdfLaTeX in order to break URLs across lines.

\documentclass[11pt]{article}

% Remove the "review" option to generate the final version.
%\usepackage[review]{ACL2023}
\usepackage{ACL2023}

% Standard package includes
\usepackage{times}
\usepackage{latexsym}

% For proper rendering and hyphenation of words containing Latin characters (including in bib files)
\usepackage[T1]{fontenc}
% For Vietnamese characters
% \usepackage[T5]{fontenc}
% See https://www.latex-project.org/help/documentation/encguide.pdf for other character sets

% This assumes your files are encoded as UTF8
\usepackage[utf8]{inputenc}

% This is not strictly necessary, and may be commented out.
% However, it will improve the layout of the manuscript,
% and will typically save some space.
\usepackage{microtype}

% This is also not strictly necessary, and may be commented out.
% However, it will improve the aesthetics of text in
% the typewriter font.
\usepackage{inconsolata}


% If the title and author information does not fit in the area allocated, uncomment the following
%
%\setlength\titlebox{10cm}
%
% and set <dim> to something 5cm or larger.

%%%%%%%%%%%%%%%%%%%%%%%%%%%%%%%%%%
\usepackage{graphicx}
\usepackage{amsfonts}
\usepackage{amsmath}
\usepackage{bigdelim}
\usepackage{diagbox}
\usepackage{amsthm}
\usepackage{makecell}
\usepackage{mathtools}
\usepackage{booktabs}
\usepackage[shortlabels]{enumitem}
\graphicspath{ {figs/} }

\theoremstyle{remark}
\newtheorem*{question}{Question}

\newcommand{\tk}[1]{\textcolor{blue}{{#1}}}
\newcommand{\sy}[1]{\textcolor{red}{{#1}}}
\newcommand{\mg}[1]{\textcolor{purple}{{#1}}}
\newcommand{\lh}[1]{\textcolor{green}{{#1}}}
\newcommand{\lc}[1]{\textcolor{green}{{#1}}}

% Rounded color box
\definecolor{light_blue}{HTML}{cfdfff}
\usepackage[most]{tcolorbox}
\tcbset{on line, 
        boxsep=1pt, left=0pt,right=0pt,top=0pt,bottom=0pt,
        colframe=white,colback=light_blue,  
        highlight math style={enhanced}
        }

\newcommand{\quash}[1]{}  %Anything in \quash is ignored
\newcommand{\gpt}{\textsc{GPT-2}}
\newcommand{\bert}{\textsc{BERT}}
\newcommand{\bertlarge}{\textsc{BERT-large}}
\newcommand{\mask}{\texttt{[MASK]}}
\newcommand{\cls}{\texttt{[CLS]}}
\newcommand{\sep}{\texttt{[SEP]}}
\newcommand{\mat}{\texttt{mat}}
\newcommand{\id}{\texttt{id}}
\newcommand{\matl}{\texttt{mat}_{\ell \rightarrow \ell'}}
\newcommand{\matattnl}{\texttt{mat\_attn}_{\ell \rightarrow \ell'}}
\newcommand{\matffl}{\texttt{mat\_ffn}_{\ell \rightarrow \ell'}}
\newcommand{\matlnl}{\texttt{mat\_ln1\_ln2}_{\ell \rightarrow \ell'}}
\newcommand{\idl}{\texttt{id}_{\ell \rightarrow \ell'}}
\newcommand{\matlL}{\texttt{mat}_{\ell \rightarrow L}}
\newcommand{\matattnlL}{\texttt{mat\_attn}_{\ell \rightarrow L}}
\newcommand{\matfflL}{\texttt{mat\_ffn}_{\ell \rightarrow L}}
\newcommand{\matlnlL}{\texttt{mat\_ln1\_ln2}_{\ell \rightarrow L}}
\newcommand{\idlL}{\texttt{id}_{\ell \rightarrow L}}

\definecolor{blue(munsell)}{rgb}{0.0, 0.5, 0.69}
%%%%%%%%%%%%%%%%%%%%%%%%%%%%%%%%%%

\title{Jump to Conclusions: Short-Cutting Transformers\\With Linear Transformations}

% Author information can be set in various styles:
% For several authors from the same institution:
% \author{Author 1 \and ... \and Author n \\
%         Address line \\ ... \\ Address line}
% if the names do not fit well on one line use
%         Author 1 \\ {\bf Author 2} \\ ... \\ {\bf Author n} \\
% For authors from different institutions:
% \author{Author 1 \\ Address line \\  ... \\ Address line
%         \And  ... \And
%         Author n \\ Address line \\ ... \\ Address line}
% To start a seperate ``row'' of authors use \AND, as in
% \author{Author 1 \\ Address line \\  ... \\ Address line
%         \AND
%         Author 2 \\ Address line \\ ... \\ Address line \And
%         Author 3 \\ Address line \\ ... \\ Address line}

\author{Alexander Yom Din$^{1}$ ~~~~~ Taelin Karidi$^{1}$ ~~~~~ Leshem Choshen$^{1}$ ~~~~~
Mor Geva$^{2}$ 
\vspace{0.2cm} \\
$^1$Hebrew University of Jerusalem ~~~ $^2$Google Research \\
\small{\texttt{\{alexander.yomdin, taelin.karidi, leshem.choshen\}@mail.huji.ac.il}}, \small{\texttt{pipek@google.com}}}

\quash{
\author{Alexander Yom Din \\
  Hebrew University of Jerusalem \\ \texttt{alexander.yomdin@mail.huji.ac.il} \\\And
  Taelin Karidi \\
  Hebrew University of Jerusalem \\
  \texttt{taelin.karidi@mail.huji.ac.il} \\\And
  Leshem Choshen \\
  Hebrew University of Jerusalem \\ \texttt{leshem.choshen@mail.huji.ac.il} \\\And
  Mor Geva \\
  Google Research \\
  \texttt{pipek@google.com} \\}
}

\begin{document}
\maketitle



\begin{abstract}
% \vspace{-1em}
The diffusion-based generative models have achieved remarkable success in text-based image generation. However, since it contains enormous randomness in generation progress, it is still challenging to apply such models for real-world visual content editing, especially in videos. 
In this paper, we propose \texttt{FateZero}, a zero-shot text-based editing method on real-world videos without per-prompt training or use-specific mask. 
\RM{Specifically, different from a pipeline of two independent inversion and then generation stages, we find the intermediate attention maps during inversions store better structure and motion information. We thus reform them to temporally casual attention and replace them in the generation progress. To further reduce the unnecessary semantic leakage of source video and enhance the editing quality, we then remix the temporally casual attentions via the cross-attention features of the source prompt as the mask.}
To edit videos consistently, we propose several techniques based on the pre-trained models. Firstly, in contrast to the straightforward DDIM inversion technique, our approach captures intermediate attention maps during inversion, which effectively retain both structural and motion information. These maps are directly fused in the editing process rather than generated during denoising. To further minimize semantic leakage of the source video, we then fuse self-attentions with a blending mask obtained by cross-attention features from the source prompt. Furthermore, we have implemented a reform of the self-attention mechanism in denoising UNet by introducing spatial-temporal attention to ensure frame consistency.
Yet succinct, our method is the first one to show the ability of zero-shot text-driven video style and local attribute editing from the trained text-to-image model. We also have a better zero-shot shape-aware editing ability based on the text-to-video model~\cite{tuneavideo}. \RM{Besides video, our unified method also achieves state-of-the-art performance in zero-shot image editing.\chenyang{Need exp or remove the zero-shot image}} Extensive experiments demonstrate our superior temporal consistency and editing capability than previous works.
% The code will be released.
% \chenyang{emphasize: our observation at inversion time} \xiaodong{replacing the bold part to the actual pipeline: \textbf{Specifically, we work on replacing and mixing the attention maps between the inversion and generation since the self-attention map keeps the structure of the original natural image and the cross-attention is semantic-related, after remixing, we replace them in the corresponding generation steps for denoising.}}
% \footnote{Since there is no general video diffusion model is publicly available, we use one-shot video generation method~(Tune-A-Video~\cite{tuneavideo}) as the pretrained video diffusion model for zero-shot video editing\xiaodong{can be removed if we actually zero-shot on video}.}.
\end{abstract}
\section{Introduction}

The ability to reason about plans is critical for performing long-horizon tasks \citep{erol1996hierarchical, sohn2018hierarchical, sharma-etal-2022-skill}, compositional generalization \citep{corona-etal-2021-modular} and generalization to unseen tasks and environments \citep{shridhar2020alfred}.
Consider a simple long-horizon planning scenario where a robot is tasked with preparing a meal and serving it on the table. 
This presents a non-trivial planning problem since the agent needs to understand the sequence of operations required to perform the task and search for the relevant objects in the unfamiliar environment by interacting with various objects. %



Large language models have been recently shown to possess commonsense knowledge about the world such as object affordances and physical dynamics \citep{ouyang2022training,chowdhery2022palm}.
Early approaches considered text based environments and fine-tuned PLMs to predict actions given the history of past observations and actions \citep{jansen-2020-visually,micheli-fleuret-2021-language,yao-etal-2020-keep}.
Recent work has used this ability to reason about plans from text instructions in simulated household environments with simplifying assumptions such as text-only environment observations or feedback \citep{huang2022language,ahn2022can,li2022pre,logeswaran-etal-2022-shot}.


We focus on \emph{visually grounded planning} with PLMs --- the ability to adapt plans based on interaction and visual feedback from the environment.
While PLMs have strong planning commonsense priors, predictions from a PLM may not be directly realizable in the environment since the observation and action spaces are unknown.
This requires \emph{grounding} the PLM in the environment and adapting it to observe visual feedback, which is highly non-trivial.
Some prior works assume the availability of a pre-trained affordance function \citep{ahn2022can} or a success detector \citep{mirchandani2021ella}.
Notably, SayCan \citep{ahn2022can} completely decouples the PLM from observation information by selecting actions that have both high affordability (through a pre-trained affordance model) and high PLM likelihood.
Although this partially addresses the grounding problem, the use of visual feedback for action affordance alone is limited.
Often an agent must choose one of many affordable actions using information from observations.
For example, a driving agent should re-navigate and possibly turn around when encountering a ``road closed'' sign, but both turning around and driving forward are indistinguishable to SayCan because they are both affordable and the PLM is blind to observations.

Another workaround explored in prior work is translating the information in the visual observations to text using a pre-trained captioning system \citep{shridhar2021alfworld,huang2022language}.
However, it can be difficult to faithfully describe an image in words and information is lost in this inherently noisy process, which limits the information available to the planner.



Recent work shows that PLMs can be adapted for various natural language tasks by inserting tunable embeddings or soft prompts at the input of the PLM (also called prompt tuning or prefix tuning)~\citep{li-liang-2021-prefix,lester-etal-2021-power}.
This approach also extends to multi-modal understanding tasks such as image captioning \citep{mokady2021clipcap} and VQA \citep{tsimpoukelli2021multimodal} where images are encoded as soft prompts and finetuned for the target task.
Transformer based architectures have also been successfully applied to offline Reinforcement Learning in recent work \citep{chen2021decision,janner2021offline,li2022pre,reid2022can}.

Taking inspiration from these works, we propose the simple approach of embedding visual observations (`visual prompts') and \textit{directly inserting them as PLM input embeddings}.
The visual encoder and PLM are jointly trained for the target task, an approach we call \textbf{\oursfull}~(\ours).
By teaching the PLM to use observations for planning in an end to end manner, we remove the dependency on external data such as captions and affordability information that was used in prior work.
We show that this simple approach performs better than prior PLM-based planning approaches on two embodied planning benchmarks based on ALFWorld~\citep{shridhar2021alfworld} and Virtualhome~\cite{puig2018virtualhome}.



\section{Related Work}

%Here we summarize prior work on transfer learning and property inference.

%\shortsection{Transfer Learning}
%%Transfer learning reuses features learned by pre-trained models for new tasks, with the pretext that inherent similarities in the generic features will be useful for the downstream tasks and hence reducing their cost of downstream training. Specifically, the downstream model trainer will use a pre-trained upstream model as the starting point for the downstream training, with inclusion of (or replacement with) the task-specific classification layer/module. The downstream model is then trained by either updating all layers of the model (including ones reused from upstream model) or freezing some earlier layers of the reused parts as the ``feature extractor'' and only updating the rest. The latter approach is more popular as the reused feature extractors can already learn useful feature representations and the training cost is also much lower and affordable for individuals with limited computational resources. We study the vulnerability of the latter transfer learning approach in this paper. 


%\shortsection{Transfer Learning} 
Several works have demonstrated risks associated with transfer learning across a variety of attack goals. Wang et al.~\cite{wang2018great} and Yao et al.~\cite{yao2019latent} consider manipulating the upstream model such that the fine-tuned downstream models contain backdoors, misclassifying test inputs that contain predefined backdoor triggers. These transfer manipulations are tailored to their particular attack goals and cannot be applied for the property inference goal considered in this paper. Zou et al.~\cite{zou2020privacy} study the threat of membership inference attacks on transfer learning, but with normally trained upstream models.  
%\dnote{its clear that the goals are different for these attacks, but how similar are the methods?} \ynote{similarity of the methods? more details about the methods? do not know what is expected here}
%In contrast, we investigate the possibility of boosting the effectiveness of property inference by manipulating the upstream model training. % Schuster et al.~\cite{schuster2020humpty} show that the attacker can modify the corpus on which the word embedding is trained such that the downstream NLP models which use that embedding will behave abnormally.

%\shortsection{Property Inference}
The risk of property inference was introduced by Ateniese et al.~\cite{ateniese2015hacking}, % introduces the threat of inferring properties of the training data from pre-trained models, 
and several subsequent works have developed property inference (also known as distribution inference) attacks~\cite{Wang2022GroupPI, suri2022formalizing, Jurez2022BlackBoxAF, Hartmann2022DistributionIR}.
% Ganju et al.~\cite{ganju2018property} and Suri and Evans~\cite{suri2022formalizing} 
These works study property inference against normally trained models, and they launch attacks using a variety of black-box and white-box attacks. All the white-box attacks use meta-classifiers, which take the permutation-invariant representation~\cite{ganju2018property} of the model parameters as the features. We use the state-of-the-art white-box attack~\cite{suri2022formalizing} in our experiments.
%We will use the state-of-the-art white-box method proposed by Ganju et al.~\cite{ganju2018property} and later extended by suri et al.~\cite{suri2022formalizing} in this paper.
%\dnote{do we use these attacks?} 
Melis et al.~\cite{melis2019exploiting} and Zhang et al.~\cite{zhang2021leakage} focus on property inference in distributed training scenarios. In their settings, the attacker is a participant in the global model training and conducts property inference using meta-classifiers that are trained on model outputs or gradients. Similarly, Suri et al.~\cite{suri2022subject} focus on federated learning settings where the attacker is a participant (or the central server) that utilizes black-box attacks for inferring membership of data from particular subjects. %\dnote{if we use black-box attacks, explain which ones, or how ours are related to previous ones} 
For our experiments, We improve the black-box meta-classifier proposed by Zhang et al.~\cite{zhang2021leakage} using the ``query tuning'' technique in Xu et al.~\cite{xu2019detecting}. 

The closest works to ours are Chase et al.~\cite{saeed} and Chaudhari et al.~\cite{Chaudhari2022SNAPEE}, which both consider a scenario where the attacker can manipulate some of the training data of the model to induce a model that significantly increases property inference risk.
% \dnote{it enables precise property inference attacks?}.
These works assume an adversary with the ability to poison the victim's training data, while the adversary in our scenario has no access to the victim's training data, and therefore, their methods are not applicable.
% \dnote{example how different from ours, and why the methods are not applicable}
%Thus, their methods are not applicable to our transfer learning scenario.
%Their methods rely on inducing certain behavior correlated with the properties to be inferred, and thus are not applicable to our transfer learning scenario. \anote{Still a bit unclear why that is the case.}
%
There are also works similar to ours that leverage ``adversarial initializations'' for attack purposes.
% \cite{grosse2019adversarial, boenisch2021curious, wen2022fishing, fowl2021robbing}.
Grosse et al.~\cite{grosse2019adversarial} focus on scenarios where the attacker can control the parameter initialization of a model, and demonstrate that the attacker can use special initializations to damage the performance of the trained model. %This attack is orthogonal to ours.
Other works \cite{boenisch2021curious, wen2022fishing, fowl2021robbing} show that the malicious central server in a federated learning protocol can reconstruct some training samples via falsifying the global model in some training rounds and then analyzing the submitted gradients. These kinds of attacks do not apply to our transfer-learning scenario since the attacker cannot access the downstream gradients, and can only manipulate the upstream training.

\iffalse %%%%%%%%%%%%%%%%%%%%%%%%%%%%%%%%

In this section, we provide the background and also the summary of prior attacks on transfer learning (Section~\ref{sec:transfer_learning}) and property inference (Section~\ref{sec:property_inference}). Then, we introduce the closely related manipulation attacks against machine learning models to boost different privacy risks in Section~\ref{sec:active_inference_attacks}.

%\anote{Do we really need a dedicated section for this? It's barely 2 paragraphs right now.}

%\dnote{the most closely related work to ours are works that attempt to amplify inference attacks by poisoning models, the two most relevant I know of are \url{https://www.computer.org/csdl/proceedings-article/sp/2022/131600b569/1CIO8nmuota} and \url{https://arxiv.org/abs/2204.00032}, but need to look thoroughly for others. We should definitely be describing this and relating it to our work, probably in the introduction. Most of what is here is Background, but should be clear what this section is for (not muddling background and related work)}

\subsection{Transfer Learning} \label{sec:transfer_learning}
Transfer learning reuses features learned by pre-trained models for new tasks, with the pretext that inherent similarities in generic features can be useful for downstream tasks, thus reducing the cost of downstream training. Specifically, the downstream model trainer uses a pre-trained upstream model as the starting point for downstream training, with the inclusion (or replacement) of task-specific classification layers/modules. The downstream model is then trained by either updating all layers of the model (including ones reused from the upstream model) or freezing some earlier layers of the reused parts as the ``feature extractor'' and only updating the rest. The latter approach is more popular as the reused feature extractors can already learn useful feature representations and the training cost is also much lower and affordable for individuals with limited computational resources. We study the vulnerability of the latter transfer learning approach in this paper. 
%mainly in two ways:  1) all the layers (including ones reused from ) and tune the full model; the other one is to freeze some earlier layers of the model as the feature extractor and only tune the rest later layers. The second update strategy could achieve better efficiency since the frozen layers can already produce meaningful feature representations~\cite{wang2018great,yao2019latent}, and we will study the transfer learning using this strategy. 

Recently, various attacks have been proposed for the transfer learning setting, but with different attack goals from ours. Wang et al.~\cite{wang2018great} generate adversarial examples against black-box student models that transfer knowledge from publicly available teacher models without repeated queries. Yao et al.~\cite{yao2019latent} propose to manipulate the upstream model such that the downstream models derived from the upstream model contain backdoors, which would misclassify test inputs that contain some predefined backdoor triggers. Zou et al.~\cite{zou2020privacy} study the threat of membership inference attacks on transfer learning and the upstream models are trained normally. In contrast, we investigate the possibility of boosting the effectiveness of property inference by manipulating the upstream model training. Schuster et al.~\cite{schuster2020humpty} show that the attacker can modify the corpus on which the word embedding is trained such that the downstream NLP models which use that embedding will behave abnormally.

%This additionally allows model trainers to achieve satisfactory performance with limited training samples, leading to reduced computational costs. The most common approach reuses parameters in the earlier layers of the pre-trained model, either by fixing them as the feature extractor or just using them for initialization, to conduct downstream training.

\subsection{Property Inference} \label{sec:property_inference}

\shortsection{Property Inference Attacks} In property inference attacks, the adversary aims to infer some sensitive properties of some data, given a model trained on it. For example, the adversary may be interested in sensitive properties like the presence of people of a specific race in the dataset~\cite{ateniese2015hacking, melis2019exploiting}), or even be curious about the 
the statistics of the training set (e.g, the ratio of people with a specific gender~\cite{saeed, ganju2018property, suri2022formalizing, zhang2021leakage}).


Ateniese et al.~\cite{ateniese2015hacking} were the first to identify the threat of inferring properties of the training data from pre-trained models. Ganju et al.~\cite{ganju2018property} and Suri and Evans~\cite{suri2022formalizing} 
study property inference against normally trained models, and they launch attacks using white-box meta-classifiers, which utilize the permutation-invariance representation~\cite{ganju2018property} of the model parameters, while other works focus on distributed training~\cite{zhang2021leakage} where the attacker is a participant in the global model training and conducts property inference using meta-classifiers trained on model outputs. Similarly, Suri et al.~\cite{suri2022subject} focus on federated learning, where the attacker is a participant (or the central server) that utilizes black-box attacks for inferring membership of data from particular subjects. Chase et al.~\cite{saeed} propose an active property inference attack for data poisoning scenarios, which we will cover and compare to in Section~\ref{sec:active_inference_attacks}.

%The closest work to ours are by Chase et al.~\cite{saeed} and Tramer et al.~\cite{tramer2022truth}. In their work, the attacker can manipulate some of the training data of the model such that a model trained (from scratch) on the poisoned data has an increased inference risk. However, their methods are not applicable to the transfer learning scenario. 
%In this work, we will focus on the property inference in transfer learning scenarios in which the attacker releases the upstream model and infer sensitive properties of the downstream models tuned from that upstream model.
% 

\shortsection{Defenses}
Defending against property inference attacks is an open problem. There are no studies in the current literature on active adversaries, and only a couple on passive ones. Ma et. al.~\cite{ma2021nosnoop} propose a defense against property inference attacks on data batches in the  collaborative learning setting. However, adversaries in the transfer-learning setting do not have access to batch-wise gradients of the downstream trainer. Chen and Ohrimenko~\cite{chen2022protecting} utilize mechanisms that add carefully-crafted noise to features to provide theoretical guarantees against inference adversaries, but focus on query-based access to the underlying dataset, not a machine learning model trained on it. These existing defenses thus do not apply to our threat model.

%propose a framework that reduces property inference to Boolean functions of individual members, posing the ratio of members satisfying the given function in a dataset as the property. These property inference attacks have since then been proposed as distribution inference attacks~\cite{suri2022formalizing}, presenting such attacks as inferring properties of the distributions used to sample datasets, differentiating them from exact inference attacks like dataset inference~\cite{maini2021dataset}. Nearly all property inference attacks use meta-classifiers to perform inference: training models on versions of datasets with and without the target property, followed by training a meta-classifier on top of these classifiers's model representations. These representations can take several forms: using model weights themselves with permutation-invariance~\cite{ganju2018property}, or model activations or logits for a generated set of query points~\cite{xu2019detecting}. However, the capability of such approaches is limited: the most that these attacks have been shown to work is medium-sized convolutional networks on the CelebA dataset~\cite{suri2022formalizing}.


\subsection{Active Privacy Attacks} \label{sec:active_inference_attacks}
% Perhaps the closely related works to ours as ones that proactively enhance the effectiveness of privacy attacks by manipulating the model training process in certain ways~\cite{saeed, melis2019exploiting, nasr2019comprehensive, tramer2022truth}. 
%shown that the adversary can, by using proactive ways, achieve stronger attacks that infer private information from deep learning systems~\cite{nasr2019comprehensive, melis2019exploiting, tramer2022truth, saeed}. In this section, we introduce the ones that are close to ours.

In the decentralized federated learning training, by submitting specially crafted gradients to the central server, malicious agents can increase membership inference risk~\cite{nasr2019comprehensive} and property inference risks~\cite{melis2019exploiting} of other benign agents' training data. However, these attacks do not apply to transfer learning scenario, as the attacker cannot control model gradients of downstream training. In the centralized setting, researchers propose attacks to poison the victim's training data such that the impacts of attribute inference and membership inference~\cite{tramer2022truth} and property inference~\cite{saeed} attacks are amplified on the poisoned model.
The ability to poison the victim's data is a threat model orthogonal to ours, since we have no access to the victim's downstream data. While there is scope to combine such approaches for stronger attacks (albeit with stronger access assumptions), we choose to focus on the scenario with no read/write access to the victim's data.

\fi %%%%%%%%%%%%%%%%%%%%%%%%%%%%%%%%

\section{Linear Shortcut Across Blocks}
\label{sec:layer_jump}

To use a hidden representation from layer $\ell<L$ as a final representation, we propose to cast it using linear regression, while skipping the computation in-between these layers. More generally, this approach can be applied to cast any $\ell$-th hidden representation to any subsequent layer $\ell'>\ell$.


\subsection{Method}
\label{subsec:methodology_linear_shortcut}

Given a source layer $\ell$ and a target layer $\ell'$ such that $0 \leq \ell < \ell' \leq L$, our goal is to learn a mapping
%$A_{\ell', \ell} \in \mathbb{R}^{d_h \times d_h}$
from hidden representations at layer $\ell$ to those at layer $\ell'$. To this end, we first collect a set of corresponding hidden representation pairs $(h^\ell, h^{\ell'})$. Concretely, we run a set $\mathcal{T}$ of input sequences through the model, and for each input $s$, we extract the hidden representations $h_{i_s}^{\ell}, h_{i_s}^{\ell'}$, where $i_s$ is a random position in $s$.
Next, we learn a matrix $A_{\ell', \ell} \in \mathbb{R}^{d_h \times d_h}$ by fitting linear regression over $\mathcal{T}$, i.e., $A_{\ell', \ell}$ is a numerical minimizer for:
$$ A \mapsto \sum_{s \in \mathcal{T}} || A \cdot h_{i_s}^\ell - h_{i_s}^{\ell'} ||^2,$$ 
and define the mapping of a representation $h$ from layer $\ell$ to layer $\ell'$ as:
\begin{equation}
\label{eq:linear_jump}
    \matl{} (h) \coloneqq A_{\ell', \ell} \cdot h.
\end{equation}


\subsection{Baseline}
\label{subsec:baseline}

We evaluate 
% our method against 
the prevalent approach of ``reading'' hidden representations directly, without any transformation. 
Namely, the propagation of a hidden representation from layer $\ell$ to layer $\ell'$ is given by the identity function, dubbed \id{}:

$$ \idl{} (h) \coloneqq h.$$

% Notably, 
This baseline 
assumes that representations at different layers operate in the same linear space.

\subsection{Quality of Fit}
\label{subsec:experiments_r2}

We first evaluate our method by measuring how well the learned linear mappings approximate the representations at the target layer. To this end, we calculate the (coordinate-averaged) $r^2$-score of our mapping's outputs with respect to the representations obtained from a full inference pass, and compare to the same for the \id{} baseline.


\paragraph{Models.}

We use \gpt{} \cite{radford2019language}, a decoder-only auto-regressive LM, with $L = 48$, $d_h = 1600$, and \bert{} \cite{devlin-etal-2019-bert}, an encoder-only model trained with masked language modeling, with $L=24$, $d_h=1024$.
% \footnote{\label{footnote:hf}We use models and data from Huggingface \cite{wolf-etal-2020-transformers,lhoest-etal-2021-datasets}.}
%For masked token prediction, we use a masked LM head pre-trained for our \bert{} model.

% \footnote{Specifically, we use the Huggingface Transformers \cite{wolf-etal-2020-transformers} implementations of all these models.}

%\sy{We use \gpt{} \cite{radford2019language}, a decoder-only auto-regressive LM, coming in four scales; $\texttt{gpt2}$ ($L = 12$, $d_h = 768$), $\texttt{gpt2-medium}$ ($L = 24$, $d_h = 1024$), $\texttt{gpt2-large}$ ($L = 36$, $d_h = 1280$) and $\texttt{gpt2-xl}$ ($L = 48$, $d_h = 1600$). Also, we use \bert{} \cite{devlin-etal-2019-bert}, an encoder-only model trained with masked language modeling, coming in two scales;  \texttt{bert-base-uncased} ($L=12$, $d_h=768$) and \texttt{bert-large-uncased} ($L=24$, $d_h=1024$). For masked token prediction, we use masked LM heads pre-trained for our models. Specifically, we use the Huggingface Transformers \cite{wolf-etal-2020-transformers} implementations of all these models. The plots presented in this section are for $48$-layered \gpt{} and $24$-layered \bert{}.}

%\sy{We use \gpt{} \cite{radford2019language}, a decoder-only auto-regressive LM, in the Huggingface \cite{wolf-etal-2020-transformers} implementation\footnote{\url{https://huggingface.co/gpt2}}, coming in four scales; $\texttt{gpt2}$ ($L = 12$, $d_h = 768$), $\texttt{gpt2-medium}$ ($L = 24$, $d_h = 1024$), $\texttt{gpt2-large}$ ($L = 36$, $d_h = 1280$) and $\texttt{gpt2-xl}$ ($L = 48$, $d_h = 1600$). Also, we use \bert{} \cite{devlin-etal-2019-bert}, an encoder-only model trained with masked language modeling, in the Hugginface implementation, coming in two scales;  \texttt{bert-base-uncased}\footnote{\url{https://huggingface.co/bert-base-uncased}} ($L=12$, $d_h=768$) and \texttt{bert-large-uncased}\footnote{\url{https://huggingface.co/bert-large-uncased}} ($L=24$, $d_h=1024$). For masked token prediction, we use the \texttt{BertForMaskedLM} heads from Huggingface, pretrained for these models. The plots presented in this section are for $48$-layered \gpt{} and $24$-layered \bert{}.}

\paragraph{Data.}
We sample random sentences from Wikipedia,
% \footref{footnote:hf} 
collecting 9,000 (resp. 3,000) sentences for the training set $\mathcal{T}$ (resp. validation set $\mathcal{V}$).\footnote{We use sentences rather than full documents to simplify the analysis.}
%\sy{We use two data sources to evaluate our method. One is Wikiepdia \cite{lhoest-etal-2021-datasets}\footnote{\url{https://huggingface.co/datasets/wikipedia}}; we use \texttt{spaCy}\footnote{\url{https://spacy.io/}} to divide documents into sentences\footnote{We use sentences rather than full documents to simplify the analysis.}\footnote{We pick randomly a Wikipedia document and then pick randomly a sentence ending in a newline character in it. \sy{[maybe this footnote is not needed?]}}, collecting 9,000 (resp. 3,000) random sentences for the training set $\mathcal{T}$ (resp. validation set $\mathcal{V}$). The second is a news article sentences dataset, the 10K English 2020 news sentences corpus
% \footnote{\url{https://downloads.wortschatz-leipzig.de/corpora/eng_news_2020_10K.tar.gz}} from the Leipzig Corpora Collection \cite{goldhahn-etal-2012-building}, which we randomly divide into a training set $\mathcal{T}$ consisting of 9,000 examples and a validation set $\mathcal{V}$ consisting of 1,000 examples.
% We truncate sentences to the maximal token length allowed by the model \mg{do we ever need to truncate? a sentence has about 10 words and the max. input len is thousands} \sy{[I surely did not need to in Leipzig, but discovered (via a transformers runtime warning) that I do need to for some (probably a minority) of the Wikipedia sentences. This probably has to do with that it is not really ``sentences" necessarily, for example, I noticed that it has some listings or something like that (bulleted items)... So some minority might get very long I guess...]}.
For each example $s$, we select a random position $i_s$ and extract the hidden representations $h_{i_s}^{\ell}$ at that position from all the layers.
For \bert{}, we first replace the input token at position $i_s$ with a \mask{} token, as our motivation is interpreting predictions, which are obtained via masked tokens in \bert{} (see \S\ref{subsec:BERT}).
Thus, in this case, the hidden representations we consider
%in the case of \bert{}
are of \mask{} tokens only.
%As we observed highly similar results for the two data sources across all our experiments, throughout the paper we will mainly report results for Wikipedia (except for \S\ref{sec:robustness}, where we cross-validate).


\begin{figure}[t]
\includegraphics[scale=0.2]{figs/r2_scores_48.pdf}
% \includegraphics[width=\columnwidth]{figs/r2_scores_48.pdf}
\caption{The coordinate-averaged $r^2$-score of $\matl{}$ (left) and $\idl{}$ (right) (\gpt{}).}
\label{fig:r2_scores}
\end{figure}


\begin{figure}[t]
\setlength{\belowcaptionskip}{-10pt}
\includegraphics[scale=0.2]{figs/bertmask_r2_scores_24.pdf}
% \includegraphics[width=\columnwidth]{figs/bertmask_r2_scores_24.pdf}
\caption{The coordinate-averaged $r^2$-score of $\matl{}$ (left) and $\idl{}$ (right) (\bert{}).}
\label{fig:bertmask_r2_scores}
\end{figure}



\paragraph{Evaluation.}
For every pair of layers $\ell, \ell'$, such that $0 \leq \ell < \ell' \leq L$, we use the training set $\mathcal{T}$ to fit linear regression as described in \S\ref{subsec:methodology_linear_shortcut}, and obtain a mapping $\matl{}$. 
Next, we evaluate the quality of $\matl{}$ as well as of $\idl{}$ using the $r^2$-coefficient, uniformly averaged over all coordinates. Concretely, we compute the $r^2$-coefficient of each of the predicted representations $\matl{} (h_{i_s}^{\ell})$ and $\idl{} (h_{i_s}^{\ell})$ versus the true representations $h_{i_s}^{\ell'}$
over all $s \in \mathcal{V}$.
%as we vary $s \in \mathcal{V}$.
%for every $s \in \mathcal{V}$.



\paragraph{Results.}
Results for \gpt{} and \bert{} are presented in Figs.~\ref{fig:r2_scores} and~\ref{fig:bertmask_r2_scores}, respectively.
In both models, \mat{} consistently yields better approximations than \id{}, as it obtains higher $r^2$-scores (in blue) across the network. 
This gap between \mat{} and \id{} is especially evident in \bert{}, where \id{} completely fails to map the representations between most layers, suggesting that hidden representations are modified  substantially by every transformer block.
Overall, this highlights the shortcoming of existing practices to inspect representations in the same linear space, and the gains from using our method to approximate future layers.
% in the network.
\section{Linear Shortcut for Language Modeling}
\label{sec:prediction}

We saw that our method approximates future hidden representations substantially better than a naive propagation. 
In this section, we will show that this improvement also translates to better predictive abilities from earlier layers. Specifically, we will use our method to estimate how often intermediate representations encode the final prediction, in the context of two fundamental LM tasks; next token prediction and masked token prediction.

\paragraph{Evaluation Metrics.}
Let $h, h' \in \mathbb{R}^{d_h}$ be a final representation and a substitute final representation obtained by some mapping, and denote by $\delta (h), \delta (h') \in \mathbb{R}^{d_v}$ their corresponding output probability distributions (obtained through projection to the output vocabulary -- see details below). 
We measure the prediction quality of $h'$ with respect to $h$ using two metrics:
\begin{itemize}
[leftmargin=*,topsep=1pt,parsep=1pt]
    \item \textbf{Precision@$k$} ($\uparrow$ is better): This checks whether the token with the highest probability according to $\delta(h')$ appears in the top-$k$ tokens according to $\delta(h)$. Namely, we sort $\delta(h)$ and assign a score of $1$ if $\arg\max(\delta(h'))$ appears in the top-$k$ tokens by $\delta(h)$, and $0$ otherwise.
    
    \item \textbf{Surprisal} ($\downarrow$ is better): We measure the minus log-probability according to $\delta(h)$, of the highest-probability token according to $\delta(h')$. Intuitively, low values mean that the model sees the substitute result as probable and hence not surprising.
\end{itemize}

\noindent We report the average Precision@$k$ and Surprisal over the validation set $\mathcal{V}$.



\subsection{Next Token Prediction}
\label{subsec:next_token_prediction_task}

Auto-regressive LMs output for every position a probability distribution over the vocabulary for the next token. Specifically, the output distribution for every position $i$ is given by $\delta (h_i^L)$, where:
\begin{equation}\label{eq:output_distribution}
    \delta (h) = \texttt{softmax} ( E^\top \cdot h) \in \mathbb{R}^{d_v}
\end{equation}
For some LMs, including \gpt{}, a layer normalization $\texttt{ln\_f}$ is applied to the final layer representation before this conversion (i.e., computing $\delta (\texttt{ln\_f}(h))$ rather than $\delta (h)$).

Recall that our goal is to measure how well this distribution can be estimated from intermediate representations, i.e. estimating $\delta (h_i^L)$ from $\delta (h_i^\ell)$ where $\ell<L$. To this end, we first run examples from the validation set through the model, while extracting for each example $s$ the hidden representation of a random position $i_s$ at every layer. Next, we apply our mappings $\matlL{}$ and the $\idlL{}$ baseline to cast the hidden representations of every layer $\ell$ to final layer substitutes (see \S\ref{sec:layer_jump}). Last, for each layer, we convert its corresponding final-layer substitute to an output distribution (Eq.~\ref{eq:output_distribution}) and compute the average Precision@$k$ (for $k=1,5,10$) and Surprisal scores with respect to the final output distribution, over the validation set.

\paragraph{Results.}
Figs.~\ref{fig:pre} and~\ref{fig:surp} show the average Precision@$k$ and Surprisal scores per layer in $48$-layered \gpt{}, respectively (the plots for the other \gpt{} models are presented in \S\ref{sec:app_scale}). Across all layers, \mat{} outperforms \id{} in terms of both scores, often by a large margin (e.g. till layer $44$ the Precision@$1$ achieved by \mat{} is bigger than that of $\id{}$ by more than $0.2$). 
This shows that linear mappings enable not just better estimation of final layer representations, but also of the predictions they induce. Moreover, the relatively high Precision@$k$ scores of \mat{} in early layers ($0.62$-$0.82$ for $k=10$, $0.52$-$0.74$ for $k=5$, and $0.28$-$0.45$ for $k=1$) suggest that early representations already encode a good estimation of the final prediction. Also, the substantially lower Surprisal scores of \mat{} compared to \id{} imply that our method allows for a more representative reading into the layer-wise prediction-formation of the model than allowed through direct projection to the vocabulary.

\begin{figure}[t]
\centering
\includegraphics[scale=0.4]{figs/pre_48.pdf}
\caption{Precision@$k$ ($k = 1,5, 10$) of $\matlL{}$ and $\idlL{}$ for next token prediction in $48$-layered \gpt{}.}
\label{fig:pre}
\end{figure}

\begin{figure}[t]
\centering
\includegraphics[scale=0.35]{figs/surp_48.pdf}
\caption{Surprisal for $\matlL$ and the baseline $\idlL{}$ ($48$-layered \gpt{} next token prediction task). A 95\% confidence interval surrounds the lines.}
\label{fig:surp}
\end{figure}

\subsection{Masked Token Prediction}
\label{subsec:BERT}

We now conduct the same experiment for the task of masked language modeling, where the model predicts a probability distribution of a masked token in the input rather than the token that follows the input. Unlike next token prediction, where the output distribution is computed from representations of varying input tokens, in masked token prediction the output is always obtained from representations of the same input token (i.e. \texttt{[MASK]}).

For this experiment, we use \bert{}, on top of which we use a pretrained masked language model head $\delta$; given a token sequence $s$, a \mask{} token inside it and its final representation $h$, $\delta (h) \in \mathbb{R}^{d_v}$
 is a probability distribution over tokens giving the model's assessment
 of the likelihood of tokens to be fitting in place of the \mask{} token in $s$.


\begin{figure}[t]
\centering
\includegraphics[scale=0.4]{figs/bertmask_pre_24.pdf}
\caption{Precision@$k$ ($k = 1,5, 10$) for  $\matlL{}$ and the baseline $\idlL{}$ ($24$-layered \bert{} masked token prediction task).}
\label{fig:bertmask_pre}
\end{figure}

\begin{figure}[t]
\centering
\includegraphics[scale=0.35]{figs/bertmask_surp_24.pdf}
\caption{Surprisal for $\matlL{}$ and the baseline $\idlL{}$ ($24$-layered \bert{} masked token prediction task). A 95\% confidence interval surrounds the lines.}
\label{fig:bertmask_surp}
\end{figure}

\paragraph{Results.}
Figs.~\ref{fig:bertmask_pre} and~\ref{fig:bertmask_surp} present the average Precision@$k$ and Surprisal scores per layer in $24$-layered \bert{} (the plots for the $12$-layered \bert{} model are presented in \S\ref{sec:app_scale}), overall showing trends similar to those observed for next token prediction in \gpt{} (\S\ref{subsec:next_token_prediction_task}). This is despite the differences between the two tasks and the considerable architectural differences between \bert{} and \gpt{}.
Notably, the superiority of \mat{} over \id{} in this setting is even more prominent; 
while \mat{}'s precision is between $0.2-0.6$ in the first ten layers (Fig.~\ref{fig:bertmask_pre}), \id{}'s precision for all values of $k$ is close to zero, again strongly indicating that our method allows for better reading into early layer hidden representations. 
More generally, \mat{} improves the Precision@$1$ of \id{} by more than $17\%$ at most layers, and unveils that a substantial amount of predictions ($>25\%$ starting from layer $3$) appear already in the very first layers.
Interestingly, the (rough) divide between the first half of layers and last half of layers for $\id{}$ in Figs.~\ref{fig:bertmask_pre},~\ref{fig:bertmask_surp} seems to align with the two-hump shape of the blue region for $\mat{}$ in Fig.~\ref{fig:bertmask_r2_scores}.

\paragraph{Analysis.}
We manually compare the predictions of our mapping $\matlL{}$ with $\idlL{}$, for a $24$-layered \bert{} model.  Concretely, we select 50 random sentences from the Leipzig dataset. Next, for each layer $\ell$, we manually analyze how many of the top-$5$ tokens according to $\matlL{}$ and $\idlL{}$ fit into context. We consider a token to fit into context if it is grammatically plausible within the sentence (see Tab.~\ref{tab:manual} for concrete examples).
In the resulting $1250$ instances (i.e. $50$ sentences $\times$ $25$ representations), we observe a substantially higher plausibility rate of $85.36\%$ for \mat{} compared to $52.8\%$ for \id{}. In fact, only in less than $4.3\%$ of the instances there are more plausible tokens among the top-$5$ tokens according to \id{} than among the top-$5$ tokens according to \mat{}, further supporting the Surprisal results above.

\begin{table*}
\footnotesize
\setlength{\belowcaptionskip}{-15pt}
\begin{tabular}{p{0.3\linewidth}ccccc}
& $\texttt{id}_{4 \rightarrow 24}$ & $\texttt{mat}_{4 \rightarrow 24}$ & $\texttt{id}_{12 \rightarrow 24}$ & $\texttt{mat}_{12 \rightarrow 24}$ & $\texttt{id}_{24 \rightarrow 24}$ \\ \midrule
\multirow{5}{=}{aldridge had shoulder surgery in \mask{}.} & fellowship & \tcbox{time} & cyclist & \tcbox{2009} & \tcbox{september} \\
& employment & \tcbox{it} & emergencies & \tcbox{2008} & \tcbox{november} \\
& agreement & her & seniors & \tcbox{2010} & \tcbox{december} \\
& \#\#ostal & them & cycling & \tcbox{2006} & \tcbox{august} \\
& \#\#com & work & \tcbox{pennsylvania} & \tcbox{2007} & \tcbox{july} \\ \midrule
\multirow{5}{=}{on your next view you will be asked to \mask{} continue reading.} & \#\#com & be & be & be & \tcbox{please} \\
& accreditation & get & undergo & \tcbox{please} & \tcbox{simply} \\ 
& $	\copyright$ & go & spartans & help & \tcbox{also} \\ 
& fellowship & \tcbox{help} & seniors & \tcbox{simply} & \tcbox{again} \\ 
& summer & have & * & say & \tcbox{immediately} \\ \bottomrule
\end{tabular}
\caption{Examples of top-$5$ predictions at layers $4$, $12$ and $24$, under the mappings $\matlL{}$ and $\idlL{}$, for a $24$-layered \bert{} model. Grammatically plausible predictions (according to a human annotator) are marked in \tcbox{blue}. Note that at layer $24$ the predictions of $\matlL{}$ and $\idlL{}$ are the same (by definition).} 
\label{tab:manual}
\end{table*}

\section{Implication to Early Exiting}
\label{sec:applications}

%The fact that it is often possible to approximate
The possibility of approximating
the final prediction already in the early layers has important implications for efficiency; applying our linear mapping instead of executing transformer blocks of quadratic time complexity, could save a substantial portion of the computation. In this section, we demonstrate this in the context of early exiting.

When 
% performing transformer model inference under 
using an early exit strategy \cite{schwartz-etal-2020-right, xin-etal-2020-deebert, schuster2022confident}, one aims at deciding dynamically at which layer to stop the computation and ``read'' the prediction from the hidden representation of that layer.
More precisely, under a confidence measure paradigm, one decides to stop the computation for a position $i$ at layer $\ell$ based on a confidence criterion, that is derived from casting the hidden representation $h_i^\ell$ as a final-layer representation and converting it to an output probability distribution. Specifically, following \citet{schuster2022confident}, a decision to exit is made if the difference between the highest and the second highest probabilities is bigger than $$ 0.9 \cdot \lambda + 0.1 \cdot {\rm exp} (-4 i / N),$$
where $N$ is the average length of the input until position $i_s$ for $s \in \mathcal{V}$, and $\lambda$ is a hyper-parameter.

\begin{figure}[t]
\setlength{\belowcaptionskip}{-10pt}
\centering
\includegraphics[width=\columnwidth]{figs/ee_gpt2bert.pdf}
\caption{Precision@$1$ with early exit and ``fixed exit'', applied to the $24$-layer \gpt{} for next token prediction (left) and the $24$-layer \bert{} for masked token prediction (right). Varying the confidence parameter $\lambda$, the $x$-coordinate is the average number of layers processed before an early exit decision is reached.}
\label{fig:ee_gpt2bert}
\end{figure}

\quash{
\begin{figure}[t]
\setlength{\belowcaptionskip}{-10pt}
\centering
\includegraphics[scale=0.35]{figs/ee_pre1_24.pdf}
\caption{Precision@$1$ for the various early exit methods, and previous ``fixed exit'' methods for comparison ($24$-layer \gpt{} next token prediction task). Varying the confidence parameter $\lambda$, the $x$-coordinate is the average number of layers processed before an early exit decision is reached.}
\label{fig:ee_pre1}
\end{figure}
}

\paragraph{Experiment.}
We assess the utility of our mapping $\matlL{}$ for early exit as a plug-and-play replacement for $\idlL{}$, through which intermediate representations are cast into final-layer representations.
We use \gpt{} for the next token prediction and \bert{} for masked token prediction (both with 24 layers).
We run each of the models over the validation set examples, while varying the confidence parameter $\lambda$ and using either $\idlL{}$ or $\matlL{}$ for casting intermediate representations.
Furthermore, we compare these early exit variants to the ``fixed exit'' strategy from \S\ref{sec:prediction}, where the computation is stopped after a pre-defined number of layers rather than relying on a dynamic decision.
We evaluate each variant in terms of both prediction's accuracy, using the Precision@$1$ metric (see \S\ref{sec:prediction}), and efficiency, measured as the average number of transformer layers processed during inference.


\paragraph{Results.}
%Figs.~\ref{fig:ee_pre1} and~\ref{fig:bertmask_ee_pre1}
Fig.~\ref{fig:ee_gpt2bert}
plots the average Precision@$1$ score against the average number of layers processed, for $24$-layer \gpt{} and $24$-layer \bert{}. For both models, under an early exit strategy our mapping \mat{} again provides a substantial improvement over \id{}.
For example, aiming at $95\%$ average precision, \mat{} saves $\sim3.3$ ($13.8$\%) layers in \gpt{} compared to only $\sim1.4$ ($5.9$\%) layers by \id{}, and $\sim4.8$ ($20$\%) layers in \bert{} versus $\sim3.5$ ($14.6$\%) layers by \id{}.
These results highlight the potential gains prominent early exit methods can obtain by using our method.
Notably, in both models and for each of the mapping methods, early exit obtains better results than fixed layer exit, as expected. 

\quash{
\begin{figure}[t]
\setlength{\belowcaptionskip}{-10pt}
\centering
\includegraphics[scale=0.35]{figs/bertmask_ee_pre1_24.pdf}
\caption{Precision@$1$ for the various early exit methods, and previous ``fixed exit'' methods for comparison ($24$-layer \bert{} masked token prediction task). Varying the confidence parameter $\lambda$, the $x$-coordinate is the average number of layers processed before an early exit decision is reached.}
\label{fig:bertmask_ee_pre1}
\end{figure}
}
\section{Linear Shortcut Across Sub-Modules}
\label{sec:submodules}

% Our experiments show that
% , despite the commonly-applied simplification by interpretability works, transformer layers do not operate in the same linear space and 
% there is a major gap in approximating future representations using an identity mapping (\S\ref{sec:layer_jump}, \S\ref{sec:prediction}).
% Here, 
In this section, we investigate whether discrepancies across layers result from specific sub-modules or are a general behaviour of all sub-modules in the network.  
This is done by extending our approach to test how well particular components in transformer blocks can be linearly approximated. 


\paragraph{Method.}

Consider \gpt{} for definiteness, then:
% we have 
$$ \texttt{b}_{\ell} = \texttt{b}_{\ell}^{\texttt{ffn}} \circ \texttt{b}_{\ell}^{\texttt{attn}}$$ 
% with
\begin{equation}\label{eq:attn} \texttt{b}^{\texttt{attn}}_{\ell} (H) = \texttt{attn}_{\ell} (\texttt{ln1}_{\ell} (H)) + H,\end{equation} 
where $\texttt{attn}_{\ell}$ is
%a multi-head self-attention
a MHSA
layer and \texttt{ln1} is a layer normalization (LN), and 
$$ \texttt{b}^{\texttt{ffn}}_{\ell} (H) = \texttt{ffn}_{\ell} (\texttt{ln2}_{\ell} (H)) + H,$$  
where $\texttt{ffn}_{\ell}$ is
%a feed-forward network
an FFN
layer and $\texttt{ln2}$ is a LN.
\quash{
Given a block $\texttt{b}_\ell$ and one of its sub-modules $\texttt{ln1}_\ell, \ \texttt{attn}_\ell, \ \texttt{ln2}_\ell$, or $\texttt{ffn}_\ell$, we fit linear regression approximating the output of the sub-module given its input and then use it in order to define mappings, as we now describe.
}
Given a block $\texttt{b}_\ell$ and one of its sub-modules $\texttt{ln1}_\ell, \ \texttt{attn}_\ell, \ \texttt{ln2}_\ell$, or $\texttt{ffn}_\ell$, we fit linear regression approximating the output of the sub-module given its input, and then use it to define mappings $\matattnl{}$, $\matlnl{}$ and $\matffl{}$.
%We provide the definition of $\matattnl{}$ below, and that of the other two in App. \ref{sec:app_submodule_skip_description}.
We provide the formal definitions of these mappings in App. \ref{sec:app_submodule_skip_description}.
\iffalse
\paragraph{$\matattnl{}$.}
%Illustrating this on $\texttt{attn}_\ell$ for definiteness,
For an input $s$, let $v^\ell_{i_s}$ be the vector at position $i_s$ in the output of $\texttt{attn}_\ell (\texttt{ln1}_\ell (H^{\ell - 1}))$. We denote by $A_\ell^{\texttt{attn}} \in \mathbb{R}^{d_h \times d_h}$ the matrix numerically minimizing 
$$ A \mapsto \sum_{s \in \mathcal{T}} || A \cdot \texttt{ln1}_\ell (h^{\ell-1}_{i_s}) - v^\ell_{i_s}||^2,$$
and define an attention sub-module replacement (Eq.~\ref{eq:attn}) by $$
\texttt{b}^{\overline{\texttt{attn}}}_\ell (h) \coloneqq A_{\ell}^{\texttt{attn}} \cdot \texttt{ln1}_\ell (h) + h. $$
We then define a mapping between two layers ${\ell \rightarrow \ell'}$ by:
$$ \matattnl{} (h) \coloneqq $$
$$ \texttt{b}^{\texttt{ffn}}_{\ell'} ( \texttt{b}^{\overline{\texttt{attn}}}_{\ell'} ( \ldots (\texttt{b}^{\texttt{ffn}}_{\ell+1} ( \texttt{b}^{\overline{\texttt{attn}}}_{\ell+1} (h)))\ldots)).$$ 
Namely, when applying each $\ell''$-th block, $\ell < \ell'' \leq \ell'$, we replace its attention sub-module $\texttt{attn}_{\ell''}$ by its linear approximation.
%In an analogous way, we consider the mappings $\matffl{}$ and $\matlnl{}$, where in the latter we perform the linear shortcut both for \texttt{ln1} and for \texttt{ln2} (see~\S\ref{sec:app_submodule_skip_description} for precise descriptions).
Importantly, unlike the original attention module, the approximation $\texttt{b}^{\overline{\texttt{attn}}}_\ell$ operates on each position independently, and therefore applying $\matattnl{}$ disables any contextualization between the layers $\ell$ and $\ell'$. Note that this is not the case for $\matffl{}$ and $\matlnl{}$, which retain the self-attention sub-modules and operate contextually.
\fi

\paragraph{Evaluation.}


We analyze the $24$-layered \gpt{}, and proceed completely analogously to \S\ref{subsec:next_token_prediction_task}, evaluating the Precision@$1$ and Surprisal metrics for the mappings $\matattnlL{}$, $\matfflL{}$ and $\matlnlL{}$.

\begin{figure}[t]
\setlength{\belowcaptionskip}{-0pt}
\centering
%\includegraphics[scale=0.2]
\includegraphics[width=\columnwidth]{figs/parts_presurp_24.pdf}
\caption{Precision@$1$ and Surprisal for the various sub-module linear mappings, and $\matlL{}$ for comparison ($24$-layer \gpt{} next token prediction task). A 95\% confidence interval surrounds the Surprisal lines.}
\label{fig:parts_presurp}
\end{figure}

\quash{
\begin{figure}[t]
\centering
\includegraphics[scale=0.4]{figs/parts_pre1_24.pdf}
\caption{Precision@$1$ for the various sub-module linear shortcut mappings, and the mapping $\matlL{}$ for comparison (\gpt{} next token prediction task).}
\label{fig:parts_pre1}
\end{figure}

\begin{figure}[t]
\centering
\includegraphics[scale=0.35]{figs/parts_surp_24.pdf}
\caption{Surprisal for the various sub-module linear shortcut mappings, and the mapping $\matlL{}$ for comparison (\gpt{} next token prediction task). A 95\% confidence interval surrounds the lines.}
\label{fig:parts_surp}
\end{figure}
}

\paragraph{Results.}
Fig.~\ref{fig:parts_presurp} shows the average Precision@$1$ and Surprisal scores per layer.
From a certain layer (\textasciitilde$7$), all sub-module mappings achieve better results than the full-block mapping $\matlL{}$. Thus, it is not just the cumulative effect of all the sub-modules in the transformer block that is amenable to linear approximation, but also individual sub-modules can be linearly approximated. 
Furthermore, the linear approximation of attention sub-modules is less harmful than that of the FFN or LN sub-modules. 
% Hypothetically, 
A possible reason is that the linear replacement of FFN or LN ``erodes'' the self-attention computation after a few layers. 
Moreover, the good performance of $\matattnlL{}$ suggests that contextualization often exhausts itself in early layers; speculatively, it is only in more delicate cases that the self-attention of late layers adds important information. Last, remark the sharp ascent of the scores for layer normalization in layers $5$-$8$, for which we do not currently see a particular reason. To conclude, we see that the possibility of linear approximation permeates
%the various
transformer components.


\section{Related Work}

Recently, there was a lot of interest in utilizing intermediate representations in transformer-based LMs, both for interpretability and for efficiency.

In the direction of interpretability, one seeks to understand the prediction construction process of the model \cite{tenney-etal-2019-bert, voita-etal-2019-bottom}.

More recent works use mechanistic interpretability and view the inference pass as a residual stream of information \cite{dar2022analyzing,geva-etal-2022-transformer}. Additionally, there are works on probing, attempting to understand what features are stored in the hidden representations \cite{adi2017finegrained, conneau-etal-2018-cram,liu-etal-2019-linguistic}. Our work is different in that it attempts to convert intermediate representations into a final-layer form, which is interpretable by design.

In the direction of efficiency, there is the thread of work on early exit, where computation is cut at a dynamically-decided earlier stage \cite{schwartz-etal-2020-right,xin-etal-2020-deebert,schuster2022confident}. Other works utilize a fixed early stage network to parallelize inference \citep{leviathan2022fast, chen2023accelerating}. However, intermediate representations are directly propagated in these works, which we show is substantially worse than our approach. Moreover, our method requires training considerably less parameters than methods such as \citet{schuster-etal-2021-consistent}, that learn a different output softmax for each intermediate layer.  

More broadly, skipping transformer layers and analyzing the linearity properties of transformer components have been discussed in prior works \cite{Zhao2021of,mickus-etal-2022-dissect,wang-etal-2022-skipbert,lamparth2023analyzing}.


\section{Conclusion and Future Work}

We present a simple and effective method for enhancing utilization of hidden representations in transformer-based LMs, that uses 
pre-fitted context-free and token-uniform linear mappings.
Through a series of experiments on different data sources, model architectures and scales, we show that our method consistently outperforms the prevalent practice of interpreting representations in the final-layer space of the model, yielding better approximations of succeeding representations and the predictions they induce, thus allowing a more faithful interpretation of the model's prediction-formation.
We demonstrate the practicality of our method for improving computation efficiency, saving a substantial amount of compute on top of prominent early exiting approaches. 
Also, by extending our method to sub-modules, 
% more specifically the attention sub-modules, 
we observe that replacing a part of the transformer inference by a non-contextual linear computation often results in a small deterioration of the prediction.
This opens new research directions for improving model efficiency,
% and parallelizability.
% including breaking the computation into several parallelizable tasks.
including breaking the computation into parallel tasks.

\section*{Limitations}

Although we see in this work that there is more linear structure to transformer inference than could be explained solely by the residual connection, we do not elucidate a reason for that. We also do not try to formulate formal criteria according to which to judge, in principle, the quality of ways of short-cutting transformer inference in-between layers. In addition, our experiments cover only English data.


%\section*{Ethics Statement}
%Scientific work published at ACL 2023 must comply with the ACL Ethics Policy.\footnote{\url{https://www.aclweb.org/portal/content/acl-code-ethics}} We encourage all authors to include an explicit ethics statement on the broader impact of the work, or other ethical considerations after the conclusion but before the references. The ethics statement will not count toward the page limit (8 pages for long, 4 pages for short papers).

\section*{Acknowledgements}

We thank Tal Schuster for constructive comments.

% Entries for the entire Anthology, followed by custom entries
\bibliography{anthology,custom}
\bibliographystyle{acl_natbib}

\appendix

\section{Descriptions of $\matattn{}$, $\matff{}$ and $\matln{}$}
\label{sec:app_submodule_skip_description}

Here we detail the definitions of the mappings $\matattnl{}$, $\matffl{}$ and $\matlnl{}$ utilized in \S\ref{sec:submodules}.

\paragraph{Description of $\matattnl{}$.}
%Illustrating this on $\texttt{attn}_\ell$ for definiteness,
For an input $s$, let $v^\ell_{i_s}$ be the vector at position $i_s$ in the output of $\texttt{attn}_\ell (\texttt{ln1}_\ell (H^{\ell - 1}))$. We denote by $A_\ell^{\texttt{attn}} \in \mathbb{R}^{d_h \times d_h}$ the matrix numerically minimizing 
$$ A \mapsto \sum_{s \in \mathcal{T}} || A \cdot \texttt{ln1}_\ell (h^{\ell-1}_{i_s}) - v^\ell_{i_s}||^2,$$
and define an attention sub-module replacement (Eq.~\ref{eq:attn}) by $$
\texttt{b}^{\overline{\texttt{attn}}}_\ell (h) \coloneqq A_{\ell}^{\texttt{attn}} \cdot \texttt{ln1}_\ell (h) + h. $$
We then define a mapping between two layers ${\ell \rightarrow \ell'}$ by:
$$ \matattnl{} (h) \coloneqq $$
$$ \texttt{b}^{\texttt{ffn}}_{\ell'} ( \texttt{b}^{\overline{\texttt{attn}}}_{\ell'} ( \ldots (\texttt{b}^{\texttt{ffn}}_{\ell+1} ( \texttt{b}^{\overline{\texttt{attn}}}_{\ell+1} (h)))\ldots)).$$ 
Namely, when applying each $\ell''$-th block, $\ell < \ell'' \leq \ell'$, we replace its attention sub-module $\texttt{attn}_{\ell''}$ by its linear approximation.
%In an analogous way, we consider the mappings $\matffl{}$ and $\matlnl{}$, where in the latter we perform the linear shortcut both for \texttt{ln1} and for \texttt{ln2} (see~\S\ref{sec:app_submodule_skip_description} for precise descriptions).
Importantly, unlike the original attention module, the approximation $\texttt{b}^{\overline{\texttt{attn}}}_\ell$ operates on each position independently, and therefore applying $\matattnl{}$ disables any contextualization between the layers $\ell$ and $\ell'$. Note that this is not the case for $\matffl{}$ and $\matlnl{}$, which retain the self-attention sub-modules and operate contextually.

\paragraph{Description of $\matffl{}$.}
Let $v^\ell_{i_s}$ be the vector at position $i_s$ in the output of $\texttt{ln2}_{\ell} (\texttt{b}_\ell^{\texttt{attn}} (H^{\ell - 1}))$, for a given input $s$. We denote by $A_\ell^{\texttt{ffn}} \in \mathbb{R}^{d_h \times d_h}$ the matrix numerically minimizing 
$$ A \mapsto \sum_{s \in \mathcal{T}} || A \cdot v^{\ell}_{i_s} - \texttt{ffn}_{\ell} (v^\ell_{i_s})||^2,$$
and define a replacement of the feed-forward sub-module $\texttt{b}_{\ell}^{\texttt{ffn}}$ by $$ \texttt{b}^{\overline{\texttt{ffn}}}_\ell (H) \coloneqq A_{\ell}^{\texttt{ffn}} \cdot \texttt{ln2}_\ell (H) + H.$$
We then define a mapping between two layers ${\ell \rightarrow \ell'}$ by:
$$ \matffl{} (H) \coloneqq $$
$$ \texttt{b}^{\overline{\texttt{ffn}}}_{\ell'} ( \texttt{b}^{\texttt{attn}}_{\ell'} ( \ldots (\texttt{b}^{\overline{\texttt{ffn}}}_{\ell+1} ( \texttt{b}^{\texttt{attn}}_{\ell+1} (H))\ldots)).$$

\paragraph{Description of $\matlnl{}$.}
Let $v^\ell_{i_s}$ be the vector at position $i_s$ in the output of $\texttt{b}^{\texttt{attn}}_{\ell} (H^{\ell - 1})$, for a given input $s$. We denote by $A_\ell^{\texttt{ln1}} \in \mathbb{R}^{d_h \times d_h}$ the matrix numerically minimizing 
$$ A \mapsto \sum_{s \in \mathcal{T}} || A \cdot h^{\ell}_{i_s} - \texttt{ln1}_{\ell} (h^\ell_{i_s})||^2$$ and we denote by $A_\ell^{\texttt{ln2}} \in \mathbb{R}^{d_h \times d_h}$ the matrix numerically minimizing $$ A \mapsto \sum_{s \in \mathcal{T}} || A \cdot v^{\ell}_{i_s} - \texttt{ln2}_{\ell} (v^\ell_{i_s})||^2.$$ We define a replacement of the block $\texttt{b}^{\texttt{attn}}_{\ell}$ by \begin{equation} \texttt{b}^{\overline{\texttt{ln1}}}_\ell (H) \coloneqq \texttt{attn}_{\ell} (A_{\ell}^{\texttt{ln1}} \cdot H) + H\end{equation} and we define a replacement of the block $\texttt{b}^{\texttt{ffn}}_{\ell}$ by \begin{equation} \texttt{b}^{\overline{\texttt{ln2}}}_\ell (H) \coloneqq \texttt{ffn}_{\ell} (A_{\ell}^{\texttt{ln2}} \cdot H) + H.\end{equation}
We then define a mapping between two layers ${\ell \rightarrow \ell'}$ by:
$$ \matlnl{} (H) \coloneqq $$
$$ \texttt{b}^{\overline{\texttt{ln2}}}_{\ell'} ( \texttt{b}^{\overline{\texttt{ln1}}}_{\ell'} ( \ldots (\texttt{b}^{\overline{\texttt{ln2}}}_{\ell+1} ( \texttt{b}^{\overline{\texttt{ln1}}}_{\ell+1} (H))\ldots)).$$


\end{document}


\vspace{1cm}


%% or
%% [B] Manual formatting (see below)
%% (i) We suggest to always provide author, title and journal data or doi:
%% in short all the informations that clearly identify a document.
%% (ii) please avoid comments such as "For a review'', "For some examples",
%% "and references therein" or move them in the text. In general, please leave only references in the bibliography and move all
%% accessory text in footnotes.
%% (iii) Also, please have only one work for each \bibitem.

% \begin{thebibliography}{99}

% \bibitem{a}
% Author,
% \emph{Title},
% \emph{J. Abbrev.} {\bf vol} (year) pg.

% \bibitem{b}
% Author,
% \emph{Title},
% arxiv:1234.5678.

% \bibitem{c}
% Author,
% \emph{Title},
% Publisher (year).

% \end{thebibliography}
\end{document}
