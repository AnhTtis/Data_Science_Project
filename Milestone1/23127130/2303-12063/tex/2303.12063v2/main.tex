\documentclass[a4paper,11pt]{article}
\pdfoutput=1
\usepackage{jheppub} % for details on the use of the package, please see the JINST-author-manual
\usepackage{lineno}
\usepackage{float}
%\linenumbers

\newcommand{\muI}{\mu_I}
\newcommand{\muB}{\mu_B}
\newcommand{\LA}{\left \langle}
\newcommand{\RA}{\right \rangle}
\newcommand{\Ns}{N_{\sigma}}
\newcommand{\Nt}{N_{\tau}}
\newcommand{\Tpc}{T_{pc}}
\newcommand{\N}{\mathcal{N}/T^3}
\newcommand{\Ob}{\mathcal{O}}
\newcommand{\M}{\mathcal{M}}
\newcommand{\hs}{\hspace{.6cm}}
\newcommand{\heq}{\hspace{.5mm}}
\renewcommand{\L}{\mathcal{L}}
\newcommand{\diff}{\Delta_I(\text{NR},0)}
\newcommand{\muS}{\mu_s}
\newcommand{\modph}{\left|\LA e^{i\Theta}\RA \right|}
\newcommand{\modmu}{\left|\muI\right|}

\newcommand{\ctBT}{(\chi_2^B)_T}
\newcommand{\ctIT}{(\chi_2^I)_T}
\newcommand{\cfBT}{(\chi_4^B)_T}
\newcommand{\cfIT}{(\chi_4^I)_T}
\newcommand{\csBT}{(\chi_6^B)_T}
\newcommand{\csIT}{(\chi_6^I)_T}
\newcommand{\ceBT}{(\chi_8^B)_T}
\newcommand{\ceIT}{(\chi_8^I)_T}

\newcommand{\ctBt}{(\chi_2^B)_2}
\newcommand{\ctIt}{(\chi_2^I)_2}
\newcommand{\cfBt}{(\chi_4^B)_2}
\newcommand{\cfIt}{(\chi_4^I)_2}
\newcommand{\csBt}{(\chi_6^B)_2}
\newcommand{\csIt}{(\chi_6^I)_2}
\newcommand{\ceBt}{(\chi_8^B)_2}
\newcommand{\ceIt}{(\chi_8^I)_2}

\newcommand{\ctBf}{(\chi_2^B)_4}
\newcommand{\ctIf}{(\chi_2^I)_4}
\newcommand{\cfBf}{(\chi_4^B)_4}
\newcommand{\cfIf}{(\chi_4^I)_4}
\newcommand{\csBf}{(\chi_6^B)_4}
\newcommand{\csIf}{(\chi_6^I)_4}
\newcommand{\ceBf}{(\chi_8^B)_4}
\newcommand{\ceIf}{(\chi_8^I)_4}

\newcommand{\ctBs}{(\chi_2^B)_6}
\newcommand{\ctIs}{(\chi_2^I)_6}
\newcommand{\cfBs}{(\chi_4^B)_6}
\newcommand{\cfIs}{(\chi_4^I)_6}
\newcommand{\csBs}{(\chi_6^B)_6}
\newcommand{\csIs}{(\chi_6^I)_6}
\newcommand{\ceBs}{(\chi_8^B)_6}
\newcommand{\ceIs}{(\chi_8^I)_6}

\newcommand{\cnBT}{(\chi_n^B)_T}
\newcommand{\cnBm}{(\chi_n^B)_m}
\newcommand{\cnIT}{(\chi_n^I)_T}
\newcommand{\cnIm}{(\chi_n^I)_m}















\newcommand{\rtBT}{(\rho_2^B)_T}
\newcommand{\rtIT}{(\rho_2^I)_T}
\newcommand{\rfBT}{(\rho_4^B)_T}
\newcommand{\rfIT}{(\rho_4^I)_T}
\newcommand{\rsBT}{(\rho_6^B)_T}
\newcommand{\rsIT}{(\rho_6^I)_T}
\newcommand{\reBT}{(\rho_8^B)_T}
\newcommand{\reIT}{(\rho_8^I)_T}

\newcommand{\rtBt}{(\rho_2^B)_2}
\newcommand{\rtIt}{(\rho_2^I)_2}
\newcommand{\rfBt}{(\rho_4^B)_2}
\newcommand{\rfIt}{(\rho_4^I)_2}
\newcommand{\rsBt}{(\rho_6^B)_2}
\newcommand{\rsIt}{(\rho_6^I)_2}
\newcommand{\reBt}{(\rho_8^B)_2}
\newcommand{\reIt}{(\rho_8^I)_2}

\newcommand{\rtBf}{(\rho_2^B)_4}
\newcommand{\rtIf}{(\rho_2^I)_4}
\newcommand{\rfBf}{(\rho_4^B)_4}
\newcommand{\rfIf}{(\rho_4^I)_4}
\newcommand{\rsBf}{(\rho_6^B)_4}
\newcommand{\rsIf}{(\rho_6^I)_4}
\newcommand{\reBf}{(\rho_8^B)_4}
\newcommand{\reIf}{(\rho_8^I)_4}

\newcommand{\rtBs}{(\rho_2^B)_6}
\newcommand{\rtIs}{(\rho_2^I)_6}
\newcommand{\rfBs}{(\rho_4^B)_6}
\newcommand{\rfIs}{(\rho_4^I)_6}
\newcommand{\rsBs}{(\rho_6^B)_6}
\newcommand{\rsIs}{(\rho_6^I)_6}
\newcommand{\reBs}{(\rho_8^B)_6}
\newcommand{\reIs}{(\rho_8^I)_6}

\newcommand{\rnBT}{(\rho_n^B)_T}
\newcommand{\rnBm}{(\rho_n^B)_m}
\newcommand{\rnIT}{(\rho_n^I)_T}
\newcommand{\rnIm}{(\rho_n^I)_m}

%\arxivnumber{1234.56789} % if you have one
\usepackage{calc}
\newlength{\depthofsumsign}
\setlength{\depthofsumsign}{\depthof{$\sum$}}
\newlength{\totalheightofsumsign}
\newlength{\heightanddepthofargument}
\newcommand{\nsum}[1][1.2]{% only for \displaystyle
    \mathop{%
        \raisebox
            {-#1\depthofsumsign+1\depthofsumsign}
            {\scalebox
                {#1}
                {$\displaystyle\sum$}%
            }
    }
}

\newcommand{\nlangle}[1][1.2]{% only for \displaystyle
    \mathop{%
        \raisebox
            {-#1\depthoflanglesign+1\depthoflanglesign}
            {\scalebox
                {#1}
                {$\displaystyle\langle$}%
            }
    }
}


\newcommand{\nrangle}[1][1.2]{% only for \displaystyle
    \mathop{%
        \raisebox
            {-#1\depthofranglesign+1\depthofranglesign}
            {\scalebox
                {#1}
                {$\displaystyle\rangle$}%
            }
    }
}

\title{\boldmath A Comparative Analysis between Unbiased Exponential Resummation and Taylor Expansion in Finite-Density QCD with a new phasefactor for Isospin }

% Collaborations

%% [A] If main author
%% \collaboration{\includegraphics[height=17mm]{collabroation-logo}\\[6pt]
%%  XXX collaboration}

%% or
%% [B] If "on behalf of"
%% \collaboration[c]{on behalf of XXX collaboration}


% Authors
% The "\note" macro will give a warning: "Ignoring empty anchor...", you can safely ignore it.

%% [A] simple case: 2 authors, same institution
\author[a,b]{Sabarnya Mitra}
%\author[a]{and Prasad Hegde}
\affiliation[a]{Centre for High Energy Physics, Indian Institute of Science, Bengaluru 560012, India}
\affiliation[b]{Fak{\"u}ltat F{\"u}r Physik, Universit{\"a}t Bielefeld 33615, Germany}
%\affiliation[a]{Fakultät für Physik, Universität Bielefeld \\ Universitätsstraße 25, 33615 Bielefeld, Germany}
%% or, e.g.
%% [B] more complex case: 4 authors, 3 institutions, 2 footnotes
%% \author[a,b]{F. Irst,\note{Now at another university}}
%% \author[c]{S. Econd,}
%% \author[a,2]{T. Hird\note{Also at Some University.}}
%% \author[c,2]{and fourth}
%% \affiliation[a]{Institution_1,\\Address, Country}
%% \affiliation[b]{Institution_2,\\Address, Country}
%% \affiliation[c]{Institution_3,\\Address, Country}

% \author{Sabarnya Mitra}
% \affiliation{Indian Institute of Science,\\
% some-street, Country}
% \affiliation{Another University,\\
% different-address, Country}

% E-mail addresses: only for the corresponding author

\emailAdd{smitra@physik.uni-bielefeld.de}
\emailAdd{sabarnyam@iisc.ac.in}

\abstract
{
 The recently introduced unbiased exponential resummation at finite chemical potential has become an important approach which promises to capture the higher order Taylor coefficients reliably appearing otherwise in the finite-density QCD Taylor series of thermodynamic observables. In this paper, we present a detailed comparative study between the usual Taylor coefficients upto eighth order and the corresponding estimates of the same coefficients obtained using unbiased exponential resummation at finite baryon and isospin chemical potentials. We also subsequently compare the different estimates of radius of convergence obtained from the Taylor expansion and the same procured from the unbiased exponential resummation scheme. We propose a new method of finding a non-trivial phasefactor estimate for isospin chemical potential and we attempt to explain the values of these different estimates of radius of convergence by observing the onset of zeroes of this gauge-ensemble average phasefactor for both baryon and isospin chemical potentials. Lastly, we also illustrate kurtosis plots describing the behaviour of overlap problem in isospin chemical potential and check if it maintains consistency with the appearance of zeros of the proposed phasefactor estimate for isospin chemical potential.    
}




\begin{document}
\maketitle
\flushbottom



\section{Introduction}
\label{sec:Intro}

\hs The strong force, one of the four fundamental forces of Nature is very well described by the quantum field theory of Quantum Chromodynamics (QCD)\,\cite{Gross:2022QCD}. An immensely important and intriguing spectacle in the paradigm of these strong interactions is the QCD phase diagram which features various interesting phases of strongly interacting matter. One of the important aspects of QCD is to explore and map this phase diagram\,\cite{Halasz:1998phasediagram, Rajagopal:1999phasediagram,Stephanov:2006phasediagram,Fukushima:2011phasediagram} as a function of temperature $T$ and baryochemical potential $\muB$. This is pivotal not only for understanding the strong dynamics at various energy scales, but also for illuminating the physics of early universe\,\cite{McGuigan:2008earlyuniverse,Castorina:2015earlyuniverse}. Despite being very robust and seemingly self-explanatory, most of this phase diagram have been constructed out from mere symmetry arguments and analyses of various QCD models. They continue to remain conjectured and await further conclusive evidences. In the quest of such evidences, one often resorts to formulating QCD on a lattice of spacetime\,\cite{Davies:2005latticeQCD,Boyle:2022latticeQCD} mostly because of its remarkable ability to successfully predict results to appreciable degree of precision. Besides offering possible signatures of unexplored phases, the non-perturbative formulation of thermodynamics in lattice QCD also enables one to obtain significant insights about the phase diagram.  Like at present, lattice simulations can well explain the manifestations at finite temperature, zero $\muB$ which resembles the vertical axis of the phase diagram. It also establishes that the phase transition between the hadronic phase and the quark-gluon plasma phases at zero $\muB$ is an analytical crossover\,\cite{Steinbrecher:2018QCDcrossover,Borsanyi:2020QCDcrossover,Bazavov:2018QCDcrossover,Li:2020QCDcrossover,Guenther:2021QCDcrossover}.


However for real finite $\muB$, lattice QCD faces a stumbling block in the shape of sign problem\,\cite{Gupta:2004signproblem,Danzer:2009signproblem,Goy:2016signproblem,Nagata:2021signproblem}. At finite $\muB$, the path integral\,\cite{Palumbo:2002pathintegral} expressing the QCD partition function $Z$ becomes complex with its measure containing a complex fermion determinant\,\cite{Nakamura:2005complexfermiondeterminant} which gives rise to the problem of complex measure. This complex measure hinders implementation of Monte-Carlo importance sampling for estimating this path integral. While reweighting\,\cite{Ejiri:2004reweighting,Li:2006reweighting} this complex measure with a real fermion determinant at zero $\muB$ makes the measure real, the observable part of the integral becomes complex which, after Monte-Carlo estimation provides a phaseangle $\theta$ and a subsequent phasefactor $\cos \theta$ for every gauge configuration of the working ensemble. The severity of this sign problem is governed by the magnitude of this $\cos \theta$ averaged over the entire ensemble, the value of which decreases towards zero for higher values of $\muB$ thereby reflecting increasing severity of the sign problem. This happens because the integrand exhibits tremendous oscillations across positive and negative real values, each of which is also large in magnitude, causing the mean to settle towards zero. This sign problem eventually leads to the breakdown of lattice QCD computations at a finite value of $\muB$ reflected by the non-monotonic behaviour of the calculated observables. This highly restricts our investigation and consequent knowledge of QCD at finite density.

Several new methods\,\cite{Bilic:1987Langevin,Aarts:2016Langevin,Kogut:2019Langevin,Sinclair:2019Langevin,Cristoforetti:2012Thimbles,Cristoforetti:2013Thimbles,Scorzato:2015Thimbles} have been introduced which can successfully avoid this sign problem, most of which unfortunately have very limited applications in QCD explicitly. In the case of QCD, the Taylor expansion around $\muB=0$\,\cite{Gavai:2003Taylor,Ejiri:2003Taylor,Gavai:2004Taylor,Gavai:2008Taylor,Miao:2008Taylor,Falcone:2010Taylor} and analytic continuation of simulations from imaginary to real $\muB$\,\cite{DElia:2002Analytic,Lombardo:2006Analytic,Sakai:2009Analytic} continue to remain the prominent methods for circumventing the sign problem and providing state-of-the-art results for QCD equation of state\,\cite{Fodor:2002EoS,Aoki:2005EoS,Miller:2006EoS,Karsch:2008EoS,Kanaya:2010EoS,Huovinen:2011EoS,Philipsen:2012EoS,Hegde:2014EoS,Bazavov:2017EoS} at finite $\muB$. Resummation approaches like  Pad{\'e}\,\cite{Cvetic:2011Pade,Pasztor:2020Pade,Bollweg:2022Pade} and exponential resummation\,\cite{Mondal:2021exponentialresummation} have been proposed, to improve the slowly convergent Taylor series results. While the former approximates the Taylor coefficients by rational functions where one is interested to find the roots and poles of these functions, the latter provides a direct estimate of $Z$ in the form of an exponential in which the argument comprises finite contributions of lower order Taylor series. Recently, the new formalism of unbiased exponential resummation\,\cite{Mitra:2023unb,Mitra:2022dae} has been introduced for obtaining a more improved QCD Equation of state at finite chemical potential, by obviating the stochastic bias\,\cite{Mitra:2022cumu} and reproducing the exact Taylor series to a given order in $\mu$, where $\mu$ is any generic flavor of chemical potential. This unbiased approach\,\cite{Mitra:2022Bonn} is paramount for recognising the genuine higher order Taylor contributions captured through this approach of resummation.

 Although the breakdown of calculations can be detected by observing the onset of the zeros of phasefactor for $\muB$, it is not the case for $\muI$ since it has no sign problem. Hence, it is not possible to identify a possible breakdown in $\muI$ by looking for the zeros of the phasefactor, which never becomes zero. Although this means that in principle, one can perform unbiased exponential resummation to all real values of $\muI$ extending to infinity, studies suggest that there is a genuine phase diagram\,\cite{deForcrand:2007isospin,Moller:2009isospin,Brandt:2017isospin,Brandt:2018isospin} in the $T-\muI$ plane which illustrates the formation of a pion condensate starting from some finite value of $\muI$ for a low $T$. This signifies that at a low $T$ surely, this formalism is supposed to have a finite radius of convergence in $\muI$ and is expected to experience a breakdown beyond that. Apart from many other objectives, this paper tries to come up with a new indicator for this purpose.





The paper is organised as follows: In Section \ref{sec:Unbiased exp}, we provide a quick overview about the Taylor expansion and the unbiased exponential resummation formalism, which constitute the two main cornerstone of comparisons in this paper. This is accompanied by a brief description about phasefactor of this resummation formalism, and the new idea of a complex phasefactor which is being proposed to identify possible breakdown for $\muI$. In Section \ref{sec:Overlap problem}, we have discussed briefly about the overlap problem and its severity along $\muI$. Starting from the details of scale setting and setup of lattice including random volume sources and gauge configurations used in constructing the Taylor coefficients, the method of error estimation for unbiased exponential resummation formalism are highlighted in Section \ref{sec:setup}. In Section \ref{sec:Results}, we present vivid discussions regarding the comparisons between the sixth and eighth order conserved charge cumulants for both $\muB$ and $\muI$ obtained from Taylor expansion and unbiased exponential resummation. We also demonstrate the same for the different estimates of the radius of convergence, and attempt to observe if the onset of zeros of the gauge ensemble averaged phasefactor $\LA \cos \theta \RA$ and complex phasefactor $\LA e^{i\theta} \RA$ can provide indications consistent with these estimates of radius of convergence for $\muB$ and $\muI$ respectively. If so, they can provide consistent indications about the start of breakdown for $\muB$ and $\muI$. In the last part of this section, we illuminate the behaviour of overlap problem in $\muI$ and see if this is compatible with the implications made by the zeros of phasefactor and their onset in $\muI$. We have concluded the paper and its discussion by providing a brief summary in Section \ref{sec:Conclusions}. Throughout this paper, we have used relativistic units ($\hbar = c = 1$) and unit Boltzmann constant, and have often denoted $\mu/T$ as $\mu$. Often we have used this notation $\mu$ in this paper to imply both $\muB$ and $\muI$.

\section{Taylor expansion and Unbiased exponential resummation}
\label{sec:Unbiased exp}

\hs In a $2+1$ flavor QCD with staggered rooted quarks, the grand-canonical partition function $Z$ with suppressed volume dependence for a given temperature $T$ and chemical potential $\mu$ is given as

\begin{equation}
    Z(T,\mu) = \int \mathcal{D}U \heq e^{-S_G\left[T,U\right]} \heq \det \M(T,\mu,U)
    \label{eq:partition function}
\end{equation}
%
with the fermionic determinant $\det \M(T,\mu,U)$ being 

\begin{equation}
\det \M(T,\mu,U) = \prod_{f=u,d,s} \big[\det \M(T,\mu_f,U)\big]^{1/4} 
\label{eq:staggered fermion}
\end{equation}
%
 In the above Eqns.\eqref{eq:partition function} and \eqref{eq:staggered fermion}, $U$ represent the gauge field configurations and functional $S_G\left[T,U\right]$ denotes the gluon action. For a thermodynamic system of volume $V$ at temperature $T$, the excess pressure $\Delta P(T,\mu)$ is given as follows:

\begin{equation}
    \frac{\Delta P(T,\mu)}{T^4} = \frac{P(T,\mu) - P(T,0)}{T^4} = \frac{1}{VT^3} \, \ln \left[\frac{Z(T,\mu)}{Z(T,0)}\right]
    \label{eq:excess pressure}
\end{equation}
%
This measure of excess pressure in Eqn.\eqref{eq:excess pressure}, scaled in powers of $T$ is dimensionless which makes it useful for calculations at finite temperature on a given lattice. 


\subsection{Taylor Expansion}


The Taylor Expansion of this excess pressure to $\mathcal{O}(\mu^N)$ is given by 

\begin{equation}
    \frac{\Delta P_N^{\text{T}}(T,\mu)}{T^4} = \sum_{n=1}^{N/2} \frac{\chi_{2n}}{(2n)!} \left(\frac{\mu}{T}\right)^{2n}
    \label{eq:Taylor excess pressure}
\end{equation}
%
where in the above Eqn.\eqref{eq:Taylor excess pressure}, $\chi_{2n}$ is the conserved charge cumulant of order $2n$. The $n^{th}$ Taylor coefficient is defined as $c_n = \chi_n/n!$. The CP symmetry of QCD instructs this Taylor series of Eqn.\eqref{eq:Taylor excess pressure} to be even in $\mu$ which implies that $N$ is even. In terms of the different correlation functions $D_n$ where the $n^{th}$ order correlation function $D_n$ is defined as 

\begin{equation}
    D_n(T) = \frac{\partial^n \ln \det \M(T,\mu)}{\partial \mu^n}\Bigg|_{\mu=0} 
    \label{eq:correlation functions}
\end{equation}
%
the first four $\chi_n$ can be expressed as follows: 

\begin{align}
    \chi_1 &= \LA D_1 \RA \notag \\
    \chi_2 &= \LA D_2 \RA + \LA D_1^2 \RA \notag \\
    \chi_3 &= \LA D_3 \RA + 3\,\LA D_2\,D_1 \RA + \LA D_1^3 \RA \notag \\
    \chi_4 &= \LA D_4 \RA + 4\,\LA D_3\,D_1 \RA + 3\,\LA D_2^2 \RA + 6\,\LA D_2\,D_1^2 \RA + \LA D_1^4 \RA 
    \label{eq:Taylor coeff and correlation funs}
\end{align}
%
Throughout this paper, the notation $\LA \mathcal{O} \RA$ represents Monte-Carlo sampling average of observable $\mathcal{O}$ over all gauge configurations in the ensemble generated at $\mu=0$. All these powers are necessarily the unbiased powers of the respective $D_n$.

\subsection{Unbiased Exponential Resummation}

\subsubsection*{Exponential Resummation}
%
The method of exponential resummation commences with estimating the partition function directly and then deducing the thermodynamic quantities successively following the subsequent thermodynamic relations. In this approach, the ratio $Z(T,\mu)/Z(T,0)$ in the above Eqn.\eqref{eq:excess pressure} to $\mathcal{O}(\mu^N)$ is given by 
\begin{equation}
    Z_N^{\text{R}}(T,\mu) \equiv \frac{Z(T,\mu)}{Z(T,0)} = \Biggl< \exp \left(\nsum_{n=1}^N \left(\frac{\mu}{T}\right)^n \frac{D_n}{n!}\right)  \Biggr>    
    \label{eq:resummed partition function}
\end{equation}
%
%  
% where the notation $\LA \cdot \RA_0$ in Eqn.\eqref{eq:resummed partition function} represent Monte-Carlo sampling average over the working ensemble of gauge field configurations, each of which has been simulated at $\mu=0$. The expectation value of an observable $\mathcal{O}$ at an arbitrary $\mu$ and temperature $T$ calculated from an ensemble of gauge configurations simulated at $\muS$ at the same $T$ is given by
% \begin{align*}
%     \Big \langle \Ob(\mu) \Big \rangle_{\muS} &= \frac{1}{Z(\muS)} \int \mathcal{D}U \heq e^{-S_G\left[U\right]} \heq \Ob(\mu,U) \heq \det \M(\muS,U)
% \end{align*}
% %
% where we have from Eqn.\eqref{eq:partition function},
% \begin{equation*}
%     Z(\muS) = \int \mathcal{D}U \heq e^{-S_G\left[U\right]} \heq \det \M(\muS,U)
% \end{equation*}
% %
% We have subdued the $T$ dependence in the ongoing discussions of the paper, since the parameter of interest in this work is $\mu$.
%
with the symbols having conventional meanings as explained above.
On an isotropic\footnote{In isotropic lattice, spatial spacing $a_{\sigma} =$ temporal spacing $a_{\tau}$.} lattice of size $N_{\sigma}^3 \cdot N_{\tau}$ in $3+1$ spacetime having $N_{\sigma}$ points in each of the $3$ spatial directions and $N_{\tau}$ points in the temporal direction, the estimate of excess pressure for exponential resummation is obtained as follows:

\begin{equation}
    \frac{\Delta P_N^{\text{R}}(T,\mu)}{T^4} = \left(\frac{N_{\tau}}{N_{\sigma}}\right)^3 \, \ln Z_N^{\text{R}}(T,\mu)
\end{equation}
%
 where the expression of $Z_N^{\text{R}}$ is given in the above Eqn.\eqref{eq:resummed partition function}. The partition function $Z$ for any real $\mu$ is real-valued by virtue of the CP symmetry of QCD. which makes the $D_n$ given in Eqn.\eqref{eq:correlation functions} real or imaginary for even or odd $n$ respectively. 
 
%  The $D_n$ are the usual temperature dependent $n$-point correlation functions given for every gauge configuration $U$ as the following  

% \begin{equation}
%     D_n(T,U) = \frac{\partial^n \ln \det \M(T,\mu,U)}{\partial (\mu/T)^n}\Bigg|_{\mu=0}
%     \label{eq:derivatives}
% \end{equation}
% 
%\subsection{Effect of finite random volume sources and stochastic bias}
\subsubsection*{Stochastic Bias}
%
 However since the fermion matrix $\M$ cannot be evaluated exactly using analytical means\,\cite{Ying:1998fermionmatrix}, these correlation functions $D_n$ require numerical estimation. This is done by considering a finite number of random volume sources for every gauge configuration and thereby estimating $D_n$ for every random source for each gauge configuration. Hence in this limit, $D_n$ of Eqn.\eqref{eq:resummed partition function} is replaced by $\overline{D_n}$, which is given as:

\begin{equation}
    \overline{D_n} = \frac{1}{N_R} \nsum_{r=1}^{N_R} D_n^{(r)}
    \label{eq:random volume sources}
\end{equation}
%
Here in Eqn.\eqref{eq:random volume sources}, $D_n^{(r)}$ is the estimate of $D_n$ in the $r^{th}$ random volume source and $N_R$ is the total number of such random volume sources. One also needs to extract the real part of this complex exponential in Eqn.\eqref{eq:resummed partition function} for determining the estimate of $Z_R^N$ preserving the CP symmetry of QCD. In this limit, the above Eqn.\eqref{eq:resummed partition function} therefore resembles

\begin{equation}
    Z_N^{\text{R}} = \text{Re} \,\LA \left[\exp \left(\nsum_{n=1}^N \left(\frac{\mu}{T}\right)^n \frac{\overline{D_n}}{n!}\right) \right] \RA
    \label{eq:modified resummed partition function}
\end{equation}
%%
where $\overline{D_n}$ is provided in Eqn.\eqref{eq:random volume sources}. Using finite $N_R$ results in stochastic bias in usual exponential resummation formula as well as provides biased estimates of different $D_n$ for every gauge configuration.
 A detailed description about this stochastic bias can be found in Ref.\,\cite{Mitra:2022cumu}. Although this bias decreases as $N_R^{-1}$, it can be very significant depending on the observable probed and the value and order of $\mu$ under consideration. It becomes highly imperative to eliminate this bias which can hinder genuine transparent understanding of underlying physics for finite-density QCD.  


\subsection{Unbiased formalism}

\hs The unbiased exponential resummation is formulated to eliminate this stochastic bias. In this formalism, the original structure of the resummation is retained with subtle modification of the argument of exponential. This is done so that on expansion in $\mu$, it produces unbiased estimates of $D_n$ and thereby reproduces Taylor series exactly to a given order in $\mu$. This is true for any flavor of chemical potential, although we have implemented this for $\muB$ and $\muI$ in this paper. For both $\muB$ and $\muI$, the unbiased formalism provides the following expression of the excess pressure $\Delta P/T^4$:

\begin{align}
    \frac{\Delta P_N^{\text{u}}(T,\mu)}{T^4} &= \frac{1}{VT^3} \ln Z_N^{\text{u}}(T,\mu) \hspace{.3cm}\text{where}\notag \\ 
    Z_N^{\text{u}}(T,\mu) &= \text{Re} \,\LA \left[\exp \left(\nsum_{n=1}^N \left(\frac{\mu}{T}\right)^n \frac{\overline{D_n^{\text{u}}}}{n!}\right) \right] \RA
    \label{eq:unbiased exponential resummation}
\end{align}
%
The above Eqn.\eqref{eq:unbiased exponential resummation} resembles Eqn.\eqref{eq:modified resummed partition function} with the notable exception that in the unbiased formalism, the argument of exponential in the expression of the unbiased partition function $Z_N^{\text{u}}$ comprises the unbiased estimates $\overline{D_n^{\text{u}}}$ of $D_n$ as the coefficient of $\mu^n$. A detailed proof along with other mathematical details of this formalism are presented in \cite{Mitra:2023unb}.


\subsection{Idea of complex isospin chemical potential and associated phasefactor}

The formalism of exponential resummation provides another significant entity which is the phasefactor. The gauge ensemble averaged phasefactor reflects the degree of $\mu$-dependent oscillations of $\det \M$. Close to the zeros of this average phasefactor, the oscillations become highly severe and may often lead to breakdown of calculations, rendering the formalism unreliable. Thus the manifestation of these zeros and their onset are interesting since they can provide a good estimate about the commencement of a possible breakdown. As mentioned regarding Eqn.\eqref{eq:modified resummed partition function}, the exponential of the complex polynomial comprising purely real (imaginary) $D_n$ for even (odd) values of $n$ yield a complex exponential. So for a complex function $z(T,\mu)$, this can be written as $e^z = R\,e^{i\theta}$ where

\begin{align*}
    R\left(T,\mu\right) = \exp{\bigg[\text{Re}\Big(z\left(T,\mu\right)\Big)\bigg]},\, \theta\left(T,\mu\right) = \text{Im}\bigg[z\left(T,\mu\right)\bigg]
\end{align*}

In the case of $\muB$, the exponential is complex for real values of $\muB$ and since one needs to extract the real part of the exponential, the phasefactor in this case is given by $\cos{\theta}$ for every gauge configuration. Hence in this situation we compute and observe the behaviour of $\LA \cos{\theta} \RA$ as a function of $\muB$. 

However for $\muI$, the exponential is always real for real $\muI$ since $D_n$ vanishes for odd $n$. Since $\theta(T,\mu)$\footnote{$\theta(T,\mu) = \sum_{n=1}^N (\frac{\mu}{T})^{2n-1} \text{Im}(D_{2n-1})$} only depends on odd $D_n$, hence $\theta=0$ for every gauge configuration for $\muI$, making $\cos{\theta}=1$. 
This makes identifying the breakdown in case of $\muI$ difficult by inspecting the behaviour of $\LA \cos{\theta} \RA$ for real $\muI$, just like $\muB$. In this paper, we propose to make $\muI$ complex which will make the resummation exponential complex. This will then provide us with a non-trivial phasefactor which will vary for various $\muI$ spanning the complex $\muI$ plane. Since CP symmetry does not guarantee partition function to be real for complex values of chemical potential, we observe the behaviour of the entire complex phasefactor $\LA e^{i\theta} \RA$ for complex $\muI$ as opposed to observing just $\LA \text{Re}(e^{i\theta}) \RA = \LA \cos{\theta} \RA$ for real $\muB$. The two dimensional phasefactor plots for $\muI$ have been constructed in this paper, by plotting the absolute value of $\LA e^{i\theta} \RA$ i.e. $\left|\LA e^{i\theta} \RA\right|$ as a function of  $\left|\muI\right|$.


\section{Overlap problem and its severity}
\label{sec:Overlap problem}

\hs In this section, we give a brief introduction to the overlap problem that becomes predominant in computations involving real $\muI$. Despite the abscence of sign problem in $\muI$, the calculations experience a genuine breakdown beyond a finite radius of convergence $\rho$.  


 While generating an ensemble of fermion and gauge field configurations at finite $\muI$ based on extrapolations from the ensemble generated at $\muI=0$, this problem arises
 when the distribution or sample comprising the ratio of fermion determinants at finite $\muI$ to zero $\muI$ i.e. $\det \M(\muI)/\det \M(0)$ becomes heavily tailed. This large tail of the distribution causes Monte-Carlo importance sampling ineffective, and renders reweighting approach inefficient in this limit. One comes across this ratio while reweighting the integrand of the path integral given in Eqn.\,\eqref{eq:partition function}, and this ratio assumes different values for different gauge configuration ensembles. In the realm of reweighting and exponential resummation where $\left|\muI\right| < \rho$, this ratio for a gauge configuration $U$ can be expressed as 

\begin{equation}
    \frac{\det \M(\muI,U)}{\det \M(0, U)} = \exp{\left[\nsum_{n=1}^{\infty} \muI^n \frac{D_n(U)}{n!}\right]}, \hs D_n(U) = \frac{\partial^n }{\partial \muI^n}\ln \det \M(\muI,U)\bigg|_{\muI=0}
    \label{eq:reweighting}
\end{equation}
%
The volume and temperature dependence of fermion determinant have been suppressed in the above Eqn.\eqref{eq:reweighting}, for the given gauge configuration $U$. These values of the above ratio for a given $\muI$ and for all such $U$ in the ensemble constitute the working sample or distribution and its tail characterises the magnitude and extent of overlap problem. A heavily tailed distribution\footnote{These distributions have large sample variance and often the sample mean is drastically different from the population mean.} will have greater extent of overlap problem. This is because in these distributions, the values radically different from the distribution mean, manifest with appreciable probability and this trait tends to change the sample statistics by a large extent. Although standard deviation can prove to be a reliable estimate characterising the heavy tail, a better quantitative measure is kurtosis $\kappa$ which is the standardised fourth order central moment. For a total of $N$ gauge field configurations, this is represented by:

\begin{equation}
    \kappa(\muI) = \frac{M_4^{\bar{x}}(\muI)}{(\sigma(\muI))^4}
\end{equation}
%
where $M_4^{\bar{x}}$ is the fourth order central moment and $\sigma$ is the standard deviation of the distribution with mean $\bar{x}$. These are defined as
\begin{align}
    M_4^{\bar{x}} = \frac{1}{N} \nsum_{i=1}^N (x_i-\bar{x})^4, \hspace{5mm} \sigma &= \left[\frac{1}{N} \nsum_{i=1}^N (x_i-\bar{x})^2\right]^{\frac{1}{2}}, \hspace{5mm} \bar{x} = \frac{1}{N} \nsum_{i=1}^N x_i
\end{align}
%
where $x_i = \left[\det \M(\muI)/\det \M(0)\right]_i$ is the value of the fermion determinant ratio obtained from $i^{th}$ configuration.
%
% given by 
%
% \begin{equation*}
%     x_i = \left[\frac{\det \M(\muI,U)}{\det \M(0, U)}\right]_i
% \end{equation*}
%
The manifestation of this overlap problem however may be different for different formalism which are adopted suitably for subsequent calculations for probing finite density QCD regime.
%\footnote{$\left|\muI\right| = \sqrt{x^2+y^2}$, $x=\text{Re}(\muI), y=\text{Im}(\muI)$}

%This formalism is performed in two bases namely,

% \vspace{.5cm}

% \subsubsection*{Chemical potential basis}

%  For real $\mu$ in chemical potential basis, the unbiased exponential resummation resembles:
%     \begin{align}   %  6
%      \frac{\Delta P_{N}^{R(\text{unb})}(T,\mu)}{T^4} &= \frac{1}{VT^3} \hspace{.5mm}\ln \hspace{1mm}\left\{ \text{Re} \left \langle \bigg[\exp \Big(A_N\left(T,\mu\right)\Big)\bigg] \right \rangle \right\} , \, \text{where} \notag \\ 
%     A_N\left(T,\mu\right) &= \nsum_{n=1}^{N} \frac{\mathcal{C}_{n}(T)}{n!}\,\left(\frac{\mu}{T}\right)^n 
%      \label{eq:mu basis}
%     \end{align}  %  6
%       %\\[0.2cm]
%       where the first four $\mathcal{C}_n (T)$ are given as follows:

% %&\hspace{3mm}
 
%  %\begin{widetext}
% \begin{align}
%     &\mathcal{C}_1 = \overline{D_1}, \notag \\
%     &\mathcal{C}_2 = \overline{D_2} + \left[\overline{(D_1)^2} - \overline{(D_1)}^2\right], \notag \\
%     &\mathcal{C}_3 = \overline{D_3} + 3\left(\overline{D_2 D_1} - \overline{D_2}\;\overline{D_1}\right) + 
%            \left(\overline{D_1^3} - 3\,\overline{(D_1)^2}\;\overline{D_1} + 2\,\left(\overline{D_1}\right)^3\right), \notag \\
%     &\mathcal{C}_4 = \overline{D_4} + 3\left(\overline{(D_2)^2} - \left(\overline{D_2}\right)^2\right)+ 4\left(\overline{D_3 D_1} - \overline{D_3}\;\overline{D_1}\right) + 6\left( \overline{D_2 (D_1)^2} - \overline{D_2}\;\overline{(D_1)^2}\right)  \notag \\
%     &- 12 \left(\overline{D_2 D_1}\;\overline{D_1} - \overline{D_2}\left(\overline{D_1}\right)^2\right) +
%      \Big(\overline{(D_1)^4} - 4\,\,\overline{(D_1)^3}\;\overline{D_1} + 12\,\overline{(D_1)^2}\left(\overline{D_1}\right)^2 \notag \\
%      &\hspace{6cm}- 6\left(\overline{D_1}\right)^4 - 3\,(\overline{(D_1)^2})^2\Big)
% \label{eq:mu coefficients}
% \end{align}
% %
%  Here in Eqn.\,\eqref{eq:mu coefficients}, $\overline{D_n^p}$ indicates average of $p^{th}$ unbiased power of\footnote{In this analysis of $\muI$, these $D_n$ are isospin correlation functions.} $D_n$ over all the $N_{R}$ random volume sources contained within a given gauge field configuration. Eqn.\,\eqref{eq:mu basis} along with Eqn.\,\eqref{eq:mu coefficients} eliminates stochastic bias upto $\Ob(\mu^4)$. In the forthcoming discussions, these $D_n$ are used to denote isospin correlation functions only.


% %\vspace{.5cm}

% \subsubsection*{Cumulant basis}

% For real $\mu$ in cumulant basis, this formalism looks as follows:

%  \begin{align}   %  6
%      &\frac{\Delta P_{N,M}^{R(\text{unb})}(T,\mu)}{T^4} = \frac{1}{VT^3} \hspace{1mm}\ln \hspace{.5mm} \left\{\text{Re} \LA \bigg[\exp \Big(W_M\big[X_N\left(T,\mu\right)\big]\Big)\bigg] \RA\right\}, \, \text{where} \notag \\ 
%      &W_M\big[X_N\left(T,\mu\right)\big] = \nsum_{m=1}^{M} \frac{\mathcal{L}_{m}\left(X_N\right)}{m!} \, ,  \hspace{3mm} X_N(T,\mu) = \sum_{n=1}^N \left(\frac{\mu}{T}\right)^n \frac{D_n}{n!}
%       \label{eq:cumulant basis}
%       \end{align}
% %
% The $\L_m$ for $1 \leq m \leq 4$ in the expression of $W_M$ of above Eqn.\eqref{eq:cumulant basis} are as follows:

% \begin{align}
%    \L_1 &= \overline{X_N}, \notag \\
%    \L_2 &= \overline{(X_N)^2} - \big(\overline{X_N}\big)^2, \notag \\
%    \L_3 &= \overline{(X_N)^3} - 3\,\big(\overline{X_N}\big)\;\overline{(X_N)^2} + 2\,\big(\overline{X_N}\big)^3, \notag \\
%    \L_4 &= \overline{(X_N)^4}- 4\,\overline{(X_N)^3}\;\big(\overline{X_N}\big)  
%          + 12\,\big(\overline{X_N}\big)^2\;\overline{(X_N)^2} - 6\,\big(\overline{X_N}\big)^4- 3\,\big(\overline{(X_N)^2}\big)^2
% \label{eq:cumulant L}
% \end{align}
% %
% Implementing the formalism in this basis following Eqn.\,\eqref{eq:cumulant basis}, using the expressions of Eqn.\,\eqref{eq:cumulant L} leads to reproducing the first $4$ cumulants exactly of unbiased cumulant expansion, where the cumulants are expressed in terms of unbiased powers. We observed that this formalism not only eliminates stochastic bias upto a finite order in $\mu$, it can also capture higher order Taylor contributions successfully. On adding more number of $\L_n$ or $\mathcal{C}_n$, one approaches exponential resummation unbiased to all orders in $\mu$ which in infinite series limit, replicates infinite Taylor series in $\mu$. The unbiased exponential resummation used in this work eliminates stochastic bias upto $\muI^4$ using $C_n$, for $n \leq 4$ in chemical potential basis and $M=4$ in cumulant basis for both second and fourth order calculations. 

% \subsection{Phasefactor analysis in unbiased exponential resummation}

% \hs When chemical potential $\mu$ is complex, the partition function $Z$ also becomes complex\footnote{The CP symmetry guarantees $Z$ to be real, only for real $\mu$.} unlike Eqns.\,\eqref{eq:mu basis} and \eqref{eq:cumulant basis}. As a result, this partition function can then be expressed in familiar polar amplitude form $R(T,\mu)\,e^{i\Theta(T,\mu)}$, where $R(T,\mu)$ is the real-valued positive definite phase-quenched reweighting factor and $e^{i\Theta}$ is the complex phasefactor, with real-valued phaseangle $\Theta(T,\mu)$. 

% \hspace{-.9cm} For chemical potential basis, 
% \begin{equation}
%     R_N(T,\mu) = \exp{\bigg[\text{Re}\Big(A_N\big(T,\mu\big)\Big)\bigg]}, \hspace{.5cm} \Theta_N(T,\mu) = \text{Im}\bigg[A_N\big(T,\mu\big)\bigg]
%     \label{eq:mu_phasefactor}
% \end{equation}
% %
% In case of cumulant basis,
% \begin{equation}
%     R_{N,M}(T,\mu) = \exp{\Bigg[\text{Re}\bigg(W_M\Big[X_N\big(T,\mu\big)\Big]\bigg)\Bigg]}, \hspace{.5cm} \Theta_{N,M}(T,\mu) = \text{Im}\bigg(W_M\Big[X_N\big(T,\mu\big)\Big]\bigg)
%     \label{eq:cumulant_phasefactor}
% \end{equation}

% In this paper, we perform analysis for finite $\muI$ and investigate the breakdown of unbiased exponential resummation. This is performed by observing the behaviour of thermodynamic observables like isospin excess pressure and isospin susceptibility across the radius of convergence estimated from the numerical determination of Lee-Yang zeros of $Z$ through Newton-Raphson iteration method. This estimate of radius of convergence is then compared with that indicated by the close-to-zero values of $\left|e^{i\Theta}\right|$. 

% Unlike baryon chemical potential $\muB$ or chemical potentials having a sign problem, there is no such problem for $\muI$ and so, the average phasefactor $\LA \cos{\Theta} \RA$ is unity for all values of real $\muI$. This makes this measure of phasefactor unsuitable for identifying and locating the breakdown, along real $\muI$. 
% The idea therefore is to make $\muI$ complex, which will make the reweighting factor, and partition function $Z$ complex. The average complex phasefactor $\LA e^{i\Theta} \RA$ is computed as a function of complex $\muI$ and we observe if we can conclude something about the breakdown of lattice computations along real $\muI$ from its behaviour. 

\section{Setup of calculations}
\label{sec:setup}

\hs In this work, we have extensively used the data generated by the HotQCD collaboration for its ongoing Taylor expansion calculations and charge fluctuations. We discuss the setup and other important relevant details of this data, in this section. 

The QCD action considered for generating the working data of the calculations in this work follows a $2+1$ flavor signature in which the strange quark is $27$ times more massive than the mass degenerate up and down quarks. This action comprises a Symanzik-improved gauge action\,\cite{Symanzik:1983gauge,Symanzik:1983gauge2nd}
and the Highly Improved Staggered Quark (HISQ) fermion action\,\cite{Gabrielli:1990Improvement,MILC:2008HISQ,MILC:2010HISQ}. Gauge field configurations of order $\Ob(10^4$ - $10^6)$ are generated in the temperature range $135$~MeV~$\lesssim~T~\lesssim$~$176$~MeV with $\Nt=8$, $12$ and $16$ and $\Ns=4\Nt$. In this work, an isotropic lattice of size $32^3 \cdot 8$ has been used in Euclidean four spacetime, which is Wick rotated from the usual $3+1$ relativistic Minkowski spacetime. Following the relation $T=(a\Nt)^{-1}$, the temperature for each $\Nt$ is varied by varying the isotropic lattice spacing $a$ through the inverse gauge coupling\footnote{$\beta=6/g^2$ where $g$ is the QCD coupling parameter.} $\beta$. For each $a$, the bare light and strange quark masses $m_l(a)$ and $m_s(a)$ are also tuned so that the pseudo-Goldstone pion and kaon masses produced become equal to the physical pion ($\pi$) and kaon ($K$) masses respectively. This fixes the line of constant physics for the lattice setup under consideration. The scale setting is determined using both the Sommer parameter $r_1$ and the kaon decay constant $f_K$. A complete description of the gauge ensembles and scale setting is provided in Ref.~\,\cite{Bollweg:2021vqf}.

To calculate the Taylor coefficients for unbiased exponential resummation, the baryon and isospin correlation functions $D_1,\dots,D_4$ are estimated stochastically using $500$ Gaussian volume sources on each gauge configuration. For a detailed derivation of these $D_n$, refer to \cite{Allton:2005Taylor}. The exponential-$\mu$ formalism is used to calculate these four derivatives, and further higher derivatives are calculated using the linear-$\mu$ formalism. Using this data, we have calculated the necessary following results to be discussed vividly in the next section for $\muB$ and $\muI$ using the methods of Taylor expansion and unbiased exponential resummation. These are computed in the range $0 \leqslant \lvert \mu_{I}/T \rvert \leqslant 2.5$, using $100K$ configurations for $\muB$ and $20K$ configurations per temperature for $\muI$. Our results have been obtained on $\Nt=8$ lattices for three temperatures namely at $T \sim 135$, $157$ and $176$ MeV. Besides describing the hadronic, crossover and QGP phases of QCD phase diagram, these temperatures have been chosen carefully as being approximately equal to $\Tpc$ and $\Tpc\pm20$~MeV, where $\Tpc=156.5(1.5)$~MeV is the chiral crossover temperature at zero baryon chemical potential $\muB$ for physical values of bare quark masses\,\cite{Steinbrecher:2018phh}. The same temperatures have been chosen as the working temperatures for $\muI$ also. 

In calculations involving unbiased exponential resummation, we have considered taking $100$ bootstrap samples of the working gauge configuration ensemble. This bootstrapping algorithm used, is based on a chosen random number generator and is implemented to calculate the errorbars associated with values of the observables for each value of $\muI$ considered. In a separate section of this paper, we have explained the setup and characteristics of the Newton-Raphson method which we have used to estimate the zeros of the partition function and identify the manifestation of a possible breakdown in calculations.


\section{Results}
\label{sec:Results}


\subsection{For baryon chemical potential}

In this section, we present a comparative study between the measures of baryon cumulants of sixth and eighth orders obtained using the usual Taylor Expansion and the measures of the same that appear on expansion of the unbiased exponential resummation formula. The Taylor coefficients\footnote{Taylor coefficient is just the charge cumulant scaled by appropriate factorial.} using the Taylor expansion are procured from the different unbiased powers of the corresponding baryon or isospin correlation functions, as given in Eqn.\eqref{eq:Taylor coeff and correlation funs}. 
As mentioned before, the manifestation of the stochastic bias has already been observed in the old exponential resummation formula\,\cite{Mondal:2021exponentialresummation} and not only we have understood the importance of eliminating this bias\,\cite{Mitra:2022cumu}, we also have come up with the unbiased exponential resummation\,\cite{Mitra:2023unb} which replicates the exact Taylor coefficients corresponding to the appropriate powers of $\mu$. Thus within its all-order series, it reproduces Taylor series upto a desired order in $\mu$ which leads to a more improved QCD Equation of state and enabling one to comprehend the true physics of finite-density QCD, to a more reliable extent. 


% As mentioned before, the formalism of the unbiased exponential resummation replicates the exact Taylor coefficients corresponding to the appropriate powers of $\mu$ and consequently produces Taylor series upto a desired order in $\mu$, which here refers to both $\muB$ and $\muI$. The manifestation of stochastic bias in the old exponential resummation formula\,\cite{Mondal:2021exponentialresummation} has already been observed and as an outcome, we have understood the importance of eliminating this bias\,\cite{Mitra:2022cumu} in obtaining the correct results thereby knowing the underlying genuine physics of finite-density QCD. 




The higher order charge cumulants or coefficients, are particularly of great significance. Owing to the higher powers of $\mu$ associated with these coefficients, their values strongly influence the behaviour of Taylor series for larger values of $\mu$ ($\mu/T>1$), which is instrumental for understanding finite-density QCD. From a computational point of view, it is very difficult and expensive to evaluate these higher order Taylor coefficients as they require calculating higher order correlation functions and also the higher unbiased powers of the required lower-order correlation functions. So, it is very challenging to evaluate these Taylor coefficients precisely and therefore would be promising if they can be ascertained using some other alternative means, from which they can be obtained with similar precision at the expense of relatively less computational complexities. In this paper, we attempt to testify this using unbiased exponential resummation of second and fourth orders for finite $\muB$ and $\muI$ and compare the subsequent results between these two approaches. 

\begin{figure}[H]
    \centering
    \includegraphics[width=.47\textwidth]{figures/CHI_compar_B_135_c6_Taylor_vs_unb.pdf}
    \quad
    \includegraphics[width=.47\textwidth]{figures/CHI_compar_B_135_c8_Taylor_vs_unb.pdf}
    \caption{Plots of sixth and eighth order charge cumulants $\chi_6^B$ (left) and $\chi_8^B$ (right) obtained using the Taylor expansion and also from the unbiased exponential resummation of second, fourth and sixth orders at $T=135$ MeV. The green points indicate the usual Taylor estimate whereas the red, blue and black lines represent the corresponding coefficients for second, fourth and sixth orders respectively.}
    \label{fig:chi 135_B sixth and eighth}
\end{figure}

Fig.\ref{fig:chi 135_B sixth and eighth} represents the plots of $\chi_6$ and $\chi_8$ obtained using the usual Taylor Expansion method and the unbiased exponential resummation approach. The $\cnBT$ depicts the estimate of $\chi_n^B$ acquired using Taylor expansion, whereas $\cnBm$ represents the same procured from unbiased exponential resummation of order $m$. The subsequent similar plots in the paper follow this nomenclature of symbols only, even for the plots of $\muI$ in the later section of the paper. While estimating $\cnBm$, all the correlation functions $D_p,\, 1 \leq p \leq m$ are included and all the higher correlation functions are excluded i.e. $D_p=0$ for $p \geq (m+1)$. So, it is quite evident that the sixth order unbiased exponential resummation contains the necessary unbiased powers of $D_n$ upto $n=6$ and hence in principle, $\csBs =\csBT$ (Eqn.\eqref{eq:Taylor coeff and correlation funs}). This is clearly demonstrated in the left plot of Fig.\ref{fig:chi 135_B sixth and eighth} where the green and the black lines representing respective $\csBT$ and $\csBs$ matches exactly, with the errorbars also in perfect alignment with each other. While there is a difference between the values and errorbars of $\csBT$ and $\csBt$ as shown by the red points, the blue points representing $\csBf$ clearly reveal that $\csBf$ obtained from the fourth order resummation remains in a very good agreement with $\csBT$. This is despite not including the baryon correlation functions $D_5$ and $D_6$ in the estimate of $\csBf$ unlike $\csBT$. 
In the case of $\chi_8^B$, all the estimates obtained from the second, fourth and sixth order resummations are more or less in a very good and commendable agreement with the Taylor estimate at $T=135$ MeV. This is quite promising in the sense that one can only estimate $D_1^B$ and $D_2^B$ and can safely rely on adopting a second order unbiased exponential resummation approach to evaluate the estimate of $\chi_8^B$ without too much of requiring to evaluate higher correlation functions from $D_3^B$ onwards. As implied by Fig.\ref{fig:chi 135_B sixth and eighth}, the resulting $\ceBt$ obtained would not differ drastically from $\ceBT$, which is the corresponding Taylor series estimate of $\chi_8^B$.

\begin{figure}[H]
    \centering
    \includegraphics[width=.47\textwidth]{figures/CHI_compar_B_157_c6_Taylor_vs_unb.pdf}
    \quad
    \includegraphics[width=.47\textwidth]{figures/CHI_compar_B_157_c8_Taylor_vs_unb.pdf} \\
    \includegraphics[width=.47\textwidth]{figures/CHI_compar_B_176_c6_Taylor_vs_unb.pdf}
    \quad
    \includegraphics[width=.47\textwidth]{figures/CHI_compar_B_176_c8_Taylor_vs_unb.pdf}
    
    \caption{Plots of sixth and eighth order charge cumulants $\chi_6^B$ (left column) and $\chi_8^B$ (right column) obtained using the Taylor expansion and also from the unbiased exponential resummation of second, fourth and sixth orders. These are obtained at $157$ (top row) and $176$ MeV (bottom row) respectively.}
    \label{fig:chi 157_B and 176_B sixth and eighth}
\end{figure}

 The same argument also holds true for the other two temperatures at $157$ and $176$ MeV respectively as shown in the Fig.\ref{fig:chi 157_B and 176_B sixth and eighth}. In both Fig.\ref{fig:chi 135_B sixth and eighth} and Fig.\ref{fig:chi 157_B and 176_B sixth and eighth}, we observe that although the second order resummation estimates $\csBt$ and $\ceBt$ of $\chi_6^B$ and $\chi_8^B$ exhibit some noticeable difference from the corresponding Taylor estimates $\csBT$ and $\ceBT$, it is from the fourth order onwards that the resummation estimate values for both $\chi_6$ and $\chi_8$ along with errorbars becomes almost perfectly identical and agrees appreciably well with the corresponding Taylor counterparts. This is found by comparing the blue and black points with the green point in Figs.\ref{fig:chi 135_B sixth and eighth} and \ref{fig:chi 157_B and 176_B sixth and eighth} where the former pair of points represent the fourth and sixth order resummation estimates and the latter illustrates the Taylor estimate.
 
 More importantly, this set of observations holds true for all the three working temperatures; $135$, $157$ and $176$ MeV which have been chosen so that they represent hadronic, crossover and plasma phases of the QCD phase diagram for physical quark and pion masses, and thereby span a major part of the phase diagram. This implies that the calculation of unbiased estimates upto $D_4$ for $\muB$ or equivalently upto $\mathcal{O}(\muB^4)$ is good enough and sufficient to reproduce Taylor series upto $\mathcal{O}(\muB^8)$ irrespective of whether or not, the stochastic bias from $D_5$ is taken care of. The question of whether this remains true for even higher powers of $\muB$ beyond $\muB^8$ is certainly a very interesting and coveted work for the future and if found true, it can constrain our tedious efforts to introduce unbiased estimates to higher orders, to a great extent. One may attribute this to the fact that the ultraviolet divergences remain in lattice calculations only upto the fourth power of $\mu$ or $\mathcal{O}(\mu^4)$, for $\mu \equiv \muB, \muI$. To cancel out these divergences, one needs to adopt the full exponential $\mu$ formalism and use all the resulting terms to evaluate all $D_n$ upto $D_4$. Beyond this that is from $D_5$ onwards, the linear $\mu$ formalism utilising only the first order $\mu$ derivatives of $\M$ suffices. Although the exponential $\mu$ formalism continues to remain valid in this domain also, one prefers to embrace the linear $\mu$ formalism for its calculational simplicity and less computational time. Owing to the prescence of these divergences, stochastic random volume estimates of $D_n,\,n \leq 4$ fluctuate a lot among the different random sources thereby contributing maximally to the resulting stochastic bias in the final results.
So for obtaining physically meaningful and genuinely reliable results, one not only must eliminate these divergences but also needs to calculate the unbiased powers of these stochastically estimated correlation functions and then using their subsequent values in the final computations. 


Apart from the estimates of these individual coefficients, we also compare and present results regarding the different estimates of radius of convergence obtained from the Taylor expansion and the unbiased exponential resummation methods. The radius of convergence of a Taylor series determines the value of the parameter of expansion ($\mu$ in this case) upto which the series approximation provides reliable results and hence, it is also significant for identifying the possible breakdown of calculations using Taylor series approximations. As stated before, owing to the CP symmetry or particle-antiparticle symmetry of QCD, the Taylor series of QCD excess pressure (see Eqn.\eqref{eq:Taylor excess pressure}) is even in $\mu \equiv \muB/\muI$ for which the radius of convergence $\rho$ in principle is given by 

\begin{equation}
    \rho = \lim_{n \to \infty} \rho_n ,\hspace{2mm} \text{where} \hspace{2mm} \rho_n=\sqrt{\frac{\chi_{2n}}{\chi_{2n+2}}}
    \label{eq:est of rad of conv}
\end{equation}
%
Although the radius of convergence is introduced mathematically in the context of Taylor series only, we have measured the different estimates of this radius of convergence $\rho$ in unbiased exponential resummation too. This is because on expansion of this resummed series in terms of $\mu$, it resembles a Taylor series whose resulting coefficients differ from the usual Taylor coefficients, depending on the resummation order and also on the given order of $\mu$. In this work, this has been done by measuring the appropriate charge cumulants of different orders using this resummation method and following Eqn.\eqref{eq:est of rad of conv}.



\begin{figure}[H]
    \centering
    \includegraphics[width=.325\textwidth]{figures/ROC_compar_B_135_all_Taylor_vs_unb.pdf}
    \includegraphics[width=.325\textwidth]{figures/ROC_compar_B_157_all_Taylor_vs_unb.pdf} 
    \includegraphics[width=.325\textwidth]{figures/ROC_compar_B_176_all_Taylor_vs_unb.pdf}
    
    \caption{Plots of the estimate of $\rho_n=(\chi_{2n}/\chi_{2n+2})$ with $n=1,2,3$ for $\muB$ obtained at $135$ (left), $157$ (center) and $176$ MeV (right). The green line represents the Taylor estimate of the above whereas the red, blue and black lines depict the same for unbiased exponential resummation approach upto second, fourth and sixth orders respectively.}
    \label{fig:all ROC 135_B 157_B and 176_B}
\end{figure}


In this work, we have measured three estimates of the radius of convergence using this unbiased exponential resummation. These estimates are $\rho_1$, $\rho_2$ and $\rho_3$ which are obtained by inserting $n=1,2,3$ in the Eqn.\eqref{eq:est of rad of conv}. 
Since the highest order Taylor series used in this work is of eighth order, the best estimate of radius of convergence is $\rho_3 = (\chi_{6}/\chi_{8})^{1/2}$. In the Fig.\ref{fig:all ROC 135_B 157_B and 176_B}, we have presented the plots of different estimates of $\rho$ as given in Eqn.\eqref{eq:est of rad of conv} for $n=1,2,3$. As before, these have been compared after obtaining them using the Taylor expansion approach and also through the method of unbiased exponential resummation for second, fourth and sixth orders in $\muB$.
The different estimates $\rho_1, \rho_2$ and $\rho_3$ of the radius of convergence have been plotted at $135$, $157$ and $176$ MeV respectively. We clearly find that although the second order resummation estimates $(\rho_n^B)_2$ for $n=1,2,3$ shown by the red points in Fig.\ref{fig:all ROC 135_B 157_B and 176_B} vary to some appreciable extent against the corresponding Taylor estimates $(\rho_n^B)_T$ illustrated by the green points, the agreement becomes appreciable and very much perfect from the fourth order onwards. Importantly, this holds true for all the three estimates $\rho_1$, $\rho_2$ and $\rho_3$ in which the agreement in $n=1$ for all $\rho$ except $(\rho_n^B)_2$ is trivial. This is because in second order resummation, $D_3$ and $D_4$ are excluded which are required otherwise for a complete estimation of $\chi_4^B$. This agreement in $\rho$ is somewhat expected to follow consequently, after our previous findings regarding Figs.\ref{fig:chi 135_B sixth and eighth} and \ref{fig:chi 157_B and 176_B sixth and eighth}. 

% \begin{figure}[H]
%     \centering
%     \includegraphics[width=.48\textwidth]{figures/ROC_compar_B_135_all_with_phasefactor_Taylor_vs_unb.pdf}
%     \includegraphics[width=.48\textwidth]{figures/ROC_phasefactor_B_135.pdf} \\
%     \includegraphics[width=.48\textwidth]{figures/ROC_compar_B_157_all_with_phasefactor_Taylor_vs_unb.pdf}
%     \includegraphics[width=.48\textwidth]{figures/ROC_phasefactor_B_157.pdf} \\
%     \includegraphics[width=.48\textwidth]{figures/ROC_compar_B_176_all_with_phasefactor_Taylor_vs_unb.pdf}
%     \includegraphics[width=.48\textwidth]{figures/ROC_phasefactor_B_176.pdf}
    
%     \caption{(Left column) Plots of the estimate of $\rho_1, \rho_2$ and $\rho_3$ for $\muB$ obtained at $135$ (top row), $157$ (middle row) and $176$ MeV (bottom row). (Right column) Plots of $\LA \cos \theta \RA$ of second and fourth orders as a function of $\muB$, for $135$ (top row), $157$ (middle row) and $176$ MeV (bottom row). For the radius of convergence plots, the green line represents the Taylor estimate of the above whereas the red, blue and black lines depict the same for unbiased exponential resummation approach utilised upto second, fourth and sixth orders respectively. For the plots of phasefactor, the red and black lines illustrate the second order and fourth order phasefactor respectively. The blue line is the zero line in these phasefactor plots, whereas the magenta line coincides with the value of $\muB$ from where $\LA \cos \theta \RA=0$.}
%     \label{fig:ROC 135,157,176_B all with phasefactor}
% \end{figure}


\begin{figure}[H]
    \centering
    \includegraphics[width=.325\textwidth]{figures/ROC_compar_B_135_all_with_phasefactor_Taylor_vs_unb.pdf}
    \includegraphics[width=.325\textwidth]{figures/ROC_compar_B_157_all_with_phasefactor_Taylor_vs_unb.pdf}
    \includegraphics[width=.325\textwidth]{figures/ROC_compar_B_176_all_with_phasefactor_Taylor_vs_unb.pdf} \\
    \includegraphics[width=.325\textwidth]{figures/ROC_phasefactor_B_135.pdf}
    \includegraphics[width=.325\textwidth]{figures/ROC_phasefactor_B_157.pdf} 
    \includegraphics[width=.325\textwidth]{figures/ROC_phasefactor_B_176.pdf}
    
    \caption{(Top row) Plots of $\rho_1^B,\rho_2^B$ and $\rho_3^B$ at $135$ (left), $157$ (middle) and $176$ MeV (right) respectively. (Bottom row) Plots of $\LA \cos \theta \RA$ as a function of $\muB$ for second and fourth orders plotted at $135$ (left), $157$ (middle) and $176$ MeV (right) respectively. The red and black points in the phasefactor plots illustrate $\LA \cos \theta \RA$ to second and fourth orders respectively with the blue line indicating the zero line. The magenta line in the top row plots illustrate the $\muB$ where the first zero of $\LA \cos \theta \RA$ manifests.}
    \label{fig:ROC 135,157,176_B all with phasefactor}
\end{figure}

% \begin{figure}[H]
%     \centering
%     \includegraphics[width=.48\textwidth]{figures/ROC_compar_B_135_all_with_phasefactor_Taylor_vs_unb.pdf}
%     \includegraphics[width=.48\textwidth]{figures/ROC_compar_B_157_all_with_phasefactor_Taylor_vs_unb.pdf} \\
%     \includegraphics[width=.48\textwidth]{figures/ROC_compar_I_135_all_with_phasefactor.pdf}
%     \includegraphics[width=.48\textwidth]{figures/ROC_compar_I_157_all_with_phasefactor.pdf}
    
%     \caption{Plots of the estimate of $(\chi_6/\chi_8)$ for $\muB$ obtained at $135$ (left), $157$ (center) and $176$ MeV (right). The green line represents the Taylor estimate of the above whereas the red, blue and black lines depict the same for unbiased exponential resummation approach utilised upto second, fourth and sixth orders respectively.}
%     \label{fig:ROC 135_B,I and 157_B,I all with phasefactor}evident from the fact that the agreement between the two approaches is excellent from the level of individual Taylor coefficients, which has been observed before during the discussion of
% \end{figure}



% \begin{figure}[H]
%     \centering
%     \includegraphics[width=.48\textwidth]{figures/ROC_compar_B_176_all_with_phasefactor_Taylor_vs_unb.pdf}
%     \includegraphics[width=.48\textwidth]{figures/ROC_compar_I_176_all_with_phasefactor.pdf} 
    
%     \caption{Plots of the estimate of $(\chi_6/\chi_8)$ for $\muB$ obtained at $135$ (left), $157$ (center) and $176$ MeV (right). The green line represents the Taylor estimate of the above whereas the red, blue and black lines depict the same for unbiased exponential resummation approach utilised upto second, fourth and sixth orders respectively.}
%     \label{fig:ROC 176_B,I all with phasefactor}
% \end{figure}





We also present our observations regarding the behaviour of the gauge ensemble averaged phasefactor $\LA \cos \theta \RA$ as a function of $\muB$ in the Fig.\ref{fig:ROC 135,157,176_B all with phasefactor}, and subsequently check if the onset of the zeros of this phasefactor happens at values of $\muB$ identical to the acquired estimates of the radius of convergence and thereby can identify the initiation of breakdown of calculations. As introduced before, this phasefactor is $e^{i\theta}$ whose real part $\cos \theta$ needs to be analysed for $\muB$ since the partition function is real for real $\muB$ which we are looking for. This is obtained from the unbiased exponential resummation where the red and black points in the plots of the lower row of Fig.\ref{fig:ROC 135,157,176_B all with phasefactor} indicate that of second order and fourth order respectively. The blue line in the lower row plots is the zero line of the phasefactor and the magenta dotted line in the upper row plots provides the value of $\muB$ from which $\LA \cos{\theta}_n \RA=0$ for $n=2,4$, as can be observed from the lower row plots. From the Fig.\ref{fig:ROC 135,157,176_B all with phasefactor}, it is explicitly observed that the zeros of $\LA \cos \theta \RA$ for both the orders agree very well with each other and also become zero starting from the same value of $\muB$. This is true despite this value being different for all the temperatures. Even though the magenta line lies somewhere between $\rho_2$ and $\rho_3$ at $176$ MeV, it shows appreciably good agreement for $T=135$ and $157$ MeV where it intersects all the three estimates within the errorbars. Significantly for these two temperatures, this line is in good agreement with the estimate of $\rho_3$ which as discussed before, is supposedly the most reliable estimate of radius of convergence out of these three estimates. This thereby demonstrates that for these two temperatures at least, the onset of the zeros of $\LA \cos{\theta}\RA$ give very agreeable values of $\rho$ and hence commendable indications about the onset of breakdown in $\muB$.


% It is interesting to observe this behaviour and its implications regarding the various estimates of the radius of convergence obtained using either the Taylor Expansion or the unbiased exponential resummation. Although for real $\muB$ the reweighting factor is complex thereby giving a non-trivial phasefactor $\LA \cos{\Theta} \RA$ varying with real $\muB$, this is not an option for real $\muI$ where the reweighting factor is purely real. Owing to this, one finds the phasefactor to be trivially unity for all gauge configurations for all values of real $\muI$.
% In this paper besides comparing different estimates with the zeros of $\LA \cos{\Theta} \RA$ for $\muB$, we attempt to complexify $\muI$ where the reweighting factor becomes complex. This then provides us with a complex phasefactor and we conduct a comparative study between the zeros of this gauge ensemble averaged phasefactor with the different estimates for $\muI$.


% \begin{figure}[H]
%     \centering
%     \includegraphics[width=.48\textwidth]{figures/ROC_compar_B_135_all_with_phasefactor_Taylor_vs_unb.pdf}
%     \includegraphics[width=.48\textwidth]{figures/ROC_phasefactor_B_135.pdf} \\
%     \includegraphics[width=.48\textwidth]{figures/ROC_compar_B_157_all_with_phasefactor_Taylor_vs_unb.pdf}
%     \includegraphics[width=.48\textwidth]{figures/ROC_phasefactor_B_157.pdf} \\
%     \includegraphics[width=.48\textwidth]{figures/ROC_compar_B_176_all_with_phasefactor_Taylor_vs_unb.pdf}
%     \includegraphics[width=.48\textwidth]{figures/ROC_phasefactor_B_176.pdf}
    
%     \caption{Plots of the estimate of $(\chi_6/\chi_8)$ for $\muB$ obtained at $135$ (left), $157$ (center) and $176$ MeV (right). The green line represents the Taylor estimate of the above whereas the red, blue and black lines depict the same for unbiased exponential resummation approach utilised upto second, fourth and sixth orders respectively.}
%     \label{fig:ROC 135,157,176_B all with phasefactor}
% \end{figure}


% This is also checked and compared with the zeros of phasefactor procured while performing unbiased exponential resummation, which is vividly illustrated in Fig.\ref{fig:ROC 135,157,176_B all with phasefactor}. The average phasefactor is calculated using second and fourth order unbiased exponential resummation, and are illustrated through red and black points respectively for all the three working temperatures. Although the location of zero for $T=176$ MeV is not very consistent with the different estimates of the radius of convergence namely $\rho_n$ for $n=1,2,3$, the agreement and consistency is appreciable for the other two temperatures at $135$ and $157$ MeV. For these two temperatures, the zero line coincides with the points within errorbars. 



\subsection{For isospin chemical potential}


\begin{figure}[H]
    \centering
    \includegraphics[width=.325\textwidth]{figures/CHI_compar_I_135_c6_Taylor_vs_unb.pdf}
    \includegraphics[width=.325\textwidth]{figures/CHI_compar_I_157_c6_Taylor_vs_unb.pdf}
    \includegraphics[width=.325\textwidth]{figures/CHI_compar_I_176_c6_Taylor_vs_unb.pdf} \\
    \hspace{.1mm}
    \includegraphics[width=.325\textwidth]{figures/CHI_compar_I_135_c8_Taylor_vs_unb.pdf}
    \includegraphics[width=.325\textwidth]{figures/CHI_compar_I_157_c8_Taylor_vs_unb.pdf}
    \includegraphics[width=.325\textwidth]{figures/CHI_compar_I_176_c8_Taylor_vs_unb.pdf}
    
    \caption{Plots for $\chi_6^I$ (top row) and $\chi_8^I$ (bottom row) obtained from the methods of Taylor expansions and unbiased exponential resummation. These are obtained for $135$ (left column), $157$ (middle column) and $176$ MeV (right column) respectively. The color nomenclature remains the same as before.}
    \label{fig:chi_6 and chi_8 135,157,176_I}
\end{figure}
% \begin{figure}[H]
%     \centering
%     \includegraphics[width=.325\textwidth]{figures/CHI_compar_I_135_c6_Taylor_vs_unb.pdf}
%     \includegraphics[width=.325\textwidth]{figures/CHI_compar_I_157_c6_Taylor_vs_unb.pdf}
%     \includegraphics[width=.325\textwidth]{figures/CHI_compar_I_176_c6_Taylor_vs_unb.pdf} \\
%     \hspace{.1mm}
%     \includegraphics[width=.325\textwidth]{figures/CHI_compar_I_135_c8_Taylor_vs_unb.pdf}
%     \includegraphics[width=.325\textwidth]{figures/CHI_compar_I_157_c8_Taylor_vs_unb.pdf}
%     \includegraphics[width=.325\textwidth]{figures/CHI_compar_I_176_c8_Taylor_vs_unb.pdf}
    
%     \caption{Plots for $\chi_6^I$ (top row) and $\chi_8^I$ (bottom row) obtained from the methods of Taylor expansions and unbiased exponential resummation. These are obtained for $135$ (left col.), $157$ (middle col.) and $176$ MeV (right col.) respectively.}
%     \label{fig:chi_6 and chi_8 135,157,176_I}
% \end{figure}


In this section, we present results for isospin chemical potential $\muI$. The comparison regarding $\chi_6$ and $\chi_8$ is illustrated for all the three working temperatures in Fig.\ref{fig:chi_6 and chi_8 135,157,176_I}. However as compared to $\muB$ we find from Fig.\ref{fig:chi_6 and chi_8 135,157,176_I}, the second order resummation estimates for $\chi_6$, namely $\csIt$ vary to a greater extent against the equivalent Taylor estimate $\csIT$. We also find that this difference reduces with increasing temperature for $\chi_6$, seemingly suggesting that the corrections introduced by the unbiased estimates of $D_n$ from $D_3$ to $D_6$, over $D_1$ and $D_2$ in the evaluation of $\csIT$ become less pronounced with increasing temperature. Despite this, we find just like $\muB$ the agreement becomes very good from fourth order onwards and as usual being $\csIs=\csIT$, the black and green points agrees almost completely along with the errorbars. In case of $\chi_8^I$ also, the disagreement is noticeable at second order but again from fourth order the extent of agreement is very good and commendable for all the three working temperatures. As before, this seem to imply that it is important maybe to just consider having unbiased estimates upto $D_4$ only.


\begin{figure}[H]
    \centering
    \includegraphics[width=.325\textwidth]{figures/ROC_compar_I_135_all_with_phasefactor.pdf}
    \includegraphics[width=.325\textwidth]{figures/ROC_compar_I_157_all_with_phasefactor.pdf}
    \includegraphics[width=.325\textwidth]{figures/ROC_compar_I_176_all_with_phasefactor.pdf} \\
    \hspace{.1mm}
    \includegraphics[width=.325\textwidth]{figures/all_order_135_exp_ph.pdf}
    \includegraphics[width=.325\textwidth]{figures/all_order_157_exp_ph.pdf}
    \includegraphics[width=.325\textwidth]{figures/all_order_176_exp_ph.pdf}
    
    \caption{(Top row) Plots of $\rho_1^I,\rho_2^I$ and $\rho_3^I$ at $135$ (left), $157$ (middle) and $176$ MeV (right) respectively. (Bottom row) Plots of $\LA e^{i\theta}\RA$ as a function of $\muI$ for second and fourth orders plotted at $135$ (left), $157$ (middle) and $176$ MeV (right) respectively. The red and black points in these bottom row plots depict the second and fourth order $\LA e^{i\theta} \RA$ respectively, whereas the magenta line in the top row plots illustrate the onset of $\LA e^{i\theta} \RA=0$.}
    \label{fig:ROC 135,157,176_I all with phasefactor_2nd}
\end{figure}


Just like $\muB$, we have also plotted the different estimates of radius of convergence in Fig.\ref{fig:ROC 135,157,176_I all with phasefactor_2nd} for $\muI$ and compared the compatibility of different estimates obtained from the Taylor expansion and unbiased exponential resummation approaches. However unlike $\muB$, we have determined the complex phasefactor $e^{i\theta}$ for complex $\muI$, computed its gauge ensemble average $\LA e^{i\theta} \RA$ and observed its behaviour for various complex $\muI$. This is because as mentioned before, the partition function does not have to be real for complex chemical potentials. We have traced a two-dimensional plot of $|\LA e^{i\Theta} \RA|$ against $\left|\muI\right|$ which is the radial distance of the complex $\muI$ from the origin in the complex $\muI$ plane. Unlike $\muB$, there is a possibility of having multiple points of phasefactor for a given $\muI/T$\footnote{This is because $x+i\,y$ and $y+i\,x$ will have same value of $\muI/T$, but can have different values of $\LA e^{i\theta} \RA$.} in the phasefactor plots of $\muI$ in Fig.\ref{fig:ROC 135,157,176_I all with phasefactor_2nd} and hence as we observe, there are more number of points as compared to $\muB$. Despite the magenta line of the Fig.\ref{fig:ROC 135,157,176_I all with phasefactor_2nd} not intersecting with $\rho_1$ and $\rho_2$ at $135$ and $157$ MeV, we find that it intersects with the estimate of $\rho_3$ for both these temperatures. Thus, the point of the first appearance of zeros of $|\LA e^{i\Theta} \RA|$ procured using second and fourth order unbiased resummation approaches, provides commendable indications about the radius of convergence and the possible breakdown at least for eighth order Taylor series. This is somewhat promising, although the agreement with the same is not so well-defined at $176$ MeV.  We may have to look out for some other indicators of radius of convergence at this temperature, or maybe the physics at this temperature is somewhat different which cannot be explained using the present method of unbiased exponential resummation and the associated phasefactor determination. It may also signify that maybe in order to capture the genuine behaviour of the Taylor series at $176$ MeV, it is important to go to even higher order calculations or it may also require the determination of even higher-than-fourth-order calculation of unbiased phasefactor.

\begin{figure}[H]
    \centering
    \includegraphics[width=.325\textwidth]{figures/NOW_all_order_135_kurtosis.pdf}
    \includegraphics[width=.325\textwidth]{figures/NOW_all_order_157_kurtosis.pdf}
    \includegraphics[width=.325\textwidth]{figures/NOW_all_order_176_kurtosis.pdf} 
    
    \caption{Plots of second and fourth order kurtosis as a function of $\muI/T$ for $135$ (left), $157$ (center) and $176$ MeV (right). The blue and black points illustrate the second and fourth order kurtosis and the red vertical line illustrates the onset of zero of $\LA e^{i\theta}\RA$ as shown above in Fig.\ref{fig:ROC 135,157,176_I all with phasefactor_2nd}.}
    \label{fig:135,157,176_I kurtosis}
\end{figure}

Lastly, we also present results manifesting the behaviour of kurtosis as a function of $\muI/T$. The kurtosis offers a quantitative measure of the overlap problem and its severity, which have been briefly discussed before. This overlap problem becomes visibly predominant for $\muI$ where there is no sign problem, and maybe the most possible reason for a breakdown. In the Fig.\ref{fig:135,157,176_I kurtosis}, we observe that both the second and fourth order kurtosis exudes a monotonically increasing behaviour for higher values of $\muI$. This is expected since the extent of overlap between distributions generated at finite $\muI$ and zero $\muI$ reduces with increasing value of the finite $\muI$, and is evident from the higher values of the associated errorbars as seen in Fig.\ref{fig:135,157,176_I kurtosis}. The red vertical line depicts the commencement of zero of $\LA e^{i\theta} \RA$, which coincides with the value of magenta line shown in Fig.\ref{fig:ROC 135,157,176_I all with phasefactor_2nd}. The Fig.\ref{fig:135,157,176_I kurtosis} demonstrates that the errorbars increases rapidly across this red line which implies that as the oscillations of the complex fermionic determinant for complex $\muI$ increases making $\LA e^{i\theta} \RA \approx 0$, the overlap problem rapidly increases. This is manifested by the increasing errorbars, non-monotonic behaviour of the kurtosis values. This is an encouraging sight as one can understand the severity of this overlap problem and associated breakdown by studying the behaviour of this newly proposed complex phasefactor and observing the onset of its zeroes or close-to-zero values. 

% of this newly proposed complex phasefactor does offer for this overlap problem is consistent for all the working temperatures, to a good extent with the onset of the zero of the proposed complex phasefactor $\left|\LA e^{i\theta}\RA\right|$. Beyond the value of $\muI$ at which the first zero of $\left|\LA e^{i\theta}\RA\right|$ appears, the overlap problem becomes very severe and the errorbars also become extremely large along with a non-monotonic behaviour of the mean values of $\kappa_2$ and $\kappa_4$.












\section{Conclusions}
\label{sec:Conclusions}


In this paper, we have conducted a detailed comparative study between the estimates of higher order charge cumulants, which have been obtained using the approaches of Taylor expansion and the unbiased exponential resummation. The estimates of these cumulants are exact in Taylor expansion, in the sense that all the necessary and appropriate correlation functions have been included with proper estimation of their unbiased powers before averaging them over the working gauge ensemble. On the other hand, correlation functions only upto order $N$ are kept in $N^{th}$ order unbiased exponential resummation and it is only from these non-zero correlation functions in this method that the respective charge cumulants of sixth and eighth orders are evaluated and estimated.

Although there still remains a lot to investigate and make further progress, it has been evident from the aforementioned results that the resummed estimates of the sixth and eighth order charge cumulants seem to give excellent agreement with the corresponding Taylor-equivalent estimates. Despite making comparisons only upto eighth order, it is found that resummation results from fourth order onwards seemingly provide some order-independent compatibility and agreement with the corresponding Taylor series results. Although a lot more needs to be done before confirming this statement, the current set of results presented in this paper seem to clearly imply that once the stochastic bias upto $\mathcal{O}(\mu^4)$ is eliminated by considering the necessary unbiased powers of $D_n$ upto $n=4$ using unbiased exponential resummation, the resummation results work very well and agree with the Taylor-ed estimates of charge cumulants of at least upto eighth orders. More importantly, this observation holds true more or less for all the three working temperatures in this paper, which are subtly selected so that they encompass a large part of the entire QCD phase diagram. This finding also remains true irrespective of whether or not, one calculates unbiased powers of the higher $D_n$ from $n=5$ onwards. So, this may possibly imply that owing to the ultraviolet divergences remaining upto $\mathcal{O}(\mu^4)$, it is only these first four correlation functions which impacts the resulting calculations maximally and therefore must be treated in an unbiased manner for procuring the behaviour of the true infinite Taylor series and for conducting a more transparent genuine probe into QCD at finite chemical potential.

We have also investigated the zeros of the gauge-ensemble average phasefactor and found that barring some exceptions at $176$ MeV, the onset of these zeros provide agreeable indications regarding the estimates of the radius of convergence and the domain of validity of the resummation results, before encountering breakdown. Apart from baryochemical potentials, this also holds true for isospin chemical potentials where we have proposed a new formula for a non-trivial phasefactor which varies with complex $\muI$ by considering complex $\muI$. We also demonstrated the nature of the overlap problem with $\muI$ and illustrated that the onset of the zeros of this newly proposed phasefactor is more or less consistent with this problem and this overlap problem becomes severely large, with excessively high errorbars beyond the point of this onset of zeros. 


\acknowledgments

 I sincerely acknowledge Prasad Hegde of Indian Institute of Science and Frithjof Karsch of Bielefeld University for highly useful discussions and stimulating constructive suggestions for this draft. I also sincerely thank all the other members of the HotQCD collaboration for their inputs and valuable insights, as well as for allowing me to use their data for the respective Taylor expansion calculations. The computations in this work have been performed on the GPU cluster at Bielefeld University, Germany. I also heartily thank the Bielefeld HPC.NRW team for their wholehearted support. 

\vspace{.1cm}

 \bibliographystyle{JHEP}
 %\bibliography{main}
 % This must be in the first 5 lines to tell arXiv to use pdfLaTeX, which is strongly recommended.
\pdfoutput=1
% In particular, the hyperref package requires pdfLaTeX in order to break URLs across lines.

\documentclass[11pt]{article}

% Remove the "review" option to generate the final version.
%\usepackage[review]{ACL2023}
\usepackage{ACL2023}

% Standard package includes
\usepackage{times}
\usepackage{latexsym}

% For proper rendering and hyphenation of words containing Latin characters (including in bib files)
\usepackage[T1]{fontenc}
% For Vietnamese characters
% \usepackage[T5]{fontenc}
% See https://www.latex-project.org/help/documentation/encguide.pdf for other character sets

% This assumes your files are encoded as UTF8
\usepackage[utf8]{inputenc}

% This is not strictly necessary, and may be commented out.
% However, it will improve the layout of the manuscript,
% and will typically save some space.
\usepackage{microtype}

% This is also not strictly necessary, and may be commented out.
% However, it will improve the aesthetics of text in
% the typewriter font.
\usepackage{inconsolata}


% If the title and author information does not fit in the area allocated, uncomment the following
%
%\setlength\titlebox{10cm}
%
% and set <dim> to something 5cm or larger.

%%%%%%%%%%%%%%%%%%%%%%%%%%%%%%%%%%
\usepackage{graphicx}
\usepackage{amsfonts}
\usepackage{amsmath}
\usepackage{bigdelim}
\usepackage{diagbox}
\usepackage{amsthm}
\usepackage{makecell}
\usepackage{mathtools}
\usepackage{booktabs}
\usepackage[shortlabels]{enumitem}
\graphicspath{ {figs/} }

\theoremstyle{remark}
\newtheorem*{question}{Question}

\newcommand{\tk}[1]{\textcolor{blue}{{#1}}}
\newcommand{\sy}[1]{\textcolor{red}{{#1}}}
\newcommand{\mg}[1]{\textcolor{purple}{{#1}}}
\newcommand{\lh}[1]{\textcolor{green}{{#1}}}
\newcommand{\lc}[1]{\textcolor{green}{{#1}}}

% Rounded color box
\definecolor{light_blue}{HTML}{cfdfff}
\usepackage[most]{tcolorbox}
\tcbset{on line, 
        boxsep=1pt, left=0pt,right=0pt,top=0pt,bottom=0pt,
        colframe=white,colback=light_blue,  
        highlight math style={enhanced}
        }

\newcommand{\quash}[1]{}  %Anything in \quash is ignored
\newcommand{\gpt}{\textsc{GPT-2}}
\newcommand{\bert}{\textsc{BERT}}
\newcommand{\bertlarge}{\textsc{BERT-large}}
\newcommand{\mask}{\texttt{[MASK]}}
\newcommand{\cls}{\texttt{[CLS]}}
\newcommand{\sep}{\texttt{[SEP]}}
\newcommand{\mat}{\texttt{mat}}
\newcommand{\id}{\texttt{id}}
\newcommand{\matl}{\texttt{mat}_{\ell \rightarrow \ell'}}
\newcommand{\matattnl}{\texttt{mat\_attn}_{\ell \rightarrow \ell'}}
\newcommand{\matffl}{\texttt{mat\_ffn}_{\ell \rightarrow \ell'}}
\newcommand{\matlnl}{\texttt{mat\_ln1\_ln2}_{\ell \rightarrow \ell'}}
\newcommand{\idl}{\texttt{id}_{\ell \rightarrow \ell'}}
\newcommand{\matlL}{\texttt{mat}_{\ell \rightarrow L}}
\newcommand{\matattnlL}{\texttt{mat\_attn}_{\ell \rightarrow L}}
\newcommand{\matfflL}{\texttt{mat\_ffn}_{\ell \rightarrow L}}
\newcommand{\matlnlL}{\texttt{mat\_ln1\_ln2}_{\ell \rightarrow L}}
\newcommand{\idlL}{\texttt{id}_{\ell \rightarrow L}}

\definecolor{blue(munsell)}{rgb}{0.0, 0.5, 0.69}
%%%%%%%%%%%%%%%%%%%%%%%%%%%%%%%%%%

\title{Jump to Conclusions: Short-Cutting Transformers\\With Linear Transformations}

% Author information can be set in various styles:
% For several authors from the same institution:
% \author{Author 1 \and ... \and Author n \\
%         Address line \\ ... \\ Address line}
% if the names do not fit well on one line use
%         Author 1 \\ {\bf Author 2} \\ ... \\ {\bf Author n} \\
% For authors from different institutions:
% \author{Author 1 \\ Address line \\  ... \\ Address line
%         \And  ... \And
%         Author n \\ Address line \\ ... \\ Address line}
% To start a seperate ``row'' of authors use \AND, as in
% \author{Author 1 \\ Address line \\  ... \\ Address line
%         \AND
%         Author 2 \\ Address line \\ ... \\ Address line \And
%         Author 3 \\ Address line \\ ... \\ Address line}

\author{Alexander Yom Din$^{1}$ ~~~~~ Taelin Karidi$^{1}$ ~~~~~ Leshem Choshen$^{1}$ ~~~~~
Mor Geva$^{2}$ 
\vspace{0.2cm} \\
$^1$Hebrew University of Jerusalem ~~~ $^2$Google Research \\
\small{\texttt{\{alexander.yomdin, taelin.karidi, leshem.choshen\}@mail.huji.ac.il}}, \small{\texttt{pipek@google.com}}}

\quash{
\author{Alexander Yom Din \\
  Hebrew University of Jerusalem \\ \texttt{alexander.yomdin@mail.huji.ac.il} \\\And
  Taelin Karidi \\
  Hebrew University of Jerusalem \\
  \texttt{taelin.karidi@mail.huji.ac.il} \\\And
  Leshem Choshen \\
  Hebrew University of Jerusalem \\ \texttt{leshem.choshen@mail.huji.ac.il} \\\And
  Mor Geva \\
  Google Research \\
  \texttt{pipek@google.com} \\}
}

\begin{document}
\maketitle



\begin{abstract}
% \vspace{-1em}
The diffusion-based generative models have achieved remarkable success in text-based image generation. However, since it contains enormous randomness in generation progress, it is still challenging to apply such models for real-world visual content editing, especially in videos. 
In this paper, we propose \texttt{FateZero}, a zero-shot text-based editing method on real-world videos without per-prompt training or use-specific mask. 
\RM{Specifically, different from a pipeline of two independent inversion and then generation stages, we find the intermediate attention maps during inversions store better structure and motion information. We thus reform them to temporally casual attention and replace them in the generation progress. To further reduce the unnecessary semantic leakage of source video and enhance the editing quality, we then remix the temporally casual attentions via the cross-attention features of the source prompt as the mask.}
To edit videos consistently, we propose several techniques based on the pre-trained models. Firstly, in contrast to the straightforward DDIM inversion technique, our approach captures intermediate attention maps during inversion, which effectively retain both structural and motion information. These maps are directly fused in the editing process rather than generated during denoising. To further minimize semantic leakage of the source video, we then fuse self-attentions with a blending mask obtained by cross-attention features from the source prompt. Furthermore, we have implemented a reform of the self-attention mechanism in denoising UNet by introducing spatial-temporal attention to ensure frame consistency.
Yet succinct, our method is the first one to show the ability of zero-shot text-driven video style and local attribute editing from the trained text-to-image model. We also have a better zero-shot shape-aware editing ability based on the text-to-video model~\cite{tuneavideo}. \RM{Besides video, our unified method also achieves state-of-the-art performance in zero-shot image editing.\chenyang{Need exp or remove the zero-shot image}} Extensive experiments demonstrate our superior temporal consistency and editing capability than previous works.
% The code will be released.
% \chenyang{emphasize: our observation at inversion time} \xiaodong{replacing the bold part to the actual pipeline: \textbf{Specifically, we work on replacing and mixing the attention maps between the inversion and generation since the self-attention map keeps the structure of the original natural image and the cross-attention is semantic-related, after remixing, we replace them in the corresponding generation steps for denoising.}}
% \footnote{Since there is no general video diffusion model is publicly available, we use one-shot video generation method~(Tune-A-Video~\cite{tuneavideo}) as the pretrained video diffusion model for zero-shot video editing\xiaodong{can be removed if we actually zero-shot on video}.}.
\end{abstract}
\section{Introduction}

The ability to reason about plans is critical for performing long-horizon tasks \citep{erol1996hierarchical, sohn2018hierarchical, sharma-etal-2022-skill}, compositional generalization \citep{corona-etal-2021-modular} and generalization to unseen tasks and environments \citep{shridhar2020alfred}.
Consider a simple long-horizon planning scenario where a robot is tasked with preparing a meal and serving it on the table. 
This presents a non-trivial planning problem since the agent needs to understand the sequence of operations required to perform the task and search for the relevant objects in the unfamiliar environment by interacting with various objects. %



Large language models have been recently shown to possess commonsense knowledge about the world such as object affordances and physical dynamics \citep{ouyang2022training,chowdhery2022palm}.
Early approaches considered text based environments and fine-tuned PLMs to predict actions given the history of past observations and actions \citep{jansen-2020-visually,micheli-fleuret-2021-language,yao-etal-2020-keep}.
Recent work has used this ability to reason about plans from text instructions in simulated household environments with simplifying assumptions such as text-only environment observations or feedback \citep{huang2022language,ahn2022can,li2022pre,logeswaran-etal-2022-shot}.


We focus on \emph{visually grounded planning} with PLMs --- the ability to adapt plans based on interaction and visual feedback from the environment.
While PLMs have strong planning commonsense priors, predictions from a PLM may not be directly realizable in the environment since the observation and action spaces are unknown.
This requires \emph{grounding} the PLM in the environment and adapting it to observe visual feedback, which is highly non-trivial.
Some prior works assume the availability of a pre-trained affordance function \citep{ahn2022can} or a success detector \citep{mirchandani2021ella}.
Notably, SayCan \citep{ahn2022can} completely decouples the PLM from observation information by selecting actions that have both high affordability (through a pre-trained affordance model) and high PLM likelihood.
Although this partially addresses the grounding problem, the use of visual feedback for action affordance alone is limited.
Often an agent must choose one of many affordable actions using information from observations.
For example, a driving agent should re-navigate and possibly turn around when encountering a ``road closed'' sign, but both turning around and driving forward are indistinguishable to SayCan because they are both affordable and the PLM is blind to observations.

Another workaround explored in prior work is translating the information in the visual observations to text using a pre-trained captioning system \citep{shridhar2021alfworld,huang2022language}.
However, it can be difficult to faithfully describe an image in words and information is lost in this inherently noisy process, which limits the information available to the planner.



Recent work shows that PLMs can be adapted for various natural language tasks by inserting tunable embeddings or soft prompts at the input of the PLM (also called prompt tuning or prefix tuning)~\citep{li-liang-2021-prefix,lester-etal-2021-power}.
This approach also extends to multi-modal understanding tasks such as image captioning \citep{mokady2021clipcap} and VQA \citep{tsimpoukelli2021multimodal} where images are encoded as soft prompts and finetuned for the target task.
Transformer based architectures have also been successfully applied to offline Reinforcement Learning in recent work \citep{chen2021decision,janner2021offline,li2022pre,reid2022can}.

Taking inspiration from these works, we propose the simple approach of embedding visual observations (`visual prompts') and \textit{directly inserting them as PLM input embeddings}.
The visual encoder and PLM are jointly trained for the target task, an approach we call \textbf{\oursfull}~(\ours).
By teaching the PLM to use observations for planning in an end to end manner, we remove the dependency on external data such as captions and affordability information that was used in prior work.
We show that this simple approach performs better than prior PLM-based planning approaches on two embodied planning benchmarks based on ALFWorld~\citep{shridhar2021alfworld} and Virtualhome~\cite{puig2018virtualhome}.



\section{Related Work}

%Here we summarize prior work on transfer learning and property inference.

%\shortsection{Transfer Learning}
%%Transfer learning reuses features learned by pre-trained models for new tasks, with the pretext that inherent similarities in the generic features will be useful for the downstream tasks and hence reducing their cost of downstream training. Specifically, the downstream model trainer will use a pre-trained upstream model as the starting point for the downstream training, with inclusion of (or replacement with) the task-specific classification layer/module. The downstream model is then trained by either updating all layers of the model (including ones reused from upstream model) or freezing some earlier layers of the reused parts as the ``feature extractor'' and only updating the rest. The latter approach is more popular as the reused feature extractors can already learn useful feature representations and the training cost is also much lower and affordable for individuals with limited computational resources. We study the vulnerability of the latter transfer learning approach in this paper. 


%\shortsection{Transfer Learning} 
Several works have demonstrated risks associated with transfer learning across a variety of attack goals. Wang et al.~\cite{wang2018great} and Yao et al.~\cite{yao2019latent} consider manipulating the upstream model such that the fine-tuned downstream models contain backdoors, misclassifying test inputs that contain predefined backdoor triggers. These transfer manipulations are tailored to their particular attack goals and cannot be applied for the property inference goal considered in this paper. Zou et al.~\cite{zou2020privacy} study the threat of membership inference attacks on transfer learning, but with normally trained upstream models.  
%\dnote{its clear that the goals are different for these attacks, but how similar are the methods?} \ynote{similarity of the methods? more details about the methods? do not know what is expected here}
%In contrast, we investigate the possibility of boosting the effectiveness of property inference by manipulating the upstream model training. % Schuster et al.~\cite{schuster2020humpty} show that the attacker can modify the corpus on which the word embedding is trained such that the downstream NLP models which use that embedding will behave abnormally.

%\shortsection{Property Inference}
The risk of property inference was introduced by Ateniese et al.~\cite{ateniese2015hacking}, % introduces the threat of inferring properties of the training data from pre-trained models, 
and several subsequent works have developed property inference (also known as distribution inference) attacks~\cite{Wang2022GroupPI, suri2022formalizing, Jurez2022BlackBoxAF, Hartmann2022DistributionIR}.
% Ganju et al.~\cite{ganju2018property} and Suri and Evans~\cite{suri2022formalizing} 
These works study property inference against normally trained models, and they launch attacks using a variety of black-box and white-box attacks. All the white-box attacks use meta-classifiers, which take the permutation-invariant representation~\cite{ganju2018property} of the model parameters as the features. We use the state-of-the-art white-box attack~\cite{suri2022formalizing} in our experiments.
%We will use the state-of-the-art white-box method proposed by Ganju et al.~\cite{ganju2018property} and later extended by suri et al.~\cite{suri2022formalizing} in this paper.
%\dnote{do we use these attacks?} 
Melis et al.~\cite{melis2019exploiting} and Zhang et al.~\cite{zhang2021leakage} focus on property inference in distributed training scenarios. In their settings, the attacker is a participant in the global model training and conducts property inference using meta-classifiers that are trained on model outputs or gradients. Similarly, Suri et al.~\cite{suri2022subject} focus on federated learning settings where the attacker is a participant (or the central server) that utilizes black-box attacks for inferring membership of data from particular subjects. %\dnote{if we use black-box attacks, explain which ones, or how ours are related to previous ones} 
For our experiments, We improve the black-box meta-classifier proposed by Zhang et al.~\cite{zhang2021leakage} using the ``query tuning'' technique in Xu et al.~\cite{xu2019detecting}. 

The closest works to ours are Chase et al.~\cite{saeed} and Chaudhari et al.~\cite{Chaudhari2022SNAPEE}, which both consider a scenario where the attacker can manipulate some of the training data of the model to induce a model that significantly increases property inference risk.
% \dnote{it enables precise property inference attacks?}.
These works assume an adversary with the ability to poison the victim's training data, while the adversary in our scenario has no access to the victim's training data, and therefore, their methods are not applicable.
% \dnote{example how different from ours, and why the methods are not applicable}
%Thus, their methods are not applicable to our transfer learning scenario.
%Their methods rely on inducing certain behavior correlated with the properties to be inferred, and thus are not applicable to our transfer learning scenario. \anote{Still a bit unclear why that is the case.}
%
There are also works similar to ours that leverage ``adversarial initializations'' for attack purposes.
% \cite{grosse2019adversarial, boenisch2021curious, wen2022fishing, fowl2021robbing}.
Grosse et al.~\cite{grosse2019adversarial} focus on scenarios where the attacker can control the parameter initialization of a model, and demonstrate that the attacker can use special initializations to damage the performance of the trained model. %This attack is orthogonal to ours.
Other works \cite{boenisch2021curious, wen2022fishing, fowl2021robbing} show that the malicious central server in a federated learning protocol can reconstruct some training samples via falsifying the global model in some training rounds and then analyzing the submitted gradients. These kinds of attacks do not apply to our transfer-learning scenario since the attacker cannot access the downstream gradients, and can only manipulate the upstream training.

\iffalse %%%%%%%%%%%%%%%%%%%%%%%%%%%%%%%%

In this section, we provide the background and also the summary of prior attacks on transfer learning (Section~\ref{sec:transfer_learning}) and property inference (Section~\ref{sec:property_inference}). Then, we introduce the closely related manipulation attacks against machine learning models to boost different privacy risks in Section~\ref{sec:active_inference_attacks}.

%\anote{Do we really need a dedicated section for this? It's barely 2 paragraphs right now.}

%\dnote{the most closely related work to ours are works that attempt to amplify inference attacks by poisoning models, the two most relevant I know of are \url{https://www.computer.org/csdl/proceedings-article/sp/2022/131600b569/1CIO8nmuota} and \url{https://arxiv.org/abs/2204.00032}, but need to look thoroughly for others. We should definitely be describing this and relating it to our work, probably in the introduction. Most of what is here is Background, but should be clear what this section is for (not muddling background and related work)}

\subsection{Transfer Learning} \label{sec:transfer_learning}
Transfer learning reuses features learned by pre-trained models for new tasks, with the pretext that inherent similarities in generic features can be useful for downstream tasks, thus reducing the cost of downstream training. Specifically, the downstream model trainer uses a pre-trained upstream model as the starting point for downstream training, with the inclusion (or replacement) of task-specific classification layers/modules. The downstream model is then trained by either updating all layers of the model (including ones reused from the upstream model) or freezing some earlier layers of the reused parts as the ``feature extractor'' and only updating the rest. The latter approach is more popular as the reused feature extractors can already learn useful feature representations and the training cost is also much lower and affordable for individuals with limited computational resources. We study the vulnerability of the latter transfer learning approach in this paper. 
%mainly in two ways:  1) all the layers (including ones reused from ) and tune the full model; the other one is to freeze some earlier layers of the model as the feature extractor and only tune the rest later layers. The second update strategy could achieve better efficiency since the frozen layers can already produce meaningful feature representations~\cite{wang2018great,yao2019latent}, and we will study the transfer learning using this strategy. 

Recently, various attacks have been proposed for the transfer learning setting, but with different attack goals from ours. Wang et al.~\cite{wang2018great} generate adversarial examples against black-box student models that transfer knowledge from publicly available teacher models without repeated queries. Yao et al.~\cite{yao2019latent} propose to manipulate the upstream model such that the downstream models derived from the upstream model contain backdoors, which would misclassify test inputs that contain some predefined backdoor triggers. Zou et al.~\cite{zou2020privacy} study the threat of membership inference attacks on transfer learning and the upstream models are trained normally. In contrast, we investigate the possibility of boosting the effectiveness of property inference by manipulating the upstream model training. Schuster et al.~\cite{schuster2020humpty} show that the attacker can modify the corpus on which the word embedding is trained such that the downstream NLP models which use that embedding will behave abnormally.

%This additionally allows model trainers to achieve satisfactory performance with limited training samples, leading to reduced computational costs. The most common approach reuses parameters in the earlier layers of the pre-trained model, either by fixing them as the feature extractor or just using them for initialization, to conduct downstream training.

\subsection{Property Inference} \label{sec:property_inference}

\shortsection{Property Inference Attacks} In property inference attacks, the adversary aims to infer some sensitive properties of some data, given a model trained on it. For example, the adversary may be interested in sensitive properties like the presence of people of a specific race in the dataset~\cite{ateniese2015hacking, melis2019exploiting}), or even be curious about the 
the statistics of the training set (e.g, the ratio of people with a specific gender~\cite{saeed, ganju2018property, suri2022formalizing, zhang2021leakage}).


Ateniese et al.~\cite{ateniese2015hacking} were the first to identify the threat of inferring properties of the training data from pre-trained models. Ganju et al.~\cite{ganju2018property} and Suri and Evans~\cite{suri2022formalizing} 
study property inference against normally trained models, and they launch attacks using white-box meta-classifiers, which utilize the permutation-invariance representation~\cite{ganju2018property} of the model parameters, while other works focus on distributed training~\cite{zhang2021leakage} where the attacker is a participant in the global model training and conducts property inference using meta-classifiers trained on model outputs. Similarly, Suri et al.~\cite{suri2022subject} focus on federated learning, where the attacker is a participant (or the central server) that utilizes black-box attacks for inferring membership of data from particular subjects. Chase et al.~\cite{saeed} propose an active property inference attack for data poisoning scenarios, which we will cover and compare to in Section~\ref{sec:active_inference_attacks}.

%The closest work to ours are by Chase et al.~\cite{saeed} and Tramer et al.~\cite{tramer2022truth}. In their work, the attacker can manipulate some of the training data of the model such that a model trained (from scratch) on the poisoned data has an increased inference risk. However, their methods are not applicable to the transfer learning scenario. 
%In this work, we will focus on the property inference in transfer learning scenarios in which the attacker releases the upstream model and infer sensitive properties of the downstream models tuned from that upstream model.
% 

\shortsection{Defenses}
Defending against property inference attacks is an open problem. There are no studies in the current literature on active adversaries, and only a couple on passive ones. Ma et. al.~\cite{ma2021nosnoop} propose a defense against property inference attacks on data batches in the  collaborative learning setting. However, adversaries in the transfer-learning setting do not have access to batch-wise gradients of the downstream trainer. Chen and Ohrimenko~\cite{chen2022protecting} utilize mechanisms that add carefully-crafted noise to features to provide theoretical guarantees against inference adversaries, but focus on query-based access to the underlying dataset, not a machine learning model trained on it. These existing defenses thus do not apply to our threat model.

%propose a framework that reduces property inference to Boolean functions of individual members, posing the ratio of members satisfying the given function in a dataset as the property. These property inference attacks have since then been proposed as distribution inference attacks~\cite{suri2022formalizing}, presenting such attacks as inferring properties of the distributions used to sample datasets, differentiating them from exact inference attacks like dataset inference~\cite{maini2021dataset}. Nearly all property inference attacks use meta-classifiers to perform inference: training models on versions of datasets with and without the target property, followed by training a meta-classifier on top of these classifiers's model representations. These representations can take several forms: using model weights themselves with permutation-invariance~\cite{ganju2018property}, or model activations or logits for a generated set of query points~\cite{xu2019detecting}. However, the capability of such approaches is limited: the most that these attacks have been shown to work is medium-sized convolutional networks on the CelebA dataset~\cite{suri2022formalizing}.


\subsection{Active Privacy Attacks} \label{sec:active_inference_attacks}
% Perhaps the closely related works to ours as ones that proactively enhance the effectiveness of privacy attacks by manipulating the model training process in certain ways~\cite{saeed, melis2019exploiting, nasr2019comprehensive, tramer2022truth}. 
%shown that the adversary can, by using proactive ways, achieve stronger attacks that infer private information from deep learning systems~\cite{nasr2019comprehensive, melis2019exploiting, tramer2022truth, saeed}. In this section, we introduce the ones that are close to ours.

In the decentralized federated learning training, by submitting specially crafted gradients to the central server, malicious agents can increase membership inference risk~\cite{nasr2019comprehensive} and property inference risks~\cite{melis2019exploiting} of other benign agents' training data. However, these attacks do not apply to transfer learning scenario, as the attacker cannot control model gradients of downstream training. In the centralized setting, researchers propose attacks to poison the victim's training data such that the impacts of attribute inference and membership inference~\cite{tramer2022truth} and property inference~\cite{saeed} attacks are amplified on the poisoned model.
The ability to poison the victim's data is a threat model orthogonal to ours, since we have no access to the victim's downstream data. While there is scope to combine such approaches for stronger attacks (albeit with stronger access assumptions), we choose to focus on the scenario with no read/write access to the victim's data.

\fi %%%%%%%%%%%%%%%%%%%%%%%%%%%%%%%%

\section{Linear Shortcut Across Blocks}
\label{sec:layer_jump}

To use a hidden representation from layer $\ell<L$ as a final representation, we propose to cast it using linear regression, while skipping the computation in-between these layers. More generally, this approach can be applied to cast any $\ell$-th hidden representation to any subsequent layer $\ell'>\ell$.


\subsection{Method}
\label{subsec:methodology_linear_shortcut}

Given a source layer $\ell$ and a target layer $\ell'$ such that $0 \leq \ell < \ell' \leq L$, our goal is to learn a mapping
%$A_{\ell', \ell} \in \mathbb{R}^{d_h \times d_h}$
from hidden representations at layer $\ell$ to those at layer $\ell'$. To this end, we first collect a set of corresponding hidden representation pairs $(h^\ell, h^{\ell'})$. Concretely, we run a set $\mathcal{T}$ of input sequences through the model, and for each input $s$, we extract the hidden representations $h_{i_s}^{\ell}, h_{i_s}^{\ell'}$, where $i_s$ is a random position in $s$.
Next, we learn a matrix $A_{\ell', \ell} \in \mathbb{R}^{d_h \times d_h}$ by fitting linear regression over $\mathcal{T}$, i.e., $A_{\ell', \ell}$ is a numerical minimizer for:
$$ A \mapsto \sum_{s \in \mathcal{T}} || A \cdot h_{i_s}^\ell - h_{i_s}^{\ell'} ||^2,$$ 
and define the mapping of a representation $h$ from layer $\ell$ to layer $\ell'$ as:
\begin{equation}
\label{eq:linear_jump}
    \matl{} (h) \coloneqq A_{\ell', \ell} \cdot h.
\end{equation}


\subsection{Baseline}
\label{subsec:baseline}

We evaluate 
% our method against 
the prevalent approach of ``reading'' hidden representations directly, without any transformation. 
Namely, the propagation of a hidden representation from layer $\ell$ to layer $\ell'$ is given by the identity function, dubbed \id{}:

$$ \idl{} (h) \coloneqq h.$$

% Notably, 
This baseline 
assumes that representations at different layers operate in the same linear space.

\subsection{Quality of Fit}
\label{subsec:experiments_r2}

We first evaluate our method by measuring how well the learned linear mappings approximate the representations at the target layer. To this end, we calculate the (coordinate-averaged) $r^2$-score of our mapping's outputs with respect to the representations obtained from a full inference pass, and compare to the same for the \id{} baseline.


\paragraph{Models.}

We use \gpt{} \cite{radford2019language}, a decoder-only auto-regressive LM, with $L = 48$, $d_h = 1600$, and \bert{} \cite{devlin-etal-2019-bert}, an encoder-only model trained with masked language modeling, with $L=24$, $d_h=1024$.
% \footnote{\label{footnote:hf}We use models and data from Huggingface \cite{wolf-etal-2020-transformers,lhoest-etal-2021-datasets}.}
%For masked token prediction, we use a masked LM head pre-trained for our \bert{} model.

% \footnote{Specifically, we use the Huggingface Transformers \cite{wolf-etal-2020-transformers} implementations of all these models.}

%\sy{We use \gpt{} \cite{radford2019language}, a decoder-only auto-regressive LM, coming in four scales; $\texttt{gpt2}$ ($L = 12$, $d_h = 768$), $\texttt{gpt2-medium}$ ($L = 24$, $d_h = 1024$), $\texttt{gpt2-large}$ ($L = 36$, $d_h = 1280$) and $\texttt{gpt2-xl}$ ($L = 48$, $d_h = 1600$). Also, we use \bert{} \cite{devlin-etal-2019-bert}, an encoder-only model trained with masked language modeling, coming in two scales;  \texttt{bert-base-uncased} ($L=12$, $d_h=768$) and \texttt{bert-large-uncased} ($L=24$, $d_h=1024$). For masked token prediction, we use masked LM heads pre-trained for our models. Specifically, we use the Huggingface Transformers \cite{wolf-etal-2020-transformers} implementations of all these models. The plots presented in this section are for $48$-layered \gpt{} and $24$-layered \bert{}.}

%\sy{We use \gpt{} \cite{radford2019language}, a decoder-only auto-regressive LM, in the Huggingface \cite{wolf-etal-2020-transformers} implementation\footnote{\url{https://huggingface.co/gpt2}}, coming in four scales; $\texttt{gpt2}$ ($L = 12$, $d_h = 768$), $\texttt{gpt2-medium}$ ($L = 24$, $d_h = 1024$), $\texttt{gpt2-large}$ ($L = 36$, $d_h = 1280$) and $\texttt{gpt2-xl}$ ($L = 48$, $d_h = 1600$). Also, we use \bert{} \cite{devlin-etal-2019-bert}, an encoder-only model trained with masked language modeling, in the Hugginface implementation, coming in two scales;  \texttt{bert-base-uncased}\footnote{\url{https://huggingface.co/bert-base-uncased}} ($L=12$, $d_h=768$) and \texttt{bert-large-uncased}\footnote{\url{https://huggingface.co/bert-large-uncased}} ($L=24$, $d_h=1024$). For masked token prediction, we use the \texttt{BertForMaskedLM} heads from Huggingface, pretrained for these models. The plots presented in this section are for $48$-layered \gpt{} and $24$-layered \bert{}.}

\paragraph{Data.}
We sample random sentences from Wikipedia,
% \footref{footnote:hf} 
collecting 9,000 (resp. 3,000) sentences for the training set $\mathcal{T}$ (resp. validation set $\mathcal{V}$).\footnote{We use sentences rather than full documents to simplify the analysis.}
%\sy{We use two data sources to evaluate our method. One is Wikiepdia \cite{lhoest-etal-2021-datasets}\footnote{\url{https://huggingface.co/datasets/wikipedia}}; we use \texttt{spaCy}\footnote{\url{https://spacy.io/}} to divide documents into sentences\footnote{We use sentences rather than full documents to simplify the analysis.}\footnote{We pick randomly a Wikipedia document and then pick randomly a sentence ending in a newline character in it. \sy{[maybe this footnote is not needed?]}}, collecting 9,000 (resp. 3,000) random sentences for the training set $\mathcal{T}$ (resp. validation set $\mathcal{V}$). The second is a news article sentences dataset, the 10K English 2020 news sentences corpus
% \footnote{\url{https://downloads.wortschatz-leipzig.de/corpora/eng_news_2020_10K.tar.gz}} from the Leipzig Corpora Collection \cite{goldhahn-etal-2012-building}, which we randomly divide into a training set $\mathcal{T}$ consisting of 9,000 examples and a validation set $\mathcal{V}$ consisting of 1,000 examples.
% We truncate sentences to the maximal token length allowed by the model \mg{do we ever need to truncate? a sentence has about 10 words and the max. input len is thousands} \sy{[I surely did not need to in Leipzig, but discovered (via a transformers runtime warning) that I do need to for some (probably a minority) of the Wikipedia sentences. This probably has to do with that it is not really ``sentences" necessarily, for example, I noticed that it has some listings or something like that (bulleted items)... So some minority might get very long I guess...]}.
For each example $s$, we select a random position $i_s$ and extract the hidden representations $h_{i_s}^{\ell}$ at that position from all the layers.
For \bert{}, we first replace the input token at position $i_s$ with a \mask{} token, as our motivation is interpreting predictions, which are obtained via masked tokens in \bert{} (see \S\ref{subsec:BERT}).
Thus, in this case, the hidden representations we consider
%in the case of \bert{}
are of \mask{} tokens only.
%As we observed highly similar results for the two data sources across all our experiments, throughout the paper we will mainly report results for Wikipedia (except for \S\ref{sec:robustness}, where we cross-validate).


\begin{figure}[t]
\includegraphics[scale=0.2]{figs/r2_scores_48.pdf}
% \includegraphics[width=\columnwidth]{figs/r2_scores_48.pdf}
\caption{The coordinate-averaged $r^2$-score of $\matl{}$ (left) and $\idl{}$ (right) (\gpt{}).}
\label{fig:r2_scores}
\end{figure}


\begin{figure}[t]
\setlength{\belowcaptionskip}{-10pt}
\includegraphics[scale=0.2]{figs/bertmask_r2_scores_24.pdf}
% \includegraphics[width=\columnwidth]{figs/bertmask_r2_scores_24.pdf}
\caption{The coordinate-averaged $r^2$-score of $\matl{}$ (left) and $\idl{}$ (right) (\bert{}).}
\label{fig:bertmask_r2_scores}
\end{figure}



\paragraph{Evaluation.}
For every pair of layers $\ell, \ell'$, such that $0 \leq \ell < \ell' \leq L$, we use the training set $\mathcal{T}$ to fit linear regression as described in \S\ref{subsec:methodology_linear_shortcut}, and obtain a mapping $\matl{}$. 
Next, we evaluate the quality of $\matl{}$ as well as of $\idl{}$ using the $r^2$-coefficient, uniformly averaged over all coordinates. Concretely, we compute the $r^2$-coefficient of each of the predicted representations $\matl{} (h_{i_s}^{\ell})$ and $\idl{} (h_{i_s}^{\ell})$ versus the true representations $h_{i_s}^{\ell'}$
over all $s \in \mathcal{V}$.
%as we vary $s \in \mathcal{V}$.
%for every $s \in \mathcal{V}$.



\paragraph{Results.}
Results for \gpt{} and \bert{} are presented in Figs.~\ref{fig:r2_scores} and~\ref{fig:bertmask_r2_scores}, respectively.
In both models, \mat{} consistently yields better approximations than \id{}, as it obtains higher $r^2$-scores (in blue) across the network. 
This gap between \mat{} and \id{} is especially evident in \bert{}, where \id{} completely fails to map the representations between most layers, suggesting that hidden representations are modified  substantially by every transformer block.
Overall, this highlights the shortcoming of existing practices to inspect representations in the same linear space, and the gains from using our method to approximate future layers.
% in the network.
\section{Linear Shortcut for Language Modeling}
\label{sec:prediction}

We saw that our method approximates future hidden representations substantially better than a naive propagation. 
In this section, we will show that this improvement also translates to better predictive abilities from earlier layers. Specifically, we will use our method to estimate how often intermediate representations encode the final prediction, in the context of two fundamental LM tasks; next token prediction and masked token prediction.

\paragraph{Evaluation Metrics.}
Let $h, h' \in \mathbb{R}^{d_h}$ be a final representation and a substitute final representation obtained by some mapping, and denote by $\delta (h), \delta (h') \in \mathbb{R}^{d_v}$ their corresponding output probability distributions (obtained through projection to the output vocabulary -- see details below). 
We measure the prediction quality of $h'$ with respect to $h$ using two metrics:
\begin{itemize}
[leftmargin=*,topsep=1pt,parsep=1pt]
    \item \textbf{Precision@$k$} ($\uparrow$ is better): This checks whether the token with the highest probability according to $\delta(h')$ appears in the top-$k$ tokens according to $\delta(h)$. Namely, we sort $\delta(h)$ and assign a score of $1$ if $\arg\max(\delta(h'))$ appears in the top-$k$ tokens by $\delta(h)$, and $0$ otherwise.
    
    \item \textbf{Surprisal} ($\downarrow$ is better): We measure the minus log-probability according to $\delta(h)$, of the highest-probability token according to $\delta(h')$. Intuitively, low values mean that the model sees the substitute result as probable and hence not surprising.
\end{itemize}

\noindent We report the average Precision@$k$ and Surprisal over the validation set $\mathcal{V}$.



\subsection{Next Token Prediction}
\label{subsec:next_token_prediction_task}

Auto-regressive LMs output for every position a probability distribution over the vocabulary for the next token. Specifically, the output distribution for every position $i$ is given by $\delta (h_i^L)$, where:
\begin{equation}\label{eq:output_distribution}
    \delta (h) = \texttt{softmax} ( E^\top \cdot h) \in \mathbb{R}^{d_v}
\end{equation}
For some LMs, including \gpt{}, a layer normalization $\texttt{ln\_f}$ is applied to the final layer representation before this conversion (i.e., computing $\delta (\texttt{ln\_f}(h))$ rather than $\delta (h)$).

Recall that our goal is to measure how well this distribution can be estimated from intermediate representations, i.e. estimating $\delta (h_i^L)$ from $\delta (h_i^\ell)$ where $\ell<L$. To this end, we first run examples from the validation set through the model, while extracting for each example $s$ the hidden representation of a random position $i_s$ at every layer. Next, we apply our mappings $\matlL{}$ and the $\idlL{}$ baseline to cast the hidden representations of every layer $\ell$ to final layer substitutes (see \S\ref{sec:layer_jump}). Last, for each layer, we convert its corresponding final-layer substitute to an output distribution (Eq.~\ref{eq:output_distribution}) and compute the average Precision@$k$ (for $k=1,5,10$) and Surprisal scores with respect to the final output distribution, over the validation set.

\paragraph{Results.}
Figs.~\ref{fig:pre} and~\ref{fig:surp} show the average Precision@$k$ and Surprisal scores per layer in $48$-layered \gpt{}, respectively (the plots for the other \gpt{} models are presented in \S\ref{sec:app_scale}). Across all layers, \mat{} outperforms \id{} in terms of both scores, often by a large margin (e.g. till layer $44$ the Precision@$1$ achieved by \mat{} is bigger than that of $\id{}$ by more than $0.2$). 
This shows that linear mappings enable not just better estimation of final layer representations, but also of the predictions they induce. Moreover, the relatively high Precision@$k$ scores of \mat{} in early layers ($0.62$-$0.82$ for $k=10$, $0.52$-$0.74$ for $k=5$, and $0.28$-$0.45$ for $k=1$) suggest that early representations already encode a good estimation of the final prediction. Also, the substantially lower Surprisal scores of \mat{} compared to \id{} imply that our method allows for a more representative reading into the layer-wise prediction-formation of the model than allowed through direct projection to the vocabulary.

\begin{figure}[t]
\centering
\includegraphics[scale=0.4]{figs/pre_48.pdf}
\caption{Precision@$k$ ($k = 1,5, 10$) of $\matlL{}$ and $\idlL{}$ for next token prediction in $48$-layered \gpt{}.}
\label{fig:pre}
\end{figure}

\begin{figure}[t]
\centering
\includegraphics[scale=0.35]{figs/surp_48.pdf}
\caption{Surprisal for $\matlL$ and the baseline $\idlL{}$ ($48$-layered \gpt{} next token prediction task). A 95\% confidence interval surrounds the lines.}
\label{fig:surp}
\end{figure}

\subsection{Masked Token Prediction}
\label{subsec:BERT}

We now conduct the same experiment for the task of masked language modeling, where the model predicts a probability distribution of a masked token in the input rather than the token that follows the input. Unlike next token prediction, where the output distribution is computed from representations of varying input tokens, in masked token prediction the output is always obtained from representations of the same input token (i.e. \texttt{[MASK]}).

For this experiment, we use \bert{}, on top of which we use a pretrained masked language model head $\delta$; given a token sequence $s$, a \mask{} token inside it and its final representation $h$, $\delta (h) \in \mathbb{R}^{d_v}$
 is a probability distribution over tokens giving the model's assessment
 of the likelihood of tokens to be fitting in place of the \mask{} token in $s$.


\begin{figure}[t]
\centering
\includegraphics[scale=0.4]{figs/bertmask_pre_24.pdf}
\caption{Precision@$k$ ($k = 1,5, 10$) for  $\matlL{}$ and the baseline $\idlL{}$ ($24$-layered \bert{} masked token prediction task).}
\label{fig:bertmask_pre}
\end{figure}

\begin{figure}[t]
\centering
\includegraphics[scale=0.35]{figs/bertmask_surp_24.pdf}
\caption{Surprisal for $\matlL{}$ and the baseline $\idlL{}$ ($24$-layered \bert{} masked token prediction task). A 95\% confidence interval surrounds the lines.}
\label{fig:bertmask_surp}
\end{figure}

\paragraph{Results.}
Figs.~\ref{fig:bertmask_pre} and~\ref{fig:bertmask_surp} present the average Precision@$k$ and Surprisal scores per layer in $24$-layered \bert{} (the plots for the $12$-layered \bert{} model are presented in \S\ref{sec:app_scale}), overall showing trends similar to those observed for next token prediction in \gpt{} (\S\ref{subsec:next_token_prediction_task}). This is despite the differences between the two tasks and the considerable architectural differences between \bert{} and \gpt{}.
Notably, the superiority of \mat{} over \id{} in this setting is even more prominent; 
while \mat{}'s precision is between $0.2-0.6$ in the first ten layers (Fig.~\ref{fig:bertmask_pre}), \id{}'s precision for all values of $k$ is close to zero, again strongly indicating that our method allows for better reading into early layer hidden representations. 
More generally, \mat{} improves the Precision@$1$ of \id{} by more than $17\%$ at most layers, and unveils that a substantial amount of predictions ($>25\%$ starting from layer $3$) appear already in the very first layers.
Interestingly, the (rough) divide between the first half of layers and last half of layers for $\id{}$ in Figs.~\ref{fig:bertmask_pre},~\ref{fig:bertmask_surp} seems to align with the two-hump shape of the blue region for $\mat{}$ in Fig.~\ref{fig:bertmask_r2_scores}.

\paragraph{Analysis.}
We manually compare the predictions of our mapping $\matlL{}$ with $\idlL{}$, for a $24$-layered \bert{} model.  Concretely, we select 50 random sentences from the Leipzig dataset. Next, for each layer $\ell$, we manually analyze how many of the top-$5$ tokens according to $\matlL{}$ and $\idlL{}$ fit into context. We consider a token to fit into context if it is grammatically plausible within the sentence (see Tab.~\ref{tab:manual} for concrete examples).
In the resulting $1250$ instances (i.e. $50$ sentences $\times$ $25$ representations), we observe a substantially higher plausibility rate of $85.36\%$ for \mat{} compared to $52.8\%$ for \id{}. In fact, only in less than $4.3\%$ of the instances there are more plausible tokens among the top-$5$ tokens according to \id{} than among the top-$5$ tokens according to \mat{}, further supporting the Surprisal results above.

\begin{table*}
\footnotesize
\setlength{\belowcaptionskip}{-15pt}
\begin{tabular}{p{0.3\linewidth}ccccc}
& $\texttt{id}_{4 \rightarrow 24}$ & $\texttt{mat}_{4 \rightarrow 24}$ & $\texttt{id}_{12 \rightarrow 24}$ & $\texttt{mat}_{12 \rightarrow 24}$ & $\texttt{id}_{24 \rightarrow 24}$ \\ \midrule
\multirow{5}{=}{aldridge had shoulder surgery in \mask{}.} & fellowship & \tcbox{time} & cyclist & \tcbox{2009} & \tcbox{september} \\
& employment & \tcbox{it} & emergencies & \tcbox{2008} & \tcbox{november} \\
& agreement & her & seniors & \tcbox{2010} & \tcbox{december} \\
& \#\#ostal & them & cycling & \tcbox{2006} & \tcbox{august} \\
& \#\#com & work & \tcbox{pennsylvania} & \tcbox{2007} & \tcbox{july} \\ \midrule
\multirow{5}{=}{on your next view you will be asked to \mask{} continue reading.} & \#\#com & be & be & be & \tcbox{please} \\
& accreditation & get & undergo & \tcbox{please} & \tcbox{simply} \\ 
& $	\copyright$ & go & spartans & help & \tcbox{also} \\ 
& fellowship & \tcbox{help} & seniors & \tcbox{simply} & \tcbox{again} \\ 
& summer & have & * & say & \tcbox{immediately} \\ \bottomrule
\end{tabular}
\caption{Examples of top-$5$ predictions at layers $4$, $12$ and $24$, under the mappings $\matlL{}$ and $\idlL{}$, for a $24$-layered \bert{} model. Grammatically plausible predictions (according to a human annotator) are marked in \tcbox{blue}. Note that at layer $24$ the predictions of $\matlL{}$ and $\idlL{}$ are the same (by definition).} 
\label{tab:manual}
\end{table*}

\section{Implication to Early Exiting}
\label{sec:applications}

%The fact that it is often possible to approximate
The possibility of approximating
the final prediction already in the early layers has important implications for efficiency; applying our linear mapping instead of executing transformer blocks of quadratic time complexity, could save a substantial portion of the computation. In this section, we demonstrate this in the context of early exiting.

When 
% performing transformer model inference under 
using an early exit strategy \cite{schwartz-etal-2020-right, xin-etal-2020-deebert, schuster2022confident}, one aims at deciding dynamically at which layer to stop the computation and ``read'' the prediction from the hidden representation of that layer.
More precisely, under a confidence measure paradigm, one decides to stop the computation for a position $i$ at layer $\ell$ based on a confidence criterion, that is derived from casting the hidden representation $h_i^\ell$ as a final-layer representation and converting it to an output probability distribution. Specifically, following \citet{schuster2022confident}, a decision to exit is made if the difference between the highest and the second highest probabilities is bigger than $$ 0.9 \cdot \lambda + 0.1 \cdot {\rm exp} (-4 i / N),$$
where $N$ is the average length of the input until position $i_s$ for $s \in \mathcal{V}$, and $\lambda$ is a hyper-parameter.

\begin{figure}[t]
\setlength{\belowcaptionskip}{-10pt}
\centering
\includegraphics[width=\columnwidth]{figs/ee_gpt2bert.pdf}
\caption{Precision@$1$ with early exit and ``fixed exit'', applied to the $24$-layer \gpt{} for next token prediction (left) and the $24$-layer \bert{} for masked token prediction (right). Varying the confidence parameter $\lambda$, the $x$-coordinate is the average number of layers processed before an early exit decision is reached.}
\label{fig:ee_gpt2bert}
\end{figure}

\quash{
\begin{figure}[t]
\setlength{\belowcaptionskip}{-10pt}
\centering
\includegraphics[scale=0.35]{figs/ee_pre1_24.pdf}
\caption{Precision@$1$ for the various early exit methods, and previous ``fixed exit'' methods for comparison ($24$-layer \gpt{} next token prediction task). Varying the confidence parameter $\lambda$, the $x$-coordinate is the average number of layers processed before an early exit decision is reached.}
\label{fig:ee_pre1}
\end{figure}
}

\paragraph{Experiment.}
We assess the utility of our mapping $\matlL{}$ for early exit as a plug-and-play replacement for $\idlL{}$, through which intermediate representations are cast into final-layer representations.
We use \gpt{} for the next token prediction and \bert{} for masked token prediction (both with 24 layers).
We run each of the models over the validation set examples, while varying the confidence parameter $\lambda$ and using either $\idlL{}$ or $\matlL{}$ for casting intermediate representations.
Furthermore, we compare these early exit variants to the ``fixed exit'' strategy from \S\ref{sec:prediction}, where the computation is stopped after a pre-defined number of layers rather than relying on a dynamic decision.
We evaluate each variant in terms of both prediction's accuracy, using the Precision@$1$ metric (see \S\ref{sec:prediction}), and efficiency, measured as the average number of transformer layers processed during inference.


\paragraph{Results.}
%Figs.~\ref{fig:ee_pre1} and~\ref{fig:bertmask_ee_pre1}
Fig.~\ref{fig:ee_gpt2bert}
plots the average Precision@$1$ score against the average number of layers processed, for $24$-layer \gpt{} and $24$-layer \bert{}. For both models, under an early exit strategy our mapping \mat{} again provides a substantial improvement over \id{}.
For example, aiming at $95\%$ average precision, \mat{} saves $\sim3.3$ ($13.8$\%) layers in \gpt{} compared to only $\sim1.4$ ($5.9$\%) layers by \id{}, and $\sim4.8$ ($20$\%) layers in \bert{} versus $\sim3.5$ ($14.6$\%) layers by \id{}.
These results highlight the potential gains prominent early exit methods can obtain by using our method.
Notably, in both models and for each of the mapping methods, early exit obtains better results than fixed layer exit, as expected. 

\quash{
\begin{figure}[t]
\setlength{\belowcaptionskip}{-10pt}
\centering
\includegraphics[scale=0.35]{figs/bertmask_ee_pre1_24.pdf}
\caption{Precision@$1$ for the various early exit methods, and previous ``fixed exit'' methods for comparison ($24$-layer \bert{} masked token prediction task). Varying the confidence parameter $\lambda$, the $x$-coordinate is the average number of layers processed before an early exit decision is reached.}
\label{fig:bertmask_ee_pre1}
\end{figure}
}
\section{Linear Shortcut Across Sub-Modules}
\label{sec:submodules}

% Our experiments show that
% , despite the commonly-applied simplification by interpretability works, transformer layers do not operate in the same linear space and 
% there is a major gap in approximating future representations using an identity mapping (\S\ref{sec:layer_jump}, \S\ref{sec:prediction}).
% Here, 
In this section, we investigate whether discrepancies across layers result from specific sub-modules or are a general behaviour of all sub-modules in the network.  
This is done by extending our approach to test how well particular components in transformer blocks can be linearly approximated. 


\paragraph{Method.}

Consider \gpt{} for definiteness, then:
% we have 
$$ \texttt{b}_{\ell} = \texttt{b}_{\ell}^{\texttt{ffn}} \circ \texttt{b}_{\ell}^{\texttt{attn}}$$ 
% with
\begin{equation}\label{eq:attn} \texttt{b}^{\texttt{attn}}_{\ell} (H) = \texttt{attn}_{\ell} (\texttt{ln1}_{\ell} (H)) + H,\end{equation} 
where $\texttt{attn}_{\ell}$ is
%a multi-head self-attention
a MHSA
layer and \texttt{ln1} is a layer normalization (LN), and 
$$ \texttt{b}^{\texttt{ffn}}_{\ell} (H) = \texttt{ffn}_{\ell} (\texttt{ln2}_{\ell} (H)) + H,$$  
where $\texttt{ffn}_{\ell}$ is
%a feed-forward network
an FFN
layer and $\texttt{ln2}$ is a LN.
\quash{
Given a block $\texttt{b}_\ell$ and one of its sub-modules $\texttt{ln1}_\ell, \ \texttt{attn}_\ell, \ \texttt{ln2}_\ell$, or $\texttt{ffn}_\ell$, we fit linear regression approximating the output of the sub-module given its input and then use it in order to define mappings, as we now describe.
}
Given a block $\texttt{b}_\ell$ and one of its sub-modules $\texttt{ln1}_\ell, \ \texttt{attn}_\ell, \ \texttt{ln2}_\ell$, or $\texttt{ffn}_\ell$, we fit linear regression approximating the output of the sub-module given its input, and then use it to define mappings $\matattnl{}$, $\matlnl{}$ and $\matffl{}$.
%We provide the definition of $\matattnl{}$ below, and that of the other two in App. \ref{sec:app_submodule_skip_description}.
We provide the formal definitions of these mappings in App. \ref{sec:app_submodule_skip_description}.
\iffalse
\paragraph{$\matattnl{}$.}
%Illustrating this on $\texttt{attn}_\ell$ for definiteness,
For an input $s$, let $v^\ell_{i_s}$ be the vector at position $i_s$ in the output of $\texttt{attn}_\ell (\texttt{ln1}_\ell (H^{\ell - 1}))$. We denote by $A_\ell^{\texttt{attn}} \in \mathbb{R}^{d_h \times d_h}$ the matrix numerically minimizing 
$$ A \mapsto \sum_{s \in \mathcal{T}} || A \cdot \texttt{ln1}_\ell (h^{\ell-1}_{i_s}) - v^\ell_{i_s}||^2,$$
and define an attention sub-module replacement (Eq.~\ref{eq:attn}) by $$
\texttt{b}^{\overline{\texttt{attn}}}_\ell (h) \coloneqq A_{\ell}^{\texttt{attn}} \cdot \texttt{ln1}_\ell (h) + h. $$
We then define a mapping between two layers ${\ell \rightarrow \ell'}$ by:
$$ \matattnl{} (h) \coloneqq $$
$$ \texttt{b}^{\texttt{ffn}}_{\ell'} ( \texttt{b}^{\overline{\texttt{attn}}}_{\ell'} ( \ldots (\texttt{b}^{\texttt{ffn}}_{\ell+1} ( \texttt{b}^{\overline{\texttt{attn}}}_{\ell+1} (h)))\ldots)).$$ 
Namely, when applying each $\ell''$-th block, $\ell < \ell'' \leq \ell'$, we replace its attention sub-module $\texttt{attn}_{\ell''}$ by its linear approximation.
%In an analogous way, we consider the mappings $\matffl{}$ and $\matlnl{}$, where in the latter we perform the linear shortcut both for \texttt{ln1} and for \texttt{ln2} (see~\S\ref{sec:app_submodule_skip_description} for precise descriptions).
Importantly, unlike the original attention module, the approximation $\texttt{b}^{\overline{\texttt{attn}}}_\ell$ operates on each position independently, and therefore applying $\matattnl{}$ disables any contextualization between the layers $\ell$ and $\ell'$. Note that this is not the case for $\matffl{}$ and $\matlnl{}$, which retain the self-attention sub-modules and operate contextually.
\fi

\paragraph{Evaluation.}


We analyze the $24$-layered \gpt{}, and proceed completely analogously to \S\ref{subsec:next_token_prediction_task}, evaluating the Precision@$1$ and Surprisal metrics for the mappings $\matattnlL{}$, $\matfflL{}$ and $\matlnlL{}$.

\begin{figure}[t]
\setlength{\belowcaptionskip}{-0pt}
\centering
%\includegraphics[scale=0.2]
\includegraphics[width=\columnwidth]{figs/parts_presurp_24.pdf}
\caption{Precision@$1$ and Surprisal for the various sub-module linear mappings, and $\matlL{}$ for comparison ($24$-layer \gpt{} next token prediction task). A 95\% confidence interval surrounds the Surprisal lines.}
\label{fig:parts_presurp}
\end{figure}

\quash{
\begin{figure}[t]
\centering
\includegraphics[scale=0.4]{figs/parts_pre1_24.pdf}
\caption{Precision@$1$ for the various sub-module linear shortcut mappings, and the mapping $\matlL{}$ for comparison (\gpt{} next token prediction task).}
\label{fig:parts_pre1}
\end{figure}

\begin{figure}[t]
\centering
\includegraphics[scale=0.35]{figs/parts_surp_24.pdf}
\caption{Surprisal for the various sub-module linear shortcut mappings, and the mapping $\matlL{}$ for comparison (\gpt{} next token prediction task). A 95\% confidence interval surrounds the lines.}
\label{fig:parts_surp}
\end{figure}
}

\paragraph{Results.}
Fig.~\ref{fig:parts_presurp} shows the average Precision@$1$ and Surprisal scores per layer.
From a certain layer (\textasciitilde$7$), all sub-module mappings achieve better results than the full-block mapping $\matlL{}$. Thus, it is not just the cumulative effect of all the sub-modules in the transformer block that is amenable to linear approximation, but also individual sub-modules can be linearly approximated. 
Furthermore, the linear approximation of attention sub-modules is less harmful than that of the FFN or LN sub-modules. 
% Hypothetically, 
A possible reason is that the linear replacement of FFN or LN ``erodes'' the self-attention computation after a few layers. 
Moreover, the good performance of $\matattnlL{}$ suggests that contextualization often exhausts itself in early layers; speculatively, it is only in more delicate cases that the self-attention of late layers adds important information. Last, remark the sharp ascent of the scores for layer normalization in layers $5$-$8$, for which we do not currently see a particular reason. To conclude, we see that the possibility of linear approximation permeates
%the various
transformer components.


\section{Related Work}

Recently, there was a lot of interest in utilizing intermediate representations in transformer-based LMs, both for interpretability and for efficiency.

In the direction of interpretability, one seeks to understand the prediction construction process of the model \cite{tenney-etal-2019-bert, voita-etal-2019-bottom}.

More recent works use mechanistic interpretability and view the inference pass as a residual stream of information \cite{dar2022analyzing,geva-etal-2022-transformer}. Additionally, there are works on probing, attempting to understand what features are stored in the hidden representations \cite{adi2017finegrained, conneau-etal-2018-cram,liu-etal-2019-linguistic}. Our work is different in that it attempts to convert intermediate representations into a final-layer form, which is interpretable by design.

In the direction of efficiency, there is the thread of work on early exit, where computation is cut at a dynamically-decided earlier stage \cite{schwartz-etal-2020-right,xin-etal-2020-deebert,schuster2022confident}. Other works utilize a fixed early stage network to parallelize inference \citep{leviathan2022fast, chen2023accelerating}. However, intermediate representations are directly propagated in these works, which we show is substantially worse than our approach. Moreover, our method requires training considerably less parameters than methods such as \citet{schuster-etal-2021-consistent}, that learn a different output softmax for each intermediate layer.  

More broadly, skipping transformer layers and analyzing the linearity properties of transformer components have been discussed in prior works \cite{Zhao2021of,mickus-etal-2022-dissect,wang-etal-2022-skipbert,lamparth2023analyzing}.


\section{Conclusion and Future Work}

We present a simple and effective method for enhancing utilization of hidden representations in transformer-based LMs, that uses 
pre-fitted context-free and token-uniform linear mappings.
Through a series of experiments on different data sources, model architectures and scales, we show that our method consistently outperforms the prevalent practice of interpreting representations in the final-layer space of the model, yielding better approximations of succeeding representations and the predictions they induce, thus allowing a more faithful interpretation of the model's prediction-formation.
We demonstrate the practicality of our method for improving computation efficiency, saving a substantial amount of compute on top of prominent early exiting approaches. 
Also, by extending our method to sub-modules, 
% more specifically the attention sub-modules, 
we observe that replacing a part of the transformer inference by a non-contextual linear computation often results in a small deterioration of the prediction.
This opens new research directions for improving model efficiency,
% and parallelizability.
% including breaking the computation into several parallelizable tasks.
including breaking the computation into parallel tasks.

\section*{Limitations}

Although we see in this work that there is more linear structure to transformer inference than could be explained solely by the residual connection, we do not elucidate a reason for that. We also do not try to formulate formal criteria according to which to judge, in principle, the quality of ways of short-cutting transformer inference in-between layers. In addition, our experiments cover only English data.


%\section*{Ethics Statement}
%Scientific work published at ACL 2023 must comply with the ACL Ethics Policy.\footnote{\url{https://www.aclweb.org/portal/content/acl-code-ethics}} We encourage all authors to include an explicit ethics statement on the broader impact of the work, or other ethical considerations after the conclusion but before the references. The ethics statement will not count toward the page limit (8 pages for long, 4 pages for short papers).

\section*{Acknowledgements}

We thank Tal Schuster for constructive comments.

% Entries for the entire Anthology, followed by custom entries
\bibliography{anthology,custom}
\bibliographystyle{acl_natbib}

\appendix

\section{Descriptions of $\matattn{}$, $\matff{}$ and $\matln{}$}
\label{sec:app_submodule_skip_description}

Here we detail the definitions of the mappings $\matattnl{}$, $\matffl{}$ and $\matlnl{}$ utilized in \S\ref{sec:submodules}.

\paragraph{Description of $\matattnl{}$.}
%Illustrating this on $\texttt{attn}_\ell$ for definiteness,
For an input $s$, let $v^\ell_{i_s}$ be the vector at position $i_s$ in the output of $\texttt{attn}_\ell (\texttt{ln1}_\ell (H^{\ell - 1}))$. We denote by $A_\ell^{\texttt{attn}} \in \mathbb{R}^{d_h \times d_h}$ the matrix numerically minimizing 
$$ A \mapsto \sum_{s \in \mathcal{T}} || A \cdot \texttt{ln1}_\ell (h^{\ell-1}_{i_s}) - v^\ell_{i_s}||^2,$$
and define an attention sub-module replacement (Eq.~\ref{eq:attn}) by $$
\texttt{b}^{\overline{\texttt{attn}}}_\ell (h) \coloneqq A_{\ell}^{\texttt{attn}} \cdot \texttt{ln1}_\ell (h) + h. $$
We then define a mapping between two layers ${\ell \rightarrow \ell'}$ by:
$$ \matattnl{} (h) \coloneqq $$
$$ \texttt{b}^{\texttt{ffn}}_{\ell'} ( \texttt{b}^{\overline{\texttt{attn}}}_{\ell'} ( \ldots (\texttt{b}^{\texttt{ffn}}_{\ell+1} ( \texttt{b}^{\overline{\texttt{attn}}}_{\ell+1} (h)))\ldots)).$$ 
Namely, when applying each $\ell''$-th block, $\ell < \ell'' \leq \ell'$, we replace its attention sub-module $\texttt{attn}_{\ell''}$ by its linear approximation.
%In an analogous way, we consider the mappings $\matffl{}$ and $\matlnl{}$, where in the latter we perform the linear shortcut both for \texttt{ln1} and for \texttt{ln2} (see~\S\ref{sec:app_submodule_skip_description} for precise descriptions).
Importantly, unlike the original attention module, the approximation $\texttt{b}^{\overline{\texttt{attn}}}_\ell$ operates on each position independently, and therefore applying $\matattnl{}$ disables any contextualization between the layers $\ell$ and $\ell'$. Note that this is not the case for $\matffl{}$ and $\matlnl{}$, which retain the self-attention sub-modules and operate contextually.

\paragraph{Description of $\matffl{}$.}
Let $v^\ell_{i_s}$ be the vector at position $i_s$ in the output of $\texttt{ln2}_{\ell} (\texttt{b}_\ell^{\texttt{attn}} (H^{\ell - 1}))$, for a given input $s$. We denote by $A_\ell^{\texttt{ffn}} \in \mathbb{R}^{d_h \times d_h}$ the matrix numerically minimizing 
$$ A \mapsto \sum_{s \in \mathcal{T}} || A \cdot v^{\ell}_{i_s} - \texttt{ffn}_{\ell} (v^\ell_{i_s})||^2,$$
and define a replacement of the feed-forward sub-module $\texttt{b}_{\ell}^{\texttt{ffn}}$ by $$ \texttt{b}^{\overline{\texttt{ffn}}}_\ell (H) \coloneqq A_{\ell}^{\texttt{ffn}} \cdot \texttt{ln2}_\ell (H) + H.$$
We then define a mapping between two layers ${\ell \rightarrow \ell'}$ by:
$$ \matffl{} (H) \coloneqq $$
$$ \texttt{b}^{\overline{\texttt{ffn}}}_{\ell'} ( \texttt{b}^{\texttt{attn}}_{\ell'} ( \ldots (\texttt{b}^{\overline{\texttt{ffn}}}_{\ell+1} ( \texttt{b}^{\texttt{attn}}_{\ell+1} (H))\ldots)).$$

\paragraph{Description of $\matlnl{}$.}
Let $v^\ell_{i_s}$ be the vector at position $i_s$ in the output of $\texttt{b}^{\texttt{attn}}_{\ell} (H^{\ell - 1})$, for a given input $s$. We denote by $A_\ell^{\texttt{ln1}} \in \mathbb{R}^{d_h \times d_h}$ the matrix numerically minimizing 
$$ A \mapsto \sum_{s \in \mathcal{T}} || A \cdot h^{\ell}_{i_s} - \texttt{ln1}_{\ell} (h^\ell_{i_s})||^2$$ and we denote by $A_\ell^{\texttt{ln2}} \in \mathbb{R}^{d_h \times d_h}$ the matrix numerically minimizing $$ A \mapsto \sum_{s \in \mathcal{T}} || A \cdot v^{\ell}_{i_s} - \texttt{ln2}_{\ell} (v^\ell_{i_s})||^2.$$ We define a replacement of the block $\texttt{b}^{\texttt{attn}}_{\ell}$ by \begin{equation} \texttt{b}^{\overline{\texttt{ln1}}}_\ell (H) \coloneqq \texttt{attn}_{\ell} (A_{\ell}^{\texttt{ln1}} \cdot H) + H\end{equation} and we define a replacement of the block $\texttt{b}^{\texttt{ffn}}_{\ell}$ by \begin{equation} \texttt{b}^{\overline{\texttt{ln2}}}_\ell (H) \coloneqq \texttt{ffn}_{\ell} (A_{\ell}^{\texttt{ln2}} \cdot H) + H.\end{equation}
We then define a mapping between two layers ${\ell \rightarrow \ell'}$ by:
$$ \matlnl{} (H) \coloneqq $$
$$ \texttt{b}^{\overline{\texttt{ln2}}}_{\ell'} ( \texttt{b}^{\overline{\texttt{ln1}}}_{\ell'} ( \ldots (\texttt{b}^{\overline{\texttt{ln2}}}_{\ell+1} ( \texttt{b}^{\overline{\texttt{ln1}}}_{\ell+1} (H))\ldots)).$$


\end{document}


\vspace{1cm}


%% or
%% [B] Manual formatting (see below)
%% (i) We suggest to always provide author, title and journal data or doi:
%% in short all the informations that clearly identify a document.
%% (ii) please avoid comments such as "For a review'', "For some examples",
%% "and references therein" or move them in the text. In general, please leave only references in the bibliography and move all
%% accessory text in footnotes.
%% (iii) Also, please have only one work for each \bibitem.

% \begin{thebibliography}{99}

% \bibitem{a}
% Author,
% \emph{Title},
% \emph{J. Abbrev.} {\bf vol} (year) pg.

% \bibitem{b}
% Author,
% \emph{Title},
% arxiv:1234.5678.

% \bibitem{c}
% Author,
% \emph{Title},
% Publisher (year).

% \end{thebibliography}
\end{document}
