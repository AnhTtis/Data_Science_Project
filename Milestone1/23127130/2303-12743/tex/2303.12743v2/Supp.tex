
\twocolumn[{%
 \centering
 \LARGE Supplementary materials for the paper \\ ``Diversified and Realistic 3D Augmentation via \\ Iterative Construction, Random Placement, and HPR Occlusion''\\[1.5em]
}]


\section{Implementation Details}
\label{sec:implementation_details_augmentation}

\subsection{Iterative Construction}
\label{sec:supp_iterative_construction}


\subsubsection{A.1.1 Indexing Candidate Objects (Pre-processing)} 

\paragraph{Canonical Pose: } A whole-body object is constructed in the canonical pose. As shown in Figure~\ref{fig:Lidar_coordinates.}, (1) we first center the object, $(\text{cx}_i, \text{cy}_i, \text{cz}_i)=(0, 0, 0)$, and (2) rotate the object to have the heading angle of $\theta_i=0$.

\begin{figure}[h!]
\centering
\includegraphics[width=0.8\columnwidth]{figs/figure7-coordinates.png}
\caption{\label{fig:Lidar_coordinates.}
An illustration of 3D coordinates. For a raw dataset, we use the LiDAR coordinates. In our study, we apply the object-level operations, including iterative construction and s-HPR, after we convert each object into the canonical pose. In other words, we center the object to have the same heading angle of $\theta_i=0$. 
Thus, when we convert the object into the canonical pose, all objects are toward the same direction.
(The figure is adapted from Figure~\ref{fig:appendix_car} of \cite{geiger2013vision}.)
}
\end{figure}


\paragraph{Size Similarity of Bounding Boxes: }
We calculate the bounding box similarity with $O_i$ for all the objects $O_j$ ($i\neq j$ and $s_i=s_j$) in the ground-truth database to select the best $2K$ candidates. The bounding box similarity is defined as below:
\begin{equation}
    \text{BoxSimilarity}(i,j) = \frac{\text{Vol}(B_i \cap B_j)}{\text{Vol}(B_i \cup B_j)},
\end{equation}
where $B_i$ is the bounding box of $O_i$ in canonical pose and $\text{Vol}(B)$ denotes the volume of $B$.


\paragraph{Partition Density: }
Among the $2K$ candidates, the best $K$ candidates that have high densities for $O_i$'s low density partitions are selected as the final candidates. We define the partitions as a set of non-overlapping subdivided regions of the object $O_i$. A fixed number of partitions is applied for each class. We evaluate the density of an $O_i$'s partition as the number of points in the partition normalized by the maximum number of points observed for the partition over all the same-class objects in the ground-truth database.


\vspace{1cm}

\subsubsection{A.1.2 Constructing a Whole-body Object (Training Time)}
\paragraph{\phantom{0}}
As explained in the main part of this work, we iteratively construct a whole-body object until the object $O_i$ can be regarded as a whole-body object. We determine the object to be a whole-body object when the proportion of the high density partitions is at least 85\%, where a partition is considered to be a high density if its density is higher than the mean density of the partition excluding the empty.
We also limit the number of iterations to be at most 20.
We provide examples of iterative construction with KITTI dataset in Figure~\ref{fig:appendix_car}, \ref{fig:appendix_ped}, and \ref{fig:appendix_cyc}. Each column corresponds to a single iteration. For each iteration, we use the augmented object in the previous iteration as the source object. At the bottom, we have also indicated the proportion of partitions whose density is higher than the mean. The final results of the whole-body object construction are indicated with red boxes. For a better illustration, we have chosen a proper heading angle $\theta_i$ for each object. The actual iterative construction process is performed in the canonical pose.

\begin{figure}[h!]
    \centering
    \subfloat[]{\includegraphics[height=0.13\paperheight]{figs/figure8-a-car-1.png}} \\
    \subfloat[]{\includegraphics[height=0.13\paperheight]{figs/figure8-b-car-2.png}}
    \caption{\label{fig:appendix_car} Two visualization examples of iterative construction for car. Construction ends after 5 iterations in (a), and 3 in (b). On the average, construction ends after 4.38 iterations for the car object.}
\end{figure}

\begin{figure}[h!]
    \centering
    \subfloat[]{\includegraphics[height=0.13\paperheight]{figs/figure9-a-ped-1.png}} \\
    \subfloat[]{\includegraphics[height=0.13\paperheight]{figs/figure9-b-ped-2.png}}
    \caption{\label{fig:appendix_ped} Two visualization examples of iterative construction for pedestrian. Construction ends after 2 iterations in both (a) and (b). On the average, construction ends after 2.09 iterations for the pedestrian object.}
\end{figure}

\begin{figure}[h!]
    \centering
    \subfloat[]{\includegraphics[height=0.13\paperheight]{figs/figure10-a-cyc-1.png}} \\
    \subfloat[]{\includegraphics[height=0.13\paperheight]{figs/figure10-b-cyc-2.png}}
    \caption{\label{fig:appendix_cyc} Two visualization examples of iterative construction for cyclist. Construction ends after 4 iterations in both (a) and (b). On the average, construction ends after 3.75 iterations for the cyclist object.}
\end{figure}

\subsection{Random placement}
The detection range is [0, 70.4]$m$ for the X axis, [$-40$, $40$]$m$ for the Y axis, and [$-3$, $1$]$m$ for the Z axis, and the size for voxelization is [0.05, 0.05, 0.1]$m$ for the three axes. 
%The meter($m$) is the SI unit of length or distance.
In our method, ten additional whole-body objects are added for each of car, pedestrian, and cyclist, and all the objects are placed by applying random location and random rotation within the detection range. By implementing collision avoidance~\cite{yan2018second}, objects with overlapped bounding boxes are excluded. 


\subsection{HPR occlusion}
For s-HPR, $R$ value needs to be sufficiently large and we set it as the diagonal length of the bounding box multiplied by 200. After fixing the radius $R$, we use the randomized location to determine the $C$ value. Therefore, $C$ is small for an object located close to the LiDAR and large for an object located far from the LiDAR. For e-HPR, $R$ and $z$ values were chosen dependent on the model. The $x$ and $y$ coordinates were set with respect to the LiDAR origin $(0,0,0)$. The parameters for four main models are listed in Table~\ref{tab:hpr_parameter}.
\begin{table}[t!]
\caption{\label{tab:hpr_parameter} The $R$ and $z$ values for the HPR occlusion.}
\vspace{-0.15cm}
\centering
\footnotesize{
\begin{tabular}{@{}cccccccc@{}}
\toprule
    Model      & R   & z \\ \midrule
    SECOND     & 300,000 & 2.0 \\
    PointRCNN  & 300,000 & 1.0 \\
    PV-RCNN    & 100,000 & 0.0 \\
    PV-RCNN++  & 100,000 & 0.0 \\
    \bottomrule
\end{tabular}

\vspace{-0.25cm}
}
\end{table}
\begin{figure}[t!]
     \centering
     \includegraphics[width=0.26\textwidth]{figs/figure11-hpr.png}
     \caption{(Adapted from Figure~\ref{fig:sparsity_with_distance} of \cite{katz2007direct}) HPR can determine the points that are visible from the viewpoint $C$. In the figure, the points inside the sphere of radius $R$ correspond to the original object and the points outside the sphere correspond to the spherically flipped object. HPR determines an original point to be visible from $C$ if its spherically flipped point lies on a convex hull.}
     \label{fig:hpr}
\end{figure}

\subsection{Conventional Data Augmentation}
For the Global Data Augmentation~(GDA), we utilized random flipping along the X axis, global scaling with a random scaling factor sampled from [0.95, 1.05], global rotation around the Z axis with a random angle sampled from [$-\pi$/4,$\pi$/4]. We also conduct the Ground-Truth Sampling~(GTS) to randomly copy and paste objects from the ground-truth database. For each of car, pedestrian, and cyclist, we copy-and-paste 20, 15, and 15 objects, respectively, for each training frame. 

\section{Experimental Details}
\label{sec:implementation_details_experiment}
\subsection{Model and Training Details: }
For the implementation of the three models~(SECOND, PointRCNN, and PV-RCNN) and Conventional Data Augmentation~(CDA), we followed the open source code base https://github.com/open-mmlab/OpenPCDet~\cite{openpcdet2020} and the original hyper-parameter configurations in there.
For PV-RCNN+, however, configuration of KITTI is not available. 
Therefore, we slightly modified PV-RCNN++'s Waymo configuration file by 
comparing it with the PV-RCNN's KITTI configuration file. All the modifications that we have made can be listed as the following. 
%
\begin{itemize}
\item WEIGHT\_DECAY is set to 0.01 instead of 0.001.
\item NUM\_REDUCED\_CHANNELS in raw\_points is set to 1 instead of 2, because KITTI has only one intensity channel in contrast to Waymo with two intensity channels.
\item NUM\_KEYPOINTS is set to 2048 instead of 4096, because average points per frame in KITTI is less than half of those in Waymo.
\end{itemize}

\subsection{Dataset and Evaluation Metric: } 
The models are evaluated using the validation dataset with the standard metric - the average precision (AP) with an IOU threshold value for the car, pedestrian, and cyclist classes of 0.7, 0.5, and 0.5, respectively. As is done most of the other works, we calculated the AP performances with 40 recall positions (R40) on three difficulties (easy, moderate, and hard) that are typically determined based on the object size, occlusion, and truncation levels.

\clearpage
\onecolumn
\section{Additional KITTI Examples}
\label{sec:appendix_examples}
We provide KITTI examples of the random placement, s-HPR, and e-HPR.

\begin{figure*}[h!]
    \label{fig:appendix_ehpr}
    \centering
    \subfloat[]{\includegraphics[width=0.9\textwidth]{figs/figure12-a.png}}\\
    \subfloat[]{\includegraphics[width=0.9\textwidth]{figs/figure12-b.png}}\\
    \subfloat[]{\includegraphics[width=0.9\textwidth]{figs/figure12-c.png}}\\
    \subfloat[]{\includegraphics[width=0.9\textwidth]{figs/figure12-d.png}}
    \caption{Examples of random placement (Left), s-HPR (Middle), and e-HPR (Right) for KITTI dataset.}
\end{figure*}

\clearpage
\section{Full Results}
\label{sec:additional_results}

\begin{table*}[h!]
\centering
\caption{Full results of Table~\ref{tab:model_agnostic}.}
\resizebox{0.9\textwidth}{!}{
\begin{tabular}{@{}llccccccccccccc@{}}
\toprule
\multicolumn{2}{c}{\multirow{2}{*}{Method}}                               & \multicolumn{3}{c}{Car} & \multicolumn{3}{c}{Pedestrian} & \multicolumn{3}{c}{Cyclist} & \multicolumn{3}{c}{Overall} & \multirow{2}{*}{Mean}           \\\cmidrule(lr){3-5}\cmidrule(lr){6-8}\cmidrule(lr){9-11}\cmidrule(lr){12-14}
\multicolumn{1}{c}{} & \multicolumn{1}{c}{}              & E        & M        & D        & E           & M          & D          & E          & M         & D         & Car     & Ped     & Cyc     &                \\ \midrule
Voxel-based          & SECOND~\cite{yan2018second}      & \textbf{90.97}    & 79.94    & 77.09    & \textbf{58.01}       & 51.88      & 47.05      & 78.50      & 56.74     & 52.83     & 82.67   & 52.31   & 62.69   & 65.89          \\
                     & SECOND with \textsc{DR.CPO}                     & 90.30    & \textbf{81.10}    & \textbf{78.04}    & 57.35       & 51.02      & 46.65      & \textbf{87.48}      & \textbf{68.94}     & \textbf{64.84}     & \textbf{83.15}   & 51.67   & \textbf{73.75}   & \textbf{69.53} \\ \midrule
Point-based          & Point-RCNN~\cite{hu2021pattern} & 90.10    & 80.41    & \textbf{78.00}    & 64.18       & 56.71      & 49.86      & \textbf{91.72}      & 72.47     & 68.18     & 82.84   & 56.92   & 77.46   & 72.40          \\
                     & Point-RCNN with \textsc{DR.CPO}                 & 89.51    & 79.82    & 77.75    & 67.05       & \textbf{59.86}      & \textbf{52.66}      & 91.48      & \textbf{73.42}     & \textbf{69.74}     & 82.36   & \textbf{59.86}   & \textbf{78.22}   & \textbf{73.48} \\ \midrule

                  
Voxel+point-based   & PV-RCNN~\cite{shi2020pv}        & \textbf{92.57}  & \textbf{84.83} & \textbf{82.69} & 64.26         & 56.67          & 51.91                                         & 88.88          & 71.95          & 66.78          & \textbf{86.70} & 57.61 & 75.87 & 73.39 \\ 
                    % & PV-RCNN (Reproduced)              & 91.80           & 84.55          & 82.38          & 63.86          & 56.99          & 51.98                     & 91.06          & 72.11          & 67.38          & 86.24 & 57.61 & 76.85 & 73.57 \\ 
                    & PV-RCNN with \textsc{DR.CPO}                    & 91.07           & 81.50          & 81.68          & \textbf{68.66} & \textbf{61.20} & \textbf{57.11} & \textbf{93.77} & \textbf{77.06} & \textbf{73.57} & 84.75 & \textbf{62.33} & \textbf{81.47} & \textbf{76.18} \\  \bottomrule
\end{tabular}
}
\vspace{-0.4cm}
\end{table*}


\begin{table*}[h!]
\caption{Full results of Table~\ref{tab:construction}.}
\centering
\resizebox{0.9\textwidth}{!}{
\begin{tabular}{@{}ccccccccccccccc@{}}
\toprule
\multirow{2}{*}{Mirroring} & \multirow{2}{*}{Adding Candidates} & \multicolumn{3}{c}{Car} & \multicolumn{3}{c}{Pedestrian} & \multicolumn{3}{c}{Cyclist} & \multicolumn{3}{c}{Overall} & \multirow{2}{*}{Mean}\\ \cmidrule(lr){3-5}\cmidrule(lr){6-8}\cmidrule(lr){9-11}\cmidrule(lr){12-14}
 & & E      & M      & D     & E        & M        & D        & E       & M       & D       & Car     & Ped     & Cyc     &                    \\ \midrule
No & None     & 91.45 & 82.13 & 81.66 & 67.77 & 62.94 & 58.32 & 92.18 & 75.02 & 72.06 & 85.08 & 63.01 & 79.75 & 75.95              \\ 
Yes & None    & 91.88 & 82.10 & 81.58 & 68.59 & 62.39 & 58.94 & 92.37 & 75.15 & 71.56 & 85.18 & 63.31 & 79.69 & 76.06         \\
Yes & Single & 91.68  & 82.18  & 81.91 & 73.21    & 67.56    & 62.78    & 93.34   & 75.67   & 72.69   & 85.25   & 67.85   & 80.57   & 77.89         \\
Yes & Multiple & 91.27  & 81.67  & 81.28 & 73.23    & 67.66    & 62.96    & 93.08   & 78.21   & 75.32   & 84.74   & 67.95   & 82.20   & 78.30         \\ \bottomrule
\end{tabular}
}
\vspace{-0.4cm}
\end{table*}


\begin{table}[h!]
\caption{Full results of Table~\ref{tab:analysis_placement1}.}
\centering
\resizebox{0.9\columnwidth}{!}{
\begin{tabular}{@{}ccccccccccccccccc@{}}
\toprule
    \multirow{2}{*}{\begin{tabular}[c]{@{}l@{}}Random \\ Location\end{tabular}}  & \multirow{2}{*}{\begin{tabular}[c]{@{}l@{}}Random \\ Rotation\end{tabular}}   & \multicolumn{3}{c}{Car} & \multicolumn{3}{c}{Pedestrian}      & \multicolumn{3}{c}{Cyclist}   & \multicolumn{3}{c}{Overall} & \multirow{2}{*}{Mean} \\\cmidrule(lr){3-5}\cmidrule(lr){6-8}\cmidrule(lr){9-11}\cmidrule(lr){12-14}
     & & E & M & D & E & M & D & E & M & D & E & M & D & \\\midrule
    
                    &              & 91.46 & 81.50 & 79.98 & 70.63 & 65.02 & 61.10 & 91.32 & 75.98 & 72.25    & 84.31 &  65.58        & 79.85     & 76.58 \\
     \checkmark     &                   & 91.37 & 81.51 & 79.95 & 72.36 & 66.65 & 62.03 & 90.44 & 75.20 & 72.31 & 84.28 &  67.01        & 79.32     & 76.87 \\ 
                    & \checkmark        & 91.91 & 82.21 & 81.85 & 71.06 & 65.81 & 61.12 & 93.69 & 79.28 & 76.52 & 85.33 &  66.00        & 83.16     & 78.16 \\
     \checkmark     & \checkmark        & 91.27 & 81.67 & 81.28 & 73.23 & 67.66 & 62.96 & 93.08 & 78.21 & 75.32 & 84.74 &  67.95        & 82.20     & 78.30 \\
    \bottomrule
\end{tabular}
}
\vspace{-0.4cm}
\end{table}

\begin{table*}[h!]
\caption{Full results of Table~\ref{tab:analysis_placement2}.}
\centering
\resizebox{0.9\textwidth}{!}{
\begin{tabular}{@{}cccccccccccccc@{}}
\toprule
\multirow{2}{*}{Rotation range} & \multicolumn{3}{c}{Car} & \multicolumn{3}{c}{Pedestrian} & \multicolumn{3}{c}{Cyclist} & \multicolumn{3}{c}{Overall} & \multirow{2}{*}{Mean}        \\ \cmidrule(lr){2-4}\cmidrule(lr){5-7}\cmidrule(lr){8-10}\cmidrule(lr){11-13}
                & E      & M      & D     & E        & M        & D        & E       & M       & D       & Car     & Ped     & Cyc     &  \\ \midrule
$(-0.25\pi, +0.25\pi)$       & 91.42  & 82.26  & 81.70 & 70.60    & 63.73    & 58.92    & 93.54   & 75.87   & 73.21   & 85.13   & 64.41   & 80.87   & 76.81   \\
$(-0.50\pi, +0.50\pi)$       & 91.80  & 82.32  & 81.96 & 71.10    & 64.82    & 59.04    & 92.47   & 75.74   & 73.19   & 85.36   & 64.99   & 80.46   & 76.94   \\
$(-0.75\pi, +0.75\pi)$     & 91.34  & 82.25  & 80.08 & 71.37    & 65.33    & 60.30    & 94.57   & 77.08   & 73.18   & 84.56   & 65.67   & 81.61   & 77.28   \\
$(-1.00\pi, +1.00\pi)$     & 91.27  & 81.67  & 81.28 & 73.23    & 67.66    & 62.96    & 93.08   & 78.21   & 75.32   & 84.74   & 67.95   & 82.20   & 78.30   \\ \bottomrule
\end{tabular}
}
\vspace{-0.4cm}
\end{table*}

\begin{table}[thb!]
\caption{Full results of Table~\ref{tab:hpr_ablation}}
\centering
\resizebox{0.9\columnwidth}{!}{
\begin{tabular}{@{}ccccccccccccccccc@{}}
\toprule
    \multirow{2}{*}{s-HPR} & \multirow{2}{*}{e-HPR} & \multicolumn{3}{c}{Car} & \multicolumn{3}{c}{Pedestrian} & \multicolumn{3}{c}{Cyclist} & \multicolumn{3}{c}{Overall} & \multirow{2}{*}{Mean} \\ \cmidrule(lr){3-5}\cmidrule(lr){6-8}\cmidrule{9-11}\cmidrule(lr){12-14}
      & & E              & M              & D              & E              & M              & D              & E              & M              & D              & Car            & Ped            & Cyc            &         \\ \midrule
                &               & 91.79 & 82.13 & 80.01 & 57.66 & 52.18 & 47.16 & 93.25 & 73.72 & 69.84 & 84.64 & 52.33 & 78.94 & 71.97 \\
                & \checkmark    & 91.50 & 81.82 & 81.61 & 69.75 & 64.58 & 60.17 & 90.16 & 76.74 & 73.77 & 84.98 & 64.83 & 80.22 & 76.68 \\
     \checkmark &               & 91.72 & 82.70 & 82.00 & 69.39 & 63.72 & 57.63 & 95.95 & 78.97 & 74.24 & 85.47 & 63.58 & 83.05 & 77.37 \\
     \checkmark & \checkmark    & 91.27 & 81.67 & 81.28 & 73.23 & 67.66 & 62.96 & 93.08 & 78.21 & 75.32 & 84.74 & 67.95 & 82.20 & 78.30 \\ 
    \bottomrule
\end{tabular}
}
\vspace{-0.4cm}
\end{table}

\section{Additional Results}
\label{sec:Supplementary_E}

\begin{table}[h!]
\caption{\label{tab:additional_experiment}Performance trade-off: by adding three additional cars into each training scene, the car detection performance can be improved at the cost of the pedestrian detection performance. Note that the mean performance remains almost the same.
}

\centering
\resizebox{0.9\columnwidth}{!}{
\begin{tabular}{@{}lccccccccccccc@{}}
\toprule
\multirow{2}{*}{Method} & \multicolumn{3}{c}{Car} & \multicolumn{3}{c}{Pedestrian} & \multicolumn{3}{c}{Cyclist} & \multicolumn{3}{c}{Overall} & \multirow{2}{*}{Mean}        \\ \cmidrule(lr){2-4}\cmidrule(lr){5-7}\cmidrule(lr){8-10}\cmidrule(lr){11-13}
                & E      & M      & D     & E        & M        & D        & E       & M       & D       & Car     & Ped     & Cyc     &  \\ \midrule
PVRCNN++ baseline          & 91.72 & 84.82 & 82.03 & 66.00 & 60.05 & 54.70 & 91.39 & 70.14 & 66.92 & 86.19 & 60.25 & 76.15 & 74.20 \\
DR.CPO                     & 91.27 & 81.67 & 81.28 & 73.23 & 67.66 & 62.96 & 93.08 & 78.21 & 75.32 & 84.74 & \textbf{67.95} & 82.20 & 78.30 \\
DR.CPO (3 additional cars) & 92.14	& 84.54	& 82.60	& 71.67	& 65.63	& 60.78	& 93.88	& 78.35 & 75.44 & \textbf{86.43} & 66.03 & 82.56 & 78.34 \\ \bottomrule
\end{tabular}
}
\vspace{-0.4cm}
\end{table}

\twocolumn

