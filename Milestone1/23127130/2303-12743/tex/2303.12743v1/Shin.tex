%File: formatting-instructions-latex-2023.tex
%release 2023.0
\documentclass[letterpaper]{article} % DO NOT CHANGE THIS
\usepackage{aaai23}  % DO NOT CHANGE THIS
\usepackage{times}  % DO NOT CHANGE THIS
\usepackage{helvet}  % DO NOT CHANGE THIS
\usepackage{courier}  % DO NOT CHANGE THIS
\usepackage[hyphens]{url}  % DO NOT CHANGE THIS
\usepackage{graphicx} % DO NOT CHANGE THIS
\urlstyle{rm} % DO NOT CHANGE THIS
\def\UrlFont{\rm}  % DO NOT CHANGE THIS
\usepackage{natbib}  % DO NOT CHANGE THIS AND DO NOT ADD ANY OPTIONS TO IT
\usepackage{caption} % DO NOT CHANGE THIS AND DO NOT ADD ANY OPTIONS TO IT
\frenchspacing  % DO NOT CHANGE THIS
\setlength{\pdfpagewidth}{8.5in}  % DO NOT CHANGE THIS
\setlength{\pdfpageheight}{11in}  % DO NOT CHANGE THIS
\usepackage{algorithm}
\usepackage{algorithmic}
\usepackage{newfloat}
\usepackage{listings}
\DeclareCaptionStyle{ruled}{labelfont=normalfont,labelsep=colon,strut=off} % DO NOT CHANGE THIS
\lstset{%
	basicstyle={\footnotesize\ttfamily},% footnotesize acceptable for monospace
	numbers=left,numberstyle=\footnotesize,xleftmargin=2em,% show line numbers, remove this entire line if you don't want the numbers.
	aboveskip=0pt,belowskip=0pt,%
	showstringspaces=false,tabsize=2,breaklines=true}
\floatstyle{ruled}
\newfloat{listing}{tb}{lst}{}
\floatname{listing}{Listing}
%
% Keep the \pdfinfo as shown here. There's no need
% for you to add the /Title and /Author tags.
\pdfinfo{
/TemplateVersion (2023.1)
}

%%%%%%%%%%
\usepackage{subfig}
\usepackage{amssymb}
\usepackage{mathtools}
\usepackage{booktabs}
\usepackage{multirow}
%%%%%%%%%%

\usepackage{hyperref}
\hypersetup{
    colorlinks=true,
    citecolor=black,
    linkcolor=black,
    filecolor=black,      
    urlcolor=blue,
}

% DISALLOWED PACKAGES
% \usepackage{authblk} -- This package is specifically forbidden
% \usepackage{balance} -- This package is specifically forbidden
% \usepackage{color (if used in text)
% \usepackage{CJK} -- This package is specifically forbidden
% \usepackage{float} -- This package is specifically forbidden
% \usepackage{flushend} -- This package is specifically forbidden
% \usepackage{fontenc} -- This package is specifically forbidden
% \usepackage{fullpage} -- This package is specifically forbidden
% \usepackage{geometry} -- This package is specifically forbidden
% \usepackage{grffile} -- This package is specifically forbidden
% \usepackage{hyperref} -- This package is specifically forbidden
% \usepackage{navigator} -- This package is specifically forbidden
% (or any other package that embeds links such as navigator or hyperref)
% \indentfirst} -- This package is specifically forbidden
% \layout} -- This package is specifically forbidden
% \multicol} -- This package is specifically forbidden
% \nameref} -- This package is specifically forbidden
% \usepackage{savetrees} -- This package is specifically forbidden
% \usepackage{setspace} -- This package is specifically forbidden
% \usepackage{stfloats} -- This package is specifically forbidden
% \usepackage{tabu} -- This package is specifically forbidden
% \usepackage{titlesec} -- This package is specifically forbidden
% \usepackage{tocbibind} -- This package is specifically forbidden
% \usepackage{ulem} -- This package is specifically forbidden
% \usepackage{wrapfig} -- This package is specifically forbidden
% DISALLOWED COMMANDS
% \nocopyright -- Your paper will not be published if you use this command
% \addtolength -- This command may not be used
% \balance -- This command may not be used
% \baselinestretch -- Your paper will not be published if you use this command
% \clearpage -- No page breaks of any kind may be used for the final version of your paper
% \columnsep -- This command may not be used
% \newpage -- No page breaks of any kind may be used for the final version of your paper
% \pagebreak -- No page breaks of any kind may be used for the final version of your paperr
% \pagestyle -- This command may not be used
% \tiny -- This is not an acceptable font size.
% \vspace{- -- No negative value may be used in proximity of a caption, figure, table, section, subsection, subsubsection, or reference
% \vskip{- -- No negative value may be used to alter spacing above or below a caption, figure, table, section, subsection, subsubsection, or reference

\setcounter{secnumdepth}{2} %May be changed to 1 or 2 if section numbers are desired.

% The file aaai23.sty is the style file for AAAI Press
% proceedings, working notes, and technical reports.
%

% Title

% Your title must be in mixed case, not sentence case.
% That means all verbs (including short verbs like be, is, using,and go),
% nouns, adverbs, adjectives should be capitalized, including both words in hyphenated terms, while
% articles, conjunctions, and prepositions are lower case unless they
% directly follow a colon or long dash
\title{Diversified and Realistic 3D Augmentation via \\ Iterative Construction, Random Placement, and HPR Occlusion}
\author{
    %Authors
    % All authors must be in the same font size and format.
    Jungwook Shin\textsuperscript{\rm 1, 2},
    Jaeill Kim\textsuperscript{\rm 1},
    Kyungeun Lee\textsuperscript{\rm 1},
    Hyunghun Cho\textsuperscript{\rm 1},
    Wonjong Rhee\textsuperscript{\rm 1, 3, 4} %\thanks{Corresponding author.}
}

\affiliations{
    % Affiliations
    \textsuperscript{\rm 1} Department of Intelligence and Information,
    Seoul National University,
    Seoul, 08826, South Korea\\
    \textsuperscript{\rm 2} SK Telecom Co., Ltd, 
    Seoul, 04539, South Korea \\    
    \textsuperscript{\rm 3} Interdisciplinary Program in Artificial Intelligence (IPAI),
    Seoul National University,
    Seoul, 08826, South Korea\\
    \textsuperscript{\rm 4} Artificial Intelligence Institute,
    Seoul National University,
    Seoul, 08826, South Korea\\
    %
    \{jungwook.shin, jaeill0704, ruddms0415, webofthink, wrhee\}@snu.ac.kr
}


\begin{document}

\maketitle

\begin{abstract}
In autonomous driving, data augmentation is commonly used for improving 3D object detection. The most basic methods include insertion of copied objects and rotation and scaling of the entire training frame. Numerous variants have been developed as well. The existing methods, however, are considerably limited when compared to the variety of the real world possibilities. In this work, we develop a diversified and realistic augmentation method that can flexibly construct a whole-body object, freely locate and rotate the object, and apply self-occlusion and external-occlusion accordingly. To improve the diversity of the whole-body object construction, we develop an iterative method that stochastically combines multiple objects observed from the real world into a single object. Unlike the existing augmentation methods, the constructed objects can be randomly located and rotated in the training frame because proper occlusions can be reflected to the whole-body objects in the final step. Finally, proper self-occlusion at each local object level and external-occlusion at the global frame level are applied using the Hidden Point Removal~(HPR) algorithm that is computationally efficient. HPR is also used for adaptively controlling the point density of each object according to the object's distance from the LiDAR. Experiment results show that the proposed \textsc{DR.CPO} algorithm is data-efficient and model-agnostic without incurring any computational overhead. Also, \textsc{DR.CPO} can improve mAP performance by 2.08\% when compared to the best 3D detection result known for KITTI dataset. The code is available at \href{https://github.com/SNU-DRL/DRCPO.git}{https://github.com/SNU-DRL/DRCPO.git}
\end{abstract}

\section{Introduction}
\label{sec:intro}
\begin{figure*}[t!]
    \centering
    \subfloat[Iterative Construction]{
    \includegraphics[width=\textwidth]{figs/figure1-a.png}
    } \\ \vspace{0.1cm}
    \subfloat[Random Placement]{
    \includegraphics[width=0.33\textwidth]{figs/figure1-b.png}}
    \subfloat[HPR Occlusion (s-HPR)]{
    \includegraphics[width=0.33\textwidth]{figs/figure1-c.png}}
    \subfloat[HPR Occlusion (e-HPR)]{
    \includegraphics[width=0.33\textwidth]{figs/figure1-d.png}}
    \vspace{0.1cm}
    \caption{Illustration of \textsc{DR.CPO} with 3D modeling objects: 
    \label{fig:overall}
\textbf{(a) Iterative construction}: \textsc{DR.CPO} constructs a whole-body 3D object via an iterative process where each source's completion candidates are pre-selected and indexed before the model training starts.
\textbf{(b) Random placement}: The constructed whole-body object can be randomly located anywhere in the frame with a random rotation as long as its bounding box does not overlap with an existing object.
\textbf{(c, d) HPR occlusion}:
(c) s-HPR: For each individual object, a self-occlusion is applied via HPR where the distance and visibility from the fixed LiDAR viewpoint are reflected for the randomly chosen location and rotation.
(d) e-HPR: For the entire frame, an external-occlusion is applied via HPR where the inter-object spacial dependency is reflected. Some of the objects can be completely removed.
}
\vspace{0.1cm}
\end{figure*}

3D object detection using point cloud has become an important research topic~\cite{zhou2018voxelnet,he2020structure,shi2020point,yang2018pixor,pan20213d}, especially in the field of autonomous driving~\cite{geiger2013vision}. While there has been a large improvement in designing network backbone~\cite{zhou2018voxelnet,lang2019pointpillars,yan2018second,shi2019part,shi2020pv,shi2021pv} and detection head~\cite{yin2021center,zheng2021cia,hu2022afdetv2}, there has been relatively less effort on devising augmentation methods specific to the point cloud~\cite{fang2020augmented,fang2021lidar,hu2020you}.

In 3D object detection, most of the models utilize global data augmentation that is a conventional method applied to all points in a valid point cloud range.
Oversampling~\cite{yan2018second} is another conventional method that has been suggested to alleviate the extreme imbalance between the foreground and background classes. Although global augmentation and oversampling can improve the detection performance, they are straightforward methods that do not consider enhancing object-level diversity.

There have been efforts to provide object-level diversity by adopting 2D image augmentation methods such as random rotation and translation~\cite{zhou2018voxelnet,lang2019pointpillars,engelcke2017vote3deep,wang2019scnet,yang2019std}. 
The domain gap between 2D image and 3D point cloud, however, is significant and the 2D augmentation methods do not fully utilize the properties specific to 3D~\cite{choi2021part}.
Therefore, it is crucial to focus on the distinct properties of point-cloud and develop specialized augmentation methods.

Point clouds have special properties that are distinct from those of images. Most importantly, point clouds exhibit complex \textit{occlusion} because the viewpoint from LiDAR is fixed. For images, occluded pixels can be easily found by correspondence. For point clouds, there are two obstacles for manipulating occlusions~\cite{xu2021behind}. At a local object level, one part of the object is occluded by another part of the same object~(\textit{self-occlusion}). At the global level, the limitation can be extended to the entire frame where a given object can be occluded by another object~(\textit{external-occlusion}). Other properties of point cloud include \textit{point density} and \textit{intensity}. In point cloud, the number of points per a unit volume is dependent on the distance to LiDAR and each point has its own intensity which is not easy to simulate~\cite{yue2018lidar}.

To deal with the properties of 3D point cloud, we propose \textsc{DR.CPO} (\textbf{D}iversified and \textbf{R}ealistic 3D augmentation via iterative \textbf{C}onstruction, random \textbf{P}lacement, and HPR \textbf{O}cclusion). It can generate infinitely diverse frames with diversified objects, placements, and occlusions without requiring the expense of additional data collection or computational overhead. \textsc{DR.CPO} consists of three steps: (1) iterative construction, (2) random placement, and (3) HPR~(Hidden Point Removal) occlusion.
First, we generate a whole-body object in a stochastic manner using the objects from the real point cloud's ground-truth database. 
Second, we randomly rotate and place objects to anywhere inside the LiDAR detection range. Thirdly, we apply self-occlusion and external-occlusion to generate realistic views based on HPR algorithm~\cite{katz2007direct,mehra2010visibility}. Applying self-occlusion not only removes the invisible parts of an object from the LiDAR, but also controls the point density depending on the distance from the LiDAR.
Then, applying external-occlusion affects the entire frame considering the inter-object spatial dependency.
The process of \textsc{DR.CPO} is illustrated in Figure~\ref{fig:overall}, and our contributions can be summarized as below.

\begin{itemize}
    \item We introduce \textsc{DR.CPO}, a data augmentation algorithm specifically developed for 3D point clouds, to provide diversified and realistic views at the level of individual objects and at the level of entire frame.
    \item We develop an iterative method for constructing realistic whole-body objects where only real observations are used.
    \item We present a computationally light occlusion method based on HPR. Our method can implement self-occlusion, external-occlusion, and density adjustment. 
    \item We show that \textsc{DR.CPO} is data efficient, model-agnostic, and computationally efficient.
    \item We achieve state-of-the-art performance on the KITTI dataset. \textsc{DR.CPO} improves mAP performance by 2.08\%.
\end{itemize}

\section{Related Works}
\label{sec:related_works}
\vspace{0.1cm}
\paragraph{3D Object Detection: }
In voxel-based methods, VoxelNet~\cite{zhou2018voxelnet} divides point clouds into voxels where linear network, such as PointNet~\cite{qi2017pointnet}, is then applied to convert the voxels into 3D tensors.
By adopting a method of sparse convolution, SECOND~\cite{yan2018second} alleviated the computational challenges in VoxelNet, and hence improved the speed.
In point-based methods, Point-RCNN~\cite{shi2019pointrcnn} directly generates 3D proposals from raw points in a bottom-up manner and, in the second stage, the proposals are refined by combining local spatial features and global semantic features learned in the first stage.
In voxel+point based methods, PV-RCNN~\cite{shi2020pv} implements a method incorporating both voxel-based operation to generate 3D proposals and point-based operation to refine the proposals.
By implementing an advanced detection head on top of PV-RCNN, PV-RCNN++~\cite{shi2021pv} has become an improved framework that is more than 2x faster and requires less memory than PV-RCNN while achieving a comparable or better performance. Considering the advantages of PV-RCNN++, we use PV-RCNN++ as the default backbone in this work.

\vspace{0.1cm}
\paragraph{Conventional 3D Augmentations: }
Among the 3D data augmentation methods that are mostly extended from 2D augmentations (e.g., translation, rotation, scaling, flipping, and removal~\cite{qi2018frustum, wang2019frustum, shi2020pv, chen2020object, zhu2019class, hahner2020quantifying, reuse2021ambiguity}),
ground-truth sampling (GTS)~\cite{yan2018second, lehner2019patch} is a simple copy-and-paste augmentation strategy and has become popular despite its lack of consideration of occlusion between objects.
Global data augmentation (GDA)~\cite{zhou2018voxelnet, hu2022afdetv2} manipulates the entire frame globally and is another effective augmentation strategy~\cite{reuse2021ambiguity}.
After the combination of GTS and GDA was proposed as a default data augmentation set in~\cite{yan2018second}, the combination was widely employed in the subsequent studies~\cite{lang2019pointpillars, shi2019pointrcnn, shi2020pv, he2020structure}.
In our work, we refer to the augmentation that consists of GTS and GDA as the Conventional Data Augmentation~(CDA).

\vspace{0.1cm}
\paragraph{Constructing Whole-body Object: }
Learning-based approaches~\cite{tchapmi2019topnet,yang2018foldingnet,xie2020grnet} used a 3D modeling dataset including a full shape, such as Shapenet~\cite{chang2015shapenet}. However, all objects in autonomous driving dataset have incomplete shapes, including occlusions and signal misses, making a learning-based approach challenging. To use learning-based approaches, a new dataset is required for training.
Rendering-based approaches~\cite{fang2020augmented} attempted to augment 3D objects using CAD models; however, the intensity attribute was ignored in their experiments. \citet{fang2021lidar} constructed shapes using raycasting and rendering based on the depth map of 3D CAD model. In these rendering-based approaches, as with data labeling, CAD models must be manually prepared.
Geometry-based approaches~\cite{xu2021behind} constructed an object using a heuristic score that was calculated by bounding box similarity, Chamfer distance, and overlapped voxel quantity. Inspired by this, we construct a whole-body using only geometric information without the use of external data.

\vspace{0.1cm}
\paragraph{Occlusion-based Augmentation:}
To consider the shape miss in LiDAR data caused by external-occlusion, signal miss, and self-occlusion~\cite{xu2021behind}, a few works have proposed occlusion-aware augmentation methods.
In \cite{hu2020you}, the destroyed information about visibility in LiDAR data was recovered by raycasting algorithm.
When inserting virtual objects for augmentation, a visibility constraint was used to ensure that the objects are placed where they should not be occluded.
The LiDAR rendering technique automatically generates virtual objects that satisfy the occlusion constraint~\cite{abu2018augmented,fang2020augmented}. Recently, \cite{fang2021lidar} rendered virtual objects with occlusion constraint and placed them in reasonable locations of real background frames using ValidMap.
In our work, because we enforce self and external occlusion via HPR after placing virtual objects, they can be located anywhere without the need of concerning visibility or ValidMap.


\section{Methods}
\label{sec:methods}

\begin{figure}[t!]
\centering
\includegraphics[width=\columnwidth]{figs/figure2-pipeline.png}
\caption{\label{fig:augmentation_pipe}
Pipeline of \textsc{DR.CPO}: In the pre-processing phase, candidates for completion are selected for each source 3D object. The chosen candidates are indexed and additionally stored in the annotated ground-truth database such that the iterative construction step can be computed efficiently. Random placement and HPR occlusion follow the iterative construction step.}
\vspace{0.1cm}
\end{figure}

\textsc{DR.CPO} aims to generate diversified and realistic augmentations for improving 3D object detection. The pipeline of \textsc{DR.CPO} is shown in Figure~\ref{fig:augmentation_pipe} where it consists of a pre-processing step and three training-time steps.
Before presenting the details, we briefly address the notations.
A point cloud~($PC$) of a frame can be considered as the union of foreground points~($FP$) and background points~($BP$), and it can be formulated as below:
\begin{equation}
    {PC} = {FP} \cup {BP}, \ FP = \cup_{i=1}^N O_i.
\end{equation}
In the equation, $O_i$ is the $i$'th object in the frame and its information can be expressed as $(P_i, s_i, B_i)$.
\begin{itemize}
\item{$P_i$ refers to a set of points, $P_i=\left\{ p_{i,1}, \dots, p_{i,T_{i}} \right\}$, where each point $p_{i,t}$ is a 4-dimensional vector $(x, y, z, r)$ with the first three as the 3D coordinates and $r$ as the point's reflection intensity.}
\item{$s_i$ is the ground-truth class label with three possible classes: $s_i\in\left\{ \text{Car}, \text{Pedestrian}, \text{Cyclist} \right\}$.} 
\item{$B_i$ contains the annotation of $O_i$'s 3D bounding box. Specifically, $B_i=(\text{cx}_i, \text{cy}_i, \text{cz}_i, l_i, w_i, h_i, \theta_i)$ with the first three as the center's coordinates, the following three as the length, width, and height of the bounding box, and $\theta_i$ as the heading angle of the bounding box.}
\end{itemize}

\subsection{Iterative Construction}
The goal of this step is to sample a source object $O_i$ from the ground-truth database and to construct it back to a whole-body object. To do so, we select another object $O_j$ that has the same class label ($s_i=s_j$) and can help fill up the missing points of $O_i$. Then, $P_i$ is complemented by additionally including the points $P_j$ into $P_i$. This complementing operation is iteratively repeated until $O_i$ is sufficiently complemented to have $P_i$ represent a whole-body object. An illustration of the iterative construction can be found in Figure~\ref{fig:overall}(a) and visualizations for KITTI can be found in the supplementary\footnote{Refer to our arXiv version for the Supplementary materials.}~(Figure 8, 9, and 10
%\ref{fig:appendix_car}, \ref{fig:appendix_ped}, and \ref{fig:appendix_cyc}
). For a given source object $O_i$, we maximize the diversity of the constructed object by making the iterative construction stochastic and by performing the iterative construction on the fly during the training. The computational overhead is minimized by performing an one-time pre-processing of the ground-truth database where the set of desirable candidates for complementing, $\{O_j\}$, are indexed for each source object $O_i$.

\vspace{0.1cm}
\subsubsection{Indexing Candidate Objects~(Pre-processing):}
In this pre-processing step, we identify and index $K$ objects as the candidates for complementing each object $O_i$ in the database. An object $O_j$ is considered to be a candidate for complementing $O_i$ if it satisfies two criteria: (1) high bounding box size similarity, and (2) high partition density for the low density partitions of $O_i$.
First, we convert all the objects into a canonical pose such that they have the same heading angle (i.e., $\theta_i=0$ for all $i$).
Then, we calculate the bounding box similarity with $O_i$ for all the objects $O_j$~($i\neq j$) in the ground-truth database to select the best $2K$ candidates. Among the $2K$ candidates, the best $K$ candidates that have high density partitions for $O_i$'s low density partitions (due to occlusion, signal miss, sparsity, etc.) are selected as the final candidates. Note that we define the partitions as the non-overlapping subdivided regions of the object $O_i$ with a fixed number of partitions for each class.
Also, we evaluate the density as the number of points per partition.
Further details can be found in Supplementary A.1.
%~\ref{sec:supp_iterative_construction}. 
$K=400$ was used for our experiments.
Partition can be considered as an extension of voxelization. In this study, we adopt the idea of partition to make it easy to analyze the density of each partition. This allows an effective method for combining multiple objects. Previously, partition was used in~\cite{choi2021part} for the purpose of applying a different augmentation technique to each partition.

\vspace{0.1cm}
\subsubsection{Constructing a Whole-body Object (Training Time):}
For a randomly selected source object $O_i$, we iteratively and stochastically construct its whole-body using the indexed $K$ objects. The details can be summarized as the following where all steps are performed in the canonical pose.
Since the car and cyclist objects are (roughly) symmetric on the x-axis in the canonical pose, (1) we first mirror their points as an additional operation before the iterative complementing. This step is skipped for the pedestrian objects.
Then, (2) for each partition, we uniform randomly sample a candidate from the $K$ candidates and add its points to $P_i$. 
(3) We iteratively repeat (2) until the object $O_i$ can be regarded as a whole-body object. We determine the object to be a whole-body when the proportion of partitions whose density is higher than the mean is at least 85\%. For a formal description and examples, see Supplementary A.1.
%~\ref{sec:supp_iterative_construction}.
Due to the stochastic nature of candidate sampling in (2), our method can generate \textit{combinatorially diverse} whole-body objects for a given source object $O_i$.

\begin{figure}[t!]
    \centering
    \includegraphics[width=0.49\textwidth]{figs/figure3-diversity.png}
    \caption{\label{fig:diversity}Diversification via placement: An iteratively constructed whole-body object can be placed at a random location, resulting in a random distance and a random viewpoint from the LiDAR. Therefore, the randomness in location provides a dimension of diversification. Also, the object can be randomly rotated and provide another dimension of diversification. Subsequently, s-HPR is applied according to the location and rotation to make the augmented point-cloud objects look realistic. The three objects constructed in Figure~\ref{fig:overall}(b) were used for this illustration.} 
\end{figure}

\subsection{Random Placement}
In the conventional data augmentation, an object from the ground-truth database must be placed at the original location with the original angle. If not, the occlusion and density of the object does not match the placement and thus causing a negative effect for the learning. \textsc{DR.CPO} does not suffer from this limitation because an object from ground-truth database is first transformed into a whole-body object and because a proper occlusion and density adjustment for the chosen location and angle are reflected in the HPR occlusion step. Taking full advantage of the flexibility, we place each whole-body object randomly and maximize the diversity of the augmentation as in the example of Figure~\ref{fig:overall}(b). To be specific, we first randomly rotate the object along the z-axis:  $\theta_i=\alpha_i$, where $\alpha_i$ is uniformly sampled between $-1.00\pi$ and $+1.00\pi$. Then, we select the location following a uniform distribution over the $(x,y)$ coordinates. 
As we will show later, the diversity shown in Figure~\ref{fig:diversity} through the randomness is essential for improving the performance and enhancing the data efficiency.

\vspace{0.2cm}
\subsection{HPR Occlusion}
To enforce self-occlusion and external-occlusion, we utilize Hidden Point Removal (HPR)~\cite{katz2007direct}.
We have chosen HPR mainly because of its computational efficiency. To the best of our knowledge, we are the first to adopt HPR for enforcing occlusion in autonomous driving.

\vspace{0.1cm}
\subsubsection{Hidden Point Removal (HPR):}
Given a set of points, $P_i$, our goal is to determine if each point $p_{i,t}$ is visible from the LiDAR viewpoint $C$. For this purpose, we use the HPR method illustrated in Figure 11
% ~\ref{fig:hpr} 
in the supplementary.
To determine whether the point $p_{i,t}$ is visible from $C$, HPR defines a sphere with a sufficiently large radius $R$ centered at origin $C$.
Spherical flipping is used to reflect every point $p_{i,t}$ internal to the sphere along the ray from $C$ to $p_{i,t}$ to its image outside the sphere, following the equation:
\begin{equation}
    \hat{p}_{i,t} = p_{i,t} + 2(R-||p_{i,t}||) \frac{p_{i,t}}{||p_{i,t}||}
\end{equation}
where $||p_{i,t}||$ is the length of $p_{i,t}$ from $C$. If we define $\hat{P}_i=\left\{ \hat{p}_{i,t} \right\}_{t=1}^{T}$, the visibility of $p_{i,t}$ is determined by whether it lies on the convex hull of $\hat{P}_{i}\cup \left\{ C \right\}$. Rigorously speaking, we choose $\epsilon$-visible points by adjusting $R$ for each class. Thanks to the efficient convex hull operation, HPR is computationally outstanding while meeting the precision required for the LiDAR data augmentation. For a full description of HPR, see \cite{katz2007direct}.


\vspace{0.1cm}
\subsubsection{s-HPR and e-HPR for Occlusion:}
We first apply HPR to perform self-occlusion~(s-HPR) and subsequently to perform external-occlusion~(e-HPR) as shown in Figure~\ref{fig:overall}(c) and \ref{fig:overall}(d). For s-HPR, HPR is applied to each individual object where the object's location and angle are accounted for. A crucial advantage of using s-HPR is that it automatically adjusts the point density of the object depending on the distance from the LiDAR. Examples are shown in Figure~\ref{fig:sparsity_with_distance}. The density adjustment is an important feature to make the augmented object's occlusion look realistic, and its effect on performance will be addressed in Section~\ref{subsec:hpr_occlusion}. Another important role of s-HPR is to reduce the number of points such that the computational burden of e-HPR can be lessened. For e-HPR, HPR is applied to the entire frame such that external-occlusion related to the inter-object dependency can be calculated. Compared to the s-HPR, this is a computationally heavy operation and the reduction in the number of points by s-HPR is beneficial for keeping e-HPR's computation burden low. Computational efficiency of \textsc{DR.CPO} will be addressed in Section~\ref{subsec:comp_efficiency}.

\begin{figure}[t!]
\centering
\includegraphics[width=\columnwidth]{figs/figure4-density.png}
\caption{\label{fig:sparsity_with_distance} 
Density adjustment with s-HPR: Depending on the object's distance from the LiDAR, point density is adjusted by s-HPR. Three examples are shown for the whole-body objects constructed from KITTI dataset.
}
\vspace{0.1cm}
\end{figure}


\section{Experiments}
\label{sec:experiments}

In this section, we show that \textsc{DR.CPO} is a data-efficient and model-agnostic augmentation method that can improve the performance of voxel, point, and voxel+point models, including the state-of-the-art voxel+point model, without incurring an extra computational cost.
For empirical evaluations, we use KITTI~\cite{geiger2013vision} dataset that is a standard 3D LiDAR point cloud benchmark and evaluate the standard metric $AP_{R40}$.
For a fair comparison, \textsc{DR.CPO} is set to follow the same class distribution as in CDA (details can be found in Table~\ref{tab:speed_comparisons}).
Unless stated otherwise, we use PVRCNN++~\cite{shi2021pv}, an open-source and state-of-the-art voxel+point method, as our default backbone model.
For training any model with \textsc{DR.CPO}, we trained the model by following the original protocol and code.
Additional details of the implementations and experimental settings are provided in Supplementary B.
%~\ref{sec:implementation_details_experiment}. 

\subsection{Data Efficiency}
\label{subsec:data-efficiency}
\begin{figure}[t!]
\centering
  \centering
  \subfloat{
    \includegraphics[width=0.5\columnwidth]{figs/figure5-CAR.pdf}
  }
  \subfloat{
    \includegraphics[width=0.5\columnwidth]{figs/figure5-Pedestrian.pdf}
  }
  \textbf{}
  \subfloat{
    \includegraphics[width=0.5\columnwidth]{figs/figure5-Cyclist.pdf}
  }
  \subfloat{
    \includegraphics[width=0.5\columnwidth]{figs/figure5-Overall.pdf}
  }
\caption{\label{fig:data_efficiency}
Data efficiency of \textsc{DR.CPO}: The training data size versus mAP score is plotted for KITTI. Except for car, \textsc{DR.CPO} achieves significant improvements over no-augmentation and CDA.}
\end{figure}

To examine the data efficiency of \textsc{DR.CPO}, we trained our model using subsets of KITTI's training examples. The results are shown in Figure~\ref{fig:data_efficiency}. Compared to the baseline (no augmentation) and CDA shown in gray and blue dotted lines, respectively, \textsc{DR.CPO} can match their performance with a much smaller data size. An exception is the class of car for CDA, and this is discussed in Section~\ref{sec:discussion}. For the overall performance, \textsc{DR.CPO} can perform as well as the baseline and CDA using only 4.60\% and 37.60\% of the training dataset, respectively.



\subsection{Model-agnostic Effectiveness}
\begin{table}[t!]
    \centering
    \caption{\label{tab:model_agnostic} \textsc{DR.CPO} is a model-agnostic augmentation method. \textsc{DR.CPO} can effectively enhance the performance of existing models that are based on voxel, point, and voxel+point representations. $AP_{R40}$ performance is shown.}
    \resizebox{\columnwidth}{!}{
    \begin{tabular}{@{}lccccc@{}}
    \toprule
    Method & Car & Pedestrian & Cyclist & Mean & Diff. \\\midrule
    SECOND~\cite{xu2021behind}      & 82.67   & 52.31   & 62.69   & 65.89 & $-$3.64 \\
    SECOND with \textsc{DR.CPO} & \textbf{83.15}   & 51.67   & \textbf{73.75}   & \textbf{69.53} &  \\ \midrule
    Point-RCNN~\cite{hu2021pattern} & 82.84   & 56.92   & 77.46   & 72.40 & $-$1.08          \\
    Point-RCNN with \textsc{DR.CPO}   & 82.36   & \textbf{59.86}   & \textbf{78.22}   & \textbf{73.48} &  \\ \midrule
    PV-RCNN~\cite{shi2020pv}& \textbf{86.70} & 57.61 & 75.87 & 73.39 & $-$2.79 \\
    PV-RCNN with \textsc{DR.CPO}& 84.75 & \textbf{62.33} & \textbf{81.47} & \textbf{76.18} &  \\  \bottomrule
    \end{tabular}
    }
    \vspace{-0.2cm}
\end{table}

\begin{table*}[thb!]
\centering
\caption{\label{tab:big_comparisons} 
Compilation of KITTI 3D LiDAR detection performance. State-of-the-art models are also included. For each column, any model performance that is superior to ours is indicated in bold.}
\resizebox{\textwidth}{!}{
\begin{tabular}{@{}lcccccccccccccc@{}}
\toprule
\multirow{2}{*}{Method} & \multicolumn{3}{c}{Car} & \multicolumn{3}{c}{Pedestrian} & \multicolumn{3}{c}{Cyclist} & \multicolumn{3}{c}{Overall} & \multirow{2}{*}{Mean} & \multirow{2}{*}{Diff.} \\ \cmidrule(lr){2-4}\cmidrule(lr){5-7}\cmidrule(lr){8-10}\cmidrule(lr){11-13}
 & E & M & D & E & M & D & E & M & D & Car & Ped & Cyc & & \\ \midrule
ContFuse~\cite{liang2018deep}                      & 83.68          & 68.78          & 61.67          & -              & -              & -              & -              & -              & -              & 71.38          & -              & -              & -              &        \\
3D IoU Loss~\cite{zhou2019iou}                     & 84.43          & 76.28          & 68.22          & -              & -              & -              & -              & -              & -              & 76.31          & -              & -              & -              &        \\
SECOND with LiDAR-Aug~\cite{fang2021lidar}         & 88.65          & 76.97          & 70.44          & 58.53          & 54.56          & 51.78          & -              & -              & -              & 78.69          & 54.96          & -              & -              &        \\
PointPillars with LiDAR-Aug~\cite{fang2021lidar}   & 87.75          & 77.83          & 74.90          & 59.99          & 55.15          & 52.66          & -              & -              & -              & 80.16          & 55.93          & -              & -              &        \\
PointRCNN with LiDAR-Aug~\cite{fang2021lidar}      & 89.56          & 79.51          & 77.89          & 67.46          & 59.06          & 56.23          & -              & -              & -              & 82.32          & 60.92          & -              & -              &        \\
PV-RCNN with LiDAR-Aug~\cite{fang2021lidar}        & 90.18          & \textbf{84.23} & 78.95          & 65.05          & 59.90          & 55.52          & -              & -              & -              & 84.45          & 60.16          & -              & -              &        \\
Part-A2-Net~\cite{guo2020deep}                     & \textbf{91.70} & \textbf{87.79} & \textbf{84.61} & -              & -              & -              & 81.91          & 68.12          & 61.92          & \textbf{88.03} & -              & 70.65          & -              &        \\
PV-RCNN with STRL~\cite{huang2021spatio}           & -              & -              & -              & -              & -              & -              & -              & -              & -              & 84.70          & 57.80          & 71.88          & 71.46          & \phantom{0}$-$6.84  \\
VoxelNet~\cite{zhou2018voxelnet}                   & 81.97          & 65.46          & 62.85          & 57.86          & 53.42          & 48.87          & 67.17          & 47.65          & 45.11          & 70.09          & 53.38          & 53.31          & 58.93          & $-$19.37 \\
StarNet~\cite{ngiam2019starnet}                    & 81.63          & 73.99          & 67.07          & 48.58          & 41.25          & 39.66          & 73.14          & 58.29          & 52.58          & 74.23          & 43.16          & 61.34          & 59.58          & $-$18.72 \\
PointPillars with PA-AUG~\cite{choi2021part}          & 83.70          & 72.48          & 68.23          & 57.38          & 51.85          & 46.91          & 70.88          & 47.58          & 44.80          & 74.80          & 52.05          & 54.42          & 60.42          & $-$17.87 \\
PV-RCNN with PA-AUG~\cite{choi2021part}            & 89.38          & 80.90          & 78.95          & 67.57          & 60.61          & 56.58          & 86.56          & 72.21          & 68.01          & 83.08          & 61.59          & 75.59          & 73.42          & \phantom{0}$-$4.88  \\
StarNet with PPBA~\cite{cheng2020improving}        & 84.16          & 77.65          & 71.21          & 52.65          & 44.08          & 41.54          & 79.42          & 61.99          & 55.34          & 77.67          & 46.09          & 65.58          & 63.12          & $-$15.18 \\
Point-GNN~\cite{guo2020deep}                       & \textbf{93.11} & \textbf{89.17} & \textbf{83.90} & 55.36          & 47.07          & 44.61          & 81.17          & 67.28          & 59.67          & \textbf{88.73} & 49.01          & 69.37          & 69.04          & \phantom{0}$-$9.26  \\
HotSpotNet~\cite{chen2020object}                   & 87.60          & 78.31          & 73.34          & \textbf{82.59} & 65.95          & 59.00          & 53.10          & 45.37          & 41.47          & 79.75          & \textbf{69.18} & 46.65          & 65.19          & $-$13.11 \\
Object as Hotspots (Dense)~\cite{chen2020object}   & 91.09          & \textbf{82.20} & 79.69          & \textbf{85.85} & 66.45          & 62.16          & 68.88          & 62.82          & 55.78          & 84.33          & \textbf{71.49} & 62.49          & 72.77          & \phantom{0}$-$5.53  \\
STD~\cite{guo2020deep}                             & \textbf{94.74} & \textbf{89.19} & \textbf{86.42} & 60.02          & 48.72          & 44.55          & 81.36          & 67.23          & 59.35          & \textbf{90.12} & 51.10          & 69.31          & 70.18          & \phantom{0}$-$8.12  \\
3DSSD~\cite{guo2020deep}                           & \textbf{92.66} & \textbf{89.02} & \textbf{85.86} & 60.54          & 49.94          & 45.73          & 85.04          & 67.62          & 61.14          & \textbf{89.18} & 52.07          & 71.27          & 70.84          & \phantom{0}$-$7.46  \\
PointRCNN~\cite{hu2021pattern}                     & 90.10          & 80.41          & 78.00          & 64.18          & 56.71          & 49.86          & 91.72          & 72.47          & 68.18          & 82.84          & 56.92          & 77.46          & 72.40          & \phantom{0}$-$5.89  \\
PV-RCNN with Pattern-Aware GT~\cite{hu2021pattern} & \textbf{92.13} & \textbf{84.79} & \textbf{82.56} & 65.99          & 58.57          & 53.66          & 90.38          & 72.03          & 67.96          & \textbf{86.49} & 59.41          & 76.79          & 74.23          & \phantom{0}$-$4.07  \\ \midrule
SECOND~\cite{xu2021behind}                         & 90.97          & 79.94          & 77.09          & 58.01          & 51.88          & 47.05          & 78.50          & 56.74          & 52.83          & 82.67          & 52.31          & 62.69          & 65.89          & $-$12.41 \\
PointPillars~\cite{xu2021behind}                   & 87.75          & 78.39          & 75.18          & 57.30          & 51.41          & 46.87          & 81.57          & 62.94          & 58.98          & 80.44          & 51.86          & 67.83          & 66.71          & $-$11.59 \\
BtcDet~\cite{xu2021behind}                         & \textbf{93.15} & \textbf{86.28} & \textbf{83.86} & 69.39          & 61.19          & 55.86          & 91.45          & 74.70          & 70.08          & \textbf{87.76} & 62.15          & 78.74          & 76.22          & \phantom{0}$-$2.08  \\
PV-RCNN~\cite{xu2021behind}                        & \textbf{92.57} & \textbf{84.83} & \textbf{82.69} & 64.26          & 56.67          & 51.91          & 88.88          & 71.95          & 66.78          & \textbf{86.70} & 57.61          & 75.87          & 73.39          & \phantom{0}$-$4.90  \\
PV-RCNN++~\cite{shi2021pv}(Reproduced)              & \textbf{91.72} & \textbf{84.82} & \textbf{82.03} & 66.00          & 60.05          & 54.70          & 91.39          & 70.14          & 66.92          & \textbf{86.19} & 60.25          & 76.15          & 74.20          & \phantom{0}$-$4.10  \\ \midrule
PV-RCNN++ with \textbf{\textsc{DR.CPO} (ours)}                                                   & 91.27          & 81.67          & 81.28          & 73.23          & \textbf{67.66} & \textbf{62.96} & \textbf{93.08} & \textbf{78.21} & \textbf{75.32} & 84.74          & 67.95          & \textbf{82.20} & \textbf{78.30} &        \\ \bottomrule
\end{tabular}
}
\vspace{0.1cm}
\end{table*}

Because \textsc{DR.CPO} is an augmentation method, we have chosen three latest and well-known models and examined if their performance can be enhanced with \textsc{DR.CPO}. The results are shown in Table~\ref{tab:model_agnostic}. In the table, SECOND is a voxel-based method, Point-RCNN is a point-based method, and PV-RCNN is a voxel+point-based method. The improvements from \textsc{DR.CPO} are 3.64\%, 1.08\%, and 2.79\%, respectively.

\subsection{Comparison with State-of-the-art Models}
Because of the improved detection head and the computationally efficient representative keypoints, we have used PV-RCNN++ as the backbone of \textsc{DR.CPO} and achieved mAP score of 78.30\%. We have compiled the previously reported records as much as we can into Table~\ref{tab:big_comparisons}, where the latest models such as SECOND, PointPillars, BtcDet, PV-RCNN, PV-RCNN++ are also included. In particular, BtcDet~\cite{xu2021behind}, which integrates multiple techniques including an extra shape occupancy network, is known to be the current state-of-the-art model. Simply by incorporating \textsc{DR.CPO} into the augmentation pipeline, the computationally efficint PV-RCNN++ can perform better than all the other models shown in Table~\ref{tab:big_comparisons} including BtcDet.
As in the data efficiency and model-agnostic effectiveness studies, the gain comes from the extraordinary improvements in pedestrian and cyclist categories. The improvements are large enough to overcome the loss in car category. We discuss the loss of car category in Section~\ref{sec:discussion}.

\subsection{Computational Efficiency}
\label{subsec:comp_efficiency}
For all of our experiments, \textsc{DR.CPO} was set to follow the same class distribution as in CDA for a fair comparison. This can be confirmed from the average object counts per frame in Table~\ref{tab:speed_comparisons}. Even with similar object counts, \textsc{DR.CPO} needs to perform additional computations including iterative construction and HPR occlusion. The computational efficiency in terms of training time per epoch, however, is not compromised as can be seen in the same table. In fact, \textsc{DR.CPO} is slightly faster than CDA. Despite the overhead of our augmentation method, the training time is more important than the data preparation time and \textsc{DR.CPO} requires less training time thanks to the reduced number of points after s-HPR and e-HPR. Overall, \textsc{DR.CPO} does not incur any additional burden when compared to the ubiquitously used CDA.

\begin{table}[t!]
\caption{\label{tab:speed_comparisons}
Object counts, the number of points, and training time per epoch.
Our method increases objects per frame without increasing the total points due to s-HPR and e-HPR operations. Overall training time does not increase because the delay in time on data preparation is offset by reduced time on processing model.
}
\centering
\resizebox{\columnwidth}{!}{
\begin{tabular}{@{}llccc@{}}
\toprule
                                &                       & No Aug. & CDA   & Ours  \\ \midrule
Average counts per frame        & Car objects           & \phantom{0.}4.00    & \phantom{.}13.70  & \phantom{.}13.40  \\
                                & Pedestrian objects    & \phantom{0.}0.60    & \phantom{.}11.80  & \phantom{.}12.90  \\
                                & Cyclist objects       & \phantom{0.}0.20    & \phantom{.}10.30  & \phantom{.}11.00  \\
Total points per frame          &                       & 18912  & 23170 & 17203 \\ \midrule
Training time per epoch (sec)   & On data preparation   & \phantom{00}2.10    & \phantom{00}4.11   & \phantom{00}9.55  \\
                                & On processing model   & 124.10  & 134.26 & 117.21 \\
                                & Total                 & 126.10  & 138.37 & 129.40 \\ \midrule
Performance (mAP)               &                       & \phantom{0}55.06  & \phantom{0}73.48 & \phantom{0}78.30 \\ \bottomrule
\end{tabular}
}
\vspace{0.1cm}
\end{table}


\section{Analysis of \textsc{DR.CPO}}
\label{sec:analysis}
In this section, we analyze \textsc{DR.CPO}'s three elements. When analyzing 
an element, the other two are kept the same as in the default setup of \textsc{DR.CPO}. 

\subsection{Iterative Construction}
\label{subsec:iter_const}
We have examined if constructing a whole-body object can indeed positively influence the performance, and the results are shown in Table~\ref{tab:construction}. By looking at the mean performance column, it can be confirmed that self-mirroring and iteration for complementing are both helpful. The gain from mirroring is only 0.11\% (the difference between the first and second rows). The performance gain is mostly due to the iteration process, and it is 95\% of the total gain.
From the car column, it can be seen that multiple iterations can actually hurt the performance of the car detection. 
For the cyclist column, however, the gain from single iteration to multiple iteration is large (1.63\%).

\subsection{Random Placement}
To analyze the effects of random location and random rotation, we have performed an ablation study and the results can be found in Table~\ref{tab:analysis_placement1}. While both are helpful as expected, the random location has improved the mean performance by 0.22\% while random rotation has improved it by 1.51\%. Apparently, the additional diversity from rotation is more effective than the additional diversity from random location. To further investigate random rotation, we have analyzed the effect of rotation range and the results are shown in 
Table~\ref{tab:analysis_placement2}. It can be observed that generally a larger rotation range is better. It is interesting to note that the largest gain is obtained when the rotation range is increased from $(-0.75\pi, +0.75\pi)$ to $(-1.00\pi, +1.00\pi)$, where the gain is 1.02\%. Rotation of $1.00\pi$ corresponds to switching the front and the back sides of an object, and thus inverting the direction of movement. 

\begin{table}[t!]
\caption{\label{tab:construction}Iterative construction: the effect of mirroring and iteration.}
\centering
\resizebox{\columnwidth}{!}{
\begin{tabular}{@{}llccccc@{}}
\toprule
Mirroring & \begin{tabular}[c]{@{}l@{}}Iteration\end{tabular} & Car & Pedestrian & Cyclist & Mean & Diff.  \\ \midrule
No & None & 85.08   & 63.01   & 79.75   & 75.95 & $-$2.35 \\
Yes & None & 85.18 & 63.31 & 79.65 & 76.06 & $-$2.24 \\
Yes & Single & 85.25 & 67.85 & 80.57 & 77.89 & $-$0.41        \\
Yes & Multiple & 84.74 & 67.95 & 82.20 & 78.30 &         \\ \bottomrule
\end{tabular}
}
\vspace{0.1cm}
\end{table}

\begin{table}[t!]
\caption{\label{tab:analysis_placement1}Random placement: ablation study of location and rotation.}
\centering
\resizebox{\columnwidth}{!}{
\begin{tabular}{@{}cccccccc@{}}
\toprule
    \begin{tabular}[c]{@{}l@{}}Random \\ Location\end{tabular}  & \begin{tabular}[c]{@{}l@{}}Random \\ Rotation\end{tabular}   & Car & Pedestrian      & Cyclist   & Mean & Diff. \\ \midrule
                    &                   & 84.31 &  65.58        & 79.85     & 76.58 & $-$1.72 \\
     \checkmark     &                   & 84.28 &  67.01        & 79.32     & 76.87 & $-$1.43 \\ 
                    & \checkmark        & 85.33 &  66.00        & 83.16     & 78.16 & $-$0.14 \\
     \checkmark     & \checkmark        & 84.74 &  67.95        & 82.20     & 78.30 &  \\
    \bottomrule
\end{tabular}
}
\end{table}

\begin{table}[t!]
\caption{\label{tab:analysis_placement2}Random placement: the effect of rotation range.}
\centering
\resizebox{\columnwidth}{!}{
\begin{tabular}{@{}ccccccc@{}}
\toprule
    Rotation range  & Car & Pedestrian & Cyclist   & Mean & Diff.\\ \midrule
    $(-0.25\pi, +0.25\pi)$ & 85.13 & 64.41 & 80.87 & 76.81 & $-$1.49 \\
    $(-0.50\pi, +0.50\pi)$ & 85.36 & 64.99 & 80.46 & 76.94 & $-$1.36 \\ 
    $(-0.75\pi, +0.75\pi)$ & 84.56 & 65.67 & 81.61 & 77.28 & $-$1.02 \\
    $(-1.00\pi, +1.00\pi)$ & 84.74 & 67.95 & 82.20 & 78.30 &  \\
    \bottomrule
\end{tabular}
}
\end{table}

\begin{table}[t!]
\caption{\label{tab:hpr_ablation}HPR occlusion: ablation study of s-HPR and e-HPR. s-HPR is responsible for density adjustment in addition to self-occlusion.}
\centering
\resizebox{\columnwidth}{!}{
\begin{tabular}{@{}cccccccc@{}}
\toprule
    s-HPR       & e-HPR          & Car   & Pedestrian  & Cyclist  & Mean & Diff. \\ \midrule
                &               & 84.64 & 52.33 & 78.94 & 71.97 & $-$6.33 \\
                & \checkmark    & 84.98 & 64.83 & 80.22 & 76.68 & $-$1.62 \\
     \checkmark &               & 85.47 & 63.58 & 83.05 & 77.37 & $-$0.93 \\
     \checkmark & \checkmark    & 84.74 & 67.95 & 82.20 & 78.30 & \\ 
    \bottomrule
\end{tabular}
}
\end{table}

\subsection{HPR Occlusion}
\label{subsec:hpr_occlusion}

To understand the effects of s-HPR and e-HPR, an ablation study was performed and the results are shown in Table~\ref{tab:hpr_ablation}. While the effectiveness of both can be confirmed, the impact is larger when s-HPR is removed (-1.62\%) than when e-HPR is removed (-0.93\%). This can be explained by s-HPR's additional functionality of adjusting density as explained in Figure~\ref{fig:sparsity_with_distance}.
When both s-HPR and e-HPR are turned off, the mean performance becomes 71.97\% and it is even worse than the CDA baseline of 74.20\% that can be found in Table~\ref{tab:big_comparisons} (mean performance of PV-RCNN++). This indicates that it is indeed undesirable to augment with whole-body objects and do not reflect the effects of occlusion. An augmented object becomes unrealistic without a proper occlusion, and it is better to avoid iterative construction and random placement when the occlusion step does not exist.

\begin{figure}[t!]
\centering
\includegraphics[width=\columnwidth]{figs/figure6-rotation.png}
\caption{\label{fig:diversity_rotation}
Diversity of the occluded shapes: Applying s-HPR to the randomly rotated objects results in a wide range of shapes depending on the object's rotation angle. For the above examples, distance is fixed at 10m and LiDAR viewpoint is always at the center.}
\end{figure}


\section{Discussion}
\label{sec:discussion}

\paragraph{Diversified and Realistic Augmentation by \textsc{DR.CPO}:}
The conventional data augmentation method is constrained in terms of the augmentation diversity it can provide - it relies on insertion of copied objects and rotation and scaling of the entire training frame. Compared to CDA, our method can provide diversified augmentations mainly via combinatorially stochastic nature of whole-body construction, random location, and random rotation. Their effectiveness for enhancing performance has been shown in Table~\ref{tab:construction}, \ref{tab:analysis_placement1}, and  \ref{tab:analysis_placement2}. 
Illustrations of diversification via placement can be found in  Figure~\ref{fig:diversity} and Figure~\ref{fig:diversity_rotation}.
Our method is also realistic in a few ways. Unlike the synthesis or rendering approaches, we construct the whole-body objects using only the real LiDAR observations collected with the real LiDAR system. Our approach can be relatively noisy but the constructed objects are guaranteed to be realistic. Another factor that makes our augmentation realistic is the adoption of HPR. It enables application of occlusion without any concern on the computational burden. In consequence, random placements of whole-body objects can become realistic by applying self-occlusion and external-occlusion with HPR.
Despite the advantages listed above, we were not able to consider the environment dependent distributions of the object location, heading angle, object class, object shape, etc. We have simply employed uniform random strategies, and this remains as a limitation of our work.  

\vspace{0.1cm}
\paragraph{Performance of Car Detection:}
Our method provides an excellent performance for pedestrian and cyclist detection as can be seen in Table~\ref{tab:big_comparisons}. The performance for car detection, however, is relatively less desirable. From the analysis tables in Section~\ref{sec:analysis}, it can be observed that the car detection performance can be easily improved by sacrificing the performance of the other two categories. For instance, the car performance can be improved from 84.74\% to 85.47\% by disabling e-HPR (see the last two rows of Table~\ref{tab:hpr_ablation}).
Similar observations can be made from Table~\ref{tab:construction}, \ref{tab:analysis_placement1}, and \ref{tab:analysis_placement2} where customizing the strategies for car detection can be beneficial for improving car detection.
In fact, we can even achieve 86.43\% of car detection performance by 
applying a simple yet effective modification. In Table 14 of Supplementary~E,
%~\ref{tab:additional_experiment} of Supplementary~\ref{sec:Supplementary_E}, 
we are showing the results for including three additional car objects into each scene. With this modification, the car performance is improved at the cost of the pedestrian performance, where the mean performance remains almost the same (slightly improved from 78.30\% to 78.34\%) and the computational overhead also remains almost the same (the total points per frame only marginally increases from 17,203 to 17,628).

Despite the obvious solutions for improving the car detection performance, we have focused on improving the mean performance only because our goal of this work is to develop a general augmentation scheme for any class of object. We didn't want to focus on tuning or developing class-dependent techniques. Adjusting the object numbers in each training scene (e.g., three more cars) is a way of controlling the performance trade-off over the three object categories, and thus it is presented only for the purpose of discussion. In this work, we have considered the mean performance as the only genuine metric of performance. 
As for the general augmentation schemes, it certainly remains open to investigate if a general modification, such as manipulating intensity of augmented points, can improve the car detection performance without harming the pedestrian and cyclist detection performance.

\vspace{0.1cm}
\paragraph{Waymo dataset:}
In the stage of iterative construction, we fill up the missing points of an object by adding points of other objects of the same class. It means that the objects of the same class should have similar shapes and characteristics for a successful construction. Unfortunately, the objects of Waymo dataset are labeled in a quite coarse manner. For instance, motorcycles and motorcyclists share the label of vehicle. Because of the need for additional labeling, we did not evaluate Waymo dataset. Nonetheless, \textsc{DR.CPO} should be effective for any dataset with decently refined categorization.


\section{Conclusion}
\label{sec:conclusion}
In this study, we proposed an augmentation method that was developed specifically for LiDAR datasets. Thanks to its diversified and realistic nature, \textsc{DR.CPO} can perform on par with no-augmentation and CDA augmentation using only 4.60\% and 37.60\% of the training data, respectively. \textsc{DR.CPO} is model-agnostic and can enhance the performance for voxel-based, point-based, and voxel+point-based models. \textsc{DR.CPO} also provides a superior performance without incurring an additional burden in computation. Experiment and analysis results show that \textsc{DR.CPO} is an excellent solution for pedestrian and cyclist detection but that there is a room for improvement for car detection.



\section*{Acknowledgement}
This work was supported by a National Research Foundation of Korea (NRF) grant funded by the Korea government (MSIT) (No. NRF-2020R1A2C2007139) and in part
by IITP grant funded by the Korea government (MSIT) [NO.2021-0-01343, Artificial Intelligence Graduate School Program (Seoul National University)].


\bibliography{Shin}

\newpage
\appendix
\section{Development Recap}
\subsection{Milestone}
A concrete milestone from 2015 to 2022 of deep learning-based camera calibration is shown in Figure \ref{fig:milestones}, spanning the main deep learning era. We classify all literature based on the uncalibrated camera model and its extended applications: standard model, distortion model, cross-view model, and cross-sensor model.%

\begin{figure*}[thb]
	\centering  \centerline{\includegraphics[width=1\linewidth]{figures/time_line.pdf}}
	\vspace{-2pt}
	\caption{A concise milestone of deep learning-based camera calibration methods. We classify all methods based on the uncalibrated camera model and its extended applications: standard model, distortion model, cross-view model, and cross-sensor model. \textbf{Standard model}: DeepFocal \cite{DeepFocal}, PoseNet \cite{PoseNet}, DeepHorizon \cite{DeepHorizon}, DeepVP \cite{DeepVP}, Chang et al.\cite{chang2018deepvp}, UprightNet \cite{UprightNet}, Lee et al. \cite{Lee}, NeurVPS~\cite{zhou2019neurvps}, Deep360Up \cite{Deep360Up}, Davidson et al. \cite{Davidson}, DeepFEPE \cite{DeepFEPE}, Baradad et al. \cite{Baradad}, Zheng et al. \cite{Zheng}, Zhu et al. \cite{Zhu}, StereoCaliNet \cite{StereoCaliNet}, SA-MobileNet \cite{SA-MobileNet}, Fang et al. \cite{Fang}, CPL \cite{CPL}, FocalPose\cite{FocalPose}, DirectionNet \cite{DirectionNet}, DVPD \cite{DVPD}, Do et al.\cite{Do}, CTRL-C \cite{CTRL-C}, SPEC \cite{SPEC}, DiffPoseNet\cite{DiffPoseNet}, SceneSqueezer\cite{SceneSqueezer}. \textbf{Distortion model}: Rong et al. \cite{Rong}, Hold-Geoffroy et al. \cite{Hold-Geoffroy}, DeepCalib \cite{DeepCalib}, URS-CNN~\cite{URS-CNN}, FishEyeRecNet \cite{FishEyeRecNet}, Shi et al. \cite{Shi}, DR-GAN \cite{DR-GAN}, STD \cite{STD}, UnFishCor \cite{UnFishCor}, BlindCor \cite{BlindCor}, RSC-Net \cite{RSC-Net}, Xue et al. \cite{Xue}, Zhuang et al. \cite{Zhuang}, Lopez et al. \cite{Lopez}, Zhao et al. \cite{Zhao}, DDM \cite{DDM}, MisCaliDet \cite{MisCaliDet}, DeepPTZ \cite{DeepPTZ}, FE-GAN \cite{FE-GAN}, PSE-GAN \cite{PSE-GAN}, RDC-Net \cite{RDC-Net}, Li et al. \cite{Li}, RDCFace \cite{RDCFace}, LaRecNet \cite{LaRecNet}, DeepUnrollNet~\cite{DeepUnrollNet}, OrdianlDistortion \cite{OrdianlDistortion}, PolarRecNet \cite{PolarRecNet}, DQN-RecNet \cite{DQN-RecNet}, Tan et al. \cite{Tan}, PCN \cite{PCN}, SIR \cite{SIR}, DaRecNet \cite{DaRecNet}, Wakai et al. \cite{Wakai}, GenCaliNet \cite{GenCaliNet}, JCD~\cite{JCD}, SS-WPC \cite{SS-WPC}, AW-RSC\cite{AW-RSC}, EvUnroll\cite{EvUnroll}, Fan\etal~\cite{fan2021inverting}, SUNet~\cite{SUNet}, CCS-Net~\cite{zhang2022learning}, FishFormer\cite{FishFormer}. \textbf{Cross-View model}: DHN \cite{DHN}, CLKN \cite{CLKN}, HierarchicalNet \cite{HierarchicalNet}, DeepFM \cite{DeepFM}, Poursaeed et al. \cite{Poursaeed}, UDHN \cite{UDHN}, PFNet \cite{PFNet}, SRHEN~\cite{SRHEN}, SSR-Net \cite{SSR-Net}, Abbas et al. \cite{Abbas}, Sha et al. \cite{Sha}, MHN \cite{MHN}, CA-UDHN \cite{CA-UDHN}, DLKFM \cite{DLKFM}, LocalTrans \cite{LocalTrans}, BasesHomo \cite{BasesHomo}, ShuffleHomoNet \cite{ShuffleHomoNet}, DAMG-Homo \cite{DAMG-Homo}, IHN \cite{IHN}, HomoGAN \cite{HomoGAN}, Liu et al. \cite{Liu}. \textbf{Cross-Sensor model}: RegNet~\cite{schneider2017regnet}, CalibNet\cite{iyer2018calibnet},  RGGNet~\cite{yuan2020rggnet}, SSI-Calib~\cite{zhu2020online}, SOIC~\cite{wang2020soic}, CalibRCNN~\cite{shi2020calibrcnn}, NetCalib~\cite{wu2021netcalib}, LCCNet~\cite{lv2021lccnet}, CFNet~\cite{lv2021cfnet}, SemAlign~\cite{liu2021semalign}, DXQ-Net~\cite{jing2022dxq}, SST-Calib~\cite{SST-Calib}, ATOP~\cite{ATOP}, FusionNet~\cite{wang2022fusionnet}, RKGCNet~\cite{RKGCNet}.}
	\label{fig:milestones}
	\vspace{-4pt}
\end{figure*}

\subsection{Statistic Analysis}
As we can observe in Figure~\ref{fig:publication_number}, the number of learning-based camera calibrations has grown since 2015 and boomed since 2019. And the learning targets are extended from the simple and pure parameters to complicated and hybrid parameters, driven by larger datasets, more reasonable learning strategies, more explicit learning representations, and more solid network architectures, etc.
\begin{figure*}[!t]
  \centering
  \includegraphics[width=.9\textwidth]{figures/publication_number.pdf}
  %\vspace{-20pt}
  \caption{A statistic analysis of deep learning-based camera calibration methods. To be specific, we summarize all literature based on the number of publications per year, calibration objectives, simulation of the dataset, and learning strategy.}
  \label{fig:publication_number}
  %\vspace{-0.3cm}
\end{figure*}

The data analysis of different learning strategies used in learning-based camera calibration is also shown in Figure~\ref{fig:publication_number}. From the statistic, six strategies have been investigated, in which supervised learning accounts for the largest majority (more than 90\%). Considering the expensive labeling works, some recent research explores liberating the training demand for camera parameters using semi-supervised learning, weakly-supervised learning, unsupervised learning, and self-supervised learning. Reinforcement learning also has been exploited to dynamically address the camera calibration problem. 

\section{Camera Model}
Researchers utilize mathematical formulas to establish camera models that describe the imaging process from a point in 3D world coordinates to its projection on a 2D image plane. Different cameras and systems correspond to different types of parametric models. In this section, we first provide a detailed formulation of the basic pinhole camera model. Then, we review more complex and useful camera models, as well as extended models studied in recent literature, to meet the advanced development of cameras and academic/industrial demands.

\subsection{Pinhole Camera Model}
The most popular and commonly applied camera model in computer vision is the pinhole camera model. It can be regarded as a geometrically accurate first-order approximation of the traditional camera. A pinhole camera has one single effective perspective because the pinhole aperture is thought to be an infinitesimal point through which all projection lines must pass.

Using a mathematical formulation, the camera model depicts the imaging process from a point in the 3D world coordinate to its projection on the 2D image plane. Assuming the homogeneous coordinates $\mathbf{P}_w = [X, Y, Z, 1]^\mathsf{T}  \in {\mathbb{R}}^{4\times1}$ and $\mathbf{P}_i =  [u, v, 1]^\mathsf{T} \in {\mathbb{R}}^{3\times1}$ denote a point in the 3D world coordinate and its corresponding point on a 2D image plane, respectively. Then, a camera model can be described by a projection mapping $M \in {\mathbb{R}}^{3\times4}$ between $\mathbf{P}_w$ and $\mathbf{P}_i$:
\begin{equation}\label{eq-general-camera-model}
\mathbf{P}_i =  M\mathbf{P}_w,
\end{equation}
where the projection can be further formed by:
\begin{equation}\label{eq-projection-ex}
\mathbf{P}_c = [\mathbf{R}\,|\,\mathbf{t}]\mathbf{P}_w,
\end{equation}
where $\mathbf{P}_c = [x_c, y_c, z_c]^\mathsf{T} \in {\mathbb{R}}^{3\times1}$ denotes a transformed point in the camera coordinate using a $3\times3$ rotation $\mathbf{R}$ and a $3$-dimension translation $\mathbf{t}$. The $3 \times 4$ matrix $[\mathbf{R}\,|\,\mathbf{t}]$ is generally named as the extrinsic camera matrix, in which the camera rotation can be further parameterized by three angles: yaw $\varphi$, pitch $\theta$, and roll $\psi$. Subsequently, the point $\mathbf{P}_c$ is projected onto a surface. This surface is represented by the pinhole camera model as a plane $z = 1$, and the normalized coordinate of the point in camera coordinate is expressed by $[x_n, y_n]^\mathsf{T} = [\frac{x_c}{z_c}, \frac{y_c}{z_c}]^\mathsf{T}$. 

Finally, the point on the normalized plane is projected onto the image plane, obtaining a pixel $\mathbf{P}_i$ by:
\begin{equation}
    \mathbf{P}_i = K [x_n, y_n, 1]^\mathsf{T},
\label{eq-projection-in}
\end{equation}
where $K\in {\mathbb{R}}^{3\times3}$ is an intrinsic camera matrix, which consists of various camera intrinsic parameters such as the focal length, skew coefficient, and image center:
\begin{equation}
K = \begin{bmatrix} f_x m_u & s & c_u \\ 
                                      0 & f_y m_v & c_v \\
                                      0 & 0 & 1\end{bmatrix},
\label{eq-intrincs}
\end{equation}
where $f_x$ and $f_y$ are the focal lengths at X-axis and Y-axis of the camera, respectively. Generally, for most cameras, $f_x = f_y$, and they are unified to $f$. $m_u$ and $m_v$ are the number of pixels per unit distance, in which $m_u = m_v$, if the image has square pixels. $s$ is the skew coefficient. A CCD sensor's pixels might not be precisely square, which would cause a slight distortion in the X or Y axes. The number of pixels on the CCD sensor per unit length in each direction is known as the skew coefficient. It would become $0$ when X-axis and Y-axis are perpendicular to each other. $[c_u, c_v]^\mathsf{T}$ is the coordinate of the image center. According to previous works and factory design, the intrinsic parameters can be refined by $s=0, m_u=m_v$ and focal length in the pixel unit, then Eq.~\eqref{eq-projection-in} can be reformulated as:    
\begin{equation}
    \mathbf{P}_i = \begin{bmatrix} f_x & 0 & c_u \\ 
                                      0 & f_y & c_v \\
                                      0 & 0 & 1\end{bmatrix} [x_n, y_n, 1]^\mathsf{T}.
\label{eq-projection-in_refined}
\end{equation}

In addition to numerical camera parameters, some geometric representations can provide useful clues for camera calibration, such as vanishing points and horizon lines. These representations establish clear relationships between image features and calibration objectives, which can alleviate the difficulty of learning conventional and implicit camera parameters.

Lines and points are both represented as three-dimensional vectors in homogeneous coordinates. The definitions for computing the line $\mathbf{l}$ that connects two points and the point $\mathbf{p}$ at the intersection of two lines can be given by:
\begin{equation}
  \mathbf{l} = \frac{\mathbf{p}_1 \times \mathbf{p}_2}{||\mathbf{p}_1 \times \mathbf{p}_2||}   \ \ \ \ \ \ \ \   \mathbf{p} = \frac{\mathbf{l}_1 \times \mathbf{l}_2}{||\mathbf{l}_1 \times \mathbf{l}_2||}
\end{equation}

There are two parameterizations of the horizon line: slope-offset ($\theta, \rho$) and left-right ($l, r$). Assuming that the viewing orientation is down the negative $z$-axis, with the positive $x$-direction to the right, and the positive $y$-direction to the up. As a result, the world viewing direction of the camera can be described by $R_c^\mathsf{T}[0,0,-1]^\mathsf{T}$. For the world vector $[0,1,0]^\mathsf{T}$ points in the zenith direction, a set of points $p$ can represent the horizon line:
\begin{equation}
  p^\mathsf{T}K^{-T}R[0,1,0]^\mathsf{T} = 0.
\end{equation}

As mentioned in Barnard \cite{barnard1983interpreting}, the normalized line direction vector $\mathbf{d}$ can be formulated for the Gaussian sphere representation of a vanishing point $\mathbf{v}$. In particular, supposed a 3D ray is described by $\mathbf{o} + \lambda \mathbf{d}$, where $\mathbf{o}$ and $\mathbf{d}$ are its origin and unit direction vector, respectively. Then, the vanishing point can be represented by $\lambda \to \infty$, in which the image coordinate is formed by $\mathbf{v} = [v_x, v_y]^T := \lim_{\lambda \to \infty} [p_x, p_y]^T\in \mathbb{R}^2$. Thus, the 3D direction of a line based on its vanishing point can be calculated by:
\begin{equation}
  \mathbf{d} = \begin{bmatrix} v_x-c_x & v_y-c_y & f \end{bmatrix}^T \in \mathbb{R}^3.  %
\end{equation}
By using $\mathbf{d}$ rather than $\mathbf{v}$, the degraded situations where $\mathbf{d}$ is parallel to the image plane are eliminated. Additionally, it provides a natural measurement for determining the separation between two vanishing points.

\subsection{Wide-angle Camera Model}
The perspective projection model, given a typical pinhole camera with focal length $f$, can be expressed as:
\begin{equation}
r = f \tan \theta, 
\label{eq-projection-ray}
\end{equation}
where $r$ denotes the projection distance between the principal point and the points in the image. $\theta$ denotes the angle between the incident ray and the optical axis of the camera. It is straightforward to determine that $\theta$ should be less than $90^{\circ}$. Without a projection point on the image plane, the incoming ray will not cross with the image plane and the pinhole camera will not be able to view anything behind. Because of their restricted field of view (FoV), most cameras cannot see all of the points in the 3D environment at the same time.

Due to the wide FoV, wide-angle cameras are increasingly widely used in computer vision and robotics tasks such as navigation, localization, and tracking. Specifically, an extra wide-angle lens called a fisheye camera is used to create a broad, hemispherical, or panoramic image. Fisheye lenses employ a specific mapping to produce convex and non-rectilinear images as opposed to images with straight lines of perspective (rectilinear images). However, the wide-angle camera violates the pinhole camera assumption and the captured image suffers from geometric distortions.

Geometric distortion induced by wide-angle cameras can generally be classified into radial distortion and tangential distortion (de-centering distortion). Radial distortion is the primary distortion in central single-view camera systems, exhibiting circular symmetry with respect to the distortion center. This distortion results in points on the image plane being moved away from their ideal location under the perspective camera model along the radial axis from the distortion center. Radial distortion models can be formulated as nonlinear functions of the radial distance \cite{fan2022wide}. On the other hand, tangential distortion occurs when the lens and image plane are not parallel. Tangential distortion, also known as de-centering distortion, is primarily caused by the lens assembly not being centered over and parallel to the image plane. Unlike radial distortion, tangential distortion has a geometric impact that is not solely along the radial axis, and can also cause rotation and skewing of the image plane with respect to the distance from the image center. The camera model with radial distortion and tangential distortion can be parameterized by:
\begin{equation}
 \left\{\begin{matrix}
    x_r &= x_d + \Bar{x}(k_1 r_d^2 + k_2 r_d^4 + k_3 r_d^6 + \cdots) \\ 
    &+ (p_1 (r_d^2 + 2\Bar{x}^2) + 2p_2\Bar{x}\Bar{y})(1+p_3r_d^2 + \cdots)
    \\ 
    y_r &= y_d + \Bar{y}(k_1 r_d^2 + k_2 r_d^4 + k_3 r_d^6 + \cdots) \\ 
    &+ (p_2 (r_d^2 + 2\Bar{y}^2) + 2p_1\Bar{x}\Bar{y})(1+p_3r_d^2 + \cdots)
    \end{matrix}\right.,   
    \label{eq-wide-angle}
\end{equation}
where $\Bar{x} = x_d - c_x$ and $\Bar{y} = y_d - c_y$. $K = (k_1, k_2, k_3, \dots)$ and $P = (p_1, p_2, p_3, \dots)$ are the radial distortion parameters and decentering distortion parameters, respectively. $r_d$ describes the radial distance from an image point to the distortion center $(c_x, c_y)$. Such an equation represents the mapping from a point $[x_d, y_d]^\mathsf{T}$ in the image captured by the wide-angle camera to that in the rectified image without the geometric distortion $[x_r, y_r]^\mathsf{T}$.

Previous works demonstrate that tangential distortion is basically insignificant and can be neglected. Moreover, as we surveyed, all learning-based camera calibration methods only consider the radial distortion for calibrating the wide-angle camera. To this end, Eq.\ref{eq-wide-angle} can be simplified by a Taylor expansion:
\begin{equation}
 \left\{\begin{matrix}
    x_r = x_d(k_1 r_d^2 + k_2 r_d^4 + k_3 r_d^6 + \cdots)
    \\ 
    y_r = y_d(k_1 r_d^2 + k_2 r_d^4 + k_3 r_d^6 + \cdots)
    \end{matrix}\right..   
    \label{eq-wide-angle-polynomial}
\end{equation}
This equation is known as the even-order polynomial model, which can also be expressed as an odd-order polynomial model by shifting the power. However, according to Wang et al. \cite{wang2009simple}, while the polynomial model is suitable for small distortions, it requires an unreasonably high number of non-zero distortion parameters for severe distortions. As an alternative, Fitzgibbon et al. \cite{fitzgibbon2001simultaneous} proposed a division model that more accurately approximates the genuine undistortion function of a common camera. For significant distortion, the division model is preferred over the polynomial model because it requires fewer terms:
\begin{equation}
 \left\{\begin{matrix}
    x_r = \frac{x_d}{k_1 r_d^2 + k_2 r_d^4 + k_3 r_d^6 + \cdots}
    \\ 
    y_r = \frac{y_d}{k_1 r_d^2 + k_2 r_d^4 + k_3 r_d^6 + \cdots}
    \end{matrix}\right..   
    \label{eq-wide-angle-division}
\end{equation}

Some classical works demonstrate the single-parameter division model (only with distortion parameter $k_1$ in Eq.\ref{eq-wide-angle-division}) seems to be sufficient for most wide-angle cameras, which has been widely applied in learning-based wide-angle camera calibration \cite{Rong, BlindCor, DeepCalib, DQN-RecNet}.

\begin{figure}[!t]
  \centering
  \includegraphics[width=.4\textwidth]{figures/global-rolling-shutter.pdf}
  %\vspace{-20pt}
  \caption{Comparison of the mechanism of global shutter camera and rolling shutter camera.}
  \label{fig:global-rolling-shutter}
  %\vspace{-0.3cm}
\end{figure}

\subsection{Rolling Shutter Camera Model}
Due to the compact design, low price, and high frame rate, numerous consumer cameras, including webcams and mobile phones, employ CMOS (complementary metal–oxide–semiconductor) sensors. But they are restricted to using rolling shutter (RS) devices. With a consistent time delay between each row, RS exposes the sensor array row by row from top to bottom, as opposed to global shutter (GS) based on CCD sensors, which simultaneously read out all rows of the sensor array. If the RS camera is moving while capturing the image, various distortions, such as skew, smear, or wobble, will break the reality of the original scene, which deviates from the pinhole camera paradigm. The unknown camera movements during the capturing process induce the so-called RS effects (also known as the jelly effect). In other words, an RS image is a row-by-row combination of GS images taken by a virtual moving GS camera throughout the camera readout time. The comparison of the RS camera and GS camera is shown in Figure~\ref{fig:global-rolling-shutter}.

The RS camera can be regarded as a high-frequency sensor that produces sparse spatial information with rich temporal coverage conveyed by distortions \cite{ait2006simultaneous}. Modeling the RS camera faces a common challenge of estimating the transformation between RS and GS images. Assume a 3D latent space-time volume captures the desired scene across the desired time period $[0, t_0]$ and creates a virtual GS image $\mathbf{I}^{\rm{GS}}$. We suppose the readout direction is from top to bottom, and then the row-by-row readout RS imaging $\mathbf{I}^{\rm{RS}}$ can be expressed by:
\begin{equation}
\mathbf{I}^{\rm{RS}} = \sum_{y=1}^H M(\mathbf{I}^{\rm{GS}}_{t_r}, y),
\label{eq-RS}
\end{equation}
where $H$ is the height of the RS image (total number of rows) and $y$ indicates the vertical coordinate. $M(\cdot, \cdot)$ masks a specific row in the GS image, in which $t_r$ represents the readout (offset) time for each row of RS.

On the other hand, by warping the RS features backward with an estimated displacement field, the GS image can be formulated by:
\begin{equation}
    \mathbf{I}^{\rm{GS}}(\mathbf{x}) = \mathbf{I}^{\rm{RS}}(\mathbf{x}+\mathbf{F}_{{GS}\rightarrow {RS}}(\mathbf{x})),
\label{eq-GS}
\end{equation}
where $\mathbf{F}_{{GS}\rightarrow {RS}} \in {\mathbb{R}}^2$ denotes the displacement field of the pixel $\mathbf{x}$ from the GS image to the RS image.

The above formulations describe the rolling shutter camera model under a short exposure scenario. When the exposure time of the camera increases, the motion blur effects occur in the captured image, jointly with the RS distortion:
\begin{equation}\label{eq:rscd}
	\mathbf{I}^{\rm{RS'}}_{(t)}[i] = \frac{1}{T}\int_{t-t_h+it_r-T/2}^{t-t_h+it_r+T/2}\mathbf{I}^{\rm{GS}}_{(t-t_h+it_r)}[i] dt,
\end{equation}
where $\mathbf{I}^{\rm{RS'}}_{(t)}[i]$ denotes the $i^{th}$ row of the RS distortion image $\mathbf{I}^{\rm{RS}}$ with the middle moment of exposure at time $t$. $T$ indicates the exposure time of camera and $t_h = (H/2)t_r$.

\subsection{Cross-View Camera Model}
The cross-view camera model is a type of multi-view camera system used in computer vision. It involves placing two or more cameras at opposite sides of a scene to capture multiple views of the same scene. This setup enables the creation of 3D reconstructions of the scene by triangulating corresponding points from multiple camera views. The cross-view camera model is commonly used in surveillance, robotics, and augmented reality applications, and provides a more accurate and complete representation of the scene than what can be achieved with a single camera. Alternatively, a camera with stable movement can also be regarded as a cross-view camera model.

In a cross-view camera model, the captured images can be used to calculate the fundamental matrix and homography matrix, which are essential tools for 3D reconstruction, image rectification, and camera calibration.

\noindent \textbf{Fundamental Matrix}
Geometric relationships between the 3D points and their projections onto the 2D plane impose constraints on the image points when two cameras capture the same 3D scene from different perspectives. This intrinsic projective geometry can be embodied by a fundamental matrix $\textbf{F}$.
\begin{equation}\label{eq-fundamental}
    \mathbf{F} = \mathbf{K_2}^{-T} [\mathbf{t}]_{\times} \mathbf{R}  \mathbf{K_1}^{-1}.
\end{equation}
Such an equation describes the epipolar geometry, where $\mathbf{K_1}$ and $\mathbf{K_2}$ indicate the intrinsic parameters of two cameras, and $\mathbf{R}$ and $[\mathbf{t}]_{\times}$ are the relative camera rotation and translation, respectively.

The fundamental matrix can be calculated from the correspondences of projected scene points by $q^T \mathbf{F} p = 0$, in which $q$ and $p$ are the matching points derived from two views. Specifically, the eight-point algorithm \cite{longuet1981computer} uses 8 point correspondences and enforces the rank-2 constraint using Singular Value
Decomposition (SVD), computing a matrix with the minimum Frobenius distance.

\noindent \textbf{Homography Matrix}
Estimating a 2D homography matrix (or projection transformation) is an elemental geometric task for a pair of images that are captured from the same planar surface in a 3D scene with different perspectives. An invertible mapping from one image plane to another with eight degrees of freedom: two each for translation, rotation, scale, and lines at infinity, is known as a homography. Supposed that the homogeneous coordinates $\mathbf{x} = [u, v, 1]^\mathsf{T}  \in {\mathbb{R}}^{3\times1}$ and $\mathbf{x}' = [u', v', 1]^\mathsf{T}  \in {\mathbb{R}}^{3\times1}$ are points from two images but indicating the same point in the 3D scene, a non-singular $3\times3$ matrix can represent a linear transformation that maps $\mathbf{x} \Leftrightarrow \mathbf{x}'$ as a planar projective transformation or homography $\mathbf{H}$:
\begin{align}
			\begin{bmatrix}
            u' \\ 
            v'   \\ 
            1  
			\end{bmatrix} 
          & \sim \begin{bmatrix}
            h_{11} & h_{12} & h_{13} \\ 
            h_{21} & h_{22} & h_{23} \\ 
            h_{31} & h_{32} & h_{33} 
			\end{bmatrix} 
            \begin{bmatrix}
            u \\ 
            v  \\ 
            1  
			\end{bmatrix}, 
\label{eq-homography}
\end{align}
where the transformation can be simplified as $\mathbf{x}' \sim \mathbf{H} \mathbf{x}$. This transformation can be rewritten by two following equations:
\begin{align}
 u' = \frac{h_{11}u + h_{12}v + h_{13}}{h_{31}u + h_{32}v + h_{33}} ;  
 v' = \frac{h_{21}u + h_{22}v + h_{23}}{h_{31}u + h_{32}v + h_{33}}.
 \label{eq-homography1}
\end{align}

Previous methods \cite{DHN, UDHN} point out that the above conventional $3\times3$ parameterization $\mathbf{H}$ is not desirable for training neural networks. Concretely, it is challenging to guarantee the non-singularity of $\mathbf{H}$ due to the significant variance in the size of the members of the $3\times3$ homography matrix. Moreover, the rotation, translation, scale, and shear components of the homography transformation are mixed in $\mathbf{H}$. For instance, the submatrix $[h_{11}\ \ h_{12}; h_{21}\ \ h_{22}]$ describes the homography's rotational term and the vector $[h_{13}, h_{23}]^T$ denotes the translation transformation. Considering the rotation and shear components typically have smaller magnitudes than the translation component, it will have a negligible impact on the loss function of the component elements, leading to an imbalance training problem with a neural network. Instead, a 4-point parameterization \cite{baker2006parameterizing} has been demonstrated to be more learning-friendly for learning-based homography estimation than the $3\times3$ parameterization. 
Supposed that the offsets of the image's vertex are $\Delta u_i = u_i' - u_i$ and $\Delta v_i = v_i' - v_i$, then the 4-point parameterization $\mathbf{\widetilde H}$ can describe a homography by:

\begin{equation}
\mathbf{\widetilde H} = \begin{pmatrix} \Delta u_{1} & \Delta v_{1} \\  \Delta u_{2} & \Delta v_{2}   \\  \Delta u_{3} & \Delta v_{3} \\  \Delta u_{4} & \Delta v_{4}  \end{pmatrix}. 
\end{equation}
The 4-point parameterization owns eight variables, which are equivalent to the matrix formulation of the homography. It is straightforward to transform from $\mathbf{\widetilde H}$ to $\mathbf{H}$ using the normalized Direct Linear Transform (DLT)~\cite{horn1987direct} if the four corners' displacement is known.

\subsection{Cross-Sensor Model}
Modern robots are often equipped with various sensors to provide a comprehensive understanding of the environment. These sensors capture scenes using different types of representations. For autonomous cars and robotics, cameras and Light Detection and Ranging sensors (LiDAR) are commonly used for vision tasks. The 3D LiDAR records long-range spatial data as sparse point clouds, while the camera captures texturally dense 2D color RGB images. Combining these sensors can facilitate 3D reconstruction and provide precise and robust perception for the robots, overcoming the limitations of individual sensors.

However, collision and vibration problems can occur when using different sensors in a robot or system. Additionally, the 3D point clouds cannot be effectively projected onto a 2D image without accurate extrinsic parameters, making it difficult to reliably correlate pixels in an image with depth information. Therefore, it is crucial to precisely calibrate the 2D-3D matching correspondences between pairs of temporally synchronized camera and LiDAR data.

The appropriate extrinsic calibration of the transformation (\textit{i.e.}, rotation and translation) between the camera and LiDAR in 6-DoF is a key condition for data fusion. To be more specific, 3D LiDAR point cloud $PC = [X, Y, Z] \in \mathbb{R}^3$ can be projected onto the image plane by transforming it into the camera coordinate using the extrinsic matrix $T$ between the camera and LiDAR as well as camera intrinsic $K$. The inverse depth and the projected 2D coordinates can be represented as $d = 1 / Z$ and $p = [u, v] \in \mathbb{R}^2$, respectively. Then, the camera-LiDAR model can be described by:
\begin{equation}
\begin{bmatrix}
    u\\ 
    v\\ 
    d
    \end{bmatrix}
    = 
    \begin{bmatrix}
        f_x(\hat{X}/\hat{Z}) + c_x \\
        f_y (\hat{Y}/\hat{Z}) + c_y \\
        1/\hat{Z}
    \end{bmatrix},
\label{eq-camera-LiDAR}
\end{equation}
where $(f_x, f_y)$ and $(c_x, c_y)$ indicate the focal lengths and the image center as listed in Eq. \ref{eq-intrincs}. $[\hat{X}, \hat{Y}, \hat{Z}]$ is the transformed point cloud $\hat{PC}$ using the estimated extrinsic matrix:
\begin{equation}
    [\hat{X}, \hat{Y}, \hat{Z}, 1]^\mathsf{T}= T [X, Y, Z, 1]^\mathsf{T}.
\end{equation}

Most deep learning works exploit the Lie algebra to parameterize the calibration camera-LiDAR extrinsic parameters. In particular, the output of the calibration network is a 1 x 6 vector $\xi = (v, \omega) \in se(3)$ in which $v$ is the translation vector, and $\omega$ is the rotation vector. To recover the original objectives, the rotation vector in $so(3)$ should be transformed to its corresponding rotation matrix. Supposed that $\omega = (\omega_1, \omega_2, \omega_3)^T$, an element $\omega \in so(3)$ can be transformed to $SO(3)$ using the exponential map by:
\begin{equation}
exp: so(3) \rightarrow SO(3); \; \hat{\omega} \mapsto e^{\hat{\omega}},
\end{equation}
where $\hat{\omega}$ and $e^{\hat{\omega}}$ denote the skew-symmetric matrix from $\omega$ and Taylor series expansion for the matrix exponential function, respectively. Then, the rotation matrix can be formed in $SO(3)$, and its Rodrigues formula is derived from the above equation by:
\begin{equation}
R = e^{\hat{\omega}} = I + \frac{\hat{\omega}}{\Vert{\omega}\Vert}\sin{\Vert{\omega}\Vert} + \frac{\hat{\omega}^2}{\Vert{\omega}\Vert^2}(1 - \cos(\Vert\omega\Vert)).
\end{equation}
Thus, the 3D rigid body transformation $T \in SE(3)$ between camera and LiDAR can be represented by:
\begin{equation}
   T = \left( \begin{array}{cc} R & t \\ 0 & 1 \end{array}\right) \text{where } R \in SO(3), t\triangleq v \in \mathbb{R}^3.
\end{equation}


\section{More Future Directions}

\subsection{Dataset}

One of the main challenges of learning-based camera calibrations is the difficulty in constructing datasets with high accuracy. This requires laborious manual intervention to obtain real-world data with labels. As we summarized, approximately 70\% of the works rely on synthesized datasets. However, the significant differences between synthesized and real-world datasets cannot be ignored, leading to domain gaps in the learned models. Therefore, the construction of a standardized, large-scale calibration dataset would significantly benefit this community. Recent works have demonstrated that well-designed learning strategies, such as semi-supervised learning~\cite{SS-WPC}, self-supervised learning~\cite{Fang, SSR-Net}, and unsupervised learning~\cite{UDHN, CA-UDHN}, can help address the demand for annotations in learning-based camera calibrations. These strategies also have the potential to discover additional calibration priors within the data itself.

\subsection{Transfer learning} 
The advancements in deep learning have led to the development of transfer learning techniques, which could facilitate the transfer of knowledge learned from one camera to another. This approach can significantly speed up and streamline the calibration process, making it more efficient and cost-effective. Transfer learning can be especially useful in applications that involve multiple cameras or mobile devices. For example, in a multi-camera system, transfer learning can be used to calibrate all the cameras using the data collected from a single camera, reducing the time and effort required for calibration. Similarly, in mobile devices, transfer learning can enable faster and more accurate calibration of the camera, resulting in improved image quality and performance.

\subsection{Robustness to noise and outliers}
Another promising application of deep learning in camera calibration is improving the robustness of calibration to noise and outliers in the data. This approach can help ensure accurate calibration even in challenging environments, with low-quality data or noisy sensor readings. Conventionally, camera calibration algorithms are sensitive to noise and outliers in the data, which can lead to significant errors in the estimated camera parameters. However, with the application of deep learning, it is possible to learn more robust and accurate models that can better handle noise and outliers in the data. For instance, regularization techniques can be used to impose constraints on the learned parameters, preventing overfitting and enhancing the generalization ability of the model. Moreover, outlier detection techniques can be used to identify and exclude data points that are likely to be outliers, reducing their impact on the calibration process. This can be achieved using various statistical and machine-learning methods, such as clustering, classification, and regression.

\subsection{Online calibration}
With the rapid development of deep learning, online camera calibration is becoming more efficient and practical. This technique involves updating the calibration parameters in real-time, allowing for better performance as the camera moves or as the environment changes. This can be achieved using deep learning algorithms that can learn the complex relationships between the camera parameters and the image data. Learning-based camera calibration has the potential to revolutionize various industries, such as robotics and augmented reality. In robotics, online calibration can improve the accuracy of robot vision, which is crucial for tasks such as object detection and manipulation. Similarly, in augmented reality, online calibration can enhance the user experience by ensuring that virtual objects are correctly aligned with the real world. This can help create more realistic and immersive AR applications, which have numerous practical applications in fields such as entertainment, education, and training.

\subsection{Multimodal calibration}
The potential of deep learning techniques in camera calibration goes beyond traditional photography and computer vision applications. It could also be applied to calibrate cameras with other sensors, such as remote sensing, infrared sensors, or radar. This advancement could lead to more precise and robust perception in various applications, including but not limited to autonomous driving, where multiple sensors are used. Incorporating deep learning-based calibration methods with multiple sensors could enhance the accuracy of the fusion of data from different sources. It could facilitate more accurate perception in challenging environments such as low-light conditions, occlusions, and adverse weather conditions. Furthermore, the ability to calibrate multiple sensors with deep learning methods could provide more reliable and consistent results compared to traditional calibration techniques.

These are a few potential directions for future research in camera calibration with deep learning. As the field continues to evolve, there may be many other exciting avenues for exploration and innovation. In addition, it is also thrilling to see how this technology will continue to impact various industries in the future.

\end{document}

